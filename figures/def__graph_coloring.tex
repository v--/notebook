\documentclass{classes/graphs}

\begin{document}
  \begin{tikzpicture}
    \GeneralizedPetersen{5}{2}

    \node[dot, fill] at (2_inner) {};
    \node[dot, fill] at (1_outer) {};
    \node[dot, fill] at (3_outer) {};
    \node[dot, fill=lightgray] at (3_inner) {};
    \node[dot, fill=lightgray] at (4_inner) {};
    \node[dot, fill=lightgray] at (0_outer) {};
  \end{tikzpicture}

  \begin{tikzpicture}[scale=1.5]
    \pgfmathsetmacro \n {6};
    \pgfmathsetmacro \limit {\n - 1};

    \foreach \i in { 0, ..., \limit } {
      % We apply a rotation so that the edge coloring would look more natural
      \node[dot] (\i) at ({(\i + 2) * 360/\n}: 1) {};
    }

    \foreach \i in { 0, ..., 3 } {
      \foreach \j in { 0, ..., \i } {
        \draw[edge] (\i) edge (\j);
      }
    }

    \foreach \i in { 4, ..., 5 } {
      \foreach \j in { 0, ..., \i } {
        \draw[edge, lightgray] (\i) edge (\j);
      }
    }
  \end{tikzpicture}
\end{document}
