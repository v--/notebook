\documentclass{classes/graphs}

\begin{document}
  \begin{tikzpicture}
    \GeneralizedPetersen[-3/4]{5}{2}

    \node[dot, fill] at (2_inner) {};
    \node[dot, fill] at (1_outer) {};
    \node[dot, fill] at (3_outer) {};
    \node[dot, fill=gray] at (3_inner) {};
    \node[dot, fill=gray] at (4_inner) {};
    \node[dot, fill=gray] at (0_outer) {};
  \end{tikzpicture}

  \begin{tikzpicture}
    \node[dot] (1) at (90:1) {};
    \node[dot] (2) at (210:1) {};
    \node[dot] (3) at (330:1) {};

    \draw[edge] (1) edge (2);
    \draw[edge] (2) edge[draw=lightgray] (3);
    \draw[edge] (3) edge[draw=lightgray] (1);
  \end{tikzpicture}
\end{document}
