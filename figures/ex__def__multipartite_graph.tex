\documentclass{classes/graphs}

\begin{document}
  \begin{tikzpicture}
    \pgfmathsetmacro \n {6};
    \pgfmathsetmacro \limit {\n - 1};

    \foreach \i in { 0, ..., \limit } {
      \pgfmathtruncatemacro \q {floor(\i / 2)};
      \pgfmathtruncatemacro \r {Mod(\i, 2)};
      \node[dot] (\i) at (\r,\q) {};
    }

    \foreach \i in { 1, ..., \limit } {
      \pgfmathtruncatemacro \j {Mod(\i - 1, \n)};
      \draw[edge] (\i) edge (\j);
    }
  \end{tikzpicture}

  \begin{tikzpicture}
    \foreach \i in { 1, 2, 7, 8, 13, 14 } {
      \node[dot] (\i) at (20 * \i:1) {};
    }
    \node[dot] (14) at (280:1) {}; % TikZ cannot find the last node for whatever reason

    \draw[edge] (1) to (8) to (13) to (2) to (7) to (14) to (1);
  \end{tikzpicture}

  \begin{tikzpicture}
    \pgfmathsetmacro \n {6};
    \pgfmathsetmacro \limit {\n - 1};

    \foreach \i in { 0, ..., \limit } {
      \pgfmathtruncatemacro \q {floor(\i / 2)};
      \pgfmathtruncatemacro \r {Mod(\i, 2)};
      \node[dot] (\i) at (\r,\q) {};
    }

    \foreach \i in { 0, ..., \limit } {
      \pgfmathtruncatemacro \j {Mod(\i + 1, \n)};
      \draw[edge] (\i) edge (\j);
    }
  \end{tikzpicture}

  \begin{tikzpicture}
    \node[dot] (0) at (0,0) {};
    \node[dot] (1) at (1,0) {};
    \node[dot] (2) at (0,-1) {};
    \node[dot] (3) at (1,-1) {};
    \node[dot] (4) at (2,0) {};

    \draw[edge] (0) edge (1);
    \draw[edge] (1) edge (2);
    \draw[edge] (2) edge (3);
    \draw[edge] (3) edge (4);
    \draw[edge] (4) edge[bend right] (0);
  \end{tikzpicture}
\end{document}
