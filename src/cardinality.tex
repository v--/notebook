\subsection{Cardinality}\label{subsec:cardinality}

\begin{Definition}\label{def:set_domination}\cite[145]{Enderton1977}
  We say that the set \( X \) is \Def{dominated by \( Y \)} and write \( \Abs{X} \leq \Abs{Y} \) if there exists an \hyperref[def:function_invertibility/injection]{injection} from \( X \) to \( Y \).
\end{Definition}

\begin{Definition}\label{def:equinumerous_sets}\cite[129]{Enderton1977}
  We say that the sets \( X \) and \( Y \) are \Def{equinumerous} and write \( X \cong Y \) if there exists a \hyperref[def:function_invertibility/bijection]{bijection} between \( X \) and \( Y \).
\end{Definition}

\begin{Theorem}[Cantor-Schröder-Bernstein]\label{thm:cantor_schroder_bernstein}\cite[147]{Enderton1977}
  If \( \Abs{X} \leq \Abs{Y} \) and \( \Abs{Y} \leq \Abs{X} \), then \( X \cong Y \).
\end{Theorem}

\begin{Proposition}\label{thm:equinumerousity_equivalence}\cite[theorem 6A]{Enderton1977}
  \hyperref[def:equinumerous_sets]{Equinumerosity} satisfies the equivalence relation \hyperref[def:equivalence_relation]{axioms} (however it is formally not an equivalence relation since we cannot define relations on the class of all sets; see \fullref{def:set_zfc}).
\end{Proposition}

\begin{Theorem}[Cantor]\label{thm:cantor_power_set_theorem}\cite[theorem 6B]{Enderton1977}
  No set \( X \) is equinumerous with its power set \( \Pow(X) \).
\end{Theorem}
\begin{proof}
  Fix some function \( f: X \to \Pow(X) \). Define the set
  \begin{equation*}
    Y \coloneqq \{ x \in X \colon x \not\in f(x) \}.
  \end{equation*}

  Note that \( Y \subseteq X \) and thus \( Y \in \Pow(X) \), however \( Y \) is not in the \hyperref[def:function]{image} \( \Img f \) and thus \( f \) is not a \hyperref[def:function_invertibility/surjection]{surjection}.

  Since \( f \) was arbitrary, we conclude that no function \( f: X \to \Pow(X) \) is a surjection and, hence, \( X \not\cong \Pow(X) \).
\end{proof}

\begin{Definition}\label{def:finite_set}\cite[133]{Enderton1977}
  A set \( A \) is \Def{finite} if it is equinumerous to a natural number as defined in \fullref{def:peano_arithmetic_models/omega} (using the convention that \( \varnothing \) corresponds to zero).

  If \( A \) is not finite, we say that it is \Def{infinite}.
\end{Definition}

\begin{Proposition}\label{thm:infinite_set_iff_equinumerous_to_proper_subset}\cite[Corollary 6D]{Enderton1977}
  A set is \hyperref[def:finite_set]{infinite} if and only if it is equinumerous to a proper subset of itself.
\end{Proposition}

\begin{Theorem}\label{thm:equinumerous_ordinal_existence}\cite[197]{Enderton1977}
  For every set, there exists at least one \hyperref[def:ordinal]{ordinal} equinumerous to it.
\end{Theorem}

\begin{Definition}\label{def:cardinal}\cite[197]{Enderton1977}
  For each set \( A \), define its \Def{cardinal} or \Def{cardinal number} \( \Card A \) as the smallest ordinal that is equinumerous to \( A \) (a smallest ordinal always exists by \fullref{def:ordinal}).

  If \( \xi \) and \( \eta \) are cardinal numbers, we define \( \xi \leq \eta \) to mean that \( \eta \) \hyperref[def:set_domination]{dominates} \( \xi \), i.e.
  \begin{equation*}
    \xi \leq \eta \iff \Abs{\xi} \leq \Abs{\eta}.
  \end{equation*}
\end{Definition}

\begin{Theorem}\label{thm:cardinal_trichotomy}\cite[theorem 6M(5)]{Enderton1977}
  If \( \xi \) and \( \eta \) are cardinals, then either
  \begin{itemize}
    \item \( \xi \leq \eta \)
    \item \( \xi = \eta \)
    \item \( \xi \geq \eta \)
  \end{itemize}
\end{Theorem}

\begin{Corollary}[Pigeonhole principle]\label{def:pigeonhole_principle}
  If the cardinality of \( X \) is greater than the cardinality of \( Y \), then there exists no injective function from \( X \) to \( Y \).
\end{Corollary}

\begin{Remark}\label{remark:cardinals}
  We can think of cardinal numbers as \enquote{choosing}\AOC a special set out of the equivalence classes obtained from \fullref{thm:equinumerousity_equivalence}.

  Since the natural numbers as defined in \fullref{def:natural_numbers} are ordinals and no two different natural numbers are equinumerous, we identify the cardinal numbers for \hyperref[def:finite_set]{finite sets} with natural numbers.

  We give special names to
  \begin{RemEnum}
    \ILabel{remark:cardinals/countable} \( \aleph_0 \coloneqq \Card(\omega) \), the \Def{cardinality of the natural numbers}.
    \ILabel{remark:cardinals/continuum} \( c \coloneqq \Card(\BR) = \Card(\Pow(\omega)) \), the \Def{cardinality of the continuum}.
  \end{RemEnum}
\end{Remark}

\begin{Proposition}\label{thm:cardinals_well_ordered}
  The class of all \hyperref[def:cardinal]{cardinals} is well-\hyperref[def:well_ordered_set]{ordered}, that is, every set of cardinals has a least element.
\end{Proposition}
\begin{proof}
  Direct consequence of \fullref{thm:ordinals_are_well_ordered} and \fullref{def:cardinal}.
\end{proof}

\begin{Hypothesis}[Continuum hypothesis]\label{hyp:continuum_hypothesis}\cite[165]{Enderton1977}
  There exists no cardinal \( \xi \) such that \( \aleph_0 < \xi < c \).
\end{Hypothesis}

\begin{Remark}\label{remark:continuum_hypothesis}\cite[165]{Enderton1977}
  \Fullref{hyp:continuum_hypothesis} has been shown by G\"odel to not be disprovable in \hyperref[def:set_zfc]{ZFC} and by Cohen to not be provable in ZFC.
\end{Remark}

\begin{Definition}\label{def:cardinal_arithmetic}
  Let \( \xi \) and \( \eta \) be cardinal numbers. We define
  \begin{DefEnum}
    \ILabel{def:cardinal_arithmetic/addition}(addition) \( \xi + \eta \coloneqq \Card(\xi \coprod \eta) \), where \( \coprod \) denotes disjoint \hyperref[def:disjoint_union]{unions}.
    \ILabel{def:cardinal_arithmetic/multiplication}(multiplication) \( \xi \cdot \eta \coloneqq \Card(\xi \times \eta) \)
    \ILabel{def:cardinal_arithmetic/exponentiation}(exponentiation) \( \xi^\eta \coloneqq \Card(\Cat{Set}(\eta, \xi)) \) (see \fullref{def:category_of_sets})
  \end{DefEnum}
\end{Definition}

\begin{Proposition}\label{thm:countable_union_of_countable_sets}\cite[Theorem6Q]{Enderton1977}
  A countable union of countable sets is countable.
\end{Proposition}
