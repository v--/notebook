\begin{definition}\label{def:logic_syntax}
  In logic, we are interested in \Def{terms} (see \fullref{def:first_order_language/term}) and \Def{formulas} (see \fullref{def:propositional_language/formula} or \fullref{def:first_order_language/formula}), which are specific strings of symbols, and their \Def{valuation} (see \fullref{def:propositional_valuation} or \fullref{def:first_order_valuation}), which maps terms and formulas into certain \hyperref[def:concrete_category]{structured set}.
\end{definition}

\begin{definition}\label{rem:syntax_and_semantics}


  A set \( \CF \) of \Def{logical formulas}. 


  \begin{RemEnum}
    \ILabel{def:propositional_model/satisfiability} If \( \varphi \) is true in \( \CA \) under every variable assignment, we say that \( \varphi \) is \Def{valid} in \( \CA \) and that \( \CA \) is a \Def{model} of \( \varphi \) and write \( \CA \models \varphi \). We extend this to sets of formulas \( \Gamma \) via conjunction.

    \ILabel{def:propositional_model/entailment} If every interpretation that satisfies all formulas in the set \( \Gamma \) also satisfies the formula \( \varphi \), we say that \( \Gamma \) \Def{entails} \( \varphi \) and write \( \Gamma \models \varphi \).

    \ILabel{def:propositional_model/tautology} If all interpretations satisfy \( \varphi \), we call \( \varphi \) a \Def{tautology}.

    \ILabel{def:propositional_model/contradiction} Dually, if no interpretations satisfy \( \varphi \), \( \varphi \) is a \Def{contradiction}.

    \ILabel{def:propositional_model/equivalence} If \( \varphi\Val{I} = \psi\Val{I} \) for every interpretation \( I \), we say that \( \varphi \) and \( \psi \) are \Def{semantically equivalent} and write \( \varphi \cong \psi \).
  \end{RemEnum}
\end{definition}
