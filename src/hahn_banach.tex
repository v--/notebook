\subsection{The Hahn-Banach theorem}\label{subsec:hahn_banach}

The Hahn-Banach theorem is an important result that can be stated differently and in different levels of generality.

\begin{theorem}[Geometric Hahn-Banach theorem/Mazur's theorem]\label{thm:geometric_hahn_banach}\mcite[24]{ИоффеТихомиров1974}
  Fix a \hyperref[def:topological_vector_space]{topological vector space} \( X \). Let \( A \subseteq X \) be an open \hyperref[def:convex_hull]{convex} set and \( L \subseteq X \) be a subspace that is disjoint from \( A \). Then there exists a continuous linear functional \( x^* \in X^* \) such that
  \begin{equation*}
    \begin{array}{l}
      \real \inprod{x^*} x > 0, x \in A \\
      \real \inprod{x^*} x = 0, x \in L
    \end{array}
  \end{equation*}

  See \fullref{rem:linear_functionals_over_c} for a justification of only considering the real part of \( x^* \).
\end{theorem}

\begin{corollary}\label{thm:hahn_banach_implies_functionals_vanish_nowhere}\mcite[24]{ИоффеТихомиров1974}
  The \hyperref[def:dual_vector_space]{dual} of a Hausdorff \hyperref[def:locally_convex_space]{locally convex space} \( X \) does not \hyperref[def:functions_vanish_nowhere]{vanish} at the nonzero vectors of \( X \).
\end{corollary}
\begin{proof}
  Fix a nonzero point \( x \in X \). The result follows from \fullref{thm:geometric_hahn_banach} with \( L \coloneqq \{ 0 \} \) and \( A \) --- any convex set containing \( x \) and not containing zero. Such a set \( A \) exists because the topology is Hausdorff and \( x \) has a neighborhood disjoint from any point in \( L \).
\end{proof}

\begin{corollary}\label{thm:hahn_banach_implies_annihilator_nontrivial}\mcite[25]{ИоффеТихомиров1974}
  The \hyperref[def:vector_space_annihilator]{annihilator} of any proper subspace of a Hausdorff \hyperref[def:locally_convex_space]{locally convex space} contains nonzero elements.
\end{corollary}
\begin{proof}
  Denote the proper subspace by \( L \subsetneq X \). Fix \( x \in X \setminus L \) and let \( A \) be a convex neighborhood of \( x \) that is disjoint from \( L \). The result follows from \fullref{thm:geometric_hahn_banach}.
\end{proof}

\begin{corollary}\label{thm:hahn_banach_implies_duality_mapping_nonempty}\mcite[25]{ИоффеТихомиров1974}
  In a \hyperref[def:norm]{normed} space \( X \), for any nonzero vector \( x \in X \) there exists a continuous functional \( x^* \in S_{X^*} \) such that \( \inprod {x^*} x = \norm x \). In other words, the duality \hyperref[def:duality_mapping]{mapping} is nonempty for any point.
\end{corollary}
\begin{proof}
  This follows from \fullref{thm:hahn_banach_implies_annihilator_nontrivial} by taking \( A \coloneqq B(x, \abs{x}) \) and \( L \coloneqq \{ 0 \} \) and then scaling the obtained functional.
\end{proof}

\begin{definition}\label{def:hyperplane_separation}\mimprovised
  We say that the sets \( A \) and \( B \) in a real or complex vector space are \term{separated} by the linear functional \( l \) if there exists a real number \( c \) such that, for every \( x \) in \( A \) and \( y \) in \( B \),
  \begin{equation}\label{eq:def:hyperplane_separation/normal}
    \real \inprod l x < c \leq \real \inprod l y.
  \end{equation}

  See \fullref{rem:linear_functionals_over_c} for a justification of only considering the real part of \( l \).

  The asymmetry in the inequality \eqref{eq:def:hyperplane_separation/normal} can be inverted by considering \( -l \) and \( -c \).

  We say that \( A \) and \( B \) are \term{strongly separated} by \( l \) if both inequalities in \eqref{eq:def:hyperplane_separation/normal} are strict:
  \begin{equation}\label{eq:def:hyperplane_separation/strong}
    \real \inprod l x < c < \real \inprod l y.
  \end{equation}

  We can regard \( l \) as a hyperplane as in \fullref{def:hyperplane/kernel}, which justifies the terminology \enquote{hyperplane separation}. It is more correct, however, to say that they are separated by the affine hyperplane \( l(x) + c \).

  In the real case, \eqref{eq:def:hyperplane_separation/strong} is equivalent to requiring that \( A \) and \( B \) are contained in opposite open half-spaces, while \eqref{eq:def:hyperplane_separation/normal} allows one of the half-spaces to be closed.
\end{definition}

\begin{theorem}[Hahn-Banach hyperplane separation theorem]\label{thm:hahn_banach_hyperplane_separation}\mcite[25]{ИоффеТихомиров1974}
  Fix a \hyperref[def:topological_vector_space]{topological vector space} \( X \). Let \( A, B \subseteq X \) be disjoint \hyperref[def:convex_hull]{convex} sets. If \( \int{A} \neq \varnothing \), there exists a continuous linear functional \hyperref[def:hyperplane_separation]{separating} \( A \) and \( B \).
\end{theorem}
