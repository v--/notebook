\subsection{The Hahn-Banach theorem}\label{subsec:hahn_banach}

The Hahn-Banach theorem is an important result that can be stated differently and in different levels of generality.

\begin{theorem}[Geometric Hahn-Banach theorem/Mazur's theorem]\label{thm:geometric_hahn_banach}\mcite[24]{Йоффе1974}
  Fix a \hyperref[def:topological_vector_space]{topological vector space} \( X \). Let \( A \subseteq X \) be an open \hyperref[def:convex_set]{convex} set and \( L \subseteq X \) be a subspace that is disjoint from \( A \). Then there exists a continuous linear functional \( x^* \in X^* \) such that
  \begin{equation*}
    \begin{array}{l}
      \real \inner{x^*} x > 0, x \in A \\
      \real \inner{x^*} x = 0, x \in L
    \end{array}
  \end{equation*}

  See \fullref{rem:linear_functionals_over_c} for a justification of only considering the real part of \( x^* \).
\end{theorem}

\begin{corollary}\label{thm:hahn_banach_implies_functionals_vanish_nowhere}\mcite[24]{Йоффе1974}
  The \hyperref[def:dual_vector_space]{dual} of a Hausdorff \hyperref[def:locally_convex_space]{locally convex space} \( X \) does not \hyperref[def:functions_vanish_nowhere]{vanish} at the nonzero vectors of \( X \).
\end{corollary}
\begin{proof}
  Fix a nonzero point \( x \in X \). The result follows from \fullref{thm:geometric_hahn_banach} with \( L \coloneqq \{ 0 \} \) and \( A \) -- any convex set containing \( x \) and not containing zero. Such a set \( A \) exists because the topology is Hausdorff and \( x \) has a neighborhood disjoint from any point in \( L \).
\end{proof}

\begin{corollary}\label{thm:hahn_banach_implies_annihilator_nontrivial}\mcite[25]{Йоффе1974}
  The \hyperref[def:vector_space_annihilator]{annihilator} of any proper subspace of a Hausdorff \hyperref[def:locally_convex_space]{locally convex space} contains nonzero elements.
\end{corollary}
\begin{proof}
  Denote the proper subspace by \( L \subsetneq X \). Fix \( x \in X \setminus L \) and let \( A \) be a convex neighborhood of \( x \) that is disjoint from \( L \). The result follows from \fullref{thm:geometric_hahn_banach}.
\end{proof}

\begin{corollary}\label{thm:hahn_banach_implies_duality_mapping_nonempty}\mcite[25]{Йоффе1974}
  In a \hyperref[def:norm]{normed} space \( X \), for any nonzero vector \( x \in X \) there exists a continuous functional \( x^* \in S_{X^*} \) such that \( \inner {x^*} x = \norm x \). In other words, the duality \hyperref[def:duality_mapping]{mapping} is nonempty for any point.
\end{corollary}
\begin{proof}
  This follows from \fullref{thm:hahn_banach_implies_annihilator_nontrivial} by taking \( A \coloneqq B(x, \abs{x}) \) and \( L \coloneqq \{ 0 \} \) and then scaling the obtained functional.
\end{proof}

\begin{theorem}[Hahn-Banach hyperplane separation theorem]\label{thm:hahn_banach_hyperplane_separation}\mcite[25]{Йоффе1974}
  Fix a \hyperref[def:topological_vector_space]{topological vector space} \( X \). Let \( A, B \subseteq X \) be disjoint \hyperref[def:convex_set]{convex} sets. If \( \int{A} \neq \varnothing \), there exists a continuous linear functional \hyperref[def:hyperplane_separation]{separating} \( A \) and \( B \).
\end{theorem}
