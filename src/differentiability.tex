\section{Nonsmooth analysis}\label{sec:nonsmooth_analysis}
\subsection{Differentiability}\label{subsec:differentiability}

Let \( X \) and \( Y \) be Hausdorff topological vector spaces\Tinyref{def:topological_vector_space}, let \( U \subseteq X \) be an open set.

\begin{definition}\label{def:derivatives}
  We fix a point \( x \in U \) and a direction \( h \in S_X \). We introduce a few definitions of derivatives. In all cases we say that the derivative (of the corresponding type) exists for \( f \) at \( x \) in the direction \( h \) and denote the derivative by either \( Df(x)(h) \) or \( f'(x)(h) \) (with accents for the different types of derivatives). The quotient under the limit sign is called a \Def{difference quotient} (of the corresponding type).

  The \Def{derivative function} is then defined as the function \( x \mapsto Df(x) \) that sends a vector \( x \) to a subset of \( \Hom(X, Y) \).

  \begin{defenum}
    \DItem{def:derivatives/classical} The classical \Def{two-sided derivative} is defined as
    \begin{equation*}
      [Df](x)(h) = f'(x)(h) \coloneqq \lim_{t \to 0} \frac {f(x + th) - f(x)} t
    \end{equation*}

    This definition is often too strict and so there exist a few generalizations.

    \DItem{def:derivatives/onesided}\cite[lemma 1.2]{Phelps1993} The one-sided (or right-hand) \Def{directional derivative} is defined as
    \begin{equation*}
      [D_+ f](x)(h) = f_+'(x)(h) \coloneqq \lim_{t \downarrow 0} \frac {f(x + th) - f(x)} t
    \end{equation*}

    It is also denoted as \( \partial^+ f(x)(h) \) in \cite[lemma 1.2]{Phelps1993}. We do not need to define a left-hand directional derivative because it would equal \( -f_+'(x)(-h) \).

    \DItem{def:derivatives/dini}\cite[definition 11.18]{Clarke2013} The upper (resp. lower) \Def{Dini derivative} is defined as
    \begin{align*}
      &[\Ol{D} f](x)(h) = \Ol{f'}(x)(h) &\coloneqq \limsup_{t \downarrow 0} \frac {f(x + th) - f(x)} t
      \\
      &[\Ul{D} f](x)(h) = \Ul{f'}(x)(h) &\coloneqq \liminf_{t \downarrow 0} \frac {f(x + th) - f(x)} t
    \end{align*}

    Dini derivatives are useful when the difference quotients are bounded but do not have a limit.

    \DItem{def:derivatives/clarke}\cite[section 10.1]{Clarke2013} The \Def{generalized Clarke derivative} is defined as
    \begin{equation*}
      [D^\circ f](x)(h)
      =
      f^\circ(x)(h)
      \coloneqq
      \limsup_{\substack{y \to x \\ t \downarrow 0}} \frac {f(y + th) - f(y)} t
      =
      \lim_{\delta \to 0} \sup_{\substack{y \in B(x, \delta) \\ t \in (0, \delta)}} \frac {f(y + th) - f(y)} t.
    \end{equation*}

    Refer to \cref{subsec:clarke_gradients} for their usefulness.
  \end{defenum}

  When \( X = \R \), the possible directions are \( h \in \{ -1, 1 \} \). We call \( D_+ f(x)(-1) \) the left derivative and \( D_+ f(x)(1) \) the right derivative of \( f \) at \( x \in \R \).
\end{definition}

\begin{definition}\label{def:differentiability}
  We fix a point \( x \in U \). We will now introduce several types of differentiability. Each is implied by the next one.

  \begin{defenum}
    \DItem{def:differentiability/one_sided} We do not introduce a special name for functions that have a one-sided derivative at \( x \) in the direction \( h \).

    \DItem{def:differentiability/two_sided}\cite[0.2.1]{Йоффе1974} If the two-sided directional derivative \( f'(x)(h) \) exists for all directions \( h \in S_X \), we call it the \Def{first variation} at \( x \) and denote the corresponding linear operator \( \delta f(x): X \to Y \).

    \DItem{def:differentiability/gateaux}\cite[definition 1.12]{Phelps1993} Let \( X \) be a Banach space. We say that \( f \) is \Def{Gateaux-differentiable} at \( x \) if there exists a continuous linear operator \( f'_G(x): X \to Y \), called the Gateaux derivative of \( f \) at \( x \), such that
    \begin{equation*}
      f'_G(x)(h) = \lim_{t \to 0} \frac {f(x + th) - f(x)} t.
    \end{equation*}

    The Gateaux derivative \( f'_G(x) \) exists precisely when the first variation \( \delta f(x) \) operator exists and is continuous. They are obviously equal.

    The Gateaux derivative is also denoted by \( df(x) \) in \cite[definition 1.12]{Phelps1993}.

    \DItem{def:differentiability/frechet}\cite[definition 1.12]{Phelps1993} Let \( X \) be a Banach space. We say that \( f \) is \Def{Frechet-differentiable} at \( x \) if there exists a continuous linear operator \( f'(x): X \to Y \), called the Frechet derivative of \( f \) at \( x \), such that for each \( \varepsilon > 0 \) there exists a radius \( \delta > 0 \) such that and for every direction \( h \in S_X \) we have
    \begin{equation*}
      \Norm{ \frac {f(x + th) - f(x)} t - f'(x)(h)} < \varepsilon \quad\forall t \in (0, \delta).
    \end{equation*}

    Note that for each \( \varepsilon > 0 \), Gateaux differentiability gives us a radius \( \delta_h > 0 \) such that
    \begin{equation*}
      \Norm{ \frac {f(x + th) - f(x)} t - f_G'(x)(h)} < \varepsilon \quad\forall t \in (0, \delta).
    \end{equation*}

    If the limit is uniform over \( h \in S_X \), i.e. if \( \sup_{h \in S_X} \delta_h < \infty \), then \( f \) is Frechet differentiable at \( x \) and \( f'(x) = f'_G(x) \).

    \DItem{def:differentiability/strong}\cite[33]{Dontchev2014} We say that \( f \) is \Def{strongly differentiable} at \( x \) if there exists a continuous linear operator \( f'(x): X \to Y \) such that
    \begin{equation*}
      \lim_{\substack{y \to x \\ z \to x}} \frac{f(y) - f(z) - f'(x)(y - z)} {\Norm{y - z}} = 0.
    \end{equation*}
  \end{defenum}
\end{definition}

\begin{example}[Weierstrass' nowhere differentiable function]\label{ex:weierstrass_nowhere_differentiable_function}\cite[\textnumero 271]{Фихтенгольц1968/2}
  Let $a \in (0, 1)$ and $b$ is a positive odd integer such that
  \begin{equation*}
    ab > 1 + \frac 3 2 \pi.
  \end{equation*}

  Define the function
  \begin{equation*}
    f(x) \coloneqq \sum_{k=0}^\infty a^k \cos(b^k \pi x).
  \end{equation*}

  \begin{figure}\label{ex:weierstrass_nowhere_differentiable_function/plot}
    \centering
    \begin{mplibcode}
      input metapost/plotting;
      u := 3cm;

      a := 0.9;
      b := 7;
      n := 4;

      vardef f_k(expr x, k) =
        pow(a, k) * cosd(pow(b, k) * pi * x)
      enddef;

      vardef f(expr x) =
        result := 0;

        for k = 1 upto n:
          result := result + f_k(x, k);
        endfor

        result / 2 % scale by 0.5 for the sake of visualization
      enddef;

      beginfig(2)
        drawarrow (-pi / 2, 0) scaled u -- (pi / 2, 0) scaled u;
        drawarrow (0, -pi / 10) scaled u -- (0, pi / 2) scaled u;

        draw path_of_plot(f, -pi / 2, pi / 2, 0.01, u);
      endfig;
    \end{mplibcode}
    \caption{Plot of the 4-th partial sum of the Weierstrass function with $a = 0.9$ and $b = 7$ from $-\pi$ to $\pi$.}
  \end{figure}

  Since \( \cos \) is bounded for real arguments and \( a \in (0, 1) \), each term is uniformly bounded by \( 1 \) and by \cref{thm:weierstrass_series_criterion}, \( f \) is continuous. However, it does not have a two-sided derivative\Tinyref{def:derivatives/classical} at any point. The proof of the latter is involved and will not be given here.
\end{example}

\begin{proposition}\label{thm:real_valued_differentiability}
  A real-valued function \( f: U \to \R \) is differentiable at \( x \) in the direction \( h \) if and only if \( \varphi(t) = f(x + th) \) is right-differentiable at \( 0 \).
\end{proposition}
\begin{proof}
  \begin{equation*}
    f_+'(x)(h) \coloneqq \lim_{t \downarrow 0} \frac {f(x + th) - f(x)} t = \varphi_+'(0)(1).
  \end{equation*}
\end{proof}
