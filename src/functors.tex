\subsection{Functors}\label{subsec:functors}

\begin{definition}\label{def:functor}\mcite[def. 1.2.1 \\ def. 1.2.10]{Leinster2016Basic}
  Fix some \hyperref[def:category]{categories} \( \cat{C} \) and \( \cat{D} \). A \term{functor} \( F: \cat{C} \to \cat{D} \) is a \hyperref[def:theory_of_graphs/quiver_homomorphism]{quiver homomorphism} between the underlying quivers that is compatible with composition and identities.

  Explicitly, a functor is a family of functions
  \begin{equation}\label{eq:def:functor_as_family_of_function}
    \begin{aligned}
      F_{\obj}:       &\obj(\cat{C}) \to \obj(\cat{D}) \\
      F_{\hom(X, Y)}: &\cat{C}(X, Y) \to \cat{D}(F_{\obj}(X), F_{\obj}(Y)),
    \end{aligned}
  \end{equation}
  where \( F_{\hom(X, Y)} \) is a distinct function for every pair of objects \( X \) and \( Y \).

  In practice, we usually define the functor as the set
  \begin{equation}\label{eq:def:functor_as_single_function}
    F \coloneqq F_{\obj} \cup \bigcup\set{ F_{\hom(X, Y)} \given A, Y \in \obj(\cat{C}) }
  \end{equation}

  Since the domains of all constituent functions are disjoint, \( F \) is again a total single-valued function. This allows us to justify the notation \( F(X) \) for objects and \( F(f) \) for morphisms.

  The definition of a functor additionally requires the following compatibility conditions to hold:
  \begin{thmenum}[series=def:functor]
    \thmitem[def:functor/CF1]{CF1} Functors must preserve identities, meaning that for any object \( X \in \cat{C} \) we must have
    \begin{equation}\label{eq:def:category/CF2}\tag{\logic{CF2}}
      F(\id_X) = \id_{F(X)}.
    \end{equation}

    \thmitem[def:functor/CF2]{CF2} Functors must preserve composition, meaning that for any pair of morphism \( f: X \to Y \) and \( g: Y \to C \) in \( \cat{C} \), we have
    \begin{equation}\label{eq:def:category/CF1}\tag{\logic{CF1}}
      F(g \bincirc f) = F(g) \bincirc F(f).
    \end{equation}
  \end{thmenum}

  \begin{thmenum}[resume=def:functor]
    \thmitem{def:functor/domain_and_codomain} We say that the category \( \cat{C} \) is the \term{domain} and \( \cat{D} \) --- the \term{codomain} of the functor \( F \). These are technically not the domain and codomain of \( F \) when regarded as a function, however it is consistent with \fullref{def:category_of_small_categories}.

    \thmitem{def:functor/endofunctor} Similarly to \fullref{def:multi_valued_function/endofunction} for functions, if the domain \( \cat{C} \) and codomain \( \cat{D} \) of a functor coincide, we say that it is an \term{endofunctor}.
  \end{thmenum}
\end{definition}
\begin{defproof}
  The definition \eqref{eq:def:functor_as_family_of_function} ensures that the quiver homomorphism conditions \eqref{eq:def:theory_of_graphs/quiver_homomorphism/head} and \eqref{eq:def:theory_of_graphs/quiver_homomorphism/tail} hold.

  Indeed, for any morphism \( f: X \to Y \) in \( \cat{C} \) we have
  \begin{equation*}
    F(\dom(f)) = F(X) = \dom(F(f)),
  \end{equation*}
  which implies \eqref{eq:def:theory_of_graphs/quiver_homomorphism/head}. We also have
  \begin{equation*}
    F(\co\dom(f)) = F(Y) = \co\dom(F(f)),
  \end{equation*}
  which implies \eqref{eq:def:theory_of_graphs/quiver_homomorphism/tail}.
\end{defproof}

\begin{remark}\label{rem:functor_size}
  It is possible that \( \cat{C} \) is \( \mscrU \)-small in the sense of \fullref{def:category_size}, but the \hyperref[def:functor]{functor} \( F \), as the set \eqref{eq:def:functor_as_single_function}, is not \( \mscrU \)-small in the sense of \fullref{def:large_and_small_sets}. Without using universes, we cannot prove the existence of any functor from the category of smalls sets to itself, for example.
\end{remark}

\begin{example}\label{ex:inclusion_functor}
  If \( \cat{D} \) is a \hyperref[def:subcategory]{subcategory} of \( \cat{C} \), then there is an \term{inclusion functor} \( \Iota: \cat{D} \to \cat{C} \), which every object and morphism of \( \cat{D} \) to itself in \( \cat{C} \).
\end{example}

\begin{example}\label{ex:constant_functor}
  Given a category \( \cat{C} \), for any object \( A \) in the nonempty category \( \cat{D} \), we can define the constant functor
  \begin{equation*}
    \begin{aligned}
      &F: \cat{C} \to \cat{D}, \\
      &F(X) \coloneqq A, \\
      &F(f: X \to Y) \coloneqq \id_A.
    \end{aligned}
  \end{equation*}
\end{example}

\begin{example}\label{ex:functors_in_set/unary}
  In \fullref{def:basic_set_operations}, we defined some operations on the category \hyperref[def:category_of_small_sets]{\( \cat{Set} \)} of small sets.

  \begin{itemize}
    \item The \hyperref[def:basic_set_operations/power_set]{power set} \( \pow: \cat{Set} \to \cat{Set} \) is a canonical example of an \hyperref[def:functor/endofunctor]{endofunctor}. Explicitly:
    \begin{equation*}
      \begin{aligned}
        &\pow: \cat{Set} \to \cat{Set}, \\
        &\pow(A) \coloneqq \set{ S \given S \subseteq A }, \\
        &\pow(f: A \to B) \coloneqq (S \mapsto f[S]). \\
      \end{aligned}
    \end{equation*}

    We must verify that it is indeed a functor. \ref{def:functor/CF2} is satisfied because
    \begin{equation*}
      \pow(g) \bincirc \pow(f) = (S \mapsto g[f[S]]) = \pow(g \bincirc f).
    \end{equation*}

    The condition \ref{df:functor/CF2} is also obviously satisfied.

    The nuance here is that we send every function \( f: A \to B \) to its \hyperref[def:multi_valued_function/set_value]{set value} \( f[S] \) of some subset \( S \) of \( A \).

    \item The \hyperref[def:basic_set_operations/union]{union} \( \bigcup \) and \hyperref[def:basic_set_operations/intersection]{intersection} \( \bigcap \) may seem to be good examples of endofunctors in \( \cat{Set} \). Unfortunately, there is no natural way to extend a morphism (function) \( f: A \to B \) to a morphism from \( \bigcup A \) to \( \bigcup B \) or \( \bigcap A \) to \( \bigcap B \).
  \end{itemize}

  Binary functors require an auxiliary definition --- \fullref{def:product_category}. We list these additional examples in \fullref{ex:functors_in_set/binary}.
\end{example}

\begin{definition}\label{def:product_category}\mcite[const. 1.1.11]{Leinster2016Basic}
  We define the \term{product category} \( \cat{C} \times \cat{D} \) of \( \cat{C} \) and \( \cat{D} \) as follows:

  \begin{itemize}
    \item The \hyperref[def:category/objects]{set of objects} is the \hyperref[def:cartesian_product]{Cartesian product}
    \begin{equation}\label{eq:def:product_category/objects}
      \obj(\cat{C} \times \cat{D}) \coloneqq \obj(\cat{C}) \times \obj(\cat{D}).
    \end{equation}

    \item The \hyperref[def:category/morphisms]{set of morphisms} from the pair of objects \( (A, X) \) to \( (B, Y) \) is the product
    \begin{equation}\label{eq:def:product_category/morphisms}
      (\cat{C} \times \cat{D})\parens[\Big]{ (A, X), (B, Y) } \coloneqq \cat{C}(A, B) \times \cat{C}(X, Y).
    \end{equation}

    \item The \hyperref[def:category/composition]{composition of the morphisms} \( (f, g) \) and \( (p, q) \) is the pairwise composition
    \begin{equation}\label{eq:def:product_category/composition}
      (f, g) \bincirc (p, q) \coloneqq (p \bincirc f, q \bincirc g).
    \end{equation}

    \item The \hyperref[def:category/identity]{identity morphism} of the pair \( (A, X) \) is simply the pair of identity morphisms \( (\id_A, \id_X) \).
  \end{itemize}
\end{definition}

\begin{example}\label{ex:functors_in_set/binary}
  We have discussed in \fullref{ex:functors_in_set/unary} endofunctors over the category \( \cat{Set} \) of small sets.

  \Fullref{def:product_category} allows us to consider some other set operations as functors from \( \cat{Set} \times \cat{Set} \) to \( \cat{Set} \):
  \begin{itemize}
    \item There is a canonical embedding given by
    \begin{equation*}
      \begin{aligned}
        &\iota: \cat{Set} \times \cat{Set} \to \cat{Set}, \\
        &\iota(A, B) \coloneqq (A, B), \\
        &\iota(f: A \to X, g: B \to Y) \coloneqq \parens[\Big]{ (x, y) \mapsto (f(x), g(y)) }.
      \end{aligned}
    \end{equation*}

    \item The family of all \hyperref[def:function_invertibility/bijective]{bijective functions} from one set to another is also a functor:
    \begin{equation*}
      \begin{aligned}
        &\op{iso}: \cat{Set} \times \cat{Set} \to \cat{Set}, \\
        &\op{iso}(A, B) \coloneqq \set{ \varphi: A \to B \given \varphi \T{is bijective} }, \\
        &\op{iso}(f: A \to X, g: B \to Y) \coloneqq (\varphi \mapsto g \bincirc \varphi \bincirc f^{-1}).
      \end{aligned}
    \end{equation*}

    The action of \( \op{iso} \) on functions can be expressed graphically as
    \begin{equation}\label{eq:ex:functors_in_set/binary/iso}
      \begin{aligned}
        \includegraphics[page=1]{figures/ex__functors_in_set__binary.pdf}
      \end{aligned}
    \end{equation}

    Then \ref{def:functor/CF2} can be seen to hold by inspecting the diagram:
    \begin{equation}\label{eq:ex:functors_in_set/binary/iso_composition}
      \begin{aligned}
        \includegraphics[page=2]{figures/ex__functors_in_set__binary.pdf}
      \end{aligned}
    \end{equation}

    This example also highlights that diagrams can be more understandable than writing out the morphisms explicitly.

    The other functor axiom \ref{def:functor/CF1} is obvious.

    \item Like the arbitrary union and intersection discussed in \fullref{ex:functors_in_set/unary}, the \hyperref[def:basic_set_operations/union]{binary union} \( A \cup B = \bigcup\set{ A, B } \) and \hyperref[def:basic_set_operations/intersection]{binary intersection} \( A \cap B = \bigcap\set{ A, B } \) of the sets \( A \) and \( B \) may seem like functors, however there is again to natural notion for how this would act on morphisms.

    We can, of course, regard them as functors, and one such definition for the binary union is
    \begin{equation*}
      \begin{aligned}
        &{\cup}: \cat{Set} \times \cat{Set} \to \cat{Set}, \\
        &{\cup} \thinspace (A, B) \coloneqq \set{ x \given x \in A \T{and} x \in B }, \\
        &{\cup} \thinspace (f: A \to X, g: B \to Y) \coloneqq (x, y) \mapsto f(x).
      \end{aligned}
    \end{equation*}

    This is not, however, a very useful definition, because it completely disregards the function \( g \).
  \end{itemize}
\end{example}

\begin{example}\label{ex:discrete_category}
  A \term{discrete category} is a category with no morphisms except for the identities. Clearly to any set there corresponds exactly one discrete category and vice versa.

  This can be expressed via functors. Denote by
  \begin{equation*}
    U: \ucat{Cat} \to \ucat{Set}
  \end{equation*}
  the forgetful functor that for any \( \mscrU \)-small category \( \cat{C} \) gives us its set of vertices \( \obj(\cat{C}) \). There is also a functor
  \begin{equation*}
    D: \ucat{Set} \to \ucat{Cat}
  \end{equation*}
  that for any \( \mscrU \)-small set \( A \) gives us the discrete category whose set of objects is \( A \).

  Furthermore, this is an adjunction of functors.
\end{example}

\begin{proposition}\label{thm:def:functor/properties}
  \hyperref[def:functor]{Functors} have the following basic properties:
  \begin{thmenum}
    \thmitem{thm:def:functor/properties/isomorphisms} Functors preserve \hyperref[def:morphism_invertibility/isomorphism]{isomorphisms}. That is, for every functor \( F: \cat{C} \to \cat{D} \) and every pair of objects \( A \) and \( B \) in \( \cat{C} \), from \( A \cong B \) it follows tha \( F(A) \cong F(B) \).
  \end{thmenum}
\end{proposition}
\begin{proof}
  \SubProofOf{thm:def:functor/properties/isomorphisms} Let \( f: A \to B \) be an isomorphism in \( \cat{C} \). Then
  \begin{equation}\label{eq:thm:def:functor/properties/isomorphisms}
    F(f^{-1}) \bincirc F(f)
    \reloset {\eqref{eq:def:category/CF2}} =
    F(f^{-1} \bincirc f)
    =
    F(\id_A)
    \reloset {\eqref{eq:def:category/CF1}} =
    \id_{F(A)},
  \end{equation}

  Hence, \( F(f^{-1}) \) is a \hyperref[morphism_invertibility/left_invertible]{left inverse} of \( F(f) \). The proof that \( F(f^{-1}) \) is a right inverse is analogous, hence \( F(f^{-1}) \) is a two-sided inverse of \( F(f) \). Therefore, \( F(f) \) is an isomorphism from \( A \) to \( B \).
\end{proof}

\begin{definition}\label{def:category_of_small_categories}
  Suppose that we are given a \hyperref[def:grothendieck_universe]{Grothendieck universe} \( \mscrU \), which is safe to assume to be the smallest suitable one as explained in \fullref{def:large_and_small_sets}.

  We denote the \hyperref[def:category]{category} of \( \mscrU \)-small \hyperref[def:category]{categories} by \( \ucat{Cat} \) or, if the universe is clear from the context, simply by \( \cat{Cat} \). See \fullref{def:category_size} for a further discussion of universes and categories.

  \begin{itemize}
    \item The \hyperref[def:category/objects]{set of objects} \( \obj(\cat{Cat}) \) is the set of all \( \mscrU \)-small categories.

    \item The \hyperref[def:category/morphisms]{set of morphisms} \( \cat{Cat}(X, Y) \) from \( X \) to \( Y \) is the set of all \hyperref[def:functor]{functors} from \( X \) to \( Y \).

    \item The \hyperref[def:category/composition]{composition of morphisms} is the \hyperref[def:multi_valued_function/composition]{function composition} of the functors regarded as the functions \eqref{eq:def:functor_as_single_function}. That is, the composition of \( F: \cat{C} \to \cat{D} \) and \( G: \cat{D} \to \cat{E} \) is the functor
    \begin{equation}\label{eq:def:category_of_small_categories/composition}
      \begin{aligned}
        &G \bincirc F: \cat{C} \to \cat{E}, \\
        &[G \bincirc F](X) \coloneqq G(F(X)), \\
        &[G \bincirc F](f) \coloneqq G(F(f)).
      \end{aligned}
    \end{equation}

    \item The \hyperref[def:category/identity]{identity morphism} on the category \( \cat{C} \) is the \term{identity functor}
    \begin{equation}\label{eq:def:category_of_small_categories/identity}
      \begin{aligned}
        &\id_{\cat{C}}: \cat{C} \to \cat{C}, \\
        &\id_{\cat{C}}(X) \coloneqq A, \\
        &\id_{\cat{C}}(f) \coloneqq f.
      \end{aligned}
    \end{equation}
  \end{itemize}
\end{definition}
\begin{defproof}
  To see that \( \ucat{Cat} \) is indeed a category, we verify the conditions \ref{def:category/C1} and \ref{def:category/C2}.

  \SubProofOf{def:category/C1} For every two \( \mscrU \)-small categories \( \cat{C} \) and \( \cat{D} \) and every functor \( F: \cat{C} \to \cat{D} \), for every object \( X \in \cat{C} \) we have
  \begin{equation*}
    [\id_{\cat{D}} \bincirc F](X)
    =
    \id_{\cat{D}}(F(X))
    =
    F(X)
    =
    F(\id_{\cat{C}}(X))
    =
    [F \bincirc \id_{\cat{C}}](X)
  \end{equation*}
  and analogously for morphisms.

  Therefore, \( \id_{\cat{C}} \) and \( \id_{\cat{D}} \) satisfy \eqref{eq:def:category/C1}.

  \SubProofOf{def:category/C2} Associativity of functor composition is inherited from the associativity of function composition.
\end{defproof}

\begin{proposition}\label{thm:category_of_small_categories_properites}
  We collect here important properties of the category \hyperref[def:category_of_small_categories]{\( \ucat{Cat} \)} of \( \mscrU \)-small categories. Most of them require forward references.
\end{proposition}

\begin{definition}\label{def:functor_image}
  The \term{image} of a functor \( F: \cat{C} \to \cat{D} \) is the \hyperref[def:quiver]{quiver} whose vertex set is
  \begin{equation*}
    V \coloneqq \set{ F(X) \given X \in \cat{C} }
  \end{equation*}
  and whose arc set is
  \begin{equation*}
    A \coloneqq \set{ F(f) \given A, Y \in \cat{C} \T{and} f \in \cat{C}(X, Y) }.
  \end{equation*}

  This quiver has no categorical structure --- it is merely a directed multigraph. As shown in \fullref{ex:functor_image_not_a_category}, imposing a categorical structure na\"ively may fail.
\end{definition}

\begin{example}\label{ex:functor_image_not_a_category}\mcite{MathSE:image_of_functor_is_not_a_category}
  \begin{figure}
    \hfill
    \includegraphics[page=1]{figures/ex__functor_image_not_a_category.pdf}
    \hfill
    \hfill
    \caption{A functor whose image is not a category.}\label{fig:ex:functor_image_not_a_category}
  \end{figure}

  Consider the functor \( F: \cat{C} \to \cat{D} \) from \cref{fig:ex:functor_image_not_a_category}.

  \begin{itemize}
    \item The solid arrows are the morphisms in \( \cat{C} \) and their images in \( F(\cat{C}) \).
    \item The squiggly arrows denote the action of the functor \( F \).
    \item The dotted arrow exists in \( \cat{D} \) as the composition of the other two arrows, however it is missing in the image \( F(\cat{C}) \). Thus, composition is not fully defined in \( F(\cat{C}) \), and \( F(\cat{C}) \) fails to be a category.
  \end{itemize}
\end{example}

\begin{remark}\label{rem:categorical_diagram_as_functor}
  Compared to \fullref{def:categorical_diagram}, a more \enquote{categorical} approach to defining diagrams is by using functors.

  Fix a category \( \cat{I} \), called an \term{index category}. A \term{diagram} in \( \cat{C} \) of shape \( \cat{I} \) is simply a functor \( D: \cat{I} \to \cat{C} \), whose domain is \( \cat{I} \).

  We do not really care about how the objects and morphisms in \( \cat{I} \) are labeled, hence we often use placeholder dots like in \eqref{eq:ex:quivers_as_functors/index/dots}.
\end{remark}

\begin{definition}\label{def:functor_invertibility}
  In connection with \fullref{def:morphism_invertibility} and \fullref{def:function_invertibility}, we introduce the following terminology:
  \begin{thmenum}
    \thmitem{def:functor_invertibility/injective_on_objects} The \hyperref[def:functor]{functor} \( F: \cat{C} \to \cat{D} \) is \term{injective on objects} if the \hyperref[def:multi_valued_function/restriction]{restriction}
    \begin{equation*}
      F\restr_{\obj(C)}: \obj(C) \to \obj(D)
    \end{equation*}
    is \hyperref[def:function_invertibility/injective]{injective}.

    That is, for every pair of objects \( X \) and \( Y \) in \( \cat{C} \), from \( F(X) = F(Y) \) it follows that \( X = Y \).

    If, instead, from \( F(X) \cong F(Y) \) it follows that \( X \cong Y \), we say that \( F \) if \term{essentially injective on objects}.

    \thmitem{def:functor_invertibility/injective_on_morphisms} The \hyperref[def:functor]{functor} \( F: \cat{C} \to \cat{D} \) is \term{injective on morphisms} if its restriction to the set
    \begin{equation*}
      \bigcup\set{ \cat{C}(X, Y) \given X, Y \in \obj(\cat{C}) }
    \end{equation*}
    of all morphisms is injective.

    That is, for every pair of morphisms \( f \) and \( g \) in \( \cat{C} \), from \( F(f) = F(g) \) it follows that \( f = g \). Note that if the morphisms are not parallel, we assume that they are not equal.

    \thmitem{def:functor_invertibility/faithful}\mcite[def. 1.2.16]{Leinster2016Basic} The functor \( F: \cat{C} \to \cat{D} \) is \term{faithful} if it is \hyperref[def:function_invertibility/injective]{injective} on \( \hom \)-sets, i.e. for all pairs of objects \( X \) and \( Y \) in \( \cat{C} \), the restriction of \( F \) to \( \cat{C}(X, Y) \) is an injective function.

    That is, for every pair of objects \( X \) and \( Y \) in \( \cat{C} \) and every pair of morphisms \( f \) and \( g \) in \( \cat{C}(X, Y) \), from \( F(f) = F(g) \) it follows that \( f = g \).

    See \fullref{thm:def:functor_invertibility/properties/injective_faithful} for how faithful functors relate to functors injective on objects or on morphisms.

    \thmitem{def:functor_invertibility/surjective_on_objects}\mcite[def. 1.3.17]{Leinster2016Basic} The functor \( F: \cat{C} \to \cat{D} \) is \term{surjective on objects} if the restriction
    \begin{equation*}
      F\restr_{\obj(C)}: \obj(C) \to \obj(D)
    \end{equation*}
    is \hyperref[def:function_invertibility/surjective]{surjective}.

    That is, for every object \( Y \) in \( \cat{D} \), there exists at least one object \( X \) in \( \cat{C} \) such that \( F(X) = Y \).

    If, instead, there exists at least one object \( X \in \cat{C} \) such that \( F(X) \cong Y \), we say that \( F \) is \term{essentially injective on objects}.

    \thmitem{def:functor_invertibility/surjective_on_morphisms} Similarly, \( F: \cat{C} \to \cat{D} \) is \term{surjective on morphisms} if its restriction to the set of all morphisms is surjective.

    That is, for every morphism \( g \) in \( \cat{D} \), there exists at least one morphism \( f \) in \( \cat{C} \) such that \( F(f) = g \).

    \thmitem{def:functor_invertibility/full}\mcite[def. 1.2.16]{Leinster2016Basic} The functor \( F: \cat{C} \to \cat{D} \) is \term{full} if it is surjective on \( \hom \)-sets, i.e. for all pairs of objects \( X \) and \( Y \) in \( \cat{C} \), the restriction of \( F \) to \( \cat{C}(X, Y) \) is a surjective function.

    That is, for every pair of objects \( X \) and \( Y \) in \( \cat{C} \) and every morphism \( g: F(X) \to F(Y) \) in \( \cat{D} \), there exists at least one morphism in \( f: X \to Y \) in \( \cat{C} \) such that \( F(f) = g \).

    \thmitem{def:functor_invertibility/fully_faithful} Finally, \( F: \cat{C} \to \cat{D} \) is \term{fully faithful} if it is both full and faithful.
  \end{thmenum}
\end{definition}

\begin{proposition}\label{thm:def:functor_invertibility/properties}
  \hyperref[def:functor]{Functors} have the following basic properties regarding their \hyperref[def:functor_invertibility]{invertibility}:

  \begin{thmenum}
    \thmitem{thm:def:functor_invertibility/properties/injective} A functor is \hyperref[def:functor_invertibility/injective_on_morphisms]{injective on morphisms} if and only if it is both \hyperref[def:functor_invertibility/injective_on_objects]{injective on objects} and \hyperref[def:functor_invertibility/faithful]{faithful}.

    \thmitem{thm:def:functor_invertibility/properties/surjective} A functor is \hyperref[def:functor_invertibility/surjective_on_morphisms]{surjective on morphisms} if and only if it is both \hyperref[def:functor_invertibility/surjective_on_objects]{surjective on objects} and \hyperref[def:functor_invertibility/full]{full}.

    \thmitem{thm:def:functor_invertibility/properties/full_subcategory} A \hyperref[def:subcategory]{subcategory} \( \cat{D} \) of \( \cat{C} \) is full in the sense of \fullref{def:subcategory} if and only if the \hyperref[ex:inclusion_functor]{inclusion functor} \( \Iota: \cat{D} \to \cat{C} \) is full in the sense of \fullref{def:functor_invertibility/full}.
  \end{thmenum}
\end{proposition}
\begin{proof}
  \SubProofOf{thm:def:functor_invertibility/properties/injective}
  \SufficiencySubProof* Let \( F: \cat{C} \to \cat{D} \) be injective on morphisms. It is trivially faithful since faithfulness is a more restrictive condition.

  To see that \( F \) is injective on objects, let \( X, Y \in \cat{C} \) and suppose that \( F(X) = F(Y) \). Then \( \id_{F(X)} = \id_{F(Y)} \) and
  \begin{equation*}
    F(\id_X)
    \reloset {\eqref{eq:def:category/CF2}} =
    \id_{F(X)}
    =
    \id_{F(Y)}
    \reloset {\eqref{eq:def:category/CF2}} =
    F(\id_Y).
  \end{equation*}

  Since \( F \) is injective on morphisms, it follows that \( \id_X = \id_Y \), hence \( X = Y \). Thus, \( F \) is injective on objects.

  \NecessitySubProof* Let \( F: \cat{C} \to \cat{D} \) be faithful and injective on objects. Let \( f: X \to Y \) and \( g: Z \to U \) be morphisms in \( \cat{C} \) such that \( F(f) = F(g) \).

  Then both \( F(f) \) and \( F(g) \) have the same domain \( F(X) = F(Z) \) and codomain \( F(Y) = F(U) \). Hence, since \( F \) is injective on objects, we have \( X = Z \) and \( Y = U \).

  Thus, \( f \) and \( g \) are both morphisms from \( X \) to \( Y \). Since \( F \) is also faithful, from \( F(f) = F(g) \) it follows that \( f = g \).

  Therefore, \( F \) is injective on morphisms.

  \SubProofOf{thm:def:functor_invertibility/properties/surjective}
  \SufficiencySubProof* Let \( F: \cat{C} \to \cat{D} \) be surjective on morphisms. It is trivially full since fullness is a more restrictive condition.

  To see that \( F \) is surjective on objects, let \( Z \in \cat{D} \). Then there exists some morphism \( f: X \to Y \) in \( \cat{C} \) such that \( F(f) = \id_Z \). We thus necessarily have \( F(X) = Z \) and \( F(Y) = Z \).

  \NecessitySubProof* Let \( F: \cat{C} \to \cat{D} \) be full and injective on objects. Let \( g: Z \to U \) be a morphism in \( \cat{D} \).

  Since \( F \) is surjective on objects, there exists preimages \( X \) of \( Z \) and \( Y \) of \( U \) under \( F \). Thus, \( g \in \cat{D}(F(X), F(Y)) \).

  Since \( F \) is also full, there exists some morphism \( f: X \to Y \) such that \( F(f) = g \).

  Therefore, \( F \) is surjective on morphisms.

  \SubProofOf{thm:def:functor_invertibility/properties/full_subcategory} Trivial.
\end{proof}

\begin{example}\label{ex:def:functor_invertibility}
  \hfill
  \begin{itemize}
    \item The power set functor described in \fullref{ex:functors_in_set/unary} is clearly \hyperref[def:functor_invertibility/injective_on_morphisms]{injective on morphisms}, hence by \fullref{thm:def:functor_invertibility/properties/injective}, it is also \hyperref[def:functor_invertibility/injective_on_objects]{injective on objects} and \hyperref[def:functor_invertibility/faithful]{faithful}.

    It is not full, nor surjective on objects.

    \item The functor \( \op{iso} \) described in \fullref{ex:functors_in_set/binary} is also injective on morphisms but neither full nor surjective on objects.

    \item The forgetful functor \( U: \ucat{Cat} \to \ucat{Set} \) discussed in \fullref{ex:discrete_category} is \hyperref[def:functor_invertibility/surjective_on_morphisms]{surjective on morphisms}, hence by \fullref{thm:def:functor_invertibility/properties/surjective}, it is also \hyperref[def:functor_invertibility/surjective_on_objects]{surjective on objects} and \hyperref[def:functor_invertibility/full]{full}.

    It is not faithful, nor injective on objects.
  \end{itemize}
\end{example}

\begin{definition}\label{def:natural_transformation}\mcite[def. 1.3.1]{Leinster2016Basic}
  Let \( F \) and \( G \) be parallel \hyperref[def:functor]{functors} from the category \( \cat{C} \) to \( \cat{D} \).

  A \term{natural transformation} \( \alpha \) from \( F \) to \( G \) is an \hyperref[def:tuple_and_cartesian_product/indexed_family]{indexed family} of
  \begin{equation}\label{eq:def:natural_transformation/family}
    \seq{ \alpha_X: F(X) \to G(X) }_{X \in \cat{C}}
  \end{equation}
  of morphisms in \( \cat{D} \) such that, for every morphism \( f: X \to Y \) in \( \cat{C} \), the following \hyperref[def:categorical_diagram]{diagram commutes}:
  \begin{equation}\label{eq:def:natural_transformation/diagram}
    \begin{aligned}
      \includegraphics[page=1]{figures/def__natural_transformation.pdf}
    \end{aligned}
  \end{equation}

  The morphisms \( \alpha_X \) are called the components of \( \alpha \). We denote natural transformations by \( \alpha: F \Rightarrow G \) and, when used in diagrams, by
  \begin{equation}\label{eq:def:natural_transformation/notation}
    \begin{aligned}
      \includegraphics[page=2]{figures/def__natural_transformation.pdf}
    \end{aligned}
  \end{equation}
\end{definition}

\begin{example}\label{ex:constant_functor_natural_transformation}\mcite{math3ma:natural_transformations}
  Let \( X \) and \( Y \) be objects in some category \( \cat{C} \). A \hyperref[def:natural_transformation]{natural transformation} between two \hyperref[ex:constant_functor]{constant functors} \( F_X \) and \( F_Y \) with a common domain \( \cat{I} \) is simply a morphism from \( X \) to \( Y \).

  Indeed, for every object \( A \in \cat{I} \), we have \( F_X(A) = X \) and \( F_Y(A) = Y \). Then, if \( \alpha: F_X \Rightarrow F_Y \) is any natural transformation, for every function \( f: A \to B \), the diagram \eqref{eq:def:natural_transformation/diagram} becomes
  \begin{equation}\label{eq:ex:constant_functor_natural_transformation}
    \begin{aligned}
      \includegraphics[page=1]{figures/ex__constant_functor_natural_transformation.pdf}
    \end{aligned}
  \end{equation}

  This diagram implies that \( \alpha_A = \alpha_B \) for any two objects \( A \) and \( B \) in \( \cat{I} \). Therefore, all components of \( \alpha \) are equal to some morphism in \( \cat{D}(X, Y) \).
\end{example}

\begin{example}\label{ex:quivers_as_functors}\mcite[exmpl. 1.3.46]{Perrone2019}
  In \fullref{def:quiver}, we have defined a quiver as a set \( V \) of vertices, a set \( A \) of arcs and two functions --- the head \( h: A \to V \)and tail \( t: A \to V \) of an arc.

  Now consider the following \hyperref[rem:categorical_diagram_as_functor]{index category} \( \cat{I}: \)
  \begin{equation}\label{eq:ex:quivers_as_functors/index/dots}
    \begin{aligned}
      \includegraphics[page=1]{figures/ex__quivers_as_functors.pdf}
    \end{aligned}
  \end{equation}

  For the sake of readability, we will give the following explicit labels in this category:
  \begin{equation}\label{eq:ex:quivers_as_functors/index/annotated}
    \begin{aligned}
      \includegraphics[page=2]{figures/ex__quivers_as_functors.pdf}
    \end{aligned}
  \end{equation}

  A quiver can then be defined as a functor \( Q: \cat{I} \to \ucat{Set} \) to the category \hyperref[def:category_of_small_sets]{\( \ucat{Set} \)} of \( \mscrU \)-small sets (for a fixed \hyperref[def:grothendieck_universe]{Grothendieck universe} \( \mscrU \)).

  A \hyperref[def:natural_transformation]{natural transformation} from the quiver \( Q: \cat{I} \to \ucat{Set} \) to \( R: \cat{I} \to \ucat{Set} \) is then a pair of functions \( f_V: Q(V) \to R(V) \) and \( f_A: Q(A) \to R(A) \) such that the following \hyperref[def:categorical_diagram]{diagrams commute}:
  \begin{equation}\label{eq:ex:quivers_as_functors/index/diagram}
    \begin{aligned}
      \includegraphics[page=3]{figures/ex__quivers_as_functors.pdf}
      \quad\quad\quad\quad
      \includegraphics[page=4]{figures/ex__quivers_as_functors.pdf}
    \end{aligned}
  \end{equation}

  See \fullref{ex:isomorphism_of_categories} for how these functors related to quivers as defined in \fullref{def:quiver}.
\end{example}

\begin{proposition}\label{thm:natural_isomorphism}\mcite{math3ma:natural_transformations}
  Let \( F \) and \( G \) be parallel \hyperref[def:functor]{functors} from the category \( \cat{C} \) to \( \cat{D} \). Then any family of isomorphisms of the form \eqref{eq:def:natural_transformation/family} is a natural transformation.

  We call such transformations \term{natural isomorphisms}.

  These are precisely the isomorphisms in the corresponding \hyperref[def:functor_category]{functor category} \( [\cat{C}, \cat{D}] \) --- see \fullref{thm:def:functor_category/properties/natural_isomorphism}.
\end{proposition}
\begin{proof}
  Fix a morphism \( f: X \to Y \) in \( \cat{C} \). Then
  \begin{equation}\label{eq:thm:natural_isomorphism/diagram}
    \begin{aligned}
      \includegraphics[page=1]{figures/thm__natural_isomorphism.pdf}
    \end{aligned}
  \end{equation}

  Therefore, the diagram \eqref{eq:def:natural_transformation/diagram} commutes.
\end{proof}

\begin{remark}\label{rem:natural_transformations_into_set}
  Let \( \cat{C} \) be an arbitrary \( \mscrU \)-small category. A \hyperref[def:natural_transformation]{natural transformation} \( \alpha \) from \( F: \cat{C} \to \ucat{Set} \) to \( G: \cat{C} \to \ucat{Set} \) is then a family of functions
  \begin{equation*}
    \seq{ \alpha_X: F(X) \to G(X) }_{X \in \cat{C}}.
  \end{equation*}

  Suppose that for every two objects \( X \) and \( Y \) in \( \cat{C} \), the functions \( \alpha_X \) and \( \alpha_Y \) agree on \( F(X) \cap F(Y) \). This is automatically satisfied in \( F(X) \) and \( F(Y) \) are disjoint whenever \( X \neq Y \).

  We can then take the set-theoretic union of \( \alpha \) to obtain the function
  \begin{equation*}
    \bigcup_{X \in \cat{C}} \alpha_X: \bigcup\set{ F(X) \given X \in \cat{C} } \to \bigcup\set{ G(X) \given X \in \cat{C} }.
  \end{equation*}

  Both the domain and codomain are sets as a consequence of \ref{def:grothendieck_universe/union}, therefore the function is well-defined in the universe \( \mscrU \). Denote it on \( \Alpha \) for brevity.

  An advantage of this is that we can define a natural transformation to be a function on a general enough set and then prove that its restrictions satisfy \eqref{eq:def:natural_transformation/diagram}.

  For example, consider the power set functor \( \pow: \ucat{Set} \to \ucat{Set} \) discussed in \fullref{ex:functors_in_set/unary}. The \hyperref[def:multi_valued_function/identity]{identity function} \( \id_\mscrU \) is then a natural transformation from the identity functor \( \id_{\ucat{Set}} \) to \( \pow \).

  Another natural transformation between the same functors is the singleton set operation \( \Sigma \) on sets defined as \( A \mapsto \set{ A } \). Note that, in this context, \( \Sigma \) operates not on the sets \( \id_{\ucat{Set}}(X) \) and \( \pow(X) \), but on their members. The diagram \eqref{eq:def:natural_transformation/diagram} becomes
  \begin{equation}\label{eq:rem:natural_transformations_into_set}
    \begin{aligned}
      \includegraphics[page=1]{figures/rem__natural_transformations_into_set.pdf}
    \end{aligned}
  \end{equation}

  This diagram commutes because, for every function \( f: X \to Y \) and every \( x \in X \), we have
  \begin{equation*}
    f[\set{ x }] = \set{ f(x) }.
  \end{equation*}
\end{remark}

\begin{definition}\label{def:functor_category}
  Let \( \cat{C} \) and \( \cat{D} \) be arbitrary \hyperref[def:category]{categories}. The \term{functor category} \( [\cat{C}, \cat{D}] \), also denoted as \( \cat{D}^{\cat{C}} \), is defined as follows:

  \begin{itemize}
    \item The \hyperref[def:category/objects]{set of objects} \( \obj([\cat{C}, \cat{D}]) \) is the set of all functors from \( \cat{C} \) to \( \cat{D} \).

    \item The \hyperref[def:category/morphisms]{set of morphisms} \( [\cat{C}, \cat{D}](F, G) \) from \( F \) to \( G \) is the set of all \hyperref[def:natural_transformation]{natural transformations} from \( F \) to \( G \).

    \item The \hyperref[def:category/composition]{composition of the morphisms} \( \alpha: F \Rightarrow G \) and \( \beta: G \Rightarrow H \) is the natural transformation \( \beta \bincirc \alpha: F \Rightarrow H \) defined in terms of componentwise morphism composition, i.e.
    \begin{equation}\label{eq:def:functor_category/composition}
      (\beta \bincirc \alpha)_X \coloneqq \beta_X \bincirc \alpha_X.
    \end{equation}

    \item The \hyperref[def:category/identity]{identity morphism} on the functor \( F: \cat{C} \to \cat{D} \) is the \term{identity natural transformation} \( \id_F: F \Rightarrow F \) with components
    \begin{equation}\label{eq:def:functor_category/identity}
      (\id_F)_X \coloneqq \underbrace{\id_{F(X)}}_{F(\id_X)}
    \end{equation}
  \end{itemize}
\end{definition}
\begin{defproof}
  Just to verify that the composition \( \beta \bincirc \alpha \) defined in \eqref{eq:def:functor_category/composition} is indeed a natural transformation from \( F \) to \( H \), note that the following diagram trivially commutes:
  \begin{equation}\label{eq:ex:quivers_as_functors/index/diagram}
    \begin{aligned}
      \includegraphics[page=1]{figures/def__functor_category.pdf}
    \end{aligned}
  \end{equation}

  Now, to see that \( [\cat{C}, \cat{D}] \) is indeed a category, we verify the conditions \ref{def:category/C1} and \ref{def:category/C2}, which are in turn inherited from the same conditions on the categories \( \cat{C} \) and \( \cat{D} \).

  \SubProofOf{def:category/C1} For every two functors \( F, G: \cat{C} \to \cat{D} \) and natural transformation \( \alpha: F \Rightarrow G \), for every object \( X \in \cat{C} \) we have
  \begin{equation*}
    \id_{G(X)} \bincirc \alpha_X
    \reloset{\eqref{def:category/C1}} =
    \alpha_X
    \reloset{\eqref{def:category/C1}} =
    \alpha_X \bincirc \id_{F(X)}
  \end{equation*}

  Therefore,
  \begin{equation*}
    \id_G \bincirc \alpha = \alpha = \alpha \bincirc \id_F
  \end{equation*}
  and, after generalizing, we obtain that \eqref{eq:def:category/C1} holds in \( [\cat{C}, \cat{D}] \).

  \SubProofOf{def:category/C2} For any quadruple \( F \), \( G \), \( H \) and \( T \) of functors from \( \cat{C} \) to \( \cat{D} \) and for every combination of natural transformations \( \alpha: F \Rightarrow G \), \( \beta: G \Rightarrow H \) and \( \gamma: H \Rightarrow T \), for every object \( X \in \cat{C} \) we have
  \begin{equation*}
    (\gamma_X \bincirc \beta_X) \bincirc \alpha_X
    \reloset{\eqref{def:category/C2}} =
    \gamma_X \bincirc (\beta_X \bincirc \alpha_X).
  \end{equation*}

  Therefore, after generalizing, we obtain that \eqref{eq:def:category/C2} holds in \( [\cat{C}, \cat{D}] \).
\end{defproof}

\begin{definition}\label{def:composition_of_natural_transformation_and_functor}
  Suppose that we have the functors and natural transformations
  \begin{equation}\label{eq:def:composition_of_natural_transformation_and_functor/diagram}
    \begin{aligned}
      \includegraphics[page=1]{figures/def__composition_of_natural_transformation_and_functor.pdf}
    \end{aligned}
  \end{equation}

  We define the \term{precomposition} natural transformation \( \alpha \) with the functor \( P \) to be the natural transformation
  \begin{equation}\label{eq:def:composition_of_natural_transformation_and_functor/pre}
    \begin{aligned}
      &\alpha \bincirc P: F \bincirc P \Rightarrow G \bincirc P, \\
      &(\alpha \bincirc P)_X \coloneqq \alpha_{P(X)}, \quad\quad\quad X \in \cat{A}.
    \end{aligned}
  \end{equation}

  Dually, but not symmetrically, we define the composition of \( \alpha \) and \( H \) as the natural transformation
  \begin{equation}\label{eq:def:composition_of_natural_transformation_and_functor/post}
    \begin{aligned}
      &H \bincirc \alpha: H \bincirc F \Rightarrow H \bincirc G, \\
      &(H \bincirc \alpha)_Y \coloneqq H(\alpha_Y), \quad\quad\quad Y \in \cat{B}.
    \end{aligned}
  \end{equation}
\end{definition}

\begin{remark}\label{rem:functor_category_size}
  If \( \cat{C} \) and \( \cat{D} \) are \( \mscrU \)-large categories in the sense of \fullref{def:category_size}, it is impossible to even construct the \hyperref[def:functor_category]{functor category} \( [\cat{C}, \cat{D}] \). This is the main motivation for the \hyperref[def:axiom_of_universes]{axiom of universes}, which is discussed in \fullref{def:large_and_small_sets} and, in relation to category theory, in \fullref{def:category_size}.
\end{remark}

\begin{example}\label{ex:isomorphism_of_categories}
  In \fullref{ex:quivers_as_functors}, we defined \hyperref[def:quiver]{quivers} as functors from a certain index category \( \cat{I} \) to \( \ucat{Set} \) (for a fixed \hyperref[def:grothendieck_universe]{Grothendieck universe} \( \mscrU \)).

  There is then an obvious correspondence between quivers as objects of \hyperref[def:category_of_small_quivers]{\( \ucat{Quiv} \)}, defined in \fullref{def:quiver}, and quivers as objects in the \hyperref[def:functor_category]{functor category} \( [\cat{I}, \ucat{Set}] \), defined in \fullref{ex:quivers_as_functors}. Indeed, given any functor \( Q: \cat{I} \to \ucat{Set} \), the quadruple
  \begin{equation*}
    \parens[\Big]{ Q(V), Q(A), Q(h), Q(t) }
  \end{equation*}
  is a quiver in the sense of \fullref{def:quiver}.

  No object in \( \ucat{Quiv} \) is formally equal to any object in \( [\cat{I}, \ucat{Set}] \) in the sense of \hyperref[def:zfc]{\logic{ZFC}}. They are, however, equivalent, as shown above, and this can be formalized by stating that the two categories are isomorphic, in the sense of \fullref{def:morphism_invertibility/isomorphism}, as objects of the category \( \ucat[\mscrV]{Cat} \), where \( \mscrV \) is a Grothendieck universe that strictly contains \( \mscrU \). We have already defined this isomorphism explicitly.

  This is an example of \term{isomorphism of categories}. In practice, if two categories are not so obviously identical, we are usually better served by \term{equivalences of categories} defined in \fullref{def:equivalence_of_categories}.
\end{example}

\begin{proposition}\label{thm:def:functor_category/properties}
  \hyperref[def:functor_category]{Functor categories} have the following basic properties:
  \begin{thmenum}
    \thmitem{thm:def:functor_category/properties/natural_isomorphism} A \hyperref[def:natural_transformation]{natural transformation} \( \alpha: F \Rightarrow G \) in the category \( [\cat{C}, \cat{D}] \) is an isomorphism if and only if every component of \( \alpha \) is an isomorphism in \( \cat{D} \).

    Note that, by \fullref{thm:natural_isomorphism}, it is sufficient for \( \alpha \) to be a family of morphisms of the form \eqref{eq:def:natural_transformation/family}.
  \end{thmenum}
\end{proposition}
\begin{proof}
  \SubProofOf{thm:def:functor_category/properties/natural_isomorphism}
  \SufficiencySubProof* Let \( \alpha: F \Rightarrow G \) be an isomorphism in \( [\cat{C}, \cat{D}] \). Then there exists some natural transformation \( \beta: G \Rightarrow F \) such that
  \begin{equation*}
    \beta \bincirc \alpha = \id_F \quad\T{and}\quad \alpha \bincirc \beta = \id_G.
  \end{equation*}

  For every object \( X \) in \( \cat{C} \), the morphism \( \alpha_X: F(X) \to G(X) \) composed with \( \beta_X: G(X) \to F(X) \) is
  \begin{equation*}
    \beta_X \bincirc \alpha_X = \id_{F(X)}.
  \end{equation*}

  Therefore, \( \alpha_X \) is left-invertible. Analogously,
  \begin{equation*}
    \alpha_X \bincirc \beta_X = \id_{F(X)}
  \end{equation*}
  and hence \( \alpha_X \) is right-invertible.

  Therefore, for every object \( X \) in \( \cat{C} \), the morphism \( \alpha_X \) is fully invertible, i.e. an isomorphism.

  \NecessitySubProof* Let \( \alpha: F \Rightarrow G \) be a natural transformation, all of which components are isomorphisms. Define the transformation \( \beta: G \Rightarrow F \) by choosing, for each object \( X \) in \( \cat{C} \), an inverse \( \beta_X \) for \( \alpha_X \).

  These components are unique by \fullref{thm:def:morphism_invertibility/properties/left_and_right}, hence we do not even need any choice.
\end{proof}
