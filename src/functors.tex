\subsection{Functors}\label{subsec:functors}

\begin{definition}\label{def:functor}\mcite[def. 1.2.1 \\ def. 1.2.10]{Leinster2016Basic}
  Fix some \hyperref[def:category]{categories} \( \cat{C} \) and \( \cat{D} \). A \term{functor} \( F: \cat{C} \to \cat{D} \) is a \hyperref[eq:def:category_of_small_quivers/homomorphism]{quiver homomorphism} between the underlying quivers that is compatible with composition and identities.

  Explicitly, a functor is a family of functions
  \begin{equation}\label{eq:def:functor_as_family_of_function}
    \begin{aligned}
      F_{\obj}:       &\obj(\cat{C}) \to \obj(\cat{D}) \\
      F_{\hom(A, B)}: &\cat{C}(A, B) \to \cat{D}(F_{\obj}(A), F_{\obj}(B)),
    \end{aligned}
  \end{equation}
  where \( F_{\hom(A, B)} \) is a distinct function for every pair of objects \( A \) and \( B \).

  In practice, we usually define the functor as the set
  \begin{equation}\label{eq:def:functor_as_single_function}
    F \coloneqq F_{\obj} \cup \bigcup\set{ F_{\hom(A, B)} \given A, B \in \obj(\cat{C}) }
  \end{equation}

  Since the domains of all constituent functions are disjoint, \( F \) is again a total single-valued function. This allows us to justify the notation \( F(A) \) for objects and \( F(f) \) for morphisms.

  \begin{thmenum}[resume=def:functor]
    \thmitem{def:functor/domain_and_codomain} We say that the category \( \cat{C} \) is the \term{domain} and \( \cat{D} \) --- the \term{codomain} of the functor \( F \). These are technically not the domain and codomain of \( F \) when regarded as a function, however it is consistent with \fullref{def:category_of_small_categories}.

    \thmitem{def:functor/endofunctor} Similarly to \fullref{def:multi_valued_function/endofunction} for functions, if the domain \( \cat{C} \) and codomain \( \cat{D} \) of a functor coincide, we say that it is an \term{endofunctor}.
  \end{thmenum}

  The definition of a functor additionally requires the following compatibility conditions to hold:
  \begin{thmenum}[series=def:functor]
    \thmitem[def:functor/CF1]{CF1} Functors must preserve identities, meaning that for any object \( A \in \cat{C} \) the following equality must hold:
    \begin{equation}\label{eq:def:functor/CF1}\tag{\logic{CF1}}
      F(\id_A) = \id_{F(A)}.
    \end{equation}

    \thmitem[def:functor/CF2]{CF2} Functors must preserve composition, meaning that for any pair of morphism \( f: A \to B \) and \( g: B \to C \) in \( \cat{C} \),
    \begin{equation}\label{eq:def:functor/CF2}\tag{\logic{CF2}}
      F(g \bincirc f) = F(g) \bincirc F(f).
    \end{equation}
  \end{thmenum}
\end{definition}
\begin{defproof}
  The definition \eqref{eq:def:functor_as_family_of_function} ensures that the quiver homomorphism conditions \eqref{eq:def:category_of_small_quivers/homomorphism/head} and \eqref{eq:def:category_of_small_quivers/homomorphism/tail} hold.

  Indeed, for any morphism \( f: A \to B \) in \( \cat{C} \) we have
  \begin{equation*}
    F(\dom(f)) = F(A) = \dom(F(f)),
  \end{equation*}
  which implies \eqref{eq:def:category_of_small_quivers/homomorphism/head}. We also have
  \begin{equation*}
    F(\co\dom(f)) = F(B) = \co\dom(F(f)),
  \end{equation*}
  which implies \eqref{eq:def:category_of_small_quivers/homomorphism/tail}.
\end{defproof}

\begin{remark}\label{rem:functor_size}
  It is possible that \( \cat{C} \) is \( \mscrU \)-small in the sense of \fullref{def:category_size}, but the \hyperref[def:functor]{functor} \( F \), as the set \eqref{eq:def:functor_as_single_function}, is not \( \mscrU \)-small in the sense of \fullref{def:large_and_small_sets}. Without using universes, we cannot prove the existence of any functor from the category of smalls sets to itself, for example.
\end{remark}

\begin{example}\label{ex:unary_functors_in_set}
  In \fullref{def:basic_set_operations}, we defined some operations on the category \hyperref[def:category_of_small_sets]{\( \cat{Set} \)} of small sets.

  \begin{thmenum}
    \thmitem{ex:unary_functors_in_set/power} The \hyperref[def:basic_set_operations/power_set]{power set} \( \pow: \cat{Set} \to \cat{Set} \) is a canonical example of an \hyperref[def:functor/endofunctor]{endofunctor}. Explicitly:
    \begin{equation*}
      \begin{aligned}
        &\pow: \cat{Set} \to \cat{Set}, \\
        &\pow(A) \coloneqq \set{ S \given S \subseteq A }, \\
        &\pow(f: A \to B) \coloneqq (S \mapsto f[S]). \\
      \end{aligned}
    \end{equation*}

    We must verify that it is indeed a functor. \ref{def:functor/CF2} is satisfied because
    \begin{equation*}
      \pow(g) \bincirc \pow(f) = (S \mapsto g[f[S]]) = \pow(g \bincirc f).
    \end{equation*}

    The condition \ref{def:functor/CF2} is also obviously satisfied.

    The nuance here is that we send every function \( f: A \to B \) to its \hyperref[def:multi_valued_function/set_value]{set value} \( f[S] \) of some subset \( S \) of \( A \).

    \thmitem{ex:unary_functors_in_set/union} The \hyperref[def:basic_set_operations/union]{union} \( \bigcup \) and \hyperref[def:basic_set_operations/intersection]{intersection} \( \bigcap \) may seem to be good examples of endofunctors in \( \cat{Set} \). Unfortunately, there is no natural way to extend a morphism (function) \( f: A \to B \) to a morphism from \( \bigcup A \) to \( \bigcup B \) or \( \bigcap A \) to \( \bigcap B \).
  \end{thmenum}
\end{example}

\begin{definition}\label{def:subcategory}\mcite[def. 1.2.18]{Leinster2016Basic}
  We call the category \( \cat{D} \) a \term{subcategory} of \( \cat{C} \) if the following hold:
  \begin{itemize}
    \item The underlying quiver \( U(\cat{D}) \) is a \hyperref[def:quiver/submodel]{subquiver} of \( U(\cat{C}) \). That is, every object in \( \cat{D} \) is an object in \( \cat{C} \) and every morphism in \( \cat{D} \) is a morphism in \( \cat{C} \).
    \item Composition and identity in \( \cat{D} \) are \hyperref[def:multi_valued_function/restriction]{restrictions} of composition and identity in \( \cat{C} \).
  \end{itemize}

  \begin{thmenum}
    \thmitem{def:subcategory/inclusion} For every subcategory there exists an \term{inclusion functor} \( \Iota: \cat{D} \to \cat{C} \), which sends every object and morphism of \( \cat{D} \) to itself in \( \cat{C} \).

    \thmitem{def:subcategory/full} We say that \( \cat{D} \) is a \term{full subcategory} if the underlying quiver \( U(\cat{C}) \) is a \hyperref[def:quiver/submodel]{full subquiver}. That is, in case \( \cat{D}(A, B) = \cat{C}(A, B) \) for every pair of objects \( A \) and \( B \) of \( \cat{D} \).

    By \fullref{thm:def:functor_invertibility/full_subcategory}, this is equivalent to the inclusion functor being \hyperref[def:functor_invertibility/full]{full}.

    \thmitem{def:subcategory/induced} Every \hyperref[rem:family_of_sets]{family} \( \mscrD \) of objects in \( \cat{C} \) induces a full subcategory \( \cat{D} \) of \( \cat{C} \), whose objects are those of \( \mscrD \) and whose morphisms are restricted to those whose domain and codomain are both in \( \mscrD \).
  \end{thmenum}
\end{definition}

\begin{remark}\label{rem:contravariant_functor}\mcite[def. 1.2.10]{Leinster2016Basic}
  We can invert the order of composition in \ref{def:functor/CF2} in the definition of a functor given in \fullref{def:functor}.
  \begin{thmenum}
    \thmitem[rem:contravariant_functor/CF2]{CF2\Textprime} We can replace \ref{def:functor/CF2} with
    \begin{equation}\label{eq:rem:contravariant_functor/CF2}\tag{\logic{CF2\Textprime}}
      F(g \bincirc f) = F(f) \bincirc F(g).
    \end{equation}
  \end{thmenum}

  This also requires some other straightforward modifications to the definition of a functor.

  A functor that satisfies \ref{rem:contravariant_functor/CF2} rather than \ref{def:functor/CF2} is called \term{contravariant}. In this context, a functor satisfying \ref{def:functor/CF2} is called \term{covariant}.

  Fortunately, a contravariant functor from \( \cat{C}^{\opcat} \) to \( \cat{D} \) is identical to a covariant functor from \( \cat{C} \) to \( \cat{D} \). Therefore, there is no formal difference between the two concepts.

  The usage of the terms are entirely dictated by context. Unless necessary, we will avoid speaking about contravariant functors to avoid confusion. Some examples where this terminology may be useful are \fullref{def:opposite_functor}, \fullref{def:hom_functor/unary} and \fullref{ex:dual_space_contravariant_functor}.
\end{remark}

\begin{example}\label{ex:dual_space_contravariant_functor}
  We can try to na\"ively define a functor that assigns to a \hyperref[def:vector_space]{vector space} its \hyperref[def:dual_vector_space]{algebraic dual}:
  \begin{equation*}
    \begin{aligned}
      &F: \cat{Vect_\BbbK} \to \cat{Vect_\BbbK}, \\
      &F(V) \coloneqq V^*, \\
      &F(f: V \to W) \coloneqq (\varphi: W \to \BbbK \mapsto \varphi \bincirc f).
    \end{aligned}
  \end{equation*}

  Unfortunately, \( F(f) \) is supposed to be a morphism from \( V^* \) to \( W^* \), but is actually a morphism from \( W^* \) to \( V^* \). This makes \( F \) a \hyperref[rem:contravariant_functor]{contravariant functor} or, equivalently, a functor from \( \cat{Vect_\BbbK}^{\opcat} \) to \( \cat{Vect_\BbbK} \).
\end{example}

\begin{definition}\label{def:discrete_category}
  A \term{discrete category} is a category with no morphisms except for the identities. Clearly to any set there corresponds exactly one discrete category and vice versa.
\end{definition}

\begin{example}\label{ex:discrete_category_adjunction}
  Denote by
  \begin{equation*}
    U: \cat{Cat} \to \cat{Set}
  \end{equation*}
  the forgetful functor that for any small category \( \cat{C} \) gives us its set of objects \( \obj(\cat{C}) \). There is also a functor
  \begin{equation*}
    D: \cat{Set} \to \cat{Cat}
  \end{equation*}
  that for any small set \( A \) gives us the \hyperref[def:discrete_category]{discrete category} whose set of objects is \( A \).

  This is actually an \hyperref[def:category_adjunction]{adjunction} --- see \fullref{ex:def:category_adjunction/set_cat}.
\end{example}

\begin{proposition}\label{thm:def:functor}
  \hyperref[def:functor]{Functors} have the following basic properties:
  \begin{thmenum}
    \thmitem{thm:def:functor/half_inverses} Functors preserve inverses. For every functor \( F: \cat{C} \to \cat{D} \) and every morphism \( f: A \to B \) in \( \cat{C} \) with a right inverse \( g: B \to A \), \( F(f) \) is a right inverse of \( F(g) \). Similarly, if \( g \) is a left inverse of \( f \), then \( F(g) \) is a left inverse of \( F(f) \).

    \thmitem{thm:def:functor/inverses} For every functor \( F: \cat{C} \to \cat{D} \) and every isomorphism \( f: A \to B \) in \( \cat{C} \),
    \begin{equation}\label{eq:thm:def:functor/inverses}
      [F(f)]^{-1} = F(f^{-1}).
    \end{equation}

    \thmitem{thm:def:functor/isomorphisms} Functors preserve \hyperref[def:morphism_invertibility/isomorphism]{isomorphisms}. That is, for every functor \( F: \cat{C} \to \cat{D} \), if \( f: A \to B \) is an isomorphism in \( \cat{C} \), \( F(f) \) is an isomorphism in \( \cat{D} \).

    Consequently, for every pair of objects \( A \) and \( B \) in \( \cat{C} \), from \( A \cong B \) it follows that \( F(A) \cong F(B) \).

    The converse sometimes also holds --- see \fullref{thm:def:functor_invertibility/fully_faithful_reflects_invertible}.
  \end{thmenum}
\end{proposition}
\begin{proof}
  \SubProofOf{thm:def:functor/half_inverses} Let \( f: A \to B \) be a right inverse of \( g: B \to A \) in \( \cat{C} \). Then
  \begin{equation*}
    F(g) \bincirc F(f)
    \reloset {\eqref{eq:def:functor/CF2}} =
    F(g \bincirc f)
    =
    F(\id_A)
    \reloset {\eqref{eq:def:functor/CF1}} =
    \id_{F(A)}.
  \end{equation*}

  Thus, \( F(f) \) is a right inverse of \( F(g) \). Since \( g \) is a left inverse of \( f \), automatically \( F(g) \) is a left inverse of \( F(f) \).

  \SubProofOf{thm:def:functor/inverses} If \( f^{-1} \) is a left inverse of \( f \), by \fullref{thm:def:functor/half_inverses} we have that \( F(f^{-1}) \) is a left inverse of \( F(f) \). But \( F(f^{-1}) \) is also a right inverse, and again by \fullref{thm:def:functor/half_inverses} \( F(g) \) is a right inverse of \( F(f^{-1}) \).

  Therefore, \( F(f^{-1}) \) is a two-sided inverse of \( F(f) \). By \fullref{thm:def:morphism_invertibility/at_most_one_inverse}, it is the only two-sided inverse, hence
  \begin{equation*}
    [F(f)]^{-1} = F(f^{-1}).
  \end{equation*}

  \SubProofOf{thm:def:functor/isomorphisms} Follows from \fullref{thm:def:functor/inverses}.
\end{proof}

\begin{definition}\label{def:category_of_small_categories}
  Suppose that we are given a \hyperref[def:grothendieck_universe]{Grothendieck universe} \( \mscrU \), which is safe to assume to be the smallest suitable one as explained in \fullref{def:large_and_small_sets}.

  We denote the \hyperref[def:category]{category} of \( \mscrU \)-small \hyperref[def:category]{categories} by \( \ucat{Cat} \) or, if the universe is clear from the context, simply by \( \cat{Cat} \). See \fullref{def:category_size} for a further discussion of universes and categories.

  \begin{itemize}
    \item The \hyperref[def:category/objects]{set of objects} \( \obj(\cat{Cat}) \) is the set of all \( \mscrU \)-small categories.

    \item The \hyperref[def:category/morphisms]{set of morphisms} \( \cat{Cat}(A, B) \) from \( A \) to \( B \) is the set of all \hyperref[def:functor]{functors} from \( A \) to \( B \).

    \item The \hyperref[def:category/composition]{composition of morphisms} is the \hyperref[def:multi_valued_function/composition]{function composition} of the functors regarded as the functions \eqref{eq:def:functor_as_single_function}. That is, the composition of \( F: \cat{C} \to \cat{D} \) and \( G: \cat{D} \to \cat{E} \) is the functor
    \begin{equation}\label{eq:def:category_of_small_categories/composition}
      \begin{aligned}
        &[G \bincirc F]: \cat{C} \to \cat{E}, \\
        &[G \bincirc F](A) \coloneqq G(F(A)), \\
        &[G \bincirc F](f) \coloneqq G(F(f)).
      \end{aligned}
    \end{equation}

    \item The \hyperref[def:category/identity]{identity morphism} on the category \( \cat{C} \) is the \term{identity functor}
    \begin{equation}\label{eq:def:category_of_small_categories/identity}
      \begin{aligned}
        &\id_{\cat{C}}: \cat{C} \to \cat{C}, \\
        &\id_{\cat{C}}(A) \coloneqq A, \\
        &\id_{\cat{C}}(f) \coloneqq f.
      \end{aligned}
    \end{equation}
  \end{itemize}
\end{definition}
\begin{defproof}
  To see that \( \ucat{Cat} \) is indeed a category, we verify the conditions \ref{def:category/C1} and \ref{def:category/C2}.

  \SubProofOf{def:category/C1} For every two \( \mscrU \)-small categories \( \cat{C} \) and \( \cat{D} \) and every functor \( F: \cat{C} \to \cat{D} \), for every object \( A \in \cat{C} \) we have
  \begin{equation*}
    [\id_{\cat{D}} \bincirc F](A)
    =
    \id_{\cat{D}}(F(A))
    =
    F(A)
    =
    F(\id_{\cat{C}}(A))
    =
    [F \bincirc \id_{\cat{C}}](A)
  \end{equation*}
  and analogously for morphisms.

  Therefore, \( \id_{\cat{C}} \) and \( \id_{\cat{D}} \) satisfy \eqref{eq:def:category/C1}.

  \SubProofOf{def:category/C2} Associativity of functor composition is inherited from the associativity of function composition.
\end{defproof}

\begin{definition}\label{def:universal_categories}
  For any \hyperref[def:grothendieck_universe]{Grothendieck universe} \( \mscrU \), the \hyperref[def:category_of_small_categories]{category of \( \mscrU \)-small categories} \( \ucat{Cat} \) has an initial and a terminal object.

  Similarly to how we use the \hyperref[def:ordinal]{ordinals} \( 0 \) and \( 1 \) to denote the initial and terminal object in the category of sets, we denote the initial category by \( \cat{0} \) and the final category by \( \cat{1} \). Note that the final category is only unique up to an isomorphism. They are identical, however, for all universes \( \mscrU \).

  These categories are precisely the \hyperref[def:discrete_category]{discrete categories} induced by the ordinals \( 0 \) and \( 1 \) as described in \fullref{thm:order_category_isomorphism}.
\end{definition}

\begin{definition}\label{def:opposite_functor}\mcite{nLab:opposite_category}
  The \term{opposite} functor of \( F: \cat{C} \to \cat{D} \) is the functor
  \begin{equation*}
    \begin{aligned}
      &F^{\opcat}: \cat{C}^{\opcat} \to \cat{D}^{\opcat} \\
      &F^{\opcat}(A) \coloneqq A \\
      &F^{\opcat}(f^{\opcat}: B \to A) \coloneqq [F(f: A \to B)]^{\opcat}.
    \end{aligned}
  \end{equation*}

  For the composition of functors, we then have
  \begin{equation}\label{eq:def:opposite_functor/composition}
    [G \bincirc F]^{\opcat} = G^{\opcat} \bincirc F^{\opcat}.
  \end{equation}

  This is somewhat in contrast to the general practice of inverting morphisms when taking opposites. Thus, for any \hyperref[def:grothendieck_universe]{Grothendieck universe} \( \mscrU \), we have the \hyperref[rem:contravariant_functor]{contravariant} \term{oppositization functor}
  \begin{equation*}
    (\anon*)^{\opcat}: \ucat{Cat}^{\opcat} \to \ucat{Cat}.
  \end{equation*}

  As an \hyperref[def:multi_valued_function/endofunction]{endofunction} on \( \obj(\ucat{Cat}) \), the oppositization functor is clearly an \hyperref[def:set_with_involution]{involution}.

  Dual functors also arise naturally in \fullref{thm:opposite_of_functor_category}.
\end{definition}

\begin{definition}\label{def:functor_image}
  The \term{image} of a functor \( F: \cat{C} \to \cat{D} \) is the \hyperref[def:quiver]{quiver} whose vertex set is
  \begin{equation*}
    V \coloneqq \set{ F(A) \given A \in \cat{C} }
  \end{equation*}
  and whose arc set is
  \begin{equation*}
    A \coloneqq \set{ F(f) \given A, B \in \cat{C} \T{and} f \in \cat{C}(A, B) }.
  \end{equation*}

  This quiver has no categorical structure --- it is merely a directed multigraph. As shown in \fullref{ex:functor_image_not_a_category}, imposing a categorical structure na\"ively may fail.
\end{definition}

\begin{example}\label{ex:functor_image_not_a_category}\mcite{MathSE:image_of_functor_is_not_a_category}
  \begin{figure}
    \hfill
    \includegraphics[page=1]{output/ex__functor_image_not_a_category.pdf}
    \hfill
    \hfill
    \caption{A functor whose image is not a category.}\label{fig:ex:functor_image_not_a_category}
  \end{figure}

  Consider the functor \( F: \cat{C} \to \cat{D} \) from \cref{fig:ex:functor_image_not_a_category}.

  \begin{itemize}
    \item The solid arrows are the morphisms in \( \cat{C} \) and their images in \( F(\cat{C}) \).
    \item The dashed arrows denote the action of the functor \( F \).
    \item The dotted arrow exists in \( \cat{D} \) as the composition of the other two arrows, however it is missing in the image \( F(\cat{C}) \). Thus, composition is not fully defined in \( F(\cat{C}) \), and \( F(\cat{C}) \) fails to be a category.
  \end{itemize}
\end{example}

\begin{definition}\label{def:categorical_diagram}
  Fix a category \( \cat{I} \), called an \term{index category}. A \term{diagram} in \( \cat{C} \) of shape \( \cat{I} \) is simply a functor \( D: \cat{I} \to \cat{C} \), whose domain is \( \cat{I} \). We sometimes identify a diagram functor with its image \( D(\cat{I}) \).

  It is often convenient to draw graphically the \hyperref[def:quiver_geometric_realization]{geometric realizations} of the \hyperref[def:quiver]{quiver} \( D(\cat{I}) \). An established convention is to allow multiple vertices representing the same object, which can be achieved formally by actually adjoining new vertices to the quiver and labeling them as in \fullref{def:weighted_set}. Other established conventions for drawing diagrams include not drawing identity morphisms and adding various visual aids. In this regard, categorical diagrams correspond to the everyday sense of the word \enquote{diagram}.

  We say that the diagram \( D \) over \( \cat{C} \) \term{commutes} if, whenever \( p = (f_1, \ldots, f_n) \) and \( q = (g_1, \ldots, g_m) \) are two \hyperref[def:quiver_path/directed]{directed paths} in \( D(\cat{I}) \) with identical endpoints and either \( n > 1 \) or \( m > 1 \), then
  \begin{equation*}
    f_n \bincirc f_{n-1} \bincirc \cdots \bincirc f_2 \bincirc f_1
    =
    g_m \bincirc g_{m-1} \bincirc \cdots \bincirc g_2 \bincirc g_1.
  \end{equation*}

  We do not really care about how the objects and morphisms in \( \cat{I} \) are labeled, hence we often use placeholder dots like in \eqref{eq:ex:quivers_as_functors/index/dots}.

  The requirement that the one of the paths is nontrivial, however, is crucial in \fullref{def:equalizers}.
\end{definition}

\begin{remark}\label{rem:inverting_isomorphisms_may_preserve_commutativity}
  Inverting isomorphisms in a \hyperref[def:categorical_diagram]{commutative diagram} may or may not preserve commutativity.

  If \( p = (f_1, \ldots, f_n) \) and \( q = (g_1, \ldots, g_m) \) are two paths in a commutative diagram, and if \( f_1 \) is invertible, then obviously
  \begin{equation*}
    f_n \bincirc \cdots \bincirc f_1 = g_m \bincirc \cdots \bincirc g_1
  \end{equation*}
  if and only if
  \begin{equation*}
    f_n \bincirc \cdots \bincirc f_2 = g_m \bincirc \cdots \bincirc g_1 \bincirc f_1
  \end{equation*}
  and similarly if \( f_n \) is invertible.

  On the other hand, consider \hyperref[def:ordinal]{ordinals} in \( \cat{Set} \). Denote by \( \iota \) the inclusion maps and by \( f: \omega^2 \to \omega \) the bijective map from \fullref{thm:omega_equinumerous_with_omega_squared}. Then the following diagram commutes:
  \begin{equation}\label{eq:rem:inverting_isomorphisms_may_preserve_commutativity/ordinals_commuting}
    \begin{aligned}
      \includegraphics[page=1]{output/rem__inverting_isomorphisms_may_preserve_commutativity.pdf}
    \end{aligned}
  \end{equation}
  but the following does not:
  \begin{equation}\label{eq:rem:inverting_isomorphisms_may_preserve_commutativity/ordinals_not_commuting}
    \begin{aligned}
      \includegraphics[page=2]{output/rem__inverting_isomorphisms_may_preserve_commutativity.pdf}
    \end{aligned}
  \end{equation}
\end{remark}

\begin{definition}\label{def:functor_invertibility}
  In connection with \fullref{def:morphism_invertibility} and \fullref{def:function_invertibility}, we introduce the following terminology:
  \begin{thmenum}
    \thmitem{def:functor_invertibility/injective_on_objects} The \hyperref[def:functor]{functor} \( F: \cat{C} \to \cat{D} \) is \term{injective on objects} if the \hyperref[def:multi_valued_function/restriction]{restriction}
    \begin{equation*}
      F\restr_{\obj(C)}: \obj(C) \to \obj(D)
    \end{equation*}
    is \hyperref[def:function_invertibility/injective]{injective}.

    That is, for every pair of objects \( A \) and \( B \) in \( \cat{C} \), from \( F(A) = F(B) \) it follows that \( A = B \).

    If, instead, from \( F(A) \cong F(B) \) it follows that \( A \cong B \), we say that \( F \) if \term{essentially injective on objects}.

    \thmitem{def:functor_invertibility/injective_on_morphisms} The \hyperref[def:functor]{functor} \( F: \cat{C} \to \cat{D} \) is \term{injective on morphisms} if its restriction to the set
    \begin{equation*}
      \bigcup\set{ \cat{C}(A, B) \given A, B \in \obj(\cat{C}) }
    \end{equation*}
    of all morphisms is injective.

    That is, for every pair of morphisms \( f \) and \( g \) in \( \cat{C} \), from \( F(f) = F(g) \) it follows that \( f = g \). Note that if the morphisms are not parallel, we assume that they are not equal.

    \thmitem{def:functor_invertibility/faithful}\mcite[def. 1.2.16]{Leinster2016Basic} The functor \( F: \cat{C} \to \cat{D} \) is \term{faithful} if it is \hyperref[def:function_invertibility/injective]{injective} on \( \hom \)-sets, i.e. for all pairs of objects \( A \) and \( B \) in \( \cat{C} \), the restriction of \( F \) to \( \cat{C}(A, B) \) is an injective function.

    That is, for every pair of objects \( A \) and \( B \) in \( \cat{C} \) and every pair of morphisms \( f \) and \( g \) in \( \cat{C}(A, B) \), from \( F(f) = F(g) \) it follows that \( f = g \).

    See \fullref{thm:def:functor_invertibility/injective} for how faithful functors relate to functors injective on objects or on morphisms.

    \thmitem{def:functor_invertibility/surjective_on_objects}\mcite[def. 1.3.17]{Leinster2016Basic} The functor \( F: \cat{C} \to \cat{D} \) is \term{surjective on objects} if the restriction
    \begin{equation*}
      F\restr_{\obj(C)}: \obj(C) \to \obj(D)
    \end{equation*}
    is \hyperref[def:function_invertibility/surjective]{surjective}.

    That is, for every object \( B \) in \( \cat{D} \), there exists at least one object \( A \) in \( \cat{C} \) such that \( F(A) = B \).

    If, instead, there exists at least one object \( A \in \cat{C} \) such that \( F(A) \cong B \), we say that \( F \) is \term{essentially surjective on objects}.

    \thmitem{def:functor_invertibility/surjective_on_morphisms} Similarly, \( F: \cat{C} \to \cat{D} \) is \term{surjective on morphisms} if its restriction to the set of all morphisms is surjective.

    That is, for every morphism \( g \) in \( \cat{D} \), there exists at least one morphism \( f \) in \( \cat{C} \) such that \( F(f) = g \).

    \thmitem{def:functor_invertibility/full}\mcite[def. 1.2.16]{Leinster2016Basic} The functor \( F: \cat{C} \to \cat{D} \) is \term{full} if it is surjective on \( \hom \)-sets, i.e. for all pairs of objects \( A \) and \( B \) in \( \cat{C} \), the restriction of \( F \) to \( \cat{C}(A, B) \) is a surjective function.

    That is, for every pair of objects \( A \) and \( B \) in \( \cat{C} \) and every morphism \( g: F(A) \to F(B) \) in \( \cat{D} \), there exists at least one morphism in \( f: A \to B \) in \( \cat{C} \) such that \( F(f) = g \).

    \thmitem{def:functor_invertibility/fully_faithful} Finally, \( F: \cat{C} \to \cat{D} \) is \term{fully faithful} if it is both full and faithful.
  \end{thmenum}
\end{definition}

\begin{proposition}\label{thm:commutative_diagrams_preserved_and_reflected}
  Functors preserve commutative diagrams and faithful functors also reflect commutative diagrams.

  More precisely, let \( \cat{C} \) be an arbitrary category, let \( D \) be a diagram in \( \cat{C} \), and let \( p = (A, f_1, \ldots, f_n) \) and \( q = (A, g_1, \ldots, g_m) \) be two \hyperref[def:quiver_path/directed]{directed paths} with the same endpoints in \( D \).

  For any functor \( F: \cat{C} \to \cat{D} \), if
  \begin{equation}\label{eq:thm:commutative_diagrams_preserved_and_reflected/source}
    f_n \bincirc \cdots \bincirc \bincirc f_1 = g_m \bincirc \cdots \bincirc \bincirc g_1,
  \end{equation}
  then
  \begin{equation}\label{eq:thm:commutative_diagrams_preserved_and_reflected/image}
    F(f_n) \bincirc \cdots \bincirc \bincirc F(f_1) = F(g_m) \bincirc \cdots \bincirc \bincirc F(g_1),
  \end{equation}

  Conversely, if \( F \) is faithful, then \eqref{eq:thm:commutative_diagrams_preserved_and_reflected/image} implies \eqref{eq:thm:commutative_diagrams_preserved_and_reflected/source}.
\end{proposition}
\begin{proof}
  Functors preserve composition by \ref{eq:def:functor/CF2}, hence \eqref{eq:thm:commutative_diagrams_preserved_and_reflected/image} follows from \eqref{eq:thm:commutative_diagrams_preserved_and_reflected/source} directly.

  Now suppose that \eqref{eq:thm:commutative_diagrams_preserved_and_reflected/image} holds for a faithful functor \( F \). \ref{eq:def:functor/CF2} allows us to reduce \eqref{eq:thm:commutative_diagrams_preserved_and_reflected/image} to
  \begin{equation*}
    F(f_n \bincirc \cdots \bincirc \bincirc f_1) = F(g_m \bincirc \cdots \bincirc \bincirc g_1).
  \end{equation*}

  Then, by injectivity of \( F \) on the morphism set \( \cat{C}(\dom(f_1), \co\dom(f_1)) \), \eqref{eq:thm:commutative_diagrams_preserved_and_reflected/source} holds.
\end{proof}

\begin{proposition}\label{thm:def:functor_invertibility}
  \hyperref[def:functor]{Functors} have the following basic properties regarding their \hyperref[def:functor_invertibility]{invertibility}:

  \begin{thmenum}
    \thmitem{thm:def:functor_invertibility/injective} A functor is \hyperref[def:functor_invertibility/injective_on_morphisms]{injective on morphisms} if and only if it is both \hyperref[def:functor_invertibility/injective_on_objects]{injective on objects} and \hyperref[def:functor_invertibility/faithful]{faithful}.

    \thmitem{thm:def:functor_invertibility/surjective} A functor is \hyperref[def:functor_invertibility/surjective_on_morphisms]{surjective on morphisms} if and only if it is both \hyperref[def:functor_invertibility/surjective_on_objects]{surjective on objects} and \hyperref[def:functor_invertibility/full]{full}.

    \thmitem{thm:def:functor_invertibility/full_subcategory} A \hyperref[def:subcategory]{subcategory} \( \cat{D} \) of \( \cat{C} \) is full in the sense of \fullref{def:subcategory} if and only if the \hyperref[def:subcategory]{inclusion functor} \( \Iota: \cat{D} \to \cat{C} \) is full in the sense of \fullref{def:functor_invertibility/full}.

    \thmitem{thm:def:functor_invertibility/preserves_inverses} Any functor preserves \hyperref[def:morphism_invertibility/left]{left}, \hyperref[def:morphism_invertibility/right]{right inverses} and \hyperref[def:morphism_invertibility/isomorphism]{two-sided inverses}.

    \thmitem{thm:def:functor_invertibility/faithful_reflects_composition} \hyperref[def:functor_invertibility/faithful]{Faithful} functors reflect composition. That is, for every functor \( F: \cat{C} \to \cat{D} \), if the following diagram commutes:
    \begin{equation}\label{eq:thm:def:functor_invertibility/faithful_reflects_composition/image}
      \begin{aligned}
        \includegraphics[page=1]{output/thm__def__functor_invertibility__properties.pdf}
      \end{aligned}
    \end{equation}
    then the following diagram are identities:
    \begin{equation}\label{eq:thm:def:functor_invertibility/faithful_reflects_composition/source}
      \begin{aligned}
        \includegraphics[page=2]{output/thm__def__functor_invertibility__properties.pdf}
      \end{aligned}
    \end{equation}

    \thmitem{thm:def:functor_invertibility/faithful_reflects_cancellative} A \hyperref[def:functor_invertibility/faithful]{faithful} functor reflects monomorphisms and epimorphisms. That is, for every functor \( F: \cat{C} \to \cat{D} \) and morphism \( f: A \to B \) in \( \cat{C} \), if \( F(f) \) is a monomorphism (resp. epimorphism), so is \( f \).

    \thmitem{thm:def:functor_invertibility/fully_faithful_reflects_identities} A \hyperref[def:functor_invertibility/fully_faithful]{fully faithful} functor reflects identities. That is, for every functor \( F: \cat{C} \to \cat{D} \) and endomorphism \( f: A \to A \) in \( \cat{C} \), if \( F(f) = \id_{F(A)} \), then \( f = \id_A \).

    \thmitem{thm:def:functor_invertibility/fully_faithful_reflects_invertible} A \hyperref[def:functor_invertibility/fully_faithful]{fully faithful} functor reflects split monomorphisms and split epimorphisms.

    That is, for every functor \( F: \cat{C} \to \cat{D} \) and morphism \( f: A \to B \) in \( \cat{C} \), if \( F(f) \) is a split monomorphism (resp. split epimorphism or isomorphism), so is \( f \).

    \thmitem{thm:def:functor_invertibility/isomorphism} A functor between \( \mscrU \)-small categories that is both injective and surjective on morphisms is itself an isomorphism in \( \ucat{Cat} \).
  \end{thmenum}
\end{proposition}
\begin{proof}
  \SubProofOf{thm:def:functor_invertibility/injective}
  \SufficiencySubProof* Let \( F: \cat{C} \to \cat{D} \) be injective on morphisms. It is trivially faithful since faithfulness is a more restrictive condition.

  To see that \( F \) is injective on objects, let \( A, B \in \cat{C} \) and suppose that \( F(A) = F(B) \). Then \( \id_{F(A)} = \id_{F(B)} \) and
  \begin{equation*}
    F(\id_A)
    \reloset {\eqref{eq:def:functor/CF2}} =
    \id_{F(A)}
    =
    \id_{F(B)}
    \reloset {\eqref{eq:def:functor/CF2}} =
    F(\id_B).
  \end{equation*}

  Since \( F \) is injective on morphisms, it follows that \( \id_A = \id_B \), hence \( A = B \). Thus, \( F \) is injective on objects.

  \NecessitySubProof* Let \( F: \cat{C} \to \cat{D} \) be faithful and injective on objects. Let \( f: A \to B \) and \( g: C \to D \) be morphisms in \( \cat{C} \) such that \( F(f) = F(g) \).

  Then both \( F(f) \) and \( F(g) \) have the same domain \( F(A) = F(C) \) and codomain \( F(B) = F(D) \). Hence, since \( F \) is injective on objects, we have \( A = C \) and \( B = D \).

  Thus, \( f \) and \( g \) are both morphisms from \( A \) to \( B \). Since \( F \) is also faithful, from \( F(f) = F(g) \) it follows that \( f = g \).

  Therefore, \( F \) is injective on morphisms.

  \SubProofOf{thm:def:functor_invertibility/surjective}
  \SufficiencySubProof* Let \( F: \cat{C} \to \cat{D} \) be surjective on morphisms. It is trivially full since fullness is a more restrictive condition.

  To see that \( F \) is surjective on objects, let \( C \in \cat{D} \). Then there exists some morphism \( f: A \to B \) in \( \cat{C} \) such that \( F(f) = \id_Z \). We thus necessarily have \( F(A) = C \) and \( F(B) = C \).

  \NecessitySubProof* Let \( F: \cat{C} \to \cat{D} \) be full and injective on objects. Let \( g: C \to D \) be a morphism in \( \cat{D} \).

  Since \( F \) is surjective on objects, there exists preimages \( A \) of \( C \) and \( B \) of \( D \) under \( F \). Thus, \( g \in \cat{D}(F(A), F(B)) \).

  Since \( F \) is also full, there exists some morphism \( f: A \to B \) such that \( F(f) = g \).

  Therefore, \( F \) is surjective on morphisms.

  \SubProofOf{thm:def:functor_invertibility/full_subcategory} Trivial.

  \SubProofOf{thm:def:functor_invertibility/preserves_inverses} Let \( g: B \to A \) be a left inverse of \( f: A \to B \). Then
  \begin{equation*}
    F(g) \bincirc F(f)
    \reloset {\eqref{eq:def:functor/CF2}} =
    F(g \bincirc f)
    =
    F(\id_A)
    \reloset {\eqref{eq:def:functor/CF1}} =
    \id_{F(A)}.
  \end{equation*}

  The case of right inverses is similar.

  \SubProofOf{thm:def:functor_invertibility/faithful_reflects_composition} Suppose that \eqref{eq:thm:def:functor_invertibility/faithful_reflects_composition/image} commutes. Then, since \( F \) is faithful and thus injective on the morphism set \( \cat{C}(A, C) \), the equality \( F(g \bincirc f) = F(g) \bincirc F(f) = F(h) \) implies that \( g \bincirc f = h \). Hence, \eqref{eq:thm:def:functor_invertibility/faithful_reflects_composition/source} also commutes.

  \SubProofOf{thm:def:functor_invertibility/faithful_reflects_cancellative} Let \( F(g) \) be a monomorphism and let \( f_1, f_2: A \to B \) be parallel morphisms such that
  \begin{equation*}
    g \bincirc f_1 = g \bincirc f_2.
  \end{equation*}

  Then, since \( F(g) \) is a monomorphism, we have that \( F(f_1) = F(f_2) \). Since \( F \) is faithful, the restriction \( F\restr_{C(A, B)} \) is injective, and \( f_1 = f_2 \).

  The proof when \( F(g) \) is an epimorphism is analogous.

  \SubProofOf{thm:def:functor_invertibility/fully_faithful_reflects_identities} If \( F: \cat{C} \to \cat{D} \) is fully faithful, for every object \( A \) in \( \cat{C} \), the identity morphism \( \id_{F(A)} \) has a unique preimage under \( F \). By \ref{def:functor/CF1}, this preimage can only be \( \id_A \).

  \SubProofOf{thm:def:functor_invertibility/fully_faithful_reflects_invertible} Let \( q \) be a left inverse of \( F(f) \). Since \( F \) is fully faithful, there exists a unique morphism \( g: B \to A \) such that \( F(g) = q \).

  Since
  \begin{equation*}
    F(g) \bincirc F(f) = \id_{F(A)},
  \end{equation*}
  by \fullref{thm:def:functor_invertibility/fully_faithful_reflects_identities} we have
  \begin{equation*}
    g \bincirc f = \id_A.
  \end{equation*}

  Therefore, \( g \) is a left inverse of \( F(f) \).

  The proof for right inverses follows from \fullref{thm:def:morphism_invertibility/inverse_interchanges}.

  From \fullref{thm:def:morphism_invertibility/left_and_right} it follows that if \( F(f) \) is an isomorphism, so is \( f \).

  \SubProofOf{thm:def:functor_invertibility/isomorphism} If \( F \) is both injective and surjective on morphisms, it is also injective and surjective on objects and hence, as a function, is bijective. Therefore, it is both left and right invertible as a consequence of \fullref{thm:function_invertibility_categorical/fully_invertible}.
\end{proof}

\begin{example}\label{ex:def:functor_invertibility}
  \hfill
  \begin{thmenum}
    \thmitem{ex:def:functor_invertibility/power} The power set functor described in \fullref{ex:unary_functors_in_set} is clearly \hyperref[def:functor_invertibility/injective_on_morphisms]{injective on morphisms}, hence by \fullref{thm:def:functor_invertibility/injective}, it is also \hyperref[def:functor_invertibility/injective_on_objects]{injective on objects} and \hyperref[def:functor_invertibility/faithful]{faithful}.

    It is not full, nor surjective on objects.

    \thmitem{ex:def:functor_invertibility/cat_to_set} The forgetful functor \( D: \ucat{Cat} \to \ucat{Set} \) discussed in \fullref{def:discrete_category} is \hyperref[def:functor_invertibility/surjective_on_morphisms]{surjective on morphisms}, hence by \fullref{thm:def:functor_invertibility/surjective}, it is also \hyperref[def:functor_invertibility/surjective_on_objects]{surjective on objects} and \hyperref[def:functor_invertibility/full]{full}.

    It is not faithful, nor injective on objects.
  \end{thmenum}
\end{example}

\begin{definition}\label{def:natural_transformation}\mcite[def. 1.3.1]{Leinster2016Basic}
  Let \( F \) and \( G \) be parallel \hyperref[def:functor]{functors} from the category \( \cat{C} \) to \( \cat{D} \).

  A \term{natural transformation} \( \alpha \) from \( F \) to \( G \) is an \hyperref[def:cartesian_product/indexed_family]{indexed family} of
  \begin{equation}\label{eq:def:natural_transformation/family}
    \seq{ \alpha_A: F(A) \to G(A) }_{A \in \cat{C}}
  \end{equation}
  of morphisms in \( \cat{D} \) such that, for every morphism \( f: A \to B \) in \( \cat{C} \), the following \hyperref[def:categorical_diagram]{diagram commutes}:
  \begin{equation}\label{eq:def:natural_transformation/diagram}
    \begin{aligned}
      \includegraphics[page=1]{output/def__natural_transformation.pdf}
    \end{aligned}
  \end{equation}

  The morphisms \( \alpha_A \) are called the components of \( \alpha \). We denote natural transformations by \( \alpha: F \Rightarrow G \) and, when used in diagrams, by
  \begin{equation}\label{eq:def:natural_transformation/notation}
    \begin{aligned}
      \includegraphics[page=2]{output/def__natural_transformation.pdf}
    \end{aligned}
  \end{equation}
\end{definition}

\begin{example}\label{ex:quivers_as_functors}\mcite[exmpl. 1.3.46]{Perrone2019}
  In \fullref{def:quiver}, we have defined a quiver as a set \( V \) of vertices, a set \( A \) of arcs and two functions --- the head \( h: A \to V \)and tail \( t: A \to V \) of an arc.

  Now consider the following \hyperref[def:categorical_diagram]{index category} \( \cat{I}: \)
  \begin{equation}\label{eq:ex:quivers_as_functors/index/dots}
    \begin{aligned}
      \includegraphics[page=1]{output/ex__quivers_as_functors.pdf}
    \end{aligned}
  \end{equation}

  For the sake of readability, we will give the following explicit labels in this category:
  \begin{equation}\label{eq:ex:quivers_as_functors/index/annotated}
    \begin{aligned}
      \includegraphics[page=2]{output/ex__quivers_as_functors.pdf}
    \end{aligned}
  \end{equation}

  A quiver can then be defined as a functor \( Q: \cat{I} \to \ucat{Set} \) to the category \hyperref[def:category_of_small_sets]{\( \ucat{Set} \)} of \( \mscrU \)-small sets (for a \hyperref[def:category_size]{fixed Grothendieck universe} \( \mscrU \)).

  A \hyperref[def:natural_transformation]{natural transformation} from the quiver \( Q: \cat{I} \to \ucat{Set} \) to \( R: \cat{I} \to \ucat{Set} \) is then a pair of functions \( f_V: Q(V) \to R(V) \) and \( f_A: Q(A) \to R(A) \) such that the following \hyperref[def:categorical_diagram]{diagrams commute}:
  \begin{equation}\label{eq:ex:quivers_as_functors/index/diagram}
    \begin{aligned}
      \includegraphics[page=3]{output/ex__quivers_as_functors.pdf}
      \quad\quad\quad\quad
      \includegraphics[page=4]{output/ex__quivers_as_functors.pdf}
    \end{aligned}
  \end{equation}

  See \fullref{ex:isomorphism_of_quiver_categories} for how these functors related to quivers as defined in \fullref{def:quiver}.
\end{example}

\begin{remark}\label{rem:natural_transformations_into_set}
  Let \( \cat{C} \) be an arbitrary \( \mscrU \)-small category. A \hyperref[def:natural_transformation]{natural transformation} \( \alpha \) from \( F: \cat{C} \to \ucat{Set} \) to \( G: \cat{C} \to \ucat{Set} \) is then a family of functions
  \begin{equation*}
    \seq{ \alpha_A: F(A) \to G(A) }_{A \in \cat{C}}.
  \end{equation*}

  Suppose that for every two objects \( A \) and \( B \) in \( \cat{C} \), the functions \( \alpha_A \) and \( \alpha_B \) agree on \( F(A) \cap F(B) \). This is automatically satisfied in \( F(A) \) and \( F(B) \) are disjoint whenever \( A \neq B \).

  We can then take the set-theoretic union of \( \alpha \) to obtain the function
  \begin{equation*}
    \bigcup_{A \in \cat{C}} \alpha_A: \bigcup\set{ F(A) \given A \in \cat{C} } \to \bigcup\set{ G(A) \given A \in \cat{C} }.
  \end{equation*}

  Both the domain and codomain are sets as a consequence of \ref{def:grothendieck_universe/union}, therefore the function is well-defined in the universe \( \mscrU \). Denote it on \( \Alpha \) for brevity.

  An advantage of this is that we can define a natural transformation to be a function on a general enough set and then prove that its restrictions satisfy \eqref{eq:def:natural_transformation/diagram}.

  For example, consider the power set functor \( \pow: \ucat{Set} \to \ucat{Set} \) discussed in \fullref{ex:unary_functors_in_set}. The \hyperref[def:multi_valued_function/identity]{identity function} \( \id_\mscrU \) is then a natural transformation from the identity functor \( \id_{\ucat{Set}} \) to \( \pow \).

  Another natural transformation between the same functors is the singleton set operation \( \Sigma \) on sets defined as \( A \mapsto \set{ A } \). Note that, in this context, \( \Sigma \) operates not on the sets \( \id_{\ucat{Set}}(A) \) and \( \pow(A) \), but on their members. The diagram \eqref{eq:def:natural_transformation/diagram} becomes
  \begin{equation}\label{eq:rem:natural_transformations_into_set}
    \begin{aligned}
      \includegraphics[page=1]{output/rem__natural_transformations_into_set.pdf}
    \end{aligned}
  \end{equation}

  This diagram commutes because, for every function \( f: A \to B \) and every \( x \in A \), we have
  \begin{equation*}
    f[\set{ x }] = \set{ f(x) }.
  \end{equation*}
\end{remark}

\begin{definition}\label{def:functor_category}
  Let \( \cat{C} \) and \( \cat{D} \) be arbitrary \hyperref[def:category]{categories}. The \term{functor category} \( [\cat{C}, \cat{D}] \), also denoted as \( \cat{D}^{\cat{C}} \), is defined as follows:

  \begin{itemize}
    \item The \hyperref[def:category/objects]{set of objects} \( \obj([\cat{C}, \cat{D}]) \) is the set of all functors from \( \cat{C} \) to \( \cat{D} \).

    \item The \hyperref[def:category/morphisms]{set of morphisms} \( [\cat{C}, \cat{D}](F, G) \) from \( F \) to \( G \) is the set of all \hyperref[def:natural_transformation]{natural transformations} from \( F \) to \( G \).

    \item The \hyperref[def:category/composition]{composition of the morphisms} \( \alpha: F \Rightarrow G \) and \( \beta: G \Rightarrow H \) is the natural transformation \( \beta \bincirc \alpha: F \Rightarrow H \) defined in terms of componentwise morphism composition, i.e.
    \begin{equation}\label{eq:def:functor_category/composition}
      (\beta \bincirc \alpha)_A \coloneqq \beta_A \bincirc \alpha_A.
    \end{equation}

    \item The \hyperref[def:category/identity]{identity morphism} on the functor \( F: \cat{C} \to \cat{D} \) is the \term{identity natural transformation} \( \id_F: F \Rightarrow F \) with components
    \begin{equation}\label{eq:def:functor_category/identity}
      (\id_F)_A \coloneqq \underbrace{\id_{F(A)}}_{F(\id_A)}
    \end{equation}
  \end{itemize}
\end{definition}
\begin{defproof}
  Just to verify that the composition \( \beta \bincirc \alpha \) defined in \eqref{eq:def:functor_category/composition} is indeed a natural transformation from \( F \) to \( H \), note that the following diagram trivially commutes:
  \begin{equation}\label{def:functor_category/composition}
    \begin{aligned}
      \includegraphics[page=1]{output/def__functor_category.pdf}
    \end{aligned}
  \end{equation}

  Now, to see that \( [\cat{C}, \cat{D}] \) is indeed a category, we verify the conditions \ref{def:category/C1} and \ref{def:category/C2}, which are in turn inherited from the same conditions on the categories \( \cat{C} \) and \( \cat{D} \).

  \SubProofOf{def:category/C1} For every two functors \( F, G: \cat{C} \to \cat{D} \) and natural transformation \( \alpha: F \Rightarrow G \), for every object \( A \in \cat{C} \) we have
  \begin{equation*}
    \id_{G(A)} \bincirc \alpha_A
    \reloset{\eqref{def:category/C1}} =
    \alpha_A
    \reloset{\eqref{def:category/C1}} =
    \alpha_A \bincirc \id_{F(A)}
  \end{equation*}

  Therefore,
  \begin{equation*}
    \id_G \bincirc \alpha = \alpha = \alpha \bincirc \id_F
  \end{equation*}
  and, after generalizing, we obtain that \eqref{eq:def:category/C1} holds in \( [\cat{C}, \cat{D}] \).

  \SubProofOf{def:category/C2} For any quadruple \( F \), \( G \), \( H \) and \( T \) of functors from \( \cat{C} \) to \( \cat{D} \) and every combination of natural transformations \( \alpha: F \Rightarrow G \), \( \beta: G \Rightarrow H \) and \( \gamma: H \Rightarrow T \), for every object \( A \in \cat{C} \) we have
  \begin{equation*}
    (\gamma_A \bincirc \beta_A) \bincirc \alpha_A
    \reloset{\eqref{def:category/C2}} =
    \gamma_A \bincirc (\beta_A \bincirc \alpha_A).
  \end{equation*}

  Therefore, after generalizing, we obtain that \eqref{eq:def:category/C2} holds in \( [\cat{C}, \cat{D}] \).
\end{defproof}

\begin{remark}\label{rem:functor_category_size}
  If \( \cat{C} \) and \( \cat{D} \) are \( \mscrU \)-large categories in the sense of \fullref{def:category_size}, we cannot construct the \hyperref[def:functor_category]{functor category} \( [\cat{C}, \cat{D}] \). This is the main motivation for the \hyperref[def:axiom_of_universes]{axiom of universes}, which is discussed in \fullref{def:large_and_small_sets} and, in relation to category theory, in \fullref{def:category_size}.
\end{remark}

\begin{example}\label{ex:isomorphism_of_quiver_categories}
  In \fullref{ex:quivers_as_functors}, we defined \hyperref[def:quiver]{quivers} as functors from a certain index category \( \cat{I} \) to \( \ucat{Set} \) (for a \hyperref[def:category_size]{fixed Grothendieck universe} \( \mscrU \)).

  There is then an obvious correspondence between quivers as objects of \hyperref[def:category_of_small_quivers]{\( \ucat{Quiv} \)}, defined in \fullref{def:quiver}, and quivers as objects in the \hyperref[def:functor_category]{functor category} \( [\cat{I}, \ucat{Set}] \), defined in \fullref{ex:quivers_as_functors}. Indeed, given any functor \( Q: \cat{I} \to \ucat{Set} \), the quadruple
  \begin{equation*}
    \parens[\Big]{ Q(V), Q(A), Q(h), Q(t) }
  \end{equation*}
  is a quiver in the sense of \fullref{def:quiver}.

  No object in \( \ucat{Quiv} \) is formally equal to any object in \( [\cat{I}, \ucat{Set}] \) in the sense of \hyperref[def:zfc]{\logic{ZFC}}. They are, however, equivalent, as shown above, and this can be formalized by stating that the two categories are isomorphic, in the sense of \fullref{def:morphism_invertibility/isomorphism}, as objects of the category \( \ucat[\mscrV]{Cat} \), where \( \mscrV \) is a Grothendieck universe that strictly contains \( \mscrU \). We have already defined this isomorphism explicitly.

  This is an example of \term{isomorphism of categories}. In practice, if two categories are not so obviously identical, we are usually better served by \term{equivalences of categories} defined in \fullref{def:category_equivalence}.
\end{example}

\begin{definition}\label{def:opposite_natural_transformation}\mcite{nLab:opposite_category}
  The \term{opposite} natural transformation of \( \alpha: F \Rightarrow G \), where \( F \) and \( G \) are functors from \( \cat{C} \) to \( \cat{D} \), is the natural transformation \( \alpha^{\opcat}: G^{\opcat} \Rightarrow F^{\opcat} \), in which we take the opposite of each component in \( \alpha \).

  Dual natural transformation arise naturally in \fullref{thm:opposite_of_functor_category}.
\end{definition}
\begin{defproof}
  The naturality diagram \eqref{eq:def:natural_transformation/diagram} commutes for \( \alpha^{\opcat} \) because all morphisms are simply reversed.
\end{defproof}

\begin{proposition}\label{thm:opposite_of_functor_category}
  For the \hyperref[def:opposite_category]{opposite} of the \hyperref[def:functor_category]{functor category} \( [\cat{C}, \cat{D}] \) we have
  \begin{equation*}
    [\cat{C}, \cat{D}]^{\opcat} = [\cat{C}^{\opcat}, \cat{D}^{\opcat}].
  \end{equation*}

  This is part of the duality principles listed in \fullref{thm:categorical_principle_of_duality}.
\end{proposition}
\begin{proof}
  In \fullref{def:opposite_functor}, we have defined the opposite functor \( F^{\opcat}: \cat{C}^{\opcat} \to \cat{D}^{\opcat} \) of \( F: \cat{C} \to \cat{D} \) in a way that allows us to regard it as an object of \( [\cat{C}^{\opcat}, \cat{D}^{\opcat}] \).

  In \fullref{def:opposite_natural_transformation}, we have defined the opposite natural transformation \( \alpha^{\opcat}: G^{\opcat} \to F^{\opcat} \) of \( \alpha: F \Rightarrow G \) in a way that allows us to regard it as a morphism of \( [\cat{C}^{\opcat}, \cat{D}^{\opcat}] \).

  Furthermore, \( \alpha^{\opcat}: G^{\opcat} \to F^{\opcat} \) reverses the direction of its morphisms, and hence it is the dual to \( \alpha \) in the category \( [\cat{C}, \cat{D}]^{\opcat} \).
\end{proof}

\begin{definition}\label{def:diagonal_functor}\mcite[143]{Leinster2016Basic}
  Given an \term{index category} \( \cat{I} \) and an arbitrary category \( \cat{C} \), for any object \( A \) in \( \cat{C} \), we can define the \term{constant functor}
  \begin{equation*}
    \begin{aligned}
      &\Delta_A^{\cat{I}}: \cat{I} \to \cat{C}, \\
      &\Delta_A^{\cat{I}}(X) \coloneqq X, \\
      &\Delta_A^{\cat{I}}(g: X \to Y) \coloneqq \id_A.
    \end{aligned}
  \end{equation*}

  Given two objects \( A \) and \( B \) in \( \cat{C} \), a natural transformation \( \alpha: \Delta_A^{\cat{I}} \Rightarrow \Delta_B^{\cat{I}} \) is an \hyperref[def:cartesian_product/indexed_family]{indexed family} that gives the same morphism for every object of the index category \( \cat{I} \).

  Indeed, the diagram \eqref{eq:def:natural_transformation/diagram} in this case becomes
  \begin{equation}\label{eq:def:diagonal_functor/nat}
    \begin{aligned}
      \includegraphics[page=1]{output/def__diagonal_functor.pdf}
    \end{aligned}
  \end{equation}

  This diagram implies that \( \alpha_A = \alpha_B \) for any two objects \( A \) and \( B \) in \( \cat{I} \). Therefore, all components of \( \alpha \) are equal to some morphism in \( \cat{C}(A, B) \).

  We can now define the \( \cat{I} \)-shaped \term{diagonal functor} on \( \cat{C} \)
  \begin{equation*}
    \begin{aligned}
      &\Delta^{\cat{I}}: \cat{C} \to [\cat{I}, \cat{C}], \\
      &\Delta^{\cat{I}}(A) \coloneqq \Delta_A^{\cat{I}}, \\
      &\Delta^{\cat{I}}(f: A \to B) \coloneqq \seq{ f: A \to B }_{k \in \cat{I}}.
    \end{aligned}
  \end{equation*}

  It is called a diagonal functor because, if \( \cat{I} \) is a discrete category of two objects, then \( \Delta_{\cat{I}} \) gives the diagonal of the \hyperref[def:product_category]{product category} \( \cat{C}^2 \) by providing, for each object \( A \) of \( \cat{C} \), the ordered pair \( (A, A) \) (and similarly for morphisms).
\end{definition}

\begin{proposition}\label{thm:natural_isomorphism}\mcite{math3ma:natural_transformations}
  Let \( F \) and \( G \) be parallel \hyperref[def:functor]{functors} from the category \( \cat{C} \) to \( \cat{D} \). The family \eqref{eq:def:natural_transformation/family} is an isomorphism in the corresponding \hyperref[def:functor_category]{functor category} \( [\cat{C}, \cat{D}] \) if and only if all of its components are isomorphisms and, for any morphism \( f: A \to B \) in \( \cat{C} \), the following diagram commutes:
  \begin{equation}\label{eq:thm:natural_isomorphism/diagram}
    \begin{aligned}
      \includegraphics[page=1]{output/thm__natural_isomorphism.pdf}
    \end{aligned}
  \end{equation}

  We say that \( \alpha \) is a \term{natural isomorphism}.
\end{proposition}
\begin{proof}
  If all components of \( \alpha \) are isomorphisms, the condition
  \begin{equation*}
    \alpha_B \bincirc F(f) = G(f) \bincirc \alpha_A
  \end{equation*}
  is equivalent to
  \begin{equation*}
    F(f) = \alpha_B^{-1} \bincirc G(f) \bincirc \alpha_A.
  \end{equation*}

  We must now show that, if \( \alpha \) is an isomorphism in \( [\cat{C}, \cat{D}] \), all of its components are isomorphisms.

  If \( \alpha: F \Rightarrow G \) is an isomorphism in \( [\cat{C}, \cat{D}] \). Then there exists some natural transformation \( \beta: G \Rightarrow F \) such that
  \begin{equation*}
    \beta \bincirc \alpha = \id_F \quad\T{and}\quad \alpha \bincirc \beta = \id_G.
  \end{equation*}

  For every object \( A \) in \( \cat{C} \), the morphism \( \alpha_A: F(A) \to G(A) \) composed with \( \beta_A: G(A) \to F(A) \) is
  \begin{equation*}
    \beta_A \bincirc \alpha_A = \id_{F(A)}.
  \end{equation*}

  Therefore, \( \alpha_A \) is left-invertible. Analogously,
  \begin{equation*}
    \alpha_A \bincirc \beta_A = \id_{F(A)}
  \end{equation*}
  and hence \( \alpha_A \) is right-invertible.

  Therefore, for every object \( A \) in \( \cat{C} \), the morphism \( \alpha_A \) is fully invertible, i.e. an isomorphism.
\end{proof}

\begin{definition}\label{def:product_category}\mcite[const. 1.1.11]{Leinster2016Basic}
  We define the \term{product category} \( \cat{C} \times \cat{D} \) of \( \cat{C} \) and \( \cat{D} \) as follows:

  \begin{itemize}
    \item The \hyperref[def:category/objects]{set of objects} is the \hyperref[def:cartesian_product]{Cartesian product}
    \begin{equation}\label{eq:def:product_category/objects}
      \obj(\cat{C} \times \cat{D}) \coloneqq \obj(\cat{C}) \times \obj(\cat{D}).
    \end{equation}

    \item The \hyperref[def:category/morphisms]{set of morphisms} from the pair of objects \( (A, X) \) to \( (B, Y) \) is the product
    \begin{equation}\label{eq:def:product_category/morphisms}
      (\cat{C} \times \cat{D})\parens[\Big]{ (A, X), (B, Y) } \coloneqq \cat{C}(A, B) \times \cat{D}(X, Y).
    \end{equation}

    \item The \hyperref[def:category/composition]{composition of the morphisms}
    \begin{align*}
      (f, r)&: (A, X) \to (B, Y) \\
      (g, s)&: (B, Y) \to (C, Z)
    \end{align*}
    is the pairwise composition
    \begin{equation}\label{eq:def:product_category/composition}
      (g, s) \bincirc (f, r) \coloneqq \underbrace{(g \bincirc f, s \bincirc r)}_{(A, X) \to (C, Z)}.
    \end{equation}

    \item The \hyperref[def:category/identity]{identity morphism} of the pair \( (A, X) \) is simply the pair of identity morphisms \( (\id_A, \id_X) \).
  \end{itemize}
\end{definition}

\begin{definition}\label{def:hom_functor}
  Let \( \cat{C} \) be a \hyperref[def:category_size]{locally \( \mscrU \)-small} category. We can regard the morphism sets \( \cat{C}(A, B) \) as a functor parameterized by objects of \( \cat{C} \).

  \begin{thmenum}
    \thmitem{def:hom_functor/binary} For any pair of morphisms \( f: B \to A \) and \( g: X \to Y \) in \( \cat{C} \), define the operator
    \begin{equation}\label{eq:def:hom_functor/t}
      \begin{aligned}
        &T_{f,g}: \cat{C}(A, X) \to \cat{C}(B, Y) \\
        &T_{f,g}(s) \mapsto g \bincirc s \bincirc f.
      \end{aligned}
    \end{equation}

    The action of \( T_{f,g} \) can be expressed graphically as
    \begin{equation}\label{eq:def:hom_functor/t_diagram}
      \begin{aligned}
        \includegraphics[page=1]{output/def__hom_functor.pdf}
      \end{aligned}
    \end{equation}

    We can now define the following \term{binary hom-functor}:
    \begin{equation}\label{eq:def:hom_functor/binary}
      \begin{aligned}
        &\cat{C}(\anon*, \anon*): \cat{C}^{\opcat} \times \cat{C} \to \ucat{Set} \\
        &\cat{C}(A, X) \coloneqq \set{ s: A \to X } \\
        &\cat{C}(f, g) \coloneqq T_{f,g}
      \end{aligned}
    \end{equation}

    \thmitem{def:hom_functor/unary} Fixing the first argument \( A \) in \eqref{eq:def:hom_functor/binary}, we instead obtain a covariant unary hom-functor:
    \begin{equation}\label{eq:def:hom_functor/unary/covariant}
      \cat{C}(A, \anon*): \cat{C} \to \ucat{Set}
    \end{equation}

    Analogously, fixing the second argument \( X \), we obtain a \hyperref[def:hom_functor/unary]{contravariant} unary hom-functor:
    \begin{equation}\label{eq:def:hom_functor/unary/contravariant}
      \cat{C}(\anon*, X): \cat{C}^{\opcat} \to \ucat{Set}
    \end{equation}
  \end{thmenum}
\end{definition}
\begin{defproof}
  It is sufficient to verify that \eqref{eq:def:hom_functor/binary} defines a functor. \ref{def:functor/CF2} can be seen to hold by inspecting the diagram:
  \begin{equation}\label{eq:def:hom_functor/inv_composition}
    \begin{aligned}
      \includegraphics[page=2]{output/def__hom_functor.pdf}
    \end{aligned}
  \end{equation}

  The other functor condition \ref{def:functor/CF1} is straightforward to prove.
\end{defproof}

\begin{proposition}\label{thm:currying_is_natural_isomorphism}
  \hyperref[def:function/currying]{Function currying} is a natural isomorphism between the functors
  \begin{align*}
    &\cat{Set}(A \times B, C)
    &\cat{Set}(A, \cat{Set}(B, C))
  \end{align*}

  More concretely, consider the following functors, which are loosely based on \fullref{def:hom_functor}:
  \begin{equation*}
    \begin{aligned}
      &V: \cat{Set}^3 \to \cat{Set} \\
      &V(A, B, C) \coloneqq \cat{Set}(A \times B, C) \\
      \Big[& V(f: X \to A, g: Y \to B, h: C \to Z) \Big](s: A \times B \to C) \coloneqq \smash{ \overbrace{ (x, y) \mapsto h\parens[\Bigg]{ s\parens[\Big]{ \underbrace{ f(x), g(y) }_{A \times B} } } }^{X \times Y \to Z} } \\
    \end{aligned}
  \end{equation*}
  and
  \begin{equation*}
    \begin{aligned}
      &W: \cat{Set}^3 \to \cat{Set} \\
      &W(A, B, C) \coloneqq \cat{Set}(A, \cat{Set}(B, C)) \\
      \Big[& W(f: X \to A, g: Y \to B, h: C \to Z) \Big](t: A \to \cat{Set}(B, C)) \coloneqq \smash{ \underbrace{ x \mapsto \overbrace{y \mapsto h\parens[\Bigg]{ \overbrace{t(f(x))}^{B \to C}\parens[\Big]{ g(y) } } }^{Y \to Z} }_{X \to \cat{Set}(Y, Z)} } \\
    \end{aligned}
  \end{equation*}

  Then the family of functions
  \begin{equation*}
    \begin{aligned}
      &\alpha: V \Rightarrow W \\
      &\alpha_{A,B,C}(s: A \times B \to C) \coloneqq a \mapsto b \mapsto s(a, b)
    \end{aligned}
  \end{equation*}
  is a \hyperref[thm:natural_isomorphism]{natural isomorphism}.
\end{proposition}
\begin{proof}
  The function \( \varphi \) is clearly invertible. Fix a triple of functions \( f: X \to A \), \( g: Y \to B \) and \( h: C \to Z \). For every \( s: A \times B \to C \) we have
  \begin{align*}
    [W(f, g, h)](\alpha_{A,B,C}(s))
    &=
    x \mapsto y \mapsto \parens[\Big]{ h\parens[\Big]{ [a \mapsto b \mapsto s(a, b)](f(x))(g(y)) } }
    = \\ &=
    x \mapsto y \mapsto h\parens[\Big]{ s(f(x), g(y)) }
    = \\ &=
    \alpha_{X,Y,Z}(V(f, g, h)),
  \end{align*}
  which proves that the following diagram commutes:
  \begin{equation}\label{eq:thm:currying_is_natural_isomorphism/diagram}
    \begin{aligned}
      \includegraphics[page=1]{output/thm__currying_is_natural_isomorphism.pdf}
    \end{aligned}
  \end{equation}
\end{proof}
