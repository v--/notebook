\subsection{Functors}\label{subsec:functors}

\begin{definition}\label{def:functor}\mcite[def. 1.2.1 \\ def. 1.2.10]{Leinster2016Basic}
  Fix some \hyperref[def:category]{categories} \( \cat{C} \) and \( \cat{D} \). A \term{functor} \( F: \cat{C} \to \cat{D} \) is a \hyperref[def:theory_of_graphs/quiver_homomorphism]{quiver homomorphism} between the underlying quivers that is compatible with composition and identities.

  Explicitly, a functor is a family of functions
  \begin{equation}\label{eq:def:functor_as_family_of_function}
    \begin{aligned}
      F_{\obj}:       &\obj(\cat{C}) \to \obj(\cat{D}) \\
      F_{\hom(X, Y)}: &\cat{C}(X, Y) \to \cat{D}(F_{\obj}(X), F_{\obj}(Y)),
    \end{aligned}
  \end{equation}
  where \( F_{\hom(X, Y)} \) is a distinct function for every pair of objects \( X \) and \( Y \).

  In practice, we usually define the functor as the set
  \begin{equation}\label{eq:def:functor_as_single_function}
    F \coloneqq F_{\obj} \cup \bigcup\set{ F_{\hom(X, Y)} \given A, Y \in \obj(\cat{C}) }
  \end{equation}

  Since the domains of all constituent functions are disjoint, \( F \) is again a total single-valued function. This allows us to justify the notation \( F(X) \) for objects and \( F(f) \) for morphisms.

  The definition of a functor additionally requires the following compatibility conditions to hold:
  \begin{thmenum}[series=def:category]
    \thmitem[def:functor/CF1]{CF1} Functors must preserve composition, meaning that for any pair of morphism \( f: X \to Y \) and \( g: Y \to C \) in \( \cat{C} \), we have
    \begin{equation}\label{eq:def:category/CF1}\tag{\logic{CF1}}
      F(g \bincirc f) = F(g) \bincirc F(f).
    \end{equation}

    \thmitem[def:functor/CF2]{CF2} Functors must preserve identities, meaning that for any object \( X \in \cat{C} \) we must have
    \begin{equation}\label{eq:def:category/CF2}\tag{\logic{CF2}}
      F(\id_X) = \id_{F(X)}.
    \end{equation}
  \end{thmenum}
\end{definition}
\begin{defproof}
  The definition \eqref{eq:def:functor_as_family_of_function} ensures that the quiver homomorphism conditions \eqref{eq:def:theory_of_graphs/quiver_homomorphism/head} and \eqref{eq:def:theory_of_graphs/quiver_homomorphism/tail} hold.

  Indeed, for any morphism \( f: X \to Y \) in \( \cat{C} \) we have
  \begin{equation*}
    F(\dom(f)) = F(X) = \dom(F(f)),
  \end{equation*}
  which implies \eqref{eq:def:theory_of_graphs/quiver_homomorphism/head}. We also have
  \begin{equation*}
    F(\co\dom(f)) = F(Y) = \co\dom(F(f)),
  \end{equation*}
  which implies \eqref{eq:def:theory_of_graphs/quiver_homomorphism/tail}.
\end{defproof}

\begin{remark}\label{rem:functor_size}
  It is possible that \( \cat{C} \) is \( \mscrU \)-small in the sense of \fullref{def:category_size}, but the \hyperref[def:functor]{functor} \( F \), as the set \eqref{eq:def:functor_as_single_function}, is not \( \mscrU \)-small in the sense of \fullref{def:large_and_small_sets}. Without using universes, we cannot prove the existence of any functor from the category of smalls sets to itself, for example.
\end{remark}

\begin{proposition}\label{thm:def:functor/properties}
  \hyperref[def:functor]{Functors} have the following basic properties:
  \begin{thmenum}
    \thmitem{thm:def:functor/properties/isomorphisms} Functors preserve \hyperref[def:morphism_invertibility/isomorphism]{isomorphisms}. That is, for every functor \( F: \cat{C} \to \cat{D} \) and every pair of objects \( A \) and \( B \) in \( \cat{C} \), from \( A \cong B \) it follows tha \( F(A) \cong F(B) \).
  \end{thmenum}
\end{proposition}
\begin{proof}
  \SubProofOf{thm:def:functor/properties/isomorphisms} Let \( f: A \to B \) be an isomorphism in \( \cat{C} \). Then
  \begin{equation}\label{eq:thm:def:functor/properties/isomorphisms}
    F(f^{-1}) \bincirc F(f)
    \reloset {\eqref{eq:def:category/CF2}} =
    F(f^{-1} \bincirc f)
    =
    F(\id_A)
    \reloset {\eqref{eq:def:category/CF1}} =
    \id_{F(A)},
  \end{equation}

  Hence, \( F(f^{-1}) \) is a \hyperref[morphism_invertibility/left_invertible]{left inverse} of \( F(f) \). The proof that \( F(f^{-1}) \) is a right inverse is analogous, hence \( F(f^{-1}) \) is a two-sided inverse of \( F(f) \). Therefore, \( F(f) \) is an isomorphism from \( A \) to \( B \).
\end{proof}

\begin{definition}\label{def:category_of_small_categories}
  Suppose that we are given a \hyperref[def:grothendieck_universe]{Grothendieck universe} \( \mscrU \), which is safe to assume to be the smallest suitable one as explained in \fullref{def:large_and_small_sets}.

  We denote the \hyperref[def:category]{category} of \( \mscrU \)-small \hyperref[def:category]{categories} by \( \cat{\mscrU-Cat} \) or, if the universe is clear from the context, simply by \( \cat{Cat} \). See \fullref{def:category_size} for a further discussion of universes and categories.

  \begin{itemize}
    \item The \hyperref[def:category/objects]{set of objects} \( \obj(\cat{Cat}) \) is the set of all \( \mscrU \)-small categories.

    \item The \hyperref[def:category/morphisms]{set of morphisms} \( \cat{Cat}(X, Y) \) from \( X \) to \( Y \) is the set of all \hyperref[def:functor]{functors} from \( X \) to \( Y \).

    \item The \hyperref[def:category/composition]{composition of morphisms} is the \hyperref[def:multi_valued_function/composition]{function composition} of the functors regarded as the functions \eqref{eq:def:functor_as_single_function}. That is, the composition of \( F: \cat{C} \to \cat{D} \) and \( G: \cat{D} \to \cat{E} \) is the functor
    \begin{equation}\label{eq:def:category_of_small_categories/composition}
      \begin{aligned}
        &G \bincirc F: \cat{C} \to \cat{E}, \\
        &[G \bincirc F](X) \coloneqq G(F(X)), \\
        &[G \bincirc F](f) \coloneqq G(F(f)).
      \end{aligned}
    \end{equation}

    \item The \hyperref[def:category/identity]{identity morphism} of the category \( \cat{C} \) is the \term{identity functor}
    \begin{equation}\label{eq:def:category_of_small_categories/identity}
      \begin{aligned}
        &\id_{\cat{C}}: \cat{C} \to \cat{C}, \\
        &\id_{\cat{C}}(X) \coloneqq A, \\
        &\id_{\cat{C}}(f) \coloneqq f.
      \end{aligned}
    \end{equation}
  \end{itemize}
\end{definition}
\begin{defproof}
  To see that \( \cat{\mscrU-Cat} \) is indeed a category, we verify the conditions \ref{def:category/C1} and \ref{def:category/C2}.

  \SubProofOf{def:category/C1} For every two \( \mscrU \)-small categories \( \cat{C} \) and \( \cat{D} \) and every functor \( F: \cat{C} \to \cat{D} \), for every object \( X \in \cat{C} \) we have
  \begin{equation*}
    [\id_{\cat{D}} \bincirc F](X)
    =
    \id_{\cat{D}}(F(X))
    =
    F(X)
    =
    F(\id_{\cat{C}}(X))
    =
    [F \bincirc \id_{\cat{C}}](X)
  \end{equation*}
  and analogously for morphisms.

  Therefore, \( \id_{\cat{C}} \) and \( \id_{\cat{D}} \) satisfy \eqref{eq:def:category/C1}.

  \SubProofOf{def:category/C2} Associativity of functor composition is inherited from the associativity of function composition.
\end{defproof}

\begin{proposition}\label{thm:category_of_small_categories_properites}
  We collect here important properties of the category \hyperref[def:category_of_small_categories]{\( \cat{\mscrU-Cat} \)} of \( \mscrU \)-small categories. Most of them require forward references.
\end{proposition}

\begin{definition}\label{def:functor_image}
  The \term{image} of a functor \( F: \cat{C} \to \cat{D} \) is the \hyperref[def:quiver]{quiver} whose vertex set is
  \begin{equation*}
    V \coloneqq \set{ F(X) \given X \in \cat{C} }
  \end{equation*}
  and whose arc set is
  \begin{equation*}
    X \coloneqq \set{ F(f) \given A, Y \in \cat{C} \T{and} f \in \cat{C}(X, Y) }.
  \end{equation*}

  This quiver has no categorical structure --- it is merely a directed multigraph. As shown in \fullref{ex:functor_image_not_a_category}, imposing a categorical structure na\"ively may fail.
\end{definition}

\begin{example}\label{ex:functor_image_not_a_category}\mcite{MathSE:image_of_functor_is_not_a_category}
  \begin{figure}
    \hfill
    \includegraphics{figures/fig__ex__functor_image_not_a_category.pdf}
    \hfill
    \hfill
    \caption{A functor whose image is not a category.}\label{fig:ex:functor_image_not_a_category}
  \end{figure}

  Consider the functor \( F: \cat{C} \to \cat{D} \) from \cref{fig:ex:functor_image_not_a_category}.

  \begin{itemize}
    \item The solid arrows are the morphisms in \( \cat{C} \) and their images in \( F(\cat{C}) \).
    \item The double arrows denote the action of the functor \( F \).
    \item The dotted arrow exists in \( \cat{D} \) as the composition of the other two arrows, however it is missing in the image \( F(\cat{C}) \). Thus, composition is not fully defined in \( F(\cat{C}) \), and \( F(\cat{C}) \) fails to be a category.
  \end{itemize}
\end{example}

\begin{remark}\label{rem:categorical_diagram_as_functor}
  Compared to \fullref{def:categorical_diagram}, a more \enquote{categorical} approach to defining diagrams is by using functors.

  Fix a category \( \cat{I} \), called an \term{index category}. A \term{diagram} of shape \( \cat{I} \) is simply a functor \( D: \cat{I} \to \cat{C} \), whose domain is \( \cat{I} \).

  We do not really care about how the objects and morphisms in \( \cat{I} \) are labeled, hence we often use placeholder dots like in \fullref{fig:ex:functor_image_not_a_category}.
\end{remark}

\begin{definition}\label{def:functor_invertibility}
  In connection with \fullref{def:morphism_invertibility} and \fullref{def:function_invertibility}, we introduce the following terminology:
  \begin{thmenum}
    \thmitem{def:functor_invertibility/injective_on_objects} The \hyperref[def:functor]{functor} \( F: \cat{C} \to \cat{D} \) is \term{injective on objects} if the \hyperref[def:multi_valued_function/restriction]{restriction}
    \begin{equation*}
      F\restr_{\obj(C)}: \obj(C) \to \obj(D)
    \end{equation*}
    is \hyperref[def:function_invertibility/injective]{injective}.

    That is, for every pair of objects \( X \) and \( Y \) in \( \cat{C} \), from \( F(X) = F(Y) \) it follows that \( X = Y \).

    If, instead, from \( F(X) \cong F(Y) \) it follows that \( X \cong Y \), we say that \( F \) if \term{essentially injective on objects}.

    \thmitem{def:functor_invertibility/injective_on_morphisms} The \hyperref[def:functor]{functor} \( F: \cat{C} \to \cat{D} \) is \term{injective on morphisms} if its restriction to the set
    \begin{equation*}
      \bigcup\set{ \cat{C}(X, Y) \given X, Y \in \obj(\cat{C}) }
    \end{equation*}
    of all morphisms is injective.

    That is, for every pair of morphisms \( f \) and \( g \) in \( \cat{C} \), from \( F(f) = F(g) \) it follows that \( f = g \). Note that if the morphisms are not parallel, we assume that they are not equal.

    \thmitem{def:functor_invertibility/faithful}\mcite[def. 1.2.16]{Leinster2016Basic} The functor \( F: \cat{C} \to \cat{D} \) is \term{faithful} if it is \hyperref[def:function_invertibility/injective]{injective} on \( \hom \)-sets, i.e. for all pairs of objects \( X \) and \( Y \) in \( \cat{C} \), the restriction of \( F \) to \( \cat{C}(X, Y) \) is an injective function.

    That is, for every pair of objects \( X \) and \( Y \) in \( \cat{C} \) and every pair of morphisms \( f \) and \( g \) in \( \cat{C}(X, Y) \), from \( F(f) = F(g) \) it follows that \( f = g \).

    See \fullref{thm:def:functor_invertibility/properties/injective_faithful} for how faithful functors relate to functors injective on objects or on morphisms.

    \thmitem{def:functor_invertibility/surjective_on_objects}\mcite[def. 1.3.17]{Leinster2016Basic} The functor \( F: \cat{C} \to \cat{D} \) is \term{surjective on objects} if the restriction
    \begin{equation*}
      F\restr_{\obj(C)}: \obj(C) \to \obj(D)
    \end{equation*}
    is \hyperref[def:function_invertibility/surjective]{surjective}.

    That is, for every object \( Y \) in \( \cat{D} \), there exists at least one object \( X \) in \( \cat{C} \) such that \( F(X) = Y \).

    If, instead, there exists at least one object \( X \in \cat{C} \) such that \( F(X) \cong Y \), we say that \( F \) is \term{essentially injective on objects}.

    \thmitem{def:functor_invertibility/surjective_on_morphisms} Similarly, \( F: \cat{C} \to \cat{D} \) is \term{surjective on morphisms} if its restriction to the set of all morphisms is surjective.

    That is, for every morphism \( g \) in \( \cat{D} \), there exists at least one morphism \( f \) in \( \cat{C} \) such that \( F(f) = g \).

    \thmitem{def:functor_invertibility/full}\mcite[def. 1.2.16]{Leinster2016Basic} The functor \( F: \cat{C} \to \cat{D} \) is \term{full} if it is surjective on \( \hom \)-sets, i.e. for all pairs of objects \( X \) and \( Y \) in \( \cat{C} \), the restriction of \( F \) to \( \cat{C}(X, Y) \) is a surjective function.

    That is, for every pair of objects \( X \) and \( Y \) in \( \cat{C} \) and every morphism \( g: F(X) \to F(Y) \) in \( \cat{D} \), there exists at least one morphism in \( f: X \to Y \) in \( \cat{C} \) such that \( F(f) = g \).

    \thmitem{def:functor_invertibility/fully_faithful} Finally, \( F: \cat{C} \to \cat{D} \) is \term{fully faithful} if it is both full and faithful.
  \end{thmenum}
\end{definition}

\begin{proposition}\label{thm:def:functor_invertibility/properties}
  \hyperref[def:functor]{Functors} have the following basic properties regarding their \hyperref[def:functor_invertibility]{invertibility}:

  \begin{thmenum}
    \thmitem{thm:def:functor_invertibility/properties/injective} A functor is \hyperref[def:functor_invertibility/injective_on_morphisms]{injective on morphisms} if and only if it is both \hyperref[def:functor_invertibility/injective_on_objects]{injective on objects} and \hyperref[def:functor_invertibility/faithful]{faithful}.

    \thmitem{thm:def:functor_invertibility/properties/surjective} A functor is \hyperref[def:functor_invertibility/surjective_on_morphisms]{surjective on morphisms} if and only if it is both \hyperref[def:functor_invertibility/surjective_on_objects]{surjective on objects} and \hyperref[def:functor_invertibility/full]{full}.
  \end{thmenum}
\end{proposition}
\begin{proof}
  \SubProofOf{thm:def:functor_invertibility/properties/injective}
  \SufficiencySubProof* Let \( F: \cat{C} \to \cat{D} \) be injective on morphisms. It is trivially faithful since faithfulness is a more restrictive condition.

  To see that \( F \) is injective on objects, let \( X, Y \in \cat{C} \) and suppose that \( F(X) = F(Y) \). Then \( \id_{F(X)} = \id_{F(Y)} \) and
  \begin{equation*}
    F(\id_X)
    \reloset {\eqref{eq:def:category/CF2}} =
    \id_{F(X)}
    =
    \id_{F(Y)}
    \reloset {\eqref{eq:def:category/CF2}} =
    F(\id_Y).
  \end{equation*}

  Since \( F \) is injective on morphisms, it follows that \( \id_X = \id_Y \), hence \( X = Y \). Thus, \( F \) is injective on objects.

  \NecessitySubProof* Let \( F: \cat{C} \to \cat{D} \) be faithful and injective on objects. Let \( f: X \to Y \) and \( g: Z \to U \) be morphisms in \( \cat{C} \) such that \( F(f) = F(g) \).

  Then both \( F(f) \) and \( F(g) \) have the same domain \( F(X) = F(Z) \) and codomain \( F(Y) = F(U) \). Hence, since \( F \) is injective on objects, we have \( X = Z \) and \( Y = U \).

  Thus, \( f \) and \( g \) are both morphisms from \( X \) to \( Y \). Since \( F \) is also faithful, from \( F(f) = F(g) \) it follows that \( f = g \).

  Therefore, \( F \) is injective on morphisms.

  \SubProofOf{thm:def:functor_invertibility/properties/surjective}
  \SufficiencySubProof* Let \( F: \cat{C} \to \cat{D} \) be surjective on morphisms. It is trivially full since fullness is a more restrictive condition.

  To see that \( F \) is surjective on objects, let \( Z \in \cat{D} \). Then there exists some morphism \( f: X \to Y \) in \( \cat{C} \) such that \( F(f) = \id_Z \). We thus necessarily have \( F(X) = Z \) and \( F(Y) = Z \).

  \NecessitySubProof* Let \( F: \cat{C} \to \cat{D} \) be full and injective on objects. Let \( g: Z \to U \) be a morphism in \( \cat{D} \).

  Since \( F \) is surjective on objects, there exists preimages \( X \) of \( Z \) and \( Y \) of \( U \) under \( F \). Thus, \( g \in \cat{D}(F(X), F(Y)) \).

  Since \( F \) is also full, there exists some morphism \( f: X \to Y \) such that \( F(f) = g \).

  Therefore, \( F \) is surjective on morphisms.
\end{proof}
