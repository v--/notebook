\subsection{Modules}\label{subsec:modules}

\begin{definition}\label{def:module}
  A \term{module} is a \hyperref[def:semimodule]{semimodule} over a \hyperref[def:ring]{ring} rather than a \hyperref[def:semiring]{semiring}.

  This makes the identity law \eqref{eq:def:semimodule/operation/scalar_multiplication_action/identity} redundant, but most metamathematical properties remain the same. The first-order theory is identical to the \hyperref[def:semimodule/theory]{theory of semimodules}.

  For a fixed ring \( R \), we denote the \hyperref[def:category_of_small_first_order_models]{category of \( \mscrU \)-small models} by \( \ucat{Mod}_R \).
\end{definition}

\begin{theorem}[Homomorphism theorem for modules]\label{thm:homomorphism_theorem_for_modules}
  Every \hyperref[def:module/homomorphism]{module homomorphism} \( \varphi: M \to N \) induces an isomorphism
  \begin{equation*}
    M / \ker \varphi \cong \img \varphi.
  \end{equation*}

  Compare this to \fullref{thm:homomorphism_theorem_for_groups} and \fullref{thm:homomorphism_theorem_for_rings}.
\end{theorem}
\begin{proof}
  \Fullref{thm:homomorphism_theorem_for_groups} defines an explicit additive group isomorphism \( \phi: M / \ker \varphi \to \img \varphi \). The module structure simply restricts \( \ker \varphi \) to be a submodule rather than an arbitrary subgroup of \( M \).
\end{proof}

\begin{proposition}\label{thm:abelian_group_is_semimodule}
  Every abelian group \( G \) is a left semimodule over \( \BbbZ \) with scalar multiplication given by \hyperref[rem:additive_magma/multiplication]{recursively defined multiplication}
  \begin{equation}\label{eq:thm:semiring_is_semimodule/operation}
    \begin{aligned}
      &\cdot: \BbbZ \times G \to G \\
      &n \cdot x \coloneqq \begin{cases}
        0_G,           &n = 0, \\
        n \cdot x + x, &n > 1, \\
        -(n \cdot x),  &n < 1.
      \end{cases}
    \end{aligned}
  \end{equation}

  Conversely, in every semimodule over \( \BbbZ \), scalar multiplication matches the recursively defined multiplication.

  Compare this result to \fullref{thm:commutative_monoid_is_semimodule}.
\end{proposition}
\begin{proof}
  Trivial refinement of \fullref{thm:commutative_monoid_is_semimodule}.
\end{proof}

\begin{definition}\label{def:module_direct_product}
  The \term{direct product} \( \prod_{k \in \mscrK} M_k \) of the indexed family \( \seq{ M_k }_{k \in \mscrK} \) of \hyperref[def:module]{modules} over the same ring is the \hyperref[def:module_direct_product]{group direct product} endowed with the componentwise scalar multiplication operation
  \begin{equation*}
    t \cdot \seq{ x_k }_{k \in \mscrK} \coloneqq \seq{ t x_k }_{k \in \mscrK}.
  \end{equation*}
\end{definition}

\begin{definition}\label{def:module_direct_sum}
  The \term{direct sum} \( \bigoplus_{k \in \mscrK} M_k \) of a family \( \seq{ M_k }_{k \in \mscrK} \) of modules over the same ring is the subgroup of the \hyperref[def:group_direct_product]{direct product} \( \prod_{k \in \mscrK} G_k \), in which only finitely many components of each tuple are distinct from the identity.
\end{definition}

\begin{proposition}\label{thm:module_categorical_limits}
  Fix a \hyperref[def:ring]{ring} \( R \). We extend \fullref{thm:abelian_group_categorical_limits} to modules rather than arbitrary abelian groups.

  \begin{thmenum}
    \thmitem{thm:module_categorical_limits/product} The \hyperref[def:discrete_category_limits]{categorical product} of the family \( \seq{ M_k }_{k \in \mscrK} \) in the category \( \cat{Mod}_R \) of small \hyperref[def:modules]{modules} is their \hyperref[def:module_direct_product]{module product} \( \prod_{k \in \mscrK} G_k \).

    \thmitem{thm:module_categorical_limits/coproduct} The \hyperref[def:discrete_category_limits]{categorical coproduct} of the family \( \seq{ M_k }_{k \in \mscrK} \) of modules is their \hyperref[def:module_direct_sum]{direct sum} \( \biguplus_{k \in \mscrK} M_k \).
  \end{thmenum}
\end{proposition}
\begin{proof}
  Trivial refinement of \fullref{thm:abelian_group_categorical_limits}.
\end{proof}

\begin{definition}\label{def:linear_combination}
  Let \( M \) be a left \( R \)-module. Like \hyperref[def:polynomial]{polynomials}, we define linear combinations to be tuples \( (t_1, t_2, \ldots, t_n) \) of scalars from \( R \). Unlike with polynomials, we are not interested in defining operations on them, but rather defining the function
  \begin{equation}\label{def:linear_combination/function}
    (x_1, \ldots, x_n) \mapsto \sum_{k=1}^n t_k x_k.
  \end{equation}

  The scalars \( t_1, \ldots, t_n \) are called the \term{coefficients} of the linear combination. A linear combination is said to be \term{trivial} if all coefficients are equal to \( 0_R \).

  For convenience, given set of vectors \( x_1, \ldots, x_n \in M \), we also call the sum \( \sum_{k=1}^n t_k x_k \) a linear combination.

  In the special case where \( R \) is a superring of \( \BbbR \), we define the following special types of linear combinations:
  \begin{thmenum}
    \thmitem{def:linear_combination/affine} an \term{affine combination} if \( \sum_{k=1}^n t_k = 1 \).
    \thmitem{def:linear_combination/conic} a \term{conic combination} if all of the coefficients are nonnegative real numbers.
    \thmitem{def:linear_combination/convex} a \term{convex combination} if it is both affine and conic.
  \end{thmenum}
\end{definition}

\begin{definition}\label{def:module_linear_dependence}
  Let \( M \) be a left \( R \)-module and let \( A \subseteq M \). We say that the set \( A \) is \term{linearly independent} if for any linear \hyperref[def:linear_combination/function]{combination}, the equality
  \begin{equation*}
    \sum_{i=1}^n t_i x_k = 0_M
  \end{equation*}
  for \( x_1, \ldots, x_n \in A \) implies that the combination is trivial.

  We say that the vectors \( x_1, \ldots, x_n \) are linearly independent if the corresponding set \( \{ x_1, \ldots, x_n \} \) is linearly independent.

  We say that \( A \) is \term{linearly dependent} if it is not linearly independent.
\end{definition}

\begin{definition}\label{def:module_hamel_basis}
  A subset \( B \) of the left \( R \)-module \( M \) is called a \term{Hamel basis} or simply \term{basis} of \( M \) if \( B \) is a \hyperref[def:partially_ordered_set_extremal_points/maximal_and_minimal_element]{minimal} (with respect to set inclusion) linearly independent subset of \( M \).
\end{definition}

\begin{definition}\label{def:free_left_module}\mcite[377]{Knapp2016BasicAlgebra}
  We say that the left \( R \)-module \( M \) is a \term{free left module} if it has a \hyperref[def:module_hamel_basis]{basis}.

  Let \( S \) be any set. If we regard \( R \) as a left module over itself, then the \hyperref[thm:module_categorical_limits/coproduct]{direct sum}
  \begin{equation*}
    F(A) \coloneqq \oplus_{s \in S} R
  \end{equation*}
  with injections \( \{ \iota_s \}_{s \in S} \) is called the free left module \term{generated by \( S \)}. Define the function
  \begin{balign*}
     & \varphi: S \to F(A)                \\
     & \varphi(s) \coloneqq \iota_s(1_R).
  \end{balign*}

  The image \( \varphi(S) \) is then a basis of \( F(A) \).

  The cardinality of the basis of a free left module \( M \) is called the \term{rank} \( \rank M \) of \( M \). \Fullref{thm:left_module_basis_cardinality} tells us that this rank is unique for commutative unital rings. If the rank of a module is finite, we say that the module is \term{finitely generated}.

  We also denote \( F(A) = \braket S \), especially in finitely generated modules.
\end{definition}

\begin{proposition}\label{thm:free_module_is_free_functor}
  The functor \( F: \cat{Set} \to \cat{Mod}_R \), defined pointwise in \fullref{def:free_left_module}, is \hyperref[def:category_adjunction]{free}.
\end{proposition}

\begin{proposition}\label{thm:left_module_basis_decomposition}
  Let \( B \) be a basis of the free left \( R \)-module \( M \). Then each element \( u \) of \( M \) can be uniquely (up to a permutation) represented as a linear \hyperref[def:linear_combination]{combination} of elements of \( B \).
\end{proposition}
\begin{proof}
  Let
  \begin{equation*}
    u = \sum_{i=1}^n t_i v_i
  \end{equation*}
  and
  \begin{equation*}
    u = \sum_{j=1}^m s_i w_i
  \end{equation*}
  be two representations of \( u \) as a linear combination over \( B \).

  Define the function
  \begin{balign*}
     & t: M \to R                                \\
     & t(v) \coloneqq \begin{cases}
      t_i, & v = v_i,                          \\
      0,   & v \not\in \{ v_1, \ldots, v_n \}.
    \end{cases}
  \end{balign*}
  and analogously for \( s: M \to R \). Define the set
  \begin{equation*}
    B' \coloneqq \{ v_1, \ldots, v_n, w_1, \ldots, w_m \}.
  \end{equation*}

  Thus,
  \begin{equation*}
    u = \sum_{b \in B'} t(b) b = \sum_{b \in B'} s(b) b
  \end{equation*}
  and
  \begin{equation*}
    0 = u - u = \sum_{b \in B'} (t(b) - s(b)) b.
  \end{equation*}

  The set \( B' \) is linearly independent as a subset of the basis \( B \), hence only a trivial linear combination can be the zero vector. This gives us
  \begin{equation*}
    t(b) = s(b), b \in B'.
  \end{equation*}

  In particular, the two decompositions of \( u \) along \( B \) are identical up to a permutation.
\end{proof}

\begin{definition}\label{def:module_basis_projection}
  Let \( M \) be a left \( R \)-module and let \( B \) be a basis of \( M \). For each \( b \in B \), we define the \text{coordinate projection functional} \( \pi_b: M \to R \) that gives us the unique coefficient in the basis decomposition. Thus, for every \( x \in M \) we have
  \begin{equation*}
    x = \sum_{b \in B} \pi_b(x) b.
  \end{equation*}

  The sum is well-defined since only finitely many terms are nonzero.

  When the basis \( B \) is finite and ordered:
  \begin{equation*}
    B = \{ b_1, \ldots, b_n \},
  \end{equation*}
  we also write
  \begin{equation*}
    x = \sum_{i=1}^n x_k b_i.
  \end{equation*}
\end{definition}
\begin{proof}
  By \fullref{thm:left_module_basis_decomposition}, this decomposition is unique given a basis \( B \).
\end{proof}

\begin{proposition}\label{thm:left_module_basis_projections_are_linear}
  The basis projection \hyperref[def:module_basis_projection]{maps} are linear.
\end{proposition}
\begin{proof}
  \SubProofOf{def:semimodule/homomorphism/homogeneity} Let \( t \in R \) and \( x \in M \). We have the unique decompositions
  \begin{balign*}
    x  & = \sum_{b \in B} \pi_b(x) b,  \\
    tx & = \sum_{b \in B} \pi_b(tx) b.
  \end{balign*}

  Since both decompositions have only finitely many terms, their difference also has only finitely many nonzero terms. Thus,
  \begin{equation*}
    0
    =
    tx - tx
    =
    t \left( \sum_{b \in B} \pi_b(x) b \right) - \sum_{b \in B} \pi_b(tx) b
    =
    \sum_{b \in B} (t \pi_b(x) - \pi_b(tx)) b.
  \end{equation*}

  Since the vectors in \( B \) are linearly independent, no nontrivial linear combination can equal the zero vector. Thus, for all \( b \in B \),
  \begin{equation*}
    t \pi_b(x) = \pi_b(tx).
  \end{equation*}

  \SubProofOf{def:semimodule/homomorphism/additivity} Analogous.
\end{proof}

\begin{theorem}\label{thm:linear_map_iff_function_on_basis}
  Let \( M \) and \( N \) be left \( R \)-modules and let \( B \) be a basis of \( M \). Then there exists a bijection between the \hyperref[def:function]{functions} from \( B \) to \( N \) and the module \hyperref[def:semimodule/homomorphism]{homomorphisms} from \( M \) to \( N \), such that each linear map is an \hyperref[def:multi_valued_function/restriction]{extension} of the corresponding function.
\end{theorem}
\begin{proof}
  Let \( \varphi: M \to N \) be a homomorphism. Define the function
  \begin{balign*}
     & f: B \to N                 \\
     & f(b) \coloneqq \varphi(b).
  \end{balign*}

  Now define the linear map
  \begin{balign*}
     & \hat \varphi: M \to N                                   \\
     & \hat \varphi(x) \coloneqq \sum_{b \in B} \pi_b(x) f(b).
  \end{balign*}

  Since the projections \( \pi_b(x) \) are linear functions by \fullref{thm:left_module_basis_projections_are_linear} and since we only use the value of \( f \) on fixed vectors, it follows that \( \hat \varphi \) is also linear.

  It remains to show that \( \varphi = \hat \varphi \). For each \( x \in M \), by linearity of \( \varphi \) we have
  \begin{equation*}
    \hat \varphi(x)
    =
    \sum_{b \in B} \pi_b(x) f(b)
    =
    \sum_{b \in B} \pi_b(x) \varphi(b)
    =
    \varphi(x).
  \end{equation*}
\end{proof}

\begin{remark}\label{rem:linear_map_iff_function_on_basis}
  \Fullref{thm:linear_map_iff_function_on_basis} is very powerful in that is allows to study linear maps given their value at only a small subset of vectors. The connection between multilinear \hyperref[def:multilinear_function]{maps} and \hyperref[def:module_tensor_product]{tensors} is based on this idea.
\end{remark}

\begin{corollary}\label{thm:linear_maps_agree_on_free_module_if_they_agree_on_basis}
  If two linear maps from the free left module \( M \) to \( N \) agree on a basis of \( M \), they agree on the whole module.
\end{corollary}

\begin{proposition}\label{thm:left_module_basis_cardinality}\mcite{ProofWiki:bases_of_free_module_have_same_cardinality}
  All bases in a free left module over a commutative unital ring have the same cardinality.
\end{proposition}

\begin{definition}\label{def:module_tensor_product}\mcite[574]{Knapp2016BasicAlgebra}
  Let \( R \) be a unital ring. Let \( M \) be a right \( R \)-module and \( N \) be a left \( R \)-module. Define the free \hyperref[def:free_abelian_group]{abelian group} \( G \) generated by the basis \( M \times N \), that is,
  \begin{equation*}
    G \coloneqq \oplus_{(m,n) \in M \times N} \BbbZ.
  \end{equation*}

  Denote by \( e_{m,n} \) the \( (m,n) \)-th basis vector and by \( t_{m,n} \) the \( (m,n) \)-th coordinate of \( t \in G \) (we can have only a finite amount of nonzero coordinates since \( G \) is a direct sum).

  We can regard \( G \) as a left \( R \)-module with scalar multiplication given by
  \begin{equation*}
    (r t)_{(m,n)} \coloneqq t_{(rm,n)}.
  \end{equation*}

  Let \( H \) be the submodule of \( G \) generated by
  \begin{itemize}
    \item \( e_{(m_1 - m_2, n)} - e_{(m_1,n)} - e_{(m_2,n)} \), \( m_1, m_2, n \in G \)
    \item \( e_{(m, n_1 - n_2)} - e_{(m,n_1)} - e_{(m,n_2)} \), \( m, n_1, n_2 \in G \)
    \item \( e_{(rm,n)} - e_{(m,rn)} \), \( m, n \in G \) and \( r \in R \)
  \end{itemize}

  Define the \term{tensor product} of \( M \) and \( N \) to be the \( R \)-module
  \begin{equation*}
    M \otimes N \coloneqq G / H.
  \end{equation*}
\end{definition}

\begin{theorem}\label{thm:tensor_product_universal_property}\mcite[thm. 10.18]{Knapp2016BasicAlgebra}
  Let \( R \) be a unital ring. Let \( M \) be a right \( R \)-module and \( N \) be a left \( R \)-module and let \( M \otimes N \) be their \hyperref[def:module_tensor_product]{tensor product} with \( q: M \times N \to M \otimes N \) being the corresponding quotient map.

  The tensor product \( M \otimes N \) satisfies the following universal mapping property: for every \( R \)-module \( K \) and any bilinear \hyperref[def:multilinear_function]{map} \( f: M \times N \to K \) there exists a unique map \( \hat f: M \otimes N \to K \) such that
  \begin{equation*}
    f = \hat f \circ q,
  \end{equation*}
  that is, the following diagram commutes:

  \begin{alignedeq}\label{thm:tensor_product_universal_property/diagram}
    \text{\todo{Add diagram}}\iffalse\begin{mplibcode}
      beginfig(1);
      input metapost/graphs;

      v1 := thelabel("$M \times N$", origin);
      v2 := thelabel("$M \otimes N$", (2, 0) scaled u);
      v3 := thelabel("$K$", (1, -1) scaled u);

      a1 := straight_arc(v1, v2);
      a2 := straight_arc(v2, v3);
      a3 := straight_arc(v1, v3);

      draw_vertices(v);
      draw_arcs(a);

      label.top("$q$", straight_arc_midpoint of a1);
      label.lrt("$\hat f$", straight_arc_midpoint of a2);
      label.llft("$f$", straight_arc_midpoint of a3);
      endfig;
    \end{mplibcode}\fi
  \end{alignedeq}
\end{theorem}

\begin{proposition}\label{thm:tensor_product_with_underlying_ring}\mcite[677]{Knapp2016BasicAlgebra}
  Let \( R \) be a unital ring (regarded as a right module) and \( B \) be a left \( R \)-module. Their \hyperref[def:module_tensor_product]{tensor product} satisfies
  \begin{equation*}
    R \otimes M \cong M.
  \end{equation*}
\end{proposition}

\begin{definition}\label{def:algebra_over_ring}\mcite[408]{Knapp2016BasicAlgebra}
  Let \( R \) be a commutative unital ring. We say that the left \( R \)-module \( A \) is an \( R \)-\term{algebra} if we define an additional bilinear \term{vector multiplication} operation
  \begin{equation*}
    \odot: A \times A \to A
  \end{equation*}
  such that for \( x, y \in M \) and \( t \in R \)
  \begin{equation*}
    t \cdot (x \odot y) = (t \cdot x) \odot y = x \odot (t \cdot y).
  \end{equation*}

  Both vector and scalar multiplication are usually denoted by juxtaposition.

  If \( \odot \) is associative, commutative, unital or invertible, we add this prefix to \( A \), e.g. A is a commutative algebra if \( \odot \) is commutative.
\end{definition}

\begin{proposition}\label{thm:functions_over_ring_form_algebra}
  Let \( X \) be an arbitrary nonempty set and \( R \) be a commutative unital ring. Define
  \begin{equation*}
    A \coloneqq \cat{Set}(X, R)
  \end{equation*}
  to be the set of all functions from \( X \) to \( R \) (see \fullref{def:category_of_small_sets}). Then \( A \) is an \( R \)-algebra with the operations being defined pointwise, that is,
  \begin{balign*}
    [f + g](x)     & \coloneqq f(x) + g(x)     \\
    [f \odot g](x) & \coloneqq f(x) \circ g(x) \\
    [rf](x)        & \coloneqq r f(x)
  \end{balign*}

  We call the algebra \( A \) the \term{algebra of functions} from \( X \) to \( R \).

  If \( X \) itself has a ring structure, we consider the set of ring \hyperref[thm:ring_homomorphism_simpler_conditions]{homomorphisms}
  \begin{equation*}
    \cat{Ring}(X, R),
  \end{equation*}
  which is usually a strict subset of \( \cat{Set}(X, R) \). This set is usually denoted by \( \hom(X, R) \).

  If \( R \) is a module, but not necessarily a ring, then \( \cat{Set}(X, R) \) is a only module since we do not necessarily have multiplication. See \fullref{def:semimodule/homomorphism}.
\end{proposition}
