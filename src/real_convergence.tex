\subsection{Real convergence}\label{subsec:real_vector_space_convergence}

\begin{theorem}[Bolzano-Weierstrass]\label{def:bolzano_weierstrass}
  Every bounded sequence in \( \BbbR \) has a \hyperref[def:net_convergence/limit]{convergent} \hyperref[def:sequence]{subsequence}.
\end{theorem}
\begin{proof}
  Let \( \{ x_k \}_{k=1}^\infty \) be a bounded sequence in \( \BbbR \) and let \( a \leq b \) be a lower and upper \hyperref[def:preordered_set/upper_lower_bound]{bound}, respectively. Construct the sequence \( \{ F_k \}_{k=1}^\infty \) of closed intervals as follows: define \( \alpha_1 \coloneqq a \) and \( \beta_1 \coloneqq b \) and, at step \( k = 1, 2, \ldots \), put
  \begin{balign*}
    F_k \coloneqq \begin{cases}
      [\alpha_k, \tfrac{\alpha_k+\beta_k} 2], & [\alpha_k, \tfrac{\alpha_k+\beta_k} 2]\text{ contains infinitely many sequence members}, \\
      [\tfrac{\alpha_k+\beta_k} 2, \beta_k],  & \text{otherwise}.
    \end{cases}
  \end{balign*}

  Then put \( \alpha_{k+1} \) and \( \beta_{k+1} \) to be the endpoints of the interval \( F_k \) and repeat with \( k+1 \) instead of \( k \). Note that for any \( k = 1, 2, \ldots \), \( \diam(F_k) = \tfrac 1 2 \diam(F_{k-1}) \), thus \( \diam(F_k) \xrightarrow[k \to \infty]{} 0 \). As in \fullref{thm:cantors_nested_compact_theorem}, it follows that if we choose\AOC a sequence
  \begin{equation*}
    x_k \in F_k, k = 1, 2, \ldots,
  \end{equation*}
  it will be a fundamental sequence. Since the space is complete, this fundamental sequence necessarily converges.
\end{proof}

\begin{theorem}\label{def:real_numbers_complete_metric_space}
  The metric space \( \BbbR \) is complete.
\end{theorem}
\begin{proof}
  Let \( \{ x_k \}_{k=1}^\infty \) be a fundamental sequence of real numbers. By \fullref{thm:fundamental_sequence_is_bounded}, the sequence is bounded. By \fullref{def:bolzano_weierstrass}, it has a convergent subsequence
  \begin{equation*}
    \{ x_{k_m} \}_{m=1}^\infty \to x.
  \end{equation*}

  By \fullref{thm:fundamental_subsequence_convergence}, the sequence itself has the same limit \( \lim_{k \to \infty} x_k = x \).
\end{proof}

\begin{proposition}\label{thm:one_sided_squeeze_lemma}
  Fix two convergent sequences \( \{ x_k \}_{k=1}^\infty \) and \( \{ y_k \}_{k=1}^\infty \) of real numbers.

  If \( x_k \leq y_k \) for all \( k = 1, 2, \ldots \), then
  \begin{equation*}
    \lim_{k \to \infty} x_k \leq \lim_{k \to \infty} y_k.
  \end{equation*}
\end{proposition}
\begin{proof}
  Denote the respective limits by \( x \) and \( y \).

  Fix \( \varepsilon > 0 \). Then by \fullref{def:net_convergence/limit}, there exist indices \( k_0 \) and \( m_0 \) such that
  \begin{balign*}
     & \abs{x - x_k} < \frac \varepsilon 2 \quad\forall k \geq k_0 \\
     & \abs{y - y_m} < \frac \varepsilon 2 \quad\forall m \geq m_0
  \end{balign*}

  Take \( k \geq \max \{ k_0, m_0 \} \). Then \( y_k \geq x_k \) and
  \begin{balign*}
    y - x
     & =
    (y - y_k) + (y_k - x_k) + (x_k - x)
    \geq \\ &\geq
    (y - y_k) + (x - x_k)
    >    \\ &>
    - \frac \varepsilon 2 - \frac \varepsilon 2
    =
    - \varepsilon.
  \end{balign*}

  Since \( \varepsilon \) was chosen arbitrary, \( y - x \) cannot equal any negative number, because otherwise\LEM we could choose another \( \varepsilon \) smaller than the magnitude of the negative number and obtain a contradiction.

  Thus \( y \geq x \).
\end{proof}

\begin{lemma}[Squeeze lemma]\label{thm:squeeze_lemma}
  Let \( I \) be a closed \hyperref[def:total_order_interval/closed]{interval} in \( \BbbR \).

  \begin{lemenum}
    \ilabel{thm:squeeze_lemma/sequences} Let \( \{ x_k \}_{k=1}^\infty, \{ x_k^- \}_{k=1}^\infty, \{ x_k^+ \}_{k=1}^\infty \) be three sequences in \( I \). If both \( \{ x_k^- \}_{k=1}^\infty \) and \( \{ x_k^+ \}_{k=1}^\infty \) converge to the same value \( \overline x \in I \) and if the following inequalities
    \begin{equation*}
      x_k^- \leq x_k \leq x_k^+
    \end{equation*}
    hold for all \( k = 1, 2, \ldots \), then the \enquote{squeezed in} sequence \( \{ x_k \}_{k=1}^\infty \) also converges to \( \overline x \).

    \ilabel{thm:squeeze_lemma/functions} Let \( f, f_-, f_+: I \to \BbbR \) be three functions and let \( \overline x \in I \). If both limits \( \lim_{x \to \overline x} f_-(x) \) and \( \lim_{x \to \overline x} f_+(x) \) converge to the same value \( \overline y \in \BbbR \) and if the following inequalities
    \begin{equation*}
      f_-(x) \leq f(x) \leq f_+(x)
    \end{equation*}
    hold for all \( x \in I \), then the \enquote{squeezed in} function \( f \) also converges to \( \overline y \) at \( \overline x \).
  \end{lemenum}
\end{lemma}
\begin{proof}
  \SubProofOf{thm:squeeze_lemma/sequences} Fix \( \varepsilon > 0 \). Then by \fullref{def:net_convergence/limit}, there exist indices \( k^- \) and \( k^+ \) such that
  \begin{balign*}
     & \abs{\overline x - x_k^-} < \frac \varepsilon 3 \quad\forall k \geq k^- \\
     & \abs{\overline x - x_k^+} < \frac \varepsilon 3 \quad\forall k \geq k^+
  \end{balign*}

  By taking \( k \geq \max \{ k^-, k^+ \} \), we obtain
  \begin{equation*}
    \abs{x_k^+ - x_k^-} \leq \abs{x_k^+ - \overline x} + \abs{\overline x - x_k^-} < \frac 2 3 \varepsilon.
  \end{equation*}

  Since \( \abs{x_k^+ - x_k} \leq \abs{x_k^+ - x_k^-} \), it follows that \( \abs{x_k^+ - x_k} < \frac 2 3 \varepsilon \).

  Thus
  \begin{equation*}
    \abs{\overline x - x_k} \leq \abs{\overline x - x_k^+} + \abs{x_k^+ - x_k} < \varepsilon.
  \end{equation*}

  \Fullref{def:net_convergence/limit} is satisfied, hence \( \{ x_k \} \) converges to \( \overline x \).

  \SubProofOf{thm:squeeze_lemma/functions} The proof is analogous to that of \fullref{thm:squeeze_lemma/sequences}, but the machinery is different. Fix \( \varepsilon > 0 \). Then by \fullref{def:local_convergence/neighborhoods}, there exist radii \( \delta^- \) and \( \delta^+ \) such that
  \begin{balign*}
     & f_-(I \cap B(\overline x, \delta^-)) \subseteq B(\overline y, \tfrac \varepsilon 3) \\
     & f_+(I \cap B(\overline x, \delta^+)) \subseteq B(\overline y, \tfrac \varepsilon 3)
  \end{balign*}

  Take \( \delta < \min \{ \delta^-, \delta^+ \} \) and \( x \in I \cap B(\overline x, \delta) \). Analogously to the proof of \fullref{thm:squeeze_lemma/sequences}, we obtain the inequality
  \begin{equation*}
    \abs{f(x) - \overline x} \leq \abs{f(x) - f^-(x)} + \abs{f^-(x) - \overline x} < \varepsilon.
  \end{equation*}

  We conclude that
  \begin{equation*}
    f(I \cap B(\overline x, \delta)) \subseteq B(\overline y, \varepsilon)
  \end{equation*}
  holds and thus by \fullref{def:local_convergence/neighborhoods}, the function \( f \) converges to \( \overline y \) at \( \overline x \).
\end{proof}

\begin{proposition}\label{thm:real_monotone_sequence_converges_iff_bounded}
  A \hyperref[def:monotone_map]{monotone} sequence of real numbers \hyperref[def:net_convergence/limit]{converges} if and only if it is \hyperref[def:metric_space/bounded_sequence]{bounded}.
\end{proposition}
\begin{proof}
  \SufficiencySubProof Let \( \{ x_k \}_{k=1}^\infty \) be a convergent monotone sequence. Denote its limit by \( x \). Fix \( \varepsilon > 0 \). By \fullref{def:net_convergence/limit}, there exists \( k_0 \) such that
  \begin{equation*}
    \abs{x - x_k} < \varepsilon \quad\forall k \geq k_0.
  \end{equation*}

  Thus \( \{ x_k \colon k \geq k_0 \} \subseteq B(x, \varepsilon) \).

  Also note that
  \begin{equation*}
    \{ x_k \colon k < k_0 \} \subseteq B(x, \max_{i < k_0} \{ \abs{x - x_k} \}).
  \end{equation*}

  We obtained that the entire sequence
  \begin{equation*}
    \{ x_k \colon k \geq 1 \} = \{ x_k \colon k < k_0 \} \cup \{ x_k \colon k \geq k_0 \}
  \end{equation*}
  is contained in a union of two balls and is therefore bounded.

  \NecessitySubProof Now let \( \{ x_k \}_{k=1}^\infty \) be a bounded monotone sequence. Denote its supremum by \( \alpha \). Note that
  \begin{equation*}
    \abs{x_n - x_m} = x_n - x_m \leq \alpha \quad\forall n \geq m.
  \end{equation*}

  Fix \( \varepsilon > 0 \). Then there exists at least one element \( x_{m_0} > \alpha - \varepsilon \) because otherwise\LEM \( \alpha \) would not be a supremum.

  Then for any index \( n \geq m_0 \) we have
  \begin{equation*}
    \abs{x_n - x_{m_0}} = x_n - x_{m_0} < \alpha - (\alpha - \varepsilon) = \varepsilon.
  \end{equation*}

  Thus \fullref{def:net_convergence/limit} is satisfied and the sequence \( \{ x_k \}_{k=1}^\infty \) converges.
\end{proof}
