\subsection{Real convergence}\label{subsec:real_vector_space_convergence}

\begin{theorem}[Bolzano-Weierstrass]\label{def:bolzano_weierstrass}
  Every bounded sequence in \( \BR \) has a convergent\Tinyref{thm:metric_topology_convergence/limit_point} subsequence\Tinyref{def:sequence}.
\end{theorem}
\begin{proof}
  Let \( \{ x_i \}_{i=1}^\infty \) be a bounded sequence in \( \BR \) and let \( a \leq b \) be a lower and upper bound\Tinyref{def:poset/upper_lower_bound}, respectively. Construct the sequence \( \{ F_i \}_{i=1}^\infty \) of closed intervals as follows: define \( \alpha_1 \coloneqq a \) and \( \beta_1 \coloneqq b \) and, at step \( k = 1, 2, \ldots \), put
  \begin{align*}
    F_k \coloneqq \begin{cases}
      [\alpha_k, \tfrac{\alpha_k+\beta_k} 2], &[\alpha_k, \tfrac{\alpha_k+\beta_k} 2]\text{ contains infinitely many sequence members}, \\
      [\tfrac{\alpha_k+\beta_k} 2, \beta_k], &\text{otherwise}.
    \end{cases}
  \end{align*}

  Then put \( \alpha_{k+1} \) and \( \beta_{k+1} \) to be the endpoints of the interval \( F_k \) and repeat with \( k+1 \) instead of \( k \). Note that for any \( i = 1, 2, \ldots \), \( \Diam(F_i) = \tfrac 1 2 \Diam(F_{i-1}) \), thus \( \Diam(F_i) \xrightarrow[i \to \infty]{} 0 \). As in \cref{thm:cantors_nested_compact_theorem}, it follows that if we choose\AOC a sequence
  \begin{equation*}
    x_i \in F_i, i = 1, 2, \ldots,
  \end{equation*}
  it will be a fundamental sequence. Since the space is complete, this fundamental sequence necessarily converges.
\end{proof}

\begin{theorem}\label{def:real_numbers_complete_metric_space}
  The metric space \( \BR \) is complete.
\end{theorem}
\begin{proof}
  Let \( \{ x_i \}_{i=1}^\infty \) be a fundamental sequence of real numbers. By \cref{thm:fundamental_sequence_is_bounded}, the sequence is bounded. By \cref{def:bolzano_weierstrass}, it has a convergent subsequence
  \begin{equation*}
    \{ x_{i_k} \}_{k=1}^\infty \to x.
  \end{equation*}

  By \cref{thm:fundamental_subsequence_convergence}, the sequence itself has the same limit \( \lim_{i \to \infty} x_i = x \).
\end{proof}

\begin{proposition}\label{thm:one_sided_squeeze_lemma}
  Fix two convergent sequences \( \{ x_i \}_{i=1}^\infty \) and \( \{ y_i \}_{i=1}^\infty \) of real numbers.

  If \( x_i \leq y_i \) for all \( i = 1, 2, \ldots \), then
  \begin{equation*}
    \lim_{i \to \infty} x_i \leq \lim_{i \to \infty} y_i.
  \end{equation*}
\end{proposition}
\begin{proof}
  Denote the respective limits by \( x \) and \( y \).

  Fix \( \varepsilon > 0 \). Then by \cref{thm:metric_topology_convergence/limit_point}, there exist indices \( i_0 \) and \( j_0 \) such that
  \begin{align*}
    &i \geq i_0 \implies \Abs{x - x_i} < \frac \varepsilon 2 \\
    &j \geq j_0 \implies \Abs{y - y_j} < \frac \varepsilon 2
  \end{align*}

  Take \( k \geq \max \{ i_0, j_0 \} \). Then \( y_k \geq x_k \) and
  \begin{align*}
    y - x
    &=
    (y - y_k) + (y_k - x_k) + (x_k - x)
    \geq \\ &\geq
    (y - y_k) + (x - x_k)
    > \\ &>
    - \frac \varepsilon 2 - \frac \varepsilon 2
    =
    - \varepsilon.
  \end{align*}

  Since \( \varepsilon \) was chosen arbitrary, \( y - x \) cannot equal any negative number, because otherwise\LEM we could choose another \( \varepsilon \) smaller than the magnitude of the negative number and obtain a contradiction.

  Thus \( y \geq x \).
\end{proof}

\begin{lemma}[Squeeze lemma]\label{thm:squeeze_lemma}
  Let \( I \) be a closed interval\Tinyref{def:total_order_interval/closed} in \( \BR \).

  \begin{lemenum}
    \DItem{thm:squeeze_lemma/sequences} Let \( \{ x_i \}_{i=1}^\infty, \{ x_i^- \}_{i=1}^\infty, \{ x_i^+ \}_{i=1}^\infty \) be three sequences in \( I \). If both \( \{ x_i^- \}_{i=1}^\infty \) and \( \{ x_i^+ \}_{i=1}^\infty \) converge to the same value \( \Ol x \in I \) and if the following inequalities
    \begin{equation*}
      x_i^- \leq x_i \leq x_i^+
    \end{equation*}
    hold for all \( i = 1, 2, \ldots \), then the \enquote{squeezed in} sequence \( \{ x_i \}_{i=1}^\infty \) also converges to \( \Ol x \).

    \DItem{thm:squeeze_lemma/functions} Let \( f, f_-, f_+: I \to \BR \) be three functions and let \( \Ol x \in I \). If both limits \( \lim_{x \to \Ol x} f_-(x) \) and \( \lim_{x \to \Ol x} f_+(x) \) converge to the same value \( \Ol y \in \BR \) and if the following inequalities
    \begin{equation*}
      f_-(x) \leq f(x) \leq f_+(x)
    \end{equation*}
    hold for all \( x \in I \), then the \enquote{squeezed in} function \( f \) also converges to \( \Ol y \) at \( \Ol x \).
  \end{lemnum}
\end{lemma}
\begin{proof}
  \begin{description}
    \RItem{thm:squeeze_lemma/sequences} Fix \( \varepsilon > 0 \). Then by \cref{thm:metric_topology_convergence/limit_point}, there exist indices \( i^- \) and \( i^+ \) such that
    \begin{align*}
      &i \geq i^- \implies \Abs{\Ol x - x_i^-} < \frac \varepsilon 3 \\
      &i \geq i^+ \implies \Abs{\Ol x - x_i^+} < \frac \varepsilon 3
    \end{align*}

    By taking \( i \geq \max \{ i^-, i^+ \} \), we obtain
    \begin{equation*}
      \Abs{x_i^+ - x_i^-} \leq \Abs{x_i^+ - \Ol x} + \Abs{\Ol x - x_i^-} < \frac 2 3 \varepsilon.
    \end{equation*}

    Since \( \Abs{x_i^+ - x_i} \leq \Abs{x_i^+ - x_i^-} \), it follows that \( \Abs{x_i^+ - x_i} < \frac 2 3 \varepsilon \).

    Thus
    \begin{equation*}
      \Abs{\Ol x - x_i} \leq \Abs{\Ol x - x_i^+} + \Abs{x_i^+ - x_i} < \varepsilon.
    \end{equation*}

    \Cref{thm:metric_topology_convergence/limit_point} is satisfied, hence \( \{ x_i \} \) converges to \( \Ol x \).

    \RItem{thm:squeeze_lemma/functions} The proof is analogous to that of \cref{thm:squeeze_lemma/sequences}, but the machinery is different. Fix \( \varepsilon > 0 \). Then by \cref{def:convergence_of_function_at_point/balls}, there exist radii \( \delta_- \) and \( \delta_+ \) such that
    \begin{align*}
      &f_-(I \cap B(\Ol x, \delta_-)) \subseteq B(\Ol y, \tfrac \varepsilon 3) \\
      &f_+(I \cap B(\Ol x, \delta_+)) \subseteq B(\Ol y, \tfrac \varepsilon 3)
    \end{align*}

    Take \( \delta < \min \{ \delta_-, \delta_+ \} \) and \( x \in I \cap B(\Ol x, \delta) \). Analogously to the proof of \cref{thm:squeeze_lemma/sequences}, we obtain the inequality
    \begin{equation*}
      \Abs{f(x) - \Ol x} \leq \Abs{f(x) - f^-(x)} + \Abs{f^-(x) - \Ol x} < \varepsilon.
    \end{equation*}

    We conclude that
    \begin{equation*}
      f(I \cap B(\Ol x, \delta)) \subseteq B(\Ol y, \varepsilon)
    \end{equation*}
    holds and thus by \cref{def:convergence_of_function_at_point/balls}, the function \( f \) converges to \( \Ol y \) at \( \Ol x \).
  \end{description}
\end{proof}

\begin{proposition}\label{thm:real_monotone_sequence_converges_iff_bounded}
  A monotone\Tinyref{def:monotone_map} sequence of real numbers converges\Tinyref{thm:metric_topology_convergence/limit_point} if and only if it is bounded\Tinyref{def:metric_space/bounded_sequence}.
\end{proposition}
\begin{proof}
  \begin{description}
    \Implies Let \( \{ x_i \}_{i=1}^\infty \) be a convergent monotone sequence. Denote its limit by \( x \). Fix \( \varepsilon > 0 \). By \cref{thm:metric_topology_convergence/limit_point}, there exists \( n_0 \) such that
    \begin{equation*}
      n \geq n_0 \implies \Abs{x - x_n} < \varepsilon.
    \end{equation*}

    Thus \( \{ x_i \colon i \geq n_0 \} \subseteq B(x, \varepsilon) \).

    Also note that
    \begin{equation*}
      \{ x_i \colon i < n_0 \} \subseteq B(x, \max_{i < n_0} \{ \Abs{x - x_i} \}).
    \end{equation*}

    We obtained that the entire sequence
    \begin{equation*}
      \{ x_i \colon i \geq 1 \} = \{ x_i \colon i < n_0 \} \cup \{ x_i \colon i \geq n_0 \}
    \end{equation*}
    is contained in a union of two balls and is therefore bounded.

    \ImpliedBy Now let \( \{ x_i \}_{i=1}^\infty \) be a bounded monotone sequence. Denote its supremum by \( \alpha \). Note that
    \begin{equation*}
      n \geq m \implies \Abs{x_n - x_m} = x_n - x_m \leq \alpha.
    \end{equation*}

    Fix \( \varepsilon > 0 \). Then there exists at least one element \( x_{m_0} > \alpha - \varepsilon \) because otherwise\LEM \( \alpha \) would not be a supremum.

    Then for any index \( n \geq m_0 \) we have
    \begin{equation*}
      \Abs{x_n - x_{m_0}} = x_n - x_{m_0} < \alpha - (\alpha - \varepsilon) = \varepsilon.
    \end{equation*}

    Thus \cref{thm:metric_topology_convergence/limit_point} is satisfied and the sequence \( \{ x_i \}_{i=1}^\infty \) converges.
  \end{description}
\end{proof}
