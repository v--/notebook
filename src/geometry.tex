\section{Geometry}\label{sec:geometry}

Geometry is the multi-millennial evolution of ancient attempts to measure everyday objects. It is possibly also the main historical justification for the gradual axiomatization of mathematics. Completely abstract results about shapes date at least as early as in Ancient Greece.

An axiomatic approach for a theory of figures in the place and space was developed by Euclid in the third century BC. Later, Hilbert, Tarski and others independently proposed axioms that fit the requirements of \hyperref[sec:mathematical_logic]{modern logical systems}. This is known today as \term{synthetic Euclidean geometry} and is mostly of theoretical interest. Modern tools based on topology and abstract algebra are easier to work with.

An important distinction between ancient and modern geometry is the introduction of coordinates in the 17th century. Descartes' idea of coordinates connects problems of algebra and geometry in such a way that most of today's mathematics seamlessly switches between algebraic and geometric interpretations of the same problem. The study of classical Greek geometry in terms of coordinates is known as \term{analytic geometry}. We will describe very basic analytic geometry in the Euclidean plane in \fullref{subsec:affine_coordinate_system}. We will then transition to more modern concepts in \fullref{subsec:vector_space_geometry} before returning to the Euclidean plane in a modernized form in \fullref{subsec:analytic_geometry_in_the_plane}.

Finally, we will very briefly touch two contemporary topics in geometry --- differential geometry in \fullref{subsec:manifolds} and algebraic geometry in \fullref{subsec:quadratic_curves}.
