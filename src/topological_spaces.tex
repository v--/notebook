\begin{definition}\label{def:topological_space}\cite[21]{Lectures:general_topology}
  Let $X$ be any set and $\Cal{T} \subseteq \Power(X)$ be a family of subsets of $X$. $\Cal{T}$ is called a \uline{topology on $X$} and the tuple $(X, \Cal{T})$ is said to be a \uline{topological space} if the following axioms are satisfied:
  \begin{description}
    \DItem{O1}{def:topological_space/O1} $\varnothing, X \in \Cal{T}$
    \DItem{O2}{def:topological_space/O2} $U, V \in \Cal{T} \implies U \cap V \in \Cal{T}$
    \DItem{O3}{def:topological_space/O3} $\Cal{T}' \subseteq \Cal{T} \implies \bigcap \Cal{T}' \in \Cal{T}$
  \end{description}

  If the topology is obvious from the context, we say that $X$ is a topological space.

  Elements of the set $X$ are called \uline{points of the topological space}, elements of $\Cal{T}$ are called \uline{open sets} and set-theoretic complements of open sets are called \uline{closed sets}.

  If $x \in U \in \Cal{T}$, we say that $U$ is a \uline{neighborhood of $x$}. Note that some authors (e.g.~\cite[38]{Kelley1955}) alternatively define neighborhoods as arbitrary sets that contain an open set that contains $x$.

  Dually, we can define the family $\Cal{F}$ of closed sets, where
  \begin{description}
    \DItem{F1}{def:topological_space/F1} $\varnothing, X \in \Cal{F}$
    \DItem{F2}{def:topological_space/F2} $U, V \in \Cal{F} \implies U \cup V \in \Cal{F}$
    \DItem{F3}{def:topological_space/F3} $\Cal{F}' \subseteq \Cal{F} \implies \bigcup \Cal{F}' \in \Cal{F}$
  \end{description}
\end{definition}

It is sometimes easier to define a topology $\Cal{T}$ via a subset of $\Cal{T}$. We will gradually construct a topology from a bare family of sets in $X$. First, we will give two definitions for a base, one on which does not require an existing topology.

\begin{definition}\label{def:topological_base}\cite[23]{Lectures:general_topology}
  Fix a topological space $(X, \Cal{T})$. We say that the family $\Cal{B} \subseteq \Cal{T}$ is a \uline{base for the topology $\Cal{T}$} if $\Cal{B}$ satisfies any of the equivalent conditions
  \begin{defenum}
    \item\label{def:topological_base/union} Every open set $U \in \Cal{T}$ is the union $U = \bigcup \Cal{B}'$ of some subset $\Cal{B}' = \Cal{B}$
    \item\label{def:topological_base/subset} For any point $x \in X$ and for any neighborhood $U$ of $x$ there exists a set $V \in \Cal{B}$ in the base such that $x \in V \subseteq U$
    \item\label{def:topological_base/axioms}
    \begin{description}
      \DItem{B1}{def:topological_base/B1} $\bigcup \Cal{B} = X$
      \DItem{B2}{def:topological_base/B2} $\forall U, V \in \Cal{B}, \forall x \in U \cap V, \exists W \in \Cal{B}: x \in W \subseteq U \cap V$
    \end{description}
    only if the topology satisfies
    \begin{equation}\label{def:topological_base/topology}
      \Cal{T} = \left\{ \bigcup \Cal{B}' \colon \Cal{B}' \subseteq \Cal{B} \right\}.
    \end{equation}
  \end{defenum}
\end{definition}
\begin{proof}
  (\ref{def:topological_base/union} $\implies$ \ref{def:topological_base/subset}) Fix a point $x \in X$ and a neighborhood $U \in \Cal{T}$ of $x$. Let $\Cal{B}'$ be a subfamily of $\Cal{B}$ such that
  \begin{align*}
    U = \bigcup \Cal{B}'.
  \end{align*}

  Then $x \in V$ for at least one $V \in \Cal{B}'$.

  (\ref{def:topological_base/subset} $\implies$ \ref{def:topological_base/union}) Fix an open set $U \in \Cal{T}$. Then for every $x \in U$, there exists a set $V_x \in \Cal{B}$ such that $x \in V_x \subseteq U$. We have
  \begin{align*}
    \bigcup_{x \in U} V_x \subseteq U \subseteq \bigcup_{x \in U} V_x,
  \end{align*}
  thus
  \begin{align*}
    U = \bigcup_{x \in U} V_x.
  \end{align*}

  (\ref{def:topological_base/union} $\implies$ \ref{def:topological_base/axioms})
  \begin{description}
    \item[\ref{def:topological_base/B1}] Implied by \cref{def:topological_base/union} with $U = X$.
    \item[\ref{def:topological_base/B2}] Fix $U, V \in \Cal{B}$. Note that $U, V \in \Cal{T}$ and $U \cap V \in \Cal{T}$. By \cref{def:topological_base/subset} (which is implied by \cref{def:topological_base/union}), for every $x \in U \cap V$, there exists $W \in \Cal{B}$ such that $x \in W \subseteq U \cap V$.
  \end{description}

  (\ref{def:topological_base/axioms} $\implies$ \ref{def:topological_base/union}) For any $U \in \Cal{T}$, by~\cref{def:topological_base/topology}, there exists a subfamily $\Cal{B}' \subseteq \Cal{T}$ such that
  \begin{align*}
    U = \bigcup \Cal{B}'.
  \end{align*}
\end{proof}

\begin{proposition}
  Let $X$ be an arbitrary set and let $\Cal{B}$ be a family of subset that satisfies \ref{def:topological_base/B1} and \ref{def:topological_base/B2}. Define $\Cal{T}$ by \cref{def:topological_base/topology}.

  Then $\Cal{T}$ is a topology on $X$. Furthermore, $\Cal{B}$ is a base of $\Cal{T}$.
\end{proposition}
\begin{proof}
  We will first prove that $\Cal{T}$ is indeed a topology.

  \begin{description}
    \item[\ref{def:topological_space/O1}] $\varnothing = \bigcup \varnothing \in \tau$ and $X = \bigcup \Cal{B} \in \Cal{T}$ (by~\ref{def:topological_base/B1})

    \item[\ref{def:topological_space/O3}] Fix $\Cal{T}' = \{ U_\alpha \colon \alpha \in A \} \subseteq \Cal{T}$. By~\cref{def:topological_base/union}, every set $U_\alpha$ has a corresponding subfamily $\Cal{B}_\alpha$ of $\Cal{B}$ such that $U_\alpha = \bigcup \Cal{B}_\alpha$.

    Define $\Cal{B}' \coloneqq \bigcup_{\alpha \in A} \Cal{B}_\alpha$. Obviously $\Cal{B}' \subseteq \Cal{B}$ and thus, by~\ref{def:topological_base/B1}, $\bigcup \Cal{B} \in \Cal{T}$.

    \item[\ref{def:topological_space/O2}] Fix $U, V \in \Cal{T}$ and families $\Cal{B}_U, \Cal{B}_V \subseteq \Cal{B}$ such that $U = \bigcup \Cal{B}_U$ and $V = \bigcup \Cal{B}_V$.

    Fix arbitrary $U' \in \Cal{B}_U$ and $V' \in \Cal{B}_V$. We will show that $U' \cap V' \in \tau$.

    By~\ref{def:topological_base/B2}, for every $x \in U' \cap V'$ there exists a neighborhood $W_x$ of $x$ such that $W \subseteq U' \cap V'$.

    The family $\Cal{B}_{U',V'} \coloneqq \{ W_x \colon x \in U' \cap V' \}$~\AOC is a subfamily of $\Cal{B}$ and thus $U' \cap V' = \bigcup \Cal{B}_{U',V'} \in \Cal{T}$.

    Hence, by~\ref{def:topological_space/O3}, $U \cap V \in \tau$.
  \end{description}

  Now, since $\Cal{B}$ satisfies \ref{def:topological_base/B1} and \ref{def:topological_base/B2} and is a subset of $\Cal{T}$, by~\cref{def:topological_base} it is a base for $\Cal{T}$.
\end{proof}

\begin{definition}\label{def:topological_subbase}\cite[23]{Lectures:general_topology}
  Fix a topological space $(X, \Cal{T})$. We say that the family $\Cal{P} \subseteq \Cal{T}$ is a \uline{subbase for the topology $\Cal{T}$} if $\FI(\Cal{P})$\Tinyref{def:finite_intersection_operator}) is a base\Tinyref{def:topological_base} of $\Cal{T}$.
\end{definition}

\begin{proposition}
  Fix a set $X$ and a family of subsets $\Cal{P} \subseteq \Power(X)$. The family $\Cal{P}' \coloneqq \Cal{P} \cup X$ is then a subbase\Tinyref{def:topological_subbase} of some topology on $X$.
\end{proposition}

\begin{definition}\label{def:topological_local_base}\cite[31]{Lectures:general_topology}
  Fix a topological space $(X, \Cal{T})$ and a point $x \in X$. We say that the family $\Cal{B}(x)$ is a \uline{local base for $\Cal{T}$ at $x$} if for every neighborhood $U$ of $x$ there exists a set $V \in \Cal{B}(x)$ such that $x \in V \subseteq U$.

  The indexed family of local bases $\{ \Cal{B}(x) \colon x \in X \}$ is called a \uline{neighborhood system} of $\Cal{T}$.
\end{definition}

\begin{proposition}\label{def:topological_local_base_axioms}\cite[31]{Lectures:general_topology}
  Fix a topological space $(X, \Cal{T})$ and a neighborhood system $\{ \Cal{B}(x) \colon x \in X \}$. The following axioms hold:
  \begin{description}
    \DItem{BN1}{def:topological_base/BN1} For every $x \in X$, $\Cal{B}(x) \neq \varnothing$ and $x \in U$ for every $U \in \Cal{B}(x)$.
    \DItem{BN2}{def:topological_base/BN2} For every $x \in X$ and for all $U, V \in \Cal{B}(x)$, $\exists W \in \Cal{B}: W \subseteq U \cap V$.
    \DItem{BN3}{def:topological_base/BN3} For all $x, y \in X$, $x \in U \in \Cal{B}(y)$ implies that there exists $V \in \Cal{B}(x)$ such that $U \subseteq V$.
  \end{description}
\end{proposition}
