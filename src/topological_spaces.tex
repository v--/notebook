\subsection{Topological spaces}\label{subsec:topological_spaces}

\begin{definition}\label{def:topological_space}\mcite[11]{Engelking1989}
  Let \( X \) be any set and \( \mscrT \subseteq \pow(X) \) be a family of subsets of \( X \). \( \mscrT \) is called a \term{topology} on \( X \) and the tuple \( (X, \mscrT) \) is said to be a \term{topological space} if the following axioms are satisfied:
  \begin{thmenum}
    \thmitem[def:topological_space/O1]{O1} \( \varnothing, X \in T \)
    \thmitem[def:topological_space/O2]{O2} \( U, V \in T \implies U \cap V \in T \)
    \thmitem[def:topological_space/O3]{O3} \( \mscrT' \subseteq T \implies \bigcap \mscrT' \in T \)
  \end{thmenum}

  If the topology is obvious from the context, we say that \( X \) is a topological space.

  Elements of the set \( X \) are called \term{points} of the topological space, elements of \( \mscrT \) are called \term{open sets} and set-theoretic complements of open sets are called \term{closed sets}.

  If \( x \in U \in T \), we say that \( U \) is a \term{neighborhood} of \( x \). Note that some authors (e.g. \cite[38]{Kelley1955}) alternatively define neighborhoods as arbitrary sets that contain an open set that contains \( x \). For simplicity, we define the subfamily
  \begin{equation*}
    \mscrT(x) \coloneqq \{ U \in T \colon x \in U \}.
  \end{equation*}

  We say that \( U \) is a \term{punctured neighborhood} of \( x \) if \( U \cup \{ x \} \) is an open set and, consequently, a neighborhood of \( x \).

  Dually, we can define the family \( \mscrF \) of closed sets, where
  \begin{thmenum}
    \thmitem[def:topological_space/F1]{F1} \( \varnothing, X \in F \)
    \thmitem[def:topological_space/F2]{F2} \( U, V \in F \implies U \cup V \in F \)
    \thmitem[def:topological_space/F3]{F3} \( \mscrF' \subseteq F \implies \bigcup \mscrF' \in F \)
  \end{thmenum}

  If \( (X, \mscrT) \) is a topological space, we denote the corresponding family of closed sets by
  \begin{equation*}
    \mscrF_\mscrT \coloneqq \{ X \setminus U \colon U \in T \}.
  \end{equation*}
\end{definition}

\begin{definition}\label{def:standard_topologies}
  On a space \( X \), we can explicitly define the following standard topologies:
  \begin{thmenum}
    \thmitem{def:standard_topologies/discrete} The \term{discrete topology} \( \mscrT \coloneqq \pow(X) \).
    \thmitem{def:standard_topologies/indiscrete} The \term{indiscrete topology} \( \mscrT \coloneqq \{ \varnothing, X \} \).
    \thmitem{def:standard_topologies/co_cardinal} For any \hyperref[def:cardinal]{cardinal} \( \xi \), the \term{co-\( \xi \) topology} \( \mscrT \coloneqq \{ A \subseteq X \colon \card A < \xi \} \) and, in particular, \term{cofinite} (\( \xi = \aleph_0 \)) and \term{cocountable} (\( \xi = c \)) topologies.
  \end{thmenum}

  For a deeper connection between discrete and indiscrete topologies, see \fullref{ex:def:category_adjunction/set_top}.
\end{definition}

\begin{proposition}\label{thm:set_open_iff_neighborhood_is_contained}
  A set \( A \) is open if and only if every point of \( A \) has a neighborhood \( U \) such that \( U \subseteq A \).
\end{proposition}
\begin{proof}
  This holds vacuously for empty sets. Assume that \( A \subseteq X \) is nonempty.

  \SufficiencySubProof Assume that \( A \) is open and let \( x_0 \in A \). Then \( A \) is a neighborhood of \( x_0 \) and the theorem holds trivially.
  \NecessitySubProof Assume that every point \( x \in A \) has a neighborhood \( U_x \) such that \( U_x \subseteq A \). Take the union
  \begin{equation*}
    B \coloneqq \cup_{x \in A} U_x.
  \end{equation*}

  Obviously \( B \subseteq A \). Aiming at a contradiction, suppose that \( y_0 \in A \setminus B \). Then \( y_0 \) has a neighborhood \( U_{y_0} \) such that \( U_{y_0} \setminus B \) is nonempty. But this is impossible by the definition of \( B \). The obtained contradiction proves \( B = A \).
\end{proof}

\begin{remark}\label{rem:abritrary_family_to_topology}
  It is sometimes easier to define a topology \( \mscrT \) via a subset of \( \mscrT \). We will gradually construct a topology from a bare family of sets in \( X \). First, we will give two definitions for a base, one on which does not require an existing topology.
\end{remark}

\begin{definition}\label{def:topological_base}\mcite[12]{Engelking1989}
  Fix a topological space \( (X, \mscrT) \). We say that the family \( \mscrB \subseteq T \) is a \term{base} for the topology \( \mscrT \) if \( \mscrB \) satisfies any of the equivalent conditions:
  \begin{thmenum}
    \thmitem{def:topological_base/union} Every open set \( U \in T \) is the union \( U = \bigcup \mscrB' \) of some subset \( \mscrB' = \mscrB \)
    \thmitem{def:topological_base/subset} For any point \( x \in X \) and for any neighborhood \( U \) of \( x \) there exists a set \( V \in B \) in the base such that \( x \in V \subseteq U \)
  \end{thmenum}
\end{definition}
\begin{proof}
  \ImplicationSubProof{def:topological_base/union}{def:topological_base/subset} Fix a point \( x \in X \) and a neighborhood \( U \in T \) of \( x \). Let \( \mscrB' \) be a subfamily of \( \mscrB \) such that
  \begin{equation*}
    U = \bigcup \mscrB'.
  \end{equation*}

  Then \( x \in V \) for at least one \( V \in B' \).

  \ImplicationSubProof{def:topological_base/subset}{def:topological_base/union} Fix an open set \( U \in T \). Then for every \( x \in U \), there exists a set \( V_x \in B \) such that \( x \in V_x \subseteq U \). We have
  \begin{equation*}
    \bigcup_{x \in U} V_x \subseteq U \subseteq \bigcup_{x \in U} V_x,
  \end{equation*}
  thus
  \begin{equation*}
    U = \bigcup_{x \in U} V_x.
  \end{equation*}
\end{proof}

\begin{proposition}\label{thm:topological_base_axioms}\mcite[12]{Engelking1989}
  Let \( X \) be an arbitrary set and let \( \mscrB \) be a family of subset that satisfies
  \begin{thmenum}
    \thmitem[thm:topological_base_axioms/B1]{B1} \( \bigcup \mscrB = X \)
    \thmitem[thm:topological_base_axioms/B2]{B2} \( \forall U, V \in B, \forall x \in U \cap V, \exists W \in B: x \in W \subseteq U \cap V \)
  \end{thmenum}

  Then the family
  \begin{balign}\label{thm:topological_base_axioms/topology}
    \mscrT \coloneqq \left\{ \bigcup \mscrB' \colon \mscrB' \subseteq B \right\}
  \end{balign}
  is a topology on \( X \). Furthermore, \( \mscrB \) is a \hyperref[def:topological_base]{base} of \( \mscrT \).

  In particular, the base on any topology satisfies \fullref{thm:topological_base_axioms/B1} -- \fullref{thm:topological_base_axioms/B2}.
\end{proposition}
\begin{proof}
  We will first prove that \( \mscrT \) is indeed a topology.

  \begin{refenum}
    \refitem{def:topological_space/O1} \( \varnothing = \bigcup \varnothing \in \tau \) and \( X = \bigcup \mscrB \in T \) (by \fullref{thm:topological_base_axioms/B1})

    \refitem{def:topological_space/O3} Fix \( \mscrT' = \{ U_\alpha \colon \alpha \in A \} \subseteq T \). By \fullref{def:topological_base/union}, every set \( U_\alpha \) has a corresponding subfamily \( \mscrB_\alpha \) of \( \mscrB \) such that \( U_\alpha = \bigcup \mscrB_\alpha \).

    Define \( \mscrB' \coloneqq \bigcup_{\alpha \in A} \mscrB_\alpha \). Obviously \( \mscrB' \subseteq B \) and thus, by \fullref{thm:topological_base_axioms/B1}, \( \bigcup \mscrB \in T \).

    \refitem{def:topological_space/O2} Fix \( U, V \in T \) and families \( \mscrB_U, \mscrB_V \subseteq B \) such that \( U = \bigcup \mscrB_U \) and \( V = \bigcup \mscrB_V \).

    Fix arbitrary \( U' \in B_U \) and \( V' \in B_V \). We will show that \( U' \cap V' \in \tau \).

    By \fullref{thm:topological_base_axioms/B2}, for every \( x \in U' \cap V' \) there exists a neighborhood \( W_x \) of \( x \) such that \( W \subseteq U' \cap V' \).

    The family \( \mscrB_{U',V'} \coloneqq \{ W_x \colon x \in U' \cap V' \} \) is a subfamily of \( \mscrB \) and thus \( U' \cap V' = \bigcup \mscrB_{U',V'} \in T \).

    Hence, by \fullref{def:topological_space/O3}, \( U \cap V \in \tau \).

    Now, for any \( U \in T \), by \fullref{thm:topological_base_axioms/topology}, there exists a subfamily \( \mscrB' \subseteq B \) such that
    \begin{equation*}
      U = \bigcup \mscrB'.
    \end{equation*}

    Hence, \( \mscrB \) is a base for \( \mscrT \).
  \end{refenum}
\end{proof}

\begin{definition}\label{def:topological_space_weight}
  We define the \term{weight} of \( (X, \mscrT) \) as the cardinal
  \begin{equation*}
    w((X, \mscrT)) \coloneqq \min \{ \abs{\mscrB} \colon \mscrB \text{ is a base for } \mscrT \}.
  \end{equation*}

  We simply write \( w(X) \) when the topology is clear from the context.

  Spaces for which \( w(X) \leq \aleph_0 \) are said to be \term{second-countable}.
\end{definition}
\begin{proof}
  The definition is correct because cardinals are well-ordered when regarded as initial ordinals.
\end{proof}

\begin{definition}\label{def:topological_subbase}\mcite[12]{Engelking1989}
  Fix a topological space \( (X, \mscrT) \). We say that the family \( P \subseteq T \) is a \term{subbase} for the topology \( \mscrT \) if the family
  \begin{equation*}
    \mscrB \coloneqq \left\{ \bigcap P' \colon P' \text{ is a nonempty \hyperref[def:set_finiteness]{finite} subset of } P \right\}
  \end{equation*}
  of finite intersections of \( P \) is a \hyperref[def:topological_base]{base} of \( \mscrT \).
\end{definition}

\begin{proposition}\label{thm:subbase_from_arbitrary_family}
  Fix a set \( X \) and a family of subsets \( P \subseteq \pow(X) \). The family \( P' \coloneqq P \cup X \) is then a \hyperref[def:topological_subbase]{subbase} of some topology on \( X \).
\end{proposition}

\begin{definition}\label{def:topological_local_base}\mcite[12]{Engelking1989}
  Fix a topological space \( (X, \mscrT) \) and a point \( x \in X \). We say that the family \( \mscrB(x) \subseteq T \) is a \term{local base} for \( \mscrT \) at \( x \) if every neighborhood of \( x \) contains a set from \( \mscrB(x) \).

  Given a base \( \mscrB \), unless explicitly noted, we consider the subfamily \( \mscrB(x) \) of all members of \( \mscrB \) containing \( x \).

  The indexed family of local bases \( \{ \mscrB(x) \colon x \in X \} \) is called a \term{neighborhood system} of \( \mscrT \).
\end{definition}

\begin{proposition}\label{thm:neighborhood_iff_union_in_topological_local_base}
  Analogously to \fullref{def:topological_base/union}, a set \( A \) containing \( x \) is a neighborhood of \( x \) if and only if \( A \) is a union of elements of the local \hyperref[def:topological_local_base]{base} \( \mscrB(x) \).
\end{proposition}
\begin{proof}
  Analogous to the proof of the equivalence in \fullref{def:topological_base}.
\end{proof}

\begin{proposition}\label{thm:topological_local_base_axioms}\mcite[13]{Engelking1989}
  Let \( X \) be an arbitrary set and let \( \{ \mscrB(x) \subseteq \pow(X) \colon x \in X \} \) be an indexed family of families of subsets of \( X \) that satisfies
  \begin{thmenum}
    \thmitem[thm:topological_local_base_axioms/BP1]{BP1} For every \( x \in X \), \( \mscrB(x) \neq \varnothing \) and \( x \in U \) for every \( U \in B(x) \).
    \thmitem[thm:topological_local_base_axioms/BP2]{BP2} For every \( x \in X \) and for all \( U, V \in B(x) \), \( \exists W \in B(x): W \subseteq U \cap V \).
    \thmitem[thm:topological_local_base_axioms/BP3]{BP3} For all \( x, y \in X \), \( x \in U \in B(y) \) implies that there exists \( V \in B(x) \) such that \( U \subseteq V \).
  \end{thmenum}

  Then the family
  \begin{equation*}
    \mscrB \coloneqq \bigcup_{x \in X} \mscrB(x)
  \end{equation*}
  is the \hyperref[thm:topological_base_axioms]{base} of some topology \( \mscrT \) on \( X \). Furthermore, \( \{ \mscrB(x) \subseteq \pow(X) \colon x \in X \} \) is a \hyperref[def:topological_local_base]{neighborhood system} for \( (X, \mscrT) \).

  In particular, the local base on any topology satisfies \fullref{thm:topological_local_base_axioms/BP1} -- \fullref{thm:topological_local_base_axioms/BP3}.
\end{proposition}

\begin{definition}\label{def:topological_space_character}
  We define the \term{character} of the point \( x \in X \) as the cardinal
  \begin{equation*}
    \chi(x) \coloneqq \min \{ \card \mscrB(x) \colon \mscrB(x) \text{ is a local base for } \mscrT \text{ at } x \}.
  \end{equation*}

  We define the \term{character} of of \( (X, \mscrT) \) as
  \begin{equation*}
    \chi((X, \mscrT)) \coloneqq \sup \{ \chi(x) \colon x \in X \}.
  \end{equation*}

  We simply write \( \chi(X) \) when the topology is clear from the context.

  Spaces for which \( \chi(X) \leq \aleph_0 \) are said to be \term{first-countable}.
\end{definition}
\begin{proof}
  The definition is correct because cardinals are well-ordered when regarded as initial ordinals.
\end{proof}

\begin{definition}\label{def:topological_local_subbase}
  Combining \fullref{def:topological_subbase} and \fullref{def:topological_local_base}, we define a \term{local subbase} for \( \mscrT \) at \( x \) to be a family \( P(x) \subseteq T \) such that every neighborhood \( U \) of \( x \) contains a finite intersection of sets from \( P(x) \).

  Given a subbase \( P \), unless explicitly noted, we consider the subfamily \( P(x) \) of all members of \( P \) containing \( x \).
\end{definition}

\begin{definition}\label{def:topological_closure_operator}\mcite[33]{Engelking1989}
  Let \( (X, \mscrT) \) be a topological space. Define the \term{closure operator}
  \begin{balign*}
     & \cl: \pow(X) \to \pow(X)                                           \\
     & \cl(A) \coloneqq \bigcap \{ F : F \in F_\mscrT, A \subseteq F \}.
  \end{balign*}
\end{definition}

\begin{proposition}\label{thm:def:topological_closure_operator/properties}
  The closure \hyperref[def:topological_closure_operator]{operator} has the following basic properties
  \begin{thmenum}
    \thmitem{thm:def:topological_closure_operator/closed} The set \( A \) is closed if and only if \( A = \cl A \).
    \thmitem{thm:def:topological_closure_operator/neighborhood_intersection} For any \( x \in X \), \( x \in \cl A \) if and only if every neighborhood of \( x \) intersects \( A \).
    \thmitem{thm:def:topological_closure_operator/monotone} \( \cl \) is \hyperref[def:partially_ordered_set/homomorphism]{monotone}, i.e. if \( A \subseteq B \), then \( \cl(A) \subseteq \cl(B) \).
  \end{thmenum}
\end{proposition}
\begin{proof}
  \SubProofOf{thm:def:topological_closure_operator/closed} The condition \( A = \cl{A} \) is equivalent to \( A \) being a closed superset of itself, which is equivalent to \( A \) being closed.

  \SubProofOf{thm:def:topological_closure_operator/neighborhood_intersection} Note that this proof relies on \fullref{def:topological_boundary}, however we do not use this property when defining the boundary.

  \SufficiencySubProof Fix \( x \in \cl{A} \) and let \( U \) be a neighborhood of \( x \). If \( x \in A \), then obviously \( x \in U \cap A \neq \varnothing \). If \( x \not\in A \), then \( U \cap A \neq \varnothing \) by \fullref{def:topological_boundary/neighborhoods}. In both cases, we obtain \( U \cap A \neq \varnothing \), which proves the statement.

  \NecessitySubProof Fix \( x \in X \) and assume that every neighborhood of \( x \) intersects \( A \). Since the case \( x \in A \) is trivial, suppose that \( x \not\in A \). By \fullref{thm:set_open_iff_neighborhood_is_contained}, every neighborhood \( U \) of \( x \) does not entirely belong to \( A \). By \fullref{def:topological_boundary/neighborhoods}, \( x \in \fr A \subseteq \cl A \).

  \SubProofOf{thm:def:topological_closure_operator/monotone} If \( A \subseteq B \), every closed superset of \( B \) is also a closed superset of \( A \).
\end{proof}

\begin{proposition}\label{thm:topological_closure_operator_axioms}\mcite[14]{Engelking1989}
  Let \( X \) be an arbitrary set and let \( \cl: \pow(X) \to \pow(X) \) be a function that satisfies
  \begin{thmenum}
    \thmitem[thm:topological_closure_operator_axioms/CO1]{CO1} \( \cl(\varnothing) = \varnothing \)
    \thmitem[thm:topological_closure_operator_axioms/CO2]{CO2} \( \forall A \in \pow(X), A \subseteq \cl(A) \)
    \thmitem[thm:topological_closure_operator_axioms/CO3]{CO3} \( \forall A, B \in \pow(X), \cl(A \cup B) = \cl(A) \cup \cl(B) \)
    \thmitem[thm:topological_closure_operator_axioms/CO4]{CO4} \( \forall A \in \pow(X), \cl(\cl(A)) = \cl(A) \)
  \end{thmenum}

  Then the family
  \begin{equation*}
    \mscrT \coloneqq \{ X \setminus F \colon F = \cl(F) \}
  \end{equation*}
  is a topology on \( X \). Furthermore, \( \cl = \cl^{\mscrT} \), where \( \cl^{\mscrT} \) is the closure \hyperref[def:topological_closure_operator]{operator} on \( (X, \mscrT) \).

  In particular, the closure operator on any topology satisfies \fullref{thm:topological_closure_operator_axioms/CO1} -- \fullref{thm:topological_closure_operator_axioms/CO4}.
\end{proposition}

\begin{definition}\label{def:interior_operator}\mcite[15]{Engelking1989}
  Let \( (X, \mscrT) \) be a topological space. Define the \term{interior operator}
  \begin{balign*}
     & \Int: \pow(X) \to \pow(X)                                     \\
     & \Int(A) \coloneqq \bigcup \{ U : U \in T, U \subseteq A \}.
  \end{balign*}
\end{definition}

\begin{proposition}\label{thm:def:interior_operator/properties}
  The interior \hyperref[def:interior_operator]{operator} has the following basic properties
  \begin{thmenum}
    \thmitem{thm:def:interior_operator/open} A set \( A \) is a topological space is open if and only if \( A = \int A \).
    \thmitem{thm:def:interior_operator/monotone} \( \int \) is \hyperref[def:partially_ordered_set/homomorphism]{monotone}, i.e. if \( A \subseteq B \), then \( \Int(A) \subseteq \Int(B) \).
  \end{thmenum}
\end{proposition}
\begin{proof}
  \SubProofOf{thm:def:interior_operator/open} Follows from \fullref{thm:def:topological_closure_operator/closed} and \fullref{thm:interior_closure_complement}.
  \SubProofOf{thm:def:interior_operator/monotone} Follows from \fullref{thm:def:topological_closure_operator/monotone} and \fullref{thm:interior_closure_complement}.
\end{proof}

\begin{proposition}\label{thm:interior_closure_complement} For every set \( A \subseteq X \) we have
  \begin{itemize}
    \item \( X \setminus \Int(A) = \cl(X \setminus A) \)
    \item \( X \setminus \cl(A) = \Int(X \setminus A) \)
  \end{itemize}
\end{proposition}
\begin{proof}
  Any open subset \( U \subseteq A \) is a closed superset of \( X \setminus A \). A point belongs to \( \Int(A) \) if it belongs to at least one open subset of \( A \), which happens if and only if it belongs to at least one closed superset of \( X \setminus A \). Therefore,
  \begin{balign*}
    X \setminus \Int(A)
     & =
    X \setminus \bigcup \{ U : U \in T, U \subseteq A \}
    =                                            \\ &=
    X \setminus \bigcup \{ F : F \in F_\mscrT, X \setminus A \subseteq F \}
    \reloset {X \setminus (X \setminus A) = A} = \\ &=
    \bigcup \{ F : F \in F_\mscrT, F \subseteq A \}.
    =                                            \\ &=
    \cl(A).
  \end{balign*}

  The other equality is obtained by noting that \( X \setminus \cl(A) = X \setminus (X \setminus \Int(A)) = \Int(A) \).
\end{proof}

\begin{proposition}\label{thm:interior_operator_axioms}
  Let \( X \) be an arbitrary set and let \( \int: \pow(X) \to \pow(X) \) be a function that satisfies
  \begin{thmenum}
    \thmitem[thm:interior_operator_axioms/IO1]{IO1} \( \Int(X) = X \)
    \thmitem[thm:interior_operator_axioms/IO2]{IO2} \( \forall A \in \pow(X), \Int(A) \subseteq A \)
    \thmitem[thm:interior_operator_axioms/IO3]{IO3} \( \forall A, B \in \pow(X), \Int(A \cap B) = \Int(A) \cap \Int(B) \)
    \thmitem[thm:interior_operator_axioms/IO4]{IO4} \( \forall A \in \pow(X), \Int(\Int(A)) = \Int(A) \)
  \end{thmenum}

  Then the family
  \begin{equation*}
    \mscrT \coloneqq \{ U \colon U = \Int(U) \}
  \end{equation*}
  is a topology on \( X \). Furthermore, \( \int = \int_\mscrT \), where \( \int_\mscrT \) is the interior \hyperref[def:interior_operator]{operator} on \( (X, \mscrT) \).

  In particular, the interior operator on any topology satisfies \fullref{thm:interior_operator_axioms/IO1} -- \fullref{thm:interior_operator_axioms/IO4}.
\end{proposition}

\begin{definition}\label{def:topological_boundary}
  For a subset \( A \) of a topological space we define its \term{boundary} \( \fr(A) \) equivalently as
  \begin{thmenum}
    \thmitem{def:topological_boundary/closure} \( \fr(A) \coloneqq \cl(A) \setminus \Int(A) \)
    \thmitem{def:topological_boundary/neighborhoods} \( \fr(A) \) is the set of all points \( x \in X \) such that every neighborhood of \( x \) intersects both \( A \) and \( X \setminus A \).
  \end{thmenum}
\end{definition}
\begin{proof}
  The equivalence of the definitions is trivial when \( \fr(A) = \varnothing \). We assume that \( \fr(A) \neq \varnothing \).

  \ImplicationSubProof{def:topological_boundary/closure}{def:topological_boundary/neighborhoods} Let \( x \in \cl(A) \setminus \Int(A) \).

  Aiming for a contradiction, suppose that there is a neighborhood \( U \) of \( x \) that does not intersect \( A \). Then \( U \subseteq X \setminus A \). Hence, \( A \subseteq X \setminus U \). Since \( X \setminus U \) is closed, it follows that \( \cl(A) \subseteq X \setminus U \) as the intersection of all closed supersets of \( A \). But \( x \not\in X \setminus U \), therefore \( x \not\in \cl(A) \), which contradicts our choice of \( x \in \cl(A) \).

  This proves that every neighborhood of \( x \) intersects \( A \).

  By passing to complements, we can reuse this to prove that every neighborhood of \( x \) intersects \( X \setminus A \) using \fullref{thm:interior_closure_complement}.

  \ImplicationSubProof{def:topological_boundary/neighborhoods}{def:topological_boundary/closure} Suppose that every neighborhood of \( x \in \fr(A) \) intersects both \( A \) and \( X \setminus A \). Therefore, no neighborhood of \( x \) is contained in neither \( A \) not \( X \setminus A \) and \( x \) belongs to neither \( \Int(A) \) nor \( \Int(X \setminus A) \). Hence,
  \begin{equation*}
    x \in (X \setminus \Int(X \setminus A)) \setminus \Int(A) \reloset {\ref{thm:interior_closure_complement}} = \cl(A) \setminus \Int(A).
  \end{equation*}
\end{proof}

\begin{proposition}\label{thm:def:topological_boundary/properties}
  The \hyperref[def:topological_boundary]{topological boundary} has the following basic properties
  \begin{thmenum}
    \thmitem{thm:def:topological_boundary/closed} \( \fr(A) \) is a closed set.
    \thmitem{thm:def:topological_boundary/not_open} If \( \fr(A) \) is not empty, it is not an open set.
    \thmitem{thm:def:topological_boundary/complement} \( \fr(A) = \fr(X \setminus A) \).
  \end{thmenum}
\end{proposition}
\begin{proof}
  \SubProofOf{thm:def:topological_boundary/closed} Note that
  \begin{equation*}
    \fr(A) = \cl(A) \setminus \Int(A) = \cl(A) \cap (X \setminus \Int(A)),
  \end{equation*}
  which is the intersection of two closed sets. Hence, \( \fr(A) \) is itself a closed set.

  \SubProofOf{thm:def:topological_boundary/not_open} Note that \( \fr(A) \) is either empty or is not open because \fullref{def:topological_boundary/neighborhoods} is incompatible with \fullref{thm:set_open_iff_neighborhood_is_contained}.

  \SubProofOf{thm:def:topological_boundary/complement} By \fullref{thm:interior_closure_complement},
  \begin{balign*}
    \fr(A)
     & =
    \cl(A) \setminus \Int(A)
    =                                                  \\ &=
    \cl(A) \cap (X \setminus \Int(A))
    \reloset {\ref{thm:interior_closure_complement}} = \\ &=
    (X \setminus \Int(X \setminus A)) \cap \cl(X \setminus A)
    =                                                  \\ &=
    \cl(X \setminus A) \setminus \Int(X \setminus A)
    =                                                  \\ &=
    \fr(X \setminus A).
  \end{balign*}
\end{proof}

\begin{definition}\label{def:topological_derived_set}\mcite[24]{Engelking1989}
  Let \( (X, \mscrT) \) be a topological space.

  \begin{thmenum}
    \thmitem{def:topological_derived_set/cluster_point} We say that the point \( x_0 \in X \) is a \term{cluster point} or an \term{accumulation point} of the set \( A \subseteq X \) if \( x \in \cl(A \setminus \{ x \}) \). It is not necessary for \( x_0 \) to belong to \( A \).

    \thmitem{def:topological_derived_set/derived_set} The set of all cluster points of \( A \) is called the \term{derived set} of \( A \) and is denoted by \( \derived(A) \).

    \thmitem{def:topological_derived_set/perfect_set} If a set equals its derived set, we call it a \term{perfect set}.

    \thmitem{def:topological_derived_set/isolated_point} Points in \( A \setminus \derived(A) \) are said to be \term{isolated points} of \( A \).

    \thmitem{def:topological_derived_set/discrete_set} If \( \derived(A) = \varnothing \), that is, if \( A \) consists of only discrete points, we say that \( A \) is a \term{discrete set}.
  \end{thmenum}
\end{definition}

\begin{proposition}\label{thm:def:derived_set/properties}
  \hyperref[def:topological_derived_set]{Derived sets} have the following basic properties
  \begin{thmenum}
    \thmitem{thm:def:derived_set/cluster_via_neighborhoods} \( x \) is a cluster point of \( A \) if and only if every neighborhood of \( x \) intersects \( A \setminus \{ x \} \)
    \thmitem{thm:def:derived_set/isolated_via_neighborhoods} \( x \) is an isolated point of \( A \) if and only if there exists a neighborhood of \( x \) that does not intersect \( A \setminus \{ x \} \)
    \thmitem{thm:def:derived_set/closed} \( \derived(A) \) is a closed set.
    \thmitem{thm:def:derived_set/closure} \( A \cup \derived(A) = \cl(A) \).
    \thmitem{thm:def:derived_set/closed_iff_contains_all_cluster_points} A set is closed if and only if it contains all of its cluster points. Compare this result to \fullref{thm:limit_point_iff_in_closure}.
    \thmitem{thm:def:derived_set/closed_iff_only_isolated_and_cluster_points} A set if closed if and only if every point is either a cluster point or an isolated point.
  \end{thmenum}
\end{proposition}
\begin{proof}
  \SubProofOf{thm:def:derived_set/cluster_via_neighborhoods} If every neighborhood \( U \) of \( x \in A \) intersects \( A \setminus \{ x \} \), by \fullref{thm:def:topological_closure_operator/neighborhood_intersection}, \( x \in \cl(A \setminus \{ x \}) \) and \( x \) is therefore a cluster point.

  \SufficiencySubProofOf{thm:def:derived_set/isolated_via_neighborhoods} Dual to \fullref{thm:def:derived_set/cluster_via_neighborhoods}.

  \SubProofOf{thm:def:derived_set/closed} Consider the complement of \( \derived(A) \). If it is empty, \( \derived(A) \) is trivially closed. Otherwise, let \( x \in X \setminus \derived(A) \).

  \begin{itemize}
    \item If \( x \) is an isolated point of \( A \), by \fullref{thm:def:derived_set/isolated_via_neighborhoods} there exists a neighborhood of \( x \) that does not intersect \( A \setminus \{ x \} \).
    \item If \( x \) is not a point of \( A \), aiming at a contradiction, assume that every neighborhood of \( x \) intersects \( A \). Then, by \fullref{def:topological_boundary/neighborhoods}, \( x \in \fr(A) \). But \( \fr(A) \subseteq \cl(A) \) and \( \cl(A) = \cl(A \setminus \{ x \}) \) because \( x \) does not belong to \( A \). Therefore, \( x \) is a cluster point of \( A \). This contradicts our assumption that \( x \not\in \derived(A) \), hence we can conclude that there exists a neighborhood of \( X \) that does not intersect \( A = A \setminus \{ x \} \).
  \end{itemize}

  In both cases, \fullref{thm:set_open_iff_neighborhood_is_contained} allows us to conclude that \( X \setminus \derived(A) \) is open and, hence, \( \derived(A) \) is closed.

  \SubProofOf{thm:def:derived_set/closure} Clearly \( A \subseteq \cl(A) \). Also
  \begin{equation*}
    \derived(A) \subseteq \bigcup_{x \in X} \cl(A \setminus \{ x \}) \subseteq \cl(A).
  \end{equation*}

  Now we will prove the reverse inclusion. Let \( x \in \cl(A) \). Then either \( x \in A \) or \( x \in \fr(A) \). Assume the latter. By \fullref{def:topological_boundary/neighborhoods}, every neighborhood \( U \) of \( x \) has points both in \( A \) and outside \( A \), therefore \( U \cap (A \setminus \{ x \}) \) is nonempty. By \fullref{thm:def:topological_closure_operator/neighborhood_intersection}, \( x \in \cl(A \setminus \{ x \}) \), that is, \( x \in \derived(A) \).

  \SubProofOf{thm:def:derived_set/closed_iff_contains_all_cluster_points}
  If \( A \) is closed, by \fullref{thm:def:derived_set/closure},
  \begin{equation*}
    A \cup \derived(A) = \cl(A) = A,
  \end{equation*}
  hence \( \derived(A) \subseteq A \).

  \NecessitySubProofOf{thm:def:derived_set/isolated_via_neighborhoods} Dual to \fullref{thm:def:derived_set/cluster_via_neighborhoods}.

  \SubProofOf{thm:def:derived_set/closed} Consider the complement of \( \derived(A) \). If it is empty, \( \derived(A) \) is trivially closed. Otherwise, let \( x \in X \setminus \derived(A) \).

  \begin{itemize}
    \item If \( x \) is an isolated point of \( A \), by \fullref{thm:def:derived_set/isolated_via_neighborhoods} there exists a neighborhood of \( x \) that does not intersect \( A \setminus \{ x \} \).
    \item If \( x \) is not a point of \( A \), aiming at a contradiction, assume that every neighborhood of \( x \) intersects \( A \). Then, by \fullref{def:topological_boundary/neighborhoods}, \( x \in \fr(A) \). But \( \fr(A) \subseteq \cl(A) \) and \( \cl(A) = \cl(A \setminus \{ x \}) \) because \( x \) does not belong to \( A \). Therefore, \( x \) is a cluster point of \( A \). This contradicts our assumption that \( x \not\in \derived(A) \), hence we can conclude that there exists a neighborhood of \( X \) that does not intersect \( A = A \setminus \{ x \} \).
  \end{itemize}

  In both cases, \fullref{thm:set_open_iff_neighborhood_is_contained} allows us to conclude that \( X \setminus \derived(A) \) is open and, hence, \( \derived(A) \) is closed.

  \SubProofOf{thm:def:derived_set/closure} Clearly \( A \subseteq \cl(A) \). Also
  \begin{equation*}
    \derived(A) \subseteq \bigcup_{x \in X} \cl(A \setminus \{ x \}) \subseteq \cl(A).
  \end{equation*}

  Now we will prove the reverse inclusion. Let \( x \in \cl(A) \). Then either \( x \in A \) or \( x \in \fr(A) \). Assume the latter. By \fullref{def:topological_boundary/neighborhoods}, every neighborhood \( U \) of \( x \) has points both in \( A \) and outside \( A \), therefore \( U \cap (A \setminus \{ x \}) \) is nonempty. By \fullref{thm:def:topological_closure_operator/neighborhood_intersection}, \( x \in \cl(A \setminus \{ x \}) \), that is, \( x \in \derived(A) \).

  \SubProofOf{thm:def:derived_set/closed_iff_contains_all_cluster_points} Assume that \( \derived(A) \subseteq A \) and, aiming at a contradiction, suppose that \( A \) is not closed. Fix a point \( x \in \cl(A) \setminus A \). By \fullref{thm:def:derived_set/closure}, this is a cluster point. By \fullref{thm:def:derived_set/cluster_via_neighborhoods}, every for neighborhood \( U \) of \( x \) the intersection \( U \cap (A \setminus \{ x \}) \subseteq U \cap A \) is nonempty. Since this holds for arbitrary neighborhoods, by \fullref{thm:def:topological_closure_operator/neighborhood_intersection}, \( A \) is closed.

  \SufficiencySubProofOf{thm:def:derived_set/closed_iff_only_isolated_and_cluster_points}
  Special case of \fullref{thm:def:derived_set/closed_iff_contains_all_cluster_points}.
  \NecessitySubProofOf{thm:def:derived_set/closed_iff_only_isolated_and_cluster_points} We already know from \fullref{thm:def:derived_set/closed_iff_contains_all_cluster_points} that it is sufficient for \( \derived(A) \) to belong to \( A \) for \( A \) to be closed. But \( A \setminus \derived(A) \) consists of all isolated points, therefore every point in \( A \) is either a cluster point or an isolated point.
\end{proof}

\begin{definition}\label{def:topologically_dense_set}\mcite[25]{Engelking1989}
  Let \( (X, \mscrT) \) be a topological space and \( A \subseteq X \) be any set. We say that \( A \) is

  \begin{thmenum}
    \thmitem{def:topologically_dense_set/dense} \term{dense} in \( X \) if \( \cl{A} = X \) (if \( X \) is assumed from the context, we simply say that \( A \) is dense).

    \thmitem{def:topologically_dense_set/codense} \term{codense} in \( X \) if \( X \setminus A \) is dense, i.e. \( \cl(X \setminus A) = X \).

    \thmitem{def:topologically_dense_set/nowhere_dense} \term{nowhere dense} in \( X \) if \( \cl(A) \) is codense, i.e. \( X = \cl(X \setminus \cl A) \reloset {\ref{thm:interior_closure_complement}} = \cl(\Int(X \setminus A)) \).

    \thmitem{def:topologically_dense_set/dense_in_itself} \term{dense in itself} if \( A \subseteq \derived(A) \), i.e. if \( A \) has no isolated points.
  \end{thmenum}

  We define the \term{density} \( d(X) \) of \( X \) to be the minimum \hyperref[def:cardinal]{cardinality} of all dense sets. If \( d(X) \leq \aleph_0 \), we say that the space is \term{separable}.
\end{definition}

\begin{proposition}\label{thm:def:dense_set/properties}
  \hyperref[def:topologically_dense_set/dense]{Dense sets} have the following basic properties:
  \begin{thmenum}
    \thmitem{thm:def:dense_set/open_intersection}\mcite[prop. 1.3.5]{Engelking1989} The set \( A \) is dense if and only if every nonempty open set intersects \( A \).
  \end{thmenum}
\end{proposition}
\begin{proof}
  \SubProofOf{thm:def:dense_set/open_intersection} Special case of \fullref{thm:def:topological_closure_operator/neighborhood_intersection}.
\end{proof}

\begin{proposition}\label{thm:def:nowhere_dense/properties}
  \hyperref[def:topologically_dense_set/nowhere_dense]{Nowhere dense sets} have the following basic properties:
  \begin{thmenum}
    \thmitem{thm:def:nowhere_dense/empty_interior} Nowhere dense sets have an empty interior
    \thmitem{thm:def:nowhere_dense/contained_in_boundary} Nowhere dense sets are entirely contained in their boundaries.
    \thmitem{thm:def:nowhere_dense/interior_of_closure} The set \( A \) is nowhere dense if and only if \( \Int(\cl(A)) = \varnothing \).
    \thmitem{thm:def:nowhere_dense/closure_contains_no_open_set} The set is nowhere dense if and only if its closure does not contain a nonempty open set.
    \thmitem{thm:def:nowhere_dense/open_subset}\mcite[prop. 1.3.5]{Engelking1989}The set \( A \) is nowhere dense if and only if every open set contains a nonempty open subset disjoint from \( A \).
    \thmitem{thm:def:nowhere_dense/subset} A subset of a nowhere dense set is nowhere dense.
    \thmitem{thm:def:nowhere_dense/homeomorphism} The \hyperref[def:homeomorphism]{homeomorphic} image of a nowhere dense set is nowhere dense.
    \thmitem{thm:def:nowhere_dense/complement_dense} A set is closed and nowhere dense if and only if its complement is open and dense.
  \end{thmenum}
\end{proposition}
\begin{proof}
  \SubProofOf{thm:def:nowhere_dense/interior_of_closure} Follows directly from \fullref{thm:interior_closure_complement}.
  \SubProofOf{thm:def:nowhere_dense/empty_interior} Follows from \fullref{thm:def:nowhere_dense/interior_of_closure} because \( \Int(A) \subseteq \Int(\cl(A)) = \varnothing \).

  \SubProofOf{thm:def:nowhere_dense/contained_in_boundary} Follows from \fullref{thm:def:nowhere_dense/empty_interior} and \fullref{def:topological_boundary/closure}.

  \SubProofOf{thm:def:nowhere_dense/closure_contains_no_open_set} By \fullref{thm:def:dense_set/open_intersection}, \( A \) is nowhere dense if and only if every nonempty open set intersects \( X \setminus \cl(A) \reloset {\ref{thm:interior_closure_complement}} = \Int(X \setminus A) \). By \fullref{thm:set_open_iff_neighborhood_is_contained}, the last condition is equivalent to every nonempty open set having a nonempty open subset in \( \Int(X \setminus A) = X \setminus \cl(A) \), which in turn implies \fullref{thm:def:nowhere_dense/closure_contains_no_open_set}.

  \SubProofOf{thm:def:nowhere_dense/subset} Let \( A \) be a nowhere dense set and let \( B \subseteq A \). Then
  \begin{equation*}
    \Int(\cl(B))
    \reloset {\ref{thm:def:interior_operator/monotone}} \subseteq
    \Int(\cl(A))
    \reloset {\ref{thm:def:nowhere_dense/interior_of_closure}} =
    \varnothing,
  \end{equation*}
  therefore \( B \) is also nowhere dense.

  \SubProofOf{thm:def:nowhere_dense/homeomorphism} Let \( f: X \to Y \) be a homeomorphic embedding (not necessarily surjective) and let \( A \subseteq X \) be a nowhere dense set. Let \( V \) be an open set in \( Y \). Then \( f^{-1}(V) \) is open in \( X \) and, by \fullref{thm:def:nowhere_dense/open_subset}, there exists an open subset \( U \subseteq f^{-1}(V) \) that is disjoint from \( A \). Therefore, \( f(U) \subseteq f(f^{-1}(V)) \reloset {\ref{thm:function_image_preimage_composition/image_of_preimage}} \subseteq V \). Furthermore, \( f(U) \) is open and \( f(U) \cap f(A) \reloset {\ref{thm:def:function_image/intersection}} = f(U \cap A) = f(\varnothing) = \varnothing \), therefore \( f(A) \) is nowhere dense.

  \SubProofOf{thm:def:nowhere_dense/complement_dense} If \( A \) is an open dense set, then \( X \setminus A \) is closed and
  \begin{equation*}
    \cl(X \setminus \cl(X \setminus A))
    =
    \cl(X \setminus (X \setminus A))
    =
    \cl(A)
    =
    X,
  \end{equation*}
  therefore \( X \setminus A \) is nowhere dense.
\end{proof}

\begin{definition}\label{def:borel_algebra}
  Fix a topological space \( X \) and \( \mscrF \subseteq \pow(X) \). Denote by \( \mscrF_\delta \) the family of all countable intersections of elements of \( \mscrF \) and by \( \mscrF_\sigma \) the family of all countable unions of elements of \( \mscrF \).

  The family \( F_\delta \) is the family of countable unions of closed sets and \( G_\sigma \) is the family of countable intersections of open sets.
\end{definition}
