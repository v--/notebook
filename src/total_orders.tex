\subsection{Total orders}\label{subsec:total_orders}

\begin{definition}\label{def:totally_ordered_set}
  We say that a partially ordered set is \term{totally ordered} if either the nonstrict order \( \leq \) is \hyperref[def:binary_relation/total]{total} or if the strict order \( < \) is \hyperref[def:binary_relation/trichotomic]{trichotomic}.

  The theory, homomorphisms and category are obtained analogously to \fullref{def:poset} but with either of these additional axiom sets.
\end{definition}
\begin{proof}
  Equivalence between nonstrict and strict total orders follows directly from the compatibility condition \eqref{def:poset/compatibility_nonstrict}.
\end{proof}

\begin{definition}\label{def:order_topology}\mcite{nLab:order_topology}
  Let \( (\mscrP, \leq) \) be a \hyperref[def:poset]{totally ordered set}. The \term{order topology} induced by \( \leq \) is the topology generated by the \hyperref[def:topological_subbase]{subbase} of open \hyperref[def:poset_interval/ray]{rays}
  \begin{equation*}
    P \coloneqq \set{ (a, \infty) \given a \in \mscrP } \cup \set{ (-\infty, b) \given b \in \mscrP }.
  \end{equation*}
\end{definition}

\begin{lemma}[Zorn's lemma]\label{thm:zorns_lemma}\mcite[thm. 6M(6)]{Enderton1977Sets}
  If any \hyperref[def:poset_chain_and_antichain]{chain} in a \hyperref[def:poset]{partially ordered set} has an upper \hyperref[def:preordered_set/upper_and_lower_bounds]{bound}, there exists a \hyperref[def:preordered_set/maximal_and_minimal_element]{maximal} set in \( X \).

  Note that Zorn's lemma is usually stated and used only for the \hyperref[thm:subsets_form_boolean_algebra]{lattice of sets}.

  This theorem is equivalent to \fullref{thm:aoc}.
\end{lemma}

\begin{definition}\label{def:well_ordered_set}\mcite[def. 63.1]{OpenLogicFull}
  A \hyperref[def:totally_ordered_set]{totally ordered set} \( (\mscrP, \leq) \) is said to be \term{well-ordered} if every nonempty set has a \hyperref[def:preordered_set/maximum_and_minimum]{minimum}.
\end{definition}

\begin{theorem}[Well-Ordering Principle]\label{thm:well_ordering_principle}\mcite[196]{Enderton1977Sets}
  Any \hyperref[def:set_zfc]{set} can be \hyperref[def:well_ordered_set]{well-ordered}.

  This theorem is equivalent to \fullref{thm:aoc}.
\end{theorem}

\begin{proposition}\label{thm:well_ordered_sets_can_be_embedded}
  If \( \mscrP \) and \( \mscrQ \) are \hyperref[def:well_ordered_set]{well-ordered sets}, then one of them can be embedded homomorphically into the other.
\end{proposition}
\begin{proof}
  We will denote the \hyperref[def:poset_interval/ray]{open initial segments} of \( x \in \mscrP \) by \( \mscrP_{<x} \).

  Define the following function:
  \begin{equation*}
    \begin{aligned}
      &\varphi: \mscrP \to \mscrQ \\
      &\varphi(x) \coloneqq \begin{cases}
        \min(\mscrQ),                               , &x = \mscrP \\
        \min(\mscrQ \setminus \varphi(\mscrP_{<x}) ), &\T{otherwise}
      \end{cases}
    \end{aligned}
  \end{equation*}

  If \( \varphi \) is defined for all \( \mscrP \), that is, if there exists to \( x \) such that \( \varphi(\mscrP_{<x}) = \mscrQ \), then \( \varphi \) is the desired embedding.

  Indeed, if \( x \leq y \), then \( \varphi(x) \leq \varphi(y) \) because the minimum in \( \varphi(y) \) ranges over a smaller set. Therefore \( \varphi \) is indeed an order homomorphism. It is also injective because in a totally ordered set, \( \varphi(x) = \varphi(y) \) is possible only if \( \mscrP_{<x} = \mscrP_{<y} \), which in turn is only possible if \( x = y \).

  If, on the contrary, \( \varphi(\mscrP_{<x}) = \mscrQ \) for some \( x \in \mscrP \), let \( x_0 \) be the smallest such element. Then we have discovered that \( \mscrQ \) equals the initial segment \( \mscrP_{<x_0} \). Since the restriction of \( \varphi \) to \( \mscrP_{<x} \) is an injective homomorphism whose image equals \( \mscrQ \), it is an isomorphism. Then the inverse function of the restriction is the desired embedding of \( \mscrQ \) into \( \mscrP \).
\end{proof}

\begin{example}\label{ex:thm:well_ordered_sets_can_be_embedded/proof}
  To demonstrate the proof of \fullref{thm:well_ordered_sets_can_be_embedded}, let \( \mscrP = \set{ p_1, p_2, \ldots, p_n } \) and \( \mscrQ = \set{ q_1, q_2, \ldots, q_m } \).

  Then \( \varphi \) is
  \begin{align*}
    \varphi(p_k)
    &=
    \begin{rcases}
      \begin{cases}
        \min(\set{ q_1, q_2, \ldots, q_m }),                                                  &k = 1 \\
        \min(\set{ q_1, q_2, \ldots, q_m } \setminus \varphi(\set{ p_1, p_2, \ldots, p_k })), &k > 1
      \end{cases}
    \end{rcases}
    = \\ &=
    \begin{rcases}
      \begin{cases}
        q_1,                                                                                                                &k = 1 \\
        \min(\set{ q_1, q_2, \ldots, q_m } \setminus \set{ q_1, q_2, \ldots, q_k }) = \min(\set{ q_k, \ldots, q_m }) = q_k, &k > 1
      \end{cases}
    \end{rcases}
    = \\ &=
    q_k.
  \end{align*}

  If \( n \leq m \), this function is total and is the desired embedding. If \( n > m \), then the restriction of \( \varphi \) to \( \set{ p_1, p_2, \ldots, p_m } \) is an isomorphism and its inverse is the desired embedding.
\end{example}
