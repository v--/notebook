\subsection{Total orders}\label{subsec:total_orders}

\begin{definition}\label{def:totally_ordered_set}
  We say that the partially ordered set \( P \) is \Def{totally ordered} if either the nonstrict order \( \leq \) is total\Tinyref{def:binary_relation/total} or if the strict order \( < \) is trichotomic\Tinyref{def:binary_relation/trichotomic}.
\end{definition}
\begin{proof}
  The two conditions are equivalent because \( x < y \) is trichotomic if and only if \( x \leq y \) is total.
\end{proof}

\begin{definition}\label{def:poset_chain}
  We call a subset \( A \subseteq X \) of a poset\Tinyref{def:poset} \( (X, \leq) \) a \Def{chain} if \( (A, \leq_A) \) is a totally ordered\Tinyref{def:totally_ordered_set} set.

  Dually, we call the subset \( A \subseteq X \) an \Def{antichain} if no two elements of \( A \) are comparable\Tinyref{def:preordered_set/comparability}.
\end{definition}

\begin{definition}\label{def:total_order_interval}\cite{nLab:order_topology}
  In a totally ordered set\Tinyref{def:poset} \( (P, \leq) \), for any \( a, b \in P \) with \( a \leq b \), we define
  \begin{defenum}
    \DItem{def:total_order_interval/closed} the \Def{closed interval}
    \begin{equation*}
      [a, b] \coloneqq \{ x \in P \colon a \leq x \leq b \}
    \end{equation*}

    \DItem{def:total_order_interval/open} the \Def{open interval}
    \begin{equation*}
      (a, b) \coloneqq \{ x \in P \colon a < x < b \}
    \end{equation*}

    \DItem{def:total_order_interval/half_open} the \Def{half-open intervals}
    \begin{align*}
      (a, b] &\coloneqq \{ x \in P \colon a < x \leq b \}
      \\
      [a, b) &\coloneqq \{ x \in P \colon a \leq x < b \}
    \end{align*}

    \DItem{def:total_order_interval/open_ray} the \Def{open rays}
    \begin{align*}
      (a, \infty) &\coloneqq \{ x \in P \colon a < x \}
      \\
      (-\infty, b) &\coloneqq \{ x \in P \colon x < b \}
    \end{align*}

    \DItem{def:total_order_interval/closed_ray} the \Def{closed rays}
    \begin{align*}
      [a, \infty) &\coloneqq \{ x \in P \colon a \leq x \}
      \\
      (-\infty, b] &\coloneqq \{ x \in P \colon x \leq b \}
    \end{align*}
  \end{defenum}
\end{definition}

\begin{definition}\label{def:function_definiteness}
  Let \( (S, 0_S) \) be a pointed set\Tinyref{def:pointed_set}, \( (P, 0_P, \leq) \) be a pointed totally ordered set. We say that the function \( f: S \to P \) is
  \begin{itemize}
    \item \Def{positive definite} if \( f(s) > 0_P \) for all \( s \in S \setminus \{ 0_S \} \) and \( f(0_S) = 0_P \).
    \item \Def{negative definite} if \( f(s) < 0_P \) for all \( s \in S \setminus \{ 0_S \} \) and \( f(0_S) = 0_P \).
    \item \Def{positive semidefinite} if \( f(s) \geq 0_P \) for all \( s \in S \setminus \{ 0_S \} \) and \( f(0_S) = 0_P \).
    \item \Def{negative semidefinite} if \( f(s) \leq 0_P \) for all \( s \in S \setminus \{ 0_S \} \) and \( f(0_S) = 0_P \).
  \end{itemize}
\end{definition}

\begin{definition}\label{def:order_topology}\cite{nLab:order_topology}
  Let \( (P, <) \) be a totally ordered set\Tinyref{def:poset}. The \Def{order topology induced by \( < \)} is the topology generated by the subbase\Tinyref{def:topological_subbase} of open rays\Tinyref{def:total_order_interval/open_ray}
  \begin{equation*}
    \Cal{P} \coloneqq \{ (a, \infty) \colon a \in P \} \cup \{ (-\infty, b) \colon b \in P \}.
  \end{equation*}
\end{definition}

\begin{lemma}[Zorn's lemma]\label{thm:zorns_lemma}\cite{nLab:zorns_lemma}
  If any chain\Tinyref{def:poset_chain} in a partially ordered set\Tinyref{def:poset} has an upper bound\Tinyref{def:preordered_set/upper_lower_bound}, there exists a maximal\Tinyref{def:preordered_set/maximal_minimal_element} set in \( X \).
\end{lemma}

\begin{definition}\label{def:well_ordered_set}
  A totally ordered set\Tinyref{def:totally_ordered_set} \( (X, \leq) \) is said to be \Def{well-ordered} if every nonempty subset \( A \subseteq X \) has a minimum\Tinyref{def:preordered_set/largest_smallest_element}.
\end{definition}

\begin{theorem}[Well-Ordering Principle]\label{thm:well_ordering_principle}\cite[196]{Enderton1977}
  Any set\Tinyref{def:set_zfc} can be well-ordered\Tinyref{def:well_ordered_set}.
\end{theorem}
