\subsection{Total orders}\label{subsec:total_orders}

\begin{definition}\label{def:totally_ordered_set}
  We say that the partially ordered set \( P \) is \term{totally ordered} if either the nonstrict order \( \leq \) is \hyperref[def:binary_relation/total]{total} or if the strict order \( < \) is \hyperref[def:binary_relation/trichotomic]{trichotomic}.

  The theory, homomorphisms and category are obtained analogously to \fullref{def:poset} but with these additional axioms.
\end{definition}
\begin{proof}
  Equivalence between nonstrict and strict total orders follows directly from the compatibility condition \eqref{def:poset/compatibility_nonstrict}.
\end{proof}

\begin{definition}\label{def:poset_chain}
  We call a subset \( A \subseteq X \) of a \hyperref[def:poset]{poset} \( (X, \leq) \) a \term{chain} if \( (A, \leq_A) \) is a totally \hyperref[def:totally_ordered_set]{ordered} set.

  Dually, we call the subset \( A \subseteq X \) an \term{antichain} if no two elements of \( A \) are \hyperref[def:preordered_set/comparability]{comparable}.
\end{definition}

\begin{definition}\label{def:total_order_interval}\mcite{nLab:order_topology}
  Fix a \hyperref[def:poset]{totally ordered set} \( (P, \leq) \) and let \( \infty \) be a sentinel symbol not in \( P \). For any \( a, b \in P \) with \( a \leq b \), we define
  \begin{thmenum}
    \thmitem{def:total_order_interval/closed} the \term{closed interval}
    \begin{equation*}
      [a, b] \coloneqq \set{ x \in P \colon a \leq x \leq b }
    \end{equation*}

    \thmitem{def:total_order_interval/open} the \term{open interval}
    \begin{equation*}
      (a, b) \coloneqq \set{ x \in P \colon a < x < b }
    \end{equation*}

    \thmitem{def:total_order_interval/half_open} the \term{half-open intervals}
    \begin{equation*}
      \begin{aligned}
        (a, b] &\coloneqq \set{ x \in P \colon a < x \leq b }
        \\
        [a, b) &\coloneqq \set{ x \in P \colon a \leq x < b }
      \end{aligned}
    \end{equation*}

    \thmitem{def:total_order_interval/open_ray} the \term{open rays}
    \begin{equation*}
      \begin{aligned}
        (a, \infty)  &\coloneqq \set{ x \in P \colon a < x }
        \\
        (-\infty, b) &\coloneqq \set{ x \in P \colon x < b }
      \end{aligned}
    \end{equation*}

    \thmitem{def:total_order_interval/closed_ray} the \term{closed rays}
    \begin{equation*}
      \begin{aligned}
        [a, \infty)  &\coloneqq \set{ x \in P \colon a \leq x }
        \\
        (-\infty, b] &\coloneqq \set{ x \in P \colon x \leq b }
      \end{aligned}
    \end{equation*}
  \end{thmenum}
\end{definition}

\begin{definition}\label{def:order_topology}\mcite{nLab:order_topology}
  Let \( (\mscrP, \leq) \) be a \hyperref[def:poset]{totally ordered set}. The \term{order topology} induced by \( \leq \) is the topology generated by the \hyperref[def:topological_subbase]{subbase} of open \hyperref[def:total_order_interval/open_ray]{rays}
  \begin{equation*}
    P \coloneqq \set{ (a, \infty) \given a \in \mscrP } \cup \set{ (-\infty, b) \given b \in \mscrP }.
  \end{equation*}
\end{definition}

\begin{lemma}[Zorn's lemma]\label{thm:zorns_lemma}\mcite[thm. 6M(6)]{Enderton1977Sets}
  If any \hyperref[def:poset_chain]{chain} in a \hyperref[def:poset]{partially ordered set} has an upper \hyperref[def:preordered_set/upper_and_lower_bounds]{bound}, there exists a \hyperref[def:preordered_set/maximal_and_minimal_element]{maximal} set in \( X \).

  Note that Zorn's lemma is usually stated and used only for the \hyperref[thm:subsets_form_boolean_algebra]{lattice of sets}.

  This theorem is equivalent to \fullref{thm:aoc}.
\end{lemma}

\begin{definition}\label{def:well_ordered_set}
  A totally ordered \hyperref[def:totally_ordered_set]{set} \( (X, \leq) \) is said to be \term{well-ordered} if every nonempty subset \( A \subseteq X \) has a \hyperref[def:preordered_set/maximum_and_minimum]{minimum}.
\end{definition}

\begin{theorem}[Well-Ordering Principle]\label{thm:well_ordering_principle}\mcite[196]{Enderton1977Sets}
  Any \hyperref[def:set_zfc]{set} can be \hyperref[def:well_ordered_set]{well-ordered}.

  This theorem is equivalent to \fullref{thm:aoc}.
\end{theorem}
