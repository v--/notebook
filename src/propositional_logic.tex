\subsection{Propositional logic}\label{subsec:propositional_logic}

\begin{definition}\label{def:propositional_alphabet}\MarginCite[102]{OpenLogic20201202}
  The \hyperref[def:language]{language} of propositional logic consists of \Def{propositional formulas}, defined in \fullref{def:propositional_formula}. These formulas are formed over a set of symbols, called the \Def{propositional alphabet}, whose constituents are described below.

  \begin{DefEnum}
    \ILabel{def:propositional_alphabet/prop} A nonempty set \( \Bold{Prop} \) of \Def{propositional variables}.

    \ILabel{def:propositional_alphabet/constants} Two \Def{propositional constants} (also known as \Def{truth values}):
    \begin{DefEnum}
      \ILabel{def:propositional_alphabet/constants/top} \Def{top} \( \top \)
      \ILabel{def:propositional_alphabet/constants/bottom} \Def{bottom} \( \bot \)
    \end{DefEnum}

    \ILabel{def:propositional_alphabet/negation} \Def{Negation} \( \neg \)
    \ILabel{def:propositional_alphabet/connectives} The set \( \Sigma \) of \Def{propositional connectives}, namely
    \begin{DefEnum}
      \ILabel{def:propositional_alphabet/connectives/conjunction} \Def{conjunction} \( \wedge \) (also known as \hyperref[def:boolean_operators]{\Def{and}} or \hyperref[def:lattice_operations]{\Def{meet}})
      \ILabel{def:propositional_alphabet/connectives/disjunction} \Def{disjunction} \( \vee \) (also known as \hyperref[def:boolean_operators]{\Def{or}} or \hyperref[def:lattice_operations]{\Def{join}})
      \ILabel{def:propositional_alphabet/connectives/implication} \Def{implication} \( \rightarrow \) (see \fullref{def:material_implication})
      \ILabel{def:propositional_alphabet/connectives/equivalence} \Def{equivalence} \( \leftrightarrow \) (see \fullref{def:equivalence})
    \end{DefEnum}

    \ILabel{def:propositional_alphabet/parentheses} Parentheses \( ( \) and \( ) \) for defining the order of operations unambiguously (see \fullref{remark:propositional_formula_parentheses}).
  \end{DefEnum}

  \Fullref{remark:minimal_propositional_language} shows we can actually utilize a smaller propositional language without losing any of its semantics.
\end{definition}

\begin{definition}\label{def:propositional_formula}
  We now define the propositional formulas themselves.

  \begin{DefEnum}
    \ILabel{def:propositional_formula/language} The set of \Def{propositional formulas} \( \CF_B \) is the language \hyperref[def:grammar_derivation/generated_language]{generated} by the grammar
    \begin{AlignedEquation}\label{eq:def:propositional_formula/grammar}
      &\Phi \to p,                 && p \in \Bold{Prop} \\
      &\Phi \to \top \mid \bot     &&                   \\
      &\Phi \to \neg \Phi          &&                   \\
      &\Phi \to (\Phi \circ \Phi), && \circ \in \Sigma.
    \end{AlignedEquation}

    This grammar is unambiguous by \fullref{thm:propositional_formulas_are_unambiguous}.

    \ILabel{def:propositional_formula/subformula} If \( \varphi \) and \( \psi \) are formulas and \( \psi \) is a \hyperref[def:language/subword]{subword} of \( \varphi \), we say that \( \psi \) is a \Def{subformula} of \( \varphi \).

    \ILabel{def:propositional_formula/variables} For each formula \( \varphi \), we inductively\IND define its \Def{variables} to be elements of the set
    \begin{equation}\label{def:propositional_formula/var}
      \Bold{Var}(\varphi) \coloneqq \begin{cases}
        \varnothing,                              & \varphi \in \{ \top, \bot \}                     \\
        \{ P \},                                  & \varphi = P \in \Bold{Prop}                      \\
        \Bold{Var}(\psi),                         & \varphi = \neg \psi                              \\
        \Bold{Var}(\psi) \cup \Bold{Var}(\theta), & \varphi = (\psi \circ \theta), \circ \in \Sigma.
      \end{cases}
    \end{equation}
  \end{DefEnum}
\end{definition}

\begin{proposition}\label{thm:propositional_formulas_are_unambiguous}
  The \hyperref[def:grammar]{grammar} \eqref{eq:def:propositional_formula/grammar} is \hyperref[def:grammar_derivation/ambiguity]{unambiguous}.
\end{proposition}
\begin{proof}
  The proof is analogous to \fullref{ex:def:formal_grammar/natural_arithmetic}.
\end{proof}

\begin{definition}\label{def:material_implication}
  Fix the formula \( \varphi \coloneqq P \rightarrow Q \). We call formulas of this form \Def{material implications}. We will use the following terminology:

  \begin{DefEnum}
    \ILabel{def:material_implication/antecedent} \( P \) the \Def{antecedent} of \( \varphi \).

    \ILabel{def:material_implication/consequent} \( Q \) the \Def{consequent} of \( \varphi \).

    \ILabel{def:material_implication/inverse} The formula \( \neg P \rightarrow \neg Q \) is the \Def{inverse} of \( \varphi \).

    \ILabel{def:material_implication/converse} The formula \( Q \rightarrow P \) is the \Def{converse} of \( \varphi \).

    \ILabel{def:material_implication/contrapositive} The formula \( \neg Q \rightarrow \neg P \) is the \Def{contrapositive} of \( \varphi \). It is \hyperref[def:propositional_interpretation/equivalence]{equivalent} to the original formula by \fullref{def:boolean_equivalences/contrapositive}.
  \end{DefEnum}
\end{definition}

\begin{remark}\label{remark:propositional_formula_parentheses}
  We use the following two conventions regarding parentheses:
  \begin{RemEnum}
    \ILabel{remark:propositional_formula_parentheses/outermost} We may skip the outermost parentheses in formulas with top-level \hyperref[def:propositional_alphabet/connectives]{connectives}, e.g. we may write \( P \wedge Q \) rather than \( (P \wedge Q) \).

    \ILabel{remark:propositional_formula_parentheses/associative} Because of the associativity of \( \wedge \) and \( \vee \) (proven in \fullref{def:boolean_equivalences/associativity}), we may skip the parentheses in chains like
    \begin{equation*}
      ( \ldots ((P_1 \wedge P_2) \wedge P_3) \wedge \ldots \wedge P_{n-1} ) \wedge P_n.
    \end{equation*}
    and instead write
    \begin{equation*}
      P_1 \wedge P_2 \wedge \ldots \wedge P_{n-1} \wedge P_n.
    \end{equation*}
  \end{RemEnum}

  These are only notations shortcuts in the \hyperref[remark:metalanguage]{metalanguage} and the formulas themselves (as abstract mathematical objects) are still assumed to contain parentheses to avoid syntactic ambiguity (see \fullref{thm:propositional_formulas_are_unambiguous}).
\end{remark}

\begin{definition}\label{def:equivalence}
  Fix the formula \( \varphi \coloneqq P \leftrightarrow Q \). We call formulas of this form \Def{logical equivalence}. Note that \eqref{eq:def:boolean_equivalences/equivalence_via_implication} holds. Despite the symmetry of \( \wedge \) in \eqref{eq:def:boolean_equivalences/equivalence_via_implication}, there is an ordering in the set \( \{ P, Q \} \) of propositions and we use this ordering. Instead of \( Q \rightarrow P \), we usually write \( P \ImpliedBy Q \). We also establish the following terminology:
  \begin{DefEnum}
    \ILabel{def:equivalence/necessary} \( P \) is a \Def{necessary condition} for \( Q \).
    \ILabel{def:equivalence/sufficient} \( Q \) is a \Def{sufficient condition} for \( P \).
  \end{DefEnum}
\end{definition}

\begin{remark}\label{remark:statements_as_implications}
  Theorems in mathematics usually take the form of a material \hyperref[def:material_implication]{implication} \( P \rightarrow Q \) or \hyperref[def:equivalence]{equivalence} \( P \leftrightarrow Q \). Therefore the terminology of \fullref{def:material_implication} and \fullref{def:equivalence} usually applies to statements about mathematics.
\end{remark}

\begin{definition}\label{def:propositional_interpretation}
  A \Def{propositional interpretation} is a function \( I: \Bold{Prop} \to \{ T, F \} \).

  We define interpretation for formulas inductively\IND as
  \begin{BreakableAlign*}
    \varphi[I] \coloneqq \begin{cases}
      T,                           & \varphi = \top                                 \\
      F,                           & \varphi = \bot                                 \\
      I(P),                        & \varphi = P \in \Bold{Prop}                    \\
      H_\neg(\psi[I]),             & \varphi = \neg \psi                            \\
      H_\circ(\psi[I], \theta[I]), & \varphi = \psi \circ \theta, \circ \in \Sigma.
    \end{cases}
  \end{BreakableAlign*}

  We introduce some directly related notions:
  \begin{DefEnum}
    \ILabel{def:propositional_interpretation/satisfiability} If \( \varphi[I] = T \), we say that \( I \) is \Def{satisfies} \( \varphi \) and write \( I \models_B \varphi \).

    \ILabel{def:propositional_interpretation/entailment} If every interpretation that satisfies the set \( \Gamma \) of formulas also satisfies the formula \( \varphi \), we say that \( \Gamma \) \Def{entails} \( \varphi \) and write \( \Gamma \models_B \varphi \).

    \ILabel{def:propositional_interpretation/tautology} If all interpretations satisfy \( \varphi \), we call \( \varphi \) a \Def{tautology}.

    \ILabel{def:propositional_interpretation/contradiction} Dually, if no interpretations satisfy \( \varphi \), \( \varphi \) is a \Def{contradiction}.

    \ILabel{def:propositional_interpretation/equivalence} If \( \varphi[I] = \psi[I] \) for any interpretation \( I \), we say that \( \varphi \) and \( \psi \) are \Def{propositional semantic equivalence} and write \( \varphi \equiv_B \psi \).
  \end{DefEnum}
\end{definition}

\begin{proposition}\label{thm:boolean_equivalence_relation}
  The \hyperref[def:propositional_interpretation/equivalence]{propositional semantic equivalence} \( \equiv_B \) is an equivalence relation on the set \( \CF_B \) of propositional formulas.
\end{proposition}
\begin{proof}
  Follows from the equivalences in \fullref{def:equivalence_relation}.
\end{proof}

\begin{theorem}\label{thm:propositional_logic_boolean_algebra}
  The \hyperref[def:equivalence_relation]{quotient set} of propositional formulas \( \CF_B / \equiv_B \) is bijective with the set of all \hyperref[def:boolean_function]{Boolean functions} of arbitrary arity.
\end{theorem}
\begin{proof}
  Trivial.
\end{proof}

\begin{definition}\label{def:propositional_substition}
  If \( \varphi \) and \( \rho \) are formulas and \( \psi \) is a subformula of \( \varphi \), we define the \Def{substition} of \( \psi \) with \( \rho \) in \( \varphi \) as
  \begin{equation*}
    \varphi[\psi \to \rho] \coloneqq \begin{cases}
      \rho,                                                    &\varphi = \psi                                                                        \\
      \varphi,                                                 &\varphi \neq \psi \text{ and } \varphi \in \{ \top, \bot \} \cup \Bold{Prop}          \\
      \neg \theta[\psi \to \rho],                              &\varphi \neq \psi \text{ and } \varphi = \neg \theta                                  \\
      (\theta_1[\psi \to \rho] \circ \theta_2[\psi \to \rho]), &\varphi \neq \psi \text{ and } \varphi = (\theta_1 \circ \theta_2), \circ \in \Sigma.
    \end{cases}
  \end{equation*}
\end{definition}

\begin{proposition}\label{thm:propositional_substition_equivalence}
  If \( \psi \) is a subformula of \( \varphi \) and if \( \psi \equiv_B \rho \), then
  \begin{equation}\label{eq:thm:propositional_substition_equivalence}
    \varphi[\psi \to \rho] \equiv_B \varphi.
  \end{equation}
\end{proposition}
\begin{proof}
  We use structural induction\IND on \( \varphi \):

  \begin{itemize}
    \item If \( \varphi = \psi \), then \( \varphi[\psi \to \rho] = \rho \) and, by definition,
    \begin{equation*}
      \varphi = \psi \equiv_B \rho = \varphi[\psi \to \rho].
    \end{equation*}

    \item If \( \varphi \in \{ \top, \bot \} \cup \Bold{Prop} \), then \( \varphi[\psi \to \rho] = \varphi \) and \eqref{eq:thm:propositional_substition_equivalence} again holds trivially.

    \item If \( \varphi = \neg \theta \), then \( \varphi[\psi \to \rho] = \neg \theta[\psi \to \rho] \). For any interpretation \( I \),
    \begin{equation*}
      \varphi[\psi \to \rho][I]
      =
      H_{\neg} (\theta[\psi \to \rho][I])
      \overset {\TIndHyp} =
      H_{\neg} (\theta[I])
      =
      \varphi[I].
    \end{equation*}

    Therefore \eqref{eq:thm:propositional_substition_equivalence} holds.

    \item If \( \varphi = (\theta_1 \circ \theta_2), \circ \in \Sigma \), then for any interpretation \( I \),
    \begin{equation*}
      \varphi[\psi \to \rho][I]
      =
      H_{\circ} (\theta_1[\psi \to \rho][I], \theta_2[\psi \to \rho][I])
      \overset {\TIndHyp} =
      H_{\circ} (\theta_1[I], \theta_2[I])
      =
      \varphi[I].
    \end{equation*}

    Therefore \eqref{eq:thm:propositional_substition_equivalence} also holds.
  \end{itemize}

  We have verified that in all cases, \eqref{eq:thm:propositional_substition_equivalence} holds.
\end{proof}

\begin{remark}\label{remark:minimal_propositional_language}
  \Fullref{thm:propositional_logic_boolean_algebra} partitions the set \( \CF_B \) of propositional formulas into equivalence classes. For \hyperref[remark:syntax_vs_semantics]{semantical} concepts, it is immaterial which element of an equivalence class we consider. The concept of a \hyperref[def:boolean_closure]{complete set of Boolean operations} allows us to represent each formula using a subset of the \hyperref[def:propositional_alphabet/constants]{propositional constants}, \hyperref[def:propositional_alphabet/negation] and \hyperref[def:propositional_alphabet/connectives]{connectives}. \Fullref{ex:thm:posts_completeness_theorem} shows some concrete commonly used complete sets of Boolean operations.

  For example, \fullref{def:boolean_equivalences/connectives_via_and_or} shows how to express the rest of the connectives via \( \neg \), \( \wedge \) and \( \vee \) and \fullref{def:boolean_equivalences/constants} shows how to express the constants via \( \wedge \) and \( \vee \). This gives rise to the \hyperref[def:conjunctive_disjunctive_normal_form]{conjunctive normal form}.

  This is useful in \hyperref[def:propositional_interpretation/satisfiability]{satisfiability} proofs that rely on \hyperref[remark:induction]{structural induction} because it allows us to consider less cases in the induction.
\end{remark}

\begin{definition}\label{def:conjunctive_disjunctive_normal_form}\mbox{}
  \begin{DefEnum}
    \ILabel{def:conjunctive_disjunctive_normal_form/literal} A \Def{literal} is either a propositional variable \( L = P \) or a negation \( L = \neg P \) of a propositional variable.

    \ILabel{def:conjunctive_disjunctive_normal_form/normal_form} A propositional formula \( \varphi \) is in \Def{conjunctive normal form} (resp. \Def{disjunctive normal form}) if \( \varphi \) is a finite conjunction of disjunctions (resp. finite disjunction of conjunctions) of literals. That is, if \( \varphi \) is in conjunctive normal form, it has the form
    \begin{equation*}
      (L_{1,1} \vee \ldots \vee L_{1,n_1}) \wedge \cdots \wedge (L_{k,1} \vee \ldots \vee L_{k,n_k}).
    \end{equation*}

    \ILabel{def:conjunctive_disjunctive_normal_form/conjunct_disjunct} A \Def{disjunct} (resp. a \Def{conjunct}) is a set of literals, the difference between the two being the context in which they are used. To each formula in conjunctive normal form there corresponds a set of disjuncts and to each formula in disjunctive normal form there corresponds a set of conjuncts.
  \end{DefEnum}
\end{definition}

\begin{proposition}\label{thm:conjunctive_normal_form_reduction}
  Every propositional formula is \hyperref[def:propositional_interpretation/equivalence]{semantically equivalent} to at least one formula in \hyperref[def:conjunctive_disjunctive_normal_form/normal_form]{conjunctive normal form}.
\end{proposition}
\begin{proof}
  We simply need to utilize substitutions based on \fullref{def:boolean_equivalences/constants} and \fullref{def:boolean_equivalences/connectives_via_and_or}. \Fullref{thm:propositional_substition_equivalence} shows that the resulting formula will be equivalent to the original one.
\end{proof}
