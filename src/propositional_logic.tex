\subsection{Propositional logic}\label{subsec:propositional_logic}

\begin{definition}\label{def:propositional_alphabet}\MarginCite[102]{OpenLogic20201202}
  The \hyperref[def:language]{language} of propositional logic consists of \Def{propositional formulas}, defined in \fullref{def:propositional_language}. These formulas are formed over a set of symbols, called the \Def{propositional alphabet}, whose constituents are described below.

  \begin{DefEnum}
    \ILabel{def:propositional_alphabet/prop} A nonempty set \( \Bold{Prop} \) of \Def{propositional variables}.

    \ILabel{def:propositional_alphabet/constants} Two \Def{propositional constants} (also known as \Def{truth values}):
    \begin{DefEnum}
      \ILabel{def:propositional_alphabet/constants/top} \Def{top} \( \top \)
      \ILabel{def:propositional_alphabet/constants/bottom} \Def{bottom} \( \bot \)
    \end{DefEnum}

    \ILabel{def:propositional_alphabet/negation} \Def{Negation} \( \neg \)
    \ILabel{def:propositional_alphabet/connectives} The set \( \Sigma \) of \Def{propositional connectives}, namely
    \begin{DefEnum}
      \ILabel{def:propositional_alphabet/connectives/conjunction} \Def{conjunction} \( \wedge \) (also known as \hyperref[def:standard_boolean_operators]{\Def{and}} or \hyperref[def:lattice_operations]{\Def{meet}})
      \ILabel{def:propositional_alphabet/connectives/disjunction} \Def{disjunction} \( \vee \) (also known as \hyperref[def:standard_boolean_operators]{\Def{or}} or \hyperref[def:lattice_operations]{\Def{join}})
      \ILabel{def:propositional_alphabet/connectives/implication} \Def{implication} \( \rightarrow \) (see \fullref{def:material_implication})
      \ILabel{def:propositional_alphabet/connectives/equivalence} \Def{equivalence} \( \leftrightarrow \) (see \fullref{def:equivalence})
    \end{DefEnum}

    \ILabel{def:propositional_alphabet/parentheses} Parentheses \( ( \) and \( ) \) for defining the order of operations unambiguously (see \fullref{remark:propositional_formula_parentheses}).
  \end{DefEnum}

  \Fullref{remark:smaller_propositional_language} shows we can actually utilize a smaller propositional language without losing any of its semantics.
\end{definition}

\begin{definition}\label{def:propositional_language}\mbox{}
  \begin{DefEnum}
    \ILabel{def:propositional_language/grammar} Consider the following \hyperref[def:backus_naur_form]{grammar schema}:
    \begin{bnf*}
      \bnfprod{variable}   {P \in \Bold{Prop}} \\
      \bnfprod{connective} {\circ \in \Sigma} \\
      \bnfprod{formula}    {\bnfpn{variable} \bnfor} \\
      \bnfmore             {\bnfts{\( \top \)} \bnfor \bnfts{\( \bot \)} \bnfor} \\
      \bnfmore             {\bnfts{\neg} \bnfpn{formula} \bnfor} \\
      \bnfmore             {\bnfts{(} \bnfsp \bnfpn{formula} \bnfsp \bnfpn{connective} \bnfsp \bnfpn{formula} \bnfsp \bnfts{)}}
    \end{bnf*}

    Note that \( \Bold{Prop} \) may be infinite, in which case the grammars may have infinitely many rules. If needed, we can circumvent this by introducing an appropriate naming convention for variables, analogously to \fullref{ex:natural_arithmetic_grammar/backus_naur_form}.

    \ILabel{def:propositional_language/formula} The set of \Def{propositional formulas} \( \CF_B \) is the language \hyperref[def:grammar_derivation/grammar_language]{generated} by this grammar schema with \( \bnfpn{formula} \) as a starting rule.

    The grammar of propositional formulas is unambiguous by \fullref{thm:propositional_formulas_are_unambiguous}.

    \ILabel{def:propositional_language/subformula} If \( \varphi \) and \( \psi \) are formulas and \( \psi \) is a \hyperref[def:language/subword]{subword} of \( \varphi \), we say that \( \psi \) is a \Def{subformula} of \( \varphi \).

    \ILabel{def:propositional_language/variables} For each formula \( \varphi \), we inductively\IND define its \Def{variables} to be elements of the set
    \begin{equation}\label{def:propositional_language/var}
      \Bold{Var}(\varphi) \coloneqq \begin{cases}
        \varnothing,                              &\varphi \in \{ \top, \bot \}                     \\
        \{ P \},                                  &\varphi = P \in \Bold{Prop}                      \\
        \Bold{Var}(\psi),                         &\varphi = \neg \psi                              \\
        \Bold{Var}(\psi) \cup \Bold{Var}(\theta), &\varphi = (\psi \circ \theta), \circ \in \Sigma.
      \end{cases}
    \end{equation}
  \end{DefEnum}
\end{definition}

\begin{proposition}\label{thm:propositional_formulas_are_unambiguous}
  The grammar of \hyperref[def:propositional_language/formula]{propositional formulas} is \hyperref[def:grammar_derivation/ambiguity]{unambiguous}.
\end{proposition}
\begin{proof}
  The proof is analogous to \fullref{ex:natural_arithmetic_grammar/derivation}.
\end{proof}

\begin{definition}\label{def:material_implication}
  Fix the formula \( \varphi \coloneqq P \rightarrow Q \). We call formulas of this form \Def{material implications}. We will use the following terminology:

  \begin{DefEnum}
    \ILabel{def:material_implication/antecedent} \( P \) the \Def{antecedent} of \( \varphi \).

    \ILabel{def:material_implication/consequent} \( Q \) the \Def{consequent} of \( \varphi \).

    \ILabel{def:material_implication/inverse} The formula \( \neg P \rightarrow \neg Q \) is the \Def{inverse} of \( \varphi \).

    \ILabel{def:material_implication/converse} The formula \( Q \rightarrow P \) is the \Def{converse} of \( \varphi \).

    \ILabel{def:material_implication/contrapositive} The formula \( \neg Q \rightarrow \neg P \) is the \Def{contrapositive} of \( \varphi \). It is \hyperref[def:propositional_interpretation/equivalence]{equivalent} to the original formula by \fullref{thm:boolean_equivalences/contrapositive}.
  \end{DefEnum}
\end{definition}

\begin{remark}\label{remark:propositional_formula_parentheses}
  We use the following two conventions regarding parentheses:
  \begin{RemEnum}
    \ILabel{remark:propositional_formula_parentheses/outermost} We may skip the outermost parentheses in formulas with top-level \hyperref[def:propositional_alphabet/connectives]{connectives}, e.g. we may write \( P \wedge Q \) rather than \( (P \wedge Q) \).

    \ILabel{remark:propositional_formula_parentheses/associative} Because of the associativity of \( \wedge \) and \( \vee \) (see \fullref{def:standard_boolean_operators}), we may skip the parentheses in chains like
    \begin{equation*}
      ( \ldots ((P_1 \wedge P_2) \wedge P_3) \wedge \ldots \wedge P_{n-1} ) \wedge P_n.
    \end{equation*}
    and instead write
    \begin{equation*}
      P_1 \wedge P_2 \wedge \ldots \wedge P_{n-1} \wedge P_n.
    \end{equation*}
  \end{RemEnum}

  These are only notations shortcuts in the \hyperref[remark:metalanguage_syntax_and_semantics]{metalanguage} and the formulas themselves (as abstract mathematical objects) are still assumed to contain parentheses to avoid syntactic ambiguity (see \fullref{thm:propositional_formulas_are_unambiguous}).
\end{remark}

\begin{definition}\label{def:equivalence}
  Fix the formula \( \varphi \coloneqq P \leftrightarrow Q \). We call formulas of this form \Def{logical equivalence}. Note that \eqref{eq:thm:boolean_equivalences/equivalence_via_implication} holds. Despite the symmetry of \( \wedge \) in \eqref{eq:thm:boolean_equivalences/equivalence_via_implication}, there is an ordering in the set \( \{ P, Q \} \) of propositions and we use this ordering. Instead of \( Q \rightarrow P \), we usually write \( P \leftarrow Q \). We also establish the following terminology:
  \begin{DefEnum}
    \ILabel{def:equivalence/necessary} \( P \) is a \Def{necessary condition} for \( Q \).
    \ILabel{def:equivalence/sufficient} \( Q \) is a \Def{sufficient condition} for \( P \).
  \end{DefEnum}
\end{definition}

\begin{remark}\label{remark:statements_as_implications}
  Theorems in mathematics usually take the form of a material \hyperref[def:material_implication]{implication} \( P \rightarrow Q \) or \hyperref[def:equivalence]{equivalence} \( P \leftrightarrow Q \). Therefore the terminology of \fullref{def:material_implication} and \fullref{def:equivalence} usually applies to statements about mathematics. We do, however, prefer the notation \( P \implies Q \) and \( P \iff Q \) when dealing with actual statements.
\end{remark}

\begin{definition}\label{def:propositional_interpretation}
  A \Def{propositional interpretation} is a function \( I: \Bold{Prop} \to \{ T, F \} \). See \fullref{def:boolean_function} and \fullref{thm:propositional_connectives_ambiguity} for remarks regarding the \hyperref[def:boolean_algebra]{Boolean algebra} \( \{ T, F \} \) and the \hyperref[def:standard_boolean_operators]{standard Boolean operators}.

  We define interpretation for formulas inductively\IND as
  \begin{BreakableAlign*}
    \varphi\Val{I} \coloneqq \begin{cases}
      T,                               &\varphi = \top                                 \\
      F,                               &\varphi = \bot                                 \\
      I(P),                            &\varphi = P \in \Bold{Prop}                    \\
      \Ol{\psi\Val{I}},                &\varphi = \neg \psi                         \\
      \psi\Val{I} \circ \theta\Val{I}, &\varphi = \psi \circ \theta, \circ \in \Sigma.
    \end{cases}
  \end{BreakableAlign*}

  We introduce some directly related notions:
  \begin{DefEnum}
    \ILabel{def:propositional_interpretation/satisfiability} If \( \varphi\Val{I} = T \), we say that \( I \) is \Def{satisfies} \( \varphi \) and write \( I \models \varphi \).

    \ILabel{def:propositional_interpretation/entailment} If every interpretation that satisfies all formulas in the set \( \Gamma \) also satisfies the formula \( \varphi \), we say that \( \Gamma \) \Def{entails} \( \varphi \) and write \( \Gamma \models \varphi \).

    \ILabel{def:propositional_interpretation/tautology} If all interpretations satisfy \( \varphi \), we call \( \varphi \) a \Def{tautology}.

    \ILabel{def:propositional_interpretation/contradiction} Dually, if no interpretations satisfy \( \varphi \), \( \varphi \) is a \Def{contradiction}.

    \ILabel{def:propositional_interpretation/equivalence} If \( \varphi\Val{I} = \psi\Val{I} \) for every interpretation \( I \), we say that \( \varphi \) and \( \psi \) are \Def{semantically equivalent} and write \( \varphi \equiv \psi \).
  \end{DefEnum}
\end{definition}

\begin{definition}\label{def:propositional_formula_induced_function}
  Let \( \varphi \) be a propositional formula and let \( \Bold{Var}(\varphi) = \{ P_1, \ldots, P_n \} \). We define the \hyperref[def:boolean_function]{Boolean function}
  \begin{AlignedEquation}\label{eq:def:propositional_formula_induced_function}
    &\Fun_\varphi: \{ T, F \}^n \to \{ T, F \} \\
    &\Fun_\varphi(x_1, \ldots, x_n) \coloneqq \varphi\Val{I},
  \end{AlignedEquation}
  where \( I \) is any interpretation that satisfies
  \begin{equation*}
    I(P_k) = x_k, k = 1, \ldots, n.
  \end{equation*}
\end{definition}

\begin{proposition}\label{thm:propositional_formulas_equivalent_iff_same_boolean_functions}
  Two propositional formulas \( \varphi \) and \( \psi \) are semantically equivalent if and only if \( \Fun_\varphi = \Fun_\psi \).
\end{proposition}
\begin{proof}
  This is simply a restatement of \fullref{def:propositional_interpretation/equivalence}.
\end{proof}

\begin{proposition}\label{thm:boolean_equivalence_relation}
  The \hyperref[def:propositional_interpretation/equivalence]{semantic equivalence} \( \equiv \) is an equivalence relation on the set \( \CF_B \) of propositional formulas.
\end{proposition}
\begin{proof}
  Follows from the equivalences in \fullref{def:equivalence_relation}.
\end{proof}

\begin{theorem}\label{thm:propositional_formula_cosets_are_boolean_functions}
  The \hyperref[def:equivalence_relation]{quotient set} of propositional formulas \( \CF_B / \equiv \) is bijective with the set of all \hyperref[def:boolean_function]{Boolean functions} of arbitrary arity.
\end{theorem}
\begin{proof}
  Follows from \fullref{thm:propositional_formulas_equivalent_iff_same_boolean_functions}.
\end{proof}

\begin{corollary}\label{thm:propositional_logic_boolean_algebra}
  The \hyperref[def:equivalence_relation]{quotient set} of propositional formulas \( \CF_B / \equiv_B \) forms a boolean \hyperref[def:boolean_algebra]{algebra} with the following:
  \begin{RefList}
    \IRef{def:binary_lattice_operations/join} Finite joins are given by \hyperref[def:propositional_alphabet/connectives/disjunction]{disjunctions \( \vee \)}.
    \IRef{def:binary_lattice_operations/meet} Finite meets are given by \hyperref[def:propositional_alphabet/connectives/conjunction]{conjunctions \( \wedge \)}.
    \IRef{def:lattice/top} The top is the equivalence class \( [\top] \) of \hyperref[def:propositional_interpretation/tautology]{tautologies}.
    \IRef{def:lattice/bottom} The bottom is the equivalence class \( [\bot] \) of \hyperref[def:propositional_interpretation/contradiction]{contradictions}.
    \IRef{def:boolean_algebra} The complement of the coset \( [\varphi] \) is the \hyperref[def:propositional_alphabet/negation]{negation} coset \( [\neg \varphi] \).
  \end{RefList}
\end{corollary}
\begin{proof}
  Follows from \fullref{thm:propositional_formula_cosets_are_boolean_functions} and \fullref{thm:boolean_functions_form_boolean_algebra}.
\end{proof}

\begin{definition}\label{def:propositional_substition}
  \begin{DefEnum}
    \ILabel{def:propositional_substition/single} We define the \Def{substition} of the propositional formula \( \psi \) with \( \rho \) in \( \varphi \) as
    \begin{equation}\label{eq:def:propositional_substition/single}
      \varphi[\theta \mapsto \rho] \coloneqq \begin{cases}
        \rho,                                                          &\varphi = \theta \\
        \varphi,                                                       &\varphi \neq \theta \T{and} \varphi \in \{ \top, \bot \} \cup \Bold{Prop} \\
        \neg \psi[\theta \mapsto \rho],                                &\varphi \neq \theta \T{and} \varphi = \neg \psi \\
        \psi_1[\theta \mapsto \rho] \circ \psi_2[\theta \mapsto \rho], &\varphi \neq \theta \T{and} \varphi = \psi_1 \circ \psi_2, \circ \in \Sigma.
      \end{cases}
    \end{equation}

    \ILabel{def:propositional_substition/simultaneous} We will now define \Def{simultaneous substition} of \( \theta_1, \ldots, \theta_n \) with \( \rho_1, \ldots, \rho_n \). We wish to avoid the case where \( \theta_k \) is a subformula of \( \rho_{k-1} \) and it accidentally gets replaced during \( \varphi[\theta_{k-1} \mapsto \rho_{k-1}][\theta_k \mapsto \rho_k] \).

    Define
    \begin{equation*}
      \Cat{Bound} \coloneqq \Bold{Var}(\rho_1) \cup \ldots \cup \Bold{Var}(\rho_n).
    \end{equation*}
    and, for each variable \( P_k \) in \( \Cat{Bound} \), pick a variable \( Q_k \) from \( \Bold{Prop} \setminus \Bold{Bound} \). Let \( m \) be the \hyperref[def:cardinal]{cardinality} of \( \Bold{Bound} \). The simultaneous substitution can now be defined as
    \begin{align*}
      \varphi[\theta_1 \mapsto \rho_1, \ldots, \theta_n \mapsto \rho_n] \coloneqq \varphi
      [\theta_1 \mapsto \rho_1[P_1 \mapsto Q_1, \ldots, P_m \mapsto Q_m]] \\
      \vdots \hspace{3cm} \\
      [\theta_n \mapsto \rho_n[P_1 \mapsto Q_1, \ldots, P_m \mapsto Q_m]] \\
      [Q_1 \mapsto P_1, \ldots, Q_m \mapsto P_m].
    \end{align*}
  \end{DefEnum}
\end{definition}

\begin{proposition}\label{thm:propositional_substition_equivalence}
  If \( \psi \) is a subformula of \( \varphi \) and if \( \psi \equiv \theta \), then
  \begin{equation}\label{eq:thm:propositional_substition_equivalence}
    \varphi[\psi \mapsto \theta] \equiv \varphi.
  \end{equation}
\end{proposition}
\begin{proof}
  We use structural induction\IND on \( \varphi \):

  \begin{itemize}
    \item If \( \varphi = \psi \), then \( \varphi[\psi \mapsto \theta] = \theta \) and, by definition,
    \begin{equation*}
      \varphi = \psi \equiv \theta = \varphi[\psi \mapsto \theta].
    \end{equation*}

    \item If \( \varphi \in \{ \top, \bot \} \cup \Bold{Prop} \), then \( \varphi[\psi \mapsto \theta] = \varphi \) and \eqref{eq:thm:propositional_substition_equivalence} again holds trivially.

    \item If \( \varphi = \neg \theta \), then \( \varphi[\psi \mapsto \theta] = \neg \theta[\psi \mapsto \theta] \). For any interpretation \( I \),
    \begin{equation*}
      \varphi[\psi \mapsto \theta]\Val{I}
      =
      \Ol{\theta[\psi \mapsto \theta]\Val{I}}
      \overset {\IndHyp} =
      \Ol{\theta\Val{I}}
      =
      \varphi\Val{I}.
    \end{equation*}

    Therefore \eqref{eq:thm:propositional_substition_equivalence} holds in this case.

    \item If \( \varphi = (\theta_1 \circ \theta_2), \circ \in \Sigma \), then for any interpretation \( I \),
    \begin{equation*}
      \varphi[\psi \mapsto \theta]\Val{I}
      =
      \theta_1[\psi \mapsto \theta]\Val{I} \circ \theta_2[\psi \mapsto \theta]\Val{I}
      \overset {\IndHyp} =
      \theta_1\Val{I} \circ \theta_2\Val{I}
      =
      \varphi\Val{I}.
    \end{equation*}

    Therefore \eqref{eq:thm:propositional_substition_equivalence} holds in this case also.
  \end{itemize}

  We have verified that \eqref{eq:thm:propositional_substition_equivalence} holds in all cases.
\end{proof}

\begin{remark}\label{remark:smaller_propositional_language}
  For \hyperref[remark:metalanguage_syntax_and_semantics]{semantical} concepts, it is immaterial which element of an equivalence class we consider. The concept of a \hyperref[def:boolean_closure]{complete set of Boolean operations} allows us to represent each formula using a strict subset of the \hyperref[def:propositional_alphabet/constants]{propositional constants}, \hyperref[def:propositional_alphabet/negation]{negation} and \hyperref[def:propositional_alphabet/connectives]{connectives}. \Fullref{ex:thm:posts_completeness_theorem} shows some concrete commonly used complete sets of Boolean operations. This is useful in certain syntactic manipulations, however it is also useful in \hyperref[def:propositional_interpretation/satisfiability]{satisfiability} proofs that rely on \hyperref[remark:induction]{structural induction} because it allows us to consider less cases in the induction.

  See the \hyperref[def:conjunctive_disjunctive_normal_form]{conjunctive normal form} for a concrete example.
\end{remark}

\begin{definition}\label{def:conjunctive_disjunctive_normal_form}\mbox{}
  \begin{DefEnum}
    \ILabel{def:conjunctive_disjunctive_normal_form/literal} A \Def{literal} is either a propositional variable \( L = P \) or a negation \( L = \neg P \) of a propositional variable.

    \ILabel{def:conjunctive_disjunctive_normal_form/normal_form} A propositional formula \( \varphi \) is in \Def{conjunctive normal form} (resp. \Def{disjunctive normal form}) if \( \varphi \) is a finite conjunction of disjunctions (resp. finite disjunction of conjunctions) of literals. That is, if \( \varphi \) is in conjunctive normal form, it has the form
    \begin{equation*}
      (L_{1,1} \vee \ldots \vee L_{1,n_1}) \wedge \cdots \wedge (L_{k,1} \vee \ldots \vee L_{k,n_k}).
    \end{equation*}

    \ILabel{def:conjunctive_disjunctive_normal_form/conjunct_disjunct} A \Def{disjunct} (resp. a \Def{conjunct}) is a set of literals, the difference between the two being the context in which they are used. To each formula in conjunctive normal form there corresponds a set of disjuncts and to each formula in disjunctive normal form there corresponds a set of conjuncts.
  \end{DefEnum}
\end{definition}

\begin{algorithm}\label{alg:conjunctive_normal_form_reduction}
  Let \( \varphi \) be any formula. We explicitly derive a formula in conjunctive normal form that is \hyperref[def:propositional_interpretation/equivalence]{semantically equivalent} to \( \varphi \).

  \begin{AlgEnum}
    \ILabel{alg:conjunctive_normal_form_reduction/constants} Choose\AOC any variable \( P \) and substitute
    \begin{align*}
      \top \mapsto P \vee \neg P && \bot \mapsto P \wedge \neg P
    \end{align*}
    to get rid of the \hyperref[def:propositional_alphabet/constants]{propositional constants}.

    \ILabel{alg:conjunctive_normal_form_reduction/iff} For any subformulas \( \psi \) and \( \theta \) of \( \varphi \), perform the substitution
    \begin{equation*}
      \psi \leftrightarrow \theta \mapsto (\psi \rightarrow \theta) \wedge (\theta \rightarrow \psi)
    \end{equation*}
    to get rid of \hyperref[def:propositional_alphabet/connectives/equivalence]{equivalence connectives}. Semantic equivalence with \( \varphi \) is then justified by \fullref{thm:boolean_equivalences/equivalence_via_implication}.

    \ILabel{alg:conjunctive_normal_form_reduction/implies} For any subformulas \( \psi \) and \( \theta \) of \( \varphi \), perform the substitution
    \begin{equation*}
      \psi \rightarrow \theta \mapsto \neg \psi \vee \theta.
    \end{equation*}
    to get rid of \hyperref[def:propositional_alphabet/connectives/implication]{implications}. Equivalence with \( \varphi \) is justified by \fullref{thm:boolean_equivalences/implication_cnf}.

    \ILabel{alg:conjunctive_normal_form_reduction/de_morgan} For any subformulas \( \psi \) and \( \theta \) of \( \varphi \), use \fullref{thm:de_morgans_laws}:
    \begin{align*}
      \neg(\psi \vee \theta) \mapsto \neg \psi \wedge \neg \theta
      &&
      \neg(\psi \wedge \theta) \mapsto \neg \psi \vee \neg \theta.
    \end{align*}

    In order to ensure that \hyperref[def:propositional_alphabet/negation]{negation} is only present before propositional variables, repeat \ref{alg:conjunctive_normal_form_reduction/de_morgan} until nothing is substituted anymore.

    \ILabel{alg:conjunctive_normal_form_reduction/distributivity} Finally, for any subformulas \( \psi \), \( \theta \) and \( \theta \) of \( \varphi \), use \hyperref[eq:def:distributive_lattice/distributivity]{distributivity} of \( \wedge \) over \( \vee \):
    \begin{equation*}
      \psi \vee (\theta \wedge \theta) \mapsto (\psi \vee \theta) \wedge (\theta \vee \theta).
    \end{equation*}

    In order to ensure that conjunction is always one level higher than disjunction, repeat \ref{alg:conjunctive_normal_form_reduction/distributivity} until nothing is substituted anymore.
  \end{AlgEnum}

  The resulting formula is in conjunctive normal form. By \fullref{thm:propositional_substition_equivalence}, it is equivalent to \( \varphi \).
\end{algorithm}
