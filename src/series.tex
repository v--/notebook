\subsection{Series}\label{subsec:series}

Here \( (X, \Norm) \) will refer to a Banach space over \( \K \).

\begin{definition}\label{def:convergent_series}
  When extending addition to a countable amount of terms, we need to impose some regularity conditions to avoid contradictions. The topologies of \( \R \) and \( \C \) are complete and allow us to define convergent and divergent series. We define series in great generality because the theory easily allows it.

  A \Def{numeric series} or simply \Def{series} is an infinite sequence \( x_0, x_1, \ldots \in X \), which we call \Def{terms}, usually written as
  \begin{equation}\label{def:convergent_series/series}
    \sum_{k=0}^\infty x_k.
  \end{equation}

  To each series, there corresponds its sequence of \Def{partial sums}
  \begin{equation*}
    S_n \coloneqq \sum_{k=0}^n x_k, n = 0, 1, 2, \ldots.
  \end{equation*}

  We can equivalently define a series as a sequence of partial sums and then recover the terms as
  \begin{equation*}
    x_k \coloneqq \begin{cases}
      S_0,           &k = 0, \\
      S_k - S_{k-1}, &k > 0
    \end{cases}
  \end{equation*}

  We say that the series \cref{def:convergent_series/series} \Def{converges} to a value \( x \) if \( \lim_{n \to \infty} S_n = x \) in the sense of \cref{thm:metric_topology_convergence}. The value \( x \) is called the \Def{sum} of the series.

  If a series does not converge, we say that it is \Def{divergent}.

  If the related series
  \begin{equation}\label{def:convergent_series/absolute_series}
    \sum_{k=0}^\infty \Norm{x_k}
  \end{equation}
  converges, we say that \cref{def:convergent_series/series} is \Def{absolutely convergent}.
\end{definition}

\begin{proposition}\label{thm:absolutely_convergent_series_is_convergent}
  An absolutely convergent series is convergent.
\end{proposition}
\begin{proof}
  Suppose that \cref{def:convergent_series/absolute_series} converges.

  By the triangle inequality, for each index \( n \) we have
  \begin{equation*}
    \Norm{\sum_{k=0}^n a_k} \leq \sum_{k=0}^n \Norm{a_k} \leq \sum_{k=0}^\infty \Norm{a_k}.
  \end{equation*}

  Thus the sequence \( \left\{ \Norm{\sum_{k=0}^{n} a_k} \right\}_{n=0}^\infty \) is a bounded (by \( \sum_{k=0}^\infty \Norm{a_k} \)) monotone sequence, which by \cref{thm:real_monotone_sequence_converges_iff_bounded} is convergent.

  Therefore the series \cref{def:convergent_series/series} is convergent.
\end{proof}

\begin{corollary}
  Convergence of the complex series \cref{def:convergent_series/series} can be established using the convergence of the nonnegative series \cref{def:convergent_series/absolute_series}.

  The convergence of the latter can be established using techniques in \cref{subsec:real_series} like \fullref{thm:cauchys_root_test} or \fullref{thm:dalamberts_ratio_test}.
\end{corollary}

\begin{proposition}\label{thm:infinitary_triangle_inequality}
  For every series \cref{def:convergent_series/series} we have
  \begin{equation}\label{thm:infinitary_triangle_inequality/inequality}
    \Norm{\sum_{k=0}^\infty a_k} \leq \sum_{k=0}^\infty \Norm{a_k},
  \end{equation}
  where both limits are allowed to be infinite.
\end{proposition}
\begin{proof}
  If the series on the right diverges, the inequality is obviously true.

  Suppose that it is convergent. By \cref{thm:absolutely_convergent_series_is_convergent}, the limit
  \cref{def:convergent_series/series} exists.

  By the triangle inequality, for each index \( n \) we have
  \begin{equation*}
    \Norm{\sum_{k=0}^n a_k} \leq \sum_{k=0}^n \Norm{a_k}.
  \end{equation*}

  By \cref{thm:one_sided_squeeze_lemma}, since both sequences are convergent, we obtain \cref{thm:infinitary_triangle_inequality/inequality}.
\end{proof}

\begin{proposition}\label{thm:convergent_series_terms_vanish}
  The terms of the convergent series \cref{def:convergent_series/series} vanish as \( k \to \infty \), that is,
  \begin{equation*}
    \lim_{k \to \infty} a_k = 0.
  \end{equation*}
\end{proposition}
\begin{proof}
  Since the series is convergent, its sequence of partial sums converges, i.e. the partial sums get arbitrarily close to each other. Then
  \begin{equation*}
    \Norm{a_n} = \Norm{S_n - S_{n-1}} \to 0.
  \end{equation*}
\end{proof}

\begin{proposition}[Cauchy's series convergence criterion]\label{thm:cauchy_series_convergence_criterion}\cite[3.22]{Rudin1991}
  The series \cref{def:convergent_series/series} converges if and only if for every \( \varepsilon > 0 \) there exists an index \( K \) such that
  \begin{equation*}
    \Norm{\sum_{k=m}^n a_k} < \varepsilon \quad\forall m, n \geq K.
  \end{equation*}
\end{proposition}
\begin{proof}
  This is simply a restatement of \cref{thm:cauchys_convergence_criterion}.
\end{proof}

\begin{proposition}[Cauchy's series continuity criterion]\label{thm:cauchy_series_continuity_criterion}\cite[\textnumero 265]{Фихтенгольц1968/2}
  Fix a topological space \( S \). Let \( \{ f_k \}_{k=0}^\infty \) be a sequence of continuous functions from \( S \) to \( X \).

  Given a corresponding sequence of scalars \( \{ a_0 \}_{k=0}^\infty \subseteq \K \), define the function \( f: S \to X \) as
  \begin{equation}\label{thm:cauchy_series_continuity_criterion/function}
    f(x) \coloneqq \sum_{k=0}^\infty a_k f_k(x).
  \end{equation}

  A sufficient condition for \( f \) to be continuous is that for every \( \varepsilon > 0 \) there exists an index \( K \) such that
  \begin{equation*}
    \Norm{\sum_{k=m}^n a_k f(x)} < \varepsilon \quad\forall m, n \geq K
  \end{equation*}
  simultaneously for all \( x \in S \).
\end{proposition}
\begin{proof}
  This is simply a restatement of \cref{def:uniform_limit_of_continuous_functions} in the style of \cref{thm:cauchy_series_convergence_criterion}.
\end{proof}

\begin{corollary}[Weierstrass' criterion]\label{thm:weierstrass_series_criterion}\cite[\textnumero 265]{Фихтенгольц1968/2}
  Let \( S \) be any set and \( \{ f_k \}_{k=0}^\infty \) be a sequence of functions from \( S \) to \( X \), not necessarily continuous. Consider the series \cref{thm:cauchy_series_continuity_criterion/function}. If
  \begin{equation*}
    \forall k \in \Z^{>0} \ \exists M_k \in \R^{>0} \ \forall x \in S : \Norm{a_k f_k(x)} < M_k
  \end{equation*}
  and if the series
  \begin{equation}\label{thm:weierstrass_series_criterion/dominating}
    \sum_{k=0}^\infty M_k
  \end{equation}
  converges, then the limit \cref{thm:cauchy_series_continuity_criterion/function} exists for every \( x \in S \) and, furthermore, the series converges absolutely and uniformly.

  In analogy to \cref{thm:positive_series_comparison}, we say that the series \cref{thm:weierstrass_series_criterion/dominating} \Def{dominates} the series \cref{thm:cauchy_series_continuity_criterion/function}.

  In particular, if the functions \( f_k(x), k = 0, 1, \ldots \) are continuous (resp. uniformly continuous), so is \( f(x) \).
\end{corollary}
\begin{proof}
  By \cref{thm:positive_series_comparison}, the series
  \begin{equation*}
    \sum_{k=0}^\infty \Norm{a_k f_k(x)}
  \end{equation*}
  converges for any \( x \in S \), hence \cref{thm:cauchy_series_continuity_criterion/function} converges absolutely for any \( x \in S \).

  Furthermore, each of the functions \( a_k f_k(x) \) is bounded by \( B(0, M_k) \) and \( M_k \) does not depend on \( x \), hence the convergence is uniform.

  The rest of the theorem follows from \cref{def:uniform_limit_of_continuous_functions}.
\end{proof}
