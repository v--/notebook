\subsection{Integer arithmetic}\label{subsec:integer_arithmetic}

\begin{definition}\label{def:integer_division}\cite[1]{Knapp2016BAlg}
  Multiplication is not invertible in the ring of integers \( \Z \), but we can instead define division with remainders (that is usually called \Def{integer division}). Let \( a, b, q, r \in \Z \) be integers\Tinyref{def:integers} and assume that the relation
  \begin{equation*}
    a = bq + r
  \end{equation*}
  holds.

  We say that \( b \) \Def{divides} \( a \) with \Def{quotient} \( q \) and \Def{remainder} \( r \).

  If the remainder \( r \) is zero, we say that \( b \) is a \Def{divisor} of \( a \) or that \( a \) is a \Def{factor} of \( b \) and write \( b \mid a \).

  A \Def{trivial} divisor of \( a \) is either \( a \), \( -a \) or a unit of \( \Z \), i.e. either \( 1 \) or \( -1 \). All other divisors are called \Def{nontrivial}.
\end{definition}

\begin{proposition}\label{thm:integer_division_algorithm}\cite[2]{Knapp2016BAlg}
  Given integers \( a, b \in \Z \) with \( b \neq 0 \), there exist unique integers \( r, q \in \Z \) such that
  \begin{equation*}
    a = bq + r.
  \end{equation*}
\end{proposition}
\begin{proof}
  Let \( Q \) be the set
  \begin{equation*}
    Q \coloneqq \{ q \in \Z : -\Abs{a} \leq bq \leq \Abs{a} \}
  \end{equation*}
\end{proof}
