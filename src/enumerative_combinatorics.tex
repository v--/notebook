\subsection{Enumerative combinatorics}\label{subsec:enumerative_combinatorics}

This subsection lists several results of various importance that don't really belong to any more consistent theory.

\begin{theorem}[Dirichlet's pigeonhole principle]\label{def:pigeonhole_principle}
  If we are given more pigeons than pigeonholes, then at least one pigeonhole must contain multiple pigeons in it.

  More formally, if \( \card(A) > \card(B) \), then there exists no injective function from \( A \) to \( B \).
\end{theorem}
\begin{proof}
  This is a corollary of \fullref{thm:set_domination_relation_trichotomy}.
\end{proof}

\begin{definition}\label{def:binomial_coefficient}
  The \term{binomial coefficient} of the \hyperref[def:integer_signum]{nonnegative integers} \( n \) and \( k \) is
  \begin{equation*}
    \binom n k \coloneqq \frac {n!} {k!(n-k)!}
  \end{equation*}

  They are motivated by \fullref{thm:binomial_theorem}.
\end{definition}

\begin{theorem}[Pascal's identity]\label{thm:pascals_identity}
  \hyperref[def:binomial_coefficient]{Binomial coefficients} have the following property:
  \begin{equation}\label{eq:thm:pascals_identity}
    \binom n k = \binom {n - 1} k + \binom {n - 1} {k - 1}.
  \end{equation}
\end{theorem}
\begin{proof}
  \begin{balign*}
    \binom {n - 1} k + \binom {n - 1} {k - 1}
    &=
    \frac {(n - 1)!} {k! (n - 1 - k)!} + \frac {(n - 1)!} {(k - 1)! (n - k)!}
    = \\ &=
    \frac {(n - 1)!} {(k - 1)! (n - 1 - k)!} \bracks*{ \frac 1 k + \frac 1 {n - k} }
    = \\ &=
    \frac {(n - 1)!} {(k - 1)! (n - 1 - k)!} \frac n {k(n - k)}
    = \\ &=
    \frac {n!} {k! (n - k)!}
    = \\ &=
    \binom n k.
  \end{balign*}
\end{proof}

\begin{theorem}[Newton's binomial theorem]\label{thm:binomial_theorem}
  If, in some \hyperref[def:semiring]{semiring}, the members \( x \) and \( y \) commute (i.e. \( xy = yx \)), then
  \begin{equation}\label{eq:thm:binomial_theorem}
    (x + y)^n = \sum_{k=0}^n \binom n k x^k y^{n-k}.
  \end{equation}
\end{theorem}
\begin{proof}
  We use induction on \( n \). For \( n = 0 \), the theorem trivially holds. Assume that the theorem holds for \( 1, \ldots, n \). Then
  \begin{balign*}
    (x + y)^{n+1}
     & =
    x (x + y)^n + y (x + y)^n
    = \\ &=
    \sum_{k=0}^n \binom n k x^{k+1} y^{n-k} + y \sum_{k=0}^n \binom n k x^k y^{n-k}
    = \\ &=
    x^{n+1} + y \sum_{k=0}^{n-1} \binom n k x^{k+1} y^{n-(k+1)} + y \sum_{k=0}^n \binom n k x^k y^{n-k}
    = \\ &=
    x^{n+1} + y \parens*{ \sum_{k=1}^n \binom n {k-1} x^k y^{n-k} + y^n \sum_{k=1}^n \binom n k x^k y^{n-k} } + y^{n+1}
    = \\ &\reloset {\eqref{eq:thm:pascals_identity}} =
    x^{n+1} + y \sum_{k=1}^n \binom {n+1} k x^k y^{n-k} + y^{n+1}
    = \\ &=
    \sum_{k=0}^n \binom {n+1} k x^k y^{(n+1)-k}.
  \end{balign*}
\end{proof}

\begin{proposition}\label{thm:xn_minus_one_factorization}
  For every \hyperref[def:ring]{ring} element \( x \) and every nonnegative integer \( n \), we have
  \begin{equation}\label{eq:thm:xn_minus_one_factorization}
    x^{n + 1} - 1 = (x - 1)(x^n + x^{n-1} + \cdots + 1) = (x - 1) \sum_{k=0}^n x^k.
  \end{equation}
\end{proposition}
\begin{proof}
  We use induction on \( n \).
  \begin{itemize}
    \item If \( n = 0 \), then
    \begin{equation*}
      x^1 - 1 = x - 1 = (x - 1) 1.
    \end{equation*}

    \item If \eqref{eq:thm:xn_minus_one_factorization} holds, then
    \begin{equation*}
      x^{n + 2} - 1
      =
      (x^{n + 1} - 1)x + (x - 1)
      =
      \parens{ (x - 1) \sum_{k=0}^n x^k }x + (x - 1)
      =
      (x - 1) \sum_{k=0}^{n+1} x^k.
    \end{equation*}
  \end{itemize}
\end{proof}

\begin{definition}\label{def:factorial}
  The \term{factorial} of a \hyperref[def:integer_signum]{nonnegative integer} \( n \) is defined recursively as
  \begin{equation*}
    n! \coloneqq \begin{cases}
      1,          &n = 0 \\
      (n - 1)! n, &n > 0.
    \end{cases}
  \end{equation*}
\end{definition}

\begin{proposition}\label{thm:gamma_function_interpolates_factorial}
  For every \hyperref[def:integer_signum]{nonnegative integer} \( n \) we have
  \begin{equation*}
    \Gamma(n + 1) \coloneqq n!,
  \end{equation*}
  where \( \Gamma \) is the Gamma function defined in \fullref{def:gamma_function}.
\end{proposition}
\begin{proof}
  We use induction on \( n \).
  \begin{itemize}
    \item If \( n = 0 \), then
    \begin{equation*}
      \Gamma(1)
      =
      \int_0^\infty x^0 e^{-x} \dl x
      =
      -e^{-x}\restr_{x=0}^\infty
      =
      -\underbrace{\lim_{x \to \infty} e^{-x}}_{0} + 1
      =
      1
      =
      0!
    \end{equation*}

    \item If \( n > 0 \) and \( \Gamma(n) = (n - 1)! \), then
    \begin{balign*}
      \Gamma(n + 1)
      &=
      \int_0^\infty x^n \cdot e^{-x} \dl x
      = \\ &=
      \underbrace{(- x^n e^{-x})\restr_{x=0}^\infty}_{-(0 - 0)} + n \int_0^\infty e^{-x} x^{n-1} \dl x
      = \\ &=
      n \Gamma(n)
      = \\ &=
      n (n - 1)!
      = \\ &=
      n!
    \end{balign*}
  \end{itemize}
\end{proof}

\begin{theorem}[Stirling's factorial approximation]\label{thm:stirlings_factorial_approximation}
  For every \hyperref[def:integer_signum]{nonnegative integer} \( n \) there exists some constant \( \theta \in (0, 1) \) such that
  \begin{equation*}
    n! = \sqrt{2 \pi n} \cdot \parens*{ \frac n e }^n \cdot e^{\frac 1 {12n + \theta}}.
  \end{equation*}
\end{theorem}
\begin{proof}
  Follows from \fullref{thm:gamma_function_interpolates_factorial} and \fullref{thm:stirlings_gamma_approximation}.
\end{proof}

\begin{remark}\label{rem:double_index_maps}
  We want to be able to map single indices to double indices and vice versa, for example for the purpose of \fullref{thm:matrix_spaces_are_tuple_spaces}. As an example, we want to be able to \enquote{linearize} an \( m \times n \) matrix such as the \( 2 \times 3 \) matrix
  \begin{equation}\label{eq:rem:double_index_maps/example/matrix}
    \begin{pmatrix}
      1 & 2 & 3 \\
      4 & 5 & 6
    \end{pmatrix}
  \end{equation}
  into the tuple
  \begin{equation}\label{eq:rem:double_index_maps/example/row_major}
    (1, 2, 3, 4, 5, 6)
  \end{equation}
  and vice versa. This is called \term{row-major order} of the elements of a matrix. The \term{column-major order} would instead be
  \begin{equation}\label{eq:rem:double_index_maps/example/column_major}
    (1, 4, 2, 5, 3, 6).
  \end{equation}

  Let \( m \) and \( n \) be \hyperref[def:integer_signum]{positive integers}. We will explicitly define functions for linearizing a matrix like \eqref{eq:rem:double_index_maps/example/matrix} into its row-major order \eqref{eq:rem:double_index_maps/example/row_major}. Define the sets
  \begin{align*}
    S &\coloneqq \overbrace{ \set{ 1, \ldots, mn - 1, mn } }^{\T{single indices}}
    \\
    D &\coloneqq \underbrace{ \set{ 1, \ldots, m } \times \set{ 1, \ldots, n } }_{\T{double indices}}
  \end{align*}
  and the mutually inverse operations
  \begin{align}
    &\begin{aligned}\label{eq:rem:double_index_maps/sharp}
      &\sharp: S \to D \\
      &\sharp(k) \coloneqq \parens[\Big]{ \quot(k - 1, m) + 1, \rem(k - 1, m) + 1 } \\
    \end{aligned}
    \\[0.5\baselineskip]
    &\begin{aligned}\label{eq:rem:double_index_maps/flat}
      &\flat: D \to S \\
      &\flat(i, j) \coloneqq (i - 1) \cdot m + (j - 1) + 1.
    \end{aligned}
  \end{align}

  The operation \( \sharp \) encodes the matrix \eqref{eq:rem:double_index_maps/example/matrix} into its row-major order \eqref{eq:rem:double_index_maps/example/row_major} and \( \flat \) does the opposite. Both operations are trivial except for the shifting needed in to allow us to use \hyperref[def:euclidean_domain]{remainders and quotients}.

  We can easily verify that \( \sharp \) is a \hyperref[def:morphism_invertibility/left_invertible]{left inverse} of \( \flat \) (note that \( j < m \)):
  \begin{align*}
    \sharp(\flat(i, j))
    &=
    \sharp\parens[\Big]{ (i - 1) \cdot m + (j - 1) + 1 }
    = \\ &=
    \parens[\Big]{ \quot(\cdots, m) + 1, \rem(\cdots, m) + 1 }
    = \\ &=
    \parens[\Big]{ (i - 1) + 1, (j - 1) + 1 }
    = \\ &=
    (i, j).
  \end{align*}

  We can just as easily verify that \( \flat \) is a \hyperref[def:morphism_invertibility/right_invertible]{right inverse} of \( \sharp \):
  \begin{align*}
    \flat(\sharp(k))
    &=
    \flat\parens[\Big]{ \quot(k, m) + 1, \rem(k, m) + 1 }
    = \\ &=
    \quot(k, m) \cdot m + \rem(k, m)
    = \\ &=
    k.
  \end{align*}

  Hence, \( \sharp \) is fully invertible with inverse \( \flat \). By \fullref{thm:function_invertibility_categorical/fully_invertible}, it is bijective.
\end{remark}
