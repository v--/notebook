\subsection{Graphs}\label{subsec:graphs}

\begin{definition}\label{def:graph}
  Fix a \hyperref[def:set]{set} \( V \), whose members we will call \term{vertices} or \term{nodes}, and a set \( E \), whose members we will call \term{arcs} or \term{edges}.

  A \term{graph} is a way to associate to every edge its \term{endpoints} in \( V \). We have different types of graphs depending on how exactly this association happens. Unfortunately, in the wild, the term \enquote{graph} may refer to any of them, except perhaps a hypergraph.

  All of these concepts are unrelated to the \hyperref[def:multi_valued_function/graph]{graph of a function}, except perhaps by accident.

  The figures here are not merely illustrative but are actually \hyperref[def:quiver_geometric_realization/embedding]{graph embeddings}.

  \begin{thmenum}[series=def:graph]
    \thmitem{def:graph/hypergraph} Suppose that we have a \hyperref[def:multi_valued_function/total]{total multi-valued function} \( \mscrE: E \multto V \) that gives us a nonempty set of endpoints for each edge. We call the triple \( H \coloneqq (V, E, \mscrE) \) a \term{hypergraph}.

    \begin{figure}[h]
      \begin{equation}\label{eq:fig:def:graph/hypergraph}
        \begin{aligned}
          \includegraphics{figures/eq__fig__def__graph__hypergraph.pdf}
        \end{aligned}
      \end{equation}
      \caption{A hypergraph containing four edges with two endpoints and one edge with four.}\label{fig:def:graph/hypergraph}
    \end{figure}

    If two edges have the same endpoints, we say that they are parallel.

    It is common to define a hypergraph as a family of nonempty sets \( E \), in which case \( V \coloneqq \bigcup E \) and \( \mscrE \) is the \hyperref[def:multi_valued_function/identity]{identity function} on \( E \).

    \thmitem{def:graph/undirected_multigraph} We say that the hypergraph \( M \coloneqq (V, E, \mscrE) \) is an \term{undirected multigraph} if, for every edge \( e \in E \), the set \( \mscrE(e) \) of endpoints has at most two vertices. If an edge has only one vertex, we call it a \term{loop}.

    \begin{figure}[h]
      \begin{equation}\label{eq:fig:def:graph/undirected_multigraph}
        \begin{aligned}
          \includegraphics{figures/eq__fig__def__graph__undirected_multigraph.pdf}
        \end{aligned}
      \end{equation}
      \caption{An undirected multigraph with two pairs of parallel edges and a loop.}\label{fig:def:graph/undirected_multigraph}
    \end{figure}

    \thmitem{def:graph/simple_undirected} We say that the undirected multigraph \( G = (V, E, \mscrE) \) is a \term{simple undirected graph} if it contains no loops or pairs of parallel edges.

    \begin{figure}[h]
      \begin{equation}\label{eq:fig:def:graph/simple_undirected}
        \begin{aligned}
          \includegraphics{figures/eq__fig__def__graph__simple_undirected.pdf}
        \end{aligned}
      \end{equation}
      \caption{An simple undirected graph.}\label{fig:def:graph/simple_undirected}
    \end{figure}

    It is common to define a simple undirected graph as a pair \( (V, E) \), where \( E \) is a set of unordered pairs \( \set{ u, v } \) of distinct vertices.

    \thmitem{def:graph/quiver} Suppose that we have a pair of \hyperref[def:function]{total single-valued functions} \( h: E \to V \), which gives the \term{head} of an edge, and \( t: E \to V \), which gives the \term{tail} of an edge. The head and tail of an edge are its endpoints.

    We call the quadruple \( Q \coloneqq (V, E, h, t) \) a \term{directed multigraph} or \term{quiver}.

    We can define a quiver as an undirected multigraph with additional functions \( h \) and \( t \). This will not help us, unfortunately. See \fullref{rem:graph_endpoint_etymology} for further discussion of the topic.

    \begin{figure}[h]
      \begin{equation}\label{eq:fig:def:graph/quiver}
        \begin{aligned}
          \includegraphics{figures/eq__fig__def__graph__quiver.pdf}
        \end{aligned}
      \end{equation}
      \caption{A quiver with a pair of parallel arcs, a pair of opposing arcs and a loop.}\label{fig:def:graph/quiver}
    \end{figure}

    We say that the edges are directed because there is a distinction between the two endpoints. It is common to reserve the term \term{arc} for directed edges and denote them as \( u \to v \). We say that \( u \) is a \term{predecessor} of \( v \) and that \( v \) is a \term{successor} of \( u \). The arcs \( u \to v \) and \( v \to u \) are \term{opposing}.

    See \fullref{rem:graph_endpoint_etymology} for a discussion of the terminology.

    The following set will be useful, especially in \fullref{sec:category_theory}:
    \begin{equation}\label{eq:def:graph/quiver/arc_set}
      M(u, v) \coloneqq \set{ e \in E \given h(e) = u \T{and} t(e) = v }.
    \end{equation}

    We call \( M(u, v) \) the \term{arc set} of \( u \) and \( v \).

    \thmitem{def:graph/simple_directed} We say that the quiver \( G = (V, E, h, t) \) is a \term{directed graph} or \term{digraph} if it has no loops or parallel arcs.

    \begin{figure}[h]
      \begin{equation}\label{eq:fig:def:graph/simple_directed}
        \begin{aligned}
          \includegraphics{figures/eq__fig__def__graph__simple_directed.pdf}
        \end{aligned}
      \end{equation}
      \caption{A simple directed graph.}\label{fig:def:graph/simple_directed}
    \end{figure}

    It is common to define a directed graph as a pair \( G = (V, E) \), where \( E \) is an \hyperref[def:binary_relation/irreflexive]{irreflexive} \hyperref[def:binary_relation]{binary relation} on \( V \).

    \thmitem{def:graph/mixed} A \term{mixed multigraph} is a tuple \( M \coloneqq (V, E, \mscrE, A, h, t) \), where \( (V, E, \mscrE) \) is an undirected multigraph and \( (V, A, h, t) \) is a directed multigraph.

    Mixed graphs are defined in an obvious way.

    \begin{figure}[h]
      \begin{equation}\label{eq:fig:def:graph/mixed}
        \begin{aligned}
          \includegraphics{figures/eq__fig__def__graph__mixed.pdf}
        \end{aligned}
      \end{equation}
      \caption{A mixed simple graph with three undirected edges.}\label{fig:def:graph/mixed}
    \end{figure}
  \end{thmenum}

  The following concepts are common for all graphs:
  \begin{thmenum}[resume=def:graph]
    \thmitem{def:graph/incidence} We say that the vertex \( v \) and the edge \( e \) are \term{incident} if the vertex \( v \) is an endpoint \( e \).

    \thmitem{def:graph/adjacent_vertices} We say that \( u \) and \( v \) are \term{adjacent vertices} if there exists an edge \( e \) such that both \( u \) and \( v \) are endpoints of \( e \).

    \thmitem{def:graph/adjacent_edges} We say that \( e_1 \) and \( e_2 \) are \term{adjacent edges} if they have a common endpoint.

    \thmitem{def:graph/order} The \term{order} \( \ord(G) \) of the graph \( G \) is the \hyperref[thm:cardinality_existence]{cardinality} of \( V \).

    We say that the graph is finite if both \( V \) and \( E \) are finite and infinite otherwise. If no parallel edges are allowed, the graph is finite if \( \ord(G) \) is finite.

    \thmitem{def:graph/degree} The \term{degree} \( \deg(v) \) of a vertex \( v \in V \) is the cardinality of set
    \begin{equation*}
      \set{ e \in E \given v \T{is an endpoint of} e }.
    \end{equation*}

    The degree \( \deg(G) \) of the graph itself is the maximum of the degrees of all vertices. It is possible that the maximum in \( \deg(G) \) is not attained if \( G \) is infinite.

    We say that the graph is \term{locally finite} if the degree of every vertex is finite.

    This is not to be confused with \hyperref[def:category_size]{locally finite categories}.
  \end{thmenum}
\end{definition}

\begin{remark}\label{rem:graph_endpoint_etymology}
  Arcs are introduced as ordered pairs, for example, in \cite[ch. 1, sec. 1.1]{GondranMinoux1984Graphs} and edges as unordered pairs a bit later in \cite[ch. 1, sec. 1.3]{GondranMinoux1984Graphs}. Other authors, e.g. \cite[sec. 5.2]{Erickson2019}, use the term \enquote{edge} to mean both an arc and an edge depending on whether the graph is directed or not.

  The terms \enquote{head} and \enquote{tail} are used, among other places, \cite[sec. 5.2]{Erickson2019}. The head and tail are also called the \term{starting}/\term{initial} endpoint and \term{final}/\term{terminal} endpoint in \cite[ch. 1, sec. 1.1]{GondranMinoux1984Graphs}.
\end{remark}

\begin{example}\label{ex:tower_diagram_graph}
  A very simple example of an infinite graph is the \hyperref[def:relation_closures/transitive]{transitive reduction} of the nonnegative integers, i.e. the simple directed graph
  \begin{equation}\label{eq:ex:tower_diagram_graph/forward}
    \begin{aligned}
      \includegraphics{figures/eq__ex__tower_diagram_graph__forward.pdf}
    \end{aligned}
  \end{equation}

  Since the graph is simple, we have
  \begin{equation*}
    \deg(n) = \begin{cases}
      1 = 0 + 1, &n = 0, \\
      2 = 1 + 1, &n > 0 \\
    \end{cases}
  \end{equation*}
  and thus \( \deg(G) = 2 \).

  We call the corresponding \hyperref[def:categorical_diagram]{categorical diagram} a \term{tower diagram} --- see \fullref{def:tower_diagram}.

  Another related graph is based on the nonpositive integers:
  \begin{equation}\label{eq:ex:tower_diagram_graph/backward}
    \begin{aligned}
      \includegraphics{figures/eq__ex__tower_diagram_graph__backward.pdf}
    \end{aligned}
  \end{equation}

  Finally, the union of the two gives us a two-sided tower:
  \begin{equation}\label{eq:ex:tower_diagram_graph/two_sided}
    \begin{aligned}
      \includegraphics{figures/eq__ex__tower_diagram_graph__two_sided.pdf}
    \end{aligned}
  \end{equation}

  All three graphs are infinite but \hyperref[def:graph/degree]{locally finite}.
\end{example}

\begin{remark}\label{rem:undirected_graphs_as_directed}
  For every \hyperref[def:graph/quiver]{quiver} \( Q \), we can define an \hyperref[def:graph/undirected_multigraph]{undirected multigraph} via \( \mscrE(e) \coloneqq \set{ h(e), t(e) } \). We call this the \term{symmetrization} \( \sym(Q) \) of the quiver. We lose information in the process, however.

  It is often convenient to do the opposite --- to represent an undirected multigraph as a quiver.

  We have discussed that the edge set \( E \) in a \hyperref[def:graph/simple_directed]{simple directed graphs} \( G = (V, E) \) is a binary relation. If this relation is \hyperref[def:binary_relation/symmetric]{symmetric}, then to every arc \( u \to v \) there corresponds an opposing arc \( v \to u \). If we consider \hyperref[def:graph/simple_directed]{undirected graphs} to be directed graphs whose arc relation is symmetric, this makes a lot of definitions hold transparently for undirected graphs.

  The symmetrization of the directed graph \( G = (V, E) \) is then the undirected graph \( \sym(G) \coloneqq (V, \cl^S(E)) \), whose edges relation is the \hyperref[def:relation_closures/symmetric]{symmetric closure} of \( E \).

  Representing \hyperref[def:graph/undirected_multigraph]{undirected multigraphs} and \hyperref[def:graph/quiver]{quivers} is more difficult --- we need to give an explicit bijective mapping from the \hyperref[eq:def:graph/quiver/arc_set]{arc set} \( Q(u, v) \) to \( Q(v, u) \) for every pair of vertices \( u \) and \( v \). Nevertheless, such a mapping often exists implicitly.
\end{remark}

\begin{definition}\label{def:theory_of_graphs}
  We have already defined directed and undirected (multi)graphs in \fullref{def:graph}. We discussed in \fullref{rem:undirected_graphs_as_directed} that it is convenient to regard undirected graphs as special directed graphs. We will now utilize this to list metamathematical properties of graphs.

  \begin{thmenum}
    \thmitem{def:theory_of_graphs/theory} The \hyperref[def:first_order_theory]{first-order theory} of simple directed graphs is a theory over the language consisting of a single infix binary relation \( \to \) containing \hyperref[def:binary_relation/irreflexive]{irreflexivity}. For simple undirected graphs, we must also add \hyperref[def:binary_relation/symmetric]{symmetry}.

    See \fullref{rem:well_founded_graphs} for how graphs are related to \hyperref[def:zfc]{Zermelo-Fraenkel set theory}.

    \thmitem{def:theory_of_graphs/simple_homomorphism} A \hyperref[def:first_order_homomorphism]{first-order homomorphism} between two simple directed graphs \( G_1 = (V_1, E_1) \) and \( G_2 = (V_2, E_2) \) is a function \( f: V_1 \to V_2 \) between their universes such that for every arc \( a \to b \) of \( G_1 \), \( f(a) \to f(b) \) is an arc of \( G_2 \).

    It is a weak homomorphism in the sense of \fullref{def:first_order_homomorphism}.

    Homomorphisms preserve symmetry, so this definition works without modification for undirected graphs.

    The term \enquote{graph embedding} usually refers to (topological) embeddings of the \hyperref[def:quiver_geometric_realization]{geometric realization} of a graph rather than (combinatorial) embeddings the graph.

    \thmitem{def:theory_of_graphs/quiver_homomorphism}\mcite[sec. II.7]{MacLane1994} Although we haven't defined \hyperref[def:graph/quiver]{quivers} via a \hyperref[def:first_order_theory]{first-order theory}, we have a canonical notion of \hyperref[def:first_order_homomorphism]{homomorphism}, although it is a pair of functions rather than a single function.

    Let \( M = (V_M, E_M) \) and \( N = (V_N, E_N) \) be quivers. A \term{homomorphism pair} from \( M \) to \( N \) is a pair of functions \( f_V: V_M \to V_N \) and \( f_E: E_M \to E_N \) such that
    \begin{align}
      h \bincirc f_e &= f_v \bincirc h \label{eq:def:theory_of_graphs/quiver_homomorphism/head} \\
      t \bincirc f_e &= f_v \bincirc t \label{eq:def:theory_of_graphs/quiver_homomorphism/tail}
    \end{align}

    \thmitem{def:theory_of_graphs/submodel} As for preordered sets, any subset \( V' \subseteq V \) of vertices of a quiver \( Q = (V, E, h, t) \) induces a quiver with the edges restricted to those with endpoints in \( V' \). That is, \( Q' = (V', E', h\restr_{V'}, t\restr_{V'}) \) is a \term{subquiver} of \( Q \) if \( V' \subseteq V \) and
    \begin{equation}\label{eq:def:theory_of_graphs/submodel}
      E' \subseteq \set{ e \in E \given h(e) \in V' \T{and} t(e) \in V' }.
    \end{equation}

    We will sometimes impose the more restrictive condition
    \begin{equation}\label{eq:def:theory_of_graphs/submodel/full}
      E' = \set{ e \in E \given h(e) \in V' \T{and} t(e) \in V' }.
    \end{equation}

    If a subgraph satisfies the more restrictive \eqref{eq:def:theory_of_graphs/submodel/full}, we will say that it is a \term{full subgraph} that it is \term{induced} by \( V' \). For simple directed graphs, these are precisely the \hyperref[def:first_order_substructure]{first-order substructures}.

    \thmitem{def:theory_of_graphs/trivial} Unlike the \hyperref[def:group/trivial]{trivial group} \( \set{ e } \) or \hyperref[def:partially_ordered_set/trivial]{empty ordered set}, which are unique up to an isomorphism, there is no single agreed upon graph called the \enquote{trivial graph}.

    An unambiguous concept is that of an \term{edgeless graph}, in which the set of arcs/edges is empty, but the set of vertices may or may not be empty. Every graph \( G \) has exactly \( 2^{\ord(G)} \) edgeless subquivers (one for each subset of \( V \)).

    For a simple directed graph, the bottom of the \hyperref[thm:substructures_form_complete_lattice]{lattice of submodels} is the \term{order-zero graph} \( (\varnothing, \varnothing) \). The order-zero graph is unique.

    The terms \term{empty graph}, \term{null graph} and \term{trivial graph} may refer to either edgeless graphs or the order-zero graph depending on the author and the situation.

    \thmitem{def:theory_of_graphs/category} Simple graphs have \hyperref[def:category_of_first_order_models]{categories of first-order models}, however these will not be interesting to us.

    Quivers are not defined via first-order logic and thus require some more elaborate definitions. Given a \hyperref[def:grothendieck_universe]{Grothendieck universe} \( \mscrU \), we denote the \hyperref[def:category]{category} of \hyperref[def:large_and_small_sets]{\( \mscrU \)-small} quivers by \( \mscrU-\cat{Quiv} \) or simply \( \cat{Quiv} \).
    \begin{itemize}
      \item The \hyperref[def:category/C1]{set of objects} \( \obj(\cat{Quiv}) \) is the set of all \( \mscrU \)-small quivers.
      \item The \hyperref[def:category/C2]{set of morphisms} \( \cat{Quiv}(M, N) \) is the set of all \hyperref[def:theory_of_graphs/quiver_homomorphism]{homomorphism pairs} from \( M \) to \( N \).

      \item The \hyperref[def:category/C3]{composition of the morphisms} \( (f_V, f_E): M \to N \) and \( (g_V, g_E): N \to K \) is the morphism \( (g_V \bincirc f_V, g_E \bincirc f_E): M \to K \).
    \end{itemize}
  \end{thmenum}
\end{definition}

\begin{remark}\label{rem:graphs_linear_algebra_and_topology}
  As we shall see, the \hyperref[def:graph/adjacent_vertices]{adjacency} and \hyperref[def:graph/incidence]{incidence} of a graph can be easily studied using linear algebra via \hyperref[def:graph_adjacency_matrix]{adjacency} and \hyperref[def:multigraph_incidence_matrix]{incidence matrices}, while the \hyperref[def:quiver_connectedness]{connectedness} of a graph can be studied using \hyperref[def:quiver_connectedness]{topology} via \hyperref[def:quiver_geometric_relization/embedding]{graph embeddings}.
\end{remark}

\begin{definition}\label{def:multigraph_incidence_matrix}\mcite[ch. 1, sec. 2.1]{GondranMinoux1984Graphs}
  Let \( M \) be a finite multigraph without loops, either \hyperref[def:graph/quiver]{directed} or \hyperref[def:graph/undirected_multigraph]{undirected}. We define the \term{incidence matrix} \( I = \seq{ a_{ve} }_{v \in V, e \in E} \) for \( M \).

  \smallskip
  \begin{minipage}[t]{0.45\textwidth}
    If \( M \) is directed,
    \begin{equation*}
      a_{ve} \coloneqq \begin{cases}
        1,  & v = h(e) \\
        -1, & v = t(e) \\
        0,  & \T{otherwise.}
      \end{cases}
    \end{equation*}
  \end{minipage}
  \hspace{0.02\textwidth}
  \begin{minipage}[t]{0.45\textwidth}
    If \( M \) is undirected,
    \begin{equation*}
      a_{ve} \coloneqq \begin{cases}
        1,  & v \T{is an endpoint of} e \\
        0,  & \T{otherwise.}
      \end{cases}
    \end{equation*}
  \end{minipage}
  \smallskip

  The two definitions are quite different but in both cases, the matrix element \( a_{ve} \) is nonzero precisely when \( v \) is an endpoint of \( e \).
\end{definition}

\begin{definition}\label{def:graph_adjacency_matrix}\mcite[ch. 1, sec. 2.3]{GondranMinoux1984Graphs}
  Let \( G = (V, E) \) be a simple \hyperref[def:graph/simple_directed]{directed} or \hyperref[def:graph/simple_directed]{undirected} graph. The \term{adjacency matrix} \( I = \seq{ a_{uv} }_{u, v \in V} \) for \( G \) has elements
  \begin{equation*}
    a_{uv} \coloneqq \begin{cases}
      1,  &(u, v) \in E \\
      0,  &\T{otherwise.}
    \end{cases}
  \end{equation*}

  Unlike the incidence matrix, the adjacency matrix regards undirected graphs as special cases of directed graphs. See also \fullref{thm:graph_undirected_iff_adjacency_matrix_is_symmetric}.
\end{definition}

\begin{example}\label{ex:graph_matrices}
  The \hyperref[def:multigraph_incidence_matrix]{incidence matrix} of the simple directed graph \eqref{eq:fig:def:graph/simple_directed} is
  \begin{equation}\label{ex:def:multigraph_incidence_matrix}
    \begin{blockarray}{cccccccc}
        & e_1       & e_2       & e_3       & e_4       & e_5       & e_6       & e_7       \\
      \begin{block}{c(ccccccc)}
      a & 1         & 1         &           &           &           &           &           \\
      b & \fbox{-1} &           & 1         &           &           &           &           \\
      c &           & \fbox{-1} &           & 1         & 1         &           &           \\
      d &           &           & \fbox{-1} & \fbox{-1} &           & 1         &           \\
      e &           &           &           &           & \fbox{-1} &           & 1         \\
      f &           &           &           &           &           & \fbox{-1} & \fbox{-1} \\
      \end{block}
    \end{blockarray}
  \end{equation}

  It can be read column-by-column. Every column contains exactly two nonzero elements whose rows correspond to the head (positive) and tail (negative).

  To obtain the incidence matrix for the \hyperref[rem:undirected_graphs_as_directed]{symmetrization} \eqref{eq:fig:def:graph/simple_undirected} of \eqref{eq:fig:def:graph/simple_directed}, we need to simply flip the sign of the boxed elements above.

  The \hyperref[def:graph_adjacency_matrix]{adjacency matrix} is
  \begin{equation}\label{ex:def:graph_adjacency_matrix}
    \begin{blockarray}{cccccccc}
        & a        & b        & c        & d        & e        & f \\
    \begin{block}{c(ccccccc)}
      a &          & 1        & 1        &          &          &   \\
      b & \fbox{1} &          &          & 1        &          &   \\
      c & \fbox{1} &          &          & 1        & 1        &   \\
      d &          & \fbox{1} & \fbox{1} &          &          & 1 \\
      e &          &          & \fbox{1} &          &          & 1 \\
      f &          &          &          & \fbox{1} & \fbox{1} &   \\
    \end{block}
    \end{blockarray}
  \end{equation}
  where the boxed elements are nonzero only in the adjacency matrix for the undirected graph.

  The matrix can be read either column-by-column or row-by-row.
  \begin{itemize}
    \item The \( v \)-th column lists the vertices \( u \) such that there is an edge \( u \to v \).
    \item The \( u \)-th row lists the vertices \( v \) such that there is an edge \( u \to v \).
  \end{itemize}
\end{example}

\begin{proposition}\label{thm:graph_undirected_iff_adjacency_matrix_is_symmetric}
  A finite simple \hyperref[def:graph]{graph} is \hyperref[def:graph/simple_undirected]{undirected} if and only if its \hyperref[def:graph_adjacency_matrix]{adjacency matrix} is \hyperref[def:symmetric_matrix]{symmetric}.
\end{proposition}
\begin{proof}
  Trivial.
\end{proof}

\begin{definition}\label{def:multigraph_vector_spaces}
  Let \( G \) be any \hyperref[def:graph]{graph}. We introduce several \hyperref[def:vector_space]{vector spaces} over \( G \) that allow us to study graphs using linear algebra.

  \begin{thmenum}
    \thmitem{def:multigraph_vector_spaces/vertex} The \term{vertex space} \( \BbbF_2^V \) is the \hyperref[def:left_module_of_tuples]{tuple vector space} of dimension \( \card(V) = \ord(G) \) over the finite \hyperref[def:field]{field} \hyperref[thm:f2_is_boolean_algebra]{\( \BbbF_2 \)}.

    Every subset \( A \subseteq V \) of vertices induces a unique vector \( \vect{A} = \seq{ \vect{A}_v }_{v \in V} \) in the \hyperref[def:multigraph_vector_spaces/vertex]{vertex space} \( \BbbF_2^V \) such that
    \begin{equation*}
      \vect{A}_v \coloneqq \begin{cases}
        1, &v \in A \\
        0, &v \not\in A
      \end{cases}
    \end{equation*}

    This vector is called the \term{characteristic vector} of \( A \). Conversely, every vector in \( \BbbF_2^V \) induces a set of vertices. If \( A \) consists of a single vertex \( u \), we write \( \vect{u} \) rather than \( \vect{\set{u}} \).

    It is important to note that, in accordance with \fullref{def:function/set_of_functions}, \( \BbbF_2^V \) is the set of all functions from \( V \) to \( \BbbF_2 \). If the graph is finite, this space is isomorphic to the tuple space \( \BbbF_2^{\ord(G)} \). Unfortunately, this isomorphism is not unique since it does not give us a choice of ordering of \( V \). Thus, even for finite graphs, we cannot regard the vectors of \( \BbbF_2^V \) as ordered tuples unless \( V \) is \hyperref[def:well_ordered_set]{well-ordered}.

    \thmitem{def:multigraph_vector_spaces/arc} Analogously, the \term{arc space} \( \BbbF_2^E \) is the \hyperref[def:left_module_of_tuples]{tuple vector space} of dimension \( \card(E) \). Every subset of \( E \) induces a unique characteristic vector in the \hyperref[def:multigraph_vector_spaces/arc]{arc space} \( \BbbF_2^E \) and vice versa. The space is motivated by \hyperref[def:quiver_directed_path]{directed paths} and their \hyperref[def:quiver_directed_path/characteristic_vector]{characteristic vectors}.

    When dealing with undirected graphs, we call \( \BbbF_2^E \) the \term{edge space}.

    \thmitem{def:multigraph_vector_spaces/oriented_arc} Finally, we introduce the more complicated \term{oriented arc space} \( \BbbF_3^E \). It is sometimes not sufficient to consider only a subset \( A \subseteq E \) but rather a pair of subsets \( A, B \subseteq E \) such that the arcs in \( A \) are \enquote{positively oriented} and \( B \) are \enquote{negatively oriented}. The terms \enquote{positively oriented} and \enquote{negatively oriented} are informal only have meaning in certain applications. The space is motivated by \hyperref[def:quiver_adjacency_chain]{adjacency chains} and their \hyperref[def:quiver_adjacency_chain/characteristic_vector]{characteristic vectors}.

    The characteristic vector \( x \) of a pair \( (A, B) \) of subsets has components
    \begin{equation*}
      \vect{(A, B)}_e \coloneqq \begin{cases}
        1,  &e \in A \\
        -1, &e \in B \\
        0,  &\T{otherwise.}
      \end{cases}
    \end{equation*}

    As for the other vector spaces, every pair \( (A, B) \) has a characteristic vector in \( \BbbF_3^E \) and every vector in \( \BbbF_3^E \) corresponds to a pair of subsets of \( E \).
  \end{thmenum}
\end{definition}

\begin{proposition}\label{thm:graphs_as_linear_transformations}
  Let \( V \) be a \hyperref[def:set_finiteness]{finite set}. Then there is a bijection between the \hyperref[def:linear_operator]{linear} \hyperref[def:endomorphism]{endomorphisms} over \( \BbbF_2 \) and the \hyperref[def:graph/simple_directed]{simple directed graphs} over \( V \) with loops allowed.
\end{proposition}
\begin{proof}
  Let \( G = (V, E) \) be a directed graph and let \( I \) be its \hyperref[def:graph_adjacency_matrix]{adjacency matrix}. Then \( I \) induces a linear endomorphism in \( \BbbF_2^n \).

  Conversely, let \( T: \BbbF_2^n \to \BbbF_2^n \) be a linear endomorphism. Define the set
  \begin{equation*}
    E \coloneqq \set{ (u, v) \in V^2 \given T(\vect{v})_u = 1 }.
  \end{equation*}

  Then the adjacency matrix of \( G = (V, E) \) induces \( T \).
\end{proof}

\begin{example}\label{ex:thm:graphs_as_linear_transformations}
  We will empirically verify the proof of \fullref{thm:graphs_as_linear_transformations}.

  Denote by \( A \) the adjacency matrix \eqref{ex:def:graph_adjacency_matrix} of \eqref{eq:fig:def:graph/simple_directed} discussed in \fullref{ex:graph_matrices}. Consider the vertex \( b \). Its characteristic vector is \( \vect{b} = (0, 1, 0, 0, 0, 0) \). Thus, the matrix product \( A \vect{b} \) simply \enquote{selects} the \( b \)-th column of \( I \), which is
  \begin{equation*}
    A \vect{b}
    =
    \begin{blockarray}{cccccrl}
      a        & b        & c        & d        & e        & f \\
    \begin{block}{(cccccr)l}
      1        & 0        & 0        & 0        & 0        & 0 \\
    \end{block}
    \end{blockarray}
    {}^T
  \end{equation*}

  The only nonzero member of \( A \vect{b} \) is the one corresponding to the vertex \( a \). This is consistent with \( a \) being the only predecessor of \( b \) in \eqref{eq:fig:def:graph/simple_directed}.
\end{example}

\begin{definition}\label{def:quiver_adjacency_chain}
  An \term{adjacency chain} or \term{undirected path} in a \hyperref[def:graph/quiver]{quiver} \( Q \) is a finite or infinite sequence of arcs
  \begin{equation}\label{eq:def:quiver_adjacency_chain}
    p \coloneqq (e_1, e_2, \ldots)
  \end{equation}
  such that every two consecutive arcs are \hyperref[def:graph/adjacency_edges]{adjacent}. That is, \( e_{k+1} \) is adjacent to \( e_k \) for every \( k = 1, 2, \ldots \).

  Note that every two arcs of \( p \) should only be adjacent, hence \( (u \to v, u \to w) \) is an adjacency chain. See \fullref{def:quiver_directed_path} for the more intuitive notion of a path.

  This is a concept that does not make sense for undirected multigraphs, for which paths are the proper notion.

  The term \enquote{chain} is used in this sense in \cite[ch. 1, sec. 3.1]{GondranMinoux1984Graphs}, for example, but in general it may refer to a path or simple path. We add the adverb \enquote{adjacency} in order to reduce ambiguity.

  \begin{thmenum}
    \thmitem{def:quiver_adjacency_chain/domain} The \term{domain} \( \dom(p) \) of \( p \) is the set of all vertices that belong to at least one arc in \( p \). We say that the path \term{visits} each member of \( \dom(p) \).

    \thmitem{def:quiver_adjacency_chain/length} The \term{length} of \( p \) is defined in an obvious way for finite paths and as \( p = \infty \) for infinite paths (here \( \infty \) is merely a more conventional symbol for denoting \hyperref[thm:omega_is_a_cardinal]{\( \aleph_0 \)}).

    Note that
    \begin{equation}\label{eq:def:quiver_adjacency_chain/length_and_domain}
      \card(\dom(p)) + 1 \leq \len(p)
    \end{equation}
    in general since a vertex can be an endpoint of many arcs.

    \thmitem{def:quiver_adjacency_chain/subchain} We say that the \hyperref[def:subsequence]{subsequence} \( q \) of \( p \) is a \term{subchain} of \( p \) if the elements of \( q \) are consecutive in \( p \), i.e. there exists some index \( n \geq 0 \) such that \( q_k = p_{n + k} \) for \( 0 < k < \len(q) \).

    \thmitem{def:quiver_adjacency_chain/endpoints} Similarly to arcs, chains have a \term{head} and \term{tail}, although both may not exist. Roughly, the head of \( p = (e_1, e_2, \ldots) \) is the endpoint of \( e_1 \) that is not an endpoint of \( e_2 \). After considering some edge cases, this definition becomes
    \begin{equation}\label{eq:def:quiver_adjacency_chain/endpoints/head}
      h(p) \coloneqq \begin{cases}
        \T{undefined}, &p = \varnothing, \\
        h(e_1),        &p = (e_1), \\
        h(e_1),        &p = e_2 \T{is opposing} e_1, \\
        v,             &v \T{is an endpoint of} e_1 \T{but not of} e_2.
      \end{cases}
    \end{equation}

    Compare this to the much simpler \eqref{eq:def:quiver_directed_path/endpoints/head}.

    The \term{tail} is defined similarly but with the caveat that an infinite chain cannot have a tail.

    The head and the tail are collectively called the \term{endpoints} of a path.

    \thmitem{def:quiver_adjacency_chain/simple} The chain \( p \) is \term{simple} if every vertex in \( \dom(p) \) is visited only once. That is, within the subquiver whose arcs are those in \( p \), the \hyperref[def:graph/degree]{degree} of every vertex is at most \( 2 \).

    An alternative characterization is that equality holds in \eqref{eq:def:quiver_adjacency_chain/length_and_domain}.

    \thmitem{def:quiver_adjacency_chain/converse} If the chain \eqref{eq:def:quiver_adjacency_chain} is finite of length \( n \), we define its \term{converse} as
    \begin{equation*}
      p^{-1} \coloneqq (e_n, e_{n-1}, \cdots, e_2, e_1).
    \end{equation*}

    \thmitem{def:quiver_adjacency_chain/orientation} We say that an arc is \term{positively oriented} if either it is the first arc \( e_1 \) in case \( h(e_1) = h(p) \) or it is the \hyperref[def:graph/adjacency]{successor} of a positively oriented arc.

    More precisely, we can recursively define the \hyperref[def:boolean_function]{predicate} \( \op{IsPos} \) determining whether the arc at position \( k \) is \term{positively oriented}:
    \begin{equation*}
      \begin{aligned}
        \op{IsPos}(k) \coloneqq \begin{cases}
          T,                            &k = 1 \T{and} h(e_1) = h(p) \\
          F,                            &k = 1 \T{and} h(e_1) \neq h(p) \\
          \op{IsPos}(k - 1),            &k > 1 \T{and} h(e_k) = t(e_k) \\
          \overline{\op{IsPos}(k - 1)}, &k > 1 \T{and} h(e_k) \neq t(e_k)
        \end{cases}
      \end{aligned}
    \end{equation*}

    All arcs that are not positively oriented are said to be \term{negatively oriented}.

    \thmitem{def:quiver_adjacency_chain/characteristic_vector} The \term{characteristic vector} \( \vect{p} \) of a chain \( p \) is defined as the characteristic vector \( \vect{(P, N)} \) in the \hyperref[def:multigraph_vector_spaces/oriented_arc]{oriented arc space} \( \BbbF_3^E \), where \( P \) is the set of positively oriented arcs and \( N \) is the set of negatively oriented arcs.

    Conversely, every vector in \( \BbbF_3^E \) identifies a path.
  \end{thmenum}
\end{definition}

\begin{definition}\label{def:quiver_directed_path}
  A \term{directed path} or simply \term{path} in a \hyperref[def:graph/quiver]{quiver} \( Q \) is an \hyperref[def:quiver_adjacency_chain]{adjacency chain} without negatively oriented arcs. In other words, a directed path is a finite or infinite sequence \eqref{eq:def:quiver_adjacency_chain} such that each arc is a \hyperref[def:graph/adjacency]{successor} of the preceding one. That is, the head \( e_{k+1} \) is the tail of \( e_k \) for every \( k = 1, 2, \ldots \).

  We regard undirected multigraphs as a special case of quivers, however the definition of a path requires some adjustments. For undirected multigraphs, we put the additional restriction that no two consecutive arcs in \( Q \) should be opposing. This is very important when considering \hyperref[def:quiver_cycle]{cycles}.

  The definitions of \hyperref[def:quiver_adjacency_chain/domain]{domain} \( \dom(p) \), \hyperref[def:quiver_adjacency_chain/length]{length} and \( \len(p) \) are inherited from chains. \hyperref[def:quiver_adjacency_chain/subchain]{Subchains} of a path are called \term{subpaths} and a path is called \term{simple} if it is a \hyperref[def:quiver_adjacency_chain/simple]{simple chain}.

  This terminology is used in \mcite[ch. 1, sec. 3.2]{GondranMinoux1984Graphs}, however in \cite[sec. 5.2]{Erickson2019} this is instead called a \term{walk} and a path is defined as a simple walk (\enquote{simple} in the sense of \fullref{def:quiver_adjacency_chain/simple}). The terms \term{trail} and \term{tour} also have related meaning, but it is unfortunately ambiguous, depending on the context. We sometimes use \enquote{directed path} in order to reduce ambiguity.

  \begin{thmenum}
    \thmitem{def:quiver_directed_path/endpoints} The definition of head and tail of a directed path inherits those of adjacency chains, however they are so much simpler that we give them explicitly. Instead of \eqref{eq:def:quiver_adjacency_chain/endpoints/head} we have
    \begin{equation}\label{eq:def:quiver_directed_path/endpoints/head}
      h(p) \coloneqq \begin{cases}
        \T{undefined}, &p = \varnothing, \\
        h(e_1),        &\T{otherwise}.
      \end{cases}
    \end{equation}

    If \( Q \) is undirected, we instead use
    \begin{equation}\label{eq:def:quiver_directed_path/endpoints/head_undirected}
      h(p) \coloneqq \begin{cases}
        \T{undefined}, &\len(p) < 2, \\
        v,             &v \T{is an endpoint of} e_1 \T{but not of} e_2.
      \end{cases}
    \end{equation}

    Tails are defined analogously.

    \thmitem{def:quiver_directed_path/characteristic_vector} The \term{characteristic vector} \( \vect{p} \) of a path \( p \) is defined as the characteristic vector of the domain \( \dom(p) \) in the \hyperref[def:multigraph_vector_spaces/arc]{arc space} \( \BbbF_2^E \). Conversely, every vector in \( \BbbF_2^E \) identifies a path.

    The vector \( \vect{p} \) is equal to the \hyperref[def:quiver_adjacency_chain/characteristic_vector]{characteristic vector} of \( p \) when regarding \( p \) as a chain. It should be noted, however, that we regard \( \vect{p} \) as a member of the arc space \( \BbbF_2^E \) and rather than the oriented arc space \( \BbbF_3^E \).

    \thmitem{def:quiver_directed_path/reachability} We say that \( p \) \term{connects} or \term{passes through} the vertices \( u \) and \( v \) if there exists a subpath \( q \) such that \( h(q) = u \) and \( t(p) = v \).

    We say that the vertex \( u \) is \term{reachable} from \( v \) if there exists a path that connects them.

    Reachability is a reflexive and transitive relation. For simple directed graphs, for which the arc set \( E \) can be regarded as a relation, reachability is the \hyperref[def:relation_closures/reflexive]{reflexive} and \hyperref[def:relation_closures/transitive]{transitive closure} of the relation \( E \). Furthermore, if the simple graph is undirected, \( E \) is symmetric and, by \fullref{thm:relation_closures_properties/symmetric_relation}, reachability is also symmetric. Thus, in an undirected graph, reachability is an equivalence relation. This is exploited in \fullref{def:quiver_connectedness}.

    This definition is generalizable to arbitrary chains, however we prefer not to do it.
  \end{thmenum}
\end{definition}

\begin{example}\label{ex:def:quiver_directed_path}
  An example of an infinite graph path is the forward tower \eqref{eq:ex:tower_diagram_graph/forward} regarded as a path of the two-sided tower \eqref{eq:ex:tower_diagram_graph/two_sided}. It is also simple since the degree of the forward tower is \( 2 \).

  Now consider again the graph \eqref{eq:fig:def:graph/simple_directed}. The solid lines in \eqref{eq:ex:def:quiver_directed_path} describe a path.
  \begin{equation}\label{eq:ex:def:quiver_directed_path}
    \begin{aligned}
      \includegraphics{figures/eq__ex__def__graph_directed_path.pdf}
    \end{aligned}
  \end{equation}

  \begin{itemize}
    \item This path corresponds to the sequence
    \begin{equation*}
      p = \parens{ \underbrace{a \to c}_{e_2}, \underbrace{c \to d}_{e_4}, \underbrace{d \to f}_{e_6} }.
    \end{equation*}

    \item Its \hyperref[def:quiver_directed_path/characteristic_vector]{characteristic vector} is
    \begin{equation*}
      \vect{p}
      =
      \begin{blockarray}{ccccccr}
        e_1 & e_2 & e_3 & e_4 & e_5 & e_6 & e_7 \\
      \begin{block}{(ccccccr)}
        0   & 1   & 0   & 1   & 0   & 1   & 0   \\
      \end{block}
      \end{blockarray}
      {}^T.
    \end{equation*}

    Furthermore, \( p \) can be identified from the characteristic vector.

    \item The \hyperref[def:quiver_directed_path/endpoints]{endpoints} are \( h(p) = h(e_1) = a \) and \( t(p) = t(e_7) = f \).

    \item It is a \hyperref[def:quiver_adjacency_chain/simple]{simple path} because \( \deg(a) = \deg(f) = 1 \) and \( \deg(c) = \deg(d) = 2 \) in the induced subgraph.

    \item The \hyperref[def:quiver_adjacency_chain/converse]{converse chain}
    \begin{equation*}
      p^{-1} = (d \to f, c \to d, a \to c),
    \end{equation*}
    is not a path since every arc is \hyperref[def:quiver_adjacency_chain/orientation]{negatively oriented}.

    \item We can conclude from \eqref{eq:def:quiver_adjacency_chain/endpoints/head} that \( h(p^{-1}) = f \) and \( t(p^{-1}) = a \).

    \item Its \hyperref[def:quiver_directed_path/characteristic_vector]{characteristic vector} is
    \begin{equation*}
      \vect{p^{-1}}
      =
      \begin{blockarray}{ccccccrl}
        e_1 & e_2 & e_3 & e_4 & e_5 & e_6 & e_7 \\
      \begin{block}{(ccccccr)l}
        0   & -1  & 0   & -1  & 0   & -1  & 0   \\
      \end{block}
      \end{blockarray}
      {}^T.
    \end{equation*}

    Furthermore, \( p^{-1} \) can be identified from the characteristic vector.
  \end{itemize}
\end{example}

\begin{proposition}\label{thm:quiver_chain_symmetrization}
  A sequence of arcs in a quiver \( Q \) is an \hyperref[def:quiver_adjacency_chain]{adjacency chain} if and only if it is a path in the \hyperref[rem:undirected_graphs_as_directed]{symmetrization} \( \sym(Q) \).
\end{proposition}
\begin{proof}
  Two arcs are adjacent if and only if one of they have a common endpoint. The endpoints of every arc in \( Q \) are the same as in the corresponding edge in \( \sym(Q) \). The only difference is in the order of the endpoints, which becomes irrelevant when only adjacency is considered.
\end{proof}

\begin{definition}\label{def:quiver_cycle}
  A \term{cycle} in a quiver is a \hyperref[def:quiver_directed_path]{path} whose head and tail coincide. It is also called a \term{closed path}.

  This definition makes sense for undirected multigraphs since we have assumed that a path does not contain consecutive opposing arcs. This additional restriction is necessary because otherwise, \( (u \to v, v \to u) \) would be a cycle.

  For quivers (but not undirected multigraphs), this is also called a \term{circuit} in \cite[ch. 1, sec. 1.4]{GondranMinoux1984Graphs},

  If the graph does not contain a cycle, it is \term{acyclic}. Directed acyclic graphs are commonly abbreviated as \term{DAG}.
\end{definition}

\begin{definition}\label{def:quiver_condensation}
  Let \( Q \) be a \hyperref[def:graph/quiver]{quiver}. Define the equivalence relation
  \begin{equation*}
    u \sim v \iff u \T{is reachable from} v \T{and} v \T{is reachable from} u.
  \end{equation*}

  Now define the binary relation \( \widetilde{E} \) on the \hyperref[def:equivalence_relation/quotient]{quotient set} \( \widetilde{V} \coloneqq V / {\sim} \) as
  \begin{equation*}
    ([u], [v]) \in \widetilde{E} \iff u \T{is reachable from} v \T{but not vice versa}.
  \end{equation*}

  It is well-defined because if \( ([u_1], [v_1]) \in \widetilde{E} \), \( u_2 \in [u_1] \) and \( v_2 \in [v_1] \), then by the transitivity of reachability we have that \( v_2 \) is reachable from \( u_2 \) and thus \( ([u_2], [v_2]) \in \widetilde{E} \).

  As discussed in \fullref{def:theory_of_graphs}, irreflexive relations over sets can be regarded as simple directed graphs. Therefore, the pair \( \widetilde{G} \coloneqq (\widetilde{V}, \widetilde{E}) \) is a directed graph. It is called the \term{condensation} of the quiver \( Q \).
\end{definition}

\begin{example}\label{ex:def:quiver_condensation}
  The \hyperref[def:quiver_condensation]{condensation} of the graph \eqref{eq:fig:def:graph/simple_directed} is (isomorphic to) the graph itself. If we add the arc \( f \to a \) to \eqref{eq:fig:def:graph/simple_directed}, the condensation would be an edgeless graph with a single vertex.

  We can add new arcs \( e_8 \) and \( e_9 \) to \eqref{eq:fig:def:graph/simple_directed} to make the example more interesting:
  \begin{equation}\label{eq:ex:def:quiver_condensation/uncondensed}
    \begin{aligned}
      \includegraphics{figures/eq__ex__def__graph_condensation__uncondensed.pdf}
    \end{aligned}
  \end{equation}

  Its condensation is
  \begin{equation}\label{eq:ex:def:quiver_condensation/condensed}
    \begin{aligned}
      \includegraphics{figures/eq__ex__def__graph_condensation__condensed.pdf}
    \end{aligned}
  \end{equation}

  These are the \hyperref[def:quiver_connectedness/strong]{strongly connected components} of \eqref{eq:ex:def:quiver_condensation/uncondensed}.
\end{example}

\begin{proposition}\label{thm:graph_condensation_is_acyclic_dag}
  The \hyperref[def:quiver_condensation]{condensation} of a quiver is a \hyperref[def:quiver_cycle]{directed acyclic graph}.
\end{proposition}
\begin{proof}
  Let \( \widetilde{G} = (\widetilde{V}, \widetilde{E}) \) be the condensation of the quiver \( Q = (V, E, h, t) \). We have already discussed in \fullref{def:quiver_condensation} that it is a directed graph. It remains to show that it is acyclic.

  Aiming at a contradiction, suppose that there exist cosets \( [u] \) and \( [v] \) and a path
  \begin{equation*}
    \parens[\Big]{ [u] \to [w_1], \cdots, [w_n] \to [v] }
  \end{equation*}
  in \( \widetilde{G} \) that connects them. We can easily prove by induction that \( v \) is reachable from \( u \), thus contradicting the definition of \( \widetilde{E} \).

  Therefore, \( \widetilde{G} \) is acyclic.
\end{proof}

\begin{proposition}\label{thm:undirected_graph_condensation}
  The \hyperref[def:quiver_condensation]{condensation} of an undirected multigraph is \hyperref[def:theory_of_graphs/trivial]{edgeless}.
\end{proposition}
\begin{proof}
  As mentioned in \fullref{def:quiver_directed_path/reachability}, reachability is a reflexive and transitive relation on the set \( V \) of vertices. It is also symmetric because the graph is undirected, hence it is an equivalence relation. It is actually equal to the relation \( {\sim} \) in \fullref{def:quiver_condensation}.

  Therefore, by definition of \( \widetilde{E} \), it is empty.
\end{proof}

\begin{definition}\label{def:quiver_connectedness}
  Let \( Q = (V, E, h, t) \) be a quiver and let \( \widetilde{G} = (\widetilde{V}, \widetilde{E}) \) be its condensed graph.

  \begin{thmenum}
    \thmitem{def:quiver_connectedness/strong}\mcite[ch. 1, sec. 3.5]{GondranMinoux1984Graphs} For every coset \( [v] \) in \( \widetilde{V} \), the \hyperref[def:theory_of_graphs/submodels]{subquiver} of \( Q \) induced by the vertices \( [v] \) is called a \term{strongly connected component} of \( Q \).

    The \term{strong connectivity number} of \( Q \) is the cardinality \( \card(\widetilde{V}) \).

    If \( Q \) has only one strongly connected component, we say that it itself is \term{strongly connected}.

    \thmitem{def:quiver_connectedness/weak}\mcite[ch. 1, sec. 3.3]{GondranMinoux1984Graphs} Similarly, the subquiver \( Q' \) of \( Q \) is called a \term{weakly connected component} if the \hyperref[rem:undirected_graphs_as_directed]{symmetrization} \( \sym(Q') \) is a strongly connected component of \( \sym(Q) \).

    The \term{weak connectivity number} of \( Q \) is the strong connectivity number of \( \sym(Q) \).

    If \( Q \) has only one weakly connected component, we say that it itself is \term{weakly connected}.
  \end{thmenum}

  Clearly the two concepts are equivalent for undirected graphs, in which case we simply say \enquote{connected component} and \enquote{connectivity number}.
\end{definition}

\begin{proposition}\label{thm:quiver_connectedness_via_chains}
  Let \( Q \) be a \hyperref[def:graph/quiver]{quiver}.

  \begin{thmenum}
    \thmitem{thm:quiver_connectedness_via_chains/strong} \( Q \) is \hyperref[def:quiver_connectedness/strong]{strongly connected} if and only if there exists a \hyperref[def:quiver_directed_path]{directed path} connecting every pair of vertices.

    \thmitem{thm:quiver_connectedness_via_chains/weak} \( Q \) is \hyperref[def:quiver_connectedness/weak]{weakly connected} if and only if there exists an \hyperref[def:quiver_adjacency_chain]{adjacency chain} connecting every pair of vertices.
  \end{thmenum}
\end{proposition}
\begin{proof}
  Follows from \fullref{thm:quiver_chain_symmetrization}.
\end{proof}

\begin{remark}\label{rem:well_founded_graphs}
  We can regard any set \( A \) in the sense of \hyperref[def:zfc]{\logic{ZFC}} as the simple directed graph \( (A, \in) \).

  The \hyperref[def:zfc/foundation]{axiom of foundation} (via \fullref{thm:set_membership_is_well_founded}) implies that the relation \( \in \) is well-founded.

  In the terminology if graph theory, this well-foundedness means that, for every vertex \( v \), there exists no infinite \hyperref[def:quiver_directed_path]{path} that ends with \( v \).

  This implies that the graph is \hyperref[def:quiver_cycle]{acyclic}. For finite graphs the converse holds, but an infinite graph this is not so. For example, the backward tower \eqref{eq:ex:tower_diagram_graph/backward} is acyclic but not well-founded.
\end{remark}

\begin{definition}\label{def:quiver_geometric_realization}
  Let \( Q \) be a \hyperref[def:graph/quiver]{quiver}. Our goal is to construct a \hyperref[def:topological_space]{topological space} that translates the connectivity properties of \( Q \) into their topological equivalents. We do this by taking a copy of the interval \( [0, 1] \) for every arc and gluing endpoints where the edges have a common endpoint.

  Consider the \hyperref[def:topological_sum]{topological sum} \( \coprod_{e \in E} [0, 1] \). Define the \hyperref[def:equivalence_relation]{equivalence relation} \( {\sim} \) on this space to hold for \( (x, e_1) \) and \( (y, e_2) \) if any of the following hold:
  \begin{align*}
    &x = 0 \T{and} y = 0 \T{and} h(e_1) = h(e_2) \\
    &x = 0 \T{and} y = 1 \T{and} h(e_1) = t(e_2) \\
    &x = 1 \T{and} y = 0 \T{and} t(e_1) = h(e_2) \\
    &x = 0 \T{and} y = 1 \T{and} t(e_1) = t(e_2) \\
    &e_1 \T{is oposing} e_2 \T{and} x = 1 - y.
  \end{align*}

  For a general quiver, the last condition requires every arc to have a unique opposite as discussed in \fullref{rem:undirected_graphs_as_directed}.

  The \hyperref[def:topological_quotient]{quotient space} \( \coprod_{e \in E} [0, 1] / {\sim} \) is called the \term{geometric realization} of \( G \).

  \begin{thmenum}
    \thmitem{def:quiver_geometric_realization/drawing} We will call any \hyperref[def:global_continuity]{continuous function} with domain \( \coprod_{e \in E} [0, 1] / {\sim} \) a \term{quiver drawing}. There is no standardized terminology for non-injective continuous images of the realization.

    \thmitem{def:quiver_geometric_realization/embedding} An injective quiver drawing is called a \term{quiver embedding}.

    \thmitem{def:quiver_geometric_realization/linear} If a quiver can be embedded into \( \BbbR \), we say that it is \term{linear}.

    \thmitem{def:quiver_geometric_realization/planar} If a quiver can be embedded into \( \BbbR^2 \), we say that it is \term{planar}.
  \end{thmenum}
\end{definition}

\begin{example}\label{ex:def:quiver_geometric_realization}
  We will give a few examples of \hyperref[def:quiver_geometric_realization]{quiver realizations}.

  \begin{thmenum}
    \thmitem{ex:def:quiver_geometric_realization/edgeless} The \hyperref[def:quiver_geometric_realization]{geometric realization} of an edgeless quiver is the empty topological space.

    \thmitem{ex:def:quiver_geometric_realization/tower} Consider the tower \eqref{eq:ex:tower_diagram_graph/forward}. We start with \( \aleph_0 \) copies of \( [0, 1] \) and glue both ends of each of them except for the first. Thus, we obtain (a space homeomorphic to)
    \begin{equation*}
      \bigcup_{k \geq 0} [k, k + 1] = [0, \infty).
    \end{equation*}

    Therefore, \eqref{eq:ex:tower_diagram_graph/forward} is a \hyperref[def:quiver_geometric_realization/linear]{linear graph}.

    \thmitem{ex:def:quiver_geometric_realization/triangle} The graph with vertices \( V = \set{ a, b, c } \) and arcs \( \set{ \overbrace{a \to b}^{e_1}, \overbrace{b \to c}^{e_2}, \overbrace{c \to a}^{e_3} } \) is more subtle.

    We start with three copies of the interval \( [0, 1] \), depicted in \eqref{eq:ex:def:quiver_geometric_realization/triangle/relization} as upward-pointing arrows, and use dashed lines to connect the endpoints that we want to glue together.
    \begin{equation}\label{eq:ex:def:quiver_geometric_realization/triangle/relization}
      \begin{aligned}
        \includegraphics{figures/eq__ex__def__graph_geometric_realization__triangle__realization.pdf}
      \end{aligned}
    \end{equation}

    After contracting the dashed lines, we obtain a topological space that can easily be \hyperref[def:quiver_geometric_realization/embedding]{embedded} into \( \BbbR^2 \). An obvious embedding corresponds to \enquote{pulling up} \( e_2 \) and \( e_3 \):
    \begin{equation}\label{eq:ex:def:quiver_geometric_realization/triangle/embedding}
      \begin{aligned}
        \includegraphics{figures/eq__ex__def__graph_geometric_realization__triangle__embedding.pdf}
      \end{aligned}
    \end{equation}

    This is only one possible embedding of the geometric realization. It is sufficient, however, for proving that the graph is \hyperref[def:quiver_geometric_realization/planar]{planar}.

    It is also clear from this example that constructing embeddings of graphs can be a very tedious task for graphs with more than a few arcs.
  \end{thmenum}
\end{example}
