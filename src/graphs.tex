\subsection{Graphs}\label{subsec:graphs}

\begin{remark}\label{rem:directed_and_undirected_graphs}
  Unfortunately, the word \enquote{graph} has at least three popular meanings within mathematics:
  \begin{itemize}
    \item The graph of a function --- \fullref{def:multi_valued_function/graph}.
    \item A directed graph --- \fullref{def:graph/directed}.
    \item An undirected graph --- \fullref{def:graph/undirected}.
  \end{itemize}

  Graphs of functions are different enough from the other two notions to not cause any confusion, however it is often not clear from the context whether \enquote{graph} refers to directed or undirected graphs. Both are formalisms corresponding to dots in the plane connected with (directed or undirected) lines, as illustrated in \fullref{ex:def:graph/directed}. Furthermore, even if it is clear whether the graph is directed or undirected, it is often unclear whether it is \hyperref[def:graph/directed/order]{finite}, whether it is \hyperref[def:graph/directed/simple]{simple} and whether it is a \hyperref[def:multigraph]{multigraph} or not. We try to be as precise as possible to avoid confusion. It doesn't help that the terminology itself is inconsistent as can be seen discussed through the definitions.

  We define undirected graphs as a special case of directed graphs. This approach does make some definitions more awkward, however we prefer it over defining the two differently. Furthermore, software implementations of undirected graphs as a special case of directed graphs are often more versatile --- see \cite[sec. 5.4]{Erickson2019} and \cite[ch. 1, sec. 2.4]{GondranMinoux1984Graphs}.
\end{remark}

\begin{definition}\label{def:graph}\mcite[ch. 1, sec. 1.1]{GondranMinoux1984Graphs}
  We will define, in parallel, directed and undirected graphs. Fix an arbitrary set \( V \).

  \begin{thmenum}[series=def:graph]
    \thmitem{def:graph/vertices} We will call the elements of \( V \) \term{vertices} or \term{nodes}.

    \thmitem{def:graph/directed} A \term{directed graph} or \term{digraph} is a pair \( G = (V, E) \), where \( E \subseteq V^2 \) is a \hyperref[def:binary_relation]{binary relation} on \( V \).

    The elements of \( E \) are called \term{arcs} or \term{arrows}. For every arc \( e = (u, v) \), the vertex \( u = \head(e) \) is called its \term{head} and \( v = \tail(e) \) is called its \term{tail}. If \( (u, v) \in E \), we say that \( u \) and \( v \) are \term{neighbors}, that they are \term{adjacent}, that \( v \) is a \term{successor} of \( u \) and that \( u \) is a \term{predecessor} of \( v \). These terms are used in, among other places, \cite[sec. 5.2]{Erickson2019}. Both \( u \) and \( v \) are called \term{endpoints} of \( e \), and the head and tail are also called the \term{starting}/\term{initial} endpoint and \term{final}/\term{terminal} endpoint. The latter terms are introduced in \cite[ch. 1, sec. 1.1]{GondranMinoux1984Graphs}.

    For readability, we sometimes use the notation \( u \to v \) rather than \( (u, v) \).

    \thmitem{def:graph/undirected} An \term{undirected graph} is a family of unordered pairs \( \set{ u, v } \) for some \( u, v \in V \) (or unary sets in the case \( u = v \)). Each unordered pair is called an \term{edge} and its members are called \term{endpoints}. Unlike arcs in directed graphs, edges have no head or tail.

    Edges are different from arcs, however it will be convenient for us to define an undirected graph as a directed graph \( G = (V, E) \), in which the relation \( E \) is \hyperref[def:binary_relation/symmetric]{symmetric}. We will thus regard each edge as either an ordered or unordered pair depending on the context. It may be confusing, but this is the adopted convention.

    We should note that some authors use the terms \term{arc} and \term{edge} as per our definitions. For example, arcs are introduced in \cite[ch. 1, sec. 1.1]{GondranMinoux1984Graphs} and edges a bit later in \cite[ch. 1, sec. 1.3]{GondranMinoux1984Graphs}. Other authors, e.g. \cite[sec. 5.2]{Erickson2019}, use the term \enquote{edge} to mean both an arc and an edge depending on whether the graph is directed or not.

    \thmitem{def:graph/symmetrization} Let \( G = (V, E) \) be a directed graph. The undirected graph \( (V, \cl^S(E)) \), whose set of edges is the \hyperref[def:derived_relations/symmetric]{symmetric closure} of \( E \), is called the \term{symmetrization} of \( G \).

    \thmitem{def:graph/order} The \term{order} \( \ord(G) \) of a (directed or undirected) graph is the \hyperref[thm:cardinality_existence]{cardinality} of \( V \). Without context, we usually assume that the set \( V \) of vertices is nonempty and \hyperref[def:set_finiteness]{finite}, however it is sometimes beneficial to consider empty or infinite graphs and hence we define \( \ord(G) \) to be a general \hyperref[def:cardinal]{cardinal number} rather than a \hyperref[rem:peano_arithmetic_zero/nonnegative]{nonnegative integer}.

    If the order \( \ord(G) \) is finite (resp. infinite), we say that \( G \) is a finite (resp. infinite) graph. See also local finiteness defined in \fullref{def:graph_incidence/degree}.

    \thmitem{def:graph/simple} An arc whose head and tail are equal is called a \term{loop}. An assumption that is often implicitly made is that a graph does not have loops. To make things more explicit, we call such graphs \term{simple}. This terminology is used in \mcite[ch. 1, sec. 1.3]{GondranMinoux1984Graphs}, for example.
  \end{thmenum}

  Graphs have the following metamathematical properties:
  \begin{thmenum}[resume=def:graph]
    \thmitem{def:graph/theory} The theory of directed graphs is an empty \hyperref[def:first_order_theory]{first-order theory} over the language consisting of a single infix binary relation \( \to \). For undirected graphs we must add \hyperref[def:binary_relation/symmetric]{symmetry} as the only axiom.

    See \fullref{rem:directed_graphs_model_theory} for how graphs are related to other theories with a similar signature.

    \thmitem{def:graph/directed/homomorphism} A \hyperref[def:first_order_homomorphism]{first-order homomorphism} between two graphs \( G_1 = (V_1, E_1) \) and \( G_2 = (V_2, E_2) \) is a function \( f: V_1 \to V_2 \) between their universes such that for every arc \( a \to b \) of \( G_1 \), \( f(a) \to f(b) \) is an arc of \( G_2 \).

    It is a weak homomorphism in the sense of \fullref{def:first_order_homomorphism}.

    Homomorphisms preserve symmetry, so this definition works without modification for undirected graphs.

    \thmitem{def:graph/directed/submodel} As for preordered sets, any subset \( V' \subseteq V \) of vertices induces a directed graph with the \( E \) relation restricted to \( V' \). That is, \( G' = (V', E') \) is a \term{subgraph} of \( G \) if \( V' \subseteq V \) and
    \begin{equation*}
      E' = \set{ e \in E \given \head(e) \in U \T{and} \tail(e) \in U }.
    \end{equation*}

    Note that any subgraph of an undirected graph is again an undirected graph.

    \thmitem{def:graph/directed/trivial} Unlike the \hyperref[def:group/trivial]{trivial group} \( \set{ e } \) or \hyperref[def:partially_ordered_set/trivial]{empty ordered set}, which are unique up to an isomorphism, there is no single agreed upon graph called the \enquote{trivial graph}.

    An unambiguous concept is that of an \term{edgeless graph}, in which the set of arcs/edges is empty, but the set of vertices may or may not be empty. Every graph \( G = (V, E) \) has exactly \( 2^{\ord(G)} \) edgeless subgraphs (one for each subset of \( V \)).

    The bottom of the \hyperref[thm:substructures_form_complete_lattice]{lattice of submodels} of \( G \) is the \term{order-zero graph} \( (\varnothing, \varnothing) \). The order-zero graph is unique.

    The terms \term{empty graph}, \term{null graph} and \term{trivial graph} may refer to either edgeless graphs or the order-zero graph depending on the author and the situation.

    \thmitem{def:graph/directed/category} We denote the \hyperref[def:category_of_first_order_models]{categories of models} by \( \cat{DGraph} \) for directed graphs and by \( \cat{UGraph} \) for undirected graphs.
  \end{thmenum}
\end{definition}

\begin{example}\label{ex:def:graph}
  Consider the directed graph
  \begin{equation}\label{eq:ex:def:graph/directed}
    \begin{aligned}
      \includegraphics{figures/eq__ex__def__directed_graph.pdf}
    \end{aligned}
  \end{equation}
  and its \hyperref[def:graph/symmetrization]{symmetrization}, the undirected graph.
  \begin{equation}\label{eq:ex:def:graph/undirected}
    \begin{aligned}
      \includegraphics{figures/eq__ex__def__undirected_graph.pdf}
    \end{aligned}
  \end{equation}

  We will use both graphs as generic examples. The few basic properties we can list now is that the graphs have finite order \( 6 \) and that they are simple graphs since they contain no loops. Note also that the edges are numbered in \hyperref[eq:def:lexicographic_order]{lexicographic order}. We will come back to this example later with more insightful comments.

  It should be noted that when defining a concrete graph, it is impractical to enumerate the vertices and edges. It is instead more understandable to work with an \hyperref[def:graph_embedding]{embedding} of the graph into \( \BbbR^2 \) like it is done here.
\end{example}

\begin{remark}\label{rem:graphs_linear_algebra_and_topology}
  As we shall see, the \hyperref[def:graph]{adjacency} and \hyperref[def:graph_incidence]{incidence} of a graph can be easily studied using linear algebra via \hyperref[def:graph_matrices/adjacency]{adjacency} and \hyperref[def:graph_matrices/incidence]{incidence matrices}, while the \hyperref[def:graph_connectedness]{connectedness} of a graph can be studied using \hyperref[def:graph_connectedness]{topology} via \hyperref[def:graph_embedding]{graph embeddings}.
\end{remark}

\begin{definition}\label{def:graph_incidence}
  Let \( G = (V, E) \) be a directed graph. We define several \hyperref[def:multi_valued_function]{multi-valued functions} from \( \pow(V) \) to \( E \):
  \begin{align*}
     w^+(A) &\coloneqq \set{ (u, v) \in E \given u \in A } \\
     w^-(A) &\coloneqq \set{ (u, v) \in E \given v \in A } \\
     w(A)   &\coloneqq w^+(A) \cup w^-(A).
  \end{align*}

  For a set \( A \) of vertices, \( w^+(A) \) gives us the set of arcs whose head is in \( A \), \( w^-(A) \) gives us the set of arcs whose tail is in \( A \) and \( w(A) \) gives us all arcs with at least one endpoint in \( A \).

  This notation is used in \cite[ch. 1, sec. 1.4]{GondranMinoux1984Graphs}. It appears not to be very conventional, and we will avoid as much as possible.

  In a simple directed graph, \( w^+(v) \) is disjoint from \( w^-(v) \) for every individual vertex \( v \). In an undirected graph \( G \), we have \( w(A) = w^+(A) = w^-(A) \) for every set \( A \) of vertices.

  \begin{thmenum}
    \thmitem{def:graph_incidence/incident_arcs} The arc \( e \) is said to be \term{incident} with the vertex \( v \) if \( e \in w(v) \). That is, if \( v \) is an endpoint of \( e \).

    \thmitem{def:graph_incidence/degree}\mcite[ch. 1, sec. 1.4]{GondranMinoux1984Graphs} The \term{degree} of a vertex \( v \) is defined as
    \begin{equation*}
      \deg(v) \coloneqq \card w(v).
    \end{equation*}

    The \term{in-degree} \( \deg^+(v) \) and \term{out-degree} \( \deg^-(v) \) are defined in an obvious way.

    The degree of the graph itself is then defined as
    \begin{equation*}
      \deg(G) \coloneqq \max_{v \in V} d(v).
    \end{equation*}

    It is possible that the maximum in \( \deg(G) \) is not attained if \( G \) is infinite. If \( G \) is infinite but \( \deg(G) \) is finite, we say that the graph \( G \) is \term{locally finite}.
  \end{thmenum}
\end{definition}

\begin{example}\label{ex:def:graph_incidence}
  In the directed graph \eqref{eq:ex:def:graph/directed}, we have
  \begin{align*}
    w^+(b) &= \set{ c, e } \\
    w^-(b) &= \set{ a } \\
    w(b)   &= \set{ a, c, e }
  \end{align*}

  The sets \( w^+(b) \) and \( w^-(b) \) are disjoint since the graph is simple. In the \hyperref[def:graph/symmetrization]{symmetrization} \eqref{eq:ex:def:graph/undirected} of \eqref{eq:ex:def:graph/directed}, all three sets are equal to \( w(b) = \set{ a, c, e } \).

  For both graphs
  \begin{equation*}
    \deg(G) = \max\set{ w(v) \given v \in V } = \deg w(b) = 3.
  \end{equation*}
\end{example}

\begin{example}\label{ex:tower_diagram_graph}
  A very simple example of an infinite graph is the \hyperref[def:derived_relations/transitive]{transitive reduction} of the nonnegative integers, i.e. the graph
  \begin{equation}\label{eq:ex:tower_diagram_graph/forward}
    \begin{aligned}
      \includegraphics{figures/eq__ex__tower_diagram_graph__forward.pdf}
    \end{aligned}
  \end{equation}

  Since the graph is simple, we have
  \begin{equation*}
    \deg(n) = \deg^-(n) + \deg^+(n) = \begin{cases}
      1 = 0 + 1, &n = 0, \\
      2 = 1 + 1, &n > 0 \\
    \end{cases}
  \end{equation*}
  and thus \( \deg(G) = 2 \).

  We call the corresponding \hyperref[def:categorical_diagram]{categorical diagram} a \term{tower diagram} --- see \fullref{def:tower_diagram}.

  Another related graph is based on the nonpositive integers:
  \begin{equation}\label{eq:ex:tower_diagram_graph/backward}
    \begin{aligned}
      \includegraphics{figures/eq__ex__tower_diagram_graph__backward.pdf}
    \end{aligned}
  \end{equation}

  Finally, the union of the two gives us a two-sided tower:
  \begin{equation}\label{eq:ex:tower_diagram_graph/two_sided}
    \begin{aligned}
      \includegraphics{figures/eq__ex__tower_diagram_graph__two_sided.pdf}
    \end{aligned}
  \end{equation}

  All three graphs are infinite but locally finite.
\end{example}

\begin{definition}\label{def:graph_matrices}
  Let \( G = (V, E) \) be a finite simple graph, either directed or undirected.
  \begin{thmenum}
    \thmitem{def:graph_matrices/incidence}\mcite[ch. 1, sec. 2.1]{GondranMinoux1984Graphs} We define the \term{incidence matrix} \( I = \seq{ a_{ve} }_{v \in V, e \in E} \) for \( G \).

    \begin{minipage}[t]{0.45\textwidth}
      For a directed graph, \( I \) has elements
      \begin{equation*}
        a_{ve} \coloneqq \begin{cases}
          1,  & v = \head(e) \\
          -1, & v = \tail(e) \\
          0,  & \T{otherwise.}
        \end{cases}
      \end{equation*}
    \end{minipage}
    \hspace{0.02\textwidth}
    \begin{minipage}[t]{0.45\textwidth}
      For an undirected graph, \( I \) has elements
      \begin{equation*}
        a_{ve} \coloneqq \begin{cases}
          1,  & v \in e \\
          0,  & \T{otherwise.}
        \end{cases}
      \end{equation*}
    \end{minipage}

    The two definitions are quite different but in both cases, the matrix element \( a_{ve} \) is nonzero precisely when \( v \) is an endpoint of \( e \).

    \thmitem{def:graph_matrices/adjacency}\mcite[ch. 1, sec. 2.3]{GondranMinoux1984Graphs} The \term{adjacency matrix} \( I = \seq{ a_{uv} }_{u, v \in V} \) for \( G \) has elements
    \begin{equation*}
      a_{uv} \coloneqq \begin{cases}
        1,  & (u, v) \in E \\
        0,  & \T{otherwise.}
      \end{cases}
    \end{equation*}

    Unlike the incidence matrix, the adjacency matrix regards undirected graphs as special cases of directed graphs. See also \fullref{thm:graph_undirected_iff_adjacency_matrix_is_symmetric}.
  \end{thmenum}
\end{definition}

\begin{example}\label{ex:def:graph_matrices}
  The \hyperref[def:graph_matrices/incidence]{incidence matrix} corresponding to the directed graph \eqref{eq:ex:def:graph/directed} is
  \begin{equation}\label{ex:def:graph_matrices/incidence}
    \begin{pNiceMatrix}[first-row,first-col]
        & 1         & 2         & 3         & 4         & 5         & 6         & 7         \\
      a & 1         & 1         &           &           &           &           &           \\
      b & \fbox{-1} &           & 1         & 1         &           &           &           \\
      c &           &           & \fbox{-1} &           & 1         &           &           \\
      d &           & \fbox{-1} &           &           &           & 1         &           \\
      e &           &           &           & \fbox{-1} &           & \fbox{-1} & 1         \\
      f &           &           &           &           & \fbox{-1} &           & \fbox{-1}
    \end{pNiceMatrix}
  \end{equation}

  It can be read column-by-column. Every column contains exactly two nonzero elements whose rows correspond to the head (positive) and tail (negative).

  To obtain the incidence matrix for the \hyperref[def:graph/symmetrization]{symmetrization} \eqref{eq:ex:def:graph/undirected} of \eqref{eq:ex:def:graph/directed}, we need to simply flip the sign of the boxed elements above.

  The \hyperref[def:graph_matrices/adjacency]{adjacency matrix} is
  \begin{equation}\label{ex:def:graph_matrices/adjacency}
    \begin{pNiceMatrix}[first-row,first-col]
        & a        & b        & c        & d        & e        & f \\
      a &          & 1        &          & 1        &          &   \\
      b & \fbox{1} &          & 1        &          & 1        &   \\
      c &          & \fbox{1} &          &          &          & 1 \\
      d & \fbox{1} &          &          &          & 1        &   \\
      e &          & \fbox{1} &          & \fbox{1} &          & 1 \\
      f &          &          & \fbox{1} &          & \fbox{1} &
    \end{pNiceMatrix}
  \end{equation}
  where the boxed elements are nonzero only in the adjacency matrix for the undirected graph.

  The matrix can be read either column-by-column or row-by-row.
  \begin{itemize}
    \item The \( v \)-th column lists the vertices \( u \) such that there is an edge \( u \to v \).
    \item The \( u \)-th row lists the vertices \( v \) such that there is an edge \( u \to v \).
  \end{itemize}
\end{example}

\begin{proposition}\label{thm:graph_undirected_iff_adjacency_matrix_is_symmetric}
  A \hyperref[def:graph/directed]{directed graph} is \hyperref[def:graph/directed]{undirected} if and only if its \hyperref[def:graph_matrices/adjacency]{adjacency matrix} is \hyperref[def:symmetric_matrix]{symmetric}.
\end{proposition}
\begin{proof}
  Trivial.
\end{proof}

\begin{definition}\label{def:graph_spaces}
  Let \( G = (V, E) \) be a finite graph. We introduce several \hyperref[def:vector_space]{vector spaces} over \( G \) that allow us to study graphs using linear algebra.

  \begin{thmenum}
    \thmitem{def:graph_spaces/vertex} The \term{vertex space} \( \BbbF_2^V \) is the \hyperref[def:left_module_of_tuples]{tuple vector space} of dimension \( \card(V) = \ord(G) \) over the finite \hyperref[def:field]{field} \hyperref[thm:f2_is_boolean_algebra]{\( \BbbF_2 \)}.

    Every subset \( A \subseteq V \) of vertices induces a unique vector \( \vect{A} = \seq{ \vect{A}_v }_{v \in V} \) in the \hyperref[def:graph_spaces/vertex]{vertex space} \( \BbbF_2^V \) such that
    \begin{equation*}
      \vect{A}_v \coloneqq \begin{cases}
        1, &v \in A \\
        0, &v \not\in A
      \end{cases}
    \end{equation*}

    This vector is called the \term{characteristic vector} of \( A \). Conversely, every vector in \( \BbbF_2^V \) induces a set of vertices. If \( A \) consists of a single vertex \( u \), we write \( \vect{u} \) rather than \( \vect{\set{u}} \).

    It is important to note that, in accordance with \fullref{def:function/set_of_functions}, \( \BbbF_2^V \) is the set of all functions from \( V \) to \( \BbbF_2 \). Since we have assumed that the graph is finite, this space is isomorphic to the tuple space \( \BbbF_2^{\ord(G)} \). Unfortunately, this isomorphism is not unique since it does not give us a choice of ordering of \( V \). Thus, we cannot regard the vectors of \( \BbbF_2^V \) as ordered tuples unless \( V \) is \hyperref[def:well_ordered_set]{well-ordered}.

    \thmitem{def:graph_spaces/edge} Analogously, the \term{edge space} \( \BbbF_2^E \) is the \hyperref[def:left_module_of_tuples]{tuple vector space} of dimension \( \card(E) \). Every subset of \( E \) induces a unique characteristic vector in the \hyperref[def:graph_spaces/edge]{edge space} \( \BbbF_2^E \) and vice versa.

    Despite its name, this vector space encodes directed arcs just as well as undirected edges.

    \thmitem{def:graph_spaces/arc} Finally, we introduce the more complicated \term{arc space} \( \BbbF_3^E \). It is sometimes not sufficient to consider only a subset \( A \subseteq E \) but rather a pair of subsets \( A, B \subseteq E \) such that the arcs in \( A \) are \enquote{positively oriented} and \( B \) are \enquote{negatively oriented}. The terms \enquote{positively oriented} and \enquote{negatively oriented} are informal only have meaning in certain applications. This is the case with \hyperref[def:graph_paths/directed_path]{simple directed paths}, for example.

    The characteristic vector \( x \) of a pair \( (A, B) \) of subsets has components
    \begin{equation*}
      \vect{(A, B)}_e \coloneqq \begin{cases}
        1,  &e \in A \\
        -1, &e \in B \\
        0,  &\T{otherwise.}
      \end{cases}
    \end{equation*}

    As for the other vector spaces, every pair \( (A, B) \) has a characteristic vector in \( \BbbF_3^E \) and every vector in \( \BbbF_3^E \) corresponds to a pair of subsets of \( E \).
  \end{thmenum}
\end{definition}

\begin{proposition}\label{thm:graphs_as_linear_transformations}
  Let \( V \coloneqq \set{ 1, \ldots, n } \) for some positive integer \( n \). Then there is a bijection between the \hyperref[def:graph/directed]{directed graphs} over \( V \) and the \hyperref[def:linear_operator]{linear} \hyperref[def:endomorphism]{endomorphisms} over \( \BbbF_2 \).
\end{proposition}
\begin{proof}
  Let \( G = (V, E) \) be a directed graph and let \( I \) be its \hyperref[def:graph_matrices/adjacency]{adjacency matrix}. Then \( I \) induces a linear endomorphism in \( \BbbF_2^n \).

  Conversely, let \( T: \BbbF_2^n \to \BbbF_2^n \) be a linear endomorphism. Define the set
  \begin{equation*}
    E \coloneqq \set{ (u, v) \in V^2 \given T(\vect{\set{v}})_u = 1 }.
  \end{equation*}

  Then the adjacency matrix of \( G = (V, E) \) induces \( T \).
\end{proof}

\begin{example}\label{ex:thm:graphs_as_linear_transformations}
  We will empirically verify the proof of \fullref{thm:graphs_as_linear_transformations}.

  Denote by \( A \) the adjacency matrix \eqref{ex:def:graph_matrices/adjacency} of \eqref{eq:ex:def:graph/directed} discussed in \fullref{ex:def:graph_matrices}. Consider the vertex \( b \). The characteristic vector of \( b \) is \( \vect{b} = (0, 1, 0, 0, 0, 0) \). Thus, the matrix product \( A \vect{b} \) simply \enquote{selects} the \( b \)-th column of \( I \), which is
  \begin{equation*}
    A \vect{b}
    =
    \begin{pNiceMatrix}[last-row]
      1        & 0        & 0        & 0        & 0        & 0 \\
      a        & b        & c        & d        & e        & f
    \end{pNiceMatrix}^T
  \end{equation*}

  The only nonzero member of \( A \vect{b} \) is the one corresponding to the vertex \( a \). This is consistent with \( a \) being the only predecessor of \( b \) in \eqref{eq:ex:def:graph/directed}.
\end{example}

\begin{definition}\label{def:graph_paths}
  Let \( G = (V, E) \) be a directed or undirected graph. We will define \term{paths}.

  \begin{thmenum}
    \thmitem{def:graph_paths/adjacent_vertices}\mcite[ch. 1, sec. 1.4]{GondranMinoux1984Graphs} Two vertices \( u \) and \( v \)  are called \term{adjacent} if there exists an arc from \( u \)  to \( v \).

    \thmitem{def:graph_paths/adjacent_arcs}\mcite[ch. 1, sec. 1.4]{GondranMinoux1984Graphs} Two arcs are called \term{adjacent} if they have a common endpoint.

    Note that we have not specified how the endpoints are related --- \( u \to v \) is adjacent to \( v \to w \) as well as to \( u \to w \).

    \thmitem{def:graph_paths/undirected_path}\mcite[ch. 1, sec. 1.3]{GondranMinoux1984Graphs} An \term{undirected path} is a finite or infinite \hyperref[def:sequence]{sequence} of arcs
    \begin{equation*}
      p \coloneqq (e_1, e_2, \ldots),
    \end{equation*}
    such that any two consecutive arcs are adjacent. That is, the arcs \( e_k \) and \( e_{k+1} \) are adjacent for every \( k = 1, 2, \ldots \).

    If the graph is undirected, we regard these as unordered edges rather than ordered arcs, which brings some nuances. We will further comment on this where necessary.

    The \term{length} of \( p \) is defined in an obvious way for finite paths and as \( p = \infty \) for infinite paths (here \( \infty \) is merely a more conventional symbol for denoting \hyperref[thm:omega_is_a_cardinal]{\( \aleph_0 \)}).

    \thmitem{def:graph_paths/endpoints} We say that \( u \) is the \term{head} of \( p \) if it is an endpoint of \( e_1 \), but not \( e_2 \).

    If \( p \) has length \( n < \infty \), we say that \( v \) is the \term{tail} of \( p \) if it is an endpoint of \( e_n \), but not \( e_{n-1} \).

    Both the head and the tail of a path may not exist. For example, the path \( \set{ u \to v, v \to u } \) has neither despite being finite. In an undirected graph, an undirected path always has an edge

    \thmitem{def:graph_paths/directed_path}\mcite[ch. 1, sec. 3.2]{GondranMinoux1984Graphs} If the tail of each non-endpoint arc in a path coincides with the head of the next arc, we say that the path is \term{directed}.

    A directed path always has a head and, if it is finite, a tail.

    Some authors (e.g. \cite[sec. 5.2]{Erickson2019}) call undirected paths \term{walks} and reserve the term \enquote{path} for simple undirected paths.

    If the head and the tail of a path coincide, we say that the path is a \term{cycle}.

    A \term{simple cycle} is a cycle where all non-endpoint vertices are distinct.

    \thmitem{def:graph_paths/directed_path}

    In the graph \fullref{ex:def:graph/directed/embedding}, \( (1, 2, 3) \) is a directed path, while \( (3, 4) \) is not.

    A directed cycle is also called a \term{circuit}.

    \thmitem{def:graph_paths/dag}\mcite[231]{Erickson2019}A \term{directed acyclic graph} or \term{dag} is a directed graph without directed cycles.
  \end{thmenum}
\end{definition}

\begin{definition}\label{def:graph_connectedness}
  Let \( G = (V, E) \) be a directed graph.

  \begin{thmenum}
    \thmitem{def:graph_connectedness/reachable_vertices}The vertex \( v \) is \term{reachable} from the vertex \( u \) if there exists a directed \hyperref[def:graph_paths/directed_path]{path} from \( u \) to \( v \).

    \thmitem{def:graph_connectedness/strongly_connected_graph}\mcite[ch. 1, sec. 3.5]{GondranMinoux1984Graphs}The graph \( G \) is \term{strongly connected} if every pair of distinct vertices are reachable, that is, if there exists a directed path between every pair of distinct vertices.

    \thmitem{def:graph_connectedness/weakly_connected_graph}\mcite[ch. 1, sec. 3.3]{GondranMinoux1984Graphs}The graph \( G \) is \term{weakly connected} if there exists an undirected path between every pair of distinct vertices.

    \thmitem{def:graph_connectedness/connected_component}\mcite[ch. 1, sec. 3.3 \\ ch. 1, sec. 3.5]{GondranMinoux1984Graphs}The subgraph \( G' \) of \( G \) is a \term{connected component} (resp. \term{strongly connected component}) if it is connected (resp. strongly connected) and there exists no connected (resp. strongly connected) subgraph of \( G \) that properly contains \( G' \).

    \thmitem{def:graph_connectedness/connectivity_number}\mcite[ch. 1, sec. 3.3 \\ ch. 1, sec. 3.5]{GondranMinoux1984Graphs}\( G \) has \term{connectivity number} (resp. \term{strong connectivity number}) \( n \) if it has \( n \) connected (resp. strongly connected) components.
  \end{thmenum}
\end{definition}

\begin{definition}\label{def:graph_adjacency}
  Let \( G = (V, E) \) be an undirected graph.

  \begin{thmenum}
    \thmitem{def:graph_adjacency/clique}\mcite[ch. 1, sec. 1.4]{GondranMinoux1984Graphs}The set \( U \subseteq V \) is called a \term{clique} if all two vertices in \( U \) are adjacent.

    \medskip

    \thmitem{def:graph_adjacency/complete_graph}\mcite[ch. 1, sec. 1.4]{GondranMinoux1984Graphs}If \( V \) itself is a clique, we say that \( G \) is a \term{complete graph}.

    \medskip

    \thmitem{def:graph_adjacency/anticlique}\mcite[120]{Erickson2019}Dually, \( U \subseteq V \) is an \term{anticlique} or \term{independent set} of vertices if no two vertices in \( U \) are adjacent.

    \thmitem{def:graph_adjacency/bipartite_graph}\mcite[7]{GondranMinoux1984Graphs}The graph is called \term{bipartite} if there exists a partition \( \{ A, B \} \) of \( V \) such that both \( A \) and \( B \) are anticliques. We also write \( G = (A, B, E) \).

    If \( G \) is undirected and if for every pair of vertices \( a \in A, b \in B \) there is an arc \( a \to b \), we say that \( G \) is a complete bipartite graph.
  \end{thmenum}
\end{definition}

\begin{remark}\label{rem:directed_graphs_model_theory}
  A (potentially infinite) \hyperref[def:graph/directed]{directed graph} \( G = (V, E) \) is defined as a set \( V \) with a binary relation \( E \). The following properties of graphs and relations are equivalent:
  \begin{center}
    \begin{tabular}{l | l}
      \hyperref[def:graph/directed]{Directed graphs}           & \hyperref[def:binary_relation]{Binary relations} \\
      \hline
      \hyperref[def:graph/directed/simple]{Simple}             & \hyperref[def:binary_relation/irreflexive]{Irreflexive} \\
      \hyperref[def:graph/undirected]{Undirected}              & \hyperref[def:binary_relation/symmetric]{Symmetric} \\
      \hyperref[def:graph_paths/undirected_path]{Acyclic}      & \hyperref[def:well_founded_relation]{Well-founded} \\
    \end{tabular}
  \end{center}

  Some properties are more subtle. For example, the edges of every \hyperref[def:graph_adjacency/complete_graph]{complete graph} are a \hyperref[def:binary_relation/transitive]{transitive relation} but not vice versa.

  From the point of view of \hyperref[subsec:first_order_models]{model theory}, over a \hyperref[def:first_order_language]{first-order language} with only a single predicate symbol every \hyperref[def:first_order_structure]{structure} is a graph and the \hyperref[def:first_order_homomorphism]{homomorphisms} are precisely the homomorphisms of these graphs. Such theories are the theory of \hyperref[def:preordered_set]{preordered sets} and \hyperref[def:zfc]{\logic{ZFC}}.

  For example, every \hyperref[def:partially_ordered_set/strict]{strictly partially ordered set} \( (\mscrP, <) \) is a simple directed graph but not vice versa.

  More interestingly, if \( \mscrV = (V, I) \) is a model of \hyperref[def:zfc]{\logic{ZFC}}, then \( \mscrV \) is \hyperref[def:graph_paths/undirected_path]{acyclic} as a consequence of the \hyperref[def:zfc/foundation]{axiom of foundation} (via \fullref{thm:set_membership_is_well_founded}). In particular, all \hyperref[def:grothendieck_universe]{Grothendieck universes} and \( V_\kappa \) for any \hyperref[def:regular_cardinal]{strongly inaccessible cardinal} \( \kappa \) are graphs (as consequences of \fullref{thm:grothendieck_universe_is_model_of_zfc} and \fullref{thm:cumulative_hierarchy_model_of_zfc}, respectfully).

  More generally, \( (A, \in) \) is a graph for every set \( A \). The edge relation is simply the membership relation restricted to \( A \). From the equivalences in the table above it follows that any \hyperref[def:ordinal]{ordinal} is an acyclic graph since it is well-founded by definition even without the axiom of foundation.
\end{remark}
