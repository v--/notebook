\subsection{Graphs}\label{subsec:graphs}

\begin{remark}\label{rem:directed_and_undirected_graphs}
  Unfortunately, the word \enquote{graph} has at least three popular meanings within mathematics:
  \begin{itemize}
    \item The graph of a function --- \fullref{def:multi_valued_function/graph}.
    \item A directed graph --- \fullref{def:directed_graph}.
    \item An undirected graph --- \fullref{def:undirected_graph}.
  \end{itemize}

  Graphs of functions are different enough from the other two notions to not cause any confusion, however it is often not clear from the context whether \enquote{graph} refers to directed or undirected graphs. Both are formalisms corresponding to dots in the plane connected with (directed or undirected) lines, as illustrated in \fullref{ex:def:directed_graph}. Furthermore, even if it is clear whether the graph is directed or undirected, it is often unclear whether it is \hyperref[def:directed_graph/order]{finite}, whether it is \hyperref[def:directed_graph/simple]{simple} and whether it is a \hyperref[def:multigraph]{multigraph} or not. We try to be as precise as possible to avoid confusion.

  We define undirected graphs as a special case of directed graphs. This approach does make some definitions more awkward, however we prefer it over defining the two differently. Furthermore, software implementations of undirected graphs as a special case of directed graphs are often more versatile --- see \cite[sec. 5.4]{Erickson2019} and \cite[ch. 1, sec. 2.4]{GondranMinoux1984Graphs}.
\end{remark}

\begin{definition}\label{def:directed_graph}\mcite[ch. 1, sec. 1.1]{GondranMinoux1984Graphs}
  A \term{directed graph} is, at least formally, a \hyperref[def:binary_relation]{binary relation} without any restrictions. It is conventional to define a graph as the pair \( G = (V, E) \), where \( V \) is an arbitrary set and \( E \subseteq V^2 \) is the relation.

  \begin{thmenum}[series=def:directed_graph]
    \thmitem{def:directed_graph/vertices} The elements of \( V \) are called \term{vertices} or \term{nodes}.

    \thmitem{def:directed_graph/order} The \term{order} \( \ord(G) \) of the graph is the \hyperref[thm:cardinality_existence]{cardinality} of \( V \). Without context, we usually assume that the set \( V \) of vertices is \hyperref[def:set_finiteness]{finite}, however it is sometimes beneficial to consider infinite graphs and hence we define \( \ord(G) \) to be a general \hyperref[def:cardinal]{cardinal number} rather than a \hyperref[rem:peano_arithmetic_zero/nonnegative]{nonnegative integer}.

    \thmitem{def:directed_graph/arcs} The elements of \( E \) are called \term{arcs}. For every arc \( e = (a, b) \), the vertex \( a \) is called its \term{head} and \( b \) is called its \term{tail}. We sometimes use the notation \( \head(e) \) and \( \tail(e) \), which notation is also fundamental for multigraphs --- see \fullref{def:multigraph}.

    For readability, we use the notation \( u \to v \) rather than \( (u, v) \) for arcs.

    \thmitem{def:directed_graph/loop} An arc whose head and tail are equal is called a \term{loop}.

    \thmitem{def:directed_graph/simple} A graph without loops is called \term{simple}.
  \end{thmenum}

  Directed graphs have the following metamathematical properties:
  \begin{thmenum}[resume=def:directed_graph]
    \thmitem{def:directed_graph/theory} The theory of directed graphs is an empty \hyperref[def:first_order_theory]{first-order theory} over the language consisting of a single infix binary relation \( \to \).

    See \fullref{rem:directed_graphs_model_theory} for some nontrivial consequences.

    \thmitem{def:directed_graph/homomorphism} A \hyperref[def:first_order_homomorphism]{first-order homomorphisms} between two graphs \( G_1 = (V_1, E_1) \) and \( G_2 = (V_2, E_2) \) is a function \( f: V_1 \to V_2 \) between their universes such that for every edge \( a \to b \) of \( G_1 \), \( f(a) \to f(b) \) is an edge of \( G_2 \).

    It is a weak homomorphism in the sense of \fullref{def:first_order_homomorphism}.

    \thmitem{def:directed_graph/submodel} As for preordered sets, any subset \( V' \subseteq V \) of vertices induces a directed graph with the \( E \) relation restricted to \( V' \). That is, \( G' = (V', E') \) is a \term{subgraph} of \( G \) if \( V' \subseteq V \) and
    \begin{equation*}
      E' = \set{ e \in E \given \head(e) \in U \T{and} \tail(e) \in U }.
    \end{equation*}

    \thmitem{def:directed_graph/trivial} Unlike the \hyperref[def:group/trivial]{trivial group} \( \set{ e } \) or \hyperref[def:partially_ordered_set/trivial]{empty ordered set}, which are unique up to an isomorphism, there is no single agreed upon graph called the \enquote{trivial graph}.

    An unambiguous concept is that of an \term{edgeless graph}, in which the set of edges is empty, but the set of vertices may or may not be empty. Every graph \( G = (V, E) \) has exactly \( 2^{\ord(G)} \) edgeless subgraphs (one for each subset of \( V \)).

    The bottom of the \hyperref[thm:substructures_form_complete_lattice]{lattice of submodels} of \( G \) is the \term{order-zero graph} \( (\varnothing, \varnothing) \). The order-zero graph is obviously unique.

    The terms \term{empty graph}, \term{null graph} and \term{trivial graph} may refer to either edgeless graphs or the order-zero graph depending on the author and the situation.

    \thmitem{def:directed_graph/category} We denote the \hyperref[def:category_of_first_order_models]{category of models} of directed graphs by \( \cat{DGraph} \).
  \end{thmenum}
\end{definition}

\begin{example}\label{ex:def:directed_graph/basics}
  Consider the directed graph
  \begin{equation}\label{eq:ex:def:directed_graph}
    \begin{aligned}
      \includegraphics{figures/eq__ex__def__directed_graph.pdf}
    \end{aligned}
  \end{equation}

  It has no remarkable properties, however it is simple enough to easily describe in detail. The few basic properties we can list now is that \eqref{eq:ex:def:directed_graph} has order \( 6 \) and is simple since it contains no loops. Note also that the edges are numbered in \hyperref[eq:def:lexicographic_order]{lexicographic order}. We will come back to this example later with more insightful comments.

  It should be noted that when defining a concrete graph, it is impractical to enumerate the vertices and edges. It is instead more understandable to work with an \hyperref[def:graph_embedding]{embedding} of the graph into \( \BbbR^2 \) like it is done in \eqref{eq:ex:def:directed_graph}.
\end{example}

\medskip

\begin{definition}\label{def:graph_matrices}
  Let \( G = (V, E) \) be a simple finite directed graph.
  \begin{thmenum}
    \thmitem{def:graph_matrices/incidence}\mcite[ch. 1, sec. 2.1]{GondranMinoux1984Graphs} The \term{incidence matrix} \( I = \{ a_{ve} \}_{v \in V, e \in E} \) of \( G \) is defined as
    \begin{equation*}
      a_{ve} \coloneqq \begin{cases}
        1,  & v \text{ is the head of } e \\
        -1, & v \text{ is the tail of } e \\
        0,  & \text{otherwise}
      \end{cases}
    \end{equation*}

    \thmitem{def:graph_matrices/adjacency}\mcite[ch. 1, sec. 2.3]{GondranMinoux1984Graphs} The \term{adjacency matrix} \( A = \{ a_{ve} \}_{u, v \in V} \) of \( G \) is defined as
    \begin{equation*}
      a_{uv} \coloneqq \begin{cases}
        1, & (u, v) \in E     \\
        0, & \text{otherwise}
      \end{cases}
    \end{equation*}
  \end{thmenum}
\end{definition}

\begin{example}\label{ex:def:directed_graph/matrices}
  The incidence matrix corresponding to \eqref{eq:ex:def:directed_graph} is
  \begin{balign*}
    \bordermatrix{
      & 1  & 2  & 3  & 4  & 5  & 6  & 7  \cr
    a & 1  & 1  &    &    &    &    & \cr
    b & -1 &    & 1  &    &    &    & \cr
    c &    &    & -1 & -1 &    & 1  & \cr
    d &    & -1 &    & 1  & 1  &    & \cr
    e &    &    &    &    & -1 &    & 1  \cr
    f &    &    &    &    &    & -1 & -1
    }
  \end{balign*}
  and the adjacency matrix is
  \begin{balign*}
    \bordermatrix{
      & a & b & c & d & e & f  \cr
    a &   & 1 &   & 1 &   & \cr
    b &   &   & 1 &   &   & \cr
    c &   &   &   &   &   & 1  \cr
    d &   &   & 1 &   & 1 & \cr
    e &   &   &   &   &   & 1  \cr
    f &   &   &   &   &   &
    }
  \end{balign*}
\end{example}

\begin{definition}\label{def:graph_paths}
  Let \( G = (V, E) \) be a directed graph.

  \begin{thmenum}
    \thmitem{def:graph_paths/adjacent_vertices} Two vertices \( u \)  and \( v \)  are called \term{adjacent} if there exists an arc from \( u \)  to \( v \) .

    \thmitem{def:graph_paths/adjacent_arcs}\mcite[ch. 1, sec. 1.4]{GondranMinoux1984Graphs} Two arcs are called \term{adjacent} if they have a common endpoint. Thus, the first three pairs of arcs are adjacent and the fourth is not (assuming all vertices are distinct):
    \begin{enumerate}
      \item \( u \to v \) and \( v \to w \)
      \item \( u \to v \) and \( u \to w \)
      \item \( u \to w \) and \( v \to w \)
      \item \( u \to u' \) and \( v \to v' \)
    \end{enumerate}

    \thmitem{def:graph_paths/undirected_path}\mcite[ch. 1, sec. 3.1]{GondranMinoux1984Graphs} An \term{undirected path} or \term{chain} is a sequence of distinct arcs
    \begin{equation*}
      p \coloneqq ( e_1, \ldots, e_n ),
    \end{equation*}
    such that any two consecutive arcs are adjacent, that is, the arcs \( e_i \) and \( e_{i+1} \) are adjacent for \( i = 1, \ldots, n - 1 \). We say that \( u \) is the \term{head} of \( p \) if it is an endpoint of \( e_1 \), but not \( e_2 \) and that \( v \) is the \term{tail} of \( p \) if it is an endpoint of \( e_n \), but not \( e_{n-1} \). The number \( n \) is called the \term{length} of the path and is denoted by \( \len p \).

    In the graph \fullref{ex:def:directed_graph/embedding}, \( (1, 2, 3) \) is a path with head \( a \), tail \( f \) and length 3, while \( (3, 4) \) is a path with head \( b \), tail \( d \) and length 2.

    If all vertices in a path are distinct, that is, if there are exactly \( n + 1 \) distinct vertices, we say that the path is \term{simple}.

    Some authors (e.g. \cite[sec. 5.2]{Erickson2019}) call undirected paths \term{walks} and reserve the term \enquote{path} for simple undirected paths.

    If the head and the tail of a path coincide, we say that the path is a \term{cycle}.

    A \term{simple cycle} is a cycle where all non-endpoint vertices are distinct.

    \thmitem{def:graph_paths/directed_path}\mcite[ch. 1, sec. 3.2]{GondranMinoux1984Graphs}If the tail of each non-endpoint arc in a path coincides with the head of the next arc, we say that the path is a \term{directed path}.

    In the graph \fullref{ex:def:directed_graph/embedding}, \( (1, 2, 3) \) is a directed path, while \( (3, 4) \) is not.

    A directed cycle is also called a \term{circuit}.

    \thmitem{def:graph_paths/dag}\mcite[231]{Erickson2019}A \term{directed acyclic graph} or \term{dag} is a directed graph without directed cycles.

    \medskip

    \thmitem{def:graph_paths/eulerian_path}\mcite[ch. 8, sec. 1.1]{GondranMinoux1984Graphs}A path (either directed or undirected) is called \term{Eulerian} if it contains every arc of the graph exactly once, that is, the path induces an ordering of the arcs. A graph with an Eulerian cycle is called an \term{Eulerian graph}.

    \thmitem{def:graph_paths/hamiltonian_path}\mcite[ch. 8, sec. 3.1]{GondranMinoux1984Graphs}A simple path (either directed or undirected) is called \term{Hamiltonian} if it contains every vertex of the graph, that is, it induces an ordering of the vertices. A graph with a Hamiltonian cycle is called a \term{Hamiltonian graph}.
  \end{thmenum}
\end{definition}

\begin{definition}\label{def:graph_incidence}
  Let \( G = (V, E) \) be a directed graph. We define the multi-valued \hyperref[def:multi_valued_function]{functions} with signature \( \pow(V) \multto E \):
  \begin{balign*}
     & w^+(A) \coloneqq \{ (u, v) \in E \colon u \in A \} \\
     & w^-(A) \coloneqq \{ (u, v) \in E \colon v \in A \} \\
     & w(A) \coloneqq w^+(A) \cup w^-(A).
  \end{balign*}

  That is, for a set \( A \) of vertices, \( w^+(A) \) gives us the set of arcs whose head is in \( A \), \( w^-(A) \) gives us the set of arcs whose tail is in \( A \) and \( w(A) \) gives us all arcs with at least one endpoint in \( A \).

  \begin{thmenum}
    \thmitem{def:graph_incidence/incident_arcs} The arc \( e \) is said to be \term{incident} with the vertex \( v \) if \( e \in w(v) \), that is, if \( v \) is an endpoint of \( e \).

    \thmitem{def:graph_incidence/degree}\mcite[ch. 1, sec. 1.4]{GondranMinoux1984Graphs}Given a vertex \( v \), the \term{degree} \( d(v) \) (resp. \term{in-degree} \( d^+(v) \) and \term{out-degree} \( d^-(v) \)) of the vertex is defined as
    \begin{equation*}
      d(v) \coloneqq \card w(v).
    \end{equation*}

    The degree of the graph is then defined as
    \begin{equation*}
      d(G) \coloneqq \max_{v \in V} d(v).
    \end{equation*}
  \end{thmenum}
\end{definition}

\begin{definition}\label{def:graph_connectivity}
  Let \( G = (V, E) \) be a directed graph.

  \begin{thmenum}
    \thmitem{def:graph_connectivity/reachable_vertices}The vertex \( v \) is \term{reachable} from the vertex \( u \) if there exists a directed \hyperref[def:graph_paths/directed_path]{path} from \( u \) to \( v \).

    \thmitem{def:graph_connectivity/strongly_connected_graph}\mcite[ch. 1, sec. 3.5]{GondranMinoux1984Graphs}The graph \( G \) is \term{strongly connected} if every pair of distinct vertices are reachable, that is, if there exists a directed path between every pair of distinct vertices.

    \thmitem{def:graph_connectivity/weakly_connected_graph}\mcite[ch. 1, sec. 3.3]{GondranMinoux1984Graphs}The graph \( G \) is \term{weakly connected} if there exists an undirected path between every pair of distinct vertices.

    \thmitem{def:graph_connectivity/connected_component}\mcite[ch. 1, sec. 3.3 \\ ch. 1, sec. 3.5]{GondranMinoux1984Graphs}The subgraph \( G' \) of \( G \) is a \term{connected component} (resp. \term{strongly connected component}) if it is connected (resp. strongly connected) and there exists no connected (resp. strongly connected) subgraph of \( G \) that properly contains \( G' \).

    \thmitem{def:graph_connectivity/connectivity_number}\mcite[ch. 1, sec. 3.3 \\ ch. 1, sec. 3.5]{GondranMinoux1984Graphs}\( G \) has \term{connectivity number} (resp. \term{strong connectivity number}) \( n \) if it has \( n \) connected (resp. strongly connected) components.

    \thmitem{def:graph_connectivity/cut}\mcite[ch. 1, sec. 3.4]{GondranMinoux1984Graphs}The set \( U \subseteq V \) of vertices is a \term{cut} (resp. \term{directed cut}) if removing \( T \) from the graph would increase the connectivity number (resp. strong connectivity number) of the graph.

    \thmitem{def:graph_connectivity/cocycle}\mcite[ch. 1, sec. 4.4]{GondranMinoux1984Graphs}The set \( F \subseteq E \) of arcs is a \term{cocycle} (resp. \term{cocircuit}) if there exists a set \( U \subseteq V \) of vertices such that \( F = w(T) \) (resp. \( F \in \{ w^+(T), w^-(T) \} \)).
  \end{thmenum}
\end{definition}

\begin{definition}\label{def:graph_adjacency}
  Let \( G = (V, E) \) be an undirected graph.

  \begin{thmenum}
    \thmitem{def:graph_adjacency/clique}\mcite[ch. 1, sec. 1.4]{GondranMinoux1984Graphs}The set \( U \subseteq V \) is called a \term{clique} if all two vertices in \( U \) are adjacent.

    \medskip

    \thmitem{def:graph_adjacency/complete_graph}\mcite[ch. 1, sec. 1.4]{GondranMinoux1984Graphs}If \( V \) itself is a clique, we say that \( G \) is a \term{complete graph}.

    \medskip

    \thmitem{def:graph_adjacency/anticlique}\mcite[120]{Erickson2019}Dually, \( U \subseteq V \) is an \term{anticlique} or \term{independent set} of vertices if no two vertices in \( U \) are adjacent.

    \thmitem{def:graph_adjacency/matching}\mcite[ch. 5, exer. 11]{GondranMinoux1984Graphs}The set \( F \subseteq E \) of arcs is a \term{matching} or \term{independent set} of arcs if no two arcs in \( F \) are adjacent.

    \thmitem{def:graph_adjacency/bipartite_graph}\mcite[7]{GondranMinoux1984Graphs}The graph is called \term{bipartite} if there exists a partition \( \{ A, B \} \) of \( V \) such that both \( A \) and \( B \) are anticliques. We also write \( G = (A, B, E) \).

    If \( G \) is undirected and if for every pair of vertices \( a \in A, b \in B \) there is an arc \( a \to b \), we say that \( G \) is a complete bipartite graph.
  \end{thmenum}
\end{definition}

\begin{definition}\label{def:undirected_graph}
  An \term{undirected graph} is a directed graph \( G = (V, E) \) where \( E \) is a symmetric relation (see \fullref{rem:directed_and_undirected_graphs}). When dealing with undirected graphs, instead of speaking about the arcs \( u \to v \) and \( v \to u \), we speak about the \term{edges} \( \{ u, v \} \). Thus, we can also define an undirected graph to be the tuple \( G = (V, E) \), where
  \begin{itemize}
    \item \( V \) is a set of \term{vertices}.
    \item \( E \subseteq \pow(V) \) is a family of unordered pairs of vertices, that is, singletons and two-element sets.
  \end{itemize}

  Defining undirected graphs as a special case of directed graphs allows us to somewhat unify their study and usage, however we need to keep in mind some remarks:
  \begin{itemize}
    \item All paths are directed and hence we only speak of \term{paths} and \term{cycles}. It is necessary, however, to not allow consecutive arcs to represent the same edge, that is, we must treat paths as sequences of edges rather than sequences of arcs.

          Otherwise, since every edge corresponds to two \enquote{inverse} arcs, for all adjacent vertices \( u \) and \( v \) the path \( (u \to v, v \to u) \) is a cycle and hence all undirected graphs would be cyclic.

    \item The incidence matrix is usually defined as \( I = \{ a_{ve} \}_{v \in V, e \in E} \), where
          \begin{equation*}
            a_{ve} \coloneqq \begin{cases}
              1, & v \text{ is an endpoint of } e \\
              0, & \text{otherwise}
            \end{cases}
          \end{equation*}

    \item The adjacency matrix is symmetric if and only if the graph is undirected.

    \item If \( G \) contains to cycles, we say that it is \term{acyclic}.

    \item The notions of connectedness and strong connectedness coincide. Connected acyclic graphs are called \term{trees}. Undirected acyclic graphs are often called \term{forests} since the connected components are trees.

    \item The in-degrees and out-degrees of vertices coincide with the degree. Also, \( w(A) = w^+(A) = w^-(A) \).
  \end{itemize}
\end{definition}

\begin{remark}\label{rem:directed_graphs_model_theory}
  A \hyperref[def:directed_graph]{directed graph} \( G = (V, E) \) is defined as a set \( V \) with a binary relation \( E \). The following properties of graphs and relations are equivalent:
  \begin{center}
    \begin{tabular}{l | l}
      \hyperref[def:directed_graph]{Directed graphs}           & \hyperref[def:binary_relation]{Binary relations} \\
      \hline
      \hyperref[def:directed_graph/simple]{Simple}             & \hyperref[def:binary_relation/irreflexive]{Irreflexive} \\
      \hyperref[def:undirected_graph]{Undirected}              & \hyperref[def:binary_relation/symmetric]{Symmetric} \\
      \hyperref[def:graph_paths/undirected_path]{Acyclic}      & \hyperref[def:well_founded_relation]{Well-founded} \\
    \end{tabular}
  \end{center}

  Some properties are more subtle. For example, the edges of every \hyperref[def:graph_adjacency/complete_graph]{complete graph} are a \hyperref[def:binary_relation/transitive]{transitive relation} but not vice versa.

  From the point of view of \hyperref[subsec:first_order_models]{model theory}, over a \hyperref[def:first_order_language]{first-order language} with only a single predicate symbol every \hyperref[def:first_order_structure]{structure} is a graph and the \hyperref[def:first_order_homomorphism]{homomorphisms} are precisely the homomorphisms of these graphs. Such theories are the theory of \hyperref[def:preordered_set]{preordered sets} and \hyperref[def:zfc]{\logic{ZFC}}.

  For example, every \hyperref[def:partially_ordered_set/strict]{strictly partially ordered set} \( (\mscrP, <) \) is a simple directed graph but not vice versa.

  More interestingly, if \( \mscrV = (V, I) \) is a model of \hyperref[def:zfc]{\logic{ZFC}}, then \( \mscrV \) is \hyperref[def:graph_paths/undirected_path]{acyclic} as a consequence of the \hyperref[def:zfc/foundation]{axiom of foundation} (via \fullref{thm:set_membership_is_well_founded}). In particular, all \hyperref[def:grothendieck_universe]{Grothendieck universes} and \( V_\kappa \) for any \hyperref[def:regular_cardinal]{strongly inaccessible cardinal} \( \kappa \) are graphs (as consequences of \fullref{thm:grothendieck_universe_is_model_of_zfc} and \fullref{thm:cumulative_hierarchy_model_of_zfc}, respectfully).

  More generally, \( (A, \in) \) is a graph for every set \( A \). The edge relation is simply the membership relation restricted to \( A \). From the equivalences in the table above it follows that any \hyperref[def:ordinal]{ordinal} is an acyclic graph since it is well-founded by definition even without the axiom of foundation.
\end{remark}
