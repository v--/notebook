\subsection{Graphs}\label{subsec:graphs}

\begin{remark}\label{rem:directed_and_undirected_graphs}
  Unfortunately, the word \enquote{graph} has at least three popular meanings within mathematics:
  \begin{itemize}
    \item The graph of a function --- \fullref{def:multi_valued_function/graph}.
    \item A directed graph --- \fullref{def:graph/directed}.
    \item An undirected graph --- \fullref{def:graph/undirected}.
  \end{itemize}

  Graphs of functions are different enough from the other two notions to not cause any confusion, however it is often not clear from the context whether \enquote{graph} refers to directed or undirected graphs. Both are formalisms corresponding to dots in the plane connected with (directed or undirected) lines, as illustrated in \fullref{ex:def:graph/directed}. Furthermore, even if it is clear whether the graph is directed or undirected, it is often unclear whether it is \hyperref[def:graph/order]{finite}, whether it is \hyperref[def:graph/simple]{simple} and whether it is a \hyperref[def:multigraph]{multigraph} or not. We try to be as precise as possible to avoid confusion. It doesn't help that the terminology itself is inconsistent as can be seen discussed through the definitions.

  We define undirected graphs as a special case of directed graphs. This approach does make some definitions more awkward, however we prefer it over defining the two differently. Furthermore, software implementations of undirected graphs as a special case of directed graphs are often more versatile --- see \cite[sec. 5.4]{Erickson2019} and \cite[ch. 1, sec. 2.4]{GondranMinoux1984Graphs}.
\end{remark}

\begin{definition}\label{def:graph}\mcite[ch. 1, sec. 1.1]{GondranMinoux1984Graphs}
  We will define, in parallel, directed and undirected graphs. Fix an arbitrary set \( V \).

  \begin{thmenum}[series=def:graph]
    \thmitem{def:graph/vertices} We will call the elements of \( V \) \term{vertices} or \term{nodes}.

    \thmitem{def:graph/directed} A \term{directed graph} or \term{digraph} is a pair \( G = (V, E) \), where \( E \subseteq V^2 \) is a \hyperref[def:binary_relation]{binary relation} on \( V \).

    The elements of \( E \) are called \term{arcs} or \term{arrows}. For every arc \( e = (u, v) \), the vertex \( u = \head(e) \) is called its \term{head} and \( v = \tail(e) \) is called its \term{tail}.

    For readability, we sometimes use the notation \( u \to v \) rather than \( (u, v) \).

    \thmitem{def:graph/adjacency} If \( (u, v) \in E \), we say that \( u \) and \( v \) are \term{neighbors}, that they are \term{adjacent}, that \( v \) is a \term{successor} of \( u \) and that \( u \) is a \term{predecessor} of \( v \). These terms are used in, among other places, \cite[sec. 5.2]{Erickson2019}. Both \( u \) and \( v \) are called \term{endpoints} of \( e \), and the head and tail are also called the \term{starting}/\term{initial} endpoint and \term{final}/\term{terminal} endpoint. The latter terms are introduced in \cite[ch. 1, sec. 1.1]{GondranMinoux1984Graphs}.

    We say that two \hi{arcs} are \term{adjacent} if they have a common endpoint. Note that by this definition, two initial endpoints may be adjacent: \( u \to v \) is adjacent to \( v \to w \) as well as to \( u \to w \).

    If every pair of vertices are adjacent, we say that \( G \) is a \term{complete graph}.

    \thmitem{def:graph/undirected} An \term{undirected graph} is a family of unordered pairs \( \set{ u, v } \) for some \( u, v \in V \) (or unary sets in the case \( u = v \)). Each unordered pair is called an \term{edge} and its members are called \term{endpoints}. Unlike arcs in directed graphs, edges have no head or tail.

    Edges are different from arcs, however it will be convenient for us to define an undirected graph as a directed graph \( G = (V, E) \), in which the relation \( E \) is \hyperref[def:binary_relation/symmetric]{symmetric}. We will thus regard each edge as either an ordered or unordered pair depending on the context. It may be confusing, but this is the adopted convention.

    We should note that some authors use the terms \term{arc} and \term{edge} as per our definitions. For example, arcs are introduced in \cite[ch. 1, sec. 1.1]{GondranMinoux1984Graphs} and edges a bit later in \cite[ch. 1, sec. 1.3]{GondranMinoux1984Graphs}. Other authors, e.g. \cite[sec. 5.2]{Erickson2019}, use the term \enquote{edge} to mean both an arc and an edge depending on whether the graph is directed or not.

    \thmitem{def:graph/symmetrization} Let \( G = (V, E) \) be a directed graph. The undirected graph \( \sym(G) \coloneqq (V, \cl^S(E)) \), whose set of edges is the \hyperref[def:relation_closures/symmetric]{symmetric closure} of \( E \), is called the \term{symmetrization} of \( G \).

    \thmitem{def:graph/order} The \term{order} \( \ord(G) \) of a (directed or undirected) graph is the \hyperref[thm:cardinality_existence]{cardinality} of \( V \). Without context, we usually assume that the set \( V \) of vertices is nonempty and \hyperref[def:set_finiteness]{finite}, however it is sometimes beneficial to consider empty or infinite graphs and hence we define \( \ord(G) \) to be a general \hyperref[def:cardinal]{cardinal number} rather than a \hyperref[rem:peano_arithmetic_zero/nonnegative]{nonnegative integer}.

    If the order \( \ord(G) \) is finite (resp. infinite), we say that \( G \) is a finite (resp. infinite) graph. See also local finiteness defined in \fullref{def:graph_incidence/degree}.

    \thmitem{def:graph/simple} An arc whose head and tail are equal is called a \term{loop}. An assumption that is often implicitly made is that a graph does not have loops.

    To make things more explicit, we call such graphs \term{simple}. This terminology is used in \mcite[ch. 1, sec. 1.3]{GondranMinoux1984Graphs}, for example.

    This condition corresponds to the \hyperref[def:binary_relation/irreflexive]{irreflexivity} of the relation \( E \).
  \end{thmenum}

  Graphs have the following metamathematical properties:
  \begin{thmenum}[resume=def:graph]
    \thmitem{def:graph/theory} The theory of directed graphs is an empty \hyperref[def:first_order_theory]{first-order theory} over the language consisting of a single infix binary relation \( \to \). For undirected graphs we must add \hyperref[def:binary_relation/symmetric]{symmetry} as the only axiom.

    See \fullref{rem:well_founded_graphs} for how graphs are related to \hyperref[def:zfc]{Zermelo-Fraenkel set theory}.

    \thmitem{def:graph/directed/homomorphism} A \hyperref[def:first_order_homomorphism]{first-order homomorphism} between two graphs \( G_1 = (V_1, E_1) \) and \( G_2 = (V_2, E_2) \) is a function \( f: V_1 \to V_2 \) between their universes such that for every arc \( a \to b \) of \( G_1 \), \( f(a) \to f(b) \) is an arc of \( G_2 \).

    It is a weak homomorphism in the sense of \fullref{def:first_order_homomorphism}.

    Homomorphisms preserve symmetry, so this definition works without modification for undirected graphs.

    The term \enquote{graph embedding} usually refers to (topological) embeddings of the \hyperref[def:graph_geometric_realization]{geometric realization} of a graph rather than (combinatorial) embeddings the graph.

    \thmitem{def:graph/submodel} As for preordered sets, any subset \( V' \subseteq V \) of vertices induces a directed graph with the \( E \) relation restricted to \( V' \). That is, \( G' = (V', E') \) is a \term{subgraph} of \( G \) if \( V' \subseteq V \) and
    \begin{equation*}
      E' = \set{ e \in E \given \head(e) \in U \T{and} \tail(e) \in U }.
    \end{equation*}

    Note that any subgraph of an undirected graph is again an undirected graph.

    \thmitem{def:graph/trivial} Unlike the \hyperref[def:group/trivial]{trivial group} \( \set{ e } \) or \hyperref[def:partially_ordered_set/trivial]{empty ordered set}, which are unique up to an isomorphism, there is no single agreed upon graph called the \enquote{trivial graph}.

    An unambiguous concept is that of an \term{edgeless graph}, in which the set of arcs/edges is empty, but the set of vertices may or may not be empty. Every graph \( G = (V, E) \) has exactly \( 2^{\ord(G)} \) edgeless subgraphs (one for each subset of \( V \)).

    The bottom of the \hyperref[thm:substructures_form_complete_lattice]{lattice of submodels} of \( G \) is the \term{order-zero graph} \( (\varnothing, \varnothing) \). The order-zero graph is unique.

    The terms \term{empty graph}, \term{null graph} and \term{trivial graph} may refer to either edgeless graphs or the order-zero graph depending on the author and the situation.

    \thmitem{def:graph/category} We denote the \hyperref[def:category_of_first_order_models]{categories of models} by \( \cat{DGraph} \) for directed graphs and by \( \cat{UGraph} \) for undirected graphs.
  \end{thmenum}
\end{definition}

\begin{example}\label{ex:def:graph}
  Consider the directed graph
  \begin{equation}\label{eq:ex:def:graph/directed}
    \begin{aligned}
      \includegraphics{figures/eq__ex__def__graph__directed.pdf}
    \end{aligned}
  \end{equation}
  and its \hyperref[def:graph/symmetrization]{symmetrization}, the undirected graph.
  \begin{equation}\label{eq:ex:def:graph/undirected}
    \begin{aligned}
      \includegraphics{figures/eq__ex__def__graph__undirected.pdf}
    \end{aligned}
  \end{equation}

  We will use both graphs as generic examples. The few basic properties we can list now is that the graphs have finite order \( 6 \) and that they are simple graphs since they contain no loops. Note also that the edges are numbered in \hyperref[eq:def:lexicographic_order]{lexicographic order}. We will come back to this example later with more insightful comments.

  It should be noted that when defining a concrete graph, it is impractical to enumerate the vertices and edges. It is instead more understandable to work with a \hyperref[def:graph_geometric_realization/drawing]{graph drawing} like it is done here.
\end{example}

\begin{remark}\label{rem:graphs_linear_algebra_and_topology}
  As we shall see, the \hyperref[def:graph]{adjacency} and \hyperref[def:graph_incidence]{incidence} of a graph can be easily studied using linear algebra via \hyperref[def:graph_matrices/adjacency]{adjacency} and \hyperref[def:graph_matrices/incidence]{incidence matrices}, while the \hyperref[def:graph_connectedness]{connectedness} of a graph can be studied using \hyperref[def:graph_connectedness]{topology} via \hyperref[def:graph_geometric_relization/embedding]{graph embeddings}.
\end{remark}

\begin{definition}\label{def:graph_incidence}
  Let \( G = (V, E) \) be a directed graph. We define several \hyperref[def:multi_valued_function]{multi-valued functions} from \( \pow(V) \) to \( E \):
  \begin{align*}
     w^+(A) &\coloneqq \set{ (u, v) \in E \given u \in A } \\
     w^-(A) &\coloneqq \set{ (u, v) \in E \given v \in A } \\
     w(A)   &\coloneqq w^+(A) \cup w^-(A).
  \end{align*}

  For a set \( A \) of vertices, \( w^+(A) \) gives us the set of arcs whose head is in \( A \), \( w^-(A) \) gives us the set of arcs whose tail is in \( A \) and \( w(A) \) gives us all arcs with at least one endpoint in \( A \).

  This notation is used in \cite[ch. 1, sec. 1.4]{GondranMinoux1984Graphs}. It appears not to be very conventional, and we will avoid as much as possible.

  In a simple directed graph, \( w^+(v) \) is disjoint from \( w^-(v) \) for every individual vertex \( v \). In an undirected graph \( G \), we have \( w(A) = w^+(A) = w^-(A) \) for every set \( A \) of vertices.

  \begin{thmenum}
    \thmitem{def:graph_incidence/incident_arcs} The arc \( e \) is said to be \term{incident} with the vertex \( v \) if \( e \in w(v) \). That is, if \( v \) is an endpoint of \( e \).

    \thmitem{def:graph_incidence/degree}\mcite[ch. 1, sec. 1.4]{GondranMinoux1984Graphs} The \term{degree} of a vertex \( v \) is defined as
    \begin{equation*}
      \deg(v) \coloneqq \card w(v).
    \end{equation*}

    The \term{in-degree} \( \deg^+(v) \) and \term{out-degree} \( \deg^-(v) \) are defined in an obvious way.

    The degree of the graph itself is then defined as
    \begin{equation*}
      \deg(G) \coloneqq \max_{v \in V} d(v).
    \end{equation*}

    It is possible that the maximum in \( \deg(G) \) is not attained if \( G \) is infinite. If \( G \) is infinite but \( \deg(G) \) is finite, we say that the graph \( G \) is \term{locally finite}.
  \end{thmenum}
\end{definition}

\begin{example}\label{ex:def:graph_incidence}
  In the directed graph \eqref{eq:ex:def:graph/directed}, we have
  \begin{align*}
    w^+(b) &= \set{ c, e } \\
    w^-(b) &= \set{ a } \\
    w(b)   &= \set{ a, c, e }
  \end{align*}

  The sets \( w^+(b) \) and \( w^-(b) \) are disjoint since the graph is simple. In the \hyperref[def:graph/symmetrization]{symmetrization} \eqref{eq:ex:def:graph/undirected} of \eqref{eq:ex:def:graph/directed}, all three sets are equal to \( w(b) = \set{ a, c, e } \).

  For both graphs
  \begin{equation*}
    \deg(G) = \max\set{ w(v) \given v \in V } = \deg w(b) = 3.
  \end{equation*}
\end{example}

\begin{example}\label{ex:tower_diagram_graph}
  A very simple example of an infinite graph is the \hyperref[def:relation_closures/transitive]{transitive reduction} of the nonnegative integers, i.e. the graph
  \begin{equation}\label{eq:ex:tower_diagram_graph/forward}
    \begin{aligned}
      \includegraphics{figures/eq__ex__tower_diagram_graph__forward.pdf}
    \end{aligned}
  \end{equation}

  Since the graph is simple, we have
  \begin{equation*}
    \deg(n) = \deg^-(n) + \deg^+(n) = \begin{cases}
      1 = 0 + 1, &n = 0, \\
      2 = 1 + 1, &n > 0 \\
    \end{cases}
  \end{equation*}
  and thus \( \deg(G) = 2 \).

  We call the corresponding \hyperref[def:categorical_diagram]{categorical diagram} a \term{tower diagram} --- see \fullref{def:tower_diagram}.

  Another related graph is based on the nonpositive integers:
  \begin{equation}\label{eq:ex:tower_diagram_graph/backward}
    \begin{aligned}
      \includegraphics{figures/eq__ex__tower_diagram_graph__backward.pdf}
    \end{aligned}
  \end{equation}

  Finally, the union of the two gives us a two-sided tower:
  \begin{equation}\label{eq:ex:tower_diagram_graph/two_sided}
    \begin{aligned}
      \includegraphics{figures/eq__ex__tower_diagram_graph__two_sided.pdf}
    \end{aligned}
  \end{equation}

  All three graphs are infinite but locally finite.
\end{example}

\begin{definition}\label{def:graph_matrices}
  Let \( G = (V, E) \) be a finite simple graph, either directed or undirected.
  \begin{thmenum}
    \thmitem{def:graph_matrices/incidence}\mcite[ch. 1, sec. 2.1]{GondranMinoux1984Graphs} We define the \term{incidence matrix} \( I = \seq{ a_{ve} }_{v \in V, e \in E} \) for \( G \).

    \begin{minipage}[t]{0.45\textwidth}
      For a directed graph, \( I \) has elements
      \begin{equation*}
        a_{ve} \coloneqq \begin{cases}
          1,  & v = \head(e) \\
          -1, & v = \tail(e) \\
          0,  & \T{otherwise.}
        \end{cases}
      \end{equation*}
    \end{minipage}
    \hspace{0.02\textwidth}
    \begin{minipage}[t]{0.45\textwidth}
      For an undirected graph, \( I \) has elements
      \begin{equation*}
        a_{ve} \coloneqq \begin{cases}
          1,  & v \in e \\
          0,  & \T{otherwise.}
        \end{cases}
      \end{equation*}
    \end{minipage}

    The two definitions are quite different but in both cases, the matrix element \( a_{ve} \) is nonzero precisely when \( v \) is an endpoint of \( e \).

    \thmitem{def:graph_matrices/adjacency}\mcite[ch. 1, sec. 2.3]{GondranMinoux1984Graphs} The \term{adjacency matrix} \( I = \seq{ a_{uv} }_{u, v \in V} \) for \( G \) has elements
    \begin{equation*}
      a_{uv} \coloneqq \begin{cases}
        1,  & (u, v) \in E \\
        0,  & \T{otherwise.}
      \end{cases}
    \end{equation*}

    Unlike the incidence matrix, the adjacency matrix regards undirected graphs as special cases of directed graphs. See also \fullref{thm:graph_undirected_iff_adjacency_matrix_is_symmetric}.
  \end{thmenum}
\end{definition}

\begin{example}\label{ex:def:graph_matrices}
  The \hyperref[def:graph_matrices/incidence]{incidence matrix} corresponding to the directed graph \eqref{eq:ex:def:graph/directed} is
  \begin{equation}\label{ex:def:graph_matrices/incidence}
    \begin{blockarray}{cccccccc}
        & 1         & 2         & 3         & 4         & 5         & 6         & 7         \\
      \begin{block}{c(ccccccc)}
      a & 1         & 1         &           &           &           &           &           \\
      b & \fbox{-1} &           & 1         & 1         &           &           &           \\
      c &           &           & \fbox{-1} &           & 1         &           &           \\
      d &           & \fbox{-1} &           &           &           & 1         &           \\
      e &           &           &           & \fbox{-1} &           & \fbox{-1} & 1         \\
      f &           &           &           &           & \fbox{-1} &           & \fbox{-1} \\
      \end{block}
    \end{blockarray}
  \end{equation}

  It can be read column-by-column. Every column contains exactly two nonzero elements whose rows correspond to the head (positive) and tail (negative).

  To obtain the incidence matrix for the \hyperref[def:graph/symmetrization]{symmetrization} \eqref{eq:ex:def:graph/undirected} of \eqref{eq:ex:def:graph/directed}, we need to simply flip the sign of the boxed elements above.

  The \hyperref[def:graph_matrices/adjacency]{adjacency matrix} is
  \begin{equation}\label{ex:def:graph_matrices/adjacency}
    \begin{blockarray}{cccccccc}
        & a        & b        & c        & d        & e        & f \\
    \begin{block}{c(ccccccc)}
      a &          & 1        &          & 1        &          &   \\
      b & \fbox{1} &          & 1        &          & 1        &   \\
      c &          & \fbox{1} &          &          &          & 1 \\
      d & \fbox{1} &          &          &          & 1        &   \\
      e &          & \fbox{1} &          & \fbox{1} &          & 1 \\
      f &          &          & \fbox{1} &          & \fbox{1} &   \\
    \end{block}
    \end{blockarray}
  \end{equation}
  where the boxed elements are nonzero only in the adjacency matrix for the undirected graph.

  The matrix can be read either column-by-column or row-by-row.
  \begin{itemize}
    \item The \( v \)-th column lists the vertices \( u \) such that there is an edge \( u \to v \).
    \item The \( u \)-th row lists the vertices \( v \) such that there is an edge \( u \to v \).
  \end{itemize}
\end{example}

\begin{proposition}\label{thm:graph_undirected_iff_adjacency_matrix_is_symmetric}
  A \hyperref[def:graph/directed]{directed graph} is \hyperref[def:graph/directed]{undirected} if and only if its \hyperref[def:graph_matrices/adjacency]{adjacency matrix} is \hyperref[def:symmetric_matrix]{symmetric}.
\end{proposition}
\begin{proof}
  Trivial.
\end{proof}

\begin{definition}\label{def:graph_spaces}
  Let \( G = (V, E) \) be a finite graph. We introduce several \hyperref[def:vector_space]{vector spaces} over \( G \) that allow us to study graphs using linear algebra.

  \begin{thmenum}
    \thmitem{def:graph_spaces/vertex} The \term{vertex space} \( \BbbF_2^V \) is the \hyperref[def:left_module_of_tuples]{tuple vector space} of dimension \( \card(V) = \ord(G) \) over the finite \hyperref[def:field]{field} \hyperref[thm:f2_is_boolean_algebra]{\( \BbbF_2 \)}.

    Every subset \( A \subseteq V \) of vertices induces a unique vector \( \vect{A} = \seq{ \vect{A}_v }_{v \in V} \) in the \hyperref[def:graph_spaces/vertex]{vertex space} \( \BbbF_2^V \) such that
    \begin{equation*}
      \vect{A}_v \coloneqq \begin{cases}
        1, &v \in A \\
        0, &v \not\in A
      \end{cases}
    \end{equation*}

    This vector is called the \term{characteristic vector} of \( A \). Conversely, every vector in \( \BbbF_2^V \) induces a set of vertices. If \( A \) consists of a single vertex \( u \), we write \( \vect{u} \) rather than \( \vect{\set{u}} \).

    It is important to note that, in accordance with \fullref{def:function/set_of_functions}, \( \BbbF_2^V \) is the set of all functions from \( V \) to \( \BbbF_2 \). Since we have assumed that the graph is finite, this space is isomorphic to the tuple space \( \BbbF_2^{\ord(G)} \). Unfortunately, this isomorphism is not unique since it does not give us a choice of ordering of \( V \). Thus, we cannot regard the vectors of \( \BbbF_2^V \) as ordered tuples unless \( V \) is \hyperref[def:well_ordered_set]{well-ordered}.

    \thmitem{def:graph_spaces/arc} Analogously, the \term{arc space} \( \BbbF_2^E \) is the \hyperref[def:left_module_of_tuples]{tuple vector space} of dimension \( \card(E) \). Every subset of \( E \) induces a unique characteristic vector in the \hyperref[def:graph_spaces/arc]{arc space} \( \BbbF_2^E \) and vice versa. The space is motivated by \hyperref[def:graph_directed_path]{directed paths} and their \hyperref[def:graph_directed_path/characteristic_vector]{characteristic vectors}.

    When dealing with undirected graphs, we call \( \BbbF_2^E \) the \term{edge space}.

    \thmitem{def:graph_spaces/oriented_arc} Finally, we introduce the more complicated \term{oriented arc space} \( \BbbF_3^E \). It is sometimes not sufficient to consider only a subset \( A \subseteq E \) but rather a pair of subsets \( A, B \subseteq E \) such that the arcs in \( A \) are \enquote{positively oriented} and \( B \) are \enquote{negatively oriented}. The terms \enquote{positively oriented} and \enquote{negatively oriented} are informal only have meaning in certain applications. The space is motivated by \hyperref[def:graph_adjacency_chain]{adjacency chains} and their \hyperref[def:graph_adjacency_chain/characteristic_vector]{characteristic vectors}.

    The characteristic vector \( x \) of a pair \( (A, B) \) of subsets has components
    \begin{equation*}
      \vect{(A, B)}_e \coloneqq \begin{cases}
        1,  &e \in A \\
        -1, &e \in B \\
        0,  &\T{otherwise.}
      \end{cases}
    \end{equation*}

    As for the other vector spaces, every pair \( (A, B) \) has a characteristic vector in \( \BbbF_3^E \) and every vector in \( \BbbF_3^E \) corresponds to a pair of subsets of \( E \).
  \end{thmenum}
\end{definition}

\begin{proposition}\label{thm:graphs_as_linear_transformations}
  Let \( V \) be a \hyperref[def:set_finiteness]{finite set}. Then there is a bijection between the \hyperref[def:graph/directed]{directed graphs} over \( V \) and the \hyperref[def:linear_operator]{linear} \hyperref[def:endomorphism]{endomorphisms} over \( \BbbF_2 \).
\end{proposition}
\begin{proof}
  Let \( G = (V, E) \) be a directed graph and let \( I \) be its \hyperref[def:graph_matrices/adjacency]{adjacency matrix}. Then \( I \) induces a linear endomorphism in \( \BbbF_2^n \).

  Conversely, let \( T: \BbbF_2^n \to \BbbF_2^n \) be a linear endomorphism. Define the set
  \begin{equation*}
    E \coloneqq \set{ (u, v) \in V^2 \given T(\vect{v})_u = 1 }.
  \end{equation*}

  Then the adjacency matrix of \( G = (V, E) \) induces \( T \).
\end{proof}

\begin{example}\label{ex:thm:graphs_as_linear_transformations}
  We will empirically verify the proof of \fullref{thm:graphs_as_linear_transformations}.

  Denote by \( A \) the adjacency matrix \eqref{ex:def:graph_matrices/adjacency} of \eqref{eq:ex:def:graph/directed} discussed in \fullref{ex:def:graph_matrices}. Consider the vertex \( b \). Its characteristic vector is \( \vect{b} = (0, 1, 0, 0, 0, 0) \). Thus, the matrix product \( A \vect{b} \) simply \enquote{selects} the \( b \)-th column of \( I \), which is
  \begin{equation*}
    A \vect{b}
    =
    \begin{blockarray}{cccccrl}
      a        & b        & c        & d        & e        & f \\
    \begin{block}{(cccccr)l}
      1        & 0        & 0        & 0        & 0        & 0 \\
    \end{block}
    \end{blockarray}
    {}^T
  \end{equation*}

  The only nonzero member of \( A \vect{b} \) is the one corresponding to the vertex \( a \). This is consistent with \( a \) being the only predecessor of \( b \) in \eqref{eq:ex:def:graph/directed}.
\end{example}

\begin{definition}\label{def:graph_adjacency_chain}
  An \term{adjacency chain} in a directed graph \( G = (V, E) \) is a finite or infinite sequence of arcs
  \begin{equation}\label{eq:def:graph_adjacency_chain}
    p \coloneqq (e_1, e_2, \ldots)
  \end{equation}
  such that every two consecutive arcs are adjacent. That is, \( e_{k+1} \) is adjacent to \( e_k \) for every \( k = 1, 2, \ldots \).

  Note that every two arcs of \( p \) should only be adjacent, hence \( (u \to v, u \to w) \) is an adjacency chain. See \fullref{def:graph_directed_path} for the more intuitive notion of a path.

  The term \enquote{chain} is used in this sense in \cite[ch. 1, sec. 3.1]{GondranMinoux1984Graphs}, for example, but in general it may refer to a path or simple path. We add the adverb \enquote{adjacency} in order to reduce ambiguity.

  \begin{thmenum}
    \thmitem{def:graph_adjacency_chain/domain} The \term{domain} \( \dom(p) \) of \( p \) is the set of all vertices that belong to at least one arc in \( p \). We say that the path \term{visits} each member of \( \dom(p) \).

    \thmitem{def:graph_adjacency_chain/length} The \term{length} of \( p \) is defined in an obvious way for finite paths and as \( p = \infty \) for infinite paths (here \( \infty \) is merely a more conventional symbol for denoting \hyperref[thm:omega_is_a_cardinal]{\( \aleph_0 \)}).

    Note that
    \begin{equation}\label{eq:def:graph_adjacency_chain/length_and_domain}
      \card(\dom(p)) + 1 \leq \len(p)
    \end{equation}
    in general since a vertex can be an endpoint of many arcs.

    \thmitem{def:graph_adjacency_chain/subchain} We say that the \hyperref[def:subsequence]{subsequence} \( q \) of \( p \) is a \term{subchain} of \( p \) if the elements of \( q \) are consecutive in \( p \), i.e. there exists some index \( n \geq 0 \) such that \( q_k = p_{n + k} \) for \( 0 < k < \len(q) \).

    \thmitem{def:graph_adjacency_chain/endpoints} Similarly to arcs, chains have a \term{head} and \term{tail}, although both may not exist. Roughly, the head of \( p = (e_1, e_2, \ldots) \) is the endpoint of \( e_1 \) that is not an endpoint of \( e_2 \). After considering some edge cases, this definition becomes
    \begin{equation}\label{eq:def:graph_adjacency_chain/endpoints/head}
      \head(p) \coloneqq \begin{cases}
        \T{undefined}, &p = \varnothing, \\
        u,             &p = (u \to v), \\
        u,             &p = (u \to v, v \to u, \cdots), \\
        v,             &v \T{is an endpoint of} e_1 \T{but not of} e_2.
      \end{cases}
    \end{equation}

    Compare this to the much simpler \eqref{eq:def:graph_directed_path/endpoints/head}.

    The \term{tail} is defined similarly but with the caveat that an infinite chain cannot have a tail.

    The head and the tail are collectively called the \term{endpoints} of a path.

    \thmitem{def:graph_adjacency_chain/simple} The chain \( p \) is \term{simple} if every vertex in \( \dom(p) \) is visited only once. That is, within the subgraph whose arcs are those in \( p \), the \hyperref[def:graph_incidence/degree]{degree} of every vertex is at most \( 2 \).

    An alternative characterization is that equality holds in \eqref{eq:def:graph_adjacency_chain/length_and_domain}.

    \thmitem{def:graph_adjacency_chain/converse} If the chain \eqref{eq:def:graph_adjacency_chain} is finite of length \( n \), we define its \term{converse} as
    \begin{equation*}
      p^{-1} \coloneqq (e_n, e_{n-1}, \cdots, e_2, e_1).
    \end{equation*}

    \thmitem{def:graph_adjacency_chain/orientation} We say that an arc is \term{positively oriented} if it is either the first arc \( e_1 \) if \( \head(e_1) = \head(p) \) or the \hyperref[def:graph/adjacency]{successor} of a positively oriented arc.

    More precisely, we can recursively define the \hyperref[def:boolean_function]{predicate} \( \op{IsPos} \) determining whether the arc at position \( k \) is \term{positively oriented}:
    \begin{equation*}
      \begin{aligned}
        \op{IsPos}(k) \coloneqq \begin{cases}
          T,                            &k = 1 \T{and} \head(e_1) = \head(p) \\
          F,                            &k = 1 \T{and} \head(e_1) \neq \head(p) \\
          \op{IsPos}(k - 1),            &k > 1 \T{and} \head(e_k) = \tail(e_k) \\
          \overline{\op{IsPos}(k - 1)}, &k > 1 \T{and} \head(e_k) \neq \tail(e_k)
        \end{cases}
      \end{aligned}
    \end{equation*}

    All arcs that are not positively oriented are said to be \term{negatively oriented}.

    \thmitem{def:graph_adjacency_chain/characteristic_vector} The \term{characteristic vector} \( \vect{p} \) of a chain \( p \) is defined as the characteristic vector \( \vect{(P, N)} \) in the \hyperref[def:graph_spaces/oriented_arc]{oriented arc space} \( \BbbF_3^E \), where \( P \) is the set of positively oriented arcs and \( N \) is the set of negatively oriented arcs.

    Conversely, every vector in \( \BbbF_3^E \) identifies a path.
  \end{thmenum}
\end{definition}

\begin{definition}\label{def:graph_directed_path}
  A \term{directed path} or simply \term{path} in a directed graph \( G = (V, E) \) is an \hyperref[def:graph_adjacency_chain]{adjacency chain} without negatively oriented arcs. In other words, a directed path is a finite or infinite sequence \eqref{eq:def:graph_adjacency_chain} such that each arc is a \hyperref[def:graph/adjacency]{successor} of the preceding one. That is, the head \( e_{k+1} \) is the tail of \( e_k \) for every \( k = 1, 2, \ldots \).

  The definitions of \hyperref[def:graph_adjacency_chain/domain]{domain} \( \dom(p) \), \hyperref[def:graph_adjacency_chain/length]{length} and \( \len(p) \) are inherited from chains. \hyperref[def:graph_adjacency_chain/subchain]{Subchains} of a path are called \term{subpaths} and a path is called \term{simple} if it is a \hyperref[def:graph_adjacency_chain/simple]{simple chain}.

  This is one of the definitions where no adaptations are needed for undirected graphs --- we simply use that we have defined undirected graphs as a special case of directed graphs.

  This terminology is used in \mcite[ch. 1, sec. 3.2]{GondranMinoux1984Graphs}, however in \cite[sec. 5.2]{Erickson2019} this is instead called a \term{walk} and a path is defined as a simple walk (\enquote{simple} in the sense of \fullref{def:graph_adjacency_chain/simple}). The terms \term{trail} and \term{tour} also have related meaning, but it is unfortunately ambiguous, depending on the context. We sometimes use \enquote{directed path} in order to reduce ambiguity.

  \begin{thmenum}
    \thmitem{def:graph_directed_path/endpoints} The definition of head and tail of a directed path inherits those of adjacency chains, however they are so much simpler that we give them explicitly. Instead of \eqref{eq:def:graph_adjacency_chain/endpoints/head} we have
    \begin{equation}\label{eq:def:graph_directed_path/endpoints/head}
      \head(p) \coloneqq \begin{cases}
        \T{undefined}, &p = \varnothing, \\
        u,             &p = (u \to v, \cdots).
      \end{cases}
    \end{equation}

    Tails are defined analogously.

    \thmitem{def:graph_directed_path/characteristic_vector} The \term{characteristic vector} \( \vect{p} \) of a path \( p \) is defined as the characteristic vector of the domain \( \dom(p) \) in the \hyperref[def:graph_spaces/arc]{arc space} \( \BbbF_2^E \). Conversely, every vector in \( \BbbF_2^E \) identifies a path.

    The vector \( \vect{p} \) is equal to the \hyperref[def:graph_adjacency_chain/characteristic_vector]{characteristic vector} of \( p \) when regarding \( p \) as a chain. It should be noted, however, that we regard \( \vect{p} \) as a member of the arc space \( \BbbF_2^E \) and rather than the oriented arc space \( \BbbF_3^E \).

    \thmitem{def:graph_directed_path/reachability} We say that \( p \) \term{connects} or \term{passes through} the vertices \( u \) and \( v \) if there exists a subpath \( q \) such that \( \head(q) = u \) and \( \tail(p) = v \).

    We say that the vertex \( u \) is \term{reachable} from \( v \) if there exists a path that connects them. Clearly reachability is the \hyperref[def:relation_closures/reflexive]{reflexive} and \hyperref[def:relation_closures/transitive]{transitive} closure of the relation \( E \).

    Furthermore, if the graph is undirected, \( E \) is symmetric and, by \fullref{thm:relation_closures_properties/symmetric_set}, reachability is also symmetric. Thus, in an undirected graph, reachability is an equivalence relation. This is exploited in \fullref{def:graph_connectedness}.

    This definition is generalizable to arbitrary chains, however we prefer not to do it.
  \end{thmenum}
\end{definition}

\begin{example}\label{ex:def:graph_directed_path}
  An example of an infinite graph path is the forward tower \eqref{eq:ex:tower_diagram_graph/forward} regarded as a path of the two-sided tower \eqref{eq:ex:tower_diagram_graph/two_sided}. It is also simple since the degree of the forward tower is \( 2 \).

  Now consider again the graph \eqref{eq:ex:def:graph/directed}. The solid lines in \eqref{eq:ex:def:graph_directed_path} describe a path.
  \begin{equation}\label{eq:ex:def:graph_directed_path}
    \begin{aligned}
      \includegraphics{figures/eq__ex__def__graph_path.pdf}
    \end{aligned}
  \end{equation}

  \begin{itemize}
    \item This path corresponds to the sequence
    \begin{equation*}
      p = \parens{ \underbrace{a \to b}_{e_1}, \underbrace{b \to e}_{e_4}, \underbrace{e \to f}_{e_7} }.
    \end{equation*}

    \item Its \hyperref[def:graph_directed_path/characteristic_vector]{characteristic vector} is
    \begin{equation*}
      \vect{p}
      =
      \begin{blockarray}{cccccrl}
        e_1 & e_2 & e_3 & e_4 & e_5 & e_6 & e_7 \\
      \begin{block}{(cccccr)l}
        1   & 0   & 0   & 1   & 0   & 0   & 1 \\
      \end{block}
      \end{blockarray}
      {}^T.
    \end{equation*}

    Furthermore, \( p \) can be identified from the characteristic vector.

    \item The \hyperref[def:graph_directed_path/endpoints]{endpoints} are \( \head(p) = \head(e_1) = a \) and \( \tail(p) = \tail(e_7) = f \).

    \item It is a \hyperref[def:graph_adjacency_chain/simple]{simple path} because \( \deg(a) = \deg(f) = 1 \) and \( \deg(b) = \deg(e) = 2 \).

    \item Every vertex (ordered alphabetically) is \hyperref[def:graph_directed_path/reachability]{reachable} from the previous except for \( a \) (which has no previous) and \( d \) (which is unreachable from \( c \)).

    \item The \hyperref[def:graph_adjacency_chain/converse]{converse chain}
    \begin{equation*}
      p^{-1} = (e \to f, b \to e, a \to b),
    \end{equation*}
    is not a path since every arc is \hyperref[def:graph_adjacency_chain/orientation]{negatively oriented}.

    \item We can conclude from \eqref{eq:def:graph_adjacency_chain/endpoints/head} that \( \head(p^{-1}) = f \) and \( \tail(p^{-1}) = a \).

    \item Its \hyperref[def:graph_directed_path/characteristic_vector]{characteristic vector} is
    \begin{equation*}
      \vect{p^{-1}}
      =
      \begin{blockarray}{ccccccrl}
        e_1 & e_2 & e_3 & e_4 & e_5 & e_6 & e_7 \\
      \begin{block}{(ccccccr)l}
        -1  & 0   & 0   & -1  & 0   & 0   & -1 \\
      \end{block}
      \end{blockarray}
      {}^T.
    \end{equation*}

    Furthermore, \( p^{-1} \) can be identified from the characteristic vector.
  \end{itemize}
\end{example}

\begin{proposition}\label{thm:graph_chain_symmetrization}
  A sequence of arcs in a directed graph \( G \) is an \hyperref[def:graph_adjacency_chain]{adjacency chain} if and only if it is a path in the \hyperref[def:graph/symmetrization]{symmetrization} \( \sym(G) \).
\end{proposition}
\begin{proof}
  Two arcs are adjacent if and only if one of they have a common endpoint. The endpoints of every arc in \( G \) are the same as in the corresponding edge \( \sym(G) \) --- the only different is in their order, which becomes irrelevant with adjacency.
\end{proof}

\begin{definition}\label{def:graph_cycle}
  A \term{directed cycle}, also called a \term{circuit} in \cite[ch. 1, sec. 1.4]{GondranMinoux1984Graphs}, is a \hyperref[def:graph_directed_path]{path} whose head and tail coincide. It is also called a \term{closed path}.

  The concept of a path transfers transparently from directed to undirected graphs, however an adaptation of the concept of a cycle should be made. An \term{undirected cycle} is a cycle that does not contain opposing arcs, i.e. if \( u \to v \) is in the cycle, \( v \to u \) is not.

  This additional restriction is necessary because in an undirected graph, if \( (u, v) \in E \), then \( (u \to v, v \to u) \) is a directed cycle.

  If a directed (resp. undirected) graph does not contain a directed (resp. undirected) cycle, we say that it is \term{acyclic}. Directed acyclic graphs are commonly abbreviated as \term{DAG}.
\end{definition}

\begin{definition}\label{def:graph_condensation}
  Let \( G = (V, E) \) be a directed graph. Define the binary relation
  \begin{equation*}
    u \sim v \iff u \T{is reachable from} v \T{and} v \T{is reachable from} u.
  \end{equation*}

  It is the \hyperref[def:relation_closures/symmetric]{symmetric closure} of the \hyperref[def:relation_closures/transitive]{transitive closure} and \hyperref[def:relation_closures/reflexive]{reflexive closure} of \( E \). Thus, \( \sim \) is an equivalence relation.

  Define the binary relation \( \widetilde{E} \) on the \hyperref[def:equivalence_relation/quotient]{quotient set} \( \widetilde{V} \coloneqq V / {\sim} \) as
  \begin{equation*}
    ([u], [v]) \in \widetilde{E} \iff u \T{is reachable from} v \T{but not vice versa}.
  \end{equation*}

  It is well-defined because if \( ([u_1], [v_1]) \in \widetilde{E} \), \( u_2 \in [u_1] \) and \( v_2 \in [v_1] \), then by the transitivity of reachability we have that \( v_2 \) is reachable from \( u_2 \) and thus \( ([u_2], [v_2]) \in \widetilde{E} \).

  The directed graph \( \widetilde{G} \coloneqq (\widetilde{V}, \widetilde{E}) \) is called the \term{condensation} of \( G \).
\end{definition}

\begin{example}\label{ex:def:graph_condensation}
  The \hyperref[def:graph_condensation] of the graph \eqref{eq:ex:def:graph/directed} is (isomorphic to) the graph itself. If we add the arc \( f \to a \) to \eqref{eq:ex:def:graph/directed}, the condensation would be an edgeless graph with a single vertex.

  We can add new arcs \( e_8 \) and \( e_9 \) to \eqref{eq:ex:def:graph/directed} to make the example more interesting:
  \begin{equation}\label{eq:ex:def:graph_condensation/uncondensed}
    \begin{aligned}
      \includegraphics{figures/eq__ex__def__graph_condensation__uncondensed.pdf}
    \end{aligned}
  \end{equation}

  Its condensation is
  \begin{equation}\label{eq:ex:def:graph_condensation/condensed}
    \begin{aligned}
      \includegraphics{figures/eq__ex__def__graph_condensation__condensed.pdf}
    \end{aligned}
  \end{equation}

  These are the \hyperref[def:graph_connectedness/strong]{strongly connected components} of \eqref{eq:ex:def:graph_condensation/uncondensed}.
\end{example}

\begin{proposition}\label{thm:undirected_graph_condensation}
  The \hyperref[def:graph_condensation]{condensation} of an undirected graph is \hyperref[def:graph/trivial]{edgeless}.
\end{proposition}
\begin{proof}
  The edge relation \( E \) is symmetric in an undirected graph. By \fullref{thm:relation_closures_properties/symmetric_set}, \hyperref[def:graph_directed_path/reachability]{reachability} is symmetric and thus is actually equal to the relation \( {\sim} \) in \fullref{def:graph_condensation}.

  Therefore, by definition of \( \widetilde{E} \), it is empty.
\end{proof}

\begin{proposition}\label{thm:graph_condensation_is_acyclic_dag}
  The \hyperref[def:graph_condensation]{condensation} of a directed graph is \hyperref[def:graph_cycle]{acyclic}.
\end{proposition}
\begin{proof}
  Let \( \widetilde{G} = (\widetilde{V}, \widetilde{E}) \) be the condensation of \( G = (V, E) \). Aiming at a contradiction, suppose that there exist cosets \( [u] \) and \( [v] \) and a path
  \begin{equation*}
    \parens[\Big]{ [u] \to [w_1], \cdots, [w_n] \to [v] }
  \end{equation*}
  in \( \widetilde{G} \) that connects them. We can easily prove by induction that \( v \) is reachable from \( u \), thus contradicting the definition of \( \widetilde{E} \).

  Therefore, \( \widetilde{G} \) is acyclic.
\end{proof}

\begin{definition}\label{def:graph_connectedness}
  Let \( G = (V, E) \) be a directed graph and \( \widetilde{G} = (\widetilde{V}, \widetilde{E}) \) be its condensed graph.

  \begin{thmenum}
    \thmitem{def:graph_connectedness/strong}\mcite[ch. 1, sec. 3.5]{GondranMinoux1984Graphs} For every coset \( [v] \) in \( \widetilde{V} \), the subgraph \( ([v], E\restr_[v]) \) of \( G \) is called a \term{strongly connected component} of \( G \).

    The \term{strong connectivity number} of \( G \) is the cardinality \( \card(\widetilde{V}) \).

    If \( G \) has only one strongly connected component, we say that it itself is \term{strongly connected}.

    \thmitem{def:graph_connectedness/weak}\mcite[ch. 1, sec. 3.3]{GondranMinoux1984Graphs} Similarly, the subgraph \( G' = (V', E') \) of \( G \) is called a \term{weakly connected component} if the \hyperref[def:graph/symmetrization]{symmetrization} \( \sym(G') \) is a strongly connected component of \( \sym(G) \).

    The \term{weak connectivity number} of \( G \) is the strong connectivity number of \( \sym(G) \).

    If \( G \) has only one weakly connected component, we say that it itself is \term{weakly connected}.
  \end{thmenum}

  Clearly the two concepts are equivalent for undirected graphs, in which case we simply say \enquote{connected component} and \enquote{connectivity number}.
\end{definition}

\begin{proposition}\label{thm:graph_connectedness_via_chains}
  Let \( G = (V, E) \) be a directed graph.

  \begin{thmenum}
    \thmitem{thm:graph_connectedness_via_chains/strong} \( G \) is \hyperref[def:graph_connectedness/strong]{strongly connected} if and only if there exists a \hyperref[def:graph_directed_path]{directed path} connecting every pair of vertices.

    \thmitem{thm:graph_connectedness_via_chains/weak} \( G \) is \hyperref[def:graph_connectedness/weak]{weakly connected} if and only if there exists an \hyperref[def:graph_adjacency_chain]{adjacency chain} connecting every pair of vertices.
  \end{thmenum}
\end{proposition}
\begin{proof}
  Follows from \fullref{thm:graph_chain_symmetrization}.
\end{proof}

\begin{remark}\label{rem:well_founded_graphs}
  We can regard any set \( A \) in the sense of \hyperref[def:zfc]{\logic{ZFC}} as the graph \( (A, \in) \).

  The \hyperref[def:zfc/foundation]{axiom of foundation} (via \fullref{thm:set_membership_is_well_founded}) states that \( (A, \in) \) is well-founded.

  In the terminology if graph theory, this well-foundedness means that there for every vertex \( v \) there exists no infinite \hyperref[def:graph_directed_path]{path} that ends with \( v \).

  This implies that the graph is \hyperref[def:graph_cycle]{acyclic}. For the finite graphs the converse holds, but an infinite graph this is not so. For example, the backward tower \eqref{eq:ex:tower_diagram_graph/backward} is acyclic but not well-founded.
\end{remark}

\begin{definition}\label{def:path_category}
  Let \( G = (V, E) \) be a \hyperref[def:graph/simple]{simple directed graph}. Then the \hyperref[def:relation_closures/transitive]{transitive closure} \( \cl^T(E) \) is a \hyperref[def:partially_ordered_set/strict]{strict partial order} on \( E \). By \fullref{thm:partial_order_category_correspondence}, there exists a unique \hyperref[def:thin_category]{thin} \hyperref[def:skeletal_category]{skeletal} category whose morphisms are the paths in \( G \).

  We will call this category the \term{path category} of \( G \).
\end{definition}

\begin{definition}\label{def:graph_geometric_realization}
  Let \( G = (V, E) \) be a \hyperref[def:graph/directed]{directed graph}. Our goal is to construct a \hyperref[def:topological_space]{topological space} that translates the connectivity properties of \( G \) into their topological equivalents. We do this taking a copy of the interval \( [0, 1] \) for every edge and gluing endpoints where the edges have a common endpoint.

  Consider the \hyperref[def:topological_sum]{topological sum} \( \coprod_{e \in E} [0, 1] \). Define the \hyperref[def:equivalence_relation]{equivalence relation} \( {\sim} \) on this space to hold for \( (x, e_1) \) and \( (y, e_2) \) if any of the following hold:
  \begin{align*}
    &x = 0 \T{and} y = 0 \T{and} h(e_1) = h(e_2) \\
    &x = 0 \T{and} y = 1 \T{and} h(e_1) = t(e_2) \\
    &x = 1 \T{and} y = 0 \T{and} t(e_1) = h(e_2) \\
    &x = 0 \T{and} y = 1 \T{and} t(e_1) = t(e_2) \\
    &x = y \T{and} h(e_1) = t(e_2) \T{and} t(e_1) = h(e_2) \quad\quad (\T{if} G \T{is undirected}).
  \end{align*}

  The \hyperref[def:topological_quotient]{quotient space} \( \coprod_{e \in E} [0, 1] / {\sim} \) is called the \term{geometric realization} of \( G \).

  \begin{thmenum}
    \thmitem{def:graph_geometric_realization/drawing} Although it is not standardized terminology, we will call any \hyperref[def:global_continuity]{continuous function} with domain \( \coprod_{e \in E} [0, 1] / {\sim} \) a \term{graph drawing}.

    \thmitem{def:graph_geometric_realization/embedding} An injective graph drawing is called a \term{graph embedding}.

    \thmitem{def:graph_geometric_realization/linear} If a graph can be embedded into \( \BbbR \), we say that it is \term{linear}.

    \thmitem{def:graph_geometric_realization/planar} If a graph can be embedded into \( \BbbR^2 \), we say that it is \term{planar}.
  \end{thmenum}
\end{definition}

\begin{example}\label{ex:def:graph_geometric_realization}
  We will give a few examples of \hyperref[def:graph_geometric_realization]{graph realizations}.

  \begin{thmenum}
    \thmitem{ex:def:graph_geometric_realization/edgeless} The \hyperref[def:graph_geometric_realization]{geometric realization} of an edgeless graph is the empty topological space.

    \thmitem{ex:def:graph_geometric_realization/tower} Consider the tower \eqref{eq:ex:tower_diagram_graph/forward}. We start with \( \aleph_0 \) copies of \( [0, 1] \) and glue both ends of each of them except for the first. Thus, we obtain (a space homeomorphic to)
    \begin{equation*}
      \bigcup_{k \geq 0} [k, k + 1] = [0, \infty).
    \end{equation*}

    Therefore, \eqref{eq:ex:tower_diagram_graph/forward} is a \hyperref[def:graph_geometric_realization/linear]{linear graph}.

    \thmitem{ex:def:graph_geometric_realization/triangle} The graph with vertices \( V = \set{ a, b, c } \) and arcs \( \set{ \overbrace{a \to b}^{e_1}, \overbrace{b \to c}^{e_2}, \overbrace{c \to a}^{e_3} } \) is more subtle.

    We start with three copies of the interval \( [0, 1] \), depicted in \eqref{eq:ex:def:graph_geometric_realization/triangle/relization} as upward-pointing arrows, and use dashed lines to connect the endpoints that we want to glue together.
    \begin{equation}\label{eq:ex:def:graph_geometric_realization/triangle/relization}
      \begin{aligned}
        \includegraphics{figures/eq__ex__def__graph_geometric_realization__triangle__realization.pdf}
      \end{aligned}
    \end{equation}

    After contracting the dashed lines, we obtain a topological space that can easily be \hyperref[def:graph_geometric_realization/embedding]{embedded} into \( \BbbR^2 \). An obvious embedding corresponds to \enquote{pulling up} \( e_2 \) and \( e_3 \):
    \begin{equation}\label{eq:ex:def:graph_geometric_realization/triangle/embedding}
      \begin{aligned}
        \includegraphics{figures/eq__ex__def__graph_geometric_realization__triangle__embedding.pdf}
      \end{aligned}
    \end{equation}

    This is only one possible embedding of the geometric realization. It is sufficient, however, for proving that the graph is \hyperref[def:graph_geometric_realization/planar]{planar}.

    It is also clear from this example that constructing embeddings of graphs can be a very tedious task for graphs with more than a few arcs.
  \end{thmenum}
\end{example}
