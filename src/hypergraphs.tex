\subsection{Hypergraphs}\label{subsec:hypergraphs}

\begin{definition}\label{def:hypergraph}
  Fix a \hyperref[def:set]{set} \( V \), whose members we will call \term{vertices} or \term{nodes}, and a disjoint from \( V \) set \( E \), whose members we will call \term{hyperedges}.

  A \term{hypergraph} is a tuple \( H = (V, E, \mscrE) \), where \( \mscrE: E \multto V \) is a \hyperref[def:multi_valued_function/total]{total multi-valued function} whose role is to give us a nonempty set of endpoints \( \mscrE(e) \) for each edge \( e \in E \).

  \begin{figure}[!ht]
    \begin{equation}\label{eq:fig:def:hypergraph}
      \begin{aligned}
        \includegraphics[page=1]{output/def__hypergraph.pdf}
      \end{aligned}
    \end{equation}
    \caption{A hypergraph containing four hyperedges of rank two and one of rank four.}\label{fig:def:hypergraph}
  \end{figure}

  \begin{thmenum}
    \thmitem{def:hypergraph/cardinality} The \term{cardinality} of a hyperedge \( e \) is the \hyperref[thm:cardinality_existence]{cardinality}
    \begin{equation*}
      \card(e) \coloneqq \card(\mscrE(e))
    \end{equation*}
    of the set of all vertices of \( e \).

    Hyperedges of rank \( 1 \) are called \term{loops} and hyperedges of rank \( 2 \) are called \term{edges}. Loops are often considered to be edges, especially in \hyperref[def:undirected_multigraph]{multigraphs}.

    \thmitem{def:hypergraph/parallel_hyperedges} We say that the hyperedges \( e \) and \( d \) are parallel if they have the same endpoints.

    \thmitem{def:hypergraph/incidence} We say that the vertex \( v \) and the hyperedge \( e \) are \term{incident} if \( v \) is an endpoint \( e \).

    \thmitem{def:hypergraph/adjacency} We say that \( v \) and \( w \) are \term{adjacent vertices} if there exists an hyperedge \( e \) such that both \( v \) and \( w \) are endpoints of \( e \).

    Similarly, we say that \( e \) and \( d \) are \term{adjacent hyperedges} if they have a common endpoint.

    \thmitem{def:hypergraph/order} The \term{order} \( \ord(H) \) of the hypergraph \( H \) is the \hyperref[thm:cardinality_existence]{cardinality} of \( V \).

    We say that the hypergraph is finite if both \( V \) and \( E \) are finite and infinite otherwise. If no parallel hyperedges are allowed, the hypergraph is finite if \( \ord(H) \) is finite.

    \thmitem{def:hypergraph/degree} The \term{degree} \( \deg(v) \) of a vertex \( v \in V \) is the cardinality of set
    \begin{equation*}
      \set{ e \in E \given v \T{is an endpoint of} e }.
    \end{equation*}

    If \( \deg(v) = 0 \), we say that \( v \) is an \term{isolated vertex}, especially in connection with \fullref{def:quiver_geometric_realization/undirected}.

    The degree \( \deg(H) \) of the hypergraph itself is the maximum of the degrees of all vertices. It is possible that the maximum in \( \deg(H) \) is not attained if \( H \) is infinite.

    We say that the graph is \term{locally finite} if the degree of every vertex is finite.

    This is not to be confused with \hyperref[def:category_size]{locally finite categories}.

    \thmitem{def:hypergraph/submodel} The hypergraph \( H' = (V', E', \mscrE') \) is a is a sub-hypergraph of \( H = (V, E, \mscrE) \) if \( V' \subseteq V \), \( E' \subseteq E \) and \( \mscrE' \) is a \hyperref[def:multi_valued_function/restriction]{restriction} of \( \mscrE \) to \( E' \).

    We say that sub-hypergraph \( H' \) is \term{full} if
    \begin{equation}\label{eq:def:hypergraph/submodel/full}
      E' = \set{ e \in E \given \mscrE(e) \subseteq V' }.
    \end{equation}

    In this case, we also say that \( V' \) \term{induces} the sub-hypergraph \( H' \).

    \thmitem{def:hypergraph/trivial} Unlike the \hyperref[def:group/trivial]{trivial group} \( \set{ e } \) or \hyperref[def:partially_ordered_set/trivial]{empty ordered set}, which are unique up to an isomorphism, there is no single agreed upon graph called the \enquote{trivial hypergraph}.

    An unambiguous concept is that of an \term{edgeless hypergraph}, in which the set of hyperedges is empty, but the set of vertices may or may not be empty. Every hypergraph \( H: E \to V \) has \( 2^{\ord(V)} \) edgeless sub-hypergraphs (one for each subset of \( V \)).

    The \term{order-zero hypergraph} \( H: \varnothing \multto \varnothing \) is the unique hypergraph that is a sub-hypergraph of all others.

    The terms \term{empty hypergraph}, \term{null hypergraph} and \term{trivial hypergraph} may refer to either edgeless graphs or the order-zero graph, depending on the author and the situation.
  \end{thmenum}
\end{definition}

\begin{proposition}\label{thm:family_of_sets_induces_hypergraph}
  Every \hyperref[rem:family_of_sets]{family} of nonempty sets \( \mscrA \) induces a \hyperref[def:hypergraph]{hypergraph} as follows:
  \begin{itemize}
    \item The set of vertices is \( \bigcup \mscrA \).
    \item The set of hyperedges is \( \mscrA \) itself.
  \end{itemize}
\end{proposition}
\begin{proof}
  The hypergraph is given by the function
  \begin{equation*}
    \begin{aligned}
      &H: \mscrA \multto \bigcup \mscrA \\
      &\mscrE(A) \coloneqq A,
    \end{aligned}
  \end{equation*}
  which is better represented as the binary relation
  \begin{equation*}
    \set{ (A, x) \given A \in \mscrA \T{and} x \in A }.
  \end{equation*}
\end{proof}

\begin{example}\label{ex:hypergraphs_not_induced_by_family}
  The hypergraph \eqref{eq:fig:def:hypergraph} is a counterexample to the converse of \fullref{thm:family_of_sets_induces_hypergraph}. It cannot be represented as a family of sets because the edges \( e_2 \) and \( \widehat{e_2} \) are \hyperref[def:hypergraph/parallel_hyperedges]{parallel} and would be represented by the same set.

  Another counterexample is a hypergraph with a nonempty vertex set and no hyperedges. The union of the endpoints of all hyperedges is empty and is thus only a strict subset of the vertices.
\end{example}

\begin{definition}\label{def:category_of_small_hypergraphs}
  Suppose that we are given a \hyperref[def:grothendieck_universe]{Grothendieck universe} \( \mscrU \), which is safe to assume to be the smallest suitable one as explained in \fullref{def:large_and_small_sets}.

  We denote the \hyperref[def:category]{category} of \( \mscrU \)-small \hyperref[def:hypergraph]{hypergraphs} by \( \ucat{HypGph} \) or, if the universe is clear from the context, simply by \( \cat{HypGph} \). See \fullref{def:category_size} for a further discussion of universes and categories.

  \begin{itemize}
    \item The \hyperref[def:category/objects]{set of objects} \( \obj(\cat{HypGph}) \) is the set of all \( \mscrU \)-small hypergraphs, i.e. the hypergraphs \( H = (V, E, \mscrE) \) such that \( V \) and \( E \) are both members of \( \mscrU \).

    \item The \hyperref[def:category/morphisms]{set of morphisms} \( \cat{HypGph}(H, G) \) from \( H \) to \( G \) is the set hypergraph homomorphisms. Given two hypergraphs \( H: E_H \multto V_H \) and \( G: E_G \multto V_G \), a \term{hypergraph homomorphism} is a pair of functions
    \begin{equation}\label{eq:def:category_of_small_hypergraphs/homomorphism}
      \begin{cases}
        f_V: V_H \to V_G \\
        f_E: E_H \to E_G
      \end{cases}
    \end{equation}
    such that, for every hyperedge \( e \in E_H \), if \( v \in V_H \) is an endpoint of \( e \), then \( f_V(v) \) is an endpoint of \( f_E(e) \).

    Note that \enquote{graph embedding} commonly refers to an embedding of its \hyperref[def:quiver_geometric_realization/undirected]{geometric realization}, hence we will avoid the term when referring to injective hypergraph homomorphisms. Furthermore, it should be clarified whether we mean \enquote{injective on vertices} or \enquote{injective on edges}, which is an important distinction in category theory --- see \fullref{def:functor_invertibility}.

    \item The \hyperref[def:category/composition]{composition of the morphisms}
    \begin{align*}
      &(f_V, f_E) \in \cat{HypGph}(A, B),
      &(g_V, g_E) \in \cat{HypGph}(B, C)
    \end{align*}
    is the morphism
    \begin{equation*}
      (g_V \bincirc f_V, g_E \bincirc f_E) \in \cat{HypGph}(A, C).
    \end{equation*}

    \item The \hyperref[def:category/identity]{identity morphism} on the hypergraph \( H = (V, E, \mscrE) \) is the pair of \hyperref[def:multi_valued_function/identity]{identity functions} \( (\id_V, \id_H) \).
  \end{itemize}
\end{definition}

\begin{definition}\label{def:hypergraph_incidence_matrix}
  Let \( H = (V, E, \mscrE) \) be a finite hypergraph. Its \term{incidence matrix}
  \begin{equation*}
    M = \seq{ M_{ve} }_{v \in V, e \in E}
  \end{equation*}
  has elements
  \begin{equation*}
    M_{ve} \coloneqq \begin{cases}
      1,  &v \T{is \hyperref[def:hypergraph/incidence]{incident} to} e \\
      0,  &\T{otherwise.}
    \end{cases}
  \end{equation*}

  Compare this definition to \fullref{def:quiver_incidence_matrix}.
\end{definition}

\begin{example}\label{ex:def:hypergraph_incidence_matrix}
  The \hyperref[def:hypergraph_incidence_matrix]{incidence matrix} of the hypergraph \eqref{eq:fig:def:hypergraph} is
  \begin{equation}\label{eq:ex:def:hypergraph_incidence_matrix}
    \begin{blockarray}{cccccc}
        & e_1 & e_2 & \widehat{e_2} & e_3 & e_4 \\
      \begin{block}{c(ccccc)}
      a & 1   & 1   & 1             &     &     \\
      b & 1   &     &               & 1   &     \\
      c &     & 1   & 1             &     & 1   \\
      d &     &     &               & 1   & 1   \\
      e &     &     &               &     & 1   \\
      f &     &     &               &     & 1   \\
      \end{block}
    \end{blockarray}
  \end{equation}

  Every column corresponds to a hyperedge and its nonzero elements are the endpoints of the hyperedge.
\end{example}

\begin{definition}\label{def:hypergraph_vector_spaces}
  Let \( H = (V, E, \mscrE) \) be a \hyperref[def:hypergraph]{hypergraph}. We introduce several \hyperref[def:vector_space]{vector spaces} that allow us to study hypergraphs using linear algebra.

  \begin{thmenum}
    \thmitem{def:hypergraph_vector_spaces/vertex} The \term{vertex space} \( \BbbF_2^V \) is the \hyperref[thm:functions_over_algebra]{function space} \( \BbbF_2^V \), where \( \BbbF_2 \) is the \hyperref[thm:finite_fields]{two-element field}.

    Every subset \( U \subseteq V \) of vertices induces a unique vector \( \vect{U} = \seq{ \vect{U}_u }_{u \in U} \) in the \hyperref[def:hypergraph_vector_spaces/vertex]{vertex space} \( \BbbF_2^V \) such that
    \begin{equation*}
      \vect{U}_u \coloneqq \begin{cases}
        1, &u \in A \\
        0, &u \not\in A
      \end{cases}
    \end{equation*}

    This vector is called the \term{characteristic vector} of \( U \). Conversely, every vector in \( \BbbF_2^V \) induces a set of vertices. If \( U \) consists of a single vertex \( u \), we write \( \vect{u} \) rather than \( \vect{\set{u}} \).

    Without a choice of ordering of \( V \), there is no canonical basis in \( \BbbF_2^V \). Thus, even for finite graphs, we cannot in general regard the vectors of \( \BbbF_2^V \) as ordered tuple.

    \thmitem{def:hypergraph_vector_spaces/edge} Analogously, the \term{hyperedge space} \( \BbbF_2^E \) is the \hyperref[thm:functions_over_algebra]{function space} \( \BbbF_2^E \). Every subset of \( E \) induces a unique characteristic vector in \( \BbbF_2^E \) and vice versa.

    The space is motivated by \hyperref[def:undirected_multigraph_path]{undirected paths} and their \hyperref[def:undirected_multigraph_path/characteristic_vector]{characteristic vectors}.
  \end{thmenum}
\end{definition}

\begin{proposition}\label{thm:graphs_as_linear_transformations}
  Let \( V = \set{ 1, \ldots, m } \) and \( E = \set{ 1, \ldots, n } \). Then there is a bijection between the \( m \times n \) \hyperref[def:array/matrix]{matrices} over \( \BbbF_2 \) and the \hyperref[def:hypergraph]{hypergraphs} \( H = (V, E, \mscrE) \).
\end{proposition}
\begin{proof}
  The \hyperref[def:hypergraph_incidence_matrix]{incidence matrix} of every hypergraph is a matrix in \( \BbbF_2^{m \times n} \).

  Conversely, let \( M = \seq{ M_{ve} } \in \BbbF_2^{m \times n} \) and define the hypergraph \( H \coloneqq (V, E, \mscrE) \), where
  \begin{equation*}
    \begin{aligned}
      &\mscrE: E \multto V \\
      &\mscrE(e) \coloneqq \set{ v \in 1, \ldots, n \given M_{ve} = 1 }.
    \end{aligned}
  \end{equation*}

  Then the adjacency matrix of \( H \) is \( M \).
\end{proof}

\begin{definition}\label{def:hypergraph_transversal}\mcite[32]{GondranMinoux1984Graphs}
  Let \( H = (V, E, \mscrE) \) be a \hyperref[def:hypergraph]{hypergraph}. We say that a set \( T \subseteq V \) of vertices is a \term{transversal} of \( H \) if it is incident to every edge of \( H \).
\end{definition}

\begin{example}\label{ex:hypergraph_vertex_set_is_transversal}
  Every hypergraph has at least one transversal since the vertex set it itself a transversal.
\end{example}

\begin{definition}\label{def:hypergraph_minimal_transversal}
  A transversal \( T \) of the hypergraph \( H = (V, E, \mscrE) \) is said to be \term{minimal} if any of the following equivalent conditions hold:
  \begin{thmenum}
    \thmitem{def:hypergraph_minimal_transversal/order} \( T \) is a minimal \hyperref[def:partially_ordered_set_extremal_points/maximal_and_minimal_element]{element} under \hyperref[def:subset]{set inclusion} in the set of all transversals of \( H \).

    \thmitem{def:hypergraph_minimal_transversal/singleton} For every vertex \( v \) in \( T \) there exists a hyperedge \( e_v \in E \) such that
    \begin{equation*}
      T \cap \mscrE(e_v) = \set{ v }.
    \end{equation*}
  \end{thmenum}
\end{definition}
\begin{proof}
  \ImplicationSubProof{def:hypergraph_minimal_transversal/order}{def:hypergraph_minimal_transversal/singleton} Let \( T \) be minimal under inclusion among all transversals.

  Fix a vertex \( v \in T \). Since \( T \) is minimal, the set \( T \setminus \set{ v } \) is not a transversal. So there exists a hyperedge \( e_v \in E \) such that \( (T \setminus \set{ v }) \cap \mscrE(e_v) = \varnothing \).

  Now since \( T \) is a transversal for \( H \), the set \( T \cap \mscrE(e_v) \) is nonempty and thus
  \begin{align*}
    T \cap \mscrE(e_x)
    &=
    \parens[\Big]{ (T \setminus \set{ v }) \cup \set{ v } } \cap \mscrE(e_v)
    \reloset {\eqref{eq:def:semilattice/distributive_lattice/finite/meet_over_join}} = \\ &=
    \underbrace{\parens[\Big]{ (T \setminus \set{ v }) \cap \mscrE(e_v) }}_{\varnothing} \cup \parens[\Big]{ \set{ v } \cap \mscrE(e_v) }
    = \\ &=
    \set{ v }.
  \end{align*}

  \ImplicationSubProof{def:hypergraph_minimal_transversal/singleton}{def:hypergraph_minimal_transversal/order} Now suppose that for every vertex \( v \in T \) there exists a hyperedge \( e_v \in E \) such that \( T \cap \mscrE(e_v) = \set{ v } \).

  Suppose that \( T \) is not minimal. Then there exists some vertex \( w \in T \) be such that \( T \setminus \set{ w } \) is a transversal. But our assertion gives us an edge \( e_u \in E \) such that \( T \cap \mscrE(e_u) = \set{ w } \). Clearly the set \( T \setminus \set{ w } \) cannot be a transversal of \( H \) since
  \begin{equation*}
    (T \setminus \set{ w }) \cap \mscrE(e_u) = \varnothing.
  \end{equation*}

  This contradiction proves that \( T \) is minimal under set inclusion.
\end{proof}

\begin{example}\label{ex:hypergraph_with_no_minimal_transversal}
  We will give an example of a hypergraph without a minimal transversal.

  For every nonnegative integer \( n \) define the set
  \begin{equation*}
    e_n \coloneqq \set{ n, n + 1, n + 2, \ldots }.
  \end{equation*}

  Let \( H \) be the hypergraph whose hyperedges are \( e_1, e_2, \cdots \). The set of vertices is \( e_1 \), hence \( e_1 \) is also a transversal.

  Now assume that \( T \) is a minimal transversal for \( H \). Since, by \fullref{thm:natural_numbers_are_well_ordered}, the natural numbers are well-ordered, \( T \) has a minimum. Let \( n_0 \coloneqq \min T \).

  But \( T \setminus \set{ n_0 } \) is also a transversal because each hyperedge \( e_n \) intersects \( T \) at infinitely many points besides \( n_0 \).

  The obtained contradiction shows that \( H \) has no minimal transversal.
\end{example}

\begin{theorem}[Hypergraph minimal transversal existence]\label{thm:hypergraphs_have_minimal_transversal}
  Every \hyperref[def:hypergraph]{hypergraph} has a \hyperref[def:hypergraph_minimal_transversal]{minimal transversal}.

  Within \hyperref[def:zfc]{\logic{ZF}}, this theorem is equivalent to the \hyperref[def:zfc/choice]{axiom of choice} --- see \fullref{thm:axiom_of_choice_equivalences/hypergraph}.
\end{theorem}
\begin{proof}
  The proof is merely a translation of the axiom of choice into the language of hypergraphs.

  \ImplicationSubProof[def:zfc/choice]{the axiom of choice}[thm:hypergraphs_have_minimal_transversal]{minimal transversal existence} Let \( H = (V, E, \mscrE) \) be a hypergraph. Then
  \begin{equation*}
    \set{ \mscrE(e) \given e \in E }
  \end{equation*}
  is a family of nonempty sets and thus there exists a set \( B \subseteq V \) such that \( B \cap \mscrE(e) \) for every hyperedge \( e \in E \). This is a minimal transversal by \fullref{def:hypergraph_minimal_transversal/singleton}.

  \ImplicationSubProof[thm:hypergraphs_have_minimal_transversal]{minimal transversal existence}[def:zfc/choice]{axiom of choice} Let \( \mscrA \) be an arbitrary family of nonempty sets. Let \( H \) be the hypergraph \hyperref[thm:family_of_sets_induces_hypergraph]{induced by} \( \mscrA \). Let \( T \) be a minimal transversal of \( H \). Then, by definition, for every hyperedge \( e \in \mscrA \), the intersection \( T \cap \mscrE(e) \) is a singleton set. Hence, \( T \) is the image of a \hyperref[def:choice_function]{choice function} for \( \mscrA \).
\end{proof}
