\subsection{Hypergraphs}\label{subsec:hypergraphs}

\begin{definition}\label{def:hypergraph}\mcite[30]{GondranMinoux1984Graphs}
  Let \( V \) be a set and \( E \subseteq \pow(V) \) be a family of nonempty subsets of \( V \). We call the pair \( H = (V, E) \) a \term{hypergraph} if \( V = \bigcup E \).

  In analogy with undirected \hyperref[def:undirected_graph]{graphs}, we call elements of \( V \) \term{vertices} and elements of \( E \) \term{edges}. If the edges have cardinality at most \( 2 \), the hypergraph is an undirected graph.
\end{definition}

\begin{definition}\label{def:hypergraph_transversal}\mcite[32]{GondranMinoux1984Graphs}
  Let \( H = (V, E) \) be a \hyperref[def:hypergraph]{hypergraph}. We say that the set \( T \subseteq V \) is a \term{transversal} if it intersects every edge of \( H \).
\end{definition}

\begin{example}\label{ex:trivial_hypergraph_transversal}
  Every hypergraph has at least one transversal since the vertex set it itself a transversal.
\end{example}

\begin{definition}\label{def:minimal_hypergraph_transversal}
  A transversal \( T \) of the hypergraph \( H = (V, E) \) is said to be \term{minimal} if any of the following equivalent conditions hold:
  \begin{thmenum}
    \thmitem{def:minimal_hypergraph_transversal/order} \( T \) is a minimal \hyperref[def:poset_extremal_points/maximal_and_minimal_element]{element} under \hyperref[rem:subset_and_membership_relations]{set inclusion} in the set of all transversals of \( H \).
    \thmitem{def:minimal_hypergraph_transversal/singleton} for every vertex \( x \) in \( T \) there exists an edge \( E_x \in E \) such that
    \begin{equation*}
      T \cap E_x = \{ x \}.
    \end{equation*}
  \end{thmenum}
\end{definition}
\begin{proof}
  \ImplicationSubProof{def:minimal_hypergraph_transversal/order}{def:minimal_hypergraph_transversal/singleton} Let \( T \) be minimal under inclusion among all transversals. Fix \( x \in T \). Since \( T \) is minimal, the set \( T \setminus \{ x \} \) is not transversal. So there exists an edge \( E_x \in E \) such that
  \begin{equation*}
    \varnothing = (T \setminus \{ x \}) \cap E_x = (T \cap E_x) \setminus \{ x \}.
  \end{equation*}

  Now since \( T \) is a transversal for \( H \), \( T \cap E_x \) is nonempty and thus
  \begin{equation*}
    T \cap E_x = \{ x \}.
  \end{equation*}

  \ImplicationSubProof{def:minimal_hypergraph_transversal/singleton}{def:minimal_hypergraph_transversal/order} Now suppose that for every \( x \in T \) there exists an edge \( E_x \in \mathcal{F} \) such that \( T \cap E_x = \{ x \} \).

  Assume that \( T \) is not minimal and let \( y \in T \) be such that \( T \setminus \{ y \} \) is a transversal. But our assertion gives us an edge \( E_y \in \mathcal{F} \) such that \( T \cap E_y = \{ y \} \). Clearly the set \( T \setminus \{ y \} \) cannot be a transversal of \( H \) since
  \begin{equation*}
    (T \setminus \{ y \}) \cap E_y = \varnothing.
  \end{equation*}

  This contradiction proves that \( T \) is minimal under set inclusion.
\end{proof}

\begin{example}\label{ex:no_minimal_set_transversal}
  Let
  \begin{equation*}
    X_n \coloneqq \{ n, n + 1, n + 2, \ldots \}, n \in \BbbZ_{>0}
  \end{equation*}
  and \( H \coloneqq (\BbbZ_{>0}, \{ X_n \colon n \in \BbbZ_{>0} \}) \).

  The hypergraph \( H \) obviously has a transversal, e.g. \( \BbbZ_{>0} \), but it does not have a minimal transversal.

  To see this, assume that \( T \) is a minimal transversal. Since the natural numbers are well-\hyperref[thm:natural_numbers_are_well_ordered]{ordered}, \( T \) has a minimum. Let \( n_0 \coloneqq \min T \).

  But \( T \setminus \{ n_0 \} \) is also a transversal because each \( X_n \) intersects \( T \) at infinitely many points besides \( n_0 \).
\end{example}

\begin{proposition}\label{thm:finite_hypergraphs_have_minimal_transversal}
  Every finite hypergraph has a minimal transversal.
\end{proposition}
\begin{proof}
  A minimal transversal must exist because the vertex set is a transversal and we can only remove finitely many elements from it.
\end{proof}
