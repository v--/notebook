\subsection{Zermelo-Fraenkel set theory}\label{subsec:zermelo_fraenkel_set_theory}

\begin{definition}\label{def:choice_function}
  Let \( \mscrA \) be a (potentially empty) family of nonempty sets. A \term{choice function} on \( \mscrA \) is a (total) \hyperref[def:function]{function} \( c: \mscrA \to \bigcup \mscrA \) such that \( c(A) \in A \) for all \( A \in \mscrA \).

  A choice function on \( \mscrA \) \enquote{chooses} an element out of each member of \( \mscrA \). We sometimes have a canonical choice function, for example in \fullref{thm:polynomial_quotient_rings_equinumerous_with_module_of_polynomials}, however for general \hyperref[def:equivalence_relation/quotient]{quotient sets} the existence of a choice function is not by any means obvious.

  The existence of a choice function for family of nonempty sets is an important axiom of \logic{ZFC} --- see \fullref{def:zfc/choice}.
\end{definition}

\begin{definition}\label{def:zfc}
  The \hyperref[def:first_order_theory]{first-order theory} commonly abbreviated as \term{\logic{ZFC}} is based on the same language as \hyperref[def:naive_set_theory]{na\"ive set theory}, but with different axioms. The three letters refer to:
  \begin{itemize}
    \item \hi{Z}ermelo, who formulated the entire theory except for the \hyperref[def:zfc/replacement]{axiom schema of replacement} and the \hyperref[def:zfc/foundation]{axiom of foundation}.
    \item\mcite[sec. 63.8]{OpenLogicFull} \hi{F}raenkel, who simultaneously with Skolem reformulated the theory within first-order logic while also introducing the axiom schema of replacement.
    \item The \hyperref[def:zfc/choice]{axiom of \hi{c}hoice}, which is part of Zermelo's original theory, but is controversial enough to attract special attention --- see \fullref{thm:axiom_of_choice_equivalences}.
  \end{itemize}

  We are usually only interested in either \logic{ZFC}, which include all axioms listed in this definition. If we wish to avoid the axiom of choice --- for example when proving the equivalences in \fullref{thm:axiom_of_choice_equivalences} --- we instead use \logic{ZF}, which excludes the axiom of choice. The abbreviation of the latter theory is inaccurate historically, but is nevertheless established.

  If we wish to instead exclude the axiom of replacement, we obtain the theory \logic{Z}, however without context it is unclear whether the axiom of choice is included in \logic{Z} or not.

  See \fullref{thm:zfc_existence_theorems} for proofs of existence of common sets.

  The full list of axioms is:
  \begin{thmenum}
    \thmitem{def:zfc/extensionality} The \term{axiom of extensionality}, as defined in \fullref{def:naive_set_theory/extensionality}. This is also the only axiom of the theory that does not deal with existence.

    \thmitem{def:zfc/specification}\mcite[sec. 62.2]{OpenLogicFull} The \term{axiom schema of specification}, also known as the axiom schema of \term{separation} or of \term{restricted comprehension}, states that given a set \( A \), any formula defines a subset of \( A \). For each formula \( \varphi \) containing neither \( \tau \) nor \( \sigma \) as free variables, the following is an axiom:
    \begin{equation}\label{eq:def:zfc/specification}
      \qforall \sigma \qexists \tau \qforall \xi (\xi \in \tau \leftrightarrow \varphi \wedge \xi \in \sigma).
    \end{equation}

    As explained in \fullref{def:naive_set_theory/unrestricted_comprehension} and \fullref{def:set_builder_notation}, this set may depend on parameters, which are other sets. We must formally take the \hyperref[thm:implicit_universal_quantification]{universal closure} of this set to quantify over all possible values for the parameters.

    Compare this axiom to \hyperref[def:naive_set_theory/unrestricted_comprehension]{unrestricted comprehension}. Informally, this axiom can be obtained by taking the result of unrestricted comprehension and intersecting it with some set \( A \). As mentioned in \fullref{def:set_builder_notation}, in set-builder notation such a set is usually denoted by
    \begin{equation*}
      \set{ x \in B \given \varphi\Bracks{x, u_1, \ldots, u_n} }.
    \end{equation*}

    Unlike unrestricted comprehension some definable subset of the universe in the metatheory no longer have a corresponding set within the object logic.

    \thmitem{def:zfc/power_set}\mcite[sec. 62.5]{OpenLogicFull} The \term{axiom of power sets} states that every set has a corresponding \hyperref[def:basic_set_operations/power_set]{power set}. Symbolically,
    \begin{equation}\label{eq:def:zfc/power_set}
      \qforall \tau \qexists \sigma \ref{eq:def:basic_set_operations/power_set/predicate}[\sigma, \tau].
    \end{equation}

    \thmitem{def:zfc/union}\mcite[sec. 62.3]{OpenLogicFull} The \term{axiom of unions} states that for every set there exists another set that is its \hyperref[def:basic_set_operations/union]{union}. Symbolically,
    \begin{equation}\label{eq:def:zfc/union}
      \qforall \tau \qexists \sigma \ref{eq:def:basic_set_operations/union/predicate}[\sigma, \tau].
    \end{equation}

    \thmitem{def:zfc/pairing}\mcite[sec. 62.4]{OpenLogicFull} The \term{axiom of pairing} states that for any sets \( A \) and \( B \) there exists another set that contains exactly \( A \) and \( B \). This is the set \( \set{ A, B } \) in set-builder notation. Symbolically,
    \begin{equation}\label{eq:def:zfc/pairing}
      \qforall \tau \qforall \sigma \qexists \rho \qforall \xi \parens[\Big]{ \xi \in \rho \leftrightarrow ( \xi \doteq \tau \vee \xi \doteq \sigma) }.
    \end{equation}

    \thmitem{def:zfc/infinity}\mcite[sec. 62.6]{OpenLogicFull} The \term{axiom of infinity} states that an \hyperref[def:inductive_set]{inductive set} exists. Symbolically,
    \begin{equation}\label{eq:def:zfc/infinity}
      \qexists \tau \ref{eq:def:inductive_set/predicate}[\tau].
    \end{equation}

    This axiom is a simple and convenient way to state that infinite sets exist. Without it, we can only deal with finite sets unless we include some other axiom to replace it.

    \thmitem{def:zfc/choice}\mcite[sec. 69.4]{OpenLogicFull} The \term{axiom of choice} states that a \hyperref[def:choice_function]{choice function} exists for any family of nonempty sets. To state the axiom via a formula, we will avoid functions and only state it in terms of the image of the choice function. That is, we will formulate that for each family \( \mscrA \) of nonempty sets there exists a set \( B \) such that \( A \cap B \) is a singleton set for each \( A \in \mscrA \). Symbolically,
    \begin{equation}\label{eq:def:zfc/choice}
      \qforall \tau \parens[\Bigg]
        {
          \parens[\Big]{ \qforall {\xi \in \tau} \neg \ref{eq:def:empty_set/predicate}[\xi] }
          \rightarrow
          \parens[\Big]{ \qexists \sigma \qforall {\xi \in \tau} \qExists {\eta \in \sigma} \eta \in \xi }
        }
    \end{equation}
    where we have used the convention regarding existence and uniqueness described in \fullref{rem:first_order_formula_conventions/exists_unique}.

    See \fullref{thm:axiom_of_choice_equivalences} for more statements equivalent to this axiom.

    \thmitem{def:zfc/replacement}\mcite[sec. 63.7]{OpenLogicFull} The \term{axiom schema of replacement} roughly states that every \hyperref[rem:function_definition]{mapping} that is definable via a formula of \logic{ZFC} is a function. As we have done for the \hyperref[def:zfc/choice]{axiom of choice}, we only formulate the axiom via the image of the function. More concretely, given a formula \( \varphi \) not containing \( \tau \) nor \( \sigma \) as free variables, the following is an axiom:
    \begin{equation}\label{eq:def:zfc/replacement}
      \qforall \tau \parens[\Bigg]
        {
          \parens[\Big]{ \qforall {\xi \in \tau} \qExists \eta \varphi }
          \rightarrow
          \parens[\Big]{ \qexists \sigma \qforall \eta \parens[\Big]{ \eta \in \sigma \leftrightarrow \qexists {\xi \in \tau} \varphi } }
        }.
    \end{equation}

    As is the case with the \hyperref[def:zfc/specification]{axiom schema of specification}, the formula \( \varphi \) may depend on parameters, in which case we use its \hyperref[thm:implicit_universal_quantification]{universal closure}.

    This axiom is useful in cases where it is impossible or at least difficult to construct a function, for example in \fullref{thm:zfc_existence_theorems/indexed_family} or \fullref{thm:hartogs_lemma}. This is the case, in general, when dealing with \hyperref[thm:zfc_existence_theorems/indexed_family]{indexed families} rather than \hyperref[def:function]{functions}.

    This is the axiom that makes \logic{ZFC} require large models --- see \fullref{thm:cumulative_hierarchy_model_of_zfc}.

    \thmitem{def:zfc/foundation}\mcite[sec. 64.4]{OpenLogicFull} The \term{axiom of foundation} states that every nonempty set contains a member disjoint from the set itself. Symbolically,
    \begin{equation}\label{eq:def:zfc/foundation}
      \qforall \tau \parens[\Big]
        {
          \neg \ref{eq:def:empty_set/predicate}[\tau]
          \rightarrow
          \qexists {\sigma \in \tau} \neg \qexists \xi \parens{ \xi \in \tau \wedge \xi \in \sigma }
        }.
    \end{equation}

    This is a very powerful axiom because it shows that set membership in \logic{ZFC} is well-founded --- see \fullref{thm:set_membership_is_well_founded}. It is equivalent to \fullref{thm:axiom_of_regularity} and is often itself called the \term{axiom of regularity}.
  \end{thmenum}
\end{definition}

\begin{proposition}\label{thm:zfc_existence_theorems}
  We will now prove that all sets we have considered up until now in \fullref{sec:set_theory} are sets in \hyperref[def:zfc]{\logic{ZFC}}. The uniqueness in all cases follows from the \hyperref[def:zfc/extensionality]{axiom of extensionality}.

  A very fundamental existence result is provided by the fact that we are assuming \hyperref[rem:standard_model_of_set_theory]{standard} and \hyperref[rem:transitive_model_of_set_theory]{transitive} models of \logic{ZFC}. Let \( \mscrV = (V, I) \) be such a model. Then if \( v \in V \) and \( u \in v \), transitivity implies that \( u \in V \). Since the model is also standard, this shows that both \( u \) and \( v \) are sets within the object theory. Thus, if \( A \) is a set within the object theory and if \( B \in A \) within the metatheory, then necessarily \( B \) itself is a set within the object theory.

  For example, \fullref{thm:zfc_existence_theorems/set_of_functions} shows that the set \( \fun(A, B) \) of functions exists within the object theory for any two sets \( A \) and \( B \) in the object theory. Therefore, every single function between \( A \) and \( B \) is a set within the object theory because it is a member of \( \fun(A, B) \).

  With that in mind, we will show the following:

  \begin{thmenum}
    \thmitem{thm:zfc_existence_theorems/subset} If \( A \) is a set, then for any formula \( \varphi \) the set \( \set{ x \in A \given \varphi[x] } \) exists and, furthermore, it is a \hyperref[def:subset]{subset} of \( A \). Only \hyperref[def:first_order_definability]{definable subsets} of \( A \) can be described in this way, however. See \fullref{thm:zfc_existence_theorems/power_set}.

    \thmitem{thm:zfc_existence_theorems/empty_set} There exists a unique \hyperref[def:empty_set]{empty set}, which we denote by \( \varnothing \).

    \thmitem{thm:zfc_existence_theorems/universe} No \hyperref[def:set]{universal set} (set of all sets) exists.

    \thmitem{thm:zfc_existence_theorems/singleton} For every set \( A \), there exists a \hyperref[rem:singleton_sets]{singleton set} \( \set{ A } \) that contains only \( A \).

    \thmitem{thm:zfc_existence_theorems/arbitrary_intersection} For any \hi{nonempty} family \( \mscrA \), the \hyperref[def:basic_set_operations/intersection]{intersection} \( \bigcap \mscrA \) exists.

    \thmitem{thm:zfc_existence_theorems/binary_intersection} For any two sets \( A \) and \( B \), their \hyperref[def:basic_set_operations/intersection]{intersection} \( A \cap B \) exists.

    \thmitem{thm:zfc_existence_theorems/arbitrary_union} For any family \( \mscrA \), the \hyperref[def:basic_set_operations/union]{union} \( \bigcup \mscrA \) exists.

    \thmitem{thm:zfc_existence_theorems/binary_union} For any two sets \( A \) and \( B \), their \hyperref[def:basic_set_operations/union]{union} \( A \cup B \) exists.

    \thmitem{thm:zfc_existence_theorems/difference} For any two sets \( A \) and \( B \), their \hyperref[def:basic_set_operations/difference]{difference} \( A \setminus B \) exists.

    \thmitem{thm:zfc_existence_theorems/power_set} For any set \( A \), its \hyperref[def:basic_set_operations/power_set]{power set} \( \pow(A) \) exists.

    As a consequence, even subsets of \( A \) which are not \hyperref[def:first_order_definability]{definable} exist.

    \thmitem{thm:zfc_existence_theorems/successor} For any set \( A \), its \hyperref[def:ordinal_successor]{successor} \( \op{succ}(A) \) exists.

    \thmitem{thm:zfc_existence_theorems/kuratowski_pair} For any two sets \( A \) and \( B \), their \hyperref[def:tuple_and_cartesian_product/kuratowski_pair]{Kuratowski pair} \( \braket{ A, B } \) exists.

    \thmitem{thm:zfc_existence_theorems/indexed_family} For any sets \( \mscrK \) and \( \mscrA \), any \hyperref[def:tuple_and_cartesian_product/indexed_family]{indexed family} \( \seq{ A_k }_{k \in \mscrK} \subseteq \mscrA \) exists.

    \thmitem{thm:zfc_existence_theorems/cartesian_product} For any indexed family \( \set{ A_k }_{k \in \mscrK} \), its \hyperref[def:tuple_and_cartesian_product]{Cartesian product} \( \bigtimes_{k \in \mscrK} A_k \) exists.

    \thmitem{thm:zfc_existence_theorems/set_of_relations} For any two sets \( A \) and \( B \), the set of all relations between \( A \) and \( B \) exists.

    \thmitem{thm:zfc_existence_theorems/set_of_functions} For any two sets \( A \) and \( B \), the set \hyperref[def:function/set_of_functions]{\( \fun(A, B) \)} exists.

    \thmitem{thm:zfc_existence_theorems/quotient_set} For any set \( A \) and any \hyperref[def:equivalence_relation]{equivalence relation} \( \cong \), the \hyperref[def:equivalence_relation/quotient]{quotient set} \( A / {\cong} \) exists.

    \thmitem{thm:zfc_existence_theorems/function_evaluation} Fix some sets \( A \) and \( B \) and some indexed family of functions \( \seq{ f_k }_{k \in \mscrK} \) where \( f: A \to B_k \) for \( k \in \mscrK \). For any \( x \in A \) the corresponding tuple \( \seq{ f_k(x) }_{k \in \mscrK} \) exists.
  \end{thmenum}
\end{proposition}
\begin{proof}
  \SubProofOf{thm:zfc_existence_theorems/subset} This is a trivial consequence of the \hyperref[def:zfc/specification]{axiom schema of specification}.

  \SubProofOf{thm:zfc_existence_theorems/empty_set} As a consequence of the \hyperref[def:zfc/infinity]{axiom of infinity}, there exists at least one inductive set. Let \( A \) be an inductive set. Then from the \hyperref[def:zfc/specification]{axiom schema of specification} it follows that
  \begin{equation*}
    \set{ x \in A \given \bot }
  \end{equation*}
  is a set. Furthermore, \( x \) belongs to this set if and only if \( x \) satisfies \( \bot \), which is impossible, hence the set is empty.

  As a consequence of the \term{axiom of extensionality}, this empty set is unique. As discussed in \fullref{def:empty_set}, we denote this unique empty set by \( \varnothing \).

  \SubProofOf{thm:zfc_existence_theorems/universe} Aiming at a contradiction, suppose that there was a universal set \( U \). Then we can easily reproduce \fullref{thm:russels_paradox} by using restricted (to \( U \)) rather than unrestricted comprehension.

  Unlike in na\"ive set theory, however, the existence of \( U \) is not an axiom of the theory. Therefore, rather than demonstrating that \logic{ZFC} is inconsistent, Russel's paradox shows that certain sets like the universal set do not exist in \logic{ZFC}.

  \SubProofOf{thm:zfc_existence_theorems/singleton} Fix a set \( A \). The set \( \set{A} \), if it exists, is equal to \( \set{A} = \set{A, A} \), by the \hyperref[def:zfc/extensionality]{axiom of extensionality}.

  Thus, by the \hyperref[def:zfc/pairing]{axiom of pairing}, the singleton set \( \set{A} = \set{A, A} \) actually exists.

  \SubProofOf{thm:zfc_existence_theorems/arbitrary_intersection} Let \( \mscrA \) be a nonempty family of sets. Their intersection \( \bigcap \mscrA \), if it exists, is a subset of every set \( A \in \mscrA \).

  Therefore, since the family \( \mscrA \) is nonempty, the \hyperref[def:zfc/specification]{axiom schema of specification} applied to any set in \( A \in \mscrA \) guarantees the existence of the intersection \( \bigcap \mscrA \). More precisely, for any \( A_0 \in \mscrA \), we can define the intersection of \( A \) as
  \begin{equation*}
    \bigcap \mscrA = \set{ x \in A_0 \given \qexists {A \in \mscrA} x \in A }.
  \end{equation*}

  \SubProofOf{thm:zfc_existence_theorems/binary_intersection} For sets \( A \) and \( B \), by the \hyperref[def:zfc/pairing]{axiom of pairing} the set \( \set{ A, B } \) exists. Then by \fullref{thm:zfc_existence_theorems/arbitrary_intersection}, the binary intersection
  \begin{equation*}
    A \cap B = \bigcap \set{ A, B }
  \end{equation*}
  also exists.

  \SubProofOf{thm:zfc_existence_theorems/arbitrary_union} The existence of arbitrary unions is merely a restatement of the \hyperref[def:zfc/union]{axiom of unions}.

  \SubProofOf{thm:zfc_existence_theorems/binary_union} Similarly to \fullref{thm:zfc_existence_theorems/binary_intersection}, for sets \( A \) and \( B \), by the \hyperref[def:zfc/pairing]{axiom of pairing} the set \( \set{ A, B } \) exists and by the \hyperref[def:zfc/union]{axiom of unions}, the binary union
  \begin{equation*}
    A \cup B = \bigcup \set{ A, B }
  \end{equation*}
  exists.

  \SubProofOf{thm:zfc_existence_theorems/difference} The difference \( A \setminus B \) is guaranteed to exist by \hyperref[def:zfc/specification]{restricted comprehension}:
  \begin{equation*}
    A \setminus B = \set{ x \in A \given x \not\in B }.
  \end{equation*}

  \SubProofOf{thm:zfc_existence_theorems/power_set} The existence of power sets is a restatement of the \hyperref[def:zfc/power_set]{axiom of power sets}.

  \SubProofOf{thm:zfc_existence_theorems/successor} The successor of \( A \) is
  \begin{equation*}
    \op{succ}(A) = \set{ A } \cup A.
  \end{equation*}

  Its existence follows from \fullref{thm:zfc_existence_theorems/singleton} and \fullref{thm:zfc_existence_theorems/binary_union}.

  \SubProofOf{thm:zfc_existence_theorems/kuratowski_pair} The existence of the Kuratowski pair
  \begin{equation*}
    \braket{ A, B } = \set{ \set{ A }, \set{ A, B } }
  \end{equation*}
  can be proven by the \hyperref[def:zfc/pairing]{axiom of pairing} applied first to \( \set{ A, B } \) and then to the pair itself.

  \SubProofOf{thm:zfc_existence_theorems/indexed_family} Given an indexed family \( \seq{ A_k }_{k \in \mscrK} \), by the \hyperref[def:zfc/replacement]{axiom schema of replacement}, there exists a set
  \begin{equation*}
    \mscrA \coloneqq \set{ A_k \given k \in \mscrK }.
  \end{equation*}

  The family \( \seq{ A_k }_{k \in \mscrK} \) is, formally, a set of Kuratowski pairs. Every pair \( \braket{ k, A_k } \) is itself a subset of \( \pow(\mscrK \bigcup \mscrA) \). The family is then a subset of \( \pow(\pow(\mscrK \bigcup \mscrA)) \). Applying the \hyperref[def:zfc/power_set]{axiom of power sets} again, we obtain that the family exits as a set.

  \SubProofOf{thm:zfc_existence_theorems/cartesian_product} By definition, the Cartesian product \( \bigtimes_{k \in \mscrK} A_k \) is a set of indexed by \( \mscrK \) families of members of the union \( \bigcup\set{ A_k \given k \in \mscrK } \).

  We can apply the \hyperref[def:zfc/power_set]{axiom of power sets} one more time in addition to those in \fullref{thm:zfc_existence_theorems/indexed_family} to obtain the set of all indexed by \( \mscrK \) families of members of this union. Then we can apply the \hyperref[def:zfc/specification]{axiom schema of specification} to restrict only to those families that satisfy the condition of \fullref{def:tuple_and_cartesian_product/product}.

  \SubProofOf{thm:zfc_existence_theorems/set_of_relations} All relations between \( A \) and \( B \) are subsets of \( A \times B \), hence elements of \( \pow(A \times B) \). The latter exists by \fullref{thm:zfc_existence_theorems/cartesian_product} and \fullref{thm:zfc_existence_theorems/power_set}.

  \SubProofOf{thm:zfc_existence_theorems/set_of_functions} The set of single-valued functions from \( A \) to \( B \) is a subset of \( \pow(A \times B) \), hence it exists by \fullref{thm:zfc_existence_theorems/set_of_relations} and \fullref{thm:zfc_existence_theorems/power_set}.

  \SubProofOf{thm:zfc_existence_theorems/quotient_set} Let \( A \) be an arbitrary set and \( \cong \) be a binary relation over \( A \). Then \( A / {\cong} \) is a subset of \( \pow(A) \) and hence it exists as a consequence of the \hyperref[def:zfc/power_set]{axiom of power sets} and the \hyperref[def:zfc/specification]{axiom schema of specification} --- see \fullref{thm:equivalence_partition/partition}.

  \SubProofOf{thm:zfc_existence_theorems/function_evaluation} The set \( \seq{ f_k }_{k \in \mscrK} \) exists because it is a member of \( \fun(\mscrK, \fun(A, B)) \), which set exists by \fullref{thm:zfc_existence_theorems/set_of_functions}.

   Then \( \seq{ f_k(x) }_{k \in \mscrK} \) is the function
   \begin{equation*}
     \begin{aligned}
       &g_x: \fun(\mscrK, B) \\
       &g_x(k) \coloneqq f_k(x).
     \end{aligned}
   \end{equation*}
\end{proof}

\begin{theorem}[Multi-valued selection existence]\label{thm:existence_of_multi_valued_function_selection}
  Every \hyperref[def:multi_valued_function/total]{total multi-valued function} has a \hyperref[def:function/selection]{selection}.

  In \hyperref[def:zfc]{\logic{ZF}} this theorem is equivalent to the \hyperref[def:zfc/choice]{axiom of choice} --- see \fullref{thm:axiom_of_choice_equivalences/selection}.
\end{theorem}
\begin{proof}
  \ImplicationSubProof[def:zfc/choice]{axiom of choice}[thm:existence_of_multi_valued_function_selection]{selection existence} Let \( F: A \multto B \) be a total multi-valued function. As described in \fullref{rem:multi_valued_functions}, we can instead take the \hyperref[def:tuple_and_cartesian_product/indexed_family]{indexed family} \( \set{ F(a) }_{a \in A} \). Denote by \( f \) the \hyperref[def:function]{single-valued function} from \( A \) to the image \( \set{ F(a) \given a \in A } \subseteq \pow(B) \) of this indexed family (see \fullref{rem:multi_valued_functions}).

  Since \( F \) is \hyperref[def:multi_valued_function/total]{total}, the family \( \img(f) = \set{ F(a) }_{a \in A} \) is a (potentially empty) family of nonempty sets. Thus, we can apply the axiom of choice to obtain a \hyperref[def:choice_function]{choice function} \( c: \img(f) \to B \).

  The composition \( c \bincirc f \) is then a single-valued function. Furthermore, we have
  \begin{equation*}
    (c \bincirc f)(a) \in f(a) = F(a)
  \end{equation*}
  so \( c \bincirc f \) is a selection of \( F \).

  This finishes the proof of the proposition. To see how it implies the axiom of choice, consider an indexed family \( \mscrA \) of nonempty sets. The identity \( \id_\mscrA \) is a single-valued function from \( \mscrA \) to itself, however we can also regard it as a multi-valued function from \( \mscrA \) to \( \bigcup \mscrA \). Any selection of this multi-valued function is then a choice function for \( \mscrA \).

  \ImplicationSubProof[thm:existence_of_multi_valued_function_selection]{selection existence}[def:zfc/choice]{axiom of choice} Fix a family \( \mscrA \) of nonempty sets. Define the function
  \begin{equation*}
    \begin{aligned}
      &F: \mscrA \to \bigcup \mscrA \\
      &F(A) \coloneqq A
    \end{aligned}
  \end{equation*}
  that sends each set in \( \mscrA \) to the corresponding subset of \( \mscrA \). In terms of relations, we have \( (A, x) \in F \) if and only if \( x \in A \). This is a total multi-valued function because every set in \( \mscrA \) is nonempty.

  Then every selection of \( F \) is a choice function for \( \mscrA \).
\end{proof}

\begin{definition}\label{def:function_invertibility_categorical}
  As a bridge between \fullref{def:function_invertibility} and \fullref{def:morphism_invertibility}, we introduce the following concepts for a function \( f: A \to B \):
  \begin{thmenum}
    \thmitem{def:function_invertibility_categorical/left} A \term{left inverse} of \( f \) is a function \( g: B \to A \) such that \( g \bincirc f = \id_A \). If \( f \) has at least one left inverse, we say that it is \term{left-invertible}.

    \thmitem{def:function_invertibility_categorical/right} Similarly, a \term{right inverse} of \( f \) is a function \( g: B \to A \) such that \( f \bincirc g = \id_B \). If \( f \) has at least one right inverse, we say that it is \term{right-invertible}.

    \thmitem{def:function_invertibility_categorical/two_sided} If \( f \) is both left-invertible and right-invertible, \( f \) is \term{invertible} or, if clarification is needed, \term{fully invertible}. Its inverse function, which is unique by \fullref{thm:morphism_invertibility_properties/left_and_right}, is then called the \term{two-sided inverse}.
  \end{thmenum}
\end{definition}

\begin{proposition}\label{thm:function_invertibility_categorical}
  We state this result here because it requires the \hyperref[def:zfc/choice]{axiom of choice} via \fullref{thm:existence_of_multi_valued_function_selection}.

  \begin{thmenum}
    \thmitem{thm:function_invertibility_categorical/empty} An \hyperref[def:multi_valued_function/empty]{empty function} is \hyperref[def:function_invertibility_categorical/left]{left-invertible} if and only if it is \hyperref[def:function_invertibility_categorical/left]{right-invertible}.

    For comparison, an empty function is always \hyperref[def:function_invertibility/injective]{injective}, but it may not be \hyperref[def:function_invertibility/surjective]{surjective}.

    \thmitem{thm:function_invertibility_categorical/nonempty_injective} A nonempty function is \hyperref[def:function_invertibility/injective]{injective} if and only if it is \hyperref[def:function_invertibility_categorical/left]{left-invertible}.

    \thmitem{thm:function_invertibility_categorical/surjective} A function is \hyperref[def:function_invertibility/surjective]{surjective} if and only if it is \hyperref[def:function_invertibility_categorical/right]{right-invertible}.

    \thmitem{thm:function_invertibility_categorical/bijective} A function is \hyperref[def:function_invertibility/bijective]{bijective} if and only if it is \hyperref[def:function_invertibility_categorical/two_sided]{fully invertible}.
  \end{thmenum}
\end{proposition}
\begin{proof}
  \SubProofOf{thm:function_invertibility_categorical/empty} Let \( \varnothing: A \to B \) be the empty function from \( A \) to \( B \). The domain \( A \) is empty because otherwise the function itself would be nonempty. It follows that the empty function is always injective.

  It is possible for the range \( B \) to be nonempty, however any function from \( B \) to \( A \) is necessarily empty because the range \( A \) is empty. Thus, there exists a function from \( B \) to \( A \) if and only if \( B = \varnothing \).

  If there does exist a function \( g \) from \( B \) to \( A \), then \( \id_A = g \bincirc f = f \bincirc g = \id_B \) with each of the four functions being empty.

  Therefore, \( f \) is left-invertible if and only it is right-invertible.

  \SubProofOf{thm:function_invertibility_categorical/nonempty_injective} Let \( f: A \to B \) be a nonempty injective function. \Fullref{def:function_invertibility/injective/inverse} states that the \hyperref[def:multi_valued_function/inverse]{inverse} \( f^{-1}: B \to A \) is a partial single-valued function.

  Fix some value \( a \in A \) and define
  \begin{equation*}
    \begin{aligned}
      &g: B \to A \\
      &g(y) \coloneqq \begin{cases}
        f^{-1}(y), &y \in f(A) \\
        a, &\T{otherwise}
      \end{cases}.
    \end{aligned}
  \end{equation*}

  This function \( g \) is a left inverse of \( f \) because, for any \( x \in A \),
  \begin{equation*}
    (g \bincirc f)(x)
    =
    g(f(x))
    =
    f^{-1}(f(x))
    =
    x.
  \end{equation*}

  We can see that \( g \) would be unique except for our choice of \( a \). We may even define \( g \) to take different values in \( A \) outside \( f(A) \). Thus, \( g \) is non-unique in general.

  Conversely, suppose that \( f: A \to B \) is not necessarily injective and let \( g: B \to A \) be a left inverse of \( f \). Let \( x_1, x_2 \) be two different points in \( A \). Since \( g \bincirc f = \id_A \), clearly \( g(f(x_1)) \neq g(f(x_2)) \). If we suppose that \( f(x_1) = f(x_2) \), we would obtain a contradiction since then \( g(f(x_1)) \) would equal \( g(f(x_2)) \). Hence, \( f(x_1) \neq f(x_2) \). Thus, shows that \( f \) is injective.

  \SubProofOf{thm:function_invertibility_categorical/surjective} Let \( f: A \to B \) be a (potentially empty) surjective function. \Fullref{def:function_invertibility/injective/inverse} states that the \hyperref[def:multi_valued_function/inverse]{inverse} \( F^{-1}: B \to A \) is a total multi-valued function. We can apply \fullref{thm:existence_of_multi_valued_function_selection} to obtain a \hyperref[def:function/selection]{selection} \( g: B \to A \) of \( F^{-1} \). This selection is a right-inverse because, for any \( y \in B \),
  \begin{equation*}
    (f \bincirc g)(y)
    =
    f(g(y))
    =
    f(F^{-1}(y))
    =
    y.
  \end{equation*}

  Conversely, suppose that \( g: B \to A \) is a right inverse of \( f: A \to B \). Let \( y \in B \). We have that \( g(y) \) is in the preimage of \( y \) under \( f \) because \( f(g(y)) = y \). Thus, the preimage is not empty for an arbitrary point in \( B \). We conclude that \( f \) is surjective.

  \SubProofOf{thm:function_invertibility_categorical/bijective} If \( f: A \to B \) is a bijective empty function, then it is surjective and, by \fullref{thm:function_invertibility_categorical/surjective}, it is right-invertible. By \fullref{thm:function_invertibility_categorical/empty}, it is also left-invertible. Thus, it is fully invertible.

  Conversely, if an empty function \( f: A \to B \) is fully invertible, by \fullref{thm:function_invertibility_categorical/empty} we have \( A = B = \varnothing \) and hence it is bijective.

  Finally, if \( f: A \to B \) is \hi{nonempty} function, then by \fullref{thm:function_invertibility_categorical/nonempty_injective} and \fullref{thm:function_invertibility_categorical/surjective} it is bijective if and only if it is fully invertible.
\end{proof}

\begin{theorem}[Axiom of choice equivalences]\label{thm:axiom_of_choice_equivalences}
  The following statements are commonly referred to as \enquote{the} \hyperref[def:zfc/choice]{axiom of choice}:
  \begin{thmenum}[series=thm:axiom_of_choice_equivalences]
    \thmitem{thm:axiom_of_choice_equivalences/choice_sets} For every family of nonempty sets \( \mscrA \) there exists a set \( B \) such that \( A \cap B \) is a singleton set for every \( A \in \mscrA \).

    \thmitem{thm:axiom_of_choice_equivalences/choice_function} Every family of nonempty sets has a corresponding \hyperref[def:choice_function]{choice function}.

    \thmitem{thm:axiom_of_choice_equivalences/choice_product} The \hyperref[def:tuple_and_cartesian_product]{Cartesian product} of a family of nonempty sets is nonempty.
  \end{thmenum}

  The following statements are equivalent to the axiom of choice, but are not conflated with it:
  \begin{thmenum}[resume=thm:axiom_of_choice_equivalences]
    \thmitem{thm:axiom_of_choice_equivalences/selection} \Fullref{thm:existence_of_multi_valued_function_selection}: Every \hyperref[def:multi_valued_function/total]{total multi-valued function} has a \hyperref[def:function/selection]{selection}.

    \thmitem{thm:axiom_of_choice_equivalences/hypergraph} \Fullref{thm:hypergraphs_have_minimal_transversal}: Every \hyperref[def:graph/hypergraph]{hypergraph} has a \hyperref[def:graph/hypergraph_minimal_transversal]{minimal transversal}.

    \thmitem{thm:axiom_of_choice_equivalences/well_ordering} \Fullref{thm:well_ordering_theorem}: Any \hyperref[def:set]{set} can be \hyperref[def:well_ordered_set]{well-ordered}.

    \thmitem{thm:axiom_of_choice_equivalences/zorns_lemma} \Fullref{thm:zorns_lemma}: If every \hyperref[def:partially_ordered_set_chain_and_antichain]{chain} in a \hyperref[def:partially_ordered_set]{partially ordered set} has an \hyperref[def:partially_ordered_set_extremal_points/upper_and_lower_bounds]{upper bound}, then the entire set has a \hyperref[def:partially_ordered_set_extremal_points/maximal_and_minimal_element]{maximal element}.

    \thmitem{thm:axiom_of_choice_equivalences/vector_space_bases} \Fullref{thm:every_vector_space_has_a_basis}: Every vector space has a \hyperref[def:left_module_hamel_basis]{basis}.

    \thmitem{thm:axiom_of_choice_equivalences/tychonoff} \Fullref{thm:tychonoffs_product_theorem}: The \hyperref[def:topological_product]{topological product} of \hyperref[def:compact_space]{compact} is compact.

    \thmitem{thm:axiom_of_choice_equivalences/krull} \Fullref{thm:krulls_theorem}: Every nontrivial \hyperref[def:semiring/commutative_unital_ring]{commutative unital ring} has a \hyperref[def:maximal_ring_ideal]{maximal ideal}.
  \end{thmenum}
\end{theorem}
\begin{proof}
  The equivalence proofs can be found in the linked theorems since that is usually the most appropriate place to put them.
\end{proof}

\begin{theorem}[Diaconescu-Goodman-Myhill theorem]\label{thm:diaconescu_goodman_myhill_theorem}\mcite[corr. 2]{Diaconescu1975}
  In \hyperref[def:zfc]{\logic{ZF}} the \hyperref[def:zfc/choice]{axiom of choice} entails the law of the excluded middle \eqref{eq:thm:minimal_propositional_negation_laws/lem}.
\end{theorem}
