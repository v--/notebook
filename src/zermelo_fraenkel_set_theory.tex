\subsection{Zermelo-Fraenkel set theory}\label{subsec:zermelo_fraenkel_set_theory}

\begin{definition}\label{def:inductive_set}
  A set is called \term{inductive} if contains the empty set and is closed under the \hyperref[def:basic_set_operations/successor]{successor operator}. The existence of (at least one) inductive set is an axiom of \logic{ZFC} --- see \fullref{def:zfc/infinity}.
\end{definition}

\begin{definition}\label{def:choice_function}
  Let \( \mscrA \) be a family of nonempty sets. A \term{choice function} on \( \mscrA \) is a (total) \hyperref[def:function]{function} \( f: \mscrA \to \bigcup \mscrA \) such that \( f(A) \in A \) for all \( A \in \mscrA \).

  A choice function on \( A \) \enquote{chooses} an element out of each member of \( \mscrA \). We sometimes have a natural choice function, for example in \fullref{thm:polynomial_quotient_rings_equinumerous_with_module_of_polynomials}, however for general \hyperref[def:equivalence_relation/quotient]{quotient sets} the existence of a choice function is not obvious.

  The existence of a choice function for family of nonempty sets is an axiom of \logic{ZFC} --- see \fullref{def:zfc/choice}.
\end{definition}

\begin{definition}\label{def:zfc}
  The \hyperref[def:first_order_theory]{first-order theory} commonly abbreviated as \term{\logic{ZFC}} is based on the same language as \hyperref[def:naive_set_theory]{na\"ive set theory} but with different axioms. The three letters refer to:
  \begin{itemize}
    \item \boldrm{\logic{Z}}ermelo, who formulated the entire theory minus the \hyperref[def:zfc/replacement]{axiom schema of replacement} and the \hyperref[def:zfc/foundation]{axiom of foundation}.
    \item\mcite[sec. 63.8]{OpenLogicFull} \boldrm{\logic{F}}raenkel, who simultaneously with Skolem reformulated the theory within first-order logic and then added the axiom schema of replacement.
    \item The \hyperref[def:zfc/choice]{axiom of \boldrm{\logic{c}}hoice}, which is part of Zermelo's original theory but is controversial enough to attract special attention --- see \fullref{subsec:axiom_of_choice}.
  \end{itemize}

  We are usually only interested in either \logic{ZFC}, which include all axioms listed in this definition. If we wish to avoid the axiom of choice --- for example when proving the equivalences in \fullref{thm:axiom_of_choice_equivalences} --- we instead use \logic{ZF}, which excludes the axiom of choice. The abbreviation of the latter theory is inaccurate historically but is nevertheless established.

  The full list of axioms is:
  \begin{thmenum}
    \thmitem{def:zfc/extensionality} The \term{axiom of extensionality}, as defined in \fullref{def:naive_set_theory/extensionality}. This is also the only axiom of the theory that does not deal with existence.

    \thmitem{def:zfc/specification}\mcite[sec. 62.2]{OpenLogicFull} The \term{axiom schema of specification}, also known as the axiom schema of \term{separation} or of \term{restricted comprehension}, states that given a set \( A \), any formula \( \varphi \) with a single free variable defines a subset of \( A \). For each such formula \( \varphi \), the following is an axiom:
    \begin{equation}\label{eq:def:zfc/specification}
      \qforall \beta \qexists \alpha \qforall \eta \parens[\Big]{ \eta \in \alpha \leftrightarrow (\varphi[\xi \mapsto \eta] \wedge \eta \in \beta) }.
    \end{equation}

    Compare this axiom to \hyperref[def:naive_set_theory/unrestricted_comprehension]{unrestricted comprehension}. Informally, this axiom can be obtained by taking the result of unrestricted comprehension
    \begin{equation*}
      \set{ x \given \varphi[\xi \mapsto x] }
    \end{equation*}
    and intersecting it with some set \( A \). As mentioned in \fullref{def:set_builder_notation}, in set-builder notation such a set is usually denoted by
    \begin{equation*}
      \set{ x \in A \given \varphi[\xi \mapsto x] }.
    \end{equation*}

    Unlike unrestricted comprehension, however we can only take subsets from an object that we know is a set and not from the universe itself.

    \thmitem{def:zfc/power_set}\mcite[sec. 62.5]{OpenLogicFull} The \term{axiom of power sets} states that \hyperref[def:basic_set_operations/power_set]{power set} of any set exists. Symbolically,
    \begin{equation}\label{eq:def:zfc/power_set}
      \qforall \alpha \qexists \beta \qforall \xi \parens[\Big]{ \xi \in \beta \leftrightarrow ( \qforall {\eta \in \xi} \eta \in \alpha ) }.
    \end{equation}

    \thmitem{def:zfc/union}\mcite[sec. 62.3]{OpenLogicFull} The \term{axiom of union} states that the \hyperref[def:basic_set_operations/union]{union} of any set exists. Symbolically,
    \begin{equation}\label{eq:def:zfc/union}
      \qforall \alpha \qexists \beta \qforall \xi \parens[\Big]{ \xi \in \beta \leftrightarrow ( \qexists {\eta \in \alpha} \xi \in \eta ) }.
    \end{equation}

    \thmitem{def:zfc/pairing}\mcite[sec. 62.4]{OpenLogicFull} The \term{axiom of pairing} states that for any sets \( A \) and \( B \), the set \( \set{ A, B } \) exists. Symbolically,
    \begin{equation}\label{eq:def:zfc/pairing}
      \qforall \alpha \qforall \beta \qexists \gamma \qforall \xi \parens[\Big]{ \xi \in \gamma \leftrightarrow ( \xi \doteq \alpha \vee \xi \doteq \beta) }.
    \end{equation}

    \thmitem{def:zfc/infinity}\mcite[sec. 62.6]{OpenLogicFull} The \term{axiom of infinity} states that an \hyperref[def:inductive_set]{inductive set} exists. Symbolically,
    \begin{equation}\label{eq:def:zfc/infinity}
      \qexists \alpha \qforall \xi \parens[\Big]
        {
          \xi \in \alpha \leftrightarrow \parens[\Big]
            {
              \hyperref[eq:def:empty_set/predicate]{\op{IsEmpty}[\xi]} \vee \parens{ \qexists \eta \hyperref[eq:def:basic_set_operations/power_set/predicate]{\op{IsSucc}[\eta, \xi]} }
            }
        }
    \end{equation}

    This axiom is a simple and convenient way to state that infinite sets exist. Without it, we can only deal with finite sets unless we include some other axiom to replace it.

    \thmitem{def:zfc/choice}\mcite[sec. 69.4]{OpenLogicFull} The \term{axiom of choice} states that a \hyperref[def:choice_function]{choice function} exists for any family of nonempty sets. To state the axiom via a formula, we will avoid functions and only state it in terms of the image of the choice function. That is, we will formulate that for each family \( \mscrA \) of nonempty sets there exists a set \( B \) such that \( A \cap B \) is a singleton set for each \( A \in \mscrA \). Symbolically,
    \begin{equation}\label{eq:def:zfc/choice}
      \qforall \alpha \parens[\Big]
        {
          \qforall {\xi \in \alpha} \neg \hyperref[eq:def:empty_set/predicate]{\op{IsEmpty}[\xi]}
          \rightarrow
          \qexists \beta \qforall {\xi \in \alpha} \qExists {\eta \in \beta} \eta \in \xi
        }
    \end{equation}
    where we have used the convention regarding existence and uniqueness described in \fullref{rem:first_order_formula_conventions/exists_unique}.

    See \fullref{subsec:axiom_of_choice} for further discussion of this axiom.

    \thmitem{def:zfc/replacement}\mcite[sec. 63.7]{OpenLogicFull} The \term{axiom schema of replacement} roughly states that every \hyperref[rem:function_definition]{mapping} that is definable via a formula of \logic{ZFC} is a function. As for the \hyperref[def:zfc/choice]{axiom of choice}, we only formulate the axiom via the image of the function. More concretely, given a formula \( \varphi \) with two free variables, the following is an axiom:
    \begin{equation}\label{eq:def:zfc/replacement}
      \qforall \alpha \parens[\Big]
        {
          \qforall {\xi \in \alpha} \qExists \eta \varphi[\xi, \eta]
          \rightarrow
          \qexists \beta \qforall \eta \parens[\Big]{ \eta \in \beta \leftrightarrow \qexists {\xi \in \alpha} \varphi[\xi, \eta] }
        }
    \end{equation}

    This is used, for example, in \fullref{thm:transfinite_recursion}, where \fullref{thm:zfc_existence_theorems} is not sufficient.

    \thmitem{def:zfc/foundation}\mcite[sec. 64.4]{OpenLogicFull} The \term{axiom of foundation} states that every nonempty set contains a member disjoint from the set itself. Symbolically,
    \begin{equation}\label{eq:def:zfc/foundation}
      \qforall \alpha \parens[\Big]
        {
          \neg \hyperref[eq:def:empty_set/predicate]{\op{IsEmpty}[\alpha]}
          \rightarrow
          \qexists {\beta \in \alpha} \neg \qexists \xi \parens{ \xi \in \alpha \wedge \xi \in \beta } }.
        }
    \end{equation}
  \end{thmenum}
\end{definition}

\begin{proposition}\label{thm:zfc_existence_theorems}
  \todo{Prove that the introduced up until now sets are sets in ZFC}.
\end{proposition}

\begin{proposition}\label{thm:intersection_of_inductive_sets}
  If \( \mscrA \) is a \hyperref[rem:family_of_sets]{family} of \hyperref[def:inductive_set]{inductive sets}, then the intersection \( \bigcup \mscrA \) is also an inductive set.
\end{proposition}
\begin{proof}
  Clearly \( \varnothing \) belongs to the intersection. Now let \( x \in \bigcup \mscrA \). Then \( \op{succ}(x) \in A \) for every \( A \in \mscrA \) and hence \( \op{succ}(x) \in \bigcup \mscrA \). Therefore \( \bigcup \mscrA \) is itself an inductive set.
\end{proof}

\begin{proposition}\label{thm:smallest_inductive_set_existence}\mcite{MathSE:smallest_inductive_set_existence}
  There exists a \hyperref[def:poset_extremal_points/maximum_and_minimum]{smallest} inductive set \( \mscrA \). We denote this set by \( \omega \).
\end{proposition}
\begin{proof}
  We cannot prove this theorem by taking the intersection of all inductive sets since the later is not a set in the set of \logic{ZFC} and we want all of our theorems to be valid in \logic{ZFC}. Hence we will use another approach.

  Let \( A \) be an inductive set. Define \( \omega \) to be the intersection of all inductive subsets of \( A \). Then \( \omega \) is contained in any other inductive set \( B \). Indeed, \( A \cap B \) is inductive and \( \omega \subseteq A \cap B \subseteq A \). Thus \( \omega \) is indeed the smallest inductive set (with respect to set inclusion).
\end{proof}

\begin{remark}\label{rem:proper_class}
  Within the context of \hyperref[def:zfc]{\logic{ZFC}}, must less sets exist compared to \hyperref[def:naive_set_theory]{na\"ive set theory}. We use the term \term{class} to refer to sets within na\"ive set theory and the term \term{set} to refer to sets within \logic{ZFC}. A \term{proper class} is then a class that is not a set. Since using classes effectively pushes us back to the inconsistent na\"ive set theory, we avoid using classes in our formalisms.

  Here is another outlook on classes. The domain \( U \) of the model in \fullref{def:set} is also a set but it is a set within the metalogic --- see \fullref{rem:set_definition_recursion}. Thus both \( U \) and \( x \in U \) are sets, however \( x \) is a set within the object logic and \( U \) is a set within the metalogic. It may happen that \( x \) is not a set within the metalogic if \( \mscrU \) is not a set-theoretic model, i.e. a set consisting only of sets. Within the object logic, we have no knowledge of the domain \( U \), much less of any properties on \( U \). We only have access to its elements. Hence \( U \) is never a set within the object logic.

  This leads us to the following alternative to the definition of a class in \fullref{def:set}: A \term{class} is a subset of \( U \) available within the object logic. This definition is tricky to formalize entirely within the object language as any such formalization must modify or enrich the carefully chosen axioms of the theory. As the definition currently stands, it captures the idea that classes are sets constructed using unrestricted comprehension on the universe.

  We avoid needing to formalize this idea by introducing multiple universes --- see \fullref{def:grothendieck_universe}.
\end{remark}
