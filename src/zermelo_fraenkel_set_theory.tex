\subsection{Zermelo-Fraenkel set theory}\label{subsec:zermelo_fraenkel_set_theory}

\begin{definition}\label{def:zfc}\mcite[271]{Enderton1977Sets}
  We introduce \term{Z}ermelo – \term{F}raenkel set theory with the axiom of choice (ZF\term{C}). Consider the \hyperref[def:first_order_language]{first-order logic language} with equality \( = \), no functional symbols and a single binary predicate \( \in \). Note that we can take the language not to have formal equality and then use \fullref{def:zfc/A1} as an axiom schema to define equality in terms of \( \in \).

  Given a unary formula \( \varphi(x) \), we can construct a (syntactic) object
  \begin{equation*}
    A = \{ a \colon \varphi(a) \}
  \end{equation*}
  that we call a \term{class}. Not all classes can be defined to have meaningful semantics (e.g. the class of all classes easily leads to paradoxes like \fullref{ex:russels_paradox}). We define sets in ZFC as classes with semantics given by a model for the following axioms (exclude \fullref{def:zfc/A8} to obtain ZF). Classes that do not satisfy these axioms are called \term{proper classes} and are often said to be \term{too big} to be sets, e.g. the class of all sets or the class of all vector spaces). In this document, our main limitation when working with classes rather than sets is not being able to talk about a class not being a member of another class, however this is also not necessary for us.

  \begin{thmenum}
    \thmitem[def:zfc/A1]{A1}(extensionality) Two sets are equal if they have the same elements (given by set membership)

    \thmitem[def:zfc/A2]{A2}(empty set) The following class is a set
    \begin{equation*}
      \varnothing \coloneqq \{ x \colon x \neq x \}.
    \end{equation*}

    \thmitem[def:zfc/A3]{A3}(pairing) If \( A \) and \( B \) are sets, then
    \begin{equation*}
      \{ A, B \}
    \end{equation*}
    is also a set. In particular, \( \{ A \} = \{ A, A \} \) is a set.

    \thmitem[def:zfc/A4]{A4}(union) If \( A \) is a set, then \( \bigcup A \) (see \fullref{def:set_union}) is also a set.

    \thmitem[def:zfc/A5]{A5}(power set) If \( A \) is a set, \( \pow(A) \) (see \fullref{def:power_set}) is also a set.

    \thmitem[def:zfc/A6]{A6}(specification) If \( A \) is a set and \( \varphi \) is a formula, then
    \begin{equation*}
      \{ x \in A \colon \varphi(x) \}
    \end{equation*}
    is a set.

    \thmitem[def:zfc/A7]{A7}(infinity) There exists an \hyperref[def:inductive_set]{inductive set}.

    \thmitem[def:zfc/A8]{A8}(choice; see \fullref{thm:axiom_of_choice_equivalences}) Let \( \mscrM \neq 0 \) and for all \( m \in M \), let \( X_m \) be a nonempty set and \( X_k \cap X_m = \varnothing \) whenever \( m \neq k \). Then there exists a set \( M \) such that for every \( m \in M \), the intersection \( M \cap X_m \) (see \fullref{def:set_intersection}) has exactly one member.

    \thmitem[def:zfc/A9]{A9}(replacement) Given a set \( X \) and a formula \( \varphi(x, y) \), if for every set \( x \in X \) there exists a unique set \( y \) such that \( \varphi(x, y) \) holds, then
    \begin{equation*}
      Y \coloneqq \{ y \colon \exists x \in X, \varphi(x, y) \}
    \end{equation*}
    is a set.

    \thmitem[def:zfc/A10]{A10}(regularity) For every nonempty set \( A \), there exists a member \( a \in A \) such that
    \begin{equation*}
      a \cap A \neq \varnothing.
    \end{equation*}
  \end{thmenum}
\end{definition}

\begin{proposition}\label{thm:zfc_no_set_is_member_of_itself}
  \todo{Prove that in ZFC no set is a member of itself}
\end{proposition}

\begin{definition}\label{def:ordinal_successor_operator}\mcite[68]{Enderton1977Sets}
  For any set \( X \), we define the \term{successor} operation
  \begin{equation*}
    S(X) \coloneqq X \cup \{ X \}.
  \end{equation*}
\end{definition}

\begin{definition}\label{def:inductive_set}\mcite[68]{Enderton1977Sets}
  A set \( A \) is called \term{inductive} if
  \begin{thmenum}
    \item \( \varnothing \in A \)
    \item \( a \in A \implies S(a) \in A \)
  \end{thmenum}
\end{definition}

\begin{definition}\label{def:smallest_inductive_set}
  The smallest \hyperref[def:inductive_set]{inductive set} is
  \begin{equation*}
    \omega \coloneqq \bigcap \{ A \colon A \text{ is an inductive set} \}.
  \end{equation*}

  The elements of \( \omega \) are
  \begin{equation*}
    \varnothing, S(\varnothing), S(S(\varnothing)),
  \end{equation*}
  where \( S \) is the set-theoretic successor operator (see \fullref{def:ordinal_successor_operator}).

  These numbers are called \term{von Neumann ordinals}.
\end{definition}

\begin{proposition}\label{thm:binary_cartesian_product_is_set}
  If \( A \) and \( B \) are sets, their product \( A \times B \) is also a set.
\end{proposition}
\begin{proof}
  Fix \( a \in A \) and \( b \in B \).
  \begin{itemize}
    \item \( \{ a \} \) is a set by \fullref{def:zfc/A6} since \( \{ a \} \subseteq A \)
    \item \( A \cup B \) is a set by \fullref{def:set_union}
    \item \( \{ a, b \} \) is a set by \fullref{def:zfc/A6} since \( \{ a \} \subseteq A \cup B \)
    \item \( (a, b) = \{ \{ a \}, \{ a, b \} \} \) is a set by \fullref{def:zfc/A6} since \( (a, b) \subseteq \pow(A \cup B) \).
  \end{itemize}

  Thus \( A \times B \) is a set since \( A \times B \subseteq \pow(\pow(A \cup B)) \).
\end{proof}
