\section{Combinatorics}\label{sec:combinatorics}

Combinatorics originated as the study of problems related to counting, although this study is now referred to as \term{enumerative combinatorics} to distinguish it from more abstract subfields. A canonical example of a counting problem is given in \fullref{ex:fibonacci_rabbits}. On the other hand, \fullref{thm:gamma_function_interpolates_factorial} provides an example of how mathematical analysis can answer questions related to counting. The latter subfield is called \term{analytic combinatorics}.

Results in combinatorics are traditionally obtained for \hyperref[def:set_finiteness]{finite sets} and \hyperref[rem:peano_arithmetic_zero/positive]{positive integers}, however many of them can be easily generalized. For example, \fullref{def:pigeonhole_principle} is stated in terms of cardinals rather than positive integers and \hyperref[def:directed_graph]{graphs} are not necessary finite.

Some basic definitions and theorems are stated in \fullref{subsec:enumerative_combinatorics} and \fullref{subsec:progressions}, however most of the section is concerned with \hyperref[rem:directed_and_undirected_graphs]{graphs} --- see \fullref{subsec:graphs}, \fullref{subsec:trees}, \fullref{subsec:multigraphs} and \fullref{subsec:hypergraphs}.

As usual, in this section \( \BbbK \) will refer to either the field \( \BbbR \) of \hyperref[def:set_of_real_numbers]{real numbers} or the field \( \BbbC \) of \hyperref[def:set_of_real_numbers]{complex numbers}. This restriction is justified by \fullref{rem:real_field_extensions}.

\begin{example}\label{ex:fibonacci_rabbits}
  A simple, but nontrivial counting problem is due to Fibonacci. In \cite{MacTutor:fibonacci}, the problem is formulated as follows:
  \begin{displayquote}
    A certain man put a pair of rabbits in a place surrounded on all sides by a wall. How many pairs of rabbits can be produced from that pair in a year if it is supposed that every month each pair begets a new pair which from the second month on becomes productive?
  \end{displayquote}

  The rabbit behavior described in ridiculously idealized. Pregnancy time, for example, seems to be nonexistent in this problem. We are interested in counting, however, and not in biology. Nonetheless, the problem as it is stated is still open to interpretations. To obtain Fibonacci's result, we must add, some further assumptions:
  \begin{itemize}
    \item The goal is to find the cumulative rabbit count, including the first pair.
    \item No crossbreeding is assumed --- each rabbit is monogamous and is part of a predefined pair.
    \item The rabbits do not die, which is a realistic assumption for a short time period in case the rabbits are taken care of.
    \item During the first two months, no rabbit is born. That is, the original pair must also wait for two months prior to producing offspring.
  \end{itemize}

  Instead of trying to give a direct answer to the problem, the usual approach is to iteratively build a sequence \( \seq{ F_k }_{k=1}^\infty \), the \term{Fibonacci sequence}, indicating the cumulative number of pairs of rabbits each month.

  As mentioned, during the first two months no offspring is produced, so we only have one pair of rabbits. Thus, \( F_1 = F_2 = 1 \). In general, the number of rabbits on month \( n \) is the sum of:
  \begin{itemize}
    \item The number of existing pairs on month \( n - 1 \), which is \( F_{n - 1} \).
    \item The number of newborn pairs, which is precisely the number of mature pairs (those born at least two months ago). This is \( F_{n - 2} \).
  \end{itemize}

  The entire sequence can thus be built using the following recursive definition:
  \begin{equation*}
    F_k \coloneqq \begin{cases}
      1,                &k = 1 \T{or} k = 2, \\
      F_{k-1} + F_{k-2} &k > 2.
    \end{cases}
  \end{equation*}

  Just to give an answer to Fibonacci's question, we will list the first 12 Fibonacci numbers:
  \begin{equation*}
    1, 1, 2, 3, 5, 8, 13, 21, 34, 55, 89, 144, 233
  \end{equation*}

  It should also be noted that outside the rabbit problem, the Fibonacci sequence is often defined to start at the zeroth month with a value of zero --- this is how it is defined in \fullref{rem:natural_number_recursion} actually. So
  \begin{equation}\label{eq:ex:fibonacci_rabbits}
    F_k \coloneqq \begin{cases}
      0,                &k = 0, \\
      1,                &k = 1, \\
      F_{k-1} + F_{k-2} &k > 2.
    \end{cases}
  \end{equation}
  is actually a more conventional definition.
\end{example}
