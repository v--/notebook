\section{Numbers}\label{sec:numbers}

Numbers are perhaps the most ubiquitous concept in mathematics. Even among non-mathematicians, division by zero or \( 0.999\ldots = 1 \) seem to be a common topic of discourse, either as a joke or a sincere misunderstanding.

The aforementioned topics were studied extensively by mathematicians and have simple justifications from the point of view of abstract mathematics:
\begin{itemize}
  \item We may want to somehow define division by zero in the \hyperref[def:set_of_real_numbers]{set \( \BbbR \) of real numbers}, however that would make it a \hyperref[def:semiring/commutative_unital_ring]{commutative unital ring} rather than a \hyperref[def:field]{field}, which, by \fullref{thm:semiring_cancellative_iff_no_zero_divisors}, would deprive us of the cancellative property of multiplication.

  Our familiar arithmetic of real numbers heavily relies on the cancellative property, therefore we simply disallow division by zero.

  \item The set \( \BbbR \) of real numbers is a uniform space and thus every \hyperref[def:fundamental_net]{fundamental sequence} is convergent by \fullref{thm:cauchys_net_convergence_criterion}. Furthermore, since \( \BbbR \) is also a \hyperref[def:separation_axioms/T2]{Hausdorff} space, by \fullref{thm:t2_iff_singleton_limits}, every fundamental sequence has a unique limit.

  Now consider the following two fundamental sequences:
  \begin{align*}
    &1, 1, 1, 1, \ldots, \\
    &0, 0.9, 0.99, \ldots.
  \end{align*}

  Their difference
  \begin{equation*}
    1, 0.1, 0.01, \ldots
  \end{equation*}
  converges to \( 0 \).

  Therefore, the two original sequences converge to the same real number, namely \( 1 \).
\end{itemize}

Unfortunately, a formal study of numbers also leads to artifacts such as the nonstandard natural numbers discussed in \fullref{rem:standard_models_of_arithmetic}.

We will describe some basic properties of the common number systems:
\begin{itemize}
  \item The \hyperref[def:set_of_natural_numbers]{set \( \BbbN \) of natural numbers} from the perspective of \fullref{sec:mathematical_logic}.
  \item The \hyperref[def:set_of_integers]{set \( \BbbZ \) of integers} from the perspective of \fullref{sec:commutative_algebra}.
  \item The \hyperref[def:set_of_rational_numbers]{set \( \BbbQ \) of rational numbers}, only briefly mentioned.
  \item The \hyperref[def:set_of_real_numbers]{set \( \BbbR \) of real numbers} from the perspective of \fullref{sec:real_analysis}.
  \item The \hyperref[def:set_of_real_numbers]{set \( \BbbC \) of complex numbers} from the perspective of \fullref{sec:complex_analysis} and \fullref{sec:commutative_algebra}.
\end{itemize}
