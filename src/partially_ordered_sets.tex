\subsection{Partially ordered sets}\label{subsec:partially_ordered_sets}

\hyperref[def:preordered_set]{Preordered sets} are simple to define and arise naturally (e.g. \fullref{def:lindenbaum_tarski_algebra}), but they have uniqueness considerations (see \fullref{ex:preorder_nonuniqueness}) that we usually want to factor out. \Fullref{thm:preorder_to_partial_order} shows that the result obtained from this factorization is a partially ordered set, which this section is dedicated to.

\begin{definition}\label{def:partially_ordered_set}
  A \term{partial order} on a set \( \mscrP \) can be defined in the following equivalent ways:
  \begin{thmenum}[series=def:partially_ordered_set]
    \thmitem{def:partially_ordered_set/nonstrict} A \hyperref[def:preordered_set]{preorder} \( \leq \) on \( \mscrP \) such that \( \leq \) is \hyperref[def:binary_relation/antisymmetric]{antisymmetric} in addition to being \hyperref[def:binary_relation/reflexive]{reflexive} and \hyperref[def:binary_relation/transitive]{transitive}. This definition is the more common one. If we wish to distinguish it from the other definition, we call such a relation a \term{nonstrict partial order}.

    \thmitem{def:partially_ordered_set/strict} An \hyperref[def:binary_relation/irreflexive]{irreflexive} and \hyperref[def:binary_relation/transitive]{transitive} binary relation \( < \) on \( \mscrP \). This relation is called a \term{strict partial order}.
  \end{thmenum}

  If both relations are present, in order for the them to be equivalent, \( \leq \) must be the union of \( < \) and the \hyperref[def:binary_relation/diagonal]{diagonal} \( \Delta \). This condition corresponds to the following axiom:
  \begin{equation}\label{def:partially_ordered_set/compatibility_nonstrict}
    (x \leq y) \leftrightarrow \parens[\Big]{(x < y) \vee (x = y)}.
  \end{equation}

  By adding \( \placeholder \wedge \neg (x = y) \) to both sides of \eqref{def:partially_ordered_set/compatibility_nonstrict}, using \fullref{thm:de_morgans_laws} and taking irreflexivity of \( < \) into account, we obtain
  \begin{equation}\label{def:partially_ordered_set/compatibility_strict}
    (x < y) \leftrightarrow \parens[\Big]{(x \leq y) \wedge \neg (x = y)}.
  \end{equation}

  A set \( \mscrP \) with any of the two types of partial orders is called a \term{partially ordered set} or simply a \term{partially ordered set}.

  The elements \( x, y \in \mscrP \) are called \term{comparable} if either \( x \leq y \) or \( y \leq x \). That is, they are comparable if they are related by \( \leq \).

  Partially ordered sets have the following metamathematical properties:
  \begin{thmenum}[resume=def:partially_ordered_set]
    \thmitem{def:partially_ordered_set/theory} Since we can interdefine nonstrict and strict orders, it makes little sense to study different theories for the two.

    In order to define the theory of partially ordered sets, we extend the language of the \hyperref[def:preordered_set/theory]{theory of preordered sets} with two binary infix predicate symbol --- \( < \) and \( > \). We then add to the axioms of the theory \eqref{def:binary_relation/antisymmetric} for \( \leq \) and either of the compatibility conditions \eqref{def:partially_ordered_set/compatibility_nonstrict} or \eqref{def:partially_ordered_set/compatibility_strict} (it is unnecessary to add both).

    We can also add \eqref{def:binary_relation/irreflexive} and \eqref{def:binary_relation/transitive} for \( < \), but that would be redundant.

    \thmitem{def:partially_ordered_set/homomorphism} We have two types of \hyperref[def:first_order_homomorphism]{first-order homomorphisms} between partially ordered sets. The homomorphisms are often called \term{order homomorphisms} or \term{monotone maps}.

    A \term{nonstrict monotone map} from \( (\mscrP, \leq_\mscrP) \) to \( (\mscrQ, \leq_\mscrQ) \) is a homomorphism for the theory containing only \( \leq \) and \( \geq \):
    \begin{equation}\label{eq:def:partially_ordered_set/homomorphism/nonstrict}
      x \leq_\mscrP y \T{implies} f(x) \leq_\mscrQ f(y).
    \end{equation}

    Nonstrict homomorphisms are used extensively in the theory of \hyperref[subsec:partially_ordered_sets]{partially ordered sets}, in particular in \hyperref[subsec:lattices]{lattice theory}.

    A \term{strict monotone map} is instead a homomorphism for the full theory:
    \begin{equation}\label{eq:def:partially_ordered_set/homomorphism/strict}
      x <_\mscrP y \T{implies} f(x) <_\mscrQ f(y).
    \end{equation}

    Strict homomorphisms are used in the theory of \hyperref[subsec:partially_ordered_sets]{totally ordered sets}, in particular for \hyperref[subsec:well_ordered_sets]{well-ordered sets} and \hyperref[subsec:ordinals]{ordinals}.

    See \fullref{thm:monotone_map_converse} for the converse implications to those in \eqref{eq:def:partially_ordered_set/homomorphism/nonstrict} and \eqref{eq:def:partially_ordered_set/homomorphism/strict}.

    In particular, if \( \mscrP \) is the preordered set of \hyperref[rem:peano_arithmetic_zero/nonnegative]{positive integers}, then we speak of \term{monotone sequences}
    \begin{equation}\label{eq:def:partially_ordered_set/homomorphism/sequence}
      \seq{ x_k }_{k=1}^\infty,
    \end{equation}
    where \( x_{k-1} \leq_Q x_k \) for all \( k \in 1, 2, 3, \ldots \).

    To elaborate, order homomorphisms can either be strict or nonstrict depending on whether we include \( < \) and \( > \) as predicate symbols in the language of the theory. We will use the convention that even when the language does include \( < \) and \( > \), by \enquote{order homomorphism} we will understand nonstrict monotone.

    Furthermore, \fullref{thm:totally_ordered_strong_homomorphism} shows that for totally ordered sets, strict order homomorphisms are precisely the strong order homomorphisms (in the sense of \fullref{rem:first_order_strong_homomorphism}).

    \thmitem{def:partially_ordered_set/submodel} As for preordered sets, any subset of a partially ordered set is itself a partially ordered set.

    \thmitem{def:partially_ordered_set/trivial} The \hyperref[thm:substructures_form_complete_lattice/bottom]{trivial partially ordered set} is the empty set (see \fullref{rem:empty_models} regarding allowing empty sets as first-order structures)..

    \thmitem{def:partially_ordered_set/category} We denote the \hyperref[def:category_of_first_order_models]{category of models} by \( \cat{Pos} \). It is a full subcategory of the \hyperref[def:preordered_set/category]{category of preordered sets}.

    \thmitem{def:partially_ordered_set/duality} The \hyperref[def:preordered_set/duality]{principle of duality for preordered sets} holds for partially ordered sets if we also swap \( < \) and \( > \).
  \end{thmenum}
\end{definition}
\begin{proof}
  \ImplicationSubProof{def:partially_ordered_set/nonstrict}{def:partially_ordered_set/strict} Let \( \leq \) be a nonstrict partial order. We will show that \( < \) is a strict partial order.

  \begin{itemize}
    \item The relation \( < \) is \hyperref[def:binary_relation/transitive]{transitive}. To see this, let \( x < y \) and \( y < z \). In particular, \( x \leq y \) and \( y \leq z \). From transitivity we have \( x \leq z \).

    Additionally, \( x \neq y \) and \( y \neq z \). Assume that \( x = z \). From reflexivity of \( \leq \) we have \( z \leq x \) and, since \( y \leq z \), from transitivity we obtain \( y \leq x \). But since \( x \leq y \), from the antisymmetry of \( \leq \), we have \( x = y \), which contradicts the assumption that \( x < y \).

    Therefore, \( x < z \).

    \item \hyperref[def:binary_relation/irreflexive]{Irreflexivity} of \( < \) follows directly from reflexivity of \( \leq \) and the compatibility condition.
  \end{itemize}

  Since the right side is false, the left side \( x < x \) is also false.

  \ImplicationSubProof{def:partially_ordered_set/strict}{def:partially_ordered_set/nonstrict} Let \( < \) be a strict partial order. We will show that \( \leq \) is a nonstrict partial order.

  \begin{itemize}
    \item To see \hyperref[def:binary_relation/reflexive]{reflexivity}, fix \( x \in \mscrP \) and assume that \( x \not\leq x \). Then \( x \neq x \) which contradicts the reflexivity of equality. Hence, \( x \leq x \).

    \item To see \hyperref[def:binary_relation/antisymmetric]{antisymmetry}, let \( x \leq y \) and \( y \leq x \), that is, either \( x = y \) or both \( x < y \) and \( y < x \) hold. Assume the latter. By the transitivity of \( \leq \), we have \( x < x \), which contradicts the irreflexivity of \( < \). Hence, \( x = y \).

    \item To see \hyperref[def:binary_relation/transitive]{transitivity}, let \( x \leq y \) and \( y \leq z \). Then we have four cases depending on which of \( x \), \( y \) and \( z \) are equal. Since both relations \( < \) and \( = \) are transitive, it follows that in all four cases \( x \leq z \).
  \end{itemize}
\end{proof}

\begin{definition}\label{def:hasse_diagram}
  It is usually easier to define small finite partially ordered sets by drawing graphs than by enumerating all relation pairs. Let \( (\mscrP, \leq) \) be a finite partially ordered set. The relation \( \leq \) may also be regarded as the set of edges of a \hyperref[def:graph/directed]{directed graph}. The graph \( (\mscrP, \red^T(\leq)) \), whose edges are the \hyperref[def:derived_relations/transitive]{transitive reduction} of \( \leq \), is called the \term{Hasse graph} or \term{Hasse diagram} of \( \mscrP \).

  The term \enquote{Hasse diagram} is usually associated with drawings. By convention, no arrowheads for denoting directions are drawn on the Hasse graph despite the graph being directed; instead, edges always point upwards. See \fullref{ex:def:hasse_diagram}.
\end{definition}

\begin{example}\label{ex:def:hasse_diagram}
  Consider the partial order over \( \set{ a, b, c, d, e } \) defined via
  \begin{equation}\label{eq:ex:def:hasse_diagram/partially ordered set}
    \underline{a \leq c},\quad \underline{a \leq d},\quad a \leq e,\quad \underline{b \leq d},\quad b \leq e,\quad \underline{d \leq e}.
  \end{equation}

  The corresponding Hasse graph includes only the underlined edges. The rest of the edges can be restored from transitivity. In this case, the Hasse graph has edges
  \begin{equation}\label{eq:ex:def:hasse_diagram/hasse_graph}
    \set{ a \to c, a \to d, b \to d, d \to e }
  \end{equation}

  \begin{figure}
    \centering
    \includegraphics{figures/ex__def__hasse_diagram.pdf}
    \caption{A drawing of the Hasse diagram \eqref{eq:ex:def:hasse_diagram/hasse_graph}}
    \label{fig:ex:def:hasse_diagram}
  \end{figure}
\end{example}

\begin{example}\label{ex:preorder_nonuniqueness}
  Consider the preordered set \( \mscrP \) in \cref{fig:ex:preorder_nonuniqueness} in which \( b \leq c \) and \( c \leq b \), but \( b \neq c \). We cannot properly draw a \hyperref[def:hasse_diagram]{Hasse diagram} because we have the restriction that \( c \) is drawn (strictly) higher than \( b \) if \( c > b \) and that \( c \) is drawn lower than \( b \) if \( c < b \). We face a similar problem formally, for example in the definition of a \hyperref[def:lindenbaum_tarski_algebra]{Lindenbaum-Tarski algebra} of a \hyperref[def:first_order_theory]{logical theory}, where the preorder \( \vdash \) allows \( \varphi \vdash \psi \) and \( \psi \vdash \varphi \), but still \( \varphi \neq \psi \). Thus, we have nonuniqueness --- every tautology is a largest element with respect to \( \vdash \), while we want to have a single largest element for the sake of building a tidier theory.

  If we are only interested in members of \( \mscrP \) up to the equivalence \eqref{eq:thm:preorder_to_partial_order/equivalence}, it is easy to factor \( \mscrP \) by the equivalence relation \eqref{eq:thm:preorder_to_partial_order/equivalence} and obtain a partially ordered set. In the language of graph theory, if we have \hyperref[def:graph_paths/directed_path]{circuits} that we may wish to avoid, we can contract each circuit into a single vertex, at which point the graph becomes acyclic.

  The formulation and proof of correctness of this process can be found in \fullref{thm:preorder_to_partial_order} and an example can be found in \cref{fig:ex:preorder_nonuniqueness}.

  \begin{figure}
    \hfill
    \includegraphics{figures/ex__preorder_nonuniqueness__preordered.pdf}
    \hfill
    \includegraphics{figures/ex__preorder_nonuniqueness__factorized.pdf}
    \hfill\hfill
    \caption{A preordered set and its induced partially ordered set.}
    \label{fig:ex:preorder_nonuniqueness}
  \end{figure}
\end{example}

\begin{proposition}\label{thm:preorder_to_partial_order}
  Let \( (\mscrP, \leq) \) be a preordered set. Define the relation \( \cong \) by
  \begin{equation}\label{eq:thm:preorder_to_partial_order/equivalence}
    x \cong y \T{if and only if} x \leq y \T{and} y \leq x.
  \end{equation}

  That is, \( \cong \) is the intersection of the relation \( \leq \) with its \hyperref[def:binary_relation/converse]{inverse}.

  Since \( \cong \) is an \hyperref[def:equivalence_relation]{equivalence relation} we can for the the quotient set \( \mscrP / \cong \). Define the relation \( \preceq \) on this quotient set by
  \begin{equation*}
    [x] \preceq [y] \T{if and only if} x \leq y.
  \end{equation*}

  The pair \( (\mscrP / \cong, \preceq) \) is then a \hyperref[def:partially_ordered_set]{partially ordered set}.
\end{proposition}
\begin{proof}
  The relation \( \preceq \) is well-defined. Indeed, let \( x \cong x' \) and \( y \cong y' \), that is, both \( x \leq x' \) and \( x' \leq x \) and similarly for \( y \). If \( x \leq y \), from transitivity \( x \leq y \leq y' \). But \( x' \leq x \), hence \( x' \leq y' \).

  It is then clear that \( \preceq \) is a partial order because it inherits reflexivity and transitivity from \( \leq \) and antisymmetry is imposed by taking quotient sets --- equality in \( \mscrP / \cong \) holds precisely when \( \cong \) holds in \( \mscrP \).

  Thus, \( (\mscrP / \cong, \preceq) \) is indeed a partially ordered set.
\end{proof}

\begin{proposition}\label{thm:partial_order_category_correspondence}
  To every \hyperref[def:partially_ordered_set]{partially ordered set} there corresponds exactly one \hyperref[def:small_and_large_categories]{small} \hyperref[def:thin_category]{thin} \hyperref[def:skeletal_category]{skeletal} category.

  Compare this result to \fullref{thm:preorder_category_correspondence}.
\end{proposition}
\begin{proof}
  The statement follows from \fullref{thm:preorder_category_correspondence} by noting that the factorization in \fullref{thm:preorder_to_partial_order} makes the corresponding category skeletal.
\end{proof}

\begin{proposition}\label{thm:order_embedding_is_strict}
  For any two \hyperref[def:preordered_set]{preordered sets} \( (\mscrP, \leq_\mscrP) \) and \( (\mscrQ, \leq_\mscrQ) \), every \hyperref[def:partially_ordered_set/homomorphism]{order embedding} of \( \mscrP \) into \( \mscrQ \) is a \hyperref[def:partially_ordered_set/homomorphism]{strict order embedding}.

  Compare this to \fullref{thm:total_order_embedding_iff_strict}
\end{proposition}
\begin{proof}
  Let \( f: \mscrP \to \mscrQ \) be an order embedding.

  Let \( x <_\mscrP y \) for some members \( x \) and \( y \) of \( \mscrP \). Since \( f \) is an order homomorphism, we have \( f(x) \leq_\mscrQ f(y) \). Since it is also injective, \( f(x) = f(y) \) implies \( x = y \), which contradicts our previous assumption.

  Therefore, \( f(x) <_\mscrQ f(y) \) and \( f \) is a strict order homomorphism.
\end{proof}

\begin{proposition}\label{thm:monotone_map_converse}
  For any (even nonstrict) monotone map \( f: \mscrP \to \mscrQ \) between partially ordered sets, if \( x \) and \( y \) are comparable elements of \( \mscrP \) we have
  \begin{equation}\label{eq:thm:monotone_map_converse}
    x <_\mscrP y \T{if and only if} f(x) <_\mscrQ f(y).
  \end{equation}
\end{proposition}
\begin{proof}
  Let \( f(x) <_\mscrQ f(y) \) and suppose that \( x \geq y \). Since \( f \) is a monotone map, we have \( f(x) \geq_\mscrQ f(y) \), which is a contradiction.
\end{proof}

\begin{definition}\label{def:partially_ordered_set_extremal_points}
  We introduce the following terminology for extremal elements of a partially ordered set \( \mscrP \). Analogous definition can be given for preordered sets, but the nonuniqueness problems outlined in \fullref{ex:preorder_nonuniqueness} highlight that there are sometimes difficulties in doing, so.

  The notions on the left and on the right are \hyperref[def:partially_ordered_set/duality]{dual}, but we discuss both nonetheless.

  \begin{thmenum}
    \thmitem{def:partially_ordered_set_extremal_points/upper_and_lower_bounds}\mcite[2]{Gratzer1978}
    \begin{minipage}[t]{0.45\textwidth}
      An \term{upper bound} for the set \( A \subseteq \mscrP \) is an element \( x_0 \in \mscrP \) such that \( x \leq x_0 \) for every \( x \in A \). Note that \( x_0 \) does not in general belong to \( A \). An upper bound is called \term{strict} if it does not belong to \( A \).

      If \( A \) has at least one upper bound, it is called \term{bounded from above}.

      Every element is vacuously an upper bound of \( A = \varnothing \).

      In \cref{fig:ex:def:hasse_diagram}, the set \( A = \set{ a, b } \) is bounded from above by both \( d \) and \( e \), but the entire partially ordered set has no upper bound.
    \end{minipage}
    \hspace{0.02\textwidth}
    \begin{minipage}[t]{0.45\textwidth}
      Dually, \( x_0 \in \mscrP \) is a \term{lower bound} of \( A \) if \( x_0 \leq x \) for every \( x \in A \). A lower bound is called \term{strict} if it does not belong to \( A \). If \( A \) has a lower bound, it is called \term{bounded from below}.

      If \( A \) is bounded both from below and from above, we say that \( A \) is \term{bounded}.

      Every element is vacuously a lower bound of \( A = \varnothing \). Hence, the empty set is bounded.

      In \cref{fig:ex:def:hasse_diagram}, the entire partially ordered set has no lower bound. The set \( A = \set{ c, d } \) is bounded from below by \( a \), but not from above, hence \( A \) is not bounded.
    \end{minipage}

    \thmitem{def:partially_ordered_set_extremal_points/maximal_and_minimal_element}
    \begin{minipage}[t]{0.45\textwidth}
      A \term{maximal element} for the set \( A \subseteq \mscrP \) is a member \( x_0 \) of \( A \) such that there is no greater element in \( A \) than \( x_0 \). More precisely, \( x_0 \) is a maximal element of \( A \) if for every element \( x \in A \) such that \( x \leq x_0 \) we have \( x = x_0 \).

      In \cref{fig:ex:def:hasse_diagram}, the entire partially ordered set has two incomparable maximal elements --- \( c \) and \( e \).
    \end{minipage}
    \hspace{0.02\textwidth}
    \begin{minipage}[t]{0.45\textwidth}
      The member \( x_0 \in A \) is a \term{minimal element} of \( A \) if for every \( x \in A \) such that \( x \leq x_0 \) we have \( x = x_0 \).

      The empty set cannot have maximal or minimal elements because it has no members.

      In \cref{fig:ex:def:hasse_diagram}, the entire partially ordered set has two incomparable minimal elements --- \( a \) and \( b \).
    \end{minipage}

    \thmitem{def:partially_ordered_set_extremal_points/maximum_and_minimum}
    \begin{minipage}[t]{0.45\textwidth}
      The \term{maximum} or \term{greatest element} of \( A \subseteq \mscrP \), if it exists, is an upper bound of \( A \) that belongs to \( A \). A maximum is necessarily a maximal element because \( x_0 \leq x \) only holds for \( x = x_0 \), which also demonstrates uniqueness of \( x_0 \). See \fullref{ex:unique_maximal_element_that_is_not_maximum} for a unique maximal element that is not a maximum.
    \end{minipage}
    \hspace{0.02\textwidth}
    \begin{minipage}[t]{0.45\textwidth}
      The \term{minimum} of \( A \), also called \term{smallest element} or \term{least element}, is a lower bound that belongs to \( A \).

      The empty set cannot have a maximum or minimum because it has no members.

      In \cref{fig:ex:def:hasse_diagram}, the entire partially ordered set has no maximum, but the set \( A = \mscrP \setminus \set{ b } = \set{ a, b, d, e } \) has \( a \) as its minimum.
    \end{minipage}

    \thmitem{def:partially_ordered_set_extremal_points/supremum_and_infimum}\mcite[2]{Gratzer1978}
    \begin{minipage}[t]{0.45\textwidth}
      The \term{supremum} \( \sup A \) of \( A \subseteq \mscrP \), if it exists, is its least upper bound of \( A \), i.e. the \hyperref[def:partially_ordered_set_extremal_points/maximum_and_minimum]{minimum} of the set of its \hyperref[def:partially_ordered_set_extremal_points/upper_and_lower_bounds]{upper bounds}.

      In \cref{fig:ex:def:hasse_diagram}, the entire partially ordered set has no upper bound, so it cannot possibly have a supremum. Obviously every supremum is a maximum, but the converse is not true. We already noted that both \( d \) and \( e \) are upper bounds of the set \( \set{ a, b } \) and since \( d \leq e \), we conclude that \( d \) is the supremum of \( \set{ a, b } \), yet the set \( \set{ a, b } \) has no maximum.
    \end{minipage}
    \hspace{0.02\textwidth}
    \begin{minipage}[t]{0.45\textwidth}
      The \term{infimum} \( \inf A \) of \( A \subseteq \mscrP \) is its greatest lower bound.

      An infimum may fail to exist because the set of lower bounds is nonempty, but has no maximum. This can happen if, for example, we had \( b \leq c \) in \cref{fig:ex:def:hasse_diagram}, in which case both \( a \) and \( b \) would be maximal lower bounds of \( \set{ c, d } \), but none of them would be a greatest lower bound because they are incomparable.
    \end{minipage}

    \thmitem{def:partially_ordered_set_extremal_points/top_and_bottom}\mcite[2]{Gratzer1978}
    \begin{minipage}[t]{0.45\textwidth}
      If it exists, the maximum of the entire partially ordered set \( \mscrP \) is usually denoted by \( \top \) and called the \term{global maximum} or \term{top} element of \( \mscrP \). Since \( \top \) is the maximum of \( \mscrP \), it is also the supremum of \( \mscrP \).

      Since every member of \( \mscrP \) is a lower bound of \( \varnothing \), the greatest lower bound is the maximum of \( \mscrP \).

      In conclusion,
      \begin{equation*}
        \top = \max \mscrP = \sup \mscrP = \inf \varnothing.
      \end{equation*}

      If \( \top \) exists, we say that the partially ordered set \( \mscrP \) itself is bounded from above.
    \end{minipage}
    \hspace{0.02\textwidth}
    \begin{minipage}[t]{0.45\textwidth}
      Dually, the minimum of \( \mscrP \) is usually denoted by \( \bot \) and called the \term{global minimum} or \term{bottom} element of \( \mscrP \).

      The supremum of the empty set is the least of the upper bounds of the empty set, i.e. the minimum of \( \mscrP \), which is \( \bot \).

      In conclusion,
      \begin{equation*}
        \bot = \min \mscrP = \inf \mscrP = \sup \varnothing.
      \end{equation*}

      If \( \bot \) exists, we say that the partially ordered set \( \mscrP \) itself is bounded from below.
    \end{minipage}
  \end{thmenum}
\end{definition}

\begin{example}\label{ex:unique_maximal_element_that_is_not_maximum}\mcite{MathSE:unique_maximal_element_that_is_not_maximum}
  For a more extreme example of the interplay between maximal elements and maxima, adjoin \( \BbbZ \) under the usual order with a new sentinel element \( \star \). Define \( \star \) to satisfy reflexivity, but not be in relation with any integer. Then \( \star \) is a unique maximal element of \( \BbbZ \cup \set{ \star } \), yet the latter set has no largest element.

  This phenomenon is impossible in finite partially ordered sets where a unique maximal element is always a maximum.
\end{example}

\begin{definition}\label{def:partially_ordered_set_chain_and_antichain}\mcite[2]{Gratzer1978}
  A \term{chain} in a partially ordered set is a subset in which every two elements are comparable. The \term{length} of a chain \( A \) is the \hyperref[def:cardinal]{cardinal number} \( \card(A) - 1 \). The \term{length} of a partially ordered set, if it exists, is the maximum among the lengths of all its chains. The length of a partially ordered set is also called its \term{height} because of how Hasse diagrams are drawn..

  An \term{antichain} is a subset in which no two elements are comparable. The \term{width} of a partially ordered set, if it exists, is the maximum among the cardinalities of its antichains.

  The height of the partially ordered set in \cref{fig:ex:def:hasse_diagram} is \( 2 \) and it is reached by the chains \( \set{ a, d, e } \) and \( \set{ b, d, e } \). The width is \( 2 \) are is reached by \( \set{ a, b } \), \( \set{ b, e } \), \( \set{ c, d } \) and \( \set{ c, e } \).
\end{definition}

\begin{definition}\label{def:partially_ordered_set_interval}
  Fix a \hyperref[def:partially_ordered_set]{partially ordered set set} \( (\mscrP, \leq) \). For any \( a, b \in P \) with \( a \leq b \), we define the following related partially ordered sets:

  \begin{thmenum}
    \thmitem{def:partially_ordered_set_interval/ray} The \term{open rays}, also called the \term{open initial segment} and \term{open final segment}, are defined as
    \begin{equation*}
      \begin{aligned}
        (a, \infty) \coloneqq \set{ x \in \mscrP \given b > a } \eqqcolon \mscrP_{>a},
        \\
        (-\infty, b) \coloneqq \set{ x \in \mscrP \given x < b } \eqqcolon \mscrP_{<b}.
      \end{aligned}
    \end{equation*}

    The notation on the left assumes that the sentinel symbols \( \infty \) and \( -\infty \) are adjoined to \( \mscrP \) as in the case of the \hyperref[def:extended_real_numbers]{extended real numbers}. This convention is widespread for unbounded ordered rings representing numbers --- for example \hyperref[def:integers]{\( \BbbZ \)}, \hyperref[def:rational_numbers]{\( \BbbQ \)} and \hyperref[def:set_of_real_numbers]{\( \BbbR \)}. The term \enquote{ray} is used in this context due to the connection with \hyperref[def:geometric_ray]{geometric rays}.

    The notation on the right is more general and is widespread for abstract partial orders, most notably \hyperref[def:well_ordered_set]{well-ordered sets}. In the latter context, they are usually referred to as \enquote{initial/final segments}.

    The \term{closed rays} and \term{closed initial/final segments} are defined analogously as
    \begin{equation*}
      \begin{aligned}
        [a, \infty) \coloneqq \set{ x \in \mscrP \given x \geq a } \eqqcolon \mscrP_{\geq a},
        \\
        (-\infty, b] \coloneqq \set{ x \in \mscrP \given x \leq b } \eqqcolon \mscrP_{\leq b}.
      \end{aligned}
    \end{equation*}

    \thmitem{def:partially_ordered_set_interval/closed} The \term{closed interval} with endpoints \( a \) and \( b \) is
    \begin{equation*}
      [a, b] \coloneqq \set{ x \in \mscrP \given a \leq x \leq b } = \mscrP_{\geq a} \cap \mscrP_{\leq b}.
    \end{equation*}

    We implicitly assume that \( a \leq b \), but this is not strictly necessary --- \( [a, b] \) is an empty set otherwise.

    \thmitem{def:partially_ordered_set_interval/open} The \term{open interval} with endpoints \( a \) and \( b \) is
    \begin{equation*}
      (a, b) \coloneqq \set{ x \in \mscrP \given a < x < b } = \mscrP_{> a} \cap \mscrP_{< b}.
    \end{equation*}

    We implicitly assume that \( a < b \), but this is also not strictly necessary.

    \thmitem{def:partially_ordered_set_interval/half_open} The \term{half-open intervals} are
    \begin{equation*}
      \begin{aligned}
        (a, b] \coloneqq \set{ x \in \mscrP \given a < x \leq b } = \mscrP_{> a} \cap \mscrP_{\leq b},
        \\
        [a, b) \coloneqq \set{ x \in \mscrP \given a \leq x < b } = \mscrP_{\geq a} \cap \mscrP_{< b}.
      \end{aligned}
    \end{equation*}
  \end{thmenum}
\end{definition}

\begin{proposition}\label{thm:partially_ordered_cofinal_equivalences}
  Let \( (\mscrP, \leq) \) be a \hyperref[def:partially_ordered_set_extremal_points/upper_and_lower_bounds]{bounded from above} totally ordered set and let \( A \subseteq \mscrP \). Then \( A \) is \hyperref[def:cofinal_set]{cofinal} if and only if it contains the \hyperref[def:partially_ordered_set_extremal_points/top_and_bottom]{top element} \( \top \).

  Compare this result with \Fullref{thm:totally_ordered_cofinal_equivalences}.
\end{proposition}
\begin{proof}
  \SufficiencySubProof Let \( A \) be a cofinal set. Then \( A \) must contain an element \( x \) such that \( \top \leq x \). But \( \top \) is a maximum and hence \( x = \top \) and thus \( \top \in A \).

  \NecessitySubProof Let \( A \) be a set containing \( \top \). Then for any \( x \in \mscrP \) we have \( x \leq \top \) and hence \( A \) is cofinal.
\end{proof}

\begin{definition}\label{def:lexicographic_order}
  Let \( (\mscrP, \leq_\mscrP) \) and \( (\mscrQ, \leq_\mscrQ) \) be partially ordered sets.

  The \term{lexicographic order} on \( \mscrP \times \mscrQ \), also know as the \term{dictionary order}, is defined as
  \begin{equation}\label{eq:def:lexicographic_order}
    (a, b) \prec (c, d) \thickspace \T{if and only if} \thickspace \parens[\Big]{ a <_\mscrP c \T{or} \parens[\Big]{ a = c \T{and} b <_\mscrQ d } }.
  \end{equation}

  The \term{reverse lexicographic order} is
  \begin{equation}\label{eq:def:lexicographic_order/reverse}
    (a, b) \prec (c, d) \thickspace \T{if and only if} \thickspace \parens[\Big]{ b <_\mscrQ d \T{or} \parens[\Big]{ b = d \T{and} a <_\mscrP c } }.
  \end{equation}

  The lexicographic order on \( \mscrP \times \mscrQ \) inherits some important properties from \( (\mscrP, \mscrP) \) and \( (\mscrQ, \leq_\mscrQ) \) as can be seen in \fullref{thm:lexicographic_order_is_partial_order,thm:total_lexicographic_order_is_total_order,thm:well_ordered_lexicographic_order_is_well_ordered}.

  We can use natural number recursion to extend this to arbitrary \( n \)-tuples --- see \fullref{ex:def:lexicographic_order} for an example.
\end{definition}

\begin{example}\label{ex:def:lexicographic_order}
  A key example for \hyperref[def:lexicographic_order]{lexicographic ordering} is real-world dictionary like a thesaurus. It is obvious that \enquote{homomorphism} should come after \enquote{axiom} and the lexicographic order on a Cartesian power of Latin alphabets suggests that \enquote{homeomorphism} should also come after \enquote{homeomorphic}.

  A slightly more relevant example for mathematics is the lexicographic ordering
  \begin{equation*}
    AB < AD < BC < CD
  \end{equation*}
  of the names of the edges of a rectangle. The reverse lexicographic ordering is
  \begin{equation*}
    AB < BC < AD < CD.
  \end{equation*}

   For the sides of a \hyperref[def:triangle]{triangle} we have \( AB < AC < BC \) for both orderings.

  The edges of the graph in \fullref{ex:def:graph/directed} are numbered in reverse lexicographic order.

  \Fullref{thm:ordinal_addition_disjoin_union} and \fullref{thm:ordinal_multiplication_cartesian_product} contain more interesting applications of lexicographic orders.
\end{example}

\begin{proposition}\label{thm:lexicographic_order_is_partial_order}
  The \hyperref[eq:def:lexicographic_order]{lexicographic} and \hyperref[eq:def:lexicographic_order/reverse]{reverse lexicographic} orders are \hyperref[def:partially_ordered_set/strict]{strict partial order} relations.

  Compare this result to \fullref{thm:total_lexicographic_order_is_total_order} and \fullref{thm:well_ordered_lexicographic_order_is_well_ordered}.
\end{proposition}
\begin{proof}
  \SubProofOf[def:binary_relation/irreflexive]{irreflexivity} Trivial.

  \SubProofOf[def:binary_relation/transitive]{transitivity} Let \( \prec \) be a lexicographic order on \( \mscrP \times \mscrQ \).

  If \( (a, b) \prec (c, d) \) and \( (c, d) \prec (e, f) \), then:
  \begin{itemize}
    \item If \( a < c \), then \( a < c \leq e \) and thus \( (a, b) \prec (e, f) \).

    \item If \( a = c \) and \( b < d \), then \( a \leq e \) and \( b < d \leq f \) and thus \( (a, b) \prec (e, f) \).
  \end{itemize}

   The proof for the reverse lexicographic order is analogous.
\end{proof}

\begin{theorem}[Zorn's lemma]\label{thm:zorns_lemma}\mcite[63]{Gratzer1978}
  If every \hyperref[def:partially_ordered_set_chain_and_antichain]{chain} in a \hyperref[def:partially_ordered_set]{partially ordered set} has an \hyperref[def:partially_ordered_set_extremal_points/upper_and_lower_bounds]{upper bound}, then the entire set has a \hyperref[def:partially_ordered_set_extremal_points/maximal_and_minimal_element]{maximal element}.

  Zorn's lemma is usually stated and used only in a \hyperref[thm:boolean_algebra_of_subsets]{lattice of sets}, however it is a more general statement in order theory.

  In \hyperref[def:zfc]{\logic{ZF}} this theorem is equivalent to the \hyperref[def:zfc/choice]{axiom of choice} --- see \fullref{thm:axiom_of_choice_equivalences/zorns_lemma}.
\end{theorem}
\begin{proof}
  \ImplicationSubProof[def:zfc/choice]{axiom of choice}[thm:zorns_lemma]{Zorn's lemma} Let \( (\mscrP, \leq) \) be a partially ordered set in which every chain has an upper bound. Aiming at a contradiction, suppose that \( \mscrP \) has no maximal element.

  Denote by \( \mscrC \) the set of all chains of \( \mscrP \) and define the multi-valued map \( F: \mscrC \to \mscrP \) that assigns to each chain the set of all strict upper bounds.

  Since \( \mscrP \) has no maximal element, every chain has a strict upper bound. That is, the function \( F \) is a total multi-valued map. By \fullref{thm:existence_of_multi_valued_function_selection}, there exists a single-valued selection \( f: \mscrC \to \mscrP \) of \( F \). We have indirectly used the axiom of choice via \fullref{thm:existence_of_multi_valued_function_selection}.

  By \fullref{thm:hartogs_lemma}, there exists a smallest \hyperref[def:ordinal]{ordinal} \( \gamma \) such that no function from \( \gamma \) to \( \mscrP \) is injective. Using \fullref{thm:bounded_transfinite_recursion}, we can define
  \begin{equation*}
    \begin{aligned}
      &g: \gamma \to A \\
      &g(\delta) \coloneqq f(\set{ g(\varepsilon) \given \varepsilon < \delta }).
    \end{aligned}
  \end{equation*}

  By construction, for every \( \delta < \gamma \) the value \( g(\delta) \) is a strict upper bound of the set \( \set{ g(\varepsilon) \given \varepsilon < \delta } \). Hence, the function \( f \) is injective, which directly contradicts our choice of \( \gamma \).

  The obtained contradiction shows that \( \mscrP \) has a maximal element.

  \ImplicationSubProof[thm:zorns_lemma]{Zorn's lemma}[def:zfc/choice]{axiom of choice} Let \( \mscrA \) be a family of nonempty sets. Let \( \mscrF \) be the set of all \hyperref[def:partial_function]{partial single-valued functions} from \( \mscrA \) to \( \bigcup \mscrA \) with the subset ordering. That is, \( f \leq g \) if \( \dom(f) \subseteq \dom(g) \) for \( f, g \in \mscrF \).

  Clearly every chain has a maximum - a total single-valued function. Then \( \mscrF \) itself has a maximal element by Zorn's lemma. This maximal element is necessarily a total function because otherwise it would not be maximal.

  Then this is the desired choice function for the family \( \mscrA \).
\end{proof}
