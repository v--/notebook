\subsection{Quivers}\label{subsec:quivers}

\begin{definition}\label{def:quiver_arc_set}
  Let \( M = (V, E, h, t) \) be a \hyperref[def:graph/quiver]{quiver}. For every pair of vertices \( u \) and \( v \), we define the \term{arc set}
  \begin{equation}\label{eq:def:quiver_arc_set}
    M(u, v) \coloneqq \set{ e \in E \given h(e) = u \T{and} t(e) = v }.
  \end{equation}

  The vertices \( u \) and \( v \) are \hyperref[def:graph/adjacent_vertices]{incident} to every arc in \( M(u, v) \) and vice versa. We shorten the \term{loop set} \( M(v, v) \) to \( M(v) \).
\end{definition}

\begin{definition}\label{def:locally_finite_quiver}
  Unlike for directed graphs, we say that the quiver \( M \) is \term{locally finite} if the arc set \( M(u, v) \) is finite for every pair of vertices \( u \) and \( v \). This is much less restrictive the local finiteness for directed graphs defined in \fullref{def:graph_incidence/locally_finite}. In a directed graph, every arc set is either empty or a singleton set, hence every directed graph is locally finite according to this definition.

  In a quiver, we have
  \begin{equation*}
    w(v)
    =
    \set[\Big]{ e \in E \given* h(e) = v \T{or} t(e) = v }
    =
    \bigcup_{w \in V} \parens[\Big]{ M(v, w) \cup M(w, v) }.
  \end{equation*}

  If the quiver has infinite order, it is possible for each arc set to be finite and for \( w(v) \) to be infinite. This would make it locally finite with respect to this definition but not with respect to \fullref{def:graph_incidence/locally_finite}.

  This definition is used in category theory and is defined in, for example, \cite{nLab:finite_category}. Local smallness is defined analogously in \cite[75]{Leinster2016Basic}.
\end{definition}

\begin{definition}\label{def:quiver_homomorphism}\mcite[sec. II.7]{MacLane1994}
  Although we haven't defined \hyperref[def:graph/quiver]{quivers} via a \hyperref[def:first_order_theory]{first-order theory}, we have a canonical notion of \hyperref[def:first_order_homomorphism]{homomorphism}, although it is a pair of functions rather than a single function.

  Let \( M = (V_M, E_M) \) and \( N = (V_N, E_N) \) be quivers. A \term{homomorphism pair} from \( M \) to \( N \) is a pair of functions \( f_V: V_M \to V_N \) and \( f_E: E_M \to E_N \) such that
  \begin{align}
    h \bincirc f_e &= f_v \bincirc h \label{eq:def:quiver_homomorphism/head} \\
    t \bincirc f_e &= f_v \bincirc t \label{eq:def:quiver_homomorphism/tail}
  \end{align}
\end{definition}

\begin{definition}\label{def:category_of_qivers}
  In relation to category theory, \hyperref[def:graph/quiver/directed]{directed quivers} are often called \term{quivers}. For this reason, given a \hyperref[def:grothendieck_universe]{Grothendieck universe} \( \mscrU \), we denote the \hyperref[def:category]{category} of \hyperref[def:large_and_small_sets]{\( \mscrU \)-small} quivers by \( \cat{Quiv} \).

  \begin{itemize}
    \item The \hyperref[def:category/C1]{set of objects} \( \obj(\cat{Quiv}) \) is the set of all \( \mscrU \)-small quivers.
    \item The \hyperref[def:category/C2]{set of morphisms} \( \cat{Quiv}(M, N) \) is the set of all \hyperref[def:quiver_homomorphism]{homomorphism pairs} from \( M \) to \( N \).

    \item The \hyperref[def:category/C3]{composition of the morphisms} \( (f_V, f_E): M \to N \) and \( (g_V, g_E): N \to K \) is the morphism \( (g_V \bincirc f_V, g_E \bincirc f_E): M \to K \).
  \end{itemize}
\end{definition}
