\subsection{Quivers}\label{subsec:quivers}

\begin{definition}\label{def:quiver}
  A \term{quiver} or \term{directed multigraph} is an extension of \hyperref[def:undirected_multigraph]{undirected multigraphs}. A quiver consists of the following:
  \begin{thmenum}
    \thmitem{def:quiver/vertices} A set \( V \) of \term{vertices}.
    \thmitem{def:quiver/arcs} A disjoint from \( V \) set \( A \) of \term{arcs}. It is conventional to call them arcs or even directed edges rather than simply edges.
    \thmitem{def:quiver/head} A \term{head} function \( h: A \to V \).
    \thmitem{def:quiver/tail} A \term{tail} function \( t: A \to V \).
    \thmitem{def:quiver/adjacency} We say that the arc \( a \) is a \term{successor} of \( b \) and that \( b \) is a \term{predecessor} of \( a \) if \( t(a) = h(b) \). We also say that \( a \) and \( b \) are \term{consecutive}.

    For vertices, we say that \( w \) is a successor of \( v \) and that \( w \) is consecutive to \( w \) if there exists an arc from \( u \) to \( v \).

    \thmitem{def:quiver/forgetful} For each quiver \( Q = (V, A, h, t) \), we have an \enquote{underlying} undirected multigraph
    \begin{equation}\label{eq:def:quiver/forgetful}
      \begin{aligned}
        &G: A \to V, \\
        &G(a) \coloneqq \set{ h(a), t(a) }.
      \end{aligned}
    \end{equation}

    All properties of \( G \) are inherited by \( Q \), although some of them do not really fit or need adaptation.

    We denote this graph by \( U(Q) \coloneqq G \) and will regard it as a forgetful functor --- see \fullref{def:multigraph_orientation}.

    \thmitem{def:quiver/opposing} One adaptation is that the arcs \( a \) and \( b \) are called \term{opposing} they are successors of each other and the term \enquote{parallel} is reserved for arcs which are parallel in the sense of \fullref{def:hypergraph/parallel_hyperedges} but not opposing.

    \thmitem{def:quiver/simple} If there are no parallel and opposing arcs or loops, we call the quiver \term{simple}. The term \term{simple directed graph} is more commonly used.

    For a simple directed graph, we often write \( G = (V, E) \), where \( E \) is an \hyperref[def:binary_relation/irreflexive]{irreflexive} binary relation.

    We say that the simple directed graph is \term{symmetric} if this relation is symmetric. For many purposes, symmetric directed graphs can be regarded as undirected --- see \fullref{rem:symmetric_directed_graphs}.

    \thmitem{def:quiver/submodel} The \term{subquivers} of \( Q \) are the precisely \hyperref[def:undirected_multigraph/submodel]{subgraphs} of \( G \). In particular, the \term{full subquivers} are the \hyperref[def:undirected_multigraph/submodel]{full subgraphs}, that are maximal among the subgraphs with the same vertices.
  \end{thmenum}

  \begin{figure}
    \begin{equation}\label{eq:fig:def:quiver}
      \begin{aligned}
        \includegraphics[page=1]{output/def__quiver.pdf}
      \end{aligned}
    \end{equation}
    \caption{A quiver with a pair of parallel arcs, a pair of opposing arcs and a loop. Removing the dashed arcs makes it a simple directed graph.}\label{fig:def:quiver}
  \end{figure}
\end{definition}

\begin{example}\label{ex:infinite_integer_graphs}
  A very simple example of an infinite quiver is the \hyperref[def:relation_closures/transitive]{transitive reduction} of the positive integers, i.e. the simple directed graph
  \begin{equation}\label{eq:ex:infinite_integer_graphs/positive}
    \begin{aligned}
      \includegraphics[page=1]{output/ex__infinite_integer_graphs.pdf}
    \end{aligned}
  \end{equation}

  Since the graph is simple, we have
  \begin{equation*}
    \deg(n) = \begin{cases}
      1 = 0 + 1, &n = 0, \\
      2 = 1 + 1, &n > 0 \\
    \end{cases}
  \end{equation*}
  and thus \( \deg(G) = 2 \).

  The \hyperref[def:categorical_diagram]{categorical diagrams} corresponding to this quiver is used to define direct limits in \fullref{def:direct_and_inverse_limits/direct}.

  Another related graph is based on the negative integers:
  \begin{equation}\label{eq:ex:infinite_integer_graphs/negative}
    \begin{aligned}
      \includegraphics[page=2]{output/ex__infinite_integer_graphs.pdf}
    \end{aligned}
  \end{equation}

  The \hyperref[def:categorical_diagram]{categorical diagrams} corresponding to this quiver is used to define inverse limits in \fullref{def:direct_and_inverse_limits/inverse}.

  Finally, the union of the two with zero added gives us the following quiver:
  \begin{equation}\label{eq:ex:infinite_integer_graphs/two_sided}
    \begin{aligned}
      \includegraphics[page=3]{output/ex__infinite_integer_graphs.pdf}
    \end{aligned}
  \end{equation}

  All three graphs are infinite but \hyperref[def:hypergraph/degree]{locally finite}.
\end{example}

\begin{definition}\label{def:category_of_small_quivers}
  Suppose that we are given a \hyperref[def:grothendieck_universe]{Grothendieck universe} \( \mscrU \), which is safe to assume to be the smallest suitable one as explained in \fullref{def:large_and_small_sets}.

  We denote the \hyperref[def:category]{category} of \( \mscrU \)-small \hyperref[def:hypergraph]{quivers} by \( \ucat{Quiv} \) or, if the universe is clear from the context, simply by \( \cat{Quiv} \). See \fullref{def:category_size} for a further discussion of universes and categories.

  A more \enquote{categorical} definition, in which this category arises naturally, is given in \fullref{ex:quivers_as_functors}.

  For simple directed graphs, we denote the category using \( \cat{SimpQuiv} \).

  \begin{itemize}
    \item The \hyperref[def:category/objects]{set of objects} \( \obj(\cat{Quiv}) \) is the set of all \( \mscrU \)-small quivers, i.e. the quivers \( Q = (V, A, h, t) \) such that \( V \) and \( A \) are both members of \( \mscrU \).

    \item The \hyperref[def:category/morphisms]{set of morphisms} \( \cat{Quiv}(Q, R) \) from \( Q \) to \( R \) is the set quiver homomorphisms. Given two quivers \( Q = (V_Q, A_Q, h_Q, t_Q) \) and \( R = (V_R, A_R, h_R, t_R) \), a \term{quiver homomorphism} is a pair of functions
    \begin{equation}\label{eq:def:category_of_small_quivers/homomorphism}
      \begin{cases}
        f_V: V_Q \to V_R \\
        f_A: A_Q \to A_R
      \end{cases}
    \end{equation}
    such that
    \begin{align}
      h_R \bincirc f_A &= f_V \bincirc h_Q \label{eq:def:category_of_small_quivers/homomorphism/head} \\
      t_R \bincirc f_A &= f_V \bincirc t_Q \label{eq:def:category_of_small_quivers/homomorphism/tail}
    \end{align}

    See \fullref{ex:quivers_as_functors} for a categorical justification of this definition, including reducing the above conditions to the diagram \eqref{eq:ex:quivers_as_functors/index/diagram}.

    Clearly every quiver homomorphism is a \hyperref[eq:def:category_of_small_hypergraphs/homomorphism]{hypergraph homomorphism}.

    Note that \enquote{quiver embedding} commonly refers to an embedding of its \hyperref[def:quiver_geometric_realization/undirected]{geometric realization}, hence we will avoid the term when referring to injective quiver homomorphisms. Furthermore, it should be clarified whether we mean \enquote{injective on vertices} or \enquote{injective on arcs}, which is an important distinction in category theory --- see \fullref{def:functor_invertibility}.

    \item The \hyperref[def:category/composition]{composition of quiver morphisms} is pointwise composition --- the same as composition of hypergraph homomorphisms.

    \item The \hyperref[def:category/identity]{identity morphism} on the quiver \( Q = (V, A, h, t) \) is the pair of \hyperref[def:multi_valued_function/identity]{identity functions} \( (\id_V, \id_A) \).
  \end{itemize}
\end{definition}

\begin{definition}\label{def:multigraph_orientation}\mcite{nLab:oriented_graph}
  Given an \hyperref[def:undirected_multigraph]{undirected (multi)graph} \( G = (V, E, \mscrE) \), constructing a quiver requires a choice of head and tail for every edge. We can view undirected graphs as equivalence classes of quivers. A head function is merely a \hyperref[def:choice_function]{choice function} on the set \( \set{ \mscrE(e) \given e \in E } \), and a choice function can be provided by the \hyperref[def:zfc/choice]{axiom of choice} even for infinite graphs.

  For a given choice function \( c \), we denote by \( O_c(G) = (V, E, h, t) \) the quiver obtained by identifying the head \( h(e) \) with \( c(\mscrE(e)) \) and the tail \( t(e) \) with the other endpoint of \( e \), if one exists, and with \( h(e) \) for loops. We call \( O_c(G) \) the \term{orientation} induced by the choice function \( c \).

  Note that \( O_c \) is not left adjoint to \( U \) --- see \fullref{ex:def:category_adjunction/multgph_quiv}.
\end{definition}

\begin{remark}\label{rem:graphs_linear_algebra_and_topology}
  As we shall see, the \hyperref[def:hypergraph/adjacency]{adjacency} and \hyperref[def:hypergraph/incidence]{incidence} of a quiver can be studied using linear algebra via \hyperref[def:graph_adjacency_matrix]{adjacency} and \hyperref[def:hypergraph_incidence_matrix]{incidence matrices}, while the \hyperref[def:quiver_connectedness]{connectedness} can be studied using \hyperref[def:quiver_connectedness]{topology} via \hyperref[def:quiver_geometric_realization/embedding]{graph embeddings}.
\end{remark}

\begin{definition}\label{def:quiver_incidence_matrix}\mcite[ch. 1, sec. 2.1]{GondranMinoux1984Graphs}
  Let \( Q = (A, E, h, t) \) be a finite quiver. Its \term{incidence matrix}
  \begin{equation*}
    M = \seq{ M_{va} }_{v \in V, a \in A}
  \end{equation*}
  has elements
  \begin{equation*}
    M_{va} \coloneqq \begin{cases}
      1,  &v = h(e) \\
      -1, &v = h(e) \\
      0,  &\T{otherwise.}
    \end{cases}
  \end{equation*}

  Compare this definition to \fullref{def:hypergraph_incidence_matrix}.
\end{definition}

\begin{example}\label{ex:quiver_matrices}
  The \hyperref[def:quiver_incidence_matrix]{incidence matrix} of the simple directed graph \eqref{eq:fig:def:quiver} is
  \begin{equation}\label{ex:quiver_matrices/incidence}
    \begin{blockarray}{cccccccc}
        & e_1       & e_2       & e_3       & e_4       & e_5       & e_6       & e_7       \\
      \begin{block}{c(ccccccc)}
      a & 1         & 1         &           &           &           &           &           \\
      b & \fbox{-1} &           & 1         &           &           &           &           \\
      c &           & \fbox{-1} &           & 1         & 1         &           &           \\
      d &           &           & \fbox{-1} & \fbox{-1} &           & 1         &           \\
      e &           &           &           &           & \fbox{-1} &           & 1         \\
      f &           &           &           &           &           & \fbox{-1} & \fbox{-1} \\
      \end{block}
    \end{blockarray}
  \end{equation}

  It can be read column-by-column. Every column contains exactly two nonzero elements whose rows correspond to the head (positive) and tail (negative).

  To obtain the incidence matrix for the underlying undirected graph \eqref{eq:fig:def:undirected_multigraph}, we need to simply flip the sign of the boxed elements above.

  The \hyperref[def:graph_adjacency_matrix]{adjacency matrix} is,
  \begin{equation}\label{ex:quiver_matrices/adjacency}
    \begin{blockarray}{cccccccc}
        & a        & b        & c        & d        & e        & f \\
    \begin{block}{c(ccccccc)}
      a &          & 1        & 1        &          &          &   \\
      b & \fbox{1} &          &          & 1        &          &   \\
      c & \fbox{1} &          &          & 1        & 1        &   \\
      d &          & \fbox{1} & \fbox{1} &          &          & 1 \\
      e &          &          & \fbox{1} &          &          & 1 \\
      f &          &          &          & \fbox{1} & \fbox{1} &   \\
    \end{block}
    \end{blockarray}
  \end{equation}
  where the boxed elements are nonzero only in the adjacency matrix for \eqref{eq:fig:def:undirected_multigraph} and not \eqref{eq:fig:def:quiver}.

  The matrix can be read either column-by-column or row-by-row.
  \begin{itemize}
    \item The \( v \)-th column lists the vertices \( u \) such that there is an arc from \( u \) to \( v \).
    \item The \( u \)-th row lists the vertices \( v \) such that there is an arc from \( v \) to \( u \).
  \end{itemize}
\end{example}

\begin{proposition}\label{thm:graph_undirected_iff_adjacency_matrix_is_symmetric}
  Every square finite matrix over the \hyperref[thm:f2_is_boolean_algebra]{two-element field} \( \BbbF_2 \) corresponds to a \hyperref[def:quiver]{quiver} without parallel arcs.

  If the matrix is symmetric, we obtain its underlying \hyperref[def:undirected_multigraph]{undirected graph}.
\end{proposition}
\begin{proof}
  Trivial.
\end{proof}

\begin{remark}\label{rem:symmetric_directed_graphs}
  It is sometimes convenient to conflate symmetric \hyperref[def:quiver/simple]{simple directed graphs} with \hyperref[def:undirected_multigraph]{simple undirected graphs}.

  This is partially justified by \fullref{thm:graph_undirected_iff_adjacency_matrix_is_symmetric}.
\end{remark}

\begin{definition}\label{def:quiver_path}
  Let \( Q = (V, A, h, t) \) be a \hyperref[def:quiver]{quiver}.

  \begin{thmenum}
    \thmitem{def:quiver_path/directed} A \term{directed path} or simply \term{path} in \( Q \) a finite or infinite sequence of \hyperref[def:quiver/adjacency]{consecutive arcs} with a specified head vertex. Formally, it is a sequence \( p = (v, e_1, e_2, \ldots) \), where \( h(e_k) = t(e_{k-1}) \) for every index \( k \).

    This is a simplification of \fullref{def:undirected_multigraph_path} since every arc has a head and tail. It is also too restrictive for some purposes.

    The definitions of \hyperref[def:undirected_multigraph_path/endpoints]{path endpoints}, \hyperref[def:undirected_multigraph_path/empty]{empty path}, \hyperref[def:undirected_multigraph_path/length]{length}, \hyperref[def:undirected_multigraph_path/domain]{domain}, \hyperref[def:undirected_multigraph_path/concatenation]{concatenation} and \hyperref[def:undirected_multigraph_path/characteristic_vector]{characteristic vector} of a directed path are inherited from \fullref{def:undirected_multigraph_path}.

    \thmitem{def:quiver_path/cycle} A \term{directed cycle} is a directed path whose endpoints coincide. The term \term{circuit} is also used, for example in \cite[ch. 1, sec. 3.2]{GondranMinoux1984Graphs}. A quiver without cycles is called \term{acyclic} and a simple acyclic directed graph is commonly abbreviated as \term{DAG}.

    \thmitem{def:quiver_path/simple} Similarly to \fullref{def:undirected_multigraph_path/simple}, we say that a directed path is \term{simple} if it contains to directed cycles.

    \thmitem{def:quiver_path/undirected} If \( p = (v, e_1, e_2, \ldots) \) is a path in the \hyperref[def:undirected_multigraph]{undirected multigraph} \( U(Q) \),
    \begin{equation*}
      D(p) \coloneqq (v, e_1, e_2, \ldots)
    \end{equation*}
    is a sequence of adjacent arcs in \( Q \) (with a specified head vertex).

    The head and tail of \( D(p) \) are defined as the head and tail of \( p \). This definition obviously differs from \fullref{def:quiver_path}.

    We say that the arc \( e_k \) of \( D(p) \) is \term{positively oriented} if the head \( h_k \) of the edge \( e_k \) in \( Q \) equals the head \( h(e_k) \) of the arc \( e_k \) in \( D(p) \). If \( e_k \) is not positively oriented, we say that it is \term{negatively oriented}.

    We conflate \( p \) and \( D(p) \) where this does not cause confusion.

    Undirected paths are also called \term{chains} in \cite[ch. 1, sec. 3.2]{GondranMinoux1984Graphs}.
  \end{thmenum}
\end{definition}

\begin{example}\label{ex:def:quiver_path}
  An example of an infinite \hyperref[def:quiver_path/directed]{directed path} is the reduced positive integer graph \eqref{eq:ex:infinite_integer_graphs/positive} regarded as a path of the reduced integer graph \eqref{eq:ex:infinite_integer_graphs/two_sided}. It is also simple since the degree of \eqref{eq:ex:infinite_integer_graphs/positive} is \( 2 \).

  Now consider again the quiver \eqref{eq:fig:def:quiver}. The solid lines in \eqref{fig:ex:def:quiver_path} describe a path.
  \begin{equation}\label{fig:ex:def:quiver_path}
    \begin{aligned}
      \includegraphics[page=1]{output/ex__def__quiver_path.pdf}
    \end{aligned}
  \end{equation}

  \begin{itemize}
    \item This path corresponds to the sequence
    \begin{equation*}
      p = \parens{ a, \underbrace{a \to c}_{e_2}, \underbrace{c \to d}_{e_4}, \underbrace{d \to f}_{e_6} }.
    \end{equation*}

    \item Its \hyperref[def:undirected_multigraph_path/characteristic_vector]{characteristic vector} is
    \begin{equation*}
      \vect{p}
      =
      \begin{blockarray}{ccccccr}
        e_1 & e_2 & e_3 & e_4 & e_5 & e_6 & e_7 \\
      \begin{block}{(ccccccr)}
        0   & 1   & 0   & 1   & 0   & 1   & 0   \\
      \end{block}
      \end{blockarray}
      {}^T.
    \end{equation*}

    Furthermore, \( p \) can be identified from the characteristic vector.

    \item The \hyperref[def:quiver_path]{endpoints} are \( h(p) = h(e_1) = a \) and \( t(p) = t(e_7) = f \).

    \item It is a \hyperref[def:undirected_multigraph_path/simple]{simple path} because \( \deg(a) = \deg(f) = 1 \) and \( \deg(c) = \deg(d) = 2 \) in the induced subgraph.

    \item The \hyperref[def:undirected_multigraph_path/converse]{converse undirected path}
    \begin{equation*}
      p^{-1} = (f, d \to f, c \to d, a \to c),
    \end{equation*}
    is not a directed path because every arc is negatively oriented.
  \end{itemize}
\end{example}

\begin{definition}\label{def:quiver_condensation}
  Let \( Q = (V, A, h, t) \) be a \hyperref[def:quiver]{quiver}. We say that the vertex \( w \) is \term{reachable} from \( v \) if there exists a \hyperref[def:quiver_path/directed]{directed path} from \( v \) to \( w \).

  Define the equivalence relation
  \begin{equation*}
    v \sim w \iff v \T{is reachable from} w \T{and} w \T{is reachable from} u.
  \end{equation*}

  Now define the binary relation \( \widetilde{A} \) on the \hyperref[def:equivalence_relation/quotient]{quotient set} \( \widetilde{V} \coloneqq V / {\sim} \) as
  \begin{equation*}
    ([u], [v]) \in \widetilde{A} \iff v \T{is reachable from} w \T{but not vice versa}.
  \end{equation*}

  It is well-defined because if \( ([v_1], [w_1]) \in \widetilde{A} \), \( v_2 \in [v_1] \) and \( w_2 \in [w_1] \), then by the transitivity of reachability we have that \( w_2 \) is reachable from \( v_2 \) and thus \( ([v_2], [w_2]) \in \widetilde{A} \).

  As discussed in \fullref{def:quiver/simple}, irreflexive relations over sets can be regarded as simple directed graphs. Therefore, the pair \( \widetilde{G} \coloneqq (\widetilde{V}, \widetilde{A}) \) is a directed graph. It is called the \term{condensation} of the quiver \( Q \).

  Compare this definition to \fullref{def:undirected_multigraph_connectedness/condensation}.
\end{definition}

\begin{example}\label{ex:def:quiver_condensation}
  The \hyperref[def:quiver_condensation]{condensation} of the quiver in \eqref{eq:fig:def:quiver} is (isomorphic to) the graph itself. If we add the arc \( f \to a \) to \eqref{eq:fig:def:quiver}, the condensation would be an edgeless quiver with a single vertex.

  We can add new arcs \( e_8 \) and \( e_9 \) to the quiver in \eqref{eq:fig:def:quiver} to make the example more interesting:
  \begin{equation}\label{eq:ex:def:quiver_condensation/uncondensed}
    \begin{aligned}
      \includegraphics[page=1]{output/ex__def__graph_condensation.pdf}
    \end{aligned}
  \end{equation}

  Its condensation is
  \begin{equation}\label{eq:ex:def:quiver_condensation/condensed}
    \begin{aligned}
      \includegraphics[page=2]{output/ex__def__graph_condensation.pdf}
    \end{aligned}
  \end{equation}

  These are the \hyperref[def:quiver_connectedness/strong]{strongly connected components} of \eqref{eq:ex:def:quiver_condensation/uncondensed}.
\end{example}

\begin{proposition}\label{thm:graph_condensation_is_acyclic_dag}
  The \hyperref[def:quiver_condensation]{condensation} of a quiver is a \hyperref[def:undirected_multigraph_path]{directed acyclic graph}.
\end{proposition}
\begin{proof}
  Let \( \widetilde{G} = (\widetilde{V}, \widetilde{E}) \) be the condensation of the quiver \( Q = (V, A, h, t) \). We have already discussed in \fullref{def:quiver_condensation} that it is a directed graph. It remains to show that it is acyclic.

  Aiming at a contradiction, suppose that there exist cosets \( [u] \) and \( [v] \) and a path
  \begin{equation*}
    \parens[\Big]{ [u] \to [w_1], \cdots, [w_n] \to [v] }
  \end{equation*}
  in \( \widetilde{G} \) that connects them. We can easily prove by induction that \( v \) is reachable from \( u \), thus contradicting the definition of \( \widetilde{A} \).

  Therefore, \( \widetilde{G} \) is acyclic.
\end{proof}

\begin{definition}\label{def:quiver_connectedness}
  Let \( Q = (V, E, h, t) \) be a quiver and let \( \widetilde{G} = (\widetilde{V}, \widetilde{E}) \) be its condensed directed graph.

  \begin{thmenum}
    \thmitem{def:quiver_connectedness/strong}\mcite[ch. 1, sec. 3.5]{GondranMinoux1984Graphs} For every coset \( [v] \) in \( \widetilde{V} \), the \hyperref[def:hypergraph/submodel]{subquiver} of \( Q \) induced by the vertices \( [v] \) is called a \term{strongly connected component} of \( Q \).

    The \term{strong connectivity number} of \( Q \) is the cardinality \( \card(\widetilde{V}) \).

    If \( Q \) has only one strongly connected component, we say that it itself is \term{strongly connected}.

    \thmitem{def:quiver_connectedness/weak}\mcite[ch. 1, sec. 3.3]{GondranMinoux1984Graphs} Similarly, the subquiver \( Q' \) of \( Q \) is called a \term{weakly connected component} if is a connected component, in the sense of \fullref{def:undirected_multigraph_connectedness/components}, of the underlying undirected multigraph \( U(Q) \).

    The \term{weak connectivity number} of \( Q \) is the connectivity number of \( U(Q) \).

    If \( Q \) has only one weakly connected component, we say that it itself is \term{weakly connected}.
  \end{thmenum}

  This definition generalizes symmetric and transitive closures of binary relations not commuting --- see \fullref{ex:thm:def:relation_closures/properties/symmetric_and_transitive}.

  Compare this definition to \fullref{def:undirected_multigraph_connectedness}.
\end{definition}

\begin{proposition}\label{thm:quiver_connectedness_via_chains}
  Let \( Q \) be a \hyperref[def:quiver]{quiver}.

  \begin{thmenum}
    \thmitem{thm:quiver_connectedness_via_chains/strong} \( Q \) is \hyperref[def:quiver_connectedness/strong]{strongly connected} if and only if there exists a \hyperref[def:quiver_path/directed]{directed path} connecting every pair of vertices.

    \thmitem{thm:quiver_connectedness_via_chains/weak} \( Q \) is \hyperref[def:quiver_connectedness/weak]{weakly connected} if and only if there exists an \hyperref[def:quiver_path/undirected]{undirected path} connecting every pair of vertices.
  \end{thmenum}
\end{proposition}
\begin{proof}
  Trivial.
\end{proof}

\begin{remark}\label{rem:well_founded_graphs}
  We can regard any set \( A \) in the sense of \hyperref[def:zfc]{\logic{ZFC}} as the simple directed graph \( (A, \in) \).

  The \hyperref[def:zfc/foundation]{axiom of foundation} (via \fullref{thm:set_membership_is_well_founded}) implies that the relation \( \in \) is well-founded.

  In the terminology if graph theory, this well-foundedness means that, for every vertex \( v \), there exists no infinite \hyperref[def:quiver_path]{path} that ends with \( v \).

  This implies that the graph is \hyperref[def:undirected_multigraph_path]{acyclic}. For finite graphs the converse holds, but an infinite graph this is not so. For example, the chain \eqref{eq:ex:infinite_integer_graphs/positive} is acyclic but not well-founded.

  Well-founded graphs are important for \fullref{thm:structural_induction}.
\end{remark}

\begin{definition}\label{def:quiver_free_category}
  Let \( Q = (V, A, h, t) \) be a \hyperref[def:quiver]{quiver}. We define the \term{free category} \( F(Q) \) generated by \( Q \) as follows:
  \begin{itemize}
    \item The \hyperref[def:category/objects]{set of objects} \( \obj(F(Q)) \) is the set of vertices \( V \).

    \item The \hyperref[def:category/morphisms]{set of morphisms} \( \cat{F(Q)}(v, w) \) is the set of all paths from \( v \) to \( w \).

    \item The \hyperref[def:category/composition]{composition of the morphisms} \( p \) and \( q \) with \( h(q) = t(p) \) is the \hyperref[def:undirected_multigraph_path/concatenation]{concatenation}
    \begin{equation*}
      q \bincirc p = p \cdot q.
    \end{equation*}

    \item The \hyperref[def:category/identity]{identity morphism} on the vertex \( v \) is the \hyperref[def:undirected_multigraph_path/empty]{empty path} at \( v \). This is the primary motivation for having a distinct empty path at every vertex.
  \end{itemize}

  Since \( F(Q) \) is already defined for every quiver \( Q \), if we also define how it acts on \hyperref[eq:def:category_of_small_quivers/homomorphism]{quiver homomorphisms}, this will make \( F \) a \hyperref[def:functor]{functor} from the category \hyperref[def:category_of_small_quivers]{\( \ucat{Quiv} \)} of \( \mscrU \)-small quivers to the category \hyperref[def:category_of_small_quivers]{\( \ucat{Cat} \)} of \( \mscrU \)-small categories, for every \hyperref[def:grothendieck_universe]{Grothendieck universe} \( \mscrU \) containing \( Q \).

  For every \( \mscrU \)-small category \( \cat{C} \) and every quiver homomorphism \( (g_V, g_A): Q \to U(\cat{C}) \), we define the following functor:
  \begin{equation}\label{eq:def:quiver_free_category/functor_from_homomorphism}
    \begin{aligned}
      &G: F(Q) \to \cat{C}, \\
      &G(v) \coloneqq g_V(v) \\
      &G(v, a_1, \ldots, a_n) \coloneqq \begin{cases}
        \id_v,                                        &n = 0, \\
        g_A(a_n) \bincirc G(v, a_1, \ldots, a_{n-1}), &n > 0,
      \end{cases}
    \end{aligned}
  \end{equation}

  The functor \( G \) \enquote{evaluates} paths in \( Q \) inside \( \cat{C} \). Put \( F(f_V, f_A) \coloneqq G \). Parameterized on \( (f_V, f_A) \), this defines \( F \) is a functor from \( \ucat{Quiv} \) to \( \ucat{Cat} \).

  We will see in \fullref{ex:def:category_adjunction/quiv_cat} that \( F \) is actually left adjoint to the forgetful functor \( U \). The new identity loops are an important part of this adjunction.
\end{definition}
