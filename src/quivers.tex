\subsection{Quivers}\label{subsec:quivers}

\begin{definition}\label{def:graph}
  A \term{quiver} or \term{directed multigraph} is an extension of \hyperref[def:undirected_multigraph]{undirected multigraphs}. A quiver consists of the following:
  \begin{thmenum}
    \thmitem{def:quiver/vertices} A set \( V \) of \term{vertices}.
    \thmitem{def:quiver/arcs} A set \( A \) of \term{arcs}. It is conventional to call them arcs or even directed edges rather than simply edges.
    \thmitem{def:quiver/head} A \term{head} function \( h: A \to V \).
    \thmitem{def:quiver/tail} A \term{tail} function \( t: A \to V \).
    \thmitem{def:quiver/adjacency} We say that the arc \( a \) is a \term{successor} of \( b \) and that \( b \) is a \term{predecessor} of \( a \) if \( t(a) = h(b) \). We also say that \( a \) and \( b \) are \term{consecutive}.

    For vertices, we say that \( v \) is a successor of \( u \) and that \( u \) and \( v \) are consecutive if there exists an arc from \( u \) to \( v \).

    \thmitem{def:quiver/forgetful} For each quiver \( Q = (V, A, h, t) \), we have an \enquote{underlying} undirected multigraph
    \begin{equation}\label{eq:def:quiver/forgetful}
      \begin{aligned}
        &G: A \to V, \\
        &G(a) \coloneqq \set{ h(a), t(a) }.
      \end{aligned}
    \end{equation}

    All properties of \( G \) are inherited by \( Q \), although some of them do not really fit or need adaptation.

    We denote this graph by \( U(Q) \) and will regard it as a forgetful functor --- see \fullref{def:multigraph_adjunction}.

    \thmitem{def:quiver/opposing} One adaptation is that the arcs \( a \) and \( b \) are called \term{opposing} they are successors of each other and the term \enquote{parallel} is reserved for arcs which are parallel in the sense of \fullref{def:hypergraph/parallel_edges} but not opposing.

    \thmitem{def:quiver/simple} If there are no parallel and opposing arcs or loops, we call the quiver \term{simple}. The term \term{simple directed graph} is more commonly used.

    A simple graph can also be regarded as a \hyperref[def:binary_relation/irreflexive]{irreflexive} binary relation. We say that the simple directed graph is \term{symmetric} if this relation is symmetric. See \fullref{rem:symmetric_directed_graphs} for how these relate to undirected graphs.
  \end{thmenum}

  \begin{figure}
    \begin{equation}\label{eq:fig:def:quiver}
      \begin{aligned}
        \includegraphics{figures/eq__fig__def__quiver.pdf}
      \end{aligned}
    \end{equation}
    \caption{A quiver with a pair of parallel arcs, a pair of opposing arcs and a loop. Removing the dashed arcs makes it a simple directed graph.}\label{fig:def:quiver}
  \end{figure}
\end{definition}

\begin{example}\label{ex:tower_diagram_graph}
  A very simple example of an infinite quiver is the \hyperref[def:relation_closures/transitive]{transitive reduction} of the nonnegative integers, i.e. the simple directed graph
  \begin{equation}\label{eq:ex:tower_diagram_graph/forward}
    \begin{aligned}
      \includegraphics{figures/eq__ex__tower_diagram_graph__forward.pdf}
    \end{aligned}
  \end{equation}

  Since the graph is simple, we have
  \begin{equation*}
    \deg(n) = \begin{cases}
      1 = 0 + 1, &n = 0, \\
      2 = 1 + 1, &n > 0 \\
    \end{cases}
  \end{equation*}
  and thus \( \deg(G) = 2 \).

  We call the corresponding \hyperref[def:categorical_diagram]{categorical diagram} a \term{tower diagram} --- see \fullref{def:tower_diagram}.

  Another related graph is based on the nonpositive integers:
  \begin{equation}\label{eq:ex:tower_diagram_graph/backward}
    \begin{aligned}
      \includegraphics{figures/eq__ex__tower_diagram_graph__backward.pdf}
    \end{aligned}
  \end{equation}

  Finally, the union of the two gives us a two-sided tower:
  \begin{equation}\label{eq:ex:tower_diagram_graph/two_sided}
    \begin{aligned}
      \includegraphics{figures/eq__ex__tower_diagram_graph__two_sided.pdf}
    \end{aligned}
  \end{equation}

  All three graphs are infinite but \hyperref[def:hypergraph/degree]{locally finite}.
\end{example}

\begin{definition}\label{def:category_of_small_quivers}
  Suppose that we are given a \hyperref[def:grothendieck_universe]{Grothendieck universe} \( \mscrU \), which is safe to assume to be the smallest suitable one as explained in \fullref{def:large_and_small_sets}.

  We denote the \hyperref[def:category]{category} of \( \mscrU \)-small \hyperref[def:hypergraph]{quivers} by \( \cat{\mscrU-Quiv} \) or, if the universe is clear from the context, simply by \( \cat{Quiv} \). For simple directed graphs, we denote the category using \( \cat{SimpQuiv} \).

  \begin{itemize}
    \item The \hyperref[def:category/objects]{set of objects} \( \obj(\cat{Quiv}) \) is the set of all \( \mscrU \)-small quivers, i.e. the quivers \( Q = (V, A, h, t) \) such that \( V \) and \( A \) are both members of \( \mscrU \).

    \item The \hyperref[def:category/morphisms]{set of morphisms} \( \cat{Quiv}(Q, R) \) from \( Q \) to \( R \) is the set quiver homomorphisms. Given two quivers \( Q = (V_G, A_G, h_G, t_G) \) and \( G = (V_G, A_G, h_G, t_G) \), a \term{quiver homomorphism} is a pair of functions
    \begin{equation}\label{eq:def:category_of_small_quivers/homomorphism}
      \begin{cases}
        f_V: V_H \to V_G \\
        f_A: A_H \to A_G
      \end{cases}
    \end{equation}
    such that
    \begin{align}
      h \bincirc f_A &= f_V \bincirc h \label{eq:def:category_of_small_quivers/homomorphism/head} \\
      t \bincirc f_A &= f_V \bincirc t \label{eq:def:category_of_small_quivers/homomorphism/tail}
    \end{align}

    Clearly every quiver homomorphism is a \hyperref[eq:def:category_of_small_hypergraphs/homomorphism]{hypergraph homomorphism}.

    \item The \hyperref[def:category/composition]{composition of quiver morphisms} is pointwise composition --- the same as composition of hypergraph homomorphisms.

    \item The \hyperref[def:category/identity]{identity morphism} of the quiver \( Q = (V, A, h, t) \) is the pair of \hyperref[def:multi_valued_function/identity]{identity functions} \( (\id_V, \id_A) \).
  \end{itemize}
\end{definition}

\begin{definition}\label{def:multigraph_adjunction}
  In \fullref{def:quiver/forgetful}, we have defined the \hyperref[rem:forgetful_functor]{forgetful functor} \( U: \hyperref[def:category_of_small_quivers]{\cat{Quiv}} \to \hyperref[def:undirected_graph]{\cat{MultGph}} \).

  Given an undirected graph \( G = (V, E, \mscrE) \), constructing a quiver requires a choice of head and tail for every edge. A head function is merely a \hyperref[def:choice_function]{choice function} on the set \( \set{ G(e) \given e \in E } \) and a choice function can be provided by the \hyperref[def:zfc/choice]{axiom of choice} even for infinite graphs.

  Given such a choice function \( c \), we can define an \hyperref[rem:forgetful_functor]{adjoint} to \( U \) functor \( D_c: \cat{MultGph} \to \cat{Quiv} \). We will occasionally use this functor.
\end{definition}

\begin{remark}\label{rem:graphs_linear_algebra_and_topology}
  As we shall see, the \hyperref[def:graph/adjacent_vertices]{adjacency} and \hyperref[def:graph/incidence]{incidence} of a quiver can be studied using linear algebra via \hyperref[def:graph_adjacency_matrix]{adjacency} and \hyperref[def:hypergraph_incidence_matrix]{incidence matrices}, while the \hyperref[def:quiver_connectedness]{connectedness} can be studied using \hyperref[def:quiver_connectedness]{topology} via \hyperref[def:quiver_geometric_relization/embedding]{graph embeddings}.
\end{remark}

\begin{definition}\label{def:quiver_incidence_matrix}\mcite[ch. 1, sec. 2.1]{GondranMinoux1984Graphs}
  Let \( Q = (A, E, h, t) \) be a finite quiver. Its \term{incidence matrix}
  \begin{equation*}
    M = \seq{ M_{va} }_{v \in V, a \in A}
  \end{equation*}
  has elements
  \begin{equation*}
    M_{va} \coloneqq \begin{cases}
      1,  &v = h(e) \\
      -1, &v = h(e) \\
      0,  &\T{otherwise.}
    \end{cases}
  \end{equation*}

  Compare this definition to \fullref{def:hypergraph_incidence_matrix}.
\end{definition}

\begin{example}\label{ex:quiver_matrices}
  The \hyperref[def:quiver_incidence_matrix]{incidence matrix} of the simple directed graph \eqref{eq:fig:def:quiver} is
  \begin{equation}\label{ex:quiver_matrices/incidence}
    \begin{blockarray}{cccccccc}
        & e_1       & e_2       & e_3       & e_4       & e_5       & e_6       & e_7       \\
      \begin{block}{c(ccccccc)}
      a & 1         & 1         &           &           &           &           &           \\
      b & \fbox{-1} &           & 1         &           &           &           &           \\
      c &           & \fbox{-1} &           & 1         & 1         &           &           \\
      d &           &           & \fbox{-1} & \fbox{-1} &           & 1         &           \\
      e &           &           &           &           & \fbox{-1} &           & 1         \\
      f &           &           &           &           &           & \fbox{-1} & \fbox{-1} \\
      \end{block}
    \end{blockarray}
  \end{equation}

  It can be read column-by-column. Every column contains exactly two nonzero elements whose rows correspond to the head (positive) and tail (negative).

  To obtain the incidence matrix for the underlying undirected graph \eqref{eq:fig:def:undirected_graph}, we need to simply flip the sign of the boxed elements above.

  The \hyperref[def:graph_adjacency_matrix]{adjacency matrix} is,
  \begin{equation}\label{ex:quiver_matrices/adjacency}
    \begin{blockarray}{cccccccc}
        & a        & b        & c        & d        & e        & f \\
    \begin{block}{c(ccccccc)}
      a &          & 1        & 1        &          &          &   \\
      b & \fbox{1} &          &          & 1        &          &   \\
      c & \fbox{1} &          &          & 1        & 1        &   \\
      d &          & \fbox{1} & \fbox{1} &          &          & 1 \\
      e &          &          & \fbox{1} &          &          & 1 \\
      f &          &          &          & \fbox{1} & \fbox{1} &   \\
    \end{block}
    \end{blockarray}
  \end{equation}
  where the boxed elements are nonzero only in the adjacency matrix for \eqref{eq:fig:def:undirected_graph} and not \eqref{eq:fig:def:quiver}.

  The matrix can be read either column-by-column or row-by-row.
  \begin{itemize}
    \item The \( v \)-th column lists the vertices \( u \) such that there is an arc from \( u \) to \( v \).
    \item The \( u \)-th row lists the vertices \( v \) such that there is an arc from \( v \) to \( u \).
  \end{itemize}
\end{example}

\begin{proposition}\label{thm:graph_undirected_iff_adjacency_matrix_is_symmetric}
  Every square finite matrix over the \hyperref[thm:f2_is_boolean_algebra]{two-element field} \( \BbbF_2 \) corresponds to a \hyperref[def:quiver]{quiver} without parallel arcs.

  If the matrix is symmetric, we obtain its underlying \hyperref[def:undirected_graph]{undirected graph}.
\end{proposition}
\begin{proof}
  Trivial.
\end{proof}

\begin{remark}\label{rem:symmetric_directed_graphs}
  It is sometimes convenient to conflate symmetric \hyperref[def:quiver/simple]{simple directed graphs} with \hyperref[def:undirected_graph]{simple undirected graphs}.

  This is partially justified by \fullref{thm:graph_undirected_iff_adjacency_matrix_is_symmetric}.
\end{remark}

\begin{definition}\label{def:quiver_path}
  A \term{directed path} or simply \term{path} in a \hyperref[def:quiver]{quiver} \( Q = (V, A, h, t) \) is a finite or infinite sequence of \hyperref[def:quiver/adjacency]{consecutive arcs}.

  This is a simplification of \fullref{def:undirected_multigraph_path} since every arc has a head and tail. Every path of the undirected multigraph \( U(Q) \) is instead called an \term{undirected path} or \term{chain} of \( Q \). The latter term is used, for example, in \cite[ch. 1, sec. 3.2]{GondranMinoux1984Graphs}.

  The definitions of domain, length and characteristic vector of a directed path are inherited from \fullref{def:undirected_multigraph_path} since they do not depend on the direction of the arcs.

  The \term{head} of a nonempty directed path is the head of its first arc. The \term{tail} of a finite nonempty directed path is the tail of its last arc. These may or may not coincide with the head and tail of the corresponding undirected path in \( U(Q) \).

  A \term{directed cycle} is a directed path whose endpoints coincide. The term \term{circuit} is also used, for example in \cite[ch. 1, sec. 3.2]{GondranMinoux1984Graphs}. A quiver without cycles is called \term{acyclic} and a simple acyclic directed graph is commonly abbreviated as \term{DAG}.
\end{definition}

\begin{example}\label{ex:def:quiver_path}
  An example of an infinite \hyperref[def:quiver_path]{directed path} is the forward tower \eqref{eq:ex:tower_diagram_graph/forward} regarded as a path of the two-sided tower \eqref{eq:ex:tower_diagram_graph/two_sided}. It is also simple since the degree of the forward tower is \( 2 \).

  Now consider again the quiver \eqref{eq:fig:def:quiver/simple}. The solid lines in \eqref{fig:ex:def:quiver_path} describe a path.
  \begin{equation}\label{fig:ex:def:quiver_path}
    \begin{aligned}
      \includegraphics{figures/fig__ex__def__quiver_path.pdf}
    \end{aligned}
  \end{equation}

  \begin{itemize}
    \item This path corresponds to the sequence
    \begin{equation*}
      p = \parens{ \underbrace{a \to c}_{e_2}, \underbrace{c \to d}_{e_4}, \underbrace{d \to f}_{e_6} }.
    \end{equation*}

    \item Its \hyperref[def:undirected_graph_path/characteristic_vector]{characteristic vector} is
    \begin{equation*}
      \vect{p}
      =
      \begin{blockarray}{ccccccr}
        e_1 & e_2 & e_3 & e_4 & e_5 & e_6 & e_7 \\
      \begin{block}{(ccccccr)}
        0   & 1   & 0   & 1   & 0   & 1   & 0   \\
      \end{block}
      \end{blockarray}
      {}^T.
    \end{equation*}

    Furthermore, \( p \) can be identified from the characteristic vector.

    \item The \hyperref[def:quiver_path/endpoints]{endpoints} are \( h(p) = h(e_1) = a \) and \( t(p) = t(e_7) = f \).

    \item It is a \hyperref[def:undirected_multigraph_path/simple]{simple path} because \( \deg(a) = \deg(f) = 1 \) and \( \deg(c) = \deg(d) = 2 \) in the induced subgraph.

    \item The \hyperref[def:undirected_multigraph_path/converse]{converse undirected path}
    \begin{equation*}
      p^{-1} = (d \to f, c \to d, a \to c),
    \end{equation*}
    is not a directed path.
  \end{itemize}
\end{example}

\begin{definition}\label{def:quiver_condensation}
  Let \( Q = (V, A, h, t) \) be a \hyperref[def:quiver]{quiver}. We say that the vertex \( v \) is \term{reachable} from \( u \) if there exists a \hyperref[def:quiver_path]{directed path} from \( u \) to \( v \).

  Define the equivalence relation
  \begin{equation*}
    u \sim v \iff u \T{is reachable from} v \T{and} v \T{is reachable from} u.
  \end{equation*}

  Now define the binary relation \( \widetilde{A} \) on the \hyperref[def:equivalence_relation/quotient]{quotient set} \( \widetilde{V} \coloneqq V / {\sim} \) as
  \begin{equation*}
    ([u], [v]) \in \widetilde{A} \iff u \T{is reachable from} v \T{but not vice versa}.
  \end{equation*}

  It is well-defined because if \( ([u_1], [v_1]) \in \widetilde{A} \), \( u_2 \in [u_1] \) and \( v_2 \in [v_1] \), then by the transitivity of reachability we have that \( v_2 \) is reachable from \( u_2 \) and thus \( ([u_2], [v_2]) \in \widetilde{A} \).

  As discussed in \fullref{def:quiver/simple}, irreflexive relations over sets can be regarded as simple directed graphs. Therefore, the pair \( \widetilde{G} \coloneqq (\widetilde{V}, \widetilde{A}) \) is a directed graph. It is called the \term{condensation} of the quiver \( Q \).

  Compare this definition to \fullref{def:undirected_multigraph_connectedness/condensation}.
\end{definition}

\begin{example}\label{ex:def:quiver_condensation}
  The \hyperref[def:quiver_condensation]{condensation} of the quiver in \eqref{eq:fig:def:quiver} is (isomorphic to) the graph itself. If we add the arc \( f \to a \) to \eqref{eq:fig:def:quiver}, the condensation would be an edgeless quiver with a single vertex.

  We can add new arcs \( e_8 \) and \( e_9 \) to the quiver in \eqref{eq:fig:def:quiver} to make the example more interesting:
  \begin{equation}\label{eq:ex:def:quiver_condensation/uncondensed}
    \begin{aligned}
      \includegraphics{figures/eq__ex__def__graph_condensation__uncondensed.pdf}
    \end{aligned}
  \end{equation}

  Its condensation is
  \begin{equation}\label{eq:ex:def:quiver_condensation/condensed}
    \begin{aligned}
      \includegraphics{figures/eq__ex__def__graph_condensation__condensed.pdf}
    \end{aligned}
  \end{equation}

  These are the \hyperref[def:quiver_connectedness/strong]{strongly connected components} of \eqref{eq:ex:def:quiver_condensation/uncondensed}.
\end{example}

\begin{proposition}\label{thm:graph_condensation_is_acyclic_dag}
  The \hyperref[def:quiver_condensation]{condensation} of a quiver is a \hyperref[def:undirected_multigraph_path]{directed acyclic graph}.
\end{proposition}
\begin{proof}
  Let \( \widetilde{G} = (\widetilde{V}, \widetilde{E}) \) be the condensation of the quiver \( Q = (V, A, h, t) \). We have already discussed in \fullref{def:quiver_condensation} that it is a directed graph. It remains to show that it is acyclic.

  Aiming at a contradiction, suppose that there exist cosets \( [u] \) and \( [v] \) and a path
  \begin{equation*}
    \parens[\Big]{ [u] \to [w_1], \cdots, [w_n] \to [v] }
  \end{equation*}
  in \( \widetilde{G} \) that connects them. We can easily prove by induction that \( v \) is reachable from \( u \), thus contradicting the definition of \( \widetilde{A} \).

  Therefore, \( \widetilde{G} \) is acyclic.
\end{proof}

\begin{definition}\label{def:quiver_connectedness}
  Let \( Q = (V, E, h, t) \) be a quiver and let \( \widetilde{G} = (\widetilde{V}, \widetilde{E}) \) be its condensed directed graph.

  \begin{thmenum}
    \thmitem{def:quiver_connectedness/strong}\mcite[ch. 1, sec. 3.5]{GondranMinoux1984Graphs} For every coset \( [v] \) in \( \widetilde{V} \), the \hyperref[def:hypergraph/submodels]{subquiver} of \( Q \) induced by the vertices \( [v] \) is called a \term{strongly connected component} of \( Q \).

    The \term{strong connectivity number} of \( Q \) is the cardinality \( \card(\widetilde{V}) \).

    If \( Q \) has only one strongly connected component, we say that it itself is \term{strongly connected}.

    \thmitem{def:quiver_connectedness/weak}\mcite[ch. 1, sec. 3.3]{GondranMinoux1984Graphs} Similarly, the subquiver \( Q' \) of \( Q \) is called a \term{weakly connected component} if is a connected component, in the sense of \fullref{def:undirected_multigraph_connectedness/components}, of the underlying undirected multigraph \( U(Q) \).

    The \term{weak connectivity number} of \( Q \) is the connectivity number of \( U(Q) \).

    If \( Q \) has only one weakly connected component, we say that it itself is \term{weakly connected}.
  \end{thmenum}

  Compare this definition to \fullref{def:undirected_multigraph_connectedness}.
\end{definition}

\begin{proposition}\label{thm:quiver_connectedness_via_chains}
  Let \( Q \) be a \hyperref[def:quiver]{quiver}.

  \begin{thmenum}
    \thmitem{thm:quiver_connectedness_via_chains/strong} \( Q \) is \hyperref[def:quiver_connectedness/strong]{strongly connected} if and only if there exists a \hyperref[def:quiver_path]{directed path} connecting every pair of vertices.

    \thmitem{thm:quiver_connectedness_via_chains/weak} \( Q \) is \hyperref[def:quiver_connectedness/weak]{weakly connected} if and only if there exists an \hyperref[def:undirected_multigraph_path]{undirected path} connecting every pair of vertices.
  \end{thmenum}
\end{proposition}
\begin{proof}
  Follows from \fullref{thm:quiver_chain_symmetrization}.
\end{proof}

\begin{remark}\label{rem:well_founded_graphs}
  We can regard any set \( A \) in the sense of \hyperref[def:zfc]{\logic{ZFC}} as the simple directed graph \( (A, \in) \).

  The \hyperref[def:zfc/foundation]{axiom of foundation} (via \fullref{thm:set_membership_is_well_founded}) implies that the relation \( \in \) is well-founded.

  In the terminology if graph theory, this well-foundedness means that, for every vertex \( v \), there exists no infinite \hyperref[def:quiver_path]{path} that ends with \( v \).

  This implies that the graph is \hyperref[def:undirected_multigraph_path]{acyclic}. For finite graphs the converse holds, but an infinite graph this is not so. For example, the backward tower \eqref{eq:ex:tower_diagram_graph/backward} is acyclic but not well-founded.
\end{remark}

\begin{definition}\label{def:graph_geometric_realization}
  Let \( Q \) be a \hyperref[def:quiver]{quiver}. Our goal is to construct a \hyperref[def:topological_space]{topological space} that translates the connectivity properties of \( Q \) into their topological equivalents. We do this by taking a copy of the interval \( [0, 1] \) for every arc and gluing endpoints where the edges have a common endpoint.

  Consider the \hyperref[def:topological_sum]{topological sum} \( \coprod_{a \in A} [0, 1] \). Define the \hyperref[def:equivalence_relation]{equivalence relation} \( {\sim} \) on this space to hold for \( (x, a) \) and \( (y, b) \) if any of the following hold:
  \begin{align*}
    &x = 0 \T{and} y = 0 \T{and} h(a) = h(b) \\
    &x = 0 \T{and} y = 1 \T{and} h(a) = t(b) \\
    &x = 1 \T{and} y = 0 \T{and} t(a) = h(b) \\
    &x = 0 \T{and} y = 1 \T{and} t(a) = t(b)
  \end{align*}

  The \hyperref[def:topological_quotient]{quotient space} \( \coprod_{a \in A} [0, 1] / {\sim} \) is called the \term{geometric realization} of \( Q \).

  \begin{thmenum}
    \thmitem{def:graph_geometric_realization/undirected} The geometric realization of an undirected multigraph requires a choice of adjoint to \( U \) functor \( D_c \) from \fullref{def:multigraph_adjunction}.

    \thmitem{def:graph_geometric_realization/drawing} We will call any \hyperref[def:global_continuity]{continuous function} with domain \( \coprod_{e \in E} [0, 1] / {\sim} \) a \term{quiver drawing}. There is no standardized terminology for non-injective continuous images of the realization.

    \thmitem{def:graph_geometric_realization/embedding} An injective quiver drawing is called a \term{quiver embedding}.

    \thmitem{def:graph_geometric_realization/linear} If a quiver can be embedded into \( \BbbR \), we say that it is \term{linear}.

    \thmitem{def:graph_geometric_realization/planar} If a quiver can be embedded into \( \BbbR^2 \), we say that it is \term{planar}.
  \end{thmenum}
\end{definition}

\begin{example}\label{ex:def:graph_geometric_realization}
  We will give a few examples of \hyperref[def:graph_geometric_realization]{quiver realizations}.

  \begin{thmenum}
    \thmitem{ex:def:graph_geometric_realization/edgeless} The \hyperref[def:graph_geometric_realization]{geometric realization} of an edgeless quiver is the empty topological space.

    \thmitem{ex:def:graph_geometric_realization/tower} Consider the tower \eqref{eq:ex:tower_diagram_graph/forward}. We start with \( \aleph_0 \) copies of \( [0, 1] \) and glue both ends of each of them except for the first. Thus, we obtain (a space homeomorphic to)
    \begin{equation*}
      \bigcup_{k \geq 0} [k, k + 1] = [0, \infty).
    \end{equation*}

    Therefore, \eqref{eq:ex:tower_diagram_graph/forward} is a \hyperref[def:graph_geometric_realization/linear]{linear graph}.

    \thmitem{ex:def:graph_geometric_realization/triangle} The graph with vertices \( V = \set{ a, b, c } \) and arcs \( \set{ \overbrace{a \to b}^{e_1}, \overbrace{b \to c}^{e_2}, \overbrace{c \to a}^{e_3} } \) is more subtle.

    We start with three copies of the interval \( [0, 1] \), depicted in \eqref{eq:ex:def:graph_geometric_realization/triangle/relization} as upward-pointing arrows, and use dashed lines to connect the endpoints that we want to glue together.
    \begin{equation}\label{eq:ex:def:graph_geometric_realization/triangle/relization}
      \begin{aligned}
        \includegraphics{figures/eq__ex__def__graph_geometric_realization__triangle__realization.pdf}
      \end{aligned}
    \end{equation}

    After contracting the dashed lines, we obtain a topological space that can easily be \hyperref[def:graph_geometric_realization/embedding]{embedded} into \( \BbbR^2 \). An obvious embedding corresponds to \enquote{pulling up} \( e_2 \) and \( e_3 \):
    \begin{equation}\label{eq:ex:def:graph_geometric_realization/triangle/embedding}
      \begin{aligned}
        \includegraphics{figures/eq__ex__def__graph_geometric_realization__triangle__embedding.pdf}
      \end{aligned}
    \end{equation}

    This is only one possible embedding of the geometric realization. It is sufficient, however, for proving that the graph is \hyperref[def:graph_geometric_realization/planar]{planar}.

    It is also clear from this example that constructing embeddings of graphs can be a very tedious task for graphs with more than a few arcs.
  \end{thmenum}
\end{example}
