\subsection{Univariate polynomials}\label{subsec:univariate_polynomials}

We will discuss here the \hyperref[def:polynomial_semiring]{polynomial ring} \( R[X] \) in one indeterminate over the nontrivial \hyperref[def:ring/commutative]{commutative ring} \( R \). We call them \term{univariate polynomials} using the general convention for function arguments from \fullref{def:multi_valued_function/arguments}. Polynomials are not functions in general, and the exact relationship between polynomials and polynomial functions is discussed in \fullref{thm:polynomial_semiring_universal_property} and \fullref{thm:functions_over_finite_fields}.

\begin{definition}\label{def:polynomial_degree}
  The \term{degree} of the nonzero univariate \hyperref[def:polynomial_semiring]{monomial} \( X^k \) is its power \( k \). More generally, the degree of a multivariate monomial \( \prod_{X \in \mscrX} X^{\gamma_X} \) in the set of indeterminates \( \mscrX \) is the \hyperref[def:multi_index]{multi-index norm} \( \norm \gamma = \sum_{X \in \mscrX} \gamma_X \).

  The degree \( \deg(p) \) of a polynomial \( p \) is the maximal degree of its nonzero monomials. For the zero polynomial, we leave the degree undefined.

  For a univariate polynomial
  \begin{equation*}
    p(X) = \sum_{k=0}^\infty a_k X^k = a_0 + a_1 X + a_2 X^2 + a_3 X^3 + \cdots,
  \end{equation*}
  the degree \( n \coloneqq \deg(p) \) allows us to write
  \begin{equation*}
    p(X) = \sum_{k=0}^n a_k X^k = a_0 + a_1 X + a_2 X^2 + \cdots + a_{n-1} X^{n-1} + a_n X^n.
  \end{equation*}

  This notation also subsumes the zero polynomial. We call \( a_n \) the \term{leading coefficient} and \( a_0 \) the \term{constant coefficient} of the polynomial.

  We introduce the following names for univariate polynomials of certain degrees:
  \begin{center}
    \begin{tabular}{l | l}
      Constant  & \( \deg(p) = 0 \) or \( p \) is the zero polynomial \\
      Linear    & \( \deg(p) = 1 \)                                   \\
      Quadratic & \( \deg(p) = 2 \)                                   \\
      Cubic     & \( \deg(p) = 3 \)                                   \\
      Quartic   & \( \deg(p) = 4 \)                                   \\
      Quintic   & \( \deg(p) = 5 \)
    \end{tabular}
  \end{center}
\end{definition}

\begin{proposition}\label{thm:def:polynomial_degree/properties}
  The \hyperref[def:polynomial_degree]{polynomial degree} has the following basic properties:
  \begin{thmenum}
    \thmitem{thm:def:polynomial_degree/properties/sum} For any two zero polynomials satisfying \( p(X) \neq -q(X) \), we have
    \begin{equation}\label{eq:thm:def:polynomial_degree/properties/sum}
      \deg (p + q) \leq \max \set{ \deg p, \deg q }.
    \end{equation}

    \thmitem{thm:def:polynomial_degree/properties/product} For any two nonzero polynomials \( p(X) \) and \( q(X) \), whose leading coefficients do not multiply to zero, we have
    \begin{equation}\label{eq:thm:def:polynomial_degree/properties/product}
      \deg (pq) = \deg p + \deg q.
    \end{equation}
  \end{thmenum}
\end{proposition}
\begin{proof}
  Fix nonzero polynomials
  \begin{align*}
    p(X) &\coloneqq \sum_{k=0}^n a_k X^k, \\
    q(X) &\coloneqq \sum_{k=0}^m b_k X^k.
  \end{align*}

  \SubProofOf{thm:def:polynomial_degree/properties/sum} Additionally assume that \( p(X) \neq -q(X) \) since otherwise \( p(X) + q(X) = 0 \) and \( \deg(p + q) \) is undefined. Thus, there exists at least one index \( k = 1, 2, \ldots \), so that \( a_k \neq b_k \). Denote by \( k_0 \) the largest such index (only finitely many are nonzero). Then
  \begin{equation*}
    a_k = b_k = 0 \T{for} k > k_0.
  \end{equation*}

  Therefore, \( \deg(p + q) = k_0 \). Note that \( k_0 \) cannot exceed both \( \deg p \) and \( \deg q \) because it corresponds to a nonzero coefficient. Thus, \( k_0 \leq \max\set{ \deg p, \deg q } \).

  \SubProofOf{thm:def:polynomial_degree/properties/product} The coefficient \( c_{n + m} \) of the product \( p(X) q(X) \) is \( a_n b_m \) by definition. By assumption, it is nonzero. Then, since \( c_{n+m+1} = 0 \), we have
  \begin{equation*}
    \deg (pq) = \deg p + \deg q.
  \end{equation*}
\end{proof}

\begin{definition}\label{def:monic_polynomial}
  We say that the nonzero univariate polynomial \( p(X) \) is \term{monic} if its leading coefficient is \( 1 \).
\end{definition}

\begin{theorem}[Euclidean division of polynomials]\label{thm:euclidean_division_of_polynomials}\mcite[prop. 1.12]{Knapp2016BasicAlgebra}
  Fix two polynomials \( f(X) \) and \( g(X) \) over the \hi{nontrivial} commutative ring \( R \), and assume that \( g(X) \) is \hyperref[def:monic_polynomial]{monic}.

  Then there exist unique polynomials \( q(X) \) and \( r(X) \), where \( r(X) \) is either zero or \( \deg r < \deg g \), such that
  \begin{equation*}
    f(X) = g(X) q(X) + r(X).
  \end{equation*}
\end{theorem}
\begin{proof}
  \SubProof{Proof of uniqueness} Suppose that
  \begin{equation*}
    a(X) = g(X)q(X) + r(X) = g(X) \widetilde{q}(X) + \widetilde{r}(X),
  \end{equation*}
  where \( r(X) \) and \( \widetilde{r}(X) \) are either zero or have degree less than \( g(X) \).

  Assume that \( r(X) \neq \widetilde{r}(X) \).

  \begin{itemize}
    \item If both \( r(X) \) and \( \widetilde{r}(X) \) are nonzero, we have
    \begin{equation*}
      g(X) \parens[\Big]{ q(X) - \widetilde{q}(X) } = -\parens[\Big]{ r(X) - \widetilde{r}(X) }.
    \end{equation*}

    Since \( g(X) \) is monic and its leading coefficient \( 1_R \) is not a zero divisor, \fullref{thm:def:polynomial_degree/properties/product} holds, and thus
    \begin{equation*}
      \deg g + \deg(q - \widetilde{q})
      \reloset {\eqref{eq:thm:def:polynomial_degree/properties/product}} =
      \deg(g (q - \widetilde{q}))
      =
      \deg(r - \widetilde{r})
      \reloset {\eqref{eq:thm:def:polynomial_degree/properties/sum}} =
      \leq \max\set{ \deg r, \deg \widetilde{r} }
      <
      \deg g,
    \end{equation*}
    which is a contradiction.

    \item If \( r(X) \) is zero but \( \widetilde{r}(X) \) is not, then
    \begin{equation*}
      g(X) q(X) = g(X) \widetilde{q}(X) + \widetilde{r}(X),
    \end{equation*}
    implying that
    \begin{equation*}
      \widetilde{r}(X) = g(X) \parens[\Big]{ q(X) - \widetilde{q}(X) }.
    \end{equation*}

    By \eqref{thm:def:polynomial_degree/properties/product}, \( \deg g \leq \widetilde{r} \), which contradicts our choice of \( \widetilde{r}(X) \).
  \end{itemize}

  \SubProof{Proof of existence} If \( f(X) \) is the zero polynomial or \( \deg f < \deg g \), define
  \begin{align*}
    q(X) &\coloneqq 0_R, \\
    r(X) &\coloneqq f(X).
  \end{align*}

  In this case, \( r(X) \) is either zero or \( \deg r = \deg f < \deg b \).

  Now suppose that \( \deg b \leq \deg f \). We will use proof by induction on \( \deg f \).

  If \( \deg f = 0 \), obviously \( \deg b = 0 \) (thus \( b = 1_R \)) and we define
  \begin{align*}
    q(X) &\coloneqq f(X), \\
    r(X) &\coloneqq 0_R.
  \end{align*}

  Assume that the result holds for \( \deg f < n \), and let \( \deg f = n \) and \( \deg b = m \). Then there exists a polynomial \( \hat f(X) \) that is either zero or \( \deg \hat f = n - 1 \) such that
  \begin{equation*}
    f(X) = a_n X^n + \hat f(X).
  \end{equation*}

  Analogously, we find \( \hat g(X) \) that is either zero or \( \deg \hat g = m - 1 \) such that
  \begin{equation*}
    g(X) = X^m + \hat g(X).
  \end{equation*}

  Thus,
  \begin{align*}
    f(X) - g(X) a_n X^{n-m}
    &=
    a_n X^n + \hat f(X) - (b_m X^m + \hat g(X)) a_n X^{n-m}
    = \\ &=
    a_n X^n + \hat f(X) - a_n X^n - \hat g(X) a_n X^{n-m}
    = \\ &=
    \underbrace{\hat f(X) - \hat g(X) a_n X^{n-m}}_{\hat r(X)}.
  \end{align*}

  Therefore, \( \hat r(X) \) is either the zero polynomial (in which case we define \( r(X) \coloneqq \hat r(X) \)) or \( \deg \hat r \leq n - 1 \). In the latter case, we can divide \( \hat r \) by \( g(X) \) to obtain \( \hat q(X) \) and \( r(X) \) such that
  \begin{equation*}
    \hat r(X) \coloneqq g(X) \hat q(X) + r(X),
  \end{equation*}
  where \( r \) is either zero or \( \deg r < \deg g \).

  Then
  \begin{align*}
    \hat r(X)                                         &= f(X) - g(X) a_n X^{n-m} \\
    g(X) \hat q(X) + r(X)                             &= f(X) - g(X) a_n X^{n-m} \\
    g(X) \left(\hat q(X) - a_n X^{n-m} \right) + r(X) &= f(X).
  \end{align*}

  Define
  \begin{equation*}
    q(X) \coloneqq \hat q(X) - a_n X^{n-m}.
  \end{equation*}

  We have obtained polynomials \( r(X) \) and \( q(X) \) where \( r(X) \) is either zero or \( \deg r < \deg g \).
\end{proof}

\begin{corollary}\label{thm:representatives_in_univariate_polynomial_quotient_set}
  Given \hyperref[def:monic_polynomial]{monic polynomial} \( g(X) \) in a nontrivial commutative ring \( R \), every coset in \( R[X] / \braket{ g(X) } \) has a unique representative that is either the zero polynomial or a polynomial of degree less than \( g(X) \).
\end{corollary}
\begin{proof}
  Let \( f(X) \) be an arbitrary polynomial. \Fullref{thm:euclidean_division_of_polynomials} gives us polynomials \( q(X) \) and \( r(X) \) so that
  \begin{equation*}
    f(X) = g(X) q(X) + r(X),
  \end{equation*}
  where \( r(X) \) is either zero or has degree less than \( g(X) \).

  Multiples of \( q(X) \) are congruent to \( 0_R \) modulo the ideal \( \braket{ q(X) } \), hence \( f(X) \) is congruent to \( r(X) \).

  By the uniqueness of \( r(X) \), the statement of the corollary follows.
\end{proof}

\begin{corollary}\label{thm:polynomial_quotient_modules_vs_quotient_algebras}
  For two nonzero monic polynomials \( p(X) \) and \( q(X) \) of the same degree, the \hyperref[def:ring/quotient]{quotient rings} \( R[X] / \braket{ p(X) } \) and \( R[X] / \braket{ q(X) } \) are isomorphic as \( R \)-modules, but may not be isomorphic as \( R \)-algebras.
\end{corollary}
\begin{proof}
  By \fullref{thm:representatives_in_univariate_polynomial_quotient_set}, for every coset in the quotient, \fullref{thm:euclidean_division_of_polynomials} gives us a unique representative of the corresponding degree. Addition and scalar multiplication must be the same in both.

  As shown in \fullref{ex:gaussian_integers} and \fullref{ex:integers_modulo_sqrt2}, however, the vector multiplication operation may differ.
\end{proof}

\begin{example}\label{ex:gaussian_integers}
  The \term{Gaussian integers} are complex numbers \( z = a + bi \) with integer real and imaginary components. We can define several isomorphic rings for the Gaussian integers.

  \begin{thmenum}
    \thmitem{ex:gaussian_integers/quotient} We can take the \hyperref[def:ring/quotient]{quotient ring} \( \BbbZ[X] / \braket{X^2 + 1} \). By \fullref{thm:representatives_in_univariate_polynomial_quotient_set}, the remainder from \fullref{thm:euclidean_division_of_polynomials} can be used as a canonical representative within the quotient. The remainder must be either a constant or a linear polynomial. That is, \( r(X) = aX + Y \).

    In order to make sense of the imposed ring structure in the quotient, we can see how multiplication modulo \( X^2 + 1 \) works. We have
    \begin{align*}
      (bX + a) (dX + c)
      &\cong
      bdX^2 + (ad + bc)X + ac
      &\pmod {X^2 + 1} \cong \\ &\cong
      bd[X^2 + 1] + [(ad + bc)X - bd + ac]
      &\pmod {X^2 + 1} \cong \\ &\cong
      (ad + bc)X + (ac - bd)
      &\pmod {X^2 + 1}. \phantom{\cong}
    \end{align*}

    This is precisely the definition of multiplication of complex numbers as given in \fullref{def:set_of_complex_numbers}. Thus,
    \begin{equation*}
      \BbbZ[X] / \braket{X^2 + 1}
    \end{equation*}
    is the desired ring of Gaussian integers.

    \thmitem{ex:gaussian_integers/evaluation} We can also \hyperref[thm:adjoining_elements_to_semiring]{adjoin} \( i \) to \( \BbbZ \) to obtain the ring \( \BbbZ[i] \).

    Given a Gaussian integer \( z = a + bi \), it corresponds to the polynomial
    \begin{equation*}
      p_z(X) \coloneqq a + bX.
    \end{equation*}

    Conversely, consider the \hyperref[thm:polynomial_semiring_universal_property]{evaluation homomorphism} \( \Phi_i: \BbbZ[X] \to \BbbC \) for the imaginary unit. Let \( p(X) \in \BbbZ[X] \). Then
    \begin{equation*}
      p(i)
      =
      \Phi_i(p)
      =
      \sum_{k=0}^n a_k i^n
      =
      \thickspace \sum_{\scriptscriptstyle{\rem(k, 4) = 0}}^n a_k - \sum_{\scriptscriptstyle{\rem(k, 4) = 2}}^n a_k + i \parens[\Bigg]{ \sum_{\scriptscriptstyle{\rem(k, 4) = 1}}^n a_k - \sum_{\scriptscriptstyle{\rem(k, 4) = 3}}^n a_k }.
    \end{equation*}

    This is clearly a Gaussian integer.

    It remains to show that multiplication in \( \BbbZ[i] \) is compatible with multiplication in \( \BbbC \). But complex \hyperref[def:set_of_complex_numbers]{multiplication} is defined to be compatible with the notation \( a + bi \), that is,
    \begin{equation*}
    (a + bi) (c + di)
    =
    ac + ibc + iad - bd
    =
    (ac - bd) + i(bc + ad).
    \end{equation*}

    Thus, the Gaussian integers are precisely the homomorphic image of \( \BbbZ[X] \) under \( \Phi_i \).
  \end{thmenum}
\end{example}

\begin{example}\label{ex:integers_modulo_sqrt2}
  Similarly to how the Gaussian integers were defined in multiple ways in \fullref{ex:gaussian_integers},
  \begin{equation*}
    \BbbZ[X] / \braket{X^2 - 2} \cong \BbbZ[\sqrt 2].
  \end{equation*}

  The gist of this example is that, even though \( \BbbZ[\sqrt 2] \) and \( \BbbZ[i] \) are isomorphic as modules, their vector multiplication operation is different. Multiplication modulo \( X^2 - 2 \) works as follows:
  \begin{balign*}
    (aX + b) (cX + bd)
    &\cong
    acX^2 + (bc + ad)X + bd
    &\pmod X^2 - 2 \cong \\ &\cong
    ac[X^2 - 2] + [(bc + ad)X + 2ac + bd]
    &\pmod X^2 - 2 \cong \\ &\cong
    (bc + ad)X + (2ac + bd)
    &\pmod X^2 - 2. \phantom{\cong}
  \end{balign*}
\end{example}

\begin{definition}\label{def:algebraic_derivative}
  Generalizing \fullref{def:differentiability} from analysis, we define the \term{algebraic derivative} of a univariate polynomial \( p(X) \) as
  \begin{equation*}
    p'(X) \coloneqq n a_n X^{n-1} + (n-1) a_{n-1} X^{n-2} + \cdots + a_2 X + a_1.
  \end{equation*}
\end{definition}

\begin{proposition}\label{thm:def:algebraic_derivative/properties}
  \hyperref[def:algebraic_derivative]{Algebraic derivatives} have the following basic properties:
  \begin{thmenum}
    \thmitem{thm:def:algebraic_derivative/properties/linear} The derivative operator \( p(X) \mapsto p'(X) \) is linear.
    \thmitem{thm:def:algebraic_derivative/properties/product} We have the product rule
    \begin{equation*}
      (pq)' = p'q + pq'.
    \end{equation*}

    \thmitem{thm:def:algebraic_derivative/properties/affine_power} The derivative of \( (X - u)^n \) is \( n(X - u)^{n-1} \).
  \end{thmenum}
\end{proposition}
\begin{proof}
  \SubProofOf{thm:def:algebraic_derivative/properties/linear} Trivial.
  \SubProofOf{thm:def:algebraic_derivative/properties/product} By \fullref{thm:def:algebraic_derivative/properties/linear}, it is enough to consider the case where both \( p(X) \) and \( q(X) \) are monomials.

  \begin{align*}
    p'(X) q(X) + p(X) q'(X)
     & =
    n a_n X^{n-1} \cdot b_m X^m + a_n X^n \cdot m b_m X^{m-1}
    =    \\ &=
    (n + m) a_n b_m X^{n+m-1}
    =    \\ &=
    (a_n b_m X^{n+m})'
    =    \\ &=
    (pq)'(X).
  \end{align*}

  \SubProofOf{thm:def:algebraic_derivative/properties/affine_power} We use induction on \( n \). The case \( n = 1 \) is obvious. Assume that the statement holds for \( 1, \ldots, n - 1 \).

  By \fullref{thm:algebraic_derivative_product_rule}, the derivative of \( (X - u)^n \) is
  \begin{align*}
    &\phantom{{}={}}
    [a(X - u)^{n-1}]' (X - u) + a(X - u)^{n-1} [(X - u)]'
    = \\ &=
    a(n-1)(X - u)^{n-2} (X - u) + a(X - u)^{n-1}
    = \\ &=
    an(X - u)^{n-1}.
  \end{align*}
\end{proof}
