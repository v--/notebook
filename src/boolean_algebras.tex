\subsection{Boolean algebras}\label{subsec:boolean_algebras}

\begin{definition}\label{def:heyting_algebra}\mcite[10]{BezhanishviliHolliday2019}
  A \term{Heyting algebra} is a \hyperref[def:semilattice/distributive_lattice]{distributive lattice} \( \mscrX \) with a binary operation \( \rightarrow \) defined as
  \begin{equation}\label{eq:def:heyting_algebra/conditional}
    (x \rightarrow y) \coloneqq \bigvee\set{ a \in \mscrX \given a \wedge x \leq y }.
  \end{equation}

  In order for the operation to be well-defined, we require that the corresponding join exists for all \( y \) and \( z \). We call this operation the \term{conditional} in analogy with the \hyperref[def:propositional_language/connectives/conditional]{propositional connective}, although this operation is usually called \enquote{implication} because of \hyperref[def:material_implication]{material implication}.

  Heyting algebras are useful for defining semantics for intuitionistic logic -- see \fullref{def:propositional_heyting_semantics}.

  \begin{thmenum}
    \thmitem{def:heyting_algebra/pseudocomplement} For any element \( x \), we define its \term{pseudocomplement} as
    \begin{equation}\label{eq:def:heyting_algebra/pseudocomplement}
      \widetilde x
      \coloneqq
      (x \rightarrow \bot)
      =
      \bigvee\set{ a \in \mscrX \given a \wedge x = \bot }.
    \end{equation}

    \thmitem{def:heyting_algebra/theory} We extend the \hyperref[def:semilattice/theory]{theory of lattices} with the binary infix functional symbol \( \rightarrow \), a unary functional symbol \( \widetilde{\placeholder} \) and the axiom \eqref{eq:def:heyting_algebra/conditional} to obtain the \term{theory of Heyting algebras}.

    \thmitem{def:heyting_algebra/homomorphism} \hyperref[def:first_order_homomorphism]{Homomorphisms} between Heyting algebras are lattice homomorphisms with the additional requirement that homomorphisms preserve conditionals.

    \thmitem{def:heyting_algebra/category} The \hyperref[def:category_of_first_order_models]{category of models} for Heyting algebras is denoted by \( \cat{Heyt} \).
  \end{thmenum}
\end{definition}

\begin{example}\label{ex:topological_space_is_heyting_algebra}
  Somewhat similar to how the power set of a nonempty set is a Boolean algebra (see \fullref{thm:subsets_form_boolean_algebra}), the topology \( \mscrT \) of a \hyperref[def:topological_space]{topological space} \( (\mscrX, \mscrT) \) is a Heyting algebra. This is actually used in topological semantics (see \fullref{def:topological_semantics}).

  Indeed,
  \begin{itemize}
    \item \hyperref[def:semilattice/join]{Arbitrary joins} are given by \hyperref[def:set_union]{unions \( \bigcap \)}.
    \item \hyperref[def:semilattice/meet]{Finite meets} are given by \hyperref[def:set_intersection]{intersections \( \bigcup \)}.
    \item The \hyperref[def:semilattice/join]{top} is the space \( \mscrX \) itself.
    \item The \hyperref[def:semilattice/meet]{bottom} is the empty set.
    \item The \hyperref[eq:def:heyting_algebra/conditional]{conditional \( U \leadsto V \)} is then
    \begin{equation*}
      \bigcup\set[\Big]{ A \in \mscrT \given \underbrace{A \cap U}_{A \setminus (X \setminus U)} \subseteq V }
      =
      \bigcup\set[\Big]{ A \in \mscrT \given A \subseteq V \cup (X \setminus U) }
      =
      \inter((X \setminus U) \cup V),
    \end{equation*}
    which is actually similar to \fullref{thm:boolean_equivalences/conditional_cnf} despite the fact that arbitrary topologies are not Boolean algebras.

    \item As a result, the \hyperref[def:heyting_algebra/pseudocomplement]{pseudocomplement} is
    \begin{equation*}
      \widetilde U = \inter(X \setminus U).
    \end{equation*}
  \end{itemize}
\end{example}

\begin{definition}\label{def:boolean_algebra}\mcite{nLab:boolean_algebra}
  A \term{Boolean algebra} is a \hyperref[def:semilattice/distributive_lattice]{distributive lattice} with an \hyperref[def:set_with_involution]{involution} \( \overline \placeholder \) satisfying
  \begin{align}
    x \vee \overline x = \top \label{eq:def:boolean_algebra/complement/join}, \\
    x \wedge \overline x = \bot \label{eq:def:boolean_algebra/complement/meet}.
  \end{align}

  The member \( \overline x \) is called the \term{complement} of \( x \). By \fullref{thm:boolean_algebra_properties/unique_complement}, the complement is unique.

  \begin{thmenum}[series=def:boolean_algebra]
    \thmitem{def:boolean_algebra/conditional} We also define the binary operation \term{conditional} (\( \rightarrow \)) via
    \begin{equation}\label{eq:def:boolean_algebra/conditional}
      (x \rightarrow y) \coloneqq (\overline x \vee y)
    \end{equation}
    in analogy with \fullref{thm:boolean_equivalences/conditional_cnf}. This operation demonstrates that Boolean algebras are a special case of \hyperref[thm:boolean_algebra_properties/heyting_algebra]{Heyting algebras}.

    \thmitem{def:boolean_algebra/biconditional} It remains to define a binary operation corresponding to the \hyperref[def:propositional_language/connectives/biconditional]{propositional biconditional}. Inspired by \fullref{thm:boolean_equivalences/biconditional_via_conditionals}, define
    \begin{equation}\label{eq:def:boolean_algebra/biconditional}
      (x \leftrightarrow y) \coloneqq (x \rightarrow y) \wedge (y \rightarrow x).
    \end{equation}
  \end{thmenum}

  \begin{thmenum}[resume=def:boolean_algebra]
    \thmitem{def:boolean_algebra/theory} To obtain the \term{theory of Boolean algebras}, we replace the unary functional symbol \( \widetilde{\placeholder} \) with \( \overline \placeholder \) in the \hyperref[def:heyting_algebra/theory]{theory of Heyting algebras} with a unary functional symbol and the axioms \eqref{eq:def:boolean_algebra/complement/join}, \eqref{eq:def:boolean_algebra/complement/meet} and \eqref{eq:def:boolean_algebra/conditional}.

    \thmitem{def:boolean_algebra/homomorphism} \hyperref[def:first_order_homomorphism]{Homomorphisms} between Boolean algebras are simply lattice homomorphisms.

    Complements are automatically preserved because for any lattice homomorphism \( \varphi \) between the Boolean algebras \( \mscrX \) and \( \mscrY \),
    \begin{equation*}
      \varphi(x) \vee_\mscrY \varphi(\overline x) = \varphi(x \vee_\mscrX \overline x) = \varphi(\top_\mscrX) = \top_\mscrY,
    \end{equation*}
    and similarly for \( \wedge \), hence, by \fullref{thm:boolean_algebra_properties/unique_complement},
    \begin{equation*}
      \varphi(\overline x) = \overline {\varphi(x)}.
    \end{equation*}

    Implications are also automatically preserved because of \eqref{eq:def:boolean_algebra/conditional}.

    \thmitem{def:boolean_algebra/category} The \hyperref[def:category_of_first_order_models]{category of models} \( \cat{Bool} \) for Boolean algebras is a full subcategories of the \hyperref[def:semilattice/category]{category \( \cat{Lat} \) of lattices} (if formulated in terms of \( \leq \) rather than operations) and a subcategory of the \hyperref[def:heyting_algebra/category]{category \( \cat{Heyt} \) of Heyting algebras}.
  \end{thmenum}
\end{definition}

\begin{example}\label{ex:boolean_algebras}
  Examples of \hyperref[def:boolean_algebra]{Boolean algebras} include:

  \begin{itemize}
    \item The cosets of propositional formulas under semantic equivalence (see \fullref{thm:propositional_formula_cosets_are_boolean_functions/boolean_algebra}).
    \item The Galois field \( \BbbF_2 \) with suitably defined operations (see \fullref{thm:f2_is_boolean_algebra}).
    \item The power set of any set, usually taken to be a space with additional structure (see \fullref{thm:subsets_form_boolean_algebra}).
  \end{itemize}
\end{example}

\begin{proposition}\label{thm:boolean_algebra_properties}
  A Boolean algebra \( \mscrX \) has the following basic properties:
  \begin{thmenum}
    \thmitem{thm:boolean_algebra_properties/unique_complement} For each \( x \in \mscrX \), the complement \( \overline x \) is unique, i.e. it is the only member of \( \mscrX \) that satisfies \eqref{eq:def:boolean_algebra/complement/join} and \eqref{eq:def:boolean_algebra/complement/meet}.

    \thmitem{thm:boolean_algebra_properties/heyting_algebra} Every Boolean algebra is a Heyting algebra with an identification given by \eqref{eq:def:boolean_algebra/conditional}.
  \end{thmenum}
\end{proposition}
\begin{proof}
  \SubProofOf{thm:boolean_algebra_properties/unique_complement} If \( y \) and \( z \) both satisfy \eqref{eq:def:boolean_algebra/complement/join}, then
  \begin{balign*}
    y
    &\reloset {\eqref{eq:thm:binary_lattice_operations/identity/meet}} =
    y \wedge \top
    = \\ &\reloset {\eqref{eq:def:boolean_algebra/complement/join}} =
    y \wedge (z \vee x)
    = \\ &\reloset {\eqref{eq:def:semilattice/distributive_lattice/finite/join_of_meets}} =
    (y \wedge z) \vee (y \wedge x)
    = \\ &\reloset {\eqref{eq:def:boolean_algebra/complement/meet}} =
    y \wedge z
    = \\ &\reloset {\eqref{eq:def:boolean_algebra/complement/meet}} =
    (x \wedge z) \vee (y \wedge z)
    = \\ &\reloset {\eqref{eq:def:semilattice/distributive_lattice/finite/join_of_meets}} =
    (x \vee y) \wedge z
    = \\ &\reloset {\eqref{eq:thm:binary_lattice_operations/identity/meet}} =
    z.
  \end{balign*}

  \SubProofOf{thm:boolean_algebra_properties/heyting_algebra} Fix any \( x, y \in \mscrX \) in a Boolean algebra \( \mscrX \). We will show that \( x \rightarrow y \) as defined in \eqref{eq:def:boolean_algebra/conditional} satisfies \eqref{eq:def:heyting_algebra/conditional}.

  Let
  \begin{equation*}
    A \coloneqq \set{ a \in \mscrX \given a \wedge x \leq y }
  \end{equation*}
  be the set from \eqref{eq:def:heyting_algebra/conditional}.

  We will show that \( \overline x \vee y \) is an \hyperref[def:preordered_set/upper_and_lower_bound]{upper bound} of \( A \).

  Fix some \( a_0 \in A \). By definition of \( A \), we have
  \begin{equation*}
    a_0 \wedge x \leq y.
  \end{equation*}

  But this means that
  \begin{equation*}
    \underbrace{(a_0 \wedge x) \vee \overline x}_{a_0 \vee \overline x} \leq y \vee \overline x.
  \end{equation*}

  Since \( a_0 \leq a_0 \vee b \) for any \( b \in \mscrX \), it follows that \( a_0 \leq y \vee \overline x \). Therefore \( \overline x \vee y \) is indeed an upper bound of \( A \).

  Also note that
  \begin{equation*}
    (\overline x \vee y) \wedge x = \underbrace{(\overline x \vee x)}_{\top} \wedge (y \wedge x) = y \wedge x \leq y,
  \end{equation*}
  hence \( \overline x \vee y \in A \).

  Thus \( \overline x \vee y \) is both an upper bound of \( A \) and an element of \( A \), i.e. it is the least upper bound of \( A \). Therefore
  \begin{equation*}
    \overline x \vee y = \bigvee A.
  \end{equation*}
\end{proof}

\begin{theorem}[De Morgan's laws]\label{thm:de_morgans_laws}
  If \( \mscrX \) is a \hyperref[def:boolean_algebra]{Boolean algebra}, the following hold for any finite \hyperref[def:indexed_family]{family} \( \{ x_k \}_{k \in \mscrK} \subseteq \mscrX \):
  \begin{align}
    \overline{\bigvee_{k \in \mscrK} x_k} = \bigwedge_{k \in \mscrK} \overline{x_k} \label{eq:thm:de_morgans_laws/complement_of_join} \\
    \overline{\bigwedge_{k \in \mscrK} x_k} = \bigvee_{k \in \mscrK} \overline{x_k} \label{eq:thm:de_morgans_laws/complement_of_meet}
  \end{align}

  If \( \mscrX \) is a complete lattice, \( \mscrK \) may be taken to be any family, not necessarily finite.
\end{theorem}
\begin{proof}
  We will only show \eqref{eq:thm:de_morgans_laws/complement_of_join} since \eqref{eq:thm:de_morgans_laws/complement_of_meet} is dual.

  In order for \( \wedge_{k \in \mscrK} \overline{x_k} \) to be the complement of \( \vee_{k \in \mscrK} x_k \), the conditions \eqref{eq:def:boolean_algebra/complement/join} and \eqref{eq:def:boolean_algebra/complement/meet} need to be satisfied.

  By distributivity,
  \begin{equation*}
    \parens*{ \bigwedge_{k \in \mscrK} \overline{x_k} } \vee \parens*{ \bigvee_{m \in \mscrK} x_m }
    \reloset {\eqref{eq:def:semilattice/distributive_lattice/arbitrary/meet_of_joins}} =
    \bigwedge_{k \in \mscrK} \parens*{ \overline{x_k} \vee \bigvee_{m \in \mscrK} x_m }
    =
    \bigwedge_{k \in \mscrK} \parens*{ \underbrace{\overline{x_k} \vee x_k}_{\top} \vee \bigvee_{\mathclap{m \in \mscrK \setminus \set{k}}} x_m }
    \reloset {\eqref{eq:thm:binary_lattice_operations/constant_absorption/join}} =
    \bigwedge_{k \in \mscrK} \top
    =
    \top,
  \end{equation*}
  which proves \eqref{eq:def:boolean_algebra/complement/join}. The proof of \eqref{eq:def:boolean_algebra/complement/meet} is analogous.
\end{proof}
