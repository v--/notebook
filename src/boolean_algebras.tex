\subsection{Boolean algebras}\label{subsec:boolean_algebras}

\begin{definition}\label{def:heyting_algebra}\mcite[10]{BezhanishviliHolliday2019}
  A \term{Heyting algebra} is a \hyperref[def:semilattice/bounded]{bounded} \hyperref[def:semilattice/distributive_lattice]{distributive lattice} \( \mscrX \) with a binary operation \( \rightarrow \) defined as
  \begin{equation}\label{eq:def:heyting_algebra/conditional}
    (x \rightarrow y) \coloneqq \bigvee\set{ a \in \mscrX \given a \wedge x \leq y }.
  \end{equation}

  In order for the operation to be well-defined, we require that the corresponding join exists for all \( y \) and \( z \). We call this operation the \term{conditional} in analogy with the \hyperref[def:propositional_language/connectives/conditional]{propositional connective}, although this operation is often called \enquote{implication} because of \hyperref[def:material_implication]{material implication}.

  Heyting algebras are useful for defining truth values for intuitionistic logic -- see \fullref{def:propositional_heyting_algebra_semantics} --- and also appear as the \hyperref[def:lindenbaum_tarski_algebra]{Lindenbaum-Tarski algebra} for propositional intuitionistic theories --- see \fullref{thm:intuitionistic_lindenbaum_tarski_algebra}.

  \begin{thmenum}
    \thmitem{def:heyting_algebra/pseudocomplement} For any element \( x \), we define its \term{pseudocomplement} as
    \begin{equation}\label{eq:def:heyting_algebra/pseudocomplement}
      \widetilde x
      \coloneqq
      (x \rightarrow \bot)
      =
      \bigvee\set{ a \in \mscrX \given a \wedge x = \bot }.
    \end{equation}

    \thmitem{def:heyting_algebra/theory} We extend the language of the \hyperref[def:semilattice/theory]{theory of lattices} with the binary infix functional symbol \( \rightarrow \) and the unary functional symbol \( \widetilde{\placeholder} \). By adding the axiom \eqref{eq:def:heyting_algebra/conditional} to the theory of bounded distributive lattices, we obtain the theory of Heyting algebras.

    \thmitem{def:heyting_algebra/submodel} The Heyting subalgebras are the \hyperref[def:semilattice/submodel]{bounded sublattices} for which the conditional is well-defined.

    \thmitem{def:heyting_algebra/trivial} The \hyperref[thm:substructures_form_complete_lattice/bottom]{trivial Heyting algebra} is simply the \hyperref[def:semilattice/trivial]{trivial bounded lattice} \( \set{ \top, \bot } \).

    \thmitem{def:heyting_algebra/homomorphism} \hyperref[def:first_order_homomorphism]{Homomorphisms} between Heyting algebras are lattice homomorphisms with the additional requirement that homomorphisms preserve conditionals.

    \thmitem{def:heyting_algebra/category} The \hyperref[def:category_of_first_order_models]{category of models} for Heyting algebras is denoted by \( \cat{Heyt} \). It is a full subcategory of the \hyperref[def:semilattice/category]{category \( \cat{Lat} \) of lattices}.

    \thmitem{def:heyting_algebra/duality} The \hyperref[def:semilattice/lattice_duality]{principle of duality for lattices} does not hold for Heyting algebras.
  \end{thmenum}
\end{definition}

\begin{example}\label{ex:topological_space_is_heyting_algebra}
  Somewhat similar to how the power set of a nonempty set is a Boolean algebra (see \fullref{thm:boolean_algebra_of_subsets}), the topology \( \mscrT \) of a \hyperref[def:topological_space]{topological space} \( (\mscrX, \mscrT) \) is a Heyting algebra. This is actually used in topological semantics (see \fullref{def:propositional_topological_semantics}).

  Indeed,
  \begin{itemize}
    \item \hyperref[def:semilattice/join]{Arbitrary joins} are given by \hyperref[def:basic_set_operations/union]{unions \( \bigcap \)}.
    \item \hyperref[def:semilattice/meet]{Finite meets} are given by \hyperref[def:basic_set_operations/intersection]{intersections \( \bigcup \)}.
    \item The \hyperref[def:semilattice/join]{top} is the entire domain \( \mscrX \).
    \item The \hyperref[def:semilattice/meet]{bottom} is the empty set.
    \item The \hyperref[eq:def:heyting_algebra/conditional]{conditional \( U \leadsto V \)} is then
    \begin{equation*}
      \bigcup\set[\Big]{ A \in T \given \underbrace{A \cap U}_{A \setminus (\mscrX \setminus U)} \subseteq V }
      =
      \bigcup\set[\Big]{ A \in T \given A \subseteq V \cup (\mscrX \setminus U) }
      =
      \inter((\mscrX \setminus U) \cup V),
    \end{equation*}
    which is actually similar to \fullref{thm:boolean_equivalences/conditional_as_disjunction} despite the fact that arbitrary topologies are not Boolean algebras.

    \item As a result, the \hyperref[def:heyting_algebra/pseudocomplement]{pseudocomplement} is
    \begin{equation*}
      \widetilde U = \inter(\mscrX \setminus U).
    \end{equation*}
  \end{itemize}
\end{example}

\begin{definition}\label{def:bounded_lattice_complement}
  Let \( \mscrX \) be a \hyperref[def:semilattice/bounded]{bounded} \hyperref[def:semilattice/lattice]{lattice} and fix an element \( x \in \mscrX \). A \term{complement} of \( x \) is an element \( y \) such that
  \begin{align}
    x \vee y = \top \label{def:bounded_lattice_complement/join}, \\
    x \wedge y = \bot \label{def:bounded_lattice_complement/meet}.
  \end{align}

  Due to the commutativity of both \( \vee \) and \( \wedge \), \( y \) is a complement of \( x \) if and only if \( x \) is a complement of \( y \).
\end{definition}

\begin{proposition}\label{thm:distributive_bounded_lattice_unique_complement}
  In a \hyperref[def:semilattice/bounded]{bounded} \hyperref[def:semilattice/distributive_lattice]{distributive lattice} \( \mscrX \), each \( x \in \mscrX \) has at most one complement.

  Thus complementation can be regarded as a \hyperref[def:partial_function]{partial operation}.
\end{proposition}
\begin{proof}
  If \( y \) and \( z \) are both complements of \( x \), then
  \begin{balign*}
    y
    &\reloset {\eqref{eq:thm:binary_lattice_operations/identity/meet}} =
    y \wedge \top
    = \\ &\reloset {\eqref{def:bounded_lattice_complement/join}} =
    y \wedge (z \vee x)
    = \\ &\reloset {\eqref{eq:def:semilattice/distributive_lattice/finite/meet_over_join}} =
    (y \wedge z) \vee (y \wedge x)
    = \\ &\reloset {\eqref{def:bounded_lattice_complement/meet}} =
    y \wedge z
    = \\ &\reloset {\eqref{def:bounded_lattice_complement/meet}} =
    (x \wedge z) \vee (y \wedge z)
    = \\ &\reloset {\eqref{eq:def:semilattice/distributive_lattice/finite/meet_over_join}} =
    (x \vee y) \wedge z
    = \\ &\reloset {\eqref{eq:thm:binary_lattice_operations/identity/meet}} =
    z.
  \end{balign*}
\end{proof}

\begin{definition}\label{def:boolean_algebra}\mcite[48]{Gratzer1978}
  A \term{Boolean algebra} is a \hyperref[def:semilattice/bounded]{bounded} \hyperref[def:semilattice/distributive_lattice]{distributive lattice} in which every element has a \hyperref[def:bounded_lattice_complement]{complement}. The complement of each element is unique due to \fullref{thm:distributive_bounded_lattice_unique_complement}. We define a unary function that gives to every element \( x \) its complement \( \overline \placeholder \). This function by definition is an \hyperref[def:set_with_involution]{involution}.

  \begin{thmenum}[series=def:boolean_algebra]
    \thmitem{def:boolean_algebra/conditional} We also define the binary operation \term{conditional} (\( \rightarrow \)) via
    \begin{equation}\label{eq:def:boolean_algebra/conditional}
      (x \rightarrow y) \coloneqq (\overline x \vee y)
    \end{equation}
    in analogy with \fullref{thm:boolean_equivalences/conditional_as_disjunction}. This operation highlights that Boolean algebras are a special case of \hyperref[thm:boolean_algebras_are_heyting_algebras]{Heyting algebras}.

    \thmitem{def:boolean_algebra/biconditional} It remains to define a binary operation corresponding to the \hyperref[def:propositional_language/connectives/biconditional]{propositional biconditional}. Inspired by \fullref{thm:boolean_equivalences/biconditional_via_conditionals}, define
    \begin{equation}\label{eq:def:boolean_algebra/biconditional}
      (x \leftrightarrow y) \coloneqq (x \rightarrow y) \wedge (y \rightarrow x).
    \end{equation}
  \end{thmenum}

  Boolean algebras have the following metamathematical structure:
  \begin{thmenum}[resume=def:boolean_algebra]
    \thmitem{def:boolean_algebra/theory} To obtain the theory of Boolean algebras, we replace the unary functional symbol \( \widetilde{\placeholder} \) with \( \overline \placeholder \) in the language of the \hyperref[def:heyting_algebra/theory]{theory of Heyting algebras} and then add the axioms \eqref{def:bounded_lattice_complement/join} and \eqref{def:bounded_lattice_complement/meet} to the theory. We may also replace \eqref{eq:def:heyting_algebra/conditional} defining \( \rightarrow \) with the simpler axiom \eqref{eq:def:boolean_algebra/conditional}.

    \thmitem{def:boolean_algebra/submodel} The Boolean subalgebras are the \hyperref[def:semilattice/submodel]{bounded sublattices} which are closed under compementation.

    \thmitem{def:boolean_algebra/trivial} The \hyperref[thm:substructures_form_complete_lattice/bottom]{trivial Boolean algebra} is simply the \hyperref[def:semilattice/trivial]{trivial bounded lattice} \( \set{ \top, \bot } \).

    \thmitem{def:boolean_algebra/homomorphism} \hyperref[def:first_order_homomorphism]{Homomorphisms} between Boolean algebras are simply lattice homomorphisms.

    Complements are automatically preserved because for any lattice homomorphism \( \varphi \) between the Boolean algebras \( \mscrX \) and \( \mscrY \),
    \begin{equation*}
      \varphi(x) \vee_\mscrY \varphi(\overline x) = \varphi(x \vee_\mscrX \overline x) = \varphi(\top_\mscrX) = \top_\mscrY,
    \end{equation*}
    and similarly for \( \wedge \), hence, due to \fullref{thm:distributive_bounded_lattice_unique_complement},
    \begin{equation*}
      \varphi(\overline x) = \overline {\varphi(x)}.
    \end{equation*}

    Implications are also automatically preserved because of \eqref{eq:def:boolean_algebra/conditional}.

    \thmitem{def:boolean_algebra/category} The \hyperref[def:category_of_first_order_models]{category of models} \( \cat{Bool} \) for Boolean algebras is a full subcategory the \hyperref[def:heyting_algebra/category]{category \( \cat{Heyt} \) of Heyting algebras}.

    \thmitem{def:boolean_algebra/duality} The \hyperref[def:semilattice/lattice_duality]{principle of duality for lattices} holds for Boolean algebras without interchanging complements.
  \end{thmenum}
\end{definition}

\begin{example}\label{ex:boolean_algebras}
  Examples of \hyperref[def:boolean_algebra]{Boolean algebras} include:

  \begin{itemize}
    \item The \hyperref[def:lindenbaum_tarski_algebra]{Lindenbaum-Tarski algebras} for propositional theories (see \fullref{thm:intuitionistic_lindenbaum_tarski_algebra}).
    \item The Galois field \( \BbbF_2 \) with suitably defined operations (see \fullref{thm:f2_is_boolean_algebra}).
    \item The power set of any set, usually taken to be a space with additional structure (see \fullref{thm:boolean_algebra_of_subsets}).
  \end{itemize}
\end{example}

\begin{proposition}\label{thm:boolean_algebras_are_heyting_algebras}
  Every \hyperref[def:boolean_algebra]{Boolean algebra} is a \hyperref[def:heyting_algebra]{Heyting algebra} with an identification given by \eqref{eq:def:boolean_algebra/conditional}.
\end{proposition}
\begin{proof}
  Fix any \( x, y \in \mscrX \) in a Boolean algebra \( \mscrX \). We will show that \( x \rightarrow y \) as defined in \eqref{eq:def:boolean_algebra/conditional} satisfies \eqref{eq:def:heyting_algebra/conditional}.

  Let
  \begin{equation*}
   A \coloneqq \set{ a \in \mscrX \given a \wedge x \leq y }
  \end{equation*}
  be the set from \eqref{eq:def:heyting_algebra/conditional}.

  We will show that \( \overline x \vee y \) is an \hyperref[def:poset_extremal_points/upper_and_lower_bounds]{upper bound} of \( A \).

  Fix some \( a_0 \in A \). By definition of \( A \), we have
  \begin{equation*}
   a_0 \wedge x \leq y.
  \end{equation*}

  But this means that
  \begin{equation*}
   \underbrace{(a_0 \wedge x) \vee \overline x}_{a_0 \vee \overline x} \leq y \vee \overline x.
  \end{equation*}

  Since \( a_0 \leq a_0 \vee b \) for any \( b \in \mscrX \), it follows that \( a_0 \leq y \vee \overline x \). Therefore \( \overline x \vee y \) is indeed an upper bound of \( A \).

  Also note that
  \begin{equation*}
   (\overline x \vee y) \wedge x = \underbrace{(\overline x \vee x)}_{\top} \wedge (y \wedge x) = y \wedge x \leq y,
  \end{equation*}
  hence \( \overline x \vee y \in A \).

  Thus \( \overline x \vee y \) is both an upper bound of \( A \) and an element of \( A \), i.e. it is the least upper bound of \( A \). Therefore
  \begin{equation*}
   \overline x \vee y = \bigvee A.
  \end{equation*}
\end{proof}

\begin{theorem}[De Morgan's laws]\label{thm:de_morgans_laws}
  If \( \mscrX \) is a \hyperref[def:boolean_algebra]{Boolean algebra}, the following hold for any finite \hyperref[def:indexed_family]{family} \( \{ x_k \}_{k \in \mscrK} \subseteq \mscrX \):
  \begin{align}
    \overline{\bigvee_{k \in \mscrK} x_k} = \bigwedge_{k \in \mscrK} \overline{x_k} \label{eq:thm:de_morgans_laws/complement_of_join} \\
    \overline{\bigwedge_{k \in \mscrK} x_k} = \bigvee_{k \in \mscrK} \overline{x_k} \label{eq:thm:de_morgans_laws/complement_of_meet}
  \end{align}

  If \( \mscrX \) is \hyperref[def:semilattice/complete]{complete}, \( \mscrK \) may be taken to be any family, not necessarily finite.
\end{theorem}
\begin{proof}
  We will only show \eqref{eq:thm:de_morgans_laws/complement_of_join} since \eqref{eq:thm:de_morgans_laws/complement_of_meet} is dual.

  In order for \( \wedge_{k \in \mscrK} \overline{x_k} \) to be the complement of \( \vee_{k \in \mscrK} x_k \), the conditions \eqref{def:bounded_lattice_complement/join} and \eqref{def:bounded_lattice_complement/meet} need to be satisfied.

  From distributivity we have
  \begin{equation*}
    \parens*{ \bigwedge_{k \in \mscrK} \overline{x_k} } \vee \parens*{ \bigvee_{m \in \mscrK} x_m }
    \reloset {\eqref{eq:def:semilattice/distributive_lattice/arbitrary/join_over_meet}} =
    \bigwedge_{k \in \mscrK} \parens*{ \overline{x_k} \vee \bigvee_{m \in \mscrK} x_m }
    =
    \bigwedge_{k \in \mscrK} \parens*{ \underbrace{\overline{x_k} \vee x_k}_{\top} \vee \bigvee_{\mathclap{m \in K \setminus \set{k}}} x_m }
    =
    \bigwedge_{k \in \mscrK} \top
    =
    \top,
  \end{equation*}
  which proves \eqref{def:bounded_lattice_complement/join}. The proof of \eqref{def:bounded_lattice_complement/meet} is analogous.
\end{proof}
