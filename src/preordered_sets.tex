\subsection{Preordered sets}\label{subsec:preordered_sets}

\begin{definition}\label{def:preordered_set}\mcite[def. 2.14]{OpenLogicFull}
  A \term{preordered set} is a set \( P \) endowed with a \hyperref[def:binary_relation/reflexive]{reflexive} and \hyperref[def:binary_relation/transitive]{transitive} \hyperref[def:binary_relation]{binary relation} \( \leq \). The relation itself is called a \term{preorder}.

  It is conventional to use the same symbol \( \leq \) as for \hyperref[def:partially_ordered_set]{partial orders}, however the lack of \hyperref[def:binary_relation/antisymmetric]{antisymmetry} may be confusing --- see \fullref{ex:preorder_nonuniqueness}.

  We define \( \geq \) as the \hyperref[def:binary_relation/converse]{inverse relation} of \( \leq \).

  Preordered sets have the following metamathematical properties:
  \begin{thmenum}
    \thmitem{def:preordered_set/theory} Consider a \hyperref[def:first_order_language]{first-order language} \( \mscrL \) with two \hyperref[rem:first_order_formula_conventions/infix]{infix} binary predicate symbols --- \( \leq \) and \( \geq \).

    The theory of preordered sets is a \hyperref[def:first_order_theory]{first-order theory} in \( \mscrL \) consisting of the axioms \eqref{eq:def:binary_relation/reflexive} and \eqref{eq:def:binary_relation/transitive} for \( \leq \) and the compatibility axiom
    \begin{equation}\label{eq:def:preordered_set/theory}
      (\xi \leq \eta) \leftrightarrow (\eta \geq \xi).
    \end{equation}

    \thmitem{def:preordered_set/homomorphism} A \hyperref[def:first_order_homomorphism]{homomorphism} from \( (P, \leq_P) \) to \( (Q, \leq_Q) \) is, explicitly, a function \( f: P \to Q \) such that
    \begin{equation}\label{eq:def:preordered_set/homomorphism}
      x \leq_P y \T{implies} f(x) \leq_Q f(y).
    \end{equation}

    These are precisely the \hyperref[eq:def:partially_ordered_set/homomorphism/nonstrict]{nonstrict monotone maps}.

    \thmitem{def:preordered_set/submodel} Since the theory contains only positive formulas over a language with no functional symbols, any subset \( A \) of the domain of a preordered set \( P \) becomes a preordered set with the induced preorder \( \leq_A \) defined as the \hyperref[def:binary_relation/restriction]{restriction} of \( \leq_P \) to only elements of \( A \).

    \thmitem{def:preordered_set/trivial} The \hyperref[thm:substructures_form_complete_lattice/bottom]{trivial preordered set} is the empty set. See \fullref{rem:empty_models} regarding allowing empty sets as first-order structures.

    \thmitem{def:preordered_set/category}  We denote the \hyperref[def:category_of_small_first_order_models]{category of \( \mscrU \)-small models} for the theory of preordered sets by \( \ucat{PreOrd} \).

    This category is equivalent to that of \( \mscrU \)-small thin categories --- see \fullref{thm:order_category_isomorphism/preordered}.

    \thmitem{def:preordered_set/duality} We define the \term{opposite preordered set} of \( (P, \leq) \) as \( (P, \geq) \).

    The \term{principle of duality} states that the formula \( \varphi \) is derivable in the \hyperref[def:preordered_set/theory]{theory of preordered sets} if and only if the opposite formula \( \varphi^{\opcat} \), in which we swap all instances of \( \leq \) and \( \geq \), is also derivable. Observe that \( \varphi \) is satisfied in a preordered set if and only if \( \varphi^{\opcat} \) is satisfied in the opposite preordered set.

    There is a actually a very simple proof. If \( \varphi \) is derivable in the theory, it is satisfied by every preordered set. Let \( P \) be a preordered set. Then \( \varphi \) is satisfied in both \( (P, \leq) \) and its opposite \( (P, \geq) \). But if \( \varphi \) is valid for the opposite preordered set \( (P, \geq) \), its opposite formula \( \varphi^{\opcat} \) is valid for the double opposite preordered set, which is \( (P, \leq) \). Since \( (P, \leq) \) was chosen arbitrarily, the opposite formula \( \varphi^{\opcat} \) is valid for every preordered set and, so it belongs to the theory of preordered sets.

    The opposite of the opposite formula of \( \varphi \) is obviously \( \varphi \). The actual replacement can be formalized by performing the \hyperref[def:first_order_substitution/term_in_formula]{simultaneous substitution}
    \begin{equation*}
      \begin{aligned}
        \varphi^{\opcat} \coloneqq \varphi[
          &\xi_1 \leq \eta_1 \mapsto \xi_1 \geq \eta_1, &&\xi_1 \geq \eta_1 \mapsto \xi_1 \leq \eta_1, \\
          &\vdots                                       &&\vdots \\
          &\xi_n \leq \eta_n \mapsto \xi_n \geq \eta_n, &&\xi_n \geq \eta_n \mapsto \xi_n \leq \eta_n]
      \end{aligned}
    \end{equation*}
    for all pairs \( (\xi_k, \eta_k) \) of free variables in \( \varphi \).

    Another form of this duality is formalized in \fullref{thm:order_category_isomorphism}.
  \end{thmenum}
\end{definition}

\begin{definition}\label{def:directed_set}\mcite[8]{Engelking1989}
  A \hyperref[def:preordered_set]{preordered set} \( P \) is called an \term{upward/right directed set} if every finite subset of \( P \) has an \hyperref[def:partially_ordered_set_extremal_points/upper_and_lower_bounds]{upper bound}, i.e. for all \( x, y \in P \) there must exist \( z \in P \) such that \( x \leq z \) and \( y \leq z \). We do not care how many upper bounds exist and how they are related, we simply need one upper bound to exist for every pair of elements of \( P \).

  \hyperref[def:preordered_set/duality]{Dually}, \( P \) is a \term{downward/left directed set} if every two elements have a lower bound.

  Directed sets are used to define nets in topological spaces, see \fullref{def:topological_net}.
\end{definition}

\begin{definition}\label{def:cofinal_set}
  A subset \( A \) of a preordered set \( (P, \leq) \) is called \term{cofinal} if for every \( x \in P \) there exists some \( y \in A \) such that \( x \leq y \).
\end{definition}

\begin{example}\label{ex:def:cofinal_set}
  We list several examples of \hyperref[def:cofinal_set]{cofinal} and non-cofinal sets.

  \begin{itemize}
    \thmitem{ex:def:cofinal_set/finite} In a finite set like \( \set{ 0, 1, 2 } \), the set \( \set{ 2 } \) containing the maximum is cofinal. This is generalized by \fullref{thm:partially_ordered_cofinal_equivalences}.

    \thmitem{ex:def:cofinal_set/integers} Consider the set \( \BbbZ \) of integers. Clearly the set \( 2\BbbZ \) of even integers is cofinal. This is generalized by \fullref{thm:totally_ordered_cofinal_equivalences}.

    \thmitem{ex:def:cofinal_set/net_convergence} Cofinal sets are important in topology because it is used to define \hyperref[def:net_convergence]{convergence of nets}.

    \thmitem{ex:def:cofinal_set/regular_cardinals} \hyperref[def:regular_cardinal]{Regular cardinals} are equal to their own \hyperref[def:cofinality]{cofinality}.
  \end{itemize}
\end{example}
