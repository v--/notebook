\subsection{Semiring ideals}\label{subsec:semiring_ideals}

When regarding \hyperref[def:semiring]{semirings} as \hyperref[def:semimodule]{semimodules} over themselves, as per \fullref{thm:semiring_is_semimodule}, we obtain several important notions.

\begin{definition}\label{def:semiring_ideal}\mimprovised
  Fix a semiring \( R \). A \term{left ideal} of \( R \) is a \hyperref[def:semimodule/submodel]{sub-semimodule} of \( R \) when regarded as a left semimodule over itself, and a \term{right ideal} is defined analogously.

  If \( I \) is both a left and right ideal of \( R \), we say that it is a \term{two-sided ideal} or simply \term{ideal}. We will almost exclusively deal with two-sided ideals.

  More explicitly, \( I \) is a two-sided ideal of \( R \) if it is a \hyperref[def:monoid/submodel]{submonoid} of the additive monoid of \( R \) that is closed under left and right multiplication, i.e. \( RI = I = IR \).

  Of course, if multiplication is commutative, every left ideal is a right ideal and there is no distinction between the two.

  \begin{thmenum}
    \thmitem{def:semiring_ideal/generated} For an arbitrary subset \( A \) of \( R \), we denote the (two-sided) \hyperref[def:semimodule/submodel]{linear span} of \( A \) by \( \braket{ A } \) and call it the ideal \term{generated} by \( S \). This is mostly useful for commutative semirings, where the linear span amounts to linear combinations due to \fullref{thm:span_via_linear_combinations}.

    This is not the same as the sub-semiring generated as a first-order substructure as described in \fullref{def:first_order_generated_substructure}.

    \thmitem{def:semiring_ideal/principal} An ideal generated by a single element is called a \term{principal ideal}.
  \end{thmenum}
\end{definition}

\begin{remark}\label{rem:semiring_ideal_as_sub_semiring}
  A proper semiring ideal is a canonical example of a non-unital sub-semiring. As a consequence of \fullref{thm:def:derived_semiring_ideals/properties/proper_ideals_containing_identity}, a proper ideal cannot contain the multiplicative identity \( 1 \), and is thus not a sub-semiring unless we allow sub-semirings to not contain \( 1 \).
\end{remark}

\begin{proposition}\label{thm:def:semiring_ideal/properties}
  \hyperref[def:semiring_ideal]{Semiring ideals} have the following basic properties:
  \begin{thmenum}
    \thmitem{thm:def:semiring_ideal/properties/proper_ideals_containing_identity} A left or right \hyperref[def:semiring_ideal]{semiring ideal} contains the multiplicative identity if and only if it is not proper. In particular, \( R = \braket{ 1_R } \).

    \thmitem{thm:def:semiring_ideal/properties/division} We have \( x \mid y \) if and only if \( \braket{y} \subseteq \braket{x} \).

    \thmitem{thm:def:semiring_ideal/properties/product_of_principal_ideals} In a \hyperref[def:semiring/commutative]{commutative semiring} \( R \), the product of the principal ideals \( \braket{x} \) and \( \braket{y} \) is \( \braket{xy} \).
  \end{thmenum}
\end{proposition}
\begin{proof}
  \SubProofOf{thm:def:semiring_ideal/properties/proper_ideals_containing_identity} Fix a semiring \( R \) and a left ideal \( I \) of \( R \). We will prove that \( 1 \in I \) if and only if \( I = R \).

  \SufficiencySubProof* Let \( 1 \in I \). Then \( 1x = x \) for any \( x \in R \), thus \( IR = R \). But \( I \) is an ideal, hence we have that \( IR = I \). Therefore, \( I = IR = R \).

  \NecessitySubProof* If \( I = R \), then obviously \( 1 \in I \).

  An analogous proof follows for the case when \( I \) is a right ideal.

  \SubProofOf{thm:def:semiring_ideal/properties/division}

  \SufficiencySubProof* Suppose that \( x \mid y \). Then there exist \( l \) and \( r \) in \( R \) such that \( y = lx = xr \). Since \( \braket{x} \) is an ideal, it is closed under multiplication, and hence \( y \in \braket{x} \). Furthermore, every left multiple of \( y \) is a multiple of \( lx \) and hence \( x \), and analogously for right multiples, hence \( \braket{y} \subseteq \braket{x} \).

  \NecessitySubProof* Suppose that \( \braket{y} \subseteq \braket{x} \). Then \( y \in \braket{x} \), and hence there exists an element \( l \) of \( R \) such that \( y = lx \). So \( x \) is a right divisor of \( y \). Analogously, \( x \) is a left divisor of \( y \). Therefore, \( x \mid y \).

  \SubProofOf{thm:def:semiring_ideal/properties/product_of_principal_ideals} \hfill
  \SufficiencySubProof* Let \( z \in \braket{x} \braket{y} \). Then there exist elements \( x_z \) of \( \braket{x} \) and \( y_z \) of \( \braket{y} \) such that \( z = x_z y_z \), and elements \( r_x \) and \( r_y \) of \( R \) such that \( x r_x = x_z \) and \( y r_y = y_z \).

  Therefore, \( z = x r_x y r_y \in \braket{xy} \).

  \NecessitySubProof* Let \( z \in \braket{xy} \). Then there exists an element \( r \) of \( R \) such that \( z = rxy = (rx)(y) \), hence \( z \in \braket{x} \braket{y} \).
\end{proof}

\begin{proposition}\label{thm:semiring_of_ideals}
  \hfill
  \begin{thmenum}
    \thmitem{thm:semiring_of_ideals/semiring} The set \( \mscrI \) of all ideals of a semiring \( R \) is itself a semiring with the set addition and multiplication from \hyperref[def:semiring/power_set]{power set semiring} \( \pow(R) \).

    \thmitem{thm:semiring_of_ideals/order} Furthermore, \( \mscrI \) is an \hyperref[def:ordered_semiring]{ordered semiring} with respect to set inclusion.

    \thmitem{thm:semiring_of_ideals/lattice} The \hyperref[def:partially_ordered_set_extremal_points/supremum_and_infimum]{supremum} of \( I \) and \( J \) is their sum \( I + J \) and their \hyperref[def:partially_ordered_set_extremal_points/supremum_and_infimum]{infimum} is their intersection \( I \cap J \). With this, \( \mscrI \) becomes a lattice.

    \thmitem{thm:semiring_of_ideals/inclusion} We have
    \begin{equation*}
      IJ \subseteq I \cap J \subseteq I + J.
    \end{equation*}
  \end{thmenum}
\end{proposition}
\begin{proof}
  \SubProofOf{thm:semiring_of_ideals/semiring} Associativity and commutativity in \( \mscrI \) are inherited from \( R \), as well as both left and right distributivity. Distributivity ensures that \( I + J \) is an ideal, while associativity of multiplication ensures that \( IJ \) is an ideal.

  \SubProofOf{thm:semiring_of_ideals/order} We must now prove that the partial order \( \subseteq \) is compatible with addition and multiplication. Suppose that \( I \subseteq J \) and let \( H \) be any ideal in \( \mscrI \). Then
  \begin{equation*}
    I + H \subseteq J + H
  \end{equation*}
  and
  \begin{equation*}
    IH \subseteq IH.
  \end{equation*}

  Therefore, \( \mscrI \) is an ordered semiring.

  \SubProofOf{thm:semiring_of_ideals/lattice} Since \( 0 \in I \), obviously \( I \subseteq I + J \), and thus \( I + J \) is an upper bound of \( I \) and \( J \). If \( H \) is any other upper bound, it must contain the sums of all elements of \( I \) and all elements of \( J \), hence \( I + J \subseteq H \). Therefore, \( \sup\set{I, J} = I + J \).

  For \( I \cap J \), it is an infimum of \( I \) and \( J \) as a consequence of \fullref{thm:boolean_algebra_of_subsets/meet}.

  \SubProofOf{thm:semiring_of_ideals/inclusion} Clearly \( I \cap J \subseteq I + J \). We will show that \( IJ \subseteq I \cap J \). If \( xy \in IJ \), then \( xy \in xJ = J \) and \( xy \in Iy = I \), hence \( xy \in I \cap J \).
\end{proof}

\begin{example}\label{ex:def:semiring_ideal}
  We list several examples of \hyperref[def:semiring_ideal]{semiring ideals}
  \begin{thmenum}
    \thmitem{ex:def:derived_semiring_ideals/prime_numbers} Consider the natural number divisibility lattice described in \fullref{thm:natural_number_divisibility_order}. By \fullref{thm:def:semiring_ideal/properties/division}, the lattice of principal ideals in \( \BbbN \) must be dual to it. Indeed, by \fullref{thm:bezout_identity}, we have that
    \begin{equation*}
      \braket{n} + \braket{m} = \braket{ \gcd(n, m) },
    \end{equation*}
    and by \fullref{thm:def:semiring_ideal/properties/division}, \( \braket{n} \cap \braket{m} \) contains the common multiples of \( n \) and \( m \), hence
    \begin{equation*}
      \braket{n} \cap \braket{m} = \braket{ \lcm(n, m) }.
    \end{equation*}

    The following Hasse diagram illustrates the duality:
    \begin{equation}\label{eq:thm:free_semimodule_universal_property/diagram}
      \begin{aligned}
        \includegraphics[page=1]{output/ex__def__semiring_ideal.pdf}
      \end{aligned}
    \end{equation}

    \thmitem{ex:def:derived_semiring_ideals/polynomial_ideals} Consider the bivariate \hyperref[def:polynomial_semiring]{polynomial semiring} \( \BbbN[X, Y] \) over natural numbers. Since \( (X + Y)^2 = X^2 + 2XY + Y^2 \), we have
    \begin{equation*}
      \braket{ X^2 + 2XY + Y^2 } \subseteq \braket{ X + Y }.
    \end{equation*}

    \thmitem{ex:def:derived_semiring_ideals/ideal_polynomials} Ideals in polynomial (semi)rings are often studied, but we can also study polynomials in ideal semirings, i.e. polynomials over the semiring \( \mscrI \) of ideals of a semiring \( R \). For example,
    \begin{equation*}
      I^2 J + H
    \end{equation*}
    is a trivariate polynomial function over \( \mscrI \).
  \end{thmenum}
\end{example}

\begin{definition}\label{def:derived_semiring_ideals}
  We introduce several important types of \hyperref[def:semiring_ideal]{semiring ideals} of the semiring \( R \).

  \begin{thmenum}
    \thmitem{def:derived_semiring_ideals/prime}\mcite[85]{Golan2010} If from \( IJ \subseteq P \) it follows that either \( I \subseteq P \) or \( J \subseteq P \), we say that \( P \) is a \term{prime ideal}.

    \thmitem{def:derived_semiring_ideals/coprime}\mcite[18]{КоцевСидеров2016} If \( I + J = R \), we say that the ideals \( I \) and \( J \) are \term{coprime}. Equivalently, \( I \) and \( J \) are coprime if their sum contains the identity.

    \thmitem{def:derived_semiring_ideals/radical}\mcite[15]{КоцевСидеров2016} The \term{radical} of an ideal \( I \) is the ideal
    \begin{equation*}
      \sqrt I \coloneqq \set{ x \in R \given \qexists {n \in \BbbZ_{>0}} x^n \in I }.
    \end{equation*}

    \thmitem{def:derived_semiring_ideals/maximal} A \term{maximal ideal} is a proper ideal that is maximal with respect to set inclusion. The maximal ideals are the predecessors of \( R \) in the lattice of ideals described in \fullref{thm:semiring_of_ideals}.
  \end{thmenum}
\end{definition}

\begin{proposition}\label{thm:def:semiring_ideal/properties}
  The ideals described in \fullref{def:derived_semiring_ideals} have the following basic properties:
  \begin{thmenum}
    \thmitem{thm:def:semiring_ideal/properties/maximal_is_prime} Every \hyperref[def:derived_semiring_ideals/maximal]{maximal ideal} is \hyperref[def:derived_semiring_ideals/prime]{prime}.

    \thmitem{thm:def:semiring_ideal/properties/nilradical} The \hyperref[def:derived_semiring_ideals/radical]{radical} \( \sqrt {\braket{0}} \) of the zero ideal is the intersection of all prime ideals. We call it the \term{nilradical} of the semiring.
  \end{thmenum}
\end{proposition}
\begin{proof}
  \SubProofOf{thm:def:semiring_ideal/properties/maximal_is_prime} Let \( M \) be a maximal ideal in the semiring \( R \) and let \( IJ \subseteq M \). Since \( M \) is maximal, \( M \subseteq M + I \subseteq R \), and at least one of the inclusion is an equality. Suppose that \( M + I = R \), i.e. \( M \) and \( I \) are \hyperref[def:derived_semiring_ideals/coprime]{coprime} ideals. Then, by the right distributivity in the \hyperref[thm:semiring_of_ideals/lattice]{semiring of ideals} of \( R \), we have
  \begin{equation*}
    J = RJ = (M + I)J = MJ + IJ \subseteq M + IJ = M.
  \end{equation*}

  Similarly, if \( M \) and \( J \) are coprime, then by left distributivity,
  \begin{equation*}
    I = IR = I(M + J) = IM + IJ \subseteq M + IJ = M.
  \end{equation*}

  \SubProofOf{thm:def:semiring_ideal/properties/nilradical}
  \SufficiencySubProof* Let \( x \in \sqrt{\braket{0}} \). That is, there exists a positive integer \( n \) such that \( x^n = 0 \). Let \( P \) be a prime ideal. We will show that \( x \in P \).

  Note that \( 0 \) necessarily belongs to \( P \). Since it is prime, either \( x^{n-1} \in P \) or \( x \in P \). If \( x^{n-1} \in P \), then either \( x^{n-2} \in P \) or \( x \in P \). Proceeding by induction on \( k \) in \( x^{n-k} \), we obtain that \( x \in P \).

  \NecessitySubProof* Conversely, let \( x \) be a member of every prime ideal. Suppose that \( x^n \neq 0 \) for every \( n \). Consider the following family of ideals:
  \begin{equation*}
    \mscrH \coloneqq \set{ I \T{is an ideal of} R \given \qforall {n \in \BbbZ_{>0}} x^n \not\in I }.
  \end{equation*}

  It is nonempty because \( \braket{0} \in \mscrH \).

  For every chain of ideals in \( \mscrH \), their union is also an ideal in \( \mscrH \). By \fullref{thm:zorns_lemma}, \( \mscrH \) has a maximal element \( H \). We will show that \( H \) is prime. From \( IJ \subseteq H \) it follows that \( IJ \in \mscrH \). If we suppose that neither \( I \) nor \( J \) belongs to \( \mscrH \), we obtain that there exist positive integers \( n_i \) and \( n_j \) such that \( x^{n_i} \in I \) and \( x^{n_j} \in J \). But \( x^{n_i + n_j} \) must then belong to \( IJ \), which contradicts \( IJ \in \mscrH \). The obtained contradiction demonstrates that \( H \) is a prime ideal. But this is impossible since, by assumption \( x \) is contained in every prime ideal, and \( x \not\in H \). So this must contradict our previous assumption that \( x^n \neq 0 \) for every \( n \).

  Therefore, \( x \) belongs to \( \sqrt{\braket{0}} \).
\end{proof}

\begin{theorem}[Maximal ideal theorem]\label{thm:maximal_ideal_theorem}\mcite[6.59]{Golan2010}
  Every proper \hyperref[def:semiring_ideal]{semiring ideal} is contained in a \hyperref[def:derived_semiring_ideals/maximal]{maximal ideal}.

  Within \hyperref[def:zfc]{\logic{ZF}}, this theorem is equivalent to the \hyperref[def:zfc/choice]{axiom of choice} --- see \fullref{thm:axiom_of_choice_equivalences/maximal_ideal}.
\end{theorem}
\begin{proof}
  We will show equivalence with \fullref{thm:zorns_lemma}.

  \ImplicationSubProof[thm:zorns_lemma]{Zorn's lemma}[thm:existence_of_multi_valued_function_selection]{maximal ideal theorem} Let \( I \) be a proper ideal in the semiring \( R \). Denote by \( \mscrH \) the set of all proper ideals in \( R \) that contain \( I \). The union of every chain in \( \mscrH \) is again an ideal, and by Zorn's lemma, \( \mscrH \) has a maximal element. That is, there exists a maximal ideal in \( \mscrH \) that contains \( I \).

  \ImplicationSubProof[thm:existence_of_multi_valued_function_selection]{maximal ideal theorem}[thm:zorns_lemma]{Zorn's lemma} Let \( (P, \leq) \) be a \hyperref[def:partially_ordered_set]{partially ordered set} in which every \hyperref[def:partially_ordered_set_chain_and_antichain]{chain} has an upper bound. We will show that \( P \) has a maximal element.
\end{proof}
