\subsection{Semiring ideals}\label{subsec:semiring_ideals}

When regarding \hyperref[def:semiring]{semirings} as \hyperref[def:semimodule]{semimodules} over themselves, as per \fullref{thm:semiring_is_semimodule}, we obtain several important notions.

\begin{definition}\label{def:semiring_ideal}\mimprovised
  Fix a semiring \( R \). A \term{left ideal} of \( R \) is a \hyperref[def:semimodule/submodel]{sub-semimodule} of \( R \) when regarded as a left semimodule over itself, and a \term{right ideal} is defined analogously.

  If \( I \) is both a left and right ideal of \( R \), we say that it is a \term{two-sided ideal} or simply \term{ideal}.

  More explicitly, \( I \) is a two-sided ideal of \( R \) if it is a \hyperref[def:monoid/submodel]{submonoid} of the additive monoid of \( R \) and that is closed under left and right multiplication, i.e. \( RI = I = IR \).

  Of course, if multiplication is commutative, every left ideal is a right ideal and there is no distinction between the two.

  \begin{thmenum}
    \thmitem{def:semiring_ideal/multiplication} If \( I \) and \( J \) are (two-sided) ideals in \( R \), we will sometimes consider their product \( IJ \) defined as \( \set{ xy \given x \in I \T{and} y \in J } \).

    This is an instance of multiplication in the \hyperref[def:magma/power_set]{power set semiring} of \( R \).

    \thmitem{def:semiring_ideal/generated} For an arbitrary subset \( S \) of \( R \), we denote the \hyperref[def:semimodule/submodel]{linear span} of \( S \) by \( \braket{ A } \) and call it the ideal \term{generated} by \( S \).

    This is not the same as the sub-semiring generated as a first-order substructure as described in \fullref{def:first_order_generated_substructure}.

    \thmitem{def:semiring_ideal/principal} An ideal generated by a single element is called a \term{principal ideal}.
  \end{thmenum}
\end{definition}

\begin{proposition}\label{thm:proper_ideals_containing_identity}
  A left or right \hyperref[def:semiring_ideal]{semiring ideal} contains the multiplicative identity if and only if it is not proper.
\end{proposition}
\begin{proof}
  Fix a semiring \( R \) and a left ideal \( I \) of \( R \). We will prove that \( 1 \in I \) if and only if \( I = R \).

  \SufficiencySubProof Let \( 1 \in I \). Then \( 1x = x \) for any \( x \in R \), thus \( IR = R \). But \( I \) is an ideal, hence we have that \( IR = I \). Therefore, \( I = IR = R \).

  \NecessitySubProof If \( I = R \), then obviously \( 1 \in I \).

  An analogous proof follows for the case when \( I \) is a right ideal.
\end{proof}

\begin{remark}\label{rem:semiring_ideal_as_sub_semiring}
  A proper semiring ideal is a canonical example of a non-unital sub-semiring. As a consequence of \fullref{thm:proper_ideals_containing_identity}, a proper ideal cannot contain the multiplicative identity \( 1 \), and is thus not a sub-semiring unless we allow sub-semirings to not contain \( 1 \).
\end{remark}

\begin{definition}\label{def:quotient_semiring}
  Consider the \hyperref[def:quotient_semimodule]{quotient semimodule} \( R / I \) of the semiring \( R \) by the ideal \( I \) and define the operation
  \begin{equation*}
    (x + I) \star (y + I) \coloneqq x \cdot y + I.
  \end{equation*}

  With this operation, \( R / I \) becomes a semiring, which we call the \term{quotient semiring} of \( R \) by \( I \).
\end{definition}
\begin{defproof}
  The multiplication operation \( \star \) on \( R / I \) inherits its properties from \( \cdot \). Hence, we only need to show that the operation is well-defined. Let \( x_1 + I = x_2 + I \) and \( y_1 + I = y_2 + I \).

  Let \( i_x \) and \( i_y \) be elements of \( I \) such that \( x_2 = x_1 + i_x \) and \( y_2 = y_1 + i_y \). Then left and right distributivity imply
  \begin{equation*}
    x_2 \cdot y_2 + I
    =
    (x_1 + i_x) \cdot (y_1 + i_y) + I
    =
    x_1 \cdot y_1 + i_x \cdot y_1 + i_y \cdot x_1 + i_x \cdot i_y + I
    =
    x_1 \cdot y_1 + I.
  \end{equation*}
\end{defproof}

\begin{definition}\label{def:semiring_kernel}\mcite[121]{Golan2010}
  The \term{kernel} \( \ker(\varphi) \) of a \hyperref[def:semiring/homomorphism]{semiring homomorphism} \( \varphi: R \to S \) is the \hyperref[def:zero_locus]{zero locus} of \( \varphi \). That is, \( \ker(\varphi) \) is the \hyperref[thm:def:function/properties/preimage]{preimage} \( \varphi^{-1}(0_S) \).

  This is in general not the kernel of \( \varphi \) when regarded as a semimodule homomorphism because \( R \) and \( S \) are semimodules over different semirings.

  Unlike \hyperref[def:pointed_set_kernel]{pointed set kernels}, \hyperref[thm:monoid_kernels]{monoid kernels}, \hyperref[thm:group_kernels]{group kernels} and \hyperref[thm:semimodule_kernels]{semimodule kernels}, this is not an instance of categorical kernels defined in \fullref{def:zero_morphisms/kernel}.
\end{definition}

\begin{proposition}\label{thm:semiring_kernel}
  The \hyperref[def:semiring_kernel]{kernel} of a semiring homomorphism is an \hyperref[def:semiring_ideal]{ideal} of \( R \).
\end{proposition}
\begin{proof}
  Consider the kernel \( \ker(\varphi) \) of the homomorphism \( \varphi: R \to S \). It is a kernel of the additive group, and by \fullref{thm:group_kernels}, it is a subgroup of \( R \). Also, if \( i \in \ker \varphi \) and \( x \in R \), then
  \begin{equation*}
    \varphi(xi)
    =
    \varphi(x) \varphi(i)
    =
    \varphi(x) 0_S
    =
    0_S.
  \end{equation*}

  Therefore, \( x \in \ker \varphi \), and \( \ker \varphi \) is an ideal of \( R \).
\end{proof}

\begin{proposition}\label{thm:product_of_semigroup_ideals_is_in_intersection}
  If \( I \) and \( J \) are semiring ideals in \( R \), so are \( IJ \) and \( I \cap J \) and, furthermore,
  \begin{equation*}
    IJ \subseteq I \cap J.
  \end{equation*}
\end{proposition}
\begin{proof}
  We first show that \( IJ \) is an ideal.

  Take \( x \in I \), \( y \in J \). If \( r \in R \), then associativity gives us
  \begin{equation*}
    r(xy) = (rx)y \in (rx)J \subseteq IJ
  \end{equation*}
  and
  \begin{equation*}
    (xy)r = x(yr) \in I(yr) \subseteq IJ.
  \end{equation*}

  Hence, \( IJ \) is closed under the semigroup operation. This makes \( IJ \) a two-sided ideal.

  If \( x \in I \cap J \) and \( r \in R \), obviously \( xr \in I \) and \( xr \in J \), hence \( xr \in I \cap J \). Then \( I \cap J \) is also a two-sided ideal.

  To obtain the inclusion
  \begin{equation*}
    IJ \subseteq I \cap J,
  \end{equation*}
  observe that \( xy \in IJ \) implies \( xy \in xJ = J \) and \( xy \in Iy = I \), hence \( xy \in I \cap J \).
\end{proof}

\begin{proposition}\label{thm:product_of_principal_ideals}
  In a \hyperref[def:semiring/commutative]{commutative semiring} \( R \), the product of the principal ideals \( \braket{x} \) and \( \braket{y} \) is \( \braket{xy} \).
\end{proposition}
\begin{proof}
  \SufficiencySubProof Let \( z \in \braket{x} \braket{y} \). Then there exist elements \( x_z \) of \( \braket{x} \) and \( y_z \) of \( \braket{y} \) such that \( z = x_z y_z \), and elements \( r_x \) and \( r_y \) of \( R \) such that \( x r_x = x_z \) and \( y r_y = y_z \).

  Therefore, \( z = x r_x y r_y \in \braket{xy} \).

  \NecessitySubProof Let \( z \in \braket{xy} \). Then there exists an element \( r \) of \( R \) such that \( z = rxy = (rx)(y) \), hence \( z \in \braket{x} \braket{y} \).
\end{proof}
