\subsection{Semiring ideals}\label{subsec:semiring_ideals}

When regarding \hyperref[def:semiring]{semirings} as \hyperref[def:semimodule]{semimodules} over themselves, as per \fullref{thm:semiring_is_semimodule}, we obtain several important notions.

\begin{definition}\label{def:semiring_ideal}\mimprovised
  Fix a semiring \( R \). A \term{left ideal} of \( R \) is a \hyperref[def:semimodule/submodel]{sub-semimodule} of \( R \) when regarded as a left semimodule over itself, and a \term{right ideal} is defined analogously.

  If \( I \) is both a left and right ideal of \( R \), we say that it is a \term{two-sided ideal} or simply \term{ideal}. More explicitly, \( I \) is a two-sided ideal of \( R \) if it is a \hyperref[def:monoid/submodel]{submonoid} of the additive monoid of \( R \) that is closed under left and right multiplication, i.e. \( RI = I = IR \).

  When \hyperref[def:ring/quotient]{quotient rings} are involved, we will have no choice but to work with two-sided ideals. If multiplication is commutative, every left ideal is a right ideal and there is no distinction between the two. Otherwise, we will usually consider left ideals by default. Right ideals are left ideals in the \hyperref[def:semiring/opposite]{opposite semiring}, and thus we lose no generality.

  \begin{thmenum}
    \thmitem{def:semiring_ideal/generated} For an arbitrary subset \( A \) of \( R \), we call the (left) \hyperref[def:semimodule/submodel]{linear span} of \( A \) the left ideal \term{generated} by \( A \). Explicitly, this is the set
    \begin{equation*}
      \sum_{a \in A} A a = \set*{ \sum_{k=1}^n t_k a_k \given n > 0 \T{and} t_k \in R \T{and} a_k \in A }.
    \end{equation*}

    If \( A = \set{ a_1, \ldots, a_n } \), we say that the ideal is \term{finitely generated} and write
    \begin{equation*}
      A a_1 + \cdots + A a_n.
    \end{equation*}

    For right ideals, this becomes
    \begin{equation*}
      a_1 A + \cdots + a_n A.
    \end{equation*}

    This is discussed in \fullref{thm:generators_via_polynomials} for the case of commutative rings, where we use the notation \( \braket{ A } \). In general rings, we are more explicit for the sake of avoiding possible confusion.

    \thmitem{def:semiring_ideal/principal} An ideal generated by a single element is called a \term{principal ideal}. In a general ring, there can be left, right and two-sided principal ideals.

    \begin{figure}[h]
      \caption{Some important ideals}\label{fig:ideal_hierarchy}
      \smallskip
      \hfill
      \begin{forest}
        for tree=
          {
            s sep=2.25cm
          }
        [
          {\hyperref[def:semiring_ideal]{ideal}}, name=ideal
            [{\hyperref[def:semiring_ideal/principal]{principal}}, name=principal]
            [
              {\hyperref[def:semiring_ideal/prime]{prime}}, name=prime
                [{\hyperref[def:semiring_ideal/maximal]{maximal}}, name=maximal]
            ]
            [{\hyperref[def:radical_ideal]{radical}}, name=radical]
        ]
        \draw[->, dashed] (prime) to node[below] {\hyperref[def:semiring/commutative]{commutative}} (radical);
        \draw[->, dashed] (prime) to[out=west, in=west] node[left] {\hyperref[def:principal_ideal_domain]{PID}} (maximal);
        \draw[->, dashed] (ideal) to[out=west, in=north] node[above] {\hyperref[def:principal_ideal_domain]{PID}} (principal);
      \end{forest}
      \hfill\hfill
    \end{figure}

    \thmitem{def:semiring_ideal/prime}\mcite[85]{Golan2010} If \( P \) is a proper ideal and if from \( IJ \subseteq P \) it follows that \( I \subseteq P \) or \( J \subseteq P \) (or both), we say that \( P \) is a \term{prime ideal}.

    When working with commutative semirings, \fullref{thm:def:semiring_ideal/properties/prime_pointwise} is instead sometimes taken as the definition of a prime ideal.

    \thmitem{def:semiring_ideal/coprime}\mcite[18]{КоцевСидеров2016} If \( I + J = R \) for proper ideals \( I \) and \( J \), we say that \( I \) and \( J \) are \term{coprime}. Equivalently, \( I \) and \( J \) are coprime if their sum contains a unit.

    \thmitem{def:semiring_ideal/maximal} A \term{maximal ideal} is a proper ideal that is maximal with respect to set inclusion. The maximal ideals are the predecessors of \( R \) in the lattice of ideals described in \fullref{thm:semiring_of_ideals}.
  \end{thmenum}
\end{definition}

\begin{remark}\label{rem:semiring_ideal_as_sub_semiring}
  A proper semiring ideal is a canonical example of a nonunital sub-semiring. As a consequence of \fullref{thm:def:semiring_ideal/properties/ideal_containing_unit}, a proper ideal cannot contain the multiplicative identity \( 1 \), and is thus not a sub-semiring unless we allow sub-semirings to not contain \( 1 \).
\end{remark}

\begin{proposition}\label{thm:def:semiring_ideal/properties}
  The left \hyperref[def:semiring_ideal]{ideals} of a semiring \( R \) have the following basic properties:
  \begin{thmenum}[series=thm:def:semiring_ideal/properties]
    \thmitem{thm:def:semiring_ideal/properties/ideal_containing_unit} An ideal contains a \hyperref[def:divisibility/zero]{unit} if and only if it is not proper. In particular, \( R = \braket{ 1_R } \).

    \thmitem{thm:def:semiring_ideal/properties/division} We have \( Rx \subseteq Ry \) if and only if \( y \) is a right divisor of \( x \). Note how \( Rx \) and \( Ry \) are \hi{left} principal ideals but \( y \) is a \hi{right} divisor.

    For commutative semirings, this reduces to \( \braket{ x } \subseteq \braket{ y } \) if and only if \( y \mid x \).

    \thmitem{thm:def:semiring_ideal/properties/union} The union of a \hyperref[eq:def:partially_ordered_set/homomorphism/sequence]{monotone sequence}
    \begin{equation*}
      I_1 \subseteq I_2 \subseteq \cdots
    \end{equation*}
    if ideals is again an ideal.

    \thmitem{thm:def:semiring_ideal/properties/maximal_is_prime} Every \hyperref[def:semiring_ideal/maximal]{maximal ideal} is \hyperref[def:semiring_ideal/prime]{prime}.

    \Fullref{thm:def:principal_ideal_domain/properties/maximal_ideal_is_prime} is a converse that holds for \hyperref[def:principal_ideal_domain]{principal ideal domains}.

    \thmitem{thm:def:semiring_ideal/properties/coprime_product} We have \( IJ \subseteq I \cap J \). The converse inclusion holds if \( R \) is \hyperref[def:semiring/commutative]{commutative} and if \( I \) and \( J \) are \hyperref[def:semiring_ideal/coprime]{coprime}.
  \end{thmenum}

  The following require \( R \) to be \hyperref[def:semiring/commutative]{commutative}:
  \begin{thmenum}[series=thm:def:semiring_ideal/properties]
    \thmitem{thm:def:semiring_ideal/properties/product_of_principal_ideals} In a commutative semiring, the product of the principal ideals \( \braket{x} \) and \( \braket{y} \) is \( \braket{xy} \).

    \thmitem{thm:def:semiring_ideal/properties/prime_pointwise} In a commutative semiring, an equivalent condition to \( P \) being \hyperref[def:semiring_ideal/prime]{prime} is that \( xy \in P \) implies \( x \in P \) or \( y \in P \) (or both).

    \thmitem{thm:def:semiring_ideal/properties/prime_is_radical} In a commutative semiring, every \hyperref[def:semiring_ideal/prime]{prime ideal} is \hyperref[def:radical_ideal]{radical}.
  \end{thmenum}
\end{proposition}
\begin{proof}
  \SubProofOf{thm:def:semiring_ideal/properties/ideal_containing_unit} We will prove that there exists a unit \( u \in I \) if and only if \( I = R \).

  \SufficiencySubProof* Let \( u \in I \) be a unit. Then \( 1_R = u^{-1} u \in I \). It follows that \( 1_R \cdot x = x \) for any \( x \in R \), thus \( IR = R \). But \( I \) is an ideal, hence we have that \( IR = I \). Therefore, \( I = IR = R \).

  \NecessitySubProof* If \( I = R \), then obviously \( 1_R \in I \).

  An analogous proof follows for the case when \( I \) is a right ideal.

  \SubProofOf{thm:def:semiring_ideal/properties/division}

  \SufficiencySubProof* Suppose that \( Rx \subseteq Ry \). Then \( x \in Ry \), and hence there exists an element \( l \) of \( R \) such that \( x = ly \). So \( y \) is a right divisor of \( x \).

  \NecessitySubProof* Suppose that \( y \) is a right divisor of \( x \). Then there exists an element \( l \) of \( R \) such that \( x = ly \). Thus, \( x \in Ry \), and hence \( Rx \subseteq Ry \).

  \SubProofOf{thm:def:semiring_ideal/properties/union} Follows from \fullref{thm:def:semimodule/properties/union}.

  \SubProofOf{thm:def:semiring_ideal/properties/maximal_is_prime} Let \( M \) be a maximal ideal and let \( IJ \subseteq M \). Since \( M \) is maximal, \( M \subseteq M + I \subseteq R \), and at least one of the inclusion is an equality. Suppose that \( M + I = R \), i.e. \( M \) and \( I \) are \hyperref[def:semiring_ideal/coprime]{coprime} ideals. Then, by the right distributivity in the \hyperref[thm:semiring_of_ideals/lattice]{semiring of ideals} of \( R \), we have
  \begin{equation*}
    J = RJ = (M + I)J = MJ + IJ \subseteq M + IJ = M.
  \end{equation*}

  Similarly, if \( M \) and \( J \) are coprime, then by left distributivity,
  \begin{equation*}
    I = IR = I(M + J) = IM + IJ \subseteq M + IJ = M.
  \end{equation*}

  \SubProofOf{thm:def:semiring_ideal/properties/coprime_product} We will first show that \( IJ \subseteq I \cap J \). If \( xy \in IJ \), then \( xy \in xJ = J \) and \( xy \in Iy = I \), hence \( xy \in I \cap J \).

  Now suppose that \( R \) is commutative and that \( I \) and \( J \) are coprime. Then there exist \( i \in I \) and \( j \in J \) such that \( i + j = 1_R \). For every \( x \in I \cap J \), we have \( ix \in IJ \) and \( xj \in IJ \). Therefore,
  \begin{equation*}
    ix + xj = (i + j)x = x.
  \end{equation*}

  \SubProofOf{thm:def:semiring_ideal/properties/product_of_principal_ideals}\hfill
  \SufficiencySubProof* Let \( z \in \braket{x} \braket{y} \). Then there exist elements \( x_z \) of \( \braket{x} \) and \( y_z \) of \( \braket{y} \) such that \( z = x_z y_z \), and elements \( r_x \) and \( r_y \) of \( R \) such that \( x r_x = x_z \) and \( y r_y = y_z \).

  Therefore,
  \begin{equation*}
    z = \underbrace{(x r_x) (y r_y)}_{(xy) (r_x r_y)} \in \braket{xy}.
  \end{equation*}

  \NecessitySubProof* Let \( z \in \braket{xy} \). Then there exists an element \( r \) of \( R \) such that \( z = rxy = (rx)(y) \), hence \( z \in \braket{x} \braket{y} \).

  \SubProofOf{thm:def:semiring_ideal/properties/prime_pointwise} Suppose that \( R \) is commutative.
  \SufficiencySubProof* Let \( P \) be prime and let \( xy \in P \). Then \( \braket{xy} \subseteq P \). By \fullref{thm:def:semiring_ideal/properties/product_of_principal_ideals}, \( \braket{x} \braket{y} \subseteq P \), and hence either \( \braket{x} \subseteq P \) or \( \braket{y} \subseteq P \). Therefore, either \( x \in P \) or \( y \in P \).

  \NecessitySubProof* Let \( P \) be an ideal such that \( xy \in P \) implies \( x \in P \) or \( y \in P \). Let \( IJ \subseteq P \) and suppose that there exist \( i \in I \setminus P \) and \( j \in J \setminus P \).

  Obviously \( ij \in I \). But since \( P \) is prime, it follows that either \( i \in P \) or \( j \in P \).

  The obtained contradiction shows that either \( I \) or \( J \) must be a subset of \( P \). Therefore, \( P \) is prime.

  \SubProofOf{thm:def:semiring_ideal/properties/prime_is_radical} Suppose that \( R \) is commutative, let \( P \) be a prime ideal and let \( x^n \in P \) for \( n > 0 \). We will show that \( x \in P \).

  We proceed via induction on \( n \). The case \( n = 1 \) is trivial. Suppose that \( x^{n-1} \in P \) implies \( x \in P \), and let \( x = x \cdot x^{n-1} \in P \). Since \( P \) is prime and \( R \) is commutative, by \fullref{thm:def:semiring_ideal/properties/prime_pointwise}, either \( x \in P \) or \( x^{n-1} \in P \). In the latter case, we use the inductive hypothesis to show that \( x \in P \).

  Generalizing on \( x \), we conclude that \( P \) is a radical ideal.
\end{proof}

\begin{proposition}\label{thm:semiring_of_ideals}
  \hfill
  \begin{thmenum}
    \thmitem{thm:semiring_of_ideals/semiring} The set \( \mscrI \) of all ideals of a semiring \( R \) is itself a semiring with the set addition and multiplication from \hyperref[def:semiring/power_set]{power set semiring} \( \pow(R) \).

    \thmitem{thm:semiring_of_ideals/order} Furthermore, \( \mscrI \) is an \hyperref[def:ordered_semiring]{ordered semiring} with respect to set inclusion.

    \thmitem{thm:semiring_of_ideals/lattice} The \hyperref[def:partially_ordered_set_extremal_points/supremum_and_infimum]{supremum} of \( I \) and \( J \) is their sum \( I + J \) and their \hyperref[def:partially_ordered_set_extremal_points/supremum_and_infimum]{infimum} is their intersection \( I \cap J \). With this, \( \mscrI \) becomes a lattice.
  \end{thmenum}
\end{proposition}
\begin{proof}
  \SubProofOf{thm:semiring_of_ideals/semiring} Associativity and commutativity in \( \mscrI \) are inherited from \( R \), as well as both left and right distributivity. Distributivity ensures that \( I + J \) is an ideal, while associativity of multiplication ensures that \( IJ \) is an ideal.

  \SubProofOf{thm:semiring_of_ideals/order} We must now prove that the partial order \( \subseteq \) is compatible with addition and multiplication. Suppose that \( I \subseteq J \) and let \( H \) be any ideal in \( \mscrI \). Then
  \begin{equation*}
    I + H \subseteq J + H
  \end{equation*}
  and
  \begin{equation*}
    IH \subseteq JH.
  \end{equation*}

  Therefore, \( \mscrI \) is an ordered semiring.

  \SubProofOf{thm:semiring_of_ideals/lattice} Since \( 0 \in I \), obviously \( I \subseteq I + J \), and thus \( I + J \) is an upper bound of \( I \) and \( J \). If \( H \) is any other upper bound, it must contain the sums of all elements of \( I \) and all elements of \( J \), hence \( I + J \subseteq H \). Therefore, \( \sup\set{I, J} = I + J \).

  For \( I \cap J \), it is an infimum of \( I \) and \( J \) as a consequence of \fullref{thm:boolean_algebra_of_subsets/meet}.
\end{proof}

\begin{definition}\label{def:gcd_and_lcm}\mimprovised
  Let \( R \) be an \hyperref[def:entire_semiring]{entire} \hyperref[def:semiring/commutative]{commutative semiring}. The ideal
  \begin{equation*}
    \sup\set{ \braket{x}, \braket{y} } = \braket{x} + \braket{y}
  \end{equation*}
  contains the common divisors of \( x \) and \( y \). If this ideal has some canonical representative \( z \), we call \( z \) the \term{greatest common divisor} of \( x \) and \( y \) and write \( z = \gcd(x, y) \). This canonical representative must be defined specifically for each semiring, but it is necessary to guarantee uniqueness.

  Dually, if
  \begin{equation*}
    \inf\set{ \braket{x}, \braket{y} } = \braket{x} \cap \braket{y}
  \end{equation*}
  has some canonical representative \( z \), we call \( z \) the \term{least common multiple} of \( x \) and \( y \) and write \( z = \lcm(x, y) \).
\end{definition}

\begin{proposition}\label{thm:natural_number_divisibility_order}
  The set \hyperref[def:set_of_natural_numbers]{\( \BbbN \)} of natural numbers is a \hyperref[def:semilattice/bounded]{bounded lattice} with respect to \hyperref[def:divisibility]{semiring divisibility}. Explicitly:
  \begin{thmenum}
    \thmitem{thm:natural_number_divisibility_order/join} The \hyperref[def:semilattice/join]{join} of \( n \) and \( m \) is their \hyperref[def:gcd_and_lcm]{least common multiple} \( \lcm(n, m) \). Well-ordering allows us to pick the smallest from all common multiple.

    \thmitem{thm:natural_number_divisibility_order/bottom} The \hyperref[def:partially_ordered_set_extremal_points/top_and_bottom]{bottom element} element is \( 1 \) since \( 1 \) divides every natural number.

    \thmitem{thm:natural_number_divisibility_order/meet} Dually, the \hyperref[def:semilattice/meet]{meet} of \( n \) and \( m \) is their \hyperref[def:gcd_and_lcm]{greatest common divisor} \( \gcd(n, m) \).

    \thmitem{thm:natural_number_divisibility_order/top} The \hyperref[def:partially_ordered_set_extremal_points/top_and_bottom]{top element} is \( 0 \) since every natural number divides \( 0 \).
  \end{thmenum}

  Furthermore, divisibility is compatible with the standard ordering in the sense that \( n \mid m \) implies \( n \leq m \).

  \begin{figure}
    \centering
    \includegraphics[page=1]{output/thm__natural_number_divisibility_order.pdf}
    \caption{A spatial \hyperref[def:hasse_diagram]{Hasse diagram} for a fragment of the \hyperref[thm:natural_number_divisibility_order]{natural number divisibility lattice}}
    \label{fig:thm:natural_number_divisibility_order/divisibility}
  \end{figure}

  \begin{figure}
    \centering
    \includegraphics[page=2]{output/thm__natural_number_divisibility_order.pdf}
    \caption{A comparison of the \hyperref[thm:natural_number_divisibility_order]{natural number divisibility lattice} and the fragment of the \hyperref[thm:semiring_of_ideals/lattice]{lattice of ideals}.}
    \label{fig:thm:natural_number_divisibility_order/ideals}
  \end{figure}
\end{proposition}
\begin{proof}
  By \fullref{thm:semiring_divisibility_order}, divisibility is a preorder.

  \SubProofOf[def:binary_relation/antisymmetric]{antisymmetry} If \( n \mid m \) and \( m \mid n \), there exist numbers \( a \) and \( b \) such that \( n = ay \) and \( m = bx \). Then \( n = abx \), and we can cancel \( n \) to obtain that \( ab = 1 \). But \( 1 \) is the only unit in \( \BbbN \), hence \( a = b = 1 \), and thus \( n = m \).

  \SubProofOf[def:semilattice/lattice]{lattice structure} By \fullref{alg:euclidean_algorithm}, every pair of integers has a positive greatest common divisor, and also a least common multiple.

  By \fullref{thm:def:semiring_ideal/properties/division}, the lattice of principal ideals in \( \BbbN \) must be dual to it. Indeed, by \fullref{thm:bezout_identity}, we have that
  \begin{equation*}
    \braket{n} + \braket{m} = \braket{ \gcd(n, m) },
  \end{equation*}
  and by \fullref{thm:def:semiring_ideal/properties/division}, \( \braket{n} \cap \braket{m} \) contains the common multiples of \( n \) and \( m \), hence
  \begin{equation*}
    \braket{n} \cap \braket{m} = \braket{ \lcm(n, m) }.
  \end{equation*}

  \SubProof{Proof that the order are compatible} If \( n \mid m \), then there exists a positive natural number \( a \) such that \( an = m \). We have
  \begin{equation*}
    an
    \reloset{\eqref{eq:def:peano_arithmetic/PA7}} =
    (a - 1)n + n
    =
    m.
  \end{equation*}

  Thus, by \eqref{eq:def:natural_number_ordering/predicate}, \( n \leq m \).
\end{proof}

\begin{example}\label{ex:def:semiring_ideal}
  We list several examples of \hyperref[def:semiring_ideal]{semiring ideals}.
  \begin{thmenum}
    \thmitem{ex:def:semiring_ideal/not_principal} The simplest example of an ideal that is not principal is the ideal \( \braket{ 2, 3 } \) in \( \BbbN \).

    To see that it is not principal, suppose that \( \braket{ n } = \braket{ 2, 3 } \) for some natural number \( n \). This implies that there exist nonzero numbers \( a \) and \( b \) such that \( n = 2a + 3b \). Hence, \( n > 2a > a \) and \( n > 3b > b \). But then neither \( 2 \) nor \( 3 \) belongs to \( \braket{ n } \), contradicting our assumption.

    \thmitem{ex:def:semiring_ideal/prime_not_maximal} The zero ideal \( \braket{ 0 } \) in \( \BbbN \) is \hyperref[def:semiring_ideal/prime]{prime} but not \hyperref[def:semiring_ideal/maximal]{maximal}.

    Indeed, since \( \BbbN \) is entire, \( \braket{ 0 } = \set{ 0 } \) and thus \( \braket{ 0 } \) is prime. But it is not maximal since it is contained in every other ideal.

    \thmitem{ex:def:semiring_ideal/natural_numbers_principal_ideals} For natural numbers, \( \braket{ n } = \braket{ m } \) implies \( n = m \).

    Indeed, by \fullref{thm:def:semiring_ideal/properties/division}, \( n \mid m \) and \( m \mid n \). Thus, there exist numbers \( a \) and \( b \) such that \( n = am \) and \( m = bn \), hence \( n = abn \). Since the semiring \( \BbbN \) is \hyperref[def:entire_semiring]{entire}, we can cancel \( n \) to obtain \( ab = 1 \). Then \( a = b = 1 \), and hence \( n = m \).

    \thmitem{ex:def:semiring_ideal/prime_numbers} A natural number \( n \) is \hyperref[def:prime_number]{prime} if and only if \( \braket{n} \) is a \hyperref[def:semiring_ideal/prime]{prime ideal} in \( \BbbN \).

    Suppose that \( n \) is prime and let \( n \mid mk \). From \fullref{thm:euclids_lemma} it follows that either \( n \mid k \) or \( n \mid m \), hence \fullref{thm:def:semiring_ideal/properties/prime_pointwise} is satisfied and \( \braket{ n } \) is a prime ideal.

    In the other direction, suppose that \( \braket{ n } \) is a prime ideal and let \( n = ab \). By \fullref{thm:def:semiring_ideal/properties/product_of_principal_ideals}, \( \braket{ n } = \braket{ a } \braket{ b } \). Since \( \braket{ n } \) is a prime ideal, either \( \braket{ a } \subseteq \braket{ n } \) or \( \braket{ b } \subseteq \braket{ n } \).

    Therefore, either \( \braket{ n } = \braket{ a } \) or \( \braket{ n } = \braket{ b } \). By \fullref{ex:def:semiring_ideal/natural_numbers_principal_ideals}, \( n = a \) or \( n = b \), which in turn implies that the other is a unit.

    Therefore, \( n \) is a prime number.
    \thmitem{ex:def:semiring_ideal/matrices} Consider the matrix algebra \( \BbbZ^{2 \times 2} \). The set
    \begin{equation*}
      \set[\Bigg]
      {
        \begin{pmatrix}
          0 & a \\
          0 & b
        \end{pmatrix}
        \given*
        a, b \in \BbbZ
      }.
    \end{equation*}
    is a left ideal. It is not a right ideal, however, because
    \begin{equation*}
      \begin{pmatrix}
        1 & 0 \\
        1 & 0
      \end{pmatrix}
      \begin{pmatrix}
        0 & 1 \\
        0 & 1
      \end{pmatrix}
      =
      \begin{pmatrix}
        0 & 1 \\
        0 & 1
      \end{pmatrix}.
    \end{equation*}

    \thmitem{ex:def:semiring_ideal/polynomial_ideals} Consider the bivariate \hyperref[def:polynomial_semiring]{polynomial semiring} \( \BbbN[X, Y] \) over natural numbers. Since \( (X + Y)^2 = X^2 + 2XY + Y^2 \), we have
    \begin{equation*}
      \braket{ X^2 + 2XY + Y^2 } \subseteq \braket{ X + Y }.
    \end{equation*}

    \thmitem{ex:def:semiring_ideal/ideal_polynomials} Ideals in polynomial (semi)rings are often studied, but we can also study polynomials in ideal semirings, i.e. polynomials over the semiring \( \mscrI \) of ideals of a semiring \( R \). For example,
    \begin{equation*}
      I^2 J + K
    \end{equation*}
    is a trivariate polynomial function over \( \mscrI \).

    \thmitem{ex:def:semiring_ideal/maximal_induced_coprime} If \( M \) is a maximal ideal and \( x \in R \setminus M \), then \( M \) and \( \braket{ x } \) are \hyperref[def:semiring_ideal/coprime]{coprime}.
  \end{thmenum}
\end{example}

\begin{definition}\label{def:multiplicative_set_in_semiring}\mcite[428]{Knapp2016BasicAlgebra}
  We call the subset of the semiring \( R \) a \term{multiplicative set} if it contains \( 1_R \) and, furthermore, it is closed under multiplication.
\end{definition}

\begin{proposition}\label{thm:complement_of_prime_ideal}
  The \hyperref[def:semiring_ideal]{ideal} \( P \) in the \hyperref[def:semiring/commutative]{commutative semiring} \( R \) is \hyperref[def:semiring_ideal/prime]{prime} if and only if \( R \setminus P \) is a \hyperref[def:multiplicative_set_in_semiring]{multiplicative set}.

  Not all multiplicative sets are obtained as complements of prime ideals --- see \fullref{ex:def:ring_localization/powers_of_two}.
\end{proposition}
\begin{proof}
  By \fullref{thm:def:semiring_ideal/properties/ideal_containing_unit}, \( P \) is a proper ideal if and only if \( 1_R \in R \setminus P \).

  By \fullref{thm:def:semiring_ideal/properties/prime_pointwise}, \( P \) is prime if and only if \( x, y \in R \setminus P \) implies \( xy \in R \setminus P \).
\end{proof}

\begin{theorem}[Maximal ideal theorem]\label{thm:maximal_ideal_theorem}\mcite[prop. 6.59]{Golan2010}
  Every proper \hyperref[def:semiring_ideal]{semiring ideal} is contained in a \hyperref[def:semiring_ideal/maximal]{maximal ideal}.

  Within \hyperref[def:zfc]{\logic{ZF}}, this theorem is equivalent to the \hyperref[def:zfc/choice]{axiom of choice} --- see \fullref{thm:axiom_of_choice_equivalences/maximal_ideal}.
\end{theorem}
\begin{proof}
  We will discuss equivalence with \fullref{thm:zorns_lemma}.

  \ImplicationSubProof[thm:zorns_lemma]{Zorn's lemma}[thm:maximal_ideal_theorem]{maximal ideal theorem} The proof is a special case of the proof (but not the statement) of \fullref{thm:maximal_ideal_theorem} for \( S = \set{ 1_R } \), which guarantees that the ideal would be proper.

  \ImplicationSubProof[thm:maximal_ideal_theorem]{maximal ideal theorem}[thm:zorns_lemma]{Zorn's lemma} In \cite{Hodges1979}, Hodges proves that the statement \enquote{every \hyperref[def:unique_factorization_domain]{unique factorization domain} has a maximal ideal} implies Zorn's lemma. We have an even stronger antecedent.
\end{proof}

\begin{definition}\label{def:radical_ideal}
  Let \( I \) be an \hyperref[def:semiring_ideal]{ideal} in the \hyperref[def:ring/commutative]{commutative ring} \( R \). The \term{radical} \( \sqrt I \) of \( I \) is a specific ideal containing \( I \) that we will define shortly. A \term{radical ideal} is an ideal that is equal to its radical. The \term{nilradical} of the ring \( R \) is \hyperref[def:radical_ideal]{radical} \( \sqrt {\braket{ 0_R }} \) of the zero ideal, whose elements we call \term{nilpotent elements}.

  The radical of the ideal \( I \) is the ideal defined equivalently through any of the following:
  \begin{thmenum}
    \thmitem{def:radical_ideal/direct}\mcite[15]{КоцевСидеров2016}
    \begin{equation}\label{eq:def:radical_ideal/direct}
      \sqrt I \coloneqq \set{ x \in R \given \qexists {n \in \BbbZ_{>0}} x^n \in I }.
    \end{equation}

    \thmitem{def:radical_ideal/intersection} \( \sqrt I \) is the intersection of all \hyperref[def:semiring_ideal/prime]{prime ideals} of \( R \) containing \( I \).
  \end{thmenum}
\end{definition}
\begin{proof}
  \SubProof{Proof of correctness of \eqref{eq:def:radical_ideal/direct}} We only need to prove that the radical \( \sqrt I \) of the ideal \( I \) is an ideal.

  Multiplicative closure is simpler. If \( x \) belongs to \( \sqrt I \), then there exists a power \( x^n \) that belongs to \( I \). Let \( r \) be any member of \( r \). Then \( rx = rx^n \in I \) since \( I \) is closed with respect to multiplication.

  Additive closure is a bit more involved. If \( x \) and \( y \) both belong to \( \sqrt I \), then there exist powers \( n \) and \( m \) such that \( x^n \in I \) and \( y^m \in I \). Let \( u \coloneqq n + m - 1 \). By \fullref{thm:binomial_theorem},
  \begin{equation*}
    (x + y)^u = \sum_{k=0}^u \binom u k x^k y^{u-k}.
  \end{equation*}

  \begin{itemize}
    \item If \( k < n \), then \( x^k y^{u-k} = (x^k y^{n - k - 1}) y^m \) and, since \( y^m \in I \), we have \( x^k y^{u-k} \in I \).
    \item If \( k \geq n \), then \( x^k y^{u-k} = x^n (x^{k-n} y^{u-k}) \) and, since \( x^n \in I \), we have \( x^k y^{u-k} \in I \).
  \end{itemize}

  Since \( I \) is closed under addition, \( (x + y)^u \in I \).

  \ImplicationSubProof{def:radical_ideal/direct}{def:radical_ideal/intersection} Let \( x \in \sqrt I \). That is, there exists a positive integer \( n \) such that \( x^n \in I \). Let \( P \) be a prime ideal containing \( I \). We will show that \( x \in P \).

  Since \( P \) is prime, by \fullref{thm:def:semiring_ideal/properties/prime_pointwise}, \( x^{n-1} \in P \) or \( x \in P \). If \( x^{n-1} \in P \), then \( x^{n-2} \in P \) or \( x \in P \). Proceeding by induction on \( k \) in \( x^{n-k} \), we eventually obtain that \( x \in P \).

  \ImplicationSubProof{def:radical_ideal/intersection}{def:radical_ideal/direct} Conversely, let \( x \) be a member of every prime ideal containing \( P \). We will show that \( x \in \sqrt I \).

  Suppose that \( x^n \not\in I \) for every \( n \) and consider the following family of ideals:
  \begin{equation*}
    \mscrH \coloneqq \set{ J \T{is an ideal of} R \T{containing} I \given \qforall {n \in \BbbZ_{>0}} x^n \not\in J }.
  \end{equation*}

  It is nonempty because \( I \in \mscrH \).

  For every chain of ideals in \( \mscrH \), their union is also an ideal in \( \mscrH \). By \fullref{thm:zorns_lemma}, \( \mscrH \) has a maximal element \( H \). We will show that \( H \) is prime.

  From \( AB \subseteq H \) it follows that \( AB \in \mscrH \). If we suppose that neither \( A \) nor \( B \) belongs to \( \mscrH \), we obtain that there exist positive integers \( n \) and \( m \) such that \( x^n \in A \) and \( x^m \in B \). But \( x^{n + m} \) must then belong to \( AB \), which contradicts \( AB \in \mscrH \). The obtained contradiction demonstrates that \( H \) is a prime ideal. But this is impossible since, by assumption \( x \) is contained in every prime ideal, and \( x \not\in H \). So this must contradict our previous assumption that \( x^n \neq 0 \) for every \( n \).

  Therefore, \( x \) belongs to \( \sqrt I \).
\end{proof}

\begin{example}\label{ex:def:radical_ideal}
  We list examples of \hyperref[def:radical_ideal]{radical ideal}.

  \begin{thmenum}
    \thmitem{ex:def:radical_ideal/natural_numbers} Suppose that the natural number \( m \) has a prime decomposition
    \begin{equation*}
      m = p_1^{k_1} \cdots p_n^{k_n}.
    \end{equation*}

    Then the radical of its principal ideal is
    \begin{equation*}
      \sqrt{ \braket{ p_1^{k_1} \cdots p_n^{k_n} } } = \braket{ p_1 \cdots p_n }.
    \end{equation*}

    Indeed, for any \( a p_1 \cdots p_n \) from the radical, with \( k \coloneqq \max\set{ k_1, \ldots, k_n } \) we have
    \begin{equation*}
      (a p_1 \cdots p_n)^k = (a p_1^{k-k_1} \cdots p_n^{k-k_n}) p_1^{k_1} \cdots p_n^{k_n}.
    \end{equation*}

    To obtain \( m \), we can take \( k = 1 \) and \( a = p_1^{k_1-1} \cdots p_n^{k_1-1} \):
    \begin{equation*}
      a p_1 \cdots p_n = p_1^{k_1} \cdots p_n^{k_n} = m.
    \end{equation*}

    Particular examples of this are
    \begin{itemize}
      \item The ideal \( { \braket{ 6 } } \) is radical. It is not prime since \( \braket{ 2 } \braket{ 3 } = \braket{ 6 } \), but neither are subsets.

      \item For any prime \( p \), \( \braket{ p } \) is radical

      \item For any prime power \( p^n \), \( \sqrt{\braket{ p^n }} = \braket{ p } \). For example, \( \sqrt{\braket{ 4 }} = \braket{ 2 } \).
    \end{itemize}

    \thmitem{ex:def:radical_ideal/matrices} Consider the matrix ring \( \BbbN^{2 \times 2} \). The matrix
    \begin{equation*}
      A \coloneqq
      \begin{pmatrix}
        0 & 1 \\
        0 & 0 \\
      \end{pmatrix}
    \end{equation*}
    is a \hyperref[thm:def:semiring_ideal/properties/nilradical]{nilpotent element} of \( \BbbN^{n \times n} \) because \( A^2 \) is the zero matrix. We refer to such matrices as \term{nilpotent matrices}.

    The transposed matrix \( A^T \) is also nilpotent. Hence, their linear combinations are also nilpotent.
  \end{thmenum}
\end{example}

\begin{definition}\label{def:noetherian_semiring}\mcite[prop. 6.16]{Golan2010}
  We say that a \hyperref[def:semiring]{semiring} is \term{left noetherian} (resp. right noetherian) if it is a left (resp. right) \hyperref[def:noetherian_semimodule]{noetherian semimodule} over itself.

  Explicitly, any of the following equivalent conditions characterize a left noetherian semiring:
  \begin{thmenum}
    \thmitem{def:noetherian_semiring/acc} Every ascending chain of left ideals stabilizes.
    \thmitem{def:noetherian_semiring/maximal} Every nonempty set of left ideals has a maximal element.
    \thmitem{def:noetherian_semiring/generated} Every left ideal is \hyperref[def:semiring_ideal/generated]{finitely generated}.
  \end{thmenum}
\end{definition}
