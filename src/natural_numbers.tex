\section{Numbers}\label{sec:numbers}
\subsection{Natural numbers}\label{subsec:natural_numbers}

\begin{definition}\label{def:peano_arithmetic}\mcite[1]{Peano1889}
  The Peano arithmetic is a \hyperref[def:first_order_language]{first-order logic language} for describing \hyperref[def:natural_numbers]{natural numbers} and their operations. Peano's original axioms referred to sets rather than first-order logic languages, however this more general framework is more suitable nowadays. It consists of:
  \begin{thmenum}
    \thmitem{def:peano_arithmetic/eq} A formal equality symbol \( \doteq \).
    \thmitem{def:peano_arithmetic/zero} A \term{zero} constant \( 0 \) (see \fullref{rem:peano_arithmetic_zero}).
    \thmitem{def:peano_arithmetic/succ} A unary functional symbol \( s \), called the \term{successor operation}.
  \end{thmenum}

  We impose the following base \hyperref[def:first_order_theory]{axioms}:
  \begin{thmenum}
    \thmitem[def:peano_arithmetic/PA1]{PA1} \( \xi \doteq \eta \leftrightarrow s(\xi) \doteq s(\eta) \)
    \thmitem[def:peano_arithmetic/PA2]{PA2} \( \neg \qexists \xi (s(\xi) \doteq 0) \)
    \thmitem[def:peano_arithmetic/PA3]{PA3} The following, called the \term{axiom schema of induction}, holds: for all formulas \( \varphi \) and all free variables \( \xi \) of \( \varphi \), we have:
    \begin{equation*}
      \parens[\Big]{ \varphi[\xi \mapsto 0] \wedge \qforall \eta (\varphi[\xi \mapsto \eta] \to \varphi[\xi \mapsto s(\eta)]) } \to \qforall \eta (\varphi[\xi \mapsto \eta]).
    \end{equation*}

    See \fullref{rem:induction} for a discussion of induction.
  \end{thmenum}

  A \hyperref[def:first_order_semantics/satisfiability]{model} of this theory that has no proper submodels, i.e. one that only consists of \( 0 \) and the \hyperref[rem:induction]{inductively defined} members \( s(0), s(s(0)), \ldots \), is said to be a \term{standard model of arithmetic}. Non-standard models are know to exist but are mostly of theoretical interest to logicians.

  For convenience, we use the familiar decimal notation
  \begin{align*}
    1  &\coloneqq s(0) \\
    2  &\coloneqq s(1) \\
       &\vdots \\
    9  &\coloneqq s(8) \\
    10 &\coloneqq s(9) \\
       &\vdots
  \end{align*}

  See \fullref{ex:natural_arithmetic_grammar/backus_naur_form} for a \hyperref[def:language]{language} that describes standard arithmetic in decimal notation.
\end{definition}

\begin{definition}\label{def:natural_numbers}
  We define the set of \term[bg=естествени числа,ru=натуральные числа]{natural numbers} \( \BbbN \) as any \hyperref[sec:set_theory]{set-theoretic} \hyperref[def:peano_arithmetic]{standard model of Peano arithmetic}. Unless noted otherwise, it is safe to assume that the model is the \hyperref[def:smallest_inductive_set]{smallest inductive set} \( \omega \), in which case the natural numbers are encoded as the \hyperref[ex:natural_numbers_models/omega]{von Nemann ordinals}
  \begin{align*}
    0 &\coloneqq \varnothing \\
    1 &\coloneqq \{ \varnothing \} \\
    2 &\coloneqq \{ \varnothing, \{ \varnothing \} \} \\
      &\vdots
  \end{align*}

  \Fullref{ex:natural_numbers_models} lists a few examples of standard models of arithmetic.
\end{definition}

\begin{remark}\label{rem:peano_arithmetic_zero}
  It is common to consider the first natural numbers to be zero (e.g. \cite[67]{Enderton1977Sets}). Peano himself, however, considered one to be the first - see \cite[1]{Peano1889}.

  Whether \( 0 \) is a natural number or not, i.e. whether \( \varnothing \) encodes \( 0 \) or \( 1 \), is a matter of convention. This does not matter much for \fullref{def:natural_numbers}, however it is of crucial importance when defining operations like \hyperref[def:natural_number_addition]{addition}.

  We choose \( \BbbN \) to start with zero, however we usually avoid referring to the set \( \BbbN \) of natural numbers directly and instead use the notations
  \begin{thmenum}
    \thmitem{rem:peano_arithmetic_zero/nonnegative} \( 0, 1, 2, \ldots \) or \( \BbbZ_{\geq 0} \) for the nonnegative \hyperref[def:integers]{integers}.
    \thmitem{rem:peano_arithmetic_zero/positive} \( 1, 2, \ldots \) or \( \BbbZ_{> 0} \) for the positive integers.
  \end{thmenum}

  The concepts \enquote{nonnegative} and \enquote{positive} come from the \hyperref[def:integers/order]{order of the integers}.
\end{remark}

\begin{example}\label{ex:natural_numbers_models}
  A \hyperref[sec:set_theory]{set-theoretic} \hyperref[def:peano_arithmetic]{standard model of Peano arithmetic} consists of a nonempty set \( N \) and relates to the axioms of Peano arithmetic as follows:
  \begin{refenum}
    \refitem{def:peano_arithmetic/eq} The equality is the usual \hyperref[def:set_zfc]{set-theoretic equality}.
    \refitem{def:peano_arithmetic/zero} There exists a constant \( 0 \in N \) acting as zero.
    \refitem{def:peano_arithmetic/succ} There exists a unary \enquote{successor} function \( s: N \to N \) such that \( s(x) \neq 0 \) for every \( x \in N \).
  \end{refenum}

  Any of the following (and many more) can be taken as a definition for the set \( \BbbN \) of \hyperref[def:natural_numbers]{natural numbers}:
  \begin{thmenum}
    \thmitem{ex:natural_numbers_models/omega} A good fit for the set of natural numbers is \hyperref[def:smallest_inductive_set]{the smallest inductive set} \( \omega \). In this case,
    \begin{refenum}
      \refitem{def:peano_arithmetic/zero} \( 0 \coloneqq \varnothing \).
      \refitem{def:peano_arithmetic/succ} \( s \) is the set-theoretic \hyperref[def:successor_operator]{successor function}.
    \end{refenum}

    \thmitem{ex:natural_numbers_models/language} Let \( N = \{ a \}^{*} \) be the \hyperref[def:language/kleene_star]{Kleene star} of the singleton alphabet \( \{ a \} \). In this case,
    \begin{refenum}
      \refitem{def:peano_arithmetic/zero} \( 0 \coloneqq \varepsilon \) is the empty word.
      \refitem{def:peano_arithmetic/succ} \( s(w) \coloneqq w \cdot a \) concatenates the letter at the end of a word.
    \end{refenum}

    \thmitem{ex:natural_numbers_models/vector_spaces} The \hyperref[def:category_of_first_order_models]{category of models} \( \cat{FinVect}_{\BbbK} \) of all \hyperref[def:vector_space_dimension]{finite-dimensional} \hyperref[def:vector_space]{vector spaces} over the \hyperref[def:field]{field} \( \BbbK \) can also be used as a model for the natural numbers. In this case,
    \begin{refenum}
      \refitem{def:peano_arithmetic/zero} \( 0 \coloneqq 0_{\BbbK} = \{ 0_{\BbbF} \} \) is the trivial (zero-dimensional) vector space.
      \refitem{def:peano_arithmetic/succ} \( s(V) \coloneqq V \oplus \BbbK \) is the \hyperref[def:left_module_direct_product]{direct sum} of the vector space \( V \) and the field \( \BbbK \) as a one-dimensional vector space.
    \end{refenum}
  \end{thmenum}
\end{example}

\begin{definition}\label{def:natural_number_addition}
  We inductively\IND define a binary \term{addition} operation \( +: \BbbN^2 \to \BbbN \) as
  \begin{equation}\label{eq:def:natural_number_addition}
    x + y \coloneqq \begin{cases}
      x,         &y = 0,     \\
      s(x + y'), &y = s(y').
    \end{cases}
  \end{equation}
\end{definition}

\begin{proposition}\label{thm:natural_numbers_form_monoid_under_addition}
  The \hyperref[def:natural_numbers]{natural numbers} \( \BbbN \) with \hyperref[def:natural_number_addition]{addition} form a \hyperref[def:magma/cancellative]{cancellative} \hyperref[def:magma/commutative]{commutative} \hyperref[def:unital_magma/associative]{monoid} with \( 0 \) as the identity.

  Furthermore, the sum of two natural numbers is nonzero if and only if both numbers are nonzero, that is,
  \begin{equation}\label{eq:thm:natural_numbers_form_monoid_under_addition/nonzero_sum}
    x + y \neq 0 \T{if and only if} x \neq 0 \T{and} y \neq 0.
  \end{equation}
\end{proposition}
\begin{proof}
  \SubProofOf[commutativity]{def:magma/commutative} Consider the sum \( x + y \). We use induction\IND on \( y \) to prove its commutativity.
  \begin{itemize}
    \item If \( y = 0 \), nested induction\IND by \( x \) yields:
    \begin{itemize}
      \item If \( x = y = 0 \), clearly \( x + y = 0 + 0 = y + x \).
      \item If \( x = s(x') \) and if the inductive hypothesis holds for \( x' \),
      \begin{balign*}
        x + y
        &=
        s(x') + 0
        \overset {\eqref{eq:def:natural_number_addition}} = \\ &=
        s(x')
        \overset {\eqref{eq:def:natural_number_addition}} = \\ &=
        s(x' + 0)
        = \\ &=
        s(x' + y)
        \overset {\IndHyp} = \\ &=
        s(y + x')
        \overset {\eqref{eq:def:natural_number_addition}} = \\ &=
        y + s(x')
        =
        y + x.
      \end{balign*}
    \end{itemize}

    \item If \( y = s(y') \) and if the inductive hypothesis holds for \( y' \), \ref{def:peano_arithmetic/PA1} yields that \( x + y = y + x \) if and only if \( x + y' = y' + x \). But the last equality is satisfied because of the inductive hypothesis, hence commutativity of \( x \) and \( y \) follows.
  \end{itemize}

  \SubProofOf[associativity]{def:magma/associative} Fix natural numbers \( x \), \( y \), \( z \). We will prove associativity by induction\IND on \( z \). If \( z = 0 \), we have
  \begin{equation*}
    (x + y) + 0
    \overset {\eqref{eq:def:natural_number_addition}} =
    (x + y)
  \end{equation*}
  and
  \begin{equation*}
    x + (y + 0)
    \overset {\eqref{eq:def:natural_number_addition}} =
    x + y.
  \end{equation*}

  If \( z \neq 0 \), the proof follows directly from \ref{def:peano_arithmetic/PA1} as in the proof of commutativity.

  \SubProofOf[identity]{def:magma_identity} We have \( x + 0 = x \) by definition and \( 0 + x = x \) by commutativity.

  \SubProofOf[cancellativity]{def:magma/cancellative} Let \( x + z = y + z \). We will prove that \( x = y \) by induction\IND. For \( z = 0 \), this is obvious. For \( z = s(z') \), we have \( x + s(z') = y + s(z') \), which by \eqref{eq:def:natural_number_addition} is equivalent to \( s(x + z') = s(y + z') \).

  By \ref{def:peano_arithmetic/PA1}, we have \( x + z' = y + z' \). The inductive hypothesis implies that \( x = y \).

  \SubProofOf[no zero divisors]{eq:thm:natural_numbers_form_monoid_under_addition/nonzero_sum} Let \( x \) and \( y = s(y') \) be nonzero numbers. By \ref{def:peano_arithmetic/PA2}, \( x + y = s(x + y') \) and, by \ref{def:peano_arithmetic/PA2}, \( x + y \neq 0 \).
\end{proof}

\begin{definition}\label{def:natural_number_multiplication}
  \hyperref[rem:additive_magma/multiplication]{Multiplication in commutative monoids} (i.e. monoid exponentiation) is defined in \fullref{def:unital_magma/exponentiation} for a natural number and a monoid member. It just to happens that, by \fullref{thm:natural_numbers_form_monoid_under_addition}, the natural numbers are themselves a monoid. Therefore we automatically have a second \term{multiplication} operation defined for any two natural numbers.

  We will use an explicit definition to avoid the possibility of circular proofs in \fullref{thm:magma_exponentiation_properties}. Explicitly, multiplication is defined as
  \begin{equation}\label{eq:def:natural_number_multiplication}
    x \cdot y \coloneqq \begin{cases}
      0,              &y = 0,     \\
      x,              &y = 1,     \\
      x + x \cdot y', &y = s(y').
    \end{cases}
  \end{equation}

  As with all \hyperref[def:semiring]{semirings}, multiplication has higher priority than addition. In the unambiguous language defined in \fullref{ex:natural_arithmetic_grammar/backus_naur_form}, this means that we can use the shorthand \( x + y \cdot z \) for \( ((x \cdot y) + z) \). We usually use juxtaposition for denoting multiplication.
\end{definition}

\begin{proposition}\label{thm:natural_numbers_form_dioid}
  The \hyperref[def:natural_numbers]{natural numbers} \( \BbbN \) with \hyperref[def:natural_number_multiplication]{multiplication} form a \hyperref[def:magma/commutative]{commutative} \hyperref[def:unital_magma/associative]{monoid} with \( 1 \) as the identity.

  When combined with addition, the natural numbers become a \hyperref[def:semiring]{dioid} \hyperref[def:semiring/no_zero_divisor]{without zero divisors}.
\end{proposition}
\begin{proof}
  \SubProofOf{def:semiring/distributivity} We use induction\IND on \( n \) to prove distributivity, i.e. \eqref{eq:def:semiring/distributivity}. If \( z = 0 \),
  \begin{equation*}
    (x + y)0
    \overset{\eqref{eq:def:natural_number_multiplication}} =
    0
    \overset{\eqref{eq:def:natural_number_addition}} =
    0 + 0
    \overset{\eqref{eq:def:natural_number_multiplication}} =
    x0 + y0.
  \end{equation*}

  If \( z = 1 \),
  \begin{equation*}
    (x + y) \cdot 1
    \overset{\eqref{eq:def:natural_number_multiplication}} =
    x + y
    \overset{\eqref{eq:def:natural_number_multiplication}} =
    x \cdot 1 + y \cdot 1.
  \end{equation*}

  If \( z = s(z') \) and if the inductive hypothesis holds for \( z' \),
  \begin{equation*}
    (x + y)z
    \overset{\eqref{eq:def:natural_number_multiplication}} =
    (x + y) + (x + y) z'
    \overset {\IndHyp} =
    x + y + xz' + yz'
    =
    (x + xz') + (y + yz')
    \overset{\eqref{eq:def:natural_number_multiplication}} =
    xz + yz.
  \end{equation*}

  \SubProofOf{def:magma/associative} With distributivity proven, associativity of multiplication follows by induction\IND. Indeed, \eqref{eq:def:magma/associative} is trivially satisfied for \( z = 0 \) and \( z = 1 \) and, when \( z = s(z') \) and \( (xy)z' = x(yz') \) for all \( x, y \in \BbbN \), we have
  \begin{equation*}
    (xy)z
    \overset{\eqref{eq:def:natural_number_multiplication}} =
    xy + (xy) z'
    \overset {\IndHyp} =
    xy + x(y z')
    \overset{\ref{def:semiring/distributivity}} =
    x(y + y z')
    \overset{\eqref{eq:def:natural_number_multiplication}} =
    x(yz).
  \end{equation*}

  \SubProofOf{def:magma/commutative} Commutativity follows analogously: the base cases are trivial and if \( y = s(y') \) and \( xy' = y'x \) for all \( x \in \BbbN \), then
  \begin{equation*}
    xy
    \overset{\eqref{eq:def:natural_number_multiplication}} =
    x + xy'
    \overset {\IndHyp} =
    x + y'x
    \overset{\eqref{def:semiring/distributivity}} =
    (1 + y') x
    \overset{\eqref{eq:def:natural_number_addition}} =
    yx
  \end{equation*}

  \SubProofOf{def:semiring/dioid} It is obvious that \( 1 \) is the multiplicative identity. With associativity and commutativity of multiplication proven, we see that \( (\BbbN, \cdot) \) is a monoid. Because of distributivity, \( (\BbbN, \cdot) \) is actually a \hyperref[def:semiring/dioid]{dioid}.

  \SubProofOf{def:semiring/no_zero_divisor}
  We will now show that \( \BbbN \) has no zero divisors. Let \( x \) and \( y \) be nonzero natural numbers. We want to show that \( xy \neq 0 \). We use induction\IND on \( y \). If \( y = 1 \), by definition \( xy = x \neq 0 \). Let \( y = s(y') \) be different from \( 1 \) and assume that the inductive hypothesis holds for \( y' \), that is, \( xy' \neq 0 \). Because addition is cancellative, if \( xy' \neq 0 \). By \eqref{eq:thm:natural_numbers_form_monoid_under_addition/nonzero_sum}, \( xy = x + xy' \neq 0 \).
\end{proof}

\begin{definition}\label{def:natural_number_ordering}
  We define an \hyperref[def:preordered_set]{order relation} \( \leq \) on \( \BbbN \) by
  \begin{equation*}
    x \leq y \T{if and only if} \exists a: x + a = y.
  \end{equation*}
\end{definition}

\begin{proposition}\label{thm:natural_numbers_are_well_ordered}
  The \hyperref[def:natural_numbers]{natural numbers} are \hyperref[def:well_ordered_set]{well-ordered} by \hyperref[def:natural_number_ordering]{\( \leq \)} with \( 0 \) being the global minimum.
\end{proposition}
\begin{proof}
  For von Neumann ordinals, this follows from the more general \fullref{thm:ordinals_are_well_ordered}.

  We will give a direct proof the \hyperref[thm:natural_numbers_form_dioid]{dioid structure} and the order defined in \fullref{def:natural_number_ordering}.

  Zero is obviously a global minimum because \( 0 + x = x \) and hence \( 0 \leq x \) for any \( x \in \BbbN \).

  \SubProofOf{def:binary_relation/reflexive} Reflexivity is trivial because for every \( x \in \BbbN \), \( x + 0 = x \) and hence \( x \leq x \).

  \SubProofOf{def:binary_relation/transitive} Transitivity follows from the associativity of addition. If \( x \leq y \leq z \), then there exist numbers \( a, b \) such that \( x + a = y \) and \( y + b = z \). But
  \begin{equation*}
    (x + a) + b = x + (a + b) = z,
  \end{equation*}
  therefore \( x \leq z \).

  \SubProofOf{def:binary_relation/antisymmetric} Antisymmetry also follows from the associativity of addition. Let \( x \leq y \) and \( y \leq x \). Then there exist numbers \( a, b \) such that \( x + a = y \) and \( y + b = x \). Therefore
  \begin{equation*}
    x = y + b = (x + a) + b = x + (a + b).
  \end{equation*}

  Since addition is cancellative, we obtain \( a + b = 0 \).

  \SubProofOf{def:binary_relation/total} Let \( x, y \in \BbbN \). We prove totality by induction on \( y \). If \( y = 0 \), then \( x = y \) and both \( x \leq y \) and \( y \neq x \) hold. If \( y = s(y') \), then by the inductive hypothesis, either \( x \leq y' \) or \( x \geq y' \).

  If \( x \leq y' \), there exists a number \( a \) such that \( x + a = y' \) and therefore \( x + s(a) = y \) and \( x \leq y \).

  If \( x \geq y' \), there exists a number \( b \) such that \( x = y' + b \). If \( b = 0 \), then \( x + 1 = y \) and \( x \leq y \). Otherwise, if \( b = s(b') \), we have \( x = y' + b = y + b' \) and \( x \geq y \).

  \SubProofOf{def:well_ordered_set} Let \( A \subseteq \BbbN \) be a nonempty set. Aiming at a contradiction, suppose\DNE that \( A \) has no \hyperref[def:preordered_set/maximum_and_minimum]{minimum}. In particular, \( 0 \not\in A \).

  We will show that \( A \) is empty by induction\IND on \( x \). Since \( 0 \not\in A \), the case \( x = 0 \) is obvious. Let \( x = s(x') \) and \( x' \not\in A \). Again, aiming at a contradiction, suppose that \( x \in A \). Then we can prove that \( x \) is the minimum of \( A \). We use induction\DNE on \( y \in A \). Clearly \( y \) is not zero. If \( y = s(y') \) and \( x \leq y' \), then \( x \leq y \) by transitivity. Therefore \( x \) is indeed the minimum of \( A \). But this contradicts our assumption that \( A \) has no minimum, therefore we conclude that \( x \not\in A \).

  We just showed that \( A \) does not contain any member of a standard model of the Peano arithmetic. Therefore \( A \) is empty, which contradicts our choice of \( A \) as an arbitrary nonempty set.

  The obtained contradiction proves that every nonempty set in \( \BbbN \) has a minimum with respect to \( \leq \).

  \SubProofOf{def:preordered_magma} We will show that the order is compatible with addition in \( \BbbN \). Let \( x, y, z \in \BbbN \) and let \( x \leq y \). Since \( x \leq y \), there exists a constant \( a \) such that \( x + a = y \). Then
  \begin{equation*}
    (x + a) + z = (x + z) + a = y + z.
  \end{equation*}

  Therefore
  \begin{equation*}
    x + z \leq y + z.
  \end{equation*}

  \SubProofOf{def:ordered_semiring} \eqref{eq:def:ordered_semiring/nonnegativity} follows trivially from \fullref{thm:natural_numbers_form_dioid}, which shows that \( \BbbN \) has \hyperref[def:semiring/no_zero_divisor]{no zero divisors}.
\end{proof}
