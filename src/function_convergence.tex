\subsection{Function convergence}\label{subsec:function_convergence}

\begin{definition}\label{def:local_convergence}
  Fix two topological spaces \( X \) and \( Y \). Let \( A \subseteq X \) be a nonempty set and let \( f: A \to Y \) be a function. We give two equivalent definitions for \( y_0 \in Y \) being a \term{limit point} of \( f \) at \( x_0 \in \cl(A) \). If \( y_0 \) is the unique limit point (e.g. in \hyperref[def:separation_axioms/T2]{Hausdorff spaces}), we write
  \begin{equation*}
    \lim_{x \to x_0} f(x) = y_0.
  \end{equation*}

  \begin{thmenum}
    \thmitem{def:local_convergence/neighborhoods}(Cauchy-style condition) For every neighborhood \( V \) of \( y_0 \) there exists a neighborhood \( U \) of \( x_0 \) such that \( f(U \cap A) \subseteq V \).

    \thmitem{def:local_convergence/nets}(Heine-style condition) For every \hyperref[def:topological_net]{net} \( \{ x_k \}_{k \in \mscrK} \subseteq A \), for which \( x_0 \) is a limit \hyperref[def:net_convergence/limit]{point}, the corresponding net \( \{ f(x_k) \}_{k \in \mscrK} \) has \( y_0 \) as a limit point.
  \end{thmenum}
\end{definition}
\begin{proof}
  \ImplicationSubProof{def:local_convergence/neighborhoods}{def:local_convergence/nets} Let \( \{ x_k \}_{k \in \mscrK} \subseteq U \) be a net  with limit point \( x_0 \). Consider the net \( \{ f(x_k) \}_{k \in \mscrK} \). Fix a neighborhood \( V \) of \( y_0 \). We need to show that \( \{ f(x_k) \}_{k \in \mscrK} \) is eventually in \( V \).

  By \fullref{def:local_convergence/neighborhoods}, there exists a neighborhood \( U \) of \( x_0 \) such that \( f(U) \subseteq V \). Since \( x_0 \) is a limit point of \( \{ x_k \}_{k \in \mscrK} \), there exists an index \( k_0 \) such that for all \( k \geq k_0 \), \( x_k \in U \) and therefore \( f(x_k) \in V \). Hence \( \{ f(x_k) \}_{k \in \mscrK} \) is eventually in \( V \).

  We conclude that \( y_0 \) is a limit point of the net \( \{ f(x_k) \}_{k \in \mscrK} \) and that the Heine-style condition is satisfied.

  \ImplicationSubProof{def:local_convergence/nets}{def:local_convergence/neighborhoods} Suppose that \fullref{def:local_convergence/nets} holds while \fullref{def:local_convergence/neighborhoods} does not. Let \( V \) be a neighborhood of \( y_0 \). Then there exists no neighborhood \( U \) of \( x_0 \) such that \( f(U) \subseteq V \).

  For any neighborhood \( U \) of \( x_0 \) and let \( y_U \in f(U) \setminus V \) and \( x_U \in f^{-1} (U) \) so that \( f(x_U) = y_U \). Consider the families
  \begin{balign*}
    \{ x_U \}_{U \in T(x_0)},
     &  &
    \{ f(x_U) \}_{U \in T(x_0)},
  \end{balign*}
  ordered by \hyperref[ex:reverse_inclusion_net]{reverse inclusion} of the neighborhoods \( \mscrT(x_0) \) of \( x_0 \).

  Note that \( x_0 \) is a limit point of \( \{ x_U \}_{U \in T(x_0)} \). By \fullref{def:local_convergence/nets}, \( y_0 \) is a limit point of \( \{ f(x_U) \}_{U \in T(x_0)} \). But this contradicts our choice of the nets because \( f(x_U) \not\in V \) for any \( U \in T(x) \).

  The obtained contradiction demonstrates that \fullref{def:local_convergence/nets} implies \fullref{def:local_convergence/neighborhoods}.
\end{proof}

\begin{proposition}\label{thm:cauchy_function_convergence_via_subbases}
  Fix two topological spaces \( X \) and \( Y \) and two points \( x_0 \in X \) and \( y_0 \in Y \). Let \( \mscrP(x_0) \) and \( \mscrP(y_0) \) be local \hyperref[def:topological_local_subbase]{subbases} for the corresponding points. Then the function \( f: X \to Y \) \hyperref[def:local_convergence]{converges} to \( y_0 \) at \( x_0 \) if and only if every \( V_P \in P(y_0) \) there exists \( U_P \in B(x_0) \) such that \( f(U_P) \subseteq V_P \).

  Compare this result to \fullref{thm:net_convergence_via_subbases}.
\end{proposition}
\begin{proof}
  \SufficiencySubProof Obvious consequence of \fullref{def:local_convergence/neighborhoods}.
  \NecessitySubProof Fix a neighborhood \( V \) of \( x \). We will show that \fullref{def:local_convergence/neighborhoods} holds.

  Let \( \{ V_k \}_{k=1}^n \subseteq P(y_0) \) be a family such that \( \bigcap_{k=1}^n V_k \subseteq V \) (such a family exists by definition of a local subbase). By the antecedent of the implication we are proving, for every \( k = 1, \ldots, n \) there exists an \( U_k \in P(x_0) \) such that \( f(U_k) \subseteq V_k \). Then \( U \coloneqq \bigcap_{k=1}^n U_k \) is a neighborhood of \( x_0 \) and, furthermore,
  \begin{equation*}
    f(U)
    =
    f\left(\bigcap_{k=1}^n U_k \right)
    \subseteq
    \bigcap_{k=1}^n f(U_k)
    \subseteq
    \bigcap_{k=1}^n V_k
    \subseteq
    V.
  \end{equation*}

  Therefore, \fullref{def:local_convergence/neighborhoods} holds.
\end{proof}
