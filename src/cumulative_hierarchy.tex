\subsection{The cumulative hierarchy}\label{subsec:cumulative_hierarchy}

We will now investigate how we can use set theory itself to build models of set theory. \Fullref{thm:cumulative_hierarchy_model_of_zfc} contains the important results.

\begin{definition}\label{def:cumulative_hierarchy}\mcite[def. 61.1]{OpenLogicFull}
  For every ordinal \( \alpha \) we recursively define
  \begin{equation*}
    V_\alpha \coloneqq \begin{cases}
      \varnothing,                                  &\alpha = 0 \\
      \pow(V_\beta),                                &\alpha = \op{succ}(\beta) \\
      \bigcup\set{ V_\beta \given \beta < \alpha }, &\alpha \T{is a limit ordinal.}
    \end{cases}
  \end{equation*}

  Formally, for every ordinal \( \alpha \) we define a \( \alpha \)-indexed transfinite sequence \( V_0, V_1, \ldots, V_\omega, \ldots \) and then use the sequence to define \( V_\alpha \). The definition does not depend on any particular ordinal \( \alpha \), however, and thus all ways to obtain \( V_\alpha \) are equivalent. This is done to circumvent the non-existence of unbounded transfinite recursion.

  Each \( V_\alpha \) is called a \term{stage} and the index \( \alpha \) of a stage is called its \term{rank}. The entire proper class of stages is called the \term{cumulative hierarchy}.

  If some set \( A \) is a subset of \( V_\alpha \) but not of \( V_\beta \) for any \( \beta < \alpha \), we say that \( \alpha \) is the \term{rank} of the set \( A \) and denote it by \( \rank(A) \). We will see in \fullref{thm:axiom_of_regularity} that every set has a rank.
\end{definition}

\begin{proposition}\label{thm:cumulative_hierarchy_properties}
  Without relying on the \hyperref[def:zfc/foundation]{axiom of foundation} and by assuming that ordinals are well-founded by definition, we can prove the following basic properties for \hyperref[def:cumulative_hierarchy]{Von Neumann's cumulative hierarchy}:
  \begin{thmenum}
    \thmitem{thm:cumulative_hierarchy_properties/transitive} Each stage \( V_\alpha \) is a transitive set.
    \thmitem{thm:cumulative_hierarchy_properties/membership} For any two ordinals \( \beta < \alpha \) we have \( V_\beta \in V_\alpha \).
    \thmitem{thm:cumulative_hierarchy_properties/rank_inequality} If \( A \in B \) and both sets have ranks, then \( \rank(A) \in \rank(B) \).
    \thmitem{thm:cumulative_hierarchy_properties/well_founded} Each stage \( V_\alpha \) is well-founded by set membership.
    \thmitem{thm:cumulative_hierarchy_properties/subsets} We have \( \beta < \alpha \) if and only if \( V_\beta \subsetneq V_\alpha \).
    \thmitem{thm:cumulative_hierarchy_properties/ordinals} For every ordinal \( \alpha \) we have \( \rank(\alpha) = \alpha \).
  \end{thmenum}
\end{proposition}
\begin{proof}
  \SubProofOf{thm:cumulative_hierarchy_properties/transitive} The statement is vacuous for \( \alpha = 0 \). Suppose that \( \alpha > 0 \), let \( A \in V_\alpha \) and \( B \in A \). We will show that \( B \in V_\alpha \).
  \begin{itemize}
    \item Suppose that \( \alpha = \op{succ}(\beta) \) and that \( V_\beta \) is a transitive set. Then \( A \in V_\alpha \) implies that \( A \subseteq V_\beta \). Thus \( B \in V_\beta \) and, since \( V_\beta \) is a transitive set, \( B \subseteq V_\beta \).

    Therefore \( B \in V_\alpha = \pow(V_\beta) \).

    \item Suppose that \( \alpha \) is a limit ordinal and that \( V_\beta \) are transitive sets for every \( \beta < \alpha \). Then \( A \in V_\alpha \) implies that \( A \) belongs to \( V_{\beta_0} \) for some \( \beta_0 < \alpha \). The inductive hypothesis implies that \( A \subseteq V_{\beta_0} \). Therefore \( A \subseteq V_\alpha \).
  \end{itemize}

  \SubProofOf{thm:cumulative_hierarchy_properties/membership} Let \( \beta < \alpha \) be some ordinals. We will show that \( V_\beta \in V_\alpha \) using induction on \( \alpha \).
  \begin{itemize}
    \item Suppose that \( \alpha \) is a successor ordinal, i.e. \( \alpha = \op{succ}(\mu) \) for some \( \mu \), and suppose that for every \( \beta < \mu \) we have \( V_\beta \in V_\mu \). Clearly \( V_\mu \in V_\alpha \) because \( V_\mu \) is a subset of itself.

    If \( \beta < \mu \), then \( V_\beta \in V_\mu \) by the inductive hypothesis and, since \( V_\alpha \) is a transitive set by \fullref{thm:cumulative_hierarchy_properties/transitive}, \( V_\beta \in V_\alpha \).

    \item Suppose that \( \alpha \) is a limit ordinal. For some fixed \( \beta_0 \in \alpha \) we have \( V_{\beta_0} \in V_{\op{succ}(\beta_0)} \) by what we have already proved. We also have
    \begin{equation*}
      V_\alpha
      =
      \bigcup_{\beta < \alpha} V_\beta,
    \end{equation*}
    hence \( V_{\beta_0} \in V_\alpha \).
  \end{itemize}

  \SubProofOf{thm:cumulative_hierarchy_properties/rank_inequality} Let \( A \in B \) be arbitrary sets for which ranks are defined. Trichotomy holds for ordinals so we have to show that \( \rank(A) \geq \rank(B) \) leads to a contradiction. Denote the ranks by \( \alpha \) and \( \beta \) for brevity.

  We have \( A \subseteq V_\alpha \) by definition and \( A \subseteq V_\beta \) since \( V_\alpha \in V_\beta \) and \( V_\beta \) is transitive. If we suppose that \( \rank(B) < \rank(A) \), this would mean that \( A \) belongs to a stage below \( V_\alpha \), which contradicts the minimality of \( \alpha = \rank(A) \).

  Now suppose that \( \beta = \rank(B) = \rank(A) = \alpha \).
  \begin{itemize}
    \item If \( \beta = 0 \), then \( A \in B \) is impossible.

    \item If \( \beta \) is a successor ordinal of \( \alpha \), then \( V_\beta = \pow(V_\alpha) \). Since \( B \subseteq V_\beta \) is a set of subsets of \( V_\alpha \), \( A \in B \) is a subset of \( V_\alpha \). Thus we have \( \rank(A) \leq \alpha < \rank(A) \), which contradicts the well-foundedness of the ordinal ordering.

    \item If \( \beta \) is a limit ordinal, then \( V_\beta = \bigcup\set{ V_\alpha \given \alpha < \beta } \). As a member of \( B \), the set \( A \) belongs to some lower stage \( V_{\alpha_0} \), which again leads to \( \rank(A) < \rank(A) \).
  \end{itemize}

  Thus it remains for \( \rank(A) < \rank(B) \).

  \SubProofOf{thm:cumulative_hierarchy_properties/well_founded} We will use induction on \( \alpha \). Suppose that \( V_\beta \) is well-founded for every \( \beta < \alpha \). Aiming at a contradiction, suppose that there exists an infinitely descending sequence \( \seq{ x_k }_{k=1}^\infty \subseteq V_\alpha \). Then the sequence \( \seq{ \rank(x_k) }_{k=0}^\infty \) of ranks is an infinitely descending set of ordinals. The \hyperref[def:transitive_closure_of_a_set]{transitive closure} of the underlying set is then an ordinal by \fullref{thm:transitive_set_of_transitive_sets_is_ordinal}. But this contradicts the well-foundedness of ordinals.

  Therefore \( V_\alpha \) must be well-founded.

  \SubProofOf{thm:cumulative_hierarchy_properties/subsets} If \( \beta < \alpha \), then from \fullref{thm:cumulative_hierarchy_properties/members} and \fullref{thm:cumulative_hierarchy_properties/transitivity} it follows that \( V_\beta \subseteq V_\alpha \). We will show that \( V_\beta \neq V_\alpha \). Fix some \( \mu < \alpha \) and suppose that \( V_\beta \subsetneq V_\mu \) holds for all \( \beta < \mu \). If \( V_\beta = V_\alpha \), this would mean that \( V_\beta \subsetneq V_\mu \subseteq V_\alpha = V_\beta \), which is a contradiction. Therefore \( V_\beta \subsetneq V_\alpha \).

  Conversely, suppose that \( V_\beta \subsetneq V_\alpha \). Since trichotomy holds for ordinals and since \( V_\beta \neq V_\alpha \), it is sufficient to show that \( \beta > \alpha \) leads to a contradiction. If \( \beta > \alpha \), from \fullref{thm:cumulative_hierarchy_properties/membership} it follows that \( V_\alpha \in V_\beta \), which implies that \( V_\alpha \subsetneq V_\alpha \). The obtained contradiction shows that \( \beta < \alpha \).

  \SubProofOf{thm:cumulative_hierarchy_properties/ordinals} We will use transfinite induction to show that \( \rank(\alpha) = \alpha \) for every ordinal.
  \begin{itemize}
    \item The case \( \alpha = 0 \) is trivial because \( \alpha = \varnothing \subseteq \varnothing = V_0 \).

    \item Suppose that \( \alpha = \op{succ}(\beta) \) is a successor ordinal and \( \rank(\beta) = \beta \). Clearly \( \beta \in \pow(V_\beta) = V_\alpha \) and \( \set{ \beta } \in V_\alpha \). Since \( V_\alpha \) is a transitive set, we also have \( \beta \subseteq V_\alpha \). Thus
    \begin{equation*}
      \alpha = \op{succ}(\beta) = \beta \cup \set{ \beta } \subseteq V_\alpha.
    \end{equation*}

    \item Suppose that \( \alpha \) is a limit ordinal and that for \( \rank(\beta) = \beta \) for all \( \beta < \alpha \). Since \( \beta \in V_{\op{succ}(\beta)} \) for any \( \beta < \alpha \), we have
    \begin{equation*}
      \alpha
      =
      \set{ \beta \T{is an ordinal} \given \beta < \alpha }
      \subseteq
      \bigcup\set{ V_{\op{succ}(\beta)} \given \beta < \alpha }
      =
      V_\alpha.
    \end{equation*}
  \end{itemize}
\end{proof}

\begin{theorem}[Axiom of regularity]\label{thm:axiom_of_regularity}\mcite[thm. 64.11]{OpenLogicFull}
  Every set belongs to a stage in \hyperref[def:cumulative_hierarchy]{von Neumann's cumulative hierarchy}.

  This statement is called the \term{axiom of regularity} and in the presence of the other axioms of \logic{ZF}, it is equivalent to the \hyperref[def:zfc/foundation]{axiom of foundation}. It is much more difficult to state in the language of set theory, however.
\end{theorem}
\begin{proof}
  \ImplicationSubProof[def:zfc/foundation]{axiom of foundation}[thm:axiom_of_regularity]{axiom of regularity} Let \( A \) be a set. By \fullref{thm:transitive_closure_of_a_set}, its \hyperref[def:transitive_closure_of_a_set]{transitive closure} \( \cl^T(A) \) is a transitive set. Define
  \begin{equation*}
    D \coloneqq \set{ B \in \cl^T(A) \given B \T{does not belong to the cumulative hierarchy} }.
  \end{equation*}

  We will show that \( D \) is empty. Assume the contrary. Then by the \hyperref[def:zfc/foundation]{axiom of foundation} there exists \( B_0 \in D \) such that \( B_0 \cap D = \varnothing \). Since \( \cl^T(A) \) is a transitive set, \( B_0 \subseteq \cl^T(A) \). Thus \( B_0 \) consists of members \( x \) of \( \cl^T(A) \) that themselves have ranks, i.e. a minimal ordinal \( \beta \) such that \( x \subseteq V_\beta \). It follows from the \hyperref[def:zfc/replacement]{axiom schema of replacement} that these ordinals form a set. Denote this set by \( C \).

  If \( x \in B_0 \), then \( x \subseteq V_\beta \) for some \( \beta \in C \) and \( x \in V_{\op{succ}(\beta)} \). Thus
  \begin{equation*}
    B_0 \subseteq \bigcup\set{ V_{\op{succ}(\beta)} \given \beta \in C }.
  \end{equation*}

  Denote the union on the right by \( \alpha \). From \fullref{thm:union_of_set_of_ordinals} it follows that \( \alpha \) is an ordinal strictly larger than the ordinals in \( C \). By \fullref{thm:cumulative_hierarchy_properties/membership} we have that \( V_{\op{succ}(\beta)} \in V_\alpha \) for every \( \beta \in C \). Thus
  \begin{equation*}
    B_0 \subseteq \bigcup\set{ V_{\op{succ}(\beta)} \given \beta \in C } \subseteq V_\alpha.
  \end{equation*}

  This contradicts our assumption that \( B_0 \) does not belongs to the cumulative hierarchy. Therefore \( D = \varnothing \) and every member of \( \cl^T(A) \) also belongs to the cumulative hierarchy. In particular, every member of \( A \) belongs to the cumulative hierarchy.

  Define
  \begin{equation*}
    \mu \coloneqq \bigcup\set{ \op{succ}(\rank(B)) \given B \in A }.
  \end{equation*}

  Since \( B \) is a member of the stage with rank \( \op{succ}(\rank(B)) \) for every \( B \in A \), with the same reasoning as above it follows that \( A \subseteq V_\mu \).

  \ImplicationSubProof[thm:axiom_of_regularity]{axiom of regularity}[def:zfc/foundation]{axiom of foundation} Let \( A \) be any nonempty set. We will show that there exists a subset of \( A \) that is disjoint from \( A \).

  The \hyperref[def:axiom_of_regularity]{axiom of regularity} ensures that \( A \) belongs to the von Neumann cumulative hierarchy. Let \( B \in A \) be a set with minimal rank.

  Suppose that \( B \cap A \) is not empty. Then there exists some set \( C \in A \setminus B \). From \fullref{thm:cumulative_hierarchy_properties/rank_inequality} it follows that \( \rank(C) < \rank(B) < \rank(A) \). But \( C \) belongs to \( A \) and has a rank strictly smaller than \( \rank(B) \), which contradicts the minimality of \( \rank(B) \).

  The obtained contradiction shows that \( B \cap A = \varnothing \).
\end{proof}

\begin{remark}\label{thm:cumulative_hierarchy_model_of_zfc}\mcite{MathSE:cumulative_hierarchy_model_of_zfc}
  It turns out that specific stages of the \hyperref[def:cumulative_hierarchy]{von Neumann's cumulative hierarchy} may act as models of \( \hyperref[def:zfc]{\logic{ZFC}} \). We will now investigate what requirements must we impose on the stage's rank in order to satisfy the necessary axioms.

  Let \( V_\alpha \) be a stage of the cumulative hierarchy. We will investigate what restrictions must we impose on \( \alpha \) in order for the \hyperref[def:first_order_structure]{first-order structure} \( (V_\alpha, \in) \) to be a model of \logic{ZFC}. To summarize, \( \alpha \) needs to be a strongly inaccessible cardinal to be a model of \logic{ZFC} but most axioms require much less than that.

  It is clear that this will be a \hyperref[rem:standard_model_of_set_theory]{standard model}. Furthermore, due to \fullref{thm:cumulative_hierarchy_properties/transitive}, it will also be a transitive model.

  The following axioms are automatically satisfied for any stage \( V_\alpha \):
  \begin{thmenum}[series=thm:cumulative_hierarchy_model_of_zfc]
    \thmitem{thm:cumulative_hierarchy_model_of_zfc/extensionality} The validity of the \hyperref[def:zfc/extensionality]{axiom of extensionality} is inherited from the metatheory.

    \thmitem{thm:cumulative_hierarchy_model_of_zfc/specification} The \hyperref[def:zfc/specification]{axiom schema of specification} is satisfied because because each axiom in the schema \hyperref[def:first_order_definability]{defines} a subset of \( V_\alpha \) and because \( V_\alpha \) is a transitive set. This does not necessarily require the axiom schema of specification in the metatheory --- we only need the subsets of \( V_\alpha \) defined in \fullref{def:first_order_definability}.

    \thmitem{thm:cumulative_hierarchy_model_of_zfc/union} The \hyperref[def:zfc/union]{axiom of unions} is satisfied because if \( A \in V_\alpha \) and \( C \in \bigcup A \), then there exists some \( B \in A \) such that \( C \in B \in A \). Since \( V_\alpha \) is a transitive set, it follows that \( C \in V_\alpha \).

    \thmitem{thm:cumulative_hierarchy_model_of_zfc/foundation} The \hyperref[def:zfc/foundation]{axiom of foundation} is satisfied for any \( V_\alpha \) due to \fullref{thm:cumulative_hierarchy_properties/well_founded}. Its validity is also inherited from the metatheory but we do not actually need the \hyperref[def:zfc/foundation]{axiom of foundation} in the metatheory.

    \thmitem{thm:cumulative_hierarchy_model_of_zfc/choice} The validity of the \hyperref[def:zfc/choice]{axiom of choice} is inherited from the metatheory.

    Indeed, for any family of sets \( \mscrA \in V_\alpha \) there exists a \hyperref[def:choice_function]{choice function} \( c: \mscrA \to \bigcup \mscrA \). The set \( \set{ c(A) \given A \in \mscrA } \) then has a lower rank by \fullref{thm:cumulative_hierarchy_properties/rank_inequality} and thus by \fullref{thm:cumulative_hierarchy_properties/subsets} it belongs to \( V_\alpha \).
  \end{thmenum}

  The rest of the axioms are satisfied whenever additional restrictions are imposed on \( \alpha \):
  \begin{thmenum}[series=thm:cumulative_hierarchy_model_of_zfc]
    \thmitem{thm:cumulative_hierarchy_model_of_zfc/power} The \hyperref[def:zfc/power_set]{axiom of power sets} is satisfied by \( V_\lambda \) for a limit ordinal \( \lambda \) if the axiom of power sets holds in the theory.

    Indeed, if \( A \in V_\lambda \) and \( A \) has rank \( \beta \), then necessarily \( \beta < \lambda \). Since \( A \subseteq V_\beta \), it follows that \( \pow(A) \subseteq \pow(V_\beta) = V_{\op{succ}(\beta)} \). But \( \op{succ}(\beta) < \lambda \) since \( \lambda \) is a limit ordinal. Therefore \( \rank(\pow(A)) = \op{succ}(\beta) < \lambda \). From \fullref{thm:cumulative_hierarchy_properties/subsets} it follows that \( V_{\op{succ}(\beta)} \subsetneq V_\lambda \) and thus \( \pow(A) \subsetneq V_\lambda \).

    \thmitem{thm:cumulative_hierarchy_model_of_zfc/pairing} The \hyperref[def:zfc/pairing]{axiom of pairing} is also satisfied by \( V_\lambda \) for a limit ordinal \( \lambda \) if the axiom of pairing holds in the theory.

    Let \( A \) and \( B \) be members of \( V_\lambda \). Let \( \beta \) be the larger of their ranks. Then \( A \) and \( B \) are also members of \( \pow(V_\beta) = V_{\op{succ}(\beta)} \) and the set \( \set{ A, B } \) is a member of \( V_{\op{succ}(\op{succ}(\beta))} \).

    Since \( \lambda \) is a limit ordinal, we have \( \op{succ}(\op{succ}(\beta)) < \lambda \) and hence \( \set{ A, B } \in V_\lambda \) by \fullref{thm:cumulative_hierarchy_properties/subsets}.

    \thmitem{thm:cumulative_hierarchy_model_of_zfc/infinity} The \hyperref[def:zfc/infinity]{axiom of infinity} is satisfied by any ordinal \( \alpha > \omega \).

    Indeed, by \fullref{thm:cumulative_hierarchy_properties/ordinals} we have \( \omega \subseteq V_\omega \) and by \fullref{thm:cumulative_hierarchy_properties/subsets} we have \( \omega \in V_\alpha \) for \( \alpha \geq \op{succ}(\omega) \).

    \thmitem{thm:cumulative_hierarchy_model_of_zfc/replacement} The \hyperref[def:zfc/replacement]{axiom schema of replacement} requires function \( f: A \to V_\alpha \) from the metatheory whose domain \( A \) is a member of \( V_\alpha \) to also be a function in the object theory. This is satisfied if \( \alpha \) is a regular cardinal due to \fullref{def:regular_cardinal/functions}.
  \end{thmenum}
\end{remark}
