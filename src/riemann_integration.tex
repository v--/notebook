\subsection{Riemann integration}\label{subsec:riemann_integration}

\begin{definition}\label{def:riemann_partition}\mcite[def. 1]{Gordon1991}
  The concept of a partition of a nonempty \hyperref[def:real_numbers]{real} \hyperref[def:total_order_interval/closed]{closed interval} \( [a, b] \) is the base for defining Riemann-style integrals.

  \begin{thmenum}
    \thmitem{def:riemann_partition/partition} A \term{Riemann partition} of \( [a, b] \) is a set
    \begin{equation*}
      \Delta \coloneqq \{ x_0, \ldots, x_n \} \subseteq [a, b]
    \end{equation*}
    that satisfies
    \begin{equation*}
      a = x_0 < x_1 < \ldots < x_n = b.
    \end{equation*}

    For brevity, we write
    \begin{equation}\label{eq:def:riemann_partition/partition}
      \Delta: a = x_0 < x_1 < \ldots < x_n = b.
    \end{equation}

    We denote the set of all partitions of \( [a, b] \) by \( \op{part}([a, b]) \).

    \thmitem{def:riemann_partition/refinement} The partition
    \begin{equation*}
      \Gamma: a = y_0 < y_1 < \ldots < y_m = b
    \end{equation*}
    is called a \term{refinement} of the partition \eqref{eq:def:riemann_partition/partition} if we have the \hyperref[def:subset]{set inclusion}
    \begin{equation}\label{eq:def:riemann_partition/refinement/inclusion}
      \{ x_0, x_1, \ldots, x_n \} \subseteq \{ y_0, y_1, \ldots, y_m \}.
    \end{equation}

    In this case, we \enquote{split} \( \Gamma \) into chains such that, for each \( k = 1, 2, \ldots, n \),
    \begin{equation}\label{def:riemann_partition/refinement/splitting}
      y_{k,j} \coloneqq \begin{cases}
        x_{k-1},                                                                          &j = 0, \\
        x_k,                                                                              &j = p_k, \\
        j\text{-th point of } \{ y_0, \ldots, y_m \} \cap [x_{k-1}, x_k], &0 < j < p_k.
      \end{cases}
    \end{equation}

    \thmitem{def:riemann_partition/diameter} Finally, the \term{diameter} of the partition \eqref{eq:def:riemann_partition/partition} is defined as
    \begin{equation}\label{eq:def:riemann_partition/diameter}
      \diam(\Delta) \coloneqq \max_{1 \leq k \leq n} (x_k - x_{k-1}).
    \end{equation}

    \thmitem{def:riemann_partition/order} We can make the set \( \op{part}([a, b]) \) of all \hyperref[def:riemann_partition/partition]{Riemann partitions} of \( [a, b] \) into a \hyperref[def:directed_set]{directed set} using two common approaches:
    \begin{thmenum}
      \thmitem{def:riemann_partition/order/refinement} Put \( \Delta \preceq_R \Gamma \) if and only if \( \Gamma \) is a \hyperref[def:riemann_partition/refinement]{refinement} of \( \Delta \). This actually makes \( (\op{part}([a, b]), \preceq_R) \) a \hyperref[def:poset]{poset}.
      \thmitem{def:riemann_partition/order/diameter} Put \( \Delta \preceq_D \Gamma \) if and only if \( \diam(\Gamma) \leq \diam(\Delta) \).
    \end{thmenum}

    \thmitem{def:riemann_partition/tagged} A \term{tagged partition} of \( [a, b] \) is a partition \eqref{eq:def:riemann_partition/partition} of \( [a, b] \) along with a choice of a \term{tag} \( \xi_k \) for each closed interval \( [x_{k-1}, x_k], k = 1, \ldots, n \). By putting \( \Xi \coloneqq \{ \xi_k \}_{k=1}^n \), we can define a tagged partition as the tuple \( (\Delta, \Xi) \). For brevity, we write
    \begin{alignedeq}\label{eq:def:riemann_partition/tagged}
      &\Delta: a = x_0 < x_1 < \ldots < x_n = b \\
      &\Xi: \xi_k \in [x_{k-1}, x_k], k = 1, \ldots, n.
    \end{alignedeq}

    We denote the set of all tagged partitions of \( [a, b] \) by \( \op{tpart}([a, b]) \). We introduce an order on \( \op{tpart}([a, b]) \) by putting
    \begin{equation*}
      (\Delta, \Xi) \preceq_R (\Gamma, \Eta) \T{if and only if} \Delta \preceq_R \Eta
    \end{equation*}
    and analogously for \( \preceq_D \). Note that \( \preceq_R \) is not a partial order in \( \op{tpart}([a, b]) \) unlike in \( \op{part}([a, b]) \).
  \end{thmenum}
\end{definition}

\begin{remark}\label{rem:set_and_riemann_partitions}
  Note that \eqref{eq:def:riemann_partition/partition} is not a partition in the sense of \fullref{def:set_partition}, however the set of intervals
  \begin{equation*}
    \Big\{ [x_0, x_1), [x_1, x_2), \ldots, [x_{n-2}, x_{n-1}), [x_{n-1}, x_n] \Big\}
  \end{equation*}
  is a set-theoretic partition of \( [a, b] \). Conversely, every finite set-theoretic partition of \( [a, b] \) gives rise to a Riemann partition in the sense of \fullref{def:riemann_partition/partition}.
\end{remark}

\begin{definition}\label{def:riemann_integral}\mcite[def. 2]{Gordon1991}
  Let \( \mscrX \) be a real \hyperref[def:separation_axioms/T2]{Hausdorff} \hyperref[def:topological_vector_space]{topological vector space}. Fix a \hyperref[def:function/single_valued]{function} \( f: [a, b] \to \mscrX \).

  The \term{Riemann sum} of \( f \) corresponding to the \hyperref[def:riemann_partition/tagged]{tagged partition} \eqref{eq:def:riemann_partition/tagged} is defined as
  \begin{equation*}
    S(f, \Delta, \Xi) \coloneqq \sum_{k=1}^n f(\xi_k) (x_k - x_{k-1}).
  \end{equation*}

  Consider the net
  \begin{equation}\label{eq:def:riemann_integral/net}
    \{ S(f, \Delta, \Xi) \}_{(\Delta, \Xi) \in \op{tpart}([a, b])}
  \end{equation}

  Both orders \fullref{def:riemann_partition/order/refinement} and \fullref{def:riemann_partition/order/diameter} on \( \op{tpart}([a, b]) \) provide equivalent convergence for Riemann sums. If the limit exists, \( f \) is said to be \term{Riemann integrable} in \( [a, b] \). We call the limit the \term{Riemann integral} of \( f \) and denote it by
  \begin{equation}\label{eq:def:riemann_integral}
    \int_a^b f(x) dx.
  \end{equation}
\end{definition}
\begin{proof}
  \ImplicationSubProof{def:riemann_partition/order/refinement}{def:riemann_partition/order/diameter} Let \( I \) be the limit \eqref{eq:def:riemann_integral} with respect to \( \preceq_R \). Fix a neighborhood \( U \) of \( 0 \). Since \eqref{eq:def:riemann_integral/net} is eventually in \( I + U \), there exists a tagged partition
  \begin{alignedeq}\label{eq:def:riemann_integral/tagged_zero}
    &\Delta_0: a = x_0^{(0)} < x_1^{(0)} < \ldots < x_n^{(0)} = b \\
    &\Xi_0: \xi_k^{(0)} \in [x_{k-1}^{(0)}, x_k^{(0)}], k = 1, \ldots, n_0.
  \end{alignedeq}
  such that \( S(f, \Gamma, \Eta) \in I + U \) if \( \Gamma \) is a refinement of \( \Delta_0 \).

  Note that \( f \) is \hyperref[def:bounded_function/bounded]{bounded}. Indeed, if\LEM it is unbounded on \( [a, b] \), then there exists a refinement \( (\Gamma, \Eta) \) of \( (\Delta_0, \Xi_0) \) such that
  \begin{equation*}
    S(f, \Gamma, \Eta) - I \not\in U.
  \end{equation*}

  But this contradicts our choice of \( \Delta_0 \). Therefore \( f \) is bounded and there exists a bounded neighborhood \( V_0 \) of \( 0 \) such that \( f([a, b]) \subseteq V_0 \) and hence \( f(x) - f(y) \in V \coloneqq V_0 - V_0 \) for all \( x, y \in [a, b] \).

  Let  \( v > 0 \) be such that \( V \subseteq vU \).

  Let \( (\Delta, \Xi) \) be a tagged partition such that \( \diam(\Delta) \leq \diam(\Delta_0) \).

  We introduce another partition \( \Gamma \coloneqq \Delta \cup \Delta_0 \). Since \( \Gamma \) is a refinement of \( \Delta_0 \), we can use a splitting similar to \eqref{def:riemann_partition/refinement/splitting} such that
  \begin{equation}\label{def:riemann_partition/subdiameter_splitting}
    S(f, \Delta_0, \Xi_0) = \sum_{k=1}^{n_0} \sum_{j=1}^{p_k} f(\xi^{(0)}_k) (y_{k,j} - y_{k,j-1}).
  \end{equation}

  Denote by \( \xi_{k,j} \) the largest tag in \( \Xi \) such that \( \xi_{k,j} \leq y_{k,j} \). Thus
  \begin{equation*}
    S(f, \Delta, \Xi) = \sum_{k=1}^{n_0} \sum_{j=1}^{p_k} f(\xi_{k,j}) (y_{k,j} - y_{k,j-1}).
  \end{equation*}

  For every \( k = 1, \ldots, n \) and every \( j = 0, \ldots, p_k \), choose\LEM an arbitrary tag
  \begin{equation*}
    \Eta: \eta_{k,j} \in [y_{k,j-1}, y_{k,j}].
  \end{equation*}

  Then we have
  \begin{balign*}
    S(f, \Delta, \Xi) - I
    &=
    S(f, \Delta, \Xi) - S(f, \Gamma, \Eta) + \underbrace{S(f, \Gamma, \Eta) - I}_{\in U}
    \in \\ &\in
    \sum_{k=1}^{n_0} \sum_{j=1}^{p_k} [ \underbrace{f(\xi_{k,j}) - f(\eta_{k,j})}_{\in V} ] (y_{k,j} - y_{k,j-1}) + U
    \subseteq \\ &\subseteq
    V \cdot \sum_{k=1}^{n_0} \underbrace{\sum_{j=1}^{p_k} (y_{k,j} - y_{k,j-1})}_{x_k - x_{k-1}} + U
    \subseteq \\ &\subseteq
    \diam(\Delta) \cdot n_0 \cdot V + U
    \subseteq \\ &\subseteq
    (\diam(\Delta) \cdot n_0 \cdot v + 1) U.
  \end{balign*}

  Let \( (\Delta_1, \Xi_1) \) be a tagged partition of \( [a, b] \) such that \( \diam(\Delta_1) \leq \min \left\{ \diam(\Delta_0), \frac 1 {v n_0} \right\} \). It follows that
  \begin{equation}\label{eq:def:riemann_integral/subdiameter_in_neighborhood}
    S(f, \Delta_1, \Xi_1) - I \subseteq 2U.
  \end{equation}

  Until now, \( U \) was fixed. Given any neighborhood \( W \) of \( 0 \), we need to choose a neighborhood \( U \) of \( 0 \) and a corresponding partition \( \Delta_1 \) such that \eqref{eq:def:riemann_integral/subdiameter_in_neighborhood} holds. Then, whenever \( \diam(\Delta) \leq \diam(\Delta_1) \), we have
  \begin{equation*}
    S(f, \Delta, \Xi) - I \subseteq 2U \subseteq W.
  \end{equation*}

  This finishes the proof.

  \ImplicationSubProof{def:riemann_partition/order/diameter}{def:riemann_partition/order/refinement} Note that if \( \Gamma \) is a refinement of \( \Delta \), clearly \( \diam(\Gamma) \leq \diam(\Delta) \). Therefore if the net \eqref{eq:def:riemann_integral/net} with respect to \( \preceq_D \) is eventually in some open set \( U \), the corresponding net with respect to \( \preceq_R \) is also eventually in \( U \). This finishes the proof.
\end{proof}

\begin{corollary}\label{thm:riemann_integrable_implies_bounded}
  A Riemann-integrable function is bounded.
\end{corollary}
\begin{proof}
  Proven in \fullref{def:riemann_integral}.
\end{proof}

\begin{definition}\label{def:darboux_integrability}\mcite[def. 17]{Gordon1991}
  Let \( (\mscrX, \rho) \) be a \hyperref[def:frechet_space]{Frechet space}. Fix a function \( f: [a, b] \to \mscrX \). Similarly to \fullref{def:riemann_integral}, choose any of the orderings \fullref{def:riemann_partition/order/refinement} and \fullref{def:riemann_partition/order/diameter} on the set of all untagged \hyperref[def:riemann_partition/partition]{Riemann partitions} \( \op{part}([a, b]) \).

  For each partition \eqref{eq:def:riemann_partition/partition}, we define its \term{oscillation} via the \hyperref[def:function_oscillation]{function oscillation} of \( f \)
  \begin{equation}\label{eq:def:darboux_integrability/oscillation}
    \omega(f, \Delta) \coloneqq \sum_{k=1}^n \omega(f, [x_{k-1}, x_k]) (x_k - x_{k-1}).
  \end{equation}

  Consider the net
  \begin{equation}\label{eq:def:darboux_integrability/net}
    \{ \omega(f, \Delta) \}_{\Delta \in \op{part}([a, b])}
  \end{equation}

  If this net converges to zero, we say that \( f \) is \term{Darboux integrable}.
\end{definition}

\begin{proposition}\label{thm:darboux_integrable_implies_riemann_integrable}
  In a \hyperref[def:banach_space]{Banach space}, \hyperref[def:darboux_integrability]{Darboux integrability} implies \hyperref[def:riemann_integral]{Riemann integrability}.
\end{proposition}
\begin{proof}
  We will show that the net \eqref{eq:def:riemann_integral/net} is fundamental. Fix \( \varepsilon > 0 \). Since \( f \) is Darboux integrable, there exists an untagged partition \( \Delta_0 \) such that, if \( \Delta \) is a refinement of \( \Delta_0 \), we have
  \begin{equation*}
    \omega(f, \Delta) < \varepsilon.
  \end{equation*}

  Let \( \Delta \) be a refinement of \( \Delta_0 \) and \( \Gamma \) be a refinement of \( \Delta \). Assume that the points of \( \Gamma \) are split as in \eqref{def:riemann_partition/refinement/splitting}. Choose arbitrary tags \( \Xi = \{ \xi_k \}_{k=1}^n \) for \( \Delta \) and \( \Eta = \{ \eta_{k,j} \}_{k=1,j=1}^{n,p_k} \) for \( \Gamma \). For the corresponding Riemann sums, we have
  \begin{balign*}
    &{}\phantom{=}{}
    \norm{S(f, \Delta, \Xi) - S(f, \Gamma, \Eta)}
    = \\ &=
    \norm{\sum_{k=1}^n f(\xi_k) (x_k - x_{k-1}) - \sum_{k=1}^n \sum_{j=1}^{p_k} f(\eta_{k,j}) (y_{k,j} - y_{k,j-1}) }
    \leq \\ &\leq
    \sum_{k=1}^n \sum_{j=1}^{p_k} \norm{f(\xi_k) - f(\eta_{k,j})} (y_{k,j} - y_{k,j-1})
    \leq \\ &\leq
    \sum_{k=1}^n \sum_{j=1}^{p_k} (y_{k,j} - y_{k,j-1}) \sup \{ \norm{f(\xi) - f(\eta)} \colon \xi, \eta \in [y_{k,j-1}, y_{k,j}] \}
    \leq \\ &\leq
    \sum_{k=1}^n \sup \{ f(\xi) - f(\eta) \colon \xi, \eta \in [x_{k-1}, x_k] \} \underbrace{\sum_{j=1}^{p_k} (y_{k,j} - y_{k,j-1})}_{x_{k-1} - x_k}
    = \\ &=
    \omega(f, \Delta)
    <
    \varepsilon.
  \end{balign*}

  Therefore the net \eqref{eq:def:riemann_integral/net} is fundamental and, since \( \mscrX \) is complete, the net converges to a limit.
\end{proof}

\begin{definition}\label{def:darboux_integral}
  Fix a real-valued function \( f: [a, b] \to \BbbR \). The \term{upper Darboux sum} corresponding to the partition \eqref{eq:def:riemann_partition/partition} is defined as
  \begin{equation*}
    \overline{S}(f, \Delta) \coloneqq \sum_{k=1}^n (x_{k-1} - x_k) \sup_{\xi \in [x_{k-1}, x_k]} f(\xi).
  \end{equation*}

  The \term{lower Darboux sum} is defined as
  \begin{equation*}
    \underline{S}(f, \Delta) \coloneqq \sum_{k=1}^n (x_{k-1} - x_k) \inf_{\xi \in [x_{k-1}, x_k]} f(\xi).
  \end{equation*}

  If the nets
  \begin{align}\label{eq:def:darboux_integral/nets}
    \{ \overline{S}(f, \Delta) \}_{\Delta \in \op{part}([a, b])}
    &&
    \{ \underline{S}(f, \Delta) \}_{\Delta \in \op{part}([a, b])}
  \end{align}
  have a common limit, we call this limit the \term{Darboux integral} of \( f \) and, analogously to \fullref{def:riemann_integral}, we denote it by
  \begin{equation*}
    \int_a^b f(x) dx.
  \end{equation*}

  This notation is justified by \fullref{thm:darboux_integral_iff_riemann_integral}.
\end{definition}

\begin{proposition}\label{thm:darboux_integrable_iff_has_darboux_integral}
  A real-valued function \( f: [a, b] \to \BbbR \) is \hyperref[def:darboux_integrability]{Darboux integrable} if and only if it has a \hyperref[def:darboux_integral]{Darboux integral}.
\end{proposition}
\begin{proof}
  Note that, given the partition \eqref{eq:def:riemann_partition/partition}, we have
  \begin{align*}
    \overline{S}(f, \Delta) - \underline{S}(f, \Delta)
    &=
    \sum_{k=1}^n (x_k - x_{k-1}) \left[ \sup_{\xi \in [x_{k-1}, x_k]} f(\xi) - \inf_{\eta \in [x_{k-1}, x_k]} f(\eta) \right]
    = \\ &=
    \sum_{k=1}^n (x_k - x_{k-1}) \left[ \sup_{\xi \in [x_{k-1}, x_k]} f(\xi) + \sup_{\eta \in [x_{k-1}, x_k]} -f(\eta) \right]
    = \\ &=
    \sum_{k=1}^n (x_k - x_{k-1}) \sup \{ f(\xi) - f(\eta) \colon \xi, \eta \in [x_{k-1}, x_k] \}
    = \\ &=
    \sum_{k=1}^n (x_k - x_{k-1}) \sup \{ \abs{f(\xi) - f(\eta)} \colon \xi, \eta \in [x_{k-1}, x_k] \}
    = \\ &=
    \omega(f, \Delta).
  \end{align*}

  Therefore the nets \eqref{eq:def:darboux_integral/nets} converge to a common limit if and only if \( \omega(f, \Delta) \xrightarrow[\Delta]{} 0 \). This finishes the proof.
\end{proof}

\begin{proposition}\label{thm:darboux_integral_iff_riemann_integral}
  A real-valued function \( f: [a, b] \to \BbbR \) has a \hyperref[def:darboux_integral]{Darboux integral} if and only if it has a \hyperref[def:riemann_integral]{Riemann integral}. Furthermore, the two integrals are equal.
\end{proposition}
\begin{proof}
  Fix \( \varepsilon > 0 \).

  \ImplicationSubProof{def:darboux_integral}{def:riemann_integral} Denote by \( I_D \) the Darboux integral of \( f \). Then there exists a partition \( \Delta_0 \) of \( [a, b] \) such that for any refinement \eqref{eq:def:riemann_partition/partition} of \( \Delta_0 \) we have
  \begin{equation*}
    \overline{S}(f, \Delta) - \underline{S}(f, \Delta) < \frac \varepsilon 2.
  \end{equation*}

  In particular, \( I_D - \underline{S}(f, \Delta) < \tfrac \varepsilon 2 \).

  Let \( \Xi \coloneqq \{ \xi_k \}_{k=1}^n \) be tags for \( \Delta \). Then
  \begin{align*}
    \abs{S(f, \Delta, \Xi) - I_D}
    &\leq
    \abs{S(f, \Delta, \Xi) - \underline{S}(f, \Delta)} - \abs{\underline{S}(f, \Delta) - I}
    \leq \\ &\leq
    \abs{\overline{S}(f, \Delta) - \underline{S}(f, \Delta)} - \abs{\underline{S}(f, \Delta) - I}
    < \\ &<
    \frac \varepsilon 2 + \frac \varepsilon 2
    = \\ &=
    \varepsilon.
  \end{align*}

  Therefore \( I_D \) is also a Riemann integral for \( f \).

  \ImplicationSubProof{def:riemann_integral}{def:darboux_integral} Denote by \( I_R \) the Riemann integral of \( f \). Then there exists a partition \eqref{eq:def:riemann_integral/tagged_zero} such that for any partition \eqref{eq:def:riemann_partition/tagged} with \( \diam(\Delta) \leq \diam(\Delta_0) \), we have
  \begin{equation*}
    \abs{S(f, \Delta, \Xi) - I_R} < \frac \varepsilon 2.
  \end{equation*}

  Since \eqref{thm:riemann_integrable_implies_bounded} is bounded, there exists a constant \( M > 0 \) such that \( \abs{f(\xi) - f(\eta)} < M \) for any \( \xi, \eta \in [a, b] \).

  Using an analogous to \eqref{def:riemann_partition/subdiameter_splitting} splitting for the refinement \( \Gamma \coloneqq \Delta \cup \Delta_0 \) of \( \Delta_0 \), we obtain
  \begin{align*}
    \overline{S}(f, \Gamma) - S(f, \Gamma, \Eta)
    &=
    \sum_{k=1}^{n_0} \sum_{k=1}^{p_k} [ \sup_{\xi \in [y_{k,j-1}, y_{k,j}]} f(\eta) - f(\eta_{k,j}) ] (y_{k,j} - y_{k,j-1})
    \leq \\ &\leq
    M \sum_{k=1}^{n_0} \sum_{k=1}^{p_k} (y_{k,j} - y_{k,j-1})
    \leq \\ &\leq
    M \cdot n_0 \cdot \diam(\Gamma).
  \end{align*}

  By choosing a tagged partition \( (\Delta_1, \Xi_1) \) with \( \diam(\Delta_1) < \min \left\{ \diam(\Delta_0), \frac \varepsilon {2 M n_0} \right\} \), we obtain
  \begin{equation*}
    \overline{S}(f, \Delta_1) - S(f, \Delta_1, \Xi) < \frac \varepsilon 2.
  \end{equation*}

  Therefore, whenever \( \diam(\Delta) \leq \diam(\Delta_1) \),
  \begin{equation*}
    \overline{S}(f, \Delta) - I_R
    =
    \overline{S}(f, \Delta) - S(f, \Delta, \Xi) + S(f, \Delta, \Xi) - I_R
    <
    \frac \varepsilon 2 + \frac \varepsilon 2
    =
    \varepsilon.
  \end{equation*}

  Thus the net \( \{ \overline{S}(f, \Delta) \}_{\Delta \in \op{part}([a, b])} \) of upper Darboux sums converges to \( I \). We can analogously show that the lower Darboux sums also converge to \( I_R \). Hence \( I_R \) is the Darboux integral of \( f \).
\end{proof}

\begin{proposition}\label{thm:countinuous_functions_integrable}
  In a \hyperref[def:frechet_space]{Frechet space} \( (\mscrX, \rho) \), \hyperref[def:global_continuity]{continuous functions} \( f: [a, b] \to \mscrX \) are \hyperref[def:darboux_integrability]{Darboux integrable}.
\end{proposition}
\begin{proof}
  Fix \( \delta > 0 \). Let \eqref{eq:def:riemann_partition/partition} be a partition of \( [a, b] \) such that \( \diam(\Delta) < \delta \). We have
  \begin{equation*}
    \omega(f, \Delta)
    =
    \sum_{k=1}^n \omega(f, [x_{k-1}, x_k]) (x_k - x_{k-1})
    \leq
    \sum_{k=1}^n \omega(f, \diam(\Delta)) \diam(\Delta)
    <
    n \omega(f, \delta) \delta.
  \end{equation*}

  Now fix \( \varepsilon > 0 \). A continuous function on a compact interval is \hyperref[def:uniform_continuity]{uniformly continuous}. By \fullref{thm:function_oscillation_properties/continuity_condition}, there exists \( \delta_0 > 0 \) such that \( \omega(f, \delta_0) < \varepsilon \). It is then enough to choose
  \begin{equation*}
    \delta \coloneqq \frac {\delta_0} {n \varepsilon}
  \end{equation*}
  to obtain
  \begin{equation*}
    \omega(f, \Delta)
    <
    n \delta \omega(f, \delta)
    =
    \delta_0 \frac {\omega(f, \delta)} {\varepsilon}
    \overset {\ref{thm:function_oscillation_properties/monotone}} \leq
    \delta_0 \frac {\omega(f, \delta_0)} {\varepsilon}
    <
    \varepsilon.
  \end{equation*}

  Therefore the same inequality holds for all partitions with diameters smaller than \( \delta \), which implies that \( f \) is Darboux integrable.
\end{proof}

\begin{proposition}\label{thm:componentwise_integration}
  Let \( f: [a, b] \to \BbbR^n \) be a function and let \( f_k, k = 1, \ldots, n \) be its components. We have that \( f \) is integrable if and only if \( f_k \) is integrable for \( k = 1, \ldots, n \). Furthermore,
  \begin{equation}\label{eq:thm:componentwise_integration}
    \bigintss_a^b \begin{pmatrix} f_1(x) \\ \vdots \\ f_n(x) \end{pmatrix} dx
    =
    \begin{pmatrix} {\displaystyle \int_a^b f_1(x)} dx \\ \vdots \\ {\displaystyle \int_a^b f_n(x) dx} \end{pmatrix}.
  \end{equation}
\end{proposition}
\begin{proof}
  \SufficiencySubProof Let \( f \) be integrable and let \( I = (I_1, \ldots, I_n)^T \) be the value of the integral. Fix \( \varepsilon > 0 \) and let \( (\Delta, \Xi) \) be a tagged partition such that
  \begin{equation*}
    \norm{I - S(f, \Delta, \Xi)} < \varepsilon.
  \end{equation*}

  Then for any \( k = 1, \ldots, n \) we have
  \begin{equation*}
    \norm{I - S(f, \Delta, \Xi)}^2
    =
    \sum_{m=1}^n \abs{I_m - S(f_m, \Delta, \Xi)}^2
    \geq
    \abs{I_k - S(f_k, \Delta, \Xi)}^2,
  \end{equation*}
  hence
  \begin{equation*}
    \abs{I_k - S(f_k, \Delta, \Xi)} < \varepsilon.
  \end{equation*}

  Therefore \( f_k \) is integrable and
  \begin{equation*}
    \int_a^b f_k(x) dx = I_k.
  \end{equation*}

  \NecessitySubProof Let \( f_k \) be integrable for \( k = 1, \ldots, n \) with value \( I_k \). Put \( I \coloneqq (I_1, \ldots, I_n)^T \). Fix \( \delta > 0 \) and let \( (\Delta_k, \Xi_k) \) be a parition such that
  \begin{equation*}
    \abs{I_k - S(f_k, \Delta_k, \Xi_k)} < \delta
  \end{equation*}

  Let \( \Gamma \coloneqq \bigcup_{k=1}^n \Delta_k \) and let \( \Eta \) be tags for \( \Gamma \). Since \( \diam(\Gamma) \leq \diam(\Delta_k) \) and since \( f_k \) is integrable, we have
  \begin{equation*}
    \abs{I_k - S(f_k, \Gamma, \Eta)} < \delta \quad\forall k = 1, \ldots, n.
  \end{equation*}

  We have
  \begin{equation*}
    \norm{I - S(f, \Gamma, \Eta)}
    =
    \sqrt{\sum_{m=1}^n \abs{I_m - S(f_m, \Gamma, \Eta)}^2}
    <
    \delta \sqrt{n}.
  \end{equation*}

  Therefore, given \( \varepsilon > 0 \), it is enough to choose \( \delta \coloneqq \frac {\varepsilon} {\sqrt n} \) to obtain a tagged partition \( (\Gamma_0, \Eta_0) \) so that for \( (\Gamma, \Eta) \) with \( \diam(\Gamma) < \diam(\Gamma_0) \) we have
  \begin{equation*}
    \norm{I - S(f, \Gamma, \Eta)} < \varepsilon.
  \end{equation*}

  This proves integrability of \( f \) and \eqref{eq:thm:componentwise_integration}.
\end{proof}
