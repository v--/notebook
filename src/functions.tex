\subsection{Functions}\label{subsec:functions}

\begin{remark}\label{rem:function_definition}
  It is not straightforward to formalize the notion of correspondence between two values. We will use the terms \term{mapping}, \term{function}, \term{transformation} and \term{operator}. We will define functions as special binary relations in \fullref{def:function}. Despite this being a standard practice, this has several drawbacks:

  \begin{itemize}
    \item There are some more mappings than the functions defined in \fullref{def:function}. For example, assigning to a set \( A \) its \hyperref[def:basic_set_operations]{power set} \( \mscrP(A) \) can be cannot be regarded as a function because its domain and range should both be the set of all sets whose existence is inconsistent with \hyperref[def:zfc]{\logic{ZFC}} by \fullref{thm:zfc_existence_theorems/member_of_itself}.

    \item We often work with spaces that have some additional structure in addition to being sets. In this case, we are often only interested in maps that preserve this structure. This is the case with \hyperref[def:group/homomorphism]{group homomorphisms}, for example.

    In terms of \hyperref[def:first_order_structure]{first-order structures}, not every function is a \hyperref[def:first_order_homomorphism]{homomorphism}.

    This is a motivating example for the benefits of \hyperref[sec:category_theory]{category theory}, where the notion of \hyperref[def:category/C2]{morphism} is able to capture this additional structure.

    It is often important to consider functions that are not homomorphisms, however. For example, function spaces over \( \BbbR \) contain some very complicated functions that are not field homomorphisms, order homomorphisms nor continuous functions and thus preserve no structure of \( \BbbR \).

    \item Several generalizations of the standard notion of a function are often used. These include \hyperref[def:multi_valued_function]{multi-valued} or \hyperref[def:partial_function]{partial functions}. Both are not function, strictly speaking.

    For simplicity of exposition, we take multi-valued functions as primitive notions and define single-valued functions as special cases. This is actually done in \cite[def. 2.31]{OpenLogicFull} and \cite[8]{Kelley1955} except that the corresponding authors conflate multi-valued functions and relations.

    \item Set-theoretic functions are often used in contexts where they do not refer to the intuitive notion of a mapping. Examples include \hyperref[def:cartesian_product]{Cartesian products} and \hyperref[def:indexed_family]{indexed families}.
  \end{itemize}
\end{remark}

\begin{definition}\label{def:multi_valued_function}
  A \term{multi-valued function} from \( A \) to \( B \) is simply a \hyperref[def:binary_relation]{binary relation} \( (F, A, B) \). For us the difference between a multi-valued function and a relation is merely in how we treat them.

  As discussed in \fullref{def:relation/graph}, the common practive of identifying a multi-valued function \( (F, A, B) \) with its \hyperref[def:relation/graph]{graph} \( F \) with no regard to its \hyperref[def:relation/signature]{signature} \( (A, B) \) has serious drawbacks that we wish to avoid.

  We use the more established notation \( F: A \rightrightarrows B \) rather than \( (F, A, B) \) and call the string of symbols \( A \rightrightarrows B \) the signature of \( F \) rather than the tuple \( (A, B) \).

  \begin{thmenum}[series=def:multi_valued_function]
    \thmitem{def:multi_valued_function/value} The \term{value} of \( F \) at \( x \) is
    \begin{equation*}
      F(x) \coloneqq \set{ y \in B \given (x, y) \in F }.
    \end{equation*}

    In case \( x \) is not a concrete value, then \( F(x) \) stands for the function \( F \) itself. In other words, \( F(x) \) refers to a member of \( B \) if \( x \) is a \hyperref[def:first_order_syntax/formula_bound_variables]{bound variable} and \( F(x) \) refers to the function \( F \) if \( x \) is a \hyperref[def:first_order_syntax/formula_free_variables]{free variable}.

    \thmitem{def:multi_valued_function/set_value} We also define the value of \( F \) at a subset \( A' \) of \( A \).
    \begin{equation*}
      F(A') \coloneqq \set{ F(x) \given x \in A' }.
    \end{equation*}

    This is also called the \term{action} of \( F \) on the set \( A' \) or the \term{image} of \( A' \) under \( F \). We also refer to \( F(x) \) as the image of \( x \) under \( F \) because \( F(x) = F(\set{x}) \).

    \thmitem{def:multi_valued_function/argument} If \( A = A_1 \times \cdots \times A_n \) is a \hyperref[def:binary_cartesian_product]{finite Cartesian product}, we denote the function \( F: X_1 \times \cdots \times X_n \rightrightarrows B \) by \( F(x_1, \ldots, x_n) \) and regard \( x_1, \ldots, x_n \) as free variables that have no assigned value. These variables are called \term{parameters} or \term{arguments}. This notion is somewhat informal and depends on the context since \( A \) can usually be represented as a Cartesian product in different ways and with different arities. For example, if \( A = B \times C \), we can write both \( F(a) \) and \( F(b, c) \) and the function has a different number of parameters in each case. In practice the number of arguments is usually clear.

    For example, in \hyperref[def:first_order_semantics]{classical first-order semantics}, to each \( n \)-ary functional symbol there corresponds an \( n \)-ary function with the unambiguous signature \( X^n \to X \).

    When working over a vector space like \( \BbbR^2 \), on the other side, depending on the context we regard \hyperref[rem:functional]{functionals} as either unary or binary functions.

    We sometimes refer to \( F \) as a \term{dependent variable} since it depends on its parameters. In this later case, we call the parameters \term{independent variables}.
  \end{thmenum}

  The following terminology is consistent with relations:
  \begin{thmenum}[resume=def:multi_valued_function]
    \thmitem{def:multi_valued_function/graph} The \term{graph} \( \gph(F) \) of \( F \) is the graph of the relation \( F \). This is consistent with \fullref{def:relation/graph}.

    \thmitem{def:multi_valued_function/domain} The \term{domain} \( \dom(F) \) of \( F \) is the set of all values for which \( \dom(F) \neq \varnothing \). This is consistent with \fullref{def:binary_relation/domain}.

    \thmitem{def:multi_valued_function/image} The \term{image} \( \img(F) \) is the set of all \( y \in B \) that belong to the set \( F(x) \) for at least one \( x \in A \). This is consistent with \fullref{def:binary_relation/image}.

    \thmitem{def:multi_valued_function/range} The \term{range} \( \range(F) \) is simply the set \( B \). This is consistent with \fullref{def:binary_relation/range}.

    \thmitem{def:multi_valued_function/restriction} The \term{restriction} of \( F: A \rightrightarrows B \) to \( A' \subseteq A \) is the multi-valued function \( F\restr_{A'}: A' \rightarrow B \) such that \( F \). We say that \( F \) is an \term{extension} of \( \restr_{A'} \). This is consistent with \fullref{def:binary_relation/restriction}.

    \thmitem{def:multi_valued_function/inverse} The \term{inverse} \( F^{-1}: B \rightrightarrows A \) of a multi-valued function \( F: A \rightrightarrows B \) is the multi-valued function in which assigns to every element \( y \) of the image \( \img(F) \) the set of all \( x \in A \) such that \( y \in F(x) \). This is consistent with \fullref{def:binary_relation/inverse}.

    \thmitem{def:multi_valued_function/composition} The \term{composition} \( G \bincirc F \) of two multi-valued functions \( F: A \rightrightarrows B \) and \( G: B \rightrightarrows C \) is the function
    \begin{equation*}
      [G \bincirc F](x) \coloneqq G(F(x)).
    \end{equation*}

    The square brackets in \( [G \bincirc F] \) are not a special notation but rather another pair of delimiters that looks different from parentheses for the sake of reducing visual clutter.

    This definition is consistent with \fullref{def:binary_relation/composition}.
  \end{thmenum}

  The following terminology is inconsistent with relations:
  \begin{thmenum}[resume=def:multi_valued_function]
    \thmitem{def:multi_valued_function/total} The term \term{total multi-valued function} means that \( \dom(F) = A \), that is, that \( F(x) \neq \varnothing \) for all \( x \in A \). This is very different from total binary relations as defined in \fullref{def:binary_relation/total}.

    \thmitem{def:multi_valued_function/arity} The \term{arity} of a multi-valued function is its number of \hyperref[def:multi_valued_function/arity]{arguments}, in case it is clear how many arguments we consider the function to have. This should not to be confused with the arity of a relation as defined in \fullref{def:relation/arity} --- multi-valued functions are always binary relations.

    \thmitem{def:multi_valued_function/symmetric} A function \( f: A \times A \to B \) is called \term{symmetric} if, for all \( x, y \in A \), we have \( f(x, y) = f(y, x) \).

    Symmetric functions should not be confused with symmetric relations as defined in \fullref{def:derived_relations/symmetric}.

    \thmitem{def:multi_valued_function/diagonal}\mcite[exer. 3.1.1]{Leinster2016Basic} The function corresponding to the diagonal relation \( \increment A \) as defined in \fullref{def:binary_relation/diagonal} is called the \term{identity function} on \( A \) and denoted by \( \id_A \). The \term{diagonal function} is instead defined as \( f: A \to A^2 \) is defined as \( f(x) \coloneqq (x, x) \). This is mostly used in category theory.
  \end{thmenum}

  We define some additional terminology:
  \begin{thmenum}[resume=def:multi_valued_function]
    \thmitem{def:multi_valued_function/endofunction} Functions from a set to itself (e.g. \( F: A \rightrightarrows A \)) are called \term{endofunctions}.

    \thmitem{def:multi_valued_function/involution} If \( F = F^{-1} \), we say that \( F \) is an \term{involution}. See \fullref{def:set_with_involution}.

    \thmitem{def:multi_valued_function/large_preimage} For a fixed set \( B' \subseteq B \), its \term{large preimage} or simply \term{preimage} under \( F: A \to B \) is the image of \( B' \) under the inverse function \( F^{-1}: B \rightrightarrows A \). For a single value \( y \in B \), we call \( F^{-1}(y) \) the \term{fiber} of \( y \) under \( F \).

    \thmitem{def:multi_valued_function/small_preimage} Analogously, we define its \term{small preimage} as
    \begin{equation*}
      F_{-1}(B') \coloneqq \set{ x \in A \colon F(x) \subseteq B' }.
    \end{equation*}

    Obviously \( F_{-1}(B') \subseteq F^{-1}(B') \).

    \thmitem{def:multi_valued_function/superposition} Although the terms \enquote{composition} and \enquote{superposition} are used interchangeably (for example in \cite[\textnumero 25]{Фихтенгольц1968Том1}), \enquote{superposition} often refers to the following special case of composition:

    If we are given the functions \( F_k: A \rightrightarrows B_k, k = 1, \ldots, n \) and \( G: B_1 \times \cdots \times B_n \rightrightarrows C \), their \term{superposition} \( H: A \rightrightarrows C \) is
    \begin{equation*}
      H(x) \coloneqq G(F_1(x), \ldots, F_n(x)).
    \end{equation*}
  \end{thmenum}
\end{definition}

\begin{proposition}\label{thm:multivalued_function_properties}
  \hyperref[def:multi_valued_function]{Multi-valued functions} have the following basic properties:

  \begin{thmenum}
    \thmitem{thm:multivalued_function_properties/associative} \hyperref[def:multi_valued_function/composition]{Composition} is associative. That is, for any three functions \( F: A \to B \), \( G: B \to C \) and \( H: C \to D \) we have
    \begin{equation*}
      H \bincirc [G \bincirc F] = [H \bincirc G] \bincirc F.
    \end{equation*}

    We will henceforth simply write \( H \bincirc G \bincirc F \).

    \thmitem{thm:multivalued_function_properties/composition_inverse} If \( F: A \to B \) and \( G: B \to C \) are \hyperref[def:multi_valued_function]{multi-valued functions}, then
    \begin{equation*}
      [G \bincirc F]^{-1} = F^{-1} \bincirc G^{-1}.
    \end{equation*}
  \end{thmenum}
\end{proposition}
\begin{proof}
  \SubProofOf{thm:multivalued_function_properties/associative} Let \( a \in A \). Then in order for \( d \in D \) to belong to \( [[H \bincirc G] \bincirc F](a) \), there must exist values \( b \in B \) and \( c \in C \) such that \( b \in F(a) \) and \( c \in G(b) \) and \( d \in H(c) \). Clearly this is also the condition for \( d \) to belong to \( [H \bincirc [G \bincirc F]](a) \).

  \SubProofOf{thm:multivalued_function_properties/composition_inverse} Since \( G \bincirc F \) has signature \( A \to C \), clearly \( [G \bincirc F]^{-1} \) has signature \( C \to A \). Let \( c \in C \).

  \begin{itemize}
    \item If \( [G \bincirc F]^{-1}(c) \) is empty, \( c \not\in \img(G \bincirc F) \), hence either \( G^{-1}(c) \) is empty or is nonempty but disjoint from \( \img(F) \). Hence \( [F^{-1} \bincirc G^{-1}](c) \) is also empty.

    \item Suppose that \( [G \bincirc F]^{-1}(c) \) is not empty and let \( a \in [G \bincirc F]^{-1}(c) \).

    By definition, there exists \( b \in B \) and such that \( b \in F(a) \) and \( c \in G(b) \). Hence \( b \in G^{-1}(c) \) and \( a \in F^{-1}(b) \), which implies that the image of \( c \) under the composition \( F^{-1} \bincirc G^{-1} \) also contains \( a \).
  \end{itemize}

  In both cases, for every \( c \in C \) we have
  \begin{equation*}
    [G \bincirc F]^{-1}(c) = [F^{-1} \bincirc G^{-1}](c).
  \end{equation*}

  Hence the two multi-valued functions are equal.
\end{proof}

\begin{definition}\label{def:function}
  Although \hyperref[def:multi_valued_function]{multi-valued functions} are very general, they are not studied nearly as extensively as single-valued functions and are not nearly as useful.

  The \hyperref[def:multi_valued_function]{multi-valued function} \( F: A \rightrightarrows B \) is called a \term{single-valued function} if \( F(x) \) is a \hyperref[rem:singleton_sets]{singleton set} for each \( x \in A \). In this case, we write \( F: A \to B \) rather than \( F: A \rightrightarrows B \).

  All terminology from \fullref{def:multi_valued_function} holds for single-valued functions.

  By convention, when both single-valued and multi-valued functions are involved, the former are denoted using lowercase letters and the latter using uppercase letters. For this reason, we will prefer lowercase letters.

  Strictly speaking, the value \( f(x) \) of a single-valued function \( f: A \to B \) is a singleton set. It is common practice (e.g. in \cite[def. 3.1]{OpenLogicFull} and \cite[10]{Kelley1955}) to define the value of a single-valued function to be an element of \( B \) rather than a subset of \( B \). Unless this would be confusing, we identify \( f(x) \) with its only element due to the convention established in \fullref{rem:singleton_sets}.

  Unless otherwise noted, we will now conflate the terms \enquote{function} and \term{single-valued function}.
  \begin{thmenum}
    \thmitem{def:function/selection}\mcite[52]{DontchevRockafellar2014} If \( f: A \to B \) is a single-valued function, \( F: A \to B \) is a multi-valued function and \( \gph(f) \subseteq \gph(F) \), we say that \( f \) is a \term{selection} of \( F \).

    \thmitem{def:function/set_of_functions} We denote the set of all single-valued functions from \( A \) to \( B \) by \( \fun(A, B) \). We may also use either \( \cat{Set}(A, B) \) (which is consistent with \hyperref[def:category_of_sets]{morphisms in the category of sets}) or by \( B^A \) (which is consistent with \hyperref[def:cardinal_arithmetic]{cardinal arithmetic}). We abbreviate \( \fun(A, A) \) as \( \fun(A) \)
  \end{thmenum}
\end{definition}

\begin{proposition}\label{thm:function_properties}
  \hyperref[def:function]{Single-valued functions} have the following properties when regarded as \hyperref[def:multi_valued_function]{multi-valued functions}:
  \begin{thmenum}
    \thmitem{thm:function_properties/total} Single-valued functions are \hyperref[def:multi_valued_function/total]{total} as multi-valued functions. As a consequence, a multi-valued function cannot have a selection unless it is total \hyperref[def:multi_valued_function/total]{total}.

    This explains why there is no established terminology for the set \( A \) analogous to \enquote{range} for \( B \) --- we rarely even need to differentiate between \( A \) and \( \dom(f) \).

    \thmitem{thm:function_properties/composition} The \hyperref[def:multi_valued_function/composition]{composition} and \hyperref[def:multi_valued_function/superposition]{superposition} of single-valued functions is a single-valued function.

    \thmitem{thm:function_properties/preimage} The \hyperref[def:multi_valued_function/large_preimage]{large preimages} and \hyperref[def:multi_valued_function/small_preimage]{small preimages} of single-valued functions are identical. We restrict ourselves to large preimages with the notation \( f^{-1}(x) \) and refer to them simply as preimages.

    Note that preimages are multi-valued in general.
  \end{thmenum}
\end{proposition}
\begin{proof}
  All of the statements are obvious.
\end{proof}

\begin{proposition}\label{thm:existence_of_multi_valued_function_selection}
  Every multi-valued function has a selection.

  This theorem is equivalent to \fullref{def:zfc/choice} --- see \fullref{thm:axiom_of_choice_equivalences/selection}.
\end{proposition}

\begin{definition}\label{def:partial_function}
  A \hyperref[def:multi_valued_function]{Multi-valued function} that is otherwise single-valued but not necessarily \hyperref[def:multi_valued_function/total]{total} is called a \term{partial function}. That is, \( f: A \to B \) is a partial function if \( f(x) \) has at most one element for every \( x \in A \).
\end{definition}

\begin{definition}\label{def:function_invertibility}
  In connection with \fullref{def:first_order_homomorphism_invertibility} and \fullref{def:morphism_invertibility}, we introduce the following terminology:
  \begin{thmenum}
    \thmitem{def:function_invertibility/injection} The (single-valued) function \( f \) is called \term{injective}, \term{left-invertible} or \term{one-to-one} if any of the following equivalent conditions hold:
    \begin{thmenum}
      \thmitem{def:function_invertibility/injection/points} Different points in \( A \) have different images under \( f \).
      \thmitem{def:function_invertibility/injection/preimage} The preimage of any point in \( B \) is either empty or a singleton.
      \thmitem{def:function_invertibility/injection/inverse} The inverse is a (single-valued) \hyperref[def:partial_function]{partial function}.
      \thmitem{def:function_invertibility/injection/categorical} There exists a function \( g: B \to A \) such that \( g \bincirc f = \id_A \).
    \end{thmenum}

    We sometimes use the categorical notation \( f: A \hookrightarrow B \).

    \thmitem{def:function_invertibility/surjection} The function \( f \) is called \term{surjective}, \term{right-invertible} or \term{onto} if any of the equivalent conditions hold:
    \begin{thmenum}
      \thmitem{def:function_invertibility/surjection/points} Each point in \( B \) is the image of at least one point in \( A \).
      \thmitem{def:function_invertibility/surjection/image} The image of \( f \) equals the range of \( f \).
      \thmitem{def:function_invertibility/surjection/inverse} The inverse is a total multi-valued function.
      \thmitem{def:function_invertibility/surjection/categorical} There exists a function \( g: B \to A \) such that \( f \bincirc g = \id_B \).
    \end{thmenum}

    We sometimes use the categorical notation \( f: A \twoheadrightarrow B \).

    \thmitem{def:function_invertibility/bijection} Finally, \( f \) is called \term{bijective} or simply \term{invertible} if any of the equivalent conditions hold:
    \begin{thmenum}
      \thmitem{def:function_invertibility/bijection/direct} It is both injective and surjective.
      \thmitem{def:function_invertibility/bijection/points} Each point in \( B \) is the image of exactly one point in \( A \).
      \thmitem{def:function_invertibility/bijection/preimage} The preimage of any point in \( B \) is a singleton.
      \thmitem{def:function_invertibility/bijection/inverse} The inverse is a single-valued total function.
      \thmitem{def:function_invertibility/bijection/categorical} There exists a function \( g: B \to A \) such that both \( g \bincirc f = \id_A \) and \( f \bincirc g = \id_B \).
    \end{thmenum}
  \end{thmenum}
\end{definition}
\begin{proof}
  \SubProofOf{def:function_invertibility/injection} First, assume that \ref{def:function_invertibility/injection/points} holds. Let \( f: A \to B \) be a function such that \( x_1 \neq x_2 \) implies \( f(x_1) \neq f(x_2) \) for all \( x_1, x_2 \in A \). The preimage of any point \( y \in B \setminus \img(f) \) is then \( \varnothing \). The contraposition to \ref{def:function_invertibility/injection/points}, states that if the points \( x_1, x_2 \in A \) are such that \( f(x_1) = f(x_2) \), then \( x_1 = x_2 \). Hence the preimage of any point in the image \( \img(f) \) is a singleton set (this part demonstrates \ref{def:function_invertibility/injection/preimage} and \ref{def:function_invertibility/injection/preimage}).

  \begin{itemize}
    \item If \( A \) is empty, then \( f \) is an empty relation and hence its inverse is also the empty function and \( f^{-1} \bincirc f = \id_A \).

    \item If \( A \) is not empty, pick any value \( x_0 \in A \). Define the single-valued total function \( g: B \to A \) as follows:
    \begin{equation*}
      g(y) \coloneqq \begin{cases}
        f^{-1}(y), &y \in \img(f) \\
        x_0,       &\T{otherwise}.
      \end{cases}
    \end{equation*}

    Then \( g \bincirc f = \id_A \) because we only care about the values in the image of \( f \). The rest of the definition of \( g \) simply ensures that it is a single-valued total function.
  \end{itemize}

  For the converse, let \( f: A \to B \) and \( g: B \to A \) be two single-valued functions such that \( g \bincirc f = \id_A \).

  \begin{itemize}
    \item If either \( A \) or \( B \) is empty, obviously \( f = g = \varnothing \) and \( g \bincirc f = \varnothing \). If we suppose that \( B \) is empty but \( A \) is not, then \( g \bincirc f = \varnothing \neq \id_A \), which contradicts our assumption. Thus if either \( A \) or \( B \) is empty, then \( A \) is necessarily empty.

    Then it holds vacuously that different points correspond to different values under \( f \) because there are no points in the domain \( A \) of \( f \).

    \item Suppose that both \( A \) and \( B \) are nonempty and fix two points \( x_1, x_2 \in A \). Since \( g \bincirc f = \id_A \), by definition of composition there exist points \( y_1, y_2 \in B \) such that \( f(x_k) = y_k \) and \( g(y_k) = x_k \) for \( k \in \set{ 1, 2 } \). But these are precisely the values \( f(x_1) \) and \( f(x_2) \) since the function \( f \) is single-valued.

    If we have \( f(x_1) = f(x_2) \), it follows that \( x_1 = g(f(x_1)) = g(f(x_2)) = x_2 \), hence \( x_1 = x_2 \). This is exactly the contraposition to \ref{def:function_invertibility/injection/points}.
  \end{itemize}

  In both cases, \ref{def:function_invertibility/injection/categorical} implies \ref{def:function_invertibility/injection/points}.

  \SubProofOf{def:function_invertibility/surjection} Let \( f: A \to B \) be a function. First suppose that \ref{def:function_invertibility/surjection/points} holds. That is, for each point \( y \in B \) there exists \( x \in A \) such that \( f(x) = y \). It is clear that \ref{def:function_invertibility/surjection/image} and \ref{def:function_invertibility/surjection/inverse} are reformulations of this condition.

  Since \( f^{-1} \) is a total multi-valued function, by \fullref{thm:existence_of_multi_valued_function_selection}, there exists a selection \( g \) of the multi-valued function \( f^{-1} \). For every \( y \in B \) we then have \( f(g(y)) = y \), hence \( f \bincirc g = \id_B \).

  \SubProofOf{def:function_invertibility/bijection} It is clear that the conditions \fullref{def:function_invertibility/bijection/points,def:function_invertibility/bijection/preimage,def:function_invertibility/bijection/inverse,def:function_invertibility/bijection/categorical} are simply conjunctions of the corresponding conditions from \fullref{def:function_invertibility/injection} and \fullref{def:function_invertibility/surjection}.

  It is worth noting however, that unlike in the proof that \fullref{def:function_invertibility/surjection/inverse} implies \fullref{def:function_invertibility/surjection/categorical}, we do not need the axiom of choice here because the inverse is single-valued.
\end{proof}

\begin{definition}\label{def:currying}
  Given the \hyperref[def:multi_valued_function/arity]{two-argument} function \( f: A \times B \to C \), we can define the function \( g: A \to \fun(B, C) \) as
  \begin{equation*}
    g(x) \coloneqq (y \mapsto f(x, y)).
  \end{equation*}

  This process is called \term{currying} after Haskell Curry.

  Currying is useful if we have somehow fixed a value \( x_0 \in A \), in which case we can \enquote{get rid} of one argument by introducing some shortcut for the function \( g(x_0): B \to C \) for the sake of reducing notational clutter. See \fullref{def:differentiability/first_variation} for an example of how this can be used in the wild.
\end{definition}

\begin{definition}\label{def:indexed_family}
  When considering finite \hyperref[rem:family_of_sets]{families of sets}, it is enough to consider \hyperref[def:binary_cartesian_product]{n-tuples}. For example, given the sets \( A_1, \ldots, A_n \), we can think of the family \( \set{ A_k }_k \) as the ordered tuple \( (A_1, \ldots, A_n) \), where the \( k \)-th coordinate of the tuple gives us the \( k \)-th set of the family.

  As explained in \fullref{def:binary_cartesian_product}, this formally requires us to have a notion of natural numbers and finiteness that we have not introduced in set theory at this point, however the intention of the discussion is clear.

  This approach requires the family to be ordered and finite. Both of these assumptions can easily be circumvented.

  A more natural approach to indexed families is given by functions. We choose a set \( \mscrK \), called the \term{index set}. An \term{indexed family} is simply a function from \( \mscrK \) into some family of sets \( \mscrA \)

  This function maps every element \( k \) of \( \mscrK \) into a set \( A_k \in \mscrA \). For convenience, this family is denoted as
  \begin{equation*}
    \set{ A_k }_{k \in \mscrK}
  \end{equation*}
  without any reference to the underlying function.

  See \fullref{rem:indexed_family_notation} for a discussion of this notation.

  For convenience we will write \( \set{ A_k }_{k \in \mscrK} \subseteq \mscrA \) despite the indexed family actually being encoded as a very different set.

  Indexed families allow us to write \( \bigcup_{k \in \mscrK} A_k \) rather than \( \bigcup \mscrA \), for example (we assume here that the underlying function is \hyperref[def:function_invertibility/surjection]{surjective}).

  A more general framework than indexed families that also considers relations between the family's elements is given by diagrams in category theory defined in \fullref{def:categorical_diagram}.
\end{definition}

\begin{example}\label{ex:def:indexed_family}
  The following are commonly encountered \hyperref[def:indexed_family]{indexed families}:

  \begin{itemize}
    \item Every n-tuple \( (x_1, \ldots, x_n) \) is an indexed family with domain \( \mscrK = \set{ 1, \ldots, n } \). Here the ordering of \( \mscrK \) plays an important role.

    \item An important corner case is when \( \mscrK \) is the empty set. Since the only possible indexing function is then the empty function, we simply say that the resulting family is empty.

    \item In continuous stochastic processes, it is convenient to consider families of random variables \( \{ X_t \}_{t \geq 0} \) indexed by the nonnegative real numbers \( \mscrK = \BbbR_{\geq 0} \). The indexing parameter is often denoted by \( t \geq 0 \) and interpreted as time.

    \item An \( n \times m \) \hyperref[def:array/matrix]{matrix} \( A = \set{ a_{i,j} } \) is a family of scalars indexed by the unordered set \( \mscrK = \set{ 1, \ldots, n } \times \set{ 1, \ldots, m } \). More general \hyperref[def:array]{arrays} are also indexed families.

    \item \hyperref[def:topological_net]{Topological nets} are indexed families where the domain is a \hyperref[def:directed_set]{directed set}.
  \end{itemize}
\end{example}

\begin{remark}\label{rem:indexed_family_notation}
  Since we denote \hyperref[def:cartesian_product]{tuples} by \( (A_1, \ldots, A_n) \), it is consistent to denote \hyperref[def:indexed_family]{indexed families} by
  \begin{equation*}
    ( A_k )_{k \in \mscrK}
  \end{equation*}
  rather than
  \begin{equation*}
    \set{ A_k }_{k \in \mscrK}.
  \end{equation*}

  This is actually done when we want to enumerate elements of a sequence, e.g. see \fullref{def:polynomial}.

  In general, however, we prefer the latter notation because
  \begin{equation*}
    \set[\Big]{ \log \parens[\Big]{ f^{(n)}(x_k) } }_{k=1}^\infty.
  \end{equation*}
  is both more conventional (in analysis) and more aesthetically pleasing than
  \begin{equation*}
    \parens[\Big]{ \log \parens[\Big]{ f^{(n)}(x_k) } }_{k=1}^\infty.
  \end{equation*}

  The difference may be more visible in simpler cases like
  \begin{balign*}
    (\sin(k))_{k \in \mscrK}
     &  &
    \{\sin(k)\}_{k \in \mscrK}.
  \end{balign*}
\end{remark}

\begin{proposition}\label{thm:function_image_preimage_composition}
  For any function \( f: A \to B \) we have
  \begin{thmenum}
    \thmitem{thm:function_image_preimage_composition/preimage_of_image} If \( A' \subseteq A \), then \( A' \subseteq f^{-1}(f(A')) \) with equality holding if \( f \) is injective.
    \thmitem{thm:function_image_preimage_composition/image_of_preimage} If \( B' \subseteq B \), then \( f(f^{-1}(B')) \subseteq B' \) with equality holding if \( f \) is surjective.
  \end{thmenum}
\end{proposition}
\begin{proof}
  \SubProofOf{thm:function_image_preimage_composition/preimage_of_image} If \( f \) is injective, then the inclusion follows directly from \fullref{def:function_invertibility/injection/categorical}. If \( f \) is not injective, then it is possible for \( f^{-1}(f(A')) \) to contain points from \( A \) which are not in \( A' \) but have the same values under \( f \), hence a strict inclusion may hold.

  \SubProofOf{thm:function_image_preimage_composition/image_of_preimage} If \( f \) is surjective, then the inclusion follows directly from \fullref{def:function_invertibility/surjection/categorical}. If \( f \) is not surjective, however, \( f(f^{-1}(B')) \) will not include points from \( B' \setminus \img(f) \) and a strict inclusion may hold.
\end{proof}

\begin{proposition}\label{thm:function_image_properties}
  \hyperref[def:multi_valued_function/set_value]{Images of sets} under \( f: A \to B \) have the following basic properties:
  \begin{thmenum}
    \thmitem{thm:function_image_properties/monotonicity} If \( A_1 \subseteq A_2 \), then \( f(A_1) \subseteq f(A_2) \).

    \thmitem{thm:function_image_properties/union} For any \hyperref[def:indexed_family]{indexed family} \( \set{ A_k }_{k \in \mscrK} \subseteq A \) of subsets of \( A \) we have the equality
    \begin{equation}\label{eq:thm:function_image_properties/union}
      f\parens*{ \bigcup_{k \in \mscrK} A_k } = \bigcup_{k \in \mscrK} f(A_k).
    \end{equation}

    \thmitem{thm:function_image_properties/intersection} For any indexed family \( \set{ A_k }_{k \in \mscrK} \) of subsets of \( A \) we have the inclusion
    \begin{equation}\label{eq:thm:function_image_properties/intersection}
      f\parens*{ \bigcap_{k \in \mscrK} A_k } \subseteq \bigcap_{k \in \mscrK} f(A_k).
    \end{equation}

    Equality in \eqref{eq:thm:function_image_properties/intersection} holds if \( f \) is \hyperref[def:function_invertibility/injection]{injective}. If \( f \) is not injective, for example if both \( A \) and \( B \) are nonempty, \( A_1 \) and \( A_2 \) are disjoint subsets of \( A \) and \( f(A_1) = f(A_2) = B \), then
    \begin{equation*}
      f(A_1 \cap A_2) = f(\varnothing) = \varnothing \subsetneq f(A_1) \cap f(A_2) = B.
    \end{equation*}

    \thmitem{thm:function_image_properties/difference} For any two subsets \( A_1 \) and \( A_2 \) of \( A \) we have the inclusion
    \begin{equation}\label{eq:thm:function_image_properties/difference}
      f(A_1) \setminus f(A_2) \subseteq f(A_1 \setminus A_2).
    \end{equation}

    Equality in \eqref{eq:thm:function_image_properties/difference} holds if \( f \) is injective. If \( f \) is not injective, for example if \( A_1 \subsetneq A_2 \) but \( f(A_1) = f(A_2) \), then
    \begin{equation*}
      f(A_1) \setminus f(A_2) = \varnothing \subsetneq f(A_1 \setminus A_2).
    \end{equation*}
  \end{thmenum}

  Compare this result to the more well-behaved properties of \hyperref[thm:function_properties/preimage]{preimages} described in \fullref{thm:function_preimage_properties}.
\end{proposition}
\begin{proof}
  \SubProofOf{thm:function_image_properties/monotonicity} If \( x \in A_1 \), then \( x \in A_2 \) and hence \( f(x) \in f(A_2) \). Therefore \( f(A_1) \subseteq f(A_2) \).

  \SubProofOf{thm:function_image_properties/union} If \( x_0 \in A_{k_0} \) for some \( k_0 \in \mscrK \), clearly
  \begin{equation*}
    f(x_0) \in f(A_{k_0}) \subseteq \bigcup_{k \in \mscrK} f(A_k).
  \end{equation*}

  Therefore
  \begin{equation*}
    f\parens*{ \bigcup_{k \in \mscrK} A_k } \subseteq \bigcup_{k \in \mscrK} f(A_k).
  \end{equation*}

  Conversely, if \( y_0 \in f(A_{k_0}) \) for some \( k_0 \in \mscrK \), by \fullref{thm:function_image_properties/monotonicity} obviously
  \begin{equation*}
    y_0 \in f\parens*{ \bigcup_{k \in \mscrK} A_k }.
  \end{equation*}

  Therefore
  \begin{equation*}
    f\parens*{ \bigcup_{k \in \mscrK} A_k } \supseteq \bigcup_{k \in \mscrK} f(A_k).
  \end{equation*}

  Hence \eqref{eq:thm:function_image_properties/union} holds.

  \SubProofOf{thm:function_image_properties/intersection} If \( x_0 \) belongs to \( \bigcap_{k \in \mscrK} A_k \), then \( x_0 \) belongs to \( A_k \) for all \( k \in \mscrK \). It follows that \( f(x_0) \) belongs to \( f(A_k) \) for all \( k \in \mscrK \) and hence to their intersection. Therefore the inclusion \eqref{eq:thm:function_image_properties/intersection} holds.

  Now suppose that \( f \) is injective. Let \( y_0 \) be a point in the intersection \( \bigcap_{k \in \mscrK} f(A_k) \). We thus have \( y_0 \in f(A_k) \) for all \( k \in \mscrK \). Since \( f \) is injective, for each \( k \in \mscrK \) there exists a unique \( x_k \in A_k \) such that \( f(x_k) = y_0 \). Again because of injectivity of \( f \), all these elements are equal because \( f(x_k) = f(x_m) = y_0 \) for \( k, m \in \mscrK \). Hence \( y_0 \in f(\bigcap_{k \in \mscrK} A_k) \).

  Therefore the reverse inclusion in \eqref{eq:thm:function_image_properties/intersection} holds if \( f \) is injective.

  \SubProofOf{thm:function_image_properties/difference} If \( f(A_1) \setminus f(A_2) \) is empty, \eqref{eq:thm:function_image_properties/difference} obviously holds. Suppose that it is nonempty and let \( y_0 \in f(A_1) \setminus f(A_2) \).

  Then there exists a point \( x_0 \in A_1 \) such that \( f(x_0) = y_0 \). It cannot be that \( x_0 \in A_2 \) because otherwise \( y_0 = f(x_0) \in f(A_2) \), which would contradict our choice of \( y_0 \). Hence \( x_0 \in A_1 \setminus A_2 \) and \( y_0 \in f(A_1 \setminus A_2) \).

  Since \( y_0 \) was chosen arbitrarily, we conclude that the inclusion \eqref{eq:thm:function_image_properties/difference} holds.

  Conversely, suppose that \( f \) is injective. If \( f(A_1 \setminus A_2) \) is empty, by \eqref{eq:thm:function_image_properties/difference} the set \( f(A_1) \setminus f(A_2) \) is also empty and the converse holds.

  Now suppose that it is nonempty and let \( y_0 \in f(A_1 \setminus A_2) \). Then there exists a point \( x_0 \in A_1 \setminus A_2 \) such that \( f(x_0) = y_0 \). Furthermore, since \( f \) is injective, \( x_0 \) is the only preimage of \( y_0 \) and hence \( f(x_0) \in f(A_1) \setminus f(A_2) \), which proves the reverse inclusion in \eqref{eq:thm:function_image_properties/difference}.
\end{proof}

\begin{proposition}\label{thm:function_preimage_properties}
  Function \hyperref[thm:function_properties/preimage]{preimages} have the following basic properties:
  \begin{thmenum}
    \thmitem{thm:function_preimage_properties/monotonicity} If \( B_1 \subseteq B_2 \), then \( f^{-1}(B_1) \subseteq f^{-1}(B_2) \).

    \thmitem{thm:function_preimage_properties/union} For any \hyperref[def:indexed_family]{indexed family} \( \set{ B_k }_{k \in \mscrK} \subseteq B \) of subsets of \( B \) we have the equality
    \begin{equation}\label{eq:thm:function_preimage_properties/union}
      f^{-1}\parens*{ \bigcup_{k \in \mscrK} B_k } = \bigcup_{k \in \mscrK} f^{-1}(B_k).
    \end{equation}

    \thmitem{thm:function_preimage_properties/intersection} For any indexed family \( \set{ B_k }_{k \in \mscrK} \) of subsets of \( B \) we have the equality
    \begin{equation}\label{eq:thm:function_preimage_properties/intersection}
      f^{-1}\parens*{ \bigcap_{k \in \mscrK} B_k } = \bigcap_{k \in \mscrK} f^{-1}(B_k).
    \end{equation}

    \thmitem{thm:function_preimage_properties/difference} For any two subsets \( B_1 \) and \( B_2 \) of \( B \) we have the equality
    \begin{equation}\label{eq:thm:function_preimage_properties/difference}
      f^{-1}(B_1) \setminus f^{-1}(B_2) = f^{-1}(B_1 \setminus B_2).
    \end{equation}
  \end{thmenum}

  Compare this result to the less well-behaved properties of images described in \fullref{thm:function_image_properties}.
\end{proposition}
\begin{proof}
  \SubProofOf{thm:function_image_properties/monotonicity} Analogous to \fullref{thm:function_image_properties/monotonicity}.

  \SubProofOf{thm:function_image_properties/union} Analogous to \fullref{thm:function_image_properties/union}.

  \SubProofOf{thm:function_image_properties/intersection} If \( y_0 \) belongs to \( \bigcap_{k \in \mscrK} B_k \), then \( y_0 \) belongs to \( B_k \) for all \( k \in \mscrK \). It follows that \( f(y_0) \subseteq f^{-1}(B_k) \) for all \( k \in \mscrK \) and hence it is also a subset of their intersection. Therefore
  \begin{equation*}
    f^{-1} \parens*{ \bigcap_{k \in \mscrK} B_k } \subseteq \bigcap_{k \in \mscrK} f^{-1}(B_k).
  \end{equation*}

  Conversely, if \( x_0 \in \bigcap_{k \in \mscrK} f^{-1}(B_k) \), it belongs to \( f^{-1}(B_k) \) for all \( k \in \mscrK \). Clearly then \( f(x_0) \in B_k \) for all \( k \in \mscrK \) and thus \( f(x_0) \in \bigcap_{k \in \mscrK} B_k \). Hence by \fullref{thm:function_preimage_properties/monotonicity} we have
  \begin{equation*}
    f^{-1}(f(x_0))
    \subseteq
    f^{-1}\parens*{ \bigcap_{k \in \mscrK} B_k }.
  \end{equation*}

  Since \( x_0 \in f^{-1}(f(x_0)) \),
  \begin{equation*}
    x_0 \in f^{-1}\parens*{ \bigcap_{k \in \mscrK} B_k }.
  \end{equation*}

  Since \( x_0 \) was chosen arbitrarily from \( \bigcap_{k \in \mscrK} f^{-1}(B_k) \), we can conclude that
  \begin{equation*}
    \bigcap_{k \in \mscrK} f^{-1}(B_k) \in f^{-1}\parens*{ \bigcap_{k \in \mscrK} B_k }.
  \end{equation*}

  Hence \eqref{eq:thm:function_preimage_properties/intersection} holds.

  \SubProofOf{thm:function_preimage_properties/difference} If \( y_0 \in B_1 \setminus B_2 \), there exists a point \( x_1 \in B_1 \) such that \( f(x_1) = y_0 \). Aiming at a contradiction, suppose that there exists a point \( x_2 \in f^{-1}(B_2) \) such that \( f(x_2) = y_0 \). Then \( y_0 = f(x_1) = f(x_2) \) implies that \( f^{-1}(y_0) \subseteq f^{-1}(B_1) \cap f^{-1}(B_2) \). \Fullref{thm:function_preimage_properties/intersection} then in turn implies that \( f^{-1}(y_0) \subseteq f^{-1}(B_1 \cap B_2) \) and hence by \fullref{thm:function_image_properties/monotonicity}
  \begin{equation*}
    y_0 = f(f^{-1}(y_0)) \in f(f^{-1}(B_1 \cap B_2)) = B_1 \cap B_2,
  \end{equation*}
  which contradicts our choice of \( y_0 \). Since the choice of \( y_0 \in B_1 \setminus B_2 \), \( x_1 \in f^{-1}(y_0) \cap B_1 \) and \( x_2 \in f^{-1}(y_0) \cap B_2 \) was arbitrary, the obtained contradiction shows that
  \begin{equation*}
    f^{-1}(B_1 \setminus B_2) \subseteq f^{-1}(B_1) \setminus f^{-1}(B_2).
  \end{equation*}

  Conversely, we have
  \begin{equation*}
    f(f^{-1}(B_1) \setminus f^{-1}(B_2))
    \reloset {\eqref{eq:thm:function_image_properties/difference}} \subseteq
    f(f^{-1}(B_1 \setminus B_2))
    \reloset {\ref{thm:function_image_preimage_composition/image_of_preimage}} \subseteq
    B_1 \setminus B_2.
  \end{equation*}

  Hence
  \begin{equation*}
    f^{-1}(B_1) \setminus f^{-1}(B_2)
    \reloset {\ref{thm:function_image_preimage_composition/preimage_of_image}} \subseteq
    f^{-1}\parens[\Big]{ f\parens[\Big]{ f^{-1}(B_1) \setminus f^{-1}(B_2) } }
    \reloset {\eqref{thm:function_preimage_properties/monotonicity}} \subseteq
    f^{-1}(B_1 \setminus B_2).
  \end{equation*}
\end{proof}

\begin{definition}\label{def:cartesian_product}
  As promised in \fullref{rem:binary_vs_arbitrary_tuples}, we will define the Cartesian products for an arbitrary indexed family of sets.

  Let \( \set{ A_k }_{k \in \mscrK} \) be a nonempty \hyperref[def:indexed_family]{indexed family} of sets. Their \term{Cartesian product} is
  \begin{equation*}
    \bigtimes_{k \in \mscrK} A_k \coloneqq \set*{ f: \mscrK \to \bigcup_{k \in \mscrK} A_k \given* \qforall{m \in \mscrK} f(m) \in A_m }.
  \end{equation*}

  The definition also makes sense when any of the sets is empty because the product itself is then empty.

  Any element of the Cartesian product is called a \term{tuple}. In particular, finite tuples are bijective with the n-tuples defined via \hyperref[def:binary_cartesian_product]{Kuratowski pairs}.
\end{definition}

\begin{definition}\label{def:disjoint_union}
  The \term{disjoint union} of the \hyperref[def:indexed_family]{indexed family} \( \set{ A_k }_{k \in \mscrK} \) of nonempty sets is
  \begin{equation*}
    \bigsqcup_{k \in \mscrK} A_k \coloneqq \set[\Big]{ (k, x) \given k \in \mscrK \T{and} x \in A_k }.
  \end{equation*}
\end{definition}
