\subsection{Functions}\label{subsec:functions}

\begin{remark}\label{rem:function_definition}
  It is not straightforward to formalize the notion of correspondence between two values. We will reserve the term \term{mapping} for this informal notion and use \term{function} in the sense of \fullref{def:function}. There are several drawbacks of using material (that is, membership-based) set theory for defining functions:

  \begin{itemize}
    \item Mappings are often more general than what can be formalized, i.e. there exist correspondences between logical \hyperref[def:first_order_syntax/formula]{formulas} and between proper \hyperref[def:set_zfc]{classes} that cannot be defined in set theory without reaching contradictions.

    \item The ambient space often has an additional structure, e.g. algebraic or topological, that is not carried by functions. This leads to definitions such as \hyperref[def:first_order_homomorphism]{homomorphism} and \hyperref[def:isometry]{isometry}. This is a motivating example for the benefits of \hyperref[sec:category_theory]{category theory}, where the notion of \hyperref[def:category/C2]{morphism} is able to capture this additional structure (see \fullref{def:category_of_sets}).

    \item Several generalizations of set-theoretic functions are often used, e.g. \hyperref[def:function]{multivalued} or \hyperref[def:function/partial]{partial functions}, however most formalisms of set theory often only concern functions.

    \item Set-theoretic functions are often used in contexts where they do not refer to the intuitive notion of a mapping, e.g. for Cartesian \hyperref[def:cartesian_product]{products} or for indexed \hyperref[def:indexed_family]{families}.
  \end{itemize}
\end{remark}

\begin{definition}\label{def:function}
  A \term{multivalued function} from \( X \) to \( Y \) is simply a \hyperref[def:binary_relation]{binary relation} \( (F, X, Y) \), the difference being in how we treat multivalued functions and relations. We use the notation \( F: X \rightrightarrows Y \).

  \begin{thmenum}[series=def:function]
    \thmitem{def:function/single_valued} The \hyperref[def:function/multivalued]{multivalued function} \( F: X \rightrightarrows Y \) is called a \term{single-valued function} or simply a \term{function} if \( F(x) \) is a \hyperref[rem:singleton_sets]{singleton set} for each \( x \in X \). In this case, we write \( F: X \to Y \) rather than \( F: X \rightrightarrows Y \).

    Formally, the value of a single-valued function is an element of \( Y \) rather than a subset of \( Y \). \Fullref{rem:singleton_sets} allows us to ignore this distinction unless is would cause confusion.

    Note that single-valued functions are total by definition, which explains why there is no established terminology for the set \( X \).

    We denote the set of all functions from \( X \) to \( Y \) by \( \fun(X, Y) \). We may also use either \( \cat{Set}(X, Y) \) (which is consistent with \fullref{def:category_of_sets}) or by \( Y^X \) (which is consistent with \hyperref[def:cardinal_arithmetic]{cardinal arithmetic}). We abbreviate \( \fun(X, X) \) as \( \fun(X) \)

    \thmitem{def:function/multivalued} We only considered multivalued functions until now. In practice, \enquote{function} usually refers to single-valued functions and we call a function multivalued if it is not single-valued.

    \thmitem{def:function/partial} A \term{partial function} is a generalization of a single-valued function which we allow to not be total, i.e. we allow \( F(x) \) to be an empty set for some \( x \in X \).

    \thmitem{def:function/selection} A \term{selection} of a multivalued function \( F: X \rightrightarrows Y \) is a single-valued function \( f: X \to Y \) such that \( \gph(f) \subseteq \gph(F) \) (see \fullref{def:function/graph}).

    \thmitem{def:function/value} We call the set \( B \) the \term{value} of \( F \) at \( x \) if \( y \in B \iff (x, y) \in f \) and use the notation \( F(x) = B \).

    \thmitem{def:function/set_value} If \( A \subseteq X \), we define the value of \( F \) at \( A \), also called the \term{action} of \( F \) on \( A \) or the \term{image} of \( A \) under \( F \), as
    \begin{equation*}
      F(A) \coloneqq \cup_{a \in A} \{ F(a) \}.
    \end{equation*}

    Even if \( Y \) is a proper class, \( f(X) \) is a set by \fullref{def:set_zfc/A9}.

    \thmitem{def:function/argument} The notion \term{function arguments} is somewhat informal. If \( X = X_1 \times \cdots \times X_n \) is a finite Cartesian product, we denote the function \( F: X_1 \times \cdots \times X_n \rightrightarrows Y \) by \( F(x_1, \ldots, x_n) \). The variables \( x_1, \ldots, x_n \) in this \hyperref[def:first_order_syntax/term]{term} are called the function's \term{parameters} or \term{arguments} or even \term{independent variables}. In the latter terminology, we say that \( F \) itself is a \term{dependent variable}, depending on the independent variables \( x_1, \ldots, x_n \).

    Note that \( X \) may to be uniquely representable as a Cartesian product, however this does not cause confusion in practice.

    Even for single-argument functions, it is conventional to write \( F(x) \) rather than simply \( F \). This does not cause confusion in practice because it is usually clear from the context whether \( x \) is a \enquote{free} or \enquote{bound} variable. See \fullref{def:first_order_syntax/formula} for a formal justification.
  \end{thmenum}

  The following terminology is consistent with \fullref{def:binary_relation}:
  \begin{thmenum}[resume=def:function]
    \thmitem{def:function/graph} The \term{graph} \( \gph(F) \) of \( F \) is the relation \( F \) itself independent from \( X \) and \( Y \) as defined in \fullref{def:relation/graph}.

    \thmitem{def:function/domain} The \term{domain} \( \dom(F) \) of \( F \) is the set of all values for which \( \dom(F) \neq \varnothing \). This is consistent with \fullref{def:binary_relation/domain}.

    \thmitem{def:function/image} The \term{image} \( \img(F) \) is the set of all \( y \in Y \) that belong to the set \( F(x) \) for at least one \( x \in X \). This is consistent with \fullref{def:binary_relation/image}.

    \thmitem{def:function/range} The \term{range} \( \range(F) \) is the set \( Y \) as defined in \fullref{def:binary_relation/range}.

    \thmitem{def:function/composition} The \term{composition} \( G \circ F \) of two functions \( F: X \rightrightarrows Y \) and \( Y \rightrightarrows Z \) is the function
    \begin{equation*}
      [G \circ F](x) \coloneqq G(F(x)).
    \end{equation*}

    This is consistent with \fullref{def:binary_relation/composition}.

    Note that the composition of single-valued functions is single-valued (see \fullref{def:category_of_sets}).
  \end{thmenum}

  The following terminology is not consistent with \fullref{def:binary_relation}:
  \begin{thmenum}[resume=def:function]
    \thmitem{def:function/total} The term \term{total multivalued function} means that \( \dom(F) = X \), that is, that \( F(x) \neq \varnothing \) for all \( x \in X \). This is not to be confused with \fullref{def:binary_relation/total}.

    \thmitem{def:function/arity} The \term{arity} of a function is its number of \hyperref[def:function/arity]{arguments}. This should not to be confused with \fullref{def:relation/arity}.

    \thmitem{def:function/inverse} The \term{inverse} \( F^{-1}: Y \rightrightarrows X \) of a multivalued function \( F: X \rightrightarrows Y \) is its \hyperref[def:binary_relation/converse]{converse} (rather than \hyperref[def:binary_relation/inverse]{inverse}) relation.

    \thmitem{def:function/diagonal} The function corresponding to the \hyperref[def:binary_relation/diagonal]{diagonal relation} is called the \term{identity}. Instead, the (single-valued) \term{diagonal function} \( f: \mscrX \to \mscrX^2 \) is defined as \( f(x) \coloneqq (x, x) \).
  \end{thmenum}

  We define some additional terminology:
  \begin{thmenum}[resume=def:function]
    \thmitem{def:function/involution} If \( F = F^{-1} \), we say that \( F \) is an \term{involution}.

    \thmitem{def:function/preimage} For a concrete set \( B \subseteq Y \), its \term{large preimage} or simply \term{preimage} under \( F: X \to Y \) is the value of \( B \) under the inverse function \( F^{-1}: Y \rightrightarrows X \). For a single value \( y \in Y \), we call \( F^{-1}(y) \) the \term{fiber} of \( y \) under \( F \).

    We define its \term{small preimage} as \( F_{-1}(B) \coloneqq \{ x \in X \colon F(x) \subseteq B \} \)

    \thmitem{def:function/extension} Let \( X \) and \( Y \) be sets and let \( A \subseteq X \). For the multivalued functions \( F: A \rightrightarrows Y \) and \( G: X \rightrightarrows Y \), we say that \( G \) is an \term{extension} of \( F \) to \( X \) and that \( F \) is a \term{restriction} of \( G \) to \( A \).

    \thmitem{def:function/superposition} Although the terms \enquote{composition} and \enquote{superposition} are used interchangeably (see \cite[44]{Enderton1977Sets} and \cite[\textnumero 25]{Фихтенгольц1968Том1}), \enquote{superposition} usually refers to the following special case of composition:

    If we are given the functions \( F_k: X \rightrightarrows Y_k, k = 1, \ldots, n \) and \( G: Y_1 \times \cdots \times Y_n \rightrightarrows Z \), their \term{superposition} \( H: X \rightrightarrows Z \) is
    \begin{equation*}
      H(x) \coloneqq G(F_1(x), \ldots, F_n(x)).
    \end{equation*}
  \end{thmenum}
\end{definition}

\begin{definition}\label{def:function_invertibility}(Compare with \fullref{def:morphism_invertibility})
  We list equivalent conditions for three types of invertibility:
  \begin{thmenum}
    \thmitem{def:function_invertibility/injection} \( f \) is called \term{injective}, \term{left-invertible}, \term{one-to-one} if any of the following equivalent conditions hold:
    \begin{thmenum}
      \thmitem{def:function_invertibility/injection/points} Different points in \( X \) have different images under \( f \).
      \thmitem{def:function_invertibility/injection/preimage} The preimage of any point in \( Y \) is either empty or a singleton.
      \thmitem{def:function_invertibility/injection/monomorphism} There exists a function \( g: Y \to X \) such that \( g \circ f = \id_X \).
      \thmitem{def:function_invertibility/injection/inverse} The inverse is a single-valued partial function.
    \end{thmenum}

    We sometimes use the \hyperref[def:morphism_invertibility/monomorphism]{monomorphism} notation \( f: X \hookrightarrow Y \).

    \thmitem{def:function_invertibility/surjection} \( f \) is called \term{surjective}, \term{right-invertible} or \term{onto} if any of the equivalent conditions hold:
    \begin{thmenum}
      \thmitem{def:function_invertibility/surjection/points} Each point in \( Y \) is the image of at least one point in \( X \).
      \thmitem{def:function_invertibility/surjection/image} The image of \( f \) equals the range of \( f \).
      \thmitem{def:function_invertibility/surjection/epimorphism} There exists a function \( g: Y \to X \) such that \( f \circ g = \id_Y \).
      \thmitem{def:function_invertibility/surjection/inverse} The inverse is a total multivalued function.
    \end{thmenum}

    We sometimes use the \hyperref[def:morphism_invertibility/epimorphism]{epimorphism} notation \( f: X \twoheadrightarrow Y \).

    \thmitem{def:function_invertibility/bijection} \( f \) is called \term{bijective} or simply \term{invertible} if any of the equivalent conditions hold:
    \begin{thmenum}
      \thmitem{def:function_invertibility/bijection/direct} it is both injective and surjective.
      \thmitem{def:function_invertibility/bijection/points} each point in \( Y \) is the image of exactly one point in \( X \).
      \thmitem{def:function_invertibility/bijection/preimage} the preimage of any point in \( Y \) is a singleton.
      \thmitem{def:function_invertibility/bijection/isomorphism} there exists a function \( g: Y \to X \) such that both \( g \circ f = \id_X \) and \( f \circ g = \id_Y \).
      \thmitem{def:function_invertibility/bijection/inverse} the inverse is a single-valued total function.
    \end{thmenum}

    We sometimes use the \hyperref[def:morphism_invertibility/isomorphism]{isomorphism} notation \( f: X \cong Y \). See also \fullref{def:equinumerous_sets}.
  \end{thmenum}
\end{definition}

\begin{definition}\label{def:endofunction}
  A function from a set to itself is called an \term{endofunction}.
\end{definition}

\begin{definition}\label{def:currying}
  This notation is abused in practice as long as it does not cause confusion.

  Given the two-argument function \( f: A \times B \to C \), we may define
  \begin{equation*}
    g(y)(x) \coloneqq f(x, y).
  \end{equation*}

  Here \( g \) is itself an operator from \( A \) to \( \cat{Set}(B, C) \). This is called \term{currying}, although the latter term is more specific and refers to lambda calculus.

  We often wish to \enquote{fix} some value \( x \), i.e. bind it using modified variable \hyperref[def:first_order_valuation/variable_assignment]{assignment}, so that \( g(y): B \to C \) is well-defined.
\end{definition}

\begin{proposition}\label{thm:function_image_properties}
  Functions images have the following basic properties (compare to \fullref{thm:function_preimage_properties}):
  \begin{thmenum}
    \thmitem{thm:function_image_properties/monotonicity} If \( A \subseteq B \), then \( f(A) \subseteq f(B) \).

    \thmitem{thm:function_image_properties/union} \( f(\bigcup_{k \in \mscrK} X_k) = \bigcup_{k \in \mscrK} f(X_k) \).

    \thmitem{thm:function_image_properties/intersection} \( f(\bigcap_{k \in \mscrK} X_k) \subseteq \bigcap_{k \in \mscrK} f(X_k) \) with equality holding if \( f \) is injective.

    \thmitem{thm:function_image_properties/difference} \( f(A \setminus B) \subseteq f(A) \setminus f(B) \) with equality holding if \( f \) is surjective.
  \end{thmenum}
\end{proposition}
\begin{proof}
  \SubProofOf{thm:function_image_properties/monotonicity} If \( x_0 \in A \), then \( x_0 \in B \) and hence \( f(x_0) \in f(B) \). Therefore \( f(A) \subseteq f(B) \).

  \SubProofOf{thm:function_image_properties/union} If \( x_0 \in X_{k_0} \) for some \( k_0 \in \mscrK \), clearly \( f(x_0) \in f(X_{k_0}) \subseteq \bigcup_{k \in \mscrK} f(X_k) \). Therefore \( f(\bigcup_{k \in \mscrK} X_k) \subseteq \bigcup_{k \in \mscrK} f(X_k) \).

  Conversely, if \( y_0 \in f(X_{k_0}) \) for some \( k_0 \in \mscrK \), by \fullref{thm:function_image_properties/monotonicity} obviously \( y_0 \in f\left( \bigcup_{k \in \mscrK} X_k \right) \). Therefore \( f(\bigcup_{k \in \mscrK} X_k) \supseteq \bigcup_{k \in \mscrK} f(X_k) \).

  \SubProofOf{thm:function_image_properties/intersection} If \( x_0 \in \bigcap_{k \in \mscrK} X_{k} \), then \( x_0 \in X_k \) for all \( k \in \mscrK \). We have \( f(x_0) \in f(X_k) \) for all \( k \in \mscrK \), therefore \( f(\bigcap_{k \in \mscrK} X_k) \subseteq \bigcap_{k \in \mscrK} f(X_k) \).

  Conversely, if \( f \) is injective and \( y_0 \in f(X_k) \) for all \( k \in \mscrK \), then there exists a unique \( x_0 \in \bigcap_{k \in X_k} \) such that \( f(x_0) = y_0 \). Therefore \( f(\bigcap_{k \in \mscrK} X_k) \supseteq \bigcap_{k \in \mscrK} f(X_k) \).

  \SubProofOf{thm:function_image_properties/difference} If \( x_0 \in A \) and \( x_0 \not\in B \), then \( f(x_0) \in f(A) \setminus f(B) \). Therefore \( f(A \setminus B) \subseteq f(A) \setminus f(B) \).

  Conversely, suppose that \( f \) is surjective. For \( y_0 \in f(A) \setminus f(B) \) there exists a \( x_0 \in A \) such that \( f(x_0) = y_0 \in f(A) \setminus B \). Since \( y_0 \not\in f(B) \), and, by surjectivity, \( y_0 \) has the preimage of \( y_0 \) has only one member \( x_0 \), we conclude that \( x_0 = \not\in f(B) \). Therefore \( f(A \setminus B) \supseteq f(A) \setminus f(B) \).
\end{proof}

\begin{proposition}\label{thm:function_preimage_properties}
  Functions \hyperref[def:function/preimage]{preimages} have the following basic properties (compare to \fullref{thm:function_image_properties}):
  \begin{thmenum}
    \thmitem{thm:function_preimage_properties/monotonicity} If \( A \subseteq B \), then \( f^{-1}(A) \subseteq f^{-1}(B) \).

    \thmitem{thm:function_preimage_properties/union} \( f^{-1}(\bigcup_{k \in \mscrK} Y_k) = \bigcup_{k \in \mscrK} f^{-1}(Y_k) \).

    \thmitem{thm:function_preimage_properties/intersection} \( f^{-1}(\bigcap_{k \in \mscrK} Y_k) = \bigcap_{k \in \mscrK} f^{-1}(Y_k) \).

    \thmitem{thm:function_preimage_properties/difference} \( f^{-1}(A \setminus B) = f^{-1}(A) \setminus f^{-1}(B) \).
  \end{thmenum}
\end{proposition}
\begin{proof}
  \SubProofOf{thm:function_image_properties/monotonicity} Analogous to \fullref{thm:function_image_properties/monotonicity}.

  \SubProofOf{thm:function_image_properties/union} Analogous to \fullref{thm:function_image_properties/union}.

  \SubProofOf{thm:function_image_properties/intersection} If \( y_0 \in \bigcap_{k \in \mscrK} Y_{k} \), then \( y_0 \in Y_k \) for all \( k \in \mscrK \). We have \( f^{-1}(y_0) \in f^{-1}(Y_k) \) for all \( k \in \mscrK \), therefore \( f^{-1}(\bigcap_{k \in \mscrK} Y_k) \subseteq \bigcap_{k \in \mscrK} f^{-1}(Y_k) \).

  Conversely, if \( x_0 \in f^{-1}(Y_k) \) for all \( k \in \mscrK \), then, since \( f \) is a function, there exists a unique \( y_0 \in \bigcap_{k \in Y_k}Y_k \) such that \( f^{-1}(y_0) = x_0 \). Therefore \( f^{-1}(\bigcap_{k \in \mscrK} X_k) \supseteq \bigcap_{k \in \mscrK} f^{-1}(X_k) \).

  \SubProofOf{thm:function_image_properties/difference} If \( y_0 \in A \) and \( y_0 \not\in B \), then \( f^{-1}(y_0) \in f^{-1}(A) \setminus f^{-1}(B) \). Therefore \( f^{-1}(A \setminus B) \subseteq f^{-1}(A) \setminus f^{-1}(B) \).

  Conversely, for \( x_0 \in f^{-1}(A) \setminus f^{-1}(B) \), since \( f \) is a function, there exists a \( y_0 \in A \) such that \( f^{-1}(y_0) = x_0 \in f^{-1}(A) \setminus f^{-1}(B) \). Since \( x_0 \not\in f^{-1}(B) \), we conclude that \( x_0 = \not\in B \). Therefore \( f^{-1}(A \setminus B) \supseteq f^{-1}(A) \setminus f^{-1}(B) \).
\end{proof}

\begin{proposition}\label{thm:function_image_preimage_composition}
  \hfill
  \begin{thmenum}
    \thmitem{thm:function_image_preimage_composition/image_first} \( A \subseteq f^{-1}(f(A)) \) with equality holding if \( f \) is injective.
    \thmitem{thm:function_image_preimage_composition/preimage_first} \( f(f^{-1}(A)) \subseteq A \) with equality holding if \( f \) is surjective
  \end{thmenum}
\end{proposition}
\begin{proof}
  \SubProofOf{thm:function_image_preimage_composition/image_first} Equality obviously holds unless the image \( f(A) \) of \( A \) contains other points except those in \( A \). In this case, \( f^{-1}(f(A)) \) may contain those points in addition to the points of \( A \). If \( f \) is injective, however, no such additional points are possible and equality indeed holds.

  \SubProofOf{thm:function_image_preimage_composition/preimage_first} Equality obviously holds unless \( A \) contains points that do not belong to the image \( \imag f \). If \( f \) is surjective, however, all point in \( A \) have preimages and equality indeed holds.
\end{proof}

\begin{definition}\label{def:indexed_family}
  When considering finite families of sets, it is enough to consider n-tuples. For example, given sets \( X_1, \ldots, X_n \), we can think of the family \( \{ X_k \}_k \) as the ordered tuple
  \begin{equation*}
    (X_1, \ldots, X_n)
  \end{equation*}
  where the \( k \)-th coordinate of the tuple gives us the \( k \)-th set of the family.

  This approach has two flaws:
  \begin{itemize}
    \item The family \term{must} be ordered since the natural numbers are ordered. Families of sets often have no obvious ordering.
    \item The family \term{must} be at most countable.
  \end{itemize}

  A more natural approach to indexed families is given by functions. We choose an arbitrary set \( \mscrK \), called the \term{index set}. Every function \( f: \mscrK \to \mathcal C \) from \( \mscrK \) into some class \( \mathcal C \) of sets is then called an \term{indexed family}. The function \( f \) maps every element \( k \) of \( \mscrK \) into a set \( X_k \coloneqq f(k) \). For convenience, this family is denoted as
  \begin{equation*}
    \{ X_k \}_{k \in \mscrK}.
  \end{equation*}

  We will write \( \{ X_k \}_{k \in \mscrK} \subseteq \mathcal{C} \), despite the net actually being an \( \mscrK \)-shaped generalized \hyperref[def:generalized_element]{element} of \( \mathcal{C} \) rather than a subset. See \fullref{rem:indexed_family_notation} for further discussion of the notation.

  A more general framework than indexed families that also considers relations between the family's elements is given by diagrams in category \hyperref[def:categorical_diagram]{theory}.
\end{definition}

\begin{example}\label{ex:indexed_families}
  \hfill
  \begin{thmenum}
    \item Every n-tuple \( (x_1, \ldots, x_n) \) is an indexed family with domain \( \mscrK = \{ 1, \ldots, n \} \).

    \item An important corner case is when \( \mscrK \) is the empty set. Since the only possible indexing function is then the empty function, we simply say that the resulting family is empty.

    \item In continuous stochastic processes, it is convenient to consider families of random variables \( \{ X_t \}_{t \geq 0} \) indexed by \( \mscrK = \BbbR^+ \). The indexing parameter is often denoted by \( t \geq 0 \) is often interpreted as time.

    \item An \( n \times m \) \hyperref[def:array/matrix]{matrix} \( A = \{ a_{i,j} \} \) is a family of scalars indexed by the unordered set \( \mscrK = \{ 1, \ldots, n \} \times \{ 1, \ldots, m \} \).

    \item \hyperref[def:topological_net]{Nets} is topology are indexed families where the domain is a directed \hyperref[def:directed_set]{set}.
  \end{thmenum}
\end{example}

\begin{definition}\label{def:sequence}
  A \term{sequence} \( \{ X_k \}_{k=1}^\infty \) is an indexed family with domain \( \mscrK = \BbbN \). Sometimes finite \( n \)-tuples are referred to as \term{finite sequences}, in which case the usual sequences are referred to as \term{infinite sequences}. See \fullref{def:topological_net}.

  We say that \( \{ X_{k_m} \}_{k=1}^\infty \) is a \term{subsequence} of \( \{ X_k \}_{k=1}^\infty \) if the sequence \( \{ k_m \}_{k=1}^\infty \) of positive integers is strictly monotone.

  Subsequences of \( \{ X_k \}_{k=1}^\infty \) are usually denoted by adding another index as a subscript, i.e. \( \{ x_{k_m} \}_{k=1}^\infty \).
\end{definition}

\begin{remark}\label{rem:indexed_family_notation}
  Since we denote \hyperref[def:cartesian_product]{tuples} as \( (x_1, \ldots, x_n) \), it is consistent to denote indexed \hyperref[def:indexed_family]{families} by
  \begin{equation*}
    ( X_k )_{k \in \mscrK}
  \end{equation*}
  instead of
  \begin{equation*}
    \{ X_k \}_{k \in \mscrK}.
  \end{equation*}

  This is actually done when we want to enumerate elements of a sequence, e.g. see \fullref{def:polynomial}.

  In general, however, we prefer the latter notation because
  \begin{equation*}
    \left\{ \log \left( f^{(n)}(x_k) \right) \right\}_{k=1}^\infty.
  \end{equation*}
  is both more conventional (in analysis) and more aesthetically pleasing than
  \begin{equation*}
    \left( \log \left( f^{(n)}(x_k) \right) \right)_{k=1}^\infty
  \end{equation*}

  The difference may be more visible in simpler cases like
  \begin{balign*}
    (\sin(k))_{k \in \mscrK}
     &  &
    \{\sin(k)\}_{k \in \mscrK}.
  \end{balign*}
\end{remark}

\begin{definition}\label{def:family_of_functions_separates_points}
  Let \( \mathcal{F} \) be a family of functions between the sets \( A \) and \( B \). We say that \( \mathcal{F} \) \term{separates points} if for every two points \( x, y \in A \) there exists a function \( f \in \mathcal{F} \) such that \( f(x) \neq f(y) \).
\end{definition}

\begin{definition}\label{def:symmetric_function}
  Fix arbitrary sets \( X \) and \( Y \). A function \( f: X \times X \to Y \) is called \term{symmetric} if, for all \( x, y \in X \), we have
  \begin{equation*}
    f(x, y) = f(y, x).
  \end{equation*}

  Symmetric functions should not be confused with \hyperref[def:derived_relations/symmetric]{symmetric relations}.
\end{definition}

\begin{definition}\label{def:fixed_point}
  Given a function \( f: A \to B \), we call \( x \in A \) a \term{fixed point} of \( f \) if \( x = f(x) \).
\end{definition}
