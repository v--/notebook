\subsection{Baire spaces}\label{subsec:baire_spaces}

\begin{remark}\label{remark:baire_categories}
  René-Louis Baire introduced the concept of \Def{Baire categories} in 1899, almost 50 years before Samuel Eilenberg and Saunders MacLane introduced \Def{categories} in \cite{MacLane1945} (see \fullref{sec:category_theory} for the latter).

  Unfortunately, topology utilizes both concepts so the word \enquote{category} should be used with caution. To circumvent this, we use alternative terminology for Baire categories.
\end{remark}

\begin{definition}\label{def:meager_set}\cite[definition 2.1]{Rudin1991}
  Any countable union of nowhere dense sets\Tinyref{def:topologically_dense_set/nowhere_dense} is called \Def{meager} or a \Def{first category set} (see \fullref{remark:baire_categories} for terminology). If a set is not meager, we call it \Def{nonmeager} or a \Def{second category set}.
\end{definition}

\begin{proposition}\label{thm:meager_set_properties}\cite[43]{Rudin1991}
  Meager sets\Tinyref{def:meager_set} have the following basic properties (compare to \fullref{thm:nowhere_dense_properties}):
  \begin{propenum}
    \DItem{thm:meager_set_properties/union} A countable union of meager sets is meager.
    \DItem{thm:meager_set_properties/subset} A subset of a meager set is meager.
    \DItem{thm:meager_set_properties/homeomorphism} The homeomorphic\Tinyref{def:homeomorphism} image of a set \( A \) is meager if and only if \( A \) itself is meager.
  \end{propenum}
\end{proposition}
\begin{proof}\mbox{}
  \begin{description}
    \RItem{thm:meager_set_properties/union} Follows from \fullref{thm:countable_union_of_countable_sets}.
    \RItem{thm:meager_set_properties/subset} Fix a meager set \( A \) and let \( B \subseteq A \). Then \( A = \bigcup_{k=1}^\infty A_k \) for some nowhere dense sets \( A_1, A_2, \ldots \). By \fullref{thm:nowhere_dense_properties/subset}, the sets \( A_1 \cap B, A_2 \cap B, \ldots \) are also nowhere dense. But
    \begin{equation*}
      B
      =
      A \cap B
      =
      \left(\bigcup_{k=1}^\infty A_k \right) \cap B
      \overset {\ref{thm:subsets_form_boolean_algebra}} =
      \bigcup_{k=1}^\infty (A_k \cap B).
    \end{equation*}

    Therefore \( B \) is also nowhere dense.

    \RItem{thm:meager_set_properties/homeomorphism}\mbox{}
    \begin{description}
      \ImpliedBy If \( A \) is meager, any homeomorphic image of \( A \) is meager by \fullref{thm:function_image_properties/union} and \fullref{thm:nowhere_dense_properties/homeomorphism}.
      \Implies If \( f: X \to Y \) is a homeomorphism and \( f(A) \) is meager for some \( A \subseteq X \), then \( A \) is the homeomorphic image of the meager set \( f(A) \) under \( f^{-1} \) and is thus meager.
    \end{description}
  \end{description}
\end{proof}

\begin{definition}\label{def:baire_space}
  A topological space is called a \Def{Baire space} if any of the following equivalent conditions hold:
  \begin{defenum}
    \DItem{def:baire_space/meager} Every nonempty open set is nonmeager.
    \DItem{def:baire_space/dense} A countable intersection of dense sets is dense.
  \end{defenum}
\end{definition}
\begin{proof}
  \Iff[def:baire_space/meager][def:baire_space/dense] Follows from \fullref{thm:nowhere_dense_properties/complement_dense} and \fullref{thm:de_morgans_laws}.
\end{proof}

\begin{proposition}\label{thm:open_subspace_of_baire_space_is_baire}
  Every open subspace of a Baire space\Tinyref{def:baire_space} is a Baire space.
\end{proposition}
\begin{proof}
  Let \( (X, \CT) \) be a Baire space and let \( (X', \CT_{X'}) \) be an open subspace\Tinyref{def:topological_subspace} with the canonical embedding \( \iota: X' \to X \). The proposition holds vacuously if \( X' = \varnothing \) so we assume that \( X' \neq \varnothing \).

  Note that \( \iota \) is continuous by definition, however it is also an open map because if \( U \in \CT_{X'} \), then \( \iota(U) = U \cap X' \) is open in \( X \) as the intersection of two open sets. Therefore it is a homeomorphic embedding and by \fullref{thm:meager_set_properties/homeomorphism}, \( U \) is meager if and only if \( \iota(U) \) is meager. Since \( X \) is a Baire space, \( \iota(U) \) is not meager and hence \( U \) is also not meager.

  We showed that every nonempty open set \( U \in \CT_{X'} \) is nonmeager, therefore \( X' \) is a Baire space.
\end{proof}

\begin{theorem}[Baire category theorem]\label{thm:baire_category_theorem}\cite{Rudin1991}
  \begin{thmenum}
    \DItem{thm:baire_category_theorem/metric} Complete metric spaces\Tinyref{def:complete_metric_space} are Baire spaces\Tinyref{def:baire_space}.
    \DItem{thm:baire_category_theorem/compact} Locally compact\Tinyref{def:locally_compact_space} Hausdorff\Tinyref{def:separation_axioms/T2}spaces are Baire spaces.
  \end{thmenum}
\end{theorem}
