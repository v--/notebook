\begin{definition}\label{def:monoidal_category}\cite{MacLane1994}[158]
  A \uline{monoidal category} is a generalization of A monoid from sets to categories. Formally, it is a category $\Bold M$ along with
  \begin{itemize}
    \item a functor $\otimes: C \times C \to C$
    \item an identity object $1 \in \Bold M$
    \item natural transformations
    \begin{itemize}
      \item[] $\alpha: ((-) \otimes (-)) \otimes (-) \cong (-) \otimes ((-) \otimes (-))$
      \item[] $\lambda: 1 \times (-) \cong (-)$
      \item[] $\rho: (-) \times 1 \cong (-)$
    \end{itemize}
  \end{itemize}
  such that
  \begin{defenum}
    \item for every object $A \in \Bold M$,
    \begin{align*}
      1 \otimes A \overset {\lambda_a} \cong A
      \\
      A \otimes 1 \overset {\rho_a} \cong A
    \end{align*}

    \item for all objects $A, B, C \in \Bold M$,
    \begin{align*}
      A \otimes (B \otimes C) \overset {\alpha_{A,B,C}} \cong (A \otimes B) \otimes C
    \end{align*}

    \item the following diagram commutes for all objects $A, B, C, D \in \Bold M$
    \begin{center}
      \begin{tikzcd}
                                                                                                                                   & (A \otimes B) \otimes (C \otimes D) \arrow[rd, "{\alpha_{(A \otimes B), C, D}}"] &                                                                                 \\
        A \otimes (B \otimes (C \otimes D)) \arrow[ru, "{\alpha_{A,B,(C \otimes D)}}"] \arrow[dd, "{\Id \otimes \alpha_{B,C,D}}"'] &                                                                                  & ((A \otimes B) \otimes C) \otimes D                                             \\
                                                                                                                                   &                                                                                  &                                                                                 \\
        A \otimes ((B \otimes C) \otimes D) \arrow[rr, "{\alpha_{A,(B \otimes C), D}}"]                                            &                                                                                  & (A \otimes (B \otimes C)) \otimes D \arrow[uu, "{\alpha_{A,B,C} \otimes \Id}"']
      \end{tikzcd}
    \end{center}

    \item the following diagram commutes for all objects $A, B \in \Bold M$
    \begin{center}
      \begin{tikzcd}
        A \otimes (1 \otimes B) \arrow[rd, "\Id \otimes \lambda_b"'] \arrow[rr, "{\alpha_{A,1,B}}"] &             & (A \otimes 1) \otimes B \arrow[ld, "\rho_a \otimes \Id"] \\
                                                                                                    & A \otimes B &
      \end{tikzcd}
    \end{center}
  \end{defenum}

  If the natural isomorphisms $\alpha$, $\lambda$ and $\rho$ are identities, we say that $\Bold M$ is a \uline{strict monoidal category}.
\end{definition}

\begin{definition}\label{def:enriched_category}\cite{MacLane1994}[180],\cite{nLab:enriched_category}
  Enriched categories provide additional structure to the morphism sets of locally small categories. The definition can be compared with\cref{def:category}. We say that $\Bold C$ is an \uline{enriched category over the small monoidal category $\Bold M$} if
  \begin{itemize}
    \item there exists a class of objects, where the membership is denoted as $A \in \Bold C$
    \item for each object $A \in \Bold C$, there exists an \uline{identity morphism} $j_A: 1 \to \Bold{C}(A, A)$
    \item for each pair of objects $A, B \in \Bold C$, there exists an object $\Bold{C}(A, B)$ in $\Bold M$
    \item for each triple of objects $A, B, C \in \Bold C$, there exists a \uline{composition morphism} in $\Bold M$:
    \begin{align*}
      \circ_{A,B,C}: {\Bold C}(B, C) \times {\Bold C}(A, B) \to {\Bold C}(A, C).
    \end{align*}
  \end{itemize}
  such that
  \begin{defenum}
    \item the following diagram commutes for all objects $A, B, C, D \in \Bold C$
    \begin{center}
      \begin{tikzcd}
         \Bold{C}(A, D)                                                                                                               &                                                                                                           \\
         \Bold{C}(C, D) \otimes \Bold{C}(A, C) \arrow[u, "\circ_{A,C,D}"]                                                             & \Bold{C}(B, D) \otimes \Bold{C}(A, B) \arrow[lu, "\circ_{A,B,D}"']                                        \\
                                                                                                                                      &                                                                                                           \\
         \Bold{C}(C, D) \otimes (\Bold{C}(B, C) \otimes \Bold{C}(A, B)) \arrow[r, "\alpha"] \arrow[uu, "{\Id \otimes \circ_{A,B,C}}"] & (\Bold{C}(C, D) \otimes \Bold{C}(B, C)) \otimes \Bold{C}(A, B) \arrow[uu, "{\circ_{B,C,D} \otimes \Id}"']
      \end{tikzcd}
    \end{center}

    \item the following diagram commutes for all objects $A, B \in \Bold M$

    \begin{center}
      \begin{tikzcd}
         \Bold{C}(B, B) \otimes \Bold{C}(A, B) \arrow[rd, "\circ_{A,B,B}"] &                & \Bold{C}(A, B) \otimes \Bold{C}(A, A) \arrow[ld, "\circ_{A,A,B}"'] \\
                                                                           & \Bold{C}(A, B) &                                                                    \\
         1 \otimes \Bold{C}(A, B) \arrow[ru, "\lambda"] \arrow[uu, "j"]    &                & \Bold{C}(A, B) \times 1 \arrow[uu, "j"'] \arrow[lu, "\rho"]
      \end{tikzcd}
    \end{center}
  \end{defenum}

  In order for monoidal categories to generalize categories, formally we need a functor $U: \Bold{M} \to \Bold{Set}$ so that morphism objects $\Bold C(A, B)$ become sets $U(\Bold C(A, B))$. This is usually defined implicitly, for example $U(\Bold C(A, B)) \coloneqq \Bold{M}(1, C(A, B))$.
\end{definition}

\begin{definition}\label{def:preadditive_category}\cite{MacLane1994}[28]
  A \uline{preadditive category $\Bold C$} is any category enriched over the category $\Bold{Ab}$ of abelian groups, such that composition
  \begin{align*}
    \circ_{A,B,C}: \Bold{Ab}(B, C) \times \Bold{Ab}(A, B) \to \Bold{Ab}(A, C)
  \end{align*}
  is bilinear, e.g. given group homomorphisms $f, f': A \to B$ and $g, g': B \to C$, we have
  \begin{align*}
    (g + g') \circ (f + f') = g \circ f + g \circ f' + g' \circ f + g' \circ f'.
  \end{align*}
\end{definition}

\begin{definition}\label{def:categorical_biproduct}\cite{MacLane1994}[189]
  Let $\Bold C$ be a preadditive category and let $X$ and $Y$ be objects in $\Bold C$. We say that $X \times Y$ is their \uline{biproduct} if there exist morphisms
  \begin{align*}
    i_X: X \to (X \times Y), && i_Y: Y \to (X \times Y), \\
    p_X: (X \times Y) \to X, && p_Y: (X \times Y) \to Y,
  \end{align*}
  such that
  \begin{align*}
    &\Id_X = p_X \circ i_X, \\
    &\Id_Y = p_Y \circ i_Y, \\
    &\Id_{X \times Y} = p_X \circ i_X + p_Y \circ i_Y
  \end{align*}
\end{definition}

\begin{definition}\label{def:abelian_category}\cite{MacLane1994}[196]
  A preadditive category $\Bold C$ is called an \uline{abelian category} if the all of following hold:
  \begin{defenum}
    \item $\Bold C$ has a zero object $0$, i.e. $\Bold C$ is a pointed category (see \cref{def:initial_final_objects})
    \item $\Bold C$ has all binary biproducts (see \cref{def:categorical_biproduct})
    \item $\Bold C$ has a kernel and a cokernel for every morphism (see \cref{def:categorical_kernel}; this requires $\Bold C$ to have all zero morphisms)
    \item every monomorphism is a kernel and every epimorphism is a cokernel (see \cref{def:morphism_invertability})
  \end{defenum}
\end{definition}

\begin{proposition}\label{def:abelian_category_morphism_factorization}\cite{MacLane1994}[proposition 8.3.1]
  In an abelian category $\Bold C$, every morphism $f: A \to B$ has a factorization $f = \Img f \circ \Coimg f$, where
  \begin{itemize}
    \item $\Img f \coloneqq \Ker(\Coker f: B \to C_1): L_1 \to B$ is a monomorphism
    \item $\Coimg f \coloneqq \Coker(\Ker f: L_2 \to A): A \to C_2$ is an epimorphism
  \end{itemize}
  Here $L_1$ and $L_2$ are the limit vertices and $C_1$ and $C_2$ are the colimit vertices as in~\cref{def:categorical_kernel}. Necessarily $L_1 \cong C_2$.
\end{proposition}

\begin{definition}\label{def:exact_morphism_pair}\cite{MacLane1994}[196]
  In an abelian category $\Bold C$, a composable pair of morphisms $f: A \to B$ and $g: B \to C$ is said to be \uline{exact at $B$} if $\Img f \equiv \Ker g$ as subobjects of $B$ (or, equivalently, $\Coker f \equiv \Coimg g$; see \cref{def:categorical_subject}).
\end{definition}

\begin{definition}\label{def:short_exact_sequence}\cite{MacLane1994}[196]
  In an abelian category $\Bold C$, the diagram
  \begin{equation}\label{def:short_exact_sequence/diagram}
    \begin{tikzcd}[baseline=(current bounding box.center)]
      0 \arrow[r, "\iota"] & A \arrow[r, "i"] & B \arrow[r, "p"] & C \arrow[r, "\pi"] & 0
    \end{tikzcd}
  \end{equation}
  is called a \uline{short exact sequence (SES)} if it is exact at $A$, $B$ and $C$ (in the sense of~\cref{def:exact_morphism_pair}).

  Equivalently, \cref{def:short_exact_sequence/diagram} is short exact if and only if $f \equiv \Ker g$ as subobjects of $B$ and $g \equiv \Coker f$ as subobjects of $C$.
\end{definition}

\begin{note}\label{note:short_exact_sequence_factorization}
  Since $0$ is an initial object, the morphism $\iota: 0 \to A$ exists and is unique. Analogously, $\pi: C \to 0$ exists and is unique. This is why $\iota$ and $\pi$ can be skipped entirely when defining short exact sequences.

  The morphism $i$ is necessarily a monomorphism (\enquote{i} stands for \enquote{injection}) since it is equivalent to a kernel and $p$ is necessarily an epimorphism (\enquote{p} stands for \enquote{projection}). When either $i$ or $p$ is obvious, they may also be skipped.

  This makes SES a good framework for describing factorization of algebraic structures, as can be seen in~\cref{ex:short_exact_sequences}.
\end{note}

\begin{definition}\label{def:exact_sequence_morphisms}\cite{MacLane1994}[198]
  Consider the two short exact sequences over the same category $\Bold C$:
  \begin{center}
    \begin{tikzcd}
      0 \arrow[r] & A \arrow[r, "i"]   & B \arrow[r, "p"]   & C \arrow[r]  & 0 \\
      0 \arrow[r] & A' \arrow[r, "i'"] & B' \arrow[r, "p'"] & C' \arrow[r] & 0
    \end{tikzcd}
  \end{center}
  We say that the triple $f = (f_A: A \to A', f_B: B \to B', f_C: C \to C')$ is a \uline{morphism of the short exact sequences} if the following diagram commutes:
  \begin{center}
    \begin{tikzcd}
      0 \arrow[r] & A \arrow[r, "i"] \arrow[d, "f_A"] & B \arrow[r, "p"] \arrow[d, "f_B"] & C \arrow[r] \arrow[d, "f_C"] & 0 \\
      0 \arrow[r] & A' \arrow[r, "i'"]                & B' \arrow[r, "p'"]                & C' \arrow[r]                 & 0
    \end{tikzcd}
  \end{center}

  If each component of $f$ is an isomorphism, we say that the short exact sequences are \uline{isomorphic}.
\end{definition}

\begin{definition}\label{def:split_exact_sequence}\cite{nLab:split_exact_sequence}
  A short exact sequence
  \begin{equation}\label{def:split_exact_sequence/short_diagram}
    \begin{tikzcd}[baseline=(current bounding box.center)]
      0 \arrow[r] & A \arrow[r, "i"] & B \arrow[r, "p"] & C \arrow[r] & 0
    \end{tikzcd}
  \end{equation}
  is said to be \uline{splitting} or \uline{split exact} if any of the following equivalent conditions hold:
  \begin{defenum}
    \item $i$ has a left inverse
    \item $p$ has a right inverse
    \item the sequence~\cref{def:split_exact_sequence/short_diagram} is isomorphic to the SES
    \begin{equation}\label{def:short_exact_sequence/split_diagram}
      \begin{tikzcd}[baseline=(current bounding box.center)]
        0 \arrow[r] & A \arrow[r] & A \otimes C \arrow[r] & C \arrow[r] & 0
      \end{tikzcd}
    \end{equation}
    with the canonical injection and projection morphisms
  \end{defenum}

  The equivalence of the three conditions is called the \uline{splitting lemma}.
\end{definition}

\begin{example}\label{ex:short_exact_sequences}
  \mbox{}
  \begin{defenum}
    \item\label{ex:short_exact_sequences/cyclic_groups} Fix a natural number $n > 0$ and consider the category of $\Bold{Ab}$ of abelian groups and the following short exact sequence:
    \begin{center}
      \begin{tikzcd}
        0 \arrow[r] & \BB{Z} \arrow[r, "n \cdot"] & \BB{Z} \arrow[r, "\lbrack \cdot \rbrack_n"] & \BB{Z} / n \BB{Z} \arrow[r] & 0
      \end{tikzcd}
    \end{center}
    where
    \begin{itemize}
      \item $i(x) \coloneqq nx$ multiplies any integer by $n$ to obtain the subgroup $n \BB{Z}$
      \item $p(x) \coloneqq [x]_n$ projects any integer into the corresponding remainder when divided by $n$
    \end{itemize}

    The (group-theoretic) image $n \BB{Z}$ of $i$ is precisely the (group-theoretic) kernel of $[\cdot]_n$. The sequence does not split since $i$ does not have a left inverse.

    \item\label{ex:short_exact_sequences/real_number_splitting} Consider the additive groups $\BB{Z}$, $\BB{R}$ and the unit circle group $S_{\BB{R}^2}$ with the group operation given by addition of polar angles and with the vector $(1, 0)^T$ as a unit.
    \begin{center}
      \begin{tikzcd}
        0 \arrow[r] & \BB{Z} \arrow[r, "i"] & \BB{R} \arrow[r, "p"] & S_{\BB{R}^2} \arrow[r] & 0
      \end{tikzcd}
    \end{center}
    where
    \begin{itemize}
      \item $i$ is the canonical embedding of $\BB{Z}$ is $\BB{R}$
      \item $p \coloneqq f \circ g$ where $g(x) \coloneqq \{ x \}$ is the fractional part of $x$ (modulo 1) and $f(x) \coloneqq (\cos(x), \sin(x))^T$ is an embedding of the interval $[0, 1)$ into the unit circle.
    \end{itemize}

    Since each integer has fractional part $0$ and $p(0) = (1, 0)^T$, the image $\Bold Z$ of $\Bold Z$ under $i$ is the kernel of the group homomorphism $p$.

    The sequence does not split since $i$ is not left-invertible.

    \item\label{ex:short_exact_sequences/vector_space_sum} The following SES of real vector spaces splits
    \begin{center}
      \begin{tikzcd}[ampersand replacement=\&]
        0 \arrow[r] \& \BB{R} \arrow[r, "{\begin{pmatrix}1 \\ 0\end{pmatrix}}"] \& \BB{R}^2 \arrow[r, "{\begin{pmatrix}0 & 1\end{pmatrix}}"] \& \BB{R} \arrow[r] \& 0
      \end{tikzcd}
    \end{center}
    since all of the following equivalent conditions hold
    \begin{itemize}
      \item $\begin{pmatrix}1 & 0\end{pmatrix}$ is a left inverse to $\begin{pmatrix}1 \\ 0\end{pmatrix}$
      \item $\begin{pmatrix}0 \\ 1\end{pmatrix}$ is a right inverse to $\begin{pmatrix}0 & 1\end{pmatrix}$
      \item $\BB{R}^2$ is a direct product and a biproduct of two copies of $\BB{R}$
    \end{itemize}

    \item\label{ex:short_exact_sequences/fundamental_theorem_of_calculus} The fundamental theorem of calculus is a splitting of the SES of vector spaces
    \begin{center}
      \begin{tikzcd}
        0 \arrow[r] & \BB{R} \arrow[r] & C^n(\BB{R}, \BB{R}) \arrow[r, "\frac d {dx}"] & C^{n-1}(\BB{R}, \BB{R}) \arrow[r] & 0.
      \end{tikzcd}
    \end{center}
  \end{defenum}
\end{example}
