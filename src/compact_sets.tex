\subsection{Compact sets}\label{subsec:compact_sets}

Let \( (X, \Cal{T}) \) be a topological space.

\begin{definition}\label{def:compact_set}\cite[40]{Deimling1985}
  The set \( A \subseteq X \) is called \Def{compact} if any of the following equivalent conditions hold:
  \begin{defenum}
    \DItem{def:compact_set/union} (\Def{Finite union property}) Every open cover of \( X \) has a finite subcover.
    \DItem{def:compact_set/intersection} (\Def{Finite intersection property}) The intersection is nonempty for every family of sets such that the intersection of any finite subfamily is nonempty.
  \end{defenum}

  If the closure if \( A \) is compact, we call \( A \)~\Def{relatively compact} or \Def{precompact} (although the term \enquote{precompact} is also used for totally bounded sets, see \ref{def:totally_bounded_set}).
\end{definition}

\begin{theorem}[Tychonoff's product theorem]\label{thm:tychonoffs_product_theorem}\cite[theorem 3.2.4]{Engelking1989}
  Let \( { (X_i, \Cal{T}_i) }_{i \in I} \) be a family of topological spaces. Their product\Tinyref{def:topological_product} \( (\prod_{i \in I} X_i, \prod_{i \in I} \Cal{T}_i) \) is compact\Tinyref{def:compact_set} if and only if \( (X_i, \Cal{T}_i) \) is compact for every \( i \in I \).

\end{theorem}
