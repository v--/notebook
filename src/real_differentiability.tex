\subsection{Real differentiability}\label{subsec:real_differentiability}

\begin{proposition}\label{thm:real_valued_differentiability}
  Let \( U \subseteq \BbbR^n \) be an open set. A real-valued function \( f: U \to \BbbR \) is differentiable at \( x \) in the direction \( h \) if and only if \( \varphi(t) = f(x + th) \) is right-differentiable at \( 0 \).
\end{proposition}
\begin{proof}
  \begin{equation*}
    f_+'(x)(h) \coloneqq \lim_{t \downarrow 0} \frac {f(x + th) - f(x)} t = \varphi_+'(0)(1).
  \end{equation*}
\end{proof}

\begin{example}[Weierstrass' nowhere differentiable function]\label{ex:weierstrass_nowhere_differentiable_function}\mcite\cite[\textnumero 271]{Фихтенгольц1968/2}
  Let \( a \in (0, 1) \) and \( b \) is a positive odd integer such that
  \begin{equation*}
    ab > 1 + \frac 3 2 \pi.
  \end{equation*}

  Define the function
  \begin{equation*}
    f(x) \coloneqq \sum_{k=0}^\infty a^k \cos(b^k \pi x).
  \end{equation*}

  \begin{figure}
    \centering
    \begin{mplibcode}
      input metapost/plotting;
      u := 3cm;

      a := 0.9;
      b := 7;
      n := 4;

      vardef f_k(expr x, k) =
      pow(a, k) * cosd(pow(b, k) * pi * x)
      enddef;

      vardef f(expr x) =
      result := 0;

      for k = 1 upto n:
      result := result + f_k(x, k);
      endfor

      result / 2 % scale by 0.5 for the sake of visualization
      enddef;

      beginfig(2)
      drawarrow (-pi / 2, 0) scaled u -- (pi / 2, 0) scaled u;
      drawarrow (0, -pi / 10) scaled u -- (0, pi / 2) scaled u;

      draw path_of_plot(f, -pi / 2, pi / 2, 0.01, u);
      endfig;
    \end{mplibcode}
    \Caption{ex:weierstrass_nowhere_differentiable_function/plot}{Plot of the 4-th partial sum of the Weierstrass function with \( a = 0.9 \) and \( b = 7 \) from \( -\pi \) to \( \pi \).}
  \end{figure}

  Since \( \cos \) is bounded for real arguments and \( a \in (0, 1) \), each term is uniformly bounded by \( 1 \) and by \fullref{thm:weierstrass_series_criterion}, \( f \) is continuous. However, it is not \hyperref[def:differentiability]{differentiable} at any point. The proof of the latter is involved and will not be given here.
\end{example}
