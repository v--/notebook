\subsection{Groups}\label{subsec:groups}

\begin{definition}\label{def:monoid_inverse}
  Let \( M \) be a \hyperref[def:monoid]{monoid}. We say that \( y \) is the \term{left inverse} (resp. \term{right inverse}) of \( x \) if \( yx = e \) (resp. \( xy = e \)).

  If \( y \) is simultaneously a left and right inverse of \( x \), we call a \term{two-sided inverse} or simply an \term{inverse} of \( x \) and denote it by \( x^{-1} \). It is unique by \fullref{def:monoid_inverse_unique}. This notation is consistent with monoid exponentiation defined in \fullref{def:monoid/exponentiation}.
\end{definition}

\begin{proposition}\label{def:monoid_inverse_unique}
  For every element \( x \) of any monoid, the (two-sided) \hyperref[def:monoid_inverse]{inverse} \( x^{-1} \) of \( x \), if it exists, is unique.
\end{proposition}
\begin{proof}
  If \( y \) and \( z \) are both inverses of \( x \), then
  \begin{equation*}
    y = ey = zxy = ze = z.
  \end{equation*}
\end{proof}

\begin{definition}\label{def:zero_morphisms}
  Let \( \cat{C} \) be a \hyperref[def:universal_objects/zero]{pointed category} with a fixed \hyperref[def:universal_objects/zero]{zero object} \( Z \).

  \begin{thmenum}
    \thmitem{def:zero_morphisms/morphism}\mcite{nLab:zero_morphism} For every pair of objects \( A \) and \( B \) in \( \cat{C} \), there exists unique morphism, called the \term{zero morphism}, that \hyperref[def:factors_through]{uniquely factors through} \( Z \):
    \begin{equation}\label{eq:def:zero_morphisms/morphism}
      \begin{aligned}
        \includegraphics[page=1]{output/def__zero_morphisms.pdf}
      \end{aligned}
    \end{equation}

    We denote this zero morphism by \( 0_{A,B} \).

    \thmitem{def:zero_morphisms/kernel} The \term{kernel} cone of a morphism \( f: A \to B \) is the \hyperref[eq:def:equalizers/equalizer]{equalizer} cone of \( f \) and \( 0_{A,B} \).

    By \fullref{thm:equalizer_invertibility}, a kernel morphism is necessarily a monomorphism. A monomorphism is \term{normal} if it is a kernel.

    \thmitem{def:zero_morphisms/cokernel} \hyperref[thm:categorical_principle_of_duality]{Dually}, the \term{cokernel} cocone of a morphism \( f: A \to B \) is the \hyperref[eq:def:equalizers/coequalizer]{coequalizer} cocone of \( f \) and \( 0_{A,B} \).

    By \fullref{thm:equalizer_invertibility}, a cokernel morphism is necessarily an epimorphism. An epimorphism is \term{normal} if it is a kernel.
  \end{thmenum}
\end{definition}

\begin{definition}\label{def:group}
  A \term{group} is a \hyperref[def:monoid]{monoid} in which every element has an \hyperref[def:monoid_inverse]{inverse}. Groups are the most well-studied and most well-behaved magmas. Many useful properties like \hyperref[thm:def:group/properties/cancellative]{cancellation} rely on associativity, so we do not consider non-associative groups.

  Groups have the following metamathematical properties:
  \begin{thmenum}
    \thmitem{def:group/theory} We can construct a \hyperref[def:first_order_theory]{first-order theory} for groups by adding a unary \hyperref[def:first_order_language/func]{functional symbol} \( (\anon)^{-1} \) to the language and the axiom
    \begin{equation}\label{eq:def:group/theory/inverse_axiom}
      \qforall \xi (\xi \cdot \xi^{-1} = e \wedge \xi^{-1} \cdot \xi = e)
    \end{equation}
    to the \hyperref[def:monoid/theory]{theory of monoids}.

    \thmitem{def:group/function_parity} A \hyperref[def:function]{function} \( \varphi: G \to H \) between two groups is called \term{even} if, for every \( x \in G \), we have
    \begin{equation}\label{eq:def:group/function_parity/even}
      \varphi(x^{-1}) = \varphi(x)
    \end{equation}
    and \term{odd} if
    \begin{equation}\label{eq:def:group/function_parity/odd}
      \varphi(x^{-1}) = \varphi(x)^{-1}.
    \end{equation}

    \thmitem{def:group/homomorphism} A \hyperref[def:first_order_homomorphism]{first-order homomorphism} between the groups \( G \) and \( H \) is an odd \hyperref[def:monoid/homomorphism]{monoid homomorphism}.

    As shown in \fullref{thm:group_homomorphism_single_condition}, however, the conditions \eqref{eq:def:pointed_set/homomorphism} and \eqref{eq:def:group/function_parity/odd} are redundant.

    \thmitem{def:group/submodel} The set \( A \subseteq G \) is a \hyperref[thm:substructure_is_model]{submodel} of \( G \) if it is a \hyperref[def:monoid/submodel]{submonoid} and if \( x \in A \) implies \( x^{-1} \in A \). We say that \( A \) is a \term{subgroup} of \( G \).

    As a consequence of \fullref{thm:positive_formulas_preserved_under_homomorphism}, the image of a group homomorphism is a subgroup of its range.

    For an arbitrary subset \( A \) of \( G \), we denote the \hyperref[def:first_order_generated_substructure]{generated submodel} by \( \braket{ A } \). In addition to the elements of \( A \), \( \braket{ A } \) contains their products and inverses, the products of their products and inverses, etc...

    The \hyperref[def:free_group]{free group} builds a group out of a plain set; furthermore, as a consequence of \fullref{thm:group_presentation}, every group is a \hyperref[def:group/quotient]{quotient} of a free group. Compare this to \hyperref[def:free_semimodule]{free (semi)modules} and \hyperref[def:polynomial_semiring]{polynomial (semi)rings}.

    \thmitem{def:group/trivial} The \hyperref[thm:substructures_form_complete_lattice/bottom]{trivial} group is the \hyperref[def:pointed_set/trivial]{trivial pointed set} \( \set{ e } \).

    \thmitem{def:group/exponentiation} We extend \hyperref[def:monoid/exponentiation]{monoid exponentiation} to all integers by setting
    \begin{equation*}
      x^{-n} \coloneqq (x^n)^{-1}.
    \end{equation*}

    This operation behaves well as shown in \fullref{thm:def:group/properties/negative_power}.

    \thmitem{def:group/category} The \hyperref[def:category_of_small_first_order_models]{category of \( \mscrU \)-small models} of groups \( \ucat{Grp} \) is \hyperref[def:concrete_category]{concrete} over \hyperref[def:monoid]{\( \ucat{Mon} \)}.

    By \fullref{thm:def:group/properties/involution}, \( \ucat{Grp} \) is also a concrete category over \( \ucat{Inv} \).

    The unique up to an isomorphism zero object in this category is the trivial group \( \set{ e } \). The \hyperref[def:zero_morphisms/morphism]{zero morphism} from \( G \) to \( H \) is
    \begin{equation*}
      \begin{aligned}
        &0_{G,H}: G \to H \\
        &0_{G,H}(x) \coloneqq e_H.
      \end{aligned}
    \end{equation*}

    We will define the \hyperref[def:free_group]{free group} functor in \fullref{subsec:free_groups}. Then from \fullref{thm:first_order_categorical_invertibility} it will follow that monomorphisms are precisely the injective homomorphisms, and that the \hyperref[def:subobject_and_quotient]{categorical subobjects} correspond to subgroups.

    Unlike in the category \hyperref[def:monoid/category]{\( \cat{Mon} \)} of monoids, in \( \cat{Grp} \) every epimorphism is surjective. We will prove this in \fullref{thm:group_epimorphisms_are_surjective}. Along with \fullref{thm:group_epimorphisms_are_normal}, this shows that the \hyperref[def:subobject_and_quotient]{categorical quotient objects} correspond to \hyperref[def:group/quotient]{quotient groups}, which we will define shortly.

    To avoid circularity, in this section, we will avoid using that monomorphisms are injective and epimorphism are surjective.

    \thmitem{def:group/kernel} The \term{kernel} of a group homomorphism \( \varphi: G \to H \) is the subgroup
    \begin{equation*}
      \ker \varphi \coloneqq \varphi^{-1}(e_H) = \set{ x \in G \given \varphi(x) = e_H }.
    \end{equation*}

    This coincides with the notion of a categorical kernel defined in \fullref{def:zero_morphisms/kernel}. Similarly to \fullref{ex:equalizers_in_set/equalizer}, \( \ker \varphi \) is the equalizer of \( \varphi \) and the zero morphism \( 0_{G,H} \).

    This equivalence holds much more generally, even for \hyperref[def:pointed_set]{pointed sets}, however speaking of kernels is only established when we have an appropriate notion of cokernels. As we will see in \fullref{def:group/quotient}, cokernels are very well-behaved for groups, but not in general. Some related problems are highlighted in \cite[ch. 8]{Golan2010}. \Fullref{thm:def:group/properties/kernel_cokernel_compatibility} expresses the compatibility between group kernels and cokernels.

    \thmitem{def:group/quotient} Consider the group homomorphism \( \varphi: G \to H \). We will find its \hyperref[def:zero_morphisms/cokernel]{cokernel}. This will highlight several very fundamental facts about groups, especially quotient groups. In practice, quotients are conveniently characterized by \fullref{thm:quotient_group_universal_property}.

    Similarly to \fullref{ex:equalizers_in_set/coequalizer}, the cokernel is an \hyperref[thm:equivalence_partition]{equivalence partition} of \( H \). The partitioning relation is different, however. An equivalence relation that is compatible with the operations of an algebraic structure is called a \term{congruence}. For the group \( H \), the equivalence relation \( \cong \) is a congruence if:
    \begin{itemize}
      \item It is compatible with the group operation: \( x \cong x' \) and \( y \cong y' \) imply \( x y \cong x' y' \).
      \item It is compatible with identities: \( e_H \cong x \) implies \( y \cong xy \) for all \( y \in H \). This easily follows from the first condition.
      \item It is compatible with inverses: \( x \cong x' \) implies \( x^{-1} \cong x'^{-1} \). This also follows from the first condition: \( x^{-1} \cong x^{-1} \) implies \( e \cong x^{-1} x' \), and thus \( x'^{-1} \cong x^{-1} \).
    \end{itemize}

    We need congruences since we are working with groups and group homomorphisms rather than sets and functions. We define \( \cong \) to be the smallest congruence relation containing
    \begin{equation*}
      \set{ (s(g), e_H) \given g \in G }.
    \end{equation*}

    Denote the partition \( H / {\cong} \) by \( Q \). Define a group operation on \( Q \) as \( [x] \cdot [y] = [xy] \).
    \begin{itemize}
      \item This operation is well-defined since group congruences are compatible with the group operation. We are thus free to denote it via juxtaposition.
      \item The coset \( [e_H] \) is the identity of \( Q \) since congruences are compatible with identities.
      \item The coset \( [x]^{-1} \) is the inverse of \( [x] \) since congruences are compatible with inverses.
    \end{itemize}

    Therefore, \( Q \) is a group and \( \pi(x) \coloneqq [x] \) is a group homomorphism. The pair \( (Q, \pi) \) is thus a categorical cokernel of \( \varphi \) by the same argument as in \fullref{ex:equalizers_in_set/coequalizer}.

    Denote the identity \( [e_H] \) by \( N \). It is a subgroup of \( H \):
    \begin{itemize}
      \item It contains the identity \( e_H \).
      \item It is closed under the group operation. Indeed, if \( [x] = [y] = N \), then
      \begin{equation*}
        [xy] = [x][y] = NN = N.
      \end{equation*}

      \item It is closed under the group inverse. Indeed, \( [x^{-1} x] = N \) for every \( x \in H \). If \( [x] = N \), then \( [x^{-1}] N = N \), and hence \( [x^{-1}] = N \).

      \item It possesses one additional important property. If \( [x] = N \), then not only \( x \in N \), but also \( y^{-1} x y \in N \) for every \( y \in H \). This holds because
      \begin{equation*}
        [y^{-1} x y]
        =
        [y^{-1}] [x] [y]
        =
        [y^{-1}] [y]
        =
        [y]^{-1} [y]
        =
        N.
      \end{equation*}
    \end{itemize}

    This last property distinguishes \( N \) from the \hyperref[def:multi_valued_function/image]{image} of \( \varphi \). A subgroup satisfying this property is called a \term{normal subgroup}. See \fullref{thm:normal_subgroup} for equivalent conditions. If the image \( \img \varphi \) is a normal subgroup of \( H \), \( \varphi \) is a normal epimorphism in the sense of \fullref{def:zero_morphisms/cokernel}.

    Obviously \( \img \varphi \subseteq N \). Since \( Q \) is a colimit, \( N \) must be the smallest normal subgroup containing \( \img \varphi \).

    It is more intriguing that \( [x] = xN \) for every \( x \in H \). This can be shown as follows:
    \begin{itemize}
      \item Suppose first that \( y \in xN \), i.e. \( y = xn \) for some \( n \in N \). Then
      \begin{equation*}
        y \in [y] = [xn] = [x] N = [x].
      \end{equation*}

      Generalizing on \( y \), we obtain that \( xN \subseteq [x] \).

      \item Conversely, let \( y \in [x] \). Obviously \( x = y (y^{-1} x) \). Then
      \begin{equation*}
        [x^{-1} y] = [x^{-1}] [y] = [x]^{-1} [y] = [x]^{-1} [x] = N.
      \end{equation*}

      Hence, \( x^{-1} y \in N \) and \( y \in xN \). Generalizing on \( y \), we obtain that \( [x] \subseteq xN \).
    \end{itemize}

    Therefore, all cosets in the quotient group \( Q = H / {\cong} \) are translations of the identity \( N \). In particular, in the notation of \hyperref[def:magma/powet_set]{power set operations}, it follows that
    \begin{equation*}
      xyN = xN yN.
    \end{equation*}

    Finally, given a normal subgroup \( N \) of an arbitrary group \( G \), we can define the \term{quotient group} \( G / N \) as the cokernel of the inclusion \( \iota: N \to G \). That is, \( G / N \) consists of the cosets \( xN \) for \( x \in G \) with the group operation \( xN yN = xyN \).

    \thmitem{def:group/simple} If the only proper \hyperref[thm:normal_subgroup_equivalences]{normal subgroup} of \( G \) is the \hyperref[def:group/trivial]{trivial subgroup} \( \set{ e_G } \), we say that \( G \) is a \term{simple group}.

    The trivial group itself is not simple, because it has no proper subgroups.
  \end{thmenum}
\end{definition}

\begin{example}\label{ex:power_set_is_not_a_group}
  The \hyperref[def:magma/power_set]{power set magma} \( \pow(G) \) of a group \( G \) is a monoid, but it is not a group unless \( G \) is trivial.
\end{example}

\begin{proposition}\label{thm:def:group/properties}
  Every \hyperref[def:group]{group} \( G \) has the following basic properties:
  \begin{thmenum}
    \thmitem{thm:def:group/properties/cancellative} The (binary) group operation is \hyperref[def:magma/cancellative]{cancellative}.
    \thmitem{thm:def:group/properties/identity_inverse} The identity \( e \) is its own inverse.
    \thmitem{thm:def:group/properties/inverse_composition} \( (xy)^{-1} = y^{-1} x^{-1} \).
    \thmitem{thm:def:group/properties/involution} \( x = (x^{-1})^{-1} \)
    \thmitem{thm:def:group/properties/negative_power} For any positive integer \( n \), \( (x^n)^{-1} = (x^{-1})^n \)
    \thmitem{thm:def:group/properties/zero_kernel} If the \hyperref[def:group/kernel]{kernel} of a group homomorphism \( \varphi: G \to H \) is trivial, then \( \varphi \) is an \hyperref[def:first_order_homomorphism_invertibility/embedding]{embedding} (injective homomorphism).

    \thmitem{thm:def:group/properties/kernel_cokernel_compatibility} For a \hyperref[def:group/quotient]{quotient group} \( G / N \) with canonical projection \( \pi(x) \coloneqq xN \), the \hyperref[def:group/kernel]{kernel} of \( \pi \) is \( N \).

    \thmitem{thm:def:group/properties/kernel_is_normal_subgroup} The kernel of a group homomorphism is a \hyperref[thm:normal_subgroup]{normal subgroup}.
  \end{thmenum}
\end{proposition}
\begin{proof}
  \SubProofOf{thm:def:group/properties/cancellative} If \( x = y \), obviously \( xz = yz \) and \( zx = zy \). Now if \( xz = yz \), we have
  \begin{equation*}
    x = x(zz^{-1}) = (xz)z^{-1} = (yz)z^{-1} = y(zz^{-1}) = y.
  \end{equation*}

  The case \( zx = zy \) is analogous.

  \SubProofOf{thm:def:group/properties/identity_inverse} \( ee = e \).
  \SubProofOf{thm:def:group/properties/inverse_composition}
  \begin{equation*}
    (xy) (y^{-1} x^{-1})
    =
    x (y y^{-1}) x^{-1}
    =
    e
    =
    y^{-1} (x^{-1} x) y
    =
    (y^{-1} x^{-1}) (xy).
  \end{equation*}

  \SubProofOf{thm:def:group/properties/involution}
  \begin{equation*}
    (x^{-1})^{-1}
    =
    x x^{-1} (x^{-1})^{-1}
    =
    x.
  \end{equation*}

  \SubProofOf{thm:def:group/properties/negative_power} Using \fullref{thm:def:group/properties/involution},
  \begin{equation*}
    x^{-n}
    =
    (x^n)^{-1}
    =
    x^{-1} \cdots x^{-1}
    =
    (x^{-1})^n.
  \end{equation*}

  \SubProofOf{thm:def:group/properties/zero_kernel} Suppose that \( \ker \varphi = \set{ e_H } \) and \( \varphi(x) = \varphi(y) \). Then
  \begin{equation*}
    e_H = \varphi(x) \varphi(y)^{-1} = \varphi(x y^{-1}).
  \end{equation*}

  Thus, \( x y^{-1} \in \ker \varphi \), and hence \( x = y \).

  Therefore, \( \varphi \) is injective.

  \SubProofOf{thm:def:group/properties/kernel_cokernel_compatibility} Trivial.

  \SubProofOf{thm:def:group/properties/kernel_is_normal_subgroup} For a homomorphism \( \varphi: G \to H \), if \( x \in \ker \varphi \), then
  \begin{equation*}
    \varphi(y^{-1} x y) = \varphi(y)^{-1} \varphi(x) \varphi(y) = \varphi(y)^{-1} \varphi(y) = e_H,
  \end{equation*}
  and thus \( y^{-1} x y \in \ker \varphi \).
\end{proof}

\begin{proposition}\label{thm:group_homomorphism_single_condition}
  A function between groups is a \hyperref[def:group/homomorphism]{group homomorphism} if and only if it satisfies \eqref{eq:def:magma/homomorphism}.
\end{proposition}
\begin{proof}
  \SufficiencySubProof \eqref{eq:def:magma/homomorphism} is required to hold by definition.

  \NecessitySubProof Let the function \( \varphi: G \to H \) satisfy \eqref{eq:def:magma/homomorphism}. Then it preserves identities, i.e. is a \hyperref[def:pointed_set/homomorphism]{pointed set homomorphism}. Indeed, we have
  \begin{equation*}
    e_H \varphi(e_G) = \varphi(e_G) = \varphi(e_G e_G) = \varphi(e_G) \varphi(e_G).
  \end{equation*}

  By \fullref{thm:def:group/properties/cancellative}, \( \varphi \) is cancellative, and hence \( e_H = \varphi(e_G) \).

  Inverses are preserved (i.e. \eqref{eq:def:group/function_parity/odd} holds) because
  \begin{equation*}
    \varphi(x^{-1})
    =
    \varphi(x^{-1}) e_H
    =
    \varphi(x^{-1}) \varphi(x) \varphi(x)^{-1}
    =
    \varphi(x^{-1} x) \varphi(x)^{-1}
    =
    e_H \varphi(x)^{-1}
    =
    \varphi(x)^{-1}.
  \end{equation*}

  Therefore, \( \varphi \) is indeed a group homomorphism.
\end{proof}

\begin{lemma}\label{thm:group_operation_induces_bijections}
  For each element \( x \) of a group \( G \), consider the function \( \varphi_x \coloneqq x \id_G \), i.e.
  \begin{equation*}
    \begin{aligned}
      &\varphi_x: G \to G \\
      &\varphi_x(y) \coloneqq x \cdot y.
    \end{aligned}
  \end{equation*}

  This is a bijective function (but not necessarily a group isomorphism).
\end{lemma}
\begin{proof}
  \SubProofOf[def:function_invertibility/injective]{injectivity} If \( y, y' \in G \) and \( \varphi_x(y) = \varphi_x(y') \), we have
  \begin{equation*}
    xy = \varphi_x(y) = \varphi_x(y') = xy'.
  \end{equation*}

  By \fullref{thm:def:group/properties/cancellative}, \( y = y' \). Therefore, \( \varphi_x \) is injective.

  \SubProofOf[def:function_invertibility/surjective]{surjectivity} If \( z \in G \), then \( z = x(x^{-1} z) \). Therefore, \( z = \varphi_x(x^{-1} z) \), and thus every member of \( G \) has a preimage. Thus, \( \varphi_x \) is surjective.
\end{proof}

\begin{definition}\label{def:subgroup_cosets}
  Let \( H \subseteq G \) be a subgroup of \( G \). Even if \( H \) is not normal, we can define the \term{left} and \term{right cosets}
  \begin{equation*}
    x H \coloneqq \set{ xh \given h \in H }
    \quad\quad
    H x \coloneqq \set{ hx \given h \in H }.
  \end{equation*}

  The \term{index} \( [G : H] \) of \( H \) in \( G \) is \hyperref[def:cardinal]{cardinality} of the family of all left cosets.

  The discussion in \fullref{def:group/quotient} can be generalized to show that  \( \set{ xH \given x \in G } \) is a \hyperref[def:set_partition]{partition} of \( G \) into \hyperref[def:equinumerosity]{equinumerous} sets. If \( H \) is not normal, this partition is not induced by a congruence, and we cannot form a quotient group using a non-normal subgroup. Nonetheless, left and right cosets still turn out useful.
\end{definition}

\begin{proposition}\label{thm:group_coset_bijection}
  The family of all \hyperref[def:subgroup_cosets]{left cosets} of a subgroup is \hyperref[def:equinumerosity]{equinumerous} to the family of all right cosets.
\end{proposition}
\begin{proof}
  Fix a subgroup \( H \) of \( G \), and consider the function \( xH \mapsto Hx \) taking left cosets to right cosets.

  It is well-defined because, if \( x H = x' H \), then there exists \( h \), such that \( x = x' h \), and thus
  \begin{equation*}
    H x' = H h^{-1} x = H x.
  \end{equation*}

  It is injective by the same converse argument, and it is surjective by definition. Therefore, it is bijective.
\end{proof}

\begin{proposition}\label{thm:normal_subgroup_equivalences}
  For a subgroup \( N \) of \( G \), the following conditions are equivalent:
  \begin{thmenum}
    \thmitem{thm:normal_subgroup_equivalences/congruence} For every element \( x \) of \( G \), we have the set equality
    \begin{equation}\label{eq:thm:normal_subgroup_equivalences/congruence}
      x^{-1} N x = N.
    \end{equation}

    This is the definition of a normal subgroup obtained in \fullref{def:group/quotient}.

    \thmitem{thm:normal_subgroup_equivalences/cosets} The partitions induced by the \hyperref[def:subgroup_cosets]{left and rights cosets} of \( N \) coincide.

    \thmitem{thm:normal_subgroup_equivalences/kernel} \( N \) is the \hyperref[thm:group_kernels]{kernel} of some group homomorphism.
  \end{thmenum}

  In particular, kernels are always normal subgroups.
\end{proposition}
\begin{proof}
  This is the group-theoretic analog to \fullref{thm:equivalence_partition}.

  \ImplicationSubProof{thm:normal_subgroup_equivalences/congruence}{thm:normal_subgroup_equivalences/cosets} For any \( x \in G \)
  \begin{equation*}
    N x = (x N x^{-1})x = x N(x^{-1}x) = x N,
  \end{equation*}
  thus every left coset is a right coset and vice versa.

  \ImplicationSubProof{thm:normal_subgroup_equivalences/cosets}{thm:normal_subgroup_equivalences/kernel} We can take the \hyperref[def:group/quotient]{canonical projection} \( \pi(x) \coloneqq x N \) as the homomorphism. By \fullref{thm:def:group/properties/kernel_cokernel_compatibility}, \( \ker \pi = N \).

  \ImplicationSubProof{thm:normal_subgroup_equivalences/kernel}{thm:normal_subgroup_equivalences/congruence} Let \( \varphi: G \to H \) be a group homomorphism and fix any \( x \in G \). Denote \( N \coloneqq \ker(f) \). By \fullref{thm:def:group/properties/kernel_is_normal_subgroup}, it is a normal subgroup in the sense of \fullref{def:group/quotient}, i.e. it satisfies \eqref{eq:thm:normal_subgroup_equivalences/congruence}.
\end{proof}

\begin{theorem}[Quotient group universal property]\label{thm:quotient_group_universal_property}\mcite[thm. II.7.12]{Aluffi2009}
  For every \hyperref[def:group]{group} \( G \) and \hyperref[thm:normal_subgroup_equivalences]{normal subgroup} \( N \), the \hyperref[def:group/quotient]{quotient group} \( G / N \) has the following \hyperref[rem:universal_mapping_property]{universal mapping property}:
  \begin{displayquote}
    Every group homomorphism \( \varphi: G \to H \) satisfying \( N \subseteq \ker \varphi \) \hyperref[def:factors_through]{uniquely factors through} \( G / N \). That is, there exists a unique homomorphism \( \widetilde{\varphi}: G / N \to H \), such that the following diagram commutes:
    \begin{equation}\label{eq:thm:quotient_group_universal_property/diagram}
      \begin{aligned}
        \includegraphics[page=1]{output/thm__quotient_group_universal_property.pdf}
      \end{aligned}
    \end{equation}

    In the case where \( N = \ker \varphi \), \( \widetilde{\varphi} \) is an \hyperref[def:first_order_homomorphism_invertibility/embedding]{embedding}.
  \end{displayquote}

  This extends to \fullref{thm:quotient_module_universal_property} and \fullref{thm:quotient_algebra_universal_property}.
\end{theorem}
\begin{proof}
  We want
  \begin{equation*}
    \widetilde{\varphi}(\pi(x)) = \widetilde{\varphi}(xN) = \varphi(x).
  \end{equation*}

  This suggests the definition
  \begin{equation*}
    \widetilde{\varphi}(xN) \coloneqq \varphi(x).
  \end{equation*}

  The homomorphism \( \widetilde{\varphi} \) is well-defined because, if \( x N = x' N \), since \( N \subseteq \ker \varphi \), we have
  \begin{equation*}
    \varphi(x)
    =
    \varphi(x) e_N
    =
    \varphi(x) \varphi(N)
    =
    \varphi(x N)
    =
    \varphi(x' N)
    =
    \cdots
    =
    \varphi(y).
  \end{equation*}

  If \( N = \ker \varphi \), the kernel of \( \widetilde{\varphi} \) is trivial. By \fullref{thm:def:group/properties/zero_kernel}, it is an injective function.
\end{proof}

\begin{corollary}\label{thm:quotient_group_by_kernel}
  Every \hyperref[def:group/homomorphism]{group homomorphism} \( \varphi: G \to H \) induces an isomorphism
  \begin{equation*}
    G / \ker \varphi \cong \img \varphi.
  \end{equation*}
\end{corollary}
\begin{proof}
  Directly follows from \fullref{thm:quotient_group_universal_property} by restricting the range of \( \widetilde{\varphi} \) to its image.
\end{proof}

\begin{corollary}\label{thm:group_epimorphisms_are_normal}
  Every surjective group homomorphism is a \hyperref[def:zero_morphisms/cokernel]{normal epimorphism}.
\end{corollary}
\begin{proof}
  Fix a group homomorphism \( \varphi: G \to H \). By \fullref{thm:quotient_group_by_kernel}, \( G / \ker \varphi \cong \img \varphi = H \). Thus, \( H \) is a \hyperref[def:zero_morphisms/cokernel]{cokernel} of the canonical inclusion \( \iota: \ker \varphi \to G \).
\end{proof}

\begin{theorem}[Quotient subgroup lattice theorem]\label{thm:quotient_subgroup_lattice_theorem}\mcite[prop. II.8.9]{Aluffi2009}
  Given a \hyperref[thm:normal_subgroup_equivalences]{normal subgroup} \( N \) of \( G \), the function \( H \mapsto H / N \) is a \hyperref[def:semilattice/homomorphism]{lattice homomorphism} between the lattice of \hyperref[def:group/submodel]{subgroups} of \( G \) containing \( N \) and the lattice of subgroups of the \hyperref[def:group/quotient]{quotient} \( G / N \).

  \begin{figure}[h]
    \centering
    \includegraphics[page=1]{output/thm__lattice_theorem_for_groups.pdf}
    \caption{The lattice of subgroups of \( G \) and the lattice of subgroups of \( G / N \).}
    \label{fig:thm:quotient_subgroup_lattice_theorem}
  \end{figure}

  This extends to \fullref{thm:quotient_submodule_lattice_theorem} and \fullref{thm:quotient_ideal_lattice_theorem}.
\end{theorem}
\begin{proof}
  \SubProofOf[def:function_invertibility/injective/equality]{injectivity} Let \( H_1 / N = H_2 / N \). Both \( H_1 / N \) and \( H_2 / N \) consist of the same cosets, hence
  \begin{equation*}
    H_1 = \bigcup (H_1 / N) = \bigcup (H_2 / N) = H_2.
  \end{equation*}

  Therefore, the map \( H \mapsto H / N \) is injective.

  \SubProofOf[def:function_invertibility/surjective/existence]{surjectivity} Fix a subgroup \( M \) of \( G / N \) and define
  \begin{equation*}
    H \coloneqq \set{ x \in G \given xN \in M }.
  \end{equation*}

  Then clearly \( H / N = M \). Therefore, the map \( H \mapsto H / N \) is surjective.

  \SubProofOf[def:semilattice/homomorphism]{lattice compatibility} The join \( \braket{ K \cup H } \) of two subgroups of \( G \) containing \( N \) must satisfy the equality
  \begin{equation}\label{eq:thm:quotient_subgroup_lattice_theorem/join}
    \underbrace{ \braket{ K \cup H } / N }_{\set{ xN \given x \in \braket{ K \cup H } }}
    =
    \underbrace{ \braket{ (K / N) \cup (H / N) } }_{\braket{ \set{ xN \given x \in K \cup H } }}.
  \end{equation}

  Verifying this amounts to noting that \( \braket{ K \cup H } \) is obtained by adding the products and inverses of any elements of \( G \) not in \( K \cup H \). Since the projection map \( \pi: G \to G / N \) is a homomorphism, the coset \( xy N \) of the product of \( x, y \in \braket{ K \cup H } \) is the product \( (xN) (yN) \) of the cosets \( xN \) and \( yN \), and analogously for inverses. Hence, adding an element \( x \in G \) to \( K \cup H \) and then taking all cosets is the same as adding the coset \( xH \) to \( (K / N) \cup (H / N) \).

  Therefore, \eqref{eq:thm:quotient_subgroup_lattice_theorem/join} holds, and thus \( H \mapsto H / N \) preserves joins in the lattice of subgroups.

  The other verifications are simpler. For meets, we have
  \begin{equation*}
    (K \cap H) / N
    =
    \set{ xN \given x \in K \cap H }
    =
    \set{ xN \given x \in K } \cap \set{ xN \cap x \in H }
    =
    (K / N) \cap (H / N).
  \end{equation*}

  Finally, it remains to show that \( H \mapsto H / N \) preserves the \hyperref[def:partially_ordered_set_extremal_points/top_and_bottom]{top and bottom elements}. This is trivial since \( G / N \) contains all possible cosets of \( N \) and is hence the top in the lattice of subgroups of \( G / N \), and \( N / N \) is the trivial group and hence the bottom.

  Therefore, \( H \mapsto H / N \) is a lattice isomorphism.
\end{proof}

\begin{theorem}[Lagrange's theorem for groups]\label{thm:lagranges_theorem_for_groups}
  Let \( H \) be a subgroup of \( G \). We have the following equality
  \begin{equation}\label{eq:thm:lagranges_theorem_for_groups/index}
    \card(G) = \card(H) \cdot [G : H].
  \end{equation}

  If \( H \) is a \hyperref[thm:normal_subgroup_equivalences]{normal subgroup}, then \( [G : H] = \card(G / H) \) and
  \begin{equation}\label{eq:thm:lagranges_theorem_for_groups/card}
    \card(G) = \card(H) \cdot \card(G / H).
  \end{equation}

  This demonstrates that there exists a bijective function between the \hyperref[def:monoid_direct_product]{direct product} \( H \times G / H \) and \( H \), however this may not be a group homomorphism --- see \fullref{ex:lagranges_theorem_for_groups/direct_product_zn}.
\end{theorem}
\begin{proof}
  By \fullref{def:subgroup_cosets}, every coset of \( G \) with respect to \( H \) is equinumerous with \( H \), and there is a total of \( [G : H] \) cosets. Therefore, \eqref{eq:thm:lagranges_theorem_for_groups/index} holds.
\end{proof}

\begin{example}\label{ex:subgroups_of_integers}
  Consider the group \( \BbbZ \) of integers with respect to addition.

  Let \( 2\BbbZ \) be the subgroup of all even integers. Then both \( \BbbZ \) and \( 2\BbbZ \) are countably infinite, but their quotient group \( \BbbZ / 2\BbbZ \) has two elements --- the set \( 2\BbbZ \) of all even integers and the set \( 2\BbbZ + 1 \) of all odd integers. Generalizations of this quotient group are discussed in \fullref{thm:group_of_integers_modulo}. \Fullref{thm:lagranges_theorem_for_groups} holds, but it gives no insight due to the absorbing properties of transfinite cardinal arithmetic described in \fullref{thm:simplified_cardinal_arithmetic}.

  Now consider the groups \( 4\BbbZ \subseteq 2\BbbZ \subseteq \BbbZ \). As a consequence of \fullref{thm:lagranges_theorem_for_groups}, \( 3\BbbZ \) is not a subgroup of \( 2\BbbZ \), and so we consider powers of \( 2 \).

  Since \( 2\BbbZ \) is a subgroup of \( \BbbZ \), the quotient \( 2\BbbZ / 4\BbbZ \) must a subgroup of \( \BbbZ / 4\BbbZ \) as a consequence of \fullref{thm:quotient_subgroup_lattice_theorem}. We may not know the structure of the quotient groups (although we do, see \fullref{thm:group_of_integers_modulo}), but we know how \( 4\BbbZ \), \( 2\BbbZ \) and \( \BbbZ \) relate to each other, and we are able to determine how the quotient groups relate to each other.
\end{example}
