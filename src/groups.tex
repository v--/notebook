\subsection{Groups}\label{subsec:groups}

\begin{definition}\label{def:unital_magma_inverse_element}
  Let \( \mscrM \) be a \hyperref[def:unital_magma]{unital magma}. We say that \( y \) is the \term{left inverse} (resp. \term{right inverse}) of \( x \) if \( yx = e \) (resp. \( xy = e \)).

  If \( y \) is simultaneously a left and right inverse of \( x \), we call a \term{two-sided inverse} or simply an \term{inverse} of \( x \) and denote it by \( x^{-1} \). It is unique by \fullref{def:unital_magma_inverse_element_unique}. This notation is consistent with \fullref{def:unital_magma/exponentiation}
\end{definition}

\begin{proposition}\label{def:unital_magma_inverse_element_unique}
  For every element \( x \) of any unital magma, the (two-sided) \hyperref[def:unital_magma_inverse_element]{inverse} \( x^{-1} \) of \( x \) is unique.
\end{proposition}
\begin{proof}
  If \( y \) and \( z \) are both inverses of \( x \), then
  \begin{equation*}
    y = ey = zxy = ze = z.
  \end{equation*}
\end{proof}

\begin{definition}\label{def:group}
  A \term{group} is a \hyperref[def:unital_magma/monoid]{monoid} in which every element has an \hyperref[def:unital_magma_inverse_element]{inverse}. Groups are the most well-studied and most well-behaved magmas. Many useful properties like \hyperref[thm:def:group/properties/cancellative]{cancellation} rely on associativity, so we do not consider non-associative groups.

  \begin{thmenum}
    \thmitem{def:group/theory} We can construct the \hyperref[def:first_order_theory]{theory of groups} by adding a unary \hyperref[def:first_order_language/func]{functional symbol} \( (\anon)^{-1} \) and the axiom
    \begin{equation}\label{eq:def:group/theory/inverse_axiom}
      \qforall \xi (\xi \cdot \xi^{-1} = e \wedge \xi^{-1} \cdot \xi = e)
    \end{equation}
    to the theory of \hyperref[def:unital_magma/monoid]{monoids}.

    \thmitem{def:group/function_parity} A \hyperref[def:function]{function} \( \varphi: \mscrG \to \mscrH \) between two groups is called \term{even} if, for every \( x \in \mscrG \), we have
    \begin{equation}\label{eq:def:group/function_parity/even}
      \varphi(x^{-1}) = \varphi(x)
    \end{equation}
    and \term{odd} if
    \begin{equation}\label{eq:def:group/function_parity/odd}
      \varphi(x^{-1}) = \varphi(x)^{-1}.
    \end{equation}

    \thmitem{def:group/homomorphism} A \hyperref[def:first_order_homomorphism]{first-order homomorphism} between the groups \( \mscrG \) and \( \mscrH \) is an odd \hyperref[def:unital_magma/homomorphism]{unital magma homomorphism}.

    As shown in \fullref{thm:group_homomorphism_single_condition}, however, the conditions \eqref{eq:def:pointed_set/homomorphism} and \eqref{eq:def:group/function_parity/odd} are redundant.

    \thmitem{def:group/submodel} The set \( S \subseteq \mscrG \) is a \hyperref[def:first_order_substructure]{first-order substructure} of \( \mscrG \) if it is a \hyperref[def:unital_magma/submodel]{unital submagma} and if \( S^{-1} = S \), where
    \begin{equation*}
      S^{-1} = \set{ s^{-1} \given s \in S }.
    \end{equation*}

    \thmitem{def:group/trivial}  The \hyperref[thm:substructures_form_complete_lattice/bottom]{trivial} group is the \hyperref[def:pointed_set/trivial]{trivial pointed set} \( \set{ e } \).

    \thmitem{def:group/category} The \hyperref[def:category_of_small_first_order_models]{category of \( \mscrU \)-small models} of groups \( \ucat{Grp} \) is \hyperref[def:concrete_category]{concrete} with respect to \hyperref[def:unital_magma/monoid]{\( \ucat{Mon} \)}.

    \thmitem{def:group/exponentiation} We extend \hyperref[def:unital_magma/exponentiation]{unital magma exponentiation} to all integers by setting
    \begin{equation*}
      x^{-n} \coloneqq (x^n)^{-1}.
    \end{equation*}

    This operation behaves well as shown in \fullref{thm:def:group/properties/negative_power}.
  \end{thmenum}
\end{definition}

\begin{proposition}\label{thm:group_homomorphism_single_condition}
  A function \( \varphi: \mscrG \to \mscrH \) between the groups \( \mscrG \) and \( \mscrH \) is a \hyperref[def:group/homomorphism]{homomorphism} if and only if it satisfies \eqref{eq:def:magma/homomorphism}.
\end{proposition}
\begin{proof}
  \SufficiencySubProof \eqref{eq:def:magma/homomorphism} is required to hold by definition.

  \NecessitySubProof Let the function \( \varphi \) satisfy \eqref{eq:def:magma/homomorphism}. Then it preserves identities (i.e. is a \hyperref[def:pointed_set/homomorphism]{pointed set homomorphism}) since
  \begin{equation*}
    e_{\mscrH} \varphi(e_{\mscrG}) = \varphi(e_{\mscrG}) = \varphi(e_{\mscrG} e_{\mscrG}) = \varphi(e_{\mscrG}) \varphi(e_{\mscrG})
  \end{equation*}
  and by \fullref{thm:def:group/properties/cancellative}, the operation is cancellative.

  Inverses are preserved (i.e. \eqref{eq:def:group/function_parity/odd} holds) because
  \begin{equation*}
    \varphi(x^{-1})
    =
    \varphi(x^{-1}) e_{\mscrH}
    =
    \varphi(x^{-1}) \varphi(x) \varphi(x)^{-1}
    =
    \varphi(x^{-1} x) \varphi(x)^{-1}
    =
    e_{\mscrH} \varphi(x)^{-1}
    =
    \varphi(x)^{-1}.
  \end{equation*}

  Therefore, \( \varphi \) is indeed a group homomorphism.
\end{proof}

\begin{proposition}\label{thm:def:group/properties}
  Every \hyperref[def:group]{group} \( \mscrG \) has the following basic properties:
  \begin{thmenum}
    \thmitem{thm:def:group/properties/cancellative} The (binary) group operation is \hyperref[def:magma/cancellative]{cancellative}.
    \thmitem{thm:def:group/properties/identity_inverse} The identity \( e \) is its own inverse.
    \thmitem{thm:def:group/properties/inverse_composition} \( (xy)^{-1} = y^{-1} x^{-1} \).
    \thmitem{thm:def:group/properties/double_inverse} \( x = (x^{-1})^{-1} \)
    \thmitem{thm:def:group/properties/negative_power} For any positive integer \( n \), \( (x^n)^{-1} = (x^{-1})^n \)
  \end{thmenum}
\end{proposition}
\begin{proof}
  \SubProofOf{thm:def:group/properties/cancellative} If \( x = y \), obviously \( xz = yz \) and \( zx = zy \). Now if \( xz = yz \), we have
  \begin{equation*}
    x = x(zz^{-1}) = (xz)z^{-1} = (yz)z^{-1} = y(zz^{-1}) = y.
  \end{equation*}

  The case \( zx = zy \) is analogous.

  \SubProofOf{thm:def:group/properties/identity_inverse} \( ee = e \).
  \SubProofOf{thm:def:group/properties/inverse_composition}
  \begin{equation*}
    (xy) (y^{-1} x^{-1})
    =
    x (y y^{-1}) x^{-1}
    =
    e
    =
    y^{-1} (x^{-1} x) y
    =
    (y^{-1} x^{-1}) (xy).
  \end{equation*}

  \SubProofOf{thm:def:group/properties/double_inverse}
  \begin{equation*}
    (x^{-1})^{-1}
    =
    x x^{-1} (x^{-1})^{-1}
    =
    x.
  \end{equation*}

  \SubProofOf{thm:def:group/properties/negative_power} Using \fullref{thm:def:group/properties/double_inverse},
  \begin{equation*}
    x^{-n}
    =
    (x^n)^{-1}
    =
    x^{-1} \cdots x^{-1}
    =
    (x^{-1})^n.
  \end{equation*}
\end{proof}

\begin{example}\label{ex:power_set_is_not_a_group}
  The \hyperref[def:magma/power_set]{power set magma} \( \pow(\mscrG) \) of a group \( \mscrG \) is a monoid, but it is not a group unless \( \mscrG \) is trivial.
\end{example}

\begin{definition}\label{def:group_cosets}
  Let \( \mscrH \subseteq \mscrG \) be a subgroup of \( \mscrG \) and let \( x \in \mscrG \). The sets
  \begin{equation*}
    x \mscrH \coloneqq \set{ xh \colon h \in \mscrH }
    \quad\quad
    \mscrH x \coloneqq \set{ hx \colon h \in \mscrH }
  \end{equation*}
  are called the left and right \term{cosets} of \( \mscrH \) with respect to \( x \). The name is justified by \fullref{thm:group_coset_partition}.

  The \hyperref[def:cardinal]{cardinality} of the set of all left cosets is called the \term{index} of \( \mscrH \) and is denoted by \( [\mscrG : \mscrH] \). By \fullref{thm:lagranges_theorem_for_groups}, the index can analogously be defined as the cardinality of all right cosets.
\end{definition}

\begin{lemma}\label{thm:group_coset_partition}
  The \hyperref[def:group_cosets]{left cosets} of a subgroup of \( \mscrG \) \hyperref[def:set_partition]{partition} \( \mscrG \). The same holds for right cosets.
\end{lemma}
\begin{proof}
  To each element \( x \in \mscrG \) there corresponds a coset \( x \in x\mscrH \) (since, as a subgroup, \( \mscrH \) contains the identity).

  Two cosets \( x\mscrH \) and \( y\mscrH \) are either disjoint or equal. Indeed, if they are not disjoint, then there exists \( z \in x\mscrH \cap y\mscrH \) and thus \( z = xa = yb \) for some \( a, b \in \mscrH \). Thus,
  \begin{equation*}
    x = x a a^{-1} = y b a^{-1}
  \end{equation*}
  and since \( b a^{-1} \in \mscrH \), we have that \( x \in y\mscrH \). Furthermore, for any \( c \in \mscrH \), we have
  \begin{equation*}
    xc = y(b a^{-1} c) \in y\mscrH,
  \end{equation*}
  hence \( x\mscrH \subseteq y\mscrH \). After obtaining the converse inclusion, we conclude \( x\mscrH = y\mscrH \).
\end{proof}

\begin{lemma}\label{thm:group_coset_bijection}
  Any two left cosets in a group are \hyperref[def:equinumerosity]{equinumerous}. The same holds for right cosets.
\end{lemma}
\begin{proof}
  Let \( \mscrH \) be a subgroup of \( \mscrG \) and let \( x, y \in \mscrG \). Then \( z \mapsto y x^{-1} z \) sends \( x\mscrH \) into \( y\mscrH \). By \fullref{thm:group_multiplication_is_bijection}, this function is a bijection.
\end{proof}

\begin{definition}\label{def:normal_subgroup}
  Let \( \mscrN \) be a subgroup of \( \mscrG \). We say that \( \mscrN \) is a normal subgroup if any of the following equivalent conditions hold:
  \begin{thmenum}
    \thmitem{def:normal_subgroup/direct} For every element \( x \) of \( \mscrG \), we have the set equality
    \begin{equation}\label{eq:def:normal_subgroup/direct}
      x \mscrN x^{-1} = \mscrN.
    \end{equation}

    \thmitem{def:normal_subgroup/cosets} The partitions induced by the left and rights cosets of \( \mscrN \) coincide.
    \thmitem{def:normal_subgroup/kernel} \( \mscrN \) is the \hyperref[def:pointed_set_kernel]{kernel} of some group homomorphism.
  \end{thmenum}

  In particular, kernels are always normal subgroups.
\end{definition}
\begin{proof}
  This is the group-theoretic analog to \fullref{thm:equivalence_partition}.

  \ImplicationSubProof{def:normal_subgroup/direct}{def:normal_subgroup/cosets} For any \( x \in \mscrG \)
  \begin{equation*}
    \mscrN x = (x \mscrN x^{-1})x = x \mscrN(x^{-1}x) = x \mscrN,
  \end{equation*}
  thus every left coset is a right coset and vice versa.

  \ImplicationSubProof{def:normal_subgroup/cosets}{def:normal_subgroup/kernel} We can take the \hyperref[def:quotient_group]{canonical projection} \( \pi(x) \coloneqq x \mscrN \) as the homomorphism. The proof of correctness in \fullref{def:quotient_group} only uses \fullref{def:normal_subgroup/cosets} and therefore does not cause circular references.

  \ImplicationSubProof{def:normal_subgroup/kernel}{def:normal_subgroup/direct} Let \( \varphi: \mscrG \to \mscrH \) be a group homomorphism and fix any \( x \in \mscrG \). Denote \( \mscrN \coloneqq \ker(f) \). Then \( x \mscrN = \mscrN x \) since
  \begin{equation*}
    \varphi(x \mscrN)
    =
    \varphi(x) \varphi(\mscrN)
    =
    \varphi(x) \varphi(e_{\mscrG})
    =
    \varphi(x)
    =
    \varphi(\mscrN) \varphi(x)
    =
    \varphi(\mscrN x)
    =
    e_{\mscrH}.
  \end{equation*}

  Thus,
  \begin{equation*}
    \varphi^{-1}(e_{\mscrH}) = \mscrN = xx^{-1}\mscrN = x \mscrN x^{-1}.
  \end{equation*}
\end{proof}

\begin{definition}\label{def:quotient_group}
  Let \( \mscrG \) be a group and \( \mscrN \) be a normal subgroup of \( \mscrG \). Define the \term{quotient group}
  \begin{equation*}
    \mscrG / \mscrN \coloneqq \{ x \mscrN \colon x \in \mscrG \}
  \end{equation*}
  with the group operation
  \begin{equation*}
    x \mscrN \star y \mscrN \coloneqq xy \mscrN.
  \end{equation*}

  Define the canonical projection homomorphism
  \begin{align*}
    &\pi: \mscrG \to \mscrG / \mscrN \\
    &\pi(x) \coloneqq x \mscrN.
  \end{align*}

  The kernel of \( \pi \) is precisely \( \mscrN \).
\end{definition}
\begin{proof}
  This definition is used in the proof of equivalence in \fullref{def:normal_subgroup}. This is why it is important to only use \fullref{def:normal_subgroup/cosets} as the definition for a normal subgroup.

  We first check that the group operations is well-defined, that is, does not depend on the choice of coset representatives. Fix \( x_1, x_2 \in \mscrG \) and \( y_1, y_2 \in \mscrG \), so that
  \begin{equation*}
    x_1 \mscrN = y_1 \mscrN
  \end{equation*}
  and
  \begin{equation*}
    x_2 \mscrN = y_2 \mscrN.
  \end{equation*}

  Since the left and right cosets coincide, we have
  \begin{equation*}
    x_1 x_2 \mscrN = x_1 \mscrN x_2 = x_1 \mscrN y_2 = y_1 \mscrN y_2 = y_1 y_2 \mscrN.
  \end{equation*}

  Thus, the operation is well-defined.

  It follows from the definition that the identity is \( e \mscrN = \mscrN \) and the inverse of \( x \mscrN \) is \( x^{-1} \mscrN \). Therefore, \( \mscrG / \mscrN \) is indeed a group. The fact that \( \pi \) is a homomorphism is also part of the definition of \( \star \).

  It remains to prove that \( \mscrN = \ker(\pi) \). Obviously \( \pi(\mscrN) = \mscrN \), so \( \mscrN \subseteq \ker(\pi) \). To see that the converse holds, assume that there exists \( x \in \ker(\pi) \setminus \mscrN \), i.e. \( \pi(x) = x\mscrN = \mscrN \), but \( x \not\in N \). Then there exists \( y \in N \) such that \( xy \in N \). The subgroup \( \mscrN \) is closed under multiplication and inverses, hence \( x = xyy^{-1} \in N \). This contradicts our assumption that \( x \not\in N \). Therefore, \( \mscrN = \ker \pi \).
\end{proof}

\begin{theorem}[Lagrange's theorem for groups]\label{thm:lagranges_theorem_for_groups}
  Let \( \mscrH \) be a subgroup of \( \mscrG \). We have the following equality
  \begin{equation}\label{eq:thm:lagranges_theorem_for_groups/index}
    \card(\mscrG) = \card(\mscrH) \cdot [\mscrG : \mscrH].
  \end{equation}

  If \( \mscrH \) is a \hyperref[def:normal_subgroup]{normal subgroup}, then \( [\mscrG : \mscrH] = \card(\mscrG / \mscrH) \) and
  \begin{equation}\label{eq:thm:lagranges_theorem_for_groups/card}
    \card(\mscrG) = \card(\mscrH) \cdot \card(\mscrG / \mscrH).
  \end{equation}

  This demonstrates that there exists a bijective function between the \hyperref[def:group_direct_product]{direct product} \( \mscrH \times \mscrG / \mscrH \) and \( \mscrH \), however this may not be a group homomorphism --- see \fullref{ex:lagranges_theorem_for_groups/direct_product_zn}.
\end{theorem}
\begin{proof}
  Follows from \fullref{thm:group_coset_partition} and \fullref{thm:group_coset_bijection}
\end{proof}

\begin{theorem}[Homomorphism theorem for groups]\label{thm:homomorphism_theorem_for_groups}
  For any \hyperref[def:group/homomorphism]{group homomorphism} \( \varphi: \mscrG \to \mscrH \), we have the isomorphism
  \begin{equation*}
    \mscrG / \ker \varphi \cong \img \varphi.
  \end{equation*}
\end{theorem}
\begin{proof}
  Denote \( \mscrN \coloneqq \ker \varphi \). Define the function
  \begin{align*}
    &\psi: \img \varphi \to \pow(\mscrG) \\
    &\psi(y) \coloneqq \varphi^{-1}(y) \mscrN.
  \end{align*}

  We will show that \( \psi \) is the desired isomorphism.

  Fix any \( y \in \img \varphi \) and \( x_1, x_2 \in \varphi^{-1}(y) \). We will first show that \( x_1 \mscrN = x_2 \mscrN \). Note that
  \begin{equation*}
    \varphi(x_1^{-1} x_2)
    =
    \varphi(x_1)^{-1} \varphi(x_2)
    =
    \varphi(x_2)^{-1} \varphi(x_2)
    =
    e_{\mscrH},
  \end{equation*}
  therefore \( x_1^{-1} x_2 \in N \). Thus,
  \begin{equation*}
    x_2 \mscrN = x_1 x_1^{-1} x_2 \mscrN = x_1 \cdot \mscrN \cdot \mscrN = x_1 \mscrN.
  \end{equation*}

  Hence, \( \varphi^{-1}(y) \mscrN \) is a coset in \( \mscrG / \mscrN \) formed by any of the elements of \( \varphi^{-1}(y) \).

  Furthermore, if \( x_1 \in x_2 \mscrN \), then there exists \( n \in N \) such that
  \begin{equation*}
    x_1 = x_2 n.
  \end{equation*}

  But \( \mscrN \) is closed under taking inverses, hence
  \begin{equation*}
    x_2 = x_1 n^{-1} \in x_1 \mscrN,
  \end{equation*}
  that is,
  \begin{equation*}
    x_1 \mscrN = x_2 \mscrN.
  \end{equation*}

  This shows that \( \psi \) is injective. It is obviously surjective because if \( x \mscrN \) is a coset, then \( \varphi(x) \in \img \varphi \). Therefore, \( \varphi \) is bijective.

  It remains to show that \( \psi \) is a homomorphism. Indeed, if \( y_1, y_2 \in \img \varphi \) and
  \begin{equation*}
    x_k \in \varphi^{-1}(y_k), k = 1, 2,
  \end{equation*}
  we have
  \begin{balign*}
    \psi(y_1) \psi(y_2)
    &=
    \varphi^{-1}(y_1) \mscrN \varphi^{-1}(y_2) \mscrN
    = \\ &=
    x_1 \mscrN x_2 \mscrN
    \reloset {\eqref{eq:def:normal_subgroup/direct}} = \\ &=
    (x_1 x_2) \mscrN
    = \\ &=
    \varphi^{-1}(y_1 y_2) \mscrN
    = \\ &=
    \psi(y_1 y_2).
  \end{balign*}
\end{proof}
