\subsection{Unital magmas}\label{subsec:unital_magmas}

\begin{definition}\label{def:magma_identity}
  An element \( e \) of a magma \( \CM \) is called a \Def{left identity} (resp. \Def{right identity}) if
  \begin{align}\label{eq:def:magma_identity}
    ex = x \T{for all} x \in \CM
    &&
    (\text{resp. } xe = x \T{for all} x \in \CM)
  \end{align}

  If \( e \) is simultaneously a left and right identity, we call a \Def{two-sided identity} or simply \Def{identity} of \( \CM \).
\end{definition}

\begin{proposition}\label{thm:magma_identity_unique}
  The two-sided \hyperref[def:magma_identity]{magma identity} \( e \) in a magma is unique.
\end{proposition}
\begin{proof}
  If \( f \) is another identity, then \( e = ef = f \).
\end{proof}

\begin{definition}\label{def:unital_magma}
  A \hyperref[def:magma]{magma} with an \hyperref[def:magma_identity]{identity} is called \Def{unital}. This makes it a \hyperref[def:pointed_set]{pointed set}. We can consider it as a pair \( (\CM, \cdot) \) rather than a triple \( (\CM, \cdot, e) \) because, by \fullref{thm:magma_identity_unique}, a two-sided identity is uniquely determined by the magma operation.

  \begin{DefEnum}
    \ILabel{def:unital_magma/theory} The theory of unital magmas consists of the axioms \eqref{eq:def:magma_identity} over the intersection of the languages of \hyperref[def:pointed_set/theory]{pointed sets} and \hyperref[def:magma/theory]{magmas}

    \ILabel{def:unital_magma/homomorphism} A \hyperref[def:first_order_homomorphism]{homomorphism} between unital magmas is a function that satisfies both \eqref{eq:def:pointed_set/homomorphism} and \eqref{eq:def:magma/homomorphism}.

    \ILabel{def:unital_magma/category} The \hyperref[def:first_order_model_category]{model category} \( \Cat{Mag}_* \) of unital magmas is \hyperref[def:concrete_category]{concrete} with respect to both \hyperref[def:pointed_set/category]{\( \Cat{Set}_* \)} and \hyperref[def:magma/category]{\( \Cat{Mag} \)}.

    \ILabel{def:unital_magma/substructure} The set \( S \subseteq \CM \) is a \hyperref[def:first_order_substructure]{unital submagma} of \( \CX \) if it is a \hyperref[def:magma/substructure]{submagma} and if \( e \in S \).

    \ILabel{def:unital_magma/trivial} A \Def{trivial unital magma} is a set \( \{ e \} \) with the \hyperref[def:function/diagonal]{identity} operation. Since it is unique up to an isomorphism, we usually speak of \enquote{the} trivial unital magma. It is the \hyperref[def:zero_objects/initial]{initial object} in \( \Cat{Mag}_* \).

    \ILabel{def:unital_magma/associative} An \hyperref[eq:def:magma/associative]{associative} unital magma is called a \Def{monoid}. The category \( \Cat{Mon} \) of monoids is a full subcategory of \( \Cat{Mag}_* \).

    \ILabel{def:unital_magma/exponentiation} We extend \hyperref[def:magma/exponentiation]{magma exponentiation} to all nonnegative integers by setting
    \begin{equation*}
      x^0 \coloneqq e.
    \end{equation*}

    \ILabel{def:unital_magma/power_set} The \hyperref[def:magma/power_set]{power set magma} \( \Pow(\CM) \) of a unital magma \( \CM \) with identity \( e \) is again a unital magma with identity \( \{ e \} \).
  \end{DefEnum}
\end{definition}

\begin{definition}\label{def:unital_magma_kernel}
  The \Def{kernel} \( \ker(\varphi) \) of a unital magma homomorphism \( \varphi: \CM \to \CN \) is the zero \hyperref[def:zero_locus]{locus} of \( \varphi \), that is, \hyperref[def:function/preimage]{preimage} \( \varphi^{-1}(e_{\CN}) \).

  It is an instance of \hyperref[def:categorical_kernel]{categorical kernels} in \hyperref[def:category_of_sets]{\( \Cat{Set} \)}. Formally, it is the \hyperref[thm:set_categorical_limits/equalizer]{equalizer} of \( \varphi \) and the constant homomorphism \( \psi(x) \coloneqq e_{\CN} \).
\end{definition}

\begin{proposition}\label{thm:unital_magma_kernel_is_submagma}
  The \hyperref[def:unital_magma_kernel]{kernel} of a unital magma homomorphism \( \varphi: \CM \to \CN \) is a \hyperref[def:first_order_substructure]{unital submagma} of \( \CM \).
\end{proposition}
\begin{proof}
  By \eqref{eq:def:pointed_set/homomorphism}, \( e_{\CM} \in \ker(\varphi) \), therefore \( \ker(\varphi) \) inherits its unital magma structure from \( \CM \). It remains only to show that it is closed under the magma operation. But this is trivial since, if \( x, y \in \ker(\varphi) \), then
  \begin{equation*}
    \varphi(xy) = \varphi(x) \varphi(y) = e_{\CN} e_{\CN} = e_{\CN}.
  \end{equation*}
\end{proof}
