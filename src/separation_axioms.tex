\subsection{Separation axioms}\label{subsec:separation_axioms}

\begin{definition}\label{def:topological_space_separation}
  Two subsets \( A, B \subseteq X \) of a topological space \( (X, \mscrT) \) are called \term{separated} or \term{separated using neighborhoods} if there exist disjoint open sets \( U \supseteq A \) and \( V \supseteq B \). In particular, two points are separated if their respective singleton sets are separated.

  We say that \( A \) and \( B \) are \term{functionally separated} if there exists a continuous function \( f: X \to [0, 1] \) such that \( f(A) = 0 \) and \( f(B) = 1 \).
\end{definition}

\begin{definition}\label{def:separation_axioms}
  We can classify topological spaces using the following separation axioms. Fix a topological space \( (X, \mathcal{T}) \).

  \begin{thmenum}
    \thmitem[def:separation_axioms/T0]{T0} (Kolmogorov) \( X \) is \( T_0 \) if for every two different points \( x, y \in X \), there exists an open set \( U \in \mathcal{T} \) such that either \( x \in U \) or \( y \in U \).
    \thmitem[def:separation_axioms/T0.5]{T0.5} \( X \) is \( T_{0.5} \) if every singleton set \( \{ x \} \) is either open or closed.
    \thmitem[def:separation_axioms/T1]{T1} (Frechet) \( X \) is \( T_1 \) if every singleton set \( \{ x \} \) is closed.
    \thmitem[def:separation_axioms/T2]{T2} (Hausdorff) \( X \) is \( T_2 \) if every two different points \( x, y \in X \) can be separated using neighborhoods, i.e. there exist disjoint open sets \( U \ni x \) and \( V \ni y \).

    \thmitem[def:separation_axioms/T3]{T3} \( X \) is \term{regular} if every point and every closed set can be separated using \hyperref[def:topological_space_separation]{neighborhoods}.

    If in addition to being regular \( X \) is \ref{def:separation_axioms/T0}, we say that \( X \) is a \( T_3 \) space.

    \thmitem[def:separation_axioms/T3.5]{T3.5} (Tychonoff) \( X \) is \term{completely regular} if every point and every closed set can be functionally \hyperref[def:topological_space_separation]{separated}.

    If in addition to being completely regular \( X \) is \ref{def:separation_axioms/T0}, we say that \( X \) is a \( T_{3.5} \) space.

    \thmitem[def:separation_axioms/T4]{T4}(Urysohn) \( X \) is \term{normal} every two closed sets \( F, G \in \mathcal{F}_{\mathcal{T}} \) can be separated using neighborhoods, i.e. there exist disjoint open sets \( U \supseteq F \) and \( V \supseteq G \).

    If in addition to being normal \( X \) is \ref{def:separation_axioms/T1}, we say that \( X \) is a \( T_4 \) space.

    \thmitem[def:separation_axioms/T5]{T5} If every subspace of a \( T_4 \) space \( X \) is \ref{def:separation_axioms/T4}, we say that \( X \) is a \( T_5 \) space or a \term{completely normal space}.

    \thmitem[def:separation_axioms/T6]{T6} If every closed set in a \( T_4 \) space \( X \) is \( G_\delta \) (see \fullref{def:borel_algebra}), we say that \( X \) is a \( T_6 \) space or a \term{perfectly normal space}.
  \end{thmenum}
\end{definition}

\begin{proposition}\label{thm:separation_axioms_cascade}
  Each numbered axiom in \fullref{def:separation_axioms} implies the previous one.
\end{proposition}

\begin{proposition}\label{thm:t2_iff_singleton_limits}
  A topological space is \hyperref[def:separation_axioms/T2]{Hausdorff} if and only if every \hyperref[def:topological_net]{net} has at most one \hyperref[def:net_convergence/limit]{limit}.
\end{proposition}
\begin{proof}
  \SufficiencySubProof Let \( X \) be Hausdorff and assume that there exists a net \( \{ x_k \}_{k \in \mscrK} \) such that \( y \) and \( z \) are not necessarily distinct limit points.

  Fix neighborhoods \( U \) of \( y \) and \( V \) of \( z \). Since both are limit points, there exist indices \( k_U \) and \( k_V \) such that \( k \geq k_U \) implies \( x_k \in U \) and \( i \geq i_k \) implies \( x_k \in V \).

  Since \( \mscrK \) is a directed set, there exists an upper bound \( k_0 \) of \( k_U \) and \( k_V \). Thus,
  \begin{equation*}
    x_k \in U \cap V \quad\forall k \geq k_0.
  \end{equation*}

  In particular, the intersection \( U \cap V \) is nonempty and is a neighborhood of both \( y \) and \( z \).

  If \( y \neq z \), then we have two distinct points such that no two neighborhoods of \( y \) and \( z \), respectively, are disjoint. This contradicts the assumption that \( X \) is Hausdorff. Thus \( y = z \).

  \NecessitySubProof Conversely, if \( X \) is not Hausdorff, then for every two distinct points \( y \) and \( z \) and every two neighborhoods \( U \ni y \) and \( V \ni z \), their intersection \( U \cap V \) is nonempty.

  Let \( \mathcal{U} \) and \( \mathcal{V} \) be the sets of all neighborhoods of \( y \) and \( z \), respectively. Since they are both partially ordered by set inclusion \( \subseteq \), define the directed set \( (\mathcal{U} \times \mathcal{V}, \leq) \) with order
  \begin{equation*}
    (U, V) \leq (U', V') \iff U \supset V \T{and} U' \supset V'.
  \end{equation*}

  For each \( (U, V) \in \mathcal{U} \times \mathcal{V} \), choose a point \( x_{(U, V)} \) from \( U \cap V \).

  Thus the net \( \{ x_{(U, V)} \}_{(U, V) \in \mathcal{U} \cap \mathcal{V}} \) has both \( y \) and \( z \) as its limit points, which contradicts our initial assumption.
\end{proof}

\begin{lemma}[Urysohn's lemma]\label{thm:urysohns_lemma}\mcite[1.5.11]{Engelking1989}
  In a \hyperref[def:separation_axioms/T4]{normal space}, every pair \( A, B \) of disjoint closed sets can be functionally \hyperref[def:topological_space_separation]{separated}.
\end{lemma}

\begin{theorem}\label{thm:separation_axioms_of_product}
  Fix is an indexed family \( \{ X_k \}_{k \in \mscrK} \) of topological spaces. Denote their \hyperref[def:topological_product]{product} by \( X \).

  \begin{thmenum}
    \thmitem{thm:separation_axioms_of_product/direct}\cite[theorem 2.3.11]{Engelking1989} If each one of \( X_k \) is a \( T_i \) space for \ref{def:separation_axioms/T0}-\ref{def:separation_axioms/T3.5}, then \( X \) is also a \( T_i \) space.

    \thmitem{thm:separation_axioms_of_product/inverse}\cite[theorem 2.3.11]{Engelking1989} If \( X \) is a \( T_i \) space for \ref{def:separation_axioms/T0}-\ref{def:separation_axioms/T6}, then each component \( X_k \) is also a \( T_i \) space.
  \end{thmenum}
\end{theorem}
