\subsection{Separation axioms}\label{subsec:separation_axioms}

\begin{definition}\label{def:topological_space_separation}
  Two subsets \( A, B \subseteq X \) of a topological space \( (X, \T) \) are called \Def{separated} or \Def{separated using neighborhoods} if there exist disjoint open sets \( U \supseteq A \) and \( V \supseteq B \). In particular, two points are separated if their respective singleton sets are separated.

  We say that \( A \) and \( B \) are \Def{functionally separated} if there exists a continuous function \( f: X \to [0, 1] \) such that \( f(A) = 0 \) and \( f(B) = 1 \).
\end{definition}

\begin{definition}\label{def:separation_axioms}
  We can classify topological spaces using the following separation axioms. Fix a topological space \( (X, \Cal{T}) \).

  \begin{description}
    \DItem{def:separation_axioms/T0}[T0] (Kolmogorov) \( X \) is \( T_0 \) if for every two different points \( x, y \in X \), there exists an open set \( U \in \Cal{T} \) such that either \( x \in U \) or \( y \in U \).
    \DItem{def:separation_axioms/T0.5}[T0.5] \( X \) is \( T_{0.5} \) if every singleton set \( \{ x \} \) is either open or closed.
    \DItem{def:separation_axioms/T1}[T1] (Frechet) \( X \) is \( T_1 \) if every singleton set \( \{ x \} \) is closed.
    \DItem{def:separation_axioms/T2}[T2] (Hausdorff) \( X \) is \( T_2 \) if every two different points \( x, y \in X \) can be separated using neighborhoods, i.e. there exist disjoint open sets \( U \ni x \) and \( V \ni y \).
    \DItem{def:separation_axioms/regular}[Regular] \( X \) is \Def{regular} if every point \( x \in X \) and every closed set \( F \in \Cal{F}_{\Cal{T}} \) can be separated using neighborhoods\Tinyref{def:topological_space_separation}.
    \DItem{def:separation_axioms/T3}[T3] \( X \) is \( T_3 \) if it is \ref{def:separation_axioms/T0} and \ref{def:separation_axioms/regular}.
    \DItem{def:separation_axioms/completely_regular}[Completely regular] \( X \) is \Def{completely regular} if every point \( x \in X \) and every closed set \( F \in \Cal{F}_{\Cal{T}} \) can be functionally separated\Tinyref{def:topological_space_separation}.
    \DItem{def:separation_axioms/T3.5}[T3.5] \( X \) if \( T_{3.5} \) the space is \ref{def:separation_axioms/T0} and \ref{def:separation_axioms/completely_regular}.
    \DItem{def:separation_axioms/normal}[Normal] (Urysohn) \( X \) is \Def{normal} every two closed sets \( F, G \in \Cal{F}_{\Cal{T}} \) can be separated using neighborhoods, i.e. there exist disjoint open sets \( U \supseteq F \) and \( V \supseteq G \).
    \DItem{def:separation_axioms/T4}[T4] \( X \) is \( T_4 \) if the space is \ref{def:separation_axioms/T1} and \ref{def:separation_axioms/normal}.
  \end{description}
\end{definition}

\begin{proposition}\label{thm:separation_axioms_cascade}
  Each numbered axiom in \cref{def:separation_axioms} implies the previous one.
\end{proposition}

\begin{proposition}\label{thm:t2_iff_singleton_limits}
  The space \( (X, \Cal{T}) \) is Hausdorff \( T_2 \)\Tinyref{def:separation_axioms/T2} if and only if every net\Tinyref{def:topological_net} has at most one limit\Tinyref{def:net_limit_point}.
\end{proposition}
\begin{proof}
  \begin{description}
    \Implies Let \( X \) be Hausdorff and assume that there exists a net \( \{ x_i \}_{i \in I} \) such that \( y \) and \( z \) are not necessarily distinct limit points.

    Fix neighborhoods \( U \) of \( y \) and \( V \) of \( z \). Since both are limit points, there exist \( i_U \) and \( i_V \) such that \( i \geq i_U \) implies \( x_i \in U \) and \( i \geq i_V \) implies \( x_i \in V \).

    Since \( I \) is a directed set, there exists an upper bound \( i_0 \) of \( i_U \) and \( i_V \). Thus,
    \begin{equation*}
      i \geq i_0 \implies x_i \in U \cap V.
    \end{equation*}

    In particular, the intersection \( U \cap V \) is nonempty and is a neighborhood of both \( y \) and \( z \).

    If \( y \neq z \), then we have two distinct points such that no two neighborhoods of \( y \) and \( z \), respectively, are disjoint. This contradicts the assumption that \( X \) is Hausdorff. Thus\LEM \( y = z \).

    \ImpliedBy Conversely, if \( X \) is not Hausdorff\LEM, then for every two distinct points \( y \) and \( z \) and every two neighborhoods \( U \ni y \) and \( V \ni z \), their intersection \( U \cap V \) is nonempty.

    Let \( \Cal{U} \) and \( \Cal{V} \) be the sets of all neighborhoods of \( y \) and \( z \), respectively. Since they are both partially ordered by set inclusion \( \subseteq \), define the directed set \( (\Cal{U} \times \Cal{V}, \leq) \) with order
    \begin{equation*}
      (U, V) \leq (U', V') \iff U \supset V \land U' \supset V'.
    \end{equation*}

    For each \( (U, V) \in \Cal{U} \times \Cal{V} \), choose\AOC a point \( x_{(U, V)} \) from \( U \cap V \).

    Thus the net \( \{ x_{(U, V)} \}_{(U, V) \in \Cal{U} \cap \Cal{V}} \) has both \( y \) and \( z \) as its limit points, which contradicts our initial assumption.
  \end{description}
\end{proof}
