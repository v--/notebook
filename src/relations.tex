\begin{definition}\label{def:relations}
  Let $\{ X_i \}_{i \in I}$ be a family of nonempty sets.
  Subsets of the form
  \begin{align*}
    \sim\; \subseteq \times_{i \in I} X_i
  \end{align*}
  are called \uline{relations}. If $I$ is a finite set of cardinality $n$, the relation is called \uline{n-ary (binary for $n = 2$, ternary for $n = 3$)}. If all $X_i$ are the same set, we say that \uline{$\sim$ is relation on X}.

  It is customary to write binary relations as $a \sim b$ instead of $(a, b) \in \sim$.
\end{definition}

\begin{definition}\label{def:orders}
  We will consider binary relations $\sim\; \subseteq X \times X$ on a nonempty set $X$.

  \begin{defenum}
    \item\label{def:order/strict_partial} The relation $<$ is called a \uline{strict partial order} if
    \begin{description}
      \DItem{Antireflexivity}{def:order/strict_partial/antireflexivity} $\lnot(a < a)$
      \DItem{Transitivity}{def:order/strict_partial/transitivity} $a < b \land b < c \implies a < c$
    \end{description}

    The binary relation $>$ is defined as $a > b \iff b < a$. Strict partial orders are rarely used compared to nonstrict partial orders (\cref{def:order/partial}).

    If any two elements are in a relation, we call the strict partial order a \uline{strict total order} or a \uline{strict linear order}.

    \item\label{def:order/preorder} The relation $\sim$ is called a \uline{preorder} if:
    \begin{description}
      \DItem{Antireflexivity}{def:order/preorder/reflexivity} $a \sim a$
      \DItem{Transitivity}{def:order/preorder/transitivity} $a \sim b \land b \sim c \implies a \sim c$
    \end{description}

    \item\label{def:order/equivalence} The preorder $=$ is called an \uline{equivalence relation} if
    \begin{description}
      \DItem{Reflexivity}{def:order/equivalence/reflexivity} $a = a$
      \DItem{Symmetry}{def:order/equivalence/symmetry} $a = b \implies b = a$
      \DItem{Transitivity}{def:order/equivalence/transitivity} $a = b \land b = c \implies a = c$
    \end{description}

    \item\label{def:order/partial} Assuming we have defined a notion of equality in our formal language, the preorder $\leq$ is called a \uline{(nonstrict) partial order} if
    \begin{description}
      \DItem{Reflexivity}{def:order/partial/reflexivity} $a \leq a$
      \DItem{Antisymmetry}{def:order/partial/symmetry} $a \leq b \land b \leq a \implies a = b$
      \DItem{Transitivity}{def:order/partial/transitivity} $a \leq b \land b \leq c \implies a \leq c$
    \end{description}

    The binary relation $\geq$ is defined as $a \geq b \iff b \leq a$.

    If any two elements are in a relation, we call the partial order a \uline{total order} or a \uline{linear order}.

    A set with a partial order is called a~\uline{partially ordered set or poset}.
  \end{defenum}
\end{definition}
