\subsection{Relations}\label{subsec:relations}

\begin{definition}\label{def:relation}
  Let \( \{ X_i \}_{i \in I} \) be a family of nonempty sets.
  Subsets of the form
  \begin{equation*}
    \sim\; \subseteq \times_{i \in I} X_i
  \end{equation*}
  are called \textbf{relations}. If \( I \) is a finite set of cardinality \( n \)\Tinyref{note:cardinals}, the relation is called \textbf{n-ary (binary for \( n = 2 \), ternary for \( n = 3 \))}. If all \( X_i \) are the same set, we say that \textbf{\( \sim \) is relation on X}.

  It is customary to write binary relations as \( a \sim b \) instead of \( (a, b) \in \sim \).
\end{definition}

\begin{note}\label{note:main_relation_types}
  We are mostly interested in either orders (see \cref{subsec:orders}) or functions (see \cref{subsec:functions}).
\end{note}

\begin{definition}\label{def:derived_relations}
  Let \( \sim \) be a binary relation on \( X \). Define
  \begin{defenum}
    \DItem{def:derived_relations/converse} the \textbf{converse of \( \sim \)} as
    \begin{equation*}
      (\sim)^{-1} \coloneqq \sim \cup \{ (y, x) \colon (x, y) \in \sim \}.
    \end{equation*}

    \DItem{def:derived_relations/reflexive} the \textbf{reflexive closure of \( \sim \)} as
    \begin{equation*}
      (\sim)^R \coloneqq \sim \cup \{ (x, x) \colon x \in X \}.
    \end{equation*}

    \DItem{def:derived_relations/symmetric} the \textbf{symmetric closure of \( \sim \)} as
    \begin{equation*}
      (\sim)^S \coloneqq \sim \cup \sim^{-1}.
    \end{equation*}

    \DItem{def:derived_relations/transitive} the \textbf{transitive closure of \( \sim \)} as
    \begin{equation*}
      (\sim)^T \coloneqq \bigcap \{ \approx \colon \approx \text{ is a binary transitive relation on \( X \) such that} \sim \subseteq \approx \}.
    \end{equation*}
  \end{defenum}
\end{definition}
\begin{proof}
\begin{itemize}
  Let \( (\sim) \) be an arbitrary binary relation on \( X \).

  \begin{description}
    \RItem{def:derived_relations/reflexive} The reflexive closure of \( \sim \) is reflexive since it contains the diagonal relation \( \{ (x, x) \colon x \in X \} \).

    \RItem{def:derived_relations/symmetric} The symmetric closure of \( \sim \) is symmetric since \( x \;(\sim)^S y \iff y \;(\sim)^S x \).

    \RItem{def:derived_relations/transitive} The transitive closure of \( \sim \) is transitive as an intersection of transitive relations.
  \end{description}
\end{itemize}
\end{proof}
