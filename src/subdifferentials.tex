\subsection{Subdifferentials}\label{subsec:subdifferentials}

Let \( X \) be a Hausdorff \hyperref[def:topological_vector_space]{topological vector space}, let \( D \subseteq X \) be an open set and \( f: D \to \BbbR \) be any function.

\begin{definition}\label{def:subdifferentials}
  We fix a point \( x \in D \). We define different types of \term{subgradients} and \term{subdifferentials}. Subgradients are linear functionals \( x^* \in X^* \) that approximate \( f \) at the point \( x \) in a certain way, and a subdifferential is the set of all subgradients of a given type.

  \begin{defenum}
    \ilabel{def:subdifferentials/convex}\mcite[59]{Clarke2013}We say that \( x^* \in X^* \) is a \term{subgradient of \( f \) at \( x \)} if for every \( y \in D \) we have
    \begin{equation*}
      f(y) - f(x) \geq \inner {x^*} {y - x}.
    \end{equation*}

    The \term{subdifferential of \( f \) at \( x \)} is denoted by \( \partial f(x) \) and is also sometimes called the \term{convex subdifferential} because of \fullref{thm:convex_iff_subdifferential_nonempty}.

    \ilabel{def:subdifferentials/clarke}\mcite[def. 10.3]{Clarke2013}We say that \( x^* \in X^* \) is a \term{Clarke (generalized) subgradient of \( f \) at \( x \)} if for every direction \( h \in X \) we have
    \begin{equation*}
      f^\circ(x)(h) \geq \inner {x^*} h,
    \end{equation*}
    where \( f^\circ(x)(h) \) is the generalized Clarke \hyperref[def:nonsmooth_derivatives/clarke]{derivative}.

    The \term{subdifferential of \( f \) at \( x \)} is denoted by \( \partial_C f(x) \). Confusingly, the Clarke subdifferential is called the \enquote{generalized gradient} by Clarke himself with no special name for the Clarke subgradients.

    See \fullref{subsec:clarke_gradients} for properties of these subgradients.

    \ilabel{def:subdifferentials/proximal}\mcite[227]{Clarke2013}We say that \( x^* \in X^* \) is a \term{proximal subgradient of \( f \) at \( x \)} if there exist \( \sigma > 0 \) and a neighborhood \( V \subseteq X \) of \( x \) such that for every \( y \in D \cap V \) we have
    \begin{equation*}
      f(y) - f(x) + \sigma \norm{y - x}^2 \geq \inner {x^*} {y - x}.
    \end{equation*}

    The \term{proximal subdifferential of \( f \) at \( x \)} is denoted by \( \partial_P f(x) \).

    \ilabel{def:subdifferentials/limiting}\mcite[def. 11.10]{Clarke2013}Suppose the following are satisfied:
    \begin{enumerate}
      \item \( \{ x_n \}_n \subseteq D \) is a sequence of points converging to \( x \)
      \item \( f(x_n) \to f(x) \) (redundant if \( f \) is continuous)
      \item \( x_n^* \) is a proximal subgradient for \( f \) at \( x_n \) for every \( n \in \BbbZ_{>0} \).
    \end{enumerate}

    If the limit \( x^* \coloneqq \lim_n x_n^* \) exists and is a continuous linear functional, we call \( x^* \) a \term{limiting subgradient of \( f \) at \( x \)}.

    The \term{limiting subdifferential of \( f \) at \( x \)} is denoted by \( \partial_P f(x) \).
  \end{defenum}
\end{definition}
