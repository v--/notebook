\section{Order theory}\label{sec:order_theory}

\term{Orders} are special \hyperref[def:binary_relation]{binary relations} which, surprisingly, are used to compare elements in a \hyperref[def:set_zfc]{set}. Order theory studies pairs \( (\mscrX, \leq) \), where \( \mscrX \) is a set and \( \leq \) is an order.

\begin{remark}\label{rem:order_infix_notation}
  We denote orders using symbols rather than letters because it is customary to write orders using infix notation \( a \leq b \) rather than \( (a, b) \in {}\leq{} \).
\end{remark}

We are interested in the following ordered sets:
\begin{center}
  \synttree
    [
      {\hyperref[def:preordered_set]{Preordered sets}}
        [{\hyperref[def:directed_set]{Directed sets}}]
        [
          {\hyperref[def:poset]{Partially ordered sets}}
            [
              {\hyperref[def:totally_ordered_set]{Totally ordered set}}
                [{\hyperref[def:well_ordered_set]{Well-ordered set}}]
            ]
            [
              {\hyperref[def:semilattice]{Semilattices}}
                [
                  {\hyperref[def:semilattice/lattice]{Lattices}}
                    [
                      {\hyperref[def:heyting_algebra]{Heyting algebras}}
                      [{\hyperref[def:boolean_algebra]{Boolean algebras}}]
                    ]
                ]
            ]
        ]
        [{\hyperref[def:equivalence_relation]{Partition (set with equivalence relation)}}]
    ]
\end{center}

General (semi)lattices also admit algebraic definitions, however these algebraic descriptions have some drawbacks:
\begin{itemize}
  \item There is no general way to extend algebraic operations from finitary to infinitary --- see \fullref{thm:binary_lattice_operations} for how this can be done in some lattices.

  \item We often implicitly rely on the order structure, for example in \hyperref[def:heyting_algebra]{the definition for Heyting algebras}.
\end{itemize}
