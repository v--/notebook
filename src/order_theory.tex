\section{Order theory}\label{sec:order_theory}

\term{Orders} are special \hyperref[def:binary_relation]{binary relations} which, surprisingly, are used to compare elements in a \hyperref[def:set]{set}. Order theory studies pairs \( X = (X, \leq) \), where \( X \) is a set and \( \leq \) is a \hyperref[def:preordered_set]{preorder}.

We denote orders using symbols rather than letters because it is customary to write orders using \hyperref[rem:first_order_formula_conventions/infix]{infix notation}, e.g. \( a \leq b \) rather than \( (a, b) \in {\leq} \).

\begin{figure}[!ht]
  \caption{Hierarchy of ordered sets}\label{fig:ordered_sets_hierarchy}
  \smallskip
  \hfill
  \begin{forest}
    [
      {\hyperref[def:preordered_set]{Preordered sets}}
        [{\hyperref[def:directed_set]{Directed sets}}]
        [
          {\hyperref[def:partially_ordered_set]{Partially ordered sets}}
            [
              {\hyperref[def:totally_ordered_set]{Totally ordered set}}
              [{\hyperref[def:well_ordered_set]{Well-ordered set}}]
            ]
            [
              {\hyperref[def:semilattice]{Semilattices}}
                [
                  {\hyperref[def:semilattice/lattice]{Lattices}}
                    [
                      {\hyperref[def:heyting_algebra]{Heyting algebras}}
                      [{\hyperref[def:boolean_algebra]{Boolean algebras}}]
                    ]
                ]
            ]
        ]
        [{\hyperref[def:equivalence_relation]{Equivalence partition}}]
      ]
  \end{forest}
  \hfill\hfill
\end{figure}

\Fullref{subsec:well_ordered_sets} is focused on set theory, and hence we have included it in \fullref{sec:set_theory} rather than here.

General (semi)lattices also admit algebraic definitions, however these algebraic descriptions have some drawbacks:
\begin{itemize}
  \item There is no general way to extend algebraic operations from finitary to infinitary. This can sometimes be circumvented if the order happens to carry more information about the partially ordered set than the algebraic operations --- see \fullref{thm:binary_lattice_operations/new_lattice}.

  \item We often implicitly rely on the order structure, for example in \hyperref[def:heyting_algebra]{the definition for Heyting algebras}.
\end{itemize}
