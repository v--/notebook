\subsection{Cardinals}\label{subsec:cardinals}

\begin{definition}\label{def:set_domination}\mcite[145]{Enderton1977Sets}
  We say that the set \( A \) is \term{dominated} by \( B \) and that \( B \) \term{dominates} \( A \) if there exists an \hyperref[def:function_invertibility/injective]{injection} from \( A \) to \( B \).
\end{definition}

\begin{definition}\label{def:equinumerous_sets}\mcite[129]{Enderton1977Sets}
  We say that the sets \( A \) and \( B \) are \term{equinumerous} if there exists a \hyperref[def:function_invertibility/bijective]{bijection} between \( A \) and \( B \).
\end{definition}

\begin{proposition}\label{thm:finite_ordinals_are_different_carinals}
  Let \( n \) and \( m \) be members of \( \omega \). The sets \( n \) and \( m \) are equinumerous if and only if \( n = m \).
\end{proposition}

\begin{definition}\label{def:infinite_set}
  In order to distinguish between finite and infinite sets, we introduce the following equivalent dual pairs definitions:
  \begin{thmenum}
    \thmitem{def:infinite_set/ordinals} We call a set \term{finite} if it is \hyperref[def:equinumerous_sets]{equinumerous} with a natural number (member of \hyperref[thm:smallest_inductive_set_existence]{\( \omega \)}) and \term{infinite} otherwise.

    \thmitem{def:infinite_set/omega} We call a set \term{infinite} if it \hyperref[def:set_domination]{dominates} the set \( \BbbN \) (encoded as \( \omega \)) and \term{finite} otherwise.

    \thmitem{def:infinite_set/dedekind} We call a set \term{Dedekind infinite} if it is equinumerous with a proper subset of itself and \term{Dedekind finite} otherwise.
  \end{thmenum}
\end{definition}
\begin{proof}
  Fix some set \( A \).

  \ImplicationSubProof{def:infinite_set/ordinals}{def:infinite_set/omega} Suppose that there exists a bijective function \( f: n \to A \) for some number \( n \in \omega \).

  Let \( g: \omega \to A \) be an arbitrary function. Suppose that it is injective. Let \( k < n \) and \( m > n \). If \( g(m) = g(k) \), then \( m = k \), which contradicts our choice of \( k \) and \( n \). Hence \( g \) cannot be injective.

  We conclude that there does not exist an injective function from \( \omega \) to \( A \).

  \ImplicationSubProof{def:infinite_set/omega}{def:infinite_set/ordinals} Let
\end{proof}

\begin{theorem}[Well-Ordering Principle]\label{thm:well_ordering_principle}\mcite[196]{Enderton1977Sets}
  Any \hyperref[def:set]{set} can be \hyperref[def:well_ordered_set]{well-ordered}.

  This theorem is equivalent to \fullref{thm:axiom_of_choice_equivalences}.
\end{theorem}

\begin{theorem}[Cantor-Schröder-Bernstein]\label{thm:cantor_schroder_bernstein}\mcite[147]{Enderton1977Sets}
  If \( \abs{X} \leq \abs{Y} \) and \( \abs{Y} \leq \abs{X} \), then \( X \cong Y \).
\end{theorem}

\medskip

\begin{proposition}\label{thm:equinumerousity_equivalence}\mcite[thm. 6A]{Enderton1977Sets}
  \hyperref[def:equinumerous_sets]{Equinumerosity} satisfies the equivalence relation \hyperref[def:equivalence_relation]{axioms} (however it is formally not an equivalence relation since we cannot define relations on the class of all sets; see \fullref{def:set}).
\end{proposition}

\begin{theorem}[Cantor]\label{thm:cantor_power_set_theorem}\mcite[thm. 6B]{Enderton1977Sets}
  No set \( X \) is equinumerous with its power set \( \pow(X) \).
\end{theorem}
\begin{proof}
  Fix some function \( f: X \to \pow(X) \). Define the set
  \begin{equation*}
    Y \coloneqq \{ x \in X \colon x \not\in f(x) \}.
  \end{equation*}

  Note that \( Y \subseteq X \) and thus \( Y \in \pow(X) \), however \( Y \) is not in the \hyperref[def:function]{image} \( \img f \) and thus \( f \) is not a \hyperref[def:function_invertibility/surjective]{surjection}.

  Since \( f \) was arbitrary, we conclude that no function \( f: X \to \pow(X) \) is a surjection and, hence, \( X \not\cong \pow(X) \).
\end{proof}

\begin{definition}\label{def:finite_set}\mcite[133]{Enderton1977Sets}
  A set \( A \) is \term{finite} if it is equinumerous to a natural number as defined in \fullref{def:set_of_natural_numbers} (using the convention that \( \varnothing \) corresponds to zero).

  If \( A \) is not finite, we say that it is \term{infinite}.
\end{definition}

\begin{proposition}\label{thm:infinite_set_iff_equinumerous_to_proper_subset}\mcite[cor. 6D]{Enderton1977Sets}
  A set is \hyperref[def:finite_set]{infinite} if and only if it is equinumerous to a proper subset of itself.
\end{proposition}

\medskip

\begin{theorem}\label{thm:equinumerous_ordinal_existence}\mcite[197]{Enderton1977Sets}
  For every set, there exists at least one \hyperref[def:ordinal]{ordinal} equinumerous to it.
\end{theorem}

\medskip

\begin{definition}\label{def:cardinal}\mcite[197]{Enderton1977Sets}
  For each set \( A \), define its \term{cardinal} or \term{cardinal number} \( \card A \) as the smallest ordinal that is equinumerous to \( A \) (a smallest ordinal always exists by \fullref{def:ordinal}).

  If \( \xi \) and \( \eta \) are cardinal numbers, we define \( \xi \leq \eta \) to mean that \( \eta \) \hyperref[def:set_domination]{dominates} \( \xi \), i.e.
  \begin{equation*}
    \xi \leq \eta \iff \abs{\xi} \leq \abs{\eta}.
  \end{equation*}
\end{definition}

\begin{theorem}\label{thm:cardinal_trichotomy}\mcite[thm. 6M(5)]{Enderton1977Sets}
  If \( \xi \) and \( \eta \) are cardinals, then either
  \begin{itemize}
    \item \( \xi \leq \eta \)
    \item \( \xi = \eta \)
    \item \( \xi \geq \eta \)
  \end{itemize}
\end{theorem}

\begin{corollary}[Pigeonhole principle]\label{def:pigeonhole_principle}
  If the cardinality of \( X \) is greater than the cardinality of \( Y \), then there exists no injective function from \( X \) to \( Y \).
\end{corollary}

\begin{remark}\label{rem:cardinals}
  We can think of cardinal numbers as \enquote{choosing} a special set out of the equivalence classes obtained from \fullref{thm:equinumerousity_equivalence}.

  Since the natural numbers as defined in \fullref{def:set_of_natural_numbers} are ordinals and no two different natural numbers are equinumerous, we identify the cardinal numbers for \hyperref[def:finite_set]{finite sets} with natural numbers.

  We give special names to
  \begin{thmenum}
    \thmitem{rem:cardinals/countably_infinite} \( \aleph_0 \coloneqq \card(\omega) \), the \term{cardinality of the natural numbers}.
    \thmitem{rem:cardinals/continuum} \( c \coloneqq \card(\BbbR) = \card(\pow(\omega)) \), the \term{cardinality of the continuum}.
  \end{thmenum}
\end{remark}

\begin{proposition}\label{thm:cardinals_well_ordered}
  The class of all \hyperref[def:cardinal]{cardinals} is \hyperref[def:well_ordered_set]{well-ordered}, that is, every set of cardinals has a least element.
\end{proposition}
\begin{proof}
  Direct consequence of \fullref{thm:ordinal_properties/set_of_ordinals_has_minimum} and \fullref{def:cardinal}.
\end{proof}

\begin{conjecture}[Continuum hypothesis]\label{hyp:continuum_hypothesis}\mcite[165]{Enderton1977Sets}
  There exists no cardinal \( \xi \) such that \( \aleph_0 < \xi < c \).
\end{conjecture}

\medskip

\begin{remark}\label{rem:continuum_hypothesis}\mcite[165]{Enderton1977Sets}
  \Fullref{hyp:continuum_hypothesis} has been shown by G\"odel to not be disprovable in \hyperref[def:set]{\logic{ZFC}} and by Cohen to not be provable in \logic{ZFC}.
\end{remark}

\begin{definition}\label{def:cardinal_arithmetic}
  Let \( \xi \) and \( \eta \) be cardinal numbers. We define
  \begin{thmenum}
    \thmitem{def:cardinal_arithmetic/addition}(addition) \( \xi + \eta \coloneqq \card(\xi \coprod \eta) \), where \( \coprod \) denotes disjoint \hyperref[def:disjoint_union]{unions}.
    \thmitem{def:cardinal_arithmetic/multiplication}(multiplication) \( \xi \cdot \eta \coloneqq \card(\xi \times \eta) \)
    \thmitem{def:cardinal_arithmetic/exponentiation}(exponentiation) \( \xi^\eta \coloneqq \card(\cat{Set}(\eta, \xi)) \) (see \fullref{def:category_of_sets})
  \end{thmenum}
\end{definition}

\begin{proposition}\label{thm:countable_union_of_countable_sets}\mcite[thm. 6Q]{Enderton1977Sets}
  A countable union of countable sets is countable.
\end{proposition}
