\subsection{Multigraphs}\label{subsec:multigraphs}

\hyperref[def:multigraph]{Multigraphs} are a generalization of \hyperref[def:graph/directed]{directed graphs} where there may be more than one arc between two vertices. Compare this to \hyperref[def:hypergraph]{hypergraphs}, where a single edge may have more than two vertices instead.

\begin{definition}\label{def:multigraph}
  As in \fullref{def:graph}, fix an arbitrary set \( V \). We will call the members of \( V \) \term{vertices}. Let \( E \) be an unrelated set whose members we will call \term{arcs}.

  Let \( \head: E \to V \) and \( \tail: E \to V \) be arbitrary \hyperref[def:function]{function}. Given an arc \( e \in E \), we will call \( \head(e) \) its \term{head} and \( \tail(e) \) its \term{tail}.

  The quadruple \( M \coloneqq (V, E, \head, \tail) \) is called a \term{multigraph}.

  \begin{thmenum}
    \thmitem{def:multigraph/arc_set} For every pair of vertices \( u \) and \( v \), we define the \term{arc set}
    \begin{equation}\label{eq:def:multigraph/arc_set}
      M(u, v) \coloneqq \set{ e \in E \given \head(e) = u \T{and} \tail(e) = v }.
    \end{equation}

    \thmitem{def:multigraph/undirected} We say that \( M \) is an \term{undirected multigraph} if \( M(u, v) \) is \hyperref[def:equinumerosity]{equinumerous} with \( M(v, u) \) for every pair of vertices \( u \) and \( v \).

    \thmitem{def:multigraph/directed} If \( M \) is not undirected, we may say that it is a \term{directed multigraph}, although this is the default assumption.

    \thmitem{def:multigraph/symmetrization} The \term{symmetrization} of \( M \) is the smallest undirected multigraph containing \( M \).

    \thmitem{def:multigraph/order} The \term{order} \( \ord(G) \) of a (directed or undirected) multigraph is the \hyperref[thm:cardinality_existence]{cardinality} of \( V \).

    If the order \( \ord(M) \) is finite (resp. infinite), we say that \( M \) is a finite (resp. infinite) multigraph.
  \end{thmenum}
\end{definition}
