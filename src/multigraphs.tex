\subsection{Multigraphs}\label{subsec:multigraphs}

\hyperref[def:multigraph]{Multigraphs} are a generalization of \hyperref[def:graph/directed]{directed graphs} where there may be more than one arc between two vertices. Compare this to \hyperref[def:hypergraph]{hypergraphs}, where a single edge may have more than two vertices instead. A lot of concepts from \fullref{subsec:graphs} are translated directly, in which cases, unfortunately, important properties are sometimes lost.

\begin{definition}\label{def:multigraph}
  As in \fullref{def:graph}, fix an arbitrary set \( V \). We will call the members of \( V \) \term{vertices}. Let \( E \) be an unrelated set whose members we will call \term{arcs}.

  Let \( \head: E \to V \) and \( \tail: E \to V \) be arbitrary \hyperref[def:function]{function}. Given an arc \( e \in E \), we will call \( \head(e) \) its \term{head} and \( \tail(e) \) its \term{tail}.

  The quadruple \( M \coloneqq (V, E, \head, \tail) \) is called a \term{multigraph} or \term{quiver}.

  \begin{thmenum}
    \thmitem{def:multigraph/arc_set} For every pair of vertices \( u \) and \( v \), we define the \term{arc set}
    \begin{equation}\label{eq:def:multigraph/arc_set}
      M(u, v) \coloneqq \set{ e \in E \given \head(e) = u \T{and} \tail(e) = v }.
    \end{equation}

    We say that the vertices \( u \) and \( v \) are \term{incident} for every arc in \( M(u, v) \) and vice versa. We shorten the \term{loop set} \( M(v, v) \) to \( M(v) \).

    \thmitem{def:multigraph/undirected} We say that \( M \) is an \term{undirected multigraph} if \( M(u, v) \) is \hyperref[def:equinumerosity]{equinumerous} with \( M(v, u) \) for every pair of vertices \( u \) and \( v \).

    \thmitem{def:multigraph/directed} If \( M \) is not undirected, we may say that it is a \term{directed multigraph}, although this is the default assumption.

    \thmitem{def:multigraph/symmetrization} The \term{symmetrization} of \( M \) is the smallest undirected multigraph containing \( M \).

    \thmitem{def:multigraph/order} The \term{order} \( \ord(G) \) of a (directed or undirected) multigraph is the \hyperref[thm:cardinality_existence]{cardinality} of \( V \).

    If the order \( \ord(M) \) is finite (resp. infinite), we say that \( M \) is a finite (resp. infinite) multigraph.

    \thmitem{def:multigraph/degree} The \term{in-degree}, \term{out-degree} and \term{degree} of a vertex is defined as in \fullref{def:graph_incidence/degree}. The \term{degree} of \( M \) is the maximum of degrees of all vertices, if the maximum is attained.

    \thmitem{def:multigraph/locally_finite} Unlike for directed graphs, we say that the multigraph \( M \) is \term{locally finite} if the arc set \( M(u, v) \) is finite for every pair of vertices \( u \) and \( v \). This is much less restrictive the local finiteness for directed graphs defined in \fullref{def:graph_incidence/locally_finite}. In a directed graph, every arc set is either empty or a singleton set, hence every directed graph is locally finite according to this definition.

    In a multigraph, we have
    \begin{equation*}
      w(v)
      =
      \set[\Big]{ e \in E \given* \head(e) = v \T{or} \tail(e) = v }
      =
      \bigcup_{w \in V} \parens[\Big]{ M(v, w) \cup M(w, v) }.
    \end{equation*}

    If the multigraph has infinite order, it is possible for each arc set to be finite and for \( w(v) \) to be infinite. This would make it locally finite with respect to this definition but not with respect to \fullref{def:graph_incidence/locally_finite}.

    This definition is used in category theory and is defined in, for example, \cite{nLab:finite_category}. Local smallness is defined analogously in \cite[75]{Leinster2016Basic}.

    \thmitem{def:multigraph/path} As for directed graphs, multigraph paths are sequences of consecutive edges as defined in \fullref{def:graph_directed_path}. Cycles and connectedness are also inherited from \fullref{def:graph_cycle} and \fullref{def:graph_connectedness}, respectively.

    \thmitem{def:multigraph/geometric_relization} The \term{geometric realization} of a multigraph \( M \) is defined identically to the geometric realization of a directed graph given in \fullref{def:graph_geometric_realization}.
  \end{thmenum}
\end{definition}

\begin{example}\label{ex:def:multigraph}
  Every \hyperref[def:directed_graph]{directed graph} is a \hyperref[def:directed_multigraph]{directed multigraph}. A simple example of a multigraph that is not a graph is the following modification of \eqref{eq:ex:def:graph/directed}:
  \begin{equation}\label{eq:ex:def:multigraph/directed}
    \begin{aligned}
      \includegraphics{figures/eq__ex__def__multigraph__directed.pdf}
    \end{aligned}
  \end{equation}

  Without the dashed arcs, \eqref{eq:ex:def:multigraph/directed} reduces to \eqref{eq:ex:def:graph/directed}.
\end{example}

\begin{definition}\label{def:multigraph_homomorphism}\mcite[sec. II.7]{MacLane1994}
  Although we haven't defined \hyperref[def:multigraph]{multigraphs} via a \hyperref[def:first_order_theory]{first-order theory}, we have a canonical notion of \hyperref[def:first_order_homomorphism]{homomorphism}, although it is a pair of functions rather than a single function.

  Let \( M = (V_M, E_M) \) and \( N = (V_N, E_N) \) be multigraphs. A \term{homomorphism pair} from \( M \) to \( N \) is a pair of functions \( f_V: V_M \to V_N \) and \( f_E: E_M \to E_N \) such that
  \begin{align}
    \head \bincirc f_e &= f_v \bincirc \head \label{eq:def:multigraph_homomorphism/head} \\
    \tail \bincirc f_e &= f_v \bincirc \tail \label{eq:def:multigraph_homomorphism/tail}
  \end{align}
\end{definition}

\begin{definition}\label{def:category_of_qivers}
  In relation to category theory, \hyperref[def:multigraph/directed]{directed multigraphs} are often called \term{quivers}. For this reason, given a \hyperref[def:grothendieck_universe]{Grothendieck universe} \( \mscrU \), we denote the \hyperref[def:category]{category} of \hyperref[def:large_and_small_sets]{\( \mscrU \)-small} quivers by \( \cat{Quiv} \).

  \begin{itemize}
    \item The \hyperref[def:category/C1]{set of objects} \( \obj(\cat{Quiv}) \) is the set of all \( \mscrU \)-small quivers.
    \item The \hyperref[def:category/C2]{set of morphisms} \( \cat{Quiv}(M, N) \) is the set of all \hyperref[def:multigraph_homomorphism]{homomorphism pairs} from \( M \) to \( N \).

    \item The \hyperref[def:category/C3]{composition of the morphisms} \( (f_V, f_E): M \to N \) and \( (g_V, g_E): N \to K \) is the morphism \( (g_V \bincirc f_V, g_E \bincirc f_E): M \to K \).
  \end{itemize}
\end{definition}
