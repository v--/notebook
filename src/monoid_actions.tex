\subsection{Monoid actions}\label{subsec:monoid_actions}

\begin{definition}\label{def:endomorphism_monoid}
  Given a \hyperref[def:category_cardinality]{locally small} \hyperref[def:category]{category} \( \Cat{C} \), we call \( \Cat{C}(A) \) the \Def{endomorphism monoid} over \( A \) and denote it by \( \End(A) \). If \( A \) is the only object in \( \Cat{C} \), we can identify the entire category \( \Cat{C} \) with the monoid \( \End(A) \).
\end{definition}

\begin{definition}\label{def:left_monoid_action}\MarginCite[159]{Knapp2016BAlg}
  Let \( \CM \) be a \hyperref[def:unital_magma/associativity]{monoid} and let \( A \) be a set or, more generally, an object in a \hyperref[def:category_cardinality]{locally small} category. A \Def{left monoid action} of \( \CM \) on \( A \) can be defined equivalently as:
  \begin{DefEnum}
    \ILabel{def:left_monoid_action/homomorphism} A homomorphism from \( \CM \) to the endomorphism monoid \( \Aut(A) \).
    \ILabel{def:left_monoid_action/indirect_homomorphism} An indexed family of morphisms \( \{ \tau_x: A \to A \}_{x \in \CM} \) such that
    \begin{equation}\label{eq:def:left_monoid_action/indirect_homomorphism}
      \tau_{xy} = \tau_x \circ \tau_y \Tforall x, y \in \CM.
    \end{equation}

    \ILabel{def:left_monoid_action/operation} A heterogeneous \hyperref[def:magma]{algebraic operation} \( \circ: \CM \times A \to A \), written using juxtaposition, such that
    \begin{DefEnum}
      \ILabel{def:left_monoid_action/operation/identity} \( e \odot a = a \) holds for any \( a \in A \).
      \ILabel{def:left_monoid_action/operation/compatibility} \( (xy) \odot a = x \odot (y \odot a) \) holds whenever \( x, y \in \CM \) and \( a \in A \).
    \end{DefEnum}

    See \fullref{remark:theory_of_left_monoid_actions} for the \hyperref[def:first_order_theory]{first order logic theory} behind this operation.
  \end{DefEnum}
\end{definition}
\begin{proof}
  \SubProofImplication{def:left_monoid_action/homomorphism}{def:left_monoid_action/indirect_homomorphism} Let \( \tau: \CM \to \End(A) \) be a homomorphism. Thus we assign a morphism \( \tau(x) \) for each member \( x \in \CM \). Furthermore, since monoid operation on \( \End(A) \) is function composition and since \( \tau \) is a homomorphism, we have
  \begin{equation*}
    \tau(xy) = \tau(x) \circ \tau(y).
  \end{equation*}

  \SubProofImplication{def:left_monoid_action/indirect_homomorphism}{def:left_monoid_action/operation} Assume that we have a morphism \( \tau_x: A \to A \) for each \( x \in \CM \) that satisfies \eqref{eq:def:left_monoid_action/indirect_homomorphism}. Define the operation
  \begin{align*}
    {}&\odot{}: \CM \times A \to A \\
    x &\odot a \coloneqq \tau_x(a).
  \end{align*}

  It satisfies the necessary axioms:
  \begin{RefList}
    \IRef{def:left_monoid_action/operation/identity} If \( a \in A \), we have
    \begin{equation*}
      e \odot a
      =
      \tau_e(a)
      =
      \Id(a)
      =
      a.
    \end{equation*}

    \IRef{def:left_monoid_action/operation/compatibility} If \( x, y \in \CM \) and \( a \in A \), we have
    \begin{equation*}
      (x y) \odot a
      =
      \tau_{x y}(a)
      =
      \tau_{x}(\tau{y}(a))
      =
      g \odot (h \odot a).
    \end{equation*}
  \end{RefList}

  \SubProofImplication{def:left_monoid_action/operation}{def:left_monoid_action/homomorphism} Assume that we have an operation \( \odot: \CM \times A \to A \) that satisfies the axioms for left action. Define the function
  \begin{BreakableAlign*}
    &\tau: \CM \to \End(A) \\
    &\tau(x) \coloneqq (a \mapsto x \cdot a).
  \end{BreakableAlign*}

  Then \( \tau \) is a monoid homomorphism because \( \tau(\varepsilon) = \Id \) and
  \begin{equation*}
    \tau(x y)
    =
    (a \mapsto xy \cdot a)
    =
    (a \mapsto x \cdot (y \cdot a))
    =
    (a \mapsto \tau(x)(\tau(y)(a))
    =
    \tau(x) \circ \tau(y).
  \end{equation*}
\end{proof}

\begin{proposition}\label{thm:monoid_is_action}
  Any \hyperref[def:unital_magma/associative]{monoid} \( \CM \) is a \hyperref[def:monoid_action]{monoid action} on itself. Every element of \( \CM \) simply corresponds to itself under this action.
\end{proposition}
\begin{proof}
  For completeness, we will verify all three definitions:

  \SubProofOf{def:left_monoid_action/homomorphism} The identity function \( \Id: \CM \to \CM \) is the identity of the endomorphism monoid \( \End(\CM) \). Thus the following function
  \begin{align*}
    &\tau: \CM \to \End(\CM) \\
    &\tau(x) \coloneqq (y \mapsto xy)
  \end{align*}
  is a monoid homomorphism. Indeed, \( \tau(e) = \Id \) and
  \begin{equation*}
    (z \mapsto xyz) = \tau(xy) = \tau(x) \circ \tau(y) = (y \mapsto xy) \circ (z \mapsto yz) = (z \mapsto xyz).
  \end{equation*}

  \SubProofOf{def:left_monoid_action/indirect_homomorphism} The proof is the same as above.

  \SubProofOf{def:left_monoid_action/operation} Define the operation
  \begin{BreakableAlign*}
    {}&\odot{}: \CM \times \CM \to \CM \\
    x &\odot y \coloneqq xy
  \end{BreakableAlign*}

  It immediately follows that
  \begin{RefList}
    \IRef{def:left_monoid_action/operation/identity} \( e \circ x = ex = x \Tforall x \in \CM \).
    \IRef{def:left_monoid_action/operation/compatibility} \( (x y) \circ z = xyz = x \circ (y \circ z) \Tforall x, y, z \in \CM \).
  \end{RefList}
\end{proof}

\begin{remark}\label{remark:theory_of_left_monoid_actions}
  In order to fit the heterogeneous operation of \hyperref[def:left_monoid_action]{left monoid actions} into the framework of \hyperref[def:first_order_model]{first order logic models}, we need the category \( \Cat{C} \) to be a \hyperref[def:first_order_model_category]{model category}. A monoid action is then obtained, by extending the theory of \( \Cat{C} \).

  \begin{RemEnum}
    \ILabel{remark:theory_of_left_monoid_actions/functions} For each \( x \in \CM \), add a unary \hyperref[def:first_order_logic_language/func]{functional symbol} \( \varphi_x \).

    \ILabel{remark:theory_of_left_monoid_actions/axiom} For each \( x, y \in \CM \), add the axiom
    \begin{equation}\label{eq:remark:theory_of_left_monoid_actions/axiom_schema}
      \forall a (\tau_{xy}(a) = \tau_x(\tau_y(a))).
    \end{equation}
  \end{RemEnum}
\end{remark}

\begin{definition}\label{def:right_monoid_action}
  We say that \( \tau: \CM \to \End(A) \) is a \Def{right monoid action} of \( \CM \) on \( A \) if the same function is a \hyperref[def:left_monoid_action]{left monoid action} of the \hyperref[def:magma/opposite]{opposite monoid} \( \CM^{-1} \) on \( A \).
\end{definition}
