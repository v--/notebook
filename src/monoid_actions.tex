\subsection{Monoid actions}\label{subsec:monoid_actions}

\begin{definition}\label{def:endomorphism_monoid}
  Given a \hyperref[def:category_cardinality]{locally small} \hyperref[def:category]{category} \( \cat{C} \), we call \( \cat{C}(A) \) the \term{endomorphism monoid} over \( A \) and denote it by \( \End(A) \). If \( A \) is the only object in \( \cat{C} \), we can identify the entire category \( \cat{C} \) with the monoid \( \End(A) \).
\end{definition}

\begin{remark}\label{rem:monoid_action_endomorphisms}
  It is tempting to define a monoid action over an object \( A \) of an arbitrary locally small category rather than over a set rather than specifying the properties of an action (e.g. \enquote{linear action}). Unfortunately, even the simplest examples of monoid actions may fail to hold nice properties. See, e.g. \fullref{thm:monoid_is_action} or \fullref{thm:cayleys_theorem}.
\end{remark}

\begin{definition}\label{def:left_monoid_action}\mcite[159]{Knapp2016BasicAlgebra}
  Let \( \mscrM \) be a \hyperref[def:unital_magma/associative]{monoid} and let \( A \) be a set. A \term{left monoid action} of \( \mscrM \) on \( A \) can be defined equivalently as:
  \begin{thmenum}
    \thmitem{def:left_monoid_action/homomorphism} A homomorphism from \( \mscrM \) to the \hyperref[def:endofunction]{endofunction} monoid \( \End(A) \).
    \thmitem{def:left_monoid_action/indirect_homomorphism} An indexed family of endofunctions \( \{ \tau_x \}_{x \in \mscrM} \) such that
    \begin{equation}\label{eq:def:left_monoid_action/indirect_homomorphism}
      \tau_{xy} = \tau_x \circ \tau_y \T{for all} x, y \in \mscrM.
    \end{equation}

    \thmitem{def:left_monoid_action/operation} A heterogeneous \hyperref[def:magma]{algebraic operation} \( \circ: \mscrM \times A \to A \), written using juxtaposition, such that
    \begin{thmenum}
      \thmitem{def:left_monoid_action/operation/identity} \( e \odot a = a \) holds for any \( a \in A \).
      \thmitem{def:left_monoid_action/operation/compatibility} \( (xy) \odot a = x \odot (y \odot a) \) holds whenever \( x, y \in \mscrM \) and \( a \in A \).
    \end{thmenum}

    See \fullref{rem:theory_of_left_monoid_actions} for the \hyperref[def:first_order_semantics/theory]{first-order logic theory} behind this operation.
  \end{thmenum}

  We say that \( \mscrM \) acts on \( A \) and optionally add adjectives, e.g. \enquote{\( \mscrM \) acts linearly/smoothly on \( A \)}.
\end{definition}
\begin{proof}
  \ImplicationSubProof{def:left_monoid_action/homomorphism}{def:left_monoid_action/indirect_homomorphism} Let \( \tau: \mscrM \to \End(A) \) be a homomorphism. Thus we assign a morphism \( \tau(x) \) for each member \( x \in \mscrM \). Furthermore, since monoid operation on \( \End(A) \) is function composition and since \( \tau \) is a homomorphism, we have
  \begin{equation*}
    \tau(xy) = \tau(x) \circ \tau(y).
  \end{equation*}

  \ImplicationSubProof{def:left_monoid_action/indirect_homomorphism}{def:left_monoid_action/operation} Assume that we have a morphism \( \tau_x: A \to A \) for each \( x \in \mscrM \) that satisfies \eqref{eq:def:left_monoid_action/indirect_homomorphism}. Define the operation
  \begin{align*}
    {}&\odot{}: \mscrM \times A \to A \\
    x &\odot a \coloneqq \tau_x(a).
  \end{align*}

  It satisfies the necessary axioms:
  \begin{refenum}
    \refitem{def:left_monoid_action/operation/identity} If \( a \in A \), we have
    \begin{equation*}
      e \odot a
      =
      \tau_e(a)
      =
      \id(a)
      =
      a.
    \end{equation*}

    \refitem{def:left_monoid_action/operation/compatibility} If \( x, y \in \mscrM \) and \( a \in A \), we have
    \begin{equation*}
      (x y) \odot a
      =
      \tau_{x y}(a)
      =
      \tau_{x}(\tau{y}(a))
      =
      g \odot (h \odot a).
    \end{equation*}
  \end{refenum}

  \ImplicationSubProof{def:left_monoid_action/operation}{def:left_monoid_action/homomorphism} Assume that we have an operation \( \odot: \mscrM \times A \to A \) that satisfies the axioms for left action. Define the function
  \begin{align*}
    &\tau: \mscrM \to \End(A) \\
    &\tau(x) \coloneqq x \id.
  \end{align*}

  Then \( \tau \) is a monoid homomorphism because \( \tau(\varepsilon) = \id \) and
  \begin{equation}\label{def:left_monoid_action/homomorphism/proof}
    \varphi(xy)
    =
    xy \id
    =
    x (y \id)
    =
    x \id (y \id)
    =
    (x \id) (y \id)
    =
    \varphi(x) \varphi(y),
  \end{equation}
\end{proof}

\begin{remark}\label{rem:theory_of_left_monoid_actions}
  In order to fit the heterogeneous operation of \hyperref[def:left_monoid_action]{left monoid actions} into the framework of \hyperref[def:first_order_semantics/satisfiability]{first-order logic models}, we need the category \( \cat{C} \) to be a \hyperref[def:first_order_model_category]{model category}. A monoid action is then obtained, by extending the theory of \( \cat{C} \).

  \begin{thmenum}
    \thmitem{rem:theory_of_left_monoid_actions/functions} For each \( x \in \mscrM \), add a unary \hyperref[def:first_order_language/func]{functional symbol} \( \varphi_x \).

    \thmitem{rem:theory_of_left_monoid_actions/axiom} For each \( x, y \in \mscrM \), add the axiom
    \begin{equation}\label{eq:rem:theory_of_left_monoid_actions/axiom_schema}
      \forall a (\tau_{xy}(a) = \tau_x(\tau_y(a))).
    \end{equation}
  \end{thmenum}
\end{remark}

\begin{definition}\label{def:right_monoid_action}
  We say that \( \tau: \mscrM \to \End(A) \) is a \term{right monoid action} of \( \mscrM \) on \( A \) if the same function is a \hyperref[def:left_monoid_action]{left monoid action} of the \hyperref[def:magma/opposite]{opposite monoid} \( \mscrM^{-1} \) on \( A \).
\end{definition}

\begin{definition}\label{def:faithful_left_monoid_action}
  A left monoid action is said to be \term{faithful} if the corresponding homomorphism is injective.
\end{definition}

\begin{proposition}\label{thm:monoid_is_action}
  Any \hyperref[def:unital_magma/associative]{monoid} \hyperref[def:left_monoid_action]{acts} on itself by \hyperref[def:endofunction]{endofunctions}.

  Compare this result to \fullref{thm:cayleys_theorem}.
\end{proposition}
\begin{proof}
  For completeness, we will verify all three definitions:

  \SubProofOf{def:left_monoid_action/homomorphism} The identity function \( \id: \mscrM \to \mscrM \) is the identity element of \( \fun(\mscrM) \). Define
  \begin{align*}
    &\tau: \mscrM \to \fun(\mscrM) \\
    &\tau(x) \coloneqq x \id
  \end{align*}

  It is a monoid homomorphism because both \eqref{eq:def:pointed_set/homomorphism} and \eqref{def:left_monoid_action/homomorphism/proof} hold.

  \SubProofOf{def:left_monoid_action/indirect_homomorphism} The proof is the same as above.

  \SubProofOf{def:left_monoid_action/operation} Define the operation
  \begin{balign*}
    {}&\odot{}: \mscrM \times \mscrM \to \mscrM \\
    x &\odot y \coloneqq xy
  \end{balign*}

  It immediately follows that
  \begin{refenum}
    \refitem{def:left_monoid_action/operation/identity} \( e \circ x = ex = x \) for all \( x \in \mscrM \).
    \refitem{def:left_monoid_action/operation/compatibility} \( (x y) \circ z = xyz = x \circ (y \circ z) \) for all \( x, y, z \in \mscrM \).
  \end{refenum}
\end{proof}

\begin{proposition}\label{thm:natural_numbers_monoid_action}
  The \hyperref[def:natural_numbers]{natural numbers} \( \BbbN \) act on any \hyperref[def:unital_magma/associative]{monoid} by \hyperref[def:unital_magma/exponentiation]{exponentiation}.

  Compare this result to \fullref{thm:integers_group_action}.
\end{proposition}
\begin{proof}
  The action itself is given by \( n \mapsto (x \mapsto x^n) \). We must only verify that it is a homomorphism.

  We verify the explicit axioms from \fullref{def:left_monoid_action/operation}:
  \begin{refenum}
    \refitem{def:left_monoid_action/operation/identity} \( x^1 = x \) for all \( x \in \mscrM \) by definition.
    \refitem{def:left_monoid_action/operation/compatibility} \( (x^n)^m = x^{nm} \) is the literal statement of \eqref{eq:thm:magma_exponentiation_properties/repeated}.
  \end{refenum}
\end{proof}
