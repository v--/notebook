\subsection{Proof derivation systems}\label{subsec:proof_derivation_systems}

\begin{definition}\label{def:proof_derivation_system}
  A \term{proof derivation system} is based on a set of formulas. These may be all the formulas of the propositional language or of a \hyperref[def:first_order_syntax]{first-order language} \( \mscrL \) or a strict subset of all the formulas (as in \fullref{def:positive_implicational_propositional_derivation_system}).

  \begin{thmenum}
    \thmitem{def:proof_derivation_system/rules} A family of \hyperref[def:relation]{relations} of potentially different arity on the set of formulas of \( \mscrL \), which we call \term{inference rules}. If \( r \) is an inference rule of arity \( n + 1 \), we say that the rule allows us to \term{infer} \( \psi \) from \( \varphi_1, \ldots, \varphi_n \) if \( (\varphi_1, \ldots, \varphi_n, \psi) \in r \). We write this symbolically as
    \begin{equation*}
      \begin{prooftree}
        \hypo{\varphi_1}
        \hypo{\cdots}
        \hypo{\varphi_n}
        \infer3[\( r_\Gamma \)]{\psi}
      \end{prooftree}
    \end{equation*}

    The formulas \( \varphi_1, \ldots, \varphi_n \) are called \term{premises} and \( \psi \) is called the \term{conclusion}.

    We usually formulate rule schemas, e.g. \hyperref[eq:def:positive_implicational_propositional_derivation_system/rules/modus_ponens]{modus ponens}, in which we replace \( \varphi \) and \( \psi \) with whatever formula has the necessary form.

    Some rules like \eqref{eq:def:first_order_derivation_system/exists/elim} are restrictive and depend on the \hyperref[def:derivation_system_derivability]{current derivation}. This is why we allow rules to be parameterized by some set \( \Gamma \) of \term{assumptions}. The rule then applies if and only if some condition is met for \( \Gamma \). This can be described formally by endowing the rule with a unary relation on sets of formulas that describes whether the rule is applicable or not.

    \thmitem{def:proof_derivation_system/axioms} A set of \hyperref[def:first_order_syntax]{formulas} in \( \mscrL \), which we call \term{logical axioms} (to distinguish them from \hyperref[def:derivation_system_derivability]{nonlogical axioms}).
  \end{thmenum}
\end{definition}

\begin{definition}\label{def:derivation_system_derivability}\mcite[def. 18.1]{OpenLogicFull}
  We now introduce the syntactic analog to \hyperref[def:first_order_semantics/entailment]{entailment}. We assume that we are working with a fixed \hyperref[def:proof_derivation_system]{derivation system} for some language \( \mscrL \).

 A sequence \( \varphi_1, \ldots, \varphi_n \) of formulas is called a \term{derivation} from the set \( \Gamma \) of formulas if, for any \( k = 1, \ldots, n \), either of the following hold:
  \begin{thmenum}
    \thmitem{def:derivation_system_derivability/logical_axiom} \( \varphi_k \) is a \hyperref[def:proof_derivation_system/axioms]{logical axiom} in the derivation system.

    \thmitem{def:derivation_system_derivability/nonlogical_axiom} \( \varphi_k \) is a \term{nonlogical axiom}, meaning that \( \varphi_k \in \Gamma \).

    \thmitem{def:derivation_system_derivability/inference} There exist indices \( i_1, \ldots, i_m < k \) such that \( \varphi_k \) can be inferred from \( \varphi_{i_1}, \ldots, \varphi_{i_m} \) using any of the \hyperref[def:proof_derivation_system/rules]{inference rules}.
  \end{thmenum}

  A derivation from \( \Gamma \) ending with the formula \( \varphi \) is also called a \term{proof} of \( \varphi \) from the nonlogical axioms \( \Gamma \). In this case, we write \( \Gamma \vdash \varphi \) and say that \( \varphi \) is a \term{theorem} of \( \Gamma \) and that \( \varphi \) is \term{derivable} from \( \Gamma \). If \( \psi \) is derivable from \( \varphi \) and vice versa, we say that \( \varphi \) and \( \psi \) are \term{interderivable}.

  If \( \Gamma = \varnothing \), we obtain \term{logical theorems} that are derived solely from the \hyperref[def:proof_derivation_system/axioms]{logical axioms} using the inference rules.

  A formula that cannot be derived by any other formula in the derivation is called an \term{assumption}. These are exactly the assumptions which we allow the rules of the derivation system to depend on. In a tree-like proof diagram like \eqref{eq:ex:def:positive_implicational_propositional_derivation_system/identity/proof} where the conclusion is the root of the tree, assumptions correspond to leaves. The conclusion of a proof can thus be derived from the set of all assumptions, some of which may happen to be logical axioms.

  This definitions are specifically tailored at derivation systems. See \fullref{def:first_order_theory} for the more general definition of a theory axiomatized by a set.
\end{definition}

\begin{proposition}\label{thm:derivation_system_transitivity}
  Given a derivation system, if \( \Gamma \vdash \varphi \), then \( \Gamma, \Delta \vdash \varphi \) for any formula \( \varphi \) and any sets \( \Gamma \) and \( \Delta \).

  If every formula in \( \Delta \) is derivable from \( \Gamma \), then the converse also holds: \( \Gamma, \Delta \vdash \varphi \) implies \( \Gamma \vdash \varphi \).
\end{proposition}
\begin{proof}
  If we can use \fullref{def:derivation_system_derivability/inference} to infer \( \varphi \) from \( \Gamma \), then adding additional axioms does not change anything.

  The second part of the theorem has a tad more complicated proof. Assume that every formula in \( \Delta \) is derivable from \( \Gamma \) and that \( \Gamma, \Delta \vdash \varphi \). Let \( \psi_1, \ldots, \psi_{n-1}, \varphi \) be a derivation of \( \varphi \) from \( \Delta \) and let \( \theta_{k,1}, \ldots, \theta_{k,m_k} \) be a derivation of \( \psi_k \) from \( \Gamma \). Then the concatenated sequence
  \begin{equation*}
    \theta_{1,1}, \ldots, \theta_{1,m_1}, \quad \theta_{2,1}, \ldots, \theta_{2,m_2}, \quad \ldots, \quad \theta_{n-1,1}, \ldots, \theta_{k-1,m_{k-1}}
  \end{equation*}
  is a derivation of \( \varphi \) from \( \Gamma \).
\end{proof}

\begin{definition}\mcite[sec. 1]{Wasilewska2010}\label{def:positive_implicational_propositional_derivation_system}
  The \term{positive implicational propositional derivation system} is extraordinarily simple. It is based on the \hyperref[def:propositional_language]{propositional language} but limited to formulas containing only the \hyperref[def:propositional_language/connectives/conditional]{conditional connective} \( \rightarrow \), without any \hyperref[def:propositional_language/constants]{propositional constants} or \hyperref[def:propositional_language/negation]{negation}.

  \begin{itemize}
    \item The system has a single \hyperref[def:proof_derivation_system/rules]{derivation rule} called \term{modus ponens} (Latin for \enquote{mode that by affirming affirms}):
    \begin{equation*}\taglabel[\textrm{MP}]{eq:def:positive_implicational_propositional_derivation_system/rules/modus_ponens}
      \begin{prooftree}
        \hypo{ \varphi}
        \hypo{ \varphi \rightarrow \psi }
        \infer2[\ref{eq:def:positive_implicational_propositional_derivation_system/rules/modus_ponens}]{ \psi }
      \end{prooftree}
    \end{equation*}

    \item The system has infinitely many \hyperref[def:proof_derivation_system/axioms]{logical axioms} but these are represented using only two axiom schemas:
    \begin{thmenum}
      \thmitem{def:positive_implicational_propositional_derivation_system/axioms/imp/intro} We can \enquote{introduce} an \hyperref[def:material_implication]{implication} whose consequent is \( \varphi \) and whose antecedent is any other formula \( \psi \):
      \begin{equation}\label{eq:def:positive_implicational_propositional_derivation_system/axioms/imp/intro}
        \varphi \rightarrow (\psi \rightarrow \varphi) \tag{\textrm{A} \( \rightarrow^+ \)}.
      \end{equation}

      \thmitem{def:positive_implicational_propositional_derivation_system/axioms/imp/trans} Implication is transitive:
      \begin{equation}\label{eq:def:positive_implicational_propositional_derivation_system/axioms/imp/trans}
        \parens[\Big]{ \varphi \rightarrow (\psi \rightarrow \theta) } \rightarrow \parens[\Big]{ (\varphi \rightarrow \psi) \rightarrow (\varphi \rightarrow \theta)} \tag{\textrm{A} \( \twoheadrightarrow \)}.
      \end{equation}
    \end{thmenum}
  \end{itemize}

  The adjective \enquote{positive} in the name of the system refers to the impossibility to negate a formula (compare to \fullref{def:positive_formula}). \enquote{Implicational} refers to the fact that all formulas are \hyperref[def:material_implication]{material implications} and the \hyperref[eq:def:positive_implicational_propositional_derivation_system/rules/modus_ponens]{sole derivation rule} is based on eliminating the connective.
\end{definition}

\begin{example}\mcite[ex. 1.1]{Wasilewska2010}\label{ex:def:positive_implicational_propositional_derivation_system/identity}
  Fix any \hyperref[def:positive_implicational_propositional_derivation_system]{positive implicational formula} \( \varphi \). We will construct a derivation of the implication
  \begin{equation}\label{eq:ex:def:positive_implicational_propositional_derivation_system/identity}
    \varphi \rightarrow \varphi.
  \end{equation}

  We derive the proof from the two logical axioms:
  \begin{equation}\label{eq:ex:def:positive_implicational_propositional_derivation_system/identity/proof}
    \begin{prooftree}[separation=3em]
      \hypo
        {
          \eqref{eq:def:positive_implicational_propositional_derivation_system/axioms/imp/intro}
        }

      \ellipsis
        {
          \( \begin{array}{l}
            \psi \mapsto (\varphi \rightarrow \varphi)
            \\
            \mbox{}
          \end{array} \)
        }
        {
          \eqref{eq:ex:propositional_positive_implicational_logic/dagger}
        }

      \hypo
        {
          \eqref{eq:def:positive_implicational_propositional_derivation_system/axioms/imp/trans}
        }

      \ellipsis
        {
          \( \begin{array}{l}
            \psi \mapsto (\varphi \rightarrow \varphi)
            \\
            \theta \mapsto \varphi
          \end{array} \)
        }
        {
          \eqref{eq:ex:propositional_positive_implicational_logic/dagger}
          \rightarrow ((\varphi \rightarrow (\varphi \rightarrow \varphi)) \rightarrow (\varphi \rightarrow \varphi))
        }

      \infer2[\ref{eq:def:positive_implicational_propositional_derivation_system/rules/modus_ponens}]{(\varphi \rightarrow (\varphi \rightarrow \varphi)) \rightarrow (\varphi \rightarrow \varphi)}

      \hypo
        {
          \eqref{eq:def:positive_implicational_propositional_derivation_system/axioms/imp/intro}
        }

      \ellipsis
        {
          \( \psi \mapsto \varphi \)
        }
        {
          \varphi \rightarrow (\varphi \rightarrow \varphi)
        }

      \infer2[\ref{eq:def:positive_implicational_propositional_derivation_system/rules/modus_ponens}]{\varphi \rightarrow \varphi}
    \end{prooftree}
  \end{equation}
  where
  \begin{equation}\label{eq:ex:propositional_positive_implicational_logic/dagger}
    \varphi \rightarrow ((\varphi \rightarrow \varphi) \rightarrow \varphi).
  \end{equation}

  The only assumptions used in the derivation were logical axioms, hence \eqref{eq:ex:def:positive_implicational_propositional_derivation_system/identity} is a logical theorem.
\end{example}

\begin{definition}\label{def:derivability_and_satisfiability}
  We introduce two notions connecting \hyperref[def:derivation_system_derivability]{derivability} and \hyperref[def:first_order_semantics/satisfiability]{satisfiability}:
  \begin{thmenum}
    \thmitem{def:derivability_and_satisfiability/soundness} If, for any closed formula \( \varphi \), derivability \( \vdash \varphi \) implies satisfiability \( \vDash \varphi \), we say that the derivation system is \term{sound} with respect to the semantical framework.

    \thmitem{def:derivability_and_satisfiability/completeness} Dually, if satisfiability \( \vDash \varphi \) implies derivability \( \vdash \varphi \) for any closed formula \( \varphi \), we say that the derivation system is \term{complete} with respect to the semantical framework.
  \end{thmenum}

  We restrict our attention to closed formulas because we wish to avoid the problems described in \fullref{rem:deduction_with_free_variables}. If we have a formula with free variables, we may simply take its \hyperref[thm:implicit_universal_quantification]{universal closure}.
\end{definition}

\begin{proposition}\label{thm:soundness_of_positive_implicational_propositional_derivation_system}\mcite[thm. 1.1]{Wasilewska2010}
  The \hyperref[def:positive_implicational_propositional_derivation_system]{positive implicational propositional derivation system} is \hyperref[def:derivability_and_satisfiability/soundness]{sound} with respect to \hyperref[def:propositional_semantics]{classical semantics}.
\end{proposition}

\begin{theorem}[Propositional syntactic deduction theorem]\label{thm:propositional_syntactic_deduction_theorem}\mcite[thm. 12.20]{OpenLogicFull}
  In the \hyperref[def:positive_implicational_propositional_derivation_system]{positive implicational derivation system}, \( \Gamma, \psi \vdash \varphi \) holds if and only if \( \Gamma \vdash \psi \rightarrow \varphi \) holds.

  This theorem also holds for propositional derivation systems which extend the positive implication system but either do not add new rules or, as in the case of \fullref{thm:minimal_natural_deduction}, add rules that reduce to axioms.
\end{theorem}
\begin{proof}
  \SufficiencySubProof Suppose that \( \Gamma, \psi \vdash \varphi \). We will use induction on the length of the derivation of \( \varphi \) from \( \Gamma \cup \set{ \psi } \).

  If the derivation consists of a single formula, then the unary finite sequence \( (\varphi) \) is this derivation. Then \( \varphi \) is either a nonlogical axiom (that is, \( \varphi \in \Gamma \) or \( \varphi = \psi \)) or a logical axiom. In all cases, the logical axiom \eqref{eq:def:positive_implicational_propositional_derivation_system/axioms/imp/intro} allows us to derive \( \psi \rightarrow \varphi \) from \( \Gamma \) using \eqref{eq:def:positive_implicational_propositional_derivation_system/rules/modus_ponens}.

  Now suppose that the theorem holds for derivations of length strictly smaller than \( n \) and suppose that there exists a derivation of \( \varphi \) from \( \Gamma \cup \set{ \psi } \) of length \( n \). It is still possible for \( \varphi \) to be a nonlogical or logical axiom, in which case the proof will be the same as in the base case. Since the only rule is \eqref{eq:def:positive_implicational_propositional_derivation_system/rules/modus_ponens}, the other possibility is that the derivation to contain the formulas \( \theta \) and \( \theta \rightarrow \varphi \) for some formula \( \theta \) that is itself derivable from \( \Gamma \cup \set{ \psi } \). But the derivation of \( \theta \) must be shorter than \( n \) and hence the inductive hypothesis holds --- that is, we have \( \Gamma \vdash \psi \rightarrow \theta \) and \( \Gamma \vdash \psi \rightarrow (\theta \rightarrow \varphi) \).

  So we now instantiate other logical axiom --- \eqref{eq:def:positive_implicational_propositional_derivation_system/axioms/imp/trans} --- to obtain
  \begin{equation*}
    \parens[\Big]{ \psi \rightarrow (\theta \rightarrow \varphi) } \rightarrow \parens[\Big]{ (\psi \rightarrow \theta) \rightarrow (\psi \rightarrow \varphi)}.
  \end{equation*}

  Finally, add this new formula to the concatenation of the derivations of \( \psi \rightarrow \theta \) and \( \psi \rightarrow (\theta \rightarrow \varphi) \) from \( \Gamma \). After applying \eqref{eq:def:positive_implicational_propositional_derivation_system/rules/modus_ponens} twice and also adding the intermediate result, we now have a derivation of \( \psi \rightarrow \varphi \) from \( \Gamma \).

  \NecessitySubProof Now suppose that \( \Gamma \vdash \psi \rightarrow \varphi \). Then we can apply \eqref{eq:def:positive_implicational_propositional_derivation_system/rules/modus_ponens} to obtain \( \varphi \) from \( \Gamma \cup \set{ \psi } \).
\end{proof}

\begin{definition}\label{def:minimal_propositional_derivation_system}\mcite[def. 55.10]{OpenLogicFull}
  While the \hyperref[def:positive_implicational_propositional_derivation_system]{positive implicational propositional derivation system} is simple, it is of more practical use to have all propositional connectives available. As it turns out, we cannot utilize \hyperref[ex:posts_completeness_theorem]{complete families of Boolean functions} unless we are dealing with \hyperref[def:propositional_semantics]{classical semantics} --- see for example \fullref{ex:heyting_semantics_lem_counterexample} and \fullref{ex:topological_semantics_lem_counterexample} for how \cref{eq:thm:boolean_equivalences/conditional_as_disjunction} fails to hold.

  Our goal is to define the \term{minimal propositional derivation system}, which would correspond to \hyperref[def:minimal_logic]{minimal logic}. It is \enquote{axiomatic} in the sense that we do not use new rules to express the rest of the propositional syntax but instead we need axiom schemas for each connective. The only exception is \hyperref[def:propositional_language/constants/verum]{\( \bot \)}, the axioms for which tend to change semantics by a lot --- see \fullref{thm:minimal_propositional_negation_laws}.

  Axioms with \( + \) in the superscript are called \term{introduction axioms} and axioms with \( - \) are called \term{elimination axioms}.

  The following axioms are essential in the sense that they cannot be defined in terms of others:
  \begin{thmenum}[series=def:minimal_propositional_derivation_system]
    \thmitem{def:minimal_propositional_derivation_system/top} The simplest axiom states that the constant \hyperref[def:propositional_language/constants/verum]{\( \top \)} is itself an axiom:
    \begin{equation}\label{eq:def:minimal_propositional_derivation_system/top/intro}
      \top \tag{\textrm{A} \( \top^+ \)}
    \end{equation}

    \thmitem{def:minimal_propositional_derivation_system/and} Axioms for \hyperref[def:propositional_language/connectives/conjunction]{conjunction}:
    \begin{align}
      \mathllap{ (\varphi \wedge \psi) } &\rightarrow \mathrlap{ \psi } \tag{\textrm{A} \( \wedge_L^- \)} \label{eq:def:minimal_propositional_derivation_system/and/elim_left} \\
      \mathllap{ (\varphi \wedge \psi) } &\rightarrow \mathrlap{ \varphi } \tag{\textrm{A} \( \wedge_R^- \)} \label{eq:def:minimal_propositional_derivation_system/and/elim_right} \\
      \mathllap{ \varphi }               &\rightarrow \mathrlap{ \parens[\Big]{ \psi \rightarrow (\varphi \wedge \psi) } } \tag{\textrm{A} \( \wedge^+ \)} \label{eq:def:minimal_propositional_derivation_system/and/intro}
    \end{align}

    \thmitem{def:minimal_propositional_derivation_system/or} Axioms for \hyperref[def:propositional_language/connectives/disjunction]{disjunction}:
    \begin{align}
      \mathllap{ \varphi }                      &\rightarrow \mathrlap{ (\varphi \vee \psi) } \tag{\textrm{A} \( \vee_L^+ \)} \label{eq:def:minimal_propositional_derivation_system/or/intro_left} \\
      \mathllap{ \psi }                      &\rightarrow \mathrlap{ (\varphi \vee \psi) } \tag{\textrm{A} \( \vee_R^+ \)} \label{eq:def:minimal_propositional_derivation_system/or/intro_right} \\
      \mathllap{ (\varphi \rightarrow \theta) } &\rightarrow \mathrlap{ \parens[\Big]{ (\psi \rightarrow \theta) \rightarrow ((\varphi \vee \psi) \rightarrow \theta) } } \tag{\textrm{A} \( \vee^- \)} \label{eq:def:minimal_propositional_derivation_system/or/elim}
    \end{align}
  \end{thmenum}

  The following axioms and are said to be \enquote{abbreviations} and do not affect semantics:
  \begin{thmenum}[resume=def:minimal_propositional_derivation_system]
    \thmitem{def:minimal_propositional_derivation_system/iff} The axioms for the biconditional is motivated via \fullref{thm:boolean_equivalences/biconditional_via_conditionals}:
    \begin{align}
      \mathllap{ (\varphi \rightarrow \psi)     } &\rightarrow \mathrlap{ \parens[\Big]{ (\psi \rightarrow \varphi) \rightarrow (\varphi \leftrightarrow \psi) } } \tag{\textrm{A} \( \leftrightarrow^+ \)} \label{def:minimal_propositional_derivation_system/iff/intro} \\
      \mathllap{ (\varphi \leftrightarrow \psi)  }&\rightarrow \mathrlap{ (\varphi \rightarrow \psi) } \tag{\textrm{A} \( \leftrightarrow_L^- \)} \label{eq:def:minimal_propositional_derivation_system/iff/elim_left} \\
      \mathllap{ (\varphi \leftrightarrow \psi) } &\rightarrow \mathrlap{ (\psi \rightarrow \varphi) } \tag{\textrm{A} \( \leftrightarrow_R^- \)} \label{eq:def:minimal_propositional_derivation_system/iff/elim_right}
    \end{align}

    \thmitem{def:minimal_propositional_derivation_system/negation} The axioms for negation is motivated via \fullref{thm:boolean_equivalences/negation_bottom}:
    \begin{align}
      \mathllap{ \neg \varphi }               &\rightarrow \mathrlap{ (\varphi \rightarrow \bot) } \tag{\textrm{A} \( \neg^- \)} \label{eq:def:minimal_propositional_derivation_system/neg/elim} \\
      \mathllap{ (\varphi \rightarrow \bot) } &\rightarrow \mathrlap{ \neg \varphi } \tag{\textrm{A} \( \neg^+ \)} \label{eq:def:minimal_propositional_derivation_system/neg/intro}
    \end{align}
  \end{thmenum}
\end{definition}

\begin{proposition}\label{thm:minimal_natural_deduction}\mcite[sec. 10.2]{OpenLogicFull}
  Formulating a derivation system only in terms of axioms and the single rule \eqref{eq:def:positive_implicational_propositional_derivation_system/rules/modus_ponens} as in \fullref{def:minimal_propositional_derivation_system} allows us to easily study the relationships between the axioms themselves. For example, in \fullref{thm:minimal_propositional_negation_laws} we study the relationships between axioms that differentiate \hyperref[def:classical_logic]{classical logic} from \hyperref[def:intuitionistic_logic]{intuitionistic} and \hyperref[def:minimal_logic]{minimal}. Such systems are called \term{axiomaic} or \term{Hilbert-style}.

  When writing actual proofs like in \fullref{ex:def:positive_implicational_propositional_derivation_system/identity}, however, it is usually more convenient to use rules collectively called \term{natural deduction}. Although natural deduction is commonly presented as an alternative to axiomatic derivations, in fact every axiom gives rise to a natural deduction rule and vice versa.

  Note that some rules have assumptions with indices, e.g. \( [\varphi]^n \) in \eqref{eq:thm:minimal_natural_deduction/imp/intro}. They are called \term{labeled assumptions} and are used in places where the premises of the rule would otherwise have conditional connectives. We say that \( [\varphi]^n \) is \term{discharged} if it is referenced by label in a rule. \enquote{Undischarged assumptions} are a collective name for both regular assumptions in the sense of \fullref{def:derivation_system_derivability} and labeled assumptions that have not been discharged. If an assumption is discharged, it becomes the \hyperref[def:material_implication/antecedent]{antecedent} of some conditional formula and is no longer an actual assumption in the sense of \fullref{def:derivation_system_derivability} but rather an application of \fullref{thm:propositional_syntactic_deduction_theorem} to bring the assumption into the object language. Thus if there are not undischarged assumptions in the derivation, the conclusion of the derivation is a logical theorem.

  \begin{thmenum}
    \thmitem{thm:minimal_natural_deduction/imp} The following rule schemas corresponds to the conditional axiom schemas in \fullref{def:positive_implicational_propositional_derivation_system}:

    \begin{minipage}[t]{0.45\textwidth}
      This rule is inspired by \eqref{eq:def:positive_implicational_propositional_derivation_system/axioms/imp/intro}:
      \begin{equation*}\taglabel[\textrm{R} \( \rightarrow^+ \)]{eq:thm:minimal_natural_deduction/imp/intro}
        \begin{prooftree}
          \hypo{ [\psi]^n }
          \ellipsis {} { \varphi }
          \infer[left label=\( n \)]1[\ref{eq:thm:minimal_natural_deduction/imp/intro}]{ \psi \rightarrow \varphi }
        \end{prooftree}
      \end{equation*}
    \end{minipage}
    \hfill
    \begin{minipage}[t]{0.45\textwidth}
      This rule is merely a renaming of \eqref{eq:def:positive_implicational_propositional_derivation_system/rules/modus_ponens}:
      \begin{equation*}\taglabel[\textrm{R} \( \rightarrow^- \)]{eq:thm:minimal_natural_deduction/imp/elim}
        \begin{prooftree}
          \hypo{ \varphi \rightarrow \psi }
          \hypo{ \varphi }
          \infer2[\ref{eq:thm:minimal_natural_deduction/imp/elim}]{ \psi }
        \end{prooftree}
      \end{equation*}
    \end{minipage}

    Note that there is no rule corresponding to \eqref{eq:def:positive_implicational_propositional_derivation_system/axioms/imp/trans} because this axiom schema follows from \eqref{eq:thm:minimal_natural_deduction/imp/intro} and \eqref{eq:thm:minimal_natural_deduction/imp/elim}. Unlike in the axiomatic derivation system where \eqref{eq:def:positive_implicational_propositional_derivation_system/axioms/imp/trans} is used to prove \fullref{thm:propositional_syntactic_deduction_theorem}, here we have a stronger connection between \( \rightarrow \) in the object language and \( \vdash \) in the metalanguage given by \eqref{eq:thm:minimal_natural_deduction/imp/intro}.

    \thmitem{thm:minimal_natural_deduction/top} The following rule corresponds to the axiom \eqref{eq:def:minimal_propositional_derivation_system/top/intro}:
    \begin{equation*}\taglabel[\textrm{R} \( \top^+ \)]{eq:thm:minimal_natural_deduction/top/intro}
      \begin{prooftree}
        \infer0[\ref{eq:thm:minimal_natural_deduction/top/intro}]{ \top }
      \end{prooftree}
    \end{equation*}

    \thmitem{thm:minimal_natural_deduction/and} The following rule schemas corresponds to the conjunction axiom schemas in \fullref{def:minimal_propositional_derivation_system/and}:

    \begin{minipage}{0.3\textwidth}
      \begin{equation*}\taglabel[\textrm{R} \( \wedge^+ \)]{eq:thm:minimal_natural_deduction/and/intro}
        \begin{prooftree}
          \hypo{ \varphi }
          \hypo{ \psi }
          \infer2[\ref{eq:thm:minimal_natural_deduction/and/intro}]{ \varphi \wedge \psi }
        \end{prooftree}
      \end{equation*}
    \end{minipage}
    \hfill
    \begin{minipage}{0.3\textwidth}
      \begin{equation*}\taglabel[\textrm{R} \( \wedge_L^- \)]{eq:thm:minimal_natural_deduction/and/elim_left}
        \begin{prooftree}
          \hypo{ \varphi \wedge \psi }
          \infer1[\ref{eq:thm:minimal_natural_deduction/and/elim_left}]{ \psi }
        \end{prooftree}
      \end{equation*}
    \end{minipage}
    \hfill
    \begin{minipage}{0.3\textwidth}
      \begin{equation*}\taglabel[\textrm{R} \( \wedge_R^- \)]{eq:thm:minimal_natural_deduction/and/elim_right}
        \begin{prooftree}
          \hypo{ \varphi \wedge \psi }
          \infer1[\ref{eq:thm:minimal_natural_deduction/and/elim_right}]{ \varphi }
        \end{prooftree}
      \end{equation*}
    \end{minipage}

    \thmitem{thm:minimal_natural_deduction/or} The following rule schemas corresponds to the disjunction axiom schemas in \fullref{def:minimal_propositional_derivation_system/or}:

    \begin{minipage}{0.3\textwidth}
      \begin{equation*}\taglabel[\textrm{R} \( \vee_L^+ \)]{eq:thm:minimal_natural_deduction/or/intro_left}
        \begin{prooftree}
          \hypo{ \varphi }
          \infer1[\ref{eq:thm:minimal_natural_deduction/or/intro_left}]{ \varphi \vee \psi }
        \end{prooftree}
      \end{equation*}
    \end{minipage}
    \hfill
    \begin{minipage}{0.3\textwidth}
      \begin{equation*}\taglabel[\textrm{R} \( \vee_R^+ \)]{eq:thm:minimal_natural_deduction/or/intro_right}
        \begin{prooftree}
          \hypo{ \psi }
          \infer1[\ref{eq:thm:minimal_natural_deduction/or/intro_right}]{ \varphi \vee \psi }
        \end{prooftree}
      \end{equation*}
    \end{minipage}
    \hfill
    \begin{minipage}{0.3\textwidth}
      \begin{equation*}\taglabel[\textrm{R} \( \vee^- \)]{eq:thm:minimal_natural_deduction/or/elim}
        \begin{prooftree}
          \hypo{ \varphi \vee \psi }
          \hypo{ [\varphi]^n }
          \ellipsis {} { \theta }
          \hypo{ [\psi]^n }
          \ellipsis {} { \theta }
          \infer[left label=\( n \)]3[\ref{eq:thm:minimal_natural_deduction/or/elim}]{ \theta }
        \end{prooftree}
      \end{equation*}
    \end{minipage}

    \thmitem{thm:minimal_natural_deduction/iff} The following rule schemas corresponds to the biconditional axiom schemas in \fullref{def:minimal_propositional_derivation_system/iff}:

    \begin{minipage}{0.3\textwidth}
      \begin{equation*}\taglabel[\textrm{R} \( \leftrightarrow^+ \)]{eq:thm:minimal_natural_deduction/iff/intro}
        \begin{prooftree}
          \hypo{ [\varphi]^n }
          \ellipsis {} { \psi }
          \hypo{ [\psi]^n }
          \ellipsis {} { \varphi }
          \infer[left label=\( n \)]2[\ref{eq:thm:minimal_natural_deduction/iff/intro}]{ \varphi \leftrightarrow \psi }
        \end{prooftree}
      \end{equation*}
    \end{minipage}
    \hfill
    \begin{minipage}{0.3\textwidth}
      \begin{equation*}\taglabel[\textrm{R} \( \leftrightarrow_L^- \)]{eq:thm:minimal_natural_deduction/iff/elim_left}
        \begin{prooftree}
          \hypo{ \varphi \leftrightarrow \psi }
          \hypo{ \psi }
          \infer2[\ref{eq:thm:minimal_natural_deduction/iff/elim_left}]{ \varphi }
        \end{prooftree}
      \end{equation*}
    \end{minipage}
    \hfill
    \begin{minipage}{0.3\textwidth}
      \begin{equation*}\taglabel[\textrm{R} \( \leftrightarrow_R^- \)]{eq:thm:minimal_natural_deduction/iff/elim_right}
        \begin{prooftree}
          \hypo{ \varphi \leftrightarrow \psi }
          \hypo{ \varphi }
          \infer2[\ref{eq:thm:minimal_natural_deduction/iff/elim_right}]{ \psi }
        \end{prooftree}
      \end{equation*}
    \end{minipage}

    \thmitem{thm:minimal_natural_deduction/negation} The following rule schemas corresponds to the negation axiom schemas in \fullref{def:minimal_propositional_derivation_system/negation}:

    \begin{minipage}{0.45\textwidth}
      \begin{equation*}\taglabel[\textrm{R} \( \neg^+ \)]{eq:thm:minimal_natural_deduction/neg/intro}
        \begin{prooftree}
          \hypo{ [\varphi]^n }
          \ellipsis {} { \bot }
          \infer[left label=\( n \)]1[\ref{eq:thm:minimal_natural_deduction/neg/intro}]{ \neg \varphi }
        \end{prooftree}
      \end{equation*}
    \end{minipage}
    \hfill
    \begin{minipage}{0.45\textwidth}
      \begin{equation*}\taglabel[\textrm{R} \( \neg^- \)]{eq:thm:minimal_natural_deduction/neg/elim}
        \begin{prooftree}
          \hypo{ \varphi }
          \hypo{ \neg \varphi }
          \infer2[\ref{eq:thm:minimal_natural_deduction/neg/elim}]{ \bot }
        \end{prooftree}
      \end{equation*}
    \end{minipage}
  \end{thmenum}
\end{proposition}
\begin{proof}
  We will prove that the axiomatic \hyperref[def:minimal_propositional_derivation_system]{minimal propositional derivation system} is equivalent to the rules of natural deduction described in this proposition.

  \SubProofOf{thm:minimal_natural_deduction/imp} Consider first the axiom \eqref{eq:def:positive_implicational_propositional_derivation_system/axioms/imp/intro}. Fix two formulas \( \varphi \) and \( \psi \). Then \( \varphi \rightarrow (\psi \rightarrow \varphi) \) is an instance of \eqref{eq:def:positive_implicational_propositional_derivation_system/axioms/imp/intro}. Thus we obtain \( \varphi \vdash \psi \rightarrow \varphi \) by applying \eqref{eq:def:positive_implicational_propositional_derivation_system/rules/modus_ponens}, which in turn shows the validity of the rule \eqref{eq:thm:minimal_natural_deduction/imp/intro}.

  The labeled assumption here is essential for showing that \eqref{eq:thm:minimal_natural_deduction/imp/intro} implies \eqref{eq:def:positive_implicational_propositional_derivation_system/axioms/imp/intro}. Without it we would have the rule
  \begin{equation*}
    \begin{prooftree}
      \hypo{ \psi }
      \infer1{ \varphi \rightarrow \psi }
    \end{prooftree}
  \end{equation*}
  which would not allow us to discharge the assumption \( \varphi \) when it is in fact immaterial for the validity of \( \psi \).

  Now we will show that \eqref{eq:def:positive_implicational_propositional_derivation_system/axioms/imp/trans} can be derived using only the rules \eqref{eq:thm:minimal_natural_deduction/imp/intro} and \eqref{eq:thm:minimal_natural_deduction/imp/elim}:
  \begin{equation}\label{eq:thm:minimal_natural_deduction/imp/trans_derivation}
    \begin{prooftree}
      \hypo{ [\varphi \rightarrow (\psi \rightarrow \theta)]^1 }
      \hypo{ [\varphi]^2 }
      \infer2[\ref{eq:thm:minimal_natural_deduction/imp/elim}]{ \psi \rightarrow \theta }

      \hypo{ [\varphi \rightarrow \psi]^3 }
      \hypo{ [\varphi]^2 }
      \infer2[\ref{eq:thm:minimal_natural_deduction/imp/elim}]{ \psi }

      \infer2[\ref{eq:thm:minimal_natural_deduction/imp/elim}]{ \theta }

      \infer[left label=\( 2 \)]1[\ref{eq:thm:minimal_natural_deduction/imp/intro}]{ \varphi \rightarrow \theta }
      \infer[left label=\( 3 \)]1[\ref{eq:thm:minimal_natural_deduction/imp/intro}]{ (\varphi \rightarrow \psi) \rightarrow (\varphi \rightarrow \theta) }
      \infer[left label=\( 1 \)]1[\ref{eq:thm:minimal_natural_deduction/imp/intro}]{ \eqref{eq:def:positive_implicational_propositional_derivation_system/axioms/imp/trans} }
    \end{prooftree}
  \end{equation}

  \SubProofOf{thm:minimal_natural_deduction/top} Obvious.

  \SubProofOf{thm:minimal_natural_deduction/and} The rule \eqref{eq:def:minimal_propositional_derivation_system/and/intro} is equivalent by more readable than proving \( \set{ \varphi, \psi } \vdash \varphi \wedge \psi \) directly. Indeed, compare it to
  \begin{equation*}
    \begin{prooftree}
      \hypo{ \varphi }
      \hypo{ \eqref{eq:def:minimal_propositional_derivation_system/and/intro} }
      \infer2[\ref{eq:def:positive_implicational_propositional_derivation_system/rules/modus_ponens}]{ \psi \rightarrow (\varphi \wedge \psi) }

      \hypo{ \psi }
      \infer2[\ref{eq:def:positive_implicational_propositional_derivation_system/rules/modus_ponens}]{ \varphi \wedge \psi },
    \end{prooftree}
  \end{equation*}
  which is a derivation of \( \varphi \wedge \psi \) from \( \set{ \varphi, \psi } \) using the axiomatic system. The other direction is also simple:
  \begin{equation}\label{eq:thm:minimal_natural_deduction/and_intro_axiom_derivation}
    \begin{prooftree}
      \hypo{ [\varphi]^1 }
      \hypo{ [\psi]^2 }
      \infer2[\ref{eq:thm:minimal_natural_deduction/and/intro}]{ \varphi \wedge \psi }
      \infer[left label=\( 2 \)]1[\ref{eq:thm:minimal_natural_deduction/imp/intro}]{ \psi \rightarrow (\varphi \wedge \psi) },
      \infer[left label=\( 1 \)]1[\ref{eq:thm:minimal_natural_deduction/imp/intro}]{ \eqref{eq:def:minimal_propositional_derivation_system/and/intro} },
    \end{prooftree}
  \end{equation}

  The other two rules are trivially connected to the corresponding axioms using a single application of \eqref{eq:def:positive_implicational_propositional_derivation_system/rules/modus_ponens}.

  \SubProofOf{thm:minimal_natural_deduction/or} For a more complicated example, consider \eqref{eq:def:minimal_propositional_derivation_system/or/elim}. We have
  \begin{equation*}
    \begin{prooftree}
      \hypo{ \eqref{eq:def:minimal_propositional_derivation_system/or/elim} }
      \hypo{ \varphi \rightarrow \theta }
      \infer2[\ref{eq:def:positive_implicational_propositional_derivation_system/rules/modus_ponens}]{ (\psi \rightarrow \theta) \rightarrow ((\varphi \vee \psi) \rightarrow \theta) },

      \hypo{ \psi \rightarrow \theta }
      \infer2[\ref{eq:def:positive_implicational_propositional_derivation_system/rules/modus_ponens}]{ (\varphi \vee \psi) \rightarrow \theta }.

      \hypo{ \varphi \vee \psi }
      \infer2[\ref{eq:def:positive_implicational_propositional_derivation_system/rules/modus_ponens}]{ \theta }.
    \end{prooftree}
  \end{equation*}

  The assumptions of this derivations are \( \varphi \rightarrow \theta \), \( \psi \rightarrow \theta \) and \( \varphi \vee \psi \). Instead of adding them directly as premises of the inference rule \eqref{eq:thm:minimal_natural_deduction/or/elim}, we replace the conditional \( \rightarrow \) with labeled assumptions that correspond to \( \varphi \vdash \theta \) and \( \psi \vdash \theta \).

  We can prove that \eqref{eq:thm:minimal_natural_deduction/or/elim} implies \eqref{eq:def:minimal_propositional_derivation_system/or/elim} analogously to \eqref{eq:thm:minimal_natural_deduction/and_intro_axiom_derivation}.

  The other two rules are again trivial to obtain from the corresponding axioms and vice versa.

  \SubProofOf{thm:minimal_natural_deduction/iff} Analogous to what we have already shown.

  \SubProofOf{thm:minimal_natural_deduction/negation} \eqref{eq:thm:minimal_natural_deduction/neg/intro} is obtained from \eqref{eq:def:minimal_propositional_derivation_system/neg/intro} by applying \eqref{eq:def:positive_implicational_propositional_derivation_system/rules/modus_ponens} once and \eqref{eq:thm:minimal_natural_deduction/neg/elim} is obtained from \eqref{eq:def:minimal_propositional_derivation_system/neg/intro} by applying \eqref{eq:def:positive_implicational_propositional_derivation_system/rules/modus_ponens} twice. Using the rules to derive the axioms is similar to \eqref{eq:thm:minimal_natural_deduction/and_intro_axiom_derivation}.
\end{proof}

\begin{proposition}\label{thm:formulas_are_derivable_iff_conjunction_is_derivable}
  In derivation systems that extend the \term{minimal propositional derivation system} we have \( \psi_1, \psi_1 \vdash \varphi \) if and only if \( (\psi_1 \wedge \psi_2) \vdash \varphi \).
\end{proposition}
\begin{proof}
  \SufficiencySubProof Let \( \psi_1, \psi_2 \vdash \varphi \). From \eqref{eq:thm:minimal_natural_deduction/and/intro} it follows that \( \psi_1, \psi_2 \vdash \psi_1 \wedge \psi_2 \).

  \Fullref{thm:derivation_system_transitivity} implies that \( \psi_1, \psi_2, (\psi_1 \wedge \psi_2) \vdash \varphi \). Applying the other direction of the same theorem, we remove \( \psi_1 \) and \( \psi_2 \) from the from the left and conclude that
  \begin{equation*}
    (\psi_1 \wedge \psi_2) \vdash \varphi.
  \end{equation*}

  \NecessitySubProof Let \( (\psi_1 \wedge \psi_2) \vdash \varphi \). From \eqref{eq:thm:minimal_natural_deduction/and/elim_left} we obtain \( (\psi_1 \wedge \psi_2) \vdash \psi_2 \) and from \eqref{eq:thm:minimal_natural_deduction/and/elim_right} we obtain \( (\psi_1 \wedge \psi_2) \vdash \psi_1 \).

  As in the other direction, by applying \fullref{thm:derivation_system_transitivity}, we get
  \begin{equation*}
    (\psi_1 \wedge \psi_2), \psi_1, \psi_2 \vdash \varphi,
  \end{equation*}
  which can be reduced to
  \begin{equation*}
    \psi_1, \psi_2 \vdash \varphi.
  \end{equation*}
\end{proof}

\begin{proposition}\label{thm:syntactic_contraposition}
  In the \term{minimal propositional derivation system} we have
  \begin{align}
    \varphi \rightarrow \psi &\vdash \neg \psi \rightarrow \neg \varphi \label{eq:thm:syntactic_contraposition/straight} \\
    \eqref{eq:thm:minimal_propositional_negation_laws/dne}, \neg \varphi \rightarrow \neg \psi &\vdash \psi \rightarrow \varphi \label{eq:thm:syntactic_contraposition/reverse}
  \end{align}
\end{proposition}
\begin{proof}
  We will only prove \eqref{eq:thm:syntactic_contraposition/straight}. The derivability \eqref{eq:thm:syntactic_contraposition/reverse} can be proved in the same way except that we would use \eqref{eq:thm:minimal_natural_deduction/neg/intro} rather than \eqref{eq:def:propositional_derivation_system/rules/dne}.

  \begin{equation*}
    \begin{prooftree}
      \hypo{ \varphi \rightarrow \psi }
      \hypo{ [\varphi]^1 }
      \infer2[\eqref{eq:thm:minimal_natural_deduction/imp/elim}]{ \psi }

      \hypo{ [\neg \psi]^2 }
      \infer2[\eqref{eq:thm:minimal_natural_deduction/neg/elim}]{ \bot }

      \infer[left label=\( 1 \)]1[\eqref{eq:thm:minimal_natural_deduction/neg/intro}]{ \neg \varphi }
      \infer[left label=\( 2 \)]1[\eqref{eq:thm:minimal_natural_deduction/imp/intro}]{ \neg \psi \rightarrow \neg \varphi }
    \end{prooftree}
  \end{equation*}
\end{proof}

\begin{theorem}\label{thm:minimal_propositional_negation_laws}
  Consider the following propositional formula schemas:
  \begin{thmenum}
    \thmitem{thm:minimal_propositional_negation_laws/dne} Double negation elimination (see \fullref{thm:boolean_equivalences/double_negation}):
    \begin{equation}\label{eq:thm:minimal_propositional_negation_laws/dne}
      \neg \neg \varphi \rightarrow \varphi \tag{DNE}.
    \end{equation}

    \thmitem{thm:minimal_propositional_negation_laws/efq} Ex falso quodlibet (Latin for \enquote{from falsity everything follows}), also known as the \term{principle of explosion}):
    \begin{equation}\label{eq:thm:minimal_propositional_negation_laws/efq}
      \bot \rightarrow \varphi \tag{EFQ}
    \end{equation}

    \thmitem{thm:minimal_propositional_negation_laws/pierce} \term{Pierce's law}:
    \begin{equation}\label{eq:thm:minimal_propositional_negation_laws/pierce}
      ((\varphi \rightarrow \psi) \rightarrow \varphi) \rightarrow \varphi \tag{Pierce}
    \end{equation}

    \thmitem{thm:minimal_propositional_negation_laws/lem} The \term{law of the excluded middle}:
    \begin{equation}\label{eq:thm:minimal_propositional_negation_laws/lem}
      \varphi \vee \neg \varphi \tag{LEM}
    \end{equation}

    \thmitem{thm:minimal_propositional_negation_laws/lnc} The \term{law of non-contradiction}:
    \begin{equation}\label{eq:thm:minimal_propositional_negation_laws/lnc}
      \neg (\varphi \wedge \neg \varphi). \tag{LNC}
    \end{equation}
  \end{thmenum}

  Assuming the \hyperref[def:minimal_propositional_derivation_system]{minimal propositional derivation system}, we have the following derivations:
  \begin{center}
    \synttree
      [
        {\eqref{eq:thm:minimal_propositional_negation_laws/dne}}
          [
            {\eqref{eq:thm:minimal_propositional_negation_laws/pierce}}
              [{\eqref{eq:thm:minimal_propositional_negation_laws/lem}}]
          ]
          [
            {\eqref{eq:thm:minimal_propositional_negation_laws/efq}}
              [{\eqref{eq:thm:minimal_propositional_negation_laws/lnc}}]
          ]
      ]
  \end{center}

  As it turns out, \eqref{eq:thm:minimal_propositional_negation_laws/lnc}, which is often associated with intuitionistic logic, is a theorem of \hyperref[def:minimal_logic]{minimal logic}.

  Conversely, \eqref{eq:thm:minimal_propositional_negation_laws/efq} and \eqref{eq:thm:minimal_propositional_negation_laws/lem} together can be used to derive \eqref{eq:thm:minimal_propositional_negation_laws/dne}.
\end{theorem}
\begin{proof}
  Most proofs are given in \cite[prop. 3]{DienerMcKubreJordens2016} and \cite[prop. 13]{DienerMcKubreJordens2016}. We will only show that \eqref{eq:thm:minimal_propositional_negation_laws/lnc} is strictly weaker than \eqref{eq:thm:minimal_propositional_negation_laws/efq}.

  For any formula \( \varphi \), we have the \hyperref[thm:minimal_natural_deduction]{natural deduction} proof that \( \eqref{eq:thm:minimal_propositional_negation_laws/lnc} \) is a tautology:
  \begin{equation*}
    \begin{prooftree}[separation=3em]
      \hypo{ [\varphi \wedge \neg \varphi]^1 }
      \infer1[\ref{eq:thm:minimal_natural_deduction/and/elim_left}]{ \varphi }

      \hypo{ [\varphi \wedge \neg \varphi]^1 }
      \infer1[\ref{eq:thm:minimal_natural_deduction/and/elim_right}]{ \neg \varphi }

      \infer2[\ref{eq:thm:minimal_natural_deduction/neg/elim}]{ \bot }

      \infer[left label=\( 1 \)]1[\ref{eq:thm:minimal_natural_deduction/neg/intro}]{ \neg (\varphi \wedge \neg \varphi) }
    \end{prooftree}
  \end{equation*}

  Hence \eqref{eq:thm:minimal_propositional_negation_laws/lnc} is a theorem of \hyperref[def:minimal_logic]{minimal logic}. If it were to imply \eqref{eq:thm:minimal_propositional_negation_laws/efq}, then minimal and intuitionistic logic would be equivalent, which would contradict \cite[prop. 3]{DienerMcKubreJordens2016}. Therefore \eqref{eq:thm:minimal_propositional_negation_laws/lnc} is indeed strictly weaker than \eqref{eq:thm:minimal_propositional_negation_laws/efq}.
\end{proof}

\smallskip

\begin{definition}\label{def:intuitionistic_propositional_derivation_system}\mcite[def. 55.10]{OpenLogicFull}
  Semantics matching the \hyperref[def:minimal_propositional_derivation_system]{minimal propositional derivation system} are not very well studied. If we extend it with the axiom \eqref{eq:thm:minimal_propositional_negation_laws/efq}, we obtain the \term{intuitionistic propositional derivation system}.

  \mcite[sec. 10.2]{OpenLogicFull} The following \hyperref[thm:minimal_natural_deduction]{natural deduction}-style rule schema can be used interchangeably with \eqref{eq:thm:minimal_propositional_negation_laws/efq}:
  \begin{equation*}\taglabel[\textrm{R} \textrm{EFQ}]{eq:def:intuitionistic_propositional_derivation_system/rules/efq}
    \begin{prooftree}
      \hypo{ \bot }
      \infer1[\ref{eq:def:intuitionistic_propositional_derivation_system/rules/efq}]{ \varphi }
    \end{prooftree}
  \end{equation*}

  The corresponding semantics are defined in \fullref{def:propositional_heyting_algebra_semantics} and their link with the derivation system is given in \fullref{thm:intuitionistic_propositional_logic_is_sound_and_complete}.
\end{definition}

\begin{definition}\label{def:propositional_heyting_algebra_semantics}\mcite[14]{BezhanishviliHolliday2019}
  We define \term{Heyting semantics} for propositional formulas similarly to how it is done with classical Boolean semantics in \fullref{def:propositional_semantics}, except that instead of using a \hyperref[def:boolean_algebra]{Boolean algebra} we use a more general \hyperref[def:heyting_algebra]{Heyting algebra}.

  Logical negations depend on complements in Boolean algebras. Since Heyting algebras do not have complements, we instead use \hyperref[def:heyting_algebra/pseudocomplement]{pseudocomplements}.

  Fix a Heyting algebra \( \mscrX = (X, \sup, \inf, T, F, \rightarrow) \). \hyperref[def:propositional_valuation/interpretation]{Propositional interpretations} in Heyting semantics may take any value in \( X \), as can \hyperref[def:propositional_valuation/formula_valuation]{formula valuations}.

  Given an interpretation \( I \) and a formula \( \varphi \), we define \( \varphi\Bracks{I} \) via \eqref{eq:def:propositional_valuation/formula_interpretation}, the sole difference being that negation valuation is defined via the pseudocomplement:
  \begin{equation*}
    (\neg \psi)\Bracks{I} \coloneqq \widetilde{\varphi\Bracks{I}}.
  \end{equation*}

  We say that \( I \) satisfies \( \varphi \) if \( \varphi\Bracks{I} = T \). Thus if the valuation of \( \varphi \) takes any value in \( H \setminus \set{ T } \), then \( I \) does not satisfy \( \varphi \), but that does not necessarily mean that \( I \) satisfies \( \neg \varphi \).

  Then \( \Gamma \) entails \( \varphi \) if, for every \( \psi \in \Gamma \) and every interpretation \( I \) in every Heyting algebra, we have \( \varphi\Bracks{I} = \psi\Bracks{I} \).

  It is important that different Heyting algebras may provide different semantics --- see \fullref{ex:heyting_semantics_lem_counterexample} for an example of what is impossible in a Boolean algebra.
\end{definition}

\begin{example}\label{ex:heyting_semantics_lem_counterexample}
  Let \( \mscrX \) be an extension of the trivial Boolean algebra \( \set{ T, F } \) with the \enquote{indeterminate} symbol \( N \). That is, the domain of \( \mscrX \) is \( \set{ F, N, T } \) and the order is \( F \leq N \leq T \).

  The pseudocomplement of \( N \) is
  \begin{equation*}
    \widetilde{N}
    \reloset {\eqref{eq:def:heyting_algebra/pseudocomplement}} =
    \sup\set{ a \in X \given a \wedge N = \bot }
    =
    F.
  \end{equation*}

  Consider any \hyperref[def:propositional_valuation]{propositional interpretation} \( I \) such that \( I(P) = N \).

  Then the valuation of \eqref{eq:thm:minimal_propositional_negation_laws/lem} is
  \begin{equation*}
    (P \vee \neg P)\Bracks{I}
    =
    \sup\set{ P\Bracks{I}, \widetilde{P\Bracks{I}} }
    =
    \sup\set{ N, \widetilde{N} }
    =
    \sup\set{ N, F }
    =
    N.
  \end{equation*}

  Therefore \eqref{eq:thm:minimal_propositional_negation_laws/lem} does not hold.
\end{example}

\begin{theorem}\label{thm:intuitionistic_propositional_logic_is_sound_and_complete}\mcite[11]{BezhanishviliHolliday2019}
  The \hyperref[def:intuitionistic_propositional_derivation_system]{intuitionistic propositional derivation system} is \hyperref[def:derivability_and_satisfiability/soundness]{sound} and \hyperref[def:derivability_and_satisfiability/completeness]{complete} with respect to \hyperref[def:propositional_heyting_algebra_semantics]{Heyting semantics}. To elaborate,
  \begin{itemize}
    \item if \( \vdash \varphi \), then \( \vDash \varphi \) for every Heyting algebra.
    \item if \( \vDash \varphi \) in every Heyting algebra, then \( \vdash \varphi \).
  \end{itemize}
\end{theorem}

\begin{definition}\label{def:propositional_topological_semantics}\mcite[15]{BezhanishviliHolliday2019}
  Since arbitrary \hyperref[def:heyting_algebra]{Heyting algebras} can be cumbersome to come up with when used for \hyperref[def:propositional_heyting_algebra_semantics]{propositional Heyting semantics}, we can instead utilize \fullref{ex:topological_space_is_heyting_algebra} and define \term{topological semantics} for some nonempty \hyperref[def:topological_space]{topological space}.

  The truth values of interpretations and valuations are then open sets in some topological space and a formula is said to be valid if its valuation is the whole space.
\end{definition}

\begin{example}\label{ex:topological_semantics_lem_counterexample}
  Let \( U \) be an open set in the standard topology in \( \BbbR \). We will examine \eqref{eq:thm:minimal_propositional_negation_laws/lem} with respect to \hyperref[def:propositional_topological_semantics]{topological semantics} for \( \BbbR \). Due to \fullref{ex:topological_space_is_heyting_algebra}, given any \hyperref[def:propositional_valuation]{propositional interpretation} \( I \) such that \( I(P) = U \), we have
  \begin{equation*}
    (P \vee \neg P)\Bracks{I}
    =
    P\Bracks{I} \cup \widetilde{P\Bracks{I}}
    =
    U \cup \widetilde{U}
    =
    U \cup \inter(\BbbR \setminus U).
  \end{equation*}

  If \( U = \varnothing \), then \( (P \vee \neg P)\Bracks{I} = \BbbR \) and \eqref{eq:thm:minimal_propositional_negation_laws/lem} holds. If \( U = (0, 1) \), then \( (P \vee \neg P)\Bracks{I} = \BbbR \setminus \set{ 0, 1 } \) and \eqref{eq:thm:minimal_propositional_negation_laws/lem} does not hold.

  Compare this result with \fullref{ex:heyting_semantics_lem_counterexample}.
\end{example}

\begin{definition}\label{def:brouwer_heyting_kolmogorov_interpretation}\mcite[sec. 55.3]{OpenLogicFull}
  Another semantics for the \hyperref[def:intuitionistic_propositional_derivation_system]{intuitionistic propositional derivation system} is the \term{Brouwer-Heyting-Kolmogorov interpretation}.

  It uses a less formal approach than \hyperref[def:propositional_heyting_algebra_semantics]{Heyting algebra semantics} that is based on the notion of a \enquote{construction}, which is also why it is sometimes called \term{constructive logic}.

  \begin{thmenum}
    \thmitem{def:brouwer_heyting_kolmogorov_interpretation/atomic} We assume that we know what constitutes a construction of propositional variables.
    \thmitem{def:brouwer_heyting_kolmogorov_interpretation/constant} There is no construction of \( \bot \) and no construction of \( \top \) is needed.
    \thmitem{def:brouwer_heyting_kolmogorov_interpretation/disjunction} A construction of \( \psi_1 \vee \psi_2 \) is a pair \( (i, M) \), where \( i = 1, 2 \) and \( M \) is a construction of \( \psi_k \) if and only if \( i = k \). The notion of a pair here is informal.
    \thmitem{def:brouwer_heyting_kolmogorov_interpretation/conjunction} A construction of \( \psi_1 \wedge \psi_2 \) is a pair \( (M_1, M_2) \), where \( M_k \) is a construction of \( \psi_k \) for \( k = 1, 2 \).
    \thmitem{def:brouwer_heyting_kolmogorov_interpretation/conditional} A construction of \( \psi_1 \rightarrow \psi_2 \) is a function that converts a construction of \( \psi_1 \) into a construction of \( \psi_2 \). The notion of a function here is informal.
  \end{thmenum}

  The negation \( \neg\psi \) that corresponds to pseudocomplements in Heyting algebra semantics corresponds to the metastatement \enquote{a construction of \( \psi \) is impossible} under the Heyting-Brouwer-Kolmogorov interpretation.

  If the set \( \Gamma \) of formulas does not derive \( \varphi \), we say that \( \varphi \) is non-constructive under the axioms \( \Gamma \).
\end{definition}

\begin{remark}\label{rem:brouwer_heyting_kolmogorov_interpretation_compatibility}
  Since the \hyperref[def:brouwer_heyting_kolmogorov_interpretation]{Heyting-Brouwer-Kolmogorov interpretation} is not very formal, we cannot properly prove its soundness or completeness with respect to the \hyperref[def:intuitionistic_propositional_derivation_system]{intuitionistic propositional derivation system}.

  Nevertheless, we generally accept the interpretation and conflate \enquote{constructive} and \enquote{intuitionistic} statements.
\end{remark}

\begin{example}\label{ex:def:brouwer_heyting_kolmogorov_interpretation/well_ordering_principle_zfc}
  \Fullref{thm:well_ordering_principle} in \hyperref[def:set]{\logic{ZFC}} does not provide a way to well-order an arbitrary set. The theorem relies on the axiom of choice, whose consequence \fullref{thm:diaconescu_goodman_myhill_theorem} implies the law of the excluded middle (LEM) assuming the nonlogical axioms of \logic{ZFC}.

  Since LEM may not hold in intuitionistic logic, it follows that both \fullref{thm:well_ordering_principle} and the axiom of choice itself should not in general hold under the Heyting-Brouwer-Kolmogorov interpretation, hence by the terminology in \fullref{def:brouwer_heyting_kolmogorov_interpretation}, \fullref{thm:well_ordering_principle} is a non-constructive theorem.
\end{example}

\begin{definition}\label{def:propositional_derivation_system}
  In order to obtain a derivation system that matches \hyperref[def:propositional_semantics]{classical propositional semantics}, we may extend the \hyperref[def:minimal_propositional_derivation_system]{minimal propositional derivation system} with \eqref{eq:thm:minimal_propositional_negation_laws/dne} or the \hyperref[def:intuitionistic_propositional_derivation_system]{intuitionistic propositional derivation system} with any statement that in conjunction with \eqref{eq:thm:minimal_propositional_negation_laws/efq} imply \eqref{eq:thm:minimal_propositional_negation_laws/dne}.

  Such an example is provided in \fullref{thm:minimal_propositional_negation_laws} --- \eqref{eq:thm:minimal_propositional_negation_laws/efq} and \eqref{eq:thm:minimal_propositional_negation_laws/lem} together imply \eqref{eq:thm:minimal_propositional_negation_laws/dne}.

  We call this, very simply, the (classical) \term{propositional derivation system}.

  \mcite[sec. 10.2]{OpenLogicFull} The following \hyperref[thm:minimal_natural_deduction]{natural deduction}-style rule schema can be used interchangeably with \eqref{eq:thm:minimal_propositional_negation_laws/dne}:
  \begin{equation*}\taglabel[\textrm{R} \textrm{DNE}]{eq:def:propositional_derivation_system/rules/dne}
    \begin{prooftree}
      \hypo{ [\neg \varphi]^n }
      \ellipsis {} { \bot }
      \infer[left label=\( n \)]1[\ref{eq:def:propositional_derivation_system/rules/dne}]{ \varphi }
    \end{prooftree}
  \end{equation*}
\end{definition}

\begin{theorem}[Glivenko's double negation theorem]\label{thm:glivenkos_double_negation_theorem}\mcite{Franks2018}
  A formula \( \varphi \) is derivable in the \hyperref[def:propositional_derivation_system]{classical propositional derivation system} if and only if it's double negation \( \neg \neg \varphi \) is derivable in the \hyperref[def:intuitionistic_propositional_derivation_system]{intuitionistic derivation system}.
\end{theorem}

\begin{theorem}\label{thm:classical_propositional_logic_is_sound_and_complete}\mcite[thm. 12.30 \\ corr. 13.7]{OpenLogicFull}
  The \hyperref[def:propositional_derivation_system]{classical propositional derivation system} is \hyperref[def:derivability_and_satisfiability/soundness]{sound} and \hyperref[def:derivability_and_satisfiability/completeness]{complete} with respect to \hyperref[def:propositional_semantics]{classical semantics}.
\end{theorem}

\medskip

\begin{definition}\label{def:first_order_derivation_system}\mcite{LeanNaturalDeduction}
  If we wish to work with first-order logic rather than merely propositional logic, we must extend the \hyperref[def:propositional_derivation_system]{classical propositional derivation system}. We call this, very simply, the (classical) \term{first-order derivation system}. We may instead extend the \hyperref[def:minimal_propositional_derivation_system]{minimal} or \hyperref[def:intuitionistic_propositional_derivation_system]{intuitionistic} derivation systems.

  Note that due to the eigenvariable condition in \eqref{eq:def:first_order_derivation_system/forall/intro} and \eqref{eq:def:first_order_derivation_system/exists/elim}, these rules cannot be reformulated as axioms because they impose conditions on the entire derivation. The other rules can easily be reduced to axioms, however.

  \begin{thmenum}
    \thmitem{def:first_order_derivation_system/eigenvariables} We first add two \hyperref[def:proof_derivation_system/rules]{derivation rules} for quantification regarding constant. Let \( \varphi \) be a formula. Let \( \xi \) be a variable that is not free in any undischarged assumption in the entire derivation (it may be free in \( \varphi \) as long as \( \varphi \) is not itself an undischarged assumption).

    We add the following two rules:

    \begin{minipage}{0.45\textwidth}
      \begin{equation*}\taglabel[\textrm{R} \( \forall^+ \)]{eq:def:first_order_derivation_system/forall/intro}
        \begin{prooftree}
          \hypo{ \varphi }
          \infer1[\ref{eq:def:first_order_derivation_system/forall/intro}]{ \qforall \xi \varphi }
        \end{prooftree}
      \end{equation*}
    \end{minipage}
    \hfill
    \begin{minipage}{0.45\textwidth}
      \begin{equation*}\taglabel[\textrm{R} \( \exists^- \)]{eq:def:first_order_derivation_system/exists/elim}
        \begin{prooftree}
          \hypo{ \qexists \xi \varphi }
          \hypo{ [\varphi]^n }
          \ellipsis {} { \psi }
          \infer[left label=\( n \)]2[\ref{eq:def:first_order_derivation_system/exists/elim}]{ \psi }
        \end{prooftree}
      \end{equation*}
    \end{minipage}

    where \( \psi \) is a formula that does not contain \( \xi \) freely.

    A variable \( \xi \) satisfying these conditions is called an \term{eigenvariable} of the rule. See \fullref{ex:def:first_order_derivation_system/eigenvariables} for further discussion of the necessity of these restrictions. At each step of the derivation the rules depend on the set of undischarged assumptions (which is allowed by \fullref{def:proof_derivation_system/rules}). Consequently, these two rule schemas cannot be formulated as axiom schemas.

    \mcite[def. 22.8]{OpenLogicFull} Note that we only impose this eigenvariable condition on undischarged assumptions, which is a concept specific for natural deduction. In order to emulate this without natural deduction, we can instead reformulate these rule schemas as:

    \begin{minipage}{0.45\textwidth}
      \begin{equation*}\taglabel[\textrm{R} \( \widetilde \forall^+ \)]{eq:def:first_order_derivation_system/forall/intro_simplified}
        \begin{prooftree}
          \hypo{ \psi \rightarrow \varphi }
          \infer1[\ref{eq:def:first_order_derivation_system/forall/intro_simplified}]{ \psi \rightarrow (\qforall \xi \varphi) }
        \end{prooftree}
      \end{equation*}
    \end{minipage}
    \hfill
    \begin{minipage}{0.45\textwidth}
      \begin{equation*}\taglabel[\textrm{R} \( \widetilde \exists^- \)]{eq:def:first_order_derivation_system/exists/elim_simplified}
        \begin{prooftree}
          \hypo{ \varphi \rightarrow \psi }
          \infer1[\ref{eq:def:first_order_derivation_system/exists/elim_simplified}]{ (\qexists \xi \varphi) \rightarrow \psi }
        \end{prooftree}
      \end{equation*}
    \end{minipage}

    where \( \xi \) must not be free in \( \psi \) nor in the set \( \Gamma \) of assumptions. The word \enquote{assumption} here is used in the sense of \fullref{def:proof_derivation_system}, i.e. formulas in \( \Gamma \) that cannot be derived from the rest.

    The implication in \eqref{eq:def:first_order_derivation_system/forall/intro_simplified} guarantees that \( \varphi \vdash \qforall \xi \varphi \) is an invalid derivation from \( \Gamma = \varnothing \), which matches the behavior of \ref{eq:def:first_order_derivation_system/forall/intro} in \fullref{ex:def:first_order_derivation_system/eigenvariables}.

    \thmitem{def:first_order_derivation_system/terms} We also add two \hyperref[def:proof_derivation_system/rules]{derivation rules} for any term \( \tau \):

    \begin{minipage}{0.45\textwidth}
      \begin{equation*}\taglabel[\textrm{R} \( \forall^- \)]{eq:def:first_order_derivation_system/forall/elim}
        \begin{prooftree}
          \hypo{ \qforall \xi \varphi }
          \infer1[\ref{eq:def:first_order_derivation_system/forall/elim}]{ \varphi[\xi \mapsto \tau] }
        \end{prooftree}
      \end{equation*}
    \end{minipage}
    \hfill
    \begin{minipage}{0.45\textwidth}
      \begin{equation*}\taglabel[\textrm{R} \( \exists^+ \)]{eq:def:first_order_derivation_system/exists/intro}
        \begin{prooftree}
          \hypo{ \varphi[\xi \mapsto \tau] }
          \infer1[\ref{eq:def:first_order_derivation_system/exists/intro}]{ \qexists \xi \varphi }
        \end{prooftree}
      \end{equation*}
    \end{minipage}

    Compare this to \fullref{thm:quantifier_satisfiability}.

    \thmitem{def:first_order_derivation_system/equality} Finally, we also add three rules for formal equality. For any terms \( \tau \) and \( \sigma \) any formula \( \varphi \) with \( \xi \in \boldop{Free}(\varphi) \), we also add the following rules:

    \begin{minipage}{0.3\textwidth}
      \begin{equation*}\taglabel[\textrm{R} \( \doteq^+ \)]{eq:def:first_order_derivation_system/equality/intro}
        \begin{prooftree}
          \infer0[\ref{eq:def:first_order_derivation_system/equality/intro}]{ \tau \doteq \tau }
        \end{prooftree}
      \end{equation*}
    \end{minipage}
    \hfill
    \begin{minipage}{0.3\textwidth}
      \begin{equation*}\taglabel[\textrm{R} \( \doteq_L^- \)]{eq:def:first_order_derivation_system/equality/elim_left}
        \begin{prooftree}
          \hypo{ \tau \doteq \sigma }
          \hypo{ \varphi[\xi \mapsto \tau] }
          \infer2[\ref{eq:def:first_order_derivation_system/equality/elim_left}]{ \varphi[\xi \mapsto \sigma] }
        \end{prooftree}
      \end{equation*}
    \end{minipage}
    \hfill
    \begin{minipage}{0.3\textwidth}
      \begin{equation*}\taglabel[\textrm{R} \( \doteq_L^+ \)]{eq:def:first_order_derivation_system/equality/elim_right}
        \begin{prooftree}
          \hypo{ \tau \doteq \sigma }
          \hypo{ \varphi[\xi \mapsto \sigma] }
          \infer2[\ref{eq:def:first_order_derivation_system/equality/elim_right}]{ \varphi[\xi \mapsto \tau] }
        \end{prooftree}
      \end{equation*}
    \end{minipage}
  \end{thmenum}
\end{definition}

\begin{example}\label{ex:def:first_order_derivation_system/eigenvariables}
  \hfill
  \begin{thmenum}
    \thmitem{ex:def:first_order_derivation_system/eigenvariables/invalid_universal_closure} We explicitly forbid the syntactic equivalent of \fullref{thm:implicit_universal_quantification} in order to avoid derivations like \fullref{ex:def:first_order_derivation_system/eigenvariables/invalid_universal}. Consider the derivation
    \begin{equation*}
      \begin{prooftree}
        \hypo{ [\varphi]^1 }
        \infer1[\ref{eq:def:first_order_derivation_system/forall/intro}]{ \qforall \xi \varphi }
      \end{prooftree}
    \end{equation*}

    The problem here is that \( \varphi \) is itself an undischarged assumption, hence \eqref{eq:def:first_order_derivation_system/forall/intro} is actually inapplicable here and the derivation is invalid.

    \thmitem{ex:def:first_order_derivation_system/eigenvariables/invalid_universal}\mcite[sec. 20.3]{OpenLogicFull} To see why the eigenvariable conditions in \fullref{def:first_order_derivation_system/eigenvariables} are essential, consider the following derivation of \( \qforall \xi \varphi \) from \( \qexists \xi \varphi \):
    \begin{equation*}
      \begin{prooftree}
        \hypo{ \qexists \xi \varphi }

        \hypo{ [\varphi]^1 }
        \infer1[\ref{eq:def:first_order_derivation_system/forall/intro}]{ \qforall \xi \varphi }

        \infer[left label=\( 1 \)]2[\ref{eq:def:first_order_derivation_system/exists/elim}]{ \qforall \xi \varphi }
      \end{prooftree}
    \end{equation*}

    This proof relies on \fullref{ex:def:first_order_derivation_system/eigenvariables/invalid_universal_closure}, which we have already demonstrated to be invalid.

    \thmitem{ex:def:first_order_derivation_system/eigenvariables/invalid_existence} Another invalid derivation, in case \( \xi \in \boldop{Free}(\varphi) \), is
    \begin{equation*}
      \begin{prooftree}
        \hypo{ \qexists \xi \varphi }

        \hypo{ [\varphi]^1 }
        \infer1{ \varphi }

        \infer[left label=\( 1 \)]2[\ref{eq:def:first_order_derivation_system/exists/elim}]{ \varphi }
      \end{prooftree}
    \end{equation*}
    \thmitem{ex:def:first_order_derivation_system/eigenvariables/universal_implies_existence} On the other hand, \( \qexists \xi \varphi \) can easily be derived from \( \qforall \xi \varphi \):
    \begin{equation*}
      \begin{prooftree}
        \hypo{ \qforall \xi \varphi }
        \infer1[\ref{eq:def:first_order_derivation_system/forall/elim}]{ \varphi = \varphi[\xi \mapsto \xi] }
        \infer1[\ref{eq:def:first_order_derivation_system/exists/intro}]{ \qexists \xi \varphi }
      \end{prooftree}
    \end{equation*}

    \thmitem{ex:def:first_order_derivation_system/eigenvariables/universal_implies_universal} It is also valid to perform the completely meaningless derivation:
    \begin{equation*}
      \begin{prooftree}
        \hypo{ \qforall \xi \varphi }
        \infer1[\ref{eq:def:first_order_derivation_system/forall/elim}]{ \varphi = \varphi[\xi \mapsto \xi] }
        \infer1[\ref{eq:def:first_order_derivation_system/forall/intro}]{ \qforall \xi \varphi }
      \end{prooftree}
    \end{equation*}
  \end{thmenum}
\end{example}

\begin{proposition}\label{thm:syntactic_first_order_quantifiers_are_dual}
  For any formula \( \varphi \) and any variable \( \xi \) over \( \mscrL \), we have the following interderivable pairs:
  \begin{align}
    \neg \qforall \xi \varphi &\T{and} \qexists \xi \neg \varphi \label{thm:syntactic_first_order_quantifiers_are_dual/negation_of_universal} \\
    \neg \qexists \xi \varphi &\T{and} \qforall \xi \neg \varphi \label{thm:syntactic_first_order_quantifiers_are_dual/negation_of_existential}
  \end{align}
\end{proposition}
\begin{proof}
  We will only show \eqref{thm:syntactic_first_order_quantifiers_are_dual/negation_of_universal}. First,

  \begin{equation*}
    \begin{prooftree}
      \hypo{ \neg \qforall \xi \varphi }
      \hypo{ [\qforall \xi \varphi]^1 }
      \infer2[\eqref{eq:thm:minimal_natural_deduction/neg/elim}]{ \bot }
      \infer[left label=\( 1 \)]1[\eqref{eq:def:propositional_derivation_system/rules/dne}]{ \qforall \xi \varphi }
      \infer1[\eqref{eq:def:first_order_derivation_system/forall/elim}]{ \varphi }

      \hypo{ [\neg \varphi]^2 }
      \infer2[\eqref{eq:thm:minimal_natural_deduction/neg/elim}]{ \bot }

      \infer[left label=\( 2 \)]1[\eqref{eq:thm:minimal_natural_deduction/neg/intro}]{ \neg \varphi }
      \infer1[\eqref{eq:def:first_order_derivation_system/exists/intro}]{ \qexists \xi \neg \varphi }
    \end{prooftree}
  \end{equation*}

  Conversely,
  \begin{equation*}
    \begin{prooftree}
      \hypo{ \qexists \xi \neg \varphi }

      \hypo{ [\qforall \xi \varphi]^1 }
      \infer1[\eqref{eq:def:first_order_derivation_system/forall/elim}]{ \varphi }

      \hypo{ [\neg \varphi]^2 }
      \infer2[\eqref{eq:thm:minimal_natural_deduction/neg/elim}]{ \bot }
      \infer[left label=\( 1 \)]1[\eqref{eq:thm:minimal_natural_deduction/neg/intro}]{ \neg \qforall \xi \varphi }

      \infer[left label=\( 2 \)]2[\eqref{eq:def:first_order_derivation_system/exists/elim}]{ \neg \qforall \xi \varphi }
    \end{prooftree}
  \end{equation*}
\end{proof}

\begin{theorem}\label{thm:classical_first_order_logic_is_sound_and_complete}\mcite[thm. 20.32 \\ corr. 23.19]{OpenLogicFull}
  The \hyperref[def:first_order_derivation_system]{classical first-order derivation system} is \hyperref[def:derivability_and_satisfiability/soundness]{sound} and \hyperref[def:derivability_and_satisfiability/completeness]{complete} with respect to \hyperref[def:first_order_semantics]{classical semantics}.

  The completeness part is known as \enquote{G\"odel's completeness theorem} and requires an elaborate proof.
\end{theorem}

\begin{theorem}[First-order syntactic deduction theorem]\label{thm:first_order_syntactic_deduction_theorem}\mcite[thm. 22.25]{OpenLogicFull}
  Let \( \Gamma \) be a set of closed formulas over some first-order language, and let \( \varphi \) and \( \psi \) be a arbitrary closed formulas.

  In the \hyperref[def:minimal_propositional_derivation_system]{first-order derivation system}, \( \Gamma, \psi \vdash \varphi \) holds if and only if \( \Gamma \vdash \psi \rightarrow \varphi \) holds.

  Compare this result with \fullref{thm:propositional_syntactic_deduction_theorem} and \fullref{thm:first_order_semantic_deduction_theorem}.
\end{theorem}
\begin{proof}
  The proof is analogous to \fullref{thm:propositional_syntactic_deduction_theorem} except that in the deduction step we can have the two additional rules \eqref{eq:def:first_order_derivation_system/forall/intro_simplified} and \eqref{eq:def:first_order_derivation_system/exists/elim_simplified} (as discussed in \fullref{def:first_order_derivation_system}, all other rules can easily be reinterpreted as axioms).

  \begin{itemize}
    \item If \eqref{eq:def:first_order_derivation_system/forall/intro_simplified} is the last rule to have been applied in the derivation of \( \varphi \) from \( \Gamma \cup \set{ \psi } \), then \( \varphi = \theta \rightarrow \qforall \xi \rho \) for some formulas \( \theta \) and \( \rho \) where \( \xi \not\in \boldop{Free}(\theta) \). Furthermore, the eigenvariable condition know that \( \xi \) is not free in \( \theta \).

    In order to apply \eqref{eq:def:first_order_derivation_system/forall/intro_simplified}, \( \theta \rightarrow \rho \) must already be part of the derivation. By the inductive hypothesis, we have \( \Gamma \vdash \psi \rightarrow (\theta \rightarrow \rho) \).

    We will use natural deduction to derive \( (\psi \wedge \theta) \rightarrow \rho \) from \( \psi \rightarrow (\theta \rightarrow \rho) \):
    \begin{equation*}
      \begin{prooftree}
        \hypo{ [\psi \wedge \theta]^1 }
        \infer1[\eqref{eq:thm:minimal_natural_deduction/and/elim_left}]{ \psi }
        \hypo{ \psi \rightarrow (\theta \rightarrow \rho) }
        \infer2[\eqref{eq:thm:minimal_natural_deduction/imp/elim}]{ \theta \rightarrow \rho }

        \hypo{ [\psi \wedge \theta]^1 }
        \infer1[\eqref{eq:thm:minimal_natural_deduction/and/elim_right}]{ \theta }
        \infer2[\eqref{eq:thm:minimal_natural_deduction/imp/elim}]{ \rho }

        \infer[left label=\( 1 \)]1[\eqref{eq:thm:minimal_natural_deduction/imp/intro}]{ (\psi \wedge \theta) \rightarrow \rho }
      \end{prooftree}
    \end{equation*}

    We can now apply \eqref{eq:def:first_order_derivation_system/forall/intro_simplified} to the formula \( (\psi \wedge \theta) \rightarrow \rho \) to obtain
    \begin{equation*}
      (\psi \wedge \theta) \rightarrow \qforall \xi \rho.
    \end{equation*}

    This application is correct because \( \xi \) is not free in neither \( \psi \) nor \( \theta \). We now go backwards:
    \begin{equation*}
      \begin{prooftree}
        \hypo{ (\psi \wedge \theta) \rightarrow \qforall \xi \rho }

        \hypo{ [\psi]^1 }
        \hypo{ [\theta]^2 }
        \infer2[\eqref{eq:thm:minimal_natural_deduction/and/intro}]{ (\psi \wedge \theta) }
        \infer2[\eqref{eq:thm:minimal_natural_deduction/imp/elim}]{ \qforall \xi \rho }
        \infer[left label=\( 2 \)]1[\eqref{eq:thm:minimal_natural_deduction/imp/intro}]{ \theta \rightarrow (\qforall \xi \rho) }
        \infer[left label=\( 1 \)]1[\eqref{eq:thm:minimal_natural_deduction/imp/intro}]{ \psi \rightarrow (\theta \rightarrow \qforall \xi \rho) }
      \end{prooftree}
    \end{equation*}

    Hence we have shown that
    \begin{equation*}
      \Gamma \vdash \psi \rightarrow \underbrace{(\theta \rightarrow \qforall \xi \rho)}_{\varphi},
    \end{equation*}
    which was our goal.

    \item If, instead, \eqref{eq:def:first_order_derivation_system/exists/elim_simplified} is the last rule to have been applied, then \( \varphi = (\qexists \xi \theta) \rightarrow \rho \), where \( \xi \not\in \boldop{Free}(\rho) \) and \( \theta \rightarrow \rho \) is part of the derivation.

    The rest follows by applying \fullref{thm:syntactic_contraposition} \( \theta \rightarrow \rho \) and then using \fullref{thm:syntactic_first_order_quantifiers_are_dual}.
  \end{itemize}
\end{proof}
