\subsection{Initial and final topologies}\label{subsec:initial_final_topologies}

\begin{definition}\label{def:category_of_small_topological_spaces}
  Since the topology \( \mscrT \) of a \hyperref[def:topological_space]{topological space} \( (\mscrX, \mscrT) \) consists of subsets of \( \mscrX \), we cannot build a \hyperref[def:first_order_theory]{first-order theory} from \fullref{def:topological_space/O1}-\fullref{def:topological_space/O3}. We can, however, explicitly describe the \hyperref[def:category]{category} \( \cat{Top} \) of topological spaces as
  \begin{refenum}
    \refitem{def:category/objects} The \hyperref[def:set]{class} of objects is the class of all topological space.
    \refitem{def:category/morphisms} The morphisms between two topological spaces are the \hyperref[def:global_continuity]{continuous functions} between them.
    \refitem{def:category/composition} Composition of morphisms is the usual \hyperref[def:multi_valued_function/composition]{function composition}.
  \end{refenum}
\end{definition}

\begin{theorem}\label{thm:top_complete_cocomplete}
  The category \( \cat{Top} \) of is both \hyperref[def:categorical_limit]{complete} and \hyperref[def:categorical_colimit]{cocomplete}.
\end{theorem}

\begin{definition}\label{def:initial_topology}\mcite{nLab:top}
  Let \( \{ (X_k, \mscrT_k) \}_{k \in \mscrK} \) be a \hyperref[def:tuple_and_cartesian_product/indexed_family]{family} of topological spaces. Let \( X \) be a bare set and let
  \begin{equation*}
    \{ f_k: X \to X_k \}_{k \in \mscrK}
  \end{equation*}
  be a family of functions.

  The topology on \( X \) generated by the subbase
  \begin{equation*}
    \mathcal{P} \coloneqq \{ f_k^{-1}(U) \colon k \in \mscrK, U \in T_k \}
  \end{equation*}
  is called the \term{initial} (or \term{weak}) topology on \( X \) generated by the family \( \{ f_k \}_{k \in \mscrK} \).

  It is the weakest topology that makes all functions in the family \( \{ f_k \}_{k \in \mscrK} \) continuous.
\end{definition}

\begin{definition}\label{def:final_topology}\mcite{nLab:top}
  Dually, if the family of functions is of the type
  \begin{equation*}
    \{ f_k: X_k \to X \}_{k \in \mscrK},
  \end{equation*}
  then we define the \term{final} (or \term{strong}) topology on \( X \) generated by the family \( \{ f_k \}_{k \in \mscrK} \) as the topology
  \begin{equation*}
    \mscrT \coloneqq \{ U \subseteq X \colon \forall k \in \mscrK, f_k^{-1}(U) \in T_k \}.
  \end{equation*}

  It is the strongest topology that makes all functions in the family \( \{ f_k \}_{k \in \mscrK} \) continuous.
\end{definition}

\begin{proposition}\label{thm:initial_final_topology_limit}\mcite{nLab:top}
  Let \( D: \boldop I \to \cat{Top} \) be a small \hyperref[def:categorical_diagram]{diagram}. For each space in the image \( D(\boldop I) \), denote the set corresponding by \( X_k \) and the corresponding topology by \( \mscrT_k \).

  The limit (resp. colimit) \( (X, \mscrT) \) of \( D \) can then be described as
  \begin{thmenum}
    \item \( (X, \{ f_k \}_{k \in \cat{I}}) = \varprojlim UD \) (resp. \( \varinjlim UD \)) is the limit (resp. colimit) in \( \cat{Set} \) of \( U \circ D \), where \( U: \boldop{Top} \to \cat{Set} \) is the forgetful functor.
    \item \( \mscrT \) is the \hyperref[def:initial_topology]{initial} (resp. \hyperref[def:final_topology]{final}) topology on \( X \) generated by the family of functions \( \{ f_k \}_{k \in \cat{I}} \).
  \end{thmenum}

  In particular, the functor \( U \) lifts limits and \hyperref[def:categorical_limit_preservation/lift]{colimits}.
\end{proposition}

\begin{definition}\label{def:topological_subspace}
  Let \( (X, \mscrT) \) be a topological space and let \( M \subseteq X \) be a subset of \( X \). The \term{topological subspace} \( (M, \mscrT_M) \) is obtained by endowing \( M \) with the topology
  \begin{equation*}
    \mscrT_M \coloneqq \{ U \cap M \colon U \in T \}.
  \end{equation*}

  The topology \( \mscrT_M \) is called the \term{subspace topology} or \term{induced topology}.

  It is the initial topology generated by the canonical embedding \( \iota: M \to X \).
\end{definition}

\begin{definition}\label{def:topological_product}
  The \term{topological product} or \term{Tychonoff product}
  \begin{equation*}
    \left( \prod_{k \in \mscrK} X_k, \prod_{k \in \mscrK} \mscrT_k \right)
  \end{equation*}
  of the family \( { (X_k, \mscrT_k) }_{k \in \mscrK} \) is simply the categorical product in the category \( \cat{Top} \) (see \fullref{def:categorical_product}). The underlying set \( \prod_{k \in \mscrK} X_k \) is the \hyperref[thm:set_categorical_limits/product]{Cartesian product} and the topology \( \prod_{k \in \mscrK} \mscrT_k \) is called the \term{product topology}.

  Let \( { (X_k, \mscrT_k) }_{k \in \mscrK} \) and \( { (Y_k, \mathcal{O}_k) }_{k \in \mscrK} \) be two families of topological spaces and let
  \begin{equation*}
    \{ f_k: X_k \to Y_k \}_{k \in \mscrK}
  \end{equation*}
  be a family of arbitrary functions between them.

  We define the \term{product \( \prod_{k \in \mscrK} f_k \) of \( \{ f_k \}_{k \in \mscrK} \)} as the function
  \begin{balign*}
     & \left(\prod_{k \in \mscrK} f_k \right): \prod_{k \in \mscrK} X_k \to \prod_{k \in \mscrK} Y_k              \\
     & \left(\prod_{k \in \mscrK} f_k \right)(\{ x_k \}_{k \in \mscrK}) \coloneqq \{ f_k (x_k) \}_{k \in \mscrK}.
  \end{balign*}

  If all of the spaces \( (X_k, \mscrT_k) \) are equal to some space \( (X, \mscrT) \), we call the product of \( \{ f_k \}_{k \in \mscrK} \) the \term{diagonal product} and denote it by
  \begin{equation*}
    \Delta_{k \in \mscrK} f_k: X \to \prod_{k \in \mscrK} Y_k.
  \end{equation*}
\end{definition}

\begin{definition}\label{def:topological_quotient}\mcite[90]{Engelking1989}
  Let \( X \) be a topological space and let \( \cong \) be an \hyperref[def:equivalence_relation]{equivalence relation} on \( X \). The \term{quotient space} \( (X, \mscrT) / \sim \) is obtained by endowing the quotient set \( X / \cong \) with the final \hyperref[def:final_topology]{topology} given by the canonical projection map \( x \mapsto [x] \).
\end{definition}

\begin{definition}\label{def:topological_sum}\mcite[74]{Engelking1989}
  The \term{topological direct sum}
  \begin{equation*}
    (\oplus_{k \in \mscrK} X_k, \oplus_{k \in \mscrK} \mscrT_k)
  \end{equation*}
  of the family \( { (X_k, \mscrT_k) }_{k \in \mscrK} \) is simply the categorical coproduct in the category \( \cat{Top} \) (see \fullref{def:categorical_coproduct}). The underlying set \( \oplus_{k \in \mscrK} X_k \) is the \hyperref[thm:set_categorical_limits/coproduct]{disjoint union} and the topology \( \oplus_{k \in \mscrK} \mscrT_k \) is called the \term{direct sum topology}.

  Let \( { (X_k, \mscrT_k) }_{k \in \mscrK} \) and \( { (Y_k, \mathcal{O}_k) }_{k \in \mscrK} \) be two families of topological spaces and let
  \begin{equation*}
    \{ f_k: X_k \to Y_k \}_{k \in \mscrK}
  \end{equation*}
  be a family of arbitrary functions between them. Let \( \iota_{X_k}: X_k \to \oplus_{k \in \mscrK} X_k \) and \( \iota_{Y_k}: Y_k \to \oplus_{k \in \mscrK} Y_k \) be the corresponding canonical embeddings.

  We define the \term{direct sum \( \oplus_{k \in \mscrK} f_k \) of \( \{ f_k \}_{k \in \mscrK} \)} as the function
  \begin{balign*}
     & (\oplus_{k \in \mscrK} f_k): \oplus_{k \in \mscrK} X_k \to \oplus_{k \in \mscrK} Y_k   \\
     & (\oplus_{k \in \mscrK} f_k){\rvert}_{X_k} \coloneqq \iota_{Y_k} \circ f_k.
  \end{balign*}

  Obviously \( \oplus_{k \in \mscrK} f_k \) is continuous whenever all \( f_k \) are continuous.

  If all of the spaces \( (Y_k, \mathcal{O}_k) \) are equal to some space \( (Y, \mathcal{O}) \), we call the direct sum of \( \{ f_k \}_{k \in \mscrK} \) simply a \term{sum} and denote it by
  \begin{equation*}
    \sum_{k \in \mscrK} f_k: \oplus_{k \in \mscrK} X_k \to Y.
  \end{equation*}
\end{definition}
