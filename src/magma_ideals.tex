\subsection{Magma ideals}\label{sec:magma_ideals}

\begin{definition}\label{def:magma_ideal}
  Let \( M \) be a magma\Tinyref{def:magma/magma} and \( I \) be a subset of \( M \). We say that \( I \) is a \Def{left ideal} (resp. \Def{right ideal}) of \( G \) if the inclusion \( IM \subseteq I \) holds, where we use the convention in \cref{remark:vector_space_set_operations}, that is,
  \begin{equation*}
    IM = \{ x \cdot i \;\colon\; x \in M, i \in I \}.
  \end{equation*}

  If \( I \) is both a left ideal and a right ideal, we say that it is a \Def{two-sided ideal}.
\end{definition}

\begin{proposition}\label{thm:magma_ideal_is_submagma}
  Every two-sided magma ideal is a submagma.
\end{proposition}
\begin{proof}
  Let \( I \) be a two-sided ideal for the magma \( M \). For \( i, j \in I \), since \( I \) is a left ideal, we have \( ij \in I \) and similarly \( ji \in I \) since \( I \) is a right ideal. Thus \( II = I \) and \( I \) is a submagma of \( M \).
\end{proof}

\begin{example}\label{ex:subgroup_is_not_ideal}
  We explicitly give a counterexample to the converse of \cref{thm:magma_ideal_is_submagma}. Define \( G \coloneqq \Z \times \Z \) to be the abelian group\Tinyref{def:abelian_group} given by pointwise addition. Define
  \begin{equation*}
    G' \coloneqq \{ (n, n) \colon n \in \Z \}.
  \end{equation*}

  The set \( G' \) is a subgroup of \( G \) since it is closed under addition and it contains the unit element \( (0, 0) \). It is not an ideal, however, since
  \begin{equation*}
    (n, n) + (n, 0) = (2n, n) \not\in G'.
  \end{equation*}
\end{example}

\begin{proposition}\label{thm:unital_magma_ideal_is_submagma_iff_contains_identity}
  A two-sided ideal of a unital magma is a unital submagma if and only if it contains the identity.
\end{proposition}
\begin{proof}
  Follows from \cref{thm:magma_ideal_is_submagma} and \cref{thm:proper_ideals_containing_identity}.
\end{proof}

\begin{proposition}\label{thm:proper_ideals_containing_identity}
  A left or right ideal of a unital magma contains the identity if and only if it is not proper.
\end{proposition}
\begin{proof}
  Let \( M \) be a unital magma and \( I \) be a left ideal of \( M \). We will prove that \( e \in I \iff I = M \).

  \begin{description}
    \Implies Let \( e \in I \). Then \( ex = x \) for any \( x \in M \), thus \( IM = M \). But \( I \) is an ideal, hence we have that \( IM = I \), thus \( I = IM = M \).
    \ImpliedBy If \( I = M \), then obviously \( e \in M = R \).
  \end{description}

  An analogous proof follows for the case when \( I \) is a right ideal.
\end{proof}

\begin{proposition}\label{thm:commutative_magma_ideals}
  In a commutative\Tinyref{def:algebraic_theory/commutativity} magma\Tinyref{def:magma/magma} \( M \), a subset \( I \subseteq M \) is a left ideal if and only if it is a right ideal. That is, in commutative magmas, it makes no sense to distinguish between left, right and two-sided ideals.
\end{proposition}
\begin{proof}
  For \( i \in I \) and \( x \in M \) by commutativity we have \( ix = xi \), thus \( I \) is a left ideal if and only if it is a right ideal.
\end{proof}

\begin{proposition}\label{thm:product_of_semigroup_ideals_is_in_intersection}
  Fix a semigroup \( G \). If \( I \) and \( J \) are two-sided ideals, so are \( IJ \) and \( I \cap J \) and
  \begin{equation*}
    IJ \subseteq I \cap J.
  \end{equation*}
\end{proposition}
\begin{proof}
  We first show that \( IJ \) is an ideal.

  Take \( i \in I \), \( j \in J \). If \( g \in G \), then associativity gives us
  \begin{equation*}
    g(ij) = (gi)j \in (gi)J \subseteq IJ
  \end{equation*}
  and
  \begin{equation*}
    (ij)g = i(jg) \in I(jg) \subseteq IJ.
  \end{equation*}

  Hence \( IJ \) is closed under the semigroup operation. This makes \( IJ \) a two-sided ideal.

  If \( i \in I \cap J \) and \( g \in G \), obviously \( ig \in I \) and \( ig \in J \), hence \( ig \in I \cap J \). Then \( I \cap J \) is also a two-sided ideal.

  For the inclusion
  \begin{equation*}
    IJ \subseteq I \cap J,
  \end{equation*}
  observe that \( ij \in IJ \) means that \( ij \in iJ = J \) and \( ij \in Ij = I \), thus \( ij \in I \cap J \) and the inclusion holds.
\end{proof}
