\subsection{Ring ideals}\label{subsec:ring_ideals}

\begin{proposition}\label{thm:semiring_ideal_iff_kernel}
  A subset of a ring is a two-sided \hyperref[def:semiring_ideal]{ideal} if and only if it is the \hyperref[def:semiring_kernel]{kernel} of some ring homomorphism.
\end{proposition}
\begin{proof}
  \SufficiencySubProof Let \( I \) be a two-sided ideal. Since it is an abelian group, \( I \) is a normal subgroup and thus we can form the quotient \hyperref[def:normal_subgroup]{group} \( R / I \) with the canonical projection
  \begin{balign*}
     & \pi: R \to R / I       \\
     & \pi(x) \coloneqq x + I
  \end{balign*}

  Multiplication in \( R \) induces multiplication in \( R / I \) by
  \begin{equation*}
    (x + I) \cdot (y + I) \coloneqq (xy + I).
  \end{equation*}

  It is well-defined since if \( x + I = x' + I \) and \( y + I = y' + I \), then
  \begin{balign*}
    (x + I) (y + I)
     & =
    xy + (Iy + xI + II)
    =    \\ &=
    xy + I
    =    \\ &=
    x'y' + I
    =    \\ &=
    x'y' + (Iy' + x'I + II)
    =    \\ &=
    (x' + I) (y' + I).
  \end{balign*}

  Thus, the ring structure on \( R \) induces a ring structure on \( R / I \).

  The canonical projection \( \pi \) is an additive group homomorphism. Since we just showed that \( \pi(xy) = \pi(x) \pi(y) \), it follows that it is also a ring homomorphism.

  It only remains to show that \( \ker(\pi) = I \). Since \( I \) is closed under addition, naturally \( I \subseteq \ker(\pi) \). Conversely, if \( x \in \ker(\pi) \), then \( \pi(x) = \pi(0) = I \), i.e. \( x \in I \). Hence, \( \ker(\pi) = I \).

  \NecessitySubProof Let \( f: R \to T \) is a ring homomorphism. We must show that \( \ker(f) \) is an ideal. If \( x \in R \) and \( y \in \ker(f) \), then
  \begin{equation*}
    f(xy) = f(x) f(y) = f(x) 0 = 0.
  \end{equation*}

  Thus, \( xy \in \ker(f) \). Similarly, we can show that \( yx \in \ker(f) \). Thus, \( R \ker(f) = \ker(f) R = \ker(f) \) and \( \ker(f) \) is a two-sided ideal.
\end{proof}
