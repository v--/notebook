\subsection{Euclidean plane}\label{subsec:euclidean_plane}

\begin{definition}\label{def:euclidean_plane}
  We call the two-dimensional \hyperref[def:euclidean_space]{Euclidean space} \( \BbbR^2 \) the \term{Euclidean plane}.
\end{definition}

\begin{remark}\label{rem:euclidean_plane_embedding}
  Every \hyperref[def:affine_plane]{affine plane} in \( \BbbR^n \) is isomorphic to \( \BbbR^2 \), and we can use the concepts from this subsection as long as we fix a plane.
\end{remark}

\begin{remark}\label{rem:xyz}
  By convention, depending on the context, in \hyperref[def:euclidean_plane]{Euclidean planes} the letters \( x \) and \( y \) have several meanings:
  \begin{itemize}
    \item The vectors of the \hyperref[def:sequence_space]{standard basis}.
    \item The corresponding \hyperref[def:euclidean_plane]{coordinate axes}.
    \item The \hyperref[def:affine_coordinate_system]{(affine) coordinates} of some arbitrary point.
  \end{itemize}

  We call the axis \( x \) the \term{abscissa} and \( y \) --- the \term{ordinate}.

  In three-dimensional \hyperref[def:euclidean_space]{Euclidean spaces}, we use the letter \( z \) to denote the third coordinate axis and call it the \term{applicata}.

  To avoid confusion, we avoid using \( x \), \( y \) and \( z \) in Euclidean planes to denote points and vectors.
\end{remark}

\begin{definition}\label{def:plane_line_equations}
  \hyperref[def:affine_line]{Lines} in \( \BbbR^2 \) are so ubiquitous that they are often represented via a variety of \hyperref[ex:equations]{equations}.

  We will start with \fullref{def:affine_line/parametric} --- the \hyperref[def:affine_operator]{affine} \hyperref[def:parametric_curve]{parametric curve}
  \begin{equation}\label{eq:def:plane_line_equations/parametric}
    l(t) = O + td
  \end{equation}

  \begin{figure}[!ht]
    \centering
    \includegraphics[align=c]{output/thm__plane_line_equations__cartessian.pdf}
    \caption{A \hyperref[def:affine_line]{line} in \( \BbbR^2 \) defined using its \hyperref[def:plane_line_equations/cartesian]{Cartesian equation}.}\label{fig:def:plane_line_equations/cartesian}
  \end{figure}

  \begin{thmenum}
    \thmitem{def:plane_line_equations/vector_parametric}\mcite[def. 2.1]{ВеселовТроицкий2002Лекции} We call \eqref{eq:def:plane_line_equations/parametric} a \term{vector parametric equation} of \( L \).

    \medspace

    \thmitem{def:plane_line_equations/scalar_parametric}\mcite[def. 2.1]{ВеселовТроицкий2002Лекции} The parametric equation \eqref{eq:def:plane_line_equations/parametric} can be rewritten as
    \begin{equation}\label{eq:def:plane_line_equations/scalar_parametric}
      \begin{cases}
         &l_x(t) = x_o + t x_d, \\
         &l_y(t) = y_o + t y_d.
      \end{cases}
    \end{equation}

    We say that these are \term{scalar parametric equations} of the line. They are non-unique by the same reason as the vector parametric equation.

    \thmitem{def:plane_line_equations/general}\mcite[exmpl. 2.4]{ВеселовТроицкий2002Лекции} The image of the scalar equations \eqref{eq:def:plane_line_equations/scalar_parametric} consists of all pairs \( (x, y) \) such that
    \begin{equation}\label{eq:def:plane_line_equations/general}
      \underbrace{ Ax + By + C }_{ p(x, y) } = 0
    \end{equation}
    for some scalars \( A \), \( B \) and \( C \), where \( A \) or \( B \) (or both) are nonzero.
    We call \eqref{eq:def:plane_line_equations/general} a \term{general equation} of the line.

    More concretely,
    \begin{equation*}
      A(o_x + td_x) + B(o_y + td_y) + C = 0
    \end{equation*}
    for all \( t \) if \( A = d_y \), \( B = -d_x \) and \( C = o_y d_x - o_x d_y \).

    Conversely, given the general equation \eqref{eq:def:plane_line_equations/general}, assuming \( A \neq 0 \), we can define the parametric equations
    \begin{equation*}
      \begin{cases}
        &l_x(t) \coloneqq -\tfrac C A - t \tfrac B A  \\
        &l_y(t) \coloneqq t.
      \end{cases}
    \end{equation*}

    The case when \( A = 0 \) and \( B \neq 0 \) is handled analogously.

    Note that multiple general equations can have the same locus --- actually all scalar multiples of \( p(x, y) \). If \( A^2 + B^2 = 1 \) in \eqref{eq:def:plane_line_equations/general}, we call it a \term{normal equation}. There are only two normal equations.

    \thmitem{def:plane_line_equations/cartesian} It is common, especially in analysis, to use the \term{Cartesian equation}
    \begin{equation}\label{eq:def:plane_line_equations/cartesian}
      y = kx + m
    \end{equation}
    for some scalars \( k \) and \( m \). We call \( k \) the \term{slope} of the line.

    It is a special case of the general equation \eqref{eq:def:plane_line_equations/general} with \( A = -k \), \( B = -1 \) and \( C = m \).

    Unlike the general equation, the Cartesian equation of a line is unique, but it cannot express vertical lines. If \( B \neq 0 \) in \eqref{eq:def:plane_line_equations/general}, we can define \( k = -\ifrac A B \) and \( m = -\ifrac C B \) to form a Cartesian equation.

    \begin{figure}[!ht]
      \centering
      \includegraphics[align=c]{output/thm__plane_line_equations__cartessian.pdf}
      \caption{A \hyperref[def:affine_line]{line} in \( \BbbR^2 \) defined using its \hyperref[def:plane_line_equations/cartesian]{Cartesian equation}.}\label{fig:def:plane_line_equations/cartesian_drawing}
    \end{figure}

    \thmitem{def:plane_line_equations/intercept} Another equation that is occasionally used is the \term{intercept equation}
    \begin{equation}\label{eq:def:plane_line_equations/intercept}
      \frac x a + \frac y b = 1
    \end{equation}
    for some nonzero real numbers \( a \) and \( b \).

    It is again a special case of the general equation \eqref{eq:def:plane_line_equations/general} with \( A = \ifrac 1 a \), \( B = \ifrac 1 b \) and \( C = -1 \). It is also unique, but it cannot express neither vertical nor horizontal lines, nor lines passing through the origin.

    Conversely, if \( A \), \( B \) and \( C \) are all nonzero in the general equation \eqref{eq:def:plane_line_equations/general}, we can define an intercept equation as \( a = -\ifrac C A \) and \( b = -\ifrac C B \).
  \end{thmenum}
\end{definition}

\begin{proposition}\label{thm:coordinates_of_directional_vector}
  If \( A x + B y + C = 0 \) is a \hyperref[def:plane_line_equations/general]{general equation} of some line, then \( (x_0, y_0) \) is a directional vector of the line if and only if
  \begin{equation*}
    A x_0 + B y_0 = 0.
  \end{equation*}
\end{proposition}
\begin{proof}
  \SufficiencySubProof Suppose that \( (x_0, y_0) \) is a directional vector, and let \( (x_1, y_1) \) be the coordinates of some point on the line. Then \( (x_0 + x_1, y_0 + y_1) \) is also a point, and
  \begin{equation*}
    A x_0 + B y_0 + C = 0 = A (x_0 + x_1) + B (y_0 + y_1) + C.
  \end{equation*}

  Then \( A x_0 + B y_0 = 0 \).

  \NecessitySubProof Suppose that \( A x_0 + B y_0 = 0 \). Let \( (x_1, y_1) \) and be a point on the line. Then
  \begin{equation*}
    A x_1 + B y_1 + C = 0 = A x_0 + B y_0.
  \end{equation*}
  and
  \begin{equation*}
    A (x_1 + t x_0) + B (y_1 + t y_0) + C = A x_1 + B y_1 + C + t (A x_0 + B y_0) = 0.
  \end{equation*}

  That is, \( (x_0, y_0) \) is a directional vector of the line.
\end{proof}

\begin{proposition}\label{thm:parallel_lines_in_plane}
  Let \( A_g x + B_g y + C_g = 0 \) and \( A_h x + B_h y + C_h = 0 \) be the \hyperref[def:plane_line_equations/general]{general equations} of the lines \( g \) and \( h \). The lines are \hyperref[def:affine_parallelism]{parallel} if and only if the vectors \( (A_g, B_g) \) and \( (A_h, B_h) \) are linearly dependent, and they coincide if and only if \( (A_g, B_g, C_g) \) and \( (A_h, B_h, C_h) \) are linearly dependent.
\end{proposition}
\begin{proof}
  \SufficiencySubProof Suppose that \( g \) and \( h \) are parallel, and let \( (x, y) \) be a directional vector for them. Then \Fullref{thm:coordinates_of_directional_vector} implies that
  \begin{equation*}
    0 = A_g x + B_g y = A_h x + B_h y.
  \end{equation*}

  Hence,
  \begin{equation*}
    \begin{pmatrix}
      A_h & B_h \\
      A_g & B_g
    \end{pmatrix}
    \begin{pmatrix}
      x \\ y
    \end{pmatrix}
    =
    \begin{pmatrix}
      0 \\ 0
    \end{pmatrix},
  \end{equation*}
  implying via \fullref{thm:matrix_invertibility} that \( (A_h, B_h) \) and \( (A_g, B_g) \) are linearly dependent.

  Finally, suppose that \( g \) and \( h \) coincide, and let \( (x_0, y_0) \) be a point. We have already shown that there exists some nonzero \( \lambda \) such that \( (A_h, B_h) = \lambda (A_g, B_g) \), thus
  \begin{equation*}
    0 = A_h x_0 + B_h y_0 + C_h = \lambda (A_g x_0 + B_g y_0 + C_g) - \lambda C_g + C_h,
  \end{equation*}
  implying \( C_h = \lambda C_g \).

  \NecessitySubProof Let \( (x_g, y_g) \) and \( (x_h, y_h) \) be directional vectors for \( g \) and \( h \). Then \Fullref{thm:coordinates_of_directional_vector} implies that
  \begin{equation*}
    A_g x_g + B_g y_g = 0 = A_h x_h + B_h y_h.
  \end{equation*}

  Suppose that the vectors \( (A_g, B_g) \) and \( (A_h, B_h) \) are linearly dependent, i.e. there exists some \( \lambda \) such that \( (A_h, B_h) \) and \( \lambda (A_g, B_g) \). Then
  \begin{equation*}
    0 = A_h x_h + B_h y_h = \lambda (A_g x_h) + \lambda (B_g y_h).
  \end{equation*}

  Hence,
  \begin{equation*}
    \begin{pmatrix}
      x_g & y_g \\
      x_h & y_h
    \end{pmatrix}
    \begin{pmatrix}
      A_g \\ B_g
    \end{pmatrix}
    =
    \begin{pmatrix}
      0 \\ 0
    \end{pmatrix}
  \end{equation*}
  implying via \fullref{thm:matrix_invertibility} that \( (x_g, y_g) \) and \( (x_h, y_h) \) are linearly dependent.

  Therefore, \( g \) and \( h \) are parallel lines.

  If, in addition, \( C_h = \lambda C_g \), then for every point \( (x_0, y_0) \) in \( g \),
  \begin{equation*}
    A_h x_0 + B_h y_0 + C_h = \lambda (A_g x_0 + B_g y_0 + C_g) = 0,
  \end{equation*}
  hence \( g \) and \( h \) coincide.
\end{proof}

\begin{proposition}\label{thm:lines_intersect_in_plane}
  Two distinct \hyperref[def:affine_line]{lines} in the Euclidean plane intersect if and only if they are not \hyperref[def:affine_parallelism]{parallel}. Furthermore, non-parallel lines intersect in exactly one point.
\end{proposition}
\begin{proof}
  Let \( A_g x + B_g y + C_g = 0 \) and \( A_h x + B_h y + C_h = 0 \) be the \hyperref[def:plane_line_equations/general]{general equations} of the lines \( g \) and \( h \). Consider the \hyperref[rem:system_of_equations]{system of linear equations}
  \begin{equation}\label{eq:thm:lines_intersect_in_plane/system}
    \begin{pmatrix}
      A_g & B_g \\
      A_h & B_h
    \end{pmatrix}
    \begin{pmatrix}
      x \\ y
    \end{pmatrix}
    =
    -
    \begin{pmatrix}
      C_g \\ C_h
    \end{pmatrix}.
  \end{equation}

  The system has a solution if and only if the lines intersect.

  \SufficiencySubProof If \( g \) and \( h \) intersect, the system \eqref{eq:thm:lines_intersect_in_plane/system} has a solution, and \fullref{thm:kroneker_capelli} implies that the vector \( (C_g, C_h) \) belongs to the column space of the matrix. Then either:
  \begin{itemize}
    \item The lines are parallel and, by \fullref{thm:parallel_lines_in_plane}, the lines coincide --- this contradicts our assumption that the lines are distinct.
    \item The lines are not parallel.
  \end{itemize}

  \NecessitySubProof Suppose that \( g \) and \( h \) are not parallel. \Fullref{thm:parallel_lines_in_plane} implies that the vectors \( (A_g, A_h) \) and \( (B_g, B_h) \) are linearly independent, and \fullref{thm:matrix_invertibility} implies that
  \begin{equation*}
    \begin{pmatrix}
      x \\ y
    \end{pmatrix}
    =
    -
    \begin{pmatrix}
      A_g & B_g \\
      A_h & B_h
    \end{pmatrix}^{-1}
    \begin{pmatrix}
      C_g \\ C_h
    \end{pmatrix}.
  \end{equation*}

  \UniquenessSubProof Suppose that \( g \) and \( h \) are not parallel. \Fullref{thm:parallel_lines_in_plane} implies that the matrix in \eqref{eq:thm:lines_intersect_in_plane/system} is invertible. Then \fullref{thm:system_of_equations_unique_solution} implies that the system has a unique solution.
\end{proof}

\begin{lemma}\label{thm:rotation_matrix_symmetry}
  For a \hyperref[def:rigid_motion/rotation]{rotation matrix}
  \begin{equation*}
    A = \begin{pmatrix}
      a & b \\
      c & d
    \end{pmatrix},
  \end{equation*}
  i.e. an \hyperref[def:unitary_matrix]{orthogonal matrix} with \hyperref[def:matrix_determinant]{determinant} \( 1 \), we have \( b = -c \) and \( d = a \). That is,
  \begin{equation*}
    A = \begin{pmatrix}
      a & -c \\
      c & a
    \end{pmatrix}
  \end{equation*}
  and
  \begin{equation*}
    a^2 + c^2 = 1.
  \end{equation*}
\end{lemma}
\begin{proof}
  Since \( A \) is orthogonal, \( ab + cd = 0 \). Also, \( \det A = ad - bc = 1 \). Either \( a \) or \( b \) or both must be nonzero, because otherwise \( A \) would be singular.
  \begin{itemize}
    \item If \( a = 0 \), then \( cd = 0 \) and hence either \( c = 0 \) or \( d = 0 \). If \( c = 0 \), then \( \det A = 0 \), which contradicts the assumption that \( \det A = 1 \). Thus, \( c \neq 0 \) and \( d = 0 \).

    Then \( b^2 = c^2 = 1 \) since the columns are normed, and also \( \det A = -bc = -1 < 0 \). Thus, \( \abs{b} = \abs{c} = 1 \) and they have different signs.

    \item If \( a \neq 0 \), then \( b = -\ifrac {cd} a \) and \( ad + \ifrac {c^2 d} a = 1 \). Multiplying both sides by \( a \), we obtain
    \begin{equation*}
      (a^2 + c^2) d = a.
    \end{equation*}

    But \( a^2 + c^2 = 1 \) because the columns of \( A \) are normed. Thus, \( d = a \) and \( c = -b \).
  \end{itemize}

  In both cases,
  \begin{equation*}
    A
    =
    \begin{pmatrix}
      a & -c \\
      c & a
    \end{pmatrix}
  \end{equation*}
  and also
  \begin{equation*}
    \det A = a^2 + c^2 = 1.
  \end{equation*}
\end{proof}

\begin{proposition}\label{thm:plane_ray_abscissa_rotation}
  Every \hyperref[def:geometric_ray]{ray} at the origin in the Euclidean plane is a \hyperref[def:rigid_motion/rotation]{rotation} of the \hyperref[rem:xyz]{abscissa}. Furthermore, this rotation is unique.
\end{proposition}
\begin{proof}
  Let \( r(t) = td \) be some ray and let \( (x, y) \) be the coordinates of the vector \( d \).

  \UniquenessSubProof Suppose that there exist rotations \( A \) and \( B \) such that
  \begin{equation*}
    \begin{pmatrix} x \\ y \end{pmatrix} = A \begin{pmatrix} 1 \\ 0 \end{pmatrix} = B \begin{pmatrix} 1 \\ 0 \end{pmatrix}.
  \end{equation*}

  \Fullref{thm:rotation_matrix_symmetry} then implies that
  \begin{equation*}
    A = B = \begin{pmatrix}
      x & -y \\
      y & x
    \end{pmatrix}.
  \end{equation*}

  \ExistenceSubProof The matrix
  \begin{equation*}
    \frac 1 {x^2 + y^2}
    \begin{pmatrix}
      x & -y \\
      y & x
    \end{pmatrix}.
  \end{equation*}
  is obviously a rotation matrix.

  Since \( (1, 0) \) are the coordinates of the abscissa basis vector,
  \begin{equation*}
    \frac 1 {x^2 + y^2}
    \begin{pmatrix}
      x & -y \\
      y & x
    \end{pmatrix}
    \begin{pmatrix}
      1 \\ 0
    \end{pmatrix}
    =
    \frac 1 {x^2 + y^2}
    \begin{pmatrix}
      x \\ y
    \end{pmatrix}.
  \end{equation*}

  This vector is \hyperref[def:geometric_ray/unidirectional]{unidirectional} with \( d \), hence its ray at the origin coincides with \( r \).
\end{proof}

\begin{proposition}\label{thm:plane_ray_rotation}
  For every point \( O \) and every pair of rays \( r(t) = O + td \) to \( s(t) = O + te \), there exists a unique \hyperref[def:rigid_motion]{rigid motion} \( f(v) \), whose linear part of \( f(v) \) is a \hyperref[def:rigid_motion/rotation]{rotation}, sending (the image of) \( r \) to \( s \). Furthermore, \( s(t) = f(r(t)) \).
\end{proposition}
\begin{proof}
  \SubProof{Proof that \( s = f \bincirc r \)} Suppose that \( f(v) = a + Tv \) is a rigid motion sending the image of \( r(t) \) to the image of \( s(t) \). That is, \( f(\img r) = \img s \), but we do not know how the functions \( f \), \( r \) and \( s \) relate.

  Suppose that \( \norm{d} = \norm{e} = 1 \). Let \( f(o) = s(\lambda) \), i.e.
  \begin{equation*}
    a + T O = f(o) = f(r(0)) = s(\lambda) = O + \lambda e.
  \end{equation*}

  Next, suppose that \( f(r(1)) = s(\mu) \). That is,
  \begin{equation*}
    f(r(1)) = a + T (o + d) = O + \mu e = s(\mu).
  \end{equation*}

  Then
  \begin{equation*}
    (a + T o) + T d = O + \mu e
  \end{equation*}
  and
  \begin{equation*}
    Td = (\mu - \lambda) e.
  \end{equation*}

  Note that \( T \) preserves norms as an orthogonal transformation, hence
  \begin{equation*}
    \underbrace{\norm{Td}}_{1} = (\mu - \lambda) \underbrace{\norm{e}}_{1}.
  \end{equation*}

  It follows that \( \mu = \lambda + 1 \) and \( e = Td \). Therefore,
  \begin{equation*}
    f(r(t)) = a + T(o + td) = (a + To) + t e.
  \end{equation*}

  Note that there exists some scalar \( \tau \) such that
  \begin{equation*}
    f(r(\tau)) = s(0) = o.
  \end{equation*}

  Then
  \begin{equation*}
    O = f(r(\tau)) = a + T O + \tau e = O + \lambda e + t e
  \end{equation*}
  and \( \lambda = -\tau \).

  Since both \( \lambda \) and \( \tau \) are nonnegative, it follows that \( \lambda = -\tau = 0 \). Then
  \begin{equation*}
    a + T O = f(r(0)) = s(\lambda) = s(0) = o,
  \end{equation*}
  hence \( a = O - T O \).

  Thus,
  \begin{equation*}
    f(v) = O - T O + Tv = O + T(v - o).
  \end{equation*}

  In particular,
  \begin{equation*}
    f(r(t)) = O + T(o + td - o) = O + t Td = O + te = s(t).
  \end{equation*}

  \UniquenessSubProof Suppose that \( f(v) \) and \( g(v) \) are rigid motions sending the ray \( r(t) \) to \( s(t) \). We have already shown that \( O \) is a fixed point of both \( f(v) \) and \( g(v) \), hence, by \fullref{thm:rigid_motion_fixed_point}, there exist some rotation operators \( F \) and \( G \) such that \( f(v) = O + F(v - o) \) and \( g(v) = O + G(v - o) \). Then
  \begin{equation*}
    \vect 0 = s(t) - s(t) = f(r(t)) - g(r(t)) = O + F(v - o) - O - G(v - o).
  \end{equation*}

  Therefore,
  \begin{equation*}
    F(v - o) = G(v - o)
  \end{equation*}
  and
  \begin{equation*}
    (G - F) v = (G - F) o.
  \end{equation*}

  Since this holds for arbitrary \( v \), it is only possible that \( (G - F) v = (G - F) O = \vect 0 \). Thus, \( F = G \) and \( f(v) = g(v) \).

  \ExistenceSubProof \Fullref{thm:plane_ray_abscissa_rotation} implies that there exists a unique rotation \( R \) sending the abscissa to \( r'(t) = td \) and \( S \) sending it to \( s'(t) = te \).

  Then
  \begin{equation*}
    S^{-1} e
    =
    S^{-1} S \begin{pmatrix} 1 \\ 0 \end{pmatrix}
    =
    R^{-1} R \begin{pmatrix} 1 \\ 0 \end{pmatrix}
    =
    R^{-1} d.
  \end{equation*}

  Hence,
  \begin{equation*}
    e = S R^{-1} d.
  \end{equation*}

  That is, \( T \coloneqq R^{-1} S \) sends \( r' \) to \( s' \). Then \( O + T(v - o) \) sends \( r \) to \( s \).
\end{proof}

\begin{proposition}\label{thm:plane_rotation_matrix}
  The map
  \begin{equation}\label{eq:thm:plane_rotation_matrix}
    \varphi
    \mapsto
    \begin{pmatrix}
      \cos \varphi & -\sin \varphi \\
      \sin \varphi & \cos \varphi
    \end{pmatrix}
  \end{equation}
  is an \hyperref[def:morphism_invertibility/right_cancellative]{epimorphism} from the real numbers under addition to the group of \hyperref[def:rigid_motion/rotation]{rotations} in \( \BbbR^2 \) under composition. The kernel of this map is the set of multiples of \( 2\pi \).

  We call \( \varphi \) the \term{angle} of the rotation; the semantics of the work \enquote{angle} are discussed in \fullref{def:angle}.

  There are other groups isomorphic to the rotation group --- see \fullref{def:circle_group}.
\end{proposition}
\begin{proof}
  \SubProof{Proof of well-definedness} The matrix \eqref{eq:thm:plane_rotation_matrix} is orthogonal, and its determinant is \( 1 \) as a consequence of \fullref{thm:trigonometric_identities/pythagorean_identity}. Hence, it induces a rotation.

  \SubProofOf[def:function_invertibility/surjective/equality]{surjectivity} Let
  \begin{equation*}
    A
    =
    \begin{pmatrix}
      a & -c \\
      c & c
    \end{pmatrix}
  \end{equation*}
  be a rotation matrix; i.e. \( \det A = a^2 + c^2 = 1 \). Every rotation matrix has this form as discussed in \fullref{thm:rotation_matrix_symmetry}.

  Define \( \varphi \) as
  \begin{equation*}
    \varphi \coloneqq \begin{cases}
      \arccos a,  &c \geq 0, \\
      -\arccos a, &c < 0.
    \end{cases}
  \end{equation*}

  Then
  \begin{equation*}
    (\sin \varphi)^2 = 1 - (\cos \varphi)^2 = 1 - a^2 = c^2.
  \end{equation*}

  \begin{itemize}
    \item If \( c \geq 0 \), then \( \varphi \in [0, \pi) \) and hence \( \sin \varphi \geq 0 \). Since the square root has nonnegative values,
    \begin{equation*}
      \sin \varphi = \sqrt{ 1 - a^2 } = c.
    \end{equation*}

    \item If \( c < 0 \), then \( \varphi \in (-\pi, 0) \) and hence \( \sin \varphi < 0 \). Thus,
    \begin{equation*}
      \sin \varphi = -\sqrt{ 1 - a^2 } = c.
    \end{equation*}
  \end{itemize}

  Therefore,
  \begin{equation*}
    \begin{pmatrix}
      \cos \varphi & -\sin \varphi \\
      \sin \varphi & \cos \varphi
    \end{pmatrix}
    =
    \begin{pmatrix}
      a & -c \\
      c & a
    \end{pmatrix}
    =
    A.
  \end{equation*}

  \SubProof{Proof of homomorphism condition} We have
  \begin{equation*}
    \cos(\varphi + \psi)
    \reloset {\eqref{eq:thm:trigonometric_identities/sum_of_angles/cos}} =
    \cos \varphi \cos \psi - \sin \varphi \sin \psi
    =
    \begin{pmatrix}
      \cos \varphi & -\sin \varphi
    \end{pmatrix}
    \begin{pmatrix}
      \cos \psi \\ \sin \psi
    \end{pmatrix}.
  \end{equation*}
  and
  \begin{equation*}
    \sin(\varphi + \psi)
    \reloset {\eqref{eq:thm:trigonometric_identities/sum_of_angles/cos}} =
    \cos \varphi \sin \psi + \sin \varphi \cos \psi
    =
    \begin{pmatrix}
      \cos \varphi & \sin \varphi
    \end{pmatrix}
    \begin{pmatrix}
      \sin \psi \\ \cos \psi
    \end{pmatrix}.
  \end{equation*}

  Then
  \begin{equation*}
    \begin{pmatrix}
      \cos \varphi & -\sin \varphi \\
      \sin \varphi & \cos \varphi
    \end{pmatrix}
    \begin{pmatrix}
      \cos \psi & -\sin \psi \\
      \sin \psi & \cos \psi
    \end{pmatrix}
    =
    \begin{pmatrix}
      \cos (\varphi + \psi) & -\sin (\varphi + \psi) \\
      \sin (\varphi + \psi) & \cos (\varphi + \psi)
    \end{pmatrix}.
  \end{equation*}

  \SubProof{Proof that kernel are multiples of \( 2\pi \)} Suppose that
  \begin{equation*}
    \begin{pmatrix}
      \cos \varphi & -\sin \varphi \\
      \sin \varphi & \cos \varphi
    \end{pmatrix}
    =
    \begin{pmatrix}
      \cos \psi & -\sin \psi \\
      \sin \psi & \cos \psi
    \end{pmatrix}.
  \end{equation*}

  Both \( \sin \) and \( \cos \) are bijective on the interval \( [0, 2\pi) \). From \fullref{thm:trigonometric_function_period} it follows that, if \( \cos \varphi = \cos \psi \), then \( 2\pi \) divides \( \varphi - \psi \).
\end{proof}

\begin{definition}\label{def:angle}
  A \term{directed angle} is an ordered pair of \hyperref[def:geometric_ray]{rays} with a common vertex. We denote the angle between the rays \( r \) and \( s \) via \( \sphericalangle(r, s) \).

  \begin{figure}[!ht]
    \centering
    \includegraphics[align=c]{output/def__angle.pdf}
    \caption{The two \hyperref[def:angle]{directed angles} \( \sphericalangle(r, s) \) and \( \sphericalangle(r, s) \) given by the rays \( r \) and \( s \).}\label{def:angle/measure/figure}
  \end{figure}

  \begin{thmenum}
    \thmitem{def:angle/measure} Denote by \( O \) the common vertex of \( r \) and \( s \). \Fullref{thm:plane_ray_rotation} implies that there exists a unique \hyperref[def:rigid_motion]{rigid motion} \( f(v) = O + T(v - o) \), where \( T \) is a \hyperref[def:rigid_motion/rotation]{rotation}, sending \( r \) to \( s \). \Fullref{thm:plane_rotation_matrix} then implies the existence of a unique number \( \varphi \in [0, 2\pi) \) entirely determining \( T \). We will call \( \varphi \) the \term{measure} of \( \sphericalangle(r, s) \).

    It is conventional to conflate an angle and its measure. For this reason, it is sometimes convenient to define angles via directional vectors rather than rays, disregarding the vertex.

    We can classify angles based on their measure as
    \begin{thmenum}
      \thmitem{def:angle/measure/zero} \term{zero} if \( \varphi = 0 \),
      \thmitem{def:angle/measure/acute} \term{acute} if \( 0 < \varphi < \ifrac \pi 2 \),
      \thmitem{def:angle/measure/right} \term{right} if \( \varphi = \tfrac \pi 2 \),
      \thmitem{def:angle/measure/obtuse} \term{obtuse} if \( \ifrac \pi 2 < \varphi < \pi \),
      \thmitem{def:angle/measure/straight} \term{straight} if \( \varphi = \pi \), in which case the angle is actually a line,
      \thmitem{def:angle/measure/reflex} \term{reflex} if \( \varphi > \pi \).
    \end{thmenum}

    \thmitem{def:angle/undirected} Given the transformation \( f(v) \) sending \( r \) to \( s \), its inverse \( f^{-1}(v) \) sends \( s \) to \( r \). Their composition is the identity, whose angle measure is a multiple \( 2\pi \). \Fullref{thm:plane_rotation_matrix} implies that the angle measures \( \sphericalangle(r, s) \) and \( \sphericalangle(s, r) \) sum to \( 2\pi \). We call the angle with the smaller measure the \term{undirected angle} between \( r \) and \( s \).
  \end{thmenum}
\end{definition}

\begin{proposition}\label{thm:cosine_of_angle_measure}
  The measure \( \alpha \) of an angle \( \sphericalangle(u, v) \) between the normed vectors \( u \) and \( v \) satisfies
  \begin{equation}\label{eq:thm:cosine_of_angle_measure}
    \cos(\alpha) = \inprod u v.
  \end{equation}

  Furthermore, the measure of \( \sphericalangle(v, u) \) also satisfies \eqref{eq:thm:cosine_of_angle_measure}, hence also the measure of the undirected angle.
\end{proposition}
\begin{proof}
  \Fullref{thm:cauchy_bunyakovsky_schwarz_inequality} implies that \( \inprod u v \) ranges between \( -1 \) and \( 1 \); hence, it is the cosine of a real number.

  \Fullref{thm:plane_ray_abscissa_rotation} gives us an angle \( \beta \) whose rotation sends the abscissa into \( u \), and a similar angle \( \gamma \) for \( v \). The coordinates of \( u \) are
  \begin{equation*}
    \begin{pmatrix}
      \cos \beta & -\sin \beta \\
      \sin \beta & \cos \beta
    \end{pmatrix}
    \begin{pmatrix}
      1 \\ 0
    \end{pmatrix}
    =
    \begin{pmatrix}
      \cos \beta \\ \sin \beta
    \end{pmatrix},
  \end{equation*}
  and similarly for \( v \).

  Then
  \begin{align*}
    \inprod u v
    &=
    \cos \beta \cos \gamma + \sin \beta \sin \gamma
    \reloset {\ref{thm:trigonometric_identities/products}} = \\ &=
    \ifrac 1 2 \parens[\Big]{ \cos(\beta - \gamma) + \cos(\beta + \gamma) + \cos(\beta - \gamma) - \cos(\beta + \gamma) }
    = \\ &=
    \cos(\beta - \gamma)
    = \\ &=
    \cos(\gamma - \beta)
    = \\ &=
    \cos(\alpha).
  \end{align*}
\end{proof}

\begin{proposition}\label{thm:adjacent_angles}
  For every triple \( r \), \( s \) and \( p \) of rays with a common vertex, we have the following \hyperref[rem:congruence_modulo_real_number]{congruence} of \hyperref[def:angle/measure]{angle measures}:
  \begin{equation*}
    \sphericalangle(r, p) \equiv \sphericalangle(r, s) + \sphericalangle(s, p) \pmod {2\pi}.
  \end{equation*}

  \begin{figure}[!ht]
    \centering
    \includegraphics[align=c]{output/thm__sum_of_angles.pdf}
    \caption{Sum of two angles.}\label{fig:thm:adjacent_angles}
  \end{figure}

  We call the angles \( \sphericalangle(r, s) \) and \( \sphericalangle(s, p) \) \term{adjacent angles}.
\end{proposition}
\begin{proof}
  Let \( f(v) = o + F(v - o) \) be the operator sending \( r \) to \( s \) given by \fullref{thm:plane_ray_rotation} and \( g(v) = o + G(v - o) \) be the map from \( s \) to \( p \). Then, when regarding the rays as parametric curves,
  \begin{equation*}
    [g \bincirc f](r(t)) = g(f(r(t))) = g(p(t)) = s(t),
  \end{equation*}
  thus \( g \bincirc f \) is an affine map sending \( r \) to \( p \).

  Furthermore,
  \begin{equation*}
    [g \bincirc f](v)
    =
    o + G(f(v) - o)
    =
    o + G(o + F(v - o) - o)
    =
    o + GF(v - o),
  \end{equation*}
  hence \( g \bincirc f \) is the unique affine map with a rotation linear map sending \( r \) to \( p \). Uniqueness is shown in \fullref{thm:plane_ray_rotation}.

  We know from \fullref{thm:plane_rotation_matrix} that the angle of the rotation \( GF \) is the sum of angles of \( G \) and \( F \). This concludes the proof.
\end{proof}

\begin{proposition}\label{thm:angles_of_crossing}
  Let \( g \) and \( h \) be \hyperref[def:crossing_lines]{crossing lines} with intersection point \( O \).

  Let \( P \) and \( Q \) be arbitrary points from \( g \) and \( h \) distinct from \( O \); let \( P' \) and \( Q' \) be their \hyperref[def:rigid_motion/point_reflection]{point reflections} about \( O \).

  Then the \hyperref[def:angle]{angles} \( \sphericalangle(\vect{OP}, \vect{OQ}) \) and \( \sphericalangle(\vect{OP'}, \vect{OQ'}) \) have equal measures, and so do \( \sphericalangle(\vect{OQ'}, \vect{OP}) \) and \( \sphericalangle(\vect{OQ}, \vect{OP'}) \). We call the pairs of angles \term{opposite}.

  We usually want to work with \hyperref[def:angle/undirected]{undirected angles}, in which case, if \( \sphericalangle(\vect{OP}, \vect{OQ}) \) is greater than \( \pi \), we \enquote{change the direction of rotation} by replacing \( P \) with \( P' \). This reduces to using angle measures \hyperref[rem:congruence_modulo_real_number]{modulo} \( \pi \).

  We call the smaller of the two undirected angle measures the \term{angle measure} \( \sphericalangle(g, h) \) of the line crossing.

  \begin{figure}[!ht]
    \centering
    \includegraphics[align=c]{output/thm__angles_of_crossing.pdf}
    \caption{Angles formed by crossing lines.}\label{fig:thm:angles_of_crossing}
  \end{figure}
\end{proposition}
\begin{proof}
  \Fullref{thm:adjacent_angles} implies that
  \begin{equation*}
    \sphericalangle(\vect{OP}, \vect{OQ}) + \sphericalangle(\vect{OP'}, \vect{OQ}) = \pi
  \end{equation*}
  and
  \begin{equation*}
    \sphericalangle(\vect{OP'}, \vect{OQ}) + \sphericalangle(\vect{OP'}, \vect{OQ'}) = \pi.
  \end{equation*}

  Then
  \begin{equation*}
    0 = \pi - \pi = \sphericalangle(\vect{OP}, \vect{OQ}) - \sphericalangle(\vect{OP'}, \vect{OQ'}),
  \end{equation*}
  hence
  \begin{equation*}
    \sphericalangle(\vect{OP}, \vect{OQ}) = \sphericalangle(\vect{OP'}, \vect{OQ'}).
  \end{equation*}

  We can prove the other equality analogously.
\end{proof}

\begin{remark}\label{rem:angle}
  To recap, the word \enquote{angle} may refer to:
  \begin{itemize}
    \item The unique real number from \( [0, 2\pi) \) inducing a rotation as shown in \fullref{thm:plane_rotation_matrix}.
    \item Either a \hyperref[def:angle]{directed angle} or its \hyperref[def:angle/measure]{measure}.
    \item Either an \hyperref[def:angle]{undirected angle} or its measure.
    \item The angle of a line crossing discussed in \fullref{thm:angles_of_crossing}.
  \end{itemize}
\end{remark}

\begin{proposition}\label{thm:angles_of_transversal}
  Let \( l \) be a \hyperref[def:transversal_line]{transversal} of the distinct \hyperref[def:affine_parallelism]{parallel lines} \( g \) and \( h \).

  Let \( R \) be the \hyperref[thm:segment_midpoint]{midpoint} of the segment \( [P, Q] \), let \( G \) be the \hyperref[def:rigid_motion/projection]{projection} of \( R \) onto \( g \) and \( H \) --- of \( R \) onto \( h \).

  If either \( P = G \) or \( Q = H \), then the other equality follows, and all angles of the line crossings are right. Otherwise, the angle \( \sphericalangle(\vect{PG}, \vect{PQ}) \) and \( \sphericalangle(\vect{QH}, \vect{QP}) \) have equal measures. We call them \term[bg=вътрешно-кръстни ъгли]{interior alternate angles}. \Fullref{thm:angles_of_crossing} implies that other pairs of angles of the crossed lines are equal.

  \begin{figure}[!ht]
    \centering
    \includegraphics[align=c]{output/thm__angles_of_transversal.pdf}
    \caption{The alternate angles of a transversal crossing parallel lines.}\label{fig:thm:angles_of_transversal}
  \end{figure}
\end{proposition}
\begin{proof}
  If \( P = G \), then \( G \) belongs to the segment \( [P, Q] \), and \fullref{thm:orthogonal_projection_minimizes_distance} implies that the line through \( P \) and \( Q \) is orthogonal to \( g \). Then it is also orthogonal to \( h \), and \( Q = H \).

  Otherwise, if \( P \neq G \), let \( G' \) be a \hyperref[def:rigid_motion/point_reflection]{point reflection} of \( G \) about \( P \) and \( Q' \) --- of \( Q \) about \( P \). Then \fullref{thm:angles_of_crossing} implies that \( \sphericalangle(\vect{PG}, \vect{PQ}) \) has equal measure with its opposite \( \sphericalangle(\vect{PG'}, \vect{PQ'}) \), which in turn is a translation of \( \sphericalangle(\vect{QH}, \vect{QP}) \).
\end{proof}

\begin{definition}\label{def:triangle}
  A \hyperref[def:simplex]{\( 2 \)-simplex} is called a \term{triangle}. This definition holds more generally than Euclidean spaces, however we restrict it because we have not defined \hyperref[def:angle]{angles} more generally.

  \begin{figure}[!ht]
    \centering
    \includegraphics[align=c]{output/def__triangle.pdf}
    \caption{An \hyperref[def:triangle/acute]{acute triangle}.}\label{fig:def:triangle}
  \end{figure}

  Given a triangle with vertices \( A \), \( B \) and \( C \), we define the associated \hyperref[def:line_segment]{line segments}, called the \term{sides} of the triangle, and its \hyperref[def:angle]{\hi{undirected} angles} as
  \begin{align*}
    a \coloneqq [B, C], && \alpha \coloneqq \sphericalangle(\vect{AB}, \vect{AC}), \\
    b \coloneqq [A, C], && \beta  \coloneqq \sphericalangle(\vect{BA}, \vect{BC}),  \\
    c \coloneqq [A, B], && \gamma \coloneqq \sphericalangle(\vect{CA}, \vect{CB}).
  \end{align*}

  We can classify triangles based on their sides as
  \begin{thmenum}
    \thmitem{def:triangle/isosceles} \term{isosceles} if at least two of its sides have equal length
    \thmitem{def:triangle/equilateral} \term{equilateral} if all of its sides have equal length
  \end{thmenum}
  or based on their angles as
  \begin{thmenum}
    \thmitem{def:triangle/acute} \term{acute} if all of its angles are \hyperref[def:angle/measure/acute]{acute}.
    \thmitem{def:triangle/right} \term{right} if at least one of the angles is \hyperref[def:angle/measure/straight]{straight}.
    \thmitem{def:triangle/obtuse} \term{obtuse} if at least one of its angles is \hyperref[def:angle/measure/obtuse]{obtuse}.
  \end{thmenum}
\end{definition}

\begin{proposition}\label{thm:sum_of_triangle_angles}
  The sum of the (measures of) the angles of any \hyperref[def:triangle]{triangle} is \( 2\pi \).
\end{proposition}
\begin{proof}
  \begin{figure}[!ht]
    \centering
    \includegraphics[align=c]{output/thm__sum_of_triangle_angles.pdf}
    \caption{The construction in the proof of \fullref{thm:sum_of_triangle_angles}.}\label{fig:thm:sum_of_triangle_angles}
  \end{figure}

  Consider a triangle with vertices \( A \), \( B \) and \( C \). Let \( g \) be the line containing \( A \) and \( B \) given by \fullref{thm:lines_intersect_in_plane}, and let \( h \) be the line through \( C \) parallel to \( g \) given by \fullref{thm:parallel_line_through_point}.

  Finally, let \( A' \) be the \hyperref[def:rigid_motion/projection]{projection} of \( A \) onto \( h \) and \( B' \) be the projection of \( B \) onto \( h \).

  \Fullref{thm:angles_of_transversal} implies that the angles \( \sphericalangle(\vect{AB}, \vect{AC}) \) and \( \sphericalangle(\vect{CA'}, \vect{CA}) \) both have measure \( \alpha \). Similarly, \( \sphericalangle(\vect{BA}, \vect{BC}) \) and \( \sphericalangle(\vect{CB'}, \vect{CB}) \) both have measure \( \beta \).

  From \fullref{thm:adjacent_angles} it follows that
  \begin{equation*}
    2\pi
    =
    \sphericalangle(\vect{CA'}, \vect{CB'})
    =
    \sphericalangle(\vect{CA'}, \vect{CA}) + \sphericalangle(\vect{CA}, \vect{CB}) + \sphericalangle(\vect{CB}, \vect{CB'})
    =
    \alpha + \gamma + \beta.
  \end{equation*}
\end{proof}
