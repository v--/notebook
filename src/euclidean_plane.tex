\subsection{Euclidean plane}\label{subsec:euclidean_plane}

\begin{remark}\label{rem:geometry_mixed_conventions}
  We use a mixture of conventions in this subsection. For example, initial capital letters like \( A \), \( B \) and \( C \) are used for both scalars in \hyperref[def:plane_line_equations/general]{general line equations} and points in \hyperref[def:triangle]{triangles}. This is not very consistent, but both are long-established conventions.
\end{remark}

\begin{definition}\label{def:plane_line_equations}
  \hyperref[def:affine_line]{Lines} in \( \BbbR^2 \) are so ubiquitous that they are often represented via a variety of \hyperref[ex:equations]{equations}. We say that the set \( L \) of points in \( \BbbR^2 \) is a \term{line} if any of the equivalent conditions hold:

  \begin{thmenum}
    \thmitem{def:plane_line_equations/vector_parametric}\mcite[def. 2.1]{ВеселовТроицкий2002Лекции} \( L \) is the image of the function
    \begin{equation}\label{eq:def:plane_line_equations/parametric}
      l(t) = o + tv
    \end{equation}

    When regarding it as a parametric curve as in \fullref{def:affine_line/parametric}, we call \eqref{eq:def:plane_line_equations/parametric} a \term{vector parametric equation} of \( L \).

    \thmitem{def:plane_line_equations/scalar_parametric}\mcite[def. 2.1]{ВеселовТроицкий2002Лекции} \( L \) is the image of \( l(t) = (l_x(t), l_y(t)) \), where
    \begin{equation}\label{eq:def:plane_line_equations/scalar_parametric}
      \begin{cases}
         &l_x(t) = o_x + t d_x  \\
         &l_y(t) = o_y + t d_y.
      \end{cases}
    \end{equation}

    We say that these are \term{scalar parametric equations} of \( L \). They are non-unique by the same reason as the vector parametric equation.

    \thmitem{def:plane_line_equations/general}\mcite[exmpl. 2.4]{ВеселовТроицкий2002Лекции} \( L \) is the set of all pairs \( (x, y) \) such that
    \begin{equation}\label{eq:def:plane_line_equations/general}
      \underbrace{ Ax + By + C }_{ p(x, y) } = 0
    \end{equation}
    for some scalars \( A \), \( B \) and \( C \) such that \( A \) or \( B \) (or both) are nonzero.

    We call \eqref{eq:def:plane_line_equations/general} a \term{general equation} of \( L \). It is an instance of \fullref{def:affine_line/algebraic}, where we regard a line as an algebraic curve.

    Note that multiple general equations can have the same locus --- actually all scalar multiples of \( p(x, y) \).

    If \( A^2 + B^2 = 1 \) in \eqref{eq:def:plane_line_equations/general}, we call it a \term{normal equation}. There are only two normal equations.

    Given the scalar parametric equations \eqref{eq:def:plane_line_equations/scalar_parametric},
    \begin{equation*}
      A(o_x + td_x) + B(o_y + td_y) + C = 0
    \end{equation*}
    for all \( t \) if \( A = d_y \), \( B = -d_x \) and \( C = o_y d_x - o_x d_y \).

    Conversely, given the general equation \eqref{eq:def:plane_line_equations/general}, assuming \( A \neq 0 \), we can define the parametric equations
    \begin{equation*}
      \begin{cases}
        &l_x(t) \coloneqq -\frac C A - t \frac B A  \\
        &l_y(t) \coloneqq t.
      \end{cases}
    \end{equation*}

    The case when \( A = 0 \) and \( B \neq 0 \) is handled analogously.

    \thmitem{def:plane_line_equations/cartesian} \( L \) is the set of all pairs \( (x, y) \) such that
    \begin{equation}\label{eq:def:plane_line_equations/cartesian}
      y = kx + m
    \end{equation}
    for some scalars \( k \) and \( m \). This is the \term{Cartesian equation} of \( L \). We call \( k \) the \term{slope} of the line.

    \begin{figure}[!ht]
      \centering
      \includegraphics[align=c]{output/thm__plane_line_equations__cartessian.pdf}
      \caption{A \hyperref[def:affine_line]{line} in \( \BbbR^2 \) defined using its \hyperref[def:plane_line_equations/cartesian]{Cartesian equation}.}\label{eq:def:plane_line_equations/cartesian_drawing}
    \end{figure}

    It is a special case of the general equation \eqref{eq:def:plane_line_equations/general} with \( A = -k \), \( B = -1 \) and \( C = m \). Unlike the general equation, the Cartesian equation of a line is unique, but it cannot express vertical lines.

    Conversely, if \( B \neq 0 \) in \eqref{eq:def:plane_line_equations/general}, we can define \( k = -\ifrac A B \) and \( m = -\ifrac C B \) to form a Cartesian equation.

    \thmitem{def:plane_line_equations/intercept} \( L \) is the set of all pairs \( (x, y) \) such that
    \begin{equation}\label{def:plane_line_equations/intercept}
      \frac x a + \frac y b = 1
    \end{equation}
    for some nonzero real numbers \( a \) and \( b \). This is the \term{intercept equation} of \( L \).

    It is again a special case of the general equation \eqref{eq:def:plane_line_equations/general} with \( A = \ifrac 1 a \), \( B = \ifrac 1 b \) and \( C = -1 \). It is also unique, but it cannot express neither vertical nor horizontal lines, nor lines passing through the origin.

    Conversely, if \( A, B, C \neq 0 \) in the general equation \eqref{eq:def:plane_line_equations/general}, we can define an intercept equation as \( a = -\ifrac C A \) and \( b = -\ifrac C B \).
  \end{thmenum}
\end{definition}

\begin{figure}[!ht]
  \centering
  \includegraphics[align=c]{output/def__angle.pdf}
  \caption{An \hyperref[def:angle/acute]{acute angle} with its measurement segments dashed.}\label{def:angle/figure}
\end{figure}

\begin{definition}\label{def:angle}
  A \term{directed angle} is a tuple of two closed \hyperref[def:geometric_ray]{rays} with a common vertex. It is a closed cone. Given two rays \( r_1, r_2 \) with a common vertex, we denote their corresponding directed angle by \( \sphericalangle(r_1, r_2) \).

  Suppose that \( r_1 \) and \( r_2 \) have scalar parametric equations
  \begin{equation*}
    r_i: t \mapsto
    \begin{cases}
      tx_i + a_i \\
      ty_i + b_i,
    \end{cases}
    i = 1, 2.
  \end{equation*}

  We write

  The condition of the rays having a common vertex is equivalent to \( a_1 = a_2 \) and \( b_1 = b_2 \). If not specified otherwise, we assume that \( a_1 = a_2 = b_1 = b_2 = 0 \).

  The \term{measure in radians} of a directed angle, often called the angle itself, is defined as the number (see \fullref{def:geometric_trigonometric_functions})
  \begin{equation}\label{eq:def:angle/measure}
    \alpha \coloneqq \rem(\arctantwo(y_2, x_2) - \arctantwo(y_1, x_1), 2\pi).
  \end{equation}

  We can classify angles based on their measure as
  \begin{thmenum}
    \thmitem{def:angle/zero} \term{zero} if \( \alpha = 0 \),
    \thmitem{def:angle/acute} \term{acute} if \( \alpha \in (0, \tfrac \pi 2) \),
    \thmitem{def:angle/right} \term{right} if \( \alpha = \tfrac \pi 2 \),
    \thmitem{def:angle/obtuse} \term{obtuse} if \( \alpha \in (\tfrac \pi 2, \pi) \),
    \thmitem{def:angle/straight} \term{straight} if \( \alpha = \pi \), in which case the angle is actually a line,
    \thmitem{def:angle/reflex} \term{reflex} if \( \alpha > \pi \).
  \end{thmenum}

  We often do not care about the order of the two rays and speak of an \term{undirected angle}. In this case, the measure of the undirected angle is the smaller of the measures of the two oriented angles. Thus, we cannot speak of straight and reflex undirected angles.
\end{definition}

\begin{definition}\label{def:triangle}
  \begin{figure}[!ht]
    \centering
    \includegraphics[align=c]{output/def__triangle.pdf}
    \caption{An \hyperref[def:triangle/acute]{acute triangle}.}\label{fig:def:triangle}
  \end{figure}

  A \term{triangle} is a triple \( (A, B, C) \) of \hyperref[rem:point]{points}, no two of which are \hyperref[rem:collinear_complanar]{collinear} (see \fullref{def:simplex/triangle} for a more general definition). The three points are called the \term{vertices} of the triangle.

  Define the associated \hyperref[def:line_segment]{line segments}, called the \term{sides} of the triangle, and its (undirected) \hyperref[def:angle]{angles} as
  \begin{balign*}
    a \coloneqq [B, C], &  & \alpha \coloneqq \sphericalangle(b, c), \\
    b \coloneqq [A, C], &  & \beta \coloneqq \sphericalangle(a, c),  \\
    c \coloneqq [A, B], &  & \gamma \coloneqq \sphericalangle(a, b).
  \end{balign*}

  Note that we defined the angles using segments rather than rays, but this is immaterial because each to each segment \( [p, q] \) there corresponds exactly one closed ray \( t \mapsto p + t q \).

  We can classify triangles based on their sides as
  \begin{thmenum}
    \thmitem{def:triangle/isosceles} \term{isosceles} if at least two of its sides have equal length
    \thmitem{def:triangle/equilateral} \term{equilateral} if all of its sides have equal length
  \end{thmenum}
  or based on their angles as
  \begin{thmenum}
    \thmitem{def:triangle/acute} \term{acute} if all of its angles are \hyperref[def:angle/acute]{acute}.
    \thmitem{def:triangle/right} \term{right} if at least one of the angles is \hyperref[def:angle/straight]{straight}.
    \thmitem{def:triangle/obtuse} \term{obtuse} if at least one of its angles is \hyperref[def:angle/obtuse]{obtuse}.
  \end{thmenum}
\end{definition}

\begin{definition}\label{def:quadratic_plane_curve}
  The \term{quadratic plane curves} are algebraic \hyperref[rem:geometric_shape/algebraic]{curves} given by a bivariate polynomial of degree \( 2 \). The \term{general equation} of a quadratic plane curve is
  \begin{equation}\label{def:quadratic_plane_curve/general_equation}
    c(x, y) \coloneqq A x^2 + B xy + C y^2 + Dx + Ey + F = 0.
  \end{equation}

  Multiple equation can correspond to the same curve. Not all general equations, however, define algebraic curves. We will not concern ourselves with the details. See \fullref{thm:classification_of_quadratic_curves} for a proof that the unit circle is an algebraic curve. It turns out that the algebraic curves given \fullref{def:quadratic_plane_curve/general_equation} are precisely the ones listed here, collectively known as \term{conic sections}. We give only canonical forms of the equations; any linear transformation of the corresponding loci is described by another general equation.

  \begin{figure}[!ht]
    \hfill
    \hfill
    \includegraphics[align=c]{output/def__conic_section__ellipse.pdf}
    \hfill
    \includegraphics[align=c]{output/def__conic_section__hyperbola.pdf}
    \hfill
    \includegraphics[align=c]{output/def__conic_section__parabola.pdf}
    \hfill
    \caption{An \hyperref[def:quadratic_plane_curve/ellipse]{ellipse}, \hyperref[def:quadratic_plane_curve/hyperbola]{hyperbola} and \hyperref[def:quadratic_plane_curve/parabola]{parabola} defined via their parametric equations. The starting point is highlighted and the direction of the parametric curves is shown.}\label{def:quadratic_plane_curve/figure}
  \end{figure}

  \begin{thmenum}
    \thmitem{def:quadratic_plane_curve/ellipse} An \term{ellipse} is a quadratic curve whose canonical equation has the form
    \begin{equation}\label{def:quadratic_plane_curve/ellipse/canonical_equation}
      c(x, y) \coloneqq \frac {x^2} {a^2} + \frac {y^2} {b^2} - 1 = 0,
    \end{equation}
    where \( a, b > 0 \).

    If \( a = b \), we say that the ellipse is a \term{circle} and we call \( a \) the circle's \term{radius}. The \term{unit circle} is defined by \( a = b = 1 \). Circles generalize to \hyperref[def:metric_space/sphere]{spheres} in metric spaces.

    \Fullref{def:pi} and \fullref{def:geometric_trigonometric_functions} logically belong here, but are extracted separately for brevity.

    We are often interested in defining ellipses via \term{scalar parametric equations} using \hyperref[def:trigonometric_functions]{trigonometric functions} as follows:
    \begin{equation}\label{def:quadratic_plane_curve/ellipse/parametric_equations}
      \begin{cases}
        x = a \cos(t) \\
        y = b \sin(t),
      \end{cases}
    \end{equation}
    where \( t \in [0, 2\pi) \).

    We will now demonstrate that \fullref{def:quadratic_plane_curve/ellipse/canonical_equation} and \fullref{def:quadratic_plane_curve/ellipse/parametric_equations} describe the same curve. First, suppose that the pair \( (x_0, y_0) \) satisfies \fullref{def:quadratic_plane_curve/ellipse/canonical_equation}. It follows from \fullref{thm:arctantwo} that \( t_0 \coloneqq \arctantwo\left(\tfrac {y_0} b, \tfrac {x_0} a \right) \) is a, solution to the \hyperref[def:quadratic_plane_curve/ellipse/parametric_equations]{parametric equations}. Conversely, if \( x_0 = a \cos(t_0) \) and \( y_0 = b \sin(t_0) \) for some \( t_0 \in [0, 2\pi) \), by \fullref{thm:trigonometric_identities/pythagorean_identity} it follows that the pair \( (x_0, y_0) \) is a root of \fullref{def:quadratic_plane_curve/ellipse/canonical_equation} and, by \fullref{thm:arctantwo}, \( t_0 \) can be restored given \( \cos(t_0) \) and \( \sin(t_0) \).

    Therefore, every point of the parametric equation \fullref{def:quadratic_plane_curve/ellipse/parametric_equations} corresponds uniquely to a, solution of the canonical equation \fullref{def:quadratic_plane_curve/ellipse/canonical_equation} and vice versa, which makes the two approaches to defining ellipses equivalent.

    \thmitem{def:quadratic_plane_curve/hyperbola} A \term{hyperbola} is a quadratic curve whose canonical equation has the form
    \begin{equation}\label{def:quadratic_plane_curve/hyperbola/canonical_equation}
      c(x, y) \coloneqq \frac {x^2} {a^2} - \frac {y^2} {b^2} - 1 = 0,
    \end{equation}
    where \( a, b > 0 \).

    Similarly to ellipses, we are can define hyperbolas via \term{scalar parametric equations} using \hyperref[def:hyperbolic_trigonometric_functions]{hyperbolic trigonometric functions} as follows:
    \begin{equation}\label{def:quadratic_plane_curve/hyperbola/parametric_equations}
      \begin{cases}
        x = a \cosh(t) \\
        y = b \sinh(t),
      \end{cases}
    \end{equation}
    where \( t \in \BbbR \). This only defines the \term{right part} of the hyperbola. The left part is defined by replacing \( a \) with \( -a \).

    \thmitem{def:quadratic_plane_curve/parabola} A \term{parabola} is a quadratic curve whose canonical equation has the form
    \begin{equation}\label{def:quadratic_plane_curve/parabola/canonical_equation}
      c(x, y) \coloneqq y^2 - 2px = 0,
    \end{equation}
    where \( p \neq 0 \).

    Unlike ellipses and hyperbolas, we do not define parametric equations. Instead, we define \( y \) as a function of \( x \) separately for the lower half-plane and upper half-plane:
    \begin{equation}\label{def:quadratic_plane_curve/parabola/cartesian_equation}
      y(x) = \pm \sqrt{2px}.
    \end{equation}
  \end{thmenum}

  Ellipses, hyperbolas and parabolas are collectively called \term{conic sections}.
\end{definition}

\begin{proposition}\label{thm:quadratic_plane_curve_canonization}
  \todo{Quadratic curve canonization}
\end{proposition}

\begin{definition}\label{def:pi}
  \begin{figure}[!ht]
    \centering
    \includegraphics{output/def__pi__upper_half_circle.pdf}
    \caption{\( \gph(y^+) \) as a parametric curve in \fullref{def:pi}.}\label{fig:def:pi/upper_half_circle}
  \end{figure}

  The definition of a circle of unit radius as the zero-locus of the polynomial \( x^2 + y^2 - 1 \) allows us to, solve a chicken-and-egg problem regarding the definitions of the number \( \pi \). It is conventional to define it as the ratio of a circle's circumference to its diameter. For a unit circle, this diameter is \( 2 \). It will be simpler for us, however, to define \( \pi \) as the radius of a half-circle's circumference since we can represent \( y \) as a function of \( x \) in the upper \hyperref[def:half_space]{half-plane} (see \ref{fig:def:pi/upper_half_circle}). Define the parametric curve
  \begin{balign*}
     & y^+: [-1, 1] \to [0, 1]          \\
     & y^+(x) \coloneqq \sqrt{1 - x^2}.
  \end{balign*}

  We use \fullref{thm:length_of_function_graph} to find the length of the graph \( \gph(y^+(x)) \). The derivative of \( y^+(x) \) is
  \begin{equation*}
    D_x[y^+(x)] = \frac{-2x}{2 \sqrt{1 - x^2}} = - \frac x {\sqrt{1 - x^2}} dx.
  \end{equation*}

  The length of the curve \( \gph(y^+) \) is thus
  \begin{equation*}
    \len(\gph(y^+)) = \int_{-1}^1 \sqrt{1 + \frac{x^2}{1 - x^2}} dx = \int_{-1}^1 \frac 1 {\sqrt{1 - x^2}} dx.
  \end{equation*}

  This justifies the definition
  \begin{equation}\label{def:pi/weierstrass_integral}
    \pi \coloneqq \int_{-1}^1 \frac 1 {\sqrt{1 - x^2}} dx.
  \end{equation}

  See \fullref{thm:trigonometric_function_basic_roots} for a proof of how this relates to the trigonometric functions and \fullref{thm:def:exponential_function/eulers_identity} as a consequence.
\end{definition}

\begin{definition}\label{def:geometric_trigonometric_functions}
  After defining the \hyperref[def:trigonometric_functions]{trigonometric functions} \( \cos(z) \) and \( \sin(z) \) analytically via power series, we will define their geometric counterparts \( \cos_G(z) \) and \( \sin_G(z) \) and show the connection between them. The actual geometric definition relies on formalisms that are far beyond our interest (see the notes in \fullref{def:euclidean_plane}).

  Fix a point \( (x_0, y_0) \) on the unit circle (that is, \( x_0^2 + y_0^2 = 1 \)) and define the points
  \begin{equation}\label{def:geometric_trigonometric_functions/vertices}
    \begin{array}{l}
      A \coloneqq (x_0, y_0), \\
      B \coloneqq (0, 0),     \\
      C \coloneqq (x_0, 0).
    \end{array}
  \end{equation}

  Consider the \hyperref[def:triangle]{triangle} formed by these vertices. \Cref{fig:def:geometric_trigonometric_functions/triangle} illustrates the situation.
  \begin{figure}[!ht]
    \hfill
    \hfill
    \includegraphics{output/def__geometric_trigonometric_functions__triangle.pdf}
    \hfill
    \includegraphics{output/def__geometric_trigonometric_functions__circle.pdf}
    \hfill
    \caption{An \enquote{abstract} right triangle in the \hyperref[def:euclidean_plane]{\hyperref[def:euclidean_space]{Euclidean plane}} with legends for geometric sines and cosines and the same triangle in \( \BbbR^2 \) connecting the origin to a point \( (x_0, y_0) \) on the unit circle.}\label{fig:def:geometric_trigonometric_functions/triangle}
  \end{figure}

  The original \enquote{geometric definition} of \( \sin_G \) and \( \cos_G \) regards them as functions of an angle rather than numeric functions. \( \sin_G \) and \( \cos_G \) are only defined for two of the angles in a right triangle. The geometric definition is
  \begin{balign*}
    \sin_G(\alpha) \coloneqq \frac{\len(b)} {\len(c)}, &  & \cos_G(\alpha) \coloneqq \frac{\len(a)} {\len(c)},
    \\
    \sin_G(\beta) \coloneqq \frac{\len(b)} {\len(c)},  &  & \cos_G(\beta) \coloneqq \frac{\len(a)} {\len(c)}.
  \end{balign*}

  In our case, \( \len(a) = y_0 \), \( \len(b) = x_0 \) and \( \len(c) = 1 \). Furthermore, \( \sin_G(\beta) \) nor \( \cos_G(
  \beta) \) are immaterial to our subsequent arguments and we only introduced them for the sake of having a full definition.

  Therefore, we conclude that
  \begin{balign*}
    \sin_G(\alpha) = x_0,
     &  &
    \cos_G(\alpha) = y_0.
  \end{balign*}

  To see that \( \sin_G \) and \( \cos_G \) are somewhat analogous to \( \sin \) and \( \cos \), notice that by \fullref{thm:arctantwo}, there exists a unique \( t_0 \coloneqq \arctantwo(y_0, x_0) \) such that
  \begin{balign*}
    \sin(t_0) = x_0,
     &  &
    \cos(t_0) = y_0.
  \end{balign*}

  Therefore, our \hyperref[def:trigonometric_functions]{analytic definition} of the trigonometric functions as numeric functions correspond to the classical geometric definition in the special case where we consider the angle near the origin in the triangle formed by the vertices \fullref{def:geometric_trigonometric_functions/vertices}. This motivates \enquote{measuring} angles using the obtained correspondence. This unit of measurement is called a \term{radian}. We say that the angle \( \alpha \) is \( t_0 \) \term{radians}. Outside of mathematics, it is more conventional to use \term{degrees}, which are obtained from radians by scaling with \( \tfrac {180} {\pi} \). That is, \( \alpha \) is \( \tfrac {180} {\pi} t_0 \) degrees.
\end{definition}
