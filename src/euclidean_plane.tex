\subsection{Euclidean plane}\label{subsec:euclidean_plane}

\begin{remark}\label{rem:geometry_mixed_conventions}
  We use a mixture of conventions in this subsection. For example, we use the letters \( x \), \( y \) and \( z \) to denote arbitrary points, however when speaking about the Euclidean plane, we use \( x \) and \( y \) to denote coordinates of points rather than points. Another example is that initial capital letters like \( A \), \( B \) and \( C \) are used for both scalars in \hyperref[def:plane_line_equations/general]{general line equations} and points in \hyperref[def:triangle]{triangles}. This is not very consistent, but both are long-established conventions.
\end{remark}

\begin{definition}\label{def:euclidean_plane}
  We call the two-dimensional \hyperref[def:euclidean_space]{Euclidean space} \( \BbbR^2 \) the \term{Euclidean plane}.
\end{definition}

\begin{remark}\label{rem:xyz}
  By convention, depending on the context, in \hyperref[def:euclidean_plane]{Euclidean planes} the letters \( x \) and \( y \) have several meanings:
  \begin{itemize}
    \item The vectors of the \hyperref[def:sequence_space]{standard basis}.
    \item The corresponding \hyperref[def:euclidean_plane]{coordinate axes}.
    \item The \hyperref[def:affine_coordinate_system]{(affine) coordinates} of some arbitrary point.
  \end{itemize}

  We call the axis \( x \) the \term{abscissa} and \( y \) --- the \term{ordinate}.

  In three-dimensional \hyperref[def:euclidean_space]{Euclidean spaces}, we use the letter \( z \) to denote the third coordinate axis and call it the \term{applicata}.
\end{remark}

\begin{definition}\label{def:plane_line_equations}
  \hyperref[def:affine_line]{Lines} in \( \BbbR^2 \) are so ubiquitous that they are often represented via a variety of \hyperref[ex:equations]{equations}.

  We will start with \fullref{def:affine_line/parametric} --- the \hyperref[def:affine_operator]{affine} \hyperref[def:parametric_curve]{parametric curve}
  \begin{equation}\label{eq:def:plane_line_equations/parametric}
    l(t) = o + td
  \end{equation}

  \begin{figure}[!ht]
    \centering
    \includegraphics[align=c]{output/thm__plane_line_equations__cartessian.pdf}
    \caption{A \hyperref[def:affine_line]{line} in \( \BbbR^2 \) defined using its \hyperref[def:plane_line_equations/cartesian]{Cartesian equation}.}\label{fig:def:plane_line_equations/cartesian}
  \end{figure}

  \begin{thmenum}
    \thmitem{def:plane_line_equations/vector_parametric}\mcite[def. 2.1]{ВеселовТроицкий2002Лекции} We call \eqref{eq:def:plane_line_equations/parametric} a \term{vector parametric equation} of \( L \).

    \medspace

    \thmitem{def:plane_line_equations/scalar_parametric}\mcite[def. 2.1]{ВеселовТроицкий2002Лекции} The parametric equation \eqref{eq:def:plane_line_equations/parametric} can be rewritten as
    \begin{equation}\label{eq:def:plane_line_equations/scalar_parametric}
      \begin{cases}
         &l_x(t) = o_x + t d_x, \\
         &l_y(t) = o_y + t d_y.
      \end{cases}
    \end{equation}

    We say that these are \term{scalar parametric equations} of the line. They are non-unique by the same reason as the vector parametric equation.

    \thmitem{def:plane_line_equations/general}\mcite[exmpl. 2.4]{ВеселовТроицкий2002Лекции} The image of the scalar equations \eqref{eq:def:plane_line_equations/scalar_parametric} consists of all pairs \( (x, y) \) such that
    \begin{equation}\label{eq:def:plane_line_equations/general}
      \underbrace{ Ax + By + C }_{ p(x, y) } = 0
    \end{equation}
    for some scalars \( A \), \( B \) and \( C \), where \( A \) or \( B \) (or both) are nonzero.
    We call \eqref{eq:def:plane_line_equations/general} a \term{general equation} of the line.

    More concretely,
    \begin{equation*}
      A(o_x + td_x) + B(o_y + td_y) + C = 0
    \end{equation*}
    for all \( t \) if \( A = d_y \), \( B = -d_x \) and \( C = o_y d_x - o_x d_y \).

    Conversely, given the general equation \eqref{eq:def:plane_line_equations/general}, assuming \( A \neq 0 \), we can define the parametric equations
    \begin{equation*}
      \begin{cases}
        &l_x(t) \coloneqq -\tfrac C A - t \tfrac B A  \\
        &l_y(t) \coloneqq t.
      \end{cases}
    \end{equation*}

    The case when \( A = 0 \) and \( B \neq 0 \) is handled analogously.

    Note that multiple general equations can have the same locus --- actually all scalar multiples of \( p(x, y) \). If \( A^2 + B^2 = 1 \) in \eqref{eq:def:plane_line_equations/general}, we call it a \term{normal equation}. There are only two normal equations.

    \thmitem{def:plane_line_equations/cartesian} It is common, especially in analysis, to use the \term{Cartesian equation}
    \begin{equation}\label{eq:def:plane_line_equations/cartesian}
      y = kx + m
    \end{equation}
    for some scalars \( k \) and \( m \). We call \( k \) the \term{slope} of the line.

    It is a special case of the general equation \eqref{eq:def:plane_line_equations/general} with \( A = -k \), \( B = -1 \) and \( C = m \).

    Unlike the general equation, the Cartesian equation of a line is unique, but it cannot express vertical lines. If \( B \neq 0 \) in \eqref{eq:def:plane_line_equations/general}, we can define \( k = -\ifrac A B \) and \( m = -\ifrac C B \) to form a Cartesian equation.

    \begin{figure}[!ht]
      \centering
      \includegraphics[align=c]{output/thm__plane_line_equations__cartessian.pdf}
      \caption{A \hyperref[def:affine_line]{line} in \( \BbbR^2 \) defined using its \hyperref[def:plane_line_equations/cartesian]{Cartesian equation}.}\label{fig:def:plane_line_equations/cartesian_drawing}
    \end{figure}

    \thmitem{def:plane_line_equations/intercept} Another equation that is occasionally used is the \term{intercept equation}
    \begin{equation}\label{eq:def:plane_line_equations/intercept}
      \frac x a + \frac y b = 1
    \end{equation}
    for some nonzero real numbers \( a \) and \( b \).

    It is again a special case of the general equation \eqref{eq:def:plane_line_equations/general} with \( A = \ifrac 1 a \), \( B = \ifrac 1 b \) and \( C = -1 \). It is also unique, but it cannot express neither vertical nor horizontal lines, nor lines passing through the origin.

    Conversely, if \( A \), \( B \) and \( C \) are all nonzero in the general equation \eqref{eq:def:plane_line_equations/general}, we can define an intercept equation as \( a = -\ifrac C A \) and \( b = -\ifrac C B \).
  \end{thmenum}
\end{definition}

\begin{definition}\label{def:angle}
  A \term{directed angle} is a pair of two \hyperref[def:geometric_ray]{rays} with a common vertex. Given two rays \( r(t) = (x_0, y_0) + t (x_r, y_r) \) and \( s(t) = (x_0, y_0) + t (x_s, y_s) \) with a common vertex, we denote their corresponding directed angle via \( \sphericalangle(r, s) \).

  \begin{figure}[!ht]
    \centering
    \includegraphics[align=c]{output/def__angle.pdf}
    \caption{An \hyperref[def:angle/acute]{acute angle} with its measurement segments dashed.}\label{def:angle/figure}
  \end{figure}

  The \term{measure in radians} of a directed angle, often called the angle itself, is defined as the number
  \begin{equation}\label{eq:def:angle/measure}
    \alpha \coloneqq \rem\parens[\Big]{ \arctantwo(y_s, x_s) - \arctantwo(y_r, x_r), 2\pi }.
  \end{equation}

  This is justified and discussed in \fullref{def:geometric_trigonometric_functions}.

  We can classify angles based on their measure as
  \begin{thmenum}
    \thmitem{def:angle/zero} \term{zero} if \( \alpha = 0 \),
    \thmitem{def:angle/acute} \term{acute} if \( 0 < \alpha < \ifrac \pi 2 \),
    \thmitem{def:angle/right} \term{right} if \( \alpha = \tfrac \pi 2 \),
    \thmitem{def:angle/obtuse} \term{obtuse} if \( \ifrac \pi 2 < \alpha < \pi \),
    \thmitem{def:angle/straight} \term{straight} if \( \alpha = \pi \), in which case the angle is actually a line,
    \thmitem{def:angle/reflex} \term{reflex} if \( \alpha > \pi \).
  \end{thmenum}

  We often do not care about the order of the two rays and speak of an \term{undirected angle} (which we may regard as a set of rays rather than an ordered pair). In this case, the measure of the undirected angle is the smaller of the measures of the two oriented angles. Thus, we cannot speak of straight and reflex undirected angles.
\end{definition}

\begin{definition}\label{def:triangle}
  \begin{figure}[!ht]
    \centering
    \includegraphics[align=c]{output/def__triangle.pdf}
    \caption{An \hyperref[def:triangle/acute]{acute triangle}.}\label{fig:def:triangle}
  \end{figure}

  A \term{triangle} is a triple \( (A, B, C) \) of \hyperref[rem:point]{points}, no two of which are \hyperref[rem:collinear_complanar]{collinear} (see \fullref{def:simplex/triangle} for a more general definition). The three points are called the \term{vertices} of the triangle.

  Define the associated \hyperref[def:line_segment]{line segments}, called the \term{sides} of the triangle, and its (undirected) \hyperref[def:angle]{angles} as
  \begin{align*}
    a \coloneqq [B, C], && \alpha \coloneqq \sphericalangle(b, c), \\
    b \coloneqq [A, C], && \beta \coloneqq \sphericalangle(a, c),  \\
    c \coloneqq [A, B], && \gamma \coloneqq \sphericalangle(a, b).
  \end{align*}

  Note that we have defined the angles using segments rather than rays, but this is immaterial because each to each segment \( [P, Q] \) there corresponds exactly one ray \( t \mapsto P + t (Q - P) \).

  We can classify triangles based on their sides as
  \begin{thmenum}
    \thmitem{def:triangle/isosceles} \term{isosceles} if at least two of its sides have equal length
    \thmitem{def:triangle/equilateral} \term{equilateral} if all of its sides have equal length
  \end{thmenum}
  or based on their angles as
  \begin{thmenum}
    \thmitem{def:triangle/acute} \term{acute} if all of its angles are \hyperref[def:angle/acute]{acute}.
    \thmitem{def:triangle/right} \term{right} if at least one of the angles is \hyperref[def:angle/straight]{straight}.
    \thmitem{def:triangle/obtuse} \term{obtuse} if at least one of its angles is \hyperref[def:angle/obtuse]{obtuse}.
  \end{thmenum}
\end{definition}
