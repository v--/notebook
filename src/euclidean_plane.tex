\subsection{Euclidean plane}\label{subsec:euclidean_plane}

\begin{definition}\label{def:euclidean_plane}
  We call the two-dimensional \hyperref[def:euclidean_space]{Euclidean space} \( \BbbR^2 \) the \term{Euclidean plane}.
\end{definition}

\begin{remark}\label{rem:xyz}
  By convention, depending on the context, in \hyperref[def:euclidean_plane]{Euclidean planes} the letters \( x \) and \( y \) have several meanings:
  \begin{itemize}
    \item The vectors of the \hyperref[def:sequence_space]{standard basis}.
    \item The corresponding \hyperref[def:euclidean_plane]{coordinate axes}.
    \item The \hyperref[def:affine_coordinate_system]{(affine) coordinates} of some arbitrary point.
  \end{itemize}

  We call the axis \( x \) the \term{abscissa} and \( y \) --- the \term{ordinate}.

  In three-dimensional \hyperref[def:euclidean_space]{Euclidean spaces}, we use the letter \( z \) to denote the third coordinate axis and call it the \term{applicata}.

  To avoid confusion, we avoid using \( x \), \( y \) and \( z \) in Euclidean planes to denote points and vectors.
\end{remark}

\begin{definition}\label{def:plane_line_equations}
  \hyperref[def:affine_line]{Lines} in \( \BbbR^2 \) are so ubiquitous that they are often represented via a variety of \hyperref[ex:equations]{equations}.

  We will start with \fullref{def:affine_line/parametric} --- the \hyperref[def:affine_operator]{affine} \hyperref[def:parametric_curve]{parametric curve}
  \begin{equation}\label{eq:def:plane_line_equations/parametric}
    l(t) = o + td
  \end{equation}

  \begin{figure}[!ht]
    \centering
    \includegraphics[align=c]{output/thm__plane_line_equations__cartessian.pdf}
    \caption{A \hyperref[def:affine_line]{line} in \( \BbbR^2 \) defined using its \hyperref[def:plane_line_equations/cartesian]{Cartesian equation}.}\label{fig:def:plane_line_equations/cartesian}
  \end{figure}

  \begin{thmenum}
    \thmitem{def:plane_line_equations/vector_parametric}\mcite[def. 2.1]{ВеселовТроицкий2002Лекции} We call \eqref{eq:def:plane_line_equations/parametric} a \term{vector parametric equation} of \( L \).

    \medspace

    \thmitem{def:plane_line_equations/scalar_parametric}\mcite[def. 2.1]{ВеселовТроицкий2002Лекции} The parametric equation \eqref{eq:def:plane_line_equations/parametric} can be rewritten as
    \begin{equation}\label{eq:def:plane_line_equations/scalar_parametric}
      \begin{cases}
         &l_x(t) = o_x + t d_x, \\
         &l_y(t) = o_y + t d_y.
      \end{cases}
    \end{equation}

    We say that these are \term{scalar parametric equations} of the line. They are non-unique by the same reason as the vector parametric equation.

    \thmitem{def:plane_line_equations/general}\mcite[exmpl. 2.4]{ВеселовТроицкий2002Лекции} The image of the scalar equations \eqref{eq:def:plane_line_equations/scalar_parametric} consists of all pairs \( (x, y) \) such that
    \begin{equation}\label{eq:def:plane_line_equations/general}
      \underbrace{ Ax + By + C }_{ p(x, y) } = 0
    \end{equation}
    for some scalars \( A \), \( B \) and \( C \), where \( A \) or \( B \) (or both) are nonzero.
    We call \eqref{eq:def:plane_line_equations/general} a \term{general equation} of the line.

    More concretely,
    \begin{equation*}
      A(o_x + td_x) + B(o_y + td_y) + C = 0
    \end{equation*}
    for all \( t \) if \( A = d_y \), \( B = -d_x \) and \( C = o_y d_x - o_x d_y \).

    Conversely, given the general equation \eqref{eq:def:plane_line_equations/general}, assuming \( A \neq 0 \), we can define the parametric equations
    \begin{equation*}
      \begin{cases}
        &l_x(t) \coloneqq -\tfrac C A - t \tfrac B A  \\
        &l_y(t) \coloneqq t.
      \end{cases}
    \end{equation*}

    The case when \( A = 0 \) and \( B \neq 0 \) is handled analogously.

    Note that multiple general equations can have the same locus --- actually all scalar multiples of \( p(x, y) \). If \( A^2 + B^2 = 1 \) in \eqref{eq:def:plane_line_equations/general}, we call it a \term{normal equation}. There are only two normal equations.

    \thmitem{def:plane_line_equations/cartesian} It is common, especially in analysis, to use the \term{Cartesian equation}
    \begin{equation}\label{eq:def:plane_line_equations/cartesian}
      y = kx + m
    \end{equation}
    for some scalars \( k \) and \( m \). We call \( k \) the \term{slope} of the line.

    It is a special case of the general equation \eqref{eq:def:plane_line_equations/general} with \( A = -k \), \( B = -1 \) and \( C = m \).

    Unlike the general equation, the Cartesian equation of a line is unique, but it cannot express vertical lines. If \( B \neq 0 \) in \eqref{eq:def:plane_line_equations/general}, we can define \( k = -\ifrac A B \) and \( m = -\ifrac C B \) to form a Cartesian equation.

    \begin{figure}[!ht]
      \centering
      \includegraphics[align=c]{output/thm__plane_line_equations__cartessian.pdf}
      \caption{A \hyperref[def:affine_line]{line} in \( \BbbR^2 \) defined using its \hyperref[def:plane_line_equations/cartesian]{Cartesian equation}.}\label{fig:def:plane_line_equations/cartesian_drawing}
    \end{figure}

    \thmitem{def:plane_line_equations/intercept} Another equation that is occasionally used is the \term{intercept equation}
    \begin{equation}\label{eq:def:plane_line_equations/intercept}
      \frac x a + \frac y b = 1
    \end{equation}
    for some nonzero real numbers \( a \) and \( b \).

    It is again a special case of the general equation \eqref{eq:def:plane_line_equations/general} with \( A = \ifrac 1 a \), \( B = \ifrac 1 b \) and \( C = -1 \). It is also unique, but it cannot express neither vertical nor horizontal lines, nor lines passing through the origin.

    Conversely, if \( A \), \( B \) and \( C \) are all nonzero in the general equation \eqref{eq:def:plane_line_equations/general}, we can define an intercept equation as \( a = -\ifrac C A \) and \( b = -\ifrac C B \).
  \end{thmenum}
\end{definition}
