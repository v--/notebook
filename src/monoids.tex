\subsection{Monoids}\label{subsec:monoids}

We list here several basic algebraic structures, we will use mostly as building blocks for more complicated structures.

As discussed in \fullref{rem:first_order_model_notation}, listing all operations explicitly is cumbersome, and we will usually avoid it.

\begin{definition}\label{def:pointed_set}\mcite[26]{MacLane1994}
  The simplest algebraic structure is a \term{pointed set}. It is simply a nonempty set \( X \) equipped with a distinguished element \( e \). It is an algebraic structure because \( e \) can be regarded as the sole value of a nullary function \( *: X^0 \to X \).

  We will call \( e \) the \term{origin} of \( X \) based on the terminology for \hyperref[def:euclidean_plane_coordinate_system/origin]{affine coordinate systems}.

  Pointed sets have the following metamathematical properties:
  \begin{thmenum}
    \thmitem{def:pointed_set/theory} Pointed sets can also be viewed as \hyperref[def:first_order_semantics/satisfiability]{models} of an empty \hyperref[def:first_order_theory]{theory} for a \hyperref[def:first_order_language]{first-order logic language} with a constant symbol, i.e. a nullary \hyperref[def:first_order_language/func]{functional symbol}.

    \thmitem{def:pointed_set/homomorphism} A \hyperref[def:first_order_homomorphism]{homomorphism} between the pointed sets \( (X, e_{X}) \) and \( (Y, e_{Y}) \) is, explicitly, a function \( \varphi: X \to Y \) that satisfies
    \begin{equation}\label{eq:def:pointed_set/homomorphism}
      \varphi(e_{X}) = e_{Y}.
    \end{equation}

    \thmitem{def:pointed_set/submodel} The set \( S \subseteq X \) is a \hyperref[thm:substructure_is_model]{submodel} of \( X \) if \( e \in S \).

    In particular, as a consequence of \fullref{thm:positive_formulas_preserved_under_homomorphism}, the image of a pointed set homomorphism is a submodel of its range.

    \thmitem{def:pointed_set/trivial} The \hyperref[thm:substructures_form_complete_lattice/bottom]{trivial} pointed set is, up to an isomorphism, the set \( \set{ e } \).

    It is a \hyperref[def:universal_objects/initial]{zero object} in \( \cat{Set_*} \) as discussed in \fullref{ex:def:universal_objects/grp}.

    \thmitem{def:pointed_set/category} We denote the \hyperref[def:category_of_small_first_order_models]{category of \( \mscrU \)-small models} for this theory by \( \ucat{Set}_* \).
  \end{thmenum}
\end{definition}

\begin{definition}\label{def:set_with_involution}\mimprovised
  A \term{set with an involution} is a \hyperref[def:set]{set} \( X \) with a unary operation \( (\anon)^{-1} \) such that
  \begin{equation*}
    (x^{-1})^{-1} = x
  \end{equation*}
  for every \( x \in X \).

  Such an operation is called, surprisingly, an \term{involution}.

  Sets with involutions have the following metamathematical properties:
  \begin{thmenum}
    \thmitem{def:set_with_involution/theory} We define the theory of sets with involution as a theory over the language consisting of a single unary functional symbol \( \anon^{-1} \) with the sole axiom
    \begin{equation}\label{eq:def:set_with_involution/theory/axiom}
      (\xi^{-1})^{-1} \doteq \xi.
    \end{equation}

    \thmitem{def:set_with_involution/homomorphism} A \hyperref[def:first_order_homomorphism]{homomorphism} between sets with involutions \( X \) and \( Y \) is a function \( \varphi: X \to Y \) satisfying
    \begin{equation}\label{eq:def:set_with_involution/homomorphism}
      \varphi(x^{-1})
      =
      \varphi(x)^{-1}.
    \end{equation}

    \thmitem{def:set_with_involution/submodel} Any subset of a set with involution is again a set with involution.

    In particular, as a consequence of \fullref{thm:positive_formulas_preserved_under_homomorphism}, the \hyperref[def:multi_valued_function/image]{image} of a homomorphism \( \varphi: X \to Y \) is a submodel of \( Y \).

    \thmitem{def:set_with_involution/trivial} The \hyperref[thm:substructures_form_complete_lattice/bottom]{trivial} set with involution is the empty set.

    \thmitem{def:set_with_involution/category} We denote the \hyperref[def:category_of_small_first_order_models]{category of \( \mscrU \)-small models} for this theory by \( \ucat{Inv} \).
  \end{thmenum}
\end{definition}

\begin{definition}\label{def:magma}
  A \term{magma} is a set \( M \) equipped with a \hyperref[def:multi_valued_function/arguments]{binary function} \( \cdot: M \times M \to M \), called the \term{magma operation}. Unless specified otherwise, we denote this operation by juxtaposition as \( xy \) instead of \( x \cdot y \).

  We often call the operation \term{multiplication} or, in the case of \hyperref[def:endomorphism_monoid]{endomorphism monoids} --- \term{composition}. See also the notes in \fullref{rem:additive_magma} regarding additive magmas and in \fullref{def:monoid_delooping} regarding the order of operands.

  \begin{thmenum}[series=def:magma]
    \thmitem{def:magma/theory} In analogy to the \hyperref[def:pointed_set/theory]{theory of pointed sets}, we can define the theory of magmas as an empty theory over a language with a single \hyperref[rem:first_order_formula_conventions/infix]{infix} binary functional symbol.

    \thmitem{def:magma/homomorphism} A \hyperref[def:first_order_homomorphism]{homomorphism} between the magmas \( (M, \cdot_{M}) \) and \( (N, \cdot_{N}) \) is, explicitly, a function \( \varphi: M \to N \) such that
    \begin{equation}\label{eq:def:magma/homomorphism}
      \varphi(x \cdot_{M} y) = \varphi(x) \cdot_{N} \varphi(y)
    \end{equation}
    for all \( x, y \in M \).

    \thmitem{def:magma/submodel} The set \( S \subseteq M \) is a \hyperref[def:first_order_substructure]{first-order submodel} of \( M \) if it is closed under the magma operation. That is, if \( x, y \in S \) implies \( xy \in S \).

    We call \( S \) a \term{submagma} of \( M \).

    As a consequence of \fullref{thm:positive_formulas_preserved_under_homomorphism}, the image of a magma homomorphism is a submagma of its range.

    \thmitem{def:magma/trivial} The \hyperref[thm:substructures_form_complete_lattice/bottom]{trivial magma} is the empty set with an empty operation. It is the unique \hyperref[def:universal_objects/initial]{zero object} in \( \cat{Mag} \).

    \thmitem{def:magma/exponentiation} We define an additional \term{exponentiation} operation for positive integers \( n \) \hyperref[rem:natural_number_recursion]{recursively} as
    \begin{equation}\label{eq:def:magma/exponentiation}
      x^n \coloneqq \begin{cases}
        x,               &n = 1 \\
        x^{n-1} \cdot x, &n > 1
      \end{cases}
    \end{equation}

    \thmitem{def:magma/power_set} It is customary to perform magma operations with sets. That is, if \( A \) and \( B \) are sets in the magma \( M \), it is customary to write
    \begin{equation*}
      A \cdot B \coloneqq \set{ a \cdot b \colon a \in A, b \in B }.
    \end{equation*}

    This actually turns the power set \( \pow(M) \) into a magma, which we will call the \term{power set magma} of \( M \). This is especially useful with the convention \fullref{rem:singleton_sets} since it allows us to write \( aB \) for \( a \in M \) and \( B \subseteq M \).

    See \fullref{thm:power_set_magma_preservation}.

    \thmitem{def:magma/category} We denote the \hyperref[def:category_of_small_first_order_models]{category of \( \mscrU \)-small models} for the theory of magmas by \( \ucat{Mag} \).

    \thmitem{def:magma/opposite} The \term{opposite magma} of \( (M, \cdot) \) is the magma \( (M, \star) \) with multiplication reversed:
    \begin{equation*}
      x \star y \coloneqq y \cdot x.
    \end{equation*}

    We denote the opposite magma by \( M^{\opcat} \). This is justified in \fullref{def:magma/opposite}.
  \end{thmenum}

  We list some additional restrictions that are often imposed on magmas.
  \begin{thmenum}[resume=def:magma]
    \thmitem{def:magma/associative} We can add the (\hyperref[thm:implicit_universal_quantification]{universal closure} of) following axiom to the theory:
    \begin{equation}\label{eq:def:magma/associative}
      (\xi \cdot \eta) \cdot \zeta = \xi \cdot (\eta \cdot \zeta).
    \end{equation}

    If \eqref{eq:def:magma/associative} is satisfied, we say that the operation \( \cdot \) and, by extension, the magma itself, are \term{associative}. Associative magmas are usually called \term{semigroups}. Associativity imposes no additional restrictions on the homomorphisms, hence semigroups are a \hyperref[def:subcategory]{full subcategory} of \( \cat{Mag} \).

    \thmitem{def:magma/commutative} Another common axiom is \term{commutativity}:
    \begin{equation}\label{eq:def:magma/commutative}
      \xi \cdot \eta \doteq \eta \cdot \xi.
    \end{equation}

    Commutative magmas also form a full subcategory. Obviously \( M = M^{-1} \) in a commutative magma.

    \thmitem{def:magma/idempotent} We say that the operation \( \cdot \) is \term{idempotent} if
    \begin{equation}\label{eq:def:magma/idempotent}
      \xi \cdot \xi \doteq \xi.
    \end{equation}

    \thmitem{def:magma/cancellative} We say that \( \cdot \) is \term{left-cancellative} if
    \begin{equation}\label{eq:def:magma/cancellative/left}
      \qforall \zeta (\zeta \cdot \xi \doteq \zeta \cdot \eta) \rightarrow \xi = \eta
      \quad
    \end{equation}
    and \term{right-cancellative} if
    \begin{equation}\label{eq:def:magma/cancellative/right}
      \qforall \zeta (\xi \cdot \zeta \doteq \eta \cdot \zeta) \rightarrow \xi = \eta
      \quad
    \end{equation}

    The operation is \term{cancellative} if it is both left and right cancellative. Cancellative magmas also form a full subcategory.
  \end{thmenum}
\end{definition}

\begin{example}\label{ex:def:magma}
  We list several examples of \hyperref[def:magma]{magmas} satisfying different properties.

  \begin{thmenum}
    \thmitem{ex:def:magma/algebraic} \hyperref[eq:def:magma/associative]{Associative} binary operations on a set are abundant and are part of the definition of essential algebraic structures like \hyperref[def:group]{groups}, \hyperref[def:semiring]{(semi)rings}, \hyperref[def:semimodule]{(semi)modules} and \hyperref[def:semilattice]{(semi)lattices}.

    These operations are \term{homogeneous} in the sense that their signature only contains a single set, unlike \hyperref[def:group_action]{group actions} and scalar products in \hyperref[def:semimodule]{(semi)modules}.

    \thmitem{ex:def:magma/composition} The quintessential example of a non-\hyperref[def:magma/commutative]{commutative} operation is \hyperref[def:multi_valued_function/composition]{composition} in any set of functions or, more generally, \hyperref[def:category/composition]{morphism composition} in any \hyperref[def:category]{category}.

    Composition is \hyperref[def:magma/associative]{associative}. \hyperref[def:magma/cancellative]{Cancellation} with respect to composition is discussed in \fullref{def:morphism_invertibility} and, for function composition, in \fullref{thm:function_invertibility_categorical}.

    \thmitem{ex:def:magma/midpoint} The midpoint operation
    \begin{equation*}
      (x, y) \mapsto \dfrac {x + y} 2
    \end{equation*}
    makes \( \BbbR \) a commutative and cancellative magma, which is not associative.
  \end{thmenum}
\end{example}

\begin{proposition}\label{thm:power_set_magma_preservation}
  \hyperref[def:magma/associative]{Associativity} and \hyperref[def:magma/commutative]{commutativity} from a magma \( M \) are preserved in \( \pow(M) \), unlike \hyperref[def:magma/cancellative]{cancellation}.
\end{proposition}
\begin{proof}
  Associativity and commutativity are obviously preserved.

  To show that cancellation is not, consider the group \hyperref[thm:group_of_integers_modulo]{\( \BbbZ_2 \)}. It is a cancellative magma by \fullref{thm:def:group/properties/cancellative}. Define the sets \( A \coloneqq \{ 0, 1 \} \) and \( B \coloneqq \{ 0 \} \). Then
  \begin{equation*}
    A + A = A = A + B,
  \end{equation*}
  however we cannot cancel \( A \) from the left because \( A \neq B \).
\end{proof}

\begin{proposition}\label{thm:magma_exponentiation_properties}
  Fix a magma \( M \). \hyperref[def:magma/exponentiation]{Magma exponentiation} in \( M \) has the following basic properties:

  \begin{thmenum}
    \thmitem{thm:magma_exponentiation_properties/commutativity} We have the following \hyperref[def:magma/commutative]{commutativity}-like property: for \( x \in M \) and \( n = 1, 2, \ldots \),
    \begin{equation}\label{eq:thm:magma_exponentiation_properties/commutativity}
      x^n = x x^{n-1} = x^{n-1} x.
    \end{equation}

    \thmitem{thm:magma_exponentiation_properties/distributivity} Exponentiation distributes over multiplication: for any member \( x \in M \) and any two positive integers \( n \) and \( m \),
    \begin{equation}\label{eq:thm:magma_exponentiation_properties/multiplication}
      x^{n + m} = x^n x^m.
    \end{equation}

    \thmitem{thm:magma_exponentiation_properties/repeated} For any member \( x \in M \) and any two positive integers \( n \) and \( m \),
    \begin{equation}\label{eq:thm:magma_exponentiation_properties/repeated}
      (x^n)^m = x^{nm}.
    \end{equation}
  \end{thmenum}
\end{proposition}
\begin{proof}
  \SubProofOf{thm:magma_exponentiation_properties/commutativity} We use induction on \( n \). The cases \( n = 1 \) and \( n = 2 \) are obvious. For \( n > 2 \), we have
  \begin{equation*}
    x^n
    \reloset {\eqref{eq:def:magma/exponentiation}} =
    x x^{n-1}
    \reloset {\T{ind.}} =
    x x^{n-2} x
    \reloset {\eqref{eq:def:magma/exponentiation}} =
    x^{n-1} x.
  \end{equation*}

  \SubProofOf{thm:magma_exponentiation_properties/distributivity} We use induction on \( n \). The case \( n = 1 \) follows directly from \eqref{eq:def:magma/exponentiation}. The case \( n > 1 \) follows from
  \begin{equation*}
    x^{n + m}
    \reloset {\eqref{eq:def:magma/exponentiation}} =
    x x^{n + (m - 1)}
    \reloset {\T{ind.}} =
    x x^{n-1} x^m
    \reloset {\eqref{eq:def:magma/exponentiation}} =
    x^n x^m.
  \end{equation*}

  \SubProofOf{thm:magma_exponentiation_properties/repeated} We use induction on \( n \). The case \( n = 1 \) is obvious and the rest follows from
  \begin{equation*}
    (x^n)^m
    \reloset {\eqref{eq:def:magma/exponentiation}} =
    x^n (x^n)^{m-1}
    \reloset {\T{ind.}} =
    x^n x^{n (m - 1)}
    \reloset {\eqref{eq:thm:magma_exponentiation_properties/multiplication}} =
    =
    x^{nm}.
  \end{equation*}
\end{proof}

\begin{definition}\label{def:ordered_magma}\mimprovised
  An \term{ordered magma}\footnote{Based on \cite[224]{Golan2010}} is a \hyperref[def:magma/commutative]{commutative} \hyperref[def:magma]{magma} \( M \) equipped with a \hyperref[def:partially_ordered_set]{partial order} \( \leq \) such that \( x \leq y \) implies \( xz \leq yz \) for every \( z \in M \).

  The condition for commutativity is not necessary, and the partial order can be generalized to a \hyperref[def:preordered_set]{preorder}, but neither generalization will be of any use for us.

  The category of small ordered magmas is a \hyperref[def:concrete_category]{concrete category} over both \hyperref[def:magma/category]{\( \cat{Mag} \)} and \hyperref[def:partially_ordered_set/category]{\( \cat{Pos} \)}.
\end{definition}

\begin{example}\label{ex:def:ordered_magma}
  We list several examples of \hyperref[def:ordered_magma]{ordered magmas}.

  \begin{thmenum}
    \thmitem{ex:def:ordered_magma/natural_numbers} The \hyperref[def:set_of_natural_numbers]{natural numbers} with addition form an ordered magma as a consequence of \fullref{thm:natural_numbers_are_well_ordered}; and so do \( \BbbZ \), \( \BbbQ \), \( \BbbR \) and \( \BbbC \).

    \thmitem{ex:def:ordered_magma/ordinal} More generally, every \hyperref[def:successor_and_limit_ordinal]{limit ordinal} as the set of all smaller ordinals is an ordered magma under \hyperref[def:ordinal_arithmetic/addition]{ordinal addition}.

    Unlike addition in natural numbers, however, ordinal addition is not commutative as shown in \fullref{ex:ordinal_addition}.

    \thmitem{ex:def:ordered_magma/semilattice} Every \hyperref[def:semilattice/join]{join-semilattice} \( (X, \vee) \) is an ordered magma with the lattice order. Indeed, if \( x \leq y \), then
    \begin{itemize}
      \item If \( z \leq x \), then
      \begin{equation*}
        x = x \vee z \leq y \vee z = y.
      \end{equation*}

      \item If \( x \leq z \leq y \), then
      \begin{equation*}
        z = x \vee z \leq y \vee z = y.
      \end{equation*}

      \item If \( z \geq y \), then
      \begin{equation*}
        z = x \vee z \leq y \vee z = z.
      \end{equation*}
    \end{itemize}

    Every \hyperref[def:semilattice/meet]{meet-semilattice} is also an ordered magma.
  \end{thmenum}
\end{example}

\begin{definition}\label{def:monoid}
  A \term{monoid} is an \hyperref[eq:def:magma/associative]{associative} \hyperref[def:magma]{magma} with a distinguished element \( e \) such that \( ex = x = xe \) for every element \( x \). Such an element is obviously unique, and we call it the \term{identity} or \term{neutral element} of the monoid. This makes monoids \hyperref[def:pointed_set]{pointed sets}.

  The requirement of associativity is conventional but not strictly necessary. Non-associative monoids will not be useful to us.

  \begin{thmenum}
    \thmitem{def:monoid/theory} The theory of monoids consists of \hyperref[eq:def:magma/associative]{associativity} and the axiom
    \begin{equation}\label{eq:def:monoid/theory/identity}
      \qforall \xi (e \cdot \xi \doteq \xi \wedge \xi \cdot e \doteq \xi)
    \end{equation}
    over the combined language of \hyperref[def:pointed_set/theory]{pointed sets} and \hyperref[def:magma/theory]{magmas}.

    \thmitem{def:monoid/homomorphism} A \hyperref[def:first_order_homomorphism]{homomorphism} between monoids is a function that satisfies both \eqref{eq:def:pointed_set/homomorphism} and \eqref{eq:def:magma/homomorphism}.

    \thmitem{def:monoid/submodel} The set \( S \subseteq X \) is a \hyperref[thm:substructure_is_model]{submodel} if \( X \) if \( e \in S \). This is equivalent to \( S \) being a pointed subset.

    We say that \( S \) is a \term{submonoid}.

    As a consequence of \fullref{thm:positive_formulas_preserved_under_homomorphism}, the image of a monoid homomorphism is a submonoid of its range.

    \thmitem{def:monoid/trivial} The \hyperref[thm:substructures_form_complete_lattice/bottom]{trivial} monoid is the \hyperref[def:pointed_set/trivial]{trivial pointed set} \( \set{ e } \).

    \thmitem{def:monoid/exponentiation} We extend \hyperref[def:magma/exponentiation]{magma exponentiation} to all nonnegative integers by defining
    \begin{equation*}
      x^0 \coloneqq e.
    \end{equation*}

    We denote the subcategory of \hyperref[def:magma/commutative]{commutative} monoids by \( \ucat{CMon} \). We usually write commutative monoids additively as explained in \fullref{rem:additive_magma}.

    \thmitem{def:monoid/power_set} The \hyperref[def:magma/power_set]{power set magma} \( \pow(M) \) of a monoid \( M \) with identity \( e \) is again a monoid with identity \( \set{ e } \).

    \thmitem{def:monoid/category} The \hyperref[def:category_of_small_first_order_models]{category of \( \mscrU \)-small models} \( \ucat{Mon} \) of monoids is \hyperref[def:concrete_category]{concrete} with respect to both \hyperref[def:pointed_set/category]{\( \ucat{Set}_* \)} and \hyperref[def:magma/category]{\( \ucat{Mag} \)}.

    We will discuss the free monoid functor in \fullref{thm:free_monoid_universal_property}. Then from \fullref{thm:first_order_categorical_invertibility} it will follow that monomorphisms are injective functions.

    For epimorphisms a similar statement does not hold, unfortunately. The embedding \( \iota: \BbbN \to \BbbZ \) is an epimorphism by \fullref{thm:monoid_grothendieck_completion_universal_property}, but it is clearly not surjective.

    \thmitem{def:monoid/opposite} The \hyperref[def:monoid_delooping]{delooping} of the \hyperref[def:magma/opposite]{opposite magma} for a monoid \( M \) is the \hyperref[def:opposite_category]{opposite category} of the delooping of \( M \). This justifies the notation \( M^{\opcat} \)
  \end{thmenum}
\end{definition}

\begin{example}\label{ex:def:monoid}
  We list several important examples \hyperref[def:monoid]{monoids}.

  \begin{thmenum}
    \thmitem{ex:def:monoid/natural_numbers} The \hyperref[def:set_of_natural_numbers]{nonnegative natural numbers} with addition form a quintessential example of a monoid. We prove in \fullref{thm:natural_number_addition_properties} that they are a monoid.

    \thmitem{ex:def:monoid/kleene_star} Another important example of a monoid is the \hyperref[def:formal_language/kleene_star]{Kleene star} \( \mscrA \) over some \hyperref[def:formal_language/alphabet]{alphabet} \( \mscrA \).

    The importance for monoid theory comes from the free monoid universal property described in \fullref{thm:free_monoid_universal_property}.

    \thmitem{ex:def:monoid/semilattice} Every \hyperref[def:semilattice/bounded]{bounded} \hyperref[def:semilattice/join]{join-semilattice} is a monoid as a consequence of \eqref{eq:thm:binary_lattice_operations/identity/meet}, and similarly for \hyperref[def:semilattice/meet]{meet-semilattice}.
  \end{thmenum}
\end{example}

\begin{example}\label{ex:monoid_cancellation_not_preserved_by_homomorphism}\mcite{MathSE:magma_cancellation_not_preserved}
  \hyperref[def:monoid/homomorphism]{Monoid homomorphisms} may not preserve the \hyperref[def:magma/cancellative]{cancellation property}. For example, the \hyperref[def:set_of_natural_numbers]{natural numbers} \( \BbbN \) are a cancellative monoid under addition, as shown in \fullref{thm:natural_number_addition_properties}, but the homomorphism
  \begin{equation*}
    \begin{aligned}
      &h: (\BbbN, +) \to (\hyperref[thm:f2_is_boolean_algebra]{\BbbZ_2}, \max) \\
      &h(n) \coloneqq \begin{cases}
        0, &n = 0 \\
        1, &n > 0
      \end{cases}
    \end{aligned}
  \end{equation*}
  does not preserve the cancellative property.

  Indeed, \( \max\set{ 0, 1 } = \max\set{ 1, 1 } \), but \( 0 \neq 1 \).
\end{example}

\begin{definition}\label{def:monoid_direct_product}
  The \term{direct product} of a family of \hyperref[def:monoid]{monoids} \( \seq{ M_k }_{k \in \mscrK} \) is their \hyperref[def:cartesian_product]{Cartesian product} \( \prod_{k \in \mscrK} M_k \) with the componentwise operation
  \begin{equation*}
    \seq{ x_k }_{k \in \mscrK} \cdot \seq{ y_k }_{k \in \mscrK}
    \coloneqq
    \seq{ x_k \cdot y_k }_{k \in \mscrK}.
  \end{equation*}

  For every index \( m \in \mscrK \), we define the canonical projection
  \begin{equation*}
    \begin{aligned}
      &\pi_m: \prod_{k \in \mscrK} M_k \to M_m \\
      &\pi_m(\seq{ x_k }_{k \in \mscrK}) \coloneqq x_m.
    \end{aligned}
  \end{equation*}

  If all \( M_k \) are equal to \( M \), we denote the direct product by \( M^\mscrK \).

  The \term{direct sum} \( \bigoplus_{k \in \mscrK} M_k \) is the submonoid of the direct product, in which only finitely many components of each tuple are distinct from the identity.

  For every index \( m \in \mscrK \), we define the canonical embedding
  \begin{equation*}
    \begin{aligned}
       &\iota_m: M_m \to \bigoplus_{k \in \mscrK} M_k \\
       &\iota_m(x_m) \coloneqq
        \begin{rcases}
          \begin{cases}
            x_m, &k = m \\
            e_k, &k \neq m
          \end{cases}
        \end{rcases}_{k \in \mscrK}
    \end{aligned}
  \end{equation*}

  If all \( M_k \) are equal to \( M \), we denote the direct sum by \( M^{\oplus \mscrK} \).

  Unlike the product, the direct sum is mostly useful for commutative monoids. The role of direct sums is discussed in \fullref{rem:binary_operation_syntax_trees/infinite/direct_sum} and, more concretely, in \fullref{thm:monoid_categorical_limits}.
\end{definition}

\begin{proposition}\label{thm:monoid_categorical_limits}
  \hfill
  \begin{thmenum}
    \thmitem{thm:monoid_categorical_limits/product} The \hyperref[def:discrete_category_limits]{categorical product} of the family \( \seq{ M_k }_{k \in \mscrK} \) in the category \hyperref[def:monoid/category]{\( \cat{Mon} \)} of monoids is their \hyperref[def:monoid_direct_product]{direct product} \( \prod_{k \in \mscrK} M_k \).

    \thmitem{thm:monoid_categorical_limits/coproduct} The \hyperref[def:discrete_category_limits]{categorical coproduct} of the family \( \seq{ M_k }_{k \in \mscrK} \) in the category \hyperref[def:monoid/category]{\( \cat{CMon} \)} of \hi{commutative} monoids is their \hyperref[def:monoid_direct_product]{direct sum} \( \bigoplus_{k \in \mscrK} M_k \).

    Compare this to the non-commutative case for groups discussed in \fullref{thm:group_categorical_limits/coproduct}.
  \end{thmenum}
\end{proposition}
\begin{proof}
  \SubProofOf{thm:monoid_categorical_limits/product} Let \( (A, \alpha) \) be a \hyperref[def:category_of_cones/cone]{cone} for the \hyperref[def:discrete_category]{discrete} \hyperref[def:categorical_diagram]{diagram} \( \seq{ M_k }_{k \in \mscrK} \). We want to define a monoid homomorphism \( l_A: A \to \prod_{k \in \mscrK} M_k \) such that, for every \( m \in \mscrK \) and \( a \in A \),
  \begin{equation*}
    \alpha_m(a) = \pi_m(l_A(a)).
  \end{equation*}

  This suggests the definition
  \begin{equation*}
    l_A(a) \coloneqq \seq{ \alpha_k(a) }_{k \in \mscrK}.
  \end{equation*}

  \SubProofOf{thm:monoid_categorical_limits/coproduct}  Let \( (A, \alpha) \) be a \hyperref[def:category_of_cones/cocone]{cocone} for the discrete diagram \( \seq{ M_k }_{k \in \mscrK} \). We want to define a monoid homomorphism \( l_A: \bigoplus_{k \in \mscrK} M_k \to A \) such that, for every \( m \in \mscrK \) and \( x \in M_m \),
  \begin{equation*}
    \alpha_m(x) = l_A(\iota_m(x)).
  \end{equation*}

  This suggests the definition
  \begin{equation*}
    l_A(\seq{ x_k }_{k \in \mscrK}) \coloneqq \prod_{k \in \mscrK}^n \alpha_k(x_k).
  \end{equation*}

  We discuss well-definedness of infinitary operations in direct sums in \fullref{rem:binary_operation_syntax_trees/infinite/direct_sum}.
\end{proof}
