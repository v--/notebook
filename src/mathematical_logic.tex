\section{Mathematical logic}\label{sec:mathematical_logic}

Mathematical logic uses mathematics to study logic and vice versa.

We start with objects that are purely logical in nature --- formulas --- which are strings of symbols that represent truth values. Formal definitions for formulas are given here using \hyperref[def:grammar]{grammars}, which in turn depend on \hyperref[def:language]{languages}. Formal definitions for truth values are given using \hyperref[def:heyting_algebra]{Heyting} and \hyperref[def:boolean_algebra]{Boolean algebras}

These definitions help us define the theory necessary to study following two important intertwined topics:
\begin{itemize}
  \item We are interested in establishing whether the formula \( \varphi \) logically entails the formula \( \psi \). This is done using \hyperref[def:first_order_derivation_system]{derivation systems} which specify precisely how we can manipulate strings of symbols. This aspect is called \term{syntactic} or \term{logical} and is the basis or \hyperref[def:first_order_derivability]{proof theory}. Formulas allow us to express statements about mathematics and proof theory allows us to systematically study the relationships between them.

  \item Given a formula, we are interested in assigning a meaning to it. Different logical systems provide different syntax that is useful for different purposes - \hyperref[def:propositional_syntax/formula]{propositional formulas} allow us to express complex relationships between propositions via \hyperref[subsec:boolean_functions]{Boolean functions} while \hyperref[def:first_order_syntax/formula]{first-order logic formulas} allow us to go one level lower and give a precise meaning to these propositions via \hyperref[def:first_order_structure]{structures}. This aspect of logic is called \term{semantical} and is the basis of \hyperref[subsec:first_order_models]{model theory}. Model theory allows us to study logical formulas using pre-existing mathematics.
\end{itemize}

There are two aspects in which logical systems are categorized:
\begin{itemize}
  \item \hyperref[subsec:propositional_logic]{Propositional} and \hyperref[subsec:first_order_logic]{first-order logic} (among others) differ in what their syntax allows us to express. This also means that they differ in what their semantics can express but, just as the syntax of first-order logic is a superset of the syntax of propositional logic, \hyperref[subsec:boolean_functions]{Boolean functions} can express relations between quantifierless atomic formulas in any \hyperref[def:first_order_structure]{structures}. In other words, semantics are identical in places where the syntax is the same.

  \item \hyperref[def:classical_logic]{Classical} and \hyperref[def:intuitionistic_logic]{intuitionistic logic} (among others) differ in their semantics and their logical derivation rules. This has two implications
  \begin{itemize}
     \item Boolean functions describe \hyperref[def:classical_logic]{classical logic}, however they fail to describe \hyperref[def:intuitionistic_logic]{intuitionistic logic} because double negation elimination \eqref{eq:thm:minimal_propositional_negation_laws/dne} no longer holds and neither do other similar statements. So, while retaining the same syntax, we must resort to much more complicated semantical frameworks like \hyperref[def:propositional_heyting_semantics]{Heyting} or \hyperref[def:propositional_topological_semantics]{topological semantics}.

     \item The proof theory that describes classical logic no longer matches the semantics, hence we must resort to other derivation systems. This turns out not to be trivial because we need a clear understanding of which logical axioms imply the others. \Fullref{subsec:first_order_proofs} lists different derivation systems and their corresponding semantics.
  \end{itemize}
\end{itemize}

The statements of mathematical logic can themselves be studied logically. We distinguish between the \term{object logic} which we study and the \term{metalogic} which we use to study it. It is possible, for example, to study intuitionistic propositional logic using classical first-order logic. The metalogic is usually less formal and its statements are written in prose for the sake of easier understanding.

Logical systems such as \( \lambda \)-calculus have entirely different syntactic and semantical properties that makes them useful for expressing computation rather than mathematical statements.

\begin{definition}\label{def:classical_logic}
  Classical logic is a vague term that we use to describe a semantic framework (see \fullref{def:propositional_semantics}) and a matching derivation system (see \fullref{def:propositional_axiomatic_derivation_system}) for \hyperref[subsec:propositional_logic]{propositional logic} and also a semantic framework (see \fullref{def:first_order_semantics}) and a matching derivation system (see \fullref{def:first_order_axiomatic_derivation_system}) for \hyperref[subsec:first_order_logic]{first-order logic}, among others.

  It is characterized by the ability to use the law of double negation elimination \eqref{eq:thm:minimal_propositional_negation_laws/dne}. A more popular (but less accurate due to \fullref{thm:minimal_propositional_negation_laws}) characterization is that the law of the excluded middle \eqref{eq:thm:minimal_propositional_negation_laws/lem} holds. Within the metalogic, this law is called the \term{principle of bivalence} and states that either a statement holds or its negation holds.
\end{definition}

\begin{definition}\label{def:intuitionistic_logic}
  Intuitionistic logic is a a generalization of \hyperref[def:classical_logic]{classical logic}. It is also called \term{constructive logic} due to the \hyperref[def:brouwer_heyting_kolmogorov_interpretation]{Brouwer-Heyting-Kolmogorov interpretation} (see \fullref{rem:brouwer_heyting_kolmogorov_interpretation_compatibility} for further discussion of the topic).

  Instead of the principle of bivalence, we have the weaker \term{principle of non-contradiction}, which is the metalogical statement corresponding to the axiom schema
  \begin{equation}\label{eq:def:intuitionistic_logic/non_contradiction}
    \neg (\varphi \wedge \neg \varphi). \tag{LNC}
  \end{equation}

  In fact, \eqref{eq:def:intuitionistic_logic/non_contradiction} follows from \eqref{eq:thm:minimal_propositional_negation_laws/efq} by introducing an implication using \eqref{eq:def:positive_implicational_propositional_derivation_system/axioms/imp_intro} with \( \neg \varphi \) as the antecedent.

  Metalogically speaking, we can only conclude that there is no statement such that both the statement and its negation are true. If the statement instead does not hold, we cannot automatically conclude that its negation holds.

  Both laws \eqref{eq:thm:minimal_propositional_negation_laws/lem} and \eqref{eq:def:intuitionistic_logic/non_contradiction} are equivalent in classical logic but their equivalence relies on double negation elimination \eqref{eq:thm:minimal_propositional_negation_laws/dne}, which is not an accepted axiom in intuitionistic logic.

  We generally accept the semantic frameworks of \hyperref[def:propositional_heyting_semantics]{Heyting algebra semantics} and \hyperref[def:propositional_topological_semantics]{topological semantics} (see \fullref{def:propositional_semantics}) and a matching derivation system (see \fullref{def:propositional_axiomatic_derivation_system}) for \hyperref[subsec:propositional_logic]{propositional logic}.
\end{definition}

\begin{remark}\label{rem:mathematical_logic_conventions}
  We will not try to completely formalize the syntax of the metalogic, i.e. the \term{metalanguage}, and we will only work in \hyperref[def:propositional_axiomatic_derivation_system]{classical metalogic}. Outside the section on logic, we will use formulas and, more generally, use object logic only in dedicated places like \fullref{def:group/theory} describing the \hyperref[def:first_order_theory]{logical theory} of groups. Most axioms like \ref{def:norm/N1}-\ref{def:norm/N3} for norms are formulated entirely within the metalanguage under the assumption that we are working within a model of set theory. To keep a clear distinction between logical formulas and non-logical axioms and, more generally, to distinguishing between logic and metalogic, we use the following conventions:

  \begin{thmenum}
    \thmitem{rem:mathematical_logic_conventions/variable_symbols} Variables in the object language are denoted by the small Greek letters, usually \( \xi, \eta, \zeta \), while variables in the metalanguage are denoted by small Latin letters, usually \( x, y, z \). If needed, we add subscripts with indices.

    \thmitem{rem:mathematical_logic_conventions/formula_term_symbols} Formulas, which we only consider in the object language, are also denoted by small Greek letters --- \( \varphi, \psi, \rho, \theta \) --- and so are terms --- \( \tau, \kappa, \mu \).

    \thmitem{rem:mathematical_logic_conventions/propositional_constants} The propositional constants denoting truth and falsity are denoted by \( \top \) and \( \bot \) in the object language and by \( T \) and \( F \) in the metalanguage. This is only for the sake of following an established convention and we still use \( \top \) and \( \bot \) in general \hyperref[def:semilattice/lattice]{lattices}.

    \thmitem{rem:mathematical_logic_conventions/connective_symbols} We usually prefer prose to symbolic quantifiers and connectives in the metalanguage. The longer arrows \( \implies \) and \( \iff \) are sometimes used within the metalogic outside of this section.

    \thmitem{rem:mathematical_logic_conventions/structure_pairs} Structures in the metalogic (i.e. sets with functions and/or relations defined on them) are usually denoted by calligraphic letters (e.g. \( \mscrX \), \( \mscrR \)) --- see \fullref{rem:first_order_model_notation} for a discussion.

    \thmitem{rem:mathematical_logic_conventions/shorthands} We additionally use syntactic shorthands like \fullref{rem:propositional_formula_parentheses} and \fullref{rem:first_order_formula_conventions} when writing formulas.
  \end{thmenum}

  Some axioms like \eqref{eq:def:magma/idempotent} are formulated within the metalogic for convenience and clarity but are used as formulas in the object language in theorems like \fullref{thm:positive_formulas_preserved_under_homomorphism}. In places like this, it is usually straightforward to translate axioms from the metalogic into logical formulas.
\end{remark}
