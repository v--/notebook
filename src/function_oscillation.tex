\subsection{Function oscillation}\label{subsec:function_oscillation}

\begin{definition}\label{def:function_oscillation}
  Let \( \mscrX \) be a nonempty set and \( (\mscrY, \rho_{\mscrY}) \) be a metric space. We define the \term{oscillation} of a function on a set as
  \begin{balign*}
     &\omega: \fun(\mscrX, \mscrY) \times \pow(\mscrX) \to [0, \infty] \\
     &\omega(f, A) \coloneqq \sup \Big\{ \rho_{\mscrY}(f(x), f(y)) \colon (x, y) \in A \Big\}.
  \end{balign*}

  In particular, if \( \mscrX \) is itself a metric space, we define its \term{modulus of continuity} \( \omega(f, \delta) \) as the oscillation of \( f \) on the ball \( B(0, \delta) \).
\end{definition}

\begin{proposition}\label{thm:function_oscillation_properties}
  The \hyperref[def:function_oscillation]{modulus of continuity} has the following basic properties:
  \begin{thmenum}
    \ilabel{thm:function_oscillation_properties/continuity_condition} \( f \) is globally \hyperref[def:uniform_continuity]{uniformly continuous} if and only if for every \( \varepsilon > 0 \) there exists \( \delta > 0 \) such that \( \omega(f, \delta) < \varepsilon \).

    \ilabel{thm:function_oscillation_properties/monotone} \( \omega(f, \delta) \) is monotone in \( \delta \).

    \ilabel{thm:function_oscillation_properties/cauchy_inequality}\mcite\cite[28]{Николов2020}For all \( \lambda, \delta > 0 \), we have the following analog of \fullref{thm:cauchy_bunyakovsky_schwarz_inequality}
    \begin{equation}\label{thm:function_oscillation_properties/cauchy_inequality/inequality}
      \omega(f, \lambda \delta) \leq \omega(f, \lambda^2) + \omega(f, \delta^2).
    \end{equation}

    \ilabel{thm:function_oscillation_properties/single_inequality}\mcite\cite[28]{Николов2020}For all \( \lambda, \delta > 0 \),
    \begin{equation}\label{thm:function_oscillation_properties/single_inequality/inequality}
      \omega(f, \lambda \delta) \leq (\lambda + 1) \omega(f, \delta).
    \end{equation}
  \end{thmenum}
\end{proposition}
\begin{proof}
  \SubProofOf{thm:function_oscillation_properties/continuity_condition} Follows directly from \fullref{def:uniform_continuity}.

  \SubProofOf{thm:function_oscillation_properties/monotone} A supremum on a larger set is larger.

  \SubProofOf{thm:function_oscillation_properties/cauchy_inequality} If \( \lambda \leq \delta \), clearly \( \lambda \delta \leq \delta^2 \). Otherwise, \( \lambda \delta < \lambda^2 \).

  Combining the two inequalities with \fullref{thm:function_oscillation_properties/monotone}, we obtain \fullref{thm:function_oscillation_properties/cauchy_inequality/inequality}.

  \SubProofOf{thm:function_oscillation_properties/single_inequality} Note that
  \begin{equation*}
    \rho_{\mscrX}(x, y) < \delta \T{implies} \rho_{\mscrY}(f(x), f(y)) < \omega(f, \delta).
  \end{equation*}

  We can multiply this by \( \lambda \) to obtain
  \begin{equation*}
    \lambda \rho_{\mscrX}(x, y) < \lambda \delta \T{implies} \lambda \rho_{\mscrY}(f(x), f(y)) < \lambda \omega(f, \delta).
  \end{equation*}

  If \( \lambda \geq 1 \), then \( \rho_{\mscrX}(x, y) \leq \lambda \rho_{\mscrX}(x, y) \) and \( \rho_{\mscrY}(f(x), f(y)) \leq \lambda \rho_{\mscrY}(f(x), f(y)) \) and hence
  \begin{equation*}
    \omega(f, \lambda \delta) \leq \lambda \omega(f, \delta).
  \end{equation*}

  Otherwise, \( \lambda < 1 \) and clearly \( \lambda \delta < \delta \), which by \fullref{thm:function_oscillation_properties/monotone} implies
  \begin{equation*}
    \omega(f, \lambda \delta) \leq \omega(f, \delta).
  \end{equation*}

  Combining the two cases, we obtain
  \begin{equation*}
    \omega(f, \lambda \delta) \leq \lambda \omega(f, \delta) + \omega(f, \delta),
  \end{equation*}
  which we wanted to prove.
\end{proof}
