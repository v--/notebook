\subsection{Group presentations}\label{subsec:group_presentations}

\begin{definition}\label{def:free_monoid}
  Let \( \mscrS \) be an arbitrary set. We associate with \( \mscrS \) its \term{free monoid} \( F(\mscrS) \coloneqq (\mscrS^{\ast}, \cdot) \), where \( \mscrS^{\ast} \) is the \hyperref[def:formal_language/kleene_star]{Kleene star} and \( \cdot \) is \hyperref[def:formal_language/concatenation]{concatenation}.
\end{definition}
\begin{proof}
  It is a monoid due to \fullref{thm:kleene_star_is_monoid}.
\end{proof}

\begin{proposition}\label{thm:free_monoid_is_free_functor}
  Consider the functor \( F: \cat{Set} \to \cat{Mon} \) defined for objects pointwise in \fullref{def:free_monoid}. For every function \( f: A \to B \), define the \hyperref[def:unital_magma/homomorphism]{monoid homomorphism}
  \begin{equation*}
    \begin{aligned}
      &F(f): F(A) \to F(B) \\
      &F(f)(x_1 \cdots x_n) \coloneqq f(x_1) \cdots f(x_n).
    \end{aligned}
  \end{equation*}

  This functor is \hyperref[def:category_adjunction]{left adjoint} to the \hyperref[def:concrete_category]{forgetful functor} \( U: \cat{Mon} \to \cat{Set} \).
\end{proposition}
\begin{proof}
  Fix a small monoid \( \mscrM \) and small set \( A \). We will first show that
  \begin{equation*}
    \cat{Mon}(F(A), \mscrM) \cong \cat{Set}(A, U(\mscrM)).
  \end{equation*}

  For every homomorphism \( \varphi: F(A) \to \mscrM \), we can define a function \( f: A \to U(\mscrM) \) by restricting \( \varphi \) to single-symbol \hyperref[def:formal_language/word]{words}.

  Conversely, for every function \( f: A \to U(\mscrM) \), we have the monoid homomorphism
  \begin{equation*}
    \begin{aligned}
      &\varphi: F(A) \to \mscrM \\
      &\varphi(x_1 \cdots x_n) \coloneqq x_1 \cdot \ldots \cdot x_n
    \end{aligned}
  \end{equation*}
  obtained by applying the monoid operation \( \cdot \) recursively to the word \( x_1 \cdots x_n \).

  This establishes a correspondence between \( \cat{Mon}(F(A), \mscrM) \) and \( \cat{Set}(A, U(\mscrM)) \). We need to establish that this is a natural transformation.
\end{proof}

\begin{definition}\label{def:free_group}
  Let \( \mscrS \) be an arbitrary set. We will now construct the \term{free group} \( F(\mscrS) \) of \( \mscrS \). The construction is similar to that of \hyperref[def:free_monoid]{free monoids}, but it is much more complicated because of special reduction rules for \hyperref[def:unital_magma_inverse_element]{inverse elements}. Refer to \cite{code:free_group_grammar_verification} for a, software implementation of the construction.

  Let \( \star \) be a \hyperref[def:formal_language/symbol]{symbol} not in \( \mscrS \). Our goal is, for each \( a \in S \), to make the word \( a{\star} \) behave like the inverse of \( a \) in a group. Rather than considering the \hyperref[def:formal_language/kleene_star]{Kleene star} \( (S \cup \{ \star \})^* \) and removing elements via \enquote{reductions} as in \cite{code:free_group_reduction_verification} and \cite[306]{Knapp2016BasicAlgebra}, we directly build a language of \term{reduced words} using the mutually recursive \hyperref[def:formal_grammar]{grammar}
  \begin{alignedeq}\label{eq:def:free_group/grammar}
    &I \to \varepsilon,           &&                        && \text{\( I \) is the initial state} \\
    &I \to S_a \mid D_a,             && a \in S              && \\
    &S_a \to a \mid a S_a,           && a \in S              && S_a \text{ does not produce words beginning with } a\star \\
    &S_a \to a D_b,               && a, b \in S, a \neq b && \\
    &D_a \to a\star S_b,          && a, b \in S, a \neq b && D_a \text{ does not produce words beginning with } a \\
    &D_a \to a\star \mid a\star D_a, && a \in S              && \\
  \end{alignedeq}

  The \term{free group} \( F(\mscrS) \) is defined to be the language of \eqref{eq:def:free_group/grammar} equipped with the inductively defined operation
  \begin{equation}\label{eq:def:free_group/operation}
    w_1 \odot w_2 \coloneqq \begin{cases}
     p \odot s, &w_1 = p a \T{and} w_2 = a\star s \text{ for some } a \in S, \\
     p \odot s, &w_1 = p a\star \T{and} w_2 = as \T{and} s \neq \star t \text{ for some } a \in S, \\
     ps,        &\text{otherwise}.
   \end{cases}
  \end{equation}

  The inverse of the word \( w = a_1 \ldots a_n \) is \( w^{-1} \coloneqq b_1 \ldots b_n \), where
   \begin{equation}\label{eq:def:free_group/inverse}
     b_{n-k+1} \coloneqq \begin{cases}
       \varnothing, &a_k = {\star} \\
       a_k{\star},  &a_k \neq {\star} \T{and} k = n \\
       a_k{\star},  &a_k \neq {\star} \T{and} a_{k+1} \neq {\star} \\
       a_k,         &a_k \neq {\star} \T{and} k \neq n \T{and} a_{k+1} = {\star} \\
     \end{cases}
   \end{equation}
   for \( k = 1, \ldots, n \).

  The group \( (F(\mscrS), \odot) \) is called the \term{free group} generated by \( \mscrS \).
\end{definition}
\begin{proof}
  The proof of the well-definedness of the group structure of \( F(\mscrS) \) is a straightforward (but tedious) application of induction.
\end{proof}

\begin{proposition}\label{thm:free_group_is_free_functor}
  The functor \( F: \cat{Set} \to \cat{Grp} \), defined pointwise in \fullref{def:free_group}, is \hyperref[def:category_adjunction]{free}.
\end{proposition}
\begin{proof}
  The outline of the proof is similar to the proof of \fullref{thm:free_monoid_is_free_functor}.
\end{proof}

\begin{definition}\label{def:group_presentation}\mcite[314]{Knapp2016BasicAlgebra}
  Let \( \mscrS \) be a set, \( F(\mscrS) \) be the \hyperref[def:free_group]{free group} and \( \mscrR \subseteq F(\mscrS) \) be a subset. Denote by \( \mscrN(\mscrR) \) the smallest normal subgroup of \( F(\mscrS) \) that includes \( \mscrR \) as a subset.

  We define the group
  \begin{equation}\label{eq:def:group_presentation/presentation}
    \mscrG = \braket{ \mscrS \mid \mscrR} \coloneqq F(\mscrS) / \mscrN(\mscrR)
  \end{equation}
  called the group with \term{generators} \( \mscrS \) and \term{relators} \( \mscrR \). The expression \eqref{eq:def:group_presentation/presentation} is called a \term{presentation} of \( \mscrG \).

  If there exists a presentation for \( \mscrG \) such that \( \mscrS \) is finite, it is called a \term{finitely generated} group. If there exists a presentation such that both \( \mscrS \) and \( \mscrR \) are finite, it is called \term{finitely presented}.

  If \( \mscrR = \varnothing \), there are no restrictions and we use the notation
  \begin{equation}\label{eq:def:group_presentation/free}
    \mscrG = \braket{ \mscrS } \coloneqq F(\mscrS)
  \end{equation}
  for the free group.
\end{definition}

\begin{theorem}\label{thm:every_group_is_representable}\mcite[prop. 7.7]{Knapp2016BasicAlgebra}
  Every group \( \mscrG \) has at least one \hyperref[def:group_presentation]{presentation}.
\end{theorem}
\begin{proof}
  Let \( \mscrG \) be an arbitrary group and let \( \mscrS \coloneqq U(\mscrG) \) be the underlying set. Let \( F(\mscrS) \) be the corresponding free group with \( \iota: \mscrS \to F(\mscrS) \) sending elements of \( \mscrS \) to singleton words in \( F(\mscrS) \). By \fullref{thm:free_group_is_free_functor}, there exists a unique homomorphism \( \varphi: F(\mscrS) \to \mscrG \) such that
  \begin{equation*}
    \text{\todo{Add diagram}}\iffalse\begin{mplibcode}
      beginfig(1);
      input metapost/graphs;

      v1 := thelabel("$\mscrS$", origin);
      v2 := thelabel("$U(F(\mscrS))$", (-1, -1) scaled u);
      v3 := thelabel("$U(G)$", (1, -1) scaled u);

      a1 := straight_arc(v1, v2);
      a2 := straight_arc(v1, v3);

      d1 := straight_arc(v2, v3);

      draw_vertices(v);
      draw_arcs(a);

      drawarrow d1 dotted;

      label.ulft("$\iota$", straight_arc_midpoint of a1);
      label.urt("$\id$", straight_arc_midpoint of a2);
      label.top("$U(\varphi)$", straight_arc_midpoint of d1);
      endfig;
    \end{mplibcode}\fi
  \end{equation*}
  that is, \( U(\varphi) \circ \iota = \id \). Thus, \( G = \mscrS \subseteq \ker \varphi \). Define \( \mscrR \coloneqq \ker \varphi \). By \fullref{def:normal_subgroup}, \( \mscrR \) is a normal subgroup of \( F(\mscrS) \), thus
  \begin{equation*}
    G = \varphi(F(\mscrS)) \cong F(\mscrS) / \ker \varphi = \braket{ \mscrS \mid \mscrR }.
  \end{equation*}
\end{proof}

\begin{definition}\label{def:cyclic_group}
  For a singleton alphabet \( \set{ a } \), we define the \term{infinite cyclic group}
  \begin{equation*}
    C \coloneqq \braket{a}
  \end{equation*}
  and, for positive integers \( n \), the \term{finite cyclic group} of \term{order} \( n \) as
  \begin{equation*}
    C_n \coloneqq \braket{a \given a^n}.
  \end{equation*}

  We use the same notation independent of \( a \) because all cyclic groups of the same order are obviously \hyperref[def:group/homomorphism]{isomorphic}.

  See \fullref{thm:cyclic_group_isomorphic_to_integers_modulo_n}.
\end{definition}

\begin{definition}\label{def:group_free_product}\mcite[323]{Knapp2016BasicAlgebra}
  The \term{free product} of a nonempty family of groups \( \seq{ \mscrX_k }_{k \in \mscrK} \) with presentations \( \braket{\mscrS_k \mid \mscrR_k}, k \in \mscrK \) is the group
  \begin{equation*}
    \Ast_{k \in \mscrK} \mscrX_k \coloneqq \braket*{ \coprod_{k \in \mscrK} \mscrS_k \given* \coprod_{k \in \mscrK} \mscrR_k },
  \end{equation*}
  where \( \coprod \) is the \hyperref[def:disjoint_union]{disjoint union}.
\end{definition}

\begin{definition}\label{def:free_abelian_group}
  A \term{free abelian group} is a \hyperref[def:free_left_module]{free} \hyperref[thm:abelian_group_iff_z_module]{\( \BbbZ \)-module}. This definition of a free abelian group is different from the definition of a \hyperref[def:free_group]{free group}.
\end{definition}

\begin{proposition}\label{thm:product_of_cyclic_groups}
  The \hyperref[def:group_direct_product]{direct product} \( C_n \times C_m \) of two \hyperref[def:cyclic_group]{cyclic groups} is cyclic if and only if \( n \) and \( m \) are \hyperref[def:coprime_numbers]{coprime}.
\end{proposition}
\begin{proof}
  Take \( (a^i, a^j) \in C_n \times C_m \).
\end{proof}

\begin{definition}\label{def:group_direct_product}
  The \term{direct product} of a nonempty family of groups \( \{ \mscrX_k \}_{k \in \mscrK} \) is their \hyperref[def:cartesian_product]{Cartesian product} \( \prod_{k \in \mscrK} \mscrX_k \) with the componentwise group operation
  \begin{equation*}
    \{ x_k \}_{k \in \mscrK} \cdot \{ y_k \}_{k \in \mscrK}
    \coloneqq
    \{ x_k \cdot y_k \}_{k \in \mscrK}.
  \end{equation*}
\end{definition}

\begin{definition}\label{def:group_direct_sum}
  The \term{direct sum} \( \bigoplus_{k \in \mscrK} \mscrX_k \) of a nonempty family of groups \( \{ \mscrX_k \}_{k \in \mscrK} \) is a subgroup of their \hyperref[def:group_direct_sum]{direct product} where, for any group element \( \{ x_k \}_{k \in \mscrK} \), only finitely many components are different from zero.

  \begin{thmenum}
    \thmitem{def:group_direct_sum/internal}\mcite[126]{Knapp2016BasicAlgebra}If all \( \{ \mscrX_k \}_{k \in \mscrK} \) are subgroups of a group \( \mscrX \), we say that \( \mscrX \) is their \term{internal direct sum} if the homomorphism
    \begin{align*}
       &\varphi: \bigoplus_{k \in \mscrK} \mscrX_k \to X \\
       &\varphi(\{ x_k \}_{k \in \mscrK}) \coloneqq \cdot_{k \in \mscrK} x_k
    \end{align*}
    is an isomorphism.

    The sum is well-defined since, by definition, there are only finitely many non-identity summands.

    \thmitem{def:group_direct_sum/external} To distinguish \( \bigoplus_{k \in \mscrK} \mscrX_k \) from \( X \), we sometimes call it the \term{external direct product}.
  \end{thmenum}
\end{definition}

\begin{proposition}\label{thm:group_categorical_limits}
  We are interested in \hyperref[def:category_of_cones/limit]{categorical limits} and \hyperref[def:category_of_cones/colimit]{colimits} in \( \cat{Grp} \). Fix an indexed family  \( \{ \mscrX_k \}_{k \in \mscrK} \) of groups.

  \begin{thmenum}
    \thmitem{thm:group_categorical_limits/product} Their \hyperref[def:discrete_category_limits]{categorical product} is their \hyperref[def:group_direct_product]{direct product} \( \prod_{k \in \mscrK} \mscrX_k \), the projection morphisms being inherited from \fullref{thm:discrete_category_limits_in_set}.

    \thmitem{thm:group_categorical_limits/coproduct} Their \hyperref[def:discrete_category_limits]{categorical coproduct} is their \hyperref[def:group_free_product]{free product} \( \Ast_{k \in \mscrK} \mscrX_k \), the embedding morphisms being
    \begin{balign*}
       &\iota_m: \mscrX_m \to \Ast_{k \in \mscrK} \mscrX_k \\
       &\iota_m(x_m) \coloneqq x_m.
    \end{balign*}
  \end{thmenum}
\end{proposition}
