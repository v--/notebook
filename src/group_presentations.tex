\subsection{Group presentations}\label{subsec:group_presentations}

\begin{definition}\label{def:free_monoid}
  Let \( \CS \) be an arbitrary set. We associate with \( \CS \) its \Def{free monoid} \( F(\CS) \coloneqq (\CS^{\ast}, \cdot) \), where \( \CS^{\ast} \) is the \hyperref[def:language/kleene_star]{Kleene star} and \( \cdot \) is \hyperref[def:language/concatenation]{concatenation}. It is a monoid due to \fullref{thm:kleene_star_is_monoid}.
\end{definition}

\begin{proposition}\label{thm:free_monoid_is_free_functor}
  The functor \( F: \Cat{Set} \to \Cat{Mon} \), defined pointwise in \fullref{def:free_monoid}, is \hyperref[def:free_functor]{free}.
\end{proposition}
\begin{proof}
  Let \( U: \Cat{Mon} \to \Cat{Set} \) be the corresponding forgetful functor and let \( \CM \in \Cat{Mon} \), \( \CS \in \Cat{Set} \). We will first show that
  \begin{equation*}
    \Cat{Mon}(F(\CS), \CM) = \Cat{Set}(\CS, U(\CM)),
  \end{equation*}
  where equality means that all the underlying functions are equal.

  Every monoid homomorphism is a function so obviously
  \begin{equation*}
    \Cat{Mon}(F(\CS), \CM) \subseteq \Cat{Set}(\CS, U(\CM)).
  \end{equation*}

  Now consider a function \( f: \CS \to U(\CM) \). Define the function
  \begin{BreakableAlign*}
     &\varphi: F(\CS) \to \CM \\
     &\varphi\left( \{ x_k \}_{k \in \CK} \right) \coloneqq \prod_{k \in \CK} x_k.
  \end{BreakableAlign*}

  Obviously \( \varphi \) is a homomorphism from \( F(\CS) \) to \( \CM \). Hence
  \begin{equation*}
    \Cat{Set}(\CS, U(\CM)) \subseteq \Cat{Mon}(F(\CS), \CM).
  \end{equation*}
\end{proof}

\begin{definition}\label{def:free_group}\MarginCite[306]{Knapp2016BAlg}
  Let \( \CS \) be an arbitrary set. We will now construct the \Def{free group} \( F(\CS) \) of \( \CS \). We shall extend \fullref{def:free_monoid} by emulating \Def{inverse elements}.

  First, regard \( \CS \) as an \hyperref[def:language/alphabet]{alphabet}. Let \( {\star} \) be a \hyperref[def:language/symbol]{symbol} not in \( \CS \). Consider the language \( W \coloneqq (\CS \cup \{ {\star} \})^{\ast} \). Our goal is, for each \( a \in \CS \), to make the word \( a{\star} \) behave like the inverse of \( a \) in a group.

  We define the \Def{reduction function}
  \begin{align*}
    &r: W \to W \\
    &r(w) \coloneqq \begin{cases}
      r(s),           &w = {\star}s, \\
      r(ps),          &w = p{\star}{\star}s, \\
      r(ps),          &w = p a{\star}a s \text{ for some } a \in \CS \Tand s \neq {\star}t, \\
      r(ps),          &w = p aa{\star} s \text{ for some } a \in \CS, \\
      w,              &\text{otherwise}.
    \end{cases}
  \end{align*}

  Correctness of \( r \) is empirically verified in \cite{code:free_group_reduction}.

  The set of \Def{reduced words} is the image \( F(\CS) \coloneqq \Img r(W) \). It forms a group under the operation
  \begin{equation*}
    u \odot w \coloneqq r(uw).
  \end{equation*}

  The inverse of a reduced word \( w = a_1 \ldots a_n \) is \( w^{-1} \coloneqq b_1 \ldots b_n \), where
   \begin{equation*}
     b_{n-k+1} \coloneqq \begin{cases}
       \varnothing, &a_k = {\star} \\
       a_k{\star},   &a_k \neq {\star} \Tand k = n \\
       a_k{\star},   &a_k \neq {\star} \Tand a_{k+1} \neq {\star} \\
       a_k,         &a_k \neq {\star} \Tand k \neq n \Tand a_{k+1} = {\star} \\
     \end{cases}.
   \end{equation*}
   for \( k = 1, \ldots, n \).

  The group \( (F(\CS), \odot) \) is called the \Def{free group} generated by \( \CS \).
\end{definition}

\begin{proposition}\label{thm:free_group_is_free_functor}
  The functor \( F: \Cat{Set} \to \Cat{Grp} \), defined pointwise in \fullref{def:free_group}, is \hyperref[def:free_functor]{free}.
\end{proposition}
\begin{proof}
  The proof is similar to the proof of \fullref{thm:free_monoid_is_free_functor}, except that the induced group structure requires some consistency proofs.
\end{proof}

\begin{definition}\label{def:group_free_product}\MarginCite[323]{Knapp2016BAlg}
  Let \( \{ \CX_k \}_{k \in \CK} \) be a nonempty family of groups.

  Consider the \hyperref[def:language/kleene_star]{Kleene star}
  \begin{equation*}
    W \coloneqq \left( \bigcup_{k \in \CK} X_k \right)^{\ast}.
  \end{equation*}

  We define the \Def{reduction function}
  \begin{align*}
    &r: W \to W \\
    &r(w) \coloneqq \begin{cases}
      r(ps),  & w = p e_k s \text{ for some } k \in \CK,                                                                              \\
      r(pts), & w = puvs, \text{ where } u \neq e_k \text{ and } v \neq e_k \text{ and } t = u \cdot_k v \text{ for some } k \in \CK, \\
      w,      & \text{otherwise}.
    \end{cases}
  \end{align*}

  The set of \Def{reduced words} is the image \( \star_{k \in \CK} \CX_k \coloneqq \Img r(W) \). It forms a group under the operation
  \begin{equation*}
    u \odot w \coloneqq r(uw).
  \end{equation*}

  The group \( (\star_{k \in \CK} X_k, \odot) \) is called the \Def{free product} of the family \( \{ \CX_k \}_{k \in \CK} \).
\end{definition}

\begin{definition}\label{def:free_abelian_group}
  A \Def{free abelian group} is a \hyperref[def:free_left_module]{free} \hyperref[thm:abelian_group_iff_z_module]{\( \BZ \)-module}. This definition of a free abelian group is different from the definition of a \hyperref[def:free_group]{free group}.
\end{definition}

\begin{definition}\label{def:group_presentation}\MarginCite[314]{Knapp2016BAlg}
  Let \( \CS \) be a set, \( F(\CS) \) be the \hyperref[def:free_group]{free group} and \( R \subseteq F(\CS) \) be a subset. Denote by \( \CN(R) \) the smallest normal subgroup of \( F(\CS) \) that includes \( R \) as a subset.

  We define the group
  \begin{equation}\label{def:group_presentation/presentation}
    \CG = \Gen{s \in \CS : r \in R} \coloneqq F(\CS) / \CN(R)
  \end{equation}
  called the group with \Def{generators} \( \CS \) and \Def{relations} \( R \). The expression \fullref{def:group_presentation/presentation} is called a \Def{presentation} of \( \CG \).

  If there exists a presentation for \( \CG \) such that \( \CS \) is finite, it is called a \Def{finitely generated} group. If there exists a presentation such that both \( \CS \) and \( R \) are finite, it is called \Def{finitely presented}.
\end{definition}

\begin{theorem}\label{thm:every_group_is_representable}\MarginCite[prop. 7.7]{Knapp2016BAlg}
  Every group \( \CG \) has at least one \hyperref[def:group_presentation]{presentation}.
\end{theorem}
\begin{proof}
  Let \( \CG \) be an arbitrary group and let \( \CS \coloneqq U(\CG) \) be the underlying set. Let \( F(\CS) \) be the corresponding free group with \( \iota: \CS \to F(\CS) \) sending elements of \( \CS \) to singleton words in \( F(\CS) \). By \fullref{thm:free_group_is_free_functor}, there exists a unique homomorphism \( \varphi: F(\CS) \to \CG \) such that
  \begin{equation*}
    \begin{mplibcode}
      beginfig(1);
      input metapost/graphs;

      v1 := thelabel("$\CS$", origin);
      v2 := thelabel("$U(F(\CS))$", (-1, -1) scaled u);
      v3 := thelabel("$U(G)$", (1, -1) scaled u);

      a1 := straight_arc(v1, v2);
      a2 := straight_arc(v1, v3);

      d1 := straight_arc(v2, v3);

      draw_vertices(v);
      draw_arcs(a);

      drawarrow d1 dotted;

      label.ulft("$\iota$", straight_arc_midpoint of a1);
      label.urt("$\Id$", straight_arc_midpoint of a2);
      label.top("$U(\varphi)$", straight_arc_midpoint of d1);
      endfig;
    \end{mplibcode}
  \end{equation*}
  that is, \( U(\varphi) \circ \iota = \Id \). Thus \( G = \CS \subseteq \ker \varphi \). Define \( R \coloneqq \ker \varphi \). By \fullref{def:normal_subgroup}, \( R \) is a normal subgroup of \( F(\CS) \), thus
  \begin{equation*}
    G = \varphi(F(\CS)) \cong F(\CS) / \ker \varphi = \Gen{ \CS \colon R }.
  \end{equation*}
\end{proof}

\begin{definition}\label{def:cyclic_group}
  For a singleton alphabet \( \{ a \} \), we define the \Def{infinite cyclic group}
  \begin{equation*}
    C \coloneqq \Gen{a}
  \end{equation*}
  and, for positive integers \( n \), the \Def{finite cyclic group}
  \begin{equation*}
    C_n \coloneqq \Gen{a \mid a^n}.
  \end{equation*}
\end{definition}

\begin{definition}\label{def:generated_subgroup}
  Let \( \CS \subseteq \CG \) be any nonempty subset of a group \( \CG \). We define the subgroup \( \Gen{\CS} \) generated by \( \CS \) using any of the equivalent definitions:
  \begin{DefEnum}
    \ILabel{def:generated_subgroup/minimal} \( \Gen{\CS} \) is the smallest subgroup of \( \CG \) that contains \( \CS \).
    \ILabel{def:generated_subgroup/presentation} \( \Gen{\CS} \) is the subgroup of \( \CG \) that is isomorphic to the free group \( F(\CS) \).
    \ILabel{def:generated_subgroup/direct} \( \Gen{\CS} \) is the subgroup
    \begin{equation*}
      \Gen{\CS} \coloneqq \Bigg\{ \prod \CS' \mid \CS' \text{ is a finite subset of } \CS \cup \CS^{-1} \Bigg\},
    \end{equation*}
    where \( \CS^{-1} \coloneqq \{ s^{-1} \mid s \in \CS \} \).
  \end{DefEnum}

  If \( \CS \) is finite, then \( \Gen \CS \) is called \Def{finitely generated}.
\end{definition}
