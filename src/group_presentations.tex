\subsection{Group presentations}\label{subsec:group_presentations}

\begin{definition}\label{def:free_monoid}
  Let \( \CS \) be an arbitrary set. We associate with \( \CS \) its \Def{free monoid} \( F(\CS) \coloneqq (\CS^{\ast}, \cdot) \), where \( \CS^{\ast} \) is the \hyperref[def:language/kleene_star]{Kleene star} and \( \cdot \) is \hyperref[def:language/concatenation]{concatenation}. It is a monoid due to \fullref{thm:kleene_star_is_monoid}.
\end{definition}

\begin{proposition}\label{thm:free_monoid_is_free_functor}
  The functor \( F: \Cat{Set} \to \Cat{Mon} \), defined pointwise in \fullref{def:free_monoid}, is \hyperref[def:free_functor]{free}.
\end{proposition}
\begin{proof}
  Let \( U: \Cat{Mon} \to \Cat{Set} \) be the corresponding forgetful functor and let \( \CM \in \Cat{Mon} \), \( \CS \in \Cat{Set} \). We will first show that
  \begin{equation*}
    \Cat{Mon}(F(\CS), \CM) = \Cat{Set}(\CS, U(\CM)),
  \end{equation*}
  where equality means that all the underlying functions are equal.

  Every monoid homomorphism is a function so obviously
  \begin{equation*}
    \Cat{Mon}(F(\CS), \CM) \subseteq \Cat{Set}(\CS, U(\CM)).
  \end{equation*}

  Now consider a function \( f: \CS \to U(\CM) \). Define the function
  \begin{BreakableAlign*}
     &\varphi: F(\CS) \to \CM \\
     &\varphi\left( \{ x_k \}_{k \in \CK} \right) \coloneqq \prod_{k \in \CK} x_k.
  \end{BreakableAlign*}

  Obviously \( \varphi \) is a homomorphism from \( F(\CS) \) to \( \CM \). Hence
  \begin{equation*}
    \Cat{Set}(\CS, U(\CM)) \subseteq \Cat{Mon}(F(\CS), \CM).
  \end{equation*}
\end{proof}

\begin{definition}\label{def:free_group}
  Let \( \CS \) be an arbitrary set. We will now construct the \Def{free group} \( F(\CS) \) of \( \CS \). The construction is similar to that of \hyperref[def:free_monoid]{free monoids} but it is much more complicated because of special reduction rules for \hyperref[def:unital_magma_inverse_element]{inverse elements}. Refer to \cite{code:free_group_grammar_verification} for a software implementation of the construction.

  Let \( \star \) be a \hyperref[def:language/symbol]{symbol} not in \( \CS \). Our goal is, for each \( a \in \CS \), to make the word \( a{\star} \) behave like the inverse of \( a \) in a group. Rather than considering the \hyperref[def:language/kleene_star]{Kleene star} \( (S \cup \{ \star \})^* \) and removing elements via \enquote{reductions} as in \cite{code:free_group_reduction_verification} and \cite[306]{Knapp2016BAlg}, we directly build a language of \Def{reduced words} using the mutually recursive \hyperref[def:grammar]{grammar}
  \begin{AlignedEquation}\label{eq:def:free_group/grammar}
    &I \to \varepsilon,           &&                        && \text{\( I \) is the initial state} \\
    &I \to S_a \mid D_a,             && a \in \CS              && \\
    &S_a \to a \mid a S_a,           && a \in \CS              && S_a \text{ does not produce words beginning with } a\star \\
    &S_a \to a D_b,               && a, b \in \CS, a \neq b && \\
    &D_a \to a\star S_b,          && a, b \in \CS, a \neq b && D_a \text{ does not produce words beginning with } a \\
    &D_a \to a\star \mid a\star D_a, && a \in \CS              && \\
  \end{AlignedEquation}

  The \Def{free group} \( F(\CS) \) is defined to be the language of \eqref{eq:def:free_group/grammar} equipped with the inductively\IND defined operation
  \begin{equation}\label{eq:def:free_group/operation}
    w_1 \odot w_2 \coloneqq \begin{cases}
     p \odot s, &w_1 = p a \T{and} w_2 = a\star s \text{ for some } a \in \CS, \\
     p \odot s, &w_1 = p a\star \T{and} w_2 = as \T{and} s \neq \star t \text{ for some } a \in \CS, \\
     ps,        &\text{otherwise}.
   \end{cases}
  \end{equation}

  The inverse of the word \( w = a_1 \ldots a_n \) is \( w^{-1} \coloneqq b_1 \ldots b_n \), where
   \begin{equation}\label{eq:def:free_group/inverse}
     b_{n-k+1} \coloneqq \begin{cases}
       \varnothing, &a_k = {\star} \\
       a_k{\star},  &a_k \neq {\star} \T{and} k = n \\
       a_k{\star},  &a_k \neq {\star} \T{and} a_{k+1} \neq {\star} \\
       a_k,         &a_k \neq {\star} \T{and} k \neq n \T{and} a_{k+1} = {\star} \\
     \end{cases}
   \end{equation}
   for \( k = 1, \ldots, n \).

  The group \( (F(\CS), \odot) \) is called the \Def{free group} generated by \( \CS \).
\end{definition}
\begin{proof}
  The proof of the well-definedness of the group structure of \( F(\CS) \) is a straightforward (but tedious) application of induction\IND.
\end{proof}

\begin{proposition}\label{thm:free_group_is_free_functor}
  The functor \( F: \Cat{Set} \to \Cat{Grp} \), defined pointwise in \fullref{def:free_group}, is \hyperref[def:free_functor]{free}.
\end{proposition}
\begin{proof}
  The outline of the proof is similar to the proof of \fullref{thm:free_monoid_is_free_functor}.
\end{proof}

\begin{definition}\label{def:group_presentation}\MarginCite[314]{Knapp2016BAlg}
  Let \( \CS \) be a set, \( F(\CS) \) be the \hyperref[def:free_group]{free group} and \( \CR \subseteq F(\CS) \) be a subset. Denote by \( \CN(\CR) \) the smallest normal subgroup of \( F(\CS) \) that includes \( \CR \) as a subset.

  We define the group
  \begin{equation}\label{eq:def:group_presentation/presentation}
    \CG = \Braket{ \CS \mid \CR} \coloneqq F(\CS) / \CN(\CR)
  \end{equation}
  called the group with \Def{generators} \( \CS \) and \Def{relators} \( \CR \). The expression \eqref{eq:def:group_presentation/presentation} is called a \Def{presentation} of \( \CG \).

  If there exists a presentation for \( \CG \) such that \( \CS \) is finite, it is called a \Def{finitely generated} group. If there exists a presentation such that both \( \CS \) and \( \CR \) are finite, it is called \Def{finitely presented}.

  If \( \CR = \varnothing \), there are no restrictions and we use the notation
  \begin{equation}\label{eq:def:group_presentation/free}
    \CG = \Braket{ \CS } \coloneqq F(\CS)
  \end{equation}
  for the free group.
\end{definition}

\begin{theorem}\label{thm:every_group_is_representable}\MarginCite[prop. 7.7]{Knapp2016BAlg}
  Every group \( \CG \) has at least one \hyperref[def:group_presentation]{presentation}.
\end{theorem}
\begin{proof}
  Let \( \CG \) be an arbitrary group and let \( \CS \coloneqq U(\CG) \) be the underlying set. Let \( F(\CS) \) be the corresponding free group with \( \iota: \CS \to F(\CS) \) sending elements of \( \CS \) to singleton words in \( F(\CS) \). By \fullref{thm:free_group_is_free_functor}, there exists a unique homomorphism \( \varphi: F(\CS) \to \CG \) such that
  \begin{equation*}
    \begin{mplibcode}
      beginfig(1);
      input metapost/graphs;

      v1 := thelabel("$\CS$", origin);
      v2 := thelabel("$U(F(\CS))$", (-1, -1) scaled u);
      v3 := thelabel("$U(G)$", (1, -1) scaled u);

      a1 := straight_arc(v1, v2);
      a2 := straight_arc(v1, v3);

      d1 := straight_arc(v2, v3);

      draw_vertices(v);
      draw_arcs(a);

      drawarrow d1 dotted;

      label.ulft("$\iota$", straight_arc_midpoint of a1);
      label.urt("$\Id$", straight_arc_midpoint of a2);
      label.top("$U(\varphi)$", straight_arc_midpoint of d1);
      endfig;
    \end{mplibcode}
  \end{equation*}
  that is, \( U(\varphi) \circ \iota = \Id \). Thus \( G = \CS \subseteq \ker \varphi \). Define \( \CR \coloneqq \ker \varphi \). By \fullref{def:normal_subgroup}, \( \CR \) is a normal subgroup of \( F(\CS) \), thus
  \begin{equation*}
    G = \varphi(F(\CS)) \cong F(\CS) / \ker \varphi = \Braket{ \CS \mid \CR }.
  \end{equation*}
\end{proof}

\begin{definition}\label{def:cyclic_group}
  For a singleton alphabet \( \{ a \} \), we define the \Def{infinite cyclic group}
  \begin{equation*}
    C \coloneqq \Braket{a}
  \end{equation*}
  and, for positive integers \( n \), the \Def{finite cyclic group} of \Def{order} \( n \) as
  \begin{equation*}
    C_n \coloneqq \Braket{a \mid a^n}.
  \end{equation*}

  We use the same notation independent of \( a \) because all cyclic groups of the same order are obviously \hyperref[def:group/homomorphism]{isomorphic}.

  See \fullref{thm:cyclic_group_isomorphic_to_integers_modulo_n}.
\end{definition}

\begin{definition}\label{def:group_free_product}\MarginCite[323]{Knapp2016BAlg}
  The \Def{free product} of a nonempty family of groups \( \{ \CX_k \}_{k \in \CK} \) with presentations \( \Braket{\CS_k \mid \CR_k}, k \in \CK \) is the group
  \begin{equation*}
    \Ast_{k \in \CK} \CX_k \coloneqq \Braket{ \coprod_{k \in \CK} \CS_k | \coprod_{k \in \CK} \CR_k },
  \end{equation*}
  where \( \coprod \) is the \hyperref[def:disjoint_union]{disjoint union}.
\end{definition}

\begin{definition}\label{def:free_abelian_group}
  A \Def{free abelian group} is a \hyperref[def:free_left_module]{free} \hyperref[thm:abelian_group_iff_z_module]{\( \BZ \)-module}. This definition of a free abelian group is different from the definition of a \hyperref[def:free_group]{free group}.
\end{definition}
