\subsection{Group presentations}\label{subsec:group_presentations}

\begin{definition}\label{def:free_monoid}\MarginCite[306]{Knapp2016BAlg}
  Let \( S \) be an arbitrary set. We associate with \( S \) its \Def{free monoid} \( F(S) \coloneqq (S^{*}, \cdot) \), where \( S^{*} \) is the Kleene star and \( \cdot \) is \hyperref[def:language]{concatenation}. It is a monoid due to \fullref{thm:set_of_all_words_is_monoid}.
\end{definition}

\begin{proposition}\label{thm:free_monoid_is_free_functor}
  The functor \( F: \Cat{Set} \to \Cat{Mon} \), defined pointwise in \fullref{def:free_monoid}, is \hyperref[def:free_functor]{free}.
\end{proposition}
\begin{proof}
  Let \( U: \Cat{Mon} \to \Cat{Set} \) be the corresponding forgetful functor and let \( M \in \Cat{Mon} \), \( S \in \Cat{Set} \). We will first show that
  \begin{equation*}
    \Cat{Mon}(M, F(S)) = \Cat{Set}(U(M), S),
  \end{equation*}
  where equality means that all the underlying functions are equal.

  Every monoid homomorphism is a function so obviously
  \begin{equation*}
    \Cat{Mon}(M, F(S)) \subseteq \Cat{Set}(U(M), S).
  \end{equation*}

  Now consider a function \( f: U(M) \to S \). Denote by \( \iota: S \to F(S) \) the canonical embedding which sends an element of \( S \) to its corresponding singleton word in \( F(S) \). Define the function
  \begin{align*}
     & \varphi: M \to F(S)               \\
     & \varphi(x) \coloneqq \iota(f(x)).
  \end{align*}

  The image \( \varphi(M) \) contains the singleton words of all elements of the image \( f(M) \). In order for \( \varphi \) to be a monoid homomorphism, we only need to define
  \begin{itemize}
    \item multiplication in \( \varphi(M) \) by concatenation, that is,
          \begin{equation*}
            \varphi(x) \varphi(y) \coloneqq \varphi(xy).
          \end{equation*}

    \item identity in \( \varphi(M) \) as the empty word.
  \end{itemize}

  Thus \( \varphi \) is a homomorphism from \( M \) to \( F(S) \). Hence
  \begin{equation*}
    \Cat{Set}(U(M), S) \subseteq \Cat{Mon}(M, F(S)).
  \end{equation*}
\end{proof}

\begin{definition}\label{def:free_group}\MarginCite[306]{Knapp2016BAlg}
  Let \( A \) be an arbitrary set. We associate with \( A \) a \Def{free group} \( F(A) \).

  First, regard \( A \) as an \hyperref[def:language]{alphabet} and define the set \( A^{-1} \) of words of the type \( a^{-1} \), where \( a \in A \) and \( \mbox{}^{-1} \) is a symbol not in \( A \). Consider the language \( W \coloneqq (A \cup A^{-1})^{*} \). If \( w = a_1 \ldots a_n \) is a word, its inverse word is
  \begin{equation*}
    w^{-1} \coloneqq a_n^{-1} \ldots a_1^{-1}.
  \end{equation*}

  By \fullref{thm:set_of_all_words_is_monoid}, \( W \) is a monoid. It is not a group, however, since \( w w^{-1} \neq \varepsilon \). This is why we define a different group operation on a subset of \( W \).

  We define the \Def{reduction function}
  \begin{align*}
     & r: W \to W                               \\
     & r(w) \coloneqq \begin{cases}
      r(ps), & w = pvs, \text{ where } v = aa^{-1} \text{ or } v = a^{-1}a \text{ for some } a \in A, \\
      w,     & \text{otherwise}.
    \end{cases}
  \end{align*}

  The set of \Def{reduced words} is the image \( F(A) \coloneqq \Img r(W) \). It forms a group under the operation
  \begin{equation*}
    u \cdot_{F(A)} w \coloneqq r(u \cdot_{W} w).
  \end{equation*}

  The group \( (F(A), \cdot_{F(A)}) \) is called the \Def{free group} generated by \( A \).
\end{definition}

\begin{proposition}\label{thm:free_group_is_free_functor}
  The functor \( F: \Cat{Set} \to \Cat{Grp} \), defined pointwise in \fullref{def:free_group}, is \hyperref[def:free_functor]{free}.
\end{proposition}
\begin{proof}
  The proof is similar to the proof of \fullref{thm:free_monoid_is_free_functor}, except that the induced group structure requires some consistency proofs.
\end{proof}

\begin{definition}\label{def:group_free_product}\MarginCite[323]{Knapp2016BAlg}
  Let \( \{ X_k \}_{k \in \CK} \) be a nonempty family of groups.

  Similarly to \fullref{def:free_group}, consider the \hyperref[def:language]{language}
  \begin{equation*}
    W \coloneqq \left( \bigcup_{k \in \CK} X_k \right)^{*}.
  \end{equation*}

  We define the \Def{reduction function}
  \begin{align*}
     & r: W \to W                                \\
     & r(w) \coloneqq \begin{cases}
      r(ps),  & w = p e_k s \text{ for some } k \in \CK,                                                                              \\
      r(pts), & w = puvs, \text{ where } u \neq e_k \text{ and } v \neq e_k \text{ and } t = u \cdot_i v \text{ for some } k \in \CK, \\
      w,      & \text{otherwise}.
    \end{cases}
  \end{align*}

  The set of \Def{reduced words} is the image \( \ast_{k \in \CK} X_k \coloneqq \Img r(W) \). It forms a group under the operation
  \begin{equation*}
    u \cdot_\ast w \coloneqq r(u \cdot_{W} w).
  \end{equation*}

  The group \( (\ast_{k \in \CK} X_k, \cdot_\ast) \) is called the \Def{free product} of the family \( \{ X_k \}_{k \in \CK} \).
\end{definition}

\begin{definition}\label{def:free_abelian_group}
  As a special case of \fullref{def:free_left_module}, since abelian groups are \( \BZ \)-modules by \fullref{thm:abelian_group_iff_z_module}, we define \Def{free abelian groups} to be free \( \BZ \)-modules.

  This definition is different from free \hyperref[def:free_group]{groups}.
\end{definition}

\begin{definition}\label{def:group_presentation}\MarginCite[314]{Knapp2016BAlg}
  Let \( S \) be a set, \( F(S) \) be the \hyperref[def:free_group]{free group} and \( R \subseteq F(S) \) be a subset. Denote by \( N(R) \) the smallest normal subgroup of \( F(S) \) that includes \( R \) as a subset.

  We define the group
  \begin{equation}\label{def:group_presentation/presentation}
    G = \Gen{s \in S : r \in R} \coloneqq F(S) / N(R)
  \end{equation}
  called the group with \Def{generators} \( S \) and \Def{relations} \( R \). The expression \fullref{def:group_presentation/presentation} is called a \Def{presentation} of \( G \).

  If there exists a presentation for \( G \) such that \( S \) is finite, it is called a \Def{finitely generated} group. If there exists a presentation such that both \( G \) and \( R \) are finite, it is called \Def{finitely presented}.
\end{definition}

\begin{theorem}\label{thm:every_group_is_representable}\MarginCite[prop. 7.7]{Knapp2016BAlg}
  Every group \( G \) has at least one \hyperref[def:group_presentation]{presentation}.
\end{theorem}
\begin{proof}
  Let \( G \) be an arbitrary group and let \( S \coloneqq U(G) \) be the underlying set. Let \( F(S) \) be the corresponding free group with \( \iota: S \to F(S) \) sending elements of \( S \) to singleton words in \( F(S) \). By \fullref{thm:free_group_is_free_functor}, there exists a unique homomorphism \( \varphi: F(S) \to G \) such that
  \begin{equation*}
    \begin{mplibcode}
      beginfig(1);
      input metapost/graphs;

      v1 := thelabel("$S$", origin);
      v2 := thelabel("$U(F(S))$", (-1, -1) scaled u);
      v3 := thelabel("$U(G)$", (1, -1) scaled u);

      a1 := straight_arc(v1, v2);
      a2 := straight_arc(v1, v3);

      d1 := straight_arc(v2, v3);

      draw_vertices(v);
      draw_arcs(a);

      drawarrow d1 dotted;

      label.ulft("$\iota$", straight_arc_midpoint of a1);
      label.urt("$\Id$", straight_arc_midpoint of a2);
      label.top("$U(\varphi)$", straight_arc_midpoint of d1);
      endfig;
    \end{mplibcode}
  \end{equation*}
  that is, \( U(\varphi) \circ \iota = \Id \). Thus \( G = S \subseteq \ker \varphi \). Define \( R \coloneqq \ker \varphi \). By \fullref{def:normal_subgroup}, \( R \) is a normal subgroup of \( F(S) \), thus
  \begin{equation*}
    G = \varphi(F(S)) \cong F(S) / \ker \varphi = \Gen{ S \colon R }.
  \end{equation*}
\end{proof}
