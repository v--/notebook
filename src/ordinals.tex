\subsection{Ordinals}\label{subsec:ordinals}

\begin{remark}\label{rem:ordinal_definition}
  Ordinals are generalizations of \hyperref[def:set_of_natural_numbers]{natural numbers}. We will find characterizing properties of the natural numbers (defined as members of \hyperref[thm:smallest_inductive_set_existence]{\( \omega \)}) so that it is clear what we want to generalize.

  Every natural number is defined as a set of other natural numbers:
  \begin{align*}
    &0 = \varnothing \\
    &1 = \set{ \varnothing } = \set{ 0 } \\
    &2 = \set{ \varnothing, \set{ \varnothing } } = \set{ 0, 1 } \\
    &3 = \set{ \varnothing, \set{ \varnothing }, \set{ \varnothing, \set{ \varnothing } } } = \set{ 0, 1, 2 }
  \end{align*}

  It just so happens that each natural number \( n \) is the set of natural numbers that are smaller with respect to the strict order relation \( < \) defined in \eqref{eq:def:natural_number_ordering/strict_predicate}.

  Therefore \( \in \) and \( < \) are equivalent on the set \( \omega \). It follows from \fullref{thm:natural_numbers_are_well_ordered} that \( \in \) is a \hyperref[def:totally_ordered_set]{strict total order} on \( \omega \). It is even \hyperref[def:well_ordered_set]{well-ordered} by \( \in \) but the latter condition is redundant due to \fullref{thm:set_membership_is_well_founded}.

  For an arbitrary set \( A \), set membership is not even a \hyperref[def:poset/strict]{strict partial order} --- irreflexivity is implied by \fullref{thm:set_membership_is_well_founded} but transitivity of \( \in \) as a binary relation on \( A \) fails to hold in general, not speaking about trichotomy.

  A very simple counterexample for transitivity of \( \in \) is a \hyperref[def:topological_space]{topological space} \( (\mscrX, \mscrT) \). Any point \( x \) in \( \mscrX \) belongs to some open set \( U \) and every open set belongs to \( \mscrT \), however it is never true (except perhaps by accident) that \( x \in \mscrT \) because \( x \) is a point and \( \mscrT \) only contains sets of points.

  In order for a set \( A \) to be a member of \( \omega \), it is not sufficient for \( \in \) to be a strict total ordered on \( A \). Except for the members of \( \omega \), another set that is totally ordered by \( \in \) is \( A = \set{ 0, 2, 4 } \).

  If we require \( A = \set{ 0, 2, 4 } \) to be a \hyperref[def:transitive_set]{transitive set}, however, it will be a natural number. Indeed, since \( 4 \) is a member of \( A \) and \( 1 \) and \( 3 \) are members of \( 4 \), then by adding \( 1 \) and \( 3 \) to \( A \) we obtain the set \( \set{ 0, 1, 2, 3, 4 } \), which by our definition of natural numbers is \( 5 \).

  Note that transitivity of the relation \( \in \) on \( A \) and transitivity of the set \( A \) itself are entirely different concepts, although we will use both. Every member of \( \omega \) is a transitive set by \fullref{thm:omega_is_transitive} and the relation \( \in \) is a strict total order by \fullref{thm:natural_numbers_are_well_ordered}.

  This is the reasoning behind our definition of an ordinal --- \fullref{def:ordinal}. From this definition it will follow that the ordinals are unique representatives of order-isomorphisms classes of well-ordered sets.

  As a final note, the above two conditions are not sufficient for \( A \) to be a member of \( \omega \) (they are too general) but if we additionally require \( A \) to be a \hyperref[def:infinite_set]{finite set}, then \( A \) will be a member of \( \omega \). We have yet to define finiteness, however.
\end{remark}

\begin{definition}\label{def:well_founded_relation}
  Let \( \prec \) be a \hyperref[def:binary_relation]{binary relation} on a set \( A \) (not necessarily satisfying any axioms).

  An \term{infinitely descending sequence} is a \hyperref[def:sequence]{sequence} \( \seq{ x_k }_{k=1}^\infty \) such that \( x_{k+1} \prec x_k \) for all \( k \in 1, 2, \ldots \). That is,
  \begin{equation*}
    \cdots \prec x_3 \prec x_2 \prec x_1.
  \end{equation*}

  The relation \( \prec \) is called \term{well-founded} if there exists no infinitely descending sequence.

  We cannot easily formulate the theory of well-founded relations as a \hyperref[def:first_order_theory]{first-order theory}. This is why we only study well-relations sets in the context of \logic{ZFC}.
\end{definition}

\begin{proposition}\label{thm:set_membership_is_well_founded}
  Set membership is \hyperref[def:well_founded_relation]{well-founded} in \logic{ZFC}. More precisely, given a set \( A \), if we regard \( \in \) as a binary relation between members of \( A \), then we would obtain that \( \in \) is a well-founded relation.

  This proposition generalizes \fullref{thm:simple_foundation_theorems}.
\end{proposition}
\begin{proof}
  The empty set is vacuously well-founded so suppose that \( A \) is nonempty. Suppose also that \( \in \) is not well-founded on \( A \). Then there exists an infinitely descending sequence \( \seq{ x_k }_{k=1}^\infty \subseteq A \) such that
  \begin{equation*}
    \cdots \in x_3 \in x_2 \in x_1.
  \end{equation*}

  Denote by \( B \) the set \( \set{ x_k \given k = 1, 2, \ldots } \). By the \hyperref[def:zfc/foundation]{axiom of foundation}, \( B \) contains a set \( C \) which is disjoint from \( B \).

  Clearly \( C \) coincides with at least one of \( x_1, x_2, \ldots \). Let \( C = x_{k_0} \). Since \( x_{k_0} \cap B = \varnothing \), then \( x_{k_0 + 1} \) is either not a member of \( x_{k_0} \) or of \( B \). But it is a member of both by our assumption that the sequence is infinitely descending.

  The obtained contradiction proves that \( \in \) is well-founded on \( A \).
\end{proof}

\begin{proposition}\label{thm:infinite_descent_partial_order}
  In a \hyperref[def:poset]{poset} \( (\mscrP, \leq) \), the strict relation \( < \) is \hyperref[def:well_founded_relation]{well-founded} if and only if every nonempty subset \( A \subseteq \mscrP \) has a \hyperref[def:poset_extremal_points/maximal_and_minimal_element]{minimal element}.
\end{proposition}
\begin{proof}
  \SufficiencySubProof Suppose that \( < \) is well-founded and let \( A \subseteq \mscrP \). Suppose that \( A \) has no minimal element.

  If \( A \) is finite, we cannot construct an infinitely descending sequence because of irreflexivity of \( < \). If \( A \) is infinite, then we can construct an infinitely descending sequence using well-founded recursion as follows:
  \begin{itemize}
    \item Let \( x_1 \) be any element of \( A \).
    \item Given \( x_k \), let \( x_{k+1} \) be any element of \( A \) such that \( x_{k+1} < x_k \). We know that such an element exists because \( x_k \) is not minimal in \( A \).
  \end{itemize}

  This construction contradicts the well-foundedness of  \( \leq \), hence \( A \) must have a minimal element.

  \NecessitySubProof Suppose that every subset of \( \mscrP \) has a minimal element. Suppose that there exists an infinitely descending sequence \( \seq{ x_k }_{k=1}^\infty \). Then the set \( \set{ x_k \given x \in \set{ 1, 2, \ldots } } \) has no minimal element, which contradicts our assumption. Hence no sequence in \( \mscrP \) is infinitely descending and \( < \) is well-founded.
\end{proof}

\begin{definition}\label{def:well_ordered_set}\mcite[def. 63.1]{OpenLogicFull}
  A \hyperref[def:totally_ordered_set]{totally ordered set} \( (\mscrP, \leq) \) is said to be \term{well-ordered} if either of the following equivalent conditions hold:
  \begin{thmenum}
    \thmitem{def:well_ordered_set/direct} Every nonempty subset of \( \mscrP \) has a \hyperref[def:poset_extremal_points/maximum_and_minimum]{minimum}.
    \thmitem{def:well_ordered_set/well_founded} The strict order \( < \) is \hyperref[def:well_founded_relation]{well-founded}.
  \end{thmenum}

  The \hyperref[def:binary_relation/irreflexive]{irreflexivity} of \( < \) is redundant because it follows from the well-foundedness --- if the strict relation is not irreflexive, then there exists some element \( x \in \mscrP \) such that \( x < x \) and thus \( \seq{ x }_{k=1}^\infty \) is an infinitely descending sequence.
\end{definition}
\begin{proof}
  The equivalence of the conditions follows from \fullref{thm:infinite_descent_partial_order} and \fullref{thm:totally_ordered_minimal_element_is_minimum}.
\end{proof}

\begin{theorem}[Well-founded induction]\label{thm:well_founded_induction}\mcite[prop. 63.3]{OpenLogicFull}
  Let \( \mscrL \) be the \hyperref[def:first_order_syntax]{first-order language} with no functional symbols and a single predicate symbol \( \prec \). We have already mentioned that we cannot formalize the concept of well-foundedness in first-order logic alone so we will work with \hyperref[def:first_order_structure]{structures} directly.

  Every \hyperref[def:well_founded]{well-founded} structure \( \mscrX = (A, I) \) over \( \mscrL \) satisfies an inductive axiom schema. For every formula \( \varphi \) over \( \mscrL \) not containing \( \eta \) nor \( \zeta \) as free variables, \( \mscrX \) satisfies
  \begin{equation}\label{eq:thm:well_founded_induction}
    \qforall \eta \parens[\Big]
      {
        \overbrace
          {
            \underbrace{ \qforall {\zeta \prec \eta} \varphi[\xi \mapsto \zeta] }_{\mathclap{\substack{\T{inductive} \\ \T{hypothesis}}}}
            \rightarrow
            \underbrace{ \varphi[\xi \mapsto \eta] }_{\mathclap{\substack{\T{inductive step} \\ \T{conclusion}}}}
          }^{\T{inductive step}}
      }
    \rightarrow
    \underbrace{ \qforall \eta \varphi[\xi \mapsto \eta] }_{\T{conclusion}}.
  \end{equation}

  See the comments in \fullref{def:peano_arithmetic/PA3} regarding variables and quantification in axiom schemas.
\end{theorem}
\begin{proof}
  Fix a formula \( \varphi \) in \( \mscrL \). Fix a variable assignment \( v \) in \( \mscrX \). We will show that the \hyperref[def:material_implication/contrapositive]{contraposition} of \eqref{eq:thm:well_founded_induction} holds in this model.

  Suppose that there exists a value \( x \in A \) such that \( \varphi\Bracks{v_{\xi \mapsto x}} = F \). That is,
  \begin{equation*}
    \qexists \eta \neg \varphi[\xi \mapsto \eta].
  \end{equation*}
  holds.

  Let \( x_0 \) be the smallest such value (which is guaranteed to exist because \( A \) is well-founded by \( \prec \)). Thus the inductive hypothesis \( \qforall {\zeta < x_0} \varphi[\xi \mapsto \zeta] \) holds but the inductive step conclusion \( \varphi[\xi \mapsto x_0] \) does not.

  Since \( v \) was chosen arbitrarily, this is true for all variable assignments in \( \mscrX \). Formally,
  \begin{equation}\label{eq:thm:well_founded_induction/contraposition}
    \mscrX
    \vDash
    \qexists \eta \neg \varphi[\xi \mapsto \eta]
    \rightarrow
    \qexists \eta \parens[\Big]
      {
        \qforall {\zeta \prec \eta} \varphi[\xi \mapsto \zeta] \wedge \neg \varphi[\xi \mapsto \eta]
      }
  \end{equation}

  This is precisely the contrapositive of \eqref{eq:thm:well_founded_induction}. Since we are working in classical logic, the contrapositive is semantically equivalent to its original implication, hence \( \mscrX \vDash \eqref{eq:thm:well_founded_induction} \).
\end{proof}

\begin{corollary}[\( \varepsilon \)-induction]\label{thm:epsilon_induction}
  For every formula \( \varphi \) over \( \mscrL \) not containing \( \eta \) nor \( \zeta \) as free variables, the following is a theorem of \logic{ZFC}:
  \begin{equation}\label{eq:thm:epsilon_induction}
    \qforall \eta \parens[\Big]
      {
        \overbrace
          {
            \underbrace{ \qforall {\zeta \in \eta} \varphi[\xi \mapsto \zeta] }_{\mathclap{\substack{\T{inductive} \\ \T{hypothesis}}}}
            \rightarrow
            \underbrace{ \varphi[\xi \mapsto \eta] }_{\mathclap{\substack{\T{inductive step} \\ \T{conclusion}}}}
          }^{\T{inductive step}}
      }
    \rightarrow
    \underbrace{ \qforall \eta \varphi[\xi \mapsto \eta] }_{\T{conclusion}}.
  \end{equation}

  This induction schema is called \enquote{\( \varepsilon \)-induction} because the set membership symbol \( \in \) is derived from \( \varepsilon \) as explained in \fullref{rem:epsilon_and_set_membership}.
\end{corollary}
\begin{proof}
  Every model of \logic{ZFC} is well-founded by \( \in \) as shown in \fullref{thm:set_membership_is_well_founded}. The corollary then follows from \fullref{thm:well_founded_induction}.
\end{proof}

\begin{definition}\label{def:ordinal}\mcite[def. 63.12]{OpenLogicFull}
  An \term{ordinal}, also known as an \term{ordinal number}, is a \hyperref[def:transitive_set]{transitive set} \( A \) such that set membership (as a binary relation on \( A \)) is a \hyperref[def:totally_ordered_set]{strict total order}. In the absence of the \hyperref[def:zfc/foundation]{axiom of foundation}, we require set membership to be a \hyperref[def:well_ordered_set]{well-order} on \( A \).

  See \fullref{rem:ordinal_definition} for a further discussion of the definition, especially the different notions of transitivity.

  By tradition, ordinals are denoted by initial small Greek letters like \( \alpha \) and \( \beta \), however we often use \( \alpha \) and \( \beta \) in logical formulas and so we prefer \( \gamma \) and \( \delta \).

  We introduce the notation \( \gamma < \delta \) for \( \gamma \in \delta \) in analogy with natural numbers. This is not a binary relation since there is not set of all ordinals by \fullref{thm:burali_forti_paradox}, however it does satisfy the properties of a well-order as shown in \fullref{thm:ordinal_properties/trichotomy}.
\end{definition}

\begin{proposition}\label{thm:omega_is_an_ordinal}
  The \hyperref[thm:smallest_inductive_set_existence_existence]{smallest inductive set \( \omega \)} is an \hyperref[def:ordinal]{ordinal}.
\end{proposition}
\begin{proof}
  From \fullref{thm:omega_is_transitive} it follows that \( \omega \) is a transitive set.

  Also, as discussed in \fullref{rem:ordinal_definition}, from \fullref{thm:natural_numbers_are_well_ordered} it follows that set membership is a strict total order on \( \omega \).

  Therefore both \( \omega \) is an \hyperref[def:ordinal]{ordinal}.
\end{proof}

\begin{proposition}\label{thm:element_of_ordinal_is_ordinal}\mcite[lemma 63.13]{OpenLogicFull}
  Every element of an ordinal is an ordinal.
\end{proposition}
\begin{proof}
  Let \( \gamma \) be an ordinal and let \( \delta \in \gamma \). We will show that \( \delta \) is an ordinal.

  By transitivity of \( \gamma \), we have \( \delta \subseteq \gamma \) and by \fullref{def:poset/submodel}, \( (\delta, \in) \) is a (strictly) totally ordered set as a \hyperref[def:first_order_substructure]{substructure} of \( (\gamma, \in) \).

  It remains to show that \( \delta \) is itself transitive. Let \( x \in \delta \). We have that \( \delta \subseteq \gamma \) since \( \gamma \) is transitive, hence \( x \in \gamma \).

  Fix \( y \in x \). Again from the transitivity of \( \gamma \) it follows that \( y \in \gamma \). Since \( \in \) is a total order on \( \gamma \), from \( y \in x \) and \( x \in \delta \) it follows that \( y \in \delta \).

  Since \( y \in x \) was chosen arbitrarily, it follows that \( x \subseteq \delta \). Since \( x \) was chosen arbitrarily, it follows that \( \delta \) is transitive.
\end{proof}

\begin{corollary}\label{thm:natural_numbers_are_ordinals}
  The natural numbers (defined as members of \hyperref[thm:smallest_inductive_set_existence]{\( \omega \)}) are ordinals.
\end{corollary}
\begin{proof}
  Follows from \fullref{thm:omega_is_an_ordinal} and \fullref{thm:element_of_ordinal_is_ordinal}.
\end{proof}

\begin{theorem}[Transfinite induction]\label{thm:transfinite_induction}\mcite[thm. 63.15]{OpenLogicFull}
  Denote by \( \op{isord} \) a formula in \hyperref[def:zfc]{\logic{ZFC}} with a single free variable \( \xi \) which is valid whenever \( \xi \) is an ordinal. Note that we cannot use the test \( \xi \in \cat{Ord} \) within the object language because it is not a member of the universe --- the domain of the category is a proper class by \fullref{thm:burali_forti_paradox}.

  For any formula \( \varphi \) in \logic{ZFC} and for every \( \xi \in \boldop{Free}(\varphi) \) the following formula is valid:
  \begin{multline}\label{eq:def:transfinite_induction}
    \qforall \eta \parens[\Big]
      {
        \op{isord}[\xi \mapsto \eta] \rightarrow
          \underbrace
            {
              \parens[\Big] { \qforall {\zeta \in \eta} \varphi[\xi \mapsto \zeta] }
            }_{\T{inductive hypothesis}}
        \rightarrow
        \overbrace{ \varphi[\xi \mapsto \eta] }^{\mathclap{\T{inductive step conclusion}}}
      }
    \rightarrow \\ \rightarrow
    \underbrace
      {
        \qforall \eta \parens[\Big] { \op{isord}[\xi \mapsto \eta] \rightarrow \varphi[\xi \mapsto \eta] }.
      }_{\T{conclusion}}
  \end{multline}

  In words, if we can prove that \( \varphi \) holds for any ordinal by assuming that it holds for all smaller ordinals, then we can conclude that \( \varphi \) holds simultaneously for all ordinals.

  Also note that this is not a special case of \fullref{thm:well_founded_induction} because there is no set of all ordinals due to \fullref{thm:burali_forti_paradox}.
\end{theorem}
\begin{proof}
  The proof is analogous to the proof of \fullref{thm:well_founded_induction} except that rather than taking elements of a fixed well-ordered set, we take ordinals.
\end{proof}

\begin{theorem}\label{thm:transfinite_recursion}
  \todo{Transfinite recursion}
\end{theorem}

\begin{proposition}\label{thm:ordinal_properties}
  \hyperref[def:ordinal]{Ordinals} have the following basic properties:
  \begin{thmenum}
    \thmitem{thm:ordinal_properties/empty_set} The empty set \( \varnothing \) is an ordinal.

    \thmitem{thm:ordinal_properties/trichotomy}\mcite[thm. 63.16]{OpenLogicFull} The order of ordinals is \hyperref[def:binary_relation/trichotomic]{trichotomic}. That is, if \( \alpha \) and \( \beta \) are ordinals, either \( \alpha < \beta \), \( \beta > \alpha \) or \( \alpha = \beta \).

    \thmitem{thm:ordinal_properties/uniqueness} If two ordinal numbers are \hyperref[def:poset/homomorphism]{order-isomorphic}, they are equal.

    \thmitem{thm:ordinal_properties/descending_sequence_of_ordinals_is_finite}\mcite[prop. 63.20]{OpenLogicFull} Every strictly decreasing sequence of ordinals is finite.

    \thmitem{thm:ordinal_properties/set_of_ordinals_has_minimum} Every nonempty set of ordinals has a minimum.

    \thmitem{thm:ordinal_properties/set_of_smaller_ordinals} Any ordinal \( \alpha \) is equal to the set
    \begin{equation}\label{eq:thm:ordinal_properties/set_of_smaller_ordinals}
      \set{ \beta \in \cat{Ord} \given \beta < \alpha }
    \end{equation}
    of smaller than \( \alpha \) ordinals. As a consequence, this set is well-defined in \logic{ZFC}.

    \thmitem{thm:ordinal_properties/transitive_set_of_ordinals} The set \( A \) is an ordinal if and only if \( A \) is a transitive set of ordinals.
  \end{thmenum}
\end{proposition}
\begin{proof}
  \SubProofOf{thm:ordinal_properties/empty_set} This statement is vacuous since there are not elements nor subsets of \( \varnothing \).

  \SubProofOf{thm:ordinal_properties/trichotomy} Let \( \alpha_0 \) be an ordinal. We will first show that for any other ordinal \( \beta \), either \( \alpha_0 \in \beta \) or \( \beta \subseteq \alpha_0 \). We will use \fullref{thm:transfinite_induction} on \( \beta \).

  Fix an ordinal \( \beta_0 \). Since the inductive hypothesis holds for all \( \beta \in \beta_0 \), we have several possibilities:
  \begin{itemize}
    \item If there exists a \( \beta \in \beta_0 \) such that \( \alpha_0 \in \beta \), then by the transitivity of the \( \in \) relation, we have \( \alpha_0 \in \beta_0 \).
    \item Otherwise, if \( \beta \subseteq \alpha_0 \) for \( \beta \in \beta_0 \), we have that \( \beta_0 \subseteq \alpha_0 \).
  \end{itemize}

  \Fullref{thm:transfinite_induction} allows us to conclude that either \( \alpha_0 \in \beta \) or \( \beta \subseteq \alpha_0 \) independent on the choice of the ordinal \( \beta \). Furthermore, since \( \alpha_0 \) was chosen arbitrarily, this conclusion does not depend on \( \alpha_0 \) and holds for any two ordinals.

  Thus if \( \alpha \) and \( \beta \) are two ordinals, we have the following mutually exclusive possibilities:
  \begin{itemize}
    \item \( \alpha \in \beta \)
    \item \( \beta \in \alpha \)
    \item Neither \( \alpha \in \beta \) nor \( \beta \in \alpha \), in which case by what we have just proved, both \( \alpha \subseteq \beta \) and \( \beta \subseteq \alpha \) hold. Therefore \( \alpha = \beta \).
  \end{itemize}

  This is the desired trichotomy.

  \SubProofOf{thm:ordinal_properties/uniqueness} Let \( \alpha \cong \beta \) be order-isomorphic ordinals. By \fullref{thm:ordinal_properties/trichotomy}, either \( \alpha \in \beta \) or \( \beta \in \alpha \) or \( \alpha = \beta \). To show that the first two cases are impossible, without loss of generality assume that \( \alpha \in \beta \). Let \( h: \alpha \to \beta \) be an embedding. We have that \( h(\alpha) = \beta \) and \( h(\alpha) \in \beta \), hence \( \beta \in \beta \). But this contradicts trichotomy because \( \beta = \beta \).

  Thus we conclude that \( \alpha = \beta \).

  \SubProofOf{thm:ordinal_properties/descending_sequence_of_ordinals_is_finite} Suppose that there exists an infinite sequence of ordinals \( \seq{ \alpha_k }_{k=1}^\infty \) such that \( \alpha_{k+1} \in \alpha_k \) for all \( k \in 1, 2, \ldots \). Thus \( \alpha_1 \) itself as a set has no minimal element with respect to \( \in \), which contradicts the definition of an ordinal.

  The obtained contradiction proves the result.

  \SubProofOf{thm:ordinal_properties/set_of_ordinals_has_minimum} Let \( A \) be a set of ordinals. By \fullref{thm:ordinal_properties/trichotomy}, this set is totally ordered by \( \in \).

  Suppose that a minimum does not exist. Let \( \alpha_1 \in A \). Since \( \alpha_1 \) is not a minimum, there exists \( \alpha_2 \in A \) such that \( \alpha_2 \in \alpha_1 \). We thus build a sequence \( \seq{ \alpha_k }_{k=1}^\infty \subseteq A \) such that \( \alpha_{k+1} \in \alpha_k \) for all \( k \in 1, 2, \ldots \). But this contradicts \fullref{thm:ordinal_properties/descending_sequence_of_ordinals_is_finite}.

  The obtained contradiction shows that \( A \) has a minimum.

  \SubProofOf{thm:ordinal_properties/set_of_smaller_ordinals} Fix an ordinal \( \alpha \). By \fullref{thm:element_of_ordinal_is_ordinal}, every element of \( \alpha \) is again an ordinal and by \fullref{thm:ordinal_properties/trichotomy} all ordinals smaller than \( \alpha \) are elements of \( \alpha \). Therefore \eqref{eq:thm:ordinal_properties/set_of_smaller_ordinals} can be reduced to the set \( \set{ \beta \given \beta \in \alpha } \), which obviously equals \( \alpha \).

  \SubProofOf{thm:ordinal_properties/transitive_set_of_ordinals} If \( A \) is an ordinal, by \fullref{thm:ordinal_properties/set_of_smaller_ordinals}, it is the set of all smaller ordinals. Since \( A \) is transitive by definition, it is a transitive set of ordinals.

  Conversely, suppose that \( A \) is a transitive set of ordinals. We must only show that it is well-ordered by \( \in \). Let \( B \subseteq A \) be a nonempty set. By \fullref{thm:ordinal_properties/set_of_ordinals_has_minimum}, \( B \) has a minimum with respect to \( \in \), hence \( A \) is well-ordered by \( \in \). Since \( A \) is a transitive set, it follows that \( A \) is an ordinal.
\end{proof}

\begin{theorem}[Burali-Forti paradox]\label{thm:burali_forti_paradox}\mcite[thm. 63.19]{OpenLogicFull}
  Assuming \logic{ZFC}, there is no set of all ordinals.
\end{theorem}
\begin{proof}
  Aiming at a contradiction, suppose that \( A \) is a containing all ordinals. If \( \alpha \in A \) and \( \beta \in \alpha \), transitivity of \( \in \) implies that \( \beta \in A \) since \( \beta \) is again an ordinal by \fullref{thm:element_of_ordinal_is_ordinal}. Thus \( A \) is a transitive set of ordinals, which \fullref{thm:ordinal_properties/transitive_set_of_ordinals} is itself an ordinal. Hence \( A \in A \).

  But this contradicts \fullref{thm:zfc_existence_theorems/member_of_itself}. Hence there is no set of all ordinals.
\end{proof}

\begin{proposition}\label{thm:order_type_existence}\mcite[thm. 63.25]{OpenLogicFull}
  Any \hyperref[def:well_ordered_set]{well-ordered set} \( (\mscrP, \leq) \) is \hyperref[def:poset/homomorphism]{order-isomorphic} to a unique ordinal. This ordinal is called the \term{order type} of \( (\mscrP, \leq) \) and is denoted by \( \ord(\mscrP) \).
\end{proposition}
\begin{proof}
  Suppose that \( (\mscrP, \leq) \) is not isomorphic to any ordinal.

  If for at least one element \( x \in P \) the \hyperref[def:poset_interval/ray]{initial segment} \( \mscrP_{<x} \) of \( \mscrP \) is not isomorphic to any ordinal, denote the smallest such element by \( x_0 \) and denote \( \mscrQ \coloneqq \mscrP_{<x_0} \) end endow it with the induced by \( \leq \) order.

  Otherwise, let \( \mscrQ \coloneqq \mscrP \).

  Then any initial segment \( \mscrP_{<x} \) of \( \mscrQ \) is isomorphic to some ordinal \( \alpha_x \), which by \fullref{thm:ordinal_properties/uniqueness} is unique.

  By \hyperref[def:zfc/choice], the following set exists and is a function:
  \begin{equation*}
    f \coloneqq \set{ (\alpha_x, x) \given x \in Q }.
  \end{equation*}

  This is obviously an order isomorphism from \( \alpha \coloneqq \dom(f) \) to \( \mscrQ \).

  In order to show that the set \( \alpha \) is an ordinal, by \fullref{thm:ordinal_properties/transitive_set_of_ordinals} we must only show that it is transitive. Let \( \beta \in \alpha \) and \( \gamma \in \beta \). Then \( \gamma \in \alpha \) since it is the preimage under \( f \) of some element of \( \mscrQ \) smaller than \( f(\beta) \). Thus \( \alpha \) is a transitive set of ordinals and hence is itself an ordinal with \( f \) as the explicit isomorphism.

  Furthermore, if \( \alpha' \) is another ordinal that is isomorphic to \( \mscrQ \) with an explicit isomorphism \( g \), then \( g \bincirc f^{-1} \) is an isomorphism from \( \alpha' \) to \( \alpha \). But by \fullref{thm:ordinal_properties/uniqueness}, isomorphic ordinals are unique and hence \( \alpha = \alpha' \).
\end{proof}
