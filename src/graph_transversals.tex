\subsection{Graph transversals}\label{subsec:graph_transversals}

\begin{definition}\label{def:graph/hypergraph_transversal}\mcite[32]{GondranMinoux1984Graphs}
  Let \( H = (V, E) \) be a \hyperref[def:graph/hypergraph]{hypergraph}. We say that the set \( T \subseteq V \) of vertices is a \term{transversal} of \( H \) if it intersects every edge of \( H \).

  It is also called the transversal of a family of nonempty sets.
\end{definition}

\begin{example}\label{ex:hypergraph_vertex_set_is_transversal}
  Every hypergraph has at least one transversal since the vertex set it itself a transversal.
\end{example}

\begin{definition}\label{def:graph/hypergraph_minimal_transversal}
  A transversal \( T \) of the hypergraph \( H = (V, E) \) is said to be \term{minimal} if any of the following equivalent conditions hold:
  \begin{thmenum}
    \thmitem{def:graph/hypergraph_minimal_transversal/order} \( T \) is a minimal \hyperref[def:partially_ordered_set_extremal_points/maximal_and_minimal_element]{element} under \hyperref[def:subset]{set inclusion} in the set of all transversals of \( H \).
    \thmitem{def:graph/hypergraph_minimal_transversal/singleton} For every vertex \( v \) in \( T \) there exists an edge \( e_v \in E \) such that
    \begin{equation*}
      T \cap e_v = \set{ v }.
    \end{equation*}
  \end{thmenum}
\end{definition}
\begin{proof}
  \ImplicationSubProof{def:graph/hypergraph_minimal_transversal/order}{def:graph/hypergraph_minimal_transversal/singleton} Let \( T \) be minimal under inclusion among all transversals.

  Fix a vertex \( v \in T \). Since \( T \) is minimal, the set \( T \setminus \set{ v } \) is not a transversal. So there exists an edge \( e_v \in E \) such that \( (T \setminus \set{ v }) \cap e_v = \varnothing \).

  Now since \( T \) is a transversal for \( H \), the set \( T \cap e_v \) is nonempty and thus
  \begin{equation*}
    T \cap e_x
    =
    \parens[\Big]{ (T \setminus \set{ v }) \cup \set{ v } } \cap e_v
    \reloset {\eqref{eq:thm:de_morgans_laws/complement_of_meet}} =
    \underbrace{\parens[\Big]{ (T \setminus \set{ v }) \cap e_v }}_{\varnothing} \cup \parens[\Big]{ \set{ v } \cap e_v }
    =
    \set{ v }.
  \end{equation*}

  \ImplicationSubProof{def:graph/hypergraph_minimal_transversal/singleton}{def:graph/hypergraph_minimal_transversal/order} Now suppose that for every vertex \( v \in T \) there exists an edge \( e_v \in E \) such that \( T \cap e_v = \set{ v } \).

  Suppose that \( T \) is not minimal. Then there exists some vertex \( u \in T \) be such that \( T \setminus \set{ u } \) is a transversal. But our assertion gives us an edge \( e_u \in E \) such that \( T \cap e_u = \set{ u } \). Clearly the set \( T \setminus \set{ u } \) cannot be a transversal of \( H \) since
  \begin{equation*}
    (T \setminus \set{ u }) \cap e_u = \varnothing.
  \end{equation*}

  This contradiction proves that \( T \) is minimal under set inclusion.
\end{proof}

\begin{example}\label{ex:hypergraph_with_no_minimal_transversal}
  We will give an example of a hypergraph without a minimal transversal.

  For every nonnegative integer \( n \) define the set
  \begin{equation*}
    e_n \coloneqq \set{ n, n + 1, n + 2, \ldots }.
  \end{equation*}

  Let \( H \) be the hypergraph whose edges are \( e_1, e_2, \cdots \). The set of vertices is \( e_1 \), hence \( e_1 \) is also a transversal.

  Now assume that \( T \) is a minimal transversal for \( H \). Since, by \fullref{thm:natural_numbers_are_well_ordered}, the natural numbers are well-ordered, \( T \) has a minimum. Let \( n_0 \coloneqq \min T \).

  But \( T \setminus \set{ n_0 } \) is also a transversal because each edge \( e_n \) intersects \( T \) at infinitely many points besides \( n_0 \).

  The obtained contradiction shows that \( H \) has no minimal transversal.
\end{example}

\begin{theorem}[Hypergraph minimal transversal existence]\label{thm:hypergraphs_have_minimal_transversal}
  Every \hyperref[def:graph/hypergraph]{hypergraph} has a \hyperref[def:graph/hypergraph_minimal_transversal]{minimal transversal}.

  In \hyperref[def:zfc]{\logic{ZF}} this theorem is equivalent to the \hyperref[def:zfc/choice]{axiom of choice} --- see \fullref{thm:axiom_of_choice_equivalences/hypergraph}.
\end{theorem}
\begin{proof}
  The proof is merely a translation of the axiom of choice into the language of hypergraphs.

  \ImplicationSubProof[def:zfc/choice]{the axiom of choice}[thm:hypergraphs_have_minimal_transversal]{minimal transversal existence} Let \( H = (V, E) \) be a hypergraph. Then \( E \) is a family of nonempty sets and thus there exists a set \( B \) such that \( e \cap B \) for every edge \( e \in E \). This is a minimal transversal by \fullref{def:graph/hypergraph_minimal_transversal/singleton}.

  \ImplicationSubProof[thm:hypergraphs_have_minimal_transversal]{minimal transversal existence}[def:zfc/choice]{axiom of choice} Let \( \mscrA \) be an arbitrary family of nonempty sets. Then \( H \coloneqq \parens*{ \bigcup \mscrA, \mscrA } \) is a hypergraph. Let \( T \) be a minimal transversal of \( H \). Then, by definition, for every edge \( e \in \mscrA \), the intersection \( T \cap e \) is a singleton set. Hence, \( T \) is the image of a \hyperref[def:choice_function]{choice function} for \( \mscrA \).
\end{proof}
