\section{Commutative algebra}\label{sec:commutative_algebra}

\begin{remark}\label{remark:polynomial_commutative_ring}
  In this whole section, \( R \) will refer to a nontrivial commutative unital ring\Tinyref{def:semiring/commutative_unital_ring}.
\end{remark}

\subsection{Prime ideals}\label{subsec:prime_ideals}

\begin{definition}\label{def:prime_ring_ideal}\cite[384]{Knapp2016BAlg}
  An ideal \( P \) is called \Def{prime} if it is proper and satisfies any of the equivalent conditions:
  \begin{defenum}
    \DItem{def:prime_ring_ideal/direct} If \( x, y \in R \) are such that \( xy \in P \), then either \( x \in P \) or \( y \in P \).
    \DItem{def:prime_ring_ideal/ideals} If \( I, J \subseteq R \) are ideals such that \( IJ \subseteq P \), then either \( I \subseteq P \) or \( J \subseteq P \).
    \DItem{def:prime_ring_ideal/quotient} The quotient \( R / P \) is an integral domain.
  \end{defenum}

  An element \( r \in R \) is called \Def{prime} if the ideal \( \Gen r \) is prime.
\end{definition}
\begin{proof}
  \Implies[def:prime_ring_ideal/direct][def:prime_ring_ideal/ideals] Fix ideals \( I, J \) of \( R \) such that \( IJ \subseteq P \).

  Assume\LEM that neither \( I \not\subseteq P \) nor \( J \not\subseteq P \). Take \( x \in I \setminus P \) and \( y \in J \setminus P \). It follows that \( xy \in P \) and either \( x \in P \) or \( y \in P \). This contradicts our assumption.

  The obtained contradiction proves that either \( I \subseteq P \) or \( J \subseteq P \).

  \Implies[def:prime_ring_ideal/ideals][def:prime_ring_ideal/quotient] Fix an ideal \( P \) such that if \( I, J \subseteq R \) are ideals and \( IJ \subseteq P \), then either \( I \subseteq P \) or \( J \subseteq P \).

  We will prove that \( R / P \) is an integral domain. If \( R \) is an integral domain, this is obvious. If not, we fix nonzero \( x, y \in R \) so that \( xy = 0 \). Thus \( [x][y] = (x + P)(y + P) = xy + P = P = [0] \). We will show that either \( x = 0 \) or \( y = 0 \).

  Consider the ideals
  \begin{align*}
    \Gen{x} &= xR, \\
    \Gen{y} &= yR.
  \end{align*}

  By \cref{thm:product_of_principal_ideals}, we have \( \Gen{x} \Gen{y} = \Gen{xy} = \Gen{0} = \{ 0 \} \).

  Since \( \Gen{x} \Gen{y} \subseteq P \), then either \( \Gen{x} \subseteq P \) or \( \Gen{y} \subseteq P \). That is, either \( [x] = 0 \) or \( [y] = 0 \).

  Thus \( R / P \) is an integral domain.

  \Implies[def:prime_ring_ideal/quotient][def:prime_ring_ideal/direct] Suppose that \( R / P \) is an integral domain. Fix \( x, y \in R \) so that \( xy \in P \). If \( x = 0 \), obviously \( x = 0 \in R \) and similarly for \( y \). Suppose that both \( x \) and \( y \) are nonzero. We will show that either \( x \in P \) or \( y \in P \).

  We have
  \begin{equation*}
    [x][y] = [xy] = xy + P = P = [0].
  \end{equation*}

  Since \( R / P \) is an integral domain, either \( [x] = [0] \) or \( [y] = [0] \). That is, either \( x \in P \) or \( y \in P \).
\end{proof}

\begin{definition}\label{def:irreducible_ring_element}\cite[388]{Knapp2016BAlg}
  A nonzero element \( r \in R \) of an integral domain is called \Def{reducible} if there exist non-invertible elements \( r_1, r_2 \in R \) such that
  \begin{equation*}
    r = r_1 r_2.
  \end{equation*}

  If \( r \) is not reducible, we say that it is\Def{irreducible}.
\end{definition}

\begin{definition}\label{def:coprime_ring_ideals}
  Two ring ideals \( I \subseteq R \) and \( J \subseteq R \) are said to be \Def{coprime} if \( I + J = R \).
\end{definition}

\begin{proposition}\label{thm:prime_implies_irreducible}\cite[389]{Knapp2016BAlg}
  All prime\Tinyref{def:prime_ring_ideal} elements in an integral domain are irreducible\Tinyref{def:irreducible_ring_element}.
\end{proposition}
\begin{proof}
  Let \( p \) be prime. Assume\LEM that \( p \) is reducible, that is, there exist non-invertible elements \( r_1, r_2 \in R \) such that
  \begin{equation*}
    p = r_1 r_2.
  \end{equation*}

  Since \( p \) is prime, it must divide either \( r_1 \) or \( r_2 \). Without loss of generality, assume that \( p | r_1 \) and \( r_1 = pc \) for some \( c \in R \).

  Then \( p = r_1 r_2 = pc r_2 \). By \cref{thm:semiring_properties/cancellable_iff_not_zero_divisor}, \( 1 = c r_2 \), which implies that \( r_2 \) is invertible with inverse \( c \). This contradicts our assumption that both \( r_1 \) and \( r_2 \) are invertible.

  The obtained contradiction proves that \( p \) is irreducible.
\end{proof}

\begin{definition}\label{def:maximal_ring_ideal}
  A two-sided ideal \( M \) is called \Def{maximal} if it is proper and satisfies any of the equivalent conditions:
  \begin{defenum}
    \DItem{def:maximal_ring_ideal/maximality} \( M \) is maximal with respect to set inclusion among proper two-sided ideals.
    \DItem{def:maximal_ring_ideal/quotient} The quotient \( R / M \) is a field.
  \end{defenum}
\end{definition}
\begin{proof}
  \Implies[def:maximal_ring_ideal/maximality][def:maximal_ring_ideal/quotient] Suppose that \( M \) is maximal among proper ideals. We will prove that every nonzero element of \( R / M \) is invertible.

  Fix \( x \not\in M \) so that \( [x] = x + M \neq M = [0] \). Define the set
  \begin{equation*}
    I \coloneqq Rx + M.
  \end{equation*}

  It is a ideal since both \( Rx \) and \( M \) are ideals. Furthermore, it contains \( M \) strictly because \( M \subseteq I \) and \( x \in I \). Since \( M \) is maximal, we have that \( I = R \).

  Hence there exists \( y \in R \) such that \( 1 = yx + M \). Hence \( [y] = y + M \) is an inverse of \( [x] \) in \( R / M \).

  Since \( [x] \in R / M \) was an arbitrary nonzero element, we conclude that \( R / M \) is a field.

  \Implies[def:maximal_ring_ideal/quotient][def:maximal_ring_ideal/maximality] Suppose that \( R / M \) is a field. Assume that \( M \) is not maximal. Then there exists a proper ideal \( I \supsetneq M \).

  Assume that \( I \neq M \) and take \( x \in I \setminus M \). Then \( x \not\in M \) and hence \( [x] \neq [0] \) and is invertible in \( R / M \). Denote by \( y \) any representative of this inverse. Thus \( [xy] - [1] = [0] \), that is, \( xy - 1 \in M \).

  Note that \( xy \in I \) because \( x \in I \) and \( y \in R \). Since \( I \) is closed under addition, it follows that \( 1 \in I \) and hence \( I = R \). But this contradicts our assumption that \( I \) is proper.

  The obtained contradiction proves that \( M \) is maximal.
\end{proof}

\begin{proposition}\label{thm:maximal_ideals_are_prime}
  Maximal ring ideals\Tinyref{def:maximal_ring_ideal} are prime\Tinyref{def:prime_ring_ideal}.
\end{proposition}
\begin{proof}
  If \( M \) is a maximal ideal of \( R \), by \cref{def:maximal_ring_ideal/quotient} \( R / M \) is a field. Thus \( R / M \) is an integral domain, which by \cref{def:prime_ring_ideal/quotient} means that \( M \) is a prime ideal.
\end{proof}

\begin{proposition}\label{thm:pid_prime_iff_irreducible}
  An element in a principal ideal domain is prime\Tinyref{def:prime_ring_ideal} if and only if it is irreducible\Tinyref{def:irreducible_ring_element}.
\end{proposition}
\begin{proof}
  \begin{description}
    \Implies Follows from \cref{thm:prime_implies_irreducible}.

    \ImpliedBy Let \( r \) be an irreducible element. Assume\LEM that \( \Gen r \) is not a maximal ideal. Then it is contained an another proper ideal, say \( I \).

    Since \( R \) is a principal ideal domain, \( I \) can be generated by only one element, say \( I = \Gen p \). Since \( r \in I \), there exists \( q \in R \) such that
    \begin{equation*}
      r = qp.
    \end{equation*}

    But \( r \) is irreducible, hence either \( p \) or \( q \) must be invertible. \( p \) cannot be invertible since \( I = \Gen p \) is a proper ideal. Thus \( q \) is invertible. Thus
    \begin{equation*}
      p = q^{-1} r
    \end{equation*}
    and \( p \in \Gen r \), that is, \( \Gen r = \Gen p = I \). This contradicts our assumption that \( \Gen r \) is not maximal.

    The obtained contradiction shows that \( \Gen r \) is a maximal, and therefore by \cref{thm:maximal_ideals_are_prime} prime, ideal.
  \end{description}
\end{proof}

\begin{theorem}[Chinese remainder theorem]\label{thm:chinese_remained_theorem}\cite[theorem 8.27]{Knapp2016BAlg}
  Let \( I_1, \ldots, I_n \) be pairwise coprime\Tinyref{def:coprime_ring_ideals} ideals. Then
  \begin{equation*}
    R / \bigcap_{i=1}^n I_n \cong R / I_1 \times \cdots \times R / I_n.
  \end{equation*}
\end{theorem}
