\subsection{Free groups}\label{subsec:free_groups}

\begin{definition}\label{def:free_monoid}
  Let \( S \) be an arbitrary set. We associate with \( S \) its \term{free monoid} \( F(S) \coloneqq (S^{\ast}, \cdot) \), where \( S^{\ast} \) is the \hyperref[def:formal_language/kleene_star]{Kleene star} and \( \cdot \) is \hyperref[def:formal_language/concatenation]{concatenation}.
\end{definition}
\begin{proof}
  It is a monoid due to \fullref{thm:kleene_star_is_monoid}.
\end{proof}

\begin{proposition}\label{thm:free_monoid_universal_property}
  For every set \( A \), denote by \( \iota_A: A \to F(A) \) the canonical inclusion function, which sends every member of \( A \) into the corresponding single-symbol word in the \hyperref[def:free_monoid]{free monoid} \( F(A) \).

  As a consequence of \fullref{thm:free_monoid_universal_property}, \( F(A) \) is the unique up to an isomorphism monoid that satisfies the following \hyperref[rem:universal_mapping_property]{universal mapping property}:
  \begin{displayquote}
    For every monoid \( \mscrM \) and every function \( f: A \to \mscrM \), there exists a unique monoid homomorphism \( \widetilde{f}: F(A) \to \mscrM \) such that the following diagram commutes:
    \begin{equation}\label{eq:def:free_monoid/diagram}
      \begin{aligned}
        \includegraphics[page=1]{output/def__free_monoid.pdf}
      \end{aligned}
    \end{equation}
  \end{displayquote}
\end{proposition}
\begin{proof}
  For every function \( f: A \to \mscrM \), we have the monoid homomorphism
  \begin{equation*}
    \begin{aligned}
      &\widetilde{f}: F(A) \to \mscrM, \\
      &\widetilde{f}(x_1 x_2 \ldots x_n) \coloneqq f(x_1) \cdot f(x_2) \cdot \ldots \cdot f(x_n)
    \end{aligned}
  \end{equation*}
  obtained by applying the monoid operation \( \cdot \) recursively to the pointwise image
  \begin{equation*}
    f(x_1) f(x_2) \ldots f(x_n)
  \end{equation*}
  of the word
  \begin{equation*}
    x_1 x_2 \ldots x_n.
  \end{equation*}

  The homomorphism \( \widetilde{f} \) is uniquely determined by the action of \( f \) on single-symbol words.
\end{proof}

\begin{corollary}\label{thm:free_monoid_functor}
  Consider the functor \( F: \cat{Set} \to \cat{Mon} \) defined for objects pointwise in \fullref{def:free_monoid}. For every function \( f: A \to B \), define the \hyperref[def:unital_magma/homomorphism]{monoid homomorphism}
  \begin{equation*}
    \begin{aligned}
      &F(f): F(A) \to F(B) \\
      &F(f)(x_1 \cdots x_n) \coloneqq f(x_1) \cdots f(x_n).
    \end{aligned}
  \end{equation*}

  This functor is \hyperref[def:category_adjunction]{left adjoint} to the \hyperref[def:concrete_category]{forgetful functor} \( U: \cat{Mon} \to \cat{Set} \).
\end{corollary}
\begin{proof}
  Follows from \fullref{thm:free_monoid_universal_property} via \fullref{rem:universal_mapping_property}.
\end{proof}

\begin{definition}\label{def:free_group}
  Let \( S \) be an arbitrary set. We will now construct the \term{free group} \( F(S) \) of \( S \). The construction is similar to that of \hyperref[def:free_monoid]{free monoids}, but it is much more complicated because of special reduction rules for \hyperref[def:unital_magma_inverse_element]{inverse elements}. Refer to \cite{code:free_group_grammar_verification} for a software implementation of the construction.

  Let \( \star \) be a \hyperref[def:formal_language/symbol]{symbol} not in \( S \). Our goal is, for each \( a \in S \), to make the word \( a{\star} \) behave like the inverse of \( a \) in a group. Rather than considering the \hyperref[def:formal_language/kleene_star]{Kleene star} \( (S \cup \{ \star \})^* \) and removing elements via \enquote{reductions} as in \cite{code:free_group_reduction_verification} and \cite[306]{Knapp2016BasicAlgebra}, we directly build a language of \term{reduced words} using the mutually recursive \hyperref[def:formal_grammar]{grammar}
  \begin{alignedeq}\label{eq:def:free_group/grammar}
    &I \to \varepsilon,           &&                        && \text{\( I \) is the initial state} \\
    &I \to S_a \mid D_a,             && a \in S              && \\
    &S_a \to a \mid a S_a,           && a \in S              && S_a \text{ does not produce words beginning with } a\star \\
    &S_a \to a D_b,               && a, b \in S, a \neq b && \\
    &D_a \to a\star S_b,          && a, b \in S, a \neq b && D_a \text{ does not produce words beginning with } a \\
    &D_a \to a\star \mid a\star D_a, && a \in S              && \\
  \end{alignedeq}

  The \term{free group} \( F(S) \) is defined to be the language of \eqref{eq:def:free_group/grammar} equipped with the inductively defined operation
  \begin{equation}\label{eq:def:free_group/operation}
    w_1 \odot w_2 \coloneqq \begin{cases}
     p \odot s, &w_1 = p a \T{and} w_2 = a\star s \text{ for some } a \in S, \\
     p \odot s, &w_1 = p a\star \T{and} w_2 = as \T{and} s \neq \star t \text{ for some } a \in S, \\
     ps,        &\text{otherwise}.
   \end{cases}
  \end{equation}

  The inverse of the word \( w = a_1 \ldots a_n \) is \( w^{-1} \coloneqq b_1 \ldots b_n \), where
   \begin{equation}\label{eq:def:free_group/inverse}
     b_{n-k+1} \coloneqq \begin{cases}
       \varnothing, &a_k = {\star} \\
       a_k{\star},  &a_k \neq {\star} \T{and} k = n \\
       a_k{\star},  &a_k \neq {\star} \T{and} a_{k+1} \neq {\star} \\
       a_k,         &a_k \neq {\star} \T{and} k \neq n \T{and} a_{k+1} = {\star} \\
     \end{cases}
   \end{equation}
   for \( k = 1, \ldots, n \).

  The group \( (F(S), \odot) \) is called the \term{free group} generated by \( S \).
\end{definition}
\begin{proof}
  The proof of the well-definedness of the group structure of \( F(S) \) is a straightforward (but tedious) application of induction.
\end{proof}

\begin{proposition}\label{thm:free_group_is_free_functor}
  The functor \( F: \cat{Set} \to \cat{Grp} \), defined pointwise in \fullref{def:free_group}, is \hyperref[def:category_adjunction]{free}.
\end{proposition}
\begin{proof}
  The outline of the proof is similar to the proof of \fullref{thm:free_monoid_universal_property}.
\end{proof}

\begin{definition}\label{def:group_presentation}\mcite[314]{Knapp2016BasicAlgebra}
  Let \( S \) be a set, \( F(S) \) be the \hyperref[def:free_group]{free group} and \( \mscrR \subseteq F(S) \) be a subset. Denote by \( \mscrN(\mscrR) \) the smallest normal subgroup of \( F(S) \) that includes \( \mscrR \) as a subset.

  We define the group
  \begin{equation}\label{eq:def:group_presentation/presentation}
    \mscrG = \braket{ S \mid \mscrR} \coloneqq F(S) / \mscrN(\mscrR)
  \end{equation}
  called the group with \term{generators} \( S \) and \term{relators} \( \mscrR \). The expression \eqref{eq:def:group_presentation/presentation} is called a \term{presentation} of \( \mscrG \).

  If there exists a presentation for \( \mscrG \) such that \( S \) is finite, it is called a \term{finitely generated} group. If there exists a presentation such that both \( S \) and \( \mscrR \) are finite, it is called \term{finitely presented}.

  If \( \mscrR = \varnothing \), there are no restrictions and we use the notation
  \begin{equation}\label{eq:def:group_presentation/free}
    \mscrG = \braket{ S } \coloneqq F(S)
  \end{equation}
  for the free group.
\end{definition}

\begin{theorem}\label{thm:every_group_is_representable}\mcite[prop. 7.7]{Knapp2016BasicAlgebra}
  Every group \( \mscrG \) has at least one \hyperref[def:group_presentation]{presentation}.
\end{theorem}
\begin{proof}
  Let \( \mscrG \) be an arbitrary group and let \( S \coloneqq U(\mscrG) \) be the underlying set. Let \( F(S) \) be the corresponding free group with \( \iota: S \to F(S) \) sending elements of \( S \) to singleton words in \( F(S) \). By \fullref{thm:free_group_is_free_functor}, there exists a unique homomorphism \( \varphi: F(S) \to \mscrG \) such that
  \begin{equation*}
    \text{\todo{Add diagram}}\iffalse\begin{mplibcode}
      beginfig(1);
      input metapost/graphs;

      v1 := thelabel("$S$", origin);
      v2 := thelabel("$U(F(S))$", (-1, -1) scaled u);
      v3 := thelabel("$U(G)$", (1, -1) scaled u);

      a1 := straight_arc(v1, v2);
      a2 := straight_arc(v1, v3);

      d1 := straight_arc(v2, v3);

      draw_vertices(v);
      draw_arcs(a);

      drawarrow d1 dotted;

      label.ulft("$\iota$", straight_arc_midpoint of a1);
      label.urt("$\id$", straight_arc_midpoint of a2);
      label.top("$U(\varphi)$", straight_arc_midpoint of d1);
      endfig;
    \end{mplibcode}\fi
  \end{equation*}
  that is, \( U(\varphi) \circ \iota = \id \). Thus, \( G = S \subseteq \ker \varphi \). Define \( \mscrR \coloneqq \ker \varphi \). By \fullref{def:normal_subgroup}, \( \mscrR \) is a normal subgroup of \( F(S) \), thus
  \begin{equation*}
    G = \varphi(F(S)) \cong F(S) / \ker \varphi = \braket{ S \mid \mscrR }.
  \end{equation*}
\end{proof}

\begin{definition}\label{def:cyclic_group}
  For a singleton alphabet \( \set{ a } \), we define the \term{infinite cyclic group}
  \begin{equation*}
    C \coloneqq \braket{a}
  \end{equation*}
  and, for positive integers \( n \), the \term{finite cyclic group} of \term{order} \( n \) as
  \begin{equation*}
    C_n \coloneqq \braket{a \given a^n}.
  \end{equation*}

  We use the same notation independent of \( a \) because all cyclic groups of the same order are obviously \hyperref[def:group/homomorphism]{isomorphic}.

  See \fullref{thm:cyclic_group_isomorphic_to_integers_modulo_n}.
\end{definition}

\begin{definition}\label{def:group_free_product}\mcite[323]{Knapp2016BasicAlgebra}
  The \term{free product} of a nonempty family of groups \( \seq{ \mscrX_k }_{k \in \mscrK} \) with presentations \( \braket{S_k \mid \mscrR_k}, k \in \mscrK \) is the group
  \begin{equation*}
    \Ast_{k \in \mscrK} \mscrX_k \coloneqq \braket*{ \coprod_{k \in \mscrK} S_k \given* \coprod_{k \in \mscrK} \mscrR_k },
  \end{equation*}
  where \( \coprod \) is the \hyperref[def:disjoint_union]{disjoint union}.
\end{definition}

\begin{definition}\label{def:free_abelian_group}
  A \term{free abelian group} is a \hyperref[def:free_left_module]{free} \hyperref[thm:abelian_group_iff_z_module]{\( \BbbZ \)-module}. This definition of a free abelian group is different from the definition of a \hyperref[def:free_group]{free group}.
\end{definition}

\begin{proposition}\label{thm:product_of_cyclic_groups}
  The \hyperref[def:group_direct_product]{direct product} \( C_n \times C_m \) of two \hyperref[def:cyclic_group]{cyclic groups} is cyclic if and only if \( n \) and \( m \) are \hyperref[def:coprime_numbers]{coprime}.
\end{proposition}
\begin{proof}
  Take \( (a^i, a^j) \in C_n \times C_m \).
\end{proof}

\begin{definition}\label{def:group_direct_product}
  The \term{direct product} of a nonempty family of groups \( \{ \mscrX_k \}_{k \in \mscrK} \) is their \hyperref[def:cartesian_product]{Cartesian product} \( \prod_{k \in \mscrK} \mscrX_k \) with the componentwise group operation
  \begin{equation*}
    \{ x_k \}_{k \in \mscrK} \cdot \{ y_k \}_{k \in \mscrK}
    \coloneqq
    \{ x_k \cdot y_k \}_{k \in \mscrK}.
  \end{equation*}
\end{definition}

\begin{definition}\label{def:group_direct_sum}
  The \term{direct sum} \( \bigoplus_{k \in \mscrK} \mscrX_k \) of a nonempty family of groups \( \{ \mscrX_k \}_{k \in \mscrK} \) is a subgroup of their \hyperref[def:group_direct_sum]{direct product} where, for any group element \( \{ x_k \}_{k \in \mscrK} \), only finitely many components are different from zero.

  \begin{thmenum}
    \thmitem{def:group_direct_sum/internal}\mcite[126]{Knapp2016BasicAlgebra}If all \( \{ \mscrX_k \}_{k \in \mscrK} \) are subgroups of a group \( \mscrX \), we say that \( \mscrX \) is their \term{internal direct sum} if the homomorphism
    \begin{align*}
       &\varphi: \bigoplus_{k \in \mscrK} \mscrX_k \to X \\
       &\varphi(\{ x_k \}_{k \in \mscrK}) \coloneqq \cdot_{k \in \mscrK} x_k
    \end{align*}
    is an isomorphism.

    The sum is well-defined since, by definition, there are only finitely many non-identity summands.

    \thmitem{def:group_direct_sum/external} To distinguish \( \bigoplus_{k \in \mscrK} \mscrX_k \) from \( X \), we sometimes call it the \term{external direct product}.
  \end{thmenum}
\end{definition}

\begin{proposition}\label{thm:group_categorical_limits}
  We are interested in \hyperref[def:category_of_cones/limit]{categorical limits} and \hyperref[def:category_of_cones/colimit]{colimits} in \( \cat{Grp} \). Fix an indexed family  \( \{ \mscrX_k \}_{k \in \mscrK} \) of groups.

  \begin{thmenum}
    \thmitem{thm:group_categorical_limits/product} Their \hyperref[def:discrete_category_limits]{categorical product} is their \hyperref[def:group_direct_product]{direct product} \( \prod_{k \in \mscrK} \mscrX_k \), the projection morphisms being inherited from \fullref{thm:discrete_category_limits_in_set}.

    \thmitem{thm:group_categorical_limits/coproduct} Their \hyperref[def:discrete_category_limits]{categorical coproduct} is their \hyperref[def:group_free_product]{free product} \( \Ast_{k \in \mscrK} \mscrX_k \), the embedding morphisms being
    \begin{balign*}
       &\iota_m: \mscrX_m \to \Ast_{k \in \mscrK} \mscrX_k \\
       &\iota_m(x_m) \coloneqq x_m.
    \end{balign*}
  \end{thmenum}
\end{proposition}
