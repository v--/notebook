\subsection{Free groups}\label{subsec:free_groups}

\begin{definition}\label{def:free_monoid}
  We associate with every set \( A \) its \term{free monoid} \( F(A) \coloneqq (A^*, \cdot) \), where \( A^* \) is the \hyperref[def:formal_language/kleene_star]{Kleene star} and \( \cdot \) is \hyperref[def:formal_language/concatenation]{concatenation}.
\end{definition}
\begin{proof}
  It is a monoid due to \fullref{thm:kleene_star_is_monoid}.
\end{proof}

\begin{proposition}\label{thm:free_monoid_universal_property}
  For every set \( A \), denote by \( \iota_A: A \to F(A) \) the canonical inclusion function, which sends every member of \( A \) into the corresponding single-symbol word in the \hyperref[def:free_monoid]{free monoid} \( F(A) \).

  Then \( F(A) \) is the unique up to an isomorphism monoid that satisfies the following \hyperref[rem:universal_mapping_property]{universal mapping property}:
  \begin{displayquote}
    For every monoid \( M \) and every function \( f: A \to M \), there exists a unique monoid homomorphism \( \widetilde{f}: F(A) \to M \) such that the following diagram commutes:
    \begin{equation}\label{eq:thm:free_monoid_universal_property/diagram}
      \begin{aligned}
        \includegraphics[page=1]{output/thm__free_monoid_universal_property.pdf}
      \end{aligned}
    \end{equation}
  \end{displayquote}
\end{proposition}
\begin{proof}
  For every function \( f: A \to M \), define the monoid homomorphism
  \begin{equation*}
    \begin{aligned}
      &\widetilde{f}: F(A) \to M, \\
      &\widetilde{f}(x_1 x_2 \ldots x_n) \coloneqq f(x_1) \cdot f(x_2) \cdot \ldots \cdot f(x_n)
    \end{aligned}
  \end{equation*}
  obtained by applying the monoid operation \( \cdot \) recursively to the pointwise image
  \begin{equation*}
    f(x_1) f(x_2) \ldots f(x_n)
  \end{equation*}
  of the word
  \begin{equation*}
    x_1 x_2 \ldots x_n.
  \end{equation*}

  The homomorphism \( \widetilde{f} \) is uniquely determined by the action of \( f \) on single-symbol words.
\end{proof}

\begin{corollary}\label{thm:free_monoid_functor}
  Consider the functor \( F: \cat{Set} \to \cat{Mon} \) defined for objects pointwise in \fullref{def:free_monoid}. For every function \( f: A \to B \), define the \hyperref[def:unital_magma/homomorphism]{monoid homomorphism}
  \begin{equation*}
    \begin{aligned}
      &F(f): F(A) \to F(B) \\
      &F(f)(x_1 x_2 \ldots x_n) \coloneqq f(x_1) f(x_2) \ldots f(x_n).
    \end{aligned}
  \end{equation*}

  This functor is \hyperref[def:category_adjunction]{left adjoint} to the \hyperref[def:concrete_category]{forgetful functor} \( U: \cat{Mon} \to \cat{Set} \).
\end{corollary}
\begin{proof}
  Follows from \fullref{thm:free_monoid_universal_property} via \fullref{rem:universal_mapping_property}.
\end{proof}

\begin{definition}\label{def:abstract_reduction_system}\mcite[def. 1.1.2]{Book1993}
  Fix an arbitrary \hyperref[def:set]{set} \( A \) and an \hyperref[rem:first_order_formula_conventions/infix]{infix} \hyperref[def:binary_relation]{binary relation} \( \to \) on \( A \). We call the operation \( \to \) a \term{reduction relation}, and the pair \( (A, \to) \) --- an \term{abstract reduction system}.

  In the case where \( A \) is the \hyperref[def:formal_language/kleene_star]{Kleene star} of some set, we also call \( (A, \to) \) a \term{string rewriting system}.

  \begin{thmenum}
    \thmitem{def:abstract_reduction_system/relations}\mcite[not. 1.1.1]{Book1993} We also introduce the following auxiliary relations:
    \begin{itemize}
      \item \( \reloset n \to \) is the \( n \)-th \hyperref[def:binary_relation/composition]{iterated composition} of \( \to \), where \( n \) is a nonnegative integer.

      \item \( \reloset + \to \) is the \hyperref[def:relation_closures/transitive]{transitive closure} of \( \to \).

      \item \( \reloset {*} \to \) is the \hyperref[def:relation_closures/reflexive]{reflexive closure} of \( \reloset + \to \).

      \item \( {\leftrightarrow} \) is the \hyperref[def:relation_closures/symmetric]{symmetric closure} of \( \to \).

      \item \( \reloset + \leftrightarrow \) is the \hyperref[def:relation_closures/transitive]{transitive closure} of \( \leftrightarrow \).

      \item \( \reloset {*} \leftrightarrow \) is the \hyperref[def:relation_closures/reflexive]{reflexive closure} of \( \reloset + \leftrightarrow \), the smallest equivalence relation containing \( \to \).
    \end{itemize}

    \thmitem{def:abstract_reduction_system/hierarchy}\mcite[def. 1.1.3]{Book1993} In analogy with \fullref{def:arborescence/ancestry}, if \( x \reloset {*} \to y \), we say that \( y \) is a \term{descendant} of \( x \) and that \( x \) is an \term{ancestor} of \( y \). If \( x \to y \), we say that \( y \) is an \term{immediate descendant}, or a \term{successor}, especially when considering \( (A, \to) \) as a \hyperref[def:quiver]{quiver}.

    We say that an element is \term{reducible} if it has an immediate descendant and \term{irreducible} otherwise.

    \thmitem{def:abstract_reduction_system/equivalent}\mcite[def. 1.1.3]{Book1993} We say that \( x \) and \( y \) are \term{equivalent} if \( x \reloset {*} \leftrightarrow y \). This is a stronger condition than \( x \reloset {*} \to y \) and \( y \reloset {*} \to x \). Indeed, the latter corresponds to \hyperref[def:quiver_connectedness/weak]{weak connectedness} of quivers and equivalence corresponds to \hyperref[def:quiver_connectedness/strong]{strong connectedness}.

    \thmitem{def:abstract_reduction_system/normal_form}\mcite[def. 1.1.5]{Book1993} Given an \hyperref[def:equivalence_relation]{equivalence relation} \( \cong \), if \( x \cong y \) and \( y \) is irreducible, we say that \( y \) is a \term{normal form} of \( x \) modulo \( \cong \), and that \( x \) is \term{normalizable} modulo \( \cong \).

    Without further context, we assume that \( \cong \) is the equivalence relation \( \reloset {*} \leftrightarrow \).

    If \( x \) has a unique normal form, we denote it by \( \op{nf} \).
  \end{thmenum}
\end{definition}

\begin{definition}\label{def:abstract_rewriting_convergence}
  Let \( (A, \to) \) be an \hyperref[def:abstract_reduction_system]{abstract reduction system}. We will list several conditions ensuring existence and uniqueness of \hyperref[def:abstract_reduction_system/normal_form]{normal forms}.

  \begin{thmenum}
    \thmitem{def:abstract_rewriting_convergence/confluence}\mcite[def. 1.1.6]{Book1993} We call the system \term{confluent} if, whenever \( x \) and \( y \) are descendants of \( w \), they have a common descendant \( z \). That is,
    \begin{equation}\label{eq:def:abstract_rewriting_convergence/confluence/diagram}
      \begin{aligned}
        \includegraphics[page=1]{output/def__abstract_rewriting_convergence.pdf}
      \end{aligned}
    \end{equation}

    If there always exists an \hi{immediate descendant} \( z \), we say that the system is \term{locally confluent}.

    \thmitem{def:abstract_rewriting_convergence/noetherian}\mcite[def. 1.1.9]{Book1993} We call the system \term{noetherian} or \term{strongly normalizable} if there exists no \term{infinitely ascending sequence}
    \begin{equation*}
      x_1 \to x_2 \to x_3 \to \cdots.
    \end{equation*}

    This is dual to \hyperref[def:well_founded_relation]{well-foundedness}.

    \thmitem{def:abstract_rewriting_convergence/convergent}\mcite[def. 1.1.11]{Book1993} Finally, we call the system \term{convergent} if it is both confluent and noetherian.

    By \fullref{thm:noetherian_rewriting_system_local_confluence}, this is equivalent to the system being \hi{locally} confluent and noetherian.
  \end{thmenum}
\end{definition}

\begin{proposition}\label{thm:noetherian_rewriting_system_local_confluence}\mcite[thm. 1.1.13]{Book1993}
  A \hyperref[def:abstract_rewriting_convergence/noetherian]{noetherian}  \hyperref[def:abstract_reduction_system]{abstract reduction system} is \hyperref[def:abstract_rewriting_convergence/confluence]{confluent} if and only if it is \hyperref[def:abstract_rewriting_convergence/confluence]{locally confluent}.
\end{proposition}

\begin{proposition}\label{thm:convergent_reduction_system_normal_forms}
  In a \hyperref[def:abstract_rewriting_convergence/convergent]{convergent} \hyperref[def:abstract_reduction_system]{abstract reduction system}, every element has a unique \hyperref[def:abstract_reduction_system/normal_form]{normal form}.
\end{proposition}
\begin{proof}
  Let \( (A, \to) \) be a convergent abstract reduction system.

  \SubProof{Proof of existence} Suppose that \( x \) has no normal form. Hence, it is reducible and has an immediate descendant \( x_1 \). Proceeding by \hyperref[rem:natural_number_recursion]{natural number recursion}, we can build a sequence
  \begin{equation*}
    x \to x_1 \to x_2 \to \cdots.
  \end{equation*}

  Such a sequence cannot exist, however, since the system is \hyperref[def:abstract_rewriting_convergence/noetherian]{noetherian}.

  \SubProof{Proof of uniqueness} Suppose that both \( y \) and \( z \) are normal forms of \( w \).

  Since the system is confluent, there exists an element \( z \) that is a descendant of both \( x \) and \( y \). But \( x \) and \( y \) are irreducible, therefore \( x = y \).
\end{proof}

\begin{definition}\label{def:free_group}
  Let \( A \) be an arbitrary set. We will now construct the \term{free group} \( F(A) \) of \( A \).

  Let \( \anon \) be an arbitrary set not in \( A \). Consider the \hyperref[def:disjoint_union]{disjoint union} \( U \coloneqq A \times \set{ +, - } \), whose members we will denote by \( a^+ \) and \( a^- \).

  On the \hyperref[def:formal_language/kleene_star]{Kleene star} \( U \), define the relation \( \to \) to hold for \( x \to y \) if there exists \( a \in A \) and words \( p \) and \( s \) such that \( y = ps \) and either \( x = p a^+ a^- s \) or \( x = p a^+ a^- s \).

  Then \( (U, \to) \) is a \hyperref[def:abstract_rewriting_convergence/convergent]{convergent} \hyperref[def:abstract_reduction_system]{abstract reduction system}. The \term{free group} \( F(S) \) is the set of \hyperref[def:abstract_reduction_system/hierarchy]{irreducible elements} in \( U \), with the operation \( v \star w \coloneqq \op{nf}(vw) \) that gives the \hyperref[def:abstract_reduction_system/normal_form]{normal form} of the concatenated string \( vw \).

  The identity is the empty word and the inverse \( v^{-1} \) of \( v \) can be characterized recursively as
  \begin{equation*}
    v^{-1} \coloneqq \begin{cases}
      \varepsilon &v = \varepsilon \\
      u^{-1} a^-  &v = u a^+ \\
      u^{-1} a^+  &v = u a^-
    \end{cases}
  \end{equation*}

  The canonical inclusion is
  \begin{equation*}
    \begin{aligned}
      &\iota_A: A \to F(A) \\
      &\iota_A(a) \coloneqq a^+.
    \end{aligned}
  \end{equation*}
\end{definition}
\begin{defproof}
  We need to prove that the system \( (U, \to) \) is convergent. Then from \fullref{thm:convergent_reduction_system_normal_forms} it will follow that every word has a unique normal form.

  \SubProofOf[def:abstract_rewriting_convergence/confluence]{local confluence} Fix two reductions \( w \to x \) and \( w \to y \).

  If \( x = y \), define \( z \coloneqq x \). Otherwise, let \( p_x \) and \( s_x \) be words such that \( w = p_x a^+ a^- s_x \) and \( x = p_x s_x \) (the case where \( w = p_x a^- a^+ s_x \) is analogous). Similarly, let \( p_y \) and \( s_y \) be words such that \( w = p_y b^+ b^- s_y \) and \( y = p_y s_y \).

  Without loss of generality, suppose that \( p_x \) is a subword of \( p_y \). Then there exists a word \( v \) such that
  \begin{equation*}
    w = p_x a^+ a^- v b^+ b^- s_y.
  \end{equation*}

  Then both \( x \) and \( y \) are reducible to \( z \coloneqq p_x v s_y \).

  This shows that the rewriting system is locally confluent.

  \SubProofOf[def:abstract_rewriting_convergence/noetherian]{noetherianity} All words are finite, and a reduction always removes two symbols. Hence, there cannot exist an infinite reduction path.
\end{defproof}

\begin{proposition}\label{thm:free_group_universal_property}
  The \hyperref[def:free_group]{free group} \( F(A) \) is the unique up to an isomorphism group that satisfies the following \hyperref[rem:universal_mapping_property]{universal mapping property}:
  \begin{displayquote}
    For every group \( G \) and every function \( f: A \to G \), there exists a unique group homomorphism \( \widetilde{f}: F(A) \to G \) such that the following diagram commutes:
    \begin{equation}\label{eq:thm:free_group_universal_property/diagram}
      \begin{aligned}
        \includegraphics[page=1]{output/thm__free_group_universal_property.pdf}
      \end{aligned}
    \end{equation}
  \end{displayquote}
\end{proposition}
\begin{proof}
  The free group operation is more complicated than the free monoid operation, however the proof of the universal mapping property is identical to the property in \fullref{thm:free_monoid_universal_property}.
\end{proof}

\begin{corollary}\label{thm:free_group_functor}
  Consider the functor \( F: \cat{Set} \to \cat{Grp} \) defined for objects pointwise in \fullref{def:free_group}. For every function \( f: A \to B \), define the \hyperref[def:group/homomorphism]{group homomorphism}
  \begin{equation*}
    \begin{aligned}
      &F(f): F(A) \to F(B) \\
      &F(f)(x_1 x_2 \ldots x_n) \coloneqq \op{nf}(f(x_1) f(x_2) \ldots f(x_n)).
    \end{aligned}
  \end{equation*}

  This functor is \hyperref[def:category_adjunction]{left adjoint} to the \hyperref[def:concrete_category]{forgetful functor} \( U: \cat{Grp} \to \cat{Set} \).
\end{corollary}
\begin{proof}
  Follows from \fullref{thm:free_group_universal_property} via \fullref{rem:universal_mapping_property}.
\end{proof}

\begin{definition}\label{def:group_presentation}\mcite[314]{Knapp2016BasicAlgebra}
  Let \( S \) be a set, \( F(S) \) be the \hyperref[def:free_group]{free group} and \( R \subseteq F(S) \) be a subset. Denote by \( N(R) \) the \hyperref[def:first_order_generated_substructure]{generated} by \( R \) \hyperref[def:normal_subgroup]{normal subgroup}.

  The group with \term{generators} \( S \) and \term{relators} \( R \) is
  \begin{equation}\label{eq:def:group_presentation/presentation}
    \braket{ S \mid R } \coloneqq F(S) / N(R).
  \end{equation}

  We call the pair \( (S, R) \) \eqref{eq:def:group_presentation/presentation} a \term{presentation} of the group \( \braket{ S \mid R } \), although we usually denote the presentation itself via \( \braket{ S \mid R } \).

  Note that the free group is generated by a set, which is different from generated subgroups defined in \fullref{def:first_order_generated_substructure}. The former defines a new group operation from scratch, while the latter uses an existing group operation.

  We say that the group \( G \) is \term{finitely generated} if it is isomorphic to \( \braket{ S \mid R } \), where \( S \) is a \hyperref[def:set_finiteness]{finite set}. If only \( R \) is finite, we call \( G \) \term{finitely presented}.

  If \( R = \varnothing \), there are no relators, and we use the following notation for the free group:
  \begin{equation}\label{eq:def:group_presentation/free}
    \braket{ S } \coloneqq F(S)
  \end{equation}
\end{definition}

\begin{definition}\label{def:cyclic_group}
  For a singleton alphabet \( \set{ a } \), we define the \term{infinite cyclic group}
  \begin{equation*}
    C \coloneqq \braket{a}
  \end{equation*}
  and, for positive integers \( n \), the \term{finite cyclic group} of \term{order} \( n \) as
  \begin{equation*}
    C_n \coloneqq \braket{a \given a^n}.
  \end{equation*}

  We use the same notation independent of \( a \) because all cyclic groups of the same order are obviously \hyperref[def:group/homomorphism]{isomorphic}.

  As shown in \fullref{thm:cyclic_group_isomorphic_to_integers_modulo_n}, cyclic groups are isomorphic to certain groups of integers, however it is still useful to have cyclic groups as a separate concept.
\end{definition}

\begin{proposition}\label{thm:every_group_is_presented}\mcite[prop. 7.7]{Knapp2016BasicAlgebra}
  Every group has at least one \hyperref[def:group_presentation]{presentation}.
\end{proposition}
\begin{proof}
  Fix an arbitrary group \( G \) and let \( S \coloneqq U(G) \) be the underlying set.

  By \fullref{thm:free_group_universal_property}, there exists a unique group homomorphism \( \varphi: F(S) \to G \) such that
  \begin{equation*}
    U(\varphi) \bincirc \iota_S = \id_S.
  \end{equation*}

  The kernel \( \ker \varphi \) is a normal subgroup of \( G \), hence by \fullref{thm:homomorphism_theorem_for_groups},
  \begin{equation*}
    G = \varphi(F(S)) \cong F(S) / \ker \varphi = \braket{ S \mid \ker \varphi }.
  \end{equation*}
\end{proof}

\begin{definition}\label{def:group_free_product}\mcite[323]{Knapp2016BasicAlgebra}
  The \term{free product} of a nonempty pairwise disjoint family of groups \( \seq{ X_k }_{k \in \mscrK} \) with \hyperref[def:group_presentation]{presentations} \( \braket{S_k \mid R_k}, k \in \mscrK \) is the group with presentation
  \begin{equation*}
    \Ast_{k \in \mscrK} X_k \coloneqq \braket*{ \bigcup_{k \in \mscrK} S_k \given[\Big] \bigcup_{k \in \mscrK} R_k }.
  \end{equation*}

  If the constituent groups are not disjoint, we may instead use \hyperref[def:disjoint_union]{disjoint unions} as
  \small
  \begin{equation*}
    \Ast_{k \in \mscrK} X_k \coloneqq \braket*{ \set[\Big]{ (k, x) \given k \in \mscrK \T{and} x \in S_k } \given[\Big] \set[\Big]{ (k, x_1) (k, x_2) \ldots (k, x_n) \colon k \in \mscrK \T{and} x_1 x_2 \ldots x_n \in R_k } }.
  \end{equation*}
  \normalsize
\end{definition}

\begin{definition}\label{def:group_direct_product}
  The \term{direct product} of a nonempty family of groups \( \seq{ X_k }_{k \in \mscrK} \) is their \hyperref[def:cartesian_product]{Cartesian product} \( \prod_{k \in \mscrK} X_k \) with the componentwise group operation
  \begin{equation*}
    \seq{ x_k }_{k \in \mscrK} \cdot \seq{ y_k }_{k \in \mscrK}
    \coloneqq
    \seq{ x_k \cdot y_k }_{k \in \mscrK}.
  \end{equation*}
\end{definition}

\begin{definition}\label{def:group_order}
  The \term{order} \( \ord(G) \) of a group \( G \) is its \hyperref[thm:cardinality_existence]{cardinality}.

  The \term{order} \( \ord(x) \) of a member \( x \) of a group is the smallest positive integer \( n \) such that \( x^n = e \), i.e. order of the \hyperref[def:cyclic_group]{cyclic subgroup} \hyperref[def:first_order_generated_substructure]{generated} by \( x \).
\end{definition}

\begin{proposition}\label{thn:product_of_cyclic_groups}
  The \hyperref[def:group_direct_product]{direct product} \( C_m \times C_n \) of two \hyperref[def:cyclic_group]{cyclic groups} is cyclic if and only if \( m \) and \( n \) are \hyperref[def:coprime_numbers]{coprime}.
\end{proposition}
\begin{proof}
  The \hyperref[def:group_order]{order} of the element \( (a, e) \) is \( m \) and the order of \( (e, a) \) is \( n \). The order of
  \begin{equation*}
    (a, a) = (a, e) \cdot (e, a)
  \end{equation*}
  is the least common multiple of \( m \) and \( n \), which equals \( mn \) if and only if \( m \) and \( n \) are coprime.
\end{proof}

\begin{proposition}\label{thm:group_categorical_limits}
  \hfill
  \begin{thmenum}
    \thmitem{thm:group_categorical_limits/product} The \hyperref[def:discrete_category_limits]{categorical product} of the family \( \seq{ G_k }_{k \in \mscrK} \) is their \hyperref[def:group_direct_product]{direct product} \( \prod_{k \in \mscrK} G_k \), the projection morphisms being
    \begin{equation*}
      \begin{aligned}
        &\pi_m: \prod_{k \in \mscrK} G_k \to G_m \\
        &\pi_m(\seq{ x_k }_{k \in \mscrK}) \coloneqq x_m.
      \end{aligned}
    \end{equation*}

    \thmitem{thm:group_categorical_limits/coproduct} The \hyperref[def:discrete_category_limits]{categorical coproduct} of the family \( \seq{ G_k }_{k \in \mscrK} \) is the \hyperref[def:group_free_product]{free product} \( \Ast_{k \in \mscrK} X_k \), the inclusion morphisms being
    \begin{equation*}
      \begin{aligned}
         &\iota_m: X_m \to \Ast_{k \in \mscrK} X_k \\
         &\iota_m(x) \coloneqq (m, x).
      \end{aligned}
    \end{equation*}
  \end{thmenum}
\end{proposition}
\begin{proof}
  \SubProofOf{thm:group_categorical_limits/product} Let \( (A, \alpha) \) be a \hyperref[def:category_of_cones/cone]{cone} for the \hyperref[def:discrete_category]{discrete} \hyperref[def:categorical_diagram]{diagram} \( \seq{ G_k }_{k \in \mscrK} \). We want to define a group homomorphism \( l_A: A \to \prod_{k \in \mscrK} G_k \) such that, for every \( m \in \mscrK \) and \( a \in A \),
  \begin{equation*}
    \alpha_m(a) = \pi_m(l_A(a)).
  \end{equation*}

  This suggests the only possible definition
  \begin{equation*}
    l_A(a) \coloneqq \seq{ \alpha_m(a) }_{k \in \mscrK}.
  \end{equation*}

  \SubProofOf{thm:group_categorical_limits/coproduct}  Let \( (A, \alpha) \) be a \hyperref[def:category_of_cones/cocone]{cocone} for the discrete diagram \( \seq{ G_k }_{k \in \mscrK} \). We want to define a group homomorphism \( l: \Ast_{k \in \mscrK} G_k \to A \) such that, for every \( m \in \mscrK \),
  \begin{equation*}
    \alpha_m(x) = l_A(\iota_m(x)).
  \end{equation*}

  This suggests the only possible definition
  \begin{equation*}
    l_A\parens[\Big]{ (k_1, x_1) (k_2, x_2) \ldots (k_n, x_n) } \coloneqq \alpha_{k_1}(x_1) \cdot \alpha_{k_2}(x_k) \cdot \ldots \cdot \alpha_{k_n}(x_n).
  \end{equation*}
\end{proof}
