\subsection{Partial orders}\label{subsec:partial_orders}

\hyperref[def:preordered_set]{Preordered sets} are simple to define and arise naturally (e.g. \fullref{def:lindenbaum_tarski_algebra}) but they have uniqueness considerations (see \fullref{ex:preorder_nonuniqueness}) that we usually want to factor out. \Fullref{thm:preorder_to_partial_order} shows that the result obtained from this factorization is a partially ordered set, which this section is dedicated to.

\begin{definition}\label{def:poset}
  Fix a set \( \mscrP \). A \term{partially ordered set} structure on \( \mscrP \), also called a \term{poset} structure, can be defined in the following equivalent ways:
  \begin{thmenum}[series=def:poset]
    \thmitem{def:poset/nonstrict} A \hyperref[def:preordered_set]{preorder} \( \leq \) on \( \mscrP \) such that \( \leq \) is \hyperref[def:binary_relation/antisymmetric]{antisymmetric} in addition to being \hyperref[def:binary_relation/reflexive]{reflexive} and \hyperref[def:binary_relation/transitive]{transitive}. This definition is the more common one. If we wish to distinguish it from the other definition, we call such a relation a \term{nonstrict partial order}.

    \thmitem{def:poset/strict} An \hyperref[def:binary_relation/irreflexive]{irreflexive} and \hyperref[def:binary_relation/transitive]{transitive} binary relation \( < \) on \( \mscrP \). This relation is called a \term{strict partial order}.
  \end{thmenum}

  If both relations are present, in order for the them to be equivalent, \( \leq \) must be the union of \( < \) and the \hyperref[def:binary_relation/diagonal]{diagonal} \( \Delta \). This condition corresponds to the following axiom:
  \begin{equation}\label{def:poset/compatibility_nonstrict}
    (x \leq y) \leftrightarrow \parens[\Big]{(x < y) \vee (x = y)}.
  \end{equation}

  By adding \( \placeholder \wedge \neg (x = y) \) to both sides of \eqref{def:poset/compatibility_nonstrict}, using \fullref{thm:de_morgans_laws} and taking irreflexivity of \( < \) into account, we obtain
  \begin{equation}\label{def:poset/compatibility_strict}
    (x < y) \leftrightarrow \parens[\Big]{(x \leq y) \wedge \neg (x = y)}.
  \end{equation}

  \begin{thmenum}[resume=def:poset]
    \thmitem{def:poset/theory} Since we can interdefine nonstrict and strict orders, it makes little sense to study different theories for the two.

    In order to define the \term{theory of posets}, we extend the language of the \hyperref[def:preordered_set/theory]{theory of preordered sets} with a binary infix predicate symbol \( < \). We then add to the theory \hyperref[def:binary_relation/antisymmetric]{Antisymmetry} for \( \leq \) and either of the compatibility condition \eqref{def:poset/compatibility_nonstrict} or \eqref{def:poset/compatibility_strict} (it is unnecessary to add both).

    We can also add \hyperref[def:binary_relation/irreflexive]{irreflexivity} and \hyperref[def:binary_relation/transitive]{transitivity} for \( < \) but that would also be redundant.

    \thmitem{def:poset/homomorphism} As for preordered sets, the \hyperref[def:first_order_homomorphism]{homomorphisms} in this theory are again \hyperref[def:preordered_set/homomorphism]{monotone maps}. If we instead build a theory of strict partial orders without \( \leq \), homomorphisms are instead the more restrictive strict monotone maps.

    \thmitem{def:poset/substructure} As for preordered sets, any subset of a poset is itself a poset.

    \thmitem{def:poset/category} We denote the \hyperref[def:category_of_first_order_models]{category of models} by \( \cat{Pos} \). It is a full subcategory of the \hyperref[def:preordered_set/category]{category of preordered sets}.
  \end{thmenum}
\end{definition}
\begin{proof}
  \ImplicationSubProof{def:poset/nonstrict}{def:poset/strict} Let \( \leq \) be a nonstrict partial order. We will show that \( < \) is a strict partial order.

  \begin{itemize}
    \item The relation \( < \) is \hyperref[def:binary_relation/transitive]{transitive}. To see this, let \( x < y \) and \( y < z \). In particular, \( x \leq y \) and \( y \leq z \). By transitivity, \( x \leq z \).

    Additionally, \( x \neq y \) and \( y \neq z \). Assume\DNE that \( x = z \). By reflexivity of \( \leq \), we have \( z \leq x \) and, since \( y \leq z \), by transitivity we obtain \( y \leq x \). But since \( x \leq y \), by the antisymmetry of \( \leq \), we have \( x = y \), which contradicts the assumption that \( x < y \).

    Therefore \( x < z \).

    \item \hyperref[def:binary_relation/irreflexive]{Irreflexivity} of \( < \) follows directly from reflexivity of \( \leq \) and the compatibility condition.
  \end{itemize}

  Since the right side is false, the left side \( x < x \) is also false.

  \ImplicationSubProof{def:poset/strict}{def:poset/nonstrict} Let \( < \) be a strict partial order. We will show that \( \leq \) is a nonstrict partial order.

  \begin{itemize}
    \item To see \hyperref[def:binary_relation/reflexive]{reflexivity}, fix \( x \in \mscrP \) and assume\DNE that \( x \not\leq x \). Then \( x \neq x \) which contradicts the reflexivity of equality. Hence \( x \leq x \).

    \item To see \hyperref[def:binary_relation/antisymmetric]{antisymmetry}, let \( x \leq y \) and \( y \leq x \), that is, either \( x = y \) or both \( x < y \) and \( y < x \) hold. Assume the latter. By the transitivity of \( \leq \), we have \( x < x \), which contradicts the irreflexivity of \( < \). Hence \( x = y \).

    \item To see \hyperref[def:binary_relation/transitive]{transitivity}, let \( x \leq y \) and \( y \leq z \). Then we have four cases depending on which of \( x \), \( y \) and \( z \) are equal. Since both relations \( < \) and \( = \) are transitive, it follows that in all four cases \( x \leq z \).
  \end{itemize}
\end{proof}

\begin{definition}\label{def:hasse_diagram}
  It is usually easier to define small finite posets by drawing graphs than by enumerating all relation pairs. Let \( (\mscrP, \leq) \) be a finite poset. The relation \( \leq \) may also be regarded as the edge set of a \hyperref[def:directed_graph]{directed graph}. The graph \( (\mscrP, \red^T(\leq)) \), whose edges are the \hyperref[def:derived_relations/transitive]{transitive reduction} of \( \leq \), is called the \term{Hasse graph} or \term{Hasse diagram} of \( (\mscrP, \leq) \).

  The term \enquote{Hasse diagram} is usually associated with drawings. By convention, no arrowheads for denoting directions are drawn on the Hasse graph despite the graph being directed; instead, edges always point upwards. See \fullref{ex:def:hasse_diagram}.
\end{definition}

\begin{example}\label{ex:def:hasse_diagram}
  Consider the partial order over \( \set{ a, b, c, d, e } \) defined via
  \begin{equation}\label{eq:ex:def:hasse_diagram/poset}
    \underline{a \leq c},\quad \underline{a \leq d},\quad a \leq e,\quad \underline{b \leq d},\quad b \leq e,\quad \underline{d \leq e}.
  \end{equation}

  The corresponding Hasse graph includes only the underlined edges. The rest of the edges can be restored from transitivity. In this case, the Hasse graph has edges
  \begin{equation}\label{eq:ex:def:hasse_diagram/hasse_graph}
    \set{ a \to c, a \to d, b \to d, d \to e }
  \end{equation}

  \begin{figure}
    \centering
    \includegraphics{figures/ex__def__hasse_diagram.pdf}
    \caption{The Hasse diagram \eqref{eq:ex:def:hasse_diagram/hasse_graph}}
    \label{fig:ex:def:hasse_diagram}
  \end{figure}
\end{example}

\begin{example}\label{ex:preorder_nonuniqueness}
  Consider the preordered set \( \mscrP \) in \ref{fig:ex:preorder_nonuniqueness} in which \( b \leq c \) and \( c \leq b \) but \( b \neq c \). We cannot properly draw a \hyperref[def:hasse_diagram]{Hasse diagram} because we have the restriction that \( c \) is drawn (strictly) higher than \( b \) if \( c \geq b \) and that \( c \) is drawn lower than \( b \) if \( c \leq b \). We face a similar problem formally, for example in logic in the definition of a \hyperref[def:lindenbaum_tarski_algebra]{Lindenbaum-Tarski algebra}, where the preorder \( \vdash \) allows \( \varphi \vdash \psi \) and \( \psi \vdash \varphi \) but still \( \varphi \neq \psi \). Thus we have nonuniqueness --- every tautology is a largest element with respect to \( \vdash \), while we want to have a single largest element for the sake of building a tidier theory.

  If we are only interested in members of \( \mscrP \) up to the equivalence \eqref{eq:thm:preorder_to_partial_order/equivalence}, it is easy to factor \( \mscrP \) by \eqref{eq:thm:preorder_to_partial_order/equivalence} and obtain a partially ordered set. In the language of graph theory, if we have \hyperref[def:graph_paths/directed_path]{circuits} that we may wish to avoid, we can contract each circuit into a single vertex, at which point the graph becomes acyclic.

  The formulation and proof of correctness of this process can be found in \fullref{thm:preorder_to_partial_order} and an example can be found in \cref{fig:ex:preorder_nonuniqueness}.

  \begin{figure}
    \hfill
    \includegraphics{figures/ex__preorder_nonuniqueness__preordered.pdf}
    \hfill
    \includegraphics{figures/ex__preorder_nonuniqueness__factorized.pdf}
    \hfill
    \hfill
    \caption{A preordered set and its induced poset.}
    \label{fig:ex:preorder_nonuniqueness}
  \end{figure}
\end{example}

\begin{proposition}\label{thm:preorder_to_partial_order}
  Let \( (\mscrP, \leq) \) be a preordered set. Define the relation \( \cong \) by
  \begin{equation}\label{eq:thm:preorder_to_partial_order/equivalence}
    x \cong y \iff x \leq y \T{and} y \leq x.
  \end{equation}

  That is, \( \cong \) is the intersection of the relation \( \leq \) with its \hyperref[def:binary_relation/converse]{converse}.

  Since \( \cong \) is an \hyperref[def:equivalence_relation]{equivalence relation} we can for the the quotient set \( \mscrP / \cong \). Define the relation \( \preceq \) on this quotient set by
  \begin{equation*}
    [x] \preceq [y] \iff x \leq y.
  \end{equation*}

  The pair \( (\mscrP / \cong, \preceq) \) is then a \hyperref[def:poset]{partially ordered set}.
\end{proposition}
\begin{proof}
  The relation \( \preceq \) is well-defined. Indeed, let \( x \cong x' \) and \( y \cong y' \), that is, both \( x \leq x' \) and \( x' \leq x \) and similarly for \( y \). If \( x \leq y \), by transitivity \( x \leq y \leq y' \). But \( x' \leq x \), hence \( x' \leq y' \).

  It is then clear that \( \preceq \) is a partial order because it inherits reflexivity and transitivity from \( \leq \) and antisymmetry is imposed by taking quotient sets --- equality in \( \mscrP / \cong \) holds precisely when \( \cong \) holds in \( \mscrP \).
\end{proof}

\begin{proposition}\label{thm:partial_order_category_correspondence}
  To every \hyperref[def:poset]{poset} there corresponds exactly one \hyperref[def:category_cardinality]{small} \hyperref[def:thin_category]{thin} \hyperref[def:skeletal_category]{skeletal} category.

  Compare this result to \fullref{thm:preorder_category_correspondence}.
\end{proposition}
\begin{proof}
  The statement follows from \fullref{thm:preorder_category_correspondence} by noting that the factorization in \fullref{thm:preorder_to_partial_order} makes the corresponding category skeletal.
\end{proof}

\begin{definition}
  We introduce the following terminology for elements and subsets of a poset \( (\mscrP, \leq) \). Analogous definition can be given for preordered sets but the nonuniqueness problems outlined in \fullref{ex:preorder_nonuniqueness} highlight that there are difficulties in doing so.

  \begin{thmenum}[resume=def:preordered_set]
    \thmitem{def:preordered_set/comparability} The elements \( x, y \in \mscrP \) are called \term{comparable} if either \( x \leq y \) or \( y \leq x \).

    \thmitem{def:preordered_set/upper_and_lower_bounds}\mcite[170]{Enderton1977Sets} The member \( x \in \mscrP \) is an \term{upper bound} for \( A \subseteq \mscrP \) if \( y \leq x \) for any \( y \in A \). Dually, \( x \) is a \term{lower bound} of \( A \) if it is an upper bound in the dual preordered set \( (\mscrP, \geq) \).

    \thmitem{def:preordered_set/bounded_set} The set \( A \subseteq \mscrP \) is called \term{bounded from above} (resp. \term{bounded from below}) if it has an upper bound (resp. lower bound). If a set is bounded from both directions, we simply say that it is \term{bounded}.

    \thmitem{def:preordered_set/maximal_and_minimal_element}\mcite[170]{Enderton1977Sets} The member \( x \in \mscrP \) is a \term{maximal element} of \( A \subseteq \mscrP \) (resp. \term{minimal element} of \( A \) in \( (\mscrP, \geq) \)) if \( x \leq y \) implies \( x = y \) for any \( y \in A \).

    \thmitem{def:preordered_set/maximum_and_minimum}\mcite[171]{Enderton1977Sets} The member \( x \in \mscrP \) is a \term{largest element} or \term{maximum} \( \max A \) of \( A \subseteq \mscrP \) (resp. \term{smallest element} or \term{minimum} \( \min A \) of \( A \) in \( (\mscrP, \geq) \)) if \( y \leq x \) for any \( y \in A \). While there may exist distinct maximal elements in \( A \), there exists at most one maximum.

    \thmitem{def:preordered_set/supremum_and_infimum}\mcite[170]{Enderton1977Sets} The member \( x \in \mscrP \) is a \term{supremum} (resp. \term{infimum}) for \( A \subseteq \mscrP \) if \( x \) is the least upper bound (resp. greatest lower bound) of \( A \), i.e. the smallest element of the set consisting of all upper bounds of \( \mscrP \). Suprema and infima may exist when neither maximal nor largest elements exist.
  \end{thmenum}
\end{definition}
