\subsection{Partial orders}\label{subsec:partial_orders}

\hyperref[def:preordered_set]{Preordered sets} are simple to define and arise naturally (e.g. \fullref{def:lindenbaum_tarski_algebra}) but they have uniqueness considerations (see \fullref{ex:preorder_nonuniqueness}) that we usually want to factor out. \Fullref{thm:preorder_to_partial_order} shows that the result obtained from this factorization is a partially ordered set, which this section is dedicated to.

\begin{definition}\label{def:poset}
  Fix a set \( \mscrP \). A \term{partially ordered set} structure on \( \mscrP \), also called a \term{poset} structure, can be defined in the following equivalent ways:
  \begin{thmenum}[series=def:poset]
    \thmitem{def:poset/nonstrict} A \hyperref[def:preordered_set]{preorder} \( \leq \) on \( \mscrP \) such that \( \leq \) is \hyperref[def:binary_relation/antisymmetric]{antisymmetric} in addition to being \hyperref[def:binary_relation/reflexive]{reflexive} and \hyperref[def:binary_relation/transitive]{transitive}. This definition is the more common one. If we wish to distinguish it from the other definition, we call such a relation a \term{nonstrict partial order}.

    \thmitem{def:poset/strict} An \hyperref[def:binary_relation/irreflexive]{irreflexive} and \hyperref[def:binary_relation/transitive]{transitive} binary relation \( < \) on \( \mscrP \). This relation is called a \term{strict partial order}.
  \end{thmenum}

  If both relations are present, in order for the them to be equivalent, \( \leq \) must be the union of \( < \) and the \hyperref[def:binary_relation/diagonal]{diagonal} \( \Delta \). This condition corresponds to the following axiom:
  \begin{equation}\label{def:poset/compatibility_nonstrict}
    (x \leq y) \leftrightarrow \parens[\Big]{(x < y) \vee (x = y)}.
  \end{equation}

  By adding \( \placeholder \wedge \neg (x = y) \) to both sides of \eqref{def:poset/compatibility_nonstrict}, using \fullref{thm:de_morgans_laws} and taking irreflexivity of \( < \) into account, we obtain
  \begin{equation}\label{def:poset/compatibility_strict}
    (x < y) \leftrightarrow \parens[\Big]{(x \leq y) \wedge \neg (x = y)}.
  \end{equation}

  The elements \( x, y \in \mscrP \) are called \term{comparable} if either \( x \leq y \) or \( y \leq x \).

  \begin{thmenum}[resume=def:poset]
    \thmitem{def:poset/theory} Since we can interdefine nonstrict and strict orders, it makes little sense to study different theories for the two.

    In order to define the \term{theory of posets}, we extend the language of the \hyperref[def:preordered_set/theory]{theory of preordered sets} with two binary infix predicate symbol --- \( < \) and \( > \). We then add to the theory \hyperref[def:binary_relation/antisymmetric]{Antisymmetry} for \( \leq \) and either of the compatibility condition \eqref{def:poset/compatibility_nonstrict} or \eqref{def:poset/compatibility_strict} (it is unnecessary to add both).

    We can also add \hyperref[def:binary_relation/irreflexive]{irreflexivity} and \hyperref[def:binary_relation/transitive]{transitivity} for \( < \) but that would also be redundant.

    \thmitem{def:poset/homomorphism} As for preordered sets, the \hyperref[def:first_order_homomorphism]{homomorphisms} in this theory are again \hyperref[def:preordered_set/homomorphism]{monotone maps}. If we instead build a theory of strict partial orders without \( \leq \), homomorphisms are instead the more restrictive strict monotone maps.

    \thmitem{def:poset/substructure} As for preordered sets, any subset of a poset is itself a poset.

    \thmitem{def:poset/category} We denote the \hyperref[def:category_of_first_order_models]{category of models} by \( \cat{Pos} \). It is a full subcategory of the \hyperref[def:preordered_set/category]{category of preordered sets}.

    \thmitem{def:poset/duality} The \hyperref[def:preordered_set/duality]{principle of duality for preordered sets} also holds for posets if we also swap \( < \) and \( > \).
  \end{thmenum}
\end{definition}
\begin{proof}
  \ImplicationSubProof{def:poset/nonstrict}{def:poset/strict} Let \( \leq \) be a nonstrict partial order. We will show that \( < \) is a strict partial order.

  \begin{itemize}
    \item The relation \( < \) is \hyperref[def:binary_relation/transitive]{transitive}. To see this, let \( x < y \) and \( y < z \). In particular, \( x \leq y \) and \( y \leq z \). By transitivity, \( x \leq z \).

    Additionally, \( x \neq y \) and \( y \neq z \). Assume\DNE that \( x = z \). By reflexivity of \( \leq \), we have \( z \leq x \) and, since \( y \leq z \), by transitivity we obtain \( y \leq x \). But since \( x \leq y \), by the antisymmetry of \( \leq \), we have \( x = y \), which contradicts the assumption that \( x < y \).

    Therefore \( x < z \).

    \item \hyperref[def:binary_relation/irreflexive]{Irreflexivity} of \( < \) follows directly from reflexivity of \( \leq \) and the compatibility condition.
  \end{itemize}

  Since the right side is false, the left side \( x < x \) is also false.

  \ImplicationSubProof{def:poset/strict}{def:poset/nonstrict} Let \( < \) be a strict partial order. We will show that \( \leq \) is a nonstrict partial order.

  \begin{itemize}
    \item To see \hyperref[def:binary_relation/reflexive]{reflexivity}, fix \( x \in \mscrP \) and assume\DNE that \( x \not\leq x \). Then \( x \neq x \) which contradicts the reflexivity of equality. Hence \( x \leq x \).

    \item To see \hyperref[def:binary_relation/antisymmetric]{antisymmetry}, let \( x \leq y \) and \( y \leq x \), that is, either \( x = y \) or both \( x < y \) and \( y < x \) hold. Assume the latter. By the transitivity of \( \leq \), we have \( x < x \), which contradicts the irreflexivity of \( < \). Hence \( x = y \).

    \item To see \hyperref[def:binary_relation/transitive]{transitivity}, let \( x \leq y \) and \( y \leq z \). Then we have four cases depending on which of \( x \), \( y \) and \( z \) are equal. Since both relations \( < \) and \( = \) are transitive, it follows that in all four cases \( x \leq z \).
  \end{itemize}
\end{proof}

\begin{definition}\label{def:hasse_diagram}
  It is usually easier to define small finite posets by drawing graphs than by enumerating all relation pairs. Let \( (\mscrP, \leq) \) be a finite poset. The relation \( \leq \) may also be regarded as the edge set of a \hyperref[def:directed_graph]{directed graph}. The graph \( (\mscrP, \red^T(\leq)) \), whose edges are the \hyperref[def:derived_relations/transitive]{transitive reduction} of \( \leq \), is called the \term{Hasse graph} or \term{Hasse diagram} of \( (\mscrP, \leq) \).

  The term \enquote{Hasse diagram} is usually associated with drawings. By convention, no arrowheads for denoting directions are drawn on the Hasse graph despite the graph being directed; instead, edges always point upwards. See \fullref{ex:def:hasse_diagram}.
\end{definition}

\begin{example}\label{ex:def:hasse_diagram}
  Consider the partial order over \( \set{ a, b, c, d, e } \) defined via
  \begin{equation}\label{eq:ex:def:hasse_diagram/poset}
    \underline{a \leq c},\quad \underline{a \leq d},\quad a \leq e,\quad \underline{b \leq d},\quad b \leq e,\quad \underline{d \leq e}.
  \end{equation}

  The corresponding Hasse graph includes only the underlined edges. The rest of the edges can be restored from transitivity. In this case, the Hasse graph has edges
  \begin{equation}\label{eq:ex:def:hasse_diagram/hasse_graph}
    \set{ a \to c, a \to d, b \to d, d \to e }
  \end{equation}

  \begin{figure}
    \centering
    \includegraphics{figures/ex__def__hasse_diagram.pdf}
    \caption{The Hasse diagram \eqref{eq:ex:def:hasse_diagram/hasse_graph}}
    \label{fig:ex:def:hasse_diagram}
  \end{figure}
\end{example}

\begin{example}\label{ex:preorder_nonuniqueness}
  Consider the preordered set \( \mscrP \) in \ref{fig:ex:preorder_nonuniqueness} in which \( b \leq c \) and \( c \leq b \) but \( b \neq c \). We cannot properly draw a \hyperref[def:hasse_diagram]{Hasse diagram} because we have the restriction that \( c \) is drawn (strictly) higher than \( b \) if \( c \geq b \) and that \( c \) is drawn lower than \( b \) if \( c \leq b \). We face a similar problem formally, for example in logic in the definition of a \hyperref[def:lindenbaum_tarski_algebra]{Lindenbaum-Tarski algebra}, where the preorder \( \vdash \) allows \( \varphi \vdash \psi \) and \( \psi \vdash \varphi \) but still \( \varphi \neq \psi \). Thus we have nonuniqueness --- every tautology is a largest element with respect to \( \vdash \), while we want to have a single largest element for the sake of building a tidier theory.

  If we are only interested in members of \( \mscrP \) up to the equivalence \eqref{eq:thm:preorder_to_partial_order/equivalence}, it is easy to factor \( \mscrP \) by \eqref{eq:thm:preorder_to_partial_order/equivalence} and obtain a partially ordered set. In the language of graph theory, if we have \hyperref[def:graph_paths/directed_path]{circuits} that we may wish to avoid, we can contract each circuit into a single vertex, at which point the graph becomes acyclic.

  The formulation and proof of correctness of this process can be found in \fullref{thm:preorder_to_partial_order} and an example can be found in \cref{fig:ex:preorder_nonuniqueness}.

  \begin{figure}
    \hfill
    \includegraphics{figures/ex__preorder_nonuniqueness__preordered.pdf}
    \hfill
    \includegraphics{figures/ex__preorder_nonuniqueness__factorized.pdf}
    \hfill
    \hfill
    \caption{A preordered set and its induced poset.}
    \label{fig:ex:preorder_nonuniqueness}
  \end{figure}
\end{example}

\begin{proposition}\label{thm:preorder_to_partial_order}
  Let \( (\mscrP, \leq) \) be a preordered set. Define the relation \( \cong \) by
  \begin{equation}\label{eq:thm:preorder_to_partial_order/equivalence}
    x \cong y \iff x \leq y \T{and} y \leq x.
  \end{equation}

  That is, \( \cong \) is the intersection of the relation \( \leq \) with its \hyperref[def:binary_relation/converse]{converse}.

  Since \( \cong \) is an \hyperref[def:equivalence_relation]{equivalence relation} we can for the the quotient set \( \mscrP / \cong \). Define the relation \( \preceq \) on this quotient set by
  \begin{equation*}
    [x] \preceq [y] \iff x \leq y.
  \end{equation*}

  The pair \( (\mscrP / \cong, \preceq) \) is then a \hyperref[def:poset]{partially ordered set}.
\end{proposition}
\begin{proof}
  The relation \( \preceq \) is well-defined. Indeed, let \( x \cong x' \) and \( y \cong y' \), that is, both \( x \leq x' \) and \( x' \leq x \) and similarly for \( y \). If \( x \leq y \), by transitivity \( x \leq y \leq y' \). But \( x' \leq x \), hence \( x' \leq y' \).

  It is then clear that \( \preceq \) is a partial order because it inherits reflexivity and transitivity from \( \leq \) and antisymmetry is imposed by taking quotient sets --- equality in \( \mscrP / \cong \) holds precisely when \( \cong \) holds in \( \mscrP \).
\end{proof}

\begin{proposition}\label{thm:partial_order_category_correspondence}
  To every \hyperref[def:poset]{poset} there corresponds exactly one \hyperref[def:category_cardinality]{small} \hyperref[def:thin_category]{thin} \hyperref[def:skeletal_category]{skeletal} category.

  Compare this result to \fullref{thm:preorder_category_correspondence}.
\end{proposition}
\begin{proof}
  The statement follows from \fullref{thm:preorder_category_correspondence} by noting that the factorization in \fullref{thm:preorder_to_partial_order} makes the corresponding category skeletal.
\end{proof}

\begin{definition}\label{def:poset_extremal_points}
  We introduce the following terminology for extremal elements of a poset \( (\mscrP, \leq) \). Analogous definition can be given for preordered sets but the nonuniqueness problems outlined in \fullref{ex:preorder_nonuniqueness} highlight that there are difficulties in doing so.

  The notions on the left and on the right are \hyperref[def:poset/duality]{dual} but we discuss both nonetheless.

  \begin{thmenum}
    \thmitem{def:poset_extremal_points/upper_and_lower_bounds}\mcite[2]{Gratzer1978}
    \begin{minipage}[t]{0.45\textwidth}
      An \term{upper bound} for the set \( A \subseteq \mscrP \) is an element \( x_0 \in \mscrP \) such that \( x \leq x_0 \) for every \( x \in A \). Note that \( x_0 \) does not in general belong to \( A \).

      If \( A \) has at least one upper bound, it is called \term{bounded from above}.

      Every element is vacuously an upper bound of \( A = \varnothing \).

      In \cref{fig:ex:def:hasse_diagram}, the set \( A = \set{ a, b } \) is bounded from above by both \( d \) and \( e \) but the entire poset has no upper bound.
    \end{minipage}
    \hspace{0.02\textwidth}
    \begin{minipage}[t]{0.45\textwidth}
      Dually, \( x_0 \in \mscrP \) is a \term{lower bound} of \( A \) if \( x_0 \leq x \) for every \( x \in A \). If \( A \) has a lower bound, it is called \term{bounded from below}.

      If \( A \) is bounded both from below and from above, we say that \( A \) is \term{bounded}.

      Every element is vacuously a lower bound of \( A = \varnothing \). Hence the empty set is bounded.

      In \cref{fig:ex:def:hasse_diagram}, the entire poset has no lower bound. The set \( A = \set{ c, d } \) is bounded from below by \( a \) but not from above, hence \( A \) is not bounded.
    \end{minipage}

    \thmitem{def:poset_extremal_points/maximal_and_minimal_element}
    \begin{minipage}[t]{0.45\textwidth}
      A \term{maximal element} for the set \( A \subseteq \mscrP \) is a member \( x_0 \) of \( A \) such that there is no greater element in \( A \) than \( x_0 \). More precisely, \( x_0 \) is a maximal element of \( A \) if for every element \( x \in A \) such that \( x \leq x_0 \) we have \( x = x_0 \).

      In \cref{fig:ex:def:hasse_diagram}, the entire poset has two incomparable maximal elements --- \( c \) and \( e \).
    \end{minipage}
    \hspace{0.02\textwidth}
    \begin{minipage}[t]{0.45\textwidth}
      The member \( x_0 \in A \) is a \term{minimal element} of \( A \) if for every \( x \in A \) such that \( x \leq x_0 \) we have \( x = x_0 \).

      The empty set cannot have maximal or minimal elements because it has no members.

      In \cref{fig:ex:def:hasse_diagram}, the entire poset has two incomparable minimal elements --- \( a \) and \( b \).
    \end{minipage}

    \thmitem{def:poset_extremal_points/maximum_and_minimum}
    \begin{minipage}[t]{0.45\textwidth}
      The \term{maximum} or \term{greatest element} of \( A \subseteq \mscrP \), if it exists, is an upper bound of \( A \) that belongs to \( A \). A maximum is necessarily a maximal element because \( x_0 \leq x \) only holds for \( x = x_0 \), which also demonstrates uniqueness of \( x_0 \). See \fullref{ex:unique_maximal_element_that_is_not_maximum} for a unique maximal element that is not a maximum.
    \end{minipage}
    \hspace{0.02\textwidth}
    \begin{minipage}[t]{0.45\textwidth}
      The \term{minimum} of \( A \), also called \term{smallest element} or \term{least element}, is a lower bound that belongs to \( A \).

      The empty set cannot have a maximum or minimum because it has no members.

      In \cref{fig:ex:def:hasse_diagram}, the entire poset has no maximum but the set \( A = \mscrP \setminus \set{ b } = \set{ a, b, d, e } \) has \( a \) as its minimum.
    \end{minipage}

    \thmitem{def:poset_extremal_points/supremum_and_infimum}\mcite[2]{Gratzer1978}
    \begin{minipage}[t]{0.45\textwidth}
      The \term{supremum} \( \sup A \) of \( A \subseteq \mscrP \), if it exists, is its least upper bound of \( A \), i.e. the \hyperref[def:poset_extremal_points/maximum_and_minimum]{minimum} of the set of its \hyperref[def:poset_extremal_points/upper_and_lower_bounds]{upper bounds}.

      In \cref{fig:ex:def:hasse_diagram}, the entire poset has no upper bound so it cannot possibly have a supremum. Obviously every supremum is a maximum but the converse is not true. We already noted that both \( d \) and \( e \) are upper bounds of the set \( \set{ a, b } \) and since \( d \leq e \), we conclude that \( d \) is the supremum of \( \set{ a, b } \), yet the set \( \set{ a, b } \) has no maximum.
    \end{minipage}
    \hspace{0.02\textwidth}
    \begin{minipage}[t]{0.45\textwidth}
      The \term{infimum} \( \inf A \) of \( A \subseteq \mscrP \) is its greatest lower bound.

      An infimum may fail to exist because the set of lower bounds is nonempty but has no maximum. This can happen if, for example, we had \( b \leq c \) in \cref{fig:ex:def:hasse_diagram}, in which case both \( a \) and \( b \) would be maximal lower bounds of \( \set{ c, d } \) but none of them would be a greatest lower bound because they are incomparable.
    \end{minipage}

    \thmitem{def:poset_extremal_points/top_and_bottom}\mcite[2]{Gratzer1978}
    \begin{minipage}[t]{0.45\textwidth}
      If it exists, the maximum of the entire poset \( \mscrP \) is usually denoted by \( \top \) and called the \term{global maximum} or \term{top} element of \( \mscrP \). Since \( \top \) is the maximum of \( \mscrP \), it is also the supremum of \( \mscrP \).

      Since every member of \( \mscrP \) is a lower bound of \( \varnothing \), the greatest lower bound is the maximum of \( \mscrP \).

      In conclusion,
      \begin{equation*}
        \top = \max \mscrP = \sup \mscrP = \inf \varnothing.
      \end{equation*}
    \end{minipage}
    \hspace{0.02\textwidth}
    \begin{minipage}[t]{0.45\textwidth}
      Dually, the minimum of \( \mscrP \) is usually denoted by \( \bot \) and called the \term{global minimum} or \term{bottom} element of \( \mscrP \).

      The supremum of the empty set is the least of the upper bounds of the empty set, i.e. the minimum of \( \mscrP \), which is \( \bot \).

      In conclusion,
      \begin{equation*}
        \bot = \min \mscrP = \inf \mscrP = \sup \varnothing.
      \end{equation*}
    \end{minipage}
  \end{thmenum}
\end{definition}

\begin{example}\label{ex:unique_maximal_element_that_is_not_maximum}\mcite{MathSE:unique_maximal_element_that_is_not_maximum}
  For a more extreme example of the interplay between maximal elements and maxima, adjoin \( \BbbZ \) under the usual order with a new sentinel element \( \star \). Define \( \star \) to satisfy reflexivity but not be in relation with any integer. Then \( \star \) is a unique maximal element of \( \BbbZ \cup \set{ \star } \), yet the latter set has no largest element.

  This phenomenon is impossible in finite posets where a unique maximal element is always a maximum.
\end{example}

\begin{definition}\label{def:poset_chain_and_antichain_and_antichain}\mcite[2]{Gratzer1978}
  A \term{chain} in a poset is a subset in which every two elements are comparable. The \term{length} of a chain \( A \) is the \hyperref[def:cardinal]{cardinal number} \( \abs{A} - 1 \). The \term{length} of a poset, if it exists, is the maximum among the lengths of all its chains. The length of a poset is also called its \term{height} because of how Hasse diagrams are drawn..

  An \term{antichain} is a subset in which no two elements are comparable. The \term{width} of a poset, if it exists, is the maximum among the cardinalities of its antichains.

  The height of the poset in \cref{fig:ex:def:hasse_diagram} is \( 2 \) and it is reached by the chains \( \set{ a, d, e } \) and \( \set{ b, d, e } \). The width is \( 2 \) are is reached by \( \set{ a, b } \), \( \set{ b, e } \), \( \set{ c, d } \) and \( \set{ c, e } \).
\end{definition}

\begin{definition}\label{def:poset_interval}\mcite{nLab:order_topology}
  Fix a \hyperref[def:poset]{poset set} \( (\mscrP, \leq) \). For any \( a, b \in \mscrP \) with \( a \leq b \), we define the following posets:

  \begin{thmenum}
    \thmitem{def:poset_interval/closed} The \term{closed interval}
    \begin{equation*}
      \mathllap{ [a, b] } \coloneqq \mathrlap{ \set{ x \in \mscrP \given a \leq x \leq b }. }
    \end{equation*}

    \thmitem{def:poset_interval/open} The \term{open interval}
    \begin{equation*}
      \mathllap{ (a, b) } \coloneqq \mathrlap{ \set{ x \in \mscrP \given a < x < b }. }
    \end{equation*}

    \thmitem{def:poset_interval/half_open} The \term{half-open intervals}
    \begin{equation*}
      \begin{aligned}
        \mathllap{ (a, b] } &\coloneqq \mathrlap{ \set{ x \in \mscrP \given a < x \leq b }, }
        \\
        \mathllap{ [a, b) } &\coloneqq \mathrlap{ \set{ x \in \mscrP \given a \leq x < b }. }
      \end{aligned}
    \end{equation*}

    \thmitem{def:poset_interval/ray} For a sentinel symbol \( \infty \) not in \( \mscrP \), define the \term{open rays}
    \begin{equation*}
      \begin{aligned}
        \mathllap{ (a, \infty) }  &\coloneqq \mathrlap{ \set{ x \in \mscrP \given b > a }, }
        \\
        \mathllap{ (-\infty, b) } &\coloneqq \mathrlap{ \set{ x \in \mscrP \given x < b } }
      \end{aligned}
    \end{equation*}
    and the \term{closed rays}
    \begin{equation*}
      \begin{aligned}
        \mathllap{ [a, \infty) }  &\coloneqq \mathrlap{ \set{ x \in \mscrP \given x \geq a }, }
        \\
        \mathllap{ (-\infty, b] } &\coloneqq \mathrlap{ \set{ x \in \mscrP \given x \leq b }. }
      \end{aligned}
    \end{equation*}

    These rays also called (open or closed) \term{initial segments} and \term{final segments}, respectively.

    We can alternatively denote initial segments by \( \mscrP_{<x} \) without introducing any sentinel symbols.
  \end{thmenum}
\end{definition}

\begin{definition}\label{def:totally_ordered_set}
  We say that a partially ordered set is \term{totally ordered} if either the nonstrict order \( \leq \) is \hyperref[def:binary_relation/total]{total} or if the strict order \( < \) is \hyperref[def:binary_relation/trichotomic]{trichotomic}.

  The theory, homomorphisms and category are obtained analogously to \fullref{def:poset} but with either of these additional axiom sets.
\end{definition}
\begin{proof}
  Equivalence between nonstrict and strict total orders follows directly from the compatibility condition \eqref{def:poset/compatibility_nonstrict}.
\end{proof}

\begin{definition}\label{def:order_topology}\mcite{nLab:order_topology}
  Let \( (\mscrP, \leq) \) be a \hyperref[def:poset]{totally ordered set}. The \term{order topology} induced by \( \leq \) is the topology generated by the \hyperref[def:topological_subbase]{subbase} of open \hyperref[def:poset_interval/ray]{rays}
  \begin{equation*}
    P \coloneqq \set{ (a, \infty) \given a \in \mscrP } \cup \set{ (-\infty, b) \given b \in \mscrP }.
  \end{equation*}
\end{definition}

\begin{lemma}[Zorn's lemma]\label{thm:zorns_lemma}\mcite[63]{Gratzer1978}
  If any \hyperref[def:poset_chain_and_antichain]{chain} in a \hyperref[def:poset]{poset} has an upper \hyperref[def:preordered_set/upper_and_lower_bounds]{bound}, there exists a \hyperref[def:preordered_set/maximal_and_minimal_element]{maximal} set in \( X \).

  Note that Zorn's lemma is usually stated and used only in a \hyperref[thm:subsets_form_boolean_algebra]{lattice of sets}.

  This theorem is equivalent to \fullref{thm:aoc}.
\end{lemma}
