\subsection{Partial orders}\label{subsec:partial_orders}

\begin{definition}\label{def:poset}
  Fix a set \( \mscrP \). A \term{partially ordered set} structure, also called a \term{poset} structure, can be defined in the following equivalent ways:
  \begin{thmenum}[series=def:poset]
    \thmitem{def:poset/nonstrict} A \hyperref[def:preordered_set]{preorder} \( \leq \) on \( \mscrP \) such that \( \leq \) is \hyperref[def:binary_relation/antisymmetric]{antisymmetric} in addition to being \hyperref[def:binary_relation/reflexive]{reflexive} and \hyperref[def:binary_relation/transitive]{transitive}. This definition is the more common one. If we wish to distinguish it from the other definition, we call such a relation a \term{nonstrict partial order}.

    \thmitem{def:poset/strict} An \hyperref[def:binary_relation/irreflexive]{irreflexive} and \hyperref[def:binary_relation/transitive]{transitive} binary relation \( < \) on \( \mscrP \). This relation is called a \term{strict partial order}.
  \end{thmenum}

  If both relations are present, in order for the them to be equivalent, \( \leq \) must be the union of \( < \) and the \hyperref[def:binary_relation/diagonal]{diagonal} \( \Delta \). This condition corresponds to the following axiom:
  \begin{equation}\label{def:poset/compatibility_nonstrict}
    (x \leq y) \leftrightarrow \parens[\Big]{(x < y) \vee (x = y)}.
  \end{equation}

  By adding \( \placeholder \wedge \neg (x = y) \) to both sides of \eqref{def:poset/compatibility_nonstrict}, using \fullref{thm:de_morgans_laws} and taking irreflexivity of \( < \) into account, we obtain
  \begin{equation}\label{def:poset/compatibility_strict}
    (x < y) \leftrightarrow \parens[\Big]{(x \leq y) \wedge \neg (x = y)}.
  \end{equation}

  \begin{thmenum}[resume=def:poset]
    \thmitem{def:poset/theory} Since we can interdefine nonstrict and strict orders, it makes little sense to study separately theories for the two. Thus to obtain the \term{theory of posets}, we add to the \hyperref[def:preordered_set/theory]{theory of preordered sets}:
    \begin{itemize}
      \item \hyperref[def:binary_relation/antisymmetric]{Antisymmetry} for \( \leq \).
      \item A single binary predicate symbol \( < \).
      \item Either of the compatibility condition \eqref{def:poset/compatibility_nonstrict} or \eqref{def:poset/compatibility_strict} (it is unnecessary to add both).
    \end{itemize}

    We can also add \hyperref[def:binary_relation/irreflexive]{irreflexivity} and \hyperref[def:binary_relation/transitive]{transitivity} for \( < \) but that would also be redundant.

    \thmitem{def:poset/homomorphism} The \hyperref[def:first_order_homomorphism]{homomorphisms} in the theory are again \hyperref[def:poset/homomorphism]{monotone maps}. If we instead build a theory of strict partial orders without \( \leq \), homomorphisms are instead the more restrictive strict monotone maps.

    \thmitem{def:poset/substructure} As for preorders, any subset of a poset is itself a poset.

    \thmitem{def:poset/category} We denote the \hyperref[def:category_of_first_order_models]{category of models} by \( \cat{Pos} \). It is a full subcategory of the \hyperref[def:preordered_set/category]{category of preordered sets}.
  \end{thmenum}
\end{definition}
\begin{proof}
  \ImplicationSubProof{def:poset/nonstrict}{def:poset/strict} Let \( \leq \) be a nonstrict partial order. We will show that \( < \) is a strict partial order.

  \begin{itemize}
    \item The relation \( < \) is \hyperref[def:binary_relation/transitive]{transitive}. To see this, let \( x < y \) and \( y < z \). In particular, \( x \leq y \) and \( y \leq z \). By transitivity, \( x \leq z \).

    Additionally, \( x \neq y \) and \( y \neq z \). Assume\DNE that \( x = z \). By reflexivity of \( \leq \), we have \( z \leq x \) and, since \( y \leq z \), by transitivity we obtain \( y \leq x \). But since \( x \leq y \), by the antisymmetry of \( \leq \), we have \( x = y \), which contradicts the assumption that \( x < y \).

    Therefore \( x < z \).

    \item \hyperref[def:binary_relation/irreflexive]{Irreflexivity} of \( < \) follows directly from reflexivity of \( \leq \) and the compatibility condition.
  \end{itemize}

  Since the right side is false, the left side \( x < x \) is also false.

  \ImplicationSubProof{def:poset/strict}{def:poset/nonstrict} Let \( < \) be a strict partial order. We will show that \( \leq \) is a nonstrict partial order.

  \begin{itemize}
    \item To see \hyperref[def:binary_relation/reflexive]{reflexivity}, fix \( x \in \mscrP \) and assume\DNE that \( x \not\leq x \). Then \( x \neq x \) which contradicts the reflexivity of equality. Hence \( x \leq x \).

    \item To see \hyperref[def:binary_relation/antisymmetric]{antisymmetry}, let \( x \leq y \) and \( y \leq x \), that is, either \( x = y \) or both \( x < y \) and \( y < x \) hold. Assume\DNE the latter. By the transitivity of \( \leq \), we have \( x < x \), which contradicts the irreflexivity of \( < \). Hence \( x = y \).

    \item To see \hyperref[def:binary_relation/transitive]{transitivity}, let \( x \leq y \) and \( y \leq z \). Then we have four cases depending on which of \( x \), \( y \) and \( z \) are equal. Since both relations \( < \) and \( = \) are transitive, it follows that in all four cases \( x \leq z \).
  \end{itemize}
\end{proof}

\begin{proposition}\label{thm:preorder_to_partial_order}
  Let \( (\mscrP, \leq) \) be a preordered set. Define the relation \( \cong \) by
  \begin{equation*}
    x \cong y \iff x \leq y \T{and} y \leq x.
  \end{equation*}

  That is, \( \cong \) is the intersection of the relation \( \leq \) with its \hyperref[def:binary_relation/converse]{converse}.

  Since \( \cong \) is an \hyperref[def:equivalence_relation]{equivalence relation} we can for the the quotient set \( \mscrP / \cong \). Define the relation \( \preceq \) on this quotient set by
  \begin{equation*}
    [x] \preceq [y] \iff x \leq y.
  \end{equation*}

  The pair \( (\mscrP / \cong, \preceq) \) is then a \hyperref[def:poset]{partially ordered set}.
\end{proposition}
\begin{proof}
  The relation \( \preceq \) is well-defined. Indeed, let \( x \cong x' \) and \( y \cong y' \), that is, both \( x \leq x' \) and \( x' \leq x \) and similarly for \( y \). If \( x \leq y \), by transitivity \( x \leq y \leq y' \). But \( x' \leq x \), hence \( x' \leq y' \).

  It is then clear that \( \preceq \) is a partial order because it inherits reflexivity and transitivity from \( \leq \) and antisymmetry is imposed by taking quotient sets --- equality in \( \mscrP / \cong \) holds precisely when \( \cong \) holds in \( \mscrP \).
\end{proof}

\begin{proposition}\label{thm:partial_order_category_correspondence}
  To every \hyperref[def:poset]{poset} there corresponds exactly one \hyperref[def:category_cardinality]{small} \hyperref[def:thin_category]{thin} \hyperref[def:skeletal_category]{skeletal} category.

  Compare this result to \fullref{thm:preorder_category_correspondence}.
\end{proposition}
\begin{proof}
  The statement follows from \fullref{thm:preorder_category_correspondence} by noting that the factorization in \fullref{thm:preorder_to_partial_order} makes the corresponding category skeletal.
\end{proof}
