\subsection{Diagonalization}\label{subsec:diagonalization}

\begin{definition}\label{def:diagonal_matrix}
  We say that the square matrix\Tinyref{def:array/matrix} \( \{ a_{i,j} \}_{i,j=1}^{n,n} \) over a field \( F \) is \Def{diagonal} if \( a_{i,j} = 0 \) whenever \( i \neq j \), that is, only the main diagonal has nonzero elements:
  \begin{equation*}
    \begin{pmatrix}
      a_{1,1} & 0       & \cdots & 0 \\
      0       & a_{2,2} & \cdots & 0 \\
      \vdots  & \vdots  & \ddots & \vdots \\
      0       & 0       & \cdots & a_{n,n}
    \end{pmatrix}.
  \end{equation*}

  If all elements along the main diagonal are \( 1 \), we call the matrix the \Def{identity matrix} of order \( n \) and denote it by
  \begin{equation*}
    E_n \coloneqq
    \begin{pmatrix}
      1       & 0       & \cdots & 0 \\
      0       & 1       & \cdots & 0 \\
      \vdots  & \vdots  & \ddots & \vdots \\
      0       & 0       & \cdots & 1
    \end{pmatrix}.
  \end{equation*}
\end{definition}
