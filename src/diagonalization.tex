\subsection{Diagonalization}\label{subsec:diagonalization}

\begin{definition}\label{def:eigenpair}\mcite[def. 4.17]{Rudin1991Functional}
  Let \( L: V \to V \) be a \hyperref[def:semimodule/homomorphism]{linear endomorphism} over the \hyperref[def:vector_space]{vector space} \( V \) over \( \BbbK \).

  An \term{eigenpair} \( (\lambda, x) \) consists of an \term{eigenvalue} \( \lambda \in \BbbK \) and a \hi{nonzero} \term{eigenvector} \( x \in V \) such that
  \begin{equation*}
    Lx = \lambda x.
  \end{equation*}
\end{definition}

\begin{remark}\label{rem:eigenpairs_via_invertibility}
  The \hyperref[def:eigenpair]{eigenpair equation}
  \begin{equation*}
    Lx = \lambda x
  \end{equation*}
  can be rewritten as
  \begin{equation*}
    (L - \lambda \cdot \id) x = \vect 0.
  \end{equation*}

  We can regard the above as a \hyperref[rem:system_of_equations]{system of equations}. By \fullref{thm:homogeneous_linear_equations_solutions}, there exists a nonzero solution, i.e. an eigenvector corresponding to \( \lambda \), if and only if the map
  \begin{equation*}
    L - \lambda \cdot \id
  \end{equation*}
  is an isomorphism.

  If \( V \) is finite-dimensional, by \fullref{thm:square_matrix_left_invertible_iff_right_invertible}, this map is injective if and only if it is surjective. Otherwise, it may fail to be either injective or surjective.
\end{remark}

\begin{definition}\label{def:characteristic_polynomial}\mimprovised
  The \term{characteristic polynomial} of a square matrix \( A \) is
  \begin{equation*}
    p(\lambda) = \det(A - \lambda I_n).
  \end{equation*}
\end{definition}

\begin{proposition}\label{thm:eigenvalues_and_characteristic_polynomials}
  The \hyperref[def:eigenpair]{eigenvalues} of a square matrix are precisely the \hyperref[def:polynomial_root]{roots} of its \hyperref[def:characteristic_polynomial]{characteristic polynomial}.
\end{proposition}
\begin{proof}
  Follows from \fullref{rem:eigenpairs_via_invertibility}.
\end{proof}

\begin{definition}\label{def:point_spectrum}\mcite[10.32]{Rudin1991Functional}
  The set of all \hyperref[def:eigenpair]{eigenvalues} of a linear endomorphism is called its \term{point spectrum}.
\end{definition}
