\subsection{Bilinear forms}\label{subsec:bilinear_forms}

\begin{definition}\label{def:bilinear_form}\cite[249]{Knapp2016BAlg}
  Let \( M \) and \( N \) be left \( R \)-modules. A \Def{bilinear form} is a multilinear\Tinyref{def:multilinear_function} function of type \( M \times N \to R \). Usually \( M = N \), in which case we say that the function is a bilinear form over \( M \).

  If instead of being linear in its second argument the function is semilinear\Tinyref{def:linear_operator}, we say that it is a \Def{sesquilinear form}.
\end{definition}

\begin{definition}\label{def:duality_pairing}
  Let \( M \) and \( N \) be left \( R \)-modules. A \Def{duality pairing} \( \Prod - -: M \times N \to R \) is a bilinear form such that both families of functions
  \begin{itemize}
    \item \( \{ m \to \Prod n m \colon n \in N \} \)
    \item \( \{ n \to \Prod n m \colon m \in M \} \)
  \end{itemize}
  vanish nowhere\Tinyref{def:functions_vanish_nowhere}.

  The \Def{canonical duality pairing} of a vector space \( V \) over \( F \) is
  \begin{align*}
    &\Prod - -: V^* \times V \to F \\
    &\Prod {x^*} x \to x^*(x).
  \end{align*}
\end{definition}

\begin{theorem}\label{thm:bilinear_form_matrix_presentation}
  Fix a commutative unital ring \( R \) and a bilinear form \( L: R^n \times R^m \to R \). Then there exists a matrix \( A \in R^{n \times m} \) such that
  \begin{equation*}
    L(x, y) \coloneqq x^T A y.
  \end{equation*}

  This matrix is called the generalized \Def{Gram matrix}.
\end{theorem}
\begin{proof}
  Denote by \( e_1, \ldots, e_n \) the basis of \( R^n \) and by \( f_1, \ldots, f_m \) the basis of \( R^m \).

  Define the matrix \( A = \{ a_{i,j} \}_{i,j=1}^{n,m} \) by
  \begin{equation*}
    a_{i,j} \coloneqq L(e_i, f_j).
  \end{equation*}

  For any fixed basis vector \( e_i, i = 1, \ldots, n \) of \( R^n \), we have
  \begin{equation*}
    L(e_i, y)
    =
    \sum_{j=1}^m y_i L(e_i, f_j)
    =
    y_i a_{(i,-)},
  \end{equation*}
  where \( a_{(i,-)} \) is the \( i \)-th row of \( A \).

  Thus for an arbitrary \( x \in R^n \)
  \begin{equation*}
    L(x, y)
    =
    \sum_{i=1}^n x_i L(e_i, y)
    =
    \sum_{i=1}^n x_i (a_{(i,-)} y)
    =
    \left( \sum_{i=1}^n x_i a_{(i,-)} \right) y
    =
    x^T A y.
  \end{equation*}
\end{proof}

\begin{corollary}\label{thm:bilinear_forms_isomorphic_to_matrices}
  Fix a commutative unital ring \( R \). The vector space of bilinear forms of type \( R^n \times R^m \to R \) is isomorphic to the matrix space \( A \in R^{n \times m} \).
\end{corollary}
