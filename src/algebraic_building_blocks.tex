\subsection{Simple algebraic structures}\label{subsec:simple_algebraic_structures}

We list here several basic algebraic structures, we will use mostly as building blocks for more complicated structures.

As discussed in \fullref{rem:first_order_model_notation}, denoting all operations explicitly is cumbersome, and we will usually avoid it.

\begin{definition}\label{def:pointed_set}\mcite[26]{MacLane1994}
  The simplest algebraic structure is a \term{pointed set}. It is simply a nonempty set \( \mscrX \) equipped with a distinguished element \( e \). It is an algebraic structure because \( e \) can be regarded as the sole value of a nullary function \( *: \mscrX^0 \to \mscrX \).

  We will call \( e \) the \term{origin} of \( \mscrX \) based on the terminology for \hyperref[def:euclidean_plane_coordinate_system/origin]{affine coordinate systems}.

  \begin{thmenum}
    \thmitem{def:pointed_set/theory} Pointed sets can also be viewed as \hyperref[def:first_order_semantics/satisfiability]{models} of an empty \hyperref[def:first_order_theory]{theory} for a \hyperref[def:first_order_language]{first-order logic language} with a constant symbol, i.e. a nullary \hyperref[def:first_order_language/func]{functional symbol}.

    \thmitem{def:pointed_set/homomorphism} A \hyperref[def:first_order_homomorphism]{homomorphism} between the pointed sets \( (\mscrX, e_{\mscrX}) \) and \( (\mscrY, e_{\mscrY}) \) is, explicitly, a function \( \varphi: \mscrX \to \mscrY \) that satisfies
    \begin{equation}\label{eq:def:pointed_set/homomorphism}
      \varphi(e_{\mscrX}) = e_{\mscrY}.
    \end{equation}

    \thmitem{def:pointed_set/submodel} The set \( S \subseteq \mscrX \) is a \hyperref[def:first_order_substructure]{substructure} if \( \mscrX \) if \( e \in S \).

    \thmitem{def:pointed_set/trivial} The \hyperref[thm:substructures_form_complete_lattice/bottom]{trivial} pointed set is, up to an isomorphism, the set \( \set{ e } \).

    It is a \hyperref[def:universal_objects/initial]{zero object} in \( \cat{Set_*} \) as discussed in \fullref{ex:def:universal_objects/grp}.

    \thmitem{def:pointed_set/category} We denote the \hyperref[def:category_of_small_first_order_models]{category of \( \mscrU \)-small models} for this theory by \( \ucat{Set}_* \).
  \end{thmenum}
\end{definition}

\begin{definition}\label{def:pointed_set_kernel}\mimprovised
  The \term{kernel} \( \ker(\varphi) \) of a \hyperref[def:pointed_set/homomorphism]{pointed set homomorphism} \( \varphi: \mscrX \to \mscrY \) is the \hyperref[def:zero_locus]{zero locus} of \( \varphi \). That is, \hyperref[thm:def:function/properties/preimage]{preimage} \( \varphi^{-1}(e_\mscrY) \).

  These are the \hyperref[def:zero_morphisms/kernel]{categorical kernels} in \hyperref[def:pointed_set/category]{\( \cat{Set_*} \)}. This is discussed in \fullref{ex:zero_morphisms_in_grp/kernel} for \hyperref[def:group]{groups}, where the argument only depends on groups being pointed sets.
\end{definition}

\begin{definition}\label{def:set_with_involution}\mimprovised
  A \term{set with an involution} is a \hyperref[def:set]{set} \( \mscrX \) with a unary operation \( (\anon)^{-1} \) such that
  \begin{equation*}
    (x^{-1})^{-1} = x
  \end{equation*}
  for every \( x \in \mscrX \).

  Such an operation is called, surprisingly, an \term{involution}.

  \begin{thmenum}
    \thmitem{def:set_with_involution/theory} We define the theory of sets with involution as a theory over the language consisting of a single unary functional symbol \( \anon^{-1} \) and the sole axiom
    \begin{equation}\label{eq:def:set_with_involution/theory/axiom}
      (\xi^{-1})^{-1} \doteq \xi.
    \end{equation}

    \thmitem{def:set_with_involution/homomorphism} A \hyperref[def:first_order_homomorphism]{homomorphism} between sets with involutions \( \mscrX \) and \( \mscrY \) is a function \( \varphi: \mscrX \to \mscrY \) satisfying
    \begin{equation}\label{eq:def:set_with_involution/homomorphism}
      \varphi(x^{-1})
      =
      \varphi(x)^{-1}.
    \end{equation}

    \thmitem{def:set_with_involution/submodel} Any subset of a set with involution is again a set with involution.

    \thmitem{def:set_with_involution/trivial} The \hyperref[thm:substructures_form_complete_lattice/bottom]{trivial} set with involution is the empty set.

    \thmitem{def:set_with_involution/category} We denote the \hyperref[def:category_of_small_first_order_models]{category of \( \mscrU \)-small models} for this theory by \( \ucat{Inv} \).
  \end{thmenum}
\end{definition}

\begin{definition}\label{def:magma}
  A \term{magma} is a set \( \mscrM \) equipped with a \hyperref[def:multi_valued_function/arguments]{binary function} \( \cdot: \mscrM \times \mscrM \to \mscrM \), called the \term{magma operation}. Unless specified otherwise, we denote this operation by juxtaposition as \( xy \) instead of \( x \cdot y \).

  We often call the operation \term{multiplication} or, in the case of \hyperref[def:endomorphism_monoid]{endomorphism monoids} --- \term{composition}. See also the notes in \fullref{rem:additive_magma} regarding additive magmas and in \fullref{def:monoid_delooping} regarding the order of operands.

  \begin{thmenum}[series=def:magma]
    \thmitem{def:magma/theory} In analogy to the \hyperref[def:pointed_set/theory]{theory of pointed sets}, we can define the theory of magmas as an empty theory over a language with a single \hyperref[rem:first_order_formula_conventions/infix]{infix} binary functional symbol.

    \thmitem{def:magma/homomorphism} A \hyperref[def:first_order_homomorphism]{homomorphism} between the magmas \( (\mscrM, \cdot_{\mscrM}) \) and \( (\mscrN, \cdot_{\mscrN}) \) is, explicitly, a function \( \varphi: \mscrM \to \mscrN \) such that
    \begin{equation}\label{eq:def:magma/homomorphism}
      \varphi(x \cdot_{\mscrM} y) = \varphi(x) \cdot_{\mscrN} \varphi(y)
    \end{equation}
    for all \( x, y \in \mscrM \).

    \thmitem{def:magma/submodel} The set \( S \subseteq \mscrM \) is a \hyperref[def:first_order_substructure]{first-order submodel} of \( \mscrM \) if it is closed under the magma operation. That is, if \( x, y \in S \) implies \( xy \in S \).

    We call \( S \) a \term{submagma} of \( \mscrM \).

    \thmitem{def:magma/category} We denote the \hyperref[def:category_of_small_first_order_models]{category of \( \mscrU \)-small models} for the theory of magmas by \( \ucat{Mag} \).

    \thmitem{def:magma/trivial} The \hyperref[thm:substructures_form_complete_lattice/bottom]{trivial magma} is the empty set with an empty operation. It is the unique \hyperref[def:universal_objects/initial]{zero object} in \( \cat{Mag} \).

    \thmitem{def:magma/opposite} The \term{opposite magma} of \( (\mscrM, \cdot) \), also called the \term{dual magma}, is the magma \( (\mscrM, \star) \) with multiplication reversed:
    \begin{equation*}
      x \star y \coloneqq y \cdot x.
    \end{equation*}

    We denote the opposite magma by \( \mscrM^{-1} \).

    \thmitem{def:magma/exponentiation} We define an additional \term{exponentiation} operation for positive integers \( n \) inductively as
    \begin{equation}\label{eq:def:magma/exponentiation}
      x^n \coloneqq \begin{cases}
        x,               & n = 1 \\
        x \cdot x^{n-1}, & n > 1
      \end{cases}
    \end{equation}

    \thmitem{def:magma/power_set} It is customary to perform magma operations with sets. That is, if \( A \) and \( B \) are sets in the magma \( \mscrM \), it is customary to write
    \begin{equation*}
      A \cdot B \coloneqq \set{ a \cdot b \colon a \in A, b \in B }.
    \end{equation*}

    This actually turns the power set \( \pow(\mscrM) \) into a magma, which we will call the \term{power set magma} of \( \mscrM \). This is especially useful with the convention \fullref{rem:singleton_sets} since it allows us to write \( aB \) for \( a \in M \) and \( B \subseteq M \).

    See \fullref{thm:power_set_magma_preservation}.
  \end{thmenum}

  We list some additional restrictions that are often imposed on magmas.
  \begin{thmenum}[resume=def:magma]
    \thmitem{def:magma/associative} We can add the (\hyperref[thm:implicit_universal_quantification]{universal closure} of) following axiom to the theory:
    \begin{equation}\label{eq:def:magma/associative}
      (\xi \cdot \eta) \cdot \zeta = \xi \cdot (\eta \cdot \zeta).
    \end{equation}

    If \eqref{eq:def:magma/associative} is satisfied, we say that the operation \( \cdot \) and, by extension, the magma itself, are \term{associative}. Associative magmas are usually called \term{semigroups}. Associativity imposes no additional restrictions on the homomorphisms, hence semigroups are a \hyperref[def:subcategory]{full subcategory} of \( \cat{Mag} \).

    \thmitem{def:magma/commutative} Another common axiom is \term{commutativity}:
    \begin{equation}\label{eq:def:magma/commutative}
      \xi \cdot \eta \doteq \eta \cdot \xi.
    \end{equation}

    Commutative magmas also form a full subcategory. Obviously \( \mscrM = \mscrM^{-1} \) in a commutative magma.

    \thmitem{def:magma/idempotent} We say that the operation \( \cdot \) is \term{idempotent} if
    \begin{equation}\label{eq:def:magma/idempotent}
      \xi \cdot \xi \doteq \xi.
    \end{equation}

    \thmitem{def:magma/cancellative} We say that \( \cdot \) is \term{left-cancellative} if
    \begin{equation}\label{eq:def:magma/cancellative/left}
      \qforall \zeta (\zeta \cdot \xi \doteq \zeta \cdot \eta) \rightarrow \xi = \eta
      \quad
    \end{equation}
    and \term{right-cancellative} if
    \begin{equation}\label{eq:def:magma/cancellative/right}
      \qforall \zeta (\xi \cdot \zeta \doteq \eta \cdot \zeta) \rightarrow \xi = \eta
      \quad
    \end{equation}

    The operation is \term{cancellative} if it is both left and right cancellative. Cancellative magmas also form a full subcategory.
  \end{thmenum}
\end{definition}

\begin{example}\label{ex:def:magma}
  \hfill
  \begin{thmenum}
    \thmitem{ex:def:magma/composition} The quintessential example of a non-\hyperref[def:magma/commutative]{commutative} operation is \hyperref[def:multi_valued_function/composition]{composition} in any set of functions or, more generally, \hyperref[def:category/composition]{morphism composition} in any \hyperref[def:category]{category}.

    Composition is \hyperref[def:magma/associative]{associative}. \hyperref[def:magma/cancellative]{Cancellation} with respect to composition is discussed in \fullref{def:morphism_invertibility} and, for function composition, in \fullref{thm:function_invertibility_categorical}.

    \thmitem{ex:def:magma/midpoint} The operation
    \begin{equation*}
      (x, y) \mapsto \dfrac {x + y} 2
    \end{equation*}
    makes \( \BbbR \) a commutative and cancellative magma, which is not associative.
  \end{thmenum}
\end{example}

\begin{remark}\label{rem:additive_magma}
  General groups often arise as \hyperref[def:automorphism_group]{automorphism groups}, which are, for the most part, non-commutative, while abelian groups are usually used as the main building block for \hyperref[def:semiring/ring]{rings} and \hyperref[def:left_module]{modules}.

  To make a further distinction, if the operation is denoted by \( \cdot \) or juxtaposition, we say that the group is a \term{multiplicative group}, and if the operation is denoted by \( + \), we say that the group is an \term{additive group}. This terminology usually, but not necessarily, coincides with the group (or, more generally, the \hyperref[def:magma]{magma}) being \hyperref[def:magma/commutative]{commutative}.

  To make things explicit, a \term{multiplicative magma} is any magma as defined in \fullref{def:magma}. Compare this to \term{additive magmas}, where
  \begin{thmenum}
    \thmitem{rem:additive_magma/addition} The magma operation is denoted by \( + \) and called \term{addition}.

    \thmitem{rem:additive_magma/multiplication} The magma \hyperref[def:magma/exponentiation]{exponentiation operation} is denoted by \( \cdot \) or juxtaposition and called \term{multiplication}. Thus, multiplication is not defined for two elements of the magma, but defined for a positive integer and an element of the magma. In the case of a \hyperref[def:magma/commutative]{commutative} \hyperref[def:unital_magma/monoid]{monoid}, if multiplication is extended to two elements of the monoid, we instead talk about \hyperref[def:semiring]{semirings}.

    \thmitem{rem:additive_magma/identity} The \hyperref[def:magma_identity]{identity} is usually denoted by \( 0 \).

    \thmitem{rem:additive_magma/inverse} If an \hyperref[def:unital_magma_inverse_element]{inverse} of \( x \) exists, it is denoted by \( -x \) rather than \( x^{-1} \).
  \end{thmenum}
\end{remark}

\begin{proposition}\label{thm:power_set_magma_preservation}
  \hyperref[def:magma/associative]{Associativity} and \hyperref[def:magma/commutative]{commutativity} from a magma \( \mscrM \) are preserved in \( \pow(\mscrM) \), unlike \hyperref[def:magma/cancellative]{cancellation}.
\end{proposition}
\begin{proof}
  Associativity and commutativity are obviously preserved.

  To show that cancellation is not, consider the group \hyperref[def:group_of_integers_modulo]{\( \BbbZ_2 \)}. It is a cancellative magma by \fullref{thm:def:group/properties/cancellative}. Define the sets \( A \coloneqq \{ 0, 1 \} \) and \( B \coloneqq \{ 0 \} \). Then
  \begin{equation*}
    A + A = A = A + B,
  \end{equation*}
  however we cannot cancel \( A \) from the left because \( A \neq B \).
\end{proof}

\begin{proposition}\label{thm:magma_exponentiation_properties}
  Fix a magma \( \mscrM \). \hyperref[def:magma/exponentiation]{Magma exponentiation} in \( \mscrM \) has the following basic properties:

  \begin{thmenum}
    \thmitem{thm:magma_exponentiation_properties/commutativity} We have the following \hyperref[def:magma/commutative]{commutativity}-like property: for \( x \in M \) and \( n = 1, 2, \ldots \),
    \begin{equation}\label{eq:thm:magma_exponentiation_properties/commutativity}
      x^n = x x^{n-1} = x^{n-1} x.
    \end{equation}

    \thmitem{thm:magma_exponentiation_properties/distributivity} Exponentiation distributes over multiplication: for any member \( x \in M \) and any two positive integers \( n \) and \( m \),
    \begin{equation}\label{eq:thm:magma_exponentiation_properties/multiplication}
      x^{n + m} = x^n x^m.
    \end{equation}

    \thmitem{thm:magma_exponentiation_properties/repeated} For any member \( x \in M \) and any two positive integers \( n \) and \( m \),
    \begin{equation}\label{eq:thm:magma_exponentiation_properties/repeated}
      (x^n)^m = x^{nm}.
    \end{equation}
  \end{thmenum}
\end{proposition}
\begin{proof}
  \SubProofOf{thm:magma_exponentiation_properties/commutativity} We use induction on \( n \). The cases \( n = 1 \) and \( n = 2 \) are obvious. For \( n > 2 \), we have
  \begin{equation*}
    x^n
    \reloset {\eqref{eq:def:magma/exponentiation}} =
    x x^{n-1}
    \reloset {\T{ind.}} =
    x x^{n-2} x
    \reloset {\eqref{eq:def:magma/exponentiation}} =
    x^{n-1} x.
  \end{equation*}

  \SubProofOf{thm:magma_exponentiation_properties/distributivity} We use induction on \( n \). The case \( n = 1 \) follows directly from \eqref{eq:def:magma/exponentiation}. The case \( n > 1 \) follows from
  \begin{equation*}
    x^{n + m}
    \reloset {\eqref{eq:def:magma/exponentiation}} =
    x x^{n + (m - 1)}
    \reloset {\T{ind.}} =
    x x^{n-1} x^m
    \reloset {\eqref{eq:def:magma/exponentiation}} =
    x^n x^m.
  \end{equation*}

  \SubProofOf{thm:magma_exponentiation_properties/repeated} We use induction on \( n \). The case \( n = 1 \) is obvious and the rest follows from
  \begin{equation*}
    (x^n)^m
    \reloset {\eqref{eq:def:magma/exponentiation}} =
    x^n (x^n)^{m-1}
    \reloset {\T{ind.}} =
    x^n x^{n (m - 1)}
    \reloset {\eqref{eq:thm:magma_exponentiation_properties/multiplication}} =
    =
    x^{nm}.
  \end{equation*}
\end{proof}

\begin{definition}\label{def:preordered_magma}
  A \term{preordered magma} is a magma \( \mscrM \) equipped with a \hyperref[def:preordered_set]{preorder} \( \leq \) such that \( x \leq y \) implies \( xz \leq yz \) and \( zx \leq zy \) for all \( z \in M \).

  The category of small preordered magmas is both \hyperref[def:magma/category]{\( \cat{Mag} \)}-\hyperref[def:concrete_category]{concrete} and \hyperref[def:preordered_set/category]{\( \cat{PreOrd} \)}-concrete.
\end{definition}

\begin{proposition}\label{thm:preordered_magma_max_distributivity}
  In a \hyperref[def:preordered_magma]{preordered magma} \( \mscrM \),
  \begin{equation}\label{eq:thm:preordered_magma_max_distributivity}
    \max \set{a b, c d} \leq \max \set{a, c} \cdot \max \set{b, d}.
  \end{equation}
\end{proposition}
\begin{proof}
  Since \( a \leq \max \set{a, c} \), then
  \begin{equation*}
    ab
    \leq
    \max \set{a, c} \cdot b
    \leq
    \max \set{a, c} \cdot \set{b, d}
  \end{equation*}

  Analogously, \( cd \leq \max \set{a, c} \cdot \set{b, d} \) and
  \begin{equation*}
    \max \set{a b, c d} \leq \max \set{a, c} \cdot \set{b, d}.
  \end{equation*}
\end{proof}

\begin{definition}\label{def:topological_magma}
  A \term{topological magma} is a magma equipped with a \hyperref[def:topological_space]{topology} such that the magma operation is continuous.

  The category of small topological magmas is both \hyperref[def:magma/category]{\( \cat{Mag} \)}-\hyperref[def:concrete_category]{concrete} and \hyperref[def:category_of_small_topological_spaces]{\( \cat{Top} \)}-concrete.
\end{definition}

\begin{definition}\label{def:magma_identity}
  An element \( e \) of a magma \( \mscrM \) is called a \term{left identity} (resp. \term{right identity}) if \( ex = x \) for all \( x \in M \) (resp. \( xe = x \) for all \( x \in M \)).

  If \( e \) is simultaneously a left and right identity, we call a \term{two-sided identity} or simply \term{identity} of \( \mscrM \).
\end{definition}

\begin{proposition}\label{thm:magma_identity_unique}
  A two-sided \hyperref[def:magma_identity]{magma identity} \( e \), if it exists, is unique.
\end{proposition}
\begin{proof}
  If \( f \) is another identity, then \( e = ef = f \).
\end{proof}

\begin{definition}\label{def:unital_magma}
  A \hyperref[def:magma]{magma} with an \hyperref[def:magma_identity]{identity} is called \term{unital}. This makes it a \hyperref[def:pointed_set]{pointed set}. We can consider it as a pair \( (\mscrM, \cdot) \) rather than a triple \( (\mscrM, \cdot, e) \) because, by \fullref{thm:magma_identity_unique}, a two-sided identity is uniquely determined by the magma operation.

  \begin{thmenum}
    \thmitem{def:unital_magma/theory} The theory of unital magmas consists of the single axiom
    \begin{equation}\label{eq:def:unital_magma/theory/identity}
      \qforall \xi (e \cdot \xi \doteq \xi \wedge \xi \cdot e \doteq \xi)
    \end{equation}
    over the combined language of \hyperref[def:pointed_set/theory]{pointed sets} and \hyperref[def:magma/theory]{magmas}.

    \thmitem{def:unital_magma/homomorphism} A \hyperref[def:first_order_homomorphism]{homomorphism} between unital magmas is a function that satisfies both \eqref{eq:def:pointed_set/homomorphism} and \eqref{eq:def:magma/homomorphism}.

    \thmitem{def:unital_magma/category} The \hyperref[def:category_of_small_first_order_models]{category of \( \mscrU \)-small models} \( \ucat{Mag}_* \) of unital magmas is \hyperref[def:concrete_category]{concrete} with respect to both \hyperref[def:pointed_set/category]{\( \ucat{Set}_* \)} and \hyperref[def:magma/category]{\( \ucat{Mag} \)}.

    \thmitem{def:unital_magma/submodel} The set \( S \subseteq X \) is a \hyperref[def:first_order_substructure]{substructure} if \( \mscrX \) if \( e \in S \). This is equivalent to \( S \) being a pointed subset.

    We say that \( S \) is a \term{unital submagma}.

    \thmitem{def:unital_magma/trivial} The \hyperref[thm:substructures_form_complete_lattice/bottom]{trivial} unital magma is the \hyperref[def:pointed_set/trivial]{trivial pointed set} \( \set{ e } \).

    \thmitem{def:unital_magma/monoid} An \hyperref[eq:def:magma/associative]{associative} unital magma is called a \term{monoid}. The category \( \cat{Mon} \) of monoids is a full subcategory of \( \cat{Mag}_* \).

    \thmitem{def:unital_magma/exponentiation} We extend \hyperref[def:magma/exponentiation]{magma exponentiation} to all nonnegative integers by defining
    \begin{equation*}
      x^0 \coloneqq e.
    \end{equation*}

    \thmitem{def:unital_magma/power_set} The \hyperref[def:magma/power_set]{power set magma} \( \pow(\mscrM) \) of a unital magma \( \mscrM \) with identity \( e \) is again a unital magma with identity \( \set{ e } \).
  \end{thmenum}
\end{definition}

\begin{example}\label{ex:monoid_cancellation_not_preserved_by_homomorphism}\mcite{MathSE:magma_cancellation_not_preserved}
  \hyperref[def:unital_magma/homomorphism]{Monoid homomorphisms} may not preserve the \hyperref[def:magma/cancellative]{cancellation property}. For example, the \hyperref[def:set_of_natural_numbers]{natural numbers} \( \BbbN \) are a cancellative monoid under addition, as shown in \fullref{thm:def:natural_number_addition/properties}), but the magma homomorphism
  \begin{equation*}
    \begin{aligned}
      &h: (\BbbN, +) \to (\hyperref[thm:galois_field_existence]{\BbbF_2}, \max) \\
      &h(n) \coloneqq \begin{cases}
        0, &n = 0 \\
        1, &n > 0
      \end{cases}
    \end{aligned}
  \end{equation*}
  does not preserve the cancellative property.

  Indeed, \( \max\set{ 0, 1} = \max\set{ 1, 1 } \), but \( 0 \neq 1 \).
\end{example}

\begin{proposition}\label{thm:unital_magma_kernel_is_submagma}
  The \hyperref[def:pointed_set_kernel]{kernel} of a unital magma homomorphism \( \varphi: \mscrM \to \mscrN \) is a \hyperref[def:first_order_substructure]{unital submagma} of \( \mscrM \).
\end{proposition}
\begin{proof}
  By \eqref{eq:def:pointed_set/homomorphism}, \( e_{\mscrM} \in \ker(\varphi) \), therefore \( \ker(\varphi) \) inherits its unital magma structure from \( \mscrM \). It remains only to show that it is closed under the magma operation. But this is trivial since, if \( x, y \in \ker(\varphi) \), then
  \begin{equation*}
    \varphi(xy) = \varphi(x) \varphi(y) = e_{\mscrN} e_{\mscrN} = e_{\mscrN}.
  \end{equation*}
\end{proof}
