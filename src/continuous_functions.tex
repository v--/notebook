\subsection{Continuous functions}\label{subsec:continuous_functions}

\begin{definition}\label{def:continuous_function}
  Let \( (X, \Cal{T}) \) and \( (Y, \Cal{O}) \) be topological spaces\Tinyref{def:topological_space}. We say that the function\Tinyref{def:function} \( f: X \to Y \) is continuous if any of the equivalent conditions hold:
  \begin{defenum}
    \DItem{def:continuous_function/direct} For every open set \( V \in \Cal{O} \), the preimage\Tinyref{def:function_preimage} \( f^{-1}(V) \) is open.
    \DItem{def:continuous_function/closed} For every closed set \( V \in \Cal{F}_{\Cal{O}} \), the preimage \( f^{-1}(V) \) is closed.
    \DItem{def:continuous_function/base} There exists a base\Tinyref{def:topological_base} \( \Cal{B}_{\Cal{O}} \subseteq \Cal{O} \), such that for every \( V \in \Cal{B}_{\Cal{O}} \), the preimage \( f^{-1}(V) \) is open.
    \DItem{def:continuous_function/subbase} There exists a subbase\Tinyref{def:topological_subbase} \( \Cal{P}_{\Cal{O}} \subseteq \Cal{O} \), such that for every \( V \in \Cal{P}_{\Cal{O}} \), the preimage \( f^{-1}(V) \) is open.
    \DItem{def:continuous_function/local_base} There exist neighborhood systems\Tinyref{def:topological_local_base} \( \{ \Cal{B}_{\Cal{T}}(x) \colon x \in X \} \) and \( \{ \Cal{B}_{\Cal{O}}(y) \colon y \in Y \} \), such that for every point \( x \in X \) and for any \( V \in \Cal{B}_{\Cal{O}}(f(x)) \), there exists a set \( U \in \Cal{B}_{\Cal{O}}(x) \) such that \( f(U) \subseteq V \).
    \DItem{def:continuous_function/closure} For every set \( A \subseteq X \), \( f(\Cl(A)) \subseteq \Cl(f(A)) \).
    \DItem{def:continuous_function/limits} For every net\Tinyref{def:topological_net} \( \{ x_i \}_{i \in I} \), we have
    \begin{equation*}
      f\left(\lim_{i \in I} x_i \right) \subseteq \lim_{i \in I} f(x_i).
    \end{equation*}
  \end{defenum}

  We denote the set of all continuous functions from \( X \) to \( Y \) by \( C(X, Y) \).
\end{definition}

\begin{definition}\label{def:homeomorphism}
  We say that the continuous function \( f: (X, \Cal{T}) \to (Y, \Cal{O}) \) is a \Def{open} (resp. \Def{closed}), if the image \( f(U) \) of an open set (resp. closed set) in \( \Cal{T} \) is open (resp. closed) in \( \Cal{O} \).

  If \( f \) is an open bijection, we say that \( f \) is a \Def{homeomorphism}. If \( f \) is only an open injection, we say that \( f \) is a \Def{homeomorphic embedding}.
\end{definition}

\begin{definition}\label{def:continuous_function_at_point}
  We say that the function \( f: X \to Y \) is continuous at the point \( \Ol x \in X \) if for every neighborhood \( V \) of \( f(\Ol x) \) there exists a neighborhood \( U \) of \( x \) such that \( f(U) \subseteq V \).

  Compare to \cref{def:convergence_of_function_at_point} and to \cref{thm:metric_space_continuity/balls}.
\end{definition}

\begin{definition}\label{def:parametric_curve}
   Let \( I \) be an interval (of any type) in \( \R \) with endpoints \( a < b \), not necessarily finite. Depending on the use case, we define a \Def{parametric curve} on \( I \) by any of the non-equivalent definitions

  \begin{defenum}
    \DItem{def:parametric_curve/function} A continuous function \( \gamma: I \to X \) is called a parametric curve.

    \DItem{def:parametric_curve/image} The image \( \Img(\gamma) \) of a parametric curve \( \gamma \) is also called a parametric curve.

    \DItem{def:parametric_curve/equivalence_class} The equivalence class of all continuous functions from \( I \) to \( X \) with
    \begin{equation*}
      \gamma \cong \beta \iff \Img(\gamma) = \Img(\beta) \text{ and the endpoints of } \gamma \text{ and } \beta \text{ coincide}
    \end{equation*}
    is also called a parametric curve.
  \end{defenum}

  The points \( \gamma(a) \) and \( \gamma(b) \) are called the \Def{endpoints} of the curve, \( \gamma(a) \) is the \Def{start} and \( \gamma(b) \) is the \Def{end}. We say that \( \gamma \) \Def{connects} \( a \) and \( b \).

  Parametric curves on \( I = [0, 1] \) are also called \Def{paths}.

  We define some fundamental types of curves:
  \begin{defenum}
    \DItem{def:parametric_curve/closed} The curve \( \gamma \) is called \Def{closed} if its endpoints coincide, i.e. \( \gamma(a) = \gamma(b) \).

    \DItem{def:parametric_curve/simple} The curve \( \gamma \) is called \Def{simple} if the function \( \gamma: I \to Y \) is injective with the possible exception of the endpoints (in which case we speak of \Def{simple closed curves}.
  \end{defenum}

  If \( \gamma: I \to X \) is a parametric curve, related curves are:
  \begin{defenum}
    \DItem{def:parametric_curve/function_graph}\cite[definition 1.20]{Иванов2017} The graph\Tinyref{def:function_graph} \( \Gph(\gamma) \) of \( \gamma \) is a the image of the curve \( \Ol{\gamma}(t, x) \coloneqq (t, \gamma(x)) \) in the topological space \( I \times X \).

    \DItem{def:parametric_curve/implicit}\cite[definition 1.24]{Иванов2017} If \( M \) is a subset of \( X \) and if there exists a curve \( \gamma: I \to X \) such that \( \Im(\gamma) = M \), we call \( M \) an \Def{implicit parametric curve}.
  \end{defenum}
\end{definition}

\begin{definition}\label{def:parametric_hypersurface}
  In analogy to \cref{def:parametric_curve} (and with the caveats of \cref{def:parametric_curve}), we define \Def{parametric hypersurfaces} as follows:

  Let \( \xi \) is a potentially infinite cardinal number, let \( \Card \A = \xi \) and let \( \{ I_\alpha \}_{\alpha \in \A} \) be a family of intervals in \( \R \). We define a parametric hypersurface to be a continuous image from the product space\Tinyref{def:topological_product} \( \prod_{\alpha \in \A} I_\alpha \) to \( Y \).

  We call \( \xi \) the \Def{dimension} of the hypersurface.
\end{definition}
