\subsection{Affine coordinate systems}\label{subsec:affine_coordinate_system}

\begin{remark}\label{rem:affine_coordinate_system_concept}
  Humans possess a strong intuition for visual information like drawings or diagrams. A drawing on a paper is only a medium for communicating ideas and data. \Fullref{fig:rem:affine_coordinate_system_concept/figures} contains some highlighted curves that our mind maps to abstract geometric figures, without considering the size limitations of the page, the precision of the drawings or the thickness of the lines.

  \begin{figure}[!ht]
    \centering
    \includegraphics[page=1]{output/rem__affine_coordinate_system_concept.pdf}
    \caption{A triangle, a circle and a line in the Euclidean plane.}\label{fig:rem:affine_coordinate_system_concept/figures}
  \end{figure}

  Our goal is to translate these mental visualizations to the language of vector spaces. We will only briefly discuss high-level concepts without formalisms at the level of \hyperref[sec:mathematical_logic]{formal logic}. We will simply not need them. Modern plane geometry based on algebra and topology is discussed in \fullref{subsec:analytic_geometry_in_the_plane}.

  In this subsection, we will proceed as follows:
  \begin{itemize}
    \item Define an affine plane in \fullref{def:affine_plane}.
    \item Define the Euclidean plane \( A_2 \) in \fullref{def:euclidean_plane} as a special affine plane.
    \item Define the set \( F_2 \) of \hi{free vectors} over \( A_2 \) in \fullref{def:euclidean_plane_free_vector}.
    \item Show that \( F_2 \) is a two-dimensional vector space over \( \BbbR \) in \fullref{thm:euclidean_plane_factorization}.
    \item Define coordinate systems that give explicit isomorphisms between \( A_2 \), \( F_2 \) and \( \BbbR^2 \) in \fullref{def:euclidean_plane_coordinate_system}.
    \item Generalize these notions in \fullref{rem:coordinate_systems}
  \end{itemize}
\end{remark}

\begin{definition}\label{def:affine_plane}\mcite[1]{Hartshorne1967}
  An \term{affine plane} consists of
  \begin{thmenum}[series=def:affine_plane]
    \thmitem{def:euclidean_plane/points} An ambient set, whose members are called \term{points}.
    \thmitem{def:euclidean_plane/lines} A family of sets of points, whose members are called \term{lines}.
  \end{thmenum}

  We define the following two relations:
  \begin{thmenum}[resume=def:affine_plane]
    \thmitem{def:affine_plane/parallel} We say that the line \( l \) is \term{parallel} to \( g \) if either \( l = g \) or if they have no points in common. We write \( l \parallel g \)

    \thmitem{def:affine_plane/collinear} We say that the points \( P_1, \ldots, P_n \) are \term{collinear} if there exists a line that contains all of them.
  \end{thmenum}

  The definition of an affine plane additionally requires the following conditions to hold:
  \begin{thmenum}
    \thmitem[def:affine_plane/A1]{A1} Given two distinct points, there exists only one line that contains both.

    \thmitem[def:affine_plane/A2]{A2} Given a line \( l \) and a point \( P \) not on \( l \), there exists exactly one line \( g \) parallel to \( l \) that contains \( P \).

    \thmitem[def:affine_plane/A3]{A3} There exist three non-collinear points.
  \end{thmenum}
\end{definition}

\begin{definition}\label{def:euclidean_plane}
  The \term{Euclidean plane} \( A_2 \) formalizes a straight infinite surface. A completely formal treatment is tedious --- Euclid's original postulates can be found in \cite[7]{Fitzpatrick2008} and Tarski's first-order axiomatization can be found in \cite{Tarski1959}.

  For our purposes, we simply assume that \( A_2 \) is an \hyperref[def:affine_plane]{affine plane} that is also a \hyperref[def:complete_metric_space]{complete metric space} with metric
  \begin{equation*}
    \op{dist}: A_2 \times A_2 \to [0, \infty).
  \end{equation*}

  These are only necessary conditions, however they sufficiently characterize the Euclidean plane for our high-level overview.

  \begin{figure}[!ht]
    \centering
    \includegraphics[page=1]{output/def__euclidean_plane.pdf}
    \caption{Three lines and three points in the Euclidean plane. The lines \( l \) and \( g \) are collinear, while the point \( R \) is between \( P \) and \( Q \).}\label{fig:def:euclidean_plane}
  \end{figure}

  We will also need the following concepts:
  \begin{thmenum}
    \thmitem{def:euclidean_plane/betweenness} We say that the point \( R \) lies \term{between} \( P \) and \( Q \) if
    \begin{equation*}
      \op{dist}(P, R) + \op{dist}(R, Q) = \op{dist}(P, Q).
    \end{equation*}

    \thmitem{def:euclidean_plane/half_plane} Every line \( l \) gives rise to two (closed) \term{half-planes} \( H^+ \) and \( H^- \) satisfying
    \begin{thmenum}
      \thmitem{def:euclidean_plane/half_plane/intersection} \( H^+ \cap H^- = l \)
      \thmitem{def:euclidean_plane/half_plane/union} \( H^+ \cup H^- = A_2 \)
      \thmitem{def:euclidean_plane/half_plane/betweenness} If \( P \) lies in \( H^+ \setminus l \) and \( Q \) lies in \( H^- \setminus l \), then there is a point \( R \in l \) between \( P \) and \( Q \)
    \end{thmenum}

    We call the sets \( H^+ \setminus l \) and \( H^- \setminus l \) \term{open} half-planes.

    Note that the superscripts \( + \) and \( - \) are only for distinguishing between the two half-planes and are not assigned based on some property of the half-planes. See \fullref{def:half_space} for a definition of a half-plane that actually has a concept of signs.

    \begin{figure}[!ht]
      \centering
      \text{\todo{Add diagram}}\iffalse\begin{mplibcode}
        input metapost/plotting;

        u := 1cm;

        beginfig(1);
        input hatching;

        path l, Hp, Hm;
        l = (0, -1) * u -- (3, 0) * u;
        draw l;

        Hp = l -- (3, 0.5) * u -- (0, 0.5) * u -- cycle;
        hatchfill Hp withcolor (45, 1mm, -.5bp);
        label.ulft("$H^+$", startpoint of l);

        Hm = l -- (3, -1.5) * u -- (0, -1.5) * u -- cycle;
        hatchfill Hm withcolor (135, 1mm, -.5bp);
        label.lrt("$H^-$", endpoint of l);
        endfig;
      \end{mplibcode}\fi

      \caption{Differently hatched half-planes in the Euclidean plane.}\label{def:euclidean_plane/bound_vector/half_plane}
    \end{figure}

    \thmitem{def:euclidean_plane/ray} Every line \( l \) and every point \( P \) give rise to two (closed) \term{rays} \( l^+ \) and \( l^- \) satisfying
    \begin{itemize}
      \thmitem{def:euclidean_plane/ray/intersection} \( l^+ \cap l^- = \set{ P } \)
      \thmitem{def:euclidean_plane/ray/union} \( l^+ \cup l^- = l \)
      \thmitem{def:euclidean_plane/ray/betweenness} If \( Q \) lies in \( l^+ \setminus \set{ P } \) and \( R \) lies in \( l^- \setminus \set{ P } \), then \( P \) is between \( Q \) and \( R \)
    \end{itemize}

    We say that \( R \) is the \term{vertex} of \( l^+ \) and \( l^- \) and that the rays are \term{opposite} of each other. We call the sets \( l^+ \setminus \set{ P } \) and \( l^- \setminus \set{ P } \) \term{open} rays.

    See \fullref{def:geometric_ray} for a definition of a ray that actually has a concept of signs.

    \begin{figure}[!ht]
      \centering
      \text{\todo{Add diagram}}\iffalse\begin{mplibcode}
        input metapost/plotting;

        u := 1cm;

        beginfig(1);
        path l, R;

        l = (0, -1) * u -- (3, 0) * u;
        drawdblarrow l;
        label.lft("$l^-$", startpoint of l);
        label.rt("$l^+$", endpoint of l);

        R = dot shifted midpoint of l;
        fill R;
        label.bot("$R$", midpoint of R);
        endfig;
      \end{mplibcode}\fi

      \caption{Opposite rays in the Euclidean plane.}\label{def:euclidean_plane/day/figure}
    \end{figure}

    \thmitem{def:euclidean_plane/rays_unidirectional} Two rays and are said to be \term{unidirectional} if the lines containing the rays are parallel and if there exists another line with respect to whom both rays are contained in the same half-plane.

    \thmitem{def:euclidean_plane/bound_vector} We call an ordered pair \( (P, Q) \) of points a \term{bound vector} and denote it via \( \vect{PQ} \). The point \( P \) is called the \term{beginning} of \( \vect{PQ} \) and \( Q \) is called the \term{end} of \( \vect{PQ} \).

    \begin{figure}[!ht]
      \centering
      \text{\todo{Add diagram}}\iffalse\begin{mplibcode}
        input metapost/plotting;

        u := 0.75cm;

        beginfig(1);
        path P, Q, R, PQ, PR;

        PQ = (0, -1) * u -- (3, 0) * u;
        drawarrow PQ;
        label.bot("$\vect{PQ}$", midpoint of PQ);

        P = dot shifted startpoint of PQ;
        fill P;
        label.bot("$P$", midpoint of P);

        Q = dot shifted endpoint of PQ;
        label.bot("$Q$", midpoint of Q);

        PR = (0, -1) * u -- (-2, 0.5) * u;
        drawarrow PR;
        label.llft("$\vect{PR}$", midpoint of PR);

        R = dot shifted endpoint of PR;
        label.llft("$R$", midpoint of R);
        endfig;
      \end{mplibcode}\fi

      \caption{Bound vectors in the Euclidean plane can be regarded as oriented line segment.}\label{def:euclidean_plane/bound_vector/figure}
    \end{figure}
  \end{thmenum}
\end{definition}

\begin{definition}\label{def:euclidean_plane_free_vector}
  We say that the bound vectors \( \vect{P_1 Q_1} \) and \( \vect{P_2 Q_2} \) in \( A_2 \) are \term{congruent} if \( \op{dist}(P_1, Q_1) = \op{dist}(P_2, Q_2) \) and if the rays \( r_i, i = 1, 2 \) beginning at \( P_i \) and containing \( Q_i \), are unidirectional.

  We define \term{free vectors} as \hyperref[thm:equivalence_partition]{equivalence classes} of bound vectors by this congruence relation. We denote the corresponding equivalence partition by \( F_2 \).
\end{definition}

\begin{theorem}\label{thm:euclidean_plane_factorization}
  The set \( F_2 \) of free vectors over \( A_2 \) is a two-dimensional \hyperref[def:vector_space]{vector space} over \( \BbbR \) with the following operations:
  \begin{thmenum}
    \thmitem{thm:euclidean_plane_factorization/sum} We define the \term{sum} of the cosets \( [\vect{PQ}] \) and \( [\vect{QR}] \) as the coset \( [\vect{PR}] \).

    \thmitem{thm:euclidean_plane_factorization/scalar_product} We define the \term{scalar multiplication} of \( \lambda \in \BbbR \) with the coset \( [\vect{PQ}] \) to be the coset \( [\vect{PR}] \), where \( \vect{PR} \) is the unique vector that is unidirectional with \( \vect{PQ} \) and \( \op{dist}(P, R) = \lambda \op{dist}(P, Q) \).
  \end{thmenum}
\end{theorem}
\begin{proof}
  Proving the well-definedness of the operations and verifying that \( F_2 \) is a two-dimensional vector space requires a lot of work and the proof is skipped.
\end{proof}

\begin{definition}\label{def:euclidean_plane_coordinate_system}
  Just because \fullref{thm:euclidean_plane_factorization} states that the set \( F_2 \) of free vectors is a vector space does not mean that we can work with it as with \( \BbbR^2 \). \Fullref{thm:modules_with_same_rank_are_isomorphic} says that \( F_2 \) is isomorphic to \( \BbbR^2 \), however the proof requires the \hyperref[def:zfc/choice]{axiom of choice}. The concrete way to select a basis in \( F_2 \) is through coordinate systems.

  Somewhat confusingly, we define coordinate systems over \( A_2 \) rather than over \( F_2 \), but this will, soon be justified.

  A \term{coordinate system} \( Oxy \) in \( A_2 \) is a choice of
  \begin{thmenum}
    \thmitem{def:euclidean_plane_coordinate_system/origin} A point \( O \in A_2 \), called the \term{origin} of the coordinate system.
    \thmitem{def:euclidean_plane_coordinate_system/basis} An \hyperref[def:partially_ordered_set]{ordered} \hyperref[def:hamel_basis]{basis} \( (x, y) \) of \( F_2 \), called the \term{basis} of \( Oxy \).
  \end{thmenum}

  What we achieve through the choice of \( O \) is that, for each point \( P \in A_2 \), we select the bound vector \( \vect{OP} \in V_2 \), called the \term{radius vector} of \( P \). This injects \( A_2 \) into \( V_2 \), however if we take the free vector \( [\vect{OP}] \), we instead obtain a bijection between \( A_2 \) and \( F_2 \).

  Now that we have a correspondence between \( A_2 \) and \( F_2 \), coordinates for the point \( P \) are defined simply as the \hyperref[def:basis_decomposition]{coordinates} of \( [\vect{OP}] \) with respect to the basis \( (x, y) \).

  Thus, the pair \( (A_2, Oxy) \) has an explicit isomorphism with \( \BbbR^2 \).

  The \term{coordinate axis} of \( x \) is the unique \hyperref[def:euclidean_plane/ray]{ray} starting at \( O \) and containing the end of \( x \). It is called the \term{abscissa}. The coordinate axis of \( y \) is called the \term{ordinate}.
\end{definition}

\begin{remark}\label{rem:coordinate_systems}
  We sketched how to embed mental images of planes into \( \BbbR^2 \), however in mathematics we are often interested in the opposite: given a set of points in \( \BbbR^2 \), visualize them on a screen or paper and then absorb the the resulting image in our brain.

  This is one of the most powerful constructions in mathematics, yet it is, so intuitive that it is not really given a lot of attention, at least until generalizations are required. Given any vector space \( V \) in the sense of \fullref{def:vector_space}, we want a way to assign a pair of numbers to each vector in \( V \). This is only possible if \( \dim V = 2 \), however we can generalize this to tuples of coordinates via bases - see \fullref{def:hamel_basis}. This well for finitely dimensional vector spaces, however we need to generalize these notion for infinitely dimensional vector spaces and general modules over \hyperref[def:module]{rings}. This allows us to generalize coordinates further to manifolds - see \fullref{def:topological_manifold}.

  See \fullref{subsec:vector_space_geometry} for immediate generalizations of the concepts introduced here.
\end{remark}
