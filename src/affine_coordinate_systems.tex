\subsection{Affine coordinate systems}\label{subsec:affine_coordinate_system}

\begin{remark}\label{rem:affine_coordinate_system_concept}
  Humans possess a strong intuition for visual information like drawings or diagrams. A drawing on a paper is only a medium for communicating ideas and data. \Fullref{fig:rem:affine_coordinate_system_concept/figures} contains some highlighted curves that our mind maps to abstract geometric figures, without considering the size limitations of the page, the precision of the drawings or the thickness of the lines.

  \begin{figure}[!ht]
    \centering
    \includegraphics{output/rem__affine_coordinate_system_concept.pdf}
    \caption{A triangle, a circle and a line in the Euclidean plane.}\label{fig:rem:affine_coordinate_system_concept/figures}
  \end{figure}

  Our goal is to translate these mental visualizations to the language of vector spaces. We will only briefly discuss high-level concepts without formalisms at the level of \hyperref[sec:mathematical_logic]{formal logic}. We will simply not need them. Modern plane geometry based on algebra and topology is discussed in \fullref{subsec:analytic_geometry_in_the_plane}.

  In this subsection, we will proceed as follows:
  \begin{itemize}
    \item Define an affine plane in \fullref{def:affine_plane}.
    \item Define the Euclidean plane \( A_2 \) in \fullref{def:euclidean_plane} as a special affine plane.
    \item Define the set \( F_2 \) of \hi{free vectors} over \( A_2 \) in \fullref{def:euclidean_plane_free_vector}.
    \item Show that \( F_2 \) is a two-dimensional vector space over \( \BbbR \) in \fullref{thm:euclidean_plane_vector_factorization}.
    \item Define coordinate systems that give explicit isomorphisms between \( A_2 \), \( F_2 \) and \( \BbbR^2 \) in \fullref{def:euclidean_plane_coordinate_system}.
    \item Generalize these notions in \fullref{rem:coordinate_systems}
  \end{itemize}
\end{remark}

\begin{definition}\label{def:affine_plane}\mcite[1]{Hartshorne1967}
  An \term{affine plane} consists of
  \begin{thmenum}[series=def:affine_plane]
    \thmitem{def:euclidean_plane/points} An ambient set, whose members are called \term{points}.
    \thmitem{def:euclidean_plane/lines} A family of sets of points, whose members are called \term{lines}.
  \end{thmenum}

  We define the following two relations:
  \begin{thmenum}[resume=def:affine_plane]
    \thmitem{def:affine_plane/parallel} We say that the line \( l \) is \term{parallel} to \( g \) if either \( l = g \) or if they have no points in common. We write \( l \parallel g \)

    \thmitem{def:affine_plane/collinear} We say that the points \( P_1, \ldots, P_n \) are \term{collinear} if there exists a line that contains all of them.
  \end{thmenum}

  The definition of an affine plane additionally requires the following conditions to hold:
  \begin{thmenum}
    \thmitem[def:affine_plane/A1]{A1} Given two distinct points, there exists only one line that contains both.

    \thmitem[def:affine_plane/A2]{A2} Given a line \( l \) and a point \( P \) not on \( l \), there exists exactly one line \( g \) parallel to \( l \) that contains \( P \).

    \thmitem[def:affine_plane/A3]{A3} There exist three non-collinear points.
  \end{thmenum}
\end{definition}

\begin{definition}\label{def:euclidean_plane}\mimprovised
  The \term{Euclidean plane} \( A_2 \) formalizes a straight infinite surface. A completely formal treatment is tedious --- Euclid's original postulates can be found in \cite[7]{Fitzpatrick2008} and Tarski's first-order axiomatization can be found in \cite{Tarski1959}.

  For our purposes, we simply assume that \( A_2 \) is an \hyperref[def:affine_plane]{affine plane} that is also a \hyperref[def:complete_metric_space]{complete metric space} with metric
  \begin{equation*}
    \op{dist}: A_2 \times A_2 \to [0, \infty).
  \end{equation*}

  These are only necessary conditions, however they sufficiently characterize the Euclidean plane for our high-level overview.

  \begin{figure}[!ht]
    \centering
    \includegraphics{output/def__euclidean_plane__basic_shapes.pdf}
    \caption{Three lines and three points in the Euclidean plane. The lines \( l \) and \( g \) are collinear, while the point \( R \) is between \( P \) and \( Q \).}\label{fig:def:euclidean_plane/basic_shapes}
  \end{figure}

  We will also need some additional concepts:
  \begin{thmenum}
    \thmitem{def:euclidean_plane/betweenness} We say that the point \( R \) lies \term{between} \( P \) and \( Q \) if
    \begin{equation*}
      \op{dist}(P, R) + \op{dist}(R, Q) = \op{dist}(P, Q).
    \end{equation*}

    \thmitem{def:euclidean_plane/half_plane} We say that the set \( H \) of points is a \term{closed half-plane} corresponding to the line \( l \) if \( H \) contains \( l \) and if, whenever \( P \) is a point in \( H \setminus l \) and \( Q \) is a point not in \( H \), there exists a point on \( l \) that lies between \( P \) and \( Q \). If \( H \) is a closed half-plane corresponding to \( l \), then \( H \setminus l \) is called an \term{open half-plane}.

    \begin{figure}[!ht]
      \centering
      \includegraphics{output/def__euclidean_plane__half_plane.pdf}
      \caption{Differently hatched half-planes in the Euclidean plane.}\label{fig:def:euclidean_plane/half_plane}
    \end{figure}

    There are two half-planes corresponding to every line. See \fullref{def:half_space} for a definition of half-planes in \hyperref[def:euclidean_space]{Euclidean spaces}, which also allows us to distinguish between a positive and negative half-plane.

    \thmitem{def:euclidean_plane/ray} We say that the set of points \( r \) is a \term{closed ray} with \term{vertex} \( V \) if \( r \) contains \( V \) and if, whenever \( P \) is a point in \( r \) distinct from \( R \) and \( Q \) is a point not in \( r \), \( V \) is between \( P \) and \( Q \). If \( r \) is a closed ray with vertex \( V \), then \( r \setminus \set{ V } \) is called an \term{open ray}.

    \begin{figure}[!ht]
      \centering
      \includegraphics{output/def__euclidean_plane__ray.pdf}
      \caption{Unidirectional and opposite rays in the Euclidean plane.}\label{fig:def:euclidean_plane/ray}
    \end{figure}

    Every ray lies on a line. There are infinitely many rays on every line and infinitely many rays for every vertex. We will further introduce relations between rays.

    We say that two rays are \term{unidirectional} if the lines containing the rays are parallel and if there exists another line with respect to whom both rays are contained in the same half-plane. If two different rays lie on the same line and have a common vertex, we say that they are \term{opposite} to each other.

    See \fullref{def:geometric_ray} for a definition of a ray in Euclidean spaces.

    \thmitem{def:euclidean_plane/bound_vector} We call an ordered pair \( (P, Q) \) of points a \term{bound vector} and denote it via \( \vect{PQ} \). We call the point \( P \) the \term{beginning} of \( \vect{PQ} \) and \( Q \) --- the \term{end} of \( \vect{PQ} \).

    We denote the set of all bound vectors over \( A_2 \) by \( V_2 \).

    \begin{figure}[!ht]
      \centering
      \includegraphics{output/def__euclidean_plane__bound_vector.pdf}
      \caption{Bound vectors in the Euclidean plane can be regarded as oriented line segment.}\label{fig:def:euclidean_plane/bound_vector}
    \end{figure}
  \end{thmenum}
\end{definition}

\begin{definition}\label{def:euclidean_plane_free_vector}\mimprovised
  We say that the \hyperref[def:euclidean_plane/bound_vector]{bound vectors} \( \vect{P_1 Q_1} \) and \( \vect{P_2 Q_2} \) in \( A_2 \) are \term{congruent} if \( \op{dist}(P_1, Q_1) = \op{dist}(P_2, Q_2) \) and if the two rays beginning at \( P_k \) and containing \( Q_k \), \( k = 1, 2 \), are unidirectional.

  We can \hyperref[thm:equivalence_partition]{partition} the set \( V_2 \) of all bound vectors by this congruence relation. We denote the corresponding partition by \( F_2 \), and we call the cosets of the partition \term{free vectors}.
\end{definition}

\begin{theorem}[Euclidean plane vector factorization]\label{thm:euclidean_plane_vector_factorization}
  The set \( F_2 \) of \hyperref[def:euclidean_plane_free_vector]{free vectors} over \( A_2 \) is a two-dimensional \hyperref[def:vector_space]{vector space} over \( \BbbR \) with the following operations:
  \begin{thmenum}
    \thmitem{thm:euclidean_plane_vector_factorization/sum} The \term{sum} of the cosets \( [\vect{PQ}] \) and \( [\vect{QR}] \) is the coset \( [\vect{PR}] \).

    \thmitem{thm:euclidean_plane_vector_factorization/scalar_product} The \term{scalar multiplication} of \( \lambda \in \BbbR \) with the coset \( [\vect{PQ}] \) is the coset \( [\vect{PR}] \), where \( \vect{PR} \) is the unique vector that is unidirectional with \( \vect{PQ} \) such that
    \begin{equation*}
      \op{dist}(P, R) = \lambda \cdot \op{dist}(P, Q).
    \end{equation*}
  \end{thmenum}
\end{theorem}
\begin{proof}
  Proving the well-definedness of the operations and verifying that \( F_2 \) is a two-dimensional vector space is tedious, and we present the theorem only for demonstrational purposes. The theorem is briefly discussed in \cite[ch. 1]{ВеселовТроицкий2002Лекции}.
\end{proof}

\begin{definition}\label{def:euclidean_plane_coordinate_system}
  Just because \fullref{thm:euclidean_plane_vector_factorization} states that the set \( F_2 \) of \hyperref[def:euclidean_plane_free_vector]{free vectors} is a vector space does not mean that we can work with it as with \( \BbbR^2 \). \Fullref{thm:modules_with_same_rank_are_isomorphic} says that \( F_2 \) is isomorphic to \( \BbbR^2 \), however we need a concrete isomorphism. The concrete way to select a basis in \( F_2 \) is through coordinate systems.

  Somewhat confusingly, we define coordinate systems over \( A_2 \) rather than over \( F_2 \), but this will soon be justified.

  A \term{coordinate system} \( Oxy \) in \( A_2 \) is a choice of
  \begin{thmenum}
    \thmitem{def:euclidean_plane_coordinate_system/origin} A point \( O \), called the \term{origin} of the coordinate system.
    \thmitem{def:euclidean_plane_coordinate_system/basis} An \hyperref[def:partially_ordered_set]{ordered} \hyperref[def:hamel_basis]{basis} \( (x, y) \) of \( F_2 \), called the \term{basis} of \( Oxy \).
  \end{thmenum}

  What we achieve through the choice of \( O \) is that, for each point \( P \), we select the bound vector \( \vect{OP} \), called the \term{radius vector} of \( P \). This injects \( A_2 \) into \( V_2 \). If we take the free vector \( [\vect{OP}] \) instead of the bound vector \( \vect{OP} \), we obtain an isomorphism between \( A_2 \) and \( F_2 \).

  Now that we have a correspondence between \( A_2 \) and \( F_2 \), coordinates for the point \( P \) are defined simply as the \hyperref[def:basis_decomposition]{coordinates} of \( [\vect{OP}] \) with respect to the basis \( (x, y) \).

  Thus, \( Oxy \) gives an explicit isomorphism between \( A_2 \) and \( \BbbR^2 \).

  The \term{coordinate axis} of \( x \) is the unique \hyperref[def:euclidean_plane/ray]{ray} starting at \( O \) and containing the end of \( x \). It is called the \term{abscissa}. The coordinate axis of \( y \) is called the \term{ordinate}.
\end{definition}

\begin{remark}\label{rem:coordinate_systems}
  We sketched how to embed mental images of planes into \( \BbbR^2 \), however we are often interested in the opposite: given a set of points in \( \BbbR^2 \), draw them on a physical medium and then absorb the resulting image in our brain.

  Coordinate systems are one of the most powerful constructions in mathematics, yet they are so intuitive that they are not really given a lot of attention in themselves. At least until generalizations are required. Euclid gives a straightforward generalization to the three-dimensional \hyperref[def:euclidean_space]{Euclidean space} --- see \cite{Fitzpatrick2008}. Beyond that, generalizations start to vary wildly --- from arbitrary \hyperref[def:vector_space]{vector spaces} to \hyperref[def:affine_algebraic_set]{algebraic varieties} and \hyperref[def:topological_manifold]{manifolds}.

  More often than not, however, we are interested in finite-dimensional real vector spaces. Given any real vector space \( V \) of dimension \( n \), we can assign an \( n \)-tuple of real numbers to each vector in \( V \) via \hyperref[def:basis_decomposition]{coordinate projections}. Thus, we can develop geometry within the framework of linear algebra rather than starting with geometric constructions like the \hyperref[def:euclidean_plane]{Euclidean plane} \( A_2 \). This has two major benefits:
  \begin{itemize}
    \item Both abstract linear algebra and matrix theory provide very powerful tools.
    \item Linear algebra allows us to generalize simply geometric constructions beyond dimensions \( 2 \) and \( 3 \).
  \end{itemize}

  We describe geometry in such a context in \fullref{subsec:vector_space_geometry} and \fullref{subsec:analytic_geometry_in_the_plane}.
\end{remark}
