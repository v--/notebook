\section{Geometry}\label{sec:geometry}

\begin{remark}\label{def:coordinates_in_geometry}
  Geometry is the multi-millennium evolution of attempts to measure parts of the earth. Ironically, it may be the main historical justification for the gradual axiomatization of mathematics. Completely abstract results about shapes date at least as early as in Ancient Greece. The important distinction between ancient geometry and modern geometry is the introduction of coordinates in the 17th century.

  An axiomatic approach for a theory of plane and solid figures was developed by Euclid in the third century BC. Later, Hilbert, Tarski and others independently proposed axioms that fit the requirements of modern logic systems. This is known today as \term{synthetic Euclidean geometry} and is mostly of theoretical interest because modern tools are easier to work with.

  Descartes' idea of coordinates connects problems of algebra and geometry in such a way that most of today's mathematics seamlessly switches between algebraic and geometric interpretations of the same problem. The study of classical Greek geometry in terms of coordinates is known as \term{analytic geometry}.
\end{remark}

\subsection{Affine coordinate systems}\label{subsec:affine_coordinate_system}

\begin{remark}\label{rem:affine_coordinate_system_concept}
  Most humans possess a strong intuition for visual information like drawings or diagrams. A paper or a painting is only a medium for communicating information and emotions. \Fullref{def:euclidean_plane/figures} contains some highlighted curves that our mind maps to abstract geometric figures, without considering the size limitations of the page, the precision of the drawings or the thickness of the lines.

  \begin{figure}[b]
    \centering
    \text{\todo{Add diagram}}\iffalse\begin{mplibcode}
      u := 1cm;

      beginfig(1);
      draw (0, -1) * u -- (3, 0) * u;
      draw (-1, 2) * u -- (3, 1) * u -- (1, 3) * u -- cycle;
      draw fullcircle scaled 1.5u shifted ((0, 0.5) * u);
      endfig;
    \end{mplibcode}\fi
    \caption{A triangle, a circle and a line in the Euclidean plane.}\label{def:euclidean_plane/figures}
  \end{figure}

  Our goal is to map these visualizations to the concept of vector spaces. Formalisms at the level of formal \hyperref[def:first_order_language]{logic} will not be stated because we only want to sketch some high-level concepts. We only give definitions that are strictly necessary, plane geometry itself is described in \fullref{subsec:analytic_geometry_in_the_plane}. We will proceed as follows:

  \begin{itemize}
    \item Define an affine plane in \fullref{def:affine_plane} with auxiliary definitions.
    \item Describe the Euclidean plane \( A_2 \) in \fullref{def:euclidean_plane} as a very special affine plane.
    \item Give additional definitions for the Euclidean plane in \fullref{def:euclidean_plane_auxiliary_definitions}.
    \item Define the set \( F_2 \) of free vectors over \( A_2 \) in \fullref{def:euclidean_plane_free_vector}.
    \item Show that \( F_2 \) is a two-dimensional vector space over \( \BbbR \) in \fullref{thm:euclidean_plane_factorization}.
    \item Define coordinate systems that give explicit isomorphisms between \( A_2 \), \( F_2 \) and \( \BbbR^2 \) in \fullref{def:euclidean_plane_coordinate_system}.
    \item Generalize these notions in \fullref{rem:coordinate_systems}
  \end{itemize}
\end{remark}

\begin{definition}\label{def:affine_plane}\mcite[1]{Hartshorne1967}
  An \term{affine plane} consists of
  \begin{itemize}
    \item a set \( X \), whose elements are called \term{points},
    \item a family of subsets of \( X \), whose members are called \term{lines}
  \end{itemize}
  with the additional relations
  \begin{itemize}
    \item a \term{parallel} relation \( l \parallel g \) for lines that holds if either \( l = g \) or if they have no points in common,
    \item a \term{collinearity} relation for a set \( B \) of points that holds if \( B \) is a subset of some line,
  \end{itemize}
  such that
  \begin{thmenum}
    \thmitem[def:affine_plane/A1]{A1} Given two distinct points, there exists only one line that contains both.
    \thmitem[def:affine_plane/A2]{A2} Given a line \( l \) and a point \( P \not\in l \), there exists exactly one line \( g \parallel l \) that contains \( P \).
    \thmitem[def:affine_plane/A3]{A3} There exist three non-collinear points.
  \end{thmenum}
\end{definition}

\begin{definition}\label{def:euclidean_plane}
  The \term{Euclidean plane} \( A_2 \) is a formalization of a straight infinite surface. An axiomatic definition can be found in \cite{nLab:euclidean_geometry}. We will use that
  \begin{itemize}
    \item The Euclidean plane \( A_2 \) is an \hyperref[def:affine_plane]{affine plane}
    \item \( A_2 \) is a \hyperref[def:complete_metric_space]{complete metric space} with distance \( \op{dist} \).
    \item There is a \term{betweenness} relation for points that says if the point \( R \) is \term{between} \( P \) and \( Q \).
  \end{itemize}

  \begin{figure}
    \centering
    \text{\todo{Add diagram}}\iffalse\begin{mplibcode}
      input metapost/plotting;

      u := 1.5cm;

      beginfig(1);
      path l, g, h, P, Q, R;
      l = (0, -1) * u -- (3, 0) * u;
      draw l;
      label.top("$l$", midpoint of l);

      g = (0, -2) * u -- (3, -1) * u;
      draw g;
      label.bot("$g$", midpoint of g);

      h = (0, 0) * u -- (3, -2) * u;
      draw h;
      label.urt("$h$", midpoint of h);

      P = dot shifted point 0.2 of h;
      fill P;
      label.llft("$P$", midpoint of P);

      Q = dot shifted point 0.8 of h;
      fill Q;
      label.llft("$Q$", midpoint of Q);

      R = dot shifted point 0.4 of h;
      fill R;
      label.llft("$R$", midpoint of R);
      endfig;
    \end{mplibcode}\fi
    \caption{Three lines and three points in the Euclidean plane. The lines \( l \) and \( g \) are collinear, while the point \( R \) is between \( P \) and \( Q \)}\label{def:affine_plane/figure}
  \end{figure}
\end{definition}

\begin{definition}\label{def:euclidean_plane_auxiliary_definitions}
  We will also need the following definitions:
  \begin{thmenum}
    \thmitem{def:affine_plane/half_plane} Every line \( l \) gives rise to two (closed) \term{half-planes} \( H^+ \) and \( H^- \) as follows:
    \begin{itemize}
      \item \( H^+ \cap H^- = l \)
      \item \( H^+ \cup H^- = A_2 \)
      \item If \( P \in H^+ \setminus l \) and \( Q \in H^- \setminus l \), then there is a point \( R \in l \) between \( P \) and \( Q \)
    \end{itemize}

    Note that the superscripts \( + \) and \( - \) are only for distinguishing between the two half-planes and are not assigned based on some property of the half-planes. See \fullref{def:half_space} for a definition of a half-plane that actually has a concept of signs.

    \begin{figure}
      \centering
      \text{\todo{Add diagram}}\iffalse\begin{mplibcode}
        input metapost/plotting;

        u := 1cm;

        beginfig(1);
        input hatching;

        path l, Hp, Hm;
        l = (0, -1) * u -- (3, 0) * u;
        draw l;

        Hp = l -- (3, 0.5) * u -- (0, 0.5) * u -- cycle;
        hatchfill Hp withcolor (45, 1mm, -.5bp);
        label.ulft("$H^+$", startpoint of l);

        Hm = l -- (3, -1.5) * u -- (0, -1.5) * u -- cycle;
        hatchfill Hm withcolor (135, 1mm, -.5bp);
        label.lrt("$H^-$", endpoint of l);
        endfig;
      \end{mplibcode}\fi

      \caption{Differently hatched half-planes in the Euclidean plane.}\label{def:affine_plane/bound_vector/half_plane}
    \end{figure}

    \thmitem{def:affine_plane/ray} Every line \( l \) and every point \( R \) give rise to two (closed) \term{rays} \( l^+ \) and \( l^- \) as follows:
    \begin{itemize}
      \item \( l^+ \cap l^- = \{ R \} \) are disjoint
      \item \( l^+ \cup l^- = l \)
      \item If \( P \in l^+ \setminus \{ R \} \) and \( Q \in l^- \setminus \{ R \} \), then \( R \) is between \( P \) and \( Q \)
    \end{itemize}

    The rays \( l^+ \) and \( l^- \) are called \term{opposite} of each other.

    We say that \( R \) is the \term{vertex} of \( l^+ \) and \( l^- \).

    See \fullref{def:geometric_ray} for a definition of a ray that actually has a concept of signs.

    \begin{figure}
      \centering
      \text{\todo{Add diagram}}\iffalse\begin{mplibcode}
        input metapost/plotting;

        u := 1cm;

        beginfig(1);
        path l, R;

        l = (0, -1) * u -- (3, 0) * u;
        drawdblarrow l;
        label.lft("$l^-$", startpoint of l);
        label.rt("$l^+$", endpoint of l);

        R = dot shifted midpoint of l;
        fill R;
        label.bot("$R$", midpoint of R);
        endfig;
      \end{mplibcode}\fi

      \caption{Opposite rays in the Euclidean plane.}\label{def:affine_plane/day/figure}
    \end{figure}

    \thmitem{def:affine_plane/rays_unidirectional} Two rays are said to be \term{unidirectional} if there exists a line distinct from the lines containing the rays, such that both rays are contained in the same half-plane with respect to the line.

    \thmitem{def:affine_plane/bound_vector} An ordered pair \( \overrightarrow{PQ} \) of points is called a \term{bound vector}. The point \( P \) is called the \term{beginning} of \( \overrightarrow{PQ} \) and \( Q \) is called the \term{end} of \( \overrightarrow{PQ} \).

    \begin{figure}
      \centering
      \text{\todo{Add diagram}}\iffalse\begin{mplibcode}
        input metapost/plotting;

        u := 0.75cm;

        beginfig(1);
        path P, Q, R, PQ, PR;

        PQ = (0, -1) * u -- (3, 0) * u;
        drawarrow PQ;
        label.bot("$\overrightarrow{PQ}$", midpoint of PQ);

        P = dot shifted startpoint of PQ;
        fill P;
        label.bot("$P$", midpoint of P);

        Q = dot shifted endpoint of PQ;
        label.bot("$Q$", midpoint of Q);

        PR = (0, -1) * u -- (-2, 0.5) * u;
        drawarrow PR;
        label.llft("$\overrightarrow{PR}$", midpoint of PR);

        R = dot shifted endpoint of PR;
        label.llft("$R$", midpoint of R);
        endfig;
      \end{mplibcode}\fi

      \caption{Bound vectors in the Euclidean plane can be regarded as oriented line segment.}\label{def:affine_plane/bound_vector/figure}
    \end{figure}
  \end{thmenum}
\end{definition}

\begin{definition}\label{def:euclidean_plane_free_vector}
  We say that the bound vectors \( \overrightarrow{P_1 Q_1} \) and \( \overrightarrow{P_2 Q_2} \) in \( A_2 \) are \term{congruent} if \( \op{dist}(P_1, Q_1) = \op{dist}(P_2, Q_2) \) and if the rays \( r_i, i = 1, 2 \) beginning at \( P_i \) and containing \( Q_i \), are unidirectional.

  We define \term{free vectors} as \hyperref[thm:equivalence_partition]{equivalence classes} of bound vectors by this congruence relation. We denote the corresponding equivalence partition by \( F_2 \).
\end{definition}

\begin{theorem}\label{thm:euclidean_plane_factorization}
  The set \( F_2 \) of free vectors over \( A_2 \) is a two-dimensional \hyperref[def:vector_space]{vector space} over \( \BbbR \) with the following operations:
  \begin{thmenum}
    \thmitem{thm:euclidean_plane_factorization/sum} We define the \term{sum} of the cosets \( [\overrightarrow{PQ}] \) and \( [\overrightarrow{QR}] \) as the coset \( [\overrightarrow{PR}] \).

    \thmitem{thm:euclidean_plane_factorization/scalar_product} We define the \term{scalar multiplication} of \( \lambda \in \BbbR \) with the coset \( [\overrightarrow{PQ}] \) to be the coset \( [\overrightarrow{PR}] \), where \( \overrightarrow{PR} \) is the unique vector that is unidirectional with \( \overrightarrow{PQ} \) and \( \op{dist}(P, R) = \lambda \op{dist}(P, Q) \).
  \end{thmenum}
\end{theorem}
\begin{proof}
  Proving the well-definedness of the operations and verifying that \( F_2 \) is a two-dimensional vector space requires a lot of work and the proof is skipped.
\end{proof}

\begin{definition}\label{def:euclidean_plane_coordinate_system}
  Just because \fullref{thm:euclidean_plane_factorization} states that the set \( F_2 \) of free vectors is a vector space does not mean that we can work with it as with \( \BbbR^2 \). \Fullref{thm:finite_dimensional_spaces_are_isomorphic} says that \( F_2 \) is isomorphic to \( \BbbR^2 \), however the proof requires the axiom of \hyperref[def:zfc/A9]{choice}. The concrete way to select a basis in \( F_2 \) is through coordinate systems.

  Somewhat confusingly, we define coordinate systems over \( A_2 \) rather than over \( F_2 \), but this will soon be justified.

  A \term{coordinate system} \( Oxy \) in \( A_2 \) is a choice of
  \begin{thmenum}
    \thmitem{def:euclidean_plane_coordinate_system/origin} A point \( O \in A_2 \), called the \term{origin} of the coordinate system.
    \thmitem{def:euclidean_plane_coordinate_system/basis} An \hyperref[def:poset]{ordered} \hyperref[def:left_module_hamel_basis]{basis} \( (x, y) \) of \( F_2 \), called the \term{basis} of \( Oxy \).
  \end{thmenum}

  What we achieve through the choice of \( O \) is that, for each point \( P \in A_2 \), we select the bound vector \( \overrightarrow{OP} \in V_2 \), called the \term{radius vector} of \( P \). This injects \( A_2 \) into \( V_2 \), however if we take the free vector \( [\overrightarrow{OP}] \), we instead obtain a bijection between \( A_2 \) and \( F_2 \).

  Now that we have a correspondence between \( A_2 \) and \( F_2 \), coordinates for the point \( P \) are defined simply as the \hyperref[def:left_module_basis_projection]{coordinates} of \( [\overrightarrow{OP}] \) with respect to the basis \( (x, y) \).

  Thus the pair \( (A_2, Oxy) \) has an explicit isomorphism with \( \BbbR^2 \).

  The \term{coordinate axis} of \( x \) is the unique \hyperref[def:affine_plane/ray]{ray} starting at \( O \) and containing the end of \( x \). It is called the \term{abscissa}. The coordinate axis of \( y \) is called the \term{ordinate}.
\end{definition}

\begin{remark}\label{rem:coordinate_systems}
  We sketched how to embed mental images of planes into \( \BbbR^2 \), however in mathematics we are often interested in the opposite: given a set of points in \( \BbbR^2 \), visualize them on a screen or paper and then absorb the the resulting image in our brain.

  This is one of the most powerful constructions in mathematics, yet it is so intuitive that it is not really given a lot of attention, at least until generalizations are required. Given any vector space \( V \) in the sense of \fullref{def:vector_space}, we want a way to assign a pair of numbers to each vector in \( V \). This is only possible if \( \dim V = 2 \), however we can generalize this to tuples of coordinates via bases - see \fullref{def:left_module_hamel_basis}. This well for finitely dimensional vector spaces, however we need to generalize these notion for infinitely dimensional vector spaces and general modules over \hyperref[def:left_module]{rings}. This allows us to generalize coordinates further to manifolds - see \fullref{def:topological_manifold}.

  See \fullref{subsec:vector_space_geometry} for immediate generalizations of the concepts introduced here.
\end{remark}
