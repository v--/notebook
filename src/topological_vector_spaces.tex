\section{Functional analysis}\label{sec:functional_analysis}
\subsection{Topological vector spaces}\label{subsec:topological_vector_spaces}

\begin{definition}\label{def:topological_vector_space}
  Let \( X \) be any vector space and let \( \Cal{T} \) be a topology on \( X \). The space \( (X, +, \cdot, \Cal{T}) \) is called a \Def{topological vector space} if the linear and topological structure agree, that is, the operations \( +: X \times X \to X \) and \( \cdot: X \times \R \to X \) are continuous with respect to \( \Cal{T} \).
\end{definition}

\begin{proposition}\label{thm:continuous_implies_locally_bounded}
  Let \( X \) and \( Y \) be real Hausdorff topological vector spaces, let \( D \subseteq X \) and let \( f: D \to Y \) be a continuous function.

  Then \( f \) is locally bounded.
\end{proposition}
\begin{proof}
  Let \( x_0 \in D \). The set \( f(x_0) + B_Y \subseteq Y \) is obviously bounded and open. Since \( f \) is continuous, \( f^{-1}(f(x_0) + B_Y) \) is also open. Even though \( f \) may not be surjective, \( f^{-1}(f(x_0) + B_Y) \) is nonempty, since it contains \( x_0 \).

  This implies that \( f^{-1}(f(x_0) + B_Y) \) is a neighborhood of \( x_0 \) with a bounded image. Hence \( f \) is locally bounded.
\end{proof}

\begin{definition}\label{def:locally_convex_space}\cite[1.8]{Rudin1991}
  We say that a topological vector space\Tinyref{def:topological_vector_space} is \Def{locally convex} if there exists a topological base\Tinyref{def:topological_base} of convex sets\Tinyref{def:linear_combination_hulls}.
\end{definition}

\begin{theorem}[Hahn-Banach]\label{thm:hahn_banach}\cite[24]{Йоффе1974}
  Fix a topological vector space\Tinyref{def:topological_vector_space} \( X \). Let \( A \subseteq X \) be an open convex set\Tinyref{def:linear_combination_hulls} and \( L \subseteq X \) be a subspace that is disjoint from \( A \). Then there exists a continuous linear functional \( x^* \in X \) such that
  \begin{equation*}
    \begin{array}{l}
      \Prod{x^*} x > 0, x \in A \\
      \Prod{x^*} x = 0, x \in L
    \end{array}
  \end{equation*}
\end{theorem}

\begin{corollary}\label{thm:hahn_banach_functionals_vanish_nowhere}
  The dual\Tinyref{def:dual_space} of a separable locally convex space\Tinyref{def:locally_convex_space} \( X \) does not vanish\Tinyref{def:functions_vanish_nowhere} at the nonzero vectors of \( X \).
\end{corollary}
\begin{proof}
  Fix a nonzero point \( x \in X \). The result follows from \cref{thm:hahn_banach} with \( L = \{ 0 \} \) and \( A \) -- any convex set containing \( x \) and not containing zero.
\end{proof}

\begin{corollary}\label{thm:hahn_banach_annihilator_nontrivial}
  The annihilator\Tinyref{def:vector_space_annihilator} of any proper subspace of a locally convex space\Tinyref{def:locally_convex_space} contains nonzero elements.
\end{corollary}
\begin{proof}
  Denote the proper subspace by \( L \subsetneq X \). Fix \( x \in X \setminus L \) and let \( A \) be a convex neighborhood of \( x \). The result follows from \cref{thm:hahn_banach}.
\end{proof}
