\section{Functional analysis}\label{sec:functional_analysis}

In this section, \( \K \) will refer to either \( \R \) or \( \C \).

\subsection{Topological vector spaces}\label{subsec:topological_vector_spaces}

\begin{remark}\label{remark:vector_space_set_operations}
  In analysis, it is customary to perform addition and scalar multiplication with sets in vector spaces. That is, if \( A \) and \( B \) are sets, it is customary to write
  \begin{align*}
    A + B \coloneqq \{ a + b \colon a \in A, b \in B \},
  \end{align*}
  and if \( \alpha \) is a scalar,
  \begin{equation*}
    \alpha A \coloneqq \{ \alpha a \colon a \in A \}.
  \end{equation*}

  If we use the convention in \cref{remark:singleton_sets}, this turns sets in vector spaces into an ill-defined algebraic structure, but it is useful when no metric is available, e.g. in topological vector spaces\Tinyref{def:topological_vector_space} or set-valued analysis.
\end{remark}

\begin{definition}\label{def:topological_vector_space}
  Let \( X \) be any vector space and let \( \T \) be a topology on \( X \). The space \( (X, +, \cdot, \T) \) is called a \Def{topological vector space} if the linear and topological structure agree, that is, the operations \( +: X \times X \to X \) and \( \cdot: X \times \R \to X \) are continuous with respect to \( \T \).

  See \cref{def:continuous_dual_space} and \cref{def:category_of_topological_vector_spaces} for more nuances.
\end{definition}

Given that a topological vector space \( X \) has both a topological and an algebraic structure, we should adapt certain definitions.

\begin{definition}\label{def:continuous_dual_space}
  We define the \Def{continuous dual space} \( X^* \) of a topological space \( X \) as the vector space of all continuous\Tinyref{def:continuous_function} linear functionals. This differs drastically from \cref{def:dual_space} because in the general case, the continuous dual space may be empty. See \cref{def:locally_convex_duality_pairing}.

  We use the same notation for both kinds of dual spaces because the meaning is usually clear from the context.
\end{definition}

\begin{definition}\label{def:category_of_topological_vector_spaces}
  The category \( \Cat{TopVect}_{\K} \) of topological vector spaces over \( \K \) is a subcategory of both \( \Cat{Top} \) and \( \Cat{Vect}_K \). Its morphisms are the continuous\Tinyref{def:continuous_function} linear maps\Tinyref{def:linear_operator}.
\end{definition}

\begin{definition}\label{def:locally_convex_space}\cite[1.8]{Rudin1991}
  We say that a topological vector space\Tinyref{def:topological_vector_space} is \Def{locally convex} if there exists a topological base\Tinyref{def:topological_base} of convex\Tinyref{def:convex_set} sets.
\end{definition}

\begin{remark}\label{def:locally_convex_duality_pairing}
  Given a Hausdorff locally convex space \( X \), \cref{thm:hahn_banach_functionals_vanish_nowhere} shows that the canonical duality pairing as defined in \cref{def:locally_convex_duality_pairing} is nondegenerate. If the space is not locally convex, we cannot guarantee that our restriction to continuous linear functionals would not interfere with our habits.
\end{remark}
