\subsection{Euclidean spaces}\label{subsec:euclidean_spaces}

\begin{definition}\label{def:euclidean_space}\mimprovised
  A \term{Euclidean space} of dimension \( n \) is the \hyperref[def:affine_space]{affine space \( \BbbR^n \)} equipped with the \hyperref[def:inner_product_space]{dot product} \( \inprod x y \coloneqq x^T y \), sometimes called the \term{Euclidean inner product}. We call \( \BbbR^2 \) the \term{Euclidean plane}.

  As described in \fullref{rem:structure_hierarchy}, this introduces a standard \hyperref[def:norm]{norm}, \hyperref[def:metric_space]{metric}, \hyperref[def:uniform_space]{uniformity} and \hyperref[def:topological_space]{topology}, which we call the \( n \)-dimensional \term{Euclidean norm} (resp. metric, uniformity or topology).

  Consider the \hyperref[def:sequence_space]{standard basis} \( e_1, \ldots, e_n \). We call the \hyperref[def:geometric_ray]{ray} at the origin with direction \( e_k \) the \( k \)-th \term{coordinate axis}.

  It is important to note that the meaning of the term \enquote{Euclidean space} varies in the literature --- see \fullref{rem:euclidean_space_etymology}.
\end{definition}

\begin{remark}\label{rem:euclidean_space_etymology}
  The term \enquote{Euclidean space} may have different meanings depending on the author. For example, \cite[sec. 24.1]{Тыртышников2004Лекции} defines Euclidean spaces as possibly infinite-dimensional real inner product spaces, while \cite[2.19]{Rudin1987RealAndComplex} restricts them to tuple spaces with the dot product, as we do. Gallier in \cite[176]{Gallier2011}, on the other hand, even makes a distinction between Euclidean vector spaces and Euclidean affine spaces, and defines the former as \hyperref[def:inner_product_space]{real inner product spaces}.

  \enquote{The} Euclidean space may also refer to the ambient space used by Euclid in \cite{Fitzpatrick2008}.
\end{remark}

\begin{proposition}\label{thm:isometry_iff_affine_orthogonal_operator}\mcite[thm. 7.1]{Treil2017}
  The endofunction \( f: \BbbR^n \to \BbbR^n \) over the real inner product space \( \BbbR^n \) is an \hyperref[def:isometry]{isometry} if and only if it is an \hyperref[def:affine_operator]{affine operator} whose linear part \( T(x) \coloneqq f(x) - f(\vect 0) \) is \hyperref[def:unitary_operator]{orthogonal}.
\end{proposition}
\begin{proof}
  \SufficiencySubProof Suppose that \( f \) is an isometry.

  First note that
  \begin{equation*}
    \inprod { f(x) } { f(y) } = \inprod x y.
  \end{equation*}

  Indeed, we have
  \begin{equation*}
    \norm{ f(x) - f(y) }^2 = \norm{ f(x) }^2 + \norm{ f(y) }^2 - 2 \inprod { f(x) } { f(y) }
  \end{equation*}
  and
  \begin{equation*}
    \norm{ x - y }^2 = \norm{x}^2 + \norm{y}^2 - 2 \inprod x y.
  \end{equation*}

  Therefore,
  \begin{equation*}
    \inprod { f(x) } { f(y) }
    =
    \frac{ \norm{ f(x) - f(y) }^2 - \norm{ f(x) }^2 - \norm{ f(y) }^2 } 2
    =
    \frac{ \norm{ x - y }^2 - \norm{ x }^2 - \norm{ y }^2 } 2
    =
    \inprod x y.
  \end{equation*}

  Now we can show additivity of \( T \):
  \begin{balign*}
    &\phantom{{}={}}
    \norm{ T(x + y) - T(x) - T(y) }^2
    = \\ &=
    \norm{ f(x + y) - f(0) - f(x) + f(0) - f(y) + f(0) }^2
    = \\ &=
    \norm{ \parens{ f(x + y) - f(x) } - \parens{ f(y) - f(0) } }^2
    = \\ &=
    \norm{ f(x + y) - f(x) }^2 + \norm{ f(y) - f(0) }^2 - 2 \inprod{ f(x + y) - f(x) } { f(y) - f(0) }
    = \\ &=
    \norm{ x + y - x }^2 + \norm{y}^2 - 2 \inprod{x + y} y + 2 \inprod x y - 2 \inprod {x + y} 0 + 2 \inprod x 0
    = \\ &=
    2 \norm{y}^2 - 2 \inprod x y - 2 \norm{y}^2 + 2 \inprod x y.
  \end{balign*}

  The norm is zero, hence \( T(x + y) = T(x) + T(y) \).

  Similarly,
  \begin{balign*}
    &\phantom{{}={}}
    \norm{ T(\lambda x) - \lambda T(x) }^2
    = \\ &=
    \norm{ f(\lambda x) - f(0) - \lambda f(x) + \lambda f(0) }^2
    = \\ &=
    \norm{ f(\lambda x) - f(0) }^2 + \norm{ \lambda f(x) - \lambda f(0) }^2 - 2 \inprod{ f(\lambda x) - f(0) } { \lambda f(x) - \lambda f(0) }
    = \\ &=
    \norm{\lambda x}^2 + \lambda^2 \norm{x}^2 - 2 \lambda \inprod{ f(\lambda x) - f(0) } { f(x) - f(0) }
    = \\ &=
    2 \lambda^2 \norm{x}^2 - 2 \lambda^2 \inprod x x.
  \end{balign*}

  This norm is also zero, hence \( T(\lambda x) = \lambda T(x) \).

  Finally, we must show that \( T \) is a unitary operator:
  \begin{balign*}
    \inprod{ T(x) }{ T(y) }
    &=
    \inprod{ f(x) - f(0) }{ f(y) - f(0) }
    = \\ &=
    \inprod{ f(x) }{ f(y) } - \inprod{ f(x) }{ f(0) } - \inprod{ f(0) }{ f(y) } + \inprod{ f(0) }{ f(0) }
    = \\ &=
    \inprod x y.
  \end{balign*}

  \NecessitySubProof Suppose that \( f(x) = Tx + f_0 \) for some unitary operator \( T \) and some vector \( f_0 \). Then
  \begin{balign*}
    \norm{ f(x) - f(y) }^2
    &=
    \norm{ Tx - f_0 - Ty + f_0 }^2
    = \\ &=
    \norm{ T(x - y) }^2
    = \\ &=
    \inprod{ T(x - y) }{ T(x - y) }
    = \\ &=
    \inprod{ x - y }{ T^{-1} T(x - y) }
    = \\ &=
    \norm{ x - y }^2.
  \end{balign*}
\end{proof}

\begin{lemma}\label{thm:dot_product_and_outer_product}
  For any field \( \BbbK \) and vectors \( x \), \( y \) and \( z \) in \( \BbbK^n \), we have
  \begin{equation*}
    x^T y z = y z^T x.
  \end{equation*}
\end{lemma}
\begin{proof}
  The \( k \)-th coordinate of \( x^T y z \) is \( (x^T y) z_k \).

  The \( k \)-th coordinate of \( y z^T x \) is
  \begin{equation*}
    \sum_{i=1}^n (y_i z_k) x_i
    =
    z_k \sum_{i=1}^n y_i x_i
    =
    (x^T y) z_k.
  \end{equation*}
\end{proof}

\begin{definition}\label{def:rigid_motion}\mimprovised
  A \term{rigid motion} is an \hyperref[def:isometry]{isometry} in an \hyperref[def:euclidean_space]{Euclidean space}. \Fullref{thm:isometry_iff_affine_orthogonal_operator} provides an equivalent characterization: a rigid motion is an affine operator whose linear part is \hyperref[def:unitary_operator]{orthogonal}. Thus, given a rigid motion \( f: \BbbR^n \to \BbbR^n \), there exists an orthogonal operator \( T: \BbbR^n \to \BbbR^n \) and a directional vector \( d \) such that
  \begin{equation*}
    f(x) = Tx + d.
  \end{equation*}

  The following are common rigid motions:
  \begin{thmenum}
    \thmitem{def:rigid_motion/translation} The \term{translation} along the \term{direction} \( d \) is
    \begin{equation*}
      f(x) = x + d.
    \end{equation*}

    \begin{figure}[!ht]
      \hfill
      \includegraphics[align=c]{output/def__rigid_motion__translation__2d.pdf}
      \hfill
      \includegraphics[align=c]{output/def__rigid_motion__translation__3d.pdf}
      \hfill
      \hfill
      \caption{Translation of the unit square in \( \BbbR^2 \) and unit cube in \( \BbbR^3 \).}\label{fig:def/rigid_motion/translation}
    \end{figure}

    The term \enquote{translation} generalizes to an arbitrary \hyperref[def:magma]{magma} via
    \begin{equation*}
      f(x) \coloneqq v \cdot x.
    \end{equation*}

    \thmitem{def:rigid_motion/rotation} A \term{rotation} is an \hyperref[def:orthogonal_operator]{orthogonal} operator with \hyperref[def:matrix_determinant]{determinant} \( 1 \).

    \begin{figure}[!ht]
      \hfill
      \includegraphics[align=c]{output/def__rigid_motion__rotation__2d.pdf}
      \hfill
      \includegraphics[align=c]{output/def__rigid_motion__rotation__3d.pdf}
      \hfill
      \hfill
      \caption{Rotation of the unit square in \( \BbbR^2 \) and unit cube in \( \BbbR^3 \).}\label{fig:def/rigid_motion/rotation}
    \end{figure}

    \thmitem{def:rigid_motion/reflection} The \term{Householder reflection} with \term{normal vector} \( v \) is, assuming \( \norm v = 1 \),
    \begin{equation*}
      f(x) \coloneqq x - (2 \inprod x v) v.
    \end{equation*}

    This can be expressed in matrix form as
    \begin{equation*}
      f(x)
      =
      x - (2 x^T v) v
      \reloset {\ref{thm:dot_product_and_outer_product}} =
      x - 2 v v^T x
      =
      (I_n - 2 v v^T) x.
    \end{equation*}

    \begin{figure}[!ht]
      \hfill
      \includegraphics[align=c]{output/def__rigid_motion__reflection__2d.pdf}
      \hfill
      \includegraphics[align=c]{output/def__rigid_motion__reflection__3d.pdf}
      \hfill
      \hfill
      \caption{Reflection of the unit square in \( \BbbR^2 \) and unit cube in \( \BbbR^3 \).}\label{fig:def/rigid_motion/reflection}
    \end{figure}
  \end{thmenum}
\end{definition}

\begin{proposition}\label{thm:rigid_motion_fixed_point}
  The point \( x_0 \) is a \hyperref[def:fixed_point]{fixed point} of the \hyperref[def:rigid_motion]{rigid motion} \( f(x) \) if and only if
  \begin{equation}\label{eq:thm:rigid_motion_fixed_point}
    f(x) = x_0 + T(x - x_0),
  \end{equation}
  where \( T \) is the linear part of \( f \).
\end{proposition}
\begin{proof}
  \SufficiencySubProof Let \( x_0 \) be a fixed point of \( f(x) \).

  Then \( T(x) \coloneqq f(x + x_0) - f(x_0) \) is a linear operator since \( f \) is affine. Thus,
  \begin{equation*}
    T(x - x_0) = f(x) - f(x_0) = f(x) - x_0,
  \end{equation*}
  and \eqref{eq:thm:rigid_motion_fixed_point} follows.

  \NecessitySubProof Suppose that \eqref{eq:thm:rigid_motion_fixed_point} holds. Then
  \begin{equation*}
    f(x_0) = x_0 + T(x_0 - x_0) = x_0.
  \end{equation*}
\end{proof}

\begin{definition}\label{def:half_space}\mcite[41]{Clarke2013}
  Given a \hyperref[def:affine_hyperplane]{hyperplane} \( H \) in \( \BbbR^n \) defined by the affine functional \( f(x) = \inprod l x - a \), its \term{closed half-spaces} are defined as
  \begin{equation*}
    H^+ \coloneqq \set{ f(x) \geq 0 } = \set{ \inprod l x \geq a }\phantom{,}
  \end{equation*}
  and
  \begin{equation*}
    H^- \coloneqq \set{ f(x) \leq 0 } = \set{ \inprod l x \leq a }.
  \end{equation*}

  In \( \BbbR^2 \), we call them \term{half-planes}.

  \begin{figure}[!ht]
    \centering
    \includegraphics{output/def__half_space.pdf}
    \caption{Half-planes}\label{fig:def:half_space}
  \end{figure}

  If the inequalities are strict, we instead obtain \term{open half-spaces}.
\end{definition}

\begin{definition}\label{def:normal_vector}\mimprovised
  A \term{normal vector} to an \hyperref[def:affine_subspace]{affine subspace} \( L \) is a nonzero vector that is \hyperref[def:orthogonality]{orthogonal} to the direction \( \vect L \).
\end{definition}

\begin{example}\label{ex:def:normal_vector}
  We list several examples of \hyperref[def:normal_vector]{normal vectors}.

  \begin{thmenum}
    \thmitem{ex:def:normal_vector/full} The space \( \BbbR^n \) as a subspace of itself has no normal vectors, since the orthogonal complement of \( \BbbR^n \) is the trivial subspace, but we explicitly require normal vectors to be nonzero.

    \thmitem{ex:def:normal_vector/empty} Conversely, every vector in \( \BbbR^n \) is normal to the trivial subspace.

    \thmitem{ex:def:normal_vector/vector_subspace} For the vector subspace \( \BbbR^k \) of \( \BbbR^n \), the normal to \( \BbbR^k \) vectors form the \hyperref[def:orthogonal_complement]{orthogonal complement}
    \begin{equation*}
      \BbbR^{n-k} \cong \set{ (0, \ldots, 0, x_{k+1}, \cdots, x_n) \given (x_1, \ldots, x_n) \in \BbbR^n }.
    \end{equation*}
  \end{thmenum}
\end{example}
