\subsection{First-order proofs}\label{subsec:first_order_proofs}

\begin{definition}\label{def:first_order_derivation_system}\MarginCite[sec. 12.1]{OpenLogic20201202}
  A \Def{first-order derivation system} for the \hyperref[def:first_order_syntax]{language} \( \CL \) consists of:

  \begin{DefEnum}
    \ILabel{def:first_order_derivation_system/axioms} A set of \hyperref[def:first_order_syntax]{formulas} in \( \CL \), which we call \Def{logical axioms} (to distinguish them with \hyperref[def:first_order_proofs/axioms_and_theorems]{nonlogical axioms}).

    \ILabel{def:first_order_derivation_system/rules} A set of \hyperref[def:relation]{relations} of potentially different arity on the set of formulas of \( \CL \), which we call \Def{inference rules}. If \( r \) is an inference rule of arity \( n \) and if \( \psi_1, \ldots, \psi_n \) and \( \varphi \) are formulas, we say that the rule allows us to \Def{infer} \( \varphi \) from \( \psi_1, \ldots, \psi_n \) if \( (\psi_1, \ldots, \psi_n, \varphi) \in r \). We write this symbolically as
    \begin{equation*}
      \begin{prooftree}
        \hypo{\psi_1}
        \hypo{\cdots}
        \hypo{\psi_n}
        \infer3[r]{\varphi}
      \end{prooftree}
    \end{equation*}
  \end{DefEnum}
\end{definition}

\begin{definition}\label{def:first_order_proofs}
  We introduce \enquote{syntactic} counterparts to the concepts from \fullref{def:first_order_semantics}.

  Given a \hyperref[def:first_order_derivation_system]{derivation system} for the language \( \CL \), we introduce the following notions:
  \begin{DefEnum}
    \ILabel{def:first_order_proofs/derivation}\MarginCite[def. 12.1]{OpenLogic20201202}A sequence \( \varphi_1, \ldots, \varphi_n \) of formulas is called a \Def{derivation} from the set \( \Gamma \) of formulas if, for any \( k = 1, \ldots, n \), either of the following hold:
    \begin{DefEnum}
      \ILabel{def:first_order_proofs/derivation/logical_axiom} \( \varphi_k \) is a \hyperref[def:first_order_derivation_system/axioms]{logical axiom} in the derivation system.
      \ILabel{def:first_order_proofs/derivation/nonlogical_axiom} \( \varphi_k \in \Gamma \)
      \ILabel{def:first_order_proofs/derivation/inference} There exist indices \( i, j < k \) such that \( \varphi_k \) can be inferred from \( \varphi_i \) and \( \varphi_j \) using any of the \hyperref[def:first_order_derivation_system/rules]{inference rules}.
    \end{DefEnum}

    Derivations are also called \Def{proofs}.

    \ILabel{def:first_order_proofs/derivability} If there exists a derivation from \( \Gamma \) such that \( \varphi \) is the last element, we say that \( \varphi \) is \Def{derivable} from \( \Gamma \).

    \ILabel{def:first_order_proofs/derivation_closure} The \Def{derivation closure} \( \Cl^{\vdash}(\Gamma) \) of the set \( \Gamma \) is the set of all formulas that can be derived from \( \Gamma \).

    \ILabel{def:first_order_proofs/completeness}\MarginCite[def. 13.1]{OpenLogic20201202}If, for every formula \( \varphi \), either \( \varphi \in \Gamma \) or \( \neg \varphi \in \Gamma \), we say that \( \Gamma \) is a \Def{complete set of formulas}. Note that we do not eliminate double negation.

    \ILabel{def:first_order_proofs/consistency}\MarginCite[def. 20.16]{OpenLogic20201202}If \( \bot \) cannot be derived from \( \Gamma \), we say that \( \Gamma \) is \Def{consistent}.

    \ILabel{def:first_order_proofs/axioms_and_theorems} If we have fixed a set \( \Alpha \) of formulas, we say that formulas in \( \Alpha \) are \Def{(nonlogical) axioms} and formulas in the closure \( \Cl^{\vdash}(\Alpha) \) are \Def{theorems}. If \( \Alpha = \varnothing \), we obtain \Def{logical theorems} that are derived solely from the \hyperref[logical axioms]{logical axioms} using the inference rules.
  \end{DefEnum}
\end{definition}

\begin{definition}\label{def:derivability_and_satisfiability}
  We introduce two notions connecting \hyperref[def:first_order_proofs/derivability]{derivability} and \hyperref[def:first_order_semantics/satisfiability]{satisfiability}:
  \begin{DefEnum}
    \ILabel{def:derivability_and_satisfiability/soundness} If, for any formula \( \varphi \), \( \vdash \varphi \) implies \( \vDash \varphi \), we say that the derivation system is \Def{sound}.

    \ILabel{def:derivability_and_satisfiability/complete} Dually, if \( \vDash \varphi \) implies \( \vdash \varphi \) for formulas \( \varphi \), we say that the derivation system is \Def{complete}.
  \end{DefEnum}
\end{definition}

\begin{definition}\label{def:hilberts_derivation_system}\MarginCite[def. 20.5 \\ def. 20.7]{OpenLogic20201202}
  \Def{Hilbert's} \hyperref[def:first_order_derivation_system]{derivation system} for the language \( \CL \) consists of: a single inference rule called \Def{modus ponens}

  \begin{RefList}
    \IRef{def:first_order_derivation_system/axioms} The following list of logical axioms (we can remove most of them - see \fullref{rem:hilberts_derivation_system_axioms}):
    \begin{BreakableAlign}
                                          \text{base}   & \text{ axioms}                                                                              \nonumber \\
      \varphi                                           &\rightarrow (\psi \rightarrow \varphi)                                                       \label{def:hilberts_derivation_system/rules/implies_intro} \\
      (\varphi \rightarrow (\psi \rightarrow \theta))   &\rightarrow ((\varphi \rightarrow \psi) \rightarrow (\psi \rightarrow \theta))               \label{def:hilberts_derivation_system/rules/implies_distributivity} \\
                                      \text{quantifier} & \text{ axioms}                                                                              \nonumber \\
      \forall \xi \centerdot \varphi                    &\rightarrow \varphi[\xi \mapsto \tau],     \hspace{0.95cm} \tau \text{ is a closed term}     \label{def:hilberts_derivation_system/rules/forall_elim} \\
      \varphi[\xi \mapsto \tau]                         &\rightarrow \exists \xi \centerdot \varphi, \hspace{1.5cm} \tau \text{ is a closed term}     \label{def:hilberts_derivation_system/rules/exists_intro} \\
                                        \text{equality} & \text{ axioms}                                                                              \nonumber \\
                                                        &\phantom{{}\rightarrow{}} (\xi \doteq \xi)                                                   \label{def:hilberts_derivation_system/rules/eq_reflexivity} \\
      (\xi \doteq \eta)                                 &\rightarrow (\tau[\zeta \mapsto \xi] \doteq \tau[\zeta \mapsto \eta])                        \label{def:hilberts_derivation_system/rules/eq_term_subst} \\
      (\xi \doteq \eta)                                 &\rightarrow (\varphi[\zeta \mapsto \xi] \rightarrow \varphi[\zeta \mapsto \eta])             \label{def:hilberts_derivation_system/rules/eq_formula_subst} \\
                                            \text{top } & \text{definition axioms}                                                                    \nonumber \\
                                                        &\phantom{{}\rightarrow{}} \top                                                               \label{def:hilberts_derivation_system/rules/top} \\
                                    \text{equivalence } & \text{definition axioms}                                                                    \nonumber \\
      (\varphi \leftrightarrow \psi)                    &\rightarrow (\varphi \rightarrow \psi)                                                       \label{def:hilberts_derivation_system/rules/equiv_elim_left} \\
      (\varphi \leftrightarrow \psi)                    &\rightarrow (\psi \rightarrow \varphi)                                                       \label{def:hilberts_derivation_system/rules/equiv_elim_right} \\
      (\varphi \rightarrow \psi)                        &\rightarrow ((\psi \rightarrow \varphi) \rightarrow (\varphi \leftrightarrow \psi))          \label{def:hilberts_derivation_system/rules/equiv_intro} \\
                                    \text{conjunction } & \text{definition axioms}                                                                    \nonumber \\
      (\varphi \wedge \psi)                             &\rightarrow \varphi                                                                          \label{def:hilberts_derivation_system/rules/and_elim_left} \\
      (\varphi \wedge \psi)                             &\rightarrow \psi                                                                             \label{def:hilberts_derivation_system/rules/and_elim_right} \\
      \varphi                                           &\rightarrow (\psi \rightarrow (\varphi \wedge \psi))                                         \label{def:hilberts_derivation_system/rules/and_intro} \\
                                    \text{disjunction } & \text{definition axioms}                                                                    \nonumber \\
      \varphi                                           &\rightarrow (\varphi \vee \psi)                                                              \label{def:hilberts_derivation_system/rules/or_elim_left} \\
      \varphi                                           &\rightarrow (\psi \vee \varphi)                                                              \label{def:hilberts_derivation_system/rules/or_elim_right} \\
      (\varphi \rightarrow \psi)                        &\rightarrow ((\theta \rightarrow \psi) \rightarrow ((\varphi \vee \theta) \rightarrow \psi)) \label{def:hilberts_derivation_system/rules/or_intro} \\
                                       \text{negation } & \text{definition axioms}                                                                    \nonumber \\
      (\varphi \to \bot)                                &\rightarrow \neg \varphi                                                                     \label{def:hilberts_derivation_system/rules/neg_falsum} \\
      (\varphi \rightarrow \psi)                        &\rightarrow ((\varphi \rightarrow \neg \psi) \rightarrow \neg \varphi)                       \label{def:hilberts_derivation_system/rules/neg_intro} \\
                         \text{principles of explosion} & \text{ (ex falso quodlibet)}                                                                \nonumber \\
      \bot                                              &\rightarrow \varphi                                                                          \label{def:hilberts_derivation_system/rules/explosion} \\
      \neg \varphi                                      &\rightarrow (\varphi \rightarrow \psi)                                                       \label{def:hilberts_derivation_system/rules/explosion_neg} \\
                     \text{double negation elimination} & \text{ (stability)}                                                                         \nonumber \\
      \neg \neg \varphi                                 &\rightarrow \varphi                                                                          \label{def:hilberts_derivation_system/rules/stab} \\
      ((\varphi \rightarrow \bot) \rightarrow \bot)     &\rightarrow \varphi                                                                          \label{def:hilberts_derivation_system/rules/stab_bot}
    \end{BreakableAlign}

    Note that rules \eqref{def:hilberts_derivation_system/rules/eq_reflexivity}-\eqref{def:hilberts_derivation_system/rules/eq_formula_subst} only make sense if the language has a formal equality symbol.

    \IRef{def:first_order_derivation_system/rules} The following three rules of inference:
    \begin{DefEnum}
      \ILabel{def:hilberts_derivation_system/rules/modus_ponens} \Def{Modus ponens}:
      \begin{equation}\label{eq:def:hilberts_derivation_system/rules/modus_ponens}
        \begin{prooftree}
          \hypo{\varphi}
          \hypo{\varphi \rightarrow \psi}
          \infer2[mp]{\psi}
        \end{prooftree}
      \end{equation}

      \ILabel{def:hilberts_derivation_system/rules/forall_intro} For \( \xi \not\in \Bold{Free}(\varphi) \), \Def{universal quantifier introduction}:
      \begin{equation}\label{eq:def:hilberts_derivation_system/rules/forall_intro}
        \begin{prooftree}
          \hypo{\varphi \rightarrow \psi}
          \infer1[\( \forall \) \text{int}]{\varphi \rightarrow \forall \xi \centerdot \psi}
        \end{prooftree}
      \end{equation}

      \ILabel{def:hilberts_derivation_system/rules/exists_intro} For \( \xi \not\in \Bold{Free}(\psi) \), \Def{existential quantifier introduction}:
      \begin{equation}\label{eq:def:hilberts_derivation_system/rules/exists_intro}
        \begin{prooftree}
          \hypo{\varphi \rightarrow \psi}
          \infer1[\( \exists \) \text{int}]{\forall \xi \centerdot \varphi \rightarrow \psi}
        \end{prooftree}
      \end{equation}
    \end{DefEnum}
  \end{RefList}
\end{definition}

\begin{proposition}\label{thm:hiblerts_derivation_system_properties}
  \hyperref[def:hilberts_derivation_system]{Hilbert's derivation system} has the following basic properties:

  \begin{PropEnum}
    \ILabel{thm:hiblerts_derivation_system_properties/one_explosion} Any of the two explosion principles axioms is derivable from the rest of the axioms without the stability axioms. In particular, we can speak of a single explosion principle in different variants (e.g. \eqref{def:hilberts_derivation_system/rules/explosion} is more conventional).

    \ILabel{thm:hiblerts_derivation_system_properties/one_stab} Any of the two stability axioms is derivable from the rest of the axioms without the explosion axioms. Removing any of the two does not affect the derivation capabilities of Hilbert's system. In particular, we can speak of a single stability axiom in different variants (e.g. \eqref{def:hilberts_derivation_system/rules/stab} is more conventional).

    \ILabel{thm:hiblerts_derivation_system_properties/stab_derives_explosion} Any of the two explosion axioms is derivable from the rest of the axioms without the other explosion axiom. In particular, removing both of the explosion principles does not affect the derivation capabilities of Hilbert's system.

    \ILabel{thm:hiblerts_derivation_system_properties/explosion_does_not_derive_stab} Neither of the stability axioms is provable from the rest of the system without the other stability axiom.
  \end{PropEnum}
\end{proposition}

\begin{theorem}[Soundness of Hilbert's derivation system]\label{thm:soundness_of_hilbers_derivation_system}\MarginCite[prop. 20.35]{OpenLogic20201202}
  \hyperref[def:hilberts_derivation_system]{Hilbert's derivation system} is \hyperref[def:derivability_and_satisfiability/soundness]{sound}.
\end{theorem}

\begin{remark}\label{rem:hilberts_derivation_system_axioms}
  Because we only need a subset of the propositional constants and connectives (see \fullref{rem:smaller_propositional_language}), we can eliminate some of the axioms in \fullref{def:first_order_derivation_system/axioms} by considering a smaller \hyperref[def:first_order_language]{first-order logic language}. As we shall see, however, by eliminating some of the rules we change the relationship between syntax and semantics.

  We will present several subsystems of \hyperref[def:hilberts_derivation_system]{Hilbert's derivation system}. We will simplify our exposition by only considering propositional proofs. This means ignoring rules \eqref{def:hilberts_derivation_system/rules/forall_elim}-\eqref{def:hilberts_derivation_system/rules/eq_formula_substs} for quantifiers and formal equality and leaving only \hyperref[def:hilberts_derivation_system/rules/modus_ponens]{modus ponens} as a rule of inference. Since \hyperref[subsec:language_of_propositional_logic]{propositional logic} is a special case of \hyperref[subsec:language_of_first_order_logic]{first-order logic} (see \fullref{rem:propositional_logic_as_first_order_logic}), this completely reduces our discussion to propositional logic.

  \begin{RemEnum}
    \ILabel{rem:hilberts_derivation_system_axioms/classical_logic} The full set of axioms are called the axioms of \Def{classical logic}. Without changing the underlying language, we can remove the explosion axioms \eqref{def:hilberts_derivation_system/rules/explosion} and \eqref{def:hilberts_derivation_system/rules/explosion_neg} and one of the two stability axioms \eqref{def:hilberts_derivation_system/rules/stab} or \eqref{def:hilberts_derivation_system/rules/stab_bot} since they are provable from the rest.

    We can actually use any of the subsets of constants, negation and connectives corresponding to the complete sets of Boolean functions in \fullref{ex:posts_completeness_theorem}. If we remove some of the constants or connectives of the language, we can also remove the corresponding axioms from \fullref{def:first_order_derivation_system/axioms}. For the rest of the remark, it would be most beneficial for us to remove everything except for the falsum \( \bot \) and material implication \( \rightarrow \). The minimal subset of axioms that still correspond to classical logic is then consists of only three rules: the two base rules \eqref{def:hilberts_derivation_system/rules/implies_intro}, \eqref{def:hilberts_derivation_system/rules/implies_distributivity} and stability \eqref{def:hilberts_derivation_system/rules/stab_bot}.

    \ILabel{rem:hilberts_derivation_system_axioms/intuitionistic_logic}\MarginCite[def 53.10]{OpenLogic20201202}If we instead consider the base rules \eqref{def:hilberts_derivation_system/rules/implies_intro}, \eqref{def:hilberts_derivation_system/rules/implies_distributivity} and the principle of explosion \eqref{def:hilberts_derivation_system/rules/explosion}, we obtain \Def{intuitionistic logic}. By \fullref{thm:hiblerts_derivation_system_properties}, we can prove the principle of explosion from stability but not vice versa, therefore we can prove less theorems in intuitionistic logic. It is said to be a \enquote{weaker} derivation system. In particular, \fullref{thm:soundness_of_hilbers_derivation_system} does not hold. As a consequence, the set \( \Set{ \rightarrow, \bot } \) is no longer sufficient to derive statements analogous to \fullref{thm:boolean_function_equivalences} involving negation \( \neg \), top \( \top \) and the rest of the connectives. The \enquote{definition} axioms in \fullref{def:hilberts_derivation_system/rules} are designed to emulate the behavior of conjunctions and disjunctions in classical logic.

    The advantage of intuitionistic logic is that we cannot perform proofs by contradiction and instead must explicitly construct objects in proofs unless we are using \fullref{thm:aoc}. For this reason, intuitionistic logic is also called constructive logic.

    \ILabel{rem:hilberts_derivation_system_axioms/minimal_logic} We can go further and remove the explosion principle from intuitionistic logic. The obtained system is called \Def{minimal logic}. In the simplest case, we only have the two base axioms \eqref{def:hilberts_derivation_system/rules/implies_intro} and \eqref{def:hilberts_derivation_system/rules/implies_distributivity}. In particular, the falsum \( \bot \) does not affect derivation and we can remove it from the language, in which case we are left with a single propositional connective \( \rightarrow \), no negation and no constants. Even after adding the propositional constants, connectives and negation to the language and after adding the \enquote{definition} axioms to the derivation system, we obtain a derivation system that is very restricted compared to Hilbert's classical logic.

    Since we are working with a simpler system, however, any derivation is minimal logic is also a derivation in intuitionistic and classical logic.
  \end{RemEnum}
\end{remark}

\begin{theorem}[Deduction theorem]\label{thm:deduction_theorem}\MarginCite[thm. 20.23]{OpenLogic20201202}
  In \hyperref[rem:hilberts_derivation_system_axioms/minimal]{Hilbert-style minimal logic} (and, hence, all of its extensions like classical logic), we have \( \Gamma \cup \{ \psi \} \vdash \varphi \) if and only if \( \Gamma \vdash \psi \rightarrow \varphi \). Compare this result to \fullref{thm:semantic_deduction_theorem}.
\end{theorem}
