\subsection{First-order proofs}\label{subsec:first_order_proofs}

\begin{definition}\label{def:first_order_axiomatic_derivation_system}\mcite[sec. 18.1]{OpenLogicFull}
  A \term{first-order derivation system} for the \hyperref[def:first_order_syntax]{language} \( \mscrL \) consists of:

  \begin{thmenum}
    \thmitem{def:first_order_axiomatic_derivation_system/rules} A set of \hyperref[def:relation]{relations} of potentially different arity on the set of formulas of \( \mscrL \), which we call \term{inference rules}. If \( r \) is an inference rule of arity \( n + 1 \), we say that the rule allows us to \term{infer} \( \varphi \) from \( \varphi_1, \ldots, \varphi_n \) if \( (\varphi_1, \ldots, \varphi_n, \psi) \in r \). We write this symbolically as
    \begin{equation*}
      \begin{prooftree}
        \hypo{\varphi_1}
        \hypo{\cdots}
        \hypo{\varphi_n}
        \infer3[r]{\psi}
      \end{prooftree}
    \end{equation*}

    A single rule is usually formulated via placeholder formulas, e.g. \hyperref[eq:def:implicational_logic/rules/modus_ponens]{modus ponens}, in which we replace \( \varphi \) and \( \psi \) with whatever formula has the necessary form.

    \thmitem{def:first_order_axiomatic_derivation_system/axioms} A set of \hyperref[def:first_order_syntax]{formulas} in \( \mscrL \), which we call \term{logical axioms} (to distinguish them from \hyperref[def:first_order_derivability/axioms_and_theorems]{nonlogical axioms}).
  \end{thmenum}

  Because of \fullref{rem:propositional_logic_as_first_order_logic}, we can consider (more restricted) derivation systems for propositional logic.
\end{definition}

\begin{definition}\label{def:first_order_derivability}\mcite[def. 18.1]{OpenLogicFull}
  We now introduce the syntactic analog to \hyperref[def:first_order_semantics/entailment]{entailment}. We assume that we are working with a fixed \hyperref[def:first_order_axiomatic_derivation_system]{derivation system} for some language \( \mscrL \).

 A sequence \( \varphi_1, \ldots, \varphi_n \) of formulas is called a \term{derivation} from the set \( \Gamma \) of formulas if, for any \( k = 1, \ldots, n \), either of the following hold:
  \begin{thmenum}
    \thmitem{def:first_order_derivability/logical_axiom} \( \varphi_k \) is a \hyperref[def:first_order_axiomatic_derivation_system/axioms]{logical axiom} in the derivation system.

    \thmitem{def:first_order_derivability/nonlogical_axiom} \( \varphi_k \) is a \term{nonlogical axiom}, meaning that \( \varphi_k \in \Gamma \).

    \thmitem{def:first_order_derivability/inference} There exist indices \( i_1, \ldots, i_m < k \) such that \( \varphi_k \) can be inferred from \( \varphi_{i_1}, \ldots, \varphi_{i_m} \) using any of the \hyperref[def:first_order_axiomatic_derivation_system/rules]{inference rules}.
  \end{thmenum}

  A derivation from \( \Gamma \) ending with the formula \( \varphi \) is also called a \term{proof} of \( \varphi \) from the nonlogical axioms \( \Gamma \). In this case, we write \( \Gamma \vdash \varphi \) and say that \( \varphi \) is a \term{theorem} of \( \Gamma \) and that \( \varphi \) is \term{derivable} from \( \Gamma \). If \( \psi \) is derivable from \( \varphi \) and vice versa, we say that \( \varphi \) and \( \psi \) are \term{interderivable}.

  If \( \Gamma = \varnothing \), we obtain \term{logical theorems} that are derived solely from the \hyperref[def:first_order_axiomatic_derivation_system/axioms]{logical axioms} using the inference rules.

  This definitions are specifically tailored at derivation systems. See \fullref{def:first_order_theory} for the more general definition of a theory axiomatized by a set.
\end{definition}

\begin{proposition}\label{thm:derivation_system_transitivity}
  In any deductive system, if \( \Gamma \vdash \varphi \), then \( \Gamma, \Delta \vdash \varphi \) for any formula \( \varphi \) and any sets \( \Gamma \) and \( \Delta \).

  If every formula in \( \Delta \) is derivable from \( \Gamma \), then the converse also holds: \( \Gamma, \Delta \vdash \varphi \) implies \( \Gamma \vdash \varphi \).
\end{proposition}
\begin{proof}
  If we can use \fullref{def:first_order_derivability/inference} to infer \( \varphi \) from \( \Gamma \), then adding additional axioms does not change anything.

  The second part of the theorem has a tad more complicated proof. Assume that every formula in \( \Delta \) is derivable from \( \Gamma \) and that \( \Gamma, \Delta \vdash \varphi \). Let \( \psi_1, \ldots, \psi_{n-1}, \varphi \) be a derivation of \( \varphi \) from \( \Delta \) and let \( \theta_{k,1}, \ldots, \theta_{k,m_k} \) be a derivation of \( \psi_k \) from \( \Gamma \). Then the concatenated sequence
  \begin{equation*}
    \theta_{1,1}, \ldots, \theta_{1,m_1}, \quad \theta_{2,1}, \ldots, \theta_{2,m_2}, \quad \ldots, \quad \theta_{n-1,1}, \ldots, \theta_{k-1,m_{k-1}}
  \end{equation*}
  is a derivation of \( \varphi \) from \( \Gamma \).
\end{proof}

\begin{definition}\mcite[sec. 1]{Wasilewska2010}\label{def:positive_implicational_propositional_derivation_system}
  The \term{positive implicational propositional derivation system} is extraordinarily simple. It is based on the \hyperref[def:propositional_language]{propositional language} but limited to only the \hyperref[def:propositional_language/connectives/conditional]{conditional connective} \( \rightarrow \), without any \hyperref[def:propositional_language/constants]{propositional constants} or \hyperref[def:propositional_language/negation]{negation}.

  \begin{itemize}
    \item The system has a single \hyperref[def:first_order_axiomatic_derivation_system/rules]{derivation rule} called \term{modus ponens} (Latin for \enquote{mode that by affirming affirms}):
    \begin{equation}\label{eq:def:implicational_logic/rules/modus_ponens}\tag{mp}
      \begin{prooftree}
        \hypo{\varphi}
        \hypo{\varphi \rightarrow \psi}
        \infer2[mp]{\psi}
      \end{prooftree}
    \end{equation}

    \item The system has infinitely many \hyperref[def:first_order_axiomatic_derivation_system/logical]{logical axioms} but only two axiom schemas:
    \begin{thmenum}
      \thmitem{def:positive_implicational_propositional_derivation_system/axioms/imp_intro} We can \enquote{introduce} an \hyperref[def:material_implication]{implication} whose consequent is \( \varphi \) and whose antecedent is any other formula \( \psi \):
      \begin{equation}\label{eq:def:positive_implicational_propositional_derivation_system/axioms/imp_intro}
        \varphi \rightarrow (\psi \rightarrow \varphi).
      \end{equation}

      \thmitem{def:positive_implicational_propositional_derivation_system/axioms/imp_distr} We can \hyperref[def:semiring/distributivity]{distribute} implications over other implications:
      \begin{equation}\label{eq:def:positive_implicational_propositional_derivation_system/axioms/imp_distr}
        \parens[\Big]{ \varphi \rightarrow (\psi \rightarrow \theta) } \rightarrow \parens[\Big]{ (\varphi \rightarrow \psi) \rightarrow (\varphi \rightarrow \theta)}.
      \end{equation}
    \end{thmenum}
  \end{itemize}

  The adjective \enquote{positive} in the name of the system refers to the impossibility to negate a formula (compare to \fullref{def:positive_universal_formula}). \enquote{Implicational} refers to the fact that all formulas are \hyperref[def:material_implication]{material implications} and the \hyperref[eq:def:implicational_logic/rules/modus_ponens]{sole derivation rule} is based on eliminating the connective.
\end{definition}

\begin{example}\mcite[ex. 1.1]{Wasilewska2010}\label{ex:propositional_positive_implicational_logic/identity}
  Fix any \hyperref[def:positive_implicational_propositional_derivation_system]{positive implicational formula} \( \varphi \). We will construct a derivation of the implication
  \begin{equation}\label{eq:ex:propositional_positive_implicational_logic/identity}
    \varphi \rightarrow \varphi.
  \end{equation}

  We derive the proof from the two logical axioms:
  \begin{equation*}
    \begin{prooftree}[separation=3em]
      \hypo
        {
          \eqref{eq:def:positive_implicational_propositional_derivation_system/axioms/imp_intro}
        }

      \ellipsis
        {
          \( \begin{array}{l}
            \psi \mapsto (\varphi \rightarrow \varphi)
            \\
            \mbox{}
          \end{array} \)
        }
        {
          \eqref{eq:ex:propositional_positive_implicational_logic/dagger}
        }

      \hypo
        {
          \eqref{eq:def:positive_implicational_propositional_derivation_system/axioms/imp_distr}
        }

      \ellipsis
        {
          \( \begin{array}{l}
            \psi \mapsto (\varphi \rightarrow \varphi)
            \\
            \theta \mapsto \varphi
          \end{array} \)
        }
        {
          \eqref{eq:ex:propositional_positive_implicational_logic/dagger}
          \rightarrow ((\varphi \rightarrow (\varphi \rightarrow \varphi)) \rightarrow (\varphi \rightarrow \varphi))
        }

      \infer2[mp]{(\varphi \rightarrow (\varphi \rightarrow \varphi)) \rightarrow (\varphi \rightarrow \varphi)}

      \hypo
        {
          \eqref{eq:def:positive_implicational_propositional_derivation_system/axioms/imp_intro}
        }

      \ellipsis
        {
          \( \psi \mapsto \varphi \)
        }
        {
          \varphi \rightarrow (\varphi \rightarrow \varphi)
        }

      \infer2[mp]{\varphi \rightarrow \varphi}
    \end{prooftree}
  \end{equation*}
  where
  \begin{equation}\label{eq:ex:propositional_positive_implicational_logic/dagger}
    \varphi \rightarrow ((\varphi \rightarrow \varphi) \rightarrow \varphi)
  \end{equation}
\end{example}

\begin{definition}\label{def:derivability_and_satisfiability}
  We introduce two notions connecting \hyperref[def:first_order_derivability/derivability]{derivability} and \hyperref[def:first_order_semantics/satisfiability]{satisfiability}:
  \begin{thmenum}
    \thmitem{def:derivability_and_satisfiability/soundness} If, for any formula \( \varphi \), derivability \( \vdash \varphi \) implies satisfiability \( \vDash \varphi \), we say that the derivation system is \term{sound} with respect to the semantical framework.

    \thmitem{def:derivability_and_satisfiability/completeness} Dually, if satisfiability \( \vDash \varphi \) implies derivability \( \vdash \varphi \) for any formula \( \varphi \), we say that the derivation system is \term{complete} with respect to the semantical framework.
  \end{thmenum}
\end{definition}

\begin{proposition}\label{thm:soundness_of_hilberts_derivation_system}\mcite[thm. 1.1]{Wasilewska2010}
  The \hyperref[def:positive_implicational_propositional_derivation_system]{positive implicational propositional derivation system} is \hyperref[def:derivability_and_satisfiability/soundness]{sound} with respect to \hyperref[def:propositional_semantics]{classical semantics}.
\end{proposition}

\begin{definition}\label{def:minimal_propositional_axiomatic_derivation_system}\mcite[def. 55.10]{OpenLogicFull}
  While the \hyperref[def:positive_implicational_propositional_derivation_system]{positive implicational propositional derivation system} is simple, it is of more practical use to have all propositional connectives available. As it turns out, we cannot utilize \hyperref[ex:posts_completeness_theorem]{complete families of Boolean functions} unless we are dealing with \hyperref[def:propositional_semantics]{classical semantics} --- see for example \fullref{ex:heyting_semantics_lem_counterexample} and \fullref{ex:topological_semantics_lem_counterexample} for how \cref{eq:thm:boolean_equivalences/conditional_cnf} fails to hold.

  Our goal is to define the \term{minimal propositional axiomatic derivation system} whose derivability we will denote by \( \vdash_M \). It is \enquote{axiomatic} in the sense that we do not use new rules to express the rest of the propositional syntax but instead we need axiom schemas for each connective. The only exception is \hyperref[def:propositional_language/constants/top]{\( \bot \)}, the axioms for which tend to change semantics by a lot --- see \fullref{def:propositional_axiomatic_derivation_system}.

  The following axioms are essential in the sense that they cannot be defined in terms of others:
  \begin{thmenum}[series=def:minimal_propositional_axiomatic_derivation_system]
    \thmitem{def:minimal_propositional_axiomatic_derivation_system/top} The simplest axiom states that the constant \hyperref[def:propositional_language/constants/top]{\( \top \)} is itself an axiom:
    \begin{equation}\label{eq:def:minimal_propositional_axiomatic_derivation_system/top_intro}
      \top
    \end{equation}

    \thmitem{def:minimal_propositional_axiomatic_derivation_system/conjunction} Axioms for \hyperref[def:propositional_language/connectives/conjunction]{conjunction}:
    \begin{align}
      (\varphi \wedge \psi) &\rightarrow \varphi \label{eq:def:minimal_propositional_axiomatic_derivation_system/and_elim_left} \\
      (\varphi \wedge \psi) &\rightarrow \psi \label{eq:def:minimal_propositional_axiomatic_derivation_system/and_elim_right} \\
      \varphi               &\rightarrow \parens[\Big]{ \psi \rightarrow (\varphi \wedge \psi) } \label{eq:def:minimal_propositional_axiomatic_derivation_system/and_intro}
    \end{align}

    \thmitem{def:minimal_propositional_axiomatic_derivation_system/disjunction} Axioms for \hyperref[def:propositional_language/connectives/disjunction]{disjunction}:
    \begin{align}
      \varphi                    &\rightarrow (\varphi \vee \psi) \label{eq:def:minimal_propositional_axiomatic_derivation_system/or_elim_left} \\
      \varphi                    &\rightarrow (\psi \vee \varphi) \label{eq:def:minimal_propositional_axiomatic_derivation_system/or_elim_right} \\
      (\varphi \rightarrow \psi) &\rightarrow \parens[\Big]{ (\theta \rightarrow \psi) \rightarrow ((\varphi \vee \theta) \rightarrow \psi) } \label{eq:def:minimal_propositional_axiomatic_derivation_system/or_intro}
    \end{align}
  \end{thmenum}

  The following axioms and are said to be \enquote{abbreviations} and do not affect semantics:
  \begin{thmenum}[resume=def:minimal_propositional_axiomatic_derivation_system]
    \thmitem{def:minimal_propositional_axiomatic_derivation_system/negation} The axioms for negation is motivated via \fullref{thm:boolean_equivalences/negation_bottom}:
    \begin{align}
      \neg \varphi &\rightarrow (\varphi \rightarrow \bot) \label{eq:def:minimal_propositional_axiomatic_derivation_system/neg_elim} \\
      (\varphi \rightarrow \bot) &\rightarrow \neg \varphi \label{eq:def:minimal_propositional_axiomatic_derivation_system/neg_intro}
    \end{align}

    \thmitem{def:minimal_propositional_axiomatic_derivation_system/equivalence} The axioms for the biconditional is motivated via \fullref{thm:boolean_equivalences/biconditional_via_conditionals}:
    \begin{align}
      (\varphi \leftrightarrow \psi) &\rightarrow (\varphi \rightarrow \psi) \label{eq:def:minimal_propositional_axiomatic_derivation_system/equiv_elim_left} \\
      (\varphi \leftrightarrow \psi) &\rightarrow (\psi \rightarrow \varphi) \label{eq:def:minimal_propositional_axiomatic_derivation_system/equiv_elim_right} \\
      (\varphi \rightarrow \psi)     &\rightarrow \parens[\Big]{ (\psi \rightarrow \varphi) \rightarrow (\varphi \leftrightarrow \psi) } \label{def:minimal_propositional_axiomatic_derivation_system/eq:equiv_intro}
    \end{align}
  \end{thmenum}
\end{definition}

\begin{theorem}\label{thm:minimal_propositional_negation_laws}
  Consider the following propositional formula schemas:
  \begin{thmenum}
    \thmitem{thm:minimal_propositional_negation_laws/dne} Double negation elimination (see \fullref{thm:boolean_equivalences/double_negation}):
    \begin{equation}\label{eq:thm:minimal_propositional_negation_laws/dne}
      \neg \neg \varphi \rightarrow \varphi \tag{DNE}.
    \end{equation}

    \thmitem{thm:minimal_propositional_negation_laws/efq} Ex falso quodlibet (Latin for \enquote{from falsity everything follows}), also known as the \term{principle of explosion}):
    \begin{equation}\label{eq:thm:minimal_propositional_negation_laws/efq}
      \bot \rightarrow \varphi \tag{EFQ}
    \end{equation}

    \thmitem{thm:minimal_propositional_negation_laws/pierce} Pierce's law:
    \begin{equation}\label{eq:thm:minimal_propositional_negation_laws/pierce}
      ((\varphi \rightarrow \psi) \rightarrow \varphi) \rightarrow \varphi \tag{Pierce}
    \end{equation}

    \thmitem{thm:minimal_propositional_negation_laws/lem} Law of the exluded middle:
    \begin{equation}\label{eq:thm:minimal_propositional_negation_laws/lem}
      \varphi \vee \neg \varphi \tag{LEM}
    \end{equation}
  \end{thmenum}

  \mcite[prop. 3]{DienerMcKubreJordens2016} Assuming the \hyperref[def:minimal_propositional_axiomatic_derivation_system]{minimal propositional axiomatic derivation system}, we have the following derivations:
  \begin{center}
    \synttree
      [
        {\eqref{eq:thm:minimal_propositional_negation_laws/dne}}
          [
            {\eqref{eq:thm:minimal_propositional_negation_laws/pierce}}
              [{\eqref{eq:thm:minimal_propositional_negation_laws/lem}}]
          ]
          [{\eqref{eq:thm:minimal_propositional_negation_laws/efq}}]
      ]
  \end{center}

  \mcite[prop. 13]{DienerMcKubreJordens2016} Conversely, \eqref{eq:thm:minimal_propositional_negation_laws/efq} and \eqref{eq:thm:minimal_propositional_negation_laws/lem} together can be used to derive \eqref{eq:thm:minimal_propositional_negation_laws/dne}.
\end{theorem}

\smallskip

\begin{definition}\label{def:intuitionistic_propositional_axiomatic_derivation_system}\mcite[def. 55.10]{OpenLogicFull}
  Semantics matching the \hyperref[def:minimal_propositional_axiomatic_derivation_system]{minimal propositional axiomatic derivation system} are not very well studied. If we extend it with the axiom \eqref{eq:thm:minimal_propositional_negation_laws/efq}, we obtain the \term{intuitionistic propositional axiomatic derivation system}.

  The corresponding semantics are defined in \fullref{def:propositional_heyting_semantics} and their link with the derivation system is given in \fullref{thm:intuitionistic_propositional_logic_is_sound_and_complete}.
\end{definition}

\begin{definition}\label{def:propositional_heyting_semantics}\mcite[14]{BezhanishviliHolliday2019}
  We define \term{Heyting semantics} for propositional formulas similarly to how it is done with classical Boolean semantics in \fullref{def:propositional_semantics}, except that instead of using a \hyperref[def:boolean_algebra]{Boolean algebra} we use a more general \hyperref[def:heyting_algebra]{Heyting algebra}.

  Logical negations depend on complements in Boolean algebras. Since Heyting algebras do not have complements, we instead use \hyperref[def:heyting_algebra/pseudocomplement]{pseudocomplements}.

  Fix a Heyting algebra \( (\mscrH, \sup, \inf, T, F, \rightarrow) \). \hyperref[def:propositional_valuation/interpretation]{Propositional interpretations} in Heyting semantics may take any value in \( \mscrH \), as can \hyperref[def:propositional_valuation/formula_semantics]{formula valuations}.

  Given an interpretation \( I \) and a formula \( \varphi \), we define \( \varphi\Bracks{I} \) via \eqref{eq:def:propositional_valuation/formula_interpretation}, the sole difference being that negation valuation is defined via the pseudocomplement:
  \begin{equation*}
    (\neg \psi)\Bracks{I} \coloneqq \widetilde{\varphi\Bracks{I}}.
  \end{equation*}

  We say that \( I \) satisfies \( \varphi \) if \( \varphi\Bracks{I} = T \). Thus if the valuation of \( \varphi \) takes any value in \( \mscrH \setminus \set{ T } \), then \( I \) does not satisfy \( \varphi \), but that does not necessarily mean that \( I \) satisfies \( \neg \varphi \).

  Then \( \Gamma \) entails \( \varphi \) if, for every \( \psi \in \Gamma \) and every interpretation \( I \) in every Heyting algebra, we have \( \varphi\Bracks{I} = \psi\Bracks{I} \).

  It is important that different Heyting algebras may provide different semantics --- see \fullref{ex:heyting_semantics_lem_counterexample} for an example of what is impossible in a Boolean algebra.
\end{definition}

\begin{example}\label{ex:heyting_semantics_lem_counterexample}
  Let \( \mscrH \coloneqq \set{ F, N, T } \) be the two-element Boolean algebra extended with the \enquote{indeterminate} symbol \( N \). Let \( F \leq N \leq T \) so that we have an order structure.

  The pseudocomplement of \( N \) is
  \begin{equation*}
    \widetilde{N}
    \overset {\eqref{eq:def:heyting_algebra/pseudocomplement}} =
    \sup\set{ a \in \mscrX \given a \wedge N = \bot }
    =
    F.
  \end{equation*}

  Consider any \hyperref[def:propositional_semantics/interpretation]{propositional interpretation} \( I \) such that \( I(P) = N \).

  Then the valuation of \eqref{eq:thm:minimal_propositional_negation_laws/lem} is
  \begin{equation*}
    (P \vee \neg P)\Bracks{I}
    =
    \sup\set{ P\Bracks{I}, \widetilde{P\Bracks{I}} }
    =
    \sup\set{ N, \widetilde{N} }
    =
    \sup\set{ N, F }
    =
    N.
  \end{equation*}

  Therefore \eqref{eq:thm:minimal_propositional_negation_laws/lem} does not hold.
\end{example}

\begin{theorem}\label{thm:intuitionistic_propositional_logic_is_sound_and_complete}\mcite[11]{BezhanishviliHolliday2019}
  The \hyperref[def:intuitionistic_propositional_axiomatic_derivation_system]{intuitionistic propositional axiomatic derivation system} is \hyperref[def:derivability_and_satisfiability/soundness]{sound} and \hyperref[def:derivability_and_satisfiability/completeness]{complete} with respect to \hyperref[def:propositional_heyting_semantics]{Heyting semantics}. To elaborate,
  \begin{itemize}
    \item if \( \vdash \varphi \), then \( \vDash \varphi \) for every Heyting algebra.
    \item if \( \vDash \varphi \) in every Heyting algebra, then \( \vdash \varphi \).
  \end{itemize}
\end{theorem}

\begin{definition}\label{def:propositional_topological_semantics}\mcite[15]{BezhanishviliHolliday2019}
  Since arbitrary \hyperref[def:heyting_algebra]{Heyting algebras} can be cumbersome to come up with when used for \hyperref[def:propositional_heyting_semantics]{propositional Heyting semantics}, we can instead utilize \fullref{ex:topological_space_is_heyting_algebra} and define \term{topological semantics} for some nonempty \hyperref[def:topological_space]{topological space}.

  The truth values of interpretations and valuations are then open sets in some topological space and a formula is said to be valid if its valuation is the whole space.
\end{definition}

\begin{example}\label{ex:topological_semantics_lem_counterexample}
  Let \( U \) be an open set in the standard topology in \( \BbbR \). We will examine \eqref{eq:thm:minimal_propositional_negation_laws/lem} with respect to \hyperref[def:propositional_topological_semantics]{topological semantics} for \( \BbbR \). Due to \fullref{ex:topological_space_is_heyting_algebra}, given any \hyperref[def:propositional_semantics/interpretation]{propositional interpretation} \( I \) such that \( I(P) = U \), we have
  \begin{equation*}
    (P \vee \neg P)\Bracks{I}
    =
    P\Bracks{I} \cup \widetilde{P\Bracks{I}}
    =
    U \cup \widetilde{U}
    =
    U \cup \inter(\BbbR \setminus U).
  \end{equation*}

  If \( U = \varnothing \), then \( (P \vee \neg P)\Bracks{I} = \BbbR \) and \eqref{eq:thm:minimal_propositional_negation_laws/lem} holds. If \( U = (0, 1) \), then \( (P \vee \neg P)\Bracks{I} = \BbbR \setminus \set{ 0, 1 } \) and \eqref{eq:thm:minimal_propositional_negation_laws/lem} does not hold.

  Compare this result with \fullref{ex:heyting_semantics_lem_counterexample}.
\end{example}

\begin{definition}\label{def:brouwer_heyting_kolmogorov_interpretation}\mcite[sec. 55.3]{OpenLogicFull}
  Another semantics for the \hyperref[def:intuitionistic_propositional_axiomatic_derivation_system]{intuitionistic propositional axiomatic derivation system} is the \term{Brouwer-Heyting-Kolmogorov interpretation}.

  It uses a less formal approach than \hyperref[def:heyting_algebra_semantics]{Heyting algebra semantics} that is based on the notion of a \enquote{construction}, which is also why it is sometimes called \term{constructive logic}.

  \begin{thmenum}
    \thmitem{def:brouwer_heyting_kolmogorov_interpretation/atomic} We assume that we know what constitutes a construction of propositional variables.
    \thmitem{def:brouwer_heyting_kolmogorov_interpretation/constant} There is no construction of \( \bot \) and no construction of \( \top \) is needed.
    \thmitem{def:brouwer_heyting_kolmogorov_interpretation/disjunction} A construction of \( \psi_1 \vee \psi_2 \) is a pair \( (i, M) \), where \( i = 1, 2 \) and \( M \) is a construction of \( \psi_k \) if and only if \( i = k \). The notion of a pair here is informal.
    \thmitem{def:brouwer_heyting_kolmogorov_interpretation/conjunction} A construction of \( \psi_1 \wedge \psi_2 \) is a pair \( (M_1, M_2) \), where \( M_k \) is a construction of \( \psi_k \) for \( k = 1, 2 \).
    \thmitem{def:brouwer_heyting_kolmogorov_interpretation/conditional} A construction of \( \psi_1 \rightarrow \psi_2 \) is a function that converts a construction of \( \psi_1 \) into a construction of \( \psi_2 \). The notion of a function here is informal.
  \end{thmenum}

  The negation \( \neg\psi \) that corresponds to pseudocomplements in Heyting algebra semantics corresponds to the metastatement \enquote{a construction of \( \psi \) is impossible} under the Heyting-Brouwer-Kolmogorov interpretation.

  If the set \( \Gamma \) of formulas does not derive \( \varphi \), we say that \( \varphi \) is non-constructive under the axioms \( \Gamma \).
\end{definition}

\begin{remark}\label{rem:brouwer_heyting_kolmogorov_interpretation_compatibility}
  Since the \hyperref[def:heyting_brouwer_kolmogorov_interpretation]{Heyting-Brouwer-Kolmogorov interpretation} is not very formal, we cannot properly prove its soundness or completeness with respect to the \hyperref[def:intuitionistic_propositional_axiomatic_derivation_system]{intuitionistic propositional axiomatic derivation system}.

  Nevertheless, we generally accept the interpretation and conflate \enquote{constructive} and \enquote{intuitionistic} statements.
\end{remark}

\begin{example}\label{ex:def:brouwer_heyting_kolmogorov_interpretation/well_ordering_principle_zfc}
  \Fullref{thm:well_ordering_principle} in \hyperref[def:set_zfc]{ZFC} does not provide a way to well-order an arbitrary set. The theorem relies on the axiom of choice, which by \fullref{diaconescu_goodman_myhill_theorem} implies the law of the excluded middle (LEM) assuming the nonlogical axioms of ZFC.

  Since LEM may not hold in intuitionistic logic, it follows that both \fullref{thm:well_ordering_principle} and the axiom of choice itself should not in general hold under the Heyting-Brouwer-Kolmogorov interpretation, hence by the terminology in \fullref{def:brouwer_heyting_kolmogorov}, it is a non-constructive theorem.
\end{example}

\begin{definition}\label{def:propositional_axiomatic_derivation_system}
  In order to obtain a derivation system that matches \hyperref[def:propositional_semantics]{classical propositional semantics}, we may extend the \hyperref[def:minimal_propositional_axiomatic_derivation_system]{minimal propositional axiomatic derivation system} with \eqref{eq:thm:minimal_propositional_negation_laws/dne} or the \hyperref[def:intuitionistic_propositional_axiomatic_derivation_system]{intuitionistic propositional axiomatic derivation system} with any statement that in conjunction with \eqref{eq:thm:minimal_propositional_negation_laws/efq} imply \eqref{eq:thm:minimal_propositional_negation_laws/dne}.

  Such an example is provided in \fullref{thm:minimal_propositional_negation_laws} --- \eqref{eq:thm:minimal_propositional_negation_laws/efq} and \eqref{eq:thm:minimal_propositional_negation_laws/lem} together imply \eqref{eq:thm:minimal_propositional_negation_laws/dne}.

  We call this, very simply, the (classical) \term{propositional axiomatic derivation system}.
\end{definition}

\begin{theorem}[Glivenko's double negation theorem]\label{thm:glivenkos_double_negation_theorem}\mcite{Franks2018}
  A formula \( \varphi \) is derivable in the \hyperref[def:propositional_axiomatic_derivation_system]{classical propositional axiomatic derivation system} if and only if it's double negation \( \neg \neg \varphi \) is derivable in the \hyperref[def:intuitionistic_propositional_axiomatic_derivation_system]{intuitionistic axiomatic derivation system}.
\end{theorem}

\begin{theorem}\label{thm:classical_propositional_logic_is_sound_and_complete}\mcite[thm. 12.30 \\ corr. 13.7]{OpenLogicFull}
  The \hyperref[def:propositional_axiomatic_derivation_system]{classical propositional axiomatic derivation system} is \hyperref[def:derivability_and_satisfiability/soundness]{sound} and \hyperref[def:derivability_and_satisfiability/completeness]{complete} with respect to \hyperref[def:propositional_semantics]{classical semantics}.
\end{theorem}

\begin{definition}\label{def:first_order_axiomatic_derivation_system}
  If we wish to work with first-order logic rather than merely propositional logic, we must extend the \hyperref[def:propositional_axiomatic_derivation_system]{classical propositional axiomatic derivation system}. We call this, very simply, the (classical) \term{first-order axiomatic derivation system}.

  \begin{itemize}
    \item\mcite[def 22.8]{OpenLogicFull} We add two \hyperref[def:first_order_axiomatic_derivation_system/rules]{derivation rules}. Both assume that \( \varphi \) and \( \psi \) are formulas such that \( \xi \not\in \boldop{Free}(\varphi) \).

    \begin{thmenum}[series=def:first_order_axiomatic_derivation_system]
      \thmitem{def:first_order_axiomatic_derivation_system/rules/forall_intro} The \hyperref[def:first_order_language/quantifiers/universal]{universal quantifier} can be introduced:
      \begin{equation}\label{eq:def:first_order_axiomatic_derivation_system/rules/forall_intro}
        \tag{\( \forall \) intro}
        \begin{prooftree}
          \hypo{\varphi \rightarrow \psi}
          \infer1[\( \forall \) intro]{\varphi \rightarrow (\qforall \xi \psi)}
        \end{prooftree}
      \end{equation}

      \thmitem{def:first_order_axiomatic_derivation_system/rules/exists_intro} The \hyperref[def:first_order_language/quantifiers/existential]{existential quantifier} can be introduced:
      \begin{equation}\label{eq:def:first_order_axiomatic_derivation_system/rules/exists_intro}
        \tag{\( \exists \) intro}
        \begin{prooftree}
          \hypo{\varphi \rightarrow \psi}
          \infer1[\( \exists \) intro]{(\qexists \xi \varphi) \rightarrow \psi}
        \end{prooftree}
      \end{equation}
    \end{thmenum}

    \item We also add a few \hyperref[def:first_order_axiomatic_derivation_system/axioms]{logical axiom} schemas.

    \begin{thmenum}[resume=def:first_order_axiomatic_derivation_system]
      \thmitem{def:first_order_axiomatic_derivation_system/axioms/terms}\mcite[def 22.7]{OpenLogicFull} First, for any formula \( \varphi \), any variable \( \xi \) and any term \( \tau \), we add the axioms
      \begin{align}
        \qforall \xi \varphi      &\rightarrow \varphi[\xi \mapsto \tau], \label{eq:def:first_order_axiomatic_derivation_system/axioms/terms/forall_elim} \\
        \varphi[\xi \mapsto \tau] &\rightarrow \qexists \xi \varphi.      \label{eq:def:first_order_axiomatic_derivation_system/axioms/terms/exists_intro}
      \end{align}

      Compare these axioms with \fullref{thm:quantifier_satisfiability}.

      \thmitem{def:first_order_axiomatic_derivation_system/axioms/equality}\mcite[20.39]{OpenLogicFull} For any variable \( \xi \), any \hyperref[def:first_order_syntax/ground_term]{ground terms} \( \tau \) and \( \sigma \) any formula \( \varphi \), we also add the following axioms:
      \begin{equation}\label{eq:def:first_order_axiomatic_derivation_system/axioms/equality}
        \begin{aligned}
                               &\phantom{{}\rightarrow{}} (\tau \doteq \tau) \\
          (\tau \doteq \sigma) &\rightarrow (\varphi[\xi \mapsto \tau] \rightarrow \varphi[\xi \mapsto \sigma])
        \end{aligned}
      \end{equation}
    \end{thmenum}
  \end{itemize}
\end{definition}

\begin{proposition}\label{thm:syntactic_implicit_universal_quantification}
  Fix a formula \( \varphi \) a variable \( \xi \) over any first-order language \( \mscrL \). The formulas \( \varphi \) and \( \qforall \xi \varphi \) are \hyperref[def:first_order_semantics/interderivable]{interderivable} in the \hyperref[def:first_order_axiomatic_derivation_system]{classical first-order axiomatic derivation system}.

  Compare this result with \fullref{thm:semantic_implicit_universal_quantification}.
\end{proposition}
\begin{proof}
  Start with \( \varphi \). Obviously \( \xi \) is not free in \( \top \), hence we can apply \eqref{eq:def:positive_implicational_propositional_derivation_system/axioms/imp_intro} with \( \psi = \top \). Then
  \begin{equation*}
    \begin{prooftree}
      \hypo{\top \rightarrow \varphi}
      \infer1[\( \forall \) intro]{\top \rightarrow \qforall \xi \varphi}
    \end{prooftree}
  \end{equation*}

  Now we can use the axiom \eqref{def:minimal_propositional_axiomatic_derivation_system/top_intro} and \eqref{eq:def:implicational_logic/rules/modus_ponens} to infer \( \qforall \xi \varphi \). Thus
  \begin{equation*}
    \varphi \vdash \qforall \xi \varphi.
  \end{equation*}

  Conversely, starting with \( \qforall \xi \varphi \), we can apply \eqref{eq:def:first_order_axiomatic_derivation_system/axioms/terms/forall_elim} with \( \tau = \xi \) to conclude that
  \begin{equation*}
    \qforall \xi \varphi \vdash \varphi.
  \end{equation*}
\end{proof}

\begin{theorem}\label{thm:classical_first_order_logic_is_sound_and_complete}\mcite[thm. 22.36 \\ corr. 23.19]{OpenLogicFull}
  The \hyperref[def:first_order_axiomatic_derivation_system]{classical first-order axiomatic derivation system} is \hyperref[def:derivability_and_satisfiability/soundness]{sound} and \hyperref[def:derivability_and_satisfiability/completeness]{complete} with respect to \hyperref[def:first_order_semantics]{classical semantics}.
\end{theorem}

\smallskip

\begin{corollary}[First-order compactness theorem]\label{thm:first_order_compactness_theorem}\mcite[thm. 23.21]{OpenLogicFull}
  A set \( \Gamma \) of first-order formulas is \hyperref[def:propositional_semantics/satisfiability]{satisfiable} if and only if every finite subset of \( \Gamma \) is satisfiable.
\end{corollary}

\begin{theorem}[Syntactic deduction theorem]\label{thm:syntactic_deduction_theorem}\mcite[thm. 22.25]{OpenLogicFull}
  In the \hyperref[def:minimal_propositional_axiomatic_derivation_system]{first-order axiomatic derivation system}, \( \Gamma, \psi \vdash \varphi \) holds if and only if \( \Gamma \vdash \psi \rightarrow \varphi \) holds.

  The theorem holds in the \hyperref[def:propositional_axiomatic_derivation_system]{classical propositional axiomatic derivation system} due to \fullref{rem:propositional_logic_as_first_order_logic} but also in the \hyperref[def:intuitionistic_propositional_axiomatic_derivation_system]{intuitionistic}, \hyperref[def:intuitionistic_propositional_axiomatic_derivation_system]{minimal} and \hyperref[def:positive_implicational_propositional_derivation_system]{positive implicational} derivation systems (see \mcite[thm. 2.2]{Wasilewska2010}).

  Compare this result to \fullref{thm:semantic_deduction_theorem}.
\end{theorem}

\begin{proposition}\label{thm:formulas_are_derivable_iff_conjunction_is_derivable}
  In derivation systems where \fullref{thm:syntactic_deduction_theorem} holds, we have \( \psi_1, \psi_1 \vdash \varphi \) if and only if \( (\psi_1 \wedge \psi_2) \vdash \varphi \).
\end{proposition}
\begin{proof}
  \SufficiencySubProof Let \( \psi_1, \psi_2 \vdash \varphi \).

  By applying the \hyperref[thm:syntactic_deduction_theorem]{deduction theorem} to \eqref{eq:def:minimal_propositional_axiomatic_derivation_system/and_intro}, we obtain
  \begin{equation*}
    \psi_1 \vdash \parens[\Big]{ \psi_2 \rightarrow (\psi_1 \wedge \psi_2) }
  \end{equation*}
  and by applying it again,
  \begin{equation}\label{eq:thm:formulas_are_derivable_iff_conjunction_is_derivable/conjunction_derivation}
    \psi_1, \psi_2 \vdash (\psi_1 \wedge \psi_2).
  \end{equation}

  \Fullref{thm:derivation_system_transitivity} implies that \( \psi_1, \psi_2, (\psi_1 \wedge \psi_2) \vdash \varphi \). Applying the same theorem again but this time removing \( \psi_1 \) and \( \psi_2 \) from the from the left due to \eqref{eq:thm:formulas_are_derivable_iff_conjunction_is_derivable/conjunction_derivation}, we conclude that
  \begin{equation*}
    (\psi_1 \wedge \psi_2) \vdash \varphi.
  \end{equation*}

  \NecessitySubProof Let \( (\psi_1 \wedge \psi_2) \vdash \varphi \). By applying the deduction theorem to \eqref{eq:def:minimal_propositional_axiomatic_derivation_system/and_elim_left} and \eqref{eq:def:minimal_propositional_axiomatic_derivation_system/and_elim_right}, we obtain that
  \begin{equation*}
    (\psi_1 \wedge \psi_2) \vdash \psi_1
  \end{equation*}
  and
  \begin{equation*}
    (\psi_1 \wedge \psi_2) \vdash \psi_2.
  \end{equation*}

  As in the other direction, by applying \fullref{thm:derivation_system_transitivity}, we obtain
  \begin{equation*}
    (\psi_1 \wedge \psi_2), \psi_1, \psi_2 \vdash \varphi,
  \end{equation*}
  which can be reduced to
  \begin{equation*}
    \psi_1, \psi_2 \vdash \varphi
  \end{equation*}
  by applying the same theorem again.
\end{proof}
