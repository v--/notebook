\subsection{First-order proofs}\label{subsec:first_order_proofs}

\begin{definition}\label{def:first_order_derivation_system}\mcite[sec. 12.1]{OpenLogic20201202}
  A \term{first-order derivation system} for the \hyperref[def:first_order_syntax]{language} \( \mscrL \) consists of:

  \begin{thmenum}
    \thmitem{def:first_order_derivation_system/rules} A set of \hyperref[def:relation]{relations} of potentially different arity on the set of formulas of \( \mscrL \), which we call \term{inference rules}. If \( r \) is an inference rule of arity \( n + 1 \), we say that the rule allows us to \term{infer} \( \varphi \) from \( \psi_1, \ldots, \psi_n \) if \( (\psi_1, \ldots, \psi_n, \varphi) \in r \) for any \( n+1 \)-tuple of formulas \( (\psi_1, \ldots, \psi_n, \varphi) \). We write this symbolically as
    \begin{equation*}
      \begin{prooftree}
        \hypo{\psi_1}
        \hypo{\cdots}
        \hypo{\psi_n}
        \infer3[r]{\varphi}
      \end{prooftree}
    \end{equation*}

    \thmitem{def:first_order_derivation_system/axioms} A set of \hyperref[def:first_order_syntax]{formulas} in \( \mscrL \), which we call \term{logical axioms} (to distinguish them with \hyperref[def:first_order_proofs/axioms_and_theorems]{nonlogical axioms}).
  \end{thmenum}
\end{definition}

\begin{definition}\label{def:first_order_proofs}
  We introduce \enquote{syntactic} counterparts to the concepts from \fullref{def:first_order_semantics}.

  Given a \hyperref[def:first_order_derivation_system]{derivation system} for the language \( \mscrL \), we introduce the following notions:
  \begin{thmenum}
    \thmitem{def:first_order_proofs/derivation}\mcite[def. 12.1]{OpenLogic20201202}A sequence \( \varphi_1, \ldots, \varphi_n \) of formulas is called a \term{derivation} from the set \( \Gamma \) of formulas if, for any \( k = 1, \ldots, n \), either of the following hold:
    \begin{thmenum}
      \thmitem{def:first_order_proofs/derivation/logical_axiom} \( \varphi_k \) is a \hyperref[def:first_order_derivation_system/axioms]{logical axiom} in the derivation system.
      \thmitem{def:first_order_proofs/derivation/nonlogical_axiom} \( \varphi_k \in \Gamma \)
      \thmitem{def:first_order_proofs/derivation/inference} There exist indices \( i_1, \ldots, i_n < k \) such that \( \varphi_k \) can be inferred from \( \varphi_{i_1}, \ldots, \varphi_{i_n} \) using any of the \hyperref[def:first_order_derivation_system/rules]{inference rules}.
    \end{thmenum}

    Derivations are also called \term{proofs}. We use the notation shortcuts from \fullref{def:propositional_semantics/satisfiability}.

    \thmitem{def:first_order_proofs/derivability} If there exists a derivation from \( \Gamma \) such that \( \varphi \) is the last element, we say that \( \varphi \) is \term{derivable} from \( \Gamma \).

    \thmitem{def:first_order_proofs/derivation_closure} The \term{derivation closure} \( \cl^{\vdash}(\Gamma) \) of the set \( \Gamma \) is the set of all formulas that can be derived from \( \Gamma \).

    \thmitem{def:first_order_proofs/completeness}\mcite[def. 13.1]{OpenLogic20201202}If, for every formula \( \varphi \), either \( \varphi \in \Gamma \) or \( \neg \varphi \in \Gamma \), we say that \( \Gamma \) is a \term{complete set of formulas}. Note that we do not eliminate double negation.

    \thmitem{def:first_order_proofs/consistency}\mcite[def. 20.16]{OpenLogic20201202}If \( \bot \) cannot be derived from \( \Gamma \), we say that \( \Gamma \) is \term{consistent}.

    \thmitem{def:first_order_proofs/axioms_and_theorems} If we have fixed a set \( A \) of formulas, we say that formulas in \( A \) are \term{(nonlogical) axioms} and formulas in the closure \( \cl^{\vdash}(A) \) are \term{theorems}. If \( A = \varnothing \), we obtain \term{logical theorems} that are derived solely from the \hyperref[def:first_order_derivation_system/axioms]{logical axioms} using the inference rules.
  \end{thmenum}
\end{definition}

\begin{definition}\label{def:derivability_and_satisfiability}
  We introduce two notions connecting \hyperref[def:first_order_proofs/derivability]{derivability} and \hyperref[def:first_order_semantics/satisfiability]{satisfiability}:
  \begin{thmenum}
    \thmitem{def:derivability_and_satisfiability/soundness} If, for any formula \( \varphi \), \( \vdash \varphi \) implies \( \vDash \varphi \), we say that the derivation system is \term{sound}.

    \thmitem{def:derivability_and_satisfiability/completeness} Dually, if \( \vDash \varphi \) implies \( \vdash \varphi \) for formulas \( \varphi \), we say that the derivation system is \term{complete}.
  \end{thmenum}
\end{definition}

\begin{definition}\label{def:propositional_implicational_logic}
  A very simple \hyperref[def:first_order_derivation_system]{derivation system} is that of \term{implicational propositional logic}. It is based on the propositional alphabet consisting only of \( \to \) and \( \bot \) (which is sufficient to express all propositional formulas by \fullref{ex:posts_completeness_theorem/implies_bot}).

  \begin{refenum}
    \refitem{def:first_order_derivation_system/rules} The system has a single rule, \term{modus ponens}:
    \begin{equation}\label{eq:def:implicational_logic/rules/modus_ponens}
      \begin{prooftree}
        \hypo{\varphi}
        \hypo{\varphi \to \psi}
        \infer2[mp]{\psi}
      \end{prooftree}
    \end{equation}

    \refitem{def:first_order_derivation_system/axioms} The system has three axioms:
    \begin{align}
      \varphi                         &\to (\psi \to \varphi)                         \label{def:implicational_logic/axioms/imp_int} \\
      (\varphi \to (\psi \to \theta)) &\to ((\varphi \to \psi) \to (\psi \to \theta)) \label{def:implicational_logic/axioms/imp_dist} \\
      (\varphi \to \psi) \to \varphi) &\to \varphi                                    \label{def:implicational_logic/axioms/pierce}
    \end{align}

    \begin{itemize}
      \item Axiom \eqref{def:implicational_logic/axioms/imp_int} is called \term{implication introduction} because it \enquote{introduces} an implication on the right.
      \item Axiom \eqref{def:implicational_logic/axioms/imp_dist} is called \term{implication distributivity} because it resembles \hyperref[def:distributive_lattice]{distributivity} of \( \to \) over itself.
      \item Axiom \eqref{def:implicational_logic/axioms/pierce} is called \term{Pierce's law}. \term{Minimal implicational logic} is the derivation system obtained by removing this rule. If, instead, we replace Pierce's law with \term{the principle of explosion}
      \begin{equation}\label{def:implicational_logic/axioms/explosion}
        \bot \to \varphi,
      \end{equation}
      the resulting derivation system is called \term{intuitionistic implicational logic}. The principle of explosion is also called \term{ex falso quodlibet} (\enquote{from falsity, everything follows}).

      To distinguish the system with \eqref{def:implicational_logic/axioms/pierce} from minimal and intuitionistic logic, we call it \term{classical implicational logic}.
    \end{itemize}
  \end{refenum}

  The rules and axioms (except for \eqref{def:implicational_logic/axioms/explosion}) consist entirely of implications, hence the name of the derivation system itself.
\end{definition}

\begin{definition}\label{def:first_order_implicational_logic}
  If we wish to work with first-order logic rather than merely propositional logic, we must extend \eqref{def:propositional_implicational_logic} as follows:

  \begin{refenum}
    \refitem{def:first_order_derivation_system/rules}\mcite[def. 20.8]{OpenLogic20201202} We add two rules. Both assume that \( \varphi \) and \( \psi \) are formulas such that \( \xi \not\in \boldop{Free}(\varphi) \):
    \begin{align}
      &\begin{prooftree}
        \hypo{\varphi \to \psi}
        \infer1[\( \forall \) \text{int}]{\varphi \to (\qforall{\xi} \psi)}
      \end{prooftree} \label{eq:def:first_order_implicational_logic/rules/forall_int}
      \\
      &\begin{prooftree}
        \hypo{\varphi \to \psi}
        \infer1[\( \exists \) \text{int}]{(\qforall{\xi} \varphi) \to \psi}
      \end{prooftree} \label{eq:def:first_order_implicational_logic/rules/exists_int}
    \end{align}

    \refitem{def:first_order_derivation_system/axioms} We add a few axioms. First, for any closed term \( \tau \),
    \begin{align}
      \qforall{\xi} \varphi     &\to \varphi[\xi \mapsto \tau], \label{eq:def:first_order_implicational_logic/axioms/forall_elim} \\
      \varphi[\xi \mapsto \tau] &\to \qexists{\xi} \varphi,     \label{eq:def:first_order_implicational_logic/axioms/exists_int}
    \end{align}

    If the language contains formal equality, we also add the following:
    \begin{alignedeq}\label{eq:def:first_order_derivation_system/axioms/forall_elim}
                          &\phantom{{}\to{}} (\xi \doteq \xi) \\
        (\xi \doteq \eta) &\to (\tau[\zeta \mapsto \xi] \doteq \tau[\zeta \mapsto \eta]) \\
        (\xi \doteq \eta) &\to (\varphi[\zeta \mapsto \xi] \to \varphi[\zeta \mapsto \eta])
    \end{alignedeq}
  \end{refenum}
\end{definition}

\begin{theorem}[Completeness of implicational logic]\label{thm:completeness_of_hilberts_derivation_system}\mcite[thm. 21.8]{OpenLogic20201202}
  \hyperref[def:first_order_implicational_logic]{First-order implication logic} is \hyperref[def:derivability_and_satisfiability/completeness]{complete}.
\end{theorem}

\smallskip

\begin{theorem}[Soundness of minimal implicational logic]\label{thm:soundness_of_hilberts_derivation_system}\mcite[thm. 1.1]{Wasilewska2010}
  \hyperref[def:first_order_implicational_logic]{Minimal (first-order) implication logic} is \hyperref[def:derivability_and_satisfiability/soundness]{sound}.
\end{theorem}

\begin{theorem}[Deduction theorem]\label{thm:deduction_theorem}\mcite[thm. 2.2]{Wasilewska2010}
  In \hyperref[def:first_order_implicational_logic]{minimal (first-order) implication logic}, \( \Gamma, \psi \vdash \varphi \) holds if and only if \( \Gamma \vdash \psi \to \varphi \) holds. Compare this result to \fullref{thm:semantic_deduction_theorem}.
\end{theorem}

\begin{theorem}[Glivenko's double negation theorem]\label{thm:glivenkos_double_negation_theorem}\mcite{Franks2018}
  A formula \( \varphi \) is derivable in \hyperref[def:first_order_implicational_logic]{classical (first-order) implication logic} if and only if \( \neg \neg \varphi \) is derivable in minimal logic, where
  \begin{equation*}
    \neg \neg \varphi \T{is a shorthand for } (\varphi \to \bot) \to \bot.
  \end{equation*}
\end{theorem}

\begin{definition}\label{def:first_order_theory}
  \todo{Add a definition for first order logic theories}
\end{definition}
