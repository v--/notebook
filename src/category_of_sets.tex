\subsection{The category of sets}\label{subsec:category_of_sets}

\begin{definition}\label{def:category_of_sets}
  We define the \hyperref[def:category]{category} \( \cat{Set} \) as follows:
  \begin{refenum}
    \refitem{def:category/objects} The \hyperref[def:set]{class} of objects is the class of all sets.

    \refitem{def:category/morphisms} The morphisms between two sets \( A, B \) are the \hyperref[def:function]{functions} \( f: A \to B \).

    \refitem{def:category/composition} Composition of morphisms is the usual \hyperref[def:multi_valued_function/composition]{function composition}.
  \end{refenum}
\end{definition}
\begin{proof}
  To see that \( \cat{Set} \) is indeed a category, we verify the axioms
  \SubProofOf{def:category/identity} For any set \( A \in \cat{Set} \), we have the identity function
  \begin{balign*}
     & \id_A: A \to A        \\
     & \id_A(x) \coloneqq A.
  \end{balign*}

  If \( f: A \to B \) is any function, for all \( x \in A \) we have
  \begin{balign*}
    [\id_B \circ f](x) = \id_B(f(x)) = f(x),
     &  &
    [f \circ \id_A](x) = f(\id_A(x)) = f(x),
  \end{balign*}
  thus \( \id_A \) and \( \id_B \) are indeed identity morphisms.

  \SubProofOf{def:category/associativity} Let \( f: A \to B \), \( g: B \to C \) and \( h: C \to D \) be arbitrary functions. For any \( x \in A \), we have
  \begin{balign*}
    [[h \circ g] \circ f](x)
    =
    [h \circ g](f(x))
    =
    h(g(f(x)))
    =
    h([g \circ f](x))
    =
    [h \circ [g \circ f]](x).
  \end{balign*}
\end{proof}

\begin{corollary}\label{thm:functions_over_set_form_monoid}
  Let \( X \) be a set. Then the family \( \cat{Set}(X) \) of all \hyperref[rem:category_obj_hom]{functions} is a \hyperref[def:unital_magma/associative]{monoid} with function composition as the operation.
\end{corollary}

\begin{proposition}\label{thm:set_is_locally_small}
  The category \( \cat{Set} \) is \hyperref[def:category_size]{locally small}.
\end{proposition}
\begin{proof}
  Since a function \( f: X \to Y \) is \hyperref[def:function]{formally} a subset of the \hyperref[def:tuple_and_cartesian_product]{Cartesian product} \( X^2 \), which is a set, it is itself a set.

  The class of all functions \( \cat{Set}(X, Y) \) from \( X \) to \( Y \) is then a subset of \( \pow(X \times Y) \), which is also a set. Thus, \( \cat{Set} \) is a locally small category.
\end{proof}

\begin{proposition}\label{thm:set_categorical_limits}
  We are interested in \hyperref[def:categorical_limit]{categorical limits} and \hyperref[def:categorical_colimit]{colimits} in \( \cat{Set} \). Fix an indexed \hyperref[def:tuple_and_cartesian_product/indexed_family]{family} \( \{ X_k \}_{k \in \mscrK} \) of sets.
  \begin{thmenum}
    \thmitem{thm:set_categorical_limits/product} Their \hyperref[def:categorical_product]{categorical product} is their \hyperref[def:tuple_and_cartesian_product]{Cartesian product} \( \prod_{k \in \mscrK} X_k \), the projection morphisms being
    \begin{balign*}
       & \pi_m: \prod_{k \in \mscrK} X_k \to X_m        \\
       & \pi_m(\{ x_k \}_{k \in \mscrK}) \coloneqq x_m.
    \end{balign*}

    \thmitem{thm:set_categorical_limits/coproduct} Their \hyperref[def:categorical_coproduct]{categorical coproduct} is their disjoint \hyperref[def:disjoint_union]{union} \( \coprod_{k \in \mscrK} X_k \), the embedding morphisms being
    \begin{balign*}
       & \iota_m: X_m \to \coprod_{k \in \mscrK} X_k \\
       & \iota_m(x_k) \coloneqq (m, x_k).
    \end{balign*}

    \thmitem{thm:set_categorical_limits/equalizer} An equalizer of two functions \( f, g: X \to Y \) in \( \cat{Set} \) is the set
    \begin{equation*}
      \{ x \in X \colon f(x) = g(x) \}.
    \end{equation*}

    Compare this with pullbacks in \( \cat{Set} \) (see \fullref{thm:set_categorical_limits/pullback}).

    \thmitem{thm:set_categorical_limits/pullback} The pullback of two functions \( f: X \to Z \) and \( g: Y \to Z \) in \( \cat{Set} \) is the set
    \begin{equation*}
      \{ (x, y) \in X \times Y \colon f(x) = g(y) \}.
    \end{equation*}

    Compare this with equalizers in \( \cat{Set} \) (see \fullref{thm:set_categorical_limits/equalizer}).

    \thmitem{thm:set_categorical_limits/coequalizer} A coequalizer of two functions \( f, g: X \to Y \) in \( \cat{Set} \) is the quotient set formed by the reflexive, symmetric and transitive closure of the relation \( x \sim y \iff f(x) = g(x) \).

    In particular, if \( \sim \subseteq X^2 \) is an \hyperref[def:equivalence_relation]{equivalence relation}) on \( X \), then the coequalizer of the two projection maps of the product \( X^2 \) is the pair \( (X / ~, \pi) \), where \( \pi \) is the quotient map
    \begin{balign*}
       & \pi: X \to X / \sim \\
       & \pi(x) = [x].
    \end{balign*}

    \thmitem{thm:set_categorical_limits/pushout} Let \( X \) and \( Y \) be two sets, let \( Z \) be a subset of \( X \) and let \( i: Z \to X \) be the inclusion map. For any function \( f: Z \to Y \), we define a pushout of \( i \) and \( f \) in \( \cat{Set} \) to be the set obtained as the quotient of the coproduct \( X \coprod Y \) and the relation \( x \sim y \iff i^{-1}(x) = f^{-1}(y) \).
  \end{thmenum}
\end{proposition}

\begin{proposition}\label{thm:set_is_monoidal}
  The category \( \cat{Set} \) is monoidal with
  \begin{itemize}
    \item the \hyperref[def:tuple_and_cartesian_product]{Cartesian product} acting as a monoidal product
    \item the singleton set \( \{ \varnothing \} \) acting as an identity object
    \item natural transformations
          \begin{balign*}
            \sigma                              & \coloneqq \id \\
            \lambda(\{ \varnothing \} \times A) & \coloneqq A   \\
            \rho(A \times \{ \varnothing \})    & \coloneqq A
          \end{balign*}
  \end{itemize}
\end{proposition}
\begin{proof}
  All conditions in \fullref{def:monoidal_category} are trivially satisfied.
\end{proof}

\begin{proposition}\label{thm:monoids_are_monoids_in_set}
  A monoid in the sense of \fullref{def:unital_magma/associative} is a monoid in \( \cat{Set} \) in the sense of \fullref{def:categorical_monoid}.
\end{proposition}
\begin{proof}
  By \fullref{thm:set_is_monoidal}, \( \cat{Set} \) is monoidal with the Cartesian product as a monoidal product. Let \( \mscrM \) be a monoid in the sense of \fullref{def:unital_magma/associative}. We define the morphism \( \mu: \mscrM \times \mscrM \to \mscrM \) to be the monoid operation and the morphism \( \eta: \{ \varnothing \} \to \mscrM \) to be the identity operation. Then the diagrams in \fullref{def:categorical_monoid} commute.

  The categorical definition of morphism between monoids in \( \cat{Set} \) is then a restatement of the definition of homomorphism of a unital magma. If \( (\mscrM, \mu, \eta) \) and \( (\mscrM', \mu', \eta') \) are monoids in \( \cat{Set} \) and \( f \) is a morphism between them, then
  \begin{equation*}
    (f \circ \mu)(x, y)
    =
    f(xy)
    =
    f(x) f(y)
    =
    (\mu' \circ (f \otimes f))(x, y)
  \end{equation*}
  and
  \begin{equation*}
    (f \circ \eta)(\{ \varnothing \})
    =
    f(e_{\mscrM})
    =
    e_{\mscrM'}
    =
    \eta'(\{ \varnothing \}).
  \end{equation*}
\end{proof}
