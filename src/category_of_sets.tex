\subsection{The category of sets}\label{subsec:category_of_sets}

\begin{definition}\label{def:category_of_sets}
  We denote by \( \Cat{Set} \) the category\Tinyref{def:category} where
  \begin{itemize}
    \item the class\Tinyref{def:set_zfc} of objects is the class of all sets\Tinyref{def:set_zfc}.
    \item the morphisms between two sets \( A, B \) are the functions \( f: A \to B \), with morphism composition being the usual function composition\Tinyref{def:function_composition}.
  \end{itemize}

  Furthermore, the category is locally small\Tinyref{def:category_cardinality}.
\end{definition}
\begin{proof}
  To see that \( \Cat{Set} \) is indeed a category, we verify the axioms
  \begin{description}
    \RItem{def:category/identity} For any set \( X \in \Cat{Set} \), we have the identity function
    \begin{align*}
      &\Id_X: X \to X \\
      &\Id_X(x) \coloneqq x.
    \end{align*}

    If \( f: X \to Y \) is any function, for all \( x \in X \) we have
    \begin{align*}
      [\Id_Y \circ f](x) = \Id_Y(f(x)) = f(x),
      &&
      [f \circ \Id_X](x) = f(\Id_X(x)) = f(x),
    \end{align*}
    thus \( \Id_x \) and \( \Id_Y \) are indeed identity morphisms.

    \RItem{def:category/associativity} Let \( f: A \to B \), \( g: B \to C \) and \( h: C \to D \) be arbitrary functions. For any \( x \in A \), we have
    \begin{align*}
      [[h \circ g] \circ f](x)
      =
      [h \circ g](f(x))
      =
      h(g(f(x)))
      =
      h([g \circ f](x))
      =
      [h \circ [g \circ f]](x).
    \end{align*}
  \end{description}

  Since a function \( f: X \to Y \) is formally\Tinyref{def:function} a subset of the product, \( X \times X \)\Tinyref{def:cartesian_product}, which is a set, it is itself a set.

  The class of all functions \( \Cat{Set}(X, Y) \) from \( X \) to \( Y \) is then a subset of \( \Power(X \times Y) \), which is also a set. Thus \( \Bold{Set} \) is a locally small category.
\end{proof}

\begin{corollary}\label{thm:functions_over_set_form_monoid}
  Let \( X \) be a set. Then the family \( \Cat{Set}(X) \) of all functions\Tinyref{remark:category_obj_hom} is a monoid\Tinyref{def:magma/monoid} with function composition as the operation.
\end{corollary}

\begin{proposition}\label{thm:set_categorical_limits}
  We are interested in categorical limits\Tinyref{def:categorical_limit} and colimits\Tinyref{def:categorical_colimit} in \( \Cat{Set} \). If \( \{ X_i \}_{i \in I} \) is an indexed family\Tinyref{def:indexed_family} of sets, then
  \begin{defenum}
    \DItem{thm:set_categorical_limits/product} their categorical product\Tinyref{def:categorical_product} is their Cartesian product\Tinyref{def:cartesian_product} \( \prod_{i \in I} X_i \), the projection morphisms being
    \begin{align*}
      &\pi_j: \prod_{i \in I} X_i \to X_j \\
      &\pi_j(\{ x_i \}_{i \in I}) \coloneqq x_j.
    \end{align*}

    \DItem{thm:set_categorical_limits/coproduct} their categorical coproduct\Tinyref{def:categorical_coproduct} is their disjoint union\Tinyref{def:disjoint_union} \( \coprod_{i \in I} X_i \), the injection morphisms being
    \begin{align*}
      &\iota_j: X_j \to \coprod_{i \in I} X_i \\
      &\iota_j(x_j) \coloneqq (j, x_j).
    \end{align*}

    \DItem{thm:set_categorical_limits/equalizer} An equalizer of two functions \( f, g: X \to Y \) in \( \Cat{Set} \) is the set
    \begin{equation*}
      \{ x \in X \colon f(x) = g(x) \}.
    \end{equation*}

    Compare this with pullbacks in \( \Cat{Set} \)\Tinyref{thm:set_categorical_limits/pullback}.

    \DItem{thm:set_categorical_limits/pullback} The pullback of two functions \( f: X \to Z \) and \( g: Y \to Z \) in \( \Cat{Set} \) is the set
    \begin{equation*}
      \{ (x, y) \in X \times Y \colon f(x) = g(y) \}.
    \end{equation*}

    Compare this with equalizers in \( \Cat{Set} \)\Tinyref{thm:set_categorical_limits/equalizer}.

    \DItem{thm:set_categorical_limits/coequalizer} A coequalizer of two functions \( f, g: X \to Y \) in \( \Cat{Set} \) is the quotient space formed by the reflexive, symmetric and transitive closure of the relation \( x \sim y \iff f(x) = g(x) \).

    In particular, if \( \sim \subseteq X^2 \) is an equivalence relation\Tinyref{def:order/equivalence}) on \( X \), then the coequalizer of the two projection maps of the product \( X^2 \) is the pair \( (X / ~, \pi) \), where \( \pi \) is the quotient map
    \begin{align*}
      &\pi: X \to X / ~ \\
      &\pi(x) = [x].
    \end{align*}

    \DItem{thm:set_categorical_limits/pushout} Let \( X \) and \( Y \) be two sets, let \( Z \) be a subset of \( X \) and let \( i: Z \to X \) be the inclusion map. For any function \( f: Z \to Y \), we define a pushout of \( i \) and \( f \) in \( \Cat{Set} \) to be the set obtained as the quotient of the coproduct \( X \coprod Y \) and the relation \( x \sim y \iff i^{-1}(x) = f^{-1}(y) \).
  \end{defenum}
\end{proposition}

\begin{proposition}\label{thm:set_is_monoidal}
  The category \( \Cat{Set} \) is monoidal with
  \begin{itemize}
    \item the Cartesian product\Tinyref{def:cartesian_product} acting as a monoidal product
    \item the singleton set \( \{ \varnothing \} \) acting as an identity object
    \item natural transformations
    \begin{align*}
      \alpha &\coloneqq \Id \\
      \lambda(\{ \varnothing \} \times A) &\coloneqq A \\
      \rho(A \times \{ \varnothing \}) &\coloneqq A
    \end{align*}
  \end{itemize}
\end{proposition}
\begin{proof}
  All conditions in \fullref{def:monoidal_category} are trivially satisfied.
\end{proof}
