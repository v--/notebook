\subsection{Convex sets}\label{subsec:convex_sets}

We will denote by \( \BbbK \) either the field of \hyperref[def:real_numbers]{real numbers} or the \hyperref[def:real_numbers]{complex numbers}. Unless otherwise noted, we are working in an \hyperref[def:affine_space]{affine space} \( (A, \vect A, \tau) \) over \( \BbbK \).

\begin{remark}\label{rem:real_field_extensions}
  When speaking about \hyperref[def:vector_space]{vector spaces} or \hyperref[def:affine_space]{affine spaces}, we usually restrict ourselves to vector spaces over \( \BbbR \) or, at most, \( \BbbC \). This restriction is not arbitrary.

  Important concepts like \hyperref[def:geometric_cone]{cones} or \hyperref[def:convex_hull]{convexity} require the field to be an extension of \( \BbbR \), and it just so happens that, by \fullref{thm:fundamental_theorem_of_algebra} and \fullref{thm:no_finite_extensions_of_closed_fields}, the only nontrivial finite \hyperref[def:field/submodel]{field extension} of \( \BbbR \) is \( \BbbC \).

  It is technically possible to work with infinite field extensions, however then we lose the concept of \hyperref[def:inner_product_space]{inner products}, which is another fundamental reasons for working with real or complex vector spaces.

  Considering only \( \BbbR \) and \( \BbbC \) leads to certain concepts being defined for complex vector spaces and then real vector spaces become a special case. This is formalized via \hyperref[def:complexification]{complexification}. For example, inner products are defined in \fullref{def:inner_product_space} differently for real and complex vector spaces, however we can transition between them due to \fullref{thm:complexification_universal_property} and \fullref{thm:complexification_of_symmetric_bilinear_form}.
\end{remark}

\begin{definition}\label{def:convex_hull}\mimprovised
  We say that an \hyperref[rem:affine_combinations]{affine combination} of points or vectors is \term{convex} if all coefficients are nonnegative.

  The \term{convex hull} of the set \( S \) is the set of all convex combinations of members of \( S \). This can be expressed succinctly by saying that, for any number \( \lambda \) in the unit interval and any vectors \( x \) and \( y \), the hull must contain
  \begin{equation}\label{eq:def:convex_hull/combination}
    \lambda x + (1 - \lambda) y.
  \end{equation}

  The convex hull is a \hyperref[def:closure_operator]{closure operator}. A set that coincides with its convex hull is called a \term{convex set}. The geometric interpretation of convex sets is given in \fullref{thm:def:convex_hull/line_segments} and drawn in \cref{fig:thm:affine_and_conic_is_convex}.
\end{definition}
\begin{proof}
  The proof that the convex hull is a closure operator is similar to that for affine hulls in \fullref{def:affine_hull}.
\end{proof}

\begin{definition}\label{def:conic_hull}\mimprovised
  A \term{conic combination} with origin \( O \), points \( x_1, \ldots, x_n \) and \hi{nonnegative} scalars \( t_1, \ldots, t_n \) is the \hyperref[rem:affine_combination]{affine combination}
  \begin{equation}\label{eq:def:conic_hull/points}
    \parens*{ 1 - \sum_{k=1}^n t_k } O + \sum_{k=1}^n t_k x_k.
  \end{equation}

  A conic combination of vectors is, instead, simply a \hyperref[rem:linear_combinations]{linear combination} with nonnegative coefficients. This relates to \eqref{eq:def:conic_hull/points} as follows:
  \begin{equation*}
    \parens*{ 1 - \sum_{k=1}^n t_k } \vect{OO} + \sum_{k=1}^n t_k \vect{O x_k}.
  \end{equation*}

  The \term{conic hull} of the set \( S \) is the set of all conic combinations of members of \( S \). This is a \hyperref[def:closure_operator]{closure operator}.

  The geometric interpretation of convex sets is given in \fullref{thm:conic_hull_is_smallest_cone} and drawn in \cref{fig:thm:affine_and_conic_is_convex}.
\end{definition}
\begin{proof}
  The proof that the conic hull is a closure operator is similar to that for affine hulls in \fullref{def:affine_hull}.
\end{proof}

\begin{proposition}\label{thm:affine_and_conic_is_convex}
  The \hyperref[def:convex_hull]{convex hull} of a set is the intersection of its \hyperref[def:affine_hull]{affine hull} and its \hyperref[def:conic_hull]{conic hull} with an arbitrary origin not in the affine hull.

  \begin{figure}[!ht]
    \centering
    \includegraphics[page=1]{output/thm__affine_and_conic_is_convex.pdf}
    \caption{The \hyperref[def:affine_hull]{affine}, \hyperref[def:conic_hull]{conic} and \hyperref[def:convex_hull]{convex} hulls of \( \set{ x, y, z } \).}\label{fig:thm:affine_and_conic_is_convex}
  \end{figure}
\end{proposition}
\begin{proof}
  It is clear that the convex hull is a subset of the affine hull and, since the coefficients sum to one, also to the conic hull for any origin point.

  Conversely, pick a point \( x \) from the intersection and consider the conic combination
  \begin{equation*}
    x = \parens*{ 1 - \sum_{k=1}^n t_k s_k } O + \sum_{k=1}^n t_k s_k,
  \end{equation*}
  where \( s_1, \ldots, s_n \) belongs to \( S \).

  Define
  \begin{equation*}
    T \coloneqq \sum_{k=1}^n t_k.
  \end{equation*}

  If \( T \neq 1 \), then
  \begin{equation*}
    x = (1 - T) O + T y
  \end{equation*}
  and
  \begin{equation*}
    O = \frac 1 {1 - T} x - \frac T {1 - T} y.
  \end{equation*}

  Both \( x \) and \( y \) are affine combinations of \( S \), hence \( O \) belongs to the affine hull of \( S \). But this contradicts our choice of \( O \).

  Therefore, \( T \) is necessarily \( 1 \) and \( x \) is a convex combination of members of \( S \).
\end{proof}

\begin{definition}\label{def:geometric_ray}\mimprovised
  A \term{ray} with \term{vertex} point \( O \) and nonzero \term{directional} vector \( d \) is the \hyperref[def:conic_hull]{conic hull} with origin \( O \) of the singleton set \( \set{ \tau_d(O) } \). It is often described, in complete analogy with \fullref{def:affine_line/parametric}, as the \hyperref[def:parametric_curve]{parametric curve}
  \begin{equation*}
    \begin{aligned}
       &r: [0, \infty) \to A \\
       &r(t) \coloneqq \tau_{td}(O).
    \end{aligned}
  \end{equation*}

  \begin{thmenum}
    \thmitem{def:geometric_ray/opposite} We say that the rays \( r(t) \) and \( s(t) \) with a common vertex are \term{opposite} if \( r(t) = s(-t) \).

    \thmitem{def:geometric_ray/unidirectional} We say that the rays \( r(t) = \tau_{t d}(O) \) and \( s(t) = \tau_{t e}(P) \) are \term{unidirectional} if there exist some \hi{positive} scalar \( \lambda \) such that \( d = \lambda e \).
  \end{thmenum}

  \begin{figure}[!ht]
    \centering
    \includegraphics[page=1]{output/def__geometric_ray.pdf}
    \caption{Unidirectional and opposite rays.}\label{fig:def:geometric_ray}
  \end{figure}
\end{definition}

\begin{definition}\label{def:geometric_cone}\mcite[20]{Clarke2013}
  A \term{cone} is a union of \hyperref[def:geometric_ray]{rays} with a common vertex.

  \Fullref{thm:def:convex_hull/conic_cone} gives a necessary and sufficient condition for a cone to coincide with its \hyperref[def:conic_hull]{conic hull}. Despite the name, this is not true in general --- a counterexample is presented in \cref{fig:def:geometric_cone}.

  \begin{figure}[!ht]
    \centering
    \includegraphics{output/def__geometric_cone.pdf}
    \caption{One cone consisting of two rays and the conic hull of two other rays.}\label{fig:def:geometric_cone}
  \end{figure}
\end{definition}

\begin{definition}\label{def:line_segment}
  A \term{line segment} between the points \( x \) and \( y \) is the \hyperref[def:parametric_curve]{parametric curve}
  \begin{equation*}
    \begin{aligned}
      &s: [0, 1] \to A \\
      &s(t) \coloneqq (1 - t) x + t y.
    \end{aligned}
  \end{equation*}

  We denote the image of \( s \) via \( [x, y] \).
\end{definition}

\begin{proposition}\label{thm:def:convex_hull}
  \hyperref[def:convex_hull]{Convex sets} have the following basic properties:

  \begin{thmenum}
    \thmitem{thm:def:convex_hull/line_segments} A set is convex if and only if it contains the entire \hyperref[def:line_segment]{line} between any two of its points.

    \thmitem{thm:def:convex_hull/conic_cone} A \hyperref[def:geometric_cone]{cone} coincides with its \hyperref[def:convex_hull]{conic hull} if and only if it is a \hyperref[def:convex_hull]{convex set}.

    \thmitem{thm:def:convex_hull/closed_under_intersections} Any nonempty intersection of convex sets is convex.
  \end{thmenum}
\end{proposition}
\begin{proof}
  \SubProofOf{thm:def:convex_hull/line_segments} Trivial.

  \SubProofOf{thm:def:convex_hull/conic_cone}
  \SufficiencySubProof* Trivial since convex combinations are conic.

  \NecessitySubProof* Fix a convex cone \( C \) with vertex \( O \) and a conic combination
  \begin{equation*}
    x \coloneqq \parens*{ 1 - \sum_{k=1}^n t_k } O + \sum_{k=1}^n t_k x_k
  \end{equation*}
  of points in \( C \). Let
  \begin{align*}
    T \coloneqq \sum_{k=1}^n t_k,
    &&
    y \coloneqq \sum_{k=1}^n \frac {t_k} T x_k.
  \end{align*}

  Then
  \begin{equation*}
    x = (1 - T) O + T y
  \end{equation*}
  and
  \begin{equation*}
    \vect{Ox} = (1 - T) \vect{OO} + T \vect{Oy}.
  \end{equation*}

  Since \( y \) is a convex combination of members of \( C \), it itself belongs to \( C \). Since \( C \) is a cone and since \( T \) is nonnegative, \( x = \tau_{T \vect{Oy}}(O) \) also belongs to \( C \).

  \SubProofOf{thm:def:convex_hull/closed_under_intersections} Trivial.
\end{proof}

\begin{definition}\label{def:simplex}
  A \( k \)-\term{simplex} is the \hyperref[def:convex_hull]{convex hull} of \( k + 1 \) \hyperref[def:affine_dependence]{affinely independent} vectors, which we call the \term{vertices} of the simplex. The convex hull of any subset of the vertices is called a \term{face} of the simplex.
\end{definition}

\begin{example}\label{ex:def:simplex}
  We list several examples of \hyperref[def:simplex]{simplices}:
  \begin{thmenum}
    \thmitem{ex:def:simplex/point} A \( 0 \)-simplex is a \hyperref[rem:point]{point}.

    Indeed, a single vector has only one possible affine combination --- itself.

    \thmitem{ex:def:simplex/line_segment} A \( 1 \)-simplex is a \hyperref[def:line_segment]{line segment}.

    Indeed, suppose that \( x \) and \( y \) are affinely independent. Then the vector \( \vect{xy} \) must be linearly independent, i.e. nonzero. Thus, \( x \) and \( y \) are affinely independent if and only \( x \neq y \).

    Given two distinct points, their convex combination is a line segment --- see \fullref{thm:def:convex_hull/line_segments}.
  \end{thmenum}
\end{example}
