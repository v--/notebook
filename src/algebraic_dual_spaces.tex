\subsection{Algebraic dual spaces}\label{subsec:algebraic_dual_spaces}

\begin{definition}\label{def:dual_vector_space}\mcite[50]{Knapp2016BasicAlgebra}
  Let \( V \) be a \hyperref[def:vector_space]{vector space} over the \hyperref[def:field]{field} \( \BbbK \). By \fullref{thm:functions_over_algebra}, the set \( \hom(V, \BbbK) \) of all \hyperref[def:semimodule/homomorphism]{linear maps} from \( V \) to the underlying field \( \BbbK \) also form a vector space over \( \BbbK \).

  We will call this space the \term{algebraic dual space} of \( V \) and denote it by \( V^* \). We will call the functions in \( V^* \) \term{linear functionals}. The prefix \enquote{algebraic} is important when confusion is possible with \hyperref[def:continuous_dual_space]{continuous dual spaces}.
\end{definition}

\begin{remark}\label{rem:dual_space_bilinear_form}\mcite[16]{ИоффеТихомиров1974}
  If \( l \) is a \hyperref[def:dual_vector_space]{linear functional} over \( V \), we often use the notation \( \inprod l x \) rather than the function notation \( l(x) \). This is an extension of the notation for \hyperref[def:inner_product_space]{inner product spaces}.

  Moreover, \( \inprod \anon \anon \) is a \hyperref[def:multilinear_function]{bilinear function} from the Cartesian product \( V^* \times V \) to \( \BbbK \). Hence, if \( V \) is isomorphic to \( V^* \), then this is precisely an inner product.
\end{remark}

\begin{remark}\label{rem:functional}
  The term \enquote{functional} as a noun has no definite meaning.

  \begin{itemize}
    \item In the context of linear algebra, and in particular \fullref{def:dual_vector_space}, the term \enquote{functional} refers to \enquote{linear functional}, i.e. a \hyperref[def:semimodule/homomorphism]{linear map} from a \hyperref[def:vector_space]{vector space} to its base field.

    This terminology can be found, for example, in \cite[50]{Knapp2016BasicAlgebra} and \cite[sec. 26.1]{Тыртышников2004Лекции}.

    \item In the context of functional analysis, \enquote{linear functional} may refer to either \hyperref[def:continuous_dual_space]{continuous linear functionals} from some \hyperref[def:topological_vector_space]{topological vector space} to its base field, or to arbitrary linear functionals.

    The former terminology can be found, for example, in \cite[def. 3.1]{Rudin1991Functional} and \cite[sec. 1.3]{Clarke2013}.

    An arbitrary map from a topological vector space to its field may also be called a functional. For example, \cite[102]{KufnerFucik1980} and \cite[223]{Deimling1985} refer to \enquote{nonlinear functionals}. \hyperref[def:minkowski_functional]{Minkowski functionals} are notoriously nonlinear.

    \item In the context of recursive functions, for example in \cite{StanfordPlato:recursive_functions}, functionals are defined as \enquote{operations which map one or more functions of type \( \BbbN^k \to \BbbN \) (possibly of different arities) to other functions}.
  \end{itemize}

  The commonality between linear algebra and functional analysis is that \enquote{functional} refers to a map from a vector space to its base field. The commonality between functional analysis and logic is that \enquote{functional} refers to a map acting on a set of functions.
\end{remark}

\begin{proposition}\label{thm:algebraic_dual_basis}
  Fix a \hyperref[def:vector_space]{vector space} \( V \) over \( \BbbK \) and a \hyperref[def:hamel_basis]{basis} \( B \) of \( V \). For every basis vector \( e \), the \hyperref[def:basis_decomposition]{projection functionals} \( \pi_e: V \to \BbbK \) maps an arbitrary vector \( v \) to its \( e \)-th coordinate.

  We can thus regard \( \pi \) as a function from \( B \) to \( B^* \), where
  \begin{equation*}
    B^* \coloneqq \set{ \pi_e \given e \in B }.
  \end{equation*}

  \begin{thmenum}
    \thmitem{thm:algebraic_dual_basis/independent} Given basis vectors \( x \) and \( y \) from \( B \), the functionals \( \pi_x \) and \( \pi_y \) are \hyperref[def:linear_dependence]{linearly independent} in \( V^* \).

    In particular, the function \( \pi \) is injective. That is, if \( \pi_x = \pi_y \), then \( x = y \).

    \thmitem{thm:algebraic_dual_basis/finite} If \( e_1, \ldots, e_n \) is a basis of \( V \), then \( \pi_{e_1}, \ldots, \pi_{e_n} \) is a basis for the \hyperref[def:dual_vector_space]{dual space} \( V^* \).

    Hence, if \( V \) is finite-dimensional, the function \( \pi \) is surjective and hence it induces an isomorphism between \( V \) and \( V^* \).

    \thmitem{thm:algebraic_dual_basis/infinite} If \( V \) is infinite-dimensional, \( \pi \) is not surjective and the dimension of \( V \) is strictly smaller than that of \( V^* \).
  \end{thmenum}
\end{proposition}
\begin{proof}
  \SubProofOf{thm:algebraic_dual_basis/independent} Let \( t_x \) and \( t_y \) be scalars such that
  \begin{equation*}
    t_x \pi_x + t_y \pi_y = 0_V.
  \end{equation*}

  Then
  \begin{equation*}
    0 = t_x \pi_x(x) + t_y \pi_y(x) = t_x \cdot 1 + t_y \cdot 0.
  \end{equation*}

  Analogously, \( t_y = 0 \). Therefore, the functionals \( \pi_x \) and \( \pi_y \) are linearly independent.

  \SubProofOf{thm:algebraic_dual_basis/finite} By \fullref{thm:algebraic_dual_basis/independent}, \( \pi_{e_1}, \ldots, \pi_{e_n} \) is a linearly independent set in \( V^* \). We will show that it spans \( V^* \).

  Let \( l \) be an arbitrary linear functional. We have
  \begin{equation*}
    l(x) = l\parens*{ \sum_{k=1}^n \pi_{e_k}(x) \cdot e_k } = \sum_{k=1}^n \pi_{e_k}(x) \cdot l\parens*{ e_k }.
  \end{equation*}

  Generalizing on \( x \),
  \begin{equation*}
    l \coloneqq \sum_{k=1}^n l(e_k) \cdot \pi_{e_k}.
  \end{equation*}

  That is, \( l \) is a linear combination of \( \pi_{e_1}, \ldots, \pi_{e_n} \). Since \( l \) was arbitrary, we conclude that \( \pi_{e_1}, \ldots, \pi_{e_n} \) is also a spanning set and hence a basis.

  \SubProofOf{thm:algebraic_dual_basis/infinite} Suppose that \( B \) is an infinite set. Hence, \( B^* \) is an infinite linearly independent set. It is not spanning, however. As an example, take the linear functional
  \begin{equation*}
    l(x) \coloneqq \sum_{e \in B} \pi_{e_k}(x).
  \end{equation*}

  It is well-defined as a \hyperref[thm:free_semimodule_universal_property]{linear extension} of \( e \mapsto 1 \) for all \( e \in B \).

  Also, it cannot be represented as a finite linear combination of elements of \( B^* \). Hence, \( B^* \) is not a spanning set of \( V^* \).
\end{proof}

\begin{proposition}\label{def:double_dual_canonical_embedding}
  Fix a vector space \( V \). We define the canonical embedding into the double dual \( V^{**} \) of \( V \) by
  \begin{balign*}
     & \Phi: V \to V^{**}                              \\
     & \Phi(x) \coloneqq (\varphi \mapsto \varphi(x)),
  \end{balign*}
  where \( \varphi \in V^* \).
\end{proposition}

\begin{remark}\label{rem:finite_dimensional_dual_space_isomorphism}
  By \fullref{thm:finite_dimensional_spaces_are_isomorphic}, the vector space \( F^n \) is isomorphic to its dual \( {F^n}^* \).

  In practice, it is sometimes useful to distinguish between vectors and functionals. This is why we regard functionals as either
  \begin{itemize}
    \item functions
    \item column vectors
    \item row vectors
  \end{itemize}
  depending on what interpretation suits us best.

  This is consistent with \fullref{thm:matrix_and_linear_function_algebras}, where we regard linear operators as matrices that act on vectors by multiplication.

  For example, if we have the \hyperref[def:differentiability]{differentiable} function \( f(x, y) = xy \), we can regard its gradient at the point \( (\overline x, \overline y) \) as the row vector
  \begin{balign*}
    f'(\overline x, \overline y) =
    \begin{pmatrix}
      \overline y & \overline x
    \end{pmatrix}.
  \end{balign*}

  This is a linear functional that can acts on regular (column) vector by multiplying them from the left.
\end{remark}

\begin{definition}\label{def:dual_linear_operator}
  We define the \term{dual linear operator} of \( L: U \to V \) as
  \begin{balign*}
     & L^*: V^* \to U^*                \\
     & L^*(v^*) \coloneqq v^* \circ L.
  \end{balign*}
\end{definition}

\begin{definition}\label{def:vector_space_annihilator}\mcite[52]{Knapp2016BasicAlgebra}
  Fix a subset \( S \subseteq V \) of a vector space \( V \) over \( F \). We define the \term{annihilator} of \( S \) as the vector space of functionals
  \begin{equation*}
    \op{ann}(S) \coloneqq \{ x^* \in V^* \colon x^*(x) = 0_F \quad\forall x \in S \}.
  \end{equation*}
\end{definition}
