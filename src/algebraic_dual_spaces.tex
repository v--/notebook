\subsection{Algebraic dual spaces}\label{subsec:algebraic_dual_spaces}

\begin{definition}\label{def:dual_vector_space}\mcite[50]{Knapp2016BasicAlgebra}
  Let \( V \) be a \hyperref[def:vector_space]{vector space} over the \hyperref[def:field]{field} \( \BbbK \). By \fullref{thm:functions_over_algebra}, the set \( \hom(V, \BbbK) \) of all \hyperref[def:semimodule/homomorphism]{linear maps} from \( V \) to the underlying field \( \BbbK \) also form a vector space over \( \BbbK \).

  We call this space the \term{algebraic dual space} of \( V \) and denote it by \( V^* \). We call the functions in \( V^* \) \term{linear functionals}. The prefix \enquote{algebraic} is important when confusion is possible with \hyperref[def:continuous_dual_space]{continuous dual spaces}.
\end{definition}

\begin{remark}\label{rem:functional}
  The term \enquote{functional} as a noun has no definite meaning.

  \begin{itemize}
    \item In the context of linear algebra, and in particular \fullref{def:dual_vector_space}, the term \enquote{functional} refers to \enquote{linear functional}, i.e. a \hyperref[def:semimodule/homomorphism]{linear map} from a \hyperref[def:vector_space]{vector space} to its base field.

    This terminology can be found, for example, in \cite[50]{Knapp2016BasicAlgebra} and \cite[sec. 26.1]{Тыртышников2004Лекции}.

    \item In the context of functional analysis, \enquote{linear functional} may refer to either \hyperref[def:continuous_dual_space]{continuous linear functionals} from some \hyperref[def:topological_vector_space]{topological vector space} to its base field, or to arbitrary linear functionals.

    This terminology can be found, for example, in \cite[def. 3.1]{Rudin1991Functional} and \cite[sec. 1.3]{Clarke2013}.

    An arbitrary map from a topological vector space to its field may also be called a functional --- for example, \cite[102]{KufnerFucik1980} and \cite[223]{Deimling1985} refer to \enquote{nonlinear functionals}. \hyperref[def:minkowski_functional]{Minkowski functionals} are notoriously nonlinear.

    \item In the context of recursive functions, for example in \cite{StanfordPlato:recursive_functions}, functionals are defined as \enquote{operations which map one or more functions of type \( \BbbN^k \to \BbbN \) (possibly of different arities) to other functions}.
  \end{itemize}

  The commonality between linear algebra and functional analysis is that \enquote{functional} refers to a map from a vector space to its base field. The commonality between functional analysis and logic is that \enquote{functional} refers to a map acting on a set of functions.
\end{remark}

\begin{definition}\label{def:duality_pairing}\mimprovised
  A \term{duality pairing} \( \inprod \anon \anon: U \times V \to \BbbK \) is a \hyperref[def:multilinear_function]{bilinear function}.

   such that, if \( \inprod u v = 0 \) for all \( u \), then \( v = 0 \).
\end{definition}

\begin{definition}\label{thm:natural_duality_pairing}
  Given a vector space \( V \), the following function is \hyperref[def:multilinear_function]{bilinear}:
  \begin{equation*}
    \begin{aligned}
      &\inprod \cdot \cdot: V^* \times X \to \BbbK \\
      &\inprod {x^*} x \mapsto x^*(x).
    \end{aligned}
  \end{equation*}
\end{definition}

\begin{proposition}\label{thm:algebraic_dual_basis}
  Fix a \hyperref[def:vector_space]{vector space} \( V \) over \( \BbbK \) and a \hyperref[def:hamel_basis]{basis} \( B \) of \( V \). For every basis vector \( e \), the \term{projection functional} \( \pi_e: V \to \BbbK \), defined in \fullref{def:basis_decomposition}, maps an arbitrary vector \( v \) to its \( e \)-th coordinate.

  \begin{thmenum}
    \thmitem{thm:algebraic_dual_basis/independent} Given basis vectors \( x \) and \( y \) from \( B \), \( \pi_x \) and \( \pi_y \) are \hyperref[def:linear_dependence]{linearly independent} in \( V^* \).

    \thmitem{thm:algebraic_dual_basis/finite} Furthermore, if \( e_1, \ldots, e_n \) is a basis of \( V \), then \( \pi_{e_1}, \ldots, \pi_{e_n} \) is a basis for the \hyperref[def:dual_vector_space]{dual space} \( V^* \).

    \thmitem{thm:algebraic_dual_basis/infinite} The set \( \set{ \pi_e \given e \in B } \) spans \( V^* \) if and only if \( V \) is finite dimensional.
  \end{thmenum}
\end{proposition}
\begin{proof}
  \SubProofOf{thm:algebraic_dual_basis/independent} Let \( t_x \) and \( t_y \) be scalars such that
  \begin{equation*}
    t_x \pi_x + t_y \pi_y = 0_V.
  \end{equation*}

  Then
  \begin{equation*}
    0 = t_x \pi_x(x) + t_y \pi_y(x) = t_x \cdot 1 + t_y \cdot 0.
  \end{equation*}

  Analogously, \( t_y = 0 \). Therefore, the functionals \( \pi_x \) and \( \pi_y \) are linearly independent.

  \SubProofOf{thm:algebraic_dual_basis/finite} Let \( l \) be an arbitrary linear functional. Then
  \begin{equation*}
    l(y)
  \end{equation*}

  \SubProofOf{thm:algebraic_dual_basis/infinite}
\end{proof}

\begin{proposition}\label{def:double_dual_canonical_embedding}
  Fix a vector space \( V \). We define the canonical embedding into the double dual \( V^{**} \) of \( V \) by
  \begin{balign*}
     & \Phi: V \to V^{**}                              \\
     & \Phi(x) \coloneqq (\varphi \mapsto \varphi(x)),
  \end{balign*}
  where \( \varphi \in V^* \).
\end{proposition}

\begin{proposition}\label{thm:finite_dimensional_dual_space_is_isomorphic}
  The dual vector space of a finite-dimensional vector space has the same dimension.
\end{proposition}
\begin{proof}
  Let \( V \) be an \( n \)-dimensional vector space over \( F \) and let \( B \) be a basis of \( V \). For each \( b \in B \), define its dual vector on \( V^* \) as the linear \hyperref[thm:quotient_module_universal_property]{extension} of the functions
  \begin{balign*}
     & \varphi: B \to F                               \\
     & \varphi(x) \coloneqq \begin{cases}
      1, & x = b    \\
      0, & x \neq b
    \end{cases}
  \end{balign*}
  from the basis to the whole space. Denote the dual basis vector of \( b \) by \( b^* \).

  We will now show that the set \( B^* \coloneqq \{ b^* \colon b \in B \} \) forms a basis of \( V^* \).

  Fix \( x^* \in V^* \). Define
  \begin{equation*}
    y^* \coloneqq \sum_{b \in B} x^*(b) b^*.
  \end{equation*}

  The linear functions \( x^* \) and \( y^* \) evidently agree on the basis \( B \). Hence, they agree on the whole space.

  Hence, \( B^* \) is a basis of \( V^* \). Note that it has the same cardinality as the basis of \( B \).
\end{proof}

\begin{remark}\label{rem:finite_dimensional_dual_space_isomorphism}
  By \fullref{thm:finite_dimensional_spaces_are_isomorphic}, the vector space \( F^n \) is isomorphic to its dual \( {F^n}^* \).

  In practice, it is sometimes useful to distinguish between vectors and functionals. This is why we regard functionals as either
  \begin{itemize}
    \item functions
    \item column vectors
    \item row vectors
  \end{itemize}
  depending on what interpretation suits us best.

  This is consistent with \fullref{thm:finite_dimensional_operators_are_isomorphic_to_matrices}, where we regard linear operators as matrices that act on vectors by multiplication.

  For example, if we have the \hyperref[def:differentiability]{differentiable} function \( f(x, y) = xy \), we can regard its gradient at the point \( (\overline x, \overline y) \) as the row vector
  \begin{balign*}
    f'(\overline x, \overline y) =
    \begin{pmatrix}
      \overline y & \overline x
    \end{pmatrix}.
  \end{balign*}

  This is a linear functional that can acts on regular (column) vector by multiplying them from the left.
\end{remark}

\begin{definition}\label{def:dual_linear_operator}
  We define the \term{dual linear operator} of \( L: U \to V \) as
  \begin{balign*}
     & L^*: V^* \to U^*                \\
     & L^*(v^*) \coloneqq v^* \circ L.
  \end{balign*}
\end{definition}

\begin{definition}\label{def:vector_space_annihilator}\mcite[52]{Knapp2016BasicAlgebra}
  Fix a subset \( S \subseteq V \) of a vector space \( V \) over \( F \). We define the \term{annihilator} of \( S \) as the vector space of functionals
  \begin{equation*}
    \op{ann}(S) \coloneqq \{ x^* \in V^* \colon x^*(x) = 0_F \quad\forall x \in S \}.
  \end{equation*}
\end{definition}
