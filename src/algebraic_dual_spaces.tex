\subsection{Algebraic dual spaces}\label{subsec:algebraic_dual_spaces}

\begin{definition}\label{def:dual_vector_space}
  Let \( V \) be a vector space over \( F \). The \hyperref[thm:functions_over_ring_form_algebra]{vector space} of all \hyperref[def:linear_operator]{linear functions} from \( V \) to \( F \) is called the \Def{algebraic dual space} of \( V \) and is denoted by \( V^* \). The functions themselves are called \Def{linear functionals}.
\end{definition}

\begin{definition}\label{def:canonical_duality_pairing}
  We denote by \( \Prod \cdot \cdot \) the \Def{canonical \hyperref[def:duality_pairing]{duality pairing}}
  \begin{BreakableAlign*}
     & \Prod \cdot \cdot: V^* \times X \to \BbbK \\
     & \Prod {x^*} x \mapsto x^*(x).
  \end{BreakableAlign*}

  We are usually not interested in other duality pairings. See \fullref{def:locally_convex_duality_pairing}, however.
\end{definition}

\begin{definition}\label{def:double_dual_canonical_embedding}
  Fix a vector space \( V \). We define the \Def{canonical embedding} into the double dual \( V^{**} \) of \( V \) by
  \begin{BreakableAlign*}
     & \Phi: V \to V^{**}                              \\
     & \Phi(x) \coloneqq (\varphi \mapsto \varphi(x)),
  \end{BreakableAlign*}
  where \( \varphi \in V^* \).
\end{definition}

\begin{proposition}\label{thm:finite_dimensional_dual_space_is_isomorphic}
  The dual vector space of a finite-dimensional vector space has the same dimension.
\end{proposition}
\begin{proof}
  Let \( V \) be an \( n \)-dimensional vector space over \( F \) and let \( B \) be a basis of \( V \). For each \( b \in B \), define its dual vector on \( V^* \) as the linear \hyperref[thm:linear_map_iff_function_on_basis]{extension} of the functions
  \begin{BreakableAlign*}
     & \varphi: B \to F                               \\
     & \varphi(x) \coloneqq \begin{cases}
      1, & x = b    \\
      0, & x \neq b
    \end{cases}
  \end{BreakableAlign*}
  from the basis to the whole space. Denote the dual basis vector of \( b \) by \( b^* \).

  We will now show that the set \( B^* \coloneqq \{ b^* \colon b \in B \} \) forms a basis of \( V^* \).

  Fix \( x^* \in V^* \). Define
  \begin{equation*}
    y^* \coloneqq \sum_{b \in B} x^*(b) b^*.
  \end{equation*}

  The linear functions \( x^* \) and \( y^* \) evidently agree on the basis \( B \). By \fullref{thm:linear_maps_agree_on_free_module_if_they_agree_on_basis}, they agree on the whole space.

  Hence \( B^* \) is a basis of \( V^* \). Note that it has the same cardinality as the basis of \( B \).
\end{proof}

\begin{remark}\label{rem:finite_dimensional_dual_space_isomorphism}
  By \fullref{thm:finite_dimensional_spaces_are_isomorphic}, the vector space \( F^n \) is isomorphic to its dual \( {F^n}^* \).

  In practice, it is sometimes useful to distinguish between vectors and functionals. This is why we regard functionals as either
  \begin{itemize}
    \item functions
    \item column vectors
    \item row vectors
  \end{itemize}
  depending on what interpretation suits us best.

  This is consistent with \fullref{thm:finite_dimensional_operators_are_isomorphic_to_matrices}, where we regard linear operators as matrices that act on vectors by multiplication.

  For example, if we have the \hyperref[def:differentiability]{differentiable} function \( f(x, y) = xy \), we can regard its gradient at the point \( (\overline x, \overline y) \) as the row vector
  \begin{BreakableAlign*}
    f'(\overline x, \overline y) =
    \begin{pmatrix}
      \overline y & \overline x
    \end{pmatrix}.
  \end{BreakableAlign*}

  This is a linear functional that can acts on regular (column) vector by multiplying them from the left.
\end{remark}

\begin{definition}\label{def:dual_linear_operator}
  We define the \Def{dual linear operator} of \( L: U \to V \) as
  \begin{BreakableAlign*}
     & L^*: V^* \to U^*                \\
     & L^*(v^*) \coloneqq v^* \circ L.
  \end{BreakableAlign*}
\end{definition}

\begin{definition}\label{def:vector_space_annihilator}\MarginCite[52]{Knapp2016BAlg}
  Annihilators in vector spaces are quite different than annihilators in modules (see \fullref{def:left_module_annihilator}).

  Fix a subset \( S \subseteq V \) of a vector space \( V \) over \( F \). We define the \Def{annihilator} of \( S \) as the vector space of functionals
  \begin{equation*}
    \Op{ann}(S) \coloneqq \{ x^* \in V^* \colon x^*(x) = 0_F \quad\forall x \in S \}.
  \end{equation*}
\end{definition}
