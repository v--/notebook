\subsection{Free objects}\label{subsec:free_objects}

\begin{remark}\label{rem:free_construction_etymology}
  Free constructions in algebra such as \hyperref[def:free_monoid]{free monoids}, \hyperref[def:free_group]{free groups} and \hyperref[def:free_left_module]{free modules} all share a common feature - they allow us to construct an algebraic structure \enquote{freely}, without imposing any restrictions on the target structure. See \fullref{def:group_presentation} for a more precise justification. Jean-Pierre Marquis in \cite{StanfordPlato:category_theory} suggests that this is the reason free constructions are called \enquote{free}.

  The category-theoretic formulation of free functors and, more generally, free objects, is based on the idea that free \hyperref[def:free_functor]{functors} \enquote{remember} what \hyperref[def:forgetful_functor]{forgetful functors} \enquote{forget}.
\end{remark}

\begin{definition}\label{def:forgetful_functor}\mcite\cite[exmpl. 1.2.3]{Leinster2016Basic}
  A functor \( U: \cat{A} \to \cat{B} \) is called \term{forgetful} if it \enquote{forgets structure}. This \enquote{definition} is informal, but often a unique free functor is clear from the context and so the definition often makes perfect sense. See \fullref{ex:forgetful_functors}.
\end{definition}

\begin{example}\label{ex:forgetful_functors}
  Examples of forgetful functors include

  \begin{thmenum}
    \item The identity functor \( U = \id \) on \( \cat{Set} \), which forgets nothing.
    \item The functors \( U: \cat{Grp} \to \cat{Set} \) and \( U: \cat{Ring} \to \cat{Set} \), which forget the group and ring operations.
    \item The functor \( U: \cat{Top} \to \cat{Set} \) from \fullref{ex:top_adjoint_functor}, which forgets the topology.
    \item More generally, given a \hyperref[def:first_order_theory]{first-order theory} \( \Gamma \) and its corresponding category \( \cat{Model}_\Gamma \), the functor \( U: \cat{Model}_\Gamma \to \cat{Set} \) forgets any additional conditions imposed on the underlying sets of the models.
  \end{thmenum}
\end{example}

\begin{definition}\label{def:free_functor}\mcite\cite{nLab:free_object}
  A left \hyperref[subsec:adjoint_functors]{adjoint} to a \hyperref[def:forgetful_functor]{forgetful functor} is called a free functor.
\end{definition}

\begin{example}\label{ex:free_functors}
  Examples of free functors include

  \begin{thmenum}
    \item The identity functor \( U = \id \) on \( \cat{Set} \).
    \item The functor \( D: \cat{Set} \to \cat{Top} \) from \fullref{ex:top_adjoint_functor}, which sends a set to its corresponding topological space.
    \item The functor \( F: \cat{Set} \to \cat{Grp} \), which sends a set to its corresponding \hyperref[def:free_group]{free group}.
    \item The functor \( F: \cat{Set} \to \cat{Mod}_R \), which sends a set \( S \) to a \hyperref[def:free_left_module]{free module} whose basis is \( S \).
  \end{thmenum}
\end{example}

\begin{definition}\label{def:free_object}\mcite\cite{nLab:free_object}
  A free object satisfies the adjoint condition at a single point. Let \( U: \cat{A} \to \cat{B} \) be a forgetful functor and let \( b \in \cat{B} \). We say that \( a \in \cat{A} \) is a \term{free object} with respect to the forgetful functor \( U \) if there exists a morphism \( \eta_{a}: b \to Ua \) that satisfy the following universal property: for every object \( a' \in \cat{A} \) and morphism \( \eta_{a'}: b \to Ua' \), there exists a unique morphism \( f: a \to a' \) such that \( U(f) \circ \eta_a = \eta_{a'} \), that is, the following diagram commutes:
  \begin{equation*}
    \todo{Add diagram}\iffalse\begin{mplibcode}
      beginfig(1);
      input metapost/graphs;

      v1 := thelabel("$b$", origin);
      v2 := thelabel("$U(a)$", (-1, -1) scaled u);
      v3 := thelabel("$U(a')$", (1, -1) scaled u);

      a1 := straight_arc(v1, v2);
      a2 := straight_arc(v1, v3);

      d1 := straight_arc(v2, v3);

      draw_vertices(v);
      draw_arcs(a);

      drawarrow d1 dotted;

      label.ulft("$\eta_{a}$", straight_arc_midpoint of a1);
      label.urt("$\eta_{a'}$", straight_arc_midpoint of a2);
      label.top("$U(f)$", straight_arc_midpoint of d1);
      endfig;
    \end{mplibcode}\fi
  \end{equation*}
\end{definition}
