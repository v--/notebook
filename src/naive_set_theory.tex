\subsection{Na\"ive set theory}\label{subsec:naive_set_theory}

Na\"ive set theory is usually defined informally by only specifying that a set is an unordered collection of objects without repetition. It turns out that this can easily be formalized as a \hyperref[def:first_order_theory]{first-order theory}, albeit an inconsistent one. Still, it is useful for introducing important concepts that can ease the introduction of \hyperref[def:zfc]{\logic{ZFC}}. The definitions we introduce and the proofs we provide will turn out to be valid in \logic{ZFC} also. In other words, we build sets that are valid in \logic{ZFC} using na\"ive set theory and later prove in \fullref{thm:zfc_existence_theorems} that the constructions are valid.

\begin{definition}\label{def:naive_set_theory}
  The \term{language of na\"ive set theory} is a \hyperref[def:first_order_syntax]{first-order language} \( \mscrL \) with only a single \hyperref[rem:first_order_formula_conventions/infix]{infix} binary relation \( \in \) called \term{set membership}. If \( \xi \in \eta \), we say that \( \xi \) is a \term{member} or \term{element} of \( \eta \) and that \( \eta \) \term{contains} \( \xi \). It is very common to use \hyperref[rem:first_order_formula_conventions/relativization]{relativization of quantifiers} with \( \in \) when dealing with sets.

  For the sake of simplicity, we will not introduce into the language any other functional or predicate symbols but will use \hyperref[rem:predicate_formula]{predicate formulas}.

  \term{Na\"ive set theory} is a \hyperref[def:first_order_theory]{first-order theory} axiomatized by the following:
  \begin{thmenum}
    \thmitem{def:naive_set_theory/extensionality}\mcite[sec. 61.1]{OpenLogicFull} The \term{axiom of extensionality}, which states that two sets are equal if and only if they have the same members. Symbolically,
    \begin{equation}\label{eq:def:naive_set_theory/extensionality}
      \parens[\Big]{ \qforall \zeta (\zeta \in \xi \leftrightarrow \zeta \in \eta) } \rightarrow \parens[\Big]{ \xi \doteq \eta }.
    \end{equation}

    As a consequence, a set is only distinguished by what it contains and thus the ordering and repetition of members of a set play no role. This axiom is also important in \logic{ZFC} --- see \fullref{def:zfc/extensionality}.

    The \hyperref[def:material_implication/converse]{converse} of \eqref{eq:def:naive_set_theory/extensionality} follows trivially by applying \fullref{thm:first_order_syntactic_deduction_theorem} to \eqref{eq:def:first_order_derivation_system/equality/elim_left}.

    \thmitem{def:naive_set_theory/unrestricted_comprehension} The \term{axiom schema of unrestricted comprehension} states that any formula \( \varphi \) with a single free variable defines a set. For each such formula \( \varphi \), the following is an axiom:
    \begin{equation}\label{eq:def:naive_set_theory/unrestricted_comprehension}
      \qexists \alpha \qforall \xi (\xi \in \alpha \leftrightarrow \varphi[\xi]).
    \end{equation}

    Compare this axiom schema to \hyperref[def:zfc/specification]{restricted comprehension}. In the context of na\"ive set theory they are equivalent because each is a special case of the other one.

    Because our goal is for all our constructions to be valid in \hyperref[def:zfc]{\logic{ZFC}}, we only use unrestricted comprehension where the existence of the set is justified by other axioms of \logic{ZFC}.
  \end{thmenum}
\end{definition}

\begin{definition}\label{def:set}
  Assume that we have a fixed model \( \mscrU = (U, I) \) of \hyperref[def:naive_set_theory]{na\"ive set theory} or \hyperref[def:zfc]{\logic{ZFC}}.  Unless it is important to make a distinction between different universes, we assume that the universe \( U \) is fixed. See \fullref{ex:skolems_paradox} for an important distinction between the mathematics within \( U \) and the metamathematics in which \( U \) is defined.

  A \term{set} is defined simply as an element of \( U \). If \( A \) is a set and \( x \in A \), we say that \( x \) is a \term{member} or \term{element} of \( A \) or, in accordance with \fullref{def:point}, a \term{point} in \( A \).
\end{definition}

\begin{definition}\label{def:set_builder_notation}
  We introduce the following syntax for unrestricted comprehension, commonly called \term{set-builder notation}:
  \begin{equation*}
    \set{ x \given \varphi[x] }.
  \end{equation*}

  Note that we break our convention described in \fullref{rem:mathematical_logic_conventions/variable_symbols} for only using Greek letters for formula variable names. This is justified by \fullref{rem:first_order_theories_in_zfc} because we interpret \( x \) as an element of some model --- an axiom defining the set \( A \). We often use looser syntax and prose in set-builder notation compared to general formulas.

  As examples such as \fullref{thm:russels_paradox} show, unrestricted comprehension can easily lead to contradictions. If \( A \) is a set, the axiom of specification \fullref{def:zfc/specification} tells us that \( \set{ x \given x \in A \wedge \varphi[x] } \) is a set within \hyperref[def:zfc]{\logic{ZFC}}. We use the following syntax for this \hyperref[def:zfc/separation]{restricted comprehension}:
  \begin{equation*}
    \set{ x \in A \given \varphi[x] }
  \end{equation*}

  Instead of using the delimiter \( \given \), we sometimes also use \( : \) or even omit the delimiter altogether and simply enumerate the members of the set:
  \begin{equation*}
    \set{ 1, 3, 9, 27 }.
  \end{equation*}

  Note that we enumerate certain numbers but only do so for illustrative purposes because the \hyperref[def:set_of_natural_numbers]{natural numbers} are not yet defined in terms of sets.

  We can also place an ellipsis if a certain pattern is obvious:
  \begin{equation*}
    \set{ 1, 3, 9, 27, \ldots }.
  \end{equation*}
\end{definition}

\begin{remark}\label{rem:multile_set_membership_shorthand}
  Within the metalogic, we often use the notation \( x_1, \ldots, x_n \in A \) to mean that \( x_k \in A \) for \( k \in \set{ 1, \ldots, n } \).
\end{remark}

\begin{definition}\label{def:empty_set}
  A very important set is the \term{empty set}
  \begin{equation*}
    \varnothing \coloneqq \set{ x \given \bot },
  \end{equation*}
  which contains no elements. We define the \hyperref[rem:predicate_formula]{predicate formula}
  \begin{equation}\label{eq:def:empty_set/predicate}
    \op{IsEmpty}[\alpha] \coloneqq \qforall \eta \neg \eta \in \alpha.
  \end{equation}
\end{definition}

\begin{theorem}[Russell's paradox]\label{thm:russels_paradox}
  \hyperref[def:naive_set_theory]{Na\"ive set theory} is \hyperref[def:proof_derivation_system_consistency]{inconsistent}. More precisely, the instance of the \hyperref[def:naive_set_theory/unrestricted_comprehension]{schema of unrestricted comprehension} with
  \begin{equation}\label{eq:thm:russels_paradox_comprehension_formula}
    \varphi = (\xi \not\in \xi)
  \end{equation}
  allows us to derive \( \bot \) in \hyperref[def:classical_logic]{classical logic}.

  Thus the set
  \begin{equation}\label{eq:thm:russels_paradox_set}
    R \coloneqq \set{ x \given x \not\in x }
  \end{equation}
  of all sets that do not contain themselves is not well-defined. Indeed, from \( R \not\in R \) it follows that \( R \in R \) and from \( R \in R \) it follows that \( R \not\in R \).
\end{theorem}
\begin{proof}
  After substituting \eqref{eq:thm:russels_paradox_comprehension_formula} in \eqref{eq:def:naive_set_theory/unrestricted_comprehension}, we obtain the following axiom of na\"ive set theory:
  \begin{equation}\label{eq:thm:russels_paradox_comprehension_axiom}
    \psi \coloneqq \qexists \eta \qforall \zeta (\zeta \in \eta \leftrightarrow \neg (\zeta \in \zeta)).
  \end{equation}

  We will show that the negation \( \neg\psi \) of \( \psi \) is also derivable in this theory. An explicit form of the negation can be obtained by utilizing the equivalences \fullref{thm:first_order_quantifiers_are_dual} and \fullref{thm:boolean_equivalences/biconditional_negation}:
  \begin{equation*}
    \neg\psi = \qforall \eta \qexists \zeta (\zeta \in \eta \leftrightarrow \zeta \in \zeta).
  \end{equation*}

  But \( \neg\psi \) is a direct consequence of the extensionality axiom \eqref{eq:def:naive_set_theory/extensionality}, hence \( \neg\psi \) is indeed derivable in na\"ive set theory.

  Hence \( \psi \) and \( \neg\psi \) are both derivable in the same theory and we can use \eqref{eq:thm:minimal_natural_deduction/neg/elim} to also derive \( \bot \). Therefore \( \bot \) is derivable in na\"ive set theory and hence it is inconsistent.
\end{proof}

\begin{remark}\label{rem:family_of_sets}
  In \hyperref[def:zfc]{\logic{ZFC}}, everything is a set. However, it is often the case that we are not interested in how a set's elements are encoded as sets and only in how they behave, e.g. when working with \hyperref[def:set_of_natural_numbers]{natural numbers}, we are interested in the elements of \( \BbbN \) and not in the way every element of \( \BbbN \) is encoded as a set.

  In order to reduce repetitiveness, sets whose elements we consider to be other sets, are often called \term{families of sets}. In particular, if all (different) sets are \hyperref[def:subset]{disjoint}, we say that the family is a \term{disjoint family}. We usually assume that the sets are nonempty.

  We often consider \hyperref[def:indexed_family]{indexed families}, i.e. sets which depend on a parameter, which further highlight our intention to distinguish between a point in a set, the set itself and some family of sets to which the latter belongs.
\end{remark}

\begin{definition}\label{def:subset}
  We say that \( A \) is a \term{subset} of \( B \) and write \( A \subseteq B \) if every member of \( A \) is a member of \( B \). If \( A \) is a subset of \( B \), we say that B is a \term{superset} of \( A \).

  If \( A \subseteq B \) and \( A \neq B \), we say that \( A \) is a \term{proper subset} of \( B \) and write \( A \subsetneq B \).

  The relation \( \subseteq \) is called the inclusion relation and it gives a partial ordering between sets. See \fullref{thm:boolean_algebra_of_subsets}. If an entire family of sets are not pairwise comparable, we say that they are \term{disjoint}.

  We also define the \hyperref[rem:predicate_formula]{predicate formula}
  \begin{equation}\label{eq:def:basic_set_operations/successor/predicate}
    \op{IsSubset}[\alpha, \beta] \coloneqq \qforall \xi (\xi \in \alpha \rightarrow \xi \in \beta),
  \end{equation}
  which is valid when \( \alpha \) is a subset of \( \beta \).
\end{definition}

\begin{remark}\label{rem:subset_notation}
  Some authors, such as \cite{Kelley1955}, use the notation \( A \subseteq B \) to mean \enquote{all elements of \( A \) belong to \( B \)}, even in the case when \( A = B \). To avoid confusion, we use the notations \( A \subseteq B \) and \( A \subsetneq B \) (see \fullref{def:subset}).
\end{remark}

\begin{remark}\label{rem:singleton_sets}
  Sets with a single elements are usually called \term{singletons}. It is sometimes convenient, especially with connection to geometry or \hyperref[def:multi_valued_function]{multi-valued functions} (e.g. when dealing with \hyperref[def:net_convergence/limit]{limits of nets} or \hyperref[def:subdifferentials]{subdifferentials}), to not distinguish between singleton sets and their corresponding element.
\end{remark}

\begin{definition}\label{def:basic_set_operations}
  We define the following operations:

  \begin{thmenum}
    \thmitem{def:basic_set_operations/intersection} The \term{intersection} of a nonempty set \( \mscrA \) is
    \begin{equation*}
      \bigcap \mscrA \coloneqq \set{ x \given \qforall {A \in \mscrA} x \in A }.
    \end{equation*}

    We leave \( \bigcap \varnothing \) undefined because it should be a \hyperref[def:poset_extremal_points/top_and_bottom]{top element} in the \hyperref[thm:boolean_algebra_of_subsets]{Boolean algebra of all sets}, but the latter does not exist because of \fullref{thm:russels_paradox}.

    For two sets \( A \) and \( B \), we define the \term{binary intersection} as
    \begin{equation*}
      A \cap B \coloneqq \bigcap \set{ A, B } = \set{ x \given x \in A \T{and} x \in B }.
    \end{equation*}

    \thmitem{def:basic_set_operations/union} Dually to \hyperref[def:basic_set_operations/intersection]{intersections}, the \term{union} of an arbitrary set \( \mscrA \) is defined as
    \begin{equation*}
      \bigcup A \coloneqq \set{ x \given \qexists {A \in \mscrA} x \in A }.
    \end{equation*}

    In particular, \( \bigcup \varnothing = \varnothing \).

    For two sets \( A \) and \( B \), we define the \term{binary union} as
    \begin{equation*}
      A \cup B \coloneqq \bigcup \set{ A, B } = \set{ x \given x \in A \T{or} x \in B }.
    \end{equation*}

    \thmitem{def:basic_set_operations/difference} The \term{difference} of the sets \( A \) and \( B \) is
    \begin{equation*}
      A \setminus B \coloneqq \set{ x \in A \given x \not\in B }.
    \end{equation*}

    \thmitem{def:basic_set_operations/power_set} The \term{power set} \( \pow(A) \) of \( A \) is the family of all subsets of \( A \). Symbolically,
    \begin{equation*}
      \pow(A) \coloneqq \set{ B \given B \subseteq A }.
    \end{equation*}

    \thmitem{def:basic_set_operations/successor} The \term{successor} \( \op{succ}(A) \) of \( A \) is the set
    \begin{equation*}
      \op{succ}(A) \coloneqq A \cup \set{ A }.
    \end{equation*}

    We also define the \hyperref[rem:predicate_formula]{predicate formula}
    \begin{equation}\label{eq:def:basic_set_operations/successor/predicate}
      \op{IsSucc}[\alpha, \beta] \coloneqq \qforall \xi \parens[\Big]{ \xi \in \alpha \leftrightarrow (\xi \in \beta \vee \xi = \beta) }.
    \end{equation}
  \end{thmenum}
\end{definition}

\begin{proposition}\label{thm:set_difference_properties}
  \hyperref[def:basic_set_operations/difference]{Set difference} has the following basic properties:
  \begin{thmenum}
    \thmitem{thm:set_difference_properties/intersection} If \( A \) and \( B \) are subsets of \( C \), then \( A \setminus B = A \cap (C \setminus B) \).

    \thmitem{thm:set_difference_properties/double_difference} If \( A \subseteq B \), then \( B \setminus (B \setminus A) = A \)
  \end{thmenum}
\end{proposition}
\begin{proof}
  \SubProofOf{thm:set_difference_properties/intersection} Since \( a \in A \) implies \( a \in C \), we have
  \begin{align*}
    A \setminus B
    &=
    \set{ x \in A \given x \not\in B }
    = \\ &=
    \set{ x \in A \given x \in C \T{and} x \not\in B }
    = \\ &=
    A \cap (C \setminus B).
  \end{align*}

  \SubProofOf{thm:set_difference_properties/double_difference} By \hyperref[thm:minimal_propositional_negation_laws/dne]{double negation elimination},
  \begin{align*}
    B \setminus (B \setminus A)
    &=
    \set[\Big]{ x \in B \given x \not\in \set{ x \in B \given x \not\in A } }
    = \\ &=
    \set{ x \in B \given x \in A }
    = \\ &=
    A.
  \end{align*}
\end{proof}

\begin{proposition}\label{thm:boolean_algebra_of_subsets}
  Let \( X \) be an arbitrary set. Then the \hyperref[def:basic_set_operations/power_set]{power set} \( \pow(A) \) endowed with the \hyperref[def:subset]{inclusion} partial order \( \subseteq \) is a \hyperref[def:semilattice/complete]{complete} \hyperref[def:boolean_algebra]{Boolean algebra}. Explicitly:

  \begin{thmenum}
    \thmitem{thm:boolean_algebra_of_subsets/join} The \hyperref[def:semilattice/join]{join} of an arbitrary family \( \mscrA \) of subsets of \( X \) is simply the \hyperref[def:basic_set_operations/union]{union} \( \bigcap \mscrA \).

    \thmitem{thm:boolean_algebra_of_subsets/top} The \hyperref[def:poset_extremal_points/top_and_bottom]{top element} is the set \( X \) itself.

    \thmitem{thm:boolean_algebra_of_subsets/meet} The \hyperref[def:semilattice/meet]{meet} of an arbitrary family \( \mscrA \) of sets is simply the \hyperref[def:basic_set_operations/intersection]{intersection} \( \bigcup \mscrA \). Unlike for a general family of sets, we have no problem defining the intersection of an empty set to be the top element \( X \).

    \thmitem{thm:boolean_algebra_of_subsets/bottom} The \hyperref[def:poset_extremal_points/top_and_bottom]{bottom element} is the empty set.

    \thmitem{thm:boolean_algebra_of_subsets/complement} The \hyperref[def:boolean_algebra]{complement} \( A^\complement \) of the subset \( A \) is the \hyperref[def:basic_set_operations/difference]{difference} \( X \setminus A \).
  \end{thmenum}
\end{proposition}
\begin{proof}
  The statements \fullref{thm:boolean_algebra_of_subsets/join,thm:boolean_algebra_of_subsets/top,thm:boolean_algebra_of_subsets/meet,thm:boolean_algebra_of_subsets/bottom} are trivial and \fullref{thm:boolean_algebra_of_subsets/complement} follows from \fullref{thm:set_difference_properties/double_difference}.
\end{proof}

\begin{remark}\label{rem:binary_vs_arbitrary_tuples}
  We give two pairs of definitions for tuples and Cartesian products. The first pair, \fullref{def:binary_cartesian_product}, is quite restricted and is mostly necessary for defining \hyperref[def:function]{functions} and ensuring that everything along the way is indeed a set. The second pair of definitions, given in \fullref{def:cartesian_product}, can then be used for arbitrary (binary and nonbinary) products.
\end{remark}

\begin{definition}\label{def:binary_cartesian_product}\mcite[def. 1.23]{OpenLogicFull}
  The \term{ordered pair} or \term{Kuratowski pair} \( (x, y) \) of the sets \( x \) and \( y \) is
  \begin{equation*}
    (x, y) \coloneqq \set{ \set{ x }, \set{ x, y } }.
  \end{equation*}

  This is a simple and widespread definition that encodes the order of \( x \) and \( y \), unlike the set \( \set{ x, y } \) for example.

  The \term{binary Cartesian product} of the sets \( A \) and \( B \) is
  \begin{equation*}
    A \times B \coloneqq \set{ (x, y) \given x \in A \T{and} y \in B }.
  \end{equation*}

  The operation \( \times \) is obviously associative hence we identify the Cartesian products \( (A \times B) \times C \) and \( A \times (B \times C) \) and call their elements ordered triples. Similarly, for each \hyperref[rem:peano_arithmetic_zero/positive]{positive integer} \( n \) we define \( n \)-tuples as elements of \( n \)-fold Cartesian product of a set with itself. This definition is tricky because we haven't yet defined natural numbers within set theory, however from a metalogical perspective (where we have natural numbers) it makes perfect sense to simply introduce countably many definitions into na\"ive set theory. This can be done as explained in \fullref{rem:first_order_theories_in_zfc}. Nonetheless, we usually prefer to use the more general \fullref{def:cartesian_product} as only use \( n \)-tuples as a temporary definition that can help us define general relations and functions.
\end{definition}

\begin{definition}\label{def:grothendieck_universe}
  \todo{Define Grothendieck universes}
\end{definition}
