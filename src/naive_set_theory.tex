\subsection{Na\"ive set theory}\label{subsec:naive_set_theory}

Na\"ive set theory is usually defined informally by only specifying that a set is an unordered collection of objects without repetition. It turns out that it is a valid \hyperref[def:first_order_theory]{first-order theory}, albeit inconsistent. Our main drawback is that within our more formal approach towards na\"ive set theory, the only possible members of sets are other sets.

\begin{definition}\label{def:naive_set_theory}
  The \term{language of set theory} is a \hyperref[def:first_order_syntax]{first-order language} \( \mscrL \) with only a single \hyperref[rem:first_order_formula_conventions/infix]{infix} binary relation \( \in \) called \term{set membership}. For the sake of simplicity, we will not introduce into the language any other functional or predicate symbols but will use the shorthands from \fullref{rem:first_order_formula_conventions}.

  \term{Na\"ive set theory} is a \hyperref[def:first_order_theory]{first-order theory} axiomatized by the following:
  \begin{thmenum}
    \thmitem{def:naive_set_theory/extensionality}\mcite[sec. 61.1]{OpenLogicFull} The \term{axiom of extensionality}
    \begin{equation}\label{eq:def:naive_set_theory/extensionality}
      \parens[\Big]{ \qforall \zeta (\zeta \in \xi \leftrightarrow \zeta \in \eta) } \rightarrow \parens[\Big]{ \xi \doteq \eta }.
    \end{equation}

    The \hyperref[def:binary_relation/converse]{converse} of \eqref{eq:def:naive_set_theory/extensionality} follows trivially from \eqref{eq:def:first_order_axiomatic_derivation_system/axioms/equality/subst}.

    The axiom states that two sets are equal if and only if they have the same elements. As a consequence, a set is only distinguished by what it contains and thus the ordering and repetition of members of a set play no role.

    \thmitem{def:naive_set_theory/unrestricted_comprehension} The \term{axiom schema of unrestricted comprehension}, which states that for any formula \( \varphi \) with \( \xi \) as a free variable, we add the following axiom
    \begin{equation}\label{eq:def:naive_set_theory/unrestricted_comprehension}
      \qexists \eta \qforall \zeta (\zeta \in \eta \leftrightarrow \varphi[\xi \mapsto \zeta]).
    \end{equation}

    Put simply, to each formula \( \varphi \) with at least one free variable \( \xi \) there corresponds a set \( \eta \) such that \( \zeta \in \eta \) precisely when \( \varphi[\xi \mapsto \zeta] \) is valid.
  \end{thmenum}
\end{definition}

\begin{definition}\label{def:set}
  Assume that we have a fixed model \( \mscrU = (U, I) \) of \hyperref[def:naive_set_theory]{na\"ive set theory} or \hyperref[def:zfc]{ZFC}.

  A \term{set} is simply an element of the universe \( U \). The rest of the definition deals with proper classes --- see \fullref{def:set_builder_notation} for a logical continuation of na\"ive set theory.

  The domain \( U \) of the model is also a set but it is a set within the metalogic --- see \fullref{rem:sets}. Thus both \( U \) and \( x \in U \) are sets, however \( x \) is a set within the object logic and \( U \) is a set within the metalogic. It may happen that \( x \) is not a set within the metalogic if \( \mscrU \) is not a set-theoretic model, i.e. a set consisting only of sets. Within the object logic, we have no knowledge of the domain \( U \), much less of any properties on \( U \). We only have access to its elements. Hence \( U \) is never a set within the object logic.

  As it turns out, it is sometimes necessary to speak about very general sets such as \enquote{the set of all sets} or \enquote{the set of all topological spaces}. These are indeed sets within na\"ive set theory but ZFC is much more restrictive and by \fullref{thm:zfc_no_set_is_member_of_itself}, there cannot be a set of all sets within a model of ZFC.

  This leads us to the following definition: A \term{class} is a subset of \( U \) available within the object logic. This definition is tricky to formalize entirely within the object language, however as it currently stands, it captures the essential idea that classes are sets constructed using unrestricted comprehension. Luckily, we rarely need classes, and mostly when defining categories, therefore we can use this definition which, as it stands, requires us occasionally cross the boundary between object logic and metalogic.

  Classes which are not sets are called \term{proper classes}. An alternative terminology is that sets are \term{\( U \)-small sets} and proper classes are \term{\( U \)-large sets}, where the prefix may be avoided if the model \( \mscrU \) is clear from the context. This terminology is consistent with \fullref{def:small_and_large_categories}.
\end{definition}

\begin{definition}\label{def:set_builder_notation}
  We introduce the following syntax for unrestricted comprehension, commonly called \term{set-builder notation}:
  \begin{equation*}
    \set{ x \given \varphi[\xi \mapsto x] }.
  \end{equation*}

  As examples such as \fullref{thm:russels_paradox} show, unrestricted comprehension can easily lead to contradictions. If \( A \) is a set, the axiom of specification \ref{def:zfc/A6} tells us that \( \set{ x \given x \in A \wedge \varphi[\xi \mapsto x] } \) is a set within ZFC. We use the following syntax for this restricted comprehension:
  \begin{equation*}
    \set{ x \in A \given \varphi[\xi \mapsto x] }
  \end{equation*}

  Instead of using the delimiter \( \given \), we sometimes also use \( : \) or even omit the delimiter altogether and simply enumerate the members of the set:
  \begin{equation*}
    \set{ 1, 3, 9, 27 }.
  \end{equation*}

  Note that we enumerate certain numbers but only do so for illustrative purposes because the \hyperref[def:set_of_natural_numbers]{natural numbers} are not yet defined in terms of sets.

  We can also place an ellipsis if a certain pattern is obvious:
  \begin{equation*}
    \set{ 1, 3, 9, 27, \ldots }.
  \end{equation*}
\end{definition}

\begin{theorem}[Russell's paradox]\label{thm:russels_paradox}
  \hyperref[def:naive_set_theory]{Na\"ive set theory} is \hyperref[def:first_order_theory/consistent]{inconsistent}. More precisely, the instance of the \hyperref[def:naive_set_theory/unrestricted_comprehension]{schema of unrestricted comprehension} with
  \begin{equation}\label{eq:thm:russels_paradox_comprehension_formula}
    \varphi = (\xi \not\in \xi)
  \end{equation}
  allows us to derive \( \bot \) in \hyperref[def:classical_logic]{classical logic}.

  Thus the set
  \begin{equation}\label{eq:thm:russels_paradox_set}
    R \coloneqq \set{ x \given x \not\in x }
  \end{equation}
  of all sets that do not contain themselves is not well-defined. Indeed, from \( R \not\in R \) it follows that \( R \in R \) and from \( R \in R \) it follows that \( R \not\in R \).
\end{theorem}
\begin{proof}
  After substituting \eqref{eq:thm:russels_paradox_comprehension_formula} in \eqref{eq:def:naive_set_theory/unrestricted_comprehension}, we obtain the following axiom of na\"ive set theory:
  \begin{equation}\label{eq:thm:russels_paradox_comprehension_axiom}
    \psi \coloneqq \qexists \eta \qforall \zeta (\zeta \in \eta \leftrightarrow \neg (\zeta \in \zeta)).
  \end{equation}

  We will show that the negation \( \neg\psi \) of \( \psi \) also belongs to na\"ive set theory. An explicit form of the negation can be obtained by using \fullref{thm:first_order_quantifiers_are_dual} and \fullref{thm:boolean_equivalences/biconditional_negation}:
  \begin{equation*}
    \neg\psi = \qforall \eta \qexists \zeta (\zeta \in \eta \leftrightarrow \zeta \in \zeta).
  \end{equation*}

  But \( \neg\psi \) is a direct consequence of the extensionality axiom \eqref{eq:def:naive_set_theory/extensionality}, hence indeed \( \neg\psi \) belongs to the na\"ive set theory.

  Now \( \psi \) and \( \neg\psi \) both belong to the same theory, which is only possible due to the principle of explosion \eqref{eq:thm:minimal_propositional_negation_laws/efq} if \( \bot \) belongs to the theory.

  Therefore \( \bot \) belongs to na\"ive set theory and hence it is inconsistent.
\end{proof}
