\subsection{Na\"ive set theory}\label{subsec:naive_set_theory}

Na\"ive set theory is usually defined informally by only specifying that a set is an unordered collection of objects without repetition. It turns out that this can easily be formalized as a \hyperref[def:first_order_theory]{first-order theory}, albeit an inconsistent one. Our main drawback is that within our more formal approach towards na\"ive set theory, the only possible members of sets are other sets.

\begin{definition}\label{def:naive_set_theory}
  The \term{language of set theory} is a \hyperref[def:first_order_syntax]{first-order language} \( \mscrL \) with only a single \hyperref[rem:first_order_formula_conventions/infix]{infix} binary relation \( \in \) called \term{set membership}. For the sake of simplicity, we will not introduce into the language any other functional or predicate symbols but will use the shorthands from \fullref{rem:first_order_formula_conventions}.

  \term{Na\"ive set theory} is a \hyperref[def:first_order_theory]{first-order theory} axiomatized by the following:
  \begin{thmenum}
    \thmitem{def:naive_set_theory/extensionality}\mcite[sec. 61.1]{OpenLogicFull} The \term{axiom of extensionality}
    \begin{equation}\label{eq:def:naive_set_theory/extensionality}
      \parens[\Big]{ \qforall \zeta (\zeta \in \xi \leftrightarrow \zeta \in \eta) } \rightarrow \parens[\Big]{ \xi \doteq \eta }.
    \end{equation}

    The \hyperref[def:binary_relation/converse]{converse} of \eqref{eq:def:naive_set_theory/extensionality} follows trivially from \eqref{eq:def:first_order_axiomatic_derivation_system/axioms/equality/subst}.

    The axiom states that two sets are equal if and only if they have the same elements. As a consequence, a set is only distinguished by what it contains and thus the ordering and repetition of members of a set play no role.

    \thmitem{def:naive_set_theory/unrestricted_comprehension} The \term{axiom schema of unrestricted comprehension}, which states that for any formula \( \varphi \) with \( \xi \) as a free variable, we add the following axiom
    \begin{equation}\label{eq:def:naive_set_theory/unrestricted_comprehension}
      \qexists \eta \qforall \zeta (\zeta \in \eta \leftrightarrow \varphi[\xi \mapsto \zeta]).
    \end{equation}

    Put simply, to each formula \( \varphi \) with at least one free variable \( \xi \) there corresponds a set \( \eta \) such that \( \zeta \in \eta \) precisely when \( \varphi[\xi \mapsto \zeta] \) is valid.
  \end{thmenum}
\end{definition}

\begin{definition}\label{def:set}
  Assume that we have a fixed model \( \mscrU = (U, I) \) of \hyperref[def:naive_set_theory]{na\"ive set theory} or \hyperref[def:zfc]{ZFC}.  Unless it is important to make a distinction between different universes, we assume that the universe \( U \) is fixed.

  A \term{set} is defined simply as an element of \( U \). Within the context of \hyperref[def:zfc]{ZFC}, sets must satisfy more restrictive axioms and we use the term \term{class} to refer to sets within na\"ive set theory. A \term{proper class} is then a class that is not a set. We avoid using classes as much as possible. See also \fullref{rem:proper_class}.
\end{definition}

\begin{remark}\label{rem:proper_class}
  The domain \( U \) of the model in \fullref{def:set} is also a set but it is a set within the metalogic --- see \fullref{rem:set_definition_recursion}. Thus both \( U \) and \( x \in U \) are sets, however \( x \) is a set within the object logic and \( U \) is a set within the metalogic. It may happen that \( x \) is not a set within the metalogic if \( \mscrU \) is not a set-theoretic model, i.e. a set consisting only of sets. Within the object logic, we have no knowledge of the domain \( U \), much less of any properties on \( U \). We only have access to its elements. Hence \( U \) is never a set within the object logic.

  This leads us to the following alternative to the definition of a class in \fullref{def:set}: A \term{class} is a subset of \( U \) available within the object logic. This definition is tricky to formalize entirely within the object language as any such formalization must modify or enrich the carefully chosen axioms of the theory. As the definition currently stands, it captures the idea that classes are sets constructed using unrestricted comprehension on the universe.

  We avoid needing to formalize this idea by introducing multiple universes --- see \fullref{def:grothendieck_universe}.
\end{remark}

\begin{definition}\label{def:set_builder_notation}
  We introduce the following syntax for unrestricted comprehension, commonly called \term{set-builder notation}:
  \begin{equation*}
    \set{ x \given \varphi[\xi \mapsto x] }.
  \end{equation*}

  Note that we break our convention described in \fullref{rem:mathematical_logic_conventions/variable_symbols} for only using Greek letters for formula variable names. This is justified by \fullref{rem:first_order_theories_in_zfc} because we interpret \( x \) as an element of some model --- an axiom defining the set \( A \).

  As examples such as \fullref{thm:russels_paradox} show, unrestricted comprehension can easily lead to contradictions. If \( A \) is a set, the axiom of specification \ref{def:zfc/A6} tells us that \( \set{ x \given x \in A \wedge \varphi[\xi \mapsto x] } \) is a set within \hyperref[def:zfc]{ZFC}. We use the following syntax for this restricted comprehension:
  \begin{equation*}
    \set{ x \in A \given \varphi[\xi \mapsto x] }
  \end{equation*}

  Instead of using the delimiter \( \given \), we sometimes also use \( : \) or even omit the delimiter altogether and simply enumerate the members of the set:
  \begin{equation*}
    \set{ 1, 3, 9, 27 }.
  \end{equation*}

  Note that we enumerate certain numbers but only do so for illustrative purposes because the \hyperref[def:set_of_natural_numbers]{natural numbers} are not yet defined in terms of sets.

  We can also place an ellipsis if a certain pattern is obvious:
  \begin{equation*}
    \set{ 1, 3, 9, 27, \ldots }.
  \end{equation*}
\end{definition}

\begin{theorem}[Russell's paradox]\label{thm:russels_paradox}
  \hyperref[def:naive_set_theory]{Na\"ive set theory} is \hyperref[def:first_order_theory/consistent]{inconsistent}. More precisely, the instance of the \hyperref[def:naive_set_theory/unrestricted_comprehension]{schema of unrestricted comprehension} with
  \begin{equation}\label{eq:thm:russels_paradox_comprehension_formula}
    \varphi = (\xi \not\in \xi)
  \end{equation}
  allows us to derive \( \bot \) in \hyperref[def:classical_logic]{classical logic}.

  Thus the set
  \begin{equation}\label{eq:thm:russels_paradox_set}
    R \coloneqq \set{ x \given x \not\in x }
  \end{equation}
  of all sets that do not contain themselves is not well-defined. Indeed, from \( R \not\in R \) it follows that \( R \in R \) and from \( R \in R \) it follows that \( R \not\in R \).
\end{theorem}
\begin{proof}
  After substituting \eqref{eq:thm:russels_paradox_comprehension_formula} in \eqref{eq:def:naive_set_theory/unrestricted_comprehension}, we obtain the following axiom of na\"ive set theory:
  \begin{equation}\label{eq:thm:russels_paradox_comprehension_axiom}
    \psi \coloneqq \qexists \eta \qforall \zeta (\zeta \in \eta \leftrightarrow \neg (\zeta \in \zeta)).
  \end{equation}

  We will show that the negation \( \neg\psi \) of \( \psi \) is also derivable. An explicit form of the negation can be obtained by using \fullref{thm:first_order_quantifiers_are_dual} and \fullref{thm:boolean_equivalences/biconditional_negation}:
  \begin{equation*}
    \neg\psi = \qforall \eta \qexists \zeta (\zeta \in \eta \leftrightarrow \zeta \in \zeta).
  \end{equation*}

  But \( \neg\psi \) is a direct consequence of the extensionality axiom \eqref{eq:def:naive_set_theory/extensionality}, hence \( \neg\psi \) is indeed derivable in na\"ive set theory.

  Now \( \psi \) and \( \neg\psi \) are both derivable in the same theory. We use \eqref{eq:def:minimal_propositional_axiomatic_derivation_system/and_intro} to obtain that \( \psi \wedge \neg\psi \) is also derivable. By applying \eqref{eq:def:positive_implicational_propositional_derivation_system/rules/modus_ponens} with \eqref{eq:thm:minimal_propositional_negation_laws/lnc}, we can also derive \( \bot \).

  Therefore \( \bot \) is derivable in na\"ive set theory and hence it is inconsistent.
\end{proof}

\begin{remark}\label{rem:family_of_sets}
  In \hyperref[def:zfc]{ZFC}, everything is a set. However, it is often the case that we are not interested in how a set's elements are encoded as sets and only in how they behave, e.g. when working with \hyperref[def:set_of_natural_numbers]{natural numbers}, we are interested in the elements of \( \BbbN \) and not in the way every element of \( \BbbN \) is encoded as a set.

  In order to reduce repetitiveness, sets whose elements we consider to be other sets, are often called \term{families of sets}. In particular, if all (different) sets are \hyperref[def:subset]{disjoint}, we say that the family is a \term{disjoint family}. We usually assume that the sets are nonempty.
\end{remark}

\begin{definition}\label{def:subset}
  We say that \( A \) is a \term{subset} of \( B \) and write \( A \subseteq B \) if every member of \( A \) is a member of \( B \). If \( A \) is a subset of \( B \), we say that B is a \term{superset} of \( A \).

  If \( A \subseteq B \) and \( A \neq B \), we say that \( A \) is a \term{proper subset} of \( B \) and write \( A \subsetneq B \).

  The relation \( \subseteq \) is called the inclusion relation and it gives a partial ordering between sets. See \fullref{thm:boolean_algebra_of_subsets}. If an entire family of sets are not pairwise comparable, we say that they are \term{disjoint}.
\end{definition}

\begin{remark}\label{rem:subset_notation}
  Some authors, such as \cite{Kelley1955}, use the notation \( A \subseteq B \) to mean \enquote{all elements of \( A \) belong to \( B \)}, even in the case when \( A = B \). To avoid confusion, we use the notations \( A \subseteq B \) and \( A \subsetneq B \) (see \fullref{def:subset}).
\end{remark}

\begin{remark}\label{rem:singleton_sets}
  Sets with a single elements are usually called \term{singletons}. It is sometimes convenient, especially with connection to geometry or \hyperref[def:function/multivalued]{multivalued functions} (e.g. when dealing with \hyperref[def:net_convergence/limit]{limits of nets} or \hyperref[def:subdifferentials]{subdifferentials}), to not distinguish between singleton sets and their corresponding element.
\end{remark}

\begin{definition}\label{def:set_operations}
  We define the following operations:

  \begin{thmenum}
    \thmitem{def:set_operations/intersection} The \term{intersection} of a nonempty set \( \mscrA \) is
    \begin{equation*}
      \bigcap \mscrA \coloneqq \set{ x \given \qforall {A \in \mscrA} x \in A }.
    \end{equation*}

    We leave \( \bigcap \varnothing \) undefined because it should be a \hyperref[def:poset_extremal_points/top_and_bottom]{top element} in the \hyperref[thm:boolean_algebra_of_subsets]{Boolean algebra of all sets}, but the latter does not exist because of \fullref{thm:russels_paradox}.

    For two sets \( A \) and \( B \), we define the \term{binary intersection} as
    \begin{equation*}
      A \cap B \coloneqq \bigcap \set{ A, B } = \set{ x \given x \in A \T{and} x \in B }.
    \end{equation*}

    \thmitem{def:set_operations/union} Dually to \hyperref[def:set_intersection]{intersections}, the \term{union} of an arbitrary set \( \mscrA \) is defined as
    \begin{equation*}
      \bigcup A \coloneqq \set{ x \given \qexists {A \in \mscrA} x \in A }.
    \end{equation*}

    In particular, \( \bigcup \varnothing = \varnothing \).

    For two sets \( A \) and \( B \), we define the \term{binary union} as
    \begin{equation*}
      A \cup B \coloneqq \bigcup \{ A, B \} = \{ x \colon x \in A \T{or} x \in B \}.
    \end{equation*}

    \thmitem{def:set_operations/difference} The \term{difference} of the sets \( A \) and \( B \) is
    \begin{equation*}
      A \setminus B \coloneqq \set{ x \in A \given x \not\in B }.
    \end{equation*}

    \thmitem{def:set_operations/power_set} For any set \( A \), define its \term{power set} \( \pow(A) \) as the set of all its subsets. Symbolically,
    \begin{equation*}
      \pow(A) \coloneqq \set{ B \given B \subseteq A }.
    \end{equation*}
  \end{thmenum}
\end{definition}

\begin{proposition}\label{thm:set_difference_properties}
  \hyperref[def:set_operations/difference]{Set difference} has the following basic properties:
  \begin{thmenum}
    \thmitem{thm:set_difference_properties/intersection} If \( A \) and \( B \) are subsets of \( C \), then \( A \setminus B = A \cap (C \setminus B) \).

    \thmitem{thm:set_difference_properties/double_difference} If \( A \subseteq B \), then \( B \setminus (B \setminus A) = A \)
  \end{thmenum}
\end{proposition}
\begin{proof}
  \SubProofOf{thm:set_difference_properties/intersection} Since \( a \in A \) implies \( a \in C \), we have
  \begin{align*}
    A \setminus B
    &=
    \set{ x \in A \given x \not\in B }
    = \\ &=
    \set{ x \in A \given x \in C \T{and} x \not\in B }
    = \\ &=
    A \cap (C \setminus B).
  \end{align*}

  \SubProofOf{thm:set_difference_properties/double_difference} By \hyperref[thm:minimal_propositional_negation_laws/dne]{double negation elimination},
  \begin{align*}
    B \setminus (B \setminus A)
    &=
    \set[\Big]{ x \in B \given x \not\in \set{ x \in B \given x \not\in A } }
    = \\ &=
    \set{ x \in B \given x \in A }
    = \\ &=
    A.
  \end{align*}
\end{proof}

\begin{proposition}\label{thm:boolean_algebra_of_subsets}
  Let \( X \) be an arbitrary set. Then the \hyperref[def:set_operations/power_set]{power set} \( \pow(A) \) endowed with the \hyperref[def:subset]{inclusion} partial order \( \subseteq \) is a \hyperref[def:lattice/complete]{complete} \hyperref[def:boolean_algebra]{Boolean algebra}. Explicitly:

  \begin{thmenum}
    \thmitem{thm:boolean_algebra_of_subsets/join} The \hyperref[def:semilattice/join]{join} of an arbitrary family \( \mscrA \) of subsets of \( X \) is simply the \hyperref[def:set_operations/union]{union} \( \bigcap \mscrA \).

    \thmitem{thm:boolean_algebra_of_subsets/top} The \hyperref[def:poset_extremal_points/top_and_bottom]{top element} is the set \( X \) itself.

    \thmitem{thm:boolean_algebra_of_subsets/meet} The \hyperref[def:semilattice/meet]{meet} of an arbitrary family \( \mscrA \) of sets is simply the \hyperref[def:set_operations/intersection]{intersection} \( \bigcup \mscrA \). Unlike for a general family of sets, we have no problem defining the intersection of an empty set to be the top element \( X \).

    \thmitem{thm:boolean_algebra_of_subsets/bottom} The \hyperref[def:poset_extremal_points/top_and_bottom]{bottom element} is the empty set.

    \thmitem{thm:boolean_algebra_of_subsets/complement} The \hyperref[def:boolean_algebra]{complement} \( A^\complement \) of the subset \( A \) is the \hyperref[def:set_operations/difference]{difference} \( X \setminus A \).
  \end{thmenum}
\end{proposition}
\begin{proof}
  The statements \fullref{thm:boolean_algebra_of_subsets/join,thm:boolean_algebra_of_subsets/top,thm:boolean_algebra_of_subsets/meet,thm:boolean_algebra_of_subsets/bottom} are trivial and \fullref{thm:boolean_algebra_of_subsets/complement} follows from \fullref{thm:set_difference_properties/double_difference}.
\end{proof}

\begin{remark}\label{rem:binary_vs_arbitrary_tuples}
  We give two pairs of definitions for tuples and Cartesian products. The first pair, \fullref{def:binary_cartesian_product}, is quite restricted and is mostly necessary for defining \hyperref[def:function]{functions} and ensuring that everything along the way is indeed a set. The second pair of definitions, given in \fullref{def:cartesian_product}, can then be used for arbitrary (binary and nonbinary) products.
\end{remark}

\begin{definition}\label{def:binary_cartesian_product}\mcite[def. 1.23]{OpenLogicFull}
  The \term{ordered pair} or \term{Kuratowski pair} \( (a, b) \) of the sets \( a \) and \( b \) is
  \begin{equation*}
    (a, b) \coloneqq \set{ \set{ a }, \set{ a, b } }.
  \end{equation*}

  This is a simple accepted definition that encodes the order of \( a \) and \( b \), unlike the set \( \set{ a, b } \) for example.

  The \term{binary Cartesian product} of the sets \( A \) and \( B \) is
  \begin{equation*}
    A \times B \coloneqq \set{ (a, b) \given a \in A \T{and} b \in B }.
  \end{equation*}
\end{definition}
