\subsection{Na\"ive set theory}\label{subsec:naive_set_theory}

Na\"ive set theory is traditionally defined informally by only specifying that a set is an unordered collection of objects without repetition. It turns out that this can easily be formalized as a \hyperref[def:first_order_theory]{first-order theory}, albeit an inconsistent one. Still, this theory is useful for introducing important concepts that can ease the introduction of more elaborate theories like \hyperref[def:zfc]{\logic{ZFC}}. The definitions we introduce and the proofs we provide will turn out to be valid in \logic{ZFC} also. In other words, we will transparently utilize \hyperref[def:naive_set_theory/unrestricted_comprehension]{unrestricted comprehension} for constructing sets and later in \fullref{thm:zfc_existence_theorems} we will prove that they exist not only in na\"ive set theory but also in \logic{ZFC}.

\begin{remark}\label{rem:pure_set_theory}
  What we lose in this formalization are objects which are not sets, usually called \term{atoms} or \term{urelements} (because of the German prefix \enquote{ur}, meaning primordial). It is not necessary for us to add such elements since we can encode everything via sets. Theories without atoms, like our versions of nai\"ve set theory and \hyperref[def:axiom_of_universes]{\logic{ZFC+U}}, are called \term{pure set theories}.
\end{remark}

\begin{definition}\label{def:naive_set_theory}
  The \term{language of na\"ive set theory} is a \hyperref[def:first_order_syntax]{first-order language} \( \mscrL \) with only a single \hyperref[rem:first_order_formula_conventions/infix]{infix} binary relation \( \in \) called \term{set membership}. If \( \xi \in \eta \), we say that \( \xi \) is a \term{member} or \term{element} of \( \eta \) and that \( \eta \) \term{contains} \( \xi \).

  For the sake of simplicity, we will not introduce into the language any other functional or predicate symbols but will use \hyperref[rem:predicate_formula]{predicate formulas} when needed mostly for formulating axioms. See the \( \ref{eq:def:grothendieck_universe/predicate}[\tau] \) predicate for an extreme example.

  \term{Na\"ive set theory} is a \hyperref[def:first_order_theory]{first-order theory} axiomatized by the following:
  \begin{thmenum}
    \thmitem{def:naive_set_theory/extensionality}\mcite[sec. 61.1]{OpenLogicFull} The \term{axiom of extensionality}, which states that two sets are equal if and only if they have the same members. Symbolically,
    \begin{equation}\label{eq:def:naive_set_theory/extensionality}
      \parens[\Big]{ \qforall \xi (\xi \in \tau \leftrightarrow \xi \in \sigma) } \rightarrow \parens[\Big]{ \tau \doteq \sigma }.
    \end{equation}

    As a consequence, a set is only distinguished by what it contains and thus the ordering and repetition of members of a set play no role. This axiom is also important in \logic{ZFC} --- see \fullref{def:zfc/extensionality}.

    It is very common when dealing with sets, as in \eqref{eq:def:naive_set_theory/extensionality}, to use \hyperref[rem:first_order_formula_conventions/relativization]{relativization of quantifiers} with \( \in \).

    As explained in \fullref{rem:mathematical_logic_conventions/quantification}, we avoid excessive universal quantification. We actually add as an axiom of the theory the \hyperref[thm:implicit_universal_quantification]{universal closure} of \eqref{eq:def:naive_set_theory/extensionality}:
    \begin{equation}\label{eq:def:naive_set_theory/extensionality_quantified}
      \qforall \tau \qforall \sigma \parens[\Bigg]{ \parens[\Big]{ \qforall \xi (\xi \in \tau \leftrightarrow \xi \in \sigma) } \rightarrow \parens[\Big]{ \tau \doteq \sigma } }.
    \end{equation}

    The \hyperref[def:material_implication/converse]{converse} of \eqref{eq:def:naive_set_theory/extensionality} obvious.

    \thmitem{def:naive_set_theory/unrestricted_comprehension} The \term{axiom schema of unrestricted comprehension} states that any formula defines a set. For each formula \( \varphi \) not containing \( \tau \) as a free variable, the following is an axiom:
    \begin{equation}\label{eq:def:naive_set_theory/unrestricted_comprehension}
      \qexists \tau \qforall \xi (\xi \in \tau \leftrightarrow \varphi).
    \end{equation}

    It is important to highlight that \( \varphi \) may have any number of free variables as long as \( \tau \) is not among them. Of course, this axiom is only interesting if \( \xi \in \boldop{Free}(\varphi) \). If \( \eta_1, \ldots, \eta_n \) are all the other free variables of \( \varphi \), then the \hyperref[thm:implicit_universal_quantification]{universal closure} of the corresponding axiom is
    \begin{equation}\label{eq:def:naive_set_theory/unrestricted_comprehension_quantified}
      \qforall {\eta_1} \cdots \qforall {\eta_n} \qexists \tau \qforall \xi (\xi \in \tau \leftrightarrow \varphi).
    \end{equation}

    In other words, the set \( \tau \) is not unique in general but depends on the free variables \( \eta_1, \ldots, \eta_n \). For this reason, they are called \term{parameters} of the axiom.

    Compare this axiom schema to \hyperref[def:zfc/specification]{restricted comprehension}. In the context of na\"ive set theory they are equivalent because each is a special case of the other one.

    Because our goal is for all our constructions to be valid in \hyperref[def:zfc]{\logic{ZFC}}, we only use unrestricted comprehension where the existence of the set is justified by other axioms of \logic{ZFC}.
  \end{thmenum}
\end{definition}

\begin{remark}\label{rem:epsilon_and_set_membership}
  The symbol \( \in \) is derived from \( \varepsilon \). Some older books like \cite{Kelley1955} even use \( \varepsilon \) for set membership. \Fullref{def:epsilon_induction} is named after set membership.
\end{remark}

\begin{definition}\label{def:set}
  Assume that we have a fixed \hyperref[rem:standard_model_of_set_theory]{standard} \hyperref[rem:transitive_model_of_set_theory]{transitive} model \( \mscrV = (V, I) \) of \hyperref[def:naive_set_theory]{na\"ive set theory} or \hyperref[def:zfc]{\logic{ZFC}}, with or without the \hyperref[def:axiom_of_universes]{axiom of universes}. We will assume \logic{ZFC+U} by default.

  We say that any member of \( V \) is a \term{set}. If \( A \) is a set and \( x \in A \), we say that \( x \) is a \term{member} or \term{element} of \( A \) or, in accordance with \fullref{def:point}, a \term{point} in \( A \).

  We usually refer to \( V \) as our \term{universe} or \term{universal set}. When working with \hyperref[def:grothendieck_universe]{Grothendieck universes}, we may wish to further distinguish between the universal set and some Grothendieck universe. Fortunately, we very rarely refer to \( V \) itself.
\end{definition}

\begin{remark}\label{rem:standard_model_of_set_theory}
  We will say that a model \( \mscrV = (V, I) \) of set theory is \term{standard} if the interpretation \( I(\in) \) of the membership predicate symbol is precisely the membership relation in the metatheory. We will only consider standard models of set theory. This is immensely important for the following reasons:

  \begin{itemize}
    \thmitem{rem:standard_model_of_set_theory/set_builder_notation} Set-builder notation relies on constructing sets in the metatheory and then using them in the object theory. If the model is not standard, then it does not hold that \( (\xi \in \eta)\Bracks{\xi \mapsto x, \eta \mapsto y} = T \) if and only if \( x \in y \).

    \thmitem{rem:standard_model_of_set_theory/skolems_paradox} It is possible that cardinality is incompatible between the object theory and metatheory --- see \fullref{ex:skolems_paradox}. We want to avoid countable sets in the metatheory to be uncountable in the object theory, for example.
  \end{itemize}

  We will therefore assume that all our models of set theory are standard. We also want their domains to be transitive sets -- see \fullref{rem:transitive_model_of_set_theory}.
\end{remark}

\begin{definition}\label{def:set_builder_notation}
  As mentioned in \fullref{rem:set_definition_recursion}, set theory somewhat blurs the line between logic and metalogic. In particular, some \hyperref[def:first_order_definability]{definable} subsets of the universe \( U \) of the fixed model \( \mscrU \) are themselves sets within the object logic.

  Fix a formula \( \varphi \) whose free variables are \( \xi \) and \( \eta_1, \ldots, \eta_n \). In the simplest case, \( n = 0 \) and \( \xi \) is the only free variable of \( \varphi \).

  Also fix an \( n \)-tuple \( u_1, \ldots, u_n \) of members of \( U \), which we will call \term{parameters}. Denote by \( A \) the subset of \( U \) consisting of members \( x \) of \( U \) such that \( \varphi\Bracks{x, u_1, \ldots, u_n} = T \).

  We introduce a special convenience notation for \( A \) called \term{set-builder notation}:
  \begin{equation*}
    A \coloneqq \set{ x \given \varphi\Bracks{x, u_1, \ldots, u_n} }.
  \end{equation*}

  Since set-builder notation is metalogical, we do not impose strict syntax rules and use prose where it is straightforward to translate it into a logical formula.

  For example, the \hyperref[def:basic_set_operations/intersection]{intersection} of the sets \( B \) and \( C \) is given by the formula \( \xi \in \eta \wedge \xi \in \zeta \), where \( B \) is a value for the parameter \( \eta \) and \( C \) is a parameter for \( \zeta \). The intersection can thus be written as
  \begin{equation*}
    A \coloneqq \set{ x \given x \in B \T{and} x \in C }.
  \end{equation*}

  Note that at this point \( A \) is a set within the metatheory and its members are sets within the object logic, however \( A \) may not be a set within the object logic and its members may not be sets within the metatheory.

  Nonetheless, within na\"ive set theory, as a consequence of the \hyperref[def:naive_set_theory/unrestricted_comprehension]{axiom schema of unrestricted comprehension}, \( A \) is also a set within the object logic. More precisely, given our choice of parameters \( u_1, \ldots, u_n \), the axiom schema instance \eqref{eq:def:naive_set_theory/unrestricted_comprehension_quantified} guarantees the existence of a member \( a \) of \( U \), such that
  \begin{equation*}
    \parens{ \xi \in \tau }\Bracks{ \tau \mapsto a, \xi \mapsto x } = T
    \T{if and only if}
    x \isinE A,
  \end{equation*}
  where we have denoted set membership within the object logic by \( \in \) and within the metatheory by \( \isinE \). We will not further use this symbol and the two membership relations will be used interchangeably.

  This is where the line between logic and metalogic blurs --- we can speak about roughly the same sets within the object logic and the metatheory.

  Examples such as \fullref{thm:russels_paradox} show that unrestricted comprehension can easily lead to an inconsistent object logic. In more elaborate set theories like \hyperref[def:zfc]{\logic{ZFC}}, we only allow restricted comprehension via the \hyperref[def:zfc/specification]{axiom schema of specification}. Instead of defining \( A \) as a set of all members of \( U \) satisfying a certain property, restricted comprehension allows us to define \( A \) as a subset not of the universe \( U \) but of some well-behaved subset \( B \) of \( U \). The corresponding notation is
  \begin{equation*}
    \set{ x \in B \given \varphi\Bracks{x, u_1, \ldots, u_n} }.
  \end{equation*}

  Of course, we may still use unrestricted comprehension of the result is guaranteed to be a set within the object logic.

  Within \logic{ZFC}, subsets of \( U \) which are not sets in the object logic are called \term{proper classes}. Sets and proper classes are collectively called \term{classes}. We avoid referencing proper classes because that can easily lead us to an inconsistent theory. See \fullref{def:large_and_small_sets} for a clever workaround.

  A bigger problem that may happen is described in \fullref{rem:transitive_model_of_set_theory}.

  Other liberties regarding set-builder notation include the following:
  \begin{itemize}
    \item We often place arbitrary terms on the left side rather than only sets. This is simply a convenient metalogical notation; the symbols that are used in these terms are often not part of the object language. For example, we write the odd integers as
    \begin{equation*}
      \set{ 2n + 1 \given n \in \BbbZ }.
    \end{equation*}

    \item Instead of using the delimiter \( \given \), we sometimes also use \( : \), especially when dealing with absolute values and norms:
    \begin{equation*}
      \set{ \abs{x} : \abs{x}^2 < 1 }
    \end{equation*}
    can be more readable than
    \begin{equation*}
      \set{ \abs{x} \given \abs{x}^2 < 1 }.
    \end{equation*}

    \item If a set has only a small finite amount of members, we usually prefer to enumerate them as
    \begin{equation*}
      \set{ 1, 3, 9, 27 }.
    \end{equation*}

    Because of the \hyperref[def:naive_set_theory/extensionality]{axiom of extensionality}, the order and repetition of the objects inside the curly braces are irrelevant. Nevertheless, using any unconventional order does not benefit us in any way.

    \item We can also place an ellipsis if a certain pattern is obvious:
    \begin{equation*}
      \set{ 1, 3, 9, 27, \ldots }.
    \end{equation*}

    This works specifically for defining countable sets.
  \end{itemize}

  Note that we have used certain numbers but this was only for illustrative purposes because even the \hyperref[def:set_of_natural_numbers]{natural numbers} are not yet defined in terms of sets.
\end{definition}

\begin{remark}\label{rem:multile_set_membership_shorthand}
  Within the metatheory, we often use the notation \( x_1, \ldots, x_n \in A \) to mean that \( x_k \in A \) for \( k \in 1, \ldots, n \).
\end{remark}

\begin{remark}\label{rem:singleton_sets}
  Sets with a single elements are usually called \term{singletons}. It is sometimes convenient, especially with connection to geometry or \hyperref[def:multi_valued_function]{multi-valued functions} (e.g. when dealing with \hyperref[def:net_convergence/limit]{limits of nets} or \hyperref[def:subdifferentials]{subdifferentials}), to not distinguish between singleton sets and their corresponding element.
\end{remark}

\begin{definition}\label{def:empty_set}
  A very important set is the \term{empty set}
  \begin{equation*}
    \varnothing \coloneqq \set{ x \given \bot },
  \end{equation*}
  which contains no elements. We will also find useful the \hyperref[rem:predicate_formula]{predicate formula}
  \begin{equation*}\taglabel[\op{IsEmpty}]{eq:def:empty_set/predicate}
    \ref{eq:def:empty_set/predicate}[\tau] \coloneqq \qforall \eta \neg \eta \in \tau.
  \end{equation*}

  We will often refer to \term{nonempty sets}, which are exactly what they sound --- sets that are not the empty set.
\end{definition}

\begin{theorem}[Russell's paradox]\label{thm:russels_paradox}
  \hyperref[def:naive_set_theory]{Na\"ive set theory} is \hyperref[def:first_order_theory_consistency]{inconsistent}. More precisely, the instance of the \hyperref[def:naive_set_theory/unrestricted_comprehension]{schema of unrestricted comprehension} with
  \begin{equation}\label{eq:thm:russels_paradox_comprehension_formula}
    \varphi = (\xi \not\in \xi)
  \end{equation}
  allows us to derive \( \bot \) in \hyperref[def:classical_logic]{classical logic}.

  Thus the set
  \begin{equation}\label{eq:thm:russels_paradox_set}
    R \coloneqq \set{ x \given x \not\in x }
  \end{equation}
  of all sets that do not contain themselves is not well-defined. Indeed, from \( R \not\in R \) it follows that \( R \in R \) and from \( R \in R \) it follows that \( R \not\in R \).
\end{theorem}
\begin{proof}
  After substituting \eqref{eq:thm:russels_paradox_comprehension_formula} in \eqref{eq:def:naive_set_theory/unrestricted_comprehension}, we obtain the following axiom of na\"ive set theory:
  \begin{equation}\label{eq:thm:russels_paradox_comprehension_axiom}
    \psi \coloneqq \qexists \tau \qforall \xi (\xi \in \tau \leftrightarrow \neg (\xi \in \xi)).
  \end{equation}

  We will show that the negation \( \neg\psi \) of \( \psi \) is also derivable in this theory. An explicit form of the negation can be obtained by utilizing the equivalences \fullref{thm:first_order_quantifiers_are_dual} and \fullref{thm:boolean_equivalences/biconditional_negation}:
  \begin{equation*}
    \neg\psi = \qforall \tau \qexists \xi (\xi \in \tau \leftrightarrow \xi \in \xi).
  \end{equation*}

  This holds when \( \xi \) and \( \tau \) take on the same value, hence it is satisfiable in na\"ive set theory. By \fullref{thm:classical_first_order_logic_is_sound_and_complete}, \( \neg\psi \) is derivable in the theory.

  Thus \( \psi \) and \( \neg\psi \) are both derivable in the same theory and we can use \eqref{eq:thm:minimal_natural_deduction/neg/elim} to also derive \( \bot \), which shows that na\"ive set theory is inconsistent.
\end{proof}

\begin{definition}\label{def:subset}
  We say that \( A \) is a \term{subset} of \( B \) and write \( A \subseteq B \) if every member of \( A \) is a member of \( B \). If \( A \) is a subset of \( B \), we say that B is a \term{superset} of \( A \).

  If \( A \subseteq B \) and \( A \neq B \), we say that \( A \) is a \term{proper subset} of \( B \) and write \( A \subsetneq B \).

  The relation \( \subseteq \) is called the inclusion relation and it gives a partial ordering between sets. See \fullref{thm:boolean_algebra_of_subsets}. If an entire family of sets are not pairwise comparable, we say that they are \term{disjoint}.

  The following \hyperref[rem:predicate_formula]{predicate formula}
  \begin{equation*}\taglabel[\op{IsSubset}]{eq:def:subset/predicate}
    \ref{eq:def:subset/predicate}[\rho, \tau] \coloneqq \qforall \xi (\xi \in \rho \rightarrow \xi \in \tau),
  \end{equation*}
  which is valid when \( \rho \) is a subset of \( \tau \), will occasionally be useful for us.
\end{definition}

\begin{remark}\label{rem:subset_notation}
  Some authors, such as \cite{Kelley1955}, use the notation \( A \subset B \) to mean \enquote{all elements of \( A \) belong to \( B \)}, even in the case when \( A = B \). To avoid confusion, we use the notations \( A \subseteq B \) and \( A \subsetneq B \) (see \fullref{def:subset}).
\end{remark}

\begin{remark}\label{rem:family_of_sets}
  In a \hyperref[rem:pure_set_theory]{pure set theory}, everything is encoded as a set. However, it is often the case that we are not interested in how a set's elements are encoded as sets and only in how they behave, e.g. when working with \hyperref[def:set_of_natural_numbers]{natural numbers}, we are interested in the elements of \( \BbbN \) and not in the way every element of \( \BbbN \) is encoded as a set.

  In order to reduce repetitiveness, sets whose elements we consider to be other sets, are often called \term{families of sets}. In particular, if all (different) sets are \hyperref[def:subset]{disjoint}, we say that the family is a \term{disjoint family}. We usually assume that the sets are nonempty.

  We often consider \hyperref[def:indexed_family]{indexed families}, i.e. sets which depend on a parameter, which further highlight our intention to distinguish between a point in a set, the set itself and some family of sets to which the latter belongs.
\end{remark}

\begin{definition}\label{def:basic_set_operations}
  We define the following operations:

  \begin{thmenum}
    \thmitem{def:basic_set_operations/intersection} The \term{intersection} of a nonempty set \( \mscrA \) is
    \begin{equation*}
      \bigcap \mscrA \coloneqq \set{ x \given \qforall {A \in \mscrA} x \in A }.
    \end{equation*}

    We also introduce the \hyperref[rem:predicate_formula]{predicate formula}
    \begin{equation*}\taglabel[\op{IsIntersection}]{eq:def:basic_set_operations/intersection/predicate}
      \ref{eq:def:basic_set_operations/intersection/predicate}[\rho, \tau] \coloneqq \qforall \xi \parens[\Big]{ \xi \in \rho \leftrightarrow \qforall {\eta \in \tau} \xi \in \eta }.
    \end{equation*}

    We leave \( \bigcap \varnothing \) undefined because it should be a \hyperref[def:partially_ordered_set_extremal_points/top_and_bottom]{top element} in the \hyperref[thm:boolean_algebra_of_subsets]{Boolean algebra of all sets}, but the latter object is an ambiguous object and does not even exist in \logic{ZFC} --- see \fullref{thm:zfc_existence_theorems/universe}. It does nonetheless satisfy \( \ref{eq:def:basic_set_operations/intersection/predicate}[\rho, \tau] \).

    For two sets \( A \) and \( B \), we define the \term{binary intersection} as
    \begin{equation*}
      A \cap B \coloneqq \bigcap \set{ A, B } = \set{ x \given x \in A \T{and} x \in B }.
    \end{equation*}

    \thmitem{def:basic_set_operations/union} Dually to \hyperref[def:basic_set_operations/intersection]{intersections}, the \term{union} of an arbitrary set \( \mscrA \) is defined as
    \begin{equation*}
      \bigcup A \coloneqq \set{ x \given \qexists {A \in \mscrA} x \in A }.
    \end{equation*}

    We define the \hyperref[rem:predicate_formula]{predicate formula}
    \begin{equation*}\taglabel[\op{IsUnion}]{eq:def:basic_set_operations/union/predicate}
      \ref{eq:def:basic_set_operations/union/predicate}[\rho, \tau] \coloneqq \qforall \xi \parens[\Big]{ \xi \in \rho \leftrightarrow \qexists {\eta \in \tau} \xi \in \rho }.
    \end{equation*}

    In particular, \( \bigcup \varnothing = \varnothing \).

    For two sets \( A \) and \( B \), we define the \term{binary union} as
    \begin{equation*}
      A \cup B \coloneqq \bigcup \set{ A, B } = \set{ x \given x \in A \T{or} x \in B }.
    \end{equation*}

    \thmitem{def:basic_set_operations/difference} The \term{difference} of the sets \( A \) and \( B \) is
    \begin{equation*}
      A \setminus B \coloneqq \set{ x \in A \given x \not\in B }.
    \end{equation*}

    We define the \hyperref[rem:predicate_formula]{predicate formula}
    \begin{equation*}\taglabel[\op{IsDifference}]{eq:def:basic_set_operations/difference/predicate}
      \ref{eq:def:basic_set_operations/difference/predicate}[\rho, \tau, \sigma] \coloneqq \qforall \xi \parens[\Big]{ \xi \in \rho \leftrightarrow \xi \in \tau \wedge \neg(\xi \in \sigma) }.
    \end{equation*}

    \thmitem{def:basic_set_operations/power_set} The \term{power set} \( \pow(A) \) of \( A \) is the family of all subsets of \( A \). Symbolically,
    \begin{equation*}
      \pow(A) \coloneqq \set{ B \given B \subseteq A }.
    \end{equation*}

    We define the \hyperref[rem:predicate_formula]{predicate formula}
    \begin{equation*}\taglabel[\op{IsPowerSet}]{eq:def:basic_set_operations/power_set/predicate}
      \ref{eq:def:basic_set_operations/powet_set/predicate}[\rho, \tau] \coloneqq \qforall \xi \parens[\Big]{ \xi \in \rho \leftrightarrow \ref{eq:def:subset/predicate}[\xi, \tau] }.
    \end{equation*}
  \end{thmenum}
\end{definition}

\begin{proposition}\label{thm:set_difference_properties}
  \hyperref[def:basic_set_operations/difference]{Set difference} has the following basic properties:
  \begin{thmenum}
    \thmitem{thm:set_difference_properties/intersection} If \( A \) and \( B \) are subsets of \( C \), then \( A \setminus B = A \cap (C \setminus B) \).

    \thmitem{thm:set_difference_properties/double_difference} If \( A \subseteq B \), then \( B \setminus (B \setminus A) = A \)
  \end{thmenum}
\end{proposition}
\begin{proof}
  \SubProofOf{thm:set_difference_properties/intersection} Since \( a \in A \) implies \( a \in C \), we have
  \begin{align*}
    A \setminus B
    &=
    \set{ x \in A \given x \not\in B }
    = \\ &=
    \set{ x \in A \given x \in C \T{and} x \not\in B }
    = \\ &=
    A \cap (C \setminus B).
  \end{align*}

  \SubProofOf{thm:set_difference_properties/double_difference} By \hyperref[thm:minimal_propositional_negation_laws/dne]{double negation elimination},
  \begin{align*}
    B \setminus (B \setminus A)
    &=
    \set[\Big]{ x \in B \given x \not\in \set{ x \in B \given x \not\in A } }
    = \\ &=
    \set{ x \in B \given x \in A }
    = \\ &=
    A.
  \end{align*}
\end{proof}

\begin{proposition}\label{thm:boolean_algebra_of_subsets}
  Let \( X \) be an arbitrary set. Then the \hyperref[def:basic_set_operations/power_set]{power set} \( \pow(X) \) endowed with the \hyperref[def:subset]{inclusion} partial order \( \subseteq \) is a \hyperref[def:semilattice/complete]{complete} \hyperref[def:boolean_algebra]{Boolean algebra}. Explicitly:

  \begin{thmenum}
    \thmitem{thm:boolean_algebra_of_subsets/join} The \hyperref[def:semilattice/join]{join} of an arbitrary family \( \mscrA \) of subsets of \( X \) is simply the \hyperref[def:basic_set_operations/union]{union} \( \bigcap \mscrA \).

    \thmitem{thm:boolean_algebra_of_subsets/top} The \hyperref[def:partially_ordered_set_extremal_points/top_and_bottom]{top element} is the set \( X \) itself.

    \thmitem{thm:boolean_algebra_of_subsets/meet} The \hyperref[def:semilattice/meet]{meet} of an arbitrary family \( \mscrA \) of sets is simply the \hyperref[def:basic_set_operations/intersection]{intersection} \( \bigcup \mscrA \). Unlike for a general family of sets, we have no problem defining the intersection of an empty set to be the top element \( X \).

    \thmitem{thm:boolean_algebra_of_subsets/bottom} The \hyperref[def:partially_ordered_set_extremal_points/top_and_bottom]{bottom element} is the empty set.

    \thmitem{thm:boolean_algebra_of_subsets/complement} The \hyperref[def:boolean_algebra]{complement} \( A^\complement \) of the subset \( A \) is the \hyperref[def:basic_set_operations/difference]{difference} \( X \setminus A \).
  \end{thmenum}

  \begin{figure}
    \hfill
    \includegraphics{figures/thm__boolean_algebra_of_subsets.pdf}
    \hfill
    \hfill
    \caption{The \hyperref[def:hasse_diagram]{Hasse diagram} of \( \pow(\set{ A, B }) \) with respect to \hyperref[def:subset]{set inclusion}}
    \label{fig:thm:boolean_algebra_of_subsets}
  \end{figure}
\end{proposition}
\begin{proof}
  \SubProofOf{thm:boolean_algebra_of_subsets/join} The union of \( \mscrA \) exists by \fullref{thm:zfc_existence_theorems/arbitrary_union} and it is itself a subset of \( \mscrA \). Every set in \( \mscrA \) is contained in \( \bigcup \mscrA \), hence it is indeed a join.

  \SubProofOf{thm:boolean_algebra_of_subsets/top} Clearly \( X \) contains every subset of \( X \).

  \SubProofOf{thm:boolean_algebra_of_subsets/join} The intersection of \( \mscrA \) exists by \fullref{thm:zfc_existence_theorems/arbitrary_union} and it is itself a subset of \( \mscrA \). Every set in \( \mscrA \) contains \( \bigcap \mscrA \), hence it is indeed a meet.

  \SubProofOf{thm:boolean_algebra_of_subsets/bottom} The empty set is contained in every set, in particular in every subset of \( A \).

  \SubProofOf{thm:boolean_algebra_of_subsets/complement} The operation \( A^\complement \) is well defined for each subset \( A \) of \( X \) due to \fullref{thm:zfc_existence_theorems/difference}.

  By definition
  \begin{equation*}
    A \vee A^\complement
    =
    A \cup (X \setminus A)
    =
    X
  \end{equation*}
  and
  \begin{equation*}
    A \vee A^\complement
    =
    A \cup (X \setminus A)
    =
    X,
  \end{equation*}
  hence \( A^\complement \) is indeed the complement of \( A \).

  Therefore \( \pow(X) \) is a Boolean algebra.
\end{proof}

\begin{remark}\label{rem:binary_vs_arbitrary_tuples}
  We give two pairs of definitions for tuples and Cartesian products. The first pair, \fullref{def:binary_cartesian_product}, is quite restricted and is mostly necessary for defining \hyperref[def:function]{functions} and ensuring that everything along the way is indeed a set. The second pair of definitions, given in \fullref{def:cartesian_product}, can then be used for arbitrary (binary and nonbinary) products.
\end{remark}

\begin{definition}\label{def:binary_cartesian_product}\mcite[def. 1.23]{OpenLogicFull}
  The \term{Kuratowski pair} or simply \term{ordered pair} \( (x, y) \) of the sets \( x \) and \( y \) is defined as
  \begin{equation*}
    (x, y) \coloneqq \set{ \set{ x }, \set{ x, y } }.
  \end{equation*}

  This is a simple and widespread definition that encodes the order of \( x \) and \( y \), unlike the set \( \set{ x, y } \) for example.

  We will use the following \hyperref[rem:predicate_formula]{predicate formula} in \( \ref{eq:def:function/predicate}[\rho, \tau, \sigma] \):
  \begin{equation*}\taglabel[\op{IsPair}]{eq:def:binary_cartesian_product/pair_predicate}
    \ref{eq:def:binary_cartesian_product/pair_predicate}[\rho, \tau, \sigma] \coloneqq \qforall \xi \parens[\Bigg]{ \xi \in \rho \leftrightarrow \parens[\Big]{ \parens[\Big]{ \qforall {\eta \in \xi} \eta \doteq \tau } \vee \parens[\Big]{ \qforall {\eta \in \xi} (\eta \doteq \tau \vee \eta \doteq \sigma) } } }
  \end{equation*}

  The \term{binary Cartesian product} of the sets \( A \) and \( B \) is
  \begin{equation*}
    A \times B \coloneqq \set{ (x, y) \given x \in A \T{and} y \in B }.
  \end{equation*}
\end{definition}

\begin{remark}\label{rem:inductive_sets}
  Induction is an important proof technique that is discussed in detail in the proof of \fullref{thm:nonzero_natural_numbers_have_predecessors}. There are more general forms of induction than \eqref{eq:def:peano_arithmetic/PA3} like \fullref{thm:well_founded_induction} and \fullref{thm:well_founded_induction}. They do, however, require concepts which in turn depend on the existence of natural numbers within set theory. As a consequence, we cannot prove \eqref{eq:def:peano_arithmetic/PA3} via \fullref{thm:well_founded_induction}.

  We will introduce a the concept of inductive sets in \fullref{def:inductive_set} and prove in \fullref{thm:omega_induction} that a special inductive set \hyperref[thm:smallest_inductive_set_existence]{\( \omega \)}, which will be the domain of our model of \( \BbbN \), allows performing inductive proofs. The technique that allows us to perform inductive proofs on \( \omega \) can be seen in the proof of \fullref{thm:omega_is_transitive}. \Fullref{thm:omega_induction} will allow us to define natural numbers without relying on metalogical induction along the way. See the proof of \fullref{thm:omega_induction} for a description of natural number induction within set theory and \fullref{rem:standard_models_of_arithmetic} for a further discussion of the use of natural numbers in the metatheory and in the object logic.

  We also introduce \term{recursion} in parallel as a technique for constructing objects. See \fullref{thm:omega_recursion}.
\end{remark}

\begin{definition}\label{def:ordinal_successor}
  The \term{successor} \( \op{succ}(A) \) of a set \( A \) is the set
  \begin{equation*}
    \op{succ}(A) \coloneqq A \cup \set{ A }.
  \end{equation*}

  It is also called the \term{ordinal successor} operation since it is an important concept in the \hyperref[subsec:ordinal]{theory of ordinals}. See \fullref{rem:def:ordinal_successor} for an example of how it naturally arises.

  The following \hyperref[rem:predicate_formula]{predicate formula}
  \begin{equation*}\taglabel[\op{IsSucc}]{eq:def:ordinal_successor/predicate}
    \ref{eq:def:ordinal_successor/predicate}[\rho, \tau] \coloneqq \qforall \xi \parens[\Big]{ \xi \in \rho \leftrightarrow (\xi \in \tau \vee \xi = \tau) },
  \end{equation*}
  which states that \( \rho \) is the successor of \( \tau \), will be useful for us when when working with \hyperref[def:inductive_set]{inductive sets}.
\end{definition}

\begin{definition}\label{def:inductive_set}
  A set is called \term{inductive} if contains the empty set and is closed under the \hyperref[def:ordinal_successor]{successor operator}.

  We introduce the following \hyperref[rem:predicate_formula]{predicate formula}
  \begin{equation*}\taglabel[\op{IsInductive}]{eq:def:inductive_set/predicate}
    \ref{eq:def:inductive_set/predicate}[\tau] \coloneqq
      \parens[\Big]{ \qexists {\xi \in \tau} \ref{eq:def:empty_set/predicate}[\xi] }
      \wedge
      \parens[\Big]{ \qforall {\xi \in \tau} \qexists {\eta \in \tau} \ref{eq:def:ordinal_successor/predicate}[\eta, \xi] }.
  \end{equation*}
\end{definition}

\begin{proposition}\label{thm:smallest_inductive_set_existence}
  There is a smallest (with respect to set inclusion) \hyperref[def:inductive_set]{inductive set}, which we denote by \( \omega \).
\end{proposition}
\begin{proof}
  We cannot directly define \( \omega \) as the intersection of all inductive sets since we want to avoid unrestricted comprehension. Fortunately, the existence of at least one inductive set \( A \) is justified by the \hyperref[def:zfc/infinity]{axiom of infinity} in \logic{ZFC} or by taking the entire universe in na\"ive set theory.

  Hence we use restricted comprehension:
  \begin{equation*}
    \omega \coloneqq \set{ x \in A \given x \T{belongs to every inductive set} }.
  \end{equation*}

  To see that \( \omega \) is itself inductive, note that \( \varnothing \in \omega \) and that if \( x \in \omega \), then it also belongs to all inductive sets and hence \( \op{succ}(x) \) also belongs to all inductive sets, proving \( \op{succ}(x) \in \omega \).
\end{proof}

\begin{theorem}[Induction for inductive sets]\label{thm:omega_induction}
  We can perform induction on the \hyperref[thm:smallest_inductive_set_existence]{smallest inductive set \( \omega \)}. That is, we can prove that some property holds for every element of \( \omega \) by proving the following:
  \begin{itemize}
    \item The property holds for \( \varnothing \)
    \item We can prove that is holds for \( \op{succ}(n) \) by assuming that it holds for some set \( n \in \omega \).
  \end{itemize}

  This is an analog of \eqref{eq:def:peano_arithmetic/PA3} and is actually used in \fullref{thm:omega_is_model_of_pa} to prove that \( \omega \) is a model of \hyperref[def:peano_arithmetic]{\logic{PA}}. Instead of an entire theorem schema, however, for this theorem it is sufficient to use one single formula. The more general induction principles that use theorem schemas cannot be proved without natural numbers, which are a model of \logic{PA} by virtue of this theorem.

  More formally, the following is a theorem of both na\"ive set theory and \hyperref[def:zfc]{\logic{ZF}}:
  \footnotesize
  \begin{equation}\label{eq:thm:omega_induction}
    \qexists \sigma
    \qforall \tau
    \parens[\Bigg]
      {
        \parens[\Bigg]
          {
            \parens[\Big]
            {
              \underbrace{ \qexists {\xi \in \tau} \ref{eq:def:empty_set/predicate}[\xi] }_{\mathclap{\T{base case}}}
            }
            \wedge
            \qforall \xi \parens[\Big]
              {
                \overbrace
                  {
                    \underbrace{ \xi \in \tau }_{\mathclap{\substack{\T{inductive} \\ \T{hypothesis}}}}
                    \rightarrow
                    \underbrace
                      {
                        \qexists {\eta \in \tau} \ref{eq:def:ordinal_successor/predicate}[\eta, \xi]
                      }_{\mathclap{\substack{\T{inductive step} \\ \T{conclusion}}}}
                  }^{\T{inductive step}}
              }
          }
        \rightarrow
        \underbrace{ \ref{eq:def:subset/predicate}[\sigma, \tau] }_{\T{conclusion}}
      }
  \end{equation}
  \normalsize
\end{theorem}
\begin{proof}
  The antecedent of (the inner formula in) \eqref{eq:thm:omega_induction} is a restatement of the predicate formula \( \ref{eq:def:inductive_set/predicate}[\tau] \). The situation resembles the \hyperref[eq:def:zfc/infinity]{axiom of infinity} but, instead of existence of an inductive set \( \tau \), it states the existence of a set \( \sigma \) such that if \( \tau \) is an inductive set, then \( \sigma \) is a subset of \( \tau \) (if we restrict \( \xi \) to range only over members of \( \sigma \), then we would obtain equality of \( \tau \) and \( \sigma \) instead). In other words, we have reduced the verification of \eqref{eq:thm:omega_induction} to showing that there exists a minimal inductive set in both na\"ive set theory and \logic{ZF}.

  We have already proved in \fullref{thm:smallest_inductive_set_existence} that our fixed model \( \mscrV = (V, I) \) of set theory has a minimal inductive set \( \omega \). Thus for any variable assignment \( v: \boldop{Var} \to V \), the modified assignment \( v_{\sigma \mapsto \omega} \) satisfies \eqref{eq:thm:omega_induction} with the outer existential quantifier removed. Hence by \fullref{def:first_order_valuation/formula_valuation}, it follows that the entire formula \eqref{eq:thm:omega_induction} is satisfied by the assignment \( v \).

  Both the assignment \( v \) and the model \( \mscrV \) were arbitrary, therefore we can conclude that \eqref{eq:thm:omega_induction} is a theorem of both na\"ive set theory and \logic{ZF}.
\end{proof}

\begin{definition}\label{def:transitive_set}
  A set \( A \) is \term{transitive} if from \( B \in A \) it follows that \( B \subseteq A \).

  See \fullref{rem:ordinal_definition} for a discussion of the motivation and terminology of transitive sets and \fullref{rem:transitive_model_of_set_theory} for their importance.

  We introduce the following \hyperref[rem:predicate_formula]{predicate formula}:
  \begin{equation*}\taglabel[\op{IsSetTransitive}]{eq:def:transitive_set/predicate}
    \ref{eq:def:transitive_set/predicate}[\tau] \coloneqq \qforall {\xi \in \tau} \qforall {\eta \in \xi} {\eta \in \tau}
  \end{equation*}
\end{definition}

\begin{proposition}\label{thm:omega_is_transitive}
  The set \( \omega \) is transitive and every member of \( \omega \) is transitive.

  This proof demonstrates usage of \fullref{thm:omega_induction}.
\end{proposition}
\begin{proof}
  \SubProof{Proof that all members of \( \omega \) are transitive} Clearly \( \varnothing \) is transitive because every member of \( \varnothing \) vacuously is a subset of \( \varnothing \).

  Now suppose that \( n \) is transitive and let \( m \in \op{succ}(n) = n \cup \set{ n } \). If \( m = n \), then \( m \in \op{succ}(n) \) by definition of the successor operation. If \( m \in n \), then \( m \subseteq n \) by the inductive hypothesis and hence also \( m \subseteq n \cup \set{ n } = \op{succ}(n) \). Thus \( \op{succ}(n) \) is also transitive.

  Therefore every member of \( \omega \) is transitive.

  \SubProof{Proof that \( \omega \) is transitive} We will show that for all members \( n \) of \( \omega \) such that \( n \subseteq \omega \).

  The case \( n = \varnothing \) is again trivial.

  Now suppose that \( n \subseteq \omega \) and let \( m \in \op{succ}(n) \). If \( m = n \), clearly \( m \subseteq \omega \). If \( m \in n \), then \( m \subseteq n \) and, since \( n \subseteq \omega \), we have \( m \subseteq \omega \) by transitivity of \( \subseteq \).

  Therefore \( \omega \) is transitive.
\end{proof}

\begin{remark}\label{rem:transitive_model_of_set_theory}
  As discussed in \fullref{def:set_builder_notation}, within the \hyperref[def:naive_set_theory/unrestricted_comprehension]{axiom schema of unrestricted comprehension} it may happen that \( U \subseteq V \) is not a set within the object logic.

  But there is a bigger problem that may happen even for \hyperref[rem:standard_model_of_set_theory]{standard models}. If \( A \in V \) and \( x \in A \) (in the metatheory), it is possible that \( x \) is not in \( V \). Therefore, if we have shown that \( A \) is a set within the object logic, it is possible that its members within the metatheory are not members in the object logic. On other words, it is possible for set membership itself to be incompatible between the metatheory and object logic.

  If \( V \) is a transitive set, however, we would not have such a problem. That is, if we construct a set \( A \) in the metatheory and show that it belongs to some set \( B \) in the object logic, then \( A \) itself would also be a set in the object logic.

  For this reason, it is very important to consider only transitive models of set theory.
\end{remark}
