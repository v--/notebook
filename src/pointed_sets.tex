\section{Group theory}\label{sec:group_theory}

\begin{remark}\label{rem:algebraic_structures}
  Modern algebra takes its roots in abstracting \hyperref[def:integers]{integers} and \hyperref[def:set_of_real_numbers]{real numbers} and their addition and multiplication. Both of these operations are commutative and, if we want to generalize their properties, it is sensible to study commutative operations.

  Another type of objects that usually fits in the same algebraic framework are \hyperref[def:function]{functions} and their \hyperref[def:multi_valued_function/composition]{composition}. Functions from a set to itself can be composed to form another function of the same type, similarly to how two integers can be added to obtain another integer. The specificity is in the non-commutativity of function composition.

  This suggests that we use the same algebraic structures to study both generalizations of numbers and generalizations of functions over a set. The first case is commutative, the second is not. This is why commutative and non-commutative algebraic structures, even though they are similarly defined, can have very different properties and applications.

  We shall not attempt to give a precise definition for an \term{algebraic structure}. There are general frameworks that attempt to systematically study different algebraic structures simultaneously, however their complexity rarely justifies their shortcomings. We will instead build standard algebraic structures from \enquote{base building blocks}.
\end{remark}

\subsection{Pointed sets}\label{subsec:pointed_sets}

\begin{definition}\label{def:pointed_set}
  The simplest algebraic structure is a \term{pointed set}. It is simply a nonempty set \( \mscrX \) equipped with an distinguished element \( e \in X \). It is an algebraic structure because \( e \) can be regarded as the, sole value of a nullary function \( \odot: X^0 \to X \).

  We will call \( e \) the \term{origin} of \( \mscrX \) based on the terminology for \hyperref[def:euclidean_plane_coordinate_system/origin]{affine coordinate systems}.

  \begin{thmenum}
    \thmitem{def:pointed_set/theory} Pointed sets can also be viewed as \hyperref[def:first_order_semantics/satisfiability]{models} of an \hyperref[def:first_order_theory]{empty theory} for a \hyperref[def:first_order_language]{first-order logic language} consisting of:
    \begin{thmenum}
      \thmitem{def:theory_of_pointed_sets/eq} A formal equality symbol \( \doteq \).
      \thmitem{def:theory_of_pointed_sets/point} A constant, i.e. a nullary \hyperref[def:first_order_language/func]{functional symbol}.
    \end{thmenum}

    This theory is called the theory of pointed sets.

    \thmitem{def:pointed_set/homomorphism} A \hyperref[def:first_order_homomorphism]{homomorphism} (based on \fullref{def:pointed_set/theory}) between the pointed sets \( (\mscrX, e_{\mscrX}) \) and \( (\mscrY, e_{\mscrY}) \) is, explicitly, a function \( \varphi: \mscrX \to \mscrY \) that satisfies
    \begin{equation}\label{eq:def:pointed_set/homomorphism}
      \varphi(e_{\mscrX}) = e_{\mscrY}.
    \end{equation}

    \thmitem{def:pointed_set/submodel} The set \( S \subseteq X \) is a \hyperref[def:first_order_substructure]{substructure} if \( \mscrX \) if \( e \in S \).

    \thmitem{def:pointed_set/category} We denote the \hyperref[def:category_of_small_first_order_models]{category of models} for the theory of pointed sets by \( \cat{Set}_* \). We denote the \hyperref[def:category_of_small_first_order_models]{category of models} for the theory of pointed sets by \( \cat{Set}_* \).
  \end{thmenum}
\end{definition}
