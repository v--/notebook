\subsection{Lagrange polynomials}\label{subsec:lagrange_polynomials}

\begin{definition}\label{def:omega_polynomial}
  Given distinct elements \( x_0, \ldots, x_n \) of the field \( \BbbK \), we form the polynomial
  \begin{equation*}
    \omega(X) \coloneqq \prod_{k=0}^n (X - x_j).
  \end{equation*}
\end{definition}

\begin{proposition}\label{thm:omega_polynomial_derivative}
  For the polynomial \( \omega \) from \fullref{def:omega_polynomial}, for \( k = 0, \ldots, n \) we have
  \begin{equation*}
    \omega'(x_j) = \prod_{\substack{j = 0 \\ j \neq k}}^n (x_j - x_k),
  \end{equation*}
  where \( \omega' \) is the algebraic \hyperref[def:algebraic_derivative]{derivative} of \( \omega \).
\end{proposition}
\begin{proof}
  Fix \( k \in \{ 0, \ldots, n \} \) and denote
  \begin{equation*}
    q(X) \coloneqq \prod_{\substack{j = 0 \\ j \neq k}}^n (X - x_j).
  \end{equation*}

  Then
  \begin{equation*}
    \omega(X) = (X - x_k) q(X)
  \end{equation*}
  so
  \begin{equation*}
    \omega'(X) = [q(X) + X q'(X)] - x_k q'(X) = q(X) + (X - x_k) q'(X).
  \end{equation*}

  So for \( x_k \) we have
  \begin{equation*}
    \omega'(x_k) = q(x_k) = \prod_{\substack{j = 0 \\ j \neq k}}^n (x_k - x_j).
  \end{equation*}
\end{proof}

\begin{theorem}[Lagrange interpolation]\label{thm:lagrange_interpolation}
  Let \( x_0, x_1, \ldots, x_n \) be pairwise distinct elements of \( \BbbK \) and \( y_0, y_1, \ldots, y_n \) be arbitrary elements of \( \BbbK \). Then there exists a unique \hyperref[def:polynomial]{polynomial} \( L_n(X) \in \pi_n \) (where \( \pi_n \) is the \hyperref[def:polynomial_free_module]{\( n \)-dimensional polynomial vector space}) such that
  \begin{equation}\label{eq:thm:lagrange_interpolation/condition}
    L_n(x_k) = y_k, k = 0, 1, \ldots, n.
  \end{equation}
\end{theorem}
\begin{proof}
  We will first show uniqueness. Let \( p, q \in \pi_n \) both satisfy \fullref{eq:thm:lagrange_interpolation/condition}. Their difference \( p - q \) is a polynomial of degree at most \( n \) that has \( n + 1 \) roots. By \fullref{thm:integral_domain_polynomial_root_limit}, \( p - q = 0 \). This proves uniqueness.

  We construct the polynomial explicitly. Define the Lagrange polynomial
  \begin{equation*}
    L_n(X) = \sum_{m=0}^n y_m \prod_{\substack{j = 0 \\ j \neq m}}^n \frac {(X - x_j)} {(x_m - x_j)}.
  \end{equation*}

  For \( k = 0, 1, \ldots, n \) we have
  \begin{equation*}
    L_n(x_k) = y_k \underbrace{\prod_{\substack{j = 0 \\ j \neq k}}^n \frac {(x_k - x_j)} {(x_k - x_j)}}_{=1} + \sum_{\substack{m = 0 \\ m \neq k}}^n y_m \overbrace{\frac{(x_k - x_m)}{(x_k - x_m)}}^{=0} \prod_{\substack{j = 0 \\ j \neq k \\ j \neq m}}^n \frac {(x_k - x_j)} {(x_m - x_j)} = y_k.
  \end{equation*}

  Therefore, \( L_n \) satisfies \eqref{eq:thm:lagrange_interpolation/condition}, which proves existence.
\end{proof}
