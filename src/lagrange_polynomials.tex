\subsection{Lagrange polynomials}\label{subsec:lagrange_polynomials}

\begin{definition}\label{def:omega_polynomial}
  Given distinct elements \( x_0, \ldots, x_n \) of the field \( \BbbK \), we form the polynomial
  \begin{equation*}
    \omega(X) \coloneqq \prod_{k=0}^n (X - x_j).
  \end{equation*}
\end{definition}

\begin{proposition}\label{thm:omega_polynomial_derivative}
  For the polynomial \( \omega \) from \fullref{def:omega_polynomial}, for \( k = 0, \ldots, n \) we have
  \begin{equation*}
    \omega'(x_j) = \prod_{\substack{j = 0 \\ j \neq k}}^n (x_j - x_k),
  \end{equation*}
  where \( \omega' \) is the \hyperref[def:algebraic_derivative]{algebraic derivative} of \( \omega \).
\end{proposition}
\begin{proof}
  Fix \( k \in \{ 0, \ldots, n \} \) and denote
  \begin{equation*}
    q(X) \coloneqq \prod_{\substack{j = 0 \\ j \neq k}}^n (X - x_j).
  \end{equation*}

  Then
  \begin{equation*}
    \omega(X) = (X - x_k) q(X)
  \end{equation*}
  so
  \begin{equation*}
    \omega'(X) = [q(X) + X q'(X)] - x_k q'(X) = q(X) + (X - x_k) q'(X).
  \end{equation*}

  So for \( x_k \) we have
  \begin{equation*}
    \omega'(x_k) = q(x_k) = \prod_{\substack{j = 0 \\ j \neq k}}^n (x_k - x_j).
  \end{equation*}
\end{proof}

\begin{theorem}[Lagrange interpolation]\label{thm:lagrange_interpolation}
  Let \( x_0, x_1, \ldots, x_n \) be pairwise distinct elements of \( \BbbK \) and let \( y_0, y_1, \ldots, y_n \) be arbitrary elements of \( \BbbK \). Then there exists a unique \hyperref[def:polynomial_semiring]{polynomial} \( L(X) \) of degree at most \( n \) such that \( L(x_k) = y_k \) for \( k = 1, \ldots, n \).
\end{theorem}
\begin{proof}
  \SubProof{Proof of uniqueness} Suppose that \( p, q \) are polynomials of degree at most \( n \) that both satisfy \( L(x_k) = y_k \) for \( k = 1, \ldots, n \). Their difference \( p - q \) is a polynomial of degree at most \( n \) that has \( n + 1 \) roots. By \fullref{thm:def:integral_domain/root_limit}, \( p - q = 0 \).

  \SubProof{Proof of existence} We will construct the polynomial explicitly. Define the Lagrange basis polynomial
  \begin{equation*}
    L(X) = \sum_{m=0}^n y_m \prod_{\substack{j = 0 \\ j \neq m}}^n \frac {(X - x_j)} {(x_m - x_j)}.
  \end{equation*}

  For \( k = 0, 1, \ldots, n \) we have
  \begin{equation*}
    L(x_k) = y_k \underbrace{\prod_{\substack{j = 0 \\ j \neq k}}^n \frac {(x_k - x_j)} {(x_k - x_j)}}_{1} + \sum_{\substack{m = 0 \\ m \neq k}}^n y_m \overbrace{\frac{(x_k - x_m)}{(x_k - x_m)}}^{0} \prod_{\substack{j = 0 \\ j \neq k \\ j \neq m}}^n \frac {(x_k - x_j)} {(x_m - x_j)} = y_k.
  \end{equation*}

  Therefore, \( L \) is the desired polynomial.
\end{proof}

\begin{theorem}[Finite field Lagrange interpolation]\label{thm:finite_field_lagrange_interpolation}\mcite{MathOF:functions_over_finite_fields}
  For a multivariate function \( f: \BbbF_q^n \to \BbbF_q \) over the \hyperref[thm:finite_fields]{finite field} \( \BbbF_q \), there exists a unique multivariate polynomial \( L(X_1, \ldots, X_n) \) such that
  \begin{itemize}
    \item For any sequence of values \( x_1, \ldots, x_n \),
    \begin{equation*}
      f(x_1, \ldots, x_n) = L(x_1, \ldots, x_n).
    \end{equation*}

    \item For every monomial \( X_1^{\gamma_n} \cdots X_n^{\gamma_n} \) of \( L \), \( \gamma_i < q \) for \( i = 1, \ldots, n \).
  \end{itemize}
\end{theorem}
\begin{proof}
  For any point \( (x_1, \ldots, x_n) \in \BbbZ_q^n \), the characteristic polynomial
  \begin{equation*}
    c(X_1, \ldots, X_n) \coloneqq \prod_{i=0}^n \parens*{ \prod_{\substack{m=0 \\ m \neq x_i}}^{q - 1} \frac {X_i - m} {x_i - m} }
  \end{equation*}
  satisfies
  \begin{equation*}
    c(y_1, \ldots, y_n) = \begin{cases}
      1, &x_i = y_i \T{for all} i = 1, \ldots, n \\
      0, &\T{otherwise.}
    \end{cases}
  \end{equation*}

  As in \fullref{thm:lagrange_interpolation}, give us the desired polynomial, a linear combination of these basis polynomials with coefficients corresponding to the values of \( f \) give us the desired polynomial.
\end{proof}
