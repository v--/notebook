\subsection{Grothendieck universes}\label{subsec:grothendieck_universes}

Instead of having one single universe, we can have multiple universes where each is contained in another one. The upside of this is that we can do category theory formally within set theory. The downside of this is that, unlike models of \hyperref[def:zfc]{\logic{ZFC}}, models of \hyperref[def:axiom_of_universes]{\logic{ZFC+U}} (\logic{ZFC} with the axiom that every set is contained in some \hyperref[def:grothendieck_universe]{Grothendieck universe}) are much less studied. In particular, this axiom requires the existence of an unbounded hierarchy of \hyperref[def:regular_cardinal]{strongly inaccessible cardinal}, unlike \( \logic{ZFC} \) for which only one such cardinal is sufficient.

\begin{definition}\label{def:grothendieck_universe}\mcite{nLab:grothendieck_universe}
  We say that the set \( U \) is a \term{Grothendieck universe} if it satisfied the following conditions:
  \begin{thmenum}
    \thmitem[def:grothendieck_universe/inductive]{AU1} It contains an \hyperref[def:inductive_set]{inductive set}. We specifically mean the \hyperref[thm:smallest_inductive_set_existence]{smallest inductive set \( \omega \)}, however is sufficient only to state the existence of one inductive set.

    \thmitem[def:grothendieck_universe/transitive]{AU2} It is a \hyperref[def:transitive_set]{transitive set}.

    \thmitem[def:grothendieck_universe/power_set]{AU3} For any \( A \in U \), the \hyperref[def:basic_set_operations/power_set]{power set} \( \pow(A) \) also belongs to \( U \).

    \thmitem[def:grothendieck_universe/union]{AU4} For any member \( A \in U \) and any \( A \)-indexed family \( \set{ B_a }_{a \in A} \subseteq U \), the union
    \begin{equation*}
      \bigcup\set{ B_a \given a \in A }
    \end{equation*}
    belongs to \( U \). This is a restriction from unions over completely arbitrary families of sets to those families that can be indexed by members of \( A \).
  \end{thmenum}

  We formalize the entire concept via the following monstrous formula:
  \begin{equation*}\taglabel[\op{IsUniverse}]{eq:def:grothendieck_universe/predicate}
    \begin{aligned}
      \mathrlap
      {
        \ref{eq:def:grothendieck_universe/predicate}[\upsilon] \coloneqq \parens[\Big]
          {
            \qexists {\tau \in \upsilon}
            \ref{eq:def:inductive_set/predicate}[\tau]
          }
        \wedge
      }
      \hspace{1cm}&
      \\ &\wedge
      \parens[\Big]
      {
        \qforall {\tau \in \upsilon}
        \ref{eq:def:subset/predicate}[\tau, \upsilon]
      }
      \thinspace \wedge \\ &\wedge
      \parens[\Big]
      {
        \qforall {\tau \in \upsilon}
        \qexists {\xi \in \upsilon}
        \ref{eq:def:basic_set_operations/power_set/predicate}[\xi, \tau]
      }
      \thinspace \wedge \\ &\wedge
      \parens[\Bigg]
      {
        \qforall {\tau \in \upsilon}
        \qforall \xi \qforall \eta
        \parens[\Bigg]
        {
          \parens[\Big]
            {
              \underbrace
                {
                  \ref{eq:def:function/predicate}[\xi, \tau, \upsilon]
                }_{\mathclap{\xi \T*{is a function} \tau \to \upsilon}}
              \wedge
              \underbrace
                {
                  \ref{eq:def:grothendieck_universe/predicate_isimage}[\eta, \xi]
                }_{ \mathclap{\eta \T*{is the image of} \xi} }
            }
            \rightarrow
            \underbrace
              {
                \qexists {\zeta \in \upsilon} \ref{eq:def:basic_set_operations/union/predicate}[\zeta, \xi]
              }_{ \bigcup \img(\xi) \in \tau
            }
        }
      },
    \end{aligned}
  \end{equation*}
  where
  \begin{equation*}\taglabel[\op{IsImage}]{eq:def:grothendieck_universe/predicate_isimage}
    \ref{eq:def:grothendieck_universe/predicate_isimage}[\chi, \tau]
    \coloneqq
    \qforall \xi
    \parens[\Big]
    {
      \xi \in \chi
      \leftrightarrow
      \underbrace
      {
        \qexists {\eta \in \tau}
        \qexists \zeta
        \ref{eq:def:tuple_and_cartesian_product/kuratowski_pair_predicate}[\eta, \zeta, \xi]
      }_{\tau(\zeta) = \xi \T*{for some} \xi \in \dom(\tau) }
    }.
  \end{equation*}
\end{definition}

\begin{definition}\label{def:axiom_of_universes}\mcite{nLab:grothendieck_universe}
  The \term{axiom of universes} states that any set is contained in a \hyperref[def:grothendieck_universe]{Grothendieck universe}. Symbolically,
  \begin{equation}\label{eq:def:axiom_of_universes}
    \begin{aligned}
      \qforall \tau \qexists \upsilon \parens[\Big]{ \ref{eq:def:grothendieck_universe/predicate}[\upsilon] \wedge \tau \in \upsilon }.
    \end{aligned}
  \end{equation}

  We usually add this theorem to \hyperref[def:zfc]{ZFC} and call the resulting \hyperref[def:first_order_theory]{logical theory} \logic{ZFC+U}.
\end{definition}

\begin{proposition}\label{thm:smallest_grothendieck_universe_existence}
  Suppose that we are working in \logic{ZFC+U}. Then there exists a smallest Grothendieck universe.

  More generally, fix a set \( A \). Then there exists a smallest Grothendieck universe containing \( A \).
\end{proposition}
\begin{proof}
  If no set \( A \) is given, we simply take \( A = \varnothing \) since it must belong to every universe by definition.

  We use a trick analogous to \fullref{thm:smallest_inductive_set_existence}.

  The \hyperref[def:axiom_of_universes]{axiom of universes} states that there exists at least one universe \( U \) that contains \( A \). Define
  \begin{equation*}
    \widehat U \coloneqq \set{ x \in U \given x \T{belongs to every Grothendieck universe} }.
  \end{equation*}

  Now that we have defined \( \widehat U \), it remains to verify that it is itself a universe. Note that every universe contains an inductive set, hence the \hyperref[thm:smallest_inductive_set_existence]{smallest inductive set \( \omega \)} belongs to all universes and thus also to their \enquote{intersection} \( \widehat U \). This rest of the verification is trivial.
\end{proof}

\begin{definition}\label{def:large_and_small_sets}\mcite{nLab:grothendieck_universe}
  Suppose \( \mscrV = (V, I) \) is a \hyperref[def:first_order_semantics/satisfiability]{model} of \logic{ZFC+U}. Let \( U \) be a fixed Grothendieck universe.

  We say that a set \( A \) is \( U \)-\term{small} if \( A \in U \) and \( U \)-\term{moderate} if \( A \subseteq U \). This situation resembles the difference between sets and proper classes described in \fullref{def:set_builder_notation}.

  A set that is not \( U \)-\term{small} is called \( U \)-\term{large}. Note that any superset of \( U \) is \( U \)-large, but not \( U \)-moderate.

  Without further context (i.e. in \enquote{ordinary mathematics}), we assume that \( U \) refers to the \hyperref[thm:smallest_grothendieck_universe_existence]{smallest Grothendieck universe} and instead of the terms \( U \)-large and \( U \)-small, we simply use the terms \term{large} and \term{small}.
\end{definition}

\begin{theorem}\label{thm:grothendieck_universe_iff_strongly_inaccessible}\mcite{Williams1969}
  The stage \( V_\kappa \) of the \hyperref[def:cumulative_hierarchy]{von Neumann's cumulative hierarchy} is a \hyperref[def:grothendieck_universe]{Grothendieck universe} for every \hyperref[def:regular_cardinal]{strongly inaccessible cardinal} \( \kappa \).

  Conversely, for every Grothendieck universe \( U \), there exists a strongly inaccessible cardinal \( \kappa \) such that \( U = V_\kappa \).
\end{theorem}
\begin{proof}
  \SufficiencySubProof Let \( \kappa \) be strongly inaccessible.

  \SubProofOf*{def:grothendieck_universe/inductive} Clearly \( V_\kappa \) contains an inductive set because \( \kappa \) is uncountable and thus
  \begin{equation*}
    \omega
    \in
    \kappa
    \reloset {\ref{thm:cumulative_hierarchy_properties/ordinals}} \subseteq
    V_\kappa.
  \end{equation*}

  \SubProofOf*{def:grothendieck_universe/transitive} The set \( V_\kappa \) is transitive as shown in \fullref{thm:cumulative_hierarchy_properties/transitive}.

  \SubProofOf*{def:grothendieck_universe/power_set} Since \( \kappa \) is a limit ordinal, \( V_\kappa \) satisfies the axiom of power sets as shown in \fullref{thm:cumulative_hierarchy_model_of_z} and thus if \( A \in V_\kappa \), then \( \pow(A) \in V_\kappa \).

  \SubProofOf*{def:grothendieck_universe/union} Fix some member \( A \in V_\kappa \) and some \( A \)-indexed family \( \set{ B_a }_{a \in A} \subseteq V_\kappa \). From \fullref{thm:strong_regular_cardinal_stage_cardinality} it follows that \( \card(A) < \kappa \) and \( \card(B_a) < \kappa \) for every \( a \in A \). Thus,
  \begin{equation*}
    \card(\set{ B_a \given a \in A }) \leq \card(A) < \kappa
  \end{equation*}
  and from \fullref{thm:regular_cardinal_stage_inverse_transitivity} it follows that
  \begin{equation*}
    \set{ B_a \given a \in A } \in V_\kappa.
  \end{equation*}

  The union
  \begin{equation*}
    \bigcap \set{ B_a \given a \in A }
  \end{equation*}
  is then a member of a lower stage, hence it also belongs to \( V_\kappa \).

  \NecessitySubProof Let \( U \) be a Grothendieck universe and let \( \alpha \) be the smallest \hi{ordinal} not in \( U \). We will first show that \( U = V_\alpha \) and gradually prove that \( \alpha \) is actually an inaccessible cardinal.

  \SubProof*{Proof that \( V_\beta \in U \) for \( \beta \in U \)} We will use \fullref{thm:bounded_transfinite_induction} on \( \beta < \alpha \).
  \begin{itemize}
    \item \ref{def:grothendieck_universe/inductive} ensures that \( U \) contains an inductive set \( A \). \ref{def:grothendieck_universe/transitive} ensures that \( A \subseteq U \) and since \( \varnothing \) is a member of every inductive set, we have \( \varnothing \in U \). Therefore, \( V_0 = \varnothing \in U \).

    \item If \( \beta < \alpha \) and \( V_\beta \in U \), then by \ref{def:grothendieck_universe/power_set} we have \( V_{\beta + 1} = \pow(V_\beta) \in U \).

    We will come back to this step a bit later, but for now note that \( V_{\beta + 1} \in U \) irregardless of whether \( \beta + 1 \in U \).

    \item If \( \lambda < \alpha \) is a limit ordinal and \( V_\beta \in U \) for every \( \beta < \lambda \), we have
    \begin{equation*}
      V_\lambda
      \reloset {\eqref{eq:def:cumulative_hierarchy}} =
      \bigcup\set{ V_\beta \given \beta < \lambda },
    \end{equation*}
    which is a \( \lambda \)-indexed union of members of \( U \). Since \( \lambda \in U \), by \ref{def:grothendieck_universe/power_set} we have \( V_\lambda \in U \).
  \end{itemize}

  \SubProof*{Proof that \( \alpha \) is a limit ordinal} In the successor case we noted that \( V_{\beta + 1} \in U \) for every \( \beta < \alpha \) irregardless of whether \( \beta + 1 < \alpha \). Since \( \rank(\beta + 1) = \beta + 1 \), it follows that \( \beta + 1 \in V_{\beta + 2} \in U \) and thus by \ref{def:grothendieck_universe/transitive}, \( \beta + 1 \in U \). Therefore, \( \alpha \) cannot be a successor ordinal --- if \( \alpha = \beta + 1 \), then \( \beta \in U \) by definition of \( \alpha \) and thus \( \beta + 1 = \alpha \in U \), which is a contradiction.

  Since \( \alpha > 0 \), it remains for \( \alpha \) to be a limit ordinal.

  \SubProof*{Proof that \( V_\alpha \subseteq U \)} By \ref{def:grothendieck_universe/transitive}, \( V_\beta \subseteq U \) for every \( \beta < \alpha \). We can conclude that
  \begin{equation*}
    V_\alpha
    \reloset {\eqref{eq:def:cumulative_hierarchy}} =
    \bigcup\set{ V_\beta \given \beta < \alpha }
    \subseteq
    U.
  \end{equation*}

  In order to show that equality holds, we must first prove that \( \alpha \) is a strongly inaccessible cardinal. But this requires some auxiliary results.

  \SubProof*{Proof that \( \set{ B } \in U \) for every \( B \in U \)} By \ref{def:grothendieck_universe/power_set} we have that \( \pow(\pow(B)) \in U \). But \( \set{ B } \subseteq \pow(B) \) and hence \( \set{ B } \in \pow(\pow(B)) \). By \ref{def:grothendieck_universe/transitive}, \( \set{ B } \in U \).

  \SubProof*{Proof that \( \kappa = \alpha \) is a cardinal} Suppose that \( \alpha \) is not a cardinal. Indeed, suppose that there exists some \( \beta < \alpha \) such that there exists a bijective function \( f: \beta \to \alpha \). Then
  \begin{equation*}
    \alpha = \bigcup\set{ \set{ f(\gamma) } \given \gamma < \beta }
  \end{equation*}
  is a \( \beta \)-indexed union of members of \( \alpha \) and hence \( \alpha \in \alpha \). But this contradicts \fullref{thm:simple_foundation_theorems/member_of_itself}.

  Therefore, \( \alpha \) is a cardinal. We will henceforth denote it by \( \kappa \) to highlight that it is a cardinal.

  \SubProof*{Proof that \( \card(B) \in U \) for every \( B \in U \)} Let \( B \in U \) and let \( f: B \to \card(B) \) be a bijective function. Then
  \begin{equation*}
    \card(B)
    =
    f[B]
    =
    \set{ f(x) \given x \in B }
    =
    \bigcup\set[\Big]{ \set{ f(x) } \given x \in B }
  \end{equation*}
   is a \( B \)-indexed union of members of \( U \) and, by \ref{def:grothendieck_universe/union}, \( \card(B) \in U \).

  \SubProof*{Proof that \( \kappa \) is a strong limit} For every \( \beta < \kappa \) by \ref{def:grothendieck_universe/power_set} we have \( \pow(\beta) \in U \). We have already show that \( \card(\pow(\beta)) \in U \) and, by \fullref{thm:cardinal_exponentiation_power_set}, we have
  \begin{equation*}
    \card(\pow(\beta)) = 2^{\card(\beta)} = 2^\beta.
  \end{equation*}

  Hence, \( 2^\beta < \kappa \) and \( \kappa \) is a strong limit.

  \SubProof*{Proof that \( \kappa \) is regular} Let \( C \subseteq \kappa \) be an unbounded set. We will show that \( \card(C) = \kappa \).

  Suppose that \( \card(C) < \kappa \). Then \( \card(C) \in U \) since \( \kappa = \alpha \) is the smallest ordinal not contained in \( U \). Let \( f: \card(C) \to C \) be a bijective function. Then
  \begin{equation*}
    C = \bigcup\set[\Big]{ \set{ f(\gamma) } \given \gamma < \card(C) }
  \end{equation*}
  and by \ref{def:grothendieck_universe/union}, \( C \in U \).

  Since \( C \) is unbounded, we have \( \sup C \geq \kappa \). But from \fullref{thm:union_of_set_of_ordinals} is follows that \( \bigcup C = \sup C \) and by \ref{def:grothendieck_universe/union}, \( \bigcup C \in U \), which implies that \( \sup C < \kappa \).

  The obtained contradiction shows that \( \card(C) = \kappa \). Since \( C \) was an arbitrary unbounded set, it follows that \( \kappa \) satisfies \fullref{def:regular_cardinal/unbounded_subsets} and is thus regular.

  \SubProof*{Proof that \( \kappa \) is uncountable} In the beginning of the proof we noted that \( \omega \subseteq U \) since \( U \) contains some inductive set \( A \). Thus,
  \begin{equation*}
    \omega = \bigcup\set[\Big]{ \set{ n } \given* n \in \omega } \cup \bigcup\set[\Big]{ \varnothing \given* A \setminus \omega }.
  \end{equation*}

  By \ref{def:grothendieck_universe/union}, \( \omega \in U \). Therefore, \( \omega < \kappa \) and since \( \kappa \) is a cardinal, we conclude that \( \kappa \) is necessary uncountable.

  \SubProof*{Proof that \( V_\kappa = U \)} Finally, now that we know that \( \kappa \) is a strongly inaccessible cardinal, we can show that equality holds in \( V_\kappa \subseteq U \).

  Aiming at a contradiction, suppose that \( U \setminus V_\kappa \) is nonempty. By the \hyperref[def:zfc/foundation]{axiom of foundation}, there exists a set \( C \in U \setminus V_\kappa \) such that
  \begin{equation*}
    C \cap (U \setminus V_\kappa) = \varnothing,
  \end{equation*}
  thus \( C \subseteq V_\kappa \). From \fullref{thm:strong_regular_cardinal_stage_cardinality} if follows that \( \card(C) < \kappa \) and from \fullref{thm:regular_cardinal_stage_inverse_transitivity} it follows that \( C \in V_\kappa \), which contradicts our choice of \( C \) as a member of \( U \setminus V_\kappa \).

  Therefore, \( V_\kappa = U \).
\end{proof}

\begin{corollary}\label{thm:grothendieck_universe_is_model_of_zfc}
  Every Grothendieck universe is a standard model of \hyperref[def:zfc]{\logic{ZFC}}.
\end{corollary}
\begin{proof}
  Follows from \fullref{thm:grothendieck_universe_iff_strongly_inaccessible} and \fullref{rem:cumulative_hierarchy_model_of_zfc}.
\end{proof}
