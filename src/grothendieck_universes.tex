\subsection{Grothendieck universes}\label{subsec:grothendieck_universes}

Instead of having one single universe, we can have multiple universes where each is contained in another one.

\begin{definition}\label{def:grothendieck_universe}\mcite{nLab:grothendieck_universe}
  We say that the set \( U \) is a \term{Grothendieck universe} if it satisfied the following conditions:
  \begin{thmenum}
    \thmitem[def:grothendieck_universe/inductive]{AU1} It contains an \hyperref[def:inductive_set]{inductive set}. We specifically mean the \hyperref[thm:smallest_inductive_set_existence]{smallest inductive set \( \omega \)}, however is sufficient only to state the existence of one inductive set.

    \thmitem[def:grothendieck_universe/transitive]{AU2} It is a \hyperref[def:transitive_set]{transitive set}.

    \thmitem[def:grothendieck_universe/power_set]{AU3} For any \( A \in U \), the \hyperref[def:basic_set_operations/power_set]{power set} \( \pow(A) \) also belongs to \( U \).

    \thmitem[def:grothendieck_universe/union]{AU4} For any member \( A \in U \) and any \( A \)-indexed family \( \set{ B_a }_{a \in A} \subseteq U \), the union
    \begin{equation*}
      \bigcup\set{ B_a \given a \in A }
    \end{equation*}
    belongs to \( U \). This is a restriction from unions over completely arbitrary families of sets to those families that can be indexed by members of \( A \).
  \end{thmenum}

  We formalize the entire concept via the following monstrous formula:
  \begin{equation*}\taglabel[\op{IsUniverse}]{eq:def:grothendieck_universe/predicate}
    \begin{aligned}
      \mathrlap
      {
        \ref{eq:def:grothendieck_universe/predicate}[\upsilon] \coloneqq \parens[\Big]
          {
            \qexists {\tau \in \upsilon}
            \ref{eq:def:inductive_set/predicate}[\tau]
          }
        \wedge
      }
      \hspace{1cm}&
      \\ &\wedge
      \parens[\Big]
      {
        \qforall {\tau \in \upsilon}
        \ref{eq:def:subset/predicate}[\tau, \upsilon]
      }
      \thinspace \wedge \\ &\wedge
      \parens[\Big]
      {
        \qforall {\tau \in \upsilon}
        \qexists {\xi \in \upsilon}
        \ref{eq:def:basic_set_operations/power_set/predicate}[\xi, \tau]
      }
      \thinspace \wedge \\ &\wedge
      \parens[\Bigg]
      {
        \qforall {\tau \in \upsilon}
        \qforall \xi \qforall \eta
        \parens[\Bigg]
        {
          \parens[\Big]
            {
              \underbrace
                {
                  \ref{eq:def:function/predicate}[\xi, \tau, \upsilon]
                }_{\mathclap{\xi \T*{is a function} \tau \to \upsilon}}
              \wedge
              \underbrace
                {
                  \ref{eq:def:grothendieck_universe/predicate_isimage}[\eta, \xi]
                }_{ \mathclap{\eta \T*{is the image of} \xi} }
            }
            \rightarrow
            \underbrace
              {
                \qexists {\zeta \in \upsilon} \ref{eq:def:basic_set_operations/union/predicate}[\zeta, \xi]
              }_{ \bigcup \img(\xi) \in \tau
            }
        }
      },
    \end{aligned}
  \end{equation*}
  where
  \begin{equation*}\taglabel[\op{IsImage}]{eq:def:grothendieck_universe/predicate_isimage}
    \ref{eq:def:grothendieck_universe/predicate_isimage}[\chi, \tau]
    \coloneqq
    \qforall \xi
    \parens[\Big]
    {
      \xi \in \chi
      \leftrightarrow
      \underbrace
      {
        \qexists {\eta \in \tau}
        \qexists \zeta
        \ref{eq:def:binary_cartesian_product/pair_predicate}[\eta, \zeta, \xi]
      }_{\tau(\zeta) = \xi \T*{for some} \xi \in \dom(\tau) }
    }.
  \end{equation*}
\end{definition}

\begin{definition}\label{def:axiom_of_universes}\mcite{nLab:grothendieck_universe}
  The \term{axiom of universes} states that any set is contained in a \hyperref[def:grothendieck_universe]{Grothendieck universe}. Symbolically,
  \begin{equation}\label{eq:def:axiom_of_universes}
    \begin{aligned}
      \qforall \tau \qexists \upsilon \parens[\Big]{ \ref{eq:def:grothendieck_universe/predicate}[\upsilon] \wedge \tau \in \upsilon }.
    \end{aligned}
  \end{equation}

  We usually add this theorem to \hyperref[def:zfc]{ZFC} and call the resulting \hyperref[def:first_order_theory]{logical theory} \logic{ZFC+U}.
\end{definition}

\begin{proposition}\label{thm:smallest_grothendieck_universe_existence}
  Suppose that we are working in \logic{ZFC+U}. Then there exists a smallest Grothendieck universe.

  More generally, fix a set \( A \). Then there exists a smallest Grothendieck universe containing \( A \).
\end{proposition}
\begin{proof}
  If no set \( A \) is given, we simply take \( A = \varnothing \) since it must belong to every universe by definition.

  We use a trick analogous to \fullref{thm:smallest_inductive_set_existence}.

  The \hyperref[def:axiom_of_universes]{axiom of universes} states that there exists at least one universe \( U \) that contains \( A \). Define
  \begin{equation*}
    \widehat U \coloneqq \set{ x \in U \given x \T{belongs to every Grothendieck universe} }.
  \end{equation*}

  Now that we have defined \( \widehat U \), it remains to verify that it is itself a universe. Note that every universe contains an inductive set, hence the \hyperref[thm:smallest_inductive_set_existence]{smallest inductive set \( \omega \)} belongs to all universes and thus also to their \enquote{intersection} \( \widehat U \). This rest of the verification is trivial.
\end{proof}

\begin{definition}\label{def:large_and_small_sets}\mcite{nLab:grothendieck_universe}
  Suppose \( \mscrV = (V, I) \) is a \hyperref[def:first_order_semantics/satisfiability]{model} of \logic{ZFC+U}. Let \( U \) be a fixed Grothendieck universe.

  We say that a set \( A \) is \( U \)-\term{small} if \( A \in U \) and \( U \)-\term{moderate} if \( A \subseteq U \). This situation resembles the difference between sets and proper classes described in \fullref{def:set_builder_notation}.

  A set that is not \( U \)-\term{small} is called \( U \)-\term{large}. Note that any superset of \( U \) is \( U \)-large but not \( U \)-moderate.

  Without further context (i.e. in \enquote{ordinary mathematics}), we assume that \( U \) refers to the \hyperref[thm:smallest_grothendieck_universe_existence]{smallest Grothendieck universe} and instead of the terms \( U \)-large and \( U \)-small, we simply use the terms \term{large} and \term{small}.
\end{definition}

\begin{theorem}\label{thm:grothendieck_universe_iff_strongly_inaccessible}\mcite{Williams1969}
  The stage \( V_\kappa \) of the \hyperref[def:cumulative_hierarchy]{von Neumann's cumulative hierarchy} is a \hyperref[def:grothendieck_universe]{Grothendieck universe} for every \hyperref[def:regular_cardinal]{strongly inaccessible cardinal} \( \kappa \).

  Conversely, for every Grothendieck universe \( U \), the cardinal \( \kappa \coloneqq \card(U) \) is strongly inaccessible and \( U = V_\kappa \).
\end{theorem}
\begin{proof}
  \SufficiencySubProof Let \( \kappa \) be strongly inaccessible.

  \SubProofOf*{def:grothendieck_universe/inductive} Clearly \( V_\kappa \) contains an inductive set because \( \kappa \) is uncountable and thus
  \begin{equation*}
    \omega
    \in
    \kappa
    \reloset {\ref{thm:cumulative_hierarchy_properties/ordinals}} \subseteq
    V_\kappa.
  \end{equation*}

  \SubProofOf*{def:grothendieck_universe/transitive} The set \( V_\kappa \) is transitive as shown in \fullref{thm:cumulative_hierarchy_properties/transitive}.

  \SubProofOf*{def:grothendieck_universe/power_set} Since \( \kappa \) is a limit ordinal, \( V_\kappa \) satisfies the axiom of power sets as shown in \fullref{thm:cumulative_hierarchy_model_of_z} and thus if \( A \in V_\kappa \), then \( \pow(A) \in V_\kappa \).

  \SubProofOf*{def:grothendieck_universe/union} Fix some member \( A \in V_\kappa \) and some \( A \)-indexed family \( \set{ B_a }_{a \in A} \subseteq V_\kappa \). From \fullref{thm:strong_regular_cardinal_stage_cardinality} it follows that \( \card(A) < \kappa \) and \( \card(B_a) < \kappa \) for every \( a \in A \). Thus
  \begin{equation*}
    \card(\set{ B_a \given a \in A }) \leq \card(A) < \kappa
  \end{equation*}
  and from \fullref{thm:regular_cardinal_stage_inverse_transitivity} it follows that
  \begin{equation*}
    \set{ B_a \given a \in A } \in V_\kappa.
  \end{equation*}

  The union
  \begin{equation*}
    \bigcap \set{ B_a \given a \in A }
  \end{equation*}
  is then a member of a lower stage, hence it also belongs to \( V_\kappa \).

  \NecessitySubProof Let \( U \) be a Grothendieck universe and let \( \kappa \coloneqq \card(U) \).

  First note that \( \kappa \) is an uncountable cardinal. Indeed, \ref{def:grothendieck_universe/inductive} ensures that \( U \) contains an inductive set \( A \), which is infinite, and \fullref{def:grothendieck_universe/power_set} ensures that \( U \) contains \( \pow(A) \). Even if \( A \) is countable, for example if \( A = \omega \), the power \( \pow(A) \) is uncountable due to \fullref{thm:cantor_power_set_theorem}. \Fullref{def:grothendieck_universe/transitive} then ensures that \( \pow(A) \subseteq U \) and hence \( \kappa = \card(U) \) is also uncountable.

  Also note that \( U \) contains the empty set because \( A \), as an inductive set, contains \( \varnothing \) and thus \( \varnothing \in A \subseteq U \).

  Now we will use \fullref{thm:bounded_transfinite_induction} on \( \alpha < \kappa \) to show that \( V_\alpha \in U \).
  \begin{itemize}
    \item If \( \alpha = 0 \), then \( V_\alpha = \varnothing \), which we have already shown to belong to \( U \).
    \item If \( V_\alpha \in U \), then from \ref{def:grothendieck_universe/power_set} it follows that
    \begin{equation*}
      V_{\alpha + 1} = \pow(V_\alpha) \in U.
    \end{equation*}

    \item Let \( \lambda < \kappa \) be a limit ordinal and suppose that \( V_\alpha \in U \) for every \( \alpha < \lambda \). Then
    \begin{equation*}
      V_\lambda
      \reloset {\eqref{eq:def:cumulative_hierarchy}} =
      \bigcup\set{ V_\alpha \given \alpha < \lambda }
    \end{equation*}
    and the latter in a \( \lambda \)-indexed union of members of \( U \). Bby \fullref{def:grothendieck_universe/union} it follows that \( V_\lambda \in \kappa \).
  \end{itemize}

  We have shown that \( V_\alpha \in U \) for \( \alpha < \kappa \). By \ref{def:grothendieck_universe/transitive}, we have \( A_\alpha \subseteq U \) for every \( \alpha \in U \). Since \( \kappa \) is infinite, \fullref{thm:cardinal_is_infinite_iff_limit_ordinal} it is a limit ordinal and thus
  \begin{equation*}
    V_\kappa
    =
    \bigcup\set{ V_\alpha \given \alpha \in \kappa }
    \subseteq
    U.
  \end{equation*}

  Aiming at a contradiction, suppose that \( U \setminus V_\kappa \) is nonempty. By \hyperref[def:zfc/foundation]{axiom of foundation} there exists a set \( B \in U \setminus V_\kappa \) such that
  \begin{equation*}
    B \cap (U \setminus V_\kappa) = \varnothing,
  \end{equation*}
  thus \( B \subseteq V_\kappa \). From \fullref{thm:strong_regular_cardinal_stage_cardinality} if follows that \( \card(B) < \kappa \) and from \fullref{thm:regular_cardinal_stage_inverse_transitivity} it follows that \( B \in V_\kappa \), which contradicts our choice of \( B \) as a member of \( U \setminus V_\kappa \).

  Therefore \( U = V_\kappa \).
\end{proof}

\begin{corollary}\label{thm:grothendieck_universe_is_model_of_zfc}
  Every Grothendieck universe is a model of \hyperref[def:zfc]{\logic{ZFC}}.
\end{corollary}
\begin{proof}
  Follows from \fullref{thm:grothendieck_universe_iff_strongly_inaccessible} and \fullref{rem:cumulative_hierarchy_model_of_zfc}.
\end{proof}
