\subsection{Grothendieck universes}\label{subsec:grothendieck_universes}

Instead of having one single universe, we can have multiple universes where each is contained in another one.

\begin{definition}\label{def:grothendieck_universe}\mcite{nLab:grothendieck_universe}
  We say that the set \( U \) is a \term{Grothendieck universe} if it satisfied the following conditions:
  \begin{thmenum}
    \thmitem{def:grothendieck_universe/inductive} It contains an \hyperref[def:inductive_set]{inductive set}. We specifically mean the \hyperref[thm:smallest_inductive_set_existence]{smallest inductive set \( \omega \)}, however is sufficient only to state the existence of one inductive set.

    This corresponds to the following \hyperref[rem:predicate_formula]{predicate formula} with \( \upsilon \) as the free variable:
    \begin{equation}\label{eq:def:grothendieck_universe/inductive}
      \qexists {\alpha \in \upsilon} \ref{eq:def:inductive_set/predicate}[\alpha].
    \end{equation}

    \thmitem{def:grothendieck_universe/transitive} \( U \) is \hyperref[def:transitive_set]{transitive}.

    Formally,
    \begin{equation}\label{eq:def:grothendieck_universe/transitive}
      \qforall {\alpha \in \upsilon} \ref{eq:def:subset/predicate}[\alpha, \upsilon]
    \end{equation}

    \thmitem{def:grothendieck_universe/power_set} For any \( A \in U \), the \hyperref[def:basic_set_operations/power_set]{power set} \( \pow(A) \) also belongs to \( U \).

    Formally,
    \begin{equation}\label{eq:def:grothendieck_universe/power_set}
      \qforall {\alpha \in \upsilon} \qexists {\xi \in \upsilon} \ref{eq:def:basic_set_operations/power_set/predicate}[\xi, \alpha]
    \end{equation}

    \thmitem{def:grothendieck_universe/union} For any member \( A \in U \) and any \( A \)-indexed family \( \set{ B_a }_{a \in A} \), the union \( \bigcup\set{ B_a \given a \in A } \) belongs to \( U \). This is a restriction from unions over completely arbitrary families of sets to those families that can be indexed by members of \( A \).

    Formally, again with \( \upsilon \) as the only free variable, this can be expressed via the \hyperref[rem:predicate_formula]{predicate formula}
    \begin{equation}\label{eq:def:grothendieck_universe/union}
      \qforall {\alpha \in \upsilon} \qforall \xi \qforall \eta \parens[\Bigg]{ \parens[\Big]{ \underbrace{ \ref{eq:def:function/predicate}[\xi, \alpha, \upsilon] }_{\mathclap{\xi \T*{is a function} \alpha \to \upsilon}} \wedge \underbrace{ \ref{eq:def:grothendieck_universe/predicate_isimage}[\eta, \xi] }_{ \mathclap{\eta \T*{is the image of} \xi} } } \rightarrow \underbrace{ \qexists {\zeta \in \upsilon} \ref{eq:def:basic_set_operations/union/predicate}[\zeta, \xi] }_{ \bigcup \img(\xi) \in \alpha } },
    \end{equation}
    where
    \begin{equation}\label{eq:def:grothendieck_universe/predicate_isimage}
      \op{IsImage}[\chi, \alpha] \coloneqq \qforall \xi \parens[\Big]{ \xi \in \chi \leftrightarrow \underbrace{ \qexists {\eta \in \alpha} \qexists \zeta \ref{eq:def:binary_cartesian_product/pair_predicate}[\eta, \zeta, \xi] }_{\alpha(\zeta) = \xi \T*{for some} \xi \in \dom(\alpha) } }
    \end{equation}
  \end{thmenum}

  We formalize the entire concept via the following monstrous conjunction:
  \begin{equation*}\taglabel[\op{IsUniverse}]{eq:def:grothendieck_universe/predicate}
    \begin{aligned}
      \ref{eq:def:grothendieck_universe/predicate}[\upsilon] &\coloneqq \eqref{eq:def:grothendieck_universe/inductive}[\upsilon] \wedge \eqref{eq:def:grothendieck_universe/transitive}[\upsilon] \wedge \eqref{eq:def:grothendieck_universe/power_set}[\upsilon] \wedge \eqref{eq:def:grothendieck_universe/union}[\upsilon]
    \end{aligned}
  \end{equation*}
\end{definition}

\begin{definition}\label{def:axiom_of_universes}\mcite{nLab:grothendieck_universe}
  The \term{axiom of universes} states that any set is contained in a \hyperref[def:grothendieck_universe]{Grothendieck universe}. Symbolically,
  \begin{equation}\label{eq:def:axiom_of_universes}
    \begin{aligned}
      \qforall \alpha \qexists \upsilon \parens[\Big]{ \ref{eq:def:grothendieck_universe/predicate}[\upsilon] \wedge \alpha \in \upsilon }.
    \end{aligned}
  \end{equation}

  We usually add this theorem to \hyperref[def:zfc]{ZFC} and call the resulting \hyperref[def:first_order_theory]{logical theory} \logic{ZFC+U}.
\end{definition}

\begin{proposition}\label{thm:smallest_grothendieck_universe_existence}
  Suppose that we are working in \logic{ZFC+U}. Then there exists a smallest Grothendieck universe.

  More generally, fix a set \( A \). Then there exists a smallest Grothendieck universe containing \( A \).
\end{proposition}
\begin{proof}
  If no set \( A \) is given, we simply take \( A = \varnothing \) since it must belong to every universe by definition.

  We use a trick analogous to \fullref{thm:smallest_inductive_set_existence}.

  The \hyperref[def:axiom_of_universes]{axiom of universes} states that there exists at least one universe \( U \) that contains \( A \). Define
  \begin{equation*}
    \widehat U \coloneqq \set{ x \in U \given x \T{belongs to every Grothendieck universe} }.
  \end{equation*}

  Now that we have defined \( \widehat U \), it remains to verify that it is itself a universe. Note that every universe contains an inductive set, hence the \hyperref[thm:smallest_inductive_set_existence]{smallest inductive set \( \omega \)} belongs to all universes and thus also to their \enquote{intersection} \( \widehat U \). This rest of the verification is trivial.
\end{proof}

\begin{definition}\label{def:large_and_small_sets}\mcite{nLab:grothendieck_universe}
  Suppose \( \mscrV = (V, I) \) is a \hyperref[def:first_order_semantics/satisfiability]{model} of \logic{ZFC+U}. Let \( U \) be a fixed Grothendieck universe.

  We say that a set \( A \) is \( U \)-\term{small} if \( A \in U \) and \( U \)-\term{moderate} if \( A \subseteq U \). This situation resembles the difference between sets and proper classes described in \fullref{def:set_builder_notation}.

  A set that is not \( U \)-\term{small} is called \( U \)-\term{large}. Note that \( U \) itself is \( U \)-large but not \( U \)-moderate.

  Without further context (i.e. in \enquote{ordinary mathematics}), we assume that \( U \) refers to the \hyperref[thm:smallest_grothendieck_universe_existence]{smallest Grothendieck universe} and instead of the terms \( U \)-large and \( U \)-small, we simply use the terms \term{large} and \term{small}.
\end{definition}
