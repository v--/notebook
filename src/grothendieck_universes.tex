\subsection{Grothendieck universes}\label{subsec:grothendieck_universes}

Instead of having one single universe, we can have multiple universes where each is contained in another one. The upside of this is that we can do category theory formally within set theory --- see the discussions in \fullref{def:category_size}. The downside of this is that, unlike models of \hyperref[def:zfc]{\logic{ZFC}}, models of \hyperref[def:axiom_of_universes]{\logic{ZFC+U}} (\logic{ZFC} with the axiom that every set is contained in some \hyperref[def:grothendieck_universe]{Grothendieck universe}) are much less studied. In particular, this axiom requires the existence of an unbounded hierarchy of \hyperref[rem:strongly_inaccessible_cardinal]{regular strong limit cardinal}, unlike \( \logic{ZFC} \) for which only one such cardinal is sufficient.

\begin{definition}\label{def:grothendieck_universe}\mcite{nLab:grothendieck_universe}
  We say that the set \( \mscrU \) is a \term{Grothendieck universe} if it satisfied the following conditions:
  \begin{thmenum}
    \thmitem[def:grothendieck_universe/nonempty]{GU1} It is \hyperref[def:empty_set]{nonempty}.

    \thmitem[def:grothendieck_universe/transitive]{GU2} It is a \hyperref[def:transitive_set]{transitive set}.

    \thmitem[def:grothendieck_universe/power_set]{GU3} For any \( A \in \mscrU \), the \hyperref[def:basic_set_operations/power_set]{power set} \( \pow(A) \) also belongs to \( \mscrU \).

    \thmitem[def:grothendieck_universe/union]{GU4} For any member \( A \in \mscrU \) and any \( A \)-indexed family \( \set{ B_a }_{a \in A} \subseteq \mscrU \), the union
    \begin{equation*}
      \bigcup\set{ B_a \given a \in A }
    \end{equation*}
    belongs to \( \mscrU \). This is a restriction from unions over completely arbitrary families of sets to those families that can be indexed by members of \( A \).
  \end{thmenum}

  We formalize the entire concept via the following monstrous formula:
  \small
  \begin{multline*}\taglabel[\op{IsUniverse}]{eq:def:grothendieck_universe/predicate}
    \ref{eq:def:grothendieck_universe/predicate}[\upsilon] \coloneqq
      \neg \ref{eq:def:empty_set/predicate}[\upsilon]
      \wedge
      \qforall {\tau \in \upsilon}
      \vast(
        \ref{eq:def:subset/predicate}[\tau, \upsilon]
        \wedge
        \parens[\Big]
          {
            \qexists {\xi \in \upsilon}
            \ref{eq:def:basic_set_operations/power_set/predicate}[\xi, \tau]
          }
        \wedge \\ \wedge
        \qforall \xi
        \qforall \eta
        \parens[\Bigg]
        {
          \parens[\Big]
            {
              \underbrace
                {
                  \ref{eq:def:function/predicate}[\xi, \tau, \upsilon]
                }_{\mathclap{\xi \T*{is a function} \tau \to \upsilon}}
              \wedge
              \underbrace
                {
                  \ref{eq:def:grothendieck_universe/predicate_isimage}[\eta, \xi]
                }_{ \mathclap{\eta \T*{is the image of} \xi} }
            }
          \rightarrow
          \underbrace
            {
              \qexists {\zeta \in \upsilon} \ref{eq:def:basic_set_operations/union/predicate}[\zeta, \xi]
            }_{ \bigcup \img(\xi) \in \upsilon}
      }
    \vast),
  \end{multline*}
  \normalsize
  where
  \begin{equation*}\taglabel[\op{IsImage}]{eq:def:grothendieck_universe/predicate_isimage}
    \ref{eq:def:grothendieck_universe/predicate_isimage}[\rho, \tau]
    \coloneqq
    \qforall \xi
    \parens[\Big]
    {
      \xi \in \rho
      \leftrightarrow
      \underbrace
      {
        \qexists {\eta \in \tau}
        \qexists \zeta
        \ref{eq:def:tuple_and_cartesian_product/kuratowski_pair_predicate}[\eta, \zeta, \xi]
      }_{\tau(\zeta) = \xi \T*{for some} \xi \in \dom(\tau) }
    }.
  \end{equation*}
\end{definition}

\begin{lemma}\label{thm:grothendieck_universe_contains_finite_sets}
  Every Grothendieck universe is a superset of universe of hereditary finite sets \hyperref[def:universe_of_hereditary_finite_sets]{\( V_\omega \)}.
\end{lemma}
\begin{proof}
  Let \( \mscrU \) be a Grothendieck universe.

  \ref{def:grothendieck_universe/nonempty} ensures that it is nonempty. Then there exists some set \( A \in \mscrU \). By \ref{def:grothendieck_universe/power_set}, \( \pow(A) \in \mscrU \). By \ref{def:grothendieck_universe/transitive}, \( \varnothing \in \pow(A) \in \mscrU \) implies \( \varnothing \in \mscrU \).

  Finally, from \ref{def:grothendieck_universe/power_set} by \fullref{thm:bounded_transfinite_induction} if follows that \( V_{n+1} = \pow(V_n) \) is a member of \( \mscrU \) for every \( n \in \omega \).

  Therefore, \( V_\omega = \bigcup \set{ V_n \given n \in \omega } \) is a subset of \( \mscrU \).
\end{proof}

\begin{definition}\label{def:axiom_of_universes}\mcite{nLab:grothendieck_universe}
  The \term{axiom of universes} states that any set is contained in a \hyperref[def:grothendieck_universe]{Grothendieck universe}. Symbolically,
  \begin{equation}\label{eq:def:axiom_of_universes}
    \begin{aligned}
      \qforall \tau \qexists \upsilon \parens[\Big]{ \ref{eq:def:grothendieck_universe/predicate}[\upsilon] \wedge \tau \in \upsilon }.
    \end{aligned}
  \end{equation}

  We usually add this theorem to \hyperref[def:zfc]{ZFC} and call the resulting \hyperref[def:first_order_theory]{logical theory} \logic{ZFC+U}.
\end{definition}

\begin{example}\label{ex:def:axiom_of_universes}
  From \fullref{thm:grothendieck_universe_iff_regular_strong_limit} it follows that the universe of hereditary finite sets \hyperref[def:universe_of_hereditary_finite_sets]{\( V_\omega \)} is a Grothendieck universe.

  The existence of other universes cannot be proven in \logic{ZFC}. For this reason, we use the \hyperref[def:axiom_of_universes]{axiom of universes}.
\end{example}

\begin{proposition}\label{thm:smallest_grothendieck_universe_existence}
  Suppose that we are working in \logic{ZFC+U}. Then for any set \( A \), there exists a smallest Grothendieck universe containing \( A \).

  More generally, fix a set \( A \). Then there exists a smallest Grothendieck universe containing \( A \).
\end{proposition}
\begin{proof}
  If no set \( A \) is given, we simply take \( A = \varnothing \) since it must belong to every universe by definition.

  We use a trick analogous to \fullref{thm:smallest_inductive_set_existence}.

  The \hyperref[def:axiom_of_universes]{axiom of universes} states that there exists at least one universe \( \mscrU \) that contains \( A \). Define
  \begin{equation*}
    \widehat \mscrU \coloneqq \set{ x \in \mscrU \given x \T{belongs to every Grothendieck universe} }.
  \end{equation*}

  Now that we have defined \( \widehat \mscrU \), it remains to verify that it is itself a universe. To show \ref{def:grothendieck_universe/nonempty}, note that \( V_\omega \in \mscrU \) by \fullref{thm:grothendieck_universe_contains_finite_sets} and hence, by \ref{def:grothendieck_universe/transitive}, \( \omega \in \mscrU \).

  The rest of the verification is trivial.
\end{proof}

\begin{definition}\label{def:large_and_small_sets}\mcite{nLab:grothendieck_universe}
  Suppose \( \mscrV = (V, I) \) is a \hyperref[def:first_order_semantics/satisfiability]{model} of \logic{ZFC+U}. Let \( \mscrU \) be a fixed Grothendieck universe.

  We say that a set \( A \) is \( \mscrU \)-\term{small} if \( A \in \mscrU \) and \( \mscrU \)-\term{moderate} if \( A \subseteq \mscrU \). This situation resembles the difference between sets and proper classes described in \fullref{def:set_builder_notation}.

  A set that is not \( \mscrU \)-\term{small} is called \( \mscrU \)-\term{large}. Note that any strict superset of \( \mscrU \) is \( \mscrU \)-large, but not \( \mscrU \)-moderate.

  Without further context (i.e. in \enquote{ordinary mathematics}), we assume that \( \mscrU \) refers to the \hyperref[thm:smallest_grothendieck_universe_existence]{smallest Grothendieck universe} that contains all sets of interest and instead of the terms \( \mscrU \)-large and \( \mscrU \)-small, we simply use the terms \term{large} and \term{small}.

  In category theory, however, if there is nothing to guarantee the existence of a larger Grothendieck universe, we cannot construct the \hyperref[def:functor_category]{functor category} of \( \mscrU \)-large categories, as discussed in \fullref{rem:functor_category_size}. This is the main motivation for the \hyperref[def:axiom_of_universes]{axiom of universes}.
\end{definition}

\begin{example}\label{ex:large_and_small_sets}
  \hfill
  \begin{itemize}
    \item A set is finite if and only if it is \( V_\omega \)-small. A set is \( V_\omega \)-moderate if and only if it is an infinite family of finite sets.

    For finite mathematics such as most of combinatorics, we rarely need to work outside of \( V_\omega \).

    \item If \( \kappa < \mu \) are regular strong limit cardinals, the stage \( V_\kappa \) of von Neumann's hierarchy is \( V_\mu \)-small by \fullref{thm:def:cumulative_hierarchy/properties/membership}.
  \end{itemize}
\end{example}

\begin{theorem}\label{thm:grothendieck_universe_iff_regular_strong_limit}\mcite{Kruse1966}
  The stage \( V_\kappa \) of the \hyperref[def:cumulative_hierarchy]{von Neumann's cumulative hierarchy} is a \hyperref[def:grothendieck_universe]{Grothendieck universe} for every \hyperref[rem:strongly_inaccessible_cardinal]{regular strong limit cardinal} \( \kappa \).

  Conversely, for every Grothendieck universe \( \mscrU \), there exists a regular strong limit cardinal \( \kappa \) such that \( \mscrU = V_\kappa \).
\end{theorem}
\begin{proof}
  \SufficiencySubProof Let \( \kappa \) be a regular strong limit cardinal.

  \SubProofOf*{def:grothendieck_universe/nonempty} Since \( \kappa \) is infinite and thus \( 0 < \kappa \), from \fullref{thm:def:cumulative_hierarchy/properties/membership} it follows that \( V_0 \in V_\kappa \).

  \SubProofOf*{def:grothendieck_universe/transitive} The set \( V_\kappa \) is transitive as shown in \fullref{thm:def:cumulative_hierarchy/properties/transitive}.

  \SubProofOf*{def:grothendieck_universe/power_set} Since \( \kappa \) is a limit ordinal, \( V_\kappa \) satisfies the axiom of power sets as shown in \fullref{thm:cumulative_hierarchy_model_of_z} and thus if \( A \in V_\kappa \), then \( \pow(A) \in V_\kappa \).

  \SubProofOf*{def:grothendieck_universe/union} Fix some member \( A \in V_\kappa \) and some \( A \)-indexed family \( \set{ B_a }_{a \in A} \subseteq V_\kappa \). From \fullref{thm:strong_regular_cardinal_stage_cardinality} it follows that \( \card(A) < \kappa \) and \( \card(B_a) < \kappa \) for every \( a \in A \). Thus,
  \begin{equation*}
    \card(\set{ B_a \given a \in A }) \leq \card(A) < \kappa
  \end{equation*}
  and from \fullref{thm:regular_cardinal_stage_inverse_transitivity} it follows that
  \begin{equation*}
    \set{ B_a \given a \in A } \in V_\kappa.
  \end{equation*}

  The union
  \begin{equation*}
    \bigcap \set{ B_a \given a \in A }
  \end{equation*}
  is then a member of a lower stage, hence it also belongs to \( V_\kappa \).

  \NecessitySubProof Let \( \mscrU \) be a Grothendieck universe and let \( \alpha \) be the smallest \hi{ordinal} not in \( \mscrU \). We will first show that \( \mscrU = V_\alpha \) and gradually prove that \( \alpha \) is actually an inaccessible cardinal.

  \SubProof*{Proof that \( V_\beta \in \mscrU \) for ordinals \( \beta \in \mscrU \)} We will use \fullref{thm:bounded_transfinite_induction} on \( \beta < \alpha \).
  \begin{itemize}
    \item From \fullref{thm:grothendieck_universe_contains_finite_sets} it follows that \( \varnothing \in \mscrU \).

    \item If \( \beta < \alpha \) and \( V_\beta \in \mscrU \), then by \ref{def:grothendieck_universe/power_set} we have \( V_{\beta + 1} = \pow(V_\beta) \in \mscrU \).

    We will come back to this step a bit later, but for now note that \( V_{\beta + 1} \in \mscrU \) regardless of whether \( \beta + 1 \in \mscrU \).

    \item If \( \lambda < \alpha \) is a limit ordinal and \( V_\beta \in \mscrU \) for every \( \beta < \lambda \), we have
    \begin{equation*}
      V_\lambda
      \reloset {\eqref{eq:def:cumulative_hierarchy}} =
      \bigcup\set{ V_\beta \given \beta < \lambda },
    \end{equation*}
    which is a \( \lambda \)-indexed union of members of \( \mscrU \). Since \( \lambda \in \mscrU \), by \ref{def:grothendieck_universe/power_set} we have \( V_\lambda \in \mscrU \).
  \end{itemize}

  \SubProof*{Proof that \( \alpha \) is a limit ordinal} In the successor case we noted that \( V_{\beta + 1} \in \mscrU \) for every \( \beta < \alpha \) regardless of whether \( \beta + 1 < \alpha \). Since \( \rank(\beta + 1) = \beta + 1 \), it follows that \( \beta + 1 \in V_{\beta + 2} \in \mscrU \) and thus by \ref{def:grothendieck_universe/transitive}, \( \beta + 1 \in \mscrU \). Therefore, \( \alpha \) cannot be a successor ordinal --- if \( \alpha = \beta + 1 \), then \( \beta \in \mscrU \) by definition of \( \alpha \) and thus \( \beta + 1 = \alpha \in \mscrU \), which is a contradiction.

  Since \( \alpha > 0 \), it remains for \( \alpha \) to be a limit ordinal.

  \SubProof*{Proof that \( V_\alpha \subseteq \mscrU \)} By \ref{def:grothendieck_universe/transitive}, \( V_\beta \subseteq \mscrU \) for every \( \beta < \alpha \). We can conclude that
  \begin{equation*}
    V_\alpha
    \reloset {\eqref{eq:def:cumulative_hierarchy}} =
    \bigcup\set{ V_\beta \given \beta < \alpha }
    \subseteq
    U.
  \end{equation*}

  In order to show that equality holds, we must first prove that \( \alpha \) is a strongly inaccessible cardinal. But this requires some auxiliary results.

  \SubProof*{Proof that \( \set{ B } \in \mscrU \) for every \( B \in \mscrU \)} By \ref{def:grothendieck_universe/power_set} we have that \( \pow(\pow(B)) \in \mscrU \). But \( \set{ B } \subseteq \pow(B) \) and hence \( \set{ B } \in \pow(\pow(B)) \). By \ref{def:grothendieck_universe/transitive}, \( \set{ B } \in \mscrU \).

  \SubProof*{Proof that \( \kappa = \alpha \) is a cardinal} Suppose that \( \alpha \) is not a cardinal. Indeed, suppose that there exists some \( \beta < \alpha \) such that there exists a bijective function \( f: \beta \to \alpha \). Then
  \begin{equation*}
    \alpha = \bigcup\set{ \set{ f(\gamma) } \given \gamma < \beta }
  \end{equation*}
  is a \( \beta \)-indexed union of members of \( \alpha \) and hence \( \alpha \in \alpha \). But this contradicts \fullref{thm:simple_foundation_theorems/member_of_itself}.

  Therefore, \( \alpha \) is a cardinal. We will henceforth denote it by \( \kappa \) to highlight that it is a cardinal.

  \SubProof*{Proof that \( \card(B) \in \mscrU \) for every \( B \in \mscrU \)} Let \( B \in \mscrU \) and let \( f: B \to \card(B) \) be a bijective function. Then
  \begin{equation*}
    \card(B)
    =
    f[B]
    =
    \set{ f(x) \given x \in B }
    =
    \bigcup\set[\Big]{ \set{ f(x) } \given x \in B }
  \end{equation*}
   is a \( B \)-indexed union of members of \( \mscrU \) and, by \ref{def:grothendieck_universe/union}, \( \card(B) \in \mscrU \).

  \SubProof*{Proof that \( \kappa \) is a strong limit} For every \( \beta < \kappa \) by \ref{def:grothendieck_universe/power_set} we have \( \pow(\beta) \in \mscrU \). We have already shown that \( \card(\pow(\beta)) \in \mscrU \) and, by \fullref{thm:cardinal_exponentiation_power_set}, we have
  \begin{equation*}
    \card(\pow(\beta)) = 2^{\card(\beta)} = 2^\beta.
  \end{equation*}

  Hence, \( 2^\beta < \kappa \) and \( \kappa \) is a strong limit.

  \SubProof*{Proof that \( \kappa \) is regular} Let \( C \subseteq \kappa \) be an unbounded set. We will show that \( \card(C) = \kappa \).

  Suppose that \( \card(C) < \kappa \). Then \( \card(C) \in \mscrU \) since \( \kappa = \alpha \) is the smallest ordinal not contained in \( \mscrU \). Let \( f: \card(C) \to C \) be a bijective function. Then
  \begin{equation*}
    C = \bigcup\set[\Big]{ \set{ f(\gamma) } \given \gamma < \card(C) }
  \end{equation*}
  and by \ref{def:grothendieck_universe/union}, \( C \in \mscrU \).

  Since \( C \) is unbounded, we have \( \sup C \geq \kappa \). But from \fullref{thm:union_of_set_of_ordinals} is follows that \( \bigcup C = \sup C \) and by \ref{def:grothendieck_universe/union}, \( \bigcup C \in \mscrU \), which implies that \( \sup C < \kappa \).

  The obtained contradiction shows that \( \card(C) = \kappa \). Since \( C \) was an arbitrary unbounded set, it follows that \( \kappa \) satisfies \fullref{def:regular_cardinal/unbounded_subsets} and is thus regular.

  \SubProof*{Proof that \( V_\kappa = U \)} Finally, now that we know that \( \kappa \) is a strongly inaccessible cardinal, we can show that equality holds in \( V_\kappa \subseteq \mscrU \).

  Aiming at a contradiction, suppose that \( U \setminus V_\kappa \) is nonempty. By the \hyperref[def:zfc/foundation]{axiom of foundation}, there exists a set \( C \in \mscrU \setminus V_\kappa \) such that
  \begin{equation*}
    C \cap (U \setminus V_\kappa) = \varnothing,
  \end{equation*}
  thus \( C \subseteq V_\kappa \). From \fullref{thm:strong_regular_cardinal_stage_cardinality} if follows that \( \card(C) < \kappa \) and from \fullref{thm:regular_cardinal_stage_inverse_transitivity} it follows that \( C \in V_\kappa \), which contradicts our choice of \( C \) as a member of \( U \setminus V_\kappa \).

  Therefore, \( V_\kappa = U \).
\end{proof}

\begin{corollary}\label{thm:grothendieck_universe_is_model_of_zfc}
  Every uncountable Grothendieck universe is a standard model of \hyperref[def:zfc]{\logic{ZFC}}.
\end{corollary}
\begin{proof}
  Follows from \fullref{thm:grothendieck_universe_iff_regular_strong_limit} and \fullref{thm:cumulative_hierarchy_model_of_zfc}.
\end{proof}
