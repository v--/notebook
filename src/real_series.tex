\subsection{Real series}\label{subsec:real_series}

\begin{proposition}\label{thm:almost_all_terms_positive_implies_absolute_convergent}
  If only finitely many coefficients in a real convergent\Tinyref{def:convergent_series} series are negative, then the series converges absolutely.
\end{proposition}
\begin{proof}
  Let \( k \) be the index of the last negative coefficient in \cref{def:convergent_series/series}. Then the series
  \begin{equation*}
    \sum_{i={n+1}}^\infty a_i
  \end{equation*}
  is absolutely convergent since every coefficient is positive. Then
  \begin{equation*}
    \sum_{i=0}^\infty \Abs{a_i} = \sum_{i=0}^n \Abs{a_i} + \sum_{i=n+1}^\infty \Abs{a_i}
  \end{equation*}
  is convergent since the first term on the right side is a finite sum and the second is a convergent series. Hence the series \cref{def:convergent_series/series} converges absolutely.
\end{proof}

\begin{corollary}\label{thm:almost_all_terms_negative_implies_absolute_convergent}
  If only finitely many coefficients in a real convergent\Tinyref{def:convergent_series} series are positive, then the series converges absolutely.
\end{corollary}

\begin{proposition}\label{thm:series_absolutely_convergent_iff_positive_and_negative_subseries_convergent}
  The series \cref{def:convergent_series/series} is absolutely convergent if and only if both series
  \begin{equation*}
    \sum_{a_i > 0} a_i
  \end{equation*}
  and
  \begin{equation*}
    \sum_{a_i < 0} a_i
  \end{equation*}
  are convergent.
\end{proposition}

\begin{theorem}[Riemann's series permutation theorem]\label{thm:riemanns_series_permutation_theorem}\cite[\textnumero 247]{Фихтенгольц1968}
  If the real series
  \begin{equation*}
    \sum_{i=0}^\infty a_i
  \end{equation*}
  is convergent\Tinyref{def:convergent_series} but not absolutely convergent, then for any extended real number \( x \in \R \cup \{ -\infty, +\infty \} \) there exists a permutation\Tinyref{def:symmetric_group} \( p \) of the coefficients \( a_0, a_1, a_2 \)
  such that
  \begin{equation*}
    \sum_{i=0}^\infty p(a_i) = x.
  \end{equation*}
\end{theorem}
\begin{proof}
  If the series is not absolutely convergent, then there exist both infinitely many positive and infinitely many negative coefficients.

  First, assume that \( x \) is finite.

  Define the permuted series
  \begin{equation*}
    \sum_{i=0}^\infty b_i
  \end{equation*}
  as follows:
  \begin{algenum}
    \DItem{thm:riemanns_series_theorem/positive} Assign to \( b_n \) only nonnegative elements of the sequence \( \{ a_i \}_{i=0}^\infty \) until \( \sum_{i=0}^n b_i \geq x \). Then go to \cref{thm:riemanns_series_theorem/negative}.
    \DItem{thm:riemanns_series_theorem/negative} Assign to \( b_n \) only negative elements of the sequence \( \{ a_i \}_{i=0}^\infty \) until \( \sum_{i=0}^n b_i \geq x \). Then go to \cref{thm:riemanns_series_theorem/positive}.
  \end{algenum}

  This mutual recursion builds a series that converges to \( x \) because the coefficients \( \{ a_i \}_{i=0}^\infty \) get arbitrarily close to each other.

  If \( x = +\infty \), we can add positive coefficients until \( \sum_{i=0}^n b_i \geq 1 \), then add a single negative coefficient, then continue adding positive coefficients until \( \sum_{i=0}^n b_i \geq 2 \) and so on.

  If \( x = -\infty \), we use the same process but with milestones of \( -1, -2, -3, \ldots \).
\end{proof}
