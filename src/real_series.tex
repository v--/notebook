\subsection{Real series}\label{subsec:real_series}

\begin{proposition}\label{thm:almost_all_terms_positive_implies_absolute_convergent}
  If only finitely many coefficients in a real convergent\Tinyref{def:convergent_series} series are negative, then the series converges absolutely.
\end{proposition}
\begin{proof}
  Let \( k \) be the index of the last negative coefficient in \cref{def:convergent_series/series}. Then the series
  \begin{equation*}
    \sum_{i={n+1}}^\infty a_i
  \end{equation*}
  is absolutely convergent since every coefficient is positive. Then
  \begin{equation*}
    \sum_{i=0}^\infty \Abs{a_i} = \sum_{i=0}^n \Abs{a_i} + \sum_{i=n+1}^\infty \Abs{a_i}
  \end{equation*}
  is convergent since the first term on the right side is a finite sum and the second is a convergent series. Hence the series \cref{def:convergent_series/series} converges absolutely.
\end{proof}

\begin{corollary}\label{thm:almost_all_terms_negative_implies_absolute_convergent}
  If only finitely many coefficients in a real convergent\Tinyref{def:convergent_series} series are positive, then the series converges absolutely.
\end{corollary}

\begin{proposition}\label{thm:series_absolutely_convergent_iff_positive_and_negative_subseries_convergent}
  The series \cref{def:convergent_series/series} is absolutely convergent if and only if both series
  \begin{equation*}
    \sum_{a_i > 0} a_i
  \end{equation*}
  and
  \begin{equation*}
    \sum_{a_i < 0} a_i
  \end{equation*}
  are convergent.
\end{proposition}

\begin{theorem}[Riemann's series permutation theorem]\label{thm:riemanns_series_permutation_theorem}\cite[\textnumero 247]{Фихтенгольц1968/2}
  If the real series
  \begin{equation*}
    \sum_{i=0}^\infty a_i
  \end{equation*}
  is convergent\Tinyref{def:convergent_series} but not absolutely convergent, then for any extended real number \( x \in \R \cup \{ -\infty, +\infty \} \) there exists a permutation\Tinyref{def:symmetric_group} \( p \) of the coefficients \( a_0, a_1, a_2 \)
  such that
  \begin{equation*}
    \sum_{i=0}^\infty p(a_i) = x.
  \end{equation*}
\end{theorem}
\begin{proof}
  If the series is not absolutely convergent, then there exist both infinitely many positive and infinitely many negative coefficients.

  First, assume that \( x \) is finite.

  Define the permuted series
  \begin{equation*}
    \sum_{i=0}^\infty b_i
  \end{equation*}
  as follows:
  \begin{algenum}
    \DItem{thm:riemanns_series_theorem/positive} Assign to \( b_n \) only nonnegative elements of the sequence \( \{ a_i \}_{i=0}^\infty \) until \( \sum_{i=0}^n b_i \geq x \). Then go to \cref{thm:riemanns_series_theorem/negative}.
    \DItem{thm:riemanns_series_theorem/negative} Assign to \( b_n \) only negative elements of the sequence \( \{ a_i \}_{i=0}^\infty \) until \( \sum_{i=0}^n b_i \geq x \). Then go to \cref{thm:riemanns_series_theorem/positive}.
  \end{algenum}

  This mutual recursion builds a series that converges to \( x \) because the coefficients \( \{ a_i \}_{i=0}^\infty \) get arbitrarily close to each other.

  If \( x = +\infty \), we can add positive coefficients until \( \sum_{i=0}^n b_i \geq 1 \), then add a single negative coefficient, then continue adding positive coefficients until \( \sum_{i=0}^n b_i \geq 2 \) and so on.

  If \( x = -\infty \), we use the same process but with milestones of \( -1, -2, -3, \ldots \).
\end{proof}

\begin{proposition}\label{thm:positive_series_comparison}\cite[\textnumero 237]{Фихтенгольц1968/2}
  Fix two nonnegative series
  \begin{equation}\label{def:positive_series_comparison/a}
    \sum_{i=0}^\infty a_i
  \end{equation}
  and
  \begin{equation}\label{def:positive_series_comparison/b}
    \sum_{i=0}^\infty b_i
  \end{equation}
  that is, series with nonnegative real coefficients. Assume that there exists an index \( i_0 \) such that
  \begin{equation*}
    i \geq i_0 \implies a_i \leq b_i.
  \end{equation*}

  Then
  \begin{thmenum}
    \DItem{thm:positive_series_comparison/b_converges} If \cref{def:positive_series_comparison/b} converges, so does \cref{def:positive_series_comparison/a}.

    \DItem{thm:positive_series_comparison/a_diverges} If \cref{def:positive_series_comparison/a} diverges, so does \cref{def:positive_series_comparison/b}.
  \end{thmenum}
\end{proposition}
\begin{proof}
  \begin{description}
    \RItem{thm:positive_series_comparison/b_converges} Suppose that \cref{def:positive_series_comparison/b} converges. Then by \cref{thm:real_monotone_sequence_converges_iff_bounded}, the sequence of partial sums is bounded. Therefore the sequence of partial sums of \cref{def:positive_series_comparison/a} is also bounded and, by \cref{thm:real_monotone_sequence_converges_iff_bounded} again, the series is convergent.

    \RItem{thm:positive_series_comparison/a_diverges} Analogous to \cref{thm:positive_series_comparison/b_converges}, but using the negation of \cref{thm:real_monotone_sequence_converges_iff_bounded}.
  \end{description}
\end{proof}

\begin{proposition}[Cauchy's root test]\label{thm:cauchys_root_test}\cite[theorem 3.33]{Rudin1991}
  Consider the nonnegative series \cref{def:positive_series_comparison/a}. Put
  \begin{equation*}
    q \coloneqq \limsup_{n \to \infty} \sqrt[n]{a_n},
  \end{equation*}
  where \( q = \infty \) if the limit does not exist. Then
  \begin{itemize}
    \item If \( q < 1 \), the series converges absolutely.
    \item If \( q > 1 \), the series diverges.
    \item If the limit does not exist (e.g. if \( a_n = n^n \)), the series diverges.
    \item If \( q = 1 \), the series may either converge or diverge.
  \end{itemize}
\end{proposition}
\begin{proof}
  The case when the limit \( q \) does not exist is obvious by the contraposition to \cref{thm:convergent_series_terms_vanish}.

  Suppose that the limit exists. Therefore there exists an index \( n_0 \) such that
  \begin{equation*}
    n \geq n_0 \implies \sqrt[n]{a_n} \leq q.
  \end{equation*}

  Thus we have the inequality
  \begin{equation*}
    n \geq n_0 \implies a_n \leq q^n.
  \end{equation*}

  Since the geometric series converges when \( q < 1 \) and diverges when \( q > 1 \), the statement of the theorem follows from \cref{thm:positive_series_comparison}.
\end{proof}

\begin{proposition}[d'Alambert's ratio test]\label{thm:dalamberts_ratio_test}\cite[theorem 3.33]{Rudin1991}
  Consider the nonnegative series \cref{def:positive_series_comparison/a}. Put
  \begin{equation*}
    q \coloneqq \limsup_{n \to \infty} \frac {a_{n+1}} {a_n},
  \end{equation*}
  where \( q = \infty \) if the limit does not exist. Then
  \begin{itemize}
    \item If \( q < 1 \), the series converges absolutely.
    \item If \( q > 1 \), the series diverges.
    \item If the limit does not exist (e.g. if \( a_n = n^n \)), the series diverges.
    \item If \( q = 1 \), the series may either converge or diverge.
  \end{itemize}
\end{proposition}
