\subsection{Lipschitz continuity}\label{subsec:lipschitz_continuity}

\begin{definition}\label{def:lipschitz_continuity}
  Let \( f: X \to Y \) be a function between metric spaces.

  \begin{thmenum}
    \labitem{def:lipschitz_continuity/holder} We say that \( f: X \to Y \) is \term{H\"older continuous} at \( x \in X \) with constant \( L \geq 0 \) and exponent \( \alpha > 0 \) if
    \begin{equation*}
      \rho_Y(f(x_1), f(x_2)) \leq L \rho_X(x_1, x_2)^\alpha \quad\forall x_1, x_2 \in X.
    \end{equation*}

    We refer to the smallest such constant, if any, as \enquote{the} H\"older constant.

    \labitem{def:lipschitz_continuity/locally_holder} We say that \( f \) is \term{locally H\"older continuous} if every point has a neighborhood where \( f \) is H\"older continuous with the same exponent, but possibly with with a different constant.

    \labitem{def:lipschitz_continuity/lipschitz} If \( \alpha = 1 \), we say that \( f \) is \term{Lipschitz continuous}.

    \labitem{def:lipschitz_continuity/contraction} If \( X = Y \) and if \( f \) is Lipschitz with constant \( L < 1 \), we call \( f \) a \term{contraction mapping}.

    \labitem{def:lipschitz_continuity/calm}\cite[53]{DontchevRockafellar2014} We say that \( f \) is \term{calm} at \( x \) if it satisfies the Lipschitz condition with one of the points fixed:
    \begin{equation*}
      \rho_Y(f(x), f(x')) \leq L \rho_X(x, x') \quad\forall x' \in X.
    \end{equation*}
  \end{thmenum}
\end{definition}

\begin{proposition}\label{thm:holder_map_is_uniformly_continuous}
  A H\"older map is uniformly continuous.
\end{proposition}
\begin{proof}
  Let \( f: X \to Y \) be a H\"older map with constant \( L \) and exponent \( \alpha \).

  Fix \( \varepsilon > 0 \). Then is enough to choose \( \delta < \sqrt[\alpha]{\frac \varepsilon L} \) so that
  \begin{equation*}
    \rho_X(x_1, x_2) < \delta \implies \rho_Y(f(x_1), f(x_2)) \leq L \rho_X(x_1, x_2)^\alpha < L \delta^\alpha < \varepsilon.
  \end{equation*}

  This implies uniform continuity.
\end{proof}

\begin{corollary}\label{thm:locally_holder_map_is_continuous}
  A locally H\"older map is continuous.
\end{corollary}

\begin{theorem}[Banach's fixed point theorem]\label{thm:banach_fixed_point_theorem}\mcite[exer. 4.3.J]{Engelking1989}
  A contraction \hyperref[def:lipschitz_continuity/contraction]{mapping} in a \hyperref[def:complete_metric_space]{complete metric space} has a unique fixed \hyperref[def:fixed_point]{point}.
\end{theorem}
\begin{proof}
  Let \( f: X \to X \) be a contraction mapping. Fix any point \( x_0 \in X \) and inductively define the sequence
  \begin{equation*}
    x_{k+1} \coloneqq f(x_k), k = 1, 2, \ldots.
  \end{equation*}

  Fix \( \varepsilon > 0 \). Since \( L < 1 \), there exists an index \( k_0 > \log_L(\varepsilon) \) such that for positive integers \( m \) and \( k > k_0 \),
  \begin{balign*}
    \rho(x_k, x_{k+m})
     & =
    \rho(f^k(x_0), f^{k+m}(x_0))
    \leq \\ &\leq
    L^k \rho(x_0, x_m)
    <    \\ &<
    \varepsilon \rho(x_0, x_m).
  \end{balign*}

  Note that
  \begin{balign*}
    \rho(x_0, x_m)
     & \leq
    \sum_{i=1}^m \rho(x_{i-1}, x_i)
    \leq    \\ &\leq
    \rho(x_0, x_1) \sum_{i=1}^m L^{i-1}
    =       \\ &=
    \rho(x_0, x_1) \frac {1 - L^m} {1 - L}
    \leq    \\ &\leq
    \rho(x_0, x_1) \frac 1 {1 - L}.
  \end{balign*}

  Thus
  \begin{equation*}
    \rho(x_k, x_{k+m}) < \frac {\varepsilon \rho(x_0, x_1)} {1 - L}.
  \end{equation*}

  The constant on the right is linear in \( \varepsilon \) and does not depend on \( k \) or \( m \), hence \( \{ x_k \}_{k=0}^\infty \) is a fundamental sequence. Since \( X \) is complete, the sequence has a limit \( x \).

  Because of the continuity of \( f \) (see \fullref{thm:holder_map_is_uniformly_continuous}),
  \begin{equation*}
    f(x) = f(\lim_{k \to \infty} x_k) = \lim_{k \to \infty} f(x_k) = \lim_{k \to \infty} x_{k+1} = x.
  \end{equation*}
\end{proof}
