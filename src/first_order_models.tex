\subsection{First-order models}\label{subsec:first_order_models}

\begin{definition}\label{def:first_order_substructure}
  Let \( (\mscrX, I) \) be a structure for the language \( \mscrL \) and let \( \mscrY \subseteq \mscrX \). We say that \( (\mscrY, I) \) is a \term{substructure} of \( (\mscrX, I) \) if it satisfies any of the following equivalent conditions:

  \begin{thmenum}
    \thmitem{def:first_order_substructure/deductive} If \( \mscrY \) is closed under function application, that is, for any functional symbol \( f \) in \( \mscrL \) with arity \( n \), we have \( I(f)(\mscrY^n) \subseteq \mscrY \).

    \thmitem{def:first_order_substructure/inductive} \( \mscrY \) is a \hyperref[def:fixed_point]{fixed point} of the operator
    \begin{alignedeq}\label{eq:def:first_order_substructure/inductive/operator}
      &T: \pow(\mscrX) \to \pow(\mscrX) \\
      &T(A) \coloneqq A \cup \set*{ x \in \mscrX \given \qexists{f \in \boldop{Fun}} \qexists{x_1, \ldots, x_{\#f} \in A} f\Bracks{x_1, \ldots, x_{\#f}} = x },
    \end{alignedeq}
    which enlarges \( A \) with the union of all image of \( A \) under functions of the language \( \mscrL \).

    Note that the formula inside \eqref{eq:def:first_order_substructure/inductive/operator} is in the metalanguage despite using syntax similar to first-order logic formulas.
  \end{thmenum}
\end{definition}
\begin{proof}
  By definition of \( T \), \( \mscrY \) if a fixed point if and only if
  \begin{equation*}
    \set*{ x \in \mscrX \given \qexists{f \in \boldop{Fun}} \qexists{x_1, \ldots, x_{\#f} \in A} f\Bracks{x_1, \ldots, x_{\#f}} = x } \subseteq \mscrY.
  \end{equation*}

  This condition is clearly satisfied if \( B \) satisfies \fullref{def:first_order_substructure/deductive}.

  If, instead \( \mscrY \) is a fixed point of \( T \), for the \( n \)-ary functional symbol \( f \in \boldop{Fun} \) and for any tuple \( x_1, \ldots, x_n \), the value \( I(f)(x_1, \ldots, x_n) \) belongs to \( \mscrY \). Therefore \fullref{def:first_order_substructure/deductive} is satisfied.
\end{proof}

\begin{proposition}\label{thm:first_order_substructure_properties}
  \hyperref[def:first_order_substructure]{First order substructures} of \( (\mscrX, I) \) over \( \mscrL \) have the following basic properties:
  \begin{thmenum}
    \thmitem{thm:first_order_substructure_properties/intersection} Let \( \{ \mscrY_k \}_{k \in \mscrK} \) be a family of substructures of \( \mscrX \). Then their \enquote{intersection} structure \( \parens*{\bigcap_{k \in \mscrK} \mscrY_k, I} \) is again a substructure of \( \mscrX \).
  \end{thmenum}
\end{proposition}
\begin{proof}
  \SubProofOf{thm:first_order_substructure_properties/intersection} For any functional symbol \( f \) in \( \mscrL \) with arity \( n \), we have
  \begin{equation*}
    I(f)\parens*{\parens*{\bigcap_{\smash{k \in \mscrK}} \mscrY_k}^n}
    \overset {\ref{thm:function_image_properties/intersection}} \subseteq
    \bigcap_{k \in \mscrK} I(f)(\mscrY_k^n) \subseteq \bigcap_{k \in \mscrK} \mscrY_k.
  \end{equation*}

  Therefore \( \parens*{\bigcap_{k \in \mscrK} \mscrY_k, I} \) is indeed a substructure of \( (\mscrX, I) \).
\end{proof}

\begin{definition}\label{def:first_order_generated_substructure}
  Let \( (\mscrX, I) \) be a structure over \( \mscrL \) and let \( A \subseteq \mscrX \) be any set. The set \( A \) is said to \term[bg=поражда,ru=порождает]{generate} the substructure \( \mscrY = (\mscrY, I) \) if it satisfies any of the equivalent statements:
  \begin{thmenum}
    \thmitem{def:first_order_generated_substructure/smallest} Out of all substructures of \( \mscrX \) whose universum contains \( A \), the universum \( \mscrY \) is the smallest with respect to \hyperref[def:subset]{set inclusion}.

    \thmitem{def:first_order_generated_substructure/intersection} \( \mscrY \) is the intersection of the universums of all substructures of \( \mscrX \) that contain \( A \).
  \end{thmenum}
\end{definition}
\begin{proof}
  Let \( \{ (\mscrY_k, I) \}_{k \in \mscrK} \) be the family of all substructures of \( (\mscrX, I) \) whose universums contain \( A \). Fix one of these substructures, say \( (\mscrY_{k_0}, I) \).

  We have the obvious inclusion
  \begin{equation*}
    \bigcap_{k \in \mscrK} \mscrY_k \subseteq \mscrY_{k_0}.
  \end{equation*}

  The reverse inclusion holds if and only if \( \mscrY_{k_0} \) is contained in each one of universums \( \mscrY_k \) for \( k \in \mscrK \). In other words, \( \mscrY_{k_0} \) is the smallest of the universums \( \set{ \mscrY_k }_{k \in \mscrK} \) with respect to set inclusion if and only if \( \mscrY_{k_0} \) equals their intersection.
\end{proof}

\begin{definition}\label{def:first_order_homomorphism}\mcite[def. 23.8]{OpenLogic20201202}
  Let \( (\mscrX, I_\mscrX) \) and \( (\mscrY, I_\mscrY) \) be structures over a common language. We say that the \hyperref[def:function]{function} \( h: \mscrX \to \mscrY \) is a \term{strong homomorphism} or simply \term{homomorphism} between \( \mscrX \) and \( \mscrY \) if it preserves all functions and relations. Explicitly:
  \begin{thmenum}
    \thmitem{def:first_order_homomorphism/functions} For any functional symbol \( f \in \boldop{Fun} \) of arity \( n \) and any tuple \( x_1, \ldots, x_n \in \mscrX \) we have
    \begin{equation*}
      h\parens[\Big]{ I_\mscrX(f)(x_1, \ldots, x_n) } = I_\mscrY(f) \parens[\Big]{ h(x_1), \ldots, h(x_n) }
    \end{equation*}

    \thmitem{def:first_order_homomorphism/predicates} For any predicate symbol \( p \in \boldop{Pred} \) of arity \( n \) and any \( x_1, \ldots, x_n \in \mscrX \),
    \begin{equation*}
      (x_1, \ldots, x_n) \in I_\mscrX(p) \T{if and only if} \parens[\Big]{ f(x_1), \ldots, f(x_n) } \in I_\mscrY(p).
    \end{equation*}

    \thmitem{def:first_order_homomorphism/predicates_weak} We can weaken \ref{def:first_order_homomorphism/predicates} to
    \begin{equation*}
      (x_1, \ldots, x_n) \in I_\mscrX(p) \T{implies} \parens[\Big]{ f(x_1), \ldots, f(x_n) } \in I_\mscrY(p),
    \end{equation*}
    in which case we obtain the notion of a \term{weak homomorphism}. Weak homomorphisms are used in e.g. \fullref{def:graph_homomorphism}.
  \end{thmenum}
\end{definition}

\begin{proposition}\label{thm:first_order_homomorphism_properties}
  \hyperref[def:first_order_homomorphism]{First-order structure homomorphisms} have the following basic properties:
  \begin{thmenum}
    \thmitem{thm:first_order_homomorphism_properties/substructure} If \( (\mscrX, I) \) is a structure and \( \mscrY = (\mscrY, I) \) is a \hyperref[def:first_order_substructure]{substructure} of \( \mscrX \), then the \term{canonical embedding} function
    \begin{equation}\label{thm:first_order_homomorphism_properties/substructure/canonical_embedding}
      \begin{aligned}
        &\iota: \mscrY \to \mscrX \\
        &\iota(y) \coloneqq y
      \end{aligned}
    \end{equation}
    is indeed a \hyperref[def:first_order_homomorphism_invertibility/projection]{homomorphism} (and thus an embedding in the sense of \fullref{def:first_order_homomorphism_invertibility}).

    \thmitem{thm:first_order_homomorphism_properties/preserves_substructures} If \( \mscrX = (\mscrX, I_\mscrX) \) and \( \mscrY = (\mscrY, I_\mscrY) \) are structures and \( h: \mscrX \to \mscrY \) is a (weak or strong) homomorphism, then the \hyperref[def:function/image]{image} \( (f(\mscrX), I_\mscrY) \) is a substructure of \( (\mscrY, I_\mscrY) \).

    \thmitem{thm:first_order_homomorphism_properties/composition} The \hyperref[def:function/composition]{composition} of two homomorphisms (resp. weak homomorphisms) is again a homomorphism (resp. weak homomorphism).
  \end{thmenum}
\end{proposition}
\begin{proof}
  \SubProofOf{thm:first_order_homomorphism_properties/substructure} The interpretation in the substructure \( \mscrY \) is the \hyperref[def:function/extension]{restriction} \( I\restr_\mscrY \) of \( I \) to \( \mscrY \) which simply restricts the domain of any predicate and function is indeed an interpretation in \( \mscrY \). Thus \( (\mscrY, I\restr_\mscrY) \) is a structure.

  Conditions \fullref{def:first_order_homomorphism/functions} and \fullref{def:first_order_homomorphism/predicates} are both satisfied since the interpretation of any function and predicate is restricted to \( \mscrY \). Thus \( \iota \) is a homomorphism.

  \SubProofOf{thm:first_order_homomorphism_properties/preserves_substructures} To prove that \( (f(\mscrX), I_\mscrY) \) is a substructure of \( (\mscrY, I_\mscrY) \), we will show that \fullref{def:first_order_substructure/deductive} holds.

  Indeed, by \fullref{def:first_order_homomorphism/functions}, for any \( n \)-ary functional symbol and any tuple \( {x_1, \ldots, x_n \in \mscrX} \), we have that
  \begin{equation*}
    I_\mscrY(f) \parens[\Big]{ h(x_1), \ldots, f(x_n) }
    \overset {\ref{def:first_order_homomorphism/functions}} =
    h\parens[\Big]{ I_\mscrX(f)(x_1, \ldots, x_n) }
    \overset {\ref{def:first_order_substructure/deductive}} \in
    h(\mscrX).
  \end{equation*}

  Since no predicates are involved in the condition of \( (h(\mscrX), I_\mscrY) \) to be a substructure of \( (\mscrY, I_\mscrY) \), this proof holds for both weak and strong homomorphisms.

  \SubProofOf{thm:first_order_homomorphism_properties/composition} Let \( h: \mscrX \mapsto \mscrY \) and \( l: \mscrY \mapsto \mscrZ \) both be homomorphisms.

  \begin{itemize}
    \item \Fullref{def:first_order_homomorphism/functions} is satisfied because for any \( n \)-ary functional symbol \( f \) and any tuple \( x_1, \ldots, x_n \in \mscrX \),
    \begin{align*}
      &\phantom{{}={}}
      (l \bincirc h) \parens[\Big]{ I_\mscrX(f)(x_1, \ldots, x_n) }
      \overset {\ref{def:first_order_homomorphism/functions}} = \\ &=
      l\parens[\Big]{ I_\mscrY(f) \parens[\Big]{ h(x_1), \ldots, h(x_n) } }
      \overset {\ref{def:first_order_homomorphism/functions}} = \\ &=
      I_{\mscrZ}(f) \parens[\Big]{ (l \bincirc h)(x_1), \ldots, (l \bincirc h)(x_n) }.
    \end{align*}

    \item \Fullref{def:first_order_homomorphism/predicates} is satisfied because for any \( n \)-ary predicate symbol \( p \) and any tuple \( x_1, \ldots, x_n \in \mscrX \),
    \begin{align*}
      &\phantom{{}\iff{}}
      (x_1, \ldots, x_n) \in I_\mscrX(p)
      \overset {\ref{def:first_order_homomorphism/predicates}} \iff \\ &\iff
      \parens[\Big]{ h(x_1), \ldots, h(x_n) } \in I_\mscrY(p)
      \overset {\ref{def:first_order_homomorphism/predicates}} \iff \\ &\iff
      \parens[\Big]{ (l \bincirc h)(x_1), \ldots, (l \bincirc h)(x_n) } \in I_{\mscrZ}(p).
    \end{align*}
  \end{itemize}
\end{proof}

\begin{definition}\label{def:first_order_homomorphism_invertibility}
  We introduce the following terminology (compare to \fullref{def:function_invertibility} and \fullref{def:morphism_invertibility}):
  \begin{thmenum}
    \thmitem{def:first_order_homomorphism_invertibility/embedding} An \term{embedding}, also called a \term{monomorphism}, is an \hyperref[def:function_invertibility/injection]{injective} homomorphism.

    \thmitem{def:first_order_homomorphism_invertibility/projection} Dually, a \term{projection}, also called an \term{epimorphism}, is an \hyperref[def:function_invertibility/surjection]{surjective} homomorphism.

    \thmitem{def:first_order_homomorphism_invertibility/isomorphism} An \term{isomorphism} is a \hyperref[def:function_invertibility/bijection]{bijective} homomorphism.

    \thmitem{def:first_order_homomorphism_invertibility/endomorphism} An \term{endomorphism} is a homomorphism that is also an \hyperref[def:endofunction]{endofunction}.

    \thmitem{def:first_order_homomorphism_invertibility/automorphism} A homomorphism that is both an endomorphism and an isomorphism is called an \term{automorphism}.
  \end{thmenum}
\end{definition}

\begin{proposition}\label{thm:first_order_homomorphism_preserves_models}\mcite[def. 23.8]{OpenLogic20201202}
  Let \( (\mscrX, I_\mscrX) \) and \( (\mscrY, I_\mscrY) \) be structures over a common language \( \mscrL \) and let \( h: \mscrX \to \mscrY \) be a homomorphism between them. Let \( \Gamma \) be a set of formulas.

  If \( (\mscrX, I_\mscrX) \models \Gamma \), then \( (h(\mscrX), I_\mscrY) \models \Gamma \).
\end{proposition}
\begin{proof}
  Let \( (\mscrX, I) \) be a model of \( \varphi \). Let \( \varphi \) be a formula in \( \mscrL \) such that
  \begin{equation*}
    \boldop{Free}(\varphi) \subseteq \set{x_1, \ldots, x_n}.
  \end{equation*}

  Using induction\IND on the structure of \( \varphi \), it is straightforward to show that \( \varphi\Bracks{x_1, \ldots, x_n} = \varphi\Bracks{h(x_1), \ldots, h(x_n)} \).

  We conclude that, if \( (\mscrX, I_\mscrX) \models \varphi \), we have \( (h(\mscrX), I_\mscrY) \models \varphi \).

  Since the formula \( \varphi \) was an arbitrary member of \( \Gamma \), the statement of the proposition follows.
\end{proof}

\begin{corollary}\label{thm:substructure_is_model}
  Any \hyperref[def:first_order_substructure]{substructure} of a model of \( \Gamma \) is again a model of \( \Gamma \).
\end{corollary}
\begin{proof}
  Follows from \fullref{thm:first_order_homomorphism_properties/substructure} and \fullref{thm:first_order_homomorphism_preserves_models}.
\end{proof}

\begin{theorem}\label{thm:functions_over_model_form_model}
  Let \( \Gamma \) be a set of \hyperref[thm:semantic_implicit_quantification]{universal formulas} over \( \mscrL \). Let \( (\mscrX, I) \) be a model of \( \Gamma \) and let \( \mscrS \) be any nonempty set. Consider the set \( \fun(\mscrS, \mscrX) \) of \hyperref[def:function/single_valued]{all set-theoretic functions} from \( \mscrS \) to \( \mscrX \).

  Define \( \iota: \mscrX \mapsto \fun(\mscrS, \mscrX) \) by sending each \( x \in \mscrX \) to the corresponding constant function in \( \fun(\mscrS, \mscrX) \).

  Define the interpretation \( \widetilde I \) on \( \fun(\mscrS, \mscrX) \) as follows:
  \begin{itemize}
    \item For each \( n \)-ary functional symbol \( f \) in \( \mscrL \), define the function
    \begin{equation*}
      \begin{aligned}
        &\widetilde I(f): \fun(\mscrS, \mscrX)^n \to \fun(\mscrS, \mscrX) \\
        &\widetilde I(f) \parens[\Big]{ k_1, \ldots, k_n } \coloneqq \parens[\Big]{ s \mapsto I(f) \parens[\Big]{ k_1(s), \ldots, k_n(s) } }.
      \end{aligned}
    \end{equation*}

    \item For each \( n \)-ary predicate symbol \( p \) in \( \mscrL \), define \( \widetilde I(p) \subseteq \fun(\mscrS, \mscrX)^n \) via
    \begin{equation*}
      \parens[\Big]{ k_1, \ldots, k_n } \in \widetilde I(p) \T{if and only if for each} s \in \mscrS \T{we have} \parens[\Big]{ k_1(s), \ldots, k_n(s) } \in I(p).
    \end{equation*}
  \end{itemize}

  Then the structure \( (\fun(\mscrS, \mscrX), \widetilde I) \) is also a model of \( \Gamma \) and \( \iota: \mscrX \to \fun(\mscrS, \mscrX) \) is a strong embedding.
\end{theorem}

\begin{definition}\label{def:first_order_model_category}
  Let \( \mscrL \) be a first-order language and let \( \Gamma \) be a set of formulas. We describe a \hyperref[def:concrete_category]{concrete category} which we will call the \term{model category} for \( \Gamma \).

  \begin{refenum}
    \refitem{def:category/C1} The \hyperref[def:set_zfc]{class} of objects is the class of all models of \( \Gamma \).
    \refitem{def:category/C2} The morphisms between two models are the \hyperref[def:first_order_homomorphism]{homomorphisms} between them.
    \refitem{def:category/C3} Composition of morphisms is the usual \hyperref[def:function/composition]{function composition}.
  \end{refenum}
\end{definition}
