\subsection{Logical theories}\label{subsec:logical_theories}

\begin{definition}\label{def:first_order_theory}\mcite[def. 17.1]{OpenLogicFull}
  The \term{closure} of the set \( \Gamma \) of formulas in the \hyperref[def:first_order_syntax]{first-order language} \( \mscrL \) is the set
  \begin{equation*}
    \cl(\Gamma) \coloneqq \set{ \varphi \in \boldop{Form} \given \Gamma \vDash \varphi }.
  \end{equation*}

  Due to \fullref{rem:propositional_logic_as_first_order_logic}, this definition also holds for formulas in propositional logic.

  The set \( \Gamma \) is called \term{closed} if it equals its own closure. A closed sets of formulas is also sometimes called a \term{theory} in e.g. \cite[def. 33.1]{OpenLogicFull} but we will not put this restriction because it is not clear what are the axioms of an arbitrary closed set of formulas. Instead, we will refer to the set of axioms itself as a theory.

  If it is necessary to distinguish between derivability and entailment, we can use the syntactic closure \( \cl^\vdash \) and semantic closure \( \cl^\vDash \).

  \begin{thmenum}
    \thmitem{def:first_order_theory/axiomatized}\mcite[def. 17.1]{OpenLogicFull} We say that the set \( \Gamma \) is \term{axiomatized} by \( \Delta \) if
    \begin{equation*}
      \Gamma = \set{ \varphi \in \boldop{Form} \given \Delta \vDash \varphi }.
    \end{equation*}

    \thmitem{def:first_order_theory/complete}\mcite[def. 33.7]{OpenLogicFull} The set \( \Gamma \) of formulas is said to be \term{complete} if every for every formula in \( \varphi \), either \( \Gamma \vDash \varphi \) or \( \Gamma \vDash \neg \varphi \).

    This is not to be confused with \fullref{def:derivability_and_satisfiability/completeness}, which defines completeness of derivation systems via how it relates to semantics.

    \thmitem{def:first_order_theory/consistent}\mcite[def. 22.6]{OpenLogicFull} The set \( \Gamma \) of formulas is said to be \term[bg=противоречива,ru=противоречивая]{inconsistent} if \( \Gamma \vDash \bot \) and \term{consistent} otherwise.

    In \hyperref[def:intuitionistic_logic]{intuitionistic logic}, every inconsistent theory obviously matches the set \( \boldop{Form} \) of all formulas because of \eqref{eq:thm:minimal_propositional_negation_laws/efq}.
  \end{thmenum}
\end{definition}

\begin{proposition}\label{thm:formulas_satisfiable_iff_consistent}
  Assuming classical logic, a set \( \Gamma \) of formulas is \hyperref[def:propositional_semantics/satisfiability]{satisfiable} if and only if it is \hyperref[def:first_order_theory/consistent]{consistent}.
\end{proposition}
\begin{proof}
  We will prove that \( \Gamma \) is unsatisfiable if and only if it is inconsistent.

  \SufficiencySubProof Assume first that \( \Gamma \) is unsatisfiable. Then for all zero models of \( \Gamma \) we can derive any universal statement, in particular that any model of \( \Gamma \) satisfies \( \bot \). Thus \( \Gamma \) is inconsistent.

  \NecessitySubProof Let \( \Gamma \) be inconsistent and suppose that \( \mscrX = (X, I) \) is a model of \( \Gamma \). Fix any valuation \( v \) in \( \mscrX \). Since \( \Gamma \vDash \bot \), obviously \( \bot\Bracks{v} = T \).

  But by \fullref{def:first_order_valuation/formula_valuation}, we have that \( \bot\Bracks{v} = F \). The obtained contradiction shows that \( \mscrX \) cannot be a model of \( \Gamma \) and since this structure was chosen arbitrarily, we conclude that \( \Gamma \) is unsatisfiable.
\end{proof}

\begin{definition}\label{def:category_of_first_order_models}
  Let \( \mscrL \) be a first-order language and let \( \Gamma \) be a nonempty set of \hyperref[def:positive_formula]{positive formulas}. We describe the \term{category of models} for \( \Gamma \) as the \hyperref[def:concrete_category]{concrete category} with the following structure:
  \begin{itemize}
    \item The \hyperref[def:category/C1]{class of objects} is the class of all models of \( \Gamma \).

    \item The \hyperref[def:category/C2]{morphisms} between two models are the \hyperref[def:first_order_homomorphism]{structure homomorphisms} between them.

    \item The \hyperref[def:category/C3]{composition of morphisms} is the usual \hyperref[def:multi_valued_function/composition]{function composition}.
  \end{itemize}
\end{definition}
\begin{proof}
  We know that any morphism in this category is defined between two models, however in general homomorphisms are not injective and the homomorphic image of a model may fail to be a model. \Fullref{thm:positive_formulas_preserved_under_homomorphism} shows that if all formulas in \( \Gamma \) are positive, the image of any homomorphism is again a model and thus an object in the category.
\end{proof}

\begin{example}\label{ex:def:category_of_first_order_models}
  This is an incomplete list of categories corresponding to \hyperref[def:first_order_theory]{first-order theories} that can be found in this document:
  \begin{itemize}
    \item The categories \hyperref[def:pointed_set/category]{\( \cat{Set}_* \)}, \hyperref[def:set_with_involution/category]{\( \cat{Inv} \)}, \hyperref[def:magma/category]{\( \cat{Mag} \)}, \hyperref[def:unital_magma/category]{\( \cat{Mag}_* \)}, \hyperref[def:unital_magma/associative]{\( \cat{Mon} \)}, \hyperref[def:group/category]{\( \cat{Grp} \)} and \hyperref[def:abelian_group]{\( \cat{Ab} \)}, whose relations are crucial for the definition and properties of groups.

    \item The categories \hyperref[def:semiring/ring]{\( \cat{Ring} \)} of rings, \hyperref[def:semiring/ring]{\( \cat{Mod}_\mscrR \)} of modules and \hyperref[def:vector_space]{\( \cat{Vect}_\BbbK \)} of vector spaces, which are both based on abelian groups.

    \item The categories \hyperref[def:category_of_topological_groups]{\( \cat{TopGrp} \)} and \hyperref[def:category_of_topological_groups]{\( \cat{TopVect}_\BbbK \)} of topological groups and vector spaces, respectively.

    \item The categories \hyperref[def:poset/category]{\( \cat{Pos} \)} in order theory and the related \hyperref[def:semilattice/category]{\( \cat{Lat} \)}, \hyperref[def:heyting_algebra/category]{\( \cat{Heyt} \)} and \hyperref[def:boolean_algebra/category]{\( \cat{Bool} \)} in lattice theory.

    \item The categories \hyperref[def:category_of_graphs]{\( \cat{DGraph} \)} and \hyperref[def:category_of_graphs]{\( \cat{UGraph} \)} of directed and undirected graphs.
  \end{itemize}

  In contrast:
  \begin{itemize}
    \item We define the category \hyperref[def:category_of_topological_spaces]{\( \cat{Top} \)}  of topological spaces and all of its related categories within set theory without a corresponding first-order theory.

    \item The category \hyperref[def:category_of_sets]{\( \cat{Set} \)} of sets with respect to either ZFC or na\"ive set theory is not the same as the category of models of set theory. Instead, it is a category within a fixed set theory. Within the metalogic of this document, we work within a fixed model of ZFC with respect to \hyperref[def:classical_logic]{classical logic}.

    \item Similarly, we do not care about models of \hyperref[def:peano_arithmetic]{Peano arithmetic} enough to study its category of models. Instead, we only use a single model and denote it by \( \BbbN \).
  \end{itemize}
\end{example}

\begin{definition}\label{def:lindenbaum_tarski_algebra}\mcite{nLab:lindenbaum_tarski_algebra}
  Assume some fixed \hyperref[def:proof_derivation_system]{derivation system} in propositional or first-order logic. Let \( \Gamma \) be a \hyperref[def:first_order_theory]{closed set of formulas} within the corresponding logic.

  Then \( (\Gamma, \vdash) \) is a \hyperref[def:preordered_set]{preordered set}. The \term{Lindenbaum-Tarski algebra} of the theory \( \Gamma \) is the poset obtained from \( (\Gamma, \vdash) \) using \fullref{thm:preorder_to_partial_order}.

  More concretely, the Lindenbaum-Tarski algebra of \( \Gamma \) is a quotient set of \( \Gamma \) by the relation \hyperref[def:derivation_system_derivability]{interderivability} and endowed with the partial order
  \begin{equation}\label{eq:def:lindenbaum_tarski_algebra/order}
    [\varphi] \leq [\psi] \T{if and only if} \varphi \vdash \psi.
  \end{equation}

  Of course, we can define the algebra using entailment rather than derivability but in the cases we consider, the two are equivalent and derivability is simpler to work with.
\end{definition}
\begin{proof}
  The correctness of \eqref{eq:def:lindenbaum_tarski_algebra/order}, i.e. the fact that the relation \( \leq \) does not depend on the choice of representatives from the quotient sets, follows from \fullref{thm:preorder_to_partial_order}.

  We must only demonstrate that \( (\Gamma, \vdash) \) is indeed a preordered set. Reflexivity of \( \vdash \) follows from \fullref{def:derivation_system_derivability/nonlogical_axiom} and transitivity follows from \fullref{thm:derivation_system_transitivity}.
\end{proof}

\begin{proposition}\label{thm:intuitionistic_lindenbaum_tarski_algebra}
  Assume that we are working in the \hyperref[def:intuitionistic_propositional_derivation_system]{intuitionistic propositional derivation system}. The \hyperref[def:lindenbaum_tarski_algebra]{Lindenbaum-Tarski algebra} of a closed set of formulas \( \Gamma \) then is a \hyperref[def:heyting_algebra]{Heyting algebra}.

  In the \hyperref[def:propositional_derivation_system]{classical derivation system}, the algebra is instead a \hyperref[def:boolean_algebra]{Boolean algebra}.

  In the \hyperref[def:minimal_propositional_derivation_system]{minimal derivation system}, we have an unbounded lattice with only a top element but no bottom. Consequently, conditionals and pseudocomplements may fail to exist.

  Explicitly:
  \begin{thmenum}
    \thmitem{thm:intuitionistic_lindenbaum_tarski_algebra/join} The \hyperref[def:semilattice/join]{join} of the equivalence classes \( [\psi_1] \) and \( [\psi_2] \) is the class \( [\psi_1 \vee \psi_2] \) of their \hyperref[def:propositional_language/connectives/disjunction]{disjunction}.

    \thmitem{thm:intuitionistic_lindenbaum_tarski_algebra/top} The class of \hyperref[def:propositional_semantics/tautology]{tautologies} \( [\top] \) is the \hyperref[def:poset_extremal_points/top_and_bottom]{top element}.

    \thmitem{thm:intuitionistic_lindenbaum_tarski_algebra/meet} Similarly to joins, the \hyperref[def:semilattice/meet]{meet} of \( [\psi_1] \) and \( [\psi_2] \) is the equivalence class \( [\psi_1 \wedge \psi_2] \) of their \hyperref[def:propositional_language/connectives/conjunction]{conjunction}.

    \thmitem{thm:intuitionistic_lindenbaum_tarski_algebra/bottom} The class of \hyperref[def:propositional_semantics/contradiction]{contradictions} \( [\bot] \) is the \hyperref[def:poset_extremal_points/top_and_bottom]{bottom element}.

    \thmitem{thm:intuitionistic_lindenbaum_tarski_algebra/conditional} The \hyperref[eq:def:heyting_algebra/conditional]{conditional} of \( [\psi_1] \) and \( [\psi_2] \) is the equivalence class \( [\psi_1 \rightarrow \psi_2] \).

    \thmitem{thm:intuitionistic_lindenbaum_tarski_algebra/complement} The \hyperref[eq:def:heyting_algebra/pseudocomplement]{pseudocomplement} \( \widetilde{[\psi]} \) of \( [\psi] \) equals \( [\neg \psi] \).

    In the classical derivation system this pseudocomplement is a complement, i.e. it satisfies \eqref{eq:def:boolean_algebra/complement/join} and \eqref{eq:def:boolean_algebra/complement/meet}.
  \end{thmenum}
\end{proposition}
\begin{proof}
  \SubProofOf{thm:intuitionistic_lindenbaum_tarski_algebra/join} We will show that \( [\psi_1 \vee \psi_2] \) is the supremum of \( [\psi_1] \) and \( [\psi_2] \).

  The inference rule \eqref{eq:thm:natural_deduction/or/intro_left} implies that \( \psi_1 \vdash \psi_1 \vee \psi_2 \) and \eqref{eq:thm:natural_deduction/or/intro_right} implies that \( \psi_1 \vdash \psi_1 \vee \psi_2 \). Thus \( \psi_1 \vee \psi_2 \) is an upper bound for both \( \psi_1 \) and \( \psi_2 \) under the ordering \( \vdash \).

  Let \( \varphi \) be any formula in \( \Gamma \) such that \( \psi_1 \vdash \varphi \), \( \psi_2 \vdash \varphi \) and \( \varphi \vdash (\psi_1 \vee \psi_2) \). Then the following instance of \eqref{eq:thm:natural_deduction/or/elim}
  \begin{equation*}
    \begin{prooftree}
      \hypo{ [\psi_1 \vee \psi_2] }
      \hypo{ [\psi_1]^n }
      \ellipsis {} { \varphi }
      \hypo{ [\psi_2]^n }
      \ellipsis {} { \varphi }
      \infer[left label=\( n \)]3[\ref{eq:thm:natural_deduction/or/elim}]{ \varphi }
    \end{prooftree}
  \end{equation*}
  demonstrates that \( \psi_1, \psi_2, (\psi_1 \vee \psi_2) \vdash \varphi \). Hence from \fullref{thm:derivation_system_transitivity} it follows that \( (\psi_1 \vee \psi_2) \vdash \varphi \).

  Obviously interderivability is not affected by the choice of representatives from \( [\psi_1] \) and \( [\psi_2] \), hence \( [\varphi] = [\psi_1 \vee \psi_2] \) and this is indeed the supremum of \( [\psi_1] \) and \( [\psi_2] \).

  \SubProofOf{thm:intuitionistic_lindenbaum_tarski_algebra/top} The inference rule \eqref{eq:thm:natural_deduction/top} shows that \( [\varphi] \leq [\top] \) for any formula \( \varphi \) and \( [\top] \) is the top element.

  \SubProofOf{thm:intuitionistic_lindenbaum_tarski_algebra/meet} Let \( \psi_1 \) and \( \psi_2 \) be any formulas in \( \Gamma \). The inference rule \eqref{eq:thm:natural_deduction/and/elim_left} implies that \( \psi_1 \wedge \psi_2 \vdash \psi_1 \) and \eqref{eq:thm:natural_deduction/and/elim_right} implies that \( \psi_1 \wedge \psi_2 \vdash \psi_2 \). Thus \( \psi_1 \wedge \psi_2 \) is a lower bound for both \( \psi_1 \) and \( \psi_2 \) under the ordering \( \vdash \).

  We must show that \( \psi_1 \wedge \psi_2 \) is interderivable with the greatest lower bound. Let \( \varphi \) be a formula in \( \Gamma \) such that \( \varphi \vdash \psi_1 \), \( \varphi \vdash \psi_2 \) and \( (\psi_1 \wedge \psi_2) \vdash \varphi \). We will show that \( \varphi \vdash (\psi_1 \wedge \psi_2) \).

  The rule \eqref{eq:thm:natural_deduction/and/intro} implies that
  \begin{equation*}
    \psi_1, \psi_2 \vdash \psi_1 \wedge \psi_2.
  \end{equation*}

  But \( \varphi \) derives both \( \psi_1 \) and \( \psi_2 \), hence by \fullref{thm:derivation_system_transitivity}
  \begin{equation*}
    \varphi \vdash (\psi_1 \wedge \psi_2).
  \end{equation*}

  Analogously to the proof of \fullref{thm:intuitionistic_lindenbaum_tarski_algebra/join}, we conclude that the choice of representatives of the equivalence classes is irrelevant and that \( [\varphi] = [\psi_1 \wedge \psi_2] \) is the infimum of \( [\psi_1] \) and \( [\psi_2] \).

  \SubProofOf{thm:intuitionistic_lindenbaum_tarski_algebra/bottom} That \( [\bot] \) is the bottom is a restatement of \eqref{eq:thm:minimal_propositional_negation_laws/efq}.

  \SubProofOf{thm:intuitionistic_lindenbaum_tarski_algebra/conditional} Restating \eqref{eq:def:heyting_algebra/conditional}, we must prove that \( [\psi_1 \rightarrow \psi_2] \) equals
  \begin{equation*}
    ([\psi_1] \rightarrow [\psi_2]) = \underbrace{\sup\set[\Big]{ [\varphi] \given* \varphi \in \Gamma \T{and} (\varphi \wedge \psi_1) \vdash \psi_2 }}_{\Phi}.
  \end{equation*}

  An equivalent condition for \( \varphi \) to be in \( \Phi \) is, due to \fullref{thm:formulas_are_derivable_iff_conjunction_is_derivable},
  \begin{equation*}
    \varphi, \psi_1 \vdash \psi_2.
  \end{equation*}

  Then by the deduction theorem,
  \begin{equation*}
    \varphi \vdash (\psi_1 \rightarrow \psi_2).
  \end{equation*}

  Thus \( [\psi_1 \rightarrow \psi_2] \) is an upper bound of \( \Psi \).

  It remains to show that \( (\psi_1 \rightarrow \psi_2) \in \Psi \). Both \( \psi_1 \) and \( (\psi_1 \rightarrow \psi_2) \) follow from the formula \( \parens[\Big]{ (\psi_1 \rightarrow \psi_2) \wedge \psi_1 } \) and by \eqref{eq:def:positive_implicational_propositional_derivation_system/rules/modus_ponens},
  \begin{equation*}
    \psi_1, (\psi_1 \rightarrow \psi_2) \vdash \psi_2.
  \end{equation*}

  By \fullref{thm:formulas_are_derivable_iff_conjunction_is_derivable},
  \begin{equation*}
    \parens[\Big]{ (\psi_1 \rightarrow \psi_2) \wedge \psi_1 } \vdash \psi_2.
  \end{equation*}

  Therefore \( [\psi_1 \rightarrow \psi_2] \in \Phi \) and it is indeed the supremum of \( \Psi \).

  \SubProofOf{thm:intuitionistic_lindenbaum_tarski_algebra/complement} The pseudocomplement \( \widetilde{[\psi]} \) is, by definition,
  \begin{equation*}
    \widetilde{[\psi]}
    =
    [\psi] \rightarrow [\bot].
  \end{equation*}

  From what we have already proved, we can conclude that \( \widetilde{[\psi]} = [\psi \rightarrow \bot] \). From \fullref{def:minimal_propositional_derivation_system/negation} it follows that the formulas \( \psi \rightarrow \bot \) and \( \neg \psi \) are interderivable, thus \( \widetilde{[\psi]} = [\neg \psi] \).

  If we are working in classical logic where \eqref{eq:thm:minimal_propositional_negation_laws/lem} holds, then
  \begin{equation*}
    \sup\set{ [\psi], \widetilde{[\psi]} }
    \reloset {\ref{thm:intuitionistic_lindenbaum_tarski_algebra/join}} =
    [\psi \vee \neg \psi]
    \reloset {\eqref{eq:thm:minimal_propositional_negation_laws/lem}} =
    [\top],
  \end{equation*}
  which proves \eqref{eq:def:boolean_algebra/complement/join}.

  The dual law \eqref{eq:def:boolean_algebra/complement/meet}
  \begin{equation*}
    \inf\set{ [\psi], \widetilde{[\psi]} }
    \reloset {\ref{thm:intuitionistic_lindenbaum_tarski_algebra/meet}} =
    [\psi \wedge \neg \psi]
    \reloset {\eqref{eq:thm:minimal_propositional_negation_laws/lnc}} =
    [\neg \top]
    =
    \widetilde{[\top]}
    \reloset {\eqref{eq:def:heyting_algebra/pseudocomplement}} =
    [\bot].
  \end{equation*}
\end{proof}

\begin{remark}\label{rem:thm:intuitionistic_lindenbaum_tarski_algebra/syntactic_proof}
  Notice that the proof of \fullref{thm:intuitionistic_lindenbaum_tarski_algebra} relies entirely on the derivation system. On the other hand, we define \hyperref[def:propositional_heyting_algebra_semantics]{Heyting semantics} for intuitionistic formulas.

  Thus we have shown that Heyting algebras arise naturally from the intuitionistic derivation system and that their role as a semantic framework is completely justified.
\end{remark}
