\subsection{Topological continuity}\label{subsec:topological_continuity}

\begin{definition}\label{def:local_continuity}
  We say that the function \( f: X \to Y \) between topological spaces is \term{continuous} at the point \( x_0 \in X \) if \( f(x_0) \) is a limit \hyperref[def:local_convergence]{point} of \( f \) at \( x_0 \).

  If limit point is unique (e.g. in \hyperref[def:separation_axioms/T2]{Hausdorff spaces}), this condition can be formulated by \enquote{interchanging} \( \lim \) and \( f \) as follows:
  \begin{equation*}
    f(x_0) = f\left( \lim_{x \to x_0} x \right) = \lim_{x \to x_0} f(x).
  \end{equation*}
\end{definition}

\begin{definition}\label{def:global_continuity}
  We say that the function \( f: X \to Y \) between topological spaces is \term{everywhere continuous} or simply \term{continuous} if and of the following conditions hold:
  \begin{thmenum}
    \thmitem{def:global_continuity/limits} \( f \) is continuous at every point of \( X \) in the sense of \fullref{def:local_continuity}.
    \thmitem{def:global_continuity/open} For every open set \( V \in T \), the \hyperref[thm:def:function/properties/preimage]{preimage} \( f^{-1}(V) \) is open.
    \thmitem{def:global_continuity/closed} For every closed set \( F \in F_{\mscrT_Y} \), the preimage \( f^{-1}(F) \) is closed.
    \thmitem{def:global_continuity/base} There exists a \hyperref[def:topological_base]{base} \( \mscrB_{\mscrT_Y} \subseteq T_Y \), such that for every \( V \in B_{\mscrT_Y} \), the preimage \( f^{-1}(V) \) is open.
    \thmitem{def:global_continuity/subbase} There exists a \hyperref[def:topological_subbase]{subbase} \( P_{\mscrT_Y} \subseteq T_Y \), such that for every \( V \in P_{\mscrT_Y} \), the preimage \( f^{-1}(V) \) is open.
    \thmitem{def:global_continuity/closure} For every set \( A \subseteq X \), \( f(\cl(A)) \subseteq \cl(f(A)) \).
  \end{thmenum}

  We denote the set of all continuous functions from \( X \) to \( Y \) by \( C(X, Y) \).
\end{definition}
\begin{proof}
  \ImplicationSubProof{def:global_continuity/limits}{def:global_continuity/open} Follows from \fullref{def:local_convergence/neighborhoods}.
  \ImplicationSubProof{def:global_continuity/open}{def:global_continuity/closed} If \( F \in F_{\mscrT_Y} \) is a closed set, \( Y \setminus F \) is open, therefore \( f^{-1}(Y \setminus F) = X \setminus f^{-1}(F) \) is also open. Hence, \( f^{-1}(F) \) is closed.
  \ImplicationSubProof{def:global_continuity/open}{def:global_continuity/base} \( \mscrT \) is a base of itself.
  \ImplicationSubProof{def:global_continuity/base}{def:global_continuity/subbase} Every base is also a subbase.
  \ImplicationSubProof{def:global_continuity/subbase}{def:global_continuity/limits} Follows from the equivalences in \fullref{def:local_convergence}.
  \ImplicationSubProof{def:global_continuity/closed}{def:global_continuity/closure} Note that
  \begin{equation*}
    A
    \reloset {\ref{thm:function_image_preimage_composition/preimage_of_image}} \subseteq
    f^{-1}(f(A))
    \reloset {\ref{thm:def:function_preimage/properties/monotonicity}} \subseteq
    f^{-1}(\cl(f(A))).
  \end{equation*}

  Apply \( f \circ \cl \) to the above chain of inclusions to obtain
  \begin{equation*}
    f(\cl(A))
    \subseteq
    f(\underbrace{\cl}_{\ref{def:global_continuity/closed}}(f^{-1}(\cl(f(A)))))
    \reloset {\ref{thm:function_image_preimage_composition/image_of_preimage}} \subseteq
    \cl(f(A)),
  \end{equation*}
  which proves the implication.

  \ImplicationSubProof{def:global_continuity/closure}{def:global_continuity/closed} Fix a closed set \( F \subseteq Y \). Then
  \begin{equation}\label{def:global_continuity/closure_implies_closed_right}
    f(\cl(f^{-1}(F)))
    \reloset {\ref{def:global_continuity/closure}} \subseteq
    \cl(f(f^{-1}(F)))
    \reloset {\ref{thm:function_image_preimage_composition/image_of_preimage}} \subseteq
    \cl(F)
    =
    F.
  \end{equation}

  Since \( \cl \) is monotone, we have
  \begin{equation}\label{def:global_continuity/closure_implies_closed_left}
    f(\cl(f^{-1}(F)))
    \supseteq
    f(f^{-1}(F))
    \reloset {\ref{thm:function_image_preimage_composition/preimage_of_image}} \supseteq
    F.
  \end{equation}

  From \eqref{def:global_continuity/closure_implies_closed_right} and \eqref{def:global_continuity/closure_implies_closed_left} it follows that
  \begin{equation*}
    F = f(\cl(f^{-1}(F))).
  \end{equation*}

  By taking the preimage, we obtain
  \begin{equation*}
    f^{-1}(F)
    =
    f^{-1}(f(\cl(f^{-1}(F))))
    \reloset {\ref{thm:function_image_preimage_composition/image_of_preimage}} \supseteq
    \cl(f^{-1}(F)).
  \end{equation*}

  Therefore, \( f^{-1}(F) \) is closed.
\end{proof}

\begin{definition}\label{def:homeomorphism}
  We say that the continuous function \( f: X \to Y \) is \term{open} (resp. \term{closed}), if the image \( f(U) \) of an open (resp. closed) in \( \mscrT_X \) set is open (resp. closed) in \( \mscrT_Y \).

  If \( f \) is an open bijection, we say that \( f \) is a \term{homeomorphism}. If \( f \) is only an open injection, we say that \( f \) is a \term{homeomorphic embedding}.
\end{definition}

\begin{definition}\label{def:parametric_curve}
  Let \( I \) be an interval (of any type) in \( \BbbR \) with endpoints \( a < b \), not necessarily finite. Depending on the use case, we define a \term{parametric curve} on \( I \) by any of the non-equivalent definitions

  \begin{thmenum}
    \thmitem{def:parametric_curve/function} A continuous function \( \gamma: I \to X \) is called a parametric curve.

    \thmitem{def:parametric_curve/image} The image \( \img(\gamma) \) of a parametric curve \( \gamma \) is also called a parametric curve.

    \thmitem{def:parametric_curve/equivalence_class} The equivalence class of all continuous functions from \( I \) to \( X \) with
    \begin{equation*}
      \gamma \cong \beta \iff \img(\gamma) = \img(\beta) \text{ and the endpoints of } \gamma \text{ and } \beta \text{ coincide}
    \end{equation*}
    is also called a parametric curve.
  \end{thmenum}

  The points \( \gamma(a) \) and \( \gamma(b) \) are called the \term{endpoints} of the curve, \( \gamma(a) \) is the \term{start} and \( \gamma(b) \) is the \term{end}. We say that \( \gamma \) \term{connects} \( a \) and \( b \).

  Parametric curves on \( I = [0, 1] \) are also called \term{paths}.

  We define some fundamental types of curves:
  \begin{thmenum}
    \thmitem{def:parametric_curve/closed} The curve \( \gamma \) is called \term{closed} if its endpoints coincide, i.e. \( \gamma(a) = \gamma(b) \).

    \thmitem{def:parametric_curve/simple} The curve \( \gamma \) is called \term{simple} if the function \( \gamma: I \to Y \) is injective with the possible exception of the endpoints (in which case we speak of \term{simple closed curves}.
  \end{thmenum}

  If \( \gamma: I \to X \) is a parametric curve, related curves are:
  \begin{thmenum}
    \thmitem{def:parametric_curve/function_graph}\mcite[def. 1.20]{ИвановТужилин2017}The \hyperref[def:multi_valued_function/graph]{graph} \( \gph(\gamma) \) of \( \gamma \) is a the image of the curve \( \overline{\gamma}(t, x) \coloneqq (t, \gamma(x)) \) in the topological space \( I \times X \).

    \thmitem{def:parametric_curve/implicit}\mcite[def. 1.24]{ИвановТужилин2017}If \( M \) is a subset of \( X \) and if there exists a curve \( \gamma: I \to X \) such that \( \imag(\gamma) = M \), we call \( M \) an \term{implicit parametric curve}.
  \end{thmenum}
\end{definition}

\begin{definition}\label{def:parametric_hypersurface}
  In analogy to \fullref{def:parametric_curve} (and with the caveats of \fullref{def:parametric_curve}), we define \term{parametric hypersurfaces} as follows:

  Let \( \xi \) is a potentially infinite cardinal number, let \( \card \mscrK = \xi \) and let \( \{ I_\alpha \}_{\alpha \in \mscrK} \) be a family of intervals in \( \BbbR \). We define a parametric hypersurface to be a continuous image from the \hyperref[def:topological_product]{product space} \( \prod_{\alpha \in \mscrK} I_\alpha \) to \( Y \).

  We call \( \xi \) the \term{dimension} of the hypersurface.
\end{definition}

\begin{definition}\label{def:fundamental_groupoid}
  \todo{Define fundamental groupoids}.
\end{definition}
