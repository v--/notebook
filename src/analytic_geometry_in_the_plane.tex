\subsection{Analytic geometry in the plane}\label{subsec:analytic_geometry_in_the_plane}

\begin{remark}\label{rem:analytic_geometry}
  Analytic geometry is a XVII-century branch of mathematics that studies geometric figures using coordinate \hyperref[rem:coordinate_systems]{systems}. The term \enquote{analytic geometry} may refer to a modern subbranch of algebraic geometry, however we refrain from using \enquote{analytic geometry} in that sense. Historically, most of these definitions were given either for the Euclidean \hyperref[def:euclidean_plane]{plane} or for the three-dimensional Euclidean space.

  Most of the definitions from \fullref{subsec:vector_space_geometry} are generalizations of concepts from analytic geometry. We will state definitions in the language of linear algebra and refrain from using synthetic (axiomatic) geometry. When working in the plane (resp. three-dimensional space), we will assume that we have fixed an \hyperref[def:orthonormal_system]{orthonormal} coordinate \hyperref[def:euclidean_plane_coordinate_system]{system} \( Oxy \) (resp. \( Oxyz \)), which allows us to visualize geometric figures.
\end{remark}

\begin{definition}\label{def:plane_line_equations}
  \hyperref[def:geometric_line]{Lines} in \( \BbbR^2 \) are so ubiquitous that they can be represented by a lot of standard \hyperref[ex:equations]{equations}.

  \begin{thmenum}
    \labitem{def:plane_line_equations/vector_parametric} When regarding a line as a parametric curve as in \fullref{def:geometric_line/parametric}, the \hyperref[def:first_order_syntax/formula]{formula}
    \begin{equation}\label{def:plane_line_equations/parametric_equation}
      l(t) = tx + a
    \end{equation}
    is called a \term{vector parametric equation}.

    \labitem{def:plane_line_equations/scalar_parametric} Given \fullref{def:plane_line_equations/parametric_equation}, the \term{scalar parametric equations} of the line are
    \begin{equation}\label{def:plane_line_equations/scalar_parametric_equations}
      \begin{cases}
         & l_1(t) = t x_1 + a_1  \\
         & l_2(t) = t x_2 + a_2.
      \end{cases}
    \end{equation}

    \labitem{def:plane_line_equations/general} When regarding a line as an algebraic curve as in \fullref{def:geometric_line/algebraic}, the equation
    \begin{equation}\label{def:plane_line_equations/general_equation}
      p(x, y) \coloneqq Ax + By + C = 0
    \end{equation}
    is called the \term{general equation} or simply \term{equation} of a line in a plane. Either \( A \) or \( B \) must be nonzero so that \( \deg(p) = 1 \).

    Note that multiple general equations can have the same locus (e.g. the entire polynomial ideal \( \braket{p} \)).

    \labitem{def:plane_line_equations/normal} If \( A^2 + B^2 = 1 \), we call \fullref{def:plane_line_equations/general_equation} a \term{normal equation}. This leaves us with only two representatives of \( \braket{p} \).

    \labitem{def:plane_line_equations/cartesian} Given \( k, m \in \BbbR \) and  \( k \neq 0 \), we define the \term{Cartesian equation} of a line:
    \begin{equation}\label{def:plane_line_equations/cartesian_equation}
      y = kx + m.
    \end{equation}

    We call \( k \) the \term{slope} of the line.

    This is a special case of \fullref{def:plane_line_equations/general} with \( A = -k \), \( B = -1 \) and \( C = m \). Unlike the general equation, the Cartesian equation of a line is unique.

    Conversely, if \( B \neq 0 \) in \fullref{def:plane_line_equations/general_equation}, we can define \( k = -\tfrac A B \) and \( m = -\tfrac C B \) to form a Cartesian equation.

    \labitem{def:plane_line_equations/intercept} Given nonzero \( a, b \in \BbbR \), we define the \term{intercept equation} of a line:
    \begin{equation}\label{def:plane_line_equations/intercept_equation}
      \frac x a + \frac y b = 1,
    \end{equation}

    This is a special case of \fullref{def:plane_line_equations/general} with \( A = \frac 1 a \), \( B = \frac 1 b \) and \( C = -1 \). The intercept equation of a line is also unique.

    If \( A, B, C \neq 0 \) in \fullref{def:plane_line_equations/general}, we can define an \term{intercept equation} as \( a = -\tfrac C A \) and \( b = -\tfrac C B \)).
  \end{thmenum}
\end{definition}

\begin{figure}
  \begin{minipage}[b]{0.40\textwidth}
    \centering
    \todo{Add diagram}\iffalse\begin{mplibcode}
      input metapost/plotting;

      u := 1.5cm;

      beginfig(1);
      path l, x_axis, y_axis;

      x_axis = (-1, 0) scaled u -- (1, 0) scaled u;
      y_axis = (0, -1) scaled u -- (0, 1) scaled u;
      l = (-1 / 2, -1) * u -- (1, 3 / 4) * u;

      drawarrow x_axis;
      label.bot("$x$", point 0.9 of x_axis);
      drawarrow y_axis;
      label.lft("$y$", point 0.9 of y_axis);
      draw l;
      label.bot("$y = kx + m$", startpoint of l);
      endfig;
    \end{mplibcode}\fi
    \caption{A \hyperref[def:geometric_line]{line} in \( \BbbR^2 \) defined using its \hyperref[def:plane_line_equations/cartesian]{Cartesian equation}.}\label{def:plane_line_equations/cartesian_equation_drawing}
  \end{minipage}
  \hspace{0.05\textwidth}
  \begin{minipage}[b]{0.40\textwidth}
    \centering
    \todo{Add diagram}\iffalse\begin{mplibcode}
      input metapost/plotting;
      u := 1.5cm;

      beginfig(1)
      drawarrow (-1 / 2, 0) scaled u -- (2, 0) scaled u;
      drawarrow (0, -1 / 2) scaled u -- (0, 2) scaled u;

      z0 = (1 / 2, 1 / 6) scaled u;
      z1 = (2, 11 / 12) scaled u;
      z2 = (1, 13 / 6) scaled u;

      draw z0 -- (x0, max(y1, y2)) dashed withdots;

      drawarrow z0 -- z1;
      draw (x0, y1) -- z1 dashed evenly;
      label.top("$x_1$", midpoint of ((x0, y1) -- z1));

      drawarrow z0 -- z2;
      draw (x0, y2) -- z2 dashed evenly;
      label.bot("$x_2$", midpoint of ((x0, y2) -- z2));
      endfig;
    \end{mplibcode}\fi
    \caption{An \hyperref[def:angle/acute]{acute angle} with its measurement segments dashed.}\label{def:angle/figure}
  \end{minipage}
\end{figure}

\begin{definition}\label{def:angle}
  A \term{directed angle} is a tuple of two closed \hyperref[def:geometric_ray]{rays} with a common vertex. It is a closed cone. Given two rays \( r_1, r_2 \) with a common vertex, we denote their corresponding directed angle by \( \sphericalangle(r_1, r_2) \).

  Suppose that \( r_1 \) and \( r_2 \) have scalar parametric equations
  \begin{equation*}
    r_i: t \mapsto
    \begin{cases}
      tx_i + a_i \\
      ty_i + b_i,
    \end{cases}
    i = 1, 2.
  \end{equation*}

  We write

  The condition of the rays having a common vertex is equivalent to \( a_1 = a_2 \) and \( b_1 = b_2 \). If not specified otherwise, we assume that \( a_1 = a_2 = b_1 = b_2 = 0 \).

  The \term{measure in radians} of a directed angle, often called the angle itself, is defined as the number (see \fullref{def:geometric_trigonometric_functions})
  \begin{equation*}
    \alpha \coloneqq \rem(\arctantwo(y_2, x_2) - \arctantwo(y_1, x_1), 2\pi).
  \end{equation*}

  We can classify angles based on their measure as
  \begin{thmenum}
    \labitem{def:angle/zero} \term{zero} if \( \alpha = 0 \),
    \labitem{def:angle/acute} \term{acute} if \( \alpha \in (0, \tfrac \pi 2) \),
    \labitem{def:angle/right} \term{right} if \( \alpha = \tfrac \pi 2 \),
    \labitem{def:angle/obtuse} \term{obtuse} if \( \alpha \in (\tfrac \pi 2, \pi) \),
    \labitem{def:angle/straight} \term{straight} if \( \alpha = \pi \), in which case the angle is actually a line,
    \labitem{def:angle/reflex} \term{reflex} if \( \alpha > \pi \).
  \end{thmenum}

  We often do not care about the order of the two rays and speak of an \term{undirected angle}. In this case, the measure of the undirected angle is the smaller of the measures of the two oriented angles. Thus we cannot speak of straight and reflex undirected angles.
\end{definition}

\begin{definition}\label{def:triangle}
  \begin{figure}
    \centering
    \todo{Add diagram}\iffalse\begin{mplibcode}
      input metapost/plotting;

      beginfig(1)
      pair A, B, C;
      path alpha, beta, gamma;

      A := origin;
      B := (3, 0) scaled u;
      C := (2, 2) scaled u;

      draw A -- B -- C -- cycle;

      alpha = fullcircle scaled (u / 2) shifted A cutbefore (A -- B) cutafter (A -- C);
      draw alpha;
      label.urt("$\alpha$", point 0.4 of alpha);

      beta = fullcircle scaled (u / 2) shifted B cutbefore (B -- C) cutafter (B -- A);
      draw beta;
      label.ulft("$\beta$", point 1.4 of beta);

      gamma = fullcircle scaled (u / 4) shifted C cutbefore (A -- C) cutafter (B -- C);
      draw gamma;
      label.bot("$\gamma$", point 0.6 of gamma);

      fill dot shifted A;
      fill dot shifted B;
      fill dot shifted C;

      label.llft("$A$", A);
      label.lrt("$B$", B);
      label.top("$C$", C);

      label.rt("$a$", midpoint of (B -- C));
      label.ulft("$b$", midpoint of (A -- C));
      label.bot("$c$", midpoint of (A -- B));
      endfig;
    \end{mplibcode}\fi
    \caption{An \hyperref[def:triangle/acute]{acute triangle}.}\label{def:triangle/figure}
  \end{figure}

  A \term{triangle} is a triple \( (A, B, C) \) of \hyperref[def:point]{points}, no two of which are \hyperref[def:collinear_complanar]{collinear} (see \fullref{def:simplex/triangle} for a more general definition). The three points are called the \term{vertices} of the triangle.

  Define the associated \hyperref[def:convex_set/line_segment]{line segments}, called the \term{sides} of the triangle, and its (undirected) \hyperref[def:angle]{angles} as
  \begin{balign*}
    a \coloneqq [B, C], &  & \alpha \coloneqq \sphericalangle(b, c), \\
    b \coloneqq [A, C], &  & \beta \coloneqq \sphericalangle(a, c),  \\
    c \coloneqq [A, B], &  & \gamma \coloneqq \sphericalangle(a, b).
  \end{balign*}

  Note that we defined the angles using segments rather than rays but this is immaterial because each to each segment \( [p, q] \) there corresponds exactly one closed ray \( t \mapsto p + t q \).

  We can classify triangles based on their sides as
  \begin{thmenum}
    \labitem{def:triangle/isosceles} \term{isosceles} if at least two of its sides have equal length
    \labitem{def:triangle/equilateral} \term{equilateral} if all of its sides have equal length
  \end{thmenum}
  or based on their angles as
  \begin{thmenum}
    \labitem{def:triangle/acute} \term{acute} if all of its angles are \hyperref[def:angle/acute]{acute}.
    \labitem{def:triangle/right} \term{right} if at least one of the angles is \hyperref[def:angle/straight]{straight}.
    \labitem{def:triangle/obtuse} \term{obtuse} if at least one of its angles is \hyperref[def:angle/obtuse]{obtuse}.
  \end{thmenum}
\end{definition}

\begin{definition}\label{def:quadratic_plane_curve}
  The \term{quadratic plane curves} are algebraic \hyperref[def:hypersurface/algebraic]{curves} given by a bivariate polynomial of degree \( 2 \). The \term{general equation} of a quadratic plane curve is
  \begin{equation}\label{def:quadratic_plane_curve/general_equation}
    c(x, y) \coloneqq A x^2 + B xy + C y^2 + Dx + Ey + F = 0.
  \end{equation}

  Multiple equation can correspond to the same curve. Not all general equations, however, define algebraic curves. We will not concern ourselves with the details. See \fullref{ex:affine_varieties} for a proof that the unit circle is an algebraic curve. It turns out that the algebraic curves given \fullref{def:quadratic_plane_curve/general_equation} are precisely the ones listed here, collectively known as \term{conic sections}. We give only canonical forms of the equations; any linear transformation of the corresponding loci is described by another general equation.

  \begin{figure}
    \begin{minipage}{0.3\textwidth}
      \centering
      \todo{Add diagram}\iffalse\begin{mplibcode}
        input metapost/plotting;

        u := 1.25cm;

        vardef scaled_sin(expr x) =
        5 / 6 * sin(x)
        enddef;

        beginfig(1)
        fill dot shifted (u, 0);

        drawarrow (-pi / 2, 0) scaled u -- (pi / 2, 0) scaled u;
        drawarrow (0, -pi / 2) scaled u -- (0, pi / 2) scaled u;

        drawarrow path_of_curve(cos, scaled_sin, -1 / 4 * pi, 3 / 4 * pi, 0.01, u);
        drawarrow path_of_curve(cos, scaled_sin, 3 / 4 * pi, 7 / 4 * pi, 0.01, u);
        endfig;
      \end{mplibcode}\fi
    \end{minipage}
    \hspace{0.02\textwidth}
    \begin{minipage}{0.3\textwidth}
      \centering
      \todo{Add diagram}\iffalse\begin{mplibcode}
        input metapost/plotting;

        u := 1.25cm;

        vardef minus_cosh(expr x) =
        -cosh(x)
        enddef;

        beginfig(1)
        drawarrow (-pi / 2, 0) scaled u -- (pi / 2, 0) scaled u;
        drawarrow (0, -pi / 2) scaled u -- (0, pi / 2) scaled u;

        drawarrow path_of_curve(cosh, sinh, -pi / 3, 0, 0.01, u);
        drawarrow path_of_curve(cosh, sinh, 0, pi / 3, 0.01, u);

        drawarrow path_of_curve(minus_cosh, sinh, -pi / 3, 0, 0.01, u);
        drawarrow path_of_curve(minus_cosh, sinh, 0, pi / 3, 0.01, u);
        endfig;
      \end{mplibcode}\fi
    \end{minipage}
    \hspace{0.02\textwidth}
    \begin{minipage}{0.3\textwidth}
      \centering
      \todo{Add diagram}\iffalse\begin{mplibcode}
        input metapost/plotting;

        u := 1.25cm;

        beginfig(1)
        fill dot;

        drawarrow (-pi / 2, 0) scaled u -- (pi / 2, 0) scaled u;
        drawarrow (0, -pi / 2) scaled u -- (0, pi / 2) scaled u;

        vardef y_upper(expr x) =
        sqrt(x)
        enddef;

        vardef y_lower(expr x) =
        -sqrt(x)
        enddef;

        drawarrow path_of_plot(y_upper, 0, pi / 3, 0.01, u);
        drawarrow path_of_plot(y_lower, 0, pi / 3, 0.01, u);
        endfig;
      \end{mplibcode}\fi
    \end{minipage}
    \caption{An \hyperref[def:quadratic_plane_curve/ellipse]{ellipse}, \hyperref[def:quadratic_plane_curve/hyperbola]{hyperbola} and \hyperref[def:quadratic_plane_curve/parabola]{parabola} defined via their parametric equations. The starting point is highlighted and the direction of the parametric curves is shown.}\label{def:quadratic_plane_curve/figure}
  \end{figure}

  \begin{thmenum}
    \labitem{def:quadratic_plane_curve/ellipse} An \term{ellipse} is a quadratic curve whose canonical equation has the form
    \begin{equation}\label{def:quadratic_plane_curve/ellipse/canonical_equation}
      c(x, y) \coloneqq \frac {x^2} {a^2} + \frac {y^2} {b^2} - 1 = 0,
    \end{equation}
    where \( a, b > 0 \).

    If \( a = b \), we say that the ellipse is a \term{circle} and we call \( a \) the circle's \term{radius}. The \term{unit circle} is defined by \( a = b = 1 \). Circles generalize to \hyperref[def:metric_space/sphere]{spheres} in metric spaces.

    \Fullref{def:pi} and \fullref{def:geometric_trigonometric_functions} logically belong here but are extracted separately for brevity.

    We are often interested in defining ellipses via \term{scalar parametric equations} using \hyperref[def:trigonometric_functions]{trigonometric functions} as follows:
    \begin{equation}\label{def:quadratic_plane_curve/ellipse/parametric_equations}
      \begin{cases}
        x = a \cos(t) \\
        y = b \sin(t),
      \end{cases}
    \end{equation}
    where \( t \in [0, 2\pi) \).

    We will now demonstrate that \fullref{def:quadratic_plane_curve/ellipse/canonical_equation} and \fullref{def:quadratic_plane_curve/ellipse/parametric_equations} describe the same curve. First, suppose that the pair \( (x_0, y_0) \) satisfies \fullref{def:quadratic_plane_curve/ellipse/canonical_equation}. It follows from \fullref{thm:arctantwo} that \( t_0 \coloneqq \arctantwo\left(\tfrac {y_0} b, \tfrac {x_0} a \right) \) is a solution to the \hyperref[def:quadratic_plane_curve/ellipse/parametric_equations]{parametric equations}. Conversely, if \( x_0 = a \cos(t_0) \) and \( y_0 = b \sin(t_0) \) for some \( t_0 \in [0, 2\pi) \), by \fullref{thm:trigonometric_identities/pythagorean_identity} it follows that the pair \( (x_0, y_0) \) is a root of \fullref{def:quadratic_plane_curve/ellipse/canonical_equation} and, by \fullref{thm:arctantwo}, \( t_0 \) can be restored given \( \cos(t_0) \) and \( \sin(t_0) \).

    Therefore every point of the parametric equation \fullref{def:quadratic_plane_curve/ellipse/parametric_equations} corresponds uniquely to a solution of the canonical equation \fullref{def:quadratic_plane_curve/ellipse/canonical_equation} and vice versa, which makes the two approaches to defining ellipses equivalent.

    \labitem{def:quadratic_plane_curve/hyperbola} A \term{hyperbola} is a quadratic curve whose canonical equation has the form
    \begin{equation}\label{def:quadratic_plane_curve/hyperbola/canonical_equation}
      c(x, y) \coloneqq \frac {x^2} {a^2} - \frac {y^2} {b^2} - 1 = 0,
    \end{equation}
    where \( a, b > 0 \).

    Similarly to ellipses, we are can define hyperbolas via \term{scalar parametric equations} using \hyperref[def:hyperbolic_trigonometric_functions]{hyperbolic trigonometric functions} as follows:
    \begin{equation}\label{def:quadratic_plane_curve/hyperbola/parametric_equations}
      \begin{cases}
        x = a \cosh(t) \\
        y = b \sinh(t),
      \end{cases}
    \end{equation}
    where \( t \in \BbbR \). This only defines the \term{right part} of the hyperbola. The left part is defined by replacing \( a \) with \( -a \).

    \labitem{def:quadratic_plane_curve/parabola} A \term{parabola} is a quadratic curve whose canonical equation has the form
    \begin{equation}\label{def:quadratic_plane_curve/parabola/canonical_equation}
      c(x, y) \coloneqq y^2 - 2px = 0,
    \end{equation}
    where \( p \neq 0 \).

    Unlike ellipses and hyperbolas, we do not define parametric equations. Instead, we define \( y \) as a function of \( x \) separately for the lower half-plane and upper half-plane:
    \begin{equation}\label{def:quadratic_plane_curve/parabola/cartesian_equation}
      y(x) = \pm \sqrt{2px}.
    \end{equation}
  \end{thmenum}

  Ellipses, hyperbolas and parabolas are collectively called \term{conic sections}.
\end{definition}

\begin{definition}\label{def:pi}
  \begin{figure}
    \centering
    \todo{Add diagram}\iffalse\begin{mplibcode}
      input metapost/plotting;

      beginfig(1)
      drawarrow (-pi / 2, 0) scaled u -- (pi / 2, 0) scaled u;
      drawarrow (0, -1 / 2) scaled u -- (0, pi / 2) scaled u;

      vardef y(expr x) =
      sqrt(1 - x ** 2)
      enddef;

      drawarrow path_of_plot(y, -1, 1, 0.01, u);
      endfig;
    \end{mplibcode}\fi
    \caption{\( \gph(y^+) \) as a parametric curve in \fullref{def:pi}.}\label{fig:def:pi/upper_half_circle}
  \end{figure}

  The definition of a circle of unit radius as the zero-locus of the polynomial \( x^2 + y^2 - 1 \) allows us to solve a chicken-and-egg problem regarding the definitions of the number \( \pi \). It is conventional to define it as the ratio of a circle's circumference to its diameter. For a unit circle, this diameter is \( 2 \). It will be simpler for us, however, to define \( \pi \) as the radius of a half-circle's circumference since we can represent \( y \) as a function of \( x \) in the upper \hyperref[def:half_space]{half-plane} (see \ref{fig:def:pi/upper_half_circle}). Define the parametric curve
  \begin{balign*}
     & y^+: [-1, 1] \to [0, 1]          \\
     & y^+(x) \coloneqq \sqrt{1 - x^2}.
  \end{balign*}

  We use \fullref{thm:length_of_function_graph} to find the length of the graph \( \gph(y^+(x)) \). The derivative of \( y^+(x) \) is
  \begin{equation*}
    D_x[y^+(x)] = \frac{-2x}{2 \sqrt{1 - x^2}} = - \frac x {\sqrt{1 - x^2}} dx.
  \end{equation*}

  The length of the curve \( \gph(y^+) \) is thus
  \begin{equation*}
    \len(\gph(y^+)) = \int_{-1}^1 \sqrt{1 + \frac{x^2}{1 - x^2}} dx = \int_{-1}^1 \frac 1 {\sqrt{1 - x^2}} dx.
  \end{equation*}

  This justifies the definition
  \begin{equation}\label{def:pi/weierstrass_integral}
    \pi \coloneqq \int_{-1}^1 \frac 1 {\sqrt{1 - x^2}} dx.
  \end{equation}

  See \fullref{thm:trigonometric_function_basic_roots} for a proof of how this relates to the trigonometric functions and \fullref{thm:exponential_function_properties/eulers_identity} as a consequence.
\end{definition}

\begin{definition}\label{def:geometric_trigonometric_functions}
  After defining the \hyperref[def:trigonometric_functions]{trigonometric functions} \( \cos(z) \) and \( \sin(z) \) analytically via power series, we will define their geometric counterparts \( \cos_G(z) \) and \( \sin_G(z) \) and show the connection between them. The actual geometric definition relies on formalisms that are far beyond our interest (see the notes in \fullref{def:euclidean_plane}).

  Fix a point \( (x_0, y_0) \) on the unit circle (that is, \( x_0^2 + y_0^2 = 1 \)) and define the points
  \begin{equation}\label{def:geometric_trigonometric_functions/vertices}
    \begin{array}{l}
      A \coloneqq (x_0, y_0), \\
      B \coloneqq (0, 0),     \\
      C \coloneqq (x_0, 0).
    \end{array}
  \end{equation}

  Consider the \hyperref[def:triangle]{triangle} formed by these vertices. \Cref{fig:def:geometric_trigonometric_functions/triangle} illustrates the situation.
  \begin{figure}
    \begin{minipage}[b]{0.4\textwidth}
      \centering
      \todo{Add diagram}\iffalse\begin{mplibcode}
        input metapost/plotting;

        u := 3.5cm;

        beginfig(1)
        pair A, B, C;
        path alpha, beta;

        t := 1;
        A := origin;
        B := (cos(t), sin(t)) scaled u;
        C := (cos(t), 0) scaled u;

        draw A -- B -- C -- cycle;

        alpha = fullcircle scaled (u / 4) shifted A cutbefore (A -- C) cutafter (A -- B);
        draw alpha;
        label.urt("$\alpha$", midpoint of alpha);
        label.lft("$\begin{rcases} \sin_G(\alpha) = \tfrac {\len b} {\len c} \\ \cos_G(\alpha) = \tfrac {\len a} {\len c} \end{rcases}$", A);

        beta = fullcircle scaled (u / 4) shifted B cutbefore (B -- A) cutafter (B -- C);
        draw beta;
        label.llft("$\beta$", point 0.7 of beta);
        label.lft("$\begin{rcases} \sin_G(\beta) = \tfrac {\len a} {\len c} \\ \cos_G(\beta) = \tfrac {\len b} {\len c} \end{rcases}$", B - (0.05, 0) * u);

        fill dot shifted A;
        fill dot shifted B;
        fill dot shifted C;

        label.bot("$A$", A);
        label.top("$B$", B);
        label.lrt("$C$", C);

        label.rt("$a$", midpoint of (B -- C));
        label.bot("$b$", midpoint of (A -- C));
        label.ulft("$c$", midpoint of (A -- B));
        endfig;
      \end{mplibcode}\fi
    \end{minipage}
    \hspace{0.05\textwidth}
    \begin{minipage}[b]{0.4\textwidth}
      \centering
      \todo{Add diagram}\iffalse\begin{mplibcode}
        input metapost/plotting;

        u := 3.5cm;

        beginfig(1)
        pair A, B, C;
        path alpha, beta;

        t := 1;
        A := origin;
        B := (cos(t), sin(t)) scaled u;
        C := (cos(t), 0) scaled u;

        drawarrow (-sin(pi/16), 0) scaled u -- (5/4, 0) scaled u;
        drawarrow (0, -sin(pi/16)) scaled u -- (0, 5/4) scaled u;

        draw path_of_curve(cos, sin, -pi/16, 9/16 * pi, 0.01, u);
        draw A -- B -- C -- cycle;

        alpha = fullcircle scaled (u / 4) shifted A cutbefore (A -- C) cutafter (A -- B);
        draw alpha;
        label.urt("$\alpha$", midpoint of alpha);

        beta = fullcircle scaled (u / 4) shifted B cutbefore (B -- A) cutafter (B -- C);
        draw beta;
        label.llft("$\beta$", point 0.7 of beta);

        fill dot shifted A;
        fill dot shifted B;
        fill dot shifted C;

        label.llft("$(0, 0)$", A);
        label.urt("$(x_0, y_0)$", B);
        label.lrt("$(x_0, 0)$", C);

        label.rt("$a$", midpoint of (B -- C));
        label.bot("$b$", midpoint of (A -- C));
        label.ulft("$c$", midpoint of (A -- B));
        endfig;
      \end{mplibcode}\fi
    \end{minipage}
    \caption{An \enquote{abstract} right triangle in the \hyperref[def:euclidean_plane]{Euclidean plane} with legends for geometric sines and cosines and the same triangle in \( \BbbR^2 \) connecting the origin to a point \( (x_0, y_0) \) on the unit circle.}\label{fig:def:geometric_trigonometric_functions/triangle}
  \end{figure}

  The original \enquote{geometric definition} of \( \sin_G \) and \( \cos_G \) regards them as functions of an angle rather than numeric functions. \( \sin_G \) and \( \cos_G \) are only defined for two of the angles in a right triangle. The geometric definition is
  \begin{balign*}
    \sin_G(\alpha) \coloneqq \frac{\len(b)} {\len(c)}, &  & \cos_G(\alpha) \coloneqq \frac{\len(a)} {\len(c)},
    \\
    \sin_G(\beta) \coloneqq \frac{\len(b)} {\len(c)},  &  & \cos_G(\beta) \coloneqq \frac{\len(a)} {\len(c)}.
  \end{balign*}

  In our case, \( \len(a) = y_0 \), \( \len(b) = x_0 \) and \( \len(c) = 1 \). Furthermore, \( \sin_G(\beta) \) nor \( \cos_G(
  \beta) \) are immaterial to our subsequent arguments and we only introduced them for the sake of having a full definition.

  Therefore we conclude that
  \begin{balign*}
    \sin_G(\alpha) = x_0,
     &  &
    \cos_G(\alpha) = y_0.
  \end{balign*}

  To see that \( \sin_G \) and \( \cos_G \) are somewhat analogous to \( \sin \) and \( \cos \), notice that by \fullref{thm:arctantwo}, there exists a unique \( t_0 \coloneqq \arctantwo(y_0, x_0) \) such that
  \begin{balign*}
    \sin(t_0) = x_0,
     &  &
    \cos(t_0) = y_0.
  \end{balign*}

  Therefore our \hyperref[def:trigonometric_functions]{analytic definition} of the trigonometric functions as numeric functions correspond to the classical geometric definition in the special case where we consider the angle near the origin in the triangle formed by the vertices \fullref{def:geometric_trigonometric_functions/vertices}. This motivates \enquote{measuring} angles using the obtained correspondence. This unit of measurement is called a \term{radian}. We say that the angle \( \alpha \) is \( t_0 \) \term{radians}. Outside of mathematics, it is more conventional to use \term{degrees}, which are obtained from radians by scaling with \( \tfrac {180} {\pi} \). That is, \( \alpha \) is \( \tfrac {180} {\pi} t_0 \) degrees.
\end{definition}
