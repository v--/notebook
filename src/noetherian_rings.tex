\subsection{Noetherian rings}\label{subsec:noetherian_rings}

\begin{definition}\label{def:noetherian_module}\mcite\cite[prop. 8.30]{Knapp2016BasicAlgebra}
  A module \( M \) over \( R \) is called \term{Noetherian} if it satisfies any of the following conditions:
  \begin{thmenum}
    \ilabel{def:noetherian_module/ascending_chain} Every strict chain of submodules of \( M \)
    \begin{equation*}
      M_1 \subsetneq M_2 \subsetneq \ldots
    \end{equation*}
    is finite.

    \ilabel{def:noetherian_module/finite_basis} Every submodule of \( M \) is finitely \hyperref[def:free_left_module]{generated}.
  \end{thmenum}
\end{definition}
\begin{proof}
  \ImplicationSubProof{def:noetherian_module/ascending_chain}{def:noetherian_module/finite_basis} We can construct a basis as follows: choose\AOC any element \( x_1 \) of \( M \). Next, choose\AOC an element \( x_2 \in M \setminus \braket {x_1} \), then \( x_3 \in M \setminus \braket {x_1, x_2} \) and so on.

  This process must stop after finitely many steps because we obtain the strict chain
  \begin{equation*}
    \braket {x_1} \subsetneq \braket {x_1, x_2} \cdots
  \end{equation*}
  of submodules.

  \ImplicationSubProof{def:noetherian_module/finite_basis}{def:noetherian_module/ascending_chain} Suppose that all submodules of \( M \) are finitely generated. Let
  \begin{equation*}
    M_1 \subsetneq M_2 \subsetneq \ldots
  \end{equation*}
  be a strictly ascending chain of submodules.

  Then the submodule \( N \coloneqq \bigcup_{i=1}^\infty M_i \) is also finitely generated. There exists a member \( M_N \) of the chain containing all the generators of \( N \). Then no further strict inclusion of modules is possible. We conclude that the chain
  \begin{equation*}
    M_1 \subsetneq M_2 \subsetneq \ldots
  \end{equation*}
  is finite.
\end{proof}

\begin{definition}\label{def:noetherian_ring}
  A \term{Noetherian ring} is an Noetherian submodule over itself, i.e. it satisfies the conditions in \fullref{def:noetherian_module} on its ideals.
\end{definition}

\begin{theorem}[Hilbert basis theorem]\label{thm:hilberts_basis_theorem}\mcite\cite[418]{Knapp2016BasicAlgebra}
  If \( R \) is Noetherian, so is \( R[X] \).
\end{theorem}

\begin{theorem}\label{thm:noetherian_rings_closed_under}
  The \hyperref[def:set_zfc]{classes} of Noetherian \hyperref[def:noetherian_module]{modules} and Noetherian \hyperref[def:noetherian_ring]{rings} are closed with respect to many operations, including the following:

  \begin{thmenum}
    \ilabel{thm:noetherian_rings_closed_under/localization}\cite[corollary 8.48]{Knapp2016BasicAlgebra} The \hyperref[def:ring_localization]{localizations} \( S^{-1} R \) of a Noetherian ring \( R \) are also Noetherian.

    \ilabel{thm:noetherian_rings_closed_under/modules}\cite[proposition 8.34]{Knapp2016BasicAlgebra} Any \hyperref[def:free_left_module]{finitely generated module} over a Noetherina ring is also \hyperref[def:noetherian_module]{Noetherian}.

    \ilabel{thm:noetherian_rings_closed_under/submodules}\cite[proposition 6.3(а)]{Коцев2016} Every submodule of a Noetherian module is Noetherian.

    \ilabel{thm:noetherian_rings_closed_under/quotients}\cite[proposition 6.3(a)]{Коцев2016} Every quotient of a Noetherian module is Noetherian.

    \ilabel{thm:noetherian_rings_closed_under/restoration}\cite[proposition 6.3(b)]{Коцев2016} If the module \( M \) has a Noetherian submodule \( N \) such that \( M / N \) is also Noetherian, then \( M \) itself is Noetherian.

    \ilabel{thm:noetherian_rings_closed_under/polynomial_ring} The polynomial ring \( R[X] \) over a Noetherian ring \( R \) is also Noetherian. See \fullref{thm:hilberts_basis_theorem}.
  \end{thmenum}
\end{theorem}

\begin{theorem}[Primary decomposition]\label{thm:primary_decomposition}\mcite\cite[thm. 10.6]{Коцев2016}
  In a Noetherian ring \( R \), every ideal \( I \) can be represented as an intersection
  \begin{equation*}
    I = \bigcap_{i=1}^n P_i
  \end{equation*}
  of finitely many primary \hyperref[def:primary_ring_ideal]{ideals}.
\end{theorem}
