\subsection{Axiom of choice}\label{subsec:axiom_of_choice}

\begin{remark}\label{remark:aoc}
  \Fullref{thm:aoc} famously leads to counterintuitive results, which is a reason why it is frowned upon. It is insanely useful in the form of its many equivalent statements (see \fullref{def:set_zfc/A8} and \fullref{sec:index}).
\end{remark}

\begin{definition}\label{def:choice_function}
  Let \( X \) be a nonempty set. A \Def{choice function} is a function of type \( f: \Pow(X) \to X \) such that, for all subsets \( A \subseteq X \),
  \begin{equation*}
    f(A) \in A.
  \end{equation*}

  That is, the function \( f \) \enquote{chooses} an element out of any subset of \( X \).
\end{definition}

\begin{theorem}[Axiom of choice]\label{thm:aoc}
  The following are equivalent:

  \begin{ThmEnum}
    \ILabel{thm:aoc/transversal}\MarginCite[thm. 6M(4)]{Enderton1977}Every hypergraph has a minimal transversal (compare to \fullref{thm:finite_hypergraphs_have_minimal_transversal} that only mentions finite sets).

    \ILabel{thm:aoc/product}\MarginCite[thm. 6M(2)]{Enderton1977}The \hyperref[def:cartesian_product]{Cartesian product} of a family of nonempty sets is nonempty.

    \medskip

    \ILabel{thm:aoc/choice}\MarginCite[thm. 6M(3)]{Enderton1977}For any nonempty set \( X \) there exists a \hyperref[def:choice_function]{choice function}.

    \medskip

    \ILabel{thm:aoc/function}\MarginCite[thm. 6M(1)]{Enderton1977}Every \hyperref[def:function/multivalued]{multivalued function} has a \hyperref[def:function/selection]{selection}.

    \ILabel{thm:aoc/cardinals} \Fullref{thm:cardinal_trichotomy}.

    \ILabel{thm:aoc/zorn} \Fullref{thm:zorns_lemma}.

    \ILabel{thm:aoc/well_ordering_principle} \Fullref{thm:well_ordering_principle}.

    \ILabel{thm:aoc/vector_space_bases} \Fullref{thm:all_vector_spaces_are_free_left_modules}. Compare this to finite-dimensional vector spaces of order \( n \) over \( \BK \), which are all isomorphic to \( \BK^n \).

    \ILabel{thm:aoc/tychonoff} \Fullref{thm:tychonoffs_product_theorem}.

    \ILabel{thm:aoc/krull} \Fullref{thm:krulls_theorem}.
  \end{ThmEnum}
\end{theorem}
