\subsection{Axiom of choice}\label{subsec:axiom_of_choice}

\begin{remark}\label{remark:aoc}
  The axiom of choice (\cref{thm:aoc}) famously leads to counterintuitive results, which is a reason why it is frowned upon. It is insanely useful in the form of its many equivalent statements (see \cref{subsec:aoc} and \cref{sec:index}).
\end{remark}

\begin{definition}\label{def:choice_function}
  Let \( X \) be a nonempty set. A \Def{choice function} is a function of type \( f: \Power(X) \to X \) such that, for all subsets \( A \subseteq X \),
  \begin{equation*}
    f(A) \in A.
  \end{equation*}

  That is, the function \( f \) \enquote{chooses} an element out of any subset of \( X \).
\end{definition}

\begin{theorem}\label{thm:aoc}
  The following are equivalent:

  \begin{thmenum}
    \DItem{thm:aoc/choice}\cite[theorem 6M(3)]{Enderton1977} For any nonempty set \( X \) there exists a choice function\Tinyref{def:choice_function}

    \DItem{thm:aoc/zorn}\cite[theorem 6M(6)]{Enderton1977} Let \( X \) be a nonempty set and consider the partial order induced by the subset relation\Tinyref{def:subset}. If for any chain\Tinyref{def:poset/chain} \( X' \subseteq X \) we have \( \bigcup X' \in X \), there exists\AOC a maximal\Tinyref{def:poset/maximal_minimal_element} set in \( X \).

    \DItem{thm:aoc/vector_space_bases} Every vector space has a basis.
  \end{thmenum}
\end{theorem}
\begin{proof}
  \Implies[thm:aoc/zorn][thm:aoc/vector_space_bases] See \cref{thm:all_vector_spaces_are_free_modules}.
\end{proof}
