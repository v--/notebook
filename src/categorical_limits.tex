\begin{definition}\label{def:categorical_cone}\cite[definition 5.1.19(a)]{Leinster2014}
  Let $\Bold A$ be a category and $\Bold I$ be an \uline{indexing} category. Let $D: \Bold I \to \Bold A$ be a functor.

  A \uline{cone} on $D$ consists of
  \begin{itemize}
    \item an object $A \in \Bold A$, called the \uline{vertex} of the cone
    \item a family of morphisms $\{ f_i: A \to D(I) \}_{i \in \Bold I}$
  \end{itemize}
  such that for all morphisms $u: i \to j$ in $\Bold I$, the following diagram commutes:
  \begin{center}
    \begin{tikzcd}
      & A \arrow[ld, "f_i"'] \arrow[rd, "f_j"] & \\
      D(i) \arrow[rr, "D(u)"'] & & D(j)
    \end{tikzcd}
  \end{center}
\end{definition}

\begin{definition}\label{def:categorical_limit}\cite[definition 5.1.19(b)]{Leinster2014}
  Let $\Bold A$ be a category and $\Bold I$ be an indexing category. The (unique up to an isomorphism, if it exists) \uline{limit of $D$} is a cone $\{ L \overset {p_i} \to D(i) \}_{i \in \Bold I}$ such that for every cone $\{ A \overset {f_i} \to D(i) \}_{i \in \Bold I}$ there exists exactly one morphism $f: A \to L$ such that $p_i \circ f = f_i, i \in \Bold I$, i.e. the following diagram commutes:
  \begin{center}
    \begin{tikzcd}
      A \arrow[rd, "f" description, dotted] \arrow[rrd, "f_i"] \arrow[rdd, "f_j"'] &                                               &   \\
                                                                                   & L \arrow[r, "p_i"'] \arrow[d, "p_j"] & D(i) \\
                                                                                   & D(j)                                             &
    \end{tikzcd}
  \end{center}
\end{definition}

\begin{definition}\label{def:categorical_product}\cite[definition 5.1.1, 5.1.7]{Leinster2014}
  If the indexing category $\Bold I$ is discrete, then any functor $D: \Bold I \to \Bold A$ is simply an indexed family $\{ X_i \}_{i \in \Bold I}$ of objects of $\Bold A$. In this case, the limit $L$ does not depend on the functor $D$. We call it the \uline{product in $\Bold A$ indexed by $\Bold I$} and denote it by $\prod_{i \in \Bold I} X_i$.

  Explicitly, the \uline{product of $\{ X \}_{i \in \Bold I}$} is an object $\prod_{i \in \Bold I} X_i$ with associated \uline{projection morphisms} $\{ p_j: \prod_{i \in \Bold I} X_i \to X_j \}_{j \in \Bold I}$, that satisfy the following universal property: for any object $A$ and any family of morphisms $\{ f_j: A \to {X_j} \}_{j \in \Bold I}$ there exists exactly one morphism $f: A \to \prod_{i \in \Bold I} X_i$ such that for every $j \in \Bold I$ we have $p_j \circ f = f_j$, i.e. the following diagram commutes:
  \begin{center}
    \begin{tikzcd}
      A \arrow[rd, "f_j"'] \arrow[rr, "f", dotted] & & \prod_{i \in \Bold I} X_i \arrow[ld, "p_j"] \\
      & X_j &
    \end{tikzcd}
  \end{center}

  The function $f$ is also denoted as $\{ f_i \}_{i \in \Bold I}$.

  In particular, for two objects $X, Y \in \Bold A$ (i.e. when $\Bold I$ is a two-object discrete category), the product is an object $X \times Y$ with projections $p_X: X \times Y \to X$ and $p_Y: X \times Y \to Y$ such that the following diagram commutes:
  \begin{center}
    \begin{tikzcd}
      A \arrow[rd, "f" description, dotted] \arrow[rrd, "f_Y"] \arrow[rdd, "f_X"'] &                                               &   \\
                                                                                   & X \times Y \arrow[r, "p_Y"'] \arrow[d, "p_X"] & Y \\
                                                                                   & X                                             &
    \end{tikzcd}
  \end{center}
\end{definition}

\begin{note}\label{note:small_categorical_product}
  If the discrete category $\Bold I$ is small, denote the set of its objects by $I$. This allows us to talk about products of families $\{ X_i \}_{i \in I}$ indexed by the set $I$ rather than the category $\Bold I$.
\end{note}

\begin{note}\label{note:categorical_product_of_empty_family}
  The product $\prod_{i \in \varnothing} X_i$ of an empty family of objects is the terminal object of the category.
\end{note}

\begin{example}\label{categorical_product/top}
  Let $\{ X_i \}_{i \in I}$ be an indexed family in the category $\Bold{Top}$ of topological spaces. Their product $\prod_{i \in I} X_i$ is the topological space whose set is the Cartessian product of the underlying sets and whose topology is called the \uline{product topology}. It is the smallest topology for which the projections are continuous functions.
\end{example}

\begin{definition}\label{def:categorical_fork}\cite[112]{Leinster2014}
  A \uline{fork} in the category $\Bold A$ is a commutative diagram of the form
  \begin{center}
    \begin{tikzcd}
      A \arrow[r, "f"] & X \arrow[r, shift left=1, "s"] \arrow[r, shift right=1, "t"'] & Y
    \end{tikzcd}
  \end{center}

  Commutativity simply means that $s \circ f = t \circ f$.
\end{definition}

\begin{definition}\label{def:categorical_equalizer}\cite[definition 5.1.11]{Leinster2014}
  Assume that the indexing category $\Bold I$ consists of two objects and two unidirectional morphisms:
  \begin{center}
    \begin{tikzcd}
      \bullet \arrow[r, shift left=1] \arrow[r, shift right=1] & \bullet
    \end{tikzcd}
  \end{center}

  Diagrams $D$ of shape $\Bold I$ are simply subcategories of $\Bold A$ of the shape
  \begin{center}
    \begin{tikzcd}
      X \arrow[r, shift left=1, "s"] \arrow[r, shift right=1, "t"'] & Y
    \end{tikzcd}
  \end{center}

  Cones with vertex $A$ are then given by commutative diagrams of shape
  \begin{center}
    \begin{tikzcd}
      & A \arrow[ld, "f"'] \arrow[rd, "g"] & \\
      X \arrow[rr, shift left=1, "s"] \arrow[rr, shift right=1, "t"'] && Y
    \end{tikzcd}
  \end{center}

  Since the morphism $g: A \to Y$ is determined uniquely by $f$ and $s$, the cones are actually forks:
  \begin{center}
    \begin{tikzcd}
      A \arrow[r, "f"] & X \arrow[r, shift left=1, "s"] \arrow[r, shift right=1, "t"'] & Y
    \end{tikzcd}
  \end{center}

  The limit $(L, l)$ of $D$ then satisfies the universal property: for any fork starting in $A \in \Bold A$, there exists a unique morphism $f: A \to L$ such that the following diagram commutes:
  \begin{center}
    \begin{tikzcd}
      A \arrow[rd, "f"] \arrow[dd, "f", dotted] & & \\
      & X \arrow[r, shift left=1, "s"] \arrow[r, shift right=1, "t"'] & Y \\
      L \arrow[ru, "l"'] & &
    \end{tikzcd}
  \end{center}

  This limit is called the \uline{equalizer of $s$ and $t$}.
\end{definition}

\begin{example}\label{ex:categorical_equalizer/top}
  An equalizer of two continuous functions $f, g: X \to Y$ in $\Bold{Top}$ is the topological space
  \begin{align*}
    \{ x \in X \colon f(x) = g(x) \}
  \end{align*}
  with the subspace topology.

  Compare this with pullbacks (see \cref{ex:categorical_pullback/top}).
\end{example}

\begin{definition}\label{def:categorical_pullback}\cite[definition 5.1.16]{Leinster2014}
  Assume that the indexing category $\Bold I$ has the shape
  \begin{center}
    \begin{tikzcd}
      \bullet \arrow[r] & \bullet & \bullet \arrow[l]
    \end{tikzcd}
  \end{center}

  Cones of shape $\Bold I$ with vertex $A$ are then given by commutative diagrams of shape
  \begin{center}
    \begin{tikzcd}
      A \arrow[r, "f_Y"] \arrow[d, "f_X"'] & Y \arrow[d, "t"] & \\
      X \arrow[r, "s"'] & Z
    \end{tikzcd}
  \end{center}

  The limit $(L, p_X, p_Y)$ of $D$ then satisfies the universal property: for any $\Bold I$-shaped diagram starting in $A \in \Bold A$, there exists a unique morphism $f: A \to L$ such that the following diagram commutes:
  \begin{center}
    \begin{tikzcd}
      A \arrow[rd, "f" description, dotted] \arrow[rrd, "f_Y"] \arrow[rdd, "f_X"'] &                                               &   \\
                                                                                   & L \arrow[r, "p_Y"'] \arrow[d, "p_X"] & Y \arrow[d, "t"] \\
                                                                                   & X \arrow[r, "s"']                             & Z
    \end{tikzcd}
  \end{center}

  This limit is called the \uline{pullback or fibered product of $s$ and $t$}.
\end{definition}

\begin{example}\label{ex:categorical_pullback/top}
  The pullback of two continuous functions $f: X \to Z$ and $g: Y \to Z$ in $\Bold{Top}$ is the topological space
  \begin{align*}
    \{ (x, y) \in X \times Y \colon f(x) = g(y) \}
  \end{align*}
  with the product topology. Compare this with equalizers (see \cref{ex:categorical_equalizer/top}).
\end{example}
