\subsection{Categorical limits}\label{subsec:categorical_limits}

\begin{definition}\label{def:comma_category}\mcite[def. 2.3.1]{Leinster2016Basic}
  Comma categories allow us to define morphisms between morphisms, which becomes useful in, for example, the concise definition of a limit in \fullref{def:category_of_cones/limit}.

  \begin{thmenum}
    \thmitem{def:comma_category/variable} We first prove the most general construction. Let \( F: \cat{A} \to \cat{C} \) and \( G: \cat{B} \to \cat{C} \) be any two functors with a common codomain. We define their \term{comma category} \( (P \downarrow G) \) as follows:

    \begin{itemize}
      \item The \hyperref[def:category/objects]{set of objects} \( \obj(F \downarrow G) \) is the set of triples \( (A, s, B) \), where \( A \in \cat{A} \), \( B \in \cat{B} \) and \( s: F(A) \to G(B) \).

      \item The \hyperref[def:category/morphisms]{set of morphisms} \( [F \downarrow G]((A, s, B), (A', s', B')) \) is the set of pairs
      \begin{equation*}
        (f: A \to A', g: B \to B'),
      \end{equation*}
      such that the following diagram commutes:
      \begin{equation}\label{eq:def:comma_category/variable}
        \begin{aligned}
          \includegraphics[page=1]{output/def__comma_category.pdf}
        \end{aligned}
      \end{equation}

      \item The \hyperref[def:category/composition]{composition of morphisms} is their pairwise composition, analogically to \hyperref[def:product_category]{product categories}.

      \item The \hyperref[def:category/identity]{identity morphism} on the object \( (A, s, B) \) is the identity pair \( (\id_A, \id_B) \).
    \end{itemize}

    \thmitem{def:comma_category/fixed} It is often the case where either \( F \) or \( G \) are constant. If, instead of \( G \), we are given an object \( X \) in \( \cat{C} \), we use the constant functor \( \Delta_X: \cat{1} \to \cat{C} \) with domain the \hyperref[def:universal_categories]{terminal category} \( \cat{1} \), in order to use the comma category \( (F \downarrow \Delta_X) \).

    We can thus simplify \fullref{def:comma_category/variable} as follows:

    \begin{minipage}[t]{0.43\textwidth}
      \begin{equation*}
        (F \downarrow X) \coloneqq (F \downarrow \Delta_X).
      \end{equation*}

      \begin{itemize}
        \item The \hyperref[def:category/objects]{set of objects} \( \obj(F \downarrow X) \) is
        \begin{equation*}
          \set{ (A, s) \given s: F(A) \to X }
        \end{equation*}

        \item The \hyperref[def:category/morphisms]{hom-set}
        \begin{equation*}
          [F \downarrow X]\parens[\Big]{ (A, s), (A, s') }
        \end{equation*}
        is the set of \( f: A \to A' \), such that
        \begin{equation}\label{eq:def:comma_category/fixed/right}
          \begin{aligned}
            \includegraphics[page=2]{output/def__comma_category.pdf}
          \end{aligned}
        \end{equation}
      \end{itemize}
    \end{minipage}
    \begin{minipage}[t]{0.43\textwidth}
      \begin{equation*}
        (X \downarrow G) \coloneqq (\Delta_X \downarrow G)
      \end{equation*}

      \begin{itemize}
        \item The \hyperref[def:category/objects]{set of objects} \( \obj(X \downarrow G) \) is
        \begin{equation*}
          \set{ (s, B) \given s: X \to F(B) }
        \end{equation*}

        \item The \hyperref[def:category/morphisms]{hom-set}
        \begin{equation*}
          [X \downarrow G]\parens[\Big]{ (s, B), (s', B') }
        \end{equation*}
        is the set of \( g: B \to B' \), such that
        \begin{equation}\label{eq:def:comma_category/fixed/left}
          \begin{aligned}
            \includegraphics[page=3]{output/def__comma_category.pdf}
          \end{aligned}
        \end{equation}
      \end{itemize}
    \end{minipage}
  \end{thmenum}
\end{definition}

\begin{definition}\label{def:category_of_cones}
  Let \( D: \cat{I} \to \cat{C} \) be a \hyperref[def:categorical_diagram]{diagram}.

  \begin{thmenum}
    \thmitem{def:category_of_cones/category} We define \term{category of cones} to \( D \) as the \hyperref[def:comma_category/fixed]{constant-functor comma category}
    \begin{equation*}
      \cat{Cone}(D) \coloneqq (\Delta \downarrow D),
    \end{equation*}
    where \( \Delta: \cat{C} \to [\cat{I}, \cat{C}] \) is the \( \cat{I} \)-shaped \hyperref[def:diagonal_functor]{diagonal functor} on \( \cat{C} \).

    \thmitem{def:category_of_cones/cone} A \term{cone} is simply a member of \( \cat{Cone}(D) \).

    Explicitly, a cone with \term{vertex} \( A \in \cat{C} \) is a pair \( (\Delta_A, \alpha) \), where \( \Delta_A \) is the constant functor at \( A \) and \( \alpha \) is a natural transformation from \( \Delta_A \) to \( D \).

    Even more explicitly, a cone is a family of morphisms
    \begin{equation}\label{eq:def:category_of_cones/cone}
      \seq{ \alpha_i: A \to D(i) }_{i \in \cat{I}}.
    \end{equation}
    satisfying a simplified naturality condition (compared to \eqref{eq:def:natural_transformation/diagram}). For every morphism \( u: i \to j \) in \( \cat{I} \), the following diagram must commute:
    \begin{equation}\label{eq:def:category_of_cones/cone_nat}
      \begin{aligned}
        \includegraphics[page=1]{output/def__category_of_cones.pdf}
      \end{aligned}
    \end{equation}

    \thmitem{def:category_of_cones/limit} A \term{limit} of the diagram \( D \) is a \hyperref[def:universal_objects/terminal]{terminal object} of the cone category \( \cat{Cone}(D) \).

    Explicitly, a limit is a cone \( (L, \lambda) \) such that, for every cone \( (A, \alpha) \), there exists a unique \hyperref[eq:def:comma_category/fixed]{cone morphism} \( l_A: (A, \alpha) \to (L, \lambda) \).

    Limits are also called \term{projective limits} and, somewhat confusingly, \term{inverse limits}.

    Even more explicitly, a limit is a cone \( (L, \lambda) \) such that, for every cone \( (A, \alpha) \), there exists a unique morphism \( l_A: A \to L \) such that following diagram commutes for every index morphism \( u: i \to j \):
    \begin{equation}\label{eq:def:category_of_cones/limit}
      \begin{aligned}
        \includegraphics[page=2]{output/def__category_of_cones.pdf}
      \end{aligned}
    \end{equation}

    From \fullref{thm:def:universal_objects/properties/terminal} and \fullref{thm:universal_objects_as_adjunctions/terminal} it follows that a limit, if it exists, is unique up to a unique isomorphism.

    \thmitem{def:category_of_cones/colimit} A \term{colimit} of the diagram \( D \) is an \hyperref[def:universal_objects/initial]{initial object} of the \term{cocone} category \( (D \downarrow \Delta) \). Refer to \fullref{def:comma_category/fixed} for how cocones differ from cones.

    There are two major differences compared to limits: a colimit is an initial object, not a terminal object, and its underlying comma category is \( (D \downarrow \Delta) \), not \( (\Delta \downarrow D) \).

    The analogous diagram to \eqref{eq:def:category_of_cones/limit} is exactly its \hyperref[rem:categorical_principle_of_duality]{dual}:
    \begin{equation}\label{eq:def:category_of_cones/colimit}
      \begin{aligned}
        \includegraphics[page=3]{output/def__category_of_cones.pdf}
      \end{aligned}
    \end{equation}

    Colimits are also called \term{inductive limits} and, somewhat confusingly, \term{direct limits}.
  \end{thmenum}
\end{definition}
