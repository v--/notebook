\subsection{Limits}\label{subsec:categorical_limits}

\begin{remark}\label{def:categorical_limit_examples}
  Examples of limits and colimits can be found in \fullref{thm:set_categorical_limits}, \fullref{thm:group_categorical_limits} and \fullref{subsec:initial_final_topologies}.
\end{remark}

\begin{definition}\label{def:diagonal_functor}\mcite[143]{Leinster2016Basic}
  Let \( \cat{I} \) be a small index category and let \( \cat{A} \) be any category. For each object \( A \in \cat{A} \), we define the functor \( \Delta A: \cat{I} \to \cat{A} \) as
  \begin{itemize}
    \item for every object \( k \in \cat{I} \), define \( \Delta A(k) = A \)
    \item for every morphism \( u: k \to \beta \), define \( \Delta A(u) = \id_A \)
  \end{itemize}

  We combine these functors for every object \( A \in \cat{A} \) to obtain the functor \( \Delta: \cat{A} \to [\cat{I}, \cat{A}] \).
\end{definition}

\begin{definition}\label{def:categorical_cone}\mcite[def. 5.1.19(a)]{Leinster2016Basic}
  Let \( \cat{A} \) be a category and \( \cat{I} \) be a \hyperref[def:categorical_diagram]{category} which we shall call an \term{index} category. Let \( D: \cat{I} \to \cat{A} \) be a diagram. A \term{cone} on \( D \) can be defined equivalently as:

  \begin{thmenum}
    \thmitem{def:categorical_cone/explicit} A family of \term{projection} morphisms \( \{ \pi_k: A \to D(k) \}_{k \in \cat{I}} \) from the \term{vertex} \( A \) such that for all morphisms \( u: k \to \beta \) in \( \cat{I} \), the following diagram commutes:
    \begin{alignedeq}\label{def:categorical_cone/universal_property}
      \text{\todo{Add diagram}}\iffalse\begin{mplibcode}
        beginfig(1);
        input metapost/graphs;

        v1 := thelabel("$A$", origin);
        v2 := thelabel("$D(k)$", (-1, -1) scaled u);
        v3 := thelabel("$D(\beta)$", (1, -1) scaled u);

        a1 := straight_arc(v1, v2);
        a2 := straight_arc(v1, v3);
        a3 := straight_arc(v2, v3);

        draw_vertices(v);
        draw_arcs(a);

        label.ulft("$\pi_k$", straight_arc_midpoint of a1);
        label.urt("$\pi_\beta$", straight_arc_midpoint of a2);
        label.bot("$D(u)$", straight_arc_midpoint of a3);
        endfig;
      \end{mplibcode}\fi
    \end{alignedeq}

    \thmitem{def:categorical_cone/natural} A natural transformation in \( [\cat{I}, \cat{A}](\Delta A, D) \).

    \thmitem{def:categorical_cone/comma} An object of the comma category \( (\Delta \downarrow D) \) (see the equivalence proof for details).
  \end{thmenum}
\end{definition}
\begin{proof}
  \EquivalenceSubProof{def:categorical_cone/explicit}{def:categorical_cone/natural} Let \( k, \beta \in \cat{I} \) and \( u: k \to \beta \). Then a natural transformation \( f \) in  satisfies the following commutative diagram:
  \begin{equation*}
    \text{\todo{Add diagram}}\iffalse\begin{mplibcode}
      beginfig(1);
      input metapost/graphs;

      v1 := thelabel("$\Delta A(k)$", (-1, 0) scaled u);
      v2 := thelabel("$\Delta A(\beta)$", (1, 0) scaled u);
      v3 := thelabel("$D(k)$", (-1, -1) scaled u);
      v4 := thelabel("$D(\beta)$", (1, -1) scaled u);

      a1 := straight_arc(v1, v2);
      a2 := straight_arc(v1, v3);
      a3 := straight_arc(v2, v4);
      a4 := straight_arc(v3, v4);

      draw_vertices(v);
      draw_arcs(a);

      label.top("$\Delta A(u)$", straight_arc_midpoint of a1);
      label.lft("$\pi_k$", straight_arc_midpoint of a2);
      label.rt("$\pi_\beta$", straight_arc_midpoint of a3);
      label.bot("$D(u)$", straight_arc_midpoint of a4);
      endfig;
    \end{mplibcode}\fi
  \end{equation*}

  Since \( \Delta A(k) = \Delta A(\beta) = A \), the above diagram is the same as \fullref{def:categorical_cone/universal_property}.

  \EquivalenceSubProof{def:categorical_cone/natural}{def:categorical_cone/comma} We can regard \( D: \cat{I} \to \cat{A} \) as an object in the functor category \( [\cat{I}, \cat{A}] \). Since \( \Delta: \cat{A} \to [\cat{I}, \cat{A}] \), an object \( (A, h) \) in \( (\Delta \downarrow D) \) consists of an object \( A \) of \( \cat{A} \) and a natural transformation from \( \Delta A \) to \( D \). The converse also applies.
\end{proof}

\begin{definition}\label{def:categorical_limit}\mcite[def. 5.1.19(b), 6.3.6]{Leinster2016Basic}
  Let \( \cat{A} \) be a category and \( \cat{I} \) be an index category. The (unique up to an isomorphism, if it exists) \term{limit} or \term{limit cone} \( \varprojlim D \) of \( D \) is a cone
  \begin{equation*}
    \{ L \reloset {\pi_k} \to D(k) \}_{k \in \cat{I}}
  \end{equation*}
  such that for every cone
  \begin{equation*}
    \{ L' \reloset {\pi_k'} \to D(k) \}_{k \in \cat{I}}
  \end{equation*}
  there exists exactly one morphism \( f: L' \to L \) such that \( f \circ \pi_k' = \pi_k, k \in \cat{I} \), i.e. the following diagram commutes:
  \begin{equation*}
    \text{\todo{Add diagram}}\iffalse\begin{mplibcode}
      beginfig(1);
      input metapost/graphs;

      v1 := thelabel("$D(k)$", origin);
      v2 := thelabel("$L'$", (-1, 1) scaled u);
      v3 := thelabel("$L$", (1, 1) scaled u);

      a1 := straight_arc(v2, v1);
      a2 := straight_arc(v3, v1);

      d1 := straight_arc(v2, v3);

      draw_vertices(v);
      draw_arcs(a);

      drawarrow d1 dotted;

      label.llft("$\pi_k$", straight_arc_midpoint of a1);
      label.lrt("$\pi_k'$", straight_arc_midpoint of a2);
      label.top("$f$", straight_arc_midpoint of d1);
      endfig;
    \end{mplibcode}\fi
  \end{equation*}

  If the diagram \( \cat{I} \) is small, its limit is called a \term{small limit}. If a category \( \cat{A} \) has all small limits, it is called \term{complete}.
\end{definition}

\begin{definition}\label{def:categorical_product}\mcite[def. 5.1.1, 5.1.7]{Leinster2016Basic}
  If the index category \( \cat{I} \) is discrete, then any diagram \( D: \cat{I} \to \cat{A} \) is simply an indexed family \( \{ X_k \}_{k \in \cat{I}} \) of objects of \( \cat{A} \). In this case, the limit \( L \) does not depend on the functor \( D \). We call it the \term{product in \( \cat{A} \) indexed by \( \cat{I} \)} and denote it by \( \prod_{k \in \cat{I}} X_k \).

  Explicitly, the \term{product} of \( \{ X_k \}_{k \in \cat{I}} \) is an object \( P \coloneqq \prod_{k \in \cat{I}} X_k \) with associated \term{projection morphisms} \( \{ \pi_k: P \to X_k \}_{k \in \cat{I}} \), that satisfy the following universal property: for any object \( P' \) and any family of morphisms \( \{ \pi_k': P' \to {X_k} \}_{k \in \cat{I}} \) there exists exactly one morphism \( f: P' \to P \) such that for every \( k \in \cat{I} \) we have \( f \circ \pi_k = \pi_k' \), i.e. the following diagram commutes:
  \begin{equation*}
    \text{\todo{Add diagram}}\iffalse\begin{mplibcode}
      beginfig(1);
      input metapost/graphs;

      v1 := thelabel("$X_k$", origin);
      v2 := thelabel("$P'$", (-1, 1) scaled u);
      v3 := thelabel("$P$", (1, 1) scaled u);

      a1 := straight_arc(v2, v1);
      a2 := straight_arc(v3, v1);

      d1 := straight_arc(v2, v3);

      draw_vertices(v);
      draw_arcs(a);

      drawarrow d1 dotted;

      label.llft("$\pi_k'$", straight_arc_midpoint of a1);
      label.lrt("$\pi_k$", straight_arc_midpoint of a2);
      label.top("$f$", straight_arc_midpoint of d1);
      endfig;
    \end{mplibcode}\fi
  \end{equation*}

  The function \( f \) is also denoted as \( \{ f_k \}_{k \in \cat{I}} \).

  In particular, for two objects \( X, Y \in \cat{A} \) (i.e. when \( \cat{I} \) is a two-object discrete category), the product is an object \( X \times Y \) with projections \( \pi_X: X \times Y \to X \) and \( \pi_Y: X \times Y \to Y \) such that for each object \( P' \) and morphisms \( \pi_X': P' \to X \) and \( \pi_Y': P' \to Y \) the following diagram commutes:
  \begin{equation*}
    \text{\todo{Add diagram}}\iffalse\begin{mplibcode}
      beginfig(1);
      input metapost/graphs;

      v1 := thelabel("$P'$", (-1, 1) scaled u);
      v2 := thelabel("$X \times Y$", origin);
      v3 := thelabel("$X$", (0, -1) scaled u);
      v4 := thelabel("$Y$", (1, 0) scaled u);

      a1 := straight_arc(v1, v3);
      a2 := straight_arc(v1, v4);
      a3 := straight_arc(v2, v3);
      a4 := straight_arc(v2, v4);

      d1 := straight_arc(v1, v2);

      draw_vertices(v);
      draw_arcs(a);

      drawarrow d1 dotted;

      label.llft("$\pi_X'$", straight_arc_midpoint of a1);
      label.urt("$\pi_Y'$", straight_arc_midpoint of a2);
      label.rt("$\pi_X$", straight_arc_midpoint of a3);
      label.bot("$\pi_Y$", straight_arc_midpoint of a4);

      fill fullcircle scaled 0.25u shifted (center d1) withcolor white;
      label("$f$", straight_arc_midpoint of d1);
      endfig;
    \end{mplibcode}\fi
  \end{equation*}
\end{definition}

\begin{remark}\label{rem:small_categorical_product}
  If the discrete category \( \cat{I} \) is small, denote the set of its objects by \( I \). This allows us to talk about products of families \( \{ X_k \}_{k \in I} \) indexed by the set \( I \) rather than the category \( \cat{I} \).
\end{remark}

\begin{remark}\label{rem:empty_categorical_product}
  The product \( \prod_{k \in \varnothing} X_k \) of an empty family of objects is the terminal object of the category.
\end{remark}

\begin{definition}\label{def:categorical_fork}\mcite[112]{Leinster2016Basic}
  A \term{fork} in the category \( \cat{A} \) is a commutative diagram of the form
  \begin{equation*}
    \text{\todo{Add diagram}}\iffalse\begin{mplibcode}
      beginfig(1);
      input metapost/graphs;

      v1 := thelabel("$A$", origin);
      v2 := thelabel("$X$", (1, 0) scaled u);
      v3 := thelabel("$Y$", (2, 0) scaled u);

      a1 := straight_arc(v1, v2);
      a2 := straight_arc_shifted(v2, v3, (0, safe_arc_spacing));
      a3 := straight_arc_shifted(v2, v3, (0, -safe_arc_spacing));

      draw_vertices(v);
      draw_arcs(a);

      label.top("$f$", straight_arc_midpoint of a1);
      label.top("$s$", straight_arc_midpoint of a2);
      label.bot("$t$", straight_arc_midpoint of a3);
      endfig;
    \end{mplibcode}\fi
  \end{equation*}

  Commutativity simply means that \( s \circ f = t \circ f \).
\end{definition}

\begin{definition}\label{def:categorical_equalizer}\mcite[def. 5.1.11]{Leinster2016Basic}
  Assume that the index category \( \cat{I} \) consists of two objects and two unidirectional morphisms:
  \begin{equation*}
    \text{\todo{Add diagram}}\iffalse\begin{mplibcode}
      beginfig(1);
      input metapost/graphs;

      v1 := thelabel("$\bullet$", (1, 0) scaled u);
      v2 := thelabel("$\bullet$", (2, 0) scaled u);

      a1 := straight_arc_shifted(v1, v2, (0, safe_arc_spacing));
      a2 := straight_arc_shifted(v1, v2, (0, -safe_arc_spacing));

      draw_vertices(v);
      draw_arcs(a);
      endfig;
    \end{mplibcode}\fi
  \end{equation*}

  Diagrams \( D \) of shape \( \cat{I} \) are simply subcategories of \( \cat{A} \) of the shape
  \begin{equation*}
    \text{\todo{Add diagram}}\iffalse\begin{mplibcode}
      beginfig(1);
      input metapost/graphs;

      v1 := thelabel("$X$", (1, 0) scaled u);
      v2 := thelabel("$Y$", (2, 0) scaled u);

      a1 := straight_arc_shifted(v1, v2, (0, safe_arc_spacing));
      a2 := straight_arc_shifted(v1, v2, (0, -safe_arc_spacing));

      draw_vertices(v);
      draw_arcs(a);

      label.top("$s$", straight_arc_midpoint of a1);
      label.bot("$t$", straight_arc_midpoint of a2);
      endfig;
    \end{mplibcode}\fi
  \end{equation*}

  Cones with vertex \( A \) are then given by commutative diagrams of shape
  \begin{equation*}
    \text{\todo{Add diagram}}\iffalse\begin{mplibcode}
      beginfig(1);
      input metapost/graphs;

      v1 := thelabel("$A$", origin);
      v2 := thelabel("$X$", (-1, -1) scaled u);
      v3 := thelabel("$Y$", (1, -1) scaled u);

      a1 := straight_arc(v1, v2);
      a2 := straight_arc(v1, v3);
      a3 := straight_arc_shifted(v2, v3, (0, safe_arc_spacing));
      a4 := straight_arc_shifted(v2, v3, (0, -safe_arc_spacing));

      draw_vertices(v);
      draw_arcs(a);

      label.ulft("$f$", straight_arc_midpoint of a1);
      label.urt("$g$", straight_arc_midpoint of a2);
      label.top("$s$", straight_arc_midpoint of a3);
      label.bot("$t$", straight_arc_midpoint of a4);
      endfig;
    \end{mplibcode}\fi
  \end{equation*}

  Since the morphism \( g: A \to Y \) is determined uniquely by \( f \) and \( s \), the cones are actually forks:
  \begin{equation*}
    \text{\todo{Add diagram}}\iffalse\begin{mplibcode}
      beginfig(1);
      input metapost/graphs;

      v1 := thelabel("$A$", origin);
      v2 := thelabel("$X$", (1, 0) scaled u);
      v3 := thelabel("$Y$", (2, 0) scaled u);

      a1 := straight_arc(v1, v2);
      a2 := straight_arc_shifted(v2, v3, (0, safe_arc_spacing));
      a3 := straight_arc_shifted(v2, v3, (0, -safe_arc_spacing));

      draw_vertices(v);
      draw_arcs(a);

      label.top("$f$", straight_arc_midpoint of a1);
      label.top("$s$", straight_arc_midpoint of a2);
      label.bot("$t$", straight_arc_midpoint of a3);
      endfig;
    \end{mplibcode}\fi
  \end{equation*}

  The limit \( (L, l) \) of \( D \) then satisfies the universal property: for any fork \( (L', l') \), there exists a unique morphism \( f: L' \to L \) such that the following diagram commutes:
  \begin{equation*}
    \text{\todo{Add diagram}}\iffalse\begin{mplibcode}
      beginfig(1);
      input metapost/graphs;

      v1 := thelabel("$X$", origin);
      v2 := thelabel("$Y$", (1, 0) scaled u);
      v3 := thelabel("$L'$", (-1, 1) scaled u);
      v4 := thelabel("$L$", (-1, -1) scaled u);

      a1 := straight_arc_shifted(v1, v2, (0, safe_arc_spacing));
      a2 := straight_arc_shifted(v1, v2, (0, -safe_arc_spacing));
      a3 := straight_arc(v3, v1);
      a4 := straight_arc(v4, v1);

      d1 := straight_arc(v3, v4);

      draw_vertices(v);
      draw_arcs(a);

      drawarrow d1 dotted;

      label.top("$s$", straight_arc_midpoint of a1);
      label.bot("$t$", straight_arc_midpoint of a2);
      label.urt("$l'$", straight_arc_midpoint of a3);
      label.lrt("$l$", straight_arc_midpoint of a4);
      label.rt("$f$", straight_arc_midpoint of d1);
      endfig;
    \end{mplibcode}\fi
  \end{equation*}

  This limit is called the \term{equalizer} of \( s \) and \( t \).
\end{definition}

\begin{definition}\label{def:categorical_pullback}\mcite[def. 5.1.16]{Leinster2016Basic}
  Assume that the index category \( \cat{I} \) has the shape
  \begin{equation*}
    \bullet \longrightarrow \bullet \longleftarrow \bullet
  \end{equation*}

  Cones of shape \( \cat{I} \) with vertex \( A \) are then given by commutative diagrams of shape
  \begin{equation*}
    \text{\todo{Add diagram}}\iffalse\begin{mplibcode}
      beginfig(1);
      input metapost/graphs;

      v1 := thelabel("$A$", origin);
      v2 := thelabel("$X$", (0, -1) scaled u);
      v3 := thelabel("$Y$", (1, 0) scaled u);
      v4 := thelabel("$Z$", (1, -1) scaled u);

      a1 := straight_arc(v1, v2);
      a2 := straight_arc(v1, v3);
      a3 := straight_arc(v2, v4);
      a4 := straight_arc(v3, v4);

      draw_vertices(v);
      draw_arcs(a);

      label.lft("$\pi_X$", straight_arc_midpoint of a1);
      label.top("$\pi_Y$", straight_arc_midpoint of a2);
      label.bot("$s$", straight_arc_midpoint of a3);
      label.rt("$t$", straight_arc_midpoint of a4);
      endfig;
    \end{mplibcode}\fi
  \end{equation*}

  The limit \( (L, \pi_X, \pi_Y) \) then satisfies the universal property: for any \( \cat{I} \)-cone \( (L', \pi_X', \pi_Y') \), there exists a unique morphism \( f: L' \to L \) such that the following diagram commutes:
  \begin{equation*}
    \text{\todo{Add diagram}}\iffalse\begin{mplibcode}
      beginfig(1);
      input metapost/graphs;

      v1 := thelabel("$L$", origin);
      v2 := thelabel("$X$", (0, -1) scaled u);
      v3 := thelabel("$Y$", (1, 0) scaled u);
      v4 := thelabel("$Z$", (1, -1) scaled u);
      v5 := thelabel("$L'$", (-1, 1) scaled u);

      a1 := straight_arc(v1, v2);
      a2 := straight_arc(v1, v3);
      a3 := straight_arc(v2, v4);
      a4 := straight_arc(v3, v4);
      a5 := straight_arc(v5, v2);
      a6 := straight_arc(v5, v3);

      d1 := straight_arc(v5, v1);

      draw_vertices(v);
      draw_arcs(a);

      drawarrow d1 dotted;

      label.rt("$\pi_X$", straight_arc_midpoint of a1);
      label.bot("$\pi_Y$", straight_arc_midpoint of a2);
      label.bot("$s$", straight_arc_midpoint of a3);
      label.rt("$t$", straight_arc_midpoint of a4);
      label.llft("$\pi_X'$", straight_arc_midpoint of a5);
      label.urt("$\pi_Y'$", straight_arc_midpoint of a6);

      fill fullcircle scaled 0.25u shifted (center d1) withcolor white;
      label("$f$", straight_arc_midpoint of d1);
      endfig;
    \end{mplibcode}\fi
  \end{equation*}

  This limit is called the \term{pullback} or \term{fibered product} of \( s \) and \( t \).
\end{definition}

\begin{definition}\label{def:categorical_cocone}\mcite[def. 5.2.1]{Leinster2016Basic}
  The dual notion of a \hyperref[def:categorical_cone]{cone} is that of a cocone. Given a category \( \cat{A} \), an index category \( \cat{I} \) and a diagram \( D: \cat{I} \to \cat{A} \), we say that the family of morphisms
  \begin{equation*}
    \{ D(k) \reloset {\iota_k} \to A \}_{k \in \cat{I}}
  \end{equation*}
  is a \term{cocone} for D if it is a cone for \( D^{-1}: \cat{I}^{-1} \to \cat{A}^{-1} \).

  Explicitly, a \term{cocone} on \( D \) consists of
  \begin{itemize}
    \item an object \( A \in \cat{A} \), called the \term{vertex} of the cocone
    \item a family of \term{coprojection} morphisms \( \{ \iota_k: D(k) \to A \}_{k \in \cat{I}} \)
  \end{itemize}
  such that for all morphisms \( u: k \to \beta \) in \( \cat{I} \), the following diagram commutes:
  \begin{alignedeq}\label{def:categorical_cocone/universal_property}
    \text{\todo{Add diagram}}\iffalse\begin{mplibcode}
      beginfig(1);
      input metapost/graphs;

      v1 := thelabel("$A$", origin);
      v2 := thelabel("$D(k)$", (-1, 1) scaled u);
      v3 := thelabel("$D(\beta)$", (1, 1) scaled u);

      a1 := straight_arc(v2, v1);
      a2 := straight_arc(v3, v1);
      a3 := straight_arc(v2, v3);

      draw_vertices(v);
      draw_arcs(a);

      label.llft("$\iota_k$", straight_arc_midpoint of a1);
      label.lrt("$\iota_\beta$", straight_arc_midpoint of a2);
      label.top("$D(u)$", straight_arc_midpoint of a3);
      endfig;
    \end{mplibcode}\fi
  \end{alignedeq}
\end{definition}

\begin{definition}\label{def:categorical_colimit}\mcite[def. 5.1.19(b)]{Leinster2016Basic}
  Analogously to \hyperref[def:categorical_limit]{limits}, we define the \term{colimit} \( \varinjlim D \) of \( D \) to be a cocone
  \begin{equation*}
    \{ D(k) \reloset {\iota_k} \to L \}_{k \in \cat{I}}
  \end{equation*}
  such that for every cocone
  \begin{equation*}
    \{ D(k) \reloset {\iota_k'} \to L' \}_{k \in \cat{I}}
  \end{equation*}
  there exists exactly one morphism \( f: L \to L' \) such that \( \iota_k' = f \circ \iota_k, k \in \cat{I} \), i.e. the following diagram commutes:
  \begin{equation*}
    \text{\todo{Add diagram}}\iffalse\begin{mplibcode}
      beginfig(1);
      input metapost/graphs;

      v1 := thelabel("$D(k)$", origin);
      v2 := thelabel("$L'$", (-1, -1) scaled u);
      v3 := thelabel("$L$", (1, -1) scaled u);

      a1 := straight_arc(v1, v2);
      a2 := straight_arc(v1, v3);

      d1 := straight_arc(v3, v2);

      draw_vertices(v);
      draw_arcs(a);

      drawarrow d1 dotted;

      label.ulft("$\iota_k'$", straight_arc_midpoint of a1);
      label.urt("$\iota_k$", straight_arc_midpoint of a2);
      label.top("$f$", straight_arc_midpoint of d1);
      endfig;
    \end{mplibcode}\fi
  \end{equation*}

  If all small colimits exist, we say that \( \cat{A} \) is a \term{cocomplete category}.
\end{definition}

\begin{definition}\label{def:cocomplete_category}
  If a category is both \hyperref[def:categorical_limit]{complete} and \hyperref[def:categorical_colimit]{cocomplete}, it is said to be a \term{cocomplete category}.
\end{definition}

\begin{definition}\label{def:categorical_coproduct}\mcite[def. 5.2.2]{Leinster2016Basic}
  If the index category \( \cat{I} \) is discrete, specifying a functor \( D: \cat{I} \to \cat{A} \) is analogous to specifying a \( \cat{I} \)-indexed family \( \{ X \}_{k \in \cat{I}} \) of objects in \( \cat{A} \) (see \fullref{def:categorical_product}).

  The \term{coproduct} or \term{categorical sum}
  \begin{equation*}
    S \coloneqq \coprod_{k \in \cat{I}} X_k = \sum_{k \in \cat{I}} X_k.
  \end{equation*}
  satisfies the following universal property: for any object \( S' \) and any family of morphisms \( \{ \iota_k': {X_k} \to S' \}_{k \in \cat{I}} \) there exists exactly one morphism \( f: S \to S' \) such that for every \( k \in \cat{I} \) we have \( \iota_k' = f \circ \iota_k \), i.e. the following diagram commutes:
  \begin{equation*}
    \text{\todo{Add diagram}}\iffalse\begin{mplibcode}
      beginfig(1);
      input metapost/graphs;

      v1 := thelabel("$X_j$", origin);
      v2 := thelabel("$S'$", (-1, 1) scaled u);
      v3 := thelabel("$S$", (1, 1) scaled u);

      a1 := straight_arc(v1, v2);
      a2 := straight_arc(v1, v3);

      d1 := straight_arc(v3, v2);

      draw_vertices(v);
      draw_arcs(a);

      drawarrow d1 dotted;

      label.llft("$\iota_k'$", straight_arc_midpoint of a1);
      label.lrt("$\iota_k$", straight_arc_midpoint of a2);
      label.top("$f$", straight_arc_midpoint of d1);
      endfig;
    \end{mplibcode}\fi
  \end{equation*}

  The function \( f \) is also denoted as \( \{ f_k \}_{k \in \cat{I}} \).

  In particular, for two objects \( X, Y \in \cat{A} \), the coproduct is an object \( X + Y \) with coprojections \( \pi_X: X \to X \times Y \) and \( \pi_Y: Y \to X \times Y \) such that for each object \( S' \) and morphisms \( \iota_X': X \to S' \) and \( \iota_Y': X \to P' \) the following diagram commutes:
  \begin{equation*}
    \text{\todo{Add diagram}}\iffalse\begin{mplibcode}
      beginfig(1);
      input metapost/graphs;

      v1 := thelabel("$S'$", (-1, 1) scaled u);
      v2 := thelabel("$X + Y$", origin);
      v3 := thelabel("$X$", (0, -1) scaled u);
      v4 := thelabel("$Y$", (1, 0) scaled u);

      a1 := straight_arc(v3, v1);
      a2 := straight_arc(v4, v1);
      a3 := straight_arc(v3, v2);
      a4 := straight_arc(v4, v2);

      d1 := straight_arc(v2, v1);

      draw_vertices(v);
      draw_arcs(a);

      drawarrow d1 dotted;

      label.llft("$\iota_X'$", straight_arc_midpoint of a1);
      label.urt("$\iota_Y'$", straight_arc_midpoint of a2);
      label.rt("$\iota_X$", straight_arc_midpoint of a3);
      label.bot("$\iota_Y$", straight_arc_midpoint of a4);

      fill fullcircle scaled 0.25u shifted (center d1) withcolor white;
      label("$f$", straight_arc_midpoint of d1);
      endfig;
    \end{mplibcode}\fi
  \end{equation*}
\end{definition}

\begin{remark}\label{rem:empty_categorical_coproduct}
  The coproduct \( \prod_{k \in \varnothing} X_k \) of an empty family of objects is the initial object of the category.
\end{remark}

\begin{definition}\label{def:categorical_coequalizer}\mcite[def. 5.2.7]{Leinster2016Basic}
  As for \hyperref[def:categorical_coequalizer]{equalizers}, assume that the index category \( \cat{I} \cong \cat{I}^{-1} \) consists of two objects and two unidirectional morphisms:
  \begin{equation*}
    \text{\todo{Add diagram}}\iffalse\begin{mplibcode}
      beginfig(1);
      input metapost/graphs;

      v1 := thelabel("$\bullet$", (1, 0) scaled u);
      v2 := thelabel("$\bullet$", (2, 0) scaled u);

      a1 := straight_arc_shifted(v1, v2, (0, safe_arc_spacing));
      a2 := straight_arc_shifted(v1, v2, (0, -safe_arc_spacing));

      draw_vertices(v);
      draw_arcs(a);
      endfig;
    \end{mplibcode}\fi
  \end{equation*}

  Cocones with vertex \( A \) are then given by commutative diagrams of shape
  \begin{equation*}
    \text{\todo{Add diagram}}\iffalse\begin{mplibcode}
      beginfig(1);
      input metapost/graphs;

      v1 := thelabel("$A$", (3, 0) scaled u);
      v2 := thelabel("$X$", (1, 0) scaled u);
      v3 := thelabel("$Y$", (2, 0) scaled u);

      a1 := straight_arc(v3, v1);
      a2 := straight_arc_shifted(v2, v3, (0, safe_arc_spacing));
      a3 := straight_arc_shifted(v2, v3, (0, -safe_arc_spacing));

      draw_vertices(v);
      draw_arcs(a);

      label.top("$f$", straight_arc_midpoint of a1);
      label.top("$s$", straight_arc_midpoint of a2);
      label.bot("$t$", straight_arc_midpoint of a3);
      endfig;
    \end{mplibcode}\fi
  \end{equation*}

  The \term{coequalizer} \( (L, l) \) then satisfies the universal property: for any \( \cat{I} \)-cocone \( (L', l') \), there exists a unique morphism \( f: L \to L' \) such that the following diagram commutes:
  \begin{equation*}
    \text{\todo{Add diagram}}\iffalse\begin{mplibcode}
      beginfig(1);
      input metapost/graphs;

      v1 := thelabel("$X$", origin);
      v2 := thelabel("$Y$", (1, 0) scaled u);
      v3 := thelabel("$L'$", (2, 1) scaled u);
      v4 := thelabel("$L$", (2, -1) scaled u);

      a1 := straight_arc_shifted(v1, v2, (0, safe_arc_spacing));
      a2 := straight_arc_shifted(v1, v2, (0, -safe_arc_spacing));
      a3 := straight_arc(v2, v3);
      a4 := straight_arc(v2, v4);

      d1 := straight_arc(v4, v3);

      draw_vertices(v);
      draw_arcs(a);

      drawarrow d1 dotted;

      label.top("$s$", straight_arc_midpoint of a1);
      label.bot("$t$", straight_arc_midpoint of a2);
      label.ulft("$l'$", straight_arc_midpoint of a3);
      label.llft("$l$", straight_arc_midpoint of a4);
      label.rt("$f$", straight_arc_midpoint of d1);
      endfig;
    \end{mplibcode}\fi
  \end{equation*}
\end{definition}

\begin{definition}\label{def:categorical_pushout}\mcite[def. 5.2.11]{Leinster2016Basic}
  A \term{pushout} in \( \cat{A} \) is a \term{pullback} in \( \cat{A}^{-1} \).

  Explicitly, the index category \( \cat{I} \) has the shape
  \begin{equation*}
    \bullet \longleftarrow \bullet \longrightarrow \bullet
  \end{equation*}

  Cocones of shape \( \cat{I} \) with vertex \( A \) are then given by commutative diagrams of shape
  \begin{equation*}
    \text{\todo{Add diagram}}\iffalse\begin{mplibcode}
      beginfig(1);
      input metapost/graphs;

      v1 := thelabel("$A$", origin);
      v2 := thelabel("$X$", (-1, 0) scaled u);
      v3 := thelabel("$Y$", (0, 1) scaled u);
      v4 := thelabel("$Z$", (-1, 1) scaled u);

      a1 := straight_arc(v2, v1);
      a2 := straight_arc(v3, v1);
      a3 := straight_arc(v4, v2);
      a4 := straight_arc(v4, v3);

      draw_vertices(v);
      draw_arcs(a);

      label.bot("$\iota_X$", straight_arc_midpoint of a1);
      label.rt("$\iota_Y$", straight_arc_midpoint of a2);
      label.lft("$s$", straight_arc_midpoint of a3);
      label.top("$t$", straight_arc_midpoint of a4);
      endfig;
    \end{mplibcode}\fi
  \end{equation*}

  The pushout \( (L, \iota_X, \iota_Y) \) of \( D \) then satisfies the universal property: for any \( \cat{I} \)-cocone \( (L', \iota_X', \iota_Y') \), there exists a unique morphism \( f: L \to L' \) such that the following diagram commutes:
  \begin{equation*}
    \text{\todo{Add diagram}}\iffalse\begin{mplibcode}
      beginfig(1);
      input metapost/graphs;

      v1 := thelabel("$L$", origin);
      v2 := thelabel("$X$", (-1, 0) scaled u);
      v3 := thelabel("$Y$", (0, 1) scaled u);
      v4 := thelabel("$Z$", (-1, 1) scaled u);
      v5 := thelabel("$L'$", (1, -1) scaled u);

      a1 := straight_arc(v2, v1);
      a2 := straight_arc(v3, v1);
      a3 := straight_arc(v4, v2);
      a4 := straight_arc(v4, v3);
      a5 := straight_arc(v2, v5);
      a6 := straight_arc(v3, v5);

      draw_vertices(v);
      draw_arcs(a);

      d1 := straight_arc(v1, v5);

      draw_vertices(v);
      draw_arcs(a);

      drawarrow d1 dotted;

      label.top("$\iota_X$", straight_arc_midpoint of a1);
      label.lft("$\iota_Y$", straight_arc_midpoint of a2);
      label.lft("$s$", straight_arc_midpoint of a3);
      label.top("$t$", straight_arc_midpoint of a4);
      label.llft("$\iota_X'$", straight_arc_midpoint of a5);
      label.urt("$\iota_Y'$", straight_arc_midpoint of a6);

      fill fullcircle scaled 0.25u shifted (center d1) withcolor white;
      label("$f$", straight_arc_midpoint of d1);
      endfig;
    \end{mplibcode}\fi
  \end{equation*}
\end{definition}

\begin{definition}\label{def:categorical_limit_preservation}\mcite[def. 5.3.1, 5.3.5]{Leinster2016Basic}
  Let \( F: \cat{A} \to \boldop B \) be a functor. We say that
  \begin{thmenum}
    \thmitem{def:categorical_limit_preservation/preserve} \( F \) \term{preserves} limits of shape \( \cat{I} \) for some index category \( \cat{I} \) if, given a \( \cat{I} \)-shaped limit cone
    \begin{equation*}
      \{ L \reloset {\pi_k} \to D(k) \}_{k \in \cat{I}},
    \end{equation*}
    its image
    \begin{equation*}
      \{ F(L) \reloset {F(\pi_k)} \to F(D(k)) \}_{k \in \cat{I}}
    \end{equation*}
    is also a limit cone. We say that \( F \) simply preserves limits if it preserves limits for every index category \( \cat{I} \).

    \thmitem{def:categorical_limit_preservation/reflect} \( F \) \term{reflects} limits of shape \( \cat{I} \) if, given any \( \cat{I} \)-shaped cone, if its image is a limit cone, then is it itself a limit cone.

    \thmitem{def:categorical_limit_preservation/create} \( F \) \term{creates} limits of shape \( \cat{I} \) if it both preserves and reflects limits.

    \thmitem{def:categorical_limit_preservation/lift} \( F \) \term{lifts} limits of shape \( \cat{I} \) if, given a diagram \( D: \cat{I} \to \boldop B \), any limit cone \( \varprojlim D \) is the image of some limit cone in \( A \).
  \end{thmenum}
\end{definition}

\begin{remark}\label{rem:categorical_colimit_preservation}
  Analogous definitions can be given for colimits.
\end{remark}
