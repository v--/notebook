\section{Category theory}\label{sec:category_theory}

Category theory studies objects via how they relate to other objects. It shifts the focus from how individual members behave and even has no concept of membership, upon which set theory is based.

This shift is evident from the following diagram, which is actually half of the proof of \fullref{thm:functor_adjoint_uniqueness}:
\begin{equation*}
  \begin{aligned}
    \includegraphics[page=1]{output/thm__functor_adjoint_uniqueness.pdf}
  \end{aligned}
\end{equation*}

We do still have individual objects, precisely the nodes of the diagram above, however we are only interested in how the nodes are related to each other. Chasing the relations in this diagram individually would require a lot more effort with little gain.

Categories can be defined \enquote{from the ground up} so that they may be used without an underlying set theory or logic. For our purposes, it will be more appropriate to define categories via \hyperref[def:quiver]{quivers}, a.k.a. directed multigraphs. This latter approach will be much more convenient for us, since we are working in \hyperref[def:axiom_of_universes]{\logic{ZFC+U}} and are only interested in categories insomuch as they are helpful to us.

Furthermore, categories are actually the primary motivation for us include the \hyperref[def:axiom_of_universes]{axiom of universes} in our metatheory that would otherwise include only the axioms of \hyperref[def:zfc]{\logic{ZFC}}. This is discussed further in \fullref{rem:functor_size} and \fullref{rem:functor_category_size}.
