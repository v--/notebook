\subsection{Totally ordered sets}\label{subsec:totally_ordered_sets}

\begin{definition}\label{def:totally_ordered_set}
  We say that a partially ordered set is \term{totally ordered} if either the nonstrict order \( \leq \) is \hyperref[def:binary_relation/total]{total} or if the strict order \( < \) is \hyperref[def:binary_relation/trichotomic]{trichotomic}.

  The theory, homomorphisms and category are obtained analogously to \fullref{def:partially_ordered_set}, but with either of these additional axiom sets.
\end{definition}
\begin{proof}
  Equivalence between nonstrict and strict total orders follows directly from the compatibility condition \eqref{def:partially_ordered_set/compatibility_nonstrict}.
\end{proof}

\begin{proposition}\label{thm:total_order_embedding_iff_strict}
  Let \( (\mscrP, \leq_\mscrP) \) and \( (\mscrQ, \leq_\mscrQ) \) be \hyperref[def:totally_ordered_set]{totally ordered sets} (more generally, \( (\mscrQ, \leq_\mscrQ) \) can be any \hyperref[def:preordered_set]{preordered set}).

  An \hyperref[def:partially_ordered_set/homomorphism]{order homomorphism} from \( \mscrP \) to \( \mscrQ \) is \hyperref[def:function_invertibility/injective]{injective} if and only if it is strict.
\end{proposition}
\begin{proof}
  \SufficiencySubProof Follows from \fullref{thm:order_embedding_is_strict}.

  \NecessitySubProof Let \( f: \mscrP \to \mscrQ \) be a strict order homomorphism and suppose that \( f(x) = f(y) \) for, some \( x \) and \( y \) in \( \mscrP \). We will use the \hyperref[def:binary_relation/trichotomic]{trichotomy} of \( <_\mscrP \).
  \begin{itemize}
    \item If \( x <_\mscrP y \), then \( f(x) <_\mscrQ f(y) \) since \( f \) is strictly monotone, which contradicts \( f(x) = f(y) \).

    \item If \( x >_\mscrP y \), similarly \( f(x) >_\mscrQ f(y) \) and we again obtain a contradiction.

    \item It remains for \( x \) to be equal to \( y \).
  \end{itemize}

  Since \( x \) and \( y \) were arbitrary, we conclude that \( f \) is injective.
\end{proof}

\begin{proposition}\label{thm:totally_ordered_strong_homomorphism}
  For totally ordered sets, \hyperref[thm:partially_ordered_set/homomorphism]{strict order homomorphisms} are precisely the \hyperref[rem:first_order_strong_homomorphism]{strong order homomorphisms}.

  Let \( (\mscrP, \leq_\mscrP) \) and \( (\mscrQ, \leq_\mscrQ) \) be \hyperref[def:totally_ordered_set]{totally ordered sets} (more generally, \( (\mscrQ, \leq_\mscrQ) \) can be any \hyperref[def:preordered_set]{preordered set}) and let \( f: \mscrP \to \mscrQ \) be an order homomorphism between them. Then \( f \) is a strict order homomorphism if and only if it is a strong order homomorphism.
\end{proposition}
\begin{proof}
  \SufficiencySubProof Suppose that \( f \) is a strict order homomorphism. We must show that \( f(x) \leq f(y) \) entails \( x \leq y \).

  If we suppose that \( x > y \) for, some \( x \) and \( y \) in \( \mscrP \), then \( f(x) > f(y) \), which contradicts our assumption \( f(x) \leq f(y) \).

  Therefore, \( f \) is a strong order homomorphism.

  \NecessitySubProof Suppose that \( f \) is a strong order homomorphism. Let \( x < y \). It follows that \( f(x) \leq f(y) \).

  Suppose that \( f(x) = f(y) \). Then both \( f(x) \leq f(y) \) and \( f(x) \geq f(y) \) hold, which in turn imply that both \( x \leq y \) and \( x \geq y \) hold since \( f \) is a strong homomorphism. Thus, we obtain \( x = y \), which contradicts our assumption that \( x < y \).

  Therefore, \( f \) is a strict order homomorphism.
\end{proof}

\begin{corollary}\label{thm:totally_ordered_strict_isomorphisms}
  A strict order embedding between totally ordered sets is an isomorphism if and only if it is bijective.
\end{corollary}
\begin{proof}
  Follows from \fullref{thm:totally_ordered_strong_homomorphism} and \fullref{thm:automorphism_without_predicate_symbols}.
\end{proof}

\begin{proposition}\label{thm:totally_ordered_minimal_element_is_minimum}
  If \( (\mscrP, \leq) \) is a \hyperref[def:totally_ordered_set]{totally ordered set} and \( A \subseteq \mscrP \) is nonempty, then any \hyperref[def:partially_ordered_set_extremal_points/maximal_and_minimal_element]{minimal element} is a \hyperref[def:partially_ordered_set_extremal_points/maximum_and_minimum]{minimum}.
\end{proposition}
\begin{proof}
  Let \( x_0 \) be a minimal element of \( A \). If \( x_0 \) is the only element of \( A \), it is clearly the minimum of \( A \). Suppose that \( A \) is not a singleton set.

  By definition of total order, for any \( x \in A \) either \( x \leq x_0 \) or \( x_0 \leq x \). If \( x \leq x_0 \), then since \( x_0 \) is a minimal element, we have \( x = x_0 \).

  Therefore, for any \( x \in A \), either \( x = x_0 \) or \( x > x_0 \) (i.e. \( x \leq x_0 \)), proving that \( x_0 \) is a minimum of \( A \).
\end{proof}

\begin{proposition}\label{thm:totally_ordered_segment_isomorphism}
  Let \( (\mscrP, \leq) \) be a \hyperref[def:totally_ordered_set]{totally-ordered set}. Let \( \mscrQ \) be the set containing the \hyperref[def:partially_ordered_set_interval/ray]{strict initial segment} \( \mscrP_{<x} \) for every member \( x \) of \( \mscrP \).

  Then \( (\mscrP, \leq) \) is \hyperref[def:partially_ordered_set/homomorphism]{strictly order-isomorphic} to \( (\mscrQ, \subseteq) \).
\end{proposition}
\begin{proof}
  Explicitly define the isomorphism
  \begin{equation*}
    \begin{aligned}
      &f: \mscrP \to \mscrQ \\
      &f(x) \coloneqq \mscrP_{<x} = \set{ y \in \mscrP \given y < x }.
    \end{aligned}
  \end{equation*}

  We will first show that \( f \) is \hyperref[def:partially_ordered_set/homomorphism]{strictly monotone}. If \( x < y \), then \( x \in \mscrP_{<y} \). But \( x \not\in \mscrP_{<x} \), hence \( \mscrP_{<x} \) is a strict subset of \( \mscrP_{<y} \). Thus, \( f \) is strictly monotone.

  The function \( f \) is injective by \fullref{thm:total_order_embedding_iff_strict} and surjective by definition. Thus, \( f \) is a strict isomorphism between \( (\mscrP, \leq) \) and \( (\mscrQ, \subseteq) \).
\end{proof}

\begin{proposition}\label{thm:totally_ordered_cofinal_equivalences}
  Let \( (\mscrP, \leq) \) be an \hyperref[def:partially_ordered_set_extremal_points/upper_and_lower_bounds]{unbounded from above} totally ordered set and let \( A \subseteq \mscrP \). Then \( A \) is \hyperref[def:cofinal_set]{cofinal} if and only if it is itself unbounded from above.

  This equivalence is useful for \hyperref[def:regular_cardinal]{regular cardinals} --- for example \fullref{thm:cardinal_cofinality}.

  Compare this result with \fullref{thm:partially_ordered_cofinal_equivalences}.
\end{proposition}
\begin{proof}
  \SufficiencySubProof Let \( A \) be a cofinal set. Suppose that it is bounded from above. Then there must exist, some \( x \in \mscrP \) such that \( x \) is an upper bound of \( A \).
  \begin{itemize}
    \item If \( x \not\in A \), this contradicts the cofinality of \( A \).
    \item If \( x \in A \), then it is a maximum. But since \( \mscrP \) is itself unbounded from above, we can find, some \( y \in \mscrP \) such that \( y \) is again an upper bound of \( A \) and \( x < y \). But this contradicts the cofinality of \( A \).
  \end{itemize}

  \NecessitySubProof Let \( A \) be unbounded from above. Suppose that is is not cofinal. Since there exists no upper bound of \( A \), for every \( x\in \mscrP \) there exists, some \( y \in A \) such that \( x \leq y \). Therefore, \( A \) is cofinal.
\end{proof}

\begin{proposition}\label{thm:total_lexicographic_order_is_total_order}
  If \( (\mscrP, \leq_\mscrP) \) and \( (\mscrQ, \leq_\mscrQ) \) are \hyperref[def:totally_ordered_set]{totally ordered sets}, then the \hyperref[eq:def:lexicographic_order]{lexicographic} and \hyperref[eq:def:lexicographic_order/reverse]{reverse lexicographic} orders on \( \mscrP \times \mscrQ \) are \hyperref[def:totally_ordered_set]{strict total order} relations.

  Compare this result to \fullref{thm:lexicographic_order_is_partial_order} and \fullref{thm:well_ordered_lexicographic_order_is_well_ordered}.
\end{proposition}
\begin{proof}
  We have already shown in \fullref{thm:lexicographic_order_is_partial_order} and these are partial orders. It only remains to check trichotomy.

  \SubProofOf[def:binary_relation/trichotomic]{trichotomy} Let \( \prec \) be the lexicographic order on \( \mscrP \times \mscrQ \). Let \( (a, b) \) and \( (c, d) \) be pairs in \( \mscrP \times \mscrQ \). Since \( <_\mscrP \) and \( <_\mscrQ \) are strict total orders, we only have the following possibilities:
  \begin{itemize}
    \item If \( a = c \) and \( b = d \), then \( (a, b) = (c, d) \).
    \item If \( a = c \) and \( b <_\mscrQ d \), then \( (a, b) \prec (c, d) \).
    \item If \( a = c \) and \( b >_\mscrQ d \), then \( (a, b) \succ (c, d) \).
    \item If \( a <_\mscrP c \), then \( (a, b) \prec (c, d) \).
    \item If \( a >_\mscrP c \), then \( (a, b) \succ (c, d) \).
  \end{itemize}

  The proof for the reverse lexicographic order is analogous.
\end{proof}

\begin{definition}\label{def:order_topology}
  Let \( \mscrP \) be a \hyperref[def:partially_ordered_set]{totally ordered set} with more than one element. The \term{order topology} induced by \( \leq \) is the topology generated by the \hyperref[def:topological_subbase]{subbase} of open \hyperref[def:partially_ordered_set_interval/ray]{rays}
  \begin{equation*}
    \mscrS \coloneqq \set[\Big]{ (a, \infty) \given a \in \mscrP } \cup \set[\Big]{ (-\infty, b) \given b \in \mscrP }.
  \end{equation*}

  The \hyperref[def:topological_base]{base} corresponding to this subbase is
  \begin{equation*}
    \mscrB = \mscrS \cup \set[\Big]{ \varnothing } \cup \set[\Big]{ (a, b) \given a, b \in \mscrP \T{and} a < b }.
  \end{equation*}

  See the proof of \hyperref[thm:topological_base_axioms/B1]{B1} for why \( \mscrP \) must have more than one element.
\end{definition}
\begin{proof}
  \SubProof{Proof of compatibility of \( \mscrS \) and \( \mscrB \)} Define
  \begin{equation*}
    \mscrC = \set*{ \bigcap S \given* S \text{ is a nonempty finite subset of } \mscrS }.
  \end{equation*}

  We will show that \( \mscrB = \mscrC \).

  Let \( B \in \mscrB \). The cases \( B \in \mscrS \) and \( B = \varnothing \) are trivial. Suppose that \( B \not\in \mscrS \). Then there exist points \( a < b \) such that
  \begin{equation*}
    B = (a, b) = (-\infty, b) \cap (a, \infty).
  \end{equation*}

  This is an intersection of members of \( \mscrS \), hence \( B \in \mscrC \). Therefore, \( \mscrB \subseteq \mscrC \).

  Now let let \( C = S_1 \cap \cdots \cap S_n \), where \( S_1, \ldots, S_n \) are members of \( \mscrS \). We will show by induction on \( n > 0 \) that \( C \in \mscrB \). The case \( n = 1 \) is trivial. Suppose that all \( n \)-ary intersections belong to \( \mscrB \) and let
  \begin{equation*}
    C = S_1 \cap \cdots \cap S_n \cap S_{n+1}.
  \end{equation*}

  By the inductive hypothesis we have that \( D \coloneqq S_1 \cap \cdots \cap S_n \) belongs to \( \mscrB \) and thus we have three cases:
  \begin{itemize}
    \item If either \( D = (a, \infty) \) or \( D = (-\infty, b) \), then \( D \in \mscrS \).
    \item If \( D = \varnothing \), then \( D = (-\infty, a) \cap (a, \infty) \) for, some \( a \in \mscrP \).
    \item If \( D = (a, b) \), then \( D = (-\infty, b) \cap (a, \infty) \).
  \end{itemize}

  In all cases both cases \( D \) and \( C = D \cap S_{n+1} \) are finite intersection of members of \( \mscrS \). Therefore, \( \mscrC \subseteq \mscrB \). Since we already have the inclusion in the other direction, we conclude that \( \mscrC = \mscrB \).

  \SubProof{Proof that \( \mscrB \) is a base} We will show that the axioms in \fullref{thm:topological_base_axioms/B1} hold.

  \SubProofOf*[thm:topological_base_axioms/B1]{B1} Let \( x \in \mscrP \).

  If \( x \) is a \hyperref[def:partially_ordered_set_extremal_points/maximum_and_minimum]{maximum}, then take any other value \( y < x \) and the set \( (y, \infty) \) will contain \( x \). We use here that there is more than one element in \( \mscrP \).

  If \( x \) is not a maximum, then \( x \) belongs to any interval \( (-\infty, y) \) whenever \( y > x \).

  In both cases there exists an interval in \( \mscrS \) containing \( x \). Thus, \( \bigcup \mscrS = \mscrP \).

  \SubProofOf*[thm:topological_base_axioms/B2]{B2} Let \( U \) and \( V \) be members of \( \mscrB \). We consider \( 14 \) cases:
  \begin{itemize}
    \item If either \( U = \varnothing \) or \( V = \varnothing \), then \( U \cap V = \varnothing \).
    \item If \( U = (-\infty, u) \) and \( V = (v, \infty) \), then
    \begin{itemize}
      \item If \( u \leq v \), then \( U \cap V = \varnothing \).
      \item If \( v < u \), then \( U \cap V = (v, u) \).
    \end{itemize}

    \item If \( U = (-\infty, u) \) and \( V = (v_1, v_2) \), then
    \begin{itemize}
      \item If \( u \leq v_1 \), then \( U \cap V = \varnothing \).
      \item If \( v_1 < v_2 \leq u \), then \( U \cap V = V =  (v_1, v_2) \).
      \item If \( v_1 \leq u < v_2 \), then \( U \cap V = (v_1, u) \).
    \end{itemize}

    \item If \( U = (u_1, u_2) \) and \( V = (v, \infty) \), then
    \begin{itemize}
      \item If \( u_2 \leq v \), then \( U \cap V = \varnothing \).
      \item If \( u_1 \leq v < u_2 \), then \( U \cap V = (v, u_2) \).
      \item If \( v \leq u_1 < u_2 \), then \( U \cap V = U = (u_1, v_1) \).
    \end{itemize}

    \item If \( U = (u_1, u_2) \) and \( V = (v_1, v_2) \), then
    \begin{itemize}
      \item If \( u_2 < v_1 \), then \( U \cap V = \varnothing \).
      \item If \( u_1 < v_1 < u_2 < v_2 \) then \( U \cap V = (v_1, u_2) \).
      \item If \( u_1 < v_1 < v_2 < u_2 \) then \( U \cap V = V = (v_1, v_2) \).
      \item If \( v_1 < u_1 < u_2 < v_2 \) then \( U \cap V = U = (u_1, u_2) \).
      \item If \( v_1 < u_1 < v_2 < u_2 \) then \( U \cap V = (u_1, v_2) \).
    \end{itemize}
  \end{itemize}

  In all cases, the intersection \( U \cap V \) belongs to \( \mscrB \).
\end{proof}

\begin{example}\label{ex:def:order_topology}
  Examples of \hyperref[def:order topology]{order topologies} include:
  \begin{itemize}
    \item The order topology on \( \mscrR \), which is equivalent to the \hyperref[def:metric_topology]{metric topology} as shown in \fullref{thm:real_metric_and_order_topologies_coincide}.

    \item All \hyperref[def:ordinal]{ordinals} greater than one induce topological spaces called the \hyperref[def:ordinal_space]{ordinal spaces}.
  \end{itemize}
\end{example}

\begin{definition}\label{def:ordinal_space}
  Let \( \alpha \) be an \hyperref[def:ordinal]{ordinal}. When regarded as the set of smaller ordinals (see \fullref{thm:ordinal_is_set_of_smaller_ordinals}), \( \alpha \) is a \hyperref[def:totally_ordered_set]{totally order set} and hence we can endow it with the \hyperref[def:order_topology]{order topology} \( \mscrT \) to obtain a \hyperref[def:topological_space]{topological space}. We call the space \( (\alpha, \mscrT) \) an \term{ordinal space}.
\end{definition}

\begin{proposition}\label{thm:limit_ordinal_order_topology}
  Fix an \hyperref[def:ordinal_space]{ordinal space} \( (\alpha, \mscrT) \).

  An nonzero ordinal \( \beta \in \alpha \) is a \hyperref[def:limit_ordinal]{limit ordinal} if and only if it is the \hyperref[def:net_convergence/limit]{limit point} of, some net of ordinals in the space \( \alpha \).
\end{proposition}
\begin{proof}
  \SufficiencySubProof Let \( \beta \) be a limit ordinal. When regarded as a subset of \( \alpha \), it is itself a topological net because it is totally ordered. We will show that \( \beta \) as a member of \( \alpha \) is a limit point as a subset of \( \alpha \).

  By \fullref{thm:net_convergence_via_subbases} it is enough to show that \( \beta \) as a net is eventually in every set of the local subbase at \( \beta \) of the order topology. This local subbase consists of all initial and final segments of \( \alpha \) that contain \( \beta \).

  With the following we exhaust the local subbase at \( \beta \): for all \( \gamma \in \alpha \) that are distinct from \( \beta \):
  \begin{itemize}
    \item If \( \gamma > \beta \), then \( \gamma \) itself is a neighborhood of \( \beta \) and a member of the subbase as an initial segment of \( \alpha \). Since the entire net \( \beta \) is contained in \( \gamma \), it is eventually in the initial segment \( \alpha_{>\gamma} \).

    \item If \( \gamma < \beta \), then the final segment \( \alpha_{>\gamma} \) is a member of the local subbase of \( \beta \). Let \( \delta \) be, some member of the net \( \beta \).
    \begin{itemize}
      \item If \( \delta > \gamma \), it is itself an ordinal such that \( \varepsilon > \delta \) implies \( \varepsilon \in \alpha_{>\gamma} \).
      \item If \( \delta \leq \gamma \), then \( \varepsilon > \op{succ}(\gamma) \) implies \( \varepsilon \in \alpha_{>\gamma} \). The successor of \( \gamma \) belongs to \( \beta \) because \( \beta \) is a limit ordinal and satisfies \fullref{def:successor_and_limit_ordinal/smaller_successor}.
    \end{itemize}

    Thus, again the net \( \beta \) is eventually in the final segment \( \alpha_{>\gamma} \).
  \end{itemize}

   We have shown that \( \beta \) as a net is eventually in, some every set in the local subbase of \( \beta \), thus it is a limit of the net.

   \NecessitySubProof Let \( \beta \) be a limit of the net \( \seq{ \gamma_k }_{k \in \mscrK} \subseteq \alpha \).

   Let \( \delta \in \beta \). Consider the neighborhood \( (\delta, \beta) \) of \( \beta \). Since \( \beta \) is a limit point, there must exist, some index \( k_0 \in \mscrK \) such that \( \gamma_k \in \alpha_{>\delta} \) for every \( k \geq k_0 \). Since \( \delta \) was an arbitrary member of \( \beta \), we conclude that \( \beta \) it satisfies \fullref{def:successor_and_limit_ordinal/smaller_successor} and is thus a limit ordinal.
\end{proof}
