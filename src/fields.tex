\subsection{Fields}\label{subsec:fields}

\begin{definition}\label{def:division_ring}\mcite[144]{Knapp2016BasicAlgebra}
  If every nonzero element of a ring is \hyperref[def:divisibility/unit]{invertible}, we call it a \term{division ring}.
\end{definition}

\begin{proposition}\label{thm:division_ring_is_entire}
  A nontrivial \hyperref[def:division_ring]{division ring} is \hyperref[def:entire_semiring]{entire}.
\end{proposition}
\begin{proof}
  Let \( xy = 0 \). If \( x \) is nonzero, multiplying both sides by \( x^{-1} \), we obtain \( y = 0 \). Analogously, \( y \neq 0 \) implies that \( x = 0 \). In all cases, either \( x \) or \( y \) is necessarily zero.

  Therefore, the ring has no nontrivial zero divisors.
\end{proof}

\begin{definition}\label{def:field}
  We will call the \hyperref[def:ring/trivial]{nontrivial} \hyperref[def:ring]{ring} \( \BbbK \) a \term{field} if any of the following equivalent conditions hold:
  \begin{thmenum}
    \thmitem{def:field/simple} \( \BbbK \) is \hyperref[def:ring/commutative]{commutative} and \hyperref[def:ring/simple]{simple}.
    \thmitem{def:field/division_ring} \( \BbbK \) is a \hyperref[def:ring/commutative]{commutative} \hyperref[def:division_ring]{division ring}.
  \end{thmenum}

  Fields have the following metamathematical properties:
  \begin{thmenum}
    \thmitem{def:field/theory} We can construct a \hyperref[def:first_order_theory]{first-order theory} for fields by adding to the \hyperref[def:semiring/theory]{theory of rings} the axioms \( \neg (0 \neq 1) \) and
    \begin{equation}\label{eq:def:field/theory/invertibility}
      \neg (\xi \doteq 0) \implies \qexists \eta (\xi \cdot \eta \doteq 1).
    \end{equation}

    These axioms are not \hyperref[def:positive_formula]{positive formulas}, hence fields automatically get worse metamathematical properties than rings, for example.

    \thmitem{def:field/homomorphism}\mcite[453]{Knapp2016BasicAlgebra} A \hyperref[def:first_order_homomorphism]{first-order homomorphism} between fields is simply a \hyperref[def:ring/homomorphism]{unital ring homomorphism}.

    \thmitem{def:field/submodel} If for two fields \( \Bbbk \) and \( \BbbK \) are have \( \Bbbk \subseteq \BbbK \), we say that \( \BbbK \) is a \term{field extension} of \( \Bbbk \) and that \( \Bbbk \) is a \term{subfield} of \( \BbbK \). In particular, if \( \BbbK = \Bbbk \), we say that the extension is trivial.

    \thmitem{def:field/category} The category of \hyperref[def:large_and_small_sets]{\( \mscrU \)-small} fields \( \ucat{Field} \) is a full subcategory of \hyperref[def:ring/category]{\( \ucat{CRing} \)} with objects restricted to fields.
  \end{thmenum}
\end{definition}
\begin{defproof}
  \ImplicationSubProof{def:field/simple}{def:field/division_ring} Suppose that \( \BbbK \) is a commutative simple ring.

  Let \( x \) be a nonzero element. Since there are no nontrivial proper ideals, the principal ideal \( \braket{ x } \) can only be \( \BbbK \). Hence, \( 1_\BbbK \in \braket{ x } \), which implies the existence of an element \( y \) such that \( xy = 1_\BbbK \).

  Therefore, \( y \) is a multiplicative inverse of \( x \). Generalizing on \( x \), we can conclude that \( \BbbK \) is a division ring.

  \ImplicationSubProof{def:field/division_ring}{def:field/simple} Suppose that \( \BbbK \) is a commutative division ring.

  Let \( I \) be a nontrivial ideal of \( \BbbK \). For any nonzero \( x \in I \), we have \( x^{-1} x \in I \) since \( I \) is closed under multiplication with elements of \( \BbbK \). Therefore, \( I \) contains the multiplicative identity \( 1_\BbbK \), and, by \fullref{thm:def:semiring_ideal/properties/proper_ideals_containing_identity}, \( \BbbK = I \).
\end{defproof}

\begin{proposition}\label{thm:field_is_euclidean_domain}
  Every \hyperref[def:field]{field} is an \hyperref[def:euclidean_domain]{Euclidean domain}.
\end{proposition}
\begin{proof}
  By \fullref{thm:division_ring_is_entire}, a field is necessarily an integral domain. Since every element of \( \BbbK \) is divisible (without remainder), the Euclidean function can be arbitrary; for definiteness, we take it to be canonically zero.
\end{proof}

\begin{proposition}\label{thm:field_of_fractions}
  Let \( D \) be an \hyperref[def:integral_domain]{integral domain}. The \hyperref[def:ring_localization]{localization} of \( D \) at the zero ideal \( \set{ 0_R } \) is a \hyperref[def:field]{field}, which we call the \term{field of fractions} of \( D \).
\end{proposition}
\begin{proof}
  By \fullref{thm:def:ring_localization/properties/prime_ideal}, the localization by the prime ideal \( \set{ 0_R } \) has  whose only maximal ideal is \( S^{-1} \set{ 0_R } \). Since \( 0_R \) is absorbing, \( S^{-1} \set{ 0_R } \) is again the zero ideal.  Therefore, it is the only proper ideal of the localization \( S^{-1} D \), and hence the localization is a \hyperref[def:ring/simple]{simple ring}.

  Since \( D \) is an integral domain, by \fullref{thm:def:ring_localization/properties/injective_inclusion}, \( S^{-1} D \) is a superring of \( D \). It is therefore a nontrivial commutative simple ring, and thus it satisfies \fullref{def:field/simple}.
\end{proof}

\begin{proposition}\label{thm:field_of_fractions_universal_property}
  The \hyperref[def:field_of_fractions]{field of fractions} \( \BbbK \) of the integral domain \( D \) satisfies the following \hyperref[rem:universal_mapping_property]{universal mapping property}:
  \begin{displayquote}
    For every field \( \BbbL \) and every ring homomorphism \( \varphi: D \to \BbbL \), \( \varphi \) \hyperref[def:factors_through]{uniquely factors through} \( \BbbK \). That is, there exists a unique field homomorphism \( \widetilde{\varphi}: \BbbK \to \BbbL \) such that the following diagram commutes:
    \begin{equation}\label{eq:thm:field_of_fractions_universal_property/diagram}
      \begin{aligned}
        \includegraphics[page=1]{output/thm__field_of_fractions_universal_property.pdf}
      \end{aligned}
    \end{equation}
  \end{displayquote}
\end{proposition}
\begin{proof}
  This is simply a special case of \fullref{thm:ring_localization_universal_property}.
\end{proof}

\begin{definition}\label{def:rational_function_field}
  The \term{field of rational algebraic functions} \( D(\mscrX) \) for the set of indeterminates \( \mscrX \) over the \hyperref[def:integral_domain]{integral domain} \( D \) is the \hyperref[thm:field_of_fractions]{field of fractions} of the \hyperref[def:polynomial_semiring]{polynomial ring} \( D[\mscrX] \).

  Despite the name, elements of the field of fractions are not actually functions, but merely formal expressions.
\end{definition}

\begin{proposition}\label{thm:adjoining_elements_to_field}
  Let \( \Bbbk \subseteq \BbbK \) be \hyperref[def:field]{fields} and let \( A \) be an arbitrary subset of \( \BbbK \).

  Let \( \Bbbk[A] \) be the ring obtained by adjoining the elements of \( A \) to \( \Bbbk \) as described in \fullref{thm:adjoining_elements_to_semiring}. The \hyperref[def:field_of_fractions]{field of fractions} of \( \Bbbk[A] \) is the smallest field extension of \( \Bbbk \) containing \( A \).

  We denote this extension by \( \Bbbk(A) \).
\end{proposition}
\begin{proof}
  It follows from \fullref{thm:adjoining_elements_to_semiring} that \( \Bbbk[A] \) is the smallest superring of \( \Bbbk \) containing \( A \). By \fullref{thm:field_of_fractions_universal_property}, \( \Bbbk(A) \) is the smallest field containing \( \Bbbk[A] \).
\end{proof}

\begin{definition}\label{def:splitting_field}
  A \term{splitting field} for a nonconstant polynomial \( f(X) \in \Bbbk[X] \) of degree \( n \) is a \hyperref[def:field_extension]{field extension} \( \BbbK \) of \( \Bbbk \) such that \( a_1, \ldots, a_n \) are roots of \( f(X) \) in \( \BbbK \) and, furthermore, \( \BbbK \) is the \hyperref[thm:field_of_fractions]{field of fractions}
  \begin{equation*}
    \BbbK \cong \Bbbk(a_1, \ldots, a_n).
  \end{equation*}

  This is the smallest field extension of \( \Bbbk \) in which the polynomial \( f(X) \) can split into linear factors. By \fullref{thm:splitting_field_existence}, splitting fields exist and are unique up to an isomorphism.
\end{definition}

\begin{proposition}\label{thm:splitting_field_existence}
  There exists a unique up to a (possibly nonunique) isomorphism \hyperref[def:splitting_field]{splitting field} for every nonconstant polynomial in one indeterminate over a field.
\end{proposition}
\begin{proof}
  \SubProof{Proof of existence}\mcite[thm 9.10 \\ thm. 9.12]{Knapp2016BasicAlgebra} We use induction on the degree of the polynomial \( f(X) = \sum_{k=0}^n a_k x^k \) over \( \Bbbk \). In the base case \( n = 1 \), \( f(X) \) is already linear, and hence \( \Bbbk \) is itself a splitting field for \( f(X) \).

  Suppose that there exist splitting fields for polynomials over \( \Bbbk \) of degree \( n - 1 \). By \fullref{thm:maximal_ideal_theorem}, the \hyperref[def:semiring_ideal]{principal ideal} \( \braket{ f(X) } \) is contained in some maximal ideal \( M \). By \fullref{thm:quotient_by_maximal_ideal}, the quotient of \( R[X] \) by \( M \) is a field.

  Define \( u_n \coloneqq \braket{ X } + M \). We have \( f(u_n) = \braket{ f(X) } + M \), hence \( u_n \) is a root of \( f \) in \( M \). By \fullref{thm:polynomial_root_iff_divisible},
  \begin{equation*}
    f(X) = (X - u_n) q(X)
  \end{equation*}
  for some polynomials \( q(X) \) and \( r(X) \), both of degree less than \( n \) (or \( r(X) = 0 \)).

  We can now apply the inductive hypothesis to obtain a splitting field of \( q(X) \). Let \( u_1, \ldots, u_{n-1} \) be the roots of \( q(X) \) in this field. We can then adjoin \( u_1, \ldots, u_n \) to the field \( \Bbbk \) to obtain a splitting field \( \Bbbk(u_1, \ldots, u_n) \) of \( f(X) \). Denote this field by \( \BbbK \).

  \SubProof{Proof of uniqueness} Suppose that, given our previous construction, \( \BbbL \) is also a splitting field for \( f(X) \).

  Again, we use induction on the degree \( n \) of \( f(X) \). The case \( n = 1 \) is again obvious.

  Suppose that any two splitting fields for polynomials over \( \Bbbk \) of degree \( n - 1 \) are isomorphic. Let \( b_n \) be a root of \( f(X) \) in \( \BbbL \) and let
  \begin{equation*}
    f(X) = (X - b_n) r(X).
  \end{equation*}

  Let \( b_1, \ldots, b_{n-1} \) be the roots of \( r(X) \). Let \( \varphi \) be an isomorphism between the subfield \( \Bbbk(a_1, \ldots, a_{n-1}) \) of \( \BbbK \) and the corresponding subfield \( \Bbbk(b_1, \ldots, b_{n-1}) \) of \( \BbbL \). It follows that
  \begin{equation*}
    \underbrace{\prod_{k=1}^{n-1} (X - b_k)}_{r(X)} = \underbrace{\prod_{k=1}^{n-1} (X - \varphi(a_k))}_{q^\varphi(X)}.
  \end{equation*}

  Therefore, we can extend \( \varphi \) to an isomorphism \( \widehat{\varphi}: \BbbK \to \BbbL \) by putting \( \widehat{\varphi}(a_n) \coloneqq b_n \).
\end{proof}

\begin{theorem}[Classification of finite fields]\label{thm:finite_fields}
  \hfill
  \begin{thmenum}
    \thmitem{thm:finite_fields/characteristic} The \hyperref[def:ring_characteristic]{characteristic} of a \hyperref[def:field]{field} with \( q \) elements is a \hyperref[def:prime_number]{prime number} \( p \), and \( q \) is a power of \( p \).

    The fields of prime cardinality are sometimes called \term{prime fields}.

    \thmitem{thm:finite_fields/splitting} Two \hyperref[def:field]{fields} with \( q \) elements are \hyperref[def:field/homomorphism]{isomorphic} as \hyperref[def:splitting_field]{splitting fields} for the polynomial
    \begin{equation*}
      X^q - X \in \BbbZ_p[X].
    \end{equation*}

    Utilizing the general conventions of identifying isomorphic objects in algebra, we denote by \( \BbbF_q \) \enquote{the} finite field with \( q \) elements. Finite fields are also called \term{Galois fields}.
  \end{thmenum}
\end{theorem}
\begin{proof}
  \SubProofOf{thm:finite_fields/characteristic} Let \( \BbbK \) be a field with \( q \) elements and let \( p \) be the \hyperref[def:ring_characteristic]{characteristic} of \( \BbbK \). Then \( \BbbZ_p \) is a subring of \( \BbbK \), and it is hence again a field.

  By \fullref{thm:quotient_by_maximal_ideal}, \( \braket{ p } \) is a maximal ideal in \( \BbbZ \), and, by \fullref{thm:def:semiring_ideal/properties/maximal_is_prime}, \( p \) is a prime number.

  By \fullref{thm:lagranges_theorem_for_groups}, \( p \) divides \( q \). But \( \BbbK / \BbbZ_p \) is again a field by \fullref{thm:quotient_ideal_lattice_theorem}, and again has prime characteristic. Continuing by induction, we eventually obtain a sequence \( p_1, \ldots, p_n \) of prime numbers such that \( q = p_1 \cdots p_n \).

  By \fullref{thm:lagranges_theorem_for_groups}, \( q \) cannot contain subgroups of prime cardinalities \( p_1 \) and \( p_2 \) unless \( p_1 = p_2 \). Hence, again by induction, we conclude that \( p_1 = \cdots = p_n \).

  Therefore, \( q = p^n \).

  \SubProofOf{thm:finite_fields/splitting} Let \( \BbbK \) be a field with \( q \) elements with characteristic \( p \). We will show that every element of \( \BbbK \) is a root of \( X^q - X \in \BbbZ_p[X] \).

  The multiplicative group of \( \BbbK \) has order \( q - 1 \). The order of a non-zero element \( a \in \BbbK \) divides \( q - 1 \) by \fullref{thm:def:group_order/properties/divides}, hence \( a^{q - 1} = 1 \). We also have \( 0^q = 0 \). Therefore, for every element of \( \BbbF_q \), we have \( a^q = a \).

  From \fullref{thm:polynomial_root_iff_divisible} it follows that
  \begin{equation*}
    X^q - X = \prod_{u \in \BbbK} (X - u).
  \end{equation*}
\end{proof}

\begin{proposition}\label{thm:functions_over_finite_fields}
  For every \hyperref[thm:finite_fields]{finite field} \hyperref[thm:ring_of_integers_modulo]{\( \BbbF_q \)} and for every \hyperref[def:polynomial_semiring]{polynomial ring} \( \BbbF_q[X_1, \ldots, X_n] \) in finitely many indeterminates, there exists an \hyperref[def:algebra_over_ring]{\( \BbbF_q \)-algebra} isomorphism
  \begin{equation*}
    \frac {\BbbF_q[X_1, \ldots, X_n]} {\braket{ X_i^q - X_i \given i = 1, \ldots, n }} \cong \fun(\BbbF_q^n, \BbbF_q),
  \end{equation*}
  where \( \fun(\BbbF_q^n, \BbbF_q) \) is the \( \BbbF_q \)-algebra \hyperref[thm:functions_over_semimodule]{of all functions} from \( \BbbF
  _q^n \) to \( \BbbF_q \).

  Furthermore, every coset of polynomials has a unique representative given by \fullref{thm:finite_field_lagrange_interpolation}.
\end{proposition}
\begin{proof}
  Consider the \hyperref[thm:polynomial_semiring_universal_property]{functional evaluation homomorphism}
  \begin{equation*}
    \Phi: \BbbF_q[X_1, \ldots, X_m] \to \fun(\BbbF_q^m, \BbbF_q).
  \end{equation*}

  By \fullref{thm:finite_field_lagrange_interpolation}, \( \Phi \) is surjective. Then, by \fullref{thm:quotient_algebra_universal_property},
  \begin{equation*}
    \BbbF_q[X_1, \ldots, X_n] / \ker \Phi \cong \fun(\BbbF_q^m, \BbbF_q).
  \end{equation*}

  We will now prove that \( \ker \Phi \) equals
  \begin{equation*}
    I \coloneqq \braket{ X_i^q - X_i \given i = 1, \ldots, n }.
  \end{equation*}

  First, let \( e: \mscrX \to \BbbF_q \) be the variable assignment that assigns \( u_1, \ldots, u_n \) to the corresponding indeterminates. By \fullref{thm:fermats_little_theorem}, for any indeterminate \( X_i \), we have
  \begin{equation*}
    \Phi_e(X_i^q - X_i) = u_i^q - u_i = 0.
  \end{equation*}

  Hence, the polynomial function \( \Phi(X_i^q - X_i) \) is the zero constant function. It follows that any linear combination of the polynomials \( X_i^q - X_i \) for \( i = 1, \ldots, n \) is also the zero function. Therefore, \( I \subseteq \ker \Phi \).

  We will prove the converse inclusion via induction on \( n \).

  In the case of a single indeterminate \( X \), for every polynomial \( f(X) \in \ker \Phi \), we know that the entirety of \( \BbbF_q \) are roots of \( f(X) \). By \fullref{thm:integral_domain_polynomial_root_limit}, \( f(X) \) has at most \( q \) roots. Hence, by \fullref{thm:polynomial_root_iff_divisible}, \( X - u \) divides \( f(X) \) for every \( u \in \BbbF_q \). We have
  \begin{equation*}
    \underbrace{\prod_{u \in \BbbF_q} (X - u)}_{\mathclap{ X^q - X \T*{by \fullref{thm:finite_fields/splitting}}}} \mid f(X),
  \end{equation*}
  and hence \( f(X) \in \braket{ X^q - X } \).

  We have, up until now, shown that the entire proposition holds for the case of one indeterminate. Suppose that the proposition holds for \( n - 1 \) indeterminates and let \( f \in \BbbF_q[X_1, \ldots, X_n] \) be a nonconstant polynomial such that \( \Phi(f) \) is the zero function. Due to \fullref{thm:def:polynomial_semiring/properties/iterated}, we can regard \( f \) as a univariate polynomial in \( X_n \)over \( \BbbF_q[X_1, \ldots, X_{n-1}] \). Thus,
  \begin{equation*}
    f(X_1, \ldots, X_n) = \sum_{k=0}^\infty \underbrace{\parens*{ \sum_\gamma a_{(k,\gamma)} X_1^{\gamma_1} X_2^{\gamma_1} \cdots X_{n-1}^{\gamma_{n-1}} }}_{s_k(X_1, \ldots, X_{n-1})} {X_n}^k,
  \end{equation*}
  where \( \gamma \) is a multi-index over the first \( n - 1 \) indeterminates.

  As a polynomial in \( X_n \), \( f \) has \( m \coloneqq (n-1)q \) roots \( s_1, \ldots, s_m \), which are themselves polynomials from \( \BbbF_q[X_1, \ldots, X_{n-1}] \). For some \( c \), we have
  \begin{equation*}
    f(X_1, \ldots, X_n) = c(X_1, \ldots, X_{n-1}) \prod_{j=1}^m (X_n - s_j(X_1, \ldots, X_{n-1}))
  \end{equation*}
  and
  \begin{equation*}
    0 = \Phi(f) = \Phi(c) \cdot \prod_{j=1}^m \parens[\Big]{ \Phi(X_n) - \Phi(s_j) }.
  \end{equation*}

  Since \( \BbbF_q[X_1, \ldots, X_{n-1}] \) is \hyperref[def:entire_semiring]{entire}, we conclude that either \( \Phi(c) \) is the zero function or \( \Phi(X_n) = \Phi(s_j) \) for at least one index \( 1 \leq j \leq m \). The latter is impossible, because \( \Phi(X_n) \) is linearly independent from polynomials in the first \( n - 1 \) variables.

  The inductive hypothesis holds for the polynomial \( c \), and \( \Phi(c) \) being the zero function implies
  \begin{equation*}
    c \in \braket{ X_i^q - X_i \given i = 1, \ldots, n - 1 } \subsetneq I.
  \end{equation*}

  Therefore, \( f \in I \) since \( f \) divides \( c \). We have chosen \( f \) to be an arbitrary member of \( \ker \Phi \), which implies \( \ker \Phi \subseteq I \).

  We have already shown that \( I \subseteq \ker \Phi \). We thus conclude that \( I = \ker \Phi \) and
  \begin{equation*}
    \BbbF_q[X_1, \ldots, X_m] / I \cong \fun(\BbbF_q^m, \BbbF_q).
  \end{equation*}
\end{proof}

\begin{definition}\label{def:transcendetal_element}\mcite[454]{Knapp2016BasicAlgebra}
  We say that the element \( a \in \BbbK \) of the field extension \( \BbbK \) of \( \Bbbk \) is \term{transcendental} over \( \BbbK \) if any of the equivalent conditions hold:
  \begin{thmenum}
    \thmitem{def:transcendetal_element/evaluation} The \hyperref[thm:polynomial_ring_universal_property]{evaluation map} \( \Phi_a: \Bbbk[X] \to \Bbbk[a] \) is injective.

    \thmitem{def:transcendetal_element/polynomial} If \( a \) is a root of some polynomial \( p(X) \in \Bbbk[X] \), then \( p(X) \) is the zero polynomial.
  \end{thmenum}

  If \( a \) is not transcendental, we say that it is \term{algebraic}. If every element of \( \BbbK \) is algebraic over \( \Bbbk \), we say that \( \BbbK \) is an \term{algebraic extension} of \( \Bbbk \).
\end{definition}
\begin{defproof}
  \ImplicationSubProof{def:transcendetal_element/evaluation}{def:transcendetal_element/polynomial} Suppose that \( \Phi_a \) is injective and that there exists a polynomial \( p(X) \) such that \( p(a) = 0 \). Then for any other polynomial \( q(X) \in \Bbbk[X] \), we have \( q(a) p(a) = 0 \), and hence either \( p \) is the zero polynomial or the evaluation map is not injective.

  \ImplicationSubProof{def:transcendetal_element/polynomial}{def:transcendetal_element/evaluation} Conversely, suppose that \( a \) is a root only of the zero polynomial. Let \( p(a) = q(a) \). Then \( a \) is a root of \( p - q \) and hence the latter is the zero polynomial. But this implies that \( p = q \). Hence, the evaluation map is injective.
\end{defproof}

\begin{proposition}\label{thm:field_is_algebraic_over_itself}
  Every field is an \hyperref[def:transcendental_element]{algebraic extension} of itself.
\end{proposition}
\begin{proof}
  Every element \( a \in \BbbK \) is a root of the polynomial \( X - a \).
\end{proof}

\begin{theorem}[Euler's constant is transcendental]\label{thm:eulers_constant_is_transcendental}
  \hyperref[def:exponential_function]{Euler's constant} \( e \) is \hyperref[def:transcendetal_element]{transcendental} over \( \BbbQ \).
\end{theorem}

\begin{theorem}[Pi is transcendental]\label{thm:pi_is_transcendental}\mcite[454]{Knapp2016BasicAlgebra}
  The number \hyperref[def:pi]{\( \pi \)} is \hyperref[def:transcendetal_element]{transcendental} over \( \BbbQ \).
\end{theorem}

\begin{example}\label{ex:polynomials_over_pi}
  \Fullref{thm:pi_is_transcendental} implies that the polynomials \( \BbbQ[X] \) can be embedded into \( \BbbR \) via \( \Phi_\pi: \BbbQ[X] \to \BbbR \). We can identify a polynomial
  \begin{equation*}
    p(X) = \sum_{i=0}^n a_k X^k
  \end{equation*}
  with rational coefficients with the number
  \begin{equation*}
    p(\pi) = \sum_{i=0}^n a_k \pi^k.
  \end{equation*}
\end{example}

\begin{definition}\label{def:finite_field_extension}
  If \( \BbbK \) is \hyperref[def:module_presentation]{finitely generated} over \( \Bbbk \) (i.e. a finite-dimensional \hyperref[def:vector_space]{vector space}), we say that \( \BbbK \) is a \term{finite extensions} of \( \Bbbk \).
\end{definition}

\begin{lemma}\label{thm:prime_field_extensions_are_algebraic}
  Every finite \hyperref[def:field_extension]{field extension} is \hyperref[def:transcendetal_element]{algebraic}.
\end{lemma}
\begin{proof}
  Let \( \BbbK \) be a field extension of \( \Bbbk \) of degree \( n \). Suppose that the evaluation map \( \Phi_a: \Bbbk[X] \to \Bbbk[a] \) is injective for some \( a \in \BbbK \). Then \( 1, a, a^2, \ldots, a^n \) is a set of \( n + 1 \) linearly independent vectors in \( \Bbbk[a] \). But \( \Bbbk[a] \) is a subspace of \( \BbbK \), and hence must have dimension at most \( n \).

  The obtained contradiction shows that \( \Phi_a \) is not injective for any \( a \in \BbbK \). Therefore, every element of \( \BbbK \) is algebraic over \( \Bbbk \).
\end{proof}

\begin{definition}\label{def:algebraically_closed_field}\mcite[prop. 9.20]{Knapp2016BasicAlgebra}
  We say that the field \( \BbbK \) is algebraically closed if any of the equivalent conditions are satisfied:
  \begin{thmenum}
    \thmitem{def:algebraically_closed_field/trivial_algebraic_extensions} \( \BbbK \) has no nontrivial algebraic \hyperref[def:transcendental_element]{extensions}.
    \thmitem{def:algebraically_closed_field/linear_irreducible_polynomials} Every irreducible polynomial in \( \BbbK[X] \) is linear.
    \thmitem{def:algebraically_closed_field/at_least_one_root} Every nonconstant polynomial in \( \BbbK[X] \) has at least one root in \( \BbbK \).
    \thmitem{def:algebraically_closed_field/factorization} Every polynomial in \( \BbbK[X] \) \hyperref[def:factorization_in_ring]{factors} into a product of linear polynomials.
    \thmitem{def:algebraically_closed_field/exactly_n_roots} Every polynomial in \( \BbbK[X] \) of degree \( n \) has exactly \( n \) roots in \( \BbbK \), counting the root multiplicities.
  \end{thmenum}
\end{definition}
\begin{proof}
  \ImplicationSubProof{def:algebraically_closed_field/trivial_algebraic_extensions}{def:algebraically_closed_field/linear_irreducible_polynomials} Let \( p(X) \) be an irreducible polynomial in \( \BbbK[X] \).

  By \fullref{thm:ufd_prime_iff_irreducible}, \( p(X) \) is a prime element, and thus \( \braket {p(X)} \) is a \hyperref[def:semiring_ideal/prime]{prime ideal} in \( \BbbK[X] \). By \fullref{thm:prime_ideals_are_maximal_in_pid}, it is also a maximal ideal. By \fullref{def:maximal_ring_ideal/quotient}, the quotient \( Q \coloneqq \BbbK[X] / \braket{ p(X) } \) is a field. As a module over \( \BbbK \), it is finitely generated because the dimension of \( Q \) is the degree of \( p(X) \).

  By \fullref{thm:prime_field_extensions_are_algebraic}, \( Q \) is an algebraic extension of \( \BbbK \). Since \( \BbbK \) has no nontrivial algebraic extensions, it follows that \( \BbbK = Q \). Thus, \( Q \) has dimension \( 1 \), and we have already discussed that \( \dim Q = \deg p \). Therefore, \( p \) is a linear polynomial.

  \ImplicationSubProof{def:algebraically_closed_field/linear_irreducible_polynomials}{def:algebraically_closed_field/at_least_one_root} Suppose that every irreducible polynomial is linear.

  By \fullref{thm:polynomials_over_unique_factorization_domain_are_unique_factorization_domain}, \( \BbbK[X] \) is a unique factorization domain, and thus there exist irreducible polynomials \( q_1(X), \ldots, q_n(X) \) and a unit \( a \) such that
  \begin{equation*}
    p(X) = a q_1(X) \cdots q_n(X).
  \end{equation*}

  By assumption, the irreducible polynomials are linear, and hence have roots. Therefore, \( p(X) \) has at least one root.

  \ImplicationSubProof{def:algebraically_closed_field/at_least_one_root}{def:algebraically_closed_field/factorization} Suppose that \( u_1 \) is a root of \( p(X) \). \Fullref{thm:polynomial_root_iff_divisible} tells us that \( p(X) \) is divisible by \( (X - u_1) \). Using induction on the degree of \( p(X) \), we can factor \( p(X) \) into
  \begin{equation*}
    p(X) = a (X - u_1) (X - u_2) \cdots (X - u_n),
  \end{equation*}
  where \( a \) is a unit of \( \BbbK \). This is the desired factorization.

  \ImplicationSubProof{def:algebraically_closed_field/factorization}{def:algebraically_closed_field/exactly_n_roots} Follows from \fullref{thm:polynomial_root_iff_divisible} by induction on the polynomial degree. By \fullref{thm:integral_domain_polynomial_root_limit}, the number of roots is bounded by \( n \).

  \ImplicationSubProof{def:algebraically_closed_field/exactly_n_roots}{def:algebraically_closed_field/trivial_algebraic_extensions} Suppose that every nonconstant polynomial of degree \( n \) has exactly \( n \) roots in \( \Bbbk \) and let \( \BbbK \) be an algebraic extension of \( \Bbbk \).

  By \fullref{thm:integral_domain_polynomial_root_limit}, every polynomial in \( \BbbK[X] \) has at most \( n \) roots. By assumption, every root of every polynomial is contained in \( \Bbbk \). Since \( \BbbK \) is algebraic over \( \Bbbk \), it follows that every element of \( \BbbK \) is a root of some polynomial. Therefore, \( \BbbK = \Bbbk \).
\end{proof}

\begin{proposition}\label{thm:no_finite_extensions_of_closed_fields}
  An \hyperref[def:algebraically_closed_field]{algebraically closed field} has no nontrivial finite extension fields.
\end{proposition}
\begin{proof}
  Follows from \fullref{def:algebraically_closed_field/trivial_algebraic_extensions} and \fullref{thm:prime_field_extensions_are_algebraic}.
\end{proof}
