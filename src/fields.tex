\subsection{Fields}\label{subsec:fields}

\begin{definition}\label{def:division_ring}\mcite[144]{Knapp2016BasicAlgebra}
  If every nonzero element of a ring is \hyperref[def:divisibility/unit]{invertible}, we call it a \term{division ring}.
\end{definition}

\begin{proposition}\label{thm:division_ring_is_entire}
  A nontrivial \hyperref[def:division_ring]{division ring} is \hyperref[def:entire_semiring]{entire}.
\end{proposition}
\begin{proof}
  Let \( xy = 0 \). If \( x \) is nonzero, multiplying both sides by \( x^{-1} \), we obtain \( y = 0 \). Analogously, \( y \neq 0 \) implies that \( x = 0 \). In all cases, either \( x \) or \( y \) is necessarily zero.

  Therefore, the ring has no nontrivial zero divisors.
\end{proof}

\begin{definition}\label{def:field}
  We will call the \hyperref[def:ring/trivial]{nontrivial} \hyperref[def:ring]{ring} \( \BbbK \) a \term{field} if any of the following equivalent conditions hold:
  \begin{thmenum}
    \thmitem{def:field/simple} \( \BbbK \) is \hyperref[def:ring/commutative]{commutative} and \hyperref[def:ring/simple]{simple}.
    \thmitem{def:field/division_ring} \( \BbbK \) is a \hyperref[def:ring/commutative]{commutative} \hyperref[def:division_ring]{division ring}.
  \end{thmenum}
\end{definition}
\begin{proof}
  \ImplicationSubProof{def:field/simple}{def:field/division_ring} Suppose that \( \BbbK \) is a commutative simple ring.

  Let \( x \) be a nonzero element. Since there are no nontrivial proper ideals, the principal ideal \( \braket{ x } \) can only be \( \BbbK \). Hence, \( 1_\BbbK \in \braket{ x } \), which implies the existence of an element \( y \) such that \( xy = 1_\BbbK \).

  Therefore, \( y \) is a multiplicative inverse of \( x \). Generalizing on \( x \), we can conclude that \( \BbbK \) is a division ring.

  \ImplicationSubProof{def:field/division_ring}{def:field/simple} Suppose that \( \BbbK \) is a commutative division ring.

  Let \( I \) be a nontrivial ideal of \( \BbbK \). For any nonzero \( x \in I \), we have \( x^{-1} x \in I \) since \( I \) is closed under multiplication with elements of \( \BbbK \). Therefore, \( I \) contains the multiplicative identity \( 1_\BbbK \), and by \fullref{thm:def:semiring_ideal/properties/proper_ideals_containing_identity}, \( \BbbK = I \).
\end{proof}

\begin{proposition}\label{thm:field_is_euclidean_domain}
  Every \hyperref[def:field]{field} is an \hyperref[def:euclidean_domain]{Euclidean domain}.
\end{proposition}
\begin{proof}
  By \fullref{thm:division_ring_is_entire}, a field is necessarily an integral domain. Since every element of \( \BbbK \) is divisible (without remainder), the Euclidean function can be arbitrary; for definiteness, we take it to be canonically zero.
\end{proof}

\begin{proposition}\label{thm:field_of_fractions}
  Let \( R \) be an \hyperref[def:integral_domain]{integral domain}. The \hyperref[def:ring_localization]{localization} of \( R \) at the zero ideal \( \set{ 0_R } \) is a \hyperref[def:field]{field}, which we call the \term{field of fractions} of \( R \).
\end{proposition}
\begin{proof}
  By \fullref{thm:def:ring_localization/properties/prime_ideal}, the localization by the prime ideal \( \set{ 0_R } \) has  whose only maximal ideal is \( S^{-1} \set{ 0_R } \). Since \( 0_R \) is absorbing, \( S^{-1} \set{ 0_R } \) is again the zero ideal.  Therefore, it is the only proper ideal of the localization \( S^{-1} R \), and hence the localization is a \hyperref[def:ring/simple]{simple ring}.

  Since \( R \) is an integral domain, by \fullref{thm:def:ring_localization/properties/injective_inclusion}, \( S^{-1} R \) is a superring of \( R \). It is therefore a nontrivial commutative simple ring, and thus it satisfies \fullref{def:field/simple}.
\end{proof}

\begin{proposition}\label{thm:prime_fields}
  For a \hyperref[def:prime_number]{prime number} \( p \), the ring \hyperref[thm:ring_of_integers_modulo]{\( \BbbZ_p \)} of integers modulo \( p \) is a \hyperref[def:field]{field}.
\end{proposition}
\begin{proof}
  Follows from \fullref{thm:cyclic_groups_are_simple} applied to the additive group.
\end{proof}

\begin{proposition}\label{thm:finite_field_characteristic}
  The \hyperref[def:semiring_characteristic]{characteristic} of a \hyperref[def:set_finiteness]{finite field} is a \hyperref[def:prime_number]{prime number}.
\end{proposition}
\begin{proof}
  Let \( \BbbF_q \) be a \hyperref[def:field]{field} of \hyperref[def:set_finiteness]{cardinality} \( q \).

  Let \( \iota: \BbbZ \to \BbbF_q \) be the \hyperref[thm:ring_characteristic_homomorphism]{unique homomorphism} from the integers. Consider the quotient ring \( \BbbZ / \ker \iota \). It is isomorphic to \( \BbbZ_c \), where \( c \) is the \hyperref[def:semiring_characteristic]{characteristic} of \( \BbbF_q \). It is also isomorphic to the image of \( \iota \) in \( \BbbF_q \).

  Every subring of \( \BbbF_q \) must be an integral domain, in particular \( \BbbZ_c \), hence by \fullref{thm:quotient_by_prime_ideal}, \( c \) must be a prime number.
\end{proof}

\begin{lemma}\label{thm:prime_field_all_root_polynomial}
  In a \hyperref[thm:prime_fields]{prime field} \( \BbbF_p \), we have the polynomial equality
  \begin{equation*}
    X^p - X = \prod_{m < p} (X - m).
  \end{equation*}
\end{lemma}
\begin{proof}
  Every nonnegative integer \( m < p \) is a root of
  \begin{equation*}
    \prod_{m < p} (X - m).
  \end{equation*}

  By \fullref{thm:fermats_little_theorem}, \( m \) is also root of \( X^p - X \).

  Therefore, their difference
  \begin{equation*}
    (X^p - X) - \prod_{m < p} (X - m)
  \end{equation*}
  has \( p + 1 \) roots. By \fullref{thm:integral_domain_polynomial_root_limit}, this is the zero polynomial.
\end{proof}

\begin{proposition}\label{thm:functions_over_prime_fields}
  For every \hyperref[thm:prime_fields]{prime field} \hyperref[thm:ring_of_integers_modulo]{\( \BbbF_p \)} and for every \hyperref[def:polynomial_semiring]{polynomial ring} \( \BbbZ_p[X_1, \ldots, X_n] \) in finitely many indeterminates, there exists an \hyperref[def:algebra_over_ring]{\( \BbbZ_p \)-algebra isomorphism}
  \begin{equation*}
    \frac {\BbbZ_n[X_1, \ldots, X_n]} {\braket{ X_i^p - X_i \given i = 1, \ldots, n }} \cong \fun(\BbbF_p^n, \BbbF_p),
  \end{equation*}
  where \( \fun(\BbbF_p^n, \BbbF_p) \) is the \( \BbbZ_p \)-algebra \hyperref[thm:functions_over_semimodule]{of all functions} from \( \BbbF_p^n \) to \( \BbbF_p \).

  Furthermore, \fullref{thm:prime_field_lagrange_interpolation} associates with any function from \( \BbbF_p^n \) to \( \BbbF_p \) a unique polynomial from the corresponding equivalence class.
\end{proposition}
\begin{proof}
  Consider the \hyperref[thm:polynomial_semiring_universal_property]{functional evaluation homomorphism}
  \begin{equation*}
    \Phi: \BbbZ_n[X_1, \ldots, X_m] \to \fun(\BbbZ_n^m, \BbbZ_n).
  \end{equation*}

  By \fullref{thm:prime_field_lagrange_interpolation}, \( \Phi \) is surjective. Then, by \fullref{thm:quotient_ring_universal_property},
  \begin{equation*}
    \BbbZ_n[X_1, \ldots, X_n] / \ker \Phi \cong \fun(\BbbZ_n^m, \BbbZ_n).
  \end{equation*}

  We will now prove that \( \ker \Phi \) equals
  \begin{equation*}
    I \coloneqq \braket[\Big]{ X_i^p - X_i \given* i = 1, \ldots, n }.
  \end{equation*}

  First, let \( f: \mscrX \to \BbbZ_p \) be the variable assignment that assigns \( x_1, \ldots, x_n \) to the corresponding indeterminates. By \fullref{thm:fermats_little_theorem}, for any indeterminate \( X_i \), we have
  \begin{equation*}
    \Phi_f(X_i^p - X_i) = x_i^p - x_i = 0.
  \end{equation*}

  Hence, the polynomial function \( \Phi(X_i^p - X_i) \) is the zero constant function. It follows that any linear combination of the polynomials \( X_i^p - X_i \) for \( i = 1, \ldots, n \) is also the zero function. Therefore, \( I \subseteq \ker \Phi \).

  We will prove the converse inclusion via induction on \( n \).

  In the case of a single indeterminate \( X \), for every polynomial \( q(X) \in \ker \Phi \), we know that every nonnegative integer \( m < p \) is a root of \( q(X) \). By \fullref{thm:integral_domain_polynomial_root_limit}, \( q(X) \) has at most \( p \) roots. Hence, these are all the roots, and by \fullref{thm:polynomial_root_iff_divisible}, \( X - m \) divides \( q(X) \) for every \( m < p \). We have
  \begin{equation*}
    \underbrace{\prod_{m=0}^{p - 1} (X - m)}_{\mathclap{ X^p - X \T*{by \fullref{thm:prime_field_all_root_polynomial}}}} \mid q(X),
  \end{equation*}
  and hence \( q(X) \in \braket{ X^p - X } \).

  We have, up until now, shown that the entire proposition holds for the case of one indeterminate. Suppose that the proposition holds for \( n - 1 \) indeterminates and let \( q \in R[X_1, \ldots, X_n] \) be such that \( \Phi(q) \) is the zero function. Due to \fullref{thm:def:polynomial_semiring/properties/iterated}, we can regard \( q \) as a member of \( R[X_1, \ldots, X_{n-1}][X_n] \). Thus,
  \begin{equation*}
    q(X_1, \ldots, X_n) = \sum_{k=0}^\infty \parens*{ \sum_\gamma a_{(k,\gamma)} X_1^{\gamma_1} X_2^{\gamma_1} \cdots X_{n-1}^{\gamma_{n-1}} } {X_n}^k,
  \end{equation*}
  where \( \gamma \) is a multi-index over the first \( n - 1 \) indeterminates.

  The inductive hypothesis holds for \( R[X_1, \ldots, X_{n-1}] \). The coefficients of \( q \) before \( X_n^k \) are polynomials from \( R[X_1, \ldots, X_{n-1}] \). If \( \Phi(q) \) is the zero function, we have several possibilities.
  \begin{itemize}
    \item If the coefficient polynomials of \( q \) evaluate to the zero function, by the inductive hypothesis,
    \begin{equation*}
      q \in \braket[\Big]{ X_i^q - X_i \given* i = 1, \ldots, n - 1 } \subsetneq I.
    \end{equation*}

    \item If there exists a coefficient polynomial \( q(X_1, \ldots, X_{n-1}) \) such that \( \Phi(q) \) is not the zero function, there must exist another coefficient polynomial \( r(X_1, \ldots, X_{n-1}) \) such that \( \Phi(q + r) \) is the zero function. Then \( q + r \in I \). Using induction on the number of nonzero coefficients of \( q \), we conclude that \( q \in I \).
  \end{itemize}
\end{proof}

\begin{definition}\label{def:field_extension}
  If \( \Bbbk \) and \( \BbbK \) are fields and \( \Bbbk \) is an \hyperref[def:ring/submodel]{unital subring} of \( \BbbK \), we say that \( \Bbbk \) is a \term{subfield} of \( \BbbK \) and that \( \BbbK \) is a \term{field extension} of \( \Bbbk \). A field extension that is not a proper superring is often called \term{trivial}.
\end{definition}

\begin{definition}\label{def:transcendetal_element}\mcite[454]{Knapp2016BasicAlgebra}
  We say that the element \( a \in \BbbK \) of the field extension \( \BbbK \) of \( \Bbbk \) is \term{transcendental} over \( \BbbK \) if any of the equivalent conditions hold:
  \begin{thmenum}
    \thmitem{def:transcendetal_element/evaluation} The \hyperref[thm:polynomial_ring_universal_property]{evaluation map} \( \Phi_a: \Bbbk[X] \to \Bbbk[a] \) is injective.

    \thmitem{def:transcendetal_element/polynomial} If \( a \) is a root of some polynomial \( p(X) \in \Bbbk[X] \), then \( p(X) \) is the zero polynomial.
  \end{thmenum}

  If \( a \) is not transcendental, we say that it is \term{algebraic}. If every element of \( \BbbK \) is algebraic over \( \Bbbk \), we say that \( \BbbK \) is an \term{algebraic extension} of \( \Bbbk \).
\end{definition}
\begin{defproof}
  \ImplicationSubProof{def:transcendetal_element/evaluation}{def:transcendetal_element/polynomial} Suppose that \( \Phi_a \) is injective and that there exists a polynomial \( p(X) \) such that \( p(a) = 0 \). Then for any other polynomial \( q(X) \in \Bbbk[X] \), we have \( q(a) p(a) = 0 \), and hence either \( p \) is the zero polynomial or the evaluation map is not injective.

  \ImplicationSubProof{def:transcendetal_element/polynomial}{def:transcendetal_element/evaluation} Conversely, suppose that \( a \) is a root only of the zero polynomial. Let \( p(a) = q(a) \). Then \( a \) is a root of \( p - q \) and hence the latter is the zero polynomial. But this implies that \( p = q \). Hence, the evaluation map is injective.
\end{defproof}

\begin{proposition}\label{thm:field_is_algebraic_over_itself}
  Every field is an \hyperref[def:transcendental_element]{algebraic extension} of itself.
\end{proposition}
\begin{proof}
  Every element \( a \in \BbbK \) is a root of the polynomial \( X - a \).
\end{proof}

\begin{theorem}[Euler's constant is transcendental]\label{thm:eulers_constant_is_transcendental}
  \hyperref[def:exponential_function]{Euler's constant} \( e \) is \hyperref[def:transcendetal_element]{transcendental} over \( \BbbQ \).
\end{theorem}

\begin{theorem}[Pi is transcendental]\label{thm:pi_is_transcendental}\mcite[454]{Knapp2016BasicAlgebra}
  The number \hyperref[def:pi]{\( \pi \)} is \hyperref[def:transcendetal_element]{transcendental} over \( \BbbQ \).
\end{theorem}

\begin{example}\label{ex:polynomials_over_pi}
  \Fullref{thm:pi_is_transcendental} implies that the polynomials \( \BbbQ[X] \) can be embedded into \( \BbbR \) via \( \Phi_\pi: \BbbQ[X] \to \BbbR \). We can identify a polynomial
  \begin{equation*}
    p(X) = \sum_{i=0}^n a_k X^k
  \end{equation*}
  with rational coefficients with the number
  \begin{equation*}
    p(\pi) = \sum_{i=0}^n a_k \pi^k.
  \end{equation*}
\end{example}

\begin{definition}\label{def:finite_field_extension}
  If \( \BbbK \) is \hyperref[def:module_presentation]{finitely generated} over \( \Bbbk \) (i.e. a finite-dimensional \hyperref[def:vector_space]{vector space}), we say that \( \BbbK \) is a \term{finite extensions} of \( \Bbbk \).
\end{definition}

\begin{lemma}\label{thm:prime_field_extensions_are_algebraic}
  Every finite \hyperref[def:field_extension]{field extension} is \hyperref[def:transcendetal_element]{algebraic}.
\end{lemma}
\begin{proof}
  Let \( \BbbK \) be a field extension of \( \Bbbk \) of degree \( n \). Suppose that the evaluation map \( \Phi_a: \Bbbk[X] \to \Bbbk[a] \) is injective for some \( a \in \BbbK \). Then \( 1, a, a^2, \ldots, a^n \) is a set of \( n + 1 \) linearly independent vectors in \( \Bbbk[a] \). But \( \Bbbk[a] \) is a subspace of \( \BbbK \), and hence must have dimension at most \( n \).

  The obtained contradiction shows that \( \Phi_a \) is not injective for any \( a \in \BbbK \). Therefore, every element of \( \BbbK \) is algebraic over \( \Bbbk \).
\end{proof}

\begin{definition}\label{def:algebraically_closed_field}\mcite[prop. 9.20]{Knapp2016BasicAlgebra}
  We say that the field \( \BbbK \) is algebraically closed if any of the equivalent conditions are satisfied:
  \begin{thmenum}
    \thmitem{def:algebraically_closed_field/trivial_algebraic_extensions} \( \BbbK \) has no nontrivial algebraic \hyperref[def:transcendental_element]{extensions}.
    \thmitem{def:algebraically_closed_field/linear_irreducible_polynomials} Every irreducible polynomial in \( \BbbK[X] \) is linear.
    \thmitem{def:algebraically_closed_field/at_least_one_root} Every nonconstant polynomial in \( \BbbK[X] \) has at least one root in \( \BbbK \).
    \thmitem{def:algebraically_closed_field/factorization} Every polynomial in \( \BbbK[X] \) \hyperref[def:factorization_in_ring]{factors} into a product of linear polynomials.
    \thmitem{def:algebraically_closed_field/exactly_n_roots} Every polynomial in \( \BbbK[X] \) of degree \( n \) has exactly \( n \) roots in \( \BbbK \), counting the root multiplicities.
  \end{thmenum}
\end{definition}
\begin{proof}
  \ImplicationSubProof{def:algebraically_closed_field/trivial_algebraic_extensions}{def:algebraically_closed_field/linear_irreducible_polynomials} Let \( p(X) \) be an irreducible polynomial in \( \BbbK[X] \).

  By \fullref{thm:ufd_prime_iff_irreducible}, \( p(X) \) is a prime element, and thus \( \braket {p(X)} \) is a \hyperref[def:semiring_ideal/prime]{prime ideal} in \( \BbbK[X] \). By \fullref{thm:prime_ideals_are_maximal_in_pid}, it is also a maximal ideal. By \fullref{def:maximal_ring_ideal/quotient}, the quotient \( Q \coloneqq \BbbK[X] / \braket{ p(X) } \) is a field. As a module over \( \BbbK \), it is finitely generated because the dimension of \( Q \) is the degree of \( p(X) \).

  By \fullref{thm:prime_field_extensions_are_algebraic}, \( Q \) is an algebraic extension of \( \BbbK \). Since \( \BbbK \) has no nontrivial algebraic extensions, it follows that \( \BbbK = Q \). Thus, \( Q \) has dimension \( 1 \), and we have already discussed that \( \dim Q = \deg p \). Therefore, \( p \) is a linear polynomial.

  \ImplicationSubProof{def:algebraically_closed_field/linear_irreducible_polynomials}{def:algebraically_closed_field/at_least_one_root} Suppose that every irreducible polynomial is linear.

  By \fullref{thm:polynomials_over_unique_factorization_domain_are_unique_factorization_domain}, \( \BbbK[X] \) is a unique factorization domain, and thus there exist irreducible polynomials \( q_1(X), \ldots, q_n(X) \) and a unit \( a \) such that
  \begin{equation*}
    p(X) = a q_1(X) \cdots q_n(X).
  \end{equation*}

  By assumption, the irreducible polynomials are linear, and hence have roots. Therefore, \( p(X) \) has at least one root.

  \ImplicationSubProof{def:algebraically_closed_field/at_least_one_root}{def:algebraically_closed_field/factorization} Suppose that \( u_1 \) is a root of \( p(X) \). \Fullref{thm:polynomial_root_iff_divisible} tells us that \( p(X) \) is divisible by \( (X - u_1) \). Using induction on the degree of \( p(X) \), we can factor \( p(X) \) into
  \begin{equation*}
    p(X) = a (X - u_1) (X - u_2) \cdots (X - u_n),
  \end{equation*}
  where \( a \) is a unit of \( \BbbK \). This is the desired factorization.

  \ImplicationSubProof{def:algebraically_closed_field/factorization}{def:algebraically_closed_field/exactly_n_roots} Follows from \fullref{thm:polynomial_root_iff_divisible} by induction on the polynomial degree. By \fullref{thm:integral_domain_polynomial_root_limit}, the number of roots is bounded by \( n \).

  \ImplicationSubProof{def:algebraically_closed_field/exactly_n_roots}{def:algebraically_closed_field/trivial_algebraic_extensions} Suppose that every nonconstant polynomial of degree \( n \) has exactly \( n \) roots in \( \Bbbk \) and let \( \BbbK \) be an algebraic extension of \( \Bbbk \).

  By \fullref{thm:integral_domain_polynomial_root_limit}, every polynomial in \( \BbbK[X] \) has at most \( n \) roots. By assumption, every root of every polynomial is contained in \( \Bbbk \). Since \( \BbbK \) is algebraic over \( \Bbbk \), it follows that every element of \( \BbbK \) is a root of some polynomial. Therefore, \( \BbbK = \Bbbk \).
\end{proof}

\begin{proposition}\label{thm:no_finite_extensions_of_closed_fields}
  An \hyperref[def:algebraically_closed_field]{algebraically closed field} has no nontrivial finite extension fields.
\end{proposition}
\begin{proof}
  Follows from \fullref{def:algebraically_closed_field/trivial_algebraic_extensions} and \fullref{thm:prime_field_extensions_are_algebraic}.
\end{proof}
