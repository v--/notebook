\subsection{Fields}\label{subsec:fields}

\begin{definition}\label{def:division_ring}\mcite[144]{Knapp2016BasicAlgebra}
  If every nonzero element of a ring is \hyperref[def:divisibility/unit]{invertible}, we call it a \term{division ring}.
\end{definition}

\begin{proposition}\label{thm:division_ring_is_entire}
  A nontrivial \hyperref[def:division_ring]{division ring} is \hyperref[def:entire_semiring]{entire}.
\end{proposition}
\begin{proof}
  Let \( xy = 0 \). If \( x \) is nonzero, multiplying both sides by \( x^{-1} \), we obtain \( y = 0 \). Analogously, \( y \neq 0 \) implies that \( x = 0 \). In all cases, either \( x \) or \( y \) is necessarily zero.

  Therefore, the ring has no nontrivial zero divisors.
\end{proof}

\begin{definition}\label{def:field}
  We will call the \hyperref[def:ring/trivial]{nontrivial} \hyperref[def:ring]{ring} \( \BbbK \) a \term{field} if any of the following equivalent conditions hold:
  \begin{thmenum}
    \thmitem{def:field/simple} \( \BbbK \) is \hyperref[def:ring/commutative]{commutative} and \hyperref[def:ring/simple]{simple}.
    \thmitem{def:field/division_ring} \( \BbbK \) is a \hyperref[def:ring/commutative]{commutative} \hyperref[def:division_ring]{division ring}.
  \end{thmenum}
\end{definition}
\begin{proof}
  \ImplicationSubProof{def:field/simple}{def:field/division_ring} Suppose that \( \BbbK \) is a commutative simple ring.

  Let \( x \) be a nonzero element. Since there are no nontrivial proper ideals, the principal ideal \( \braket{ x } \) can only be \( \BbbK \). Hence, \( 1_\BbbK \in \braket{ x } \), which implies the existence of an element \( y \) such that \( xy = 1_\BbbK \).

  Therefore, \( y \) is a multiplicative inverse of \( x \). Generalizing on \( x \), we can conclude that \( \BbbK \) is a division ring.

  \ImplicationSubProof{def:field/division_ring}{def:field/simple} Suppose that \( \BbbK \) is a commutative division ring.

  Let \( I \) be a nontrivial ideal of \( \BbbK \). For any nonzero \( x \in I \), we have \( x^{-1} x \in I \) since \( I \) is closed under multiplication with elements of \( \BbbK \). Therefore, \( I \) contains the multiplicative identity \( 1_\BbbK \), and by \fullref{thm:def:semiring_ideal/properties/proper_ideals_containing_identity}, \( \BbbK = I \).
\end{proof}

\begin{proposition}\label{thm:field_is_euclidean_domain}
  Every \hyperref[def:field]{field} is an \hyperref[def:euclidean_domain]{Euclidean domain}.
\end{proposition}
\begin{proof}
  By \fullref{thm:division_ring_is_entire}, a field is necessarily an integral domain. Since every element of \( \BbbK \) is divisible (without remainder), the Euclidean function can be arbitrary; for definiteness, we take it to be canonically zero.
\end{proof}

\begin{lemma}\label{thm:ring_of_integers_modulo_zero_divisors}
  In the ring \hyperref[thm:ring_of_integers_modulo]{\( \BbbZ_n \)} of integers modulo \( n \), a nonzero number \( m \) is a \hyperref[def:divisibility]{zero divisor} if and only if \( m \) is not \hyperref[def:coprime_numbers]{coprime} to \( n \).
\end{lemma}
\begin{proof}
  Simply note that \( m \odot k = 0 \) for nonzero \( m \) and \( k \) if and only if \( n \mid mk \).
\end{proof}

\begin{proposition}\label{thm:ring_of_integers_modulo_prime_is_field}
  For a \hyperref[def:prime_number]{prime number} \( p \), the ring \hyperref[thm:ring_of_integers_modulo]{\( \BbbZ_p \)} of integers modulo \( p \) is a \hyperref[def:field]{field}.
\end{proposition}
\begin{proof}
  By \fullref{thm:ring_of_integers_modulo}, \( \BbbZ_p \) is a commutative ring. By \fullref{thm:ring_of_integers_modulo_zero_divisors}, it has no zero divisors for prime \( p \).

  Hence, it only remains to show that every number \( m < p \) has a multiplicative inverse in \( \BbbZ_p \). \Fullref{thm:bezout_identity} gives us integers \( a, b \in \BbbZ \) such that
  \begin{equation*}
    am + bp = \gcd(m, p) = 1.
  \end{equation*}

  Without loss of generality, we can assume that \( a < p \). We can take \( \rem(a, p) \) otherwise.

  We have
  \begin{equation*}
    m \odot a = \rem(ma, p) = \rem(ma + bp, p) = \rem(1, p) = 1.
  \end{equation*}

  Then \( a \) is a multiplicative inverse of \( m \).
\end{proof}

\begin{definition}\label{def:field_extension}
  If \( \Bbbk \) and \( \BbbK \) are fields and \( \Bbbk \) is an \hyperref[def:ring/submodel]{unital subring} of \( \BbbK \), we say that \( \Bbbk \) is a \term{subfield} of \( \BbbK \) and that \( \BbbK \) is a \term{field extension} of \( \Bbbk \). A field extension that is not a proper superring is often called \term{trivial}.

  \begin{thmenum}
    \thmitem{def:field_extension/dimension} A field extension \( \BbbK \) of \( \Bbbk \) is a \hyperref[def:vector_space]{vector space} over \( \Bbbk \). Of particular importance are extensions of finite dimension, which we call \term{finite field extensions}.

    \thmitem{def:field_extension/generated_extension} For members \( x_1, \ldots, x_n \) of \( \BbbK \), we can introduce the following:
    \begin{itemize}
      \item The ring \( \Bbbk[x_1, \ldots, x_n] \) obtained by evaluating \hyperref[thm:polynomial_ring_universal_property]{polynomials}.
      \item The field \( \Bbbk(x_1, \ldots, x_n) \) obtained by evaluating \hyperref[def:rational_algebraic_function]{rational algebraic functions}.
    \end{itemize}
  \end{thmenum}
\end{definition}

\begin{remark}\label{rem:adjoint_extension_field}
  Any field of \hyperref[def:rational_algebraic_function]{rational algebraic functions} over \( \BbbK \) is always a field extension of \( \BbbK \). We say that the field \( \BbbK(X) \) is obtained from \( \BbbK \) by \term{adjoining} a new element \( X \). Although formally \( X \) is a polynomial, we regard it as a symbol in the sense of \fullref{def:formal_language}.
\end{remark}

\begin{definition}\label{def:transcendetal_element}\mcite[454]{Knapp2016BasicAlgebra}
  We say that the element \( a \in \BbbK \) of the field extension \( \BbbK \) of \( \Bbbk \) is \term{transcendental} over \( \BbbK \) if any of the equivalent conditions hold:
  \begin{thmenum}
    \thmitem{def:transcendetal_element/evaluation} The \hyperref[thm:polynomial_ring_universal_property]{evaluation map} \( \Phi_a: \Bbbk[X] \to \Bbbk[a] \) is injective.

    \thmitem{def:transcendetal_element/polynomial} If \( a \) is a root of some polynomial \( p(X) \in \Bbbk[X] \), then \( p(X) \) is the zero polynomial.
  \end{thmenum}

  If \( a \) is not transcendental, we say that it is \term{algebraic}. If every element of \( \BbbK \) is algebraic over \( \Bbbk \), we say that \( \BbbK \) is an \term{algebraic extension} of \( \Bbbk \).
\end{definition}
\begin{defproof}
  \ImplicationSubProof{def:transcendetal_element/evaluation}{def:transcendetal_element/polynomial} Suppose that \( \Phi_a \) is injective and that there exists a polynomial \( p(X) \) such that \( p(a) = 0 \). Then for any other polynomial \( q(X) \in \Bbbk[X] \), we have \( q(a) p(a) = 0 \), and hence either \( p \) is the zero polynomial or the evaluation map is not injective.

  \ImplicationSubProof{def:transcendetal_element/polynomial}{def:transcendetal_element/evaluation} Conversely, suppose that \( a \) is a root only of the zero polynomial. Let \( p(a) = q(a) \). Then \( a \) is a root of \( p - q \) and hence the latter is the zero polynomial. But this implies that \( p = q \). Hence, the evaluation map is injective.
\end{defproof}

\begin{proposition}\label{thm:field_is_algebraic_over_itself}
  Every field is an \hyperref[def:transcendental_element]{algebraic extension} of itself.
\end{proposition}
\begin{proof}
  Every element \( a \in \BbbK \) is a root of the polynomial \( X - a \).
\end{proof}

\begin{theorem}[Euler's constant is transcendental]\label{thm:eulers_constant_is_transcendental}
  \hyperref[def:exponential_function]{Euler's constant} \( e \) is \hyperref[def:transcendetal_element]{transcendental} over \( \BbbQ \).
\end{theorem}

\begin{theorem}[Pi is transcendental]\label{thm:pi_is_transcendental}\mcite[454]{Knapp2016BasicAlgebra}
  The number \hyperref[def:pi]{\( \pi \)} is \hyperref[def:transcendetal_element]{transcendental} over \( \BbbQ \).
\end{theorem}

\begin{example}\label{ex:polynomials_over_pi}
  \Fullref{thm:pi_is_transcendental} implies that the polynomials \( \BbbQ[X] \) can be embedded into \( \BbbR \) via \( \Phi_\pi: \BbbQ[X] \to \BbbR \). We can identify a polynomial
  \begin{equation*}
    p(X) = \sum_{i=0}^n a_k X^k
  \end{equation*}
  with rational coefficients with the number
  \begin{equation*}
    p(\pi) = \sum_{i=0}^n a_k \pi^k.
  \end{equation*}
\end{example}

\begin{lemma}\label{thm:finite_field_extensions_are_algebraic}
  Every finite \hyperref[def:field_extension]{field extension} is \hyperref[def:transcendetal_element]{algebraic}.
\end{lemma}
\begin{proof}
  Let \( \BbbK \) be a field extension of \( \Bbbk \) of degree \( n \). Suppose that the evaluation map \( \Phi_a: \Bbbk[X] \to \Bbbk[a] \) is injective for some \( a \in \BbbK \). Then \( 1, a, a^2, \ldots, a^n \) is a set of \( n + 1 \) linearly independent vectors in \( \Bbbk[a] \). But \( \Bbbk[a] \) is a subspace of \( \BbbK \), and hence must have dimension at most \( n \).

  The obtained contradiction shows that \( \Phi_a \) is not injective for any \( a \in \BbbK \). Therefore, every element of \( \BbbK \) is algebraic over \( \Bbbk \).
\end{proof}

\begin{definition}\label{def:algebraically_closed_field}\mcite[prop. 9.20]{Knapp2016BasicAlgebra}
  We say that the field \( \BbbK \) is algebraically closed if any of the equivalent conditions are satisfied:
  \begin{thmenum}
    \thmitem{def:algebraically_closed_field/trivial_algebraic_extensions} \( \BbbK \) has no nontrivial algebraic \hyperref[def:transcendental_element]{extensions}.
    \thmitem{def:algebraically_closed_field/linear_irreducible_polynomials} Every irreducible polynomial in \( \BbbK[X] \) is linear.
    \thmitem{def:algebraically_closed_field/at_least_one_root} Every nonconstant polynomial in \( \BbbK[X] \) has at least one root in \( \BbbK \).
    \thmitem{def:algebraically_closed_field/factorization} Every polynomial in \( \BbbK[X] \) \hyperref[def:factorization_in_ring]{factors} into a product of linear polynomials.
    \thmitem{def:algebraically_closed_field/exactly_n_roots} Every polynomial in \( \BbbK[X] \) of degree \( n \) has exactly \( n \) roots in \( \BbbK \), counting the root multiplicities.
  \end{thmenum}
\end{definition}
\begin{proof}
  \ImplicationSubProof{def:algebraically_closed_field/trivial_algebraic_extensions}{def:algebraically_closed_field/linear_irreducible_polynomials} Let \( p(X) \) be an irreducible polynomial in \( \BbbK[X] \).

  By \fullref{thm:ufd_prime_iff_irreducible}, \( p(X) \) is a prime element, and thus \( \braket {p(X)} \) is a \hyperref[def:prime_ideal]{prime ideal} in \( \BbbK[X] \). By \fullref{thm:prime_ideals_are_maximal_in_pid}, it is also a maximal ideal. By \fullref{def:maximal_ring_ideal/quotient}, the quotient \( Q \coloneqq \BbbK[X] / \braket{ p(X) } \) is a field. As a module over a field, it is a vector space, and thus a field extension of \( \BbbK \). Furthermore, it is a finite field extension because the dimension of \( Q \) is the degree of \( p(X) \).

  By \fullref{thm:finite_field_extensions_are_algebraic}, \( Q \) is an algebraic extension of \( \BbbK \). Since \( \BbbK \) has no nontrivial algebraic extensions, it follows that \( \BbbK = Q \). Thus, \( Q \) has dimension \( 1 \), and we have already discussed that \( \dim Q = \deg p \). Therefore, \( p \) is a linear polynomial.

  \ImplicationSubProof{def:algebraically_closed_field/linear_irreducible_polynomials}{def:algebraically_closed_field/at_least_one_root} Suppose that every irreducible polynomial is linear.

  By \fullref{thm:polynomials_over_unique_factorization_domain_are_unique_factorization_domain}, \( \BbbK[X] \) is a unique factorization domain, and thus there exist irreducible polynomials \( q_1(X), \ldots, q_n(X) \) and a unit \( a \) such that
  \begin{equation*}
    p(X) = a q_1(X) \cdots q_n(X).
  \end{equation*}

  By assumption, the irreducible polynomials are linear, and hence have roots. Therefore, \( p(X) \) has at least one root.

  \ImplicationSubProof{def:algebraically_closed_field/at_least_one_root}{def:algebraically_closed_field/factorization} Suppose that \( u_1 \) is a root of \( p(X) \). \Fullref{thm:polynomial_root_iff_divisible} tells us that \( p(X) \) is divisible by \( (X - u_1) \). Using induction on the degree of \( p(X) \), we can factor \( p(X) \) into
  \begin{equation*}
    p(X) = a (X - u_1) (X - u_2) \cdots (X - u_n),
  \end{equation*}
  where \( a \) is a unit of \( \BbbK \). This is the desired factorization.

  \ImplicationSubProof{def:algebraically_closed_field/factorization}{def:algebraically_closed_field/exactly_n_roots} Follows from \fullref{thm:polynomial_root_iff_divisible} by induction on the polynomial degree. By \fullref{thm:integral_domain_polynomial_root_limit}, the number of roots is bounded by \( n \).

  \ImplicationSubProof{def:algebraically_closed_field/exactly_n_roots}{def:algebraically_closed_field/trivial_algebraic_extensions} Suppose that every nonconstant polynomial of degree \( n \) has exactly \( n \) roots in \( \Bbbk \) and let \( \BbbK \) be an algebraic extension of \( \Bbbk \).

  By \fullref{thm:integral_domain_polynomial_root_limit}, every polynomial in \( \BbbK[X] \) has at most \( n \) roots. By assumption, every root of every polynomial is contained in \( \Bbbk \). Since \( \BbbK \) is algebraic over \( \Bbbk \), it follows that every element of \( \BbbK \) is a root of some polynomial. Therefore, \( \BbbK = \Bbbk \).
\end{proof}

\begin{proposition}\label{thm:no_finite_extensions_of_closed_fields}
  An \hyperref[def:algebraically_closed_field]{algebraically closed field} has no nontrivial finite extension fields.
\end{proposition}
\begin{proof}
  Follows from \fullref{def:algebraically_closed_field/trivial_algebraic_extensions} and \fullref{thm:finite_field_extensions_are_algebraic}.
\end{proof}

\begin{proposition}\label{thm:f2_is_boolean_algebra}
  The finite field \( \BbbZ_2 \) is \hyperref[def:partially_ordered_set/homomorphism]{order-isomorphic} to the two-element \hyperref[def:boolean_algebra/trivial]{Boolean algebra} \( \set{ \top, \bot } \).
\end{proposition}
\begin{proof}
  We identify \( 0 \) with \( \bot \) and \( 1 \) with \( \top \). The verification itself is straightforward.
\end{proof}

\begin{algorithm}\label{alg:finite_field_polynomial_reduction}
  Consider the field \( \BbbZ_p \) for some prime number \( p \) and the nonzero polynomial
  \begin{equation*}
    f(X_1, \ldots, X_n) \coloneqq \sum_{k_1=0}^{m_1} \cdots \sum_{k_n=0}^{m_n} a_{k_1,\ldots,k_n} X_1^{k_1} \cdots X_n^{k_n}.
  \end{equation*}

  We will build a polynomial \( \hat f(X_1, \ldots, X_n) \) of degree at most \( n - 1 \) that corresponds to the same function. In terms of the \hyperref[thm:polynomial_ring_universal_property]{evaluation homomorphism}
  \begin{equation*}
    \Phi: \BbbZ_p[X_1, \ldots, X_n] \to \fun(\BbbZ_p^n, \BbbZ_p),
  \end{equation*}
  this means that
  \begin{equation*}
    \Phi(f) = \Phi(\hat f).
  \end{equation*}

  This can be achieved by grouping some of the coefficients. The univariate case is more understandable, so we will initially consider the polynomial
  \begin{equation*}
    g(X) \coloneqq \sum_{k=0}^m a_k X^k.
  \end{equation*}

  We will now use (and later prove) that the following two univariate polynomials evaluate to the same function:
  \begin{equation}\label{eq:alg:finite_field_polynomial_reduction/reduction}
    \Phi(X^s) = \Phi(X^{\rho(s)}),
  \end{equation}
  where
  \begin{align*}
    &\rho: \set{ 1, 2, 3, \ldots } \to \set{ 1, 2, \ldots, p - 1 } \\
    &\rho(s) \coloneqq \begin{cases}
      p - 1,          &(p - 1) \mid s, \\
      \rem(s, p - 1), &\T{otherwise}
    \end{cases}
  \end{align*}

  Obviously \( \rho(s) = s \) for \( s < p \). It follows from the linearity of \( \Phi \) that
  \begin{equation*}
    \hat g(X)
    \coloneqq
    \sum_{k=0}^m a_k X^{r(k)}
    =
    \sum_{j=0}^{p-1} \left( \sum_{k \in r^{-1}(j)} a_k \right) X^j
  \end{equation*}
  is the desired polynomial.

  In the multivariate case, the reduced polynomial is
  \begin{equation*}
    \hat f(X_1, \ldots, X_n) \coloneqq \sum_{j_1=0}^{p-1} \cdots \sum_{j_n=0}^{p-1} \left( \sum_{k_1 \in r^{-1}(j_1)} \cdots \sum_{k_n \in r^{-1}(j_n)} a_{k_1,\ldots,k_n} \right) X_1^{k_1} \cdots X_n^{k_n}.
  \end{equation*}

  Correctness follows, again, from the linearity of \( \Phi \).
\end{algorithm}
\begin{proof}[Proof of \eqref{eq:alg:finite_field_polynomial_reduction/reduction}]
  By \fullref{thm:fermats_little_theorem},
  \begin{equation}\label{eq:alg:finite_field_polynomial_reduction/fermat}
    \Phi(X^p) = \Phi(X).
  \end{equation}

  We now use induction on \( q \coloneqq \quot(s, p) \).
  \begin{itemize}
    \item If \( q = 0 \), then \( s < p \) and no reduction is necessary.
    \item If \( q > 0 \), division gives us \( s = (p - 1) q + r \). We have two cases
    \begin{itemize}
      \item If \( (p - 1) \mid s \), i.e. if \( r = 0 \), then
      \begin{balign*}
        \Phi(X^s)
        &=
        \Phi(X^{(p-1)q})
        = \\ &=
        \Phi(X^{(p-1)(q-1) + (p-1)})
        = \\ &=
        \Phi(X^{(p-1)(q-1)}) \Phi(X^{p-1})
        \reloset {\T{ind.}} = \\ &=
        \Phi(X^{p-1}) \Phi(X^{p-1})
        = \\ &=
        \Phi(X^{2(p-1)})
        = \\ &=
        \Phi(X^{p+(p-2)})
        = \\ &=
        \Phi(X^p) \Phi(X^{p-2})
        \reloset {\eqref{eq:alg:finite_field_polynomial_reduction/fermat}} = \\ &=
        \Phi(X^{p-1}).
      \end{balign*}

      \item If \( r > 0 \), we have
      \begin{balign*}
        \Phi(X^s)
        &=
        \Phi(X^{(p-1)q+r})
        = \\ &=
        \Phi(X^{[(p-1)(q-1) + r] + (p-1)})
        = \\ &=
        \Phi(X^{(p-1)(q-1)+r}) \Phi(X^{p-1})
        \reloset {\T{ind.}} = \\ &=
        \Phi(X^r) \Phi(X^{p-1})
        = \\ &=
        \Phi(X^{r-1}) \Phi(X^p)
        \reloset {\eqref{eq:alg:finite_field_polynomial_reduction/fermat}} = \\ &=
        \Phi(X^{r-1}) \Phi(X)
        = \\ &=
        \Phi(X^r).
      \end{balign*}
    \end{itemize}
  \end{itemize}
\end{proof}
