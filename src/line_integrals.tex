\subsection{Line integrals}\label{subsec:line_integrals}

\begin{definition}\label{def:length_of_parametric_curve}
  \begin{figure}
    \centering
    \begin{mplibcode}
      input metapost/plotting;

      u := 4cm;

      beginfig(1)
      pair p[];

      vardef y(expr x) =
      if x < 1 / 3:
      -sqrt(1 - (2 * x * 4 / 3 - 1) ** 2) / 2
      else:
      -sqrt(1 - (x * 4 / 3 - 1 / 3) ** 2) / 2
      fi
      enddef;

      p[1] := (0, y(0)) scaled u;
      p[2] := (1 / 4, y(1 / 4)) scaled u;
      p[3] := (2 / 3, y(2 / 3)) scaled u;
      p[4] := (1, y(1)) scaled u;

      draw path_of_plot(y, 0, 1, 0.01, u);
      draw p[1] -- p[2] -- p[3] -- p[4];

      label.lft("$\gamma(0)$", p[1]);
      fill dot shifted p[1];

      label.bot("$\gamma(\tfrac 1 4)$", p[2]);
      fill dot shifted p[2];

      label.bot("$\gamma(\tfrac 2 3)$", p[3]);
      fill dot shifted p[3];

      label.rt("$\gamma(1)$", p[4]);
      fill dot shifted p[4];
      endfig;
    \end{mplibcode}
    \Caption{def:length_of_parametric_curve/approximation}{A rough approximation of a curve using three line segments.}.
  \end{figure}

  Let \( X \) be a \hyperref[def:banach_space]{Banach space} and let \( \gamma: [a, b] \to X \) be a \hyperref{def:parametric_curve}{parametric curve}. In order to find the curve's length, we use the following procedure:
  \begin{itemize}
    \item Fix a nonnegative integer \( n \).

    \item Choose \( n - 1 \) points \( c_1, \ldots, c_{n-1} \) in \( [0, 1] \) and order them ascendingly. Define \( c_0 \coloneqq 0 \) and \( c_n \coloneqq 1 \). We will call the tuple \( c \coloneqq (c_0, \ldots, c_n) \) a \term{partition} of \( [0, 1] \) because it partitions \( [0, 1] \) into the subintervals \( [c_{k-1}, c_k], k = 1, \ldots, n \).

          Note that this choice does not actually require the axiom of choice since we will universally quantify all partitions.

    \item Use the partition \( c \) to build linear approximation to \( \gamma \) using the \hyperref[def:convex_set/line_segment]{line segments} \( [\gamma(c_{k-1}), \gamma(c_k)], k = 1, \ldots, n \).

    \item Find the total length of the approximation as
          \begin{equation*}
            \len_c (\gamma) \coloneqq \sum_{k=1}^n \norm{\gamma(c_k) - \gamma(c_{k-1})}.
          \end{equation*}

    \item Build a \hyperref[def:directed_set]{directed set} of all partitions by introducing an order that depends only the size of the tuples (i.e. we declare all partitions with the same size as equal).

    \item Using the defined directed set, build a \hyperref[def:topological_net]{net} that assigns to each partition the length \( \len_c(\gamma) \). The \hyperref[def:net_convergence/limit]{limit} of this net, if it exists, is called the \term{length} of the curve \( \gamma \) and is denoted by \( \len(\gamma) \).
  \end{itemize}

  If the curve \( \gamma \) has a length, it is called \term{rectifiable}.
\end{definition}

\begin{proposition}\label{thm:length_of_smooth_curves}
  For a \hyperref[def:differentiability/frechet]{differentiable} parametric curve \( \gamma: [a, b] \to X \) we have
  \begin{equation*}
    \len(\gamma) \coloneqq \int_a^b \norm{\gamma'(x)} dx.
  \end{equation*}
\end{proposition}
\begin{proof}
  By the mean value theorem, when constructing the length in \fullref{def:length_of_parametric_curve}, for each \( k = 1, \ldots, n \) there exists a point \( \xi_k \in [c_{k-1}, c_k] \) such that
  \begin{equation*}
    \gamma(c_k) - \gamma(c_{k-1}) = \gamma'(\xi_k) (c_k - c_{k-1}).
  \end{equation*}

  Therefore
  \begin{equation*}
    \len_n (\gamma)
    =
    \sum_{k=1}^{n+1} \norm{\gamma(c_k) - \gamma(c_{k-1})}
    =
    \sum_{k=1}^{n+1} \norm{\gamma'(\xi_k)} (c_k - c_{k-1})
    \to
    \int_a^b \norm{\gamma'(x)} dx.
  \end{equation*}
\end{proof}

\begin{corollary}\label{thm:length_of_function_graph}
  The \hyperref[def:length_of_parametric_curve]{length} of the \hyperref[def:function/graph]{graph} of a \hyperref[def:differentiability/frechet]{differentiable} function \( f: [a, b] \to \BbbR \), if it exists, is given by
  \begin{equation*}
    \len(\gph(f)) \coloneqq \int_a^b \sqrt{1 + f'(x)} dx.
  \end{equation*}
\end{corollary}
\begin{proof}
  Apply \fullref{thm:length_of_smooth_curves} for the parametric curve \( \gamma(x) \coloneqq = \gph(y^+(x)) = (x, f(x)) \).
\end{proof}
