\subsection{Progressions}\label{subsec:progressions}

Progressions are an elementary concept that happens to be useful quite often. There is no definition of progression, but rather the term \enquote{progression} refers to specific recursively defined \hyperref[def:sequence]{sequences}.

\begin{definition}\label{def:arithmetic_progression}
  The \term{arithmetic progression} with \term{base} \( a_0 \) and \term{difference} \( d \) is the sequence
  \begin{equation}\label{eq:def:arithmetic_progression}
    a_k \coloneqq \begin{cases}
      a_0,         & k = 0, \\
      a_{k-1} + d, & k > 0.
    \end{cases}
  \end{equation}

  Clearly every index \( k \geq 0 \) we have the closed form representation \( a_k = a_0 + kd \).
\end{definition}

\begin{proposition}\label{thm:arithmetic_progression_partial_sums}
  The \hyperref[def:convergent_series]{series} constructed from the arithmetic progression \eqref{eq:def:arithmetic_progression} has partial sums
  \begin{equation}\label{eq:thm:arithmetic_progression_partial_sums}
    \sum_{k=0}^n a_k = \frac {(n + 1) (a_n - a_0)} 2.
  \end{equation}

  In the special case where \( a_0 = 0 \) and \( d = 1 \), this reduces to
  \begin{equation}\label{eq:thm:arithmetic_progression_partial_sums/integers}
    \sum_{k=0}^n k = \sum_{k=1}^n k = \frac {n (n + 1)} 2.
  \end{equation}
\end{proposition}
\begin{proof}
  \begin{balign*}
    2 \sum_{k=0}^n a_k
     & =
    2 \sum_{k=0}^n (a_0 + kd)
    =    \\ &=
    \sum_{k=0}^n (a_0 + kd) + \sum_{k=0}^n (a_0 + (n-k)d)
    =    \\ &=
    \sum_{k=0}^n (2 a_0 + nd)
    =    \\ &=
    (n + 1) (a_0 + a_n).
  \end{balign*}
\end{proof}

\begin{definition}\label{def:geometric_progression}
  The \term{geometric progression} with \term{base} \( a_0 \) and \term{denominator} \( q \) is the sequence
  \begin{equation}\label{eq:def:geometric_progression}
    a_k \coloneqq \begin{cases}
      a_0,       & k = 0, \\
      a_{k-1} q, & k > 0.
    \end{cases}
  \end{equation}

  Clearly every index \( k \geq 0 \) we have the closed form representation \( a_k = a_0 q^k \).

  The \hyperref[def:convergent_series]{series}
  \begin{equation}\label{eq:def:geometric_progression/series}
    \sum_{k=0}^\infty a_k = a_0 \sum_{k=0}^\infty q^k.
  \end{equation}
  is called the \term{geometric series} for \( q \). Without loss of generality, we will assume \( a_0 = 1 \) when speaking about geometric series.
\end{definition}

\begin{proposition}\label{thm:geometric_series_properties}
  The geometric series \eqref{eq:def:geometric_progression/series} has the following basic properties:
  \begin{thmenum}
    \thmitem{thm:geometric_series_properties/finite_sum} For all \( q \in \BbbC \setminus \{ 1 \} \), the geometric series \eqref{eq:def:geometric_progression/series} has partial sums
    \begin{equation}\label{thm:geometric_progression/partial_sum}
      \sum_{k=0}^n q^k = \frac {1 - q^{n+1}} {1 - q}.
    \end{equation}

    Compare this to \fullref{thm:xn_minus_one_factorization}.

    \thmitem{thm:geometric_series_properties/degenerate} In the degenerate case \( q = 1 \), the progression itself is constant, and its partial sums are instead
    \begin{equation}\label{thm:geometric_progression/degenerate}
      \sum_{k=0}^n q^k = n + 1.
    \end{equation}

    \thmitem{thm:geometric_series_properties/series_sum_exterior} For \( \abs{q} \geq 1 \), the geometric series diverges.

    \thmitem{thm:geometric_series_properties/series_sum_interior} For \( 0 < \abs{q} < 1 \), the geometric series converges absolutely with sum
    \begin{equation}\label{thm:geometric_progression/series_sum}
      \sum_{k=0}^\infty q^k = \frac 1 {1 - q}.
    \end{equation}
  \end{thmenum}
\end{proposition}
\begin{proof}
  \SubProofOf{thm:geometric_series_properties/finite_sum} Follows from \fullref{thm:xn_minus_one_factorization}.
  \SubProofOf{thm:geometric_series_properties/degenerate} Obvious.

  \SubProofOf{thm:geometric_series_properties/series_sum_exterior} For \( q = 1 \), \fullref{thm:geometric_series_properties/degenerate} implies that the series diverges because it grows indefinitely. If \( \abs{q} = 1 \) and \( q \neq 1 \), the integer powers \( q^k \) are rotations around the complex plane unit circle, which do not tend to a limit. Hence, the series diverges again.

  When \( \abs{q} > 1 \), \( \abs{q^n} \) grows indefinitely with \( n \), and it follows that
  \begin{equation*}\label{thm:geometric_progression/cauchy_partial_sum}
    \sum_{k=m}^n q^k
    =
    q^m \sum_{k=0}^{n-m} q^k
    =
    q^m \frac {1 - q^{n-m+1}} {1 - q}
    =
    \frac {q^m - q^{n+1}} {1 - q}.
  \end{equation*}
  can get arbitrarily large. Therefore, in this case the series also diverges.

  \SubProofOf{thm:geometric_series_properties/series_sum_interior} Fix \( q \in B(0, 1) \). Since only \( q^{n + 1} \) depends on \( n \) in \eqref{thm:geometric_progression/partial_sum}, we obtain \eqref{thm:geometric_progression/series_sum} by simply noting that \( q^n \to 0 \) when \( n \to \infty \).
\end{proof}

\begin{example}\label{ex:n_ary_decomposition}
  A simple but important practical example of a \hyperref[thm:geometric_series_properties/series]{geometric series} is
  \begin{equation}\label{eq:ex:n_ary_decomposition/binary}
    \sum_{k=0}^\infty \frac 1 {2^k} = \frac 1 {1 - \sfrac 1 2} = 2.
  \end{equation}

  Note that if the series starts at \( k = 1 \) instead of \( k = 0 \), it sums to \( 1 \). This is often applied in analysis indirectly via \fullref{thm:continuous_function_series_powers_of_two}.

  Another application of \eqref{eq:ex:n_ary_decomposition/binary} is showing that \( 0.\overline{1} = 2 \) in the binary number system. More generally, for the \( n \)-ary number system we have
  \begin{equation*}\label{eq:ex:n_ary_decomposition/general}
    \sum_{k=0}^\infty \parens*{ \frac {n-1} n }^k = \frac 1 {1 - \ifrac {(n-1)} n} = n.
  \end{equation*}
\end{example}

\begin{remark}\label{rem:progressions_and_interest}
  In this example we exploit the equivalence between the closed form representations in \fullref{def:arithmetic_progression} and \fullref{def:geometric_progression} and the corresponding inductive definitions. The equivalences are obvious from a mathematical standpoint, however outside of mathematics they have highly nontrivial consequences. Indeed, they highlight the difference between simple interest and compound interest.

  As an example, a savings account with \( 1000\$ \) with a simple monthly interest of \( 2\% \) will earn \( 240\$ \) over a year:
  \begin{equation*}
    1000 (1 + 12 \cdot \sfrac 2 {100}) = 1240.
  \end{equation*}

  The same account with a compound interest of \( 2\% \) will earn a bit more - about \( 268\$ \):
  \begin{equation*}
    1000 (1 + \sfrac 2 {100})^{12} \approx 1268.24.
  \end{equation*}

  Over the course of ten years, however, simple interest will earn a total of \( 2400\$ \), while compound interest will earn \( \approx 9765\$ \).

  The difference between linear and exponential growth appears staggering in a real world situation even though the difference may not be very noticeable short-term.
\end{remark}

\begin{definition}\label{def:harmonic_progression}
  The \term{harmonic progression} with \term{base} \( a_0 \) and \term{difference} \( d \) is the sequence
  \begin{equation}\label{eq:def:harmonic_progression}
    a_k \coloneqq \frac 1 {a_0 + kd}.
  \end{equation}

  That is, each term is the reciprocal of the corresponding term in an \hyperref[def:arithmetic_progression]{arithmetic progression} with the same base and difference. In order for \eqref{eq:def:harmonic_progression} to be well-defined, either
  \begin{itemize}
    \item \( d = 0 \) and \( a_0 \neq 0 \), which turns \eqref{eq:def:harmonic_progression} into the constant sequence \( \seq{ \sfrac 1 {a_0} }_{k=0}^\infty \).
    \item \( d \neq 0 \), in which case
    \begin{equation*}
      a_k = \frac d {\sfrac {a_0} d + k}.
    \end{equation*}

    Thus, if \( d \neq 0 \), \( \ifrac {a_0} d \) must not be a negative integer unless we are satisfied with only the first \( -\ifrac {a_0} d \) terms of the progression existing.
  \end{itemize}

  Furthermore, the series may only start at \( k = 0 \) if \( a_0 \neq 0 \).

  \begin{thmenum}
    \thmitem{def:harmonic_progression/harmonic_series} In the special case where \( d = 1 \) and \( a_0 = 0 \) (so the progression starts at \( k = 1 \)), the series
    \begin{equation}\label{eq:def:harmonic_progression/harmonic_series}
      \sum_{k=1}^\infty \frac 1 k = 1 + \frac 1 2 + \frac 1 3 + \frac 1 4 + \cdots.
    \end{equation}
    is called \hi{the} \term{harmonic series}. It diverges as shown in \fullref{thm:harmonic_series_diverges}, which make it much less useful in practice, however it is an important enough example that it has a dedicated name.

    \thmitem{def:harmonic_progression/hyperharmonic_series} For any \( s \in \BbbC \), the series
    \begin{equation}\label{eq:def:harmonic_progression/hyperharmonic_series}
      \sum_{k=1}^\infty \frac 1 {k^s}.
    \end{equation}
    is called the \term{hyperharmonic series}.

    Unlike the harmonic series, the hyperharmonic series sometimes converges --- see \fullref{thm:hyperharmonic_series_convergence}.
  \end{thmenum}
\end{definition}

\begin{remark}\label{rem:harmonic_progression_recursive_form}
  Unlike \fullref{def:arithmetic_progression} and \fullref{def:geometric_progression}, we have defined the harmonic progressions via closed-form expressions. Indeed, the equivalent inductive definition is more awkward to work with:
  \begin{equation*}
    a_k \coloneqq \begin{cases}
      \ifrac 1 {a_0},                    & k = 0, \text{ only defined if } a_0 \neq 0, \\
      \ifrac 1 {(a_0 + d)},                & k = 1,                                      \\
      \ifrac 1 {(\sfrac 1 {a_{k-1}} + d)}, & k > 0.
    \end{cases}
  \end{equation*}
\end{remark}

\begin{proposition}\label{thm:harmonic_series_diverges}
  The harmonic series \eqref{eq:def:harmonic_progression/harmonic_series} diverges.
\end{proposition}
\begin{proof}
  Define the series
  \begin{equation*}
    1 + \frac 1 2 + \underbrace{\frac 1 4 + \frac 1 4}_{\sfrac 1 2} + \underbrace{\frac 1 8 + \frac 1 8 + \frac 1 8 + \frac 1 8}_{\sfrac 1 2} + \underbrace{\frac 1 {16} + \cdots + \frac 1 {16}}_{\sfrac 1 2} + \cdots.
  \end{equation*}

  It is divergent as the sum of infinitely many \( \sfrac 1 2 \). Furthermore, it is dominated by the harmonic series:
  \begin{align*}
    &1 + \frac 1 2 + \frac 1 3 + \frac 1 4 + \frac 1 5 + \frac 1 6 + \frac 1 7 + \frac 1 8 + \frac 1 9 \thinspace + \cdots + \frac 1 {16} + \cdots
    \\
    &1 + \frac 1 2 + \overbrace{ \frac 1 4 + \frac 1 4 }^{\sfrac 1 2} + \overbrace{ \frac 1 8 + \frac 1 8 + \frac 1 8 + \frac 1 8 }^{\sfrac 1 2} + \overbrace{ \frac 1 {16} + \cdots + \frac 1 {16} }^{\sfrac 1 2} + \cdots
  \end{align*}

  Thus, by \fullref{thm:positive_series_comparison}, the harmonic series also diverges.
\end{proof}

\begin{proposition}\label{thm:hyperharmonic_series_convergence}
  The hyperharmonic series \eqref{eq:def:harmonic_progression/hyperharmonic_series} converges for real \( s > 1 \).
\end{proposition}
\begin{proof}
  We use the integral test:
  \begin{equation*}
    \int_1^\infty \frac 1 {x^{1 + \varepsilon}} \dl x
    =
    -\frac 1 {\varepsilon x^\varepsilon}\restr_{x=1}^\infty
    =
    \lim_{x \to \infty} \frac 1 \varepsilon \parens*{ 1 - \frac 1 {x^\varepsilon} }
    =
    \frac 1 \varepsilon.
  \end{equation*}
\end{proof}
