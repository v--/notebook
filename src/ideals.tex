\subsection{Ideals}\label{subsec:ideals}

\begin{definition}\label{def:magma_ideal}
  Let \( M \) be a magma\Tinyref{def:magma/magma} and \( I \) be a subset of \( M \). We say that \( I \) is a \Def{left ideal} (resp. \Def{right ideal}) of \( G \) if the inclusion \( IM \subseteq I \) holds, where we use the convention in \cref{remark:vector_space_set_operations}, that is,
  \begin{equation*}
    IM = \{ x \cdot i \;\colon\; x \in M, i \in I \}.
  \end{equation*}

  If \( I \) is both a left ideal and a right ideal, we say that it is a \Def{two-sided ideal}.
\end{definition}

\begin{proposition}\label{thm:magma_ideal_is_submagma}
  Every two-sided magma ideal is a submagma.
\end{proposition}
\begin{proof}
  Let \( I \) be a two-sided ideal for the magma \( M \). For \( i, j \in I \), since \( I \) is a left ideal, we have \( ij \in I \) and similarly \( ji \in I \) since \( I \) is a right ideal. Thus \( II = I \) and \( I \) is a submagma of \( M \).
\end{proof}

\begin{example}\label{ex:subgroup_is_not_ideal}
  We explicitly give a counterexample to the converse of \cref{thm:magma_ideal_is_submagma}. Define \( G \coloneqq \Z \times \Z \) to be the abelian group\Tinyref{def:magma/abelian_group} given by pointwise addition. Define
  \begin{equation*}
    G' \coloneqq \{ (n, n) \colon n \in \Z \}.
  \end{equation*}

  The set \( G' \) is a subgroup of \( G \) since it is closed under addition and it contains the unit element \( (0, 0) \). It is not an ideal, however, since
  \begin{equation*}
    (n, n) + (n, 0) = (2n, n) \not\in G'.
  \end{equation*}
\end{example}

\begin{proposition}\label{thm:unital_magma_ideal_is_submagma_iff_contains_identity}
  A two-sided ideal of a unital magma is a unital submagma if and only if it contains the identity.
\end{proposition}
\begin{proof}
  Follows from \cref{thm:magma_ideal_is_submagma} and \cref{thm:proper_ideals_containing_identity}.
\end{proof}

\begin{proposition}\label{thm:proper_ideals_containing_identity}
  A left or right ideal of a unital magma contains the identity if and only if it is not proper.
\end{proposition}
\begin{proof}
  Let \( M \) be a unital magma and \( I \) be a left ideal of \( M \). We will prove that \( e \in I \iff I = M \).

  \begin{description}
    \Implies Let \( e \in I \). Then \( ex = x \) for any \( x \in M \), thus \( IM = M \). But \( I \) is an ideal, hence we have that \( IM = I \), thus \( I = IM = M \).
    \ImpliedBy If \( I = M \), then obviously \( e \in M = R \).
  \end{description}

  An analogous proof follows for the case when \( I \) is a right ideal.
\end{proof}

\begin{proposition}\label{thm:commutative_magma_ideals}
  In a commutative\Tinyref{def:algebraic_theory/commutativity} magma\Tinyref{def:magma/magma} \( M \), a subset \( I \subseteq M \) is a left ideal if and only if it is a right ideal. That is, in commutative magmas, it makes no sense to distinguish between left, right and two-sided ideals.
\end{proposition}
\begin{proof}
  For \( i \in I \) and \( x \in M \) by commutativity we have \( ix = xi \), thus \( I \) is a left ideal if and only if it is a right ideal.
\end{proof}

\begin{definition}\label{def:semiring_ideal}
  Let \( R \) be a semiring and \( I \) be a subset of \( R \) (not necessarily a subsemiring). We say that \( I \) is an \Def{ideal} (left, right or two-sided) if \( (I, +) \) is a subgroup of \( (R, +) \) and \( (I, \cdot) \) is a magma ideal\Tinyref{def:magma_ideal} of \( (R, \cdot) \).
\end{definition}

\begin{proposition}\label{thm:semiring_ideal_is_nonunital_subsemiring}
  Two-sided semiring ideals are subsemirings. In the special case where the semiring is unital, an ideal is a unital subsemiring if and only if it is not a proper ideal.
\end{proposition}
\begin{proof}
  Follows from \cref{thm:magma_ideal_is_submagma} and \cref{thm:unital_magma_ideal_is_submagma_iff_contains_identity}.
\end{proof}

\begin{definition}\label{thm:semiring_ideal_iff_kernel}
  A subset of a semiring is a two-sided ideal\Tinyref{def:semiring_ideal} if and only if it is the kernel\Tinyref{def:semiring_kernel} of some ring homomorphism.
\end{definition}
\begin{proof}
  \Implies Let \( I \) be a two-sided ideal. Since it is an abelian group, \( I \) is a normal subgroup and thus we can form the quotient group\Tinyref{def:normal_subgroup} \( R / I \) with the canonical projection
  \begin{align*}
    &\pi: R \to R / I \\
    &\pi(x) \coloneqq x + I
  \end{align*}

  Multiplication in \( R \) induces multiplication in \( R / I \) by
  \begin{equation*}
    (x + I) \cdot (y + I) \coloneqq (xy + I).
  \end{equation*}

  It is well defined since if \( x + I = x' + I \) and \( y + I = y' + I \), then
  \begin{align*}
    (x + I) (y + I)
    &=
    xy + (Iy + xI + II)
    = \\ &=
    xy + I
    = \\ &=
    x'y' + I
    = \\ &=
    x'y' + (Iy' + x'I + II)
    = \\ &=
    (x' + I) (y' + I).
  \end{align*}

  Thus the ring structure on \( R \) induces a ring structure on \( R / I \).

  The canonical projection \( \pi \) is an additive group homomorphism. Since we just showed that \( \pi(xy) = \pi(x) \pi(y) \), it follows that it is also a ring homomorphism.

  It only remains to show that \( \ker(\pi) = I \). Since \( I \) is closed under addition, naturally \( I \subseteq \ker(\pi) \). Conversely, if \( x \in \ker(\pi) \), then \( \pi(x) = \pi(0) = I \), i.e. \( x \in I \). Hence \( \ker(\pi) = I \).

  \ImpliedBy Let \( f: R \to T \) is a ring homomorphism. We must show that \( \ker(f) \) is an ideal. If \( i \in \ker(f) \) and \( x \in R \), then
  \begin{equation*}
    f(ix) = f(i) f(x) = 0 f(x) = 0.
  \end{equation*}

  Thus \( ix \in \ker(f) \). Similarly, we can show that \( xi \in \ker(f) \). Thus \( R \ker(f) = \ker(f) R = \ker(f) \) and \( \ker(f) \) is a two-sided ideal.
\end{proof}

\begin{definition}\label{def:generated_ring_ideal}
  Let \( R \) be a commutative ring so that left and right ideals coincide. Let \( S \subseteq R \) be any nonempty subset of \( R \). We define the ideal generated by \( S \) equivalently as either
  \begin{defenum}
    \DItem{def:generated_ring_ideal/minimal} the smallest ideal of \( R \) that contains \( S \).
    \DItem{def:generated_ring_ideal/direct} the ideal
    \begin{equation*}
      \Gen S \coloneqq \left\{ \Prod {\vec r} {\vec s} \mid \vec r \in R^n, \vec s \in S^n, n \in \Z^{>0} \right\}
    \end{equation*}
    of finite linear combinations.

    \DItem{def:generated_ring_ideal/polynomials} the ideal
    \begin{equation*}
      \Gen S \coloneqq \left\{ p(s_1, s_2, \ldots, s_n) \mid s_1, \ldots, s_n \in S, p \in R[X_1, \ldots, X_n], n \in \Z^{>0} \right\}
    \end{equation*}
  \end{defenum}

  If \( S \) is finite, then \( \Gen S \) is called \Def{finitely generated}. If \( S = \{ s_1, \ldots, s_n \} \), then
  \begin{equation*}
    \Gen S = s_1 R + s_2 R + \cdots s_n R.
  \end{equation*}
\end{definition}

\begin{definition}\label{def:principal_ideal}
  If an ideal \( I \) is generated\Tinyref{def:generated_ring_ideal} by a single element, it is called a \Def{principal ideal}.
\end{definition}

\begin{proposition}\label{thm:product_of_principal_ideals}
  In a commutative unital ring \( R \) the product of the principal ideals \( \Gen{x} \) and \( \Gen{y} \) is \( \Gen{xy} \).
\end{proposition}

\begin{definition}\label{def:prime_ring_ideal}\cite[384]{Knapp2016BAlg}
  An ideal \( P \) in a commutative unital ring \( R \) is called \Def{prime} if it is proper and satisfies any of the equivalent conditions:
  \begin{defenum}
    \DItem{def:prime_ring_ideal/direct} If \( x, y \in R \) are such that \( xy \in P \), then either \( x \in P \) or \( y \in P \).
    \DItem{def:prime_ring_ideal/ideals} If \( I, J \subseteq R \) are ideals such that \( IJ \subseteq P \), then either \( I \subseteq P \) or \( J \subseteq P \).
    \DItem{def:prime_ring_ideal/quotient} The quotient \( R / P \) is an integral domain.
  \end{defenum}

  An element \( r \in R \) is called \Def{prime} if the ideal \( \Gen r \) is prime.
\end{definition}
\begin{proof}
  \Implies[def:prime_ring_ideal/direct][def:prime_ring_ideal/ideals] Fix ideals \( I, J \) of \( R \) such that \( IJ \subseteq P \).

  Assume\LEM that neither \( I \not\subseteq P \) nor \( J \not\subseteq P \). Take \( x \in I \setminus P \) and \( y \in J \setminus P \). It follows that \( xy \in P \) and either \( x \in P \) or \( y \in P \). This contradicts our assumption.

  The obtained contradiction proves that either \( I \subseteq P \) or \( J \subseteq P \).

  \Implies[def:prime_ring_ideal/ideals][def:prime_ring_ideal/quotient] Fix an ideal \( P \) such that if \( I, J \subseteq R \) are ideals and \( IJ \subseteq P \), then either \( I \subseteq P \) or \( J \subseteq P \).

  We will prove that \( R / P \) is an integral domain. If \( R \) is an integral domain, this is obvious. If not, we fix nonzero \( x, y \in R \) so that \( xy = 0 \). Thus \( [x][y] = (x + P)(y + P) = xy + P = P = [0] \). We will show that either \( x = 0 \) or \( y = 0 \).

  Consider the ideals
  \begin{align*}
    \Gen{x} &= xR, \\
    \Gen{y} &= yR.
  \end{align*}

  By \cref{thm:product_of_principal_ideals}, we have \( \Gen{x} \Gen{y} = \Gen{xy} = \Gen{0} = \{ 0 \} \).

  Since \( \Gen{x} \Gen{y} \subseteq P \), then either \( \Gen{x} \subseteq P \) or \( \Gen{y} \subseteq P \). That is, either \( [x] = 0 \) or \( [y] = 0 \).

  Thus \( R / P \) is an integral domain.

  \Implies[def:prime_ring_ideal/quotient][def:prime_ring_ideal/direct] Suppose that \( R / P \) is an integral domain. Fix \( x, y \in R \) so that \( xy \in P \). If \( x = 0 \), obviously \( x = 0 \in R \) and similarly for \( y \). Suppose that both \( x \) and \( y \) are nonzero. We will show that either \( x \in P \) or \( y \in P \).

  We have
  \begin{equation*}
    [x][y] = [xy] = xy + P = P = [0].
  \end{equation*}

  Since \( R / P \) is an integral domain, either \( [x] = [0] \) or \( [y] = [0] \). That is, either \( x \in P \) or \( y \in P \).
\end{proof}

\begin{definition}\label{def:irreducible_ring_element}\cite[388]{Knapp2016BAlg}
  A nonzero element \( r \in R \) of an integral domain is called \Def{reducible} if there exist non-invertible elements \( r_1, r_2 \in R \) such that
  \begin{equation*}
    r = r_1 r_2.
  \end{equation*}

  If \( r \) is not reducible, we say that it is\Def{irreducible}.
\end{definition}

\begin{definition}\label{def:coprime_ring_ideal}
  Two ring ideals \( I \subseteq R \) and \( J \subseteq R \) are said to be \Def{coprime} if \( I + J = R \).
\end{definition}

\begin{proposition}\label{thm:prime_implies_irreducible}\cite[389]{Knapp2016BAlg}
  All prime\Tinyref{def:prime_ring_ideal} elements in an integral domain are irreducible\Tinyref{def:irreducible_ring_element}.
\end{proposition}
\begin{proof}
  Let \( p \) be prime. Assume\LEM that \( p \) is reducible, that is, there exist non-invertible elements \( r_1, r_2 \in R \) such that
  \begin{equation*}
    p = r_1 r_2.
  \end{equation*}

  Since \( p \) is prime, it must divide either \( r_1 \) or \( r_2 \). Without loss of generality, assume that \( p | r_1 \) and \( r_1 = pc \) for some \( c \in R \).

  Then \( p = r_1 r_2 = pc r_2 \). By \cref{thm:semiring_properties/cancellable_iff_not_zero_divisor}, \( 1 = c r_2 \), which implies that \( r_2 \) is invertible with inverse \( c \). This contradicts our assumption that both \( r_1 \) and \( r_2 \) are invertible.

  The obtained contradiction proves that \( p \) is irreducible.
\end{proof}

\begin{definition}\label{def:maximal_ring_ideal}
  A two-sided ideal \( M \) in a commutative unital ring \( R \) is called \Def{maximal} if it is proper and satisfies any of the equivalent conditions:
  \begin{defenum}
    \DItem{def:maximal_ring_ideal/maximality} \( M \) is maximal with respect to set inclusion among proper two-sided ideals.
    \DItem{def:maximal_ring_ideal/quotient} The quotient \( R / M \) is a field.
  \end{defenum}
\end{definition}
\begin{proof}
  \Implies[def:maximal_ring_ideal/maximality][def:maximal_ring_ideal/quotient] Suppose that \( M \) is maximal among proper ideals. We will prove that every nonzero element of \( R / M \) is invertible.

  Fix \( x \not\in M \) so that \( [x] = x + M \neq M = [0] \). Define the set
  \begin{equation*}
    I \coloneqq Rx + M.
  \end{equation*}

  It is a ideal since both \( Rx \) and \( M \) are ideals. Furthermore, it contains \( M \) strictly because \( M \subseteq I \) and \( x \in I \). Since \( M \) is maximal, we have that \( I = R \).

  Hence there exists \( y \in R \) such that \( 1 = yx + M \). Hence \( [y] = y + M \) is an inverse of \( [x] \) in \( R / M \).

  Since \( [x] \in R / M \) was an arbitrary nonzero element, we conclude that \( R / M \) is a field.

  \Implies[def:maximal_ring_ideal/quotient][def:maximal_ring_ideal/maximality] Suppose that \( R / M \) is a field. Assume that \( M \) is not maximal. Then there exists a proper ideal \( I \supsetneq M \).

  Assume that \( I \neq M \) and take \( x \in I \setminus M \). Then \( x \not\in M \) and hence \( [x] \neq [0] \) and is invertible in \( R / M \). Denote by \( y \) any representative of this inverse. Thus \( [xy] - [1] = [0] \), that is, \( xy - 1 \in M \).

  Note that \( xy \in I \) because \( x \in I \) and \( y \in R \). Since \( I \) is closed under addition, it follows that \( 1 \in I \) and hence \( I = R \). But this contradicts our assumption that \( I \) is proper.

  The obtained contradiction proves that \( M \) is maximal.
\end{proof}

\begin{proposition}\label{thm:maximal_ideals_are_prime}
  Maximal ring ideals\Tinyref{def:maximal_ring_ideal} are prime\Tinyref{def:prime_ring_ideal}.
\end{proposition}
\begin{proof}
  If \( M \) is a maximal ideal of \( R \), by \cref{def:maximal_ring_ideal/quotient} \( R / M \) is a field. Thus \( R / M \) is an integral domain, which by \cref{def:prime_ring_ideal/quotient} means that \( M \) is a prime ideal.
\end{proof}

\begin{proposition}\label{thm:pid_prime_iff_irreducible}
  An element in a principal ideal domain is prime\Tinyref{def:prime_ring_ideal} if and only if it is irreducible\Tinyref{def:irreducible_ring_element}.
\end{proposition}
\begin{proof}
  \begin{description}
    \Implies Follows from \cref{thm:prime_implies_irreducible}.

    \ImpliedBy Let \( r \) be an irreducible element. Assume\LEM that \( \Gen r \) is not a maximal ideal. Then it is contained an another proper ideal, say \( I \).

    Since \( R \) is a principal ideal domain, \( I \) can be generated by only one element, say \( I = \Gen p \). Since \( r \in I \), there exists \( q \in R \) such that
    \begin{equation*}
      r = qp.
    \end{equation*}

    But \( r \) is irreducible, hence either \( p \) or \( q \) must be invertible. \( p \) cannot be invertible since \( I = \Gen p \) is a proper ideal. Thus \( q \) is invertible. Thus
    \begin{equation*}
      p = q^{-1} r
    \end{equation*}
    and \( p \in \Gen r \), that is, \( \Gen r = \Gen p = I \). This contradicts our assumption that \( \Gen r \) is not maximal.

    The obtained contradiction shows that \( \Gen r \) is a maximal, and therefore by \cref{thm:maximal_ideals_are_prime} prime, ideal.
  \end{description}
\end{proof}
