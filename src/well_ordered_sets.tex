\subsection{Well-ordered sets}\label{subsec:well_ordered_sets}

This section would perhaps fit in \fullref{sec:order_theory}, however it deals with well-foundedness of set membership and with induction principles which are more fitting here.

\begin{definition}\label{def:well_founded_relation}
  Let \( \prec \) be a \hyperref[def:binary_relation]{binary relation} on a set \( A \) (not necessarily satisfying any axioms).

  An \term{infinitely descending sequence} is a \hyperref[def:sequence]{sequence} \( \seq{ x_k }_{k=1}^\infty \) such that \( x_{k+1} \prec x_k \) for all \( k \in 1, 2, \ldots \). That is,
  \begin{equation*}
    \cdots \prec x_3 \prec x_2 \prec x_1.
  \end{equation*}

  The relation \( \prec \) is called \term{well-founded} if there exists no infinitely descending sequence.

  We cannot easily formulate the theory of well-founded relations as a \hyperref[def:first_order_theory]{first-order theory}. This is why we only study well-relations sets in the context of \logic{ZFC}.
\end{definition}

\begin{proposition}\label{thm:set_membership_is_well_founded}
  Set membership is \hyperref[def:well_founded_relation]{well-founded} in \logic{ZFC}. More precisely, given a set \( A \), if we regard \( \in \) as a binary relation between members of \( A \), then we would obtain that \( \in \) is a well-founded relation.

  This proposition generalizes \fullref{thm:simple_foundation_theorems}.
\end{proposition}
\begin{proof}
  The empty set is vacuously well-founded, so suppose that \( A \) is nonempty. Suppose also that \( \in \) is not well-founded on \( A \). Then there exists an infinitely descending sequence \( \seq{ x_k }_{k=1}^\infty \subseteq A \) such that
  \begin{equation*}
    \cdots \in x_3 \in x_2 \in x_1.
  \end{equation*}

  Denote by \( B \) the set \( \set{ x_k \given k = 1, 2, \ldots } \). By the \hyperref[def:zfc/foundation]{axiom of foundation}, \( B \) contains a set \( C \) which is disjoint from \( B \).

  Clearly \( C \) coincides with at least one of \( x_1, x_2, \ldots \). Let \( C = x_{k_0} \). Since \( x_{k_0} \cap B = \varnothing \), then \( x_{k_0 + 1} \) is either not a member of \( x_{k_0} \) or of \( B \). But it is a member of both by our assumption that the sequence is infinitely descending.

  The obtained contradiction proves that \( \in \) is well-founded on \( A \).
\end{proof}

\begin{proposition}\label{thm:infinite_descent_partial_order}
  In a \hyperref[def:partially_ordered_set]{partially ordered set} \( (\mscrP, \leq) \), the strict relation \( < \) is \hyperref[def:well_founded_relation]{well-founded} if and only if every nonempty subset \( A \subseteq \mscrP \) has a \hyperref[def:partially_ordered_set_extremal_points/maximal_and_minimal_element]{minimal element}.
\end{proposition}
\begin{proof}
  \SufficiencySubProof Suppose that \( < \) is well-founded and let \( A \subseteq \mscrP \). Suppose that \( A \) has no minimal element.

  If \( A \) is finite, we cannot construct an infinitely descending sequence because of irreflexivity of \( < \). If \( A \) is infinite, then we can construct an infinitely descending sequence using well-founded recursion as follows:
  \begin{itemize}
    \item Let \( x_1 \) be any element of \( A \).
    \item Given \( x_k \), let \( x_{k+1} \) be any element of \( A \) such that \( x_{k+1} < x_k \). We know that such an element exists because \( x_k \) is not minimal in \( A \).
  \end{itemize}

  This construction contradicts the well-foundedness of  \( \leq \), hence \( A \) must have a minimal element.

  \NecessitySubProof Suppose that every subset of \( \mscrP \) has a minimal element. Suppose that there exists an infinitely descending sequence \( \seq{ x_k }_{k=1}^\infty \). Then the set \( \set{ x_k \given x \in \set{ 1, 2, \ldots } } \) has no minimal element, which contradicts our assumption. Hence, no sequence in \( \mscrP \) is infinitely descending and \( < \) is well-founded.
\end{proof}

\begin{definition}\label{def:well_ordered_set}\mcite[def. 63.1]{OpenLogicFull}
  A \hyperref[def:totally_ordered_set]{totally ordered set} \( (\mscrP, \leq) \) is said to be \term{well-ordered} if either of the following equivalent conditions hold:
  \begin{thmenum}
    \thmitem{def:well_ordered_set/direct} Every nonempty subset of \( \mscrP \) has a \hyperref[def:partially_ordered_set_extremal_points/maximum_and_minimum]{minimum}.
    \thmitem{def:well_ordered_set/well_founded} The strict order \( < \) is \hyperref[def:well_founded_relation]{well-founded}.
  \end{thmenum}

  The \hyperref[def:binary_relation/irreflexive]{irreflexivity} of \( < \) is redundant because it follows from the well-foundedness --- if the strict relation is not irreflexive, then there exists some element \( x \in \mscrP \) such that \( x < x \) and thus \( \seq{ x }_{k=1}^\infty \) is an infinitely descending sequence.
\end{definition}
\begin{proof}
  The equivalence of the conditions follows from \fullref{thm:infinite_descent_partial_order} and \fullref{thm:totally_ordered_minimal_element_is_minimum}.
\end{proof}

\begin{theorem}[Well-founded induction]\label{thm:well_founded_induction}\mcite[prop. 63.3]{OpenLogicFull}
  Let \( \mscrL \) be the \hyperref[def:first_order_syntax]{first-order language} with no functional symbols and a single predicate symbol \( \prec \). We have already mentioned that we cannot formalize the concept of well-foundedness in first-order logic alone, so we will work with \hyperref[def:first_order_structure]{structures} directly.

  Every \hyperref[def:well_founded]{well-founded} structure \( \mscrX = (A, I) \) over \( \mscrL \) satisfies an inductive axiom schema. For every formula \( \varphi \) over \( \mscrL \) not containing \( \eta \) nor \( \zeta \) as free variables, \( \mscrX \) satisfies
  \begin{equation}\label{eq:thm:well_founded_induction}
    \qforall \eta
    \parens[\Big]
      {
        \overbrace
          {
            \underbrace{ \parens[\Big]{ \qforall {\zeta \prec \eta} \varphi[\xi \mapsto \zeta] } }_{\mathclap{\substack{\T{inductive} \\ \T{hypothesis}}}}
            \rightarrow
            \underbrace{ \varphi[\xi \mapsto \eta] }_{\mathclap{\substack{\T{inductive step} \\ \T{conclusion}}}}
          }^{\T{inductive step}}
      }
    \rightarrow
    \underbrace{ \qforall \eta \varphi[\xi \mapsto \eta] }_{\T{conclusion}}.
  \end{equation}

  See the comments in \fullref{def:peano_arithmetic/PA3} regarding variables and quantification in axiom schemas and \fullref{rem:induction} for a general discussion of induction.

  In the special case where \( \mscrX = \BbbN \), this is called \term{strong induction} compared to the usual natural number induction \eqref{eq:def:peano_arithmetic/PA3}. This is discussed in \fullref{rem:induction/well_founded}.
\end{theorem}
\begin{proof}
  Fix a formula \( \varphi \) in \( \mscrL \). Fix a variable assignment \( v \) in \( \mscrX \). We will show that the \hyperref[def:material_implication/contrapositive]{contraposition} of \eqref{eq:thm:well_founded_induction} holds in this model.

  Suppose that there exists a value \( x \in A \) such that \( \varphi\Bracks{v_{\xi \mapsto x}} = F \). That is,
  \begin{equation*}
    \qexists \eta \neg \varphi[\xi \mapsto \eta].
  \end{equation*}
  holds.

  Let \( x_0 \) be the smallest such value (which is guaranteed to exist because \( A \) is well-founded by \( \prec \)). Thus, the inductive hypothesis \( \qforall {\zeta < x_0} \varphi[\xi \mapsto \zeta] \) holds, but the inductive step conclusion \( \varphi[\xi \mapsto x_0] \) does not.

  Since \( v \) was chosen arbitrarily, this is true for all variable assignments in \( \mscrX \). Formally,
  \begin{equation}\label{eq:thm:well_founded_induction/contraposition}
    \mscrX
    \vDash
    \qexists \eta \neg \varphi[\xi \mapsto \eta]
    \rightarrow
    \qexists \eta
    \parens[\Big]
      {
        \qforall {\zeta \prec \eta} \varphi[\xi \mapsto \zeta] \wedge \neg \varphi[\xi \mapsto \eta]
      }
  \end{equation}

  This is precisely the contrapositive of \eqref{eq:thm:well_founded_induction}. Since we are working in classical logic, the contrapositive is semantically equivalent to its original implication, hence \( \mscrX \vDash \eqref{eq:thm:well_founded_induction} \).
\end{proof}

\begin{theorem}[Epsilon induction]\label{thm:epsilon_induction}
  For every formula \( \varphi \) in the language of set theory not containing \( \eta \) nor \( \zeta \) as free variables, the following is a theorem of \logic{ZFC}:
  \begin{equation*}
    \qforall \eta
    \parens[\Big]
      {
        \overbrace
          {
            \underbrace{ \parens[\Big]{ \qforall {\zeta \in \eta} \varphi[\xi \mapsto \zeta] } }_{\mathclap{\substack{\T{inductive} \\ \T{hypothesis}}}}
            \rightarrow
            \underbrace{ \varphi[\xi \mapsto \eta] }_{\mathclap{\substack{\T{inductive step} \\ \T{conclusion}}}}
          }^{\T{inductive step}}
      }
    \rightarrow
    \underbrace{ \qforall \eta \varphi[\xi \mapsto \eta] }_{\T{conclusion}}.
  \end{equation*}

  This induction schema is called \enquote{\( \varepsilon \)-induction} because the set membership symbol \( \in \) is derived from \( \varepsilon \) as explained in \fullref{rem:epsilon_and_set_membership}.

  See the comments in \fullref{def:peano_arithmetic/PA3} regarding variables and quantification in axiom schemas and \fullref{rem:induction} for a general discussion of induction.
\end{theorem}
\begin{proof}
  Every model of \logic{ZFC} is well-founded by \( \in \) due to \fullref{thm:set_membership_is_well_founded}. The corollary then follows from \fullref{thm:well_founded_induction}.
\end{proof}

\begin{lemma}\label{thm:well_ordered_embedding_inflationary}
  Any \hyperref[def:partially_ordered_set/homomorphism]{order embedding} on a well-ordered set is \hyperref[def:inflationary_function]{inflationary}.

  That is, if \( (\mscrP, \leq) \) is a well-ordered set and \( f: \mscrP \to \mscrP \) is an order embedding, then \( x \leq f(x) \) for any \( x \in \mscrP \).
\end{lemma}
\begin{proof}
  We proceed by induction on \( \mscrP \). Fix \( x_0 \) and suppose that \( y \leq f(y) \) for all \( y < x_0 \).

  Aiming at a contradiction, suppose that \( f(x_0) < x_0 \). Thus, there exists some \( y_0 < x_0 \) such that \( f(x_0) = y_0 \). By the inductive hypothesis we have \( f(x_0) = y_0 \leq f(y_0) \). Thus, either \( f(x_0) < f(y_0) \), which contradicts that \( f \) is an order homomorphism, or \( f(x_0) = f(y_0) \), which contradicts the injectivity of \( f \).

  The obtained contradiction demonstrates that \( x \leq f(x) \) for all \( x \in \mscrP \).
\end{proof}

\begin{proposition}\label{thm:well_ordered_isomorphism_is_unique}
  There is at most one isomorphism between any pair of well-ordered sets.
\end{proposition}
\begin{proof}
  Fix two well-ordered sets \( (\mscrP, \leq_\mscrP) \) and \( (\mscrQ, \leq_\mscrQ) \) and let \( f: \mscrP \to \mscrQ \) and \( g: \mscrP \to \mscrQ \) be two order isomorphisms.

  For any member \( x \) of \( \mscrP \) we have the following possibilities:
  \begin{itemize}
    \item If \( f(x) <_\mscrQ g(x) \), then \( g^{-1}(f(x)) <_\mscrP x \) and \( f(g^{-1}(f(x))) <_\mscrQ f(x) \). Then we can use \hyperref[rem:natural_number_recursion]{natural number recursion} to build an infinitely descending sequence of members of \( \mscrQ \). But this is a contradiction because \( \mscrQ \) is well-founded.

    \item We can build a similar sequence if \( g(x) <_\mscrQ f(x) \).

    \item It remains for \( f(x) = g(x) \) to hold.
  \end{itemize}

  Since \( x \in \mscrP \) was arbitrary, we conclude that \( f = g \).
\end{proof}

\begin{proposition}\label{thm:well_ordered_lexicographic_order_is_well_ordered}
  If \( (\mscrP, \leq_\mscrP) \) and \( (\mscrQ, \leq_\mscrQ) \) are \hyperref[def:well_ordered_set]{well-ordered sets}, then the \hyperref[eq:def:lexicographic_order]{lexicographic} and \hyperref[eq:def:lexicographic_order/reverse]{reverse lexicographic} orders on \( \mscrP \times \mscrQ \) are well-ordering relations.

  Compare this result to \fullref{thm:lexicographic_order_is_partial_order} and \fullref{thm:total_ordered_lexicographic_order_is_total}.
\end{proposition}
\begin{proof}
  We have already shown in \fullref{thm:total_ordered_lexicographic_order_is_total} and these total orders. It only remains to check well-foundedness.

  \SubProofOf[def:well_founded_relation]{well-foundedness} Let \( \prec \) be the lexicographic order on \( \mscrP \times \mscrQ \).

  Suppose that there exists an infinitely descending sequence
  \begin{equation*}
    \cdots \prec (a_3, b_3) \prec (a_2, b_2) \prec (a_1, b_1).
  \end{equation*}

  Since \( \mscrP \) is well-founded, the corresponding sequence \( \seq{ a_k }_{k=1}^\infty \). Then there exists an index \( k_0 \) such that \( a_{k_0} = a_k \) for \( k \geq k_0 \). Therefore, the sequence \( \set{ b_k }_{k=k_0}^\infty \) must be infinitely descending. But this contradicts the well-foundedness of \( \mscrQ \).

  Therefore, no infinitely descending sequence exists in the totally ordered set \( (\mscrP \times \mscrQ, \prec) \) and thus it is well-ordered.
\end{proof}
