\section{Set theory}\label{sec:set_theory}

Sets are ubiquitous in mathematics yet the theory if sets is far from simple. Attention needs to be put to give consistent definitions, especially regarding classes that are used, for example, in the \hyperref[def:category]{definition of a category}.

We first use the simplicity of \hyperref[def:naive_set_theory]{na\"ive set theory} to introduce some fundamental definitions so that the more sophisticated \hyperref[def:zfc]{Zermelo-Fraenkel set theory} can be easily formulated.

We then proceed to define the \hyperref[def:category_of_sets]{category of sets} and to discuss the controversial axiom of choice with all of its incarnations given in \fullref{thm:axiom_of_choice_equivalences}.

Finally, we discuss \hyperref[def:ordinal]{ordinals} which we use for transfinite induction and recursion and \hyperref[def:cardinal]{cardinals} which we use in many definitions like \hyperref[def:topological_space_character]{character of a topological space} or the \hyperref[def:poset_chain_and_antichain]{width and height of a poset}.

\begin{remark}\label{rem:set_definition_recursion}
  The relation between \hyperref[subsec:first_order_logic]{first-order logic} and set theory is remarkable. On one hand, set theory studies axiomatically defined sets and the relations between axioms such as \hyperref[def:zfc]{\logic{ZFC}} by utilizing first-order logic, among other things.

  On the other hand, the \hyperref[def:first_order_semantics]{classical first-order semantical framework} is defined in terms of \hyperref[def:first_order_structure]{first-order structures}, which in turn are defined in terms of sets.

  We do the following:
  \begin{itemize}
    \item We build certain large sets within set theory that can be used as models of \logic{ZFC}.
    \item Given such a set, we can define it as \enquote{the} model of \logic{ZFC} we are interested in.
    \item Use this model within our \hyperref[rem:metalogic]{metalogical framework} for definitions such as \fullref{def:language}, which are used to define \hyperref[subsec:first_order_logic]{first-order logic} itself (see \fullref{rem:language_definitions_using_sets}).
  \end{itemize}

  The interesting part is that \hyperref[def:naive_set_theory]{na\"ive set theory} is inconsistent due to \fullref{thm:russels_paradox} and we do not know whether \hyperref[def:zfc]{\logic{ZFC}} is consistent. It is possible that the set theory which we use within the metalogic in order to provide models of \logic{ZFC} is itself inconsistent. In that case, due to \eqref{eq:thm:minimal_propositional_negation_laws/efq}, even a proof of consistency of \logic{ZFC} could be invalid.

  Another interesting note is \fullref{ex:skolems_paradox}.
\end{remark}

\begin{remark}\label{rem:first_order_theories_in_zfc}
  Instead of studying first-order theories like the \hyperref[def:group/theory]{theory of groups}, we can instead reformulate the definition within a set theory like \hyperref[def:zfc]{\logic{ZFC}} and add a \hyperref[rem:predicate_formula]{predicate formula} \( \op{IsGroup} \) with one free variable which is valid only for groups. That is, \( \op{IsGroup}[\tau] \) is a tautology if and only if the ground term \( \tau \) satisfies the set-based definition of a group given in \fullref{def:group}.

  This is a natural approach and we usually assume it implicitly. Furthermore, it makes no sense to speak about concepts like the \hyperref[thm:substructures_form_complete_lattice]{lattice of subgroups} or the \hyperref[def:cardinal]{cardinality} of a group otherwise (we may use categorical definitions but our \hyperref[def:category]{definition of a category} also depends on an ambient set theory). This is also a natural framework for defining \hyperref[def:topological_space]{topological spaces}.

  Thus, roughly, set theory allows us to use \hyperref[rem:higher_order_logic]{higher-order relations and types} in first-order logic.
\end{remark}
