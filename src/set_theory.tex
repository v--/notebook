\section{Set theory}\label{sec:set_theory}

Sets are ubiquitous in mathematics yet the theory if sets is far from simple. Attention needs to be put to give consistent definitions, especially regarding classes that are used, for example, in the \hyperref[def:category]{definition of a category}.

We first use the simplicity of \hyperref[def:naive_set_theory]{na\"ive set theory} to introduce some fundamental definitions so that the more sophisticated \hyperref[def:zfc]{Zermelo-Fraenkel set theory} can be easily formulated.

We then proceed to define the \hyperref[src/category_of_sets]{category of sets} and to discuss the controversial axiom of choice with all of its incarnations given in \fullref{thm:axiom_of_choice_equivalences}.

Finally, we discuss \hyperref[src/ordinals]{ordinals} which we use for transfinite induction and recursion and \hyperref[src/cardinals]{cardinals} which we use in many definitions like \hyperref[def:topological_space_character]{character of a topological space} or the \hyperref[def:poset_chain_and_antichain]{width and height of a poset}.

\begin{remark}\label{rem:sets}
  The relation between \hyperref[first_order_logic]{first-order logic} and set theory is remarkable. On one hand, set theory studies axiomatically defined sets and the relations between axioms such as \hyperref[def:zfc]{ZFC} by utilizing first-order logic, among other things.

  On the other hand, the \hyperref[def:first_order_semantics]{classical first-order semantical framework} is defined in terms of \hyperref[def:first_order_structure]{first-order structures}, which in turn are defined in terms of sets.

  We do the following:
  \begin{itemize}
    \item We build certain large sets within set theory that can be used as models of ZFC.
    \item Given such a set, we can define it as \enquote{the} model of ZFC we are interested in.
    \item Use this model within our \hyperref[rem:metalogic]{metalogical framework} for definitions such as \fullref{def:language}, which are used to define \hyperref[sec:first_order_logic]{first-order logic} itself (see \fullref{rem:language_definitions_using_sets}).
  \end{itemize}
\end{remark}
