\subsection{Hausdorff distance}\label{subsec:hausdorff_distance}

Let \( (X, \mu) \) be a metric space\Tinyref{def:metric_space}.

\begin{definition}\label{def:hausdorff_distance}\cite[144]{Dontchev2014}
  Fix two sets \( E \subseteq X \) and \( F \subseteq X \).

  The~\textbf{excess of \( E \) beyond \( F \)} is defined as
  \begin{align}\label{equ:hausdorff_excess_def}
    &e: \Power X \times \Power X \to \BB{R} \cup \{ \infty \} \\ \nonumber
    &e(E, F) \coloneqq \begin{cases}
      +\infty, &E = \varnothing, D = \varnothing \\
      0, &E = \varnothing, D \neq \varnothing \\
      \sup_{x \in E} \Dist(x, F) \stackrel{(\ref{equ:hausdorff_excess_def})}{=} \inf \{\delta > 0 \colon E \subseteq F_\delta \}, &E \neq \varnothing \nonumber\refstepcounter{equation}
    \end{cases}
  \end{align}
  where \( F_\delta \coloneqq \{ y \in X \colon \Dist(y, F) \leq \delta \} \).

  The~\textbf{Pompeiu-Hausdorff distance} or simply~\textbf{Hausdorff} distance between them is then defined as
  \begin{align*}
    h(E, F) \coloneqq \max\{ e(E, F), e(F, E) \} = \inf \{\delta > 0 \colon E \subseteq F_\delta, F \subseteq E_\delta \}.
  \end{align*}
\end{definition}
\begin{proof}(of \ref{equ:hausdorff_excess_def})
  Note that the set
  \begin{align*}
    F_{e(E, F)} = \{ x \in X \colon \Dist(x, F) \leq \sup_{x \in E} \Dist(x, F) \}
  \end{align*}
  obviously includes \( E \).

  Now let \( \delta > 0 \) be any real number that satisfies \( E \subseteq F_\delta \), i.e.
  \begin{align*}
    E \subseteq F_\delta = \{ x \in X \colon \Dist(x, F) \leq \delta \},
  \end{align*}
  which implies that
  \begin{align*}
    e(E, F) = \sup_{x \in E} \Dist(x, F) \leq \delta.
  \end{align*}
\end{proof}

\begin{proposition}
  The Hausdorff distance is a metric on the nonempty compact subsets of \( X \).
\end{proposition}
\begin{proof}
  Let \( E \), \( F \) and \( G \) be nonempty compact subsets of \( X \).

  The function \( h \) is nonnegative. Since we exclude empty and unbounded sets, We do not care about infinite values.

  \begin{description}
    \item[Identity] Obviously \( h(E, E) = 0 \). If \( h(E, F) = 0 \), then there exists no point of \( E \) outside of \( F \) and vice versa, hence \( E = F \).
    \item[Symmetry] This follows from the symmetry of the \( \max \) function.
    \item[Subadditivity] For any point \( y \in X \), we have
    \begin{align*}
      \Dist(x, G)
      =
      \inf_{z \in G} \mu(x, z)
      \leq
      \mu(x, y) + \inf_{y \in G} \mu(y, z)
      =
      \mu(x, y) + \Dist(y, G).
    \end{align*}

    Select \( y \in F \) that minimizes the distance \( \mu(x, y) \) over \( F \) (compactness allows us), so that % TODO: Prove Weierstrass' theorem
    \begin{align*}
      \Dist(x, G)
      \leq
      \mu(x, y) + \Dist(y, G)
      =
      \Dist(x, F) + \Dist(y, G)
      \leq
      \Dist(x, F) + e(F, G).
    \end{align*}

    It now follows that
    \begin{align*}
      e(E, G)
      &=
      \inf \{\delta > 0 \colon E \subseteq G_\delta \}
      = \\ &=
      \inf \{\delta > 0 \colon E \subseteq \{ x \in X \colon \Dist(x, G) \leq \delta \}
      \leq \\ &\leq
      \inf \{\delta > 0 \colon E \subseteq \{ x \in X \colon \Dist(x, F) + e(F, G) \leq \delta, y \in X \}
      = \\ &=
      e(F, G) + \inf \{\delta > 0 \colon E \subseteq F_\delta \}
      = \\ &=
      e(F, G) + e(E, F).
    \end{align*}
  \end{description}
\end{proof}
