\subsection{Integers}\label{subsec:integers}

\begin{definition}\label{def:set_of_integers}
  The set \( \BbbZ \) of \term{integers} is defined as the Grothendieck \hyperref[thm:monoid_grothendieck_completion_universal_property]{completion} of the commutative monoid \( (\BbbN, +) \).

  Let \( \oplus \), \( \odot \) and \( \leq_N \) be the operations in \( \BbbN \) (see \fullref{thm:natural_number_multiplication_properties}). Since either \( x \in \BbbZ \) or \( -x \in \BbbZ \) is isomorphic to a natural number, we extend the operations to \( \BbbZ \) as follows:
  \begin{thmenum}
    \item addition is defined in the completion.
    \item multiplication is defined as follows:
          \begin{equation*}
            x \cdot y \coloneqq \begin{cases}
              x \odot y,    & x \geq 0 \iff y \geq 0      \\
              (-x) \odot y, & x < 0 \text{ and } y \geq 0 \\
              x \odot (-y), & x \geq 0 \text{ and } y < 0
            \end{cases}
          \end{equation*}

    \thmitem{def:set_of_integers/order} the order is inherited
    \item the additional absolute \hyperref[def:absolute_value]{value} operation is defined as
          \begin{equation*}
            \abs{x} \coloneqq \begin{cases}
              x,  & x \geq 0, \\
              -x, & x < 0.
            \end{cases}
          \end{equation*}
  \end{thmenum}
\end{definition}
\begin{proof}
  The proof that multiplication and absolute values are well-defined can be done similarly to the proof of \fullref{thm:monoid_grothendieck_completion_universal_property}.

  Integer multiplication obviously generalizes natural number multiplication.
\end{proof}

\begin{proposition}\label{thm:integers_are_euclidean_domain}
  The \hyperref[def:integral_domain]{domain} of integers \( \BbbZ \) is \hyperref[def:euclidean_domain]{Euclidean} with \( \delta(n) \coloneqq \abs{n} \). Furthermore, the remainder and quotient are unique.
\end{proposition}
\begin{proof}
  Let \( a, b \in \BbbZ \) and \( b \neq 0 \). Suppose that \( b > 0 \). Define
  \begin{balign*}
    q & \coloneqq \max \{ q \in \BbbZ \colon bq \leq \abs{a} \} \\
    r & \coloneqq a - bq.
  \end{balign*}

  It remains to show that either \( r = 0 \) or \( \delta(r) < \delta(b) \).

  Note that \( r = 0 \) if and only if \( b \) is a \hyperref[def:divisibility]{divisor} of \( a \).

  Suppose that \( b \) is not a divisor of \( a \). Note that \( g \geq 0 \) and \( r > 0 \). If \( r \geq b \), this would imply \( r - b \geq 0 \) and
  \begin{equation*}
    a = bq + r = b(q + 1) + (r - b) \geq b(q + 1),
  \end{equation*}
  which would contradict the maximality of \( q \). Thus, \( r < b \).

  It remains to show uniqueness. Suppose that \( a = bq + r = bq' + r' \). Then
  \begin{equation*}
    0 = b(q - q') + (r - r').
  \end{equation*}

  Thus, \( b \mid r - r' \). But \( -b < r - r' < b \), which implies that \( r = r' \). Thus, also implies that \( q = q' \) since \( b \neq 0 \).

  Now suppose that \( b < 0 \). Define
  \begin{balign*}
    q & \coloneqq \min \{ q \in \BbbZ \colon -\abs{a} \leq bq \} \\
    r & \coloneqq a - bq.
  \end{balign*}

  Suppose that \( b \) is not a divisor of \( a \). Note that \( q \geq 0 \) and \( r < 0 \). If \( r \leq b \), this would imply \( r - b \leq 0 \) and
  \begin{equation*}
    a = bq + r = b(q + 1) + (r - b) \leq b(q + 1),
  \end{equation*}
  which would contradict the minimality of \( q \). Thus, \( r > b \) and, since both \( b \) and \( r \) are negative, \( \abs{r} < \abs{b} \).

  To see uniqueness, suppose that \( a = bq + r = bq' + r' \). Thus, \( b \mid r - r' \). But \( -\abs{b} = b < r - r' < -b = \abs{b} \), which implies that \( r = r' \) and \( q = q' \).

  In both cases, we obtained unique integers \( q \) and \( r \) such that \( a = bq + r \) with \( \abs{r} < \abs{b} \).
\end{proof}

\begin{remark}\label{rem:units_in_rings_etymology}
  An integer \( a \) is divisible by \( b \neq 0 \) if there exists a number \( q \) such that
  \begin{equation*}
    a = qb.
  \end{equation*}

  Obviously \( q = (-q)(-1) \), so the following also holds:
  \begin{equation*}
    a = [(-q)(-1)]b = (-q)(-b),
  \end{equation*}
  hence \( a \) is also divisible by \( -b \).

  For any nonzero number \( b = 1 \cdot b \) that divides \( a \), the number \( -b = (-1) \cdot b \) also divides \( a \). Both \( 1 \) and \( -1 \) have unit norm (that is, \( \abs{1} = \abs{-1} = 1 \)), so it is reasonable to call them \enquote{units}. They are the only integers \( e \) with the property that if \( b | a \), then \( eb | a \). This is probably the reason why invertible elements in arbitrary rings are named \enquote{units}. Another reason is that invertible elements divide the multiplicative identity, commonly denoted by \( 1 \).

  Consider fields, in which all nonzero elements are units. It makes no sense to speak of divisibility whatsoever because any real number \( a \) is divisible by any nonzero real number \( b \). Putting \( q \coloneqq \frac a b \) satisfies the divisibility condition. Now if \( e \) is any unit in \( \BbbR \), we have
  \begin{equation*}
    a = qb = q(e^{-1} e) b = (qe^{-1}) (eb),
  \end{equation*}
  hence \( eb \) also divides \( a \).
\end{remark}

\begin{lemma}[Euclid's lemma]\label{thm:euclids_lemma}
  An \hyperref[def:set_of_integers]{integer} is \hyperref[def:semiring_ideal/prime]{prime} if and only if it is irreducible.
\end{lemma}
\begin{proof}
  Follows from \fullref{thm:ufd_prime_iff_irreducible}.
\end{proof}

\begin{definition}\label{def:prime_number}
  Despite negative integers being prime \hyperref[thm:euclids_lemma]{elements} of the ring \( \BbbZ \), we only call positive prime integers \term{prime numbers}. That is, a positive integer is prime if it has no divisors except \( 1 \) and itself.

  Non-prime integers are called \term{composite numbers}.
\end{definition}

\begin{definition}\label{def:coprime_numbers}
  Two integers \( n, m \) are called \term{coprime} (see \fullref{def:semiring_ideal/coprime}) if \( \gcd(n, m) = 1 \).
\end{definition}

\begin{theorem}[Fundamental theorem of arithmetic]\label{thm:fundamental_theorem_of_arithmetic}
  Every positive integer greater than \( 2 \) can be \hyperref[def:irreducible_factorization]{factored} into a product of \hyperref[def:prime_number]{prime} powers.
\end{theorem}
\begin{proof}
  The ring \( \BbbZ \) is an Euclidean domain by \fullref{thm:integers_are_euclidean_domain}, which is a principal ideal domain by \fullref{thm:euclidean_domain_is_pid}, which is a unique factorization domain by \fullref{thm:pid_is_ufd}.
\end{proof}

\begin{definition}\label{def:eulers_totient_function}
  For any positive integer \( n \), denote by \( \varphi(n) \) the number of strictly smaller than \( n \) positive integers that are \hyperref[def:coprime_numbers]{coprime} to \( n \). We call \( \varphi: \BbbZ_{>0} \to \BbbZ_{\geq 0} \) \term{Euler's totient function}.
\end{definition}

\begin{proposition}\label{thm:def:eulers_totient_function}
  \hyperref[def:eulers_totient_function]{Euler's totient function} \( \varphi \) has the following basic properties:
  \begin{thmenum}
    \thmitem{thm:def:eulers_totient_function/one} \( \varphi(1) = 0 \).
    \thmitem{thm:def:eulers_totient_function/prime} If \( p \) is \hyperref[def:prime_number]{prime}, then \( \varphi(p) = p - 1 \).
    \thmitem{thm:def:eulers_totient_function/zn} The \hyperref[def:semiring]{multiplicative group} \( \BbbZ_n^\times \) of the ring \hyperref[thm:ring_of_integers_modulo]{\( \BbbZ_n \)} of integers modulo \( n > 1 \) has order \( \varphi(n) \).
  \end{thmenum}
\end{proposition}
\begin{proof}
  \SubProofOf{thm:def:eulers_totient_function/one} There are no positive integers smaller than \( 1 \).

  \SubProofOf{thm:def:eulers_totient_function/prime} Every positive integer smaller than \( p \) is coprime to \( p \), and there are exactly \( p - 1 \) positive integers smaller than \( p \) --- \( 1, 2, \ldots, p - 1 \).

  \SubProofOf{thm:def:eulers_totient_function/zn} Follows from \fullref{thm:multiplicative_group_of_integers_modulo}.
\end{proof}

\begin{theorem}[Euler's totient theorem]\label{thm:eulers_totient_theorem}
  For positive integers \( n \) and \( x \), we have
  \begin{equation*}
    x^{\varphi(n)} \cong 1 \pmod n,
  \end{equation*}
  where \( \varphi \) is \hyperref[def:eulers_totient_function]{Euler's totient function}.
\end{theorem}
\begin{proof}
  For \( n = 1 \), this is obvious from \fullref{thm:def:eulers_totient_function/one}. Suppose that \( n > 1 \).

  Consider the \hyperref[def:semiring]{multiplicative group} \( \BbbZ_n^\times \) of the ring \hyperref[thm:ring_of_integers_modulo]{\( \BbbZ_n \)} of integers modulo \( n \). For some positive integer \( x < n \), consider the \hyperref[def:cyclic_group]{cyclic subgroup} \( \set{ 1, x, x^2, \ldots } \) (modulo \( n \)). It is necessarily finite as a subgroup of \( \BbbZ_n^\times \). Furthermore, by \fullref{thm:lagranges_theorem_for_groups}, its order \( k \) divides the order of \( \BbbZ_n^\times \). By \fullref{thm:def:eulers_totient_function/zn}, the order of \( \BbbZ_n^\times \) is \( \varphi(n) \).

  We have \( x^k \cong 1 \pmod n \) since \( k \) is the order of a cyclic group. If \( \varphi(n) = km \), then
  \begin{equation*}
    x^{\varphi(n)}
    =
    x^{km}
    \reloset {\eqref{eq:thm:magma_exponentiation_properties/repeated}} =
    (x^k)^m
    \cong
    1^m
    \pmod n.
  \end{equation*}
\end{proof}

\begin{proposition}\label{thm:division_modulo}
  Given positive integers \( n \) and \( m \), we can apply \fullref{alg:euclidean_division_of_integers} to obtain \( n = a \varphi(m) + b \), where \( \varphi \) is \hyperref[def:eulers_totient_function]{Euler's totient function}.

  Then
  \begin{equation*}
    x^n \cong x^b \pmod m,
  \end{equation*}
\end{proposition}
\begin{proof}
  By \fullref{thm:eulers_totient_theorem}, \( x^{\varphi(m)} \cong 1 \pmod m \). Then
  \begin{equation*}
    x^n = (x^{\varphi(m)})^a x^b \cong x^b \pmod m.
  \end{equation*}
\end{proof}

\begin{theorem}[Fermat's little theorem]\label{thm:fermats_little_theorem}
  For a \hyperref[def:prime_number]{prime number} \( p \) and for any positive integer \( x \), we have
  \begin{equation*}
    x^p \cong x \pmod p.
  \end{equation*}
\end{theorem}
\begin{proof}
  If \( p \mid x \), then both \( x^p \) and \( x \) and congruent to \( 0 \) modulo \( p \).

  Otherwise, by \fullref{thm:eulers_totient_theorem}, we have \( x^{\varphi(p) + 1} \cong x \pmod p \), and by \fullref{thm:def:eulers_totient_function/prime}, we have \( \varphi(p) + 1 = p \).
\end{proof}
