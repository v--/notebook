\subsection{Transfinite arithmetic}\label{subsec:ordinal_and_cardinal_arithmetic}

Our purpose is to extend natural number arithmetic to ordinals and cardinals. It turns out that the two are rather different. We will first introduce some additional concepts, however.

\begin{definition}
  Let \( A \) and \( B \) be sets of \hyperref[def:ordinal]{ordinals}. We say
\end{definition}

\begin{definition}\label{def:ordinal_arithmetic}
  We recursively define arithmetic operations for arbitrary \hyperref[def:ordinal]{ordinals} as extensions of the corresponding operations of \hyperref[def:peano_arithmetic]{Peano arithmetic}.

  \begin{thmenum}
    \thmitem{def:ordinal_arithmetic/sum}\mcite[lemma 66.6]{OpenLogicFull} The \term{sum} of \( \alpha \) and \( \beta \) extends \eqref{eq:def:peano_arithmetic/PA4} and \eqref{eq:def:peano_arithmetic/PA5} with a case for limit ordinals:
    \begin{equation}\label{eq:def:ordinal_arithmetic/sum}
      \alpha + \beta \coloneqq \begin{cases}
        \alpha,                                            &\beta = 0 \\
        \op{succ}(\alpha + \gamma),                        &\beta = \op{succ}(\gamma)\\
        \sup\set{ \alpha + \gamma \given \gamma < \beta }, &\beta \T{is a limit ordinal}\\
      \end{cases}
    \end{equation}

    \thmitem{def:ordinal_arithmetic/product}\mcite[lemma 66.13]{OpenLogicFull} Analogously, the \term{product} of \( \alpha \) and \( \beta \) extends \eqref{eq:def:peano_arithmetic/PA6} and \eqref{eq:def:peano_arithmetic/PA7}:
    \begin{equation}\label{eq:def:ordinal_arithmetic/product}
      \alpha \cdot \beta \coloneqq \begin{cases}
        0,                                                     &\beta = 0 \\
        \alpha \cdot \gamma + \alpha,                          &\beta = \op{succ}(\gamma)\\
        \sup\set{ \alpha \cdot \gamma \given \gamma < \beta }, &\beta \T{is a limit ordinal}\\
      \end{cases}
    \end{equation}

    \thmitem{def:ordinal_arithmetic/exponentiation}\mcite[lemma 66.16]{OpenLogicFull} Exponentiation extends \fullref{def:unital_magma/exponentiation}:
    \begin{equation}\label{eq:def:ordinal_arithmetic/exponentiation}
      \alpha^\beta \coloneqq \begin{cases}
        1,                                               &\beta = 0 \\
        \alpha^\gamma \cdot \alpha,                      &\beta = \op{succ}(\gamma)\\
        \sup\set{ \alpha^\gamma \given \gamma < \beta }, &\beta \T{is a limit ordinal}\\
      \end{cases}
    \end{equation}
  \end{thmenum}
\end{definition}

\begin{proposition}\label{thm:ordinal_addition_disjoin_union}
  For any two ordinals \( \alpha \) and \( \beta \), their \hyperref[def:ordinal_arithmetic/addition]{sum} satisfies
  \begin{equation*}
    \alpha + \beta = \ord(\alpha \sqcup \beta, \prec),
  \end{equation*}
  where \( \prec \) is the \hyperref[def:lexicographic_order]{lexicographic order} on the \hyperref[def:disjoin_union]{disjoint union} \( \alpha \sqcup \beta \).
\end{proposition}
% \begin{proof}
%   We will construct an explicit \hyperref[def:partially_ordered_set/homomorphism]{order isomorphism} between \( (\alpha + \beta, \in) \) and \( (\alpha \sqcup \beta, \prec) \).

%   Define
%   \begin{equation*}
%     \begin{aligned}
%       &f: (\alpha \sqcup \beta) \to (\alpha + \beta) \\
%       &f(i, \gamma) \coloneqq \begin{cases}
%         \gamma,          &i = 0 \\
%         \alpha + \gamma, &i = 1.
%       \end{cases}
%     \end{aligned}
%   \end{equation*}

%   By \fullref{thm:totally_ordered_strict_isomorphisms} it is sufficient to show that \( f \) is a bijective strict monotone map.

%   \SubProofOf[eq:def:partially_ordered_set/homomorphism/strict]{strict monotonicity} Let \( (i, \gamma) < (j, \delta) \) be some members of \( \alpha \sqcup \beta \).
%   \begin{itemize}
%     \item If \( i = j = 0 \), then \( f(i, \gamma) = \gamma < \delta = f(j, \delta) \).
%     \item If \( i = j = 1 \), then \( f(i, \gamma) = \alpha + \gamma < \alpha + \delta = f(j, \delta) \).
%     \item If \( i = 0 \) and \( j = 1 \), then \( f(0, \gamma) = \gamma < \alpha + \delta = f(1, \delta) \).
%   \end{itemize}

%   \SubProofOf[def:function_invertibility/injective]{injectivity} If \( f(\gamma) = f(\delta) \), then
%   \begin{equation*}

%   \end{equation*}

%   \SubProofOf[def:function_invertibility/bijective]{bijectivity}
% \end{proof}

\begin{proposition}\label{thm:ordinal_addition_properties}
  Ordinal number addition is \hyperref[def:magma/associative]{associative}, \hyperref[def:magma/cancellative]{left cancellative} and \hyperref[def:preordered_magma]{compatible} with ordinal ordering.

  As in \fullref{thm:ordinals_are_well_ordered}, we adapt the corresponding axioms due to \fullref{thm:burali_forti_paradox}. The more concrete result is:
  \begin{thmenum}
    \thmitem{thm:ordinal_addition_properties/associative} For any three ordinals \( \alpha \), \( \beta \) and \( \gamma \) we have
    \begin{equation*}
      (\alpha + \beta) + \gamma = \alpha + (\beta + \gamma).
    \end{equation*}

    \thmitem{thm:ordinal_addition_properties/cancellative} For any three ordinals \( \alpha \), \( \beta \) and \( \gamma \) such that \( \gamma + \alpha = \gamma + \beta \), we have \( \alpha = \beta \).

    \thmitem{thm:ordinal_addition_properties/ordering} For any three ordinals \( \alpha \), \( \beta \) and \( \gamma \) such that \( \alpha < \beta \) we have \( \alpha + \gamma < \beta + \gamma \).
  \end{thmenum}

   See \fullref{ex:ordinal_addition} for counterexamples to \hyperref[def:magma/cancellative]{right cancellation} and \hyperref[def:magma/commutativity]{commutativity}.
\end{proposition}
% \begin{proof}
%   \SubProofOf[thm:ordinal_addition_properties/cancellative]{cancellation} We will use induction on \( \gamma \). Note that the base and successor cases are handles in the same way as \fullref{thm:natural_number_addition_properties}. Hence only the limit case remains.

%   Let \( \lambda \) be a limit ordinal and suppose that cancellation holds for \( \gamma < \lambda \) and arbitrary \( \alpha \) and \( \beta \). Let \( \lambda + \alpha = \lambda + \beta \). Then


% Denote this common value by \( \sigma \). Thus for any \( \delta < \lambda \) we have
% \begin{equation*}
%   \alpha + \delta
%   \leq
%   \sup\set{ \alpha + \gamma \given \gamma < \lambda }
%   \alpha + \lambda
%   \reloset {\eqref{eq:def:ordinal_arithmetic/sum}} =
%   \sigma
% \end{equation*}
% and analogously for \( \beta + \lambda \). Therefore
% \begin{equation*}
%   \alpha + \delta \leq \sigma \T{if and only if} \beta + \delta \leq \sigma.
% \end{equation*}

% Because the ordinals are trichotomic, it follows that \( \alpha + \delta = \beta + \delta \). By the inductive hypothesis, \( \alpha + \delta = \beta + \delta \) implies that \( \alpha = \beta \).

%   % Hence cancellation holds.

%   \SubProofOf[thm:ordinal_addition_properties/associative]{associativity}

%   Let \( \lambda \) be a limit ordinal and suppose that \eqref{eq:thm:ordinal_addition_properties/associative} holds for \( \gamma < \lambda \) and for arbitrary \( \alpha \) and \( \beta \). Then
%   \begin{align*}
%     (\alpha + \beta) + \lambda
%     &=
%     \sup\set{ (\alpha + \beta) + \gamma \given \gamma < \lambda }
%     = \\ &=
%     \sup\set{ \alpha + (\beta + \gamma) \given \gamma < \lambda }
%     \reloset {\ref{thm:ordinal_addition_supremum}} = \\ &=
%     \alpha + \sup\set{ \beta + \gamma \given \gamma < \lambda }
%     = \\ &=
%     \alpha + (\beta + \lambda).
%   \end{align*}
% \end{proof}

\begin{example}\label{ex:ordinal_addition}
  \begin{thmenum}
    \thmitem{ex:ordinal_addition/right_cancellative} \( 0 + \omega = 1 + \omega = \omega \), however \( 0 \neq 1 \).
    \thmitem{ex:ordinal_addition/commutative} \( \omega + 1 \neq 1 + \omega = \omega \).
  \end{thmenum}
\end{example}

\begin{proposition}\label{thm:ordinal_multiplication_disjoin_union}
  For any two ordinals \( \alpha \) and \( \beta \), their \hyperref[def:ordinal_arithmetic/multiplication]{product} satisfies
  \begin{equation*}
    \alpha \cdot \beta = \ord(\alpha \times \beta, \prec),
  \end{equation*}
  where \( \prec \) is the \hyperref[def:lexicographic_order]{lexicographic order} on the \hyperref[def:cartesian_product]{Cartesian product} \( \alpha \times \beta \).
\end{proposition}

\begin{definition}\label{def:cardinal_arithmetic}
  Fix two ordinals \( \kappa \) and \( \mu \). We will define arithmetic operations for them.

  \begin{thmenum}
    \thmitem{def:cardinal_arithmetic/sum} We define their \term{sum} as
    \begin{equation*}
      \kappa + \mu \coloneqq \card(\kappa \sqcup \mu),
    \end{equation*}
    where \( \kappa \sqcup \mu \) is their \hyperref[def:disjoint_union]{disjoint union}.

    \thmitem{def:cardinal_arithmetic/product} We define their \term{product} as
    \begin{equation*}
      \kappa \cdot \mu \coloneqq \card(\kappa \times \mu).
    \end{equation*}

    \thmitem{def:cardinal_arithmetic/exponentiation} We define \term{exponentiation} as
    \begin{equation*}
      \kappa^\mu \coloneqq \card(\fun(\kappa, \mu)).
    \end{equation*}
  \end{thmenum}
\end{definition}

\begin{proposition}\label{thm:cardinal_arithmetic_properties}
  \hyperref[def:cardinal_arithmetic]{Cardinal arithmetic} has the following basic properties:

  \begin{thmenum}
    \thmitem{thm:cardinal_arithmetic_properties/addition} Cardinal addition is associative and commutative.

    \thmitem{thm:cardinal_arithmetic_properties/multiplication} Cardinal multiplication is associative and commutative.

    \thmitem{thm:cardinal_arithmetic_properties/power_set} For every set \( A \)
    \begin{equation*}
      \card(\pow(A)) = 2^{\card(A)}.
    \end{equation*}
  \end{thmenum}
\end{proposition}
