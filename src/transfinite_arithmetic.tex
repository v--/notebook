\subsection{Transfinite arithmetic}\label{subsec:ordinal_and_cardinal_arithmetic}

Our purpose is to extend natural number arithmetic to ordinals and cardinals. It turns out that the two are rather different.

\begin{definition}\label{def:disjoint_union}
  The \term{disjoint union} of the \hyperref[def:indexed_family]{indexed family} \( \seq{ A_k }_{k \in \mscrK} \) of nonempty sets is
  \begin{equation*}
    \bigsqcup_{k \in \mscrK} A_k \coloneqq \set[\Big]{ (k, x) \given k \in \mscrK \T{and} x \in A_k }.
  \end{equation*}
\end{definition}

\begin{definition}\label{def:ordinal_arithmetic}
  Fix two ordinals \( \alpha \) and \( \beta \). We will define arithmetic operations for them.

  \begin{thmenum}
    \thmitem{def:ordinal_arithmetic/sum} We define their \term{sum} as
    \begin{equation*}
      \alpha + \beta \coloneqq \ord(\alpha \sqcup \beta, \prec),
    \end{equation*}
    where \( (\gamma_1, k_1) \prec (\gamma_2, k_2) \) if either \( k_1 < k_2 \) or both \( k_1 = k_2 \) and \( \gamma_1 < \gamma_2 \).

    \thmitem{def:ordinal_arithmetic/product} We define their \term{product} as
    \begin{equation*}
      \alpha \cdot \beta \coloneqq \ord(\alpha \times \beta),
    \end{equation*}
    where \( (\gamma_1, \delta_1) \prec (\gamma_2, \delta_2) \) if either \( \gamma_1 < \gamma_2 \) or both \( \gamma_1 = \gamma_2 \) and \( \delta_1 < \delta_2 \).

    \thmitem{def:ordinal_arithmetic/exponentiation} We define \term{exponentiation} as
    \begin{equation*}
      \alpha^\beta \coloneqq \ord\parens[\Big]{ \set[\Big]{ f: \beta \to \alpha \given \qexists {n \in \omega} \rank(f^{-1}(0)) = n } },
    \end{equation*}
    where \( f \prec g \) if \( f(\gamma_0) < g(\gamma_0) \) for \( \gamma_0 \coloneqq \max\set{ \gamma \in \beta \given f(\gamma) \neq g(\gamma) } \).
  \end{thmenum}
\end{definition}

\begin{proposition}\label{thm:ordinal_arithmetic_properties}
  \hyperref[def:ordinal_arithmetic]{Ordinal arithmetic} has the following basic properties:

  \begin{thmenum}
    \thmitem{thm:ordinal_arithmetic_properties/addition}\mcite[lemma 66.7]{OpenLogicFull} Ordinal addition is associative and cancellative.

    \medskip

    \thmitem{thm:ordinal_arithmetic_properties/addition_commutativity}\mcite[prop. 66.8]{OpenLogicFull} Ordinal addition is not commutative.

    \medskip

    \thmitem{thm:ordinal_arithmetic_properties/multiplication}\mcite[lemma 66.14]{OpenLogicFull} Ordinal multiplication is associative and cancellative and distributes over addition.

    \medskip

    \thmitem{thm:ordinal_arithmetic_properties/multiplication_commutativity}\mcite[prop. 66.15]{OpenLogicFull} Ordinal multiplication is not commutative.
  \end{thmenum}
\end{proposition}

\begin{definition}\label{def:cardinal_arithmetic}
  Fix two ordinals \( \kappa \) and \( \mu \). We will define arithmetic operations for them.

  \begin{thmenum}
    \thmitem{def:cardinal_arithmetic/sum} We define their \term{sum} as
    \begin{equation*}
      \kappa + \mu \coloneqq \card(\kappa \sqcup \mu),
    \end{equation*}
    where \( \kappa \sqcup \mu \) is their \hyperref[def:disjoint_union]{disjoint union}.

    \thmitem{def:cardinal_arithmetic/product} We define their \term{product} as
    \begin{equation*}
      \kappa \cdot \mu \coloneqq \card(\kappa \times \mu).
    \end{equation*}

    \thmitem{def:cardinal_arithmetic/exponentiation} We define \term{exponentiation} as
    \begin{equation*}
      \kappa^\mu \coloneqq \card(\fun(\kappa, \mu)).
    \end{equation*}
  \end{thmenum}
\end{definition}

\begin{proposition}\label{thm:cardinal_arithmetic_properties}
  \hyperref[def:cardinal_arithmetic]{Cardinal arithmetic} has the following basic properties:

  \begin{thmenum}
    \thmitem{thm:cardinal_arithmetic_properties/addition} Cardinal addition is associative and commutative.

    \thmitem{thm:cardinal_arithmetic_properties/multiplication} Cardinal multiplication is associative and commutative.

    \thmitem{thm:cardinal_arithmetic_properties/power_set} For every set \( A \)
    \begin{equation*}
      \card(\pow(A)) = 2^{\card(A)}.
    \end{equation*}
  \end{thmenum}
\end{proposition}
