\subsection{Transfinite arithmetic}\label{subsec:ordinal_and_cardinal_arithmetic}

Our purpose is to extend natural number arithmetic to ordinals and cardinals. It turns out that the two are rather different. We will first introduce some additional concepts, however.

\begin{definition}
  Let \( A \) and \( B \) be sets of \hyperref[def:ordinal]{ordinals}. We say
\end{definition}

\begin{definition}\label{def:ordinal_arithmetic}
  We recursively define arithmetic operations for arbitrary \hyperref[def:ordinal]{ordinals} as extensions of the corresponding operations of \hyperref[def:peano_arithmetic]{Peano arithmetic}.

  \begin{thmenum}
    \thmitem{def:ordinal_arithmetic/sum}\mcite[lemma 66.6]{OpenLogicFull} The \term{sum} of \( \alpha \) and \( \beta \) extends \eqref{eq:def:peano_arithmetic/PA4} and \eqref{eq:def:peano_arithmetic/PA5} with a case for limit ordinals:
    \begin{equation}\label{eq:def:ordinal_arithmetic/sum}
      \alpha + \beta \coloneqq \begin{cases}
        \alpha,                                            &\beta = 0 \\
        \op{succ}(\alpha + \gamma),                        &\beta = \op{succ}(\gamma)\\
        \sup\set{ \alpha + \gamma \given \gamma < \beta }, &\beta \T{is a limit ordinal}\\
      \end{cases}
    \end{equation}

    \thmitem{def:ordinal_arithmetic/product}\mcite[lemma 66.13]{OpenLogicFull} Analogously, the \term{product} of \( \alpha \) and \( \beta \) extends \eqref{eq:def:peano_arithmetic/PA6} and \eqref{eq:def:peano_arithmetic/PA7}:
    \begin{equation}\label{eq:def:ordinal_arithmetic/product}
      \alpha \cdot \beta \coloneqq \begin{cases}
        0,                                                     &\beta = 0 \\
        \alpha \cdot \gamma + \alpha,                          &\beta = \op{succ}(\gamma)\\
        \sup\set{ \alpha \cdot \gamma \given \gamma < \beta }, &\beta \T{is a limit ordinal}\\
      \end{cases}
    \end{equation}

    \thmitem{def:ordinal_arithmetic/exponentiation}\mcite[lemma 66.16]{OpenLogicFull} Exponentiation extends \fullref{def:unital_magma/exponentiation}:
    \begin{equation}\label{eq:def:ordinal_arithmetic/exponentiation}
      \alpha^\beta \coloneqq \begin{cases}
        1,                                               &\beta = 0 \\
        \alpha^\gamma \cdot \alpha,                      &\beta = \op{succ}(\gamma)\\
        \sup\set{ \alpha^\gamma \given \gamma < \beta }, &\beta \T{is a limit ordinal}\\
      \end{cases}
    \end{equation}
  \end{thmenum}
\end{definition}

\begin{proposition}\label{thm:ordinal_addition_properties}
  Ordinal number addition is \hyperref[def:magma/associative]{associative} and \hyperref[def:magma/cancellative]{cancellative}.

  As in \fullref{thm:ordinals_are_well_ordered}, we adapt the corresponding axioms due to \fullref{thm:burali_forti_paradox}. The more concrete result is:
  \begin{thmenum}
    \thmitem{thm:ordinal_addition_properties/associative} For any three ordinals \( \alpha \), \( \beta \) and \( \gamma \) we have
    \begin{equation}\label{eq:thm:ordinal_addition_properties/associative}
      (\alpha + \beta) + \gamma = \alpha + (\beta + \gamma).
    \end{equation}

    \thmitem{thm:ordinal_addition_properties/cancellative} For any three ordinals \( \alpha \), \( \beta \) and \( \gamma \) such that \( \alpha + \gamma = \beta + \gamma \), we have \( \alpha = \beta \).
  \end{thmenum}

   See \fullref{ex:ordinal_addition_commutativity} for a counterexample of \hyperref[def:magma/commutativity]{commutativity}.
\end{proposition}

\begin{definition}\label{def:cardinal_arithmetic}
  Fix two ordinals \( \kappa \) and \( \mu \). We will define arithmetic operations for them.

  \begin{thmenum}
    \thmitem{def:cardinal_arithmetic/sum} We define their \term{sum} as
    \begin{equation*}
      \kappa + \mu \coloneqq \card(\kappa \sqcup \mu),
    \end{equation*}
    where \( \kappa \sqcup \mu \) is their \hyperref[def:disjoint_union]{disjoint union}.

    \thmitem{def:cardinal_arithmetic/product} We define their \term{product} as
    \begin{equation*}
      \kappa \cdot \mu \coloneqq \card(\kappa \times \mu).
    \end{equation*}

    \thmitem{def:cardinal_arithmetic/exponentiation} We define \term{exponentiation} as
    \begin{equation*}
      \kappa^\mu \coloneqq \card(\fun(\kappa, \mu)).
    \end{equation*}
  \end{thmenum}
\end{definition}

\begin{proposition}\label{thm:cardinal_arithmetic_properties}
  \hyperref[def:cardinal_arithmetic]{Cardinal arithmetic} has the following basic properties:

  \begin{thmenum}
    \thmitem{thm:cardinal_arithmetic_properties/addition} Cardinal addition is associative and commutative.

    \thmitem{thm:cardinal_arithmetic_properties/multiplication} Cardinal multiplication is associative and commutative.

    \thmitem{thm:cardinal_arithmetic_properties/power_set} For every set \( A \)
    \begin{equation*}
      \card(\pow(A)) = 2^{\card(A)}.
    \end{equation*}
  \end{thmenum}
\end{proposition}
