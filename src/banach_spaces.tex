\subsection{Banach spaces}\label{subsec:banach_spaces}

\begin{definition}\label{def:banach_space}
  A \Def{Banach space} is a normed\Tinyref{def:norm} vector space\Tinyref{def:vector_space} which is also a complete metric spaces\Tinyref{def:complete_metric_space} with the metric induced by the norm\Tinyref{def:norm_induced_metric}.
\end{definition}

\begin{example}\label{ex:noncomplete_normed_space}\cite{MathCounterExamples:noncomplete_normed_space}
  Consider the polynomial algebra\Tinyref{def:algebra_of_polynomials} \( \R[x] \) as a vector space with the supremum norm. We will show that it is not complete. Define the sequence
  \begin{equation*}
    p_n(x) \coloneqq \sum_{k=0}^n \frac{x^k} {2^k}, n = 1, 2, \ldots.
  \end{equation*}

  Then the limit of the sequence in \( C([0, 1]) \) is the power series
  \begin{equation*}
    \lim_{n \to \infty} p_n(x)
    =
    \sum_{k=0}^n \frac{x^k} {2^k}
    =
    \frac 2 {2 - x}.
  \end{equation*}

  Since \( \R[x] \) is a subspace of \( C([0, 1]) \), we conclude that \( \R[x] \) has fundamental sequence, but we just demonstrated that its limit is not in \( \R[x] \).
\end{example}

\begin{definition}\label{def:dual_norm}
  Fix two nonempty Banach spaces \( (X, \Norm{\cdot}_X) \) and \( (Y, \Norm{\cdot}_Y) \). We define the \Def{operator norm} \( \Norm{\cdot}_{\Hom(X, Y)} \) on \( \Hom(X, Y) \) equivalently as
  \begin{defenum}
    \DItem{def:dual_norm/sup_unit_sphere}
    \begin{equation*}
      \Norm{L}_{\Hom(X, Y)} \coloneqq \sup_{\Norm{x}_X = 1} \Norm{Lx}_Y.
    \end{equation*}

    \DItem{def:dual_norm/sup_unit_ball}
    \begin{equation*}
      \Norm{L}_{\Hom(X, Y)} \coloneqq \sup_{\Norm{x}_X < 1} \Norm{Lx}_Y.
    \end{equation*}

    \DItem{def:dual_norm/sup_nonzero}
    \begin{equation*}
      \Norm{L}_{\Hom(X, Y)} \coloneqq \sup_{x \neq 0_X} \frac {\Norm{Lx}_Y} {\Norm{x}_X}.
    \end{equation*}

    \DItem{def:dual_norm/inf}
    \begin{equation*}
      \Norm{L}_{\Hom(X, Y)} \coloneqq \inf \left\{ c \geq 0 \colon \Norm{Lx}_Y \leq c \Norm{x}_X \right\}.
    \end{equation*}
  \end{defenum}

  In particular, this induces a norm on \( X^* \).
\end{definition}
