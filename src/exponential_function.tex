\subsection{Exponential function}\label{subsec:exponential_function}

\begin{definition}\label{def:exponential_function}
  We define the \term{exponential function}
  \begin{equation}\label{def:exponential_function/series}
    \exp(z) \coloneqq \sum_{k=0}^\infty \frac {z^k} {k!}
  \end{equation}
  and \term{Euler's number}
  \begin{equation*}
    e \coloneqq \exp(1) = \sum_{i=0}^k \frac 1 {k!}.
  \end{equation*}

  \Fullref{thm:exponential_function_properties/interpolates_power} justifies the notation \( e^z = \exp(z) \).
\end{definition}
\begin{proof}
  We will show that \( \exp(z) \) converges everywhere. By \fullref{thm:power_series_radius_of_convergence}, the radius of convergence is
  \begin{equation*}
    \limsup_{k \to \infty} \frac {k!} {(k-1)!}
    =
    \limsup_{k \to \infty} k
    =
    +\infty
  \end{equation*}

  Hence the radius of convergence of \( \exp(x) \) is infinite.
\end{proof}

\begin{proposition}\label{thm:exponential_function_properties}
  The exponential function \( \exp(z) \) has the following basic properties (not that we do not use the notation \( e^z \) here in order to reduce confusion with yet-undefined power \hyperref[def:power_function]{functions}):

  \begin{thmenum}
    \labitem{thm:exponential_function_properties/eulers_identity} (Euler's identity)
    \begin{equation*}
      \exp(i \pi) = -1.
    \end{equation*}

    \labitem{thm:exponential_function_properties/derivative} \( \exp(z) \) is its own derivative.

    \labitem{thm:exponential_function_properties/homomorphism} \( \exp(x + y) = \exp(x) \exp(y) \). Stated in another way, \( \exp \) is a homomorphism from the additive group of \( \BbbC \) to the multiplicative group.

    \labitem{thm:exponential_function_properties/interpolates_power} The notation \( \exp(x) \) is consistent with iterated multiplication as defined in \fullref{def:semiring/dioid}, that is, \( \exp(n) = \underbrace{e \cdot \ldots \cdot e}_{n \text{times}} \) and for positive integers \( n \), \( \exp(n) =  \) and \( \exp(-n) =\tfrac 1 {\exp(n)} \).

    \labitem{thm:exponential_function_properties/negative_power}
    \begin{equation*}
      \exp(z) = \frac 1 {\exp(-z)}.
    \end{equation*}

    \labitem{thm:exponential_function_properties/real_positive} For real \( t \), \( e^t \) is a positive real number.

    \labitem{thm:exponential_function_properties/conjugate} \( \overline{\exp(z)} = \exp(\overline{z}) \).

    \labitem{thm:exponential_function_properties/unit_circle} For any \( c \in \BbbR \), the function \( t \mapsto \exp(it) \) is a bijection between any half-open interval \( [c, c + 2\pi) \) and the unit circle in \( \BbbC \).

    \labitem{thm:exponential_function_properties/real_bijective} \( t \mapsto \exp(t) \) is a bijection from \( \BbbR \) to \( [0, \infty) \).

    \labitem{thm:exponential_function_properties/bijective} For any \( c \in \BbbR \), \( \exp(z) \) is a bijection between the strip \( S \coloneqq \{ a + bi \colon c \leq b < c + 2\pi \} \) and the complex plane \( \BbbC \setminus \{ 0 \} \).

    \labitem{thm:exponential_function_properties/periodic} \( \exp(z) \) is \( 2i\pi \)-\hyperref[def:periodic_function]{periodic}.

    \labitem{thm:exponential_function_properties/compound_interest} For nonnegative real \( t \geq 0 \) we have
    \begin{equation*}
      \exp(t) = \lim_{n \to \infty} \left(1 + \frac t n \right)^n
    \end{equation*}
  \end{thmenum}
\end{proposition}
\begin{proof}
  \SubProofOf{thm:exponential_function_properties/eulers_identity} By \fullref{eq:thm:trigonometric_function_basic_roots/pi} and \fullref{thm:exponential_trigonometric_identities/eulers_formula}, we have
  \begin{equation*}
    \exp(i\pi) = \cos(\pi) + i\sin(\pi) = -1.
  \end{equation*}

  \SubProofOf{thm:exponential_function_properties/derivative} Follows from \fullref{thm:power_series_are_locally_uniform_convergent} and \fullref{thm:derivative_limit_exchange}.

  \SubProofOf{thm:exponential_function_properties/homomorphism} The Cauchy product of \( \exp(x) \) and \( \exp(y) \) is
  \begin{balign*}
    \exp(x) \exp(y)
     & =
    \left( \sum_{k=0}^\infty \frac {x^k} {k!} \right) \left( \sum_{k=0}^\infty \frac {y^k} {k!} \right)
    =                                       \\ &=
    \sum_{k=0}^\infty \sum_{m=0}^k \frac {x^m} {m!} \frac {x^{k-m}} {(k-m)!}
    =                                       \\ &=
    \sum_{k=0}^\infty \frac 1 {k!} \sum_{m=0}^k \binom{k}{m} x^m y^{k-m}
    \overset {\ref{thm:binomial_theorem}} = \\ &=
    \sum_{k=0}^\infty \frac {(x + y)^k} {k!}
    =
    \exp(x + y).
  \end{balign*}

  \SubProofOf{thm:exponential_function_properties/interpolates_power} We use induction\IND on \( n \) to prove \( \exp(n) = e^n \). The case \( \exp(0) = 1 \) is obvious. If we assume that \( \exp(n) = e^n \), by \fullref{thm:exponential_function_properties/homomorphism}, we have
  \begin{equation*}
    \exp(n + 1)
    =
    \exp(n) \exp(1)
    =
    e^n \cdot e
    =
    e^{n+1}.
  \end{equation*}

  Note that this works for negative \( n \) too.

  \SubProofOf{thm:exponential_function_properties/negative_power} Note that
  \begin{equation*}
    1 = \exp(0) = \exp(z - z) = \exp(z) \exp(-z),
  \end{equation*}
  hence
  \begin{equation*}
    \exp(-z) = \frac 1 {\exp(z)}.
  \end{equation*}

  \SubProofOf{thm:exponential_function_properties/real_positive} For \( t > 0 \), the following
  \begin{equation*}
    \exp(t) = \sum_{k=0}^\infty \frac {t^k} {k!}
  \end{equation*}
  is a series of positive real numbers. To see its convergence, we apply \fullref{thm:dalamberts_ratio_test}:
  \begin{equation*}
    \frac {t^k} {k!} \cdot \frac {(k-1)!} {t^{k-1}}
    =
    \frac t k
    \xrightarrow[k \to \infty]{} 0.
  \end{equation*}

  Thus \( \exp(t) \) is a nonnegative real number. Furthermore, since the sequence of partial sums is monotone, \( \exp(t) \) cannot be zero. Hence for \( t > 0 \), we have \( \exp(t) > 0 \).

  Notice that \( \exp(t) \exp(-t) > 0 \), hence if \( \exp(t) > 0 \), then \( \exp(-t) > 0 \).

  \SubProofOf{thm:exponential_function_properties/conjugate} By \fullref{thm:exponential_trigonometric_identities/eulers_formula},
  \begin{balign*}
    \overline{\exp(a + bi)}
     & \overset {\ref{thm:exponential_function_properties/homomorphism}} =
    \overline{\exp(a) \exp(bi)}
    \overset {\ref{thm:exponential_function_properties/real_positive}} =   \\ &=
    \exp(a) \overline{(\cos(b) + i\sin(b))}
    =                                                                      \\ &=
    \exp(a) (\cos(b) - i\sin(b))
    \overset {\ref{thm:power_series_parity}} =                             \\ &=
    \exp(a) (\cos(-b) + i\sin(-b))
    =                                                                      \\ &=
    \exp(a) \exp(-bi)
    =                                                                      \\ &=
    \exp(a - bi)
    =                                                                      \\ &=
    \exp(\overline{a + bi}).
  \end{balign*}

  \SubProofOf{thm:exponential_function_properties/periodic} By \fullref{thm:exponential_function_properties/eulers_identity},
  \begin{equation*}
    \exp(x + 2i\pi) = \exp(x) \exp(2i\pi) = \exp(x).
  \end{equation*}

  Furthermore, this is also the minimal period. If we assume\LEM that \( \sin(x) \) has another period, say \( p \in (0, 2\pi) \), we would have \( \sin(p) = \sin(0) = 0 \) and \fullref{thm:trigonometric_identities/pythagorean_identity} would imply that \( \cos(p) \in \{ -1, 1 \} \). But then \( \cos(p) \) would be an extreme point for \( \cos \), which is not possible because \( \cos \) is convex in \( [0, 2\pi] \) and only has three extremal points -- \( 0, \pi, 2\pi \).

  \SubProofOf{thm:exponential_function_properties/unit_circle} For \( c, t \in \BbbR \) we have
  \begin{equation*}
    \abs{\exp(it)}
    =
    \abs{\cos(t) + i\sin(t)}
    =
    \sqrt{\cos(t)^2 + \sin(t)^2}
    \overset {\eqref{eq:thm:trigonometric_identities/pythagorean_identity}} =
    1.
  \end{equation*}

  Furthermore, if \( r \) is another real number,
  \begin{equation}
    \exp(ir)
    =
    \exp(i(t + (r - t)))
    =
    \exp(it) \exp(i(r - t)).
  \end{equation}

  It follows that \( \exp(ir) \neq \exp(it) \) if and only if \( \exp(i(r - t)) \neq 0 \). If \( t, r \in [c, c + 2\pi) \) and \( t \neq r \), this is satisfied.

  Hence \( t \mapsto \exp(it) \) is indeed an injection of \( [c, c + 2\pi) \) into the unit circle of \( \BbbC \). It is also a surjection because of the intermediate value theorem.

  \SubProofOf{thm:exponential_function_properties/real_bijective} First, assume\LEM that \( e^t \) is not injective on \( \BbbR \). Then there exist \( t, r \in \BbbR \), \( t \neq r \), such that \( e^t = e^r \). By \fullref{thm:exponential_function_properties/real_positive}, both are positive real numbers. In particular, we can divide by \( e^t \) to obtain
  \begin{equation*}
    1
    =
    \frac {e^r} {e^t}
    \overset {\ref{thm:exponential_function_properties/negative_power}} =
    =
    e^r e^{-t}
    \overset {\ref{thm:exponential_function_properties/homomorphism}} =
    e^{r - t}.
  \end{equation*}

  We know that \( e^0 = 1 \) from \fullref{thm:exponential_function_properties/interpolates_power}. Thus it is enough to show that \( e^t = 1 \) if and only if \( t = 0 \).

  Assume\LEM that \( e^t = 1 \) holds for some \( t > 0 \). The partial sums are monotonely increasing so in order for them to converge to \( 1 \), for any fixed index \( n \) we must have
  \begin{balign*}
    0  & \leq \sum_{k=0}^n \frac {t^k} {k!} = 1 + \sum_{k=1}^n \frac {t^k} {k!} \leq 1, \\
    -1 & \leq \sum_{k=1}^n \frac {t^k} {k!} \leq 0.
  \end{balign*}

  But \( \sum_{k=1}^n \frac {t^k} {k!} > 0 \) because \( t > 0 \). The obtained contradiction proves that \( e^t \neq 1 \) for positive \( t \).

  For negative \( t \), note that
  \begin{equation*}
    e^t e^{-t} = 1.
  \end{equation*}

  Since \( -t \) is positive, \( e^{-t} \neq 1 \) and hence \( e^t \neq 1 \).

  Therefore the function \( t \mapsto e^t \) is injective on \( \BbbR \). It is also surjective onto \( \BbbR^{>0} \) because of the intermediate value theorem.

  \SubProofOf{thm:exponential_function_properties/bijective} Fix \( a + bi \in S_c \), that is, \( b \in [c, c + 2\pi) \). By \fullref{thm:exponential_function_properties/homomorphism},
  \begin{equation*}
    e^{a + bi} = e^a e^{bi}.
  \end{equation*}

  By \fullref{thm:exponential_function_properties/unit_circle}, \( b \mapsto e^{bi} \) is injective for \( b \in [c, c + 2\pi) \) and by \fullref{thm:exponential_function_properties/real_bijective}, \( a \mapsto e^a \) is injective on \( \BbbR \). It follows that their product is also injective.

  \SubProofOf{thm:exponential_function_properties/compound_interest}\mcite\cite[3.31]{Rudin1976Principles}By \fullref{thm:binomial_theorem},
  \begin{balign*}
    \left(1 + \frac t n \right)^n
     & =
    \sum_{k=0}^n \binom{n}{k} \left(\frac t n\right)^k 1^{n-k}
    =    \\ &=
    \sum_{k=0}^n \frac {n!} {(n-k)! k!} \frac {t^k} {n^k}
    =    \\ &=
    \sum_{k=0}^n \frac {n!} {(n-k)! n^k} \frac {t^k} {k!}
    =    \\ &=
    \sum_{k=0}^n \left[ \prod_{j=1}^k \left(1 - \frac {k+j} n \right) \right] \frac {t^k} {k!}.
  \end{balign*}

  Fix an index \( m \). Since the series is nonnegative, there exists an index \( N \) such that for \( n \geq N \)
  \begin{equation*}
    \sum_{k=0}^m \frac {t^k} {k!}
    \leq
    \sum_{k=0}^n \left[ \prod_{j=1}^k \left(1 - \frac {k+j} n \right) \right] \frac {t^k} {k!}.
  \end{equation*}

  Note that
  \begin{equation*}
    \left[ \prod_{j=1}^k \left(1 - \frac {k+j} n \right) \right] \frac {t^k} {k!}
    \leq
    \frac {t^k} {k!},
  \end{equation*}
  hence
  \begin{equation*}
    \sum_{k=0}^m \frac {t^k} {k!}
    \leq
    \sum_{k=0}^n \left[ \prod_{j=1}^k \left(1 - \frac {k+j} n \right) \right] \frac {t^k} {k!}
    \leq
    \sum_{k=0}^n \frac {t^k} {k!}.
  \end{equation*}

  By \fullref{thm:squeeze_lemma},
  \begin{equation*}
    \lim_{n \to \infty} \left(1 + \frac t n \right)^n
    =
    \lim_{n \to \infty} \sum_{k=0}^n \frac {t^k} {k!}
    =
    \exp(t).
  \end{equation*}
\end{proof}

\begin{definition}\label{def:logarithm}
  Fix \( c \in \BbbR \). Unless specified otherwise, we assume \( c = 0 \).

  We define the \term{natural logarithm} \( \log(x) \) as \hyperref[def:function/inverse]{inverse function} of \( e^x \) from \( \BbbC \setminus \{ 0 \} \) to the strip \( S_c \coloneqq \{ a + bi \colon c \leq b < c + 2\pi \} \).

  We also define the \term{base \( b \) logarithm} \( \log_b(x) \) for \( b > 0 \) over the same domain as
  \begin{equation*}
    \log_b(x) \coloneqq \frac {\log(x)} {\log(b)}.
  \end{equation*}
\end{definition}
\begin{proof}
  The well-definedness follows from \fullref{thm:exponential_function_properties/bijective}.
\end{proof}

\begin{proposition}\label{thm:logarithm_properties}
  \hfill
  \begin{thmenum}
    \labitem{thm:logarithm_properties/homomorphism} \( \log(xy) = \log(x) \log(y) \)
  \end{thmenum}
\end{proposition}
\begin{proof}
  \SubProofOf{thm:logarithm_properties/homomorphism} Follows from \fullref{thm:exponential_function_properties/homomorphism}.
\end{proof}

\begin{definition}\label{def:power_function}
  For each positive real number \( y > 0 \), we define the \term{power function}
  \begin{equation*}
    x^y \coloneqq e^{y \ln x}
  \end{equation*}
  as a function of \( x \).
\end{definition}

\begin{proposition}\label{thm:power_function_properties}
  \hfill
  \begin{thmenum}
    \labitem{thm:power_function_properties/composition} \( (x^y)^z = x^{yz} \).
    \labitem{thm:power_function_properties/derivative} \( D_x(x^y) = \log(x) x^y \).
  \end{thmenum}
\end{proposition}
\begin{proof}
  \SubProofOf{thm:power_function_properties/composition}
  \begin{equation*}
    (x^y)^z
    =
    e^{z \log(e^{y \log(x)})}
    =
    e^{z y \log(x)}
    =
    x^{yz}.
  \end{equation*}

  \SubProofOf{thm:power_function_properties/derivative} Using the chain rule for differentiation, we obtain
  \begin{equation*}
    D_x(x^y) = D_x(e^{\log(y) x}) = \log(x) e^{\log(y) x} = \log(x) x^y.
  \end{equation*}
\end{proof}
\begin{proposition}\label{thm:exponential-trigonometric_identities}
  We have the following exponential-trigonometric identities:
  \labitem{thm:exponential_trigonometric_identities/eulers_formula} (Euler's formula) For any \( z \in \BbbC \),
  \begin{equation}\label{thm:exponential_trigonometric_identities/eulers_formula/identity}
    e^{iz} = \cos(z) + i \sin(z).
  \end{equation}

  \labitem{thm:exponential_trigonometric_identities/inverse_eulers_formula} (Inverse Euler's identities) For any \( z \in \BbbC \),
  \begin{balign}
    \sin(z) & = \real(e^z) = \frac {e^{iz} - e^{-iz}} {2i} \label{thm:exponential_trigonometric_identities/inverse_eulers_formula/sin} \\
    \cos(z) & = \imag(e^z) = \frac {e^{iz} + e^{-iz}} 2 \label{thm:exponential_trigonometric_identities/inverse_eulers_formula/cos}
  \end{balign}

  \labitem{thm:exponential_trigonometric_identities/de_moivre} (De Moivre's formula) For any complex number \( z \) and any nonnegative integer \( n \),
  \begin{equation}\label{thm:exponential_trigonometric_identities/de_moivre/identity}
    (\cos(z) + i \sin(z))^n = \cos(nz) + i \sin(nz).
  \end{equation}
\end{proposition}
\begin{proof}
  \SubProofOf{thm:exponential_trigonometric_identities/eulers_formula} Simply note that \fullref{def:exponential_function} is a termwise sum of \fullref{def:trigonometric_functions/sine} and \fullref{def:trigonometric_functions/cosine}, therefore \fullref{thm:exponential_trigonometric_identities/eulers_formula/identity} holds.

  \SubProofOf{thm:exponential_trigonometric_identities/inverse_eulers_formula} Follows from \fullref{thm:exponential_trigonometric_identities/eulers_formula}.

  \SubProofOf{thm:exponential_trigonometric_identities/de_moivre} From \fullref{thm:exponential_trigonometric_identities/eulers_formula},
  \begin{equation*}
    (\cos(z) + i \sin(z))^n
    =
    {e^{iz}}^n
    \overset {\ref{thm:power_function_properties/composition}} {=}
    =
    e^{i(zn)}
    =
    \cos(nz) + i \sin(nz).
  \end{equation*}
\end{proof}
