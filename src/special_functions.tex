\subsection{Special functions}\label{subsec:special_functions}

\begin{definition}\label{def:gamma_function}
  The \term{Gamma function} is
  \begin{equation*}
    \begin{aligned}
      \Gamma: \set{ z \in \BbbC \given \real(z) > 0 } \to \BbbC \\
      \Gamma(z) \coloneqq \int_0^\infty x^{z-1} e^{-x} \dl x
    \end{aligned}
  \end{equation*}
\end{definition}

\begin{theorem}[Stirling's gamma approximation]\label{thm:stirlings_gamma_approximation}\mcite[\textnumero 257]{Фихтенгольц1968Том2}
  For every \hyperref[rem:peano_arithmetic_zero/nonnegative]{nonnegative integer} \( n \) there exists some constant \( \theta \in (0, 1) \) such that
  \begin{equation*}
    \Gamma(n + 1) = \sqrt{2 \pi n} \parens*{ \frac n e }^n \cdot e^{\frac \theta {12n}}.
  \end{equation*}
\end{theorem}
