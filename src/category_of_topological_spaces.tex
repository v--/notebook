\subsection{Category of topological spaces}\label{subsec:category_of_topological_spaces}

\begin{definition}\label{def:category_of_small_frames}\mcite[43]{Johnstone1983}
  Suppose that we are given a \hyperref[def:grothendieck_universe]{Grothendieck universe} \( \mscrU \), which is safe to assume to be the smallest suitable one as explained in \fullref{def:large_and_small_sets}. We describe the \term{category of \( \mscrU \)-small frames} as the following \hyperref[rem:concrete_categories]{concrete category}:

  \begin{itemize}
    \item The \hyperref[def:category/objects]{objects} are the \( \mscrU \)-small \hyperref[def:semilattice/complete]{complete lattices} in which arbitrary meets distribute over finite joins, i.e. \eqref{eq:def:semilattice/distributive_lattice/arbitrary/meet_over_join} holds. We call such lattices \term{frames}.

    \item The \hyperref[def:category/morphisms]{morphisms} between two frames are the functions between them preserving finite meets and arbitrary joins. We call such functions \term{frame homomorphisms}.
  \end{itemize}
\end{definition}

\begin{proposition}\label{thm:topological_spaces_are_frames}
  The topology of a topological space is a \hyperref[def:category_of_small_frames]{frame}.
\end{proposition}
\begin{proof}
  Trivial.
\end{proof}

\begin{remark}\label{rem:topology_frame_homomorphism}
  Consider the continuous function \( f: X \to Y \) between topological spaces.

  \Fullref{thm:def:function_preimage/union} and \fullref{thm:def:function_preimage/intersection} imply that the inverse \( f^{-1}: Y \to X \) preserves arbitrary unions and intersections and is hence a \hyperref[def:category_of_small_frames]{frame homomorphism} from \( \mscrT_Y \) to \( \mscrT_X \).

  This motivates defining \hyperref[def:category_of_small_locales]{locales}.
\end{remark}

\begin{definition}\label{def:category_of_small_locales}\mcite[43]{Johnstone1983}
  We call the \hyperref[def:opposite_category]{opposite category} of the \hyperref[def:category_of_small_frames]{category of \( \mscrU \)-small frames} the \term{category of \( \mscrU \)-small locales}.
\end{definition}

\begin{remark}\label{rem:picking_a_point_from_a_locale}
  The topology of the one-element topological space is a \hyperref[thm:binary_boolean_algebra]{two-element Boolean algebra}.

  The \hyperref[def:category_of_small_locales]{locale homomorphism} \( f^{\opcat}: \set{ \top, \bot } \to L \) then corresponds to a continuous function from the one-element space to some other space \( X \) whose topology is isomorphic to \( L \). This reduces to picking a point from \( X \).
\end{remark}

\begin{lemma}\label{thm:frame_homomorphism_kernel}\mcite[45]{Johnstone1983}
  Fix a \hyperref[thm:binary_boolean_algebra]{two-element Boolean algebra} \( \set{ \top, \bot } \). It is vacuously a \hyperref[def:category_of_small_locales]{frame}. Fix also an arbitrary locale \( L \).

  Then the \hyperref[def:semilattice/homomorphism]{lattice homomorphisms} \( f: L \to \set{ \top, \bot } \) is a \hyperref[def:category_of_small_frames]{frame homomorphism} if and only if \( f^{-1}(\top) \) is a \hyperref[def:lattice_ideal/prime]{completely prime filter}.
\end{lemma}
\begin{proof}
  \SufficiencySubProof Suppose that \( f: L \to \set{ \top, \bot } \) is a frame homomorphism.

  We will first show that \( f^{-1}(\top) \) is a filter.
  \begin{itemize}
    \item It is closed under meets. Let \( f(a) = f(b) = \top \). Then, since \( f \) preserves meets,
    \begin{equation*}
      f(a \wedge b) = f(a) \wedge f(b) = \top \wedge \top = \top.
    \end{equation*}

    \item It is closed under joins with elements of \( L \). Let \( a \in L \) and \( f(b) = \top \). Then
    \begin{equation*}
      f(a \vee b) = f(a) \vee f(b) = f(a) \vee \top = \top.
    \end{equation*}
  \end{itemize}

  It remains to show that \( f^{-1}(\top) \) is completely prime. Let
  \begin{equation*}
    \top = f\parens*{ \bigvee_{k \in \mscrK} a_k } = \bigvee_{k \in \mscrK} f(a_k).
  \end{equation*}

  If \( f(a_k) = \bot \) for every \( k \in \mscrK \),
  \begin{equation*}
    \bigvee_{k \in \mscrK} f(a_k) = \bot.
  \end{equation*}

  Hence, there exists some \( k \in \mscrK \) such that \( f(a_k) = \top \).

  Therefore, \( f^{-1}(\top) \) is a completely prime filter.

  \NecessitySubProof Suppose that \( f \) is a lattice homomorphism and that \( f^{-1}(\top) \) is a completely prime filter. We will show that \( f \) preserves arbitrary joins.

  Let \( \seq{ a_k }_{k \in \mscrK} \) be some family of members of \( L \).
  \begin{itemize}
    \item Suppose that \( f(a_k) = \bot \) for each \( k \in \mscrK \).

    If \( f\parens*{ \bigvee_{k \in \mscrK} a_k } = \top \), since \( f^{-1}(\top) \) is completely prime filter, there exists some \( k_0 \in \mscrK \) such that \( f(a_{k_0}) = \top \). The obtained contradiction shows that
    \begin{equation*}
      f\parens*{ \bigvee_{k \in \mscrK} a_k } = \bot = \bigvee_{k \in \mscrK} f(a_k)
    \end{equation*}

    \item Suppose that \( f(a_{k_0}) = \top \) for some \( k_0 \in \mscrK \).

    Then, since \( f \) preserves finite joins,
    \begin{equation*}
      f\parens*{ \bigvee_{k \in \mscrK} a_k }
      =
      f\parens*{ \bigvee_{k \neq k_0} a_k } \vee f(a_{k_0})
      =
      f\parens*{ \bigvee_{k \neq k_0} a_k } \vee \top
      =
      \top
      =
      \bigvee_{k \in \mscrK} f(a_k).
    \end{equation*}
  \end{itemize}
\end{proof}

\begin{remark}\label{rem:prime_elements_of_locale}
  We discussed in \fullref{rem:picking_a_point_from_a_locale} how to \enquote{pick a point} from a locale via locale maps from the two-element locale. \Fullref{thm:frame_homomorphism_kernel} and \fullref{def:lattice_prime_element} imply that these functions correspond exactly to prime elements of the locale.
\end{remark}

\begin{proposition}\label{thm:locale_to_topology}\mcite[45]{Johnstone1983}
  We define the set \( X \) of \term{points} of a \hyperref[def:locale]{locale} \( L \) as the set of all \hyperref[def:lattice_prime_element]{prime elements} of \( L \).

  Now the elements of \( L \) must become neighborhoods of these points. For each \( a \in L \), define the \enquote{closed} set
  \begin{equation*}
    F_a \coloneqq \set{ x \in X \given x \leq a }.
  \end{equation*}

  Finally, define
  \begin{equation*}
    \mscrF \coloneqq \set{ F_a \given a \in L }.
  \end{equation*}

  Then \( \mscrF \) is a family of closed sets for a topology on \( X \).
\end{proposition}
\begin{proof}
  The union and intersection of an arbitrary family of \enquote{closed} sets is again closed because \( L \) is complete. We only need to show \ref{def:topological_space/F2}.

  First note that \( \bot \) is vacuously a prime element and that \( L \) has no elements strictly less than \( \bot \). Hence, \( F_\bot = \varnothing \) and thus \( \mscrF \) contains the empty set.

  The top \( \top \) is also prime, and \( F_\top = L \), implying that \( \mscrF \) also contains the set of all points.

  Therefore, \( \mscrF \) induces a topology on \( X \).
\end{proof}
