\subsection{Analytic geometry}\label{subsec:analytic_geometry}

\begin{remark}\label{remark:analytic_geometry}
  Analytic geometry is a XVII-century branch of mathematics that studies geometric figures using coordinate systems\Tinyref{remark:coordinate_systems}. The term \enquote{analytic geometry} may refer to a modern subbranch of algebraic geometry, however we refrain from using \enquote{analytic geometry} in that sense. Historically, most of these definitions were given either for the Euclidean plane\Tinyref{def:euclidean_plane} or for the three-dimensional Euclidean space\Tinyref{def:euclidean_plane}\Tinyref{remark:coordinate_systems/euclidean_space}.

  Most of the definitions extend to arbitrary-dimensional vector spaces. We will state definitions in the language of linear algebra and refrain from using synthetic (axiomatic) geometry. When working in the plane (resp. three-dimensional space), we will assume that we have fixed an orthonormal\Tinyref{def:orthonormal_system} coordinate system\Tinyref{def:euclidean_plane_coordinate_system} \( Oxy \) (resp. \( Oxyz \)), which allows us to visualize geometric figures.
\end{remark}

Unless noted otherwise, \( X \) will be a vector space over \( \R \).

\begin{definition}\label{def:geometric_figure}
  A \Def{geometric figure} is an informal notion that refers to certain special subsets of a vector space, usually defined in a coordinate-independent manner. This subsection lists some important figures. If two figures intersect as sets, we say that they are \Def{incident}. The simplest figures are the \Def{points}, which are simply the singleton sets - see \cref{def:simplex/point}. We use the convention \cref{remark:singleton_sets} and, unless the distinction is important, we do not distinguish between singleton sets and their only element.
\end{definition}

\begin{definition}\label{def:zero_locus}
  Let \( S \) be an arbitrary set and let \( M \) be a unital\Tinyref{def:algebraic_theory/identity} magma\Tinyref{def:magma/magma} with identity \( e \).

  The \Def{zero locus} or \Def{set of zeros} of a function \( f: S \to M \) is the preimage
  \begin{equation*}
    f^{-1}(e) = \{ x \in X \colon f(x) = e \}.
  \end{equation*}

  In practice, \( M \) is usually a ring\Tinyref{def:semiring/ring} or a module\Tinyref{def:left_module}, in which case the zero locus is defined for the additive group, i.e.
  \begin{equation*}
    f^{-1}(0_M) = \{ x \in X \colon f(x) = 0_M \}.
  \end{equation*}

  See also \cref{def:unital_magma_kernel}, \cref{def:semiring_kernel} and \cref{def:left_module_kernel}.
\end{definition}

\begin{definition}\label{def:geometric_curve}
  A \Def{curve} can have different meanings depending on the context. Some important definitions are

  \begin{defenum}
    \DItem{def:geometric_curve/parametric_curve}\cite[definition 1.2]{Иванов2017} In \cref{def:parametric_curve}, we defined a curve to be an continuous function \( \gamma: I \to Y \) from a real interval \( I \) into a topological space \( Y \). We usually want \( Y \) to be a manifold\Tinyref{def:topological_manifold}.

    \Cref{def:parametric_curve} already discusses how the term \enquote{curve} can refer to \( \gamma \) itself, to its image or to an equivalence class of curves.

    \DItem{def:geometric_curve/function_graph}\cite[definition 1.20]{Иванов2017} The graph\Tinyref{def:function_graph} \( \Gph(\gamma) \) of a parametric curve is a curve in the topological space \( I \times Y \).

    \DItem{def:geometric_curve/implicit}\cite[definition 1.24]{Иванов2017} Let \( M \) be the zero locus\Tinyref{def:zero_locus} of \( F: X \to \K \). If there exists a curve \( \gamma: I \to X \) such that \( \Im(M) = I \), we call \( M \) an \Def{implicit curve}.

    \DItem{def:geometric_curve/smooth} A \Def{smooth curve} is a Frechet\Tinyref{def:differentiability/frechet} differentiable parametric curve \( \gamma: I \to X \) over a Banach space\Tinyref{def:banach_space} \( X \). A \Def{regular curve} is a smooth curve whose derivative is nonzero.

    \DItem{def:geometric_curve/algebraic} An \Def{algebraic curve} is an affine variety of dimension one\Tinyref{def:affine_variety}. In the simplest case that will interest us here, a plain algebraic curve will be the zero locus of a bivariate polynomial \( p(x, y) \).
  \end{defenum}
\end{definition}

\begin{definition}\label{def:geometric_line}
  We give several equivalent definitions of a \Def{line} in a real vector space \( X \).

  \begin{defenum}
    \DItem{def:geometric_line/parametric} A \Def{line} in \( X \) is a parametric curve \( l: X \to X \) of the form
    \begin{equation}\label{def:geometric_line/parametric_equation}
      l(t) = tx + a,
    \end{equation}
    where \( x, a \in X \) and \( t \in \R \).

    We call \cref{def:geometric_line/parametric_equation} the \Def{vector parametric equation} and \( x \) the \Def{directional vector} of the line.

    If \( X = \R^n \), we can also define the \Def{scalar parametric equations}
    \begin{equation}\label{def:geometric_line/scalar_parametric_equations}
      \begin{cases}
        &l_1(t) = t x_1 + r_1 \\
        &\vdots \\
        &l_n(t) = t x_n + r_n
      \end{cases}
    \end{equation}

    \DItem{def:geometric_line/ray} If we restrict \( t \geq 0 \) or \( t \leq 0 \) in \cref{def:geometric_line/parametric_equation}, we get a \Def{closed ray} with a vertex at \( a \) instead of a line and an \Def{open ray} if the inequalities are strict.

    \DItem{def:geometric_line/algebraic_curve} A \Def{line} in \( X \) is an algebraic curve\Tinyref{def:geometric_curve/algebraic} of a degree \( 1 \) polynomial. In the plane, this is the zero locus of a function \( p: \R^2 \to \R \) of the form
    \begin{equation}\label{def:geometric_line/general_equation}
      p(x, y) \coloneqq Ax + By + C = 0,
    \end{equation}
    where either \( A \neq 0 \) or \( B \neq 0 \).

    We call \cref{def:geometric_line/general_equation} the \Def{general equation} or simply \Def{equation} of a line in a plane. Note that multiple general equations can have the same locus (e.g. the entire ideal \( \Gen{p} \)). We can select a concrete representative: if \( A^2 + B^2 = 1 \), we call \cref{def:geometric_line/general_equation} a \Def{normal equation}.

    If \( B \neq 0 \), \cref{def:geometric_line/general_equation} is equivalent to the \Def{Cartesian equation} of the form
    \begin{equation}\label{def:geometric_line/cartesian_equation}
      y = kx + m,
    \end{equation}
    where \( k, m \in \R \) and the \Def{slope} \( k \neq 0 \) (in our case, \( k = -\tfrac A B \) and \( m = -\tfrac C B \)) and, if \( A, B, C \neq 0 \), to the \Def{intercept equation} of the form
    \begin{equation}\label{def:geometric_line/intercept_equation}
      \frac x a + \frac y b = 1,
    \end{equation}
    where \( a, b \in \R \setminus \{ 0 \} \) (in our case, \( a = -\tfrac C A \) and \( b = -\tfrac C B \)).

    \begin{figure}
      \centering
      \begin{mplibcode}
        input metapost/plotting;

        u := 1.5cm;

        beginfig(1);
          path l, x_axis, y_axis;

          x_axis = (-1, 0) scaled u -- (1, 0) scaled u;
          y_axis = (0, -1) scaled u -- (0, 1) scaled u;
          l = (-1 / 2, -1) * u -- (1, 3 / 4) * u;

          drawarrow x_axis;
          label.bot("$x$", point 0.9 of x_axis);
          drawarrow y_axis;
          label.lft("$y$", point 0.9 of y_axis);
          draw l;
          label.lrt("$y = kx + m$", endpoint of l);
        endfig;
      \end{mplibcode}
      \caption{A line in \( \R^2 \) defined using its Cartesian equation}\label{def:geometric_line/cartesian_equation_drawing}
    \end{figure}
  \end{defenum}
\end{definition}

\begin{definition}\label{def:geometric_cone}
  An open (resp. closed) \Def{cone} is a collection of open (resp. closed) rays\Tinyref{def:geometric_line/ray} with a common vertex, called the \Def{vertex} of the cone.

  Unless explicitly noted otherwise, we assume that the vertex of the cone is \( 0 \) because every cone is a translation of a cone centered at \( 0 \).
\end{definition}

\begin{definition}\label{def:angle}
  A \Def{directed angle} is a tuple of two closed rays\Tinyref{def:geometric_line/ray} with a common vertex. It is a closed cone.

  Suppose that the rays have scalar parametric equations
  \begin{equation*}
    t \mapsto
    \begin{cases}
      tx_i + a_i \\
      ty_i + b_i,
    \end{cases}
    i = 1, 2.
  \end{equation*}

  The condition of the rays having a common vertex is equivalent to \( a_1 = a_2 \) and \( b_1 = b_2 \). If not specified otherwise, we assume that \( a_1 = a_2 = b_1 = b_2 = 0 \).

  \begin{figure}
    \centering
    \begin{mplibcode}
      input metapost/plotting;
      u := 1.5cm;

      beginfig(1)
        drawarrow (-1 / 2, 0) scaled u -- (2, 0) scaled u;
        drawarrow (0, -1 / 2) scaled u -- (0, 2) scaled u;

        z0 = (1 / 2, 1 / 6) scaled u;
        z1 = (2, 11 / 12) scaled u;
        z2 = (1, 13 / 6) scaled u;

        draw z0 -- (x0, max(y1, y2)) dashed withdots;

        drawarrow z0 -- z1;
        draw (x0, y1) -- z1 dashed evenly;
        label.top("$x_1$", midpoint of ((x0, y1) -- z1));

        drawarrow z0 -- z2;
        draw (x0, y2) -- z2 dashed evenly;
        label.bot("$x_2$", midpoint of ((x0, y2) -- z2));
      endfig;
    \end{mplibcode}
    \caption{An angle with the measurement segments marked.}\label{def:angle/figure}
  \end{figure}

  The \Def{measure} of a directed angle, often called the angle itself, can be defined as the number
  \begin{equation*}
    \alpha \coloneqq \Rem\left(\arccos\left(\frac {x_2} {\sqrt{x_2^2 + y_2^2}} \right) - \arccos\left(\frac {x_1} {\sqrt{x_1^2 + y_1^2}} \right), 2\pi \right).
  \end{equation*}

  We can classify angles based on their measure as
  \begin{defenum}
    \DItem{def:angle/zero} \Def{zero} if \( \alpha = 0 \),
    \DItem{def:angle/acute} \Def{acute} if \( \alpha < \tfrac \pi 2 \),
    \DItem{def:angle/right} \Def{right} if \( \alpha = \tfrac \pi 2 \),
    \DItem{def:angle/obtuse} \Def{obtuse} if \( \alpha \in (\tfrac \pi 2, \pi) \),
    \DItem{def:angle/straight} \Def{straight} if \( \alpha = \pi \), in which case the angle is actually a line,
    \DItem{def:angle/reflex} \Def{reflex} if \( \alpha > \pi \).
  \end{defenum}

  We often do not care about the order of the two rays and speak of an \Def{undirected angle}. In this case, the measure of the undirected angle is the smaller of the measures of the two oriented angles. Thus we can only speak of straight and reflex directed angles.
\end{definition}

\begin{definition}\label{def:hyperplane}
  A \Def{linear hyperplane} or simply \Def{hyperplane} in \( X \) is the zero locus\Tinyref{def:zero_locus} of some linear functional.

  An \Def{affine hyperplane} in \( X \) is the zero locus of an affine functional\Tinyref{def:affine_operator}, that is, the zero locus of a function of the form
  \begin{equation*}
    \begin{array}{l}
      l: X \to \R \\
      l(x) = \Prod {x^*} x + a
    \end{array}
  \end{equation*}
  for some \( x^* \in X^* \) and \( a \in \R \).

  Any hyperplane gives rise to two half-spaces
  \begin{equation*}
    H^+ \coloneqq \{ l(x) \geq 0 \} = \{ \Prod {x^*} x \geq -a \}
  \end{equation*}
  and
  \begin{equation*}
    H^- \coloneqq \{ l(x) \leq 0 \} = \{ \Prod {x^*} x \leq -a \}.
  \end{equation*}

  Note that \( X = \Img(l) \cup H^+ \cup H^- \).
\end{definition}

\begin{example}\label{ex:hyperplanes}
  Affine hyperplanes in \( \R^2 \) are lines\Tinyref{def:geometric_line} and affine hyperplanes in \( \R^3 \) are planes\Tinyref{remark:coordinate_systems/plane}.

  Linear hyperplanes in \( \R^2 \) are the lines incident to origin \( (0, 0) \) and linear hyperplanes in \( \R^3 \) are the planes incident to \( (0, 0, 0) \).
\end{example}

\begin{definition}\label{def:polyhedron}
  A \Def{polyhedron} is an intersection of half-spaces\Tinyref{def:hyperplane}.
\end{definition}

\begin{definition}\label{def:quadratic_plane_curve}
  The \Def{quadratic plane curves} are algebraic curves\Tinyref{def:geometric_curve/algebraic} given by a bivariate polynomial of degree \( 2 \). The \Def{general equation} of a quadratic plane curve is
  \begin{equation}\label{def:quadratic_plane_curve/general_equation}
    c(x, y) \coloneqq A x^2 + B xy + C y^2 + Dx + Ey + F = 0.
  \end{equation}

  Multiple equation can correspond to the same locus. Furthermore, unlike with lines\Tinyref{def:geometric_line/algebraic_curve}, we sometimes wish to classify quadratic curves up to linear bijections. We are usually interested in some canonical form of the equation that is unique for the curve. The quadratic curves that are interesting to use are
  \begin{defenum}
    \DItem{def:quadratic_plane_curve/ellipse} An \Def{ellipse} is a quadratic curve whose canonical equation has the form
    \begin{equation}\label{def:quadratic_plane_curve/ellipse/canonical_equation}
      c(x, y) \coloneqq \frac {x^2} {a^2} + \frac {x^y} {b^2} - 1 = 0.
    \end{equation}

    If \( a = b \), we say that the ellipse is a \Def{circle}. Circles generalize to spheres\Tinyref{def:metric_space/sphere} in metric spaces.

    We are often interested in defining ellipses via \Def{scalar parametric equations} using trigonometric functions\Tinyref{def:trigonometric_functions} as follows:
    \begin{equation}\label{def:quadratic_plane_curve/ellipse/parametric_equations}
      \begin{cases}
        x = a \cos(t) \\
        y = b \sin(t),
      \end{cases}
    \end{equation}
    where \( t \in [0, 2\pi) \).

    \begin{figure}
      \centering
      \begin{mplibcode}
        input metapost/plotting;

        vardef scaled_cos(expr x) =
          2 * cos(x)
        enddef;

        beginfig(1)
          fill dot shifted (2u, 0);

          drawarrow (-pi, 0) scaled u -- (pi, 0) scaled u;
          drawarrow (0, -pi / 2) scaled u -- (0, pi / 2) scaled u;

          drawarrow path_of_curve(scaled_cos, sin, -1 / 4 * pi, 3 / 4 * pi, 0.01, u);
          drawarrow path_of_curve(scaled_cos, sin, 3 / 4 * pi, 7 / 4 * pi, 0.01, u);
        endfig;
      \end{mplibcode}
      \caption{An ellipse defined via its parametric equations. The starting point is highlighted and the direction of the parametric curves is shown.}\label{def:quadratic_plane_curve/ellipse/parametric_equations_figure}
    \end{figure}

    \DItem{def:quadratic_plane_curve/hyperbola} A \Def{hyperbola} is a quadratic curve whose canonical equation has the form
    \begin{equation}\label{def:quadratic_plane_curve/hyperbola/canonical_equation}
      c(x, y) \coloneqq \frac {x^2} {a^2} + \frac {y^2} {b^2} + 1 = 0.
    \end{equation}

    Similarly to ellipses, we are often interested in defining hyperbolas via \Def{scalar parametric equations} using hyperbolic trigonometric functions\Tinyref{def:hyperbolic_trigonometric_functions} as follows:
    \begin{equation}\label{def:quadratic_plane_curve/hyperbola/parametric_equations}
      \begin{cases}
        x = a \cosh(t) \\
        y = b \sinh(t),
      \end{cases}
    \end{equation}
    where \( t \in \R \). This only defines the \Def{right part} of the hyperbola. The left part is defined by replacing \( a \) with \( -a \).

    \begin{figure}
      \centering
      \begin{mplibcode}
        input metapost/plotting;

        vardef minus_cosh(expr x) =
          -cosh(x)
        enddef;

        beginfig(1)
          drawarrow (-pi / 2, 0) scaled u -- (pi / 2, 0) scaled u;
          drawarrow (0, -pi / 2) scaled u -- (0, pi / 2) scaled u;

          drawarrow path_of_curve(cosh, sinh, -pi / 3, 0, 0.01, u);
          drawarrow path_of_curve(cosh, sinh, 0, pi / 3, 0.01, u);

          drawarrow path_of_curve(minus_cosh, sinh, -pi / 3, 0, 0.01, u);
          drawarrow path_of_curve(minus_cosh, sinh, 0, pi / 3, 0.01, u);
        endfig;
      \end{mplibcode}
      \caption{A hyperbola defined via its parametric equations.}\label{def:quadratic_plane_curve/hyperbola/parametric_equations_figure}
    \end{figure}

    \DItem{def:quadratic_plane_curve/parabola} A \Def{parabola} is a quadratic curve whose canonical equation has the form
    \begin{equation}\label{def:quadratic_plane_curve/parabola/canonical_equation}
      c(x, y) \coloneqq y^2 - 2px = 0,
    \end{equation}
    where \( p \neq 0 \).

    Unlike ellipses and hyperbolas, we can define \( y \) as a function of \( x \) separately for the lower half-plane and upper half-plane:
    \begin{equation}\label{def:quadratic_plane_curve/parabola/cartesian_equation}
      y(x) = \pm \sqrt{2px}.
    \end{equation}

    \begin{figure}
      \centering
      \begin{mplibcode}
        input metapost/plotting;

        beginfig(1)
          fill dot;

          drawarrow (-pi / 2, 0) scaled u -- (pi / 2, 0) scaled u;
          drawarrow (0, -pi / 2) scaled u -- (0, pi / 2) scaled u;

          vardef y_upper(expr x) =
            sqrt(x)
          enddef;

          vardef y_lower(expr x) =
            -sqrt(x)
          enddef;

          drawarrow path_of_plot(y_upper, 0, pi / 3, 0.01, u);
          drawarrow path_of_plot(y_lower, 0, pi / 3, 0.01, u);
        endfig;
      \end{mplibcode}
      \caption{A parabola defined via its parametric equations.}\label{def:quadratic_plane_curve/parabola/parametric_equations_figure}
    \end{figure}
  \end{defenum}

  Ellipses, hyperbolas and parabolas are collectively called \Def{conic sections}.
\end{definition}

\begin{definition}\label{def:convex_set}\mbox{}
  \begin{defenum}
    \DItem{def:convex_set/line_segment} Given two points \( x, y \in X \), we define the \Def{line segment} between \( x \) and \( y \) as the parametric curve \( t \mapsto tx + (1-t)y, t \in [0, 1] \). The image
    \begin{equation*}
      [x, y] \coloneqq \{ tx + (1-t)y \colon t \in [0, 1] \}
    \end{equation*}
    of this parametric curve is called the \Def{convex hull} of \( x \) and \( y \). We usually use the term \enquote{line segment} to refer to the convex hull.

    \DItem{def:convex_set/hull} We define convex hull \( \Conv A \) of a set \( A \subseteq X \) as the union of all line segments with endpoints in \( A \).

    \DItem{def:convex_set/set} We call a set \Def{convex} if it coincides with its convex hull, that is, if it contains the line segment between any two of its points.
  \end{defenum}
\end{definition}

\begin{proposition}\label{thm:convex_set_properties}
  Convex sets\Tinyref{def:convex_set} have the following basic properties:

  \begin{thmenum}
    \DItem{thm:convex_set_properties/closed_under_combinations} A convex set is closed under convex combinations\Tinyref{def:linear_combination/convex}.
    \DItem{thm:convex_set_properties/cone_closed_under_combinations} A closed convex\Tinyref{def:convex_set} cone\Tinyref{def:geometric_cone} is closed under conic combinations\Tinyref{def:linear_combination/conic}.
    \DItem{thm:convex_set_properties/closed_under_intersections} Any intersection of convex sets is convex.
  \end{thmenum}
\end{proposition}
\begin{proof}
  \begin{description}
    \RItem{thm:convex_set_properties/closed_under_combinations} Fix a convex set \( C \). Let \( \sum_{k=1}^n t_k x_k \) be a convex combination of elements of \( C \).

    We will use induction\IND on \( n \). If \( n = 1 \), this is obvious. If \( n = 2 \), this is given by definition. Assume that it is true for \( n - 1 \). Denote \( s \coloneqq \sum_{k=1}^{n-1} t_k \). If \( s = 0 \), take another convex combination in order to handle all the possible cases of the induction. Suppose \( s \neq 0 \). Then
    \begin{equation*}
      \sum_{k=1}^n t_k x_k
      =
      s \sum_{k=1}^n \frac {t_k} s x_k
      =
      s \underbrace{\sum_{k=1}^{n-1} \frac {t_k} s x_k}_{\eqqcolon y} + t_n x_n.
    \end{equation*}

    By the inductive hypothesis, \( y \in C \). Note that \( s \in [0, 1] \) and that \( s + t_n = 1 \) by definitions of \( s \). Then \( s y + t_n x_n \) is a binary convex combination that we know is contained in \( C \) by definition.

    \RItem{thm:convex_set_properties/cone_closed_under_combinations} Fix a cone \( C \). Let \( \sum_{k=1}^n t_k x_k \) be a conic combination of elements of \( C \). Each vector \( x_k \) lies on a closed ray, say \( r_k \), thus \( t_k x_k \) also lies on \( r_k \).

    Therefore we only need to show that the sum of two elements \( x_1, x_2 \in C \) is again in \( C \). This is true because \( x_1 + x_2 \) is a convex combination of \( 2x_1 \in r_1 \) and \( 2x_2 \in r_2 \).

    \RItem{thm:convex_set_properties/closed_under_intersections} Let \( X = \cap_{\alpha \in \A} X_\alpha \) be an intersection of convex sets. Take \( x, y \in X \) and \( t \in [0, 1] \). Then \( tx + (1-t)y \in X_\alpha \) for all \( \alpha \in \A \), hence \( tx + (1-t)y \in X \). Therefore \( X \) is convex.
  \end{description}
\end{proof}

\begin{definition}\label{def:simplex}
  A \( k \)-\Def{simplex} is the convex hull\Tinyref{def:convex_set/hull} of \( k + 1 \) affinely independent\Tinyref{def:left_module_linear_dependence} vectors called the \Def{vertices} of the simplex. The convex hull of any subset of the vertices is called a \Def{face} of the simplex.

  \begin{defenum}
    \DItem{def:simplex/point} A \( 0 \)-simplex is a point as defined in \cref{def:geometric_figure}.
    \DItem{def:simplex/line_segment} A \( 1 \)-simplex is a line segment as defined in \cref{def:convex_set/line_segment}.
    \DItem{def:simplex/triangle} A \( 2 \)-simplex is called a \Def{triangle}.
    \DItem{def:simplex/tetrahedron} A \( 3 \)-simplex is called a \Def{tetrahedron}.
  \end{defenum}
\end{definition}
