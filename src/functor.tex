\begin{definition}\label{def:functor}(\cite[definitions 1.2.1, 1.2.10]{Leinster2014})
  Let $\Cal A$ and $\Cal B$ be categories. A \uline{(covariant) functor} $F: \Cal A \to \Cal B$ consists of:
  \begin{itemize}
    \item a function $\Obj \Cal A \to \Obj \Cal B$, written as $A \mapsto F(A)$.
    \item for each $A, B \in \Obj \Cal A$, a function
    \begin{align*}
      \Cal{A}(A, B) \to \Cal{B}(F(A), F(B)),
    \end{align*}
    written as $f \mapsto F(f)$.
  \end{itemize}
  such that
  \begin{defenum}
    \item\label{def:functor/composition_axiom} $A \overset f \mapsto B \overset g \mapsto C$ implies $F(g \circ f) = F(g) \circ F(f)$.
    \item\label{def:functor/identity_axiom} $A \in \Obj \Cal A$ implies $F(\Id_A) = \Id_{F(A)}$.
  \end{defenum}

  If we replace the axiom~\cref{def:functor/composition_axiom} with
  \begin{defenum}
    \item[b')]\label{def:functor/contravariant_composition_axiom} $A \overset f \mapsto B \overset g \mapsto C$ implies $F(g \circ f) = F(f) \circ F(g)$,
  \end{defenum}
  we call $F$ a \uline{contravariant functor}. Equivalently, $F: \Cal A \to \Cal B$ is contravariant if and only if $F: \Cal{A}^{\Op} \to \Cal B$ is covariant.
\end{definition}

\begin{definition}\label{def:forgetful_functor}(\cite[example 1.2.3]{Leinster2014})
  An informal notion is that of the \uline{forgetful functor}. A functor $F: \Cal A \to \Cal B$ is called forgetful if the images $F(A)$ of objects $A \in \Obj \Cal A$ have \enquote{less structure} than $A$. For example, a functor which sends topological spaces to their underlying sets is forgetful since it \enquote{forgets} about the topological structure.
\end{definition}

\begin{definition}\label{def:free_functor}(\cite[example 1.2.4]{Leinster2014})
  Another informal notion, which is dual to \cref{def:forgetful_functor}, is that of a \uline{free functor}. In contrast to forgetful functors which \enquote{remove structure}, free functors \enquote{add structure}. For example, a functor which sends a set to its corresponding discrete topological space is a free functor.
\end{definition}

\begin{definition}\label{def:presheaf}(\cite[definition 1.2.15]{Leinster2014})
  A \uline{presheaf} on the category $\Cal A$ is a contravariant functor
  \begin{align*}
    F: \Cal A \to \Cat{Set}.
  \end{align*}
\end{definition}

\begin{example}\label{ex:topological_space_presheaf}(\cite[24]{Leinster2014})
  Let $(X, \tau)$ be a topological space. Form the category $\Cal C$ from the poset $(\tau, \subseteq)$ as in \cref{def:standard_categories/ord}. Presheaves on $\Cal C$ are also called presheaves on the topological space $(X, \tau)$.

  Let $(Y, \rho)$ be another topological space. Then the map
  \begin{align*}
    &F: \tau \to C(\tau, Y) \\
    &F(U) = C(U, Y) = \{ f: U \mapsto Y, f \text{ is continuous} \}
  \end{align*}
  is a presheaf.
\end{example}

\begin{definition}\label{def:faithful_full_functors}(\cite[definition 1.2.16]{Leinster2014})
  A functor $F: \Cal A \to \Cal B$ is called \uline{faithful} (resp. \uline{full}) if the map
  \begin{align*}
    \Cal{A}(A, B) &\to \Cal{B}(F(A), F(B)) \\
    f &\mapsto F(f)
  \end{align*}
  is injective (resp. surjective) (see \cref{def:function_invertability}).
\end{definition}

\begin{example}\label{def:subcategory_functors}(\cite[25]{Leinster2014})
  Let $\Cal B$ be a subcategory of $\Cal A$. We define the inclusion functor $I: \Cal B \to \Cal A$ by sending each object and each morphism of $\Cal B$ to itself within $\Cal A$.

  Then $I$ is faithful and, if the subcategory $\Cal B$ is full (see \cref{def:subcategory}), then $I$ is also full.
\end{example}

\begin{definition}\label{def:natural_transformation}(\cite[definition 1.3.1]{Leinster2014})
  Let $\Cal A$ and $\Cal B$ be categories and let $F$ and $G$ be functors from $\Cal A$ to $\Cal B$.

  A \uline{natural transformation} $\alpha: F \to G$ is a family $\{ \alpha_A \}_{A \in \Cal A}$ of morphisms in $\Cal B$ such that for every morphism $f: A \to B$ in $\Cal A$, the diagram
  \begin{center}
    \begin{tikzcd}
      F(A) \arrow[r, "F(f)"] \arrow[d, "\alpha_A"] & F(B) \arrow[d, "\alpha_B"] \\
      G(A) \arrow[r, "G(f)"]                       & G(B)
    \end{tikzcd}
  \end{center}
  commutes.

  The morphisms $\alpha_A$ are called the components of $\alpha$. We denote natural transformations using
  \begin{center}
    \begin{tikzcd}[column sep=huge]
      \Cal A
        \arrow[r, bend left, "F"]{}[name=F]{}
        \arrow[r, bend right, "G"']{}[name=G]{} &
      \Cal B
        \arrow[shorten <= 0.5em, Rightarrow,to path={(F) -- node[label=right:$\alpha$] {} (G)}]{}
    \end{tikzcd}
  \end{center}

  Composition of natural transformations is defined in an obvious way and is sometimes called \uline{vertical composition}.
\end{definition}

\begin{definition}\label{def:functor_category}
  Given categories $\Cal A$ and $\Cal B$, we define their \uline{functor category} ${\Cal B}^{\Cal A}$ by
  \begin{itemize}
    \item $\Obj {\Cal B}^{\Cal A}$ are functors $F: \Cal A \to \Cal B$.
    \item ${\Cal B}^{\Cal A}(F, G)$ are the natural transformations from $F$ to $G$.
  \end{itemize}

  When $\Cal A$ is a discrete category, then ${\Cal B}^{\Cal A}$ is called a \uline{product category} indexed by $\Cal A$. In particular, if $\Cal A$ is finite of cardinality $n$, we can also use the notation
  \begin{align*}
    {\Cal B}^{\Cal A} = {\Cal B}^n = \Cal B \times \ldots \times \Cal B.
  \end{align*}

  If the natural transformation $\alpha$ is an isomorphism in ${\Cal B}^{\Cal A}$, we say that the categories $\Cal A$ and $\Cal B$ are \uline{naturally isomorphic} and write $\Cal A \cong \Cal B$.
\end{definition}

\begin{definition}\label{def:category_equivalence}\cite[definition 1.3.15]{Leinster2014}
  An \uline{equivalence} between the categories $\Cal A$ and $\Cal B$ consists of a pair of functors $F, G: \Cal A \to \Cal B$ and a pair of natural isomorphisms
  \begin{align*}
    \xi: \Id_{\Cal A} \to G \circ F,
    &&
    \eta: F \circ G \to \Id_{\Cal B}.
  \end{align*}

  If an equivalence between $\Cal A$ and $\Cal B$ exists, we say that \uline{the categories $\Cal A$ and $\Cal B$ are equivalent} and write $\Cal A \simeq \Cal B$.

  An equivalence of the form $\Cal{A}^{\Op} \simeq \Cal{B}$ is called a \uline{duality} between $\Cal A$ and $\Cal B$ and we say that \uline{$\Cal A$ is dual to $\Cal B$} (\cite[example 1.3.22]{Leinster2014}).
\end{definition}

\begin{definition}\label{def:natural_transformation_horizontal_composition}\cite[remarks 1.3.24]{Leinster2014}
  Let $\Cal A$, $\Cal B$ and $\Cal C$ be categories, $F, G: \Cal A \to \Cal C$ and $H, K: \Cal B \to \Cal C$ be functors and $\beta: F \to G$ and $\gamma: H \to K$ be natural transformations.
  \begin{center}
    \begin{tikzcd}[column sep=huge]
      \Cal A
        \arrow[r, bend left, "F"]{}[name=F]{}
        \arrow[r, bend right, "G"']{}[name=G]{} &
      \Cal B
        \arrow[shorten <= 0.5em, Rightarrow,to path={(F) -- node[label=right:$\beta$] {} (G)}]{}
        \arrow[r, bend left, "H"]{}[name=H]{}
        \arrow[r, bend right, "K"']{}[name=K]{} &
      \Cal C
        \arrow[shorten <= 0.5em, Rightarrow,to path={(H) -- node[label=right:$\gamma$] {} (K)}]{}
    \end{tikzcd}
  \end{center}

  We define the natural transformation
  \begin{align*}
    \alpha \coloneqq \gamma * \beta: H \circ F \to K \circ G,
  \end{align*}
  called \uline{horizontal composition of $\beta$ and $\gamma$}, defined by
  \begin{align*}
    \alpha_A \coloneqq \gamma_{G(A)} \circ H(\beta_A) = K(\beta_A) \circ \gamma_{F(A)}.
  \end{align*}
\end{definition}
