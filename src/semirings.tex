\subsection{Rings}\label{subsec:rings}

We will start by defining semirings, and to do that we will first motivate distributivity.

\begin{proposition}\label{thm:monoid_distributivity}
  Fix a \hyperref[def:magma/commutative]{commutative} \hyperref[def:monoid]{monoid} \( (R, +, \cdot) \), where \( +: R \times R \to R \) is the monoid operation and \( \cdot: \BbbN \times R \to R \) is \hyperref[rem:additive_magma/multiplication]{recursively defined multiplication}.

  We have the following property, which we call \term{distributivity} of \( \cdot \) over \( + \):
  \begin{equation}\label{eq:thm:monoid_distributivity}
    n \cdot (x + y) = n \cdot x + n \cdot y.
  \end{equation}
\end{proposition}
\begin{proof}
  We use induction on \( n \). The case \( n = 0 \) is trivial. Suppose that \eqref{eq:thm:monoid_distributivity} holds. Then
  \begin{equation*}
    (n + 1) \cdot (x + y)
    \reloset {\eqref{eq:def:magma/exponentiation}} =
    n \cdot (x + y) + (x + y)
    \reloset {\T{ind.}} =
    n \cdot x + n \cdot y + (x + y)
    \reloset {\eqref{eq:def:magma/exponentiation}} =
    (n + 1) \cdot x + (n + 1) \cdot y.
  \end{equation*}
\end{proof}

\begin{definition}\label{def:semiring}\mcite[ch. 3, sec. 2.1]{GondranMinoux1984Graphs}
  A \term{semiring} is a \hyperref[def:magma/commutative]{commutative} \hyperref[def:monoid]{monoid} \( (R, +) \) with a second \hyperref[def:magma/associative]{associative} operation \( \cdot: R \times R \to R \) called \term{multiplication}, which extends multiplication with natural numbers. The precise compatibility axioms are listed in \fullref{def:semiring/theory}.

  Although not strictly necessary, it will be convenient for us to assume that multiplication has an identity. If a multiplicative identity does not exist, we call \( (R, +, \cdot) \) a \term{nonunital semiring}. We will not use nonunital semirings, but it is important to acknowledge their existence. In this context, if an identity exists, we will sometimes call \( (R, + \cdot) \) a \term{unital semiring}.

  We call \( (R, +) \) the \term{additive monoid} and \( (R, \cdot) \) the \term{multiplicative monoid} of the semiring.

  Semirings have the following metamathematical properties:
  \begin{thmenum}
    \thmitem{def:semiring/theory} The \hyperref[def:first_order_theory]{first-order theory} for semirings extends the \hyperref[def:monoid/theory]{theory of monoids}.

    First, we add another \hyperref[rem:first_order_formula_conventions/infix]{infix} binary functional symbol \( \cdot \) and a constant \( 1 \). This notation is justified by \fullref{thm:semiring_characteristic_homomorphism}.

    We then extend the theory of monoids with \hyperref[def:magma/commutative]{commutativity} for \( + \), \hyperref[def:magma/associative]{associativity} for \( \cdot \), and the following axioms:
    \begin{thmenum}
      \thmitem{def:semiring/left_distributivity} Multiplication on the left distributes over addition:
      \begin{equation}\label{eq:def:semiring/left_distributivity}
        \xi \cdot (\eta + \zeta) \doteq \xi \cdot \eta + \xi \cdot \zeta.
      \end{equation}

      \thmitem{def:semiring/right_distributivity} Multiplication on the right also distributes over addition:
      \begin{equation}\label{eq:def:semiring/right_distributivity}
        (\xi + \eta) \cdot \zeta \doteq \xi \cdot \zeta + \eta \cdot \zeta.
      \end{equation}

      If multiplication is commutative, right distributivity follows from left distributivity.

      \thmitem{def:semiring/absorption} Zero is an absorbing element:
      \begin{equation}\label{eq:def:semiring/absorption}
        \xi \cdot 0 \doteq 0 \wedge 0 \cdot \xi \doteq 0.
      \end{equation}

      \thmitem{def:semiring/identity} We also restate the identity axiom \eqref{eq:def:monoid/theory/identity} for the multiplicative unit \( 1 \) to highlight its connection with \eqref{eq:def:semiring/absorption}:
      \begin{equation}\label{eq:def:semiring/identity}
        \xi \cdot 1 \doteq \xi \wedge 1 \cdot \xi \doteq \xi.
      \end{equation}
    \end{thmenum}

    \thmitem{def:semiring/homomorphism} A \hyperref[def:first_order_homomorphism]{first-order homomorphism} between the semirings \( R \) and \( S \) is a function \( \varphi: R \to S \) that is a \hyperref[def:monoid/homomorphism]{monoid homomorphism} both for their additive monoids also for their multiplicative monoids.

    For semirings, \( \varphi \) must be a monoid homomorphism for the multiplicative monoids also. That is, it must preserve the multiplicative identity \( 1 \).

    \thmitem{def:semiring/submodel} The set \( S \subseteq R \) is a \hyperref[thm:substructure_is_model]{submodel} of \( R \) if it is a both \hyperref[def:monoid/submodel]{submonoid} of the additive monoid and also of the multiplicative monoid.

    As a consequence of \fullref{thm:positive_formulas_preserved_under_homomorphism}, the \hyperref[def:multi_valued_function/image]{image} of a group homomorphism \( \varphi: R \to S \) is a subgroup of \( S \).

    For an arbitrary set \( S \), we denote the \hyperref[def:first_order_generated_substructure]{generated submodel} by \( \braket{ S } \).

    \thmitem{def:semiring/trivial} The \hyperref[thm:substructures_form_complete_lattice/bottom]{trivial} semiring is the \hyperref[def:pointed_set/trivial]{trivial pointed set} \( \set{ 0 } \).

    See \fullref{ex:trivial_semiring} for some properties of the trivial semiring.

    \thmitem{def:semiring/category} The \hyperref[def:category_of_small_first_order_models]{category of \( \mscrU \)-small models} for semirings does not have an established name. It is \hyperref[def:monoid]{\( \ucat{CMon} \)}-\hyperref[def:concrete_category]{concrete} with the forgetful functor taking the additive monoids.

    \thmitem{def:semiring/exponentiation} As we shall see in \fullref{thm:semiring_characteristic_homomorphism}, multiplication in \( \cdot \) extends multiplication by natural numbers in the abelian group \( (R, +) \). We do have a third operation, however --- \hyperref[def:monoid/exponentiation]{monoid exponentiation} in \( (R, \cdot) \).

    For any integer \( n \), we have the fundamental property \( 1^n = 1 \).

    \thmitem{def:semiring/commutative} If multiplication is commutative, we call the semiring itself \term{commutative}.
  \end{thmenum}
\end{definition}

\begin{example}\label{ex:trivial_semiring}
  A \hyperref[def:semiring/homomorphism]{semiring} is trivial if and only if \( 0_R = 1_R \). This follows from \eqref{eq:def:semiring/absorption} and \eqref{eq:def:semiring/identity}.

  As a consequence, if \( \varphi: \set{ 0 } \to R \) is a \hyperref[def:semiring/homomorphism]{semiring} homomorphism, \( R \) is a trivial semiring. This is further strengthened by \fullref{thm:semiring_embedding_preserves_characterstic}.
\end{example}

\begin{proposition}\label{thm:semiring_characteristic_homomorphism}
  For every \hyperref[def:semiring/identity]{semiring}, multiplication extends the abelian group multiplication.

  More precisely, denote the additive identity by \( 0_R \) and the multiplicative identity by \( 1_R \). Define the following semiring homomorphism:
  \begin{equation}\label{eq:thm:semiring_characteristic_homomorphism}
    \begin{aligned}
      &\iota: \BbbN \to R \\
      &\iota(n) \coloneqq \begin{cases}
        0_R                &n = 0, \\
        \iota(n - 1) + 1_R &n > 0.
      \end{cases}
    \end{aligned}
  \end{equation}

  This is the unique homomorphism from \( \BbbN \) to \( R \). Furthermore, we have the following analogue to \eqref{eq:def:magma/exponentiation}:
  \begin{equation}\label{eq:thm:semiring_characteristic_homomorphism/multiplication}
    \iota(n) \cdot x \coloneqq \begin{cases}
      0_R,                      &n = 0, \\
      \iota(n - 1) \cdot x + x, &n > 1.
    \end{cases}
  \end{equation}
\end{proposition}
\begin{proof}
  First note that \eqref{eq:thm:semiring_characteristic_homomorphism/multiplication} follows from \eqref{eq:thm:semiring_characteristic_homomorphism} via \hyperref[def:semiring/right_distributivity]{right distributivity}.

  It remains to show that \( \iota \) is a monoid homomorphism, and that it is unique. Clearly \( \iota(0) = 0_R \) and \( \iota(1) = 1_R \). Proving \( \iota(n + m) = \iota(n) + \iota(m) \) and \( \iota(nm) = \iota(n) \cdot \iota(m) \) can be done via nested induction.

  Now suppose \( \varphi: \BbbN \to R \) is a homomorphism. It is clear that \( \varphi(0) = 0_R \) and \( \varphi(1) = 1_R \), and also
  \begin{equation*}
    \varphi(n + 1) = \varphi(n) + \varphi(1) = \varphi(n) + 1_R.
  \end{equation*}

  This implies \( \iota = \varphi \).
\end{proof}

\begin{definition}\label{def:semiring_characteristic}\mimprovised
  We define the \term{characteristic} \( \op{char}(R) \) of a semiring \( R \) via the following equivalent definitions:
  \begin{thmenum}
    \thmitem{def:semiring_characteristic/homomorphism} \( \op{char}(R) \) is the nonnegative integer \( n \) such that
    \begin{equation*}
      n \BbbZ = \ker\iota,
    \end{equation*}
    where \( \iota \) is the homomorphism from the natural numbers defined via \eqref{eq:thm:semiring_characteristic_homomorphism}.

    \thmitem{def:semiring_characteristic/direct} \( \op{char}(R) \) is the smallest positive integer \( n \) such that \( n \cdot 1_R = 0_R \) and \( \op{char}(R) = 0 \) if \( 0_R \) cannot be obtained in this way.
  \end{thmenum}
\end{definition}
\begin{defproof}
  \EquivalenceSubProof{def:semiring_characteristic/homomorphism}{def:semiring_characteristic/direct} Let \( n \) be such that
  \begin{equation*}
    n \BbbZ = \ker\iota.
  \end{equation*}

  In particular, \( \iota(0) = \iota(n) \).

  If \( n = 0 \), \( \ker\iota \) is a trivial group and \( \iota \) is an embedding. Then there cannot exist a positive integer \( n \) such that
  \begin{equation*}
    n \cdot 1_R = 0_R.
  \end{equation*}

  Otherwise, \( n \) is the smallest positive integer such that
  \begin{equation*}
    n \cdot 1_R = 0 \cdot 1_R = 0_R.
  \end{equation*}
\end{defproof}

\begin{proposition}\label{thm:semiring_embedding_preserves_characterstic}
  If \( \varphi: R \to T \) is a \hyperref[def:semiring/homomorphism]{semiring embedding}, then the \hyperref[def:semiring_characteristic]{characteristic} of \( R \) is equal to that of \( T \).
\end{proposition}
\begin{proof}
  First suppose that \( R \) has positive characteristic \( n \). Then \( n \cdot 1_R = 0_R \), which implies \( n \cdot \varphi(1_R) = \varphi(0_R) \), hence \( \op{char}(T) \leq n \). But \( \varphi \) is an embedding, hence if \( k \cdot 1_R \neq 0_R \), then
  \begin{equation*}
    k \cdot \varphi(1_R) \neq \varphi(0_R).
  \end{equation*}

  This implies that \( \op{char}(T) \geq \op{char}(R) \), which in turn shows that \( \op{char}(T) = \op{char}(R) \).

  If \( R \) has characteristic zero, then \( \iota: \BbbN \to R \) is an embedding and thus \( \varphi \bincirc \iota: \BbbN \to T \) is also an embedding. It is unique as shown in \fullref{thm:semiring_characteristic_homomorphism}. Therefore, \( T \) also has characteristic zero.
\end{proof}

\begin{example}\label{ex:semiring_characteristic}
  The following are examples of \hyperref[def:semiring_characteristic]{semiring characteristics}:

  \begin{thmenum}
    \thmitem{ex:semiring_characteristic/nonnegative_integers} The zero-based \hyperref[def:set_of_natural_numbers]{natural numbers} \( \BbbN \) have characteristic \( \op{char}(\BbbN) = 0 \) because \( \iota \) is an isomorphism. Consequently, by \fullref{thm:semiring_embedding_preserves_characterstic}, any supersemiring of \( \BbbN \) has characteristic zero, most notably the ring of integers \( \BbbZ \) and the fields \( \BbbQ \), \( \BbbR \), \( \BbbC \).

    \thmitem{ex:semiring_characteristic/integers_modulo} The ring \hyperref[def:ring_of_integers_modulo]{\( \BbbZ_n \)} of integers modulo \( n \) has characteristic \( \op{char}(\BbbZ_n) = n \) because of \fullref{thm:integers_modulo_isomorphic_to_quotient_group}.

    \thmitem{ex:semiring_characteristic/polynomial_ring} An \hyperref[def:algebra_over_ring]{algebra} \( A \) over a nontrivial commutative unital ring \( R \) has characteristic \( \op{char}(A) = R \) because of the canonical embedding of \( R \) in \( A \). In particular, polynomial \hyperref[def:algebra_of_polynomials]{rings} \( R[X] \) have the same characteristic as their ring.

    \thmitem{ex:semiring_characteristic/galois_fields} The \hyperref[thm:galois_field_existence]{Galois field} \( \BbbF_{p^n} \) has characteristic \( p \) because it is a field extension of \( \BbbF_p \).
  \end{thmenum}
\end{example}

\begin{proposition}\label{thm:category_of_semirings_properties}
  The \hyperref[def:semiring/category]{category of semirings} has the following properties:
  \begin{thmenum}
    \thmitem{thm:category_of_semirings_properties/initial} The \hyperref[def:set_of_integers]{ring of integers} \( \BbbZ \) is an \hyperref[def:universal_objects/initial]{initial object}.

    \thmitem{thm:category_of_semirings_properties/terminal} The trivial semiring \( \set{ 0 } \) is an \hyperref[def:universal_objects/terminal]{terminal object}.
  \end{thmenum}
\end{proposition}
\begin{proof}
  \SubProofOf{thm:category_of_semirings_properties/initial} Follows from \fullref{thm:semiring_characteristic_homomorphism}.
  \SubProofOf{thm:category_of_semirings_properties/terminal} Follows from \fullref{ex:trivial_semiring}.
\end{proof}

\begin{definition}\label{def:semiring_division}\mimprovised
  Fix an arbitrary element \( x \) in a \hyperref[def:semiring]{semiring} \( R \). If, for some nonzero element \( y \), there exists a pair \( l \) and \( r \) of nonzero elements such that \( l y = y r = x \), we say that \( y \) is a \term{divisor} of \( x \) and that \( x \) is a \term{multiple} or \term{factor} of \( y \). We denote this by \( y \mid x \). Divisors of \( 1 \) are sometimes called \term{units}.

  The requirement that \( y \), \( l \) and \( r \) are nonzero is essential. Zero divisors are extensively studied, and without this requirement, every element of \( R \) would be a zero divisor due to \hyperref[def:semiring/absorption]{absorption}. Since \( 0 \) does not divide any nonzero element, we do not lose anything useful with this restriction.
\end{definition}

\begin{proposition}\label{thm:semiring_cancellative_iff_no_zero_divisors}
  An element of a commutative ring is a \hyperref[def:semiring_division]{zero divisor} if and only if it is not cancellable. That is, \( x \mid 0 \) if and only if \( xy = xz \) does not imply \( y = z \).
\end{proposition}
\begin{proof}
  Zero is obviously a zero divisor, and it is not cancellable. Let \( x \) be a nonzero element.

  \SufficiencySubProof Suppose that \( x \) is a zero divisor and let \( y \) be such that \( xy = 0 \). For any element \( z \), we have
  \begin{equation*}
    xy = 0 = x(yz).
  \end{equation*}

  But \( y \neq yz \) unless \( z = 1 \). Thus, \( x \) is not cancellable.

  \NecessitySubProof Suppose that \( x \) is cancellable.

  Suppose also that \( xy = 0 \) for some nonzero \( y \). Then \( xy = x0 \), which implies \( y = 0 \). But this contradicts out choice of \( y \).

  Thus, \( x \) is not a zero divisor.
\end{proof}

\begin{definition}\label{def:semiring_kernel}
  The \term{kernel} \( \ker(f) \) of a semiring homomorphism \( f: R \to S \) is the \hyperref[def:zero_locus]{zero locus} of \( f \), that is, \hyperref[thm:def:function/properties/preimage]{preimage} \( f^{-1}(0_S) \).

  It is an instance of \fullref{def:zero_morphisms/kernel}.
\end{definition}

\begin{definition}\label{def:quotient_semiring}
  Let \( R \) be a ring and \( I \) be an ideal of \( M \). Define the \term{quotient ring} to be the quotient \hyperref[def:quotient_left_module]{module} when considering \( R \) as a module over itself.
\end{definition}

\begin{theorem}\label{thm:homomorphism_theorem_for_rings}
  Let \( \varphi: R \to T \) be a homomorphism of rings. We have the isomorphism
  \begin{equation*}
    R / \ker \varphi \cong \img \varphi.
  \end{equation*}
\end{theorem}
\begin{proof}
  Special case of \fullref{thm:homomorphism_theorem_for_left_modules}.
\end{proof}

\begin{definition}\label{def:tropical_semiring}\mcite{nLab:tropical_semiring}
  Fix a partially \hyperref[def:partially_ordered_set]{ordered} \hyperref[def:abelian_group]{abelian group} \( (M, +, \leq) \). Let \( \infty \) be a sentinel symbol not in \( M \). Define
  \begin{equation*}
    T \coloneqq M \cup \{ \infty \}
  \end{equation*}
  with operations
  \begin{balign*}
     & \oplus: T \times T \to T                        \\
     & x \oplus y \coloneqq \begin{cases}
      \min \{ x, y \}, & x \neq \infty \text{ and } y \neq \infty \text{ and they are comparable}, \\
      \infty,          & x = \infty \text{ or } y = \infty
    \end{cases} \\
    \\
     & \odot: T \times T \to T                         \\
     & x \odot y \begin{cases}
      x + y,  & x \neq \infty \text{ and } y \neq \infty, \\
      \infty, & x = \infty \text{ or } y = \infty
    \end{cases}
  \end{balign*}

  This makes \( (T, \oplus, \odot) \) into a \hyperref[def:semiring/identity]{semiring} with additive identity \( \infty \) and multiplicative identity \( 0 \). We call \( (T, \oplus, \odot) \) the \( \min \)-\term{tropical semiring} or simply the \term{tropical semiring} over \( M \). We define the \( \max \)-\term{tropical semiring} analogously by simply replacing \( \min \) with \( \max \).
\end{definition}

\begin{definition}\label{def:endomorphism_semiring}
  Let \( (X, +) \) be an monoid and let \( \End(X) \) be set of endomorphism over \( X \). We define two operations:
  \begin{itemize}
    \item Pointwise addition \( [f + g](x) \coloneqq f(x) + g(x) \).
    \item Multiplication by composition \( [fg](x) \coloneqq f(g(x)) \).
  \end{itemize}

  These operations make \( \End(X) \) into a semiring. If \( X \) is a group, then \( \End(X) \) is a ring.

  If \( X \) is a semiring, we define \( \End(X) \) to be a set of semiring endomorphisms (that is, we want the additive group homomorphisms to preserve multiplication and units). Then \( \End(X) \) is again a semiring and, if \( X \) is a unital ring, so is \( \End(X) \).
\end{definition}

\begin{definition}\label{def:ordered_semiring}
  Extending \fullref{def:preordered_magma} to (semi)rings, we define a \term{preordered semiring} to be a semiring \( \BbbR \) with a magma preorder \( \leq \) that additionally satisfies
  \begin{equation}\label{eq:def:ordered_semiring/nonnegativity}
    0 \leq y \T{and} 0 \leq y \T{implies} 0 \leq xy.
  \end{equation}
\end{definition}
