\subsection{Semirings}\label{subsec:semirings}

We will start by defining semirings, and to do that we will first motivate distributivity.

\begin{proposition}\label{thm:monoid_distributivity}
  Fix an \hyperref[rem:additive_magma/multiplication]{additive} \hyperref[def:monoid]{monoid} \( (R, +, \cdot) \), where \( +: R \times R \to R \) is the monoid operation and \( \cdot: \BbbN \times R \to R \) is defined via \eqref{eq:rem:additive_magma/multiplication}.

  We have the following property, which we call \term{distributivity} of \( \cdot \) over \( + \):
  \begin{equation}\label{eq:thm:monoid_distributivity}
    n \cdot (x + y) = n \cdot x + n \cdot y.
  \end{equation}
\end{proposition}
\begin{proof}
  We use induction on \( n \). The case \( n = 0 \) is trivial. Suppose that \eqref{eq:thm:monoid_distributivity} holds. Then
  \begin{equation*}
    (n + 1) \cdot (x + y)
    \reloset {\eqref{eq:def:magma/exponentiation}} =
    n \cdot (x + y) + (x + y)
    \reloset {\T{ind.}} =
    n \cdot x + n \cdot y + (x + y)
    \reloset {\eqref{eq:def:magma/exponentiation}} =
    (n + 1) \cdot x + (n + 1) \cdot y.
  \end{equation*}
\end{proof}

\begin{definition}\label{def:semiring}\mcite[1]{Golan2010}
  A \term{semiring} is a \hyperref[def:magma/commutative]{commutative} \hyperref[def:monoid]{monoid} \( (R, +) \) with a second \hyperref[def:magma/associative]{associative} operation \( \cdot: R \times R \to R \) called \term{multiplication}, which extends multiplication with natural numbers. The precise compatibility axioms are listed in \fullref{def:semiring/theory} because they fit nicely into first-order logic (unlike the \hyperref[def:semimodule/theory]{theory of semimodules}, for example, for which we prefer expressing these conditions in the metalogic).

  Although not strictly necessary, it will be convenient for us to assume that multiplication has an identity. If a multiplicative identity does not exist, we call \( (R, +, \cdot) \) a \term{nonunital semiring}. A canonical example of a nonunital semiring is a \hyperref[def:semiring_ideal]{semiring ideal}. We will not use nonunital semirings, but it is important to acknowledge their existence. In this context, if an identity exists, we will sometimes call \( (R, + \cdot) \) a \term{unital semiring}.

  We call \( (R, +) \) the \term{additive monoid} and \( (R, \cdot) \) the \term{multiplicative monoid} of the semiring.

  Semirings have the following metamathematical properties:
  \begin{thmenum}
    \thmitem{def:semiring/theory} The \hyperref[def:first_order_theory]{first-order theory} for semirings extends the \hyperref[def:monoid/theory]{theory of monoids}.

    First, we add another \hyperref[rem:first_order_formula_conventions/infix]{infix} binary functional symbol \( \cdot \) and a constant \( 1 \). The notation for the constant is justified by \fullref{thm:semiring_characteristic_homomorphism}.

    We then extend the theory of monoids with \hyperref[def:magma/commutative]{commutativity} for \( + \), \hyperref[def:magma/associative]{associativity} for \( \cdot \), and the following axioms:
    \begin{thmenum}
      \thmitem{def:semiring/left_distributivity} Multiplication on the left distributes over addition:
      \begin{equation}\label{eq:def:semiring/left_distributivity}
        \xi \cdot (\eta + \zeta) \doteq \xi \cdot \eta + \xi \cdot \zeta.
      \end{equation}

      \thmitem{def:semiring/right_distributivity} Multiplication on the right also distributes over addition:
      \begin{equation}\label{eq:def:semiring/right_distributivity}
        (\xi + \eta) \cdot \zeta \doteq \xi \cdot \zeta + \eta \cdot \zeta.
      \end{equation}

      If multiplication is commutative, right distributivity follows from left distributivity.

      \thmitem{def:semiring/absorption} Zero is an absorbing element:
      \begin{equation}\label{eq:def:semiring/absorption}
        \xi \cdot 0 \doteq 0 \wedge 0 \cdot \xi \doteq 0.
      \end{equation}

      \thmitem{def:semiring/identity} We also restate the identity axiom \eqref{eq:def:monoid/theory/identity} for the multiplicative unit \( 1 \) to highlight its connection with \eqref{eq:def:semiring/absorption}:
      \begin{equation}\label{eq:def:semiring/identity}
        \xi \cdot 1 \doteq \xi \wedge 1 \cdot \xi \doteq \xi.
      \end{equation}
    \end{thmenum}

    \thmitem{def:semiring/homomorphism} A \hyperref[def:first_order_homomorphism]{first-order homomorphism} from the semiring \( R \) to \( T \) is a function \( \varphi: R \to T \) that is a \hyperref[def:monoid/homomorphism]{monoid homomorphism} both for their additive monoids also for their multiplicative monoids.

    \thmitem{def:semiring/submodel} The set \( S \subseteq R \) is a \hyperref[thm:substructure_is_model]{submodel} of \( R \) if it is a both \hyperref[def:monoid/submodel]{submonoid} of the additive monoid and also of the multiplicative monoid. We call \( S \) a \term{sub-semiring}.

    As a consequence of \fullref{thm:positive_formulas_preserved_under_homomorphism}, the image of a semiring homomorphism is a sub-semiring of its range.

    For an arbitrary set \( S \), we denote the \hyperref[def:first_order_generated_substructure]{generated submodel} by \( \braket{ S } \).

    \thmitem{def:semiring/trivial} The \hyperref[thm:substructures_form_complete_lattice/bottom]{trivial} semiring is the \hyperref[def:pointed_set/trivial]{trivial pointed set} \( \set{ 0 } \).

    See \fullref{ex:def:semiring/trivial} for some properties of the trivial semiring.

    \thmitem{def:semiring/exponentiation} As we shall see in \fullref{thm:semiring_characteristic_homomorphism}, multiplication in \( \cdot \) extends left multiplication with natural numbers in the monoid \( (R, +) \). We do have a third operation, however --- \hyperref[def:monoid/exponentiation]{monoid exponentiation} in \( (R, \cdot) \).

    For any integer \( n \), we have the fundamental property \( 1^n = 1 \).

    \thmitem{def:semiring/commutative} If multiplication is commutative, we call the semiring itself \term{commutative}. Unless multiplication corresponds to function composition, most semirings we will encounter will be commutative.

    Notable exceptions to this rule are \hyperref[def:ordinal]{ordinals}. A \hyperref[def:successor_and_limit_ordinal]{limit ordinal} \( \alpha \), regarded as the set of all smaller ordinals, is a semiring. It is not commutative, however, as shown in \fullref{ex:ordinal_addition}.

    \thmitem{def:semiring/power_set} Similarly to power set magmas defined in \fullref{def:magma/power_set}, the power set \( \pow(R) \) of a semiring is also a semiring with the operations
    \begin{align*}
      A \oplus B &\coloneqq \set{ x + y \given x \in A \T{and} y \in B } \\
      A \odot B  &\coloneqq \set{ x \cdot y \given x \in A \T{and} y \in B }
    \end{align*}

    \thmitem{def:semiring/category} The corresponding \hyperref[def:category_of_small_first_order_models]{category of \( \mscrU \)-small models} \( \ucat{SRing} \) is \hyperref[def:concrete_category]{concrete} over \hyperref[def:monoid]{\( \ucat{CMon} \)} with the forgetful functor taking the additive monoids. We denote the category of commutative semirings by \( \cat{CSRing} \).

    \thmitem{def:semiring/opposite}\mcite[555]{Knapp2016BasicAlgebra} The \term{opposite semiring} of \( (R, +, \cdot) \) is the semiring \( (R, +, \star) \), with multiplication defined as \( x \star y = y \cdot x \).
  \end{thmenum}
\end{definition}

\begin{remark}\label{rem:semiring_etymology}
  In \fullref{def:semiring}, we require semirings to have both an additive identity and a multiplicative identity. This is not consistent with semigroups defined in \fullref{def:magma/associative}, which in general do not have identities.

  \cite[ch. 3]{GondranMinoux1984Graphs} suggest using \enquote{dioid} (short for \enquote{double monoid}) instead of \enquote{semiring}. \cite[xi]{Golan2010} describes how the term \enquote{dioid} may refer to semirings with idempotent addition, i.e. a general form of the tropical semirings defined in \fullref{def:tropical_semiring}.

  We thus prefer using the term \enquote{semiring} as we have defined it in \fullref{def:semiring}.
\end{remark}

\begin{example}\label{ex:def:semiring}
  We list several examples of \hyperref[def:semiring]{semirings} that are not \hyperref[def:ring]{rings}.

  \begin{thmenum}
    \thmitem{ex:def:semiring/trivial} A \hyperref[def:semiring/homomorphism]{semiring} is trivial if and only if \( 0_R = 1_R \). This follows from \eqref{eq:def:semiring/absorption} and \eqref{eq:def:semiring/identity}.

    As a consequence, if \( \varphi: \set{ 0 } \to R \) is a \hyperref[def:semiring/homomorphism]{semiring} homomorphism, \( R \) is a trivial semiring. This is further strengthened by \fullref{thm:semiring_embedding_preserves_characterstic}.

    \thmitem{ex:def:semiring/natural_numbers} The \hyperref[def:set_of_natural_numbers]{natural numbers} are the quintessential example of a semiring. We prove in \fullref{thm:natural_number_multiplication_properties} that they are a semiring.

    \thmitem{ex:def:semiring/ordinals} Every \hyperref[def:successor_and_limit_ordinal]{limit ordinal} is a monoid under addition, as discussed in \fullref{ex:def:semiring/ordinals}, however it is not commutative.

    \hyperref[def:cardinal_arithmetic/addition]{Cardinal addition} is commutative, however, and hence for every \hyperref[def:successor_and_limit_cardinal/weak_limit]{limit cardinal} \( \kappa \), the set of all cardinals smaller than \( \kappa \) are a semiring.

    \thmitem{ex:def:semiring/lattice} We discussed in \fullref{ex:def:monoid/semilattice} that in a \hyperref[def:semilattice/bounded]{bounded lattice} \( (X, \vee, \wedge, \top, \bot) \), both \( (X, \vee, \bot) \) and \( (X, \wedge, \top) \) are monoids.

    As a consequence of \fullref{thm:bounded_lattice_absorbing}, \( \bot \) is absorbing with respect to \( \wedge \) and \( \top \) with respect to \( \vee \). Therefore, if the lattice is \hyperref[def:semilattice/distributive_lattice]{distributive}, as a consequence of \fullref{thm:bounded_lattice_absorbing}, both \( (X, \vee, \wedge) \) and \( (X, \wedge, \vee) \) are semirings.

    We refer to these semirings are the positive and negative semiring of the lattice. This terminology comes from \fullref{ex:def:ordered_semiring/lattice}.
  \end{thmenum}
\end{example}

\begin{definition}\label{def:tropical_semiring}\mcite[exmpl. 1.12]{Golan2010}
  Consider the additive monoid \( (\BbbN, +) \) of natural numbers or, more generally, an \hyperref[def:ordered_magma]{ordered} \hyperref[def:magma/commutative]{commutative} \hyperref[def:monoid]{monoid} \( (M, +, \leq) \).

  We adjoin a \hyperref[def:partially_ordered_set_extremal_points/top_and_bottom]{top element} \( \infty \) to \( M \) that is absorbing with respect to addition. That is, \( x + \infty = \infty \) for every \( x \in M \).

  The \( \min \)-plus semiring over \( M \) is the triple \( (M \cup \set{ \infty }, \min, +) \). The \hyperref[def:partially_ordered_set_extremal_points/maximum_and_minimum]{minimum} as a binary operation plays the role of semiring addition, with \( \infty \) as the zero element. The usual addition in \( M \) extended with \( \infty \) plays the role of semiring multiplication, with \( 0 \) as the multiplicative identity.

  We analogously define the \( \max \)-plus semiring, adjoining a \hyperref[def:partially_ordered_set_extremal_points/top_and_bottom]{bottom element} \( -\infty \) rather than a top element \( \infty \).

  We will sometimes use \enquote{tropical semiring} to refer to either type of semirings. See \fullref{rem:tropical_semiring_etymology}.
\end{definition}
\begin{defproof}
  We will only show \hyperref[def:semiring/left_distributivity]{distributivity}. If \( x \leq y \), since \( \leq \) is compatible with \( + \), we have
  \begin{equation*}
    \underbrace{\min\set{ x, y }}_{x} + z = x + z \leq y + z.
  \end{equation*}

  Therefore,
  \begin{equation*}
    \min\set{ x, y } + z = \min\set{ x + z , y + z }.
  \end{equation*}
\end{defproof}

\begin{remark}\label{rem:tropical_semiring_etymology}
  \hyperref[def:tropical_semiring]{\( \min \)-plus} and \( \max \)-plus semirings are sometimes referred to as the \term{tropical semirings}. This term is ambiguous, unfortunately, but it gives rise to the terms \enquote{tropical geometry} and \enquote{tropical optimization}.

  According to \cite{Pin1994}, the name \enquote{tropical semiring} is a dedication to the Brazilian-born Imre Simon. The paper also introduces the terms \enquote{tropical integers}, \enquote{tropical reals}, etc. \cite[3]{Golan2010} refers to the more general notion of additively-idempotent semirings. Both reserve the term \enquote{tropical semiring} for the case where \( M = \BbbN \). \cite[ch. 3]{GondranMinoux1984Graphs} does not explicitly use the word \enquote{tropical}, but instead refers to semirings as \enquote{dioids}, and the latter term sometimes refers to additively-idempotent semirings.
\end{remark}

\begin{proposition}\label{thm:semiring_characteristic_homomorphism}
  For every \hyperref[def:semiring/identity]{semiring}, multiplication extends the abelian group multiplication.

  More precisely, denote the additive identity by \( 0_R \) and the multiplicative identity by \( 1_R \). Define the following semiring homomorphism:
  \begin{equation}\label{eq:thm:semiring_characteristic_homomorphism}
    \begin{aligned}
      &\iota: \BbbN \to R \\
      &\iota(n) \coloneqq \begin{cases}
        0_R                &n = 0, \\
        \iota(n - 1) + 1_R &n > 0.
      \end{cases}
    \end{aligned}
  \end{equation}

  This is the unique homomorphism from \( \BbbN \) to \( R \). Furthermore, we have the following analogue to \eqref{eq:def:magma/exponentiation}:
  \begin{equation}\label{eq:thm:semiring_characteristic_homomorphism/multiplication}
    \iota(n) \cdot x \coloneqq \begin{cases}
      0_R,                      &n = 0, \\
      \iota(n - 1) \cdot x + x, &n > 1.
    \end{cases}
  \end{equation}
\end{proposition}
\begin{proof}
  First note that \eqref{eq:thm:semiring_characteristic_homomorphism/multiplication} follows from \eqref{eq:thm:semiring_characteristic_homomorphism} via \hyperref[def:semiring/right_distributivity]{right distributivity}.

  It remains to show that \( \iota \) is a monoid homomorphism, and that it is unique. Clearly \( \iota(0) = 0_R \) and \( \iota(1) = 1_R \). Proving \( \iota(n + m) = \iota(n) + \iota(m) \) and \( \iota(nm) = \iota(n) \cdot \iota(m) \) can be done via nested induction.

  Now suppose \( \varphi: \BbbN \to R \) is a homomorphism. It is clear that \( \varphi(0) = 0_R \) and \( \varphi(1) = 1_R \), and also
  \begin{equation*}
    \varphi(n + 1) = \varphi(n) + \varphi(1) = \varphi(n) + 1_R.
  \end{equation*}

  This implies \( \iota = \varphi \).
\end{proof}

\begin{definition}\label{def:semiring_characteristic}\mimprovised
  We define the \term{characteristic} \( \op{char}(R) \) of a semiring \( R \) via the following equivalent definitions:
  \begin{thmenum}
    \thmitem{def:semiring_characteristic/homomorphism} \( \op{char}(R) \) is the nonnegative integer \( n \) such that
    \begin{equation*}
      n \BbbZ = \ker\iota,
    \end{equation*}
    where \( \iota \) is the homomorphism from the natural numbers defined via \eqref{eq:thm:semiring_characteristic_homomorphism}.

    \thmitem{def:semiring_characteristic/direct} \( \op{char}(R) \) is the smallest positive integer \( n \) such that \( n \cdot 1_R = 0_R \) and \( \op{char}(R) = 0 \) if \( 0_R \) cannot be obtained in this way.
  \end{thmenum}
\end{definition}
\begin{defproof}
  \EquivalenceSubProof{def:semiring_characteristic/homomorphism}{def:semiring_characteristic/direct} Let \( n \) be such that
  \begin{equation*}
    n \BbbZ = \ker\iota.
  \end{equation*}

  In particular, \( \iota(0) = \iota(n) \).

  If \( n = 0 \), \( \ker\iota \) is a trivial group and \( \iota \) is an embedding. Then there cannot exist a positive integer \( n \) such that
  \begin{equation*}
    n \cdot 1_R = 0_R.
  \end{equation*}

  Otherwise, \( n \) is the smallest positive integer such that
  \begin{equation*}
    n \cdot 1_R = 0 \cdot 1_R = 0_R.
  \end{equation*}
\end{defproof}

\begin{proposition}\label{thm:semiring_embedding_preserves_characterstic}
  If \( \varphi: R \to T \) is a \hyperref[def:semiring/homomorphism]{semiring embedding}, then the \hyperref[def:semiring_characteristic]{characteristic} of \( R \) is equal to that of \( T \).
\end{proposition}
\begin{proof}
  First suppose that \( R \) has positive characteristic \( n \). Then \( n \cdot 1_R = 0_R \), which implies \( n \cdot \varphi(1_R) = \varphi(0_R) \), hence \( \op{char}(T) \leq n \). But \( \varphi \) is an embedding, hence if \( k \cdot 1_R \neq 0_R \), then
  \begin{equation*}
    k \cdot \varphi(1_R) \neq \varphi(0_R).
  \end{equation*}

  This implies that \( \op{char}(T) \geq \op{char}(R) \), which in turn shows that \( \op{char}(T) = \op{char}(R) \).

  If \( R \) has characteristic zero, then \( \iota: \BbbN \to R \) is an embedding and thus \( \varphi \bincirc \iota: \BbbN \to T \) is also an embedding. It is unique as shown in \fullref{thm:semiring_characteristic_homomorphism}. Therefore, \( T \) also has characteristic zero.
\end{proof}

\begin{example}\label{ex:semiring_characteristic}
  The following are examples of \hyperref[def:semiring_characteristic]{semiring characteristics}:

  \begin{thmenum}
    \thmitem{ex:semiring_characteristic/nonnegative_integers} The zero-based \hyperref[def:set_of_natural_numbers]{natural numbers} \( \BbbN \) have characteristic \( \op{char}(\BbbN) = 0 \) because \( \iota \) is an isomorphism. Consequently, by \fullref{thm:semiring_embedding_preserves_characterstic}, any semiring extension of \( \BbbN \) has characteristic zero, most notably the ring of integers \( \BbbZ \) and the fields \( \BbbQ \), \( \BbbR \), \( \BbbC \).

    \thmitem{ex:semiring_characteristic/integers_modulo} The ring \hyperref[thm:ring_of_integers_modulo]{\( \BbbZ_n \)} of integers modulo \( n \) has characteristic \( \op{char}(\BbbZ_n) = n \) because of \fullref{thm:integers_modulo_isomorphic_to_quotient_group}.

    \thmitem{ex:semiring_characteristic/polynomial_ring} An \hyperref[def:algebra_over_semiring]{algebra} \( A \) over a nontrivial commutative unital ring \( R \) has the same characteristic as \( R \) because of the canonical embedding of \( R \) in \( A \). In particular, polynomial \hyperref[def:algebra_of_polynomials]{rings} \( R[X] \) have the same characteristic as their ring.
  \end{thmenum}
\end{example}

\begin{proposition}\label{thm:category_of_semirings_properties}
  The \hyperref[def:semiring/category]{category of semirings} has the following basic properties:
  \begin{thmenum}
    \thmitem{thm:category_of_semirings_properties/initial} The \hyperref[def:set_of_integers]{ring of integers} \( \BbbZ \) is an \hyperref[def:universal_objects/initial]{initial object}.

    \thmitem{thm:category_of_semirings_properties/terminal} The trivial semiring \( \set{ 0 } \) is an \hyperref[def:universal_objects/terminal]{terminal object}.
  \end{thmenum}
\end{proposition}
\begin{proof}
  \SubProofOf{thm:category_of_semirings_properties/initial} Follows from \fullref{thm:semiring_characteristic_homomorphism}.
  \SubProofOf{thm:category_of_semirings_properties/terminal} Follows from \fullref{ex:def:semiring/trivial}.
\end{proof}

\begin{definition}\label{def:ordered_semiring}\mcite[224]{Golan2010}
  An \term{ordered semiring} is a \hyperref[def:magma/commutative]{commutative} semiring \( R \) with a \hyperref[def:partially_ordered_set]{partial order} \( \leq \) such that \( (R, +) \) is an \hyperref[def:ordered_magma]{ordered magma} and, additionally, \( x \leq y \) and \( 0 \leq z \) imply \( xz \leq yz \).

  As in \fullref{def:ordered_magma}, the commutativity condition can be avoided, but then we would need to also require \( zx \leq zy \).

  If the semiring is \hyperref[def:totally_ordered_set]{totally ordered}, we can use the usual terminology that is conventional for real numbers:
  \begin{itemize}
    \item \( x \) is \term{positive} if \( x > 0 \).
    \item \( x \) is \term{nonnegative} if \( x \geq 0 \).
    \item \( x \) is \term{negative} if \( x < 0 \).
    \item \( x \) is \term{nonpositive} if \( x \leq 0 \).
  \end{itemize}
\end{definition}

\begin{example}\label{ex:def:ordered_semiring}
  We list several examples of \hyperref[def:ordered_semirings]{ordered semirings}.

  \begin{thmenum}
    \thmitem{ex:def:ordered_semiring/natural_numbers} The \hyperref[def:set_of_natural_numbers]{natural numbers} form an ordered semiring as shown in \fullref{thm:natural_numbers_are_well_ordered}.

    \thmitem{ex:def:ordered_semiring/lattice} We discussed in \fullref{ex:def:semiring/lattice} that a \hyperref[def:semilattice/bounded]{bounded} \hyperref[def:semilattice/distributive_lattice]{distributive} \hyperref[def:semilattice/lattice]{lattice} \( (X, \vee, \wedge) \) can be regarded as a semiring, and so can its opposite lattice.

    We discussed in \fullref{ex:def:ordered_magma/semilattice} that both \( (X, \vee) \) and \( (X, \wedge) \) are \hyperref[def:ordered_magma]{ordered magmas}. Both \( (X, \vee, \wedge) \) and \( (X, \wedge, \vee) \) vacuously satisfy the condition from \fullref{def:ordered_semiring}, which makes them ordered semirings.

    All elements of the ordered semiring \( (X, \vee, \wedge) \) are nonnegative and all elements of \( (X, \wedge, \vee) \) are nonpositive. With a slight abuse of notation, we refer to them as the \term{positive} and \term{negative} semirings of the lattice.
  \end{thmenum}
\end{example}

\begin{definition}\label{def:divisibility}\mimprovised
  Fix an arbitrary element \( x \) in a \hyperref[def:semiring]{semiring}. If there exist elements \( l \) and \( r \) such that \( x = lr \), we say that \( l \) is a \term{left divisor} of \( x \), and that \( r \) is a \term{right divisor}.

  In a \hyperref[def:semiring/commutative]{commutative semiring}, the two notions coincide, and we simply use the term \enquote{divisor}. If \( x \) is a divisor of \( y \), we write \( x \mid y \) and say that \( y \) is a \term{multiple} or \term{factor} of \( x \). Most rings we will encounter will be commutative, but it is useful to have the weaker notions of left and right divisors.

  \begin{thmenum}
    \thmitem{def:divisibility/zero}\mcite[4]{Golan2010} Divisors of \( 0 \) are called \term{zero divisors}. Due to \hyperref[def:semiring/absorption]{absorption}, every semiring element is a zero divisor. If \( lr = 0 \) for nonzero \( l \) and \( r \), we say that \( l \) (resp. \( r \)) is a \term{nontrivial} left zero divisor (resp. nontrivial right zero divisor)

    \thmitem{def:divisibility/unit} Divisors of \( 1 \) are called \term{invertible}, since they are precisely the \hyperref[def:monoid_inverse]{monoid inverses} under multiplication. They are also sometimes called \term{units} --- this is discussed in \fullref{rem:units_in_rings_etymology}.

    \thmitem{def:divisibility/associates} If \( x \mid y \) and \( y \mid x \), we say that \( x \) and \( y \) are \term{associates}.

    We restrict this definition only to commutative semirings.

    Another common restriction is to entire semirings, in which case \fullref{thm:associates_in_entire_semirings} states an equivalent condition.
  \end{thmenum}
\end{definition}

\begin{example}\label{ex:def:divisibility}
  \hfill
  \begin{thmenum}
    \thmitem{ex:def:divisibility/integers} The positive integers are commutative and their left and right divisors coincide. They have no \hyperref[def:divisibility/zero]{nontrivial zero divisors} as a consequence of \fullref{thm:natural_number_multiplication_properties}.

    \thmitem{ex:def:divisibility/matrix_zero_divisors} A simple example of nontrivial zero divisors is given by the matrix algebra \( \BbbZ^{2 \times 2} \). We have
    \begin{equation*}
      \underbrace
      {
        \begin{pmatrix}
          0 & 1 \\
          0 & 0
        \end{pmatrix}
      }_{L}
      \underbrace
      {
        \begin{pmatrix}
          0 & 0 \\
          0 & 1
        \end{pmatrix}
      }_{R}
      =
      \begin{pmatrix}
        0 & 0 \\
        0 & 0
      \end{pmatrix}.
    \end{equation*}

    Therefore, \( L \) is a left zero divisor and \( R \) is a right zero divisor. The two do not commute because
    \begin{equation*}
      \underbrace
      {
        \begin{pmatrix}
          0 & 0 \\
          0 & 1
        \end{pmatrix}
      }_{R}
      \underbrace
      {
        \begin{pmatrix}
          0 & 1 \\
          0 & 0
        \end{pmatrix}
      }_{L}
      =
      \begin{pmatrix}
        0 & 0 \\
        1 & 0
      \end{pmatrix}.
    \end{equation*}

    Nevertheless, \( RLRL \) is the zero matrix, so \( R \) is a left zero divisor and \( L \) is a right zero divisor.
  \end{thmenum}
\end{example}

\begin{proposition}\label{thm:semiring_cancellative_iff_no_zero_divisors}
  An element of a \hyperref[def:semiring/commutative]{commutative semiring} is cancellable if and only if it is not a \hyperref[def:divisibility]{zero divisor}. That is, \( x \mid 0 \) if and only if \( xy = xz \) does not imply \( y = z \).
\end{proposition}
\begin{proof}
  Let \( x \) be a nonzero element.

  \SufficiencySubProof Suppose that \( x \) is a zero divisor and let \( y \) be such that \( xy = 0 \). For any element \( z \), we have
  \begin{equation*}
    xy = 0 = x(yz).
  \end{equation*}

  But \( y \neq yz \) unless \( z = 1 \). Thus, \( x \) is not cancellable.

  \NecessitySubProof Suppose that \( x \) is cancellable.

  Suppose also that \( xy = 0 \) for some nonzero \( y \). Then \( xy = x0 \), which implies \( y = 0 \). But this contradicts out choice of \( y \).

  Thus, \( x \) is not a zero divisor.
\end{proof}

\begin{definition}\label{def:entire_semiring}
  We say that the \hyperref[def:semiring]{semiring} \( R \) is \term{entire} if any of the following equivalent conditions hold:
  \begin{thmenum}
    \thmitem{def:entire_semiring/zero_divisors}\mcite[4]{Golan2010} \( R \) has no \hyperref[def:divisibility/zero]{nontrivial zero divisors}.
    \thmitem{def:entire_semiring/cancellation} \( R \setminus \set{ 0_R } \) is a \hyperref[def:magma/cancellative]{cancellative} \hyperref[def:monoid]{monoid} with respect to multiplication.
  \end{thmenum}
\end{definition}
\begin{defproof}
  The equivalence follows from \fullref{thm:semiring_cancellative_iff_no_zero_divisors}.
\end{defproof}

\begin{definition}\label{def:zerosumfree}\mcite[4]{Golan2010}
  We say that an \hyperref[rem:additive_magma]{additive} \hyperref[def:monoid]{monoid} is \term{zerosumfree} is no nonzero element has an additive inverse. That is, if \( x + y = 0 \) implies \( x = y = 0 \).
\end{definition}

\begin{example}\label{ex:def:zerosumfree}
  We list several examples of \hyperref[def:zerosumfree]{zerosumfree} semirings:
  \begin{thmenum}
    \thmitem{ex:def:zerosumfree/natural_numbers} By \fullref{thm:natural_number_addition_properties}, the natural numbers are zerosumfree.

    \thmitem{ex:def:zerosumfree/lattice} We discussed in \fullref{ex:def:semiring/lattice} that every bounded distributive lattice \( (X, \vee, \wedge) \) has two associated semirings.

    We will show that the positive semiring \( (X, \vee, \wedge) \) is zerosumfree. The proof only relies on \( \vee \) being idempotent. Suppose that \( x \vee y = \bot \). Then
    \begin{equation*}
      \bot
      =
      x \vee y
      \reloset {\eqref{eq:def:magma/idempotent}} =
      (x \vee x) \vee y
      \reloset {\eqref{eq:def:magma/associative}} =
      x \vee (x \vee y)
      =
      x \vee \bot
      \reloset {\eqref{eq:thm:binary_lattice_operations/identity/join}} =
      x.
    \end{equation*}

    Therefore, \( x = \bot \). But \( \bot \vee y = y \), hence \( x \vee y = \bot \) implies \( y = \bot \).

    This demonstrates that the positive semiring is zerosumfree.

    \thmitem{ex:def:zerosumfree/tropical} The \hyperref[def:tropical_semiring]{\( \min \)-plus semiring} \( (\BbbN \cup \set{ \infty }, \min, +) \) discussed in \fullref{ex:def:semiring/tropical} is also zerosumfree. Indeed, \( \min \) is idempotent, and the proof is analogous to the one for lattices in \fullref{ex:def:zerosumfree/lattice}.
  \end{thmenum}
\end{example}

\begin{proposition}\label{thm:associates_in_entire_semirings}
  In an \hyperref[def:divisibility/zero]{entire commutative semiring}, \( x \) and \( y \) are associates if and only if there exists a \hyperref[def:divisibility/zero]{unit} \( u \) such that \( x = uy \).

  This is sometimes taken as the definition of associate elements.
\end{proposition}
\begin{proof}
  \NecessitySubProof If \( x \mid y \), there exists an element \( a \) of \( R \) such that \( ax = y \). If \( y \mid x \), there exists \( b \) such that \( x = by \).

  Then \( x = by = bax \). If \( x \) is not a zero divisor, by \fullref{thm:semiring_cancellative_iff_no_zero_divisors}, we can cancel it to obtain \( 1_R = ba \).

  Similarly, \( y = ax = aby \). Since \( y \) is not a zero divisor, \( 1_R = ab \).

  This implies that \( a \) is both a left and right unit, and so is \( b \).

  \SufficiencySubProof Trivial.
\end{proof}

\begin{proposition}\label{thm:semiring_divisibility_order}
  In an \hyperref[def:entire_semiring]{entire} \hyperref[def:semiring/commutative]{commutative semiring}, the \hyperref[def:divisibility]{divisibility} relation is a \hyperref[def:preordered_set]{preorder}.

  It is not a partial order in general. To avoid the nonuniqueness problems described in \fullref{ex:preorder_nonuniqueness}, we instead prefer working with ideals. See \fullref{thm:semiring_of_ideals/lattice} for the general approach and \fullref{def:gcd_and_lcm} for how we can sometimes extract elements from the ideals.
\end{proposition}
\begin{proof}
  Fix a semiring \( R \).

  \SubProofOf[def:binary_relation/reflexive]{reflexivity} Clearly every element of \( R \) divides itself.

  \SubProofOf[def:binary_relation/transitive]{transitivity} Let \( x \mid y \mid z \). Then there exist elements \( a \) and \( b \) such that \( y = a x \) and \( z = b y \). Hence, \( z = (ba) x \) and \( x \mid z \).
\end{proof}
