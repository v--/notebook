\subsection{Category equivalences}\label{subsec:category_equivalences}

\begin{remark}\label{rem:category_similarity}
  We have the following notions for expressing that two categories \( \cat{C} \) and \( \cat{D} \) are similar:

  \begin{thmenum}
    \thmitem{rem:category_similarity/equality} Obviously, if \( \cat{C} = \cat{D} \) are equal, they are similar.

    \thmitem{rem:category_similarity/isomorphism} A slightly less obvious notion is \term{isomorphism of categories}. This is an isomorphism, in the sense of \fullref{def:morphism_invertibility/isomorphism}, in the category \hyperref[def:category_of_small_categories]{\( \ucat{Cat} \)} of \( \mscrU \)-small categories for a suitable \hyperref[def:grothendieck_universe]{Grothendieck universe} \( \mscrU \). That is, \( \cat{C} \) and \( \cat{D} \) are isomorphic if there exists an invertible functor between them.

    We rarely distinguish between objects and arrows of isomorphic categories, even if we do not have strict equality in the sense of the \hyperref[def:zfc/extensionality]{axiom of extensionality} in \hyperref[def:zfc]{\logic{ZFC}}.

    Examples of isomorphic categories include \fullref{thm:order_category_isomorphism} and \fullref{ex:isomorphism_of_categories}.

    \thmitem{rem:category_similarity/equivalence} A weaker but very useful notion is \term{category equivalence} defined in \fullref{def:category_equivalence}.
  \end{thmenum}
\end{remark}

\begin{definition}\label{def:category_equivalence}\mcite[def. 1.3.15]{Leinster2016Basic}
  An \term{equivalence} between the \hyperref[def:category]{categories} \( \cat{C} \) and \( \cat{D} \) is a quadruple
  \begin{equation}\label{eq:def:category_equivalence/signature}
    \begin{aligned}
                F &: \cat{C} \to \cat{D}, \\
                G &: \cat{D} \to \cat{C}, \\
             \eta &: \id_{\cat{C}} \Rightarrow G \bincirc F, \\
      \varepsilon &: F \bincirc G \Rightarrow \id_{\cat{D}},
    \end{aligned}
  \end{equation}
  where \( \eta \) and \( \varepsilon \) are \hyperref[thm:natural_isomorphism]{natural isomorphisms}.

  We call \( \eta \) the \term{unit} of the equivalence and \( \varepsilon \) the \term{counit}.

  If \( (F, G, \eta, \varepsilon) \) is an equivalence, we say that \( \cat{C} \) and \( \cat{D} \) are equivalent categories.

  Note that an equivalence is not an \hyperref[def:category_adjunction]{adjunction}, they simply have a common setup.
\end{definition}

\begin{remark}\label{rem:category_equivalence}
  As discussed in \fullref{rem:category_similarity/isomorphism}, isomorphisms of categories as objects in \( \cat{Cat} \) are often too restrictive for our needs. Every isomorphism is an equivalence with \( G \) simply being the inverse of \( F \) and \( \eta \) and \( \varepsilon \) being the corresponding \hyperref[eq:def:functor_category/identity]{identity natural transformations}.
\end{remark}

\begin{proposition}\label{thm:discrete_category_equivalence}
  Let \( \cat{C} \) and \( \cat{D} \) be \hyperref[def:discrete_category]{discrete categories}. Then \( \cat{C} \) and \( \cat{D} \) are \hyperref[def:category_equivalence]{equivalent} if and only if the underlying sets \( \obj(\cat{C}) \) and \( \obj(\cat{D}) \) are \hyperref[def:equinumerosity]{equinumerous}.
\end{proposition}
\begin{proof}
  \SufficiencySubProof Suppose that \( (F, G, \eta, \varepsilon) \) be a category equivalence.

  The unit natural transformation \( \eta: \id_{\cat{C}} \Rightarrow G \bincirc F \) consists of a morphism
  \begin{equation*}
    \eta_A: A \to [G \bincirc F](A)
  \end{equation*}
  for every object \( A \) of \( \cat{C} \). Since the only morphisms in \( \cat{C} \) are the identities, it follows that \( \eta_A = \id_A \) and hence \( [G \bincirc F](A) = A \). In particular, this implies that \( \eta \) is the \hyperref[eq:def:functor_category/identity]{identity natural transformation} on \( \id_{\cat{C}} \) and that the restriction \( G\restr_{\obj(D)} \) is a left inverse of \( F\restr_{\obj(\cat{C})} \).

  Similarly, for the counit \( \varepsilon: F \bincirc G \Rightarrow \id_{\cat{D}} \), for every object \( X \) in \( \cat{D} \) we have \( \varepsilon_X = \id_X \) and hence \( [F \bincirc G](X) = X \). Thus, \( \eta \) is the identity natural transformation on \( \id_{\cat{D}} \) and \( G\restr_{\obj(D)} \) is a right inverse of \( F\restr_{\obj(\cat{C})} \).

  Therefore, the sets \( \obj(\cat{C}) \) and \( \obj(D) \) are equinumerous.

  \NecessitySubProof Suppose that \( F: \obj(\cat{C}) \to \obj(D) \) is a bijective function. Then it is an isomorphism in the category \( \ucat{Cat} \) for an appropriate universe \( \mscrU \), hence it induces an equivalence between \( \cat{C} \) and \( \cat{D} \).
\end{proof}

\begin{proposition}\label{thm:inverse_of_category_equivalence}
  If \( (F, G, \eta, \varepsilon) \) is an \hyperref[def:category_equivalence]{equivalence of categories}, so is \( (G, F, \varepsilon^{-1}, \eta^{-1}) \).
\end{proposition}
\begin{proof}
  Trivial.
\end{proof}

\begin{proposition}\label{thm:equivalence_induces_fully_faithful_and_essentially_surjective_functor}
  In any \hyperref[def:category_equivalence]{category equivalence} \( (F, G, \eta, \varepsilon) \), the functor \( F \) is \hyperref[def:functor_invertibility/fully_faithful]{fully faithful} and \hyperref[def:functor_invertibility/surjective_on_objects]{essentially surjective on objects}.

  The converse of this statement is \fullref{thm:fully_faithful_and_essentially_surjective_functor_induces_equivalence}.
\end{proposition}
\begin{proof}
  \SubProofOf[def:functor_invertibility/surjective_on_objects]{essential surjectivity} For any object \( X \) in \( \cat{D} \), \( A \coloneqq G(X) \) is an object in \( \cat{C} \).

  By definition of category equivalence, the morphism
  \begin{equation*}
    \varepsilon_X: \underbrace{[F \bincirc G](X)}_{F(A)} \to X
  \end{equation*}
  is an isomorphism.

  Therefore, for every object \( X \) in \( \cat{D} \), there exists some object \( A \) in \( \cat{C} \) such that \( F(A) \cong X \). Thus, \( F \) is essentially surjective.

  \SubProofOf[def:functor_invertibility/faithful]{faithfulness} Fix some objects \( A \) and \( B \) in \( \cat{C} \). Let \( r: A \to B \) and \( s: A \to B \) be morphisms such that \( F(p) = F(q) \).

  From the naturality of \( \eta \) it follows that the following diagram commutes:
  \begin{equation}\label{eq:thm:equivalence_induces_fully_faithful_and_essentially_surjective_functor/faithfullness}
    \begin{aligned}
      \includegraphics[page=1]{figures/thm__category_equivalence_is_fully_faithful_and_essentially_surjective.pdf}
    \end{aligned}
  \end{equation}

  Therefore, \( \eta_B \bincirc r = \eta_B \bincirc s \) and, since \( \eta_B \) is left-cancellative, \( r = s \).

  \SubProofOf[def:functor_invertibility/full]{fullness} Fix some objects \( A \) and \( B \) in \( \cat{C} \). Let \( s: F(A) \to F(B) \) be an arbitrary morphism.

  We can now define a function \( r: A \to B \) via the composition
  \begin{equation}\label{eq:thm:equivalence_induces_fully_faithful_and_essentially_surjective_functor/fullness/def}
    \begin{aligned}
      \includegraphics[page=2]{figures/thm__category_equivalence_is_fully_faithful_and_essentially_surjective.pdf}
    \end{aligned}
  \end{equation}

  Again from the naturality of \( \eta \), it follows that the following diagram commutes:
  \begin{equation}\label{eq:thm:equivalence_induces_fully_faithful_and_essentially_surjective_functor/fullness/nat_eta}
    \begin{aligned}
      \includegraphics[page=3]{figures/thm__category_equivalence_is_fully_faithful_and_essentially_surjective.pdf}
    \end{aligned}
  \end{equation}

  Therefore,
  \begin{equation*}
    (\eta_B^{-1} \bincirc G(q) \bincirc \eta_A) = s = (\eta_B^{-1} \bincirc [G \bincirc F](p) \bincirc \eta_A).
  \end{equation*}

  After performing the necessary composition to cancel \( \eta_A \) and \( \eta_B^{-1} \), we obtain
  \begin{equation*}
     G(q) = (\eta_B \bincirc s \bincirc \eta_A^{-1}) = [G \bincirc F](p).
  \end{equation*}

  From the naturality of \( \varepsilon \), it follows that the following diagram commutes:
  \begin{equation}\label{eq:thm:equivalence_induces_fully_faithful_and_essentially_surjective_functor/fullness/nat_varepsilon}
    \begin{aligned}
      \includegraphics[page=4]{figures/thm__category_equivalence_is_fully_faithful_and_essentially_surjective.pdf}
    \end{aligned}
  \end{equation}

  Therefore,
  \begin{equation*}
   r = (\varepsilon_B \bincirc \underbrace{[F \bincirc G](q)}_{[F \bincirc G \bincirc F](p)} \bincirc \varepsilon_A^{-1}) = F(p).
  \end{equation*}
\end{proof}

\begin{theorem}[Fully faithful and essentially surjective functor induces equivalence]\label{thm:fully_faithful_and_essentially_surjective_functor_induces_equivalence}
  Every \hyperref[def:functor_invertibility/fully_faithful]{fully faithful} and \hyperref[def:functor_invertibility/surjective_on_objects]{essentially surjective on objects} functor induces a \hyperref[def:category_equivalence]{category equivalence}.

  More precisely, given a functor \( F: \cat{C} \to \cat{D} \) that is fully faithful and essentially surjective on objects, there exists a functor \( G: \cat{D} \to \cat{C} \) and \hyperref[def:natural_transformation]{natural transformations}
  \begin{align*}
    \eta        &: \id_{\cat{C}} \Rightarrow G \bincirc F, \\
    \varepsilon &: F \bincirc G \Rightarrow \id_{\cat{D}},
  \end{align*}
  such that the quadruple \( (F, G, \eta, \varepsilon) \) is a \hyperref[def:category_equivalence]{category equivalence}.

  In \hyperref[def:zfc]{\logic{ZF}}, this theorem is equivalent to the \hyperref[def:zfc/choice]{axiom of choice} --- see \fullref{thm:axiom_of_choice_equivalences/fully_faithful_essentially_surjective}.

  We prove the converse of this statement separately in \fullref{thm:equivalence_induces_fully_faithful_and_essentially_surjective_functor}.
\end{theorem}
\begin{proof}
  \ImplicationSubProof[def:zfc/choice]{the axiom of choice}[thm:fully_faithful_and_essentially_surjective_functor_induces_equivalence]{functors induce equivalences} Suppose that the axiom of choice holds and let \( F \) be a fully faithful functor that is surjective on objects.

  From essential surjectivity of \( F \), it follows that for every object \( X \) in \( \cat{D} \), the preimage of \( X \) under \( F \) is nonempty. The preimage of \( X \) is the set \( \mscrA_X \) of objects in \( \cat{C} \) such that \( A \in \mscrA_X \) if and only if \( F(A) \cong X \). We use the axiom of choice on the family \( \set{ \mscrA_X }_{X \in \cat{D}} \) to select a single preimage for every \( X \), which we denote by \( G(X) \).

  Again using the axiom of choice, we pick an isomorphism \( \varepsilon_X: F(G(X)) \to X \) for every \( X \).

  We have defined a function \( G \) from \( \obj(\cat{D}) \) to \( \obj(\cat{C}) \). In order to \( G \) to become a functor, we must extend it to morphisms. Let \( X \) and \( Y \) be objects in \( \cat{D} \) and \( s: X \to Y \) be any morphism.

  Consider the morphism
  \begin{equation*}
    \varepsilon_Z^{-1} \bincirc r \bincirc \varepsilon_X: [F \bincirc G](X) \to [F \bincirc G](Y).
  \end{equation*}

  Since \( F \) is fully faithful, there exists a unique morphism \( s \) in \( \cat{D}(G(X), G(Y)) \) such that the following diagram commutes:
  \begin{equation}\label{eq:thm:fully_faithful_and_essentially_surjective_functor_induces_equivalence/inverse_morphism_definition}
    \begin{aligned}
      \includegraphics[page=1]{figures/thm__fully_faithful_and_essentially_surjective_functor_induces_equivalence.pdf}
    \end{aligned}
  \end{equation}

  We define \( G(q) \coloneqq s \).

  In order to prove that \( G \) is a functor, we need to show that \ref{def:functor/CF1} and \ref{def:functor/CF2} hold.

  For \ref{def:functor/CF1}, note that the following diagram commutes for any object \( X \) in \( \cat{D} \):
  \begin{equation}\label{eq:thm:fully_faithful_and_essentially_surjective_functor_induces_equivalence/identity}
    \begin{aligned}
      \includegraphics[page=2]{figures/thm__fully_faithful_and_essentially_surjective_functor_induces_equivalence.pdf}
    \end{aligned}
  \end{equation}

  Note that \eqref{eq:thm__fully_faithful_and_essentially_surjective_functor_induces_equivalence/inverse_morphism_definition} also commutes if we replace \( [F \bincirc G](\id_X) \) with \( F(\id_{G(X)}) \). Since \( F \) is fully faithful, this morphism is unique and it follows that
  \begin{equation*}
    [F \bincirc G](\id_X) = F(\id_{G(X)}).
  \end{equation*}

  For \ref{def:functor/CF2}, analogously, given morphisms \( s: X \to Y \) and \( r: Y \to Z \), the following diagram commutes:
  \begin{equation}\label{eq:thm:fully_faithful_and_essentially_surjective_functor_induces_equivalence/composition}
    \begin{aligned}
      \includegraphics[page=3]{figures/thm__fully_faithful_and_essentially_surjective_functor_induces_equivalence.pdf}
    \end{aligned}
  \end{equation}

  We have implicitly used \fullref{thm:inverting_isomorphisms_preserves_commutativity} above.

  By the same uniqueness argument used for \ref{def:functor/CF1}, we conclude that
  \begin{equation*}
    G(r \bincirc q) = G(r) \bincirc G(q).
  \end{equation*}

  We have shown that \( G \) is a functor. Furthermore, \( \varepsilon \) is a natural transformation since, for any morphism \( s: X \to Y \) in \( \cat{D} \), the following diagram commutes:
  \begin{equation}\label{eq:thm:fully_faithful_and_essentially_surjective_functor_induces_equivalence/varepsilon}
    \begin{aligned}
      \includegraphics[page=4]{figures/thm__fully_faithful_and_essentially_surjective_functor_induces_equivalence.pdf}
    \end{aligned}
  \end{equation}

  To show that \( F \) induces an equivalence, it now only remains to define a unit natural transformation \( \eta: \id_{\cat{C}} \to G \bincirc F \). For every object \( A \) in \( \cat{C} \) we have an isomorphism
  \begin{equation*}
    \varepsilon_{F(A)}^{-1}: F(A) \to [F \bincirc G \bincirc F](A).
  \end{equation*}

  Using \( G(\varepsilon_{F(A)}^{-1}) \) will get us nowhere. Fortunately, \( F \) is fully faithful, so there is a bijective function
  \begin{equation*}
    \varphi: \cat{D}(F(A), [F \bincirc G \bincirc F](A)) \to \cat{C}(A, [F \bincirc G](A)).
  \end{equation*}

  Hence, we can define
  \begin{equation*}
    \eta_A \coloneqq \varphi(\varepsilon_{F(A)}^{-1})
  \end{equation*}
  so that \( F(\eta_A) = \varepsilon_{F(A)}^{-1} \). By \fullref{thm:def:functor_invertibility/properties/fully_faithful_reflects_isomorphisms}, \( \eta_A \) is also an isomorphism.

  By naturality of \( F(\eta_A) \), the following diagram commutes:
  \begin{equation}\label{eq:thm:fully_faithful_and_essentially_surjective_functor_induces_equivalence/varepsilon_image_nat}
    \begin{aligned}
      \includegraphics[page=5]{figures/thm__fully_faithful_and_essentially_surjective_functor_induces_equivalence.pdf}
    \end{aligned}
  \end{equation}

  Hence, by \fullref{thm:commutative_diagrams_preserved_and_reflected}, the following diagram also commutes:
  \begin{equation}\label{eq:thm:fully_faithful_and_essentially_surjective_functor_induces_equivalence/varepsilon_source_nat}
    \begin{aligned}
      \includegraphics[page=6]{figures/thm__fully_faithful_and_essentially_surjective_functor_induces_equivalence.pdf}
    \end{aligned}
  \end{equation}

  Therefore, the quadruple \( (F, G, \eta, \varepsilon) \) is an equivalence of categories.

  \ImplicationSubProof[thm:fully_faithful_and_essentially_surjective_functor_induces_equivalence]{functors induce equivalences}[def:zfc/choice]{the axiom of choice} Let \( \mscrA \) be a family of nonempty sets. Let \( \cat{D} \) be the \hyperref[def:discrete_category]{discrete category} induced by \( \mscrA \).

  Define the category \( \cat{C} \) as follows:
  \begin{itemize}
    \item The \hyperref[def:category/objects]{set of objects} \( \obj(\cat{C}) \) is the \hyperref[def:disjoint_union]{disjoint union} \( \bigsqcup_{A \in \mscrA} A \).

    \item The \hyperref[def:category/morphisms]{set of morphisms} \( \cat{C}((A, x), (B, y)) \) has a single morphism if \( A = B \) and no morphisms otherwise. This single morphism can be encoded as the triple \( (A, x, y) \).

    \item The \hyperref[def:category/composition]{composition of the morphisms} \( (A, x, y) \) and \( (A, y, z) \) is the morphism \( (A, x, z) \).

    \item The \hyperref[def:category/identity]{identity morphism} on the object \( (A, x) \in \cat{C} \) is \( (A, x, x) \).
  \end{itemize}

  Define the functor
  \begin{equation*}
    \begin{aligned}
      &F: \cat{C} \to \cat{D} \\
      &F(A, x) \coloneqq A \\
      &F(A, x, y) \coloneqq \id_A
    \end{aligned}
  \end{equation*}
  that maps each point \( x \in A \in \mscrA \) into the set \( A \) it belongs to. We have taken the disjoint union of \( \mscrA \) since otherwise there may not be a canonical choice of set \( A \) for \( F \) to send \( x \) to. Thus, the functor is surjective on objects (not essentially surjective but actually surjective).

  Note that \( \cat{D}(F(A, x), F(B, y)) \) has a single morphism if \( A = B \) and is empty otherwise. From this it follows that \( F \) is fully faithful.

  Therefore, \( F \) induces a \hyperref[def:category_equivalence]{category equivalence} \( (F, G, \eta, \varepsilon) \). The functor \( G \) chooses an object \( (A, x) \) of \( \cat{C} \) for each object \( A \) of \( \cat{D} \). This induces a \hyperref[def:choice_function]{choice function} on \( \mscrA \).

  We have shown that the axiom of choice holds.
\end{proof}

\begin{definition}\label{def:groupoid}
  A \term{groupoid} is a category whose only morphisms are \hyperref[def:morphism_invertibility/isomorphism]{isomorphisms}.
\end{definition}

\begin{definition}\label{def:monoid_delooping}\mcite[def. 1.1.7]{Perrone2019}
  Let \( (\mscrM, \cdot, e) \) be a \hyperref[def:monoid]{monoid}. The \term{delooping} \( \cat{B}_\mscrM \) of \( \mscrM \) is the following category:
  \begin{itemize}
    \item The \hyperref[def:category/objects]{set of objects} \( \obj(\cat{B}_\mscrM) \) is the singleton set \( \set{ \anon } \), where \( \anon \) is any set not in the set of all morphisms \( \mscrM \).

    \item The only \hyperref[def:category/morphisms]{set of morphisms} \( \cat{B}_\mscrM(\anon) \) is the underlying set \( \mscrM \) of the monoid.

    \item The \hyperref[def:category/composition]{composition of the morphisms} \( x \) and \( y \) is the multiplication:
    \begin{equation*}
      y \bincirc x \coloneqq y \cdot x.
    \end{equation*}

    Note how we write composition in the same order as multiplication. This may seem to contradict the general convention, however it is consistent with groups being regarded as sets of invertible transformations.

    \item The \hyperref[def:category/identity]{identity morphism} is \( e \).
  \end{itemize}
\end{definition}

\begin{proposition}\label{thm:delooping_of_group}
  The \hyperref[def:delooping]{delooping} of a \hyperref[def:group]{group} \( \mscrG \) is a \hyperref[def:groupoid]{groupoid}.

  There is a restricted form of a converse --- see \fullref{thm:connected_delooping}.
\end{proposition}
\begin{proof}
  Trivial.
\end{proof}

\begin{definition}\label{def:connected_category}\mcite[exer. 1.5.20]{Perrone2019}
  A \hyperref[def:category]{category} is \term{connected} if there exists a morphism between any two objects. That is, for a connected category \( \cat{C} \), given two objects \( X \) and \( Y \), either \( \cat{C}(X, Y) \) or \( \cat{C}(Y, X) \) is nonempty.

  Connected categories are used in \fullref{thm:connected_delooping} and \fullref{thm:order_category_isomorphism/totally_ordered}.
\end{definition}

\begin{proposition}\label{thm:connected_delooping}
  For every \hyperref[def:connected_category]{connected} \hyperref[def:groupoid]{groupoid} \( \cat{G} \) there exists a \hyperref[def:group]{group} \( \mscrG \) such that \( \cat{G} \) is \hyperref[def:category_equivalence]{equivalent} to the \hyperref[def:monoid_delooping]{delooping} \( \cat{B}_\mscrG \). Furthermore, if \( \cat{G} \) has only one object, then this equivalence is an \hyperref[rem:category_similarity/isomorphism]{isomorphism}.

  See \fullref{thm:delooping_of_group} for a much simpler converse.
\end{proposition}
\begin{proof}
  Define the group \( \mscrG \) as follows:
  \begin{itemize}
    \item Let the underlying set of \( \mscrG \) be the set of all morphisms of \( \cat{G} \).
    \item Define \( x \cdot y = z \) to hold whenever the following diagram commutes:
    \begin{equation}\label{eq:thm:connected_delooping/mult}
      \begin{aligned}
        \includegraphics[page=1]{figures/thm__connected_delooping.pdf}
      \end{aligned}
    \end{equation}
    \item Pick the identity of \( \mscrG \) out of the identity morphisms in \( \mscrG \).
    \item Let the inverse of \( x \) be the inverse morphism \( x^{-1} \).
  \end{itemize}

  Finally, define the \hyperref[ex:constant_functor]{constant functor} \( F: \cat{G} \to \cat{B}_\mscrG \). It is obviously surjective on objects.

  For every morphism \( x: \anon \to \anon \) in the delooping, there exists a unique \( f: X \to Y \) in \( \cat{G} \) such that \( F(f) = x \). Hence, \( f \) is surjective on morphisms, and by \fullref{thm:def:functor_invertibility/properties/surjective}, full and surjective on objects.

  Furthermore, for any objects \( X \) and \( Y \) in \( \cat{G} \), we have
  \begin{equation*}
    \cat{G}(X, Y) \subseteq \cat{B}_\mscrG(F(X), F(Y)).
  \end{equation*}

  Hence, \( G \) is faithful. Therefore, by \fullref{thm:fully_faithful_and_essentially_surjective_functor_induces_equivalence}, the groupoid \( \cat{G} \) is equivalent to the delooping \( \cat{B}_\mscrG \).

  Furthermore, if \( \cat{G} \) has only one object, it is also injective on objects and, by \fullref{thm:def:functor_invertibility/properties/injective}, injective on morphisms. In this case, we have an isomorphism due to \fullref{thm:def:functor_invertibility/properties/isomorphism}.
\end{proof}

\begin{definition}\label{def:skeletal_category}\mcite[91]{MacLane1994}
  The category \( \cat{S} \) is called \term{skeletal} if the only isomorphisms in \( \cat{S} \) are the identity morphisms.

  If \( \cat{S} \) is a subcategory of \( \cat{C} \) and if they are \hyperref[def:category_equivalence]{equivalent}, we say that \( \cat{S} \) is a \term{skeleton} of \( \cat{C} \).
\end{definition}

\begin{theorem}[Category skeleton existence]\label{thm:category_skeleton_existence}
  Every \hyperref[def:category]{category} has a \hyperref[def:skeletal_category]{skeleton}.

  In \hyperref[def:zfc]{\logic{ZF}}, this theorem is equivalent to the \hyperref[def:zfc/choice]{axiom of choice} --- see \fullref{thm:axiom_of_choice_equivalences/skeletons}.
\end{theorem}
\begin{proof}
  \ImplicationSubProof[def:zfc/choice]{the axiom of choice}[thm:category_skeleton_existence]{skeleton existence} Suppose that the axiom of choice holds and let \( F \) be a fully faithful functor that is surjective on objects.

  Fix a category \( \cat{C} \). We will build a subcategory \( \cat{S} \) of \( \cat{C} \) whose inclusion functor \( \Iota: \cat{S} \to \cat{C} \) is essentially surjective and fully faithful. By \fullref{thm:fully_faithful_and_essentially_surjective_functor_induces_equivalence}, this is sufficient for \( \cat{S} \) and \( \cat{C} \) to be equivalent.

  In order for \( \Iota \) to be a full functor, \( \cat{S} \) must be a full subcategory. Therefore, when building \( \cat{S} \), we can only remove objects and must preserve the morphism sets for the remaining objects.

  Denote by \( \obj(\cat{C}) / \cong \) the quotient of \( \cat{C} \) by the isomorphism relation. Using the axiom of choice, we can obtain a \hyperref[def:choice_function]{choice function} \( c: \obj(\cat{C}) / \cong \to \obj(\cat{C}) \).

  Define \( \cat{S} \) as the subcategory induced by the image \( c[\obj(\cat{C}) / \cong] \).

  Now consider the \hyperref[def:functor/inclusion]{inclusion functor} \( \Iota: \cat{S} \to \cat{C} \). For every pair \( X \) and \( Y \) of objects in \( \cat{S} \), clearly
  \begin{equation*}
    \cat{S}(X, Y) = \cat{C}(\Iota(X), \Iota(Y)).
  \end{equation*}

  Hence, \( \Iota \) is fully faithful.

  Now let \( X \) and \( Y \) be objects of \( \cat{C} \). Since the objects of \( \cat{S} \) were chosen from isomorphism classes of \( \cat{C} \), there exist objects \( X' \) and \( Y' \) in \( \cat{S} \) that are isomorphic to \( X \) and \( Y \), correspondingly. Hence, \( \Iota \) is essentially surjective.

  Therefore, \( F \) satisfies \fullref{thm:fully_faithful_and_essentially_surjective_functor_induces_equivalence}, from which is follows that \( \cat{C} \) and \( \cat{S} \) are equivalent.

  \ImplicationSubProof[thm:category_skeleton_existence]{skeleton existence}[def:zfc/choice]{the axiom of choice} Suppose that every category has a skeleton.

  Let \( \mscrA \) be a family of nonempty sets. Construct a category \( \cat{C} \)from the \hyperref[def:disjoint_union]{disjoint union} \( \bigsqcup_{A \in \mscrA} A \), where a morphism exists only between members of the same set. This construction is performed in detail in the proof of \fullref{thm:fully_faithful_and_essentially_surjective_functor_induces_equivalence}.

  Then \( \cat{C} \) has a skeleton \( \cat{S} \). All morphisms in \( \cat{C} \) are isomorphisms, hence the set \( \obj(\cat{S}) \) contains exactly one representative for each set in the family \( \mscrA \).

  More precisely, define the set
  \begin{equation*}
    S \coloneqq \set{ x \given (A, x) \in \obj(\cat{S}) }.
  \end{equation*}

  Then \( S \) satisfies \fullref{thm:axiom_of_choice_equivalences/choice_sets}.

  Since the family \( \mscrA \) is arbitrary, we conclude that the axiom of choice holds.
\end{proof}

\begin{remark}\label{rem:skeletons_and_thin_categories}
  Rather than defining representatives of equivalence classes, as in \fullref{thm:category_skeleton_existence}, we can define morphisms between the equivalence classes themselves, as in \fullref{thm:preorder_to_partial_order}.

  This does not require the \hyperref[def:zfc/choice]{axiom of choice}, but is rarely applicable, unfortunately. One case where it is applicable is in \hyperref[def:thin_category]{thin categories} --- see \fullref{thm:order_category_isomorphism}.
\end{remark}

\begin{definition}\label{def:thin_category}\mcite{nLab:thin_category}
  A \hyperref[def:category]{category} is \term{thin} if any two parallel morphisms are equal.

  This is equivalent to saying that the function for every two objects \( A \) and \( B \) in \( \cat{P} \), whenever the set \( \cat{P}(A, B) \) is at most a singleton.

  As shown in \fullref{ex:preorder_nonuniqueness} and discussed in \fullref{thm:order_category_isomorphism}, a thin category may not be \hyperref[def:skeletal_category]{skeletal}.

  Thin categories are often conflated with preordered sets due to \fullref{thm:order_category_isomorphism/preordered}.
\end{definition}

\begin{theorem}[Ordered sets as categories]\label{thm:order_category_isomorphism}
  Regarding ordered sets, we have the following \hyperref[rem:category_similarity/isomorphism]{category isomorphisms} (assuming a \hyperref[def:category_size]{fixed Grothendieck universe} \( \mscrU \)):
  \begin{thmenum}
    \thmitem{thm:order_category_isomorphism/preordered} The categories \hyperref[def:preordered_set/category]{\( \ucat{PreOrd} \)} and \( \ucat{Thin} \), where
    \begin{itemize}
      \item \( \ucat{PreOrd} \) is the category of \( \mscrU \)-small \hyperref[def:preordered_set]{preordered sets} and (nonstrict) \hyperref[def:preordered_set/homomorphism]{monotone maps}.

      \item \( \ucat{Thin} \) is the subcategory of \hyperref[def:category_of_small_categories]{\( \ucat{Cat} \)} \hyperref[def:subcategory]{induced} by \hyperref[def:thin_category]{thin categories}, i.e. the category of \( \mscrU \)-small thin categories.
    \end{itemize}

    \thmitem{thm:order_category_isomorphism/partially_ordered} The categories \hyperref[def:partially_ordered_set/category]{\( \ucat{Pos} \)} and \( \ucat{SkelThin} \), where
    \begin{itemize}
      \item \( \ucat{Pos} \) is the subcategory of \( \ucat{PreOrd} \) induced by (nonstrict) \hyperref[def:partially_ordered_set]{partially ordered sets}.

      \item \( \ucat{SkelThin} \) is the subcategory of \( \ucat{Thin} \) induced by \hyperref[def:skeletal_category]{skeletal categories}, i.e. the category of \( \mscrU \)-small thin skeletal categories.
    \end{itemize}

    \thmitem{thm:order_category_isomorphism/totally_ordered} The categories \hyperref[def:totally_ordered_set]{\( \ucat{Tos} \)} and \( \ucat{ConnSkelThin} \), where
    \begin{itemize}
      \item \( \ucat{Tos} \) is the subcategory of \( \ucat{Pos} \) induced by (nonstrict) \hyperref[def:totally_ordered_set]{totally ordered sets}.

      \item \( \ucat{ConnSkelThin} \) is the subcategory of \( \ucat{SkelThin} \) induced by \hyperref[def:connected_category]{connected categories}, i.e. the category of \( \mscrU \)-small thin skeletal connected categories.
    \end{itemize}
  \end{thmenum}
\end{theorem}
\begin{proof}
  \SubProofOf{thm:order_category_isomorphism/preordered}
  \SubProof*{Proof that preorders induce thin categories} Let \( (\mscrP, \leq) \) be a \( \mscrU \)-small preordered set. Let \( \cat{P} \) be the \hyperref[def:quiver_free_category]{free category} obtained by regarding \( (\mscrP, \leq) \) as a \hyperref[def:quiver]{quiver}. Explicitly, the category \( \cat{P} \) is built as follows:
  \begin{itemize}
    \item The \hyperref[def:category/objects]{set of objects} \( \obj(\cat{P}) \) is simply \( \mscrP \).

    \item The \hyperref[def:category/morphisms]{set of morphisms} \( \cat{P}(x, y) \) consists of the tuple \( (x, y) \) if \( x \leq y \) and is empty otherwise.

    \item The \hyperref[def:category/composition]{composition of the morphisms} \( (x, y) \) and \( (y, z) \) is simply \( (x, y) \). This is well-defined because of the \hyperref[def:binary_relation/transitive]{transitivity} of \( \leq \).

    \item The \hyperref[def:category/identity]{identity morphism} on the object \( x \in \cat{C} \) is \( (x, x) \). This is well-defined because of the \hyperref[def:binary_relation/reflexive]{reflexivity} of \( \leq \).
  \end{itemize}

  The category is clearly thin because \( \leq \) is a binary relation and ordered tuples with the same elements are equal.

  Now let \( f: \mscrP \to \mscrQ \) be a nonstrict monotone map. It induces the functor
  \begin{equation*}
    \begin{aligned}
      &F: \cat{P} \to \cat{Q} \\
      &F(x) \coloneqq f(x) \\
      &F(x, y) \coloneqq (f(x), f(y)).
    \end{aligned}
  \end{equation*}

  \ref{def:functor/CF1} is immediate and \ref{def:functor/CF2} follows from \eqref{eq:def:preordered_set/homomorphism}, hence \( F \) is indeed a functor.

  \SubProof*{Proof that thin categories induce preorders} Let \( \cat{P} \) be a \( \mscrU \)-small category. Define the binary relation
  \begin{equation*}
    X \leq Y \T{if and only if} \cat{P}(X, Y) \neq \varnothing.
  \end{equation*}

  This is a binary relation over the set \( \mscrP \coloneqq \obj(\cat{P}) \). It is reflexive because of the existence of identity morphisms in \( \cat{P} \) and transitive because of the requirement that the composition of compatible morphisms exists.

  Therefore, \( (\mscrP, \leq) \) is a preordered set.

  Given a functor \( F: \cat{P} \to \cat{Q} \), the restriction \( F\restr_{\obj(\cat{P})} \) is a monotone map from \( (\mscrP, \leq_\mscrP) \) to \( (\mscrQ, \leq_\mscrQ) \).

  Indeed, if \( X \leq Y \) for \( X, Y \in \mscrP \), then \( \cat{P}(X, Y) \neq \varnothing \). Hence, \( \cat{P}(F(X), F(Y)) \neq \varnothing \) and \( F(X) \leq F(Y) \).

  \SubProof*{Proof of isomorphism} We have implicitly defined a functor from \( \cat{PreOrd} \) to \( \cat{Thin} \) and vice versa. Isomorphism of \( \cat{PreOrd} \) and \( \cat{Thin} \) requires that these functors are mutually inverse.

  We need to prove that the induced preordered set for the induced thin category of a preordered set is the same as the original. In the other direction, we need to prove that the induced thin category for the induced preordered set of a thin category is the same as the original.

  Both of these proofs are trivial but nevertheless the fact that we need to perform such a check is important.

  \SubProofOf{thm:order_category_isomorphism/partially_ordered} We have already shown in \fullref{thm:order_category_isomorphism/preordered} the isomorphism between \( \ucat{PreOrd} \) and \( \ucat{Thin} \).

  For a thin category, there is an isomorphism between \( x \) and \( y \) if and only if there is both a morphism from \( x \) to \( y \) and one from \( y \) to \( x \). This isomorphism may not be unique as a consequence of \fullref{ex:preorder_nonuniqueness}. Uniqueness requires \( x = y \) to hold in this case, which is in turn equivalent to partial order \hyperref[def:binary_relation/antisymmetry]{antisymmetry}.

  \SubProofOf{thm:order_category_isomorphism/totally_ordered} We have already shown in \fullref{thm:order_category_isomorphism/partially_ordered} the isomorphism between \( \ucat{Pos} \) and \( \ucat{SkelThin} \).

  \hyperref[def:connected_category]{Connectedness} of a category \( \cat{P} \) is then equivalent to \hyperref[def:binary_relation/total]{totality} of \( (\mscrP, \leq) \).
\end{proof}
