\subsection{First-order logic}\label{subsec:first_order_logic}

\begin{definition}\label{def:first_order_language}\mcite[sec. 15.2]{OpenLogicFull}
  The idea of first-order predicate logic is to create a formal language whose semantics (given by structures) support boolean operations and can quantify over all elements of an ambient universe. Unlike in \hyperref[subsec:propositional_logic]{propositional logic}, there are different first-order logic languages.

  The alphabet for a \term{first-order predicate logic \hyperref[def:language]{language}} \( \mscrL \) extends that of \hyperref[subsec:propositional_logic]{propositional logic} and consists of two types of symbols (note that \fullref{rem:propositional_language_is_alphabet} holds here also).

  \begin{description}
    \item[Logical symbols]
    \hfill
    \begin{thmenum}[series=def:first_order_language]
      \thmitem{def:first_order_language/propositional} The entirety of the \hyperref[subsec:propositional_logic]{propositional logic alphabet}.

      \thmitem{def:first_order_language/var} A nonempty at most \hyperref[rem:cardinals/countably_infinite]{countable} alphabet of variables \( \boldop{Var} \), usually denoted by small greek letters \( \xi_1, \xi_2, \ldots \) or \( \xi, \eta, \zeta \) --- see \fullref{rem:mathematical_logic_conventions/variable_symbols}.

      \thmitem{def:first_order_language/quantifiers} The quantifiers \( Q = \set{ \forall, \exists } \):
      \begin{thmenum}
        \thmitem{def:first_order_language/quantifiers/universal} The \term{universal quantifier} \( \forall \).
        \thmitem{def:first_order_language/quantifiers/existential} The \term{existential quantifier} \( \exists \).
        \thmitem{def:first_order_language/quantifiers/dot} The dot \( . \) for separating a quantifier from its formula.
      \end{thmenum}

      The dot is not itself a quantifier and is not formally necessary --- we use it only for readability.

      \thmitem{def:first_order_language/equality} A symbol for \term{formal equality} \( \doteq \). Equality is sometimes omitted by logicians but examples of first-order languages without formal equality are obscure.
    \end{thmenum}

    \item[Non-logical symbols]
    \hfill
    \begin{thmenum}[resume=def:first_order_language]
      \thmitem{def:first_order_language/func} A set of functional symbols, \( \boldop{Fun} \), whose elements are usually denoted by \( f_1, f_2, \ldots \) or \( f, g \) or by symbols like \( \otimes \). In the latter case we usually use infix notation. Each functional symbol has an associated natural number called its \term{arity}, denoted by \( \# f \). Functional symbols with a zero arity are called \term{constants}.

      \thmitem{def:first_order_language/pred} A set of predicate symbols, \( \boldop{Pred} \), whose elements are usually denoted by \( p_1, p_2, \ldots \) or by symbols like \( \leq \). Again, in the latter case we use infix notation. Predicate symbols also have an associated arity. Predicate symbols with zero arity are called \term{propositional variables}.
    \end{thmenum}
  \end{description}

  The logical symbols are common for all first-order languages. Thus first-order languages differ by their non-logical symbols. The collection of functional and predicate symbols of a language are sometimes called its \term{signature}.
\end{definition}

\begin{definition}\label{def:first_order_syntax}
  Similarly to the \hyperref[def:propositional_syntax]{syntax of propositional logic}, we define the \term{syntax} of a fixed first-order language \( \mscrL \).

  \begin{thmenum}
    \thmitem{def:first_order_syntax/grammar_schema} Consider the following \hyperref[def:backus_naur_form]{grammar schema}:
    \begin{bnf*}
      \bnfprod{variable}        {v \in \boldop{Var}} \\
      \bnfprod{connective}      {\bincirc \in \Sigma} \\
      \bnfprod{quantifier}      {\bnfts{\( \forall \)} \bnfor \bnfts{\( \exists \)}} \\
      \bnfprod{unary function}  {f \in \boldop{Fun}, \#f = 1} \\
      \bnfmore                  {\vdots} \\
      \bnfprod{n-ary function}  {f \in \boldop{Fun}, \#f = n \text{ (standalone rule for each \( n \))}} \\
      \bnfmore                  {\vdots} \\
      \bnfprod{unary predicate} {p \in \boldop{Pred}, \#p = 1} \\
      \bnfmore                  {\vdots} \\
      \bnfprod{n-ary predicate} {p \in \boldop{Pred}, \#p = n \text{ (standalone rule for each \( n \))}} \\
      \bnfmore                  {\vdots} \\
      \bnfprod{term}            {\bnfpn{variable} \bnfor} \\
      \bnfmore                  {\bnfpn{unary function} \bnfsp \bnfts{(} \bnfsp \bnfpn{term} \bnfsp \bnfts{)} \bnfor} \\
      \bnfmore                  {\hspace{3cm} \vdots} \\
      \bnfmore                  {\bnfpn{n-ary function} \bnfsp \underbrace{\bnfts{(} \bnfsp \bnfpn{term} \bnfts{,} \bnfsk \bnfts{,} \bnfpn{term} \bnfsp \bnfts{)}}_{n \text{ terms}} \bnfor} \\
      \bnfmore                  {\hspace{3cm} \vdots} \\
      \bnfprod{atomic formula}  {\bnfts{\( \top \)} \bnfor \bnfts{\( \bot \)} \bnfor} \\
      \bnfmore                  {\bnfts{(} \bnfsp \bnfpn{term} \bnfsp \bnfts{\( \doteq \)} \bnfsp \bnfpn{term} \bnfsp \bnfts{)} \bnfor} \\
      \bnfmore                  {\bnfpn{unary predicate} \bnfsp \bnfts{(} \bnfsp \bnfpn{term} \bnfsp \bnfts{)} \bnfor} \\
      \bnfmore                  {\hspace{3cm} \vdots} \\
      \bnfmore                  {\bnfpn{n-ary predicate} \bnfsp \underbrace{\bnfts{(} \bnfsp \bnfpn{term} \bnfts{,} \bnfsk \bnfts{,} \bnfpn{term} \bnfsp \bnfts{)}}_{n \text{ terms}} \bnfor} \\
      \bnfmore                  {\hspace{3cm} \vdots} \\
      \bnfprod{formula}         {\bnfpn{atomic formula} \bnfor} \\
      \bnfmore                  {\bnfts{\( \neg \)} \bnfpn{formula} \bnfor} \\
      \bnfmore                  {\bnfts{(} \bnfsp \bnfpn{formula} \bnfsp \bnfpn{connective} \bnfsp \bnfpn{formula} \bnfsp \bnfts{)} \bnfor} \\
      \bnfmore                  {\bnfpn{quantifier} \bnfsp \bnfpn{variable} \bnfsp \bnfts{.} \bnfsp \bnfpn{formula}}
    \end{bnf*}

    In practice, we usually only have functions and predicates of specific arities. Note that we can have infinitely many functions but only finitely many different arities. The \hyperref[rem:theory_of_left_monoid_actions]{theory of monoid actions} is an example of a first-order language with infinitely many unary functional symbols, one constant and one binary functional symbol.

    If we need the grammars to have a finite set of rules, except for having only finitely many different arities, we need to introduce appropriate naming conventions for variables, functions and predicates, analogously to the \hyperref[def:propositional_syntax/grammar_schema]{grammar schema of propositional logic}.

    We use the conventions in \fullref{rem:propositional_formula_parentheses} regarding parentheses by extending them wherever appropriate.

    In order to simplify notation, we also use the conventions in \fullref{rem:first_order_formula_conventions}.

    \thmitem{def:first_order_syntax/term} The set \( \boldop{Term} \) of \term{terms} in \( \mscrL \) is the language \hyperref[def:grammar_derivation/grammar_language]{generated} by this grammar schema with \( \bnfpn{term} \) as a starting rule.

    The grammar of first-order terms is unambiguous by \fullref{thm:first_order_terms_and_formulas_are_unambiguous}.

    \thmitem{def:first_order_syntax/subterm} If \( \tau \) and \( \kappa \) are terms and \( \kappa \) is a \hyperref[def:language/subword]{subword} of \( \tau \), we say that \( \kappa \) is a \term{subterm} of \( \tau \).

    \thmitem{def:first_order_syntax/term_variables} For each term \( \tau \), we define its variables as
    \begin{equation}\label{eq:def:first_order_syntax/term_variables}
      \boldop{Var}(\tau) \coloneqq \begin{cases}
        \xi,                                                        &\tau = \xi \in \boldop{Var}, \\
        \boldop{Var}(\tau_1) \cup \ldots \cup \boldop{Var}(\tau_n), &\tau = f(\tau_1, \ldots, \tau_n).
      \end{cases}
    \end{equation}

    As in \fullref{def:propositional_syntax/variables}, \( \boldop{Var} \) is ordered by the position of the first occurrence of a variable.

    \thmitem{def:first_order_syntax/ground_term} A term \( \tau \) is called a \term{ground term} if \( \boldop{Var}(\tau) = \varnothing \). Ground terms are also called \term{closed terms}.

    \thmitem{def:first_order_syntax/atomic_formula} The set \( \boldop{Atom} \) of \term{atomic formulas} in \( \mscrL \) is the language \hyperref[def:grammar_derivation/grammar_language]{generated} by this grammar schema with \( \bnfpn{atomic formula} \) as a starting rule.

    \thmitem{def:first_order_syntax/formula} The set \( \boldop{Form} \) of \term{formulas} in \( \mscrL \) is the language \hyperref[def:grammar_derivation/grammar_language]{generated} by this grammar schema with \( \bnfpn{formula} \) as a starting rule.

    The \term{atomic formulas} are the ones generated from \( \bnfpn{atomic formula} \).

    The grammar of first-order formulas is unambiguous by \fullref{thm:first_order_terms_and_formulas_are_unambiguous}.

    See \fullref{ex:first_order_substitution} for examples of first-order formulas.

    \thmitem{def:first_order_syntax/subformula} If \( \varphi \) and \( \psi \) are formulas and \( \psi \) is a \hyperref[def:language/subword]{subword} of \( \varphi \), we say that \( \psi \) is a \term{subformula} of \( \varphi \).

    \thmitem{def:first_order_syntax/formula_terms} If \( \varphi \) is a formula, if \( \tau \) is a term and if \( \tau \) is a \hyperref[def:language/subword]{subword} of \( \varphi \), we say that \( \tau \) is a \term{term} of \( \varphi \).

    \thmitem{def:first_order_syntax/formula_free_variables} The \term{free variables} of a formula are defined as
    \begin{equation}\label{eq:def:first_order_syntax/formula_free_variables}
      \boldop{Free}(\varphi) \coloneqq \begin{cases}
        \varnothing,                                                &\varphi \in \set{ \top, \bot } \\
        \boldop{Var}(\tau_1) \cup \ldots \cup \boldop{Var}(\tau_n), &\varphi = p(\tau_1, \ldots, \tau_n) \\
        \boldop{Var}(\tau_1) \cup \boldop{Var}(\tau_2),             &\varphi = \tau_1 \doteq \tau_2, \\
        \boldop{Free}(\psi),                                        &\varphi = \neg \psi, \\
        \boldop{Free}(\psi_1) \cup \boldop{Free}(\psi_2),           &\varphi = \psi_1 \bincirc \psi_2, \bincirc \in \Sigma, \\
        \boldop{Free}(\psi) \setminus \set{ \xi },                  &\varphi = \quantifier{Q}{\xi} \psi, Q \in \set{ \forall, \exists }
      \end{cases}
    \end{equation}

    \thmitem{def:first_order_syntax/ground_formula} A formula \( \varphi \) is called a \term{ground formula} if \( \boldop{Free}(\varphi) = \varnothing \). Ground formulas are also called \term{closed formulas} or \term{sentences} (unlike in propositional logic where all formulas are called sentences --- see \fullref{def:propositional_syntax/ground_formula}).

    Some authors restrict their interest to only closed formulas but this does not benefit us in any way due to \fullref{thm:quantifier_satisfiability}.

    \thmitem{def:first_order_syntax/formula_bound_variables} Dually, the \term{bound variables} of a formula are defined as
    \begin{equation}\label{eq:def:first_order_syntax/formula_bound_variables}
      \boldop{Bound}(\varphi) \coloneqq \begin{cases}
        \varnothing,                                        &\varphi \text{ is atomic}, \\
        \boldop{Bound}(\psi),                               &\varphi = \neg \psi, \\
        \boldop{Bound}(\psi_1) \cup \boldop{Bound}(\psi_2), &\varphi = \psi_1 \bincirc \psi_2, \bincirc \in \Sigma, \\
        \boldop{Bound}(\psi) \cup \set{ \xi },              &\varphi = \quantifier{Q}{\xi} \psi, Q \in \set{ \forall, \exists }.
      \end{cases}
    \end{equation}

    \thmitem{def:first_order_syntax/formula_variables} Finally, the set of all variables of a formula \( \varphi \) is
    \begin{equation}\label{eq:def:first_order_syntax/formula_variables}
      \boldop{Var}(\varphi) \coloneqq \boldop{Free}(\varphi) \cup \boldop{Bound}(\varphi).
    \end{equation}
  \end{thmenum}
\end{definition}

\begin{proposition}\label{thm:first_order_terms_and_formulas_are_unambiguous}
  The grammars of \hyperref[def:first_order_syntax/term]{first-order terms} and of \hyperref[def:first_order_syntax/formula]{first-order formulas} are \hyperref[def:grammar_derivation/ambiguity]{unambiguous}.
\end{proposition}
\begin{proof}
  The proof is more complicated but similar in spirit to \fullref{thm:propositional_formulas_are_unambiguous}.
\end{proof}

\begin{remark}\label{rem:first_order_formula_conventions}
  In order to simplify exposition, we use the following conventions
  \begin{thmenum}
    \thmitem{rem:first_order_formula_conventions/infix} Binary functional symbols are often written using \term{infix notation}, i.e.
    \begin{equation*}
      \zeta \doteq \xi + \eta
    \end{equation*}
    rather than the \term{prefix notation}
    \begin{equation*}
      \zeta \doteq +(\xi, \eta).
    \end{equation*}

    This also applies to predicates --- we write
    \begin{equation*}
      \xi < \eta
    \end{equation*}
    rather than
    \begin{equation*}
      <(\xi, \eta).
    \end{equation*}

    \thmitem{rem:first_order_formula_conventions/quantifier_predicates} If \( \sim \) is a binary predicate, to further shorten notation, we write
    \begin{equation*}
      \qforall {\xi \sim \eta} \varphi
    \end{equation*}
    as a shorthand for
    \begin{equation*}
      \qforall \xi (\xi \sim \eta \rightarrow \varphi).
    \end{equation*}
  \end{thmenum}

  As for \fullref{rem:propositional_formula_parentheses}, both of these conventions exist only in the metalanguage and the formulas themselves are assumed to have the former form within the object language.
\end{remark}

\begin{definition}\label{def:first_order_structure}\mcite[def. 16.1]{OpenLogicFull}
  Fix a first-order logic language \( \mscrL \). A \term{structure} for \( \mscrL \) is a pair \( (\mscrX, I) \), where
  \begin{thmenum}
    \thmitem{def:first_order_structure/set} \( \mscrX \) is a nonempty set called the \term{domain} or \term{universe} of the structure \( (\mscrX, I) \). See \fullref{rem:empty_models}.

    \thmitem{def:first_order_structure/interpretation} The \term{interpretation} \( I \) of the structure \( (\mscrX, I) \) is a \hyperref[def:function/single_valued]{function} that is defined on the signature of \( \mscrL \) and satisfies the following conditions:
    \begin{thmenum}
      \thmitem{def:first_order_structure/interpretation/function} For every \( n \)-ary function symbol \( f \), its interpretation is a \hyperref[def:function]{function} of type \( I(f): \mscrX^n \to \mscrX \).

      \thmitem{def:first_order_structure/interpretation/predicate} For every \( n \)-ary predicate \( p \), its interpretation is a an n-ary \hyperref[def:relation]{relation} \( I(p) \subseteq \mscrX^n \). This relation corresponds to all tuples of values that satisfy the predicate within the structure.
    \end{thmenum}
  \end{thmenum}
\end{definition}

\begin{remark}\label{rem:first_order_model_notation}
  In first-order logic, \hyperref[def:first_order_semantics/satisfiability]{models} are defined as tuples \( (\mscrX, I) \). Outside of logic, models are usually denoted more handwavily since either a part of the interpretation or the whole interpretation is clear from the context. We say that the set \( \mscrX \) is \term{endowed} with the structure \( I \) (see \hyperref[def:category_of_first_order_models]{model categories}).

  For example, in the sense of first-order structures, the language of the \hyperref[def:group/theory]{theory of groups} has a signature consisting of three functional symbols and no predicate symbols. A structure for this language is thus the same as a quadruple \( (\mscrG, e, (\placeholder)^{-1}, \cdot) \). Since group operations are named using a convention, we go as far as to say that \( \mscrG \) itself is a group and then simply use subscripts like \( e_{\mscrG} \) to avoid ambiguity.

  A popular convention is to use compatible letters like \( \mscrX = (X, I) \) to denote a structure. This works great in very simple cases, however in the case of ordered Banach algebras, this becomes \( \mscrX = (X, \BbbK, +, \cdot, \odot, \norm \placeholder, \leq) \). It is more conventional in this case to say \enquote{Let \( (\mscrX, \norm \placeholder) \) be an ordered Banach algebra}, assuming that \( \mscrX \) itself is a vector space and that \( \leq \) is implicitly defined. For this reason, we choose to denote the domains of structures using calligraphic letters and specify whatever other structure is necessary for our understanding.
\end{remark}

\begin{definition}\label{def:first_order_valuation}
  Fix a structure \( (\mscrX, I) \) for a first-order logic language \( \mscrL \).

  \begin{thmenum}
    \thmitem{def:first_order_valuation/variable_assignment}\mcite[def. 16.7]{OpenLogicFull} A \term{variable assignment} for the variables of \( \mscrL \) is any function \( v: \boldop{Var} \to \mscrX \) (loosely similar to \hyperref[def:propositional_valuation/interpretation]{propositional interpretations}).

    \thmitem{def:first_order_valuation/modified_assignment} For every variable \( \xi \) and every domain element \( x \in \mscrX \) we also define the \term{modified assignment} at \( \xi \) with \( x \):
    \begin{equation*}
      v_{\xi \mapsto x}(\zeta) \coloneqq \begin{cases}
        x,        &\zeta = \xi, \\
        v(\zeta), &\zeta \neq \xi.
      \end{cases}
    \end{equation*}

    We can also modify the value at \( \xi \) with another variable, e.g.
    \begin{equation*}
      v_{\xi \mapsto \eta}(\zeta) \coloneqq \begin{cases}
        v(\eta),  &\zeta = \xi, \\
        v(\zeta), &\zeta \neq \xi.
      \end{cases}
    \end{equation*}

    Inductively\IND,
    \begin{equation*}
      v_{\xi_1 \mapsto x_1, \ldots, \xi_n \mapsto x_n}(\eta) \coloneqq ((\ldots(v_{\xi_1 \mapsto x_1})\ldots)_{\xi_n \mapsto x_n})(\eta).
    \end{equation*}

    Except for semantics of quantification, these are also used in other places like \fullref{thm:renaming_assignment_compatibility} and \fullref{rem:first_order_formula_valuation_without_variable_assignment}.

    \thmitem{def:first_order_valuation/term_valuation}\mcite[def. 16.8]{OpenLogicFull} The \term{valuation} of a term \( \tau \) is a value in the domain \( \mscrX \) given by
    \begin{equation}\label{eq:def:first_order_valuation/term_valuation}
      \tau\Bracks{v} \coloneqq \begin{cases}
        v(\xi),                                           &\tau = \xi \in \boldop{Var}, \\
        I(f)(\tau_1\Bracks{v}, \ldots, \tau_n\Bracks{v}), &\tau = f(\tau_1, \ldots, \tau_n).
      \end{cases}
    \end{equation}

    \thmitem{def:first_order_valuation/formula_valuation}\mcite[def. 16.11]{OpenLogicFull} We extend the classical propositional valuations from \fullref{def:propositional_valuation}. The (classical) \term{valuation} of a formula \( \varphi \) is a \hyperref[def:boolean_value]{Boolean value} given by
    \begin{equation}\label{eq:def:first_order_valuation/formula_valuation}
      \varphi\Bracks{v} \coloneqq \begin{cases}
        T,                                                                   &\varphi = \top, \\
        F,                                                                   &\varphi = \bot, \\
        \tau_1\Bracks{v} = \tau_2\Bracks{v},                                 &\varphi = \tau_1 \doteq \tau_2, \\
        (\tau_1\Bracks{v}, \ldots, \tau_n\Bracks{v}) \in I(p),               &\varphi = p(\tau_1, \ldots, \tau_n), \\
        \overline{\psi\Bracks{v}},                                           &\varphi = \neg \psi, \\
        \psi_1\Bracks{v} \bincirc \psi_2\Bracks{v},                          &\varphi = \psi_1 \bincirc \psi_2, \bincirc \in \Sigma, \\
        \bigvee\set{ \psi\Bracks{v_{\xi \mapsto x}} \given x \in \mscrX },   &\varphi = \qforall \xi \psi, \\
        \bigwedge\set{ \psi\Bracks{v_{\xi \mapsto x}} \given x \in \mscrX }, &\varphi = \qexists \xi \psi.
      \end{cases}
    \end{equation}

    The rules for evaluating constants, negations and connectives are a direct extension of the \hyperref[def:propositional_valuation/formula_valuation]{rules for propositional logic}.

    It is important that the domain is nonempty because \( \bigwedge\varnothing = T \), which directly contradicts our intent of defining \( \exists \) as a quantifier for existence.
  \end{thmenum}
\end{definition}

\begin{remark}\label{rem:first_order_formula_valuation_without_variable_assignment}
  Somewhat similar to \fullref{rem:propositional_formula_valuation_without_variable_assignment}, if we know that \( \boldop{Free}(\varphi) \subseteq \{ \xi_1, \ldots, \xi_n \} \), we know that the \hyperref[def:first_order_valuation/formula_valuation]{valuation} \( \varphi\Bracks{v} \) only depends on the values \( v(\xi_1), \ldots, v(\xi_n) \). This allows us to introduce the shorthand
  \begin{equation}\label{eq:rem:first_order_formula_valuation_without_variable_assignment/long}
    \varphi\Bracks{\xi_1 \mapsto x_1, \ldots, \xi_n \mapsto x_n}
  \end{equation}
  or even
  \begin{equation}\label{eq:rem:first_order_formula_valuation_without_variable_assignment/short}
    \varphi\Bracks{x_1, \ldots, x_n}
  \end{equation}
  for
  \begin{equation*}
    \varphi\Bracks{v_{\xi_1 \mapsto x_1, \ldots, \xi_n \mapsto x_n}}
  \end{equation*}
  because the variable assignment \( v \) plays no role here.

  When using either of these shorthand notations, we implicitly assume that \( \boldop{Free}(\varphi) \subseteq \set{ \xi_1, \ldots, \xi_n } \).

  The shorter notation \eqref{eq:rem:first_order_formula_valuation_without_variable_assignment/short} is useful when \( \varphi \) is a predicate because this translates to
  \begin{equation*}
    p\Bracks{x_1, \ldots, x_n} = T \T{if and only if} (x_1, \ldots, x_n) \in I(p).
  \end{equation*}

  Of course, we avoid this notation for formulas like \( p(f(\xi)) \) because \( p\Bracks{x} \) would mean \( I(p)(I(f)(x)) \) rather than \( I(p)(x) \), which would be confusing.

  We also apply this notation for terms and, in particular, functions.
\end{remark}

\begin{definition}\label{def:first_order_definability}
  Fix a \hyperref[def:first_order_syntax]{first-order language} \( \mscrL \) and a \hyperref[def:first_order_structure]{structure} \( (\mscrX, I) \) on \( \mscrL \).

  We say that the set \( A \subseteq \mscrX^n \) is \term{definable} using the \hyperref[def:first_order_syntax]{formula} \( \varphi \) if, assuming \( {\boldop{Free}(\varphi) \subseteq \set{ \xi_1, \ldots, \xi_n }} \), we have
  \begin{equation*}
    \varphi\Bracks{\xi_1 \mapsto x_1, \ldots, \xi_n \mapsto x_n} = T \T{if and only if} (x_1, \ldots, x_n) \in A.
  \end{equation*}
\end{definition}

\begin{definition}\label{def:first_order_equation}
  A \term{first-order equation} is a formula of the form
  \begin{equation}\label{eq:def:first_order_equation}
    f(\xi_1, \ldots, \xi_n) \doteq g(\xi_1, \ldots, \xi_n),
  \end{equation}
  where both \( f(\xi_1, \ldots, \xi_n) \) and \( g(\xi_1, \ldots, \xi_n) \) are functional symbols with the same free variables.

  Given a structure \( (\mscrX, I) \), we call the elements of the set defined by this formula \term{solutions}. That is, we say that the tuple \( (x_1, \ldots, x_n) \) is a solution to the equation \eqref{eq:def:first_order_equation} if
  \begin{equation*}
    f\Bracks{x_1, \ldots, x_n} = g\Bracks{x_1, \ldots, x_n}.
  \end{equation*}
\end{definition}

\begin{example}\label{ex:equations}
  A remarkable portion of mathematics concerns the study of different types of equations (even though they are not generally restricted to \hyperref[def:first_order_equation]{equations in first-order logic}). The reason for this is that equations provide a simple way to specify rich semantic structure using simple syntactic objects.

  \begin{itemize}
    \item Matrix theory can be regarded as the study of linear equations. See \fullref{subsec:matrices}.
    \item Differential equations is aptly named since it studies equations in functional spaces concerning functions and their derivatives. See \fullref{sec:diffeq}.
    \item Roots of generalized derivatives are studied in optimization. See \fullref{sec:nonsmooth_analysis}.
    \item Diophantine equations are studied in number theory. See \fullref{subsec:integers}.
    \item Fixed points of functions are studied in different branches of mathematics. See \fullref{thm:banach_fixed_point_theorem} or \fullref{thm:knaster_tarski_theorem}.
    \item Affine varieties, which are sets of roots of polynomials, are studied in algebraic geometry. See \fullref{subsec:affine_varieties}.
  \end{itemize}
\end{example}

\begin{definition}\label{def:first_order_semantics}
  Fix a first-order logic language \( \mscrL \). We introduce notions analogous to \hyperref[def:propositional_semantics]{propositional semantics}:
  \begin{thmenum}
    \thmitem{def:first_order_semantics/satisfiability}\mcite[def. 16.11]{OpenLogicFull} Given a \hyperref[def:first_order_structure]{structure} \( (\mscrX, I) \), a \hyperref[def:first_order_valuation/variable_assignment]{variable assignment} \( v \) and a set \( \Gamma \) of \hyperref[def:first_order_syntax/formula]{first-order formulas}, we say that the variable assignment \( v \) \term{satisfies} \( \Gamma \) and we write \( (\mscrX, I) \vDash_v \Gamma \) if, for every formula \( \gamma \in \Gamma \) we have \( \gamma\Bracks{v} = T \).

    If every variable assignment in \( (\mscrX, I) \) satisfies \( \Gamma \), we say that \( \mscrX \) itself satisfies \( \Gamma \) or that \( \mscrX \) is a \term{model} of \( \Gamma \) and write \( (\mscrX, I) \vDash \Gamma \) (or simply \( \mscrX \vDash \Gamma \) if the interpretation is clear from the context).

    Analogously to \fullref{def:propositional_semantics/satisfiability}, we say that \( \Gamma \) is satisfiable if there exists a model for \( \Gamma \).

    \thmitem{def:first_order_semantics/entailment} We say that the set of formulas \( \Gamma \) \term{entails} the set of formulas \( \Delta \) and write \( \Gamma \vDash \Delta \) if every model of \( \Gamma \) is also a model of \( \Delta \).

    \thmitem{def:first_order_semantics/tautology} The formula \( \varphi \) is a \term{tautology} if every structure is a model of \( \varphi \).

    \thmitem{def:first_order_semantics/contradiction} Dually, \( \varphi \) is a \term{contradiction} is no structure is a model of \( \varphi \).

    \thmitem{def:first_order_semantics/equivalence} As in the simplest case with \hyperref[def:propositional_semantics/equivalence]{propositional semantical equivalence}, we say that \( \Gamma \) and \( \Delta \) are \term{semantically equivalent} and write \( \Gamma \gleichstark \Delta \) if both \( \Gamma \vDash \Delta \) and \( \Delta \vDash \Gamma \).

    \thmitem{def:first_order_semantics/equisatisfiability} Again as in the simplest case with \hyperref[def:propositional_semantics/equisatisfiability]{propositional equisatisfiability}, we say that the sets of formulas \( \Gamma \) and \( \Delta \) are \term{equisatisfiable} when it holds that \( \Gamma \) is satisfiable if and only if \( \Delta \) is satisfiable.

    \Fullref{thm:quantifier_satisfiability/existential} provides an important example of equisatisfiable formulas that are not equivalent.
  \end{thmenum}
\end{definition}

\begin{remark}\label{rem:propositional_logic_as_first_order_logic}
  It is now clear that the \hyperref[subsec:propositional_logic]{propositional logic language} can be regarded as a degenerate first-order logic language with no at most countably many predicates, all of arity 0, and no functional symbols. Thus first-order logic is indeed an extension of propositional logic.
\end{remark}

\begin{remark}\label{rem:empty_models}
   If we allow for the domain \( \mscrX \) of a structure \( (\mscrX, I) \) to be empty, we would have to reformulate a lot of important theorems (e.g. see the proof of \fullref{thm:renaming_assignment_compatibility/formulas}), which would complicate compatibility between semantics and \hyperref[def:first_order_derivation_system]{derivation systems}.

   We sometimes allow empty models. For example, the \hyperref[def:set_zfc/A2]{empty set} is a \hyperref[def:topological_space]{topological space}) and the initial object in the \hyperref[def:category_of_topological_spaces]{category \( \cat{Top} \) of topological spaces}. We must be careful, however, not to rely too much on metalogical theorems unless we explicitly refer to nonempty topological spaces.

   The \hyperref[def:magma/theory]{theory of magmas} provides an example of a first-order theory that admits empty models.
\end{remark}

\begin{definition}\label{def:first_order_substitution}
  As in \hyperref[subsec:propositional_logic]{propositional logic}, we sometimes want to perform substitution, however we have different types of syntactic objects (terms and formulas) which have different substitution rules. The notion of free and bound variables further complicates us --- see for example the problems outlined in \fullref{rem:first_order_substitution_renaming_justification}. In particular, this means that an analogous to \fullref{thm:propositional_substitution_equivalence} theorem cannot longer justify substitution as it is done in \fullref{alg:conjunctive_normal_form_reduction} --- we can have weaker statements as in \fullref{thm:propositional_substitution_equivalence} that implicitly rely on variable renaming in order to hold. This implies that it is of no practical use to define substitution of a first-order subformula inside another formula as it is done in \fullref{def:propositional_substitution}. Instead, we concert ourselves with substituting variables --- propositional variables with first-order formulas and first-order variables with first-order terms. Furthermore, since this does not complicate us, we allow substituting arbitrary terms rather than only first-order variables.

  While substituting a propositional variable is the syntactic analog to applying \hyperref[def:boolean_function]{Boolean functions} to different variables or propositional constants, substituting a first-order variable can express applying \hyperref[def:function/single_valued]{arbitrary functions} to different first-order variables or arbitrary constants. For example, in a suitable language, we can apply \( \log(x) \) to the constant \( e \) by substituting \( x \) with \( e \) to obtain the ground term \( \log(e) \).

  As in \fullref{def:propositional_substitution}, we define different kinds of (single) \term{substitution} in more generality that in e.g. \cite[def. 15.25]{OpenLogicFull}. Where applicable, \term{simultaneous substitution} is defined via the same trick as in \fullref{def:propositional_substitution}.

  \begin{thmenum}
    \thmitem{def:first_order_substitution/propositional} Let \( \varphi \) be a \hyperref[def:propositional_syntax/formula]{propositional formula} with variables \( \boldop{Var}(\varphi) = \set{ P_1, \ldots, P_n } \). For brevity, denote \( V \coloneqq \boldop{Var}(\varphi) \). Let \( \Theta = \set{ \theta_1, \ldots, \theta_n } \) be a set of \hyperref[def:first_order_syntax/formula]{first-order formulas}.

    It does not make sense to replace a single propositional variable by a single formula. Furthermore, a first-order formula \( \theta_k \) cannot possibly contain any of the propositional variables \( P_1, \ldots, P_n \). This allows us to introduce a simplification of the simultaneous substitution based on \eqref{eq:def:propositional_substitution/single} as
    \begin{equation}\label{eq:def:first_order_substitution/propositional}
      \varphi[V \mapsto \Theta] \coloneqq \begin{cases}
        \varphi,                                                    &\varphi \in \set{ \top, \bot } \\
        \theta_k,                                                   &\varphi = \theta_k \T{for some} k = 1, \ldots, n \\
        \neg \psi[V \mapsto \Theta],                                &\varphi = \neg \psi \\
        \psi_1[V \mapsto \Theta] \bincirc \psi_2[V \mapsto \Theta], &\varphi = \psi_1 \bincirc \psi_2, \bincirc \in \Sigma.
      \end{cases}
    \end{equation}

    As in \fullref{def:propositional_substitution}, it is not strictly necessary for any of the variables to belong to \( \boldop{Var}(\varphi) \).

    \thmitem{def:first_order_substitution/term_in_term} We define the substitution of the \hyperref[def:first_order_syntax/term]{first-order term} \( \kappa \) with \( \mu \) in the term \( \tau \) as
    \begin{equation}\label{eq:def:first_order_substitution/term_in_term}
      \tau[\kappa \mapsto \mu] \coloneqq \begin{cases}
        \mu,                                                               &\tau = \kappa, \\
        \tau,                                                              &\tau \neq \kappa \T{and} \tau \in \boldop{Var}, \\
        f(\tau_1[\kappa \mapsto \mu], \ldots, \tau_n[\kappa \mapsto \mu]), &\tau \neq \kappa \T{and} \tau = f(\tau_1, \ldots, \tau_n).
      \end{cases}
    \end{equation}

    It is not strictly necessary for \( \kappa \) to be a \hyperref[def:first_order_syntax/subterm]{subterm} of \( \tau \).

    \thmitem{def:first_order_substitution/term_in_formula} This case is more complicated. We define the substitution of the term \( \kappa \) with the term \( \mu \) in the first-order formula \( \varphi \) as
    \begin{equation}\label{eq:def:first_order_substitution/term_in_formula}
      \varphi[\kappa \mapsto \mu] \coloneqq \begin{cases}
        \varphi,                                                           &\varphi \in \set{ \top, \bot }, \\
        p(\tau_1[\kappa \mapsto \mu], \ldots, \tau_n[\kappa \mapsto \mu]), &\varphi = p(\tau_1, \ldots, \tau_n), \\
        \tau_1[\kappa \mapsto \mu] \doteq \tau_2[\kappa \mapsto \mu],      &\varphi = \tau_1 \doteq \tau_2, \\
        \neg \psi[\kappa \mapsto \mu],                                     &\varphi = \neg \psi, \\
        \psi_1[\kappa \mapsto \mu] \bincirc \psi_2[\xi \mapsto \mu],       &\varphi = \psi_1 \bincirc \psi_2, \bincirc \in \Sigma, \\
        (\dagger),                                                         &\varphi = \quantifier{Q}{\xi} \psi, Q \in \set{ \forall, \exists },
      \end{cases}
    \end{equation}
    where
    \begin{empheq}[left=(\dagger) \coloneqq \empheqlbrace]{align}
      &\varphi,                                                                        && \xi \in \boldop{Var}(\kappa), \label{eq:def:first_order_substitution/term_in_formula/quantifiers/trivial} \\
      &\quantifier{Q}{\xi} \parens[\Big]{\psi[\kappa \mapsto \mu]},                    && \xi \not\in \boldop{Var}(\kappa) \cup \boldop{Var}(\mu), \label{eq:def:first_order_substitution/term_in_formula/quantifiers/direct} \\
      &\quantifier{Q}{\eta} \parens[\Big]{\psi[\xi \mapsto \eta][\kappa \mapsto \mu]}, && \xi \not\in \boldop{Var}(\kappa) \T{and} \xi \in \boldop{Var}(\mu) \T{and} &\label{eq:def:first_order_substitution/term_in_formula/quantifiers/renaming} \\
                                                                                      &&& \eta \not\in \boldop{Var}(\kappa) \cup \boldop{Var}(\mu) \cup \boldop{Var}(\psi). \nonumber
    \end{empheq}

    In \eqref{eq:def:first_order_substitution/term_in_formula/quantifiers/renaming}, we chose a new variable \( \eta \). We implicitly assume that there exist enough variables in the language so that we can find \( \eta \) that satisfies the condition. In order to fully avoid nondeterminism in the choice of \( \eta \), we can pick\AOC a well-ordering on the set \( \boldop{Var} \) and always choose \( \eta \) to be the smallest variable not present in \( \varphi \). This rule is called \term{renaming of the bound variables} \( \xi \) to \( \eta \) and is done to mitigate capturing as described in \fullref{rem:first_order_substitution_renaming_justification/capturing}.

    We could avoid the rule for renaming (as it is done in \cite[def. 15.25]{OpenLogicFull}), however renaming both free and bound variables is natural and is often done in practice. For example, consider the \hyperref[def:peano_arithmetic]{Peano arithmetic} formula \enquote{there exists \( n \) such that \( nm \) is even}. Note that the bound variable \( n \) is renamed to \( k \) and the free variable \( m \) to \( n \) in the larger formula \enquote{for every \( n \) there exists \( k \) such that \( kn \) is even}.

    The rule \eqref{eq:def:first_order_substitution/term_in_formula/quantifiers/trivial} may seem redundant but when doing inductive proofs (e.g. the proof of \fullref{thm:renaming_assignment_compatibility}), we usually need to separately consider the cases where \( \xi \in \boldop{Var}(\kappa) \) and \( \xi \not\in \boldop{Var}(\kappa) \setminus \boldop{Var}(\mu) \) and the rule \eqref{eq:def:first_order_substitution/term_in_formula/quantifiers/direct} being trivial simplifies the proofs.

    See \fullref{rem:first_order_substitution_parentheses} regarding the additional parentheses in \eqref{eq:def:first_order_substitution/term_in_formula/quantifiers/renaming}.

    See \fullref{ex:first_order_substitution} for examples of applying the different quantifier rules.
  \end{thmenum}
\end{definition}

\begin{remark}\label{rem:first_order_substitution_renaming_justification}
  The renaming rule \eqref{eq:def:first_order_substitution/term_in_formula/quantifiers/renaming} is designed to mitigate the following two problems (compared to \eqref{eq:def:first_order_substitution/term_in_formula/quantifiers/direct}):

  \begin{thmenum}
    \thmitem{rem:first_order_substitution_renaming_justification/capturing} Renaming mitigates \enquote{capturing} free variables as in
    \begin{equation*}
      \parens[\Big]{ \qforall \eta p(\xi, \eta) }[\xi \mapsto \eta] = \qforall \eta p(\eta, \eta)
    \end{equation*}
    by instead producing, up to a choice of new variables, the formula
    \begin{equation*}
      \parens[\Big]{ \qforall \eta p(\xi, \eta) }[\xi \mapsto \eta] = \qforall \zeta p(\eta, \zeta).
    \end{equation*}

    \thmitem{rem:first_order_substitution_renaming_justification/colliding} Renaming mitigates \enquote{colliding} multiple bound variables as in
    \begin{equation*}
      \parens[\Big]{ \qforall \xi \qforall \eta p(\xi, \eta) }[\xi \mapsto \eta] = \qforall \xi \qforall \eta p(\eta, \eta)
    \end{equation*}
    by instead producing, up to a choice of new variables, the formula
    \begin{equation*}
      \parens[\Big]{ \qforall \xi \qforall \eta p(\xi, \eta) }[\xi \mapsto \eta] = \qforall \zeta \qforall \sigma p(\zeta, \sigma).
    \end{equation*}
  \end{thmenum}
\end{remark}

\begin{remark}\label{rem:first_order_substitution_parentheses}
  When performing \hyperref[def:propositional_substitution]{substitution}, it is sometimes convenient to add additional parentheses to avoid ambiguity. For example, while parentheses around quantifier expressions are not necessary by the syntax of first-order logic, adding such parentheses helps avoid the ambiguity in
  \begin{equation*}
    \qforall \xi p(\xi, \eta) [\eta \mapsto \zeta].
  \end{equation*}

  Instead, we either write
  \begin{equation*}
    \parens[\Big]{ \qforall \xi p(\xi, \eta) } [\eta \mapsto \zeta]
  \end{equation*}
  or
  \begin{equation*}
    \qforall \xi \parens[\Big]{ p(\xi, \eta)[\eta \mapsto \zeta] }.
  \end{equation*}

  This convention is only part of the metasyntax and the parentheses are not part of the syntax of the formulas themselves.
\end{remark}

\begin{example}\label{ex:first_order_substitution}
  The following term substitutions should justify the distinct cases in \eqref{eq:def:first_order_substitution/term_in_formula}:
  \begin{thmenum}
    \thmitem{ex:first_order_substitution/1} The trivial case without actual substitution:
    \begin{align*}
      &\phantom{{}={}}
      \parens[\Big]{\qforall \xi p(\xi, \eta)}[\xi \mapsto \eta]
      \overset {\eqref{eq:def:first_order_substitution/term_in_formula/quantifiers/trivial}} = \\ &=
      \qforall \xi p(\xi, \eta).
    \end{align*}

    \Fullref{ex:first_order_substitution/5} demonstrates that this does not work for nested substitution.

    \thmitem{ex:first_order_substitution/2} A simple substitution without renaming:
    \begin{align*}
      &\phantom{{}={}}
      \parens[\Big]{\qforall \xi p(\xi, \eta)}[\eta \mapsto \zeta]
      \overset {\eqref{eq:def:first_order_substitution/term_in_formula/quantifiers/direct}} = \\ &=
      \qforall \xi \parens[\Big]{p(\xi, \eta)[\eta \mapsto \zeta]}
      \overset {\eqref{eq:def:first_order_substitution/term_in_formula}} = \\ &=
      \qforall \xi p(\xi, \zeta).
    \end{align*}

    \thmitem{ex:first_order_substitution/3} A simple renaming without actual substitution:
    \begin{align*}
      &\phantom{{}={}}
      \parens[\Big]{\qforall \xi p(\xi, \eta)}[\eta \mapsto \xi]
      \overset {\eqref{eq:def:first_order_substitution/term_in_formula/quantifiers/renaming}} = \\ &=
      \qforall \zeta \parens[\Big]{p(\xi, \eta)[\xi \mapsto \zeta][\eta \mapsto \xi]}
      \overset {\eqref{eq:def:first_order_substitution/term_in_formula}} = \\ &=
      \qforall \zeta p(\zeta, \xi).
    \end{align*}

    \thmitem{ex:first_order_substitution/4} \Fullref{ex:first_order_substitution/3} but with \( \mu \) in \eqref{eq:def:first_order_substitution/term_in_formula} containing \( \xi \) indirectly:
    \begin{align*}
      &\phantom{{}={}}
      \parens[\Big]{\qforall \xi p(\xi, \eta)}[\eta \mapsto f(\xi)]
      \overset {\eqref{eq:def:first_order_substitution/term_in_formula/quantifiers/renaming}} = \\ &=
      \qforall \zeta \parens[\Big]{p(\xi, \eta)[\xi \mapsto \zeta][\eta \mapsto f(\xi)]}
      \overset {\eqref{eq:def:first_order_substitution/term_in_formula}} = \\ &=
      \qforall \zeta p(\zeta, f(\xi)).
    \end{align*}

    \thmitem{ex:first_order_substitution/5} Only renaming with multiple quantifiers which shows the limitations of \eqref{eq:def:first_order_substitution/term_in_formula/quantifiers/trivial}:
    \begin{align*}
      &\phantom{{}={}}
      \parens[\Big]{\qforall \eta \qexists \xi p(\xi, \eta)}[\xi \mapsto \eta]
      \overset {\eqref{eq:def:first_order_substitution/term_in_formula/quantifiers/renaming}} = \\ &=
      \qforall \zeta \parens*{ \parens[\Big]{ \qforall \xi p(\xi, \eta) }[\eta \mapsto \zeta][\xi \mapsto \eta]}
      \overset {\eqref{eq:def:first_order_substitution/term_in_formula}} = \\ &=
      \qforall \zeta \parens*{ \parens[\Big]{ \qforall \xi p(\xi, \zeta) }[\xi \mapsto \eta]}
      \overset {\eqref{eq:def:first_order_substitution/term_in_formula/quantifiers/trivial}} = \\ &=
      \qforall \zeta \qforall \xi p(\xi, \zeta).
    \end{align*}

    \thmitem{ex:first_order_substitution/6} Both renaming and substitution with multiple quantifiers:
    \begin{align*}
      &\phantom{{}={}}
      \parens[\Big]{ \qforall \eta (p(\xi, \eta) \vee \qforall \xi p(\xi, \eta)) }[\xi \mapsto \eta]
      \overset {\eqref{eq:def:first_order_substitution/term_in_formula/quantifiers/renaming}} = \\ &=
      \qforall \zeta \parens*{ \parens[\Big]{ p(\xi, \eta) \vee \qexists \xi p(\xi, \eta) }[\eta \mapsto \zeta][\xi \mapsto \eta] }
      \overset {\eqref{eq:def:first_order_substitution/term_in_formula}} = \\ &=
      \qforall \zeta \parens*{ p(\eta, \zeta) \vee \parens[\Big]{ \qexists \xi p(\xi, \eta) }[\eta \mapsto \zeta][\xi \mapsto \eta] }
      \overset {\eqref{eq:def:first_order_substitution/term_in_formula/quantifiers/direct}} = \\ &=
      \qforall \zeta \parens*{ p(\eta, \zeta) \vee \parens[\Big]{ \qexists \xi p(\xi, \zeta) }[\xi \mapsto \eta] }
      \overset {\eqref{eq:def:first_order_substitution/term_in_formula/quantifiers/trivial}} = \\ &=
      \parens[\Big]{ \qforall \zeta p(\eta, \zeta) } \vee \parens[\Big]{ \qexists \xi p(\xi, \zeta) }.
    \end{align*}

    \thmitem{ex:first_order_substitution/7} Substitution of more general terms than variables with renaming of term's variables:
    \begin{align*}
      &\phantom{{}={}}
      \parens*{\qforall \xi p(\xi, \eta, f(\eta))}[f(\eta) \mapsto g(\eta, \xi)]
      \overset {\eqref{eq:def:first_order_substitution/term_in_formula/quantifiers/renaming}} = \\ &=
      \qforall \zeta \parens[\Big]{p(\xi, \eta, f(\eta))[\xi \mapsto \zeta][f(\eta) \mapsto g(\eta, \xi)]}
      \overset {\eqref{eq:def:first_order_substitution/term_in_formula}} = \\ &=
      \qforall \zeta p(\zeta, \eta, g(\eta, \xi)).
    \end{align*}
  \end{thmenum}
\end{example}

\begin{proposition}\label{thm:renaming_assignment_compatibility}
  We will show how \hyperref[rem:first_order_substitution_renaming_justification]{syntactic renaming} is compatible with a certain \enquote{semantic renaming}.

  Fix a \hyperref[def:first_order_syntax]{first-order language} \( \mscrL \), a \hyperref[def:first_order_structure]{structure} \( (\mscrX, I) \) on \( \mscrL \) and a \hyperref[def:first_order_valuation/variable_assignment]{variable assignment} \( v \) in \( (\mscrX, I) \).

  \begin{thmenum}
    \thmitem{thm:renaming_assignment_compatibility/terms} For any term \( \tau \) and any two variables \( \xi \) and \( \eta \), we have
    \begin{equation}\label{eq:thm:renaming_assignment_compatibility/terms}
      \tau\Bracks{v_{\xi \mapsto \eta}}
      =
      \parens[\Big]{ \tau[\xi \mapsto \eta] }\Bracks{v}.
    \end{equation}

    \thmitem{thm:renaming_assignment_compatibility/formulas} For any formula \( \varphi \), any variable \( \xi \) and any other variable \( \eta \) not in \( \boldop{Var}(\varphi) \) we have
    \begin{equation}\label{eq:thm:renaming_assignment_compatibility/formulas}
      \varphi\Bracks{v_{\xi \mapsto \eta}}
      =
      \parens[\Big]{ \varphi[\xi \mapsto \eta] }\Bracks{v}.
    \end{equation}
  \end{thmenum}
\end{proposition}
\begin{proof}
  In both cases, we use structural induction on the definition of the substitution\IND.

  \SubProofOf{thm:renaming_assignment_compatibility/terms}

  \begin{itemize}
    \item If \( \tau = \xi \), then
    \begin{equation*}
      \tau[\xi \mapsto \eta] = \xi[\xi \mapsto \eta] = \eta
    \end{equation*}
    and \eqref{eq:thm:renaming_assignment_compatibility/terms} follows directly.

    \item If \( \tau \) is a variable and \( \tau \neq \xi \), then
    \begin{equation*}
      \tau[\xi \mapsto \eta] = \tau
    \end{equation*}
    and \eqref{eq:thm:renaming_assignment_compatibility/terms} again holds trivially.

    \item If \( \tau = f(\tau_1, \ldots, \tau_n) \) and if the inductive hypothesis holds for \( \tau_1, \ldots, \tau_n \), then
    \begin{balign*}
      \parens[\Big]{ \tau[\xi \mapsto \eta] }\Bracks{v}
      &=
      \parens[\Big]{ f(\tau_1[\xi \mapsto \eta], \ldots, \tau_n[\xi \mapsto \eta]) }\Bracks{v}
      = \\ &=
      I(f) \parens[\Bigg]{ \parens[\Big]{ \tau_1[\xi \mapsto \eta] }\Bracks{v}, \ldots, \parens[\Big]{ \tau_1[\xi \mapsto \eta] }\Bracks{v} }
      \overset {\IndHyp} = \\ &=
      I(f) \parens[\Big]{ \tau_1\Bracks{v_{\xi \mapsto \eta}}, \ldots, \tau_n\Bracks{v_{\xi \mapsto \eta}} }
      = \\ &=
      \parens[\Big]{ f(\tau_1, \ldots, \tau_n) }\Bracks{v_{\xi \mapsto \eta}}
      = \\ &=
      \tau\Bracks{v_{\xi \mapsto \eta}}.
    \end{balign*}
  \end{itemize}

  In all cases, \eqref{eq:thm:renaming_assignment_compatibility/terms} holds.

  \SubProofOf{thm:renaming_assignment_compatibility/formulas}
  \hfill
  \begin{itemize}
    \item If \( \varphi \in \set{ \top, \bot } \), then \( \varphi \) has no subterms and thus \eqref{eq:thm:renaming_assignment_compatibility/formulas} holds vacuously.

    \item If \( \varphi = p(\tau_1, \ldots, \tau_n) \), then by \eqref{eq:thm:renaming_assignment_compatibility/terms} for all \( \tau_k \) we have
    \begin{equation*}
      \parens[\Big]{ \tau_k[\xi \mapsto \eta] }\Bracks{v} = \tau_k\Bracks{v_{\xi \mapsto \eta}}
    \end{equation*}
    and thus
    \begin{equation*}
      I(p)\parens*{ \parens[\Big]{ \tau_1[\xi \mapsto \eta] }\Bracks{v}, \ldots, \parens[\Big]{ \tau_1[\xi \mapsto \eta] }\Bracks{v} }
      \overset {\IndHyp} =
      I(p)\parens[\Big]{ \tau_1\Bracks{v_{\xi \mapsto \eta}}, \ldots, \tau_n\Bracks{v_{\xi \mapsto \eta}} }.
    \end{equation*}

    Therefore
    \begin{balign*}
      \parens[\Big]{ \varphi[\xi \mapsto \eta] }\Bracks{v}
      &=
      \parens[\Big]{ p(\tau_1[\xi \mapsto \eta], \ldots, \tau_n[\xi \mapsto \eta]) }\Bracks{v}
      = \\ &=
      \parens[\Big]{ p(\tau_1, \ldots, \tau_n) }\Bracks{v_{\xi \mapsto \eta}}
      = \\ &=
      \varphi\Bracks{v_{\xi \mapsto \eta}}.
    \end{balign*}

    \item The case \( \varphi = \tau_1 \doteq \tau_2 \) is proved analogously.

    \item The cases \( \varphi = \neg \psi \) and \( \varphi = \psi_1 \bincirc \psi_2 \) are proved in an straightforward manner.

    \item Let \( \varphi = \qforall \zeta \psi \), where the inductive hypothesis holds for \( \psi \). We consider three cases
    \begin{itemize}
      \item Suppose that \( \zeta = \xi \). By definition, we have
      \begin{equation*}
        \varphi[\xi \mapsto \eta]
        =
        \varphi,
      \end{equation*}
      hence \eqref{eq:thm:renaming_assignment_compatibility/formulas} holds trivially.

      \item Suppose that \( \zeta \neq \xi \). It follows that
      \begin{equation*}
        \varphi[\xi \mapsto \eta]
        =
        \qforall \zeta \parens[\Big]{ \psi[\xi \mapsto \eta] }.
      \end{equation*}

      \begin{itemize}
        \item If \( \parens[\Big]{\varphi[\xi \mapsto \eta]}\Bracks{v} = T \), by definition of \hyperref[def:first_order_valuation/formula_valuation]{quantifier formula valuation}, for any \( x \in \mscrX \) we have
        \begin{equation}\label{eq:thm:renaming_assignment_compatibility/formulas/true_modified_assignment}
          \parens[\Bigg]{\underbrace{\qforall \zeta \parens[\Big]{ \psi[\xi \mapsto \eta] }}_{\varphi[\xi \mapsto \eta]} }\Bracks{v}
          =
          \parens[\Big]{\psi[\xi \mapsto \eta]}\Bracks{v_{\zeta \mapsto x}}
          =
          T.
        \end{equation}

        On the other hand, by the inductive hypothesis,
        \begin{equation*}
          \parens[\Big]{ \psi[\xi \mapsto \eta] }\Bracks{v} = \psi\Bracks{v_{\xi \mapsto \eta}}
        \end{equation*}
        and, as a special case, for any \( x \in \mscrX \),
        \begin{equation}\label{eq:thm:renaming_assignment_compatibility/formulas/ind_hyp_modified_assignment}
          \parens[\Big]{ \psi[\xi \mapsto \eta] }\Bracks{v_{\zeta \mapsto x}} = \psi\Bracks{v_{\xi \mapsto \eta, \zeta \mapsto x}}.
        \end{equation}

        Combining \eqref{eq:thm:renaming_assignment_compatibility/formulas/true_modified_assignment} and \eqref{eq:thm:renaming_assignment_compatibility/formulas/ind_hyp_modified_assignment}, we obtain
        \begin{equation*}
          \parens[\Big]{ \varphi[\xi \mapsto \eta] }\Bracks{v}
          =
          \parens[\Big]{ \psi[\xi \mapsto \eta] }\Bracks{v_{\zeta \mapsto x}}
          =
          \underbrace{\psi\Bracks{v_{\xi \mapsto \eta, \zeta \mapsto x}}}_{T \T{for all} x \in \mscrX}
          =
          \varphi\Bracks{v_{\xi \mapsto \eta}},
        \end{equation*}
        which proves the case.

        \item If \( \parens[\Big]{\varphi[\xi \mapsto \eta]}\Bracks{v} = F \), then there exists \( x \in \mscrX \) such that
        \begin{equation*}
          \parens[\Big]{\psi[\xi \mapsto \eta]}\Bracks{v_{\zeta \mapsto x}} = F.
        \end{equation*}

        Since \eqref{eq:thm:renaming_assignment_compatibility/formulas/ind_hyp_modified_assignment} holds by the inductive hypothesis, we have
        \begin{equation*}
          \psi\Bracks{v_{\xi \mapsto \eta, \zeta \mapsto x}} = F
        \end{equation*}
        for the same \( x \).

        It follows that \( \varphi\Bracks{v_{\xi \mapsto \eta}} = F \), which proves the case.
      \end{itemize}
    \end{itemize}

    \item We can prove the case \( \varphi = \qexists \zeta \psi \) using double negation on the previous case.
  \end{itemize}

  In all cases, \eqref{eq:thm:renaming_assignment_compatibility/formulas} holds.
\end{proof}

\begin{proposition}\label{thm:first_order_substitution_equivalence}
  Analogously to \fullref{thm:propositional_substitution_equivalence}, we will show that all substitutions defined in \fullref{def:first_order_substitution} types of substitution preserve the corresponding \hyperref[def:first_order_semantics]{semantics}.

  By induction\IND, this proposition also holds for \hyperref[def:propositional_substitution/simultaneous]{simultaneous substitution}.

  Fix a \hyperref[def:first_order_structure]{structure} \( (\mscrX, I) \) and a \hyperref[def:first_order_valuation/variable_assignment]{variable assignment} \( v \).

  \begin{thmenum}
    \thmitem{thm:first_order_substitution_equivalence/propositional} As in \fullref{def:first_order_substitution/propositional}, let \( \varphi \) be a \hyperref[def:propositional_syntax/formula]{propositional formula} with variables \( {V = \set{ P_1, \ldots, P_n }} \) and let \( \Theta = \set{ \theta_1, \ldots, \theta_n } \) be a set of \hyperref[def:first_order_syntax/formula]{first-order formulas}.

    Furthermore, let \( J \) be a \hyperref[def:propositional_valuation/interpretation]{propositional interpretation} such that, for all \( k = 1, \ldots, n \),
    \begin{equation}\label{eq:thm:first_order_substitution_equivalence/propositional/compatibility}
      P_k\Bracks{J} = \theta_k\Bracks{v}.
    \end{equation}

    Then
    \begin{equation}\label{eq:thm:first_order_substitution_equivalence/propositional}
      \parens[\Big]{ \varphi[V \mapsto \Theta] }\Bracks{v} = \varphi \Bracks{J}.
    \end{equation}

    In particular, \( \vDash \varphi \) (in the sense of \fullref{def:propositional_semantics/tautology}) implies \( \vDash \varphi[V \mapsto \Theta] \) (in the sense of \fullref{def:first_order_semantics/tautology}).

    \thmitem{thm:first_order_substitution_equivalence/term_in_term} Let \( \tau \) be a \hyperref[def:first_order_syntax/term]{first-order term} and let \( \kappa \) be a \hyperref[def:first_order_syntax/subterm]{subterm} of \( \tau \). Let \( \mu \) be another term such that
    \begin{equation}\label{eq:thm:first_order_substitution_equivalence/term_in_term/compatibility}
      \mu\Bracks{v} = \kappa\Bracks{v}.
    \end{equation}

    Then
    \begin{equation}\label{eq:thm:first_order_substitution_equivalence/term_in_term}
      \tau[\kappa \mapsto \mu]\Bracks{v} = \tau\Bracks{v}.
    \end{equation}

    \thmitem{thm:first_order_substitution_equivalence/term_in_formula} Let \( \varphi \) be a \hyperref[def:first_order_syntax/formula]{first-order formula} and let \( \kappa \) be a \hyperref[def:first_order_syntax/formula_terms]{term of \( \varphi \)}. Let \( \mu \) be another term such that
    \begin{equation}\label{eq:thm:first_order_substitution_equivalence/term_in_formula/compatibility}
      \mu\Bracks{v} = \kappa\Bracks{v}.
    \end{equation}

    Then
    \begin{equation}\label{eq:thm:first_order_substitution_equivalence/term_in_formula}
      \varphi[\kappa \mapsto \mu]\Bracks{v} = \varphi\Bracks{v}.
    \end{equation}
  \end{thmenum}
\end{proposition}
\begin{proof}
  In all cases, we use structural induction by the definition of the substitution\IND. The inductive hypothesis for a formula is that the proposition holds for arbitrary substitutions and valuations.

  \SubProofOf{thm:first_order_substitution_equivalence/propositional} Let \( \varphi \) be a propositional formula.
  \begin{itemize}
    \item If \( \varphi \in \set{ \top, \bot } \), no substitution is performed and thus \eqref{eq:thm:first_order_substitution_equivalence/propositional} holds trivially.

    \item If \( \varphi = P_k \) for some \( k = 1, \ldots, n \), then follows \eqref{eq:thm:first_order_substitution_equivalence/propositional} from \eqref{eq:thm:first_order_substitution_equivalence/propositional/compatibility}.

    \item If \( \varphi = \neg \psi \) and if the inductive hypothesis holds for \( \psi \), then
    \begin{equation*}
      \parens[\Big]{ \psi[V \mapsto \Theta] }\Bracks{v}
      =
      \overline{\parens[\Big]{ \psi[V \mapsto \Theta] }\Bracks{v}}
      \overset {\IndHyp} =
      \overline{\psi \Bracks{J}}
      =
      \varphi \Bracks{J}.
    \end{equation*}

    \item If \( \varphi = \psi_1 \bincirc \psi_2, \bincirc \in \Sigma \) and if the inductive hypothesis holds for both \( \psi_1 \) and \( \psi_2 \), then
    \begin{equation*}
      \parens[\Big]{ \psi[V \mapsto \Theta] }\Bracks{v}
      =
      \parens[\Big]{ \psi_1[\rho \mapsto \theta] }\Bracks{v} \bincirc \parens[\Big]{ \psi_2[\rho \mapsto \theta] }\Bracks{v}
      \overset {\IndHyp} =
      \psi_1 \Bracks{J} \bincirc \psi_2\Bracks{J}
      =
      \varphi\Bracks{J}.
    \end{equation*}
  \end{itemize}

  In all cases, \eqref{eq:thm:first_order_substitution_equivalence/propositional} holds.

  \SubProofOf{thm:first_order_substitution_equivalence/term_in_term} The proof is identical to that of \fullref{thm:renaming_assignment_compatibility/terms}.

  \SubProofOf{thm:first_order_substitution_equivalence/term_in_formula} The proof is identical to that of \fullref{thm:renaming_assignment_compatibility/formulas} except for the special cases where \hyperref[rem:first_order_substitution_renaming_justification]{renaming} occurs, i.e. \( \varphi = \qforall \xi \psi \) and \( \varphi = \qexists \xi \psi \), where
  \begin{itemize}
    \item \( \xi \in \boldop{Free}(\mu) \).
    \item \( \eta \not\in \boldop{Var}(\kappa) \cup \boldop{Var}(\mu) \cup \boldop{Var}(\psi) \).
    \item The inductive hypothesis holds for \( \psi \).
  \end{itemize}

  We will only show the case \( \varphi = \qforall \xi \psi \) since the existential case is handled similarly.

  Since \( \xi \in \boldop{Free}(\mu) \), we have
  \begin{equation*}
    \varphi[\kappa \mapsto \mu]
    =
    \qforall \eta \parens[\Big]{ \psi[\xi \mapsto \eta][\kappa \mapsto \mu] },
  \end{equation*}
  which does not allow us to use the inductive hypothesis directly.

  We proceed to prove the statement by nested induction\IND on the number of quantifiers. We have already shown the case of \( 0 \) quantifiers. Suppose that the statement holds for all formulas with strictly less than \( n \) quantifiers and suppose that \( \varphi \) has exactly \( n \) quantifiers.

  Furthermore, for formulas with \( n \) quantifiers with \( \forall \) as the outermost one, the non-renaming cases \eqref{eq:def:first_order_substitution/term_in_formula/quantifiers/trivial} and \eqref{eq:def:first_order_substitution/term_in_formula/quantifiers/direct} hold. Therefore, since \( \eta \not\in \boldop{Free}(\mu) \),
  \begin{equation}\label{eq:thm:first_order_substitution_equivalence/term_in_formula/nested_induction}
    \begin{aligned}
      &\phantom{{}={}}
      \varphi[\kappa \mapsto \mu]\Bracks{v}
      = \\ &=
      \parens[\Bigg]{\qforall \eta \parens[\Big]{ \psi[\xi \mapsto \eta][\kappa \mapsto \mu] }}\Bracks{v}
      \overset {\eqref{eq:def:first_order_substitution/term_in_formula/quantifiers/direct}} = \\ &=
      \parens[\Bigg]{ \parens*{ \qforall \eta \parens[\Big]{ \psi[\xi \mapsto \eta] } }[\kappa \mapsto \mu] }\Bracks{v}
      \overset {\IndHyp} = \\ &=
      \parens[\Bigg]{\qforall \eta \parens[\Big]{ \psi[\xi \mapsto \eta] } }\Bracks{v},
    \end{aligned}
  \end{equation}
  where we have implicitly used that \( \psi \) has \( n - 1 \) quantifiers.

  On the other hand, by \fullref{thm:renaming_assignment_compatibility/formulas},
  \begin{equation*}
    \parens[\Big]{ \psi[\xi \mapsto \eta] }\Bracks{v} = \psi\Bracks{v_{\xi \mapsto \eta}}
  \end{equation*}
  and, in particular, for any \( x \in \mscrX \),
  \begin{equation*}
    \parens[\Big]{ \psi[\xi \mapsto \eta] }\Bracks{v_{\eta \mapsto x}}
    =
    \psi\Bracks{v_{\xi \mapsto \eta,\eta \mapsto x}}
    \ClapOverset {\eta \not\in \boldop{Var}(\psi)} =
    \psi\Bracks{v_{\xi \mapsto x}}
  \end{equation*}

  Hence
  \begin{equation*}
    \underbrace{ \parens[\Bigg]{\qforall \eta \parens[\Big]{ \psi[\xi \mapsto \eta] } }\Bracks{v} }_{\varphi[\xi \mapsto \eta]\Bracks{v}}
    =
    \underbrace{ \parens[\Big]{\qforall \xi \psi }\Bracks{v_{\xi \mapsto \eta}} }_{\varphi\Bracks{v_{\xi \mapsto \eta}}}.
  \end{equation*}

  This proves \eqref{eq:thm:first_order_substitution_equivalence/term_in_formula}.
\end{proof}

\begin{proposition}\label{thm:first_order_quantifiers_are_dual}
  For any formula \( \varphi \) and any variable \( \xi \) over \( \mscrL \), we have the following equivalence:
  \begin{align}
    \neg \qforall \xi \varphi &\gleichstark \qexists \xi \neg \varphi \label{thm:first_order_quantifiers_are_dual/negation_of_universal} \\
    \neg \qexists \xi \varphi &\gleichstark \qforall \xi \neg \varphi \label{thm:first_order_quantifiers_are_dual/negation_of_existential}
  \end{align}
\end{proposition}
\begin{proof}
  The two equivalences are connected using \hyperref[thm:boolean_equivalences/double_negation]{double negation}. We will only prove \eqref{thm:first_order_quantifiers_are_dual/negation_of_universal}.

  Let \( (\mscrX, I) \) be a structure over \( \mscrL \) and let \( v \) be a variable assignment. Then
  \begin{align*}
    (\neg \qforall \xi \varphi)\Bracks{v}
    &=
    \overline{(\qforall \xi \varphi)\Bracks{v}}
    = \\ &=
    \overline{\bigwedge\set{ \varphi\Bracks{v_{\xi \mapsto x}} \given x \in \mscrX }}
    \overset {\eqref{eq:thm:de_morgans_laws/complement_of_meet}} = \\ &=
    \bigvee\set{ \overline{\varphi\Bracks{v_{\xi \mapsto x}}} \given x \in \mscrX }
    = \\ &=
    \bigvee\set{ (\neg \varphi)\Bracks{v_{\xi \mapsto x}} \given x \in \mscrX }
    = \\ &=
    (\qexists \xi \neg \varphi)\Bracks{v}.
  \end{align*}
\end{proof}

\begin{proposition}\label{thm:semantic_implicit_universal_quantification}
  For any formula \( \varphi \) and any variable \( \xi \) over \( \mscrL \), the formulas \( \varphi \) and \( \qforall \xi \varphi \) are \hyperref[def:first_order_semantics/equivalent]{semantically equivalent}.

  This allows us to skip quantifiers when writing formulas without changing their validity. Given a formula \( \varphi \) with free variables \( \xi_1, \ldots, \xi_n \), we call
  \begin{equation*}
    \qforall {\xi_1} \cdots \qforall {\xi_n} \varphi
  \end{equation*}
  its \term{universal closure} and say that \( \varphi \) itself is \term{implicitly universally quantified}. Universal closures of quantifierless formulas are called \term{universal formulas}.

  Compare this result with \fullref{thm:syntactic_implicit_universal_quantification}.
\end{proposition}
\begin{proof}
  Let \( (\mscrX, I) \) be a structure that satisfies \( \varphi \). Let \( v \) be a variable assignment in \( (\mscrX, I) \). Then for any \( x \in \mscrX \), the modified variable assignment \( v_{\xi \mapsto x} \) also satisfies \( \varphi \), i.e.
  \begin{equation*}
    \varphi\Bracks{v} = \varphi\Bracks{v_{\xi \mapsto x}} = T.
  \end{equation*}

  Thus \( (\mscrX, I) \) is also a model for \( \qforall \xi \varphi \).

  Conversely, suppose that \( (\mscrX, I) \) satisfies \( \qforall \xi \varphi \) and \( v \) is any variable assignment. Then
  \begin{equation*}
    \varphi\Bracks{v_{\xi \mapsto x}} = T
  \end{equation*}
  for any \( x \), including \( x \coloneqq v(\xi) \). Thus
  \begin{equation*}
    \varphi\Bracks{v_{\xi \mapsto v(\xi)}} = \varphi\Bracks{v} = T.
  \end{equation*}

  Therefore \( (\mscrX, I) \) is also a model for \( \varphi \).
\end{proof}

\begin{proposition}\label{thm:quantifier_satisfiability}
  Let \( \mscrL \) be a first-order language, \( \varphi \) be a formula, \( \xi \) be a variable and \( \tau \) be a \hyperref[def:first_order_syntax/ground_term]{ground term} in \( \mscrL \). The following hold:

  \begin{thmenum}
    \thmitem{thm:quantifier_satisfiability/universal} \( \qforall \xi \varphi \vDash \varphi[\xi \mapsto \tau] \),

    \thmitem{thm:quantifier_satisfiability/existential} \( \varphi[\xi \mapsto \tau] \vDash \qexists \xi \varphi \).

    Furthermore, the formulas \( \varphi[\xi \mapsto \tau] \) and \( \qexists \xi \varphi \) are actually equisatisfiable.
  \end{thmenum}

  See also \fullref{def:first_order_axiomatic_derivation_system/axioms/terms} for the syntactic counterpart to this proposition.
\end{proposition}
\begin{proof}
  The proof is very straightforward but the technical details make it look a bit more complicated.

  Let \( (\mscrX, I) \) be any structure and let \( v \) be a variable assignment in that structure. Let \( t \coloneqq \tau\Bracks{v} \). Since \( \tau \) is a ground term, we have
  \begin{equation*}
    \xi\Bracks{v_{\xi \mapsto t}} = \tau\Bracks{v_{\xi \mapsto t}},
  \end{equation*}
  hence from \fullref{thm:first_order_substitution_equivalence/term_in_formula} we have
  \begin{equation*}
    \varphi\Bracks{v_{\xi \mapsto t}}
    =
    \varphi[\xi \mapsto \tau]\Bracks{v_{\xi \mapsto t}}.
  \end{equation*}

  But \( \xi \) does not occur freely in \( \varphi[\xi \mapsto \tau] \), hence
  \begin{equation*}
    \varphi[\xi \mapsto \tau]\Bracks{v_{\xi \mapsto t}}
    =
    \varphi[\xi \mapsto \tau]\Bracks{v}.
  \end{equation*}

  \SubProofOf{thm:quantifier_satisfiability/universal} If \( v \) satisfies \( \qforall \xi \varphi \), then for any variable \( x \), the valuation \( v_{\xi \mapsto x} \). Since \( t \) is a variable, we have
  \begin{equation}\label{eq:thm:quantifier_satisfiability/universal/valuation}
    \varphi\Bracks{v_{\xi \mapsto t}}
    =
    \varphi[\xi \mapsto \tau]\Bracks{v}
    =
    T.
  \end{equation}

  Hence \( v \) satisfies \( \varphi[\xi \mapsto \tau] \). Since \( v \) was arbitrary, we conclude that \( (\mscrX, I) \) is a model of \( \varphi[\xi \mapsto \tau] \) also.

  \SubProofOf{thm:quantifier_satisfiability/existential} If \( v \) satisfies \( \varphi[\xi \mapsto \tau] \), then \eqref{eq:thm:quantifier_satisfiability/universal/valuation} holds and hence \( v \) satisfies \( \qexists \xi \varphi \).

  To prove equisatisfiability, suppose that \( v \) satisfies \( \qexists \xi \varphi \). Then there exists a value \( x \) such that
  \begin{equation}\label{eq:thm:quantifier_satisfiability/existential/existence_modified_assignment}
    \varphi\Bracks{v_{\xi \mapsto x}} = T.
  \end{equation}

  Define the interpretation \( \widetilde I \) as \( I \) modified at all constants in \( \tau \) so that \( \tau\Bracks{v} = x \). It remains to show that the structure \( (\mscrX, \widetilde I) \) is a model of \( \varphi[\xi \mapsto \tau] \).

  Since the new structure has the same domain, \( v \) is a variable assignment in this new structure. Nevertheless, we will denote it by \( \widetilde v \) in order to distinguish between valuations in the two structures. As in the proof of the other direction, we have
  \begin{equation*}
    \xi\Bracks{\widetilde v_{\xi \mapsto x}}
    =
    \tau\Bracks{\widetilde v_{\xi \mapsto x}}
  \end{equation*}

  Hence
  \begin{equation*}
    \varphi[\xi \mapsto \tau]\Bracks{\widetilde v_{\xi \mapsto x}}
    \overset {\ref{thm:first_order_substitution_equivalence/term_in_formula}} =
    \varphi\Bracks{\widetilde v_{\xi \mapsto x}}
    \overset {\eqref{eq:thm:quantifier_satisfiability/existential/existence_modified_assignment}} =
    T
  \end{equation*}

  Since \( \xi \) does not occur freely in \( \varphi[\xi \mapsto \tau] \), it follows that
  \begin{equation*}
    \varphi[\xi \mapsto \tau]\Bracks{\widetilde v} = \varphi[\xi \mapsto \tau]\Bracks{\widetilde v_{\xi \mapsto x}} = T.
  \end{equation*}

  Since \( \widetilde v \) was chosen arbitrarily, it follows that \( (\mscrX, \widetilde I) \) is a model of \( \varphi[\xi \mapsto c] \).
\end{proof}

\begin{theorem}[Semantic deduction theorem]\label{thm:semantic_deduction_theorem}\mcite[thm. 16.29]{OpenLogicFull}
  The entailment \( \Gamma, \psi \vDash \varphi \) holds if and only if \( \Gamma \vDash \psi \to \varphi \) holds. Compare this result to \fullref{thm:syntactic_deduction_theorem}.
\end{theorem}
