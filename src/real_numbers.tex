\subsection{Real numbers}\label{subsec:real_numbers}

\begin{definition}\label{def:rational_numbers}
  The \term{rational numbers} \( \BbbQ \) are the field of \hyperref[def:ring_localization]{fractions} of the \hyperref[def:integers]{integers}. Both operations from \( \BbbZ \) are inherited in \( \BbbQ \) and all nonzero elements in \( \BbbQ \) are now invertible, which makes \( \BbbQ \) a field.
\end{definition}

\begin{definition}\label{def:real_numbers}
  The \term{real numbers} \( \BbbR \) are the metric space \hyperref[def:complete_metric_space]{completion} of \( \BbbQ \) with respect to the absolute value. Unfortunately, real numbers are used for defining metric spaces, so we cannot rely on the theory of metric spaces. This can be circumvented by
  \begin{enumerate}
    \item Regarding \( \BbbQ \) as a \hyperref[def:uniform_space]{uniform space}.
    \item Using uniform space \hyperref[thm:uniform_space_completion]{completion} to obtain \( \BbbR \).
    \item Defining metric spaces.
    \item Showing that \( \BbbR \) is a metric space.
    \item Using \fullref{def:complete_metric_space/uniform} to automatically verify that \( \BbbR \) is complete as a metric space.
  \end{enumerate}
\end{definition}

\begin{definition}\label{def:extended_real_numbers}
  We are sometimes interested in \term{extended real numbers}. These can be any of the three sets
  \begin{itemize}
    \item \( \BbbR \cup \{ +\infty \} \),
    \item \( \BbbR \cup \{ -\infty \} \),
    \item \( \BbbR \cup \{ -\infty, +\infty \} \),
  \end{itemize}
  where \( -\infty \) and \( +\infty \) are both sentinel values that act as the \hyperref[def:partially_ordered_set_extremal_points/maximum_and_minimum]{greatest} and/or least real number.

  We generally avoid performing arithmetic operations on \( \pm \infty \), however it is sometimes convenient to define
  \begin{balign*}
    x + (+\infty)     & \coloneqq +\infty, x \in \BbbR
                      &                              &
    x \cdot (+\infty) & \coloneqq +\infty, x \in \BbbR
  \end{balign*}

  We leave the operations
  \begin{balign*}
     & (-\infty) + (+\infty)
     & (-\infty) \cdot (+\infty)
  \end{balign*}
  undefined.

  With these operations, the extended real numbers are no longer a \hyperref[def:field]{field}.
\end{definition}

\begin{definition}\label{def:floor_ceiling_functions}
  Let \( x \in \BbbR \) be a real number. In analogy with \fullref{def:divisibility}, we define its \term{floor}
  \begin{equation*}
    \floor(x) \coloneqq \max \{ n \in \BbbZ : n \leq x \},
  \end{equation*}
  its \term{ceiling}
  \begin{equation*}
    \ceil(x) \coloneqq \min \{ n \in \BbbZ : n \geq x \}
  \end{equation*}
  and its \term{fractional part}
  \begin{equation*}
    \op{frac}(x) \coloneqq x - \floor(x).
  \end{equation*}
\end{definition}

\begin{proposition}\label{thm:reals_not_algebraically_closed}
  The field \( \BbbR \) is not algebraically \hyperref[def:algebraically_closed_field]{closed}.

  In particular, the polynomial \( x^2 + 1 \) has no root.
\end{proposition}
\begin{proof}
  Assume that \( \BbbR \) is algebraically closed and that the polynomial \( x^2 + 1 \) has at least one root. Denote one of them by \( u \).

  By the \hyperref[def:binary_relation/trichotomic]{trichotomy} of the order \( < \) of \( \BbbR \), we have either \( u < 1 \) or \( u > 1 \) since \( u \neq 1 \).

  If \( u < 0 \), then \( u^2 = -1 < 0 \), which is impossible because the image of \( x \mapsto x^2 \) is the interval \( [0, \infty) \).

  If \( u > 0 \), then \( u^2 = -1 < 0 = 0 \), which is also impossible because \( x \mapsto x^2 \) is monotone on \( [0, \infty) \).

  Thus, \( u \) is not a root of \( x^2 + 1 \) and \( \BbbR \) is not algebraically closed.
\end{proof}

\begin{definition}\label{def:signum}
  We define the \term{signum} function \( \sgn: \BbbR \to \{ -1, 0, 1 \} \) as
  \begin{equation*}
    \sgn(x) \coloneqq \begin{cases}
      1,  & x > 0, \\
      0,  & x = 0, \\
      -1, & x < 0.
    \end{cases}
  \end{equation*}
\end{definition}

\begin{remark}\label{rem:congruence_modulo_real_number}
  Over the real numbers, \hyperref[def:congruence_modulo_normal_subgroup]{congruence modulo subgroups} is occasionally useful. For example, the \hyperref[def:angle]{angles} \( \alpha \) and \( \beta \coloneqq \alpha + 2\pi \) correspond to the same rotation, and we can write
  \begin{equation*}
    \alpha \cong \beta \pmod 2\pi
  \end{equation*}
  to mean that \( \beta - \alpha \) is a multiple of \( 2\pi \).

  This should not be confused with congruence modulo ideals, because over a field the ideal generated by a nonzero number is the field itself.
\end{remark}
