\subsection{Boolean functions}\label{subsec:boolean_functions}

\begin{definition}\label{def:boolean_value}
  Fix a two-element set \( \set{ T, F } \). We can think of \( T \) as a value denoting truth and \( F \) as denoting falsity. See \fullref{rem:mathematical_logic_conventions/propositional_constants} for notation conventions.

  There is a natural Boolean algebra structure on \( \set{ T, F } \) where \( T \) is the \hyperref[def:samilattice/join]{top} and \( F \) is the \hyperref[def:samilattice/meet]{bottom} and the operations are defined in an obvious way. By \fullref{thm:binary_lattices_are_isomorphic}, there is a unique (up to isomorphism) two-element Boolean algebra. Without further context, we can assume that \( \set{ T, F } \) the \hyperref[thm:galois_field_existence]{Galois field} \( \BbbF_2 \) where we identify \( F \) with \( 0 \) and \( T \) with \( 1 \).
\end{definition}

\begin{definition}\label{def:boolean_function}
  We call functions from any set to \( \set{ T, F } \) \term{Boolean-valued} and functions from \( \set{ T, F }^n \) to \( \set{ T, F } \) \term{Boolean}.
\end{definition}

\begin{definition}\label{def:boolean_closure}
  Fix a set \( \mscrB \) of Boolean functions of arbitrary arities.

  The \term{closure} \( \cl{\mscrB} \) of \( \mscrB \) is defined inductively as follows:
  \begin{itemize}
    \item If \( f \in \mscrB \), then \( f \in \cl{\mscrB} \)
    \item If \( f_k(x_1, \ldots, x_n) \in \cl{\mscrB}, k = 1, \ldots, m \) and \( g(x_1, \ldots, x_m) \in \cl{\mscrB} \), then their \hyperref[def:function/superposition]{superposition}
    \begin{equation*}
      h(x_1, \ldots, x_n) \coloneqq g(f_1(x_1, \ldots, x_n), \ldots, f_m(x_1, \ldots, x_n))
    \end{equation*}
    is also in \( \cl{\mscrB} \).
  \end{itemize}

  We say that \( \mscrB \) is \term{closed} if \( \cl{\mscrB} = \mscrB \) and \term{complete} if \( \cl{\mscrB} \) is the set of all Boolean functions of arbitrary arity.

  See \fullref{thm:propositional_formula_cosets_are_boolean_functions/boolean_algebra} for a consequence of the fact that the set of all Boolean functions is a Boolean algebra.
\end{definition}

\begin{definition}\label{def:zhegalkin_polynomial}
  A \term{Zhegalkin polynomial} is a \hyperref[def:polynomial]{polynomial} in the \hyperref[thm:galois_field_existence]{Galois field} \( \BbbF_2 \).

  Due to \fullref{thm:polynomial_embedding_behavior}, we may restrict ourselves to polynomials with no powers higher than \( 1 \). For example, for binary operations, we may restrict ourselves to the Zhegalkin polynomials
  \begin{equation}\label{eq:def:zhegalkin_polynomial/binary_polynomial}
    f(x, y) = axy \oplus bx \oplus cy \oplus d,
  \end{equation}
  where \( a, b, c, d \in \BbbF_2 \), since these are exactly the polynomials that correspond to unique binary Boolean functions.

  Using this conventions, we can avoid distinguishing between polynomials and polynomial functions (see \fullref{rem:polynomials_vs_polynomial_functions}).
\end{definition}

\begin{definition}\label{def:standard_boolean_operators}
  Unlike \hyperref[def:function/single_valued]{arbitrary functions}, \hyperref[def:boolean_function]{Boolean functions} only have a small finite number of possible values that can easily be enumerated.

  Out of the following binary operations, \( \vee \), \( \wedge \) and \( \overline{\placeholder} \) form the \hyperref[thm:f2_is_boolean_algebra]{Boolean algebra structure} on \( \BbbF_2 \) and \( \oplus \) and \( \wedge \) form the \hyperref[def:field]{field structure} on \( \BbbF_2 \). The operations \( \rightarrow \) and \( \leftrightarrow \) are also defined in any \hyperref[def:boolean_algebra]{Boolean algebra}.

  \begin{center}
    \begin{tabular}{c | c || c c | c c c c c c}
      \( x \) & \( \overline{x} \) & \( x \) & \( y \) & \( x \vee y \) & \( x \oplus y \)    & \( x \wedge y \) & \( x \rightarrow y \)   & \( x \leftrightarrow y \) \\
      \hline
              & not \( x \)        &         &         & \( x \) or \( y \)  & \( x \) xor \( y \) & \( x \) and \( y \)     & \( x \) implies \( y \) & \( x \) iff \( y \)       \\
      \hline
      \( F \) & \( T \)            & \( F \) & \( F \) & \( F \)             & \( F \)             & \( F \)                 & \( T \)                 & \( T \)                   \\
      \( T \) & \( F \)            & \( F \) & \( T \) & \( F \)             & \( T \)             & \( F \)                 & \( T \)                 & \( F \)                   \\
              &                    & \( T \) & \( F \) & \( F \)             & \( T \)             & \( F \)                 & \( F \)                 & \( F \)                   \\
              &                    & \( T \) & \( T \) & \( T \)             & \( F \)             & \( T \)                 & \( T \)                 & \( T \)                   \\
      \hline
              & \( x \oplus 1 \) &         &         & \( xy \oplus x \oplus y \) & \( x \oplus y \)    & \( xy \)            & \( xy \oplus x \oplus 1 \) & \( x \oplus y \oplus 1 \)
    \end{tabular}
  \end{center}

  See \fullref{thm:boolean_equivalences} for direct consequences of these definitions.
\end{definition}

\begin{definition}\label{def:boolean_functions_in_f2}
  Fix a \hyperref[def:boolean_function]{Boolean function} \( f(x_1, \ldots, x_n) \) in the \hyperref[thm:galois_field_existence]{Galois field} \( \BbbF_2 \),

  \begin{thmenum}
    \thmitem{def:boolean_function_in_f2/dual} Its \term{dual function} is
    \begin{equation*}
      \overline{f}(x_1, \ldots, x_n) \coloneqq \overline{f(\overline{x_1}, \ldots, \overline{x_n})}.
    \end{equation*}

    \thmitem{def:boolean_functions_in_f2/self_dual} \( f \) is \term{self-dual} if it is its own \hyperref[def:boolean_function_in_f2/dual]{dual}.

    \thmitem{def:boolean_functions_in_f2/truth_preserving} \( f \) is \term{truth-preserving} if \( f(1, \ldots, 1) = 1 \).

    \thmitem{def:boolean_functions_in_f2/falsity_preserving} \( f \) is \term{falsity-preserving} if \( f(0, \ldots, 0) = 0 \).

    \thmitem{def:boolean_functions_in_f2/monotone} \( f \) is \term{monotone} if, for any two tuples of arguments \( x_1, \ldots, x_n \in \BbbF_2 \) and \( y_1, \ldots, y_n \in \BbbF_2 \), the inequalities \( x_k \leq y_k \) for all \( k = 1, \ldots, n \) imply that
    \begin{equation*}
      f(x_1, \ldots, x_n) \leq f(y_1, \ldots, y_n).
    \end{equation*}

    \thmitem{def:boolean_functions_in_f2/linear} \( f \) is \term{linear} if its \hyperref[def:zhegalkin_polynomial]{Zhegalkin polynomial} is linear, i.e. has only monomials of degree \( 0 \) or \( 1 \). In the case of binary Boolean functions, this means that the coefficient \( a \) in \eqref{eq:def:zhegalkin_polynomial/binary_polynomial} is zero.
  \end{thmenum}
\end{definition}

\begin{theorem}[Post's completeness theorem]\label{thm:posts_completeness_theorem}\mcite{Martin1990}
  The family \( \mscrB \) of Boolean functions is \hyperref[def:boolean_closure]{complete} if and only if all of the following conditions are satisfied:
  \begin{thmenum}
    \thmitem{thm:posts_completeness_theorem/truth_preserving} \( \mscrB \) contains a function that is not \hyperref[def:boolean_functions_in_f2/truth_preserving]{truth-preserving}.
    \thmitem{thm:posts_completeness_theorem/falsity_preserving} \( \mscrB \) contains a function that is not \hyperref[def:boolean_functions_in_f2/falsity_preserving]{falsity-preserving}.
    \thmitem{thm:posts_completeness_theorem/self_dual} \( \mscrB \) contains a function that is not \hyperref[def:boolean_functions_in_f2/self_dual]{self-dual}.
    \thmitem{thm:posts_completeness_theorem/monotone} \( \mscrB \) contains a function that is not \hyperref[def:boolean_functions_in_f2/monotone]{monotone}.
    \thmitem{thm:posts_completeness_theorem/linear} \( \mscrB \) contains a function that is not \hyperref[def:boolean_functions_in_f2/linear]{linear}.
  \end{thmenum}
\end{theorem}

\begin{example}\label{ex:posts_completeness_theorem}
  We give examples of complete sets of Boolean functions in \( \BbbF_2 \).

  \begin{thmenum}
    \thmitem{ex:posts_completeness_theorem/and_or} The archetypic example of a complete set of Boolean functions is the triple \( B_\vee, B_\wedge, \overline{\placeholder} \) that forms the Boolean algebra structure on \( \BbbF_2 \).

    We verify that the conditions of \fullref{thm:posts_completeness_theorem} are satisfied:
    \begin{refenum}
      \refitem{thm:posts_completeness_theorem/truth_preserving} \( \overline{\placeholder} \) is not truth-preserving.
      \refitem{thm:posts_completeness_theorem/falsity_preserving} \( \overline{\placeholder} \) is not falsity-preserving.
      \refitem{thm:posts_completeness_theorem/self_dual} Neither \( \vee \) nor \( \wedge \) are self-dual. In fact, by \fullref{thm:de_morgans_laws}, \( \wedge \) is the dual of \( \vee \) and vice versa.
      \refitem{thm:posts_completeness_theorem/monotone} \( \overline{\placeholder} \) is not monotone.
      \refitem{thm:posts_completeness_theorem/linear} Neither \( \vee \) nor \( \wedge \) are linear.
    \end{refenum}

    Thus \( \set{ \wedge, \vee, \overline{\placeholder} } \) is indeed a complete set of Boolean functions. Note that having both \( \vee \) and \( \wedge \) is redundant and we usually include both for symmetry. The families \( \set{ \wedge, \overline{\placeholder} } \) and \( \set{ \vee, \overline{\placeholder} } \) are both complete.

    This is used for \hyperref[alg:conjunctive_normal_form_reduction]{conjunctive normal forms}.

    \thmitem{ex:posts_completeness_theorem/nand} We can go even further and have a single binary Boolean function generate all others. We will use the function
    \begin{equation}\label{eq:ex:posts_completeness_theorem/nand}
      (x \uparrow y) \coloneqq \overline{x \wedge y} = xy \oplus 1.
    \end{equation}

    This operation is called \term{Sheffer's stroke} or \term{nand} (\enquote{not and}).

    We have
    \begin{equation*}
      \begin{array}{ccc}
        \overline{x} = (x \uparrow 1)
        &
        \T{and}
        &
        (x \wedge y) = \overline{x \uparrow y},
      \end{array}
    \end{equation*}
    which allows us to reduce the case to \fullref{ex:posts_completeness_theorem/and_or}. We conclude that the singleton set \( \set{ \uparrow } \) is a complete set of Boolean operations.

    \thmitem{ex:posts_completeness_theorem/conditional_negation} Another commonly used complete family is \( \set{ \rightarrow, \overline{\placeholder} } \).
    We verify that the conditions of \fullref{thm:posts_completeness_theorem} are satisfied:
    \begin{refenum}
      \refitem{thm:posts_completeness_theorem/truth_preserving} \( \overline{\placeholder} \) is not truth-preserving.
      \refitem{thm:posts_completeness_theorem/falsity_preserving} \( \rightarrow \) is not falsity-preserving because \( (F \rightarrow F) = T \).
      \refitem{thm:posts_completeness_theorem/self_dual} \( \rightarrow \) is not self-dual because \( \overline{\overline{x} \rightarrow \overline{y}} = (y \rightarrow x) \neq (x \rightarrow y) \).
      \refitem{thm:posts_completeness_theorem/monotone} \( \rightarrow \) is not monotone because \( F \rightarrow T = F \).
      \refitem{thm:posts_completeness_theorem/linear} \( \rightarrow \) is not linear.
    \end{refenum}

    \thmitem{ex:posts_completeness_theorem/conditional_bottom} Given the family \( \set{ B_\rightarrow, F } \), we can define
    \begin{equation*}
      \overline{x} \coloneqq (x \rightarrow F),
    \end{equation*}
    which shows that \( \set{ \rightarrow, F } \) is also a complete family.

    This is used in \hyperref[def:propositional_implicational_logic]{propositional implicational logic}.
  \end{thmenum}
\end{example}
