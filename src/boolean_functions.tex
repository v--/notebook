\subsection{Boolean functions}\label{subsec:boolean_functions}

\begin{definition}\label{def:boolean_value}
  Fix a two-element set \( \set{ T, F } \). We can think of \( T \) as a value denoting truth and \( F \) as denoting falsity. See \fullref{rem:mathematical_logic_conventions/propositional_constants} for notation conventions.

  There is a natural \hyperref[def:boolean_algebra]{Boolean algebra} structure on \( \set{ T, F } \) where \( T \) is the \hyperref[def:partially_ordered_set_extremal_points/top_and_bottom]{top} and \( F \) is the \hyperref[def:partially_ordered_set_extremal_points/top_and_bottom]{bottom} and the operations are defined in an obvious way.

  By \fullref{thm:binary_boolean_algebra}, all two-element Boolean algebras are isomorphic, so our choice of symbols does not matter much. We do sometimes identify \( T \) with \( 1 \) and \( F \) with \( 0 \) in the \hyperref[thm:finite_fields]{prime field} \( \BbbF_2 \). The latter is a Boolean algebra as discussed in \fullref{thm:f2_is_boolean_algebra}.
\end{definition}

\begin{definition}\label{def:boolean_function}
  We call functions from any set to \( \set{ T, F } \) (Boolean-valued) \term{predicates} and functions from \( \set{ T, F }^n \) to \( \set{ T, F } \) \term{Boolean functions}.
\end{definition}

\begin{remark}\label{rem:boolean_valued_functions_and_predicates}
  \hyperref[def:boolean_function]{Boolean-valued functions} and \hyperref[def:relation]{relations} represent the same concept. In particular, the relation \( R \subseteq A_1 \times \cdots \times A_n \) corresponds to a unique Boolean-valued function
  \begin{equation*}
    \begin{aligned}
      &f: X_1 \times \cdots \times X_n \to \set{ T, F } \\
      &f(x_1, \ldots, x_n) = \begin{cases}
        T, &(x_1, \ldots, x_n) \in R, \\
        F, &\T{otherwise}
      \end{cases}
    \end{aligned}
  \end{equation*}
  and vice versa.
\end{remark}

\begin{definition}\label{def:boolean_closure}
  Fix a set \( B \) of Boolean functions of arbitrary arities.

  The \term{closure} \( \cl{B} \) of \( B \) is defined \hyperref[rem:structural_recursion_and_induction]{recursively} as follows:
  \begin{itemize}
    \item If \( f \in B \), then \( f \in \cl{B} \)
    \item If \( f_k(x_1, \ldots, x_n) \in \cl{B} \) for \( k = 1, \ldots, m \) and if \( g(x_1, \ldots, x_m) \in \cl{B} \), then their \hyperref[def:multi_valued_function/superposition]{superposition}
    \begin{equation*}
      h(x_1, \ldots, x_n) \coloneqq g(f_1(x_1, \ldots, x_n), \ldots, f_m(x_1, \ldots, x_n))
    \end{equation*}
    is also in \( \cl{B} \).
  \end{itemize}

  We say that \( B \) is \term{closed} if \( \cl{B} = B \) and \term{complete} if \( \cl{B} \) is the set of all Boolean functions of arbitrary arity.

  If \( B \) is complete, then from \fullref{thm:functions_over_model_form_model} it follows that \( B \) is a Boolean algebra. This is used in \fullref{thm:lindenmaum_tarski_algebra_of_full_propositional_logic/bijection}.
\end{definition}

\begin{definition}\label{def:zhegalkin_polynomial}
  A \term{Zhegalkin polynomial} is a \hyperref[def:polynomial_semiring]{polynomial} in the \hyperref[thm:finite_fields]{prime field} \( \BbbF_2 \). Due to \fullref{thm:functions_over_prime_fields}, however, we restrict ourselves to polynomials with \hyperref[def:square_free]{square-free} monomials.

  For example, for every binary Boolean function there exist coefficients \( a, b, c, d \in \BbbF_2 \) such that
  \begin{equation}\label{eq:def:zhegalkin_polynomial/binary_polynomial}
    f(x, y) = axy \oplus bx \oplus cy \oplus d.
  \end{equation}
\end{definition}

\begin{definition}\label{def:standard_boolean_operators}
  Unlike \hyperref[def:function]{arbitrary functions}, \hyperref[def:boolean_function]{Boolean functions} only have a small finite number of possible values that can easily be enumerated.

  Out of the following binary operations, \( \vee \), \( \wedge \) and \( \overline{\anon} \) form the \hyperref[def:boolean_algebra]{Boolean algebra} structure on \( \BbbF_2 \) and \( \oplus \) and \( \wedge \) form the \hyperref[def:field]{field} structure on \( \BbbF_2 \). The operations \( \rightarrow \) and \( \leftrightarrow \) are also defined in any \hyperref[def:boolean_algebra]{Boolean algebra}.

  \begin{center}
    \begin{tabular}{c | c || c c | c c c c c c}
      \( x \) & \( \overline{x} \) & \( x \) & \( y \) & \( x \vee y \)             & \( x \oplus y \)    & \( x \wedge y \)        & \( x \rightarrow y \)   & \( x \leftrightarrow y \) \\
      \hline
              & not \( x \)        &         &         & \( x \) or \( y \)         & \( x \) xor \( y \) & \( x \) and \( y \)     & \( x \) implies \( y \) & \( x \) iff \( y \)       \\
      \hline
      \( F \) & \( T \)            & \( F \) & \( F \) & \( F \)                    & \( F \)             & \( F \)                 & \( T \)                 & \( T \)                   \\
      \( T \) & \( F \)            & \( F \) & \( T \) & \( F \)                    & \( T \)             & \( F \)                 & \( T \)                 & \( F \)                   \\
              &                    & \( T \) & \( F \) & \( F \)                    & \( T \)             & \( F \)                 & \( F \)                 & \( F \)                   \\
              &                    & \( T \) & \( T \) & \( T \)                    & \( F \)             & \( T \)                 & \( T \)                 & \( T \)                   \\
      \hline
              & \( x \oplus 1 \)   &         &         & \( xy \oplus x \oplus y \) & \( x \oplus y \)    & \( xy \)            & \( xy \oplus x \oplus 1 \) & \( x \oplus y \oplus 1 \)
    \end{tabular}
  \end{center}

  See \fullref{thm:boolean_equivalences} for direct consequences of these definitions.
\end{definition}

\begin{definition}\label{def:boolean_functions_in_f2}\mcite[I.1.\S6]{Яблонский1986}
  Fix a \hyperref[def:boolean_function]{Boolean function} \( f(x_1, \ldots, x_n) \) in the \hyperref[thm:finite_fields]{prime field} \( \BbbF_2 \).

  \begin{thmenum}
    \thmitem{def:boolean_function_in_f2/dual} Its \term{dual function} is
    \begin{equation*}
      \overline{f}(x_1, \ldots, x_n) \coloneqq \overline{f(\overline{x_1}, \ldots, \overline{x_n})}.
    \end{equation*}

    \thmitem{def:boolean_functions_in_f2/self_dual} \( f \) is \term{self-dual} if it is its own \hyperref[def:boolean_function_in_f2/dual]{dual}.

    \thmitem{def:boolean_functions_in_f2/truth_preserving} \( f \) is \term{truth-preserving} if \( f(T, \ldots, T) = T \).

    \thmitem{def:boolean_functions_in_f2/falsity_preserving} \( f \) is \term{falsity-preserving} if \( f(F, \ldots, F) = F \).

    \thmitem{def:boolean_functions_in_f2/monotone} \( f \) is \term{monotone} if, for any two tuples of arguments \( x_1, \ldots, x_n \in \BbbF_2 \) and \( y_1, \ldots, y_n \in \BbbF_2 \), the inequalities \( x_k \leq y_k \) for all \( k \in \set{ 1, \ldots, n } \) imply that
    \begin{equation*}
      f(x_1, \ldots, x_n) \leq f(y_1, \ldots, y_n).
    \end{equation*}

    \thmitem{def:boolean_functions_in_f2/linear} \( f \) is \term{linear} if its \hyperref[def:zhegalkin_polynomial]{Zhegalkin polynomial} is linear, i.e. has only monomials of degree \( 0 \) or \( 1 \). In the case of binary Boolean functions, this means that the coefficient \( a \) in \eqref{eq:def:zhegalkin_polynomial/binary_polynomial} is zero.
  \end{thmenum}
\end{definition}

\begin{theorem}[Post-Yablonsky completeness theorem]\label{thm:posts_completeness_theorem}\mcite[thm. I.1.7]{Яблонский1986}
  The family \( B \) of Boolean functions is \hyperref[def:boolean_closure]{complete} if and only if all of the following conditions are satisfied:
  \begin{thmenum}
    \thmitem{thm:posts_completeness_theorem/truth_preserving} \( B \) contains a function that is not \hyperref[def:boolean_functions_in_f2/truth_preserving]{truth-preserving}.
    \thmitem{thm:posts_completeness_theorem/falsity_preserving} \( B \) contains a function that is not \hyperref[def:boolean_functions_in_f2/falsity_preserving]{falsity-preserving}.
    \thmitem{thm:posts_completeness_theorem/self_dual} \( B \) contains a function that is not \hyperref[def:boolean_functions_in_f2/self_dual]{self-dual}.
    \thmitem{thm:posts_completeness_theorem/monotone} \( B \) contains a function that is not \hyperref[def:boolean_functions_in_f2/monotone]{monotone}.
    \thmitem{thm:posts_completeness_theorem/linear} \( B \) contains a function that is not \hyperref[def:boolean_functions_in_f2/linear]{linear}.
  \end{thmenum}
\end{theorem}

\begin{example}\label{ex:thm:posts_completeness_theorem}
  We give examples of complete sets of Boolean functions in \( \BbbF_2 \).

  \begin{thmenum}
    \thmitem{ex:thm:posts_completeness_theorem/and_or} The archetypic example of a complete set of Boolean functions is the triple \( \vee, \wedge, \overline{\anon} \) that forms the Boolean algebra structure on \( \BbbF_2 \).

    We verify that the conditions of \fullref{thm:posts_completeness_theorem} are satisfied:
    \begin{refenum}
      \refitem{thm:posts_completeness_theorem/truth_preserving} \( \overline{\anon} \) is not truth-preserving.
      \refitem{thm:posts_completeness_theorem/falsity_preserving} \( \overline{\anon} \) is not falsity-preserving.
      \refitem{thm:posts_completeness_theorem/self_dual} Neither \( \vee \) nor \( \wedge \) are self-dual. In fact, due to \fullref{thm:de_morgans_laws}, \( \wedge \) is the dual of \( \vee \) and vice versa.
      \refitem{thm:posts_completeness_theorem/monotone} \( \overline{\anon} \) is not monotone.
      \refitem{thm:posts_completeness_theorem/linear} Neither \( \vee \) nor \( \wedge \) have linear Zhegalkin polynomials.
    \end{refenum}

    Thus, \( \set{ \wedge, \vee, \overline{\anon} } \) is indeed a complete set of Boolean functions. Note that having both \( \vee \) and \( \wedge \) is redundant and we usually include both for symmetry. The families \( \set{ \wedge, \overline{\anon} } \) and \( \set{ \vee, \overline{\anon} } \) are both complete.

    This is utilized for \hyperref[def:cnf_and_dnf]{conjunctive and disjunctive normal forms}.

    \thmitem{ex:thm:posts_completeness_theorem/nand} We can go even further and have a single binary Boolean function generate all others. We will use the function
    \begin{equation}\label{eq:ex:thm:posts_completeness_theorem/nand}
      (x \uparrow y) \coloneqq \overline{x \wedge y} = xy \oplus 1.
    \end{equation}

    This operation is called \term{Sheffer's stroke} or \term{nand} (\enquote{not and}).

    We have
    \begin{equation*}
      \begin{array}{ccc}
        \overline{x} = (x \uparrow 1)
        &
        \T{and}
        &
        (x \wedge y) = \overline{x \uparrow y},
      \end{array}
    \end{equation*}
    which allows us to reduce the case to \fullref{ex:thm:posts_completeness_theorem/and_or}. We conclude that the singleton set \( \set{ \uparrow } \) is a complete set of Boolean operations.

    \thmitem{ex:thm:posts_completeness_theorem/conditional_negation} Another commonly used complete family is \( \set{ \rightarrow, \overline{\anon} } \).
    We verify that the conditions of \fullref{thm:posts_completeness_theorem} are satisfied:
    \begin{refenum}
      \refitem{thm:posts_completeness_theorem/truth_preserving} \( \overline{\anon} \) is not truth-preserving.
      \refitem{thm:posts_completeness_theorem/falsity_preserving} \( \rightarrow \) is not falsity-preserving because \( (F \rightarrow F) = T \).
      \refitem{thm:posts_completeness_theorem/self_dual} \( \rightarrow \) is not self-dual because \( \overline{\overline{x} \rightarrow \overline{y}} \reloset {\eqref{eq:thm:boolean_equivalences/contrapositive}} = (y \rightarrow x) \neq (x \rightarrow y) \).
      \refitem{thm:posts_completeness_theorem/monotone} \( \rightarrow \) is not monotone because \( F \rightarrow T = F \).
      \refitem{thm:posts_completeness_theorem/linear} \( \rightarrow \) doesn't have a linear Zhegalkin polynomial.
    \end{refenum}

    \thmitem{ex:thm:posts_completeness_theorem/conditional_bottom} Given the family \( \set{ \rightarrow, F } \), we can define
    \begin{equation*}
      \overline{x} \coloneqq (x \rightarrow F),
    \end{equation*}
    which shows that \( \set{ \rightarrow, F } \) is also a complete family.
  \end{thmenum}
\end{example}
