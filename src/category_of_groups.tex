\subsection{Category of groups}\label{subsec:category_of_groups}

\begin{definition}\label{def:group_direct_product}
  The \term{direct product} of a nonempty family of groups \( \{ \mscrX_k \}_{k \in \mscrK} \) is their \hyperref[def:cartesian_product]{Cartesian product} \( \prod_{k \in \mscrK} \mscrX_k \) with the componentwise group operation
  \begin{equation*}
    \{ x_k \}_{k \in \mscrK} \cdot \{ y_k \}_{k \in \mscrK}
    \coloneqq
    \{ x_k \cdot y_k \}_{k \in \mscrK}.
  \end{equation*}
\end{definition}

\begin{proposition}\label{thm:product_of_cyclic_groups}
  The \hyperref[def:group_direct_product]{direct product} \( C_n \times C_m \) of two \hyperref[def:cyclic_group]{cyclic groups} is cyclic if and only if \( n \) and \( m \) are \hyperref[def:coprime_numbers]{coprime}.
\end{proposition}
\begin{proof}
  Take \( (a^i, a^j) \in C_n \times C_m \).
\end{proof}

\begin{definition}\label{def:group_direct_sum}
  The \term{direct sum} \( \bigoplus_{k \in \mscrK} \mscrX_k \) of a nonempty family of groups \( \{ \mscrX_k \}_{k \in \mscrK} \) is a subgroup of their \hyperref[def:group_direct_sum]{direct product} where, for any group element \( \{ x_k \}_{k \in \mscrK} \), only finitely many components are different from zero.

  \begin{thmenum}
    \thmitem{def:group_direct_sum/internal}\mcite[126]{Knapp2016BasicAlgebra}If all \( \{ \mscrX_k \}_{k \in \mscrK} \) are subgroups of a group \( \mscrX \), we say that \( \mscrX \) is their \term{internal direct sum} if the homomorphism
    \begin{align*}
       &\varphi: \bigoplus_{k \in \mscrK} \mscrX_k \to X \\
       &\varphi(\{ x_k \}_{k \in \mscrK}) \coloneqq \cdot_{k \in \mscrK} x_k
    \end{align*}
    is an isomorphism.

    The sum is well-defined since, by definition, there are only finitely many non-identity summands.

    \thmitem{def:group_direct_sum/external} To distinguish \( \bigoplus_{k \in \mscrK} \mscrX_k \) from \( X \), we sometimes call it the \term{external direct product}.
  \end{thmenum}
\end{definition}

\begin{proposition}\label{thm:group_categorical_limits}
  We are interested in \hyperref[def:categorical_limit]{categorical limits} and \hyperref[def:categorical_colimit]{colimits} in \( \cat{Grp} \). Fix an indexed family  \( \{ \mscrX_k \}_{k \in \mscrK} \) of groups.

  \begin{thmenum}
    \thmitem{thm:group_categorical_limits/product} Their \hyperref[def:categorical_product]{categorical product} is their \hyperref[def:group_direct_product]{direct product} \( \prod_{k \in \mscrK} \mscrX_k \), the projection morphisms being inherited from \fullref{thm:set_categorical_limits/product}.

    \thmitem{thm:group_categorical_limits/coproduct} Their \hyperref[def:categorical_coproduct]{categorical coproduct} is their \hyperref[def:group_free_product]{free product} \( \Ast_{k \in \mscrK} \mscrX_k \), the embedding morphisms being
    \begin{balign*}
       &\iota_m: \mscrX_m \to \Ast_{k \in \mscrK} \mscrX_k \\
       &\iota_m(x_m) \coloneqq x_m.
    \end{balign*}
  \end{thmenum}
\end{proposition}
