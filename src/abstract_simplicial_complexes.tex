\begin{definition}\label{def:abstract_simplicial_complex}\cite[definition 2.1]{Carlsson2009}
  An \uline{abstract simplicial complex} is a pair $(V, \Sigma)$, where
  \begin{itemize}
    \item $V$ is a finite set
    \item $\sigma \in \Sigma$ and $\tau \subseteq \sigma$ implies $\tau \in \Sigma$.
  \end{itemize}
\end{definition}

\begin{definition}\label{def:simplicial_complex}
  A \uline{simplicial complex} in $\BB{R}^n$ is a set $K$ of simplices\Tinyref{def:simplex}, such that
  \begin{itemize}
    \item For any simplex $S \in K$, any face of $S$ is also in $K$.
    \item The intersection of any two simplices $S_1$ and $S_2$ of $K$ is either empty or is a face of both $S_1$ and $S_2$.
  \end{itemize}
\end{definition}

\begin{proposition}\label{thm:abstract_simplicial_complex_iff_simplicial_complex}
  Let $(V, \Sigma)$ be an abstract simplicial complex\Tinyref{def:abstract_simplicial_complex} and let $v_1, \ldots, v_n$ be an ordering of elements of $V$. Define the map $E(v_k) \coloneqq e_k, k = 1, \ldots, n$, where $e_k$ are the corresponding basis vectors in $\BB{R^n}$. Then the set
  \begin{align*}
    K \coloneqq \{ \Conv E(S) \colon S \in \Sigma \}
  \end{align*}
  is a simplicial complex\Tinyref{def:simplicial_complex}.

  Conversely, if $K$ is a simplicial complex in $\BB{R^n}$, denote by $V$ the set of all vertices of simplices in $K$ and
  \begin{align*}
    \Sigma \coloneqq \{ U \subseteq V \colon \Conv U \in K \}.
  \end{align*}

  Then $(V, \Sigma)$ is an abstract simplicial complex.
\end{proposition}
