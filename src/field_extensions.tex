\subsection{Field extensions}\label{subsec:field_extensions}

\begin{definition}\label{def:splitting_field}
  A \term{splitting field} for a nonconstant polynomial \( f(X) \in \Bbbk[X] \) of degree \( n \) is the smallest \hyperref[def:field/submodel]{field extension} \( \BbbK \) of \( \Bbbk \) in which \( f(X) \) has \( n \) \hyperref[def:polynomial_root]{roots}. That is,
  \begin{equation*}
    \BbbK \cong \Bbbk(a_1, \ldots, a_n),
  \end{equation*}
  where \( a_1, \ldots, a_n \) are roots of \( f(X) \).

  By \fullref{thm:splitting_field_existence}, splitting fields exist and are unique up to an isomorphism.
\end{definition}

\begin{proposition}\label{thm:splitting_field_existence}
  There exists a unique up to an (possibly nonunique) isomorphism \hyperref[def:splitting_field]{splitting field} for every nonconstant polynomial in one indeterminate over a field.
\end{proposition}
\begin{proof}
  \SubProof{Proof of existence}\mcite[thm 9.10 \\ thm. 9.12]{Knapp2016BasicAlgebra} We use induction on the degree of the polynomial \( f(X) = \sum_{k=0}^n a_k x^k \) over \( \Bbbk \). In the base case \( n = 1 \), \( f(X) \) is already linear, and hence \( \Bbbk \) is itself a splitting field for \( f(X) \).

  Suppose that there exist splitting fields for polynomials over \( \Bbbk \) of degree \( n - 1 \). By \fullref{thm:maximal_ideal_theorem}, the \hyperref[def:semiring_ideal]{principal ideal} \( \braket{ f(X) } \) is contained in some maximal ideal \( M \). By \fullref{thm:quotient_by_maximal_ideal}, the quotient of \( R[X] \) by \( M \) is a field.

  Define \( u_n \coloneqq \braket{ X } + M \). We have \( f(u_n) = \braket{ f(X) } + M \), hence \( u_n \) is a root of \( f \) in \( M \). Then
  \begin{equation*}
    f(X) = (X - u_n) q(X)
  \end{equation*}
  for some polynomials \( q(X) \) and \( r(X) \), both of degree less than \( n \) (or \( r(X) = 0 \)).

  We can now apply the inductive hypothesis to obtain a splitting field of \( q(X) \). Let \( u_1, \ldots, u_{n-1} \) be the roots of \( q(X) \) in this field. We can then adjoin \( u_1, \ldots, u_n \) to the field \( \Bbbk \) to obtain a splitting field \( \Bbbk(u_1, \ldots, u_n) \) of \( f(X) \). Denote this field by \( \BbbK \).

  \SubProof{Proof of uniqueness} Suppose that, given our previous construction, \( \BbbL \) is also a splitting field for \( f(X) \).

  Again, we use induction on the degree \( n \) of \( f(X) \). The case \( n = 1 \) is again obvious.

  Suppose that any two splitting fields for polynomials over \( \Bbbk \) of degree \( n - 1 \) are isomorphic. Let \( b_n \) be a root of \( f(X) \) in \( \BbbL \) and let
  \begin{equation*}
    f(X) = (X - b_n) r(X).
  \end{equation*}

  Let \( b_1, \ldots, b_{n-1} \) be the roots of \( r(X) \). Let \( \varphi \) be an isomorphism between the subfield \( \Bbbk(a_1, \ldots, a_{n-1}) \) of \( \BbbK \) and the corresponding subfield \( \Bbbk(b_1, \ldots, b_{n-1}) \) of \( \BbbL \). It follows that
  \begin{equation*}
    \underbrace{\prod_{k=1}^{n-1} (X - b_k)}_{r(X)} = \underbrace{\prod_{k=1}^{n-1} (X - \varphi(a_k))}_{q^\varphi(X)}.
  \end{equation*}

  Therefore, we can extend \( \varphi \) to an isomorphism \( \widehat{\varphi}: \BbbK \to \BbbL \) by putting \( \widehat{\varphi}(a_n) \coloneqq b_n \).
\end{proof}

\begin{theorem}[Classification of finite fields]\label{thm:finite_fields}
  \hfill
  \begin{thmenum}
    \thmitem{thm:finite_fields/characteristic} The \hyperref[def:ring_characteristic]{characteristic} of a \hyperref[def:field]{field} with \( q \) elements is a \hyperref[def:prime_number]{prime number} \( p \), and \( q \) is a power of \( p \).

    The fields of prime cardinality are sometimes called \term{prime fields}.

    \thmitem{thm:finite_fields/prime_field} For a prime number \( p \), the ring \hyperref[thm:ring_of_integers_modulo]{\( \BbbZ_p \)} of integers modulo \( p \) is a field.

    \thmitem{thm:finite_fields/splitting} All \hyperref[def:field]{fields} with \( q \) elements are \hyperref[def:field/homomorphism]{isomorphic} as \hyperref[def:splitting_field]{splitting fields} for the polynomial
    \begin{equation*}
      X^q - X \in \BbbZ_p[X].
    \end{equation*}

    Utilizing the general conventions of identifying isomorphic objects in algebra, we denote by \( \BbbF_q \) \enquote{the} finite field with \( q \) elements. Finite fields are also called \term{Galois fields}.

    Every member of \( \BbbF_q \) is a root of \( X^q - X \).
  \end{thmenum}
\end{theorem}
\begin{proof}
  \SubProofOf{thm:finite_fields/characteristic} Let \( \BbbK \) be a field with \( q \) elements and let \( p \) be the \hyperref[def:ring_characteristic]{characteristic} of \( \BbbK \). Then \( \BbbZ_p \) is a subring of \( \BbbK \). By \fullref{thm:multiplicative_group_of_integers_modulo}, \( \BbbZ_p \) is a field.

  By \fullref{thm:quotient_by_maximal_ideal}, \( \braket{ p } \) is a maximal ideal in \( \BbbZ \), and, by \fullref{thm:def:semiring_ideal/maximal_is_prime}, \( p \) is a prime number.

  By \fullref{thm:lagranges_theorem_for_groups}, \( p \) divides \( q \). But \( \BbbK / \BbbZ_p \) is again a field by \fullref{thm:quotient_ideal_lattice_theorem}, and again has prime characteristic. Continuing by induction, we eventually obtain a sequence \( p_1, \ldots, p_n \) of prime numbers such that
  \begin{equation*}
    q = p_1 \cdots p_n.
  \end{equation*}

  By \fullref{thm:lagranges_theorem_for_groups}, \( q \) cannot contain subgroups of prime cardinalities \( p_1 \) and \( p_2 \) unless \( p_1 = p_2 \). Hence, again by induction, we conclude that
  \begin{equation*}
    p_1 = \cdots = p_n.
  \end{equation*}

  Therefore, \( q = p^n \).

  \SubProofOf{thm:finite_fields/prime_field} Follows from \fullref{thm:multiplicative_group_of_integers_modulo}.

  \SubProofOf{thm:finite_fields/splitting} Let \( \BbbK \) be a field with \( q \) elements with characteristic \( p \). We will show that every element of \( \BbbK \) is a root of \( X^q - X \in \BbbZ_p[X] \).

  The multiplicative group of \( \BbbK \) has order \( q - 1 \). The order of a non-zero element \( a \in \BbbK \) divides \( q - 1 \) by \fullref{thm:def:group_order/divides}, hence \( a^{q - 1} = 1 \pmod q \). We also have \( 0^q = 0 \). Therefore, for every element of \( \BbbF_q \), we have \( a^q = a \).

  Then
  \begin{equation*}
    X^q - X = \prod_{u \in \BbbK} (X - u).
  \end{equation*}
\end{proof}

\begin{proposition}\label{thm:functions_over_prime_fields}
  For every \hyperref[thm:finite_fields]{finite field} \( \BbbF_q \) and every \hyperref[def:polynomial_algebra]{polynomial ring} \( \BbbF_q[X_1, \ldots, X_n] \) in finitely many indeterminates, there exists an \( \BbbF_q \)-\hyperref[def:algebra_over_ring]{algebra} isomorphism
  \begin{equation*}
    \frac {\BbbF_q[X_1, \ldots, X_n]} {\braket{ X_i^q - X_i \given i = 1, \ldots, n }} \cong \fun(\BbbF_q^n, \BbbF_q),
  \end{equation*}
  where \( \fun(\BbbF_q^n, \BbbF_q) \) is the \hyperref[thm:functions_over_algebra]{\( \BbbF_q \)-algebra of all functions} from \( \BbbF
  _q^n \) to \( \BbbF_q \).

  Furthermore, every coset of polynomials has a unique representative given by \fullref{thm:finite_field_lagrange_interpolation}.
\end{proposition}
\begin{proof}
  Consider the \hyperref[thm:polynomial_algebra_universal_property]{functional evaluation homomorphism}
  \begin{equation*}
    \Phi: \BbbF_q[X_1, \ldots, X_m] \to \fun(\BbbF_q^m, \BbbF_q).
  \end{equation*}

  By \fullref{thm:finite_field_lagrange_interpolation}, \( \Phi \) is surjective. Then, by \fullref{thm:quotient_algebra_universal_property},
  \begin{equation*}
    \BbbF_q[X_1, \ldots, X_n] / \ker \Phi \cong \fun(\BbbF_q^m, \BbbF_q).
  \end{equation*}

  We will now prove that \( \ker \Phi \) equals
  \begin{equation*}
    I \coloneqq \braket{ X_i^q - X_i \given i = 1, \ldots, n }.
  \end{equation*}

  First, let \( e: \mscrX \to \BbbF_q \) be the variable assignment that assigns \( u_1, \ldots, u_n \) to the corresponding indeterminates. By \fullref{thm:finite_fields/splitting}, every member of \( \BbbF_q \) is a root of \( X_i^q - X_i \). Then, for any indeterminate \( X_i \),
  \begin{equation*}
    \Phi_e(X_i^q - X_i) = u_i^q - u_i = 0 \pmod q.
  \end{equation*}

  Hence, the polynomial function \( \Phi(X_i^q - X_i) \) is the zero constant function. It follows that any linear combination of the polynomials \( X_i^q - X_i \) for \( i = 1, \ldots, n \) is also the zero function. Therefore, \( I \subseteq \ker \Phi \).

  We will prove the converse inclusion via induction on \( n \).

  In the case of a single indeterminate \( X \), for every polynomial \( f(X) \in \ker \Phi \), we know that the entirety of \( \BbbF_q \) are roots of \( f(X) \). By \fullref{thm:def:integral_domain/root_limit}, \( f(X) \) has at most \( q \) roots. Hence, \( X - u \) divides \( f(X) \) for every \( u \in \BbbF_q \). We have
  \begin{equation*}
    \underbrace{\prod_{u \in \BbbF_q} (X - u)}_{\mathclap{ X^q - X \T*{by \fullref{thm:finite_fields/splitting}}}} \mid f(X),
  \end{equation*}
  and hence \( f(X) \in \braket{ X^q - X } \).

  We have, up until now, shown that the entire proposition holds for the case of one indeterminate. Suppose that the proposition holds for \( n - 1 \) indeterminates and let \( f \in \BbbF_q[X_1, \ldots, X_n] \) be a nonconstant polynomial such that \( \Phi(f) \) is the zero function. Due to \fullref{thm:def:polynomial_algebra/iterated}, we can regard \( f \) as a univariate polynomial in \( X_n \)over \( \BbbF_q[X_1, \ldots, X_{n-1}] \). Thus,
  \begin{equation*}
    f(X_1, \ldots, X_n) = \sum_{k =0}^\infty \underbrace{\parens*{ \sum_\gamma a_{(k,\gamma)} X_1^{\gamma_1} X_2^{\gamma_1} \cdots X_{n-1}^{\gamma_{n-1}} }}_{s_k(X_1, \ldots, X_{n-1})} {X_n}^k,
  \end{equation*}
  where \( \gamma \) is a multi-index over the first \( n - 1 \) indeterminates.

  As a polynomial in \( X_n \), \( f \) has \( m \coloneqq (n-1)p \) roots \( s_1, \ldots, s_m \), which are themselves polynomials from \( \BbbF_q[X_1, \ldots, X_{n-1}] \). For some \( c \), we have
  \begin{equation*}
    f(X_1, \ldots, X_n) = c(X_1, \ldots, X_{n-1}) \prod_{j=1}^m (X_n - s_j(X_1, \ldots, X_{n-1}))
  \end{equation*}
  and
  \begin{equation*}
    0 = \Phi(f) = \Phi(c) \cdot \prod_{j=1}^m \parens[\Big]{ \Phi(X_n) - \Phi(s_j) }.
  \end{equation*}

  Since \( \BbbF_q[X_1, \ldots, X_{n-1}] \) is \hyperref[def:entire_semiring]{entire}, we conclude that either \( \Phi(c) \) is the zero function or \( \Phi(X_n) = \Phi(s_j) \) for at least one index \( 1 \leq j \leq m \). The latter is impossible, because \( \Phi(X_n) \) is linearly independent from polynomials in the first \( n - 1 \) variables.

  The inductive hypothesis holds for the polynomial \( c \), and \( \Phi(c) \) being the zero function implies
  \begin{equation*}
    c \in \braket{ X_i^q - X_i \given i = 1, \ldots, n - 1 } \subsetneq I.
  \end{equation*}

  Therefore, \( f \in I \) since \( f \) divides \( c \). We have chosen \( f \) to be an arbitrary member of \( \ker \Phi \), which implies \( \ker \Phi \subseteq I \).

  We have already shown that \( I \subseteq \ker \Phi \). We thus conclude that \( I = \ker \Phi \) and
  \begin{equation*}
    \BbbF_q[X_1, \ldots, X_m] / I \cong \fun(\BbbF_q^m, \BbbF_q).
  \end{equation*}
\end{proof}

\begin{definition}\label{def:transcendental_element}
  We say that the element \( a \in \BbbK \) of the field extension \( \BbbK \) of \( \Bbbk \) is \term{transcendental} over \( \BbbK \) if it is \hyperref[def:algebraic_dependence]{algebraically independent}.

  If \( a \) is not transcendental, we say that it is \term{algebraic}. If every element of \( \BbbK \) is algebraic over \( \Bbbk \), we say that \( \BbbK \) is an \term{algebraic extension} of \( \Bbbk \).
\end{definition}

\begin{proposition}\label{thm:field_is_algebraic_over_itself}
  Every field is an \hyperref[def:transcendental_element]{algebraic extension} of itself.
\end{proposition}
\begin{proof}
  Every element \( a \in \BbbK \) is a root of the polynomial \( X - a \).
\end{proof}

\begin{theorem}[Euler's constant is transcendental]\label{thm:eulers_constant_is_transcendental}
  \hyperref[def:exponential_function]{Euler's constant} \( e \) is \hyperref[def:transcendental_element]{transcendental} over \( \BbbQ \).
\end{theorem}

\begin{theorem}[Pi is transcendental]\label{thm:pi_is_transcendental}\mcite[454]{Knapp2016BasicAlgebra}
  The number \hyperref[def:pi]{\( \pi \)} is \hyperref[def:transcendental_element]{transcendental} over \( \BbbQ \).
\end{theorem}

\begin{example}\label{ex:polynomials_over_pi}
  \Fullref{thm:pi_is_transcendental} implies that the polynomials \( \BbbQ[X] \) can be embedded into \( \BbbR \) via \( \Phi_\pi: \BbbQ[X] \to \BbbR \). We can identify a polynomial
  \begin{equation*}
    p(X) = \sum_{i=0}^n a_k X^k
  \end{equation*}
  with rational coefficients with the number
  \begin{equation*}
    p(\pi) = \sum_{i=0}^n a_k \pi^k.
  \end{equation*}
\end{example}

\begin{definition}\label{def:finite_field_extension}
  If \( \BbbK \) is \hyperref[thm:vector_space_dimension]{finite-dimensional vector space} over \( \Bbbk \), we say that \( \BbbK \) is a \term{finite extensions} of \( \Bbbk \).
\end{definition}

\begin{lemma}\label{thm:finite_field_extensions_are_algebraic}
  Every \hyperref[def:finite_field_extension]{finite field extension} is \hyperref[def:transcendental_element]{algebraic}.
\end{lemma}
\begin{proof}
  Let \( \BbbK \) be a field extension of \( \Bbbk \). Consider the evaluation map \( \Phi_a: \Bbbk[X] \to \Bbbk[u] \) for some \( u \in \BbbK \).

  Since the polynomials \( X^k \) for \( k = 0, 1, 2, \ldots \) form a basis for \( \Bbbk[X] \). If \( \Phi_a \) is injective, then \( \Phi_a(X_k) \) are linearly independent over \( \BbbK \). But \( \BbbK \) has finite dimension over \( \Bbbk \).

  The obtained contradiction shows that \( \Phi_a \) is not injective.
\end{proof}

\begin{definition}\label{def:algebraically_closed_field}\mcite[prop. 9.20]{Knapp2016BasicAlgebra}
  We say that the field \( \BbbK \) is algebraically closed if any of the equivalent conditions are satisfied:
  \begin{thmenum}
    \thmitem{def:algebraically_closed_field/trivial_algebraic_extensions} \( \BbbK \) has no nontrivial \hyperref[def:transcendental_element]{algebraic extensions}.
    \thmitem{def:algebraically_closed_field/linear_irreducible_polynomials} Every irreducible polynomial in \( \BbbK[X] \) is linear.
    \thmitem{def:algebraically_closed_field/at_least_one_root} Every nonconstant polynomial in \( \BbbK[X] \) has at least one root in \( \BbbK \).
    \thmitem{def:algebraically_closed_field/factorization} Every polynomial in \( \BbbK[X] \) \hyperref[def:irreducible_factorization]{factors} into a product of linear polynomials.
    \thmitem{def:algebraically_closed_field/exactly_n_roots} Every polynomial in \( \BbbK[X] \) of degree \( n \) has exactly \( n \) roots in \( \BbbK \), counting the root multiplicities.
  \end{thmenum}
\end{definition}
\begin{proof}
  \ImplicationSubProof{def:algebraically_closed_field/trivial_algebraic_extensions}{def:algebraically_closed_field/linear_irreducible_polynomials} Let \( p(X) \) be an irreducible polynomial in \( \BbbK[X] \).

  Since \( \BbbK[X] \) is a unique factorization domain, it satisfies \fullref{def:unique_factorization_domain/primes_and_ideals}, and hence \( p(X) \) is a prime element. Thus, \( \braket {p(X)} \) is a \hyperref[def:semiring_ideal/prime]{prime ideal} in \( \BbbK[X] \).

  Since \( \BbbK[X] \) is a principal ideal domain, by \fullref{thm:def:principal_ideal_domain/prime_ideal_is_maximal}, \( \braket{ p(X) } \) is also a maximal ideal. By \fullref{thm:quotient_by_maximal_ideal}, the quotient \( Q \coloneqq \BbbK[X] / \braket{ p(X) } \) is a field. The vectors \( 1, X, X^2, \cdots, X^n \) for a basis of \( Q \) over \( \BbbK \), where \( n \) is the degree of \( p(X) \).

  By \fullref{thm:finite_field_extensions_are_algebraic}, \( Q \) is an algebraic extension of \( \BbbK \). Since \( \BbbK \) has no nontrivial algebraic extensions, it follows that \( \BbbK = Q \). Thus, \( Q \) is unidimensional, and we have already discussed that \( \dim Q = \deg p \). Therefore, \( p \) is a linear polynomial.

  \ImplicationSubProof{def:algebraically_closed_field/linear_irreducible_polynomials}{def:algebraically_closed_field/at_least_one_root} Suppose that every irreducible polynomial is linear.

  By \fullref{thm:def:unique_factorization_domain/polynomial_ring}, \( \BbbK[X] \) is a unique factorization domain, and thus there exist irreducible polynomials \( q_1(X), \ldots, q_n(X) \) and a unit \( a \) such that
  \begin{equation*}
    p(X) = a q_1(X) \cdots q_n(X).
  \end{equation*}

  By assumption, the irreducible polynomials are linear, and hence have roots. Therefore, \( p(X) \) has at least one root.

  \ImplicationSubProof{def:algebraically_closed_field/at_least_one_root}{def:algebraically_closed_field/factorization} Suppose that \( u_1 \) is a root of \( p(X) \). Then \( p(X) \) is divisible by \( (X - u_1) \). Using induction on the degree of \( p(X) \), we can factor \( p(X) \) into
  \begin{equation*}
    p(X) = a (X - u_1) (X - u_2) \cdots (X - u_n),
  \end{equation*}
  where \( a \) is a unit of \( \BbbK \). This is the desired factorization.

  \ImplicationSubProof{def:algebraically_closed_field/factorization}{def:algebraically_closed_field/exactly_n_roots} Follows from the equivalence in \fullref{def:polynomial_root} by induction on the polynomial degree. By \fullref{thm:def:integral_domain/root_limit}, the number of roots is bounded by \( n \).

  \ImplicationSubProof{def:algebraically_closed_field/exactly_n_roots}{def:algebraically_closed_field/trivial_algebraic_extensions} Suppose that every nonconstant polynomial of degree \( n \) has exactly \( n \) roots in \( \Bbbk \) and let \( \BbbK \) be an algebraic extension of \( \Bbbk \).

  By \fullref{thm:def:integral_domain/root_limit}, every polynomial in \( \BbbK[X] \) has at most \( n \) roots. By assumption, every root of every polynomial is contained in \( \Bbbk \). Since \( \BbbK \) is algebraic over \( \Bbbk \), it follows that every element of \( \BbbK \) is a root of some polynomial. Therefore, \( \BbbK = \Bbbk \).
\end{proof}

\begin{proposition}\label{thm:no_finite_extensions_of_closed_fields}
  An \hyperref[def:algebraically_closed_field]{algebraically closed field} has no nontrivial finite field extensions.
\end{proposition}
\begin{proof}
  Follows from \fullref{thm:finite_field_extensions_are_algebraic} applied to \fullref{def:algebraically_closed_field/trivial_algebraic_extensions}.
\end{proof}

\begin{theorem}[{Weak \term[en=zero locus theorem]{nullstellensatz}}]\label{thm:weak_nullstellensatz}
  Let \( \mscrK \) be an \hyperref[def:algebraically_closed_field]{algebraically closed field} and let \( \BbbK[X_1, \ldots, X_n] \) be its \hyperref[def:polynomial_algebra]{polynomial ring} in \( n \) indeterminates.

  The ideal \( M \) of \( \BbbK[X_1, \ldots, X_n] \) is \hyperref[def:semiring_ideal/maximal]{maximal} if and only if there exist elements \( u_1, \ldots, u_n \) of \( \BbbK \) such that
  \begin{equation*}
    M = \braket{ X_1 - u_1, \ldots, X_n - u_n }.
  \end{equation*}
\end{theorem}
