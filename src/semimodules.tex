\subsection{Semimodules}\label{subsec:semimodules}

Semimodules are generalizations of monoid actions. Notation and terminology-wise, semimodules are somewhat special in that they are very much influenced by linear algebra and analysis, where vector spaces are crucial.

\begin{definition}\label{def:semimodule}
  Fix a \hyperref[def:semiring]{semiring} \( R \), whose elements we will call \term{scalars}. A \term{left semimodule} over \( R \) is an \hyperref[rem:additive_magma]{additive} \hyperref[def:abelian_group]{abelian group} \( M \), whose elements we will call \term{vectors}, endowed with a heterogeneous binary operation \( \cdot: R \times M \to M \) called \term{scalar multiplication} that satisfies the following conditions:
  \begin{thmenum}
    \thmitem{def:semimodule/scalar_multiplication_action} Scalar multiplication is a \hyperref[def:monoid_action]{monoid action} of the multiplicative monoid \( (R, \cdot_R) \) on \( M \). The following conditions correspond to \eqref{eq:def:monoid_action/family/identity} and \eqref{eq:def:monoid_action/family/compatibility}:
    \begin{align}
      &1 \cdot x = x, \label{eq:def:semimodule/scalar_multiplication_action/identity} \\
      &(r \cdot_R s) \cdot x = r \cdot (s \cdot x). \label{eq:def:semimodule/scalar_multiplication_action/compatibility}
    \end{align}

    The latter can be regarded as a form of associativity.

    \thmitem{def:semimodule/scalar_addition_distributivity} Scalar addition distributes over scalar multiplication:
    \begin{equation}\label{eq:def:semimodule/scalar_addition_distributivity}
      (r +_R s) \cdot x = r \cdot x + s \cdot x.
    \end{equation}

    \thmitem{def:semimodule/vector_addition_distributivity} Vector addition distributes over Scalar multiplication:
    \begin{equation}\label{eq:def:semimodule/vector_addition_distributivity}
      r \cdot (x + y) = r \cdot x + r \cdot y.
    \end{equation}

    \thmitem{def:semimodule/absorption} The scalar and vector zeros are compatible:
    \begin{equation}\label{eq:def:semimodule/absorption}
      0_R \cdot x = 0_M = r \cdot 0_M.
    \end{equation}
  \end{thmenum}

  In practice, we use the same symbol for both scalar and vector addition, and we denote both scalar and vector multiplication via juxtaposition.

  We can analogously define \term{right semimodules} with an operation \( \cdot: M \times R \to M \). This distinction is unnecessary for \hyperref[def:magma/commutative]{commutative} semirings. Without further context, by \enquote{semimodule} we will mean \enquote{left semimodule}, even for non-commutative semirings.

  \begin{thmenum}
    \thmitem{def:semimodule/theory} In order to fit the heterogeneous operation \( \cdot \) into the framework of \hyperref[def:first_order_semantics/satisfiability]{first-order logic models}, we can extend the \hyperref[def:monoid/theory]{theory of monoids} by adding, for every semiring element \( r \), a unary \hyperref[def:first_order_language/func]{functional symbol} \( m_r \). All conditions can then be reformulated via this operation. For example, \eqref{eq:def:semimodule/scalar_multiplication_action/compatibility} corresponds to the axiom schema
    \begin{equation*}
      m_{rs}(\xi) = m_r(m_s(\xi)).
    \end{equation*}

    \thmitem{def:semimodule/homomorphism} A \hyperref[def:first_order_homomorphism]{first-order homomorphism} between two semimodules \( M \) and \( N \) over the same semiring \( R \) is a function \( \varphi: M \to N \) that is a \hyperref[def:monoid/homomorphism]{monoid homomorphism} and satisfies \( \varphi \bincirc m_r^M = m_r^N \bincirc \varphi \).

    This can be expressed more clearly via the following two conditions, which we call \term{additivity} and \term{homogeneity}:
    \begin{align}
      \varphi(x + y) &= \varphi(x) + \varphi(y) \label{def:semimodule/homomorphism/additive} \\
         \varphi(rx) &= r \varphi(x) \label{def:semimodule/homomorphism/homogeneity}
    \end{align}

    Functions satisfying additivity and homogeneity are commonly called \term{linear}. These are a central object of study in \hyperref[sec:linear_algebra]{linear algebra} and, to a lesser extent, (linear) \hyperref[sec:functional_analysis]{functional analysis}.

    \thmitem{def:semimodule/submodel} The set \( S \subseteq M \) is a \hyperref[thm:substructure_is_model]{submodel} of \( S \) if it is a \hyperref[def:monoid/submodel]{submonoid} of \( M \) such that \( m_r(M) \subseteq M \) for every \( r \in R \). We say that \( S \) is closed under addition and scalar multiplication.

    As a consequence of \fullref{thm:positive_formulas_preserved_under_homomorphism}, the \hyperref[def:multi_valued_function/image]{image} of a semimodule homomorphism \( \varphi: M \to N \) is a sub-semimodule of \( S \).

    For an arbitrary set \( S \), we denote the \hyperref[def:first_order_generated_substructure]{generated submodel} by \( \linspan{ S } \) and call it the \term{linear span} of \( S \).

    \thmitem{def:semiring/trivial} The \hyperref[thm:substructures_form_complete_lattice/bottom]{trivial} semimodule is the \hyperref[def:pointed_set/trivial]{trivial pointed set} \( \set{ 0 } \).

    \thmitem{def:semiring/category} For a fixed semiring \( R \), the \hyperref[def:category_of_small_first_order_models]{category of \( \mscrU \)-small models} \( \ucat{SMod}_R \) is \hyperref[def:concrete_category]{concrete} over \hyperref[def:monoid]{\( \ucat{Mon} \)}.
  \end{thmenum}
\end{definition}
