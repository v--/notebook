\subsection{Semimodules}\label{subsec:semimodules}

Semimodules are generalizations of monoid actions. Notation and terminology-wise, semimodules are somewhat special in that they are very much influenced by linear algebra and analysis, where vector spaces are crucial.

\begin{definition}\label{def:endomorphism_semiring}
  Let \( M \) be a \hyperref[def:magma/commutative]{commutative} \hyperref[def:monoid]{monoid} and let \( \End(M) \) be the \hyperref[def:endomorphism_monoid]{monoid of monoid endomorphisms} over \( M \).

  Define addition in \( \End(M) \) pointwise as \( [f + g](x) \coloneqq f(x) + g(x) \). Then \( \End(M) \) with pointwise addition and composition is a \hyperref[def:semiring]{semiring}, which we call the \term{endomorphism semiring} over \( M \).
\end{definition}

\begin{definition}\label{def:semimodule}
  Fix a \hyperref[def:semiring]{semiring} \( R \), whose elements we will call \term{scalars}, and an \hyperref[rem:additive_magma]{additive} \hyperref[def:magma/commutative]{commutative} \hyperref[def:monoid]{monoid} \( M \), whose elements we will call \term{vectors}.

  We say that \( M \) is a \term{semimodule} over \( R \) if they are compatible in any of the equivalent ways listed below. Analogously to \hyperref[def:monoid_action]{monoid actions}, if \( R \) is not commutative, we distinguish between left and right semimodules.

  \begin{thmenum}[series=def:semimodule]
    \thmitem{def:semimodule/action} A left semimodule is a \hyperref[def:semiring/homomorphism]{homomorphism} from \( R \) to the \hyperref[def:endomorphism_semiring]{endomorphism semiring} \( \End(M) \). A right semimodule is a homomorphism from the \hyperref[def:semiring/duality]{dual semiring} \( R^{-1} \) to \( \End(M) \).

    This definition is concise and natural, but unfortunately not very useful.

    \thmitem{def:semimodule/operation} The usual way to define a left semimodule is via a binary operation \( \cdot: R \times M \to M \) called \term{scalar multiplication} that satisfies the following conditions:
    \begin{thmenum}
      \thmitem{def:semimodule/operation/scalar_multiplication_action} Scalar multiplication is a \hyperref[def:monoid_action]{monoid action} of the multiplicative monoid \( (R, \cdot_R) \) on \( M \). The following conditions correspond to \eqref{eq:def:monoid_action/family/identity} and \eqref{eq:def:monoid_action/family/compatibility}:
      \begin{align}
        &1 \cdot x = x, \label{eq:def:semimodule/operation/scalar_multiplication_action/identity} \\
        &(r \cdot_R s) \cdot x = r \cdot (s \cdot x). \label{eq:def:semimodule/operation/scalar_multiplication_action/compatibility}
      \end{align}

      The latter can be regarded as a form of associativity.

      \thmitem{def:semimodule/operation/scalar_addition_distributivity} Scalar addition distributes over scalar multiplication:
      \begin{equation}\label{eq:def:semimodule/operation/scalar_addition_distributivity}
        (r +_R s) \cdot x = r \cdot x + s \cdot x.
      \end{equation}

      \thmitem{def:semimodule/operation/vector_addition_distributivity} Vector addition distributes over Scalar multiplication:
      \begin{equation}\label{eq:def:semimodule/operation/vector_addition_distributivity}
        r \cdot (x + y) = r \cdot x + r \cdot y.
      \end{equation}

      \thmitem{def:semimodule/operation/absorption} The scalar and vector zeros are compatible:
      \begin{equation}\label{eq:def:semimodule/operation/absorption}
        0_R \cdot x = 0_M = r \cdot 0_M.
      \end{equation}
    \end{thmenum}

    In practice, we use the same symbol for both scalar and vector addition, and we denote both scalar and vector multiplication via juxtaposition.
  \end{thmenum}

  Semimodules have the following metamathematical properties:
  \begin{thmenum}[resume=def:semimodule]
    \thmitem{def:semimodule/theory} In order to fit the heterogeneous operation \( \cdot \) into the framework of \hyperref[def:first_order_semantics/satisfiability]{first-order logic models}, we can extend the \hyperref[def:monoid/theory]{theory of monoids} by adding, for every semiring element \( r \), a unary \hyperref[def:first_order_language/func]{functional symbol} \( m_r \). All conditions can then be reformulated via this operation. For example, \eqref{eq:def:semimodule/operation/scalar_multiplication_action/compatibility} corresponds to the axiom schema
    \begin{equation*}
      m_{rs}(\xi) = m_r(m_s(\xi)).
    \end{equation*}

    \thmitem{def:semimodule/homomorphism} A \hyperref[def:first_order_homomorphism]{first-order homomorphism} between two semimodules \( M \) and \( N \) over the same semiring \( R \) is a function \( \varphi: M \to N \) that is a \hyperref[def:monoid/homomorphism]{monoid homomorphism} and satisfies \( \varphi \bincirc m_r^M = m_r^N \bincirc \varphi \).

    This can be expressed more clearly via the following two conditions, which we call \term{additivity} and \term{homogeneity}:
    \begin{align}
      \varphi(x + y) &= \varphi(x) + \varphi(y) \label{def:semimodule/homomorphism/additive} \\
         \varphi(rx) &= r \varphi(x) \label{def:semimodule/homomorphism/homogeneity}
    \end{align}

    Functions satisfying additivity and homogeneity are commonly called \term{linear}. These are a central object of study in \hyperref[sec:linear_algebra]{linear algebra} and, to a lesser extent, (linear) \hyperref[sec:functional_analysis]{functional analysis}.

    \thmitem{def:semimodule/submodel} The set \( A \subseteq M \) is a \hyperref[thm:substructure_is_model]{submodel} of \( A \) if it is a \hyperref[def:monoid/submodel]{submonoid} of \( M \) such that \( rM = m_r[M] \subseteq M \) for every \( r \in R \). We say that \( A \) is closed under addition and scalar multiplication.

    As a consequence of \fullref{thm:positive_formulas_preserved_under_homomorphism}, the \hyperref[def:multi_valued_function/image]{image} of a semimodule homomorphism \( \varphi: M \to N \) is a sub-semimodule of \( A \).

    For an arbitrary set \( A \), we denote the \hyperref[def:first_order_generated_substructure]{generated submodel} by \( \linspan{ A } \) and call it the \term{linear span} of \( A \).

    \thmitem{def:semimodule/trivial} The \hyperref[thm:substructures_form_complete_lattice/bottom]{trivial} semimodule is the \hyperref[def:pointed_set/trivial]{trivial pointed set} \( \set{ 0 } \).

    \thmitem{def:semimodule/category} For a fixed semiring \( R \), the \hyperref[def:category_of_small_first_order_models]{category of \( \mscrU \)-small models} \( \ucat{SMod}_R \) is \hyperref[def:concrete_category]{concrete} over \hyperref[def:monoid]{\( \ucat{Mon} \)}.
  \end{thmenum}
\end{definition}
\begin{proof}
  \ImplicationSubProof{def:semimodule/action}{def:semimodule/operation} Fix a semiring homomorphism \( \varphi: R \to \End(M) \) and define the operation \( r \cdot x \coloneqq \varphi(r)(x) \).

  We will verify that all conditions from \fullref{def:semimodule/operation} hold for this operation.

  \begin{itemize}
    \item By definition, \( \varphi \) is a monoid action of \( (R, \cdot) \) on \( (M, \bincirc) \).

    \item Distributivity of scalar addition holds because \( \varphi \) is a \hyperref[def:magma/homomorphism]{magma homomorphism} from \( (R, +) \) to \( (M, +) \).

    \item Distributivity of vector addition holds because, for each \( r \), \( \varphi(r) \) is a magma endomorphism of \( (M, +) \).

    \item Since \( \varphi \) is a monoid homomorphism from \( (R,  +) \) to \( (R, \cdot) \), it preserves identities and hence
    \begin{equation*}
      0_R \cdot x = \varphi(0_R)(x) = [y \mapsto 0_M](x) = 0_M.
    \end{equation*}

    This proves half of \fullref{def:semimodule/operation/absorption}.

    \item Since, for each \( r \), \( \varphi(r) \) is a monoid endomorphism of \( (M, +) \), we have
    \begin{equation*}
      r \cdot 0_M = \varphi(r)(0_M) = 0_M.
    \end{equation*}

    This proves the other half of \fullref{def:semimodule/operation/absorption}.
  \end{itemize}

  \ImplicationSubProof{def:semimodule/operation}{def:semimodule/action} Let \( \cdot: R \times M \to M \) be an operation satisfying all conditions from \fullref{def:semimodule/operation}. Define the function \( \varphi(r) \coloneqq (x \mapsto r \cdot x) \). We will show that this is a semiring homomorphism.

  It preserves both identities because
  \begin{equation*}
    \varphi(0_R) = (x \mapsto 0) = 0_{\End(M)}
  \end{equation*}
  and
  \begin{equation*}
    \varphi(1_R) = (x \mapsto x) = \id_M.
  \end{equation*}

  We must also show that it preserves both binary operations. Clearly
  \begin{equation*}
    \varphi(r + s)
    =
    (x \mapsto (r + s) x)
    \reloset {\eqref{eq:def:semiring/right_distributivity}} =
    (x \mapsto r x + s x)
    =
    (x \mapsto r x) + (x \mapsto s x)
    =
    \varphi(r) + \varphi(s).
  \end{equation*}

  For multiplication, we have
  \begin{equation*}
    \varphi(rs)
    =
    (x \mapsto (rs)x)
    \reloset {\eqref{eq:def:magma/associative}} =
    (x \mapsto r(sx))
    =
    \parens[\Big]{ x \mapsto \varphi(r)\parens[\Big]{ \varphi(s)(x) } }
    =
    \varphi(r) \bincirc \varphi(s).
  \end{equation*}
\end{proof}

\begin{proposition}\label{thm:semiring_is_semimodule}
  Every semiring is both a left and right semimodule over itself with scalar multiplication given by the semiring multiplication.
\end{proposition}
\begin{proof}
  Fix a semiring \( R \). We will show that \( \cdot \) satisfied the conditions in \fullref{def:semimodule/operation}.
  \begin{itemize}
    \item The identity law \eqref{eq:def:semimodule/operation/scalar_multiplication_action/identity} holds because \( 1 \) is a multiplicative identity of \( M \).
    \item The associativity-like law \eqref{eq:def:semimodule/operation/scalar_multiplication_action/compatibility} follows from associativity of multiplication.
    \item The two distributivity laws \eqref{eq:def:semimodule/operation/scalar_addition_distributivity} and \eqref{eq:def:semimodule/operation/vector_addition_distributivity} follow from left and right distributivity on \( R \).
    \item The absorption law \eqref{eq:def:semimodule/operation/absorption} follows from absorption on semirings.
  \end{itemize}

  All the above also hold for right semimodules rather than left.
\end{proof}

\begin{proposition}\label{thm:commutative_monoid_is_semimodule}
  Every commutative monoid \( M \) is a left semimodule over \( \BbbN \) with scalar multiplication given by \hyperref[rem:additive_magma/multiplication]{recursively defined multiplication}
  \begin{equation}\label{eq:thm:semiring_is_semimodule/operation}
    \begin{aligned}
      &\cdot: \BbbN \times M \to M \\
      &n \cdot x \coloneqq \begin{cases}
        0_M,           &n = 0, \\
        n \cdot x + x, &n > 1.
      \end{cases}
    \end{aligned}
  \end{equation}

  Conversely, in every semimodule over \( \BbbN \), scalar multiplication matches the recursively defined multiplication.

  Compare this result to \fullref{thm:abelian_group_is_semimodule}.
\end{proposition}
\begin{proof}
  \SufficiencySubProof Let \( M \) be a commutative monoid. The operation \( \cdot: \BbbN \times M \to M \) defined in \fullref{thm:semiring_characteristic_homomorphism} satisfies the conditions in \fullref{def:semimodule/operation} as either a direct consequence of the definition or as a consequence of \fullref{thm:monoid_distributivity}.

  \NecessitySubProof Let \( M \) be a semimodule over \( \BbbN \). We will use induction to show that \eqref{eq:thm:semiring_is_semimodule/operation} holds.
  \begin{itemize}
    \item For \( n = 0 \), this follows from the absorption law \eqref{eq:def:semimodule/operation/absorption}.
    \item If \( n \cdot x = n \cdot x + x \), then by scalar distributivity, \( (n + 1) \cdot x = n \cdot x + 1 \cdot x \). The multiplicative identity law \eqref{eq:def:semimodule/operation/scalar_multiplication_action/identity} then shows that \( 1 \cdot x = x \), which concludes our proof.
  \end{itemize}
\end{proof}

\begin{proposition}\label{thm:semimodule_kernels}
  For a \hyperref[def:semimodule/homomorphism]{semimodule homomorphism} \( \varphi: M \to N \), the preimage of \( 0_N \), which is the \hyperref[thm:monoid_kernels]{kernel} of the additive monoid homomorphism induced by \( \varphi \), is a \hyperref[def:semimodule/submodel]{sub-semimodule} of \( M \).

  Consequently, these are the \hyperref[def:zero_morphisms/kernel]{categorical kernels} in \hyperref[def:monoid/category]{\( \cat{SMod}_R \)}.
\end{proposition}
\begin{proof}
  Follows from \fullref{thm:monoid_kernels} and the observation that if \( \varphi(x) = e_H \), then
  \begin{equation*}
    \varphi(r \cdot x)
    =
    r \cdot \varphi(x)
    =
    r \cdot 0_N
    \reloset {\eqref{eq:def:semimodule/operation/absorption}} =
    0_N.
  \end{equation*}
\end{proof}

\begin{definition}\label{def:quotient_semimodule}
  Let \( M \) be a semimodule over \( R \) and let \( N \) be a sub-semimodule of \( M \). We define the \term{quotient semimodule}
  \begin{equation*}
    M / N \coloneqq \set{ x + N \given x \in M }
  \end{equation*}
  with the operations
  \begin{align*}
    (x + N) \oplus (y + N) &\coloneqq x + y + N \\
    r \odot (x + N)  &\coloneqq r \cdot x + N.
  \end{align*}

  We also define the homomorphism
  \begin{align*}
    &\pi: G \to G / N \\
    &\pi(x) \coloneqq x N,
  \end{align*}
  which we call the \term{canonical projection}.
\end{definition}

\begin{definition}\label{def:subtractive_sub_semimodule}\mcite[66]{Golan2010}
  We say that the \hyperref[def:semimodule/submodel]{sub-semimodule} \( N \) of \( M \) is \term{subtractive} if \( x \in N \) and \( x + y \in N \) imply \( y \in N \).

  This definition also applies to more general subsets than sub-semimodules.
\end{definition}

\begin{proposition}\label{thm:semimodule_kernel_of_canonical_projection}
  If \( N \) is a \hyperref[def:subtractive_sub_semimodule]{subtractive sub-semimodule} of \( M \), the kernel of the canonical projection \( \pi: M \to M / N \) is \( N \).

  Compare this result to \fullref{thm:group_kernel_of_canonical_projection}.
\end{proposition}
\begin{proof}
  Obviously \( \pi(N) = N \), so \( N \subseteq \ker(\pi) \).

  To see that the converse holds, suppose that \( \ker(\pi) \setminus N \) is nonempty and let \( x \) be a member of the latter set. Then \( \pi(x) = x + N = N \), but \( x \not\in N \).

  There must exist a member \( y \in N \) such that \( x + y \in N \). The sub-semimodule \( N \) is subtractive, hence \( x \in N \). This contradicts our assumption that \( x \not\in N \). Therefore, \( N = \ker \pi \).
\end{proof}
