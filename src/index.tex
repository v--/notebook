% arara: pdflatex: { shell: true, interaction: nonstopmode }
% arara: biber
% arara: pdflatex: { shell: true }

\documentclass[numbers=endperiod, bibliography=totocnumbered]{scrartcl}

% Base packages
\usepackage[T2A]{fontenc}
\usepackage[utf8]{inputenc}
\usepackage[english]{babel}
\usepackage[pdfencoding=unicode]{hyperref}
\usepackage[citestyle=authoryear]{biblatex}
\usepackage{csquotes}

% Base math packages
\usepackage{amsmath}
\usepackage{amssymb}
\usepackage{amsthm}
\usepackage{mathtools}

% Misc packages
\usepackage{ulem} % Line-breaking underlines
\usepackage{import} % Importing nested files

% Custom packages
\usepackage{common/macros}
\usepackage{common/theorem_styles}

% Bibliography
\addbibresource{references.bib}

% Document
\title{Almanac}
\author{Ianis Vasilev, \Email{ianis@ivasilev.net}}
\date{}

% https://tex.stackexchange.com/questions/171999/overfull-hbox-in-biblatex
\emergencystretch=1em

\begin{document}

\maketitle

\begin{abstract}
  This is an ever-expanding collection of definitions and theorems from various topics in mathematics that I have written in detail for the sake of better understanding them. The notes are quite useful for later reference. Most of the notions were new to me at the time of writing, but sometimes I come back to a fundamental notion and expand upon it with additional commentary. I do not use a strict note-taking process and so the document structure follows a natural evolution.
\end{abstract}

\tableofcontents

\section{Analysis}
\subsection{Asplund spaces}
\import{analysis/asplund/}{dentable_set_definitions.tex}
\import{analysis/asplund/}{weak_dentable_sets_are_dentable.tex}
\import{analysis/asplund/}{asplund_space_definition.tex}

\subsection{Nonsmooth analysis}
\import{analysis/nonsmooth/}{generalized_gradient_definition.tex}
\import{analysis/nonsmooth/}{generalized_gradient_existence.tex}

\printbibliography

\end{document}
