\subsection{Categories}\label{subsec:categories}

Categories are sometimes defined \enquote{from the ground up} so that they may be used without an underlying set theory or logic. For our purposes, it will be more appropriate to define categories via \hyperref[def:graph/quiver]{quivers}, a.k.a. directed multigraphs. This latter approach is appropriate in \hyperref[def:axiom_of_universes]{\logic{ZFC+U}}.

\begin{definition}\label{def:category}\mcite[def. 1.1.1]{Leinster2016Basic}
  A \term{category} is a \hyperref[def:graph/quiver]{quiver} \( \cat{C} \) equipped with a \hyperref[def:partial_function]{partial operation} \( \bincirc \) on the arrows of \( \cat{C} \) and another operation \( \id \) that selects a distinguished arrow for each vertex.

  \begin{thmenum}[series=def:category]
    \thmitem{def:category/objects} We call the vertices of the quiver \term{objects} and denote the set of all objects by \( \obj(\cat{C}) \). We will often write \( A \in \cat{C} \) as a shorthand for \( A \in \obj(\cat{C}) \).

    \thmitem{def:category/morphisms} We call the arrows of the quiver \term{morphisms}. If \( f \) is a morphism, we call its head its \term{domain} \( \dom(f) \) and its tail its \term{codomain} \( \co\dom(f) \). We denote a morphism from \( A \) to \( B \) by \( f: A \to B \) or \( A \reloset f \to B \).

    We call the \hyperref[eq:def:graph/quiver/arc_set]{arc set} \( \cat{C}(A, B) \) of all morphisms from \( A \) to \( B \) a \term{morphism set} or \term{\( \boldop{hom} \)-set}. We use the shorthand \( \cat{C}(A) \) for \( \cat{C}(A, A) \).

    Another established notation is \( \op{hom}(A, B) \) instead of \( \cat{C}(A, B) \).

    \thmitem{def:category/composition} We require the \term{composition} \( \bincirc \) of the arrows \( f \) and \( g \) to be defined only if \( \co\dom(f) = \dom(g) \). In this case, we require \( g \bincirc f \) to be a morphism from \( \dom(f) \) to \( \co\dom(g) \).

    Note how the order of \( f \) and \( g \) may seem confusing: we write the composition of \( f: A \to B \) and \( g: B \to C \) as \( g \bincirc f: A \to C \). This is set up so that it matches \hyperref[def:multi_valued_function/composition]{function composition}.

    This order of composition is used in \cite[7]{MacLane1994}, \cite[def. 1.1.1]{Leinster2016Basic} and \cite[def 3.1.]{Aluffi2009}.

    \thmitem{def:category/identity} We denote the \term{identity morphism} of an object \( A \) by \( \id_A \).
  \end{thmenum}

  The definition of a category additionally requires the following conditions to hold:
  \begin{thmenum}[resume=def:category]
    \thmitem[def:category/C1]{C1} For any morphism \( f: A \to B \), the identities \( \id_A \) and \( \id_B \) must satisfy
    \begin{equation}\label{eq:def:category/C1}\tag{\logic{C1}}
      f \bincirc \id_A = \id_B \bincirc f = f.
    \end{equation}

    \thmitem[def:category/C2]{C2} Composition must be associative. That is, for each triple of morphism \( f \in \cat{C}(A, B) \), \( g \in \cat{C}(B, C) \) and \( h \in \cat{C}(C, D) \), the following must hold:
    \begin{equation}\label{eq:def:category/C2}\tag{\logic{C2}}
      (h \bincirc g) \bincirc f = h \bincirc (g \bincirc f).
    \end{equation}
  \end{thmenum}
\end{definition}

\begin{example}\label{ex:def:category}
  Examples of categories include:

  \begin{itemize}
    \item The category \( \cat{Set} \) of \hyperref[def:large_and_small_sets]{small} \hyperref[def:set]{sets} and \hyperref[def:function]{functions} defined in \fullref{def:category_of_small_sets}.

    \item The category \( \cat{Cat} \) of small categories defined in \fullref{def:category_of_small_categories}.

    \item All the \hyperref[def:category_of_small_first_order_models]{categories of small first-order models} listed in \fullref{ex:def:category_of_small_first_order_models}

    \item The category \( \cat{Top} \) of small \hyperref[def:topological_space]{topological spaces} and \hyperref[def:global_continuity]{continuous functions} defined in \fullref{def:category_of_topological_spaces}.

    \item For every topological space, the fundamental groupoid defined in \fullref{def:fundamental_groupoid}.

    \item The category \( \cat{Quiv} \) of small \hyperref[def:quiver]{quivers} defined in \fullref{def:theory_of_graphs/category}.

    \item For every quiver, the path category defined in \fullref{def:quiver_path_category}.

    \item For every \hyperref[def:preordered_set]{preordered set}, the induced category defined in \fullref{thm:preorder_category_correspondence}.
  \end{itemize}
\end{example}

\begin{definition}\label{def:category_size}
  As can be seen from \fullref{ex:def:category}, some of the categories we are working with, like \( \cat{Set} \), contain as objects all \hyperref[def:large_and_small_sets]{small sets}. As mentioned in \fullref{def:large_and_small_sets}, the concepts of a small set is defined relative to the smallest Grothendieck universe that suits our needs.

  \Fullref{thm:russels_paradox} demonstrates that the set of all sets easily leads to a paradox, which is the reason we restrict our attention only to sets within some Grothendieck universe. This universe is implicit by default, however we will occasionally need to make it explicit.

  We will say that the category \( \cat{C} \) is \term{locally \( \mscrU \)-small} if the morphism set \( \cat{C}(A, B) \) is \( \mscrU \)-small for every pair of objects \( A \) and \( B \). If, in addition, the set \( \obj(\cat{C}) \) of objects is also \( \mscrU \)-small, we will say that the category \( \cat{C} \) is \term{\( \mscrU \)-small}. If a category is not \( \mscrU \)-small, we say that it is \term{\( \mscrU \)-large}.

  Universes are crucial to be able to do a lot of categorical constructions within set theory, for example \hyperref[def:functors]{functors}, \hyperref[def:product_category]{product categories} and thus \hyperref[def:functor_category]{functor categories}.

  Note that, even if a category is \( \mscrU \)-small, the category itself as a tuple \( (\mscrQ, \bincirc, \id) \) (see \fullref{def:category}) may not be a \( \mscrU \)-small set.

  Also note that, in a locally small category, it is possible for the set of all morphisms to be \( \mscrU \)-large. This is impossible for small categories due to \ref{def:grothendieck_universe/union}.

  As usual, we skip the prefix \enquote{\( \mscrU \)-} unless it is important and simply speak of \enquote{large categories} or \enquote{locally small categories}.

  In particular, finite and locally finite categories are ones who are \( V_\omega \)-small and \( V_\omega \)-locally small for the universe of hereditary finite sets \hyperref[def:universe_of_hereditary_finite_sets]{\( V_\omega \)}.

  Note that this notion of local finiteness is unrelated to local finiteness of graphs defined in \fullref{def:graph/degree}.
\end{definition}

\begin{definition}\label{def:category_of_small_sets}
  Suppose that we are given a \hyperref[def:grothendieck_universe]{Grothendieck universe} \( \mscrU \), which is safe to assume to be the smallest suitable one as explained in \fullref{def:large_and_small_sets}.

  We denote the \hyperref[def:category]{category} of \( \mscrU \)-small \hyperref[def:set]{sets} by \( \cat{\mscrU-Set} \) or, if the universe is clear from the context, simply by \( \cat{Set} \).

  \begin{itemize}
    \item The \hyperref[def:category/objects]{set of objects} \( \obj(\cat{Set}) \) is the set of all \( \mscrU \)-small sets, i.e. all members of \( \mscrU \).

    \item The \hyperref[def:category/morphisms]{set of morphisms} \( \cat{Set}(A, B) \) from \( A \) to \( B \) is the set \hyperref[def:function/set_of_functions]{\( \fun(A, B) \)} of all total single-valued functions from \( A \) to \( B \).

    \item The \hyperref[def:category/composition]{composition of morphisms} is the usual \hyperref[def:multi_valued_function/composition]{function composition}.

    \item The \hyperref[def:category/identity]{identity morphism} of the set \( A \) is the \hyperref[def:multi_valued_function/identity]{identity function}
    \begin{equation*}
      \begin{aligned}
        &\id_A: A \to A \\
        &\id_A(x) \coloneqq x.
      \end{aligned}
    \end{equation*}
  \end{itemize}
\end{definition}
\begin{defproof}
  To see that \( \cat{\mscrU-Set} \) is indeed a category, we verify the conditions \ref{def:category/C1} and \ref{def:category/C2}.

  \SubProofOf{def:category/C1} For every two sets \( A, B \in \mscrU \) and every function \( f: A \to B \), for all \( x \in A \) we have
  \begin{equation*}
    [\id_B \bincirc f](x)
    =
    \id_B(f(x))
    =
    f(x)
    =
    f(\id_A(x))
    =
    [f \bincirc \id_A](x).
  \end{equation*}

  Therefore, \( \id_A \) and \( \id_B \) satisfy \eqref{eq:def:category/C1}.

  \SubProofOf{def:category/C2} Associativity of function composition is proved in \fullref{thm:multivalued_function_properties/associative}.
\end{defproof}

\begin{proposition}\label{thm:category_of_small_sets_properites}
  We collect here important properties of the category \hyperref[def:category_of_small_sets]{\( \cat{\mscrU-Set} \)} of \( \mscrU \)-small sets. Most of them require forward references.

  \begin{thmenum}
    \thmitem{thm:category_of_small_sets_properites/large} It is a \( \mscrU \)-large category in the sense of \fullref{def:category_size} because \( \mscrU \) itself is the set of objects and, defined as a \hyperref[def:graph/quiver]{quiver} with additional operations, the category is a \( \mscrU \)-large set in the sense of \fullref{def:large_and_small_sets}.

    \thmitem{thm:category_of_small_sets_properites/locally_small} It is a \hyperref[def:category_size]{\( \mscrU \)-locally small category} because \( \mscrU \) is a model of \hyperref[def:zfc]{\( \logic{ZFC} \)} and \fullref{thm:zfc_existence_theorems/set_of_functions} holds.

    \thmitem{thm:category_of_small_sets_properites/morphism_invertibility} All \hyperref[def:morphism_invertibility/right_cancellative]{epimorphisms} and \hyperref[def:multi_valued_function/empty]{nonempty} \hyperref[def:morphism_invertibility/left_cancellative]{monomorphisms} \hyperref[def:morphism_invertibility/left_invertible]{split} and are precisely the \hyperref[def:function_invertibility/surjective]{surjective} and nonempty \hyperref[def:function_invertibility/injective]{injective functions}, respectively.

    This is stated in \fullref{thm:function_invertibility_categorical}. See also \fullref{thm:epimorphisms_split_in_set}.

    \thmitem{thm:category_of_small_sets_properites/zero_objects} The empty set \( \varnothing \) is an \hyperref[def:zero_objects/initial]{initial object} and the singleton set \( \set{ A } \) is a \hyperref[def:zero_objects/terminal]{terminal object} for every \( A \in \cat{\mscrU-Set} \). No \hyperref[def:zero_objects/zero]{zero objects} exist in \( \cat{\mscrU-Set} \) by \fullref{thm:zero_object_properties/no_zero}, \( \cat{\mscrU-Set} \).

    This is discussed in \fullref{ex:def:zero_objects}.
  \end{thmenum}
\end{proposition}

\begin{definition}\label{def:morphism_invertibility}
  In connection with \fullref{def:function_invertibility} and \fullref{def:first_order_homomorphism_invertibility}, we introduce the following terminology:
  \begin{thmenum}
    \thmitem{def:morphism_invertibility/left_cancellative} The morphism \( g: B \to C \) is \term{left-cancellative} if, for any pair of morphisms \( f_1, f_2: A \to B \), the equality \( g \bincirc f_1 = g \bincirc f_2 \) implies \( f_1 = f_2 \).

    Left-cancellative morphisms are also called \term{monic morphisms} or \term{monomorphisms}.

    A conventional notation for monomorphisms is \( g: B \hookrightarrow C \).

    \thmitem{def:morphism_invertibility/left_invertible} The morphism \( f: A \to B \) is \term{left-invertible} if there exists a morphism \( g: B \to A \) such that \( g \bincirc f = \id_A \). We call \( g \) a \term{left inverse} of \( f \).

    Left-invertible morphisms are sometimes called \term{split monomorphisms} because they \enquote{split} the identity \( \id_A \) into a composition of \( f \) and \( g \).

    \thmitem{def:morphism_invertibility/right_cancellative} The morphism \( f: A \to B \) is \term{right-cancellative} if, for any pair of morphisms \( g_1, g_2: B \to C \), the equality \( g \bincirc f_1 = g \bincirc f_2 \) implies \( g_1 = g_2 \).

    Right-cancellative morphisms are also called \term{epic morphisms} or \term{epimorphisms}.

    A conventional notation for epimorphisms is \( f: A \twoheadrightarrow B \).

    \thmitem{def:morphism_invertibility/right_invertible} The morphism \( g: A \to B \) is \term{right-invertible} if there exists a morphism \( g: B \to A \) such that \( f \bincirc g = \id_B \). We call \( g \) a \term{right inverse} of \( f \).

    Right-invertible morphisms are sometimes called \term{split epimorphisms} because they \enquote{split} the identity \( \id_B \) into a composition of \( g \) and \( f \).

    \thmitem{def:morphism_invertibility/isomorphism} The morphism \( f: A \to B \) is \term{fully invertible} it is both left-invertible and right-invertible. By \fullref{thm:morphism_invertibility_properties/left_and_right}, in this case, there exists a unique morphism \( f^{-1}: B \to A \) that is a \term{two-sided inverse}, i.e. it is both a left inverse and a right inverse.

    A fully invertible morphism is usually called an \term{isomorphism}. If there exists an isomorphism between \( A \) and \( B \), we say that they are \term{isomorphic}.

    \thmitem{def:morphism_invertibility/endomorphism} A morphism \( f: A \to A \) from an object to itself is called an \term{endomorphism}.

    \thmitem{def:morphism_invertibility/automorphism} A morphism that is both an endomorphism and an isomorphism is called an \term{automorphism}.
  \end{thmenum}
\end{definition}

\begin{example}\label{ex:def:morphism_invertibility}
  \Fullref{thm:function_invertibility_categorical} characterizes the cancellative and invertible morphisms defined in \fullref{def:morphism_invertibility} for \hyperref[def:category_of_small_sets]{\( \cat{Set} \)} in terms of \hyperref[def:function_invertibility/injective]{injectivity} and \hyperref[def:function_invertibility/injective]{surjectivity}.

  A very simple example of a monomorphism which does not split is the empty function with nonempty domain. These are discussed in \fullref{thm:function_invertibility_categorical/empty}.

  \Fullref{thm:surjective_functions_are_right_invertible} is important enough to have a categorical interpretation via \fullref{thm:epimorphisms_split_in_set}, where its relation to the \hyperref[def:zfc/choice]{axiom of choice} is also discussed.
\end{example}

\begin{proposition}\label{thm:morphism_invertibility_properties}
  Morphisms have the following basic properties regarding their \hyperref[def:morphism_invertibility]{invertibility}:

  \begin{thmenum}
    \thmitem{thm:morphism_invertibility_properties/split_monomorphism} Any \hyperref[def:morphism_invertibility/left_invertible]{left-invertible morphism} is \hyperref[def:morphism_invertibility/left_cancellative]{left-cancellative}.

    In more categorical terms, every split monomorphism is a monomorphism.

    \thmitem{thm:morphism_invertibility_properties/split_epimorphism} Any \hyperref[def:morphism_invertibility/right_invertible]{right-invertible morphism} is \hyperref[def:morphism_invertibility/right_cancellative]{right-cancellative}.

    In more categorical terms, every split epimorphism is an epimorphism.

    \thmitem{thm:morphism_invertibility_properties/at_most_one_inverse}\mcite[exer. 1.1.13]{Leinster2016Basic} Any morphism has at most one two-sided inverse.

    \thmitem{thm:morphism_invertibility_properties/left_and_right} If a morphism is both left-invertible and right-invertible, the two inverses are equal and the morphism is fully invertible.

    \thmitem{thm:morphism_invertibility_properties/cancellative_composition} The composition of two monomorphisms (resp. epimorphisms) is again a monomorphism (resp. epimorphism).

    \thmitem{thm:morphism_invertibility_properties/invertible_composition} The composition of two split monomorphisms (resp. epimorphisms) is again a split monomorphism (resp. epimorphism).

    Compare this result to \fullref{thm:function_composition_invertibility}.
  \end{thmenum}
\end{proposition}
\begin{proof}
  \SubProofOf{thm:morphism_invertibility_properties/split_monomorphism} Suppose that \( g: B \to C \) is left-invertible with inverse \( h: C \to B \). Suppose that \( f_1, f_2: A \to B \) are morphisms such that
  \begin{equation*}
    g \bincirc f_1 = g \bincirc f_2.
  \end{equation*}

  Then
  \begin{equation*}
    f_1
    \reloset {\eqref{eq:def:category/C1}} =
    \id_B \bincirc f_1
    =
    (h \bincirc g) \bincirc f_1
    \reloset {\eqref{eq:def:category/C2}} =
    h \bincirc (g \bincirc f_1)
    =
    h \bincirc (g \bincirc f_2)
    =
    \cdots
    =
    f_2.
  \end{equation*}

  \SubProofOf{thm:morphism_invertibility_properties/split_epimorphism} The proof is analogous to \fullref{thm:morphism_invertibility_properties/split_monomorphism}.

  \SubProofOf{thm:morphism_invertibility_properties/at_most_one_inverse} If \( f: A \to B \) has no inverse, it vacuously has at most one inverse.

  Now assume that \( f: A \to B \) has two inverses \( g_1: B \to A \) and \( g_2: B \to A \):
  \begin{align*}
    g_1 \bincirc f = \id_A &&& f \bincirc g_1 = \id_B, \\
    g_2 \bincirc f = \id_A &&& f \bincirc g_2 = \id_B.
  \end{align*}

  Then
  \begin{equation*}
    g_1
    \reloset {\eqref{eq:def:category/C1}} =
    g_1 \bincirc \id_B
    =
    g_1 \bincirc (f \bincirc g_2)
    \reloset {\eqref{eq:def:category/C2}} =
    (g_1 \bincirc f) \bincirc g_2
    =
    \id_A \bincirc g_2
    \reloset {\eqref{eq:def:category/C1}} =
    g_2.
  \end{equation*}

  \SubProofOf{thm:morphism_invertibility_properties/left_and_right} Suppose that \( f: A \to B \) has a left-inverse \( l: B \to A \) and a right-inverse \( r: B \to A \). Then
  \begin{equation*}
    r
    \reloset {\eqref{eq:def:category/C1}} =
    \id_B \bincirc r
    =
    (l \bincirc f) \bincirc r
    \reloset {\eqref{eq:def:category/C2}} =
    l \bincirc (f \bincirc r)
    =
    l \bincirc \id_A
    \reloset {\eqref{eq:def:category/C1}} =
    l.
  \end{equation*}

  \SubProofOf{thm:morphism_invertibility_properties/cancellative_composition} Let \( g: B \to C \) and \( h: C \to D \) be monomorphisms (left-cancellative).

  Let \( f_1, f_2: A \to B \) be two arbitrary morphisms with codomain \( B \). Suppose that
  \begin{equation*}
    (h \bincirc g) \bincirc f_1 = (h \bincirc g) \bincirc f_2.
  \end{equation*}

  Then, by \ref{def:category/C2},
  \begin{equation*}
    h \bincirc (g \bincirc f_1) = h \bincirc (g \bincirc f_2).
  \end{equation*}

  Since \( h \) is left-cancellative, it follows that
  \begin{equation*}
    g \bincirc f_1 = g \bincirc f_2.
  \end{equation*}

  Since \( g \) is also left-cancellative, \( f_1 = f_2 \).

  Therefore, \( h \bincirc g \) is a monomorphism.

  The proof for composition of epimorphisms is identical.

  \SubProofOf{thm:morphism_invertibility_properties/invertible_composition} Let \( f: A \to B \) and \( g: B \to C \) be split monomorphisms (left-invertible).

  Then there exist left inverses \( l_f: B \to A \) and \( l_g: C \to B \) of \( f \) and \( g \), respectfully. We have
  \begin{equation*}
    (l_f \bincirc l_g) \bincirc (g \bincirc f)
    =
    \reloset {\eqref{eq:def:category/C2}} =
    l_f \bincirc (l_g \bincirc g) \bincirc f
    =
    l_f \bincirc \id_B \bincirc f
    \reloset {\eqref{eq:def:category/C1}} =
    l_f \bincirc f
    =
    \id_A.
  \end{equation*}

  Therefore, \( g \bincirc f \) is also left-invertible.

  The proof for composition of split epimorphisms is identical.
\end{proof}

\begin{theorem}[Epimorphisms split in Set]\label{thm:epimorphisms_split_in_set}
  Every \hyperref[def:morphism_invertibility/right_cancellative]{epimorphism} in \hyperref[def:category_of_small_sets]{\( \cat{Set} \)} splits. That is, all epimorphisms in \( \cat{Set} \) are \hyperref[def:morphism_invertibility/right_invertible]{split epimorphisms}.

  Assuming the existence of the \hyperref[def:grothendieck_universe]{Grothendieck universe} containing \( \cat{Set} \), in \hyperref[def:zfc]{\logic{ZF}} this theorem is equivalent to the \hyperref[def:zfc/choice]{axiom of choice} --- see \fullref{thm:axiom_of_choice_equivalences/selection}.

  Since not every epimorphism splits in a general category, this theorem is sometimes considered to be a categorical statement of the axiom of choice, which holds in some categories but not in others.
\end{theorem}
\begin{proof}
  By \fullref{thm:function_invertibility_categorical/right_cancellative}, a function is an epimorphism if and only if it is surjective. Thus, the theorem is equivalent to \fullref{thm:surjective_functions_are_right_invertible}.
\end{proof}

\begin{definition}\label{def:zero_objects}\mcite[def. 2.1.7]{Leinster2016Basic}
  Fix a category \( \cat{C} \).

  \begin{thmenum}
    \thmitem{def:zero_objects/initial} We call the object \( A \in \cat{C} \) an \term{initial object} if for any other object \( B \in \cat{C} \) there exists a unique morphism \( f: A \to B \),

    \thmitem{def:zero_objects/final} Analogously, we call the object \( B \in \cat{C} \) a \term{terminal object} or \term{final object} if for any other object \( A \in \cat{C} \) there exists a unique morphism \( f: A \to B \).

    \thmitem{def:zero_objects/zero}\mcite{nLab:pointed_category} If \( A \) is both an initial and a terminal object, we say that \( A \) is a \term{zero object}. A category with a distinguished zero object is called a \term{pointed category}.
  \end{thmenum}
\end{definition}

\begin{example}\label{ex:def:zero_objects}
  In the category \hyperref[def:category_of_small_sets]{\( \cat{Set} \)} of small sets, for any set \( A \) there is a unique \hyperref[def:multi_valued_function/empty]{empty function} from \( \varnothing \) to \( A \). Therefore, \( \varnothing \) is an \hyperref[def:zero_objects/initial]{initial object} in \( \cat{Set} \).

  For any set \( A \), there is a unique function that contracts any set \( B \) to \( \set{ A } \). Therefore, every singleton set is a \hyperref[def:zero_objects/final]{final object} in \( \cat{Set} \).

  By \fullref{thm:zero_object_properties/no_zero}, \( \cat{Set} \) has no zero object.

  In the category \hyperref[def:group/category]{\( \cat{Grp} \)} of small groups, the \hyperref[def:group/trivial]{trivial group} is a \hyperref[def:zero_objects/zero]{zero object}. Indeed, it can be embedded into any other group and any group can be contracted into the corresponding trivial group. Furthermore, all trivial groups are isomorphic.
\end{example}

\begin{proposition}\label{thm:zero_object_properties}
  \hfill
  \begin{thmenum}
    \thmitem{thm:zero_object_properties/initial} An \hyperref[def:zero_objects/initial]{initial object} is unique up to an isomorphism.
    \thmitem{thm:zero_object_properties/final} A \hyperref[def:zero_objects/initial]{terminal object} is also unique up to an isomorphism.
    \thmitem{thm:zero_object_properties/zero} If a category has an initial and a terminal object and if they are isomorphic, then both are zero objects.

    In particular, a zero object is unique up to an isomorphism.

    \thmitem{thm:zero_object_properties/no_zero} If an initial and a terminal object exists and are not isomorphic, then there exist no zero objects.
  \end{thmenum}
\end{proposition}
\begin{proof}
  \SubProofOf{thm:zero_object_properties/initial} Suppose that \( A \) and \( B \) are both initial objects in \( \cat{C} \). Then there exist morphisms \( f: A \to B \) and \( g: B \to A \). Their composition \( g \bincirc f \) is an \hyperref[def:morphism_invertibility/endomorphism]{endomorphism} on \( A \).

  But there exists a unique \hyperref[def:morphism_invertibility/endomorphism]{endomorphism} on \( A \), which must be the identity \( \id_A \). Thus, \( g \bincirc f = \id_A \) and \( g \) is a left inverse of \( f \).

  We can analogously show that \( g \) is a right inverse of \( f \). Therefore, \( f \) is fully invertible, and \( A \) and \( B \) are isomorphic.

  \SubProofOf{thm:zero_object_properties/final} The proof is analogous to \fullref{thm:zero_object_properties/initial}.

  \SubProofOf{thm:zero_object_properties/zero} Suppose that \( A \) is an initial object and that \( B \) is a final object in \( \cat{C} \). Let \( f: A \to B \) be an isomorphism between them.

  Let \( C \in \cat{C} \) be any other object and let \( g: C \to B \) be the unique morphism to \( B \). Then \( f^{-1} \bincirc g: C \to A \) is a morphism from \( C \) to \( A \). The inverse \( f^{-1}: B \to A \) is unique by \fullref{thm:morphism_invertibility_properties/at_most_one_inverse}, therefore its composition with \( g: C \to B \) is also unique. Hence any object has a unique morphism to \( A \). This makes \( A \) a terminal object and thus a zero object.

  We can analogously show that \( B \) is a zero object.

  \SubProofOf{thm:zero_object_properties/no_zero} By \fullref{thm:zero_object_properties/zero}, all zero objects are isomorphic. By \fullref{thm:zero_object_properties/initial}, all initial objects are isomorphic and analogously for terminal objects. Hence, if a zero object exists, all initial objects are isomorphic to all terminal objects.

  If some initial object is not isomorphic to some terminal object, then by contraposition it follows that no zero object exists.
\end{proof}

\begin{definition}\label{def:dual_category}\mcite[def. 1.1.9]{Leinster2016Basic}
  The \term{dual category} of \( \cat{C} \), also called the \term{opposite category}, is obtained from \( \cat{C} \) by \enquote{reversing} all arrows.

  Formally, the category \( \cat{C}^{-1} \) is defined as follows:
  \begin{itemize}
    \item The \hyperref[def:category/objects]{set of objects} \( \obj(\cat{C}^{-1}) \) is the set of objects \( \obj(\cat{C}) \) of \( \cat{C} \).

    \item The \hyperref[def:category/morphisms]{set of morphisms} \( \cat{C}^{-1}(A, B) \) is the set \( \cat{C}(B, A) \). Thus, any morphism \( f_{\cat{C}^{-1}}: A \to B \) in the dual category \( \cat{C}^{-1} \) is a morphism \( f_{\cat{C}}: B \to A \) in \( \cat{C} \).

    The lower index here is used solely to distinguish between \( f \) being regarded as a morphism of \( \cat{C} \) and of \( \cat{C}^{-1} \).

    \item The \hyperref[def:category/composition]{composition of the morphisms}
    \begin{align*}
      f_{\cat{C}^{-1}} &\in \cat{C}^{-1}(A, B) = \cat{C}(B, A) \\
      g_{\cat{C}^{-1}} &\in \cat{C}^{-1}(B, C) = \cat{C}(C, B)
    \end{align*}
    is the morphism
    \begin{equation*}
      g_{\cat{C}^{-1}} \bincirc f_{\cat{C}^{-1}} \coloneqq f_{\cat{C}} \bincirc g_{\cat{C}} \in \cat{C}(C, A) = \cat{C}^{-1}(A, C).
    \end{equation*}

    \item The \hyperref[def:category/identity]{identity morphism} of the object \( A \in \cat{C} \) is again \( \id_A \).
  \end{itemize}
\end{definition}

\begin{example}\label{ex:def:dual_category}
  A morphism \( f: A \to B \) in the category \( \cat{Set}^{-1} \) is a function from the set \( B \) to the set \( A \). Note that this is not a function in general.
\end{example}

\begin{definition}\label{def:subcategory}\mcite[def. 1.2.18]{Leinster2016Basic}
  We call the category \( \cat{D} \) a \term{subcategory} of \( \cat{C} \) if the following hold:
  \begin{itemize}
    \item Every object in \( \cat{D} \) is an object in \( \cat{C} \).
    \item Every morphism in \( \cat{D} \) is a morphism in \( \cat{D} \).
    \item Composition in \( \cat{D} \) is a \hyperref[def:multi_valued_function/restriction]{restriction} of composition in \( \cat{C} \).
    \item The identity morphisms in \( \cat{D} \) are the same as those of \( \cat{C} \).
  \end{itemize}

  In case \( \cat{D}(A, B) = \cat{C}(A, B) \) for every pair of objects \( A \) and \( B \) of \( \cat{D} \), we say that \( \cat{D} \) is a \term{full subcategory} of \( \cat{C} \).

  The full subcategories of \( \cat{C} \) are precisely the \hyperref[eq:def:theory_of_graphs/submodel/full]{full subquivers} of \( \cat{C} \).
\end{definition}

\begin{definition}\label{def:skeletal_category}\mcite[91]{MacLane1994}
  The category \( \cat{C} \) is called \term{skeletal} if the only isomorphisms in \( \cat{C} \) are equalities.

  A skeletal \hyperref[def:subcategory]{subcategory} is called a \term{skeleton}.
\end{definition}

\begin{theorem}\label{thm:skeletal_subcategory_existence}
  Every \hyperref[def:category]{category} has a \hyperref[def:skeletal_category]{skeletal subcategory}.
\end{theorem}
\begin{proof}
  Fix category \( \cat{C} \). Let \( \mscrM \) be the set of all morphisms of \( \cat{C} \). Define the function
  \begin{equation*}
    \begin{aligned}
      &I: \obj(\cat{C})^2 \to \mscrM \\
      &I(A, B) \coloneqq \set{ f \in \cat{C}(A, B) \given f \T{is an isomorphism} }.
    \end{aligned}
  \end{equation*}

  Let \( D \subseteq \obj(\cat{C})^2 \) be the set of pairs of objects \( A \) and \( B \) such that \( I(A, B) \) is nonempty.

  Finally, use the \hyperref[def:zfc/choice]{axiom of choice} to obtain a \hyperref[def:choice_function]{choice function} \( c \) for the family of nonempty sets
  \begin{equation*}
    \seq{ I(A, B) }_{A, B \in D}.
  \end{equation*}

  Define the subcategory \( \cat{S} \) with the same objects and with the following morphism sets:
  \begin{equation*}
    \cat{S}(A, B) \coloneqq \begin{cases}
      \cat{C}(A, B),                                     &I(A, B) = \varnothing \\
      [\cat{C}(A, B) \setminus I(A, B)] \cup c(I(A, B)), &I(A, B) \neq \varnothing \\
    \end{cases}
  \end{equation*}
\end{proof}
