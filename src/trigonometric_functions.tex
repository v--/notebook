\subsection{Trigonometric functions}\label{subsec:trigonometric_functions}

\begin{definition}\label{def:trigonometric_functions}
  We define the two basic \term{trigonometric functions}. They are also called \term{circular trigonometric functions} to distinguish them from the hyperbolic trigonometric functions defined and motivated in \fullref{def:hyperbolic_trigonometric_functions}.

  \begin{thmenum}
    \ilabel{def:trigonometric_functions/sine} The \term{sine} function, also called the \term{sinus} function, is
    \begin{equation*}
      \sin(z)
      \coloneqq
      -i \sum_{m \text{ is odd}}^\infty \frac {i^m z^m} {m!}
      =
      -i \sum_{k=0}^\infty \frac {i^{2k+1} z^{2k+1}} {(2k + 1)!}
      =
      \sum_{k=0}^\infty \frac {i^{2k} z^{2k+1}} {(2k + 1)!}
    \end{equation*}

    \ilabel{def:trigonometric_functions/cosine} The \term{cosine} function, also called the \term{cosinus} function, is
    \begin{equation*}
      \cos(z)
      \coloneqq
      \sum_{m \text{ is even}}^\infty \frac {i^m z^m} {m!}
      =
      \sum_{k=0}^\infty \frac {i^{2k} z^{2k}} {(2k)!}.
    \end{equation*}
  \end{thmenum}

  \Fullref{def:geometric_trigonometric_functions} justifies the term \enquote{angle} for the \hyperref[def:function/argument]{parameter} of the trigonometric functions.
\end{definition}
\begin{proposition}\label{thm:trigonometric_function_properties}
  The \hyperref[def:trigonometric_functions]{main trigonometric functions} have the following basic properties:
  \begin{thmenum}
    \ilabel{thm:trigonometric_function_properties/convergence} Both \( \sin(z) \) and \( \cos(z) \) converge in the entire complex plane.
    \ilabel{thm:trigonometric_function_properties/parity} \( \sin(z) \) is an odd function and \( \cos(z) \) is an even function.
    \ilabel{thm:trigonometric_function_properties/derivative} \( \sin'(z) = \cos(z) \) and \( \cos'(z) = -\sin(z) \) for all \( z \in \BbbC \).
  \end{thmenum}
\end{proposition}
\begin{proof}
  \SubProofOf{thm:trigonometric_function_properties/convergence} Note that the zero coefficients in the expansion of either \( \sin \) or \( \cos \) do not alter convergence. Therefore, by \fullref{thm:power_series_radius_of_convergence}, the radius of convergence is
  \begin{equation*}
    \limsup_{k \to \infty} \frac {\abs{i^{k-1} k!}} {\abs{i^k (k-1)!}}
    =
    \limsup_{k \to \infty} k
    =
    +\infty.
  \end{equation*}

  \SubProofOf{thm:trigonometric_function_properties/parity} Follows from \fullref{thm:power_series_parity}.

  \SubProofOf{thm:trigonometric_function_properties/derivative} Follows from \fullref{thm:power_series_are_locally_uniform_convergent} and \fullref{thm:derivative_limit_exchange}.
\end{proof}

\begin{proposition}\label{thm:trigonometric_identities}
  We have the following basic trigonometric identities:
  \begin{thmenum}
    \ilabel{thm:trigonometric_identities/pythagorean_identity} (Pythagorean identity) For any \( z \in \BbbC \),
    \begin{equation}\label{eq:thm:trigonometric_identities/pythagorean_identity}
      \sin(z)^2 + \cos(z)^2 = 1.
    \end{equation}

    \ilabel{thm:trigonometric_identities/products} (Products) For \( x, y \in \BbbC \),
    \begin{balign}
      2 \sin(x) \sin(y) & = \cos(x - y) - \cos(x + y) \label{eq:thm:trigonometric_identities/products/ss}  \\
      2 \cos(x) \cos(y) & = \cos(x - y) + \cos(x + y) \label{eq:thm:trigonometric_identities/products/cc}  \\
      2 \sin(x) \cos(y) & = \sin(x - y) + \sin(x + y) \label{eq:thm:trigonometric_identities/products/sc}  \\
      2 \cos(x) \sin(y) & = -\sin(x - y) + \sin(x + y) \label{eq:thm:trigonometric_identities/products/cs}
    \end{balign}

    \ilabel{thm:trigonometric_identities/sums} (Sums) For \( x, y \in \BbbC \),
    \begin{balign}
      \sin(x) + \sin(y) & = 2 \cos\left(\frac{x - y} 2 \right) \sin\left(\frac{x + y} 2 \right) \label{eq:thm:trigonometric_identities/sums/sin_sum}   \\
      \sin(x) - \sin(y) & = 2 \sin\left(\frac{x - y} 2 \right) \cos\left(\frac{x + y} 2 \right) \label{eq:thm:trigonometric_identities/sums/sin_diff}  \\
      \cos(x) + \cos(y) & = 2 \cos\left(\frac{x - y} 2 \right) \cos\left(\frac{x + y} 2 \right) \label{eq:thm:trigonometric_identities/sums/cos_sum}   \\
      \cos(x) - \cos(y) & = -2 \sin\left(\frac{x - y} 2 \right) \sin\left(\frac{x + y} 2 \right) \label{eq:thm:trigonometric_identities/sums/cos_diff}
    \end{balign}

    \ilabel{thm:trigonometric_identities/sum_of_angles} (Sum of angles) For \( x, y \in \BbbC \),
    \begin{balign}
      \sin(x + y) & = \cos(x) \sin(y) + \cos(x) \sin(y) \label{eq:thm:trigonometric_identities/sum_of_angles/sin} \\
      \cos(x + y) & = \cos(x) \cos(y) - \sin(x) \sin(y) \label{eq:thm:trigonometric_identities/sum_of_angles/cos}
    \end{balign}
  \end{thmenum}
\end{proposition}
\begin{proof}
  We first use \hyperref[def:algebra_of_polynomials/polynomial_multiplication]{Cauchy multiplication} for the power series \( \cos(v) \) and \( \cos(w) \):
  \begin{balign}
    \cos(v) \cos(w)
    &=
    \left( \sum_{k=0}^\infty \frac {i^{2k} v^{2k}} {(2k)!} \right) \Ast \left( \sum_{k=0}^\infty \frac {i^{2k} w^{2k}} {(2k)!} \right)
    = \nonumber \\ &=
    \sum_{k=0}^\infty \sum_{m=0}^k \frac {i^{2m} v^{2m}} {(2m)!} \frac {i^{2(k-m)} w^{2(k-m)}} {(2(k-m))!}
    = \nonumber \\ &=
    \sum_{k=0}^\infty \frac {i^{2k}} {(2k)!} \sum_{m=0}^k \binom {2k} {2m} v^{2m} w^{2(k-m)}. \label{eq:thm:trigonometric_identities/cos_product}
  \end{balign}

  Analogously,
  \begin{balign}
    \sin(v) \sin(w)
    &=
    (-i) (-i) \left( \sum_{k=0}^\infty \frac {i^{2k+1} v^{2k+1}} {(2k+1)!} \right) \Ast \left( \sum_{k=0}^\infty \frac {i^{2k+1} w^{2k+1}} {(2k+1)!} \right)
    = \nonumber \\ &=
    -\sum_{k=0}^\infty \sum_{m=0}^k \frac {i^{2m+1} v^{2m+1}} {(2m+1)!} \frac {i^{2(k-m)+1} w^{2(k-m)+1}} {(2(k-m)+1)!}
    = \nonumber \\ &=
    -\sum_{k=0}^\infty \frac {i^{2(k+1)}} {(2(k+1))!} \sum_{m=0}^k \binom {2(k+1)} {2m+1} v^{2m+1} w^{2(k-m)+1}
    = \nonumber \\ &=
    -\sum_{k=1}^\infty \frac {i^{2k}} {(2k)!} \sum_{m=0}^{k-1} \binom {2k} {2m+1} v^{2m+1} w^{2k-(2m+1)}. \label{eq:thm:trigonometric_identities/sin_product}
  \end{balign}

  \SubProofOf{thm:trigonometric_identities/pythagorean_identity} From \eqref{eq:thm:trigonometric_identities/cos_product} and \eqref{eq:thm:trigonometric_identities/sin_product} we have
  \begin{equation*}
    \sin(z)^2 + \cos(z)^2
    =
    1 + \sum_{k=1}^\infty \frac {i^{2k} z^{2k}} {(2k)!} \underbrace{\left[-\sum_{m=0}^{k-1} \binom {2k} {2m+1} + \sum_{m=0}^k \binom {2k} {2m} \right]}_{\eqqcolon a_k}.
  \end{equation*}

  It remains to show that the expression \( a_k \) equals zero for all \( k = 1, 2, \ldots \). We have
  \begin{equation*}
    a_k
    =
    \sum_{m=0}^k \binom {2k} {2m} - \sum_{m=0}^{k-1} \binom {2k} {2m+1}
    =
    \sum_{m=0}^k (-1)^m \binom {2k} m
    \overset {\ref{thm:binomial_theorem}} =
    1 - 1 = 0.
  \end{equation*}

  \Fullref{eq:thm:trigonometric_identities/pythagorean_identity} follows.

  \SubProofOf{thm:trigonometric_identities/products} We will only prove \eqref{eq:thm:trigonometric_identities/products/cc} because the other identities are proved analogously. We have
  \begin{balign*}
    \cos(v - w) + \cos(v + w)
    &=
    \sum_{k=0}^\infty \frac {i^{2k}} {(2k)!} \left[(v - w)^{2k} + (v + w)^{2k} \right]
    \overset {\ref{thm:binomial_theorem}} = \\ &=
    \sum_{k=0}^\infty \frac {i^{2k}} {(2k)!} \sum_{m=0}^{2k} \binom {2k} m v^{2k-m} w^m \left[ (-1)^m + 1 \right]
    = \\ &=
    2 \sum_{k=0}^\infty \frac {i^{2k}} {(2k)!} \sum_{m=0}^{2k} \binom {2k} {2m} v^{2(k-m)} w^{2m}
    \overset {\eqref{eq:thm:trigonometric_identities/cos_product}} = \\ &=
    2 \cos(v) \cos(w).
  \end{balign*}

  \SubProofOf{thm:trigonometric_identities/sums} Fix some \( v, w \in \BbbC \) and define
  \begin{balign*}
    x \coloneqq \frac {v + w} 2
    &&
    y \coloneqq \frac {v - w} 2
  \end{balign*}
  so that \( v = x + y \) and \( w = x - y \).

  The identity \eqref{eq:thm:trigonometric_identities/sums/sin_sum} the follows from \eqref{eq:thm:trigonometric_identities/products/sc} applied to \( x \) and \( y \). The other identities are proved analogously.

  \SubProofOf{thm:trigonometric_identities/sum_of_angles} We will only prove \eqref{eq:thm:trigonometric_identities/sum_of_angles/sin} because \eqref{eq:thm:trigonometric_identities/sum_of_angles/cos} is proved analogously. From \eqref{eq:thm:trigonometric_identities/products/cs},
  \begin{balign*}
    \sin(x + y)
     & =
    2 \cos(x) \sin(y) + \sin(x - y)
    \overset {\eqref{eq:thm:trigonometric_identities/products/sc}} = \\ &=
    2 \cos(x) \sin(y) + 2 \cos(x) \sin(y) - \sin(x + y).
  \end{balign*}

  After dividing by \( 2 \), we obtain \eqref{eq:thm:trigonometric_identities/sum_of_angles/sin}.
\end{proof}

\begin{lemma}\label{thm:trigonometric_function_basic_roots}
  We have the following important special values:
  \begin{balign}
    \sin(0) = 0,   &  & \cos(0) = 1,    \label{eq:thm:trigonometric_function_basic_roots/zero} \\
    \sin(\pi) = 0, &  & \cos(\pi) = -1. \label{eq:thm:trigonometric_function_basic_roots/pi}
  \end{balign}
\end{lemma}
\begin{proof}
\eqref{eq:thm:trigonometric_function_basic_roots/zero} follows directly from \fullref{def:trigonometric_functions}.
  Now consider the \hyperref[def:function/extension]{restriction} of \( \cos \) to the real line. Since \( \cos(0) \neq 0 \) and \( \cos \) is continuously differentiable as a power series, in some neighborhood \( U \) of \( 0 \) we have \( 0 \not\in \cos(U) \). Therefore the inverse function theorem holds and there exists a neighborhood \( V \subseteq U \) of \( 1 \) such that the continuously differentiable function \( f: V \to \BbbR \) is the inverse of \( \cos \) in \( V \) (we have not yet defined \hyperref[def:inverse_trigonometric_functions/arccos]{\( \arccos \)}). If \( y = \cos(x) \), then
  \begin{equation*}
    Df(y)
    =
    \frac 1 {D\cos(x)}
    =
    \frac 1 {-\sin(x)}
    \overset {\ref{thm:trigonometric_identities/pythagorean_identity}} =
    -\frac 1 {\sqrt{1 - y^2}},
    \quad y \in \cos(V).
  \end{equation*}

  The derivative is actually well-defined and continuous anywhere except for \( y \in \{ -1, 1 \} \). Therefore, for any \( \alpha \in (-1, 1) \),
  \begin{equation*}
    f(y) = f(\alpha) - \int_{\alpha}^y \frac 1 {\sqrt{1 - t^2}} dt, \quad y \in [\alpha, 1).
  \end{equation*}

  We already know that \( \cos(0) = 1 \), hence \( f(1) = 0 \) and, since \( f(y) \) is given by a convergent integral in \( [\alpha, 1) \), we can extend this interval to \( [\alpha, 1] \).

  By taking \( y = \alpha \), we obtain
  \begin{equation*}
    f(y) - f(-y) = -\int_{-y}^y \frac 1 {\sqrt{1 - t^2}} dt, \quad y \in [-1, 1].
  \end{equation*}

  Note that by \hyperref[def:pi]{our definition} of \( \pi \),
  \begin{equation*}
    \pi
    =
    \int_{-1}^1 \frac 1 {\sqrt{1 - t^2}} dt
    =
    -[\underbrace{f(1)}_{=0} - f(-1)]
    =
    f(-1).
  \end{equation*}

  Hence \( \cos(\pi) = -1 \). From \fullref{thm:trigonometric_identities/pythagorean_identity},
  \begin{equation*}
    \abs{\sin(\pi)} = \sqrt{1 - \cos(\pi)^2} = 0,
  \end{equation*}
  proving that \( \sin(\pi) = 0 \).

  This concludes the proof of \eqref{eq:thm:trigonometric_function_basic_roots/pi}.
\end{proof}

\begin{definition}\label{def:periodic_function}
  A function \( f: \mscrG \to \mscrH \) between \hyperref[def:abelian_group] abelian groups is called \term{periodic} with \term{period} \( p \in \mscrG \) if, for all \( x \in \mscrG \), we have \( f(x) = f(x + r) \).

  The \term{base period} of a function is the \hyperref[def:preordered_set/largest_smallest_element]{least} of all periods, if a minimum exists. When referring to \enquote{the period}, we mean the base period.

  We can define periods for arbitrary magmas rather than abelian groups but the definition would make it difficult to talk about the base period.
\end{definition}

\begin{theorem}\label{thm:trigonometric_function_period}
  Both \( \sin(z) \) and \( \cos(z) \) are \( 2\pi \)-periodic.
\end{theorem}
\begin{proof}
  We will temporarily restrict ourselves to the real line. Since \( \cos(x) \) is continuous, \( \cos^{-1}(\{ 0 \}) \) is a closed set by \fullref{thm:weierstrass_extreme_value_theorem} there exists a minimum \( \gamma \) of \( [0, \pi] \cap \cos^{-1}(\{ 0 \}) \).

  Since, by \fullref{thm:trigonometric_function_basic_roots}, \( \cos(0) = 1 \), it follows that \( \cos(x) > 0, x \in (-\gamma, \gamma) \). Therefore its primitive function \( \sin(x) \) increases on the same interval. It is also continuous, hence by \fullref{thm:trigonometric_identities/pythagorean_identity}, \( \sin(\gamma) = 1 \) because \( \cos(\gamma) = 0 \).

  Because \( \sin \) is an odd function, \( \sin(-\gamma) = \sin(\gamma) = -1 \).

  From \hyperref[def:pi]{our definition} of \( \pi \) it follows that
  \begin{equation*}
    \pi
    =
    \int_{-1}^1 \frac 1 {1 - t^2} dt
    =
    \int_{-\gamma}^\gamma \frac {\cos(\varphi)} {1 - \sin(\varphi)^2} d\varphi
    \overset {\ref{thm:trigonometric_identities/pythagorean_identity}} =
    \int_{-\gamma}^\gamma d\varphi
    =
    2\gamma.
  \end{equation*}

  In order for a number \( p \) to be a period of \( \sin \), we need to have \( \sin(p) = \sin(0) = 0 \). But we showed that \( \sin(x) \) is increasing from \( 0 \) to \( \tfrac \pi 2 \) and cannot possibly contain zeros in that interval. Hence \( p > \tfrac \pi 2 \).

  We also have \( \cos(\tfrac \pi 2) = 0 \). By \fullref{thm:trigonometric_identities/sums},
  \begin{equation*}
    \sin(\tfrac \pi 2 + x)
    =
    \sin(\tfrac \pi 2) \cos(x) + \cos(\tfrac \pi 2) \sin(x)
    =
    \cos(x).
  \end{equation*}

  Since \( \cos \) is positive on \( [0, \tfrac \pi 2) \), \( \sin \) is positive on \( [\tfrac \pi 2, \pi) \).

  We already showed in \fullref{thm:trigonometric_function_basic_roots} that \( \sin(\pi) = 0 \).

  It follows that the minimal period of \( \sin \) is either \( \pi \) or a multiple of \( \pi \). It cannot be \( \pi \) since \( \cos(\pi) \neq \cos(0) \), therefore it must be \( 2\pi \).
\end{proof}

\begin{definition}\label{def:hyperbolic_trigonometric_functions}
  In analogy with \fullref{thm:exponential_trigonometric_identities/inverse_eulers_formula}, we define \term{hyperbolic trigonometric functions}.

  \begin{thmenum}
    \ilabel{def:hyperbolic_trigonometric_functions/sine} The \term{hyperbolic sine} function:
    \begin{equation*}
      \sinh(x) \coloneqq - \frac {e^x - e^{-x}} 2 \\
    \end{equation*}

    \ilabel{def:hyperbolic_trigonometric_functions/cosine} The \term{hyperbolic cosine} function:
    \begin{equation*}
      \cosh(x) \coloneqq \frac {e^x + e^{-x}} 2
    \end{equation*}
  \end{thmenum}

  Compare \fullref{def:quadratic_plane_curve/ellipse/parametric_equations} and \fullref{def:quadratic_plane_curve/hyperbola/parametric_equations} for a justification of the naming.
\end{definition}

\begin{definition}\label{def:derived_trigonometric_functions}
  In addition to \( \sin(z) \) and \( \cos(z) \), we define two additional functions, also called \enquote{trigonometric}.

  \begin{thmenum}
    \ilabel{def:derived_trigonometric_functions/tan} The \hyperref[def:function/partial]{partial} \term{tangent} function, also called \term{tangens}, is
    \begin{equation*}
      \tan(z) \coloneqq \frac {\sin(z)} {\cos(z)}.
    \end{equation*}

    It is defined in \( \BbbC \setminus (\tfrac \pi 2 + \pi\BbbZ) \).

    \ilabel{def:derived_trigonometric_functions/cot} The \hyperref[def:function/partial]{partial} \term{cotangent function}, also called \term{cotangens}, is
    \begin{equation*}
      \cot(z) \coloneqq \frac {\cos(z)} {\sin(z)}.
    \end{equation*}

    It is defined in \( \BbbC \setminus \pi\BbbZ \).
  \end{thmenum}
\end{definition}

\begin{definition}\label{def:inverse_trigonometric_functions}
  We can define \term{inverse trigonometric functions}. We will thus restrict ourselves only to real numbers. Fix an integer \( k \). Unless noted otherwise, we assume \( k = 0 \).

  \begin{thmenum}
    \ilabel{def:inverse_trigonometric_functions/arcsin} The \term{arcus sinus} function \( \arcsin(x) \) is defined as the \hyperref[def:function/inverse]{inverse function} of \( \sin(x) \) (see \fullref{def:trigonometric_functions/sine}) from \( [-1, 1] \) to \( \left[(k - \tfrac 1 2) \pi, (k + \tfrac 1 2) \pi \right) \).

    \ilabel{def:inverse_trigonometric_functions/arccos} The \term{arcus cosinus} function \( \arccos(x) \) is defined as the inverse of \( \cos(x) \) (see \fullref{def:trigonometric_functions/cosine}) from \( [-1, 1] \) to \( (k\pi, (k + 1)\pi) \).

    \ilabel{def:inverse_trigonometric_functions/arctan} The \term{arcus tangens} function \( \arctan(x) \) is defined as the inverse of \( \tan(x) \) (see \fullref{def:derived_trigonometric_functions/tan}) from \( \BbbR \) to \( \left((k - \tfrac 1 2) \pi, (k + \tfrac 1 2) \pi \right) \).

    \ilabel{def:inverse_trigonometric_functions/arccot} The \term{arcus cotangens} function \( \arccot(x) \) is defined as the inverse of \( \cot(x) \) (see \fullref{def:derived_trigonometric_functions/cot}) from \( \BbbR \) to \( (k\pi, (k + 1)\pi) \).

    \ilabel{def:inverse_trigonometric_functions/arctantwo} The \term{two-argument arcus tangens} function \( \arctantwo(y, x) \) is a bit special, however it is very useful in practice - see \fullref{thm:arctantwo}. It is defined as
    \begin{align*}
       &\arctantwo: \BbbR^2 \setminus \{ 0 \} \to [2k\pi, 2k\pi + 2)  \\
       &\arctantwo(y, x) \coloneqq \begin{dcases}
        \arctan \parens*{ \tfrac y x },       &x > 0                  \\
        \arctan \parens*{ \tfrac y x } + \pi, &x < 0 \T{and} y \geq 0 \\
        \arctan \parens*{ \tfrac y x } - \pi, &x < 0 \T{and} y < 0    \\
        \pi,                                 &x = 0 \T{and} y \geq 0 \\
        -\pi,                                &x = 0 \T{and} y < 0.
      \end{dcases}
    \end{align*}
  \end{thmenum}
\end{definition}

\begin{proposition}\label{thm:arctantwo}
  Fix an integer \( k \). Given \( (x_0, y_0) \in S_{\BbbR^2} \), \( t_0 \coloneqq \arctantwo(y_0, x_0) \) is the unique solution to the equation
  \begin{equation}\label{thm:arctantwo/equation}
    \begin{cases}
      x_0 = \cos(t) \\
      y_0 = \sin(t)
    \end{cases}
  \end{equation}
  in \( t \in [2k\pi, 2k\pi + 2) \).
\end{proposition}
