\subsection{Lattices}\label{subsec:lattices}

\begin{definition}\label{def:semilattice}\mcite[3]{Gratzer1978}
  Lattices are \hyperref[def:partially_ordered_set]{partially ordered sets} in which \hyperref[def:partially_ordered_set_extremal_points/supremum_and_infimum]{suprema and infima} are taken as basic operations called \enquote{joins} and \enquote{meets}. See \fullref{rem:lattice_operation_etymology} for a discussion of the operation names. This shifts the focus from ordering to operations, i.e. from predicates to functions.

  Joins and meets may also be defined axiomatically as binary operations rather than via some partial order, however this restricts us to taking suprema of finite sets and prevents us from taking the supremum of an arbitrary set. In other words, it is possible for the order to carry more information than joins and meets. See \fullref{thm:binary_lattice_operations/new_lattice} for a discussion. Unless explicitly noted otherwise, we assume that lattices have their partial order defined.

  \begin{thmenum}[series=def:semilattice]
    \thmitem{def:semilattice/join} A \term{join-semilattice} is a \hyperref[def:partially_ordered_set_extremal_points/top_and_bottom]{bounded from above} partially ordered set in which every finite supremum exists. The operation itself is denoted by \( \vee \) and referred to as \term{join} and rather than supremum. In contrast to suprema, joins are usually written in \hyperref[rem:first_order_formula_conventions/infix]{infix} notation, e.g. \( x \vee y \vee z \) rather than \( \sup\set{ x, y, z } \).

    \thmitem{def:semilattice/meet} Analogously, a \term{meet-semilattice} is a partially ordered set in which every finite infimum exists. The infimum is denoted by \( \wedge \) and called \term{meet}.

    \thmitem{def:semilattice/bounded} A \term{bounded semilattice} is a semilattice that is \hyperref[def:partially_ordered_set_extremal_points/top_and_bottom]{bounded} as a \hyperref[def:partially_ordered_set]{partially ordered set}, either \hi{from below} for join-semilattices or \hi{from above} for meet-semilattices.

    \thmitem{def:semilattice/complete}\mcite[24]{Gratzer1978} A semilattice is said to be \term{countably complete} or \( \sigma \)-\term{complete} if the corresponding operation is defined for \hyperref[def:set_countability/at_most_countable]{at most countable} sets rather than only finite ones.

    A semilattice is said to be \term{complete} if the corresponding operation is defined for arbitrary sets.

    Finite semilattices are clearly complete.

    \thmitem{def:semilattice/lattice} A \term{lattice} is a partially ordered set which is both a join-semilattice and a meet-semilattice. It is called \term{bounded} if both semilattices are bounded, i.e. if the partially ordered set itself is \hyperref[def:partially_ordered_set_extremal_points/top_and_bottom]{bounded}. It is called \term{join-complete} if the join-semilattice is complete, \term{meet-complete} if the meet-semilattice is complete, and simply \term{complete} if both semilattices are complete.

    \thmitem{def:semilattice/distributive_lattice}\mcite[30]{Gratzer1978} A lattice is said to be \term{distributive} if the following two conditions hold:
    \begin{align}
      x \vee (y \wedge z) &= (x \vee y) \wedge (x \vee z) \label{eq:def:semilattice/distributive_lattice/finite/join_over_meet} \\
      x \wedge (y \vee z) &= (x \wedge y) \vee (x \wedge z) \label{eq:def:semilattice/distributive_lattice/finite/meet_over_join}.
    \end{align}

    If the lattice is \hyperref[def:semilattice/complete]{complete}, the above conditions are not enough. A complete lattice \( L \) it is said to be \term{distributive} if any of the following more general distributive axioms hold for every \( x \in L \) and \hyperref[def:cartesian_product/indexed_family]{family} \( \seq{ y_k }_{k \in \mscrK} \subseteq L \):
    \begin{align}
      x \vee \parens*{ \bigwedge_{k \in \mscrK} y_k } &= \bigwedge_{k \in \mscrK} \parens{ x \vee y_k } \label{eq:def:semilattice/distributive_lattice/arbitrary/join_over_meet} \\
      x \wedge \parens*{ \bigvee_{k \in \mscrK} y_k } &= \bigvee_{k \in \mscrK} \parens{ x \wedge y_k } \label{eq:def:semilattice/distributive_lattice/arbitrary/meet_over_join}
    \end{align}
  \end{thmenum}

  Lattices have the following metamathematical properties:
  \begin{thmenum}[resume=def:semilattice]
    \thmitem{def:semilattice/theory} The language of the theory of lattices consists of the language of the \hyperref[def:partially_ordered_set/theory]{theory of partially ordered sets} with the addition of the binary infix functional symbols \( \vee \) and \( \wedge \). If we only want to restrict ourselves to semilattices, we can add only one of the two operations as functional symbols. If we wish to study \hyperref[def:semilattice/bounded]{bounded lattices}, as it is often done, we must also add the constants \( \top \) and \( \bot \).

    For meet-semilattices, we add the following axiom schema to the theory to ensure compatibility between infima and meets (we use \( \mathbin\& \) to denote \hyperref[def:propositional_language/connectives/conjunction]{logical conjunction} to avoid symbol collision with meets):
    \begin{equation}\label{eq:def:semilattice/theory/meet_compat}
      \parens[\Big]{ \xi \wedge \eta \doteq \alpha } \leftrightarrow \parens[\Big]{ \alpha \leq \xi \mathbin\& \alpha \leq \eta \mathbin\& \qforall \alpha ((\alpha \leq \xi \mathbin\& \alpha \leq \eta) \rightarrow \alpha \leq \alpha) }
    \end{equation}
    and, for bounded meet-semilattices, the following axiom to ensure that \( \top \) is indeed the maximum:
    \begin{equation}\label{eq:def:semilattice/theory/bottom_compat}
      \qforall \xi (\xi \leq \top).
    \end{equation}

    Analogous axioms need to be added for join-semilattices.

    We cannot properly express the theory of complete (semi)lattices as an extension of this theory since we must define join and meet as unary operations on subsets of the domain rather than binary operations on members of the domain. Complete semilattices can instead be defined within \hyperref[def:zfc]{\logic{ZFC}}.

    \thmitem{def:semilattice/submodel} Unlike for partially ordered sets, whose submodels are discussed in \fullref{def:partially_ordered_set/submodel}, not every subset of a semilattice is a sub-semilattice because \( \vee \) and \( \wedge \) are now regarded as functional symbols. A sub-(semi)lattice must be closed under joins and meets. The axiom \eqref{eq:def:semilattice/theory/meet_compat} is not a positive formula, but does not cause trouble itself as it merely specifies compatibility of \( \leq \) and \( \wedge \).

    For bounded semilattices, the relevant constants should formally be present in any bounded sub-semilattice. This is sometimes inconvenient, however. For convenience, we will distinguish between \enquote{sublattices} and \enquote{bounded sublattices}. Similarly, we distinguish between \enquote{sublattice} and \enquote{complete sublattice}.

    \thmitem{def:semilattice/trivial} The \hyperref[def:trivial_structure]{trivial} join-semilattice and the trivial meet-semilattice are the empty set. The trivial bounded join-semilattice is the singleton \( \set{ \bot } \) and the trivial bounded meet-semilattice is \( \set{ \top } \).

    The trivial bounded lattice consists only of one element, which is both the top and the bottom.

    The \hyperref[thm:substructures_form_complete_lattice/bottom]{initial substructure} of a bounded lattice is instead \( \set{ \top, \bot } \). Note that \( \top = \bot \) if and only if the algebra is trivial. Unless trivial, the initial substructures of different bounded lattices are isomorphic since homomorphisms must preserve constants.

    \thmitem{def:semilattice/homomorphism} \hyperref[def:first_order_homomorphism]{Homomorphisms} between (semi)lattices are the monotone maps that preserve joins, meets and constants.

    Homomorphisms of complete lattices need to preserve arbitrary joins and meets.

    \begin{figure}[!ht]
      \centering
      \includegraphics[page=1]{output/def__semilattice.pdf}
      \caption{A monotone map between lattices, which is not a lattice homomorphism}
      \label{fig:def:semilattice/homomorphism/monotone_map_not_homomorphism}
    \end{figure}

    As we shall see in \fullref{thm:lattice_homomorphism_is_monotone}, the requirement of monotonicity is redundant.

    \thmitem{def:semilattice/category} The \hyperref[def:category_of_small_first_order_models]{categories of \( \mscrU \)-small models} for (semi)lattices are subcategories of \hyperref[def:partially_ordered_set/category]{\( \ucat{Pos} \)}. We only give a special name for the category \( \ucat{Lat} \) of lattices.

    \thmitem{def:semilattice/duality} The \hyperref[def:partially_ordered_set/duality]{principle of duality for partially ordered sets} holds for lattices if we also swap the binary operations \( \vee \) and \( \wedge \).

    If the lattice is bounded, we must additionally swap the constants \( \top \) and \( \bot \).

    If the lattice is bounded from only one side, the principle of duality does not hold unless we restrict ourselves to formulas that do not contain the constants.
  \end{thmenum}
\end{definition}

\begin{example}\label{ex:complete_sublattice_of_incomplete_lattice}
  Any finite \hyperref[def:semilattice/submodel]{sublattice} is \hyperref[def:semilattice/complete]{complete}, even if the ambient lattice is not.
\end{example}

\begin{remark}\label{rem:lattice_operation_etymology}
  The terms \hyperref[def:semilattice/join]{\enquote{join}} for \( \vee \) and \hyperref[def:semilattice/meet]{\enquote{meet}} for \( \wedge \) are notoriously difficult to remember. A helpful accident is the ability to write \enquote{meet} as \enquote{\( \wedge \wedge \)eet}.
\end{remark}

\begin{proposition}\label{thm:binary_lattice_operations}
  Let \( (P, \leq) \) be a partially ordered set.

  \begin{thmenum}
    \thmitem{thm:binary_lattice_operations/semilattices} If it is a \hyperref[def:semilattice/join]{join-semilattice} (resp. \hyperref[def:semilattice/meet]{meet-semilattice}), then \( \vee \) (resp. \( \wedge \)) is \hyperref[def:magma/associative]{associative}, \hyperref[def:magma/commutative]{commutative} and \hyperref[def:magma/idempotent]{idempotent} when considered as a binary operation.

    \thmitem{thm:binary_lattice_operations/identity} If \( P \) is a (semi)lattice, the constants act as \hyperref[def:monoid]{monoid identities}. That is, for each \( x \in P \),
    \begin{align}
      x \vee \bot = x \label{eq:thm:binary_lattice_operations/identity/join} \\
      x \wedge \top = x \label{eq:thm:binary_lattice_operations/identity/meet}
    \end{align}

    \thmitem{thm:binary_lattice_operations/absorption} If \( P \) is a lattice, then the following absorption laws hold:
    \begin{align}
      x \vee (x \wedge y) &= x \label{eq:thm:binary_lattice_operations/absorption/join} \\
      x \wedge (x \vee y) &= x \label{eq:thm:binary_lattice_operations/absorption/meet}.
    \end{align}

    \thmitem{thm:binary_lattice_operations/compatibility} The following conditions for compatibility with \( \leq \) hold:
    \begin{align}
      x \leq y &\T{if and only if} x \vee y = y \label{eq:thm:binary_lattice_operations/compatibility/join} \\
      x \leq y &\T{if and only if} x \wedge y = x \label{eq:thm:binary_lattice_operations/compatibility/meet}.
    \end{align}

    \thmitem{thm:binary_lattice_operations/new_lattice} If \( A \) is an arbitrary \hyperref[def:set]{set} and if \( \vee \) is a binary operation that is associative, commutative and idempotent (the conclusion of \fullref{thm:binary_lattice_operations/semilattices}), then \( (A, \leq) \) is a join-semilattice with an ordering defined by \eqref{eq:thm:binary_lattice_operations/compatibility/join}. If there exists a distinguished element \( \bot \) such that \eqref{eq:thm:binary_lattice_operations/identity/meet} holds, then \( (A, \leq) \) is bounded.

    A completely analogous statement holds for meet-semilattices.

    If \( (A, \leq) \) is both a join-semilattice and meet-semilattice and if \( \vee \) and \( \wedge \) satisfy the absorption conditions \eqref{eq:thm:binary_lattice_operations/absorption/join} and \eqref{eq:thm:binary_lattice_operations/absorption/meet}, then \( (A, \leq) \) is a lattice. Furthermore, proving idempotence for \( \vee \) or \( \wedge \) is unnecessary because both follow from the absorption conditions.

    It may turn out that \( (A, \leq) \) is a complete lattice under this definition. This can allow us, for example, to transparently extend the binary operations join and meet into infinitary operations.
  \end{thmenum}
\end{proposition}
\begin{proof}
  \SubProofOf{thm:binary_lattice_operations/semilattices} Suprema and infima are obviously associative and commutative as binary operations because ordering is immaterial for pure sets and \( x \vee y \) is defined as \( \sup\set{ x, y } \).

  Idempotence is also obvious because \( x \vee x = \sup\set{ x } = x \).

  \SubProofOf{thm:binary_lattice_operations/identity} Obvious since \( \bot \leq x \leq \top \) for all \( x \in P \).

  \SubProofOf{thm:binary_lattice_operations/absorption} If we rewrite \eqref{eq:thm:binary_lattice_operations/absorption/join} using suprema and infima, we obtain
  \begin{equation*}
    \sup\set{ x, \inf\set{ y, x } } = x.
  \end{equation*}

  If \( x \leq y \), then \( \inf\set{ y, x } = x \) and \( \sup\set{ x, \inf\set{ y, x } } = \sup\set{ x, x } = x \).

  If \( x \geq y \), then \( \inf\set{ y, x } = y \) and \( \sup\set{ x, \inf\set{ y, x } } = \sup\set{ x, y } = x \).

  This proves \eqref{eq:thm:binary_lattice_operations/absorption/join}. Since \( \wedge \) is \( \vee \) in the \hyperref[def:preordered_set/duality]{opposite partially ordered set}, \eqref{eq:thm:binary_lattice_operations/absorption/meet} follows automatically.

  \SubProofOf{thm:binary_lattice_operations/compatibility} We have
  \begin{equation*}
    x \vee y
    =
    \sup\set{ x, y }
    =
    \begin{cases}
      y, &x \leq y \\
      x, &x > y
    \end{cases}
  \end{equation*}
  and dually for \( \wedge \).

  \SubProofOf{thm:binary_lattice_operations/new_lattice} Since the binary join and/or meet are defined for all members of the set \( A \), it is indeed a join-semilattice because all finite joins and meets exist by definition.

  Idempotence of \( \vee \) follows from \eqref{eq:thm:binary_lattice_operations/absorption/meet}:
  \begin{equation*}
    x \vee x = x \vee (x \wedge (x \vee x)) = x
  \end{equation*}
  and dually for \( \wedge \).
\end{proof}

\begin{corollary}\label{thm:lattice_homomorphism_is_monotone}
  If a function \( f: L \to M \) between \hyperref[def:semilattice]{(semi)lattices} preserves either joins or meets, it is \hyperref[def:partially_ordered_set/homomorphism]{monotone}.

  Thus, the requirement for \hyperref[def:semilattice/homomorphism]{lattice homomorphisms} to be monotone is redundant.
\end{corollary}
\begin{proof}
  If the function \( f \) preserves joins and if \( x < y \), by \eqref{eq:thm:binary_lattice_operations/compatibility/join} we have \( x \vee y = y \) and thus
  \begin{equation*}
    f(x \vee y) = f(y),
  \end{equation*}
  which again by \eqref{eq:thm:binary_lattice_operations/compatibility/join} implies \( f(x) \leq f(y) \).
\end{proof}

\begin{proposition}\label{thm:bounded_lattice_absorbing}
  In any \hyperref[def:semilattice/bounded]{bounded lattice}, \( \bot \) is absorbing with respect to meets and \( \top \) with respect to joins. That is, \( \bot \wedge x = \bot \) and \( \top \vee x = \top \).
\end{proposition}
\begin{proof}
  Obvious when the lattice is regarded as a partially ordered set.
\end{proof}

\begin{definition}\label{def:fixed_point}
  Given a \hyperref[def:function]{function} \( f: A \to A \) between arbitrary sets, we call \( x \in A \) a \term{fixed point} of \( f \) if \( x = f(x) \).
\end{definition}

\begin{theorem}[Knaster-Tarski theorem]\label{thm:knaster_tarski_theorem}
  The \hyperref[def:fixed_point]{fixed points} of a \hyperref[def:partially_ordered_set/homomorphism]{monotone} \hyperref[def:multi_valued_function/endofunction]{endofunction} in a \hyperref[def:semilattice/lattice]{complete lattice} form a complete sublattice. In particular, the function has at least one fixed point.
\end{theorem}
\begin{proof}
  Let \( (L, \leq) \) be a complete lattice and let \( \varphi: L \to L \) be a monotone function. Define
  \begin{equation*}
    D \coloneqq \set{ x \in L \given f(x) \leq x }.
  \end{equation*}

  We know that \( D \) is nonempty because \( \top \in D \).

  Since the lattice is complete, we can take \( l \coloneqq \inf D \). Note that \( f(l) \) is a lower bound of \( D \) because for any \( y \in D \) we have
  \begin{equation*}
    f(l) \leq f(y) \leq y.
  \end{equation*}

  But \( l \) is the largest lower bound of \( D \), hence
  \begin{equation}\label{eq:thm:knaster_tarski/f_lower}
    f(l) \leq l.
  \end{equation}

  Therefore, \( f(f(l)) \leq f(l) \) and \( f(l) \in D \). Hence, \( l \) is a lower bound for \( \set{ f(l) } \) and
  \begin{equation}\label{eq:thm:knaster_tarski/f_upper}
    l \leq f(l).
  \end{equation}

  From \eqref{eq:thm:knaster_tarski/f_lower} and \eqref{eq:thm:knaster_tarski/f_upper} it follows that \( l = f(l) \), that is, \( l \) is a fixed point of \( f \).

  Denote by \( F \) the set of all fixed points of \( L \). We just showed that \( F \) is nonempty. Let \( G \subseteq F \). We will show that the infimum and supremum of \( G \) is in \( F \).

  Denote
  \begin{equation*}
    l_G \coloneqq \inf G.
  \end{equation*}

  For any \( g \in G \) we have \( l_G \leq g \). From monotonicity of \( f \),
  \begin{equation*}
    f(l_G) \leq f(g) = g,
  \end{equation*}
  therefore \( f(l_G) \leq l_G \) because \( l_G \) is the greatest lower bound of \( G \). But, from monotonicity of \( f \), we have \( l_G \leq f(l_G) \). Therefore, \( f(l_G) = l_G \) and \( l_G \in F \).

  We can analogously show that \( \sup G \in F \) and conclude that \( (F, \leq) \) is itself a complete lattice.
\end{proof}

\begin{remark}\label{rem:lattice_categorical_product}
  The existence of finite joins and meets is equivalent to the existence of finite products and coproducts in the respective \hyperref[def:thin_category]{thin category} defined in \fullref{thm:order_category_isomorphism/partially_ordered}.
\end{remark}

\begin{definition}\label{def:square_free}
  An element \( x \) of a \hyperref[def:semiring/commutative]{commutative semiring} is said to be \term{square-free} if \( y \mid x \) implies that \( z^2 \not\mid x \).
\end{definition}

\begin{remark}\label{rem:lattice_polynomials}
  Let \( L \) be a \hyperref[def:semilattice/bounded]{bounded} \hyperref[def:semilattice/distributive_lattice]{distributive lattice}. We will discuss polynomials over \( L \).

  By \fullref{ex:def:semiring/lattice}, \( L \) induces a positive and a negative commutative semiring. Given a set \( \mscrX \) of indeterminates, we can form the \hyperref[def:polynomial_algebra]{polynomial semiring} \( L[\mscrX] \) over the positive semiring. Suppose that we are given an evaluation \( f: \mscrX \to L^\mscrX \). Consider the \hyperref[thm:polynomial_algebra_universal_property]{evaluation homomorphism} \( \Phi_f: L[\mscrX] \to L^\mscrX \).
  \begin{equation*}
    \Phi_f(X^2) = \Phi_f(X).
  \end{equation*}

  Hence, we can limit ourselves to \hyperref[def:square_free]{square-free} monomials. That is, monomials of the form \( \prod_{X \in \mscrX} X^{\gamma_X} \), where \( \gamma_X \) is either \( 0 \) or \( 1 \). More succinctly, since every monomial has finitely many indeterminates of positive power, it can be written as a finite meet \( X_1 \wedge \cdots \wedge X_n \). A polynomial is then a finite join of finite meets of indeterminates and constants.

  There is a nuance, however. In \cite[def. I.4.2]{Gratzer1978}, a multivariate lattice polynomial is defined to consist only of indeterminates; for example
  \begin{equation*}
    p(X, Y, Z) = (X \wedge Y) \vee (X \wedge Z) \vee (Y \wedge Z).
  \end{equation*}

  This excludes coefficients before the monomials, hence making the definition distinct from the general notion of a polynomial over a commutative semiring. In \cite{Marichal2007}, polynomials with coefficients in front of the monomials are called \term{weighted lattice polynomials}. For example, a weighted polynomial is
  \begin{equation*}
    q(X, Y, Z) = (a \wedge X \wedge Y) \vee (b \wedge X \wedge Z) \vee (c \wedge Y \wedge Z).
  \end{equation*}

  Compare \( q(X, Y, Z) \) to \( p(X, Y, Z) \). We will refer to unweighted polynomials by default.

  We can analogously define polynomials over the negative semiring of \( L \), e.g.
  \begin{equation*}
    r(X, Y, Z) = (X \vee Y) \wedge (X \vee Z) \wedge (Y \vee Z).
  \end{equation*}

  The latter polynomials are related to but distinct from \hyperref[def:cnf_and_dnf]{conjunctive normal forms}, while the former - to \hyperref[def:cnf_and_dnf]{disjunctive normal forms}.
\end{remark}

\begin{example}\label{ex:lattice_polynomials}
  We list several examples of \hyperref[rem:lattice_polynomials]{lattice polynomials}:
  \begin{thmenum}
    \thmitem{ex:lattice_polynomials/distributivity} The distributivity axiom \eqref{eq:def:semilattice/distributive_lattice/finite/meet_over_join} implies that the polynomial
    \begin{equation*}
      X \wedge (Y \vee Z)
    \end{equation*}
    over the positive semiring evaluates to the same trivariate function as the polynomial
    \begin{equation*}
      (X \wedge Y) \vee (X \wedge Z)
    \end{equation*}
    over the negative semiring.

    \thmitem{ex:lattice_polynomials/boolean} A lattice polynomial for the two-element boolean algebra \hyperref[def:boolean_value]{\( \set{ T, F } \)} corresponds to a \hyperref[def:boolean_function]{boolean function}. We cannot express negation without introducing auxiliary polynomials as a consequence of \fullref{ex:thm:posts_completeness_theorem/and_or}. \hyperref[def:positive_formula]{Positive formulas} in \hyperref[def:cnf_and_dnf]{conjunctive/disjunctive normal form}, however, correspond exactly to lattice polynomials with square-free monomials.

    For example, \hyperref[def:standard_boolean_operators]{exclusive or} \( \oplus \) can be expressed via the \hyperref[def:propositional_syntax/formula]{propositional formula}
    \begin{equation*}
      P \vee Q \vee (P \wedge Q).
    \end{equation*}

    This corresponds to a bivariate polynomial over the negative semiring of \( \set{ T, F } \).
  \end{thmenum}
\end{example}

\begin{proposition}\label{thm:lattice_divisibility}
  Identify the \hyperref[def:semilattice/bounded]{bounded} \hyperref[def:semilattice/distributive_lattice]{distributive lattice} with its \hyperref[ex:def:semiring/lattice]{positive semilattice}.

  \begin{thmenum}
    \thmitem{thm:lattice_divisibility/divides} We have \( x \mid y \) if and only if \( x \geq y \).

    \thmitem{thm:lattice_divisibility/prime} The element \( p \) is \hyperref[def:domain_divisibility/prime]{prime} if and only if \( x \wedge y \leq p \) implies that \( x \leq p \) or \( y \leq p \) (or both).

    This leads to \fullref{def:lattice_prime_element}.
  \end{thmenum}
\end{proposition}
\begin{proof}
  \SubProofOf{thm:lattice_divisibility/divides}
  \SufficiencySubProof* Suppose that \( p \) divides \( x \). Then there exists some \( y \) for which \( x = p \wedge y \).
  \begin{itemize}
    \item If \( y \leq p \), then \( x = p \wedge y = y \leq p \).
    \item If \( y > p \), then \( x = p \wedge y = p \).
  \end{itemize}

  In both cases, \( p \geq x \).

  \NecessitySubProof* Suppose that \( p \geq x \). Then \eqref{eq:thm:binary_lattice_operations/compatibility/meet} implies that \( x = p \wedge x \), hence \( p \mid x \).

  \SubProofOf{thm:lattice_divisibility/prime} Reformulation of \fullref{def:domain_divisibility/prime/direct} with respect to \fullref{thm:lattice_divisibility/divides}.
\end{proof}

\begin{definition}\label{def:lattice_ideal}
  We will define concepts analogous to those from \fullref{subsec:semiring_ideals}, but adapted to \hyperref[def:semilattice/lattice]{lattices}. \Fullref{thm:lattice_divisibility} provides motivation. In some cases the notions coincide --- see \fullref{thm:semilattice_ideal_as_semiring_ideal}.

  Fix some lattice \( L \).

  \begin{thmenum}
    \thmitem{def:lattice_ideal/ideal}\mcite[17]{Gratzer1978}
    \begin{minipage}[t]{0.45\textwidth}
      We call the subset \( I \) a \term{lattice ideal} it is closed under joins and if \( i \in I \) and \( l \in L \) together imply \( i \wedge l \in I \).

      Since this also implies that \( I \) is closed under meets, it follows that ideals are automatically \hyperref[def:semilattice/submodel]{sublattices}.
    \end{minipage}
    \hspace{0.02\textwidth}
    \begin{minipage}[t]{0.45\textwidth}
      \hyperref[def:partially_ordered_set/duality]{Dually}, we call the subset \( F \) a \term{filter} if it is closed under meets and if \( f \in F \) and \( l \in L \) together imply \( f \vee l \in F \).
    \end{minipage}

    \thmitem{def:lattice_ideal/principal}
    \begin{minipage}[t]{0.45\textwidth}
      We define the \term{principal ideal} of \( x \) as the set
      \begin{equation*}
        \set{ y \in L \given y \leq x }.
      \end{equation*}
    \end{minipage}
    \hspace{0.02\textwidth}
    \begin{minipage}[t]{0.45\textwidth}
      Dually, we define the \term{principal filter} of \( x \) as
      \begin{equation*}
        \set{ y \in L \given y \geq x }.
      \end{equation*}

      In \hyperref[def:boolean_algebra]{Boolean algebras}, they are also called \enquote{principal ultrafilters} because of \fullref{thm:principal_ultrafilter}.
    \end{minipage}

    \thmitem{def:lattice_ideal/prime}\mcite[17]{Gratzer1978}
    \begin{minipage}[t]{0.45\textwidth}
      We call a proper ideal \( I \) \term{prime} if \( {x \wedge y \in I} \) implies that \( x \in I \) or \( y \in I \) (or both) and \term{completely prime} if \( \bigwedge_{k \in \mscrK} x_k \in I \) implies that \( x_k \in I \) for at least one \( k \in \mscrK \).
    \end{minipage}
    \hspace{0.02\textwidth}
    \begin{minipage}[t]{0.45\textwidth}
      Dually, we call a proper filter \( F \) \term{prime} if \( x \vee y \in F \) implies that \( x \in F \) or \( y \in F \) and \term{completely prime} if \( \bigvee_{k \in \mscrK} x_k \in I \) implies that \( x_k \in I \) for at least one \( k \in \mscrK \).
    \end{minipage}

    \thmitem{def:lattice_ideal/maximal}
    \begin{minipage}[t]{0.45\textwidth}
      We call a proper ideal \term{maximal} if it is \hyperref[def:partially_ordered_set_extremal_points/maximal_and_minimal_element]{maximal} with respect to set inclusion.
    \end{minipage}
    \hspace{0.02\textwidth}
    \begin{minipage}[t]{0.45\textwidth}
      Dually, we call a proper filter is \term{maximal} if it is maximal with respect to inclusion.

      Unlike for semirings, it is possible for a maximal ideal or filter not to be prime.
    \end{minipage}

    \thmitem{def:lattice_ideal/submodel}
    \begin{minipage}[t]{0.45\textwidth}
      We say that \( J \) is a \term{subideal} of \( I \) if both are ideals and if \( J \subseteq I \).
    \end{minipage}
    \hspace{0.02\textwidth}
    \begin{minipage}[t]{0.45\textwidth}
      Dually, we say that \( G \) is a \term{subfilter} of \( F \) if both are filters and if \( G \subseteq F \).
    \end{minipage}
  \end{thmenum}
\end{definition}

\begin{definition}\label{def:closed_ordered_set}
  We say that the subset \( S \) of the \hyperref[def:partially_ordered_set]{partially ordered set} \( (P, \leq) \) is \term{upward closed} if, for every \( s \in S \), \( s \leq x \) implies \( x \in S \).

  We define \term{downward closed} sets analogously.
\end{definition}

\begin{proposition}\label{thm:def:lattice_ideal}
  The \hyperref[def:lattice_ideal]{ideals and filters} of a lattice have the following basic properties:
  \begin{thmenum}
    \thmitem{thm:def:lattice_ideal/directed_and_closed} A set is an ideal if and only if it is \hyperref[def:directed_set]{upward directed} and \hyperref[def:closed_ordered_set]{downward closed}.

    Analogously, a set is a filter if it is \hyperref[def:directed_set]{downward directed} and \hyperref[def:closed_ordered_set]{upward closed}.

    \thmitem{thm:def:lattice_ideal/prime} The complement of a \hyperref[def:lattice_ideal/prime]{prime ideal} (resp. \hyperref[def:lattice_ideal/prime]{prime filter}) is a filter (resp. ideal).

    \thmitem{thm:def:lattice_ideal/completely_prime} A set is a prime \hyperref[def:lattice_ideal/principal]{principal ideal} (resp. filter) if and only if its complement is \hyperref[def:lattice_ideal/prime]{completely prime filter} (resp. \hyperref[def:lattice_ideal/prime]{completely prime ideal}).
  \end{thmenum}
\end{proposition}
\begin{proof}
  We will prove the statements only for complements of ideals. The statements for complements of filters will follow by \hyperref[def:semilattice/duality]{duality}.

  \SubProofOf{thm:def:lattice_ideal/directed_and_closed}

  \SufficiencySubProof* Suppose that \( I \) is an ideal.

  \begin{itemize}
    \item It is upward directed because, if \( x \in I \) and \( y \in I \), then, since \( I \) is closed under joins, \( x \vee y \in I \).
    \item It is downward closed because, if \( i \in I \) and \( x \leq i \), then, since \( I \) is closed under meets with arbitrary elements of the lattice, \( x = i \wedge x \in I \)
  \end{itemize}

  \NecessitySubProof* Conversely, suppose that \( I \) is upward directed and downward closed.

  \begin{itemize}
    \item It is closed under joins because if \( x \in I \) and \( y \in I \), then directedness gives us an upper bound \( z \). Then \( x \vee y \leq z \), and since \( I \) is downward closed, we conclude that \( x \vee y \in I \).

    \item It is closed under meets with elements of the lattice because if \( i \in I \), then \( x \wedge i \leq i \) and downward closedness implies \( x \wedge i \in I \).
  \end{itemize}

  \SubProofOf{thm:def:lattice_ideal/prime} Let \( I \) be a prime ideal of the lattice \( L \). Its complement is a filter because:
  \begin{itemize}
    \item It is closed under meets. Let \( x, y \in L \setminus I \) and suppose that \( x \wedge y \in I \). Then, since \( I \) is a prime ideal, \( x \in I \) or \( y \in I \). The obtained contradiction shows that \( x \wedge y \in L \setminus I \).

    \item It is closed under joins with elements of \( L \). Let \( l \in L \) and \( f \in L \setminus I \) and suppose that \( l \vee f \in I \). Then \( f \leq l \vee f \leq p \), implying that \( f \in I \). The obtained contradiction shows that \( l \vee f \in L \setminus I \).
  \end{itemize}

  \SubProofOf{thm:def:lattice_ideal/completely_prime}

  \SufficiencySubProof* Suppose that \( I \) be a principal prime ideal with generator \( p \).

  Its complement is a filter as a consequence of \fullref{thm:def:lattice_ideal/prime}. We must show that it is completely prime.

  Suppose that \( \bigvee_{k \in \mscrK} x_k \in L \setminus I \) and that \( x_k \not\in F \) for every \( k \in \mscrK \). Then \( p \) as the generator for \( I \) is an upper bound for the set \( \set{ x_k \given k \in \mscrK } \). Since \( \bigvee_{k \in \mscrK} x_k \) is the least upper bound, \( x \) must dominate it. That is, \( \bigvee_{k \in \mscrK} x_k \) must belong to \( I \), which contradicts our assumption that it belongs to \( L \setminus I \).

  Therefore, \( I \) is a completely prime filter.

  \NecessitySubProof* Now suppose that \( I \) is completely prime. Define
  \begin{equation*}
    f_0 \coloneqq \bigwedge L \setminus I
  \end{equation*}
  and let
  \begin{equation*}
    F \coloneqq \set{ x \in L \given x \geq f_0 }
  \end{equation*}
  be the principal filter of \( f_0 \). Obviously \( L \setminus I \subseteq F \).

  Conversely, let \( y \in F \), i.e. \( y \geq f_0 \). Suppose that \( y \in I \). Since \( f_0 \leq y \) and \( I \) is an ideal, it follows that \( f_0 \in I \). Since \( I \) is completely prime, there exists at least one point of \( L \setminus I \) that is also a point of \( I \).

  The obtained contradiction shows that \( y \in F \) implies that \( y \in L \setminus I \). Hence, \( F \subseteq L \setminus I \).

  Therefore, \( F = L \setminus I \).
\end{proof}

\begin{proposition}\label{thm:semilattice_ideal_as_semiring_ideal}
  If we identify the \hyperref[def:semilattice/bounded]{bounded} \hyperref[def:semilattice/distributive_lattice]{distributive lattice} with its \hyperref[ex:def:semiring/lattice]{positive semilattice}, then the \hyperref[def:lattice_ideal/ideal]{lattice ideals} are precisely the \hyperref[def:semiring_ideal]{semiring ideals}.

  Moreover, \hyperref[def:lattice_ideal/prime]{prime} and \hyperref[def:lattice_ideal/maximal]{maximal} lattice ideals are \hyperref[def:semiring_ideal/prime]{prime} and \hyperref[def:semiring_ideal/maximal]{maximal} semiring ideals.

  The same holds for \hyperref[def:lattice_ideal/ideal]{lattice filters} if we instead consider the \hyperref[ex:def:semiring/lattice]{negative semilattice}.
\end{proposition}
\begin{proof}
  Straightforward.
\end{proof}

\begin{example}\label{ex:def:lattice_ideal}
  We list examples of \hyperref[def:lattice_ideal]{lattice ideals and filters}:
  \begin{thmenum}
    \thmitem{ex:def:lattice_ideal/trivial} The simplest ideal is \( \set{ \bot } \) and the simplest filter is \( \set{ \top } \).

    We call them the \term{trivial ideal} and \term{trivial filter}.

    \thmitem{ex:def:lattice_ideal/lattice} Consider the (zero-based) natural number divisibility lattice from \fullref{thm:natural_number_divisibility_lattice}. For any natural number \( n \), the semiring ideal
    \begin{equation*}
      \braket{ n } \coloneqq \set{ 0, n, 2n, 3n, \ldots }
    \end{equation*}
    is a principal filter in the divisibility lattice.

    Indeed,
    \begin{itemize}
      \item \( \braket{ n } \) contains the top \( 0 \).
      \item \( \braket{ n } \) contains the meet \( \gcd(a, b) \) of every two members of \( \braket{ n } \).
      \item If \( n \mid a \) and \( b \) is any natural number, then \( n \mid a \mid \lcm(a, b) \), meaning that the join \( \lcm(a, b) \) also belongs to \( \braket{ n } \).
    \end{itemize}

    If \( p \) is a \hyperref[def:prime_number]{prime number}, then \( \braket{ p } \) is a \hyperref[def:semiring_ideal/maximal]{maximal semiring ideal}, and hence also a \hyperref[def:semiring_ideal/prime]{prime semiring ideal}, and hence also a \hyperref[def:lattice_ideal/prime]{prime lattice ideal}.

    The corresponding filter is more complicated. For any natural number \( n \), the set \( D_n \) of all \hyperref[def:divisibility]{divisors} of \( n \) is a lattice ideal.

    \begin{figure}[!ht]
      \centering
      \includegraphics[page=1]{output/ex__lattice_ideals.pdf}
      \caption{A Hasse diagram for the divisors of \( 24 \).}
      \label{fig:ex:def:lattice_ideal/lattice}
    \end{figure}

    Indeed,
    \begin{itemize}
      \item The bottom \( 1 \) divides \( n \).
      \item If \( a \) and \( b \) divide \( n \), their product \( ab \) also does, and hence their join \( \lcm(a, b) \) also does.
      \item If \( a \mid n \) and \( b \) is any natural number, then \( \gcd(a, b) \mid a \mid n \).
    \end{itemize}

    Furthermore, \( D_n \) is a \hyperref[def:semiring_ideal]{principal ideal} since it can be obtained as \( \set{ n \wedge m \mid m \in \BbbN } \).

    \thmitem{ex:def:lattice_ideal/subgroups} Given a \hyperref[def:group]{group} \( G \) and a proper \hyperref[thm:normal_subgroup_equivalences]{normal subgroup} \( N \), consider the \hyperref[thm:substructures_form_complete_lattice]{lattice of subgroups} \( L_G \) of \( G \) and the lattice \( L_N \) of subgroups of \( N \). Note that the top \( N \) of \( L_N \) is not the top of \( L_G \), hence \( L_N \) is a lattice but not a bounded sublattice of \( L_G \).

    Then \( L_N \) is an ideal on \( L \).
    \begin{itemize}
      \item \( L_N \) contains the bottom \( \set{ e } \).
      \item \( L_N \) contains the join \( \braket{ K \cup H } \) of every two subgroups of \( N \).
      \item \( L_N \) contains the meet \( K \cap H \) of \( K \in L_N \) and \( H \in L_G \) since \( K \cap H \subseteq K \subseteq N \).
    \end{itemize}

    By \hyperref[def:lattice/duality]{duality}, the sublattice of subgroups containing \( N \) (rather than contained in \( N \)) is a filter (rather than an ideal).
  \end{thmenum}
\end{example}

\begin{definition}\label{def:lattice_prime_element}
  We say that the element \( p \) of the lattice \( L \) is \term{prime} if any of the following equivalent conditions hold:
  \begin{thmenum}
    \thmitem{def:lattice_prime_element/direct} If \( x \wedge y \leq p \), then \( x \leq p \) or \( y \leq p \) (or both).
    \thmitem{def:lattice_prime_element/ideals} The \hyperref[def:lattice_ideal/principal]{principal ideal} of \( p \) is \hyperref[def:lattice_ideal/prime]{prime}.
    \thmitem{def:lattice_prime_element/filters} The complement of the \hyperref[def:lattice_ideal/principal]{principal ideal} of \( p \) is a \hyperref[def:lattice_ideal/prime]{completely prime filter}.
  \end{thmenum}
\end{definition}
\begin{defproof}
  \ImplicationSubProof{def:lattice_prime_element/direct}{def:lattice_prime_element/ideals} Suppose that \( y \wedge z \leq p \) implies \( y \leq z \) or \( y \leq p \).

  The principal ideal \( I \) of \( p \) consists of all elements of \( L \) that are dominated by \( p \). Thus, if \( y \wedge z \in I \), then \( y \wedge z \leq p \) and hence \( y \leq p \), in which case \( y \in I \), or \( z \leq p \).

  Hence, \( I \) is a principal ideal.

  \EquivalenceSubProof{def:lattice_prime_element/ideals}{def:lattice_prime_element/filters} Follows from \fullref{thm:def:lattice_ideal/completely_prime}.

  \ImplicationSubProof{def:lattice_prime_element/filters}{def:lattice_prime_element/direct} Let \( I \) be the ideal generated by \( p \) and suppose that its complement \( L \setminus I \) is a completely prime filter.

  If \( x \wedge y \in I \), and if both \( x \) and \( y \) belong to \( L \setminus I \), then, since as a filter \( L \setminus I \) is closed under meets, it follows that \( x \wedge y \in L \setminus I \).

  The obtained contradiction shows that if \( x \wedge y \in I \), then \( x \in I \) or \( y \in I \).
\end{defproof}
