\subsection{Lattices}\label{subsec:lattices}

\begin{definition}\label{def:semilattice}\mcite{nLab:lattice}
  Lattices are \hyperref[def:poset]{posets} in which \hyperref[def:preordered_set/supremum_and_infimum]{suprema and infima} are taken as basic operations called \enquote{joins} and \enquote{meets}. This shifts the focus and makes lattices feel more \enquote{algebraic} compared to arbitrary ordered sets. See \fullref{rem:lattice_operation_etymology} for a discussion of the operation names.

  Joins and meets may also be defined axiomatically as binary operations (see \fullref{thm:binary_lattice_operations}) rather than via some partial order, however this restricts us to taking suprema of finite sets and prevents us from taking the supremum of an arbitrary set. See \fullref{thm:binary_lattice_operations/new_lattice} for a discussion. Unless explicitly noted otherwise, we assume that lattices have their partial order defined.

  \begin{thmenum}[series=def:semilattice]
    \thmitem{def:semilattice/join} A \term{join-semilattice} is a \hyperref[def:preordered_set/bounded_set]{bounded from above} poset in which every finite supremum exists. The operation itself is denoted by \( \vee \) and referred to as \term{join} and rather than supremum. In contrast to suprema, joins are usually written in \hyperref[rem:order_infix_notation]{infix} notation, e.g. \( x \vee y \vee z \) rather than \( \sup\set{ x, y, z } \).

    The boundedness from above is equivalent to the existence of a global maximum which we will denote by \( \top \coloneqq \bigvee \mscrX \) and call the \term{top} element.

    \thmitem{def:semilattice/meet} Analogously, a \term{meet-semilattice} is a bounded from below poset in which every finite infimum exists. The infimum is denoted by \( \wedge \) and called \term{meet} and the global minimum is denoted by \( \bot \coloneqq \bigwedge \mscrX \).

    \thmitem{def:semilattice/complete} A semilattice is said to be \term{complete} if the corresponding operation is defined for all infinite sets rather than only finite ones.

    \thmitem{def:semilattice/lattice} A \term{lattice} is a poset which is both a join-semilattice and a meet-semilattice. It is complete if both semilattices are complete. If finite joins and meets exist but the poset is unbounded, we sometimes call it a \term{pseudolattice}.

    \thmitem{def:semilattice/distributive_lattice}\mcite{nLab:distributive_lattice} A lattice is said to be \term{distributive} if the following two distributive conditions hold:
    \begin{align}
      x \vee (y_1 \wedge y_2) &= (x \vee y_1) \wedge (y \vee y_2) \label{eq:def:semilattice/distributive_lattice/finite/meet_of_joins} \\
      x \wedge (y_1 \vee y_2) &= (x \wedge y_1) \vee (y \wedge y_2) \label{eq:def:semilattice/distributive_lattice/finite/join_of_meets}.
    \end{align}

    If the lattice is \hyperref[def:semilattice/complete]{complete}, the above conditions are not enough. A complete lattice \( \mscrX \) it is said to be \term{distributive} if any of the following more general distributive axioms hold for any \( x \in \mscrX \) and \hyperref[def:indexed_family]{family} \( \set{ y_k }_{k \in \mscrK} \subseteq \mscrX \):
    \begin{align}
      x \vee \parens*{ \bigwedge_{k \in \mscrK} y_k } &= \bigwedge_{k \in \mscrK} \parens{ x \vee y_k } \label{eq:def:semilattice/distributive_lattice/arbitrary/meet_of_joins} \\
      x \wedge \parens*{ \bigvee_{k \in \mscrK} y_k } &= \bigvee_{k \in \mscrK} \parens{ x \wedge y_k } \label{eq:def:semilattice/distributive_lattice/arbitrary/join_of_meets}
    \end{align}
  \end{thmenum}

  Lattices have the following metamathematical structure:
  \begin{thmenum}[resume=def:semilattice]
    \thmitem{def:semilattice/theory} We extend the \hyperref[def:poset/theory]{theory of posets} with a single \hyperref[rem:order_infix_notation]{infix} binary functional symbol to obtain the \term{theories of semilattices} and with both symbols to obtain the \term{theory of lattices}.

    For meet-semilattices, we add the following axiom schema to ensure compatibility between infima and meets (we use \( \wedgeonwedge \) to denote \hyperref[def:propositional_language/connectives/conjunction]{logical conjunction} to avoid symbol collision with meets):
    \begin{equation}\label{eq:def:semilattice/theory/meet_compat}
      (\xi \wedge \eta \doteq \alpha) \leftrightarrow (\alpha \leq \xi \wedgeonwedge \alpha \leq \eta \wedgeonwedge \qforall \alpha ((\alpha \leq \xi \wedgeonwedge \alpha \leq \eta) \rightarrow \alpha \leq \alpha))
    \end{equation}
    and the following axiom to ensure that \( \bot \) is indeed the minimum:
    \begin{equation}\label{eq:def:semilattice/theory/bottom_compat}
      \qforall \xi (\bot \leq \xi).
    \end{equation}

    Analogous axioms need to be added for meet-semilattices.

    First-order logic cannot properly express the \term{theory of complete (semi)lattices} since we must define join and meet as unary operations on subsets of \( \mscrX \) rather than binary operations on members of \( \mscrX \).

    \thmitem{def:semilattice/homomorphism} \hyperref[def:first_order_homomorphism]{Homomorphisms} between (semi)lattices are simply the monotone maps.

    Alternatively, without referring to the order, we can characterize homomorphisms as functions preserving joins, meets and constants. No axioms follow automatically as in \fullref{thm:group_homomorphism_single_condition}.

    \thmitem{def:semilattice/category} The \hyperref[def:category_of_first_order_models]{model categories} for (semi)lattices are full subcategories of \hyperref[def:poset/category]{\( \cat{Pos} \)}.
  \end{thmenum}
\end{definition}

\begin{remark}\label{rem:lattice_operation_etymology}
  The terms \hyperref[thm:binary_lattice_operations/join]{\enquote{join}} for \( \vee \) and \hyperref[thm:binary_lattice_operations/meet]{\enquote{meet}} for \( \wedge \) are notoriously difficult to remember. A helpful accident is the ability to write \enquote{meet} as \enquote{\( \wedge \wedge \)eet}.
\end{remark}

\begin{proposition}\label{thm:binary_lattice_operations}
  Let \( (\mscrX, \leq) \) be a poset.

  \begin{thmenum}
    \thmitem{thm:binary_lattice_operations/semilattices} If it is a \hyperref[def:semilattice/join]{join-semilattice} (resp. \hyperref[def:semilattice/meet]{meet-semilattice}), then \( \vee \) (resp. \( \wedge \)) is \hyperref[def:magma/associative]{associative}, \hyperref[def:magma/commutative]{commutative} and \hyperref[def:magma/idempotent]{idempotent} when considered as a binary operation.

    \thmitem{thm:binary_lattice_operations/identity} The constants act as \hyperref[def:magma_identity]{magma identities}. That is, for each \( x \in \mscrX \),
    \begin{align}
      x \vee \bot = x \label{eq:thm:binary_lattice_operations/identity/join} \\
      x \wedge \top = x \label{eq:thm:binary_lattice_operations/identity/meet}
    \end{align}

    \thmitem{thm:binary_lattice_operations/absorption} If \( (\mscrX, \vee, \wedge) \) is a lattice, then the following absorption laws hold:
    \begin{align}
      x \vee (x \wedge y) &= x \label{eq:thm:binary_lattice_operations/absorption/join} \\
      x \wedge (x \vee y) &= x \label{eq:thm:binary_lattice_operations/absorption/meet}.
    \end{align}

    \thmitem{thm:binary_lattice_operations/compatibility} The following conditions for compatibility with \( \leq \) hold:
    \begin{align}
      x \leq y &\iff x \vee y = y \label{eq:thm:binary_lattice_operations/compatibility/join} \\
      x \leq y &\iff x \wedge y = x \label{eq:thm:binary_lattice_operations/compatibility/meet}.
    \end{align}

    \thmitem{thm:binary_lattice_operations/new_lattice} If \( \mscrS \) is an arbitrary \hyperref[def:set_zfc]{set}, if \( \vee \) is a binary operation that is associative, commutative and idempotent (the conclusion of \fullref{thm:binary_lattice_operations/semilattices}) and if there exists a distinguished element \( \top \) such that \eqref{eq:thm:binary_lattice_operations/identity/join} holds, then \( (\mscrS, \vee) \) is a join-semilattice with an ordering defined by \eqref{eq:thm:binary_lattice_operations/compatibility/join}. A completely analogous statement holds for meet-semilattices.

    If \( \mscrS \) is both a join-semilattice and meet-semilattice and if \( \vee \) and \( \wedge \) satisfy the absorption conditions \eqref{eq:thm:binary_lattice_operations/absorption/join} and \eqref{eq:thm:binary_lattice_operations/absorption/meet}, then \( \mscrS \) is a lattice. Furthermore, proving idempotence for \( \vee \) or \( \wedge \) is unnecessary because both follow from the absorption conditions.

    It may turn out that \( \mscrS \) is a complete lattice under this definition. This can allow us, for example, to transparently extend the binary operations join and meet into infinitary operations.
  \end{thmenum}
\end{proposition}
\begin{proof}
  \SubProofOf{thm:binary_lattice_operations/semilattices} Suprema and infima are obviously associative and commutative as binary operations because ordering is immaterial for pure sets and \( x \vee y \) is defined as \( \sup\set{ x, y } \).

  Idempotence is also obvious because \( x \vee x = \sup\set{ x } = x \).

  \SubProofOf{thm:binary_lattice_operations/identity} Obvious since \( \bot \leq x \leq \top \) for all \( x \in \mscrX \).

  \SubProofOf{thm:binary_lattice_operations/absorption} If we rewrite \eqref{eq:thm:binary_lattice_operations/absorption/join} using suprema and infima, we obtain
  \begin{equation*}
    \sup\set{ x, \inf\set{ y, x } } = x.
  \end{equation*}

  If \( x \leq y \), then \( \inf\set{ y, x } = x \) and \( \sup\set{ x, \inf\set{ y, x } } = \sup\set{ x, x } = x \).

  If \( x \geq y \), then \( \inf\set{ y, x } = y \) and \( \sup\set{ x, \inf\set{ y, x } } = \sup\set{ x, y } = x \).

  This proves \eqref{eq:thm:binary_lattice_operations/absorption/join}. Since \( \wedge \) is \( \vee \) in the \hyperref[def:preordered_set/duality]{dual poset}, \eqref{eq:thm:binary_lattice_operations/absorption/meet} follows automatically.

  \SubProofOf{thm:binary_lattice_operations/compatibility} We have
  \begin{equation*}
    x \vee y
    =
    \sup\set{ x, y }
    =
    \begin{cases}
      y, &x \leq y \\
      x, &x > y
    \end{cases}
  \end{equation*}
  and dually for \( \wedge \).

  \SubProofOf{thm:binary_lattice_operations/new_lattice} Since the binary join and/or meet are defined for all members of the set \( \mscrS \), it is indeed a join-semilattice because all finite joins and meets exist by definition.

  Idempotence of \( \vee \) follows from \eqref{eq:thm:binary_lattice_operations/absorption/meet}:
  \begin{equation*}
    x \vee x = x \vee (x \wedge (x \vee x)) = x
  \end{equation*}
  and dually for \( \wedge \).
\end{proof}

\begin{proposition}\label{thm:binary_lattices_are_isomorphic}
  All two-element \hyperref[def:semilattice/lattice]{lattices} are isomorphic.
\end{proposition}
\begin{proof}
  Since the lattices are complete, they include at least two elements - the \hyperref[def:semilattice/top]{top} and the \hyperref[def:semilattice/bottom]{bottom}. Since by the statement of the proposition they have exactly two elements, their top and the bottom are their only elements.

  Obviously \( \bot \leq \top \) in any lattice, hence the obvious bijection between the lattices is in fact an \hyperref[def:preordered_set/homomorphism]{order isomorphism}.
\end{proof}

\begin{proposition}[Knaster-Tarski]\label{thm:knaster_tarski_theorem}
  The \hyperref[def:fixed_point]{fixed points} of a \hyperref[def:preordered_set/homomorphism]{monotone} \hyperref[def:endofunction]{endofunction} in a \hyperref[def:lattice]{complete lattice} form a complete sublattice. In particular, the function has at least one fixed point.
\end{proposition}
\begin{proof}
  Let \( \mscrX \) be a complete lattice and let \( \varphi: \mscrX \to \mscrX \) be monotone. Define
  \begin{equation*}
    L \coloneqq \{ x \in \mscrX \colon f(x) \leq x \}.
  \end{equation*}

  We know that \( L \) is nonempty because \( \top \in L \).

  Since the lattice is complete, we can take \( l \coloneqq \inf L \). Note that \( f(l) \) is a lower bound of \( L \) because for any \( y \in L \) we have
  \begin{equation*}
    f(l) \leq f(y) \leq y.
  \end{equation*}

  But \( l \) is the largest lower bound of \( L \), hence
  \begin{equation}\label{eq:thm:knaster_tarski/f_lower}
    f(l) \leq l.
  \end{equation}

  Therefore \( f(f(l)) \leq f(l) \) and \( f(l) \in L \). Hence \( l \) is a lower bound for \( \{ f(l) \} \) and
  \begin{equation}\label{eq:thm:knaster_tarski/f_upper}
    l \leq f(l).
  \end{equation}

  From \eqref{eq:thm:knaster_tarski/f_lower} and \eqref{eq:thm:knaster_tarski/f_upper} it follows that \( l = f(l) \), that is, \( l \) is a fixed point of \( f \).

  Denote by \( F \) the set of all fixed points of \( \mscrX \). We just showed that \( F \) is nonempty. Let \( G \subseteq F \). We will show that the infimum and supremum of \( G \) is in \( F \).

  Denote
  \begin{equation*}
    l_G \coloneqq \inf G.
  \end{equation*}

  For any \( g \in G \) we have \( l_G \leq g \). By monotonicity of \( f \),
  \begin{equation*}
    f(l_G) \leq f(g) = g,
  \end{equation*}
  therefore \( f(l_G) \leq l_G \) because \( l_G \) is the greatest lower bound of \( G \). But, by monotonicity of \( f \), we have \( l_G \leq f(l_G) \). Therefore \( f(l_G) = l_G \) and \( l_G \in F \).

  We can analogously show that \( \sup G \in F \). We conclude that \( F \) is itself a complete lattice.
\end{proof}

\begin{remark}\label{def:lattice_categorical_product}
  The existence of finite joins and meets is equivalent to the existence of finite products and coproducts in the respective poset category defined in \fullref{thm:partial_order_category_correspondence}.
\end{remark}
