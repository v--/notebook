\subsection{Lattices}\label{subsec:lattices}

\begin{definition}\label{def:semilattice}\mcite[3]{Gratzer1978}
  Lattices are \hyperref[def:partially_ordered_set]{partially ordered sets} in which \hyperref[def:partially_ordered_set_extremal_points/supremum_and_infimum]{suprema and infima} are taken as basic operations called \enquote{joins} and \enquote{meets}. See \fullref{rem:lattice_operation_etymology} for a discussion of the operation names. This shifts the focus from ordering to operations, i.e. from predicates to functions.

  Joins and meets may also be defined axiomatically as binary operations rather than via some partial order, however this restricts us to taking suprema of finite sets and prevents us from taking the supremum of an arbitrary set. In other words, it is possible for the order to carry more information than joins and meets. See \fullref{thm:binary_lattice_operations/new_lattice} for a discussion. Unless explicitly noted otherwise, we assume that lattices have their partial order defined.

  \begin{thmenum}[series=def:semilattice]
    \thmitem{def:semilattice/join} A \term{join-semilattice} is a \hyperref[def:partially_ordered_set_extremal_points/top_and_bottom]{bounded from above} partially ordered set in which every finite supremum exists. The operation itself is denoted by \( \vee \) and referred to as \term{join} and rather than supremum. In contrast to suprema, joins are usually written in \hyperref[rem:first_order_formula_conventions/infix]{infix} notation, e.g. \( x \vee y \vee z \) rather than \( \sup\set{ x, y, z } \).

    \thmitem{def:semilattice/meet} Analogously, a \term{meet-semilattice} is a partially ordered set in which every finite infimum exists. The infimum is denoted by \( \wedge \) and called \term{meet}.

    \thmitem{def:semilattice/bounded} A \term{bounded semilattice} is a semilattice that is \hyperref[def:partially_ordered_set_extremal_points/top_and_bottom]{bounded} as a \hyperref[def:partially_ordered_set]{partially ordered set}, either \hi{from below} for join-semilattices or \hi{from} above for meet-semilattices.

    \thmitem{def:semilattice/complete}\mcite[24]{Gratzer1978} A semilattice is said to be \term{complete} if the corresponding operation is defined for arbitrary sets rather than only finite ones.

    Finite semilattices are clearly complete, as well as bounded semilattices.

    \thmitem{def:semilattice/lattice} A \term{lattice} is a partially ordered set which is both a join-semilattice and a meet-semilattice. It is called \term{bounded} if both semilattices are bounded, i.e. if the partially ordered set itself is \hyperref[def:partially_ordered_set_extremal_points/top_and_bottom]{bounded}. It is called \term{complete} if both semilattices are complete.

    \thmitem{def:semilattice/distributive_lattice}\mcite[30]{Gratzer1978} A lattice is said to be \term{distributive} if the following two conditions hold:
    \begin{align}
      x \vee (y \wedge z) &= (x \vee y) \wedge (x \vee z) \label{eq:def:semilattice/distributive_lattice/finite/join_over_meet} \\
      x \wedge (y \vee z) &= (x \wedge y) \vee (x \wedge z) \label{eq:def:semilattice/distributive_lattice/finite/meet_over_join}.
    \end{align}

    If the lattice is \hyperref[def:semilattice/complete]{complete}, the above conditions are not enough. A complete lattice \( X \) it is said to be \term{distributive} if any of the following more general distributive axioms hold for every \( x \in X \) and \hyperref[def:cartesian_product/indexed_family]{family} \( \seq{ y_k }_{k \in \mscrK} \subseteq X \):
    \begin{align}
      x \vee \parens*{ \bigwedge_{k \in \mscrK} y_k } &= \bigwedge_{k \in \mscrK} \parens{ x \vee y_k } \label{eq:def:semilattice/distributive_lattice/arbitrary/join_over_meet} \\
      x \wedge \parens*{ \bigvee_{k \in \mscrK} y_k } &= \bigvee_{k \in \mscrK} \parens{ x \wedge y_k } \label{eq:def:semilattice/distributive_lattice/arbitrary/meet_over_join}
    \end{align}
  \end{thmenum}

  Lattices have the following metamathematical properties:
  \begin{thmenum}[resume=def:semilattice]
    \thmitem{def:semilattice/theory} The language of the theory of lattices consists of the language of the \hyperref[def:partially_ordered_set/theory]{theory of partially ordered sets} with the addition of the binary infix functional symbols \( \vee \) and \( \wedge \). If we only want to restrict ourselves to semilattices, we can add only one of the two operations as functional symbols. If we wish to study \hyperref[def:semilattice/bounded]{bounded lattices}, as it is often done, we must also add the constants \( \top \) and \( \bot \).

    For meet-semilattices, we add the following axiom schema to the theory to ensure compatibility between infima and meets (we use \( \mathbin\& \) to denote \hyperref[def:propositional_language/connectives/conjunction]{logical conjunction} to avoid symbol collision with meets):
    \begin{equation}\label{eq:def:semilattice/theory/meet_compat}
      \parens[\Big]{ \xi \wedge \eta \doteq \alpha } \leftrightarrow \parens[\Big]{ \alpha \leq \xi \mathbin\& \alpha \leq \eta \mathbin\& \qforall \alpha ((\alpha \leq \xi \mathbin\& \alpha \leq \eta) \rightarrow \alpha \leq \alpha) }
    \end{equation}
    and, for bounded meet-semilattices, the following axiom to ensure that \( \top \) is indeed the maximum:
    \begin{equation}\label{eq:def:semilattice/theory/bottom_compat}
      \qforall \xi (\xi \leq \top).
    \end{equation}

    Analogous axioms need to be added for join-semilattices.

    We cannot properly express the theory of complete (semi)lattices as an extension of this theory since we must define join and meet as unary operations on subsets of the domain rather than binary operations on members of the domain. Complete semilattices can instead be defined within \hyperref[def:zfc]{\logic{ZFC}}.

    \thmitem{def:semilattice/submodel} Unlike for partially ordered sets, whose submodels are discussed in \fullref{def:partially_ordered_set/submodel}, not every subset of a semilattice is a sub-semilattice because \( \vee \) and \( \wedge \) are now regarded as functional symbols. The axiom \eqref{eq:def:semilattice/theory/meet_compat} is not a positive formula, but does not cause trouble itself as it merely specifies compatibility of \( \leq \) and \( \wedge \).

    For bounded semilattices, the relevant constants should be present in any bounded sub-semilattice.

    \thmitem{def:semilattice/trivial} The \hyperref[thm:substructures_form_complete_lattice/bottom]{trivial join-semilattice} and the trivial meet-semilattice are the empty set. The trivial bounded join-semilattice is the singleton \( \set{ \bot } \) and the trivial bounded meet-semilattice is \( \set{ \top } \).

    The trivial bounded lattice satisfies \( \top = \bot \), which implies that it consists of one element.

    Note that the elements \( \top \) and \( \bot \) formally differ between different semilattices, however all trivial bounded lattices are isomorphic and hence it makes sense to speak of \enquote{the} bounded lattice.

    \thmitem{def:semilattice/homomorphism} \hyperref[def:first_order_homomorphism]{Homomorphisms} between (semi)lattices are simply the monotone maps.

    Alternatively, without referring to the order, we can characterize homomorphisms as functions preserving joins, meets and constants. No axioms follow automatically as in \fullref{thm:group_homomorphism_single_condition}.

    \thmitem{def:semilattice/category} The \hyperref[def:category_of_small_first_order_models]{categories of \( \mscrU \)-small models} for (semi)lattices are full subcategories of \hyperref[def:partially_ordered_set/category]{\( \ucat{Pos} \)}. We only give a special name for the category \( \ucat{Lat} \) of lattices.

    \thmitem{def:semilattice/lattice_duality} The \hyperref[def:partially_ordered_set/opposite]{principle of duality for partially ordered sets} holds for lattices if we also swap the binary operations \( \vee \) and \( \wedge \).

    If the lattice is bounded, we must additionally swap the constants \( \top \) and \( \bot \).

    If the lattice is bounded from only one side, the principle of duality does not hold unless we restrict ourselves to formulas that do not contain the constants.
  \end{thmenum}
\end{definition}

\begin{remark}\label{rem:lattice_operation_etymology}
  The terms \hyperref[def:semilattice/join]{\enquote{join}} for \( \vee \) and \hyperref[def:semilattice/meet]{\enquote{meet}} for \( \wedge \) are notoriously difficult to remember. A helpful accident is the ability to write \enquote{meet} as \enquote{\( \wedge \wedge \)eet}.
\end{remark}

\begin{proposition}\label{thm:binary_lattice_operations}
  Let \( (P, \leq) \) be a partially ordered set.

  \begin{thmenum}
    \thmitem{thm:binary_lattice_operations/semilattices} If it is a \hyperref[def:semilattice/join]{join-semilattice} (resp. \hyperref[def:semilattice/meet]{meet-semilattice}), then \( \vee \) (resp. \( \wedge \)) is \hyperref[def:magma/associative]{associative}, \hyperref[def:magma/commutative]{commutative} and \hyperref[def:magma/idempotent]{idempotent} when considered as a binary operation.

    \thmitem{thm:binary_lattice_operations/identity} If \( P \) is a (semi)lattice, the constants act as \hyperref[def:monoid]{monoid identities}. That is, for each \( x \in P \),
    \begin{align}
      x \vee \bot = x \label{eq:thm:binary_lattice_operations/identity/join} \\
      x \wedge \top = x \label{eq:thm:binary_lattice_operations/identity/meet}
    \end{align}

    \thmitem{thm:binary_lattice_operations/absorption} If \( P \) is a lattice, then the following absorption laws hold:
    \begin{align}
      x \vee (x \wedge y) &= x \label{eq:thm:binary_lattice_operations/absorption/join} \\
      x \wedge (x \vee y) &= x \label{eq:thm:binary_lattice_operations/absorption/meet}.
    \end{align}

    \thmitem{thm:binary_lattice_operations/compatibility} The following conditions for compatibility with \( \leq \) hold:
    \begin{align}
      x \leq y &\T{if and only if} x \vee y = y \label{eq:thm:binary_lattice_operations/compatibility/join} \\
      x \leq y &\T{if and only if} x \wedge y = x \label{eq:thm:binary_lattice_operations/compatibility/meet}.
    \end{align}

    \thmitem{thm:binary_lattice_operations/new_lattice} If \( A \) is an arbitrary \hyperref[def:set]{set} and if \( \vee \) is a binary operation that is associative, commutative and idempotent (the conclusion of \fullref{thm:binary_lattice_operations/semilattices}), then \( (A, \leq) \) is a join-semilattice with an ordering defined by \eqref{eq:thm:binary_lattice_operations/compatibility/join}. If there exists a distinguished element \( \bot \) such that \eqref{eq:thm:binary_lattice_operations/identity/meet} holds, then \( (A, \leq) \) is bounded.

    A completely analogous statement holds for meet-semilattices.

    If \( (A, \leq) \) is both a join-semilattice and meet-semilattice and if \( \vee \) and \( \wedge \) satisfy the absorption conditions \eqref{eq:thm:binary_lattice_operations/absorption/join} and \eqref{eq:thm:binary_lattice_operations/absorption/meet}, then \( (A, \leq) \) is a lattice. Furthermore, proving idempotence for \( \vee \) or \( \wedge \) is unnecessary because both follow from the absorption conditions.

    It may turn out that \( (A, \leq) \) is a complete lattice under this definition. This can allow us, for example, to transparently extend the binary operations join and meet into infinitary operations.
  \end{thmenum}
\end{proposition}
\begin{proof}
  \SubProofOf{thm:binary_lattice_operations/semilattices} Suprema and infima are obviously associative and commutative as binary operations because ordering is immaterial for pure sets and \( x \vee y \) is defined as \( \sup\set{ x, y } \).

  Idempotence is also obvious because \( x \vee x = \sup\set{ x } = x \).

  \SubProofOf{thm:binary_lattice_operations/identity} Obvious since \( \bot \leq x \leq \top \) for all \( x \in P \).

  \SubProofOf{thm:binary_lattice_operations/absorption} If we rewrite \eqref{eq:thm:binary_lattice_operations/absorption/join} using suprema and infima, we obtain
  \begin{equation*}
    \sup\set{ x, \inf\set{ y, x } } = x.
  \end{equation*}

  If \( x \leq y \), then \( \inf\set{ y, x } = x \) and \( \sup\set{ x, \inf\set{ y, x } } = \sup\set{ x, x } = x \).

  If \( x \geq y \), then \( \inf\set{ y, x } = y \) and \( \sup\set{ x, \inf\set{ y, x } } = \sup\set{ x, y } = x \).

  This proves \eqref{eq:thm:binary_lattice_operations/absorption/join}. Since \( \wedge \) is \( \vee \) in the \hyperref[def:preordered_set/opposite]{opposite partially ordered set}, \eqref{eq:thm:binary_lattice_operations/absorption/meet} follows automatically.

  \SubProofOf{thm:binary_lattice_operations/compatibility} We have
  \begin{equation*}
    x \vee y
    =
    \sup\set{ x, y }
    =
    \begin{cases}
      y, &x \leq y \\
      x, &x > y
    \end{cases}
  \end{equation*}
  and dually for \( \wedge \).

  \SubProofOf{thm:binary_lattice_operations/new_lattice} Since the binary join and/or meet are defined for all members of the set \( A \), it is indeed a join-semilattice because all finite joins and meets exist by definition.

  Idempotence of \( \vee \) follows from \eqref{eq:thm:binary_lattice_operations/absorption/meet}:
  \begin{equation*}
    x \vee x = x \vee (x \wedge (x \vee x)) = x
  \end{equation*}
  and dually for \( \wedge \).
\end{proof}

\begin{proposition}\label{thm:bounded_lattice_absorbing}
  In any \hyperref[def:semilattice/bounded]{bounded lattice}, \( \bot \) is absorbing with respect to meets and \( \top \) with respect to joins. That is, \( \bot \wedge x = \bot \) and \( \top \vee x = \top \).
\end{proposition}
\begin{proof}
  Obvious when the lattice is regarded as a partially ordered set.
\end{proof}

\begin{definition}\label{def:fixed_point}
  Given a \hyperref[def:function]{function} \( f: A \to A \) between arbitrary sets, we call \( x \in A \) a \term{fixed point} of \( f \) if \( x = f(x) \).
\end{definition}

\begin{theorem}[Knaster-Tarski theorem]\label{thm:knaster_tarski_theorem}
  The \hyperref[def:fixed_point]{fixed points} of a \hyperref[def:partially_ordered_set/homomorphism]{monotone} \hyperref[def:multi_valued_function/endofunction]{endofunction} in a \hyperref[def:semilattice/lattice]{complete lattice} form a complete sublattice. In particular, the function has at least one fixed point.
\end{theorem}
\begin{proof}
  Let \( (X, \leq) \) be a complete lattice and let \( \varphi: X \to X \) be a monotone function. Define
  \begin{equation*}
    L \coloneqq \{ x \in X \colon f(x) \leq x \}.
  \end{equation*}

  We know that \( L \) is nonempty because \( \top \in L \).

  Since the lattice is complete, we can take \( l \coloneqq \inf L \). Note that \( f(l) \) is a lower bound of \( L \) because for any \( y \in L \) we have
  \begin{equation*}
    f(l) \leq f(y) \leq y.
  \end{equation*}

  But \( l \) is the largest lower bound of \( L \), hence
  \begin{equation}\label{eq:thm:knaster_tarski/f_lower}
    f(l) \leq l.
  \end{equation}

  Therefore, \( f(f(l)) \leq f(l) \) and \( f(l) \in L \). Hence, \( l \) is a lower bound for \( \{ f(l) \} \) and
  \begin{equation}\label{eq:thm:knaster_tarski/f_upper}
    l \leq f(l).
  \end{equation}

  From \eqref{eq:thm:knaster_tarski/f_lower} and \eqref{eq:thm:knaster_tarski/f_upper} it follows that \( l = f(l) \), that is, \( l \) is a fixed point of \( f \).

  Denote by \( F \) the set of all fixed points of \( X \). We just showed that \( F \) is nonempty. Let \( G \subseteq F \). We will show that the infimum and supremum of \( G \) is in \( F \).

  Denote
  \begin{equation*}
    l_G \coloneqq \inf G.
  \end{equation*}

  For any \( g \in G \) we have \( l_G \leq g \). From monotonicity of \( f \),
  \begin{equation*}
    f(l_G) \leq f(g) = g,
  \end{equation*}
  therefore \( f(l_G) \leq l_G \) because \( l_G \) is the greatest lower bound of \( G \). But, from monotonicity of \( f \), we have \( l_G \leq f(l_G) \). Therefore, \( f(l_G) = l_G \) and \( l_G \in F \).

  We can analogously show that \( \sup G \in F \) and conclude that \( (F, \leq) \) is itself a complete lattice.
\end{proof}

\begin{remark}\label{rem:lattice_categorical_product}
  The existence of finite joins and meets is equivalent to the existence of finite products and coproducts in the respective \hyperref[def:thin_category]{thin category} defined in \fullref{thm:order_category_isomorphism/partially_ordered}.
\end{remark}
