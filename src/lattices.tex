\subsection{Lattices}\label{subsec:lattices}

\begin{remark}\label{remark:infinite_lattice_operations}
  Suprema and infima in posets can be used to define operations named \hyperref[def:binary_lattice_operations/join]{\Def{join}} and \hyperref[def:binary_lattice_operations/meet]{\Def{meet}}, however there are also axioms for binary joins and \hyperref[def:binary_lattice_operations]{meets}. If we are interested in infinitary joins and meets, however, we need to use the poset definition. This can be accomplished indirectly by
  \begin{enumerate}
    \item Defining joins and meets axiomatically,
    \item Using them to define a partial order,
    \item Using the partial order to define infinitary joins and meets
  \end{enumerate}
\end{remark}

\begin{remark}\label{remark:lattice_operation_etymology}
  The terms \hyperref[def:binary_lattice_operations/join]{\enquote{join}} for \( \vee \) and \hyperref[def:binary_lattice_operations/meet]{\enquote{meet} for \( \wedge \)} are notoriously difficult to remember. A helpful accident is the ability to write \enquote{meet} as \enquote{\( \wedge \wedge \)eet}.
\end{remark}

\begin{definition}\label{def:binary_lattice_operations}
  Fix an arbitrary set \( \CX \) and let \( x, y \in \CX \). Define two \hyperref[def:magma/associative]{associative} and \hyperref[def:magma/commutative]{commutative} binary operations:
  \begin{DefEnum}
    \ILabel{def:binary_lattice_operations/join} The \Def{join} of \( x \) and \( y \), denoted as \( x \vee y \).
    \ILabel{def:binary_lattice_operations/meet} The \Def{meet} of \( x \) and \( y \), denoted as \( x \wedge y \).
    \ILabel{def:binary_lattice_operations/absorption} The following \Def{absorption laws} hold:
    \begin{BreakableAlign*}
      x \vee (y \wedge x) = x
       &  &
      x \wedge (y \vee x) = x.
    \end{BreakableAlign*}
  \end{DefEnum}
\end{definition}

\begin{proposition}\label{thm:binary_lattice_operations_poset}
  Given a set \( \CX \) with \hyperref[def:binary_lattice_operations/join]{joins} and \hyperref[def:binary_lattice_operations/meet]{meets}, we can define the \hyperref[def:poset]{partial order relation} \( \leq \) as either
  \begin{equation*}
    x \leq y \iff x \vee y = y
  \end{equation*}
  or
  \begin{equation*}
    x \leq y \iff x \wedge y = x.
  \end{equation*}

  Both definitions of \( \leq \) are equivalent and induce on \( \CX \) a poset structure that is compatible with \fullref{def:lattice_operations}.
\end{proposition}
\begin{proof}
  We will first show that the two definitions of \( \leq \) are equivalent. If \( x = x \wedge y \), then
  \begin{equation*}
    y
    \overset {\ref{def:binary_lattice_operations/absorption}} =
    y \vee (x \wedge y)
    =
    y \vee x
    \overset {\ref{def:magma/commutative}} =
    x \vee y.
  \end{equation*}

  Now we will prove that \( \leq \), defined as \( x \leq y \iff x \vee y = y \), is indeed a partial order.
  \SubProofOf{def:binary_relation/reflexive} Direct consequence of \fullref{thm:binary_lattice_operations_properties/idempotence}.
  \SubProofOf{def:binary_relation/antisymmetric} Let \( x \leq y \) and \( y \leq x \), that is, \( x \vee y = y \) and \( y \vee x = x \). By \fullref{def:magma/commutative}, \( x = y \vee x = x \vee y = y \).
  \SubProofOf{def:binary_relation/transitive} Let \( x \leq y \) and \( y \leq z \), that is, \( x \vee y = y \) and \( y \vee z = z \). Then, by \fullref{def:magma/associative},
  \begin{equation*}
    z = y \vee z = (x \vee y) \vee z = x \vee (y \vee z) = x \vee z.
  \end{equation*}
\end{proof}

\begin{proposition}\label{thm:binary_lattice_operations_properties}
  If \( (\CX, \vee, \wedge) \) is a set with a \hyperref[def:binary_lattice_operations/join]{binary join} and a \hyperref[def:binary_lattice_operations/meet]{binary meet}, the following properties hold:
  \begin{DefEnum}
    \ILabel{thm:binary_lattice_operations_properties/idempotence} Both operations are \hyperref[def:algebraic_theory/idempotent_element]{idempotent}, i.e. \( x \vee x = x = x \wedge x \).
  \end{DefEnum}
\end{proposition}
\begin{proof}
  \SubProofOf{thm:binary_lattice_operations_properties/idempotence} \fullref{def:binary_lattice_operations/absorption} implies that
  \begin{equation*}
    x \vee x = x \vee (x \wedge (x \vee x)) = x
  \end{equation*}
  and analogously \( x \wedge x = x \).
\end{proof}

\begin{definition}\label{def:lattice_operations}
  Let \( (\CX, \leq) \) be a \hyperref[def:poset]{poset}. We define \Def{joins} \( \vee \) and \Def{meets} \( \wedge \) as the \hyperref[def:function/partial]{partial} functions
  \begin{align*}
    &{}\vee{}: \Pow(\CX) \to \CX
    &&{}\wedge{}: \Pow(\CX) \to \CX
    \\
    & \vee(A) \coloneqq \sup A
    && \wedge(A) \coloneqq \inf A.
  \end{align*}

  For finite sets, we usually use the infix notation \( x_1 \vee \ldots \vee x_n \) instead of \( \vee \{ x_1, \ldots, x_n \} \).
\end{definition}
\begin{proof}
  We first show that \( \vee \) and \( \wedge \) satisfy \fullref{def:binary_lattice_operations}. Since suprema and infima are obviously associative and commutative, it remains only to show \fullref{def:binary_lattice_operations/absorption}, that is, for any comparable \( x, y \in \CX \),
  \begin{BreakableAlign*}
    x = \sup \{x, \inf \{ x, y \} \}
     &  &
    x = \inf \{x, \sup \{ x, y \} \}.
  \end{BreakableAlign*}

  If \( x \leq y \), then
  \begin{BreakableAlign*}
    \sup \{ x, \inf \{ x, y \} \} = \sup \{ x, x \} = x
     &  &
    \inf \{ x, \sup \{ x, y \} \} = \inf \{ x, y \} = x.
  \end{BreakableAlign*}

  If \( y \leq x \), then
  \begin{BreakableAlign*}
    \sup \{ x, \inf \{ x, y \} \} = \sup \{ x, y \} = x
     &  &
    \inf \{ x, \sup \{ x, y \} \} = \inf \{ x, x \} = x.
  \end{BreakableAlign*}

  We now show that if the partial order \( \leq \) was defined using binary joins and meets as in \fullref{def:binary_lattice_operations}, then the original join \( \vee \) and meet \( \wedge \) are compatible with the binary \( \sup \) and \( \inf \).

  Fix \( x, y \in \CX \). Since the functions \( \vee \) and \( \wedge \) are total, all binary suprema and infima exist. If \( \sup \{ x, y \} = z \), then \( z \) is the least element such that both \( x \leq z \) and \( y \leq z \). Thus
  \begin{equation*}
    x \vee z = z = y \vee z.
  \end{equation*}

  Hence
  \begin{equation*}
    x \vee y = (x \vee (z \wedge x)) \vee y = x \vee z \vee y = x \vee (z \vee z) \vee y = \\ = (x \vee z) \vee (z \vee y) = z \vee z = z.
  \end{equation*}

  Conversely, if \( x \vee y = z \), by \fullref{def:binary_lattice_operations/absorption},
  \begin{equation*}
    x \vee z = (x \wedge (y \vee x)) \vee z = (x \wedge z) \vee z = z,
  \end{equation*}
  thus \( x \leq z \). Analogously, \( y \leq z \).

  If we assume that there exists \( t \in \CX \) such that \( x \leq t \leq z \) and \( y \leq t \leq z \), then
  \begin{equation*}
    t = t \vee x = t \vee x \vee y = t \vee z = z.
  \end{equation*}

  Thus \( z = \sup \{ x, y \} \).

  The equivalence between binary \( \inf \) and \( \wedge \) can be obtained analogously.
\end{proof}

\begin{definition}\label{def:lattice}
  A poset \( (\CX, \leq) \) is called a \Def{lattice} if the following are satisfied:

  \begin{DefEnum}
    \ILabel{def:lattice/top} The \hyperref[def:preordered_set/largest_smallest_element]{global maximum} \( \vee \CX \) exists. We call it the \Def{top element} \( \top \).

    \ILabel{def:lattice/bottom} The \hyperref[def:preordered_set/largest_smallest_element]{global minimum} \( \wedge \CX \) exists. We call it the \Def{bottom element} \( \bot \).

    \ILabel{def:lattice/finite_join} All finite \hyperref[def:binary_lattice_operations]{joins} exist.
    \ILabel{def:lattice/finite_meet} All finite \hyperref[def:binary_lattice_operations]{meets} exist.
  \end{DefEnum}

  If the join and meet are defined axiomatically, all finite joins and meets necessarily exist.

  If the last two properties hold for all joins and meets, not necessarily finite, we say that the lattice is a \Def{complete lattice}.
\end{definition}

\begin{proposition}[Knaster-Tarski]\label{thm:knaster_tarski}
  The \hyperref[def:fixed_point]{fixed points} of a \hyperref[def:monotone_map]{monotone} \hyperref[def:endofunction]{endofunction} in a \hyperref[def:lattice]{complete lattice} form a complete sublattice. In particular, the function has at least one fixed point.
\end{proposition}
\begin{proof}
  Let \( \CX \) be a complete lattice and let \( \varphi: \CX \to \CX \) be monotone. Define
  \begin{equation*}
    L \coloneqq \{ x \in \CX \colon f(x) \leq x \}.
  \end{equation*}

  We know that \( L \) is nonempty because \( \top \in L \).

  Since the lattice is complete, we can take \( l \coloneqq \inf L \). Note that \( f(l) \) is a lower bound of \( L \) because, for any \( y \in L \),
  \begin{equation*}
    \varphi(l) \leq \varphi(y) \leq y.
  \end{equation*}

  But \( l \) is the largest lower bound of \( L \), hence
  \begin{equation}\label{eq:thm:knaster_tarski/f_lower}
    f(l) \leq l.
  \end{equation}

  Therefore \( f(f(l)) \leq f(l) \) and \( f(l) \in L \). Hence \( l \) is a lower bound for \( \{ f(l) \} \) and
  \begin{equation}\label{eq:thm:knaster_tarski/f_upper}
    l \leq f(l).
  \end{equation}

  From \eqref{eq:thm:knaster_tarski/f_lower} and \eqref{eq:thm:knaster_tarski/f_upper} it follows that \( l = f(l) \), that is, \( l \) is a fixed point of \( f \).

  Denote by \( F \) the set of all fixed points of \( \CX \). We just showed that \( F \) is nonempty. Let \( G \subseteq F \). We will show that \( G \) has an infimum and supremum in \( F \).

  Denote
  \begin{equation*}
    l_G \coloneqq \inf G.
  \end{equation*}

  For any \( g \in G \) we have \( l_G \leq g \). By monotonicity of \( f \),
  \begin{equation*}
    f(l_G) \leq f(g) = g,
  \end{equation*}
  therefore \( f(l_G) \leq l_G \) because \( l_G \) is the greatest lower bound of \( G \). But, by monotonicity of \( f \), we have \( l_G \leq f(l_G) \). Therefore \( f(l_G) = l_G \) and \( l_G \in F \).

  We can analogously show that \( \sup G \in F \). We conclude that \( F \) is itself a complete lattice.
\end{proof}

\begin{proposition}\label{thm:lattice_homomorphism}
  A \hyperref[def:first_order_homomorphism]{homomorphism} of the lattices
  \begin{equation*}
    (\CX, \vee_{\CX}, \wedge_{\CX}, \top_{\CX}, \bot_{\CX})
  \end{equation*}
  and
  \begin{equation*}
    (\CY, \vee_{\CY}, \wedge_{\CY}, \top_{\CY}, \bot_{\CY})
  \end{equation*}
  is a function \( \varphi: \CX \to \CY \) such that
  \begin{PropEnum}
    \ILabel{thm:lattice_homomorphism/respects_joins} \( f(x \vee_{\CX} y) = f(x) \vee_{\CY} f(y) \)
    \ILabel{thm:lattice_homomorphism/respects_meets} \( f(x \wedge_{\CX} y) = f(x) \wedge_{\CY} f(y) \)
    \ILabel{thm:lattice_homomorphism/respects_top} \( f(\top_{\CX}) = \top_{\CY} \)
    \ILabel{thm:lattice_homomorphism/respects_bot} \( f(\bot_{\CX}) = \bot_{\CY} \)
  \end{PropEnum}

  That is, no axioms follow automatically as in \fullref{thm:group_homomorphism_single_condition}.

  For Boolean \hyperref[def:boolean_algebra]{algebras}, however, complements are automatically preserved.
\end{proposition}

\begin{remark}\label{def:lattice_categorical_product}
  The existence of finite joins and meets is equivalent to the existence of finite products and coproducts in the corresponding category by \fullref{thm:partial_order_category_correspondence}.
\end{remark}

\begin{definition}\label{def:distributive_lattice}\MarginCite{nLab:distributive_lattice}
  A lattice \( (\CX, \vee, \wedge, \top, \bot) \) is called a \Def{distributive lattice} if any of the following two equivalent infinitary \hyperref[def:algebraic_theory/distributivity]{distributive} axioms hold for any \hyperref[def:indexed_family]{family} \( \{ x_k \}_{k \in \CK} \subseteq \CX \) and any \( y \in \CX \):
  \begin{align}\label{eq:def:distributive_lattice/distributivity}
    y \wedge \left( \bigvee_{k \in \CK} x_k \right) = \bigwedge_{k \in \CK} (y \vee x_k)
     &  &
    y \vee \left( \bigwedge_{k \in \CK} x_k \right) = \bigvee_{k \in \CK} (y \wedge x_k).
  \end{align}
\end{definition}

\begin{definition}\label{def:boolean_algebra}\MarginCite{nLab:boolean_algebra}
  Let \( \CX \) be a distributive lattice. A \Def{complement} of \( x \) is an element \( y \) of \( \CX \) such that
  \begin{align}\label{eq:def:boolean_algebra/complement}
    x \vee y = \top, && x \wedge y = \bot.
  \end{align}

  Since in a distributive lattice complements are unique (see \fullref{thm:boolean_algebra_properties/unique_complement}), the complement of \( x \) is denoted by \( \Ol x \). If all elements of \( \CX \) have complements, then \( (\CX, \vee, \wedge, \top, \bot, \Ol \cdot) \) is called a \Def{Boolean algebra}.
\end{definition}

\begin{example}\label{ex:boolean_algebras}
  Examples of \hyperref[def:boolean_algebra]{Boolean algebras} include:

  \begin{itemize}
    \item The cosets of propositional formulas under semantic equivalence (see \fullref{thm:propositional_logic_boolean_algebra}).
    \item The Galois field \( \BF_2 \) with suitably defined operations (see \fullref{thm:f2_is_boolean_algebra}).
    \item The power set of any set, usually taken to be a space with additional structure (see \fullref{thm:subsets_form_boolean_algebra}).
  \end{itemize}
\end{example}

\begin{proposition}\label{thm:boolean_algebra_properties}
  A Boolean algebra \( \CX \) has the following basic properties:
  \begin{PropEnum}
    \ILabel{thm:boolean_algebra_properties/unique_complement} For each \( x \in \CX \), there exists a unique complement \( \Ol x \).
    \ILabel{thm:boolean_algebra_properties/double_complement} For each \( x \in \CX \), we have \( x = \Ol{\Ol x} \).
  \end{PropEnum}
\end{proposition}
\begin{proof}
  \SubProofOf{thm:boolean_algebra_properties/unique_complement} If \( y \) and \( z \) are both complements of \( x \), then
  \begin{BreakableAlign*}
    y
    =
    y \wedge \top
    =
    y \wedge (z \vee x)
    =
    (y \wedge z) \vee (y \wedge x)
    = \\ =
    y \wedge z
    =
    (x \wedge z) \vee (y \wedge z)
    =
    (x \vee y) \wedge z
    =
    z.
  \end{BreakableAlign*}

  \SubProofOf{thm:boolean_algebra_properties/double_complement} Fix \( x \in \CX \). We have
  \begin{BreakableAlign*}
    \Ol{\Ol x}
    =
    \Ol{\Ol x} \wedge \top
    =
    \Ol{\Ol x} \wedge (\Ol x \vee x)
    =
    (\Ol{\Ol x} \wedge \Ol x) \vee (\Ol{\Ol x} \wedge x)
    = \\ =
    \Ol{\Ol x} \wedge x
    =
    (\Ol{\Ol x} \wedge x) \vee (\Ol x \wedge x)
    =
    (\Ol{\Ol x} \vee \Ol x) \wedge x
    =
    x.
  \end{BreakableAlign*}
\end{proof}

\begin{theorem}[De Morgan's laws]\label{thm:de_morgans_laws}
  If \( \CX \) is a \hyperref[def:boolean_algebra]{Boolean algebra}, the following hold for any \hyperref[def:indexed_family]{family} \( \{ x_k \}_{k \in \CK} \subseteq \CX \):
  \begin{align*}
    \Ol{\bigvee_{k \in \CK} x_k} = \bigwedge_{k \in \CK} \Ol{x_k}
    &&
    \Ol{\bigwedge_{k \in \CK} x_k} = \bigvee_{k \in \CK} \Ol{x_k}
  \end{align*}
\end{theorem}
\begin{proof}
  We will first show
  \begin{equation}\label{thm:de_morgans_laws/equation}
    \Ol{\bigvee_{k \in \CK} x_k} = \bigwedge_{k \in \CK} \Ol{x_k}
  \end{equation}

  In order for \( \wedge_{k \in \CK} \Ol{x_k} \) to be the complement of \( \vee_{k \in \CK} x_k \), the equations in \fullref{def:boolean_algebra} need to be satisfied.

  By distributivity,
  \begin{BreakableAlign*}
    \left( \bigwedge_{k \in \CK} \Ol{x_k} \right) \vee \left( \bigvee_{\beta \in \CK} x_\beta \right)
     & =
    \bigvee_{k \in \CK} \left( \Ol{x_k} \wedge \bigvee_{\beta \in \CK} x_\beta \right)
    =                                                                                                          \\ &=
    \bigvee_{k \in \CK} \bigwedge_{\beta \in \CK} (\Ol{x_k} \vee x_\beta)
    =                                                                                                          \\ &=
    \bigvee_{k \in \CK} \left( (\Ol{x_k} \vee x_k) \wedge \bigwedge_{\substack{\beta \in \CK                   \\ \beta \neq k}} (\Ol{x_k} \vee x_\beta) \right)
    =                                                                                                          \\ &=
    \bigvee_{k \in \CK} \left( \top \wedge \bigwedge_{\substack{\beta \in \CK                                  \\ \beta \neq k}} (\Ol{x_k} \vee x_\beta) \right)
    =                                                                                                          \\ &=
    \left( \bigwedge_{k \in \CK} \top \right) \vee \bigvee_{k \in \CK} \left( \bigvee_{\substack{\beta \in \CK \\ \beta \neq k}} (\Ol{x_k} \vee x_\beta) \right)
    =                                                                                                          \\ &=
    \top.
  \end{BreakableAlign*}

  Analogously,
  \begin{equation*}
    \left( \bigwedge_{k \in \CK} \Ol{x_k} \right) \wedge \left( \bigvee_{\beta \in \CK} x_\beta \right) = \bot.
  \end{equation*}

  This proves \fullref{thm:de_morgans_laws/equation}.

  The other law follows from this one by double \hyperref[thm:boolean_algebra_properties]{negation}:
  \begin{equation*}
    \Ol{\bigwedge_{k \in \CK} x_k}
    =
    \Ol{\bigwedge_{k \in \CK} \Ol{\Ol{x_k}}}
    =
    \Ol{\Ol{\bigvee_{k \in \CK} \Ol{x_k}}}
    =
    \bigvee_{k \in \CK} \Ol{x_k}.
  \end{equation*}
\end{proof}

\begin{proposition}\label{thm:binary_boolean_algebras_are_isomorphic}
  All two-element Boolean \hyperref[def:boolean_algebra]{algebras} are isomorphic.
\end{proposition}
\begin{proof}
  We will show that the two-element Boolean algebra
  \begin{BreakableAlign*}
     & \left( \Set{ \top, \bot }, \vee, \wedge, \top, \bot, \Ol \cdot \right)
  \end{BreakableAlign*}
  is isomorphic to the \hyperref[thm:galois_field_existence]{Galois field} \( \BF_2 \) with the Boolean algebra structure defined in \fullref{thm:f2_is_boolean_algebra}.

  There is an obvious bijection
  \begin{equation*}
    \varphi(x) \coloneqq \begin{cases}
      1, & x = \top \\
      0, & x = \bot
    \end{cases}.
  \end{equation*}

  We only need to verify that \( \varphi \) preserves meets and joints. But this is trivial:
  \begin{BreakableAlign*}
    \varphi(x \vee y)
     & =
    \begin{rcases}
      \begin{cases}
        0, & x = y = 0        \\
        1, & \text{otherwise}
      \end{cases}
    \end{rcases}
    =
    x \oplus y \oplus xy.
    \\
    \varphi(x \wedge y)
     & =
    \begin{rcases}
      \begin{cases}
        1, & x = y = 1        \\
        0, & \text{otherwise}
      \end{cases}
    \end{rcases}
    =
    xy.
  \end{BreakableAlign*}
\end{proof}

\begin{proposition}\label{thm:functions_into_boolean_algebra_form_boolean_algebra}
  The set \( \CF \coloneqq \Fun(\CS, \CX) \) of all functions from any set \( \CS \) into a Boolean algebra also form a Boolean algebra with the operations defined pointwise as
  \begin{RefList}
    \IRef{def:binary_lattice_operations/join} \( f \vee_{\CF} g \coloneqq (x \mapsto f(x) \vee_{\CX} g(x)) \)
    \IRef{def:binary_lattice_operations/meet} \( f \wedge_{\CF} g \coloneqq (x \mapsto f(x) \wedge_{\CX} g(x)) \)
    \IRef{def:lattice/top} \( \top_{\CF} \coloneqq (x \mapsto \top_{\CX}) \)
    \IRef{def:lattice/bottom} \( \bot_{\CF} \coloneqq (x \mapsto \bot_{\CX}) \)
    \IRef{eq:def:boolean_algebra/complement} \( \Ol{f \ }^{\CF} \coloneqq (x \mapsto \Ol{f(x)}^{\CX}) \)
  \end{RefList}

  Because of the canonical embedding \( \iota: \CX \to \Fun(\CS, \CX) \) that sends each member of \( \CX \) to the corresponding constant function, we can regard \( \CX \) is a subset of \( \Fun(\CS, \CX) \) and further ignore the distinction between the two Boolean algebra structures.
\end{proposition}
\begin{proof}
  Verifying that \( \CF \) is a Boolean algebra is trivial.
\end{proof}
