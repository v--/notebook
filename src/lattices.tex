\section{Lattices}\label{sec:lattices}

\begin{note}\label{note:infinite_join_meet}
  Suprema and infima in posets can be used to define operations named \underLine{joins} and \underLine{meets}\Tinyref{def:join_meet}, however there are also axioms for binary joins and meets\Tinyref{def:binary_join_meet}. If we are interested in infinitary joins and meets, however, we need to use the poset definition. This can be accomplished indirectly by
  \begin{enumerate}
    \item defining joins and meets axiomatically
    \item using them to define a partial order
    \item using the partial order to define infinitary joins and meets
  \end{enumerate}
\end{note}

\begin{definition}\label{def:binary_join_meet}
  Fix an arbitrary set $X$ and let $x, y \in X$. Define two binary operations
  \begin{itemize}
    \item the \underLine{join of $x$ and $y$}, $x \lor y$
    \item the \underLine{meet of $x$ and $y$}, $x \land y$
  \end{itemize}
  such that,
  \begin{description}
    \DItem{associativity}{def:binary_join_meet/associativity}
    \begin{align*}
      (x \lor y) \lor z = x \lor (y \lor z)
      &&
      (x \land y) \land z = x \land (y \land z).
    \end{align*}

    \DItem{commutativity}{def:binary_join_meet/commutativity}
    \begin{align*}
      x \lor y = y \lor x
      &&
      x \land y = y \land x.
    \end{align*}

    \DItem{absorption}{def:binary_join_meet/absorption}
    \begin{align*}
      x \lor (y \land x) = x
      &&
      x \land (y \lor x) = x.
    \end{align*}
  \end{description}

  We can use joins to define the partial order relation\Tinyref{def:order/partial}
  \begin{align*}
    x \leq y \iff x \lor y = y,
  \end{align*}
  thus $X$ is automatically a poset and this structure is compatible with~\cref{def:join_meet}.
\end{definition}
\begin{proof}
  We will prove that $\leq$ is indeed a partial order.
  \begin{description}
    \item[\ref{def:order/partial/reflexivity}] Direct consequence of \cref{thm:binary_join_meet_properties/idempotence}.
    \item[\ref{def:order/partial/antisymmetry}] Let $x \leq y$ and $y \leq x$, that is, $x \lor y = y$ and $y \lor x = x$. By~\ref{def:binary_join_meet/commutativity}, $x = y \lor x = x \lor y = y$.
    \item[\ref{def:order/partial/transitivity}] Let $x \leq y$ and $y \leq z$, that is, $x \lor y = y$ and $y \lor z = z$. Then, by~\ref{def:binary_join_meet/associativity},
    \begin{align*}
      z = y \lor z = (x \lor y) \lor z = x \lor (y \lor z) = x \lor z.
    \end{align*}
  \end{description}
\end{proof}

\begin{note}\label{note:binary_join_meet_order}
  We can analogously define $x \leq y \iff x \land y = x$. The resulting partial order would be the same.
\end{note}

\begin{proposition}\label{thm:binary_join_meet_properties}
  If $(X, \lor, \land)$ is a set with a binary join and a meet\Tinyref{def:binary_join_meet}, the following properties hold:
  \begin{defenum}
    \item\label{thm:binary_join_meet_properties/idempotence} both operations are \underLine{idempotent}, i.e. $x \lor x = x = x \land x$
  \end{defenum}
\end{proposition}
\begin{proof}
  \begin{description}
    \item[\ref{thm:binary_join_meet_properties/idempotence}] \ref{def:binary_join_meet/absorption} implies that
    \begin{align*}
      x \lor x = x \lor (x \land (x \lor x)) = x
    \end{align*}
    and analogously $x \land x = x$.
  \end{description}
\end{proof}

\begin{definition}\label{def:join_meet}
  Let $(X, \leq)$ be a poset\Tinyref{def:poset}. We define \underLine{joins $\lor$} and \underLine{meets $\land$} as the partial\Tinyref{def:function/partial} functions
  \begin{align*}
    &\lor: \Power(X) \to X
    &&\land: \Power(X) \to X
    \\
    &\lor(A) \coloneqq \sup X
    &&\land(A) \coloneqq \inf X.
  \end{align*}

  For finite sets, we usually use the infix notation $x_1 \lor \ldots \lor x_n$ instead of $\lor \{ x_1, \ldots, x_n \}$.
\end{definition}
\begin{proof}
  We first show that $\lor$ and $\land$ satisfy \cref{def:binary_join_meet}. Since suprema and infima are obviously associative and commutative, it remains only to show \ref{def:binary_join_meet/absorption}, that is, for any comparable $x, y \in X$,
  \begin{align*}
    x = \sup \{x, \inf \{ x, y \} \}
    &&
    x = \inf \{x, \sup \{ x, y \} \}.
  \end{align*}

  If $x \leq y$, then
  \begin{align*}
    \sup \{ x, \inf \{ x, y \} \} = \sup \{ x, x \} = x
    &&
    \inf \{ x, \sup \{ x, y \} \} = \inf \{ x, y \} = x.
  \end{align*}

  If $y \leq x$, then
  \begin{align*}
    \sup \{ x, \inf \{ x, y \} \} = \sup \{ x, y \} = x
    &&
    \inf \{ x, \sup \{ x, y \} \} = \inf \{ x, x \} = x.
  \end{align*}

  We now show that if the partial order $\leq$ was defined using binary joins and meets as in \cref{def:binary_join_meet}, then the original join $\lor$ and meet $\land$ are compatible with the binary $\sup$ and $\inf$.

  Fix $x, y \in X$. Since the functions $\lor$ and $\land$ are total, all binary suprema and infima exist. If $\sup \{ x, y \} = z$, then $z$ is the least element such that both $x \leq z$ and $y \leq z$. Thus
  \begin{align*}
    x \lor z = z = y \lor z.
  \end{align*}

  Hence
  \begin{align*}
    x \lor y = (x \lor (z \land x)) \lor y = x \lor z \lor y = x \lor (z \lor z) \lor y = \\ = (x \lor z) \lor (z \lor y) = z \lor z = z.
  \end{align*}

  Conversely, if $x \lor y = z$, by~\ref{def:binary_join_meet/absorption},
  \begin{align*}
    x \lor z = (x \land (y \lor x)) \lor z = (x \land z) \lor z = z,
  \end{align*}
  thus $x \leq z$. Analogously, $y \leq z$.

  If we assume that there exists $t \in X$ such that $x \leq t \leq z$ and $y \leq t \leq z$, then
  \begin{align*}
    t = t \lor x = t \lor x \lor y = t \lor z = z.
  \end{align*}

  Thus $z = \sup \{ x, y \}$.

  The equivalence between binary $\inf$ and $\land$ can be obtained analogously.
\end{proof}

\begin{definition}\label{def:lattice}
  A poset $(X, \leq)$ is called a \underLine{lattice} if it has
  \begin{description}
    \DItem{bottom}{def:lattice/top} a \underLine{top element} $\top$, such that $\top = \lor X$ (in particular, $\lor X$ exists).
    \DItem{top}{def:lattice/bottom} a \underLine{bottom element} $\bot$, such that $\bot = \land X$.
    \DItem{finite joins}{def:lattice/join} all finite joins\Tinyref{def:binary_join_meet} exist.
    \DItem{finite meets}{def:lattice/meet} all finite meets\Tinyref{def:binary_join_meet} exist.
  \end{description}

  If the join and meet are defined axiomatically, all finite joins and meets necessarily exist.

  If the last two properties hold for all joins and meets, not necessarily finite, we say that the lattice is a \underLine{full lattice}.
\end{definition}

\begin{note}\label{def:lattice_categorical_product}
  The existence of finite joins and meets is equivalent to the existence of finite products and coproducts in the corresponding category by \cref{thm:poset_iff_poset_category}.
\end{note}

\begin{definition}\label{def:distributive_lattice}\cite{nLab:distributive_lattice}
  A lattice $(X, \top, \bot, \lor, \land)$ is called a \underLine{distributive lattice} if any of the following two equivalent distributive axioms hold for all $x, y, z \in X$:
  \begin{description}
    \DItem{distributivity}{def:distributive_lattice/distributivity}
    \begin{align*}
      x \land (y \lor z) = (x \land y) \lor (x \land z)
      &&
      x \lor (y \land z) = (x \lor y) \land (x \lor z).
    \end{align*}
  \end{description}
\end{definition}

\begin{definition}\label{def:boolean_algebra}\cite{nLab:boolean_algebra}
  Let $X$ be a distributive lattice. A \underLine{complement} of $x$ is an element $y$ of $X$ such that
  \begin{align*}
    x \lor y = \top && x \land y = \bot.
  \end{align*}

  Since in a distributive lattice complements are unique\Tinyref{thm:boolean_algebra_properties/unique_complement}, the complement of $x$ is denoted by $\neg x$. If all elements of $X$ have complements, then $(X, \top, \bot, \lor, \land, \neg)$ is called a \underLine{Boolean algebra}.
\end{definition}

\begin{example}\label{ex:boolean_algebra}
  By \cref{thm:propositional_logic_boolean_algebra}, the equivalence classes of propositional formulas under semantic equivalence form a Boolean algebra.
\end{example}

\begin{proposition}\label{thm:boolean_algebra_properties}
  A Boolean algebra $X$ has the following basic properties:
  \begin{defenum}
    \item\label{thm:boolean_algebra_properties/unique_complement} For each $x \in X$, there exists a unique complement $\neg x$.
    \item\label{thm:boolean_algebra_properties/double_complement} For each $x \in X$, we have $x = \neg \neg x$.
  \end{defenum}
\end{proposition}
\begin{proof}\mbox{}
  \begin{itemize}
    \item[\ref{thm:boolean_algebra_properties/unique_complement}] If $y$ and $z$ are both complements of $x$, then
    \begin{align*}
      y
      =
      y \land \top
      =
      y \land (z \lor x)
      =
      (y \land z) \lor (y \land x)
      = \\ =
      y \land z
      =
      (x \land z) \lor (y \land z)
      =
      (x \lor y) \land z
      =
      z.
    \end{align*}

    \item[\ref{thm:boolean_algebra_properties/double_complement}] Fix $x \in X$. We have
    \begin{align*}
      \neg \neg x
      =
      \neg \neg x \land \top
      =
      \neg \neg x \land (\neg x \lor x)
      =
      (\neg \neg x \land \neg x) \lor (\neg \neg x \land x)
      = \\ =
      \neg \neg x \land x
      =
      (\neg \neg x \land x) \lor (\neg x \land x)
      =
      (\neg \neg x \lor \neg x) \land x
      =
      x.
    \end{align*}
  \end{itemize}
\end{proof}
