\subsection{Category adjunctions}\label{subsec:category_adjunctions}

\begin{remark}\label{rem:adjoint_functors}\cite{StanfordPlato:category_theory}
  When the functor \( G: \cat{D} \to \cat{C} \) is left adjoint to \( F: \cat{C} \to \cat{D} \) and \( F \) is not invertible, then \( G \) finds a \enquote{generalized inverse} under \( F \) for every object in \( \cat{C} \) that try to \enquote{act the same} with respect to morphisms.
\end{remark}

\begin{definition}\label{def:category_adjunction}\mcite[sec. 2.2]{Leinster2016Basic}
  An \term{adjunction} between the \hyperref[def:category]{categories} \( \cat{C} \) and \( \cat{D} \) is a quadruple \eqref{eq:def:category_equivalence/signature}, where \( \eta \) and \( \varepsilon \) are not necessarily isomorphisms, satisfying the condition that the following diagrams commute:
  \begin{equation}\label{eq:def:category_adjunction/diagrams}
    \begin{aligned}
      \includegraphics[page=1]{figures/def__category_adjunction.pdf}
      \quad\quad\quad\quad
      \includegraphics[page=2]{figures/def__category_adjunction.pdf}
    \end{aligned}
  \end{equation}

  The cross-composition of natural transformation and functors used above is defined in \fullref{def:composition_of_natural_transformation_and_functor}.

  We call the \hyperref[def:natural_transformation]{natural transformation} \( \eta \) the \term{unit} of the adjunction and \( \varepsilon \) the \term{counit}. If \( (F, G, \eta, \varepsilon) \) is an adjunction, we say that \( F \) is \term{left adjoint} to \( G \) and, correspondingly, that \( G \) is \term{right adjoint} to \( F \).

  A conventional notation for adjoint functors is \( F \dashv G \).

  Note that an adjunction is not an \hyperref[def:category_equivalence]{equivalence}, they simply have a common setup.
\end{definition}
