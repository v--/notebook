\subsection{Preorders}\label{subsec:preorders}

\begin{definition}\label{def:preordered_set}\mcite{nLab:preorder}
  A \hyperref[def:binary_relation]{binary relation} \( \leq \) on a set \( \mscrP \) is called a \term{preorder} if it is \hyperref[def:binary_relation/reflexive]{reflexive} and \hyperref[def:binary_relation/transitive]{transitive}.

  A \term{preordered set} is a set \( \mscrP \) equipped with a preorder \( \leq \). It is conventional to use the same symbol as for \hyperref[def:poset]{partial orders}, however the lack of \hyperref[def:binary_relation/antisymmetric]{antisymmetry} may be confusing --- see \fullref{ex:preorder_nonuniqueness}.

  \begin{thmenum}[series=def:preordered_set]
    \thmitem{def:preordered_set/theory} Consider a \hyperref[def:first_order_language]{first-order language} \( \mscrL \) with a single \hyperref[rem:order_infix_notation]{infix} binary predicate symbol \( \leq \). The \hyperref[def:first_order_semantics/theory]{theory} in \( \mscrL \) axiomatized by \hyperref[def:binary_relation/reflexive]{reflexivity} and \hyperref[def:binary_relation/transitive]{transitivity} for \( \leq \) is called the \term{theory of preordered sets}.

    \thmitem{def:preordered_set/homomorphism} A \hyperref[def:first_order_homomorphism]{homomorphism} between the preordered sets \( (\mscrP, \leq_\mscrP) \) and \( (\mscrQ, \leq_\mscrQ) \) is, explicitly, a function \( \varphi: \mscrP \to \mscrQ \) such that
    \begin{equation}\label{eq:def:preordered_set/homomorphism}
      x \leq_\mscrP y \T{implies} \varphi(x) \leq_\mscrQ \varphi(y)
    \end{equation}

    Such as function is called \term{monotone} or \term{order-preserving} or simply an \term{order homomorphism}.

    If
    \begin{equation*}
      x \neq y \T{implies} \varphi(x) \neq \varphi(y),
    \end{equation*}
    we call the function \( \varphi \) \term{strictly monotone}.

    In particular, if \( \mscrP \) is the set of \hyperref[rem:peano_arithmetic_zero/nonnegative]{nonnegative integers}, then we speak of \term{monotone sequences}
    \begin{equation*}
      \{ x_k \}_{k=1}^\infty,
    \end{equation*}
    where \( x_{k-1} \leq_Q x_k \) for all \( k = 1, 2, 3, \ldots \).

    \thmitem{def:preordered_set/substructure} Since the theory contains only positive formulas over a language with no functional symbols, any subset \( A \) of a preordered set \( (X, \leq) \) becomes a preordered set with the induced preorder \( \leq_A \) defined as the restriction of \( \leq \) to only elements of \( A \).

    \thmitem{def:preordered_set/category} We give no special name for the \hyperref[def:category_of_first_order_models]{category of models} for the theory of preordered sets.

    \thmitem{def:preordered_set/dual} The partially ordered set \( (\mscrP, \geq) \) where \( \geq \) is the \hyperref[def:binary_relation/converse]{converse relation}, is the called the \term{dual preordered set}. See \fullref{def:thin_category} for a discussion of the duality.
  \end{thmenum}
\end{definition}

\begin{definition}\label{def:directed_set}\mcite[8]{Engelking1989}
  A \hyperref[def:preordered_set]{preordered set} \( (\mscrP, \leq) \) is called a \term{directed set} if every finite subset of \( \mscrP \) has an upper \hyperref[def:preordered_set/upper_and_lower_bounds]{bound}, i.e. for all \( x, y \in \mscrP \) there exists \( z \in \mscrP \) such that \( x \leq z \) and \( y \leq z \).

  There is no established name for the relation itself.

  Note that \( \{ x, y \} \) may not have a supremum, i.e. the set of its upper bounds may not have a \hyperref[def:preordered_set/maximum_and_minimum]{smallest element}. Thus the upper bound condition is strictly weaker than every two-element set having a supremum.

  Directed sets are used to define nets in topological spaces, see \fullref{def:topological_net}.
\end{definition}

\begin{definition}\label{def:thin_category}\mcite{nLab:thin_category}
  A \hyperref[def:category]{category} \( \cat{P} \) is called a \term{thin category} if, for every two objects \( A, B \in \boldop{P} \), whenever \( f, g \in \boldop{P}(A, B) \), we have \( f = g \).

  If \( \cat{P} \) is locally small, this is equivalent to saying that any set of morphisms \( \boldop{P}(A, B) \) is at most a singleton.
\end{definition}

\begin{proposition}\label{thm:preorder_category_correspondence}
  To every \hyperref[def:preordered_set]{preordered set} there corresponds exactly one \hyperref[def:category_cardinality]{small} \hyperref[def:thin_category]{thin} category.

  Furthermore, \hyperref[def:preordered_set/supremum_and_infimum]{infima} correspond to categorical \hyperref[def:categorical_product]{products}, suprema to \hyperref[def:categorical_coproduct]{coproducts} and dual \hyperref[def:preordered_set/dual]{preordered sets} correspond to dual \hyperref[def:opposite_category]{categories}.

  Compare this result to \fullref{thm:partial_order_category_correspondence}.
\end{proposition}
\begin{proof}
  \SufficiencySubProof Let \( (P, \leq) \) be a preordered set. We define the category \( \cat{P} \) as follows:
  \begin{itemize}
    \item The \hyperref[def:category/C1]{class of objects} in \( \cat{P} \) is the set \( P \).
    \item The \hyperref[def:category/C2]{set of morphisms} between two elements \( x, y \in \cat{P} \) is the singleton \( \{ (x, y) \} \) when \( x \leq y \) and the empty set otherwise.
    \item The \hyperref[def:category/C3]{composition} of two morphisms \( (x, y) \) and \( (y, z) \) is simply \( (x, z) \) (such a morphism exists by transitivity of \( \leq \)).

    The axiom \ref{def:category/identity} follows from reflexivity of \( \leq \) and the axiom \ref{def:category/associativity} is trivial.
  \end{itemize}

  We showed that \( \cat{P} \) is indeed a category. We will only prove the equivalence of products and infima since the argument for suprema and coproducts is completely analogous.

  Let \( p \) be the categorical product of the set \( A \subseteq P \). Then \( p \leq x \) for all \( x \in A \), hence it is a lower bound. If \( q \) is another lower bound, then by definition of product, there exists a unique morphism \( q \leq p \). Therefore \( p = q \) is the infimum.

  \NecessitySubProof Now assume that \( \cat{P} \) is a thin small category. Define the relation \( \leq \) on the set \( \cat{P} \) as
  \begin{equation*}
    x \leq y \T{if and only if} \cat{P}(x, y) \neq \varnothing.
  \end{equation*}

  This is indeed a preorder because
  \begin{itemize}
    \item \( \leq \) is \hyperref[def:binary_relation/reflexive]{reflexivity} because of the existence of identity morphisms.
    \item \( \leq \) is \hyperref[def:binary_relation/transitive]{transitive} since if \( x \leq y \) and \( y \leq z \), composition of morphisms gives us \( x \leq z \).
  \end{itemize}

  Note that the infimum of a set \( A \subseteq \cat{P} \) (if it exists) has a unique morphism \( \inf A \) such that \( \inf A \leq x \) for any \( x \in A \). If \( y \leq x \) for all \( x \in A \) is another \hyperref[def:categorical_cone]{cone}, then necessarily \( y \leq \inf A \). Therefore the infimum is the categorical product.

  We proved that for each partially ordered set there corresponds at least one thin small category and vice versa. The fact that to each poset corresponds at most one category \( \cat{P} \) is obvious. Therefore we have a correspondence between the two.

  Duality is also obvious from our constructions.
\end{proof}
