\subsection{Manifolds}\label{subsec:manifolds}

\begin{definition}\label{def:atlas}\mcite[def. 12.1]{ИвановТужилин2017}
  Let \( X \) be a \hyperref[def:topological_space]{topological space} and \( Y \) be a \hyperref[def:topological_vector_space]{topological vector space}.

  A \term{coordinate chart} on \( X \) over \( Y \) is a pair \( (U_\alpha, \varphi_\alpha) \), where
  \begin{itemize}
    \item \( U_\alpha \subseteq X \) is a \hyperref[def:connected_space]{connected} open set.
    \item \( \varphi_\alpha: U_\alpha \to Y \) is a homeomorphic \hyperref[def:homeomorphism]{embedding}, called a \term{coordinate homeomorphism}.
  \end{itemize}

  An \term{atlas} on \( X \) over \( Y \) is an indexed \hyperref[def:tuple_and_cartesian_product/indexed_family]{family} \( \{ (U_\alpha, \varphi_\alpha) \}_{\alpha \in \mscrK} \) of charts such that the family \( \{ U_\alpha \}_{\alpha \in \mscrK} \) is a \hyperref[def:set_partition]{cover} of \( X \). If \( Y = \BbbK^n \) for \( \BbbK \in \{ \BbbR, \BbbC \} \), we say that \( X \) is a real (resp. complex) manifold of dimension \( n \).

  For any two coordinate homeomorphisms \( \varphi_\alpha \) and \( \varphi_\beta \) in an atlas, the function restriction of the composition \( \varphi_\alpha \circ \varphi_\alpha^{-1} \) to \( U_\alpha \cap U_\beta \) is a homeomorphism from \( \varphi_\alpha(U_\alpha \cap U_\beta) \) to \( \varphi_\beta(U_\alpha \cap U_\beta) \), called a \term{transition map}.
\end{definition}

\begin{definition}\label{def:topological_manifold}\mcite[def. 12.4]{ИвановТужилин2017}
  We call the topological space \( X \) a \term{topological manifold} if it has a countable \hyperref[def:atlas]{atlas}. If \( Y = \BbbR^n \), we say that \( X \) is a manifold of dimension \( n \).
\end{definition}

\begin{definition}\label{def:differentiable_manifold}\mcite[def. 12.6]{ИвановТужилин2017}
  We call the \hyperref[def:topological_manifold]{topological manifold} \( X \) a \term{smooth manifold} of type \( C^k \) if the transition maps are \( k \)-times continuously \hyperref[def:differentiability/frechet]{Frechet} differentiable.

  We also allow \( k = \infty \) for infinitely differentiable transition maps \( k = \omega \) for analytic transition maps.
\end{definition}
