\subsection{Manifolds}\label{subsec:manifolds}

\begin{remark}\label{rem:smooth_manifold}
  In the theory of \hyperref[def:topological_manifold]{manifolds}, the term \enquote{smooth} refers not to \enquote{infinitely differentiable} but to \enquote{\( r \) times differentiable} for some integer \( r \) depending on the context.
\end{remark}

\begin{definition}\label{def:smooth_curve}\mcite[8]{ИвановТужилин2017}
  A \term{smooth curve} is a \hyperref[def:parametric_curve]{parametric curve} \( \gamma: I \to \BbbR^n \) that is \( r \) times pointwise \hyperref[def:differentiability/frechet]{differentiable}. If
  \begin{equation*}
    \gamma(t) = \begin{pmatrix}
      \gamma_1(t) \\
      \vdots \\
      \gamma_n(t)
    \end{pmatrix},
  \end{equation*}
  then the \( k \)-th pointwise derivative is at \( t \in I \) is the vector
  \begin{equation*}
    \gamma^{(k)}(t) = \begin{pmatrix}
      \gamma_1^{(k)}(t) \\
      \vdots \\
      \gamma_n^{(k)}(t)
    \end{pmatrix}.
  \end{equation*}

  \begin{thmenum}
    \thmitem{def:smooth_curve/speed} If \( \gamma \) does not intersect itself at the point \( \gamma(t_0) \), we call \( \gamma'(t_0) \) the \term{speed vector} and norm \( \norm{\gamma'(t_0)} \) the \term{speed} of \( \gamma \) at the point \( \gamma(t_0) \).

    If the curve intersects itself at this point, it has multiple speed values, and the concept is ambiguous unless the vectors have equal norms.

    \thmitem{def:smooth_curve/regular} We say that the curve is \term{regular} if the speed vector is always nonzero.

    \thmitem{def:smooth_curve/natural_parameter} We say that the \hyperref[def:reparametrization]{reparametrization} \( t: J \to I \) is a \term{natural parameter} if, at any point \( t \in J \), we have \( \norm{\gamma'(t)} = 1 \).

    Of course, \( t \) can be the identity, in which case we say that the original parametrization is natural.
  \end{thmenum}
\end{definition}

\begin{proposition}\label{thm:regular_reparametrization}
  A \hyperref[def:function_regular_at_point]{regular} \hyperref[def:parametric_curve_reparametrization]{reparametrization} of a \hyperref[def:smooth_curve/regular]{regular} \hyperref[def:parametric_curve]{parametric curve} is regular.
\end{proposition}
\begin{proof}
  The composition of regular functions is regular.
\end{proof}

\begin{proposition}\label{thm:natural_reparametrization}
  Every \hyperref[def:smooth_curve/regular]{regular} \hyperref[def:parametric_curve]{parametric curve} has a \hyperref[def:smooth_curve/natural_parameter]{natural parametrization}.
\end{proposition}
\begin{proof}
  Consider the regular curve \( \gamma: I \to \BbbR^n \). Define the reparametrization
  \begin{equation*}
    s(t) \coloneqq \int_0^t \norm{\gamma'(\tau)} \cdot \dl \tau
  \end{equation*}
  over the same interval.

  \Fullref{thm:fundamental_theorem_of_calculus} implies that
  \begin{equation*}
    s'(t) = \norm{\gamma'(t)}.
  \end{equation*}

  Since \( \gamma \) is regular, \( t'(s) \) is nonzero. Hence, its inverse has a derivative
  \begin{equation*}
    t'(s) = \frac 1 {\norm{\gamma'(s)}}.
  \end{equation*}

  For \( \delta(s) \coloneqq \gamma(t(s)) \), \fullref{thm:chain_rule} implies
  \begin{equation*}
    \delta'(s)
    =
    t'(s) \cdot \gamma'(s)
    =
    \frac {\gamma'(s)} {\norm{\gamma'(s)}}.
  \end{equation*}

  Since \( t_0 \) was arbitrary, we conclude that this holds for the entire interval \( I \).
\end{proof}

\begin{example}\label{ex:natural_reparametrization_of_line}
  We can demonstrate the trick from \fullref{thm:natural_reparametrization} to reparametrize the line \( \gamma(t) = O + td \). Define
  \begin{equation*}
    s(t) \coloneqq \int_0^t \norm{\gamma'(\tau)} \cdot \dl \tau = t \norm{d}.
  \end{equation*}

  Its inverse is
  \begin{equation*}
    t(s) = \frac s {\norm{d}}.
  \end{equation*}

  Then
  \begin{equation*}
    \gamma(t(s)) = O + s \frac d {\norm{d}},
  \end{equation*}
  and this is obviously a natural reparametrization.
\end{example}

\begin{lemma}\label{thm:natural_reparametrization_lemma}
  For two \hyperref[def:smooth_curve/natural_parameter]{natural parameters} \( t_1: J_1 \to I \) and \( t_2: J_2 \to I \) of \( \gamma: I \to \BbbR^n \), there exists some constant \( a \) such that \( t_2(s) = t_1(a + s) \) for all \( s \in J_2 \).
\end{lemma}
\begin{proof}
  Define
  \begin{equation*}
    \delta_1(s) \coloneqq \gamma(t_1(s))
  \end{equation*}
  and similarly for \( \delta_2(s) \).

  We have
  \begin{equation*}
    \delta_1'(s) = t_1'(s) \cdot \gamma'(t_1(s)).
  \end{equation*}

  Since \( \delta_1 \) is natural, it follows that
  \begin{equation*}
    1 = \norm{\delta_1'(s)} = \abs{t_1'(s)} \cdot \norm{\gamma'(s)},
  \end{equation*}
  and similarly for \( \delta_2(s) \)

  Furthermore, since \( t_1 \) and \( t_2 \) are strictly monotone, their derivatives positive. Therefore,
  \begin{equation*}
    t_1'(s) = t_2'(s) = \frac 1 {\norm{\gamma'(s)}}.
  \end{equation*}

  Then \( t_1'(s) - t_2'(s) = 0 \), implying that the different between \( t_1(s) \) and \( t_2(s) \) is a constant.
\end{proof}

\begin{definition}\label{def:smooth_curve_curvature}
  If the naturally parametrized curve \( \gamma: I \to \BbbR^n \) does not intersect itself at the point \( \gamma(t_0) \), we call \( \gamma^\dprime(t_0) \) the \term{acceleration vector} and the norm \( \norm{\gamma^\dprime(t_0)} \) the \term{curvature} of \( \gamma \) at \( \gamma(t_0) \). \Fullref{thm:natural_reparametrization_lemma} ensures that this definition does not depend on the parametrization, as long as it is natural.

  If the curve intersects itself at \( \gamma(t_0) \), the concept is ambiguous.
\end{definition}

\begin{example}\label{ex:def:smooth_curve}
  We list several examples of \hyperref[def:smooth_curve]{smooth curves}:
  \begin{thmenum}
    \thmitem{ex:def:smooth_curve/line} An \hyperref[def:affine_line]{affine line} \( l(t) = O + td \) has the entire real line as its domain.

    The \hyperref[def:smooth_curve/speed]{speed} at any point is \( \norm{d} \). Hence, the curve is regular, but it may not be naturally parametrized. If it is not, we can obviously replace \( d \) with a normed vector and the parametrization would become natural.

    The second derivative is the zero vector, and hence the \hyperref[def:smooth_curve/curvature]{curvature} is zero.

    \thmitem{ex:def:smooth_curve/circle} Consider the \hyperref[def:circle]{circle} \( \gamma: [0, 2\pi) \to \BbbR^2 \) with \hyperref[eq:def:ellipse/parametric_equation]{parametric equation}
    \begin{equation*}
      \begin{cases}
        x = x_0 + r \cdot \cos \varphi, \\
        y = y_0 + r \cdot \sin \varphi.
      \end{cases}
    \end{equation*}

    The first derivative is
    \begin{equation*}
      \gamma'(\varphi) = r \cdot (-\sin \varphi, \cos \varphi).
    \end{equation*}

    The norm of this derivative is \( r \). A natural parameter is
    \begin{equation*}
      \varphi(\psi) \coloneqq \frac \psi r
    \end{equation*}
    so that, by \fullref{thm:chain_rule},
    \begin{equation*}
      \gamma'(\varphi(\psi)) = r \cdot \frac 1 r \cdot (-\sin \varphi(\psi), \cos \varphi(\psi)).
    \end{equation*}

    The second derivative is then
    \begin{equation*}
      \gamma^\dprime(\varphi(\psi)) = \frac 1 r \cdot (-\cos \varphi(\psi), -\sin \varphi(\psi)).
    \end{equation*}

    Therefore, the curvature of \( \gamma \) is \( \ifrac 1 r \).
  \end{thmenum}
\end{example}

\begin{definition}\label{def:atlas}\mcite[def. 12.1]{ИвановТужилин2017}
  Let \( X \) be a \hyperref[def:topological_space]{topological space} and \( Y \) be a \hyperref[def:topological_vector_space]{topological vector space}.

  A \term{coordinate chart} on \( X \) over \( Y \) is a pair \( (U_\alpha, \varphi_\alpha) \), where
  \begin{itemize}
    \item \( U_\alpha \subseteq X \) is a \hyperref[def:connected_space]{connected} open set.
    \item \( \varphi_\alpha: U_\alpha \to Y \) is a homeomorphic \hyperref[def:homeomorphism]{embedding}, called a \term{coordinate homeomorphism}.
  \end{itemize}

  An \term{atlas} on \( X \) over \( Y \) is an indexed \hyperref[def:cartesian_product/indexed_family]{family} \( \{ (U_\alpha, \varphi_\alpha) \}_{\alpha \in \mscrK} \) of charts such that the family \( \{ U_\alpha \}_{\alpha \in \mscrK} \) is a \hyperref[def:set_partition]{cover} of \( X \). If \( Y = \BbbK^n \) for \( \BbbK \in \{ \BbbR, \BbbC \} \), we say that \( X \) is a real (resp. complex) manifold of dimension \( n \).

  For any two coordinate homeomorphisms \( \varphi_\alpha \) and \( \varphi_\beta \) in an atlas, the function restriction of the composition \( \varphi_\alpha \circ \varphi_\alpha^{-1} \) to \( U_\alpha \cap U_\beta \) is a homeomorphism from \( \varphi_\alpha(U_\alpha \cap U_\beta) \) to \( \varphi_\beta(U_\alpha \cap U_\beta) \), called a \term{transition map}.
\end{definition}

\begin{definition}\label{def:topological_manifold}\mcite[def. 12.4]{ИвановТужилин2017}
  We call the topological space \( X \) a \term{topological manifold} if it has a countable \hyperref[def:atlas]{atlas}. If \( Y = \BbbR^n \), we say that \( X \) is a manifold of dimension \( n \).
\end{definition}

\begin{definition}\label{def:differentiable_manifold}\mcite[def. 12.6]{ИвановТужилин2017}
  We call the \hyperref[def:topological_manifold]{topological manifold} \( X \) a \term{smooth manifold} of type \( C^k \) if the transition maps are \( k \)-times continuously \hyperref[def:differentiability/frechet]{Frechet} differentiable.

  We also allow \( k = \infty \) for infinitely differentiable transition maps \( k = \omega \) for analytic transition maps.
\end{definition}
