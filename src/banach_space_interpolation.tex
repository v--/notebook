\subsection{Banach space interpolation}\label{subsec:banach_space_interpolation}

\begin{definition}\label{def:interpolated_topological_vector_space}\MarginCite[24]{Bergh1976}
  Let \( \BK \) be either the \hyperref[def:real_numbers]{field \( \BR \) of real numbers} or the \hyperref[def:real_numbers]{field \( \BC \) of complex numbers}.

  \begin{DefEnum}
    \ILabel{def:interpolated_topological_vector_space/compatibility} We say that two \hyperref[def:topological_vector_space]{topological vector spaces} \( \CX_0 \) and \( \CX_1 \) are \Def{compatible} if they can both be \hyperref[def:first_order_homomorphism/embedding]{embedded} \hyperref[def:global_continuity]{continuously} into a \hyperref[def:separation_axioms/T2]{Hausdorff} topological vector space \( \CU \), in which case we can regard them as subspaces of \( \CU \).

    In particular, both \( \CX_0 \) and \( \CX_1 \) are Hausdorff. We write \( \Ol{\CX} \coloneqq (\CX_0, \CX_1) \).

    \ILabel{def:interpolated_topological_vector_space/intersection} Denote by
    \begin{equation*}
      \Delta \Ol{\CX} \coloneqq \CX_0 \cap \CX_1
    \end{equation*}
    the \Def{intersection} of \( \CX_0 \) and \( \CX_1 \) (when regarded as subspaces of \( \CU \)).

    \ILabel{def:interpolated_topological_vector_space/sum} Denote by
    \begin{equation*}
      \Sigma \Ol{\CX} \coloneqq \( \CX_0 + \CX_1 \)
    \end{equation*}
    the \Def{sum} of \( \CX_0 \) and \( \CX_1 \). If \( x \in \Sigma \Ol{\CX} \), then there exist (possibly nonunique) vectors \( x_0 \in \CX_0 \) and \( x_1 \in \CX_1 \) such that \( x = x_0 + x_1 \).

    \ILabel{def:interpolated_topological_vector_space/intermediate_space} Let \( \Ol{\CX} \) be a pair of compatible spaces. We say that the space \( \CX \) is an \Def{intermediate} space for \( \Ol{\CX} \) if \( \Delta \Ol{\CX} \subseteq \CX \subseteq \Sigma \Ol{\CX} \).

    \ILabel{def:interpolated_topological_vector_space/morphisms} We introduce \hyperref[def:category/morphisms]{morphisms} between two compatible pairs \( \Ol{\CX} \) and \( \Ol{\CY} \) that are, strictly speaking, not \hyperref[def:function]{functions}. We identify an \Def{operator} \( \Ol{T}: \Ol{\CX} \to \Ol{\CY} \) between compatible pairs with a function \( T \) from \( \Sigma \Ol{\CX} \) to \( \Sigma \Ol{\CY} \). Unlike for functions between \( \Sigma \Ol{\CX} \) and \( \Sigma \Ol{\CY} \), however, we say that \( \Ol{T} \) has a property (e.g. linearity, continuity, boundedness or any property of functions in \( \Cat{TopVect}_\BK \)) if the corresponding property holds for the restrictions
    \begin{align*}
      T_0: \CX_0 \to \CY_0
      &&
      T_1: \CX_1 \to \CY_1
    \end{align*}
    of \( T \) to the underlying spaces. Note that the requirement for \( \Ol{T} \) to be continuous is weaker than the same requirement for \( T \). In practice, we conflate an operator \( \Ol{T} \) and its underlying function \( T \) and do not use different notations.

    \ILabel{def:interpolated_topological_vector_space/category} If \( \Cat{C} \) is a \hyperref[def:subcategory]{subcategory} of the \hyperref[def:category_of_topological_vector_spaces]{category \( \Cat{TopVect}_\BK \)} of topological vector spaces. We define the category \( \Cat{Interp}_{\Cat{C}} \) as follows:
    \begin{RefList}
      \IRef{def:category/objects} The \hyperref[def:set_zfc]{class} of objects is the class of all pairs of \hyperref[def:interpolated_topological_vector_space/compatibility]{compatible spaces}.
      \IRef{def:category/morphisms} The morphisms between two compatible pairs are the continuous linear operators \( T: \Ol{\CX} \to \Ol{\CY} \) between them, as defined in \fullref{def:interpolated_topological_vector_space/morphisms}.
      \IRef{def:category/composition} Composition of morphisms is the usual \hyperref[def:function/composition]{function composition} if we regard a morphism \( T: \Ol{\CX} \to \Ol{\CY} \) as a function from \( \Sigma \Ol{\CX} \) to \( \Sigma \Ol{\CY} \).
    \end{RefList}

    \ILabel{def:interpolated_topological_vector_space/interpolation_space} We say that the intermediate spaces \( \CX \) for \( \Ol{\CX} \) and \( \CY \) for \( \Ol{\CY} \) are \Def{interpolation spaces} with respect to \( \Ol{\CX} \) and \( \Ol{\CY} \) if, for any bounded linear \hyperref[def:interpolated_topological_vector_space/morphisms]{operator} \( T: \Ol{\CX} \to \Ol{\CY} \), we have \( T(\CX) \subseteq \CY \).

    In particular, an intermediate space \( \CX \) of \( \CX \) is an interpolation space for \( \CX \) if it is invariant under any bounded linear operator \( T \), i.e. \( T(\CX) \subseteq \CX \).
  \end{DefEnum}
\end{definition}
