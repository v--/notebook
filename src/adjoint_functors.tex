\subsection{Adjoint functors}\label{subsec:adjoint_functors}

\begin{definition}\label{def:adjoint_functor}\cite[definition 2.1.1]{Leinster2014}
  Let \( \Bold A \) and \( \Bold B \) be locally small categories and \( F: \Bold A \to \Bold B \) and \( G: \Bold B \to \Bold A \) be functors. Further assume that for every \( A \in \Bold A \) and \( B \in \Bold B \) we have an isomorphism
  \begin{equation*}
    \Cat{A}(A, G(B)) \overset {\varphi_{A, B}} {\cong} \Bold{B}(F(A), B),
  \end{equation*}
  where \( \Cat{A}(A, G(B)) \) and \( \Bold{B}(F(A), B) \) are regarded as objects in \( \Bold{Set} \).

  Given a morphism \( f: A \to G(B) \), we define the \Def{transpose \( \Ol f \) of \( f \)} as
  \begin{align*}
    &\Ol f: F(A) \to B \\
    &\Ol f \coloneqq \varphi_{A, B} (f).
  \end{align*}

  Dually, given a morphism \( g: F(A) \to B \), we define
  \begin{align*}
    &\Ol g: A \to G(B) \\
    &\Ol g \coloneqq \varphi_{A, B}^{-1} (g).
  \end{align*}

  We say that the isomorphism \( \varphi_{A, B} \) is \Def{natural} if,  given \( A' \in \Bold A \) and morphisms \( f: A \to G(B) \) and \( p: A' \to A \), we have
  \begin{equation*}
    \Ol{f \circ p} = \Ol f \circ F(p),
  \end{equation*}
  and, given \( B' \in \Bold B \) and morphisms \( g: F(A) \to B \) and \( q: B \to B' \), we have
  \begin{equation*}
    \Ol{q \circ g} = G(q) \circ \Ol g.
  \end{equation*}

  In this case, we say that \( F \) is \Def{left-adjoint} to \( G \) and \( G \) is \Def{right-adjoint} to \( F \), and write \( F \dashv G \).
\end{definition}

\begin{example}\label{ex:top_adjoint_functor}\cite[example 2.1.5]{Leinster2014}
  Consider the functors
  \begin{itemize}
    \item \( U: \Cat{Top} \to \Bold{Set} \), which maps topological spaces to their underlying sets.
    \item \( D: \Cat{Set} \to \Bold{Top} \), which maps sets to topological spaces equipped with the discrete topology\Tinyref{def:standard_topologies/discrete}.
    \item \( I: \Cat{Set} \to \Bold{Top} \), which maps sets to topological spaces equipped with the indiscrete topology\Tinyref{def:standard_topologies/indiscrete}.
  \end{itemize}

  Let \( T \in \Cat{Top} \) and \( S \in \Bold{Set} \).

  Let \( f: T \to I(S) \) be any continuous function and \( g: U(T) \to S \) be any function.

  Denote by \( \Ol f: U(T) \to S \) the function between sets, corresponding to \( f \) and by \( \Ol g: T \to I(S) \) the corresponding function between the topological spaces \( T \) and \( I(S) \). Since any function into an indiscrete topological space is \( T \) is continuous, we have that \( \Ol g \) is a morphism \( T \to I(S) \).

  Thus \( \Ol{\Ol f} = f \) and \( \Ol{\Ol g} = g \) and we have a natural isomorphism between \( \Cat{Set}(U(T), S) \) and \( \Bold{Top}(T, I(S)) \). This proves that \( U \dashv I \).

  Similarly, since any function from a discrete space is continuous, we have that \( D \dashv U \).

  Hence \( D \dashv U \dashv I \).
\end{example}
