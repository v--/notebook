\begin{note}
  The definitions here are somewhat informal because of set-theoretic difficulties.
\end{note}

\begin{definition}\label{def:category}(\cite[definition 1.1.1]{Leinster2014})
  A \uline{category} $C$ consists of
  \begin{itemize}
    \item a (set-theoretic) class $\Obj(\CCal)$ of \uline{objects}
    \item for each pair of objects $A, B \in \Obj(\CCal)$, a class $\CCal(A, B)$ of \uline{morphisms} (also called \uline{arrows})
    \item for each triple of objects $A, B, C \in \Obj(\CCal)$, a function
    \begin{align*}
      \circ: \CCal(B, C) \times \CCal(A, B) \to \CCal(A, C)
    \end{align*}
    called the composition $g \circ f$ of $f \in \CCal(A, B)$ and $g \in \CCal(B, C)$. The reverse order comes from composition of functions.
  \end{itemize}
  such that
  \begin{enumerate}
    \item for each object $A \in \Obj(\CCal)$, there exists an identity morphism $\Id_A \in \CCal(A, A)$.
    \item composition is associative, i.e. for each $f \in \CCal(A, B)$, $g \in \CCal(B, C)$ and $h \in \CCal(C, D)$, we have
    \begin{align*}
      (h \circ g) \circ f = h \circ (g \circ f).
    \end{align*}
    \item for each $f \in \CCal(A, B)$, we have
    \begin{align*}
      f \circ \Id_A = \Id_B \circ f = f.
    \end{align*}
  \end{enumerate}
\end{definition}

\begin{definition}[Invertability (compare with \cref{def:function_invertability})]\label{def:morphism_invertability}
  Let $f: A \to B$ be a morphism in some category $\CCal$.

  \begin{itemize}
    \item $f$ is called \uline{left-invertible} if there exists a morphism $g: B \to A$ such that $g \circ f = \Id_A$. In this case we call $g$ a \uline{left inverse of $f$}.

    \item $f$ is called \uline{right-invertible} if there exists a morphism $g: B \to A$ such that $f \circ g = \Id_B$. In this case we call $g$ a \uline{right inverse of $f$}.

    \item $f$ is called \uline{invertible} or an \uline{isomorphism} if there exists a morphism $g: B \to A$ that is both a left and a right inverse of $f$. In this case we call $g$ a (two-sided) \uline{inverse of $f$} and we say that the objects $A$ and $B$ are isomorphic.
  \end{itemize}
\end{definition}

\begin{proposition}\label{ex:indiscrete_topology_universal_property}(\cite[exercise 1.1.13]{Leinster2014})
  A morphism $f: A \to B$ in any category $\CCal$ can have at most one inverse.
\end{proposition}
\begin{proof}
  If $f$ has no inverse, it has at most one inverse and the theorem follows.

  Now assume that $f$ has two inverses $g$ and $h$, i.e.
  \begin{align*}
    g \circ f = \Id_A && &f \circ g = \Id_B,
    \\
    h \circ f = \Id_A && &f \circ h = \Id_B.
  \end{align*}

  It follows that $g = h$ since
  \begin{align*}
    g
    =
    g \circ \Id_B
    =
    g \circ (f \circ h)
    =
    (g \circ f) \circ h
    =
    \Id_A \circ h
    =
    h.
  \end{align*}
\end{proof}

\begin{example}\label{ex:indiscrete_topology_universal_property}(\cite[exercise 0.10]{Leinster2014})
  Let $S$ be a set. The indiscrete topological space $I(S)$ and the canonical projection $p: I(S) \to S$ are characterized by the universal property \enquote{for any topological space $X$ and any function $f: X \to S$, there exists a unique continuous function $\tilde f$ such that $p \circ \tilde f = f$}

  \begin{figure}[ht]
    \center
    \begin{tikzcd}
    S & I(S) \arrow[l, "p"'] \\
      & \forall X \arrow[lu, "\forall \text{ functions } f"] \arrow[u, "\exists! \text{ continuous } \tilde f"']
    \end{tikzcd}
  \end{figure}
\end{example}
\begin{proof}
  Obviously $I(S)$ and $p$ exist. Assume they are not unique. Let the topological space $Y$ and the function $r: Y \to S$ satisfy the same universal property.

  Then by the universal property, there exist unique continuous functions $\tilde p: I(S) \to Y$ and $\tilde r: Y \to I(S)$ such that
  \begin{align*}
    r \circ \tilde p = p
    &&
    p \circ \tilde r = r.
  \end{align*}

  Hence $p = r \circ \tilde p = p \circ \tilde r \circ \tilde p$ and $\tilde r \circ \tilde p = \Id_{I(S)}$.

  Analogously, $r = p \circ \tilde r = r \circ \tilde p \circ \tilde r$, so $\tilde p \circ \tilde r = \Id_Y$.

  Thus $\tilde r$ and $\tilde p$ are mutually inverse and $I(S)$ is isomorphic to $Y$.
\end{proof}
