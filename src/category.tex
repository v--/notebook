\begin{note}
  The definitions here are somewhat informal because of set-theoretic difficulties.
\end{note}

\begin{definition}\label{def:category}(\cite[definition 1.1.1]{Leinster2014})
  A \uline{category} $C$ consists of
  \begin{itemize}
    \item a (set-theoretic) class $\Obj(\Cal C)$ of \uline{objects}
    \item for each pair of objects $A, B \in \Obj(\Cal C)$, a class ${\Cal C}(A, B)$ of \uline{morphisms} (also called \uline{arrows})
    \item for each triple of objects $A, B, C \in \Obj(\Cal C)$, a function
    \begin{align*}
      \circ: {\Cal C}(B, C) \times {\Cal C}(A, B) \to {\Cal C}(A, C)
    \end{align*}
    called the composition $g \circ f$ of $f \in {\Cal C}(A, B)$ and $g \in {\Cal C}(B, C)$. The reverse order comes from composition of functions.
  \end{itemize}
  such that
  \begin{defenum}
    \item for each object $A \in \Obj(\Cal C)$, there exists an identity morphism $\Id_A \in {\Cal C}(A, A)$.
    \item composition is associative, i.e. for each $f \in {\Cal C}(A, B)$, $g \in {\Cal C}(B, C)$ and $h \in {\Cal C}(C, D)$, we have
    \begin{align*}
      (h \circ g) \circ f = h \circ (g \circ f).
    \end{align*}
    \item for each $f \in {\Cal C}(A, B)$, we have
    \begin{align*}
      f \circ \Id_A = \Id_B \circ f = f.
    \end{align*}
  \end{defenum}

  If there are no morphisms in $\Cal C$ besides identity morphisms, we say that $\Cal C$ is a \uline{discrete category} (\cite[example 1.1.18(b)]{Leinster2014}).
\end{definition}

\begin{definition}\label{def:standard_categories}
  \begin{defenum}
    \item\label{def:standard_categories/set} The category $\Cat{Set}$ consists of all sets, the morphisms being all functions
    \item\label{def:standard_categories/top} The category $\Cat{Top}$ consists of all topological spaces, the morphisms being continuous functions
    \item\label{def:standard_categories/grp} The category $\Cat{Grp}$ consists of all groups, the morphisms being group homomorphisms
    \item\label{def:standard_categories/vect} The category $\Cat{Vect}_k$ consists of all vector spaces over a field $k$, the morphisms being linear maps
    \item\label{def:standard_categories/ord} Given a partially ordered set $(X, \leq)$ (see \cref{def:orders}), we can form the category $\Cal{C}$ by defining
    \begin{itemize}
      \item the objects to be $\Obj \Cal{C} \coloneqq X$
      \item for any $a, b \in X$, the morphisms are $f: a \to b \iff a \leq b$.
    \end{itemize}
  \end{defenum}
\end{definition}

\begin{definition}[Invertability (compare with \cref{def:function_invertability})]\label{def:morphism_invertability}
  Let $f: A \to B$ be a morphism in some category $\Cal C$.

  \begin{itemize}
    \item $f$ is called \uline{left-invertible} if there exists a morphism $g: B \to A$ such that $g \circ f = \Id_A$. In this case we call $g$ a \uline{left inverse of $f$}.

    \item $f$ is called \uline{right-invertible} if there exists a morphism $g: B \to A$ such that $f \circ g = \Id_B$. In this case we call $g$ a \uline{right inverse of $f$}.

    \item $f$ is called \uline{invertible} or an \uline{isomorphism} if there exists a morphism $g: B \to A$ that is both a left and a right inverse of $f$. In this case we call $g$ a (two-sided) \uline{inverse of $f$} and we say that the objects $A$ and $B$ are isomorphic.
  \end{itemize}
\end{definition}

\begin{proposition}\label{ex:at_most_one_isomorphism}(\cite[exercise 1.1.13]{Leinster2014})
  A morphism $f: A \to B$ in any category $\Cal C$ can have at most one inverse.
\end{proposition}
\begin{proof}
  If $f$ has no inverse, it has at most one inverse and the theorem follows.

  Now assume that $f$ has two inverses $g$ and $h$, i.e.
  \begin{align*}
    g \circ f = \Id_A && &f \circ g = \Id_B,
    \\
    h \circ f = \Id_A && &f \circ h = \Id_B.
  \end{align*}

  It follows that $g = h$ since
  \begin{align*}
    g
    =
    g \circ \Id_B
    =
    g \circ (f \circ h)
    =
    (g \circ f) \circ h
    =
    \Id_A \circ h
    =
    h.
  \end{align*}
\end{proof}

\begin{example}\label{ex:indiscrete_topology_universal_property}(\cite[exercise 0.10]{Leinster2014})
  Let $S$ be a set. The indiscrete topological space $I(S)$ and the canonical projection $p: I(S) \to S$ are characterized by the universal property \enquote{for any topological space $X$ and any function $f: X \to S$, there exists a unique continuous function $\tilde f$ such that $p \circ \tilde f = f$}

  \begin{figure}[ht]
    \center
    \begin{tikzcd}
    S & I(S) \arrow[l, "p"'] \\
      & \forall X \arrow[lu, "\forall \text{ functions } f"] \arrow[u, "\exists! \text{ continuous } \tilde f"']
    \end{tikzcd}
  \end{figure}
\end{example}
\begin{proof}
  Obviously $I(S)$ and $p$ exist. Assume they are not unique. Let the topological space $Y$ and the function $r: Y \to S$ satisfy the same universal property.

  Then by the universal property, there exist unique continuous functions $\tilde p: I(S) \to Y$ and $\tilde r: Y \to I(S)$ such that
  \begin{align*}
    r \circ \tilde p = p
    &&
    p \circ \tilde r = r.
  \end{align*}

  Hence $p = r \circ \tilde p = p \circ \tilde r \circ \tilde p$ and $\tilde r \circ \tilde p = \Id_{I(S)}$.

  Analogously, $r = p \circ \tilde r = r \circ \tilde p \circ \tilde r$, so $\tilde p \circ \tilde r = \Id_Y$.

  Thus $\tilde r$ and $\tilde p$ are mutually inverse and $I(S)$ is isomorphic to $Y$.
\end{proof}

\begin{definition}\label{def:dual_category}(\cite[construction 1.1.9]{Leinster2014})
  The \uline{dual category} of $\Cal C$ is the category $\Cal C^{\Op}$ such that
  \begin{defenum}
    \item $\Obj(\Cal C^{\Op}) = \Obj(\Cal C)$
    \item $f \in \Cal{C}^{\Op}(A, B) \iff f \in \Cal{C}(B, A)$, i.e. the morphisms are reversed.
  \end{defenum}
\end{definition}

\begin{example}
  The category $\Cat{Set}^{\Op}$ has a morphism $f: A \to B$ precisely when there exists a function $f$ from the set $B$ to the set $A$. If $f: A \to B$ is not invertible, then $f$ is not a function.
\end{example}

\begin{definition}\label{def:subcategory}(\cite[definition 1.2.18]{Leinster2014})
  We call the category $\Cal B$ a \uline{subcategory} of $\Cal A$ if
  \begin{itemize}
    \item $\Obj \Cal B \subseteq \Obj \Cal A$
    \item $f \in \Cal{B}(A, B) \implies f \in \Cal{A}(A, B)$
  \end{itemize}

  In case $\Cal{A}(A, B) = \Cal{A}(A, B)$ for all objects $A, B \in \Obj \Cal B$, we say that $\Cal B$ is a full subcategory.
\end{definition}
