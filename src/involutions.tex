\subsection{Involutions}\label{subsec:involutions}

\begin{definition}\label{def:set_with_involution}
  A \term{set with an involution} is quite literally a \hyperref[def:set]{set} \( \mscrX \) with an \hyperref[def:multi_valued_function/involution]{involution}, usually denoted by \( (\placeholder)^{-1} \) or \( \overline \placeholder \).

  \begin{thmenum}
    \thmitem{def:set_with_involution/theory} We define the theory of sets with involution as a theory over the language consisting of a single unary functional symbol \( i \) with the, sole axiom
    \begin{equation}\label{eq:def:set_with_involution/theory/axiom}
      (x^{-1})^{-1} \doteq x.
    \end{equation}

    \thmitem{def:set_with_involution/homomorphism} A \hyperref[def:first_order_homomorphism]{homomorphism} between sets with involutions \( \mscrX \) and \( \mscrY \) is a function \( \varphi: \mscrX \to \mscrY \) satisfying
    \begin{equation}\label{eq:def:set_with_involution/homomorphism}
      \varphi(x^{-1})
      =
      \varphi(x)^{-1}.
    \end{equation}

    \thmitem{def:set_with_involution/substructure} Any subset of a set with involution is again a set with involution.

    \thmitem{def:set_with_involution/category} We denote the \hyperref[def:category_of_small_first_order_models]{category of models} for the theory of sets with involutions by \( \cat{Inv} \).
  \end{thmenum}
\end{definition}
