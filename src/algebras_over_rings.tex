\subsection{Algebras over rings}\label{subsec:algebras_over_rings}

\begin{definition}\label{def:algebra_over_ring}
  An \term{algebra} over a commutative ring \( R \) rather than over a \hyperref[def:algebra_over_semiring]{semiring} exhibits some more interesting metamathematical properties.

  \begin{thmenum}
    \thmitem{def:algebra_over_ring/theory} The first-order theory is identical to the \hyperref[def:agebra_over_semiring/theory]{theory of algebras over semimodules}.

    \thmitem{def:algebra_over_ring/homomorphism} A \hyperref[def:first_order_homomorphism]{first-order homomorphism} between two \( R \)-algebras \( A \) and \( B \) is a \hyperref[def:semimodule/homomorphism]{linear map} that preserves multiplication. This is the same as for semirings.

    \thmitem{def:algebra_over_ring/submodel} The set \( A \subseteq M \) is a \hyperref[thm:substructure_is_model]{submodel} of \( M \) if it is a \hyperref[def:monoid/submodel]{submodule} of \( M \) that is closed under algebra multiplication.

    As a consequence of \fullref{thm:positive_formulas_preserved_under_homomorphism}, the image of an \( R \)-algebra homomorphism is a subalgebra of its range.

    \thmitem{def:algebra_over_ring/trivial} The \hyperref[thm:substructures_form_complete_lattice/bottom]{trivial} \( R \)-algebra is the \hyperref[def:pointed_set/trivial]{trivial pointed set} \( \set{ 0_R } \).

    \thmitem{def:algebra_over_ring/category} For a fixed ring \( R \), we denote the \hyperref[def:category_of_small_first_order_models]{category of \( \mscrU \)-small models} by \( \ucat{Alg}_R \). It is concrete with respect to both \( \ucat{CRing} \) and \( \ucat{Mod}_R \).

    Unfortunately, these categories are not as well-behaved as categories of modules.

    \thmitem{def:algebra_over_ring/kernel} The \term{kernel} of an \( R \)-algebra homomorphism \( \varphi: M \to N \) is its \hyperref[def:zero_locus]{zero locus} \( \varphi^{-1}(0_N) \).

    The kernel of a homomorphism is a both a submodule and a subring of \( M \). It is the kernel of the underlying group, ring and module, and the \hyperref[def:zero_morphisms/cokernel]{categorical kernel} in the category of modules.

    \thmitem{def:algebra_over_ring/quotient} The \hyperref[def:zero_morphisms/cokernel]{categorical cokernel} of an \( R \)-homomorphism \( \varphi: M \to N \) in the category \( \cat{Alg}_R \) is simply \hyperref[def:ring/quotient]{quotient ring} \( N / \img \varphi \).

    More generally, let \( I \) be a two-sided ideal of \( M \). We identify each member \( x \) of \( R \) with its embedding \( x \cdot 1_R \). Since \( I \) is closed under multiplication with elements of \( M \), it is also closed under multiplication with elements of \( R \). Thus, \( I \) is a submodule of the \( R \)-module \( M \).

    Thus, given an ideal \( I \) of \( M \), \( M / I \) is both a quotient ring and a quotient module.
  \end{thmenum}
\end{definition}

\begin{proposition}\label{thm:ring_is_integer_algebra}
  The categories \( \hyperref[def:ring/category]{\cat{Ring}} \) of rings and \( \hyperref[def:algebra_over_ring/category]{\cat{Alg}_\BbbZ} \) of integer algebras are \hyperref[rem:category_similarity/isomorphism]{isomorphic}.

  Compare this result to \fullref{thm:abelian_group_is_module}.
\end{proposition}
\begin{proof}
  Follows from \fullref{thm:commutative_monoid_is_semimodule} by noting that, as in the proof of \fullref{thm:semiring_is_algebra}, distributivity implies bilinearity.
\end{proof}

\begin{theorem}[Quotient algebra universal property]\label{thm:quotient_algebra_universal_property}
  For every \hyperref[def:algebra_over_ring]{unital \( R \)-algebra} \( M \) and \hyperref[def:semiring_ideal]{ideal} \( I \) of \( M \), the \hyperref[def:algebra_over_ring/quotient]{quotient algebra} \( M / I \) has the following \hyperref[rem:universal_mapping_property]{universal mapping property}:
  \begin{displayquote}
    Every ring homomorphism \( \varphi: M \to N \) satisfying \( I \subseteq \ker \varphi \) \hyperref[def:factors_through]{uniquely factors through} \( M / I \). That is, there exists a unique homomorphism \( \widetilde{\varphi}: M / I \to N \), such that the following diagram commutes:
    \begin{equation}\label{eq:thm:quotient_algebra_universal_property/diagram}
      \begin{aligned}
        \includegraphics[page=1]{output/thm__quotient_algebra_universal_property.pdf}
      \end{aligned}
    \end{equation}

    In the case where \( I = \ker \varphi \), \( \widetilde{\varphi} \) is an \hyperref[def:first_order_homomorphism_invertibility/embedding]{embedding}.
  \end{displayquote}

  Compare this result to \fullref{thm:quotient_group_universal_property} and \fullref{thm:quotient_module_universal_property}.
\end{theorem}
\begin{proof}
  Simple refinement of \fullref{thm:quotient_group_universal_property}.
\end{proof}

\begin{definition}\label{def:algebra_presentation}
  For a \hyperref[def:commutative_ring]{commutative ring} \( R \), a \term{presentation} of the \( R \)-\hyperref[def:algebra_over_ring]{algebra} \( M \) is a surjective \hyperref[def:module/homomorphism]{homomorphism} \( \varphi: R[\mscrS] \to M \) (epimorphisms may be too general), where \( R[\mscrS] \) is a \hyperref[def:polynomial_semiring]{polynomial ring} with indeterminates \( \mscrS \).

  By \fullref{thm:quotient_algebra_universal_property},
  \begin{equation*}
    M = \img \varphi \cong R[\mscrS] / \ker \varphi.
  \end{equation*}

  Analogously to \hyperref[def:group_presentation]{group presentations}, we say that \( M \) is finitely generated/related/presented if there exists an appropriate presentation.
\end{definition}

\begin{theorem}[Quotient ideal lattice theorem]\label{thm:quotient_ideal_lattice_theorem}
  Given a \hyperref[def:semiring_ideal]{two-sided ideal} \( I \) of the \hyperref[def:algebra_over_ring]{unital \( R \)-algebra} \( M \), the function \( N \mapsto N / I \) is a \hyperref[def:semilattice/homomorphism]{lattice isomorphism} between the lattice of \hyperref[def:ring/submodel]{subrings} of \( M \) containing \( I \) and the lattice of subrings of the \hyperref[def:ring/quotient]{quotient} \( M / I \).

  Furthermore, the sublattices of \hyperref[def:semiring_ideal/prime]{prime}, \hyperref[def:semiring_ideal/maximal]{maximal} or \hyperref[def:radical_ideal]{radical} ideals are also isomorphic.

  Note that, by \fullref{thm:ring_is_integer_algebra}, usual rings can be regarded as \( \BbbZ \)-algebras.

  Compare this result to \fullref{thm:quotient_subgroup_lattice_theorem} and \fullref{thm:quotient_submodule_lattice_theorem}.
\end{theorem}
\begin{proof}
  \SubProof{Proof for general ideals} Simple refinement of \fullref{thm:quotient_subgroup_lattice_theorem}.

  \SubProof{Proof for prime ideals}
  \SufficiencySubProof* Suppose that \( P \) is a prime ideal in \( M \) containing \( I \). Let \( A \) and \( B \) be ideals of \( P \) containing \( I \), such that \( (A / I) (B / I) \subseteq P / I \). By the general ideal lattice isomorphism, \( AB \subseteq P \), which implies that \( A \subseteq P \) or \( B \subseteq P \). Again using the general isomorphism, we conclude that \( A / I \subseteq P / I \) or \( B / I \subseteq P / I \), meaning that \( P / I \) is a prime ideal in \( M / I \).

  \NecessitySubProof* Suppose that \( P / I \) is a prime ideal in \( M / I \) and let \( AB \subseteq P \). Note that neither \( A \) nor \( B \) do not necessarily contain \( I \), but \( A + I \) and \( B + I \) do. Furthermore, \( A + I \subseteq P \) and \( B + I \subseteq P \).

  We have
  \begin{equation*}
    [(A + I) / I][(B + I) / I]
    \reloset {\ref{thm:def:semiring_ideal/properties/coprime_product}} \subseteq
    [(A + I) / I] \cap [(B + I) / I]
    \subseteq
    P / I.
  \end{equation*}

  Since \( P / I \) is prime, \( (A + I) / I \subseteq P / I \) or \( (B + I) / I \subseteq P / I \). Again using the general lattice isomorphism, we conclude that \( A \subseteq A + I \subseteq P \) or \( B \subseteq B + I \subseteq P \).

  \SubProof{Proof for maximal ideals} Trivial consequence of the general lattice isomorphism.

  \SubProof{Proof for radical ideals} Correspondence of radical ideals follows from correspondence of prime ideals since any radical ideal satisfies \fullref{def:radical_ideal/intersection} and thus equals the intersection of all prime ideals containing it.
\end{proof}

\begin{corollary}\label{thm:quotient_by_maximal_ideal}
  The two-sided ideal \( I \) of the \hyperref[def:ring]{ring} \( R \) is \hyperref[def:semiring_ideal/maximal]{maximal} if and only if the \hyperref[def:ring/quotient]{quotient} \( R / M \) is a \hyperref[def:ring/simple]{simple ring}.

  See \fullref{thm:quotient_by_prime_ideal} for the corresponding statement for \hyperref[def:semiring_ideal/prime]{prime ideals} in commutative rings.
\end{corollary}
\begin{proof}
  Since \( M \) is maximal, only \( M \) and \( R \) are ideals of \( R \) containing \( M \). Therefore, by \fullref{thm:quotient_ideal_lattice_theorem}, \( R / M \) has only two ideals.
\end{proof}
