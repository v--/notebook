\subsection{Algebras over rings}\label{subsec:algebras_over_rings}

\begin{definition}\label{def:algebra_over_ring}
  An \term{algebra} over a commutative ring \( R \) rather than over a \hyperref[def:algebra_over_semiring]{semiring} exhibits some more interesting metamathematical properties.

  \begin{thmenum}
    \thmitem{def:algebra_over_ring/theory} The first-order theory is identical to the \hyperref[def:agebra_over_semiring/theory]{theory of algebras over semimodules}.

    \thmitem{def:algebra_over_ring/homomorphism} A \hyperref[def:first_order_homomorphism]{first-order homomorphism} between two \( R \)-algebras \( A \) and \( B \) is a \hyperref[def:semimodule/homomorphism]{linear map} that preserves multiplication. This is the same as for semirings.

    \thmitem{def:algebra_over_ring/submodel} The set \( A \subseteq M \) is a \hyperref[thm:substructure_is_model]{submodel} of \( M \) if it is a \hyperref[def:monoid/submodel]{submodule} of \( M \) that is closed under algebra multiplication.

    As a consequence of \fullref{thm:positive_formulas_preserved_under_homomorphism}, the image of an \( R \)-algebra homomorphism is a subalgebra of its range.

    \thmitem{def:algebra_over_ring/trivial} The \hyperref[thm:substructures_form_complete_lattice/bottom]{trivial} \( R \)-algebra is the \hyperref[def:pointed_set/trivial]{trivial pointed set} \( \set{ 0_R } \).

    \thmitem{def:algebra_over_ring/category} For a fixed ring \( R \), we denote the \hyperref[def:category_of_small_first_order_models]{category of \( \mscrU \)-small models} by \( \ucat{Alg}_R \). It is concrete with respect to both \( \ucat{CRing} \) and \( \ucat{Mod}_R \).

    Unfortunately, these categories are not as well-behaved as categories of modules.

    \thmitem{def:algebra_over_ring/kernel} The \term{kernel} of an \( R \)-algebra homomorphism \( \varphi: M \to N \) is its \hyperref[def:zero_locus]{zero locus} \( \varphi^{-1}(0_N) \).

    The kernel of a homomorphism is a both a submodule and a subring of \( M \). It is the kernel of the underlying group, ring and module, and the \hyperref[def:zero_morphisms/cokernel]{categorical kernel} in the category of modules.

    \thmitem{def:algebra_over_ring/quotient} The \hyperref[def:zero_morphisms/cokernel]{categorical cokernel} of an \( R \)-homomorphism \( \varphi: M \to N \) in the category \( \cat{Alg}_R \) is simply \hyperref[def:ring/quotient]{quotient ring} \( N / \img \varphi \).

    More generally, let \( I \) be a two-sided ideal of \( M \). We identify each member \( x \) of \( R \) with its embedding \( x \cdot 1_R \). Since \( I \) is closed under multiplication with elements of \( M \), it is also closed under multiplication with elements of \( R \). Thus, \( I \) is a submodule of the \( R \)-module \( M \).

    Thus, given an ideal \( I \) of \( M \), \( M / I \) is both a quotient ring and a quotient module.
  \end{thmenum}
\end{definition}

\begin{proposition}\label{thm:ring_is_integer_algebra}
  The categories \( \hyperref[def:ring/category]{\cat{Ring}} \) of rings and \( \hyperref[def:algebra_over_ring/category]{\cat{Alg}_\BbbZ} \) of integer algebras are \hyperref[rem:category_similarity/isomorphism]{isomorphic}.

  Compare this result to \fullref{thm:abelian_group_is_module}.
\end{proposition}
\begin{proof}
  Follows from \fullref{thm:commutative_monoid_is_semimodule} by noting that, as in the proof of \fullref{thm:semiring_is_algebra}, distributivity implies bilinearity.
\end{proof}

\begin{theorem}[Quotient algebra universal property]\label{thm:quotient_algebra_universal_property}
  For every \hyperref[def:algebra_over_ring]{unital \( R \)-algebra} \( M \) and \hyperref[def:semiring_ideal]{ideal} \( I \) of \( M \), the \hyperref[def:algebra_over_ring/quotient]{quotient algebra} \( M / I \) has the following \hyperref[rem:universal_mapping_property]{universal mapping property}:
  \begin{displayquote}
    Every ring homomorphism \( \varphi: M \to N \) satisfying \( I \subseteq \ker \varphi \) \hyperref[def:factors_through]{uniquely factors through} \( M / I \). That is, there exists a unique homomorphism \( \widetilde{\varphi}: M / I \to N \), such that the following diagram commutes:
    \begin{equation}\label{eq:thm:quotient_algebra_universal_property/diagram}
      \begin{aligned}
        \includegraphics[page=1]{output/thm__quotient_algebra_universal_property.pdf}
      \end{aligned}
    \end{equation}

    In the case where \( I = \ker \varphi \), \( \widetilde{\varphi} \) is an \hyperref[def:first_order_homomorphism_invertibility/embedding]{embedding}.
  \end{displayquote}

  Compare this result to \fullref{thm:quotient_group_universal_property} and \fullref{thm:quotient_module_universal_property}.
\end{theorem}
\begin{proof}
  Simple refinement of \fullref{thm:quotient_group_universal_property}.
\end{proof}

\begin{definition}\label{def:algebra_presentation}
  For a \hyperref[def:commutative_ring]{commutative ring} \( R \), a \term{presentation} of the \( R \)-\hyperref[def:algebra_over_ring]{algebra} \( M \) is a surjective \hyperref[def:module/homomorphism]{homomorphism} \( \varphi: R[\mscrS] \to M \) (epimorphisms may be too general), where \( R[\mscrS] \) is a \hyperref[def:polynomial_algebra]{polynomial ring} with indeterminates \( \mscrS \).

  By \fullref{thm:quotient_algebra_universal_property},
  \begin{equation*}
    M = \img \varphi \cong R[\mscrS] / \ker \varphi.
  \end{equation*}

  Analogously to \hyperref[def:group_presentation]{group presentations}, we say that \( M \) is finitely generated/related/presented if there exists an appropriate presentation.
\end{definition}

\begin{theorem}[Quotient ideal lattice theorem]\label{thm:quotient_ideal_lattice_theorem}
  Given a \hyperref[def:semiring_ideal]{two-sided ideal} \( I \) of the \hyperref[def:algebra_over_ring]{unital \( R \)-algebra} \( M \), the function \( N \mapsto N / I \) is a \hyperref[def:semilattice/homomorphism]{lattice isomorphism} between the \hyperref[thm:substructures_form_complete_lattice]{lattice of subalgebras} of \( M \) containing \( I \) and the lattice of subalgebras of the \hyperref[def:ring/quotient]{quotient} \( M / I \).

  Furthermore, the subsets of \hyperref[def:semiring_ideal/prime]{prime}, \hyperref[def:semiring_ideal/maximal]{maximal} or \hyperref[def:radical_ideal]{radical} ideals are \hyperref[def:partially_ordered_set/homomorphism]{order-isomorphic}.

  Note that, by \fullref{thm:ring_is_integer_algebra}, usual rings can be regarded as \( \BbbZ \)-algebras.

  Compare this result to \fullref{thm:quotient_subgroup_lattice_theorem} and \fullref{thm:quotient_submodule_lattice_theorem}.
\end{theorem}
\begin{proof}
  \SubProof{Proof for general ideals} Simple refinement of \fullref{thm:quotient_subgroup_lattice_theorem}.

  \SubProof{Proof for prime ideals}
  \SufficiencySubProof* Suppose that \( P \) is a prime ideal in \( M \) containing \( I \). Let \( A \) and \( B \) be ideals of \( P \) containing \( I \), such that \( (A / I) (B / I) \subseteq P / I \). By the general ideal lattice isomorphism, \( AB \subseteq P \), which implies that \( A \subseteq P \) or \( B \subseteq P \). Again using the general isomorphism, we conclude that \( A / I \subseteq P / I \) or \( B / I \subseteq P / I \), meaning that \( P / I \) is a prime ideal in \( M / I \).

  \NecessitySubProof* Suppose that \( P / I \) is a prime ideal in \( M / I \) and let \( AB \subseteq P \). Note that neither \( A \) nor \( B \) do not necessarily contain \( I \), but \( A + I \) and \( B + I \) do. Furthermore, \( A + I \subseteq P \) and \( B + I \subseteq P \).

  We have
  \begin{equation*}
    [(A + I) / I][(B + I) / I]
    \reloset {\ref{thm:def:semiring_ideal/coprime_product}} \subseteq
    [(A + I) / I] \cap [(B + I) / I]
    \subseteq
    P / I.
  \end{equation*}

  Since \( P / I \) is prime, \( (A + I) / I \subseteq P / I \) or \( (B + I) / I \subseteq P / I \). Again using the general lattice isomorphism, we conclude that \( A \subseteq A + I \subseteq P \) or \( B \subseteq B + I \subseteq P \).

  \SubProof{Proof for maximal ideals} Trivial consequence of the general lattice isomorphism.

  \SubProof{Proof for radical ideals} Correspondence of radical ideals follows from correspondence of prime ideals since any radical ideal satisfies \fullref{def:radical_ideal/intersection} and thus equals the intersection of all prime ideals containing it.
\end{proof}

\begin{corollary}\label{thm:quotient_by_maximal_ideal}
  The two-sided ideal \( I \) of the \hyperref[def:ring]{ring} \( R \) is \hyperref[def:semiring_ideal/maximal]{maximal} if and only if the \hyperref[def:ring/quotient]{quotient} \( R / M \) is a \hyperref[def:ring/simple]{simple ring}.

  In particular, if \( R \) is commutative, \( R / M \) is a \hyperref[def:field]{field}.

  See \fullref{thm:quotient_by_prime_ideal} for the corresponding statement for \hyperref[def:semiring_ideal/prime]{prime ideals} in commutative rings.
\end{corollary}
\begin{proof}
  Since \( M \) is maximal, only \( M \) and \( R \) are ideals of \( R \) containing \( M \). Therefore, by \fullref{thm:quotient_ideal_lattice_theorem}, \( R / M \) has only two ideals. The converse also follows from the lattice theorem.
\end{proof}

\begin{definition}\label{def:noetherian_semimodule}\mcite[prop. 6.16]{Golan2010}
  We say that an \( R \)-\hyperref[def:semimodule]{semimodule} is \term{noetherian} if any of the following equivalent conditions hold:
  \begin{thmenum}
    \thmitem{def:noetherian_semimodule/acc} Every ascending chain of \( R \)-sub-semimodules stabilizes. That is,
    \begin{equation*}
      N_1 \subseteq N_2 \subseteq N_3 \cdots
    \end{equation*}
    implies that there exists an index \( k_0 \) such that \( N_k = N_{k_0} \) for \( k > k_0 \).

    This condition is sometimes abbreviated as ACC (ascending chain condition).

    \thmitem{def:noetherian_semimodule/maximal} Every nonempty family of \( R \)-sub-semimodules has a \hyperref[def:partially_ordered_set_extremal_points/maximal_and_minimal_element]{maximal element}.

    \thmitem{def:noetherian_semimodule/generated} Every \( R \)-sub-semimodule is \hyperref[def:module_presentation]{finitely generated}, i.e. is the \hyperref[def:semimodule/submodel]{linear span} of finitely many elements.
  \end{thmenum}
\end{definition}
\begin{defproof}
  \EquivalenceSubProof{def:noetherian_semimodule/acc}{def:noetherian_semimodule/maximal} Follows from the equivalences in \fullref{def:well_ordered_set} adapted to the \hyperref[def:semilattice/duality]{opposite} of the \hyperref[thm:substructures_form_complete_lattice]{lattice of \( R \)-sub-semimodules}.

  \ImplicationSubProof{def:noetherian_semimodule/maximal}{def:noetherian_semimodule/generated} Suppose that every nonempty family of sub-semimodules has a maximal element and let \( N \) be a sub-semimodule.

  Let \( K \coloneqq \linspan{ x_1, \ldots, x_n } \) be maximal in the family of all finitely-generated \( R \)-sub-semimodules. Adding any particular element from \( N \) does not change \( K \), because otherwise it would not be maximal. Thus, \( K = N \).

  \ImplicationSubProof{def:noetherian_semimodule/generated}{def:noetherian_semimodule/acc} Suppose that every \( R \)-sub-semimodule is finitely generated and let
  \begin{equation*}
    N_1 \subseteq N_2 \subseteq N_3 \cdots
  \end{equation*}
  be a chain of \( R \)-sub-semimodules.

  Suppose that for every positive integer \( n \), there exists an \( R \)-sub-semimodule \( N_k \) in this chain with more than \( n \) elements. Then the union
  \begin{equation*}
    \bigcup_{k=1}^\infty N_k,
  \end{equation*}
  which by \fullref{thm:def:semimodule/union} is also an \( R \)-sub-semimodule, contains infinitely many elements, contradicting our initial assumption.

  Therefore, every ascending chain of \( R \)-sub-semimodules stabilizes.
\end{defproof}

\begin{proposition}\label{thm:def:noetherian_semimodule}
  \hyperref[def:noetherian_semimodule]{Noetherian modules} over an arbitrary ring \( R \) have the following basic properties:
  \begin{thmenum}
    \thmitem{thm:def:noetherian_semimodule/submodule} If \( M \) is noetherian, then every \( R \)-submodule of \( M \) also is.
    \thmitem{thm:def:noetherian_semimodule/quotient}\mcite[prop. 6.3b)]{КоцевСидеров2016} Let \( N \) be an \( R \)-\hyperref[def:module/submodel]{submodule} of \( M \). Then \( M \) is noetherian if and only if both \( N \) and their \hyperref[def:module/quotient]{quotient} \( M / N \) are.
  \end{thmenum}
\end{proposition}
\begin{proof}
  \SubProofOf{thm:def:noetherian_semimodule/submodule} Trivial.
  \SubProofOf{thm:def:noetherian_semimodule/quotient} By \fullref{thm:quotient_submodule_lattice_theorem}, every chain of submodules of \( M / N \) corresponds to a chain of submodules in \( M \). Thus, if \( M \) is noetherian, clearly \( M / N \) also is.

  Conversely, suppose that both \( N \) and \( M / N \) are noetherian. Let
  \begin{equation}\label{eq:thm:def:noetherian_semimodule/quotient/chain}
    K_1 \subseteq K_2 \subseteq K_3 \subseteq \cdots
  \end{equation}
  be an ascending chain of \( R \)-submodule of \( M \). Then
  \begin{equation}\label{eq:thm:def:noetherian_semimodule/quotient/chain/intersection}
    K_1 \cap N \subseteq K_2 \cap N \subseteq K_3 \cap N \subseteq \cdots
  \end{equation}
  is an ascending chain of \( R \)-submodules of \( N \) and
  \begin{equation}\label{eq:thm:def:noetherian_semimodule/quotient/chain/quotient}
    (K_1 + N) / N \subseteq (K_2 + N) / N \subseteq (K_3 + N) / N \subseteq \cdots
  \end{equation}
  is an ascending chain of \( R \)-submodule of \( M / N \).

  Both \eqref{eq:thm:def:noetherian_semimodule/quotient/chain/intersection} and \eqref{eq:thm:def:noetherian_semimodule/quotient/chain/quotient} stabilize. Let \( n \) be an index such that, for every positive integer \( k \), \( K_n \cap N = K_{n + k} \cap N \) and \( (K_n + N) / N = (K_{n + k} + N) / N \). For a fixed \( k \), we will show that \( K_n = K_{n + k} \).

  Let \( x \in K_{n + k} \). If \( x \in N \), then \( x \in K_n \) since \( K_n \cap N = K_{n + k} \cap N \). Suppose that \( x \in K_{n + k} \setminus N \). For any \( n \in N \), we have \( x + n \in K_{n + k} + N \), and hence
  \begin{equation*}
    x + n + N = x + N \in (K_{n + k} + N) / N = (K_n + N) / N.
  \end{equation*}

  Then there exists some \( y \in K_n \) such that \( x - y \in N \). Actually
  \begin{equation*}
    x - y \in K_{n + k} \cap N = K_n \cap N.
  \end{equation*}

  Since both \( y \) and \( x - y \) are in \( K_n \), so is their sum \( x \). Generalizing on \( x \), we conclude that \( K_n = K_{n + k} \).

  Therefore, the chain \eqref{eq:thm:def:noetherian_semimodule/quotient/chain} stabilizes, implying that \( M \) is noetherian.
\end{proof}

\begin{definition}\label{def:noetherian_semiring}\mcite[prop. 6.16]{Golan2010}
  We say that a (not necessarily commutative) \hyperref[def:semiring]{semiring} is \term{left noetherian} (resp. right noetherian) if it is a left (resp. right) \hyperref[def:noetherian_semimodule]{noetherian semimodule} over itself.

  Explicitly, any of the following equivalent conditions characterize a left noetherian semiring:
  \begin{thmenum}
    \thmitem{def:noetherian_semiring/acc} Every ascending chain of left ideals stabilizes.
    \thmitem{def:noetherian_semiring/maximal} Every nonempty set of left ideals has a maximal element.
    \thmitem{def:noetherian_semiring/generated} Every left ideal is \hyperref[def:semiring_ideal/generated]{finitely generated}.
  \end{thmenum}
\end{definition}

\begin{proposition}\label{thm:noetherian_free_module}
  For a \hyperref[def:noetherian_semiring]{noetherian ring} \( R \), the \hyperref[def:free_semimodule]{free module} \hyperref[def:standard_basis]{\( R^n \)} is a \hyperref[def:noetherian_semimodule]{noetherian module}.
\end{proposition}
\begin{proof}
  We will use induction on \( n \). The cases \( n = 0 \) and \( n = 1 \) are trivial.

  Suppose that \( R^{n-1} \) is noetherian. We can identify \( R \) with the submodule of \( R^n \) generated by the vector \( (0, \ldots, 0, 1) \). Two vectors \( \seq{ x_k }_{i=1}^n \) and  \( \seq{ y_k }_{i=1}^n \) in \( R^n \) belong to this submodule if and only if \( x_k = y_k \) for \( k = 1, \ldots, n - 1 \).

  By \fullref{thm:quotient_equality_via_difference}, these vectors get mapped to the same vector in the quotient \( R^n / R \). Then \( R^n / R \cong R^{n-1} \), which is noetherian by the inductive hypothesis. By \fullref{thm:def:noetherian_semimodule/quotient}, \( R^{n-1} \) is noetherian if and only if \( R^n \) is noetherian.

  Therefore, \( R^n \) is noetherian.
\end{proof}

\begin{lemma}\label{thm:surjective_endomorphism_over_noetherian_module}
  Every surjective endomorphism \( f: M \to M \) of a noetherian \( R \)-module \( M \) is an isomorphism.
\end{lemma}
\begin{proof}
  Consider the equation
  \begin{equation*}
    f(f(x)) = 0_M.
  \end{equation*}

  It is obviously satisfied for \( x \in \ker f \), but it is also possible that \( f(x) \neq 0_M \) while \( f(f(x)) = 0_M \). Therefore,
  \begin{equation*}
    \ker f \subseteq \ker f^2 \subseteq \ker f^3 \subseteq \cdots,
  \end{equation*}
  where \( f^k \) is \( k \)-fold iterated composition.

  Since \( M \) is noetherian, this chain stabilizes. Suppose that \( \ker f^n = \ker f^{n + k} \) for every positive integer \( k \).

  Let \( y \in \ker f^n \). Since \( f \) is surjective, so is \( f^n \), and hence there exists some \( x \) be such that \( f^n(x) = y \). Then \( f^n(y) = f^n(f^n(x)) = 0_M \). But \( \ker f^n = \ker f^{2n} \), hence \( x \in \ker f^n \). Therefore, \( y = f^n(x) = 0 \).

  It follows that \( f^n \) has a trivial kernel. Then so does \( f \). By \fullref{thm:def:group/zero_kernel}, this implies that \( f \) is injective, and hence an isomorphism.
\end{proof}

\begin{proposition}\label{thm:surjective_endomorphism_in_free_module}
  Consider the \hyperref[def:free_semimodule]{free module} \hyperref[def:standard_basis]{\( R^n \)} for a \hyperref[def:noetherian_semiring]{noetherian ring} \( R \). If the endomorphism \( \varphi: R^n \to R^n \) is surjective, then it is also injective and hence an automorphism.
\end{proposition}
\begin{proof}
  Follows from \fullref{thm:noetherian_free_module} and \fullref{thm:surjective_endomorphism_over_noetherian_module}.
\end{proof}

\begin{theorem}[Hilbert's basis theorem]\label{thm:hilberts_basis_theorem}\mcite[thm. 7.4]{КоцевСидеров2016}
  If \( R \) is a \hyperref[def:noetherian_semiring]{noetherian commutative ring}, then so is \( R[X] \).
\end{theorem}
\begin{proof}
  Let \( I \subseteq R[X] \) be an arbitrary ideal. We will prove that \( I \) is finitely generated.

  Denote by \( L \) the set of all leading coefficients of polynomials in \( I \). The leading coefficient of the product \( p(X) q(X) \) of univariate polynomials is the product of their leading coefficients, hence \( L \) is an ideal as a consequence of \( I \) being an ideal.

  As a consequence of \( R \) being noetherian, \( L \) is finitely generated. Suppose that \( L = \set{ l_1, \ldots, l_n } \).

  For every generator \( l_k \), there exists a polynomial \( p_k(X) \) in \( I \) whose leading coefficient is \( l_k \). Denote by \( d_k \) the degree of \( p_k \) and let \( d \) be the maximum of the degrees. We will show that \( I \) itself is equal to the sum of the finitely generated ideals
  \begin{equation*}
    J \coloneqq \underbrace{ \braket{ p_1, \ldots, p_n } + \braket{ X, X^2, \ldots, X^d } }_{ \braket{ p_1, \ldots, p_n, X, X^2, \ldots, X^d } }.
  \end{equation*}

  Let \( f(X) \) be some polynomial from \( I \) whose leading term is \( l X^m \).

  We proceed by induction on \( m \) to show that \( f(X) \) belongs to \( J \).
  \begin{itemize}
    \item If \( m \leq d \), then \( f(X) \) belongs to the second ideal \( \braket{ X, X^2, \ldots, X^d } \).

    \item Suppose that \( m > d \) and that every polynomial in \( I \) of degree less than \( m \) belongs to \( J \).

    Since \( l \in L \), it is a linear combination \( l = \sum_{k=1}^n t_k l_k \) with coefficients in \( R \). Consider the polynomial
    \begin{equation*}
      p(X) \coloneqq \sum_{k=1}^n t_k p_k(X) X^{m - d_k}.
    \end{equation*}

    Define \( r(X) \coloneqq f(X) - p(X) \). Since \( p(X) \) belongs to \( I \), \( r(X) \) does too. It is a polynomial in \( I \) of degree less than \( m \), hence it belongs to \( J \). Then
    \begin{equation*}
      f(X) = \underbrace{ p(X) }_{\mathclap{\braket{ p_1(X) \cdots, p_n(X) }}} + \overbrace{ r(X) }^J.
    \end{equation*}

    Hence, \( f(X) \in J \).
  \end{itemize}

  Our choice of polynomial \( f(X) \in I \) was arbitrary. Therefore,
  \begin{equation*}
    I \subseteq \braket{ p_1, \ldots, p_n } + \braket{ X, X^2, \ldots, X^d } \subseteq I,
  \end{equation*}
  demonstrating that \( I \) is finitely generated.
\end{proof}

\begin{definition}\label{def:algebraic_dependence}\mimprovised
  Let \( M \) be an \( R \)-\hyperref[def:algebra_over_ring]{algebra} over a \hyperref[def:ring/commutative]{commutative ring} \( R \) and fix a subset \( B \subseteq M \). We say that the elements of \( B \) are \term{algebraically independent} if any of the following conditions hold:

  \begin{thmenum}
    \thmitem{def:algebraic_dependence/direct} If \( \seq{ u_b }_{b \in B} \) is a root of some polynomial with indeterminates \( \set{ X_b \given b \in B } \) and coefficients in \( R \), then it is the zero polynomial.

    \thmitem{def:algebraic_dependence/evaluation} The \hyperref[thm:polynomial_algebra_universal_property]{evaluation map} \( \Phi_B: R[X_b \given b \in B] \to M \) is injective.
  \end{thmenum}

  Unsurprisingly, if the elements of \( B \) are not \term{algebraically independent}, we say that they are \term{algebraically dependent}.

  Compare this concept to \hyperref[def:linear_dependence]{linear dependence}.
\end{definition}
\begin{defproof}
  \ImplicationSubProof{def:algebraic_dependence/direct}{def:algebraic_dependence/evaluation} Suppose that \( \Phi_B \) is injective and that there exists a polynomial \( p(X_b \given b \in B) \) such that \( \Phi_B(p) = 0_M \). Then for any other polynomial \( q(X_x \given b \in B) \), we have \( \Phi_B(p q) = 0_M \), and hence either \( p \) is the zero polynomial or the evaluation map is not injective.

  \ImplicationSubProof{def:algebraic_dependence/evaluation}{def:algebraic_dependence/direct} Conversely, suppose that \( B \) is a root only of the zero polynomial. Let \( \Phi_B(p) = \Phi_B(q) \). Then \( B \) is a root of \( p - q \) and hence the latter is the zero polynomial. But this implies that \( p = q \). Hence, the evaluation map is injective.
\end{defproof}

\begin{proposition}\label{thm:def:algebraic_dependence}
  \hyperref[def:algebraic_dependence]{Algebraic (in)dependence} for the \hyperref[def:integral_domain]{integral domain} \( D \) has the following basic properties:
  \begin{thmenum}
    \thmitem{thm:def:algebraic_dependence/n_independent} Monomials for different indeterminates are algebraically independent over \( D \).

    \thmitem{thm:def:algebraic_dependence/two_univariate_dependent}\mcite{MathOF:n_plus_one_polynomials_algebraically_dependent} Every two univariate polynomials in \( D \) are algebraically dependent over \( D \).

    \thmitem{thm:def:algebraic_dependence/n_plus_one_dependent} Every \( n + 1 \) polynomials in \( D[X_1, \ldots, X_n] \) are algebraically dependent over \( D \).
  \end{thmenum}
\end{proposition}
\begin{proof}
  \SubProofOf{thm:def:algebraic_dependence/n_independent} Suppose that the monomials \( X_1, \ldots, X_n \) are algebraically dependent over \( D \). Then there exists some nonzero polynomial \( f(Y_1, \ldots, Y_n) \) such that the evaluation \( \Phi_{X_1, \dots, X_n}(f) \) is the zero polynomial. But the evaluation simply renames the variables, hence \( f \) itself is zero. But we have assumed that it is nonzero.

  The obtained contradiction demonstrates that \( X_1, \ldots, X_n \) are algebraically independent over \( D \).

  \SubProofOf{thm:def:algebraic_dependence/two_univariate_dependent} Fix polynomials \( p(X) \) and \( q(X) \) over \( D \). We will construct a polynomial \( f(Y, Z) \) over \( D \) such that \( \Phi_{p,q}(f) = 0 \).

  If \( p(X) \) is zero, simply define \( f(Y, Z) \coloneqq Z \). If \( q(X) \) is zero, put \( f(Y, Z) \coloneqq Y \).

  Suppose that both are nonzero; denote by \( n \) be the degree of \( p(X) \) and by \( m \) the degree of \( q(X) \). We will consider polynomials of the form \( p^l q^k \).

  Fix a positive integer \( d \). We want the degree of \( p^l q^k \) to be at most \( d \). If
  \begin{align*}
    l < \frac d {2n} && k < \frac d {2m},
  \end{align*}
  then, by \fullref{thm:def:polynomial_degree/product},
  \begin{equation*}
    \deg(p^l q^k) = nl + km < \frac d 2 + \frac d 2 = d.
  \end{equation*}

  These polynomials are all in
  \begin{equation*}
    L_d \coloneqq \linspan\set{ 1, X, X^2, X^3, \ldots, X^{d-1} }.
  \end{equation*}

  This is a module of \hyperref[thm:commutative_module_rank]{rank} \( d \).

  Furthermore, there are \( \ifrac {d^2} {(4nm)} \) such polynomials. If \( d > 4nm \), there are more polynomials of the form \( p^l q^k \) than the \hyperref[thm:commutative_module_rank]{rank} of \( L_d \). Hence, every \( d + 1 \) such polynomials are linearly dependent, and hence there exists some linear combination
  \begin{equation*}
    a_1 p^{l_1} q^{k_1} + \cdots + a_{d+1} p^{l_{d+1}} q^{k_{d+1}} = 0.
  \end{equation*}

  We can thus define the following polynomial in \( D[Y, Z] \):
  \begin{equation*}
    f(Y, Z) \coloneqq a_1 Y^{l_1} Z^{k_1} + \cdots + a_{d+1} Y^{l_{d+1}} Z^{k_{d+1}}.
  \end{equation*}

  Then clearly \( \Phi_{p,q}(f) = 0 \), so \( p \) and \( q \) are algebraically dependent over \( D \).

  \SubProofOf{thm:def:algebraic_dependence/n_plus_one_dependent} Let \( p_1, \ldots, p_{n+1} \) be polynomials in \( D[X_1, \ldots, X_{n-1}][X_n] \). By \fullref{thm:def:algebraic_dependence/two_univariate_dependent}, the polynomials \( p_n \) and \( p_{n+1} \) are algebraically dependent over \( D[X_1, \ldots, X_{n-1}] \).

  Let \( f(Y_n, Y_{n+1}) \) be a polynomial in \( D[X_1, \ldots, X_{n-1}][Y_n, Y_{n+1}] \) such that \( \Phi_{p_n,p_{n+1}}(f) = 0 \). The coefficients of \( f \) are themselves polynomials. Let
  \begin{equation*}
    \widehat{f}(Y_1, \ldots, Y_{n-1}, Y_n, Y_{n+1})
  \end{equation*}
  be the polynomial obtained from
  \begin{align*}
    f(X_1, \ldots, X_{n-1}, Y_n, Y_{n+1})
  \end{align*}
  by renaming the corresponding variables.

  Then \( \Phi_{p_1,\ldots,p_{n+1}}(f) = 0 \). Therefore, \( p_1, \ldots, p_{n+1} \) are algebraically dependent over \( D \).
\end{proof}
