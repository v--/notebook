\subsection{Group actions}\label{subsec:group_actions}

\begin{definition}\label{def:endomorphism_monoid}
  For every object \( A \) in an arbitrary \hyperref[def:category]{category} \( \cat{C} \), the set \( \cat{C}(A) \) is a \hyperref[def:unital_magma/monoid]{monoid} with morphism composition as the monoid operation and \( \id_A \) as the monoid identity.

  Outside of \hyperref[sec:category_theory]{category theory}, whenever the category \( \cat{C} \) is clear from the context, we call \( \cat{C}(A) \) the \term{endomorphism monoid} over \( A \) and denote it by \( \End(A) \).
\end{definition}

\begin{definition}\label{def:monoid_action}
  Let \( \mscrM \) be a \hyperref[def:unital_magma/monoid]{monoid} and let \( A \) be a set or, more generally, an object in some \hyperref[def:concrete_category]{concrete category} \( \cat{C} \).

  We will define monoid actions of \( \mscrM \) on \( A \), which we will sometimes call \term{left monoid actions}. There are also \term{right monoid actions}, which are defined as left monoid actions of the \hyperref[def:group/duality]{dual group} \( G^{\opcat} \).

  A \term{monoid action} can be defined equivalently as:
  \begin{thmenum}
    \thmitem{def:monoid_action/operation} A heterogeneous \hyperref[def:magma]{binary operation} \( \star: \mscrM \times A \to A \) such that
    \begin{thmenum}
      \thmitem[def:monoid_action/operation/identity]{MA1} For every \( a \in A \),
      \begin{equation}\label{eq:def:monoid_action/operation/identity}\tag{\logic{MA1}}
        e \star a = a.
      \end{equation}

      \thmitem[def:monoid_action/operation/compatibility]{MA2} For every \( x, y \in M \) and \( a \in A \),
      \begin{equation}\label{eq:def:monoid_action/operation/compatibility}\tag{\logic{MA2}}
        (xy) \star a = x \star (y \star a).
      \end{equation}
    \end{thmenum}

    See \fullref{rem:theory_of_monoid_actions} for the \hyperref[def:first_order_theory]{first-order logic theory} behind this operation.

    \thmitem{def:monoid_action/homomorphism} A \hyperref[def:unital_magma/homomorphism]{homomorphism} from \( \mscrM \) to the \hyperref[def:endomorphism_monoid]{endomorphism monoid} \( \End(A) \).

    \thmitem{def:monoid_action/functor} A \hyperref[def:functor]{functor} from the \hyperref[def:monoid_delooping]{delooping} \( \cat{B}_\mscrM \) to \( \cat{C} \) (right actions are \hyperref[rem:contravariant_functor]{contravariant functors}).
  \end{thmenum}
\end{definition}
\begin{proof}
  \ImplicationSubProof{def:monoid_action/operation}{def:monoid_action/homomorphism} Suppose that we have an operation \( \star: \mscrM \times A \to A \) that satisfies the axioms for left action. Define the function
  \begin{equation*}
    \begin{aligned}
      &\tau: \mscrM \to \End(A) \\
      &\tau(x) \coloneqq a \mapsto x \star a.
    \end{aligned}
  \end{equation*}

  Then \( \tau \) is a monoid homomorphism because \ref{def:monoid_action/operation/identity} implies \( \tau(e) = \id_A \) and
  \begin{equation}\label{def:monoid_action/homomorphism/proof}
    [\tau(xy)](a)
    =
    xy \star a
    \reloset {\eqref{eq:def:monoid_action/operation/compatibility}} =
    x (y \star a)
    =
    [\tau(x)]\parens[\Big]{ [\tau(y)](a) }
    =
    [\tau(x) \bincirc \tau(y)](a).
  \end{equation}

  \ImplicationSubProof{def:monoid_action/homomorphism}{def:monoid_action/functor} Suppose that we have a monoid homomorphism \( \varphi: \mscrM \to \End(A) \). Define the functor
  \begin{equation*}
    \begin{aligned}
      &F: \cat{B}_\mscrM \to \cat{C} \\
      &F(\anon) \coloneqq A \\
      &F(x) \coloneqq \varphi(x).
    \end{aligned}
  \end{equation*}

  This is indeed a functor because \eqref{eq:def:functor/CF1} follows from \eqref{eq:def:pointed_set/homomorphism} and \eqref{eq:def:functor/CF2} follows from \eqref{eq:def:magma/homomorphism}.

  \ImplicationSubProof{def:monoid_action/functor}{def:monoid_action/operation} Suppose that we have a functor \( F: \cat{B}_\mscrM \to \cat{C} \). Define the operation
  \begin{equation*}
    \begin{aligned}
      &\star: \mscrM \times A \to A \\
      &x \star a \coloneqq F(x)(a).
    \end{aligned}
  \end{equation*}

  It satisfies the necessary axioms:
  \begin{itemize}
    \item \ref{def:monoid_action/operation/identity} holds: for every \( a \in A \), we have
    \begin{equation*}
      e \star a
      =
      F(e)(a)
      \reloset {\eqref{eq:def:functor/CF1}} =
      \id_A(a)
      =
      a.
    \end{equation*}

    \item \ref{def:monoid_action/operation/compatibility} holds: for every \( x, y \in M \) and \( a \in A \), we have
    \begin{equation*}
      (x y) \star a
      =
      F(x y)(a)
      \reloset {\eqref{eq:def:functor/CF2}} =
      [F(x) \bincirc F(y)](a)
      =
      F(x)\parens[\Big]{ F(y)(a) }
      =
      x \star (y \star a).
    \end{equation*}
  \end{itemize}
\end{proof}

\begin{remark}\label{rem:theory_of_monoid_actions}
  In order to fit the heterogeneous operation of \hyperref[def:monoid_action]{left monoid actions} into the framework of \hyperref[def:first_order_semantics/satisfiability]{first-order logic models}, we need the category \( \cat{C} \) to be a \hyperref[def:category_of_small_first_order_models]{category of small models}. A monoid action is then obtained by extending the corresponding.

  \begin{thmenum}
    \thmitem{rem:theory_of_monoid_actions/functions} For every \( x \in \mscrM \), add a unary \hyperref[def:first_order_language/func]{functional symbol} \( \tau_x \).

    \thmitem{rem:theory_of_monoid_actions/axiom} For every pair \( x, y \in \mscrM \), add the axiom
    \begin{equation}\label{eq:rem:theory_of_monoid_actions/axiom_schema}
      \qforall \xi \parens[\Big]{ \tau_{xy}(\xi) \doteq \tau_x(\tau_y(\xi)) }.
    \end{equation}
  \end{thmenum}
\end{remark}

\begin{proposition}\label{thm:monoid_is_action}
  Any \hyperref[def:unital_magma/monoid]{monoid} \hyperref[def:monoid_action]{acts} on itself by \hyperref[def:multi_valued_function/endofunction]{endofunctions}.

  Compare this result to \fullref{thm:cayleys_theorem}.
\end{proposition}
\begin{proof}
  We will show that \( \cdot \) satisfies \fullref{def:monoid_action/operation} holds. This is both a left and right action. It immediately follows that
  \begin{itemize}
    \item \ref{def:monoid_action/operation/identity} follows from \eqref{eq:def:unital_magma/theory/identity}.

    \item \ref{def:monoid_action/operation/compatibility} follows from associativity.
  \end{itemize}
\end{proof}

\begin{proposition}\label{thm:natural_numbers_monoid_action}
  The \hyperref[def:set_of_natural_numbers]{natural numbers} \( \BbbN \) (\hyperref[rem:peano_arithmetic_zero]{with zero}) act on any \hyperref[def:unital_magma/monoid]{monoid} by \hyperref[def:unital_magma/exponentiation]{exponentiation}.

  Compare this result to \fullref{thm:integers_group_action}.
\end{proposition}
\begin{proof}
  The action is given by \( n \mapsto (x \mapsto x^n) \). This action satisfies \fullref{def:monoid_action/operation} because \ref{def:monoid_action/operation/identity} is obvious and \ref{def:monoid_action/operation/compatibility} follows from \fullref{thm:magma_exponentiation_properties/repeated}.
\end{proof}

\begin{definition}\label{def:automorphism_group}
  For every object \( A \) in a \hyperref[def:groupoid]{groupoid} \( \cat{G} \), the set \( \cat{G}(A) \) is a \hyperref[def:group]{group} with morphism composition as the group operation.

  As for \hyperref[def:endomorphism_monoid]{endomorphism monoids}, whenever the groupoid \( \cat{G} \) is clear from the context, we call \( \cat{G}(A) \) the \term{automorphism group} over \( A \) and denote it by \( \End(A) \).
\end{definition}

\begin{definition}\label{def:symmetric_group}
  We define the \term{symmetric group} of order \( n \) is the group
  \begin{equation*}
    S_n \coloneqq \aut(\set{ 1, 2, \ldots, n })
  \end{equation*}
  of all bijective functions from the set \( \set{ 1, 2, \ldots, n } \) to itself.

  \begin{thmenum}
    \thmitem{def:symmetric_group/permutation} We call members of \( S_n \) \term{permutations}.

    \thmitem{def:symmetric_group/inversion} We say that the pair \( (p(k), p(m)) \) is an \term{inversion} of the permutation \( p \) if \( k < m \) and \( p(k) > p(m) \).

    \thmitem{def:symmetric_group/permutation_parity} We say that a permutation is \term{even} (resp. \term{odd}) depending on whether it has an even or odd number of inversions.

    \thmitem{def:symmetric_group/permutation_sign} We define the \term{sign} of a permutation as
    \begin{equation*}
      \begin{aligned}
         &\sgn: S_n \to \set{ -1, 1 }, \\
         &\sgn(p) \coloneqq \begin{cases}
          1,  &p \T{is even} \\
          -1, &p \T{is odd}
        \end{cases}
      \end{aligned}
    \end{equation*}

    \thmitem{def:symmetric_group/alternating} We call the subgroup of all even permutations the \term{alternating group} of order \( n \); we denote it by \( A_n \).
  \end{thmenum}
\end{definition}

\begin{definition}\label{def:group_action}
  Let \( G \) be a \hyperref[def:group]{group} and let \( A \) be a set or, more generally, an object in some \hyperref[def:concrete_category]{concrete category} \( \cat{C} \).

  We will define group actions analogously to \hyperref[def:monoid_action]{monoid actions}, with the same remarks regarding left and right group actions.

  A \term{group action} can be defined equivalently as:
  \begin{thmenum}
    \thmitem{def:group_action/operation} A heterogeneous \hyperref[def:magma]{binary operation} \( \star: \mscrM \times A \to A \) such that
    \begin{thmenum}
      \thmitem[def:group_action/operation/identity]{GA1} For every \( a \in A \),
      \begin{equation}\label{eq:def:group_action/operation/identity}\tag{\logic{GA1}}
        e \star a = a.
      \end{equation}

      \thmitem[def:group_action/operation/compatibility]{GA2} For every \( x, y \in M \) and \( a \in A \),
      \begin{equation}\label{eq:def:monoid_action/operation/compatibility}\tag{\logic{GA2}}
        (xy) \star a = x \star (y \star a).
      \end{equation}
    \end{thmenum}

    See \fullref{rem:theory_of_monoid_actions} for the \hyperref[def:first_order_theory]{first-order logic theory} behind this operation.

    Note that \eqref{eq:def:group_action/operation/identity} does not follow from \eqref{eq:def:monoid_action/operation/compatibility}.

    \thmitem{def:group_action/homomorphism} A \hyperref[def:group/homomorphism]{homomorphism} from \( G \) to the \hyperref[def:automorphism_group]{automorphism group} \( \aut(A) \).

    \thmitem{def:group_action/functor} A \hyperref[def:functor]{functor} from the \hyperref[def:monoid_delooping]{delooping} \( \cat{B}_G \) to \( \cat{C} \) (right actions are \hyperref[rem:contravariant_functor]{contravariant functors}).
  \end{thmenum}
\end{definition}
\begin{proof}
  The proof of equivalence is simple; it is similar to \fullref{def:monoid_action}.
\end{proof}

\begin{lemma}\label{thm:group_multiplication_is_bijection}
  For each element \( x \) of a group \( G \), consider the function \( \varphi_x \coloneqq x \id_G \), i.e.
  \begin{equation*}
    \begin{aligned}
      &\varphi_x: G \to G \\
      &\varphi_x(y) \coloneqq xy.
    \end{aligned}
  \end{equation*}

  This is a bijective function (but not necessarily a group isomorphism).
\end{lemma}
\begin{proof}
  \SubProofOf[def:function_invertibility/injective]{injectivity} If \( y_1, y_2 \in G \) and \( \varphi_x(y_1) = \varphi_x(y_2) \), we have
  \begin{equation*}
    xy_1 = \varphi_x(y_1) = \varphi_x(y_2) = xy_2.
  \end{equation*}

  By \fullref{thm:def:group/properties/cancellative}, \( y_1 = y_2 \). Therefore, \( \varphi_x \) is injective.

  \SubProofOf[def:function_invertibility/surjective]{surjectivity} If \( z \in G \), then \( z = x(x^{-1} z) \). Therefore, \( z = \varphi_x(x^{-1} z) \), and thus every member of \( G \) has a preimage. Thus, \( \varphi_x \) is surjective.
\end{proof}

\begin{theorem}[Cayley's theorem]\label{thm:cayleys_theorem}
  Any \hyperref[def:group]{group} \hyperref[def:group_action]{acts} on itself by \hyperref[def:multi_valued_function/endofunction]{endofunctions}.

  Compare this result to \fullref{thm:monoid_is_action}.
\end{theorem}
\begin{proof}
  Follows directly from \fullref{thm:monoid_is_action} and \fullref{thm:group_multiplication_is_bijection}.
\end{proof}

\begin{proposition}\label{thm:integers_group_action}
  The \hyperref[def:set_of_integers]{integers} \( \BbbZ \) act on any \hyperref[def:group]{group} by \hyperref[def:group/exponentiation]{exponentiation}.

  Compare this result to \fullref{thm:natural_numbers_monoid_action}.
\end{proposition}
\begin{proof}
  Follows from \fullref{thm:natural_numbers_monoid_action}.
\end{proof}
