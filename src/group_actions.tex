\subsection{Group actions}\label{subsec:group_actions}

\begin{definition}\label{def:group_action}\cite[159]{Knapp2016BAlg}
  Let \( G \) be a group and let \( A \) be an arbitrary set. A \Def{group action} of \( G \) on \( A \) can be defined equivalently as
  \begin{defenum}
    \DItem{def:group_action/homomorphism} a homomorphism from \( G \) to the symmetric group \( S(A) \).
    \DItem{def:group_action/indirect_homomorphism} an assignment of a bijection \( \tau_x: A \to A \) for each \( x \in G \) such that
    \begin{equation*}
      \tau_{xy} = \tau_x \circ \tau_y.
    \end{equation*}
    \DItem{def:group_action/left_multiplication} an operation\Tinyref{def:algebraic_theory} \( \circ: G \times A \to A \), written using juxtaposition, such that
    \begin{itemize}
      \item \( (g_1 g_2) \circ a = g_1 \circ (g_2 \circ a) \) whenever \( g_1, g_2 \in G \) and \( a \in A \)
      \item \( e \circ a = a \) for any \( a \in A \)
    \end{itemize}
  \end{defenum}

  We also refer to \ref{def:group_action/left_multiplication} as \Def{left group actions} to distinguish them from \Def{right group actions} (which are defined in the obvious way).
\end{definition}

\begin{example}\label{ex:group_actions}
  \begin{itemize}\mbox{}
    \item Any group \( G \) is a group action on itself. Every element of \( G \) simply corresponds to itself under this action. We will verify all three definitions:
    \begin{description}
      \RItem{def:group_action/homomorphism} The identity function \( \Id: G \to G \) is the identity of the symmetric group \( S(G) \) (where \( G \) is considered as a set here). Thus the following function
      \begin{align*}
        &\varphi: G \to S(G) \\
        &\varphi(x) \coloneqq (y \to xy)
      \end{align*}
      is a group homomorphism from \( G \) to \( S(G) \). Indeed, \( \varphi(e) = \Id \) and
      \begin{equation*}
        (z \to xyz) = \varphi(xy) = \varphi(x) \circ \varphi(y) = (y \to xy) \circ (z \to yz) = (z \to xyz)
      \end{equation*}

      \RItem{def:group_action/indirect_homomorphism} The proof is the same as above.

      \RItem{def:group_action/left_multiplication} Define the operation
      \begin{align*}
        &\circ: G \times G \to G \\
        &x \circ y \coloneqq xy
      \end{align*}

      Thus
      \begin{itemize}
        \item \( (x y) \circ z = xyz = x \circ (y \circ z) \) whenever \( x, y, z \in G \)
        \item \( e \circ x = ex = x \) for any \( x \in G \)
      \end{itemize}
    \end{description}

    \item The group of integers \( \Z \) acts on any group \( G \) by sending \( g \) to its power \( g^n \).
  \end{itemize}
\end{example}
