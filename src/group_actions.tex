\subsection{Group actions}\label{subsec:group_actions}

\begin{definition}\label{def:endomorphism_monoid}
  For every object \( X \) in an arbitrary \hyperref[def:category]{category} \( \cat{C} \), the set \( \cat{C}(X) \) is a \hyperref[def:unital_magma/monoid]{monoid} with morphism composition as the monoid operation and \( \id_X \) as the monoid identity.

  Outside of \hyperref[sec:category_theory]{category theory}, whenever the category \( \cat{C} \) is clear from the context, we call \( \cat{C}(X) \) the \term{endomorphism monoid} over \( X \) and denote it by \( \End(X) \).
\end{definition}

\begin{definition}\label{def:monoid_action}
  Let \( M \) be a \hyperref[def:unital_magma/monoid]{monoid} and let \( X \) be an object in some \hyperref[def:category]{category} \( \cat{C} \).

  We will define monoid actions of \( M \) on \( X \), which we will sometimes call \term{left monoid actions}. There are also \term{right monoid actions}, which are defined as left monoid actions of the \hyperref[def:group/duality]{dual group} \( G^{\opcat} \).

  A \term{monoid action} can be defined equivalently as:
  \begin{thmenum}
    \thmitem{def:monoid_action/family} An \hyperref[def:cartesian_product/indexed_family]{indexed family} \( \seq{ \Phi_m }_{m \in M} \) of \hyperref[def:morphism_invertibility/endomorphism]{endomorphisms} of \( X \) such that
    \begin{align}
      &\Phi_e = \id_X, \label{eq:def:monoid_action/family/identity}\tag{\logic{MA1}} \\
      &\Phi_{mn} = \Phi_m \bincirc \Phi_n. \label{eq:def:monoid_action/family/compatibility}\tag{\logic{MA2}}
    \end{align}

    This defines a function \( \Phi: M \times A \to A \).

    \thmitem{def:monoid_action/homomorphism} A \hyperref[def:unital_magma/homomorphism]{homomorphism} from \( M \) to the \hyperref[def:endomorphism_monoid]{endomorphism monoid} \( \End(X) \).

    \thmitem{def:monoid_action/functor} A \hyperref[def:functor]{functor} from the \hyperref[def:monoid_delooping]{delooping} \( \cat{B}_M \) to \( \cat{C} \) (right actions are \hyperref[rem:contravariant_functor]{contravariant functors}).
  \end{thmenum}
\end{definition}
\begin{proof}
  \ImplicationSubProof{def:monoid_action/family}{def:monoid_action/homomorphism} Suppose that we have an indexed family \( \seq{ \Phi_m }_{m \in M} \) of endomorphisms of \( A \) that satisfies the axioms for left action. Regard this indexed family as a function \( \Phi: M \to \End(X) \).

  Then \( \Phi \) is a monoid homomorphism because \ref{eq:def:monoid_action/family/identity} implies \( \Phi(e) = \id_X \) and \eqref{eq:def:monoid_action/family/compatibility} implies
  \begin{equation*}
    \Phi(mn) = \Phi(m) \bincirc \Phi(n).
  \end{equation*}

  \ImplicationSubProof{def:monoid_action/homomorphism}{def:monoid_action/functor} Suppose that we have a monoid homomorphism \( \Phi: M \to \End(X) \). Define the functor
  \begin{equation*}
    \begin{aligned}
      &F: \cat{B}_M \to \cat{C} \\
      &F(\anon) \coloneqq X \\
      &F(m) \coloneqq \Phi(m).
    \end{aligned}
  \end{equation*}

  This is indeed a functor because \eqref{eq:def:functor/CF1} follows from \eqref{eq:def:pointed_set/homomorphism} and \eqref{eq:def:functor/CF2} follows from \eqref{eq:def:magma/homomorphism}.

  \ImplicationSubProof{def:monoid_action/functor}{def:monoid_action/family} Suppose that we have a functor \( F: \cat{B}_M \to \cat{C} \). Let \( X \coloneqq F(\anon) \) and define the \( M \)-indexed family
  \begin{equation*}
    \begin{aligned}
      &\Phi_m: X \to X \\
      &\Phi_m \coloneqq F(m).
    \end{aligned}
  \end{equation*}

  It satisfies the necessary axioms:
  \begin{itemize}
    \item \ref{eq:def:monoid_action/family/identity} holds:
    \begin{equation*}
      \Phi_e
      =
      F(e)
      \reloset {\eqref{eq:def:functor/CF1}} =
      \id_A.
    \end{equation*}

    \item \ref{eq:def:monoid_action/family/compatibility} holds: for every pair \( m, n \in M \), we have
    \begin{equation*}
      \Phi_{mn}
      =
      F(mn)
      \reloset {\eqref{eq:def:functor/CF2}} =
      F(m) \bincirc F(n)
      =
      \Phi_m \bincirc \Phi_n
    \end{equation*}
  \end{itemize}
\end{proof}

\begin{proposition}\label{thm:monoid_is_action}
  Every \hyperref[def:unital_magma/monoid]{monoid} \hyperref[def:monoid_action]{acts} on itself via the family of functions \( h \mapsto g \cdot h \) indexed by \( g \). These functions are not monoid homomorphisms in general.

  Compare this result to \fullref{thm:cayleys_theorem}.
\end{proposition}
\begin{proof}
  The family satisfies \fullref{def:monoid_action/family}:
  \begin{itemize}
    \item \ref{eq:def:monoid_action/family/identity} follows from \eqref{eq:def:unital_magma/theory/identity}.

    \item \ref{eq:def:monoid_action/family/compatibility} follows from associativity:
    \begin{equation*}
      [h \mapsto g_1 \cdot h] \bincirc [h \mapsto g_2 \cdot h] = [h \mapsto g_1 \cdot (g_2 \cdot h)] = [h \mapsto (g_1 \cdot g_2) \cdot h].
    \end{equation*}
  \end{itemize}
\end{proof}

\begin{proposition}\label{thm:exponentiation_monoid_action}
  The \hyperref[def:set_of_natural_numbers]{natural numbers} \( \BbbN \) (\hyperref[rem:peano_arithmetic_zero]{with zero}) act on any \hyperref[def:unital_magma/monoid]{monoid} by \hyperref[def:unital_magma/exponentiation]{exponentiation} via the family of function \( g \mapsto g^n \) indexed by \( n \in \BbbN \).

  Compare this result to \fullref{thm:exponentiation_group_action}.
\end{proposition}
\begin{proof}
  This family satisfies \fullref{def:monoid_action/family}:
  \begin{itemize}
    \item \ref{eq:def:monoid_action/family/identity} is obvious.
    \item \ref{eq:def:monoid_action/family/compatibility} follows from \fullref{thm:magma_exponentiation_properties/repeated}.
  \end{itemize}
\end{proof}

\begin{definition}\label{def:automorphism_group}
  For every object \( X \) in a \hyperref[def:groupoid]{groupoid} \( \cat{G} \), the set \( \cat{G}(X) \) is a \hyperref[def:group]{group} with morphism composition as the group operation.

  As for \hyperref[def:endomorphism_monoid]{endomorphism monoids}, whenever the groupoid \( \cat{G} \) is clear from the context, we call \( \cat{G}(X) \) the \term{automorphism group} over \( X \) and denote it by \( \End(X) \).
\end{definition}

\begin{definition}\label{def:symmetric_group}
  We define the \term{symmetric group} of order \( n \) is the group
  \begin{equation*}
    S_n \coloneqq \aut(\set{ 1, 2, \ldots, n })
  \end{equation*}
  of all bijective functions from the set \( \set{ 1, 2, \ldots, n } \) to itself.

  \begin{thmenum}
    \thmitem{def:symmetric_group/permutation} We call members of \( S_n \) \term{permutations}. It is common to write a permutation \( p \) as
    \begin{equation*}
      \begin{pmatrix}
        1    & \cdots & n \\
        p(1) & \cdots & p(n)
      \end{pmatrix}
    \end{equation*}

    \thmitem{def:symmetric_group/cycle} If there exists a finite sequence \( (k_1, \ldots, k_m) \) of distinct numbers such that \( p(k_m) = k_1 \) and \( p(k_{i+1}) = p(k_i) \) for each \( i < m \), we say that the permutation is \term{cyclic} or a \term{cycle} of order \( m \). For brevity, we denote this cycle by \( \cycle{k_1, \ldots, k_m} \). We associate a \term{length} with each cycle defined in an obvious way.

    Cycles of length \( 2 \) are also called \term{transpositions}. Every \hyperref[def:symmetric_group/cycle]{transposition} is an \hyperref[def:set_with_involution]{involution}, and thus \( t = t^{-1} \) (as permutations).

    \thmitem{def:symmetric_group/disjoint_cycle} If \( \cycle{k_1, \ldots, k_m} \) and \( \cycle{s_1, \ldots, s_l} \) are two cycles and if the sets \( \set{ k_1, \ldots, k_m } \) and \( \set{ s_1, \ldots, s_l } \) are disjoint, we say that the cycles themselves are \term{disjoint}.

    Two disjoint cycles commute. That is,
    \begin{equation*}
      \cycle{k_1, \ldots, k_m} \bincirc \cycle{s_1, \ldots, s_l} = \cycle{s_1, \ldots, s_l} \bincirc \cycle{k_1, \ldots, k_m}.
    \end{equation*}
  \end{thmenum}
\end{definition}

\begin{proposition}\label{thm:cycle_transposition_decomposition}
  Every cycle \( \cycle{ k_1, \ldots, k_m } \) can be decomposed into the product of transpositions
  \begin{equation*}
    \cycle{ k_1, \ldots, k_m } = \cycle{ k_1, k_m } \bincirc \cycle{ k_1, k_{m-1} } \bincirc \cdots \bincirc \cycle{ k_1, k_2 }.
  \end{equation*}
\end{proposition}
\begin{proof}
  Trivial.
\end{proof}

\begin{definition}\label{def:group_action}
  Let \( G \) be a \hyperref[def:group]{group} and let \( X \) be an object in some \hyperref[def:concrete_category]{concrete category} \( \cat{C} \).

  We will define group actions as a special case of \hyperref[def:monoid_action]{monoid actions}, with the same remarks regarding left and right group actions.

  X \term{group action} can be defined equivalently as:
  \begin{thmenum}
    \thmitem{def:group_action/family} An \hyperref[def:cartesian_product/indexed_family]{indexed family} \( \seq{ \Phi_x }_{x \in G} \) of \hyperref[def:morphism_invertibility/isomorphism]{isomorphisms} of \( X \) such that, for every pair \( g, h \in G \),
    \begin{equation}\label{eq:def:group_action/family/compatibility}\tag{\logic{GA}}
      \Phi_{gh} = \Phi_g \bincirc \Phi_h.
    \end{equation}

    This defines a function \( \Phi: G \times A \to A \).

    \thmitem{def:group_action/homomorphism} A \hyperref[def:group/homomorphism]{homomorphism} from \( G \) to the \hyperref[def:automorphism_group]{automorphism group} \( \aut(X) \).

    \thmitem{def:group_action/functor} A \hyperref[def:functor]{functor} from the \hyperref[def:monoid_delooping]{delooping} \( \cat{B}_G \) to \( \cat{C} \) (right actions are \hyperref[rem:contravariant_functor]{contravariant functors}).
  \end{thmenum}
\end{definition}
\begin{proof}
  The proof of equivalence is simple; it is similar to \fullref{def:monoid_action}.
\end{proof}

\begin{lemma}\label{thm:group_operation_is_bijection}
  For each element \( g \) of a group \( G \), consider the function \( \varphi_g \coloneqq g \id_G \), i.e.
  \begin{equation*}
    \begin{aligned}
      &\varphi_x: G \to G \\
      &\varphi_x(y) \coloneqq x \cdot y.
    \end{aligned}
  \end{equation*}

  This is a bijective function (but not necessarily a group isomorphism).
\end{lemma}
\begin{proof}
  \SubProofOf[def:function_invertibility/injective]{injectivity} If \( y_1, y_2 \in G \) and \( \varphi_x(y_1) = \varphi_x(y_2) \), we have
  \begin{equation*}
    xy_1 = \varphi_x(y_1) = \varphi_x(y_2) = xy_2.
  \end{equation*}

  By \fullref{thm:def:group/properties/cancellative}, \( y_1 = y_2 \). Therefore, \( \varphi_x \) is injective.

  \SubProofOf[def:function_invertibility/surjective]{surjectivity} If \( z \in G \), then \( z = x(x^{-1} z) \). Therefore, \( z = \varphi_x(x^{-1} z) \), and thus every member of \( G \) has a preimage. Thus, \( \varphi_x \) is surjective.
\end{proof}

\begin{theorem}[Cayley's theorem]\label{thm:cayleys_theorem}
  Every \hyperref[def:group]{group} \hyperref[def:group_action]{acts} on itself via the family of functions \( y \mapsto x \cdot y \) indexed by \( x \). These functions are not group homomorphisms in general.

  Compare this result to \fullref{thm:monoid_is_action}.
\end{theorem}
\begin{proof}
  Follows directly from \fullref{thm:monoid_is_action} and \fullref{thm:group_operation_is_bijection}.
\end{proof}

\begin{proposition}\label{thm:exponentiation_group_action}
  The \hyperref[def:set_of_integers]{integers} \( \BbbZ \) act on any \hyperref[def:group]{group} by \hyperref[def:unital_magma/exponentiation]{exponentiation} via the family of function \( g \mapsto g^n \) indexed by \( n \in \BbbZ \).

  Compare this result to \fullref{thm:exponentiation_monoid_action}.
\end{proposition}
\begin{proof}
  Follows from \fullref{thm:exponentiation_monoid_action}.
\end{proof}

\begin{definition}\label{def:group_action_orbit}\mcite[163]{Knapp2016BasicAlgebra}
  The \term{orbit} of \( x \) under the \hyperref[def:group_action]{group action} \( \Phi: G \to \End(X) \) is the set
  \begin{equation*}
    \set{ \Phi_g(x) \given g \in G }.
  \end{equation*}

  This is the set of all members of \( X \) \enquote{reachable} from \( x \) via the action.

  The relation \( g \sim h \) on \( G \), defined to hold if \( g \) and \( h \) have the same \hyperref[def:group_action_orbit]{orbit}, is an \hyperref[def:equivalence_relation]{equivalence relation}. The quotient set \( G / {\sim} \) is a partition of \( G \) into sets called \term{orbits}.
\end{definition}

\begin{example}\label{ex:plane_rotation_action_orbits}
  Consider the action of the additive group of \( \BbbR \) on \( \BbbR^2 \) given by the \hyperref[def:euclidean_transformation/rotation]{rotation} \hyperref[def:array/matrix]{matrices}
  \begin{equation*}
    \Phi_r \coloneqq \begin{pmatrix}
      \cos(r)  & \sin(r) \\
      -\sin(r) & \cos(r) \\
    \end{pmatrix}
  \end{equation*}

  Fix a nonzero vector \( (x, y)^T \) in \( \BbbR^2 \) with norm \( l \). Since rotation matrices are orthogonal, they preserve norms. Furthermore, given a vector of norm \( l \), with the angle \( r \) defined via \eqref{eq:def:angle/measure}, \( \Phi_r^{-1} \) sends the vector to \( (x, y)^T \).

  The \hyperref[def:group_action_orbit]{orbit} of \( (x, y)^T \) is thus a \hyperref[def:quadratic_plane_curve/ellipse]{circle} at the origin with radius \( l \).
\end{example}

\begin{proposition}\label{thm:group_conjugation_action}\mcite[165]{Knapp2016BasicAlgebra}
  Every \hyperref[def:group]{group} \hyperref[def:group_action]{acts} on itself via the \term{conjugation automorphisms} defined as
  \begin{equation*}
    \begin{aligned}
      &\Phi_g: G \to G, \\
      &\Phi_g(h) \coloneqq g h g^{-1}.
    \end{aligned}
  \end{equation*}
\end{proposition}
\begin{proof}
  Trivial.
\end{proof}

\begin{definition}\label{def:inner_and_outer_automorphisms}\mcite[exer. 6.12.15]{Knapp2016BasicAlgebra}
  Denote by \( \Phi: G \to \aut(G) \) the \hyperref[thm:group_conjugation_action]{conjugation action} on the group \( G \).

  The \term{inner automorphisms group} of \( G \) is
  \begin{equation*}
    \op{inn}(G) \coloneqq \set{ \Phi_g \given g \in G }.
  \end{equation*}

  The \term{outer automorphism group} is the \hyperref[def:quotient_group]{quotient group}
  \begin{equation*}
    \op{out}(G) \coloneqq \aut(G) / \op{inn}(G).
  \end{equation*}
\end{definition}
\begin{defproof}
  We will show that \( \op{inn}(G) \) is a \hyperref[def:normal_subgroup]{normal subgroup} of \( \aut(G) \).

  Fix a member \( g \) of \( G \) and define the inner automorphism
  \begin{equation*}
    \varphi(h) \coloneqq gh^{-1}g.
  \end{equation*}

  Let \( \psi \) be an arbitrary automorphism of \( G \). Then
  \begin{equation*}
    [\varphi \bincirc \varphi \bincirc \varphi^{-1}](h)
    =
    \varphi(g \varphi^{-1}(h) g^{-1})
    =
    \varphi(g) h \varphi(g^{-1}),
  \end{equation*}
  and thus \( \varphi \bincirc \varphi \bincirc \varphi^{-1} \) is again an inner automorphism.
\end{defproof}

\begin{definition}\label{def:permutation_cycle_decomposition}\mimprovised
  Let \( S_n \) be a \hyperref[def:symmetric_group]{symmetric group} and let \( p \) be a permutation in \( S_n \). Consider the \hyperref[def:group_action]{group action}
  \begin{equation*}
    \begin{aligned}
      &\Phi^{(p)}: \braket{ p } \times \set{ 1, \ldots, n } \to \set{ 1, \ldots, n } \\
      &\Phi^{(p)}_p(k) \coloneqq p(k)
    \end{aligned}
  \end{equation*}
  of the \hyperref[def:first_order_generated_substructure]{generated subgroup} \( \braket{ p } \).

  The \hyperref[def:group_action_orbit]{orbit} of \( k \) under \( \Phi^{(p)} \) is the set
  \begin{equation*}
    O_p(k) \coloneqq \set{ p^m(k) \given m \in \BbbZ }
  \end{equation*}
  of all numbers reachable from \( k \) via iterated application of \( p \) or \( p^{-1} \). The family
  \begin{equation*}
    \set{ O_p(k) \given k = 1, \ldots, n }
  \end{equation*}
  of orbits partitions \( 1, \ldots, n \) into disjoint subsets.

  Each orbit \( O \) has a smallest element \( o \); and \( O_p(o) = O \). This smallest element uniquely identifies a cycle
  \begin{equation*}
    \cycle{o, p(o), p^2(o), \ldots, p^{c-1}(o)},
  \end{equation*}
  where \( c \) is the cardinality of \( O_p(o) \).

  Cycles of length \( 1 \) are not useful to us since they represent fixed points of \( p \); we ignore these cycles. We obtain a unique set of disjoint cycles for every permutation \( p \) in \( S_n \), which we call the \term{cycle decomposition} of \( p \). We denote this set by \( C_p \).
\end{definition}
\begin{defproof}
  We must prove that \( C_p \) is family of disjoint sets. Let \( O_p(k_1) \) and \( O_p(k_2) \) be two orbits.

  Suppose that there exists a number \( k \in O_p(k_2) \cap O_p(k_1) \). Then there exist \( m_1, m_2 \leq n \) such that \( k = p^{m_1}(k_1) \) and \( k = p^{m_2}(k_2) \). Then
  \begin{equation*}
    k_1 = p^{-m_1}(k) = p^{-m_1}(p^{m_2}(k_2)) = p^{m_2 - m_1}(k_2),
  \end{equation*}
  and thus \( O_p(k_2) \subseteq O_p(k_1) \). We can analogously obtain the converse inclusion.

  Therefore, if two orbits have a nonempty intersection, they are equal.
\end{defproof}

\begin{proposition}\label{thm:permutation_decomposition}
  Given a permutation \( p \), for every ordering \( c_1, \ldots, c_m \) of the \hyperref[def:permutation_cycle_decomposition]{cycle decomposition} of \( p \), we have
  \begin{equation*}
    p = c_1 \bincirc \cdots \bincirc c_m.
  \end{equation*}
\end{proposition}
\begin{proof}
  We will use induction on \( m \). The case \( m = 0 \) is trivial. For the inductive hypothesis, note that \( c_1^{-1} p \) has as cycles \( c_2, \ldots, c_m \) because all the numbers in \( c_1 \) are fixed points of \( c_1^{-1} p \). Thus, the inductive hypothesis holds for \( c_1^{-1} p \).

  From
  \begin{equation*}
    c_1^{-1} p = c_2 \bincirc \cdots \bincirc c_m
  \end{equation*}
  it follows that
  \begin{equation*}
    p = c_1 \bincirc \cdots \bincirc c_m.
  \end{equation*}
\end{proof}

\begin{definition}\label{def:permutation_parity}
  Let \( C_p \) be the \hyperref[def:permutation_cycle_decomposition]{cycle decomposition} of \( p \). We say that \( p \) is an \term{even permutation} if the sum
  \begin{equation*}
    \sum_{c \in C_p} (\len(c) - 1)
  \end{equation*}
  is a an even number; and likewise for \term{odd permutations}.

  Using the decomposition of individual cycles from \fullref{thm:cycle_transposition_decomposition}, we obtain that the above is precisely the total number of transpositions in the decomposition of all the cycles.

  We correspondingly define the \term{sign} of a permutation as
  \begin{equation*}
    \begin{aligned}
       &\sgn: S_n \to \set{ -1, 1 }, \\
       &\sgn(p) \coloneqq \begin{cases}
        1,  &p \T{is even} \\
        -1, &p \T{is odd}
      \end{cases}
    \end{aligned}
  \end{equation*}
\end{definition}

\begin{definition}\label{def:alternating_group}
  The \term{alternating group} \( A_n \) of order \( n \) is the subgroup of all \hyperref[def:permutation_parity]{even permutation} in the \hyperref[def:symmetric_group]{symmetric group} \( S_n \).
\end{definition}

\begin{example}\label{ex:s3_and_a3}
  The \hyperref[def:symmetric_group]{symmetric group} \( S_3 \) contains the following \hyperref[def:symmetric_group/permutation]{permutations}:
  \begin{equation*}
    S_3
    \coloneqq
    \set*
    {
      \begin{pmatrix}
        1 & 2 & 3 \\
        1 & 2 & 3
      \end{pmatrix},
      \underbrace
        {
          \begin{pmatrix}
            1 & 2 & 3 \\
            2 & 1 & 3
          \end{pmatrix}
        }_{
          \cycle{ 1, 2 }
        },
      \underbrace
        {
          \begin{pmatrix}
            1 & 2 & 3 \\
            2 & 3 & 1
          \end{pmatrix}
        }_{
          \cycle{ 1, 3 } \bincirc* \cycle{ 1, 2 }
        },
      \underbrace
        {
          \begin{pmatrix}
            1 & 2 & 3 \\
            3 & 2 & 1
          \end{pmatrix}
        }_{
          \cycle{ 1, 3 }
        },
      \underbrace
        {
          \begin{pmatrix}
            1 & 2 & 3 \\
            3 & 1 & 2
          \end{pmatrix}
        }_{
          \cycle{ 1, 2 } \bincirc* \cycle{ 1, 3 }
        },
      \underbrace
        {
          \begin{pmatrix}
            1 & 2 & 3 \\
            1 & 3 & 2
          \end{pmatrix}
        }_{
          \cycle{ 2, 3 }
        }
    }
  \end{equation*}

  The \hyperref[def:alternating_group]{alternating group} \( A_3 \) consists of the first, third and fifth permutations, which decompose into an even number of transpositions.

  The \hyperref[def:automorphism_group]{automorphism group} \( \aut(A_3) \) consists of:
  \begin{itemize}
    \item Conjugation with all three elements gives the identity \( \id_{A_3} \). Thus, the \hyperref[def:inner_and_outer_automorphisms]{inner automorphism group} is trivial.

    \item The map \( p \mapsto p^{-1} \), which takes the inverse function of a permutation, is then an \hyperref[def:inner_and_outer_automorphisms]{outer automorphism}.
  \end{itemize}

  Compare this to the situation in \( S_3 \):
  \begin{itemize}
    \item All permutations except for \( \cycle{ 1, 3 } \bincirc \cycle{ 1, 2 } \) and \( \cycle{ 1, 2 } \bincirc \cycle{ 1, 3 } \) are involutions.
    \item The map \( p \mapsto p^{-1} \) is given by conjugation with \( \cycle{ 1, 3 } \).
    \item Every conjugation automorphism is unique, and thus
    \begin{equation*}
      \op{inn}(S_3) = \aut(S_3) \cong S_3.
    \end{equation*}

    In particular, the outer automorphism group \( \op{out}(S_3) \) is trivial.
  \end{itemize}
\end{example}

\begin{definition}\label{def:dynamical_system}\mimprovised
  Suppose that \( \cat{C} \) is a \hyperref[def:concrete_category]{concrete category} and let \( X \) be an object of \( \cat{C} \).

  A \term{dynamical system} is a \hyperref[def:monoid_action]{monoid action} \( \Phi: T \times X \to X \). We call \( X \) the \term{phase space} of the system. In applications, we interpret the monoid \( T \) as \term{time} and consider it to be \hyperref[rem:additive_magma]{additive}. We call \( \Phi \) the \term{evolution function} of the system.

  We will call the system \term{discrete} if \( T \) is the monoid of \hyperref[def:set_of_natural_numbers]{natural numbers} (with zero) and \term{continuous} if \( T \) is the monoid of nonnegative \hyperref[def:set_of_real_numbers]{real numbers}, with or without \hyperref[def:extended_real_numbers]{an infinite element}.

  The monoid \( T \) can theoretically be a \hyperref[def:group]{group}, in which case we consider \hyperref[def:group_action]{group actions}, however negative time is not as often needed in practice.
\end{definition}

\begin{example}\label{ex:abstract_reductions_as_dynamical_systems}
  Every \hyperref[def:abstract_reduction_system]{abstract reduction system} \( (A, \to) \) gives rise to many \hyperref[def:dynamical_system]{discrete dynamical system}. Let \( c \) be a \hyperref[def:choice_function]{choice function} for \( A \). We can inductively define the evolution function:
  \begin{equation*}
    \begin{aligned}
      \Phi: \BbbN \times A \to A \\
      \Phi_t(x) \coloneqq \begin{cases}
        x,                          &t = 0, \\
        x                           &t = 1 \T{and} x \T{is irreducible}, \\
        c(\set{ y \given x \to y }) &t = 1 \T{and} x \T{is reducible}, \\
        \Phi_1(\Phi_{t-1}(x))       &t > 1
      \end{cases}
    \end{aligned}
  \end{equation*}

  This shows how abstract reduction systems can be used to model computation. As long as we have a procedure for deciding which branch of computation to follow, we can regard the reduction system as a deterministic time-dependent computation device. If the reduction system is \hyperref[def:abstract_rewriting_convergence/convergent]{convergent}, the choice function \( c \) should be irrelevant since we will eventually reach the only end of the computation.

  If, instead, we want to try all possible branches simultaneously, the device becomes nondeterministic and no longer fits as nicely in the abstraction of dynamical systems.
\end{example}

\begin{definition}\label{def:dynamical_system_trajectory}
  Fix a \hyperref[def:dynamical_system]{dynamical system} with evolution function \( \Phi: T \times X \to X \).

  A \term{trajectory} in a starting at the \term{initial state} \( x_0 \in X \) is an \hyperref[def:cartesian_product/indexed_family]{indexed family} \( \seq{ x_t }_{t \in T} \) obtained as
  \begin{equation*}
    x_t \coloneqq \Phi_t(x_0).
  \end{equation*}

  The condition \ref{eq:def:monoid_action/family/identity} ensures that \( \Phi_0(x_0) = x_0 \), and \ref{eq:def:monoid_action/family/compatibility} ensures that
  \begin{equation*}
    x_{t + s}
    =
    \Phi_{t + s}(x_0)
    =
    \Phi_t(x_s).
  \end{equation*}

  For discrete dynamical systems, trajectories are sequences.
\end{definition}

\begin{example}\label{ex:abstract_reduction_trajectories}
  In \fullref{ex:abstract_reductions_as_dynamical_systems} we described how, given a choice function \( c \), an abstract reduction system \( (A, \to) \) can be regarded as a discrete dynamical system with evolution function \( \Phi: \BbbN \times A \to A \).

  A \hyperref[def:dynamical_system_trajectory]{trajectory} in this system is a reduction path. Without the choice function, trajectories are general \hyperref[def:arborescence]{arborescences} rather than paths, with every branching of the tree corresponding to some parallel computation.
\end{example}
