\subsection{Group actions}\label{subsec:group_actions}

\begin{definition}\label{def:endomorphism_monoid}
  For every object \( X \) in an arbitrary \hyperref[def:category]{category} \( \cat{C} \), the set \( \cat{C}(X) \) is a \hyperref[def:monoid]{monoid} with morphism composition as the monoid operation and \( \id_X \) as the monoid identity.

  Outside of \hyperref[sec:category_theory]{category theory}, whenever the category \( \cat{C} \) is clear from the context, we call \( \cat{C}(X) \) the \term{endomorphism monoid} over \( X \) and denote it by \( \End(X) \).
\end{definition}

\begin{definition}\label{def:monoid_action}
  Let \( M \) be a \hyperref[def:monoid]{monoid} and let \( X \) be an object in some \hyperref[def:category]{category} \( \cat{C} \).

  We will define monoid actions of \( M \) on \( X \), which we will sometimes call \term{left monoid actions}. There are also \term{right monoid actions}, which are only briefly mentioned in \fullref{def:monoid_action/functor}.

  A \term{monoid action} can be defined equivalently as:
  \begin{thmenum}
    \thmitem{def:monoid_action/homomorphism} A \hyperref[def:monoid/homomorphism]{homomorphism} from \( M \) to the \hyperref[def:endomorphism_monoid]{endomorphism monoid} \( \End(X) \).

    \thmitem{def:monoid_action/functor} A \hyperref[def:functor]{functor} from the \hyperref[def:monoid_delooping]{delooping} \( \cat{B}_M \) to \( \cat{C} \).

    Right actions are \hyperref[rem:contravariant_functor]{contravariant functors}.

    \thmitem{def:monoid_action/family} An \hyperref[def:cartesian_product/indexed_family]{indexed family} \( \seq{ \Phi_m }_{m \in M} \) of \hyperref[def:morphism_invertibility/endomorphism]{endomorphisms} of \( X \) such that
    \begin{align}
      &\Phi_e = \id_X, \label{eq:def:monoid_action/family/identity}\tag{\logic{MA1}} \\
      &\Phi_{mn} = \Phi_m \bincirc \Phi_n. \label{eq:def:monoid_action/family/compatibility}\tag{\logic{MA2}}
    \end{align}

    This defines a function \( \Phi: M \times A \to A \).
  \end{thmenum}
\end{definition}
\begin{proof}
  \ImplicationSubProof{def:monoid_action/homomorphism}{def:monoid_action/functor} Suppose that we have a monoid homomorphism \( \Phi: M \to \End(X) \). Define the functor
  \begin{equation*}
    \begin{aligned}
      &F: \cat{B}_M \to \cat{C} \\
      &F(\anon) \coloneqq X \\
      &F(m) \coloneqq \Phi(m).
    \end{aligned}
  \end{equation*}

  This is indeed a functor because \eqref{eq:def:functor/CF1} follows from \eqref{eq:rem:pointed_set/homomorphism} and \eqref{eq:def:functor/CF2} follows from \eqref{eq:def:magma/homomorphism}.

  \ImplicationSubProof{def:monoid_action/functor}{def:monoid_action/family} Suppose that we have a functor \( F: \cat{B}_M \to \cat{C} \). Let \( X \coloneqq F(\anon) \) and define the \( M \)-indexed family
  \begin{equation*}
    \begin{aligned}
      &\Phi_m: X \to X \\
      &\Phi_m \coloneqq F(m).
    \end{aligned}
  \end{equation*}

  It satisfies the necessary axioms:
  \begin{itemize}
    \item \ref{eq:def:monoid_action/family/identity} holds:
    \begin{equation*}
      \Phi_e
      =
      F(e)
      \reloset {\eqref{eq:def:functor/CF1}} =
      \id_A.
    \end{equation*}

    \item \ref{eq:def:monoid_action/family/compatibility} holds: for every pair \( m, n \in M \), we have
    \begin{equation*}
      \Phi_{mn}
      =
      F(mn)
      \reloset {\eqref{eq:def:functor/CF2}} =
      F(m) \bincirc F(n)
      =
      \Phi_m \bincirc \Phi_n
    \end{equation*}
  \end{itemize}

  \ImplicationSubProof{def:monoid_action/family}{def:monoid_action/homomorphism} Suppose that we have an indexed family \( \seq{ \Phi_m }_{m \in M} \) of endomorphisms of \( A \) that satisfies the axioms for left action. Regard this indexed family as a function \( \Phi: M \to \End(X) \).

  Then \( \Phi \) is a monoid homomorphism because \ref{eq:def:monoid_action/family/identity} implies \( \Phi(e) = \id_X \) and \eqref{eq:def:monoid_action/family/compatibility} implies
  \begin{equation*}
    \Phi(mn) = \Phi(m) \bincirc \Phi(n).
  \end{equation*}
\end{proof}

\begin{proposition}\label{thm:monoid_is_action}
  Every \hyperref[def:monoid]{monoid} \hyperref[def:monoid_action]{acts} on itself via the family of functions \( h \mapsto g \cdot h \) indexed by \( g \). These functions are not monoid homomorphisms in general.

  Compare this result to \fullref{thm:cayleys_theorem}.
\end{proposition}
\begin{proof}
  The family satisfies \fullref{def:monoid_action/family}:
  \begin{itemize}
    \item \ref{eq:def:monoid_action/family/identity} follows from \eqref{eq:def:monoid/theory/identity}.

    \item \ref{eq:def:monoid_action/family/compatibility} follows from associativity:
    \begin{equation*}
      [h \mapsto g_1 \cdot h] \bincirc [h \mapsto g_2 \cdot h] = [h \mapsto g_1 \cdot (g_2 \cdot h)] = [h \mapsto (g_1 \cdot g_2) \cdot h].
    \end{equation*}
  \end{itemize}
\end{proof}

\begin{proposition}\label{thm:exponentiation_monoid_action}
  The \hyperref[def:natural_numbers]{natural numbers} \( \BbbN \) (\hyperref[rem:peano_arithmetic_zero]{with zero}) act on any \hyperref[def:monoid]{monoid} by \hyperref[def:monoid/exponentiation]{exponentiation} via the family of function \( g \mapsto g^n \) indexed by \( n \in \BbbN \).

  Compare this result to \fullref{thm:exponentiation_group_action}.
\end{proposition}
\begin{proof}
  This family satisfies \fullref{def:monoid_action/family}:
  \begin{itemize}
    \item \ref{eq:def:monoid_action/family/identity} is obvious.
    \item \ref{eq:def:monoid_action/family/compatibility} follows from \fullref{thm:magma_exponentiation_properties/repeated}.
  \end{itemize}
\end{proof}

\begin{definition}\label{def:automorphism_group}
  For every object \( X \) in a \hyperref[def:groupoid]{groupoid} \( \cat{G} \), the set \( \cat{G}(X) \) is a \hyperref[def:group]{group} with morphism composition as the group operation.

  Similarly to \hyperref[def:endomorphism_monoid]{endomorphism monoids}, whenever the groupoid \( \cat{G} \) is clear from the context, we call \( \cat{G}(X) \) the \term{automorphism group} over \( X \) and denote it by \( \aut(X) \).
\end{definition}

\begin{definition}\label{def:symmetric_group}
  We call the \hyperref[def:automorphism_group]{automorphism group} of a set \( A \) the \term{symmetric group}\footnote{The term possibly comes from symmetric functions defined in \fullref{def:symmetric_function}} on \( A \) and denote it by \( S(A) \). The group \( S(A) \) consists of bijective functions, which we call \term{permutations}.

  Rather than for arbitrary sets, we often consider symmetric group
  \begin{equation*}
    S_n \coloneqq S(\set{ 1, 2, \ldots, n }).
  \end{equation*}

  It is conventional to call \( S_n \) the \enquote{symmetric group on \( n \) letters}.

  \begin{thmenum}
    \thmitem{def:symmetric_group/permutation} It is common to write a permutation \( \sigma \) in \( S_n \) as
    \begin{equation*}
      \begin{pmatrix}
        1         & \cdots & n \\
        \sigma(1) & \cdots & \sigma(n)
      \end{pmatrix}
    \end{equation*}

    \thmitem{def:symmetric_group/cycle} If there exists a finite sequence \( k_1, \ldots, k_m \) of distinct numbers such that \( \sigma(k_m) = k_1 \) and \( \sigma(k_{i+1}) = \sigma(k_i) \) for each \( i < m \), we say that the permutation is \term{cyclic} or a \term{cycle} of \term{length} \( m \). For brevity, we denote this cycle by \( \cycle{k_1, \cdots, k_m} \).

    We call cycles of length \( 1 \) \term{trivial cycles} or \term{loops} and cycles of length \( 2 \) --- \term{transpositions}. Every \hyperref[def:symmetric_group/cycle]{transposition} is an \hyperref[def:set_with_involution]{involution}, and thus a transposition is equal to its inverse permutation.

    \thmitem{def:symmetric_group/disjoint_cycle} If \( \cycle{k_1, \cdots, k_m} \) and \( \cycle{s_1, \cdots, s_l} \) are two cycles and if the sets \( \set{ k_1, \ldots, k_m } \) and \( \set{ s_1, \ldots, s_l } \) are disjoint, we say that the cycles themselves are \term{disjoint}.

    Two disjoint cycles commute. That is,
    \begin{equation*}
      \cycle{k_1, \cdots, k_m} \bincirc \cycle{s_1, \cdots, s_l} = \cycle{s_1, \cdots, s_l} \bincirc \cycle{k_1, \cdots, k_m}.
    \end{equation*}
  \end{thmenum}
\end{definition}

\begin{proposition}\label{thm:cycle_transposition_decomposition}
  Every nontrivial cycle \( \cycle{ k_1, \cdots, k_m } \) can be decomposed into the product of transpositions
  \begin{equation*}
    \cycle{ k_1, \cdots, k_m } = \cycle{ k_1, k_m } \bincirc \cycle{ k_1, k_{m-1} } \bincirc \cdots \bincirc \cycle{ k_1, k_2 }.
  \end{equation*}
\end{proposition}
\begin{proof}
  Trivial.
\end{proof}

\begin{proposition}\label{thm:symmetric_group_cardinality}
  The \hyperref[def:symmetric_group]{symmetric group} \( S_n \) has \( n! \) elements.
\end{proposition}
\begin{proof}
  We use induction on \( n \). The case \( n = 1 \) is trivial. Suppose that \( S_{n-1} \) has \( (n-1)! \) elements. Then \( S_n \) is obtained by permuting \( n \) with each element of \( S_{n-1} \). That is,
  \begin{equation*}
    S_n = \set{ \cycle{ k, n } \bincirc \sigma \given \sigma \in S_{n-1} \T{and} 1 \leq k \leq n }.
  \end{equation*}

  It follows that
  \begin{equation*}
    \card(S_n) = n \cdot \card(S_{n-1}) = n (n-1)! = n!.
  \end{equation*}
\end{proof}

\begin{definition}\label{def:group_action}
  Let \( G \) be a \hyperref[def:group]{group} and let \( X \) be an object in some \hyperref[def:concrete_category]{concrete category} \( \cat{C} \).

  We will define group actions as a special case of \hyperref[def:monoid_action]{monoid actions}, with the same remarks regarding left and right group actions.

  A \term{group action} can be defined equivalently as:
  \begin{thmenum}
    \thmitem{def:group_action/homomorphism} A \hyperref[def:group/homomorphism]{homomorphism} from \( G \) to the \hyperref[def:automorphism_group]{automorphism group} \( \aut(X) \).

    Right actions are homomorphisms from the \hyperref[def:monoid/opposite]{opposite} group \( G^{\opcat} \) to \( \aut(X) \).

    \thmitem{def:group_action/functor} A \hyperref[def:functor]{functor} from the \hyperref[def:monoid_delooping]{delooping} \( \cat{B}_G \) to \( \cat{C} \).

    Right actions are \hyperref[rem:contravariant_functor]{contravariant functors}.

    \thmitem{def:group_action/family} An \hyperref[def:cartesian_product/indexed_family]{indexed family} \( \seq{ \Phi_x }_{x \in G} \) of \hyperref[def:morphism_invertibility/isomorphism]{isomorphisms} of \( X \) such that, for every pair \( g, h \in G \),
    \begin{equation}\label{eq:def:group_action/family/compatibility}\tag{\logic{GA}}
      \Phi_{gh} = \Phi_g \bincirc \Phi_h.
    \end{equation}

    This defines a function \( \Phi: G \times A \to A \).
  \end{thmenum}
\end{definition}
\begin{proof}
  The proof of equivalence is simple; it is similar to \fullref{def:monoid_action}.
\end{proof}

\begin{theorem}[Cayley's theorem]\label{thm:cayleys_theorem}
  Every \hyperref[def:group]{group} \hyperref[def:group_action]{acts} on itself via the family of functions \( y \mapsto x \cdot y \) indexed by \( x \). These functions are not group homomorphisms in general.

  Compare this result to \fullref{thm:monoid_is_action}.
\end{theorem}
\begin{proof}
  Follows directly from \fullref{thm:monoid_is_action} and \fullref{thm:group_operation_induces_bijections}.
\end{proof}

\begin{proposition}\label{thm:exponentiation_group_action}
  The \hyperref[def:integers]{integers} \( \BbbZ \) act on any \hyperref[def:group]{group} by \hyperref[def:monoid/exponentiation]{exponentiation} via the family of function \( g \mapsto g^n \) indexed by \( n \in \BbbZ \).

  Compare this result to \fullref{thm:exponentiation_monoid_action}.
\end{proposition}
\begin{proof}
  Follows from \fullref{thm:exponentiation_monoid_action}.
\end{proof}

\begin{definition}\label{def:group_action_orbit}\mcite[163]{Knapp2016BasicAlgebra}
  The \term{orbit} of \( x \) under the \hyperref[def:group_action]{group action} \( \Phi: G \to \End(X) \) is the set
  \begin{equation*}
    \set{ \Phi_g(x) \given g \in G }.
  \end{equation*}

  This is the set of all members of \( X \) \enquote{reachable} from \( x \) via the action.

  The relation \( g \sim h \) on \( G \), defined to hold if \( g \) and \( h \) have the same \hyperref[def:group_action_orbit]{orbit}, is an \hyperref[def:equivalence_relation]{equivalence relation}. The quotient set \( G / {\sim} \) is a partition of \( G \) into sets called \term{orbits}.
\end{definition}

\begin{example}\label{ex:plane_rotation_action_orbits}
  Consider the action of the additive group of \( \BbbR \) on \( \BbbR^2 \) given by the \hyperref[def:rigid_motion/rotation]{rotation} \hyperref[def:array/matrix]{matrices}
  \begin{equation*}
    \Phi_r \coloneqq \begin{pmatrix}
      \cos(r)  & \sin(r) \\
      -\sin(r) & \cos(r) \\
    \end{pmatrix}
  \end{equation*}

  Fix a nonzero vector \( (x, y)^T \) in \( \BbbR^2 \) with norm \( l \). Since rotation matrices are orthogonal, they preserve norms. Furthermore, given a vector of norm \( l \), with the angle \( r \) defined via \eqref{eq:def:angle/measure}, \( \Phi_r^{-1} \) sends the vector to \( (x, y)^T \).

  The \hyperref[def:group_action_orbit]{orbit} of \( (x, y)^T \) is thus a \hyperref[def:quadratic_plane_curve/ellipse]{circle} at the origin with radius \( l \).
\end{example}

\begin{proposition}\label{thm:group_conjugation_action}\mcite[165]{Knapp2016BasicAlgebra}
  Every \hyperref[def:group]{group} \hyperref[def:group_action]{acts} on itself via the \term{conjugation automorphisms} defined as
  \begin{equation*}
    \begin{aligned}
      &\Phi_g: G \to G, \\
      &\Phi_g(h) \coloneqq g h g^{-1}.
    \end{aligned}
  \end{equation*}
\end{proposition}
\begin{proof}
  Trivial.
\end{proof}

\begin{definition}\label{def:inner_and_outer_automorphisms}\mcite[exer. 6.12.15]{Knapp2016BasicAlgebra}
  Denote by \( \Phi: G \to \aut(G) \) the \hyperref[thm:group_conjugation_action]{conjugation action} on the group \( G \).

  The \term{inner automorphisms group} of \( G \) is
  \begin{equation*}
    \op{inn}(G) \coloneqq \set{ \Phi_g \given g \in G }.
  \end{equation*}

  The \term{outer automorphism group} is the \hyperref[def:group/quotient]{quotient group}
  \begin{equation*}
    \op{out}(G) \coloneqq \aut(G) / \op{inn}(G).
  \end{equation*}
\end{definition}
\begin{defproof}
  We will show that \( \op{inn}(G) \) is a \hyperref[thm:normal_subgroup_equivalences]{normal subgroup} of \( \aut(G) \).

  Fix a member \( g \) of \( G \) and define the inner automorphism
  \begin{equation*}
    \varphi(h) \coloneqq gh^{-1}g.
  \end{equation*}

  Let \( \psi \) be an arbitrary automorphism of \( G \). Then
  \begin{equation*}
    [\varphi \bincirc \psi \bincirc \varphi^{-1}](h)
    =
    \varphi(g \varphi^{-1}(h) g^{-1})
    =
    \varphi(g) h \varphi(g^{-1}),
  \end{equation*}
  and thus \( \varphi \bincirc \psi \bincirc \varphi^{-1} \) is again an inner automorphism.
\end{defproof}

\begin{proposition}\label{thm:permutation_decomposition_existence}
  Let \( S_n \) be a \hyperref[def:symmetric_group]{symmetric group} and let \( \sigma \) be a permutation in \( S_n \). Then there exists a finite sequence \( c_1, \ldots, c_m \) of nontrivial \hyperref[def:symmetric_group/disjoint_cycle]{disjoint cycles} such that
  \begin{equation*}
    \sigma = c_1 \bincirc \cdots \bincirc c_m.
  \end{equation*}

  The case where \( \sigma \) is the identity permutation corresponds to \( m = 0 \).

  We call this a \term{cycle decomposition} of \( \sigma \). Via \fullref{thm:cycle_transposition_decomposition}, this permutation can further be decomposed into a composition of individual transpositions. We call the latter a \term{transposition decomposition} of \( \sigma \).

  Neither decomposition is unique. Nonetheless, existence is sufficient for most practical purposes, including permutation parity defined in \fullref{def:permutation_parity}.
\end{proposition}
\begin{proof}
  Suppose that \( \sigma \) is not the identity.

  Consider the \hyperref[def:group_action]{group action}
  \begin{equation*}
    \begin{aligned}
      &\Phi^{(\sigma)}: \braket{ \sigma } \times \set{ 1, \ldots, n } \to \set{ 1, \ldots, n } \\
      &\Phi^{(\sigma)}_\sigma(k) \coloneqq \sigma(k)
    \end{aligned}
  \end{equation*}
  of the \hyperref[def:first_order_generated_substructure]{generated subgroup} \( \braket{ \sigma } \).

  The \hyperref[def:group_action_orbit]{orbit} of \( k \) under \( \Phi^{(\sigma)} \) is the set
  \begin{equation*}
    O_\sigma(k) \coloneqq \set{ \sigma^m(k) \given m \in \BbbZ }
  \end{equation*}
  of all numbers reachable from \( k \) via iterated application of \( \sigma \) or \( \sigma^{-1} \). The family
  \begin{equation*}
    \set{ O_\sigma(k) \given k = 1, \ldots, n }
  \end{equation*}
  of orbits partitions \( 1, \ldots, n \) into disjoint subsets.

  Each orbit \( O \) has a smallest element \( o \); and \( O_\sigma(o) = O \). This smallest element uniquely identifies a cycle
  \begin{equation*}
    \cycle{o, \sigma(o), \sigma^2(o), \ldots, \sigma^{c-1}(o)},
  \end{equation*}
  where \( c \) is the cardinality of \( O_\sigma(o) \).

  Ignoring the trivial cycles, we obtain a unique set of disjoint nontrivial cycles for every permutation \( \sigma \) in \( S_n \), which we call the \term{cycle decomposition} of \( \sigma \). Denote this set by \( C_\sigma \).

  We must prove that \( C_\sigma \) is family of disjoint sets. Let \( O_\sigma(k_1) \) and \( O_\sigma(k_2) \) be two orbits.

  Suppose that there exists a number \( k \in O_\sigma(k_2) \cap O_\sigma(k_1) \). Then there exist \( m_1, m_2 \leq n \) such that \( k = \sigma^{m_1}(k_1) \) and \( k = \sigma^{m_2}(k_2) \). Then
  \begin{equation*}
    k_1 = \sigma^{-m_1}(k) = \sigma^{-m_1}(\sigma^{m_2}(k_2)) = \sigma^{m_2 - m_1}(k_2),
  \end{equation*}
  and thus \( O_\sigma(k_2) \subseteq O_\sigma(k_1) \). We can analogously obtain the converse inclusion.

  Therefore, if two orbits have a nonempty intersection, they are equal.

  Finally, we will use induction on \( m \) to show that these cycles give us \( \sigma \). The case \( m = 0 \) is trivial. For the inductive hypothesis, note that \( c_1^{-1} \sigma \) has as cycles \( c_2, \ldots, c_m \) because all the numbers in \( c_1 \) are fixed points of \( c_1^{-1} \sigma \). Thus, the inductive hypothesis holds for \( c_1^{-1} \sigma \).

  From
  \begin{equation*}
    c_1^{-1} \sigma = c_2 \bincirc \cdots \bincirc c_m
  \end{equation*}
  it follows that
  \begin{equation*}
    \sigma = c_1 \bincirc \cdots \bincirc c_m.
  \end{equation*}
\end{proof}

\begin{lemma}\label{thm:permutation_parity_correctness}
  If a \hyperref[def:symmetric_group/permutation]{permutation} \( \sigma \in S_n \) can be decomposed into \hyperref[def:symmetric_group/cycle]{transpositions} as both
  \begin{equation}\label{eq:thm:permutation_parity_correctness/k}
    \sigma = \underbrace{\cycle{ k_1, k_2 } \bincirc \cycle{ k_3, k_4 } \bincirc \cdots \bincirc \cycle{ k_{2n-1}, k_{2n} }}_{n \T*{transpositions}}
  \end{equation}
  and
  \begin{equation}\label{eq:thm:permutation_parity_correctness/l}
    \sigma = \underbrace{\cycle{ l_1, l_2 } \bincirc \cycle{ l_3, l_4 } \bincirc \cdots \bincirc \cycle{ l_{2m-1}, l_{2m} }}_{m \T*{transpositions}},
  \end{equation}
  then \( n - m \) is an even number.
\end{lemma}
\begin{proof}
  We will use induction on \( n \). First consider the base case \( n = 0 \). Then \( \sigma \) is the identity. Hence, every transposition in \eqref{eq:thm:permutation_parity_correctness/l} should be present twice so that its action cancels out. Therefore, \( m \) is an even number.

  Now suppose that the statement holds for \( n - 1 \). Add (compose on the right) the last transposition \( \cycle{ k_{2n-1}, k_{2n} } \) of \eqref{eq:thm:permutation_parity_correctness/k} to both \eqref{eq:thm:permutation_parity_correctness/k} and \eqref{eq:thm:permutation_parity_correctness/l}. The obtained permutations are obviously equal. Furthermore, since \( \cycle{ k_{2n-1}, k_{2n} } \) is its own inverse, we can just as well remove \( \cycle{ k_{2n-1}, k_{2n} } \) from \eqref{eq:thm:permutation_parity_correctness/k} to obtain a decomposition into \( n - 1 \) (rather than \( n + 1 \)) transpositions.

  By the inductive hypothesis, \( (n - 1) - (m + 1) = n - m - 2 \) is an even number. Therefore, \( n - m \) is also an even number.
\end{proof}

\begin{definition}\label{def:permutation_parity}
  We say that a \hyperref[def:symmetric_group/permutation]{permutation} \( \sigma \in S_n \) is \term{even} (resp. \term{odd}) if a decomposition of \( \sigma \) into \hyperref[def:symmetric_group/cycle]{transpositions} has an even (resp. odd) number of transpositions.

  By \fullref{thm:permutation_decomposition_existence}, such a decomposition exists. By \fullref{thm:permutation_parity_correctness}, all such decompositions yield the same parity even if they decompose into a differing number of transpositions.

  We correspondingly define the \term{sign} of a permutation as
  \begin{equation*}
    \begin{aligned}
       &\sgn: S_n \to \BbbZ, \\
       &\sgn(\sigma) \coloneqq \begin{cases}
        1,  &\sigma \T{is even} \\
        -1, &\sigma \T{is odd}
      \end{cases}
    \end{aligned}
  \end{equation*}
\end{definition}

\begin{definition}\label{def:alternating_group}
  The \term{alternating group} \( A_n \) on \( n \) letters is the subgroup of all \hyperref[def:permutation_parity]{even permutation} in the \hyperref[def:symmetric_group]{symmetric group} \( S_n \).
\end{definition}

\begin{proposition}\label{thm:alternating_group_cardinality}
  The \hyperref[def:alternating_group]{alternating group} \( A_n \) has \( \ifrac {n!} 2 \) elements.
\end{proposition}
\begin{proof}
  The proof is similar to that of \fullref{thm:symmetric_group_cardinality}, but is a little different.

  We use induction on \( n \). The case \( n = 1 \) is trivial. Suppose that \( A_{n-1} \) has \( \ifrac {(n-1)!} 2 \) elements. Then
  \begin{equation*}
    A_n = \set{ \cycle{ k, n } \bincirc \sigma \given \sigma \in S_{n-1} \setminus A_{n-1} \T{and} 1 \leq k \leq n }.
  \end{equation*}

  We obtain \( A_n \) by taking all the odd permutations in \( S_{n-1} \) and composing them with one new transposition. It follows that
  \begin{equation*}
    \card(A_n) = n \cdot \card(S_{n-1} \setminus A_{n-1}) = n \frac {(n-1)!} 2 = \frac {n!} 2.
  \end{equation*}
\end{proof}

\begin{example}\label{ex:s3_and_a3}
  The \hyperref[def:symmetric_group]{symmetric group} \( S_3 \) contains the following \hyperref[def:symmetric_group/permutation]{permutations}:
  \begin{equation*}
    S_3
    \coloneqq
    \set[\vast]
    {
      \begin{pmatrix}
        1 & 2 & 3 \\
        1 & 2 & 3
      \end{pmatrix},
      \underbrace
        {
          \begin{pmatrix}
            1 & 2 & 3 \\
            2 & 1 & 3
          \end{pmatrix}
        }_{
          \cycle{ 1, 2 }
        },
      \underbrace
        {
          \begin{pmatrix}
            1 & 2 & 3 \\
            2 & 3 & 1
          \end{pmatrix}
        }_{
          \cycle{ 1, 2, 3 }
        },
      \underbrace
        {
          \begin{pmatrix}
            1 & 2 & 3 \\
            3 & 2 & 1
          \end{pmatrix}
        }_{
          \cycle{ 1, 3 }
        },
      \underbrace
        {
          \begin{pmatrix}
            1 & 2 & 3 \\
            3 & 1 & 2
          \end{pmatrix}
        }_{
          \cycle{ 1, 3, 2 }
        },
      \underbrace
        {
          \begin{pmatrix}
            1 & 2 & 3 \\
            1 & 3 & 2
          \end{pmatrix}
        }_{
          \cycle{ 2, 3 }
        }
    }
  \end{equation*}

  We can observe the following:
  \begin{itemize}
    \item The permutations \( \cycle{ 1, 2, 3 } \) and \( \cycle{ 1, 3, 2 } \) are inverses of each other and all other permutations are involutions.

    \item Every conjugation automorphism is unique. This can be verified explicitly. Therefore, the \hyperref[def:inner_and_outer_automorphisms]{inner automorphism group} \( \op{inn}(S_3) \) is isomorphic to \( S_3 \).

    \item The \hyperref[def:alternating_group]{alternating group} \( A_3 \) consists of the identity and the odd-length cycles \( (1, 2, 3) \) and \( (1, 3, 2) \).

    \item When restricted to \( A_3 \), all conjugation automorphisms are trivial. This can be verified explicitly. Therefore, the inner automorphism group \( \op{inn}(A_3) \) is trivial, and hence
    \begin{equation*}
      \aut(A_3) \cong \op{out}(A_3).
    \end{equation*}

    \item The map \( \sigma \mapsto \sigma^{-1} \), which fixes the identity and exchanges the two other permutations, is an automorphism of \( A_3 \). It is distinct from the identity, hence it is an outer automorphism.

    This map is given by the restriction of the conjugation \( \sigma \mapsto \cycle{1, 2} \sigma \cycle{1, 2} \) to \( A_3 \). It is an inner automorphism of \( S_3 \), but an outer automorphism of \( A_3 \).
  \end{itemize}
\end{example}

\begin{proposition}\label{thm:group_epimorphisms_are_surjective}\mcite[exer. I.5.5]{MacLane1994}
  Every \hyperref[def:morphism_invertibility/right_cancellative]{epimorphism} in \hyperref[def:group/category]{\( \cat{Grp} \)} is \hyperref[def:function_invertibility/surjective]{surjective}.
\end{proposition}
\begin{proof}
  Let \( \varphi: G \to H \) be an epimorphism and suppose that it is not surjective. Let \( M \) be the smallest normal subgroup of \( H \) containing \( \img \varphi \).

  If \( M \) has index \( 2 \) in \( H \), consider the quotient map \( \pi: H \to H / M \) and the constant map \( c(h) \coloneqq M \). Then
  \begin{equation*}
    \pi \bincirc \varphi = c \bincirc \varphi.
  \end{equation*}

  Since \( \varphi \) is an epimorphism, we have \( \pi = c \). But we have deliberately taken \( \pi \) and \( c \) so that \( \pi \neq c \). The obtained contradiction shows that \( M \) must have an index greater than \( 2 \).

  Let \( M \), \( uM \) and \( vM \) be different cosets. Define \( \sigma: H \to H \) as the \hyperref[def:symmetric_group/permutation]{permutation} on \( H \) that exchanges \( xu \) with \( xv \) for every \( x \in M \). Define the homomorphism
  \begin{equation*}
    \begin{aligned}
      &\psi: H \to S(H) \\
      &\psi(h) \coloneqq (x \mapsto hx),
    \end{aligned}
  \end{equation*}
  where \( S(H) \) is the \hyperref[def:symmetric_group]{symmetric group}.

  This is indeed a homomorphism by \fullref{thm:cayleys_theorem}. By \fullref{thm:group_conjugation_action}, another homomorphism is
  \begin{equation*}
    \begin{aligned}
      &\theta: H \to S(H) \\
      &\theta(h) \coloneqq \sigma^{-1} \bincirc \psi(h) \bincirc \sigma.
    \end{aligned}
  \end{equation*}

  Since \( \sigma \) fixes the members of \( M \) in-place, we have \( \theta(h)\restr_M = \psi(h)\restr_M \). Since \( M \) contains the image of \( \varphi \), this implies
  \begin{equation*}
    \psi \bincirc \varphi = \theta \bincirc \varphi.
  \end{equation*}

  Since \( \varphi \) is an epimorphism, we have \( \psi = \theta \). But we have deliberately constructed \( \psi \) and \( \theta \) such that \( \psi \neq \theta \). The obtained contradiction shows that \( \img \varphi \) cannot be a strict subgroup of \( G \). Therefore, \( \varphi \) must be surjective.
\end{proof}

\begin{definition}\label{def:dynamical_system}\mimprovised
  Suppose that \( \cat{C} \) is a \hyperref[def:concrete_category]{concrete category} and let \( X \) be an object of \( \cat{C} \).

  A \term{dynamical system} is a \hyperref[def:monoid_action]{monoid action} \( \Phi: T \times X \to X \). We call \( X \) the \term{phase space} of the system. In applications, we interpret the monoid \( T \) as \term{time} and consider it to be \hyperref[rem:additive_magma]{additive}. We call \( \Phi \) the \term{evolution function} of the system.

  \begin{thmenum}
    \thmitem{def:dynamical_system/ifs} If \( T \) is either the additive monoid of the \hyperref[def:natural_numbers]{natural numbers} or the additive group of the \hyperref[def:integers]{integers}, we say that the dynamical system has \term{discrete time}.

    Due to \eqref{eq:def:monoid_action/family/compatibility}, \( \Phi_{n+1} = \Phi_n \bincirc \Phi_1 \) for any integer \( n \). Using \hyperref[rem:induction/peano_arithmetic]{natural number induction} and \fullref{thm:def:group/negative_power}, we can show that \( \Phi_n = \Phi_1^n \) for every integer \( n \).

    Therefore, the entire evolution function of a discrete-time dynamical system is determined by a single endofunction \( \varphi: X \to X \). For this reason, we also refer to discrete-time dynamical systems as \term{iterated function systems}.

    \thmitem{def:dynamical_system/semiflow} If \( T \) is the additive monoid of \hyperref[def:real_numbers]{real numbers}, with or without \hyperref[def:extended_real_numbers]{an infinite element}, we say that the system is a \term{semiflow}.

    \thmitem{def:dynamical_system/flow} If \( T \) is the additive group of \hyperref[def:real_numbers]{real numbers}, we say that the system is a \term{flow}.
  \end{thmenum}

  We will call the system \term{discrete} if \( T \) is the monoid of zero-based \hyperref[def:natural_numbers]{natural numbers} and \term{continuous} if \( T \) is the monoid of nonnegative \hyperref[def:real_numbers]{real numbers}, with or without \hyperref[def:extended_real_numbers]{an infinite element}.

  The monoid \( T \) can theoretically be a \hyperref[def:group]{group}, in which case we consider \hyperref[def:group_action]{group actions}, however negative time is not as often needed in practice.
\end{definition}

\begin{definition}\label{def:dynamical_system_trajectory}
  Fix a \hyperref[def:dynamical_system]{dynamical system} with evolution function \( \Phi: T \times X \to X \).

  A \term{trajectory} in a starting at the \term{initial state} \( x_0 \in X \) is an \hyperref[def:cartesian_product/indexed_family]{indexed family} \( \seq{ x_t }_{t \in T} \) obtained as
  \begin{equation*}
    x_t \coloneqq \Phi_t(x_0).
  \end{equation*}

  The condition \ref{eq:def:monoid_action/family/identity} ensures that \( \Phi_0(x_0) = x_0 \), and \ref{eq:def:monoid_action/family/compatibility} ensures that
  \begin{equation*}
    x_{t + s}
    =
    \Phi_{t + s}(x_0)
    =
    \Phi_t(x_s).
  \end{equation*}

  For discrete dynamical systems, trajectories are sequences.
\end{definition}
