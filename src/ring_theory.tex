\section{Ring theory}\label{sec:ring_theory}

As discussed in \fullref{rem:additive_magma}, commutative and non-commutative groups are quite different despite having similar definitions. Rings are extensions of \hyperref[def:abelian_group]{abelian groups}, which allow multiplication with more than members of \( \BbbZ \).

For commutative rings, this second operation is often truly an extension of \fullref{def:magma/exponentiation} to arbitrary ring elements. For noncommutative ring, this second operation is usually given by function composition.

This section also describes \hyperref[def:module]{modules}, which are important both as generalizations of \hyperref[def:vector_space]{vector spaces} and as a tool to study rings. \hyperref[def:ring_ideal]{Ring ideals} are instances of submodules, for example.

In an attempt to encompass the \hyperref[def:natural_number]{natural numbers}, \hyperref[def:lattice_ideal]{lattice ideals} and \hyperref[def:polynomial_semiring]{polynomials} and \hyperref[def:array/matrix]{matrices} over \hyperref[def:tropical_semiring]{tropical semirings}, we have chosen to use \hyperref[def:semiring]{semirings}, \hyperref[def:semimodule]{semimodules} and \hyperref[def:semiring_ideal]{semiring ideals} as fundamental notions.

\begin{figure}[h]
  \caption{Some important (semi)rings}\label{fig:ring_hierarchy}
  \smallskip
  \hfill
  \begin{forest}
    [
      {\hyperref[def:semiring]{semiring}}
        [{\hyperref[def:noetherian_semiring]{noetherian}}, name=noetherian]
        [
          {\hyperref[def:zerosumfree]{zerosumfree}}
            [{\hyperref[def:semilattice/distributive_lattice]{distributive lattice}}, name=lattice]
        ]
        [
          {\hyperref[def:semiring/commutative]{commutative}}, name=commutative
            [
              {\hyperref[def:integral_domain]{integral domain}}, name=domain
                [
                  {\hyperref[def:greatest_common_divisor_domain]{greatest common divisor domain}}
                    [
                      {\hyperref[def:unique_factorization_domain]{unique factorization domain}}, name=ufd
                        [
                          {\hyperref[def:principal_ideal_domain]{principal ideal domain}}, name=pid
                            [
                              {\hyperref[def:euclidean_domain]{Euclidean domain}}
                              [{\hyperref[def:field]{field}}, name=field]
                            ]
                        ]
                    ]
                ]
            ]
        ]
        [{\hyperref[def:entire_semiring]{entire}}, name=entire]
        [
          {\hyperref[def:ring]{ring}}, name=ring
          [{\hyperref[def:division_ring]{division ring}}, name=division]
          [{\hyperref[def:ring/simple]{simple ring}}, name=simple]
        ]
    ]
    \draw[-] (ring) to (domain);
    \draw[-] (entire) to (domain);
    \draw[-] (commutative) to (lattice);
    \draw[-] (division) to[out=south, in=east] (field);
    \draw[-] (simple) to[out=south, in=east] (field);
    \draw[-] (noetherian) to[out=south, in=west] (pid);
  \end{forest}
  \hfill\hfill
\end{figure}
