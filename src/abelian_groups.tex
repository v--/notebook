\subsection{Abelian groups}\label{subsec:abelian_groups}

\begin{definition}\label{def:abelian_group}
  A \hyperref[def:magma/commutative]{commutative} \hyperref[def:group]{group} is usually called an \term{abelian group}. We denote by \( \cat{Ab} \) the category of abelian groups.
\end{definition}

\begin{remark}\label{rem:additive_magma}
  General groups often arise as \hyperref[def:automorphism_group]{automorphism groups}, which are, for the most part, non-commutative, while abelian groups are usually used as the main building block for \hyperref[def:ring]{rings} and \hyperref[def:module]{modules}.

  To make a further distinction, if the operation is denoted by \( \cdot \) or juxtaposition, we say that the group is a \term{multiplicative group}, and if the operation is denoted by \( + \), we say that the group is an \term{additive group}. This terminology usually, but not necessarily, coincides with the group (or, more generally, the \hyperref[def:magma]{magma}) being \hyperref[def:magma/commutative]{commutative}.

  To make things explicit, a \term{multiplicative magma} is any magma as defined in \fullref{def:magma}. Compare this to \term{additive magmas}, where
  \begin{thmenum}
    \thmitem{rem:additive_magma/addition} The magma operation is denoted by \( + \) and called \term{addition}.

    \thmitem{rem:additive_magma/multiplication} The magma \hyperref[def:magma/exponentiation]{exponentiation operation} \( x^n \) is denoted by \( n \cdot x \) or juxtaposition and called \term{multiplication}. Thus, multiplication is not defined for two elements of the magma, but defined for a positive integer and an element of the magma. That is,
    \begin{equation}\label{eq:rem:additive_magma/multiplication}
      \begin{aligned}
        &\cdot: \cdot: \BbbN \times R \to R \\
        &n \cdot x \coloneqq \begin{cases}
          0_M,           &n = 0, \T{initial condition if} M \T{is a monoid} \\
          x,             &n = 1, \T{initial condition if} M \T{is not a monoid} \\
          n \cdot x + x, &n > 1 \\
          -(n \cdot x),  &n < 0, \\
        \end{cases}
      \end{aligned}
    \end{equation}

    In the case of a \hyperref[def:magma/commutative]{commutative} \hyperref[def:monoid]{monoid}, if multiplication is extended to two elements of the monoid, we instead talk about \hyperref[def:semiring]{semirings}.

    \thmitem{rem:additive_magma/identity} The \hyperref[def:monoid]{identity} is usually denoted by \( 0 \).

    \thmitem{rem:additive_magma/inverse} If an \hyperref[def:monoid_inverse_element]{inverse} of \( x \) exists, it is denoted by \( -x \) rather than \( x^{-1} \).
  \end{thmenum}
\end{remark}

\begin{proposition}\label{thm:abelian_outer_automorphism_group}
  In an \hyperref[def:abelian_group]{abelian group}, the full \hyperref[def:automorphism_group]{automorphism group} \( \aut(G) \) is isomorphic to the \hyperref[def:inner_and_outer_automorphisms]{outer automorphism group} \( \op{out}(G) \).
\end{proposition}
\begin{proof}
  If the group operation is \hyperref[def:magma/commutative]{commutative}, then \( xyx^{-1} = yxx^{-1} = y \), which makes the \hyperref[def:inner_and_outer_automorphisms]{conjugation action} trivial. Thus, the \hyperref[def:inner_and_outer_automorphisms]{inner automorphism group} \( \op{int}(G) \) is trivial, and hence \( \aut(G) \cong \op{out}(G) \).
\end{proof}

\begin{proposition}\label{thm:abelian_normal_subgroups}
  All subgroups of an abelian group are \hyperref[def:normal_subgroup]{normal}.
\end{proposition}
\begin{proof}
  Let \( G \) be abelian and \( H \) be a subgroup of \( G \). Then \( x H x^{-1} = xx^{-1} H = H \) for any \( x \in G \) and thus \( H \) is normal.
\end{proof}

\begin{proposition}\label{thm:group_of_integers_modulo}
  The \hyperref[def:set_of_integers]{integers} \( \BbbZ \) form an abelian group under addition. For every positive integer \( n \), we define the group
  \begin{equation*}
    \BbbZ_n \coloneqq \set{ 0, 1, \ldots, n - 1 }
  \end{equation*}
  with the operation
  \begin{equation*}
    x \oplus y \coloneqq \rem(x + y, n)
  \end{equation*}
  so that
  \begin{equation*}
    x \oplus y \cong x + y \pmod n.
  \end{equation*}

  The group \( \BbbZ_n \) is called the \term{group of integers modulo} \( n \). Compare this result with \fullref{thm:ring_of_integers_modulo}.
\end{proposition}
\begin{proof}
  We will prove that \( \BbbZ_n \) is an abelian group.

  \SubProofOf[def:magma/associative]{associativity} Addition in \( \BbbZ_n \) is associative since
  \begin{balign*}
    (x \oplus y) \oplus z
    &=
    \rem((x \oplus y) + z, n)
    = \\ &=
    \rem(\rem(x + y, n) + z, n)
    = \\ &=
    \rem(x + y - n \quot(x + y, n) + z, n)
    = \\ &=
    \rem(x + y + z, n)
    = \\ &=
    \ldots
    = \\ &=
    x \oplus (y \oplus z).
  \end{balign*}

  \SubProofOf[def:monoid]{identity} The zero is the identity.

  \SubProofOf[def:monoid_inverse_element]{inverse} Fix \( x \in \BbbZ_n \). If \( x = 0 \), its inverse is \( 0 \). If \( x > 0 \), its inverse is \( n - x \) since \( n - x \in \BbbZ_n \) and
  \begin{equation*}
    x \oplus (n - x) = x + (n - x) - n = 0.
  \end{equation*}

  \SubProofOf[def:magma/commutative]{commutativity} Follows from
  \begin{equation*}
    x \oplus y
    =
    \rem(x + y, n)
    =
    \rem(y + x, n)
    =
    y \oplus x.
  \end{equation*}
\end{proof}

\begin{proposition}\label{thm:integers_modulo_isomorphic_to_quotient_group}
  The group \( \BbbZ_n \) of \hyperref[thm:group_of_integers_modulo]{integers modulo \( n \)} is isomorphic to the quotient of \( \BbbZ \) by \( n\BbbZ = \set{ nz \given z \in \BbbZ } \). That is,
  \begin{equation*}
    \BbbZ_n \cong \BbbZ / n\BbbZ.
  \end{equation*}
\end{proposition}
\begin{proof}
  Define the function
  \begin{align*}
    &\varphi: \BbbZ_n \to \BbbZ / n\BbbZ  \\
    &\varphi(x) \coloneqq x + n\BbbZ.
  \end{align*}

  It is a homomorphism because
  \begin{balign*}
    \varphi(x \oplus y)
    &=
    \varphi(\rem(x + y, n))
    = \\ &=
    \varphi(x + y - n \quot(x + y, n))
    = \\ &=
    x + y - n \quot(x + y, n) + n\BbbZ
    = \\ &=
    x + y + n\BbbZ
    = \\ &=
    (x + n\BbbZ) + (y + n\BbbZ)
    = \\ &=
    \varphi(x) + \varphi(y).
  \end{balign*}

  Furthermore, this shows that \( \varphi \) is also an isomorphism.
\end{proof}

\begin{example}\label{ex:lagranges_theorem_for_groups/direct_product_zn}
  \Fullref{thm:lagranges_theorem_for_groups} and \fullref{thm:integers_modulo_isomorphic_to_quotient_group} imply that, for any positive integer \( n \), \( (nm, k) \mapsto nm + k \) is a bijection between \( n \BbbZ \times \BbbZ_n \) and \( \BbbZ \). This bijection, however, is not necessarily a group isomorphism because \eqref{eq:def:magma/homomorphism} may not hold.

  Consider the tuples \( (nm_1, k_1) \) and \( (nm_2, k_2) \)  in \( n \BbbZ \times \BbbZ_n \). We have
  \begin{equation*}
    (nm_1, k_1) + (nm_2, k_2) = (nm_1 + nm_2, \rem(k_1 + k_2, n)).
  \end{equation*}

  Therefore, if \( k_1 + k_2 \geq n \),
  \begin{equation*}
    nm_1 + nm_2 + \rem(k_1 + k_2, n) < (nm_1 + k_1) + (nm_2 + k_2).
  \end{equation*}
\end{example}

\begin{proposition}\label{thm:cyclic_group_isomorphic_to_integers_modulo_n}
  The \hyperref[def:cyclic_group]{cyclic group} \( C_n \) is isomorphic to the group \hyperref[thm:group_of_integers_modulo]{\( \BbbZ_n \)} of integers modulo \( n \).
\end{proposition}
\begin{proof}
  The homomorphism
  \begin{equation*}
    \begin{aligned}
      &\varphi: \BbbZ_n \to C_n \\
      &\varphi(k) \coloneqq a^k,
    \end{aligned}
  \end{equation*}
  and the analogous homomorphism for the infinite group, are isomorphisms.
\end{proof}

\begin{definition}\label{def:group_direct_sum}
  The \term{direct sum} \( \bigoplus_{k \in \mscrK} G_k \) of a family \( \seq{ G_k }_{k \in \mscrK} \) of (not necessarily abelian) groups is the subgroup of the \hyperref[def:group_direct_product]{direct product} \( \prod_{k \in \mscrK} G_k \), in which only finitely many components of each tuple are distinct from the identity.
\end{definition}

\begin{proposition}\label{thm:abelian_group_categorical_limits}
  \hfill
  \begin{thmenum}
    \thmitem{thm:abelian_group_categorical_limits/product} As for the category \hyperref[def:group/category]{\( \cat{Grp} \)} of small group, the \hyperref[def:discrete_category_limits]{categorical product} of the family \( \seq{ G_k }_{k \in \mscrK} \) in the category \( \cat{Ab} \) of small \hyperref[def:abelian_group]{abelian groups} is their \hyperref[def:group_direct_product]{direct product} \( \prod_{k \in \mscrK} G_k \).

    This is discussed as proved in \fullref{thm:group_categorical_limits/product}; and the proof holds in \( \cat{Ab} \).

    \thmitem{thm:abelian_group_categorical_limits/coproduct} The \hyperref[def:discrete_category_limits]{categorical coproduct} of the family \( \seq{ G_k }_{k \in \mscrK} \) of \hyperref[def:abelian_group]{abelian groups} is their \hyperref[def:group_direct_sum]{direct sum} \( \biguplus_{k \in \mscrK} X_k \), the inclusion morphisms being
    \begin{equation*}
      \begin{aligned}
         &\iota_m: X_m \to \oplus_{k \in \mscrK} X_k \\
         &\iota_m(x_m) \coloneqq \begin{cases}
          \begin{rcases}
            x_m, &k = m \\
            0_k, &k \neq m
          \end{rcases}_{k \in \mscrK}
        \end{cases}
      \end{aligned}
    \end{equation*}

    Compare this to \fullref{thm:group_categorical_limits/coproduct}.
  \end{thmenum}
\end{proposition}
\begin{proof}
  \SubProofOf{thm:abelian_group_categorical_limits/coproduct} Let \( (A, \alpha) \) be a \hyperref[def:category_of_cones/cocone]{cocone} for the discrete diagram \( \seq{ G_k }_{k \in \mscrK} \). We want to define a group homomorphism \( l: \bigoplus_{k \in \mscrK} G_k \to A \) such that, for every \( m \in \mscrK \),
  \begin{equation*}
    \alpha_m(x) = l_A(\iota_m(x)).
  \end{equation*}

  This suggests the definition
  \begin{equation*}
    l_A(\seq{ x_k }_{k \in \mscrK}) \coloneqq \sum\set{ \alpha_k(x_k) \given k \in \mscrK \T{and} x_k \neq e_k }.
  \end{equation*}

  It is now evident why both commutativity and the assumption about only finitely many non-identity components is crucial.
\end{proof}

\begin{definition}\label{def:monoid_grothendieck_completion}\mcite{nLab:grothendieck_group_of_a_commutative_monoid}
  Let \( M \) be a \hyperref[def:magma/commutative]{commutative} \hyperref[def:monoid]{monoid}. Define the \hyperref[def:equivalence_relation]{equivalence relation} \( \sim \) on tuples of members of \( M \) to hold for \( (x_1, x_2) \sim (x_1', x_2') \) if there exists an element \( a \) of \( M \) such that
  \begin{equation*}
    x_1 + x_2' + a = x_1' + x_2 + a.
  \end{equation*}

  Define addition on the \hyperref[thm:equivalence_partition]{equivalence partition} \( G \coloneqq (M \times M) / \sim \) componentwise as
  \begin{equation*}
    [(x_1, x_2)] \oplus [(y_1, y_2)] \coloneqq [(x_1 + y_1, x_2 + y_2)]
  \end{equation*}
  and fix a canonical embedding
  \begin{equation*}
    \begin{aligned}
      &\iota_M: M \to G \\
      &\iota_M(m) \coloneqq [(m, 0)].
    \end{aligned}
  \end{equation*}

  We call the obtained \hyperref[def:abelian_group]{abelian group} \( (G, \oplus) \) the \term{Grothendieck completion} of \( M \).
\end{definition}
\begin{defproof}
  \SubProof{Proof that \( \sim \) is an equivalence relation}
  \SubProofOf*[def:binary_relation/reflexive]{reflexivity}
  \begin{equation*}
    (x_1, x_2) \cong (x_1', x_2') \iff x_1 + x_2' + 0 = x_1' + x_2 + 0
  \end{equation*}

  \SubProofOf*[def:binary_relation/symmetric]{symmetry} By commutativity,
  \begin{balign*}
    (x_1, x_2) \cong (x_1', x_2')
    &\iff
    \qexists a x_1 + x_2' + a = x_1' + x_2 + a
    \\ &\iff
    \qexists a x_1' + x_2 + a = x_1 + x_2' + a
    \\ &\iff
    (x_1', x_2') \cong (x_1, x_2)
  \end{balign*}

  \SubProofOf*[def:binary_relation/transitive]{transitivity} Suppose that \( (x_1, x_2) \cong (x_1', x_2') \) and \( (x_1', x_2') \cong (x_1^\dprime, x_2^\dprime) \). Thus, there exist elements \( a \) and \( b \) of \( M \) such that
  \begin{equation*}
    x_1 + x_2' + a = x_1' + x_2 + a \T{and} x_1' + x_2^\dprime + b = x_1^\dprime + x_2' + b
  \end{equation*}

  Summing both sides, we obtain
  \begin{equation*}
    x_1 + x_1' + x_2' + x_2^\dprime + a + b = x_1' + x_1^\dprime + x_2 + x_2' + a + b
  \end{equation*}

  We reorder both sides to obtain
  \begin{equation*}
    (x_1 + x_2^\dprime) + (x_1' + x_2' + a + b) = (x_1^\dprime + x_2) + (x_1' + x_2' + a + b),
  \end{equation*}
  which implies \( (x_1, x_1^\dprime) \cong (x_2, x_2^\dprime) \).

  \SubProof{Proof that \( (G, \oplus) \) is an abelian group}

  \SubProof*{Proof that \( \oplus \) is well-defined} The addition operation on \( G \) does not depend on the representative of the equivalence class. Indeed, let \( (x_1, x_2) \sim (x_1', x_2') \) and \( (y_1, y_2) \sim (y_1', y_2') \). Then there exist \( a \) and \( b \) such that
  \begin{align*}
    x_1 + x_2' + a &= x_1' + x_2 + a, \\
    y_1 + y_2' + b &= y_1' + y_2 + b.
  \end{align*}

  When added combined, these give
  \begin{equation*}
    (x_1 + y_1) + (x_2' + y_2') + (a + b)
    =
    (x_1' + y_1') + (x_2 + y_2) + (a + b),
  \end{equation*}
  which implies that
  \begin{equation*}
    (x_1 + y_1, x_2 + y_2) \sim (x_1' + y_1', x_2' + y_2').
  \end{equation*}

  \SubProofOf*[def:magma/associative]{associativity} Associativity of multiplication in \( G \) is inherited from multiplication in \( M \).

  \SubProofOf*[def:monoid]{identity} The equivalence class \( [(0, 0)] \) is an identity in \( G \) and contains the pairs \( (x, x) \) of identical elements.

  \SubProofOf*[def:monoid_inverse_element]{inverse} For each member \( (x_1, x_2) \in M \times M \), its inverse is \( (x_2, x_1) \) because
  \begin{equation*}
    [(x_1, x_2)] \oplus [(x_2, x_1)] = [(x_1 + x_2, x_2 + y_1)],
  \end{equation*}
  which, by commutativity, belongs to \( [(0, 0)] \).

  \SubProofOf*[def:magma/commutative]{commutativity} Commutativity of the group operation \( \oplus \) is also inherited from the monoid operation \( + \).
\end{defproof}

\begin{proposition}\label{thm:monoid_grothendieck_completion_universal_property}\mcite{nLab:grothendieck_group_of_a_commutative_monoid}
  The \hyperref[def:monoid_grothendieck_completion]{Grothendieck completion} \( \overline{M} \) of a commutative monoid \( M \) satisfies the following \hyperref[rem:universal_mapping_property]{universal mapping property}:
  \begin{displayquote}
    For every abelian group \( G \) and every monoid homomorphism \( \varphi: M \to G \), there exists a unique group homomorphism \( \widetilde{\varphi}: \overline{M} \to G \) such that the following diagram commutes:
    \begin{equation}\label{eq:thm:monoid_grothendieck_completion_universal_property/diagram}
      \begin{aligned}
        \includegraphics[page=1]{output/thm__monoid_grothendieck_completion_universal_propety.pdf}
      \end{aligned}
    \end{equation}
  \end{displayquote}

  Via \fullref{rem:universal_mapping_property}, \( \overline{\anon} \) becomes \hyperref[def:category_adjunction]{left adjoint} to the \hyperref[def:concrete_category]{forgetful functor} \( U: \cat{Ab} \to \cat{CMon} \).

  Compare this result to \fullref{thm:semiring_grothendieck_completion_universal_property}.
\end{proposition}
\begin{proof}
  Let \( \varphi: M \to G \) be a monoid homomorphism into an abelian group \( G \). We want to define a homomorphism \( \overline{\varphi} \) such that
  \begin{equation*}
    \overline{\varphi}(\iota_M(x)) = \overline{\varphi}([(x, 0)]) = \varphi(x).
  \end{equation*}

  Each equivalence class \( C \) in \( G \) has a unique member \( x \) such that \( (x, 0) \in C \), hence the above condition is well-posed.

  Fix pairs \( (x_1, x_2) \) and \( (x_1', x_2') \) from \( M \times M \). Suppose that \( (x_1, x_2) \sim (x_1', x_2') \). Then there exists \( a \in M \) such that
  \begin{equation*}
    x_1 + x_2' + a = x_1' + x_2 + a.
  \end{equation*}

  An additional restriction on \( \overline{\varphi} \) is then
  \begin{equation*}
    \overline{\varphi}\parens[\Big]{ [(x_1, x_1')] }
    =
    \overline{\varphi}\parens[\Big]{ [(x_2, x_2')] }.
  \end{equation*}

  This uniquely determines \( \overline{\varphi} \) as
  \begin{equation*}
    \overline{\varphi}([(x, y)]) \coloneqq \varphi(x) - \varphi(y).
  \end{equation*}
\end{proof}

\begin{definition}\label{def:group_commutator}\mcite[313]{Knapp2016BasicAlgebra}
  Let \( G \) be an arbitrary group. We define the \term{commutator} of the elements \( x \) and \( y \) as
  \begin{equation*}
    [x, y] \coloneqq xyx^{-1}y^{-1}.
  \end{equation*}

  The \term{commutator subgroup} \( [G, G] \) of \( G \) is the subgroup \hyperref[def:group/submodel]{generated} by all the commutators in \( G \).
\end{definition}

\begin{proposition}\label{thm:quotient_by_commutator_subgroup}\mcite[prop. 7.4]{Knapp2016BasicAlgebra}
  The commutator group \( [G, G] \) of any group \( G \) is \hyperref[def:normal_subgroup]{normal} and the quotient \( G / C \) is \hyperref[def:abelian_group]{abelian}.
\end{proposition}
\begin{proof}
  Normality easily follows from
  \begin{equation*}
    a xyx^{-1}y^{-1} a^{-1}
    =
    (a x a^{-1}) (a y a^{-1}) (a x a^{-1})^{-1} (a y a^{-1})^{-1}.
  \end{equation*}

  Then for the cosets \( a C \) and \( b C \) we have
  \begin{equation*}
    a C \cdot b C
    =
    a b C
    =
    a b (b^{-1} a^{-1} b a) C
    =
    b a C.
  \end{equation*}

  Therefore, the quotient group \( G / C \) is abelian.
\end{proof}

\begin{definition}\label{def:free_abelian_group}\mimprovised
  We associate with every set \( A \) its \term{free abelian group} \( F(A) \) defined as the set
  \begin{equation*}
    F(A) \coloneqq \set{ s: A \to \BbbZ \given s \T{has finitely many nonzero values} }
  \end{equation*}
  with addition inherited from \( \BbbZ \).

  We may equivalently reformulate the definition using \hyperref[def:weighted_set/multiset]{multisets} of integers, however this will not benefit us.

  The canonical inclusion function is
  \begin{equation*}
    \begin{aligned}
      &\iota_A: A \to F(A) \\
      &\iota_A(x) \coloneqq \parens[\Bigg]
        {
          y \mapsto \begin{rcases}
            \begin{cases}
              1, &y = x \\
              0, &y \neq x
            \end{cases}
          \end{rcases}
        }
    \end{aligned}
  \end{equation*}

  Compare this definition to free groups defined in \fullref{def:free_group}.
\end{definition}
\begin{defproof}
  This is an abelian group as a consequence of \fullref{thm:functions_over_model_form_model}.
\end{defproof}

\begin{proposition}\label{thm:free_abelian_group_universal_property}
  The \hyperref[def:free_abelian_group]{free abelian group} \( F(A) \) is the unique up to an isomorphism abelian group that satisfies the following \hyperref[rem:universal_mapping_property]{universal mapping property}:
  \begin{displayquote}
    For every abelian group \( G \) and every function \( f: A \to G \), there exists a unique group homomorphism \( \widetilde{f}: F(A) \to G \) such that the following diagram commutes:
    \begin{equation}\label{eq:thm:free_abelian_group_universal_property/diagram}
      \begin{aligned}
        \includegraphics[page=1]{output/thm__free_abelian_group_universal_property.pdf}
      \end{aligned}
    \end{equation}
  \end{displayquote}

  Via \fullref{rem:universal_mapping_property}, \( F \) becomes \hyperref[def:category_adjunction]{left adjoint} to the \hyperref[def:concrete_category]{forgetful functor} \( U: \cat{Set} \to \cat{Ab} \).
\end{proposition}
\begin{proof}
  For every function \( f: A \to G \), we want
  \begin{equation*}
    \widehat{f}(\iota_A(x)) = f(x).
  \end{equation*}

  This implies the definition
  \begin{equation*}
    \widehat{f}(s) \coloneqq \sum_{x \in A} s(x) \cdot f(x).
  \end{equation*}

  Note that the expression \( s(x) \cdot f(x) \) is \hyperref[rem:additive_magma/multiplication]{additive notation} for group exponentiation.

  As in \fullref{thm:abelian_group_categorical_limits/coproduct}, it becomes evident why we need the assumptions of commutativity and finitely many nonzero values of \( s \).
\end{proof}
