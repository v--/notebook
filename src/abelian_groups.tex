\subsection{Abelian groups}\label{subsec:abelian_groups}

\begin{definition}\label{def:abelian_group}
  A \hyperref[def:magma/commutative]{commutative} \hyperref[def:group]{group} is usually called an \term{abelian group}. It is conventional to use \hyperref[rem:additive_magma]{additive notation} for abelian groups.

  We denote by \( \cat{Ab} \) the category of abelian groups. It is the quintessential example of \hyperref[def:abelian_category]{abelian categories} (see \fullref{thm:ab_is_abelian}).
\end{definition}

\begin{proposition}\label{thm:abelian_normal_subgroups}
  All subgroups of an abelian group are \hyperref[def:normal_subgroup]{normal}.
\end{proposition}
\begin{proof}
  Let \( \mscrG \) be abelian and \( \mscrH \) be a subgroup of \( \mscrG \). Then \( x \mscrG x^{-1} = xx^{-1} \mscrH = \mscrH \) for any \( x \in G \) and thus \( \mscrH \) is normal.
\end{proof}

\begin{definition}\label{def:group_of_integers_modulo}
  The \hyperref[def:integers]{integers} \( \BbbZ \) notoriously form an abelian group under addition. Fix a positive integer \( n \). We define the group
  \begin{equation*}
    \BbbZ_n \coloneqq \{ 0, 1, \ldots, n - 1 \}
  \end{equation*}
  with the operation
  \begin{equation*}
    x \oplus y \coloneqq \rem(x + y, n)
  \end{equation*}
  so that
  \begin{equation*}
    x \oplus y \cong x + y \pmod n.
  \end{equation*}

  The group \( \BbbZ_n \) is called the \term{group of integers modulo} \( n \).
\end{definition}
\begin{proof}
  We will prove that \( \BbbZ_n \) is an abelian group.

  \SubProofOf{def:magma/associative} Addition in \( \BbbZ_n \) is associative since
  \begin{balign*}
    (x \oplus y) \oplus z
    &=
    \rem((x \oplus y) + z, n)
    = \\ &=
    \rem(\rem(x + y, n) + z, n)
    = \\ &=
    \rem(x + y - n \quot(x + y, n) + z, n)
    = \\ &=
    \rem(x + y + z, n)
    = \\ &=
    \ldots
    = \\ &=
    x \oplus (y \oplus z).
  \end{balign*}

  \SubProofOf{def:unital_magma} The zero is obviously the identity.

  \SubProofOf{def:unital_magma_inverse_element} Fix \( x \in \BbbZ_n \). If \( x = 0 \), its inverse is \( 0 \). If \( x > 0 \), its inverse is \( n - x \) since \( n - x \in \BbbZ_n \) and
  \begin{equation*}
    x \oplus (n - x) = x + (n - x) - n = 0.
  \end{equation*}

  \SubProofOf{def:magma/commutative} Commutativity follows from
  \begin{equation*}
    x \oplus y
    =
    \rem(x + y, n)
    =
    \rem(y + x, n)
    =
    y \oplus x.
  \end{equation*}
\end{proof}

\begin{proposition}\label{thm:integers_modulo_isomorphic_to_quotient_group}
  The group \( \BbbZ_n \) of \hyperref[def:group_of_integers_modulo]{integers modulo \( n \)} is isomorphic to the quotient of \( \BbbZ \) by \( n\BbbZ = \{ nz : z \in \BbbZ \} \), i.e.
  \begin{equation*}
    \BbbZ_n \cong \BbbZ / n\BbbZ.
  \end{equation*}
\end{proposition}
\begin{proof}
  Define the function
  \begin{align*}
    &\varphi: \BbbZ_n \to \BbbZ / n\BbbZ  \\
    &\varphi(x) \coloneqq x + n\BbbZ.
  \end{align*}

  It is a homomorphism because
  \begin{balign*}
    \varphi(x \oplus y)
    &=
    \varphi(\rem(x + y, n))
    = \\ &=
    \varphi(x + y - n \quot(x + y, n))
    = \\ &=
    x + y - n \quot(x + y, n) + n\BbbZ
    = \\ &=
    x + y + n\BbbZ
    = \\ &=
    (x + n\BbbZ) + (y + n\BbbZ)
    = \\ &=
    \varphi(x) + \varphi(y).
  \end{balign*}

  Furthermore, this shows that \( \varphi \) is also an isomorphism.
\end{proof}

\begin{example}\label{ex:lagranges_theorem_for_groups/direct_product_zn}
  \Fullref{thm:lagranges_theorem_for_groups} and \fullref{thm:integers_modulo_isomorphic_to_quotient_group} imply that, for any positive integer \( n \), there exists a bijection between \( n \BbbZ \times \BbbZ_n \) and \( \BbbZ \).

  This bijection, however, is not necessarily a group isomorphism because \eqref{thm:integers_modulo_isomorphic_to_quotient_group} may not hold.

  Let \( f: \BbbZ \to n \BbbZ \times \BbbZ_n \) be a bijection. Then, for every integer \( k \) there exist integers \( m_k \in \BbbZ \) and \( r_k \in \BbbZ_n \) such that
  \begin{equation*}
    f(k) = (n m_k, r_k).
  \end{equation*}

  For \( (mn, p) \in  \)
\end{example}

\begin{proposition}\label{thm:cyclic_group_isomorphic_to_integers_modulo_n}
  Let \( C \) be a cyclic \hyperref[def:cyclic_group]{group}. If \( C \) is finite of order \( n \), it is isomorphic to the group \( \BbbZ_n \) of integers modulo \( n \) (see \fullref{def:group_of_integers_modulo}).
\end{proposition}
\begin{proof}
  The homomorphism
  \begin{balign*}
     &\varphi: \BbbZ_n \to C_n \\
     &\varphi(k) \coloneqq a^k
  \end{balign*}
  and the analogous homomorphism for the infinite group, is an isomorphism.
\end{proof}

\begin{proposition}\label{thm:abelian_group_categorical_limits}
  We are interested in \hyperref[def:categorical_limit]{categorical limits} and \hyperref[def:categorical_colimit]{colimits} in the category \( \cat{Ab} \). Fix an indexed family  \( \{ \mscrX_k \}_{k \in \mscrK} \) of abelian groups.
  \begin{thmenum}
    \thmitem{thm:abelian_group_categorical_limits/product} Their \hyperref[def:categorical_product]{categorical product} is the direct product as inherited from \fullref{thm:group_categorical_limits}.

    \thmitem{thm:abelian_group_categorical_limits/coproduct} Their \hyperref[def:categorical_coproduct]{categorical coproduct} is the \hyperref[def:group_direct_product]{direct sum} \( \oplus_{k \in \mscrK} \mscrX_k \), the embedding morphisms being
    \begin{align*}
       &\iota_m: \mscrX_m \to \oplus_{k \in \mscrK} \mscrX_k \\
       &\iota_m(x_m) \coloneqq \begin{dcases}
        \begin{drcases}
          x_m, &k = m \\
          0_k, &k \neq m
        \end{drcases}
      \end{dcases}_{k \in \mscrK}.
    \end{align*}
  \end{thmenum}
\end{proposition}

\begin{remark}\label{rem:abelian_group_biproducts}
  Since \( \cat{Ab} \) is a subcategory of \( \cat{Grp} \), by \fullref{thm:group_categorical_limits} we have that for abelian groups the notions of \hyperref[def:group_free_product]{free product} and direct sum coincide.

  By \fullref{thm:preadditive_category_biproducts}, finite direct products and finite direct sums of abelian groups coincide as biproducts. This is also obvious by definition, even for nonabelian groups. What is not obvious, however, is that finite free products and finite direct products coincide for abelian groups.
\end{remark}

\begin{proposition}\label{thm:ab_is_monoidal}
  The category \( \cat{Ab} \) has two monoidal structures. It is monoidal with
  \begin{itemize}
    \item the \hyperref[def:group_direct_product]{direct sum} acting as a monoidal product
    \item the trivial group \( \{ e \} \) acting as an identity object
    \item natural transformations
          \begin{balign*}
            \sigma                    & \coloneqq \id \\
            \lambda(\{ e \} \times A) & \coloneqq A   \\
            \rho(A \times \{ e \})    & \coloneqq A
          \end{balign*}
  \end{itemize}
  and with
  \begin{itemize}
    \item the \hyperref[def:left_module_tensor_product]{tensor product} (see \fullref{thm:abelian_group_iff_z_module}) acting as a monoidal product
    \item the integers \( (\BbbZ, +) \) acting as an identity object
    \item natural transformations
          \begin{balign*}
            \sigma  & \coloneqq \id \\
            \lambda & \coloneqq \id \\
            \rho    & \coloneqq \id
          \end{balign*}
  \end{itemize}
\end{proposition}
\begin{proof}
  All conditions in \fullref{def:monoidal_category} are trivially satisfied for the direct sum structure.

  The other conditions are also satisfied due to \fullref{thm:tensor_product_with_underlying_ring}.
\end{proof}

\begin{proposition}\label{thm:ab_is_abelian}
  The category \( \cat{Ab} \) of \hyperref[def:abelian_group]{abelian group} is \hyperref[def:abelian_category]{abelian} (enriched with the direct sum monoidal \hyperref[thm:ab_is_monoidal]{structure}).
\end{proposition}
\begin{proof}
  The category \( \cat{Ab} \) is \hyperref[def:enriched_category]{enriched} over itself in an obvious way.

  Composition is bilinear because it is the usual function composition. Hence \( \cat{Ab} \) is \hyperref[def:preadditive_category]{preadditive}.

  Finite products and coproducts exist by \fullref{thm:group_categorical_limits}, thus \( \cat{Ab} \) is \hyperref[def:additive_category]{additive}.

  Every homomorphism \( f: G \to H \) has a kernel \( \ker(f) \) (the usual kernel in the sense of \fullref{def:unital_magma_kernel}) and a cokernel
  \begin{equation*}
    \co\ker(f) \coloneqq H / f(G).
  \end{equation*}

  Furthermore, because of the equivalences in \fullref{def:normal_subgroup}, all \hyperref[def:first_order_homomorphism_invertibility/embedding]{embeddings} have trivial kernels and all \hyperref[def:first_order_homomorphism_invertibility/projection]{projections} have trivial cokernels.

  Thus \( \cat{Ab} \) is abelian.
\end{proof}

\begin{proposition}\label{thm:monoid_completion_to_abelian_group}\mcite{nLab:grothendieck_group_of_a_commutative_monoid}
  Every \hyperref[def:magma/commutative]{commutative} \hyperref[def:unital_magma/associative]{monoid} can be \hyperref[def:first_order_homomorphism_invertibility/embedding]{embedded} into an abelian group using the \term{Grothendieck completion} presented in the proof.
\end{proposition}
\begin{proof}
  Let \( M \) be a commutative monoid. Define the relation \( \cong \) on tuples of members of \( M \) as
  \begin{equation*}
    (x_1, x_2) \cong (y_1, y_2) \iff \exists a: x_1 + y_2 + a = y_1 + x_2 + a.
  \end{equation*}

  This is an equivalence relation because
  \SubProofOf{def:binary_relation/reflexive}
  \begin{equation*}
    (x_1, x_2) \cong (x_1, x_2) \iff x_1 + x_2 + 0 = x_1 + x_2 + 0
  \end{equation*}

  \SubProofOf{def:binary_relation/symmetric} By commutativity,
  \begin{balign*}
    (x_1, x_2) \cong (y_1, y_2)
     & \iff
    \exists a: x_1 + y_2 + a = y_1 + x_2 + a
    \\ &\iff
    \exists a: y_1 + x_2 + a = x_1 + y_2 + a
    \\ &\iff
    (y_1, y_2) \cong (x_1, x_2)
  \end{balign*}

  \SubProofOf{def:binary_relation/transitive} Let \( (x_1, x_2) \cong (y_1, y_2) \) and \( (y_1, y_2) \cong (z_1, z_2) \). Thus there exist \( a, b \in \BbbN \) such that
  \begin{equation*}
    [x_1 + y_2 + a = y_1 + x_2 + a] \T{and} [y_1 + z_2 + b = z_1 + y_2 + b]
  \end{equation*}

  Summing both sides, we have
  \begin{equation*}
    x_1 + y_2 + a + y_1 + z_2 + b = y_1 + x_2 + a + z_1 + y_2 + b
  \end{equation*}

  We reorder both sides to obtain
  \begin{equation*}
    (x_1 + z_2) + (y_1 + y_2 + a + b) = (x_2 + z_1) + (y_1 + y_2 + a + b),
  \end{equation*}
  which implies \( (x_1, x_2) \cong (z_1, z_2) \).

  Define \( G \coloneqq M^2 / \cong \) to be the equivalence partition\fullref{thm:equivalence_partition} of \( M \times M \). Define addition in \( G \) on members of \( M \times M \) by
  \begin{equation*}
    (x_1, x_2) + (y_1, y_2)
    \coloneqq
    (x_1 + y_1, x_2 + y_2).
  \end{equation*}

  This addition does not depend on the representative of the equivalence class since \( (x_1, x_2) \cong (x_1', x_2') \) and \( (y_1, y_2) \cong (y_1', y_2') \) implies the existence of \( k, m \in \BbbN \), such that
  \begin{balign*}
    x_1 + x_2' + a&= x_2 + x_1' + a,
    y_1 + y_2' + b&= y_2 + y_1' + b,
  \end{balign*}
  which, when combined, give
  \begin{balign*}
    (x_1 + x_2' + a) + (y_1 + y_2' + b)
    &=
    (x_2 + x_1' + a) + (y_2 + y_1' + b)
    \\
    (x_1 + y_1) + (x_2' + y_2') + (a + b)
    &=
    (x_2 + y_2) + (y_1 + x_1) + (a + b).
  \end{balign*}

  This implies
  \begin{balign*}
    (x_1 + y_1, x_2 + y_2)
    \cong
    (x_1' + y_1', x_2' + y_2').
  \end{balign*}

  The equivalence class \( [(0, 0)] \) is obviously an identity in \( G \) and contains exactly the pairs \( (x, x) \) of identical elements.

  For each member \( (x_1, x_2) \in M \times M \) we define its inverse as \( (x_2, x_1) \). It is indeed an inverse since
  \begin{equation*}
    (x_1, x_2) + (x_2, x_1) = (x_1 + x_2, x_2 + x_1),
  \end{equation*}
  which, by commutativity, belongs to \( [(0, 0)] \).

  If \( (x_1, x_2) \cong (x_1', x_2') \), then
  \begin{equation*}
    (x_1, x_2) + (x_2', x_1')
    =
    (x_1 + x_2', x_2 + x_1'),
  \end{equation*}
  where the two representatives of a pair of inverses are equal because of the equivalence \( \cong \).

  Thus \( + \) is a well-defined commutative operation on \( G \) with identity, making it an abelian group.

  Furthermore, the function
  \begin{balign*}
     & \varphi: M \to G              \\
     & \varphi(x) \coloneqq [(x, 0)]
  \end{balign*}
  is a monoid homomorphism, hence \( M \) is indeed embedded in the group. Furthermore, any group that embeds \( G \) must also embed \( M \) since \( G \setminus \varphi(M) \) consists only of the \enquote{inverse} elements of \( \varphi(M) \).
\end{proof}

\begin{definition}\label{def:group_commutator}
  Let \( G \) be any group. The commutator of \( x, y \in G \) is defined as
  \begin{equation*}
    [x, y] \coloneqq xyx^{-1}y^{-1}.
  \end{equation*}

  The commutator subgroup of \( G \) is the subgroup \hyperref[def:group_presentation]{generated} by all the commutators in \( G \).
\end{definition}

\begin{proposition}\label{thm:quotient_by_commutator_subgroup}\mcite[prop. 7.4]{Knapp2016BasicAlgebra}
  The commutator group \( G' \) of any group \( G \) is \hyperref[def:normal_subgroup]{normal} and the quotient \( G / G' \) is \hyperref[def:abelian_group]{abelian}.
\end{proposition}
