\subsection{Rings}\label{subsec:rings}

\begin{definition}
  The following are special kinds of semirings:
  \begin{thmenum}[resume=def:semiring]
    \thmitem{def:semiring/no_zero_divisor} If \( xy = 0 \) whenever \( x \) and \( y \) are nonzero, we say that the semiring has \term{no zero divisors}.

    \thmitem{def:semiring/ring} \term{Rings} are semirings with invertible addition, i.e. \( (R, +) \) forms an \hyperref[def:abelian_group]{abelian group}. The category of rings is denoted by \( \cat{Ring} \).

    \thmitem{def:semiring/unital_ring} \term{Unital rings} are \hyperref[def:semiring/ring]{rings} in which multiplication is unital, i.e. \( R, \cdot \) is a \hyperref[def:monoid]{monoid} with identity \( 1 \). Some authors define all rings to be unital. Invertible elements under multiplication are called \term{units} (see \fullref{rem:units_in_rings_etymology}) and the operation itself is called \term{division}.

    We may additionally require that the ring is nontrivial, so that \( 0 \neq 1 \) (see \fullref{ex:trivial_semiring}).

    \thmitem{def:semiring/commutative_ring} \term{Commutative rings} are \hyperref[def:semiring/ring]{rings} in which multiplication is \hyperref[def:magma/commutative]{commutative}, i.e. \( R, \cdot \) is a commutative semigroup.

    \thmitem{def:semiring/commutative_unital_ring} \term{Commutative unital rings} are (obviously) both commutative and unital and as usually assumed to be nontrivial. Despite being ubiquitous, they do not have an established one-word name.

    We are mostly interested in these rings because if \( R \) is a commutative unital ring, so is its polynomial \hyperref[def:algebra_of_polynomials]{ring} \( R[X] \) and, by induction, multivariate polynomial \hyperref[def:multivariate_polynomial]{rings} (\( R[X, Y] = R[X][Y] \)). So we are interested in properties of \( R \) that are preserves by \( R[X] \). This is the reason, so many types of commutative rings and ideals are studied. We are interested in integral domains when we speak of divisibility and factorization and since zero is absorbing, it does not interact nicely with factorization. Refer to \fullref{sec:commutative_algebra}.

    \thmitem{def:semiring/integral_domain} \term{Integral domains} are nontrivial commutative unital \hyperref[def:semiring/commutative_unital_ring]{rings} with no zero \hyperref[def:semiring_division]{divisors}. This implies that \( 1 \neq 0 \) because otherwise \( 1 \cdot 1 = 0 \) and \( 1 \) would be a zero divisor. This in turn implies that \fullref{ex:trivial_semiring}.

    \Fullref{thm:semiring_cancellative_iff_no_zero_divisors} shows that a commutative unital ring \( R \) is an integral domain if and only if its multiplication is cancellable.

    \thmitem{def:semiring/unique_factorization_domain} \term{Unique factorization domains} are integral domains in which every element has unique \hyperref[def:factorization_in_ring]{factorization} exists.

    \thmitem{def:semiring/principal_ideal_domain} \term{Principal ideal domains} are integral \hyperref[def:semiring/integral_domain]{domains} in which every \hyperref[def:semiring_ideal]{ideal} is \hyperref[def:principal_ideal]{principal}.

    By \fullref{thm:pid_is_ufd}, every principal ideal domain is a unique factorization domain.

    \thmitem{def:semiring/euclidean_domain} \term{Euclidean domains} are integral \hyperref[def:semiring/integral_domain]{domains} which allow division with remainders (see \fullref{def:euclidean_domain}).

    By \fullref{thm:euclidean_domain_is_pid}, every Euclidean domain is a principal ideal domain.

    \thmitem{def:semiring/division_ring} \term{Division rings} are nontrivial unital \hyperref[def:semiring/unital_ring]{rings} in which all nonzero elements are units, i.e. \( (F \setminus \{ 0 \}, \cdot) \) is a \hyperref[def:group]{group}. The nontriviality is a requirement because we want \( 1 \) not to be a zero divisor (see discussion in \fullref{def:semiring/integral_domain}).

    The multiplicative inverse of an element in a division ring is called its \term{reciprocal}.

    In order to fit multiplicative invertibility as an axiom, we can use the following formula:
    \begin{equation*}
      \forall \xi ((\xi \doteq 0) \lor \exists \eta (\xi \cdot \eta \doteq 1))
    \end{equation*}
    or add an additional operation \( (\cdot)^{-1} \) that inverts all nonzero elements and fixes zero, that is,
    \begin{equation*}
      \forall \xi (\xi \cdot \xi^{-1} \doteq 1),
    \end{equation*}
    where we define \( 0^{-1} = 0 \). This is only a formalism since \( 0 \) is not actually \enquote{invertible}, but it is required if we wish to avoid existential quantifiers.

    \thmitem{def:semiring/field} \term{Fields} are \hyperref[def:magma/commutative]{commutative} division \hyperref[def:semiring/division_ring]{rings}, i.e. \( (F \setminus \{ 0 \}, \cdot) \) is an \hyperref[def:abelian_group]{abelian group}. The category of fields is denoted by \( \cat{Field} \).
  \end{thmenum}
\end{definition}

\begin{example}\label{ex:semirings}
  We give examples and counterexamples of semirings. Note that the order of definitions in \fullref{def:semiring} is not preserved.

  \begin{refenum}
    \refitem{def:semiring/euclidean_domain} The base building block for the examples will be the ring \( (\BbbZ, +, \cdot) \) of \hyperref[def:set_of_integers]{integers}, which itself is an Euclidean domain (see \fullref{def:set_of_integers}).

    Another example of Euclidean domains are the polynomial rings over a field (see \fullref{thm:polynomials_over_field_are_euclidean_domain}).

    \refitem{def:semiring/identity} By removing additive inverses from the integers, we obtain the semiring \( (\BbbN, +, \cdot) \) of \hyperref[def:set_of_natural_numbers]{natural numbers}.

    Take only the commutative monoid \( (\BbbN, +) \) of \hyperref[def:set_of_natural_numbers]{natural numbers} with addition. The endomorphism semiring
    \begin{equation*}
      \End(\BbbN)
    \end{equation*}
    is a noncommutative semiring.

    Tropical \hyperref[def:tropical_semiring]{semirings} are another example noncommutative semirings.

    \refitem{def:semiring} Simple examples of semirings (but not rings) without unity are proper semiring ideals. For example, the ideal \( 2\BbbN \) of positive even numbers is a non-unital semiring.

    Another example can be given by taking a subsemiring, but not a unital subsemiring of the endomorphism semiring \( \End(\BbbN, \oplus) \), for example the functions
    \begin{balign*}
       & f_n: \BbbN \to \BbbN        \\
       & f_n(x) \coloneqq x + n,
    \end{balign*}
    where \( n > 0 \). They are closed under composition and thus form a semiring themselves, but they do not contain the identity function, so the semiring is not unital.

    \refitem{def:semiring/unital_ring} The endomorphism rings \( \End(G) \) for any abelian group \( G \) are unital, but non-commutative rings. This includes the matrix space \( R^{n \times n} \) (see \fullref{thm:finite_dimensional_operators_are_isomorphic_to_matrices}).

    \refitem{def:semiring/ring} Consider the Banach space \( C_0(\BbbC) \) of complex functions vanishing at \hyperref[def:function_spaces/c0]{infinity}. If we take addition to be pointwise addition and multiplication to be composition, then \( C_0(\co) \) becomes a non-commutative ring with no multiplicative identity because \( C_0(\co) \) does not contain the identity function.

    \refitem{def:semiring/commutative_ring} For an example of a commutative ring without unit, consider again the Banach space \( C_0(\co) \), however define multiplication as pointwise function multiplication rather than by composition. The constant function \( f(x) = 1 \) does not vanish at infinity, hence \( (C_0(\co), +, \cdot) \) is a commutative, but not unital ring.

    \refitem{def:semiring/commutative_unital_ring} We are mostly interested in different types of commutative unital rings since the polynomials over well-behaved commutative rings preserve this behavior. This is important because multivariate \hyperref[def:multivariate_polynomial]{polynomials} are defined inductively as polynomials over polynomial rings. See \fullref{thm:geometric_nullstellensatz} for an application.

    An example of a nontrivial commutative unital ring that has zero divisors is the matrix algebra \( \BbbZ^{n \times n} \) over the integers. It is a ring under addition and matrix multiplication. We have
    \begin{equation*}
      \begin{pmatrix}
        1 & 0 \\
        1 & 0
      \end{pmatrix}
      \begin{pmatrix}
        0 & 0 \\
        0 & 1
      \end{pmatrix}
      =
      \begin{pmatrix}
        0 & 0 \\
        0 & 0
      \end{pmatrix},
    \end{equation*}
    thus there are zero divisors in \( \BbbZ^{n \times n} \).

    \refitem{def:semiring/integral_domain}\cite[388]{Knapp2016BasicAlgebra} The integral domain \( \BbbZ[\sqrt{-5}] \) is not a unique factorization domain because
    \begin{equation*}
      6 = (1 + \sqrt{-5}) (1 - \sqrt{-5}) = 2 \cdot 3.
    \end{equation*}

    Note that \( \BbbZ[\sqrt{-5}] \) is an integral domain by \fullref{thm:polynomials_over_integral_domain_are_integral_domain}.

    \refitem{def:semiring/unique_factorization_domain}\cite{ProofWiki:polynomials_in_integers_is_not_principal_ideal_domain} The unique factorization domain \( \BbbZ[X] \) is not a principal ideal domain.

    Note that \( \BbbZ[X] \) is a unique factorization domain by \fullref{thm:polynomials_over_integral_domain_are_integral_domain}.

    Consider the ideal \( I \) of polynomials with an even constant term.

    Assume that \( I \) is generated by the polynomial \( p(X) \in \BbbZ[X] \). Since \( 2 \in I \), then \( p(X) \) divides \( 2 \), so \( p(X) \in \{ -2, -1, 1, 2 \} \), that is \( p(X) \) is a unit of \( \BbbZ[X] \). But then \( I = \braket{p(X)} = \BbbZ[X] \), which contradicts the definition of \( I \).

    The obtained contradiction proves that \( \BbbZ[X] \) is not a principal ideal domain.

    \refitem{def:semiring/principal_ideal_domain} Principal ideal domains are not Euclidean domains in general. Such domains are discussed in \cite{Anderson1988}.

    \refitem{def:semiring/division_ring} The \hyperref[def:ring_localization]{localization} of a noncommutative ring over its nonzero cancellative elements (characterized by \fullref{thm:ring_localization_universal_property}), if it exists, forms a division ring.

    \refitem{def:semiring/field} The canonical examples of fields include the rational \hyperref[def:set_of_rational_numbers]{numbers} \( \BbbQ \), the \hyperref[def:set_of_real_numbers]{real numbers} \( \BbbR \) and the \hyperref[def:set_of_complex_numbers]{complex numbers} \( \BbbC \).

    More generally, any nontrivial commutative unital can be embedded in a field by \fullref{def:field_of_fractions}.
  \end{refenum}
\end{example}

\begin{proposition}\label{thm:ring_homomorphism_simpler_conditions}
  A function \( f: R \to S \) between the rings \( R \) and \( S \) is a homomorphism in the sense of \fullref{def:first_order_homomorphism} if and only if for any \( x, y \in R \) it satisfies
  \begin{equation}\label{thm:ring_homomorphism_simpler_conditions/condition}
    \begin{dcases}
      f(x + y) & = f(x) + f(y), \\
      f(xy)    & = f(x) f(y),   \\
      f(1_R)   & = 1_S.
    \end{dcases}
  \end{equation}

  Note that the last condition is only for unital rings.

  In other words, if a function satisfies \fullref{thm:ring_homomorphism_simpler_conditions/condition}, the following are automatically satisfied:
  \begin{itemize}
    \item \( f(0_R) = 0_S \)
    \item for all \( x \in R \), we have \( f(-x) = -f(x) \)
    \item for all units \( x \in R \), we have \( f(x^{-1}) = f(x)^{-1} \)
  \end{itemize}
\end{proposition}
\begin{proof}
  Since \( (R, +) \) and \( (S, +) \) are groups, the first two equalities from \fullref{thm:group_homomorphism_single_condition}.

  The proof of \( f(x^{-1}) = f(x)^{-1} \) is analogous to \fullref{thm:group_homomorphism_single_condition}.
\end{proof}

\begin{definition}\label{def:ring_of_integers_modulo}
  The \hyperref[def:set_of_integers]{integers} \( \BbbZ \) form a ring under addition and multiplication. Fix a positive integer \( n > 1 \). We extend the group \( \BbbZ_n \) of integers modulo \( n \) (see \fullref{def:group_of_integers_modulo}) with the operation
  \begin{equation*}
    x \odot y \coloneqq \rem(xy, n).
  \end{equation*}

  The ring \( \BbbZ_n \) is called the \term{ring of integers modulo} \( n \).
\end{definition}
\begin{proof}
  Note that
  \begin{balign*}
     & \phantom{\cong}\thickspace \rem(x, n) \rem(y, n)
     & \pmod n \cong                           \\ &\cong
    (x - n \quot(x, n)) (y - n \quot(y, n))
     & \pmod n \cong                           \\ &\cong
    xy - n \quot(x, n) - n \quot(y, n) + n^2 \quot(x, n) \quot(y, n)
     & \pmod n \cong                           \\ &\cong
    xy
     & \pmod n. \phantom{\cong}
  \end{balign*}

  The proof that multiplication in \( \BbbZ_n \) is associative, unital and commutative becomes trivial.

  We will prove that multiplication distributes over addition. Fix \( x, y, z \in \BbbZ_n \). We have
  \begin{balign*}
    (x \oplus y) \odot z
     & =
    \rem((x \oplus y) z, n)
    =    \\ &=
    \rem(\rem(x + y, n) z, n)
    =    \\ &=
    \rem((x + y - n \quot(x + y, n)) z, n)
    =    \\ &=
    \rem((x + y)z, n).
  \end{balign*}
  and
  \begin{balign*}
    (x \odot z) \oplus (y \odot z)
     & =
    \rem([(x \odot z) + (y \odot z)], n)
    =    \\ &=
    \rem([xz - n \quot(xz, n) + yz - n \quot(yz, n)], n)
    =    \\ &=
    \rem(xz + yz, n)
    =    \\ &=
    \rem((x + y)z, n).
  \end{balign*}

  Hence,
  \begin{equation*}
    (x \oplus y) \odot z = (x \odot z) \oplus (y \odot z).
  \end{equation*}
\end{proof}

\begin{definition}\label{def:opposite_ring}\mcite[555]{Knapp2016BasicAlgebra}
  The opposite ring \( R^{-1} \) of \( R \) is defined as the same abelian group with the order of multiplication reversed. They are obviously isomorphic for commutative rings.
\end{definition}

\begin{definition}\label{def:ring_commutator}
  Let \( R \) be a ring. The commutator of \( x, y \in R \) is defined as
  \begin{equation*}
    [x, y] \coloneqq xy - yx.
  \end{equation*}

  The commutator ideal of \( R \) is the ideal \hyperref[def:generated_ring_ideal]{generated} by all the commutators in \( G \).
\end{definition}

\begin{proposition}\label{thm:quotient_by_commutator_ideal}
  The quotient \( R / I \) of any unital \hyperref[def:semiring/unital_ring]{ring} \( R \) by its commutator ideal \( I \) is \hyperref[def:semiring/commutative_ring]{commutative}.
\end{proposition}
