\section{Ring theory}\label{sec:ring_theory}
\subsection{Rings}\label{subsec:rings}

\begin{remark}\label{remark:rings}
  As for groups (see \fullref{remark:additive_group}), commutative and non-commutative rings are quite different despite having similar definitions.

  Rather than regarding rings as standalone algebraic structures, it is often convenient to regard rings as abelian groups with an additional second operation that extends multiplication by integers as defined in \fullref{def:magma_exponentiation} to arbitrary elements of the ring.

  For noncommutative ring, this second operation is usually given by function composition and for commutative rings, this operation is often truly an extension of \fullref{def:magma_exponentiation} to arbitrary ring elements.
\end{remark}

\begin{definition}\label{def:semiring}
  A \Def{semiring} \( (R, +, \cdot) \) is an algebraic \hyperref[def:algebraic_theory]{structure} with two binary operations:
  \begin{DefEnum}[series=def:semiring]
    \ILabel{def:semiring/addition} The \Def{addition} \( + \) is \hyperref[def:magma/associative]{associative}, \hyperref[def:unital_magma]{unital} and \hyperref[def:magma/commutative]{commutative}, i.e. \( (R, +) \) is a commutative \hyperref[def:unital_magma/associative]{monoid}. The additive identity is usually denoted by \( 0 \). We denote this monoid by \( R^+ \).

    \ILabel{def:semiring/multiplication} The \Def{multiplication} \( \cdot \) (usually written using juxtaposition) is \hyperref[def:magma/associative]{associative}, i.e. \( (R, \cdot) \) a \hyperref[def:magma/associative]{semigroup}. We denote this semigroup by \( R^\times \).

    \ILabel{def:semiring/distributivity} We require that \( \cdot \) \Def{distributes} over \( + \), i.e.
    \begin{align}\label{eq:def:semiring/distributivity}
      (x + y)z = xz + yz
      &&
      x(y + z) = xy + xz
    \end{align}
  \end{DefEnum}

  The \Def{trivial semiring} consists only of the additive identity \( \{ 0 \} \).

  We say that \( x \) is \Def{nilpotent} if \( x^n = 0 \) for some nonnegative integer \( n \).

  The following are special kinds of semirings:
  \begin{DefEnum}[resume=def:semiring]
    \ILabel{def:semiring/dioid} \Def{Dioids} are unital semirings, that is, both \( (R, +) \) and \( (R, \cdot) \) are monoids. The term \enquote{semiring} is sometimes reserved for dioids while \enquote{hemiring} is used for what we call semirings.

    We define integer exponentiation as a shorthand for iterated multiplication. That is, if multiplication is invertible, for \( r \in R \) and \( n \in \BZ \) we define
    \begin{equation*}
      r^n \coloneqq \begin{cases}
        1,                      & n = 0  \\
        r^{n - 1} \cdot r,      & n > 0  \\
        r^{n + 1} \cdot r^{-1}, & n < 0.
      \end{cases}
    \end{equation*}

    If multiplication is not invertible, we skip defining negative exponents.

    \ILabel{def:semiring/no_zero_divisor} If \( xy = 0 \) whenever \( x \) and \( y \) are nonzero, we say that the semiring has \Def{no zero divisors}.

    \ILabel{def:semiring/ring} \Def{Rings} are semirings with invertible addition, i.e. \( (R, +) \) forms an \hyperref[def:abelian_group]{abelian group}. The category of rings is denoted by \( \Cat{Ring} \).

    \ILabel{def:semiring/unital_ring} \Def{Unital rings} are \hyperref[def:semiring/ring]{rings} in which multiplication is unital, i.e. \( R, \cdot \) is a \hyperref[def:unital_magma/associative]{monoid} with identity \( 1 \). Some authors define all rings to be unital. Invertible \hyperref[def:algebraic_theory/invertibile_element]{elements} under multiplication are called \Def{units} (see \fullref{remark:units_in_rings_etymology}) and the operation itself is called \Def{division}.

    We may additionally require that the ring is nontrivial so that \( 0 \neq 1 \) (see \fullref{thm:semiring_properties/identities_are_equal_iff_trivial_ring}).

    \ILabel{def:semiring/commutative_ring} \Def{Commutative rings} are \hyperref[def:semiring/ring]{rings} in which multiplication is \hyperref[def:magma/commutative]{commutative}, i.e. \( R, \cdot \) is a commutative semigroup.

    \ILabel{def:semiring/commutative_unital_ring} \Def{Commutative unital rings} are (obviously) both commutative and unital and as usually assumed to be nontrivial. Despite being ubiquitous, they do not have an established one-word name.

    We are mostly interested in these rings because if \( R \) is a commutative unital ring, so is its polynomial \hyperref[def:algebra_of_polynomials]{ring} \( R[X] \) and, by induction\IND, multivariate polynomial \hyperref[def:multivariate_polynomial]{rings} (\( R[X, Y] = R[X][Y] \)). So we are interested in properties of \( R \) that are preserves by \( R[X] \). This is the reason so many types of commutative rings and ideals are studied. We are interested in integral domains when we speak of divisibility and factorization and since zero is absorbing, it does not interact nicely with factorization. Refer to \fullref{sec:commutative_algebra}.

    \ILabel{def:semiring/integral_domain} \Def{Integral domains} are nontrivial commutative unital \hyperref[def:semiring/commutative_unital_ring]{rings} with no zero \hyperref[def:commutative_ring_division]{divisors}. This implies that \( 1 \neq 0 \) because otherwise \( 1 \cdot 1 = 0 \) and \( 1 \) would be a zero divisor. This in turn implies that \fullref{thm:semiring_properties/identities_are_equal_iff_trivial_ring}.

    \Fullref{thm:semiring_properties/cancellable_iff_not_zero_divisor} shows that a commutative unital ring \( R \) is an integral domain if and only if its multiplication is cancellable.

    \ILabel{def:semiring/unique_factorization_domain} \Def{Unique factorization domains} are integral domains in which every element has unique \hyperref[def:factorization_in_ring]{factorization} exists.

    \ILabel{def:semiring/principal_ideal_domain} \Def{Principal ideal domains} are integral \hyperref[def:semiring/integral_domain]{domains} in which every \hyperref[def:semiring_ideal]{ideal} is \hyperref[def:principal_ideal]{principal}.

    By \fullref{thm:pid_is_ufd}, every principal ideal domain is a unique factorization domain.

    \ILabel{def:semiring/euclidean_domain} \Def{Euclidean domains} are integral \hyperref[def:semiring/integral_domain]{domains} which allow division with remainders (see \fullref{def:euclidean_domain}).

    By \fullref{thm:euclidean_domain_is_pid}, every Euclidean domain is a principal ideal domain.

    \ILabel{def:semiring/division_ring} \Def{Division rings} are nontrivial unital \hyperref[def:semiring/unital_ring]{rings} in which all nonzero elements are units, i.e. \( (F \setminus \{ 0 \}, \cdot) \) is a \hyperref[def:group]{group}. The nontriviality is a requirement because we want \( 1 \) not to be a zero divisor (see discussion in \fullref{def:semiring/integral_domain}).

    The multiplicative inverse of an element in a division ring is called its \Def{reciprocal}.

    In order to fit multiplicative invertibility as an axiom for \fullref{def:algebraic_theory}, we can use the following formula:
    \begin{equation*}
      \forall \xi ((\xi \doteq 0) \lor \exists \eta (\xi \cdot \eta \doteq 1))
    \end{equation*}
    or add an additional operation \( (\cdot)^{-1} \) that inverts all nonzero elements and fixes zero, that is,
    \begin{equation*}
      \forall \xi (\xi \cdot \xi^{-1} \doteq 1),
    \end{equation*}
    where we define \( 0^{-1} = 0 \). This is only a formalism since \( 0 \) is not actually \enquote{invertible}, but it is required if we wish to avoid existential quantifiers.

    \ILabel{def:semiring/field} \Def{Fields} are \hyperref[def:magma/commutative]{commutative} division \hyperref[def:semiring/division_ring]{rings}, i.e. \( (F \setminus \{ 0 \}, \cdot) \) is an \hyperref[def:abelian_group]{abelian group}. The category of fields is denoted by \( \Cat{Field} \).
  \end{DefEnum}
\end{definition}

\begin{example}\label{ex:semirings}
  We give examples and counterexamples of semirings. Note that the order of definitions in \fullref{def:semiring} is not preserved.

  \begin{RefList}
    \IRef{def:semiring/euclidean_domain} The base building block for the examples will be the ring \( (\BZ, +, \cdot) \) of \hyperref[def:integers]{integers}, which itself is an Euclidean domain (see \fullref{def:integers}).

    Another example of Euclidean domains are the polynomial rings over a field (see \fullref{thm:polynomials_over_field_are_euclidean_domain}).

    \IRef{def:semiring/dioid} By removing additive inverses from the integers, we obtain the dioid \( (\BN, +, \cdot) \) of \hyperref[def:natural_numbers]{natural numbers}.

    Take only the commutative monoid \( (\BN, +) \) of \hyperref[def:natural_numbers]{natural numbers} with addition. The endomorphism semiring
    \begin{equation*}
      \End(\BN)
    \end{equation*}
    is a noncommutative dioid.

    Tropical \hyperref[def:tropical_semiring]{semirings} are another example noncommutative dioids.

    \IRef{def:semiring} Simple examples of semirings (but not rings) without unity are proper semiring ideals. For example, the ideal \( 2\BN \) of positive even numbers is a non-unital semiring.

    Another example can be given by taking a subsemiring but not a unital subsemiring of the endomorphism dioid \( \End(\BN, \oplus) \), for example the functions
    \begin{BreakableAlign*}
       & f_n: \BN \to \BN        \\
       & f_n(x) \coloneqq x + n,
    \end{BreakableAlign*}
    where \( n > 0 \). They are closed under composition and thus form a semiring themselves, but they do not contain the identity function so the semiring is not unital.

    \IRef{def:semiring/unital_ring} The endomorphism rings \( \End(G) \) for any abelian group \( G \) are unital but non-commutative rings. This includes the matrix space \( R^{n \times n} \) (see \fullref{thm:finite_dimensional_operators_are_isomorphic_to_matrices}).

    \IRef{def:semiring/ring} Consider the Banach space \( C_0(\BC) \) of complex functions vanishing at \hyperref[def:function_spaces/c0]{infinity}. If we take addition to be pointwise addition and multiplication to be composition, then \( C_0(\Co) \) becomes a non-commutative ring with no multiplicative identity because \( C_0(\Co) \) does not contain the identity function.

    \IRef{def:semiring/commutative_ring} For an example of a commutative ring without unit, consider again the Banach space \( C_0(\Co) \), however define multiplication as pointwise function multiplication rather than by composition. The constant function \( f(x) = 1 \) does not vanish at infinity, hence \( (C_0(\Co), +, \cdot) \) is a commutative but not unital ring.

    \IRef{def:semiring/commutative_unital_ring} We are mostly interested in different types of commutative unital rings since the polynomials over well-behaved commutative rings preserve this behavior. This is important because multivariate \hyperref[def:multivariate_polynomial]{polynomials} are defined inductively as polynomials over polynomial rings. See \fullref{thm:geometric_nullstellensatz} for an application.

    An example of a nontrivial commutative unital ring that has zero divisors is the matrix algebra \( \BZ^{n \times n} \) over the integers. It is a ring under addition and matrix multiplication. We have
    \begin{equation*}
      \begin{pmatrix}
        1 & 0 \\
        1 & 0
      \end{pmatrix}
      \begin{pmatrix}
        0 & 0 \\
        0 & 1
      \end{pmatrix}
      =
      \begin{pmatrix}
        0 & 0 \\
        0 & 0
      \end{pmatrix},
    \end{equation*}
    thus there are zero divisors in \( \BZ^{n \times n} \).

    \IRef{def:semiring/integral_domain}\cite[388]{Knapp2016BAlg} The integral domain \( \BZ[\sqrt{-5}] \) is not a unique factorization domain because
    \begin{equation*}
      6 = (1 + \sqrt{-5}) (1 - \sqrt{-5}) = 2 \cdot 3.
    \end{equation*}

    Note that \( \BZ[\sqrt{-5}] \) is an integral domain by \fullref{thm:polynomials_over_integral_domain_are_integral_domain}.

    \IRef{def:semiring/unique_factorization_domain}\cite{ProofWiki:polynomials_in_integers_is_not_principal_ideal_domain} The unique factorization domain \( \BZ[X] \) is not a principal ideal domain.

    Note that \( \BZ[X] \) is a unique factorization domain by \fullref{thm:polynomials_over_integral_domain_are_integral_domain}.

    Consider the ideal \( I \) of polynomials with an even constant term.

    Assume\LEM that \( I \) is generated by the polynomial \( p(X) \in \BZ[X] \). Since \( 2 \in I \), then \( p(X) \) divides \( 2 \) so \( p(X) \in \{ -2, -1, 1, 2 \} \), that is \( p(X) \) is a unit of \( \BZ[X] \). But then \( I = \Gen{p(X)} = \BZ[X] \), which contradicts the definition of \( I \).

    The obtained contradiction proves that \( \BZ[X] \) is not a principal ideal domain.

    \IRef{def:semiring/principal_ideal_domain} Principal ideal domains are not Euclidean domains in general. Such domains are discussed in \cite{Anderson1986}.

    \IRef{def:semiring/division_ring} The \hyperref[def:ring_localization]{localization} of a noncommutative ring over its nonzero cancellative elements (characterized by \fullref{thm:ring_localization_universal_property}), if it exists, forms a division ring.

    \IRef{def:semiring/field} The canonical examples of fields include the rational \hyperref[def:rational_numbers]{numbers} \( \BQ \), the \hyperref[def:real_numbers]{real numbers} \( \BR \) and the \hyperref[def:complex_numbers]{complex numbers} \( \BC \).

    More generally, any nontrivial commutative unital can be embedded in a field by \fullref{def:field_of_fractions}.
  \end{RefList}
\end{example}

\begin{proposition}\label{thm:semiring_properties}
  Any semiring \( R \) has the following basic properties:
  \begin{PropEnum}
    \ILabel{thm:semiring_properties/zero_is_absorbing} Multiplication by \( 0 \) is \hyperref[def:algebraic_theory/absorbing_element]{absorbing}, that is, \( x0 = 0x = 0 \) for any \( x \in R \).
    \ILabel{thm:semiring_properties/one_exponents} For any integer \( n \), \( 1^n = 1 \).
    \ILabel{thm:semiring_properties/identities_are_equal_iff_trivial_ring} In a unital \hyperref[def:semiring/unital_ring]{ring}, the additive and multiplicative identities are equal if and only if the ring is trivial.
    \ILabel{thm:semiring_properties/cancellable_iff_not_zero_divisor} An element \( x \in R \) of a commutative ring is a zero divisor if and only if it is cancellable (with respect to multiplication).
  \end{PropEnum}
\end{proposition}
\begin{proof}
  \SubProofOf{thm:semiring_properties/zero_is_absorbing} Follows from \fullref{thm:left_module_properties/ring_zero_is_absorbing} and \fullref{thm:left_module_properties/module_zero_is_absorbing}.
  \SubProofOf{thm:semiring_properties/one_exponents} By definition, \( r \cdot 1 = r \) for any \( r \in R \). Hence for \( 1 \cdot 1 = 1 \). Proceeding by induction\IND, we can show that for \( 1^n = 1 \) for positive \( n \). For negative \( n \), since \( 1 \) is its own inverse, we also have \( 1^n = 1 \).

  \SubProofOf{thm:semiring_properties/identities_are_equal_iff_trivial_ring}
  Let \( 0 = 1 \) in a unital ring \( R \). Let \( r \in R \). Then, by \fullref{thm:semiring_properties/zero_is_absorbing},
  \begin{equation*}
    r = 1r = 0r = 0.
  \end{equation*}

  Thus \( r = 0 \). Since \( r \) was arbitrary, we conclude that \( R = \{ 0 \} \) is the trivial ring.

  \SubProofOf{thm:semiring_properties/zero_is_absorbing} Follows from \fullref{thm:left_module_properties/ring_zero_is_absorbing} and \fullref{thm:left_module_properties/module_zero_is_absorbing}.

  \SubProofOf{thm:semiring_properties/one_exponents} By definition, \( r \cdot 1 = r \) for any \( r \in R \). Hence for \( 1 \cdot 1 = 1 \). Proceeding by induction\IND, we can show that for \( 1^n = 1 \) for positive \( n \). For negative \( n \), since \( 1 \) is its own inverse, we also have \( 1^n = 1 \).

  \SubProofOf{thm:semiring_properties/identities_are_equal_iff_trivial_ring} The trivial ring only has one element, hence \( 0 = 1 \).

  \SufficiencyOf{thm:semiring_properties/cancellable_iff_not_zero_divisor}
  Suppose that \( x \in R \) is not cancellable. Then there exist \( y \neq z \) for which \( xy = xz \). We have \( y - z \neq 0 \) and
  \begin{equation*}
    x(y - z) = xy - xz = 0.
  \end{equation*}

  Thus \( x \) is a zero divisor.

  \NecessityOf{thm:semiring_properties/cancellable_iff_not_zero_divisor} If \( x \) is a zero divisor, fix \( y \in R \) such that \( xy = 0 \). For any \( z \in R \) we have
  \begin{equation*}
    xy = 0 = x(yz)
  \end{equation*}
  but \( y \neq yz \) in general.

  Thus \( x \) is not cancellable.
\end{proof}

\begin{definition}\label{def:semiring_characteristic}
  Fix a nontrivial unital semiring \( R \) (the nontrivial condition is essential because of \fullref{thm:embedding_preserves_characteristic}). We define its \Def{characteristic} \( \Char(R) \) via any of the equivalent definitions:
  \begin{DefEnum}
    \ILabel{def:semiring_characteristic/direct} \( \Char(R) \) is the smallest number of times \( 1_R \) must be added to itself in order to obtain \( 0_R \) and zero if \( 0_R \) cannot be obtained in this way.

    \ILabel{def:semiring_characteristic/homomorphism} The semiring \( \BZ_{\geq 0} \) of nonnegative integers can be embedded into \( R \) via the unique homomorphism
    \begin{BreakableAlign*}
       & \iota: \BZ_{\geq 0} \to R                     \\
       & \iota(k) \coloneqq \begin{cases}
        0_R,                & k = 0 \\
        \iota(k - 1) + 1_R, & k > 1
      \end{cases}
    \end{BreakableAlign*}
    that adds the identity \( 1_R \) to itself \( n \) times. This allows us to use the positive integers in any semiring. Note that if addition in \( R \) is invertible (that is, if \( R \) is a ring), we can embed all integers by defining \( \iota(n) \coloneqq -\iota(-n) \) for \( n < 0 \).

    We define \( \Char(R) \) as the positive integer \( n \) such that
    \begin{equation*}
      n\BZ_{\geq 0} \cong \ker\iota.
    \end{equation*}
  \end{DefEnum}
\end{definition}
\begin{proof}
  We will show that \( \iota \) is unique. Let \( \varphi: \BZ_{\geq 0} \to R \) be another embedding. Obviously
  \begin{equation*}
    \iota(1_{\BZ}) = 1_R = \varphi(1_{\BZ}).
  \end{equation*}

  Then by induction\IND on \( k \) we can show that
  \begin{equation*}
    \iota(k) = \underbrace{1_R + \cdots + 1_R}_{k \text{ times }} = \varphi(k).
  \end{equation*}

  Hence \( \iota = \varphi \).
\end{proof}

\begin{proposition}\label{thm:embedding_preserves_characteristic}
  If \( R \) is a nontrivial unital semiring and \( T \) is a unital supersemiring of \( R \), then
  \begin{equation*}
    \Char(T) = \Char(R).
  \end{equation*}
\end{proposition}
\begin{proof}
  Note that if \( R \) was the trivial unital semiring, its characteristic would be \( 1 \) and yet the characteristic of \( T \) could be nonzero. So the theorem would not hold if \( R \) was trivial.

  Now assume that \( R \) is nontrivial and let \( \iota: \BZ_{\geq 0} \to R \) be the embedding defining \( \Char(R) \). Then it is obviously also an embedding of \( \BZ_{\geq 0} \) into \( T \). Since \( 0_R = 0_T \), then \( \ker \iota = \ker \varphi = n\BZ_{\geq 0} \), where \( n = \Char(R) \).
\end{proof}

\begin{example}\label{ex:semiring_characteristic}
  We find examples of semiring characteristics via the embedding \( \iota \) defined in \fullref{def:semiring_characteristic}:

  \begin{ExEnum}
    \ILabel{ex:semiring_characteristic/nonnegative_integers} The zero-based \hyperref[def:natural_numbers]{natural numbers} \( \BN \) have characteristic \( \Char(\BN) = 0 \) because \( \iota \) is an isomorphism. Consequently, any supersemiring of \( \BN \) has characteristic zero, most notably the integers \( \BZ \) and the fields \( \BQ \), \( \BR \), \( \BC \).

    \ILabel{ex:semiring_characteristic/integers_modulo} The integers modulo \( n \) (see \fullref{def:ring_of_integers_modulo}) have characteristic \( \Char(\BZ_n) = n \) because of \fullref{thm:integers_modulo_isomorphic_to_quotient_group}.

    \ILabel{ex:semiring_characteristic/polynomial_ring} An \hyperref[def:algebra_over_ring]{algebra} \( A \) over a nontrivial commutative unital ring \( R \) has characteristic \( \Char(A) = R \) because of the canonical embedding of \( R \) in \( A \). In particular, polynomial \hyperref[def:algebra_of_polynomials]{rings} \( R[X] \) have the same characteristic as their ring.

    \ILabel{ex:semiring_characteristic/galois_fields} The \hyperref[thm:galois_field_existence]{Galois field} \( \BF_{p^n} \) has characteristic \( p \) because it is a field extension of \( \BF_p \).
  \end{ExEnum}
\end{example}

\begin{definition}\label{def:semiring_kernel}
  The \Def{kernel} \( \ker(f) \) of a semiring homomorphism \( f: R \to S \) is the zero \hyperref[def:zero_locus]{locus} of \( f \), that is, \hyperref[def:function/preimage]{preimage} \( f^{-1}(0_S) \).

  It is an instance of \fullref{def:categorical_kernel}.
\end{definition}

\begin{definition}\label{def:quotient_semiring}
  Let \( R \) be a ring and \( I \) be an ideal of \( M \). Define the \Def{quotient ring} to be the quotient \hyperref[def:quotient_left_module]{module} when considering \( R \) as a module over itself.
\end{definition}

\begin{theorem}\label{thm:homomorphism_theorem_for_rings}
  Let \( \varphi: R \to T \) be a homomorphism of rings. We have the isomorphism
  \begin{equation*}
    R / \ker \varphi \cong \Img \varphi.
  \end{equation*}
\end{theorem}
\begin{proof}
  Special case of \fullref{thm:homomorphism_theorem_for_left_modules}.
\end{proof}

\begin{proposition}\label{thm:ring_homomorphism_simpler_conditions}
  A function \( f: R \to S \) between the rings \( R \) and \( S \) is a homomorphism in the sense of \fullref{def:first_order_homomorphism} if and only if for any \( x, y \in R \) it satisfies
  \begin{equation}\label{thm:ring_homomorphism_simpler_conditions/condition}
    \begin{dcases}
      f(x + y) & = f(x) + f(y), \\
      f(xy)    & = f(x) f(y),   \\
      f(1_R)   & = 1_S.
    \end{dcases}
  \end{equation}

  Note that the last condition is only for unital rings.

  In other words, if a function satisfies \fullref{thm:ring_homomorphism_simpler_conditions/condition}, the following are automatically satisfied:
  \begin{itemize}
    \item \( f(0_R) = 0_S \)
    \item for all \( x \in R \), we have \( f(-x) = -f(x) \)
    \item for all units \( x \in R \), we have \( f(x^{-1}) = f(x)^{-1} \)
  \end{itemize}
\end{proposition}
\begin{proof}
  Since \( (R, +) \) and \( (S, +) \) are groups, the first two equalities from \fullref{thm:group_homomorphism_single_condition}.

  The proof of \( f(x^{-1}) = f(x)^{-1} \) is analogous to \fullref{thm:group_homomorphism_single_condition}.
\end{proof}

\begin{definition}\label{def:ring_of_integers_modulo}
  The \hyperref[def:integers]{integers} \( \BZ \) form a ring under addition and multiplication. Fix a positive integer \( n > 1 \). We extend the group \( \BZ_n \) of integers modulo \( n \) (see \fullref{def:group_of_integers_modulo}) with the operation
  \begin{equation*}
    x \odot y \coloneqq \Rem(xy, n).
  \end{equation*}

  The ring \( \BZ_n \) is called the \Def{ring of integers modulo} \( n \).
\end{definition}
\begin{proof}
  Note that
  \begin{BreakableAlign*}
     & \phantom{\equiv}\; \Rem(x, n) \Rem(y, n)
     & \pmod n \equiv                           \\ &\equiv
    (x - n \Quot(x, n)) (y - n \Quot(y, n))
     & \pmod n \equiv                           \\ &\equiv
    xy - n \Quot(x, n) - n \Quot(y, n) + n^2 \Quot(x, n) \Quot(y, n)
     & \pmod n \equiv                           \\ &\equiv
    xy
     & \pmod n. \phantom{\equiv}
  \end{BreakableAlign*}

  The proof that multiplication in \( \BZ_n \) is associative, unital and commutative becomes trivial.

  We will prove that multiplication distributes over addition. Fix \( x, y, z \in \BZ_n \). We have
  \begin{BreakableAlign*}
    (x \oplus y) \odot z
     & =
    \Rem((x \oplus y) z, n)
    =    \\ &=
    \Rem(\Rem(x + y, n) z, n)
    =    \\ &=
    \Rem((x + y - n \Quot(x + y, n)) z, n)
    =    \\ &=
    \Rem((x + y)z, n).
  \end{BreakableAlign*}
  and
  \begin{BreakableAlign*}
    (x \odot z) \oplus (y \odot z)
     & =
    \Rem([(x \odot z) + (y \odot z)], n)
    =    \\ &=
    \Rem([xz - n \Quot(xz, n) + yz - n \Quot(yz, n)], n)
    =    \\ &=
    \Rem(xz + yz, n)
    =    \\ &=
    \Rem((x + y)z, n).
  \end{BreakableAlign*}

  Hence
  \begin{equation*}
    (x \oplus y) \odot z = (x \odot z) \oplus (y \odot z).
  \end{equation*}
\end{proof}

\begin{definition}\label{def:tropical_semiring}\MarginCite{nLab:tropical_semiring}
  Fix a partially \hyperref[def:poset]{ordered} \hyperref[def:abelian_group]{abelian group} \( (M, +, \leq) \). Let \( \infty \) be a sentinel symbol not in \( M \). Define
  \begin{equation*}
    T \coloneqq M \cup \{ \infty \}
  \end{equation*}
  with operations
  \begin{BreakableAlign*}
     & \oplus: T \times T \to T                        \\
     & x \oplus y \coloneqq \begin{cases}
      \min \{ x, y \}, & x \neq \infty \text{ and } y \neq \infty \text{ and they are comparable}, \\
      \infty,          & x = \infty \text{ or } y = \infty
    \end{cases} \\
    \\
     & \odot: T \times T \to T                         \\
     & x \odot y \begin{cases}
      x + y,  & x \neq \infty \text{ and } y \neq \infty, \\
      \infty, & x = \infty \text{ or } y = \infty
    \end{cases}
  \end{BreakableAlign*}

  This makes \( (T, \oplus, \odot) \) into a \hyperref[def:semiring/dioid]{dioid} with additive identity \( \infty \) and multiplicative identity \( 0 \). We call \( (T, \oplus, \odot) \) the \( \min \)-\Def{tropical semiring} or simply the \Def{tropical semiring} over \( M \). We define the \( \max \)-\Def{tropical semiring} analogously by simply replacing \( \min \) with \( \max \).
\end{definition}

\begin{definition}\label{def:semiring_direct_product}
  Let \( \{ X_k \}_{k \in \CK} \) be a nonempty family of rings.

  Analogously to \fullref{def:group_direct_product}, we define their \Def{direct product} as the ring \( \prod_{k \in \CK} X_k \), the operations defined componentwise as
  \begin{BreakableAlign*}
     & \{ x_k \}_{k \in \CK} + \{ y_k \}_{k \in \CK}
    \coloneqq
    \{ x_k + y_k \}_{k \in \CK},                         \\
     & \{ x_k \}_{k \in \CK} \cdot \{ y_k \}_{k \in \CK}
    \coloneqq
    \{ x_k \cdot y_k \}_{k \in \CK}.
  \end{BreakableAlign*}

  We define their \Def{direct sum} as the subring of \( \prod_{k \in \CK} X_k \) (see \fullref{def:semiring_direct_product}) where only finitely many components of any ring element are different from zero.
\end{definition}

\begin{proposition}\label{thm:ring_categorical_limits}
  We are interested in \hyperref[def:categorical_limit]{categorical limits} and \hyperref[def:categorical_colimit]{colimits} in \( \Cat{Ring} \). Fix an indexed family  \( \{ X_k \}_{k \in \CK} \) of rings.
  \begin{DefEnum}
    \ILabel{thm:ring_categorical_limits/product} Their \hyperref[def:categorical_product]{categorical product} is their direct \hyperref[def:semiring_direct_product]{product} \( \prod_{k \in \CK} X_k \), the projection morphisms being inherited from \fullref{thm:set_categorical_limits/product}.
  \end{DefEnum}
\end{proposition}

\begin{definition}\label{def:opposite_ring}\MarginCite[555]{Knapp2016BAlg}
  The opposite ring \( R^{-1} \) of \( R \) is defined as the same abelian group with the order of multiplication reversed. They are obviously isomorphic for commutative rings.
\end{definition}

\begin{definition}\label{def:ring_commutator}
  Let \( R \) be a ring. The commutator of \( x, y \in R \) is defined as
  \begin{equation*}
    [x, y] \coloneqq xy - yx.
  \end{equation*}

  The commutator ideal of \( R \) is the ideal \hyperref[def:generated_ring_ideal]{generated} by all the commutators in \( G \).
\end{definition}

\begin{proposition}\label{thm:quotient_by_commutator_ideal}
  The quotient \( R / I \) of any unital \hyperref[def:semiring/unital_ring]{ring} \( R \) by its commutator ideal \( I \) is \hyperref[def:semiring/commutative_ring]{commutative}.
\end{proposition}

\begin{definition}\label{def:endomorphism_dioid}
  Let \( (X, +) \) be an monoid and let \( \End(X) \) be set of endomorphism over \( X \). We define two operations:
  \begin{itemize}
    \item Pointwise addition \( [f + g](x) \coloneqq f(x) + g(x) \).
    \item Multiplication by composition \( [fg](x) \coloneqq f(g(x)) \).
  \end{itemize}

  These operations make \( \End(X) \) into a dioid. If \( X \) is a group, then \( \End(X) \) is a ring.

  If \( X \) is a dioid, we define \( \End(X) \) to be a set of dioid endomorphisms (that is, we want the additive group homomorphisms to preserve multiplication and units). Then \( \End(X) \) is again a dioid and, if \( X \) is a unital ring, so is \( \End(X) \).
\end{definition}

\begin{definition}\label{def:function_support}
  The \Def{support} of a function \( f: S \to R \) from a set \( S \) to a semiring \( R \) is the set
  \begin{equation*}
    \Supp(f) \coloneqq \{ x \in S \colon f(x) \neq 0_R \}.
  \end{equation*}
\end{definition}

\begin{definition}\label{def:functions_vanish_nowhere}
  Let \( \Cal{F} \) be a family of functions from a set \( S \) to a ring \( R \). We say that \( \Cal{F} \) \Def{vanishes nowhere} if for every \( x \in S \) there exists a function \( f \in \Cal{F} \) such that \( f(x) \neq 0_R \).
\end{definition}

\begin{definition}\label{def:ordered_semiring}
  Extending \fullref{def:preordered_magma} to (semi)rings, we define a \Def{preordered semiring} to be a semiring \( \BR \) with a magma preorder \( \leq \) that additionally satisfies
  \begin{equation}\label{eq:def:ordered_semiring/nonnegativity}
    0 \leq y \T{and} 0 \leq y \T{implies} 0 \leq xy.
  \end{equation}
\end{definition}
