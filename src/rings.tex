\subsection{Rings}\label{subsec:rings}

\begin{definition}\label{def:ring}
  A \term{ring} is a \hyperref[def:semiring]{semiring} with additive inverses. More precisely, this means that the additive monoid is a group.

  As for semirings, rings can also be nonunital, with \hyperref[def:semiring_ideal]{ring ideals} being the main example.

  Rings have the following metamathematical properties:
  \begin{thmenum}
    \thmitem{def:ring/theory} We can construct a \hyperref[def:first_order_theory]{first-order theory} for rings by adding a unary functional symbol \( -\anon \) and the group inversion axiom \eqref{eq:def:group/theory/inverse_axiom} to the \hyperref[def:semiring/theory]{theory of semirings}.

    \thmitem{def:ring/homomorphism} A \hyperref[def:first_order_homomorphism]{first-order homomorphism} between the rings \( R \) and \( S \) is a \hyperref[def:semiring/homomorphism]{semiring homomorphism} \( \varphi: R \to S \) that additionally preserves additive inverses.

    As shown in \fullref{thm:group_homomorphism_single_condition}, this condition is not only redundant, but it also automatically follows that \( \varphi(0_R) = 0_S \).

    \thmitem{def:ring/submodel} The set \( A \subseteq R \) is a \hyperref[thm:substructure_is_model]{submodel} of \( R \) if it is a \hyperref[def:subtractive_sub_semimodule]{subtractive} \hyperref[def:semiring]{sub-semiring} of \( R \).

    Since \( 0 \in A \), \( x \in A \) implies that \( -x \in A \) because \( x + -x = 0 \).

    As a consequence of \fullref{thm:positive_formulas_preserved_under_homomorphism}, the \hyperref[def:multi_valued_function/image]{image} of a ring homomorphism \( \varphi: G \to H \) is a subring of \( H \).

    \thmitem{def:ring/trivial} The \hyperref[thm:substructures_form_complete_lattice/bottom]{trivial} ring is the \hyperref[def:pointed_set/trivial]{trivial pointed set} \( \set{ 0 } \).

    \thmitem{def:ring/category} The corresponding \hyperref[def:category_of_small_first_order_models]{category of \( \mscrU \)-small models} \( \ucat{Ring} \) is \hyperref[def:concrete_category]{concrete} over \hyperref[def:monoid]{\( \ucat{SRing} \)}.

    \thmitem{def:ring/commutative} If multiplication is commutative, we call the ring itself \term{commutative}. Unless multiplication corresponds to function composition, most rings we will encounter will be commutative. We denote the category of commutative rings by \( \cat{CRing} \).

    \Fullref{sec:commutative_algebra} is entirely devoted to commutative unital rings.
  \end{thmenum}
\end{definition}

\begin{proposition}\label{thm:ring_characteristic_homomorphism}
  Similarly to how \( \BbbN \) is an \hyperref[def:universal_objects/initial]{initial object} in the category \hyperref[def:semiring/category]{\( \cat{SRing} \)} of semirings, \( \BbbZ \) is an initial object in the category \hyperref[def:ring/category]{\( \cat{Ring} \)} of rings.
\end{proposition}
\begin{proof}
  Follows from \fullref{thm:semiring_characteristic_homomorphism} with the addition of \( (-n)x = -nx \).
\end{proof}

\begin{proposition}\label{def:ring_of_integers_modulo}
  For a positive integer \( n > 1 \), we extend the group \hyperref[thm:group_of_integers_modulo]{\( \BbbZ_n \)} of integers modulo \( n \) with the operation
  \begin{equation*}
    x \odot y \coloneqq \rem(xy, n).
  \end{equation*}

  The ring \( \BbbZ_n \) is called the \term{ring of integers modulo} \( n \).
\end{proposition}
\begin{proof}
  Note that
  \begin{balign*}
    &\phantom{{}\cong{}} \rem(x, n) \rem(y, n)
    &\cong \pmod n \\ &\cong
    (x - n \quot(x, n)) (y - n \quot(y, n))
    &\cong \pmod n \\ &\cong
    xy - n \quot(x, n) - n \quot(y, n) + n^2 \quot(x, n) \quot(y, n)
    &\cong \pmod n \\ &\cong
    xy
    &
  \end{balign*}

  The proof that multiplication in \( \BbbZ_n \) is associative, unital and commutative becomes trivial.

  We will prove that multiplication distributes over addition. Fix \( x, y, z \in \BbbZ_n \). We have
  \begin{balign*}
    (x \oplus y) \odot z
     & =
    \rem((x \oplus y) z, n)
    =    \\ &=
    \rem(\rem(x + y, n) z, n)
    =    \\ &=
    \rem((x + y - n \quot(x + y, n)) z, n)
    =    \\ &=
    \rem((x + y)z, n).
  \end{balign*}
  and
  \begin{balign*}
    (x \odot z) \oplus (y \odot z)
     & =
    \rem([(x \odot z) + (y \odot z)], n)
    =    \\ &=
    \rem([xz - n \quot(xz, n) + yz - n \quot(yz, n)], n)
    =    \\ &=
    \rem(xz + yz, n)
    =    \\ &=
    \rem((x + y)z, n).
  \end{balign*}

  Hence,
  \begin{equation*}
    (x \oplus y) \odot z = (x \odot z) \oplus (y \odot z).
  \end{equation*}
\end{proof}

\begin{theorem}[Homomorphism theorem for rings]\label{thm:homomorphism_theorem_for_rings}
  Every \hyperref[def:ring/homomorphism]{ring homomorphism} \( \varphi: R \to S \) induces an isomorphism
  \begin{equation*}
    R / \ker \varphi \cong \img \varphi.
  \end{equation*}

  Compare this to \fullref{thm:homomorphism_theorem_for_groups} and \fullref{thm:homomorphism_theorem_for_modules}. The latter theorem is a generalization of this one.
\end{theorem}
\begin{proof}
  \Fullref{thm:homomorphism_theorem_for_groups} defines an explicit additive group isomorphism \( \phi: R / \ker \varphi \to \img \varphi \). The ring structure simply restricts \( \ker \varphi \) to be an ideal rather than an arbitrary subgroup of the additive group of \( R \).
\end{proof}

\begin{proposition}\label{thm:semiring_grothendieck_completion}
  The \hyperref[def:monoid_grothendieck_completion]{Grothendieck completion} \( \overline{R} \) of the additive monoid of a \hyperref[def:semiring]{semiring} \( R \) becomes a \hyperref[def:ring/commutative]{commutative ring} with the operation
  \begin{equation*}
    [(x_1, x_2)] \odot [(y_1, y_2)] \coloneqq [(x_1 \cdot y_1 + x_2 \cdot y_2, x_1 \cdot y_2 + x_2 \cdot y_1)].
  \end{equation*}

  This definition is motivated in the proof of \fullref{thm:semiring_grothendieck_completion_universal_property}.
\end{proposition}
\begin{proof}
  Multiplication on \( R \) does not depend on the representative of the equivalence class. Indeed, let \( (x_1, x_2) \sim (x_1', x_2') \) and \( (y_1, y_2) \sim (y_1', y_2') \). Then there exist \( a \) and \( b \) such that
  \begin{align*}
    x_1 + x_2' + a &= x_1' + x_2 + a, \\
    y_1 + y_2' + b &= y_1' + y_2 + b.
  \end{align*}

  A tedious check, which is easily automated via software, reveals that there exists a constant \( c \) such that
  \begin{align*}
    &\phantom{{}={}}
    (x_1 \cdot y_1 + x_2 \cdot y_2) + (x_1' \cdot y_1' + x_2' \cdot y_2') + c
    = \\ &=
    \cdots
    = \\ &=
    \parens[\Big]{ (x_1 + x_2' + a) + (x_1' + x_2 + a) } \cdot \parens[\Big]{ (y_1 + y_2' + b) + (y_1' + y_2 + b) }
    = \\ &=
    \cdots
    = \\ &=
    (x_1 \cdot y_1 + x_2 \cdot y_2) + (x_1' \cdot y_1' + x_2' \cdot y_2') + c
  \end{align*}

  Therefore,
  \begin{equation*}
    (x_1 \cdot y_1 + x_2 \cdot y_2, x_1 \cdot y_2 + x_2 \cdot y_1) \sim (x_1' \cdot y_1' + x_2' \cdot y_2', x_1' \cdot y_2' + x_2' \cdot y_1').
  \end{equation*}

  All conditions imposed on multiplication in \( \overline{R} \) are inherited from \( R \) and are proved directly.
\end{proof}

\begin{proposition}\label{thm:semiring_grothendieck_completion_universal_property}\mcite[80]{OpenLogicFull}
  The \hyperref[thm:semiring_grothendieck_completion]{Grothendieck completion} \( \overline{R} \) of a commutative semiring \( R \) satisfies the following \hyperref[rem:universal_mapping_property]{universal mapping property}:
  \begin{displayquote}
    For every commutative ring \( S \) and every semiring homomorphism \( \varphi: R \to S \), there exists a unique ring homomorphism \( \widetilde{\varphi}: \overline{R} \to S \) such that the following diagram commutes:
    \begin{equation}\label{eq:thm:semiring_grothendieck_completion_universal_property/diagram}
      \begin{aligned}
        \includegraphics[page=1]{output/thm__semiring_grothendieck_completion_universal_property.pdf}
      \end{aligned}
    \end{equation}
  \end{displayquote}

  Via \fullref{rem:universal_mapping_property}, \( \overline{\anon} \) becomes \hyperref[def:category_adjunction]{left adjoint} to the \hyperref[def:concrete_category]{forgetful functor} \( U: \cat{CRing} \to \cat{CSRing} \).

  Compare this result to \fullref{thm:monoid_grothendieck_completion_universal_property}.
\end{proposition}
\begin{proof}
  \Fullref{thm:monoid_grothendieck_completion_universal_property} suggests the definition
  \begin{equation*}
    \overline{\varphi}([(x, y)]) \coloneqq \varphi(x) - \varphi(y).
  \end{equation*}

  We must only show that \( \overline{\varphi} \) is a ring homomorphism. Clearly
  \begin{equation*}
    \overline{\varphi}([(1, 0)]) = \varphi(1) - \varphi(0),
  \end{equation*}
  which implies that \( \varphi \) preserves multiplicative identities. Also,
  \begin{balign*}
    \overline{\varphi}\parens[\Big]{ [(x_1, x_2)] \odot [(y_1, y_2)] }
    &=
    \overline{\varphi}\parens[\Big]{ [(x_1 \cdot y_1 + x_2 \cdot y_2, x_1 \cdot y_2 + x_2 \cdot y_1)] }
    = \\ &=
    \varphi(x_1 \cdot y_1 + x_2 \cdot y_2) - \varphi(x_1 \cdot y_2 + x_2 \cdot y_1)
    = \\ &=
    \varphi(x_2) \parens[\Big]{ \varphi(y_2) - \varphi(y_1) } - \varphi(x_1) \parens[\Big]{ \varphi(y_2) - \varphi(y_1) }
    = \\ &=
    \parens[\Big]{ \varphi(x_2) - \varphi(x_1) } \parens[\Big]{ \varphi(y_2) - \varphi(y_1) }
    = \\ &=
    \overline{\varphi}\parens[\Big]{ [(x_1, x_2)] } \overline{\varphi}\parens[\Big]{ [(y_1, y_2)] }.
  \end{balign*}
\end{proof}

\begin{definition}\label{def:ring_commutator}
  Let \( R \) be an arbitrary ring. We define the \term{commutator} of the elements \( x \) and \( y \) as
  \begin{equation*}
    [x, y] \coloneqq xy - yx.
  \end{equation*}

  The \term{commutator ideal} \( [R, R] \) of \( R \) is the ideal \hyperref[def:generated_ring_ideal]{generated} by all the commutators in \( G \).
\end{definition}

\begin{proposition}\label{thm:quotient_by_commutator_ideal}
  The \hyperref[def:quotient_semiring]{quotient ring} of any ring \( R \) by its commutator ideal \( [R, R] \) is a \hyperref[def:ring/commutative]{commutative ring}.
\end{proposition}
\begin{proof}
  For \( x \) and \( y \) in \( R \), since \( yx - xy \in I \), we have
  \begin{equation*}
    (x + I) (y + I)
    =
    (xy + I)
    =
    (xy + yx - xy + I)
    =
    (yx + I)
    =
    (y + I) (x + I).
  \end{equation*}
\end{proof}
