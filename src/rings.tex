\subsection{Rings}\label{subsec:rings}

\begin{definition}\label{def:ring}
  A \term{ring} is a \hyperref[def:semiring]{semiring} with additive inverses. More precisely, this means that the additive monoid is a group.

  As for semirings, rings can also be nonunital, with \hyperref[def:semiring_ideal]{ring ideals} being the main example.

  Rings have the following metamathematical properties:
  \begin{thmenum}
    \thmitem{def:ring/theory} We can construct a \hyperref[def:first_order_theory]{first-order theory} for rings by adding a unary functional symbol \( - \) and the group inversion axiom \eqref{eq:def:group/theory/inverse_axiom} to the \hyperref[def:semiring/theory]{theory of semirings}.

    \thmitem{def:ring/homomorphism} A \hyperref[def:first_order_homomorphism]{first-order homomorphism} between the rings \( R \) and \( T \) is a \hyperref[def:semiring/homomorphism]{semiring homomorphism} \( \varphi: R \to T \) that additionally preserves additive inverses.

    As shown in \fullref{thm:group_homomorphism_single_condition}, this condition is not only redundant, but the structure of a ring rather than semiring also automatically implies that \( \varphi(0_R) = 0_S \).

    \thmitem{def:ring/submodel} The set \( A \subseteq R \) is a \hyperref[thm:substructure_is_model]{submodel} of \( R \) if it is both a \hyperref[def:semiring]{sub-semiring} of \( R \) and an additive submonoid of \( R \).

    As a consequence of \fullref{thm:positive_formulas_preserved_under_homomorphism}, the image of a ring homomorphism is a subring of its range.

    \thmitem{def:ring/trivial} The \hyperref[thm:substructures_form_complete_lattice/bottom]{trivial} ring is the \hyperref[def:pointed_set/trivial]{trivial pointed set} \( \set{ 0 } \).

    \thmitem{def:ring/commutative} If multiplication is commutative, we call the ring itself \term{commutative}. Unless multiplication corresponds to function composition, most rings we will encounter will be commutative.

    \thmitem{def:ring/category} The corresponding \hyperref[def:category_of_small_first_order_models]{category of \( \mscrU \)-small models} \( \ucat{Ring} \) is \hyperref[def:concrete_category]{concrete} over \hyperref[def:monoid]{\( \ucat{SRing} \)}. We denote the category of commutative rings by \( \cat{CRing} \).

    Unlike the category \hyperref[def:group/category]{\( \cat{Grp} \)} of groups, \( \cat{Ring} \) is not as well-behaved. Nevertheless, kernels and quotients of rings are commonly established concepts.

    The category of unital rings does not have a zero object, but the category of nonunital rings does, and we will sometimes consider nonunital ring homomorphisms between unital rings. That is, ring homomorphisms that may not preserve the multiplicative identity.

    \thmitem{def:ring/kernel} The \term{kernel} of a ring homomorphism \( \varphi: R \to T \) is simply its \hyperref[def:zero_locus]{zero locus} \( \varphi^{-1}(0_S) \). This is precisely the kernel of the additive group in the sense of \fullref{def:group/kernel}, and the \hyperref[def:zero_morphisms/cokernel]{categorical kernel} in the category of nonunital rings.

    Furthermore, \( \ker \varphi \) is an ideal of \( R \) because, if \( x \in \ker \varphi \),
    \begin{equation*}
      \varphi(xy)
      =
      \varphi(x) \varphi(y)
      =
      0_S \varphi(y)
      =
      0_S,
    \end{equation*}
    and thus \( xy \in \ker \varphi \).

    Despite being categorical kernels only in the category of nonunital rings, the kernel is defined and used mainly for unital ring homomorphism.

    \thmitem{def:ring/quotient} The \hyperref[def:zero_morphisms/cokernel]{categorical cokernel} of homomorphism \( \varphi: R \to T \) in the category of nonunital rings is, similarly to the case for groups in \fullref{def:group/quotient}, a partition of \( T \) induced by the image of \( \varphi \).

    This is not merely the cokernel \( T / \img \varphi \) of the additive group, however. Multiplication induces an additional restriction on congruences: \( x \cong x' \) and \( y \cong y' \) together imply \( x y \cong x' y' \). Hence, \( [x][y] = [xy] \).  Denote the coset \( [0_S] \) by \( I \). We have \( I[x] = [0x] = I \), therefore the cokernel inherits absorption from \( T \).

    Additive subgroups of \( T \) that absorb multiplication are precisely the \hyperref[def:semiring_ideal]{ideals} of \( T \). Hence, \( I \) is the ideal generated by \( \img \varphi \). From the general case for groups it follows that quotient ring cosets have the form \( x + I \).

    Finally, given an ideal \( I \) of an arbitrary ring \( R \), we can define the \term{quotient ring} \( R / I \) as the cokernel of the inclusion \( \iota: I \to R \). That is, \( R / I \) consists of the cosets \( x + I \) for \( x \in R \). In practice, quotients are conveniently characterized by \fullref{thm:quotient_ring_universal_property}.

    Somewhat similarly to \fullref{thm:def:group/properties/kernel_cokernel_compatibility} for groups, the kernel \( \ker \pi \) of the canonical projection \( \pi(x) \coloneqq x + I \) is the ideal \( I \) itself.

    Despite being \hyperref[def:zero_morphisms/cokernel]{categorical cokernel} only in the category of nonunital rings, the quotient \( R / I \) is defined and used mainly for unital rings.

    Fortunately, for unital ring \( R \), the quotient \( R / I \) is also unital. The projection morphism is an epimorphism by \fullref{thm:equalizer_invertibility}, and hence \( R / I \) is a \hyperref[def:subobject_and_quotient]{categorical quotient object}.

    \thmitem{def:ring/simple} Analogously to \hyperref[def:group/simple]{simple groups}, if the only proper \hyperref[def:semiring_ideal]{ideal} of \( R \) is the \hyperref[def:ring/trivial]{trivial ideal} \( \set{ 0_R } \), we say that \( R \) is a \term{simple ring}.

    The trivial ring itself is not simple, because it has no proper ideals.
  \end{thmenum}
\end{definition}

\begin{proposition}\label{thm:ring_characteristic_homomorphism}
  Similarly to how \( \BbbN \) is an \hyperref[def:universal_objects/initial]{initial object} in the category \hyperref[def:semiring/category]{\( \cat{SRing} \)} of semirings, \( \BbbZ \) is an initial object in the category \hyperref[def:ring/category]{\( \cat{Ring} \)} of rings.
\end{proposition}
\begin{proof}
  Follows from \fullref{thm:semiring_characteristic_homomorphism} with the addition of \( (-n)x = -nx \).
\end{proof}

\begin{proposition}\label{thm:ring_of_integers_modulo}
  For a positive integer \( n > 1 \), we extend the group \hyperref[thm:group_of_integers_modulo]{\( \BbbZ_n \)} of integers modulo \( n \) with the operation
  \begin{equation*}
    x \odot y \coloneqq \rem(xy, n).
  \end{equation*}

  Then \( \BbbZ_n \) is a \hyperref[def:ring/commutative]{commutative ring} called the \term{ring of integers modulo} \( n \).
\end{proposition}
\begin{proof}
  Note that
  \begin{balign*}
    &\phantom{{}\cong{}} \rem(x, n) \rem(y, n)
    &\cong \pmod n \\ &\cong
    (x - n \quot(x, n)) (y - n \quot(y, n))
    &\cong \pmod n \\ &\cong
    xy - n \quot(x, n) - n \quot(y, n) + n^2 \quot(x, n) \quot(y, n)
    &\cong \pmod n \\ &\cong
    xy
    &
  \end{balign*}

  The proof that multiplication in \( \BbbZ_n \) is associative, unital and commutative becomes trivial.

  We will prove that multiplication distributes over addition. Fix \( x, y, z \in \BbbZ_n \). We have
  \begin{balign*}
    (x \oplus y) \odot z
     & =
    \rem((x \oplus y) z, n)
    =    \\ &=
    \rem(\rem(x + y, n) z, n)
    =    \\ &=
    \rem((x + y - n \quot(x + y, n)) z, n)
    =    \\ &=
    \rem((x + y)z, n).
  \end{balign*}
  and
  \begin{balign*}
    (x \odot z) \oplus (y \odot z)
     & =
    \rem([(x \odot z) + (y \odot z)], n)
    =    \\ &=
    \rem([xz - n \quot(xz, n) + yz - n \quot(yz, n)], n)
    =    \\ &=
    \rem(xz + yz, n)
    =    \\ &=
    \rem((x + y)z, n).
  \end{balign*}

  Hence,
  \begin{equation*}
    (x \oplus y) \odot z = (x \odot z) \oplus (y \odot z).
  \end{equation*}
\end{proof}

\begin{theorem}[Quotient ring universal property]\label{thm:quotient_ring_universal_property}
  For every \hyperref[def:ring]{ring} \( R \) and \hyperref[def:semiring_ideal]{ideal} \( I \), the \hyperref[def:ring/quotient]{quotient ring} \( R / I \) has the following \hyperref[rem:universal_mapping_property]{universal mapping property}:
  \begin{displayquote}
    Every ring homomorphism \( \varphi: R \to T \) satisfying \( I \subseteq \ker \varphi \) \hyperref[def:factors_through]{uniquely factors through} \( R / I \). That is, there exists a unique homomorphism \( \widetilde{\varphi}: R / I \to T \) such that the following diagram commutes:
    \begin{equation}\label{eq:thm:quotient_ring_universal_property/diagram}
      \begin{aligned}
        \includegraphics[page=1]{output/thm__quotient_ring_universal_property.pdf}
      \end{aligned}
    \end{equation}

    In the case where \( I = \ker \varphi \), \( \widetilde{\varphi} \) is an \hyperref[def:first_order_homomorphism_invertibility/embedding]{embedding}.
  \end{displayquote}

  Compare this result to \fullref{thm:quotient_group_universal_property} and \fullref{thm:quotient_module_universal_property}.
\end{theorem}
\begin{proof}
  Simple refinement of \fullref{thm:quotient_group_universal_property}.
\end{proof}

\begin{theorem}[Quotient subring lattice theorem]\label{thm:quotient_subring_lattice_theorem}
  Given a \hyperref[def:semiring_ideal]{two-sided ideal} \( I \) of the \hyperref[def:ring]{ring} \( R \), the function \( T \mapsto T / I \) is a \hyperref[def:semilattice/homomorphism]{lattice isomorphism} between the lattice of \hyperref[def:ring/submodel]{subrings} of \( R \) containing \( I \) and the lattice of subrings of the \hyperref[def:ring/quotient]{quotient} \( R / I \).

  Furthermore, the sublattices of \hyperref[def:semiring_ideal/prime]{prime}, \hyperref[def:semiring_ideal/maximal]{maximal} or \hyperref[def:radical_ideal]{radical} ideals are also isomorphic.

  Compare this result to \fullref{thm:quotient_subgroup_lattice_theorem} and \fullref{thm:quotient_submodule_lattice_theorem}.
\end{theorem}
\begin{proof}
  \SubProof{Proof for general ideals} Simple refinement of \fullref{thm:quotient_subgroup_lattice_theorem}.

  \SubProof{Proof for prime ideals}
  \SufficiencySubProof* Suppose that \( P \) is a prime ideal in \( R \) containing \( I \). Let \( A \) and \( B \) be ideals of \( P \) containing \( I \) such that \( (A / I) (B / I) \subseteq P / I \). By \fullref{thm:quotient_subring_lattice_theorem}, \( AB \subseteq P \), which implies that \( A \subseteq P \) or \( B \subseteq P \). Again using \fullref{thm:quotient_subring_lattice_theorem}, we conclude that \( A / I \subseteq P / I \) or \( B / I \subseteq P / I \), meaning that \( P / I \) is a prime ideal in \( R / I \).

  \NecessitySubProof* Suppose that \( P / I \) is a prime ideal in \( R / I \) and let \( AB \subseteq P \). Note that neither \( A \) nor \( B \) do not necessarily contain \( I \), but \( A + I \) and \( B + I \) do. Furthermore, \( A + I \subseteq P \) and \( B + I \subseteq P \). Hence, by \fullref{thm:quotient_subring_lattice_theorem}, \( [(A + I) / I][(B + I) / I] \subseteq P / I \), and since \( P / I \) is prime, \( (A + I) / I \subseteq P / I \) or \( (B + I) / I \subseteq P / I \). Again using \fullref{thm:quotient_subring_lattice_theorem}, we conclude that \( A \subseteq A + I \subseteq P \) or \( B \subseteq B + I \subseteq P \).

  \SubProof{Proof for maximal ideals} Trivial consequence of \fullref{thm:quotient_subring_lattice_theorem}.

  \SubProof{Proof for radical ideals} Correspondence of radical ideals follows from correspondence of prime ideals since any radical ideal satisfies \fullref{def:radical_ideal/intersection} and thus equals the intersection of all prime ideals containing it.
\end{proof}

\begin{proposition}\label{thm:quotient_by_maximal_ideal}
  The two-sided ideal \( M \) of the \hyperref[def:ring]{ring} \( R \) is \hyperref[def:semiring_ideal/maximal]{maximal} if and only if the \hyperref[def:ring/quotient]{quotient} \( R / M \) is a \hyperref[def:ring/simple]{simple ring}.

  See \fullref{thm:quotient_by_prime_ideal} for the corresponding statement for \hyperref[def:semiring_ideal/prime]{prime ideals} in commutative rings.
\end{proposition}
\begin{proof}
  Since \( M \) is maximal, only \( M \) and \( R \) are ideals of \( R \) containing \( M \). Therefore, by \fullref{thm:quotient_subring_lattice_theorem}, \( R / M \) has only two ideals.
\end{proof}

\begin{proposition}\label{thm:semiring_grothendieck_completion}\mcite[80]{OpenLogicFull}
  The \hyperref[def:monoid_grothendieck_completion]{Grothendieck completion} \( \overline{R} \) of the additive monoid of a \hyperref[def:semiring]{semiring} \( R \) becomes a \hyperref[def:ring/commutative]{commutative ring} with the operation
  \begin{equation*}
    [(a, b)] \odot [(c, d)] \coloneqq [(ac + bd, ad + bc)].
  \end{equation*}

  This definition is motivated in the proof of \fullref{thm:semiring_grothendieck_completion_universal_property}.
\end{proposition}
\begin{proof}
  Multiplication on \( R \) does not depend on the representative of the equivalence class. Indeed, let \( (a, b) \sim (a', b') \) and \( (c, d) \sim (c', d') \). Then there exist \( u \) and \( v \) such that
  \begin{align*}
    a + b' + u &= a' + b + u, \\
    c + d' + v &= c' + d + v.
  \end{align*}

  Then
  \begin{align*}
    &\phantom{{}={}}
    \hi{ac} + b'c + uc + a'd + \hi{bd} + ud + a'c + \hi{a'd'} + a'v + \hi{b'c'} + b'd + b'v
    = \\ &=
    (a + b' + u)c + (a' + b + u)d + a'(c + d' + v) + b'(c' + d + v)
    = \\ &=
    (a' + b + u)c + (a + b' + u)d + a'(c' + d + v) + b'(c + d' + v)
    = \\ &=
    a'c + \hi{bc} + uc + \hi{ad} + b'd + ud + \hi{a'c'} + a'd + a'v + b'c + \hi{b'd'} + b'v.
  \end{align*}

  Therefore,
  \begin{equation*}
    (a \cdot c + b \cdot d, a \cdot d + b \cdot c) \sim (a' \cdot c' + b' \cdot d', a' \cdot d' + b' \cdot c').
  \end{equation*}

  Associativity, commutativity and distributivity in \( \overline{R} \) are inherited from \( R \).
\end{proof}

\begin{proposition}\label{thm:semiring_grothendieck_completion_universal_property}
  The \hyperref[thm:semiring_grothendieck_completion]{Grothendieck completion} \( \overline{R} \) of a commutative semiring \( R \) satisfies the following \hyperref[rem:universal_mapping_property]{universal mapping property}:
  \begin{displayquote}
    For every commutative ring \( T \) and every semiring homomorphism \( \varphi: R \to T \), there exists a unique ring homomorphism \( \widetilde{\varphi}: \overline{R} \to T \) such that the following diagram commutes:
    \begin{equation}\label{eq:thm:semiring_grothendieck_completion_universal_property/diagram}
      \begin{aligned}
        \includegraphics[page=1]{output/thm__semiring_grothendieck_completion_universal_property.pdf}
      \end{aligned}
    \end{equation}
  \end{displayquote}

  Via \fullref{rem:universal_mapping_property}, \( \overline{\anon} \) becomes \hyperref[def:category_adjunction]{left adjoint} to the \hyperref[def:concrete_category]{forgetful functor} \( U: \cat{CRing} \to \cat{CSRing} \).

  Compare this result to \fullref{thm:monoid_grothendieck_completion_universal_property}.
\end{proposition}
\begin{proof}
  \Fullref{thm:monoid_grothendieck_completion_universal_property} suggests the definition
  \begin{equation*}
    \overline{\varphi}([(a, b)]) \coloneqq \varphi(a) - \varphi(b).
  \end{equation*}

  We must only show that \( \overline{\varphi} \) is a ring homomorphism. Clearly
  \begin{equation*}
    \overline{\varphi}([(1, 0)]) = \varphi(1) - \varphi(0),
  \end{equation*}
  which implies that \( \varphi \) preserves multiplicative identities. Also,
  \begin{balign*}
    \overline{\varphi}\parens[\Big]{ [(a, b)] \odot [(c, d)] }
    &=
    \overline{\varphi}\parens[\Big]{ [(a \cdot b + c \cdot d, a \cdot d + b \cdot c)] }
    = \\ &=
    \varphi(a \cdot b + c \cdot d) - \varphi(a \cdot d + b \cdot c)
    = \\ &=
    \varphi(c) \parens[\Big]{ \varphi(d) - \varphi(b) } - \varphi(a) \parens[\Big]{ \varphi(d) - \varphi(b) }
    = \\ &=
    \parens[\Big]{ \varphi(c) - \varphi(a) } \parens[\Big]{ \varphi(d) - \varphi(b) }
    = \\ &=
    \overline{\varphi}\parens[\Big]{ [(a, c)] } \overline{\varphi}\parens[\Big]{ [(b, d)] }.
  \end{balign*}
\end{proof}

\begin{definition}\label{def:ring_commutator}
  Let \( R \) be an arbitrary ring. We define the \term{commutator} of the elements \( x \) and \( y \) as
  \begin{equation*}
    [x, y] \coloneqq xy - yx.
  \end{equation*}

  The \term{commutator ideal} \( [R, R] \) of \( R \) is the ideal \hyperref[def:generated_ring_ideal]{generated} by all the commutators in \( G \).
\end{definition}

\begin{proposition}\label{thm:ring_abelianization_universal_property}\mcite[prop. 7.4]{Knapp2016BasicAlgebra}
  The quotient \( R / [R, R] \) of a ring \( R \) by its commutator ideal \( [R, R] \) is a commutative ring, which we call the \term{abelianization} of \( R \), and satisfies the following \hyperref[rem:universal_mapping_property]{universal mapping property}:
  \begin{displayquote}
    For every commutative ring \( T \) and every ring homomorphism \( \varphi: R \to T \), \( \varphi \) \hyperref[def:factors_through]{uniquely factors through} \( R / [R, R] \). That is, there exists a unique ring homomorphism \( \widetilde{\varphi}: R / [R, R] \to T \) such that the following diagram commutes:
    \begin{equation}\label{eq:thm:ring_abelianization_universal_property/diagram}
      \begin{aligned}
        \includegraphics[page=1]{output/thm__ring_abelianization_universal_property.pdf}
      \end{aligned}
    \end{equation}
  \end{displayquote}

  Via \fullref{rem:universal_mapping_property}, the abelianization functor becomes \hyperref[def:category_adjunction]{left adjoint} to the \hyperref[def:concrete_category]{forgetful functor} \( U: \cat{CRing} \to \cat{Ring} \).

  Compare this result to \fullref{thm:group_abelianization_universal_property}.
\end{proposition}
\begin{proof}
  This is a refinement of \fullref{thm:group_abelianization_universal_property}, and we only need to show that \( R / [R, R] \) is a commutative ring. For \( x \) and \( y \) in \( R \), since \( yx - xy \in I \), we have
  \begin{equation*}
    (x + I) (y + I)
    =
    (xy + I)
    =
    (xy + yx - xy + I)
    =
    (yx + I)
    =
    (y + I) (x + I).
  \end{equation*}
\end{proof}

\begin{definition}\label{def:multiplicative_set_in_ring}\mcite[428]{Knapp2016BasicAlgebra}
  We call the subset \( A \) of the ring \( R \) a \term{multiplicative set} if it contains \( 1_R \) and, furthermore, it is closed under multiplication.
\end{definition}

\begin{proposition}\label{thm:complement_of_prime_ideal}
  The \hyperref[def:semiring_ideal]{ideal} \( P \) in the \hyperref[def:ring/commutative]{commutative ring} \( R \) is \hyperref[def:semiring_ideal/prime]{prime} if and only if \( R \setminus P \) is a \hyperref[def:multiplicative_set_in_ring]{multiplicative set}.

  Not all multiplicative sets are obtained as complements of prime ideals --- see \fullref{ex:def:ring_localization/powers_of_two}.
\end{proposition}
\begin{proof}
  By \fullref{thm:def:semiring_ideal/properties/proper_ideals_containing_identity}, \( P \) is a proper ideal if and only if \( 1_R \in R \setminus P \).

  By \fullref{thm:def:semiring_ideal/properties/prime_pointwise}, \( P \) is prime if and only if \( x, y \in R \setminus P \) implies \( xy \in R \setminus P \).
\end{proof}

\begin{definition}\label{def:ring_localization}\mcite[428]{Knapp2016BasicAlgebra}
  Let \( R \) be a \hyperref[def:ring/commutative]{commutative ring} and let \( S \subseteq R \) be a \hyperref[def:multiplicative_set_in_ring]{multiplicative set}.

  Define the equivalence relation \( (r, s) \sim (r', s') \) on \( R \times S \) to hold if and only if there exists some \( u \in S \) such that \( u r s' = u r' s \).

  Define the ring
  \begin{equation*}
    S^{-1} R \coloneqq R \times S / \sim,
  \end{equation*}
  whose cosets we will denote by \( \ifrac r s \) rather than \( [(r, s)] \), with operations
  \begin{align*}
    \frac a b + \frac c d     &\coloneqq \frac {a d + b c} {b d}, \\
    \frac a b \cdot \frac c d &\coloneqq \frac {a c} {b d},
  \end{align*}
  and a canonical inclusion
  \begin{equation*}
    \begin{aligned}
      &\iota: R \to S^{-1} R \\
      &\iota(r) \coloneqq \frac r {1_R}.
    \end{aligned}
  \end{equation*}

  This ring is called the \term{localization} of \( R \) with respect to \( A \) and denote it by \( S^{-1} R \). In case \( S \) is the \hyperref[thm:boolean_algebra_of_subsets/complement]{complement} of a \hyperref[def:semiring_ideal/prime]{prime ideal}, we denote the localization by \( R_P \) (or \( R_p \) if \( P = \braket{ p } \)).

  The image under \( \iota \) of every element \( s \) of \( S \) is invertible in \( S^{-1} R \), and we call the inverse \( \ifrac {1_R} s \) the \term{reciprocal} of \( s \).

  This construction is very similar to the \hyperref[def:monoid_grothendieck_completion]{Grothendieck completion} of a monoid or semiring, although with notable differences --- the set \( S \) may be a strict subset of \( R \), and addition in the Grothendieck completion corresponds to multiplication in the localization, while addition in the completion has no analogy.
\end{definition}
\begin{defproof}
  The proof that \( {\sim} \) is an equivalence relation is the same as in \fullref{def:monoid_grothendieck_completion}.

  We will show that both operations are well-defined. Let \( u ab' = u a'b \), meaning that \( (a, b) \sim (a', b') \) and hence \( \ifrac a b = \ifrac {a'} {b'} \), and let \( v cd' = v c'd \).

  For addition, we have
  \begin{align*}
    u v (ad + bc) b' d'
    &=
    v dd' (u ab') + u bb' (v cd')
    = \\ &=
    v dd' (u a'b) + u bb' (v c'd)
    = \\ &=
    u v (a'd' + b'c') b d,
  \end{align*}
  hence \( (ad + bc, bd) \sim (a'd' + b'c', b'd') \).

  The proof for correctness of multiplication is the same as the proof of correctness of addition in \fullref{def:monoid_grothendieck_completion}.
\end{defproof}

\begin{proposition}\label{thm:ring_localization_universal_property}\mcite[431]{Knapp2016BasicAlgebra}
  The \hyperref[def:ring_localization]{localization} of \( R \) by \( S \) satisfies the following \hyperref[rem:universal_mapping_property]{universal mapping property}:
  \begin{displayquote}
    For every commutative ring \( T \) and every ring homomorphism \( \varphi: R \to T \) such that \( \varphi(s) \) is invertible in \( T \) for every \( s \in S \), \( \varphi \) \hyperref[def:factors_through]{uniquely factors through} \( S^{-1} R \). That is, there exists a unique ring homomorphism \( \widetilde{\varphi}: S^{-1} R \to T \) such that the following diagram commutes:
    \begin{equation}\label{eq:thm:ring_localization_universal_property/diagram}
      \begin{aligned}
        \includegraphics[page=1]{output/thm__ring_localization_universal_property.pdf}
      \end{aligned}
    \end{equation}
  \end{displayquote}
\end{proposition}
\begin{proof}
  The condition suggests the definition
  \begin{equation*}
    \widetilde{\varphi}\parens*{ \frac r s } \coloneqq \varphi(r) \varphi(s)^{-1}.
  \end{equation*}
\end{proof}

\begin{example}\label{ex:def:ring_localization}
  We list several examples of \hyperref[def:ring/commutative]{commutative ring} \hyperref[def:ring_localization]{localization}.

  \begin{thmenum}
    \thmitem{ex:def:ring_localization/zero} If \( S \) contains \( 0_R \), then \( S^{-1} R \) is the trivial ring.

    \thmitem{ex:def:ring_localization/powers_of_two} The localization \( S^{-1} \BbbZ \) by the set \( S \coloneqq \set{ 2^n \given n \geq 0 } \) is (a ring isomorphic to) the rational numbers with denominators that are powers of two. This is an example of a multiplicative set that is not the complement of a prime ideal.

    This ring is isomorphic to the ring \( \BbbZ[\ifrac 1 2] \) obtained by \hyperref[rem:adjoining_via_polynomials]{adjoining} the rational number \( \ifrac 1 2 \) to \( \BbbZ \).

    \thmitem{ex:def:ring_localization/prime_number} Let \( p \) be a \hyperref[def:prime_number]{prime number}. The localization \( S^{-1} \BbbZ \) by \( S \coloneqq \BbbZ \setminus \braket{ p } \) is (a ring isomorphic to) the rational numbers with denominators coprime to \( p \).

    For \( p = 2 \), this localization consists of rational numbers whose denominator is an odd number.
  \end{thmenum}
\end{example}

\begin{proposition}\label{thm:def:ring_localization/properties}
  \hyperref[def:ring_localization]{Ring localization} has the following basic properties:

  \begin{thmenum}
    \thmitem{thm:def:ring_localization/properties/image_of_ideal}\mcite[432]{Knapp2016BasicAlgebra} Localization preserves \hyperref[def:semiring_ideal]{ideals}. More precisely, given a commutative ring \( R \), a multiplicative set \( S \) and an ideal \( I \), the set
    \begin{equation*}
      S^{-1} I \coloneqq \set*{ \frac r s \given* r \in I \T{and} s \in S }
    \end{equation*}
    is an ideal of the localization \( S^{-1} R \).

    \thmitem{thm:def:ring_localization/properties/injective_inclusion} The canonical inclusion \( \iota: R \to S^{-1} R \) is injective if and only if \( S \) contains no zero divisors.

    \thmitem{thm:def:ring_localization/properties/prime_ideal}\mcite[exer. 4.2a)]{КоцевСидеров2016} The localization \( R_P \) by a \hyperref[def:semiring_ideal/prime]{prime ideal} \( P \) has a unique maximal ideal \( S^{-1} P \) (here \( S \coloneqq R \setminus P \)).
  \end{thmenum}
\end{proposition}
\begin{proof}
  \SubProofOf{thm:def:ring_localization/properties/image_of_ideal} Trivial since \( S \) is closed under multiplication.

  \SubProofOf{thm:def:ring_localization/properties/injective_inclusion} Let \( sr = 0 \) for \( s \in S \). Then \( \iota(s) = \ifrac s {1_R} \) is invertible in \( S^{-1} R \) and hence
  \begin{equation*}
    \frac {0_R} {1_R}
    =
    \frac {sr} {1_R}
    =
    \frac {1_R} s \cdot \frac {sr} {1_R}
    =
    \frac r {1_R}.
  \end{equation*}

  Hence, \( \iota(r) = \iota(0_R) \).

  It follows that \( \iota \) is injective if and only if \( S \) contains no zero divisors.

  \SubProofOf{thm:def:ring_localization/properties/prime_ideal} In the localization \( R_P \) be a prime ideal, all members of \( P \) become invertible. Hence, a maximal ideal cannot contain members of \( P \). By \fullref{thm:def:ring_localization/properties/image_of_ideal}, \( S^{-1} P \) is an ideal, therefore it must be the largest proper ideal.
\end{proof}
