\documentclass{classes/notebook}

% Bibliography
\addbibresource{bibliography/books.bib}
\addbibresource{bibliography/papers.bib}
\addbibresource{bibliography/articles.bib}
\addbibresource{bibliography/nlab.bib}
\addbibresource{bibliography/plato.bib}
\addbibresource{bibliography/proofwiki.bib}
\addbibresource{bibliography/mathcounterexamples.bib}
\addbibresource{bibliography/mathse.bib}
\addbibresource{bibliography/mathof.bib}
\addbibresource{bibliography/social.bib}
\addbibresource{bibliography/misc.bib}

% Document
\title{Notebook}
\subtitle{\url{https://github.com/v--/notebook}}
\author{Ianis Vasilev}

% Failsafe means "use defauls if no file is found", see common/git_commit_info.sty
\GitCommitInfoReadFailsafe {git-commit-info}
{
  \date
  {
    \normalsize
    \textbf{Date:} \GitCommitInfoDate \\
    \textbf{Commit:} \texttt{\GitCommitInfoHash}
  }
}

\begin{document}
  \maketitle

  \begin{abstract}
    This ever-expanding document started as a set of study notes and exercises and gradually outgrew itself to become more encyclopedic. Having all these notes in one place is quite helpful for both expressing my own thoughts clearly and for later reference. It is also helpful for tracking connections between seemingly unrelated concepts --- the entire document is hyperlinked. Even though I aim at understanding every concept in the way it is meant to be used, most concepts are presented insomuch as they are relevant to my previous experience.

    Since these are study notes, they will naturally have a lot of errors, so read them with caution. The document claims no referential nor pedagogical value. Everything is written at the level of abstraction I am comfortable with. Furthermore, some sections in the document are much more polished than others. Feel free to contact me if something in this document happens to distress you.

    I tried putting citations on as many things as possible. The citations themselves are usually put in the left margin. If there is no citation on a definition or theorem, that means that I have either recalled it from memory or discovered it on my own. I try to mark my own definitions to distinguish them from one taken from other sources. Many of the unoriginal definitions and theorems are restated. The simple proofs are mostly original, and the difficult ones are, often loosely, based on proofs from the places cited. Some proofs simply state "Trivial." in order to distinguish themselves from the proofs that are omitted for other reasons.

    Last but not least, some auxiliary programs are provided in the code subdirectory within the document's source.

    See the remarks in the introduction of \fullref{sec:mathematical_logic} and \fullref{sec:set_theory} for clarification regarding seemingly arbitrary conventions.
  \end{abstract}

  \newpage
  \addcontentsline{toc}{section}{Contents}
  \tableofcontents
  \newpage

  \chapter{Numbers}\label{ch:numbers}

Numbers are perhaps the most ubiquitous concept in mathematics. Even among non-mathematicians, division by zero or \( 0.\oline{9} = 1 \) seem to be a common topic of discourse, either as a joke or a sincere misunderstanding.

The aforementioned topics were studied extensively by mathematicians and have simple justifications from the point of view of abstract mathematics:
\begin{itemize}
  \item We may want to somehow define division by zero in the \hyperref[def:real_numbers]{field \( \BbbR \) of real numbers}, however that would introduce \hyperref[def:divisibility/zero]{zero divisors} and hence deprive \( \BbbR \) of being an \hyperref[def:integral_domain]{integral domain}. The cancellative property of multiplication would not hold, and hence \( xy = zy \) would not imply that \( x = z \) when \( y \neq 0 \).

  Our familiar arithmetic of real numbers heavily relies on the cancellative property, therefore we simply disallow division by zero.

  \item The set \( \BbbR \) of real numbers is a uniform space and thus every \hyperref[def:fundamental_net]{fundamental sequence} is convergent by \fullref{thm:cauchys_net_convergence_criterion}. Furthermore, since \( \BbbR \) is also a \hyperref[def:separation_axioms/T2]{Hausdorff} space, by \fullref{thm:t2_iff_singleton_limits}, every fundamental sequence has a unique limit.

  Now consider the following two fundamental sequences:
  \begin{align*}
    &1, 1, 1, 1, \ldots \\
    &0, 0.9, 0.99, \ldots
  \end{align*}

  Their absolute difference
  \begin{equation*}
    1, 0.1, 0.01, \ldots
  \end{equation*}
  converges to \( 0 \).

  Therefore, the two original sequences converge to the same real number, namely \( 1 \).

  This example is generalized in \fullref{ex:def:real_number_radix_expansion/geometric}.
\end{itemize}

Unfortunately, a formal study of numbers also leads to artifacts such as the nonstandard natural numbers discussed in \fullref{rem:standard_models_of_arithmetic}.

We will build and describe from the perspective of \fullref{ch:order_theory} and \fullref{ch:ring_theory} the following:
\begin{itemize}
  \item \Fullref{sec:natural_numbers}, built upon \fullref{ch:mathematical_logic}.
  \item \Fullref{sec:integers}.
  \item \Fullref{sec:rational_numbers}.
  \item \Fullref{sec:real_numbers}.
  \item \Fullref{sec:complex_numbers}.
\end{itemize}

Additionally, we will describe syntactic presentations of numbers in \fullref{sec:positional_number_systems}.

  \subsection{Natural numbers}\label{subsec:natural_numbers}

\paragraph{Peano arithmetic}

\begin{definition}\label{def:peano_arithmetic}\mcite[sec. 4.1]{Hinman2005Logic}
  \term{Peano arithmetic} (commonly abbreviated as \logic{PA}) is a \hyperref[def:first_order_theory]{theory} in \hyperref[subsec:first_order_logic]{first-order predicate logic} for describing \hyperref[def:natural_numbers]{natural numbers} and their operations. It can also be formulated in \hyperref[rem:higher_order_logic]{second-order logic} or entirely within \hyperref[sec:set_theory]{set theory} (especially considering that we are working inside an ambient \hyperref[rem:standard_model_of_set_theory]{standard} \hyperref[rem:transitive_model_of_set_theory]{transitive} model of \hyperref[def:axiom_of_universes]{\logic{ZFC+U}}), however in this monograph we try to state, if possible, the first-order logic formulation of a theory.

  Peano's original axioms from \cite[1]{Peano1889PA} used sets as fundamental notions. We prefer using first-order logic, as it is done by Peter Hinman in \incite[sec. 4.1]{Hinman2005Logic}. This allows us to use isolate concepts related to natural numbers that do not depend on sets. We do work with a model of Peano arithmetic --- see \fullref{def:natural_numbers} --- and we distinguish statements about that model and statements about the logical theory presented here.

  The \hyperref[def:first_order_language]{language} of the theory consists of
  \begin{thmenum}[series=def:peano_arithmetic]
    \thmitem{def:peano_arithmetic/zero} A constant \( 0 \) for representing \term{zero}. We can alternatively require a constant for \( 1 \), but this would lead to worse metamathematical properties as discussed in \fullref{rem:peano_arithmetic_zero}.

    \thmitem{def:peano_arithmetic/succ} A unary \hyperref[def:first_order_language/fun]{functional symbol} \( s \), called the \term{successor operation}.

    The successor function is only a technicality used for establishing basic properties and for defining addition and multiplication, both in this subsection and in \fullref{def:omega_operations}.

    We will only use the abstract successor operation prior to proving the familiar properties of addition and multiplication, although we will later use the \hyperref[def:ordinal_successor]{ordinal successor operation} for building a model of \logic{PA} --- see \fullref{thm:omega_is_model_of_pa}.

    \thmitem{def:peano_arithmetic/plus} An \hyperref[rem:first_order_formula_conventions/infix]{infix} binary functional symbol \( + \) for denoting \term{addition}.

    See \fullref{thm:natural_number_addition_properties} for the algebraic properties of natural number addition.

    \thmitem{def:peano_arithmetic/mult} Another infix binary functional symbol \( \cdot \) for denoting \term{multiplication}. Outside the \hyperref[con:metalogic]{object language} we usually use juxtaposition instead.

    See \fullref{thm:natural_number_multiplication_properties} for the algebraic properties of natural number multiplication.

    In the unambiguous language defined in \fullref{ex:natural_number_arithmetic_grammar/schema}, right-hand side of axiom \eqref{eq:def:peano_arithmetic/PA7} should instead be parenthesized as \( ((\syny \cdot \synx) + \synx) \). However, we want to avoid excessive parentheses in formulas as per our convention \fullref{rem:propositional_formula_parentheses}. Hence, we prefer relying on the convention that, as with all \hyperref[def:semiring]{semirings}, multiplication has higher priority than addition. Furthermore, we often use juxtaposition for denoting multiplication.
  \end{thmenum}

  We impose the following base \hyperref[def:first_order_theory/axiomatized]{axioms}:
  \begin{thmenum}[resume=def:peano_arithmetic]
    \thmitem[def:peano_arithmetic/PA1]{PA1} The successor function is \hyperref[thm:function_invertibility_categorical/nonempty_left_invertible]{injective}. This can be stated as follows:
    \begin{equation}\label{eq:def:peano_arithmetic/PA1}\tag{\logic{PA1}}
      s(\synx) \syneq s(\syny) \synimplies \synx \syneq \syny.
    \end{equation}

    We use here the convention for implicit universal quantification described in \fullref{rem:mathematical_logic_conventions/quantification}.

    \thmitem[def:peano_arithmetic/PA2]{PA2} Zero is not the successor of any natural number. Symbolically,
    \begin{equation}\label{eq:def:peano_arithmetic/PA2}\tag{\logic{PA2}}
      \synneg \qexists \synx (s(\synx) \syneq 0).
    \end{equation}

    \thmitem[def:peano_arithmetic/PA3]{PA3} The \term{axiom schema of induction} roughly states that for a property to hold for all natural numbers it is sufficient for the following two conditions to be met:
    \begin{itemize}
      \item The property holds for \( 0 \).
      \item We can prove that is holds for any number by assuming that it holds for its predecessor.
    \end{itemize}

    See our proof of \fullref{thm:nonzero_natural_numbers_have_predecessors} for a detailed discussion.

    To describe this formally, we state that, for any \hyperref[def:first_order_syntax/formula_variables]{variables} \( \synx \) and \( \syny \) and any formula \( \varphi \) not containing \underline{\( \syny \)} as a \hyperref[def:first_order_syntax/formula_free_variables]{free variable}, the following is an axiom:
    \begin{equation}\label{eq:def:peano_arithmetic/PA3}\tag{\logic{PA3}}
      \parens[\Big]
        {
          \underbrace{\varphi[\synx \mapsto 0]}_{\T{base case}}
          \synwedge
          \qforall \syny \parens[\Big]
            {
              \overbrace
                {
                  \underbrace{ \varphi[\synx \mapsto \syny] }_{\mathclap{\substack{\T{inductive} \\ \T{hypothesis}}}}
                  \synimplies
                  \underbrace{ \varphi[\synx \mapsto s(\syny)] }_{\mathclap{\substack{\T{inductive step} \\ \T{conclusion}}}}
                }^{\T{inductive step}}
            }
        }
      \synimplies
      \underbrace{ \qforall \syny \varphi[\synx \mapsto \syny] }_{\T{conclusion}}.
    \end{equation}

    It is important to highlight that \( \varphi \) may have any set of free variables, as long as \( \syny \) is not among them. As explained in \fullref{rem:mathematical_logic_conventions/quantification}, we avoid excessive universal quantification. Of course, the axiom is only interesting if \( \synx \in \op*{Free}(\varphi) \). If \( \synz_1, \ldots, \synz_n \) are all the other free variables of \( \varphi \), then the \hyperref[def:universal_closure]{universal closure} of the corresponding axiom is
    \small
    \begin{equation}\label{eq:def:peano_arithmetic/PA3_quantified}\tag{\logic{PA3'}}
      \qforall {\synz_1} \cdots \qforall {\synz_n}
      \parens[\Bigg]
      {
        \parens[\Big]
          {
            \underbrace{\varphi[\synx \mapsto 0]}_{\T{base case}}
            \synwedge
            \qforall \syny \parens[\Big]
              {
                \overbrace
                  {
                    \underbrace{ \varphi[\synx \mapsto \syny] }_{\mathclap{\substack{\T{inductive} \\ \T{hypothesis}}}}
                    \synimplies
                    \underbrace{ \varphi[\synx \mapsto s(\syny)] }_{\mathclap{\substack{\T{inductive step} \\ \T{conclusion}}}}
                  }^{\T{inductive step}}
              }
          }
        \synimplies
        \underbrace{ \qforall \syny \varphi[\synx \mapsto \syny] }_{\T{conclusion}}
      }.
    \end{equation}
    \normalsize

    Thus, the axiom holds for any assignment for the variables \( \synz_1, \ldots, \synz_n \). We call these variables \term{parameters}. Parameters in axiom schemas are further discussed in \fullref{def:set_builder_notation} in relation to comprehension in set theory.

    See \fullref{con:induction} for a more detailed discussion of induction in general and \fullref{thm:omega_recursion} for the corresponding recursion principle.
  \end{thmenum}

  The theory we obtain without the binary operations and with only the axioms \eqref{eq:def:peano_arithmetic/PA1}-\eqref{eq:def:peano_arithmetic/PA3} is itself sometimes used in derivative forms, for example as \enquote{Peano systems} in \cite[70]{Enderton1977Sets}. The operations are defined inductively, however, and there is no way for us to formalize them within a first-order object theory without adding them to the language and theory itself.

  \begin{thmenum}[resume=def:peano_arithmetic]
    \thmitem[def:peano_arithmetic/PA4+5]{PA4+5} The next two axioms inductively define how addition is supposed to work:
    \begin{align}
      \synx + 0       &\syneq \synx           \label{eq:def:peano_arithmetic/PA4}\tag{\logic{PA4}} \\
      \synx + s(\syny) &\syneq s(\synx + \syny) \label{eq:def:peano_arithmetic/PA5}\tag{\logic{PA5}}
    \end{align}

    \thmitem[def:peano_arithmetic/PA6+7]{PA6+7} The final two axioms are for multiplication:
    \begin{align}
      \synx \cdot 0       &\syneq 0                    \label{eq:def:peano_arithmetic/PA6}\tag{\logic{PA6}} \\
      \synx \cdot s(\syny) &\syneq \synx \cdot \syny + \synx \label{eq:def:peano_arithmetic/PA7}\tag{\logic{PA7}}
    \end{align}
  \end{thmenum}
\end{definition}

\begin{remark}\label{rem:peano_arithmetic_zero}
  It is common to consider the first natural numbers to be \( 0 \). This is done by Peter Hinman in \incite[sec. 4.1]{Hinman2005Logic} and by \incite[71]{Enderton1977Sets}. Peano himself, however, considered \( 1 \) to be the first natural number - see \cite[1]{Peano1889PA}.

  Whether \( 0 \) is considered to be a natural number is a matter of convention. The operations defined via \eqref{eq:def:peano_arithmetic/PA4}-\eqref{eq:def:peano_arithmetic/PA7} can be modified to work if \( 1 \) was instead the first natural number.

  We make choose for \( \BbbN \) to start with \( 0 \), however we try not to refer to the set \( \BbbN \) of natural numbers and instead rely on the concepts \enquote{nonnegative} and \enquote{positive} integers formally defined in \fullref{def:integer_ordering}.
\end{remark}

\begin{definition}\label{def:natural_numbers}
  We define the set of \term[bg=естествени числа (\cite[371]{ГеновМиховскиМоллов1991Алгебра}), ru=натуральные числа (\cite[def. 11.1]{АлександровМаркушевичХинчинЭнциклопедия1951Том1})]{natural numbers} \( \BbbN \) as the \hyperref[thm:smallest_inductive_set_existence]{smallest inductive set} \( \omega \) with the \hyperref[def:first_order_structure/interpretation]{interpretation} described in \fullref{thm:omega_is_model_of_pa}.

  We do not depend on any particular properties of \( \omega \), but we use it because our construction of it is careful and purposely does not use natural numbers to avoid circularity. We are working in an ambient \hyperref[rem:standard_model_of_set_theory]{standard} \hyperref[rem:transitive_model_of_set_theory]{transitive} model of \hyperref[def:axiom_of_universes]{\logic{ZFC+U}} and hence we will conflate \( \BbbN \) with \( \omega \) as sets.

  We use the usual decimal notation
  \begin{align*}
    0 &\coloneqq \varnothing \\
    1 &\coloneqq \op{sc}(\varnothing) = \set{ \varnothing } \\
    2 &\coloneqq \op{sc}(\op{sc}(\varnothing)) = \set{ \varnothing, \set{ \varnothing } } \\
      &\vdots
  \end{align*}
  for numbers, formalized in \fullref{def:positional_number_system/decimal}, and continue to use the notation functional symbols from \fullref{def:peano_arithmetic}, however we now use them to denote the corresponding interpretations in the structure \( \BbbN \).
\end{definition}
\begin{comments}
  \item See \fullref{ex:natural_number_arithmetic_grammar/schema} for a simple \hyperref[def:formal_grammar/schema]{grammar schema} that produces numeric symbols in their decimal notation.
\end{comments}

\begin{remark}\label{rem:standard_models_of_arithmetic}
  At this point, we have two kinds of natural numbers:
  \begin{itemize}
    \item We have natural numbers within the metatheory. This is our mental model of the natural numbers, and it is used for distinguishing between \enquote{unary} functional symbols like \( s \) and \enquote{binary} functional symbols like \( + \). This is mostly used within logic itself.

    \item We have the set of natural numbers \( \BbbN \) defined in \fullref{def:natural_numbers}. These are the numbers which we have defined formally, whose properties we study and the numbers which we use in the entire monograph. The properties of \( \BbbN \) help us develop a better mental model, which in turn changes our perception of the natural numbers within the metatheory.
  \end{itemize}

  We want the two sets of natural numbers to coincide. This is important when talking about, for example, \hyperref[def:sequence]{sequences}. If a number in \( \BbbN \) is not a natural number within the metatheory, we say that it is \term{nonstandard}. The existence of nonstandard models is guaranteed by \fullref{thm:upward_lowenheim_skolem_theorem}. There cannot be numbers in the metatheory that are not in \( \BbbN \) because a model of \logic{PA} cannot have a finite domain and the natural numbers are the smallest metalogical infinite set.

  A model of \logic{PA} which contains precisely the numbers in the metatheory is called a \term{standard model}. For the purpose of this monograph, it is sufficient to accept the convention that \( \BbbN \) is a standard model of \logic{PA}.
\end{remark}

\paragraph{Algebraic structure of \( \BbbN \)}

\begin{proposition}\label{thm:nonzero_natural_numbers_have_predecessors}
  Every nonzero natural number has a unique predecessor. More precisely, zero has no predecessor and for any nonzero number \( n \) there exists a unique number \( m \) such that \( n = s(m) \). We will denote this predecessor by \( p(n) \).
\end{proposition}
\begin{proof}
  This proof is exemplar because it clearly demonstrates both the distinction between inductive and deductive reasoning and the role of the main three axioms.

  \begin{itemize}
    \item The axiom \eqref{eq:def:peano_arithmetic/PA1} states that the function \( s \) is injective. By the equivalences in \fullref{def:function_invertibility/injective}, its \hyperref[def:set_valued_map/inverse]{inverse set-valued map} is actually a \hyperref[def:set_valued_map/partial]{single-valued partial function}. Denote this inverse by \( p \).

    \item The axiom \eqref{eq:def:peano_arithmetic/PA2} states that the function \( s \) is not surjective. By the equivalences in \fullref{def:function_invertibility/surjective}, the inverse \( p \) is not a \hyperref[def:set_valued_map/partial]{total function}.
  \end{itemize}

  What we have shown up until this point in the proof is deductive --- we have restated the first two axioms of \logic{PA} and used some equivalent conditions that allowed us to deduce properties of the inverse function \( p \) of \( s \). We did all of this by following the precise rules of \hyperref[def:classical_logic]{classical logic} described formally in \fullref{subsec:natural_deduction}. This reasoning emulates \hyperref[inf:thm:axiomatic_derivation_as_natural_deduction/mp]{modus ponens}.

  Now we will show that every nonzero natural number has a predecessor, that is, that the function \( p \) is not defined only at \( 0 \). To highlight the logical structure of this proof, we will use \hyperref[def:first_order_natural_deduction_system]{first-order natural deduction} rather than work with the model \( \BbbN \) of \logic{PA}.

  Denote by \( \theta \) the formula
  \begin{equation*}
    \synx \syneq 0 \synvee \qexists \synz (\synx \syneq s(\synz)).
  \end{equation*}

  Clearly \( \synx \) is the only free variable in \( \theta \). We want to derive the formula \( \qforall \syny \theta[\synx \mapsto \syny] \) from the axioms of \logic{PA}.

  In this part of the proof we will use inductive reasoning. This will highlight that \eqref{eq:def:peano_arithmetic/PA3} is not an axiom schema about specifying properties, but rather about introducing a proof technique that does not hold for general \hyperref[def:first_order_theory]{logical theories}. We will not attempt to prove \( \qforall \syny \theta[\synx \mapsto \syny] \) directly. Instead, we will prove a more complicated formula that is easier to prove and then by one of the many induction principles, it will follow that our desired result holds.

  We can deduce the following theorem without premises:
  \begin{equation*}
    \begin{prooftree}
      \infer0[\eqref{eq:def:first_order_natural_deduction_system/equality/intro}]{ (\synx \syneq 0)[\synx \mapsto 0] }
      \infer1[\ref{inf:def:propositional_natural_deduction_systems/or/intro_left}]{ \theta[\synx \mapsto 0] }

      \infer0[\eqref{eq:def:first_order_natural_deduction_system/equality/intro}]{ (\synx \syneq s(\synz))[\synz \mapsto \syny, \synx \mapsto s(\syny)] }
      \infer1[\eqref{eq:def:first_order_natural_deduction_system/exists/intro}]{ \parens[\Big]{ \qexists \synz (\synx \syneq s(\synz) }[\synx \mapsto s(\syny)] }
      \infer1[\ref{inf:def:propositional_natural_deduction_systems/or/intro_right}]{ \theta[\synx \mapsto s(\syny)] }
      \infer1[\ref{inf:def:propositional_natural_deduction_systems/imp/intro}]{ \theta[\synx \mapsto \syny] \synimplies \theta[\synx \mapsto s(\syny)] }
      \infer1[\eqref{eq:def:first_order_natural_deduction_system/forall/intro}]{ \qforall \syny (\theta[\synx \mapsto \syny] \synimplies \theta[\synx \mapsto s(\syny)]) }

      \infer2[\ref{inf:def:propositional_natural_deduction_systems/and/intro}]{ \theta[\synx \mapsto 0] \synwedge \qforall \syny \parens[\Big] { \theta[\synx \mapsto \syny] \synimplies \theta[\synx \mapsto s(\syny)] } }
    \end{prooftree}
  \end{equation*}

  This is precisely the antecedent of the instance of \eqref{eq:def:peano_arithmetic/PA3} with \( \varphi = \theta \). By \fullref{thm:syntactic_deduction_theorem} we have
  \begin{equation*}
    \eqref{eq:def:peano_arithmetic/PA3} \vdash \qforall \syny \theta[\synx \mapsto \syny].
  \end{equation*}

  When interpreted in \( \BbbN \), this formula \( \qforall \syny \theta[\synx \mapsto \syny] \) simply states that every natural number is either zero or has a predecessor. The statement does not concern itself with uniqueness nor with whether \( 0 \) has a predecessor.

  But we have already shown uniqueness --- the predecessor function \( p \) is a partial single-valued function. And we have shown that \( p \) is not defined at zero. The last part of the proof shows \( p \) is defined for all nonzero values.

  We may choose to define \( p \) at zero by giving it a sentinel value. This is precisely the technique we use in \fullref{thm:function_invertibility_categorical/nonempty_left_invertible} to show that \( s \) has a left inverse if it is injective.

  We can use only \eqref{eq:def:peano_arithmetic/PA1} to show that \( s \) is injective and then pick \( p \) to be any of its left inverses. We also want \( p \) to be as close as possible to a right inverse, however. The latter, as we have seen, is more tricky.
\end{proof}

\begin{proposition}\label{thm:natural_number_addition_properties}
  The \hyperref[def:natural_numbers]{natural numbers} \( \BbbN \) with \hyperref[def:peano_arithmetic/plus]{addition} form a \hyperref[def:binary_operation/cancellative]{cancellative} \hyperref[def:binary_operation/commutative]{commutative} \hyperref[def:zerosumfree]{zerosumfree} \hyperref[def:monoid]{monoid} with \( 0 \) as the neutral element.
\end{proposition}
\begin{comments}
  \item This generalizes to \fullref{thm:cardinal_addition_algebraic_properties} and \fullref{thm:cardinal_addition_algebraic_properties}.
\end{comments}
\begin{proof}
  \SubProofOf[def:binary_operation/commutative]{commutativity} Consider the sum \( n + m \). We use induction on \( m \) to prove its commutativity.
  \begin{itemize}
    \item If \( m = 0 \), nested induction by \( n \) yields:
    \begin{itemize}
      \item If \( n = m = 0 \), clearly \( n + m = 0 + 0 = m + n \).
      \item If the inductive hypothesis holds for its predecessor \( p(n) \),
      \begin{balign*}
        n + m
        &=
        n + 0
        = \\ &=
        s(p(n)) + 0
        \reloset {\eqref{eq:def:peano_arithmetic/PA4}} = \\ &=
        s(p(n))
        \reloset {\eqref{eq:def:peano_arithmetic/PA4}} = \\ &=
        s(p(n) + 0)
        = \\ &=
        s(p(n) + m)
        \reloset {\T{ind.}} = \\ &=
        s(m + p(n))
        \reloset {\eqref{eq:def:peano_arithmetic/PA5}} = \\ &=
        m + s(p(n))
        =
        m + n.
      \end{balign*}
    \end{itemize}

    \item If \( m \neq 0 \) and if the inductive hypothesis holds for \( p(m) \), \eqref{eq:def:peano_arithmetic/PA1} yields that \( n + m = m + n \) if and only if \( n + p(m) = p(m) + n \). But the last equality is satisfied because of the inductive hypothesis, hence commutativity of \( n \) and \( m \) follows.
  \end{itemize}

  \SubProofOf[def:binary_operation/associative]{associativity} Fix natural numbers \( n \), \( m \), \( k \). We will prove associativity by induction on \( k \). If \( k = 0 \), we have
  \begin{equation*}
    (n + m) + 0
    \reloset {\eqref{eq:def:peano_arithmetic/PA4}} =
    n + m
    \reloset {\eqref{eq:def:peano_arithmetic/PA4}} =
    n + (m + 0).
  \end{equation*}

  If \( k \neq 0 \), the proof follows directly from \eqref{eq:def:peano_arithmetic/PA1} as in our proof of commutativity.

  \SubProofOf[def:monoid]{neutral element} We have \( n + 0 = n \) by \eqref{eq:def:peano_arithmetic/PA4} and \( 0 + n = n \) by commutativity.

  \SubProofOf[def:binary_operation/cancellative]{cancellation} Let \( n + k = m + k \). We will prove that \( n = m \) by induction. This is obvious for \( k = 0 \). For \( k \neq 0 \) we have
  \begin{equation*}
    n + s(p(k))
    =
    n + k
    =
    m + k
    =
    m + s(p(k)),
  \end{equation*}
  which by \eqref{eq:def:peano_arithmetic/PA5} is equivalent to \( s(n + p(k))) = s(m + p(k))) \).

  By \eqref{eq:def:peano_arithmetic/PA1}, we have \( n + p(k) = m + p(k) \). The inductive hypothesis implies that \( n = m \).

  \SubProof[def:zerosumfree]{Proof that \( \BbbN \) is zerosumfree} We will use induction on \( m \) in \( n + m = 0 \).
  \begin{itemize}
    \item If \( m = 0 \), then \( n + m = n \) by \eqref{eq:def:peano_arithmetic/PA4}, and hence \( n = 0 \).

    \item If \( m > 0 \), then \( n + m = s(n + p(m)) \) by \eqref{eq:def:peano_arithmetic/PA5}, and by \eqref{eq:def:peano_arithmetic/PA2}, \( n + m \neq 0 \).
  \end{itemize}

  Therefore, \( n + m = 0 \) if and only if \( n = m = 0 \).
\end{proof}

\begin{remark}\label{rem:natural_number_multiplication}
  \hyperref[con:additive_semigroup/multiplication]{Multiplication in commutative monoids} (i.e. monoid exponentiation) is defined in \fullref{def:monoid/exponentiation} for a natural number and a monoid member. It just to happens that, by \fullref{thm:natural_number_addition_properties}, the natural numbers are themselves a monoid. We cannot rely on \fullref{thm:semigroup_exponentiation_properties}, however, if we want to avoid circular definitions and proofs.

  Having multiplication as part of the signature of \hyperref[def:peano_arithmetic]{Peano arithmetic} allows us to avoid this circularity.
\end{remark}

\begin{proposition}\label{thm:natural_number_multiplication_properties}
  The \hyperref[def:natural_numbers]{natural numbers} \( \BbbN \) with \hyperref[def:peano_arithmetic/mult]{multiplication} form a \hyperref[def:binary_operation/commutative]{commutative} \hyperref[def:monoid]{monoid} with \( 1 \) as the neutral element.

  When combined with addition, the natural numbers become an \hyperref[def:entire_semiring]{entire} \hyperref[def:semiring/commutative]{commutative semiring}.
\end{proposition}
\begin{comments}
  \item This generalizes to \fullref{thm:ordinal_multiplication_algebraic_properties} and \fullref{thm:cardinal_multiplication_algebraic_properties}.
\end{comments}
\begin{proof}
  \SubProof{Proof that \( 1 \) is a \hyperref[def:monoid]{neutral element}} Multiplication by \( 1 \) on the right preserves any natural number:
  \begin{equation*}
     n \cdot 1
     \reloset{\eqref{eq:def:peano_arithmetic/PA7}} =
     n \cdot 0 + n
     \reloset{\eqref{eq:def:peano_arithmetic/PA6}} =
     0 + n
     =
     n.
  \end{equation*}

  Multiplication from the left is handled by induction. Indeed, the case \( n = 0 \) is trivial and for nonzero \( n \) we have
  \begin{equation*}
     1 \cdot n
     \reloset{\eqref{eq:def:peano_arithmetic/PA7}} =
     1 \cdot p(n) + 1
     \reloset{\eqref{eq:def:peano_arithmetic/PA6}} =
     p(n) + 1
     =
     n.
  \end{equation*}

  \SubProofOf[def:semiring/left_distributivity]{distributivity} We will prove that \( (n + m)k = n \cdot k + n \cdot k \) with induction on \( k \).

  If \( k = 0 \),
  \begin{equation*}
    (n + m) \cdot 0
    \reloset{\eqref{eq:def:peano_arithmetic/PA6}} =
    0
    \reloset{\eqref{eq:def:peano_arithmetic/PA4}} =
    0 + 0
    \reloset{\eqref{eq:def:peano_arithmetic/PA6}} =
    n \cdot 0 + m \cdot 0.
  \end{equation*}

  For all nonzero \( k \), if the inductive hypothesis holds for \( p(k) \), then
  \begin{balign*}
    (n + m) \cdot k
    &\reloset{\eqref{eq:def:peano_arithmetic/PA7}} =
    (n + m) + (n + m) \cdot p(k)
    \reloset {\T{ind.}} = \\ &=
    (n + m) + n \cdot p(k) + n \cdot p(k)
    = \\ &=
    (n + n \cdot p(k)) + (m + m \cdot p(k))
    \reloset{\eqref{eq:def:peano_arithmetic/PA7}} = \\ &=
    n \cdot k + m \cdot k.
  \end{balign*}

  \SubProofOf[def:binary_operation/associative]{associativity} With distributivity proven, associativity of multiplication follows by induction. Indeed,
  \begin{equation*}
    (n \cdot m) \cdot k = n \cdot (m \cdot k)
  \end{equation*}
  is trivially satisfied for \( k = 0 \) and for all nonzero \( k \), whenever the inductive hypothesis holds for all \( n, m \in \BbbN \), it follows that
  \begin{balign*}
    (n \cdot m) \cdot k
    &\reloset{\eqref{eq:def:peano_arithmetic/PA7}} =
    n \cdot m + (n \cdot m) \cdot p(k)
    \reloset {\T{ind.}} = \\ &=
    n \cdot m + n \cdot (m \cdot p(k))
    \reloset{\eqref{eq:def:semiring/left_distributivity}} = \\ &=
    n \cdot (m + m \cdot p(k))
    \reloset{\eqref{eq:def:peano_arithmetic/PA7}} = \\ &=
    n \cdot (m \cdot k).
  \end{balign*}

  \SubProofOf[def:binary_operation/commutative]{commutativity} By induction on \( m \) we prove
  \begin{equation*}
    n \cdot m = m \cdot n.
  \end{equation*}

  The base case is trivial for nonzero \( m \).

  If \( n \cdot p(m) = p(m) \cdot n \) for all \( n \in \BbbN \), then
  \begin{equation*}
    n \cdot m
    \reloset{\eqref{eq:def:peano_arithmetic/PA7}} =
    n + n \cdot p(m)
    \reloset {\T{ind.}} =
    n + p(m) \cdot n
    \reloset{\eqref{def:semiring/left_distributivity}} =
    (1 + p(m)) \cdot n
    \reloset{\eqref{eq:def:peano_arithmetic/PA5}} =
    m \cdot n.
  \end{equation*}

  \SubProof{Proof of no zero divisors} We will now show that \( \BbbN \) has no zero divisors.

  If \( m = 0 \), then \( n \cdot m = 0 \) by \eqref{eq:def:peano_arithmetic/PA6}. If \( n = 0 \), then by induction on \( m \) we can easily show that \( n \cdot m = n \cdot p(m) + n = 0 + 0 = 0 \).

  Conversely, let \( n \cdot m = 0 \). By induction on \( m \), either \( m = 0 \) or we have \( n \cdot m = n \cdot p(m) + n \), in which case both \( n \cdot p(m) \) and \( n \) are zero because \( \BbbN \) is zerosumfree. Thus, if we assume that \( m \neq 0 \), then we can conclude that \( n = 0 \).
\end{proof}

\begin{remark}\label{rem:natural_number_successor_via_addition}
  In \cite[1]{Peano1889PA}, Peano defined an \enquote{\( n \mapsto n + 1 \)} operation rather than a successor operation. It has since become common practice to instead define a \enquote{successor} operation, define addition and then show that the two are compatible:
  \begin{equation*}
    n + 1
    \reloset {\eqref{eq:def:peano_arithmetic/PA5}} =
    s(n + 0)
    \reloset {\eqref{eq:def:peano_arithmetic/PA4}} =
    s(n).
  \end{equation*}

  The predecessor operation then corresponds to integer subtraction by \( 1 \). We avoid subtraction in this subsection --- we are only interested in the fact that every nonzero natural number \( n \) has a predecessor \( m \) such that \( n = m + 1 \).

  It is dangerous to conflate \( n + 1 \) and \( s(n) \) until we have proved the familiar properties of addition. We have already done so in \fullref{thm:natural_number_addition_properties}, however, and will further avoid mentioning directly the operations \( s(n) \) and \( p(n) \).

  See \fullref{rem:ordinal_successor_via_addition} for the more general case of ordinal addition.
\end{remark}

\paragraph{Order structure of \( \BbbN \)}

\begin{definition}\label{def:natural_numbers_ordering}
  We can define the familiar order \( \leq \) on the natural numbers via addition as the \hyperref[con:predicate_formula]{predicate formula}
  \begin{equation}\label{eq:def:natural_numbers_ordering/predicate}
    \alpha \leq \beta \coloneqq \qexists \synx \parens[\Big]{ \alpha + \synx \syneq \beta }.
  \end{equation}

  We use the infix notation for convenience, however we do not assume that \( \leq \) is part of the language of \logic{PA} (as explained in \fullref{rem:first_order_formula_conventions/necessary_signature}).

  The following relation
  \begin{equation}\label{eq:def:natural_numbers_ordering/strict_predicate}
    \alpha < \beta \coloneqq \qexists {\synx \neq 0} \parens[\Big]{ \alpha + \synx \syneq \beta }.
  \end{equation}
  is then connected to \( \leq \) via \eqref{eq:def:preordered_set/compatibility_strict}.
\end{definition}
\begin{comments}
  \item For the specific model of \logic{PA} based on the smallest inductive set \( \omega \), the latter relation \( < \) is equivalent to \( \in \) as discussed in \fullref{rem:ordinal_definition}.
  \item We will show in \fullref{thm:natural_numbers_are_well_ordered} that \( \BbbN \) is \hyperref[def:totally_ordered_set]{total ordered} (even \hyperref[def:well_ordered_set]{well-ordered}) with \( \leq \) as the nonstrict order and \( < \) as the strict order.
\end{comments}
\begin{defproof}
  We will show that \( n < m \) if and only if \( n \leq m \) and \( n \neq m \).

  \SufficiencySubProof Suppose that \( n < m \). Then there exists some nonzero \( a \) such that \( n + a = m \). In particular, we have \( n \leq m \). If we suppose that \( n = m \), then \( n + a = m \) and since addition is cancellative, it would follow that \( a = 0 \), contradicting the assumption that \( a \) is nonzero.

  Therefore, \( n \leq m \) and \( n \neq m \).

  \NecessitySubProof Suppose that \( n \leq m \) and \( n \neq m \). Then there exists some \( a \) such that \( n + a = m \) If we suppose that \( a = 0 \), then we would obtain that \( n = m \), which would contradict our choice of \( n \) and \( m \).

  Therefore, \( n < m \).
\end{defproof}

\begin{proposition}\label{thm:natural_numbers_are_well_ordered}
  The natural numbers are \hyperref[def:well_ordered_set]{well-ordered} by the relation \( < \) defined by \eqref{eq:def:natural_numbers_ordering/strict_predicate}.

  Furthermore, \( \BbbN \) is an \hyperref[def:ordered_semiring]{ordered semiring}, that is, the nonstrict order \( \leq \) is compatible with addition and multiplication.
\end{proposition}
\begin{proof}
  As discussed in \fullref{def:well_ordered_set}, in order to show that \( < \) well-orders \( \BbbN \), we only need to show that \( < \) is \hyperref[def:binary_relation/transitive]{transitive}, satisfies \hyperref[def:binary_relation/trichotomy]{trichotomy} and does not allow an strictly descending sequence.

  \SubProofOf[def:binary_relation/transitive]{transitivity} Let \( n < m \) and \( m < k \). Then there exist nonzero numbers \( a \) such that \( n + a = m \) and \( b \) such that \( m + b = k \). Thus, \( n + a + b = k \), which demonstrates that \( n \leq k \). Furthermore, because \( \BbbN \) is zerosumfree, it also follows that \( a + b \neq 0 \).

  Therefore, \( n < k \).

  \SubProofOf[def:binary_relation/trichotomy]{trichotomy} Let \( n \) and \( m \) be natural numbers.

  We have already shown in \fullref{def:natural_numbers_ordering} that due to \eqref{eq:def:preordered_set/compatibility_strict} the equality \( n = m \) holds if and only if neither \( n < m \) nor \( n > m \).

  Aiming at a contradiction, suppose that both \( n < m \) and \( n > m \) hold. There must exist nonzero numbers \( a \) and \( b \) such that \( n + a = m \) and \( n = m + b \). Then
  \begin{equation*}
    n = m + b = (n + a) + b.
  \end{equation*}

  Since addition is cancellative, we have \( a + b = 0 \). Therefore, \( n = m \), which as we have shown is incompatible with neither \( n < m \) nor \( n > m \).

  Therefore, at most one of the three conditions \( n = m \), \( n < m \) or \( n > m \) holds.

  We will use induction on \( m \) to show that at least one of the conditions hold. If \( m = 0 \), then either \( n = 0 \) and \( m = n \) or \( n \neq 0 \) and \( m < n \). Now suppose that the inductive hypothesis holds for \( m \). We will show that it also holds for \( m + 1 \).
  \begin{itemize}
    \item If \( n = m \), then clearly \( n < m + 1 \).
    \item If \( n < m \), then since \( m < m + 1 \) by transitivity of \( < \), we have \( n < m + 1 \).
    \item If \( n > m \), then there exists some nonzero \( a \) such that \( n = m + a \). If \( a = 1 \), then \( n = m + 1 \). If \( a \) is neither \( 0 \) nor \( 1 \), then \( n > m + 1 \).
  \end{itemize}

  \SubProofOf[def:well_founded_relation]{well-foundedness} We will show by induction on \( n \) that an strictly descending sequence ending at \( n \) cannot exist.

  If \( n = 0 \), by \eqref{eq:def:peano_arithmetic/PA2} \( n \) has no predecessor and thus there cannot exist a natural number \( m \) such that \( m < n \).

  Now assume that the inductive hypothesis holds for \( n \) and suppose that there exists an strictly descending sequence ending in \( n + 1 \):
  \begin{equation}\label{eq:thm:natural_numbers_are_well_ordered/descending_chain}
    \cdots < k < m < n + 1.
  \end{equation}

  If \( m = n \), it follows that
  \begin{equation*}
    \cdots < k < n
  \end{equation*}
  is an strictly descending sequence ending in \( n \).

  If \( m < n \), then
  \begin{equation*}
    \cdots < k < m < n
  \end{equation*}
  is again an strictly descending sequence ending in \( n \).

  By the inductive hypothesis, a descending sequence ending at \( n \) cannot exist, therefore neither does \eqref{eq:thm:natural_numbers_are_well_ordered/descending_chain}.

  \SubProofOf[def:ordered_semigroup]{compatibility with addition} We will show that the nonstrict order is compatible with addition in \( \BbbN \). Let \( n \leq m \) and let \( k \) be an arbitrary natural number. Since \( n \leq m \), there exists a number \( a \) such that \( n + a = m \). Then
  \begin{equation*}
    m + k = (n + a) + k = (n + k) + a.
  \end{equation*}

  Therefore,
  \begin{equation*}
    n + k \leq m + k.
  \end{equation*}

  \SubProofOf[def:ordered_semiring]{compatibility with multiplication} If \( n \geq 0 \) and \( m \geq 0 \), then \( n \cdot m \geq 0 \) for the simple reason that all natural numbers are greater than or equal to zero.
\end{proof}

\begin{proposition}\label{thm:natural_numbers_confinal_equivalence}
  For a set of natural numbers, the following are equivalent:
  \begin{thmenum}
    \thmitem{thm:natural_numbers_confinal_equivalence/cofinal} It is \hyperref[def:cofinal_set]{cofinal}.
    \thmitem{thm:natural_numbers_confinal_equivalence/unbounded} It is \hyperref[def:extremal_points/bounds]{unbounded from above}.
    \thmitem{thm:natural_numbers_confinal_equivalence/infinite} It is \hyperref[def:set_finiteness]{infinite}.
  \end{thmenum}
\end{proposition}
\begin{proof}
  Let \( A \) be a set of nonnegative integers.

  \ImplicationSubProof{thm:natural_numbers_confinal_equivalence/cofinal}{thm:natural_numbers_confinal_equivalence/unbounded} \Fullref{thm:def:totally_ordered_set/cofinal_iff_unbounded} implies that, if a set of integers is cofinal, then it is unbounded from above.

  \ImplicationSubProof{thm:natural_numbers_confinal_equivalence/unbounded}{thm:natural_numbers_confinal_equivalence/infinite} If a set \( A \) of nonnegative integers is unbounded from above, we can recursively build a strictly increasing sequence of elements of \( A \). This sequence has infinitely many elements, hence so does \( A \).

  \ImplicationSubProof{thm:natural_numbers_confinal_equivalence/infinite}{thm:natural_numbers_confinal_equivalence/cofinal} If \( A \) is infinite, then, for any integer \( n \), \( A \) contains only finitely many elements smaller than \( n \). But \( A \) is infinite, hence it must contain at least one element greater than or equal to \( n \).
\end{proof}

\begin{proposition}\label{thm:natural_number_divisibility_lattice}
  The semiring \hyperref[def:natural_numbers]{\( \BbbN \)} of natural numbers is a \hyperref[def:extremal_points/bounds]{bounded lattice} with respect to \hyperref[def:divisibility]{semiring divisibility}. Explicitly:
  \begin{thmenum}
    \thmitem{thm:natural_number_divisibility_lattice/join} The \hyperref[def:lattice/join]{join} of \( n \) and \( m \) is their unique \hyperref[def:lcm]{least common multiple}, denoted by \( \lcm(n, m) \).

    \thmitem{thm:natural_number_divisibility_lattice/bottom} The \hyperref[def:extremal_points/top_and_bottom]{bottom element} is \( 1 \) since \( 1 \) divides every natural number.

    \thmitem{thm:natural_number_divisibility_lattice/meet} Dually, the \hyperref[def:lattice/meet]{meet} of \( n \) and \( m \) is their unique \hyperref[def:gcd]{greatest common divisor}, denoted by \( \gcd(n, m) \) and constructed via \fullref{alg:euclidean_algorithm}.

    \thmitem{thm:natural_number_divisibility_lattice/top} The \hyperref[def:extremal_points/top_and_bottom]{top element} is \( 0 \) since every natural number divides \( 0 \).
  \end{thmenum}

  Furthermore, divisibility is compatible with the standard ordering in the sense that, if \( m \) is nonzero, then \( n \mid m \) implies \( n \leq m \).

  \begin{figure}[!ht]
    \centering
    \includegraphics[page=1]{output/thm__natural_number_divisibility_order}
    \caption{A spatial \hyperref[def:hasse_diagram]{Hasse diagram} for a fragment of the \hyperref[thm:natural_number_divisibility_lattice]{natural number divisibility lattice}}
    \label{fig:thm:natural_number_divisibility_lattice/divisibility}
  \end{figure}

  \begin{figure}[!ht]
    \centering
    \includegraphics[page=2]{output/thm__natural_number_divisibility_order}
    \caption{A comparison of the \hyperref[thm:natural_number_divisibility_lattice]{divisibility lattice} of \( \BbbN \) and the \hyperref[thm:semiring_of_ideals/lattice]{lattice of ideals} of \( \BbbZ \).}
    \label{fig:thm:natural_number_divisibility_lattice/ideals}
  \end{figure}
\end{proposition}
\begin{proof}
  By \fullref{thm:semiring_divisibility_order}, divisibility is a preorder.

  \SubProofOf[def:binary_relation/antisymmetric]{antisymmetry} If \( n \mid m \) and \( m \mid n \), there exist numbers \( a \) and \( b \) such that \( n = ay \) and \( m = bx \). Then \( n = abx \), and we can cancel \( n \) to obtain that \( ab = 1 \). But \( 1 \) is the only invertible element in \( \BbbN \), hence \( a = b = 1 \), and thus \( n = m \).

  \SubProofOf[def:lattice]{lattice structure} By \fullref{alg:euclidean_algorithm}, every pair of integers has a positive greatest common divisor, and also a least common multiple.

  By \fullref{thm:def:semiring_ideal/division}, the lattice of principal ideals in \( \BbbN \) must be dual to it. Indeed, by \eqref{eq:def:bezout_domain/identity}, we have that
  \begin{equation*}
    \braket{n} + \braket{m} = \braket{ \gcd(n, m) },
  \end{equation*}
  and by \fullref{thm:def:semiring_ideal/division}, \( \braket{n} \cap \braket{m} \) contains the common multiples of \( n \) and \( m \), hence
  \begin{equation*}
    \braket{n} \cap \braket{m} = \braket{ \lcm(n, m) }.
  \end{equation*}

  \SubProof{Proof that the orders are compatible} If \( n \mid m \), then there exists a positive natural number \( a \) such that \( an = m \). We have
  \begin{equation*}
    an
    \reloset{\eqref{eq:def:peano_arithmetic/PA7}} =
    (a - 1)n + n
    =
    m.
  \end{equation*}

  Thus, by \eqref{eq:def:natural_numbers_ordering/predicate}, \( n \leq m \).
\end{proof}

  \section{Integers}\label{sec:integers}

\paragraph{Ring of integers}

\begin{definition}\label{def:integers}\mimprovised
  We define the \hyperref[def:ring]{ring} \( \BbbZ \) of \term[bg=цели числа (\cite[374]{ГеновМиховскиМоллов1991Алгебра}), ru=целые числа (\cite[def. 20.1]{АлександровМаркушевичХинчин1951ЭнциклопедияТом1})]{integers} as the \hyperref[thm:grothendieck_semiring_completion]{Grothendieck completion} of the \hyperref[def:semiring/commutative]{commutative semiring} \( \BbbN \) of zero-based \hyperref[def:natural_numbers]{natural numbers}.
\end{definition}
\begin{comments}
  \item We consider the following an indispensable part of \( \BbbZ \):
  \begin{itemize}
    \item The ordering defined in \fullref{def:integer_ordering}, which makes \( \BbbZ \) an \hyperref[def:ordered_semiring]{ordered (semi)ring}.
    \item The Euclidean domain structure discussed in \fullref{rem:integer_domain_chain}.
    \item The order topology discussed in \fullref{thm:order_topology_on_integers_is_discrete}.
  \end{itemize}
\end{comments}

\begin{proposition}\label{thm:integers_are_integral_domain}
  The ring of integers is \hyperref[def:entire_semiring]{entire}, i.e. an \hyperref[def:integral_domain]{integral domain}.
\end{proposition}
\begin{proof}
  Suppose that \( nm = 0 \).

  \begin{itemize}
    \item If \( n = \iota(a) \) and \( m = \iota(b) \), since \( \BbbN \) is entire and since \( \iota \) is a homomorphism, then \( nm = \iota(a) \cdot \iota(b) = \iota(ab) = \iota(0_\BbbN) \) implies that at least one of \( a \) or \( b \) is zero.
    \item If \( n = \iota(a) \) and \( m = -\iota(b) \), this reduces to the previous case since \( nm = 0 = n(-m) \).
    \item The other cases are similar.
  \end{itemize}

  Therefore, \( n = 0 \) or \( m = 0 \).
\end{proof}

\paragraph{Order of integers}

Rather than defining an order from first principles and then using the signum function from \fullref{def:totally_ordered_ring_signum} with its established properties, it will be more convenient for us to define the signum function, prove its properties, and then define the order itself.

\begin{lemma}\label{thm:integer_signum_lemma}
  Consider the canonical embedding \( \iota: \BbbN \to \BbbZ \). For every nonzero integer \( n \), there either exists a unique natural number \( a \) such that \( n = \iota(a) \), or a unique natural number \( a \) such that \( n = -\iota(a) \).
\end{lemma}
\begin{proof}
  \ExistenceSubProof Due to the trichotomy on natural numbers shown in \fullref{def:natural_numbers_ordering}, we have the following mutually exclusive possibilities:
  \begin{itemize}
    \item If \( a = b \), then \( x = [(a, a)] = [(0_\BbbN, 0_\BbbN)] \), hence \( x \) is zero.
    \item If \( a < b \), by \fullref{def:natural_numbers_ordering}, there exists some natural number \( c > 0 \) such that \( a + c = 0_\BbbN + b \). Then
    \begin{equation*}
      x = [(a, b)] = [(c, 0_\BbbN)] = \iota(c).
    \end{equation*}

    \item If \( a > b \), there exists some natural number \( d \) such that \( 0_\BbbN + a = b + d \). Then
    \begin{equation*}
      x = [(a, b)] = [(0_\BbbN, d)] = -[(d, 0_\BbbN)] = -\iota(d).
    \end{equation*}
  \end{itemize}

  \UniquenessSubProof If \( \iota(a) = \iota(b) \), then there exists some natural number \( u \) such that
  \begin{equation*}
     a + 0_\BbbN + u = 0_\BbbN + b + u.
  \end{equation*}

  By \fullref{thm:natural_number_addition_properties}, \( \BbbN \) is cancellative, so \( a = b \).

  \SubProof{Proof of exclusive conditions} Finally, suppose that \( n = \iota(a) = -\iota(b) \). Then \( a + b = 0_\BbbN \). Since \( \BbbN \) is \hyperref[def:zerosumfree]{zerosumfree} by \fullref{thm:natural_number_addition_properties}, it follows that \( a = b = 0_\BbbN \), and hence \( n \) is zero.

  Therefore, if \( n \) is nonzero, either \( n = \iota(a) \) for some \( a \) or \( n = -\iota(b) \) for some \( b \).
\end{proof}

\begin{definition}\label{def:integer_signum}\mimprovised
  Consider the canonical embedding \( \iota: \BbbN \to \BbbZ \). Define the \term{signum} function
  \begin{equation*}
    \begin{aligned}
      &\sgn: \BbbZ \to \BbbZ \\
      &\sgn(n) \coloneqq \begin{cases}
        0,  &n = \iota(0_\BbbN), \\
        1,  &n = \iota(a) \T{for some nonzero natural number} a, \\
        -1, &n = -\iota(a) \T{for some nonzero natural number} a.
      \end{cases}
    \end{aligned}
  \end{equation*}

  This is a well-defined \hyperref[def:set_valued_map/partial]{total function} due to \fullref{thm:integer_signum_lemma}.
\end{definition}
\begin{comments}
  \item We will later show that this matches the signum function on \( \BbbZ \) defined in \fullref{def:totally_ordered_ring_signum}.
\end{comments}

\begin{proposition}\label{thm:def:integer_signum}
  \hyperref[def:integer_signum]{Integer signs} have the following basic properties:
  \begin{thmenum}
    \thmitem{thm:def:integer_signum/sum} If two integers have the same sign, their sum also has the same sign.

    This is an analog of \fullref{thm:def:ordered_semiring/sum}.

    \thmitem{thm:def:integer_signum/product} \( \sgn(nm) = \sgn(n) \cdot \sgn(m) \).

    This is an analog of \fullref{thm:def:totally_ordered_ring_signum}.

    \thmitem{thm:def:integer_signum/inverse} \( \sgn(-n) = -\sgn(n) \).

    This is an analog of \fullref{thm:ordered_ring_inversion}.
  \end{thmenum}
\end{proposition}
\begin{proof}
  \SubProofOf{thm:def:integer_signum/sum} Fix two integers \( n \) and \( m \).

  \begin{itemize}
    \item If \( n \) and \( m \) are both zero, their sum is also zero.
    \item If \( n = \iota(a) \) and \( m = \iota(b) \), then \( n + m = \iota(a + b) \), so \( \sgn(n + m) \) is also \( 1 \).
    \item If \( n = -\iota(a) \) and \( m = -\iota(b) \) are both negative, then \( n + m = -\iota(a + b) \), so \( \sgn(n + m) \) is also \( -1 \).
  \end{itemize}

  \SubProofOf{thm:def:integer_signum/product} Fix two integers \( n \) and \( m \).

  \begin{itemize}
    \item If either \( n \) or \( m \) is zero, the product \( nm \) is again zero, so
    \begin{equation*}
      \sgn(nm) = \sgn(n) \cdot \sgn(m) = 0.
    \end{equation*}

    \item If \( n = \iota(a) \) and \( m = \iota(b) \) are both nonzero, then \( nm = \iota(ab) \) and \( \sgn(nm) \geq 0 \). Furthermore, \( nm \) cannot be zero because \( \BbbZ \) is \hyperref[def:entire_semiring]{entire}. Hence,
    \begin{equation*}
      \sgn(nm) = \sgn(n) \cdot \sgn(m) = 1.
    \end{equation*}

    \item If \( n = \iota(a) \) and \( m = -\iota(b) \) are both nonzero, then \( n(-m) = \iota(ab) \). Furthermore, \( nm = -n(-m) = -\iota(ab) \), and hence \( \sgn(nm) \leq 0 \). Again, \( nm \) cannot be zero because \( \BbbZ \) is entire. Then
    \begin{equation*}
      \sgn(nm) = \underbrace{\sgn(n)}_1 \cdot \underbrace{\sgn(m)}_{-1} = -1.
    \end{equation*}

    The case where \( \sgn(n) = -1 \) and \( \sgn(m) = 1 \) follows from this one due to commutativity.

    \item If \( n = -\iota(a) \) and \( m = -\iota(b) \) are both nonzero, then \( nm = (-n)(-m) = \iota(ab) \) and \( \sgn(nm) \geq 0 \). Thus,
    \begin{equation*}
      \sgn(nm) = \underbrace{\sgn(n)}_{-1} \cdot \underbrace{\sgn(m)}_{-1} = 1.
    \end{equation*}
  \end{itemize}

  \SubProofOf{thm:def:integer_signum/inverse} From \fullref{thm:def:integer_signum/product} it follows that \( \sgn(-n) = \sgn(-1) \cdot \sgn(n) = -\sgn(n) \).
\end{proof}

\begin{definition}\label{def:integer_ordering}\mimprovised
  We extend the \hyperref[def:natural_numbers_ordering]{natural number ordering} \( \leq_\BbbN \) to the \hyperref[def:integers]{integers} \( \BbbZ \) via the following truth table:
  \begin{equation*}
    \begin{array}{c !{\thickspace} c !{\thickspace} c}
      \toprule
      \sgn(n)    & \sgn(m)    & n \leq m         \\
      \midrule
      0 \T{or} 1 & 0 \T{or} 1 & n \leq_\BbbN m   \\
      0 \T{or} 1 & -1         & F                \\
      -1         & 0 \T{or} 1 & T                \\
      -1         & -1         & -m \leq_\BbbN -n \\
      \bottomrule
    \end{array}
  \end{equation*}
\end{definition}
\begin{comments}
  \item We will now use terms like \enquote{positive} and \enquote{negative} from \fullref{def:ordered_semiring}, although we are yet to prove that this ordering is compatible with the algebraic structure.
\end{comments}

\begin{proposition}\label{thm:def:integer_ordering}
  \hyperref[def:integer_ordering]{Integer ordering} has the following basic properties:
  \begin{thmenum}
    \thmitem{thm:def:integer_ordering/positive} \( n < 0 \) if and only if \( \sgn(n) = 1 \).

    This partially ensures compatibility with the signum function from \fullref{def:totally_ordered_ring_signum}.

    \thmitem{thm:def:integer_ordering/negative} \( 0 < n \) if and only if \( \sgn(n) = -1 \).

    This also ensures compatibility with the signum function from \fullref{def:totally_ordered_ring_signum}.

    \thmitem{thm:def:integer_ordering/inverse} \( n \leq m \) if and only if \( -m \leq -n \).

    This is an analog of \fullref{thm:ordered_ring_order_inversion}

    \thmitem{thm:def:integer_ordering/total} It is a \hyperref[def:totally_ordered_set]{total order}.

    \thmitem{thm:def:integer_ordering/ordered_ring} It makes \( \BbbZ \) an \hyperref[def:ordered_semiring]{ordered (semi)ring}.
  \end{thmenum}
\end{proposition}
\begin{proof}
  \SubProofOf{thm:def:integer_ordering/positive} Trivial.
  \SubProofOf{thm:def:integer_ordering/negative} Trivial.

  \SubProofOf{thm:def:integer_ordering/inverse} Let \( n \leq m \).
  \begin{itemize}
    \item If \( n \) and \( m \) are both either nonnegative or negative, then \( -m \leq -n \) by definition.
    \item If \( n \) is negative and \( m \) is not, then by \fullref{thm:def:integer_signum/inverse} \( -m \) is negative and \( -n \) is not. Hence, \( -m \leq -n \).
  \end{itemize}

  The converse direction of the proof is identical.

  \SubProofOf{thm:def:integer_ordering/total} The order \( \leq \) is clearly defined for every pair of integers, so it remains to show that it is a partial order.

  \SubProofOf*[def:binary_relation/reflexive]{reflexivity} If \( n \) is positive, then \( n \leq n \) since \( n \leq_\BbbN n \). Otherwise, \( n \leq n \) since \( -n \leq_\BbbN -n \).

  \SubProofOf*[def:binary_relation/antisymmetric]{antisymmetry} Suppose that \( n \leq m \) and \( m \leq n \).
  \begin{itemize}
    \item If \( n \) and \( m \) are both nonnegative, then \( n = m \) from the antisymmetry of the natural number ordering.
    \item If \( n \) and \( m \) are both negative, then \( -n = -m \) from the antisymmetry of the natural number ordering.
    \item If \( n \) is nonnegative and \( m \) is negative, we have \( m \leq n \) but not \( n \leq m \). This contradicts our assumption.
    \item If \( n \) is negative and \( m \) is nonnegative, we have \( n \leq m \) but not \( m \leq n \). This contradicts our assumption.
  \end{itemize}

  \SubProofOf*[def:binary_relation/transitive]{transitivity} Suppose that \( n \leq m \) and \( m \leq k \).
  \begin{itemize}
    \item If \( n \), \( m \) and \( k \) are nonnegative, then \( n \leq k \) from the transitivity of the natural number ordering.
    \item If \( n \), \( m \) and \( k \) are negative, then \( -k \leq -n \) and, by \fullref{thm:def:integer_ordering/inverse}, \( n \leq k \).
    \item If \( n \) is negative and \( k \) is nonnegative, then \( n \leq k \) by definition.
  \end{itemize}

  \SubProofOf{thm:def:integer_ordering/ordered_ring} We will show that \( \BbbZ \) is an ordered ring:
  \SubProofOf*[def:ordered_semigroup]{addition compatibility} If \( n \leq m \), we will prove that \( n + k \leq m + k \) for any integer \( k \).

  We have \( n + (m - n) = m \), and hence \( n + (m - n) + k = m + k \). \( n \leq m \) implies that \( m - n \) is nonnegative, hence can use induction on it to show that \( n + k \leq m + k \). More precisely, we will use induction on \( s \) to show that \( n \leq n + s \) implies that \( n + k \leq n + s + k \).

  \begin{itemize}
    \item The case \( s = 0 \) is trivial.
    \item If \( n + k \leq n + s + k \), then we have the following cases:
    \begin{itemize}
      \item If \( n + s + k \) is nonnegative, then \( n + s + k \leq n + (s + 1) + k \) by \fullref{thm:natural_numbers_are_well_ordered}, even without the inductive hypothesis.
      \item If \( n + s + k \) is negative, then \( -n - s - k \) is positive, and, again by \fullref{thm:natural_numbers_are_well_ordered},
      \begin{equation*}
        -n - (s + 1) - k \leq -n - (s + 1) - k + 1 = -n - s - k.
      \end{equation*}

      \Fullref{thm:def:integer_ordering/inverse} then implies that
      \begin{equation*}
        n + s + k \leq n + (s + 1) + k.
      \end{equation*}
    \end{itemize}

    In both cases,
    \begin{equation*}
      n + k \leq n + s + k \leq n + (s + 1) + k.
    \end{equation*}
  \end{itemize}

  \SubProofOf*[def:ordered_semiring]{multiplication compatibility} If \( n \leq m \), we will prove that \( nk \leq mk \) for any nonnegative integer \( k \).

  \begin{itemize}
    \item If both \( n \) and \( m \) are nonnegative, the statement follows from \fullref{thm:natural_numbers_are_well_ordered}.
    \item If both \( n \) and \( m \) are negative, \fullref{thm:def:integer_ordering/inverse} implies that \( -m \leq -n \), then the previous case implies that \( (-m)k \leq (-n)k \), and finally \( nk \leq mk \).
    \item If \( n \) is negative and \( m \) is nonnegative, we have two cases:
      \begin{itemize}
        \item The case \( k = 0 \) is trivial due to the absorption property \eqref{eq:def:semiring/absorption}.

        \item If \( k > 0 \), \fullref{thm:def:integer_signum/product} implies that
        \begin{equation*}
          \sgn(nk) = \sgn(n) \sgn(k) = \sgn(n) < \sgn(m) = \sgn(m) \sgn(k) = \sgn(mk),
        \end{equation*}
        hence \( nk \) is negative and \( mk \) is nonnegative.
    \end{itemize}
  \end{itemize}
\end{proof}

\paragraph{Topology of integers}

\begin{proposition}\label{thm:order_topology_on_integers_is_discrete}
  The \hyperref[def:order_topology]{order topology} on the set of integers is \hyperref[def:discrete_topology]{discrete}.
\end{proposition}
\begin{proof}
  The base \eqref{eq:def:order_topology/base} contains, among others, intervals of the form \( \set{ n } = (n - 1, n + 1) \) for any integer \( n \). The union of such sets give all possible subsets of \( \BbbZ \).
\end{proof}

\begin{corollary}\label{thm:order_topology_on_natural_numbers_is_discrete}
  The \hyperref[def:order_topology]{order topology} on the set of natural numbers is \hyperref[def:discrete_topology]{discrete}.
\end{corollary}
\begin{proof}
  Follows from the same argument as our proof in \fullref{thm:order_topology_on_integers_is_discrete} by noting that the subbase \eqref{eq:def:order_topology/subbase} contains the interval \( \set{ 0 } = (-\infty, 1) \).
\end{proof}

\paragraph{Integer division}

\begin{definition}\label{def:integer_absolute_value}\mimprovised
  We will define absolute values for general complex numbers in \fullref{def:complex_absolute_value}, but we will find absolute values useful for proving some structural properties of the ring of integers. So, we will introduce the auxiliary definition
  \begin{equation*}
    \begin{aligned}
      &\abs{\anon}_\BbbZ: \BbbZ \to \BbbZ \\
      &\abs{n}_\BbbZ \coloneqq \sgn(n) \cdot n.
    \end{aligned}
  \end{equation*}
\end{definition}

\begin{remark}\label{rem:integer_division_uniqueness}
  Fix two integers \( n \) and \( m \), and assume that \( m \) is nonzero. There are multiple ways to define \hyperref[def:euclidean_domain]{Euclidean division} of \( n \) by \( m \), that is, an algorithm producing integers \( q \) and \( r \) such that
  \begin{equation*}
    n = mq + r
  \end{equation*}
  and
  \begin{equation}\label{eq:rem:integer_division_uniqueness/rem_inequality}
    0 \leq \abs{r}_\BbbZ < \abs{m}_\BbbZ.
  \end{equation}

  We have chosen the \hyperref[def:integer_absolute_value]{integer absolute value} as a Euclidean degree functions. We will not consider other possibilities.

  Because of the restriction \eqref{eq:rem:integer_division_uniqueness/rem_inequality} on \( r \), we have two possibilities: \( r \geq 0 \) and \( r \leq 0 \).

  \begin{thmenum}
    \thmitem{rem:integer_division_uniqueness/max} \incite[23]{Jacobson1985AlgebraPart1}, \incite[thm. A-2.2]{Rotman2015AlgebraVol1}, \incite[prop. 1.1]{Knapp2016BasicAlgebra}, \incite[83]{Aluffi2009Algebra} and \incite[sec. 2.12]{Тыртышников2007ЛинАлгебра} assume that
    \begin{equation*}
      0 \leq r < \abs{m}_\BbbZ.
    \end{equation*}

    This is a commonly accepted convention among mathematicians, and this is the choice we make for \fullref{alg:integer_division} and \cite{notebook:code}.

    Based on \cite[prop. 1.1]{Knapp2016BasicAlgebra}, we can define the quotient as
    \begin{equation}\label{eq:rem:integer_division_uniqueness/max/q_cases}
      q_{\max} \coloneqq \begin{cases}
        \max\set{ k \in \BbbZ \given km \leq n }, &m > 0, \\
        \min\set{ k \in \BbbZ \given km \leq n }, &m < 0
      \end{cases}
    \end{equation}
    or, more succinctly,
    \begin{equation}\label{eq:rem:integer_division_uniqueness/max/q}
      q_{\max} \coloneqq \sgn(m) \cdot \max\set[\Big]{ \sgn(m) \cdot k \given* k \in \BbbZ \T{and} km \leq n }.
    \end{equation}

    Then \( m \cdot q_{\max} \leq n \), hence the remainder
    \begin{equation*}
      r_{\max} \coloneqq n - q_{\max} m
    \end{equation*}
    is nonnegative.

    This definition handles different signs as follows:
    \begin{equation*}
      \begin{array}{*{4}{c}}
        \toprule
        n   & m  & q_{\max} & r_{\max} \\
        \midrule
        10  & 3  & 3        & 1        \\
        10  & -3 & -3       & 1        \\
        -10 & 3  & -4       & 2        \\
        -10 & -3 & 4        & 2        \\
        \bottomrule
      \end{array}
    \end{equation*}

    \thmitem{rem:integer_division_uniqueness/trunc} The C programming language standard, \cite[66]{ISO:9899:2018}, suggests using \enquote{truncation towards zero}, which it describes as \enquote{the algebraic quotient with any fractional part discarded}.

    The quotient can be defined as follows:
    \begin{equation}\label{eq:rem:integer_division_uniqueness/trunc/q}
      q_{\T{trunc}} \coloneqq \sgn(mn) \cdot \max\set[\Big]{ k \geq 0 \given* k \cdot \abs{m}_\BbbZ \leq \abs{n}_\BbbZ }.
    \end{equation}

    The sign of the remainder
    \begin{equation*}
      r_{\T{trunc}} \coloneqq n - q_{\T{trunc}} m
    \end{equation*}
    is either \( 0 \) or it matches the sign of \( n \). Indeed,
    \begin{align*}
      n - m \cdot q_{\T{trunc}}
      &=
      \sgn(n) \abs{n}_\BbbZ - \underbrace{\sgn(m) \abs{m}_\BbbZ \cdot \sgn(mn)}_{\sgn(n) \cdot \abs{m}_\BbbZ} \cdot \max\set{ k \geq 0 \given k \cdot \abs{m}_\BbbZ \leq \abs{n}_\BbbZ }
      = \\ &=
      \sgn(n) \parens[\Big]{ \abs{n}_\BbbZ - \underbrace{\abs{m}_\BbbZ \cdot \max\set{ k \geq 0 \given k \cdot \abs{m}_\BbbZ \leq \abs{n}_\BbbZ }}_{\leq \abs{n}_\BbbZ} },
    \end{align*}

    Division with truncation handles different signs as follows:
    \begin{equation*}
      \begin{array}{*{4}{c}}
        \toprule
        n   & m  & q_{\T{trunc}} & r_{\T{trunc}} \\
        \midrule
        10  & 3  & 3             & 1             \\
        10  & -3 & -3            & 1             \\
        -10 & 3  & -3            & -1            \\
        -10 & -3 & 3             & -1            \\
        \bottomrule
      \end{array}
    \end{equation*}

    \thmitem{rem:integer_division_uniqueness/dist} The standard for floating-point arithmetic, \cite[31]{IEEE:754:2019}, defines the quotient of \( n \) and \( m \) as the integer \( q_{\T{dist}} \) minimizing
    \begin{equation}\label{eq:rem:integer_division_uniqueness/dist/q}
      \abs{n - m \cdot q_{\T{dist}}}_\BbbZ,
    \end{equation}
    with the special case that if two values are equally close, the even one must be chosen.

    \begin{itemize}
      \item It is sometimes the case that \eqref{eq:rem:integer_division_uniqueness/trunc/q} minimizes this distance, like above where \( \abs{n}_\BbbZ = 10 \) and \( \abs{3}_\BbbZ \). In general, we have
      \begin{equation*}
        \abs{ n - m \cdot q_{\T{trunc}} }_\BbbZ
        =
        \abs[\Big]{ \abs{n}_\BbbZ - \abs{m}_\BbbZ \cdot \max\set[\Big]{ k \geq 0 \given* k \cdot \abs{m}_\BbbZ \leq \abs{n}_\BbbZ } }_\BbbZ.
      \end{equation*}

      \item In other cases, like when the shortest distance is ambiguous, the two may differ:
      \begin{equation*}
        \begin{array}{c c !{\quad} c c !{\quad} c c}
          \toprule
          n  & m & q_{\T{dist}} & r_{\T{dist}} & q_{\T{trunc}} & r_{\T{trunc}} \\
          \midrule
          1  & 2 & 0            & 1            & 0             & 1             \\
          -1 & 2 & 0            & -1           & 0             & -1            \\
          3  & 2 & 2            & -1           & 1             & 1             \\
          -3 & 2 & -2           & 1            & -1            & -1            \\
          \bottomrule
        \end{array}
      \end{equation*}

      \item Even if the shortest distance is not ambiguous, the two may again differ:
      \begin{equation*}
        \begin{array}{c c !{\quad} c c !{\quad} c c}
          \toprule
          n   & m   & q_{\T{dist}} & r_{\T{dist}} & q_{\T{trunc}} & r_{\T{trunc}} \\
          \midrule
          10  & 12  & 1            & -2           & 0             & 10            \\
          -10 & 12  & -1           & 2            & 0             & -10           \\
          10  & -12 & -1           & -2           & 0             & 10            \\
          -10 & -12 & 1            & 2            & 0             & -10           \\
          \bottomrule
        \end{array}
      \end{equation*}
    \end{itemize}

    The Python programming language uses \enquote{floor division}, discussed below, however it provides the function \identifier{math.remainder}, which computes \( r_{\T{dist}} \coloneqq n - q_{\T{dist}} m \). On the other hand, even though the ECMAScript standard \cite[\S 6.1.6.1.5]{ECMA:262} dictates that remainders should follow the aforementioned \cite{IEEE:754:2019}, which defines quotients via \( q_{\T{dist}} \), actual ECMAScript implementations like V8 v12.9.43 and SpiderMonkey v115 use \( q_{\T{trunc}} \).

    \thmitem{rem:integer_division_uniqueness/floor} Python has \enquote{floor division}, where the quotient of \( n \) and \( m \) is defined as the \hyperref[def:real_floor_ceiling]{floor} of the \hyperref[def:rational_numbers]{rational number} \( n / m \). This is discussed in \cite{PythonDocs:3.12:math} \cite{PEP:238}.

    We can define floor division as follows:
    \begin{equation}\label{eq:rem:integer_division_uniqueness/floor/q_cases}
      q_{\floor} \coloneqq \begin{cases}
        \max\set{ k \in \BbbZ \given km \leq n }, &m > 0, \\
        \max\set{ k \in \BbbZ \given km \geq n }, &m < 0
      \end{cases}
    \end{equation}
    or, more succinctly,
    \begin{equation}\label{eq:rem:integer_division_uniqueness/floor/q}
      q_{\floor} \coloneqq \max\set[\Big]{ k \in \BbbZ \given* k \cdot \abs{m}_\BbbZ \leq \sgn(m) \cdot n }.
    \end{equation}

    The sign of the remainder
    \begin{equation*}
      r_{\floor} \coloneqq n - q_{\floor} m
    \end{equation*}
    is either \( 0 \) or it matches the sign of \( m \). Indeed,
    \begin{itemize}
      \item \( m > 0 \) implies that
      \begin{equation*}
        n - m \cdot q_{\floor}
        =
        n - \underbrace{m \cdot \max\set{ k \in \BbbZ \given km \leq n }}_{\leq n}
        \geq
        0.
      \end{equation*}

      \item \( m < 0 \) implies that
      \begin{equation*}
        n - m \cdot q_{\floor}
        =
        n - \underbrace{m \cdot \max\set{ k \in \BbbZ \given km \geq n }}_{\geq n}
        \leq
        0.
      \end{equation*}
    \end{itemize}

    Floor division handles different signs as follows:
    \begin{equation*}
      \begin{array}{*{4}{c}}
        \toprule
        n   & m  & q_{\floor} & r_{\floor} \\
        \midrule
        10  & 3  & 3          & 1          \\
        10  & -3 & -4         & -2         \\
        -10 & 3  & -4         & 2          \\
        -10 & -3 & 3          & -1         \\
        \bottomrule
      \end{array}
    \end{equation*}

    Finally, we note that \( q_{\max} \) and \( q_{\floor} \) coincide for \( m > 0 \) and, if \( m < 0 \),
    \begin{align*}
      -q_{\max}
      &=
      -\min\set{ k \in \BbbZ \given km \leq n }
      = \\ &=
      -\min\set{ -(-k) \in \BbbZ \given -km \geq -n }
      = \\ &=
      \max\set{ -k \in \BbbZ \given (-k)m \geq -n }
      = \\ &=
      \max\set{ k \in \BbbZ \given km \geq -n },
    \end{align*}
    which is the floor quotient of \( -n \) and \( m \).
  \end{thmenum}
\end{remark}
\begin{comments}
  \item All these quotient algorithms can be found in \identifier{arithmetic.divisibility} in \cite{notebook:code}.
\end{comments}

\begin{algorithm}[Integer division]\label{alg:integer_division}
  Fix two integers \( n \) and \( m \), and assume that \( m \) is nonzero. There exists a unique pair \( \quot(n, m) \) and \( \rem(n, m) \) of integers such that
  \begin{equation}
    n = m \cdot \quot(n, m) + \rem(n, m)
  \end{equation}
  and
  \begin{equation*}
    0 \leq \rem(n, m) < \abs{m}.
  \end{equation*}

  We simply define
  \begin{equation}\label{eq:alg:integer_division/quot}
    \quot(n, m) \coloneqq \begin{cases}
      \max\set{ k \in \BbbZ \given km \leq n }, &m > 0, \\
      \min\set{ k \in \BbbZ \given km \leq n }, &m < 0
    \end{cases}
  \end{equation}
  and remainder
  \begin{equation}\label{eq:alg:integer_division/rem}
    \rem(n, m) \coloneqq n - m \cdot \quot(n, m).
  \end{equation}
\end{algorithm}
\begin{comments}
  \item We use the notation \( \quot(n, m) \) and \( \rem(n, m) \) from \fullref{def:euclidean_domain}.
  \item This is \( q_{\max} \) and \( r_{\max} \) from \fullref{rem:integer_division_uniqueness}, where Euclidean division of integers is extensively discussed.
\end{comments}
\begin{defproof}
  \UniquenessSubProof Suppose that \( n = mq + r = mq' + r' \), where \( 0 \leq r < \abs{m}_\BbbZ \) and \( 0 \leq r' < \abs{m}_\BbbZ \). Then
  \begin{equation*}
    m(q - q') = -(r - r').
  \end{equation*}

  Thus, \( m \) divides \( r - r' \). Then \( m \) divides \( r \) and \( r' \) contradicting the assumption that \( \abs{r}_\BbbZ < \abs{m}_\BbbZ \).
\end{defproof}

\begin{proposition}\label{thm:integers_are_euclidean_domain}
  The \hyperref[def:integers]{ring of integers} is an \hyperref[def:euclidean_domain]{Euclidean domain} with division given by \fullref{alg:integer_division} and degree function \( \abs{\anon}_\BbbZ \).
\end{proposition}
\begin{proof}
  By \fullref{thm:integers_are_integral_domain}, \( \BbbZ \) is entire. The Euclidean domain structure is described by \fullref{alg:integer_division}.
\end{proof}

\begin{proposition}\label{thm:alg:integer_division}
  \Fullref{alg:integer_division} has the following basic properties:
  \begin{thmenum}
    \thmitem{thm:alg:integer_division/nested_quot} For positive integers \( b \) and \( c \) and an arbitrary integer \( a \) we have
    \begin{equation}\label{eq:thm:alg:integer_division/nested_quot}
      \quot(\quot(a, b), c) = \quot(a, bc).
    \end{equation}
  \end{thmenum}
\end{proposition}
\begin{proof}
  \SubProofOf{thm:alg:integer_division/nested_quot} We have
  \begin{equation}\label{eq:thm:alg:integer_division/nested_quot/proof/simple}
    a = bc \cdot \quot(a, bc) + \rem(a, bc).
  \end{equation}

  Also,
  \begin{equation*}
    a = b \cdot \quot(a, b) + \rem(a, b)
  \end{equation*}
  and
  \begin{align*}
    \quot(a, b) = c \cdot \quot(\quot(a, b), c) + \rem(\quot(a, b), c),
  \end{align*}
  thus
  \begin{equation}\label{eq:thm:alg:integer_division/nested_quot/proof/nested}
    a = bc \cdot \quot(\quot(a, b), c) + b \cdot \rem(\quot(a, b), c) + \rem(a, b).
  \end{equation}

  We have
  \begin{equation*}
    b \cdot \rem(\quot(a, b), c) + \rem(a, b)
    \leq
    b(c - 1) + (b - 1)
    =
    bc - 1,
  \end{equation*}
  hence, the sum of the remainders in \eqref{eq:thm:alg:integer_division/nested_quot/proof/nested} do not contribute to the quotient.

  Because of the uniqueness in \fullref{alg:integer_division}, we conclude from \eqref{eq:thm:alg:integer_division/nested_quot/proof/simple} and \eqref{eq:thm:alg:integer_division/nested_quot/proof/nested} that
  \begin{equation*}
    \quot(a, bc) = \quot(\quot(a, b), c).
  \end{equation*}
\end{proof}

\begin{lemma}[Bezout's lemma]\label{thm:bezout_lemma}
  For any two integers \( n \) and \( m \), there exist \( a \) and \( b \) such that
  \begin{equation*}
    \gcd(n, m) = an + bm.
  \end{equation*}
\end{lemma}
\begin{proof}
  Since \( \BbbZ \) is a principal ideal domain, it is also a Bezout domain by \fullref{def:principal_ideal_domain/bezout}.
\end{proof}

\begin{remark}\label{rem:integer_domain_chain}
  We have the following hierarchy:
  \begin{itemize}
    \item \( \BbbZ \) is a \hyperref[def:euclidean_domain]{Euclidean domain} as shown in \fullref{thm:integers_are_euclidean_domain}.
    \item As an Euclidean domain, \( \BbbZ \) is a \hyperref[def:principal_ideal_domain]{principal ideal domain} as a consequence of \fullref{thm:def:euclidean_domain/pid}.
    \item As a PID, \( \BbbZ \) is, by definition, both a \hyperref[def:factorial_domain]{factorial domain} and a \hyperref[def:bezout_domain]{Bezout domain}.
    \item As a factorial domain (or as a Bezout domain), \( \BbbZ \) is a \hyperref[def:gcd_domain]{GCD domain}.
  \end{itemize}
\end{remark}

\begin{remark}\label{rem:integer_gcd}
  We discuss in \fullref{rem:choice_of_associates}, in general \hyperref[def:gcd_domain]{GCD domain}, the \hyperref[def:gcd]{greatest common divisors} are not unique.

  The \hyperref[def:gcd]{greatest common divisor} of integers of is, by convention, \hi{positive}. This leaves a canonical choice for both the greatest common divisor and the least common multiple. By \fullref{thm:natural_number_divisibility_lattice}, the \hyperref[thm:semiring_divisibility_order]{divisibility order} of positive integers is compatible with the usual \hyperref[def:integer_ordering]{integer ordering}.

  \Fullref{alg:euclidean_algorithm} allows us to explicitly compute both GCDs and LCMs.
\end{remark}

\paragraph{Prime numbers}

\begin{definition}\label{def:prime_number}\mcite[2]{Apostol1976AnalyticNumberTheory}
  A \term[ru=простое число (\cite[45]{Зорич2019АнализТом1})]{prime number} is an integer greater than \( 1 \) whose only \hi{positive} proper \hyperref[def:divisibility]{divisors} is \( 1 \). Non-prime integers greater than \( 1 \) are called \term{composite numbers}.
\end{definition}

\begin{remark}\label{rem:prime_numbers}
  The definition of a prime number given in \fullref{def:prime_number} is standard, however it seems quite inconsistent with \fullref{sec:integral_domains}.

  First, \fullref{sec:integral_domains} actually defines \hyperref[def:domain_divisibility/irreducible]{irreducible elements} rather than \hyperref[def:domain_divisibility/prime]{prime elements} of the domain \( \BbbZ \). Second, if \( p \) is a prime number, \( -p \) is also a prime number.

  Fortunately, prime and irreducible elements coincide in \hyperref[def:gcd_domain]{GCD domains} due to \fullref{thm:def:gcd_domain/irreducible_is_prime} and \fullref{thm:def:gcd_domain/irreducible_is_prime}. Unfortunately, calling negative prime elements of \( \BbbZ \) \enquote{prime numbers} is not accepted.

  Coprime integers are, fortunately, defined as in general GCD domains via \fullref{def:coprime_elements}.
\end{remark}

\begin{lemma}[Euclid's lemma]\label{thm:euclids_lemma}
  If \( p \) is a \hyperref[def:prime_number]{prime number}, then \( p \mid nm \) implies \( p \mid n \) or \( p \mid m \).
\end{lemma}
\begin{proof}
  Since \( \BbbZ \) is a GCD domain, the lemma follows from \fullref{thm:def:gcd_domain/irreducible_is_prime}.
\end{proof}

\begin{theorem}[Fundamental theorem of arithmetic]\label{thm:fundamental_theorem_of_arithmetic}
  Every integer greater than \( 1 \) can be \hyperref[def:irreducible_factorization]{factored} into a product of \hyperref[def:prime_number]{prime} powers.
\end{theorem}
\begin{comments}
  \item The factorization is unique when regarded as a \hyperref[def:multiset]{multiset} of prime numbers.
\end{comments}
\begin{proof}
  As discussed in \fullref{rem:integer_domain_chain}, \( \BbbZ \) is a factorial domain.
\end{proof}

\begin{definition}\label{def:coprime_numbers}
  We say that the integers \( n \) and \( m \) are \term[ru=взаимно простые (числа) (\cite[45]{Зорич2019АнализТом1}), en=coprime / relatively prime (\cite[231]{Rosen2018DiscreteHandbook})]{coprime} if any of the following equivalent conditions hold:
  \begin{thmenum}
    \thmitem{def:coprime_numbers/abstract} They are \hyperref[def:coprime_elements]{coprime elements} of \( \BbbZ \).
    \thmitem{def:coprime_numbers/concrete}\mcite[45]{Зорич2019АнализТом1} Their only \hi{positive} common divisor is \( 1 \).
  \end{thmenum}
\end{definition}

\begin{proposition}\label{thm:n_plus_1_coprime}
  Given any integer greater \( n \) than \( 1 \), the numbers \( n \) and \( n + 1 \) are \hyperref[def:coprime_elements]{coprime}.
\end{proposition}
\begin{proof}
  Let \( m \) be a common divisor of \( n \) and \( n + 1 \). Then there exists some integer \( k \) such that \( n = km \) and some integer \( l \) such that \( n + 1 = lm \). Then
  \begin{equation*}
    n + 1 = lm = km + 1,
  \end{equation*}
  thus \( m \) divides \( 1 \).

  Therefore, \( m = 1 \).
\end{proof}

\begin{proposition}\label{thm:primality_via_coprimality}
  The (positive) integer \( n \) is \hyperref[def:prime_number]{prime} if and only if \( n \) is \hyperref[def:coprime_numbers]{coprime} with every \( m < n \).
\end{proposition}
\begin{proof}
  \SufficiencySubProof Trivial.
  \NecessitySubProof If \( m \mid n \), then \( \gcd(m, n) = 1 \), hence \( m = 1 \).
\end{proof}

\paragraph{Prime counting}

\begin{definition}\label{def:arithmetic_function}\mcite[24]{Apostol1976AnalyticNumberTheory}
  We say that a \hyperref[def:function]{function} is \term[en=arithmetical function (\cite[24]{Apostol1976AnalyticNumberTheory})]{arithmetic} if its domain is the set of positive integers.
\end{definition}

\begin{definition}\label{def:int_sqrt}\mimprovised
  Based on \cite{PythonDocs:3.12:math}, we will introduce an \term{integer square root} \hyperref[def:arithmetic_function]{arithmetic function}
  \begin{equation*}
    \op{isqrt}(n) \coloneqq \min\set{ k = 1, 2, \ldots \given k^2 \leq n }.
  \end{equation*}

  If \( n = \op{isqrt}(n)^2 \), we call \( n \) a \term[en=perfect square (\cite[example 1.7.1]{Rosen2019DiscreteMathematics})]{perfect square}.
\end{definition}
\begin{comments}
  \item In terms of the real number \( n \)-th root function, which we will discuss in \fullref{def:nth_root}, we can define the integer square root as
  \begin{equation*}
    \op{isqrt}(n) \coloneqq \floor\parens[\Big]{ \sqrt{n} }.
  \end{equation*}
\end{comments}

\begin{proposition}\label{thm:int_sqrt_leq}
  If for some positive integers we have \( n = ab \), we have the following possibilities:
  \begin{thmenum}
    \thmitem{thm:int_sqrt_leq/equal} \( a = \op{isqrt}(n) = b \).
    \thmitem{thm:int_sqrt_leq/a_leq} \( a \leq \op{isqrt}(n) < b \).
    \thmitem{thm:int_sqrt_leq/b_leq} \( a > \op{isqrt}(n) \geq b \).
  \end{thmenum}
\end{proposition}
\begin{proof}
  We have the following options in general for \( n = ab \):
  \begin{itemize}
    \item If \( a = b \), then \( n \) is a perfect square, hence \fullref{thm:int_sqrt_leq/equal} holds.
    \item If \( a < b \), then we must investigate how they are related to \( \op{isqrt}(n) \).

    If we suppose that \( b \leq \op{isqrt}(n) \), then \( ab < b^2 \leq \op{isqrt}(n)^2 \leq n \), which is a contradiction.

    Similarly, if we suppose that \( a > \op{isqrt}(n) \), then both \( a \geq \op{isqrt}(n) + 1 \), hence
    \begin{equation*}
      (\op{isqrt}(n) + 1)^2 < ab = n,
    \end{equation*}
    which contradicts the maximality of \( \op{isqrt}(n) \).

    This leads to \fullref{thm:int_sqrt_leq/a_leq}.

    \item If \( a > b \), we can argue by \fullref{thm:preorder_duality} to conclude that \fullref{thm:int_sqrt_leq/b_leq} holds.
  \end{itemize}
\end{proof}

\begin{proposition}\label{thm:prime_number_sqrt}
  An integer \( n > 1 \) is \hyperref[def:prime_number]{prime} if and only if no integer \( k \) with \( 1 < k \leq \op{isqrt}(n) \) divides \( n \).
\end{proposition}
\begin{comments}
  \item A consequence of this is that a brute-force primality check, in which we check whether \( k \) divides \( n \) for all \( k < n \), the condition on \( k \) can be simplified to \( k \leq \op{isqrt}(n) \).
\end{comments}
\begin{proof}
  \SufficiencySubProof Suppose that \( n \) is prime. Since \( n > 1 \), then \( \op{isqrt}(n) < n \), and thus \( 1 < k \leq \op{isqrt}(n) \) implies \( k < n \). Then, since \( n \) is prime, no such \( k \) can divide \( n \) since its only positive divisors are \( 1 \) and \( n \) itself.

  \NecessitySubProof Suppose that, for some \( n > 1 \), from \( 1 < k \leq \op{isqrt}(n) \) it follows that \( k \) does not divide \( n \).

  Let \( n = ab \). \Fullref{thm:int_sqrt_leq} implies that either \( a \leq \op{isqrt}(n) \), in which case \( a = 1 \) and \( b = n \), or \( b \leq \op{isqrt}(n) \), in which case \( b = 1 \) and \( a = n \).

  Therefore, we conclude that \( n \) is prime.
\end{proof}

\begin{corollary}\label{thm:prime_number_sqrt_prime}
  An integer \( n > 1 \) is \hyperref[def:prime_number]{prime} if and only if no \hi{prime} \( p \) with \( 1 < p \leq \op{isqrt}(n) \) divides \( n \).
\end{corollary}
\begin{proof}
  Follows from \fullref{thm:prime_number_sqrt} by noticing that \( k \) divides \( n \) whenever the prime factors of \( p \) do.
\end{proof}

\begin{definition}\label{def:prime_counting_function}\mcite[8]{Apostol1976AnalyticNumberTheory}
  Denote by \( \pi(n) \) the \hyperref[def:arithmetic_function]{arithmetic function} giving the number of primes less than or equal to \( n \).
\end{definition}

\begin{algorithm}[Sieve of Eratosthenes]\label{alg:sieve_of_eratosthenes}\mcite[exer. 4.5.4.8]{Knuth1997ArtVol2}
  We can construct approximations to the sequence
  \begin{equation*}
    s_k \coloneqq \begin{cases}
      1, &k \T{is prime}, \\
      0, &\T{otherwise}.
    \end{cases}
  \end{equation*}

  Via recursion on \( n \), we will build the sequence \( \seq{ s_{n,k} }_{k=1}^\infty \) such that, for \( 1 \leq k \leq n^2 \), \( s_{n,k} \) is \( 1 \) if \( k \) is prime and \( 0 \) otherwise.

  \begin{thmenum}
    \thmitem{alg:sieve_of_eratosthenes/one} We start with the sequence
    \begin{equation*}
      s_{1,k} \coloneqq \begin{cases}
        0, &k = 1, \\
        1, &\T{otherwise.}
      \end{cases}
    \end{equation*}

    \thmitem{alg:sieve_of_eratosthenes/recursive} For \( n > 1 \), let \( s_{n,k} = s_{n-1,k} \) if \( s_{n-1,n} = 0 \) and otherwise let
    \begin{equation*}
      s_{n,k} \coloneqq \begin{cases}
        0,         &k > n \T{and} n \T{divides} k \\
        s_{n-1,k}, &\T{otherwise.}
      \end{cases}
    \end{equation*}
  \end{thmenum}
\end{algorithm}
\begin{comments}
  \item This algorithm can be found as \identifier{arithmetic.primes.build_erathostenes_sieve} in \cite{notebook:code}.
\end{comments}
\begin{proof}
  Correctness follows from \fullref{thm:prime_number_sqrt_prime}.
\end{proof}

\begin{proposition}\label{thm:prime_counting_sieve}
  We can use \fullref{alg:sieve_of_eratosthenes} to compute the \hyperref[def:prime_counting_function]{prime counting function}:
  \begin{equation*}
    \pi(n) \coloneqq \sum_{k=1}^n s_{\op{isqrt}(n),k}
  \end{equation*}
\end{proposition}
\begin{comments}
  \item We can use \fullref{thm:inclusion_exclusion_principle} to compute \( \pi(n) \) --- see \fullref{ex:thm:inclusion_exclusion_principle/eratosthenes}.
\end{comments}
\begin{proof}
  Trivial.
\end{proof}

\paragraph{Euler's totient theorem}

\begin{definition}\label{def:eulers_totient_function}\mcite[25]{Apostol1976AnalyticNumberTheory}
  Denote by \( \varphi(n) \) the \hyperref[def:arithmetic_function]{arithmetic function} giving the number of strictly smaller than \( n \) positive integers that are \hyperref[def:coprime_elements]{coprime} to \( n \). We call \( \varphi \) \term{Euler's totient function}.
\end{definition}
\begin{comments}
  \item \Fullref{thm:inclusion_exclusion_totient} provides a useful expression in terms of prime factors.
\end{comments}

\begin{proposition}\label{thm:def:eulers_totient_function}
  \hyperref[def:eulers_totient_function]{Euler's totient function} \( \varphi \) has the following basic properties:
  \begin{thmenum}
    \thmitem{thm:def:eulers_totient_function/one} \( \varphi(1) = 0 \).
    \thmitem{thm:def:eulers_totient_function/prime} If \( p \) is \hyperref[def:prime_number]{prime}, then \( \varphi(p) = p - 1 \).
    \thmitem{thm:def:eulers_totient_function/zn} The \hyperref[def:semiring]{multiplicative group} \( \BbbZ_n^\times \) of the ring \hyperref[def:ring_of_integers_modulo]{\( \BbbZ_n \)} of integers modulo \( n > 1 \) has order \( \varphi(n) \).
  \end{thmenum}
\end{proposition}
\begin{proof}
  \SubProofOf{thm:def:eulers_totient_function/one} There are no positive integers smaller than \( 1 \).

  \SubProofOf{thm:def:eulers_totient_function/prime} Every positive integer smaller than \( p \) is coprime to \( p \), and there are exactly \( p - 1 \) positive integers smaller than \( p \) --- \( 1, 2, \ldots, p - 1 \).

  \SubProofOf{thm:def:eulers_totient_function/zn} Follows from \fullref{thm:multiplicative_group_of_integers_modulo}.
\end{proof}

\begin{theorem}[Euler's totient theorem]\label{thm:eulers_totient_theorem}
  For positive coprime integers \( n \) and \( x \), we have
  \begin{equation*}
    x^{\varphi(n)} \cong 1 \pmod n,
  \end{equation*}
  where \( \varphi \) is \hyperref[def:eulers_totient_function]{Euler's totient function}.
\end{theorem}
\begin{proof}
  This is vacuous for \( n = 1 \) since all integers are equal modulo \( 1 \).

  Suppose that \( n > 1 \). First, use \fullref{alg:integer_division} to obtain integers \( q \) and \( y < n \) such that
  \begin{equation*}
    x = nq + y.
  \end{equation*}

  Since \( x \) is, by assumption, coprime with \( n \), then \( y \) is also coprime with \( n \). Indeed, every common divisor \( d \) of \( y \) and \( n \) is also a common divisor \( x \), and the largest such possible value is \( \gcd(n, x) = 1 \).

  Now consider the \hyperref[def:semiring]{multiplicative group} \( \BbbZ_n^\times \) of the ring \hyperref[def:ring_of_integers_modulo]{\( \BbbZ_n \)} of integers modulo \( n \) and the \hyperref[def:cyclic_group]{cyclic subgroup} \( \set{ 1, y, y^2, \ldots } \) (modulo \( n \)). It is necessarily finite as a subgroup of \( \BbbZ_n^\times \). Furthermore, by \fullref{thm:lagranges_subgroup_theorem}, its order \( k \) divides the order of \( \BbbZ_n^\times \). By \fullref{thm:def:eulers_totient_function/zn}, the order of \( \BbbZ_n^\times \) is \( \varphi(n) \).

  We have \( y^k \cong 1 \pmod n \) since \( k \) is the order of a cyclic group. If \( \varphi(n) = km \), then
  \begin{equation*}
    y^{\varphi(n)}
    =
    y^{km}
    \reloset {\eqref{eq:thm:semigroup_exponentiation_properties/repeated}} =
    (y^k)^m
    \cong
    1^m
    \pmod n.
  \end{equation*}
\end{proof}

\begin{corollary}\label{thm:division_modulo}
  Given positive integers \( n \) and \( m \), we can apply \fullref{alg:integer_division} to obtain \( n = q \varphi(m) + r \), where \( \varphi \) is \hyperref[def:eulers_totient_function]{Euler's totient function}.

  Then, for a positive integer \( x \) coprime to \( m \), we have
  \begin{equation*}
    x^n \cong x^r \pmod m,
  \end{equation*}
\end{corollary}
\begin{proof}
  By \fullref{thm:eulers_totient_theorem}, \( x^{\varphi(m)} \cong 1 \pmod m \). Then
  \begin{equation*}
    x^n = (x^{\varphi(m)})^q x^r \cong x^r \pmod m.
  \end{equation*}
\end{proof}

\begin{example}\label{ex:division_modulo}
  The integers \( 9 \) and \( 10 \) are coprime. We have \( \varphi(9) = 6 \) and \( 1000 = 166 \cdot 6 + 4 \). By \fullref{thm:division_modulo},
  \begin{equation*}
    9^{1000} \cong 9^4 = 6561 \cong 1 \pmod {10}.
  \end{equation*}

  We can thus vastly simplify finding the last digit of the decimal representation of \( 9^{1000} \).
\end{example}

\begin{theorem}[Fermat's little theorem]\label{thm:fermats_little_theorem}
  For a \hyperref[def:prime_number]{prime number} \( p \) and for any positive integer \( x \), we have
  \begin{equation*}
    x^p \cong x \pmod p.
  \end{equation*}
\end{theorem}
\begin{proof}
  If \( p \mid x \), then both \( x^p \) and \( x \) and congruent to \( 0 \) modulo \( p \).

  Otherwise, by \fullref{thm:eulers_totient_theorem}, we have \( x^{\varphi(p) + 1} \cong x \pmod p \), and by \fullref{thm:def:eulers_totient_function/prime}, we have \( \varphi(p) + 1 = p \).
\end{proof}

  \section{Rational numbers}\label{sec:rational_numbers}

\paragraph{Field of rational numbers}

\begin{definition}\label{def:rational_numbers}
  We define the \hyperref[def:field]{field} \( \BbbQ \) of \term[bg=рационални числа (\cite[18]{Тагамлицки1971Диф}), ru=рациональные числа (\cite[def. 22.1]{АлександровМаркушевичХинчин1951ЭнциклопедияТом1})]{rational numbers} as the \hyperref[def:field_of_fractions]{field of fractions} of the \hyperref[def:ring]{ring} \( \BbbZ \) of \hyperref[def:integers]{integers}.
\end{definition}
\begin{comments}
  \item We will prefer representations of rational numbers \hyperref[def:lowest_terms]{in lowest terms} with positive denominators. \Fullref{def:lowest_terms} implies that such a representation is unique.

  \item We consider the following an indispensable part of \( \BbbQ \):
  \begin{itemize}
    \item The ordering defined in \fullref{def:rational_numbers_ordering}, which makes \( \BbbQ \) an \hyperref[def:ordered_semiring]{ordered (semi)ring}.
    \item The order topology defined via \fullref{def:order_topology}.
  \end{itemize}
\end{comments}

\paragraph{Ordering of rational numbers}

\begin{definition}\label{def:rational_numbers_ordering}\mcite[lemma 5QH]{Enderton1977Sets}
  We extend the \hyperref[def:integer_ordering]{integer ordering} \( \leq_\BbbZ \) to the \hyperref[def:rational_numbers]{rational numbers} \( \BbbQ \) as follows:
  \begin{equation*}
    \frac a b \leq \frac c d \quad\T{if}\quad ad \leq_\BbbZ bc.
  \end{equation*}
\end{definition}

\begin{definition}\label{def:archimedean_field}\mcite[22]{Kelley1975Topology}
  We say that a \hyperref[def:totally_ordered_set]{totally} \hyperref[def:ordered_semiring]{ordered} \hyperref[def:field]{field} is \term{Archimedean} if, for every pair of \hyperref[def:ordered_semiring_positivity]{positive} field elements \( x \) and \( y \), there exists a positive integer \( n \) such that \( nx > y \).
\end{definition}
\begin{comments}
  \item We can take \( n \) to be minimal so that \( nx > y \geq (n-1)x \) holds.
\end{comments}

\begin{proposition}\label{thm:def:rational_numbers_ordering}
  \hyperref[def:rational_numbers_ordering]{Rational number ordering} has the following basic properties:
  \begin{thmenum}
    \thmitem{thm:def:rational_numbers_ordering/inverse} We have
    \begin{equation*}
      \frac a b \leq \frac c d \T{if and only if} -\frac c d \leq -\frac a b.
    \end{equation*}

    \thmitem{thm:def:rational_numbers_ordering/total} It is a \hyperref[def:totally_ordered_set]{total order}.
    \thmitem{thm:def:rational_numbers_ordering/dense} As a totally ordered set, \( \BbbQ \) is \hyperref[def:dense_total_order]{dense-in-itself}.
    \thmitem{thm:def:rational_numbers_ordering/ordered_ring} It makes \( \BbbQ \) an \hyperref[def:ordered_semiring]{ordered (semi)ring}.
    \thmitem{thm:def:rational_numbers_ordering/archimedean} It makes \( \BbbQ \) an \hyperref[def:archimedean_field]{Archimedean field}.

    \thmitem{thm:def:rational_numbers_ordering/archimedean_exponentiation} For every pair \( a / b > 1 \) and \( c / d > 0 \), there exists some nonnegative integer \( n \) such that \( (a / b)^n > c / d \).

    \thmitem{thm:def:rational_numbers_ordering/power_monotone} If \( 0 \leq a / b < c / d \), then, for any positive integer \( n \), we have \( (a / b)^n < (c / d)^n \).

    \thmitem{thm:def:rational_numbers_ordering/reciprocal} For positive rational numbers \( a / b \) and \( c / d \), we have
    \begin{equation}\label{eq:thm:def:rational_numbers_ordering/reciprocal}
      \frac a b < \frac c d \T{if and only if} \frac b a > \frac d c.
    \end{equation}
  \end{thmenum}
\end{proposition}
\begin{proof}
  \SubProofOf{thm:def:rational_numbers_ordering/inverse} Let \( a / b \leq c / d \). Then \( ad \leq_\BbbZ bc \). \Fullref{thm:def:integer_ordering/inverse} implies that \( -bc \leq_\BbbZ -ad \), and hence \( -c / d \leq - a / b \).

  The converse direction of the proof is identical.

  \SubProofOf{thm:def:rational_numbers_ordering/total} All necessary conditions for \( \leq \) to be a total order follow from the analogous property for integers --- \fullref{thm:def:integer_ordering/total} --- similarly to our proof in \fullref{thm:def:rational_numbers_ordering/inverse}.

  \SubProofOf{thm:def:rational_numbers_ordering/dense} Let
  \begin{equation*}
    \frac a b < \frac c d.
  \end{equation*}

  Then \( ad < bc \), hence
  \begin{equation*}
    \frac a b = \frac {2d \cdot a} {2d \cdot b} < \frac {ad + bc} {2bd} < \frac {2b \cdot c} {2b \cdot d} = \frac c d.
  \end{equation*}

  \SubProofOf{thm:def:rational_numbers_ordering/ordered_ring} We will show that \( \BbbQ \) is an ordered ring:
  \SubProofOf*[def:ordered_semigroup]{addition compatibility} If \( a / b \leq c / d \), we will prove that, for any rational number \( e / f \),
  \begin{equation}\label{eq:thm:def:rational_numbers_ordering/ordered_ring/additive_hypothesis}
    \underbrace{\frac a b + \frac e f}_{\frac {af + be} {bf}} \leq \underbrace{\frac c d  + \frac e f}_{\frac {cf + de} {df}}.
  \end{equation}

  We have \( ad \leq bc \). Because \( \BbbZ \) is itself an ordered ring by \fullref{thm:def:integer_ordering/ordered_ring}, this inequality will be preserved if we multiply it by \( f^2 \) and then add \( bdef \) so that
  \begin{equation*}
    (af + be) df = adf^2 + bdef \leq bcf^2 + bdef (cf + de) bf.
  \end{equation*}

  Thus, \eqref{eq:thm:def:rational_numbers_ordering/ordered_ring/additive_hypothesis} follows.

  \SubProofOf*[def:ordered_semiring]{multiplication compatibility} If \( a / b \leq c / d \), we will prove that, for any rational number \( e / f \),
  \begin{equation}\label{eq:thm:def:rational_numbers_ordering/ordered_ring/multiplicative_hypothesis}
    \frac a b \cdot \frac e f \leq \frac c d \cdot \frac e f.
  \end{equation}

  We have \( ad \leq bc \). Then we can multiply by \( ef \) to obtain \( ae \cdot df \leq ce \cdot bf \), which is a restatement of \eqref{eq:thm:def:rational_numbers_ordering/ordered_ring/multiplicative_hypothesis}.

  \SubProofOf{thm:def:rational_numbers_ordering/archimedean} If both \( a / b \) and \( c / d \) are positive, all \( a \), \( b \), \( c \) and \( d \) are positive integers. We are looking for a positive integer \( n \) such that
  \begin{equation*}
    n \cdot \frac a b > \frac c d,
  \end{equation*}
  that is,
  \begin{equation*}
    n \cdot ad > bc.
  \end{equation*}

  \Fullref{alg:integer_division} gives us integers \( q \) and \( r \), where \( 0 \leq r < ad \) and
  \begin{equation*}
    bc = q \cdot ad + r.
  \end{equation*}

  Since \( bc \leq q \cdot ad \), we can conclude that \( bc < (q + 1) \cdot ad \).

  \SubProofOf{thm:def:rational_numbers_ordering/archimedean_exponentiation} We will use an auxiliary statement: for any nonnegative integer \( n \), we have
  \begin{equation}\label{eq:thm:def:rational_numbers_ordering/archimedean_exponentiation/aux}
    \parens[\Big]{ \frac a b }^n > n\parens[\Big]{ \frac a b - 1 }.
  \end{equation}

  The base case for \( n = 0 \) is trivial. If we suppose that \eqref{eq:thm:def:rational_numbers_ordering/archimedean_exponentiation/aux} holds for some concrete \( n \), then
  \small
  \begin{equation*}
    \parens[\Big]{ \frac a b }^{n+1} = \parens[\Big]{ \frac a b }^n\parens[\Big]{ \frac a b - 1 } + \parens[\Big]{ \frac a b }^n > \parens[\Big]{ \frac a b }^n\parens[\Big]{ \frac a b - 1 } + n\parens[\Big]{ \frac a b - 1 } > \parens[\Big]{ \frac a b - 1 } + n\parens[\Big]{ \frac a b - 1 } = (n + 1)\parens[\Big]{ \frac a b - 1 }.
  \end{equation*}
  \normalsize

  Having that in mind, \fullref{thm:def:rational_numbers_ordering/archimedean} implies that, for some positive integer \( n \), we have
  \begin{equation*}
    n\parens[\Big]{\frac a b - 1} > \frac c d.
  \end{equation*}

  Then \eqref{eq:thm:def:rational_numbers_ordering/archimedean_exponentiation/aux} implies that \( (a / b)^n > c / d \).

  \SubProofOf{thm:def:rational_numbers_ordering/power_monotone} Suppose that \( 0 < a / b < c / d \). We will show by induction on positive integers \( n \) that \( (a / b)^n < (c / d)^n \).

  The case \( n = 1 \) is trivial. If \( (a / b)^n < (c / d)^n \) holds, then we can multiply both sides by \( c / d \) so that
  \begin{equation}\label{eq:thm:def:rational_numbers_ordering/power_monotone/right}
    \frac c d \cdot \parens[\Big]{ \frac a b }^n < \parens[\Big]{ \frac c d }^{n+1}.
  \end{equation}

  We can also multiply \( a / b < c / d \) by \( (a / b)^n \):
  \begin{equation}\label{eq:thm:def:rational_numbers_ordering/power_monotone/left}
    \parens[\Big]{ \frac a b }^{n + 1} < \frac c d \cdot \parens[\Big]{ \frac a b }^n.
  \end{equation}

  The result follows by combining \eqref{eq:thm:def:rational_numbers_ordering/power_monotone/right} and \eqref{eq:thm:def:rational_numbers_ordering/power_monotone/left}.

  \SubProofOf{thm:def:rational_numbers_ordering/reciprocal} Both directions have identical proofs, so we will only prove that \( a / b < c / d \) implies \( b / a > d / c \).

  This is done simply by multiplying both sides by \( {bd} / {ac} \):
  \begin{equation*}
    \frac d c = \frac {a \cdot bd} {b \cdot ac} < \frac {c \cdot bd} {d \cdot ac} = \frac b a.
  \end{equation*}
\end{proof}

\begin{proposition}\label{thm:rational_number_signum}\mimprovised
  For the \hyperref[def:signum]{signum} function of \( \BbbQ \), we have
  \begin{equation*}
    \sgn_\BbbQ\parens*{ \frac a b } = \frac {\sgn_\BbbZ(a)} {\sgn_\BbbZ(b)}.
  \end{equation*}
\end{proposition}
\begin{proof}
  Simple case-by-case verification.
\end{proof}

\begin{algorithm}[Rational number power bisection]\label{alg:rational_number_power_bisection}
  Fix a positive integer \( n \) and two positive rational numbers
  \begin{equation*}
    \frac a b < \frac c d.
  \end{equation*}

  We will find positive integers \( e \) and \( f \) such that
  \begin{equation}\label{eq:alg:rational_power_bisection/condition}
    \frac a b < \frac {e^n} {f^n} < \frac c d.
  \end{equation}

  \begin{thmenum}
    \thmitem{alg:rational_number_power_bisection/mixed} If
    \begin{equation*}
      \frac a b < 1 < \frac c d,
    \end{equation*}
    we halt the algorithm with \( e = f = 1 \).

    \thmitem{alg:rational_number_power_bisection/small} Suppose that
    \begin{equation*}
      \frac c d \leq 1.
    \end{equation*}

    Then \( c \leq d \) and \( a < b \).

    \begin{thmenum}
      \thmitem{alg:rational_number_power_bisection/small/init} Define
      \begin{align*}
        l_0 \coloneqq \frac a b,
        &&
        u_0 \coloneqq 1,
        &&
        m_0 \coloneqq \frac 1 2 (l_0 + u_0).
      \end{align*}

      We have
      \begin{equation*}
        l_0^n = \parens[\Big]{ \frac a b }^n < \frac a b
      \end{equation*}
      and
      \begin{equation*}
        \frac c d \leq 1 = u_0^n.
      \end{equation*}

      Therefore, the interval between \( l_0^n \) and \( u_0^n \) contains both \( a / b \) and \( c / d \).

      \thmitem{alg:rational_number_power_bisection/small/step} If, at step \( k \geq 0 \), we have
      \begin{equation*}
        \frac a b < m_k^n < \frac c d,
      \end{equation*}
      then we can halt the algorithm with \( m_k \).

      \begin{figure}[!ht]
        \begin{subcaptionblock}{\textwidth}
          \centering
          \includegraphics[page=1]{output/alg__rational_number_power_bisection__small}
        \end{subcaptionblock}

        \begin{subcaptionblock}{\textwidth}
          \centering
          \includegraphics[page=2]{output/alg__rational_number_power_bisection__small}
        \end{subcaptionblock}

        \begin{subcaptionblock}{\textwidth}
          \centering
          \includegraphics[page=3]{output/alg__rational_number_power_bisection__small}
        \end{subcaptionblock}

        \caption{An illustration of \fullref{alg:rational_number_power_bisection/small}.}
        \label{fig:alg:rational_number_power_bisection/small}
      \end{figure}

      If \( m_k^n \leq a / b \), let
      \begin{align*}
        l_{k+1} \coloneqq m_k,
        &&
        u_{k+1} \coloneqq u_k,
        &&
        m_{k+1} \coloneqq \frac 1 2 (l_{k+1} + u_{k+1}).
      \end{align*}

      Otherwise, if \( m_k^n \geq c / d \), let
      \begin{align*}
        l_{k+1} \coloneqq l_k,
        &&
        u_{k+1} \coloneqq m_k,
        &&
        m_{k+1} \coloneqq \frac 1 2 (l_{k+1} + u_{k+1}).
      \end{align*}

      In this case the interval between \( l_{k+1}^n \) and \( u_{k+1}^n \) contains \( a / b \).
    \end{thmenum}

    \thmitem{alg:rational_number_power_bisection/large} Suppose that
    \begin{equation*}
      \frac a b \geq 1.
    \end{equation*}

    Then \fullref{thm:ordered_ring/reciprocal_inversion} implies that
    \begin{equation*}
      \frac d c < \frac b a \leq 1,
    \end{equation*}
    and we can use \fullref{alg:rational_number_power_bisection/small} to find integers \( e \) and \( f \) such that
    \begin{equation*}
      \frac d c < \frac {e^n} {f^n} < \frac b a.
    \end{equation*}

    We can thus halt the algorithm with \( f / e \).
  \end{thmenum}
\end{algorithm}
\begin{comments}
  \item This algorithm can be found as \identifier{arithmetic.rational.power_bisection} in \cite{notebook:code}.
\end{comments}
\begin{proof}
  We must only prove correctness of \fullref{alg:rational_number_power_bisection/small}.

  We start with a wide enough interval \( (l_0, u_0) \) such that \( (l_0^n, u_0^n) \) contains both \( a / b \) and \( c / d \).

  At step \( k \geq 0 \), in case the midpoint \( m_k \) does not satisfy \eqref{eq:alg:rational_power_bisection/condition}, we move on to smaller interval. We choose either \( (m_k, u_k) \) or \( (l_k, m_k) \) as the new interval depending on whether \( m_k^n \) is less than \( a / b \) or greater than \( c / d \). With this new interval, \( (l_k^n, l_n^n) \) again contains both \( a / b \) and \( c / d \);

  Since the size of the interval reduces at each step, at some point the algorithm must halt as the interval will become less than the distance between \( a / b \) and \( c / d \);
\end{proof}

\paragraph{Dedekind incompleteness}

\begin{proposition}\label{thm:nth_root_is_not_rational}
  For positive integers \( n \) and \( m \), where \( m \) is \hyperref[def:square_free_element]{square-free}, there exists no \hyperref[def:rational_numbers]{rational number} whose \( n \)-th power equals \( m \).
\end{proposition}
\begin{comments}
  \item \Fullref{thm:real_nth_root_is_irrational} establishes that this fails for \hyperref[def:real_numbers]{real numbers}.
  \item We will use coprime numbers in this proof, although we will defer their definition to \fullref{sec:prime_numbers}. At this point, we simply regard them as coprime elements of the Bezout domain \( \BbbZ \).
\end{comments}
\begin{proof}
  Aiming at a contradiction, suppose that there exists a rational number (in lowest terms) \( a / b \) whose \( n \)-th power is \( m \).

  Then \( a^n = m \cdot b^n \). \Fullref{thm:def:coprime_elements/gcd_power} implies that \( a^n \) and \( b^n \) are coprime since \( a \) and \( b \) are. But \( b^n \) divides \( a^n \), which is only possible if \( b^n = 1 \) and hence \( b = 1 \).

  Then \( a^n = a^{n - 2} a^2 = m \), which contradicts our assumption that \( m \) is square-free.

  Therefore, no rational number raised to the \( n \)-th power equals \( m \).
\end{proof}

\begin{corollary}\label{thm:rational_numbers_not_dedekind_complete}
  The \hyperref[def:rational_numbers]{rational numbers} are not Dedekind complete.
\end{corollary}
\begin{proof}
  Fix a square-free \( m \) and consider the \hyperref[def:dedekind_cut]{Dedekind cut} \( (A, B) \), where
  \begin{align*}
    A \coloneqq \set*{ \frac a b \given* \parens[\Big]{ \frac a b }^2 \leq m }
    &&
    B \coloneqq \set*{ \frac a b \given* \parens[\Big]{ \frac a b }^2 \geq m }.
  \end{align*}

  \Fullref{thm:nth_root_is_not_rational} implies that neither \( A \) has a minimum nor \( B \) has a maximum.

  Therefore, the rational numbers are not Dedekind complete.
\end{proof}

  \subsection{Real numbers}\label{subsec:real_numbers}

\paragraph{Field of real numbers}

\begin{definition}\label{def:real_numbers}\mcite[sec. I.V.]{Beman1901Dedekind}
  We define the set \( \BbbR \) of \term[bg=реални числа (\cite[ch. I]{Тагамлицки1971Диф}), ru=вещественные числа (\cite[\textnumero 3]{ФихтенгольцОсновыТом1}) / действительные числа (\cite[36]{Александров1977Введение})]{real numbers} as the \hyperref[def:dedekind_completion]{Dedekind completion} of the set \( \BbbQ \) of \hyperref[def:rational_numbers]{rational numbers}.
\end{definition}
\begin{comments}
  \item We consider the following an indispensable part of \( \BbbR \):
  \begin{itemize}
    \item The field structure defined in \fullref{def:real_number_arithmetic}.
    \item The order topology defined via \fullref{def:order_topology}.
  \end{itemize}

  \item Our definition of Dedekind completion uses \hyperref[def:dedekind_macnielle_closure]{Dedekind-MacNeille-closed sets}, which has certain consequences compared to other authors --- see \fullref{rem:dedekind_completion_through_dedekind_macneille_closures}.
\end{comments}

\begin{definition}\label{def:real_number_arithmetic}\mimprovised
  For each pair of \hyperref[def:real_numbers]{real numbers} \( R \) and \( Q \), considered as \hyperref[def:dedekind_macnielle_closure]{Dedekind-MacNeille-closed sets} of rational numbers, we define the base operations:
  \begin{thmenum}
    \thmitem{def:real_number_arithmetic/zero} Zero:
    \begin{equation}\label{eq:def:real_number_arithmetic/zero}
      \begin{split}
        \mathllap{O} &\coloneqq \mathrlap{\set{ s \given s \leq 0 }}.
      \end{split}
    \end{equation}

    \thmitem{def:real_number_arithmetic/additive_inverse} Additive inverses:
    \begin{equation}\label{eq:def:real_number_arithmetic/additive_inverse}
      \begin{split}
        \mathllap{-P} \coloneqq \mathrlap{\set{ -p \given p \in P }^L}.
      \end{split}
    \end{equation}

    \thmitem{def:real_number_arithmetic/addition} Addition:
    \begin{equation}\label{eq:def:real_number_arithmetic/addition}
      \begin{split}
        \mathllap{P + Q} \coloneqq \mathrlap{\set{ p + q \given p \in P \T{and} q \in Q }^{UL}}.
      \end{split}
    \end{equation}

    \thmitem{def:real_number_arithmetic/one} One:
    \begin{equation}\label{eq:def:real_number_arithmetic/one}
      \begin{split}
        \mathllap{I} \coloneqq \mathrlap{\set{ s \given s \leq 1 }}.
      \end{split}
    \end{equation}

    \thmitem{def:real_number_arithmetic/multiplicative_inverse} Multiplicative inverses (for \( P \neq O \)):
    \begin{equation}\label{eq:def:real_number_arithmetic/multiplicative_inverses}
      \mathllap{P^{-1}} \coloneqq \mathrlap{\set{ p^{-1} \given p \in P }^L}.
    \end{equation}

    \thmitem{def:real_number_arithmetic/multiplication} Multiplication:
    \begin{equation}\label{eq:def:real_number_arithmetic/multiplication}
      P \cdot Q \coloneqq \begin{cases}
        O,                                                                             &O = P \T{or} O = Q \\
        \set{ p \cdot q \given r \in P \setminus O \T{and} q \in Q \setminus O }^{UL}, &O \subsetneq P \T{and} O \subsetneq Q \\
        (-P) \cdot Q,                                                                  &O \supsetneq P \T{and} O \subsetneq Q \\
        P \cdot (-Q),                                                                  &O \subsetneq P \T{and} O \supsetneq Q \\
        (-P) \cdot (-Q),                                                               &O \supsetneq P \T{and} O \supsetneq Q \\
      \end{cases}
    \end{equation}
  \end{thmenum}
\end{definition}
\begin{comments}
  \item \incite[113]{Enderton1977Sets} instead defines the arithmetic operations for cuts as discussed in \fullref{rem:dedekind_completion_through_dedekind_macneille_closures}, which makes addition simpler, but nearly all other operations more difficult.
  \item The complications in \eqref{eq:def:real_number_arithmetic/multiplication} arise from the fact that \( I \) contains all negative rational numbers and, when multiplied by \( -1 \), these become positive. Thus, if multiplication is always treated like in the first case, multiplying \( I \) by \( -1 \) would give the entire set \( \BbbR \).
\end{comments}

\begin{proposition}\label{thm:real_numbers_are_a_field}
  The operations from \fullref{def:real_number_arithmetic} make the set \( \BbbR \) of \hyperref[def:real_numbers]{real numbers} a \hyperref[def:field]{field}. Furthermore, the \hyperref[def:order_homomorphism/embedding]{order embedding} \( \iota: \BbbQ \to \BbbR \) from \fullref{def:dedekind_macnielle_completion} is a \hyperref[def:field/homomorphism]{field embedding}.
\end{proposition}
\begin{proof}
  \SubProof{Proof that addition is associative} First note that the set of sums \( r + q \), where \( r \in R \) and \( q \in Q \), is downward closed. Indeed, if \( s \leq r + q \), then \( s = (r + q) - s_0 \), where \( s_0 \coloneqq (r + q) - s \). The latter is nonnegative, hence \( q - s_0 \leq q \) and thus \( q - s_0 \in Q \). Then \( s = r + (q - s_0) \), where \( r \) belongs to \( R \) and \( q - s_0 \) belongs to \( Q \). Thus, the set \( \set{ r + q \given r \in R \T{and} q \in Q } \) is downward closed.

  \Fullref{thm:def:dedekind_macnielle_closure/point_in_closure} then implies that every member of \( R + Q \) is either a sum of the form \( r + q \) for \( r \in R \) and \( q \in Q \), or the supremum of such sums.

  Now let \( x \in (P + Q) + R \). We have two possibilities:
  \begin{itemize}
    \item If \( x = s + r \) for some \( s \in P + Q \) and \( r \in R \), we again have two possibilities:
    \begin{itemize}
      \item If \( s = p + q \) for some \( p \in P \) and \( q \in Q \), then \( x = (p + q) + r \). Then obviously \( x = p + (q + r) \) is a member of \( P + (Q + R) \).

      \item If \( s = \sup(P + Q) \), consider some \( s' < s \) and let \( r' \coloneqq r - (s - s') \). Since \( R \) is downward closed, it contains \( r' \). Since \( s' \) is not the supremum of \( P + Q \) (but belongs to the latter set), by \fullref{thm:def:dedekind_macnielle_closure/point_in_closure}, there exist some \( p' \in P \) and \( q' \in Q \) such that \( s' = p' + q' \). Then \( x = (p' + q') + r' \), which again implies \( x \in P + (Q + R) \).
    \end{itemize}

    \item Suppose that \( x = \sup((P + Q) + R) \).

    Let \( x' < x \) so that \( x' \) belongs to \( (P + Q) + R \) and \( x = x' + (x - x') \). Then there exist some \( p' \in P \), \( q' \in Q \) and \( r' \in R \) such that \( x' = (p' + q') + r' \). Clearly then \( x' \) belongs to \( P + (Q + R) \).

    Thus, every rational number strictly smaller than \( x \) belongs to \( P + (Q + R) \). But \( x \) is the supremum of such numbers, and \fullref{thm:def:dedekind_macnielle_closure/closed_elements} implies that this supremum must also belong to \( P + (Q + R) \).
  \end{itemize}

  We have shown that \( (P + Q) + R \subseteq P + (Q + R) \). The converse inclusion can be proven symmetrically. Hence, we conclude that addition of real numbers is associative.

  \SubProof{Proof that addition is commutative} Follows directly from commutativity of addition of rational numbers.

  \SubProof{Proof that addition has a neutral element} We will show that \( P + O = P \).

  Since \( p + 0 = p \), clearly \( P \subseteq P + O \).

  For the converse, note that \Fullref{thm:def:dedekind_macnielle_closure/closed_elements} implies that \( P \) is downward closed, hence \( p \in P \) and \( o < 0 \) imply that \( p + o \in P \). Since \( P \) is also closed under taking suprema of subsets for which suprema exist, it follows that \( P + O \subseteq P \).

  \SubProof{Proof that addition is invertible} We will show that \( P + (-P) = O \).

  A rational number \( p \) is in \( P \) if and only if \( -p \) is an upper bound of \( -P \), hence \( 0 \) is an upper bound of \( P + (-P) \), implying that \( P + (-P) \subseteq O \).

  For the converse, we will consider several cases.

  \SubProof*{Proof for \( P = O \)} \( P \) is simply the principal ideal of \( 0 \), and so is \( -P \). Any negative number \( s < 0 \) is then in \( P \), and, since \( 0 \) is in \( -P \), we have that \( s \) is in \( P + (-P) \).

  \SubProof*{Proof for \( O \subsetneq P \)} Let \( s < 0 \) and let \( n \) be the smallest positive integer such that \( (-s)n \) is an upper bound of \( P \). At least one such integer exists because, given some upper bound \( u \) of \( P \), \fullref{thm:def:rational_number_ordering/archimedean}, gives us a positive integer such that \( (-s)m > u \).

  Then \( (-s)n \) is in \( -P \), \( (-s)n + s \) is in \( P \), and their sum is \( s \). Hence, \( s \in P + (-P) \).

  \SubProof*{Proof for \( P \subsetneq O \)} Similarly to the above case, we can take the smallest \( n \) such that \( sn \) is an upper bound of \( P \). Then \( -sn \) is in \( -P \) and \( sn + s \) is in \( P \), and their sum is \( s \).

  \SubProof{Proof that multiplication is commutative} Follows directly from commutativity of multiplication of rational numbers.

  \SubProof{Proof that multiplication respects additive inverses} We will show that
  \begin{equation}\label{eq:thm:real_numbers_are_a_field/mult_additive_inverses}
    P \cdot (-Q) = (-P) \cdot Q = -(P \cdot Q).
  \end{equation}

  The case where either \( P = O \) or \( Q = O \) is trivial, and we will not consider it.

  \SubProof*{Proof for \( O \subsetneq P \) and \( O \subsetneq Q \)} Since \( -Q \) is the additive inverse of \( Q \), we have \( O \supsetneq -Q \), and thus
  \begin{equation*}
    P \cdot (-Q)
    \reloset {\eqref{eq:def:real_number_arithmetic/multiplication}} =
    -(P \cdot (--Q))
    =
    -(P \cdot Q).
  \end{equation*}

  Similarly,
  \begin{equation*}
    (-P) \cdot Q
    \reloset {\eqref{eq:def:real_number_arithmetic/multiplication}} =
    -((--P) \cdot Q)
    =
    -(P \cdot Q).
  \end{equation*}

  \SubProof*{Proof for \( O \supsetneq P \) and \( O \subsetneq Q \)}
  \begin{equation*}
    P \cdot (-Q)
    \reloset {\eqref{eq:def:real_number_arithmetic/multiplication}} =
    (-P) \cdot (--Q)
    =
    (-P) \cdot Q
    \reloset {\eqref{eq:def:real_number_arithmetic/multiplication}} =
    -((--P) \cdot Q)
    =
    -(P \cdot Q).
  \end{equation*}

  \SubProof*{Proof for \( O \subsetneq P \) and \( O \supsetneq Q \)} Follows from the above via commutativity.

  \SubProof*{Proof for \( O \supsetneq P \) and \( O \supsetneq Q \)}
  \begin{equation*}
    P \cdot (-Q)
    \reloset {\eqref{eq:def:real_number_arithmetic/multiplication}} =
    ((-P) \cdot (-Q))
    \reloset {\eqref{eq:def:real_number_arithmetic/multiplication}} =
    (-P) \cdot Q
  \end{equation*}
  and
  \begin{equation*}
    -(P \cdot Q)
    =
    --((-P) \cdot (-Q))
    =
    (-P) \cdot (-Q).
  \end{equation*}

  \SubProof{Proof that multiplication is associative} We have many cases, but fortunately most of them are similar. We will only consider two, excluding the trivial case where \( P = O \) or \( Q = O \) or \( R = O \).

  \SubProof*{Proof for \( O \subsetneq P \), \( O \subsetneq Q \) and \( O \subsetneq R \)} First note that \( O \subsetneq (P \cdot Q) \cdot R \) and \( O \subsetneq P \cdot (Q \cdot R) \) Indeed, there exist positive numbers \( p \in P \) and \( q \in Q \) such that \( pq \in P \cdot Q \), and thus \( O \subsetneq P \cdot Q \). Similarly, there exists a positive number \( r \in R \) such that \( pqr \in (P \cdot Q) \cdot R \). But \( pqr \) also belongs to \( P \cdot (Q \cdot R) \). Hence, both sets strictly contain \( O \).

  Now that we have show that the two products have the same sign, we will show that they are equal.

  Let \( x \in (P \cdot Q) \cdot R \) and \( x > 0 \). As in the case of addition, \fullref{thm:def:dedekind_macnielle_closure/point_in_closure} implies that we have the following cases:
  \begin{itemize}
    \item If there exist some \( s \in P \cdot Q \) and \( r \in R \) such that \( x = s \cdot r \), we again have two possibilities:
    \begin{itemize}
      \item If there exist some \( p \in P \) and \( q \in Q \) such that \( s = p \cdot q \), then \( x = (p \cdot q) \cdot r = p \cdot (q \cdot r) \) belongs to \( P \cdot (Q \cdot R) \).

      \item If \( s = \sup(P \cdot Q) \), consider some positive \( s' < s \) and let \( r' \coloneqq s' \cdot s^{-1} \cdot r \). Then \( x = s \cdot r = s' \cdot r' \). Furthermore, since \( s' < s \), the quotient \( \ifrac {s'} s \) is less than \( 1 \), hence \( r' < r \) and \( r' \) belongs to \( R \).

      Then \( s' \) is not the supremum of \( P \cdot Q \), and hence there exist positive numbers \( p' \in P \) and \( q' \in Q \) such that \( s' = p' \cdot q' \) and thus \( x' = (p' \cdot q') \cdot r' \).

      It is now clear that \( x \in P \cdot (Q \cdot R) \).
    \end{itemize}

    \item Suppose that \( x = \sup((P \cdot Q) \cdot R) \).

    Fix some \( x' \) such that \( 0 < x' < x \). Then there exist some positive \( p' \in P \), \( q' \in Q \) and \( r' \in R \) such that \( x' = (p' \cdot q') \cdot r' \). Clearly then \( x' \in P \cdot (Q \cdot R) \).

    Thus, every positive member of \( (P \cdot Q) \cdot R \) belongs to \( P \cdot (Q \cdot R) \). Since \( x \) is the supremum of such members, it follows that \( x \) must also belong to \( P \cdot (Q \cdot R) \), which is closed under suprema of subsets that have suprema.
  \end{itemize}

  We have shown that \( (P \cdot Q) \cdot R \subseteq P \cdot (Q \cdot R) \).  The converse inclusion can be proven symmetrically.

  \SubProof*{Proof for \( O \supsetneq P \), \( O \subsetneq Q \) and \( O \subsetneq R \)} We use \eqref{eq:thm:real_numbers_are_a_field/mult_additive_inverses} to reduce this case to the first:
  \begin{equation*}
    P \cdot (Q \cdot R)
    \reloset {\eqref{eq:thm:real_numbers_are_a_field/mult_additive_inverses}} =
    -[(-P) \cdot (Q \cdot R)]
    =
    -[((-P) \cdot Q) \cdot R]
    \reloset {\eqref{eq:thm:real_numbers_are_a_field/mult_additive_inverses}} =
    -[(-(P \cdot Q)) \cdot R]
    \reloset {\eqref{eq:thm:real_numbers_are_a_field/mult_additive_inverses}} =
    (P \cdot Q) \cdot R.
  \end{equation*}

  \SubProof*{Proof for all other cases} The other cases follow similarly from the first via \eqref{eq:thm:real_numbers_are_a_field/mult_additive_inverses}.

  \SubProof{Proof that multiplication has a neutral element} This is similar to the proof for addition. We will show that \( P \cdot I = P \).

  Since \( p \cdot 1 = p \), clearly \( P \subseteq P \cdot I \).

  For the converse, let \( x \in P \cdot I \). Note that \( O \cdot I = O \), hence we can consider the nontrivial cases where \( P \neq O \). First suppose that \( O \subsetneq P \).
  \begin{itemize}
    \item If \( x = p \cdot i \) for some positive \( p \in P \) and \( i \in I \), then \( 0 < i < 1 \) and thus \( x = pi < p \) and \( x \in P \).
    \item If \( x = \sup(P \cdot I) \), it is the supremum of members of \( P \) and is thus itself in \( P \).
  \end{itemize}

  Therefore, if \( O = P \) or \( O \subsetneq P \), then \( P = O \cdot P \). If \( O \supsetneq P \), then
  \begin{equation*}
    P \cdot I = -[(-P) \cdot I] = -(-P) = P.
  \end{equation*}

  \SubProof{Proof that multiplication is invertible} We will show that \( P \cdot P^{-1} = I \), somewhat similarly to how we did it for addition.

  A rational number \( p \) is in \( P \) if and only if \( p^{-1} \) is an upper bound of \( P^{-1} \), hence \( 1 = p \cdot p^{-1} \) is an upper bound of \( P \cdot P^{-1} = I \), implying that \( P \cdot P^{-1} \subsetneq I \).

  We will now show that every positive rational number less than \( 1 \) belongs to \( P \cdot P^{-1} \), and it will follow that so does their Dedekind-MacNeille closure \( I \).

  \SubProof*{Proof for \( O \subsetneq P \)} Let \( 0 < s < 1 \).
  \begin{itemize}
    \item If \( s^{-1} \) is in \( P^{-1} \), let \( n \) be the smallest positive integer such that \( s^{-n} \) is an upper bound of \( P^{-1} \). This is well-defined because \fullref{thm:def:rational_number_ordering/reciprocal} implies that \( s^{-1} > 1 \) and thus \( s^{-n} \) increases with \( n \). Furthermore, \fullref{thm:def:rational_number_ordering/archimedean_exponentiation} implies that, for some positive integer \( n \), \( s^{-n} \) must be an upper bound, that is, at least one upper bound of this form exists.

    Then \( s^{-n+1} \) is in \( P^{-1} \). The number \( n \) is necessarily greater than \( 1 \) because \( s^{-1} \) itself belongs to \( P^{-1} \).

    Furthermore, since \( s^{-n} \) is an upper bound of \( P^{-1} \), its reciprocal belongs to \( P \). Then
    \begin{equation*}
      s = s^n \cdot s^{-n+1} \in P \cdot P^{-1}.
    \end{equation*}

    \item If \( s^{-1} \) is an upper bound of \( P^{-1} \), let \( n \) be the smallest positive integer such that \( s^n \) is in \( P^{-1} \).

    \begin{itemize}
      \item If \( n = 1 \), then \( s \) belongs to both \( P \) and \( P^{-1} \). Either \( P \) or \( P^{-1} \) must contain \( 1 \), thus \( s = s \cdot 1 = 1 \cdot s \) belongs to their product \( P \cdot P^{-1} \).

      \item If \( n > 1 \), then \( s^{n-1} \) is an upper bound of \( P^{-1} \), and thus \( s^{-(n-1)} \) is in \( P \). Thus,
      \begin{equation*}
        s^{-(n-1)} \cdot s^n = s.
      \end{equation*}
    \end{itemize}
  \end{itemize}

  \SubProof*{Proof for \( P \subsetneq O \)} We can show similarly by considering \( -s^n \) rather than \( s^n \).

  \SubProof{Proof of distributivity} We will show that \( P(Q + R) = PQ + PR \). Skipping the trivial cases, we only need to consider the case where \( O \subsetneq P \) and \( O \subsetneq Q + R \) because the rest will follow via \eqref{eq:def:real_number_arithmetic/multiplication}.

  Again, \fullref{thm:def:dedekind_macnielle_closure/point_in_closure} gives us several possibilities for \( x \in P(Q + R) \):
  \begin{itemize}
    \item If \( x = p \cdot s \) for some \( p \in P \) and \( s \in Q + R \), then we have two possibilities again:
    \begin{itemize}
      \item If \( s = q + r \) for some \( q \in Q \) and \( r + R \), then \( x = p(q + r) \) and distributivity on the rational numbers imply \( x \in PQ + PR \).

      \item If \( s = \sup(Q + R) \), since \( p \) is positive, as in our proofs of associativity we have \( x = p' \cdot (q' + r') \) for some \( q' \in Q \) and \( r' \in Q \), and distributivity on the rational again implies \( x \in PQ + PR \).
    \end{itemize}

    \item If \( x = \sup(P(Q + R)) \), then, since \( O \subsetneq P(Q + R) \), we can again conclude that \( x' < x \) implies \( x' \in PQ + PR \), which in turn implies that \( x \in PR + PR \).
  \end{itemize}

  Therefore, \( P(Q + R) \subseteq PQ + PR \).

  Conversely, if \( x \in PQ + PR \), then:
  \begin{itemize}
    \item If \( x = q_0 + r_0 \) for \( q_0 \in PQ \) and \( r_0 \in PR \), then both are positive, and we have more possibilities:
    \begin{itemize}
      \item If \( q_0 = p_q q \) for positive \( p_q \in P \) and \( q \in Q \) and if similarly \( r_0 = p_r r \), let \( p \) be the larger of the two. Then
      \begin{equation*}
        x = p\parens[\Big]{ \underbrace{\frac {p_q} p}_{\leq 1} q + \underbrace{\frac {p_r} p}_{\leq 1} r },
      \end{equation*}
      hence \( x \in P(Q + R) \).

      \item If \( q_0 = p_q q \) as above but \( r_0 = \sup(PR) \), we must consider \( r_0' < r_0 \) and argue that \( x \) is the supremum by \( r_0' \).

      \item The case \( q = \sup(PQ) \) and \( r_0 = p_r r \) is symmetric.

      \item If \( q_0 = \sup(PQ) \) and \( r_0 = \sup(PR) \), we must simultaneously consider \( q_0' < q_0 \) and \( r_0' < r_0 \).
    \end{itemize}

    \item If \( x = \sup(PR + PR) \), then we must consider \( x' < x \) and, as in the proofs of associativity, argue that \( x \) is the supremum by \( x' \).
  \end{itemize}

  Therefore, \( PQ + PR \subseteq P(Q + R) \).

  \SubProof{Proof of field embedding} Clearly the map \( \iota(p) = \BbbQ_{\leq p} \) is injective as an order embedding. We must show that it is a \hyperref[def:field/homomorphism]{field homomorphism}. Clearly \( O = \iota(0) \) and \( I = \iota(1) \).

  Furthermore, \( \iota(p + q) \) has \( p + q \) as its supremum, while \( \iota(p) \) and \( \iota(q) \) have \( p \) and \( q \) correspondingly. Then \( \iota(p) + \iota(q) \) also has \( p + q \) as its supremum, and hence
  \begin{equation*}
    \iota(p + q) = \iota(p) = \iota(q).
  \end{equation*}

  The base case of multiplication can be handled similarly and then extended via \eqref{eq:thm:real_numbers_are_a_field/mult_additive_inverses}.
\end{proof}

\paragraph{Extended real numbers}

\begin{definition}\label{def:extended_real_numbers}\mimprovised
  We are sometimes interested in \term{extended real numbers}. This is the \hyperref[def:dedekind_macnielle_completion]{Dedekind-MacNeille completion} of the \hyperref[def:rational_numbers]{rational numbers}.

  Note that in \fullref{def:dedekind_completion} we defined the Dedekind completion of an unbounded totally ordered set as its Dedekind-MacNeille completion without the top and bottom elements. Thus, the extended real numbers are the real numbers with the additional elements \( \infty = +\infty \), corresponding to the entire set \( \BbbQ \), and \( -\infty \), corresponding to the empty set \( \varnothing \).
\end{definition}

\begin{proposition}\label{thm:extended_real_numbers_are_not_field}
  Consider the \hyperref[def:extended_real_numbers]{extended real numbers} \( \BbbR \cup \set{ -\infty, \infty } \). If we extend \hyperref[def:real_number_arithmetic/addition]{addition} or \hyperref[def:real_number_arithmetic/multiplication]{multiplication} of real numbers to \( \infty \) and/or \( -\infty \) in any way, the obtained set is no longer an \hyperref[def:ordered_semiring]{ordered field}.
\end{proposition}
\begin{proof}
  Since \( 0 < 1 \), in an ordered field, \fullref{thm:def:ordered_semiring/strict_sum} implies that \( \infty = 0 + \infty < 1 + \infty \). But \( \infty \) is the top element.

  Similarly, for multiplication, since \( 1 < 2 \), we have \( \infty < 2 \cdot \infty \), which is again a contradiction.

  The result for \( -\infty \) instead of \( \infty \) is \hyperref[def:semilattice/duality]{dual}.
\end{proof}

\paragraph{Irrational numbers}

\begin{definition}\label{def:irrational_numbers}\mimprovised
  \term[ru=иррациональные числа (\cite[36]{Александров1977Введение})]{Irrational numbers} are \hyperref[def:real_numbers]{real numbers} that are not (the embedding of) \hyperref[def:rational_numbers]{rational numbers}.
\end{definition}
\begin{comments}
  \item These correspond exactly to Dedekind-MacNeille-closed sets of rational numbers that have no supremum.
\end{comments}

\begin{definition}\label{def:nth_root}\mimprovised
  For any \hyperref[def:integers]{integer} \( n \geq 2 \) and any \hi{nonnegative} \hyperref[def:real_numbers]{real number} \( P \), we define the \( n \)-th \term{root} of the \( P \) as the number
  \begin{equation*}
    \sqrt[n]{ P } = O \cup \set{ q > 0 \given q^n \in P }.
  \end{equation*}

  It is customary to simply write \( \sqrt{ P } \) when \( n = 2 \).
\end{definition}
\begin{defproof}
  We must show that \( \sqrt[n]{ P } \) is always \hyperref[def:dedekind_macnielle_closure]{Dedekind-MacNeille closed}.

  If \( P = O \), then \( \sqrt[n]{ P } = O \) and there is nothing to prove. Suppose that \( O \subsetneq P \).

  First note that if \( q > 0 \) is in \( \sqrt[n]{ P } \) and \( 0 < r < q \), \fullref{thm:def:rational_number_ordering/power_monotone} implies that \( r^n < q^n \) and, since \( P \) is \hyperref[def:closed_ordered_subset]{downward closed}, we have \( r^n \in P \) and hence \( r \in \sqrt[n]{ P } \). Thus, since it also contains all nonpositive numbers, \( \sqrt[n]{ P } \) is downward closed.

  Furthermore, suppose that \( q_0 \) is the supremum of \( \sqrt[n]{ P } \) and that \( q_0 \) does not belong to \( \sqrt[n]{ P } \). Then \( q_0^n \) is not in \( P \), but it is a supremum of \( P \). But \( P \) must contain its supremum. The obtained contradiction shows that, if \( \sqrt[n]{ P } \) has a supremum, the latter must belong to \( \sqrt[n]{ P } \).

  \Fullref{thm:def:dedekind_macnielle_closure/closed_elements_totally_ordered} then implies that \( \sqrt[n]{ P } \) is Dedekind-MacNeille closed.
\end{defproof}

\begin{proposition}\label{thm:nth_root_is_irrational}
  For any \hyperref[def:prime_number]{prime number} \( p \), consider its embedding into the \hyperref[def:real_numbers]{real numbers}
  \begin{equation*}
    P = \set{ s \in \BbbQ \given s \leq p }.
  \end{equation*}

  We claim that, for integers \( n > 2 \), the \( n \)-th \hyperref[def:nth_root]{root} \( \sqrt[n]{ P } \) is \hyperref[def:irrational_numbers]{irrational}.
\end{proposition}
\begin{proof}
   If \( \sqrt[n]{ P } \) is rational, then by definition it has a supremum. Denote this supremum by \( s \). Then \( s^n \) belongs to \( P \). Furthermore, \( s^n \) must be the supremum of \( P \) due to \fullref{thm:def:rational_number_ordering/power_monotone}. But \( p \) is the supremum of \( P \), and \fullref{thm:nth_root_is_not_rational} establishes that there exists no rational number \( q \) such that \( q^n = p \). But \( p = s^n \).

   The obtained contradiction shows that \( \sqrt[n]{ P } \) is irrational.
\end{proof}

\begin{proposition}\label{thm:power_of_nth_root}
  The \( n \)-th \hyperref[def:nth_root]{root} of \( P \) satisfies
  \begin{equation}\label{eq:thm:def:nth_root}
    \parens[\Big]{ \sqrt[n]{ P } }^n = P.
  \end{equation}
\end{proposition}
\begin{proof}
  We will use induction on \( m \geq 1 \) to show that
  \begin{equation}\label{eq:thm:def:nth_root/aux}
    \sqrt[n]{ P }^m = \set{ p^m \given p > 0 \T{and} p^n \in P }^{UL}.
  \end{equation}

  The base case \( m = 1 \) is the definition of \( n \)-th root. Suppose that \eqref{eq:thm:def:nth_root/aux} holds for some fixed \( m \). If \( \sqrt[n]{ P }^m \) has a supremum, then this supremum multiplied by any positive \( q \) in \( \sqrt[n]{ P } \) does not exceed the supremum of \( \sqrt[n]{ P }^{m+1} \). Hence,
  \begin{equation*}
    \sqrt[n]{ P }^{m+1} = \sqrt[n]{ P }^m \cdot \sqrt[n]{ P } = \set[\Big]{ p^m \cdot q \given p > 0 \T{and} q > 0 \T{and} p^n \in P \T{and} q^n \in P }^{UL}
  \end{equation*}

  Note that if \( p > q \), we have \( p^{m+1} > p^m \cdot q \). Since the same restrictions are placed on both \( p \) and \( q \), any upper bound of numbers of the form \( p^m \cdot q \) is also an upper bound of numbers of the form \( p^{m+1} \) and vice versa. Thus,
  \begin{equation*}
    \sqrt[n]{ P }^{m+1}
    =
    \set[\Big]{ p^{m+1} \given p > 0 \T{and} p^n \in P }^{UL}.
  \end{equation*}

  Then \eqref{eq:thm:def:nth_root} follows for \( m = n \).
\end{proof}

\paragraph{Ceiling and floor}

\begin{definition}\label{def:real_floor_ceiling}
  For any \hyperref[def:real_numbers]{real number} \( P \), we define its \term{floor} as
  \begin{equation*}
    \floor(P) \coloneqq \max(\BbbZ \cap P)
  \end{equation*}
  and its \term{ceiling} as
  \begin{equation*}
    \ceil(P) \coloneqq \min(\BbbZ \setminus P).
  \end{equation*}
\end{definition}
\begin{defproof}
  The floor of \( P \) is well-defined because the set \( \BbbZ \cap P \) is bounded from above and thus the set of strict upper bounds is \hyperref[rem:well_founded_relation]{well-founded}. If \( n \) is the least upper bound of \( \BbbZ \cap P \), then \( n - 1 \) is the floor.

  The ceiling is well-defined because \( \BbbZ \setminus P \) is by itself well-founded and thus has a minimum.
\end{defproof}

\begin{proposition}\label{thm:real_floor_ceiling_interval}
  For any real number \( P \), we have
  \begin{equation*}
    \floor(P) \subseteq P \subseteq \ceil(P).
  \end{equation*}

  \begin{thmenum}
    \thmitem{thm:real_floor_ceiling_interval/integer} If \( P \) is an integer, then
    \begin{equation*}
      \floor(P) = P = \ceil(P).
    \end{equation*}

    \thmitem{thm:real_floor_ceiling_interval/non_integer} If \( P \) is not an integer, then
    \begin{equation*}
      \floor(P) \subsetneq P \subsetneq \ceil(P).
    \end{equation*}
  \end{thmenum}
\end{proposition}
\begin{proof}
  If \( n \) is the ceiling, i.e. the smallest integer such that \( P \subseteq \iota(n) \), then \( \iota(n - 1) \subseteq P \) as it is smaller.

  Furthermore, if \( \iota(m) \subseteq P \), either \( \iota(m) = \iota(n) \) and thus \( P = \iota(n) \), or \( \iota(m) \leq \iota(n - 1) \). Hence, either \( \iota(n) \) or \( \iota(n - 1) \) is the floor.
\end{proof}

\paragraph{Polynomials}

\begin{proposition}\label{thm:nth_root_polynomial}
  The \( n \)-th \hyperref[def:nth_root]{root} \( \sqrt[n]{ P } \) of a nonnegative \hyperref[def:real_numbers]{real number} \( P \) is a \hyperref[def:polynomial_root]{polynomial root} of the \hyperref[def:polynomial_algebra/polynomial]{polynomial} \( X^n - P \).

  If \( n \) is even, the additive inverse \( -\sqrt[n]{ P } \) is also a polynomial root.
\end{proposition}
\begin{comments}
  \item This proposition highlights how the phrase \enquote{the root of \( P \)} is possibly non-unique unless we clarify what the word \enquote{root} means.
\end{comments}
\begin{proof}
  Follows from \fullref{thm:power_of_nth_root}.
\end{proof}

\begin{proposition}\label{thm:x2_plus_one_no_root}
  The \hyperref[def:polynomial_algebra/polynomial]{polynomial} \( X^2 + 1 \) has no \hyperref[def:polynomial_root]{root} over the \hyperref[def:real_numbers]{real numbers}.
\end{proposition}
\begin{comments}
  \item We denote here by \( 1 \) the integer \( 1 \) and by \( I \) the corresponding real number. We take a polynomial over the integers and show that it does not have a root over the reals.
\end{comments}
\begin{proof}
  For every real number \( P \), by \hyperref[def:binary_relation/trichotomic]{trichotomy}, we have the following possibilities for \( P^2 \):
  \begin{itemize}
    \item If \( P = O \), then \( P^2 = O^2 = 0 \).
    \item If \( P \supsetneq O \), then \( P^2 \supsetneq O \).
    \item If \( P \subsetneq O \), then again \( P^2 = (-P)^2 \supsetneq O \).
  \end{itemize}

  In all cases, \( P^2 \supsetneq O \), and thus \( P^2 + I \supsetneq 1 \). Therefore, \( P \) cannot be a root of the (integer) polynomial \( X^2 + 1 \).

  Since \( P \) was arbitrary, we conclude that the polynomial \( X^2 + 1 \) has no real root.
\end{proof}

\begin{proposition}\label{thm:reals_not_algebraically_closed}
  The field \( \BbbR \) of \hyperref[def:real_numbers]{real numbers} is not \hyperref[def:algebraically_closed_field]{algebraically closed}.
\end{proposition}
\begin{proof}
  Follows from \fullref{thm:x2_plus_one_no_root}.
\end{proof}

  \section{Complex numbers}\label{sec:complex_numbers}

\Fullref{thm:ordered_field_not_algebraically_closed} implies that the \hyperref[def:field]{field} \( \BbbR \) of \hyperref[def:real_numbers]{real numbers} is not \hyperref[def:algebraically_closed_field]{algebraically closed}. This motivates the introduction of complex numbers.

\paragraph{Complex numbers}

\begin{definition}\label{def:complex_numbers}\mimprovised
  We define the \hyperref[def:field]{field} \( \BbbC \) of \term[bg=комплексни числа (\cite[296]{ИлинСадовничиСендов1984АнализТом1}), ru=комплексные числа (\cite[12]{Маркушевич1967АналитическиеФункцииТом1}), en=complex numbers (\cite[1]{Ahlfors1979ComplexAnalysis})]{complex numbers} as the \hyperref[def:algebra_over_ring/quotient]{quotient algebra}
  \begin{equation}\label{eq:def:complex_numbers/quotient}
    \BbbC \coloneqq \BbbR[i] / \braket{ i^2 + 1 }.
  \end{equation}

  It is indeed a field because, by \fullref{thm:axn_byn_irreducible} and \fullref{thm:homogeneous_polynomial_constant}, the polynomial \( i^2 + 1 \) is irreducible in \( \BbbR \) and, by \fullref{thm:quotient_by_irreducible_polynomial}, \( \BbbC \) is a \hyperref[def:field/submodel]{field extension} of \( \BbbR \) of \hyperref[def:field_extension_degree]{degree} \( 2 \).

  Here \( i \) is an indeterminate with no inherent semantics --- its semantics stem from the algebraic properties of the defined quotient. \Fullref{thm:representatives_in_univariate_polynomial_quotient_set} allows us to unambiguously identify each coset in \( \BbbC \) with a canonical representative; thus, we regard members of \( \BbbC \) as linear polynomials with real coefficients:
  \begin{equation}\label{eq:def:complex_numbers/number}
    z = a + bi.
  \end{equation}

  The constant coefficient is written first by convention since we embed \( \BbbR \) as constant polynomials.

  We denote the constant coefficient of \( z \) by \( \real z \) and call it the \term[bg=реална част (\cite[296]{ИлинСадовничиСендов1984АнализТом1}), ru=действительная часть (\cite[12]{Маркушевич1967АналитическиеФункцииТом1}), en=real (part) (\cite[1]{Ahlfors1979ComplexAnalysis})]{real part}. We denote the linear coefficient by \( \imag z \) and call it the \term[bg=имагинерна част (\cite[296]{ИлинСадовничиСендов1984АнализТом1}), ru=мнимая часть (\cite[12]{Маркушевич1967АналитическиеФункцииТом1}), en=imaginary part (\cite[1]{Ahlfors1979ComplexAnalysis})]{imaginary part}. If \( \real z = 0 \), we say that \( z \) is \term[ru=чисто мнимое (число) (\cite[12]{Маркушевич1967АналитическиеФункцииТом1}), en=purely imaginary (number) (\cite[1]{Ahlfors1979ComplexAnalysis})]{purely imaginary}. We call \( i \) the \term[en=imaginary unit (\cite[1]{Ahlfors1979ComplexAnalysis}), ru=мнимая единица (\cite[12]{Маркушевич1967АналитическиеФункцииТом1})]{imaginary unit}.

  Since, by definition of quotient ring, we have \( i^2 + 1 = 0 \), it follows that
  \begin{equation}\label{eq:def:complex_numbers/i_square}
    i^2 = -1.
  \end{equation}

  Multiplication in \( \BbbC \) is thus given by
  \begin{equation}\label{eq:def:complex_numbers/multiplication}
    (a + bi) (c + di) = ac + adi + bci + bdi^2 = (ac - bd) + (ad + bc) i.
  \end{equation}
\end{definition}
\begin{comments}
  \item We consider the following an indispensable part of \( \BbbR \):
  \begin{itemize}
    \item Complex conjugation defined in \fullref{def:complex_conjugation}.
    \item The absolute value defined in \fullref{def:complex_absolute_value}, as well as the induced topology.
    \item Trigonometric forms of complex numbers defined in \fullref{def:complex_numbers_trigonometric_form}.
    \item The \hyperref[def:algebraically_closed_field]{algebraic closure} shown in \fullref{thm:fundamental_theorem_of_algebra}.
  \end{itemize}

  \item As a consequence of \fullref{thm:ordered_field_not_algebraically_closed}, the complex numbers cannot be \hyperref[def:totally_ordered_set]{totally ordered} in a way that would make it an \hyperref[def:ordered_semiring]{ordered ring}.

  \item The definition as a quotient ring is given after knowledge of complex numbers is already assumed, for which reason complex numbers are often defined in a more elementary form --- by endowing \( \BbbR^2 \) with multiplication defined by \eqref{eq:def:complex_numbers/multiplication}. The quotient ring construction is given in \incite[example III.4.8]{Aluffi2009Algebra}, where it is shown to be equivalent to the vector space construction.

  We discuss an alternative definition via matrices in \fullref{thm:complex_numbers_as_matrices}.
\end{comments}

\begin{proposition}\label{thm:complex_numbers_as_matrices}
  The subset of the \hyperref[thm:matrix_algebra]{matrix algebra} \( \BbbR^{2 \times 2} \) of matrices of the form
  \begin{equation}\label{eq:thm:complex_numbers_as_matrices}
    \begin{pmatrix}
      a  & b \\
      -b & a
    \end{pmatrix}
  \end{equation}
  is a \hyperref[def:algebra_over_ring/submodel]{subalgebra} and is isomorphic to the \( \BbbR \)-algebra \( \BbbC \) of \hyperref[def:complex_numbers]{complex numbers}.
\end{proposition}
\begin{proof}
  Denote by \( C \) the subset of all matrices of the form \eqref{eq:thm:complex_numbers_as_matrices}.

  It is closed under addition and scalar multiplication and contains both the additive and multiplicative identities. It is also closed under additive inverses. Therefore, \( C \) is a vector subspace of \( \BbbR^{2 \times 2} \). The function sending \( a + bi \) to \eqref{thm:complex_numbers_as_matrices} is a vector space isomorphism from \( \BbbC \) to \( C \).

  Furthermore, we have
  \begin{equation*}
    \begin{pmatrix}
      a  & b \\
      -b & a
    \end{pmatrix}
    \cdot
    \begin{pmatrix}
      c  & d \\
      -d & c
    \end{pmatrix}
    =
    \begin{pmatrix}
      ac - bd  & ad + bc \\
      -ad - bc & -bd + ac.
    \end{pmatrix}
  \end{equation*}

  Thus, \( C \) is closed under multiplication, making it a subalgebra of \( \BbbR^{2 \times 2} \). Comparing the product with \eqref{eq:def:complex_numbers/multiplication}, we conclude that \( C \) is isomorphic to \( \BbbC \) as an algebra.
\end{proof}

\paragraph{Complex conjugation}

\begin{definition}\label{def:complex_conjugation}\mcite[]{Маркушевич1967АналитическиеФункцииТом1}
  We define the \term[bg=(комплексно) спрегнато (число) (\cite[298]{ИлинСадовничиСендов1984АнализТом1}), ru=комплексно сопряжённые (числа), en=complex conjugate (\cite[7]{Ahlfors1979ComplexAnalysis})]{complex conjugate} of the \hyperref[def:complex_numbers]{complex number} \( a + bi \) as \( a - bi \).
\end{definition}
\begin{comments}
  \item It follows from \fullref{thm:quadratic_conjugate_algebraic_element} that, when \( b \) is nonzero, \( a \pm bi \) are \hyperref[def:conjugate_algebraic_element]{conjugate algebraic elements} over \( \BbbR \).
\end{comments}

\begin{proposition}\label{thm:def:complex_conjugation}
  \hyperref[def:complex_conjugation]{Complex conjugation} has the following basic properties:
  \begin{thmenum}
    \thmitem{thm:def:complex_conjugation/components} We have
    \begin{align*}
      \real z = \frac {z + \oline z} 2
      &&\T{and}&&
      \imag z = \frac {z - \oline z} {2i}.
    \end{align*}

    \thmitem{thm:def:complex_conjugation/automorphism} Conjugation is a \hyperref[def:field/homomorphism]{field automorphism} of the complex numbers.
    \thmitem{thm:def:complex_conjugation/fixed} The \hyperref[def:function_fixed_point]{fixed points} under complex conjugation are the \hyperref[def:real_numbers]{real numbers}.
  \end{thmenum}
\end{proposition}
\begin{proof}
  \SubProofOf{thm:def:complex_conjugation/components} We have
  \begin{equation*}
    \frac {a + bi + a - bi} 2 = a = \real(a + bi)
  \end{equation*}
  and similarly
  \begin{equation*}
    \frac {a + bi - a + bi} {2i} = b = \imag(a + bi)
  \end{equation*}

  \SubProofOf{thm:def:complex_conjugation/automorphism} Straightforward.
  \SubProofOf{thm:def:complex_conjugation/fixed} Every real number is clearly a fixed point.

  Conversely, we can cancel \( a \) in \( a + bi = a - bi \) to obtain \( 2bi = 0 \). Since \( 2 \neq 0 \) and \( i \neq 0 \), it remains for \( b = 0 \).
\end{proof}

\paragraph{Absolute values}

\begin{definition}\label{def:absolute_value}\mcite[def. 9.1]{Jacobson1989BasicAlgebraII}
  An \term{absolute value} on a \hyperref[def:field]{field} \( \BbbK \) is a map \( \abs{\anon}: \BbbK \to \BbbR \) that satisfies the following properties:
  \begin{thmenum}
    \thmitem{def:absolute_value/positive_definite} \hyperref[def:real_function_definiteness]{Positive definiteness}: \( \abs{x} \geq 0 \) for all \( x \) and \( \abs{x} = 0 \) if and only if \( x = 0_\BbbK \).
    \thmitem{def:absolute_value/multiplicative} \hyperref[def:multiplicative_function]{Multiplicativity}: \( \abs{xy} = \abs{x} \cdot \abs{y} \).
    \thmitem{def:absolute_value/subadditive} \hyperref[def:additive_function/sub]{Subadditivity}: \( \abs{x + y} \leq \abs{x} + \abs{y} \).
  \end{thmenum}
\end{definition}
\begin{comments}
  \item Absolute values are used to define \hyperref[def:norm]{norms}, however they are also special cases of norms.
  \item Unless otherwise noted, for complex numbers we will assume that \( \abs{\anon} \) refers to the absolute value defined in \fullref{def:complex_absolute_value}.
\end{comments}

\begin{definition}\label{def:complex_absolute_value}
  We define an \hyperref[def:absolute_value]{absolute value} for \hyperref[def:complex_numbers]{complex numbers} as
  \begin{equation}\label{eq:def:complex_absolute_value}
    \abs{z} \coloneqq \sqrt{z \cdot \oline z},
  \end{equation}
  where \( \oline z \) is the \hyperref[def:complex_conjugation]{complex conjugate} of \( z \).

  More explicitly, if \( z = a + bi \), then
  \begin{equation}\label{eq:def:complex_absolute_value/algebraic}
    \abs{a + bi} = \sqrt{(a + bi) \cdot (a - bi)} = \sqrt{a^2 + b^2}.
  \end{equation}
\end{definition}
\begin{comments}
  \item We will define square roots for general complex numbers in \fullref{def:principal_real_square_root}. At this point we only need square roots for real numbers as defined in \fullref{def:principal_nonnegative_nth_root}.
\end{comments}
\begin{defproof}
  \SubProofOf[def:real_function_definiteness]{positive definiteness} Clearly \( \abs{0} = 0 \). If \( a \neq 0 \), then \fullref{thm:ordered_ring_power} implies that \( a^2 \) is positive and \( b^2 \) is nonnegative, while \fullref{thm:def:ordered_semiring/strict_sum} implies that their sum is positive. Similarly, if \( b \neq 0 \), the sum \( a^2 + b^2 \) is strictly positive.

  \SubProofOf[def:absolute_value/multiplicative]{multiplicativity} We have
  \begin{balign*}
    \abs{(a + bi) \cdot (c + di)}
    &=
    \abs{(ac - bd) + (ad + bc)i}
    = \\ &=
    \sqrt{(ac - bd)^2 + (ad + bc)^2}
    = \\ &=
    \sqrt{a^2 c^2 - \cancel{2abcd} + b^2 d^2 + a^2 d^2 + \cancel{2abcd} + b^2 c^2}
    = \\ &=
    \sqrt{a^2 (c^2 + d^2) + b^2 (d^2 + c^2)}
    = \\ &=
    \sqrt{(a^2 + b^2) \cdot (c^2 + d^2)}
    \reloset {\eqref{eq:thm:def:principal_nonnegative_nth_root/multiplicative}} = \\ &=
    \sqrt{a^2 + b^2} \cdot \sqrt{c^2 + d^2}
    = \\ &=
    \abs{a + bi} \cdot \abs{c + di}.
  \end{balign*}

  \SubProofOf[def:additive_function/sub]{subadditivity} We have
  \begin{balign*}
    \abs{(a + bi) + (c + di)}
    &=
    (a + c)^2 + (b + d)^2
    = \\ &=
    (a^2 + b^2) + (c^2 + d^2) + 2(ac + bd)
    = \\ &=
    \abs{a + bi} + \abs{c + di} + 2(ac + bd).
  \end{balign*}
\end{defproof}

\begin{proposition}\label{thm:complex_multiplicative_inverse}
  The \hyperref[def:semiring]{multiplicative inverse} of the nonzero \hyperref[def:complex_numbers]{complex number} \( z = a + bi \) is
  \begin{equation}\label{eq:thm:complex_multiplicative_inverse}
    \frac {\oline z} {{\abs z}^2} = \frac {a - bi} {\sqrt{a^2 + b^2}}
  \end{equation}
\end{proposition}
\begin{proof}
  Trivial considering how we defined the complex absolute value.
\end{proof}

\paragraph{Square roots}

\begin{definition}\label{def:principal_real_square_root}\mimprovised
  We extend \hyperref[def:nth_root]{principal square roots} as defined in \fullref{def:principal_nonnegative_nth_root} to arbitrary \hyperref[def:real_numbers]{real numbers} by defining, for any \( x > 0 \),
  \begin{equation}\label{eq:def:principal_real_square_root/negative}
    \sqrt {-x} \coloneqq i \sqrt x.
  \end{equation}
\end{definition}
\begin{comments}
  \item We avoid defining square roots for arbitrary complex numbers, or every higher degree roots for negative real numbers, since it is difficult to make a decision which root should be principal. \incite[16]{АлександровМаркушевичХинчинИПр1951ЭнциклопедияТом1} defines principal roots for any complex number based on their \hyperref[def:complex_numbers_trigonometric_form]{trigonometric form}, however we prefer to avoid such a choice.
\end{comments}

\begin{proposition}\label{thm:real_quadratic_polynomial_roots}
  The \hyperref[def:root_of_polynomial]{roots} of the \hyperref[def:real_numbers]{real} \hyperref[def:polynomial_degree_terminology]{quadratic} \hyperref[def:polynomial_algebra]{polynomial} \( p(X) = a X^2 + b X + c \) are
  \begin{equation}\label{eq:thm:real_quadratic_polynomial_roots}
    \frac {-b \pm \sqrt{b^2 - 4ac}} {2a}.
  \end{equation}
\end{proposition}
\begin{proof}
  Follows from \fullref{thm:quadratic_polynomial_roots}.
\end{proof}

\begin{corollary}\label{thm:real_quadratic_discriminant}
  Fix a real quadratic polynomial \( p(X) = a X^2 + b X + c \). Based on its \hyperref[def:discriminant]{discriminant} \( D(p) = b^2 - 4ac \), we can infer the following:
  \begin{thmenum}
    \thmitem{thm:real_quadratic_discriminant/pos} \( D(p) > 0 \) if and only if \( p(X) \) has two simple real roots.
    \thmitem{thm:real_quadratic_discriminant/eq} \( D(p) = 0 \) if and only if \( p(X) \) has a real double root.
    \thmitem{thm:real_quadratic_discriminant/neg} \( D(p) < 0 \) if and only if \( p(X) \) has two simple purely imaginary roots.
  \end{thmenum}
\end{corollary}

\paragraph{Algebraic numbers}

In this paragraph, we also regard real numbers as atoms and not as lower cuts.

\begin{definition}\label{def:algebraic_number}\mcite[222]{ГеновМиховскиМоллов1991Алгебра}
  We say that a \hyperref[def:complex_numbers]{complex number} is \term[bg=алгебрично (число), ru=алгебраическое (число) (\cite[358]{Курош1968КурсВысшейАлгебры}), en=algebraic (number) (\cite[16]{Carothers2000RealAnalysis})]{algebraic} if it is an \hyperref[def:algebraic_element]{algebraic element} over the field \( \BbbQ \) of \hyperref[def:rational_numbers]{rational numbers}.
\end{definition}

\begin{definition}\label{def:transcendental_number}\mcite[222]{ГеновМиховскиМоллов1991Алгебра}
  We say that a \hyperref[def:complex_numbers]{complex number} is \term[bg=трансцендентно (число), ru=трансцендентное (число) (\cite[358]{Курош1968КурсВысшейАлгебры}), en=transcendental (number) (\cite[277]{Jacobson1985BasicAlgebraI})]{algebraic} if it is not \hyperref[def:algebraic_number]{algebraic}.
\end{definition}

\begin{theorem}[Lindemann-Weierstrass theorem]\label{thm:lindemann_weierstrass}\mcite[277]{Jacobson1985BasicAlgebraI}
  Let \( \alpha_1, \ldots, \alpha_n \) be \hyperref[def:complex_numbers]{complex numbers} that are \hyperref[def:algebraic_number]{algebraic} over \( \BbbQ \). If they are \hyperref[def:linear_dependence]{linearly independent} over \( \BbbQ \), their complex exponentials \( e^{u_1}, \ldots, e^{u_n} \) are \hyperref[def:algebraic_dependence]{algebraically independent} over \( \BbbQ \).
\end{theorem}
\begin{comments}
  \item We will not prove this theorem. A proof can be found in \cite[277]{Jacobson1985BasicAlgebraI}.
\end{comments}

\begin{corollary}\label{thm:eulers_constant_is_transcendental}
  \hyperref[def:exponential_function]{Euler's constant} \( e \) is \hyperref[def:transcendental_number]{transcendental}.
\end{corollary}
\begin{proof}
  Since \( 1 \) is by itself linearly independent over \( \BbbQ \), \fullref{thm:lindemann_weierstrass} implies that \( e \) is by algebraically independent over \( \BbbQ \). Hence, there exists no rational polynomial whose root is \( e \), and thus \( e \) satisfies \fullref{def:transcendental_element}.
\end{proof}

\begin{corollary}\label{thm:pi_is_transcendental}\mcite[454]{Knapp2016BasicAlgebra}
  The number \hyperref[def:pi]{\( \pi \)} is \hyperref[def:transcendental_number]{transcendental}.
\end{corollary}
\begin{proof}
  Suppose that \( \pi \) is algebraic over \( \BbbQ \).

  The complex unit \( i \) is algebraic by construction --- it has minimal polynomial \( X^2 + 1 \). Then \( i\pi \) is also algebraic because it belongs to \( \BbbQ[\pi][i] \), which must be finite-dimensional.

  Furthermore, \( i\pi \) is by itself linearly independent over \( \BbbQ \) --- it can only be linearly dependent if it is the zero vector, which it is not. So \fullref{thm:lindemann_weierstrass} implies that \( e^{i\pi} \) is transcendental. But \( e^{i\pi} \) equals \( -1 \), which is a root of the integer polynomial \( X + 1 \).

  The obtained contradiction shows that our initial assumption of \( \pi \) being algebraic over \( \BbbQ \) is false; that is, \( \pi \) must be transcendental.
\end{proof}

\begin{example}\label{ex:polynomials_over_pi}
  \Fullref{thm:pi_is_transcendental} implies that the polynomial ring \( \BbbQ[X] \) can be embedded into \( \BbbR \) via \( \Phi_\pi: \BbbQ[X] \to \BbbR \). We can thus identify a rational polynomial
  \begin{equation*}
    f(X) = \sum_{i=0}^n a_k X^k
  \end{equation*}
  with the number
  \begin{equation*}
    f(\pi) = \sum_{i=0}^n a_k \pi^k.
  \end{equation*}
\end{example}

\paragraph{Fundamental theorem of algebra}

\begin{theorem}[Fundamental theorem of algebra]\label{thm:fundamental_theorem_of_algebra}
  The field \( \BbbC \) of \hyperref[def:complex_numbers]{complex numbers} is \hyperref[def:algebraically_closed_field]{algebraically closed}.
\end{theorem}
\begin{proof}
  \todo{Prove}
\end{proof}


  \chapter{Real analysis}\label{ch:real_analysis}

Real analysis is concerned with \hyperref[def:function]{functions} with values in \hyperref[def:euclidean_space]{Euclidean spaces}. These include the real line \( \BbbR \), plane \( \BbbR^2 \) or space \( \BbbR^3 \), which are the classic spaces from \fullref{sec:euclidean_plane}.

Important aspects of real analysis include:
\begin{itemize}
  \item \hyperref[def:local_continuity]{Continuity}, which we discuss extensively in \fullref{ch:general_topology} and \fullref{ch:metric_spaces}, and aggregate in \fullref{sec:topology_of_euclidean_spaces} and \fullref{sec:real_convergence}.

  \item \hyperref[def:differentiability]{Differentiability}, which we discuss in \fullref{sec:differentiability} from \fullref{ch:functional_analysis} and discuss briefly in \fullref{sec:differentiability}, and, in a generalized form, in \fullref{sec:nonsmooth_derivatives}.

  \item Integration, which we discuss in \fullref{sec:riemann_integration}.

  \item Special functions, which we delegate to \fullref{ch:complex_analysis} via \fullref{sec:special_functions}.

  \item Convex functions, which we discuss in \fullref{sec:convex_functions}.
\end{itemize}

  \section{Topology of Euclidean spaces}\label{sec:topology_of_euclidean_spaces}

\begin{theorem}\label{thm:real_metric_and_order_topologies_coincide}
  For the real numbers, the \hyperref[def:metric_topology]{metric} and \hyperref[def:order_topology]{order topologies} coincide.
\end{theorem}
\begin{proof}
  The metric topology \( \mscrT_M \) is generated by the \hyperref[def:topological_base]{base}
  \begin{equation*}
    \mathcal{B} \coloneqq \{ B(x, r) \colon x \in \BbbR, r \in \BbbR_{>0} \}
  \end{equation*}
  and the order topology \( \mscrT_O \) is generated by the \hyperref[def:topological_subbase]{subbase}
  \begin{equation*}
    \mathcal{P} \coloneqq \{ (a, \infty) \colon a \in \BbbR \} \cup \{ (\infty, b) \colon b \in \BbbR \}.
  \end{equation*}

  The inclusion \( \mathcal{B} \subseteq FI(\mathcal{P}) \) is obvious since any ball \( B(x, r) \) is the intersection of the two rays
  \begin{equation*}
    B(x, r) = (x - r, \infty) \cap (-\infty, x + r).
  \end{equation*}

  Thus, \( \mscrT_M \subseteq T_O \). We now only need to show that \( \mathcal{B} \) is a base for \( \mscrT_O \).

  Let \( U \in T_O \). Since \( FI(\mathcal{P}) \) is a base for \( \mscrT_O \), there \hyperref[def:topological_base/union]{exists} a family \( \{ U_i \}_{i \in I} \subseteq FI(\mathcal{P}) \) such that
  \begin{equation*}
    U = \bigcup_{i \in I} U_i.
  \end{equation*}

  We only need to express every \( U_i \) as a union of balls from \( \mathcal{B} \). There are several possibilities:
  \begin{itemize}
    \item if \( U_i \) is the open interval \( (a, \infty) \),
          \begin{equation*}
            (a, \infty) = \bigcup_{i=1}^\infty B(a + i, 1).
          \end{equation*}

    \item if \( U_i \) is the open interval \( (-\infty, b) \),
          \begin{equation*}
            (-\infty, b) = \bigcup_{i=1}^\infty B(b - i, 1).
          \end{equation*}

    \item if \( U_i \) is the intersection \( (a, \infty) \cap (-\infty, b), a < b \),
          \begin{equation*}
            (a, \infty) \cap (-\infty, b) = B(\tfrac {a + b} 2, \tfrac {b - a} 2)
          \end{equation*}

    \item if \( U_i \) is the empty set,
          \begin{equation*}
            \varnothing = \bigcup \varnothing \text{ (see \cref{def:basic_set_operations/union})}.
          \end{equation*}
  \end{itemize}

  Thus, \( U_i \) is the union of an at most countable amount of balls. The countable union of countable sets is again countable, hence by \cref{def:topological_base/union}, \( \mathcal{B} \) is a base for \( \mscrT_O \).
\end{proof}

\begin{proposition}\label{thm:rn_bounded_iff_totally_bounded}
  A set in \( \BbbR^n \) is totally \hyperref[def:totally_bounded_set]{bounded} if and only if it is \hyperref[def:metric_space/bounded_set]{bounded}.
\end{proposition}
\begin{proof}
  \SufficiencySubProof Follows from \cref{thm:totally_bounded_sets_are_bounded}.
  \NecessitySubProof Let \( A \) be a bounded set in \( \BbbR^n \) and let \( B(x, r) \) be a ball containing \( A \). Fix \( \varepsilon > 0 \).

  Denote by \( e_1, \ldots, e_n \) the \hyperref[def:hamel_basis]{basis} of \( \BbbR^n \). Denote by \( m \) the smallest integer such that \( m \varepsilon \geq r \).

  We can create a grid around \( B(x, r) \) as follows:

  Define the set
  \begin{equation*}
    \left\{ x + \sum_{i=1}^n [k_i \varepsilon] e_i \colon \forall i = 1, \ldots, n: k_i = 1, \ldots, m \right\}.
  \end{equation*}

  is finite. Furthermore, it is an \( \varepsilon \)-net of \( A \). Indeed, let \( y \in A \). Denote its coordinates along \( e_1, \ldots, e_n \) by \( y_1, \ldots, y_n \). Then \( y \) is contained in the ball
  \begin{equation*}
    B\left(x + \sum_{i=1}^n [\ceil(y_i) \varepsilon] e_i, \varepsilon \right).
  \end{equation*}
\end{proof}

\begin{theorem}[Heine-Borel theorem]\label{thm:heine_borel}
  A set in \( \BbbR^n \) is compact in the sense of \cref{def:compact_space} if and only if it is closed and bounded.
\end{theorem}
\begin{proof}
  Follows from \cref{thm:rn_bounded_iff_totally_bounded} and \cref{thm:complete_metric_space_compact_conditions/closed_totally_bounded}.
\end{proof}

\begin{proposition}\label{thm:real_supremum_of_closure}
  The supremum (resp. infimum) of a set \( A \subseteq \BbbR \), if it exists, is equal to the supremum (resp. infimum) of \( \cl A \).
\end{proposition}
\begin{proof}
  \SufficiencySubProof Denote by \( M \) the supremum of \( A \). Assume that it is not a supremum of \( \cl A \), that is, there exists an upper bound \( M' \) of \( \cl A \) such that \( M' < M \). But this is impossible because \( A \subseteq \cl A \).

  Therefore, \( M \) is a supremum of \( \cl A \).

  \NecessitySubProof Denote by \( M \) the supremum of \( \cl A \). Assume that it is not a supremum of \( A \), that is, there exists an upper bound \( M' \) of \( A \) such that \( M' < M \).

  Let \( \{ x_i \}_{i=1}^\infty \subseteq A \) be a sequence that converges to \( M \). Then
  \begin{equation*}
    x_i < M' < M.
  \end{equation*}

  By \cref{thm:squeeze_lemma/sequences}, we have \( M' = M \), which contradicts our choice of \( M' \). Thus, \( M \) is the supremum of \( A \).
\end{proof}

\begin{proposition}\label{thm:real_bounded_set_has_supremum}
  Every nonempty \hyperref[def:metric_space/bounded_set]{bounded set} in \( \BbbR \) has a supremum and infimum.
\end{proposition}
\begin{proof}
  Let \( A \subseteq \BbbR \) be a nonempty bounded set. By \fullref{thm:heine_borel}, the set \( \cl A \) is compact. By \fullref{thm:weierstrass_extreme_value_theorem}, the identity function \( \id: \BbbR \to \BbbR \) attains its minimum \( m \) and maximum \( M \) on \( \cl A \). Note that both \( m \) and \( M \) do not have to belong to \( A \), but \( m \) is a lower bound and \( M \) is an upper bound of the set \( A \).

  If we take any other upper bound \( M' \) of \( A \), then by \cref{thm:real_supremum_of_closure},
  \begin{equation*}
    M' \geq \sup A = \sup \cl A = M.
  \end{equation*}

  Hence, \( M \) is the least upper bound of \( A \).

  We can analogously prove that \( m \) is the greatest lower bound of \( A \).
\end{proof}

  \section{Real-valued functions}\label{sec:real_valued_functions}

\paragraph{Definiteness}

\begin{definition}\label{def:real_function_definiteness}\mimprovised
  We say that a real-valued function \( f: X \to \BbbR \) on an \hyperref[con:additive_semigroup]{additive semigroup} \( X \) is \term[bg=положително положително дефинитна (квадратична форма) (\cite[86]{Обрешков1962ВисшаАлгебра}), ru=положительно определённая (квадратичная форма) (\cite[def. 1.4.7]{Кострикин2000АлгебраТом2}), en=positive definite (function) (\cite[17; 271]{Clarke2013OptimalControl})]{positive definite} if \( f(0_X) = 0 \) and \( f(x) > 0 \) for all nonzero \( x \).

  If we allow \( f(x) = 0 \) for nonzero \( x \), we instead say that \( f \) is \term[ru=положительно полуопрделённая (квадратичная форма) (\cite[def. 1.4.7]{Кострикин2000АлгебраТом2}), en=positive semidefinite (bilinear form) (\cite[107]{Knapp2016BasicAlgebra})]{positive semidefinite}.

  If the inequality is reversed, i.e. if \( f(x) < 0 \), we instead say that the function is \term[bg=отрицателно дефинитна (квадратична форма) (\cite[86]{Обрешков1962ВисшаАлгебра}), ru=отрицательно определённая (квадратичная форма) (\cite[def. 1.4.7]{Кострикин2000АлгебраТом2}), en=negative (semi)definite (bilinear form) (\cite[107]{Knapp2016BasicAlgebra})]{negative (semi)definite}.
\end{definition}

\begin{remark}\label{rem:real_function_definiteness_terminology}
  The terminology regarding \hyperref[def:real_function_definiteness]{function definiteness} is given in very different levels of generality across the literature:
  \begin{itemize}
    \item \incite[17; 271]{Clarke2013OptimalControl} gives inline definitions for positive definite real-valued functions (as is our case).

    \item \incite[def. 1.4.7]{Кострикин2000АлгебраТом2}, \incite[152]{Фаддеев1984Алгебра} and \incite[208]{Treil2017LinearAlgebra} gives an explicit definition for positive and negative (semi)definite finite dimensional \hyperref[thm:quadratic_forms]{quadratic forms}. Kostrikin require the forms to be nondegenerate.

    \item \incite[217]{Винберг2014Алгебра} and \incite[86]{Обрешков1962ВисшаАлгебра} give an explicit definition for both positive definite and negative definite real-valued quadratic forms.

    \item \incite[\S 25.2]{Тыртышников2007ЛинейнаяАлгебра} gives an explicit definition for positive definite complex-valued matrices. Later in \cite[\S 33.6], he also gives a definition for positive semidefinite (complex-valued) matrices.

    \item \incite[107]{Knapp2016BasicAlgebra} gives explicit definitions for both positive and negative (semi)definite self-adjoint linear operators.

    \item \incite[374]{FriedbergInselSpence2018LinearAlgebra} give explicit definitions for positive definite and semidefinite self-adjoint linear operators and matrices.

    \item \incite[578]{Lang2002Algebra} gives an explicit definition for both positive definite and negative definite finite-dimensional symmetric bilinear forms.

    \item \incite[exerc. 6.4.1]{Jacobson1985AlgebraPart1} and \incite[def. 8.1.1]{Berger1987GeometryVol1} give inline definitions for positive definite symmetric bilinear forms. \incite*[378]{Berger1987GeometryVol1} also gives an inline definition for positive definite quadratic forms.

    \item \incite[230]{Roman2005LinearAlgebra} give an explicit definition for \enquote{positive} (what we call positive semidefinite) positive definite self-adjoint linear operators.

    \item \incite[295]{Тагамлицки1971Диф} gives an explicit definition for positive definite two-dimensional quadratic forms.

    \item \incite[def. 6.5.1]{Savage2008Computability} gives an explicit definition for positive definite real-valued symmetric \hyperref[def:array/matrix]{matrices}.

    \item \incite[246]{Strang2023LinearAlgebra} gives an explicit definition for positive semidefinite and semidefinite real-valued symmetric matrices in terms of their eigenvalues.

    \item \incite[59]{DontchevRockafellar2014SolutionMappings} give an inline definition for \hyperref[def:affine_operator]{affine maps}.
  \end{itemize}
\end{remark}

\paragraph{Homogeneous functions}

\begin{definition}\label{def:real_homogeneous_function}\mimprovised
  For real-valued functions we can extend the definition of homogeneous function from \fullref{def:homogeneous_function}. Fix \hyperref[def:topological_vector_space]{topological vector spaces} \( X \) and \( Y \).

  We say that a function \( f: X \to Y \) is \term{homogeneous} of degree \( \alpha > 0 \) if, for every scalar \( r \) and ever vector \( x \), we have
  \begin{equation}\label{eq:def:real_homogeneous_function}
    f(rx) = r^\alpha \cdot f(x).
  \end{equation}

  Compared to \fullref{def:homogeneous_function}, we allow the degree \( \alpha \) to be an arbitrary positive real number rather than only a positive integer.

  \begin{thmenum}
    \thmitem{def:real_homogeneous_function/positive}\mcite[rem. 2.10]{HugWeil2020ConvexGeometry} We say that \( f: X \to \BbbR \) is \term{positive homogeneous} (of degree \( d \)) if \eqref{eq:def:homogeneous_function} holds for \( r \geq 0 \) (but not necessarily when \( r < 0 \)).

    \thmitem{def:real_homogeneous_function/absolute} We say that \( f: X \to \BbbR \) is \term{absolutely homogeneous} (of degree \( d \)) if
    \begin{equation}\label{eq:def:real_homogeneous_function/absolute}
      f(rx) = \abs{r}^d \cdot f(x).
    \end{equation}
  \end{thmenum}
\end{definition}
\begin{comments}
  \item We generalize this definition from \incite[def. 2.3.1]{HillePhillips1996FunctionalAnalysis}, who define homogeneous functions between \hyperref[def:topological_vector_space]{topological vector spaces} without degrees, and \incite[416]{Зорич2019АнализТом1}, who defines homogeneous real-valued functions on \hyperref[def:euclidean_space]{Euclidean spaces} with positive real-valued degrees.
\end{comments}

\begin{remark}\label{rem:positive_homogeneous_function}
  Terminology regarding \hyperref[def:real_homogeneous_function/positive]{positive} and \hyperref[def:real_homogeneous_function/absolute]{absolutely} homogeneous functions differs across the literature.
  \begin{itemize}
    \item \incite[rem. 2.10]{HugWeil2020ConvexGeometry} use \enquote{positive homogeneous} like us, but restrict themselves to Euclidean spaces.

    \item \incite[def. 7.12.2]{HillePhillips1996FunctionalAnalysis} generalize positive homogeneous functions to cones rather than the interval \( [0, \infty) \), but only for \hyperref[def:additive_function/sub]{subadditive functions}.

    \item \incite[95]{DontchevRockafellar2014SolutionMappings} use \enquote{positive homogeneity} for functions between Euclidean spaces in the case \( r > 0 \).

    \item \incite[3]{Clarke2013OptimalControl} uses \enquote{positive homogeneity} for what we call \enquote{absolute homogeneity}.

    \item \incite[416]{Зорич2019АнализТом1} uses \enquote{положительно однородная функция} (\enquote{positive homogeneous function}) for what we call \enquote{absolutely homogeneous}.

    \item \incite[80]{КанторовичАкилов1984ФункАнализ} uses \enquote{однородная функция} (\enquote{homogeneous function}) for what we call \enquote{absolutely homogeneous}.

    \item \incite[12]{МагарилИльяевТихомиров2002ВыпуклыйАнализ} use \enquote{однородная функция} (\enquote{homogeneous function}) for what we call \enquote{positively homogeneous}.
  \end{itemize}
\end{remark}

\paragraph{...}

\begin{definition}\label{def:functions_vanish_nowhere}
  Let \( \mathcal{F} \) be a family of functions from a set \( S \) to a ring \( R \). We say that \( \mathcal{F} \) \term{vanishes nowhere} if for every \( x \in S \) there exists a function \( f \in \mathcal{F} \) such that \( f(x) \neq 0_R \).
\end{definition}

\begin{definition}\label{def:epigraph}
  Let \( X \) be an arbitrary set. The \term{epigraph} of the function \( f: X \to \BbbR \) is defined as
  \begin{equation*}
    \epi f \coloneqq \{ (x, r) \in X \times \BbbR \colon r \geq f(x) \},
  \end{equation*}
\end{definition}

\begin{definition}\label{def:vector_field}
  \todo{Define vector fields}
\end{definition}

  \section{Real convergence}\label{sec:real_convergence}

\begin{theorem}[Bolzano-Weierstrass]\label{def:bolzano_weierstrass}
  Every bounded sequence in \( \BbbR \) has a \hyperref[def:net_limit_point]{convergent} \hyperref[def:sequence]{subsequence}.
\end{theorem}
\begin{proof}
  Let \( \seq{ x_k }_{k=1}^\infty \) be a bounded sequence in \( \BbbR \) and let \( a \leq b \) be a lower and upper \hyperref[def:extremal_points/bounds]{bound}, respectively. Construct the sequence \( \{ F_k \}_{k=1}^\infty \) of closed intervals as follows: define \( \alpha_1 \coloneqq a \) and \( \beta_1 \coloneqq b \) and, at step \( k = 1, 2, \ldots \), put
  \begin{balign*}
    F_k \coloneqq \begin{cases}
      [\alpha_k, \tfrac{\alpha_k+\beta_k} 2], & [\alpha_k, \tfrac{\alpha_k+\beta_k} 2]\text{ contains infinitely many sequence members}, \\
      [\tfrac{\alpha_k+\beta_k} 2, \beta_k],  & \text{otherwise}.
    \end{cases}
  \end{balign*}

  Then put \( \alpha_{k+1} \) and \( \beta_{k+1} \) to be the endpoints of the interval \( F_k \) and repeat with \( k+1 \) instead of \( k \). Note that for any \( k = 1, 2, \ldots \), \( \diam(F_k) = \tfrac 1 2 \diam(F_{k-1}) \), thus \( \diam(F_k) \xrightarrow[k \to \infty]{} 0 \). As in \fullref{thm:cantors_nested_compact_theorem}, it follows that if we choose a sequence
  \begin{equation*}
    x_k \in F_k, k = 1, 2, \ldots,
  \end{equation*}
  it will be a fundamental sequence. Since the space is complete, this fundamental sequence necessarily converges.
\end{proof}

\begin{theorem}\label{def:real_numbers_complete_metric_space}
  The metric space \( \BbbR \) is complete.
\end{theorem}
\begin{proof}
  Let \( \seq{ x_k }_{k=1}^\infty \) be a fundamental sequence of real numbers. By \fullref{thm:fundamental_sequence_is_bounded}, the sequence is bounded. By \fullref{def:bolzano_weierstrass}, it has a convergent subsequence
  \begin{equation*}
    \{ x_{k_m} \}_{m=1}^\infty \to x.
  \end{equation*}

  By \fullref{thm:fundamental_subsequence_convergence}, the sequence itself has the same limit \( \lim_{k \to \infty} x_k = x \).
\end{proof}

\begin{proposition}\label{thm:one_sided_squeeze_lemma}
  Fix two convergent sequences \( \seq{ x_k }_{k=1}^\infty \) and \( \seq{ y_k }_{k=1}^\infty \) of real numbers.

  If \( x_k \leq y_k \) for all \( k = 1, 2, \ldots \), then
  \begin{equation*}
    \lim_{k \to \infty} x_k \leq \lim_{k \to \infty} y_k.
  \end{equation*}
\end{proposition}
\begin{proof}
  Denote the respective limits by \( x \) and \( y \).

  Fix \( \varepsilon > 0 \). Then by \fullref{def:net_limit_point}, there exist indices \( k_0 \) and \( m_0 \) such that
  \begin{balign*}
     & \abs{x - x_k} < \frac \varepsilon 2 \quad\forall k \geq k_0 \\
     & \abs{y - y_m} < \frac \varepsilon 2 \quad\forall m \geq m_0
  \end{balign*}

  Take \( k \geq \max \{ k_0, m_0 \} \). Then \( y_k \geq x_k \) and
  \begin{balign*}
    y - x
     & =
    (y - y_k) + (y_k - x_k) + (x_k - x)
    \geq \\ &\geq
    (y - y_k) + (x - x_k)
    >    \\ &>
    - \frac \varepsilon 2 - \frac \varepsilon 2
    =
    - \varepsilon.
  \end{balign*}

  Since \( \varepsilon \) was chosen arbitrary, \( y - x \) cannot equal any negative number, because otherwise we could choose another \( \varepsilon \) smaller than the magnitude of the negative number and obtain a contradiction.

  Thus, \( y \geq x \).
\end{proof}

\begin{lemma}[Squeeze lemma]\label{thm:squeeze_lemma}
  Let \( I \) be a closed \hyperref[def:order_interval/closed]{interval} in \( \BbbR \).

  \begin{thmenum}
    \thmitem{thm:squeeze_lemma/sequences} Let \( \seq{ x_k }_{k=1}^\infty, \{ x_k^- \}_{k=1}^\infty, \{ x_k^+ \}_{k=1}^\infty \) be three sequences in \( I \). If both \( \{ x_k^- \}_{k=1}^\infty \) and \( \{ x_k^+ \}_{k=1}^\infty \) converge to the same value \( \oline x \in I \) and if the following inequalities
    \begin{equation*}
      x_k^- \leq x_k \leq x_k^+
    \end{equation*}
    hold for all \( k = 1, 2, \ldots \), then the \enquote{squeezed in} sequence \( \seq{ x_k }_{k=1}^\infty \) also converges to \( \oline x \).

    \thmitem{thm:squeeze_lemma/functions} Let \( f, f_-, f_+: I \to \BbbR \) be three functions and let \( \oline x \in I \). If both limits \( \lim_{x \to \oline x} f_-(x) \) and \( \lim_{x \to \oline x} f_+(x) \) converge to the same value \( \oline y \in \BbbR \) and if the following inequalities
    \begin{equation*}
      f_-(x) \leq f(x) \leq f_+(x)
    \end{equation*}
    hold for all \( x \in I \), then the \enquote{squeezed in} function \( f \) also converges to \( \oline y \) at \( \oline x \).
  \end{thmenum}
\end{lemma}
\begin{proof}
  \SubProofOf{thm:squeeze_lemma/sequences} Fix \( \varepsilon > 0 \). Then by \fullref{def:net_limit_point}, there exist indices \( k^- \) and \( k^+ \) such that
  \begin{balign*}
     & \abs{\oline x - x_k^-} < \frac \varepsilon 3 \quad\forall k \geq k^- \\
     & \abs{\oline x - x_k^+} < \frac \varepsilon 3 \quad\forall k \geq k^+
  \end{balign*}

  By taking \( k \geq \max \{ k^-, k^+ \} \), we obtain
  \begin{equation*}
    \abs{x_k^+ - x_k^-} \leq \abs{x_k^+ - \oline x} + \abs{\oline x - x_k^-} < \frac 2 3 \varepsilon.
  \end{equation*}

  Since \( \abs{x_k^+ - x_k} \leq \abs{x_k^+ - x_k^-} \), it follows that \( \abs{x_k^+ - x_k} < \frac 2 3 \varepsilon \).

  Thus,
  \begin{equation*}
    \abs{\oline x - x_k} \leq \abs{\oline x - x_k^+} + \abs{x_k^+ - x_k} < \varepsilon.
  \end{equation*}

  \Fullref{def:net_limit_point} is satisfied, hence \( \{ x_k \} \) converges to \( \oline x \).

  \SubProofOf{thm:squeeze_lemma/functions} The proof is analogous to that of \fullref{thm:squeeze_lemma/sequences}, but the machinery is different. Fix \( \varepsilon > 0 \). Then by \fullref{def:local_convergence/neighborhoods}, there exist radii \( \delta^- \) and \( \delta^+ \) such that
  \begin{balign*}
     & f_-(I \cap B(\oline x, \delta^-)) \subseteq B(\oline y, \tfrac \varepsilon 3) \\
     & f_+(I \cap B(\oline x, \delta^+)) \subseteq B(\oline y, \tfrac \varepsilon 3)
  \end{balign*}

  Take \( \delta < \min \{ \delta^-, \delta^+ \} \) and \( x \in I \cap B(\oline x, \delta) \). Analogously to our proof of \fullref{thm:squeeze_lemma/sequences}, we obtain the inequality
  \begin{equation*}
    \abs{f(x) - \oline x} \leq \abs{f(x) - f^-(x)} + \abs{f^-(x) - \oline x} < \varepsilon.
  \end{equation*}

  We conclude that
  \begin{equation*}
    f(I \cap B(\oline x, \delta)) \subseteq B(\oline y, \varepsilon)
  \end{equation*}
  holds and thus by \fullref{def:local_convergence/neighborhoods}, the function \( f \) converges to \( \oline y \) at \( \oline x \).
\end{proof}

\begin{proposition}\label{thm:real_monotone_sequence_converges_iff_bounded}
  A \hyperref[def:order_function]{monotone} sequence of real numbers \hyperref[def:net_limit_point]{converges} if and only if it is \hyperref[def:metric_space/bounded_sequence]{bounded}.
\end{proposition}
\begin{proof}
  \SufficiencySubProof Let \( \seq{ x_k }_{k=1}^\infty \) be a convergent monotone sequence. Denote its limit by \( x \). Fix \( \varepsilon > 0 \). By \fullref{def:net_limit_point}, there exists \( k_0 \) such that
  \begin{equation*}
    \abs{x - x_k} < \varepsilon \quad\forall k \geq k_0.
  \end{equation*}

  Thus, \( \{ x_k \colon k \geq k_0 \} \subseteq B(x, \varepsilon) \).

  Also note that
  \begin{equation*}
    \{ x_k \colon k < k_0 \} \subseteq B(x, \max_{i < k_0} \{ \abs{x - x_k} \}).
  \end{equation*}

  We obtained that the entire sequence
  \begin{equation*}
    \{ x_k \colon k \geq 1 \} = \{ x_k \colon k < k_0 \} \cup \{ x_k \colon k \geq k_0 \}
  \end{equation*}
  is contained in a union of two balls and is therefore bounded.

  \NecessitySubProof Now let \( \seq{ x_k }_{k=1}^\infty \) be a bounded monotone sequence. Denote its supremum by \( \alpha \). Note that
  \begin{equation*}
    \abs{x_n - x_m} = x_n - x_m \leq \alpha \quad\forall n \geq m.
  \end{equation*}

  Fix \( \varepsilon > 0 \). Then there exists at least one element \( x_{m_0} > \alpha - \varepsilon \) because otherwise \( \alpha \) would not be a supremum.

  Then for any index \( n \geq m_0 \) we have
  \begin{equation*}
    \abs{x_n - x_{m_0}} = x_n - x_{m_0} < \alpha - (\alpha - \varepsilon) = \varepsilon.
  \end{equation*}

  Thus, \fullref{def:net_limit_point} is satisfied and the sequence \( \seq{ x_k }_{k=1}^\infty \) converges.
\end{proof}

  \subsection{Real differentiability}\label{subsec:real_differentiability}

\begin{proposition}\label{thm:real_valued_differentiability}
  Let \( U \subseteq \BbbR^n \) be an open set. A real-valued function \( f: U \to \BbbR \) is differentiable at \( x \) in the direction \( h \) if and only if \( \varphi(t) = f(x + th) \) is right-differentiable at \( 0 \).
\end{proposition}
\begin{proof}
  \begin{equation*}
    f_+'(x)(h) \coloneqq \lim_{t \downarrow 0} \frac {f(x + th) - f(x)} t = \varphi_+'(0)(1).
  \end{equation*}
\end{proof}

\begin{example}[Weierstrass' nowhere differentiable function]\label{ex:weierstrass_nowhere_differentiable_function}\mcite[\S 271]{Фихтенгольц1968ОсновыТом2}
  Let \( a \in (0, 1) \) and \( b \) is a positive odd integer such that
  \begin{equation*}
    ab > 1 + \frac 3 2 \pi.
  \end{equation*}

  Define the function
  \begin{equation*}
    f(x) \coloneqq \sum_{k=0}^\infty a^k \cos(b^k \pi x).
  \end{equation*}

  \begin{figure}[!ht]
    \centering
    \includegraphics{output/ex__weierstrass_nowhere_differentiable_function}
    \caption
    {
      Plot of the third partial sum of the Weierstrass function with \( a = 0.9 \) and \( b = 7 \) from \( -\pi / 8 \) to \( \pi / 8 \).
    }
    \label{fig:ex:weierstrass_nowhere_differentiable_function/plot}
  \end{figure}

  Since \( \cos \) is bounded for real arguments and \( a \in (0, 1) \), each term is uniformly bounded by \( 1 \) and by \fullref{thm:weierstrass_series_criterion}, \( f \) is continuous. However, it is not \hyperref[def:differentiability]{differentiable} at any point. The proof of the latter is involved and will not be given here.
\end{example}

\begin{theorem}[Leibniz' rule]\label{thm:leibniz_rule}
  \todo{Prove}.
\end{theorem}

\begin{theorem}[Lagrange's mean value theorem]\label{thm:lagranges_mean_value_theorem}
  \todo{Prove}.
\end{theorem}

\begin{theorem}[Constant rank theorem]\label{thm:constant_rank_theorem}
  \todo{Prove}.
\end{theorem}

\begin{theorem}[Inverse function theorem]\label{thm:inverse_function_theorem}
  \todo{Prove}.
\end{theorem}

\begin{theorem}[Fundamental theorem of calculus]\label{thm:fundamental_theorem_of_calculus}
  \todo{Prove}.
\end{theorem}

  \section{Real series}\label{sec:real_series}

\begin{proposition}\label{thm:almost_all_terms_positive_implies_absolute_convergent}
  If only finitely many coefficients in a real \hyperref[def:convergent_series]{convergent} series are negative, then the series converges absolutely.
\end{proposition}
\begin{proof}
  Let \( N \) be the index of the last negative coefficient in \eqref{def:convergent_series/series}. Then the series
  \begin{equation*}
    \sum_{k={N+1}}^\infty a_k
  \end{equation*}
  is absolutely convergent since every coefficient is positive. Then
  \begin{equation*}
    \sum_{k=0}^\infty \abs{a_k} = \sum_{k=0}^N \abs{a_k} + \sum_{k=N+1}^\infty \abs{a_k}
  \end{equation*}
  is convergent since the first term on the right side is a finite sum and the second is a convergent series. Hence, the series \eqref{def:convergent_series/series} converges absolutely.
\end{proof}

\begin{corollary}\label{thm:almost_all_terms_negative_implies_absolute_convergent}
  If only finitely many coefficients in a real \hyperref[def:convergent_series]{convergent} series are positive, then the series converges absolutely.
\end{corollary}

\begin{theorem}[Riemann's series permutation theorem]\label{thm:riemanns_series_permutation_theorem}\mcite[\S 247]{Фихтенгольц1968ОсновыТом2}
  If the real series
  \begin{equation*}
    \sum_{k=0}^\infty a_k
  \end{equation*}
  is \hyperref[def:convergent_series]{convergent}, but not absolutely convergent, then for any extended real number \( x \in \BbbR \cup \{ -\infty, +\infty \} \) there exists a \hyperref[def:symmetric_group]{permutation} \( p \) of the coefficients \( a_0, a_1, a_2 \)
  such that
  \begin{equation*}
    \sum_{k=0}^\infty p(a_k) = x.
  \end{equation*}
\end{theorem}
\begin{proof}
  If the series is not absolutely convergent, then there exist both infinitely many positive and infinitely many negative coefficients.

  First, assume that \( x \) is finite.

  Define the permuted series
  \begin{equation*}
    \sum_{k=0}^\infty b_k
  \end{equation*}
  as follows:
  \begin{thmenum}
    \thmitem{thm:riemanns_series_theorem/positive} Assign to \( b_n \) only nonnegative elements of the sequence \( \{ a_k \}_{k=0}^\infty \) until \( \sum_{k=0}^n b_k \geq x \). Then go to \ref{thm:riemanns_series_theorem/negative}.
    \thmitem{thm:riemanns_series_theorem/negative} Assign to \( b_n \) only negative elements of the sequence \( \{ a_k \}_{k=0}^\infty \) until \( \sum_{k=0}^n b_k \geq x \). Then go to \ref{thm:riemanns_series_theorem/positive}.
  \end{thmenum}

  This mutual recursion builds a series that converges to \( x \) because the coefficients \( \{ a_k \}_{k=0}^\infty \) get arbitrarily close to each other.

  If \( x = +\infty \), we can add positive coefficients until \( \sum_{k=0}^n b_k \geq 1 \), then add a single negative coefficient, then continue adding positive coefficients until \( \sum_{k=0}^n b_k \geq 2 \), and so on.

  If \( x = -\infty \), we use the same process, but with milestones of \( -1, -2, -3, \ldots \).
\end{proof}

\begin{example}\label{ex:riemanns_series_permutation_theorem/alternating_harmonic_series}\mcite[\S 247]{Фихтенгольц1968ОсновыТом2}
  Consider the alternating harmonic series \eqref{eq:ex:harmonic_series/alternating}. Denote its sum by \( a \).

  We can rearrange this series by repeating two negative terms and a single positive term as follows:
  \begin{equation}\label{ex:riemanns_series_permutation_theorem/alternating_harmonic_series/rearranged}
    1 - \frac 1 2 - \frac 1 4 + \frac 1 3 - \frac 1 6 - \frac 1 8 + \cdots
    =
    \sum_{m=1}^\infty \left( \frac 1 {2m - 1} - \frac 1 {4m - 2} - \frac 1 {4m} \right).
  \end{equation}

  Note that \fullref{ex:riemanns_series_permutation_theorem/alternating_harmonic_series/rearranged} is equivalent to
  \begin{equation*}
    \sum_{m=1}^\infty \left( \frac 1 {2m - 1} - \frac 1 {4m - 2} - \frac 1 {4m} \right)
    =
    \sum_{m=1}^\infty \left( \frac 1 {4m - 2} - \frac 1 {4m} \right)
    =
    \frac 1 2 \sum_{m=1}^\infty \left( \frac 1 {2m - 1} - \frac 1 {42} \right)
    =
    \frac a 2.
  \end{equation*}
\end{example}

\begin{proposition}\label{thm:positive_series_comparison}\mcite[\S 237]{Фихтенгольц1968ОсновыТом2}
  Fix two nonnegative series
  \begin{equation}\label{def:positive_series_comparison/a}
    \sum_{k=0}^\infty a_k
  \end{equation}
  and
  \begin{equation}\label{def:positive_series_comparison/b}
    \sum_{k=0}^\infty b_k
  \end{equation}
  that is, series with nonnegative real coefficients. Assume that there exists an index \( K \) such that
  \begin{equation*}
    a_k \leq b_k \quad\forall k \geq K.
  \end{equation*}

  We say that the series \fullref{def:positive_series_comparison/b} \term{dominates} the series \fullref{def:positive_series_comparison/a}.

  Then
  \begin{thmenum}
    \thmitem{thm:positive_series_comparison/b_converges} If \fullref{def:positive_series_comparison/b} converges, so does \fullref{def:positive_series_comparison/a}.

    \thmitem{thm:positive_series_comparison/a_diverges} If \fullref{def:positive_series_comparison/a} diverges, so does \fullref{def:positive_series_comparison/b}.
  \end{thmenum}
\end{proposition}
\begin{proof}
  \SubProofOf{thm:positive_series_comparison/b_converges} Suppose that \fullref{def:positive_series_comparison/b} converges. Then by \fullref{thm:real_monotone_sequence_converges_iff_bounded}, the sequence of partial sums is bounded. Therefore, the sequence of partial sums of \fullref{def:positive_series_comparison/a} is also bounded and, by \fullref{thm:real_monotone_sequence_converges_iff_bounded} again, the series is convergent.

  \SubProofOf{thm:positive_series_comparison/a_diverges} Analogous to \fullref{thm:positive_series_comparison/b_converges}, but using the negation of \fullref{thm:real_monotone_sequence_converges_iff_bounded}.
\end{proof}

\begin{proposition}[Cauchy's root test]\label{thm:cauchys_root_test}\mcite[thm. 3.33]{Rudin1976AnalysisPrinciples}
  Consider the nonnegative series \fullref{def:positive_series_comparison/a}. Put
  \begin{equation*}
    q \coloneqq \limsup_{k \to \infty} \sqrt[k]{a_k},
  \end{equation*}
  where \( q = \infty \) if the limit does not exist. Then
  \begin{itemize}
    \item If \( q < 1 \), the series converges.
    \item If \( q > 1 \), the series diverges.
    \item If the limit does not exist (e.g. if \( a_k = k^k \)), the series diverges.
    \item If \( q = 1 \), the series may either converge or diverge.
  \end{itemize}
\end{proposition}
\begin{proof}
  The case when the limit \( q \) does not exist is obvious by the contraposition to \fullref{thm:convergent_series_terms_vanish}.

  Suppose that the limit exists. Therefore, there exists an index \( K \) such that
  \begin{equation*}
    \sqrt[k]{a_k} \leq q \quad\forall k \geq K.
  \end{equation*}

  Thus, we have the inequality
  \begin{equation*}
    a_k \leq q^k \quad\forall k \geq K.
  \end{equation*}

  The statement of the theorem now follows from comparison (\fullref{thm:positive_series_comparison}) with the geometric series \eqref{eq:def:geometric_series}.
\end{proof}

\begin{proposition}[d'Alambert's ratio test]\label{thm:dalamberts_ratio_test}\mcite[thm. 3.33]{Rudin1976AnalysisPrinciples}
  Consider the nonnegative series \fullref{def:positive_series_comparison/a}. Put
  \begin{equation*}
    q \coloneqq \limsup_{k \to \infty} \frac {a_{k+1}} {a_k},
  \end{equation*}
  where \( q = \infty \) if the limit does not exist. Then
  \begin{itemize}
    \item If \( q < 1 \), the series converges.
    \item If there exists an index \( k_0 \) such that \( \frac {a_{k+1}} {a_k} \geq 1 \) for all \( k \geq k_0 \), the series diverges.
    \item If the limit does not exist (e.g. if \( a_k = k! \)), the series diverges.
  \end{itemize}
\end{proposition}
\begin{proof}
  All cases except for \( q < 1 \) are obvious by the contraposition to \fullref{thm:convergent_series_terms_vanish}.

  Suppose that the limit exists. Therefore, there exists an index \( k_0 \) such that
  \begin{equation*}
    a_{k+1} \leq q a_k \quad\forall k \geq k_0.
  \end{equation*}

  Thus,
  \begin{equation*}
    a_{k_0 + m} \leq q^m a_{k_0} \quad\forall m \geq \BbbZ^{\geq 0}.
  \end{equation*}

  Convergence now follows from comparison (\fullref{thm:positive_series_comparison}) of the geometric series \eqref{eq:def:geometric_series} with the subseries of \fullref{def:positive_series_comparison/a} obtained by trimming the first \( k_0 \) elements.
\end{proof}

\begin{proposition}\label{rem:nonnegative_series_convergence_test_equivalence}
  The values of \( q \) in \fullref{thm:cauchys_root_test} and in \fullref{thm:dalamberts_ratio_test} are equal.
\end{proposition}
\begin{proof}
  If we assume that they are not equal, then the same series would have to be convergent and divergent simultaneously in some region.
\end{proof}

\begin{definition}\label{def:alternating_series}
  Series of the form
  \begin{equation}\label{def:alternating_series/series}
    \pm \sum_{k=0}^\infty (-1)^k a_k,
  \end{equation}
  where all \( a_k, k = 0, 1, \ldots \) are nonnegative, are called \term{alternating}.
\end{definition}

\begin{proposition}[Leibniz' alternating series test]\label{thm:leibniz_alternating_series_test}
  Consider the alternating series \fullref{def:alternating_series}. If the sequence of terms \( \{ a_k \}_{k=0}^\infty \) decreases monotonically and if \( \lim_{k \to \infty} a_k = 0 \), then the series converges.
\end{proposition}

\begin{theorem}\label{thm:weierstrass_series_criterion_nessessity}\mcite[\S 268]{Фихтенгольц1968ОсновыТом2}
  \Fullref{thm:weierstrass_series_criterion} is a necessary condition for nonnegative real functions.
\end{theorem}

\begin{proposition}\label{thm:nonnegative_series_inequality}
  The sum of a series with nonnegative coefficients is greater than or equal to any of its partial sums.
\end{proposition}
\begin{proof}
  Trivial.
\end{proof}

\begin{corollary}\label{thm:absolute_convergence_to_zero}
  A complex series converges absolutely to zero if and only if all coefficients are zero.
\end{corollary}
\begin{proof}
  \SufficiencySubProof Let
  \begin{equation*}
    \sum_{k=0}^\infty \abs{a_k} = 0.
  \end{equation*}

  From \fullref{thm:nonnegative_series_inequality} it follows that \( \abs{a_k} \leq 0 \) for every index \( k \).

  \NecessitySubProof Trivial.
\end{proof}

  \subsection{Riemann integration}\label{subsec:riemann_integration}

\begin{definition}\label{def:riemann_partition}\mcite[def. 1]{Gordon1991BanachSpaceIntegration}
  The concept of a partition of a nonempty \hyperref[def:real_numbers]{real} \hyperref[def:order_interval/closed]{closed interval} \( [a, b] \) is the base for defining Riemann-style integrals.

  \begin{thmenum}
    \thmitem{def:riemann_partition/partition} A \term{Riemann partition} of \( [a, b] \) is a set
    \begin{equation*}
      \Delta \coloneqq \{ x_0, \ldots, x_n \} \subseteq [a, b]
    \end{equation*}
    that satisfies
    \begin{equation*}
      a = x_0 < x_1 < \ldots < x_n = b.
    \end{equation*}

    For brevity, we write
    \begin{equation}\label{eq:def:riemann_partition/partition}
      \Delta: a = x_0 < x_1 < \ldots < x_n = b.
    \end{equation}

    We denote the set of all partitions of \( [a, b] \) by \( \op{part}([a, b]) \).

    \thmitem{def:riemann_partition/refinement} The partition
    \begin{equation*}
      \Gamma: a = y_0 < y_1 < \ldots < y_m = b
    \end{equation*}
    is called a \term{refinement} of the partition \eqref{eq:def:riemann_partition/partition} if we have the \hyperref[def:subset]{set inclusion}
    \begin{equation}\label{eq:def:riemann_partition/refinement/inclusion}
      \{ x_0, x_1, \ldots, x_n \} \subseteq \{ y_0, y_1, \ldots, y_m \}.
    \end{equation}

    In this case, we \enquote{split} \( \Gamma \) into chains such that, for each \( k = 1, 2, \ldots, n \),
    \begin{equation}\label{def:riemann_partition/refinement/splitting}
      y_{k,j} \coloneqq \begin{cases}
        x_{k-1},                                                                          &j = 0, \\
        x_k,                                                                              &j = p_k, \\
        j\text{-th point of } \{ y_0, \ldots, y_m \} \cap [x_{k-1}, x_k], &0 < j < p_k.
      \end{cases}
    \end{equation}

    \thmitem{def:riemann_partition/diameter} Finally, the \term{diameter} of the partition \eqref{eq:def:riemann_partition/partition} is defined as
    \begin{equation}\label{eq:def:riemann_partition/diameter}
      \diam(\Delta) \coloneqq \max_{1 \leq k \leq n} (x_k - x_{k-1}).
    \end{equation}

    \thmitem{def:riemann_partition/order} We can make the set \( \op{part}([a, b]) \) of all \hyperref[def:riemann_partition/partition]{Riemann partitions} of \( [a, b] \) into a \hyperref[def:directed_set]{directed set} using two common approaches:
    \begin{thmenum}
      \thmitem{def:riemann_partition/order/refinement} Put \( \Delta \preceq_R \Gamma \) if and only if \( \Gamma \) is a \hyperref[def:riemann_partition/refinement]{refinement} of \( \Delta \). This actually makes \( (\op{part}([a, b]), \preceq_R) \) a \hyperref[def:partially_ordered_set]{partially ordered set}.
      \thmitem{def:riemann_partition/order/diameter} Put \( \Delta \preceq_D \Gamma \) if and only if \( \diam(\Gamma) \leq \diam(\Delta) \).
    \end{thmenum}

    \thmitem{def:riemann_partition/tagged} A \term{tagged partition} of \( [a, b] \) is a partition \eqref{eq:def:riemann_partition/partition} of \( [a, b] \) along with a choice of a \term{tag} \( \xi_k \) for each closed interval \( [x_{k-1}, x_k], k = 1, \ldots, n \). By putting \( \Xi \coloneqq \{ \xi_k \}_{k=1}^n \), we can define a tagged partition as the tuple \( (\Delta, \Xi) \). For brevity, we write
    \begin{equation}\label{eq:def:riemann_partition/tagged}
      \begin{aligned}
        &\Delta: a = x_0 < x_1 < \ldots < x_n = b \\
        &\Xi: \xi_k \in [x_{k-1}, x_k], k = 1, \ldots, n.
      \end{aligned}
    \end{equation}

    We denote the set of all tagged partitions of \( [a, b] \) by \( \op{tpart}([a, b]) \). We introduce an order on \( \op{tpart}([a, b]) \) by putting
    \begin{equation*}
      (\Delta, \Xi) \preceq_R (\Gamma, \Eta) \T{if and only if} \Delta \preceq_R \Eta
    \end{equation*}
    and analogously for \( \preceq_D \). Note that \( \preceq_R \) is not a partial order in \( \op{tpart}([a, b]) \) unlike in \( \op{part}([a, b]) \).
  \end{thmenum}
\end{definition}

\begin{remark}\label{rem:set_and_riemann_partitions}
  Note that \eqref{eq:def:riemann_partition/partition} is not a partition in the sense of \fullref{def:set_partition}, however the set of intervals
  \begin{equation*}
    \Big\{ [x_0, x_1), [x_1, x_2), \ldots, [x_{n-2}, x_{n-1}), [x_{n-1}, x_n] \Big\}
  \end{equation*}
  is a set-theoretic partition of \( [a, b] \). Conversely, every finite set-theoretic partition of \( [a, b] \) gives rise to a Riemann partition in the sense of \fullref{def:riemann_partition/partition}.
\end{remark}

\begin{definition}\label{def:riemann_integral}\mcite[def. 2]{Gordon1991BanachSpaceIntegration}
  Let \( X \) be a real \hyperref[def:separation_axioms/T2]{Hausdorff} \hyperref[def:topological_vector_space]{topological vector space}. Fix a \hyperref[def:function]{function} \( f: [a, b] \to X \).

  The \term{Riemann sum} of \( f \) corresponding to the \hyperref[def:riemann_partition/tagged]{tagged partition} \eqref{eq:def:riemann_partition/tagged} is defined as
  \begin{equation*}
    S(f, \Delta, \Xi) \coloneqq \sum_{k=1}^n f(\xi_k) (x_k - x_{k-1}).
  \end{equation*}

  Consider the net
  \begin{equation}\label{eq:def:riemann_integral/net}
    \{ S(f, \Delta, \Xi) \}_{(\Delta, \Xi) \in \op{tpart}([a, b])}
  \end{equation}

  Both orders \fullref{def:riemann_partition/order/refinement} and \fullref{def:riemann_partition/order/diameter} on \( \op{tpart}([a, b]) \) provide equivalent convergence for Riemann sums. If the limit exists, \( f \) is said to be \term{Riemann integrable} in \( [a, b] \). We call the limit the \term{Riemann integral} of \( f \) and denote it by
  \begin{equation}\label{eq:def:riemann_integral}
    \int_a^b f(x) dx.
  \end{equation}
\end{definition}
\begin{proof}
  \ImplicationSubProof{def:riemann_partition/order/refinement}{def:riemann_partition/order/diameter} Let \( I \) be the limit \eqref{eq:def:riemann_integral} with respect to \( \preceq_R \). Fix a neighborhood \( U \) of \( 0 \). Since \eqref{eq:def:riemann_integral/net} is eventually in \( I + U \), there exists a tagged partition
  \begin{equation}\label{eq:def:riemann_integral/tagged_zero}
    \begin{aligned}
      &\Delta_0: a = x_0^{(0)} < x_1^{(0)} < \ldots < x_n^{(0)} = b \\
      &\Xi_0: \xi_k^{(0)} \in [x_{k-1}^{(0)}, x_k^{(0)}], k = 1, \ldots, n_0.
    \end{aligned}
  \end{equation}
  such that \( S(f, \Gamma, \Eta) \in I + U \) if \( \Gamma \) is a refinement of \( \Delta_0 \).

  Note that \( f \) is \hyperref[def:bounded_function/bounded]{bounded}. Indeed, if it is unbounded on \( [a, b] \), then there exists a refinement \( (\Gamma, \Eta) \) of \( (\Delta_0, \Xi_0) \) such that
  \begin{equation*}
    S(f, \Gamma, \Eta) - I \not\in U.
  \end{equation*}

  But this contradicts our choice of \( \Delta_0 \). Therefore, \( f \) is bounded and there exists a bounded neighborhood \( V_0 \) of \( 0 \) such that \( f([a, b]) \subseteq V_0 \) and hence \( f(x) - f(y) \in V \coloneqq V_0 - V_0 \) for all \( x, y \in [a, b] \).

  Let  \( v > 0 \) be such that \( V \subseteq vU \).

  Let \( (\Delta, \Xi) \) be a tagged partition such that \( \diam(\Delta) \leq \diam(\Delta_0) \).

  We introduce another partition \( \Gamma \coloneqq \Delta \cup \Delta_0 \). Since \( \Gamma \) is a refinement of \( \Delta_0 \), we can use a splitting similar to \eqref{def:riemann_partition/refinement/splitting} such that
  \begin{equation}\label{def:riemann_partition/subdiameter_splitting}
    S(f, \Delta_0, \Xi_0) = \sum_{k=1}^{n_0} \sum_{j=1}^{p_k} f(\xi^{(0)}_k) (y_{k,j} - y_{k,j-1}).
  \end{equation}

  Denote by \( \xi_{k,j} \) the largest tag in \( \Xi \) such that \( \xi_{k,j} \leq y_{k,j} \). Thus,
  \begin{equation*}
    S(f, \Delta, \Xi) = \sum_{k=1}^{n_0} \sum_{j=1}^{p_k} f(\xi_{k,j}) (y_{k,j} - y_{k,j-1}).
  \end{equation*}

  For every \( k = 1, \ldots, n \) and every \( j = 0, \ldots, p_k \), choose an arbitrary tag
  \begin{equation*}
    \Eta: \eta_{k,j} \in [y_{k,j-1}, y_{k,j}].
  \end{equation*}

  Then we have
  \begin{balign*}
    S(f, \Delta, \Xi) - I
    &=
    S(f, \Delta, \Xi) - S(f, \Gamma, \Eta) + \underbrace{S(f, \Gamma, \Eta) - I}_{\in U}
    \in \\ &\in
    \sum_{k=1}^{n_0} \sum_{j=1}^{p_k} [ \underbrace{f(\xi_{k,j}) - f(\eta_{k,j})}_{\in V} ] (y_{k,j} - y_{k,j-1}) + U
    \subseteq \\ &\subseteq
    V \cdot \sum_{k=1}^{n_0} \underbrace{\sum_{j=1}^{p_k} (y_{k,j} - y_{k,j-1})}_{x_k - x_{k-1}} + U
    \subseteq \\ &\subseteq
    \diam(\Delta) \cdot n_0 \cdot V + U
    \subseteq \\ &\subseteq
    (\diam(\Delta) \cdot n_0 \cdot v + 1) U.
  \end{balign*}

  Let \( (\Delta_1, \Xi_1) \) be a tagged partition of \( [a, b] \) such that \( \diam(\Delta_1) \leq \min \left\{ \diam(\Delta_0), \frac 1 {v n_0} \right\} \). It follows that
  \begin{equation}\label{eq:def:riemann_integral/subdiameter_in_neighborhood}
    S(f, \Delta_1, \Xi_1) - I \subseteq 2U.
  \end{equation}

  Until now, \( U \) was fixed. Given any neighborhood \( W \) of \( 0 \), we need to choose a neighborhood \( U \) of \( 0 \) and a corresponding partition \( \Delta_1 \) such that \eqref{eq:def:riemann_integral/subdiameter_in_neighborhood} holds. Then, whenever \( \diam(\Delta) \leq \diam(\Delta_1) \), we have
  \begin{equation*}
    S(f, \Delta, \Xi) - I \subseteq 2U \subseteq W.
  \end{equation*}

  This finishes the proof.

  \ImplicationSubProof{def:riemann_partition/order/diameter}{def:riemann_partition/order/refinement} Note that if \( \Gamma \) is a refinement of \( \Delta \), clearly \( \diam(\Gamma) \leq \diam(\Delta) \). Therefore, if the net \eqref{eq:def:riemann_integral/net} with respect to \( \preceq_D \) is eventually in some open set \( U \), the corresponding net with respect to \( \preceq_R \) is also eventually in \( U \). This finishes the proof.
\end{proof}

\begin{corollary}\label{thm:riemann_integrable_implies_bounded}
  A Riemann-integrable function is bounded.
\end{corollary}
\begin{proof}
  Proven in \fullref{def:riemann_integral}.
\end{proof}

\begin{definition}\label{def:darboux_integrability}\mcite[def. 17]{Gordon1991BanachSpaceIntegration}
  Let \( (X, \rho) \) be a \hyperref[def:frechet_space]{Frechet space}. Fix a function \( f: [a, b] \to X \). Similarly to \fullref{def:riemann_integral}, choose any of the orderings \fullref{def:riemann_partition/order/refinement} and \fullref{def:riemann_partition/order/diameter} on the set of all untagged \hyperref[def:riemann_partition/partition]{Riemann partitions} \( \op{part}([a, b]) \).

  For each partition \eqref{eq:def:riemann_partition/partition}, we define its \term{oscillation} via the \hyperref[def:function_oscillation]{function oscillation} of \( f \)
  \begin{equation}\label{eq:def:darboux_integrability/oscillation}
    \omega(f, \Delta) \coloneqq \sum_{k=1}^n \omega(f, [x_{k-1}, x_k]) (x_k - x_{k-1}).
  \end{equation}

  Consider the net
  \begin{equation}\label{eq:def:darboux_integrability/net}
    \{ \omega(f, \Delta) \}_{\Delta \in \op{part}([a, b])}
  \end{equation}

  If this net converges to zero, we say that \( f \) is \term{Darboux integrable}.
\end{definition}

\begin{proposition}\label{thm:darboux_integrable_implies_riemann_integrable}
  In a \hyperref[def:banach_space]{Banach space}, \hyperref[def:darboux_integrability]{Darboux integrability} implies \hyperref[def:riemann_integral]{Riemann integrability}.
\end{proposition}
\begin{proof}
  We will show that the net \eqref{eq:def:riemann_integral/net} is fundamental. Fix \( \varepsilon > 0 \). Since \( f \) is Darboux integrable, there exists an untagged partition \( \Delta_0 \) such that, if \( \Delta \) is a refinement of \( \Delta_0 \), we have
  \begin{equation*}
    \omega(f, \Delta) < \varepsilon.
  \end{equation*}

  Let \( \Delta \) be a refinement of \( \Delta_0 \) and \( \Gamma \) be a refinement of \( \Delta \). Assume that the points of \( \Gamma \) are split as in \eqref{def:riemann_partition/refinement/splitting}. Choose arbitrary tags \( \Xi = \{ \xi_k \}_{k=1}^n \) for \( \Delta \) and \( \Eta = \{ \eta_{k,j} \}_{k=1,j=1}^{n,p_k} \) for \( \Gamma \). For the corresponding Riemann sums, we have
  \begin{balign*}
    &\phantom{{}={}}
    \norm{S(f, \Delta, \Xi) - S(f, \Gamma, \Eta)}
    = \\ &=
    \norm{\sum_{k=1}^n f(\xi_k) (x_k - x_{k-1}) - \sum_{k=1}^n \sum_{j=1}^{p_k} f(\eta_{k,j}) (y_{k,j} - y_{k,j-1}) }
    \leq \\ &\leq
    \sum_{k=1}^n \sum_{j=1}^{p_k} \norm{f(\xi_k) - f(\eta_{k,j})} (y_{k,j} - y_{k,j-1})
    \leq \\ &\leq
    \sum_{k=1}^n \sum_{j=1}^{p_k} (y_{k,j} - y_{k,j-1}) \sup \{ \norm{f(\xi) - f(\eta)} \colon \xi, \eta \in [y_{k,j-1}, y_{k,j}] \}
    \leq \\ &\leq
    \sum_{k=1}^n \sup \{ f(\xi) - f(\eta) \colon \xi, \eta \in [x_{k-1}, x_k] \} \underbrace{\sum_{j=1}^{p_k} (y_{k,j} - y_{k,j-1})}_{x_{k-1} - x_k}
    = \\ &=
    \omega(f, \Delta)
    <
    \varepsilon.
  \end{balign*}

  Therefore, the net \eqref{eq:def:riemann_integral/net} is fundamental and, since \( X \) is complete, the net converges to a limit.
\end{proof}

\begin{definition}\label{def:darboux_integral}
  Fix a real-valued function \( f: [a, b] \to \BbbR \). The \term{upper Darboux sum} corresponding to the partition \eqref{eq:def:riemann_partition/partition} is defined as
  \begin{equation*}
    \oline{S}(f, \Delta) \coloneqq \sum_{k=1}^n (x_{k-1} - x_k) \sup_{\xi \in [x_{k-1}, x_k]} f(\xi).
  \end{equation*}

  The \term{lower Darboux sum} is defined as
  \begin{equation*}
    \underline{S}(f, \Delta) \coloneqq \sum_{k=1}^n (x_{k-1} - x_k) \inf_{\xi \in [x_{k-1}, x_k]} f(\xi).
  \end{equation*}

  If the nets
  \begin{align}\label{eq:def:darboux_integral/nets}
    \{ \oline{S}(f, \Delta) \}_{\Delta \in \op{part}([a, b])}
    &&
    \{ \underline{S}(f, \Delta) \}_{\Delta \in \op{part}([a, b])}
  \end{align}
  have a common limit, we call this limit the \term{Darboux integral} of \( f \) and, analogously to \fullref{def:riemann_integral}, we denote it by
  \begin{equation*}
    \int_a^b f(x) dx.
  \end{equation*}

  This notation is justified by \fullref{thm:darboux_integral_iff_riemann_integral}.
\end{definition}

\begin{proposition}\label{thm:darboux_integrable_iff_has_darboux_integral}
  A real-valued function \( f: [a, b] \to \BbbR \) is \hyperref[def:darboux_integrability]{Darboux integrable} if and only if it has a \hyperref[def:darboux_integral]{Darboux integral}.
\end{proposition}
\begin{proof}
  Note that, given the partition \eqref{eq:def:riemann_partition/partition}, we have
  \begin{align*}
    \oline{S}(f, \Delta) - \underline{S}(f, \Delta)
    &=
    \sum_{k=1}^n (x_k - x_{k-1}) \left[ \sup_{\xi \in [x_{k-1}, x_k]} f(\xi) - \inf_{\eta \in [x_{k-1}, x_k]} f(\eta) \right]
    = \\ &=
    \sum_{k=1}^n (x_k - x_{k-1}) \left[ \sup_{\xi \in [x_{k-1}, x_k]} f(\xi) + \sup_{\eta \in [x_{k-1}, x_k]} -f(\eta) \right]
    = \\ &=
    \sum_{k=1}^n (x_k - x_{k-1}) \sup \{ f(\xi) - f(\eta) \colon \xi, \eta \in [x_{k-1}, x_k] \}
    = \\ &=
    \sum_{k=1}^n (x_k - x_{k-1}) \sup \{ \abs{f(\xi) - f(\eta)} \colon \xi, \eta \in [x_{k-1}, x_k] \}
    = \\ &=
    \omega(f, \Delta).
  \end{align*}

  Therefore, the nets \eqref{eq:def:darboux_integral/nets} converge to a common limit if and only if \( \omega(f, \Delta) \xrightarrow[\Delta]{} 0 \). This finishes the proof.
\end{proof}

\begin{proposition}\label{thm:darboux_integral_iff_riemann_integral}
  A real-valued function \( f: [a, b] \to \BbbR \) has a \hyperref[def:darboux_integral]{Darboux integral} if and only if it has a \hyperref[def:riemann_integral]{Riemann integral}. Furthermore, the two integrals are equal.
\end{proposition}
\begin{proof}
  Fix \( \varepsilon > 0 \).

  \ImplicationSubProof{def:darboux_integral}{def:riemann_integral} Denote by \( I_D \) the Darboux integral of \( f \). Then there exists a partition \( \Delta_0 \) of \( [a, b] \) such that for any refinement \eqref{eq:def:riemann_partition/partition} of \( \Delta_0 \) we have
  \begin{equation*}
    \oline{S}(f, \Delta) - \underline{S}(f, \Delta) < \frac \varepsilon 2.
  \end{equation*}

  In particular, \( I_D - \underline{S}(f, \Delta) < \tfrac \varepsilon 2 \).

  Let \( \Xi \coloneqq \{ \xi_k \}_{k=1}^n \) be tags for \( \Delta \). Then
  \begin{align*}
    \abs{S(f, \Delta, \Xi) - I_D}
    &\leq
    \abs{S(f, \Delta, \Xi) - \underline{S}(f, \Delta)} - \abs{\underline{S}(f, \Delta) - I}
    \leq \\ &\leq
    \abs{\oline{S}(f, \Delta) - \underline{S}(f, \Delta)} - \abs{\underline{S}(f, \Delta) - I}
    < \\ &<
    \frac \varepsilon 2 + \frac \varepsilon 2
    = \\ &=
    \varepsilon.
  \end{align*}

  Therefore, \( I_D \) is also a Riemann integral for \( f \).

  \ImplicationSubProof{def:riemann_integral}{def:darboux_integral} Denote by \( I_R \) the Riemann integral of \( f \). Then there exists a partition \eqref{eq:def:riemann_integral/tagged_zero} such that for any partition \eqref{eq:def:riemann_partition/tagged} with \( \diam(\Delta) \leq \diam(\Delta_0) \), we have
  \begin{equation*}
    \abs{S(f, \Delta, \Xi) - I_R} < \frac \varepsilon 2.
  \end{equation*}

  Since \eqref{thm:riemann_integrable_implies_bounded} is bounded, there exists a constant \( M > 0 \) such that \( \abs{f(\xi) - f(\eta)} < M \) for any \( \xi, \eta \in [a, b] \).

  Using an analogous to \eqref{def:riemann_partition/subdiameter_splitting} splitting for the refinement \( \Gamma \coloneqq \Delta \cup \Delta_0 \) of \( \Delta_0 \), we obtain
  \begin{align*}
    \oline{S}(f, \Gamma) - S(f, \Gamma, \Eta)
    &=
    \sum_{k=1}^{n_0} \sum_{k=1}^{p_k} [ \sup_{\xi \in [y_{k,j-1}, y_{k,j}]} f(\eta) - f(\eta_{k,j}) ] (y_{k,j} - y_{k,j-1})
    \leq \\ &\leq
    M \sum_{k=1}^{n_0} \sum_{k=1}^{p_k} (y_{k,j} - y_{k,j-1})
    \leq \\ &\leq
    M \cdot n_0 \cdot \diam(\Gamma).
  \end{align*}

  By choosing a tagged partition \( (\Delta_1, \Xi_1) \) with \( \diam(\Delta_1) < \min \left\{ \diam(\Delta_0), \frac \varepsilon {2 M n_0} \right\} \), we obtain
  \begin{equation*}
    \oline{S}(f, \Delta_1) - S(f, \Delta_1, \Xi) < \frac \varepsilon 2.
  \end{equation*}

  Therefore, whenever \( \diam(\Delta) \leq \diam(\Delta_1) \),
  \begin{equation*}
    \oline{S}(f, \Delta) - I_R
    =
    \oline{S}(f, \Delta) - S(f, \Delta, \Xi) + S(f, \Delta, \Xi) - I_R
    <
    \frac \varepsilon 2 + \frac \varepsilon 2
    =
    \varepsilon.
  \end{equation*}

  Thus, the net \( \{ \oline{S}(f, \Delta) \}_{\Delta \in \op{part}([a, b])} \) of upper Darboux sums converges to \( I \). We can analogously show that the lower Darboux sums also converge to \( I_R \). Hence, \( I_R \) is the Darboux integral of \( f \).
\end{proof}

\begin{proposition}\label{thm:countinuous_functions_integrable}
  In a \hyperref[def:frechet_space]{Frechet space} \( (X, \rho) \), \hyperref[def:global_continuity]{continuous functions} \( f: [a, b] \to X \) are \hyperref[def:darboux_integrability]{Darboux integrable}.
\end{proposition}
\begin{proof}
  Fix \( \delta > 0 \). Let \eqref{eq:def:riemann_partition/partition} be a partition of \( [a, b] \) such that \( \diam(\Delta) < \delta \). We have
  \begin{equation*}
    \omega(f, \Delta)
    =
    \sum_{k=1}^n \omega(f, [x_{k-1}, x_k]) (x_k - x_{k-1})
    \leq
    \sum_{k=1}^n \omega(f, \diam(\Delta)) \diam(\Delta)
    <
    n \omega(f, \delta) \delta.
  \end{equation*}

  Now fix \( \varepsilon > 0 \). A continuous function on a compact interval is \hyperref[def:uniform_continuity]{uniformly continuous}. By \fullref{thm:def:function_oscillation/continuity_condition}, there exists \( \delta_0 > 0 \) such that \( \omega(f, \delta_0) < \varepsilon \). It is then enough to choose
  \begin{equation*}
    \delta \coloneqq \frac {\delta_0} {n \varepsilon}
  \end{equation*}
  to obtain
  \begin{equation*}
    \omega(f, \Delta)
    <
    n \delta \omega(f, \delta)
    =
    \delta_0 \frac {\omega(f, \delta)} {\varepsilon}
    \reloset {\ref{thm:def:function_oscillation/monotone}} \leq
    \delta_0 \frac {\omega(f, \delta_0)} {\varepsilon}
    <
    \varepsilon.
  \end{equation*}

  Therefore, the same inequality holds for all partitions with diameters smaller than \( \delta \), which implies that \( f \) is Darboux integrable.
\end{proof}

\begin{proposition}\label{thm:componentwise_integration}
  Let \( f: [a, b] \to \BbbR^n \) be a function and let \( f_k, k = 1, \ldots, n \) be its components. We have that \( f \) is integrable if and only if \( f_k \) is integrable for \( k = 1, \ldots, n \). Furthermore,
  \begin{equation}\label{eq:thm:componentwise_integration}
    \bigintss_a^b \begin{pNiceMatrix} f_1(x) \\ \vdots \\ f_n(x) \end{pNiceMatrix} dx
    =
    \begin{pNiceMatrix} {\displaystyle \int_a^b f_1(x)} dx \\ \vdots \\ {\displaystyle \int_a^b f_n(x) dx} \end{pNiceMatrix}.
  \end{equation}
\end{proposition}
\begin{proof}
  \SufficiencySubProof Let \( f \) be integrable and let \( I = (I_1, \ldots, I_n)^T \) be the value of the integral. Fix \( \varepsilon > 0 \) and let \( (\Delta, \Xi) \) be a tagged partition such that
  \begin{equation*}
    \norm{I - S(f, \Delta, \Xi)} < \varepsilon.
  \end{equation*}

  Then for any \( k = 1, \ldots, n \) we have
  \begin{equation*}
    \norm{I - S(f, \Delta, \Xi)}^2
    =
    \sum_{m=1}^n \abs{I_m - S(f_m, \Delta, \Xi)}^2
    \geq
    \abs{I_k - S(f_k, \Delta, \Xi)}^2,
  \end{equation*}
  hence
  \begin{equation*}
    \abs{I_k - S(f_k, \Delta, \Xi)} < \varepsilon.
  \end{equation*}

  Therefore, \( f_k \) is integrable and
  \begin{equation*}
    \int_a^b f_k(x) dx = I_k.
  \end{equation*}

  \NecessitySubProof Let \( f_k \) be integrable for \( k = 1, \ldots, n \) with value \( I_k \). Put \( I \coloneqq (I_1, \ldots, I_n)^T \). Fix \( \delta > 0 \) and let \( (\Delta_k, \Xi_k) \) be a parition such that
  \begin{equation*}
    \abs{I_k - S(f_k, \Delta_k, \Xi_k)} < \delta
  \end{equation*}

  Let \( \Gamma \coloneqq \bigcup_{k=1}^n \Delta_k \) and let \( \Eta \) be tags for \( \Gamma \). Since \( \diam(\Gamma) \leq \diam(\Delta_k) \) and since \( f_k \) is integrable, we have
  \begin{equation*}
    \abs{I_k - S(f_k, \Gamma, \Eta)} < \delta \quad\forall k = 1, \ldots, n.
  \end{equation*}

  We have
  \begin{equation*}
    \norm{I - S(f, \Gamma, \Eta)}
    =
    \sqrt{\sum_{m=1}^n \abs{I_m - S(f_m, \Gamma, \Eta)}^2}
    <
    \delta \sqrt{n}.
  \end{equation*}

  Therefore, given \( \varepsilon > 0 \), it is enough to choose \( \delta \coloneqq \frac {\varepsilon} {\sqrt n} \) to obtain a tagged partition \( (\Gamma_0, \Eta_0) \), so that for \( (\Gamma, \Eta) \) with \( \diam(\Gamma) < \diam(\Gamma_0) \) we have
  \begin{equation*}
    \norm{I - S(f, \Gamma, \Eta)} < \varepsilon.
  \end{equation*}

  This proves integrability of \( f \) and \eqref{eq:thm:componentwise_integration}.
\end{proof}

  \subsection{Total variation}\label{subsec:total_variation}

\begin{definition}\label{def:riemann_stieltjes_integral}
  The most common generalization of the \hyperref[def:riemann_integral]{Riemann integral} is the Riemann-Stieltjes integral. It is not as well-behaved, hence we will give the most general definition and not attempt to prove equivalences.

  Let \( X \) be a real \hyperref[def:separation_axioms/T2]{Hausdorff} \hyperref[def:topological_vector_space]{topological vector space}. Fix two \hyperref[def:function]{functions} \( f, \alpha: [a, b] \to X \).

  The \term{Riemann sum} of \( f \) with respect to \( \alpha \) corresponding to the \hyperref[def:riemann_partition/tagged]{tagged partition} \eqref{eq:def:riemann_partition/tagged} is defined as
  \begin{equation*}
    S(f, \alpha, \Delta, \Xi) \coloneqq \sum_{k=1}^n f(\xi_k) (\alpha(x_k) - \alpha(x_{k-1})).
  \end{equation*}

  The limit of the net
  \begin{equation}\label{eq:def:riemann_stieltjes_integral/net}
    \{ S(f, \alpha, \Delta, \Xi) \}_{(\Delta, \Xi) \in \op{tpart}([a, b])},
  \end{equation}
  if it exists, is called the \term{Riemann-Stieltjes integral} of \( f \) with respect to \( \alpha \) and is denoted by
  \begin{equation*}
    \int_a^b f(x) d \alpha(x).
  \end{equation*}
\end{definition}

  \subsection{Nonsmooth derivatives}\label{subsec:nonsmooth_derivatives}

\begin{remark}\label{rem:nonsmooth_analysis}
  Nonsmooth analysis studies generalized differentiability for functions which are not necessarily differentible. The generalized derivatives (see \fullref{subsec:nonsmooth_derivatives}) are not linear, which motivates the study of subdifferentials (see \fullref{subsec:subdifferentials}).

  Both optimization in Euclidean spaces and infinite-dimensional optimization studies (not necessarily linear) real-valued functionals. Hence, we are only concerned with studying real-valued topological vector spaces.
\end{remark}

\begin{remark}\label{rem:nonsmooth_differentiability}
  Unlike in \fullref{def:differentiability}, we do not introduce terminology for differentiability because actual differentiability refers to linear approximations of \( f: U \to Y \) with some consistency properties. We will say that \enquote{\( f \) has a Clarke derivative at \( x_0 \) in the direction \( h \)} rather than \enquote{\( f \) is Clarke differentiable at \( x_0 \) in the direction \( h \)}.
\end{remark}

\begin{definition}\label{def:nonsmooth_derivatives}
  Let \( X \) be a real Hausdorff \hyperref[def:topological_vector_space]{topological vector spaces} and let \( U \subseteq X \) be an open set.

  We fix a point \( x_0 \in U \) and a direction \( h \in X \).

  \begin{thmenum}
    \thmitem{def:nonsmooth_derivatives/directional}\Fullref{def:differentiability} already introduced directional derivatives. Here we introduce a special notation for them:
    \begin{equation*}
      D_h^+ f(x_0) = f_+'(x_0)(h) \coloneqq \lim_{t \downarrow 0} \frac {f(x_0 + th) - f(x_0)} t.
    \end{equation*}

    \thmitem{def:nonsmooth_derivatives/dini}\mcite[definition 11.18]{Clarke2013} The upper (resp. lower) \term{Dini derivative} is defined as
    \begin{balign*}
      \overline{D}_h f(x_0) = \overline{f'}(x_0)(h) & \coloneqq \limsup_{t \downarrow 0} \frac {f(x_0 + th) - f(x_0)} t
      \\
      \underline{D}_h f(x_0) = \underline{f'}(x_0)(h) & \coloneqq \liminf_{t \downarrow 0} \frac {f(x_0 + th) - f(x_0)} t
    \end{balign*}

    Dini derivatives are useful when the difference quotients are bounded, but do not have a limit.

    \thmitem{def:nonsmooth_derivatives/clarke}\mcite[section 10.1]{Clarke2013} The \term{generalized Clarke derivative} is defined as
    \begin{equation*}
      D_h^\circ f(x_0)
      =
      f^\circ(x_0)(h)
      \coloneqq
      \limsup_{\substack{y \to x_0 \\ t \downarrow 0}} \frac {f(y + th) - f(y)} t.
    \end{equation*}

    Refer to \fullref{subsec:clarke_gradients} for their usefulness.
  \end{thmenum}
\end{definition}

  \section{Convex functions}\label{sec:convex_functions}

Let \( X \) be a Hausdorff \hyperref[def:topological_vector_space]{topological vector space} and \( D \) be a \hyperref[def:convex_hull]{convex} subset of \( X \).

\begin{definition}\label{def:convex_functions}
  A function \( f: D \to \BbbR \) is called \term{convex} if any of the following equivalent conditions hold:

  \begin{thmenum}
    \thmitem{def:convex_functions/ineq} For any two points \( x, y \in D \) and any \( t \in [0, 1] \) we have
    \begin{equation*}
      f(tx + (1-t)y) \leq tf(x) + (1-t)f(y).
    \end{equation*}

    \thmitem{def:convex_functions/epi} The \hyperref[def:epigraph]{epigraph}
    \begin{equation*}
      \epi f \coloneqq \{ (x, a) \in X \times \BbbR \colon f(x) \leq a \}
    \end{equation*}
    is convex.
  \end{thmenum}

  If \( -g \) is convex for some function \( g: D \to \BbbR \), we call \( g \) \term{concave}.

  Note that definitions do not require any topological structure on \( X \). Most of their properties, however, require a topology.
\end{definition}
\begin{proof}
  Let \( x, y \in D \) and let \( t \in [0, 1] \).

  \ImplicationSubProof{def:convex_functions/ineq}{def:convex_functions/epi} Let \( \epi f \) be a convex set. Obviously \( (x, f(x)) \in D \) and \( (y, f(y)) \in D \). By the convexity of \( \epi f \), we have
  \begin{equation*}
    f(tx + (1-t)y) \leq tf(x) + (1-t)f(y).
  \end{equation*}

  Thus, \( f \) is a convex function.

  \ImplicationSubProof{def:convex_functions/epi}{def:convex_functions/ineq} Let \( f \) be convex. Let \( a \geq f(x) \) and \( b \geq f(y) \), so that \( (x, a) \in \epi f \) and \( (y, b) \in \epi f \). Hence,
  \begin{equation*}
    f(tx + (1-t)y) \leq tf(x) + (1-t)f(y) \leq ta + (1-t)b,
  \end{equation*}
  which implies that
  \begin{equation*}
    (tx + (1-t)y, ta + (1-t)b) \in \epi f.
  \end{equation*}

  Thus, \( \epi f \) is a convex set.
\end{proof}

\begin{definition}\label{def:affine_operators_concave_and_convex}
  \hyperref[def:affine_operator]{Affine functions} \( f: X \to \BbbR \) are simultaneously convex and concave.
\end{definition}

\begin{proposition}\label{thm:convex_subdifferential_is_convex_and_weak*_closed}\mcite[exer. 1.10]{Phelps1993Convex}
  For any convex function \( f \) and any \( x \in D \), the set \( \partial f(x) \) is convex and weak* closed.
\end{proposition}
\begin{proof}
  Fix \( x \in D \). If \( \partial f(x) \) is empty, then the theorem is trivially true.

  Suppose it is nonempty and \( y^*, z^* \in \partial f(x) \). For any \( x \in D \) we then have
  \begin{balign*}
     & \inprod{y^*} {x - x} \leq f(x) - f(x), \\
     & \inprod{z^*} {x - x} \leq f(x) - f(x).
  \end{balign*}

  Fix \( t \in [0, 1] \) and \( x \in D \). It follows that
  \begin{balign*}
    \inprod{t y^* + (1-t) z^*} {x - x}
     & =
    t \inprod{y^*} {x - x} + (1-t) \inprod{z^*} {x - x}
    \leq \\ &\leq
    t [f(x) - f(x)] + (1-t) [f(x) - f(x)]
    =    \\ &=
    f(x) - f(x),
  \end{balign*}
  thus \( t y^* + (1-t)z^* \in \partial f(x) \) and hence \( \partial f(x) \) is convex.

  To prove weak*-closedness, we consider the decomposition
  \begin{balign*}
    \partial f(x)
     & =
    \{ x^* \in E^* \colon \forall x \in D, \inprod {x^*} {x - x} \leq f(x) - f(x) \}
    =    \\ &=
    \bigcap_{x \in D} \{ x^* \in E^* \colon \inprod {x^*} {x - x} \leq f(x) - f(x) \}
    =    \\ &=
    \bigcap_{x \in D} L(x)^{-1} (-\infty, f(x) - f(x)],
  \end{balign*}
  where
  \begin{balign*}
     & L: E \to E^{**}                  \\
     & L(x)(x^*) = \inprod {x^*} {x - x}.
  \end{balign*}

  For each \( x \in E \), the functionals \( L(x) \) are weak*-to-weak continuous because the image \( L(E) \subseteq E^{**} \) is isometrically isomorphic to a translation of \( E \). Hence, the preimage \( L(x)^{-1} (-\infty, f(x) - f(x)] \) is closed and \( \partial f(x) \) is weak*-closed as the intersection of weak*-closed sets.
\end{proof}

\begin{lemma}\label{thm:convex_difference_quotient_grows}
  For every point \( x \in X \) and every direction \( h \in S_X \) the difference quotient is a monotone function of \( t > 0 \), i.e. for \( 0 < s < t \)
  \begin{balign*}
    \frac {f(x + sh) - f(x)} s
    \leq
    \frac {f(x + th) - f(x)} t
  \end{balign*}
\end{lemma}
\begin{proof}
  \begin{balign*}
    \frac {f(x + sh) - f(x)} s
    =
    \frac t s \frac {f(x + \frac s t t h) - f(x)} t
    =
    \frac t s \frac {f\left(\frac s t (x + th) + (1 - \frac s t) x \right) - f(x)} t
    \leq \\ \leq
    \frac t s \frac {\frac s t f(x + t h) + (1 - \frac s t) f(x) - f(x)} t
    =
    \frac t s \frac s t \frac {f(x + th) - f(x)} t
    =
    \frac {f(x + th) - f(x)} t
  \end{balign*}
\end{proof}

\begin{proposition}\label{thm:convex_one_sided_derivatives_exist}
  For every point \( x \in X \) and every direction \( h \in S_X \) the one-sided derivative \( f_+'(x)(h) \) exists.
\end{proposition}
\begin{proof}
  We use the convexity of \( f \) to obtain
  \begin{balign*}
    f(x) = f \left(x + \frac {th} 2 - \frac {th} 2 \right) \leq \frac {f(x + th) + f(x - th)} 2,
    \\
    0 \leq [f(x - th) - f(x)] + [f(x + th) - f(x)],
    \\
    -[f(x - th) - f(x)] \leq [f(x + th) - f(x)],
    \\
    -\frac {f(x + t(-h)) - f(x)} t \leq \frac {f(x + th) - f(x)} t,
  \end{balign*}
  thus the difference quotient in \( f_+'(x)(h) \) is bounded below by the difference quotient for \( -f_+'(x)(-h) \).

  \Fullref{thm:convex_difference_quotient_grows} implies that the right difference quotient is non-increasing, thus both limits exist and
  \begin{equation*}
    -f_+'(x)(-h) \leq f_+'(x)(h).
  \end{equation*}
\end{proof}

\begin{proposition}\label{thm:convex_one_sided_derivatives_sublinear}
  For every point \( x \in X \) and every direction \( h \in S_X \) the one-sided derivative \( f_+'(x)(h) \) is a \hyperref[def:sublinear_functional]{sublinear functional}.
\end{proposition}
\begin{proof}
  \SubProofOf{def:sublinear_functional/subadditivity} It follows directly from
  \begin{balign*}
    \frac {f(x + t(a + b)) - f(x)} t
     & =
    \frac {f(\tfrac 1 2 (x + 2ta) + \tfrac 1 2 (x + 2tb)) - f(x)} t
    \leq \\ &\leq
    \frac {\tfrac 1 2 f(x + 2ta) + \tfrac 1 2 f(x + 2tb) - f(x)} t
    =    \\ &=
    \frac {f(x + 2ta) - f(x)} {2t} + \frac {f(x + 2tb) - f(x)} {2t}.
  \end{balign*}

  \SubProofOf{def:sublinear_functional/positive_homogeneity} For \( \lambda > 0 \) the equality \( f_+'(x)(\lambda h) = \lambda f_+'(x)(h) \) follows from
  \begin{balign*}
    \frac {f(x + t \lambda h) - f(x)} t
    =
    \lambda \frac {f(x + t \lambda h) - f(x)} {t \lambda}
  \end{balign*}
\end{proof}

\begin{corollary}\label{thm:convex_one_sided_derivative_negative_inequality}
  \begin{equation*}
    -f_+'(x)(-h) \leq f_+'(x)(h)
  \end{equation*}
\end{corollary}
\begin{proof}
  \begin{equation*}
    0 = f_+'(x)(h + (-h)) \leq f_+'(x)(h) + f_+'(x)(-h)
  \end{equation*}
\end{proof}

\begin{proposition}\label{thm:convex_iff_subdifferential_nonempty}
  The continuous function \( f: D \to X \) is convex if and only if its subdifferential \( \partial f(x) \) (see \fullref{def:subdifferentials/convex}) is nonempty for every \( x \) in \( D \).
\end{proposition}
\begin{proof}
  \todo{Prove}.
\end{proof}

\begin{proposition}
  \label{thm:convex_one_sided_derivative_is_max}
  For every direction \( h \in S_X \), we have that
  \begin{equation*}
    f_+'(x)(h) = \max\{ \inprod {x^*} h \colon x^* \in \partial f(x) \}.
  \end{equation*}
\end{proposition}

\begin{theorem}\label{thm:singleton_subdifferential_implies_gateaux}
  If \( f \) is continuous and if the subdifferential \( \partial f(x) \) at \( x \in X \) is a singleton with element \( x^* \), then \( f \) is Gateaux differentiable at \( x \) and \( f_G'(x) = x^* \).
\end{theorem}
\begin{proof}
  Let \( h \in S_X \) be arbitrary. \Fullref{thm:convex_one_sided_derivatives_exist} implies that the one-sided derivatives \( f_+'(x)(-h) \) and \( f_+'(x)(h) \) exist and
  \begin{equation*}
    -f_+'(x)(-h) \leq f_+'(x)(h).
  \end{equation*}

  Assume that \( f \) is not Gateaux differentiable at \( x \), i.e. for some \( h_0 \in X \), we have a strict inequality. Then by \fullref{thm:convex_one_sided_derivative_is_max}
  \begin{balign*}
    \min\{ \inprod {x^*} {h_0} \colon x^* \in \partial f(x) \}
    =
    -\max\{ \inprod {x^*} {-h_0} \colon x^* \in \partial f(x) \}
    =
    -f_+'(x)(-h_0)
    < \\ <
    f_+'(x)(h_0)
    =
    \max\{ \inprod {x^*} {h_0} \colon x^* \in \partial f(x) \},
  \end{balign*}
  which implies that there is more that one functional \( x^* \in \partial_C f(x) \). This contradicts the assumption of the theorem.

  Thus, \( f \) is Gateaux differentiable at \( x \).
\end{proof}

\begin{theorem}\label{thm:rn_continuous_convex_partial_derivatives_imply_gateaux}\mcite[exer. 1.15(b]{Phelps1993Convex})
  In \( \BbbR^n \), the existence of the partial derivatives at \( x \) for a continuous convex function \( f: D \to \BbbR \) at a point \( x \in D \) implies Gateaux differentiability.
\end{theorem}
\begin{proof}
  Let \( D \subseteq \BbbR^n \) be an open and convex set and let \( f: D \to \BbbR \) be continuous and convex. Then \( f_+'(x) \) exists everywhere by \fullref{thm:convex_one_sided_derivatives_exist} and is a subdifferential functional by \fullref{thm:convex_one_sided_derivatives_sublinear}.

  Let \( e_1, \ldots, e_n \) be the canonical basis for \( \BbbR^n \).

  The partial derivatives
  \begin{balign*}
    \frac {\partial f} {\partial x_i} (x)
    \coloneqq
    \lim_{t \to 0} \frac {f(x + t e_i) - f(x)} t
    =
    f_+'(x)(e_i)
  \end{balign*}
  exist, hence the projections of \( f_+'(x) \) along the coordinate exes are linear.

  Define line linear functional
  \begin{equation*}
    l(h) \coloneqq \sum_{i=1}^n h_i \inprod{\frac {\partial f} {\partial x_i} (x)} h,
  \end{equation*}
  where \( h_1, \ldots, h_n \) are the coordinates of \( h \) along \( e_1, \ldots, e_n \).

  We will show that \( l \cong f_+' \). Fix \( h \in S_X \). We have
  \begin{balign}\label{thm:rn_continuous_convex_partial_derivatives_imply_gateaux/diff_dominated}
    f_+'(x)(h)
     & =
    f_+'(x)\left(\sum_{i=1}^n h_i e_i \right)
    \reloset {\text{sublinearity}} \leq \nonumber      \\ &\leq
    \sum_{i=1}^n f_+'(x)(h_i e_i)
    \reloset {\text{linearity along } e_i} = \nonumber \\ &=
    \sum_{i=1}^n h_i f_+'(x)(e_i)
    =
    \sum_{i=1}^n h_i \inprod{\frac {\partial f} {\partial x_i} (x)} h.
  \end{balign}

  Thus,
  \begin{balign*}
    \inprod l h
    =
    -\inprod l {-h}
    \reloset {\ref{thm:rn_continuous_convex_partial_derivatives_imply_gateaux/diff_dominated}} \leq
    -f_+'(x)(-h)
    \reloset {\text{\ref{thm:convex_one_sided_derivative_negative_inequality}}} \leq
    f_+'(x)(h)
    \reloset {\ref{thm:rn_continuous_convex_partial_derivatives_imply_gateaux/diff_dominated}} \leq
    \inprod l h,
  \end{balign*}
  i.e. \( f_+'(x)(h) = \inprod l h \) for all \( h \in S_X \), hence \( f_+'(x) \) is a linear functional and \( f \) is Gateaux differentiable at \( x \).
\end{proof}

\begin{theorem}\label{thm:rn_continuous_convex_gateaux_implies_frechet}\mcite[exer. 1.15(a]{Phelps1993Convex})
  In \( \BbbR^n \), Gateaux differentiability of a continuous convex function \( f: D \to \BbbR \) at a point \( x \in D \) implies Frechet differentiability.
\end{theorem}
\begin{proof}
  Since \( f \) is Gateaux differentiable (\fullref{def:differentiability/gateaux}) at \( x \), the derivative \( f'(x) = f_+'(x) \) is linear.

  Because \( f \) is continuous and convex, it is locally Lipschitz with constant \( L \) in some \( \delta \)-ball with center \( x \).

  Suppose that \( f \) is not Frechet differentiable at \( x \). Inverting the condition in \fullref{def:differentiability/frechet}, we obtain that there exist \( \varepsilon > 0 \) and a sequence \( \{ h_n \}_n \subseteq B(x, \delta) \setminus \{ 0 \} \) such that \( \norm{h_n} \to 0 \) and yet for all \( n \in \BbbZ_{>0} \),
  \begin{balign}\label{thm:rn_continuous_convex_gateaux_implies_frechet/frechet_assumption}
    \abs{f(x + h_n) - f(x) - \inprod{f'(x)} {h_n}} > \varepsilon \norm{h_n}.
  \end{balign}

  Define
  \begin{balign*}
    t_n \coloneqq \norm{h_n}
     &  &
    u_n \coloneqq \frac{h_n} {\norm {h_n}}.
  \end{balign*}

  Obviously \( t_{n_k} \downarrow 0 \). The vectors \( h_n \) are linearly independent since otherwise \( f \) would not be Gateaux differentiable at \( x \), hence \( u_n \) are not all equal.

  Since \( S_{\BbbR^n} \) is compact, by the Bolzano-Weierstrass theorem, there exists a convergent subsequence \( \{ u_{n_k} \}_k \underset {k \to \infty} \to u_0 \) of \( \{ u_n \}_n \). We have

  \begin{balign}\label{thm:rn_continuous_convex_gateaux_implies_frechet/frechet_estimate}
     & \phantom= \abs{\frac {f(x + t_{n_k} u_{n_k}) - f(x)} {t_{n_k}} - \inprod{f'(x)} {u_{n_k}}}
    \leq \nonumber
    \abs{\frac {f(x + t_{n_k} u_{n_k}) - f(x + t_{n_k} u_0)} {t_{n_k}}} +                       \\ &+ \abs{\frac {f(x + t_{n_k} u_0) - f(x)} {t_{n_k}} - \inprod{f'(x)} {u_0}} + \abs{\inprod{f'(x)} {u_0 - u_{n_k}}}
    \leq \nonumber                                                                              \\ &\leq
    L \norm{u_{n_k} - u_0} + \abs{\frac {f(x + t_{n_k} u_0) - f(x)} {t_{n_k}} - \inprod{f'(x)} {u_0}} + \norm{f'(x)} \norm{u_0 - u_{n_k}}.
  \end{balign}

  Fix \( \delta > 0 \). Because of the Gateaux differentiable of \( f \) at \( x \), we can pick \( k_0 \) such that
  \begin{equation*}
    \abs{\frac {f(x + t_{n_{k_0}} u_0) - f(x)} {t_{n_{k_0}}} - \inprod{f'(x)} {u_0}} < \delta.
  \end{equation*}

  Because \( \{ u_{n_k} \}_k \) converges to \( u_0 \), we can choose \( k_1 \) such that
  \begin{equation*}
    \norm{u_0 - u_{n_{k_1}}} < \delta.
  \end{equation*}

  Thus, for \( k > \max \{ k_0, k_1 \} \), \fullref{thm:rn_continuous_convex_gateaux_implies_frechet/frechet_estimate} is bounded by
  \begin{balign*}
    \abs{\frac {f(x + t_{n_k} u_{n_k}) - f(x)} {t_{n_k}} - \inprod{f'(x)} {u_{n_k}}}
    \leq
    (L + 1 + \norm{f'(x)}) \delta.
  \end{balign*}

  It suffices to choose \( \delta > 0 \), so that
  \begin{equation*}
    \delta < \frac 1 {L + 1 + \norm{f'(x)}}
  \end{equation*}
  in order to have, for \( k > \max \{ k_0, k_1 \} \),
  \begin{equation*}
    \abs{\frac {f(x + t_{n_k} u_{n_k}) - f(x)} {t_{n_k}} - \inprod{f'(x)} {u_{n_k}}} < \varepsilon.
  \end{equation*}

  But this contradicts \fullref{thm:rn_continuous_convex_gateaux_implies_frechet/frechet_assumption}, hence \( f \) is Frechet differentiable at \( x \).
\end{proof}

\begin{corollary}\label{thm:rn_continuous_convex_partial_derivatives_imply_frechet}
  In \( \BbbR^n \), the existence of the partial derivatives at \( x \) for a continuous convex function \( f: D \to \BbbR \) at a point \( x \in D \) is equivalent to Frechet differentiability.
\end{corollary}
\begin{proof}
  A direct consequence of and \fullref{thm:rn_continuous_convex_partial_derivatives_imply_gateaux} and \fullref{thm:rn_continuous_convex_gateaux_implies_frechet}.
\end{proof}

\begin{theorem}\label{thm:rn_continuous_convex_frechet_almost_everywhere}\mcite[exer. 1.17]{Phelps1993Convex}
  In \( \BbbR^n \), continuous convex functions \( f: D \to \BbbR \) are differentiable almost everywhere.
\end{theorem}
\begin{proof}
  For all \( h \in S_X \) and small enough \( t > 0 \) we define
  \begin{balign*}
     & \varphi_h^t: D \to \BbbR
     & \varphi_h^t(x) \coloneqq \frac {f(x + th) - f(x)} t
  \end{balign*}
  and \( \varphi_h(x) \coloneqq f_+'(x)(h) = \lim_{t \downarrow 0} \varphi_h^t(x) \).

  Considered as functions of \( x \), \( \varphi_h^t \) are obviously continuous hence Borel measurable, and so \( \varphi_h \) is also Borel measurable.

  Denote by
  \begin{balign*}
    B_h
    \coloneqq
    \{ x \in D \colon -f_+'(x)(-h) < f_+'(x)(h) \}
    =
    \{ x \in D \colon -\varphi_{-h}(x) - \varphi_h(x) < 0 \}
  \end{balign*}
  the set of points \( x \in D \) where the one-sided derivative \( f_+'(x)(h) \) is not linear, given a fixed direction \( h \in S_X \). If \( B_h \) is nonempty, \( f \) is not differentiable at \( x \).

  The sets \( B_h \) are Borel sets since they are the preimages of \( (-\infty, 0) \) under a Borel function. We will show that it is a null set for every direction \( h \).

  Fix \( h \in S_X \). Denote by \( \delta_x \coloneqq \sup \{ t > 0 \colon x + th \in D \} \).

  The function \( t \mapsto f(x + th) \) is a convex function of one variable. By \cite[theorem 1.16]{Phelps1993Convex}, it is differentiable \( \mu_1 \)-almost everywhere in \( [0, \delta_x) \), where \( \mu_m \) is the Lebesgue \( m \)-measure.

  Denote
  \begin{balign*}
     & H \coloneqq \linspan\{ h \} \cong \BbbR^1,
    \\
     & H^\perp \cong \BbbR^{n-1} \text{ - the orthogonal complement of \( H \) in \( \BbbR^n \)},
    \\
     & L_x \coloneqq \{ x + th, 0 \leq t < \delta_x \} - half-open segments in D.
  \end{balign*}

  THe whole domain \( D \) can be represented as \( D = \cup \{ L_x \colon x \in H^\perp \} \).

  We can now use Fubini's theorem to show that \( B_h \) is a null set:
  \begin{balign*}
    \mu_n(B_h)
    =
    \int_{B_h} dz
    =
    \int_{\BbbR^n = H^\perp \oplus H} \chi_{B_h} (z) dz
    =
    \int_{H^\perp} \int_{L_x} \chi_{B_h} (y) dy dx
    = \\ =
    \int_{H^\perp} \mu_1(B_h \cap L_x) dx
    =
    \int_{H^\perp} 0 dx
    =
    0.
  \end{balign*}

  Hence, for all \( h \in S_X \), \( -f_+'(x)(-h) = f_+'(x)(h) \) for almost all \( x \in D \).

  In particular, if \( e_1, \ldots, e_n \) is the canonical basis of \( \BbbR^n \), the \( i \)-th partial derivative \( \frac{\partial f} {\partial x_i} (x) \) exists only in \( D \ B_{e_i} \).

  The gradient
  \begin{equation*}
    \nabla f(x) = \left( \frac{\partial f} {\partial x_1} (x), \ldots, \frac{\partial f} {\partial x_n} (x) \right)
  \end{equation*}
  then exists in
  \begin{equation*}
    \hat D \coloneqq (D \ B_{e_1}) \cap \ldots \cap (D \ B_{e_n}) = D \setminus \left( \bigcup_{i=1}^n B_{e_i} \right).
  \end{equation*}

  \Fullref{thm:rn_continuous_convex_partial_derivatives_imply_frechet} then implies that \( f \) is Frechet differentiable in \( \hat D \), i.e. almost everywhere in \( D \).
\end{proof}

  \section{Subdifferentials}\label{sec:subdifferentials}

Let \( X \) be a Hausdorff \hyperref[def:topological_vector_space]{topological vector space}, let \( D \subseteq X \) be an open set and \( f: D \to \BbbR \) be any function.

\begin{definition}\label{def:subdifferentials}
  We fix a point \( x \in D \). We define different types of \term{subgradients} and \term{subdifferentials}. Subgradients are linear functionals \( x^* \in X^* \) that approximate \( f \) at the point \( x \) in a certain way, and a subdifferential is the set of all subgradients of a given type.

  \begin{thmenum}
    \thmitem{def:subdifferentials/convex}\mcite[59]{Clarke2013OptimalControl}We say that \( x^* \in X^* \) is a \term{subgradient of \( f \) at \( x \)} if for every \( y \in D \) we have
    \begin{equation*}
      f(y) - f(x) \geq \inprod {x^*} {y - x}.
    \end{equation*}

    The \term{subdifferential of \( f \) at \( x \)} is denoted by \( \partial f(x) \) and is also sometimes called the \term{convex subdifferential} because of \cref{thm:convex_iff_subdifferential_nonempty}.

    \thmitem{def:subdifferentials/clarke}\mcite[def. 10.3]{Clarke2013OptimalControl}We say that \( x^* \in X^* \) is a \term{Clarke (generalized) subgradient of \( f \) at \( x \)} if for every direction \( h \in X \) we have
    \begin{equation*}
      f^\circ(x)(h) \geq \inprod {x^*} h,
    \end{equation*}
    where \( f^\circ(x)(h) \) is the generalized Clarke \hyperref[def:nonsmooth_derivatives/clarke]{derivative}.

    The \term{subdifferential of \( f \) at \( x \)} is denoted by \( \partial_C f(x) \). Confusingly, the Clarke subdifferential is called the \enquote{generalized gradient} by Clarke himself with no special name for the Clarke subgradients.

    See \fullref{sec:clarke_gradients} for properties of these subgradients.

    \thmitem{def:subdifferentials/proximal}\mcite[227]{Clarke2013OptimalControl}We say that \( x^* \in X^* \) is a \term{proximal subgradient of \( f \) at \( x \)} if there exist \( \sigma > 0 \) and a neighborhood \( V \subseteq X \) of \( x \) such that for every \( y \in D \cap V \) we have
    \begin{equation*}
      f(y) - f(x) + \sigma \norm{y - x}^2 \geq \inprod {x^*} {y - x}.
    \end{equation*}

    The \term{proximal subdifferential of \( f \) at \( x \)} is denoted by \( \partial_P f(x) \).

    \thmitem{def:subdifferentials/limiting}\mcite[def. 11.10]{Clarke2013OptimalControl}Suppose the following are satisfied:
    \begin{enumerate}
      \item \( \{ x_n \}_n \subseteq D \) is a sequence of points converging to \( x \)
      \item \( f(x_n) \to f(x) \) (redundant if \( f \) is continuous)
      \item \( x_n^* \) is a proximal subgradient for \( f \) at \( x_n \) for every \( n \in \BbbZ_{>0} \).
    \end{enumerate}

    If the limit \( x^* \coloneqq \lim_n x_n^* \) exists and is a continuous linear functional, we call \( x^* \) a \term{limiting subgradient of \( f \) at \( x \)}.

    The \term{limiting subdifferential of \( f \) at \( x \)} is denoted by \( \partial_P f(x) \).
  \end{thmenum}
\end{definition}

  \section{Clarke generalized gradients}\label{sec:clarke_gradients}

Let \( X \) be a Banach space and \( f: X \to \BbbR \) be locally Lipschitz.

\begin{definition}\label{def:clarke_gradient}\mcite[def. 10.3]{Clarke2013OptimalControl}
  Let \( x \in X \) and \( U \subseteq X \) be a neighborhood of x where \( f \) is \( L \)-Lipschitz, i.e.

  \begin{equation*}
    \forall y, z \in U, \abs{f(y) - f(z)} \leq L \norm{y - z}.
  \end{equation*}

  We use the Clarke generalized \hyperref[def:nonsmooth_derivatives/clarke]{derivative},
  \begin{equation*}
    f^\circ(x)(h) \coloneqq \limsup_{\substack{y \to x \\ t \downarrow 0}} \frac {f(y + th) - f(y)} t
  \end{equation*}

  We define the \term{generalized gradient of \( f \) at \( x \)} to be the set
  \begin{equation*}
    \partial_C f(x) \coloneqq \{ x^* \in X^* \colon \forall h \in X, f^\circ(x)(h) \geq \inprod {x^*} h \}.
  \end{equation*}

  We say that the vector \( h \) is a \term{descent direction of \( f \) at \( x \)} if
  \begin{equation*}
    \limsup_{t \downarrow 0} \frac {f(x + th) - f(x)} t < 0.
  \end{equation*}
\end{definition}

\begin{proposition}\label{thm:clarke_derivative_exists}
  The generalized derivative of a locally Lipschitz function \( f: X \to \BbbR \) exists for every \( x \in X \).
\end{proposition}
\begin{proof}
  Let \( x, h \in X \) and let \( U \) be a neighborhood of \( x \) where the Lipschitz condition holds with the constant \( L_U \). Then there exists \( \delta_0 > 0 \) such that \( B(x, \delta_0) \subseteq U \).

  Define \( \delta_1 \coloneqq \frac 1 2 \min \left\{\delta_0, \frac {\delta_0} {\norm h} \right\} < \delta_0 \), so that for \( y \in B(x, \delta_1) \) and \( t \in (0, \delta_1) \) we have
  \begin{balign*}
    \norm{(y + th) - x}
    \leq
    \norm{y - x} + t \norm h
    \leq
    \delta_1 + \delta_1 \norm h
    \leq
    \begin{cases}
      \frac {\delta_0} 2 (1 + \norm h),           & \norm h \leq 1 \\
      \frac {\delta_0} {2 \norm h} (1 + \norm h), & \norm h > 1.
    \end{cases}
  \end{balign*}

  In both cases we get that \( y + th \in B(x, \delta_0) \).

  The generalized derivative in \( x \) in the direction \( h \in X \) is then norm-bounded by
  \begin{balign*}
    \abs{f^\circ(x)(h)}
    =
    \abs{\limsup_{\substack{y \to x                             \\ t \downarrow 0}} \frac {f(y + th) - f(y)} t}
    =
    \abs{\lim_{\delta \to 0} \sup_{\substack{y \in B(x, \delta) \\ t \in (0, \delta)}} \frac {f(y + th) - f(y)} t}
    \leq                                                        \\ \leq
    \abs{\sup_{\substack{y \in B(x, \delta_1)                   \\ t \in (0, \delta_1)}} \frac {f(y + th) - f(y)} t}
    \leq
    \sup_{\substack{y \in B(x, \delta_1)                        \\ t \in (0, \delta_1)}} \frac {\abs{f(y + th) - f(y)}} t
    \leq                                                        \\ \leq
    \sup_{\substack{y \in B(x, \delta_1)                        \\ t \in (0, \delta_1)}} \frac {\norm{(y + th) - (y)}} t
    =
    \norm h.
  \end{balign*}

  The fact that \( f \) is locally Lipschitz gave us that the supremum is taken over a bounded set and thus the generalized derivative exists.
\end{proof}


  \section{Complex analysis}\label{sec:complex_analysis}

Complex analysis an extension of \fullref{sec:real_analysis}, which is concerned with studying functions with values in \hyperref[thm:vector_space_dimension]{finite-dimensional} \hyperref[def:complex_numbers]{complex} \hyperref[def:hilbert_space]{Hilbert spaces} \( \BbbC^n \) rather than Euclidean spaces \( \BbbR^n \). A lot of results are different, however much of \fullref{sec:real_analysis} is delegated here because it holds in greater generality. For these general results, through the section, \( \BbbK \) will refer to either \( \BbbR \) or \( \BbbC \).

Despite complex analysis being a very rich field, we are mostly concerned with special functions, to which we dedicate the following sections:
\begin{itemize}
  \item \Fullref{subsec:power_series}
  \item \Fullref{subsec:trigonometric_functions}
  \item \Fullref{subsec:exponential_function}
  \item \Fullref{subsec:trigonometric_polynomials}
  \item \Fullref{subsec:special_functions}
\end{itemize}

  \section{Complex functions}\label{sec:complex_functions}

\begin{definition}\label{def:sequence_spaces}
  We will define multiple Banach spaces of sequences over \( \BbbC \).

  \begin{thmenum}
    \thmitem{def:sequence_spaces/c00} The simplest nontrivial sequence space is that of all sequences with only finitely many nonzero elements. It is denoted by \( c_{00} \). It can be defined as
    \begin{equation*}
      c_{00} \coloneqq \bigcup_{i=1}^\infty \BbbC^k,
    \end{equation*}
    where \( \BbbC^k \) is the corresponding space of tuples.

    This space can be generalized to modules over \hyperref[def:module]{semirings}.
  \end{thmenum}
\end{definition}

\begin{definition}\label{def:function_spaces}
  We will define multiple Banach spaces of functions over \( \BbbK \).

  \begin{thmenum}
    \thmitem{def:function_spaces/c0} Define the set of functions \term{vanishing at infinity}:
    \begin{equation*}
      C_0(\BbbC) \coloneqq \{ f: \BbbC \to \BbbC \colon f(x) \xrightarrow[\abs{x} \to \infty]{} 0 \}.
    \end{equation*}

    \thmitem{def:function_spaces/c} Fix \hyperref[def:topological_space]{topological space} \( X \). The set \( C(X) = C(X, \BbbK) \) of all \( \BbbK \)-valued continuous functions on \( X \) in a Banach space over \( \BbbK \).
  \end{thmenum}
\end{definition}

\begin{theorem}[Arzela-Ascoli]\label{thm:arzela_ascoli}\mcite[cor. 10.49]{Knapp2016BasicAlgebra}
  Let \( X \) be a \hyperref[def:compact_space]{compact} \hyperref[def:separation_axioms/T2]{Hausdorff} space.

  A family \( \mscrF \subseteq C(X, \BbbR) \) of continuous real-valued functions is totally \hyperref[def:totally_bounded_set]{bounded} if and only if it is pointwise \hyperref[def:bounded_function/pointwise]{bounded} and \hyperref[def:function_set_continuity/equicontinuous]{equicontinuous}.
\end{theorem}

  \subsection{Series}\label{subsec:series}

Here \( (X, \norm) \) will refer to a Banach space over \( \BbbK \).

\begin{definition}\label{def:convergent_series}
  When extending addition to a countable amount of terms, we need to impose some regularity conditions to avoid contradictions. The topologies of \( \BbbR \) and \( \BbbC \) are complete and allow us to define convergent and divergent series. We define series in great generality because the theory easily allows it.

  A \term{numeric series} or simply \term{series} is an infinite sequence \( x_0, x_1, \ldots \in X \), which we call \term{terms}, usually written as
  \begin{equation}\label{def:convergent_series/series}
    \sum_{k=0}^\infty x_k.
  \end{equation}

  To each series, there corresponds its sequence of \term{partial sums}
  \begin{equation*}
    S_n \coloneqq \sum_{k=0}^n x_k, n = 0, 1, 2, \ldots
  \end{equation*}

  We can equivalently define a series as a sequence of partial sums and then recover the terms as
  \begin{equation*}
    x_k \coloneqq \begin{cases}
      S_0,           & k = 0, \\
      S_k - S_{k-1}, & k > 0
    \end{cases}
  \end{equation*}

  We say that the series \eqref{def:convergent_series/series} \term{converges} to a value \( x \) if \( \lim_{n \to \infty} S_n = x \) in the sense of \fullref{def:net_convergence/limit}. The value \( x \) is called the \term{sum} of the series.

  If a series does not converge, we say that it is \term{divergent}.

  If the related series
  \begin{equation}\label{def:convergent_series/absolute_series}
    \sum_{k=0}^\infty \norm{x_k}
  \end{equation}
  converges, we say that \eqref{def:convergent_series/series} is \term{absolutely convergent}.
\end{definition}

\begin{example}\label{ex:series}
  Several examples of series are
  \begin{itemize}
    \item An absolutely convergent series is \eqref{thm:geometric_progression/series_sum}.
    \item A divergent series is the harmonic series \eqref{eq:ex:harmonic_series/harmonic}.
    \item A convergent, but not absolutely convergent series is the alternating harmonic series \eqref{eq:ex:harmonic_series/alternating}.
  \end{itemize}
\end{example}

\begin{proposition}\label{thm:absolutely_convergent_series_is_convergent}
  An absolutely convergent series is convergent.
\end{proposition}
\begin{proof}
  Suppose that \eqref{def:convergent_series/absolute_series} converges.

  By the triangle inequality, for each index \( n \) we have
  \begin{equation*}
    \norm{\sum_{k=0}^n x_k} \leq \sum_{k=0}^n \norm{x_k} \leq \sum_{k=0}^\infty \norm{x_k}.
  \end{equation*}

  Thus, the sequence \( \left\{ \norm{\sum_{k=0}^{n} x_k} \right\}_{n=0}^\infty \) is a bounded (by \( \sum_{k=0}^\infty \norm{x_k} \)) monotone sequence, which by \fullref{thm:real_monotone_sequence_converges_iff_bounded} is convergent.

  Therefore, the series \eqref{def:convergent_series/series} is convergent.
\end{proof}

\begin{remark}\label{rem:establish_series_convergence_by_absolute_series}
  Convergence of the series \eqref{def:convergent_series/series} can be established using the convergence of the nonnegative series \eqref{def:convergent_series/absolute_series}.

  The convergence of the latter can be established using techniques in \fullref{subsec:real_series} like \fullref{thm:cauchys_root_test} or \fullref{thm:dalamberts_ratio_test}.
\end{remark}

\begin{proposition}\label{thm:infinitary_triangle_inequality}
  For every series \eqref{def:convergent_series/series} we have
  \begin{equation}\label{thm:infinitary_triangle_inequality/inequality}
    \norm{\sum_{k=0}^\infty x_k} \leq \sum_{k=0}^\infty \norm{x_k},
  \end{equation}
  where both limits are allowed to be infinite.
\end{proposition}
\begin{proof}
  If the series on the right diverges, the inequality is obviously true.

  Suppose that it is convergent. By \fullref{thm:absolutely_convergent_series_is_convergent}, the limit
  \eqref{def:convergent_series/series} exists.

  By the triangle inequality, for each index \( n \) we have
  \begin{equation*}
    \norm{\sum_{k=0}^n x_k} \leq \sum_{k=0}^n \norm{x_k}.
  \end{equation*}

  By \fullref{thm:one_sided_squeeze_lemma}, since both sequences are convergent, we obtain \fullref{thm:infinitary_triangle_inequality/inequality}.
\end{proof}

\begin{proposition}\label{thm:convergent_series_terms_vanish}
  The terms of the convergent series \eqref{def:convergent_series/series} vanish as \( k \to \infty \), that is,
  \begin{equation*}
    \lim_{k \to \infty} x_k = 0.
  \end{equation*}
\end{proposition}
\begin{proof}
  Since the series is convergent, its sequence of partial sums converges, i.e. the partial sums get arbitrarily close to each other. Then
  \begin{equation*}
    \norm{x_n} = \norm{S_n - S_{n-1}} \to 0.
  \end{equation*}
\end{proof}

\begin{theorem}\label{thm:product_of_series_convergence}
  Consider two convergent series
  \begin{equation}\label{thm:product_of_series_convergence/a}
    A \coloneqq \sum_{k=0}^\infty x_k
  \end{equation}
  and
  \begin{equation}\label{thm:product_of_series_convergence/b}
    B \coloneqq \sum_{k=0}^\infty y_k.
  \end{equation}

  If either \fullref{thm:product_of_series_convergence/a} or \fullref{thm:product_of_series_convergence/b} converges absolutely, then
  \begin{equation}\label{thm:product_of_series_convergence/prod}
    \sum_{k=0}^\infty \sum_{m=0}^k x_m y_{k-m} = AB.
  \end{equation}
\end{theorem}

\begin{proposition}[Cauchy's series convergence criterion]\label{thm:cauchy_series_convergence_criterion}\mcite[3.22]{Rudin1976Principles}
  The series \eqref{def:convergent_series/series} converges if and only if for every \( \varepsilon > 0 \) there exists an index \( k_0 \) such that
  \begin{equation*}
    \norm{\sum_{k=m}^n x_k} < \varepsilon \quad\forall m, n \geq k_0.
  \end{equation*}
\end{proposition}
\begin{proof}
  This is simply a restatement of \fullref{thm:cauchys_net_convergence_criterion}.
\end{proof}

\begin{proposition}[Cauchy's series continuity criterion]\label{thm:cauchy_series_continuity_criterion}\mcite[\textnumero 265]{ФихтенгольцОсновыТом2}
  Fix a topological space \( A \) and a set \( S \subseteq A \). Let \( \{ f_k \}_{k=0}^\infty \) be a sequence of continuous functions from \( S \) to \( X \).

  Define the function \( f: S \to X \) as
  \begin{equation}\label{thm:cauchy_series_continuity_criterion/function}
    f(x) \coloneqq \sum_{k=0}^\infty f_k(x).
  \end{equation}

  A sufficient condition for \( f \) to be continuous in \( S \) is that for every \( \varepsilon > 0 \) there exists an index \( K \) such that
  \begin{equation*}
    \norm{\sum_{k=m}^n f(x)} < \varepsilon \quad\forall m, n \geq K
  \end{equation*}
  simultaneously for all \( x \in S \).
\end{proposition}
\begin{proof}
  This is simply a restatement of \fullref{thm:uniform_limit_of_continuous_functions} in the style of \fullref{thm:cauchy_series_convergence_criterion}.
\end{proof}

\begin{corollary}[Weierstrass' series criterion]\label{thm:weierstrass_series_criterion}\mcite[\textnumero 265]{ФихтенгольцОсновыТом2}
  Let \( S \) be any set and \( \{ f_k \}_{k=0}^\infty \) be a sequence of functions from \( S \) to \( X \). Consider the series \fullref{thm:cauchy_series_continuity_criterion/function}. If
  \begin{equation*}
    \forall k \in \BbbZ^{>0} \ \exists M_k \in \BbbR^{>0} \ \forall x \in S : \norm{f_k(x)} < M_k
  \end{equation*}
  and if the series
  \begin{equation}\label{thm:weierstrass_series_criterion/dominating}
    \sum_{k=0}^\infty M_k
  \end{equation}
  converges, then the limit \fullref{thm:cauchy_series_continuity_criterion/function} exists for every \( x \in S \) and, furthermore, the series converges absolutely and uniformly.

  In analogy to \fullref{thm:positive_series_comparison}, we say that the series \fullref{thm:weierstrass_series_criterion/dominating} \term{dominates} the series \fullref{thm:cauchy_series_continuity_criterion/function}.

  In particular, if \( S \) has a topology and the functions \( f_k(x), k = 0, 1, \ldots \) are continuous (resp. uniformly continuous), so is \( f(x) \).
\end{corollary}
\begin{proof}
  By \fullref{thm:positive_series_comparison}, the series
  \begin{equation*}
    \sum_{k=0}^\infty \norm{f_k(x)}
  \end{equation*}
  converges for any \( x \in S \), hence \fullref{thm:cauchy_series_continuity_criterion/function} converges absolutely for any \( x \in S \).

  Furthermore, each of the functions \( f_k(x) \) is bounded by \( B(0, M_k) \) and \( M_k \) does not depend on \( x \), hence the convergence is uniform.

  The rest of the theorem follows from \fullref{thm:uniform_limit_of_continuous_functions}.
\end{proof}

\begin{corollary}\label{thm:continuous_function_series_powers_of_two}
  Let \( X \subseteq \BbbR \) be a nonempty set. Consider the series of real-valued real functions
  \begin{equation}\label{thm:continuous_function_series_powers_of_two/series}
    f(x) \coloneqq \sum_{k=0}^\infty \frac {f_k(x)} {2^k},
  \end{equation}
  where \( \{ f_k \}_{k=0}^\infty \subseteq B_{C(X)} \) is a sequence of continuous functions bounded in \( [-1, 1] \).

  Then \( f(x) \) is defined and continuous for all \( x \in X \).
\end{corollary}
\begin{proof}
  For \( \abs{x} \leq 1 \), the series is dominated by the geometric series \fullref{eq:ex:n_ary_decomposition/binary}, which sums to \( 2 \), hence by \fullref{thm:weierstrass_series_criterion} \( f(x) \) is continuous in the interval \( [-1, 1] \).

  Note that
  \begin{equation*}
    f(2x) \coloneqq \sum_{k=0}^\infty \frac x {2^{k-1}} = 2 f(x),
  \end{equation*}
  hence the series \fullref{thm:continuous_function_series_powers_of_two/series} also converges for \( \abs{x} \leq 2 \).

  By induction on \( n \), we show that \( f(2^n x) = 2^n f(x) \) and thus \( f(x) \) is continuous in \( B(0, 2^n) \), therefore also on the entire real line \( \BbbR \).
\end{proof}

\begin{example}\label{thm:weierstrass_series_criterion/counterexample}\mcite[\textnumero 266]{ФихтенгольцОсновыТом2}
  Consider the real series
  \begin{equation*}
    f(x) \coloneqq \sum_{k=0}^\infty x^k (1 - x).
  \end{equation*}

  It converges for \( \abs{x} < 1 \) because it is dominated by a convergent geometric series.

  For \( x \in (0, 1) \),
  \begin{equation*}
    f(x)
    =
    \sum_{k=0}^\infty x^k (1 - x)
    =
    \sum_{k=0}^\infty x^k - \sum_{k=1}^\infty x^k
    =
    1.
  \end{equation*}

  But
  \begin{equation*}
    \lim_{t \uparrow 1} f(x) = 1 \neq 0 = f(1) = f(\lim_{t \uparrow 1} t).
  \end{equation*}

  This shows that \( f(x) \) is not continuous, despite every term being continuous.

  By contraposition to \fullref{thm:weierstrass_series_criterion}, it follows that no series that dominates \( f(x) \) converges.
\end{example}

\begin{theorem}\label{thm:uniform_limit_exchange}\mcite[\textnumero 268]{ФихтенгольцОсновыТом2}
  Fix a uniform space \( (A, \mscrU) \) and let \( S \subseteq A \). Let \( f_k: S \to X, k = 0, 1, \ldots \) be a sequence of functions and assume that \( x_0 \in M \) is a limit point of each of these functions.

  \begin{thmenum}
    \thmitem{thm:uniform_limit_exchange/sequence} If the sequence \( \{ f_k \}_{k=0}^\infty \) converges uniformly on \( S \), we can exchange the limits
    \begin{equation*}
      \lim_{x \to x_0} \lim_{k \to \infty} f_k(x)
      =
      \lim_{k \to \infty} \lim_{x \to x_0} f_k(x).
    \end{equation*}

    \thmitem{thm:uniform_limit_exchange/series} If the series \fullref{thm:cauchy_series_continuity_criterion/function} converges uniformly on \( S \), we can exchange the limits
    \begin{equation*}
      \lim_{x \to x_0} \sum_{k=0}^\infty f_k(x)
      =
      \sum_{k=0}^\infty \lim_{x \to x_0} f_k(x).
    \end{equation*}
  \end{thmenum}
\end{theorem}

\begin{remark}\label{rem:thm:uniform_limit_exchange_continuity}
  If the functions \( f_k \) in \fullref{thm:uniform_limit_exchange/series} are continuous at \( x_0 \), we have the additional equality
  \begin{equation}\label{thm:uniform_limit_exchange/continuous_equality}
    \lim_{x \to x_0} f(x)
    =
    \lim_{x \to x_0} \sum_{k=0}^\infty f_k(x)
    =
    \sum_{k=0}^\infty \lim_{x \to x_0} f_k(x)
    \reloset * =
    \sum_{k=0}^\infty f_k\left(\lim_{x \to x_0} x \right)
    =
    f\left(\lim_{x \to x_0} x \right),
  \end{equation}
  thus \( f \) is continuous at \( x_0 \). The continuity actually follows from \fullref{thm:cauchy_series_continuity_criterion} directly.
\end{remark}

\begin{corollary}\label{thm:riemann_intergral_limit_exchange}\mcite[\textnumero 269]{ФихтенгольцОсновыТом2}
  Let \( \{ f_k \}_{k=0}^\infty \subseteq C([a, b], \BbbR) \).

  \begin{thmenum}
    \thmitem{thm:riemann_intergral_limit_exchange/sequence} If the sequence \( \{ f_k \}_{k=0}^\infty \) converges uniformly, then
    \begin{equation*}
      \lim_{k \to \infty} \int_a^b f_k(x) dx = \int_a^b \lim_{k \to \infty} f_k(x) dx.
    \end{equation*}

    \thmitem{thm:riemann_intergral_limit_exchange/series} If the series \fullref{thm:cauchy_series_continuity_criterion/function} converges uniformly, then
    \begin{equation*}
      \int_a^b f(x) dx = \int_a^b \sum_{k=0}^\infty f_k(x) dx = \sum_{k=0}^\infty \int_a^b f_k(x) dx.
    \end{equation*}
  \end{thmenum}
\end{corollary}
\begin{proof}
  \SubProofOf{thm:riemann_intergral_limit_exchange/sequence} Assume that the sequence \( \{ f_k \}_{k=0}^\infty \) converges uniformly to \( f \). Then by \fullref{thm:uniform_limit_exchange}, we note that for any index \( k \), the difference \( r_k(x) \coloneqq f(x) - f_k(x) \) is continuous, hence integrable, and
  \begin{equation*}
    \int_a^b f(x) dx = \int_a^b f_k(x) dx + \int_a^b r_k(x) dx.
  \end{equation*}

  Because of the uniform convergence, for any \( \delta > 0 \) and there exist an index \( k_0 \) such that
  \begin{equation*}
    \abs{f(x) - f_k(x)} = \abs{r_k(x)} < \delta \quad\forall k \geq k_0, \forall x \in [a, b].
  \end{equation*}

  Then
  \begin{equation*}
    \abs{\int_a^b f(x) dx - \int_a^b f_k(x) dx} = \abs{\int_a^b r_k(x) dx} < (b - a) \delta.
  \end{equation*}

  Given \( \varepsilon > 0 \), we define \( \delta \coloneqq \frac \varepsilon {b - a} \) to obtain an index \( k_0 \) such that
  \begin{equation*}
    \abs{\int_a^b f(x) dx - \int_a^b f_k(x) dx} = \abs{\int_a^b r_k(x) dx} < \varepsilon \quad\forall k \geq k_0.
  \end{equation*}

  Thus, \fullref{def:net_convergence/limit} is satisfied and equality holds.

  \SubProofOf{thm:riemann_intergral_limit_exchange/series} This is a special case of \fullref{thm:riemann_intergral_limit_exchange/sequence}.
\end{proof}

\begin{corollary}\label{thm:derivative_limit_exchange}\mcite[thm. 7.17]{Rudin1976Principles}
  Let \( \{ f_k \}_{k=0}^\infty \subseteq C^1([a, b], \BbbR) \). Suppose that the series \fullref{thm:cauchy_series_continuity_criterion/function} converges for at least one point \( x_0 \in [a, b] \).

  \begin{thmenum}
    \thmitem{thm:derivative_limit_exchange/sequence} If the sequence \( \{ D f_k \}_{k=0}^\infty \) of derivatives converges uniformly, then \( \{ f_k \}_{k=0}^\infty \) also converges uniformly, its limit is differentiable in \( (a, b) \) and
    \begin{equation*}
      D\left(\lim_{k \to \infty} f_k(x) \right) = \lim_{k \to \infty} D f_k(x).
    \end{equation*}

    \thmitem{thm:derivative_limit_exchange/series} If the series of derivatives
    \begin{equation}\label{thm:derivative_limit_exchange/derivative_series}
      \sum_{k=0}^\infty D f_k(x)
    \end{equation}
    converges uniformly, then \fullref{thm:cauchy_series_continuity_criterion/function} converges uniformly, is differentiable in \( (a, b) \) and
    \begin{equation*}
      D\left(\sum_{k=0}^\infty f_k(x)\right) = \sum_{k=0}^\infty D f_k(x).
    \end{equation*}
  \end{thmenum}
\end{corollary}
\begin{proof}
  \SubProofOf{thm:derivative_limit_exchange/sequence} Fix \( \varepsilon > 0 \). Since the sequence \( \{ f_k \}_{k=0}^\infty \) converges for \( x_0 \), there exists an index \( k_0 \) such that
  \todo{Prove complex case}
  \begin{equation*}
    \abs{f_m(x_0) - f_n(x_0)} < \varepsilon \quad\forall m, n \geq k_0.
  \end{equation*}

  Furthermore, there exists an index \( k_1 \) such that
  \begin{equation*}
    \abs{D f_m(x) - D f_n(x)} < \varepsilon \quad\forall x \in [a, b] \ \forall m, n \geq k_0.
  \end{equation*}

  Fix \( m, n \geq k_0 \) and \( x \in [a, b] \). Note that the function \( f_m - f_n \) is differentiable and thus by the mean value theorem, there exists \( \xi \) between \( x_0 \) and \( x \) such that
  \begin{equation*}
    \frac {[f_m(x) - f_n(x)] - [f_m(x_0) - f_n(x_0)]} {x - x_0} = D f_m(\xi) - D f_n(\xi).
  \end{equation*}

  Thus,
  \begin{balign*}
    \abs{f_m(x) - f_n(x)}
     & \leq
    \abs{f_m(x_0) - f_n(x_0)} + (x - x_0)\abs{D f_m(\xi) - D f_n(\xi)}
    <       \\ &<
    2 (x - x_0) \varepsilon
    \leq    \\ &\leq
    2 (b - a) \varepsilon.
  \end{balign*}

  Therefore, the limit \( \lim_{k\to\infty} f_k(x) \) exists. Since \( x \) was arbitrary and \( 2 (b - a) \varepsilon \) does not depend on \( x \), we conclude that
  \begin{equation*}
    f(x) \coloneqq \lim_{k\to\infty} f_k(x)
  \end{equation*}
  is uniformly convergent on \( [a, b] \).

  By the Newton-Leibniz theorem, for the sequence \( \{ D f_k \}_{k=0}^\infty \) of derivatives we have
  \begin{equation*}
    \lim_{k \to \infty} \int_a^x D f_k(t) dt
    =
    \lim_{k \to \infty} [f_k(x) - f_k(a)]
    =
    \lim_{k \to \infty} f_k(x) - \lim_{k \to \infty} f_k(a)
    =
    f(x) - f(a).
  \end{equation*}

  Differentiating both sides, we obtain
  \begin{equation*}
    D\left(\lim_{k \to \infty} \int_a^x D f_k(t) dt \right)
    =
    D\left(\lim_{k \to \infty} f_k(x) dt \right)
    =
    D f(x).
  \end{equation*}

  \Fullref{thm:riemann_intergral_limit_exchange/sequence} allows us to conclude that
  \begin{equation*}
    D f(x)
    =
    D\left(\lim_{k \to \infty} \int_a^x D f_k(t) dt \right)
    =
    D \int_a^x \lim_{k \to \infty} D f_k(t) dt
    =
    \lim_{k \to \infty} D f_k(x).
  \end{equation*}

  \SubProofOf{thm:derivative_limit_exchange/series} This is a special case of \fullref{thm:riemann_intergral_limit_exchange/sequence}.
\end{proof}

\begin{example}\label{ex:harmonic_series}
  We list several important series related to \hyperref[def:harmonic_progression]{harmonic progressions}.

  \begin{thmenum}
    \thmitem{ex:harmonic_series/harmonic} The series
    \begin{equation}\label{eq:ex:harmonic_series/harmonic}
      \sum_{k=1}^\infty \frac 1 k = 1 + \frac 1 2 + \frac 1 3 + \frac 1 4 + \cdots
    \end{equation}
    is called \hi{the} \term{harmonic series}. It diverges as shown in \fullref{thm:harmonic_series_diverges}, which make it much less useful in practice, however it is an important enough example that it has a dedicated name.

    \thmitem{ex:harmonic_series/alternating} The series
    \begin{equation}\label{eq:ex:harmonic_series/alternating}
      \sum_{k=1}^\infty \frac {(-1)^k} k
      =
      \sum_{m=1}^\infty \parens*{ \frac 1 {2m - 1} - \frac 1 {2m} }
      =
      1 - \frac 1 2 + \frac 1 3 - \frac 1 4 + \cdots
    \end{equation}
    is called the \term{alternating harmonic series}. It converges, but not absolutely --- \fullref{thm:alternating_harmonic_series_convergence}.

    \thmitem{ex:harmonic_series/hyperharmonic} For any \( s \in \BbbC \), the series
    \begin{equation}\label{eq:ex:harmonic_series/hyperharmonic}
      \sum_{k=1}^\infty \frac 1 {k^s}.
    \end{equation}
    is called the \term{hyperharmonic series}.

    Unlike the harmonic series, the hyperharmonic series sometimes converges --- see \fullref{thm:hyperharmonic_series_convergence}.
  \end{thmenum}
\end{example}

\begin{proposition}\label{thm:harmonic_series_diverges}
  The harmonic series \eqref{eq:ex:harmonic_series/harmonic} diverges.
\end{proposition}
\begin{proof}
  Define the series
  \begin{equation*}
    1 + \frac 1 2 + \underbrace{\frac 1 4 + \frac 1 4}_{\sfrac 1 2} + \underbrace{\frac 1 8 + \frac 1 8 + \frac 1 8 + \frac 1 8}_{\sfrac 1 2} + \underbrace{\frac 1 {16} + \cdots + \frac 1 {16}}_{\sfrac 1 2} + \cdots
  \end{equation*}

  It is divergent as the sum of infinitely many \( \sfrac 1 2 \). Furthermore, it is dominated by the harmonic series:
  \begin{align*}
    &1 + \frac 1 2 + \frac 1 3 + \frac 1 4 + \frac 1 5 + \frac 1 6 + \frac 1 7 + \frac 1 8 + \frac 1 9 \thinspace + \cdots + \frac 1 {16} + \cdots
    \\
    &1 + \frac 1 2 + \overbrace{ \frac 1 4 + \frac 1 4 }^{\sfrac 1 2} + \overbrace{ \frac 1 8 + \frac 1 8 + \frac 1 8 + \frac 1 8 }^{\sfrac 1 2} + \overbrace{ \frac 1 {16} + \cdots + \frac 1 {16} }^{\sfrac 1 2} + \cdots
  \end{align*}

  Thus, by \fullref{thm:positive_series_comparison}, the harmonic series also diverges.
\end{proof}

\begin{proposition}\label{thm:alternating_harmonic_series_convergence}\mcite[\textnumero 247]{ФихтенгольцОсновыТом2}
  Consider the alternating harmonic series \eqref{eq:ex:harmonic_series/alternating}.

  Compare the series with \eqref{eq:ex:harmonic_series/harmonic}. Note that, by \fullref{thm:leibniz_alternating_series_test}, the series is convergent. It is not absolutely convergent, because the harmonic series \eqref{eq:ex:harmonic_series/harmonic} is divergent.
\end{proposition}

\begin{proposition}\label{thm:hyperharmonic_series_convergence}
  The hyperharmonic series \eqref{ex:harmonic_series/hyperharmonic} converges all \( s \in \BbbC \) with \( \real(s) > 1 \).
\end{proposition}
\begin{proof}
  Let \( s = (1 + \varepsilon) + bi \). We use the integral test on the series \( \sum_{k=1}^\infty \abs{k}^{-s} \):
  \begin{equation*}
    \int_1^\infty \frac 1 {\abs{x^s}} \dl x
    =
    \int_1^\infty \frac 1 {x^{1 + \varepsilon} \underbrace{\abs{x^{bi}}}_{1}} \dl x
    =
    -\frac 1 {\varepsilon x^\varepsilon}\Big\restr_{x=1}^\infty
    =
    \frac 1 \varepsilon \lim_{x \to \infty} \parens*{ 1 - \frac 1 {x^\varepsilon} }
    =
    \frac 1 \varepsilon.
  \end{equation*}

  The integral is finite, hence the hyperharmonic series is absolutely convergent.
\end{proof}

  \subsection{Power series}\label{subsec:power_series}

\begin{definition}\label{def:convergent_power_series}
  Let \( \BbbK\Bracks{X} \) be the space of formal power series defined in \fullref{def:formal_power_series}.

  To each formal power series
  \begin{equation*}
    \sum_{k=0}^\infty a_k X^k
  \end{equation*}
  there corresponds a function, called a \term{power series}
  \begin{equation}\label{def:convergent_power_series/series}
    p(x) \coloneqq \sum_{k=0}^\infty a_k x^k.
  \end{equation}

  We sometimes slightly generalize this notion slightly by using a \enquote{shift} by \( \alpha \in \BbbK \): define the function
  \begin{equation}\label{def:convergent_power_series/shifted_series}
    p(x) \coloneqq \sum_{k=0}^\infty a_k (x - \alpha)^k.
  \end{equation}

  If the limit exists (as a \hyperref[def:convergent_series]{numeric series}) for a certain \( x \in \BbbK \), we say that the series \term{converges} at \( x \).

  The series is no longer \enquote{formal} because it is now a proper function instead of an abstract algebraic object, although a power series may only be defined in a subset of \( \BbbK \) (that is, a \hyperref[def:set_valued_map/partial]{partial function}).
\end{definition}

\begin{theorem}\label{thm:power_series_radius_of_convergence}
  For every power series \eqref{def:convergent_power_series/series}, there exists a nonnegative extended real number \( r \in [0, +\infty] \), called its \term{radius of convergence}, such that \eqref{def:convergent_power_series/series} converges absolutely if \( \abs{x} < r \) and diverges if \( \abs{x} > r \).

  The behavior of the series is more complicated when \( \abs{x} = r \) (unless \( r = 0 \), in which case the power series converges if and only if \( x = 0 \)).
\end{theorem}
\begin{proof}
  Define
  \begin{equation*}
    q \coloneqq \limsup_{n \to \infty} \sqrt[n]{\abs{a_n}},
  \end{equation*}
  where we put \( q = +\infty \) if the limit does not exist. We have
  \begin{equation*}
    \limsup_{n \to \infty} \sqrt[n]{\abs{x^n a_n}} = \abs{x} q.
  \end{equation*}

  By \fullref{thm:cauchys_root_test}, \eqref{def:convergent_power_series/series} converges absolutely if \( \abs{z} q < 1 \) and diverges if \( \abs{z} q > 1 \).

  Thus, \( r \coloneqq \tfrac 1 q \) is the desired radius of convergence.

  Note that we may also use \fullref{thm:dalamberts_ratio_test} for finding the same radius of convergence by \fullref{rem:nonnegative_series_convergence_test_equivalence}.
\end{proof}

\begin{definition}\label{def:real_function_parity}\mcite[\textnumero 115]{ФихтенгольцОсновыТом1}
  We say that the \hyperref[def:function]{function} \( f: \BbbR \to \BbbR \) is \term[ru=чётная, en=even (\cite[170]{Carothers2000})]{even} if, for every real number \( x \), we have
  \begin{equation}\label{eq:def:real_function_parity/even}
    f(x^{-1}) = f(x)
  \end{equation}
  and \term[ru=нечётная, en=odd (\cite[170]{Carothers2000})]{odd} if
  \begin{equation}\label{eq:def:real_function_parity/odd}
    f(x^{-1}) = f(x)^{-1}.
  \end{equation}
\end{definition}

\begin{proposition}\label{thm:power_series_parity}
  Power series of the form
  \begin{equation}\label{thm:power_series_parity/odd}
    f_o(z) \coloneqq \sum_{m \text{ is odd}} a_m z^m = \sum_{k=0}^\infty a_{2k+1} z^{2k+1}
  \end{equation}
  are \hyperref[def:real_function_parity]{odd functions} and power series of the form
  \begin{equation}\label{thm:power_series_parity/even}
    f_e(z) \coloneqq \sum_{m \text{ is even}} a_m z^m = \sum_{k=0}^\infty a_{2k} z^{2k}
  \end{equation}
  are even functions.
\end{proposition}
\begin{proof}
  If \eqref{thm:power_series_parity/odd} converges for \( z \in \BbbC \),
  \begin{equation*}
    f_o(-z)
    =
    \sum_{k=0}^\infty a_{2k+1} (-z)^{2k+1}
    =
    \sum_{k=0}^\infty a_{2k+1} (-1)^{2k+1} z^{2k+1}
    =
    - \sum_{k=0}^\infty a_{2k+1} z^{2k+1}
    =
    - f_o(z).
  \end{equation*}

  Analogously, since \( (-1)^{2k} = 1 \), we have \( f_e(-z) = f_e(z) \).
\end{proof}

\begin{proposition}\label{thm:power_series_are_locally_uniform_convergent}
  A power series is \hyperref[def:function_net_convergence/locally_uniform]{locally uniformly convergent} in the interior of its domain of convergence.
\end{proposition}
\begin{proof}
  Assume that the series \eqref{def:convergent_power_series/series} converges inside the ball \( B(0, R) \). Fix \( x \in B(0, R) \) and \( R_x < R - \abs{x} \). Then the geometric series
  \begin{equation*}
    \sum_{k=0}^\infty a_k R_x^k
  \end{equation*}
  converges and dominates \eqref{def:convergent_power_series/series} in the ball \( B(x, R_x) \). Thus, by \fullref{thm:weierstrass_series_criterion}, \eqref{def:convergent_power_series/series} converges uniformly in \( B(x, R_x) \).

  Since the choice of \( x \in B(0, R) \) was arbitrary, we conclude that \eqref{def:convergent_power_series/series} is locally uniformly convergent.
\end{proof}

\begin{theorem}\label{thm:series_termwise_operations}
  Suppose that the power series \eqref{def:convergent_power_series/series} has a (potentially infinite) radius of convergence \( R \).

  \begin{thmenum}
    \thmitem{thm:series_termwise_operations/differentiation} \( p(x) \) is differentiable in \( B(0, R) \) and can be differentiated termwise as
    \begin{equation}\label{thm:series_termwise_operations/derivative}
      p'(x) = \sum_{k=0}^\infty a_{k+1} (k+1) x^k.
    \end{equation}

    Furthermore, \( p'(x) \) has the same radius of convergence as \( p(x) \).

    \thmitem{thm:series_termwise_operations/integration} If the series is real and \( \abs{x} < R \), \( p(x) \) is integrable in \( [0, x] \) (or \( [x, 0] \)) and can be integrated termwise as
    \begin{equation}\label{thm:series_termwise_operations/primitive}
      \int_0^x p(t) dt = \sum_{k=0}^\infty a_k \frac {x^{k+1}} {k+1}.
    \end{equation}
  \end{thmenum}
\end{theorem}
\begin{proof}
  \SubProofOf{thm:series_termwise_operations/differentiation} Note that the right-hand side of \fullref{thm:series_termwise_operations/derivative} is a power series. Furthermore, its radius of convergence is, by \fullref{thm:power_series_radius_of_convergence},
  \begin{equation*}
    \lim_{k \to \infty} \abs{\frac {a_{k+1} (k+1) x^k} {a_{k+2} (k+2) x^{k+1}}}
    =
    \abs{x} \lim_{k \to \infty} \frac {k+1} {k+2} \abs{\frac {a_{k+1}} {a_{k+2}}}
    =
    R.
  \end{equation*}

  Fix \( x \in B(0, R) \) and choose \( r \in (\abs{x}, R) \). Both series are uniformly convergent in \( B(0, r) \). By \fullref{thm:derivative_limit_exchange/sequence}, the equality \fullref{thm:series_termwise_operations/derivative} holds in \( B(0, r) \), hence it also holds for \( x \).

  \SubProofOf{thm:series_termwise_operations/integration} Analogously to \fullref{thm:series_termwise_operations/differentiation}, we conclude that the right-hand side of \fullref{thm:series_termwise_operations/primitive} is a power series with radius of convergence \( R \).

  The rest follows directly from \fullref{thm:riemann_intergral_limit_exchange}.
\end{proof}

  \subsection{Trigonometric functions}\label{subsec:trigonometric_functions}

\begin{definition}\label{def:pi}
  We define the real number \( \pi \) via any of the following equivalent definitions:
  \begin{thmenum}
    \thmitem{def:pi/circle} The ratio of a \hyperref[def:circle]{circle}'s \hyperref[def:circumference]{circumference} to its \hyperref[def:metric_space/diameter]{diameter}.

    \thmitem{def:pi/integral} The integral
    \begin{equation}\label{eq:def:pi/integral}
      \int_{x=-1}^1 \frac 1 { \sqrt{1 - x^2} } \cdot \dl x.
    \end{equation}
  \end{thmenum}
\end{definition}
\begin{defproof}
  Consider some circle \( C \) with center \( O = (x, y) \) and radius \( r \). We will show that the ratio of the circumference to the diameter is always \eqref{eq:def:pi/integral}.

  We start by calculating the circumference. Consider the standard equation
  \begin{equation*}
    (x - x_0)^2 + (y - y_0)^2 = r^2.
  \end{equation*}

  We have
  \begin{equation*}
    r^2
    =
    (x - x_0)^2 + (y - y_0)^2
    =
    (x - x_0)^2 + y^2 - 2 y y_0 + y_0^2.
  \end{equation*}

  Then
  \begin{equation*}
    y^2 + (-2y_0) y + [(x - x_0)^2 + y_0^2 - r^2] = 0.
  \end{equation*}
  and
  \begin{equation*}
    y = \frac {2y_0 \pm \sqrt{ 4y_0^2 - 4(x - x_0)^2 - 4y_0^2 + 4r^2 }} 2 = y_0 \pm \sqrt{ r^2 - (x - x_0)^2 }.
  \end{equation*}

  This allows us to express \( y \) as two functions of \( x \):
  \begin{equation*}
    \begin{aligned}
      &y^\pm: [-r + x_0, r + x_0] \to \BbbR, \\
      &y^\pm(x) \coloneqq y_0 \pm \sqrt{ r^2 - (x - x_0)^2 }.
    \end{aligned}
  \end{equation*}

  \begin{figure}[!ht]
    \centering
    \includegraphics[align=c]{output/def__pi__upper_half_circle.pdf}
    \caption{The graph of \( y^+(x) \).}\label{fig:def:pi/upper_half_circle}
  \end{figure}

  \Fullref{thm:length_of_function_graph} implies that the \hyperref[def:arc_length]{curve length} of the graph \( \gph(y^\pm) \) is
  \begin{equation*}
    \int_{x=-r + x_0}^{r + x_0} \sqrt{ 1 + [\Dl_x y^\pm(x)]^2 } \dl x.
  \end{equation*}

  We have
  \begin{equation*}
    \Dl_x y^\pm(x)
    =
    \Dl_x \sqrt{ r^2 - (x - x_0)^2 }
    =
    -\frac {2(x - x_0)} {2\sqrt{ r^2 - (x - x_0)^2 }}
    =
    -\frac {x - x_0} {\sqrt{ r^2 - (x - x_0)^2 }}.
  \end{equation*}

  Then
  \begin{align*}
    \int_{x=-r + x_0}^{r + x_0} \sqrt{ 1 + [\Dl_x y^\pm(x)]^2 } \cdot \dl x
    &=
    \int_{x=-r + x_0}^{r + x_0} \sqrt{ 1 + \frac {(x - x_0)^2} { r^2 - (x - x_0)^2 } } \cdot \dl x
    = \\ &=
    \int_{x=-r + x_0}^{r + x_0} \sqrt{ \frac {r^2} { r^2 - (x - x_0)^2 } } \cdot \dl x
    = \\ &=
    \int_{x=-r}^r \sqrt{ \frac {r^2} { \sqrt{r^2 - x^2} } } \cdot \dl x
    = \\ &=
    \int_{x=-r}^r \frac r { \sqrt{r^2 - x^2} } \cdot \dl x
    = \\ &=
    r \cdot \int_{x=-1}^1 \frac 1 { \sqrt{1 - x^2} } \cdot \dl x.
  \end{align*}

  Summing the curve lengths of \( \gph(y^+) \) and \( \gph(y^-) \), we obtain that the circumference of \( C \) is
  \begin{equation*}
    2r \cdot \int_{x=-1}^1 \frac 1 { \sqrt{1 - x^2} } \cdot \dl x.
  \end{equation*}

  The ratio to the diameter \( 2r \) is simply the underlying integral \eqref{eq:def:pi/integral}.
\end{defproof}

\begin{definition}\label{def:trigonometric_functions}
  We define the two basic \term{trigonometric functions}. They are also called \term{circular trigonometric functions} to distinguish them from the hyperbolic trigonometric functions defined and motivated in \fullref{def:hyperbolic_trigonometric_functions}.

  \begin{thmenum}
    \thmitem{def:trigonometric_functions/sine} The \term{sine} function, also called the \term{sinus} function, is
    \begin{equation*}
      \sin(z)
      \coloneqq
      -i \sum_{m \text{ is odd}}^\infty \frac {i^m z^m} {m!}
      =
      -i \sum_{k=0}^\infty \frac {i^{2k+1} z^{2k+1}} {(2k + 1)!}
      =
      \sum_{k=0}^\infty \frac {i^{2k} z^{2k+1}} {(2k + 1)!}
    \end{equation*}

    \thmitem{def:trigonometric_functions/cosine} The \term{cosine} function, also called the \term{cosinus} function, is
    \begin{equation*}
      \cos(z)
      \coloneqq
      \sum_{m \text{ is even}}^\infty \frac {i^m z^m} {m!}
      =
      \sum_{k=0}^\infty \frac {i^{2k} z^{2k}} {(2k)!}.
    \end{equation*}
  \end{thmenum}

  \Fullref{thm:right_triangle_trigonometric_functions} justifies the term \enquote{angle} for the \hyperref[def:multi_valued_function/arguments]{parameter} of the trigonometric functions.
\end{definition}

\begin{proposition}\label{thm:def:trigonometric_function}
  The \hyperref[def:trigonometric_functions]{main trigonometric functions} have the following basic properties:
  \begin{thmenum}
    \thmitem{thm:def:trigonometric_function/convergence} Both \( \sin(z) \) and \( \cos(z) \) converge in the entire complex plane.
    \thmitem{thm:def:trigonometric_function/parity} \( \sin(z) \) is an odd function and \( \cos(z) \) is an even function.
    \thmitem{thm:def:trigonometric_function/derivative} \( \sin'(z) = \cos(z) \) and \( \cos'(z) = -\sin(z) \) for all \( z \in \BbbC \).
  \end{thmenum}
\end{proposition}
\begin{proof}
  \SubProofOf{thm:def:trigonometric_function/convergence} Note that the zero coefficients in the expansion of either \( \sin \) or \( \cos \) do not alter convergence. Therefore, by \fullref{thm:power_series_radius_of_convergence}, the radius of convergence is
  \begin{equation*}
    \limsup_{k \to \infty} \frac {\abs{i^{k-1} k!}} {\abs{i^k (k-1)!}}
    =
    \limsup_{k \to \infty} k
    =
    +\infty.
  \end{equation*}

  \SubProofOf{thm:def:trigonometric_function/parity} Follows from \fullref{thm:power_series_parity}.

  \SubProofOf{thm:def:trigonometric_function/derivative} Follows from \fullref{thm:power_series_are_locally_uniform_convergent} and \fullref{thm:derivative_limit_exchange}.
\end{proof}

\begin{proposition}\label{thm:trigonometric_identities}
  We have the following basic trigonometric identities:
  \begin{thmenum}
    \thmitem{thm:trigonometric_identities/pythagorean_identity} (Pythagorean identity) For any \( z \in \BbbC \),
    \begin{equation}\label{eq:thm:trigonometric_identities/pythagorean_identity}
      \sin(z)^2 + \cos(z)^2 = 1.
    \end{equation}

    \thmitem{thm:trigonometric_identities/products} (Products) For \( x, y \in \BbbC \),
    \begin{balign}
      2 \sin(x) \sin(y) &= \cos(x - y) - \cos(x + y) \label{eq:thm:trigonometric_identities/products/ss}  \\
      2 \cos(x) \cos(y) &= \cos(x - y) + \cos(x + y) \label{eq:thm:trigonometric_identities/products/cc}  \\
      2 \sin(x) \cos(y) &= \sin(x - y) + \sin(x + y) \label{eq:thm:trigonometric_identities/products/sc}  \\
      2 \cos(x) \sin(y) &= -\sin(x - y) + \sin(x + y) \label{eq:thm:trigonometric_identities/products/cs}
    \end{balign}

    \thmitem{thm:trigonometric_identities/sums} (Sums) For \( x, y \in \BbbC \),
    \begin{balign}
      \sin(x) + \sin(y) &= 2 \cos\left(\frac{x - y} 2 \right) \sin\left(\frac{x + y} 2 \right) \label{eq:thm:trigonometric_identities/sums/sin_sum} \\
      \sin(x) - \sin(y) &= 2 \sin\left(\frac{x - y} 2 \right) \cos\left(\frac{x + y} 2 \right) \label{eq:thm:trigonometric_identities/sums/sin_diff} \\
      \cos(x) + \cos(y) &= 2 \cos\left(\frac{x - y} 2 \right) \cos\left(\frac{x + y} 2 \right) \label{eq:thm:trigonometric_identities/sums/cos_sum} \\
      \cos(x) - \cos(y) &= -2 \sin\left(\frac{x - y} 2 \right) \sin\left(\frac{x + y} 2 \right) \label{eq:thm:trigonometric_identities/sums/cos_diff}
    \end{balign}

    \thmitem{thm:trigonometric_identities/sum_of_angles} (Sum of angles) For \( x, y \in \BbbC \),
    \begin{balign}
      \sin(x + y) &= \cos(x) \sin(y) + \sin(x) \cos(y) \label{eq:thm:trigonometric_identities/sum_of_angles/sin} \\
      \cos(x + y) &= \cos(x) \cos(y) - \sin(x) \sin(y) \label{eq:thm:trigonometric_identities/sum_of_angles/cos}
    \end{balign}
  \end{thmenum}
\end{proposition}
\begin{proof}
  We first use Cauchy multiplication for the power series \( \cos(v) \) and \( \cos(w) \):
  \begin{balign}
    \cos(v) \cos(w)
    &=
    \left( \sum_{k=0}^\infty \frac {i^{2k} v^{2k}} {(2k)!} \right) \Ast \left( \sum_{k=0}^\infty \frac {i^{2k} w^{2k}} {(2k)!} \right)
    = \nonumber \\ &=
    \sum_{k=0}^\infty \sum_{m=0}^k \frac {i^{2m} v^{2m}} {(2m)!} \frac {i^{2(k-m)} w^{2(k-m)}} {(2(k-m))!}
    = \nonumber \\ &=
    \sum_{k=0}^\infty \frac {i^{2k}} {(2k)!} \sum_{m=0}^k \binom {2k} {2m} v^{2m} w^{2(k-m)}. \label{eq:thm:trigonometric_identities/cos_product}
  \end{balign}

  Analogously,
  \begin{balign}
    \sin(v) \sin(w)
    &=
    (-i) (-i) \left( \sum_{k=0}^\infty \frac {i^{2k+1} v^{2k+1}} {(2k+1)!} \right) \Ast \left( \sum_{k=0}^\infty \frac {i^{2k+1} w^{2k+1}} {(2k+1)!} \right)
    = \nonumber \\ &=
    -\sum_{k=0}^\infty \sum_{m=0}^k \frac {i^{2m+1} v^{2m+1}} {(2m+1)!} \frac {i^{2(k-m)+1} w^{2(k-m)+1}} {(2(k-m)+1)!}
    = \nonumber \\ &=
    -\sum_{k=0}^\infty \frac {i^{2(k+1)}} {(2(k+1))!} \sum_{m=0}^k \binom {2(k+1)} {2m+1} v^{2m+1} w^{2(k-m)+1}
    = \nonumber \\ &=
    -\sum_{k=1}^\infty \frac {i^{2k}} {(2k)!} \sum_{m=0}^{k-1} \binom {2k} {2m+1} v^{2m+1} w^{2k-(2m+1)}. \label{eq:thm:trigonometric_identities/sin_product}
  \end{balign}

  \SubProofOf{thm:trigonometric_identities/pythagorean_identity} From \eqref{eq:thm:trigonometric_identities/cos_product} and \eqref{eq:thm:trigonometric_identities/sin_product} we have
  \begin{equation*}
    \sin(z)^2 + \cos(z)^2
    =
    1 + \sum_{k=1}^\infty \frac {i^{2k} z^{2k}} {(2k)!} \underbrace{\left[-\sum_{m=0}^{k-1} \binom {2k} {2m+1} + \sum_{m=0}^k \binom {2k} {2m} \right]}_{\eqqcolon a_k}.
  \end{equation*}

  It remains to show that the expression \( a_k \) equals zero for all \( k = 1, 2, \ldots \). We have
  \begin{equation*}
    a_k
    =
    \sum_{m=0}^k \binom {2k} {2m} - \sum_{m=0}^{k-1} \binom {2k} {2m+1}
    =
    \sum_{m=0}^k (-1)^m \binom {2k} m
    \reloset {\ref{thm:binomial_theorem}} =
    1 - 1 = 0.
  \end{equation*}

  \Fullref{eq:thm:trigonometric_identities/pythagorean_identity} follows.

  \SubProofOf{thm:trigonometric_identities/products} We will only prove \eqref{eq:thm:trigonometric_identities/products/cc} because the other identities are proved analogously. We have
  \begin{balign*}
    \cos(v - w) + \cos(v + w)
    &=
    \sum_{k=0}^\infty \frac {i^{2k}} {(2k)!} \left[(v - w)^{2k} + (v + w)^{2k} \right]
    \reloset {\ref{thm:binomial_theorem}} = \\ &=
    \sum_{k=0}^\infty \frac {i^{2k}} {(2k)!} \sum_{m=0}^{2k} \binom {2k} m v^{2k-m} w^m \left[ (-1)^m + 1 \right]
    = \\ &=
    2 \sum_{k=0}^\infty \frac {i^{2k}} {(2k)!} \sum_{m=0}^{2k} \binom {2k} {2m} v^{2(k-m)} w^{2m}
    \reloset {\eqref{eq:thm:trigonometric_identities/cos_product}} = \\ &=
    2 \cos(v) \cos(w).
  \end{balign*}

  \SubProofOf{thm:trigonometric_identities/sums} Fix some \( v, w \in \BbbC \) and define
  \begin{balign*}
    x \coloneqq \frac {v + w} 2
    &&
    y \coloneqq \frac {v - w} 2
  \end{balign*}
  so that \( v = x + y \) and \( w = x - y \).

  The identity \eqref{eq:thm:trigonometric_identities/sums/sin_sum} the follows from \eqref{eq:thm:trigonometric_identities/products/sc} applied to \( x \) and \( y \). The other identities are proved analogously.

  \SubProofOf{thm:trigonometric_identities/sum_of_angles} We will only prove \eqref{eq:thm:trigonometric_identities/sum_of_angles/sin} because \eqref{eq:thm:trigonometric_identities/sum_of_angles/cos} is proved analogously. From \eqref{eq:thm:trigonometric_identities/products/cs},
  \begin{balign*}
    \sin(x + y)
     & =
    2 \cos(x) \sin(y) + \sin(x - y)
    \reloset {\eqref{eq:thm:trigonometric_identities/products/sc}} = \\ &=
    2 \cos(x) \sin(y) + 2 \cos(x) \sin(y) - \sin(x + y).
  \end{balign*}

  After dividing by \( 2 \), we obtain \eqref{eq:thm:trigonometric_identities/sum_of_angles/sin}.
\end{proof}

\begin{proposition}\label{thm:trigonometric_function_basic_roots}
  We have the following important special values:
  \begin{align}
    \sin(0) = 0,                && \cos(0) = 1,               \label{eq:thm:trigonometric_function_basic_roots/zero} \\
    \sin(\ifrac \pi 2) = 1,     && \cos(\ifrac \pi 2) = 0,    \label{eq:thm:trigonometric_function_basic_roots/half_pi} \\
    \sin(\pi) = 0,              && \cos(\pi) = -1,            \label{eq:thm:trigonometric_function_basic_roots/pi} \\
    \sin(\ifrac {3\pi} 2) = -1, && \cos(\ifrac {3\pi} 2) = 0, \label{eq:thm:trigonometric_function_basic_roots/sesqui_pi}
  \end{align}
\end{proposition}
\begin{proof}
  \SubProofOf{eq:thm:trigonometric_function_basic_roots/zero} Follows directly from \fullref{def:trigonometric_functions}.

  \SubProofOf{eq:thm:trigonometric_function_basic_roots/pi} Now consider the \hyperref[def:multi_valued_function/restriction]{restriction} of \( \cos \) to the real line. Since \( \cos(0) \neq 0 \) and \( \cos \) is continuously differentiable as a power series, in some neighborhood \( U \) of \( 0 \) we have \( 0 \not\in \cos(U) \). Therefore, the inverse function theorem holds and there exists a neighborhood \( V \subseteq U \) of \( 1 \) such that the continuously differentiable function \( f: V \to \BbbR \) is the inverse of \( \cos \) in \( V \) (we have not yet defined \hyperref[def:inverse_trigonometric_functions/arccos]{\( \arccos \)}). If \( y = \cos(x) \), then
  \begin{equation*}
    Df(y)
    =
    \frac 1 {D\cos(x)}
    =
    \frac 1 {-\sin(x)}
    \reloset {\ref{thm:trigonometric_identities/pythagorean_identity}} =
    -\frac 1 {\sqrt{1 - y^2}},
    \quad y \in \cos(V).
  \end{equation*}

  The derivative is actually well-defined and continuous anywhere except for \( y \in \{ -1, 1 \} \). Therefore, for any \( \alpha \in (-1, 1) \),
  \begin{equation*}
    f(y) = f(\alpha) - \int_{\alpha}^y \frac 1 {\sqrt{1 - t^2}} dt, \quad y \in [\alpha, 1).
  \end{equation*}

  We already know that \( \cos(0) = 1 \), hence \( f(1) = 0 \) and, since \( f(y) \) is given by a convergent integral in \( [\alpha, 1) \), we can extend this interval to \( [\alpha, 1] \).

  By taking \( y = \alpha \), we obtain
  \begin{equation*}
    f(y) - f(-y) = -\int_{-y}^y \frac 1 {\sqrt{1 - t^2}} dt, \quad y \in [-1, 1].
  \end{equation*}

  Note that by \hyperref[def:pi]{our definition} of \( \pi \),
  \begin{equation*}
    \pi
    =
    \int_{-1}^1 \frac 1 {\sqrt{1 - t^2}} dt
    =
    -[\underbrace{f(1)}_{=0} - f(-1)]
    =
    f(-1).
  \end{equation*}

  Hence, \( \cos(\pi) = -1 \). From \fullref{thm:trigonometric_identities/pythagorean_identity},
  \begin{equation*}
    \abs{\sin(\pi)} = \sqrt{1 - \cos(\pi)^2} = 0,
  \end{equation*}
  proving that \( \sin(\pi) = 0 \).

  This concludes our proof of \eqref{eq:thm:trigonometric_function_basic_roots/pi}.

  \SubProofOf{eq:thm:trigonometric_function_basic_roots/half_pi} Note that
  \begin{equation*}
    0
    \reloset {\eqref{eq:thm:trigonometric_function_basic_roots/pi}} =
    \sin(\pi)
    =
    \sin(\ifrac \pi 2 + \ifrac \pi 2)
    \reloset {\eqref{eq:thm:trigonometric_identities/sum_of_angles/sin}} =
    \cos(\ifrac \pi 2)^2 \sin(\ifrac \pi 2)^2,
  \end{equation*}
  hence either \( \cos(\ifrac \pi 2) \) or \( \sin(\ifrac \pi 2) \) is zero.

  If \( \sin(\ifrac \pi 2) = 0 \), then \( \cos(\ifrac \pi 2) = 1 \). But
  \begin{equation*}
    -1
    \reloset {\eqref{eq:thm:trigonometric_function_basic_roots/pi}} =
    \cos(\pi)
    =
    \cos(\ifrac \pi 2 + \ifrac \pi 2)
    \reloset {\eqref{eq:thm:trigonometric_identities/sum_of_angles/cos}} =
    \cos(\ifrac \pi 2)^2 - \sin(\ifrac \pi 2)^2
    =
    1 - 0,
  \end{equation*}
  which is a contradiction.

  Hence, it follows that \( \cos(\ifrac \pi 2) = 0 \) and \( \sin(\ifrac \pi 2) = 1 \).

  \SubProofOf{eq:thm:trigonometric_function_basic_roots/sesqui_pi} We have
  \begin{equation*}
    \sin(\ifrac {3\pi} 2)
    =
    \sin(\pi + \ifrac \pi 2)
    \reloset {\eqref{eq:thm:trigonometric_identities/sum_of_angles/sin}} =
    \sin(\pi) \cos(\ifrac \pi 2) + \cos(\pi) \sin(\ifrac \pi 2)
    \reloset {\eqref{eq:thm:trigonometric_function_basic_roots/pi}} =
    -\sin(\ifrac \pi 2)
    \reloset {\eqref{eq:thm:trigonometric_function_basic_roots/half_pi}} =
    -1
  \end{equation*}
  and
  \begin{equation*}
    \cos(\ifrac {3\pi} 2)
    =
    \cos(\pi + \ifrac \pi 2)
    \reloset {\eqref{eq:thm:trigonometric_identities/sum_of_angles/cos}} =
    \cos(\pi) \cos(\ifrac \pi 2) - \sin(\pi) \sin(\ifrac \pi 2)
    \reloset {\eqref{eq:thm:trigonometric_function_basic_roots/pi}} =
    \cos(\ifrac \pi 2)
    =
    0.
  \end{equation*}
\end{proof}

\begin{definition}\label{def:periodic_function}
  A function \( f: G \to H \) between \hyperref[def:abelian_group] abelian groups is called \term{periodic} with \term{period} \( p \in G \) if, for all \( x \in G \), we have \( f(x) = f(x + r) \).

  The \term{base period} of a function is the \hyperref[def:extremal_points/maximum_and_minimum]{least} of all periods, if a minimum exists. When referring to \enquote{the period}, we mean the base period.

  We can define periods for arbitrary magmas rather than abelian groups, but the definition would make it difficult to talk about the base period.
\end{definition}

\begin{theorem}\label{thm:trigonometric_function_period}
  Both \( \sin(z) \) and \( \cos(z) \) are \( 2\pi \)-periodic.
\end{theorem}
\begin{proof}
  We will temporarily restrict ourselves to the real line. Since \( \cos(x) \) is continuous, \( \cos^{-1}(\{ 0 \}) \) is a closed set by \fullref{thm:weierstrass_extreme_value_theorem} there exists a minimum \( \gamma \) of \( [0, \pi] \cap \cos^{-1}(\{ 0 \}) \).

  Since, by \fullref{thm:trigonometric_function_basic_roots}, \( \cos(0) = 1 \), it follows that \( \cos(x) > 0, x \in (-\gamma, \gamma) \). Therefore, its primitive function \( \sin(x) \) increases on the same interval. It is also continuous, hence by \fullref{thm:trigonometric_identities/pythagorean_identity}, \( \sin(\gamma) = 1 \) because \( \cos(\gamma) = 0 \).

  Because \( \sin \) is an odd function, \( \sin(-\gamma) = \sin(\gamma) = -1 \).

  From \hyperref[def:pi]{our definition} of \( \pi \) it follows that
  \begin{equation*}
    \pi
    =
    \int_{-1}^1 \frac 1 {1 - t^2} dt
    =
    \int_{-\gamma}^\gamma \frac {\cos(\varphi)} {1 - \sin(\varphi)^2} d\varphi
    \reloset {\ref{thm:trigonometric_identities/pythagorean_identity}} =
    \int_{-\gamma}^\gamma d\varphi
    =
    2\gamma.
  \end{equation*}

  In order for a number \( p \) to be a period of \( \sin \), we need to have \( \sin(p) = \sin(0) = 0 \). But we showed that \( \sin(x) \) is increasing from \( 0 \) to \( \tfrac \pi 2 \) and cannot possibly contain zeros in that interval. Hence, \( p > \tfrac \pi 2 \).

  We also have \( \cos(\tfrac \pi 2) = 0 \). By \fullref{thm:trigonometric_identities/sums},
  \begin{equation*}
    \sin(\tfrac \pi 2 + x)
    =
    \sin(\tfrac \pi 2) \cos(x) + \cos(\tfrac \pi 2) \sin(x)
    =
    \cos(x).
  \end{equation*}

  Since \( \cos \) is positive on \( [0, \tfrac \pi 2) \), \( \sin \) is positive on \( [\tfrac \pi 2, \pi) \).

  We already showed in \fullref{thm:trigonometric_function_basic_roots} that \( \sin(\pi) = 0 \).

  It follows that the minimal period of \( \sin \) is either \( \pi \) or a multiple of \( \pi \). It cannot be \( \pi \) since \( \cos(\pi) \neq \cos(0) \), therefore it must be \( 2\pi \).
\end{proof}

\begin{proposition}\label{thm:trigonometric_function_period_identities}
  For any complex number \( z \), we have
  \begin{align}
    \sin(z \pm \ifrac \pi 2) &= \pm \cos(z) \label{eq:thm:trigonometric_function_period_identities/half/cos} \\
    \cos(z \pm \ifrac \pi 2) &= \pm \sin(z) \label{eq:thm:trigonometric_function_period_identities/half/sin}
  \end{align}
  and
  \begin{align}
    \sin(z \pm \pi) &= -\cos(z) \label{eq:thm:trigonometric_function_period_identities/full/cos} \\
    \cos(z \pm \pi) &= -\sin(z) \label{eq:thm:trigonometric_function_period_identities/full/sin}
  \end{align}
\end{proposition}
\begin{proof}
  For the first pair of identities, we have
  \begin{equation*}
    \sin(z \pm \ifrac \pi 2)
    \reloset {\eqref{eq:thm:trigonometric_identities/sum_of_angles/sin}} =
    \cos(z) \sin(\pm \ifrac \pi 2) + \sin(z) \cos(\pm \ifrac \pi 2)
    \reloset {\eqref{eq:thm:trigonometric_function_basic_roots/half_pi}} =
    \cos(z) \cdot (\pm 1) + \sin(z) \cdot 0
    =
    \pm \cos(z)
  \end{equation*}
  and
  \begin{equation*}
    \cos(z \pm \ifrac \pi 2)
    \reloset {\eqref{eq:thm:trigonometric_identities/sum_of_angles/sin}} =
    \cos(z) \cos(\pm \ifrac \pi 2) + \sin(z) \sin(\pm \ifrac \pi 2)
    \reloset {\eqref{eq:thm:trigonometric_function_basic_roots/half_pi}} =
    \cos(z) \cdot 0 + \sin(z) \cdot (\pm 1)
    =
    \pm \sin(z).
  \end{equation*}

  The second pair is analogous.
\end{proof}

\begin{definition}\label{def:derived_trigonometric_functions}
  In addition to \( \sin(z) \) and \( \cos(z) \), we define two additional functions, also called \enquote{trigonometric}.

  \begin{thmenum}
    \thmitem{def:derived_trigonometric_functions/tan} The \hyperref[def:partial_function]{partial} \term{tangent} function, also called \term{tangens}, is
    \begin{equation*}
      \tan(z) \coloneqq \frac {\sin(z)} {\cos(z)}.
    \end{equation*}

    It is defined in \( \BbbC \setminus (\tfrac \pi 2 + \pi\BbbZ) \).

    \thmitem{def:derived_trigonometric_functions/cot} The \hyperref[def:partial_function]{partial} \term{cotangent function}, also called \term{cotangens}, is
    \begin{equation*}
      \cot(z) \coloneqq \frac {\cos(z)} {\sin(z)}.
    \end{equation*}

    It is defined in \( \BbbC \setminus \pi\BbbZ \).
  \end{thmenum}
\end{definition}

\begin{definition}\label{def:inverse_trigonometric_functions}
  We can define \term{inverse trigonometric functions}. We will thus restrict ourselves only to real numbers. Fix an integer \( k \). Unless noted otherwise, we assume \( k = 0 \).

  \begin{thmenum}
    \thmitem{def:inverse_trigonometric_functions/arcsin} The \term{arcus sinus} function \( \arcsin(x) \) is defined as the \hyperref[def:multi_valued_function/inverse]{inverse function} of \( \sin(x) \) (see \fullref{def:trigonometric_functions/sine}) from \( [-1, 1] \) to \( \left[(k - \tfrac 1 2) \pi, (k + \tfrac 1 2) \pi \right) \).

    \thmitem{def:inverse_trigonometric_functions/arccos} The \term{arcus cosinus} function \( \arccos(x) \) is defined as the inverse of \( \cos(x) \) (see \fullref{def:trigonometric_functions/cosine}) from \( [-1, 1] \) to \( (k\pi, (k + 1)\pi) \).

    \thmitem{def:inverse_trigonometric_functions/arctan} The \term{arcus tangens} function \( \arctan(x) \) is defined as the inverse of \( \tan(x) \) (see \fullref{def:derived_trigonometric_functions/tan}) from \( \BbbR \) to \( \left((k - \tfrac 1 2) \pi, (k + \tfrac 1 2) \pi \right) \).

    \thmitem{def:inverse_trigonometric_functions/arccot} The \term{arcus cotangens} function \( \arccot(x) \) is defined as the inverse of \( \cot(x) \) (see \fullref{def:derived_trigonometric_functions/cot}) from \( \BbbR \) to \( (k\pi, (k + 1)\pi) \).
  \end{thmenum}
\end{definition}

\begin{proposition}\label{thm:def:inverse_trigonometric_function}
  The \hyperref[def:inverse_trigonometric_functions]{inverse trigonometric functions} have the following basic properties:

  \begin{thmenum}
    \thmitem{thm:def:inverse_trigonometric_function/sin_of_arctan} For any real number \( x \),
    \begin{equation}\label{eq:thm:def:inverse_trigonometric_function/sin_of_arctan}
      \sin(\arctan(x)) = \frac x {\sqrt{1 + x^2}}.
    \end{equation}

    \thmitem{thm:def:inverse_trigonometric_function/cos_of_arctan} For any real number \( x \),
    \begin{equation}\label{eq:thm:def:inverse_trigonometric_function/cos_of_arctan}
      \cos(\arctan(x)) = \frac 1 {\sqrt{1 + x^2}}.
    \end{equation}

    \thmitem{thm:def:inverse_trigonometric_function/sum_of_arccos}\mcite{MathSE:sum_of_angles} For any pair of real numbers \( x \) and \( y \) in \( [-1, 1] \),
    \begin{equation}\label{eq:thm:def:inverse_trigonometric_function/sum_of_arccos}
      \arccos(x) + \arccos(y) = \arccos\parens*{ xy - \sqrt{ (1 - x)^2 (1 - y)^2 } }.
    \end{equation}
  \end{thmenum}
\end{proposition}
\begin{proof}
  \SubProofOf{thm:def:inverse_trigonometric_function/sin_of_arctan} If \( x = 0 \), then \( \arctan x = 0 \) and \( \sin(\arctan x) = 0 \).

  Otherwise,
  \begin{equation*}
    \sin(\arctan x)^2
    =
    \tan(\arctan x)^2 \cdot \cos(\arctan x)^2
    =
    x^2 \cdot (1 - \sin(\arctan x)^2).
  \end{equation*}

  Then
  \begin{equation*}
    \frac 1 {x^2}
    =
    \frac {1 - \sin(\arctan x)^2} {\sin(\arctan x)^2}
    =
    \frac 1 {\sin(\arctan x)^2} - 1
  \end{equation*}
  and
  \begin{equation*}
    \frac 1 {\sin(\arctan x)^2} = \frac {x^2 + 1} {x^2}.
  \end{equation*}

  Therefore,
  \begin{equation*}
    \sin(\arctan x) = \frac x {\sqrt{x^2 + 1}}.
  \end{equation*}

  \SubProofOf{thm:def:inverse_trigonometric_function/cos_of_arctan} If \( x = 0 \), then \( \arctan x = 0 \) and \( \cos(\arctan x) = 1 \).

  Otherwise,
  \begin{equation*}
    \cos(\arctan x)^2
    =
    \frac {\sin(\arctan x)^2} {\tan(\arctan x)^2}
    =
    \frac {\sin(\arctan x)^2} {x^2}
    =
    \frac {1 - \cos(\arctan x)^2} {x^2}.
  \end{equation*}

  Then
  \begin{equation*}
    x^2
    =
    \frac {1 - \cos(\arctan x)^2} {\cos(\arctan x)^2}
    =
    \frac 1 {\cos(\arctan x)^2} - 1
  \end{equation*}
  and
  \begin{equation*}
    \frac 1 {\cos(\arctan x)^2} = x^2 + 1.
  \end{equation*}

  Therefore,
  \begin{equation*}
    \cos(\arctan x) = \frac 1 {\sqrt{x^2 + 1}}.
  \end{equation*}

  \SubProofOf{thm:def:inverse_trigonometric_function/sum_of_arccos} Let \( \alpha \coloneqq \arccos(x) \) and \( \beta \coloneqq \arccos(y) \). Then
  \begin{balign*}
    \arccos(x) + \arccos(y)
    &=
    \arccos\parens*{ \cos(\alpha + \beta) }
    = \\ &\reloset {\eqref{eq:thm:trigonometric_identities/sum_of_angles/cos}} =
    \arccos\parens*{ \cos \alpha \cos \beta + \sin \alpha \sin \beta }
    = \\ &=
    \arccos\parens*{ \cos \alpha \cos \beta + \sqrt{ (1 - \cos(\alpha)^2) (1 - \cos(\beta)^2) } }
    = \\ &=
    \arccos\parens*{ x y + \sqrt{ (1 - x^2) (1 - y^2) } }.
  \end{balign*}
\end{proof}

  \subsection{Exponential function}\label{subsec:exponential_function}

\begin{definition}\label{def:exponential_function}
  We define the \term{exponential function}
  \begin{equation}\label{def:exponential_function/series}
    \exp(z) \coloneqq \sum_{k=0}^\infty \frac {z^k} {k!}
  \end{equation}
  and \term{Euler's number}
  \begin{equation*}
    e \coloneqq \exp(1) = \sum_{i=0}^k \frac 1 {k!}.
  \end{equation*}

  \Fullref{thm:def:exponential_function/interpolates_power} justifies the notation \( e^z = \exp(z) \).
\end{definition}
\begin{proof}
  We will show that \( \exp(z) \) converges everywhere. By \fullref{thm:power_series_radius_of_convergence}, the radius of convergence is
  \begin{equation*}
    \limsup_{k \to \infty} \frac {k!} {(k-1)!}
    =
    \limsup_{k \to \infty} k
    =
    +\infty
  \end{equation*}

  Hence, the radius of convergence of \( \exp(x) \) is infinite.
\end{proof}

\begin{proposition}\label{thm:def:exponential_function}
  The exponential function \( \exp(z) \) has the following basic properties (not that we do not use the notation \( e^z \) here in order to reduce confusion with yet-undefined power \hyperref[def:power_function]{functions}):

  \begin{thmenum}
    \thmitem{thm:def:exponential_function/eulers_identity} (Euler's identity)
    \begin{equation*}
      \exp(i \pi) = -1.
    \end{equation*}

    \thmitem{thm:def:exponential_function/derivative} \( \exp(z) \) is its own derivative.

    \thmitem{thm:def:exponential_function/homomorphism} \( \exp(x + y) = \exp(x) \exp(y) \). Stated in another way, \( \exp \) is a homomorphism from the additive group of \( \BbbC \) to the multiplicative group.

    \thmitem{thm:def:exponential_function/interpolates_power} The notation \( \exp(x) \) is consistent with iterated multiplication as defined in \fullref{def:semiring/identity}, that is, \( \exp(n) = \underbrace{e \cdot \ldots \cdot e}_{n \text{times}} \) and for positive integers \( n \), \( \exp(n) =  \) and \( \exp(-n) =\tfrac 1 {\exp(n)} \).

    \thmitem{thm:def:exponential_function/negative_power}
    \begin{equation*}
      \exp(z) = \frac 1 {\exp(-z)}.
    \end{equation*}

    \thmitem{thm:def:exponential_function/real_positive} For real \( t \), \( e^t \) is a positive real number.

    \thmitem{thm:def:exponential_function/conjugate} \( \overline{\exp(z)} = \exp(\overline{z}) \).

    \thmitem{thm:def:exponential_function/unit_circle} For any \( c \in \BbbR \), the function \( t \mapsto \exp(it) \) is a bijection between any half-open interval \( [c, c + 2\pi) \) and the unit circle in \( \BbbC \).

    \thmitem{thm:def:exponential_function/real_bijective} \( t \mapsto \exp(t) \) is a bijection from \( \BbbR \) to \( [0, \infty) \).

    \thmitem{thm:def:exponential_function/bijective} For any \( c \in \BbbR \), \( \exp(z) \) is a bijection between the strip \( S \coloneqq \{ a + bi \colon c \leq b < c + 2\pi \} \) and the complex plane \( \BbbC \setminus \{ 0 \} \).

    \thmitem{thm:def:exponential_function/periodic} \( \exp(z) \) is \( 2i\pi \)-\hyperref[def:periodic_function]{periodic}.

    \thmitem{thm:def:exponential_function/compound_interest} For nonnegative real \( t \geq 0 \) we have
    \begin{equation*}
      \exp(t) = \lim_{n \to \infty} \left(1 + \frac t n \right)^n
    \end{equation*}
  \end{thmenum}
\end{proposition}
\begin{proof}
  \SubProofOf{thm:def:exponential_function/eulers_identity} By \fullref{eq:thm:trigonometric_function_basic_roots/pi} and \fullref{thm:exponential_trigonometric_identities/eulers_formula}, we have
  \begin{equation*}
    \exp(i\pi) = \cos(\pi) + i\sin(\pi) = -1.
  \end{equation*}

  \SubProofOf{thm:def:exponential_function/derivative} Follows from \fullref{thm:power_series_are_locally_uniform_convergent} and \fullref{thm:derivative_limit_exchange}.

  \SubProofOf{thm:def:exponential_function/homomorphism} The Cauchy product of \( \exp(x) \) and \( \exp(y) \) is
  \begin{balign*}
    \exp(x) \exp(y)
     & =
    \left( \sum_{k=0}^\infty \frac {x^k} {k!} \right) \left( \sum_{k=0}^\infty \frac {y^k} {k!} \right)
    =                                       \\ &=
    \sum_{k=0}^\infty \sum_{m=0}^k \frac {x^m} {m!} \frac {x^{k-m}} {(k-m)!}
    =                                       \\ &=
    \sum_{k=0}^\infty \frac 1 {k!} \sum_{m=0}^k \binom{k}{m} x^m y^{k-m}
    \reloset {\ref{thm:binomial_theorem}} = \\ &=
    \sum_{k=0}^\infty \frac {(x + y)^k} {k!}
    =
    \exp(x + y).
  \end{balign*}

  \SubProofOf{thm:def:exponential_function/interpolates_power} We use induction on \( n \) to prove \( \exp(n) = e^n \). The case \( \exp(0) = 1 \) is obvious. If we assume that \( \exp(n) = e^n \), by \fullref{thm:def:exponential_function/homomorphism}, we have
  \begin{equation*}
    \exp(n + 1)
    =
    \exp(n) \exp(1)
    =
    e^n \cdot e
    =
    e^{n+1}.
  \end{equation*}

  Note that this works for negative \( n \) too.

  \SubProofOf{thm:def:exponential_function/negative_power} Note that
  \begin{equation*}
    1 = \exp(0) = \exp(z - z) = \exp(z) \exp(-z),
  \end{equation*}
  hence
  \begin{equation*}
    \exp(-z) = \frac 1 {\exp(z)}.
  \end{equation*}

  \SubProofOf{thm:def:exponential_function/real_positive} For \( t > 0 \), the following
  \begin{equation*}
    \exp(t) = \sum_{k=0}^\infty \frac {t^k} {k!}
  \end{equation*}
  is a series of positive real numbers. To see its convergence, we apply \fullref{thm:dalamberts_ratio_test}:
  \begin{equation*}
    \frac {t^k} {k!} \cdot \frac {(k-1)!} {t^{k-1}}
    =
    \frac t k
    \xrightarrow[k \to \infty]{} 0.
  \end{equation*}

  Thus, \( \exp(t) \) is a nonnegative real number. Furthermore, since the sequence of partial sums is monotone, \( \exp(t) \) cannot be zero. Hence, for \( t > 0 \), we have \( \exp(t) > 0 \).

  Notice that \( \exp(t) \exp(-t) > 0 \), hence if \( \exp(t) > 0 \), then \( \exp(-t) > 0 \).

  \SubProofOf{thm:def:exponential_function/conjugate} By \fullref{thm:exponential_trigonometric_identities/eulers_formula},
  \begin{balign*}
    \overline{\exp(a + bi)}
     & \reloset {\ref{thm:def:exponential_function/homomorphism}} =
    \overline{\exp(a) \exp(bi)}
    \reloset {\ref{thm:def:exponential_function/real_positive}} =   \\ &=
    \exp(a) \overline{(\cos(b) + i\sin(b))}
    =                                                                      \\ &=
    \exp(a) (\cos(b) - i\sin(b))
    \reloset {\ref{thm:power_series_parity}} =                             \\ &=
    \exp(a) (\cos(-b) + i\sin(-b))
    =                                                                      \\ &=
    \exp(a) \exp(-bi)
    =                                                                      \\ &=
    \exp(a - bi)
    =                                                                      \\ &=
    \exp(\overline{a + bi}).
  \end{balign*}

  \SubProofOf{thm:def:exponential_function/periodic} By \fullref{thm:def:exponential_function/eulers_identity},
  \begin{equation*}
    \exp(x + 2i\pi) = \exp(x) \exp(2i\pi) = \exp(x).
  \end{equation*}

  Furthermore, this is also the minimal period. If we assume that \( \sin(x) \) has another period, say \( p \in (0, 2\pi) \), we would have \( \sin(p) = \sin(0) = 0 \) and \fullref{thm:trigonometric_identities/pythagorean_identity} would imply that \( \cos(p) \in \{ -1, 1 \} \). But then \( \cos(p) \) would be an extreme point for \( \cos \), which is not possible because \( \cos \) is convex in \( [0, 2\pi] \) and only has three extremal points --- \( 0, \pi, 2\pi \).

  \SubProofOf{thm:def:exponential_function/unit_circle} For \( c, t \in \BbbR \) we have
  \begin{equation*}
    \abs{\exp(it)}
    =
    \abs{\cos(t) + i\sin(t)}
    =
    \sqrt{\cos(t)^2 + \sin(t)^2}
    \reloset {\eqref{eq:thm:trigonometric_identities/pythagorean_identity}} =
    1.
  \end{equation*}

  Furthermore, if \( r \) is another real number,
  \begin{equation}
    \exp(ir)
    =
    \exp(i(t + (r - t)))
    =
    \exp(it) \exp(i(r - t)).
  \end{equation}

  It follows that \( \exp(ir) \neq \exp(it) \) if and only if \( \exp(i(r - t)) \neq 0 \). If \( t, r \in [c, c + 2\pi) \) and \( t \neq r \), this is satisfied.

  Hence, \( t \mapsto \exp(it) \) is indeed an injection of \( [c, c + 2\pi) \) into the unit circle of \( \BbbC \). It is also a surjection because of the intermediate value theorem.

  \SubProofOf{thm:def:exponential_function/real_bijective} First, assume that \( e^t \) is not injective on \( \BbbR \). Then there exist \( t, r \in \BbbR \), \( t \neq r \), such that \( e^t = e^r \). By \fullref{thm:def:exponential_function/real_positive}, both are positive real numbers. In particular, we can divide by \( e^t \) to obtain
  \begin{equation*}
    1
    =
    \frac {e^r} {e^t}
    \reloset {\ref{thm:def:exponential_function/negative_power}} =
    =
    e^r e^{-t}
    \reloset {\ref{thm:def:exponential_function/homomorphism}} =
    e^{r - t}.
  \end{equation*}

  We know that \( e^0 = 1 \) from \fullref{thm:def:exponential_function/interpolates_power}. Thus, it is enough to show that \( e^t = 1 \) if and only if \( t = 0 \).

  Assume that \( e^t = 1 \) holds for some \( t > 0 \). The partial sums are monotonely increasing, so in order for them to converge to \( 1 \), for any fixed index \( n \) we must have
  \begin{balign*}
    0  & \leq \sum_{k=0}^n \frac {t^k} {k!} = 1 + \sum_{k=1}^n \frac {t^k} {k!} \leq 1, \\
    -1 & \leq \sum_{k=1}^n \frac {t^k} {k!} \leq 0.
  \end{balign*}

  But \( \sum_{k=1}^n \frac {t^k} {k!} > 0 \) because \( t > 0 \). The obtained contradiction proves that \( e^t \neq 1 \) for positive \( t \).

  For negative \( t \), note that
  \begin{equation*}
    e^t e^{-t} = 1.
  \end{equation*}

  Since \( -t \) is positive, \( e^{-t} \neq 1 \) and hence \( e^t \neq 1 \).

  Therefore, the function \( t \mapsto e^t \) is injective on \( \BbbR \). It is also surjective onto \( \BbbR^{>0} \) because of the intermediate value theorem.

  \SubProofOf{thm:def:exponential_function/bijective} Fix \( a + bi \in S_c \), that is, \( b \in [c, c + 2\pi) \). By \fullref{thm:def:exponential_function/homomorphism},
  \begin{equation*}
    e^{a + bi} = e^a e^{bi}.
  \end{equation*}

  By \fullref{thm:def:exponential_function/unit_circle}, \( b \mapsto e^{bi} \) is injective for \( b \in [c, c + 2\pi) \) and by \fullref{thm:def:exponential_function/real_bijective}, \( a \mapsto e^a \) is injective on \( \BbbR \). It follows that their product is also injective.

  \SubProofOf{thm:def:exponential_function/compound_interest}\mcite[3.31]{Rudin1976Principles}By \fullref{thm:binomial_theorem},
  \begin{balign*}
    \left(1 + \frac t n \right)^n
     & =
    \sum_{k=0}^n \binom{n}{k} \left(\frac t n\right)^k 1^{n-k}
    =    \\ &=
    \sum_{k=0}^n \frac {n!} {(n-k)! k!} \frac {t^k} {n^k}
    =    \\ &=
    \sum_{k=0}^n \frac {n!} {(n-k)! n^k} \frac {t^k} {k!}
    =    \\ &=
    \sum_{k=0}^n \left[ \prod_{j=1}^k \left(1 - \frac {k+j} n \right) \right] \frac {t^k} {k!}.
  \end{balign*}

  Fix an index \( m \). Since the series is nonnegative, there exists an index \( N \) such that for \( n \geq N \)
  \begin{equation*}
    \sum_{k=0}^m \frac {t^k} {k!}
    \leq
    \sum_{k=0}^n \left[ \prod_{j=1}^k \left(1 - \frac {k+j} n \right) \right] \frac {t^k} {k!}.
  \end{equation*}

  Note that
  \begin{equation*}
    \left[ \prod_{j=1}^k \left(1 - \frac {k+j} n \right) \right] \frac {t^k} {k!}
    \leq
    \frac {t^k} {k!},
  \end{equation*}
  hence
  \begin{equation*}
    \sum_{k=0}^m \frac {t^k} {k!}
    \leq
    \sum_{k=0}^n \left[ \prod_{j=1}^k \left(1 - \frac {k+j} n \right) \right] \frac {t^k} {k!}
    \leq
    \sum_{k=0}^n \frac {t^k} {k!}.
  \end{equation*}

  By \fullref{thm:squeeze_lemma},
  \begin{equation*}
    \lim_{n \to \infty} \left(1 + \frac t n \right)^n
    =
    \lim_{n \to \infty} \sum_{k=0}^n \frac {t^k} {k!}
    =
    \exp(t).
  \end{equation*}
\end{proof}

\begin{definition}\label{def:logarithm}
  Fix \( c \in \BbbR \). Unless specified otherwise, we assume \( c = 0 \).

  We define the \term{natural logarithm} \( \log(x) \) as \hyperref[def:multi_valued_function/inverse]{inverse function} of \( e^x \) from \( \BbbC \setminus \{ 0 \} \) to the strip \( S_c \coloneqq \{ a + bi \colon c \leq b < c + 2\pi \} \).

  We also define the \term{base \( b \) logarithm} \( \log_b(x) \) for \( b > 0 \) over the same domain as
  \begin{equation*}
    \log_b(x) \coloneqq \frac {\log(x)} {\log(b)}.
  \end{equation*}
\end{definition}
\begin{proof}
  The well-definedness follows from \fullref{thm:def:exponential_function/bijective}.
\end{proof}

\begin{proposition}\label{thm:def:logarithm}
  \hfill
  \begin{thmenum}
    \thmitem{thm:def:logarithm/homomorphism} \( \log(xy) = \log(x) \log(y) \)
  \end{thmenum}
\end{proposition}
\begin{proof}
  \SubProofOf{thm:def:logarithm/homomorphism} Follows from \fullref{thm:def:exponential_function/homomorphism}.
\end{proof}

\begin{definition}\label{def:power_function}
  For each positive real number \( y > 0 \), we define the \term{power function}
  \begin{equation*}
    x^y \coloneqq e^{y \ln x}
  \end{equation*}
  as a function of \( x \).
\end{definition}

\begin{proposition}\label{thm:def:power_function}
  \hfill
  \begin{thmenum}
    \thmitem{thm:def:power_function/composition} \( (x^y)^z = x^{yz} \).
    \thmitem{thm:def:power_function/derivative} \( D_x(x^y) = \log(x) x^y \).
  \end{thmenum}
\end{proposition}
\begin{proof}
  \SubProofOf{thm:def:power_function/composition}
  \begin{equation*}
    (x^y)^z
    =
    e^{z \log(e^{y \log(x)})}
    =
    e^{z y \log(x)}
    =
    x^{yz}.
  \end{equation*}

  \SubProofOf{thm:def:power_function/derivative} Using the chain rule for differentiation, we obtain
  \begin{equation*}
    D_x(x^y) = D_x(e^{\log(y) x}) = \log(x) e^{\log(y) x} = \log(x) x^y.
  \end{equation*}
\end{proof}
\begin{proposition}\label{thm:exponential-trigonometric_identities}
  We have the following exponential-trigonometric identities:
  \thmitem{thm:exponential_trigonometric_identities/eulers_formula} (Euler's formula) For any \( z \in \BbbC \),
  \begin{equation}\label{thm:exponential_trigonometric_identities/eulers_formula/identity}
    e^{iz} = \cos(z) + i \sin(z).
  \end{equation}

  \thmitem{thm:exponential_trigonometric_identities/inverse_eulers_formula} (Inverse Euler's identities) For any \( z \in \BbbC \),
  \begin{balign}
    \sin(z) & = \real(e^z) = \frac {e^{iz} - e^{-iz}} {2i} \label{thm:exponential_trigonometric_identities/inverse_eulers_formula/sin} \\
    \cos(z) & = \imag(e^z) = \frac {e^{iz} + e^{-iz}} 2 \label{thm:exponential_trigonometric_identities/inverse_eulers_formula/cos}
  \end{balign}

  \thmitem{thm:exponential_trigonometric_identities/de_moivre} (De Moivre's formula) For any complex number \( z \) and any nonnegative integer \( n \),
  \begin{equation}\label{thm:exponential_trigonometric_identities/de_moivre/identity}
    (\cos(z) + i \sin(z))^n = \cos(nz) + i \sin(nz).
  \end{equation}
\end{proposition}
\begin{proof}
  \SubProofOf{thm:exponential_trigonometric_identities/eulers_formula} Simply note that \fullref{def:exponential_function} is a termwise sum of \fullref{def:trigonometric_functions/sine} and \fullref{def:trigonometric_functions/cosine}, therefore \fullref{thm:exponential_trigonometric_identities/eulers_formula/identity} holds.

  \SubProofOf{thm:exponential_trigonometric_identities/inverse_eulers_formula} Follows from \fullref{thm:exponential_trigonometric_identities/eulers_formula}.

  \SubProofOf{thm:exponential_trigonometric_identities/de_moivre} From \fullref{thm:exponential_trigonometric_identities/eulers_formula},
  \begin{equation*}
    (\cos(z) + i \sin(z))^n
    =
    {e^{iz}}^n
    \reloset {\ref{thm:def:power_function/composition}} {=}
    =
    e^{i(zn)}
    =
    \cos(nz) + i \sin(nz).
  \end{equation*}
\end{proof}


\begin{definition}\label{def:hyperbolic_trigonometric_functions}
  In analogy with \fullref{thm:exponential_trigonometric_identities/inverse_eulers_formula}, we define \term{hyperbolic trigonometric functions}.

  \begin{thmenum}
    \thmitem{def:hyperbolic_trigonometric_functions/sine} The \term{hyperbolic sine} function:
    \begin{equation*}
      \sinh(x) \coloneqq \frac {e^x - e^{-x}} 2
    \end{equation*}

    \thmitem{def:hyperbolic_trigonometric_functions/cosine} The \term{hyperbolic cosine} function:
    \begin{equation*}
      \cosh(x) \coloneqq \frac {e^x + e^{-x}} 2
    \end{equation*}
  \end{thmenum}
\end{definition}
\begin{thmcomment}
  Unlike \( \sin \) and \( \cos \), which are used in the standard parametric equation \eqref{eq:def:ellipse/parametric_equation} of an \hyperref[def:ellipse]{ellipse}, \( \sinh \) and \( \cosh \) are used in the standard parametric equation \eqref{eq:def:hyperbola/parametric_equation} of a \hyperref[def:hyperbola]{hyperbola}.
\end{thmcomment}

\begin{proposition}\label{thm:hyperbolic_identities}
  We have the following basic hyperbolic identities:
  \begin{thmenum}
    \thmitem{thm:hyperbolic_identities/pythagorean_identity} For any \( z \in \BbbC \),
    \begin{equation}\label{eq:thm:hyperbolic_identities/pythagorean_identity}
      \sinh(z)^2 - \cosh(z)^2 = -1.
    \end{equation}

    \thmitem{thm:hyperbolic_identities/inverse} For any \( z \in \BbbC \),
    \begin{align}
      &\sinh^{-1}(z) = \ln(z + \sqrt{ z^2 + 1 }), \label{eq:thm:hyperbolic_identities/inverse/sinh} \\
      &\cosh^{-1}(z) = \ln(z + \sqrt{ z^2 - 1 }). \label{eq:thm:hyperbolic_identities/inverse/cosh}
    \end{align}
  \end{thmenum}
\end{proposition}
\begin{proof}
  \SubProofOf{thm:hyperbolic_identities/pythagorean_identity}
  \begin{equation*}
    \sinh(z)^2 - \cosh(z)^2
    =
    \frac {e^{2z} - 2e^z e^{-z} + e^{-2z} - e^{2z} - 2e^z e^{-z} - e^{-2z}} 4
    =
    -1
  \end{equation*}

  \SubProofOf{thm:hyperbolic_identities/inverse} Suppose that
  \begin{equation*}
    z = \sinh(x) = \frac {e^x - e^{-x}} 2.
  \end{equation*}

  Then
  \begin{equation*}
    e^{2x} - 2z e^x - 1 = 0.
  \end{equation*}

  It follows that
  \begin{equation*}
    e^x = \frac {2z \pm \sqrt{ 4z^2 - 4 }} 2 = z \pm \sqrt{ z^2 - 1 }.
  \end{equation*}

  Then
  \begin{equation*}
    \sinh^{-1}(z) = \ln\parens*{ z + \sqrt{ z^2 - 1 } }.
  \end{equation*}

  The proof for \( \cosh \) is similar.
\end{proof}

  \subsection{Trigonometric polynomials}\label{subsec:trigonometric_polynomials}

\begin{definition}\label{def:trigonometric_polynomial}
  We define the \term{trigonometric polynomials} over \( \BbbC \) as the \hyperref[def:ring_of_laurent_polynomials]{Laurent polynomials} \( \BbbC[e^{iz}] \). A trigonometric polynomial \( p \in \BbbC[e^{iz}] \) can be written as
  \begin{equation}\label{def:trigonometric_polynomial/exponential}
    p(z) = \sum_{k \in \BbbZ} c_k e^{ikz}
  \end{equation}
  or, using \hyperref[thm:exponential_trigonometric_identities/eulers_formula]{Euler's formula}, rewritten in the more conventional notation (see \cite[1]{Боянов2008} or \cite[88]{Rudin1986RealAndComplex}):
  \begin{equation}\label{def:trigonometric_polynomial/trigonometric}
    p(z) = a_0 + \sum_{k=1}^\infty [ a_k \cos(kz) + b_k \sin(kz) ],
  \end{equation}
  where we denote \( a_k \coloneqq c_k \) and \( b_k \coloneqq ic_k \).

  In particular, when using \fullref{def:trigonometric_polynomial/trigonometric}, we may regard the coefficients \( \{ a_k \}_{k=0}^\infty \) and \( \{ b_k \}_{k=1}^\infty \) as either real or complex, which is a downside of \fullref{def:trigonometric_polynomial/exponential}.

  Denote by \( \tau_n(\BbbK) \) the vector space of all trigonometric polynomials of degree at most \( n \) with coefficients in \( \BbbK \). We also introduce the subspaces \( \tau_n^\alpha{\BbbK} \) of those polynomials which \( a_0 = 0 \).
\end{definition}

  \subsection{Special functions}\label{subsec:special_functions}

\begin{definition}\label{def:gamma_function}
  The \term[bg=гама (функция) (\cite[280]{Тагамлицки1978Инт}), ru=гамма функция (\cite[406]{Зорич2019Том2})]{Gamma function} \( \Gamma: \BbbR_{>0} \to \BbbR \) can be defined via the following equivalent definitions:
  \begin{thmenum}
    \thmitem{def:gamma_function/direct}\mcite[\S 8.17]{Rudin1976Principles}
    \begin{equation}\label{eq:def:gamma_function/direct}
      \Gamma(x) \coloneqq \int_0^\infty t^{x-1} e^{-t} \dl t
    \end{equation}

    \thmitem{def:gamma_function/characterization}\mcite[\S 8.19]{Rudin1976Principles} \( \Gamma \) is the unique function satisfying the following conditions:
    \begin{thmenum}
      \thmitem{def:gamma_function/characterization/recurrent} \( \Gamma(x + 1) = x \Gamma(x) \).
      \thmitem{def:gamma_function/characterization/starting} \( \Gamma(1) = 1 \).
      \thmitem{def:gamma_function/characterization/log_convex} \( \log \Gamma \) is \hyperref[def:convex_functions]{convex}.
    \end{thmenum}
  \end{thmenum}
\end{definition}

\begin{theorem}[Stirling's gamma approximation]\label{thm:stirlings_gamma_approximation}\mcite[24]{Райков2009Артин}
  For the \( \Gamma \) function we have
  \begin{equation}\label{eq:thm:stirlings_gamma_approximation}
    \Gamma(x) = \sqrt{2 \pi} \cdot x^{x - 1 / 2} \cdot e^{-x + \mu(x)},
  \end{equation}
  where
  \begin{equation}\label{eq:thm:stirlings_gamma_approximation/mu}
    \mu(x) \coloneqq \sum_{k=0}^\infty \parens[\Big]{ \parens[\Big]{ x + k + \frac 1 2 } \ln\parens[\Big]{ 1 + \frac 1 {x + k} } - 1 }.
  \end{equation}

  Furthermore,
  \begin{equation}\label{eq:thm:stirlings_gamma_approximation/mu_inequality}
    0 < \mu(x) < \frac 1 {12x}.
  \end{equation}
\end{theorem}
\begin{comments}
  \item This approximation can be found as \identifier{numeric.gamma.stirling} in \cite{code}.
\end{comments}

  \section{Norms}\label{sec:norms}

\paragraph{The triangle inequality}

\begin{definition}\label{def:triangle_inequality}\mimprovised
  For a binary real-valued function \( d: X^2 \to G \) from an arbitrary set \( X \) to an \hyperref[def:ordered_semigroup]{ordered} \hyperref[con:additive_semigroup]{additive (semi)groups}, we will refer to the \hyperref[def:inequality]{inequality}
  \begin{equation}\label{eq:def:triangle_inequality}
    d(x, z) \leq d(x, y) + d(y, z).
  \end{equation}
  as the \term[ru=неравенство треугольника (\cite[436]{Натансон1974ВещественныйАнализ}), en=triangle inequality (\cite[\S 1.4]{Rudin1987RealAndComplexAnalysis})]{triangle inequality}.
\end{definition}
\begin{comments}
  \item Due to the relation to \hyperref[def:additive_function/sub]{subadditive functions} shown in \fullref{thm:subadditivity_to_triangle_inequality}, the subadditivity inequality is often called \enquote{the triangle inequality} in relation to norms. This is discussed in \fullref{rem:triangle_inequality_terminology}.
\end{comments}

\begin{remark}\label{rem:triangle_inequality_terminology}
  \enquote{The triangle inequality} is generally used to refer to \eqref{eq:def:triangle_inequality}, however, due the relation to \hyperref[def:additive_function/sub]{subadditive functions} shown in \fullref{thm:subadditivity_to_triangle_inequality} and \fullref{thm:triangle_inequality_to_subadditivity}, it is sometimes used to refer to \eqref{eq:def:additive_function/sub}. As can be seen from our breakdown, this refers exclusively for subadditivity of norms, where the distinction is blurred.

  \begin{itemize}
    \item Authors using the term for both norms and metrics include
    \incite[4]{Yoshida1980FunctionalAnalysis} and \cite[30]{Yoshida1980FunctionalAnalysis},
    \incite[3]{WheedenZygmund2015MeasureAndIntegral} and \cite[191]{WheedenZygmund2015MeasureAndIntegral},
    \incite[4.3.3]{Tao2022AnalysisI} and
    \incite[thm. 9.2.4]{Roman2005AdvancedLinearAlgebra} and \cite[thm. 9.4.3]{Roman2005AdvancedLinearAlgebra}.
    \incite[94]{Rotman2015AdvancedModernAlgebraPart1} and \cite[673]{Rotman2015AdvancedModernAlgebraPart1}.

    \item Authors using the term when defining metrics include
    \incite[217]{Birkhoff1967LatticeTheory},
    \incite[48]{Carothers2000RealAnalysis},
    \incite[118]{Kelley1975GeneralTopology},
    \incite[248]{Engelking1989GeneralTopology},
    \incite[40]{Schechter1997AnalysisHandbook},
    \incite[24]{Edwards1965FunctionalAnalysis},
    \incite[55]{HalmosGivant2009BooleanAlgebras},
    \incite[436]{Натансон1974ВещественныйАнализ},
    \incite[15]{ЛюстерникСоболев1965ФункциональныйАнализ} as \enquote{аксиома треугольника} (triangle axiom).

    \item Authors using the term for what we call the triangle inequality for metrics in other contexts include
    \incite[161]{Арнольд2014ОДУ},
    \incite[53]{Боровков1999ТеорияВероятностей},
    \incite[13]{Зорич2019АнализЧасть2},
    \incite[\S 1.4]{Rudin1987RealAndComplexAnalysis},
    \incite[362]{Grätzer2011LatticeTheory} and
    \incite[\S 9.1.1.1]{Berger1987GeometryI}.

    \item Authors using the term when defining norms include
    \incite[16]{Clarke2013OptimalControl},
    \incite[12]{FabianEtAl2001FunctionalAnalysis} and
    \incite[151]{Folland1999RealAnalysis}.

    \item Authors using the term for norms in other contexts include
    \incite[122]{Treil2017LinearAlgebraDoneWrong},
    \incite[7]{BerghLöfström1976InterpolationSpaces},
    \incite[9]{Ahlfors1979ComplexAnalysis},
    \incite[21]{Malliavin1995IntegrationAndProbability},
    \incite[578]{Lang2002Algebra},
    \incite[13]{Strang2023LinearAlgebra},
    \incite[def. 1.1.1]{Хелемский2014ФункциональныйАнализ},
    \incite[112]{Deimling1985NonlinearFA} and
    \incite[605]{Knapp2016BasicAlgebra}.
  \end{itemize}
\end{remark}

\begin{definition}\label{def:translation_invariant_binary_function}\mimprovised
   We say that a binary function between \hyperref[def:ordered_semigroup]{ordered} \hyperref[con:additive_semigroup]{additive (semi)groups} is \term{translation-invariant} if
  \begin{equation*}
    f(x, y) = f(x + d, y + d).
  \end{equation*}
  for any \( d \).
\end{definition}
\begin{comments}
  \item We generalize the definition of a translation-invariant metric from \cite[88]{Deimling1985NonlinearFA}.
\end{comments}

\begin{proposition}\label{thm:subadditivity_to_triangle_inequality}
  Fix two \hyperref[con:additive_semigroup]{additive (semi)groups} \( G \) and \( H \), and suppose that \( H \) is \hyperref[def:ordered_semigroup]{ordered}.

  Let \( f: G \to H \) be any function. Define
  \begin{equation*}
    \begin{aligned}
      &d: G^2 \to H, \\
      &d(x, y) \coloneqq f(y - x).
    \end{aligned}
  \end{equation*}

  Then \( d \) is \hyperref[def:translation_invariant_binary_function]{translation-invariant}. Furthermore, if \( f \) is \hyperref[def:additive_function/sub]{subadditive}, then \( d \) satisfies the \hyperref[def:triangle_inequality]{triangle inequality}.
\end{proposition}
\begin{proof}
  That \( d \) is translation-invariant follows from
  \begin{equation*}
    d(x + d, y + d) = f(y + d - x - d) = f(y - x) = d(x, y).
  \end{equation*}

  Furthermore, if \( f \) is subadditive, then
  \begin{equation*}
    d(x, z) = f(z - x) = f(z - y + y - x) \leq f(z - y) + f(y - x) = d(y, z) + d(x, y).
  \end{equation*}
\end{proof}

\begin{proposition}\label{thm:triangle_inequality_to_subadditivity}
  As in \fullref{thm:subadditivity_to_triangle_inequality}, fix two \hyperref[con:additive_semigroup]{additive (semi)groups} \( G \) and \( H \), and suppose that \( H \) is \hyperref[def:ordered_semigroup]{ordered}.

  Let \( d: G^2 \to H \) be a \hyperref[def:translation_invariant_binary_function]{translation-invariant} function. Define
  \begin{equation*}
    \begin{aligned}
      &f: G \to \BbbR, \\
      &f(x) \coloneqq d(0, x).
    \end{aligned}
  \end{equation*}

  If \( d \) satisfies the \hyperref[def:triangle_inequality]{triangle inequality}, then \( f \) is \hyperref[def:additive_function/sub]{subadditive}.
\end{proposition}
\begin{proof}
  We have
  \begin{equation*}
    f(x + y) = d(0, x + y) = d(-x, y) \leq \underbrace{d(-x, 0)}_{d(0, x)} + d(0, y) = f(x) + f(y).
  \end{equation*}
\end{proof}

\paragraph{Norms}

\begin{remark}\label{rem:normed_fields}
  Norms generalize distances of points in a plane, while absolute values generalize the absolute value over either \( \BbbR \) or \( \BbbC \). The axioms themselves differ minimally. Absolute values in a field are multiplicative norms over the field, however we cannot define absolute values in terms of norms since absolute values are needed for defining norms. Still, we will refer to fields with absolute values as \term{normed fields}.
\end{remark}

\begin{definition}\label{def:norm}
  Let \( M \) be an \( R \)-module with absolute value \( \abs{\cdot} \). We say that the function \( \norm{\cdot}: M \to \BbbR_{\geq 0} \) is a \term{norm} if
  \begin{thmenum}
    \thmitem[def:norm/N1]{N1}(identity) \( x = 0_M \) if and only if \( \norm x = 0_{\BbbR} \)

    \thmitem[def:norm/N2]{N2}(absolute homogeneity)
    \begin{equation*}
      \norm{t x} = \abs{t} \norm{x} \text{ for all } t \in R \text{ and } x \in M
    \end{equation*}

    \thmitem[def:norm/N3]{N3}(subadditivity)
    \begin{equation*}
      \norm{x + y} \leq \norm{x} + \norm{y} \text{ for all } x, y \in M
    \end{equation*}
  \end{thmenum}

  If we remove \fullref{def:norm/N1}, then \( \norm{\cdot} \) is called a \term{seminorm}.

  If instead \( V \) is an \hyperref[def:algebra_over_semiring]{associative} and \( \norm{\cdot} \) satisfies the additional axiom
  \begin{thmenum}
    \thmitem{def:norm/multiplicativity}(multiplicativity)
    \begin{equation*}
      \norm{xy} = \norm{x} \cdot \norm{y} \text{ for all } x, y \in M,
    \end{equation*}
  \end{thmenum}
  we say that it is a \term{multiplicative norm}.
\end{definition}

\begin{definition}\label{def:normed_vector}
  \todo{Normed vectors}
\end{definition}

\begin{definition}\label{def:norm_induced_metric}
  A norm \( \norm \cdot \) on a real or complex vector space \( V \) induces the \hyperref[def:vector_space]{metric}
  \begin{balign*}
     & \rho: V \times V \to \BbbR_{\geq 0}  \\
     & \rho(x, y) \coloneqq \norm{x - y}.
  \end{balign*}
\end{definition}
\begin{proof}
  The function is positive definite since \( \norm \cdot \) is positive definite; we will show that the function is a metric.

  \SubProofOf{def:metric_space/M1} Follows from \fullref{def:norm/N1}.

  \SubProofOf{def:metric_space/M2} By \fullref{def:norm/N2},
  \begin{equation*}
    \rho(x, y) = \norm{ x - y } = \norm{ (-1) (y - x) } = \abs{-1} \norm{y - x} = \rho(y, x).
  \end{equation*}

  \SubProofOf{def:metric_space/M3}
  \begin{equation*}
    \rho(x, y) + \rho(y, z) = \norm{x - y} + \norm{y - z} \geq \norm{x - z} = \rho(x, z).
  \end{equation*}
\end{proof}

\begin{definition}\label{def:duality_mapping}\mcite[example 2.26]{Phelps1993ConvexDifferentiability}
  We define the \term{duality mapping}
  \begin{balign*}
     & D: E \multto X^*,                                                                                              \\
     & D(x) \coloneqq \{ x^* \in X^* \colon \norm x = \norm {x^*} \text{ and } \inprod{x^*} x = \norm {x^*} \norm x \}.
  \end{balign*}

  We will usually use this mapping for unit vectors, so we may as well consider its restriction to the unit spheres, where
  \begin{balign*}
     & D': S_X \multto S_{X^*},                                       \\
     & D'(x) \coloneqq \{ x^* \in S_{X^*} \colon \inprod{x^*} x = 1 \}.
  \end{balign*}
\end{definition}

\begin{definition}\label{def:smooth_norm}\mcite[def. 2.36]{Phelps1993ConvexDifferentiability}
  The norm \( \norm \cdot \) on \( X \) is called \term{smooth} if any of  if for each \( x \in S_X \) the duality mapping is single-valued.
\end{definition}

\begin{definition}\label{def:rotund_norm}\mcite[def. 2.36]{Phelps1993ConvexDifferentiability}
  The norm \( \norm \cdot \) on \( X \) is called \term{rotund} or \term{strictly convex} if any of the following equivalent conditions hold:
  \begin{thmenum}
    \thmitem{def:rotund_norm/no_sphere_segments} There are no line segments in the unit sphere \( S_X \).
    \thmitem{def:rotund_norm/least_norm} Every convex subset of \( X \) has at most one point of least norm.
    \thmitem{def:rotund_norm/linearly_dependent}
    \begin{balign}\label{def:rotund_norm/linearly_dependent/equation}
      \norm{x + y} = \norm x + \norm y \implies x \text{ and } y \text{ are linearly dependent}.
    \end{balign}
  \end{thmenum}
\end{definition}
\begin{proof}
  \ImplicationSubProof{def:rotund_norm/no_sphere_segments}{def:rotund_norm/least_norm} Let the norm in \( E \) be rotund and let \( C \subseteq E \) be a (potentially empty) convex set. We will prove that \( C \) contains at most one point of least norm.

  If \( C \) is empty or otherwise contains no element of least norm, trivially contains at most one point of least norm.

  Now let \( C \) contain at least one element \( x \in C \) of least norm. Assume that \( y \in C \) is another element of least norm. Necessarily \( \norm x = \norm y \).

  Fix \( t \in (0, 1) \) and define \( z \coloneqq tx + (1-t)y \). Since \( C \) is convex, it contains \( z \). Since \( x \) and \( y \) are elements of least norm, we have \( \norm z \geq \norm x \). By the triangle inequality,
  \begin{balign*}
    \norm{z}
    =
    \norm{tx + (1-t)y}
    \leq
    t \norm x + (1-t) \norm y
    =
    \norm{x},
  \end{balign*}
  thus \( \norm z = \norm x \).

  This implies that the entire segment \( [x, y] \) are elements of least norm in \( C \). Hence, the segment \( [x, y] \) is contained in the sphere \( \norm x S_E \), which contradicts the rotundity of the norm \( \norm{\cdot} \).

  Hence, \( C \) contains at most one element of least norm.

  \ImplicationSubProof{def:rotund_norm/least_norm}{def:rotund_norm/no_sphere_segments} Let every convex set \( C \subseteq E \) have at most one element of least norm.

  Assume that the norm \( \norm{\cdot} \) is not rotund. Then the unit sphere \( S_E \) contains a line segment \( [x, y], x \neq y \). The set \( [x, y] \) is compact and, by the Weierstrass extreme value theorem, the norm attains its minimum on the segment in a point \( z \in [x, y] \). Since the segment is also convex and we assumed that convex sets have at most one element of least norm, it follows that this element \( z \) is unique.

  Then for any point \( s \in [x, y], s \neq z \), we have \( \norm s > \norm z = 1 \), thus \( s \) cannot be an element of the unit sphere. The obtained contradiction shows that the norm \( \norm{\cdot} \) is rotund.

  \ImplicationSubProof{def:rotund_norm/no_sphere_segments}{def:rotund_norm/linearly_dependent} Let \( E \) be rotund let \( x, y \in E \) be distinct vectors such that
  \begin{balign}\label{def:rotund_norm/linearly_dependent/assumption}
    \norm{x + y} = \norm x + \norm y.
  \end{balign}

  If either of them is the zero vector, then they are trivially linearly dependent.

  Assume that both \( x \) and \( y \) are nonzero and define
  \begin{balign*}
    \xi \coloneqq \frac x {\norm x}
     &  &
    \eta \coloneqq \frac y {\norm y}
     &  &
    t \coloneqq \frac {\norm x} {\norm{x + y}}
  \end{balign*}

  \Fullref{def:rotund_norm/linearly_dependent/assumption} implies that
  \begin{equation*}
    1 - t = 1 - \frac {\norm x} {\norm{x + y}} = \frac {\norm{x + y} - \norm x} {\norm{x + y}} = \frac {\norm y} {\norm{x+y}}.
  \end{equation*}

  Since both \( \xi \) and \( \eta \) are in \( S_E \), by rotundity, their convex combination
  \begin{equation*}
    \nu \coloneqq t \xi + (1-t)\eta
  \end{equation*}
  should not be contained in \( S_E \) unless \( \xi = \eta \).

  Calculating the norm, we obtain
  \begin{balign*}
    \norm{\nu}
     & =
    \norm{t \xi + (1-t)\eta}
    =    \\ &=
    \norm{\frac {\norm x \xi} {\norm{x + y}} + \frac {\norm y \eta} {\norm{x + y}}}
    =    \\ &=
    \norm{\frac {x + y} {\norm{x + y}}}
    = 1,
  \end{balign*}
  hence \( \nu \in S_E \). Thus, \( \xi = \eta \) and \( x = \frac {\norm x} {\norm y} y \), so \( x \) and \( y \) are linearly dependent.

  \ImplicationSubProof{def:rotund_norm/linearly_dependent}{def:rotund_norm/no_sphere_segments} Let \fullref{def:rotund_norm/linearly_dependent/equation} hold and fix \( x, y \in S_E, t \in (0, 1) \). Define \( z \coloneqq tx + (1-t)y \).
  First, assume that the vectors \( tx \) and \( (1-t)y \) satisfy the left part of \fullref{def:rotund_norm/linearly_dependent/equation}, i.e.
  \begin{equation*}
    \norm z = \norm{tx + (1-t)y} = t \norm x + (1-t) \norm y = 1.
  \end{equation*}

  This does not refute rotundity since \( x \) and \( y \) are not necessarily distinct. It follows from \fullref{def:rotund_norm/linearly_dependent/equation} that \( tx \) and \( (1-t)y \) are linearly dependent, hence \( x \) and \( y \) are also linearly dependent. Since \( x \) and \( y \) both have unit norm, either \( y = x \) or \( y = -x \).

  If we assume that \( y = -x \), then
  \begin{balign*}
    \norm z
    =
    \norm{tx + (1-t)y}
    =
    (2t - 1) \norm x
    =
    2t - 1,
  \end{balign*}
  which is only possible if \( t = 1 \) since \( \norm z = 1 \). But \( t \) is strictly less than 1.

  Hence, \( y \neq -x \) and the only remaining possibility is that \( y = x \).

  Now assume that the vectors \( tx \) and \( (1-t)y \) do not satisfy the left part of \fullref{def:rotund_norm/linearly_dependent/equation}. This implies \( \norm z < 1 \). Thus, \( x \) and \( y \) are necessarily distinct, but \( z \) is not contained in the unit sphere and the segment \( [x, y] \) is not contained in \( S_E \).

  We have shown that \( x, y \in S_E \) implies that either \( y = x \) or that the segment \( [x, y] \) is not contained in \( S_E \), thus the norm in \( E \) is rotund.
\end{proof}

\begin{theorem}\label{thm:smooth_rotund_norm_duality}\mcite[exerc. 2.37(a)]{Phelps1993ConvexDifferentiability}
  If the norm in a Banach space \( X \) is such that its dual norm in \( X^* \) is rotund (resp. smooth), then it is itself smooth (resp. rotund).
\end{theorem}
\begin{proof}
  \begin{enumerate}
    \item First, let the dual norm \( \norm{\cdot}^* \) be rotund and assume that \( \norm{\cdot} \) is not smooth.

          Fix \( x \in S_X \). Since \( D(x) \) is nonempty (by \fullref{thm:hahn_banach_implies_duality_mapping_nonempty}) and since \( \norm{\cdot} \) is not smooth, then there exist two different functionals \( x^*, y^* \in D(x) \), such that
          \begin{balign*}
            \inprod {x^*} x
            =
            \inprod {y^*} x
            =
            1.
          \end{balign*}

          We will show that the segment \( [x^*, y^*] \) is contained in \( S_{X^*} \), i.e. that the dual norm is not rotund.

          Fix any \( t \in (0, 1) \) and define \( z^* \coloneqq t x^* + (1-t) y^* \). We only need to show that \( \norm{z^*} = 1 \).

          By the triangle inequality, we have
          \begin{balign*}
            \norm{z^*}
            =
            \norm{t x^* + (1-t) y^*}
            \leq
            t \norm{x^*} + (1-t) \norm{y^*}
            =
            t + (1-t)
            =
            1.
          \end{balign*}

          For the reverse inequality, note that
          \begin{balign*}
            \norm{z^*}
            \geq
            \inprod {z^*} x
            =
            t \inprod {x^*} x + (1-t) \inprod {y^*} x
            =
            t + (1-t)
            =
            1,
          \end{balign*}
          thus \( \norm{z^*} = 1 \). Hence, \( [x^*, y^*] \) is contained in \( S_{X^*} \) and the dual space is not smooth. The obtained contradiction proves that the norm in \( X \) is rotund.

    \item Now let the dual norm \( \norm{\cdot}^* \) be smooth and assume that \( \norm{\cdot} \) is not rotund. Then there exist points \( x, y \in S_X \) such that the while segment \( [x, y] \) is contained in \( S_X \).

          Fix \( t \in (0, 1) \) and define \( z \coloneqq tx + (1-t)y \in S_X \). Denote by \( J: X \to X^{**} \) the canonical embedding into the double-dual. By \fullref{thm:hahn_banach_implies_duality_mapping_nonempty}, there exists a functional \( z^* \in X^* \), such that
          \begin{balign*}
            \inprod {J(z)} {z^*}
            =
            \inprod{z^*} z
            =
            1.
          \end{balign*}

          Because the dual norm \( \norm{\cdot}^* \) is smooth, we cannot have \( \inprod{J(x)} {z^*} =  \inprod{z^*} x = 1 \) or \( \inprod{J(y)} {z^*} = \inprod{z^*} y = 1 \) and since \( \norm{z^*} = 1 \), necessarily
          \begin{equation*}
            \inprod{z^*} x < 1 \text{ and } \inprod{z^*} y < 1.
          \end{equation*}

          If follows that
          \begin{balign*}
            1
            =
            \inprod{z^*} z
            =
            t \inprod{z^*} x + (1-t) \inprod{z^*} y
            <
            t + (1-t)
            =
            1,
          \end{balign*}
          which is a contradiction. Hence, \( \norm{\cdot} \) is rotund.
  \end{enumerate}
\end{proof}

\begin{proposition}\label{thm:hilbert_space_smooth_rotund}\mcite[exerc. 2.37(c)]{Phelps1993ConvexDifferentiability}
  Norms in Hilbert spaces are both smooth and rotund.
\end{proposition}
\begin{proof}
  Let \( X \) be a Hilbert space, i.e. the norm is generated by an inner product and, due to Riesz's theorem, we identify the space \( X \) with its continuous dual \( X^* \).

  To prove that \( X \) is rotund, choose \( x, y \in S_X, x \neq y \). We will show that the segment \( [x, y] \) is not contained in \( S_X \).

  If \( x \) and \( y \) are linearly dependent, necessarily \( y = -x \) and all non-trivial convex combinations of \( x \) and \( y \) are contained in the open unit ball, hence \( [x, y] \not\subseteq S_X \).

  Not let \( x \) and \( y \) be linearly independent. By the Cauchy-Bunyakovsky-Schwarz inequality, we have
  \begin{balign}\label{eq:hilbert_cauchy_inequality}
    \inprod x y \leq \abs{\inprod x y} < \norm x \norm y = 1.
  \end{balign}

  Fix \( t \in (0, 1) \) and let \( z \coloneqq tx + (1-t)y \). We will show that \( z \not\in S_X \). Indeed,
  \begin{balign*}
    \norm{z}^2
    =
    \inprod z z
     & =
    t^2 \norm x^2 + t(1-t) \inprod x y + (1-t) t \inprod y x + (1-t)^2 \norm y^2
    =    \\ &=
    t^2 + (1-t)^2 + 2 t(1-t) \inprod x y
    <    \\ &\reloset {(\ref{eq:hilbert_cauchy_inequality})} <
    t^2 + (1-t)^2 + 2 t(1-t)
    =    \\ &=
    t^2 + 1 - 2t + t^2 + 2t - t^2
    =
    1.
  \end{balign*}

  Thus, \( \norm{z}^2 < 1 \) and \( \norm z < 1 \) and \( z \not\in S_X \).

  In both cases, no interior point of the segment \( [x, y] \) is contained in \( S_X \), hence the norm in \( X \) is rotund.

  Since we identify \( X \) with its dual, the norm in \( X^* \) is also rotund and by \fullref{thm:smooth_rotund_norm_duality}, the norm in \( X \) is also smooth.
\end{proof}

\begin{example}\label{thm:c0_l1_not_smooth_rotund}\mcite[exerc. 2.37(c)]{Phelps1993ConvexDifferentiability}
  The norms in \( c_0 \) and \( l^1 \) are neither smooth nor rotund.
\end{example}
\begin{proof}
  Consider the space \( c_0 \) of all real sequences that converge to zero equipped with the uniform norm
  \begin{equation*}
    \norm{x}_{c_0} \coloneqq \sup_i \abs{x_i}.
  \end{equation*}

  Note that the dual space of \( c_0 \) is (isometrically isomorphic to) the space \( l^1 \) of absolutely summable sequences with norm
  \begin{equation*}
    \norm{x}_{l^1} \coloneqq \sum_i \abs{x_i}.
  \end{equation*}

  Let \( \{ e_n \}_{n=1}^\infty \) be the canonical basis of \( c_0 \), i.e. the coordinates \( e^{(i)}_n \) of \( e_n \) are given by the Dirac delta function, \( e^{(i)}_n \coloneqq \delta_{i,n} \).

  For every natural \( n \geq 1 \), define \( x_n \) to be the same as \( e_n \) except that the first coordinate of \( x_n \) is always \( 1 \).

  The corresponding norms of \( e_n \) are all equal to 1 and the norms of \( x_n \) are
  \begin{balign*}
    \norm{x_n}_{c_0} = 1
     &  &
    \norm{x_n}_{l^1} = 2.
  \end{balign*}

  For every \( n \) we have
  \begin{equation*}
    \inprod {e_1} {x_n} = \inprod {e_n} {x_n} = 1,
  \end{equation*}
  hence \( J_{c_0}(x_n) \) has at least two elements \( e_1 \) and \( e_n \) and the norm in \( c_0 \) is not smooth.

  Given that \( \{ x_1, x_2, \ldots \} \subseteq S_{c_0} \), consider the convex combinations of \( x_2 \) and \( x_3 \):
  \begin{balign*}
    tx_2 + (1-t)x_3
    =
    (1, t, (1-t), 0, 0, \ldots).
  \end{balign*}

  Evidently \( tx_2 + (1-t)x_3 \in S_{c_0} \) for every \( t \in (0, 1) \), hence the norm in \( c_0 \) is not rotund.

  The contrapositions to the statements in \fullref{thm:smooth_rotund_norm_duality} say that if \( X \) is not rotund (resp. smooth), then the dual space \( X^* \) is not smooth (resp. rotund). Thus, \( l^1 \) is neither smooth or rotund as the dual of \( c_0 \).
\end{proof}

\begin{definition}\label{def:bilinear_form_induced_norm}
  Let \( V \) be a real or complex \hyperref[def:inner_product_space]{inner product space} with product \( \inprod \cdot \cdot \). We define its induced \hyperref[def:norm]{norm} as
  \begin{balign*}
     & \norm \cdot : V \to \BbbR_{\geq 0}    \\
     & \norm x \coloneqq \sqrt{\inprod x x}.
  \end{balign*}

  If \( V \) is a real inner product space, the induced norm is a square root of the induced quadratic \hyperref[thm:quadratic_forms]{form} of \( \inprod \cdot \cdot \).
\end{definition}
\begin{proof}
  We will only prove the complex case because the real case is identical, but slightly simpler.

  Note that \( \norm \cdot \) is well-defined.

  Now we will show that it is a norm.
  \SubProofOf{def:norm/N1} Follows from the positive definiteness of \( \inprod \cdot \cdot \)

  \SubProofOf{def:norm/N2} For \( t \in \BbbC \) and \( x \in V \) we have
  \begin{equation*}
    \norm{tx} = \sqrt{\inprod{tx} {tx}} = \abs{t} \sqrt{\inprod x x} = \abs t \norm x.
  \end{equation*}

  \SubProofOf{def:norm/N3} For \( x, y \in V \) we have
  \begin{balign*}
    \norm{x + y}^2
     & =
    \inprod{x + y} {x + y}
    =                                                            \\ &=
    \inprod x x + \inprod x y + \inprod y x + \inprod y y
    =                                                            \\ &=
    \norm{x}^2 + 2 \real \inprod x y + \norm{y}^2
    \leq                                                         \\ &\leq
    \norm{x}^2 + 2 \abs{\real \inprod x y} + \norm{y}^2
    \reloset {\ref{thm:cauchy_bunyakovsky_schwarz_inequality}} = \\ &=
    \norm{x}^2 + 2 \norm x \norm y + \norm{y}^2
    =
    (\norm{x} + \norm{y})^2
  \end{balign*}

  Therefore,
  \begin{equation*}
    \norm{x + y} \leq \norm x + \norm y.
  \end{equation*}
\end{proof}

\paragraph{Seminorms}

\begin{definition}\label{def:seminorm}
  \todo{Define seminorms}
\end{definition}

\paragraph{\( p \)-norms}

\begin{definition}\label{def:conjugate_exponent}\mcite[def. 3.4]{Rudin1987RealAndComplexAnalysis}
  We say that the positive real numbers \( p \) and \( q \) are \term{conjugate exponents} if
  \begin{equation}\label{eq:def:conjugate_exponent}
    \frac 1 p  + \frac 1 q = 1.
  \end{equation}
\end{definition}

\begin{proposition}\label{thm:def:conjugate_exponent}
  \hyperref[def:conjugate_exponent]{Conjugate exponents} have the following basic properties:
  \begin{thmenum}
    \thmitem{thm:def:conjugate_exponent/larger_than_one}\mcite[63]{Rudin1987RealAndComplexAnalysis} For any pair \( p \) and \( q \) of conjugate exponents, both \( p > 1 \) and \( q > 1 \).

    \thmitem{thm:def:conjugate_exponent/holder_lemma}\mcite[\S 22.3]{Тыртышников2007ЛинейнаяАлгебра} For nonnegative real numbers \( a \), \( b \), \( p \) and \( q \), where \( q \) is conjugate to \( q \), we have
    \begin{equation}\label{eq:thm:def:conjugate_exponent/holder_lemma}
      ab \leq \frac {a^p} p + {b^q} q.
    \end{equation}
  \end{thmenum}
\end{proposition}
\begin{proof}
  \SubProofOf{thm:def:conjugate_exponent/larger_than_one} We can multiply \eqref{eq:def:conjugate_exponent} by \( pq \) to obtain
  \begin{equation*}
    q + p = pq.
  \end{equation*}

  Then \( q = p(q - 1) \), thus
  \begin{equation*}
    p = \frac q {q - 1} > 1.
  \end{equation*}

  We similarly conclude that \( q > 1 \).

  \SubProofOf{thm:def:conjugate_exponent/holder_lemma} \Fullref{thm:def:logarithm/convex} implies that the \hyperref[def:logarithm]{logarithm function} is \hyperref[def:convex_function]{concave}, thus
  \begin{equation*}
    \ln(ab)
    \reloset {\ref{thm:def:logarithm/homomorphism}} =
    \ln(a) + \ln(b)
    \reloset {\ref{thm:def:logarithm/power}} =
    \frac {\ln(a^p)} p + \frac {\ln(b^p)} q
    \leq
    \ln\parens*{ \frac {a^p} p + \frac {b^p} q }.
  \end{equation*}

  \Fullref{thm:def:exponential_function/monotone} implies that the exponential function is monotonic, hence \eqref{eq:thm:def:conjugate_exponent/holder_lemma} follows.
\end{proof}

\begin{theorem}[H\"older inequality for p-norms]\label{thm:holder_inequality_for_p_norms}\mcite[\S 22.3]{Тыртышников2007ЛинейнаяАлгебра}
  Fix a \hyperref[def:field]{field} \( \BbbK \) with an \hyperref[def:absolute_value]{absolute value} \( \abs\anon \). Fix also scalars \( x_1, \ldots, x_n \) and \( y_1, \ldots, y_n \).

  For \hyperref[def:conjugate_exponent]{conjugate exponents} \( p \) and \( q \), the following \hyperref[def:inequality]{inequality} holds:
  \begin{equation}\label{eq:thm:holder_inequality_for_p_norms}
    \abs*{ \sum_{k=1}^n x_k y_k } \leq \parens*{ \sum_{k=1}^n \abs{x_k}^p }^{1 / p} \cdot \parens*{ \sum_{k=1}^n \abs{x_k}^q }^{1 / q}.
  \end{equation}
\end{theorem}
\begin{proof}
  Let
  \begin{align*}
    a \coloneqq \parens*{ \sum_{k=1}^n \abs{x_k}^p }^{1 / p}
    &&\T{and}&&
    b \coloneqq \parens*{ \sum_{k=1}^n \abs{y_k}^q }^{1 / q}.
  \end{align*}

  Obviously both are nonnegative.

  If \( a = 0 \), then, by positive definiteness of the absolute value, we have \( x_1 = \cdots x_n = 0 \). The inequality then becomes trivial. The case \( b = 0 \) is analogous.

  Suppose that both \( a \) and \( b \) are nonzero.

  For a fixed \( k \), we can apply \fullref{thm:def:conjugate_exponent/holder_lemma} to the quotients \( \abs{ x_k } / a \) and \( \abs{ y_k } / b \):
  \begin{equation*}
    \parens*{ \frac {\abs{ x_k }} a } \cdot \parens*{ \frac {\abs{ y_k }} b }
    \leq
    \frac {\abs{ x_k }^p} {p \cdot a^p} + \frac {\abs{ y_k }^q} {q \cdot b^q}.
  \end{equation*}

  Summing the above, we obtain
  \begin{equation*}
    \frac {\sum_{k=1}^n \abs{ x_k y_k }} {ab}
    \leq
    \frac 1 p + \frac 1 q = 1.
  \end{equation*}

  Then \eqref{eq:thm:holder_inequality_for_p_norms} follows by multiplying by \( ab \).
\end{proof}

\begin{theorem}[Minkowski inequality for p-norms]\label{thm:minkowski_inequality_for_p_norms}\mcite[\S 22.3]{Тыртышников2007ЛинейнаяАлгебра}
  Fix a \hyperref[def:field]{field} \( \BbbK \) with an \hyperref[def:absolute_value]{absolute value} \( \abs\anon \). Fix also scalars \( x_1, \ldots, x_n \) and \( y_1, \ldots, y_n \).

  Then, for any real number \( p \geq 1 \), the following \hyperref[def:inequality]{inequality} holds:
  \begin{equation}\label{eq:thm:minkowski_inequality_for_p_norms}
    \parens*{ \sum_{k=1}^n \abs{x_k + y_k}^p }^{1 / p} \leq \parens*{ \sum_{k=1}^n \abs{x_k}^p }^{1 / p} + \parens*{ \sum_{k=1}^n \abs{y_k}^p }^{1 / p}.
  \end{equation}
\end{theorem}
\begin{proof}
  If \( p = 1 \), \eqref{eq:thm:minkowski_inequality_for_p_norms} follows directly from the subadditivity of the absolute value.

  Otherwise, \( p > 1 \), and
  \begin{equation*}
    \sum_{k=1}^n \abs{x_k + y_k}^p
    =
    \sum_{k=1}^n \abs{x_k + y_k}^{p - 1} \cdot \abs{x_k + y_k}
    \leq
    \sum_{k=1}^n \abs{x_k} \cdot \abs{x_k + y_k}^{p - 1} + \sum_{k=1}^n \abs{y_k} \cdot \abs{x_k + y_k}^{p - 1}.
  \end{equation*}

  Noting that \( q \coloneqq p / (p - 1) \) is the conjugate exponent of \( p \), we can apply \fullref{thm:holder_inequality_for_p_norms} to the summands above. For the first summand we have
  \begin{align*}
    \sum_{k=1}^n \abs{x_k} \cdot \abs{x_k + y_k}^{p - 1}
    \leq
    \parens*{ \sum_{k=1}^n \abs{x_k}^p }^{1 / p} \cdot \parens*{ \sum_{k=1}^n \abs{x_k + y_k}^{(p - 1) \cdot q} }^{1 / q}
  \end{align*}

  Furthermore,
  \begin{equation}\label{eq:thm:minkowski_inequality_for_p_norms/proof}
    \parens*{ \sum_{k=1}^n \abs{x_k + y_k}^{(p - 1) \cdot q} }^{1 / q}
    =
    \parens*{ \sum_{k=1}^n \abs{x_k + y_k}^p }^{(p - 1) / p}.
  \end{equation}

  With both summands, we obtain
  \begin{equation*}
    \sum_{k=1}^n \abs{x_k + y_k}^p
    \leq
    \parens*{ \parens*{ \sum_{k=1}^n \abs{x_k}^p }^{1 / p} + \parens*{ \sum_{k=1}^n \abs{x_k}^p }^{1 / p} } \cdot \parens*{ \sum_{k=1}^n \abs{x_k + y_k}^p }^{(p - 1) / p}.
  \end{equation*}

  It remains to divide by \eqref{eq:thm:minkowski_inequality_for_p_norms/proof} to obtain \eqref{eq:thm:minkowski_inequality_for_p_norms}.
\end{proof}

\begin{definition}\label{def:p_norm}\mcite[\S 22.4]{Тыртышников2007ЛинейнаяАлгебра}
  Fix a \hyperref[def:field]{field} \( \BbbK \) with an \hyperref[def:absolute_value]{absolute value} \( \abs\anon \). For any positive real number \( p \), we define the \term[ru=\( p \)-норма, en=\( p \)-norm (\cite[example 1.2.3(b)]{AtkinsonHan2001TheoreticalNumericalAnalysis})]{\( p \)-norm} in the \hyperref[def:sequence_space]{coordinate space} \( \BbbK^n \) as
  \begin{equation}\label{eq:def:p_norm}
    \norm{ x }_p \coloneqq \parens*{ \sum_{k=1}^n \abs{ x_k }^p }^{1 / p}.
  \end{equation}

  We also allow \( p \) to be the \hyperref[def:extended_real_number]{extended real number} \( \infty \). In this case we call it the \term[en=maximum norm (\cite[example 1.2.3(b)]{AtkinsonHan2001TheoreticalNumericalAnalysis})]{maximum norm} or \term{uniform norm} and define it as
  \begin{equation}\label{eq:def:p_norm/uniform}
    \norm{ x }_\infty \coloneqq \max\set{ \abs{ x_1 }, \ldots, \abs{ x_n } }
  \end{equation}

  \begin{figure}[!ht]
    \centering
    \includegraphics[page=1]{output/def__p_norm}
    \caption{The \hyperref[def:circle]{unit circles} of the \( 1 \)-norm (lightest), \( 2 \)-norm, \( 5 \)-norm and \( \infty \)-norm (darkest) in the \hyperref[def:euclidean_plane]{Euclidean plane}.}\label{fig:def:p_norm}
  \end{figure}
\end{definition}
\begin{comments}
  \item The restriction \( p \geq 1 \) is necessary because \fullref{thm:holder_inequality_for_p_norms} and its consequence \fullref{thm:minkowski_inequality_for_p_norms} fail to hold otherwise.

  \item We generalize the definition by \incite[\S 22.4]{Тыртышников2007ЛинейнаяАлгебра} from \( \BbbC \) and by \incite[example 1.2.3(b)]{AtkinsonHan2001TheoreticalNumericalAnalysis} from \( \BbbR \) to a more general field \( \BbbK \) with absolute value.
\end{comments}
\begin{defproof}
  Both \hyperref[def:real_function_definiteness]{positive definiteness} and \hyperref[def:real_homogeneous_function/absolute]{absolute homogeneity} follow from the analogous properties of the absolute value.

  \hyperref[def:additive_function/sub]{Subadditivity} follows from \fullref{thm:minkowski_inequality_for_p_norms} for \( p < \infty \) and is trivial when \( p = \infty \).
\end{defproof}


  \chapter{Functional analysis}\label{ch:functional_analysis}

\Fullref{ch:real_analysis} and \fullref{ch:complex_analysis} study certain functions with values in \hyperref[thm:vector_space_dimension]{finite-dimensional} \hyperref[def:hilbert_space]{Hilbert spaces}. Functional analysis studies spaces of functions arising from real or complex analysis. These spaces are mostly infinite-dimensional, and a lot of results hold for general infinite-dimensional \hyperref[def:vector_space]{vector spaces}.

Nevertheless, we will only study vector spaces over \hyperref[def:real_numbers]{\( \BbbR \)} or \hyperref[def:complex_numbers]{\( \BbbC \)}. This is justified by \cref{rem:real_field_extensions}. As in \fullref{ch:complex_analysis}, through this section, \( \BbbK \) will refer to either \( \BbbR \) or \( \BbbC \).

  \section{Topological groups}\label{sec:topological_groups}

\begin{definition}\label{def:topological_semigroup}
  \todo{Topological semigroups}
\end{definition}

\begin{definition}\label{def:topological_group}
  Let \( G \) be any \hyperref[def:group]{group} and let \( \mscrT \) be a topology on \( G \). The tuple \( (G, \cdot, \mscrT) \) is called a \term{topological group} if the group structure and topological structure agree, that is, the operations \( \cdot: X \times X \to X \) and \( (-)^{-1}: X \to X \) are continuous with respect to \( \mscrT \).

  See \fullref{rem:hausdorff_topological_groups} and \fullref{def:category_of_topological_groups} for more nuances.
\end{definition}

\begin{remark}\label{rem:hausdorff_topological_groups}
  It is conventional to require the topology in a topological group to be \( T_1 \) (see \fullref{def:separation_axioms}). We will not do this due to our goal of not assuming more than is necessary.

  Due to \fullref{thm:topological_group_t0_iff_t3.5}, it is immaterial whether we require the topology to be \( T_0 \) or \( T_{3.5} \) or anywhere in between. It is customary to call the space \enquote{Hausdorff} (although stronger separation axioms actually hold) and require \( T_1 \) to hold (since it is simple to state).

  We will explicitly mention when we want a topological group to be Hausdorff. This is usually, so when we speak of convergence.
\end{remark}

\begin{definition}\label{def:category_of_topological_groups}
  The category \( \cat{TopGrp} \) of topological groups is a subcategory of both \( \cat{Top} \) and \( \cat{Grp} \). Its morphisms are the \hyperref[def:global_continuity]{continuous} group \hyperref[thm:group_homomorphism_single_condition]{homomorphisms}.
\end{definition}

\begin{proposition}\label{thm:neighborhood_translations_in_topological_groups}
  Fix \( x, y \in G \) in a topological group \( G \). If \( U \) is a neighborhood of \( x \), then both \( V = yx^{-1} U \) and \( W = U x^{-1}y \) are neighborhoods of \( y \).
\end{proposition}
\begin{proof}
  Since the group operations are continuous, for fixed \( x \) and \( y \), the function \( f(z) \coloneqq xy^{-1}z \) is continuous.

  Note that \( U = f(V) \), hence \( V \) is the preimage of \( U \) under \( f \) and it follows from the continuity of \( f \) that \( V \) is open.

  Since \( x \in U \), \( yx^{-1}x = ye = y \in V \). Therefore, \( V \) is a neighborhood of \( y \).

  The proof that \( W \) is a neighborhood of \( y \) is analogous.
\end{proof}

\begin{corollary}\label{thm:origin_neighborhoods_in_topological_groups}
  In a topological group \( G \), every neighborhood is a translation of e neighborhood of the origin \( e \).
\end{corollary}

\begin{remark}\label{rem:origin_neighborhoods_in_topological_groups}
  \Fullref{thm:origin_neighborhoods_in_topological_groups} provides a lot of uniformity by allowing us to only consider neighborhoods of zero when working with topological groups.
\end{remark}

\begin{proposition}\label{thm:topological_group_t0_iff_t3.5}
  If a topological group is \( T_0 \), it is automatically \( T_{3.5} \).
\end{proposition}

\begin{proposition}\label{thm:topological_group_uniform_space}
  A Hausdorff topological group \( G \) can be made into a uniform space by the families of entourages
  \begin{balign*}
     & V^l_A \coloneqq \{ (x, y) \in G \times G \colon x^{-1} y \in A \}, \\
     & V^r_A \coloneqq \{ (x, y) \in G \times G \colon x y^{-1} \in A \},
  \end{balign*}
  where \( A \) is a \hyperref[def:neighborhood_set_types/symmetric]{symmetric} neighborhood of the origin \( e \).

  If \( G \) is abelian, the two families of entourages coincide.
\end{proposition}

\begin{proposition}\label{thm:limits_are_topological_group_homomorphisms}
  If \( \{ a_\alpha \}_{\alpha \in \mscrK} \) and \( \{ b_\alpha \}_{\alpha \in \mscrK} \) are \hyperref[def:topological_net]{nets} in a Hausdorff topological group \( X \) that converge to \( a \) and \( b \), correspondingly, then \( a_\alpha b_\alpha \to a b \).
\end{proposition}
\begin{proof}
  Special case of \fullref{thm:linearity_of_sequence_limits}.
\end{proof}

  \section{Topological vector spaces}\label{sec:topological_vector_spaces}

\begin{definition}\label{def:topological_vector_space}
  Let \( X \) be any vector space and let \( \mscrT \) be a topology on \( X \). The space \( (X, +, \cdot, \mscrT) \) is called a \term{topological vector space} if the linear and topological structure agree, that is, the operations \( +: X \times X \to X \) and \( \cdot: X \times \BbbR \to X \) are continuous with respect to \( \mscrT \).

  Both the additive group \( (X, +) \) and the multiplicative group \( (X \setminus \{ 0 \}, \cdot) \) are \hyperref[def:topological_group]{topological groups}. We regard \( X \) as a subgroup of its additive topological group.

  See \fullref{rem:hausdorff_topological_groups}, \fullref{def:continuous_dual_space} and \fullref{def:category_of_topological_vector_spaces} for more nuances.
\end{definition}

Given that a topological vector space \( X \) has both a topological and an algebraic structure, we should adapt certain definitions.

\begin{definition}\label{def:continuous_dual_space}
  We define the \term{continuous dual space} \( X^* \) of a topological space \( X \) as the vector space of all \hyperref[def:global_continuity]{continuous} linear functionals. This differs drastically from \fullref{def:dual_vector_space} because in the general case, the continuous dual space may be trivial, i.e. only contain the zero functional. See \fullref{def:locally_convex_duality_pairing}.

  We use the same notation for both the algebraic dual spaces and the continuous dual space because the meaning is usually clear from the context. In particular, hyperplanes as defined in \fullref{def:affine_hyperplane} are only relevant to continuous linear functionals.
\end{definition}

\begin{definition}\label{def:category_of_topological_vector_spaces}
  The category \( \cat{TopVect}_{\BbbK} \) of topological vector spaces over \( \BbbK \) is a subcategory of both \( \cat{Top} \) and \( \cat{Vect}_K \). Its morphisms are the \hyperref[def:global_continuity]{continuous} linear \hyperref[def:semimodule/homomorphism]{maps}.
\end{definition}

\begin{remark}\label{rem:origin_neighborhoods_in_topological_vector_spaces}
  As in \fullref{rem:origin_neighborhoods_in_topological_groups}, we are only interested in neighborhoods of the origin \( 0 \) since any neighborhood \( U \) of \( x \) is simply a translation of the neighborhood \( U - x \) of the origin.
\end{remark}

\begin{proposition}\label{thm:topological_vector_space_is_uniform}
  A Hausdorff topological vector space \( X \) is a uniform space with the families of entourages
  \begin{balign*}
     & V_A \coloneqq \{ (x, y) \in X \times X \colon x - y \in A \},
  \end{balign*}
  where \( A \) is a \hyperref[def:neighborhood_set_types/symmetric]{symmetric} neighborhood of the origin \( 0 \).
\end{proposition}
\begin{proof}
  Follows from \fullref{thm:topological_group_uniform_space}.
\end{proof}

\begin{proposition}\label{thm:linearity_of_sequence_limits}
  If \( \{ a_\alpha \}_{\alpha \in \mscrK} \) and \( \{ b_\alpha \}_{\alpha \in \mscrK} \) are \hyperref[def:topological_net]{nets} in a Hausdorff topological vector space \( X \) that converge to \( a \) and \( b \), correspondingly, then
  \begin{thmenum}
    \thmitem{thm:linearity_of_sequence_limits/addition} \( a_\alpha + b_\alpha \to a + b \).
    \thmitem{thm:linearity_of_sequence_limits/scalar_multiplication} \( \lambda a_\alpha \to \lambda a \) for any scalar \( \lambda \in \BbbK \).
  \end{thmenum}
\end{proposition}
\begin{proof}
  Fix a neighborhood \( U \) of \( 0 \) and fix an index \( \alpha_0 \) such that for \( \alpha \geq \alpha_0 \) we have both \( a - a_\alpha \in U \) and \( b - b_\alpha \in U \).

  \SubProofOf{thm:linearity_of_sequence_limits/addition} For addition, we have
  \begin{equation*}
    (a + b) - (a_\alpha + b_\alpha) = (a - a_\alpha) + (b - b_\alpha) \in 2U.
  \end{equation*}

  \SubProofOf{thm:linearity_of_sequence_limits/scalar_multiplication} For scalar multiplication, we have
  \begin{equation*}
    \lambda a - \lambda a_\alpha \in \lambda U.
  \end{equation*}

  In both cases the containing neighborhood does not depend on \( \alpha \), hence the nets converge to their desired values.
\end{proof}

\begin{corollary}\label{thm:linearity_of_function_limits}
  If \( f, g: X \to Y \) are continuous functions between topological vector spaces, then for any point \( x_0 \in X \) we have
  \begin{equation*}
    \lim_{x \to x_0} (f(x) + g(x)) = \lim_{x \to x_0} f(x) + \lim_{x \to x_0} g(x)
  \end{equation*}
  and for any \( \lambda \in \BbbK \)
  \begin{equation*}
    \lim_{x \to x_0} \lambda f(x) = \lambda \lim_{x \to x_0} f(x).
  \end{equation*}
\end{corollary}

\begin{definition}\label{def:locally_convex_space}\mcite[1.8]{Rudin1991FuncAnalysis}
  We say that a \hyperref[def:topological_vector_space]{topological vector space} is \term{locally convex} if there exists a \hyperref[def:topological_base]{topological base} of \hyperref[def:convex_hull]{convex} sets.
\end{definition}

\begin{remark}\label{def:locally_convex_duality_pairing}
  Given a Hausdorff locally convex space \( X \), \fullref{thm:hahn_banach_implies_functionals_vanish_nowhere} shows that the canonical duality pairing as defined in \fullref{def:locally_convex_duality_pairing} is nondegenerate. If the space is not locally convex, we cannot guarantee that the pairing will be nondegenerate and our restriction to continuous linear functionals could interfere with our habits of working with linear functionals.
\end{remark}

\begin{definition}\label{def:sublinear_functional}
  We say that \( f: X \to \BbbR \) is a \term{sublinear functional} if it satisfies
  \begin{thmenum}
    \thmitem{def:sublinear_functional/subadditivity}(subadditivity) \( f(x + y) \leq f(x) + f(y) \) for any \( x, y \in X \).
    \thmitem{def:sublinear_functional/positive_homogeneity}(positive homogeneity) \( f(tx) \leq t f(x) \) for any \( t > 0 \) and \( x \in X \).
  \end{thmenum}

  Compare this definition to \fullref{def:semimodule/homomorphism}.
\end{definition}

\begin{definition}\label{def:neighborhood_set_types}
  The following topology-independent definitions are often used for neighborhoods in a topological vector space \( X \):

  \begin{thmenum}
    \thmitem{def:neighborhood_set_types/absorbing} \( A \) is \term{absorbing} if \( \bigcup_{k=0}^\infty kA = X \).
    \thmitem{def:neighborhood_set_types/symmetric} \( A \) is \term{symmetric} if \( -A = A \).
    \thmitem{def:neighborhood_set_types/balanced} \( A \) is \term{balanced} if \( tA \subseteq A \) for any \( t \in [0, 1] \).
  \end{thmenum}
\end{definition}

  \section{The Hahn-Banach theorem}\label{sec:hahn_banach}

The Hahn-Banach theorem is an important result that can be stated differently and in different levels of generality.

\begin{theorem}[Geometric Hahn-Banach theorem/Mazur's theorem]\label{thm:geometric_hahn_banach}\mcite[24]{ИоффеТихомиров1974ЭкстремЗадачи}
  Fix a \hyperref[def:topological_vector_space]{topological vector space} \( X \). Let \( A \subseteq X \) be an open \hyperref[def:convex_hull]{convex} set and \( L \subseteq X \) be a subspace that is disjoint from \( A \). Then there exists a continuous linear functional \( x^* \in X^* \) such that
  \begin{equation*}
    \begin{aligned}
      \real \inprod{x^*} x > 0, &x \in A \\
      \real \inprod{x^*} x = 0, &x \in L
    \end{aligned}
  \end{equation*}

  See \fullref{rem:linear_functionals_over_c} for a justification of only considering the real part of \( x^* \).
\end{theorem}

\begin{corollary}\label{thm:hahn_banach_implies_functionals_vanish_nowhere}\mcite[24]{ИоффеТихомиров1974ЭкстремЗадачи}
  The \hyperref[def:dual_vector_space]{dual} of a Hausdorff \hyperref[def:locally_convex_space]{locally convex space} \( X \) does not \hyperref[def:functions_vanish_nowhere]{vanish} at the nonzero vectors of \( X \).
\end{corollary}
\begin{proof}
  Fix a nonzero point \( x \in X \). The result follows from \fullref{thm:geometric_hahn_banach} with \( L \coloneqq \{ 0 \} \) and \( A \) --- any convex set containing \( x \) and not containing zero. Such a set \( A \) exists because the topology is Hausdorff and \( x \) has a neighborhood disjoint from any point in \( L \).
\end{proof}

\begin{corollary}\label{thm:hahn_banach_implies_annihilator_nontrivial}\mcite[25]{ИоффеТихомиров1974ЭкстремЗадачи}
  The \hyperref[def:vector_space_annihilator]{annihilator} of any proper subspace of a Hausdorff \hyperref[def:locally_convex_space]{locally convex space} contains nonzero elements.
\end{corollary}
\begin{proof}
  Denote the proper subspace by \( L \subsetneq X \). Fix \( x \in X \setminus L \) and let \( A \) be a convex neighborhood of \( x \) that is disjoint from \( L \). The result follows from \fullref{thm:geometric_hahn_banach}.
\end{proof}

\begin{corollary}\label{thm:hahn_banach_implies_duality_mapping_nonempty}\mcite[25]{ИоффеТихомиров1974ЭкстремЗадачи}
  In a \hyperref[def:norm]{normed} space \( X \), for any nonzero vector \( x \in X \) there exists a continuous functional \( x^* \in S_{X^*} \) such that \( \inprod {x^*} x = \norm x \). In other words, the duality \hyperref[def:duality_mapping]{mapping} is nonempty for any point.
\end{corollary}
\begin{proof}
  This follows from \fullref{thm:hahn_banach_implies_annihilator_nontrivial} by taking \( A \coloneqq B(x, \abs{x}) \) and \( L \coloneqq \{ 0 \} \) and then scaling the obtained functional.
\end{proof}

\begin{definition}\label{def:hyperplane_separation}
  We again restrict our attention to real affine spaces in order to define \term{hyperplane separation}.

  \begin{thmenum}
    \thmitem{def:hyperplane_separation/nonstrict}\mcite[def. 3.3]{Gallier2011Geometry} We say that two sets of points are \term{separated} by a \hyperref[def:affine_hyperplane]{hyperplane} if the sets belongs to different closed half-spaces with respect to the hyperplane.

    More concretely, the sets \( A \) and \( B \) are strongly separated by a hyperplane \( H \) if there exists a parametrization \( f(x) = \inprod l x - c \) such that, whenever \( x \in A \) and \( y \in B \), we have
    \begin{equation}\label{eq:def:hyperplane_separation/nonstrict}
      \inprod l x \leq c \leq \inprod l y.
    \end{equation}

    \begin{figure}[!ht]
      \centering
      \includegraphics[page=1]{output/def__hyperplane_separation__nonstrict}
      \caption{A \hyperref[def:hyperplane_separation/nonstrict]{separating line} between \hyperref[def:circle]{circles}.}\label{fig:def:hyperplane_separation/nonstrict}
    \end{figure}

    \thmitem{def:hyperplane_separation/strong}\mcite[def. 3.3]{Gallier2011Geometry} If the sets belong to different \hi{open} half-spaces, we say that they are \term{strongly separated}. The inequalities then become
    \begin{equation}\label{eq:def:hyperplane_separation/strong}
      \inprod l x < c < \inprod l y.
    \end{equation}

    \begin{figure}[!ht]
      \centering
      \includegraphics[page=1]{output/def__hyperplane_separation__strong}
      \caption{Parallel \hyperref[def:hyperplane_separation/strong]{strongly separating lines} between \hyperref[def:circle]{circles}.}\label{fig:def:hyperplane_separation/strong}
    \end{figure}

    \thmitem{def:hyperplane_separation/supporting}\mcite[def. 3.4]{Gallier2011Geometry} If \( B \) consists of a single point \( x_0 \) from \( A \), we say that a separating hyperplane is a \term{supporting hyperplane} for \( x_0 \) in \( A \). For each \( x \in A \) we have
    \begin{equation}\label{eq:def:hyperplane_separation/supporting}
      \inprod l x \leq \inprod l {x_0}.
    \end{equation}

    \begin{figure}[!ht]
      \centering
      \includegraphics[page=1]{output/def__hyperplane_separation__supporting}
      \caption{Multiple \hyperref[def:hyperplane_separation/supporting]{supporting lines} for a \hyperref[def:circle]{circle}.}\label{fig:def:hyperplane_separation/supporting}
    \end{figure}
  \end{thmenum}
\end{definition}

\begin{theorem}[Hahn-Banach hyperplane separation theorem]\label{thm:hahn_banach_hyperplane_separation}\mcite[25]{ИоффеТихомиров1974ЭкстремЗадачи}
  Fix a \hyperref[def:topological_vector_space]{topological vector space} \( X \). Let \( A, B \subseteq X \) be disjoint \hyperref[def:convex_hull]{convex} sets. If \( \int{A} \neq \varnothing \), there exists a continuous linear functional \hyperref[def:hyperplane_separation]{separating} \( A \) and \( B \).
\end{theorem}

  \section{Frechet spaces}\label{sec:frechet_spaces}

\begin{definition}\label{def:frechet_space}\mcite[1.8 (f)]{Rudin1991FuncAnalysis}
  An \term{F-space} is a \hyperref[thm:uniform_space_completion]{complete} \hyperref[def:metric_topology]{metrizable} \hyperref[def:topological_vector_space]{topological vector space}. We can assume that an F-space is a tuple \( (X, \rho) \), where \( \rho \) is a \hyperref[def:complete_metric_space]{complete} \hyperref[def:translation_invariant_metric]{translation-invariant} \hyperref[def:metric_space]{metric}.

  A \term{Frechet space} is a \hyperref[def:locally_convex_space]{locally convex} F-space.
\end{definition}

  \subsection{Banach spaces}\label{subsec:banach_spaces}

\begin{definition}\label{def:banach_space}
  A \term{Banach space} is a \hyperref[def:norm]{normed} \hyperref[def:vector_space]{vector space} which is also a \hyperref[def:complete_metric_space]{complete metric spaces} with the metric induced by the \hyperref[def:norm_induced_metric]{norm}.
\end{definition}

\begin{definition}\label{def:topological_duality_pairing}
  Let \( M \) and \( N \) be left \( R \)-modules. A \term{duality pairing} \( \inprod \cdot \cdot: M \times N \to R \) is a \hyperref[def:degenerate_bilinear_form]{nondegenerate} bilinear form.

  The \term{canonical duality pairing} of a vector space \( V \) over \( F \) is
  \begin{balign*}
     & \inprod \cdot \cdot: V^* \times V \to F \\
     & \inprod {x^*} x \mapsto x^*(x).
  \end{balign*}
\end{definition}

\begin{example}\label{ex:noncomplete_normed_space}\mcite{MathCounterExamples:noncomplete_normed_space}
  Consider the polynomial \hyperref[def:polynomial_algebra]{algebra} \( \BbbR[x] \) as a vector space with the supremum norm. We will show that it is not complete. Define the sequence
  \begin{equation*}
    p_n(x) \coloneqq \sum_{k=0}^n \frac{x^k} {2^k}, n = 1, 2, \ldots
  \end{equation*}

  Then the limit of the sequence in \( C([0, 1]) \) is the power series
  \begin{equation*}
    \lim_{n \to \infty} p_n(x)
    =
    \sum_{k=0}^n \frac{x^k} {2^k}
    =
    \frac 2 {2 - x}.
  \end{equation*}

  Since \( \BbbR[x] \) is a subspace of \( C([0, 1]) \), we conclude that \( \BbbR[x] \) has fundamental sequence, but we just demonstrated that its limit is not in \( \BbbR[x] \).
\end{example}

\begin{definition}\label{def:dual_norm}
  Fix two nonempty Banach spaces \( (X, \norm{\cdot}_X) \) and \( (Y, \norm{\cdot}_Y) \). We define the \term{operator norm} \( \norm{\cdot}_{\hom(X, Y)} \) on \( \hom(X, Y) \) equivalently as
  \begin{thmenum}
    \thmitem{def:dual_norm/sup_unit_sphere}
    \begin{equation*}
      \norm{L}_{\hom(X, Y)} \coloneqq \sup_{\norm{x}_X = 1} \norm{Lx}_Y.
    \end{equation*}

    \thmitem{def:dual_norm/sup_unit_ball}
    \begin{equation*}
      \norm{L}_{\hom(X, Y)} \coloneqq \sup_{\norm{x}_X < 1} \norm{Lx}_Y.
    \end{equation*}

    \thmitem{def:dual_norm/sup_nonzero}
    \begin{equation*}
      \norm{L}_{\hom(X, Y)} \coloneqq \sup_{x \neq 0_X} \frac {\norm{Lx}_Y} {\norm{x}_X}.
    \end{equation*}

    \thmitem{def:dual_norm/inf}
    \begin{equation*}
      \norm{L}_{\hom(X, Y)} \coloneqq \inf \left\{ c \geq 0 \colon \norm{Lx}_Y \leq c \norm{x}_X \right\}.
    \end{equation*}
  \end{thmenum}

  In particular, this induces a norm on \( X^* \).
\end{definition}

\begin{definition}\label{def:banach_space_support_function}\mcite[exmpl. 3.2(a)]{Phelps1993})
  Let \( X \) be a Banach space.

  We define the \term{support function \( \sigma_{A^*} \) for the set of functionals \( A^* \subseteq X^* \)} by
  \begin{balign*}
     & \sigma_{A^*}: X \to \BbbR \cup \{ \infty \}                             \\
     & \sigma_{A^*}(x) \coloneqq \sup \{ \inprod {x^*} x \colon x^* \in A^* \}
  \end{balign*}

  and the \term{weak* support function \( \sigma^*_A \) for the set of points \( A \subseteq X \)} by
  \begin{balign*}
     & \sigma^*_A: X^* \to \BbbR \cup \{ \infty \}                          \\
     & \sigma^*_A(x^*) \coloneqq \sup \{ \inprod {x^*} x \colon x \in A \}.
  \end{balign*}
\end{definition}

\begin{definition}\label{def:banach_space_slice}\mcite[def. 2.17]{Phelps1993}
  Given a linear functional \( x^* \), a nonempty subset \( A \) of \( X \) and a \term{diameter} \( \alpha > 0 \), the value \( S(x^*, A, \alpha) \) is called a \term{slice} of \( A \), where
  \begin{balign*}
     & S: X^* \times \pow(X) \times \BbbR_{>0} \mapsto \pow(A)                                      \\
     & S(x^*, A, \alpha) \coloneqq \{ x \in A \colon \inprod {x^*} x > \sigma_A^*(x^*) - \alpha \}.
  \end{balign*}

  We define a weak* slice of \( A^* \subseteq X^* \) as \( S^*(x, A^*, \alpha) \), where
  \begin{balign*}
     & S^*: X \times \pow(X) \times \BbbR_{>0} \mapsto \pow(A)                                            \\
     & S^*(x, A^*, \alpha) \coloneqq \{ x^* \in A^* \colon \inprod {x^*} x > \sigma_{A^*}(x) - \alpha \}.
  \end{balign*}

  If we need to make the underlying space explicit, we will use \( S_X(x^*, A, \alpha) \) and \( S_X^*(x, A^*, \alpha) \).
\end{definition}

\begin{proposition}
  If \( \{ a_k \}_{k=1}^\infty \) and \( \{ b_k \}_{k=1}^\infty \) are sequences a in a \hyperref[def:banach_space]{Banach} \hyperref[def:algebra_over_semiring]{algebra} \( X \), that converge to \( a \) and \( b \), correspondingly, then \( a_k b_k \to a b \).
\end{proposition}
\begin{proof}
  Let \( \delta > 0 \) and let \( k_0 \) be an index such that for \( k \geq k_0 \) we have both \( \norm{a - a_k} < \delta \) and \( \norm{b - b_k} < \delta \). Then
  \begin{balign*}
    ab - a_k b_k
     & =
    (ab - a b_k) + (a b_k - a_k b) + (a_k b - a_k b_k)
    =    \\ &=
    a (b - b_k) + (a b_k - ab + ab - a_k b) + (-a_k)(b_k - b)
    =    \\ &=
    a (b - b_k) + a \underbrace{(b_k - b)} + (a - a_k) b + (-a_k)\underbrace{(b_k - b)}
    =    \\ &=
    a \underbrace{(b - b_k)}_{\in B(0, \delta)} + \underbrace{(a - a_k)}_{\in B(0, \delta)} \underbrace{(b_k - b)}_{\in B(0, \delta)} + \underbrace{(a - a_k)}_{\in B(0, \delta)} b.
  \end{balign*}

  Therefore, \( \norm{ab - a_k b_k} < \delta^2 + \norm{a + b} \delta \). If we require \( \delta \) to be strictly less than \( 1 \), we obtain \( \delta^2 < \delta \) and \( \norm{ab - a_k b_k} < (1 + \norm{a + b}) \delta \).

  Given an arbitrary \( \varepsilon > 0 \), we can choose \( \delta = \tfrac {\min \{\varepsilon, 1 \}} {1 + \norm{a + b}} \) in order to have \( \norm{ab - a_k b_k} < \varepsilon \) for some large enough \( k \).

  Therefore, \( a_k b_k \to a b \).
\end{proof}

  \subsection{Hilbert spaces}\label{subsec:hilbert_spaces}

\begin{definition}\label{def:hilbert_space}
  A \term{Hilbert space} is an \hyperref[def:inner_product_space]{inner product space} which is also a \hyperref[def:complete_metric_space]{complete metric space} with the metric induced by the inner product.
\end{definition}

\begin{definition}\label{def:orthonormal_system}
  A set of vectors \( A \) in a Hilbert space \( X \) is called an \term{orthonormal system} if \( A \)
  \begin{equation*}
    \inprod x y \coloneqq \begin{cases}
      1, & x = y,    \\
      0, & x \neq y.
    \end{cases}
  \end{equation*}

  It is a special case of an \hyperref[def:orthogonality]{orthogonal system}. We are usually interested in \term{orthogonal bases}.
\end{definition}

  \subsection{Reflexive spaces}\label{subsec:reflexive_spaces}

  \section{Asplund spaces}\label{sec:asplund_spaces}

\begin{definition}\label{def:asplund_space}
  The Banach space \( X \) is called an Asplund (resp. weak Asplund) space if any of the following equivalent conditions hold:

  \begin{thmenum}
    \thmitem{def:asplund_space/differentiable_on_dense_subset}\mcite[thm. 2.14]{Phelps1993ConvexDifferentiability}Every continuous convex function on a convex open subset \( D \) of \( X \) is Frechet (resp. Gateaux) differentiable at a dense \( G_\delta \) subset of \( D \).

    \thmitem{def:asplund_space/radon_nikodym}\mcite[def. 5.2]{Phelps1993ConvexDifferentiability}The dual space \( X^* \) has the Radon-Nikodym property.

    \medskip

    \thmitem{def:asplund_space/exposed_points}\mcite[thm. 5.12]{Phelps1993ConvexDifferentiability}Every nonempty weak* compact convex subset of \( X^* \) is the weak* closed convex hull of its weak* strongly exposed points.
  \end{thmenum}
\end{definition}

  \subsection{Minkowski functionals}\label{subsec:minkowski_functionals}

\begin{definition}\label{def:minkowski_functional}
  Let \( A \) is an \hyperref[def:neighborhood_set_types/absorbing]{absorbing} \hyperref[def:convex_hull]{convex} set.

  We define the corresponding \term{Minkowski functional}
  \begin{balign*}
     & \rho_A: X \to [0, \infty),
     & \rho_A(x) = \inf \{ t > 0 \colon x \in tA \}.
  \end{balign*}
\end{definition}
\begin{proof}
  We will prove that \( \rho_A(x) \) is always a nonnegative real number. Obviously
  \begin{equation*}
    \rho_A(x) \geq 0
  \end{equation*}
  since the infimum over \( \BbbR_{>0} \) is \( 0 \).

  Now fix \( x \in X \). Since \( A \) is an absorbing set, there exists \( t_0 > 0 \) such that \( t_0 x \in A \). We need to take the infimum of all such numbers. This infimum exists since \( \BbbR \) is complete and the set over which we take the minimum is bounded.
\end{proof}

  \subsection{Dentable sets}\label{subsec:dentable_sets}

\begin{definition}\label{def:dentability}\mcite[def. 5.1]{Phelps1993Convex}
  A subset \( A \) of a Banach space \( X \) is called \term{dentable} if it admits slices of arbitrarily small diameter, i.e. for every \( \varepsilon > 0 \) there exist a functional \( x^* \in X^* \) and a diameter \( \alpha > 0 \), such that \( \diam S(x^*, A, \alpha) < \varepsilon \).

  Weak* dentability is defined in an obvious way.
\end{definition}

\begin{definition}\label{def:radon-nikodym-property}\mcite[def. 5.2]{Phelps1993Convex}
  The space \( X \) is said to have the \term{Radon-Nikodym property (RNP)} if every nonempty bounded set \( A \) of \( X \) is dentable.
\end{definition}

\begin{proposition}\label{thm:weak_dentable_sets_are_dentable}
  Let \( X \) be a Banach space and \( A^* \subseteq X^* \) be a weak*-dentable set. Then \( A^* \) is dentable in \( X^* \).
\end{proposition}
\begin{proof}
  Let \( \varepsilon > 0 \) and let \( x \in X \) and \( \alpha > 0 \) be such that \( \diam S^*(x, A^*, \alpha) < \varepsilon \).
  We denote by \( J(x) \) the embedding of \( x \in X \) into the double-dual \( X^{**} \) and by \( T(J(x), A^*, \alpha) \) the slice of \( A^* \) in \( X^* \). We have that
  \begin{balign*}
    S^*(x, A^*, \alpha)
     & =
    \{ x^* \in A^* \colon \inprod {x^*} x > \sigma_{A^*}(x) - \alpha \}
    =    \\ &=
    \{ x^* \in A^* \colon \inprod {x^*} x > \sup \{ \inprod {y^*} x \colon y^* \in A^* \} - \alpha \}
    =    \\ &=
    \{ x^* \in A^* \colon \inprod {J(x)} {x^*} > \sup \{ \inprod {J(x)} {y^*} \colon y^* \in A^* \} - \alpha \}
    =
    T(J(x), A^*, \alpha),
  \end{balign*}

  Since \( J \) is an isometry, this equality implies that
  \begin{equation*}
    \diam T(J(x), A^*, \alpha) = \diam S(x, A^*, \alpha) < \varepsilon.
  \end{equation*}

  Hence, \( A^* \) admits arbitrarily small slices in \( X^* \), i.e. it is dentable in \( X^* \).
\end{proof}

  \subsection{Differentiability}\label{subsec:differentiability}

Let \( X \) and \( Y \) be Hausdorff \hyperref[def:topological_vector_space]{topological vector spaces} and let \( U \subseteq X \) be an open set.

\begin{definition}\label{def:differentiability}
  Out goal is to study the following \hyperref[def:partial_function]{partial} operator:
  \begin{balign}\label{def:differentiability/partial_operator}
     & \partial: \cat{Set}(U, Y) \times U \times X \to Y                             \\
     & \partial(f, x, h) \coloneqq \lim_{t \downarrow 0} \frac {f(x + th) - f(x)} t.
  \end{balign}

  We implicitly assume that \( t \neq 0 \) because otherwise the definition would not make sense.

  We only use the operator \( \partial \) inside this definition. See \fullref{rem:derivative_notation} for a discussion of derivative notation.

  The quotient under the limit sign is called a \term{difference quotient}.

  For each function \( f: U \to Y \), each point \( x_0 \in X \) and each \enquote{direction} vector \( x_0 \in X \), we want to obtain a value in \( Y \), which we will call the \term{directional derivative} of \( f \) at \( x_0 \) in the direction \( h \). Note that \( h \) is allowed to range over \( X \).

  The existence of a directional derivative is already a harsh condition, however we impose even harsher restrictions

  \begin{thmenum}
    \thmitem{def:differentiability/first_variation}\mcite[sec. 0.2.1]{ИоффеТихомиров1974}If, for fixed \( f \) and \( x_0 \), the directional derivative \( \partial(f, x_0, h) \) exists for all directions \( h \), we define the \term{first variation} of \( f \) at \( x_0 \) as
    \begin{balign*}
       & \delta f(x_0): X \to Y                            \\
       & [\delta f(x_0)](h) \coloneqq \partial(f, x_0, h).
    \end{balign*}

    Within its domain of definition of \( \delta \), which is stricter than that of \( \partial \), the operator \( \delta \) is a \hyperref[def:function/currying]{currying} of \( \partial \). We are interested in how the operator \( \delta f(x) \) varies as \( x \) varies.

    Note that \( \delta f(x_0) \) is an operator from \( X \) to \( Y \) even if \( f \) is a function from \( U \subsetneq X \) to \( Y \).

    Note that, in general, the first variation operator \( \delta f(x_0) \) is not linear - for example, by \fullref{thm:convex_one_sided_derivatives_sublinear}, the first variation of a general convex functions is, at most, sublinear. See \fullref{subsec:nonsmooth_derivatives} for how \enquote{nonlinear derivatives} are handled.

    \thmitem{def:differentiability/gateaux}\mcite[sec. 0.2.1]{ИоффеТихомиров1974}If the first variation \( \delta f(x_0) \) is a continuous linear operator, we say that \( f \) is \term{Gateaux differentiable} or \term{weakly differentiable} at \( x_0 \) with \term{Gateaux derivative} \( f'_G(x_0) \coloneqq \delta f(x_0) \).

    Since \( f'(x_0) \) is linear in \( h \), we can replace \( t \downarrow 0 \) with \( t \to 0 \) in \fullref{def:differentiability/partial_operator} and reformulate this condition of Gateaux differentiability as the existence of a continuous linear operator \( \Lambda: X \to Y \) such that
    \begin{equation}\label{def:differentiability/gateaux/condition}
      \Lambda(h) = \lim_{t \to 0} \frac {f(x_0 + h) - f(x_0)} t.
    \end{equation}

    If \( \Lambda \) exists, we usually denote it by \( D_G f(x_0) \) or \( f_G'(x_0) \) and call it the \term{Gateaux derivative} of \( f \) at \( x_0 \). See \fullref{rem:derivative_notation} for a discussion of the notation.

    \thmitem{def:differentiability/frechet}\mcite[sec. 0.2.1]{ИоффеТихомиров1974}We now restrict our attention to \hyperref[def:banach_space]{Banach spaces}. We say that \( f \) is \term{Frechet differentiable} or \term{strongly differentiable} at \( x_0 \) if there exists a continuous linear operator \( \Lambda: X \to Y \) such that
    \begin{equation}\label{def:differentiability/frechet/condition}
      \lim_{h \to 0} \frac {\norm{f(x_0 + h) - f(x_0) - \Lambda(h)}_Y} {\norm{h}_X} = 0.
    \end{equation}

    If \( \Lambda \) exists, we usually denote it by \( D f(x_0) \) or \( f(x_0) \) and call it the \term{Frechet derivative} of \( f \) at \( x_0 \). See \fullref{rem:derivative_notation} for a discussion of the notation.

    Note that \fullref{def:differentiability/gateaux/condition} uses convergence in the topology of \( Y \) while \fullref{def:differentiability/frechet/condition} uses convergence in \( \BbbR \). We discuss in \fullref{rem:gateaux_vs_frechet} how Frechet differentiability is a special \enquote{uniform} case of Gateaux differentiability.

    \thmitem{def:differentiability/strict}\mcite[33]{DontchevRockafellar2014}If there exists a continuous linear operator \( \Lambda \) such that
    \begin{equation}\label{def:differentiability/strict/condition}
      \lim_{\substack{y \to x_0 \\ z \to x_0}} \frac {\norm{f(y) - f(z) - \Lambda (y - z)}_Y} {\norm{y - z}_X} = 0,
    \end{equation}
    we say that \( f \) is \term{strictly differentiable} at \( x_0 \).
  \end{thmenum}
\end{definition}

\begin{remark}\label{rem:derivative_notation}
  The following are standard notations for derivatives (some of the comments are based on \cite[146]{Фихтенгольц1968Том2}):
  \begin{thmenum}
    \thmitem{rem:derivative_notation/lagrange} We already used \term{Lagrange's notation} \( f'(x_0) \) and \( f_G'(x_0) \) in \fullref{def:differentiability}. Brevity is the only benefit of this notation. It becomes convenient when the functions have no name is a burden for directional derivatives.

    The second and third derivatives of \( f \) at \( x_0 \) are denoted as \( f^{''}(x_0) \) and \( f^{'''}(x_0) \) and the \( n \)-th derivative of is denoted as \( f^{(n)} \).

    See \fullref{def:nonsmooth_derivatives} for variations of this notation.

    \thmitem{rem:derivative_notation/newton} Newton's notation is similar to that of Leibniz, but depends on placing dots on top of \( f \), e.g. \( \ddot{f}(x_0) \coloneqq f''(x_0) \). This is used in areas like mathematical physics, however it has not become standard in more pure areas of analysis.

    \thmitem{rem:derivative_notation/euler} We use \term{Euler's notation} \( Df(x_0) \coloneqq f'(x_0) \) for more complicated expressions, e.g. \fullref{thm:derivative_limit_exchange}. The main benefit of this notation is that is allows to express differentiation as an operator, similar to what we defined in \fullref{def:differentiability}. The directional derivative of \( f \) at \( x_0 \) in the direction \( h \) is denoted as \( D_h f(x_0) \). Iterated differentiation corresponds to the standard notation for group composition: the \( n \)-th derivative at \( x_0 \) is denoted as \( D^n f(x_0) \).

    We also use other letters in the superscripts like \( D^G f(x_0) \) for Gateaux derivatives, \( D^\circ f(x_0) \) for Clarke's generalized derivatives, etc.

    \thmitem{rem:derivative_notation/phelps} Some authors like \cite{Phelps1993} use a variation of Euler's notation with \( \partial \) instead of \( D \). For example, directional derivatives are introduced as \( \partial^+ f(x_0)(h) \) in \cite[lemma 1.2]{Phelps1993}. This is consistent with the standard notation for subdifferentials - see \fullref{def:subdifferentials}, however Euler's notation appears to be more widely adoped.

    \thmitem{rem:derivative_notation/leibniz} The Leibniz notation for the derivative \( f'(x_0) \) is
    \begin{equation*}
      \diff f x (x_0) \coloneqq D f(x_0).
    \end{equation*}

    This notation is used extensively in integral calculus, however it is often confusing when manipulating derivatives. The fraction notation is unjustified in anything, but trivial cases and the partial derivative notation
    \begin{equation*}
      \diffp f x (x_0) \coloneqq D_x f(x_0)
    \end{equation*}
    is even more confusing.

    Note also that this depends on the convention of having variable names.
  \end{thmenum}
\end{remark}

\begin{remark}\label{rem:gateaux_vs_frechet}
  We will compare Gateaux \hyperref[def:differentiability/gateaux]{differentiability} with Frechet \hyperref[def:differentiability/frechet]{differentiability}. Let \( X \) and \( Y \) be Banach spaces, let \( U \subseteq X \) be an open set and let \( f: U \to Y \) be an arbitrary function. Fix a point \( x_0 \in U \).

  The continuous linear operator \( \Lambda: X \to Y \) is a Gateaux derivative if, for every \( \varepsilon > 0 \) and every direction \( h \in X \) there exists \( \delta_G^h > 0 \) such that
  \begin{equation}\label{rem:gateaux_vs_frechet/gateaux}
    \norm{\frac {f(x_0 + th) - f(x_0)} t - \inprod \Lambda h}_Y < \varepsilon \quad\forall t \in (0, \delta_G^h).
  \end{equation}

  In order for \( \Lambda \) to be a Frechet derivative, for every \( \varepsilon > 0 \) there must exist a \( \delta_F > 0 \), so that
  \begin{equation*}
    \frac{\norm{f(x_0 + h) - f(x_0) - \inprod \Lambda h}_Y} {\norm{h}_X} < \varepsilon \quad\forall h \in B(0, \delta_F) \setminus \{ 0 \},
  \end{equation*}
  which can be restated as
  \begin{equation}\label{rem:gateaux_vs_frechet/frechet}
    \norm{\frac {f(x_0 + th) - f(x_0)} t - \inprod \Lambda h}_Y < \varepsilon \quad\forall t \in (0, \delta_F) \ \forall h \in S_X.
  \end{equation}

  By comparing \fullref{rem:gateaux_vs_frechet/gateaux} to \fullref{def:differentiability/frechet}, we conclude that \( f \) is Frechet differentiable at \( x_0 \) if \( \inf_{h \in S_X} \delta^h_G > 0 \), that is, if \( f \) is Gateaux differentiable and the convergence of the Gateaux derivative is uniform on \( h \in S_X \).

  In particular, Frechet differentiability implies Gateaux differentiability.
\end{remark}

\begin{definition}\label{def:function_regular_at_point}
  We say that a function is \term{regular} at a point if its derivative at that point is nonzero.
\end{definition}

\begin{theorem}[Chain rule]\label{thm:chain_rule}
  \todo{Prove.}
\end{theorem}

  \subsection{Banach space interpolation}\label{subsec:banach_space_interpolation}

\begin{definition}\label{def:interpolated_topological_vector_space}\mcite[24]{BerghLofstrom1976}
  Let \( \BbbK \) be either the \hyperref[def:real_numbers]{field \( \BbbR \) of real numbers} or the \hyperref[def:real_numbers]{field \( \BbbC \) of complex numbers}.

  \begin{thmenum}
    \thmitem{def:interpolated_topological_vector_space/compatibility} We say that two \hyperref[def:topological_vector_space]{topological vector spaces} \( X_0 \) and \( X_1 \) are \term{compatible} if they can both be \hyperref[def:morphism_invertibility/left_cancellative]{embedded} \hyperref[def:global_continuity]{continuously} into a \hyperref[def:separation_axioms/T2]{Hausdorff} topological vector space \( \mscrU \), in which case we can regard them as subspaces of \( \mscrU \).

    In particular, both \( X_0 \) and \( X_1 \) are Hausdorff. We write \( \overline{X} \coloneqq (X_0, X_1) \).

    \thmitem{def:interpolated_topological_vector_space/intersection} Denote by
    \begin{equation*}
      \Delta \overline{X} \coloneqq X_0 \cap X_1
    \end{equation*}
    the \term{intersection} of \( X_0 \) and \( X_1 \) (when regarded as subspaces of \( \mscrU \)).

    \thmitem{def:interpolated_topological_vector_space/sum} Denote by
    \begin{equation*}
      \Sigma \overline{X} \coloneqq ( X_0 + X_1 )
    \end{equation*}
    the \term{sum} of \( X_0 \) and \( X_1 \). If \( x \in \Sigma \overline{X} \), then there exist (possibly nonunique) vectors \( x_0 \in X_0 \) and \( x_1 \in X_1 \) such that \( x = x_0 + x_1 \).

    \thmitem{def:interpolated_topological_vector_space/intermediate_space} Let \( \overline{X} \) be a pair of compatible spaces. We say that the space \( X \) is an \term{intermediate} space for \( \overline{X} \) if \( \Delta \overline{X} \subseteq X \subseteq \Sigma \overline{X} \) with continuous linear inclusions.

    \thmitem{def:interpolated_topological_vector_space/morphisms} We introduce \hyperref[def:category/morphisms]{morphisms} between two compatible pairs \( \overline{X} \) and \( \overline{Y} \) that are, strictly speaking, not \hyperref[def:function]{functions} between the pairs themselves. We define an \term{operator} \( T: \overline{X} \to \overline{Y} \) between compatible pairs to be a function \( T \) from \( \Sigma \overline{X} \) to \( \Sigma \overline{Y} \) that satisfies the additional conditions
    \begin{align*}
      T(X_0) \subseteq Y_0
      &&
      T(X_1) \subseteq Y_1.
    \end{align*}

    \thmitem{def:interpolated_topological_vector_space/category} If \( \cat{C} \) is a \hyperref[def:subcategory]{subcategory} of the category \hyperref[def:category_of_topological_vector_spaces]{\( \cat{TopVect}_{\BbbK} \)} of topological vector spaces. We define the category \( \cat{Interp}_{\cat{C}} \) as the product category \( \cat{TopVect}_{\BbbK} \times \cat{TopVect}_{\BbbK} \). More explicitly:
    \begin{refenum}
      \refitem{def:category/objects} The \hyperref[def:set]{class} of objects is the class of all pairs of \hyperref[def:interpolated_topological_vector_space/compatibility]{compatible spaces}.
      \refitem{def:category/morphisms} The morphisms between two compatible pairs are the \hyperref[def:interpolated_topological_vector_space/morphisms]{continuous linear operators} \( T: \overline{X} \to \overline{Y} \) between them.
      \refitem{def:category/composition} Composition of morphisms is the usual \hyperref[def:multi_valued_function/composition]{function composition} if we regard a morphism \( T: \overline{X} \to \overline{Y} \) as a function from \( \Sigma \overline{X} \) to \( \Sigma \overline{Y} \).
    \end{refenum}

    \thmitem{def:interpolated_topological_vector_space/interpolation_space} We say that the intermediate spaces \( X \) for \( \overline{X} \) and \( Y \) for \( \overline{Y} \) are a pair of \term{interpolation spaces} with respect to \( \overline{X} \) and \( \overline{Y} \) if, for any continuous linear \hyperref[def:interpolated_topological_vector_space/morphisms]{operator} \( T: \overline{X} \to \overline{Y} \) between the compatible pairs, we have \( T(X) \subseteq Y \).
  \end{thmenum}
\end{definition}

\begin{lemma}\label{thm:preordered_magma_max_distributivity}
  In a \hyperref[def:ordered_magma]{preordered magma} \( M \),
  \begin{equation}\label{eq:thm:preordered_magma_max_distributivity}
    \max \set{a b, c d} \leq \max \set{a, c} \cdot \max \set{b, d}.
  \end{equation}
\end{lemma}
\begin{proof}
  Since \( a \leq \max \set{a, c} \), then
  \begin{equation*}
    ab
    \leq
    \max \set{a, c} \cdot b
    \leq
    \max \set{a, c} \cdot \set{b, d}
  \end{equation*}

  Analogously, \( cd \leq \max \set{a, c} \cdot \set{b, d} \) and
  \begin{equation*}
    \max \set{a b, c d} \leq \max \set{a, c} \cdot \set{b, d}.
  \end{equation*}
\end{proof}

\begin{proposition}\label{def:banach_space_sum_and_intersection_norms}\mcite[24]{BerghLofstrom1976}
  Let \( X \coloneqq (X_0, X_1) \) be a \hyperref[def:interpolated_topological_vector_space/compatibility]{compatible pair} of \hyperref[def:banach_space]{Banach spaces}.

  \begin{thmenum}
    \thmitem{def:banach_space_sum_and_intersection_norms/intersection} The intersection \( \Delta \overline{X} = X_0 \cap X_1 \) is a Banach space with norm
    \begin{equation}\label{eq:def:banach_space_sum_and_intersection_norms/intersection}
      \norm{x}_{\Delta \overline{X}} \coloneqq \max \set{ \norm{x}_{X_0}, \norm{x}_{X_1} }.
    \end{equation}

    \thmitem{def:banach_space_sum_and_intersection_norms/sum} The sum \( \Sigma \overline{X} = X_0 + X_1 \) is a Banach space with norm
    \begin{equation}\label{eq:def:banach_space_sum_and_intersection_norms/sum}
      \norm{x}_{\Delta \overline{X}} \coloneqq \inf \set{ \norm{x_0}_{X_0} + \norm{x_1}_{X_1} : x_0 + x_1 = x }.
    \end{equation}
  \end{thmenum}
\end{proposition}
\begin{proof}
  \SubProofOf{def:banach_space_sum_and_intersection_norms/intersection} We will first show that \( \norm{x}_{\Delta \overline{X}} \) is indeed a norm.
  \begin{refenum}
    \refitem{def:norm/N1} We have
    \begin{equation}\label{eq:def:banach_space_sum_and_intersection_norms/intersection/zero}
      \norm{x}_{\Delta \overline{X}} = \max \set{ \norm{x}_{X_0}, \norm{x}_{X_1} } = 0 \T{if and only if} \norm{x}_{X_0} = \norm{x}_{X_1} = 0.
    \end{equation}

    Clearly \( 0 \) belongs to both \( X_0 \) and \( X_1 \) hence to their intersection. Therefore, \eqref{eq:def:banach_space_sum_and_intersection_norms/intersection/zero} is satisfied if and only if \( x = 0 \).

    \refitem{def:norm/N2} Absolute homogeneity follows from
    \begin{equation*}
      \norm{tx}_{\Delta \overline{X}}
      =
      \max \set{ \norm{tx}_{X_0}, \norm{tx}_{X_1} }
      \reloset {\ref{def:norm/N2}} =
      \abs{t} \max \set{ \norm{x}_{X_0}, \norm{x}_{X_1} }
      =
      \abs{t} \norm{x}_{\Delta \overline{X}}.
    \end{equation*}

    \refitem{def:norm/N3} Subadditivity follows from
    \begin{balign*}
      \norm{x + y}_{\Delta \overline{X}}
      &=
      \max \set{ \norm{x + y}_{X_0}, \norm{x + y}_{X_1} }
      \reloset {\ref{def:norm/N3}} \leq \\ &\leq
      \max \set{ \norm{x}_{X_0} + \norm{y}_{X_0}, \norm{x}_{X_1} + \norm{y}_{X_1} }
      \reloset {\ref{eq:thm:preordered_magma_max_distributivity}} \leq \\ &\leq
      \max \set{ \norm{x}_{X_0}, \norm{x}_{X_1} } + \max \set{ \norm{y}_{X_0}, \norm{y}_{X_1} }
      = \\ &=
      \norm{x}_{\Delta \overline{X}} + \norm{y}_{\Delta \overline{X}}.
    \end{balign*}
  \end{refenum}

  We will now show the completeness of \( \norm{\cdot}_{\Delta \overline{X}} \) directly. Let \( \seq{ x_k }_{k=1}^\infty \subseteq \Delta \overline{X} \) be a \hyperref[def:fundamental_net]{fundamental sequence}. Both \( X_0 \) and \( X_1 \) are complete, therefore \( \seq{ x_k }_{k=1}^\infty \) converges to the same value. Both are subspaces of \( \mscrU \), therefore the limit of the sequence is the same in both. In particular, it belongs to the intersection \( \Delta \overline{X} \).

  Denote the limit of \( \seq{ x_k }_{k=1}^\infty \) by \( x \). Let \( \varepsilon > 0 \) and let \( k_0 \) be an index such that both \( \norm{x_k - \xi_0}_{X_0} < \varepsilon \) and \( \norm{x_k - \xi_1}_{X_1} < \varepsilon \) whenever \( k \geq k_0 \). Then, for any \( k \geq k_0 \),
  \begin{equation*}
    \norm{x_k - x}_{\Delta \overline{X}}
    =
    \max\set{\norm{x_k - x}_{X_0}, \norm{x_k - x}_{X_1}}
    <
    \varepsilon.
  \end{equation*}

  Therefore, the sequence \( \seq{ x_k }_{k=1}^\infty \) converges to \( x_0 \) in \( \Delta \overline{X} \).

  \SubProofOf{def:banach_space_sum_and_intersection_norms/sum} Again, we will first show that \( \norm{x}_{\Sigma \overline{X}} \) is indeed a norm.
  \begin{refenum}
    \refitem{def:norm/N1} Analogously to \ref{def:banach_space_sum_and_intersection_norms/sum},
    \begin{equation*}
      \norm{x}_{\Sigma \overline{X}} = \inf \set{ \norm{x_0}_{X_0} + \norm{x_1}_{X_1} : x_0 + x_1 = x } = 0
    \end{equation*}
    if and only if
    \begin{equation*}
      \norm{x}_{X_0} = \norm{x}_{X_1} = 0.
    \end{equation*}

    \refitem{def:norm/N2} Absolute homogeneity follows from
    \begin{equation*}
      \norm{tx}_{\Sigma \overline{X}}
      =
      \inf \set{ \norm{tx_0}_{X_0} + \norm{tx_1}_{X_1} | x_0 + x_1 = x }
      \reloset {\ref{def:norm/N2}} =
      \abs{t} \norm{x}_{\Sigma \overline{X}}.
    \end{equation*}

    \refitem{def:norm/N3} Subadditivity follows from
    \begin{align*}
      &\phantom{{}={}}
      \norm{x + y}_{\Sigma \overline{X}}
      = \\ &=
      \inf \set{ \left( \norm{x_0}_{X_0} + \norm{x_1}_{X_1} \right) + \left( \norm{y_0}_{X_0} + \norm{y_1}_{X_1} \right) | \substack{\textstyle{x_0 + x_1 = x} \\ \textstyle{y_0 + y_1 = y}} }
      \leq \\ &\leq
      \norm{x}_{\Sigma \overline{X}} + \norm{y}_{\Sigma \overline{X}}.
    \end{align*}
  \end{refenum}

  It remains to prove the completeness of \( \norm{\cdot}_{\Sigma \overline{X}} \). Let \( \{ x_0^{(k)} + x_1^{(k)} \}_{k=1}^\infty \subseteq \Sigma \overline{X} \) be a \hyperref[def:fundamental_net]{fundamental sequence}. Fix \( \varepsilon > 0 \). Then there exists an index \( m_0 \) such that \( k, m \geq k_0 \) implies
  \begin{equation*}
    \norm{x_0^{(k)} + x_1^{(k)} - x_0^{(m)} + x_1^{(m)}}_{\Sigma \overline{X}} < \varepsilon.
  \end{equation*}

  But
  \begin{equation*}
    \norm{x_0^{(k)} - x_0^{(m)}}_{X_0}
    \leq
    \norm{\left(x_0^{(k)} + x_1^{(k)} \right) - \left( x_0^{(m)} + x_1^{(m)} \right)}_{\Sigma \overline{X}}
    <
    \varepsilon,
  \end{equation*}
  hence the sequence \( \{ x_0^{(k)} \}_{k=1}^\infty \) is fundamental. Since \( X_0 \) is complete, this sequence has a limit, which we will denote by \( \xi_0 \). We define \( \xi_1 \) analogously.

  With the same \( \varepsilon \), denote by \( k_0 \) an index such that both \( \norm{x_0^{(k)} - \xi_0}_{X_0} < \tfrac \varepsilon 2 \) and \( \norm{x_1^{(k)} - \xi_1}_{X_1} < \tfrac \varepsilon 2 \) whenever \( k \geq k_0 \).

  Then
  \begin{balign*}
    &\phantom{{}={}}
    \norm{\left( x_0^{(k)} + x_1^{(k)} \right) - \left( \xi_0 + \xi_1 \right)}_{\Sigma \overline{X}}
    = \\ &=
    \inf \set{ \norm{x_0}_{X_0} + \norm{x_1}_{X_1} : x_0 + x_1 = \left( x_0^{(k)} + x_1^{(k)} \right) - \left( \xi_0 + \xi_1 \right) }
    \leq \\ &\leq
    \norm{x_0^{(k)} - \xi_0}_{X_0} + \norm{x_1^{(k)} - \xi_1}_{X_1}
    <
    \tfrac \varepsilon 2 + \tfrac \varepsilon 2
    =
    \varepsilon.
  \end{balign*}

  Therefore, \( \xi_0 + \xi_1 \) is the limit of the sequence \( \{ x_0^{(k)} + x_1^{(k)} \}_{k=1}^\infty \subseteq \Sigma \overline{X} \) in \( \Sigma \overline{X} \).
\end{proof}

\begin{example}\label{thm:lp_interpolation_spaces/definition}
  The spaces \( L^p(\BbbR) \) are interpolation spaces for the pair \( (L^1(\BbbR), L^\infty(\BbbR)) \). The pair is compatible because both are subspaces of the space \( S(\BbbR) \) of all Lebesgue-measurable real function with metric
  \begin{equation*}
    \rho(f, g) \coloneqq \int_{\BbbR} \frac {\abs{f(x) - g(x)}} {1 + \abs(f(x) - g(x))} d\lambda.
  \end{equation*}
\end{example}

\begin{definition}\label{def:lebesgue_space}\cite[6]{BerghLofstrom1976}
  Let \( \mu: U \to [0, \infty] \) be a positive measure and \( p \) be a positive real number. The \term{Lebesgue space} \( L_p \) is defined as the set of bounded functions \( f: U \mapsto \BbbK \) such that the norm
  \begin{equation*}
    \norm{f}_{L_p} \coloneqq \begin{dcases}
      \parens[\Big]{\int_U \abs{f(t)}^p dt}^{1/p}, &0 < p < \infty \\
      \ess\sup_{t \in U} \abs{f(t)} , &p = \infty
    \end{dcases}
  \end{equation*}
\end{definition}

\begin{theorem}[The Riesz-Thorin interpolation theorem]\label{thm:riesz_thorin}\mcite[24]{BerghLofstrom1976}
  Fix two measure spaces \( (U, \mu) \) and \( (V, \nu) \). Let \( T: S(U, \mu) \to S(V, \nu) \) be a continuous linear map between the corresponding spaces of measurable functions.

  Suppose that for some real numbers \( p_0, p_1, q_0, q_1 \geq 1 \) we have
  \begin{equation*}
    T(L^{p_j}(U, \mu)) \subseteq T(L^{q_j}(V, \nu)), j = 0, 1.
  \end{equation*}

  Additionally, let \( \theta \in (0, 1) \) and
  \begin{align*}
    \frac 1 p = \frac {1 - \theta} {p_0} + \frac {\theta} {p_1}
    &&
    \frac 1 q = \frac {1 - \theta} {q_0} + \frac {\theta} {q_1}.
  \end{align*}

  Then
  \begin{equation*}
    T(L^p(U, \mu)) \subseteq L^q(V, \nu)
  \end{equation*}
  and
  \begin{equation*}
    \norm{T}_{\hom(L^p, L^q)} \leq \norm{T}_{\hom(L^{p_0}, L^{q_0})}^{1 - \theta} \norm{T}_{\hom(L^{p_1}, L^{q_1})}^\theta.
  \end{equation*}
\end{theorem}

\begin{definition}\label{def:distribution_function}\cite[6]{BerghLofstrom1976}
  Let \( \mu: U \to [0, \infty] \) be a positive measure and \( p \) be a positive real number.

  \begin{thmenum}
    \thmitem{def:distribution_function/distribution_function} Given a scalar-valued function \( f: U \mapsto \BbbK \), we define its \term{distribution function} as
    \begin{align*}
      &m_f: [0, \infty] \to \BbbK \\
      &m_f(\sigma) \coloneqq \mu(\set{ x :  > \sigma }).
    \end{align*}

    \thmitem{def:distribution_function/rearrangement} We define the \term{decreasing rearrangement} of \( f \) as
    \begin{equation*}
      f^*(t) \coloneqq \inf\set{ \sigma : m_f(\sigma) \leq t }.
    \end{equation*}

    \thmitem{def:distribution_function/lorenz_space} The \( (p, q)-\)Lorenz space, for potentially infinite positive \( q > 0 \), is the set of functions \( f: U \mapsto \BbbK \) for which the quasinorm
    \begin{equation*}
      \norm{f}_{L_{p,q}} \coloneqq \begin{dcases}
        \parens[\Big]{\int_0^\infty \parens[\Big]{\frac {f^*(\tau)} {\tau^p}}^q \frac {d t} t}^{\frac 1 q}, &1 \leq q < \infty \\
        \ess\sup\parens[\Big]{\frac {f^*(t)} {t^p}}, &q = \infty
      \end{dcases}
    \end{equation*}
    is finite.

    In particular, when \( q = \infty \), we use the notation
    \begin{equation*}
      \norm{f}_{L^{p*}} \coloneqq \parens[\Big]{p \int_0^\infty \sigma^p m_f(\sigma) \frac {d \sigma} \sigma}^{\frac 1 p}
    \end{equation*}
  \end{thmenum}
\end{definition}

\begin{theorem}[The Marcinkiewicz interpolation theorem]
  Fix two measure spaces \( (U, \mu) \) and \( (V, \nu) \). Let \( T: S(U, \mu) \to S(V, \nu) \) be a continuous linear map between the corresponding spaces of measurable functions.

  Suppose that for some real numbers \( p_0, p_1, q_0, q_1 \geq 1 \) we have
  \begin{equation*}
    T(L^{p_j}(U, \mu)) \subseteq T(L^{q_j *}(V, \nu)), j = 0, 1.
  \end{equation*}

  Additionally, let \( \theta \in (0, 1) \) and
  \begin{align*}
    \frac 1 p = \frac {1 - \theta} {p_0} + \frac {\theta} {p_1}
    &&
    \frac 1 q = \frac {1 - \theta} {q_0} + \frac {\theta} {q_1}.
  \end{align*}

  Then, if \( p \leq q \),
  \begin{equation*}
    T(L^p(U, \mu)) \subseteq L^q(V, \nu)
  \end{equation*}
  and
  \begin{equation*}
    \norm{T}_{\hom(L^p, L^q)} \leq C_\theta \norm{T}_{\hom(L^{p_0}, L^{q_0*})}^{1 - \theta} \norm{T}_{\hom(L^{p_1}, L^{q_1*})}^\theta
  \end{equation*}
  for some constant \( C_\theta \).
\end{theorem}

\begin{definition}\label{def:banach_interpolation_space_exponent}\mcite[27]{BerghLofstrom1976}
  Let \( \overline{X} \coloneqq ( X_0, X_1 ) \) and \( \overline{Y} \coloneqq ( Y_0, Y_1 ) \) be compatible pairs of Banach spaces. If \( X \) and \( Y \) are a pair of interpolation spaces and, additionally, the inequality
  \begin{equation}\label{da:def:banach_interpolation_space_exponent}
    \norm{T}_{\hom(X, Y)} \leq C \norm{T}_{\hom(X_0, Y_0)}^{1-\theta} \cdot \norm{T}_{\hom(X_1, Y_1)}^{\theta}
  \end{equation}
  holds for some constant \( C > 0 \) and \( \theta \in [0, 1] \), we say that the pair \( (X, Y) \) are \term{interpolation spaces of exponent} \( \theta \).

  If, additionally, \( C = 1 \), we say that \( (X, Y) \) is an \term{exact pair} of interpolation spaces.
\end{definition}

\begin{definition}\label{def:k_functional}\mcite[38]{BerghLofstrom1976}
  Let \( \overline{X} \coloneqq ( X_0, X_1 ) \) be a compatible pair of Banach spaces. Instead of the norm \( \norm{\cdot}_{X_1} \) in \( X_1 \), we can consider \hyperref[def:equivalent_metrics]{equivalent norms} of the type \( t\norm{\cdot}_{X_1} \) for \( t \geq 0 \). Furthermore, we can also introduce equivalent norms in \( \Sigma \overline{X} \) via the \term{\( K \)-functional}
  \begin{equation}\label{eq:def:k_functional}
    \begin{aligned}
      &K: (0, \infty) \times {\Sigma \overline{X}} \\
      &K(t, x) \coloneqq \inf \set{ \norm{x_0}_{X_0} + t\norm{x_1}_{X_1} : x_0 + x_1 = x }.
    \end{aligned}
  \end{equation}

  See \fullref{def:k_functional_properties/equivalent_norm} for a proof that \( x \mapsto K(t, x) \) for a fixed \( t \geq 0 \) is an equivalent norm in the sum \( \Sigma \overline{X} \).
\end{definition}

\begin{proposition}\label{def:k_functional_properties}\mcite[38]{BerghLofstrom1976}
  The \hyperref[def:k_functional]{\( K \)-functional} has the following basic properties:

  \begin{thmenum}
    \thmitem{def:k_functional_properties/basic} For any fixed \( x \in \Sigma \overline{X} \), the function \( t \mapsto K(t, x) \) is positive, \hyperref[def:partially_ordered_set/homomorphism]{monotone} and \hyperref[def:convex_functions]{concave}.

    \thmitem{def:k_functional_properties/inequality} For positive real numbers \( t, s > 0 \), we have the following inequality:
    \begin{equation}\label{eq:def:k_functional_properties/inequality}
      K(t, x) \leq \max\set{1, \frac t s} K(s, x).
    \end{equation}

    \thmitem{def:k_functional_properties/equivalent_norm} For any fixed \( t > 0 \), the function \( x \mapsto K(t, x) \) is an \hyperref[def:equivalent_metrics]{equivalent norm} in the sum \( \Sigma \overline{X} \).
  \end{thmenum}
\end{proposition}
\begin{proof}
  \SubProofOf{def:k_functional_properties/basic} That \( t \mapsto K(t, x) \) is positive is a slight generalization of \fullref{def:norm/N1}, which can be proved as in \fullref{def:banach_space_sum_and_intersection_norms/sum}.

  Monotonicity follows from the monotonicity of the infimum.

  To see that \( t \mapsto K(t, x) \) is concave, fix \( x \), \( \lambda \in [0, 1] \) and \( t, s > 0 \). We have
  \begin{align*}
    &\phantom{{}={}}
    K(\lambda t + (1 - \lambda) s, x)
    = \\ &=
    \inf \set{ \norm{x_0}_{X_0} + (\lambda t + (1 - \lambda) s)\norm{x_1}_{X_1} | x_0 + x_1 = x }
    = \\ &=
    \inf \set{ \lambda \left(\norm{x_0}_{X_0} + t \norm{x_1}_{X_1} \right) + (1 - \lambda) \left(\norm{x_0}_{X_0} + s \norm{x_1}_{X_1} \right) | x_0 + x_1 = x }
    \geq \\ &\geq
    \lambda K(t, x) + (1 - \lambda) K(s, x).
  \end{align*}

  \SubProofOf{def:k_functional_properties/inequality} Fix positive real numbers \( t, s > 0 \).
  \begin{itemize}
    \item If \( t \leq s \), by monotonicity we have
    \begin{equation}\label{eq:def:k_functional_properties/inequality/monotonicity}
      K(t, x) \leq K(s, x)
    \end{equation}

    \item If \( t > s \), we use concavity with
    \begin{equation*}
      s = \frac s t t + \left(1 - \frac s t \right) 0
    \end{equation*}
    to obtain
    \begin{equation*}
      K(s, x) \geq \frac s t K(t, x) + \left(1 - \frac s t \right) K(0, x).
    \end{equation*}

    By positivity of \( K \), we have \( K(t, x) = 0 \) if and only if \( t = 0 \), hence
    \begin{equation}\label{eq:def:k_functional_properties/inequality/concavity}
      K(t, x) \leq \frac t s K(s, x).
    \end{equation}
  \end{itemize}

  Combining \eqref{eq:def:k_functional_properties/inequality/monotonicity} and \eqref{eq:def:k_functional_properties/inequality/concavity}, we obtain \eqref{eq:def:k_functional_properties/inequality}.

  \SubProofOf{def:k_functional_properties/equivalent_norm} That \( x \mapsto K(t, x) \) for a fixed \( t > 0 \) is a slight generalization of the proof in \fullref{def:banach_space_sum_and_intersection_norms/sum}.

  That the norms \( \norm{\cdot}_{\Sigma \overline{X}} \) and \( K(t, \cdot) \) are equivalent follows from \eqref{eq:def:k_functional_properties/inequality} with \( s = 1 \) for the upper bound and \( t = 1, s = t \) for the lower bound. That is,
  \begin{equation*}
    \min\set{1, t} \underbrace{K(1, x)}_{\norm{x}_{\Sigma \overline{X}}} \leq K(t, x) \leq \max\set{1, t} \underbrace{K(1, x)}_{\norm{x}_{\Sigma \overline{X}}}.
  \end{equation*}
\end{proof}

\begin{example}\label{thm:lp_interpolation_spaces/k_functional}
  The \hyperref[def:k_functional]{\( K \)-functional} for the pair \( (L_1(\BbbR), L_\infty(\BbbR)) \) from \fullref{thm:lp_interpolation_spaces/definition} is
  \begin{equation*}
    K(t, f) \coloneqq \int_0^t f^*(\tau) d\tau.
  \end{equation*}
\end{example}

\begin{definition}\label{def:lorenz_quasinorm}
  For \( \theta \in \BbbR \), \( q \in (0, \infty] \) and nonnegative functions \( g: [0, \infty) \to [0, \infty] \) we define
  \begin{equation}\label{eq:def:lorenz_quasinorm}
    \Phi_{\theta,q}(g) \coloneqq \begin{dcases}
      \left( \int_0^\infty \left( \frac {g(\tau)} {\tau^\theta} \right)^q \frac {d\tau} \tau \right)^{\tfrac 1 q}, &0 < q < \infty \\
      \ess\sup_{t \geq 0} \left( \frac {g(t)} {t^\theta} \right),                                                &q = \infty
    \end{dcases}
  \end{equation}
  and
  \begin{equation}\label{eq:def:lorenz_quasinorm/gamma}
    \gamma_{\theta,q} \coloneqq \Phi_{\theta,q}(\min(t, 1)).
  \end{equation}
\end{definition}

\begin{proposition}\label{thm:def:lorenz_quasinorm}
  The function \hyperref[def:lorenz_quasinorm]{\( \Phi_{\theta,q} \)} has the following basic properties:

  \begin{thmenum}
    \thmitem{thm:def:lorenz_quasinorm/reciprocal} For \( s > 0 \) and \( h(t) \coloneqq g(\tfrac t s) \) we have
    \begin{equation}\label{eq:thm:def:lorenz_quasinorm/reciprocal}
      \Phi_{\theta,q}(h) = \frac 1 {s^{\theta}} \Phi_{\theta,q}(g).
    \end{equation}

    \thmitem{thm:def:lorenz_quasinorm/gamma} For finite \( q \) we have
    \begin{equation}\label{eq:thm:def:lorenz_quasinorm/gamma}
      \gamma_{\theta,q} = \left( \frac 1 {q \theta (1 - \theta)} \right)^{\tfrac 1 q}.
    \end{equation}
  \end{thmenum}
\end{proposition}
\begin{proof}
  \SubProofOf{thm:def:lorenz_quasinorm/reciprocal} The case \( q = \infty \) is obvious. For \( 0 < q < \infty \), we have
  \begin{balign*}
    \Phi_{\theta,q}(h)
    &=
    \left( \int_0^\infty \left( \frac {h(\tau)} {\tau^\theta} \right)^q \frac {d\tau} \tau \right)^{\tfrac 1 q}
    = \\ &=
    \left( \frac 1 {s^{\theta q}} \int_0^\infty \left( \frac {g(\tfrac \tau s)} {\left(\tfrac \tau s \right)^\theta} \right)^q \frac {d{\tfrac \tau s}} {\tfrac \tau s} \right)^{\tfrac 1 q}
    = \\ &=
    \frac 1 {s^{\theta}} \Phi_{\theta,q}(g).
  \end{balign*}

  \SubProofOf{thm:def:lorenz_quasinorm/gamma} We can raise \( \gamma_{\theta,q} \) to the \( q \)-th power for brevity of notation:
  \begin{balign*}
    \gamma_{\theta,q}^q
    &=
    \Phi_{\theta,q}(\min(t, 1))^q
    = \\ &=
    \int_0^1 \left( \frac {\tau} {\tau^\theta} \right)^q \frac {d\tau} \tau + \int_1^\infty \left( \frac {1} {\tau^\theta} \right)^q \frac {d\tau} \tau
    = \\ &=
    \int_0^1 \tau^{(1 - \theta) q - 1} d\tau + \int_1^\infty \tau^{-\theta q - 1} d\tau
    = \\ &=
    \frac {1 - 0} {(1 - \theta) q} + \frac {\lim_{\tau \to \infty} \tau^{-\theta q} - 1} {-\theta q}
    = \\ &=
    \frac 1 {(1 - \theta) q} - \frac 1 {-\theta q}
    = \\ &=
    \frac {-\theta q - (1 - \theta) q} {(1 - \theta) (-\theta) q^2}
    = \\ &=
    \frac 1 {(1 - \theta) \theta q}
  \end{balign*}
\end{proof}

\begin{definition}\label{def:k_functional_interpolation_space}\mcite[40]{BerghLofstrom1976}
  Let \( \overline{X} \coloneqq ( X_0, X_1 ) \) be a compatible pair of Banach spaces.

  For \( \theta \in (0, \infty) \), \( q \in (0, \infty] \), we introduce the following norm:
  \begin{equation}\label{eq:def:k_functional_interpolation_space/norm}
    \norm{x}_{\theta,q,K} \coloneqq \Phi_{\theta,q}(K(t, x)).
  \end{equation}

  The subspace of \( \Sigma\overline{X} \) for which this norm is finite is denoted by either
  \begin{align*}
    K_{\theta,q}(\overline{X})
    &&
    X_{\theta,q,K}.
  \end{align*}
\end{definition}

\begin{theorem}\label{thm:k_functional_interpolation}\mcite[thm. 3.1.2]{BerghLofstrom1976}
  Let \( \theta \in (0, 1) \) and \( q \in (0, \infty] \). The space \( X_{\theta,q,K} \) defined in \fullref{eq:def:k_functional_interpolation_space/norm} is an \hyperref[def:banach_interpolation_space_exponent]{exact interpolation space} of exponent \( \theta \). Furthermore,
  \begin{equation}\label{eq:thm:k_functional_interpolation/inequality}
    K(s, x) \leq (\gamma_{\theta,q})^{-1} s^\theta \norm{x}_{\theta,q,K}.
  \end{equation}
\end{theorem}
\begin{proof}
  Note that \( K(s, x) \) is a norm on \( \Sigma \overline{X} \) by \fullref{def:k_functional_properties/equivalent_norm}. Therefore, \( \norm{\cdot}_{\theta,q,K} \), the composition of \( K(s, x) \) with the \hyperref[def:lorenz_quasinorm]{Lorenz quasinorm} \( \Phi_{\theta,q} \), is a norm.

  We denote by \( X_{\theta,q,K} \) the space consisting of all vectors from \( \Sigma \overline{X} \) where the norm \eqref{eq:def:k_functional_interpolation_space/norm} is finite.

  From \eqref{eq:def:k_functional_properties/inequality} it follows that
  \begin{equation*}
    \min(1, \tfrac t s) K(s, x) \leq K(t, x)
  \end{equation*}
  and hence
  \begin{equation*}
    \underbrace{\Phi_{\theta,q}}_{\text{depends on } t} \parens[\Big]{ \min(1, \tfrac t s) K(s, x) } \leq \underbrace{\Phi_{\theta,q}(K(s, x))}_{\text{norm in } X_{\theta,q,K}}.
  \end{equation*}

  By \eqref{eq:thm:def:lorenz_quasinorm/reciprocal}, we have
  \begin{equation*}
    \Phi_{\theta,q}(\min(1, \tfrac t s)) = \tfrac 1 {s^\theta} \underbrace{\Phi_{\theta,q}(\min(1, t))}_{\hyperref[eq:def:lorenz_quasinorm/gamma]{\gamma_{\theta,q}}},
  \end{equation*}
  and \eqref{eq:thm:k_functional_interpolation/inequality} follows.

  It remains to show that \( X_{\theta,q,K} \) is an exact interpolation space of exponent \( \theta \).

  Note that \( K(1, x) = \norm{x}_{\Sigma \overline{X}} \) and thus \eqref{eq:thm:k_functional_interpolation/inequality} with \( s = 1 \) implies that
  \begin{equation*}
    \gamma_{\theta,q} \norm{x}_{\Sigma \overline{X}} \leq \norm{x}_{\theta,q,K},
  \end{equation*}
  which shows that \( X_{\theta,q,K} \) can be embedded continuously into \( \Sigma \overline{X} \).

  On the other hand, for \( x \in \Delta \overline{X} \) we have
  \begin{equation*}
    K(t, x) \leq \norm{x} \leq \norm{x}_{\Delta \overline{X}} \T{since} x = x + 0
  \end{equation*}
  and
  \begin{equation*}
    K(t, x) \leq \norm{x} \leq t \norm{x}_{\Delta \overline{X}} \T{since} x = 0 + x.
  \end{equation*}

  Therefore,
  \begin{equation*}
    K(t, x) \leq \min(1, t) \norm{x}_{\Delta \overline{X}},
  \end{equation*}
  which after applying \( \Phi_{\theta,q} \) becomes
  \begin{equation*}
    \norm{x}_{\theta,q,K} \leq \gamma_{\theta,q} \norm{x}_{\Delta \overline{X}}.
  \end{equation*}

  Hence, we have the chain of continuous linear inclusions of Banach spaces
  \begin{equation*}
    \Delta \overline{X} \subseteq X \subseteq \Sigma \overline{X}.
  \end{equation*}

  Finally, to show that \( X \) is an interpolation space of exponent \( \theta \), fix a linear operator \( T: \overline{X} \mapsto \overline{CY} \) between compatible pairs and let \( Y \) be an intermediate space for \( \overline{CY} \).

  Then
  \begin{align*}
    K(t, Tx)_{\overline{CY}}
    &\leq
    \inf \set{ \norm{y_0}_{Y_0} + t \norm{y_1}_{Y_1} : y_0 + y_1 = Tx }
    \leq \\ &\leq
    \inf \set{ \norm{T}_{\hom(X_0, Y_0)} \norm{x_0}_{X_0} + t \norm{T}_{\hom(X_1, Y_1)} \norm{x_1}_{Y_1} : x_0 + x_1 = x }
    = \\ &=
    \norm{T}_{\hom(X_0, Y_0)} K\parens[\Bigg]{\frac {\norm{T}_{\hom(X_1, Y_1)}} {\norm{T}_{\hom(X_0, Y_0)}} t, x}.
  \end{align*}

  By applying \( \Phi_{\theta,q} \) to both sides and using \eqref{eq:thm:def:lorenz_quasinorm/reciprocal}, we obtain
  \begin{equation*}
    \norm{Tx}_{\overline{Y}_{\theta,q,K}}
    \leq
    {\norm{T}_{\hom(X_1, Y_1)}}^{1 - \theta} {\norm{T}_{\hom(X_0, Y_0)}}^{\theta} \norm{x}_{\overline{Y}_{\theta,q,K}}.
  \end{equation*}

  Thus, \( X \) satisfies \fullref{def:banach_interpolation_space_exponent} for being an exact interpolating space with exponent \( \theta \).
\end{proof}

\begin{definition}\label{def:discrete_k_interpolation_space}
  For positive numbers \( \theta \) and \( q \), we denote by \( \lambda^{\theta,q} \) the set of all doubly-infinite real sequences \( \seq{ x_k }_{k=-\infty}^\infty \) such that the norm
  \begin{equation}\label{eq:def:discrete_k_interpolation_space}
    \norm{\seq{ x_k }_{k=-\infty}^\infty}_{\lambda^{\theta,q}} \coloneqq \left( \sum_{k=-\infty}^\infty \left( \frac {\abs{x_k}} {2^{k\theta}} \right)^q \right)^{\tfrac 1 q}
  \end{equation}
  is finite.
\end{definition}

\begin{theorem}\label{thm:discrete_k_interpolation}\mcite[lemma 3.1.3]{BerghLofstrom1976}
  The vector \( x \in \Sigma\overline{X} \) belongs to \hyperref[def:k_functional_interpolation_space]{\( X_{\theta,q,K} \)} if and only if the sequence \( \seq{ x_k }_{k=-\infty}^\infty \) defined as
  \begin{equation}\label{eq:thm:discrete_k_interpolation/sequence}
    x_k \coloneqq K(2^k, x)
  \end{equation}
  belongs to \hyperref[def:discrete_k_interpolation_space]{\( \lambda^{\theta,q} \)}.

  Furthermore, for any integer \( k \) the following inequalities hold:
  \begin{equation}\label{eq:thm:discrete_k_interpolation/inequalities}
    \frac 1 {2^\theta} \ln 2 \norm{x_k}_{\lambda^{\theta,q}}
    \leq
    \norm{x}_{\theta,q,K}
    \leq
    2 \cdot \ln 2 \norm{x_k}_{\lambda^{\theta,q}}.
  \end{equation}
\end{theorem}
\begin{proof}
  We have
  \begin{equation*}
    \norm{x}_{\theta,q,K}^q
    =
    \int_0^\infty \left( \frac {K(\tau, x)} {\tau^\theta} \right)^q \frac {d\tau} \tau
    =
    \sum_{k=-\infty}^\infty \int_{2^k}^{2^{k+1}} \left( \frac {K(\tau, x)} {\tau^\theta} \right)^q \frac {d\tau} \tau.
  \end{equation*}

  By \fullref{def:k_functional_properties/inequality}, for each integer \( k \),
  \begin{equation*}
    K(2^k, x) \leq 2 K(2^k, x).
  \end{equation*}

  By the \hyperref[def:k_functional_properties/basic]{monotonicity} of \( K \), for \( t \in [2^k, 2^{k+1}] \) we have
  \begin{equation*}
    K(2^k, x) \leq K(t, x) \leq 2 K(2^k, x).
  \end{equation*}

  Denote \( x_k \coloneqq K(2^k, x) \). For \( 2^k \leq t \leq 2^{k+1} \) we have
  \begin{equation*}
    \frac{x_k}{2^{(k+1)\theta}} \leq \frac{K(t, x)}{t^\theta} \leq 2 \frac{x_k}{2^{k\theta}}
  \end{equation*}

  Therefore,
  \begin{align*}
    \norm{x}_{\theta,q,K}^q
    &=
    \sum_{k=-\infty}^\infty \int_{2^k}^{2^{k+1}} \left( \frac {K(\tau, x)} {\tau^\theta} \right)^q \frac {d\tau} \tau
    \leq \\ &\leq
    2^q \sum_{k=-\infty}^\infty \left(\frac{x_k}{2^{k\theta}} \right)^q \cdot \ln \tau \mid_{\tau=2^k}^{2^{k+1}}
    = \\ &=
    \ln 2 \cdot 2^q \sum_{k=-\infty}^\infty \left(\frac{x_k}{2^{k\theta}} \right)^q
    = \\ &=
    2^q \ln 2 \norm{\seq{ x_k }_{k=-\infty}^\infty}_{\lambda^{\theta,q}}^q
  \end{align*}
  and similarly for the lower bound.
\end{proof}

\begin{definition}\label{def:e_functional}\mcite[174]{BerghLofstrom1976}
  Let \( \overline{X} = (X_0, X_1) \) be a compatible pair of Banach spaces. Let \( x \in \Sigma \overline{X} \). Put
  \begin{equation}\label{eq:def:e_functional}
    \begin{aligned}
      &E: (0, \infty) \times {\Sigma \overline{X}} \\
      &E(t, x) \coloneqq \inf \set{ \norm{x - x_0}_{X_1} : \norm{x_0}_{X_0} \leq t }.
    \end{aligned}
  \end{equation}
\end{definition}

\begin{proposition}\label{thm:def:e_functional}\mcite[lemma 7.1.3]{BerghLofstrom1976}
  When \( \overline{X} \) are quasi-Banach spaces, the \hyperref[def:e_functional]{\( E \)-functional} has the following basic properties:

  \begin{thmenum}
    \thmitem{def:k_functional_properties/decreasing} For fixed \( x \in \Sigma\overline{X} \), the function \( t \mapsto E(t, x) \) is decreasing.

    \thmitem{def:k_functional_properties/subaditive} For \( \varepsilon \in (0, 1) \), we have
    \begin{equation*}
      E(t, x + y) \leq E(\varepsilon t, x) + E((1 + \varepsilon) t, y).
    \end{equation*}

    \thmitem{def:k_functional_properties/positive} \( x = 0 \) if and only if \( E(t, x) = 0 \) for all \( t > 0 \).

    \thmitem{def:k_functional_properties/k_functional_connection}\mcite[thm. 7.1.4]{BerghLofstrom1976}
    \begin{equation*}
      E(t, x) = \sup \set{ \frac {K(s, x) - t} s : s > 0 }.
    \end{equation*}
  \end{thmenum}
\end{proposition}

\begin{definition}\label{def:approximation_space}\mcite[def. 7.1.5]{BerghLofstrom1976}
  Let \( \overline{X} = (X_0, X_1) \) be a compatible pair of Banach spaces. We define an \term{approximation space} \( E_{\alpha,q}(\overline{X}) \) for \( x \in \Sigma\overline{X} \) as the space of all members of \( \Sigma\overline{X} \) for which the following norm
  \begin{equation}\label{eq:def:approximation_space/norm}
    \norm{x}_{\alpha,q,E} \coloneqq \Phi_{-\alpha,q}(E(t,a))
  \end{equation}
  is finite.

  Here \( \alpha \) and \( q \) are both positive real numbers and \( q \) is potentially \( \infty \).
\end{definition}

\begin{theorem}\label{thm:interpolation_space_and_approximation_space}\mcite[thm. 7.1.6]{BerghLofstrom1976}
  Let \( X \) be a compatible pair of Banach spaces. Let \( \alpha \) and \( q \) be positive real numbers and define
  \begin{align*}
    \theta \coloneqq \frac 1 {\alpha + 1},
    &&
    r \coloneqq \theta q.
  \end{align*}

  Then
  \begin{equation*}
    (E_{\alpha,\theta q}(\overline{X}))^\theta = K_{\theta,q}(\overline{X}).
  \end{equation*}
\end{theorem}

\begin{theorem}\label{thm:interpolation_space_and_approximation_space_reiteration}\mcite[thm. 7.1.8]{BerghLofstrom1976}
  Let \( X \) be a compatible pair of Banach spaces. Let \( \theta, \alpha_0, \alpha_1, r_0, r_1 \) and \( q \) be positive real numbers such that \( \alpha_0 \neq \alpha_1 \) and define \( r \coloneqq \theta q \) and
  \begin{align*}
    \alpha \coloneqq (1 - \theta) \alpha_0 + \theta \alpha_1,
    &&
    \beta \coloneqq - \frac {\alpha_1 - \alpha} {\alpha_0 - \alpha}.
  \end{align*}

  Then
  \begin{equation*}
    K_{\theta,q}(E_{\alpha_0,r_0}(\overline{X}), E_{\alpha_1,r_1}(\overline{X})) = E_{\alpha,q}(\overline{X})
  \end{equation*}
  and
  \begin{equation*}
    E_{\beta,r}(E_{\alpha_0,r_0}(\overline{X}), E_{\alpha_1,r_1}(\overline{X}))^\theta = E_{\alpha,q}(\overline{X}).
  \end{equation*}
\end{theorem}


  \section{Approximation theory}\label{sec:approximation_theory}

Approximation theory studies how real-valued functions can be approximated by more well-behaved functions. In its modern form, these include inequalities and optimization in function spaces.

  \section{Lagrange polynomials}\label{sec:lagrange_polynomials}

\begin{definition}\label{def:omega_polynomial}
  Given distinct elements \( x_0, \ldots, x_n \) of the field \( \BbbK \), we form the polynomial
  \begin{equation*}
    \omega(X) \coloneqq \prod_{k=0}^n (X - x_j).
  \end{equation*}
\end{definition}

\begin{proposition}\label{thm:omega_polynomial_derivative}
  For the polynomial \( \omega \) from \fullref{def:omega_polynomial}, for \( k = 0, \ldots, n \) we have
  \begin{equation*}
    \omega'(x_j) = \prod_{\substack{j = 0 \\ j \neq k}}^n (x_j - x_k),
  \end{equation*}
  where \( \omega' \) is the \hyperref[def:algebraic_derivative]{algebraic derivative} of \( \omega \).
\end{proposition}
\begin{proof}
  Fix \( k \in \{ 0, \ldots, n \} \) and denote
  \begin{equation*}
    q(X) \coloneqq \prod_{\substack{j = 0 \\ j \neq k}}^n (X - x_j).
  \end{equation*}

  Then
  \begin{equation*}
    \omega(X) = (X - x_k) q(X)
  \end{equation*}
  so
  \begin{equation*}
    \omega'(X) = [q(X) + X q'(X)] - x_k q'(X) = q(X) + (X - x_k) q'(X).
  \end{equation*}

  So for \( x_k \) we have
  \begin{equation*}
    \omega'(x_k) = q(x_k) = \prod_{\substack{j = 0 \\ j \neq k}}^n (x_k - x_j).
  \end{equation*}
\end{proof}

\begin{theorem}[Lagrange interpolation]\label{thm:lagrange_interpolation}
  Let \( x_0, x_1, \ldots, x_n \) be pairwise distinct elements of \( \BbbK \) and let \( y_0, y_1, \ldots, y_n \) be arbitrary elements of \( \BbbK \). Then there exists a unique \hyperref[def:polynomial_algebra/polynomials]{polynomial} \( L(X) \) of degree at most \( n \) such that \( L(x_k) = y_k \) for \( k = 1, \ldots, n \).
\end{theorem}
\begin{proof}
  \UniquenessSubProof Suppose that \( p, q \) are polynomials of degree at most \( n \) that both satisfy \( L(x_k) = y_k \) for \( k = 1, \ldots, n \). Their difference \( p - q \) is a polynomial of degree at most \( n \) that has \( n + 1 \) roots. By \fullref{thm:def:integral_domain/root_limit}, \( p - q = 0 \).

  \ExistenceSubProof We will construct the polynomial explicitly. Define the Lagrange basis polynomial
  \begin{equation*}
    L(X) = \sum_{m=0}^n y_m \prod_{\substack{j = 0 \\ j \neq m}}^n \frac {(X - x_j)} {(x_m - x_j)}.
  \end{equation*}

  For \( k = 0, 1, \ldots, n \) we have
  \begin{equation*}
    L(x_k) = y_k \underbrace{\prod_{\substack{j = 0 \\ j \neq k}}^n \frac {(x_k - x_j)} {(x_k - x_j)}}_{1} + \sum_{\substack{m = 0 \\ m \neq k}}^n y_m \overbrace{\frac{(x_k - x_m)}{(x_k - x_m)}}^{0} \prod_{\substack{j = 0 \\ j \neq k \\ j \neq m}}^n \frac {(x_k - x_j)} {(x_m - x_j)} = y_k.
  \end{equation*}

  Therefore, \( L \) is the desired polynomial.
\end{proof}

\begin{theorem}[Finite field Lagrange interpolation]\label{thm:finite_field_lagrange_interpolation}\mcite{MathOF:functions_over_finite_fields}
  For a multivariate function \( f: \BbbF_q^n \to \BbbF_q \) over the \hyperref[def:finite_field]{finite field} \( \BbbF_q \), there exists a unique multivariate polynomial \( L(X_1, \ldots, X_n) \) such that
  \begin{itemize}
    \item For any sequence of values \( x_1, \ldots, x_n \),
    \begin{equation*}
      f(x_1, \ldots, x_n) = L(x_1, \ldots, x_n).
    \end{equation*}

    \item For every monomial \( X_1^{\gamma_n} \cdots X_n^{\gamma_n} \) of \( L \), \( \gamma_i < q \) for \( i = 1, \ldots, n \).
  \end{itemize}
\end{theorem}
\begin{proof}
  For any point \( (x_1, \ldots, x_n) \in \BbbZ_q^n \), the characteristic polynomial
  \begin{equation*}
    c(X_1, \ldots, X_n) \coloneqq \prod_{i=0}^n \parens*{ \prod_{\substack{m=0 \\ m \neq x_i}}^{q - 1} \frac {X_i - m} {x_i - m} }
  \end{equation*}
  satisfies
  \begin{equation*}
    c(y_1, \ldots, y_n) = \begin{cases}
      1, &x_i = y_i \T{for all} i = 1, \ldots, n \\
      0, &\T{otherwise.}
    \end{cases}
  \end{equation*}

  As in \fullref{thm:lagrange_interpolation}, give us the desired polynomial, a linear combination of these basis polynomials with coefficients corresponding to the values of \( f \) give us the desired polynomial.
\end{proof}

  \section{Bernstein inequalities}\label{sec:bernstein_inequalities}

\begin{definition}\label{def:real_function_space_operators}
  Consider the \hyperref[def:function]{operator} \( T: C([a, b]) \to C([a, b]) \).

  \begin{thmenum}
    \thmitem{def:real_function_space_operators/positive} If \( f([a, b]) \subseteq [0, \infty) \) implies \( T(f)([a, b]) \subseteq [0, \infty) \), we say that \( T \) is \term{positive}.

    \thmitem{def:real_function_space_operators/monotone} If \( f(x) \leq g(x) \) for all \( x \in [a, b] \) implies \( T(f)(x) \leq T(g)(x) \) for all \( x \in [a, b] \), we say that \( T \) is \term{monotone}.
  \end{thmenum}
\end{definition}

\begin{definition}\label{def:periodic_function_space}\mcite[44]{Николов2020АпроксимацииЛекции}
  We denote by \( \tilde{C}([a, b]) \) the subspace of \( C([a, b]) \) consisting of all continuous functions in \( [a, b] \) which are periodic with minimal period \( b - a \).
\end{definition}

\begin{definition}\label{def:approximation_error}\mcite[44]{Николов2020АпроксимацииЛекции}
  We introduce two operators.

  \begin{thmenum}
    \thmitem{def:approximation_error/algebraic} The \term{algebraic approximation error}
    \begin{balign*}
      E_n: C([a, b]) \to [0, \infty] \\
      E_n(f) \coloneqq \inf_{p \in \pi_n} \norm{f - p}.
    \end{balign*}

    \thmitem{def:approximation_error/trigonometric} The \term{trigonometric approximation error}
    \begin{balign*}
      \tilde{E}_n: C([a, b]) \to [0, \infty] \\
      \tilde{E}_n(f) \coloneqq \inf_{p \in \tau_n} \norm{f - p}.
    \end{balign*}
  \end{thmenum}
\end{definition}

\begin{theorem}[Jackson's trigonometric theorem]\label{thm:jacksons_trigonometric_theorem}\mcite[47]{Николов2020АпроксимацииЛекции}
  For \( f \in \tilde{C}[-\pi, \pi] \) we have
  \begin{equation*}
    \tilde{E}_n(f) \leq \frac {6^{k+1}} {n^k} \omega\left(f^{(k)}, \frac 1 n \right).
  \end{equation*}
\end{theorem}

\begin{theorem}[Szego’s inequality]\label{thm:szegos_trigonometric_inequality}\mcite[55]{Николов2020АпроксимацииЛекции}
  For any nonnegative integer \( n \) and any \( s \in S_{\tau_n} \), we have
  \begin{equation}\label{eq:thm:szegos_trigonometric_inequality}
    [s'(\theta)]^2 + n^2 s^2 (\theta) \leq n^2 \quad\forall \theta \in [-\pi, \pi].
  \end{equation}
\end{theorem}
\begin{proof}
  Fix \( n = 1, 2, \ldots \) and \( \alpha \in [-1, 1] \).

  For brevity, denote \( c(\theta) \coloneqq \cos(n \theta) \). Let \( \theta_s \) and \( \theta_c \) be numbers in \( [-\pi, \pi] \) such that
  \begin{equation*}
    s(\theta_s) + c(\theta_c) = \alpha.
  \end{equation*}

  We will show that
  \begin{equation}\label{eq:thm:szegos_trigonometric_inequality/abs}
    \abs{s'(\theta_s)} \leq \abs{c'(\theta_c)}
  \end{equation}

  This will, in turn, imply that
  \begin{equation*}
    [s'(\theta_s)]^2
    \leq
    [c'(\theta_c)]^2
    =
    n^2 [\sin(n \theta_c)]^2
    \reloset {\ref{thm:trigonometric_identities/pythagorean_identity}}
    =
    n^2 [1 - \cos(n \theta_c)^2]
    =
    n^2 [1 - s(\theta_s)]
  \end{equation*}
  which is equivalent to \eqref{eq:thm:szegos_trigonometric_inequality}.

  Now we will prove \eqref{eq:thm:szegos_trigonometric_inequality/abs}. If \( \theta_c \) is a critical point of \( c \), i.e. if \( r'(\theta_c) = 0 \), then \( c'(\theta_c) = s'(\theta_s) \) and \eqref{eq:thm:szegos_trigonometric_inequality/abs} holds. Suppose that \( \theta_c \) is not a critical point. Denote by
  \begin{equation*}
    \theta_n \coloneqq \tfrac k n \pi, k = -n, -n+1, \ldots, n-2, n-1.
  \end{equation*}
  the extrema of \( c(\theta) \) in \( [-\pi, \pi) \).

  Define the auxiliary function
  \begin{equation*}
    r(\theta) \coloneqq c(\theta) - s(\theta - \theta_c + \theta_s).
  \end{equation*}

  We now have \( r(\theta_c) = c(\theta_c) - s(\theta_s) = 0 \).

  Furthermore, since \( \norm{s} = 1 \), then \( \abs{s(\theta)} \leq 1 \) for all \( \theta \in [-\pi, \pi) \). Therefore, \( r(\theta_k) \leq 0 \) for all odd \( k \) and \( r(\theta_k) \geq 0 \) for all even \( k \).

  If \( \theta_c \) coincides with any of the extrema \( \theta_k \), then \( r(\theta_c) \) holds trivially. Suppose that \( \theta_c \) is between \( \theta_{k-1} \) and \( \theta_k \). Without loss of generality, assume that \( k \) is even.

  By the intermediate value theorem, there exists a zero of \( r \) between \( r(\theta_c) \) and \( r(\theta_k) \). If \( r(\theta_c) < 0 \), then \( \theta_c \) is a local minimum and hence there exists a point \( \theta_{c'} \) between \( r(\theta_{k-1}) \) and \( \theta_c \) such that \( r(\theta_{c'}) = r(\theta_c) = 0 \). But this would imply that \( r \) has more than \( 2n \) different roots in the interval \( [-\pi, \pi) \), which is a contradiction.
\end{proof}

\begin{corollary}[Bernstein's trigonometric inequality]\label{thm:bernsteins_trigonometric_inequality}\mcite[53]{Николов2020АпроксимацииЛекции}
  For any nonnegative integer \( n \) and any \( s \in \tau_n \) we have
  \begin{equation}\label{eq:thm:bernsteins_trigonometric_inequality}
    \abs{s'(\theta)} \leq n \norm{s} \quad\forall \theta \in [-\pi, \pi].
  \end{equation}
\end{corollary}
\begin{proof}
  The case \( n = 0 \) is trivial. If \( n > 0 \), for any \( s \in \tau_n \), we can apply \eqref{eq:thm:szegos_trigonometric_inequality} to \( \frac s {\norm s} \) to obtain
  \begin{equation}
    [s'(\theta)]^2 + n^2 s^2 (\theta) \leq \norm{s}^2 n^2 \quad\forall \theta \in [-\pi, \pi].
  \end{equation}

  \eqref{eq:thm:bernsteins_trigonometric_inequality} follows directly.
\end{proof}

\begin{theorem}[Bernstein's trigonometric theorem]\label{thm:bernsteins_trigonometric_theorem}\mcite[55]{Николов2020АпроксимацииЛекции}
  Let \( f \in \tilde(C)[-\pi, \pi] \) and
  \begin{equation*}
    \tilde{E}_n(f) \leq \frac A {n^{k+\alpha}} \quad n = 0, 1, 2, \ldots,
  \end{equation*}
  where \( A \in \BbbR \) and \( \alpha \in (0, 1) \).

  Then \( f \in C^{(k)}[-\pi, \pi] \) and \( f^{(k)} \) is \( \alpha \)-H\"older.
\end{theorem}
\begin{proof}
  Since \( \tilde{E}_n(f) \) is bounded by \( \frac A {n^{k+\alpha}} \) on a compact interval, there exists a sequence \( \{ s_k \}_{k=0}^\infty \) such that
  \begin{equation*}
    \norm{f - s_n} \leq A {n^{k+\alpha}}.
  \end{equation*}

  Define the sequence of polynomials
  \begin{equation*}
    v_j \coloneqq \begin{cases}
      s_1,                  &j = 0, \\
      s_{2^j} - s_{2^{j-1}} &j > 0.
    \end{cases}
  \end{equation*}

  It is now clear that
  \begin{equation*}
    \norm{f - \sum_{j=0}^n v_j} \xrightarrow[]{j \to \infty} 0
  \end{equation*}
  because
  \begin{equation*}
    \abs{f(\theta) - \sum_{j=0}^n v_j (\theta)}
    =
    \abs{f(\theta) - s_{2^n}(\theta)}
    \leq
    A {n^{k+\alpha}}.
  \end{equation*}

  For each term of the series, we have
  \begin{equation*}
    \abs{v_j(\theta)}
    \leq
    \abs{f(\theta) - s_{2^j}(\theta)} + \abs{f(\theta) - s_{2^{j-1}}(\theta)}
    \leq
    \frac A {2^{j(k + \alpha)}} + \frac A {2^{(j - 1) (k + \alpha)}}.
  \end{equation*}

  By setting \( B \coloneqq A (2^{k + \alpha}) \), we obtain
  \begin{equation*}
    \abs{v_j(\theta)} \leq \frac B {2^{j(k + \alpha)}}.
  \end{equation*}

  By \fullref{thm:bernsteins_trigonometric_inequality},
  \begin{equation*}
    \norm{v_j^{(r)}} \leq 2^{jr} \norm{v_j} \leq \frac B {2^{j(k - r + \alpha)}}.
  \end{equation*}

  Then
  \begin{equation*}
    \sum_{j=0}^\infty v_j^{(r)} (\theta)
  \end{equation*}
  converges uniformly, therefore
  \begin{equation*}
    f^{(r)}(\theta) = \sum_{j=0}^\infty v_j^{(r)} (\theta).
  \end{equation*}
\end{proof}

\begin{theorem}[Bernstein's algebraic inequality]\label{thm:bernsteins_algebraic_inequality}\mcite[59]{Николов2020АпроксимацииЛекции}
  For any nonnegative integer \( n \) and any \( p \in \pi_n \) and \( x \in (a, b) \) we have
  \begin{equation*}
    \abs{p'(x)} \leq n \frac 1 {(b - a)(b - x)} \norm{p}.
  \end{equation*}
\end{theorem}

\begin{theorem}[Bernstein's algebraic theorem]\label{thm:bernsteins_algebraic_theorem}\mcite[60]{Николов2020АпроксимацииЛекции}
  Let \( f \in C[a, b] \) and
  \begin{equation*}
    E_n(f) \leq \frac A {n^{k+\alpha}} \quad n = 0, 1, 2, \ldots,
  \end{equation*}
  where \( A \in \BbbR \) and \( \alpha \in (0, 1) \).

  Then \( f \in C^{(k)}(a, b) \) and \( f^{(k)} \) is \( \alpha \)-H\"older in every \( [a_1, b_1] \) such that \( a_1 > a \) and \( b_1 < b \).
\end{theorem}


  \section{Measure theory}\label{sec:measure_theory}

\begin{definition}\label{def:sigma_complete_lattice}\mcite[77]{Jech2003}
  We say that a \hyperref[def:semilattice/lattice]{lattice} is \term{countably complete} or \( \sigma \)-\term{complete} if its operations are defined for \hyperref[def:set_countability/countably_infinite]{countably infinite} sets.
\end{definition}

\begin{remark}\cite{MathOF:what_does_the_sigma_in_sigma_algebra_stand_for}
  The prefix \enquote{\( \sigma \)-} stands for \enquote{closed under countable unions}.

  Felix Hausdorff chose the letter \( \sigma \) because of the German word \enquote{summe}, which he used for unions, and \( \delta \) because of the German word \enquote{durchschnitt}, which he used for intersections.
\end{remark}

\begin{definition}\label{def:sigma_algebra}
  \todo{Define}.
\end{definition}

\begin{definition}\label{def:measure}
  \todo{Define}.
\end{definition}

\begin{definition}\label{def:lebesgue_measure}
  \todo{Define}.
\end{definition}


  \section{General topology}\label{sec:general_topology}

The study of topology began by abstracting away geometrical notions like \hyperref[def:metric_space]{distances}, \hyperref[def:angle]{angles} and \hyperref[def:regular_curve_curvature]{curvature} until we were left with a vague notion of points having certain neighborhoods. The obtained notions turned out to be powerful enough to find many uses outside of analysis and geometry.

We begin by discussing \hyperref[def:topological_space]{topological spaces} in \fullref{subsec:topological_spaces}, then convergence in \fullref{subsec:net_convergence}, \fullref{subsec:filter_convergence} and \fullref{subsec:function_convergence}, and finally \hyperref[def:global_continuity]{continuous functions} in \fullref{subsec:topological_continuity}. We then discuss the \hyperref[def:category_of_small_topological_spaces]{category of topological spaces} in \fullref{subsec:category_of_topological_spaces}.

We then turn to separation axioms in \fullref{subsec:separation_axioms}, connectedness in \fullref{subsec:connected_spaces} and compactness in \fullref{subsec:compact_spaces}. Some of the notions discussed are shown in \cref{fig:topological_space_kind_hierarchy}. We also discuss \hyperref[def:baire_space]{Baire spaces} in \fullref{subsec:baire_spaces}.

Finally, we discuss \hyperref[def:uniform_space]{uniform spaces} in \fullref{subsec:uniform_spaces}.

Topology is used extensively through the document --- almost all concepts from \cref{fig:topological_space_structure_hierarchy} are defined in other sections, and notions like \hyperref[def:ordinal_space]{ordinal spaces} or \hyperref[def:quiver_geometric_realization]{quiver geometric realizations} are not included in the figure.

\begin{figure}[!ht]
  \caption{Some important kinds of topological spaces}\label{fig:topological_space_kind_hierarchy}
  \smallskip
  \hfill
  \begin{forest}
    [
      {\hyperref[def:topological_space]{topological} \\ \hyperref[def:topological_space]{space}}, align=center, name=topological
        [
          {\hyperref[def:metacompact_space]{metacompact}}, name=metacompact,
          [{\hyperref[def:paracompact_space]{paracompact}}, name=paracompact]
        ]
        [
          {\hyperref[def:locally_compact_space]{locally} \\ \hyperref[def:locally_compact_space]{compact}}, align=center, name=locally_compact
          [{\hyperref[def:compact_space]{compact}}, tier=path_connected, name=compact]
        ]
        [{\hyperref[def:sequentially_compact_space]{sequentially} \\ \hyperref[def:sequentially_compact_space]{compact}}, align=center, name=sequentially_compact]
        [
          {\hyperref[def:locally_connected_space]{locally} \\ \hyperref[def:locally_connected_space]{connected}}, align=center, name=locally_connected
            [
              {\hyperref[def:connected_space]{connected}}, name=connected
              [{\hyperref[def:path_connected_space]{path} \\ \hyperref[def:path_connected_space]{connected}}, align=center, tier=path_connected, name=path_connected]
            ]
            [{\hyperref[def:locally_path_connected_space]{locally path} \\ \hyperref[def:locally_path_connected_space]{connected}}, align=center, name=locally_path_connected]
        ]
        [{\hyperref[def:topological_space_density]{separable}}, align=center, name=separable]
        [
          {\hyperref[def:topological_space_character]{first-} \\ \hyperref[def:topological_space_character]{countable}}, align=center, name=first_countable
            [{\hyperref[def:topological_space_weight]{second-} \\ {\hyperref[def:topological_space_weight]{countable}}}, align=center, name=second_countable]
        ]
        [{\hyperref[def:baire_space]{Baire}}, align=center, name=baire]
    ]
    \draw[-] (path_connected) to (locally_path_connected);
    \draw[-] (second_countable) to (separable);
    \draw[-] (compact) to (paracompact);
    \draw[-] (compact) to (sequentially_compact);
  \end{forest}
  \hfill\hfill
\end{figure}

\begin{figure}[!ht]
  \caption{Some topological spaces with additional structure}\label{fig:topological_space_structure_hierarchy}
  \smallskip
  \hfill
  \begin{forest}
    [
      {\hyperref[def:topological_space]{topological} \\ \hyperref[def:topological_space]{space}}, align=center, name=topological
        [
          {\hyperref[def:uniform_space]{uniform} \\ \hyperref[def:uniform_space]{space}}, align=center, name=uniform
            [
              {\hyperref[def:metric_space]{metric} \\ \hyperref[def:metric_space]{space}}, align=center, name=metric
                [{\hyperref[def:complete_metric_space]{complete} \\ \hyperref[def:complete_metric_space]{metric space}}, align=center, name=complete_metric]
            ]
            [
              {\hyperref[def:complete_uniform_space]{complete} \\ \hyperref[def:complete_uniform_space]{uniform space}}, align=center, name=complete_uniform]
                [
                  {\hyperref[def:topological_group]{topological} \\ \hyperref[def:topological_group]{group}}, align=center, name=group
                    [
                      {\hyperref[def:topological_vector_space]{topological} \\ \hyperref[def:topological_vector_space]{vector space}}, align=center, name=vector_space
                        [
                          {\hyperref[def:locally_convex_space]{locally convex} \\ \hyperref[def:locally_convex_space]{space}}, align=center, name=locally_convex
                            [
                              {\hyperref[def:frechet_space]{Frechet} \\ \hyperref[def:frechet_space]{space}}, align=center, name=frechet
                                [
                                  {\hyperref[def:banach_space]{Banach} \\ \hyperref[def:banach_space]{space}}, align=center, name=banach
                                    [{\hyperref[def:banach_algebra]{Banach} \\ \hyperref[def:banach_algebra]{algebra}}, align=center, name=banach_algebra]
                                    [
                                      {\hyperref[def:hilbert_space]{Hilbert} \\ \hyperref[def:hilbert_space]{space}}, align=center, name=hilbert
                                    ]
                                ]
                            ]
                            [
                              {\hyperref[def:normed_vector_space]{normed} \\ \hyperref[def:normed_vector_space]{space}}, align=center, name=normed
                                [
                                  {\hyperref[def:inner_product_space]{inner product} \\ \hyperref[def:inner_product_space]{space}}, align=center, name=inner
                                ]
                            ]
                        ]
                        [
                          {\hyperref[def:ordered_topological_vector_space]{ordered topological} \\ \hyperref[def:ordered_topological_vector_space]{vector space}}, align=center, tier=ordered, name=ordered_vector_space
                            [
                              {\hyperref[def:topological_vector_lattice]{topological} \\ \hyperref[def:topological_vector_lattice]{vector lattice}}, align=center, name=topological_lattice
                                [
                                  {\hyperref[def:normed_vector_lattice]{normed} \\ \hyperref[def:normed_vector_lattice]{vector lattice}}, align=center, name=normed_lattice
                                  [{\hyperref[def:euclidean_space]{Euclidean} \\ \hyperref[def:euclidean_space]{space}}, align=center, name=euclidean, before drawing tree={y-=4em}]
                                ]
                            ]
                        ]
                    ]
                ]
            ]
            [
              {\hyperref[def:topological_manifold]{topological} \\ {\hyperref[def:topological_manifold]{manifold}}}, align=center, name=manifold
              [
                {\hyperref[def:smooth_manifold]{smooth} \\ {\hyperref[def:smooth_manifold]{manifold}}}, align=center, name=smooth_manifold,
                  [{\hyperref[def:riemannian_manifold]{Riemannian} \\ {\hyperref[def:riemannian_manifold]{manifold}}}, align=center, tier=ordered, name=riemannian_manifold]
              ]
            ]
        ]
    ]
    \draw[-] (complete_metric) to (complete_uniform);
    \draw[-] (frechet) to (complete_metric);
    \draw[-] (banach) to (normed);
    \draw[-] (hilbert) to (inner);
    \draw[-] (normed_lattice) to (normed);
    \draw[-] (vector_space) to (manifold);
    \draw[-] (euclidean) to (hilbert);
    \draw[-] (euclidean) to (banach_algebra.south east);
    \draw[-] (euclidean) to (riemannian_manifold);
  \end{forest}
  \hfill\hfill
\end{figure}

  \section{Topological spaces}\label{sec:topological_spaces}

\begin{remark}\label{rem:topologies_and_sigma_algebras_as_lattices}
  Some structures like \hyperref[def:topological_space]{topologies} and \hyperref[def:sigma_algebra]{sigma algebras} are \hyperref[def:lattice/submodel]{bounded sublattices} of the \hyperref[thm:boolean_algebra_of_subsets]{power set Boolean algebra} of the ambient space. This allows us to study certain aspects of \hyperref[ch:general_topology]{general topology} and \hyperref[ch:measure_theory]{measure theory} via \hyperref[sec:lattices]{lattice theory}.
\end{remark}

\begin{definition}\label{def:topological_space}\mcite[11]{Engelking1989GeneralTopology}
  A \term{topology} on a set \( X \) is a \hyperref[def:lattice]{join-complete} \hyperref[def:lattice/submodel]{bounded sublattice} of the \hyperref[thm:boolean_algebra_of_subsets]{power set Boolean algebra} of \( X \).

  This is commonly expressed via the following three conditions on an arbitrary family \( \mscrT \) of subsets:
  \begin{thmenum}
    \thmitem[def:topological_space/O1]{O1} \( \mscrT \) contains both \( \varnothing \) and \( X \).
    \thmitem[def:topological_space/O2]{O2} \( \mscrT \) is closed under finite intersections: \( U, V \in \mscrT \) implies \( U \cap V \in \mscrT \).
    \thmitem[def:topological_space/O3]{O3} \( \mscrT \) is closed under arbitrary unions: \( \mscrT' \subseteq \mscrT \) implies \( \bigcup \mscrT' \in \mscrT \).
  \end{thmenum}

  If \( \mscrT \) is a topology on \( X \), we call \( (X, \mscrT) \) a \term{topological space}. When the topology is obvious from the context, we refer to \( X \) itself as a topological space.

  Elements of \( X \) are called \term{points} of the topological space, elements of \( \mscrT \) are called \term{open sets} and \hyperref[thm:boolean_algebra_of_subsets/complement]{set-theoretic complements} of open sets are called \term{closed sets}. Sets that are both open and closed are called \term{clopen}.
\end{definition}
\begin{comments}
  \item We will later prove that open sets are \hyperref[def:function_fixed_point]{fixed points} of the \hyperref[def:topological_interior_operator]{topological interior operator} and closed sets are fixed points of the \hyperref[def:topological_closure_operator]{topological closure operator}.
\end{comments}

\begin{definition}\label{def:topological_space_ordering}\mcite[16]{Engelking1989GeneralTopology}
  The family of all topologies on a set \( X \) is \hyperref[def:partially_ordered_set]{partially ordered} by set inclusion. If \( \mscrO \subseteq \mscrT \) are topologies on \( X \), we say that \( \mscrO \) is \term[bg=груба, ru=грубая]{coarser} than \( \mscrT \) and that \( \mscrT \) is \term{finer} than \( \mscrO \).
\end{definition}
\begin{comments}
  \item As we shall see in \fullref{thm:lattice_of_topologies}, this is a \hyperref[def:partially_ordered_set]{partial ordering} in the \hyperref[thm:lattice_of_topologies]{lattice of topologies} on \( X \).
\end{comments}

\begin{definition}\label{def:discrete_set}
  \todo{Define discrete sets}
\end{definition}

\begin{definition}\label{def:discrete_topology}\mcite[37]{Kelley1975GeneralTopology}
  The \term{discrete topology} on the set \( X \) is simply the power set Boolean algebra \( \pow(X) \).
\end{definition}
\begin{comments}
  \item It is the in the \hyperref[def:topological_space_ordering]{finest} topology in the \hyperref[thm:lattice_of_topologies]{lattice of topologies} on \( X \), and is \hyperref[def:category_adjunction]{left adjoint} to the \hyperref[def:concrete_category]{forgetful functor} \( U: \cat{Top} \to \cat{Set} \).

  \item In \hyperref[def:connected_space]{connected spaces}, being closed implies not being open. At the other end, under the \hyperref[def:discrete_topology]{discrete topology}, every set is open.
\end{comments}

\begin{definition}\label{def:indiscrete_topology}\mcite[37]{Kelley1975GeneralTopology}
  The \term{trivial} or \term{indiscrete topology} on the set \( X \) is simply the two-element subalgebra \( \set{ 0, X } \) of \( \pow(X) \).
\end{definition}
\begin{comments}
  \item It is the in the \hyperref[def:topological_space_ordering]{coarsest} topology in the \hyperref[thm:lattice_of_topologies]{lattice of topologies} on \( X \), and is \hyperref[def:category_adjunction]{right adjoint} to the \hyperref[def:concrete_category]{forgetful functor} \( U: \cat{Top} \to \cat{Set} \).
\end{comments}

\begin{proposition}\label{thm:empty_set_discrete_and_indiscrete_topologies}
  The topology for the empty set and for any singleton set is unique.
\end{proposition}
\begin{proof}
  In both cases, it is irrelevant whether we consider the \hyperref[def:discrete_topology]{discrete} or \hyperref[def:indiscrete_topology]{indiscrete} topology.
\end{proof}

\begin{definition}\label{def:sierpinski_space}\mcite[example 2.1.7(e)]{Perrone2021CategoryTheoryNotes}
  The \hyperref[thm:two_element_lattice]{two-element lattice} \( \set{ \top, \bot } \) endowed with the topology
  \begin{equation*}
    \set[\Big]{ \varnothing, \set{ \top }, \set{ \top, \bot } }
  \end{equation*}
  is called the \term{Sierpi\'nski space}.

  \begin{figure}[!ht]
    \hfill
    \includegraphics[page=1]{output/def__sierpinski_space}
    \hfill\hfill
    \caption{A \hyperref[def:hasse_diagram]{Hasse diagram} of the \hyperref[def:discrete_topology]{discrete} and the \hyperref[def:sierpinski_space]{Sierpi\'nski topologies} on \( \set{ \top, \bot } \). }
    \label{fig:def:sierpinski_space}
  \end{figure}
\end{definition}

\begin{remark}\label{rem:constructing_topologies}
  Among others, we have the following ways of constructing topologies:
  \begin{thmenum}
    \thmitem{rem:constructing_topologies/open} By specifying all open sets --- see \fullref{def:topological_space}.

    \thmitem{rem:constructing_topologies/closed} By specifying all closed sets --- see \fullref{thm:topology_from_closed_sets}.

    \thmitem{rem:constructing_topologies/base} By specifying only a \hyperref[def:topological_base]{base} of open sets --- see \fullref{thm:topology_from_base}.

    \thmitem{rem:constructing_topologies/subbase} By specifying only some family of open sets --- see \fullref{thm:topology_from_subbase}.

    This is the approach we use for order topologies in \fullref{def:order_topology}.

    \thmitem{rem:constructing_topologies/local_base} By specifying a \hyperref[def:topological_local_base]{local base} of open sets at every point --- see \fullref{thm:topology_from_local_base}.

    This is the approach we use for metric spaces in \fullref{def:metric_topology} and uniform spaces in \fullref{def:uniform_topology}.

    \thmitem{rem:constructing_topologies/local_subbase} By specifying a \hyperref[def:topological_local_subbase]{local subbase} of open sets at every point --- see \fullref{thm:topology_from_local_subbase}.

    \thmitem{rem:constructing_topologies/closure} By specifying a \hyperref[def:topological_closure_operator]{closure operator} --- see \fullref{thm:topology_from_closure_operator}.

    \thmitem{rem:constructing_topologies/interior} By specifying an \hyperref[def:topological_interior_operator]{interior operator} --- see \fullref{thm:topology_from_interior_operator}.
  \end{thmenum}
\end{remark}

\begin{proposition}\label{thm:topology_from_closed_sets}
  If \( \mscrF \) is a \hyperref[def:lattice/submodel]{bounded sublattice} of the \hyperref[thm:boolean_algebra_of_subsets]{power set Boolean algebra} of \( X \) that is closed under arbitrary intersections, then \( \mscrT \coloneqq \set{ X \setminus F \given F \in \mscrF } \) is a \hyperref[def:topological_space]{topology} on \( X \). This allows us to construct a topology by specifying the closed sets in advance.

  This is commonly expressed via the following three conditions on an arbitrary family \( \mscrF \) of subsets:
  \begin{thmenum}
    \thmitem[thm:topology_from_closed_sets/C1]{C1} \( \mscrF \) contains both \( \varnothing \) and \( X \).
    \thmitem[thm:topology_from_closed_sets/C2]{C2} \( \mscrF \) is closed under finite unions: \( U, V \in \mscrF \) implies \( U \cup V \in \mscrF \).
    \thmitem[thm:topology_from_closed_sets/C3]{C3} \( \mscrF \) is closed under arbitrary intersections: \( \mscrF' \subseteq \mscrF \) implies \( \bigcap \mscrF' \in \mscrF \).
  \end{thmenum}
\end{proposition}
\begin{proof}
  We will show that the axioms \ref{thm:topology_from_closed_sets/C1} --- \ref{thm:topology_from_closed_sets/C3} imply \ref{def:topological_space/O1} --- \ref{def:topological_space/O3}. The converse is also true, but we will not give a proof because it is straightforward.

  \SubProofOf{def:topological_space/O1} Follows directly from \ref{thm:topology_from_closed_sets/C1}.
  \SubProofOf{def:topological_space/O2} If \( F, G \in \mscrF \), then \ref{thm:topology_from_closed_sets/C2} implies \( F \cup G \in \mscrF \).

  By \fullref{thm:de_morgans_laws_for_sets},
  \begin{equation*}
    X \setminus (F \cup G) = (X \setminus F) \cap (X \setminus G),
  \end{equation*}
  hence \( (X \setminus F), (X \setminus G) \in \mscrT \) implies \( X \setminus (F \cup G) \in \mscrT \).

  \SubProofOf{def:topological_space/O3} Follows from \ref{thm:topology_from_closed_sets/C3} via \fullref{thm:de_morgans_laws_for_sets}.
\end{proof}

\begin{definition}\label{def:topological_neighborhood}\mcite[def. 4.1]{Александров1977ОбщаяТопология}
  Let \( A \) be a nonempty set in a topological space \( (X, \mscrT) \). We say that the set \( U \) is a \term[bg=околност,ru=окрестность]{neighborhood} of \( A \) if \( A \subseteq U \) and \( U \) is open. The definition usually applies to singleton sets \( A = \set{ x } \), in which case we refer to \enquote{neighborhoods of a point}.
\end{definition}
\begin{comments}
  \item Some authors --- for example \incite[12]{Engelking1989GeneralTopology}, \incite[7]{Rudin1991FunctionalAnalysis} and \incite[def. 4.1]{Александров1977ОбщаяТопология} --- stick to this definition, while others define neighborhoods to be an arbitrary superset of what we call a neighborhood --- for example \incite[8]{Kelley1975GeneralTopology}.
\end{comments}

\begin{definition}\label{def:neighborhood_filter}\mcite[25]{SteenSeebach1995CounterexamplesInTopology}
  Given a set \( A \) in a topological space \( (X, \mscrT) \), we define the \term{neighborhood filter} of \( A \) as the family
  \begin{equation*}
    \mscrT(A) \coloneqq \set{ U \in \mscrT \given A \subseteq U }.
  \end{equation*}
\end{definition}
\begin{comments}
  \item For a singleton set \( A = \set{ x } \), \( \mscrT(x) \) is a \hyperref[def:filter_ordering]{coarser} than the \hyperref[def:principal_ultrafilter]{principal ultrafilter} of \( x \).
\end{comments}
\begin{defproof}
  The neighborhood filter \( \mscrT(A) \) is a \hyperref[def:lattice_ideal]{filter} in \( \mscrT \). Indeed:
  \begin{itemize}
    \item The intersection of finitely many neighborhoods of \( A \) is again a neighborhood of \( A \).
    \item The union of a neighborhood of \( A \) with any open set is again a neighborhood of \( A \).
  \end{itemize}
\end{defproof}

\begin{example}\label{ex:def:topological_neighborhood}
  We list several examples of \hyperref[def:topological_neighborhood]{neighborhood filters}:
  \begin{thmenum}
    \thmitem{ex:def:topological_neighborhood/discrete} With the \hyperref[def:discrete_topology]{discrete topology} on \( X \), the \hyperref[def:topological_neighborhood]{neighborhood filter} of the point \( x \) is the \hyperref[def:principal_ultrafilter]{principal ultrafilter} of \( x \).

    \thmitem{ex:def:topological_neighborhood/indiscrete} At the other end, with the \hyperref[def:indiscrete_topology]{indiscrete topology} on \( X \), the \hyperref[def:topological_neighborhood]{neighborhood filter} of any point \( x \) is the family \( \set{ X } \).
  \end{thmenum}
\end{example}

\begin{definition}\label{def:topological_base}
  Fix a topological space \( (X, \mscrT) \). We say that the family \( \mscrB \subseteq \mscrT \) is a \term{base} for the topology \( \mscrT \) if \( \mscrB \) satisfies any of the equivalent conditions:
  \begin{thmenum}
    \thmitem{def:topological_base/union}\mcite[12]{Engelking1989GeneralTopology} Every open set \( U \in \mscrT \) is the union \( U = \bigcup \mscrB' \) of some subset \( \mscrB' \subseteq \mscrB \).

    \medskip

    \thmitem{def:topological_base/subset}\mcite[12]{Engelking1989GeneralTopology} For any point \( x \in X \) and for any neighborhood \( U \) of \( x \), there exists a set \( V \in \mscrB \) in the base such that \( x \in V \subseteq U \).
  \end{thmenum}
\end{definition}
\begin{proof}
  \ImplicationSubProof{def:topological_base/union}{def:topological_base/subset} Fix a point \( x \in X \) and a neighborhood \( U \) of \( x \). Let \( \mscrB' \) be a subfamily of \( \mscrB \) such that
  \begin{equation*}
    U = \bigcup \mscrB'.
  \end{equation*}

  Then \( x \in V \) for at least one \( V \in \mscrB' \) by definition of union.

  \ImplicationSubProof{def:topological_base/subset}{def:topological_base/union} Fix an open set \( U \) and suppose that, for every \( x \in U \), there exists a set \( V_x \in \mscrB \) such that \( x \in V_x \subseteq U \).

  Thus,
  \begin{equation*}
    \bigcup_{x \in U} V_x \subseteq U \subseteq \bigcup_{x \in U} V_x
  \end{equation*}
  and
  \begin{equation*}
    U = \bigcup_{x \in U} V_x.
  \end{equation*}
\end{proof}

\begin{example}\label{ex:def:topological_base}
  We list several examples of \hyperref[def:topological_base]{topological bases}:
  \begin{thmenum}
    \thmitem{ex:def:topological_base/topology} Every topology is itself a base. The topology with the empty set removed is also a base.

    \thmitem{ex:def:topological_base/superset} If \( \mscrB \) is a base of \( \mscrT \), then any set between \( \mscrB \) and \( \mscrT \) is also a base.

    \thmitem{ex:def:topological_base/indiscrete} The family \( \set{ X } \) is a base for the \hyperref[def:indiscrete_topology]{indiscrete topology} on a nonempty set \( X \). It is the only proper base.

    For an empty space \( X = \varnothing \), the indiscrete topology is \( \set{ \varnothing } \), and hence the empty family is a base.

    \thmitem{ex:def:topological_base/sierpinski} The \hyperref[def:sierpinski_space]{Sierpi\'nski space} also has only two bases: the topology itself and the set
    \begin{equation*}
      \set[\Big]{ \set{ \top }, \set{ \top, \bot } }.
    \end{equation*}

    \thmitem{ex:def:topological_base/discrete} For the \hyperref[def:discrete_topology]{discrete topology} \( \pow(X) \) on a set \( X \), the following is a base:
    \begin{equation*}
      \mscrB = \set[\Big]{ \set{ x } \given x \in X }.
    \end{equation*}

    Indeed, every subset of \( X \) is a union of members of \( \mscrB \).

    \thmitem{ex:def:topological_base/not_closed_under_intersections} A topological base may or may not be closed under finite intersections. For example, consider the \hyperref[def:order_topology]{order topology} on \( \BbbR \) with base \eqref{eq:def:order_topology/base}.

    We can remove any open interval \( (a, b) \) from the base and the result would still be a base because, for \( a < x < y < b \),
    \begin{equation*}
      (a, b) = (a, y) \cup (x, b).
    \end{equation*}

    Then the base would not be closed under intersections because
    \begin{equation*}
      (a, b) = (-\infty, b) \cap (a, \infty).
    \end{equation*}

    This leads to \ref{thm:topology_from_base/B2}.
  \end{thmenum}
\end{example}

\begin{proposition}\label{thm:topology_from_base}\mcite[12]{Engelking1989GeneralTopology}
  Let \( X \) be an arbitrary set and let \( \mscrB \) be a family of subsets that satisfies the conditions
  \begin{thmenum}
    \thmitem[thm:topology_from_base/B1]{B1} \( \mscrB \) \hyperref[def:set_cover]{covers} \( X \), i.e. \( \bigcup \mscrB = X \).
    \thmitem[thm:topology_from_base/B2]{B2} If \( U, V \in \mscrB \) and if \( x \in U \cap V \), there exists \( W \in \mscrB \) such that \( x \in W \) and \( W \subseteq U \cap V \).
  \end{thmenum}

  Then the family
  \begin{equation}\label{eq:thm:topology_from_base/topology}
    \mscrT \coloneqq \set*{ \bigcup \mscrB' \given \mscrB' \subseteq \mscrB }
  \end{equation}
  is the coarsest topology on \( X \) containing \( \mscrB \).

  Furthermore, \( \mscrB \) satisfies \fullref{def:topological_base/union} is thus a \hyperref[def:topological_base]{base} of \( \mscrT \). We say that the base \( \mscrB \) \term{generates} \( \mscrT \).
\end{proposition}
\begin{proof}
  \SubProof{Proof that \( \mscrT \) is a topology}

  \SubProofOf*{def:topological_space/O1} \( \varnothing = \bigcup \varnothing \) and, by \ref{thm:topology_from_base/B1}, \( X = \bigcup \mscrB \).

  \SubProofOf*{def:topological_space/O3} Pick a subfamily \( \mscrT' \subseteq \mscrT \). For each \( U \in \mscrT' \), there exists a subfamily \( \mscrB_U' \) such that \( U = \bigcup \mscrB_U' \). Define
  \begin{equation*}
    \mscrB'
    \coloneqq
    \bigcup_{U \in \mscrT'} \mscrB_U'
    =
    \set{ B \in \mscrB \given \qexists {U \in \mscrT'} B \in \mscrB_U' }.
  \end{equation*}

  It is a subset of \( \mscrB \). Furthermore,
  \begin{align*}
    \bigcup \mscrT'
    &=
    \set{ x \in X \given \qexists{U \in \mscrT'} x \in U }
    = \\ &=
    \set{ x \in X \given \qexists{U \in \mscrT'} x \in \bigcup \mscrB_U' }
    = \\ &=
    \set{ x \in X \given \qexists{U \in \mscrT'} \qexists{B \in \mscrB_U'} x \in B }
    = \\ &=
    \set{ x \in X \given \qexists{B \in \mscrB'} x \in B }
    =
    \bigcup \mscrB'.
  \end{align*}

  Hence, \( \bigcup\mscrT' \in \mscrT \).

  \SubProofOf*{def:topological_space/O2} Fix \( U, V \in T \). Then there exist families \( \mscrB_U, \mscrB_V \subseteq B \) such that \( U = \bigcup \mscrB_U' \) and \( V = \bigcup \mscrB_V' \).

  Furthermore, by \ref{thm:topology_from_base/B2}, for every \( x \in U' \cap V' \) there exists a neighborhood \( W_x \) of \( x \) such that \( W \subseteq U' \cap V' \). We have
  \begin{equation*}
    U' \cap V' \subseteq \bigcup\set{ W_x \given x \in U' \cap V' } \subseteq U' \cap V'.
  \end{equation*}

  Then
  \begin{align*}
    U \cap V
    &=
    \parens*{ \bigcup \mscrB_U } \cap \parens*{ \bigcup \mscrB_V }
    = \\ &=
    \set{ x \in X \given x \in \bigcup \mscrB_U \T{and} x \in \bigcup \mscrB_V }
    = \\ &=
    \set{ x \in X \given \qexists{ U' \in \mscrB_U' } x \in U' \T{and} \qexists{ V' \in \mscrB_V' } x \in V' }
    \reloset {\ref{thm:pulling_quantifiers_out}} = \\ &=
    \set{ x \in X \given \qexists{ U' \in \mscrB_U' } \qexists{ V' \in \mscrB_V' } x \in U' \cap V' }
    = \\ &=
    \bigcup\set{ U' \cap V' \given U' \in \mscrB_U' \T{and} V' \in \mscrB_V' }.
    = \\ &=
    \bigcup\set{ W_x \given U' \in \mscrB_U' \T{and} V' \in \mscrB_V' \T{and} x \in U' \cap V' }.
  \end{align*}

  Therefore, \( U \cap V \) is a union of members of \( \mscrB \), and is thus a member of \( \mscrT \).

  \SubProof{Proof of minimality of \( \mscrT \)} Let \( \mscrO \) be another topology on \( X \) containing \( \mscrB \).

  Since \( \mscrO \) must be closed under arbitrary unions, and since \( \mscrB \) may not be closed under arbitrary unions, it follows that \( \mscrO \) must contain the union \( \bigcup \mscrB' \) of an arbitrary subset \( \mscrB' \subseteq \mscrB \).

  Therefore, \( \mscrT \) is necessarily a subset of \( \mscrO \).
\end{proof}

\begin{proposition}\label{thm:base_can_generate_topology}
  Every \hyperref[def:topological_base]{base} satisfies \ref{thm:topology_from_base/B1} and \ref{thm:topology_from_base/B2}.
\end{proposition}
\begin{proof}
  Let \( (X, \mscrT) \) be a topological space and let \( \mscrB \) be a base for \( \mscrT \).

  \SubProofOf{thm:topology_from_base/B1} Since \( X \) is always a member of the topology, \fullref{def:topological_base/union} implies that \( X \) is a union.

  \SubProofOf{thm:topology_from_base/B2} Let \( U \) and \( V \) be members of \( \mscrB \). Since \( U \cap V \in \mscrT \), there exists a family \( \mscrB' \subseteq \mscrB \) such that \( U \cap V = \bigcup \mscrB' \).

  Thus, for every point \( x \) of \( U \cap V \), by definition of set union there exists a member \( W \) of \( \mscrB' \) such that \( x \in W \). Obviously \( W \subseteq U \cap V \).
\end{proof}

\begin{proposition}\label{thm:set_open_iff_neighborhood_is_contained}
  Let \( \mscrB \) be a \hyperref[def:topological_base]{base} of the space \( (X, \mscrT) \).

  A set \( A \subseteq X \) is open if and only if every point of \( A \) has a neighborhood in \( \mscrB \) contained in \( A \).
\end{proposition}
\begin{proof}
  This holds vacuously for empty sets. Assume that \( A \subseteq X \) is nonempty.

  \SufficiencySubProof Assume that \( A \) is open and let \( x_0 \in A \). Then \( A \) is a neighborhood of \( x_0 \), hence there exists some \( V \in \mscrB \) such that \( x_0 \in V \subseteq A \).

  \NecessitySubProof Assume that every point \( x \in A \) has a neighborhood \( U_x \) in \( \mscrB \) that \( U_x \subseteq A \). Take the union
  \begin{equation*}
    V \coloneqq \bigcup_{x \in A} U_x.
  \end{equation*}

  Obviously \( V \subseteq A \). Conversely, since \( x \) belongs to \( U_x \) for every \( x \in A \), it follows that their union \( V \) is contained in \( A \).

  We conclude that \( A = V \) is open as a union of open sets.
\end{proof}

\begin{definition}\label{def:topological_space_weight}
  We define the \term{weight} of the topological space \( (X, \mscrT) \) as the \hyperref[def:cardinal]{cardinal}
  \begin{equation*}
    w(X, \mscrT) \coloneqq \min \set{ \card(\mscrB) \given \mscrB \T{is a base for} \mscrT }.
  \end{equation*}

  We simply write \( w(X) \) when the topology is clear from the context.

  Spaces for which \( w(X) \leq \hyperref[def:aleph_hierarchy]{\aleph_0} \) are said to be \term{second-countable}.

  See \fullref{thm:topological_space_countability} for how weight relates to other measurements of the space's \enquote{size}.
\end{definition}
\begin{defproof}
  The definition is correct because cardinals are well-ordered when regarded as initial ordinals.
\end{defproof}

\begin{example}\label{ex:def:topological_space_weight}
  We list several examples of \hyperref[def:topological_space_weight]{topological space weights}:
  \begin{thmenum}
    \thmitem{ex:def:topological_space_weight/indiscrete} We have shown in \fullref{ex:def:topological_base/indiscrete} that \( \set{ X } \) is the unique base of the indiscrete topology on a nonempty set \( X \) that is not the topology itself. Hence, the weight of any indiscrete topology is \( 1 \).

    The weight of the indiscrete topology on the empty set is \( 0 \).

    \thmitem{ex:def:topological_space_weight/sierpinski} Similarly to \fullref{ex:def:topological_space_weight/indiscrete}, \fullref{ex:def:topological_base/sierpinski} allows us to conclude that the weight of the Sierpi\'nski space is \( 2 \).

    \thmitem{ex:def:topological_space_weight/discrete} Therefore, the weight of the discrete topology on \( X \) is \( \card X \).

    Let \( \mscrB \) be an arbitrary base of the \hyperref[def:discrete_topology]{discrete topology} \( \pow(X) \) on \( X \).

    There are \( 2^{\card \mscrB} \) subfamilies of \( \mscrB \). The union of two distinct subfamilies may be the same, hence
    \begin{equation*}
      \card(\pow X) = 2^{\card X} \leq 2^{\card \mscrB}.
    \end{equation*}

    \Fullref{thm:cardinal_exponentiation_comparison} then implies that \( \card X \leq \card \mscrB \). The base discussed in \fullref{ex:def:topological_base/discrete} has the cardinality of \( X \) itself, hence it is possible to attain the equality.
  \end{thmenum}
\end{example}

\begin{theorem}\label{thm:base_has_subset_of_minimal_weight}[Alexandrov-Urysohn theorem]
  Every \hyperref[def:topological_base]{base} of a topological space has a subset that is a base of minimal cardinality.
\end{theorem}
\begin{comments}
  \item \incite{Александров1977ОбщаяТопология} calls this theory \enquote{the} Alexandrov-Urysohn theorem. The infinite case of the theorem can be found as \enquote{theorem 2} in his earlier paper \cite{АлександровУрысон1950КомпактныеПространства} with Pavel Urysohn, although this theorem is not the focus of their paper.
\end{comments}
\begin{proof}
  Let \( \mscrB \) and \( \mscrC \) be bases of \( (X, \mscrT) \). Suppose that \( \mscrC \) has minimal cardinality. We will find a subset \( \mscrB_0 \) of \( \mscrB \) that is itself a base if minimal cardinality.

  \SubProof{Proof if weight is finite} Suppose that \( \mscrC \) is finite. We will show that \( \mscrC \subseteq \mscrB \).

  For each set \( A \), denote by \( \mscrC(A) \) the subfamily of \( \mscrC \) of all sets containing \( A \). Also, denote by \( C_x \) the intersection of \( \mscrC(x) \), i.e. all sets in \( \mscrC \) that contain \( x \).

  Note that \( C_x \) is a member of \( \mscrB \). Indeed, there must exist some \( U \in \mscrB \) such that \( x \in U \subseteq C_x \). Then there exists some \( V \in \mscrC \) such that \( x \in V \subseteq U \). That is, we have \( C_x \subseteq V \subseteq U \), and hence \( U = C_x \). Then
  \begin{equation*}
    \set{ C_x \given x \in X } \subseteq \mscrB.
  \end{equation*}

  Since \( \mscrB \) was an arbitrary base, we conclude that
  \begin{equation*}
    \set{ C_x \given x \in X } \subseteq \mscrC.
  \end{equation*}

  Now fix some nonempty set \( V \) in \( \mscrC \). Then \( C_x \subseteq V \) for every \( x \in V \), hence
  \begin{equation*}
    V = \bigcup_{x \in V} C_x.
  \end{equation*}

  If \( C_x \neq V \) for every \( x \in V \), then \( \mscrC \setminus \set{ V } \) is also a base, contradicting the minimality of \( \mscrC \). Hence, \( V = C_x \) for at least one \( x \in V \), implying that
  \begin{equation*}
    \mscrC = \set{ C_x \given x \in X }.
  \end{equation*}

  We conclude that \( \mscrC \subseteq \mscrB \).

  \SubProof{Proof if weight is infinite} Suppose that \( \mscrC \) is infinite, and consider the following set:
  \begin{equation*}
    \mscrB_0 \coloneqq \set{ B \in \mscrB \given \qexists {U \in \mscrC} \qexists {V \in \mscrC} U \subseteq B \subseteq V }.
  \end{equation*}

  \Fullref{thm:simplified_cardinal_arithmetic/infinite} implies that
  \begin{equation*}
    \card\mscrB_0 \leq \card \mscrC^2 = \card \mscrC = w(X, \mscrT).
  \end{equation*}

  We will show that \( \mscrB_0 \) is a base. Let \( O \) be an arbitrary nonempty open set from \( \mscrT \), and let \( x \in O \). Then there exists some set \( V \) from \( \mscrC \) such that \( x \in V \subseteq O \). Since \( V \) is open, there exists some nonempty set \( B \) from \( \mscrB \) such that \( x \in B \subseteq V \), and similarly there exists a set \( U \) in \( \mscrB \) such that \( x \in U \subseteq V \). That is, \( U \subseteq B \subseteq V \subseteq O \). Then \( B \in \mscrB_0 \). We have found a neighborhood \( B \) of \( x \) from \( \mscrB_0 \) contained in \( O \) for an arbitrary open set \( O \) and point \( x \) in \( O \).

  Therefore, \( \mscrB_0 \) satisfies \fullref{def:topological_base/subset} is thus a base of \( (X, \mscrT) \).
\end{proof}

\begin{definition}\label{def:topological_subbase}\mimprovised
  Fix a topological space \( (X, \mscrT) \). We say that the family \( \mscrS \subseteq \mscrT \) is a \term[ru=предбаза (\cite[def. 4.7]{Александров1977ОбщаяТопология})]{subbase} for the topology \( \mscrT \) if \ref{thm:topology_from_base/B1} holds and, instead of requiring \ref{thm:topology_from_base/B2}, we require that, for every nonempty \( U \in \mscrT \) and every \( x \in U \), there exists some subfamily \( V_1, \ldots, V_n \in \mscrS \) such that
  \begin{equation}\label{eq:def:topological_subbase}
    x \in V_1 \cap \cdots \cap V_n \subseteq U.
  \end{equation}
\end{definition}
\begin{comments}
  \item This is a twist around the usual definition requiring that the family of all finite intersections of members of \( \mscrS \) is a \hyperref[def:topological_base]{basis}. The latter definition is given by \incite[12]{Engelking1989GeneralTopology}, \incite[48]{Kelley1975GeneralTopology} and \incite[def. 4.7]{Александров1977ОбщаяТопология}.
\end{comments}

\begin{example}\label{ex:def:topological_subbase}
  We list several examples of \hyperref[def:topological_subbase]{topological subbases}:
  \begin{thmenum}
    \thmitem{ex:def:topological_subbase/topology} Every topology is itself a subbase, as well as any base.

    \thmitem{ex:def:topological_subbase/superset} If \( \mscrS \) is a subbase of \( \mscrT \), then any set between \( \mscrS \) and \( \mscrT \) is also a subbase.

    \thmitem{ex:def:topological_subbase/indiscrete} The family \( \set{ X } \) is a subbase for the \hyperref[def:indiscrete_topology]{indiscrete topology} on any set \( X \).

    \thmitem{ex:def:topological_subbase/discrete} For the \hyperref[def:discrete_topology]{discrete topology} \( \pow(X) \) on a set \( X \) with at least three elements, the following is a subbase:
    \begin{equation*}
      \mscrS = \set[\Big]{ \set{ x, y } \given x \in X \T{and} y \in X \T{and} x \neq y }.
    \end{equation*}

    Indeed, \fullref{ex:def:topological_base/discrete} implies that
    \begin{equation*}
      \mscrB = \set[\Big]{ \set{ x } \given x \in X }
    \end{equation*}
    is a base, and, for \( y \neq z \) distinct from \( x \),
    \begin{equation*}
      \set{ x } = \set{ x, y } \cap \set{ x, z }.
    \end{equation*}

    \thmitem{ex:def:topological_subbase/order} \hyperref[def:order_topology]{Order topologies} have a base \eqref{eq:def:order_topology/base} constructed from the subbase \eqref{eq:def:order_topology/subbase}.
  \end{thmenum}
\end{example}

\begin{proposition}\label{thm:topology_from_subbase}
  Fix a set \( X \) and an arbitrary family of subsets \( \mscrS \) of \( X \) that \hyperref[def:set_cover]{covers} \( X \). Then the family
  \begin{equation}\label{eq:thm:topology_from_subbase/base}
    \mscrB \coloneqq \set*{ \bigcap \mscrS' \colon \mscrS' \T{is a finite nonempty subfamily of} \mscrS }
  \end{equation}
  of finite intersections of \( \mscrS \) generates via \fullref{thm:topology_from_base} the coarsest topology \( \mscrT \) containing \( \mscrS \).

  Furthermore, \( \mscrS \) is a \hyperref[def:topological_subbase]{subbase} of the topology. We say that the subbase \( \mscrS \) \term{generates} the topology.
\end{proposition}
\begin{comments}
  \item We make no claim that \( \mscrB \) is the smallest base containing \( \mscrS \) --- \fullref{ex:def:topological_base/not_closed_under_intersections} provides a counterexample.
\end{comments}
\begin{proof}
  \Fullref{thm:topology_from_base} already shows well-definedness and minimality of the topology with respect to the base. Minimality with respect to the subbase can be shown similarly.
\end{proof}

\begin{remark}\label{rem:subbase_and_empty_intersection}
  In the context of an ambient topological space, the intersection of empty sets discussed in \fullref{def:basic_set_operations/intersection} is safe to use. It can be used, for example, in \eqref{eq:thm:topology_from_subbase/base} to drop the requirement \ref{thm:topology_from_base/B1}, but it would also complicate \fullref{thm:global_subbase_to_local_subbase}.

  We prefer not to use empty intersections because of possible ambiguity.
\end{remark}

\begin{proposition}\label{thm:lattice_of_topologies}
  The family \( \mfrakT \) of all \hyperref[def:topological_space]{topologies} on a set \( X \) is a \hyperref[def:lattice]{complete bounded lattice} with respect to the ordering from \fullref{def:topological_space_ordering}. Explicitly:

  \begin{thmenum}
    \thmitem{thm:lattice_of_topologies/join} The \hyperref[def:lattice/join]{join} of an arbitrary family \( \mfrakT' \) of topologies on \( X \) is the topology generated by the family \( \bigcup \mfrakT' \) via \fullref{thm:topology_from_subbase}.

    \thmitem{thm:lattice_of_topologies/top} The \hyperref[def:extremal_points/top_and_bottom]{top element} is the finest topology, the \hyperref[def:discrete_topology]{discrete topology} on \( X \).

    \thmitem{thm:lattice_of_topologies/meet} The \hyperref[def:lattice/meet]{meet} of an arbitrary family \( \mfrakT' \) of topologies on \( X \) is simply their intersection \( \bigcap \mfrakT' \).

    \thmitem{thm:lattice_of_topologies/bottom} The \hyperref[def:extremal_points/top_and_bottom]{bottom element} is the coarsest topology, \hyperref[def:indiscrete_topology]{indiscrete topology} on \( X \).
  \end{thmenum}
\end{proposition}
\begin{proof}
  \SubProofOf{thm:lattice_of_topologies/join} Minimality follows from \fullref{thm:topology_from_subbase}.

  \SubProofOf{thm:lattice_of_topologies/top} Every topology is by definition a sublattice of the discrete topology.

  \SubProofOf{thm:lattice_of_topologies/meet} We have shown in \fullref{thm:boolean_algebra_of_subsets/meet} that \( \bigcap \mfrakT' \) the meet of \( \mfrakT' \) in the Boolean algebra of sets. It remains only to show that \( \bigcap \mfrakT' \) is a topology.

  \SubProofOf*{def:topological_space/O1} Every topology must contain the empty set and the space itself, hence so does the intersection of any family of topologies.

  \SubProofOf*{def:topological_space/O2} Any finite family of sets from \( \bigcap \mfrakT' \) belongs to every topology in \( \mfrakT' \), hence their intersection also belongs to every topology in \( \mfrakT' \).

  \SubProofOf*{def:topological_space/O3} Similarly, any family of sets from \( \bigcap \mfrakT' \) belongs to every topology in \( \mfrakT' \), hence their union also belongs to every topology in \( \mfrakT' \).

  \SubProofOf{thm:lattice_of_topologies/bottom} The intersection of all topologies on \( X \) is the initial sublattice.
\end{proof}

\begin{definition}\label{def:topological_local_base}\mcite[13]{Engelking1989GeneralTopology}
  Fix a topological space \( (X, \mscrT) \) and a point \( x_0 \in X \). We say that the family \( \mscrB(x_0) \) of open sets is a \term{local base} for \( \mscrT \) at \( x_0 \) if, for any point \( x \in X \) and for any neighborhood \( U \) of \( x \), there exists a set \( V \in \mscrB(x_0) \) in the local base such that \( x \in V \subseteq U \).

  When given a local base at each point, we call the collection \( \seq{ \mscrB(x) }_{x \in X} \) a \term{neighborhood system} of the topology \( \mscrT \).
\end{definition}
\begin{comments}
  \item This definition nearly replicates \fullref{def:topological_base/subset}.
\end{comments}

\begin{proposition}\label{thm:global_base_to_local_base}
  For any \hyperref[def:topological_base]{base} \( \mscrB \), the subfamily \( \mscrB(x) \) of all members of \( \mscrB \) containing \( x \) is a \hyperref[def:topological_local_base]{local base} at \( x \).
\end{proposition}
\begin{proof}
  Let \( U \) be some neighborhood of \( x \). Then there exists some neighborhood \( V \) of \( x \) such that \( V \subseteq U \). Since \( U \) is in the neighborhood filter \( \mscrT(x) \) and \( V \) is in \( \mscrB(x) \), after generalizing on \( U \) it follows that \( \mscrB(x) \) is a prefilter for \( \mscrT(x) \).
\end{proof}

\begin{example}\label{ex:def:topological_local_base}
  We list several examples of \hyperref[def:topological_local_base]{topological local bases}:
  \begin{thmenum}
    \thmitem{ex:def:topological_local_base/sierpinski} The \hyperref[def:sierpinski_space]{Sierpi\'nski space} has a local base \( \set{ \set{ \top } } \) at \( \top \) and \( \set{ \set{ \top, \bot } } \) at \( \bot \).

    \thmitem{ex:def:topological_local_base/discrete} \fullref{ex:def:topological_base/discrete} implies that, for the \hyperref[def:discrete_topology]{discrete topology} \( \pow(X) \) on a set \( X \), a simple local base is \( \mscrB(x) = \set{ \set{ x } } \) for every point \( x \).

    \thmitem{ex:def:topological_local_base/metric} For a \hyperref[def:metric_space]{metric space}, we have an explicit countable local base \eqref{def:metric_topology/integer_base} and an explicit uncountable local base \eqref{def:metric_topology/real_base}.
  \end{thmenum}
\end{example}

\begin{proposition}\label{thm:topology_from_local_base}\mcite[13]{Engelking1989GeneralTopology}
  Let \( X \) be an arbitrary set and let
  \begin{equation*}
    \seq{ \mscrB(x) }_{x \in X}
  \end{equation*}
  be an indexed family of families of subsets of \( X \) that satisfies the conditions
  \begin{thmenum}
    \thmitem[thm:topology_from_local_base/BP1]{BP1} For every \( x \in X \), the family \( \mscrB(x) \) is nonempty and every member \( U \in B(x) \) contains \( x \).
    \thmitem[thm:topology_from_local_base/BP2]{BP2} If \( x \in U \in B(y) \), then there exists some \( V \in B(x) \) such that \( V \subseteq U \).
    \thmitem[thm:topology_from_local_base/BP3]{BP3} For every pair \( U, V \in B(x) \), there exists some \( W \in B(x) \) such that \( W \subseteq U \cap V \).
  \end{thmenum}

  The family
  \begin{equation*}
    \mscrB \coloneqq \bigcup_{x \in X} \mscrB(x)
  \end{equation*}
  generates via \fullref{thm:topology_from_base} the coarsest topology \( \mscrT \) containing \( \mscrB(x) \) for every \( x \in X \).

  Furthermore, \( \mscrB(x) \) is a \hyperref[def:topological_local_base]{local base} for \( \mscrT \) at \( x \).
\end{proposition}
\begin{proof}
  \SubProof{Proof that \( \mscrB \) satisfies \fullref{thm:topology_from_base}}

  \SubProofOf*{thm:topology_from_base/B1} \ref{thm:topology_from_local_base/BP1} implies that \( x \in U \) for every \( U \in \mscrB(x) \), hence the union over \( x \) of all neighborhoods is \( X \).

  \SubProofOf*{thm:topology_from_base/B2} Let \( U, V \in \mscrB \). Suppose that \( U \in \mscrB(u) \) and \( V \in \mscrB(v) \).

  If \( U \cap V \) is empty, \ref{thm:topology_from_base/B2} is vacuously satisfied.

  Otherwise, let \( w \in U \cap V \). \ref{thm:topology_from_local_base/BP2} implies that there exist sets \( U' \in \mscrB(w) \) and \( V' \in \mscrB(w) \) such that \( U' \subseteq U \) and \( V' \subseteq V \). \ref{thm:topology_from_local_base/BP3} implies that there exists some \( W \in \mscrB(w) \) such that
  \begin{equation*}
    W \subseteq U' \cap V' \subseteq U \cap V.
  \end{equation*}

  \SubProof{Proof of minimality of \( \mscrT \)} The proof is similar to that in \fullref{thm:topology_from_base}.

  \SubProof{Proof that \( \mscrB(x) \) is a local base} Consider the \hyperref[def:topological_neighborhood]{neighborhood filter} \( \mscrT(x) \). Let \( U \in \mscrT(x) \).

  By construction, \( U \) is a union of some nonempty subfamily \( \mscrB' \subseteq \mscrB \). Since \( U \) contains \( x \), then there exists some nonempty subfamily \( \mscrB^\dprime \) of \( \mscrB' \) of points containing \( x \). \ref{thm:topology_from_base/B1} implies that \( \mscrB^\dprime \) is a subset of \( \mscrB(x) \).

  Then \( V \subseteq U \) for every set \( V \) in \( \mscrB^\dprime \subseteq \mscrB(x) \).
\end{proof}

\begin{proposition}\label{thm:local_base_can_generate_topology}
  Every family of \hyperref[def:topological_local_base]{local bases} satisfies \ref{thm:topology_from_local_base/BP1} -- \ref{thm:topology_from_local_base/BP3}.
\end{proposition}
\begin{proof}
  Let \( (X, \mscrT) \) be a topological space and let \( \mscrB(x) \) be a local base for \( \mscrT \) at \( x \). We know that \( \mscrB(x) \) is nonempty, consists of neighborhoods of \( x \) and every neighborhood of \( x \) has an open subset in \( \mscrB(x) \).

  \SubProofOf*{thm:topology_from_local_base/BP1} Since \( X \) is a neighborhood of every point, by definition, \( \mscrB(x) \) contains some neighborhood \( U_x \subseteq X \) of \( x \). Hence, \( \mscrB(x) \) is nonempty.

  Furthermore, as a subset of \( \mscrT(x) \), every set in \( \mscrB(x) \) contains \( x \).

  \SubProofOf*{thm:topology_from_local_base/BP2} Suppose that \( x \in U \in \mscrB(y) \).

  Then \( U \) is a neighborhood of \( x \), and thus \( \mscrB(x) \) contains some open subset \( V \) of \( U \).

  \SubProofOf*{thm:topology_from_local_base/BP3} If \( U, V \in \mscrB(x) \), then \( U \cap V \) is a neighborhood of \( x \) and hence \( \mscrB(x) \) contains some open subset \( W \) of \( U \cap V \).
\end{proof}

\begin{definition}\label{def:topological_space_character}\mcite[12]{Engelking1989GeneralTopology}
  Fix a topological space \( (X, \mscrT) \). We define the \term{character} of the point \( x \in X \) as the \hyperref[def:cardinal]{cardinal}
  \begin{equation*}
    \chi(X, \mscrT, x) \coloneqq \min \set{ \card \mscrB(x) \given \mscrB(x) \T{is a local base for} \mscrT \T{at} x }.
  \end{equation*}

  We define the \term{character} of the entire space \( (X, \mscrT) \) as
  \begin{equation*}
    \chi(X, \mscrT) \coloneqq \sup \set{ \chi(x) \given x \in X }.
  \end{equation*}

  We simply write \( \chi(X) \) when the topology is clear from the context.

  Spaces for which \( \chi(X) \leq \hyperref[def:aleph_hierarchy]{\aleph_0} \) are said to be \term{first-countable}.
\end{definition}
\begin{comments}
  \item The supremum of cardinals is well-defined as a consequence of \fullref{thm:union_of_set_of_cardinals}.
  \item See \fullref{thm:topological_space_countability} for how weight relates to other measurements of the space's \enquote{size}.
\end{comments}

\begin{example}\label{ex:def:topological_space_character}
  We list several examples of \hyperref[def:topological_space_character]{topological space characters}:
  \begin{thmenum}
    \thmitem{ex:def:topological_space_character/indiscrete} We have shown in \fullref{ex:def:topological_base/indiscrete} that \( \set{ X } \) is the unique base of the indiscrete topology on a nonempty set \( X \) that is not the topology itself. Hence, the character of each point is \( 1 \), and the character of the space itself is also \( 1 \).

    \thmitem{ex:def:topological_space_character/sierpinski} \Fullref{ex:def:topological_local_base/sierpinski} implies that the character of both points in the Sierpi\'nski space is \( 1 \), and thus character of the entire space is \( 1 \).

    \thmitem{ex:def:topological_space_character/discrete} Similarly, \fullref{ex:def:topological_local_base/discrete} implies that the character of each point with respect to the discrete topology is \( 1 \), and hence the character of the space itself is \( 1 \).

    \thmitem{ex:def:topological_space_character/metric} \Fullref{thm:def:metric_topology/first_countable} shows that the character of a \hyperref[def:metric_space]{metric space} is \( \aleph_0 \).
  \end{thmenum}
\end{example}

\begin{definition}\label{def:topological_local_subbase}\mimprovised
  Similarly to how we have defined \hyperref[def:topological_subbase]{global subbases} related to \hyperref[def:topological_base]{global bases}, we can also define local subbases.

  Fix a topological space \( (X, \mscrT) \) and a point \( x_0 \in X \). We say that the family \( \mscrS(x_0) \) of open sets is a \term{local subbase} for \( \mscrT \) at \( x_0 \) if, for any point \( x \in X \) and for any neighborhood \( U \) of \( x \), there exists a family \( V_1, \ldots, V_n \in \mscrS(x_0) \) in the subbase such that
  \begin{equation}\label{eq:def:topological_local_subbase}
    V_1 \cap \cdots \cap V_n \subseteq U.
  \end{equation}
\end{definition}
\begin{comments}
  \item Unlike in \eqref{eq:def:topological_local_subbase}, in \eqref{eq:def:topological_local_subbase} it is implicit that the sets \( V_1, \ldots, V_n \) all contain \( x_0 \) and hence also have a nonempty intersection.
\end{comments}

\begin{proposition}\label{thm:global_subbase_to_local_subbase}
  For any \hyperref[def:topological_subbase]{subbase} \( \mscrS \), the subfamily \( \mscrS(x) \) of all members of \( \mscrS \) containing \( x \) is a \hyperref[def:topological_local_subbase]{local subbase} at \( x \).
\end{proposition}
\begin{proof}
  Follows from \fullref{thm:global_base_to_local_base} once we take finite intersections of \( \mscrS \) and \( \mscrS(x) \).
\end{proof}

\begin{proposition}\label{thm:topology_from_local_subbase}
  Let \( X \) be an arbitrary set and let
  \begin{equation*}
    \seq{ \mscrS(x) }_{x \in X}
  \end{equation*}
  be an indexed family of families of subsets of \( X \) that satisfies the condition \ref{thm:topology_from_local_base/BP1} and \ref{thm:topology_from_local_base/BP2} (without \ref{thm:topology_from_local_base/BP3}).

  For each \( x \in X \), define
  \begin{equation*}
    \mscrB(x) \coloneqq \set*{ \bigcap \mscrS' \given \mscrS' \T{is a finite nonempty subfamily of} \mscrS(x) }.
  \end{equation*}

  Then the family \( \seq{ \mscrB(x) }_{x \in X} \) generates via \fullref{thm:topology_from_local_base} the coarsest topology \( \mscrT \) containing \( \mscrS(x) \) for every \( x \in X \) (but the local bases may not be minimal).

  Furthermore, \( \mscrB(x) \) is a \hyperref[def:topological_local_base]{local base} for \( \mscrT \) at \( x \).
\end{proposition}
\begin{proof}
  \SubProof{Proof that \( \mscrB \) satisfies \fullref{thm:topology_from_local_base}} We require \ref{thm:topology_from_local_base/BP1} and \ref{thm:topology_from_local_base/BP2} to hold, hence it remains only to show \ref{thm:topology_from_local_base/BP3}.

  Let \( U, V \in \mscrB(x) \). Both are intersections of finitely many elements of \( \mscrS(x) \), hence \( U \cap V \) also is. Then \( U \cap V \in \mscrB(x) \).

  \SubProof{Proof of minimality of \( \mscrT \)} The proof is similar to that in \fullref{thm:topology_from_subbase}.

  \SubProof{Proof that \( \mscrB(x) \) is a local subbase} Follows from \fullref{thm:topology_from_local_base} by noting that every member of \( \mscrB(x) \) is a finite intersection of members of \( \mscrS(x) \).
\end{proof}

\begin{proposition}\label{thm:local_subbase_can_generate_topology}
  Every family of \hyperref[def:topological_local_subbase]{local subbases} satisfies \ref{thm:topology_from_local_base/BP1} and \ref{thm:topology_from_local_base/BP2}.
\end{proposition}
\begin{proof}
  Both properties are proven in \fullref{thm:local_base_can_generate_topology} and the proof works here.
\end{proof}

\begin{definition}\label{def:topological_closure_operator}\mcite[thm. 1.1.6]{Engelking1989GeneralTopology}
  Let \( (X, \mscrT) \) be a topological space. Define the \term{topological closure} of \( A \) as the intersection of all closed sets containing \( A \):
  \begin{equation*}
    \begin{aligned}
      &\cl: \pow(X) \to \pow(X) \\
      &\cl(A) \coloneqq \bigcap \set{ X \setminus U \given U \in \mscrT \T{and} A \subseteq X \setminus U }.
    \end{aligned}
  \end{equation*}

  It is a \hyperref[def:moore_closure_operator]{Moore closure operator} in the sense of \fullref{def:moore_closure_operator}.
\end{definition}
\begin{comments}
  \item As an intersection of closed set, \( \cl(A) \) is closed. It is actually the smallest closed set containing \( A \) because it is a subset of every closed set containing \( A \). In particular, the \hyperref[def:function_fixed_point]{fixed points} of the operator are the closed sets of \( \mscrT \).
\end{comments}
\begin{defproof}
  \SubProof{Proof that \( H \) is \SubProofOf[def:extensive_function]{extensive}} Since \( A \) belongs to any closed superset, it follows that \( A \subseteq \cl(A) \).

  \SubProofOf[def:idempotent_function]{idempotence} Since \( \cl(A) \) is the smallest closed set containing \( \cl(A) \), we conclude that \( \cl(\cl(A)) = \cl(A) \).

  \SubProofOf[def:order_function/preserving]{monotonicity} If \( A \subseteq B \), then every closed superset of \( B \) is also a closed superset of \( A \). Hence, \( \cl(A) \subseteq \cl(B) \).
\end{defproof}

\begin{proposition}\label{thm:topology_from_closure_operator}
  Let \( X \) be an arbitrary set and let \( \cl: \pow(X) \to \pow(X) \) be a function that satisfies \term{Kuratowski's axioms}:
  \begin{thmenum}
    \thmitem[thm:topology_from_closure_operator/CO1]{CO1} \( \cl(\varnothing) = \varnothing \),
    \thmitem[thm:topology_from_closure_operator/CO2]{CO2} It is \hyperref[def:extensive_function]{extensive}: \( A \subseteq \cl(A) \),
    \thmitem[thm:topology_from_closure_operator/CO3]{CO3} It preserves finite unions: \( \cl(A \cup B) = \cl(A) \cup \cl(B) \),
    \thmitem[thm:topology_from_closure_operator/CO4]{CO4} It is \hyperref[def:binary_operation/idempotent]{idempotent}: \( \cl(\cl(A)) = \cl(A) \).
  \end{thmenum}

  Then the family
  \begin{equation*}
    \mscrF \coloneqq \set{ A \subseteq X \given A = \cl(A) }
  \end{equation*}
  generates via \fullref{thm:topology_from_closed_sets} the topology
  \begin{equation*}
    \mscrT \coloneqq \set{ X \setminus F \given F \in \mscrF }
  \end{equation*}
  for which \( \mscrF \) is the family of closed sets.

  Furthermore, \( \cl \) is the \hyperref[def:topological_closure_operator]{topological closure operator} on \( (X, \mscrT) \).
\end{proposition}
\begin{proof}
  \SubProofOf[def:order_function/preserving]{monotonicity} \ref{thm:topology_from_closure_operator/CO3} implies that, if \( A \subseteq B \), then
  \begin{equation*}
    \cl(B) = \cl(A \cup B \setminus A) = \cl(A) \cup \cl(B \setminus A).
  \end{equation*}

  Hence, \( \cl(A) \subseteq \cl(B) \).

  \SubProof{Proof that \( \mscrF \) satisfies \fullref{thm:topology_from_closed_sets}}

  \SubProofOf*{thm:topology_from_closed_sets/C1} \ref{thm:topology_from_closure_operator/CO1} implies that \( \varnothing \in \mscrF \). \ref{thm:topology_from_closure_operator/CO2} implies that \( X \subseteq \cl(X) \), hence \( X = \cl(X) \) and thus \( X \in \mscrF \).

  \SubProofOf*{thm:topology_from_closed_sets/C2} \ref{thm:topology_from_closure_operator/CO3} implies that, if \( F, G \in \mscrF \), then
  \begin{equation*}
    F \cup G = \cl(F) \cup \cl(G) = \cl(F \cup G),
  \end{equation*}
  and thus \( F \cup G \in \mscrF \).

  \SubProofOf*{thm:topology_from_closed_sets/C3} Let \( \mscrF' \subseteq \mscrF \). \ref{thm:topology_from_closure_operator/CO2} implies that
  \begin{equation*}
    \bigcap \mscrF' \subseteq \cl\parens*{ \bigcap \mscrF' }.
  \end{equation*}

  Conversely, monotonicity implies that, for any \( F_0 \in \mscrF' \),
  \begin{equation*}
    \cl\parens*{ \bigcap \mscrF' }
    \subseteq
    \cl(F_0)
    =
    F_0.
  \end{equation*}

  Therefore,
  \begin{equation*}
    \cl\parens*{ \bigcap \mscrF' } \subseteq \bigcap \mscrF'.
  \end{equation*}

  \SubProof{Proof that \( \cl \) is the topological closure operator} Fix a set \( A \) and let \( \mscrF' \subset \mscrF \) be the family of closed sets containing \( A \). Then
  \begin{equation*}
    \cl(A) \subseteq \cl\parens*{ \bigcap \mscrF' } = \bigcap \mscrF'.
  \end{equation*}

  \ref{thm:topology_from_closure_operator/CO2} implies that \( A \subseteq \cl(A) \). \ref{thm:topology_from_closure_operator/CO4} implies that \( \cl(A) \) is a closet set. Hence, \( \cl(A) \) belongs to \( \mscrF' \) and
  \begin{equation*}
    \bigcap \mscrF' \subseteq \cl(A).
  \end{equation*}

  We conclude that \( \cl(A) \) is the closure operator for the induced topology.
\end{proof}

\begin{proposition}\label{thm:topological_closure_operator_can_generate_topology}
  Every \hyperref[def:topological_closure_operator]{topological closure operator} satisfies \ref{thm:topology_from_closure_operator/CO1} -- \ref{thm:topology_from_closure_operator/CO4}.
\end{proposition}
\begin{proof}
  \SubProofOf{thm:topology_from_closure_operator/CO1} Since \( \varnothing \) is closed, \( \cl(\varnothing) = \varnothing \).
  \SubProofOf{thm:topology_from_closure_operator/CO2} Already proven in \fullref{def:topological_closure_operator}.
  \SubProofOf{thm:topology_from_closure_operator/CO3} Fix sets \( A \) and \( B \). Note that \( \cl(A \cup B) \) is a closed set containing both \( A \) and \( B \), hence
  \begin{align*}
    \cl(A) \subseteq \cl(A \cup B)
    &&
    \cl(B) \subseteq \cl(A \cup B).
  \end{align*}

  Then
  \begin{equation*}
    \cl(A) \cup \cl(B) \subseteq \cl(A \cup B).
  \end{equation*}

  Conversely, \( \cl(A) \cup \cl(B) \) is a closed set containing both \( A \) and \( B \), and hence \( A \cup B \), thus
  \begin{equation*}
    \cl(A \cup B) \subseteq \cl(A) \cup \cl(B).
  \end{equation*}

  We conclude that \( \cl \) preserves finite unions.

  \SubProofOf{thm:topology_from_closure_operator/CO4} Already proven in \fullref{def:topological_closure_operator}.
\end{proof}

\begin{definition}\label{def:topological_interior_operator}\mcite[thm. 1.1.6]{Engelking1989GeneralTopology}
  Let \( (X, \mscrT) \) be a topological space. Define the \term{interior} of \( A \) as the union of all open sets contained in \( A \):
  \begin{equation*}
    \begin{aligned}
      &\Int: \pow(X) \to \pow(X) \\
      &\Int(A) \coloneqq \bigcup \set{ U \given U \in \mscrT \T{and} U \subseteq A }.
    \end{aligned}
  \end{equation*}

  Points that belong to \( \Int(A) \) are called \term{interior points} of \( A \). \Fullref{thm:properties_via_bases/interior} provides a characterization of interior points.

  The interior is not a \hyperref[def:moore_closure_operator]{closure operator} --- it is \hyperref[def:order_function/preserving]{monotone} and \hyperref[def:binary_operation/idempotent]{idempotent}, but instead of being \hyperref[def:extensive_function]{extensive}, it satisfies \ref{thm:topology_from_interior_operator/IO2}. It does have a deep relationship with the \hyperref[def:topological_closure_operator]{topological closure operator} -- see \fullref{thm:interior_closure_complement}.
\end{definition}
\begin{comments}
  \item As a union of open sets, \( \Int(A) \) is open. It is actually the largest open set containing \( A \) because it is a superset of every closed set contained in \( A \). In particular, the \hyperref[def:function_fixed_point]{fixed points} of the operator are the open sets of \( \mscrT \).
\end{comments}
\begin{defproof}
  \SubProof{Proof of \( \Int(A) \subseteq A \)} Since \( A \) contains any open subset, it follows that \( \Int(A) \subseteq A \).

  \SubProofOf[def:idempotent_function]{idempotence} Since \( \Int(A) \) is the largest an open set contained in \( \Int(A) \), we conclude that \( \Int(\Int(A)) = \Int(A) \)

  \SubProofOf[def:order_function/preserving]{monotonicity} If \( A \subseteq B \), then every open subset of \( A \) is also an open subset of \( B \). Hence, \( \Int(A) \subseteq \Int(B) \).
\end{defproof}

\begin{proposition}\label{thm:interior_closure_complement}
  The \hyperref[def:topological_closure_operator]{topological closure operator} and the \hyperref[def:topological_interior_operator]{topological interior operator} on the same space are related as follows:
  \begin{align}
    X \setminus \Int(A) = \cl(X \setminus A)
    &&
    X \setminus \cl(A) = \Int(X \setminus A)
    \label{eq:thm:interior_closure_complement}
  \end{align}
\end{proposition}
\begin{proof}
  We have
  \begin{balign*}
    X \setminus \Int(A)
    &=
    X \setminus \bigcup \set[\Big]{ U \given* U \in \mscrT \T{and} U \subseteq A }
    \reloset {\eqref{eq:thm:de_morgans_laws_for_sets/complement_of_union}} = \\ &=
    \bigcap \set[\Big]{ X \setminus U \given* U \in \mscrT \T{and} U \subseteq A }
    = \\ &=
    \bigcap \set[\Big]{ X \setminus U \given* U \in \mscrT \T{and} X \setminus A \subseteq X \setminus U }
    = \\ &=
    \cl(X \setminus A).
  \end{balign*}

  The other equality is obtained by noting that
  \begin{equation*}
    X \setminus \cl(A)
    =
    X \setminus (X \setminus \Int(A))
    \reloset {\ref{thm:set_difference/double_difference}} =
    \Int(A).
  \end{equation*}
\end{proof}

\begin{proposition}\label{thm:topology_from_interior_operator}
  Let \( X \) be an arbitrary set and let \( \Int: \pow(X) \to \pow(X) \) be a function that satisfies the conditions:
  \begin{thmenum}
    \thmitem[thm:topology_from_interior_operator/IO1]{IO1} \( \Int(X) = X \)
    \thmitem[thm:topology_from_interior_operator/IO2]{IO2} \( \Int(A) \subseteq A \)
    \thmitem[thm:topology_from_interior_operator/IO3]{IO3} It preserves finite intersections: \( \Int(A \cap B) = \Int(A) \cap \Int(B) \)
    \thmitem[thm:topology_from_interior_operator/IO4]{IO4} It is \hyperref[def:binary_operation/idempotent]{idempotent}: \( \Int(\Int(A)) = \Int(A) \).
  \end{thmenum}

  Then the family
  \begin{equation*}
    \mscrT \coloneqq \set{ A \subseteq X \given A = \cl(A) }
  \end{equation*}
  is a topology on \( X \).

  Furthermore, \( \Int \) is the \hyperref[def:topological_interior_operator]{topological interior operator} on \( (X, \mscrT) \).
\end{proposition}
\begin{proof}
  Note that \( \cl(A) \coloneqq X \setminus \Int(A) \) satisfies \ref{thm:topology_from_interior_operator/IO1} -- \ref{thm:topology_from_interior_operator/IO4}. Then \( \mscrT \) is a topology, \( \cl(A) \) is the closure operator for \( \mscrT \), and \fullref{thm:interior_closure_complement} implies that \( \Int(A) \) is the interior operator for \( \mscrT \).
\end{proof}

\begin{proposition}\label{thm:topological_interior_operator_can_generate_topology}
  Every \hyperref[def:topological_interior_operator]{topological interior operator} satisfies \ref{thm:topology_from_interior_operator/IO1} -- \ref{thm:topology_from_interior_operator/IO4}.
\end{proposition}
\begin{proof}
  Follows from \fullref{thm:interior_closure_complement} and \fullref{thm:topological_closure_operator_can_generate_topology}.
\end{proof}

\begin{proposition}\label{thm:topology_is_heyting_algebra}
  The topology \( \mscrT \) of a \hyperref[def:topological_space]{topological space} \( (X, \mscrT) \) is a \hyperref[def:complete_lattice]{complete} \hyperref[def:heyting_algebra]{Heyting algebra}.

  Explicitly:
  \begin{thmenum}
    \thmitem{thm:topology_is_heyting_algebra/join} \hyperref[def:lattice/join]{Arbitrary joins} are given by \hyperref[def:basic_set_operations/union]{unions}.

    \thmitem{thm:topology_is_heyting_algebra/meet} \hyperref[def:lattice/meet]{Finite meets} are given by \hyperref[def:basic_set_operations/intersection]{intersections}.

    \thmitem{thm:topology_is_heyting_algebra/top} The \hyperref[def:extremal_points/top_and_bottom]{top element} is the entire domain \( X \).

    \thmitem{thm:topology_is_heyting_algebra/bottom} The \hyperref[def:extremal_points/top_and_bottom]{bottom element} is the empty set.

    \thmitem{thm:topology_is_heyting_algebra/relative_pseudocomplement} The \hyperref[def:heyting_algebra]{relative pseudocomplement} \( U \rightarrow V \) is then
    \begin{equation*}
      \bigcup\set[\Big]{ A \in \mscrT \given \underbrace{A \cap U}_{A \setminus (X \setminus U)} \subseteq V }
      =
      \bigcup\set[\Big]{ A \in \mscrT \given A \subseteq V \cup (X \setminus U) }
      =
      \Int((X \setminus U) \cup V).
    \end{equation*}

    This is similar to \fullref{thm:classical_equivalences/conditional_as_disjunction} despite the fact that arbitrary topologies are not Boolean algebras.

    \thmitem{thm:topology_is_heyting_algebra/pseudocomplement} As a result, the \hyperref[def:heyting_algebra/pseudocomplement]{pseudocomplement} is
    \begin{equation*}
      \oline U = \Int(X \setminus U).
    \end{equation*}
  \end{thmenum}
\end{proposition}
\begin{comments}
  \item This can be used for topological propositional semantics defined in \fullref{def:propositional_semantics/topological}.
\end{comments}
\begin{proof}
  Trivial.
\end{proof}

\begin{definition}\label{def:topological_boundary_operator}
  Let \( (X, \mscrT) \) be a topological space. Define the \term{boundary operator}
  \begin{equation*}
    \begin{aligned}
      &\fr: \pow(X) \to \pow(X), \\
      &\fr(A) \coloneqq \cl(A) \setminus \Int(A).
    \end{aligned}
  \end{equation*}

  Points that belong to \( \fr(A) \) are called \term{boundary points} of \( A \). \Fullref{thm:properties_via_bases/boundary} provides a characterization of boundary points.
\end{definition}

\begin{proposition}\label{thm:def:topological_boundary_operator}
  The \hyperref[def:topological_boundary_operator]{topological boundary} has the following basic properties:
  \begin{thmenum}
    \thmitem{thm:def:topological_boundary_operator/intersection} \( \fr(A) = \cl(A) \cap \cl(X \setminus A) \).
    \thmitem{thm:def:topological_boundary_operator/complement} \( \fr(A) = \fr(X \setminus A) \).
    \thmitem{thm:def:topological_boundary_operator/closed} \( \fr(A) \) is a closed set.
  \end{thmenum}
\end{proposition}
\begin{proof}
  \SubProofOf{thm:def:topological_boundary_operator/intersection} We have
  \begin{balign*}
    \fr(A)
    &=
    \cl(A) \setminus \Int(A)
    \reloset {\ref{thm:set_difference/intersection}} = \\ &=
    \cl(A) \cap (X \setminus \Int(A))
    \reloset {\ref{thm:interior_closure_complement}} = \\ &=
    \cl(A) \cap \cl(X \setminus A).
  \end{balign*}

  \SubProofOf{thm:def:topological_boundary_operator/complement} Follows from \fullref{thm:def:topological_boundary_operator/intersection}.

  \SubProofOf{thm:def:topological_boundary_operator/closed} \Fullref{thm:def:topological_boundary_operator/intersection} represents \( \fr(A) \) as the intersection of two closed sets. Hence, \( \fr(A) \) is itself closed.
\end{proof}

\begin{definition}\label{def:set_cluster_point}\mcite[24]{Engelking1989GeneralTopology}
  Let \( (X, \mscrT) \) be a topological space. We say that the point \( x \in X \) is a \term{cluster point} or an \term{accumulation point} of the set \( A \subseteq X \) if \( x \in \cl(A \setminus \set{ x }) \).

  Points of \( A \) that are not cluster points are said to be \term{isolated points} of \( A \).

\end{definition}
\begin{comments}
  \item It is not necessary for \( x \) to belong to \( A \) in order for \( x \) to be a cluster point of \( A \).
  \item This is related to a similar notion for \hyperref[def:topological_net]{nets}, \fullref{def:net_cluster_point}, via \fullref{thm:closed_iff_contains_all_net_cluster_points}.
\end{comments}

\begin{proposition}\label{thm:def:set_cluster_point}
  \hyperref[def:set_cluster_point]{Cluster points} and \hyperref[def:set_cluster_point]{isolated points} in \( (X, \mscrT) \) have the following basic properties:
  \begin{thmenum}
    \thmitem{thm:def:set_cluster_point/isolated} A point \( x \in A \) is isolated if and only if \( A \setminus \set{ x } \) is closed.

    \thmitem{thm:def:set_cluster_point/discrete} A topological space is \hyperref[def:discrete_topology]{discrete} if and only if all of its points are \hyperref[def:set_cluster_point]{isolated}.

    This motivates \fullref{def:discrete_set}.
  \end{thmenum}
\end{proposition}
\begin{proof}
  \SubProofOf{thm:def:set_cluster_point/isolated} Closed sets are fixed points of the closure operator.
  \SubProofOf{thm:def:set_cluster_point/discrete} \Fullref{thm:def:set_cluster_point/isolated} implies that \( x \) is isolated in \( X \) if and only if \( X \setminus \set{ x } \) is closed, i.e. if \( \set{ x } \) is open. \Fullref{ex:def:topological_base/discrete} shows that \( \set[\Big]{ \set{ x } } \) is a basis of a discrete space.
\end{proof}

\begin{example}\label{ex:def:set_cluster_point}
  We list several examples of \hyperref[def:set_cluster_point]{cluster points} of sets:
  \begin{thmenum}
    \thmitem{ex:def:set_cluster_point/sierpinski} In the \hyperref[def:sierpinski_space]{Sierpi\'nski space}, \( \bot \) is an isolated point because
    \begin{equation*}
      \cl(\set{ \top, \bot } \setminus \set{ \bot }) = \cl(\set{ \top }) = \set{ \top },
    \end{equation*}
    while \( \top \) is a cluster point because
    \begin{equation*}
      \cl(\set{ \bot }) = \set{ \top, \bot }.
    \end{equation*}

    \thmitem{ex:def:set_cluster_point/euclidean} If \( A \) is a finite set in an \hyperref[def:metric_space]{metric space}, then \( A \) is a \hyperref[thm:def:set_cluster_point/discrete]{discrete set}. That is, all of its points are isolated. This follows from \fullref{thm:def:set_cluster_point/isolated} by noting that any finite set is closed (a metric space is \ref{def:separation_axioms/T1}).
  \end{thmenum}
\end{example}

\begin{definition}\label{def:derived_set}\mcite[25]{Engelking1989GeneralTopology}
  We define the \term{derived set} of a point \( A \) as the set of all cluster points of \( A \). We will denote it via \( \drv(A) \).
\end{definition}

\begin{proposition}\label{thm:union_with_derived_set}
  For any set \( A \) in a topological space,
  \begin{equation*}
    A \cup \drv(A) = \cl(A).
  \end{equation*}
\end{proposition}
\begin{proof}
  Clearly \( A \subseteq \cl(A) \). Furthermore,
  \begin{equation*}
    \drv(A) \subseteq \bigcup_{x \in X} \cl(A \setminus \set{ x }) \subseteq \cl(A).
  \end{equation*}

  Hence,
  \begin{equation*}
    A \cup \drv(A) \subseteq \cl(A).
  \end{equation*}

  Now let \( x \in \cl(A) \). If \( x \not\in A \), then \( A = A \setminus \set{ x } \), hence \( x \in \cl(A) = \cl(A \setminus \set{ x }) \). That is, if \( x \in \cl(A) \), it is a point in \( A \) or a point of \( \drv(A) \) (or both). Therefore,
  \begin{equation*}
    \cl(A) \subseteq A \cup \drv(A).
  \end{equation*}
\end{proof}

\begin{corollary}\label{thm:cluster_point_characterization}
  A set is closed if and only if it contains all of its cluster points.
\end{corollary}
\begin{comments}
  \item Compare this result to \fullref{thm:cluster_point_iff_in_closure}.
\end{comments}
\begin{proof}
  \Fullref{thm:union_with_derived_set} implies that \( \cl(A) = A \cup \drv(A) \).

  If \( A \) is closed, then \( A = \cl(A) = A \cup \drv(A) \) and thus \( \drv(A) \) is a subset of \( A \).

  Conversely, if \( \drv(A) \subseteq A \), then \( \cl(A) = A \cup \drv(A) = A \), implying that \( A \) is closed.
\end{proof}

\begin{definition}\label{def:topologically_dense_set}\mcite[25]{Engelking1989GeneralTopology}
  We say that the set \( A \) in a topological space is \term{dense} in \( B \) if \( \cl(A) = B \) and \term{nowhere dense} if \( \Int(\cl(A)) = \varnothing \).

  If we do not specify the set \( B \), for example if we say \enquote{\( A \) is dense}, we assume that \( B \) is the entire space.
\end{definition}

\begin{proposition}\label{thm:def:topologically_dense_set}
  \hyperref[def:topologically_dense_set]{Dense} and \hyperref[def:topologically_dense_set]{nowhere dense} sets have the following basic properties:
  \begin{thmenum}
    \thmitem{thm:def:topologically_dense_set/no_proper_closed_set} A set is dense if and only if no proper closed set contains it.
    \thmitem{thm:def:topologically_dense_set/dense_superset} A superset of a dense set is dense.
    \thmitem{thm:def:topologically_dense_set/nowhere_dense_subset} A subset of a nowhere dense set is nowhere dense.
    \thmitem{thm:def:topologically_dense_set/nowhere_dense_is_in_boundary} Nowhere dense sets are entirely contained in their boundaries.
    \thmitem{thm:def:topologically_dense_set/complement} A set is nowhere dense if and only if the complement of its closure is dense.
  \end{thmenum}
\end{proposition}
\begin{proof}
  \SubProofOf{thm:def:topologically_dense_set/no_proper_closed_set} \( \cl(A) \) is the entire space if and only if it is the smallest closed set containing \( A \).

  \SubProofOf{thm:def:topologically_dense_set/dense_superset} The closure operator is monotone.

  \SubProofOf{thm:def:topologically_dense_set/nowhere_dense_subset} Let \( A \) be a nowhere dense set and let \( B \subseteq A \). Then
  \begin{equation*}
    \Int(\cl(B))
    \subseteq
    \Int(\cl(A))
    =
    \varnothing,
  \end{equation*}
  therefore \( B \) is also nowhere dense.

  \SubProofOf{thm:def:topologically_dense_set/nowhere_dense_is_in_boundary} If \( A \) is nowhere dense, its interior is empty. Then
  \begin{equation*}
    A \subseteq \cl(A) = \Int(A) \cup \fr(A) = \fr(A).
  \end{equation*}

  \SubProofOf{thm:def:topologically_dense_set/complement} \Fullref{thm:interior_closure_complement} implies that
  \begin{equation*}
    \cl(X \setminus \cl(A)) = X \setminus \Int(\cl(A)).
  \end{equation*}

  Then \( \cl(X \setminus \cl(A)) = X \) if and only if \( \Int(\cl(A)) = \varnothing \). That is, the complement of \( \cl(A) \) is dense if and only if \( A \) is nowhere dense.
\end{proof}

\begin{example}\label{ex:def:topologically_dense_set}
  We list several examples of \hyperref[def:topologically_dense_set]{dense} and \hyperref[def:topologically_dense_set]{nowhere dense} sets:
  \begin{thmenum}
    \thmitem{ex:def:topologically_dense_set/discrete} Under the \hyperref[def:discrete_topology]{discrete topology}, no proper subset is dense because each of them is closed. There are no nonempty nowhere dense sets because \( \Int(\cl(A)) = A \) for every set \( A \).

    \thmitem{ex:def:topologically_dense_set/indiscrete} On the other end, under the \hyperref[def:indiscrete_topology]{indiscrete topology}, every nonempty subset is dense because none of them is closed except for the empty set. There are no nonempty nowhere dense sets because \( \Int(\cl(A)) \) is the entire space.

    \thmitem{ex:def:topologically_dense_set/metric} Every \hyperref[def:metric_space]{metric space} is dense in its \hyperref[thm:metric_space_completion]{completion} as shown in \fullref{thm:metric_space_is_dense_in_completion}.

    \thmitem{ex:def:topologically_dense_set/finite} Let \( A \) be an \hyperref[def:set_countability/at_most_countable]{at most countable} set in some \hyperref[def:euclidean_space]{Euclidean spaces}.

    Every set in the local base \eqref{def:metric_topology/real_base} is uncountable, hence, \( A \) does not contain any of them. Then \( A \) does not contain any open set. Since \( A \) is closed, \( A = \cl(A) \), and \( \Int(A) = \varnothing \) implies that \( \Int(\cl(A)) = \varnothing \). Therefore, \( A \) is nowhere dense.

    \Fullref{thm:properties_via_bases/nowhere_dense} generalizes similar reasoning to arbitrary topological spaces.
  \end{thmenum}
\end{example}

\begin{definition}\label{def:topological_space_density}\mcite[12]{Engelking1989GeneralTopology}
  We define the \term{density} of the topological space \( (X, \mscrT) \) as the \hyperref[def:cardinal]{cardinal}
  \begin{equation*}
    d(X, \mscrT) \coloneqq \min \set{ \card D \given D \T{is dense} }.
  \end{equation*}

  We simply write \( d(X) \) when the topology is clear from the context.

  Spaces for which \( d(X) \leq \hyperref[def:aleph_hierarchy]{\aleph_0} \) are said to be \term{separable}.
\end{definition}
\begin{comments}
  \item See \fullref{thm:topological_space_countability} for how weight relates to other measurements of the space's \enquote{size}.
\end{comments}

\begin{example}\label{ex:def:topological_space_density}
  We list several examples of \hyperref[def:topological_space_density]{topological space density}:
  \begin{thmenum}
    \thmitem{ex:def:topological_space_density/discrete} Under the \hyperref[def:discrete_topology]{discrete topology}, the density of a space equals its cardinality. Indeed, \fullref{ex:def:topologically_dense_set/discrete} implies that only the space itself is dense.

    Then the space is separable if and only if it is countable.

    \thmitem{ex:def:topological_space_density/indiscrete} On the other end, under the \hyperref[def:indiscrete_topology]{indiscrete topology}, \fullref{ex:def:topologically_dense_set/indiscrete} shows that every nonempty set is dense, hence the density of the space is \( 1 \).
  \end{thmenum}
\end{example}

\begin{proposition}\label{thm:topological_space_countability}
  For any topological space \( (X, \mscrT) \), the \hyperref[def:topological_space_weight]{weight} \( w(X) \) \hyperref[def:equinumerosity]{dominates} both the \hyperref[def:topological_space_character]{character} \( \chi(X) \) and the \hyperref[def:topological_space_density]{density} \( d(X) \).
\end{proposition}
\begin{comments}
  \item In particular, every \hyperref[def:topological_space_weight]{second-countable space} is \hyperref[def:topological_space_character]{first-countable} and \hyperref[def:topological_space_density]{separable}.
\end{comments}
\begin{proof}
  \SubProof{Proof that \( \chi(X) \leq w(X) \)} Let \( \mscrB \) be \hyperref[def:topological_base]{base} of minimal cardinality \( w(X) \). Then \( \card \mscrB(x) \leq \card \mscrB \) for any
  \begin{equation*}
    \mscrB(x) = \set{ U \in \mscrB \given x \in U }.
  \end{equation*}

  \Fullref{thm:global_base_to_local_base} implies that \( \seq{ \mscrB(x) }_{x \in X} \) is a family of local bases. Then
  \begin{equation*}
    \chi(X, \mscrT, x) \leq \card \mscrB(x) \leq \card \mscrB = w(X)
  \end{equation*}
  and
  \begin{equation*}
    \chi(X, \mscrT) = \sup_{x \in X} \chi(X, \mscrT, x) \leq \sup_{x \in X} \card \mscrB(x) \leq \card \mscrB = w(X).
  \end{equation*}

  \SubProof{Proof that \( d(X) \leq w(X) \)} The \hyperref[def:zfc/choice]{axiom of choice} gives us a \hyperref[def:choice_function]{choice function}
  \begin{equation*}
    c: \mscrB \setminus \set{ \varnothing } \to X.
  \end{equation*}

  Let \( D \) be the image of this function. Clearly
  \begin{equation*}
    \card D = \card \mscrB \setminus \set{ \varnothing } \leq \card \mscrB = w(X).
  \end{equation*}

  We will show that \( D \) is \hyperref[def:topologically_dense_set]{dense} in \( X \). Indeed, if \( F \) is a proper closed set, then \( X \setminus F \) is open, hence it contains a set \( U \) from \( \mscrB \). The point \( c(U) \) then belongs to \( D \), implying that \( F \) does not belong to \( D \). The only closed set containing \( D \) is \( X \) itself, and \fullref{thm:def:topologically_dense_set/no_proper_closed_set} implies that \( D \) is dense.

  Therefore,
  \begin{equation*}
    d(X) \leq \card D \leq w(X).
  \end{equation*}
\end{proof}

\begin{proposition}\label{thm:properties_via_bases}
  We present characterizations of several definitions entirely via \hyperref[def:topological_local_base]{local bases}. Let \( \set{ \mscrB(x) }_{x \in X} \) be a family of local bases of the space \( (X, \mscrT) \).

  \begin{thmenum}
    \thmitem{thm:properties_via_bases/open} A set \( A \) is open if and only if, for every point \( x \in A \), there must exist a neighborhood in \( \mscrB(x) \) contained in \( A \).

    \thmitem{thm:properties_via_bases/boundary} \( x \) belongs to the \hyperref[def:topological_boundary_operator]{boundary} of \( A \) if and only if every neighborhood in \( \mscrB(x) \) intersects both \( A \) and \( X \setminus A \).

    \thmitem{thm:properties_via_bases/closure} \( x \) belongs to the \hyperref[def:topological_closure_operator]{closure} of \( A \) if and only if every neighborhood in \( \mscrB(x) \) intersects \( A \).

    \thmitem{thm:properties_via_bases/interior} \( x \) belongs to the \hyperref[def:topological_interior_operator]{interior} of \( A \) if and only if \( \mscrB(x) \) contains a set entirely in \( A \).

    \thmitem{thm:properties_via_bases/cluster} \( x \) is a \hyperref[def:set_cluster_point]{cluster point} of \( A \) if and only if every neighborhood in \( \mscrB(x) \) intersects \( A \setminus \set{ x } \).

    \thmitem{thm:properties_via_bases/isolated} \( x \) is an \hyperref[def:set_cluster_point]{isolated point} of \( A \) if and only \( \mscrB(x) \) contains a set \( U \) disjoint from \( A \setminus \set{ x } \).

    \thmitem{thm:properties_via_bases/dense} The set \( A \) is \hyperref[def:topologically_dense_set]{dense} in \( B \) if and only if, for every point \( x \in B \), every neighborhood in \( \mscrB(x) \) intersects \( A \).

    \thmitem{thm:properties_via_bases/nowhere_dense} The set \( A \) is \hyperref[def:topologically_dense_set]{nowhere dense} if and only if, for every point \( x \in A \), all members of \( \mscrB(x) \) are supersets of \( A \).

    This statement also holds of we consider \hyperref[def:topological_local_subbase]{local subbases} rather than local bases.
  \end{thmenum}
\end{proposition}
\begin{proof}
  \SubProofOf{thm:properties_via_bases/open}

  \SufficiencySubProof* Suppose that \( A \) is open and let \( x \in A \). Then \( A \) is a neighborhood of \( x \).

  Since \( \mscrB(x) \) is a local base \( x \), there exists some neighborhood \( U \in \mscrB(x) \) such that \( U \subseteq A \).

  \NecessitySubProof* Suppose that, for every point \( x \in A \), there must exist a neighborhood in \( \mscrB(x) \) contained in \( A \). Let \( \seq{ U_x }_{x \in A} \) be a collection of such neighborhoods. Then
  \begin{equation*}
    A = \bigcup_{x \in A} U_x,
  \end{equation*}
  implying that \( A \) is open as a union of open sets.

  \SubProofOf{thm:properties_via_bases/boundary}

  \SufficiencySubProof* Let \( x \in \fr(A) \).

  Aiming for a contradiction, suppose that there is a neighborhood \( U \in \mscrB(x) \) that is disjoint from \( A \). Then \( A \subseteq X \setminus U \). Since \( X \setminus U \) is closed, it follows that \( \cl(A) \subseteq X \setminus U \). The point \( x \) is not in \( X \setminus U \), therefore \( x \not\in \cl(A) \). But this contradicts \fullref{thm:def:topological_boundary_operator/intersection} because we have assumed that \( x \in \fr(A) \), and \( \fr(A) \subseteq \cl(A) \).

  The obtained contradiction shows that every neighborhood in \( \mscrB(x) \) intersects \( A \).

  By passing to complements, we can reuse this to prove that every neighborhood of \( x \) intersects \( X \setminus A \) using \fullref{thm:interior_closure_complement}.

  \NecessitySubProof* Suppose that every neighborhood in \( \mscrB(x) \) intersects both \( A \) and \( X \setminus A \). \Fullref{thm:properties_via_bases/open} then implies that \( x \) belongs to neither \( \Int(A) \) nor \( \Int(X \setminus A) \). Therefore,
  \begin{equation*}
    x
    \in
    X \setminus \parens[\Big]{ \Int(A) \cup \Int(X \setminus A) }
    \reloset {\eqref{eq:thm:de_morgans_laws_for_sets/complement_of_union}} =
    \cl(X \setminus A) \cap \cl(A)
    =
    \fr(A).
  \end{equation*}

  \SubProofOf{thm:properties_via_bases/closure}
  \SufficiencySubProof* Let \( x \in \cl(A) \), and let \( U \in \mscrB(x) \).

  \Fullref{thm:def:topological_boundary_operator/intersection} implies that \( \cl(A) = A \cup \fr(A) \).
  \begin{itemize}
    \item If \( x \in A \), then \( x \in U \cap A \), and hence \( U \) intersects \( A \).
    \item If \( x \in \fr(A) \), then \fullref{thm:properties_via_bases/boundary} implies that \( U \) intersects \( A \).
  \end{itemize}

  Therefore, every neighborhood in \( \mscrB(x) \) intersects \( A \).

  \NecessitySubProof* Let \( x \in X \) and suppose that every neighborhood in \( \mscrB(x) \) intersects \( A \).

  \begin{itemize}
    \item If \( x \in A \), clearly \( x \in \cl(A) \).
    \item If \( x \in X \setminus A \), every neighborhood in \( \mscrB(x) \) intersects both \( A \) and \( X \setminus A \), and \fullref{thm:properties_via_bases/boundary} implies that \( x \in \fr(A) \).
  \end{itemize}

  In both cases, \( x \in \cl(A) \).

  \SubProofOf{thm:properties_via_bases/interior} \Fullref{thm:interior_closure_complement} implies that \( x \in \Int(A) \) if and only if \( x \in X \setminus \cl(X \setminus A) \). \Fullref{thm:properties_via_bases/closure} implies that \( x \in X \setminus \cl(X \setminus A) \) if and only if some neighborhood in \( \mscrB(x) \) is disjoint from \( X \setminus A \), i.e. entirely in \( A \).

  \SubProofOf{thm:properties_via_bases/cluster} Follows from \fullref{thm:properties_via_bases/closure}.

  \SubProofOf{thm:properties_via_bases/isolated} \hyperref[def:conditional_formula/inverse]{Inverse} of \fullref{thm:properties_via_bases/cluster}.

  \SubProofOf{thm:properties_via_bases/dense} Follows from \fullref{thm:properties_via_bases/closure}.

  \SubProofOf{thm:properties_via_bases/nowhere_dense}

  \SufficiencySubProof* Suppose that \( A \) is nowhere dense. Let \( x \in A \) and let \( U \) be a neighborhood in \( \mscrB(x) \). Since \( \Int(\cl(A)) = \varnothing \), \( U \) is not contained in \( \cl(A) \).

  Generalizing on \( U \), it follows that every neighborhood in \( \mscrB(x) \) is a superset of \( \cl(A) \).

  \NecessitySubProof* Suppose that, for every point \( x \in A \), all members of \( \mscrB(x) \) are supersets of \( \cl(A) \). Then \( \cl(A) \) contains no nonempty open sets. Therefore, its interior is empty.

  If \( \mscrB(x) \) is only a local subbase, then the result also follows because any intersection of members of \( \mscrB(x) \) is also a superset of \( \cl(A) \).
\end{proof}

  \subsection{Net convergence}\label{subsec:net_convergence}

\paragraph{Nets}

\begin{definition}\label{def:topological_net}\mcite[65]{Kelley1975}
  A \term{net} or \term{generalized sequence} is a family of elements of a \hyperref[def:topological_space]{topological space} \hyperref[def:cartesian_product/indexed_family]{indexed} by a \hyperref[def:directed_set]{directed set}.

  We reuse the notation for indexed families from \fullref{def:cartesian_product/indexed_family}:
  \begin{equation*}
    \set{ x_k }_{k \in \mscrK}
  \end{equation*}
\end{definition}
\begin{comments}
  \item This definition does not require a topology, but, unlike sequences, this concept of a net is only discussed in the context of topological spaces.
\end{comments}

\begin{definition}\label{def:net_eventually_in}\mcite[65]{Kelley1975}
  We say that the \hyperref[def:topological_net]{net} \( \set{ x_k }_{k \in \mscrK} \) is \term{eventually in} the set \( A \) if there exists an index \( k_0 \in \mscrK \) such that, for every \( k \geq k_0 \), we have \( x_k \in A \).
\end{definition}

\begin{proposition}\label{thm:eventually_in_implies_nonempty}
  If the net \( \set{ x_k }_{k \in \mscrK} \) is \hyperref[def:net_eventually_in]{eventually in} some set \( A \), then both \( \mscrK \) and \( A \) are nonempty.
\end{proposition}
\begin{proof}
  By definition, there exists some index \( k_0 \in \mscrK \) such that \( x_{k_0} \in A \).
\end{proof}

\begin{proposition}\label{thm:sequence_eventally_in}
  An \hyperref[def:sequence]{infinite sequence} is eventually in some set if and only if only finitely many sequence elements are outside the set.
\end{proposition}
\begin{proof}
  A sequence \( \seq{ x_k }_{k=1}^\infty \) is eventually in \( A \) if there exists a positive integer \( k_0 \) such that \( x_k \in A \) whenever \( k \geq k_0 \). Since the positive integers are totally ordered, the condition is equivalent to requiring that there exists some upper limit for the set \( \set{ k \given x_k \not\in A } \) --- every such limit can be taken to be \( k_0 \). \Fullref{thm:natural_number_cofinal_subsets} implies that the latter condition is equivalent to requiring that only finitely many sequence elements are outside \( A \).
\end{proof}

\begin{definition}\label{def:net_frequently_in}\mcite[65]{Kelley1975}
  We say that the \hyperref[def:topological_net]{net} \( \set{ x_k }_{k \in \mscrK} \) is \term{frequently in} the set \( A \) if, for every index \( k_0 \in \mscrK \), there exists some \( k \geq k_0 \) such that \( x_k \in A \).
\end{definition}

\begin{proposition}\label{thm:empty_index_set_implies_frequently_in}
  A net \( \set{ x_k }_{k \in \varnothing} \) over the empty set is \hyperref[def:net_frequently_in]{frequently in} any set.
\end{proposition}
\begin{proof}
  Any condition with quantifier \enquote{for every index in \( \varnothing \)} is vacuously true.
\end{proof}

\begin{proposition}\label{thm:eventually_in_implies_frequently_in}
  If a net is \hyperref[def:net_eventually_in]{eventually in} a set, it is also \hyperref[def:net_frequently_in]{frequently in} it.
\end{proposition}
\begin{proof}
  Suppose that \( \seq{ x_k }_{k \in \mscrK} \) is eventually in \( A \). Then there exists some index \( k_e \in \mscrK \) such that \( x_k \in A \) for \( k \geq k_e \).

  Fix some arbitrary index \( k_f \in \mscrK \). Then, since \( \mscrK \) is directed, there must exist an upper bound \( k_u \) for \( k_e \) and \( k_f \).

  Since \( k_u \geq k_e \), we have \( x_{k_u} \in A \). Therefore, \( \seq{ x_k }_{k \in \mscrK} \) is frequently in \( A \).
\end{proof}

\begin{proposition}\label{thm:eventually_and_frequently_in}
  The net \( \seq{ x_k }_{k \in \mscrK} \) in the space \( X \) is \hyperref[def:net_eventually_in]{eventually in} \( A \) if and only if it is not \hyperref[def:net_frequently_in]{frequently in} \( X \setminus A \).
\end{proposition}
\begin{proof}
  \Fullref{thm:relativized_first_order_quantifiers_are_dual} implies that
  \begin{equation*}
    \qexists {k_0 \in \mscrK} \qforall {k \geq k_0} x_k \in A
  \end{equation*}
  if and only if
  \begin{equation*}
    \neg \qforall {k_0 \in \mscrK} \qexists {k \geq k_0} \neg (x_k \in A).
  \end{equation*}
\end{proof}

\begin{proposition}\label{thm:cofinal_iff_frequently_in}
  Let \( \mscrK \) be a directed set, let \( A \subseteq \mscrK \) be a subset and let \( \seq{ x_k }_{k \in \mscrK} \) be a net.

  Then \( A \) is \hyperref[def:cofinal_set]{cofinal} if and only if \( x \) is \hyperref[def:net_frequently_in]{frequently in} the \( \set{ x_a \given a \in A } \).
\end{proposition}
\begin{proof}
  The net \( \seq{ x_k }_{k \in \mscrK} \) is frequently in \( \set{ x_a \given a \in A } \) if
  \begin{equation*}
    \qforall {k_0 \in \mscrK} \qexists {a_0 \in \mscrK} k \leq a_0 \T{and} x_{a_0} \in A,
  \end{equation*}
  which can be rewritten as
  \begin{equation*}
    \qforall {k_0 \in \mscrK} \qexists {a_0 \in A} k \leq a_0.
  \end{equation*}

  This is exactly the condition for \( A \) to be cofinal in \( \mscrK \).
\end{proof}

\begin{example}\label{ex:def:topological_net}
  We list several examples of \hyperref[def:topological_net]{nets}:
  \begin{thmenum}
    \thmitem{ex:def:topological_net/empty} The most basic example of a net is the empty set. Indeed, an empty set is vacuously directed.

    It is also vacuously \hyperref[def:net_frequently_in]{frequently in} any set as a consequence of \fullref{thm:empty_index_set_implies_frequently_in}, but it is not \hyperref[def:net_eventually_in]{eventually in} any set because of \fullref{thm:eventually_in_implies_nonempty}.

    \thmitem{ex:def:topological_net/finite_sequence} Any \hyperref[def:sequence]{finite sequence} \( x_1, \ldots, x_n \) is a net.

    It is eventually in (and hence also frequently in) \( \set{ x_n } \) and any of its supersets.

    \thmitem{ex:def:topological_net/sequence} For a finite sequence it suffices for \( A \) to contain the last element, but for an infinite sequence \( x_1, x_2, x_3, \ldots \), this condition is more complicated. \Fullref{thm:cofinal_iff_frequently_in} implies that the sequence is frequently in some set \( A \) if and only if \( A \) contains a \hyperref[def:cofinal_set]{cofinal subset} of the sequence (when regarding the sequence as a well-ordered set).

    \thmitem{ex:def:topological_net/reverse} Given any point \( x_0 \), its \hyperref[def:neighborhood_filter]{neighborhood filter} \( \mscrT(x_0) \) is naturally a \hyperref[def:semilattice/lattice]{lattice} with respect to inclusion. Indeed, the union and intersection of any pair of neighborhoods of \( x_0 \) is again a neighborhood of \( x_0 \).

    We can also consider the \hyperref[def:semilattice/duality]{dual lattice} of \( \mscrT(x_0) \), that is, the \enquote{reverse inclusion lattice} where \( V \leq U \) if \( V \supseteq U \).

    The latter is useful as an index set of nets, for example in the proofs of \fullref{thm:def:net_limit_point/cluster_point_subnet_limit} and \fullref{thm:cluster_point_iff_in_closure} or the proof of equivalence in \fullref{def:local_convergence}.
  \end{thmenum}
\end{example}

\paragraph{Subnets}

It is unfortunately an established convention for subsequences to be defined differently compared to subnets. There are actually different definitions of subnets --- see \cite{Lehuta2009} --- but we will base our exposition on that of \incite[ch. 2]{Kelley1975}.

\begin{definition}\label{def:subsequence}\mcite[63]{Kelley1975}
  We say that
  \begin{equation}\label{eq:def:subsequence/sub}
    y_1, y_2, y_3, y_4, \ldots
  \end{equation}
  is a \term{subsequence} of
  \begin{equation}\label{eq:def:subsequence/original}
    x_1, x_2, x_3, x_4, \ldots
  \end{equation}
  if there exists a strictly increasing sequence of indices
  \begin{equation*}
    i_1 < i_2 < i_3 < i_4 < \cdots
  \end{equation*}
  such that \( y_k = x_{i_k} \) for every \( k = 1, 2, \ldots \).
\end{definition}
\begin{comments}
  \item While Kelley in \cite[63]{Kelley1975} calls this definition \enquote{unnecessarily stringent} before introducing nets, other authors like \incite[def. 3.5]{Rudin1976Principles} prefer this definition.

  \item If we regard \( x \) and \( y \) as functions from \( \BbbZ_{>0} \) to \( X \), then \( y \) is a subsequence of \( x \) if and only if there exists a \hyperref[def:order_function/preserving]{strictly increasing} inclusion \hyperref[def:function/endofunction]{endofunction} \( \iota: \BbbZ_{>0} \to \BbbZ_{>0} \) such that \( y = x \cdot \iota \).
\end{comments}

\begin{definition}\label{def:subnet}\mcite[70]{Kelley1975}
  We say that the net \( \seq{ y_m }_{m \in \mscrM} \) is a \term{subnet} of \( \seq{ x_k }_{k \in \mscrK} \) if there exists a family \( \seq{ k_m }_{m \in \mscrM} \) of indices from \( \mscrK \) such that
  \begin{thmenum}
    \thmitem[def:subnet/composition]{SN1} For every \( m \in \mscrM \), we have \( x_{k_m} = y_m \).

    This condition ensures that the subnet consists of members of the supernet.

    \thmitem[def:subnet/cofinality]{SN2} For every \( k_0 \in \mscrK \), there exists some index \( m_0 \in \mscrM \) such that, for every \( m \geq m_0 \), we have \( k_m \geq k_0 \).

    This condition ensures that the subnet \enquote{follows} the direction of the supernet. It involves only the index sets and not the actual nets.
  \end{thmenum}
\end{definition}
\begin{comments}
  \item \ref{def:subnet/cofinality} states that the subnet must be eventually in the tail \( \set{ x_k \given k \geq k_0 } \) for every \( k_0 \in \mscrK \).
\end{comments}

\begin{example}\label{ex:def:subnet}
  We list several examples of \hyperref[def:subnet]{subnets}:
  \begin{thmenum}
    \thmitem{ex:def:subnet/empty} The empty net has no empty subnet.

    \thmitem{ex:def:subnet/subsequence} Consider the \hyperref[def:sequence]{sequence}
    \begin{equation*}
      x_1, x_2, x_3, \ldots
    \end{equation*}

    The sequence
    \begin{equation*}
      x_1, x_3, x_5, \ldots
    \end{equation*}
    with only odd indices is a \hyperref[def:subsequence]{subsequence}, and so is
    \begin{equation*}
      x_2, x_3, x_4, \ldots
    \end{equation*}

    The above are also subnets.

    \thmitem{ex:def:subnet/padded_sequence} The following is a subnet but not subsequence:
    \begin{equation*}
      \underbrace{0}_{y_1}, \underbrace{x_1}_{y_2}, \underbrace{x_2}_{y_3}, \underbrace{x_3}_{y_4}, \underbrace{x_4}_{y_5}, \cdots.
    \end{equation*}

    Indeed, consider the family of indices
    \begin{equation*}
      k_m = \begin{cases}
        0,     &m = 1, \\
        m - 1, &m > 1.
      \end{cases}
    \end{equation*}

    The family is not monotone, yet it defines a subnet because it satisfies \fullref{def:subnet}:
    \begin{itemize}
      \item Since \( y_m = x_{k_m} \) for every \( m = 1, 2, \ldots \), \ref{def:subnet/composition} holds.
      \item For every \( k_0 \), we have an index \( m_0 = k_0 + 1 \) such that \( k_m = m - 1 \geq m_0 - 1 = k_0 \) for \( m \geq m_0 \); hence \ref{def:subnet/cofinality} also holds.
    \end{itemize}

    \thmitem{ex:def:subnet/real_subnet} Consider again some sequence
    \begin{equation*}
      x_1, x_2, x_3, \ldots
    \end{equation*}

    A subnet needs not be a sequence. For example, consider the \hyperref[def:totally_ordered_set]{totally ordered set} \( \BbbR_{>0} \) of positive \hyperref[def:real_numbers]{real numbers}. For every positive real number \( r \), let \( k_r \coloneqq \ceil(r) \) be the smallest integer not smaller than \( r \), and define the net \( y_r \coloneqq x_{k_r} \).

    Then \ref{def:subnet/composition} is automatically satisfied.

    Fix some positive integer \( k_0 \). Then we have an index \( r_0 \coloneqq k_0 \) such that, for \( r \geq r_0 \),
    \begin{equation*}
      k_r = \ceil(r) \geq \ceil(r_0) = r_0 = k_0.
    \end{equation*}

    Therefore, \ref{def:subnet/cofinality} holds and \( \seq{ y_r }_{r \in \BbbR} \) is a subnet.

    \thmitem{ex:def:subnet/divisibility_subnet} Furthermore, a subnet of a sequence needs not even be totally ordered. Given some sequence, if we order its index set by divisibility\footnote{The natural number divisibility lattice is described in \fullref{thm:natural_number_divisibility_lattice}.} rather than by the conventional order, we will obtain a subnet of the original.

    Indeed, for every positive integer \( k_0 \), we have an index \( m_0 \coloneqq k_0 \) such that, for \( m_0 \mid m \), we have \( m \geq k_0 \).
  \end{thmenum}
\end{example}

\paragraph{Cluster points}

We will now work in an ambient topological space \( (X, \mscrT) \).

\begin{definition}\label{def:net_cluster_point}\mcite[71]{Kelley1975}
  We say that the point \( x_0 \) is an \term{cluster point} or \term{accumulation point} of the net \( \seq{ x_k }_{k \in \mscrK} \) if it is \hyperref[def:net_frequently_in]{frequently in} every neighborhood of \( x_0 \).
\end{definition}
\begin{comments}
  \item \Fullref{thm:net_cluster_point_base} shows that it is sufficient to only consider bases.
  \item This is related to a similar notion for subsets, \fullref{def:set_cluster_point}, via \fullref{thm:closed_iff_contains_all_net_cluster_points}.
\end{comments}

\begin{example}\label{ex:def:net_cluster_point}
  We give several examples of \hyperref[def:net_cluster_point]{net cluster points}:

  \begin{thmenum}
    \thmitem{ex:def:net_cluster_point/empty} \Fullref{thm:empty_index_set_implies_frequently_in} implies that the empty net \( \seq{ x_k }_{k \in \varnothing} \) is frequently in every set, including every neighborhood of every point.

    Therefore, every point is a cluster point of the empty net.

    \thmitem{ex:def:net_cluster_point/finite_sequence} As discussed in \fullref{ex:def:topological_net/finite_sequence}, any finite sequence \( x_1, x_2, \ldots, x_n \) is frequently in \( \set{ x_n } \) and its supersets, hence also in every neighborhood of \( x_n \).

    The last element of a finite sequence is thus a cluster point. Furthermore, it is the only cluster point of the sequence.

    \thmitem{ex:def:net_cluster_point/sequence} An easy example of a sequence with two cluster points is
    \begin{equation*}
      0, 1, 0, 1, 0, 1, 0, 1, \cdots
    \end{equation*}

    It is frequently in \( \set{ 0 } \) and in \( \set{ 1 } \), hence in all of their neighborhoods (independent of the topology). Thus, both \( 0 \) and \( 1 \) are cluster points.

    \thmitem{ex:def:net_cluster_point/sierpinski} Consider the \hyperref[def:sierpinski_space]{Sierpinski space} \( \set{ \top, \bot } \). The constant sequence
    \begin{equation*}
      \top, \top, \top, \top, \top, \top
    \end{equation*}
    obviously has \( \top \) as a cluster point since it is frequently in the neighborhoods \( \set{ \top } \) and \( \set{ \top, \bot } \) of \( \top \).

    Furthermore, since \( \set{ \top, \bot } \) is the only neighborhood of \( \bot \), the latter is also a cluster point of the sequence (despite not belonging to it).

    \thmitem{ex:def:net_cluster_point/natural_divisibility} Consider the sequence
    \begin{equation*}
      p_1, p_2, p_3, \cdots,
    \end{equation*}
    where \( p_k \) is the smallest \hyperref[def:prime_number]{prime number} that divides \( k \). For completeness, let \( p_1 \coloneqq 1 \).

    The first terms are
    \begin{equation*}
      1, 2, 3, 2, 5, 2, 7, 2, 3, 2, 11, 2, 13, 2, 3, \ldots
    \end{equation*}

    The sequence is frequently in \( \set{ 2 } \) since, for any number \( n \), \( 2n \) is a strictly larger even number. Analogously, it is frequently in \( \set{ p } \) for every prime \( p \) because, for any number \( n \), \( pn \) is a strictly larger number divisible by \( p \).

    Therefore, under any topology, the prime numbers are cluster points, and no other numbers except \( 0 \) and \( 1 \) occur.

    \thmitem{ex:def:net_cluster_point/sin} Consider the net \( \seq{ \sin(r) }_{r \in \BbbR} \).

    Every real number in the interval \( [-1, 1] \) is a cluster point under any topology. Indeed, fix \( x_0 \in [-1, 1] \).

    For any index \( r_0 \in \BbbR \), let \( n_0 \) be the smallest integer such that \( r_0 \leq 2 n_0 \pi \). Then
    \begin{equation*}
      \sin(r_0) = \sin(\underbrace{2 n_0 \pi + \arcsin(x_0)}_r).
    \end{equation*}

    Thus, \( r \geq r_0 \) and \( \sin(r) = x_0 \), meaning that the net \( \seq{ \sin(r) }_{r \in \BbbR} \) is frequently in \( \set{ x_0 } \) (and hence any neighborhood of \( x_0 \)).

    Therefore, every number in the interval \( [-1, 1] \) is a cluster point for the net.
  \end{thmenum}
\end{example}

\begin{proposition}\label{thm:net_cluster_point_base}
  Fix some point \( x_0 \) in a topological space and let \( \mscrB(x) \) be a \hyperref[def:topological_local_base]{local base} at \( x_0 \).

  Then \( x_0 \) is a \hyperref[def:net_cluster_point]{cluster point} of the net \( \seq{ x_k }_{k \in \mscrK} \) if and only if it is \hyperref[def:net_frequently_in]{frequently in} every set of \( \mscrB(x) \).
\end{proposition}
\begin{proof}
  \SufficiencySubProof Every set in \( \mscrB(x) \) is itself a neighborhood of \( x_0 \) by \ref{thm:topology_from_local_base/BP1}, hence if \( x_0 \) is a cluster point, the net is frequently in every set from \( \mscrB \) containing \( x_0 \).

  \NecessitySubProof Suppose that the net is frequently in every set in \( \mscrB(x) \).

  Let \( U \) be a neighborhood of \( x_0 \). By definition of a local base, there exists some set \( V \) in \( \mscrB(x) \) such that
  \begin{equation*}
    x_0 \in V \subseteq U.
  \end{equation*}

  For every index \( k_0 \in \mscrK \) then there exists some \( k \geq k_0 \) such that \( x_k \in V \subseteq U \).

  Since \( U \) was arbitrary, we conclude that \( x_0 \) is a cluster point of the net.
\end{proof}

\begin{proposition}\label{thm:cluster_point_iff_in_closure}\mcite[prop. 1.6.3]{Engelking1989}
  A point \( x_0 \) belongs to the closure of the set \( A \) if and only if there exists a net \( \seq{ x_k }_{k \in \mscrK} \) of members of \( A \) for which \( x_0 \) is a cluster point.
\end{proposition}
\begin{proof}
  The statement holds vacuously if \( A \) is the empty set. Assume that \( A \) is nonempty.

  \SufficiencySubProof Suppose that \( x_0 \) is in the closure of \( A \).
  \begin{itemize}
    \item If \( x_0 \) belongs to \( A \) itself, then the singleton net \( (x_0) \) has \( x_0 \) as a cluster point.
    \item Otherwise, \fullref{thm:union_with_derived_set} implies that \( x_0 \) belongs to the boundary of \( A \). \Fullref{thm:properties_via_bases/boundary} then implies that, for every neighborhood \( U \) of \( x_0 \), there exists some point \( x_U \) from \( A \). We can thus build a reverse inclusion net \( \seq{ x_U }_{U \in \mscrT(x_0)} \) in the style of \fullref{ex:def:topological_net/reverse} that is frequently in every neighborhood of \( x_0 \). The point \( x_0 \) is then a cluster point of this net.
  \end{itemize}

  In both cases, we have constructed a net with elements of \( A \) for which \( x_0 \) is a cluster point.

  \NecessitySubProof Let \( x_0 \) be a cluster point of \( \seq{ x_k }_{k \in \mscrK} \).

  Let \( F \) be any closed superset of \( A \). We will show that \( x_0 \) belongs to \( F \), thus allowing us to conclude that \( x_0 \) is in the closure of \( A \).

  Aiming at a contradiction, suppose that \( x_0 \) belongs to \( X \setminus F \). Note that the latter is then a neighborhood of \( x_0 \).

  Since \( x_0 \) is a cluster point of \( \seq{ x_k }_{k \in \mscrK} \), there exists some index \( k_0 \in \mscrK \) such that \( x_{k_0} \) belongs to \( X \setminus F \). But the net consists of members of \( A \), hence \( x_0 \) belongs to
  \begin{equation*}
    A \cap (X \setminus F) \subseteq F \cap (X \setminus F) = \varnothing.
  \end{equation*}

  The obtained contradiction shows that \( x_0 \) belongs to every closed superset of \( A \), hence also to their intersection --- the closure of \( A \).
\end{proof}

\begin{corollary}\label{thm:closed_iff_contains_all_net_cluster_points}
  A set is closed if and only if it contains the cluster points of all of its nets.
\end{corollary}
\begin{proof}
  By \fullref{thm:cluster_point_characterization}, the set \( A \) is closed if and only if it contains all of its cluster points in the sense of \fullref{def:set_cluster_point}. By \fullref{thm:cluster_point_iff_in_closure}, this is equivalent to \( A \) containing all cluster points of its nets in the sense of \fullref{def:net_cluster_point}.
\end{proof}

\paragraph{Limit points}

Assume again that we work in an ambient topological space \( X \).

\begin{definition}\label{def:net_limit_point}\mcite[66]{Kelley1975}
  We say that the point \( x_0 \) is a \term{limit point} of the net \( \seq{ x_k }_{k \in \mscrK} \) if it is \hyperref[def:net_eventually_in]{eventually in} every neighborhood of \( x_0 \).

  We say that a net is \term{convergent} if it has a limit point and \term{divergent} otherwise.
\end{definition}
\begin{comments}
  \item \Fullref{thm:net_limit_point_subbase} shows that it is sufficient to only consider subbases.

  \item In general, there can exist multiple limit points --- see \fullref{ex:def:net_limit_point}. In Hausdorff spaces, however, limits are unique as shown in \fullref{thm:t2_iff_singleton_limits}.
\end{comments}

\begin{example}\label{ex:def:net_limit_point}
  We give several examples of \hyperref[def:net_limit_point]{net limit points}:

  \begin{thmenum}
    \thmitem{ex:def:net_limit_point/empty} The empty net cannot have limit points as a consequence of \fullref{thm:eventually_in_implies_nonempty}.

    \thmitem{ex:def:net_limit_point/finite_sequence} As in the case of cluster points in \fullref{ex:def:net_cluster_point/finite_sequence}, a finite sequence \( x_1, \ldots, x_n \) can have only one limit point --- \( x_n \).

    \thmitem{ex:def:net_limit_point/sin} We saw in \fullref{ex:def:net_cluster_point/sin} that the net \( \seq{ \sin(r) }_{r \in \BbbR} \) has a lot of cluster points. It does not have a limit point, however.

    Indeed, suppose that \( x_0 \) is a limit point. Fix some neighborhood \( U \) of \( x_0 \) in the topological space \( [-1, 1] \) that is not the whole space.

    There exists some \( r_0 \) such that \( \sin(r) \in U \) for \( r \geq r_0 \). But
    \begin{equation*}
      \sin([r_0, r_0 + 2\pi)) = [-1, 1]
    \end{equation*}
    and we have explicitly assumed that \( U \) is a strict subset of \( [-1, 1] \).

    Therefore, no point in \( [-1, 1] \) is a limit point of the net \( \seq{ \sin(r) }_{r \in \BbbR} \).

    \thmitem{ex:def:net_limit_point/sierpinski_constant} As in \fullref{ex:def:net_cluster_point/sierpinski}, consider the \hyperref[def:sierpinski_space]{Sierpinski space} \( \set{ \top, \bot } \) and the constant sequence
    \begin{equation*}
      \top, \top, \top, \top, \top, \top.
    \end{equation*}

    Both \( \top \) and \( \bot \) are cluster points, but they are also limit points. They are distinct and even \hyperref[def:topologically_indistinguishable]{topologically distinguishable}.

    \thmitem{ex:def:net_limit_point/sierpinski_alternating} Consider instead the Sierpinski space with the alternating sequence
    \begin{equation*}
      \top, \bot, \top, \bot, \top, \bot.
    \end{equation*}

    Both \( \top \) and \( \bot \) are cluster points. Furthermore, \( \bot \) is a limit point, while \( \top \) is not.

    \thmitem{ex:def:net_limit_point/multiple} Limits of sequences need not be unique in general topological spaces. Let \( X = \set{ a, b } \) be a binary set with the \hyperref[def:indiscrete_topology]{indiscrete topology} \( \set{ \varnothing, X } \).

    Define the following \hyperref[def:sequence]{sequence}
    \begin{equation*}
      x_k \coloneqq \begin{cases}
        a, &k \T{is even}, \\
        b, &k \T{is odd}.
      \end{cases}
    \end{equation*}

    The only neighborhood of \( y \), the whole space \( X \), contains all members of the sequence, therefore \( y \) is a limit point of the sequence. The same is true for \( z \), however.
  \end{thmenum}
\end{example}

\begin{proposition}\label{thm:def:net_limit_point}
  \hyperref[def:net_limit_point]{Net limit points} have the following basic properties:

  \begin{thmenum}
    \thmitem{thm:def:net_limit_point/limit_point_is_cluster_point} Every limit point is a cluster point.

    The converse is true for converging nets in \hyperref[def:separation_axioms/T2]{Hausdorff spaces}, but not in general --- see \fullref{ex:def:net_limit_point/sierpinski_alternating}.

    \thmitem{thm:def:net_limit_point/cluster_point_in_subnet} Every cluster point of a subnet is also a cluster point of the entire net.

    \thmitem{thm:def:net_limit_point/limit_and_subnet_limit} A point is a limit point of a net if and only if it is a limit point of all \hyperref[def:subnet]{subnets}.

    \thmitem{thm:def:net_limit_point/cluster_point_subnet_limit} A point is a cluster point of a net if and only if it is a limit point of some subnet.

    \thmitem{thm:def:net_limit_point/cluster_and_limit_point} Given a limit point and a cluster point of the same net, every neighborhood of the first intersects every neighborhood of the second and vice versa.

    In a general topological space, the two points may still be \hyperref[def:topologically_indistinguishable]{topologically distinguishable} as shown in \fullref{ex:def:net_limit_point/sierpinski_constant}.
  \end{thmenum}
\end{proposition}
\begin{proof}
  \SubProofOf{thm:def:net_limit_point/limit_point_is_cluster_point} Follows from \fullref{thm:eventually_in_implies_frequently_in}.

  \SubProofOf{thm:def:net_limit_point/cluster_point_in_subnet} Let \( \seq{ y_m }_{k \in \mscrK} \) be a subnet of \( \seq{ x_k }_{k \in \mscrK} \) with index translation \( \seq{ k_m }_{m \in \mscrM} \). Suppose that \( y_0 \) is a cluster point of \( \seq{ y_m }_{k \in \mscrK} \).

  Let \( U \) be a neighborhood of \( y_0 \) and fix some index \( k_0 \in \mscrK \). \ref{def:subnet/cofinality} implies that there exists some index \( m_0 \) such that \( k_m \geq k_0 \) for \( m \geq m_0 \).

  Since \( y_0 \) is a cluster point of \( \seq{ k_m }_{m \in \mscrM} \), there exists some index \( m' \geq m_0 \) such that the point \( y_{m'} = x_{k_m'} \) belongs to \( U \). Since we have \( k_m' \geq k_0 \), we conclude that the entire net \( \seq{ x_k }_{k \in \mscrK} \) is frequently in \( U \).

  The neighborhood \( U \) was chosen arbitrarily, hence we conclude that \( y_0 \) is a cluster point of \( \seq{ x_k }_{k \in \mscrK} \).

  \SubProofOf{thm:def:net_limit_point/limit_and_subnet_limit}

  \SufficiencySubProof* Let \( x_0 \) be a limit point of \( \seq{ x_k }_{k \in \mscrK} \) and let \( \seq{ y_m }_{m \in \mscrM} \) be a subnet with index translation \( \seq{ k_m }_{m \in \mscrM} \).

  Let \( U \) be a neighborhood of \( x_0 \). Then there exists an index \( k_0 \) such that \( x_k \in U \) for \( k \geq k_0 \). Since \( \seq{ y_m }_{m \in \mscrM} \) is a subnet, \ref{def:subnet/cofinality} states that there exists some index \( m_0 \) such that \( k_m \geq k_0 \) for \( m \geq m_0 \).

  We showed that the subnet is eventually in an arbitrary neighborhood of \( x_0 \), implying that \( x_0 \) is a limit point of the subnet.

  \NecessitySubProof* Suppose that \( x_0 \) is a limit point of all subnets of \( \seq{ x_k }_{k \in \mscrK} \). Every net is a subnet of itself, hence \( x_0 \) is a limit point of \( \seq{ x_k }_{k \in \mscrK} \).

  \SubProofOf{thm:def:net_limit_point/cluster_point_subnet_limit}

  \SufficiencySubProof* Let \( x_0 \) be a cluster point of the net \( \seq{ x_k }_{k \in \mscrK} \).

  For every neighborhood \( U \) of \( x_0 \), there exists some index \( k_U \) such that \( x_{k_U} \in U \).

  Consider the \hyperref[ex:def:topological_net/reverse]{dual filter} \( (\mscrT(x_0), \supseteq) \). For every concrete neighborhood \( U_0 \), there exists some index \( k_{U_0} \) such that \( x_{k_V} \in U_0 \) whenever the neighborhood \( V \) is a subset of \( U_0 \), that is, whenever \( V \geq U_0 \).

  Therefore, the net \( \seq{ k_U }_{U \in \mscrT(x_0)} \) is a subnet of \( \seq{ x_k }_{k \in \mscrK} \).

  Furthermore, it converges to \( x_0 \) because it is eventually in each neighborhood of \( x_0 \).

  \NecessitySubProof* Follows from \fullref{thm:def:net_limit_point/limit_point_is_cluster_point} and \fullref{thm:def:net_limit_point/cluster_point_in_subnet}.

  \SubProofOf{thm:def:net_limit_point/cluster_and_limit_point} Let \( l \) be a limit point of \( \seq{ x_k }_{k \in \mscrK} \) and let \( c \) be a cluster point.

  Let \( U \) be a neighborhood of \( l \). Then there exists an index \( k_l \) such that \( x_k \in U \) for \( k > k_l \).

  Let \( V \) be a neighborhood of \( c \). Then there exists an index \( k_c \geq k_l \) such that \( x_{k_c} \in V \). Furthermore, \( x_{k_c} \in U \cap V \).

  Therefore, \( U \) and \( V \) intersect. Since they are arbitrary neighborhoods, we conclude that the proposition holds.
\end{proof}

\begin{proposition}\label{thm:net_limit_point_subbase}
  Fix some point \( x_0 \) in a topological space and let \( \mscrS(x) \) be a \hyperref[def:topological_subbase]{subbase} at \( x_0 \).

  Then \( x_0 \) is a \hyperref[def:net_limit_point]{limit point} of the net \( \seq{ x_k }_{k \in \mscrK} \) if and only if it is \hyperref[def:net_eventually_in]{eventually in} every set of \( \mscrS(x) \).
\end{proposition}
\begin{proof}
  \SufficiencySubProof Every set in \( \mscrS \) is itself a neighborhood of \( x_0 \), hence if \( x_0 \) is a limit point, the net is eventually in every set from \( \mscrS \) containing \( x_0 \).

  \NecessitySubProof Suppose that the net is eventually in every set in \( \mscrS \) containing \( x_0 \).

  Let \( U \) be a neighborhood of \( x_0 \). By definition of local subbase, there exists some family \( V_1, \ldots, V_n \) in \( \mscrS \) such that
  \begin{equation*}
    x_0 \in V_1 \cap \cdots \cap V_n \subseteq U.
  \end{equation*}

  For every \( i = 1, \ldots, n \), there exists some \( k_i \in \mscrK \) such that \( k \geq k_i \) such that \( x_k \in V_i \subseteq U \).

  Since \( \mscrK \) is a directed set, there exists some upper bound \( k_0 \) for \( k_1, \ldots, k_n \). Then, if \( k \geq k_0 \), we have
  \begin{equation*}
    x_0 \in V_1 \cap \cdots \cap V_n \subseteq U.
  \end{equation*}

  Since \( U \) was arbitrary, we conclude that \( x_0 \) is a limit point of the net.
\end{proof}

\begin{proposition}\label{thm:first_countable_space_limit_points}
  Let \( X \) be a \hyperref[def:topological_space_character]{first-countable space} and let \( \seq{ x_k }_{k \in \mscrK} \) be a net in \( X \). Suppose that \( x_0 \) is a \hyperref[def:net_cluster_point]{cluster point} of the net.

  Then there exists an \hyperref[def:order_function/preserving]{order-preserving} sequence of indices
  \begin{equation*}
    k_1 \leq_\mscrK k_2 \leq_\mscrK k_3 \leq_\mscrK k_4 \leq_\mscrK \cdots
  \end{equation*}
  from \( K \) such that \( x_0 \) is a \hyperref[def:net_limit_point]{limit point} of the \hyperref[def:sequence]{sequence}
  \begin{equation*}
    x_{k_1}, x_{k_2}, x_{k_3}, x_{k_4}, \ldots.
  \end{equation*}
\end{proposition}
\begin{comments}
  \item First-countable spaces, such as \hyperref[def:metric_space]{metric spaces}, allow restricting ourselves to only \hyperref[def:sequence]{sequences} rather than arbitrary \hyperref[def:topological_net]{nets}.
  \item If \( \mscrK \) is a \hyperref[def:totally_ordered_set]{totally ordered set}, this sequence is a \hyperref[def:subnet]{subnet}.
\end{comments}
\begin{proof}
  Consider any countable neighborhood basis \( U_1, U_2, U_3, \ldots \) of \( x_0 \). Define the following sequence of sets:
  \begin{equation*}
    V_m \coloneqq \cap_{i=1}^m U_i.
  \end{equation*}

  Then we have the reverse inclusion chain
  \begin{equation*}
    V_1 \supseteq V_2 \supseteq V_3 \supseteq \cdots
  \end{equation*}

  Every element of this chain is a neighborhood of \( x_0 \). Fix an arbitrary index \( k_0 \in K \) --- the concrete choice does not matter; it is only a technicality. Since \( x_0 \) is a cluster point of the net \( \seq{ x_k }_{k \in \mscrK} \), there exists some index \( k_1 \geq k_0 \) such that \( x_{k_1} \in V_1 \).

  For \( i > 1 \), we similarly pick an index such that \( k_i \geq k_{i-1} \) and \( x_{k_i} \in V_i \).

  It remains to show that \( x_0 \) is a cluster point of the obtained sequence
  \begin{equation*}
    x_{k_1}, x_{k_2}, x_{k_3}, x_{k_4}, \ldots
  \end{equation*}

  Let \( W \) be an arbitrary neighborhood of \( x_0 \). By definition of local filter, there exists an index \( j_W \) such that \( U_{j_W} \subseteq W \). Then, whenever \( j \geq j_W \),
  \begin{equation*}
    x_{k_j} \in V_j = U_j \cap U_{j_W} \cap \cdots \subseteq U_{j_W} \subseteq W.
  \end{equation*}

  Thus, the sequence is eventually in \( W \). Generalizing on \( W \), we conclude that \( x_0 \) is a limit point of the sequence.
\end{proof}

  \section{Filter convergence}\label{sec:filter_convergence}

\paragraph{Filters}

\begin{definition}\label{def:topological_filter}\mcite[52]{Engelking1989Topology}
  By a \enquote{filter} in a topological space \( (X, \mscrT) \), we will mean a \hyperref[def:lattice_ideal]{lattice-theoretic filter} in the \hyperref[thm:boolean_algebra_of_subsets]{power set Boolean algebra} \( \pow(X) \).
\end{definition}

\begin{definition}\label{def:filter_ordering}\mcite[52]{Engelking1989Topology}
  If \( F \subseteq G \) for two \hyperref[def:topological_filter]{filters} over the same space, we say that \( G \) is \term{finer} than \( F \) and \( F \) is \term{coarser} than \( G \).
\end{definition}
\begin{comments}
  \item The terminology is inspired by the \hyperref[def:topological_space_ordering]{lattice of topologies} on a given set.
\end{comments}

  \subsection{Function convergence}\label{subsec:function_convergence}

\begin{definition}\label{def:local_convergence}
  Fix two topological spaces \( X \) and \( Y \). Let \( A \subseteq X \) be a nonempty set and let \( f: A \to Y \) be a function. We give two equivalent definitions for \( y_0 \in Y \) being a \term{limit point} of \( f \) at \( x_0 \in \cl(A) \). If \( y_0 \) is the unique limit point (e.g. in \hyperref[def:separation_axioms/T2]{Hausdorff spaces}), we write
  \begin{equation*}
    \lim_{x \to x_0} f(x) = y_0.
  \end{equation*}

  \begin{thmenum}
    \thmitem{def:local_convergence/neighborhoods}(Cauchy-style condition) For every neighborhood \( V \) of \( y_0 \) there exists a neighborhood \( U \) of \( x_0 \) such that \( f(U \cap A) \subseteq V \).

    \thmitem{def:local_convergence/nets}(Heine-style condition) For every \hyperref[def:topological_net]{net} \( \seq{ x_k }_{k \in \mscrK} \subseteq A \), for which \( x_0 \) is a limit \hyperref[def:net_convergence/limit]{point}, the corresponding net \( \{ f(x_k) \}_{k \in \mscrK} \) has \( y_0 \) as a limit point.
  \end{thmenum}
\end{definition}
\begin{proof}
  \ImplicationSubProof{def:local_convergence/neighborhoods}{def:local_convergence/nets} Let \( \seq{ x_k }_{k \in \mscrK} \subseteq U \) be a net  with limit point \( x_0 \). Consider the net \( \{ f(x_k) \}_{k \in \mscrK} \). Fix a neighborhood \( V \) of \( y_0 \). We need to show that \( \{ f(x_k) \}_{k \in \mscrK} \) is eventually in \( V \).

  By \fullref{def:local_convergence/neighborhoods}, there exists a neighborhood \( U \) of \( x_0 \) such that \( f(U) \subseteq V \). Since \( x_0 \) is a limit point of \( \seq{ x_k }_{k \in \mscrK} \), there exists an index \( k_0 \) such that for all \( k \geq k_0 \), \( x_k \in U \) and therefore \( f(x_k) \in V \). Hence, \( \{ f(x_k) \}_{k \in \mscrK} \) is eventually in \( V \).

  We conclude that \( y_0 \) is a limit point of the net \( \{ f(x_k) \}_{k \in \mscrK} \) and that the Heine-style condition is satisfied.

  \ImplicationSubProof{def:local_convergence/nets}{def:local_convergence/neighborhoods} Suppose that \fullref{def:local_convergence/nets} holds while \fullref{def:local_convergence/neighborhoods} does not. Let \( V \) be a neighborhood of \( y_0 \). Then there exists no neighborhood \( U \) of \( x_0 \) such that \( f(U) \subseteq V \).

  For any neighborhood \( U \) of \( x_0 \) and let \( y_U \in f(U) \setminus V \) and \( x_U \in f^{-1} (U) \), so that \( f(x_U) = y_U \). Consider the families
  \begin{balign*}
    \{ x_U \}_{U \in T(x_0)},
     &  &
    \{ f(x_U) \}_{U \in T(x_0)},
  \end{balign*}
  ordered by \hyperref[ex:reverse_inclusion_net]{reverse inclusion} of the neighborhoods \( \mscrT(x_0) \) of \( x_0 \).

  Note that \( x_0 \) is a limit point of \( \{ x_U \}_{U \in T(x_0)} \). By \fullref{def:local_convergence/nets}, \( y_0 \) is a limit point of \( \{ f(x_U) \}_{U \in T(x_0)} \). But this contradicts our choice of the nets because \( f(x_U) \not\in V \) for any \( U \in T(x) \).

  The obtained contradiction demonstrates that \fullref{def:local_convergence/nets} implies \fullref{def:local_convergence/neighborhoods}.
\end{proof}

\begin{proposition}\label{thm:cauchy_function_convergence_via_subbases}
  Fix two topological spaces \( X \) and \( Y \) and two points \( x_0 \in X \) and \( y_0 \in Y \). Let \( P(x_0) \) and \( P(y_0) \) be local \hyperref[def:topological_local_subbase]{subbases} for the corresponding points. Then the function \( f: X \to Y \) \hyperref[def:local_convergence]{converges} to \( y_0 \) at \( x_0 \) if and only if every \( V_P \in P(y_0) \) there exists \( U_P \in B(x_0) \) such that \( f(U_P) \subseteq V_P \).

  Compare this result to \fullref{thm:net_convergence_via_subbases}.
\end{proposition}
\begin{proof}
  \SufficiencySubProof Obvious consequence of \fullref{def:local_convergence/neighborhoods}.
  \NecessitySubProof Fix a neighborhood \( V \) of \( x \). We will show that \fullref{def:local_convergence/neighborhoods} holds.

  Let \( \{ V_k \}_{k=1}^n \subseteq P(y_0) \) be a family such that \( \bigcap_{k=1}^n V_k \subseteq V \) (such a family exists by definition of a local subbase). By the antecedent of the implication we are proving, for every \( k = 1, \ldots, n \) there exists an \( U_k \in P(x_0) \) such that \( f(U_k) \subseteq V_k \). Then \( U \coloneqq \bigcap_{k=1}^n U_k \) is a neighborhood of \( x_0 \) and, furthermore,
  \begin{equation*}
    f(U)
    =
    f\left(\bigcap_{k=1}^n U_k \right)
    \subseteq
    \bigcap_{k=1}^n f(U_k)
    \subseteq
    \bigcap_{k=1}^n V_k
    \subseteq
    V.
  \end{equation*}

  Therefore, \fullref{def:local_convergence/neighborhoods} holds.
\end{proof}

  \subsection{Topological continuity}\label{subsec:topological_continuity}

\begin{definition}\label{def:local_continuity}
  We say that the function \( f: X \to Y \) between topological spaces is \term{continuous} at the point \( x_0 \in X \) if \( f(x_0) \) is a limit \hyperref[def:local_convergence]{point} of \( f \) at \( x_0 \).

  If limit point is unique (e.g. in \hyperref[def:separation_axioms/T2]{Hausdorff spaces}), this condition can be formulated by \enquote{interchanging} \( \lim \) and \( f \) as follows:
  \begin{equation*}
    f(x_0) = f\left( \lim_{x \to x_0} x \right) = \lim_{x \to x_0} f(x).
  \end{equation*}
\end{definition}

\begin{definition}\label{def:global_continuity}
  We say that the function \( f: X \to Y \) between topological spaces is \term{everywhere continuous} or simply \term{continuous} if and of the following conditions hold:
  \begin{thmenum}
    \thmitem{def:global_continuity/limits} \( f \) is continuous at every point of \( X \) in the sense of \fullref{def:local_continuity}.
    \thmitem{def:global_continuity/open} For every open set \( V \in T \), the \hyperref[thm:def:function/preimage]{preimage} \( f^{-1}(V) \) is open.
    \thmitem{def:global_continuity/closed} For every closed set \( F \in F_{\mscrT_Y} \), the preimage \( f^{-1}(F) \) is closed.
    \thmitem{def:global_continuity/base} There exists a \hyperref[def:topological_base]{base} \( \mscrB_{\mscrT_Y} \subseteq T_Y \), such that for every \( V \in B_{\mscrT_Y} \), the preimage \( f^{-1}(V) \) is open.
    \thmitem{def:global_continuity/subbase} There exists a \hyperref[def:topological_subbase]{subbase} \( P_{\mscrT_Y} \subseteq T_Y \), such that for every \( V \in P_{\mscrT_Y} \), the preimage \( f^{-1}(V) \) is open.
    \thmitem{def:global_continuity/closure} For every set \( A \subseteq X \), \( f(\cl(A)) \subseteq \cl(f(A)) \).
  \end{thmenum}

  We denote the set of all continuous functions from \( X \) to \( Y \) by \( C(X, Y) \).
\end{definition}
\begin{proof}
  \ImplicationSubProof{def:global_continuity/limits}{def:global_continuity/open} Follows from \fullref{def:local_convergence/neighborhoods}.
  \ImplicationSubProof{def:global_continuity/open}{def:global_continuity/closed} If \( F \in F_{\mscrT_Y} \) is a closed set, \( Y \setminus F \) is open, therefore \( f^{-1}(Y \setminus F) = X \setminus f^{-1}(F) \) is also open. Hence, \( f^{-1}(F) \) is closed.
  \ImplicationSubProof{def:global_continuity/open}{def:global_continuity/base} \( \mscrT \) is a base of itself.
  \ImplicationSubProof{def:global_continuity/base}{def:global_continuity/subbase} Every base is also a subbase.
  \ImplicationSubProof{def:global_continuity/subbase}{def:global_continuity/limits} Follows from the equivalences in \fullref{def:local_convergence}.
  \ImplicationSubProof{def:global_continuity/closed}{def:global_continuity/closure} Note that
  \begin{equation*}
    A
    \reloset {\ref{thm:function_image_preimage_composition/preimage_of_image}} \subseteq
    f^{-1}(f(A))
    \reloset {\ref{thm:def:function_preimage/monotonicity}} \subseteq
    f^{-1}(\cl(f(A))).
  \end{equation*}

  Apply \( f \circ \cl \) to the above chain of inclusions to obtain
  \begin{equation*}
    f(\cl(A))
    \subseteq
    f(\underbrace{\cl}_{\ref{def:global_continuity/closed}}(f^{-1}(\cl(f(A)))))
    \reloset {\ref{thm:function_image_preimage_composition/image_of_preimage}} \subseteq
    \cl(f(A)),
  \end{equation*}
  which proves the implication.

  \ImplicationSubProof{def:global_continuity/closure}{def:global_continuity/closed} Fix a closed set \( F \subseteq Y \). Then
  \begin{equation}\label{def:global_continuity/closure_implies_closed_right}
    f(\cl(f^{-1}(F)))
    \reloset {\ref{def:global_continuity/closure}} \subseteq
    \cl(f(f^{-1}(F)))
    \reloset {\ref{thm:function_image_preimage_composition/image_of_preimage}} \subseteq
    \cl(F)
    =
    F.
  \end{equation}

  Since \( \cl \) is monotone, we have
  \begin{equation}\label{def:global_continuity/closure_implies_closed_left}
    f(\cl(f^{-1}(F)))
    \supseteq
    f(f^{-1}(F))
    \reloset {\ref{thm:function_image_preimage_composition/preimage_of_image}} \supseteq
    F.
  \end{equation}

  From \eqref{def:global_continuity/closure_implies_closed_right} and \eqref{def:global_continuity/closure_implies_closed_left} it follows that
  \begin{equation*}
    F = f(\cl(f^{-1}(F))).
  \end{equation*}

  By taking the preimage, we obtain
  \begin{equation*}
    f^{-1}(F)
    =
    f^{-1}(f(\cl(f^{-1}(F))))
    \reloset {\ref{thm:function_image_preimage_composition/image_of_preimage}} \supseteq
    \cl(f^{-1}(F)).
  \end{equation*}

  Therefore, \( f^{-1}(F) \) is closed.
\end{proof}

\begin{definition}\label{def:homeomorphism}
  We say that the continuous function \( f: X \to Y \) is \term{open} (resp. \term{closed}), if the image \( f(U) \) of an open (resp. closed) in \( \mscrT_X \) set is open (resp. closed) in \( \mscrT_Y \).

  If \( f \) is an open bijection, we say that \( f \) is a \term{homeomorphism}. If \( f \) is only an open injection, we say that \( f \) is a \term{homeomorphic embedding}.
\end{definition}

\begin{definition}\label{def:fundamental_groupoid}
  \todo{Define fundamental groupoids}.
\end{definition}

  \subsection{Category of topological spaces}\label{subsec:category_of_topological_spaces}

\begin{definition}\label{def:category_of_small_topological_spaces}
  Since the topology \( \mscrT \) of a \hyperref[def:topological_space]{topological space} \( (X, \mscrT) \) consists of subsets of \( X \), we cannot build a \hyperref[def:first_order_theory]{first-order theory} from \fullref{def:topological_space/O1}-\fullref{def:topological_space/O3}. We can, however, explicitly describe the \hyperref[def:category]{category} \( \cat{Top} \) of topological spaces as
  \begin{refenum}
    \refitem{def:category/objects} The \hyperref[def:set]{class} of objects is the class of all topological space.
    \refitem{def:category/morphisms} The morphisms between two topological spaces are the \hyperref[def:global_continuity]{continuous functions} between them.
    \refitem{def:category/composition} Composition of morphisms is the usual \hyperref[def:set_valued_map/composition]{function composition}.
  \end{refenum}
\end{definition}

\begin{theorem}\label{thm:top_complete_cocomplete}
  The category \( \cat{Top} \) of is both \hyperref[def:category_of_cones/limit]{complete} and \hyperref[def:category_of_cones/colimit]{cocomplete}.
\end{theorem}

\begin{definition}\label{def:initial_topology}\mcite{nLab:top}
  Let \( \{ (X_k, \mscrT_k) \}_{k \in \mscrK} \) be a \hyperref[def:cartesian_product/indexed_family]{family} of topological spaces. Let \( X \) be a bare set and let
  \begin{equation*}
    \{ f_k: X \to X_k \}_{k \in \mscrK}
  \end{equation*}
  be a family of functions.

  The topology on \( X \) generated by the subbase
  \begin{equation*}
    \mathcal{P} \coloneqq \{ f_k^{-1}(U) \colon k \in \mscrK, U \in T_k \}
  \end{equation*}
  is called the \term{initial} (or \term{weak}) topology on \( X \) generated by the family \( \{ f_k \}_{k \in \mscrK} \).

  It is the weakest topology that makes all functions in the family \( \{ f_k \}_{k \in \mscrK} \) continuous.
\end{definition}

\begin{definition}\label{def:final_topology}\mcite{nLab:top}
  Dually, if the family of functions is of the type
  \begin{equation*}
    \{ f_k: X_k \to X \}_{k \in \mscrK},
  \end{equation*}
  then we define the \term{final} (or \term{strong}) topology on \( X \) generated by the family \( \{ f_k \}_{k \in \mscrK} \) as the topology
  \begin{equation*}
    \mscrT \coloneqq \{ U \subseteq X \colon \forall k \in \mscrK, f_k^{-1}(U) \in T_k \}.
  \end{equation*}

  It is the strongest topology that makes all functions in the family \( \{ f_k \}_{k \in \mscrK} \) continuous.
\end{definition}

\begin{proposition}\label{thm:initial_final_topology_limit}\mcite{nLab:top}
  Let \( D: \op* I \to \cat{Top} \) be a small \hyperref[def:categorical_diagram]{diagram}. For each space in the image \( D(\op* I) \), denote the set corresponding by \( X_k \) and the corresponding topology by \( \mscrT_k \).

  The limit (resp. colimit) \( (X, \mscrT) \) of \( D \) can then be described as
  \begin{thmenum}
    \item \( (X, \{ f_k \}_{k \in \cat{I}}) = \varprojlim UD \) (resp. \( \varinjlim UD \)) is the limit (resp. colimit) in \( \cat{Set} \) of \( U \circ D \), where \( U: \op*{Top} \to \cat{Set} \) is the forgetful functor.
    \item \( \mscrT \) is the \hyperref[def:initial_topology]{initial} (resp. \hyperref[def:final_topology]{final}) topology on \( X \) generated by the family of functions \( \{ f_k \}_{k \in \cat{I}} \).
  \end{thmenum}
\end{proposition}

\begin{definition}\label{def:topological_subspace}
  Let \( (X, \mscrT) \) be a topological space and let \( M \subseteq X \) be a subset of \( X \). The \term{topological subspace} \( (M, \mscrT_M) \) is obtained by endowing \( M \) with the topology
  \begin{equation*}
    \mscrT_M \coloneqq \{ U \cap M \colon U \in T \}.
  \end{equation*}

  The topology \( \mscrT_M \) is called the \term{subspace topology} or \term{induced topology}.

  It is the initial topology generated by the canonical embedding \( \iota: M \to X \).
\end{definition}

\begin{definition}\label{def:topological_product}
  The \term{topological product} or \term{Tychonoff product}
  \begin{equation*}
    \left( \prod_{k \in \mscrK} X_k, \prod_{k \in \mscrK} \mscrT_k \right)
  \end{equation*}
  of the family \( { (X_k, \mscrT_k) }_{k \in \mscrK} \) is simply the categorical product in the category \( \cat{Top} \) (see \fullref{def:discrete_category_limits}). The underlying set \( \prod_{k \in \mscrK} X_k \) is the \hyperref[thm:discrete_category_limits_in_set]{Cartesian product} and the topology \( \prod_{k \in \mscrK} \mscrT_k \) is called the \term{product topology}.

  Let \( { (X_k, \mscrT_k) }_{k \in \mscrK} \) and \( { (Y_k, \mathcal{O}_k) }_{k \in \mscrK} \) be two families of topological spaces and let
  \begin{equation*}
    \{ f_k: X_k \to Y_k \}_{k \in \mscrK}
  \end{equation*}
  be a family of arbitrary functions between them.

  We define the \term{product \( \prod_{k \in \mscrK} f_k \) of \( \{ f_k \}_{k \in \mscrK} \)} as the function
  \begin{balign*}
     & \left(\prod_{k \in \mscrK} f_k \right): \prod_{k \in \mscrK} X_k \to \prod_{k \in \mscrK} Y_k              \\
     & \left(\prod_{k \in \mscrK} f_k \right)(\seq{ x_k }_{k \in \mscrK}) \coloneqq \{ f_k (x_k) \}_{k \in \mscrK}.
  \end{balign*}

  If all of the spaces \( (X_k, \mscrT_k) \) are equal to some space \( (X, \mscrT) \), we call the product of \( \{ f_k \}_{k \in \mscrK} \) the \term{diagonal product} and denote it by
  \begin{equation*}
    \Delta_{k \in \mscrK} f_k: X \to \prod_{k \in \mscrK} Y_k.
  \end{equation*}
\end{definition}

\begin{definition}\label{def:topological_quotient}\mcite[90]{Engelking1989}
  Let \( X \) be a topological space and let \( \cong \) be an \hyperref[def:equivalence_relation]{equivalence relation} on \( X \). The \term{quotient space} \( (X, \mscrT) / \sim \) is obtained by endowing the quotient set \( X / \cong \) with the final \hyperref[def:final_topology]{topology} given by the canonical projection map \( x \mapsto [x] \).
\end{definition}

\begin{definition}\label{def:topological_sum}\mcite[74]{Engelking1989}
  The \term{topological direct sum}
  \begin{equation*}
    (\oplus_{k \in \mscrK} X_k, \oplus_{k \in \mscrK} \mscrT_k)
  \end{equation*}
  of the family \( { (X_k, \mscrT_k) }_{k \in \mscrK} \) is simply the categorical coproduct in the category \( \cat{Top} \) (see \fullref{def:discrete_category_limits}). The underlying set \( \oplus_{k \in \mscrK} X_k \) is the \hyperref[thm:discrete_category_limits_in_set]{disjoint union} and the topology \( \oplus_{k \in \mscrK} \mscrT_k \) is called the \term{direct sum topology}.

  Let \( { (X_k, \mscrT_k) }_{k \in \mscrK} \) and \( { (Y_k, \mathcal{O}_k) }_{k \in \mscrK} \) be two families of topological spaces and let
  \begin{equation*}
    \{ f_k: X_k \to Y_k \}_{k \in \mscrK}
  \end{equation*}
  be a family of arbitrary functions between them. Let \( \iota_{X_k}: X_k \to \oplus_{k \in \mscrK} X_k \) and \( \iota_{Y_k}: Y_k \to \oplus_{k \in \mscrK} Y_k \) be the corresponding canonical embeddings.

  We define the \term{direct sum \( \oplus_{k \in \mscrK} f_k \) of \( \{ f_k \}_{k \in \mscrK} \)} as the function
  \begin{balign*}
     & (\oplus_{k \in \mscrK} f_k): \oplus_{k \in \mscrK} X_k \to \oplus_{k \in \mscrK} Y_k   \\
     & (\oplus_{k \in \mscrK} f_k){\rvert}_{X_k} \coloneqq \iota_{Y_k} \circ f_k.
  \end{balign*}

  Obviously \( \oplus_{k \in \mscrK} f_k \) is continuous whenever all \( f_k \) are continuous.

  If all of the spaces \( (Y_k, \mathcal{O}_k) \) are equal to some space \( (Y, \mathcal{O}) \), we call the direct sum of \( \{ f_k \}_{k \in \mscrK} \) simply a \term{sum} and denote it by
  \begin{equation*}
    \sum_{k \in \mscrK} f_k: \oplus_{k \in \mscrK} X_k \to Y.
  \end{equation*}
\end{definition}

\begin{definition}\label{def:borel_hierarchy}
  \todo{Define}.
\end{definition}

\begin{definition}\label{def:category_of_small_frames}\mcite[43]{Johnstone1983}
  Suppose that we are given a \hyperref[def:grothendieck_universe]{Grothendieck universe} \( \mscrU \), which is safe to assume to be the smallest suitable one as explained in \fullref{def:large_and_small_sets}. We describe the \term{category of \( \mscrU \)-small frames} as the following \hyperref[rem:concrete_categories]{concrete category}:

  \begin{itemize}
    \item The \hyperref[def:category/objects]{objects} are the \( \mscrU \)-small \hyperref[def:complete_lattice]{complete lattices} in which arbitrary meets distribute over finite joins. We call such lattices \term{frames}.

    \item The \hyperref[def:category/morphisms]{morphisms} between two frames are the functions between them preserving finite meets and arbitrary joins. We call such functions \term{frame homomorphisms}.
  \end{itemize}
\end{definition}

\begin{proposition}\label{thm:topological_spaces_are_frames}
  The topology of a topological space is a \hyperref[def:category_of_small_frames]{frame}.
\end{proposition}
\begin{proof}
  Trivial.
\end{proof}

\begin{remark}\label{rem:topology_frame_homomorphism}
  Consider the continuous function \( f: X \to Y \) between topological spaces.

  \Fullref{thm:function_preimage_properties/union} and \fullref{thm:function_preimage_properties/intersection} imply that the inverse \( f^{-1}: Y \to X \) preserves arbitrary unions and intersections and is hence a \hyperref[def:category_of_small_frames]{frame homomorphism} from \( \mscrT_Y \) to \( \mscrT_X \).

  This motivates defining \hyperref[def:category_of_small_locales]{locales}.
\end{remark}

\begin{definition}\label{def:category_of_small_locales}\mcite[43]{Johnstone1983}
  We call the \hyperref[def:opposite_category]{opposite category} of the \hyperref[def:category_of_small_frames]{category of \( \mscrU \)-small frames} the \term{category of \( \mscrU \)-small locales}.
\end{definition}

\begin{remark}\label{rem:picking_a_point_from_a_locale}
  The topology of the one-element topological space is a \hyperref[thm:two_element_lattice]{two-element lattice}.

  The \hyperref[def:category_of_small_locales]{locale homomorphism} \( f^\oppos: \set{ \top, \bot } \to L \) then corresponds to a continuous function from the one-element space to some other space \( X \) whose topology is isomorphic to \( L \). This reduces to picking a point from \( X \).
\end{remark}

\begin{lemma}\label{thm:frame_homomorphism_kernel}\mcite[45]{Johnstone1983}
  Fix a \hyperref[thm:two_element_lattice]{two-element lattice} \( \set{ \top, \bot } \). It is vacuously a \hyperref[def:category_of_small_locales]{frame}. Fix also an arbitrary locale \( L \).

  Then the \hyperref[def:lattice/homomorphism]{lattice homomorphisms} \( f: L \to \set{ \top, \bot } \) is a \hyperref[def:category_of_small_frames]{frame homomorphism} if and only if \( f^{-1}(\top) \) is a \hyperref[def:lattice_ideal/prime]{completely prime filter}.
\end{lemma}
\begin{proof}
  \SufficiencySubProof Suppose that \( f: L \to \set{ \top, \bot } \) is a frame homomorphism.

  We will first show that \( f^{-1}(\top) \) is a filter.
  \begin{itemize}
    \item It is closed under meets. Let \( f(a) = f(b) = \top \). Then, since \( f \) preserves meets,
    \begin{equation*}
      f(a \wedge b) = f(a) \wedge f(b) = \top \wedge \top = \top.
    \end{equation*}

    \item It is closed under joins with elements of \( L \). Let \( a \in L \) and \( f(b) = \top \). Then
    \begin{equation*}
      f(a \vee b) = f(a) \vee f(b) = f(a) \vee \top = \top.
    \end{equation*}
  \end{itemize}

  It remains to show that \( f^{-1}(\top) \) is completely prime. Let
  \begin{equation*}
    \top = f\parens*{ \bigvee_{k \in \mscrK} a_k } = \bigvee_{k \in \mscrK} f(a_k).
  \end{equation*}

  If \( f(a_k) = \bot \) for every \( k \in \mscrK \),
  \begin{equation*}
    \bigvee_{k \in \mscrK} f(a_k) = \bot.
  \end{equation*}

  Hence, there exists some \( k \in \mscrK \) such that \( f(a_k) = \top \).

  Therefore, \( f^{-1}(\top) \) is a completely prime filter.

  \NecessitySubProof Suppose that \( f \) is a lattice homomorphism and that \( f^{-1}(\top) \) is a completely prime filter. We will show that \( f \) preserves arbitrary joins.

  Let \( \seq{ a_k }_{k \in \mscrK} \) be some family of members of \( L \).
  \begin{itemize}
    \item Suppose that \( f(a_k) = \bot \) for each \( k \in \mscrK \).

    If \( f\parens*{ \bigvee_{k \in \mscrK} a_k } = \top \), since \( f^{-1}(\top) \) is completely prime filter, there exists some \( k_0 \in \mscrK \) such that \( f(a_{k_0}) = \top \). The obtained contradiction shows that
    \begin{equation*}
      f\parens*{ \bigvee_{k \in \mscrK} a_k } = \bot = \bigvee_{k \in \mscrK} f(a_k)
    \end{equation*}

    \item Suppose that \( f(a_{k_0}) = \top \) for some \( k_0 \in \mscrK \).

    Then, since \( f \) preserves finite joins,
    \begin{equation*}
      f\parens*{ \bigvee_{k \in \mscrK} a_k }
      =
      f\parens*{ \bigvee_{k \neq k_0} a_k } \vee f(a_{k_0})
      =
      f\parens*{ \bigvee_{k \neq k_0} a_k } \vee \top
      =
      \top
      =
      \bigvee_{k \in \mscrK} f(a_k).
    \end{equation*}
  \end{itemize}
\end{proof}

\begin{remark}\label{rem:prime_elements_of_locale}
  We discussed in \fullref{rem:picking_a_point_from_a_locale} how to \enquote{pick a point} from a locale via locale maps from the two-element locale. \Fullref{thm:frame_homomorphism_kernel} and \fullref{def:lattice_prime_element} imply that these functions correspond exactly to prime elements of the locale.
\end{remark}

\begin{proposition}\label{thm:locale_to_topology}\mcite[45]{Johnstone1983}
  We define the set \( X \) of \term{points} of a \hyperref[def:category_of_small_locales]{locale} \( L \) as the set of all \hyperref[def:lattice_prime_element]{prime elements} of \( L \).

  Now the elements of \( L \) must become neighborhoods of these points. For each \( a \in L \), define the \enquote{closed} set
  \begin{equation*}
    F_a \coloneqq \set{ x \in X \given x \leq a }.
  \end{equation*}

  Finally, define
  \begin{equation*}
    \mscrF \coloneqq \set{ F_a \given a \in L }.
  \end{equation*}

  Then \( \mscrF \) is a family of closed sets for a topology on \( X \).
\end{proposition}
\begin{proof}
  The union and intersection of an arbitrary family of \enquote{closed} sets is again closed because \( L \) is complete. We only need to show \ref{thm:topology_from_closed_sets/C2}.

  First note that \( \bot \) is vacuously a prime element and that \( L \) has no elements strictly less than \( \bot \). Hence, \( F_\bot = \varnothing \) and thus \( \mscrF \) contains the empty set.

  The top \( \top \) is also prime, and \( F_\top = L \), implying that \( \mscrF \) also contains the set of all points.

  Therefore, \( \mscrF \) induces a topology on \( X \).
\end{proof}

  \section{Separation axioms}\label{sec:separation_axioms}

\begin{definition}\label{def:topological_space_separation}
  Two subsets \( A, B \subseteq X \) of a topological space \( (X, \mscrT) \) are called \term{separated} or \term{separated using neighborhoods} if there exist disjoint open sets \( U \supseteq A \) and \( V \supseteq B \). In particular, two points are separated if their respective singleton sets are separated.

  We say that \( A \) and \( B \) are \term{functionally separated} if there exists a continuous function \( f: X \to [0, 1] \) such that \( f(A) = 0 \) and \( f(B) = 1 \).
\end{definition}

\begin{definition}\label{def:topologically_indistinguishable}\mimprovised
  We say that two points in a topological space are \term{topologically indistinguishable} if they have the same \hyperref[def:neighborhood_filter]{neighborhood filters}, i.e. every neighborhood of one is a neighborhood of the other.
\end{definition}

\begin{definition}\label{def:separation_axioms}
  We can classify topological spaces using the following separation axioms. Fix a topological space \( (X, \mathcal{T}) \).

  \begin{thmenum}
    \thmitem[def:separation_axioms/T0]{T0} (Kolmogorov) \( X \) is \( T_0 \) if for every two different points \( x, y \in X \), there exists an open set \( U \in \mathcal{T} \) such that either \( x \in U \) or \( y \in U \).
    \thmitem[def:separation_axioms/T0.5]{T0.5} \( X \) is \( T_{0.5} \) if every singleton set \( \set{ x } \) is either open or closed.
    \thmitem[def:separation_axioms/T1]{T1} (Frechet) \( X \) is \( T_1 \) if every singleton set \( \set{ x } \) is closed.
    \thmitem[def:separation_axioms/T2]{T2} (Hausdorff) \( X \) is \( T_2 \) if every two different points \( x, y \in X \) can be separated using neighborhoods, i.e. there exist disjoint open sets \( U \ni x \) and \( V \ni y \).

    \thmitem[def:separation_axioms/T3]{T3} \( X \) is \term{regular} if every point and every closed set can be separated using \hyperref[def:topological_space_separation]{neighborhoods}.

    If in addition to being regular \( X \) is \ref{def:separation_axioms/T0}, we say that \( X \) is a \( T_3 \) space.

    \thmitem[def:separation_axioms/T3.5]{T3.5} (Tychonoff) \( X \) is \term{completely regular} if every point and every closed set can be functionally \hyperref[def:topological_space_separation]{separated}.

    If in addition to being completely regular \( X \) is \ref{def:separation_axioms/T0}, we say that \( X \) is a \( T_{3.5} \) space.

    \thmitem[def:separation_axioms/T4]{T4}(Urysohn) \( X \) is \term{normal} every two closed sets \( F, G \in \mathcal{F}_{\mathcal{T}} \) can be separated using neighborhoods, i.e. there exist disjoint open sets \( U \supseteq F \) and \( V \supseteq G \).

    If in addition to being normal \( X \) is \ref{def:separation_axioms/T1}, we say that \( X \) is a \( T_4 \) space.

    \thmitem[def:separation_axioms/T5]{T5} If every subspace of a \( T_4 \) space \( X \) is \ref{def:separation_axioms/T4}, we say that \( X \) is a \( T_5 \) space or a \term{completely normal space}.

    \thmitem[def:separation_axioms/T6]{T6} If every closed set in a \( T_4 \) space \( X \) is \( G_\delta \) (see \fullref{def:borel_hierarchy}), we say that \( X \) is a \( T_6 \) space or a \term{perfectly normal space}.
  \end{thmenum}
\end{definition}

\begin{proposition}\label{thm:separation_axioms_cascade}
  Each numbered axiom in \fullref{def:separation_axioms} implies the previous one.
\end{proposition}

\begin{proposition}\label{thm:t2_iff_singleton_limits}
  A topological space is \hyperref[def:separation_axioms/T2]{Hausdorff} if and only if every \hyperref[def:topological_net]{net} has at most one \hyperref[def:net_limit_point]{limit}.
\end{proposition}
\begin{proof}
  \SufficiencySubProof Let \( X \) be Hausdorff and assume that there exists a net \( \seq{ x_k }_{k \in \mscrK} \) such that \( y \) and \( z \) are not necessarily distinct limit points.

  Fix neighborhoods \( U \) of \( y \) and \( V \) of \( z \). Since both are limit points, there exist indices \( k_U \) and \( k_V \) such that \( k \geq k_U \) implies \( x_k \in U \) and \( i \geq i_k \) implies \( x_k \in V \).

  Since \( \mscrK \) is a directed set, there exists an upper bound \( k_0 \) of \( k_U \) and \( k_V \). Thus,
  \begin{equation*}
    x_k \in U \cap V \quad\forall k \geq k_0.
  \end{equation*}

  In particular, the intersection \( U \cap V \) is nonempty and is a neighborhood of both \( y \) and \( z \).

  If \( y \neq z \), then we have two distinct points such that no two neighborhoods of \( y \) and \( z \), respectively, are disjoint. This contradicts the assumption that \( X \) is Hausdorff. Thus, \( y = z \).

  \NecessitySubProof Conversely, if \( X \) is not Hausdorff, then for every two distinct points \( y \) and \( z \) and every two neighborhoods \( U \ni y \) and \( V \ni z \), their intersection \( U \cap V \) is nonempty.

  Let \( \mathcal{U} \) and \( \mathcal{V} \) be the sets of all neighborhoods of \( y \) and \( z \), respectively. Since they are both partially ordered by set inclusion \( \subseteq \), define the directed set \( (\mathcal{U} \times \mathcal{V}, \leq) \) with order
  \begin{equation*}
    (U, V) \leq (U', V') \iff U \supset V \T{and} U' \supset V'.
  \end{equation*}

  For each \( (U, V) \in \mathcal{U} \times \mathcal{V} \), choose a point \( x_{(U, V)} \) from \( U \cap V \).

  Thus, the net \( \{ x_{(U, V)} \}_{(U, V) \in \mathcal{U} \cap \mathcal{V}} \) has both \( y \) and \( z \) as its limit points, which contradicts our initial assumption.
\end{proof}

\begin{lemma}[Urysohn's lemma]\label{thm:urysohns_lemma}\mcite[1.5.11]{Engelking1989GeneralTopology}
  In a \hyperref[def:separation_axioms/T4]{normal space}, every pair \( A, B \) of disjoint closed sets can be functionally \hyperref[def:topological_space_separation]{separated}.
\end{lemma}

\begin{theorem}\label{thm:separation_axioms_of_product}
  Fix is an indexed family \( \{ X_k \}_{k \in \mscrK} \) of topological spaces. Denote their \hyperref[def:topological_product]{product} by \( X \).

  \begin{thmenum}
    \thmitem{thm:separation_axioms_of_product/direct}\mcite[theorem 2.3.11]{Engelking1989GeneralTopology} If each one of \( X_k \) is a \( T_i \) space for \ref{def:separation_axioms/T0}-\ref{def:separation_axioms/T3.5}, then \( X \) is also a \( T_i \) space.

    \medskip

    \thmitem{thm:separation_axioms_of_product/inverse}\mcite[theorem 2.3.11]{Engelking1989GeneralTopology} If \( X \) is a \( T_i \) space for \ref{def:separation_axioms/T0}-\ref{def:separation_axioms/T6}, then each component \( X_k \) is also a \( T_i \) space.
  \end{thmenum}
\end{theorem}

  \section{Connected spaces}\label{sec:connected_spaces}

\begin{definition}\label{def:connected_space}\mcite[thm. 6.1.1]{Engelking1989Topology}
  We say that the topological space \( X \) is \term{connected} if it satisfies any of the following equivalent conditions:
  \begin{thmenum}
    \thmitem{def:connected_space/open_union} If \( X = X_1 \cup X_2 \) and \( X_1, X_2 \) are disjoint open sets, either \( X_1 \) or \( X_2 \) is empty.

    \thmitem{def:connected_space/closed_union} If \( X = X_1 \cup X_2 \) and \( X_1, X_2 \) are disjoint closed sets, either \( X_1 \) or \( X_2 \) is empty.

    \thmitem{def:connected_space/separated_union} If \( X = X_1 \cup X_2 \) and \( X_1, X_2 \) are \hyperref[def:topological_space_separation]{separated}, either \( X_1 \) or \( X_2 \) is empty.

    \thmitem{def:connected_space/clopen} The only subsets of \( X \) that are both open and closed are \( \varnothing \) and \( X \).

    \thmitem{def:connected_space/discrete_mapping} Every continuous mapping \( f: X \to \{ 0, 1 \} \) into the two-point discrete space is constant.
  \end{thmenum}
\end{definition}

\begin{definition}\label{def:locally_connected_space}\mcite[exerc. 6.3.3]{Engelking1989Topology}
  We say that \( X \) is \term{locally connected} if for every point \( x \in X \) and every neighborhood \( U \) of \( x \) there exists a connected set \( C \subseteq U \) such that \( x \in \Int(C) \).
\end{definition}

\begin{definition}\label{def:path_connected_space}\mcite[exerc. 6.3.9]{Engelking1989Topology}
  We say that a topological space is \term{path connected} if every two points can be connected via a \hyperref[def:parametric_curve]{path}.
\end{definition}

\medskip

\begin{definition}\label{def:locally_path_connected_space}\mcite[exerc. 6.3.10]{Engelking1989Topology}
  We say that \( X \) is \term{locally path connected} if for every point \( x \in X \) and every neighborhood \( U \) of \( x \) there exists a neighborhood \( V \) of \( x \) such that for any \( y \in V \) there exists a path \( \gamma: [0, 1] \to U \) connecting \( x \) with \( y \).
\end{definition}

\begin{proposition}\label{thm:homomorphism_preserves_connectedness}
  If \( X \) is connected and \( f: X \to Y \) is a homeomorphism, then \( Y \) is also connected.
\end{proposition}
\begin{proof}
  Let \( Y = Y_1 \cup Y_2 \), where \( Y_1 \) and \( Y_2 \) are disjoint and open.

  Note that the preimages \( \gamma^{-1}(Y_1) \) and \( \gamma^{-1}(Y_2) \) are open and disjoint, hence \( X = \gamma^{-1}(Y_1) \cup \gamma^{-1}(Y_2) \). But \( X \) is connected and by \fullref{def:connected_space/open_union}, either \( \gamma^{-1}(Y_1) \) or \( \gamma^{-1}(Y_2) \) is the null set. Thus, either \( Y_1 \) and \( Y_2 \) is the null set and, again, by \fullref{def:connected_space/open_union}, \( Y \) is connected.
\end{proof}

\begin{proposition}\label{thm:path_connected_implies_connected}
  Any path connected space is connected.
\end{proposition}
\begin{proof}
  Let \( X = X_1 \cup X_2 \), where \( X_1 \) and \( X_2 \) are disjoint and open.

  Assume that both are nonzero and take \( x_1 \in X_1, x_2 \in X_2 \). Then there exists a path \( \gamma: I \to X \) with endpoints \( x_1 \) and \( x_2 \). Note that the preimages \( \gamma^{-1}(X_1) \) and \( \gamma^{-1}(X_2) \) are nonempty and open, hence cannot be separated by \fullref{def:connected_space/separated_union}. But this contradicts the disjointedness of \( X_1 \) and \( X_2 \).

  The obtained contradiction proves that \( X \) is connected.
\end{proof}

  \subsection{Compact spaces}\label{subsec:compact_spaces}

\begin{definition}\label{def:centered_family}\mcite[123]{Engelking1989}
  The nonempty family \( \mscrF \) of subsets of the topological space \( X \) is said to be a \term{centered family of sets} or to have the \term{finite intersection property} if the intersection \( F_1 \cap \cdots \cap F_n \) of any finite collection of sets is nonempty.
\end{definition}

\begin{definition}\label{def:sequentially_compact_space}
  \todo{Define}.
\end{definition}

\begin{definition}\label{def:compact_space}\mcite[123]{Engelking1989}
  The space \( X \) is called \term{compact} if any of the following equivalent finiteness conditions hold:
  \begin{thmenum}
    \thmitem{def:compact_space/finite_subcover} Every open cover of \( X \) has a finite subcover.
    \thmitem{def:compact_space/centered_family} Every centered \hyperref[def:centered_family]{family} \( \mscrF \) of closed subsets of \( X \) has a nonempty intersection.
    \thmitem{def:compact_space/convergent_nets} Every \hyperref[def:topological_net]{net} has a cluster point or, \hyperref[thm:def:net_convergence/cluster_point_iff_subnet_limit_point]{equivalently}, a \hyperref[def:net_convergence]{convergent} subnet. This property is also called \enquote{generalized sequential compactness} or, when restricted to sequences instead of general nets, simply \enquote{sequential compactness}.
  \end{thmenum}
\end{definition}
\begin{proof}
  \ImplicationSubProof{def:compact_space/finite_subcover}{def:compact_space/centered_family} Assume that every open cover of \( X \) has a finite subcover. Let \( \mscrF \) be a centered family of closed subsets of \( X \). Aiming at a contradiction, suppose that \( \bigcap \mscrF = \varnothing \). Then
  \begin{equation*}
    X
    =
    X \setminus \bigcap \mscrF
    =
    \bigcup_{F \in F} (X \setminus F),
  \end{equation*}
  which has a finite subcover indexed by, say, \( \mscrF' \subseteq F \). But \( \mscrF \) is a centered family and \( \bigcap \mscrF' \) is nonempty, hence
  \begin{equation*}
    X
    =
    \bigcup_{F \in F'} (X \setminus F)
    =
    X \setminus \bigcap \mscrF'
    \neq
    X.
  \end{equation*}

  The obtained contradiction shows that \( \bigcap \mscrF \) is nonempty.

  \ImplicationSubProof{def:compact_space/centered_family}{def:compact_space/finite_subcover} Assume that every centered family of closed sets has a nonempty intersection. Let \( \{ U_k \}_{k \in \mscrK} \) be an open cover of \( X \). By putting \( F_k \coloneqq U_k \) for all \( k \in \mscrK \), we obtain a family \( \{ F_k \}_{k \in \mscrK} \) of closed sets with an empty intersection. Therefore, it is not a centered family. Then there exists at least one finite subfamily \( \{ F_k \}_{k \in \mscrK'} \) with an empty intersection. The complement of this subfamily is then a finite cover of \( X \), which proves our statement.

  \ImplicationSubProof{def:compact_space/finite_subcover}{def:compact_space/convergent_nets} Assume that every open cover of \( X \) has a finite subcover. Fix a net \( \seq{ x_k }_{k \in \mscrK} \subseteq X \).

  Aiming at a contradiction, suppose that the net has no cluster points. For any point \( x \in X \) and any neighborhood \( U_x \) of \( x \), the net is not frequently in \( U_x \). Obviously \( \{ U_x \}_{x \in X} \) is an infinite open cover of \( X \). Then it has a finite subcover indexed by, say, \( X' \subseteq X \).

  Therefore, every element of the net \( \seq{ x_k }_{k \in \mscrK} \) if contained in one of the finitely many neighborhoods \( \{ U_x \}_{x \in X'} \) and the net itself is frequently in at least one of the neighborhoods.

  Thus, one of the finitely many points in \( X' \) is a cluster point of \( \seq{ x_k }_{k \in \mscrK} \).

  \ImplicationSubProof{def:compact_space/convergent_nets}{def:compact_space/centered_family}\mcite[thm. 3.1.23]{Engelking1989} Assume that every net has a cluster point. Let \( \{ F_k \}_{k \in \mscrK} \) be a central family of closed sets.

  Denote by \( \mathcal C \) the family of all finite subsets of \( \mscrK \). For each \( C \in \mathcal C \), the set \( \bigcap_{c \in C} F_c \) is a finite intersection of members of a central family and is hence nonempty. Choose an element \( x_C \in \bigcap_{c \in C} F_c \) for each \( C \in \mathcal C \).

  If we order \( \mathcal C \) by reverse inclusion, that is, if \( C \leq C' \iff \bigcap_{c \in C'} F_c \subseteq \bigcap_{c \in C} F_c \), then the family \( \{ x_C \}_{C \in \mathcal C} \) becomes a net. Our assumption is that it has a cluster point, say \( x_0 \).

  It remains to show that \( x_0 \) belongs to the intersection of the centered family \( \{ F_k \}_{k \in \mscrK} \) itself. Fix an element \( F_0 \) of this family and denote by \( C_0 \) the singleton set \( \{ k_0 \} \in \mathcal C \). Because \( x_0 \) is a cluster point, for every neighborhood \( U \) of \( x_0 \), there exists an index \( C \in \mathcal C \) such that \( C \geq C_0 \) and \( x_C \in U \). Then \( x_C \in \bigcap_{c \in C} F_c \subseteq \bigcap_{c \in C_0} F_c = F_0 \).

  Therefore, \( F_0 \cap U \neq \varnothing \) for all neighborhoods \( U \) of \( x_0 \). By \fullref{thm:properties_via_bases/closure}, \( x_0 \in F_0 \). Since \( F_0 \) was an arbitrary set from the centered family \( \{ F_k \}_{k \in \mscrK} \), we conclude that the intersection of the family is not empty. This proves the theorem.
\end{proof}

\begin{remark}\label{rem:precompact_set}
  If the closure of \( A \) is compact, we call \( A \) \term{relatively compact} or \term{precompact} (although the term \enquote{precompact} is also used for totally bounded sets, see \fullref{def:totally_bounded_set}).
\end{remark}

\begin{proposition}\label{thm:def:compact_space}
  \hyperref[def:compact_space]{Compact spaces} have the following basic properties:
  \begin{thmenum}
    \thmitem{thm:def:compact_space/closed} If \( X \) is any topological space and \( A \subseteq X \) is a compact subspace, then \( A \) is closed in \( X \).
    \thmitem{thm:def:compact_space/closed_subspace} A closed subspace of a compact space is compact.
  \end{thmenum}
\end{proposition}
\begin{proof}
  \SubProofOf{thm:def:compact_space/closed} If \( A \) is compact, by \fullref{def:compact_space/convergent_nets} every net in \( A \) has a cluster point in \( A \). By \fullref{thm:cluster_point_characterization}, \( A \) is closed in \( X \).

  \SubProofOf{thm:def:compact_space/closed_subspace} Let \( X \) be \hyperref[def:compact_space]{compact} and let \( X' \subseteq X \) be a closed subspace.

  Fix an open cover \( \{ U_k \}_{k \in \mscrK} \) of \( X' \). By definition of the subspace \hyperref[def:topological_subspace]{topology} for each \( U_k \) there exists a set \( V_k \) that is open in \( X \) and \( U_k = V_k \cap X' \). Since \( X' \) is closed, \( X \setminus X' \) is open and hence the family \( \{ V_k \}_{k \in \mscrK} \) together with \( X \setminus X' \) is an open cover of the space \( X \).

  By \fullref{def:compact_space/finite_subcover}, there exists a finite subcover \( \{ V_k \}_{k \in \mscrK'} \) that, along with \( X \setminus X' \), still covers \( X' \). Therefore, \( X' \) is also compact.
\end{proof}

\begin{theorem}[Tychonoff's product theorem]\label{thm:tychonoffs_product_theorem}\mcite[thm. 3.2.4]{Engelking1989}
  Let \( { (X_k, \mscrT_k) }_{k \in \mscrK} \) be a family of topological spaces. Their \hyperref[def:topological_product]{product} \( (\prod_{k \in \mscrK} X_k, \prod_{k \in \mscrK} \mscrT_k) \) is \hyperref[def:compact_space]{compact} if and only if \( (X_k, \mscrT_k) \) is compact for every \( k \in \mscrK \).
\end{theorem}
\begin{comments}
  \item Within \hyperref[def:zfc]{\logic{ZF}}, this theorem is equivalent to the \hyperref[def:zfc/choice]{axiom of choice} --- see \fullref{thm:axiom_of_choice_equivalences/tychonoff}.
\end{comments}

\begin{theorem}[Weierstrass' extreme value theorem]\label{thm:weierstrass_extreme_value_theorem}\mcite[cor. 3.2.9]{Engelking1989}
  Let \( X \) be a compact topological space and let \( f: X \to \BbbR \) be a continuous function into the \hyperref[def:real_numbers]{real numbers}.

  Then \( f \) is \hyperref[def:metric_space/bounded_function]{bounded} and there exist \( m, M \in X \) such that
  \begin{balign*}
    f(m) = \min_{x \in X} f(x)
     &  &
    f(M) = \max_{x \in X} f(x).
  \end{balign*}
\end{theorem}

\begin{definition}\label{def:locally_compact_space}\mcite[148]{Engelking1989}
  A topological space is called \term{locally compact} if every point has a relatively compact neighborhood.
\end{definition}

\begin{definition}\label{def:paracompact_space}
  \todo{Define}.
\end{definition}

\begin{definition}\label{def:metacompact_space}
  \todo{Define}.
\end{definition}

  \subsection{Baire spaces}\label{subsec:baire_spaces}

\begin{remark}\label{rem:baire_categories}
  René-Louis Baire introduced the concept of \term{Baire categories} in 1899, almost 50 years before Samuel Eilenberg and Saunders MacLane introduced \term{categories} in \cite{EilenbergMacLane1945Equivalences} (see \fullref{sec:category_theory} for the latter).

  Unfortunately, topology utilizes both concepts, so the word \enquote{category} should be used with caution. To circumvent this, we use alternative terminology for Baire categories.
\end{remark}

\begin{definition}\label{def:meager_set}\mcite[def. 2.1]{Rudin1991Functional}
  Any countable union of \hyperref[def:topologically_dense_set]{nowhere dense sets} is called \term{meager} or a \term{first category set} (see \fullref{rem:baire_categories} for terminology). If a set is not meager, we call it \term{nonmeager} or a \term{second category set}.
\end{definition}

\begin{proposition}\label{thm:def:meager_set}\mcite[43]{Rudin1991Functional}
  \hyperref[def:meager_set]{Meager sets} have the following basic properties:
  \begin{thmenum}
    \thmitem{thm:def:meager_set/union} A countable union of meager sets is meager.
    \thmitem{thm:def:meager_set/subset} A subset of a meager set is meager.
    \thmitem{thm:def:meager_set/homeomorphism} The \hyperref[def:homeomorphism]{homeomorphic} image of a set \( A \) is meager if and only if \( A \) itself is meager.
  \end{thmenum}
\end{proposition}
\begin{proof}
  \SubProofOf{thm:def:meager_set/union} Follows from \fullref{thm:countably_infinite_union_of_countably_infinite_sets}.
  \SubProofOf{thm:def:meager_set/subset} Fix a meager set \( A \) and let \( B \subseteq A \). Then \( A = \bigcup_{k=1}^\infty A_k \) for some nowhere dense sets \( A_1, A_2, \ldots \). By \fullref{thm:def:topologically_dense_set/nowhere_dense_subset}, the sets \( A_1 \cap B, A_2 \cap B, \ldots \) are also nowhere dense. But
  \begin{equation*}
    B
    =
    A \cap B
    =
    \left(\bigcup_{k=1}^\infty A_k \right) \cap B
    \reloset {\ref{thm:boolean_algebra_of_subsets}} =
    \bigcup_{k=1}^\infty (A_k \cap B).
  \end{equation*}

  Therefore, \( B \) is also nowhere dense.

  \SubProofOf{thm:def:meager_set/homeomorphism}
\end{proof}

\begin{definition}\label{def:baire_space}
  A topological space is called a \term{Baire space} if any of the following equivalent conditions hold:
  \begin{thmenum}
    \thmitem{def:baire_space/meager} Every nonempty open set is nonmeager.
    \thmitem{def:baire_space/dense} A countable intersection of dense sets is dense.
  \end{thmenum}
\end{definition}
\begin{proof}
  \EquivalenceSubProof{def:baire_space/meager}{def:baire_space/dense} Follows from \fullref{thm:def:topologically_dense_set/complement} and \fullref{thm:de_morgans_laws}.
\end{proof}

\begin{proposition}\label{thm:open_subspace_of_baire_space_is_baire}
  Every open subspace of a \hyperref[def:baire_space]{Baire space} is a Baire space.
\end{proposition}
\begin{proof}
  Let \( (X, \mscrT) \) be a Baire space and let \( (X', \mscrT_{X'}) \) be an open \hyperref[def:topological_subspace]{subspace} with the canonical embedding \( \iota: X' \to X \). The proposition holds vacuously if \( X' = \varnothing \), so we assume that \( X' \neq \varnothing \).

  Note that \( \iota \) is continuous by definition, however it is also an open map because if \( U \in T_{X'} \), then \( \iota(U) = U \cap X' \) is open in \( X \) as the intersection of two open sets. Therefore, it is a homeomorphic embedding and by \fullref{thm:def:meager_set/homeomorphism}, \( U \) is meager if and only if \( \iota(U) \) is meager. Since \( X \) is a Baire space, \( \iota(U) \) is not meager and hence \( U \) is also not meager.

  We showed that every nonempty open set \( U \in T_{X'} \) is nonmeager, therefore \( X' \) is a Baire space.
\end{proof}

\begin{theorem}[Baire category theorem]\label{thm:baire_category_theorem}\mcite{Rudin1991Functional}
  \begin{thmenum}
    \thmitem{thm:baire_category_theorem/metric} \hyperref[def:complete_metric_space]{Complete metric spaces} are \hyperref[def:baire_space]{Baire spaces}.
    \thmitem{thm:baire_category_theorem/compact} \hyperref[def:locally_compact_space]{Locally compact} \hyperref[def:separation_axioms/T2]{Hausdorff} spaces are Baire spaces.
  \end{thmenum}
\end{theorem}

  \subsection{Uniform spaces}\label{subsec:uniform_spaces}

\begin{remark}\label{rem:entourage_notation}
  \hyperref[def:uniform_space]{Uniform spaces} are an extension of both \hyperref[def:metric_space]{metric spaces} and \hyperref[def:topological_group]{topological groups} (including \hyperref[def:topological_vector_space]{topological vector spaces}). They are topological spaces that are \enquote{uniform} in the sense that different parts of the space behave the same.

  In metric spaces, we use the notation \( \mu(x, y) < \varepsilon \) to mean that \( x \) and \( y \) are close (at a distance less than \( \varepsilon \)).

  In (\hyperref[rem:additive_semigroup]{additive}) topological groups, we instead have linear operations and use \( x - y \in U \) to mean that \( x \) and \( y \) are close (their difference belongs to some neighborhood of \( 0 \)).

  A proper generalization needs to make both metric spaces and topological groups feel natural as special cases. Generalizing metric space \hyperref[def:metric_space/ball]{balls} or neighborhoods of \hyperref[thm:origin_neighborhoods_in_topological_groups]{zero} are nice options which unfortunately introduces some asymmetry since, for example for metric spaces, \( \mu(x, y) < \varepsilon \) can be written as either \( y \in B(x, \varepsilon) \) or \( x \in B(y, \varepsilon) \). This approach does not go far beyond what general topological spaces offer as a notation.

  \cite[section 8]{Engelking1989} uses the notation \( \abs{x - y} < V \) to mean that \( x \) and \( y \) belong to the \hyperref[def:entourage]{entourage} \( V \). This is a bit confusing because no absolute \hyperref[def:absolute_value]{value} nor subtraction are defined in uniform spaces. We find it simpler to not introduce any special notation beyond that of \hyperref[def:relation]{relations}, and so we denote the same by \( (x, y) \in V \).
\end{remark}

\begin{definition}\label{def:entourage}\mcite[sec. 8.1]{Engelking1989}
  Let \( X \) be a set. For two binary \hyperref[def:relation]{relations} \( V \) and \( U \) on \( X \) we define their sum as
  \begin{equation*}
    V + U \coloneqq \{ (x, z) \colon \exists y \in X: (x, y) \in U, (y, z) \in V \}
  \end{equation*}
  and \( nV \) by \( n \)-fold iterative addition.

  For any relation \( V \), we denote by \( -V \) the \hyperref[def:binary_relation/inverse]{inverse relation}.

  A relation \( V \) on \( X \) is called an \term{entourage} if \( V \) is \hyperref[def:binary_relation/reflexive]{reflexive} and \hyperref[def:binary_relation/symmetric]{symmetric}.

  In analogy to \hyperref[def:metric_space]{metric spaces}, we define
  \begin{thmenum}
    \thmitem{def:entourage/ball} We define the \term{open ball} or simply \term{ball} with \term{center} \( x \) and \term{radius} \( V \) to be the set
    \begin{equation*}
      B(x, V) \coloneqq \{ y \in X \colon (y, x) \in V \}.
    \end{equation*}

    \thmitem{def:entourage/bounded_set} We say that the set \( S \subseteq X \) is \term{bounded} if it is contained in some ball.
  \end{thmenum}
\end{definition}

\begin{proposition}\label{thm:entourage_simulates_metric}\mcite[sec. 8.1]{Engelking1989}
  Using the notation of \fullref{def:entourage}, we obtain properties similar to those of metrics:
  \begin{thmenum}
    \refitem{def:metric_space/M1} \( (x, x) \in V \)
    \refitem{def:metric_space/M2} \( (x, y) \in V \) if and only if \( (y, x) \in V \)
    \refitem{def:metric_space/M3} If \( (x, y) \in U \) and \( (y, z) \in V \), then \( (x, y) \in U + V \).
  \end{thmenum}
\end{proposition}

\begin{definition}\label{def:uniform_space}\mcite[sec. 8.1]{Engelking1989}
  A \term{uniform space} is a set \( X \) with a nonempty family \( \mscrV \) of \hyperref[def:entourage]{entourages} on \( X \) such that
  \begin{thmenum}
    \thmitem[def:uniform_space/U1]{U1} If \( V \in V \) and \( V \subseteq W \) for some entourage \( W \) on \( X \), then \( W \in V \).
    \thmitem[def:uniform_space/U2]{U2} If \( V_1, V_2 \in V \), then \( V_1 \cap V_2 \in V \).
    \thmitem[def:uniform_space/U3]{U3} For every \( V \in V \) there exists \( W \in V \) such that \( 2W \subseteq V \).
    \thmitem[def:uniform_space/U4]{U4} \( \bigcap \mscrV = \Delta_X \), where \( \Delta_X \) is the \hyperref[def:binary_relation/diagonal]{diagonal relation}.
  \end{thmenum}

  The family \( \mscrV \) is called a \term{uniform structure} or \term{uniformity} on \( X \).
\end{definition}

\begin{definition}\label{def:uniform_topology}
  Let \( (X, \mscrV) \) be a \hyperref[def:uniform_space]{uniform space}. We define its \term{uniform topology} or \term{induced topology} as the \hyperref[def:topological_space]{topology} generated by the \hyperref[def:topological_local_base]{neighborhood filter}
  \begin{equation*}
    \mathcal{B}(x) \coloneqq \{ B(x, V) \colon V \in V \}.
  \end{equation*}

  If for some topological space \( (X, \mscrT) \) there exists a uniformity such that \( \mscrT \) is its induced topology, we say that the topology \( \mscrT \) is \term{uniformizable}.
\end{definition}
\begin{proof}
  This proof of correctness does not actually rely on the uniform structure (except for \( \mscrV \) being nonempty), but rather on the properties of entourages.

  It is indeed a neighborhood filter because
  \begin{refenum}
    \refitem{thm:topology_from_local_base/BP1} Every \hyperref[def:entourage]{entourage} is reflexive, hence \( x \) is contained in every ball in \( \mathcal{B}(x) \).

    \refitem{thm:topology_from_local_base/BP3} For \( B(x, U) \) and \( B(x, V) \) we have
    \begin{balign*}
      B(x, U \cap V)
       & =
      \{ y \in X \colon (x, y) \in U \cap V \}
      =    \\ &=
      \{ y \in X \colon (x, y) \in U \text{ and } (x, y) \in V \}
      =    \\ &=
      B(x, U) \cap B(x, V).
    \end{balign*}

    \refitem{thm:topology_from_local_base/BP2} Fix \( x, y \in X \) and a ball \( B(y, V) \in \mathcal{B}(y) \) that contains \( x \). We will show that \( B(y, V) \subseteq B(x, 2V) \).

    Fix \( z \in B(y, V) \). We have \( (z, y) \in V \). Then \( (z, x) \in V + V = 2V \). Since \( z \in B(y, V) \) was arbitrary, we conclude that \( B(y, V) \subseteq B(x, 2V) \).
  \end{refenum}
\end{proof}

\begin{theorem}\label{thm:tychonoff_spaces_are_uniformizable}\mcite[thm. 8.1.20]{Engelking1989}
  A topological space is \hyperref[def:uniform_topology]{uniformizable} if and only if it is a \hyperref[def:separation_axioms/T3.5]{Tychonoff space}.
\end{theorem}

\begin{definition}\label{def:uniform_space_base}
  Fix a uniform space \( (X, \mscrV) \). The subfamily \( \mscrB \subseteq V \) if entourages is called a \term{base} for \( \mscrV \) if every entourage \( V \in V \) contains a member of \( \mscrV \).
\end{definition}

\begin{definition}\label{thm:uniform_space_base_axioms}\mcite[prop. 8.1.14]{Engelking1989}
  Let \( X \) be an arbitrary set and let \( \mscrB \) be a family of entourages satisfying the following axioms:
  \begin{thmenum}
    \thmitem[thm:uniform_space_base_axioms/BU1]{BU1} If \( V_1, V_2 \in B \), there exists an entourage \( V \in B \) such that \( V \subseteq V_1 \cap V_2 \).
    \thmitem[thm:uniform_space_base_axioms/BU2]{BU2} For every \( V \in B \) there exists \( W \in B \) such that \( 2W \subseteq V \).
    \thmitem[thm:uniform_space_base_axioms/BU3]{BU3} \( \bigcap \mscrB = \Delta_X \)
  \end{thmenum}

  Then the family of entourages
  \begin{balign}\label{thm:uniform_space_base_axioms/uniformity}
    \mscrV \coloneqq \left\{ V \subseteq X \times X \colon \exists B \in B: B \in V \T{and} V \text{ is reflexive and symmetric} \right\}
  \end{balign}
  is a uniform structure on \( X \). Furthermore, \( \mscrB \) is a \hyperref[def:uniform_space_base]{base} of \( \mscrV \).

  In particular, the base on any topology satisfies \fullref{thm:uniform_space_base_axioms/BU1} -- \fullref{thm:uniform_space_base_axioms/BU2}.
\end{definition}

\begin{lemma}\label{thm:uniform_space_neighborhood_contains_ball}
  In a uniform space \( (X, \mscrV) \), for every neighborhood \( A \) (in the topology) of a point \( x_0 \in X \) there exists an entourage \( V \in V \) such that \( B(x_0, V) \subseteq A \).
\end{lemma}
\begin{proof}
  By \fullref{def:uniform_topology} and \fullref{def:topological_base/union}, \( A \) is a union of balls centered at \( x_0 \). For any ball \( B(x_0, V) \) of this union, we have \( B(x_0, V) \subseteq A \).
\end{proof}

\begin{proposition}\label{thm:def:uniform_topology}
  The \hyperref[def:uniform_topology]{uniform topology} \( \mscrT \) on \( (X, \mscrV) \) the following basic properties:
  \begin{thmenum}
    \thmitem{thm:def:uniform_topology/ball_is_open} All \hyperref[def:entourage/ball]{balls} are open sets.
    \thmitem{thm:def:uniform_topology/neighborhood_contains_ball} Every neighborhood of every point a ball centered at that point.
  \end{thmenum}
\end{proposition}
\begin{proof}
  \SubProofOf{thm:def:uniform_topology/ball_is_open} We defined the balls to be the base of the uniform topology, therefore they are open.
  \refitem{thm:def:uniform_topology/neighborhood_contains_ball} Fix a point \( x_0 \). It is a trivial consequence of \fullref{def:topological_base/subset} that every neighborhood of \( x_0 \) contains some ball centered at a point that is not necessarily \( x_0 \). By \fullref{thm:topology_from_local_base/BP3}, this ball contains another ball centered at \( x_0 \).
\end{proof}

\begin{proposition}\label{thm:uniform_space_local_convergence}
  Fix a topological space \( (X, \mscrT) \) and a uniform space \( (Y, \mscrU) \). Let \( A \subseteq X \) be a nonempty set and let \( f: A \to Y \) be a function. Then \( y_0 \) is a limit point of \( f \) at \( x_0 \in X \) in the sense of \fullref{def:local_continuity} if and only if
  \begin{equation}\label{thm:uniform_space_local_convergence/topological_source}
    \forall V \in V \ \exists A \in T(x_0) : x \in A \implies (f(x), y_0) \in V.
  \end{equation}

  If instead, \( (X, \mscrU) \) is a uniform space, then \( y_0 \) is a limit point of \( f \) at \( x_0 \in X \) if and only if
  \begin{equation}\label{thm:uniform_space_local_convergence/uniform_source}
    \forall V \in V \ \exists U \in U : (x, x_0) \in U \implies (f(x), y_0) \in V.
  \end{equation}

  Note that the limit point may not be unique because uniform spaces are not \hyperref[def:separation_axioms/T2]{Hausdorff} in general.
\end{proposition}
\begin{proof}
  We will only prove \fullref{thm:uniform_space_local_convergence/uniform_source} because our proof of \fullref{thm:uniform_space_local_convergence/topological_source} is a special case.

  \SufficiencySubProof Suppose that \( y_0 \) is a limit point of \( f \) at \( x_0 \) and fix a neighborhood \( B \) of \( y_0 \). Then there exists a neighborhood \( A \) of \( x_0 \) such that \( f(A) \subseteq B \).

  Fix an entourage \( V \in V \). Then \( B(y_0, V) \) is also a neighborhood of \( y_0 \). By \fullref{thm:uniform_space_neighborhood_contains_ball} and \fullref{def:uniform_space/U2}, there exists an entourage \( V' \subseteq V \) such that \( B(f(x), V') \subseteq B \cap B(y_0, V) \).

  Fix an entourage \( U \in U \) such that \( B(x_0, U) \subseteq A \). Then for any \( x \in X \), if \( (x, x_0) \in U \), we have \( (f(x), y_0) \in V' \). But \( V' \subseteq V \), therefore
  \begin{equation*}
    (x, x_0) \in U \implies (f(x), y_0) \in V.
  \end{equation*}

  This concludes the proof.

  \NecessitySubProof Fix a neighborhood \( B \) of \( y_0 \) and an entourage \( V \in V \) such that \( B(x_0, V) \subseteq B \) (see \fullref{thm:uniform_space_neighborhood_contains_ball} for a justification). Then there exists \( U \in U \) such that
  \begin{equation*}
    (x, x_0) \in U \implies (f(x), y_0) \in V.
  \end{equation*}

  Therefore, \( A \coloneqq B(x_0, U) \) is a neighborhood of \( x_0 \) such that \( f(A) \subseteq B \).
\end{proof}

\begin{corollary}\label{thm:uniform_space_local_continuity}
  A function \( f: (X, \mscrV) \to (Y, \mscrU) \) between uniform spaces is continuous at \( x_0 \in X \) if and only if
  \begin{equation*}
    \forall V \in V \ \exists U \in U : (x, x_0) \in U \implies (f(x), f(x_0)) \in V.
  \end{equation*}
\end{corollary}

\begin{definition}\label{def:bounded_function}
  Fix a set \( X \) and a \hyperref[def:uniform_space]{uniform space} \( (Y, \mscrV) \). Fix a function \( f: X \to Y \).

  \begin{thmenum}
    \thmitem{def:bounded_function/bounded} We say that the function \( f: X \to Y \) is \term{bounded} if \( f(X) \) is a bounded set, that is, if there exists a \hyperref[def:entourage/ball]{ball} \( B(y, V) \) such that \( f(X) \subseteq B(y, V) \).

    \thmitem{def:bounded_function/bounded_family} We say that the family of functions \( \mscrF \) from \( X \) to \( Y \) is \term{bounded} at \( x_0 \) if there exists a ball \( B(y, V) \) such that the set \( \mscrF(x_0) \coloneqq \{ f(x_0) \colon f \in F \} \) is contained in \( B(y, V) \).

    \thmitem{def:bounded_function/pointwise} We say that \( \mscrF \) is \term{pointwise bounded} on the set \( S \subseteq X \) if
    \begin{equation*}
      \forall x \in S \ \exists B(y, V) : \mscrF(x) \subseteq B(y, V).
    \end{equation*}

    \thmitem{def:bounded_function/uniform} We say that \( \mscrF \) is \term{uniformly bounded} on \( S \subseteq X \) if
    \begin{equation*}
      \exists B(y, V) \ \forall x \in S : \mscrF(x) \subseteq B(y, V).
    \end{equation*}

    \thmitem{def:bounded_function/locally_bounded} If there is a topology \( \mscrT \) on \( X \), we say that the function \( f: X \to Y \) is \term{locally bounded} if there exists an entourage \( V \in V \) such that for each neighborhood \( A \in T(x) \) we have \( \diam{f(A)} < V \).
  \end{thmenum}
\end{definition}

\begin{proposition}\label{thm:continuous_implies_locally_bounded}
  Let \( (X, \mscrT) \) be a topological space and \( (Y, \mscrV) \) be a uniform space. Any \hyperref[thm:uniform_space_local_convergence/topological_source]{continuous function} from \( X \) to \( Y \) is locally \hyperref[def:bounded_function/locally_bounded]{bounded}.
\end{proposition}
\begin{proof}
  Trivial.
\end{proof}

\begin{definition}\label{def:function_net_convergence}
  Fix a set \( X \) and a \hyperref[def:uniform_space]{uniform space} \( (Y, \mscrV) \). Let \( \{ f_k \}_{k \in \mscrK} \) be a \hyperref[def:topological_net]{net} of functions from \( X \) to \( Y \). We say that \( \{ f_k \}_{k \in \mscrK} \) \term{converges pointwise} to the function \( f \) and write \( f_k \to f \) if
  \begin{equation}\label{def:function_net_convergence/pointwise}
    \forall V \in V \ \underbrace{\forall x \in X \ \exists k_0 \in \mscrK} : k \geq k_0 \implies (f_k(x), f(x)) \in V
  \end{equation}
  and that \( \{ f_k \}_{k \in \mscrK} \) \term{converges uniformly} to \( f \) and write \( f_k \multto f \) if
  \begin{equation}\label{def:function_net_convergence/uniform}
    \forall V \in V \ \underbrace{\exists k_0 \in \mscrK \ \forall x \in X} : k \geq k_0 \implies (f_k(x), f(x)) \in V
  \end{equation}

  In the special case where \( X \) is a topological space with topology \( \mscrT \), we call the sequence \( \{ f_k \}_{k \in \mscrK} \) \term{locally uniformly convergent} (see \cite{ProofWiki:locally_uniform_convergence}) if every point in \( S \) has a neighborhood in which the sequence converges uniformly. Symbolically,
  \begin{equation}\label{def:function_net_convergence/locally_uniform}
    \forall V \in V \ \forall x_0 \in S \ \exists A \in T(x_0) \ \exists k_0 \in \mscrK \ \forall x \in A : k \geq k_0 \implies (f_k(x), f(x)) \in V.
  \end{equation}

  If the index \( k_0 \) does not depend on the neighborhood \( A \) and the point \( x_0 \), then this is equivalent to uniform convergence. It is still more powerful than pointwise convergence. For example, \hyperref[def:convergent_power_series]{power series} are locally uniformly convergent on the interior of their domain of convergence - see \fullref{thm:power_series_are_locally_uniform_convergent}.

  A slightly weaker notion is that of \term{compact convergence} (see \cite{ProofWiki:compact_convergence}), which is defined as uniform convergence on any compact subset. Symbolically,
  \begin{equation}\label{def:function_net_convergence/compact}
    \forall V \in V \ \forall \text{ compact } C \subseteq S \ \exists k_0 \in \mscrK \ \forall x \in C : k \geq k_0 \implies (f_k(x), f(x)) \in V.
  \end{equation}
\end{definition}

\begin{definition}\label{def:uniform_continuity}\mcite[435]{Engelking1989}
  Fix two \hyperref[def:uniform_space]{uniform spaces} \( (X, \mscrU) \) and \( (Y, \mscrV) \). We say that the function \( f: X \to Y \) if is \term{uniformly continuous} on the set \( S \subseteq X \) if
  \begin{equation}\label{def:uniform_continuity/uniform}
    \forall V \in V \ \underbrace{\exists U \in U \ \forall x_1, x_2 \in S} : (x_1, x_2) \in U \implies (f(x_1), f(x_2)) \in V.
  \end{equation}

  Compare this to \term{pointwise continuity} on \( S \), which is defined by \fullref{thm:uniform_space_local_convergence/uniform_source} as convergence for any \( x_1 \in X \):
  \begin{equation}\label{def:uniform_continuity/pointwise}
    \forall V \in V \ \underbrace{\forall x_1, x_2 \in S \ \exists U \in U} : (x_1, x_2) \in U \implies (f(x_1), f(x_2)) \in V.
  \end{equation}
\end{definition}

\begin{definition}\label{def:function_set_continuity}\mcite[285]{BouziadTroallic2004}
  Fix a topological space \( (X, \mscrT) \) and a \hyperref[def:uniform_space]{uniform space} \( (Y, \mscrV) \). We say that the family \( \mscrF \) of functions from \( X \) to \( Y \) is \term{functionwise continuous} at \( x_0 \in X \) if
  \begin{equation}\label{def:function_set_continuity/functionwise}
    \forall V \in V \ \underbrace{\forall f \in F \ \exists A \in T(x_0)} : f(A) \subseteq B(f(x_0), V),
  \end{equation}
  and \term{equicontinuous} at \( x_0 \in X \) if
  \begin{equation}\label{def:function_set_continuity/equicontinuous}
    \forall V \in V \ \underbrace{\exists A \in T(x_0) \ \forall f \in F} : f(A) \subseteq B(f(x_0), V).
  \end{equation}

  In the special case where \( (X, \mscrU) \) is a uniform space, then we can define \term{uniform equicontinuity} of the family \( \mscrF \) on the set \( S \subseteq X \) as
  \begin{equation}\label{def:function_set_continuity/uniform_equicontinuous}
    \forall V \in V \ \underbrace{\exists U \in U \ \forall f \in F \ \forall x_1, x_2 \in S} : (x_1, x_2) \in U \implies (f(x_1), f(x_2)) \in V
  \end{equation}

  Compare this to \term{pointwise equicontinuity} of \( \mscrF \) on \( S \), as defined by \fullref{def:function_set_continuity/equicontinuous} for all \( x_1, x_2 \in S \),
  \begin{equation}\label{def:function_set_continuity/pointwise_equicontinuous}
    \forall V \in V \ \underbrace{\forall x_1, x_2 \in S \ \exists U \in U \ \forall f \in F} : (x_1, x_2) \in U \implies (f(x_1), f(x_2)) \in V
  \end{equation}
  to \term{functionwise uniform continuity} of \( \mscrF \) on \( S \), which is defined by \fullref{def:uniform_continuity/uniform} for all \( f \in F \),
  \begin{equation}\label{def:function_set_continuity/uniform_functionwise}
    \forall V \in V \ \underbrace{\forall f \in F \ \exists U \in U \ \forall x_1, x_2 \in S} : (x_1, x_2) \in U \implies (f(x_1), f(x_2)) \in V
  \end{equation}
  and to \term{functionwise pointwise continuity} of \( \mscrF \) on \( S \), i.e. regular continuity as defined by \fullref{thm:uniform_space_local_continuity} for all \( x_1, x_2 \in S \) and all \( f \in F \),
  \begin{equation}\label{def:function_set_continuity/functionwise_pointwise}
    \forall V \in V \ \underbrace{\forall f \in F \ \forall x_1, x_2 \in S \ \exists U \in U} : (x_1, x_2) \in U \implies (f(x_1), f(x_2)) \in V
  \end{equation}
\end{definition}

\begin{proposition}\label{thm:uniform_limit_of_continuous_functions}
  \hfill
  \begin{thmenum}
    \thmitem{thm:uniform_limit_of_continuous_functions/continuous} A locally \hyperref[def:function_net_convergence]{uniform limit} of functions continuous at a \hyperref[thm:uniform_space_local_continuity]{point} is continuous at that point.
    \thmitem{thm:uniform_limit_of_continuous_functions/uniform} A uniform limit of functions uniformly \hyperref[def:uniform_continuity]{continuous} on a set is uniformly continuous on the set.
  \end{thmenum}
\end{proposition}
\begin{proof}
  The two proofs are similar, but have a lot of subtle differences.

  Fix uniform spaces \( (X, \mscrU) \) and \( (Y, \mscrV) \). Let \( \{ f_k \}_{k \in \mscrK} \) be a \hyperref[def:topological_net]{net} of functions from \( S \subseteq X \) to \( (Y, \mscrV) \).

  \SubProofOf{thm:uniform_limit_of_continuous_functions/continuous} Assume that the functions \( f_k, k \in \mscrK \) are continuous and that they converge to the function \( f \) locally \hyperref[def:function_net_convergence/locally_uniform]{uniformly}.

  Fix an entourage \( W \in V \) and use \fullref{def:uniform_space/U3} to obtain \( V \subseteq W \) such that \( 3V \subseteq W \).

  . and a point \( x_0 \in S \). Let \( A \) be a neighborhood of \( x_0 \). From locally \hyperref[def:function_net_convergence]{uniform convergence}, there exists an index \( k_0 \in \mscrK \) such that
  \begin{equation*}
    \forall k > k_0 \ \forall x \in A : (f_k(x), f(x)) \in V.
  \end{equation*}

  Fix \( k > k_0 \). From \hyperref[def:function_net_convergence/locally_uniform]{uniform continuity}, there exists an entourage \( U \in U \) such that
  \begin{equation*}
    \forall x \in S : (x, x_0) \in U \implies (f_k(x_0), f_k(x)) \in V.
  \end{equation*}

  Combining the last two inequalities, we note that for any \( x \in A \),
  \begin{itemize}
    \item \( (f(x_0), f(x)) \in V \),
    \item \( (f_k(x_0), f(x_0)) \in V \),
    \item \( (f_k(x), f(x)) \in V \),
  \end{itemize}
  thus by applying the triangle inequality in \fullref{thm:entourage_simulates_metric} twice, we obtain
  \begin{equation*}
    (f(x_0), f(x)) \in 3V \subseteq W \quad\forall x \in A \cap B(x_0, U).
  \end{equation*}

  Given an entourage \( W \in V \), we found a neighborhood \( A \cap B(x_0, U) \) of \( x_0 \) such that \fullref{thm:uniform_space_local_convergence/topological_source} is satisfied. Thus, \( f \) is continuous at \( x_0 \).

  \SubProofOf{thm:uniform_limit_of_continuous_functions/continuous} Assume that the functions \( f_k, k \in \mscrK \) are uniformly continuous and that they converge to \( f \) \hyperref[def:function_net_convergence/locally_uniform]{uniformly}.

  As in \fullref{thm:uniform_limit_of_continuous_functions/continuous}, fix entourages \( V, W \in V \) such that \( 3V \subseteq W \). From \hyperref[def:uniform_continuity]{uniform continuity},
  \begin{equation*}
    \forall k \in \mscrK \ \exists U \in U \ \forall x_1, x_2 \in S : (x_1, x_2) \in U \implies (f_k(x_1), f_k(x_2)) \in V.
  \end{equation*}

  From \hyperref[def:function_net_convergence]{uniform convergence}, there exists an index \( k_0 \in \mscrK \) such that
  \begin{equation*}
    \forall k > k_0 \ \forall x \in S : (f_k(x), f(x)) \in V.
  \end{equation*}

  Fix an index \( k > k_0 \) and let \( U \in U \) be such that
  \begin{equation}\label{thm:uniform_limit_of_continuous_functions/uniform/continuity}
    \forall x_1, x_2 \in S : (x_1, x_2) \in U \implies (f_k(x_1), f_k(x_2)) \in V.
  \end{equation}

  For any two points \( x_1, x_2 \in S \), we also have that
  \begin{equation}\label{thm:uniform_limit_of_continuous_functions/uniform/convergence}
    (f(x_i), f_k(x_i)) \in V, i = 1, 2.
  \end{equation}

  Analogously to \fullref{thm:uniform_limit_of_continuous_functions/continuous}, from \eqref{thm:uniform_limit_of_continuous_functions/uniform/continuity} and \eqref{thm:uniform_limit_of_continuous_functions/uniform/convergence}, we obtain
  \begin{equation*}
    \forall x_1, x_2 \in S : (x_1, x_2) \in U \implies (f(x_1), f(x_2)) \in 3V \subseteq W.
  \end{equation*}

  Thus, the entourage \( U \) depends on \( W \) and not on \( x_1 \) and \( x_2 \). Technically, it also depends on \( k_0 \), however we are only concerned with existence and not uniqueness. Hence, \( f \) is uniformly continuous.
\end{proof}

\begin{definition}\label{def:category_of_uniform_spaces}
  Uniform spaces and \hyperref[def:uniform_continuity]{uniformly continuous functions} form a subcategory of \( \cat{Top} \) (see \fullref{def:category_of_small_topological_spaces}). We denote this category by \( \cat{Met} \).
\end{definition}

\begin{definition}\label{def:fundamental_net}
  A \hyperref[def:topological_net]{net} \( \seq{ x_k }_{k \in \mscrK} \) in a uniform space \( (X, \mscrV) \) is called a \term{fundamental net} or \term{Cauchy net} if
  \begin{equation*}
    \forall V \in V \ \exists k_0 \in \mscrK \ \forall k, m \geq k_0 : (x_k, x_m) \in V.
  \end{equation*}
\end{definition}

\begin{lemma}\label{thm:convergent_net_is_fundamental}
  A net in a uniform space that has a limit \hyperref[def:net_limit_point]{point} is \hyperref[def:fundamental_net]{fundamental}.
\end{lemma}
\begin{proof}
  If \( x_0 \) is a limit point of the net \( \seq{ x_k }_{k \in \mscrK} \), the net is eventually in every ball \( B(x_0, V) \), which implies \fullref{def:fundamental_net}.
\end{proof}

\begin{definition}\label{def:complete_uniform_space}\mcite[446]{Engelking1989}
  A uniform space is called \term{complete} if it is \hyperref[def:separation_axioms/T2]{Hausdorff} and if every \hyperref[def:fundamental_net]{fundamental net} \hyperref[def:net_limit_point]{converges}.

  The \term{completion} of uniform space \( (X, \mscrV) \) is a (\hyperref[def:uniform_continuity]{uniformly continuous}) \hyperref[def:morphism_invertibility/left_cancellative]{embedding} \( f: X \to Y \) into a complete uniform space \( (Y, \mscrU) \) such that \( \img(X) \) is \hyperref[def:topologically_dense_set]{dense} in \( Y \).
\end{definition}

\begin{theorem}[Uniform space completion]\label{thm:uniform_space_completion}\mcite[thm. 8.3.12]{Engelking1989}
  Every uniform space has a unique (up to an isomorphism) \hyperref[def:complete_uniform_space]{completion}.

  See also \fullref{thm:metric_space_completion}.
\end{theorem}

\begin{theorem}[Cauchy's net convergence criterion]\label{thm:cauchys_net_convergence_criterion}
  A net in a complete \hyperref[def:complete_uniform_space]{uniform space} is convergent if and only if it \hyperref[def:fundamental_net]{fundamental}.

  Explicitly, a net \( \seq{ x_k }_{k \in \mscrK} \) in a complete uniform space \( (X, \mscrV) \) is \hyperref[def:net_limit_point]{convergent} if and only if for every entourage \( V \in V \) there exists an index \( k_0 \) such that
  \begin{equation*}
    (x_k, x_m) \in V \quad\forall k, m \geq k_0.
  \end{equation*}
\end{theorem}
\begin{proof}
  \SufficiencySubProof Given by \fullref{thm:convergent_net_is_fundamental}
  \NecessitySubProof Given by \fullref{def:complete_uniform_space}
\end{proof}


  \chapter{Metric spaces}\label{ch:metric_spaces}

Metric spaces are \hyperref[def:uniform_space]{uniform} \hyperref[def:topological_space]{topological spaces} with a lot of desirable properties. Most of real analysis that does not rely on certain algebraic structures generalizes well to metric spaces.

  \subsection{Metric topology}\label{subsec:metric_topology}

\begin{definition}\label{def:metric_space}\mcite[248]{Engelking1989}
  A \term{metric space} is a set \( X \) along with a nonnegative real-valued function \( \rho: X \times X \to [0, \infty) \), called a \term{metric}, also called the \term{distance function}, such that
  \begin{thmenum}[series=def:metric_space]
    \thmitem[def:metric_space/M1]{M1} \( \rho(x, y) = 0 \iff x = y \)
    \thmitem[def:metric_space/M2]{M2}(symmetry) \( \rho(x, y) = \rho(y, x) \)
    \thmitem[def:metric_space/M3]{M3}(triangle inequality) \( \rho(x, y) \leq \rho(x, z) + \rho(z, y) \)
  \end{thmenum}

  If instead of \ref{def:metric_space/M1} we have the weaker condition
  \begin{thmenum}[resume=def:metric_space]
    \thmitem[def:metric_space/pseudometric_identity]{M1'} \( \forall x \in X, \rho(x, x) = 0 \),
  \end{thmenum}
  we call \( \rho \) a \term{pseudometric} and \( (X, \rho) \) a \term{pseudometric space}.

  \begin{thmenum}
    \thmitem{def:metric_space/subspace} If \( A \subseteq X \) is a set, then the restriction \( (A, \rho{\rvert_A}) \) is a metric space and it is called a \term{subspace} of \( X \).

    \thmitem{def:metric_space/ball} Define the function
    \begin{balign*}
       & B: X \times (0, \infty) \to \pow(X),                   \\
       & B(x, r) \coloneqq \{ y \in X \colon \rho(x, y) < r \}.
    \end{balign*}

    The set \( B(x, r) \) is called an \term{open ball} or simply \term{ball} with \term{center} \( x \) and \term{radius} \( r \).

    The ball \( B = B(0, 1) \) is called the \term{unit ball}.

    \thmitem{def:metric_space/closed_ball} The set
    \begin{equation*}
      \overline{B(x, r)} \coloneqq \cl(B(x, r))
    \end{equation*}
    is called the \term{closed ball} with center \( x \) and radius \( r \).

    \thmitem{def:metric_space/sphere} The set
    \begin{equation*}
      S(x, r) \coloneqq \fr{B(x, r)}
    \end{equation*}
    is called the \term{sphere} with center \( x \) and radius \( r \).

    \thmitem{def:metric_space/bounded_set} A set \( A \subseteq X \) is called \term{bounded} if it is contained in some ball \( B(x, r) \).

    \thmitem{def:metric_space/bounded_sequence} A \hyperref[def:sequence]{sequence} \( \seq{ x_k }_{k=1}^\infty \subseteq X \) is called \term{bounded} if the corresponding set \( \{ x_k \colon k = 1, 2, \ldots \} \) is \hyperref[def:metric_space/bounded_set]{bounded}.

    \thmitem{def:metric_space/bounded_metric} If every set is bounded, we say that the metric itself is bounded.

    \thmitem{def:metric_space/bounded_function} We say that a function \( f: S \to X \) from a set \( S \) to a metric space \( (X, \rho) \) is \term{bounded} if its image \( f(S) \) is a bounded set in \( (X, \rho) \).

    \thmitem{def:metric_space/diameter} Define the function
    \begin{balign*}
       & \diam: \pow(X) \to [0, \infty],                             \\
       & \diam(A) \coloneqq \sup \{ \rho(x, y) \colon x, y \in A \},
    \end{balign*}
    whose values include the nonnegative extended real \hyperref[def:extended_real_numbers]{numbers}.

    If it exists, we call the number \( \diam(A) \) the \term{diameter of \( A \)}.

    \thmitem{def:metric_space/distance} Define the function
    \begin{balign*}
       & \op{dist}: X \times \pow(X) \to [0, \infty),                    \\
       & \op{dist}(x, A) \coloneqq \inf \{ \rho(x, a) \colon a \in A \}.
    \end{balign*}

    We call the number \( \op{dist}(x, A) \) the \term{distance from the point \( x \) to the set \( A \)}. We use the convention that the infimum of an empty set of real numbers is \( +\infty \), hence \( \op{dist}(x, \varnothing) = \infty \).
  \end{thmenum}
\end{definition}

\begin{proposition}\label{thm:pseudometric_to_metric}
  Let \( (X, \rho) \) be a \hyperref[def:metric_space]{pseudometric space}. Define the equivalence relation
  \begin{equation*}
    x \cong y \iff \rho(x, y) = 0.
  \end{equation*}

  Then the following metric on the \hyperref[thm:equivalence_partition]{quotient set} \( M \coloneqq X / \cong \)
  \begin{balign*}
     & \rho: M \times M \to [0, \infty)    \\
     & \rho([x], [y]) \coloneqq \rho(x, y)
  \end{balign*}
  is well-defined.
\end{proposition}
\begin{proof}
  The function \( \rho \) is well-defined since, if \( x \) and \( y \) both belong to the same equivalence class \( [x] \), then \( \rho(x) = \rho(y) \). Thus, \( \rho \) does not depend on the choice of representatives.

  Additionally, \( \rho \) is a metric since \( \rho([x], [y]) = 0 \) implies that \( [x] = [y] \), that is, \( \rho(x, y) = 0 \).
\end{proof}

\begin{proposition}\label{rem:bounded_set_metric_order_equivalence}
  A set \( A \) in a metric space \( (X, \rho) \) is \hyperref[def:metric_space/bounded_set]{bounded} if and only if the set \( \{ \rho(a, b) \colon a, b \in A \} \) is bounded as a \hyperref[def:extremal_points/upper_and_lower_bounds]{partially ordered set}.
\end{proposition}

\begin{definition}\label{def:metric_topology}\mcite[249]{Engelking1989}
  Let \( (X, \rho) \) be a metric space. We define the \term{metric topology}\( \mscrT \) , also called the \term{induced topology}, as the \hyperref[def:topological_space]{topology} generated by the \hyperref[def:topological_local_base]{neighborhood system}
  \begin{equation}\label{def:metric_topology/integer_base}
    \mathcal{B}(x) \coloneqq \{ B(x, \tfrac 1 n) \colon n = 1, 2, 3, \ldots \}.
  \end{equation}

  If for some topological space \( (X, \mscrT) \) there exists a metric such that \( \mscrT \) is its induced topology, we say that the topology \( \mscrT \) is \term{metrizable}.

  It is often conventional to consider the alternative (larger) base
  \begin{equation}\label{def:metric_topology/real_base}
    \mathcal{B}'(x) \coloneqq \{ B(x, \varepsilon) \colon \varepsilon > 0 \}.
  \end{equation}
\end{definition}
\begin{proof}
  This is indeed a neighborhood system as it satisfies \ref{thm:topology_from_local_base/BP1}-\ref{thm:topology_from_local_base/BP3}:

  \begin{refenum}
    \refitem{thm:topology_from_local_base/BP1} Every point \( x \) belongs to any ball centered at \( x \).

    \refitem{thm:topology_from_local_base/BP3} Fix \( x \in X \) and two balls \( B(x, \tfrac 1 n) \) and \( B(x, \tfrac 1 m) \). Then
    \begin{equation*}
      B(x, \tfrac 1 {\max\{ n, m \}}) \subseteq B(x, n) \cap B(x, m).
    \end{equation*}

    \refitem{thm:topology_from_local_base/BP2} Fix \( x, y \in X \) and let \( x \in B(y, \tfrac 1 n) \), i.e. \( \rho(x, y) < \tfrac 1 n \).

    \begin{figure}[!ht]
      \centering
      \includegraphics[page=1]{output/def__metric_topology__nested_balls.pdf}
      \caption{There is a nested ball around every point in an open ball}\label{def:metric_topology/nested_balls}
    \end{figure}

    Define \( m \) to be the smallest positive integer such that
    \begin{equation*}
      \tfrac 1 m \leq \min\{ \rho(x, \tfrac 1 n), \tfrac 1 n - \rho(x, \tfrac 1 n) \}.
    \end{equation*}

    Note that \( m \) exists since the positive integers are \hyperref[def:well_founded_relation]{well-founded}.

    Let \( z \in B(x, \tfrac 1 m) \). There are two cases:
    \begin{itemize}
      \item If \( \rho(x, y) \leq \tfrac 1 {2n} \), then
            \begin{balign*}
              \rho(z, y)
              \leq
              \rho(z, x) + \rho(x, y)
              <
              \tfrac 1 m + \rho(x, y)
              \leq
              \tfrac 1 n + \tfrac 1 n
              \leq
              2 \tfrac 1 {2n}
              =
              \tfrac 1 n.
            \end{balign*}

      \item If \( \rho(x, y) > \tfrac 1 {2n} \), then
            \begin{balign*}
              \rho(z, y)
              \leq
              \rho(z, x) + \rho(x, y)
              <
              \tfrac 1 m + \rho(x, y)
              \leq
              (\tfrac 1 n - \rho(x, y)) + \rho(x, y)
              =
              \tfrac 1 n.
            \end{balign*}
    \end{itemize}

    In both cases, \( B(x, \tfrac 1 m) \subseteq B(y, \tfrac 1 n) \).
  \end{refenum}
\end{proof}

\begin{proposition}\label{thm:def:metric_topology}
  The metric topology \( \mscrT \) on \( X \) induced by \( \rho \) has the following properties:
  \begin{thmenum}
    \thmitem{thm:def:metric_topology/ball_is_open} For every point \( x \in X \) and any radius \( r > 0 \), the ball \( B(x, r) \) is an open set and, hence, a neighborhood of \( x \).
    \thmitem{thm:def:metric_topology/first_countable} \( \mscrT \) is first-countable.
    \thmitem{thm:def:metric_topology/sequential} \( \mscrT \) is sequential.
    \thmitem{thm:def:metric_topology/hausdorff} \( \mscrT \) is Hausdorff.
  \end{thmenum}
\end{proposition}
\begin{proof}
  \SubProofOf{thm:def:metric_topology/ball_is_open} Obvious from \fullref{def:metric_topology}.

  \SubProofOf{thm:def:metric_topology/first_countable} Since \fullref{def:metric_topology} involves generating a topology using a neighborhood system of countable local neighborhoods, \( \mscrT \) is first-countable.

  \SubProofOf{thm:def:metric_topology/sequential} Follows from \fullref{thm:def:metric_topology/first_countable} and \fullref{thm:first_countable_spaces_are_sequential}.

  \SubProofOf{thm:def:metric_topology/hausdorff} Let \( x, y \in X \) be distinct points. Define
  \begin{equation*}
    r \coloneqq \dfrac 1 2 \rho(x, y),
  \end{equation*}
  so that
  \begin{equation*}
    B(x, r) \cap B(y, r) = \varnothing.
  \end{equation*}
\end{proof}

\begin{definition}\label{def:metric_uniformity}
  Let \( (X, \rho) \) be a metric space.

  We define the \term{metric uniformity} \( \mscrV \), also called the \term{induced uniformity}, as the \hyperref[def:uniform_space]{uniformity} generated by the countable \hyperref[thm:uniform_space_base_axioms]{base}
  \begin{equation}\label{def:metric_uniformity/integer_base}
    \mathcal{B} \coloneqq \{ V_n \colon n = 1, 2, \ldots \},
  \end{equation}
  where
  \begin{equation*}
    V_n \coloneqq \rho^{-1}([0, \tfrac 1 n)).
  \end{equation*}

  As for the \hyperref[def:metric_topology]{metric topology}, we can instead consider the base
  \begin{equation}\label{def:metric_uniformity/real_base}
    \mathcal{B}' \coloneqq \{ \rho^{-1}([0, \varepsilon)) \colon \varepsilon > 0 \}.
  \end{equation}
\end{definition}
\begin{proof}
  Each relation \( V_r \) is obviously an entourage by \ref{def:metric_space/M1} and \ref{def:metric_space/M2}. We will prove that \( \mathcal{B} \) is indeed a uniform space base.

  \begin{refenum}
    \refitem{thm:uniform_space_base_axioms/BU1} For nonnegative integers \( n, m \) we have
    \begin{equation*}
      V_n \cap V_m
      =
      \{ (x, y) \in X \times X \colon \rho(x, y) < \tfrac 1 n \T{and} \rho(x, y) < \tfrac 1 m \}
      =
      V_{\max\{ n, m \}}.
    \end{equation*}

    Pick any integer \( k \geq \max\{ n, m \} \), so that
    \begin{equation*}
      V_k \subseteq V_n \cap V_m.
    \end{equation*}

    \refitem{thm:uniform_space_base_axioms/BU2} Fix \( V_n \in V \) and \( m \coloneqq 2n \). By the triangle inequality, we have that if \( \rho(x, y) < m \) and \( \rho(y, z) < m \), then
    \begin{equation*}
      \rho(x, z) \leq \rho(x, y) + \rho(y, z) < \tfrac 1 m + \tfrac 1 m = \tfrac 1 n.
    \end{equation*}

    Thus,
    \begin{equation*}
      V_m + V_m
      =
      \left\{ (x, z) \colon \exists y \in X: \rho(x, y) < \tfrac 1 m \T{and} \rho(y, z) < \tfrac 1 m \right\}
      \subseteq
      V_n
    \end{equation*}

    \refitem{thm:uniform_space_base_axioms/BU3} \ref{def:metric_space/M1} implies that
    \begin{equation*}
      \bigcap \mscrB = \lim_{n \to \infty} \rho^{-1}([0, \tfrac 1 n)) = \Delta_X.
    \end{equation*}
  \end{refenum}
\end{proof}

\begin{proposition}\label{thm:metric_topology_coincides_with_uniform_topology}
  The \hyperref[def:metric_topology]{metric topology} and the \hyperref[def:uniform_topology]{uniform topology} from the \hyperref[def:metric_uniformity]{metric uniformity} coincide.
\end{proposition}

\begin{theorem}\label{thm:countable_uniform_base_implies_metrizable}\mcite[thm. 8.1.21]{Engelking1989}
  A uniform space \( X \) is metrizable if and only \( w(X) \leq \aleph_0 \).
\end{theorem}

\medskip

\begin{definition}\label{def:isometry}\mcite[253]{Engelking1989}
  Let \( (X, \rho) \) and \( (Y, \nu) \) be two \hyperref[def:metric_space]{metric spaces}. We say that the function \( f: X \to Y \) is a \term{distance preserving map} or \term{isometry} or \term{isometric embedding} if
  \begin{equation*}
    \forall x, y \in X, \rho(x, y) = \nu(f(x), f(y)).
  \end{equation*}

  If \( f \) is bijective, we say that \( X \) and \( Y \) are \term{isometric}.
\end{definition}

\begin{proposition}\label{thm:isometry_is_injective}
  An \hyperref[def:isometry]{isometry} \( f: (X, \rho) \to (Y, \nu) \) is always injective.
\end{proposition}
\begin{proof}
  If \( f(x) = f(x') \), then by \fullref{def:metric_space/M1}, \( x = x' \).
\end{proof}

\begin{definition}\label{def:category_of_metric_spaces}
  Metric spaces and monotone maps form a subcategory of \( \cat{Unif} \) (see \fullref{def:category_of_uniform_spaces}). We denote this category by \( \cat{Met} \).
\end{definition}

\begin{definition}\label{def:equivalent_metrics}
  Two metrics \( \rho \) and \( \nu \) on the set \( X \) are said to be \term{equivalent} if \( \rho \) and \( \nu \) have the same \hyperref[def:metric_topology]{metric topology}. They are said to be \term{strongly equivalent} if there exist constants \( \alpha, \beta \in \BbbR \) such that for every \( x, y \in X \) we have
  \begin{equation*}
    \alpha \nu(x, y) \leq \rho(x, y) \leq \beta \nu(x, y).
  \end{equation*}
\end{definition}

\begin{remark}\label{rem:metric_space_convergence}
  All types of convergence from \fullref{subsec:net_convergence}, \fullref{subsec:topological_continuity} and \fullref{subsec:uniform_spaces} hold in metric spaces using the \hyperref[def:metric_topology]{metric topology} and \hyperref[def:metric_uniformity]{metric uniformity} structure.

  It is conventional to prefer the bases \fullref{def:metric_topology/real_base} and \fullref{def:metric_uniformity/real_base} to the bases \fullref{def:metric_topology/integer_base} and \fullref{def:metric_topology/integer_base}.

  For example, given two metric spaces \( X \) and \( Y \), continuity of \( f: X \to Y \) at \( x_0 \in X \) (see \fullref{def:local_continuity}) is usually written using the \enquote{epsilon-delta notation} as
  \begin{equation*}
    \forall \varepsilon > 0 \ \exists \delta > 0 : \rho_X(x, x_0) < \delta \implies \rho_Y(f(x), f(x_0)) < \varepsilon
  \end{equation*}
  for any \( x \in X \).
\end{remark}

\begin{definition}\label{def:translation_invariant_metric}
  A \hyperref[def:metric_space]{metric} \( \rho \) on a \hyperref[def:magma]{magma} \( G \) is said to be \term{left translation-invariant} if
  \begin{equation*}
    \rho(ax, ay) = \rho(x, y) \quad\forall a, x, y \in G
  \end{equation*}
  and \term{right translation-invariant} if
  \begin{equation*}
    \rho(xa, ya) = \rho(x, y) \quad\forall a, x, y \in G.
  \end{equation*}

  If \( \rho \) is both left and right translation invariant (e.g. for commutative magmas), we simply say that \( \rho \) is \term{translation invariant}.
\end{definition}

  \subsection{Complete metric spaces}\label{subsec:metric_convergence}

\begin{definition}\label{def:complete_metric_space}
  A metric space is said to be \term{complete} if
  \begin{thmenum}
    \thmitem{def:complete_metric_space/sequences} Every fundamental sequence converges.
    \thmitem{def:complete_metric_space/uniform} It is complete as a uniform space in the sense of \fullref{def:complete_uniform_space}
  \end{thmenum}
\end{definition}
\begin{proof}
  The equivalence is due to \fullref{thm:def:metric_topology/sequential} and \fullref{thm:def:metric_topology/hausdorff}.
\end{proof}

\begin{proposition}\label{thm:fundamental_sequence_is_bounded}
  In a metric space, any \hyperref[def:fundamental_net]{fundamental sequence} \( \{ x_k \}_{i=1}^n \) is \hyperref[def:metric_space/bounded_sequence]{bounded}.
\end{proposition}
\begin{proof}
  Since the set
  \begin{equation*}
    I \coloneqq \{ x_k \colon k \leq k_0 \}
  \end{equation*}
  is finite, it has a finite \hyperref[def:metric_space/diameter]{diameter}.

  Fix \( \varepsilon > 0 \). Since the sequence is fundamental, there exists an index \( k_0 \) such that
  \begin{equation*}
    \rho(x_k, x_m) < \varepsilon \quad\forall k, m \geq k_0.
  \end{equation*}

  We are only interested in the case \( \rho(x_{k_0}, x_m) < \varepsilon \).

  Let \( k < k_0 \) and \( m \geq k_0 \). Then
  \begin{equation*}
    \rho(x_k, x_m) \leq \rho(x_k, x_{k_0}) + \rho(x_{k_0}, x_m) < \diam(I) + \varepsilon,
  \end{equation*}
  which is a finite number.

  Thus, the distance between any two elements of the sequence is finite and the sequence is bounded.
\end{proof}

\begin{proposition}\label{thm:fundamental_subsequence_convergence}
  In any \hyperref[def:complete_metric_space]{metric space}, a \hyperref[def:fundamental_net]{fundamental sequence} converges to a value if and only if it has a subsequence that converges to the same value.
\end{proposition}
\begin{proof}
  Let \( (X, \rho) \) be a metric space and let \( \{ x_k \}_{k=1}^\infty \) be a fundamental sequence.

  \SufficiencySubProof Obvious
  \NecessitySubProof Assume that the subsequence \( \{ x_{k_n} \}_{n=1}^\infty \) converges to \( x \). Fix \( \varepsilon > 0 \). There exist \( k_0 \) and \( n_0 \) such that
  \begin{balign*}
     & \rho(x_k, x_m) < \tfrac \varepsilon 2 \quad\forall k, m \geq k_0
     & \rho(x, x_{k_n}) < \tfrac \varepsilon 2 \quad\forall n \geq n_0.
  \end{balign*}

  Fix \( k \geq k_0 \) and let \( n \geq n_0 \) be such that \( k_n \geq k_0 \). Then
  \begin{equation*}
    \rho(x, x_k) \leq \rho(x, x_{k_n}) + \rho(x_{k_n}, x_k) < \varepsilon.
  \end{equation*}

  Since \( \varepsilon \) was arbitrary, we conclude that \( \lim_{k \to \infty} x_k = \lim_{n \to \infty} x_{k_n} = x \).
\end{proof}

\begin{lemma}\label{thm:metric_space_completion_uniqueness}
  Let \( X \) be a metric space. If both \( f: X \to Y \) and \( g: X \to Z \) are \hyperref[def:complete_metric_space]{completions} of \( X \), then \( Y \) and \( Z \) are isometric.
\end{lemma}
\begin{proof}
  Let \( y \in Y \) and let \( \{ x_k \}_{k \to \infty} \subseteq X \) be a sequence such that
  \begin{equation*}
    f(x_k) \xrightarrow[k \to \infty]{} y.
  \end{equation*}

  Such a sequence exists since \( f(X) \) is dense in \( Y \).

  Define \( z \coloneqq \lim_{k \to \infty} g(x_k) \). Since both \( f \) and \( g \) are isometries, \( z \) does not depend on the choice of sequence \( \{ x_k \}_{k \to \infty} \) such that \( f(x_k) \to y \). Furthermore, if \( z \in Z \) is given rather than \( y \in Y \), an analogous process allows us to determine \( y \) uniquely based on \( z \).

  Thus, we have a bijective isometry between \( Y \) and \( Z \).
\end{proof}

\begin{theorem}[Metric space completion]\label{thm:metric_space_completion}
  Every metric space has a unique (up to an isometry) \hyperref[def:complete_metric_space]{completion}.

  This is a special case of \fullref{thm:uniform_space_completion} that we prove fully.
\end{theorem}
\begin{proof}
  Let \( (X, \rho) \) be a metric space. Uniqueness of the completion follows from \fullref{thm:metric_space_completion_uniqueness}. We will only show existence.

  \begin{thmenum}
    \thmitem{thm:metric_space_completion/part_a} First, we build the pseudometric space \( (F, \rho) \). We deal with fundamental sequences and isometries in pseudometric spaces, where the definitions, however, does not change.

    Define \( F \) to be the set of all fundamental \hyperref[def:fundamental_net]{sequences} in X. Define the pseudometric
    \begin{balign*}
       & \rho: F \times F \to \BbbR_{\geq 0}                                                                               \\
       & \rho\left( \{ x_k \}_{k=1}^\infty, \{ y_k \}_{k=1}^\infty \right) \coloneqq \lim_{k \to \infty} \rho(x_k, y_k).
    \end{balign*}

    We first show that is well-defined as a \hyperref[def:function]{function}. Let \( \{ x_k \}_{k=1}^\infty \) and \( \{ y_k \}_{k=1}^\infty \) be two sequences. Fix \( \varepsilon > 0 \). Then there exists an \( k_0 \) such that
    \begin{equation*}
      \rho(x_k, x_k) < \tfrac \varepsilon 2 \text{ and } \rho(y_k, y_m) <  \quad\forall k, m \geq k_0.tfrac \varepsilon 2.
    \end{equation*}

    Fix \( k, m \geq k_0 \). Then
    \begin{equation*}
      \rho(x_k, y_k) \leq \rho(x_k, x_m) + \rho(x_m, y_m) + \rho(y_m, y_k) < \rho(x_m, y_m) + \varepsilon,
    \end{equation*}
    hence
    \begin{equation*}
      \abs{\rho(x_k, y_k) - \rho(x_m, y_m)} < \varepsilon.
    \end{equation*}

    Thus, the sequence \( \{ \rho(x_k, y_k) \}_{k=1}^\infty \) is fundamental and, by \fullref{def:real_numbers_complete_metric_space}, it is convergent.

    Now we check that \( \rho \) is indeed a pseudometric:
    \SubProofOf{def:metric_space/pseudometric_identity} For every sequence \( x \in F \),
    \begin{equation*}
      \rho(x, x) = \lim_{k \to \infty} \rho(x_k, x_k) = 0.
    \end{equation*}
    \SubProofOf{def:metric_space/M2} For all sequences \( x, y \in F \),
    \begin{equation*}
      \rho(x, y) = \lim_{k \to \infty} \rho(x_k, y_k) = \lim_{k \to \infty} \rho(y_k, x_k) = \rho(y, x).
    \end{equation*}

    \SubProofOf{def:metric_space/M3} For all sequences \( x, y, z \in F \),
    \begin{equation*}
      \rho(x, z) = \lim_{k \to \infty} \rho(x_k, z_i) \leq \lim_{k \to \infty} \rho(x_k, y_k) + \lim_{k \to \infty} \rho(y_k, z_i) = \rho(x, y) + \rho(y, z).
    \end{equation*}

    \thmitem{thm:metric_space_completion/part_b} We prove that every fundamental sequence in \( (F, \rho) \) is convergent.

    Let \( \{ c^{(k)} \}_{k=1}^\infty \) be a fundamental sequence (of sequences) in \( (F, \rho) \). Thus, for every \( k = 1, 2, \ldots \), there exists an index \( n_k \) such that
    \begin{equation*}
      \rho(c_m^{(k)}, c_{n_k}^{(k)}) < \tfrac 1 k \quad\forall m \geq n_k.
    \end{equation*}

    Define the sequence
    \begin{equation*}
      d_k \coloneqq c_{n_k}^{(k)}, k = 1, 2, \ldots
    \end{equation*}

    To see that it is fundamental, fix \( \varepsilon > 0 \). Now since the sequence \( \{ c^{(k)} \} \) in \( F \) is fundamental, there exists \( k_0 \) such that
    \begin{equation*}
      \rho(c^{(k)}, c^{(m)}) = \lim_{k \to \infty} \rho(c_k^{(k)}, c_k^{(m)}) < \frac \varepsilon 2 \quad\forall k, m \geq k_0.
    \end{equation*}

    Let \( m_0 \geq k_0 \) be an index such that
    \begin{equation*}
      \frac 2 {m_0} < \frac \varepsilon 2.
    \end{equation*}

    Fix \( k \geq m \geq m_0 \). Let \( l \geq \max \{ n_k, n_m \} \) be such that
    \begin{equation*}
      \rho(c_l^{(k)}, c_l^{(m)}) < \frac \varepsilon 2.
    \end{equation*}

    Then
    \begin{balign*}
      \rho(d_k, d_m)
       & =
      \rho(c_{n_k}^{(k)}, c_{n_m}^{(m)})
      \leq \\ &\leq
      \rho(c_{n_k}^{(k)}, c_l^{(k)}) + \rho(c_l^{(k)}, c_l^{(m)}) + \rho(c_l^{(m)}, c_{n_m}^{(m)})
      \leq \\ &\leq
      \frac 1 k + \frac \varepsilon 2 + \frac 1 m
      \leq
      \frac 2 m + \frac \varepsilon 2
      <
      \varepsilon.
    \end{balign*}

    Thus, we have
    \begin{equation*}
      \rho(d_k, d_m) < \varepsilon \quad\forall k \geq m \geq m_0,
    \end{equation*}
    which proves that the sequence \( \{ d_k \}_{k=1}^\infty \) is fundamental in \( (X, \rho) \).

    Now it remains to show that \( c^{(k)} \xrightarrow[k \to \infty]{} d \) in \( (F, \rho) \).

    Fix \( \varepsilon > 0 \) and let \( k_0 \) be such that
    \begin{equation*}
      \frac 1 {k_0} \leq \frac \varepsilon 2.
    \end{equation*}
    and
    \begin{equation*}
      \rho(d_k, d_m) < \frac \varepsilon 2 \quad\forall k, m \geq k_0.
    \end{equation*}

    Now fix \( i \geq k_0 \). We have, for all \( k \geq i \),
    \begin{balign*}
      \rho(c_{n_k}^{(k)}, d_k)
       & =
      \rho(c_{n_k}^{(k)}, c_{n_k}^{(k)})
      \leq \\ &\leq
      \rho(c_{n_k}^{(k)}, c_{n_k}^{(k)}) + \rho(c_{n_k}^{(k)}, c_{n_k}^{(k)})
      =    \\ &=
      \rho(c_{n_k}^{(k)}, c_{n_k}^{(k)}) + \rho(d_k, d_k)
      <    \\ &<
      \frac 1 k + \frac \varepsilon 2
      <
      \varepsilon.
    \end{balign*}

    Hence,
    \begin{balign*}
      \rho(c^{(k)}, d)
      =
      \lim_{k \to \infty} \rho(c_k^{(k)}, d_k)
      =
      \lim_{k \to \infty} \rho(c_k^{(k)}, c_{n_k}^{(k)})
      <
      \varepsilon.
    \end{balign*}

    Thus, given \( \varepsilon > 0 \), we found an index \( k_0 \) such that
    \begin{equation*}
      \rho(c^{(k)}, d) < \varepsilon \quad\forall g \geq k_0.
    \end{equation*}

    Thus, \( d = \lim_{k \to \infty} c^{(k)} \) and \( (F, \rho) \) is a complete pseudometric space.

    \thmitem{thm:metric_space_completion/part_c} We construct an isometry of \( (X, \rho) \) into \( (F, \rho) \).

    Define the function
    \begin{balign*}
       & \iota: X \to F                        \\
       & \iota(x) \coloneqq (x, x, x, \ldots),
    \end{balign*}
    which sends each element of \( X \) into the corresponding constant sequence in \( F \).

    It is an \hyperref[def:isometry]{isometry} since
    \begin{equation*}
      \rho(\iota(x),\iota(y)) = \lim_{k \to \infty} \rho(x, y) = \rho(x, y).
    \end{equation*}

    \thmitem{thm:metric_space_completion/part_d} We show that the image \( \iota(X) \) is dense in \( (F, \rho) \).

    Fix the fundamental sequence \( y \coloneqq \{ y_k \}_{k=1}^\infty \). Define the sequence \( x \) of sequences
    \begin{equation*}
      x^{(k)} \coloneqq \iota(y_k), i = 1, 2, \ldots
    \end{equation*}

    It is fundamental in \( (F, \rho) \) since \( e \) is an isometry and since \( y \) is fundamental in \( (X, \rho) \).

    Fix \( \varepsilon > 0 \). Let \( k_0 \) be such that
    \begin{equation*}
      \rho(y_k, y_m) < \varepsilon \quad\forall k, m \geq k_0.
    \end{equation*}

    For \( i, k \geq k_0 \), we have
    \begin{balign*}
      \rho(x_k^{(k)}, y_k)
      \leq
      \rho(x_k^{(k)}, y_k) + \rho(y_k, y_k)
      =
      0 + \rho(y_k, y_k)
      <
      \varepsilon,
    \end{balign*}
    hence
    \begin{equation*}
      \rho(x^{(k)}, y) = \lim_{k \to \infty} \rho(x_k^{(k)}, y_k) < \varepsilon.
    \end{equation*}

    We conclude that \( x^{(k)} \xrightarrow[k \to \infty]{} y \) in \( (F, \rho) \), which implies that \( e(X) \) is dense in \( (F, \rho) \).

    \thmitem{thm:metric_space_completion/part_e} We build a complete metric space \( (C, \nu) \) from \( (F, \rho) \).

    We use \fullref{thm:pseudometric_to_metric} to construct a complete metric space \( (C, \nu) \) from the complete pseudometric space \( (F, \rho) \).

    We adapt \( \iota \) to the equivalence classes on \( C \):
    \begin{balign*}
       & \hat\iota: X \to C                 \\
       & \hat\iota(x) \coloneqq [\iota(x)].
    \end{balign*}

    Thus, \( \hat\iota \) embeds \( X \) into the complete metric space \( C \).
  \end{thmenum}
\end{proof}

\begin{proposition}\label{thm:metric_space_is_dense_in_completion}
  Every \hyperref[def:metric_space]{metric space} is dense in its \hyperref[thm:metric_space_completion]{completion}.
\end{proposition}

\begin{theorem}[Cantor's nested compact theorem]\label{thm:cantors_nested_compact_theorem}
  A descending sequence of nonempty compact sets \( F_1 \supseteq F_2 \supseteq \ldots \) in a complete metric space such that \( \diam(F_i) \to 0 \) intersects at exactly one point (compare with \fullref{thm:noncompact_kuratowskis_lemma}).
\end{theorem}
\begin{proof}
  Choose an element \( x_k \in F_k \) for any \( i = 1, 2, \ldots \). Then the sequence \( \{ x_k \}_{k=1}^\infty \) is fundamental. To see this, let \( \varepsilon > 0 \) and let \( k_0 \) be an index such that \( \diam(F_{k_0}) < \varepsilon \). Then if \( j \geq i \geq k_0 \), \( x_m \) is contained in \( F_k \) and \( \rho(x_k, x_m) < \varepsilon \). Thus, the sequence is indeed fundamental and, since the space is complete, it has a limit point \( x \).

  The point \( x \) is contained in every set \( F_k, i = 1, 2, \ldots \) since all of the sets \( F_k \) are closed (by \fullref{thm:complete_metric_space_compact_conditions}) and contain their limit \hyperref[thm:limit_point_iff_in_closure]{points}. Thus,
  \begin{equation*}
    x \in \bigcap_{k=1}^\infty F_k.
  \end{equation*}

  Furthermore,
  \begin{equation*}
    \diam\left( \bigcap_{k=1}^\infty F_k \right) = 0,
  \end{equation*}
  hence \( x \) is the only point in the intersection.
\end{proof}

  \subsection{Hausdorff distance}\label{subsec:hausdorff_distance}

Let \( (X, \mu) \) be a \hyperref[def:complete_metric_space]{complete metric space}.

\begin{definition}\label{def:hausdorff_distance}\mcite[144]{DontchevRockafellar2014SolutionMappings}
  Fix two sets \( E \subseteq X \) and \( F \subseteq X \).

  The \term{excess} of \( E \) beyond \( F \) is defined as
  \begin{balign*}
     & e: \pow X \times \pow X \to \BbbR \cup \{ \infty \} \\
     & e(E, F) \coloneqq \begin{cases}
      +\infty,                                                                                    & E = \varnothing, D = \varnothing                      \\
      0,                                                                                          & E = \varnothing, D \neq \varnothing                   \\
      \sup_{x \in E} \op{dist}(x, F) \reloset{*}{=} \inf \{\delta > 0 \colon E \subseteq F_\delta \}, & E \neq \varnothing \nonumber\refstepcounter{equation}
    \end{cases}
  \end{balign*}
  where \( F_\delta \coloneqq \{ y \in X \colon \op{dist}(y, F) \leq \delta \} \).

  The \term{Pompeiu-Hausdorff distance} or simply \term{Hausdorff} distance between them is then defined as
  \begin{equation*}
    h(E, F) \coloneqq \max\{ e(E, F), e(F, E) \} = \inf \{\delta > 0 \colon E \subseteq F_\delta, F \subseteq E_\delta \}.
  \end{equation*}
\end{definition}
\begin{proof}(of the equality \( * \))
  Note that the set
  \begin{equation*}
    F_{e(E, F)} = \{ x \in X \colon \op{dist}(x, F) \leq \sup_{x \in E} \op{dist}(x, F) \}
  \end{equation*}
  obviously includes \( E \).

  Now let \( \delta > 0 \) be any real number that satisfies \( E \subseteq F_\delta \), i.e.
  \begin{equation*}
    E \subseteq F_\delta = \{ x \in X \colon \op{dist}(x, F) \leq \delta \},
  \end{equation*}
  which implies that
  \begin{equation*}
    e(E, F) = \sup_{x \in E} \op{dist}(x, F) \leq \delta.
  \end{equation*}
\end{proof}

\begin{proposition}\label{thm:hausdorff_distance_is_metric}
  The Hausdorff distance is a metric on the nonempty compact subsets of \( X \).
\end{proposition}
\begin{proof}
  Let \( E \), \( F \) and \( G \) be nonempty compact subsets of \( X \).

  The function \( h \) is nonnegative. Since we exclude empty and unbounded sets, We do not care about infinite values.

  \SubProofOf{def:metric_space/M1} Obviously \( h(E, E) = 0 \). If \( h(E, F) = 0 \), then there exists no point of \( E \) outside \( F \) and vice versa, hence \( E = F \).

  \SubProofOf{def:metric_space/M2} This follows from the symmetry of the \( \max \) function.

  \SubProofOf{def:metric_space/M3} For any point \( y \in X \), we have
  \begin{balign*}
    \op{dist}(x, G)
    =
    \inf_{z \in G} \mu(x, z)
    \leq
    \mu(x, y) + \inf_{y \in G} \mu(y, z)
    =
    \mu(x, y) + \op{dist}(y, G).
  \end{balign*}

  Select \( y \in F \) that minimizes the distance \( \mu(x, y) \) over \( F \) (compactness allows us), so that \todo{Prove \hyperref[thm:weierstrass_extreme_value_theorem]{Weierstrass' theorem}}
  \begin{balign*}
    \op{dist}(x, G)
    \leq
    \mu(x, y) + \op{dist}(y, G)
    =
    \op{dist}(x, F) + \op{dist}(y, G)
    \leq
    \op{dist}(x, F) + e(F, G).
  \end{balign*}

  It now follows that
  \begin{balign*}
    e(E, G)
     & =
    \inf \{\delta > 0 \colon E \subseteq G_\delta \}
    =    \\ &=
    \inf \{\delta > 0 \colon E \subseteq \{ x \in X \colon \op{dist}(x, G) \leq \delta \}
    \leq \\ &\leq
    \inf \{\delta > 0 \colon E \subseteq \{ x \in X \colon \op{dist}(x, F) + e(F, G) \leq \delta, y \in X \}
    =    \\ &=
    e(F, G) + \inf \{\delta > 0 \colon E \subseteq F_\delta \}
    =    \\ &=
    e(F, G) + e(E, F).
  \end{balign*}
\end{proof}

  \subsection{Totally bounded sets}\label{subsec:totally_bounded_sets}

Let \( (X, \rho) \) be a \hyperref[def:metric_space]{metric space}.

\begin{definition}\label{def:epsilon_net}
  We say that \( E \subseteq X \) is an \( \varepsilon \)-\term{net} for the set \( A \subseteq X \) if
  \begin{equation}
    A \subseteq \bigcup_{x \in E} B(x, \varepsilon).
  \end{equation}
\end{definition}

\begin{definition}\label{def:totally_bounded_set}
  The space \( A \subseteq X \) is called \term{totally bounded} if any of the following equivalent conditions hold:

  \begin{thmenum}
    \thmitem{def:totally_bounded_set/sets} For every \( \varepsilon > 0 \) there exists a finite cover of \( A \) with sets with diameter at most \( \varepsilon \).
    \thmitem{def:totally_bounded_set/epsilon_net} For every \( \varepsilon > 0 \) there exists a finite \hyperref[def:epsilon_net]{\( \varepsilon \)-net} of \( A \).
    \thmitem{def:totally_bounded_set/zero_noncompactness/sets} \hyperref[def:noncompactness_measures/sets]{Kuratowski's noncompactness measure} \( \alpha(A) \) is zero.
    \thmitem{def:totally_bounded_set/zero_noncompactness/balls} The \hyperref[def:noncompactness_measures/balls]{ball noncompactness measure} \( \beta(A) \) is zero.
    \thmitem{def:totally_bounded_set/fundamental_subsequences} Every sequence in \( A \) admits a \hyperref[def:fundamental_net]{fundamental subsequence}.
  \end{thmenum}

  Totally bounded sets are sometimes called \term{\hyperref[def:compact_space]{precompact}} because of \fullref{thm:metric_compact_iff_sequentially_compact}. This equivalence requires the metric space to be complete, however.
\end{definition}
\begin{proof}
  \EquivalenceSubProof{def:totally_bounded_set/sets}{def:totally_bounded_set/zero_noncompactness/sets} Straightforward.

  \EquivalenceSubProof{def:totally_bounded_set/epsilon_net}{def:totally_bounded_set/zero_noncompactness/balls} Straightforward.

  \ImplicationSubProof{def:totally_bounded_set/epsilon_net}{def:totally_bounded_set/sets} Given \( \varepsilon > 0 \), any cover of \( A \) with balls of radius \( \frac \varepsilon 2 \) is a cover with sets of diameter \( \varepsilon \).

  \ImplicationSubProof{def:totally_bounded_set/sets}{def:totally_bounded_set/epsilon_net} Fix \( \varepsilon > 0 \) and \( \rho \in (0, \varepsilon) \) and let \( A_1, \ldots, A_n \subseteq \pow X \) be a finite cover of \( A \) with sets of diameter at most \( \rho \).

  Choose a point \( x_k \) from every \( A_k \), \( k = 1, \ldots, n \). We then have that for every \( k = 1, \ldots, n \),
  \begin{balign*}
    A_k \subseteq \cl B(x_k, \rho) \subsetneq B(x_k, \varepsilon)
    \\
    \implies A \subseteq \bigcup_{k=1}^n A_k \subseteq \bigcup_{k=1}^n B(x_k, \rho) \subsetneq \bigcup_{k=1}^n B(x_k, \varepsilon),
  \end{balign*}
  hence \( x_1, \ldots, x_n \) are centers of \( \varepsilon \)-balls that cover \( A \).

  \ImplicationSubProof{def:totally_bounded_set/epsilon_net}{def:totally_bounded_set/fundamental_subsequences} Let \( \{ x_n \} \subseteq A \) be any sequence.

  If we assume that \( \{ x_n \} \) has no fundamental subsequence, then there exists \( \varepsilon_0 > 0 \) such that \( \rho(x_k, x_m) > \varepsilon_0 \) for any \( n, m \in \BbbZ_{>0} \).

  Consider a finite cover of \( A \) with \( \varepsilon_0 \)-balls. By the pigeonhole principle, at least one of the balls contains more than one element of the sequence, which contradicts the assumption that all elements of the sequence have a distance of at least \( \varepsilon_0 \).

  Hence, an arbitrary sequence in \( A \) has a fundamental subsequence.

  \ImplicationSubProof{def:totally_bounded_set/fundamental_subsequences}{def:totally_bounded_set/epsilon_net} Assume that there exists \( \varepsilon_0 > 0 \), such that \( A \) admits no finite cover by \( \varepsilon_0 \)-balls.

  Define \( x_1 \in X, x_2 \in X \setminus B(x_1, \varepsilon_0), \ldots \), so that every two elements of the sequence \( \{ x_n \} \) have a distance of at least \( \varepsilon_0 \). But then the sequence is does not admit a fundamental subsequence, which contradicts our assumption.

  This contradiction proves that \( A \) admits a finite cover by \( \varepsilon \)-balls for every \( \varepsilon > 0 \).
\end{proof}

\begin{corollary}\label{thm:metric_space_compact_iff_closed_totally_bounded}
  Assume that \( X \) is complete. The set \( A \subseteq X \) is sequentially compact if and only if it is closed and totally bounded.
\end{corollary}
\begin{proof}
  The property that every sequence has a fundamental subsequence is equivalent to sequential compactness for a closed set in a complete metric space.
\end{proof}

\begin{proposition}\label{thm:totally_bounded_sets_are_bounded}
  Totally bounded sets are bounded.
\end{proposition}
\begin{proof}
  Fix a totally bounded set \( A \subseteq X \). Let \( \varepsilon > 0 \) and let \( x_1, x_2, \ldots, x_n \) be a finite \hyperref[def:totally_bounded_set/epsilon_net]{\( \varepsilon \)-net} of \( A \). The distance between two points of the \( \varepsilon \)-net is at most \( 2\varepsilon \). Then
  \begin{equation*}
    A \subseteq \bigcup_{i=1}^n B(x_i, \varepsilon) \subseteq B(x_i, 2 n \varepsilon).
  \end{equation*}

  Hence, \( A \) is \hyperref[def:metric_space/bounded_set]{bounded}.
\end{proof}

\begin{proposition}\label{thm:closure_of_totally_bounded_is_totally_bounded}
  If a set \( A \subseteq X \) is totally bounded, then, so is its closure \( \cl A \).
\end{proposition}
\begin{proof}
  Let \( \varepsilon > 0 \) and \( \rho \in (0, \varepsilon) \) and let \( x_1, \ldots, x_n \in X \) be the centers of a cover of \( A \) with \( \rho \)-balls.

  If \( y \) is a point in \( \cl A \setminus A \), there exists a point \( z \in A \) with \( \rho(y, z) < \varepsilon - \rho \). Let \( x_k \in A \) be one of the centers whose \( \rho \)-balls contain \( z \). We then have that \( y \in B(x_k, \varepsilon) \) since
  \begin{equation*}
    \rho(x_k, z) \leq \rho(x_k, y) + \rho(y, z) < \rho + \varepsilon - \rho = \varepsilon.
  \end{equation*}

  Hence, the balls \( \cl B(x_k, \varepsilon) \) cover \( \cl A \), i.e.
  \begin{equation*}
    \cl A \subseteq \bigcup_{k=1}^n B(x_k, \varepsilon).
  \end{equation*}
\end{proof}

\begin{lemma}[Lebesgue's covering lemma]\label{thm:lebesgues_covering_lemma}
  Assume that \( X \) is complete. Let \( A \subseteq X \) be sequentially compact. Given an open cover \( \mathcal{F} \subseteq \pow A \), there exists a number \( \delta > 0 \) such that every \( \delta \)-ball with a center in \( A \) is contained in some set of the cover \( \mathcal{F} \).
\end{lemma}
\begin{proof}
  Assume that no such number \( \delta > 0 \) exists. Then for any natural number \( n \in \BbbZ_{>0} \), there exists an element \( x_n \in A \) such that the ball \( B(x_n, \frac 1 n) \) is not contained in any set of the cover \( \mathcal{F} \). Since \( A \) is sequentially compact, the sequence \( \{ x_n \}_n \) contains a convergent subsequence \( \{ x_{n_k} \}_k \).

  Define
  \begin{equation*}
    x \coloneqq \lim_{k \to \infty} x_{n_k}.
  \end{equation*}

  Let \( E \) be a set in \( \mathcal{F} \) that contains \( x \). Since \( E \) is open, there exists some radius \( r > 0 \) such that \( B(x, r) \subseteq E \).

  Choose any \( k_0 > \frac 2 r \) such that \( \rho(x_{n_{k_0}}, x) < \frac r 2 \). By the triangle inequality,
  \begin{equation*}
    B \left(x_{n_k}, \frac 1 k \right) \subsetneq B \left(x_k, \frac r 2 \right) \subseteq B(x, r) \subseteq E,
  \end{equation*}
  which contradicts the choice of the sequence \( \{ x_n \}_n \).

  Hence, there exists a \( \delta > 0 \) such that for every \( x \in A \), the ball \( B(x, \delta) \) is contained in some element \( E \) of the cover \( \mathcal{F} \).
\end{proof}

\begin{theorem}\label{thm:metric_compact_iff_sequentially_compact}
  Assume that \( X \) is complete. The set \( A \subseteq X \) is compact if and only if it is sequentially compact.
\end{theorem}
\begin{proof}
  \SufficiencySubProof Let \( \mathcal{F} \subseteq \pow X \) be an open cover of \( A \).

  By \fullref{thm:lebesgues_covering_lemma}, there exists \( \delta > 0 \) such that for every \( x \in A \), the ball \( B(x, \delta) \) is contained in some set of the cover \( \mathcal{F} \). Let \( x_1, \ldots, x_n \) be a cover of \( A \) with \( \delta \)-balls.

  For each \( k = 1, \ldots, n \) we have that the ball \( B(x_k, \delta) \) is contained in some set \( E_k \in \mathcal{F} \). Hence, \( E_1, \ldots, E_n \) is a finite subcover of \( A \), because
  \begin{equation*}
    A \subseteq \bigcup_{k=1}^\infty B(x_k, \delta) \subseteq \bigcup_{k=1}^\infty E_k.
  \end{equation*}

  Thus, \( A \) is compact.

  \NecessitySubProof Let \( A \) be compact. Fix \( \varepsilon > 0 \) and take the cover
  \begin{equation*}
    \mathcal{F} \coloneqq \{ B(a, \varepsilon) \colon a \in A \}.
  \end{equation*}

  By compactness of \( A \), there exists a finite subcover. Thus, a finite cover of \( A \) with \( \varepsilon \)-balls exists for every \( \varepsilon > 0 \). \Fullref{def:totally_bounded_set} then implies that total boundedness is equivalent to sequential compactness because \( X \) is complete and \( A \) is closed.
\end{proof}

\begin{corollary}\label{thm:complete_metric_space_compact_conditions}
  The following are equivalent for a set \( A \) in complete metric space:
  \begin{thmenum}
    \thmitem{thm:complete_metric_space_compact_conditions/compact} \( A \) is \hyperref[def:compact_space]{compact}
    \thmitem{thm:complete_metric_space_compact_conditions/sequentially_compact} \( A \) is sequentially \hyperref[def:compact_space/convergent_nets]{compact}.
    \thmitem{thm:complete_metric_space_compact_conditions/closed_totally_bounded} \( A \) is closed and totally \hyperref[def:totally_bounded_set]{bounded}.
  \end{thmenum}
\end{corollary}
\begin{proof}
  \EquivalenceSubProof{thm:complete_metric_space_compact_conditions/compact}{thm:complete_metric_space_compact_conditions/sequentially_compact} The equivalence is given by \fullref{thm:metric_compact_iff_sequentially_compact}.

  \ImplicationSubProof{thm:complete_metric_space_compact_conditions/sequentially_compact}{thm:complete_metric_space_compact_conditions/closed_totally_bounded} The equivalence is given by \fullref{thm:metric_space_compact_iff_closed_totally_bounded}.
\end{proof}

  \subsection{Noncompactness measures}\label{subsec:noncompactness_measures}

\begin{definition}\label{def:noncompactness_measures}\mcite[def. 7.1]{Deimling1985}
  Let \( (X, \rho) \) be a \hyperref[def:metric_space]{metric space} and let \( \mscrB \) be the family of \hyperref[def:metric_space/bounded_set]{bounded sets} in \( X \). We define the following functions
  \begin{thmenum}
    \thmitem{def:noncompactness_measures/sets} The \term{Kuratowski measure of noncompactness},
    \begin{balign*}
       & \alpha: \mscrB \to [0, \infty)                                                                                                            \\
       & \alpha(A) \coloneqq \inf \{d > 0 \colon \exists U_1, \ldots, U_n \subseteq X: \diam {U_k} < d \T{and} A \subseteq \bigcup_{k=1}^n U_k \}
    \end{balign*}

    \thmitem{def:noncompactness_measures/balls} The \term{ball measure of noncompactness},
    \begin{balign*}
       & \beta: \mscrB \to [0, \infty)                                                                                   \\
       & \beta(A) \coloneqq \inf \{r > 0 \colon \exists x_1, \ldots, x_2 \in X: A \subseteq \cup_{k=1}^n B(x_k, r) \}
    \end{balign*}
  \end{thmenum}
\end{definition}

\begin{example}\label{ex:noncompactness_measures}\mcite[exer. 7.3]{Deimling1985}
  Consider the subsets \( B_2 \subseteq B_3 \subseteq B_1 \subseteq C([0, 1]) \), defined by
  \begin{balign*}
    B_1 & \coloneqq \left\{
    x \in C([0, 1]) \colon \begin{aligned}
      0 \leq t \leq 1 \implies 0 \leq x(t) \leq 1 \\
      x(0) = 0, x(1) = 1                          \\
    \end{aligned}
    \right\}
    \\
    B_2 & \coloneqq \left\{
    x \in B_1 \colon \begin{aligned}
      0 \leq t \leq \frac 1 2 \implies 0 \leq x(t) \leq \frac 1 2 \\
      \frac 1 2 \leq t \leq 1 \implies \frac 1 2 \leq x(t) \leq 1
    \end{aligned}
    \right\}
    \\
    B_3 & \coloneqq \left\{
    x \in B_1 \colon \begin{aligned}
      0 \leq t \leq \frac 1 2 \implies 0 \leq x(t) \leq \frac 2 3 \\
      \frac 1 2 \leq t \leq 1 \implies \frac 1 3 \leq x(t) \leq 1
    \end{aligned}
    \right\}
  \end{balign*}

  Then \( \alpha(B_1) = 1, \alpha(B_2) = \frac 1 2, \alpha(B_3) = \frac 1 3 \) and \( \beta(B_1) = \beta(B_2) = \beta(B_3) = \frac 1 2 \).
\end{example}
\begin{proof}
  Since the distance between any two functions from \( B_1 \) is at most 1, we have that \( \diam B_1 = 1 \) and \( \alpha(B_1) \leq 1 \).

  Fix \( \varepsilon > 0 \). For any function \( f \in B_1 \), continuity of \( f \) gives us a radius \( \delta_f > 0 \) such that
  \begin{equation*}
    x < 2 \delta_f \implies f(x) < \varepsilon.
  \end{equation*}

  \begin{figure}[!ht]
    \centering
    \includegraphics[page=1]{output/ex__noncompactness_measures}
    \caption{The operator \( T_\varepsilon \) adds \enquote{spikes} to functions.}\label{fig:ex:noncompactness_measures/spike_plot}
  \end{figure}

  Define
  \begin{balign*}
    T_\varepsilon(f)(x) \coloneqq \begin{cases}
      \frac x {\delta_f},                                       & 0 \leq x < \delta_f          \\
      f(\delta_f) + [1 - f(\delta_f)] (2 - \frac x {\delta_f}), & \delta_f \leq x < 2 \delta_f \\
      f(x),                                                     & x \geq 2 \delta_f,
    \end{cases}
  \end{balign*}
  so that
  \begin{balign*}
    \norm{T_\varepsilon(f) - f}
    \geq
    T_\varepsilon(f) (\delta_f) - f(\delta_f)
    =
    1 - f(\delta_f)
    >
    1 - \varepsilon.
  \end{balign*}

  Additionally, because \( \delta_{T_\varepsilon(f)} < \delta_f \), we have that \( f(\delta_{T_\varepsilon(f)}) < \varepsilon \) and
  \begin{balign*}
    \norm{T_\varepsilon(T_\varepsilon(f)) - f}
    \geq
    T_\varepsilon(T_\varepsilon(f)) (\delta_{T_\varepsilon(f)}) - f(\delta_{T_\varepsilon(f)})
    =
    1 - f(\delta_{T_\varepsilon(f)})
    >
    1 - \varepsilon.
  \end{balign*}

  Thus, proceeding by induction, we see that for any \( m = 1, 2, \ldots \)
  \begin{equation*}
    \norm{T_\varepsilon^m(f) - f} > 1 - \varepsilon,
  \end{equation*}
  where \( T_\varepsilon^m \) denotes repeated application of \( T_\varepsilon \).

  Consider the sequence
  \begin{equation*}
    \{ T_\varepsilon^k(f) \}_{k=0}^\infty = \{ f, T_\varepsilon(f), T_\varepsilon(T_\varepsilon(f)), \ldots \}.
  \end{equation*}

  We can easily see that the distance between any two elements of the sequence, say \( T_\varepsilon^k(f) \) and \( T_\varepsilon^{k+m}(f) \), is strictly greater that \( 1 - \varepsilon \), i.e.
  \begin{balign*}
    \norm{T_\varepsilon^k(f) - T_\varepsilon^{k+m}(f)}
    =
    \norm{T_\varepsilon^k(f) - T_\varepsilon^m(T_\varepsilon^k(f))}
    >
    1 - \varepsilon.
  \end{balign*}

  Hence, \( B_1 \) cannot be covered by a finite \( (1-\varepsilon) \)-net and \( \alpha(B_1) \geq 1 - \varepsilon \). Since \( \varepsilon > 0 \) can be made arbitrarily small, this implies that \( \alpha(B_1) \geq 1 \) and, because we already have the reverse inequality, \( \alpha(B_1) = 1 \).

  In the set \( B_2 \), the maximum distance between two functions is \( \frac 1 2 \), thus \( \diam(B_2) = \frac 1 2 \) and \( \alpha(B_2) \leq \frac 1 2 \). We can then define an operator similar to \( T_\varepsilon \) that creates \enquote{spikes} of height \( \frac 1 2 \) to prove the reverse inequality, obtaining
  \begin{equation*}
    \alpha(B_2) = \frac 1 2.
  \end{equation*}

  Finally, the set \( B_3 \) has diameter \( \frac 2 3 \) and hence \( \alpha(B_3) = \frac 2 3 \).

  The ball measure for \( B_1 \) satisfies the inequalities
  \begin{equation*}
    \frac 1 2 \leq \beta(B_1) \leq 1.
  \end{equation*}

  Additionally, \( B_1 \) is strictly contained in the ball centered in the constant function \( \frac 1 2 \) with radius \( \frac 1 2 \), which implies that \( \beta(B_1) \leq \frac 1 2 \), hence \( \beta(B_1) = \frac 1 2 \).

  For \( B_2 \) we have
  \begin{equation*}
    \frac 1 4 \leq \beta(B_2) \leq \frac 1 2.
  \end{equation*}

  Assume that for some \( \varepsilon > 0 \) the set \( B_2 \) can be covered by a finite set of balls with centers \( \{ f_1, \ldots, f_n \} \subsetneq C([0, 1]) \) and radius \( \frac 1 2 - \varepsilon \).

  Because of continuity, we can find a radius \( \delta > 0 \) such that for all \( f_k, k = 1, \ldots, n \) we have
  \begin{equation*}
    x \in \left[\tfrac {1 - \delta} 2, \tfrac {1 + \delta} 2 \right] \implies \abs{f_k(x) - f_k(\tfrac 1 2)} < \varepsilon.
  \end{equation*}

  Consider the function
  \begin{balign*}
    g(x) \coloneqq \begin{cases}
      0,                                & 0 \leq x < \frac {1 - \delta} 2,                       \\
      \frac{2x + \delta - 1} {2\delta}, & \frac {1 - \delta} 2 \leq x \leq \frac {1 + \delta} 2, \\
      1,                                & \frac {1 + \delta} 2 < x \leq 1.
    \end{cases}
  \end{balign*}

  \begin{figure}[!ht]
    \centering
    \includegraphics[page=2]{output/ex__noncompactness_measures}
    \caption{The function \( g \) always has points that are far enough from all \( f_k, k = 1, \ldots, n \).}\label{fig:ex:noncompactness_measures/sigmoid_plot}
  \end{figure}

  If \( f_k(\tfrac 1 2) \geq \frac 1 2 \), then \( f_k(\tfrac {1 - \delta} 2) > \tfrac 1 2 - \varepsilon \) and
  \begin{equation*}
    \norm{f_k - g} \geq f_k(\tfrac {1 - \delta} 2) - g(\tfrac {1 - \delta} 2) = f_k(\tfrac {1 - \delta} 2) > \tfrac 1 2 - \varepsilon.
  \end{equation*}

  Analogously, if \( f_k(\tfrac 1 2) < \frac 1 2 \), then \( f_k(\tfrac {1 + \delta} 2) < \tfrac 1 2 + \varepsilon \) and
  \begin{equation*}
    \norm{g - f_k} \geq g(\tfrac {1 + \delta} 2) - f_k(\tfrac {1 + \delta} 2) = 1 - f_k(\tfrac {1 + \delta} 2) > \tfrac 1 2 - \varepsilon.
  \end{equation*}

  Thus, for every \( k = 1, \ldots, n \) we have
  \begin{equation*}
    \norm{g - f_k} > \frac 1 2 - \varepsilon,
  \end{equation*}
  i.e. \( g \) in not contained in a ball of radius \( \frac 1 2 - \varepsilon \) around any of the centers \( f_1, \ldots, f_n \).

  Hence, \( \beta(B_2) \geq \frac 1 2 \), which implies \( \beta(B_2) = \frac 1 2 \). Because of the inclusion \( B_2 \subsetneq B_3 \subsetneq B_1 \), we have
  \begin{equation*}
    \frac 1 2 = \beta(B_2) \leq \beta(B_3) \leq \beta(B_1) = \frac 1 2,
  \end{equation*}
  hence \( \beta(B_3) = \frac 1 2 \).
\end{proof}

\begin{theorem}[Kuratowski's noncompactness lemma]\label{thm:noncompact_kuratowskis_lemma}\mcite[exer. 7.4]{Deimling1985}
  Let \( X \) be a Banach space and \( \{ A_n \}_n \) be a decreasing sequence of nonempty closed subsets such that \( \alpha(A_n) \to 0 \). Then \( A \coloneqq \bigcap_n A_n \) is nonempty and compact.
\end{theorem}
\begin{proof}
  The set \( A \) is compact because it is closed as the intersection of closed sets and \( \alpha(A) \leq \alpha(A_n) \to 0 \), hence \( \alpha(A) = 0 \).

  It remains to show that \( A \) is nonempty.
  Choose any sequence \( \{ x_n \}_n \) where \( x_n \in A_n \). Since any finite set is compact, we have that for any \( k \geq 1 \)
  \begin{balign*}
    \alpha(\{ x_n \}_{n \geq 1})
    =
    \max\{ \alpha(\{ x_n \}_{n < k}), \alpha(\{ x_n \}_{n \geq k}) \}
    =
    \alpha(\{ x_n \}_{n \geq k})
    \leq
    \alpha(A_k) \to 0,
  \end{balign*}
  hence the set \( \{ x_n \colon n \geq 1 \} \) is compact and thus sequentially compact. We can choose a convergent subsequence \( \{ x_{n_k} \}_k \) of \( \{ x_n \}_n \) whose limit lies in every \( A_n \) (since they are closed) and, consequently, in their intersection \( A \). So \( A \) is nonempty.
\end{proof}

  \subsection{Lipschitz continuity}\label{subsec:lipschitz_continuity}

\begin{definition}\label{def:lipschitz_continuity}
  Let \( f: X \to Y \) be a function between metric spaces.

  \begin{thmenum}
    \thmitem{def:lipschitz_continuity/holder} We say that \( f: X \to Y \) is \term{H\"older continuous} at \( x \in X \) with constant \( L \geq 0 \) and exponent \( \alpha > 0 \) if
    \begin{equation*}
      \rho_Y(f(x_1), f(x_2)) \leq L \rho_X(x_1, x_2)^\alpha \quad\forall x_1, x_2 \in X.
    \end{equation*}

    We refer to the smallest such constant, if any, as \enquote{the} H\"older constant.

    \thmitem{def:lipschitz_continuity/locally_holder} We say that \( f \) is \term{locally H\"older continuous} if every point has a neighborhood where \( f \) is H\"older continuous with the same exponent, but possibly with with a different constant.

    \thmitem{def:lipschitz_continuity/lipschitz} If \( \alpha = 1 \), we say that \( f \) is \term{Lipschitz continuous}.

    \thmitem{def:lipschitz_continuity/contraction} If \( X = Y \) and if \( f \) is Lipschitz with constant \( L < 1 \), we call \( f \) a \term{contraction mapping}.

    \thmitem{def:lipschitz_continuity/calm}\cite[53]{DontchevRockafellar2014} We say that \( f \) is \term{calm} at \( x \) if it satisfies the Lipschitz condition with one of the points fixed:
    \begin{equation*}
      \rho_Y(f(x), f(x')) \leq L \rho_X(x, x') \quad\forall x' \in X.
    \end{equation*}
  \end{thmenum}
\end{definition}

\begin{proposition}\label{thm:holder_map_is_uniformly_continuous}
  A H\"older map is uniformly continuous.
\end{proposition}
\begin{proof}
  Let \( f: X \to Y \) be a H\"older map with constant \( L \) and exponent \( \alpha \).

  Fix \( \varepsilon > 0 \). Then is enough to choose \( \delta < \sqrt[\alpha]{\frac \varepsilon L} \), so that
  \begin{equation*}
    \rho_X(x_1, x_2) < \delta \implies \rho_Y(f(x_1), f(x_2)) \leq L \rho_X(x_1, x_2)^\alpha < L \delta^\alpha < \varepsilon.
  \end{equation*}

  This implies uniform continuity.
\end{proof}

\begin{corollary}\label{thm:locally_holder_map_is_continuous}
  A locally H\"older map is continuous.
\end{corollary}

\begin{theorem}[Banach's fixed point theorem]\label{thm:banach_fixed_point_theorem}\mcite[exer. 4.3.J]{Engelking1989}
  A contraction \hyperref[def:lipschitz_continuity/contraction]{mapping} in a \hyperref[def:complete_metric_space]{complete metric space} has a unique fixed \hyperref[def:fixed_point]{point}.
\end{theorem}
\begin{proof}
  Let \( f: X \to X \) be a contraction mapping. Fix any point \( x_0 \in X \) and inductively define the sequence
  \begin{equation*}
    x_{k+1} \coloneqq f(x_k), k = 1, 2, \ldots
  \end{equation*}

  Fix \( \varepsilon > 0 \). Since \( L < 1 \), there exists an index \( k_0 > \log_L(\varepsilon) \) such that for positive integers \( m \) and \( k > k_0 \),
  \begin{balign*}
    \rho(x_k, x_{k+m})
     & =
    \rho(f^k(x_0), f^{k+m}(x_0))
    \leq \\ &\leq
    L^k \rho(x_0, x_m)
    <    \\ &<
    \varepsilon \rho(x_0, x_m).
  \end{balign*}

  Note that
  \begin{balign*}
    \rho(x_0, x_m)
     & \leq
    \sum_{i=1}^m \rho(x_{i-1}, x_i)
    \leq    \\ &\leq
    \rho(x_0, x_1) \sum_{i=1}^m L^{i-1}
    =       \\ &=
    \rho(x_0, x_1) \frac {1 - L^m} {1 - L}
    \leq    \\ &\leq
    \rho(x_0, x_1) \frac 1 {1 - L}.
  \end{balign*}

  Thus,
  \begin{equation*}
    \rho(x_k, x_{k+m}) < \frac {\varepsilon \rho(x_0, x_1)} {1 - L}.
  \end{equation*}

  The constant on the right is linear in \( \varepsilon \) and does not depend on \( k \) or \( m \), hence \( \{ x_k \}_{k=0}^\infty \) is a fundamental sequence. Since \( X \) is complete, the sequence has a limit \( x \).

  Because of the continuity of \( f \) (see \fullref{thm:holder_map_is_uniformly_continuous}),
  \begin{equation*}
    f(x) = f(\lim_{k \to \infty} x_k) = \lim_{k \to \infty} f(x_k) = \lim_{k \to \infty} x_{k+1} = x.
  \end{equation*}
\end{proof}

  \section{Function oscillation}\label{sec:function_oscillation}

\begin{definition}\label{def:function_oscillation}
  Let \( X \) be a nonempty set and \( (Y, \rho_{Y}) \) be a metric space. We define the \term{oscillation} of a function on a set as
  \begin{balign*}
     &\omega: \fun(X, Y) \times \pow(X) \to [0, \infty] \\
     &\omega(f, A) \coloneqq \sup \Big\{ \rho_{Y}(f(x), f(y)) \colon (x, y) \in A \Big\}.
  \end{balign*}

  In particular, if \( X \) is itself a metric space, we define its \term{modulus of continuity} \( \omega(f, \delta) \) as the oscillation of \( f \) on the ball \( B(0, \delta) \).
\end{definition}

\begin{proposition}\label{thm:def:function_oscillation}
  The \hyperref[def:function_oscillation]{modulus of continuity} has the following basic properties:
  \begin{thmenum}
    \thmitem{thm:def:function_oscillation/continuity_condition} \( f \) is globally \hyperref[def:uniform_continuity]{uniformly continuous} if and only if for every \( \varepsilon > 0 \) there exists \( \delta > 0 \) such that \( \omega(f, \delta) < \varepsilon \).

    \thmitem{thm:def:function_oscillation/monotone} \( \omega(f, \delta) \) is monotone in \( \delta \).

    \thmitem{thm:def:function_oscillation/cauchy_inequality}\mcite[28]{Николов2020АпроксимацииЛекции}For all \( \lambda, \delta > 0 \), we have the following analog of \fullref{thm:cauchy_bunyakovsky_schwarz_inequality}
    \begin{equation}\label{thm:def:function_oscillation/cauchy_inequality/inequality}
      \omega(f, \lambda \delta) \leq \omega(f, \lambda^2) + \omega(f, \delta^2).
    \end{equation}

    \thmitem{thm:def:function_oscillation/single_inequality}\mcite[28]{Николов2020АпроксимацииЛекции}For all \( \lambda, \delta > 0 \),
    \begin{equation}\label{thm:def:function_oscillation/single_inequality/inequality}
      \omega(f, \lambda \delta) \leq (\lambda + 1) \omega(f, \delta).
    \end{equation}
  \end{thmenum}
\end{proposition}
\begin{proof}
  \SubProofOf{thm:def:function_oscillation/continuity_condition} Follows directly from \cref{def:uniform_continuity}.

  \SubProofOf{thm:def:function_oscillation/monotone} A supremum on a larger set is larger.

  \SubProofOf{thm:def:function_oscillation/cauchy_inequality} If \( \lambda \leq \delta \), clearly \( \lambda \delta \leq \delta^2 \). Otherwise, \( \lambda \delta < \lambda^2 \).

  Combining the two inequalities with \cref{thm:def:function_oscillation/monotone}, we obtain \cref{thm:def:function_oscillation/cauchy_inequality/inequality}.

  \SubProofOf{thm:def:function_oscillation/single_inequality} Note that
  \begin{equation*}
    \rho_{X}(x, y) < \delta \T{implies} \rho_{Y}(f(x), f(y)) < \omega(f, \delta).
  \end{equation*}

  We can multiply this by \( \lambda \) to obtain
  \begin{equation*}
    \lambda \rho_{X}(x, y) < \lambda \delta \T{implies} \lambda \rho_{Y}(f(x), f(y)) < \lambda \omega(f, \delta).
  \end{equation*}

  If \( \lambda \geq 1 \), then \( \rho_{X}(x, y) \leq \lambda \rho_{X}(x, y) \) and \( \rho_{Y}(f(x), f(y)) \leq \lambda \rho_{Y}(f(x), f(y)) \) and hence
  \begin{equation*}
    \omega(f, \lambda \delta) \leq \lambda \omega(f, \delta).
  \end{equation*}

  Otherwise, \( \lambda < 1 \) and clearly \( \lambda \delta < \delta \), which by \cref{thm:def:function_oscillation/monotone} implies
  \begin{equation*}
    \omega(f, \lambda \delta) \leq \omega(f, \delta).
  \end{equation*}

  Combining the two cases, we obtain
  \begin{equation*}
    \omega(f, \lambda \delta) \leq \lambda \omega(f, \delta) + \omega(f, \delta),
  \end{equation*}
  which we wanted to prove.
\end{proof}


  \section{Geometry}\label{sec:geometry}

Humans possess a strong intuition for visual information like drawings or diagrams. A drawing on a paper is only a medium for communicating ideas and data. \Fullref{fig:sec:geometry/figures} contains some highlighted curves that our mind maps to abstract geometric figures, without considering the size limitations of the page, the precision of the drawings or the thickness of the lines.

\begin{figure}[!ht]
  \centering
  \includegraphics{output/sec__geometry__figures}
  \caption{A \hyperref[def:triangle]{triangle}, a \hyperref[def:circle]{circle} and a \hyperref[def:affine_line]{line} in the \hyperref[def:euclidean_space]{Euclidean plane}.}\label{fig:sec:geometry/figures}
\end{figure}

Our goal is to introduce formalisms for these mental visualizations. An axiomatic approach for a theory of figures in the place and space was developed by Euclid in the third century BC and can be found in \cite{Fitzpatrick2008EuclidsElements}. A few millennia later, several mathematicians proposed systems of axioms that fit the requirements of \hyperref[sec:mathematical_logic]{modern logical systems}. Tarski's system can be found in \cite{Tarski1959ElementaryGeometry}. This is known today as \term{synthetic Euclidean geometry} and is mostly of theoretical interest.

An important distinction between ancient and modern geometry is the introduction of coordinates in the 17th century. Descartes' idea of coordinates connects problems of algebra and geometry in such a way that most of today's mathematics seamlessly switches between algebraic and geometric interpretations of the same problem. The study of classical Greek geometry in terms of coordinates is known as \term{analytic geometry}.

A modern interpretation if the ideas behind analytic geometry leads to \hyperref[def:affine_space]{affine spaces}, which we will discuss in \fullref{subsec:affine_spaces} and \fullref{subsec:convex_sets}, to Euclidean spaces discussed in \fullref{subsec:euclidean_spaces} and to the Euclidean plane discussed in \fullref{subsec:euclidean_plane}, \fullref{subsec:triangles} and \fullref{subsec:quadratic_plane_curves}. As part of our discussion of the \hyperref[def:euclidean_plane]{Euclidean plane}, we will briefly introduce modern concepts related to differential geometry in \fullref{subsec:parametric_curves} and algebraic geometry in \fullref{subsec:quadratic_plane_curves}.

  \subsection{Affine spaces}\label{subsec:affine_spaces}

We provide here general definitions and statements about affine spaces over arbitrary fields.

\begin{definition}\label{def:affine_space}\mcite[def. 2.1.1]{Gallier2011}
  An \term{affine space} over the \hyperref[def:field]{field} \( \BbbK \) is a triple \( (A, \vect A, \tau) \), where
  \begin{thmenum}[series=def:affine_space]
    \thmitem{def:affine_space/points} \( A \) is a set, whose members we call \term{points}.

    \thmitem{def:affine_space/vectors} \( \vect A \) is a \hyperref[def:vector_space]{vector space} over \( \BbbK \), whose members we call \term{vectors}. We also call them \term{free vectors}, as opposed to \term{bound vectors}, which are simply pairs of points --- a \term{beginning} and an \term{end}.

    \thmitem{def:affine_space/action} \( \tau: \vect A \times A \to A \) is a \hyperref[def:group_action]{group action} of the additive group of \( \vect A \) that links bound and free vectors. For a bound vector from \( x \) to \( y \), we require that there exists a unique free vector \( v \) satisfying \( y = \tau_v(x) \). We also say that \( y \) is a \term{translation} of \( x \) in the \term{direction} \( v \).

    We will use the notation \( \vect{xy} \) for this unique free vector.
  \end{thmenum}

  We define the \term{dimension} of \( A \) as the vector space dimension of \( \vect A \).

  Usually \( A = \vect A \) and \( \tau_v(x) = x + v \), in which case the distinction between \enquote{points} and \enquote{vectors} is only informal. We call this the \term{natural affine structure} on \( \vect A \). The free vector \( \vect{xy} \) is then \( y - x \). We implicitly associate the natural affine structure with every vector space --- usually the \hyperref[def:sequence_space]{tuple vector space} \( \BbbK^n \).

  \begin{figure}[!ht]
    \hfill
    \hfill
    \includegraphics[align=c]{output/def__affine_space__vectors}
    \hfill
    \includegraphics[align=c]{output/def__affine_space__points}
    \hfill
    \caption{Vectors acting on points in an \hyperref[def:affine_space]{affine space}.}\label{fig:def:affine_space}
  \end{figure}
\end{definition}

\begin{example}\label{ex:x2_as_affine_space}
  Simple yet nontrivial example of \hyperref[def:affine_space]{affine spaces} are \hyperref[def:set_valued_map/graph]{function graphs}.
  \begin{itemize}
    \item The set of points is the following subset of \( \BbbR^2 \):
    \begin{equation*}
      A \coloneqq \set{ (x, y) \given y = x^2 }.
    \end{equation*}

    \begin{figure}[!ht]
      \centering
      \includegraphics{output/ex__x2_as_affine_space}
      \caption{The affine space from \cref{ex:x2_as_affine_space}.}\label{fig:ex:x2_as_affine_space}
    \end{figure}

    \item The set of vectors is the unidimensional vector space \( \BbbR \).

    \item The action of \( \BbbR \) on \( A \) is
    \begin{equation*}
      \tau_t(x, y) \coloneqq (x + t, y + t^2).
    \end{equation*}
  \end{itemize}
\end{example}

\begin{definition}\label{def:affine_coordinate_system}
  Let \( (A, \vect A, \tau) \) be an \hyperref[def:affine_space]{affine space} over \( \BbbK \), let \( O \) be a distinguished point of \( A \) and let \( \vect E \) be a \hyperref[def:hamel_basis]{basis} of \( \vect A \). We call the pair \( (O, E) \) an \term{affine coordinate system} of \( A \) with \term{origin} \( O \) and \term{basis} \( \vect E \). If \( \vect E \) is \hyperref[def:orthogonality]{orthonormal}, we say that the coordinate system itself is orthonormal. Usually \( \vect A \) is finite dimensional, in which case \( \vect E = \set{ e_1, \ldots, e_n } \), and we denote the coordinate system by \( O e_1 \cdots e_n \).

  For every point \( x \) in \( A \), we call the free vector \( \vect{Ox} \) its \term{radius vector} with respect to the origin \( O \). This gives us an explicit isomorphism between the set \( A \) of points and the set \( \vect A \) of free vectors.

  The choice of basis \( \vect E \) then gives an isomorphism between \( A \) and the \hyperref[def:free_semimodule]{free vector space} \( \BbbK^{\oplus E} \). If \( \vect E = \set{ e_1, \ldots, e_n } \), the corresponding coordinate system allows us to associate a tuple of coordinates from \( \BbbK^n \) for each point in \( A \).

  Using the notation \( x_v \coloneqq \pi_v(\vect{Ox}) \), we associate the family \( \seq{ x_v \given v \in \vect E } \) to every point \( x \) and call its members the \term{affine coordinates} of \( x \) with respect to the coordinate system. We have
  \begin{equation}\label{eq:def:affine_coordinate_system/coordinates}
    \vect{Ox} = \sum_{e \in \vect E} x_e \cdot e.
  \end{equation}

  This relates to \hyperref[def:barycentric_coordinate_system]{barycentric coordinate systems} via \fullref{thm:affine_and_barycentric_coordinate_systems}.
\end{definition}

\begin{remark}\label{rem:affine_combinations}
  We cannot define the familiar addition and scalar multiplication of points that does not depend on a choice of coordinate system. We can, however, define a certain type of linear combination of points, which we call \term{affine combinations}.

  Let \( (A, \vect A, \tau) \) be an \hyperref[def:affine_space]{affine space} over \( \BbbK \), let \( x_1, \ldots, x_n \) be points in \( A \) and let \( t_1, \ldots, t_n \) be scalars that sum to \( 1_\BbbK \).

  For any two origin points \( O \) and \( P \), we have
  \begin{equation*}
    \underbrace{ \sum_{k=1}^n t_k \vect{O x_k} }_{v_O}
    =
    \sum_{k=1}^n t_k \parens[\Big]{ \vect{OP} + \vect{P x_k} }
    =
    \underbrace{ \parens*{ \sum_{k=1}^n t_k } }_{1_\BbbK} \vect{OP} + \underbrace{ \sum_{k=1}^n t_k \vect{P x_k} }_{v_P}.
  \end{equation*}

  Therefore,
  \begin{equation*}
    \tau_{v_O}(O) = \tau_{\vect{OP} + v_P}(O) = \tau_{v_P}(\tau_{\vect{OP}}(O)) = \tau_{v_P}(P).
  \end{equation*}

  We denote this common value by
  \begin{equation*}
    \sum_{k=1}^n t_k x_k.
  \end{equation*}

  We also use the term \enquote{affine combination} for linear combinations of vectors whose coefficients sum to one.

  Formally, it is sufficient to consider only affine combinations of two points, i.e.
  \begin{equation}\label{eq:rem:affine_combinations}
    \lambda x + (1 - \lambda) y.
  \end{equation}

  The affine combination of \( n \) points can then be defined via \hyperref[rem:natural_number_recursion]{recursion}:
  \begin{equation*}
    \sum_{k=1}^n t_k x_k = \begin{dcases}
      x_n,                                                          &n = 1 \\
      t_1 x_1 + (1 - t_1) \sum_{k=2}^n \frac {t_k} {(1 - t_1)} x_k, &n > 1
    \end{dcases}
  \end{equation*}
\end{remark}

\begin{proposition}\label{thm:rem:affine_combinations}
  \hyperref[rem:affine_combinations]{Affine combinations} have the following basic properties:
  \begin{thmenum}
    \thmitem{thm:rem:affine_combinations/inverse_of_action} For any point \( x \) and vector \( v \),
    \begin{equation*}
      \tau_v^{-1}(x) = \tau_{-v}(x).
    \end{equation*}

    \thmitem{thm:rem:affine_combinations/inverse} For any two points \( x \) and \( y \),
    \begin{equation*}
      \vect{xy} = -\vect{xy}.
    \end{equation*}

    \thmitem{thm:rem:affine_combinations/chasles} For any three points \( x \), \( y \) and \( z \),
    \begin{equation}\label{eq:thm:rem:affine_combinations/chasles}
      \vect{xz} = \vect{xy} + \vect{yz}.
    \end{equation}

    This is called Chasles' identity.

    \thmitem{thm:rem:affine_combinations/vectors_to_points} If we are given the vector affine combination \( w = \lambda u + (1 - \lambda) v \), for any point \( x \) we have the point affine combination
    \begin{equation*}
      \tau_w(x) = \lambda \cdot \tau_u(x) + (1 - \lambda) \cdot \tau_v(x).
    \end{equation*}
  \end{thmenum}
\end{proposition}
\begin{proof}
  \SubProofOf{thm:rem:affine_combinations/inverse_of_action} If \( y = \tau_v(x) \), then
  \begin{equation*}
    x = \tau_{-v + v}(x) = \tau_{-v}(y).
  \end{equation*}

  \SubProofOf{thm:rem:affine_combinations/inverse} Follows from \fullref{thm:rem:affine_combinations/inverse_of_action}.

  \SubProofOf{thm:rem:affine_combinations/chasles}
  \begin{equation*}
    \tau_{\vect{xy} + \vect{yz}}(x)
    =
    \tau_{\vect{yz} + \vect{xy}}(x)
    =
    \tau_{\vect{yz}}(\tau_{\vect{xy}}(x))
    =
    \tau_{\vect{yz}}(y)
    =
    z
    =
    \tau_{\vect{xz}}(x).
  \end{equation*}

  Then \eqref{eq:thm:rem:affine_combinations/chasles} follows.

  \SubProofOf{thm:rem:affine_combinations/vectors_to_points} Follows from the definition by noting that, for any point \( x \),
  \begin{equation*}
    w = \lambda u + (1 - \lambda) v
    =
    \vect{x, \tau_w(x)} = \lambda \cdot \vect{x, \tau_u(x)} + (1 - \lambda) \cdot \vect{x, \tau_v(x)}.
  \end{equation*}
\end{proof}

\begin{definition}\label{def:affine_hull}\mimprovised
  The \term{affine hull} of the set \( S \) of points is the set of all arbitrary (i.e. not necessarily binary) \hyperref[rem:affine_combinations]{affine combinations} of members of \( S \). It is a \hyperref[def:moore_closure_operator]{Moore closure operator}.
\end{definition}
\begin{defproof}
  We will show that the affine hull operator \( H: \pow(A) \to \pow(A) \) is a closure operator.

  \SubProofOf[def:extensive_function]{extensiveness} Obviously \( S \subseteq H(S) \).

  \SubProofOf[def:idempotent_function]{idempotence} Consider the affine combination \( \lambda x + (1 - \lambda) y \) of points from \( H(S) \). Suppose that \( x = t_x a_x + (1 - t_x) b_x \) and \( y = t_y a_y + (1 - t_y) b_y \), where \( a_x \), \( a_y \), \( b_x \) and \( b_y \) are points from \( S \).

  Then
  \begin{equation*}
    \lambda x + (1 - \lambda) y
    =
    \lambda t_x u_x + \lambda (1 - t_x) v_x + (1 - \lambda) t_y u_y + (1 - \lambda) (1 - t_y) v_y.
  \end{equation*}

  This is an affine combination of members of \( S \). Therefore, \( A(A(S)) = A(S) \).

  \SubProofOf[def:order_function/preserving]{monotonicity} If \( S_1 \subseteq S_2 \), then \( H(S_2) \) contains the affine combinations of the members of \( S_1 \) in addition to others. Hence, \( H(S_1) \subseteq H(S_2) \).
\end{defproof}

\begin{definition}\label{def:affine_subspace}\mcite[25]{Gallier2011}
  Let \( B \) be a set of points in the \hyperref[def:affine_space]{affine space} \( (A, \vect A, \tau) \). Fix an origin point \( O \) in \( B \) and define the set
  \begin{equation}\label{eq:def:affine_subspace/direction}
    \vect B \coloneqq \set{ \vect{Ob} \given b \in B \T{and} b \neq O }.
  \end{equation}

  We present two equivalent conditions under which the set \( \vect B \) does not depend on the choice of origin. If any of them holds, we say that \( \vect B \) is the \term{direction} of \( B \) and that \( (B, \vect B, \tau) \) is an \term{affine subspace} of \( (A, \vect A, \tau) \).

  \begin{thmenum}
    \thmitem{def:affine_subspace/hull} If \( B \) coincides with its \hyperref[def:affine_hull]{affine hull}.
    \thmitem{def:affine_subspace/linear} If \( \vect B \) is a \hyperref[def:module/submodel]{vector subspace} of \( \vect A \).
  \end{thmenum}
\end{definition}
\begin{defproof}
  \SubProof{Proof of independence of origin} Let
  \begin{align*}
    \vect B_O \coloneqq \set{ \vect{Ob} \given b \in B \T{and} b \neq O },
    &&
    \vect B_P \coloneqq \set{ \vect{Pb} \given b \in B \T{and} b \neq P }.
  \end{align*}

  Let \( x \) be a point in \( B \) and consider the vector \( \vect{P x} \) from \( \vect B_P \). We have
  \begin{equation*}
    \vect{P x}
    \reloset {\ref{thm:rem:affine_combinations/chasles}} =
    \vect{PO} + \vect{O x}
    \reloset {\ref{thm:rem:affine_combinations/inverse}} =
    \vect{OO} - \vect{OP} + \vect{O x}.
  \end{equation*}

  Thus,
  \begin{equation*}
    \tau_{\vect{Px}}(O) = O - P + x.
  \end{equation*}

  The latter is an affine combination of points of \( B \), hence it is again a point of \( B \). That is, \( \vect{Px} \) is the free vector that takes \( O \) to \( O - P + x \), and hence \( \vect{Px} \) belongs to \( \vect B_O \).

  Generalizing on \( x \), we conclude that \( \vect B_P \subseteq \vect B_O \). The converse inclusion follows automatically.

  \ImplicationSubProof{def:affine_subspace/hull}{def:affine_subspace/linear} Suppose that \( B \) coincides with its affine hull.

  Let \( x \) and \( y \) be points in \( B \). Note that \( z \coloneqq x + y - O \) is an affine combination and thus belongs to \( B \). Then
  \begin{equation*}
    \vect{Ox} + \vect{Oy}
    =
    \vect{Ox} + \vect{Oy} - \vect{OO}
    =
    \vect{Oz}
    \in
    \vect B.
  \end{equation*}

  Now let \( x \) be a point in \( B \) and \( \lambda \) be a scalar. Then \( y \coloneqq \lambda x + (1 - \lambda) O \) is an affine combination and thus belongs to \( B \). Hence,
  \begin{equation*}
    \lambda \vect{Oy}
    =
    \lambda \vect{Oy} + (1 - \lambda) \vect{OO}
    =
    \vect{Oy}
    \in
    \vect B.
  \end{equation*}

  Therefore, \( \vect B \) is closed under linear combinations and is thus a subspace of \( \vect A \).

  \ImplicationSubProof{def:affine_subspace/linear}{def:affine_subspace/hull} Fix some point \( O \) in \( B \) and suppose that \( \vect B \) is a vector subspace of \( \vect A \).

  Fix two points \( x \) and \( y \) in \( B \) and some scalar \( \lambda \). Then there exists some point \( z \) in \( B \) such that
  \begin{equation*}
    \vect{Oz} = \lambda \vect{Ox} + (1 - \lambda) \vect{Oy}.
  \end{equation*}

  Then \( z = \lambda x + (1 - \lambda) y \), which implies that \( B \) is closed under affine combinations.
\end{defproof}

\begin{proposition}\label{thm:affine_subspace_of_subspace}
  \hyperref[def:affine_subspace]{Affine subspaces} of affine subspaces of \( A \) are subspaces of \( A \).
\end{proposition}
\begin{proof}
  Trivial.
\end{proof}

\begin{definition}\label{def:affine_parallelism}\mcite[25]{Gallier2011}
  We say that two \hyperref[def:affine_subspace]{affine subspaces} of a common ambient space are \term{parallel} if their directions coincide. We write \( B \parallel C \).
\end{definition}

\begin{proposition}\label{thm:parallel_subspace_through_point}
  For every point \( x \) and every \hyperref[def:affine_subspace]{affine subspace} \( A \), there exists a unique subspace \hyperref[def:affine_parallelism]{parallel} to \( A \) passing through \( x \).

  The case of an \hyperref[def:affine_line]{affine line} is a restatement of Euclid's \enquote{Proposition \( \beta. \)} (see \cite[8]{Fitzpatrick2008Euclid}):
  \begin{quote}
    To place a straight-line equal to a given straight-line at a given point (as an extremity).
  \end{quote}
\end{proposition}
\begin{proof}
  Define the set
  \begin{equation*}
    B \coloneqq \set{ \tau_v(x) \given v \in \vect A }.
  \end{equation*}

  It obviously contains \( x \) and obviously the directions \( \vect A \) and \( \vect B \) coincide, hence \( A \) and \( B \) are parallel.

  It remains to show uniqueness. Suppose that \( C \) also contains \( x \) and is parallel to \( B \). Let \( y \) be a point of \( B \). Then \( \vect{yx} \) is also a vector in \( \vect C \), hence \( y = \tau_{\vect{yx}}(x) \) is a point in \( C \). Therefore, \( B = C \).
\end{proof}

\begin{definition}\label{def:affine_space_of_solutions}\mimprovised
  Consider the \hyperref[rem:system_of_equations]{system} of \( n \) linear equations and \( m \) variables over the field \( \BbbK \) in matrix form \( Ax = b \). If at least one solution \( x_0 \) exists, then all solutions are contained in the set
  \begin{equation}\label{eq:def:affine_space_of_solutions}
    \set{ x_0 + x \given x \in \ker A }.
  \end{equation}

  This is an \hyperref[def:affine_subspace]{affine subspace} of the domain \( \BbbK^m \) of \( A \). We call it the \term{solution space} of the system. If no solutions exist, the solution space should be an empty set, hence formally not an affine space.
\end{definition}
\begin{defproof}
  We will prove that the set \eqref{eq:def:affine_space_of_solutions} is precisely the set of solutions of the system.

  If \( x \in \ker A \), then
  \begin{equation*}
    A(x_0 + x) = Ax_0 + Ax = Ax_0.
  \end{equation*}

  Conversely, if \( Ax_1 = b \), then \( A(x_1 - x_0) = \vect 0 \), and thus \( x_1 - x_0 \in \ker A \). Then \( x_1 = x_0 + (x_1 - x_0) \) belongs to the aforementioned solution space.
\end{defproof}

\begin{proposition}\label{thm:system_of_equations_unique_solution}
  The \hyperref[rem:system_of_equations]{system of linear equations} \( Ax = b \) has a unique solution if and only if \( A \) is invertible.
\end{proposition}
\begin{proof}
  \SufficiencySubProof If \( Ax = b \) has a unique solution, then the \hyperref[def:affine_space_of_solutions]{solution space} has dimension zero, the kernel of \( A \) is empty, and \( A \) is invertible.

  \NecessitySubProof If \( A \) is invertible, then \( x = A^{-1} b \) is a solution.
\end{proof}

\begin{definition}\label{def:affine_dependence}\mimprovised
  We say that the set of points \( E \) is \term{affinely independent} if any of the following conditions holds:

  \begin{thmenum}
    \thmitem{def:affine_dependence/hull} The \hyperref[def:affine_hull]{affine hull} of \( E \) strictly contains the affine hulls of any subset of \( E \).

    \thmitem{def:affine_dependence/combinations} Given a sequence of points \( O, e_1, \ldots, e_n \) from \( E \), the conditions
    \begin{align*}
      \sum_{k=1}^n t_k = 0,
      &&
      \sum_{k=1}^n t_k \vect{O e_k} = \vect 0
    \end{align*}
    together imply that \( t_1 = \cdots = t_n \).

    \thmitem{def:affine_dependence/linear}\mcite[def. 2.4]{Gallier2011} Given any origin point \( O \) from \( E \), the set
    \begin{equation*}
      \set{ \vect{O e} \given e \in E \T{and} e \neq O }
    \end{equation*}
    is linearly independent.

    If this condition holds for \( O \), it also holds if we replace \( O \) with any other point in \( E \).
  \end{thmenum}
\end{definition}
\begin{proof}
  \SubProofOf{def:affine_dependence/linear} Suppose that
  \begin{equation*}
    \set{ \vect{O e} \given e \in E \T{and} e \neq O }
  \end{equation*}
  is linearly independent. We will show that
  \begin{equation*}
    \set{ \vect{P e} \given e \in E \T{and} e \neq P }
  \end{equation*}
  is also linearly independent for any \( P \neq O \). Let
  \begin{equation*}
    \sum_{k=1}^n t_k \vect{P e_k} = \vect 0.
  \end{equation*}

  \Fullref{thm:rem:affine_combinations/chasles} implies
  \begin{equation*}
    \sum_{k=1}^n t_k \vect{P e_k}
    =
    \sum_{k=1}^n t_k \vect{O e_k} + \sum_{k=1}^n t_k \vect{P O}
    =
    \sum_{k=1}^n t_k \vect{O e_k} + \parens*{ -\sum_{k=1}^n t_k } \vect{O P}.
  \end{equation*}

  Since the vectors \( \vect{O e_1}, \cdots, \vect{O e_n}, \vect{O P} \) are linearly independent, we conclude that
  \begin{equation*}
    t_1 = \cdots = t_n = 0.
  \end{equation*}

  Thus, the set
  \begin{equation*}
    \set{ \vect{P b} \given e \in E \T{and} e \neq P }
  \end{equation*}
  is linearly independent.

  \ImplicationSubProof{def:affine_dependence/hull}{def:affine_dependence/combinations} Suppose that the affine hull of \( E \) strictly contains the affine hull of any subset.

  Let
  \begin{align*}
    \sum_{k=1}^n t_k = 0,
    &&
    \sum_{k=1}^n t_k \vect{O e_k} = \vect 0.
  \end{align*}

  Suppose that \( t_{k_0} \) is nonzero for some index \( k_0 \). Then
  \begin{equation*}
    t_{k_0} = - \sum_{k \neq k_0} t_k
  \end{equation*}
  and
  \begin{equation*}
    \vect{O e_{k_0}} = -\sum_{k \neq k_0} \frac {t_k} {t_{k_0}} \vect{O e_k}.
  \end{equation*}

  Thus, \( \vect{O e_{k_0}} \) is an affine combination of other vectors from \( E \), and hence the affine hull of \( E \setminus \set{ e_{k_0} } \) coincides with the affine hull of \( E \). This contradicts our initial assumption.

  Therefore,
  \begin{equation*}
    t_1 = \cdots = t_n = 0.
  \end{equation*}

  \ImplicationSubProof{def:affine_dependence/combinations}{def:affine_dependence/linear} Suppose that \fullref{def:affine_dependence/combinations} holds for \( E \). Fix some point \( O \) from \( E \). We will show that the set
  \begin{equation*}
    \set{ \vect{O e} \given e \in E \T{and} e \neq O }
  \end{equation*}
  is linearly independent.

  Aiming at a contradiction, suppose that
  \begin{equation*}
    \sum_{k=1}^n t_k \vect{O e_k} = \vect 0.
  \end{equation*}

  Let
  \begin{equation*}
    T \coloneqq \sum_{k=1}^n t_k.
  \end{equation*}

  Suppose that \( T \) is nonzero. Then
  \begin{equation*}
    \sum_{k=1}^n t_k \vect{O e_k} + (-T) \vect 0 = \vect 0.
  \end{equation*}

  The coefficients of this combination sum to \( 0 \). Therefore, by our assumption \fullref{def:affine_dependence/combinations}, they are all equal to zero. But the last coefficient is \( -T \), which we have assumed is nonzero.

  The obtained contradiction shows that \( T = 0 \). Then again from \fullref{def:affine_dependence/combinations}, we have
  \begin{equation*}
    t_1 = \cdots = t_n = 0.
  \end{equation*}

  Therefore,
  \begin{equation*}
    \set{ \vect{O e} \given e \in E \T{and} e \neq O }
  \end{equation*}
  is a linearly independent set of vectors.

  \ImplicationSubProof{def:affine_dependence/linear}{def:affine_dependence/hull} Let \( A \) be a strict subset of \( E \). Fix some point \( O \) from \( E \setminus A \) and suppose that the set
  \begin{equation*}
    \set{ \vect{O e} \given b \in E \T{and} b \neq O }
  \end{equation*}
  is linearly independent. We will show that \( O \) is not in the affine hull of \( A \).

  Aiming at a contradiction, suppose that \( O \) is an affine combination of members of \( A \):
  \begin{equation*}
    O = \sum_{k=1}^n t_k a_k.
  \end{equation*}

  We have
  \begin{equation*}
    \vect 0
    =
    \vect{O O} - \sum_{k=1}^n t_k \vect {O a_k}
    =
    \sum_{k=1}^n t_k \vect{O a_k},
  \end{equation*}
  which, due to linear independence, implies that
  \begin{equation*}
    t_1 = \cdots = t_n = 0.
  \end{equation*}

  Therefore, \( O \) is not an affine combination of members of \( A \) since the coefficients do not sum to \( 1 \).
\end{proof}

\begin{proposition}\label{thm:linear_and_affine_bases}
  Let \( (A, \vect A, \tau) \) be an \hyperref[def:affine_space]{affine space}. Let \( E \) be a set of points with a fixed origin point \( O \) and let
  \begin{equation*}
    \vect E \coloneqq \set{ \vect{Oe} \given e \in E }.
  \end{equation*}

  \begin{thmenum}
    \thmitem{thm:linear_and_affine_bases/dependence} The vectors in \( \vect E \) are \hyperref[def:linear_dependence]{linearly independent} if and only if the points in \( E \) are \hyperref[def:affine_dependence]{affinely independent}.

    \thmitem{thm:linear_and_affine_bases/hulls} The \hyperref[def:semimodule/submodel]{linear span} of \( \vect E \) is \( \vect A \) if and only if the \hyperref[def:affine_hull]{affine hull} of \( E \) is \( A \).
  \end{thmenum}
\end{proposition}
\begin{proof}
  \SubProofOf{thm:linear_and_affine_bases/dependence} This is the statement of \fullref{def:affine_dependence/linear}.

  \SubProofOf{thm:linear_and_affine_bases/hulls}
  \SufficiencySubProof* Suppose that the linear span of \( \vect E \) is \( \vect A \). Fix some point \( x \) in \( A \). Then \( \vect{Ox} \) is a linear combination of members of \( \vect E \):
  \begin{equation*}
    \vect{Ox} = \sum_{k=1}^n t_k \vect{O x_k}.
  \end{equation*}

  Then
  \begin{equation*}
    x = \parens*{ 1 - \sum_{k=1}^n t_k } O + \sum_{k=1}^n t_k x_k.
  \end{equation*}

  \NecessitySubProof* Suppose that the affine hull of \( E \) is \( A \). Let \( v \) be some vector in \( \vect A \) and let \( x \coloneqq \tau_v(O) \). We have
  \begin{equation*}
    x = \sum_{k=1}^n t_k e_k
  \end{equation*}
  for some points \( e_1, \ldots, e_n \) from \( E \) and some scalars that sum to one. Then
  \begin{equation*}
    v = \vect{Ox} = \sum_{k=1}^n t_k \vect{O e_k}.
  \end{equation*}

  We have shown that any vector in \( \vect A \) is a linear combination of vectors in \( \vect E \).
\end{proof}

\begin{definition}\label{def:barycentric_coordinate_system}
  Let \( (A, \vect A, \tau) \) be an \hyperref[def:affine_space]{affine space} over \( \BbbK \), let \( E \) be a set of \hyperref[def:affine_dependence]{affinely independent} whose \hyperref[def:affine_hull]{affine hull} is \( A \). Let \( O \) be a distinguished origin point from \( E \). We call the pair \( (O, E) \) a \term{barycentric coordinate system} of \( A \) with \term{origin} \( O \) and \term{basis} \( E \). We call the points in \( E \) the \term{basis points} of the coordinate system.

  \Fullref{thm:linear_and_affine_bases} implies that the set
  \begin{equation}\label{eq:def:barycentric_coordinate_system/linear_basis}
    \vect E \coloneqq \set{ \vect{Oe} \given e \in E }
  \end{equation}
  is a linear basis of \( \vect A \).

  We associate with the point \( x \) its \term{barycentric coordinates} with respect to the coordinate system:
  \begin{equation*}
    x_e \coloneqq \begin{cases}
      \pi_{\vect{Oe}}(\vect{Ox}) &e \in E \setminus \set{ O }, \\
      1 - \sum_{e \neq O} x_e    &e = O.
    \end{cases}
  \end{equation*}

  We have
  \begin{equation}\label{eq:def:barycentric_coordinate_system/coordinates}
    x = \sum_{e \in E} x_e \cdot e = \parens*{ 1 - \sum_{e \neq O} x_e } \cdot O + \sum_{e \neq O} x_e \cdot e.
  \end{equation}

  This relates to \hyperref[def:affine_coordinate_system]{affine coordinate systems} via \fullref{thm:affine_and_barycentric_coordinate_systems}.
\end{definition}

\begin{proposition}\label{thm:affine_and_barycentric_coordinate_systems}
  Let \( E \) be a set of points in the affine space \( (A, \vect A, \tau) \) and define \( \vect E \) as in \eqref{eq:def:barycentric_coordinate_system/linear_basis}.

  Then the pair \( (O, \vect E) \) is an \hyperref[def:affine_coordinate_system]{affine coordinate system} if and only if \( (O, E) \) is a \hyperref[def:barycentric_coordinate_system]{barycentric coordinate system}. Furthermore, the coordinates are directly related because
  \begin{equation*}
    x = \parens*{ 1 - \sum_{e \neq O} x_e } \cdot O + \sum_{e \neq O} x_e \cdot e
  \end{equation*}
  and
  \begin{equation*}
    \vect{Ox} = \parens*{ 1 - \sum_{e \neq O} x_e } \cdot \underbrace{\vect{OO}}_{\vect 0} + \sum_{e \neq O} x_e \cdot \vect{Oe}.
  \end{equation*}
\end{proposition}
\begin{proof}
  Trivial.
\end{proof}

\begin{definition}\label{def:affine_operator}
  Let \( (A, \vect A, \tau) \) and \( (B, \vect B, \sigma) \) be \hyperref[def:affine_space]{affine spaces}. We say that the function \( {f: \vect A \to W} \) between points is an \term{affine operator} if any of the following equivalent conditions hold:
  \begin{thmenum}
    \thmitem{def:affine_operator/combination} \( f \) preserves \hyperref[def:affine_hull]{affine combinations}. That is, for any scalar \( \lambda \),
    \begin{equation}\label{eq:def:affine_operator/combination}
      f\parens[\Big]{ \lambda x + (1 - \lambda) y } = \lambda f(x) + (1 - \lambda) f(y).
    \end{equation}

    \thmitem{def:affine_operator/translation} For a fixed origin point \( O \), the following function is a linear operator:
    \begin{equation}\label{eq:def:affine_operator/translation/general}
      \begin{aligned}
        &T: \vect A \to W \\
        &T(v) \coloneqq \vect{ f(O), f(\tau_v(O)) }.
      \end{aligned}
    \end{equation}

    It is important to note that \( T \), if it is linear for \( O \), does not depend on the choice of \( O \).

    When \( A = \vect A \) and \( B = \vect B \), this function has the simpler form
    \begin{equation}\label{eq:def:affine_operator/translation/natural}
      T(v) \coloneqq f(v) - f(\vect 0).
    \end{equation}
  \end{thmenum}
\end{definition}
\begin{defproof}
  \SubProofOf{def:affine_operator/translation} Suppose that \( T_O(v) = \vect{ f(O), f(\tau_v(O)) } \) is linear in \( v \). Then
  \begin{balign*}
    T_O(v)
    &=
    \vect{ f(O), f(\tau_v(O)) }
    = \\ &=
    \vect{ f(\tau_{v-v}(O)), f(\tau_v(O)) }
    = \\ &=
    -\vect{ f(\tau_v(O)), f(\tau_{v-v}(O)) }
    = \\ &=
    -\vect{ f(P), f(\tau_{-v}(P)) }
    = \\ &=
    -T_P(-v)
    = \\ &=
    T_P(v).
  \end{balign*}

  Therefore, the choice of point in the definition of \( T(v) \) is irrelevant.

  \ImplicationSubProof{def:affine_operator/combination}{def:affine_operator/translation} Suppose that \eqref{eq:def:affine_operator/combination} holds. Fix some point \( x \).

  \SubProofOf[eq:def:semimodule/homomorphism/additive]{additivity} \Fullref{thm:rem:affine_combinations/vectors_to_points} implies
  \begin{equation*}
    \tau_{u + v}(O)
    =
    \tau_{u + v - 0}(O)
    =
    \tau_u(O) + \tau_v(O) - O.
  \end{equation*}

  Since \( f \) preserves affine combinations, we have
  \begin{equation*}
    f(\tau_{u + v}(O))
    =
    f(\tau_u(O) + \tau_v(O) - O)
    =
    f(\tau_u(O)) + f(\tau_v(O)) - f(O).
  \end{equation*}

  Then
  \begin{balign*}
    T(u + v)
    &=
    \vect{ f(O), f(\tau_{u + v}(O)) }
    = \\ &=
    \vect{ f(O), f(\tau_u(O)) + f(\tau_v(O)) - f(O) }
     = \\ &=
    \vect{ f(O), f(\tau_u(O)) } + \vect{ f(O), f(\tau_v(O)) } - \vect{ f(O), f(O) }
    = \\ &=
    \vect{ f(O), f(\tau_u(O)) } + \vect{ f(O), f(\tau_v(O)) }
    = \\ &=
    T(u) + T(v).
  \end{balign*}

  \SubProofOf[eq:def:semimodule/homomorphism/homogeneity]{homogeneity} Similarly,
  \begin{equation*}
    T(\lambda v)
    =
    \vect{ f(O) , f(\tau_{\lambda v}(O)) }
    =
    \lambda \vect{ f(O) , f(\tau_v(O)) } + (1 - \lambda) \vect{ f(O) , f(O) }
    =
    \lambda T(v).
  \end{equation*}

  \ImplicationSubProof{def:affine_operator/translation}{def:affine_operator/combination} Suppose that \( T \) is linear. Fix two points \( x \) and \( y \), a scalar \( \lambda \) and an origin point \( O \). Let \( z \coloneqq \lambda x + (1 - \lambda) y \) and \( P \coloneqq f(O) \).

  Then
  \begin{equation*}
    \vect{Oz} = \lambda \cdot \vect{Ox} + (1 - \lambda) \cdot \vect{Oy}.
  \end{equation*}

  Thus,
  \begin{equation*}
    T(\vect{Oz}) = \vect{ f(O), f(\tau_{\vect{Oz}}(O)) } = \vect{ P f(z) }.
  \end{equation*}

  On the other hand,
  \begin{equation*}
    T(\vect{Oz})
    =
    \lambda T(\vect{Ox}) + (1 - \lambda) T(\vect{Oy})
    =
    \lambda \vect{ P f(x) } + (1 - \lambda) \vect{ P f(y) }.
  \end{equation*}

  Therefore,
  \begin{equation*}
    f(z) = \lambda f(x) + (1 - \lambda) f(y).
  \end{equation*}
\end{defproof}

\begin{proposition}\label{thm:image_of_affine_operator}
  The image of an \hyperref[def:affine_operator]{affine operator} \( f: A \to B \) is an \hyperref[def:affine_subspace]{affine subspace} of \( B \).
\end{proposition}
\begin{proof}
  Trivial.
\end{proof}

\begin{proposition}\label{thm:affine_operator_fixed_point}
  The point \( x_0 \) is a \hyperref[def:fixed_point]{fixed point} of the \hyperref[def:affine_operator]{affine endofunction} \( f: A \to A \) if and only if
  \begin{equation}\label{eq:thm:affine_operator_fixed_point}
    f(x) = \tau_{T(\vect{x_0 x})}(x_0),
  \end{equation}
  where \( T \) is the linear part of \( f \).
\end{proposition}
\begin{proof}
  \SufficiencySubProof Let \( x_0 \) be a fixed point of \( f(x) \). Then
  \begin{equation*}
    T(v) \coloneqq \vect{x_0,f(\tau_v(x_0)}
  \end{equation*}
  is a linear operator since \( f \) is affine. Thus,
  \begin{equation*}
    T(\vect{x_0 x}) = \vect{x_0,f(x)}.
  \end{equation*}

  Therefore,
  \begin{equation*}
    f(x) = \tau_{\vect{x_0,f(x)}}(x_0) = \tau_{T(\vect{x_0 x})}(x_0)
  \end{equation*}

  \NecessitySubProof Suppose that \eqref{eq:thm:affine_operator_fixed_point} holds. Then
  \begin{equation*}
    f(x_0) = \tau_{T(\vect 0)}(x_0) = x_0.
  \end{equation*}
\end{proof}

\begin{definition}\label{def:category_of_small_affine_spaces}
  Suppose that we are given a \hyperref[def:grothendieck_universe]{Grothendieck universe} \( \mscrU \), which is safe to assume to be the smallest suitable one as explained in \fullref{def:large_and_small_sets}. We describe the \term{category of \( \mscrU \)-small affine spaces} as the following \hyperref[rem:concrete_categories]{concrete category}:

  \begin{itemize}
    \item The \hyperref[def:category/objects]{set of objects} is the set of all \hyperref[def:affine_space]{affine spaces} \( (A, \vect A, \tau) \), whose set of points \( A \) is \( \mscrU \)-small.

    \item The \hyperref[def:category/morphisms]{morphisms} from \( (A, \vect A, \tau) \) to \( (B, \vect B, \sigma) \) are the \hyperref[def:affine_operator]{affine operators} from \( A \) to \( B \).
  \end{itemize}
\end{definition}

\begin{remark}\label{rem:geometric_shape}
  A \term{geometric shape} is an informal notion that refers to certain special sets of points in an affine space. Shapes in two-dimensional spaces are called \term{figures} and shapes in three dimensions are called \term{bodies}.

  Important kinds of shapes include
  \begin{itemize}
    \item Other affine spaces whose points are points of the ambient space.
    \item Affine varieties, defined in \fullref{def:affine_algebraic_set}.
    \item Parametric curves, defined in \fullref{def:parametric_curve}.
  \end{itemize}

  All the above have the concept of dimensions. Unidimensional shapes are called \term{curves}; bidimensional shapes --- \term{surfaces}. In spaces of finite dimension \( n \), shapes of dimension \( n - 1 \) are called \term{hypersurfaces}.

  When two geometric shapes intersect, we say that they are \term{incident}. This relates to \hyperref[def:graph_incidence]{graph incidence} via \hyperref[def:graph_geometric_realization]{geometric realizations}.
\end{remark}

\begin{definition}\label{def:affine_line}\mimprovised
  We say that the set \( L \) of points in an \hyperref[def:affine_space]{affine space} \( (A, \vect A, \tau) \) of dimension at least two is an \term{affine line} or simply \term{line} if any of the following equivalent conditions hold:

  \begin{thmenum}
    \thmitem{def:affine_line/subspace} \( L \) is a unidimensional \hyperref[def:affine_subspace]{affine subspace}.

    \thmitem{def:affine_line/parametric} There exists a point \( o \), called the \term{origin}, and a nonzero vector \( d \), called the \term{directional} vector, such that \( L \) is the image of the \hyperref[def:affine_operator]{affine function}
    \begin{equation}\label{eq:def:affine_line/parametric}
      \begin{aligned}
        &l: \BbbK \to A \\
        &l(t) \coloneqq \tau_{t d}(o).
      \end{aligned}
    \end{equation}

    We refer to the function \( l \) as a \term{parametrization} of \( L \).
  \end{thmenum}
\end{definition}
\begin{defproof}
  \SubProofOf{def:affine_line/parametric} \Fullref{thm:rem:affine_combinations/vectors_to_points} implies that \( l \) satisfies \fullref{def:affine_operator/combination} and is therefore an affine function.

  \ImplicationSubProof{def:affine_line/subspace}{def:affine_line/parametric} Suppose that \( L \) is a unidimensional affine subspace and fix two points \( o \) and \( a \) from \( L \).

  Let \( x \) be any point in \( L \). Since \( L \) is unidimensional, the vectors \( \vect{ox} \) and \( \vect{oa} \) are linearly dependent. Suppose that \( \vect{ox} = t \cdot \vect{oa} \). Define \( d \coloneqq \vect{oa} \). Then
  \begin{equation*}
    x = \tau_{\vect{ox}}(o) = \tau_{td}(o).
  \end{equation*}

  We conclude that \( L \) is the image of the function \( l(t) = \tau_{td}(o) \).

  \ImplicationSubProof{def:affine_line/parametric}{def:affine_line/subspace} Suppose that \( l(t) = \tau_{td}(o) \) is an affine function that \( L = l(\BbbK) \). \Fullref{thm:image_of_affine_operator} implies that \( L \) is an affine subspace of \( A \).

  The points \( o \), \( l(t_x) \) and \( l(t_y) \) are affinely dependent because, either \( t_y = 0 \) and \( o = l(t_y) \) or
  \begin{equation*}
    \vect{o,l(t_x)} = t_x d = \frac {t_x} {t_y} t_y d = \frac {t_x} {t_y} \vect{o,l(t_y)}.
  \end{equation*}

  Therefore, \( L \) is an affine space of dimension at most one. It is nonzero, hence it is also a space of dimension at least one.
\end{defproof}

\begin{definition}\label{def:collinear_points}\mimprovised
  We say that a set of points in an affine space of dimension at least two is \term{collinear} if there exists an \hyperref[def:affine_line]{affine line} that contains them.
\end{definition}

\begin{proposition}\label{thm:pair_of_points_is_collinear}
  Two or fewer distinct points are always collinear. Furthermore, there is exactly one line passing through two distinct points.

  The case of exactly two points is a restatement of Euclid's first postulate (see \cite[7]{Fitzpatrick2008Euclid}):
  \begin{quote}
    Let it have been postulated to draw a straight-line from any point to any point.
  \end{quote}
\end{proposition}
\begin{proof}
  If we have fewer than two points, we may choose additional points so that we have two.

  Given two distinct points \( o \) and \( a \), we simply define \( d \coloneqq \vect{oa} \) and \( l(t) \coloneqq \tau_{td}(o) \). Then \( o = l(0) \) and \( a = l(1) \).
\end{proof}

\begin{proposition}\label{thm:crossing_lines}
  If two lines intersect in more than one point, they coincide.
\end{proposition}
\begin{proof}
  Let \( g \) and \( h \) be two lines, and let \( d \) and \( e \) be directional vectors for them. Suppose that \( x \) and \( y \) are distinct intersection points, and let \( y = x + td \) and \( y = x + re \). Then
  \begin{equation*}
    \vect 0 = y - y = td - er,
  \end{equation*}
  hence \( d \) and \( e \) are linearly dependent. Therefore, \( g \) and \( h \) coincide.
\end{proof}

\begin{definition}\label{def:crossing_lines}
  We say that two lines \term{cross} if they intersect and are not \hyperref[def:affine_parallelism]{parallel}. \Fullref{thm:crossing_lines} implies that they should intersect in exactly one point (because otherwise they would coincide). We call this point the \term{crossing point}.
\end{definition}

\begin{definition}\label{def:transversal_line}\mimprovised
  A \term{transversal line} for two distinct \hyperref[def:affine_line]{affine lines} is a third line that crosses both of them.
\end{definition}

\begin{definition}\label{def:affine_plane}\mimprovised
  Planes are particularly important yet simple \hyperref[rem:geometric_shape]{surfaces}. We say that the set \( \Pi \) of points in an \hyperref[def:affine_space]{affine space} \( (A, \vect A, \tau) \) of dimension at least three is an \term{affine plane} or simply \term{plane} if any of the following equivalent conditions hold:

  \begin{thmenum}
    \thmitem{def:affine_plane/subspace} \( \Pi \) is a two-dimensional \hyperref[def:affine_subspace]{affine subspace}.

    \thmitem{def:affine_plane/parametric} There exists a point \( o \), called the \term{origin}, and linearly independent vectors \( d \) and \( e \), called the \term{directions}, such that \( \Pi \) is the image of the \hyperref[def:affine_operator]{affine function}
    \begin{equation}\label{eq:def:affine_plane/parametric}
      \begin{aligned}
        &\pi: \BbbK^2 \to A \\
        &\pi(t, r) \coloneqq \tau_{t d + r e}(o).
      \end{aligned}
    \end{equation}

    We refer to the function \( \pi \) as a \term{parametrization} of \( L \).
  \end{thmenum}
\end{definition}
\begin{defproof}
  \SubProofOf{def:affine_plane/parametric} \Fullref{thm:rem:affine_combinations/vectors_to_points} implies that \( \pi \) satisfies \fullref{def:affine_operator/combination} and is therefore an affine function.

  \ImplicationSubProof{def:affine_plane/subspace}{def:affine_plane/parametric} Suppose that \( \Pi \) is a two-dimensional affine subspace and fix three points \( o \), \( a \) and \( b \) from \( \Pi \).

  Let \( x \) be any point in \( \Pi \). Since \( \Pi \) is two-dimensional, the vectors \( \vect{ox} \), \( \vect{oa} \) and \( \vect{ob} \) are linearly dependent. Suppose that \( \vect{ox} = t \cdot \vect{oa} + r \cdot \vect{ob} \). Define \( d \coloneqq \vect{oa} \) and \( e \coloneqq \vect{ob} \). Then
  \begin{equation*}
    x = \tau_{\vect{ox}}(o) = \tau_{td + re}(o).
  \end{equation*}

  We conclude that \( \Pi \) is the image of the function \( \pi(t) \coloneqq \tau_{td + re}(o) \).

  \ImplicationSubProof{def:affine_plane/parametric}{def:affine_plane/subspace} Suppose that \( \pi(t, r) = \tau_{td + re}(o) \) is an affine function such that \( \Pi = \pi(\BbbK, \BbbK) \). \Fullref{thm:image_of_affine_operator} implies that \( \Pi \) is an affine subspace of \( A \). Furthermore, the points \( o \), \( \pi(0, 1) \) and \( \pi(1, 0) \) are affinely independent, hence \( \Pi \) has dimension at least two.

  We will show that \( \Pi \) has dimension at most two by demonstrating that the points \( o \), \( \pi(t_x, r_x) \), \( \pi(t_y, r_y) \) and \( \pi(t_z, r_z) \) are affinely dependent. Consider the matrix
  \begin{equation*}
    \begin{pmatrix}
      t_y & t_z \\
      r_y & r_z
    \end{pmatrix}.
  \end{equation*}

  Its determinant is \( t_y r_z - t_z r_y \).

  \begin{itemize}
    \item If \( t_y r_z - t_z r_y = 0 \), then the points \( \pi(t_y, r_y) \) and \( \pi(t_z, r_z) \) are affinely dependent. We have several possibilities
    \begin{itemize}
      \item If \( t_y = r_y = 0 \), then \( \pi(t_y, r_y) = o \).

      \item If \( t_y = 0 \) and \( r_y \neq 0 \), then from \( t_z r_y = 0 \) it follows that \( t_z = 0 \).

      But \( r_z = \frac {r_z} {r_y} r_y \). Hence,
      \begin{equation*}
        \pi(t_z, r_z) = \tau_{r_z / r_y}(\pi(t_y, r_y)).
      \end{equation*}

      \item If \( t_y \neq 0 \), then
      \begin{equation*}
        t_y r_z = t_y \frac {t_z} {t_y} r_y,
      \end{equation*}
      which implies that \( r_z = t_z / t_y r_y \). Hence,
      \begin{equation*}
        \pi(t_z, r_z) = \tau_{t_z / t_y}(\pi(t_y, r_y)).
      \end{equation*}
    \end{itemize}

    \item Otherwise, we utilize \fullref{thm:inverse_of_2x2_matrix} to solve the \hyperref[rem:system_of_equations]{system of equations}
    \begin{equation*}
      \begin{pmatrix}
        t_y & t_z \\
        r_y & r_z
      \end{pmatrix}
      \begin{pmatrix}
        \lambda \\
        \mu
      \end{pmatrix}
      =
      \begin{pmatrix}
        t_x \\ r_x
      \end{pmatrix}.
    \end{equation*}

    \begin{equation*}
      \begin{pmatrix}
        \lambda \\
        \mu
      \end{pmatrix}
      =
      \frac 1 {t_y r_z - t_z r_y}
      \begin{pmatrix}
        r_z  & -t_z \\
        -r_y & t_y
      \end{pmatrix}
      \begin{pmatrix}
        t_x \\ r_x
      \end{pmatrix}
      =
      \frac 1 {t_y r_z - t_z r_y}
      \begin{pmatrix}
        t_x r_z - t_z r_x \\
        t_y r_x - t_x r_y
      \end{pmatrix}
    \end{equation*}

    Therefore,
    \begin{equation*}
      t_x d + r_x e
      =
      \frac {t_x r_z - t_z r_x} {t_y r_z - t_z r_y} (t_y d + r_y e) + \frac {t_y r_x - t_x r_y} {t_y r_z - t_z r_y} (t_z d + r_z e).
    \end{equation*}

    That is, \( \vect{o,\pi(t_x,r_x)} \) is a linear combination of \( \vect{o,\pi(t_y,r_y)} \) and \( \vect{o,\pi(t_z,r_z)} \).
  \end{itemize}
\end{defproof}

\begin{definition}\label{def:coplanar_points}\mimprovised
  We say that a set of points in an affine space of dimension at least three is \term{coplanar} if there exists an \hyperref[def:affine_plane]{affine plane} that contains them.
\end{definition}

\begin{proposition}\label{thm:triple_of_points_is_coplanar}
  Three or fewer distinct points are \hyperref[def:coplanar_points]{coplanar}. Furthermore, there is exactly one plane passing through three non-collinear points.
\end{proposition}
\begin{proof}
  If we have fewer than three points, we may choose additional points so that we have three.

  Given three distinct points \( o \), \( a \) and \( b \), we simply define \( d \coloneqq \vect{oa} \).
  \begin{itemize}
    \item If \( \vect{ob} = \lambda d \), choose \( e \) as any vector linearly independent from \( d \).
    \item Otherwise, let \( e \coloneqq \vect{ob} \).
  \end{itemize}

  Define \( \pi(t, r) \coloneqq \tau_{td + re}(o) \). Then \( o = \pi(0, 0) \) and \( a = \pi(1, 0) \) and, depending on whether \( \vect{ob} = td \), either \( b = \pi(\lambda, 0) \) or \( b = \pi(0, 1) \).
\end{proof}

\begin{proposition}\label{thm:two_lines_are_coplanar}
  Two lines are always coplanar. Furthermore, if the lines are not \hyperref[def:affine_parallelism]{parallel}, there is exactly one plane containing them.
\end{proposition}
\begin{proof}
  Follows from \fullref{thm:triple_of_points_is_coplanar}.
\end{proof}

\begin{definition}\label{def:normal_vector}\mimprovised
  A \term{normal vector} for an \hyperref[def:affine_subspace]{affine subspace} \( L \) is a nonzero vector that is \hyperref[def:orthogonality]{orthogonal} to every vector in the direction \( \vect L \).
\end{definition}

\begin{example}\label{ex:def:normal_vector}
  We list several examples of \hyperref[def:normal_vector]{normal vectors}.

  \begin{thmenum}
    \thmitem{ex:def:normal_vector/full} The space \( \BbbR^n \) as a subspace of itself has no normal vectors, since the orthogonal complement of \( \BbbR^n \) is the trivial subspace, but we explicitly require normal vectors to be nonzero.

    \thmitem{ex:def:normal_vector/empty} Conversely, every vector in \( \BbbR^n \) is normal for the trivial subspace.

    \thmitem{ex:def:normal_vector/vector_subspace} For the vector subspace \( \BbbR^k \) of \( \BbbR^n \), the normal for \( \BbbR^k \) vectors form the \hyperref[def:orthogonal_complement]{orthogonal complement}
    \begin{equation*}
      \BbbR^{n-k} \cong \set{ (0, \ldots, 0, x_{k+1}, \cdots, x_n) \given (x_1, \ldots, x_n) \in \BbbR^n }.
    \end{equation*}
  \end{thmenum}
\end{example}

\begin{definition}\label{def:affine_hyperplane}
  Hyperplanes are particularly important yet simple \hyperref[rem:geometric_shape]{hypersurfaces}. We say that the set \( H \) of points in an \hyperref[def:affine_space]{affine space} \( (A, \vect A, \tau) \) of finite dimension \( n \) is an \term{affine hyperplane} or simply \term{hyperplane} if any of the following equivalent conditions hold:

  \begin{thmenum}
    \thmitem{def:affine_hyperplane/subspace} \( H \) is an \hyperref[def:affine_subspace]{affine subspace} of dimension \( n - 1 \).

    \thmitem{def:affine_hyperplane/functional} There exists a nontrivial \hyperref[def:affine_operator]{affine} \hyperref[rem:functional]{functional} \( h: A \to \BbbK \) such that
    \begin{equation*}
      H = h^{-1}(0) = \set{ x \in A \given h(x) = 0 }.
    \end{equation*}
  \end{thmenum}
\end{definition}
\begin{defproof}
  \ImplicationSubProof{def:affine_hyperplane/subspace}{def:affine_hyperplane/functional} Let \( H \) be an affine subspace of dimension \( n - 1 \). Fix a \hyperref[def:barycentric_coordinate_system]{barycentric coordinate system} \( O e_1 \ldots e_{n-1} \) in \( H \) and let \( e_n \in A \setminus H \) be affinely independent from them. Denoting the decomposition of \( x \) by \( x = x_1 e_1 + \cdots + x_n e_n \), define
  \begin{equation*}
    \begin{aligned}
      &h: A \to \BbbK \\
      &h(x_1, \ldots, x_n) \coloneqq x_n.
    \end{aligned}
  \end{equation*}

  Then \( h^{-1}(0) \) is precisely the set of all points that do not depend on \( e_n \). This is an affine space of dimension \( n - 1 \) that contains \( H \) as a subset; hence \( h^{-1}(0) = H \).

  \ImplicationSubProof{def:affine_hyperplane/functional}{def:affine_hyperplane/subspace} Suppose that \( h: A \to \BbbK \) is an affine functional and let \( H \coloneqq h^{-1}(0) \).

  Since \( h \) is an affine operator, any affine combination in \( H \) is mapped to zero and hence again belongs to \( H \). Thus, \( H \) is an affine subspace of \( A \).

  We must show that \( H \) has dimension \( n - 1 \). First note that, since \( h \) is an affine operator, for a fixed origin point \( O \) the operator \( T(v) \coloneqq h(\tau_v(O)) - h(O) \) is a linear map from \( \vect A \) to \( \BbbK \). The direction \( \vect H \) of \( H \) is the kernel of \( T \).

  Since \( h \) is nontrivial, from \fullref{thm:rank_nullity_theorem} it follows that
  \begin{equation*}
    n = \dim \vect A = \dim \vect H + \dim \img h = \dim \vect H + 1.
  \end{equation*}

  Therefore, the dimension of \( H \) is \( n - 1 \).
\end{defproof}

  \section{Convex sets}\label{sec:convex_sets}

We will denote by \( \BbbK \) either the field of \hyperref[def:real_numbers]{real numbers} or the \hyperref[def:real_numbers]{complex numbers}. Unless otherwise noted, we are working in an \hyperref[def:affine_space]{affine space} \( (A, \vect A, \tau) \) over \( \BbbK \).

\begin{remark}\label{rem:real_field_extensions}
  When speaking about \hyperref[def:vector_space]{vector spaces} or \hyperref[def:affine_space]{affine spaces}, we usually restrict ourselves to vector spaces over \( \BbbR \) or, at most, \( \BbbC \). This restriction is not arbitrary.

  Important concepts like \hyperref[def:geometric_cone]{cones} or \hyperref[def:convex_hull]{convexity} require the field to be an extension of \( \BbbR \), and it just so happens that, by \fullref{thm:fundamental_theorem_of_algebra} and \fullref{thm:no_finite_extensions_of_closed_fields}, the only nontrivial finite \hyperref[def:field/submodel]{field extension} of \( \BbbR \) is \( \BbbC \).

  It is technically possible to work with infinite field extensions, however then we lose the concept of \hyperref[def:inner_product_space]{inner products}, which is another fundamental reasons for working with real or complex vector spaces.

  Considering only \( \BbbR \) and \( \BbbC \) leads to certain concepts being defined for complex vector spaces and then real vector spaces become a special case. This is formalized via \hyperref[def:complexification]{complexification}. For example, inner products are defined in \fullref{def:inner_product_space} differently for real and complex vector spaces, however we can transition between them due to \fullref{thm:complexification_universal_property} and \fullref{thm:complexification_of_symmetric_bilinear_form}.

  Furthermore, complex linear functionals are entirely defined by their real parts as discussed in \fullref{rem:complex_linear_functional}.
\end{remark}

\begin{definition}\label{def:convex_hull}\mimprovised
  We say that an \hyperref[def:affine_combinations]{affine combination} of points or vectors is \term{convex} if all coefficients are nonnegative.

  The \term{convex hull} of the set \( S \) is the set of all convex combinations of members of \( S \). This can be expressed succinctly by saying that, for any number \( \lambda \) in the unit interval and any vectors \( x \) and \( y \), the hull must contain
  \begin{equation}\label{eq:def:convex_hull/combination}
    \lambda x + (1 - \lambda) y.
  \end{equation}

  The convex hull is a \hyperref[def:moore_closure_operator]{Moore closure operator}. A set that coincides with its convex hull is called a \term{convex set}. The geometric interpretation of convex sets is given in \fullref{thm:def:convex_hull/line_segments}. The relationship with affine and conic hulls is discussed in \fullref{thm:affine_and_conic_is_convex}.
\end{definition}
\begin{proof}
  The proof that the convex hull is a closure operator is similar to that for affine hulls in \fullref{def:affine_hull}.
\end{proof}

\begin{definition}\label{def:conic_hull}\mimprovised
  A \term{conic combination} with origin \( O \), points \( x_1, \ldots, x_n \) and \hi{nonnegative} scalars \( t_1, \ldots, t_n \) is the \hyperref[def:affine_combinations]{affine combination}
  \begin{equation}\label{eq:def:conic_hull/points}
    \parens*{ 1 - \sum_{k=1}^n t_k } O + \sum_{k=1}^n t_k x_k.
  \end{equation}

  A conic combination of vectors is, instead, simply a \hyperref[def:linear_combination]{linear combination} with nonnegative coefficients. This relates to \eqref{eq:def:conic_hull/points} as follows:
  \begin{equation*}
    \parens*{ 1 - \sum_{k=1}^n t_k } \vect{OO} + \sum_{k=1}^n t_k \vect{O x_k}.
  \end{equation*}

  The \term{conic hull} of the set \( S \) is the set of all conic combinations of members of \( S \). This is also a \hyperref[def:moore_closure_operator]{Moore closure operator}.

  The geometric interpretation of convex sets is given in \fullref{thm:def:convex_hull/line_segments}. and drawn in \cref{fig:thm:affine_and_conic_is_convex}.
\end{definition}
\begin{proof}
  The proof that the conic hull is a closure operator is similar to that for affine hulls in \fullref{def:affine_hull}.
\end{proof}

\begin{proposition}\label{thm:affine_and_conic_is_convex}
  The \hyperref[def:convex_hull]{convex hull} of a set is the intersection of its \hyperref[def:affine_hull]{affine hull} and its \hyperref[def:conic_hull]{conic hull} with an arbitrary origin not in the affine hull.

  \begin{figure}[!ht]
    \centering
    \includegraphics[page=1]{output/thm__affine_and_conic_is_convex}
    \caption{The \hyperref[def:affine_hull]{affine}, \hyperref[def:conic_hull]{conic} and \hyperref[def:convex_hull]{convex} hulls of three points.}\label{fig:thm:affine_and_conic_is_convex}
  \end{figure}
\end{proposition}
\begin{proof}
  It is clear that the convex hull is a subset of the affine hull and, since the coefficients sum to one, also to the conic hull for any origin point.

  Conversely, pick a point \( x \) from the intersection and consider the conic combination
  \begin{equation*}
    x = \parens*{ 1 - \sum_{k=1}^n t_k s_k } O + \sum_{k=1}^n t_k s_k,
  \end{equation*}
  where \( s_1, \ldots, s_n \) belongs to \( S \).

  Define
  \begin{equation*}
    T \coloneqq \sum_{k=1}^n t_k.
  \end{equation*}

  If \( T \neq 1 \), then
  \begin{equation*}
    x = (1 - T) O + T y
  \end{equation*}
  and
  \begin{equation*}
    O = \frac 1 {1 - T} x - \frac T {1 - T} y.
  \end{equation*}

  Both \( x \) and \( y \) are affine combinations of \( S \), hence \( O \) belongs to the affine hull of \( S \). But this contradicts our choice of \( O \).

  Therefore, \( T \) is necessarily \( 1 \) and \( x \) is a convex combination of members of \( S \).
\end{proof}

\begin{definition}\label{def:geometric_ray}\mimprovised
  A \term{ray} with \term{vertex} point \( O \) and nonzero \term{directional} vector \( d \) is the \hyperref[def:conic_hull]{conic hull} with origin \( O \) of the singleton set \( \set{ \tau_d(O) } \). It is often described, in complete analogy with \fullref{def:affine_line/parametric}, as the \hyperref[def:parametric_curve]{parametric curve}
  \begin{equation*}
    \begin{aligned}
       &r: [0, \infty) \to A \\
       &r(t) \coloneqq \tau_{td}(O).
    \end{aligned}
  \end{equation*}

  \begin{thmenum}
    \thmitem{def:geometric_ray/opposite} We say that the rays \( r(t) = \tau_{t d}(O) \) and \( s(t) = \tau_{t e}(O) \) with a common vertex \( O \) are \term{opposite} if \( d / \norm d = -e / \norm e \).

    \thmitem{def:geometric_ray/unidirectional} We say that the rays \( r(t) = \tau_{t d}(O) \) and \( s(t) = \tau_{t e}(P) \) are \term{unidirectional} if there exist some \hi{positive} scalar \( \lambda \) such that \( d = \lambda e \).
  \end{thmenum}

  \begin{figure}[!ht]
    \centering
    \includegraphics[page=1]{output/def__geometric_ray}
    \caption{Unidirectional and opposite rays.}\label{fig:def:geometric_ray}
  \end{figure}
\end{definition}

\begin{proposition}\label{thm:hyperplane_via_ray}
  For every \hyperref[def:affine_hyperplane]{hyperplane} \( H \) in \( \BbbK^n \) for \( n > 1 \) there exists a \hyperref[def:geometric_ray]{ray} \( r(t) = O + td \) such that
  \begin{equation}\label{eq:thm:hyperplane_via_ray}
    H = \set{ \tau_v(O) \given \inprod v d = 0 }.
  \end{equation}

  It automatically follows that \( d \) is a \hyperref[def:normal_vector]{normal vector} for \( H \).

  Conversely, the set \eqref{eq:thm:hyperplane_via_ray} is a hyperplane for every ray \( r(t) \).
\end{proposition}
\begin{proof}
  \SufficiencySubProof Let \( H \) be a hyperplane. Then the direction \( \vect H \) has dimension \( n - 1 \). Fix some point in \( O \) from \( H \) and some vector \( d \) from the \hyperref[def:orthogonal_complement]{orthogonal complement} of \( \vect H \).

  Then for any point \( x \) in \( H \) distinct from \( O \), we have \( x = \tau_{\vect{Ox}}(O) \). Furthermore, by definition \( \vect{Ox} \) and \( d \) are orthogonal. Then
  \begin{equation*}
    H \subseteq \set{ \tau_v(O) \given \inprod v d = 0 }.
  \end{equation*}

  Conversely, for every vector \( v \) orthogonal to \( d \), i.e. for every vector \( v \) from \( \vect H \), the point \( \tau_v(O) \) belongs to \( H \). It follows that
  \begin{equation*}
    \set{ \tau_v(O) \given \inprod v d = 0 } \subseteq H.
  \end{equation*}

  The equality \eqref{eq:thm:hyperplane_via_ray} now follows.

  \NecessitySubProof Let \( r(t) = O + td \) be some ray. \Fullref{thm:rank_nullity_theorem} implies that \hyperref[def:orthogonal_complement]{orthogonal complement} of \( \linspan\set{ d } \) has dimension \( n - 1 \). Then
  \begin{equation*}
    O + \linspan\set{ d }^\perp
  \end{equation*}
  is a hyperplane.
\end{proof}

\begin{definition}\label{def:geometric_cone}\mcite[20]{Clarke2013OptimalControl}
  A \term{cone} is a union of \hyperref[def:geometric_ray]{rays} with a common vertex.

  \Fullref{thm:def:convex_hull/conic_cone} gives a necessary and sufficient condition for a cone to coincide with its \hyperref[def:conic_hull]{conic hull}. Despite the name, this is not true in general --- a counterexample is presented in \cref{fig:def:geometric_cone}.

  \begin{figure}[!ht]
    \centering
    \includegraphics{output/def__geometric_cone}
    \caption{One cone consisting of two rays and the conic hull of two other rays.}\label{fig:def:geometric_cone}
  \end{figure}
\end{definition}

\begin{definition}\label{def:line_segment}
  A \term{line segment} between the distinct points \( x \) and \( y \) is the \hyperref[def:parametric_curve]{parametric curve}
  \begin{equation*}
    \begin{aligned}
      &s: [0, 1] \to A \\
      &s(t) \coloneqq x + t \vect{xy} = t y + (1 - t) x.
    \end{aligned}
  \end{equation*}

  The \hyperref[def:totally_ordered_set]{total order} on \( [0, 1] \) induces a total order on the image of \( s \). This allows us to use the notation for \hyperref[def:order_interval]{intervals}, i.e. \( [x, y] \), \( [x, y) \), \( (x, y] \) and \( (x, y) \).
\end{definition}

\begin{proposition}\label{thm:def:convex_hull}
  \hyperref[def:convex_hull]{Convex sets} have the following basic properties:

  \begin{thmenum}
    \thmitem{thm:def:convex_hull/line_segments} A set is convex if and only if it contains the entire \hyperref[def:line_segment]{line segment} between any two of its points.

    \thmitem{thm:def:convex_hull/affine_subspace} \hyperref[def:affine_subspace]{Affine subspaces} are convex.

    \thmitem{thm:def:convex_hull/convex_in_subspace} Convex sets in \hyperref[def:affine_subspace]{affine subspaces} are also convex in the ambient space.

    \thmitem{thm:def:convex_hull/conic_cone} A \hyperref[def:geometric_cone]{cone} coincides with its \hyperref[def:convex_hull]{conic hull} if and only if it is a \hyperref[def:convex_hull]{convex set}.

    \thmitem{thm:def:convex_hull/closed_under_intersections} Any nonempty intersection of convex sets is convex.
  \end{thmenum}
\end{proposition}
\begin{proof}
  \SubProofOf{thm:def:convex_hull/line_segments} Trivial.

  \SubProofOf{thm:def:convex_hull/affine_subspace} Convex combinations are affine, and subspaces are closed under affine combinations, hence they are also closed under convex combinations.

  \SubProofOf{thm:def:convex_hull/convex_in_subspace} Trivial.

  \SubProofOf{thm:def:convex_hull/conic_cone}
  \SufficiencySubProof* Trivial since convex combinations are conic.

  \NecessitySubProof* Fix a convex cone \( C \) with vertex \( O \) and a conic combination
  \begin{equation*}
    x \coloneqq \parens*{ 1 - \sum_{k=1}^n t_k } O + \sum_{k=1}^n t_k x_k
  \end{equation*}
  of points in \( C \). Let
  \begin{align*}
    T \coloneqq \sum_{k=1}^n t_k,
    &&
    y \coloneqq \sum_{k=1}^n \frac {t_k} T x_k.
  \end{align*}

  Then
  \begin{equation*}
    x = (1 - T) O + T y
  \end{equation*}
  and
  \begin{equation*}
    \vect{Ox} = (1 - T) \vect{OO} + T \vect{Oy}.
  \end{equation*}

  Since \( y \) is a convex combination of members of \( C \), it itself belongs to \( C \). Since \( C \) is a cone and since \( T \) is nonnegative, \( x = \tau_{T \vect{Oy}}(O) \) also belongs to \( C \).

  \SubProofOf{thm:def:convex_hull/closed_under_intersections} Trivial.
\end{proof}

\begin{definition}\label{def:half_space}\mcite[41]{Clarke2013OptimalControl}
  We again restrict our attention to real affine spaces. Given an affine functional \( f(x) \), its \term{closed half-spaces} are
  \begin{align*}
    H^+ \coloneqq \set{ f(x) \geq 0 },
    &&
    H^- \coloneqq \set{ f(x) \leq 0 }.
  \end{align*}

  The affine functional \( g(x) \coloneqq -f(x) \) defines the same half-spaces, but swaps \( H^+ \) and \( H^- \).

  We say that the half-spaces are separated by the common \hyperref[def:affine_hyperplane]{hyperplane} parameterized by both \( f(x) \) and \( g(x) \). An arbitrary hyperplane induces two half-spaces, although we need additional information to systematically choose a \enquote{positive} and \enquote{negative} half-space.

  In \( \BbbR^2 \), we call them \term{half-planes}.

  \begin{figure}[!ht]
    \centering
    \includegraphics{output/def__half_space__half_plane}
    \caption{Half-planes}\label{fig:def:half_space/half_plane}
  \end{figure}

  If the inequalities are strict, we instead obtain \term{open half-spaces}.
\end{definition}

\begin{proposition}\label{thm:half_spaces_are_convex}
  \hyperref[def:half_space]{Half-spaces} (and half-spaces of subspaces) are \hyperref[def:convex_hull]{convex}.
\end{proposition}
\begin{proof}
  Let \( f \) be an affine functional. If \( f(x) \leq 0 \) and \( f(y) \leq 0 \), then
  \begin{equation*}
    f(\lambda x + (1 - \lambda) y)
    =
    \lambda f(x) + (1 - \lambda) f(y)
    \leq
    \lambda 0 + (1 - \lambda) 0
    =
    0.
  \end{equation*}
\end{proof}

\begin{definition}\label{def:extremal_point}\mcite[def. 3.6]{Gallier2011Geometry}
  We say that a point \( x \) is \term{extremal} for a \hyperref[def:convex_hull]{convex set} \( A \) if any of the following equivalent conditions hold:

  \begin{thmenum}
    \thmitem{def:extremal_point/combination} If \( x \) is a convex combination of points of \( A \), then \( x \) coincides with one of them.

    \thmitem{def:extremal_point/segment} If \( x \) belongs to some segment with endpoints in \( A \), then \( x \) is an endpoint of the segment.

    \thmitem{def:extremal_point/difference} The set \( A \setminus \set{ x } \) is convex.
  \end{thmenum}
\end{definition}
\begin{defproof}
  \ImplicationSubProof{def:extremal_point/segment}{def:extremal_point/combination} Follows easily via \hyperref[con:induction/peano_arithmetic]{natural number induction}.

  \ImplicationSubProof{def:extremal_point/combination}{def:extremal_point/segment} Special case.

  \ImplicationSubProof{def:extremal_point/segment}{def:extremal_point/difference} Suppose that every segment containing \( x \) has \( x \) as an endpoint.

  Let \( y \) and \( z \) be points in \( A \setminus \set{ x } \) and consider the convex combination \( \lambda y + (1 - \lambda) z \).

  \begin{itemize}
    \item If \( \lambda y + (1 - \lambda) z = x \), then by our assumption \( x \) is either \( y \) or \( z \), and hence \( y \) and \( z \) cannot both be points of \( A \setminus \set{ x } \).

    \item Otherwise, if \( \lambda y + (1 - \lambda) z \neq x \), then it belongs to \( A \) because the latter is convex, and to \( A \setminus \set{ x } \) because \( \lambda y + (1 - \lambda) z \neq x \).
  \end{itemize}

  \ImplicationSubProof{def:extremal_point/difference}{def:extremal_point/segment} Suppose that \( A \setminus \set{ x } \) is convex. Also suppose that \( x = \lambda y + (1 - \lambda) z \) for some points \( y \) and \( z \) in \( A \) and some \( \lambda \) in the unit interval.

  If \( 0 < \lambda < 1 \), then \( x \) splits the segment into the half-open segments \( [z, x) \) and \( (x, y] \). Their union does not contain \( x \). But \( z \) and \( y \) belong to \( A \setminus \set{ x } \), hence \( x \) must also belong to \( A \setminus \set{ x } \) in order for \( A \setminus \set{ x } \) to be convex.

  But we have assumed that \( A \setminus \set{ x } \) is convex, hence \( \lambda \) is either \( 0 \) or \( 1 \).
\end{defproof}

\begin{example}\label{ex:def:extremal_point}
  We give several examples of \hyperref[def:extremal_point]{extremal points}:
  \begin{thmenum}
    \thmitem{ex:def:extremal_point/one} Every point \( x \) is extremal for the \hyperref[rem:singleton_sets]{singleton set} \( \set{ x } \) because the empty set is vacuously convex.

    \thmitem{ex:def:extremal_point/segment} The extremal points of a \hyperref[def:line_segment]{line segment} \( [x, y] \) are \( x \) and \( y \). This is a restatement of \fullref{def:extremal_point/segment}.

    \thmitem{ex:def:extremal_point/subspace} An affine subspace \( L \) has no extremal points unless \( \dim L = 0 \).

    Indeed, suppose that \( x \) is extremal. Then, given any point \( y \), define \( z \coloneqq \tau_{2 \vect{yx}}(y) \). Then
    \begin{equation*}
      \vect{yz} = 2 \vect{yx},
    \end{equation*}
    hence
    \begin{equation*}
      \vect{yx} = \frac {\vect{yz}} 2 = \frac {\vect{yy} + \vect{yz}} 2
    \end{equation*}
    and
    \begin{equation*}
      x = \frac {y + z} 2.
    \end{equation*}

    Thus, \( L \setminus \set{ x } \) is not convex, which contradicts our assumption that \( x \) is extremal.

    Therefore, there are no extremal points in \( L \).
  \end{thmenum}
\end{example}

\begin{proposition}\label{thm:extremal_points_of_convex_hull}
  The \hyperref[def:extremal_point]{extremal points} of \( \conv A \) are a subset of \( A \).
\end{proposition}
\begin{proof}
  Let \( x \) be an extremal point of \( \conv A \). Then there exist some points \( x_1, \ldots, x_n \) in \( A \) and convex coefficients \( t_1, \ldots, t_n \) such that
  \begin{equation*}
    x = \sum_{k=1}^n t_k x_k.
  \end{equation*}

  Then, by \fullref{def:extremal_point/combination}, \( x \) coincides with one of \( x_1, \ldots, x_n \). Hence, it belongs to \( A \).
\end{proof}

\begin{definition}\label{def:convex_polytope}\mimprovised
  A \term{convex polytope} is a nonempty intersection of finitely many \hyperref[def:half_space]{half-spaces}.

  We refer to the \hyperref[def:extremal_point]{extremal points} of the polytope as \term{vertices}.
\end{definition}

\begin{definition}\label{def:degenerate_polytope}\mimprovised
\end{definition}

\begin{proposition}\label{thm:def:convex_polytope}
  \hyperref[def:convex_polytope]{Convex polytopes} have the following basic properties:
  \begin{thmenum}
    \thmitem{thm:def:convex_polytope/convex} Convex polytopes are, surprisingly, \hyperref[def:convex_hull]{convex sets}.
    \thmitem{thm:def:convex_polytope/non_collinear_vertices} No three vertices are \hyperref[def:collinear_points]{collinear}.
  \end{thmenum}
\end{proposition}
\begin{proof}
  \SubProofOf{thm:def:convex_polytope/convex} Follows from \fullref{thm:half_spaces_are_convex} and \fullref{thm:def:convex_hull/closed_under_intersections}.

  \SubProofOf{thm:def:convex_polytope/non_collinear_vertices} If three vertices \( x \), \( y \) and \( z \) are collinear, they lie on the same line, and one of them is a convex combination of the other two, making it not an extremal point.
\end{proof}

\begin{definition}\label{def:simplex}\mimprovised
  A \( k \)-\term{simplex} is the \hyperref[def:convex_hull]{convex hull} of \( k + 1 \) \hyperref[def:affine_dependence]{affinely independent} points, which we call vertices. The convex hull of any subset of the vertices is called a \term{face} of the simplex.

  \Fullref{thm:def:simplex/extremal} implies that the \hyperref[def:extremal_point]{extremal points} of a simplex are its vertices, hence it is always possible to uniquely determine the vertices given only the simplex as a set of points. \Fullref{thm:def:simplex/polytope} then implies that this terminology is consistent is that for \hyperref[def:convex_polytope]{convex polytopes}.
\end{definition}

\begin{definition}\label{def:standard_simplex}\mimprovised
  \todo{Define standard simplices}
\end{definition}

\begin{remark}\label{def:simplex_terminology}
\end{remark}

\begin{example}\label{ex:def:simplex}
  We list several examples of \hyperref[def:simplex]{simplices}:
  \begin{thmenum}
    \thmitem{ex:def:simplex/point} A \( 0 \)-simplex is a point.

    Indeed, a single vector has only one possible affine combination --- itself.

    \thmitem{ex:def:simplex/line_segment} A \( 1 \)-simplex is a \hyperref[def:line_segment]{line segment}.

    Indeed, suppose that \( x \) and \( y \) are affinely independent. Then the vector \( \vect{xy} \) must be linearly independent, i.e. nonzero. Thus, \( x \) and \( y \) are affinely independent if and only \( x \neq y \).

    Therefore, given two distinct points, their convex combination is a line segment --- see \fullref{thm:def:convex_hull/line_segments}.

    \thmitem{ex:def:simplex/triangle} A \( 2 \)-simplex is a \hyperref[def:triangle]{triangle} --- this is actually the definition we will use in \fullref{def:triangle}.
  \end{thmenum}
\end{example}

\begin{proposition}\label{thm:def:simplex}
  \hyperref[def:simplex]{Simplices} have the following basic properties:
  \begin{thmenum}
    \thmitem{thm:def:simplex/polytope} Simplices are \hyperref[def:convex_polytope]{convex polytopes}.
    \thmitem{thm:def:simplex/extremal} The \hyperref[def:extremal_point]{extremal points} of a simplex are precisely its vertices.
  \end{thmenum}
\end{proposition}
\begin{proof}
  \SubProofOf{thm:def:simplex/polytope} Let \( \conv\set{ x_0, x_1, \ldots, x_n } \) be an \( n \)-simplex.

  Let \( L \) be the affine subspace generated by \( x_0, \cdots, x_n \). Since they are affinely independent, they form a \hyperref[def:barycentric_coordinate_system]{barycentric coordinate system} for \( L \). Then \( \vect{x_0 x_1}, \cdots, \vect{x_0 x_n} \) is an ordered basis for the direction \( \vect L \).

  \begin{equation*}
    l_k(x) \coloneqq \begin{cases}
      1 - \sum_{i=1}^n \inprod {\vect{x_0 x_i}} {\vect{x_0 x}}, &k = 0, \\
      \inprod {\vect{x_0 x_k}} {\vect{x_0 x}},                  &k > 0
    \end{cases}.
  \end{equation*}

  Then \( l_0, l_1, \ldots, l_n \) are affine functionals that determine the barycentric coordinates of \( x \):
  \begin{equation*}
    x = \sum_{k=0}^n l_k(x) \cdot x_k.
  \end{equation*}

  This combination is convex if \( l_k(x) \geq 0 \) for \( k = 0, \ldots, n \). Therefore,
  \begin{equation*}
    \conv\set{ x_0, x_1, \ldots, x_n } = \bigcap_{k=0}^n \set{ x \in X \given l_k(x) \geq 0 }.
  \end{equation*}

  \SubProofOf{thm:def:simplex/extremal} Let \( \conv\set{ x_0, x_1, \ldots, x_n } \) be an \( n \)-simplex.

  Consider some convex combination
  \begin{equation*}
    x = \sum_{k=0}^n t_k \cdot x_k.
  \end{equation*}

  This is the unique convex combination of \( x_0, \ldots, x_n \) that determines \( x \) because the vertices determine a barycentric coordinate system.

  In order for \( x \) to be extremal, we obtain that \( x \) is one of the vertices. Conversely, if \( x = x_k \), then because of the uniqueness of the coefficients it follows that \( t_k = 1 \) and
  \begin{equation*}
    t_0 = \cdots = t_{k-1} = t_{k+1} = \cdots = t_n = 0.
  \end{equation*}
\end{proof}

\begin{definition}\label{def:polytope}
  \todo{Define polytopes}
\end{definition}

\begin{definition}\label{def:parallelotope}
  \todo{Define parallelotopes}
\end{definition}

\begin{definition}\label{def:axis_aligned_parallelotope}
  \todo{Define axis-aligned parallelotopes}
\end{definition}

\begin{definition}\label{def:polygon}
  \todo{Define polygons}
\end{definition}

\begin{definition}\label{def:regular_polygon}
  \todo{Define polygons}
\end{definition}

\begin{definition}\label{def:parallelotope_terminology}
\end{definition}

\begin{definition}\label{def:hypercube}
  \todo{Define hypercubes}.
\end{definition}

\begin{definition}\label{def:hypercube_terminology}
\end{definition}

\begin{definition}\label{def:unit_hypercube}
  \todo{Define unit hypercubes}
\end{definition}

\begin{proposition}\label{thm:volume_of_parallelotope}
  \todo{Volumes of parallelotopes}
\end{proposition}

  \section{Euclidean spaces}\label{sec:euclidean_spaces}

\begin{definition}\label{def:euclidean_space}\mimprovised
  A \term[ru=Евклидово пространство (\cite[40]{Вулих1973ВещественныйАнализ})]{Euclidean space} of dimension \( n \) is the \hyperref[def:affine_space]{affine space \( \BbbR^n \)} equipped with the \hyperref[def:inner_product_space]{dot product} \( \inprod x y \coloneqq x^T y \), sometimes called the \term{Euclidean inner product}. We call \( \BbbR^2 \) the \term{Euclidean plane}.

  As described in \fullref{rem:structure_hierarchy}, this introduces a standard \hyperref[def:norm]{norm}, \hyperref[def:metric_space]{metric}, \hyperref[def:uniform_space]{uniformity} and \hyperref[def:topological_space]{topology}, which we call the \( n \)-dimensional \term{Euclidean norm} (resp. metric, uniformity or topology).

  Consider the \hyperref[def:sequence_space]{standard basis} \( e_1, \ldots, e_n \). We call the \hyperref[def:geometric_ray]{ray} at the origin with direction \( e_k \) the \( k \)-th \term[ru=координатная ось (\cite[39]{Вулих1973ВещественныйАнализ})]{coordinate axis}.

\end{definition}
\begin{comments}
  \item The term \enquote{Euclidean space} arose as a generalization of the three-dimensional space discussed in \cite{Fitzpatrick2008EuclidsElements}; thus, its definition varies in the literature --- see \fullref{rem:euclidean_space_etymology}.
\end{comments}

\begin{remark}\label{rem:euclidean_space_etymology}
  The term \enquote{Euclidean space} may have different meanings depending on the author.
  \begin{itemize}
    \item It is defined as \( \BbbR^n \) with the Euclidean inner product by
    \incite[\S 2.19]{Rudin1987RealAndComplexAnalysis},
    \incite[example A-7.1]{Rotman2015AdvancedModernAlgebraPart1},
    \incite[xv]{HugWeil2020ConvexGeometry},
    \incite[1]{WheedenZygmund2015MeasureAndIntegral},
    \incite[example 1.1.6]{Tao2022AnalysisII},
    \incite[13]{HillePhillips1996FunctionalAnalysis},
    \incite[11]{Tao2011MeasureTheory},
    \incite[444]{ИльинСадовничийСендов1985АнализТом1},
    \incite[39]{Вулих1973ВещественныйАнализ} and
    \incite[105]{Обрешков1962ВисшаАлгебра}

    \item It is defined as a finite-dimensional \hyperref[def:inner_product_space]{real inner product space} by
    \incite[\S 8.1.1]{Berger1987GeometryI} and
    \incite[121]{Halmos1974VectorSpaces}.

    \item It is defined as a real inner product space, possibly infinite-dimensional, by
    \incite[\S 24.1]{Тыртышников2007ЛинейнаяАлгебра},
    \incite[def. 5.5.1]{Винберг2014КурсАлгебры} and
    \incite[216]{Станилов1974АналитичнаГеометрия}.
  \end{itemize}

  \enquote{The} Euclidean space may also refer to the ambient space used by Euclid in \cite{Fitzpatrick2008EuclidsElements}.
\end{remark}

\begin{definition}\label{def:congruent_shapes}
  We say that two \hyperref[con:geometric_shape]{shapes} in an \hyperref[def:euclidean_space]{Euclidean space} are \term{congruent} if they are \hyperref[def:isometry]{isometric}.
\end{definition}

\begin{proposition}\label{thm:isometry_iff_affine_orthogonal_operator}\mcite[thm. 7.1]{Treil2017LinearAlgebraDoneWrong}
  The endofunction \( f: \BbbR^n \to \BbbR^n \) over the real inner product space \( \BbbR^n \) is an \hyperref[def:isometry]{isometry} if and only if it is an \hyperref[def:affine_operator]{affine operator} whose linear part \( T(x) \coloneqq f(x) - f(\vect 0) \) is \hyperref[def:unitary_operator]{orthogonal}.
\end{proposition}
\begin{proof}
  \SufficiencySubProof Suppose that \( f \) is an isometry.

  First note that
  \begin{equation*}
    \inprod { f(x) } { f(y) } = \inprod x y.
  \end{equation*}

  Indeed, we have
  \begin{equation*}
    \norm{ f(x) - f(y) }^2 = \norm{ f(x) }^2 + \norm{ f(y) }^2 - 2 \inprod { f(x) } { f(y) }
  \end{equation*}
  and
  \begin{equation*}
    \norm{ x - y }^2 = \norm{x}^2 + \norm{y}^2 - 2 \inprod x y.
  \end{equation*}

  Therefore,
  \begin{equation*}
    \inprod { f(x) } { f(y) }
    =
    \frac{ \norm{ f(x) - f(y) }^2 - \norm{ f(x) }^2 - \norm{ f(y) }^2 } 2
    =
    \frac{ \norm{ x - y }^2 - \norm{ x }^2 - \norm{ y }^2 } 2
    =
    \inprod x y.
  \end{equation*}

  Now we can show additivity of \( T \):
  \begin{balign*}
    &\phantom{{}={}}
    \norm{ T(x + y) - T(x) - T(y) }^2
    = \\ &=
    \norm{ f(x + y) - f(0) - f(x) + f(0) - f(y) + f(0) }^2
    = \\ &=
    \norm{ \parens{ f(x + y) - f(x) } - \parens{ f(y) - f(0) } }^2
    = \\ &=
    \norm{ f(x + y) - f(x) }^2 + \norm{ f(y) - f(0) }^2 - 2 \inprod{ f(x + y) - f(x) } { f(y) - f(0) }
    = \\ &=
    \norm{ x + y - x }^2 + \norm{y}^2 - 2 \inprod{x + y} y + 2 \inprod x y - 2 \inprod {x + y} 0 + 2 \inprod x 0
    = \\ &=
    2 \norm{y}^2 - 2 \inprod x y - 2 \norm{y}^2 + 2 \inprod x y.
  \end{balign*}

  The norm is zero, hence \( T(x + y) = T(x) + T(y) \).

  Similarly,
  \begin{balign*}
    &\phantom{{}={}}
    \norm{ T(\lambda x) - \lambda T(x) }^2
    = \\ &=
    \norm{ f(\lambda x) - f(0) - \lambda f(x) + \lambda f(0) }^2
    = \\ &=
    \norm{ f(\lambda x) - f(0) }^2 + \norm{ \lambda f(x) - \lambda f(0) }^2 - 2 \inprod{ f(\lambda x) - f(0) } { \lambda f(x) - \lambda f(0) }
    = \\ &=
    \norm{\lambda x}^2 + \lambda^2 \norm{x}^2 - 2 \lambda \inprod{ f(\lambda x) - f(0) } { f(x) - f(0) }
    = \\ &=
    2 \lambda^2 \norm{x}^2 - 2 \lambda^2 \inprod x x.
  \end{balign*}

  This norm is also zero, hence \( T(\lambda x) = \lambda T(x) \).

  Finally, we must show that \( T \) is a unitary operator:
  \begin{balign*}
    \inprod{ T(x) }{ T(y) }
    &=
    \inprod{ f(x) - f(0) }{ f(y) - f(0) }
    = \\ &=
    \inprod{ f(x) }{ f(y) } - \inprod{ f(x) }{ f(0) } - \inprod{ f(0) }{ f(y) } + \inprod{ f(0) }{ f(0) }
    = \\ &=
    \inprod x y.
  \end{balign*}

  \NecessitySubProof Suppose that \( f(x) = Tx + f_0 \) for some unitary operator \( T \) and some vector \( f_0 \). Then
  \begin{balign*}
    \norm{ f(x) - f(y) }^2
    &=
    \norm{ Tx - f_0 - Ty + f_0 }^2
    = \\ &=
    \norm{ T(x - y) }^2
    = \\ &=
    \inprod{ T(x - y) }{ T(x - y) }
    = \\ &=
    \inprod{ x - y }{ T^{-1} T(x - y) }
    = \\ &=
    \norm{ x - y }^2.
  \end{balign*}
\end{proof}

\begin{lemma}\label{thm:dot_product_and_outer_product}
  For any field \( \BbbK \) and vectors \( x \), \( y \) and \( z \) in \( \BbbK^n \), we have
  \begin{equation*}
    x^T y z = y z^T x.
  \end{equation*}
\end{lemma}
\begin{proof}
  The \( k \)-th coordinate of \( x^T y z \) is \( (x^T y) z_k \).

  The \( k \)-th coordinate of \( y z^T x \) is
  \begin{equation*}
    \sum_{i=1}^n (y_i z_k) x_i
    =
    z_k \sum_{i=1}^n y_i x_i
    =
    (x^T y) z_k.
  \end{equation*}
\end{proof}

\begin{definition}\label{def:rigid_motion}\mimprovised
  A \term{rigid motion} is an \hyperref[def:isometry]{isometry} in an \hyperref[def:euclidean_space]{Euclidean space}. \Fullref{thm:isometry_iff_affine_orthogonal_operator} provides an equivalent characterization: a rigid motion is an affine operator whose linear part is \hyperref[def:unitary_operator]{orthogonal}. Thus, given a rigid motion \( f: \BbbR^n \to \BbbR^n \), there exists an orthogonal operator \( T: \BbbR^n \to \BbbR^n \) and a directional vector \( d \) such that
  \begin{equation*}
    f(x) = Tx + d.
  \end{equation*}

  The following are common rigid motions:
  \begin{thmenum}
    \thmitem{def:rigid_motion/translation} The \term{translation} along the \term{direction} \( d \) is
    \begin{equation*}
      f(x) = x + d.
    \end{equation*}

    \begin{figure}[!ht]
      \hfill
      \includegraphics[align=c]{output/def__rigid_motion__translation__2d}
      \hfill
      \includegraphics[align=c]{output/def__rigid_motion__translation__3d}
      \hfill
      \hfill
      \caption{Translation of the unit square in \( \BbbR^2 \) and unit cube in \( \BbbR^3 \).}\label{fig:def:rigid_motion/translation}
    \end{figure}

    The term \enquote{translation} generalizes to an arbitrary \hyperref[def:semigroup]{semigroup} via
    \begin{equation*}
      f(x) \coloneqq v \cdot x.
    \end{equation*}

    \thmitem{def:rigid_motion/rotation} A \term{rotation} about the point \( O \) is an affine operator with fixed point \( O \) and an \hyperref[def:unitary_operator]{orthogonal} linear part with \hyperref[def:matrix_determinant]{determinant} \( 1 \).

    \begin{figure}[!ht]
      \hfill
      \includegraphics[align=c]{output/def__rigid_motion__rotation__2d}
      \hfill
      \includegraphics[align=c]{output/def__rigid_motion__rotation__3d}
      \hfill
      \hfill
      \caption{Rotation of the unit square in \( \BbbR^2 \) and unit cube in \( \BbbR^3 \).}\label{fig:def:rigid_motion/rotation}
    \end{figure}

    \thmitem{def:rigid_motion/householder_reflection} The \term{Householder reflection} through the \hyperref[def:geometric_ray]{ray} \( r \) with origin \( O \) and unit directional vector \( v \) is
    \begin{equation*}
      f(x) \coloneqq x - 2 \inprod {x - O} v v.
    \end{equation*}

    The vector \( v \) is interpreted as a \hyperref[def:normal_vector]{normal vector} of the \hyperref[def:affine_hyperplane]{affine hypersurface} \( O + \braket{ v }^\perp \).

    This can be expressed in matrix form as
    \begin{equation*}
      f(x)
      =
      x - (2 (x - O)^T v) \cdot v
      \reloset {\ref{thm:dot_product_and_outer_product}} =
      x - 2 v v^T (x - O)
      =
      2 v v^T O + (I_n - 2 v v^T) x.
    \end{equation*}

    \begin{figure}[!ht]
      \hfill
      \includegraphics[align=c]{output/def__rigid_motion__householder_reflection__2d}
      \hfill
      \includegraphics[align=c]{output/def__rigid_motion__householder_reflection__3d}
      \hfill
      \hfill
      \caption{Householder reflection of the unit square in \( \BbbR^2 \) and unit cube in \( \BbbR^3 \).}\label{fig:def:rigid_motion/householder_reflection}
    \end{figure}

    \thmitem{def:rigid_motion/point_reflection} Similarly, the \term{point reflection} through \( O \) is
    \begin{equation*}
      f(x) \coloneqq x - 2\vect{Ox} = 2O - x.
    \end{equation*}

    \begin{figure}[!ht]
      \hfill
      \includegraphics[align=c]{output/def__rigid_motion__point_reflection__2d}
      \hfill
      \includegraphics[align=c]{output/def__rigid_motion__point_reflection__3d}
      \hfill
      \hfill
      \caption{Point reflection of the unit square in \( \BbbR^2 \) and unit cube in \( \BbbR^3 \).}\label{fig:def:rigid_motion/point_reflection}
    \end{figure}
  \end{thmenum}
\end{definition}

\begin{proposition}\label{thm:lebesgue_measure_invariant_under_rigid_motions}
  The \hyperref[def:lebesgue_measure]{Lebesgue measure} of a set is invariant under \hyperref[def:rigid_motion]{rigid motions}.
\end{proposition}
\begin{proof}
  \todo{Prove.}
\end{proof}

\begin{definition}\label{def:argmin_argmax}
  Given a \hyperref[def:function]{plain function} \( f: A \to B \) from a \hyperref[def:set]{plain set} \( A \) to a \hyperref[def:partially_ordered_set]{partially ordered set} \( (B, \leq) \), if, for a unique value \( a_0 \in A \), we have
  \begin{equation*}
    f(a_0) = \min\set{ f(a) \given a \in A },
  \end{equation*}
  we denote this value \( a_0 \) via
  \begin{equation*}
    \argmin_{a \in A} f(a).
  \end{equation*}

  \hyperref[thm:preorder_duality]{Duality} the allows us to define
  \begin{equation*}
    \argmax_{a \in A} f(a).
  \end{equation*}
\end{definition}

\begin{definition}\label{def:orthogonal_projection}
  Let \( L \) be an affine subspace of \( \BbbR^n \). Let \( Oe_1 \cdots e_m \) be an orthogonal \hyperref[def:affine_coordinate_system]{affine coordinate system} of \( L \). The \term{orthogonal projection} or \term{shadow} onto \( L \) is the affine operator
  \begin{equation*}
    \begin{aligned}
      &\pi: \BbbR^n \to L, \\
      &\pi(x) \coloneqq O + \sum_{k=1}^m \inprod {\vect{Ox}} {e_k} \cdot e_k.
    \end{aligned}
  \end{equation*}

  The definition does not depend on the choice of coordinate system. \Fullref{thm:def:orthogonal_projection} lists important properties of orthogonal projections.

  \begin{figure}[!ht]
    \hfill
    \includegraphics[align=c]{output/def__orthogonal_projection}
    \hfill
    \hfill
    \caption{Orthogonal projection of the unit cube in \( \BbbR^3 \) onto an affine plane.}\label{fig:def:orthogonal_projection}
  \end{figure}
\end{definition}
\begin{defproof}
  Let \( O e_1\cdots e_m \) and \( P f_1 \cdots f_m \) be affine coordinate systems in \( L \). Then, for any \( x \) in \( \BbbR^n \),
  \begin{align*}
    \sum_{i=1}^m \inprod {\vect{Px}} {f_i} \cdot f_i
    &=
    \sum_{i=1}^m \inprod {\vect{Px}} {f_i} \cdot \parens*{ \sum_{j=1}^m \inprod {f_i} {e_j} \cdot e_j }
    = \\ &=
    \sum_{j=1}^m \parens*{ \sum_{i=1}^m \inprod {\vect{Px}} {f_i} \inprod {f_i} {e_j} } \cdot e_j
    = \\ &=
    \sum_{j=1}^m \inprod* {\vect{Px}} {\sum_{i=1}^m \inprod {f_i} {e_j} \cdot f_i} \cdot e_j
    = \\ &=
    \sum_{j=1}^m \inprod {\vect{Px}} {e_j} \cdot e_j.
  \end{align*}

  Furthermore,
  \begin{equation*}
    \vect{Px} = \vect{PO} + \vect{Ox}.
  \end{equation*}

  Therefore, since \( \vect{PO} \) is a vector from the direction \( \vect L \),
  \begin{align*}
    P + \sum_{i=1}^m \inprod {\vect{Px}} {f_i} \cdot f_i
    &=
    P + \sum_{j=1}^m \inprod {\vect{Px}} {e_j} \cdot e_j
    = \\ &=
    P + \sum_{j=1}^m \inprod {\vect{PO}} {e_j} \cdot e_j + \sum_{j=1}^m \inprod {\vect{Ox}} {e_j} \cdot e_j
    = \\ &=
    P + \vect{PO} + \sum_{j=1}^m \inprod {\vect{Ox}} {e_j} \cdot e_j.
  \end{align*}
\end{defproof}

\begin{proposition}\label{thm:def:orthogonal_projection}
  The \hyperref[def:orthogonal_projection]{orthogonal projection} \( \pi: \BbbR^n \to L \) has the following basic properties:

  \begin{thmenum}
    \thmitem{thm:def:orthogonal_projection/fixed} Every point in \( L \) is \hyperref[def:function_fixed_point]{fixed} under \( \pi \).

    \thmitem{thm:def:orthogonal_projection/normal} The vector \( \vect{x,\pi(x)} = \pi(x) - x \) is \hyperref[def:normal_vector]{normal} for \( L \), i.e. orthogonal to every vector in the \hyperref[def:affine_subspace]{direction} \( \vect L \).

    \thmitem{thm:def:orthogonal_projection/distance} The point \( \pi(x) \) is the unique point in \( L \) that minimizes the distance to \( x \). That is, that unique point in \( L \) such that
    \begin{equation*}
      \norm{ x - \pi(x) } = \min_{y \in L} \norm{ x - y }.
    \end{equation*}

    This gives an alternative characterization of \( \pi \) as
    \begin{equation}\label{eq:thm:def:orthogonal_projection/distance/argmin}
      \pi(x) \coloneqq \operatorname*{\hyperref[def:argmin_argmax]{argmin}}_{y \in L} \norm{ x - y }.
    \end{equation}
  \end{thmenum}
\end{proposition}
\begin{proof}
  Let \( O e_1 \cdots e_m \) be an orthonormal affine coordinate system in \( L \) and let \( e_{m+1}, \cdots, e_n \) be an expansion of \( e_1, \ldots, e_m \) to an orthonormal basis of \( \BbbR^n \). Let
  \begin{equation*}
    \pi(x) \coloneqq O + \sum_{k=1}^m \inprod {\vect{Ox}} {e_k} \cdot e_k.
  \end{equation*}

  \SubProofOf{thm:def:orthogonal_projection/fixed} For every point \( x \) in \( L \), \( \pi \) simply gives an expansion along the coordinate system \( O e_1 \cdots e_m \).

  \SubProofOf{thm:def:orthogonal_projection/normal} Let \( x \) be a point in \( L \) and let \( v \) be a vector in \( \vect L \). Then
  \begin{equation*}
    v = \sum_{k=1}^m \inprod v {e_k} \cdot e_k
  \end{equation*}
  and
  \begin{equation*}
    \inprod x v
    =
    \sum_{k=1}^m \inprod v {e_k} \cdot \inprod x {e_k}
    =
    \sum_{k=1}^m \inprod x {e_k} \cdot \inprod v {e_k}
    =
    \inprod {\pi(x)} v.
  \end{equation*}

  Therefore,
  \begin{equation*}
    \inprod {\pi(x) - x} v
    =
    \inprod {\pi(x)} v - \inprod x v
    =
    0.
  \end{equation*}

  \SubProofOf{thm:def:orthogonal_projection/distance} From \fullref{thm:def:orthogonal_projection/normal} it follows that, for \( y \in L \),
  \begin{equation*}
    \norm{ x - y }^2
    =
    \norm{ x - \pi(x) + \pi(x) - y }^2
    \reloset {\ref{thm:generalized_pythagorean_theorem}} =
    \norm{ x - \pi(x) }^2 + \norm{ \pi(x) - y }^2.
  \end{equation*}

  Since norms are nonnegative, the minimum along \( y \in L \) is obtained if \( \pi(x) = y \).
\end{proof}

\begin{definition}\label{def:distance_to_subspace}
  We define the \term{distance} \( \op{dist}(x, L) \) between a point \( x \) and an \hyperref[def:affine_subspace]{affine subspace} \( A \) as \( \norm{ x - \pi_L(x) } \).
\end{definition}

\begin{definition}\label{def:equidistant_points}
  We say that a set of points \( A \) is \enquote{equidistant} from \( x \) if every point \( y \) in \( A \) has equal distance \( \norm{y - x} \) from \( x \).
\end{definition}

\begin{proposition}\label{thm:segment_midpoint}
  Let \( x \) and \( y \) be distinct points. The convex combination \( (x + y) / 2 \) is the unique point of the \hyperref[def:line_segment]{segment} \( [x, y] \) that is \hyperref[def:equidistant_points]{equidistant} from both \( x \) and \( y \). We call it the \term{midpoint} of the segment.
\end{proposition}
\begin{proof}
  Let \( z = \lambda x + (1 - \lambda) y \) be a convex combination that is equidistant from \( x \) and from \( y \).

  Then
  \begin{equation*}
    \norm{ z - x }
    =
    \norm{ \lambda x + (1 - \lambda) y - x }
    =
    \norm{ (\lambda - 1) x + (1 - \lambda) y }
    =
    (1 - \lambda) \norm{ y - x }.
  \end{equation*}

  Analogously,
  \begin{equation*}
    \norm{ z - y }
    =
    \norm{ \lambda x + (1 - \lambda) y - y }
    =
    \norm{ \lambda x - \lambda y }
    =
    \lambda \norm{ x - y }.
  \end{equation*}

  From the equidistance assumption,
  \begin{equation*}
    (1 - \lambda) \norm{ y - x } = \lambda \norm{ x - y }.
  \end{equation*}

  Since the points \( x \) and \( y \) are distinct, \( \norm{ y - x } \neq 0 \), hence we can cancel it to obtain
  \begin{equation*}
    1 - \lambda = \lambda,
  \end{equation*}
  implying that \( \lambda = 1 / 2 \).
\end{proof}

  \section{Parametric curves}\label{sec:parametric_curves}

\begin{definition}\label{def:parametric_curve}\mimprovised
  A \term{parametric curve} in a \hyperref[def:topological_space]{topological space} \( X \) is a set \( C \) of points for which there exists a \hyperref[def:global_continuity]{continuous function} \( \gamma: I \to X \), where \( I \) is a nonempty interval in \( \BbbR \) with potentially infinite endpoints, such that \( \gamma(I) = C \).

  Multiple functions correspond to the same curve, the most trivial example being \( t \mapsto \gamma(-t) \). We call any concrete function \( \gamma: I \to X \) a \term{parametrization} of \( C \). We prefer the interval \( I \) to be either \( [0, 1] \), \( (-\infty, 0] \), \( [0, \infty) \) or \( (-\infty, \infty) \).

  \begin{figure}[!ht]
    \centering
    \includegraphics{output/def__parametric_curve}
    \caption{Part of the Lissajous curve parametrized by \( (\sin(2t), \cos(3t)) \) for \( 3 \pi / 4 \leq t \leq 9 \pi / 4 \).}\label{fig:def:parametric_curve}
  \end{figure}
\end{definition}

\begin{definition}\label{def:simple_curve}\mimprovised
  We say that a \hyperref[def:parametric_curve]{parametric curve} is \term{simple} if it has an injective parametrization.
\end{definition}

\begin{example}\label{ex:def:simple_curve}
  \hfill
  \begin{thmenum}
    \thmitem{ex:def:simple_curve/segment} \hyperref[def:line_segment]{Line segments} in \hyperref[def:topological_vector_space]{topological vector spaces} are simple curves. The segment between \( x \) and \( y \) can be parametrized via the bijective function
    \begin{equation*}
      \begin{aligned}
        &\gamma: [0, 1] \to X \\
        &\gamma(t) \coloneqq ty + (1 - t)x.
      \end{aligned}
    \end{equation*}

    \thmitem{ex:def:simple_curve/lissajous} The Lissajous curve
    \begin{equation*}
      \gamma \coloneqq \begin{pmatrix} \sin(2t) \\ \cos(3t) \end{pmatrix}
    \end{equation*}
    for \( 0 \leq t \leq 2\pi \) is not simple. Part of this curve is shown in \cref{fig:def:parametric_curve}.

    To see that it is not simple, note that the curve is zero at both \( t = \pi / 2 \) and \( t = 3\pi / 2 \). For small enough \( \varepsilon > 0 \), the restrictions of \( \gamma(t) \) to an \( \varepsilon \)-neighborhood around \( \pi / 2 \) and around \( 3\pi / 2 \) are simple curves that intersect only at zero --- see \fullref{fig:ex:def:simple_curve/lissajous}. An injective parametrization of \( \gamma \) cannot be continuous because it needs to pass through zero from different directions. But parametrizations are continuous by definition.

    Therefore, \( \gamma \) has no injective parametrization and is thus not a simple curve.

    \begin{figure}[!ht]
      \centering
      \includegraphics{output/ex__def__simple_curve__lissajous}
      \caption{Self-intersection at zero of a Lissajous curve.}\label{fig:ex:def:simple_curve/lissajous}
    \end{figure}
  \end{thmenum}
\end{example}

\begin{example}\label{ex:curve_moving_backwards}
  One thing we wish to avoid is \enquote{moving backwards} during parametrization of a \hyperref[def:parametric_curve]{parametric curve}.

  Consider the \hyperref[def:line_segment]{line segment}
  \begin{equation*}
    \gamma(t) \coloneqq (1 - t) x + ty,
  \end{equation*}
  where \( 0 \leq t \leq 1 \).

  \begin{figure}[!ht]
    \centering
    \includegraphics{output/ex__curve_moving_backwards}
    \caption{A line segment being traversed twice in opposite directions.}\label{fig:ex:curve_moving_backwards}
  \end{figure}

  We can also parametrize it via the function
  \begin{equation*}
    \delta(s) \coloneqq \begin{cases}
      (1 - s)x + sy,       &0 \leq s \leq 1, \\
      (s - 1)x + (2 - s)y, &1 \leq s \leq 2.
    \end{cases}
  \end{equation*}

  In this case, we have the symmetry \( \delta(1 + s) = \delta(1 - s) \) for \( 0 \leq s \leq 1 \). Hence, when regarding the parameter as time, after moving from \( x \) to \( y \), we start \enquote{moving backwards} via the same path.

  This can cause problems with certain definitions. For example, if we regard merely continuous periodic curves as \hyperref[def:closed_curve]{closed}, \( \delta \) can be extended to a periodic function, making the segment a closed curve.

  Fortunately, this can be avoided easily by requiring that \( \delta \) is \hyperref[def:differentiability/frechet]{differentiable}. In our case,
  \begin{equation*}
    \delta'(s) = \begin{cases}
      -x + y, &0 \leq s < 1, \\
      x - y,  &1 < s \leq 2.
    \end{cases}
  \end{equation*}

  But \( \delta \) is not differentiable at \( s = 1 \). On the other hand, \( \gamma \) is differentiable, and it avoids \enquote{moving backwards}.

  Among other pathologies, this leads to \fullref{def:smooth_curve}.
\end{example}

\begin{definition}\label{def:smooth_curve}\mcite[8]{ИвановТужилин2017ДифГеометрия}
  An \( r \) times \term{smooth curve} is a \hyperref[def:parametric_curve]{parametric curve} for which there exists an \( r \) times \hyperref[def:differentiability/frechet]{differentiable} parametrization \( \gamma: I \to \BbbR^n \). If
  \begin{equation*}
    \gamma(t) = \begin{pmatrix}
      \gamma_1(t) \\
      \vdots \\
      \gamma_n(t)
    \end{pmatrix},
  \end{equation*}
  then the \( k \)-th pointwise derivative is at \( t \in I \) is the vector
  \begin{equation*}
    \gamma^{(k)}(t) = \begin{pmatrix}
      \gamma_1^{(k)}(t) \\
      \vdots \\
      \gamma_n^{(k)}(t)
    \end{pmatrix}.
  \end{equation*}

  By \enquote{parametrization of a smooth curve} we implicitly assume one that is \( r \) times differentiable.

  We call the first derivative \( \gamma'(t) \) --- \term{speed vector} or \term{tangent vector} and \( \gamma^\dprime(t) \) --- the \term{acceleration vector}. Both concepts obviously depend on the parametrization.
\end{definition}

\begin{remark}\label{rem:smooth_curve}
  We will use the term \enquote{smooth} in the context of \hyperref[def:smooth_curve]{smooth curves} and implicitly mean \( r \)-times smooth for large enough \( r \). It is possible for us to require the curve to be infinitely differentiable, in which case we write \( r = \infty \).
\end{remark}

\begin{proposition}\label{thm:tangent_vector_dependent}
  Let \( \gamma: I \to \BbbR^n \) and \( \delta: J \to \BbbR^n \) be two parametrizations of the same smooth curve. Let \( \gamma(t_0) = \delta(s_0) \). Furthermore, suppose that the curve does not intersect itself at this point. That is, suppose that \( s_0 \) is the only solution to \( \gamma(t_0) = \delta(s) \).

  Then the speed vectors \( \gamma'(t_0) \) and \( \delta'(s_0) \) are linearly dependent.
\end{proposition}
\begin{proof}
  If either \( \gamma'(t_0) \) or \( \delta'(s_0) \) is the zero vector, they are vacuously linearly dependent.

  Suppose that both are nonzero. \Fullref{thm:inverse_function_theorem} implies that there exist some neighborhoods \( I \) of \( t_0 \) and \( U \) of \( \gamma(t_0) \) in which there exists a continuously differentiable local inverse \( g: U \to I \) of \( \gamma \). Let \( J \) be the preimage of \( U \) under \( \delta \). Then \( t \coloneqq g \bincirc \delta \) is a smooth map from \( J \) to \( I \). Hence, for \( s \in J \),
  \begin{equation*}
    \delta(s) = \gamma(t(s)).
  \end{equation*}

  Consequently,
  \begin{equation*}
    \delta'(s) = \gamma'(t(s)) \cdot t'(s).
  \end{equation*}

  We have assumed that \( s_0 \) is the only parameter for which \( t_0 = t(s_0) \). We conclude that \( \gamma'(t_0) \) and \( \delta'(s_0) \) are linearly dependent.
\end{proof}

\begin{definition}\label{def:tangent_line}
  Let \( \gamma: I \to \BbbR^n \) be a parametrization of a smooth curve. Suppose that the curve does not intersect itself at \( \gamma(t_0) \) (restrict the domain if necessary). Also suppose that \( \gamma'(t_0) \) is not zero.

  Then we define the \term{tangent line} at \( \gamma(t_0) \) as the \hyperref[def:affine_line]{affine line} with origin \( \gamma(t_0) \) and direction vector \( \gamma'(t_0) \).

  \begin{figure}[!ht]
    \centering
    \includegraphics{output/def__tangent_line}
    \caption{Several tangent lines to the same curve at different points}\label{fig:def:tangent_line}
  \end{figure}

  \Fullref{thm:tangent_vector_dependent} implies that tangent lines, unlike tangent vectors, do not depend on the parametrization.

  If the curve intersects itself at \( \gamma(t_0) \), the concept is ambiguous.
\end{definition}

\begin{definition}\label{def:closed_curve}\mimprovised
  We say that the \hyperref[def:smooth_curve]{smooth curve} \( C \) is \term{closed} if it can be parametrized via a smooth \hyperref[def:periodic_function]{periodic function} \( \gamma: \BbbR \to X \).

  Suppose that \( \gamma \) has period \( p \). Then every interval \( [a, a + p) \) is also a parametrization of \( C \), and for convenience we often take the closed interval \( [a, a + p] \). In particular, if \( \gamma: [a, b) \) is a closed curve that is also \hyperref[def:simple_curve]{simple}, we sometimes refer to \( \gamma: [a, b] \to X \) as a \enquote{simple closed curve}. Here \( \gamma(a) = \gamma(b) \) hods for \( a \neq b \) and \( \gamma \) not injective, however the curve admits an injective parametrization obtained by simply removing \( b \) from the domain of \( \gamma \).

  Of course, different periodic parametrizations may have different periods.

  Conversely, if \( \delta: [a, b] \to X \) is some parametrization of \( C \) such that \( \delta(a) = \delta(b) \), then
  \begin{equation*}
    \begin{aligned}
      &\widetilde{\delta}: \BbbR \to X \\
      &\widetilde{\delta}(t) \coloneqq \delta(a + \rem(t - a, b - a)).
    \end{aligned}
  \end{equation*}
  is a differentiable periodic function.

  Consequently, for a general closed curve, we have no obvious choice of parametrization.

  \begin{figure}[!ht]
    \hfill
    \includegraphics{output/def__closed_curve__a}
    \hfill
    \hfill
    \includegraphics{output/def__closed_curve__b}
    \hfill
    \hfill
    \caption{Two different parametrizations of the same closed trefoil curve.}\label{fig:def:closed_curve}
  \end{figure}
\end{definition}

\begin{example}\label{ex:def:closed_curve}
  \hfill
  \begin{thmenum}
    \thmitem{ex:def:closed_curve/segment} \hyperref[def:line_segment]{Line segments} are not closed curves because the canonical parametrization is not periodic. Furthermore, differentiability avoids the pathologies discussed in \fullref{ex:curve_moving_backwards}, which would technically make line segments closed curves.

    \thmitem{ex:def:closed_curve/lissajous} The parametrization
    \begin{equation*}
      \gamma \coloneqq \begin{pmatrix} \sin(2t) \\ \cos(3t) \end{pmatrix}
    \end{equation*}
    of a Lissajous curve is \( 2\pi \)-periodic, hence the curve is closed.
  \end{thmenum}
\end{example}

\begin{example}\label{ex:curve_staying_in_place}
  We have avoided \hyperref[ex:curve_moving_backwards]{\enquote{moving backwards}} by requiring that curves are \hyperref[def:smooth_curve]{smooth}, however we have not avoided \enquote{staying in place}.

  For example, the graph of the absolute value function can be parametrized as follows:
  \begin{equation*}
    \gamma(t) \coloneqq \begin{cases}
      ( t^2     , t^2 )^T,     &-1 \leq t \leq 0, \\
      ( 0       , 0 )^T,       &0 < t < 1, \\
      ( (t-1)^2 , (t-1)^2 )^T, &1 \leq t \leq 2, \\
    \end{cases}
  \end{equation*}

  Its derivative is
  \begin{equation*}
    \gamma'(t) = \begin{cases}
      ( 2t     , 2t )^T,     &-1 \leq t \leq 0, \\
      ( 0      , 0 )^T,      &0 < t < 1, \\
      ( 2(t-1) , 2(t-1) )^T, &1 \leq t \leq 2, \\
    \end{cases}
  \end{equation*}

  Clearly \( \gamma(t) \) is continuously differentiable. We can consider \( t^{2r} \) instead of \( t^2 \) in order to make it \( 2r - 1 \) times continuously differentiable.

  We do have, however, \( \gamma(t_1) = \gamma(t_2) \) for \( 0 \leq t_1 \leq t_2 \leq 1 \). This is one of the things we generally wish to avoid when working with curves. For this reason, \hyperref[def:regular_curve]{regular curves} impose the additional requirement that \( \gamma'(t) \) should be nonzero at any point.
\end{example}

\begin{definition}\label{def:curve_speed}
  Given a parametrization of a \hyperref[def:smooth_curve]{smooth curve}, its \term{speed} at a point is defined as the norm of its speed vector at that point. The concept depends on the parametrization, and for this reason we prefer \hyperref[thm:natural_parametrization_existence]{natural parameters} with constant speed.
\end{definition}

\begin{example}\label{ex:nonregular_curve}
  Consider the smooth curve
  \begin{equation*}
    \gamma(t)
    \coloneqq
    \begin{pmatrix}
      t^3 \\ t^2
    \end{pmatrix},
  \end{equation*}
  where \( t \) is any real number. For any \( t \), the speed vector is
  \begin{equation*}
    \gamma'(t)
    \coloneqq
    \begin{pmatrix}
      3t^2 \\ 2t
    \end{pmatrix}.
  \end{equation*}

  \begin{figure}[!ht]
    \centering
    \includegraphics{output/ex__nonregular_curve}
    \caption{The curve \( (t^3, t^2) \) has a cusp at zero.}\label{fig:ex:nonregular_curve}
  \end{figure}

  The \hyperref[def:curve_speed]{speed} is
  \begin{equation*}
    \norm{\gamma'(t)} = \abs{t} \sqrt{9 t^2 + 4}.
  \end{equation*}

  Fix some \( \varepsilon > 0 \). Then, for any point \( t \) such that \( 0 < 9 t^2 + 4 < \varepsilon^2 \), we have
  \begin{equation*}
    \norm{\gamma'(t)} < \varepsilon.
  \end{equation*}

  Hence, we technically avoid the \enquote{staying in place} discussed in \fullref{ex:curve_staying_in_place}, but the speed of movement becomes arbitrarily slow until it eventually reaches zero.

  In this particular case, the curve does not allow a parametrization with entirely nonzero speed. Indeed, suppose that there exists a parametrization \( \delta: J \to \BbbR^2 \) that has nonzero speed at every point. Let \( s_0 \) be a value for which \( \delta(s_0) = \gamma(0) = (0, 0) \).

  \Fullref{thm:inverse_function_theorem} implies that there exist neighborhoods \( (a, b) \) of \( s_0 \) and \( U \) of \( (0, 0) \) in which \( \delta \) is invertible via some continuously differentiable function \( f: U \to (a, b) \).

  For small enough \( t > 0 \), the point \( \gamma(t) = (t^3, t^2) \) is in \( U \), hence \( a < f(\gamma(t)) < b \). Define
  \begin{equation*}
    s(t) \coloneqq f(\gamma(t)).
  \end{equation*}

  Then
  \begin{equation*}
    s'(t) = f'(\gamma(t)) \cdot \gamma'(t).
  \end{equation*}

  This derivative is zero at \( t = 0 \), implying that \( s(t) \) is constant. But \( \gamma \) and \( f \) are both injective, hence \( s(t) \) should also be injective.

  The obtained contradiction shows that \( \delta \) must have zero speed at \( 0 \).
\end{example}

\begin{definition}\label{def:regular_curve}
  A \hyperref[def:smooth_curve]{smooth curve} is called \term{regular} if it has a parametrization whose speed is always nonzero.
\end{definition}

\begin{example}\label{ex:nonregular_parametrization}\mcite[rem. 1.16]{ИвановТужилин2017ДифГеометрия}
  Consider the \hyperref[def:smooth_curve]{smooth curve}
  \begin{equation*}
    \gamma(t)
    \coloneqq
    \begin{pmatrix}
      t \\ t^2
    \end{pmatrix},
  \end{equation*}
  where \( t \) is any real number. For any \( t \), the speed vector is
  \begin{equation*}
    \gamma'(t)
    \coloneqq
    \begin{pmatrix}
      1 \\ 2t
    \end{pmatrix}.
  \end{equation*}

  It is never the zero vector, hence the curve is \hyperref[def:regular_curve]{regular}.

  We can also reparametrize via \( t(s) = s^3 \) so that
  \begin{equation*}
    \delta(s)
    \coloneqq
    \gamma(t(s))
    =
    \begin{pmatrix}
      s^3 \\ s^6
    \end{pmatrix}.
  \end{equation*}

  Then the derivative
  \begin{equation*}
    \delta'(s)
    =
    \begin{pmatrix}
      3t^2 \\ 6t^5
    \end{pmatrix}
  \end{equation*}
  is the zero vector when \( t = 0 \). Hence, \( \delta \) is not a regular parametrization of a regular curve.

  \begin{figure}[!ht]
    \centering
    \includegraphics{output/ex__nonregular_parametrization}
    \caption{Part of the curve \( (t, t^2) \) from \( t = -1 \) to \( t = 1 \).}\label{fig:ex:nonregular_parametrization}
  \end{figure}
\end{example}

\begin{proposition}\label{thm:natural_parametrization_existence}
  Every \hyperref[def:regular_curve]{regular} \hyperref[def:parametric_curve]{parametric curve} has a parametrization for which the \hyperref[def:curve_speed]{speed} is always \( 1 \). We call such a parametrization a \term{natural parametrization} or, due to \fullref{thm:length_of_smooth_curves}, an \term{arc length parametrization}.
\end{proposition}
\begin{proof}
  Consider the regular curve \( \gamma: I \to \BbbR^n \). Define the function
  \begin{equation*}
    s(t) \coloneqq \int_0^t \norm{\gamma'(\tau)} \cdot \dl \tau
  \end{equation*}
  over the same interval.

  \Fullref{thm:fundamental_theorem_of_calculus} implies that
  \begin{equation*}
    s'(t) = \norm{\gamma'(t)}.
  \end{equation*}

  Since \( \gamma \) is regular, \( t'(s) \) is nonzero. Hence, its inverse has a derivative
  \begin{equation*}
    t'(s) = \frac 1 {\norm{\gamma'(s)}}.
  \end{equation*}

  For \( \delta(s) \coloneqq \gamma(t(s)) \), \fullref{thm:chain_rule} implies
  \begin{equation*}
    \delta'(s)
    =
    t'(s) \cdot \gamma'(s)
    =
    \frac {\gamma'(s)} {\norm{\gamma'(s)}}.
  \end{equation*}

  Since \( t_0 \) was arbitrary, we conclude that this holds for the entire interval \( I \).
\end{proof}

\begin{example}\label{ex:natural_reparametrization_of_line}
  We can demonstrate the trick from \fullref{thm:natural_reparametrization_existence} to reparametrize the line \( \gamma(t) = O + td \). Define
  \begin{equation*}
    s(t) \coloneqq \int_0^t \norm{\gamma'(\tau)} \cdot \dl \tau = t \norm{d}.
  \end{equation*}

  Its inverse is
  \begin{equation*}
    t(s) = \frac s {\norm{d}}.
  \end{equation*}

  Then
  \begin{equation*}
    \gamma(t(s)) = O + s \frac d {\norm{d}},
  \end{equation*}
  and this is obviously a natural reparametrization.
\end{example}

\begin{proposition}\label{thm:natural_reparametrization_existence}
  For two \hyperref[thm:natural_parametrization_existence]{natural parameters} \( t_1: J_1 \to I \) and \( t_2: J_2 \to I \) of \( \gamma: I \to \BbbR^n \), there exists some constant \( a \) such that \( J_2 = a + J_1 \) and, for all \( s \in J_2 \), \( t_2(s) = t_1(a + s) \).
\end{proposition}
\begin{proof}
  Define
  \begin{equation*}
    \delta_1(s) \coloneqq \gamma(t_1(s))
  \end{equation*}
  and similarly for \( \delta_2(s) \).

  We have
  \begin{equation*}
    \delta_1'(s) = t_1'(s) \cdot \gamma'(t_1(s)).
  \end{equation*}

  Since \( \delta_1 \) is natural, it follows that
  \begin{equation*}
    1 = \norm{\delta_1'(s)} = \abs{t_1'(s)} \cdot \norm{\gamma'(s)},
  \end{equation*}
  and similarly for \( \delta_2(s) \)

  Furthermore, since \( t_1 \) and \( t_2 \) are strictly monotone, their derivatives positive. Therefore,
  \begin{equation*}
    t_1'(s) = t_2'(s) = \frac 1 {\norm{\gamma'(s)}}.
  \end{equation*}

  Then \( t_1'(s) - t_2'(s) = 0 \), implying that the different between \( t_1(s) \) and \( t_2(s) \) is a constant.
\end{proof}

\begin{definition}\label{def:regular_curve_curvature}
  If the \hyperref[thm:natural_reparametrization_existence]{naturally parametrized} curve \( \gamma: I \to \BbbR^n \) does not intersect itself at the point \( \gamma(t_0) \), we call the norm \( \norm{\gamma^\dprime(t_0)} \) of the \hyperref[def:smooth_curve]{acceleration vector} the \term{curvature} of \( \gamma \) at \( \gamma(t_0) \). \Fullref{thm:natural_reparametrization_existence} ensures that this definition does not depend on the parametrization, as long as it is natural.

  If the curve intersects itself at \( \gamma(t_0) \), the concept is ambiguous.
\end{definition}

\begin{proposition}\label{thm:line_curvature}
  The \hyperref[def:regular_curve_curvature]{curvature} of the \hyperref[thm:natural_reparametrization_existence]{naturally parametrized} \( \gamma: I \to \BbbR^n \) is always \( 0 \) if and only if \( \gamma(t) = O + td \).

  That is, \( \gamma(t) \) is an
  \begin{itemize}
    \item \hyperref[def:affine_line]{affine line} in case \( I \) is unbounded.
    \item \hyperref[def:geometric_ray]{ray} in case \( I \) is bounded from one side.
    \item \hyperref[def:affine_line]{affine line} in case \( I \) is bounded from both sides.
  \end{itemize}
\end{proposition}
\begin{proof}
  \SufficiencySubProof Suppose that \( \norm{\gamma^\dprime(t)} = 0 \). Then the acceleration vector is zero, implying that the speed vector \( d \coloneqq \gamma'(t) \) is a constant and \( \gamma \) itself is
  \begin{equation*}
    \gamma(t) = O + t d
  \end{equation*}
  for some point \( O \).

  Determining both \( O \) and \( d \) requires additional information.

  \NecessitySubProof Clearly \( \gamma'(t) = d \) and thus \( \gamma^\dprime(t) = \vect 0 \).
\end{proof}

\begin{definition}\label{def:arc_length}
  \begin{figure}[!ht]
    \centering
    \includegraphics[page=1]{output/def__parametric_curve_length__approximation}
    \caption{A rough approximation of a curve using three line segments.}\label{def:arc_length/approximation}.
  \end{figure}

  Let \( X \) be a \hyperref[def:banach_space]{Banach space}, let \( C \) be a \hyperref[def:parametric_curve]{parametric curve} and let \( \gamma: [a, b] \to X \) be some parametrization of \( C \).

  For each \hyperref[def:riemann_partition/tagged]{tagged Riemann partition}
  \begin{equation*}
    \begin{aligned}
      &\Delta: a = t_0 < t_1 < \ldots < t_n = b \\
      &\Tau: \tau_k \in [t_{k-1}, t_k], k = 1, \ldots, n,
    \end{aligned}
  \end{equation*}
  assign
  \begin{equation*}
    \len(\gamma, \Delta, \Tau) \coloneqq \sum_{k=1}^n \norm{\gamma(\tau_k) - \gamma(\tau_{k-1})}.
  \end{equation*}

  If the limit over all tagged partition with respect to the order \fullref{def:riemann_partition/order/diameter} exists, we call it the \term{arc length} of \( C \) with respect to \( \gamma \) and say that \( C \) is \term{rectifiable}.

  This concept unfortunately depends on the parametrization in the general case.
\end{definition}

\begin{proposition}\label{thm:length_of_smooth_curves}
  For a \hyperref[def:smooth_curve]{smooth curve} \( C \) with parametrization \( \gamma: [a, b] \to \BbbR^n \), we have
  \begin{equation*}
    \len(\gamma) = \int_a^b \norm{\gamma'(t)} \dl t.
  \end{equation*}
\end{proposition}
\begin{proof}
  By \fullref{thm:lagranges_mean_value_theorem}, given a Riemann partition
  \begin{equation*}
    \Delta: a = t_0 < \cdots < t_n = b,
  \end{equation*}
  for each \( k = 1, \ldots, n \) there exists a point \( \tau_k \in [t_{k-1}, t_k] \) such that
  \begin{equation*}
    \gamma(t_k) - \gamma(t_{k-1}) = \gamma'(\tau_k) (t_k - t_{k-1}).
  \end{equation*}

  Then
  \begin{equation*}
    \len(\gamma, \Delta, \Tau)
    =
    \sum_{k=1}^n \norm{\gamma(t_k) - \gamma(t_{k-1})}
    =
    \sum_{k=1}^n \norm{\gamma'(\tau_k)} (t_k - t_{k-1}),
  \end{equation*}
  which is a standard Riemann sum for the function \( t \mapsto \norm{\gamma'(t)} \). Therefore, \fullref{thm:countinuous_functions_integrable} implies that the curve is rectifiable and
  \begin{equation*}
    \len(\gamma) = \int_a^b \norm{\gamma'(t)} \dl t.
  \end{equation*}
\end{proof}

\begin{corollary}\label{thm:length_of_piecewise_smooth_curves}
  The \hyperref[def:arc_length]{arc length} of a piecewise smooth curve is the sum of the lengths of its components.
\end{corollary}

\begin{corollary}\label{thm:length_of_function_graph}
  The \hyperref[def:arc_length]{length} of the \hyperref[def:set_valued_map/graph]{graph} of a \hyperref[def:differentiability/frechet]{differentiable} function \( f: [a, b] \to \BbbR \), if it exists, is given by
  \begin{equation*}
    \len(\gph(f)) \coloneqq \int_a^b \sqrt{1 + [f'(x)]^2} \dl x.
  \end{equation*}
\end{corollary}
\begin{proof}
  Apply \fullref{thm:length_of_smooth_curves} for the parametric curve \( t \mapsto (t, f(t)) \).
\end{proof}

\begin{definition}\label{def:perimeter}
  Consider a \hyperref[con:geometric_shape]{geometric shape}, i.e. a set of points in an Euclidean space. Suppose that the \hyperref[def:topological_boundary_operator]{boundary} of the shape is a \hyperref[def:parametric_curve]{parametric curve}. Suppose also that the curve is \hyperref[def:arc_length]{rectifiable}.

  In this case, we call the arc length of this curve the \term{perimeter} of the shape.
\end{definition}

  \section{Euclidean plane}\label{sec:euclidean_plane}

\begin{definition}\label{def:euclidean_plane}
  We call the two-dimensional \hyperref[def:euclidean_space]{Euclidean space} \( \BbbR^2 \) the \term{Euclidean plane}.
\end{definition}

\begin{remark}\label{rem:euclidean_plane_embedding}
  Every \hyperref[def:affine_plane]{affine plane} in \( \BbbR^n \) is isomorphic to \( \BbbR^2 \), and we can use the concepts from this subsection as long as we fix a plane.
\end{remark}

\begin{remark}\label{rem:xyz}
  By convention, depending on the context, in \hyperref[def:euclidean_plane]{Euclidean planes} the letters \( x \) and \( y \) have several meanings:
  \begin{itemize}
    \item The vectors of the \hyperref[def:coordinate_space]{standard basis}.
    \item The corresponding \hyperref[def:euclidean_plane]{coordinate axes}.
    \item The \hyperref[def:affine_coordinate_system]{(affine) coordinates} of some arbitrary point.
  \end{itemize}

  We call the axis \( x \) the \term{abscissa} and \( y \) --- the \term{ordinate}.

  In three-dimensional \hyperref[def:euclidean_space]{Euclidean spaces}, we use the letter \( z \) to denote the third coordinate axis and call it the \term{applicata}.

  To avoid confusion, we avoid using \( x \), \( y \) and \( z \) in Euclidean planes to denote points and vectors.
\end{remark}

\begin{definition}\label{def:plane_line_equations}
  \hyperref[def:affine_line]{Lines} in \( \BbbR^2 \) are so ubiquitous that they are often represented via a variety of \hyperref[def:equation]{equations}.

  We will start with \cref{def:affine_line/parametric} --- the \hyperref[def:affine_operator]{affine} \hyperref[def:parametric_curve]{parametric curve}
  \begin{equation}\label{eq:def:plane_line_equations/parametric}
    l(t) = O + td
  \end{equation}

  \begin{figure}[!ht]
    \centering
    \includegraphics[align=c]{output/thm__plane_line_equations__cartessian}
    \caption{A \hyperref[def:affine_line]{line} in \( \BbbR^2 \) defined using its \hyperref[def:plane_line_equations/cartesian]{Cartesian equation}.}\label{fig:def:plane_line_equations/cartesian}
  \end{figure}

  \begin{thmenum}
    \thmitem{def:plane_line_equations/vector_parametric} We call \eqref{eq:def:plane_line_equations/parametric} a \term{vector parametric equation} of \( L \).

    \medspace

    \thmitem{def:plane_line_equations/scalar_parametric} The parametric equation \eqref{eq:def:plane_line_equations/parametric} can be rewritten as
    \begin{equation}\label{eq:def:plane_line_equations/scalar_parametric}
      \begin{cases}
         &l_x(t) = x_o + t x_d, \\
         &l_y(t) = y_o + t y_d.
      \end{cases}
    \end{equation}

    We say that these are \term{scalar parametric equations} of the line. They are non-unique by the same reason as the vector parametric equation.

    \thmitem{def:plane_line_equations/general}\mcite[sec. 9.9]{Тыртышников2007ЛинейнаяАлгебра} The image of the scalar equations \eqref{eq:def:plane_line_equations/scalar_parametric} consists of all pairs \( (x, y) \) such that
    \begin{equation}\label{eq:def:plane_line_equations/general}
      \underbrace{ Ax + By + C }_{ p(x, y) } = 0
    \end{equation}
    for some scalars \( A \), \( B \) and \( C \), where \( A \) or \( B \) (or both) are nonzero.
    We call \eqref{eq:def:plane_line_equations/general} a \term{general equation} of the line.

    More concretely,
    \begin{equation*}
      A(o_x + td_x) + B(o_y + td_y) + C = 0
    \end{equation*}
    for all \( t \) if \( A = d_y \), \( B = -d_x \) and \( C = o_y d_x - o_x d_y \).

    Conversely, given the general equation \eqref{eq:def:plane_line_equations/general}, assuming \( A \neq 0 \), we can define the parametric equations
    \begin{equation*}
      \begin{cases}
        &l_x(t) \coloneqq -\tfrac C A - t \tfrac B A  \\
        &l_y(t) \coloneqq t.
      \end{cases}
    \end{equation*}

    The case when \( A = 0 \) and \( B \neq 0 \) is handled analogously.

    Note that multiple general equations can have the same locus --- actually all scalar multiples of \( p(x, y) \). If \( A^2 + B^2 = 1 \) in \eqref{eq:def:plane_line_equations/general}, we call it a \term{normal equation}. There are only two normal equations.

    \thmitem{def:plane_line_equations/cartesian} It is common, especially in analysis, to use the \term{Cartesian equation}
    \begin{equation}\label{eq:def:plane_line_equations/cartesian}
      y = kx + m
    \end{equation}
    for some scalars \( k \) and \( m \). We call \( k \) the \term{slope} of the line.

    It is a special case of the general equation \eqref{eq:def:plane_line_equations/general} with \( A = -k \), \( B = -1 \) and \( C = m \).

    Unlike the general equation, the Cartesian equation of a line is unique, but it cannot express vertical lines. If \( B \neq 0 \) in \eqref{eq:def:plane_line_equations/general}, we can define \( k = -A / B \) and \( m = -C / B \) to form a Cartesian equation.

    \thmitem{def:plane_line_equations/intercept} Another equation that is occasionally used is the \term{intercept equation}
    \begin{equation}\label{eq:def:plane_line_equations/intercept}
      \frac x a + \frac y b = 1
    \end{equation}
    for some nonzero real numbers \( a \) and \( b \).

    It is again a special case of the general equation \eqref{eq:def:plane_line_equations/general} with \( A = 1 / a \), \( B = 1 / b \) and \( C = -1 \). It is also unique, but it cannot express neither vertical nor horizontal lines, nor lines passing through the origin.

    Conversely, if \( A \), \( B \) and \( C \) are all nonzero in the general equation \eqref{eq:def:plane_line_equations/general}, we can define an intercept equation as \( a = -C / A \) and \( b = -C / B \).
  \end{thmenum}
\end{definition}

\begin{proposition}\label{thm:coordinates_of_directional_vector}
  If \( A x + B y + C = 0 \) is a \hyperref[def:plane_line_equations/general]{general equation} of some line, then \( (x_0, y_0) \) is a directional vector of the line if and only if
  \begin{equation*}
    A x_0 + B y_0 = 0.
  \end{equation*}
\end{proposition}
\begin{proof}
  \SufficiencySubProof Suppose that \( (x_0, y_0) \) is a directional vector, and let \( (x_1, y_1) \) be the coordinates of some point on the line. Then \( (x_0 + x_1, y_0 + y_1) \) is also a point, and
  \begin{equation*}
    A x_0 + B y_0 + C = 0 = A (x_0 + x_1) + B (y_0 + y_1) + C.
  \end{equation*}

  Then \( A x_0 + B y_0 = 0 \).

  \NecessitySubProof Suppose that \( A x_0 + B y_0 = 0 \). Let \( (x_1, y_1) \) and be a point on the line. Then
  \begin{equation*}
    A x_1 + B y_1 + C = 0 = A x_0 + B y_0.
  \end{equation*}
  and
  \begin{equation*}
    A (x_1 + t x_0) + B (y_1 + t y_0) + C = A x_1 + B y_1 + C + t (A x_0 + B y_0) = 0.
  \end{equation*}

  That is, \( (x_0, y_0) \) is a directional vector of the line.
\end{proof}

\begin{proposition}\label{thm:parallel_lines_in_plane}
  Let \( A_g x + B_g y + C_g = 0 \) and \( A_h x + B_h y + C_h = 0 \) be the \hyperref[def:plane_line_equations/general]{general equations} of the lines \( g \) and \( h \). The lines are \hyperref[def:affine_parallelism]{parallel} if and only if the vectors \( (A_g, B_g) \) and \( (A_h, B_h) \) are linearly dependent, and they coincide if and only if \( (A_g, B_g, C_g) \) and \( (A_h, B_h, C_h) \) are linearly dependent.
\end{proposition}
\begin{proof}
  \SufficiencySubProof Suppose that \( g \) and \( h \) are parallel, and let \( (x, y) \) be a directional vector for them. Then \Cref{thm:coordinates_of_directional_vector} implies that
  \begin{equation*}
    0 = A_g x + B_g y = A_h x + B_h y.
  \end{equation*}

  Hence,
  \begin{equation*}
    \begin{pmatrix}
      A_h & B_h \\
      A_g & B_g
    \end{pmatrix}
    \begin{pmatrix}
      x \\ y
    \end{pmatrix}
    =
    \begin{pmatrix}
      0 \\ 0
    \end{pmatrix},
  \end{equation*}
  implying via \cref{thm:matrix_invertibility_via_determinants} that \( (A_h, B_h) \) and \( (A_g, B_g) \) are linearly dependent.

  Finally, suppose that \( g \) and \( h \) coincide, and let \( (x_0, y_0) \) be a point. We have already shown that there exists some nonzero \( \lambda \) such that \( (A_h, B_h) = \lambda (A_g, B_g) \), thus
  \begin{equation*}
    0 = A_h x_0 + B_h y_0 + C_h = \lambda (A_g x_0 + B_g y_0 + C_g) - \lambda C_g + C_h,
  \end{equation*}
  implying \( C_h = \lambda C_g \).

  \NecessitySubProof Let \( (x_g, y_g) \) and \( (x_h, y_h) \) be directional vectors for \( g \) and \( h \). Then \Cref{thm:coordinates_of_directional_vector} implies that
  \begin{equation*}
    A_g x_g + B_g y_g = 0 = A_h x_h + B_h y_h.
  \end{equation*}

  Suppose that the vectors \( (A_g, B_g) \) and \( (A_h, B_h) \) are linearly dependent, i.e. there exists some \( \lambda \) such that \( (A_h, B_h) \) and \( \lambda (A_g, B_g) \). Then
  \begin{equation*}
    0 = A_h x_h + B_h y_h = \lambda (A_g x_h) + \lambda (B_g y_h).
  \end{equation*}

  Hence,
  \begin{equation*}
    \begin{pmatrix}
      x_g & y_g \\
      x_h & y_h
    \end{pmatrix}
    \begin{pmatrix}
      A_g \\ B_g
    \end{pmatrix}
    =
    \begin{pmatrix}
      0 \\ 0
    \end{pmatrix}
  \end{equation*}
  implying via \cref{thm:matrix_invertibility_via_determinants} that \( (x_g, y_g) \) and \( (x_h, y_h) \) are linearly dependent.

  Therefore, \( g \) and \( h \) are parallel lines.

  If, in addition, \( C_h = \lambda C_g \), then for every point \( (x_0, y_0) \) in \( g \),
  \begin{equation*}
    A_h x_0 + B_h y_0 + C_h = \lambda (A_g x_0 + B_g y_0 + C_g) = 0,
  \end{equation*}
  hence \( g \) and \( h \) coincide.
\end{proof}

\begin{proposition}\label{thm:lines_intersect_in_plane}
  Two distinct \hyperref[def:affine_line]{lines} in the Euclidean plane intersect if and only if they are not \hyperref[def:affine_parallelism]{parallel}. Furthermore, non-parallel lines intersect in exactly one point.
\end{proposition}
\begin{proof}
  Let \( A_g x + B_g y + C_g = 0 \) and \( A_h x + B_h y + C_h = 0 \) be the \hyperref[def:plane_line_equations/general]{general equations} of the lines \( g \) and \( h \). Consider the \hyperref[def:system_of_linear_equations]{system of linear equations}
  \begin{equation}\label{eq:thm:lines_intersect_in_plane/system}
    \begin{pmatrix}
      A_g & B_g \\
      A_h & B_h
    \end{pmatrix}
    \begin{pmatrix}
      x \\ y
    \end{pmatrix}
    =
    -
    \begin{pmatrix}
      C_g \\ C_h
    \end{pmatrix}.
  \end{equation}

  The system has a solution if and only if the lines intersect.

  \SufficiencySubProof If \( g \) and \( h \) intersect, the system \eqref{eq:thm:lines_intersect_in_plane/system} has a solution, and \fullref{thm:kroneker_capelli} implies that the vector \( (C_g, C_h) \) belongs to the column space of the matrix. Then either:
  \begin{itemize}
    \item The lines are parallel and, by \cref{thm:parallel_lines_in_plane}, the lines coincide --- this contradicts our assumption that the lines are distinct.
    \item The lines are not parallel.
  \end{itemize}

  \NecessitySubProof Suppose that \( g \) and \( h \) are not parallel. \Cref{thm:parallel_lines_in_plane} implies that the vectors \( (A_g, A_h) \) and \( (B_g, B_h) \) are linearly independent, and \cref{thm:matrix_invertibility_via_determinants} implies that
  \begin{equation*}
    \begin{pmatrix}
      x \\ y
    \end{pmatrix}
    =
    -
    \begin{pmatrix}
      A_g & B_g \\
      A_h & B_h
    \end{pmatrix}^{-1}
    \begin{pmatrix}
      C_g \\ C_h
    \end{pmatrix}.
  \end{equation*}

  \UniquenessSubProof Suppose that \( g \) and \( h \) are not parallel. \Cref{thm:parallel_lines_in_plane} implies that the matrix in \eqref{eq:thm:lines_intersect_in_plane/system} is invertible. Then \cref{thm:system_of_equations_unique_solution} implies that the system has a unique solution.
\end{proof}

\begin{proposition}\label{thm:normal_vector_of_line}
  The vector with coordinates \( (A, B) \) is \hyperref[def:normal_vector]{normal} for the line with \hyperref[def:plane_line_equations/general]{general equation} \( A x + B y + C = 0 \).
\end{proposition}
\begin{proof}
  The vector \( (-B, A) \) is directional for the line as a consequence of \cref{thm:coordinates_of_directional_vector}. Obviously
  \begin{equation*}
    A(-B) + BA = 0.
  \end{equation*}
\end{proof}

\begin{proposition}\label{thm:plane_rotation_matrix}
  \hyperref[def:rigid_motion/rotation]{Rotations} about the origin in the Euclidean plain are \hyperref[def:unitary_matrix]{orthogonal} \( 2 \times 2 \) matrices with \hyperref[def:matrix_determinant]{determinant} \( 1 \), i.e. the \hyperref[def:unitary_groups]{special orthogonal group} \( \grp{SO}(2) \). We call them \term{(plane) rotation matrices}.

  These are the matrices with entries
  \begin{equation}\label{eq:thm:plane_rotation_matrix}
    \begin{pmatrix}
      a & -b \\
      b & a
    \end{pmatrix},
  \end{equation}
  where \( a^2 + b^2 = 1 \).
\end{proposition}
\begin{defproof}
  Let
  \begin{equation*}
    A = \begin{pmatrix}
      a & c \\
      b & d
    \end{pmatrix}
  \end{equation*}
  be an orthogonal matrix with determinant \( 1 \).

  Since \( A \) is orthogonal, its rows are orthogonal. That is, \( ab + cd = 0 \).

  Also, \( \det A = ad - bc = 1 \). Either \( a \) or \( b \) or both must be nonzero, because otherwise \( A \) would be singular.
  \begin{itemize}
    \item If \( a = 0 \), then \( cd = 0 \) and hence either \( c = 0 \) or \( d = 0 \). If \( c = 0 \), then \( \det A = 0 \), which contradicts the assumption that \( \det A = 1 \). Thus, \( c \neq 0 \) and \( d = 0 \).

    Then \( b^2 = c^2 = 1 \) since the columns are normed, and also \( \det A = -bc = -1 < 0 \). Thus, \( \abs{b} = \abs{c} = 1 \) and they have different signs.

    \item If \( a \neq 0 \), then \( b = -cd / a \) and \( ad + c^2 d / a = 1 \). Multiplying both sides by \( a \), we obtain
    \begin{equation*}
      (a^2 + c^2) d = a.
    \end{equation*}

    But \( a^2 + c^2 = 1 \) because the columns of \( A \) are normed. Thus, \( d = a \). It then follows that \( c = -b \).
  \end{itemize}

  In both cases,
  \begin{equation*}
    A
    =
    \begin{pmatrix}
      a & -b \\
      b & a
    \end{pmatrix}
  \end{equation*}
  and also
  \begin{equation*}
    \det A = a^2 + b^2 = 1.
  \end{equation*}
\end{defproof}

\begin{proposition}\label{thm:plane_ray_abscissa_rotation}
  Every \hyperref[def:geometric_ray]{ray} at the origin in the Euclidean plane is a \hyperref[def:rigid_motion/rotation]{rotation} about the origin of the \hyperref[rem:xyz]{abscissa}. Furthermore, this rotation is unique.
\end{proposition}
\begin{proof}
  Let \( r(t) = td \) be some ray and let \( (x, y) \) be the coordinates of the vector \( d \).

  \UniquenessSubProof Suppose that there exist \hyperref[thm:plane_rotation_matrix]{rotation matrices} \( A \) and \( B \) such that
  \begin{equation*}
    \begin{pmatrix} x \\ y \end{pmatrix} = A \begin{pmatrix} 1 \\ 0 \end{pmatrix} = B \begin{pmatrix} 1 \\ 0 \end{pmatrix}.
  \end{equation*}

  Rotation matrices have the form \eqref{eq:thm:plane_rotation_matrix}, and in this case
  \begin{equation*}
    A = B = \begin{pmatrix}
      x & -y \\
      y & x
    \end{pmatrix}.
  \end{equation*}

  \ExistenceSubProof The matrix
  \begin{equation*}
    \frac 1 {x^2 + y^2}
    \begin{pmatrix}
      x & -y \\
      y & x
    \end{pmatrix}.
  \end{equation*}
  is obviously a rotation matrix.

  Since \( (1, 0) \) are the coordinates of the abscissa basis vector,
  \begin{equation*}
    \frac 1 {x^2 + y^2}
    \begin{pmatrix}
      x & -y \\
      y & x
    \end{pmatrix}
    \begin{pmatrix}
      1 \\ 0
    \end{pmatrix}
    =
    \frac 1 {x^2 + y^2}
    \begin{pmatrix}
      x \\ y
    \end{pmatrix}.
  \end{equation*}

  This vector is \hyperref[def:geometric_ray/unidirectional]{unidirectional} with \( d \), hence its ray at the origin coincides with \( r \).
\end{proof}

\begin{proposition}\label{thm:plane_ray_rotation}
  For every point \( O \) and every pair of rays \( r(t) = O + td \) to \( s(t) = O + te \) with \( \norm{d} = \norm{e} \), there exists a unique \hyperref[def:rigid_motion/rotation]{rotation} \( f(v) \) through \( O \) sending the image of \( r \) to \( s \). Furthermore, \( s(t) = f(r(t)) \).
\end{proposition}
\begin{proof}
  \SubProof{Proof that \( s = f \bincirc r \)} Suppose that \( \norm{d} = \norm{e} = 1 \). Let \( f(v) = O + T(v - O) \) be a rotation sending the image of \( r(t) \) to the image of \( s(t) \). That is, \( f(\img r) = \img s \), but we do not know how the functions \( f \), \( r \) and \( s \) relate.

  We have
  \begin{equation*}
    f(r(0)) = f(O) = O = s(0).
  \end{equation*}

  Then
  \begin{equation*}
    f(r(t)) = f(O + td) = O + T(O + td - O) = O + t Td.
  \end{equation*}

  Since \( f(\img r) = \img s \), there exists some positive number \( \lambda \) such that \( f(r(1)) = s(\lambda) \). That is,
  \begin{equation*}
    O + Td = f(r(1)) = s(\lambda) = O + \lambda e.
  \end{equation*}

  Note that \( T \) preserves norms as an orthogonal transformation, hence
  \begin{equation*}
    \underbrace{\norm{Td}}_{1} = \lambda \underbrace{\norm{e}}_{1}.
  \end{equation*}

  It follows that \( \lambda = 1 \) and \( e = Td \). Therefore,
  \begin{equation*}
    f(r(t)) = O + t Td = O + te = s(t).
  \end{equation*}

  \UniquenessSubProof Suppose that \( f(v) = O + F(v - O) \) and \( g(v) = O + G(v - O) \) are rotations about \( O \) sending the ray \( r(t) \) to \( s(t) \). Then
  \begin{equation*}
    \vect 0 = s(t) - s(t) = f(r(t)) - g(r(t)) = O + F(v - O) - O - G(v - O).
  \end{equation*}

  Therefore,
  \begin{equation*}
    F(v - O) = G(v - O)
  \end{equation*}
  and
  \begin{equation*}
    (G - F) v = (G - F) O.
  \end{equation*}

  Since this holds for arbitrary \( v \), it is only possible that \( (G - F) v = (G - F) O = \vect 0 \). Thus, \( F = G \) and \( f(v) = g(v) \).

  \ExistenceSubProof \Cref{thm:plane_ray_abscissa_rotation} implies that there exists a unique rotation \( R \) sending the abscissa to \( \widehat{r}(t) = td \) and \( S \) sending it to \( \widehat{s}(t) = te \).

  Then
  \begin{equation*}
    S^{-1} e
    =
    S^{-1} S \begin{pmatrix} 1 \\ 0 \end{pmatrix}
    =
    R^{-1} R \begin{pmatrix} 1 \\ 0 \end{pmatrix}
    =
    R^{-1} d.
  \end{equation*}

  Hence,
  \begin{equation*}
    e = S R^{-1} d.
  \end{equation*}

  That is, \( T \coloneqq R^{-1} S \) sends \( \widehat{r} \) to \( \widehat{s} \). Then \( v \mapsto O + T(v - O) \) sends \( r \) to \( s \).
\end{proof}

\begin{proposition}\label{thm:plane_rotation_matrix_angle}
  The map
  \begin{equation}\label{eq:thm:plane_rotation_matrix_angle}
    \varphi
    \mapsto
    \begin{pmatrix}
      \cos \varphi & -\sin \varphi \\
      \sin \varphi & \cos \varphi
    \end{pmatrix}
  \end{equation}
  is an \hyperref[def:morphism_invertibility/right_cancellative]{epimorphism} from the real numbers under addition to the group of \hyperref[thm:plane_rotation_matrix]{plane rotation matrices} under composition. The kernel of this map is the set of multiples of \( 2\pi \).

  We call \( \varphi \) the \term{angle} of the rotation; the semantics of the word \enquote{angle} are discussed in \cref{def:angle}.

  There are other groups isomorphic to the rotation group --- see \cref{def:circle_group}.
\end{proposition}
\begin{proof}
  \SubProof{Proof of well-definedness} The matrix \eqref{eq:thm:plane_rotation_matrix_angle} is orthogonal, and its determinant is \( 1 \) as a consequence of \cref{thm:trigonometric_identities/pythagorean_identity}. Hence, it induces a rotation.

  \SubProofOf[def:function_invertibility/surjective/equality]{surjectivity} Let
  \begin{equation*}
    A
    =
    \begin{pmatrix}
      a & -b \\
      b & c
    \end{pmatrix}
  \end{equation*}
  be a rotation matrix; i.e. \( \det A = a^2 + b^2 = 1 \). Every rotation matrix has this form as discussed in \cref{thm:plane_rotation_matrix}.

  Define \( \varphi \) as
  \begin{equation*}
    \varphi \coloneqq \begin{cases}
      \arccos a,         &b \geq 0, \\
      2 \pi - \arccos a, &b < 0.
    \end{cases}
  \end{equation*}

  Then
  \begin{equation*}
    (\sin \varphi)^2 = 1 - (\cos \varphi)^2 = 1 - a^2 = b^2.
  \end{equation*}

  \begin{itemize}
    \item If \( b \geq 0 \), then \( \varphi \in [0, \pi] \) and hence \( \sin \varphi \geq 0 \). Since the square root has nonnegative values,
    \begin{equation*}
      \sin \varphi = \sqrt{ 1 - a^2 } = b.
    \end{equation*}

    \item If \( b < 0 \), then \( \varphi \in [\pi, 2\pi) \) and hence \( \sin \varphi < 0 \). Thus,
    \begin{equation*}
      \sin \varphi = -\sqrt{ 1 - a^2 } = b.
    \end{equation*}
  \end{itemize}

  Therefore,
  \begin{equation*}
    \begin{pmatrix}
      \cos \varphi & -\sin \varphi \\
      \sin \varphi & \cos \varphi
    \end{pmatrix}
    =
    \begin{pmatrix}
      a & -b \\
      b & a
    \end{pmatrix}
    =
    A.
  \end{equation*}

  \SubProof{Proof of homomorphism condition} We have
  \begin{equation*}
    \cos(\varphi + \psi)
    \reloset {\eqref{eq:thm:trigonometric_identities/sum_of_angles/cos}} =
    \cos \varphi \cos \psi - \sin \varphi \sin \psi
    =
    \begin{pmatrix}
      \cos \varphi & -\sin \varphi
    \end{pmatrix}
    \begin{pmatrix}
      \cos \psi \\ \sin \psi
    \end{pmatrix}.
  \end{equation*}
  and
  \begin{equation*}
    \sin(\varphi + \psi)
    \reloset {\eqref{eq:thm:trigonometric_identities/sum_of_angles/cos}} =
    \cos \varphi \sin \psi + \sin \varphi \cos \psi
    =
    \begin{pmatrix}
      \cos \varphi & \sin \varphi
    \end{pmatrix}
    \begin{pmatrix}
      \sin \psi \\ \cos \psi
    \end{pmatrix}.
  \end{equation*}

  Then
  \begin{equation*}
    \begin{pmatrix}
      \cos \varphi & -\sin \varphi \\
      \sin \varphi & \cos \varphi
    \end{pmatrix}
    \begin{pmatrix}
      \cos \psi & -\sin \psi \\
      \sin \psi & \cos \psi
    \end{pmatrix}
    =
    \begin{pmatrix}
      \cos (\varphi + \psi) & -\sin (\varphi + \psi) \\
      \sin (\varphi + \psi) & \cos (\varphi + \psi)
    \end{pmatrix}.
  \end{equation*}

  \SubProof{Proof that kernel are multiples of \( 2\pi \)} Suppose that
  \begin{equation*}
    \begin{pmatrix}
      \cos \varphi & -\sin \varphi \\
      \sin \varphi & \cos \varphi
    \end{pmatrix}
    =
    \begin{pmatrix}
      \cos \psi & -\sin \psi \\
      \sin \psi & \cos \psi
    \end{pmatrix}.
  \end{equation*}

  Both \( \sin \) and \( \cos \) are bijective on the interval \( [0, 2\pi) \). From \cref{thm:trigonometric_function_period} it follows that, if \( \cos \varphi = \cos \psi \), then \( 2\pi \) divides \( \varphi - \psi \).
\end{proof}

\begin{definition}\label{def:angle}\mimprovised
  A \term{directed angle} is an ordered pair of \hyperref[def:geometric_ray]{rays} with a common vertex. We say that the rays are the \term{sides} of the angle and denote the angle with sides \( r \) and \( s \) via \( \angle(r, s) \).

  \begin{figure}[!ht]
    \centering
    \includegraphics[align=c]{output/def__angle}
    \caption{The two \hyperref[def:angle]{directed angles} \( \angle(r, s) \) and \( \angle(r, s) \) given by the rays \( r \) and \( s \).}\label{def:angle/measure/figure}
  \end{figure}

  \begin{thmenum}
    \thmitem{def:angle/measure} Denote by \( O \) the common vertex of \( r \) and \( s \). \Cref{thm:plane_ray_rotation} implies that there exists a unique \hyperref[def:rigid_motion/rotation]{rotation} \( f(v) = O + T(v - O) \) sending \( r \) to \( s \). \Cref{thm:plane_rotation_matrix_angle} then implies the existence of a unique number \( \varphi \in [0, 2\pi) \) entirely determining \( T \). We will call \( \varphi \) the \term{measure} of \( \angle(r, s) \) and denote it by \( \measuredangle(r, s) \)

    We can classify angles based on their measure as
    \begin{thmenum}
      \thmitem{def:angle/measure/zero} \term{zero} if \( \varphi = 0 \),

      \medspace

      \thmitem{def:angle/measure/acute}\mcite[\S 16]{Hadamard2008LessonsInGeometryVol1} \term[bg=остър (\cite[9]{Гюзелев1873Геометрія}), ru=острый (\cite[\S 22]{Киселёв2004Геометрия})]{acute} if \( 0 < \varphi < \pi / 2 \),

      \medspace

      \thmitem{def:angle/measure/right} \term[bg=прав (\cite[9]{Гюзелев1873Геометрія}), ru=прямой (\cite[\S 22]{Киселёв2004Геометрия})]{right} if \( \varphi = \tfrac \pi 2 \),

      \medspace

      \thmitem{def:angle/measure/obtuse}\mcite[\S 16]{Hadamard2008LessonsInGeometryVol1} \term[bg=тъп (\cite[9]{Гюзелев1873Геометрія}), ru=тупой (\cite[\S 22]{Киселёв2004Геометрия})]{obtuse} if \( \pi / 2 < \varphi < \pi \),

      \medspace

      \thmitem{def:angle/measure/straight}\mcite[\S 15]{Hadamard2008LessonsInGeometryVol1} \term{straight} if \( \varphi = \pi \), in which case the angle is actually a line,

      \medspace

      \thmitem{def:angle/measure/reflex} \term{reflex} if \( \varphi > \pi \).
    \end{thmenum}

    \begin{figure}[!ht]
      \centering
      \includegraphics[align=c]{output/def__angle__measure__right}
      \caption{A \hyperref[def:angle/measure/right]{right angle} is conventionally denoted via dots.}\label{fig:def:angle/measure/right}
    \end{figure}

    \thmitem{def:angle/vectors} It is conventional to conflate an angle and its measure. For this reason, it is sometimes convenient to define angles via directional vectors rather than rays, disregarding the vertex.

    We call vectors determining the rays \term{directional vectors} of the angle.

    \thmitem{def:angle/undirected} Given the transformation \( f(v) \) sending \( r \) to \( s \), its inverse \( f^{-1}(v) \) sends \( s \) to \( r \). Their composition is the identity, whose angle measure is a multiple \( 2\pi \). \Cref{thm:plane_rotation_matrix_angle} implies that the angle measures \( \angle(r, s) \) and \( \angle(s, r) \) sum to \( 2\pi \). We call the angle with the smaller measure the \term{undirected angle} between \( r \) and \( s \).
  \end{thmenum}
\end{definition}

\begin{proposition}\label{thm:angle_measure_swap}
  \hyperref[def:angle/measure]{Directed angle measures} satisfy
  \begin{equation}\label{eq:thm:angle_measure_swap}
    \measuredangle(u, v) = 2\pi - \measuredangle(u, v).
  \end{equation}
\end{proposition}
\begin{proof}
  Trivial.
\end{proof}

\begin{proposition}\label{thm:cosine_of_angle_measure}
  The \hyperref[def:angle/measure]{measure} of an angle \( \angle(u, v) \) between the normed vectors \( u \) and \( v \) satisfies
  \begin{equation}\label{eq:thm:cosine_of_angle_measure/cos}
    \cos \measuredangle(u, v) = \inprod u v.
  \end{equation}

  Furthermore, denoting by \( y_u \) and \( y_v \) the ordinates of \( u \) and \( v \),
  \begin{equation}\label{eq:thm:cosine_of_angle_measure/arccos}
    \measuredangle(u, v) = \begin{cases}
      \arccos \inprod u v,        &y_v \geq y_u, \\
      2\pi - \arccos \inprod u v, &y_v < y_u.
    \end{cases}
  \end{equation}
\end{proposition}
\begin{proof}
  \Fullref{thm:cauchy_bunyakovsky_schwarz_inequality} implies that \( \inprod u v \) ranges between \( -1 \) and \( 1 \); hence, it is the cosine of a real number.

  \Cref{thm:plane_ray_abscissa_rotation} gives us an angle measure \( \beta \) whose rotation sends the abscissa into \( u \), and a similar angle measure \( \gamma \) for \( v \). Then \( \angle(u, v) = \gamma - \beta \) The coordinates of \( u \) are
  \begin{equation*}
    \begin{pmatrix}
      \cos \beta & -\sin \beta \\
      \sin \beta & \cos \beta
    \end{pmatrix}
    \begin{pmatrix}
      1 \\ 0
    \end{pmatrix}
    =
    \begin{pmatrix}
      \cos \beta \\ \sin \beta
    \end{pmatrix},
  \end{equation*}
  and similarly for \( v \).

  Then
  \begin{align*}
    \inprod u v
    &=
    \cos \beta \cos \gamma + \sin \beta \sin \gamma
    \reloset {\ref{thm:trigonometric_identities/products}} = \\ &=
    \tfrac 1 2 [\cos(\beta - \gamma) + \cos(\beta + \gamma) + \cos(\beta - \gamma) - \cos(\beta + \gamma)]
    = \\ &=
    \cos(\beta - \gamma)
    = \\ &=
    \cos(\gamma - \beta)
    = \\ &=
    \cos \measuredangle(u, v).
  \end{align*}

  If \( y_v \geq y_u \), then \( \measuredangle(u, v) = \arccos \inprod u v \). We have
  \begin{equation*}
    \abs{\sin \measuredangle(u, v)}
    =
    \abs{\sin \arccos \inprod u v}
    =
    \sqrt{1 - \inprod u v^2}.
  \end{equation*}

  Since \( 0 \leq \measuredangle(u, v) \leq \pi \), we have \( \sin \measuredangle(u, v) \geq 0 \) and
  \begin{equation*}
    \sin \measuredangle(u, v) = \sqrt{1 - \inprod u v^2}.
  \end{equation*}

  Otherwise, \( \measuredangle(u, v) = 2\pi - \arccos \inprod u v \). We have
  \begin{equation*}
    \abs{\sin \measuredangle(u, v)}
    \reloset{\ref{thm:trigonometric_function_period}} =
    \abs{\sin(-\arccos \inprod u v)}
    \reloset{\ref{thm:def:trigonometric_function/parity}} =
    \abs{-\sin(\arccos \inprod u v)}
    =
    \sqrt{1 - \inprod u v^2}.
  \end{equation*}

  Since \( \pi \leq \measuredangle(u, v) \leq 2\pi \), we have \( \sin \measuredangle(u, v) \leq 0 \) and
  \begin{equation*}
    \sin \measuredangle(u, v) = -\sqrt{1 - \inprod u v^2}.
  \end{equation*}
\end{proof}

\begin{proposition}\label{thm:arctantwo}
  The measure of a directed angle between the abscissa and the ray with directional vector \( v = (x, y) \) is
  \begin{equation}
    \angle(Ox, v) = \arctantwo(y, x),
  \end{equation}
  where
  \begin{equation*}
    \begin{aligned}
       &\arctantwo: \BbbR^2 \setminus \set{ \vect 0 } \to [-\pi, \pi), \\
       &\arctantwo(y, x) \coloneqq \begin{cases}
        \arctan(y / x),                     &x > 0, \\
        \arctan(y / x) + \sgn(y) \cdot \pi, &x < 0 \T{and} y \neq 0, \\
        \pi,                                     &x < 0 \T{and} y = 0, \\
        \sgn(y) \cdot \pi / 2,              &x = 0. \\
      \end{cases}
    \end{aligned}
  \end{equation*}

  We take the branch of \( \arctan \) with values \( [-\pi / 2, \pi / 2) \).

  The name of the function comes from the \( \logic{FORTRAN} \) programming language and the suffix \enquote{2} highlights that the function has two arguments rather than one.
\end{proposition}
\begin{proof}
  \hfill
  \begin{itemize}
    \item If \( x > 0 \), then
    \begin{equation*}
      \cos(\arctantwo(y, x))
      =
      \cos(\arctan(y / x))
      \reloset {\eqref{eq:thm:def:inverse_trigonometric_function/cos_of_arctan}} =
      \frac 1 {\sqrt{1 + y^2 / x^2}}
      =
      \frac x {\sqrt{x^2 + y^2}}
    \end{equation*}
    and
    \begin{equation*}
      \sin(\arctantwo(y, x))
      =
      \sin(\arctan(y / x))
      \reloset {\eqref{eq:thm:def:inverse_trigonometric_function/sin_of_arctan}} =
      \frac {y / x} {\sqrt{1 + y^2 / x^2}}
      =
      \frac y {\sqrt{x^2 + y^2}}.
    \end{equation*}

    Then \( \arctantwo(y, x) \) is the angle of the unique rotation sending the abscissa to \( v \).

    \item If \( x < 0 \) and \( y \neq 0 \), then
    \begin{align*}
      \cos(\arctantwo(y, x))
      &=
      \cos(\arctan(y / x) + \sgn(y) \cdot \pi)
      \reloset {\eqref{eq:thm:trigonometric_function_period_identities/full/cos}} = \\ &=
      -\cos(\arctan(y / x))
      = \\ &=
      -\frac 1 {\sqrt{1 + y^2 / x^2}}
      = \\ &=
      -\frac 1 {1 / ((-x)) \cdot \sqrt{x^2 + y^2}}
      = \\ &=
      \frac x {\sqrt{x^2 + y^2}}.
    \end{align*}
    and similarly
    \begin{equation*}
      \sin(\arctantwo(y, x))
      =
      \sin(\arctan(y / x)  + \sgn(y) \cdot \pi)
      \reloset {\eqref{eq:thm:trigonometric_function_period_identities/full/sin}} =
      -\sin(\arctan(y / x))
      =
      \cdots
      =
      \frac y {x^2 + y^2}.
    \end{equation*}

    \item If \( x < 0 \) and \( y = 0 \), then
    \begin{equation*}
      \cos(\arctantwo(y, x)) = \cos \pi = -1
    \end{equation*}
    and
    \begin{equation*}
      \sin(\arctantwo(y, x)) = \sin \pi = 0.
    \end{equation*}

    \item If \( x = 0 \), then
    \begin{equation*}
      \cos(\arctantwo(y, x)) = \cos(\sgn(y) \cdot \pi / 2) = 0.
    \end{equation*}
    and
    \begin{equation*}
      \sin(\arctantwo(y, x)) = \sin(\sgn(y) \cdot \pi / 2) = \sgn(y) \cdot 1.
    \end{equation*}
  \end{itemize}
\end{proof}

\begin{definition}\label{def:perpendicularity}\mimprovised
  We say that two vectors are \term{perpendicular} if the \hyperref[def:angle/undirected]{undirected angle} between them is \hyperref[def:angle/measure/right]{right}.

  \begin{thmenum}
    \thmitem{def:perpendicularity/directional} The definition also applies to rays and lines if we take their directional vectors.
    \thmitem{def:perpendicularity/subspace} The term \enquote{perpendicular from a point to an affine subspace} refers to the segment from the point to its \hyperref[def:orthogonal_projection]{orthogonal projection} onto the subspace.
  \end{thmenum}
\end{definition}
\begin{comments}
  \item This is the geometric analog of \hyperref[def:orthogonality]{orthogonality}.
\end{comments}

\begin{proposition}\label{thm:perpendicular_iff_orthogonal}
  Two vectors are \hyperref[def:orthogonality]{orthogonal} if and only if they are \hyperref[def:perpendicularity/directional]{perpendicular}.
\end{proposition}
\begin{proof}
  For the angle \( \angle(u, v) \), we have
  \begin{equation*}
    \cos \measuredangle(u, v)
    \reloset {\eqref{eq:thm:cosine_of_angle_measure/cos}} =
    \inprod u v.
  \end{equation*}

  Then \( \measuredangle(u, v) = \pi / 2 \) if and only if \( \cos(\measuredangle(u, v)) = 0 \) if and only if \( \inprod u v = 0 \).
\end{proof}

\begin{proposition}\label{thm:straight_iff_opposite_rays}
  An angle is \hyperref[def:angle/measure/straight]{straight} if and only if its sides are \hyperref[def:geometric_ray/opposite]{opposite rays}.
\end{proposition}
\begin{proof}
  \SufficiencySubProof Suppose that the angle \( \angle(r, s) \) is straight. Let \( O \) be the vertex and let \( r(t) = O + td \) and \( s(t) = O + te \) be parametrizations of the rays.

  We have assumed that the measure of the angle is \( \pi \). \Cref{thm:cosine_of_angle_measure} implies that
  \begin{equation*}
    \frac {\inprod d e} {\norm{d} \cdot \norm{e}} = \cos(\pi) = -1.
  \end{equation*}

  \Fullref{thm:cauchy_bunyakovsky_schwarz_inequality} implies that \( d \) and \( e \) are linearly dependent. We have two possibilities:
  \begin{itemize}
    \item If \( e = {\norm e} / {\norm d} \cdot d \), then \( r(t) \) and \( s(t) \) describe the same set of points, and thus the rotation between them is the identity matrix. But the identity matrix induces an angle with measure zero, and we have assumed that the measure is \( \pi \).

    \item It remains for \( e \) to be \( -{\norm e} / {\norm d} \cdot d \).
  \end{itemize}

  Therefore, the rays \( r(t) \) and \( s(t) \) are opposite.

  \NecessitySubProof Suppose that \( r(t) = O + td \) and \( s(t) = O + te \) are opposite rays. Denote the measure of \( \angle(r, s) \) via \( \alpha \). Then
  \begin{equation*}
    \cos(\alpha)
    =
    \frac {\inprod d e} {\norm{d} \cdot \norm{e}}
    =
    -\frac {\inprod d {\norm{e} d}} {\norm{d} \cdot \norm{d} \cdot \norm{e}}
    =
    -\frac {\inprod d d} {\norm{d}^2}
    =
    -1.
  \end{equation*}

  The only number in \( [0, 2\pi) \) with this property is \( \pi \). Hence, the angle \( \angle(r, s) \) is straight.
\end{proof}

\begin{definition}\label{def:adjacent_angles}\mcite[\S 10]{Hadamard2008LessonsInGeometryVol1}
  We say that two \hyperref[def:angle]{undirected angles} with a common vertex are \term[bg=съседен (\cite[\S 5]{Гюзелев1873Геометрія}), ru=смежный (\cite[\S 22]{Киселёв2004Геометрия})]{adjacent} if they have a common side and if their other sides lie on different \hyperref[def:half_space]{half-planes} with respect to (the line containing) their common side.

  We avoid introducing similar terminology for directed angles.

  \begin{figure}[!ht]
    \centering
    \includegraphics[align=c]{output/def__adjacent_angles}
    \caption{The undirected angles \( \angle(r, p) \) and \( \angle(p, s) \) are adjacent.}\label{fig:def:adjacent_angles}
  \end{figure}
\end{definition}

\begin{definition}\label{def:sum_of_angles}
  Fix three rays \( p \), \( r \) and \( s \) with a common vertex \( O \). We call the (directed) angle \( \angle(r, s) \) the \term[bg=сума (\cite[8]{Гюзелев1873Геометрія}), ru=сумма (\cite[\S 15]{Киселёв2004Геометрия})]{sum} of \( \angle(r, p) \) and \( \angle(p, s) \).

  If the angles are undirected, we only consider sums if the addends are \hyperref[def:adjacent_angles]{adjacent}.

  \begin{figure}[!ht]
    \hfill
    \includegraphics[align=c]{output/def__sum_of_angles__acute}
    \hfill
    \includegraphics[align=c]{output/def__sum_of_angles__obtuse}
    \hfill
    \hfill
    \caption{Sum of two directed angles. The first is also a sum of the corresponding undirected angles.}\label{fig:def:sum_of_angles}
  \end{figure}
\end{definition}

\begin{proposition}\label{thm:sum_of_angles_measure}
  The \hyperref[def:sum_of_angles]{angle sum} of \( \angle(r, p) \) and \( \angle(p, s) \) satisfies the following \hyperref[rem:congruence_modulo_real_number]{congruence} of \hyperref[def:angle/measure]{angle measures}:
  \begin{equation*}
    \measuredangle(r, s) \equiv \measuredangle(r, p) + \measuredangle(p, s) \pmod {2\pi}.
  \end{equation*}
\end{proposition}
\begin{proof}
  Let \( f(v) = o + F(v - o) \) be the operator sending \( r \) to \( p \) given by \cref{thm:plane_ray_rotation} and \( g(v) = o + G(v - o) \) be the map from \( p \) to \( s \). Then, when regarding the rays as parametric curves,
  \begin{equation*}
    [g \bincirc f](r(t)) = g(f(r(t))) = g(p(t)) = s(t),
  \end{equation*}
  thus \( g \bincirc f \) is an affine map sending \( r \) to \( s \).

  Furthermore,
  \begin{equation*}
    [g \bincirc f](v)
    =
    o + G(f(v) - o)
    =
    o + G(o + F(v - o) - o)
    =
    o + GF(v - o),
  \end{equation*}
  hence \cref{thm:plane_ray_rotation} implies that \( g \bincirc f \) is the \hi{unique} affine map with a rotation sending \( r \) to \( s \).

  We know from \cref{thm:plane_rotation_matrix_angle} that the angle of the rotation \( GF \) is the sum of angles of \( G \) and \( F \). This concludes the proof.
\end{proof}

\begin{corollary}\label{thm:angle_of_opposite_ray}
  Fix two rays \( r \) and \( s \) and denote by \( -r \) the \hyperref[def:geometric_ray/opposite]{opposite ray} of \( r \). Then
  \begin{equation}\label{eq:thm:angle_of_opposite_ray}
    \measuredangle(-r, s) = \measuredangle(r, s) + \pi \pmod (2\pi).
  \end{equation}

  Furthermore,
  \begin{align}
    \sin \measuredangle(-r, s) &= - \sin \measuredangle(r, s), \label{eq:thm:angle_of_opposite_ray/sin} \\
    \cos \measuredangle(-r, s) &= - \cos \measuredangle(r, s). \label{eq:thm:angle_of_opposite_ray/cos}
  \end{align}
\end{corollary}
\begin{proof}
  \Cref{thm:sum_of_angles_measure} then implies that
  \begin{equation*}
    \measuredangle(-r, s) = \measuredangle(-r, r) + \measuredangle(r, s).
  \end{equation*}

  \Cref{thm:straight_iff_opposite_rays} implies that
  \begin{equation*}
    \measuredangle(-r, r) = \pi.
  \end{equation*}

  This concludes the proof of \eqref{eq:thm:angle_of_opposite_ray}.

  Then \eqref{eq:thm:angle_of_opposite_ray/sin} and \eqref{eq:thm:angle_of_opposite_ray/cos} follow from \cref{thm:trigonometric_function_period_identities}.
\end{proof}

\begin{definition}\label{def:vertical_angles}\mcite[\S 12]{Hadamard2008LessonsInGeometryVol1}
  Let \( g \) and \( h \) be \hyperref[def:crossing_lines]{crossing lines} with intersection point \( O \).

  Let \( P \) and \( Q \) be arbitrary points from \( g \) and \( h \) distinct from \( O \); let \( P' \) and \( Q' \) be their \hyperref[def:rigid_motion/point_reflection]{point reflections} through \( O \).

  \begin{figure}[!ht]
    \centering
    \includegraphics[align=c]{output/def__vertical_angles}
    \caption{The angles \( \angle(\vect{OP}, \vect{OQ}) \) and \( \angle(\vect{OP'}, \vect{OQ'}) \) are \hyperref[def:vertical_angles]{vertical}.}\label{fig:def:vertical_angles}
  \end{figure}

  We say that \( \angle(\vect{OP}, \vect{OQ}) \) and \( \angle(\vect{OP'}, \vect{OQ'}) \) are a pair of \term[bg=вертикални/срещуположни ъгли (\cite[\S 7]{Гюзелев1873Геометрія}), ru=вертикальные углы (\cite[\S 26]{Киселёв2004Геометрия})]{vertical angles}, and similarly for \( \angle(\vect{OP}, \vect{OQ'}) \) and \( \angle(\vect{OP'}, \vect{OQ}) \).
\end{definition}

\begin{proposition}\label{thm:vertical_angles_are_equal}
  Every two \hyperref[def:vertical_angles]{vertical angles} have equal \hyperref[def:angle/measure]{measures}.
\end{proposition}
\begin{proof}
  \Cref{thm:perpendicular_iff_orthogonal} implies that \( \measuredangle(\vect{OP}, \vect{OP'}) = \pi \).

  \Cref{thm:sum_of_angles_measure} implies that
  \begin{equation*}
    \measuredangle(\vect{OP}, \vect{OQ}) + \measuredangle(\vect{OP'}, \vect{OQ}) = \measuredangle(\vect{OP}, \vect{OP'}) = \pi
  \end{equation*}
  and similarly
  \begin{equation*}
    \measuredangle(\vect{OP'}, \vect{OQ}) + \measuredangle(\vect{OP'}, \vect{OQ'}) = \pi.
  \end{equation*}

  Then
  \begin{equation*}
    0 = \pi - \pi = \measuredangle(\vect{OP}, \vect{OQ}) - \measuredangle(\vect{OP'}, \vect{OQ'}),
  \end{equation*}
  hence
  \begin{equation*}
    \measuredangle(\vect{OP}, \vect{OQ}) = \measuredangle(\vect{OP'}, \vect{OQ'}).
  \end{equation*}

  We can prove the other equality analogously.
\end{proof}

\begin{proposition}\label{thm:angle_with_inverse_vectors}
  For any three \hyperref[def:affine_dependence]{affinely independent} points \( P \), \( Q \) and \( R \), the angles \( \angle(\vect{PQ}, \vect{PR}) \) and \( \angle(\vect{RP}, \vect{RQ}) \) are \hyperref[def:vertical_angles]{vertical}.
\end{proposition}
\begin{comments}
  \item In particular, by \cref{thm:vertical_angles_are_equal}, they have equal measures.
\end{comments}
\begin{proof}
  We have
  \begin{equation*}
    \vect{PQ} = Q - P = 2Q - P - Q = \vect{Q(2Q - P)},
  \end{equation*}
  where \( 2Q - P \) is the \hyperref[def:rigid_motion/point_reflection]{point reflection} of \( P \) through \( Q \).

  Similarly, we have \( \vect{PR} = \vect{R(2R - P)} \).

  Furthermore, the two vectors are linearly independent since the points are affinely independent.

  Then the angles \( \angle(\vect{PQ}, \vect{PR}) \) and
  \begin{equation*}
    \angle(\vect{QP}, \vect{RP}) = \angle(\vect{Q(2Q - P)}, \vect{R(2R - P)})
  \end{equation*}
  are vertical.
\end{proof}

\begin{remark}\label{rem:angle}
  To recap, the word \enquote{angle} may refer to:
  \begin{itemize}
    \item The unique real number from \( [0, 2\pi) \) inducing a rotation as shown in \cref{thm:plane_rotation_matrix_angle}.
    \item Either a \hyperref[def:angle]{directed angle} or its \hyperref[def:angle/measure]{measure}.
    \item Either an \hyperref[def:angle]{undirected angle} or its measure.
    \item The angle of a line crossing discussed in \cref{thm:vertical_angles_are_equal}.
  \end{itemize}
\end{remark}

\begin{definition}\label{def:angles_of_transversal}\mcite[\S 37]{Hadamard2008LessonsInGeometryVol1}
  Let \( l \) be a \hyperref[def:transversal_line]{transversal} of the distinct lines \( g \) and \( h \). Suppose that, if \( g \) and \( h \) intersect, \( l \) does not pass through their intersection.

  \begin{figure}[!ht]
    \centering
    \includegraphics[align=c]{output/def__angles_of_transversal}
    \caption{A transversal crossing two lines with some of the distances shortened.}\label{fig:def:angles_of_transversal}
  \end{figure}

  Let \( P \) be the crossing point of \( g \) and \( l \) and let \( P' \) be the \hyperref[def:orthogonal_projection]{orthogonal projection} of \( P \) onto \( h \).

  Similarly, let \( Q \) be the crossing point of \( h \) and \( l \) and let \( Q' \) be the projection of \( Q \) onto \( g \).

  We introduce the following terminology for the eight (undirected) angles of the intersection:
  \begin{thmenum}
    \thmitem{def:angles_of_transversal/interior} We say that an angle is \term{interior} if it lies in the intersection of the half-plane of \( g \) containing \( Q \) and the half-plane of \( h \) containing \( P \).

    For example, the angles \( \angle(\vect{PQ}, \vect{PQ'}) \) and \( \angle(\vect{QP}, \vect{QP'}) \) are interior.

    \thmitem{def:angles_of_transversal/exterior} We say that an angle is \term{exterior} if it is not interior.

    For example, the angles \( \angle(\vect{P(2P - Q)}, \vect{PQ'}) \) and \( \angle(\vect{Q(2Q - P)}, \vect{QP'}) \) are exterior.

    \thmitem{def:angles_of_transversal/alternate} We say that two \hi{non-\hyperref[def:adjacent_angles]{adjacent}} angles are \term[bg={кръстни, кръстосани (\cite[\S 28]{Гюзелев1873Геометрія})}, ru=накрестлежащие (\cite[\S 72]{Киселёв2004Геометрия})]{alternate} if they lie in different half-planes with respect to \( l \) and are both either interior or exterior.

    For example, \( \angle(\vect{PQ}, \vect{PQ'}) \) and \( \angle(\vect{QP}, \vect{QP')}) \) is a pair of interior alternate angles, while their corresponding pair of \hyperref[def:vertical_angles]{vertical angles} \( \angle(\vect{P(2P - Q)}, \vect{P(2P - Q')}) \) and \( \angle(\vect{Q(2Q - P)}, \allowbreak \vect{Q(2Q - P')}) \) is exterior alternate.

    \thmitem{def:angles_of_transversal/corresponding} We say that two \hi{non-\hyperref[def:adjacent_angles]{adjacent}} angles are \term[bg=съответни (\cite[\S 28]{Гюзелев1873Геометрія}), ru=соответственные (\cite[\S 72]{Киселёв2004Геометрия})]{corresponding} if they lie in different in the same half-plane with respect to \( l \) and one of them is internal, while the other is external.

    For example, the \( \angle(\vect{PQ}, \vect{PQ'}) \) and \( \angle(\vect{Q(2Q - P)}, \vect{Q(2Q - P')}) \) is a pair of corresponding angles.
  \end{thmenum}
\end{definition}

\begin{proposition}\label{thm:angles_of_transversal_parallel_lines}
  Let \( l \) be a \hyperref[def:transversal_line]{transversal} of the distinct lines \( g \) and \( h \). Suppose that, if \( g \) and \( h \) intersect, \( l \) does not pass through their intersection.

  Then the following are equivalent:
  \begin{thmenum}
    \thmitem{thm:angles_of_transversal_parallel_lines/parallel} The lines \( g \) and \( h \) are \hyperref[def:affine_parallelism]{parallel}.
    \thmitem{thm:angles_of_transversal_parallel_lines/alternate} Any two \hyperref[def:angles_of_transversal/alternate]{alternate angles} have equal measures.
    \thmitem{thm:angles_of_transversal_parallel_lines/corresponding} Any two \hyperref[def:angles_of_transversal/corresponding]{corresponding angles} have equal measures.
  \end{thmenum}
\end{proposition}
\begin{proof}
  \ImplicationSubProof{thm:angles_of_transversal_parallel_lines/parallel}{thm:angles_of_transversal_parallel_lines/alternate} Suppose that \( g \) and \( h \) are parallel. The vectors \( \vect{PQ'} \) and \( \vect{P'Q} \) are collinear and there exists some scalar \( a \) such that \( \vect{P'Q} = a \cdot \vect{PQ'} \).

  It follows from \cref{thm:def:orthogonal_projection/normal} it follows that \( \vect{PP'} \) is normal for \( h \). Then \( \vect{PP'} \) is orthogonal to \( \vect{P'Q} \), and hence also to \( \vect{PQ'} \). \Cref{thm:perpendicular_iff_orthogonal} implies that the angle \( \angle(\vect{PQ'}, \vect{PP'}) \) is right.

  \Cref{thm:sum_of_angles_measure} implies that
  \begin{equation*}
    \underbrace{\measuredangle(\vect{PQ'}, \vect{PP'})}_{\pi / 2} = \measuredangle(\vect{PQ'}, \vect{PQ}) + \measuredangle(\vect{PQ}, \vect{PP'}) \pmod {2\pi}.
  \end{equation*}

  All angles are undirected and thus have a measure at most \( \pi \). It follows that both \( \angle(\vect{PQ'}, \vect{PQ}) \) and \( \angle(\vect{PQ}, \vect{PP'}) \) must be acute.

  Similarly, we can conclude that \( \angle(\vect{QP'}, \vect{QP}) \) is acute, and hence so is its vertical \( \angle(P'Q, PQ) \).

  The value of \( \cos(t) \) for \( 0 < t < \pi / 2 \) is positive. Hence, \( \cos(\measuredangle(\vect{PQ}, \vect{P'Q})) \) and \( \cos(\measuredangle(\vect{PQ}, \vect{PQ'})) \) are both positive because both angles are acute.

  Since \( \vect{P'Q} = a \cdot \vect{PQ'} \), \cref{thm:cosine_of_angle_measure} implies that
  \begin{equation*}
    \cos(\measuredangle(\vect{PQ}, \vect{P'Q}))
    =
    \frac {\inprod {\vect{PQ}} {\vect{P'Q}}} {\norm{\vect{PQ}} \cdot \norm{\vect{P'Q}}}
    =
    \frac a {\abs{a}} \cdot \frac {\inprod {\vect{PQ}} {\vect{PQ'}}} {\norm{\vect{PQ}} \cdot \norm{\vect{PQ'}}}
    =
    \sgn(a) \cdot \cos(\measuredangle(\vect{PQ}, \vect{PQ'}))
  \end{equation*}

  Both cosines are positive, hence \( a \) must itself be positive. We conclude that \( \vect{P'Q} = \vect{PQ'} \).

  Therefore,
  \begin{equation*}
    \measuredangle(\vect{PQ}, \vect{PQ'})
    =
    \measuredangle(\vect{PQ}, \vect{P'Q})
    \reloset {\ref{thm:angle_with_inverse_vectors}} =
    \measuredangle(\vect{QP}, \vect{QP'}).
  \end{equation*}

  This shows that two of the internal alternate angles are equal. \Cref{thm:vertical_angles_are_equal} implies that their vertical angles, which are external alternate, are also equal. \Cref{thm:sum_of_angles_measure} and \cref{thm:perpendicular_iff_orthogonal} imply that the other pairs of alternate angles are equal.

  \ImplicationSubProof{thm:angles_of_transversal_parallel_lines/alternate}{thm:angles_of_transversal_parallel_lines/corresponding} If two alternate angles have equal measures, from \cref{thm:vertical_angles_are_equal} it follows that the first angle and the vertical angle of the second also have equal measures. The aforementioned two angles are corresponding.

  \ImplicationSubProof{thm:angles_of_transversal_parallel_lines/corresponding}{thm:angles_of_transversal_parallel_lines/parallel} Suppose that some pair of corresponding angles have equal measures. Then \cref{thm:vertical_angles_are_equal} implies that their vertical angles also have equal measures, and \cref{thm:sum_of_angles_measure} and \cref{thm:perpendicular_iff_orthogonal} imply that the other pairs of corresponding angles also do. Hence, every two corresponding angles have equal measures.

  In particular, \( \angle(\vect{PQ}, \vect{PQ'}) \) and \( \angle(\vect{Q(2Q - P)}, \vect{Q(2Q - P')}) \) have equal measures, the latter angle being the same as to \( \angle(\vect{PQ}, \vect{P'Q'}) \). Hence,
  \begin{equation*}
    \measuredangle(\vect{PQ}, \vect{PQ'}) = \measuredangle(\vect{PQ}, \vect{P'Q}).
  \end{equation*}

  The angle \( \angle(\vect{QP}, \vect{P'Q}) = \angle(\vect{QP}, \vect{Q(2Q - P')}) \) is adjacent to \( \angle(\vect{QP}, \vect{QP'}) \) and the two sum to a straight angle. Hence,
  \begin{equation*}
    \underbrace{\measuredangle(\vect{QP}, \vect{QP'})}_{\measuredangle(\vect{PQ}, \vect{P'Q})} = \pi - \measuredangle(\vect{QP}, \vect{P'Q}),
  \end{equation*}
  thus, combining with the above gives us
  \begin{equation*}
    \measuredangle(\vect{PQ}, \vect{PQ'}) = \measuredangle(\vect{PQ}, \vect{P'Q}) = \pi - \measuredangle(\vect{QP}, \vect{P'Q}).
  \end{equation*}

  Aiming at a contradiction, suppose that \( g \) and \( h \) intersect at some point \( R \). Consider the \( \angle(\vect{PQ}, \vect{PQ'}) = \angle(\vect{PQ}, \vect{PR}) \).

  We will now do a partial proof of \cref{thm:sum_of_triangle_angles}. Consult \cref{fig:thm:sum_of_triangle_angles}.

  Let \( k \) be a line through \( R \) parallel to \( l \) and let \( P^\dprime \) and \( Q^\dprime \) be the orthogonal projections of \( P \) and \( Q \) on \( k \). Note that \( R \) lies on the line segment between \( P^\dprime \) and \( Q^\dprime \) because the orthogonal projection reduces to a translation.

  Then \cref{thm:angles_of_transversal_parallel_lines/parallel} holds for \( g \) intersecting the parallel lines \( k \) and \( l \), and hence \cref{thm:angles_of_transversal_parallel_lines/alternate} implies that the angles \( \angle(\vect{PQ}, \vect{PR}) \) and \( \angle(\vect{RP}, \vect{RP^\dprime}) \) have equal measures as interior alternate angles.

  Similarly, we conclude that \( \angle(\vect{QP}, \vect{QR}) \) and \( \angle(\vect{RQ}, \vect{RQ^\dprime}) \) have equal measures.

  Since \( \angle(\vect{RP^\dprime}, \vect{RQ^\dprime}) \) is a straight angle, it follows from \cref{thm:sum_of_angles_measure} that
  \begin{equation*}
    \angle(\vect{RP}, \vect{RQ}) = \pi - \angle(\vect{PQ}, \vect{PR}) - \angle(\vect{QP}, \vect{QR})
  \end{equation*}

  We have the following possibilities:
  \begin{itemize}
    \item If \( \vect{PR} \) is unidirectional with \( \vect{PQ'} \), then \( \measuredangle(\vect{PQ}, \vect{PR}) = \measuredangle(\vect{PQ}, \vect{PQ'}) \) and \( \measuredangle(\vect{QP}, \vect{QR}) = \measuredangle(\vect{QP}, \vect{P'Q}) = \pi - \measuredangle(\vect{PQ}, \vect{PQ'}) \), and thus \( \measuredangle(\vect{RP}, \vect{RQ}) = 0 \).

    \item Otherwise, we instead have \( \measuredangle(\vect{PQ}, \vect{PR}) = \measuredangle(\vect{PQ}, \vect{Q'P}) \) and \( \measuredangle(\vect{QP}, \vect{QR}) = \measuredangle(\vect{QP}, \vect{QP'}) = \pi - \measuredangle(\vect{PQ}, \vect{Q'P}) \) and thus \( \measuredangle(\vect{RP}, \vect{RQ}) = 0 \).
  \end{itemize}

  In both cases, we obtain that \( \angle(\vect{RP}, \vect{RQ}) \) is a zero angle, and thus the lines \( g \) and \( h \) coincide. But this is not possible since we have assumed that they are distinct.

  The obtained contradiction shows that \( g \) and \( h \) do not intersect.
\end{proof}

  \subsection{Triangles}\label{subsec:triangles}

We will work in the \hyperref[def:euclidean_plane]{Euclidean plane}.

\begin{definition}\label{def:triangle}
  A \hyperref[def:simplex]{\( 2 \)-simplex} is called a \term{triangle}. This definition holds more generally than Euclidean spaces, however we restrict it because we have not defined \hyperref[def:angle]{angles} more generally.

  \begin{figure}[!ht]
    \centering
    \includegraphics[align=c]{output/def__triangle.pdf}
    \caption{An \hyperref[def:triangle/measure/acute]{acute triangle}.}\label{fig:def:triangle}
  \end{figure}

  Given a triangle with vertices \( A \), \( B \) and \( C \), we usually consider its \hyperref[def:angle]{\hi{undirected} angles}
  \begin{align*}
    \alpha &\coloneqq \sphericalangle(\vect{AB}, \vect{AC}), \\
    \beta  &\coloneqq \sphericalangle(\vect{BA}, \vect{BC}), \\
    \gamma &\coloneqq \sphericalangle(\vect{CA}, \vect{CB}).
  \end{align*}

  We say that the segment \( [B, C] \) and the angle \( \sphericalangle(\vect{AB}, \vect{AC}) \) are \term{opposite} to \( A \), and similarly for the other segments and angles.

  We can classify triangles based on their sides as
  \begin{thmenum}
    \thmitem{def:triangle/isosceles} \term{isosceles} if at least two of its sides have equal length
    \thmitem{def:triangle/equilateral} \term{equilateral} if all of its sides have equal length
  \end{thmenum}
  or based on their angles as
  \begin{thmenum}
    \thmitem{def:triangle/measure/acute} \term{acute} if all of its angles are \hyperref[def:angle/measure/acute]{acute}.
    \thmitem{def:triangle/right} \term{right} if at least one of the angles is \hyperref[def:angle/measure/straight]{straight}.
    \thmitem{def:triangle/measure/obtuse} \term{obtuse} if at least one of its angles is \hyperref[def:angle/measure/obtuse]{obtuse}.
  \end{thmenum}
\end{definition}

\begin{proposition}\label{thm:sum_of_triangle_angles}
  The sum of the (measures of) the angles of any \hyperref[def:triangle]{triangle} is \( \pi \).
\end{proposition}
\begin{proof}
  \begin{figure}[!ht]
    \centering
    \includegraphics[align=c]{output/thm__sum_of_triangle_angles.pdf}
    \caption{The construction in our proof of \fullref{thm:sum_of_triangle_angles}.}\label{fig:thm:sum_of_triangle_angles}
  \end{figure}

  Consider a triangle with vertices \( A \), \( B \) and \( C \). Let \( g \) be the line containing \( A \) and \( B \) given by \fullref{thm:lines_intersect_in_plane}, and let \( h \) be the line through \( C \) parallel to \( g \) given by \fullref{thm:parallel_subspace_through_point}.

  Finally, let \( A' \) be the \hyperref[def:orthogonal_projection]{projection} of \( A \) onto \( h \) and \( B' \) be the projection of \( B \) onto \( h \).

  \Fullref{thm:angles_of_transversal} implies that the angles \( \sphericalangle(\vect{AB}, \vect{AC}) \) and \( \sphericalangle(\vect{CA'}, \vect{CA}) \) both have measure \( \alpha \). Similarly, \( \sphericalangle(\vect{BA}, \vect{BC}) \) and \( \sphericalangle(\vect{CB'}, \vect{CB}) \) both have measure \( \beta \).

  From \fullref{thm:adjacent_angles} it follows that
  \begin{equation*}
    \pi
    =
    \sphericalangle(\vect{CA'}, \vect{CB'})
    =
    \sphericalangle(\vect{CA'}, \vect{CA}) + \sphericalangle(\vect{CA}, \vect{CB}) + \sphericalangle(\vect{CB}, \vect{CB'})
    =
    \alpha + \gamma + \beta.
  \end{equation*}
\end{proof}

\begin{corollary}\label{thm:right_triangle}
  A \hyperref[def:triangle/right]{right triangle} has exactly one \hyperref[def:angle/measure/right]{right angle}.

  \begin{figure}[!ht]
    \centering
    \includegraphics[align=c]{output/thm__right_triangle.pdf}
    \caption{A \hyperref[def:triangle/right]{right triangle} with hypotenuse \( [A, B] \).}\label{fig:thm:right_triangle}
  \end{figure}

  We call the opposite side of the right angle the \term[bg=хипотенуза,ru=гипотенуза]{hypotenuse} and the other two sides --- \term[bg=катет,ru=катет]{legs}.
\end{corollary}
\begin{proof}
  There can be only one right angle because, by \fullref{thm:sum_of_triangle_angles}, the other two angles must sum to \( \ifrac \pi 2 \), and one of the angles being zero would violate affine independence of the vertices.
\end{proof}

\begin{remark}\label{rem:polar_coordinate_system}
  A \term{polar coordinate system} given by the ray \( r(t) = O + td \) assigns to each point \( P \neq O \) a unique pair \( (l, \rho) \) of real numbers, where \( l \) is the distance between \( O \) and \( P \) and \( \rho \) is the \hyperref[def:angle]{angle} between \( d \) and \( \vect{OP} \). The number \( l \) is also called the \term{radius} of \( O \).

  \begin{figure}[!ht]
    \centering
    \includegraphics[align=c]{output/rem__polar_coordinate_system.pdf}
    \caption{The \hyperref[rem:polar_coordinate_system]{polar coordinates} of the point \( P \) with respect to the ray \( r \).}\label{fig:rem:polar_coordinate_system}
  \end{figure}

  For simplicity, suppose that \( \norm d = 1 \). Then \fullref{thm:cosine_of_angle_measure} implies that
  \begin{equation*}
    l \cdot \cos \rho = l \cdot \frac {\inprod d {\vect{OP}}} {\norm{\vect{OP}}} = \inprod d {\vect{OP}}.
  \end{equation*}

  Let \( e \) be a rotation of the normed ordinate directional vector by \( \varphi \). Then it is a rotation of the normed abscissa directional vector by \( \varphi + \ifrac \pi 2 \). We have
  \begin{equation*}
    l \cdot \sin \rho
    \reloset {\eqref{eq:thm:trigonometric_function_period_identities/half/sin}} =
    l \cdot \cos(\rho + \ifrac \pi 2)
    =
    \inprod e {\vect{OP}}.
  \end{equation*}

  Therefore, the coordinates of \( P \) with respect to the \hyperref[def:affine_coordinate_system]{affine coordinate system} \( Ode \) are
  \begin{equation}\label{eq:rem:polar_coordinate_system/simple}
    \begin{cases}
      x = l \cdot \cos(\varphi), \\
      y = l \cdot \sin(\varphi).
    \end{cases}
  \end{equation}

  The polar coordinates are unique up to a choice of interval for \( \varphi \).
\end{remark}

\begin{proposition}\label{thm:right_triangle_trigonometric_functions}
  Consider the \hyperref[def:triangle/right]{right triangle} with vertices \( A \), \( B \) and \( C \) and angles \( \alpha \), \( \beta \), \( \gamma \). Without loss of generality, suppose that \( [A, B] \) is the \hyperref[thm:right_triangle]{hypotenuse} .

  \begin{figure}[!ht]
    \centering
    \includegraphics[align=c]{output/thm__right_triangle_trigonometric_functions.pdf}
    \caption{The construction from \fullref{thm:right_triangle_trigonometric_functions}.}\label{fig:thm:right_triangle_trigonometric_functions}
  \end{figure}

  Then
  \begin{align*}
    \sin \alpha = \frac {\norm{\vect{BC}}} {\norm{\vect{AB}}}
    &&
    \cos \alpha = \frac {\norm{\vect{AC}}} {\norm{\vect{AB}}}.
  \end{align*}

  These are the geometric definitions of the trigonometric functions.
\end{proposition}
\begin{proof}
  Let
  \begin{equation*}
    D \coloneqq A + B - C = 2 \frac {A + B} 2 - C
  \end{equation*}
  be the \hyperref[def:rigid_motion/point_reflection]{point reflection} of \( C \) through the \hyperref[thm:segment_midpoint]{midpoint} of \( [A, B] \).

  We have \( \vect{AD} = B - C = \vect{BC} \). Since \( \gamma \) is a right angle, we have
  \begin{equation*}
    0
    =
    \inprod{\vect{AC}} {\vect{BC}}
    =
    \inprod{\vect{AC}} {\vect{AD}}.
  \end{equation*}

  Therefore, the angles \( \sphericalangle(\vect{AC},\vect{AD}) \) and \( \sphericalangle(\vect{BC},\vect{BD}) \) are right.

  Furthermore, \Fullref{thm:angles_of_transversal} implies that
  \begin{equation*}
    \alpha = \sphericalangle(\vect{AB},\vect{AC}) = \sphericalangle(\vect{BA},\vect{BD}).
  \end{equation*}

  Then, considering the \hyperref[rem:polar_coordinate_system]{polar coordinate system} with origin \( A \) and directional vector \( \vect{AC} \), we obtain
  \begin{equation*}
    \norm{AC} = \norm{AB} \cdot \cos \alpha
  \end{equation*}
  and similarly
  \begin{equation*}
    \norm{BC} = \norm{AD} = \norm{AB} \cdot \sin \alpha.
  \end{equation*}
\end{proof}

\begin{definition}\label{def:triangle_median}\mimprovised
  The \term{median} of the side \( [A, B] \) of a triangle \( \conv\set{ A, B, C } \) is the segment from \( C \) to the \hyperref[thm:segment_midpoint]{midpoint} \( M_C \coloneqq \ifrac {A + B} 2 \) of \( [A, B] \). This generalizes in an obvious way to the other sides.
\end{definition}

\begin{proposition}\label{thm:medicenter}
  The \hyperref[def:triangle_median]{medians} of the triangle \( \conv\set{ A, B, C } \) intersect in the point
  \begin{equation*}
    M \coloneqq \frac {A + B + C} 3,
  \end{equation*}
  which we call the \term{medicenter} of the triangle.

  \begin{figure}[!ht]
    \centering
    \includegraphics[align=c]{output/thm__medicenter.pdf}
    \caption{The \hyperref[thm:medicenter]{medicenter} of an \hyperref[def:triangle/measure/acute]{acute triangle}.}\label{fig:thm:medicenter}
  \end{figure}
\end{proposition}
\begin{defproof}
  The median of \( [A, B] \) has directional vector
  \begin{equation*}
    \frac {A + B} 2 - C
    =
    \frac {A + B - 2C} 2
    =
    \frac {(A - C) + (B - C)} 2
    =
    \frac {\vect{AC} + \vect{BC}} 2.
  \end{equation*}

  We take \( \vect{AC} + \vect{BC} \) for simplicity. Then we must find a scalar \( t \) such that
  \begin{equation*}
    M = C + t \vect{AC} + t \vect{BC} = t A + t B + (1 - 2t) C.
  \end{equation*}

  This is obviously satisfied by \( t = \ifrac 1 3 \). Furthermore, generalizing this on the other medians, we obtain that \( M \) is indeed their intersection.
\end{defproof}

\begin{definition}\label{def:triangle_altitude}
  The \term{altitude} of a vertex in a triangle is the \hyperref[def:perpendicularity]{perpendicular} from the vertex to its opposite line. We call the projection the \term{foot} of the altitude.

  For the triangle \( \conv\set{ A, B, C } \), the projection of \( A \) onto the line through \( B \) and \( C \) is
  \begin{equation*}
    O_A
    \coloneqq
    B + \frac {\inprod {\vect{BA}} {\vect{BC}}} {\norm{\vect{BC}}^2} \cdot \vect{BC}
    \reloset {\ref{thm:right_triangle_trigonometric_functions}} =
    B + \cos \beta \cdot \frac {\norm{\vect{BA}}} {\norm{\vect{BC}}} \cdot \vect{BC}.
  \end{equation*}

  The corresponding altitude is the segment \( [A, O_A] \).
\end{definition}

\begin{proposition}\label{thm:orthocenter}\mimprovised
  The \hyperref[def:triangle_altitude]{altitudes} of a triangle intersect in a point, which we call the \term{orthocenter}.

  \begin{figure}[!ht]
    \centering
    \includegraphics[align=c]{output/thm__orthocenter.pdf}
    \caption{The \hyperref[thm:orthocenter]{orthocenter} of an \hyperref[def:triangle/measure/acute]{acute triangle}.}\label{fig:thm:orthocenter}
  \end{figure}
\end{proposition}
\begin{defproof}
  Consider the triangle \( \conv\set{ A, B, C } \).

  We will first show that the vectors \( \vect{A O_A} \) and \( \vect{B O_B} \) are linearly independent. Note that \( \vect{B O_B} \) must be perpendicular to the line through \( A \) and \( C \). If \( \vect{A O_A} \) and \( \vect{B O_B} \) are linearly dependent, then  \( \vect{A O_A} \) must be perpendicular to both the line through \( A \) and \( C \) and the line through \( B \) and \( C \). This implies that the lines are parallel, which contradicts the affine independence of \( A \), \( B \) and \( C \).

  Therefore, it follows from \fullref{def:crossing_lines} that the lines \( A + t \vect{A O_A} \) and \( B + t \vect{B O_B} \) intersect in exactly one point.

  Let \( O \) be the intersection point
  \begin{equation*}
    O = A + t_0 \vect{A O_A} = B + r_0 \vect{B O_B}.
  \end{equation*}

  It remains to show that the vector \( \vect{CO} \) is orthogonal to \( \vect{AB} \). We have
  \begin{align*}
    \vect 0 - \vect 0
    &=
    \inprod {\vect{AO}} {\vect{BC}} - \inprod {\vect{BO}} {\vect{AC}}
    = \\ &=
    \inprod {\vect{CA}} {\vect{BC}} + \inprod {\vect{AO}} {\vect{BC}} - \inprod {\vect{CA}} {\vect{BC}} - \inprod {\vect{BO}} {\vect{AC}}
    = \\ &=
    \inprod {\vect{CA} + \vect{AO}} {\vect{BC}} - \inprod {\vect{CB} + \vect{BO}} {\vect{AC}}
    = \\ &=
    \inprod {\vect{CO}} {\vect{BC}} - \inprod {\vect{CO}} {\vect{AC}}
    = \\ &=
    \inprod {\vect{CO}} {\vect{BC} - \vect{AC}}
    = \\ &=
    \inprod {\vect{CO}} {\vect{BA}}.
  \end{align*}

  Therefore, \( O \) is the intersection of the altitudes.
\end{defproof}

\begin{proposition}\label{thm:acute_triangle_orthocenter}
  In an \hyperref[def:triangle/measure/acute]{acute triangle}, the \hyperref[thm:orthocenter]{orthocenter} lies in the interior of the triangle.

  In particular, the foot of each \hyperref[def:triangle_altitude]{altitude} lies on its opposite side.
\end{proposition}
\begin{proof}
  Consider the acute triangle \( \conv\set{ A, B, C } \) with angles \( \alpha \), \( \beta \) and \( \gamma \). The cosine of each angles is positive because the angles are acute.

  Consider the projection
  \begin{equation*}
    O_C = A + \cos \alpha \cdot \frac {\norm{\vect{AC}}} {\norm{\vect{AB}}} \cdot \vect{AB}.
  \end{equation*}

  The corresponding altitude \( \vect{A O_C} \) is unidirectional with \( \vect{AB} \) because
  \begin{equation*}
    \cos \alpha = \frac {\inprod{ \vect{AB} } { \vect{AC}} } { \norm{\vect{AB}} \cdot \norm{\vect{AC}} } > 0.
  \end{equation*}

  Furthermore, analogously we obtain that \( \vect{B O_C} \) is unidirectional with \( \vect{BA} \). Therefore, \( O_C \) lies on \( [A, B] \).

  It also follows similarly that \( O_A \) belongs to \( [B, C] \) and \( O_B \) belongs to \( [A, C] \).

  The interior of the segments \( [A, A_C] \), \( [B, B_C] \) and \( [C, O_C] \) entirely belong to the interior of the triangle. Their intersection must then also belong to the interior.
\end{proof}

\begin{proposition}\label{thm:perimeter_of_triangle}
  The \hyperref[def:perimeter]{perimeter} of a \hyperref[def:triangle]{triangle} is the sum of the lengths of its sides.
\end{proposition}
\begin{proof}
  Denote the triangle by \( T \) and its vertices by \( A \), \( B \) and \( C \). We will first determine the \hyperref[def:topological_boundary_operator]{topological boundary} \( \fr T \).

  Consider a point \( P \) of \( T \). Let \( P_a \) be the \hyperref[def:orthogonal_projection]{orthogonal projection} of \( P \) onto the line through \( B \) and \( C \), \( P_b \) --- onto the line through \( A \) and \( C \) and \( P_c \) --- onto the line through \( A \) and \( B \).

  If \( P = P_a \), then \( P \) lies on the segment \( [B, C] \). One of the half-spaces separated by the line through \( B \) and \( C \) contains \( T \), the other one doesn't. Hence, every ball around \( P \) contains points from both half-spaces, and hence a point from \( T \) and a point outside \( T \). Thus, \( P \) lies on the boundary of \( T \).

  The cases \( P = P_b \) and \( P = P_c \) are identical. We conclude that the segments \( [A, B] \), \( [B, C] \) and \( [C, A] \) lie on the boundary.

  If \( P \not\in \set{ P_a, P_b, P_c } \), then \( T \) entirely contains the ball with center \( P \) and radius
  \begin{equation*}
    \varepsilon < \min\set{ \norm{\vect{P P_a}}, \norm{\vect{P P_b}}, \norm{\vect{P P_c}} }.
  \end{equation*}

  Indeed, if \( \norm{\vect{PQ}} < \varepsilon \) for some point \( Q \), then
  \begin{equation*}
    \norm{\vect{P P_a}}
    =
    \norm{\vect{P Q} + \vect{Q P_a}}
    \leq
    \norm{\vect{P Q}} + \norm{\vect{Q P_a}}
    <
    \varepsilon + \norm{\vect{Q P_a}}.
  \end{equation*}

  Hence,
  \begin{equation*}
    \norm{\vect{Q P_a}} < \norm{\vect{P P_a}} - \varepsilon < \norm{\vect{P P_a}},
  \end{equation*}
  implying that \( P \) and \( Q \) lie in the same half-plane with respect to the line through \( B \) and \( C \).

  By considering the other sides of \( T \), we conclude that \( Q \) is in \( T \), making \( P \) an interior point.

  Therefore, the topological boundary of \( T \) is \( [A, B] \cup [B, C] \cup [C, A] \). It can be described as the parametric curve
  \begin{equation*}
    \gamma(t) \coloneqq \begin{cases}
      A + t \vect{AB},       &0 \leq t < 1, \\
      B + (t - 1) \vect{BC}, &1 \leq t < 2, \\
      C + (t - 2) \vect{CA}, &2 \leq t \leq 3.
    \end{cases}
  \end{equation*}

  \Fullref{thm:length_of_piecewise_smooth_curves} implies that the perimeter of this curve is
  \begin{equation*}
    \norm{\vect{AB}} + \norm{\vect{BC}} + \norm{\vect{CA}}.
  \end{equation*}
\end{proof}

\begin{definition}\label{def:trapezoid}\mimprovised
  A \term[bg=трапец,ru=трапеция]{trapezoid} is the \hyperref[def:convex_hull]{convex hull} of four points \( A \), \( B \), \( C \) and \( D \) called \term{vertices}, no three of which are \hyperref[def:collinear_points]{collinear}, such that, up to a relabeling,
  \begin{equation*}
    \frac {\vect{AB}} {\norm{\vect{AB}}} = -\frac {\vect{CD}} {\norm{\vect{CD}}}.
  \end{equation*}

  \begin{figure}[!ht]
    \centering
    \includegraphics[align=c]{output/def__trapezoid.pdf}
    \caption{A \hyperref[def:trapezoid]{trapezoid}.}\label{fig:def:trapezoid}
  \end{figure}

  We call the \hyperref[def:line_segment]{line segments} \( [A, B] \), \( [B, C] \), \( [C, D] \) and \( [D, A] \) \term{sides}, the sides \( [A, B] \) and \( [C, D] \) --- \term{bases} of the trapezoid, and the segments \( [A, C] \) and \( [B, D] \) --- \term{diagonals}.

  \begin{equation*}
    \vect{AB} = -\frac {\norm{\vect{AB}}} {\norm{\vect{CD}}} \vect{CD}.
  \end{equation*}
\end{definition}

\begin{proposition}\label{thm:diagonal_trapezoid_triangulation}
  Consider a \hyperref[def:trapezoid]{trapezoid} with vertices \( A \), \( B \), \( C \) and \( D \). Then
  \begin{equation*}
    \underbrace{\conv\set{ A, B, C, D }}_{\T{trapezoid}} = \conv\set{ A, B, C } \cup \conv\set{ A, D, C }.
  \end{equation*}

  We call this decomposition the \term{main diagonal triangulation} of the trapezoid.

  \begin{figure}[!ht]
    \centering
    \includegraphics[align=c]{output/thm__trapezoid_diagonal_triangulation.pdf}
    \caption{\hyperref[thm:diagonal_trapezoid_triangulation]{Diagonal trapezoid triangulation}.}\label{fig:thm:diagonal_trapezoid_triangulation}
  \end{figure}
\end{proposition}
\begin{proof}
  No three vertices of the trapezoid are collinear, hence \( \conv\set{ A, B, C } \) and \( \conv\set{ A, D, C } \) are indeed triangles. Obviously every convex combination of a subset of the vertices belongs to the trapezoid, hence
  \begin{equation*}
    \conv\set{ A, B, C } \cup \conv\set{ A, D, C } \subseteq \conv\set{ A, B, C, D }.
  \end{equation*}

  For the converse, first note that
  \begin{equation*}
    \vect{CD} = -\frac {\norm{\vect{CD}}} {\norm{\vect{AB}}} \vect{AB},
  \end{equation*}
  thus
  \begin{equation*}
    D = C + \vect{CD} = C - \frac {\norm{\vect{CD}}} {\norm{\vect{AB}}} \vect{AB} = \frac {\norm{\vect{CD}}} {\norm{\vect{AB}}} A - \frac {\norm{\vect{CD}}} {\norm{\vect{AB}}} B + C.
  \end{equation*}

  Then, for any convex combination
  \begin{equation*}
    P = aA + bB + cC + dD
  \end{equation*}
  we have
  \begin{equation*}
    P
    =
    \parens*{ a + d \frac {\norm{\vect{CD}}} {\norm{\vect{AB}}} } A +
    \parens*{ b - d \frac {\norm{\vect{CD}}} {\norm{\vect{AB}}} } B +
    (c + d) C.
  \end{equation*}

  This is a convex combination of \( A \), \( B \) and \( C \) if
  \begin{equation*}
    b < d \frac {\norm{\vect{CD}}} {\norm{\vect{AB}}}.
  \end{equation*}

  Otherwise, note that
  \begin{equation*}
    B = A + \vect{AB} = A - \frac {\norm{\vect{AB}}} {\norm{\vect{CD}}} \vect{CD} = A + \frac {\norm{\vect{AB}}} {\norm{\vect{CD}}} C - \frac {\norm{\vect{AB}}} {\norm{\vect{CD}}} D,
  \end{equation*}
  hence the following is a convex combination:
  \begin{equation*}
    P
    =
    (a + b) A +
    \parens*{ c + b \frac {\norm{\vect{AB}}} {\norm{\vect{CD}}} } C +
    \parens*{ d - b \frac {\norm{\vect{AB}}} {\norm{\vect{CD}}} } D.
  \end{equation*}

  Therefore, every point in \( \conv\set{ A, B, C, D } \) is either in \( \conv\set{ A, B, C } \) or in \( \conv\set{ A, D, C } \).
\end{proof}

\begin{proposition}\label{thm:trapezoid_is_polytope}
  A \hyperref[def:trapezoid]{trapezoid} is a \hyperref[def:convex_polytope]{convex polytope} whose \hyperref[def:extremal_point]{extremal points} are its vertices.
\end{proposition}
\begin{proof}
  \Fullref{thm:extremal_points_of_convex_hull} implies that the extremal points are a subset of the vertices. \Fullref{thm:diagonal_trapezoid_triangulation} implies that all the vertices are extremal points of the triangles in the main diagonal triangulation. If we assume that a vertex belongs to the interior of an edge, then we would obtain that the three points are collinear. Hence, all four vertices are extremal points.

  We must also show that trapezoids are convex polytopes. Triangles are simplices and simplices are convex polytopes. Both of the half-planes corresponding to the main diagonal line are used, one for each triangle. The union of the two triangles is the intersection of all half-planes except those two. Therefore, the trapezoid is a convex polytope.
\end{proof}

\begin{proposition}\label{thm:perimeter_of_trapezoid}
  The \hyperref[def:perimeter]{perimeter} of a \hyperref[def:trapezoid]{trapezoid} is the sum of the lengths of its sides.
\end{proposition}
\begin{proof}
  \Fullref{thm:diagonal_trapezoid_triangulation} and \fullref{thm:perimeter_of_triangle} imply that the boundary of the trapezoid consists of its sides. The diagonal, which is a side of both triangles in the main diagonal triangulation, now lies in the interior.

  Therefore, the perimeter is the sum of the sides.
\end{proof}

\begin{definition}\label{def:parallelogram}
  A \term[bg=успоредник,ru=параллелограмм]{parallelogram} is a \hyperref[def:trapezoid]{trapezoid} whose two non-base sides are also parallel.

  \begin{figure}[!ht]
    \centering
    \includegraphics[align=c]{output/def__parallelogram.pdf}
    \caption{A generic \hyperref[def:parallelogram]{parallelogram}.}\label{fig:def:parallelogram}
  \end{figure}

  \begin{thmenum}
    \thmitem{def:parallelogram/rhombus} A \term{rhombus} is a parallelogram whose sides all have equal lengths.

    \thmitem{def:parallelogram/rectangle} A \term{rectangle} is a parallelogram where the sides with common vertices are \hyperref[def:perpendicularity]{perpendicular}.

    \thmitem{def:parallelogram/square} A \term{square} is a parallelogram that is both a rhombus and a rectangle.

    \begin{figure}[!ht]
      \hfill
      \includegraphics[align=c]{output/def__parallelogram__rhombus.pdf}
      \hfill
      \hfill
      \includegraphics[align=c]{output/def__parallelogram__rectangle.pdf}
      \hfill
      \hfill
      \includegraphics[align=c]{output/def__parallelogram__square.pdf}
      \hfill
      \hfill
      \caption{A \hyperref[def:parallelogram/rhombus]{rhombus}, \hyperref[def:parallelogram/rectangle]{rectangle} and \hyperref[def:parallelogram/square]{square}.}\label{fig:def:parallelogram/rhombus}
    \end{figure}
  \end{thmenum}
\end{definition}

\begin{definition}\label{def:figure_area}\mimprovised
  The \term{area} of a \hyperref[rem:geometric_shape]{geometric figure} is its \hyperref[def:lebesgue_measure]{Lebesgue measure}.
\end{definition}

\begin{proposition}\label{thm:area_of_triangle}
  The \hyperref[def:figure_area]{area} of a \hyperref[def:triangle]{triangle} with vertices \( A \), \( B \) and \( C \) and angles \( \alpha \), \( \beta \) and \( \gamma \) is
  \begin{equation}\label{eq:thm:area_of_triangle}
    \frac 1 2 \cdot \norm{\vect{AB}} \cdot \norm{\vect{AC}} \cdot \sin \alpha = \frac 1 2 \cdot \norm{\vect{BA}} \cdot \norm{\vect{BC}} \cdot \sin \beta = \frac 1 2 \cdot \norm{\vect{CA}} \cdot \norm{\vect{CB}} \cdot \sin \gamma.
  \end{equation}
\end{proposition}
\begin{proof}
  We will only prove the theorem for \( \alpha \). The other cases follow automatically by simply relabeling the vertices.

  \SubProof{Proof when \( \alpha \) is right} Consider the \hyperref[def:rigid_motion/point_reflection]{point reflection}
  \begin{equation*}
    A' \coloneqq B + C - A
  \end{equation*}
  of the \hyperref[thm:segment_midpoint]{midpoint} \( \ifrac {B + C} 2 \) of \( [B, C] \) through \( A \).

  Then the vertices \( A \), \( B \), \( C \) and \( A' \) form a \hyperref[def:parallelogram/rectangle]{rectangle}. Indeed,
  \begin{equation*}
    \vect{BA'} = \vect{BB} + \vect{BC} - \vect{BA} = \vect{AB} + \vect{BC} = \vect{AC}
  \end{equation*}
  and
  \begin{equation*}
    \vect{CA'} = \vect{CB} + \vect{CC} - \vect{CA} = \vect{AC} + \vect{CB} = \vect{AB}.
  \end{equation*}

  \Fullref{thm:angles_of_transversal} then implies that the angles of the corners are right. Hence, this is indeed a rectangle.

  The Lebesgue measure is invariant under rotations and translations as a consequence of \fullref{thm:lebesgue_measure_invariant_under_rigid_motions}. Hence, we can translate our rectangle by \( -A \) and rotate by \( -\sphericalangle(\vect{AB}, (1, 0)) \) to obtain the rectangle as the Cartesian product of intervals
  \begin{equation*}
    \bracks*{ 0, \norm{\vect{AB}} } \times \bracks*{ 0, \norm{\vect{AC}} }.
  \end{equation*}

  This rectangle has Lebesgue measure \( \norm{\vect{AB}} \cdot \norm{\vect{AC}} \).

  The triangle \( \conv\set{ A', B, C } \) is the \hyperref[def:rigid_motion/householder_reflection]{Householder reflection} of \( \conv\set{ A, B, C } \) about the line through \( B \) and \( C \). Hence, both triangles have equal measures, and the sum of their measures is the measure of the rectangle \( \conv\set{ A, B, A', C } \). Therefore, the area of \( \conv\set{ A, B, C } \) is
  \begin{equation*}
    \frac 1 2 \cdot \norm{\vect{AB}} \cdot \norm{\vect{AC}} \cdot \underbrace{ \sin(\alpha) }_{1}.
  \end{equation*}

  \begin{figure}[!ht]
    \centering
    \includegraphics[align=c]{output/thm__area_of_triangle__right.pdf}
    \caption{Finding the \hyperref[def:figure_area]{area} of a triangle via a \hyperref[def:angle/measure/right]{right angle}.}\label{fig:thm:area_of_triangle/right}
  \end{figure}

  \SubProof{Proof when \( \alpha \) is acute} Let \( O_C \) be the foot of the \hyperref[def:triangle_altitude]{altitude} of \( C \). That is,
  \begin{equation*}
    \vect{A O_C}
    =
    \frac {\inprod {\vect{AC}} {\vect{AB}}} {\norm{\vect{AB}}^2} \cdot \vect{AB}
    =
    \frac {\norm{\vect{AC}}} {\norm{\vect{AB}}} \cdot \cos \alpha \cdot \vect{AB}.
  \end{equation*}

  The triangle \( \conv\set{ A, O_C, C } \) is right, hence its area is
  \begin{equation*}
    \frac 1 2 \cdot \norm{\vect{ O_CA}} \cdot \norm{\vect{O_C C}}.
  \end{equation*}

  Similarly, the area of the triangle \( \conv\set{ B, O_C, C } \) is
  \begin{equation*}
    \frac 1 2 \cdot \norm{\vect{O_C B}} \cdot \norm{\vect{O_C C}}.
  \end{equation*}

  \Fullref{thm:right_triangle_trigonometric_functions} implies that \( \norm{\vect{O_C C}} = \norm{\vect{AC}} \cdot \sin \alpha \).

  \begin{itemize}
    \item If \( \norm{A O_C} > \norm{A B} \), then \( O_C \) lies outside the triangle, and we need to subtract the area of \( \conv\set{ B, C, O_C } \) from that of \( \conv\set{ A, C, O_C } \). Both are right triangles, and we have already shown that \eqref{eq:thm:area_of_parallelogram} holds for right angles. The area of \( \conv\set{ A, B, C } \) is then
    \begin{equation*}
      \frac 1 2 \cdot \parens[\Big]{ \norm{\vect{A O_C}} - \norm{\vect{O_C B}} } \cdot \norm{\vect{O_C C}} = \frac 1 2 \cdot \norm{\vect{AB}} \cdot \norm{\vect{AC}} \cdot \sin \alpha.
    \end{equation*}

    \item If \( \norm{A O_C} \leq \norm{A B} \), then \( O_C \) lies in the triangle, and we need to add rather than subtract. The area of \( \conv\set{ A, B, C } \) is then
    \begin{equation*}
      \frac 1 2 \cdot \parens[\Big]{ \norm{\vect{A O_C}} + \norm{\vect{O_C B}} } \cdot \norm{\vect{O_C C}} = \frac 1 2 \cdot \norm{\vect{AB}} \cdot \norm{\vect{AC}} \cdot \sin \alpha.
    \end{equation*}
  \end{itemize}

  \begin{figure}[!ht]
    \hfill
    \hfill
    \includegraphics[align=c]{output/thm__area_of_triangle__acute__full.pdf}
    \hfill
    \includegraphics[align=c]{output/thm__area_of_triangle__acute__partial.pdf}
    \hfill
    \caption{Finding the \hyperref[def:figure_area]{area} of a triangle via an \hyperref[def:angle/measure/acute]{acute angle}.}\label{fig:thm:area_of_triangle/measure/acute}
  \end{figure}

  \SubProof{Proof when \( \alpha \) is obtuse} Consider the angle \( \sphericalangle(\vect{A O_C}, \vect{A C}) \). \Fullref{thm:adjacent_angles} implies that its measure is \( \pi - \alpha \). Then
  \begin{equation*}
    \sin \alpha
    \reloset {\eqref{eq:thm:trigonometric_function_period_identities/full/sin}} =
    -\sin(\pi - \alpha)
    =
    -\sin \sphericalangle(\vect{A O_C}, \vect{A C}).
  \end{equation*}

  \Fullref{thm:right_triangle_trigonometric_functions} in turn implies that
  \begin{equation*}
    \sin \sphericalangle(\vect{A O_C}, \vect{A C})
    =
    \frac {\norm{\vect{C O_C}}} {\norm{\vect{CA}}}.
  \end{equation*}

  The point \( O_C \) is outside the triangle, and \( \vect{O_C A} \) is unidirectional with \( \vect{AB} \). Therefore, the area of \( \conv\set{ A, B, C } \) is the area of \( \conv\set{ O_C, B, C } \) minus the area of \( \conv\set{ O_C, A, C } \). Both are right triangles, and we have already shown that \eqref{eq:thm:area_of_parallelogram} holds for right angles. Hence, the area of \( \conv\set{ A, B, C } \) is
  \begin{equation*}
    \frac 1 2 \cdot \parens[\Big]{ \norm{\vect{O_C B}} - \norm{\vect{O_C A}} } \cdot \norm{\vect{O_C C}}
    =
    \frac 1 2 \cdot \parens[\Big]{ -\norm{\vect{B A}} } \cdot \parens[\Big]{ -\sin \alpha \cdot \norm{\vect{CA}} }
    =
    \frac 1 2 \cdot \norm{\vect{AB}} \cdot \norm{\vect{AC}} \cdot \sin \alpha.
  \end{equation*}

  \begin{figure}[!ht]
    \centering
    \includegraphics[align=c]{output/thm__area_of_triangle__obtuse.pdf}
    \caption{Finding the \hyperref[def:figure_area]{area} of a triangle via an \hyperref[def:angle/measure/obtuse]{obtuse angle}.}\label{fig:thm:area_of_triangle/measure/obtuse}
  \end{figure}
\end{proof}

\begin{proposition}\label{thm:area_of_parallelogram}
  The \hyperref[def:figure_area]{area} of a \hyperref[def:parallelogram]{parallelogram} with vertices \( A \), \( B \), \( C \) and \( D \) is the product of the lengths of adjacent sides:
  \begin{equation}\label{eq:thm:area_of_parallelogram}
    \norm{\vect{AB}} \cdot \norm{\vect{AD}} \cdot \sin\sphericalangle(\vect{AB}, \vect{AD}).
  \end{equation}
\end{proposition}
\begin{proof}
  For a general parallelogram, we instead take the \hyperref[thm:diagonal_trapezoid_triangulation]{main diagonal triangulation} producing the triangles \( \conv\set{ A, B, C } \) and \( \conv\set{ A, D, C } \), and note that \( \conv\set{ A, D, C } \) is obtained from \( \conv\set{ A, B, C } \) through reflection about the line through \( A \) and \( C \). \Fullref{thm:lebesgue_measure_invariant_under_rigid_motions} implies that the two triangles have the same measure.

  \Fullref{thm:area_of_triangle} implies that the measure of \( \conv\set{ A, B, C } \) is
  \begin{equation*}
    \frac 1 2 \cdot \norm{\vect{AB}} \cdot \norm{\vect{AD}} \cdot \sin\sphericalangle(\vect{AB}, \vect{AD}).
  \end{equation*}

  Multiplying by \( 2 \), \eqref{eq:thm:area_of_parallelogram} follows.
\end{proof}

\begin{theorem}[Law of sines]\label{thm:law_of_sines}
  For any \hyperref[def:triangle]{triangle} with vertices \( A \), \( B \) and \( C \) and angles \( \alpha \), \( \beta \) and \( \gamma \), we have
  \begin{equation}\label{eq:thm:law_of_sines}
    \frac {\sin \alpha} {\norm{\vect{BC}}}
    =
    \frac {\sin \beta} {\norm{\vect{AC}}}
    =
    \frac {\sin \gamma} {\norm{\vect{AB}}}.
  \end{equation}
\end{theorem}
\begin{proof}
  Follows from \fullref{thm:area_of_triangle} because
  \begin{equation*}
    \frac 1 2 \cdot \norm{\vect{AB}} \cdot \norm{\vect{AC}} \cdot \sin \alpha
    =
    \frac 1 2 \cdot \norm{\vect{BA}} \cdot \norm{\vect{BC}} \cdot \sin \beta
  \end{equation*}
  implies
  \begin{equation*}
    \frac {\sin \alpha} {\norm{\vect{BC}}}
    =
    \frac {\sin \beta} {\norm{\vect{AC}}}.
  \end{equation*}
\end{proof}

\begin{theorem}[Law of cosines]\label{thm:law_of_cosines}
  For any \hyperref[def:triangle]{triangle} with vertices \( A \), \( B \) and \( C \) and angles \( \alpha \), \( \beta \) and \( \gamma \), we have
  \begin{equation}\label{eq:thm:law_of_cosines}
    \norm{\vect{BC}}^2 = \norm{\vect{AB}}^2 + \norm{\vect{AC}}^2 - 2 \cdot \norm{\vect{AB}} \cdot \norm{\vect{AC}} \cdot \cos \alpha,
  \end{equation}
  and similarly for the other sides and angles.

  The simplified case with a right angle is known as \enquote{Pythagoras' theorem}. We have stated \fullref{thm:pythagoras_theorem} in more generality.
\end{theorem}
\begin{proof}
  Clearly
  \begin{equation*}
    \norm{\vect{BC}}^2
    =
    \norm{\vect{AC} - \vect{AB}}^2
    =
    \norm{\vect{AC}}^2 - 2 \inprod{\vect{AC}}{\vect{AB}} + \norm{\vect{AB}}^2.
  \end{equation*}

  Then \eqref{eq:thm:law_of_cosines} follows from \fullref{thm:cosine_of_angle_measure}.
\end{proof}

  \section{Quadratic plane curves}\label{sec:quadratic_plane_curves}

This subsection is dedicated to \hyperref[def:parametric_curve]{curves} in \( \BbbR^2 \) described by \hyperref[def:polynomial_degree]{quadratic polynomials}. We first formalize the meaning of \hyperref[def:affine_algebraic_set/variety]{algebraic variety} and \hyperref[def:affine_algebraic_set/curve]{algebraic curve}, but do not go into algebraic geometry beyond that. After that, we restrict ourselves to the classical theory of quadratic curves.

\begin{definition}\label{def:quadratic_plane_curve}\mimprovised
  A \term{quadratic plane curve} is the \hyperref[def:root_of_polynomial]{set of zeros} of a bivariate quadratic polynomial in the \hyperref[def:euclidean_plane]{Euclidean plane} \( \BbbR^2 \).

  Those that are \hyperref[def:affine_algebraic_set/curve]{algebraic curves} are precisely the \hyperref[def:ellipse]{ellipses}, \hyperref[def:hyperbola]{hyperbolas} and \hyperref[def:parabola]{parabolas} with their standard equations, as well as the imaginary ellipses described in \fullref{ex:imaginary_ellipse}. We call the other quadratic plane curves \term{degenerate}.

  \Fullref{alg:canonization_of_quadratic_plane_curves} describes a procedure for finding an \hyperref[def:affine_coordinate_system]{affine coordinate system} in \( \BbbR^2 \) in which we can easily classify the type of algebraic curve. \Fullref{thm:change_of_polynomial_basis} ensures that irreducible polynomials are invariant under affine changes of coordinates.

  \begin{figure}[!ht]
    \hfill
    \hfill
    \includegraphics[align=c]{output/def__conic_section__ellipse}
    \hfill
    \includegraphics[align=c]{output/def__conic_section__hyperbola}
    \hfill
    \includegraphics[align=c]{output/def__conic_section__parabola}
    \hfill
    \caption{An \hyperref[def:ellipse]{ellipse}, \hyperref[def:hyperbola]{hyperbola} and \hyperref[def:parabola]{parabola} defined via their parametric equations. The starting point is highlighted and the direction of the parametric curves is shown.}\label{fig:def:quadratic_plane_curve}
  \end{figure}
\end{definition}

\begin{proposition}\label{thm:quadratic_polynomial_irreducibility}
  For a quadratic polynomial \( p(X, Y) \) in two indeterminates over an \hyperref[def:algebraically_closed_field]{algebraically closed} \hyperref[def:field]{field}, the \hyperref[def:affine_algebraic_set]{affine algebraic set} \( \mscrV(\braket{ p(X, Y) }) \) is an \hyperref[def:affine_algebraic_set/variety]{algebraic curve} if and only if \( p(X, Y) \) is an \hyperref[def:domain_divisibility/irreducible]{irreducible polynomial}.
\end{proposition}
\begin{proof}
  For a field \( \BbbK \), by \fullref{thm:polynomial_ring_over_gcd_domain}, \( \BbbK[X, Y] \) is also a GCD domain, and, by \fullref{thm:def:gcd_domain/irreducible_is_prime}, every \hyperref[def:domain_divisibility/irreducible]{irreducible polynomial} is \hyperref[def:domain_divisibility/prime]{prime}.

  Thus, if \( p(X, Y) \) is a nonconstant \hyperref[def:domain_divisibility/irreducible]{irreducible polynomial}, then the following \hyperref[def:hasse_diagram]{Hasse diagram} shows how the principal ideal of \( \braket{ p(X, Y) } \) relates to other prime ideals
  \begin{equation*}
    \begin{aligned}
      \includegraphics[page=1]{output/ex__quadratic_curves}
    \end{aligned}
  \end{equation*}

  By \fullref{thm:def:krull_dimension/polynomial_ring}, the \hyperref[def:krull_dimension]{Krull dimension} of \( \BbbK[X, Y] \) is \( 2 \). Hence, by \fullref{thm:lattice_theorem_for_ideals}, the ascending sequence of \hyperref[def:ring/quotient]{quotients}
  \begin{equation*}
    \frac {\braket{ p(X, Y) }} {\braket{ p(X, Y) }} \subsetneq \frac {\braket{ X, Y }} {\braket{ p(X, Y) }} \subsetneq \frac {\BbbK[X, Y]} {\braket{ p(X, Y) }}
  \end{equation*}
  is a maximal ascending sequence of prime ideals.

  Therefore, the coordinate ring of \( \mscrV(\braket{ p(X, Y) }) \) is unidimensional, and hence the affine algebraic set itself is an \hyperref[def:affine_algebraic_set/curve]{algebraic curve}.
\end{proof}

\begin{example}\label{ex:real_algebraic_curves}
  Suppose that affine varieties are defined for fields that are not algebraically closed.

  The polynomial \( X^2 + Y^2 \) is irreducible \( \BbbR[X, Y] \) as shown in \fullref{thm:axn_byn_irreducible}. \Fullref{thm:quadratic_polynomial_irreducibility} implies that the set
  \begin{equation*}
    \mscrV(\braket{ X^2 + Y^2 }) = \set{ (0, 0) }
  \end{equation*}
  is an algebraic curve. It is not a curve intuitively, but it fits the definition.

  To avoid degenerate cases like this one, we restrict the definition of affine variety to algebraically closed fields. In this case, \( X^2 + Y^2 = (X - iY) (X + iY) \) is a reducible polynomial in \( \BbbC[X, Y] \), and hence does not induce an algebraic curve.

  We have described in \fullref{rem:real_affine_varieties} our approach to algebraic curves in \( \BbbR^2 \). It still has pathologies --- see \fullref{ex:imaginary_ellipse} --- but much less of them.
\end{example}

\begin{definition}\label{def:lame_curve}\mimprovised
  A \term{Lam\'{e} curve} of degree \( n \) over the \hyperref[def:field]{field} \( \BbbK \) is the \hyperref[def:algebraic_curve]{algebraic curve} given by the \hyperref[def:algebraic_equation]{algebraic equation}
  \begin{equation}\label{eq:def:lame_curve}
    \frac {X^n} {a^n} + \frac {Y^n} {b^n} = 1.
  \end{equation}
\end{definition}
\begin{comments}
  \item Our definition is based on \cite[179]{Савелов1960ПлоскиеКривые}, however we generalize the definition from the real numbers to arbitrary fields at the cost of restricting \( n \) an integer. Gabriel Lam\'{e} himself used these curves to prove a special case of \fullref{thm:fermats_last_theorem}.
\end{comments}
\begin{defproof}
  \Fullref{thm:axn_byn_czn_irreducible} implies that the trivariate polynomial
  \begin{equation*}
    \frac {X^n} {a^n} + \frac {Y^n} {b^n} - Z^n
  \end{equation*}
  is irreducible, and \fullref{thm:homogeneous_polynomial_constant} implies that the bivariate polynomial from \eqref{eq:def:lame_curve} is irreducible.

  Then \fullref{thm:quadratic_polynomial_irreducibility} implies that the corresponding set of solutions is an algebraic curve.
\end{defproof}

\begin{definition}\label{def:fermat_curve}\mimprovised
  A \term{Fermat curve} of degree \( n > 1 \) over the \hyperref[def:field]{field} \( \BbbK \) is the \hyperref[def:lame_curve]{Lam\'{e} curve}
  \begin{equation}\label{eq:def:fermat_curve}
    X^n + Y^n = 1.
  \end{equation}
\end{definition}
\begin{comments}
  \item For this definition, we generalize the conventional equation of a circle. Fermat curves relate to \fullref{thm:fermats_last_theorem} via \fullref{thm:fermat_curve_rational_points_via_fermat_triples}.
\end{comments}

\begin{definition}\label{def:ellipse}
  An \term{ellipse} is, up to a choice of \hyperref[def:affine_coordinate_system]{affine coordinate system}, the \hyperref[def:root_of_polynomial]{set of zeros} of the polynomial
  \begin{equation}\label{eq:def:ellipse/polynomial}
    \frac {X^2} {a^2} + \frac {Y^2} {b^2} - 1
  \end{equation}
  for some real numbers \( a > b > 0 \), which we call the big and small \term{radii}.

  The polynomial is unique in the sense that no two polynomials of the form \eqref{eq:def:hyperbola/polynomial} have the same set of zeros. In particular, the radii are well-defined.

  It is an \hyperref[def:affine_algebraic_set/curve]{algebraic curve} because, by \fullref{thm:axn_byn_czn_irreducible} and \fullref{thm:homogeneous_polynomial_constant}, the polynomial is irreducible over \( \BbbC \).

  \begin{thmenum}
    \thmitem{def:ellipse/standard_equation} The set of zeros of \eqref{eq:def:ellipse/polynomial} consists of all pairs \( (x, y) \) of real numbers satisfying
    \begin{equation}\label{eq:def:ellipse/standard_equation}
      \frac {x^2} {a^2} + \frac {y^2} {b^2} = 1.
    \end{equation}

    We call \eqref{eq:def:ellipse/standard_equation} the \term{standard equation} of the ellipse.

    \thmitem{def:ellipse/parametric_equation} Ellipses can also be described via the \hyperref[def:parametric_curve]{parametric curve}
    \begin{equation}\label{eq:def:ellipse/parametric_equation}
      \begin{cases}
        x = a \cdot \cos \varphi, \\
        y = b \cdot \sin \varphi,
      \end{cases}
    \end{equation}
    where \( \varphi \in [0, 2\pi) \).
  \end{thmenum}

  \thmitem{def:ellipse/eccentricity} The \term{linear eccentricity} of the ellipse is
  \begin{equation*}
    c \coloneq \sqrt{ a^2 - b^2 },
  \end{equation*}
  and the \term{eccentricity} is
  \begin{equation*}
    e \coloneqq \frac c a.
  \end{equation*}

  The eccentricity of an ellipse is always in the interval \( [0, 1) \).

  \thmitem{def:ellipse/foci} The \term{foci} of the ellipse are the points \( F_1 \coloneqq (-c, 0) \) and \( F_2 \coloneqq (c, 0) \).

  \thmitem{def:ellipse/focal_equation} We can also describe an ellipse as a set of points \( P \) such that
  \begin{equation}\label{eq:def:ellipse/focal_equation}
    \norm{\vect{F_1 P}} + \norm{\vect{F_2 P}} = 2a.
  \end{equation}

  We call \eqref{eq:def:ellipse/focal_equation} the \term{focal equation} of the ellipse. Focal equations have the benefit of not depending on the coordinate system.
\end{definition}
\begin{defproof}
  \ImplicationSubProof{def:ellipse/standard_equation}{def:ellipse/parametric_equation} Suppose that \( (x, y) \) satisfies the standard equation \eqref{eq:def:ellipse/standard_equation}.

  Since \( \abs{x} \leq a \), the ratio \( x / a \) is always between \( -1 \) and \( 1 \). Thus, \( \arccos(x / a) \) is defined, and is a number between \( 0 \) and \( \pi \).

  \begin{itemize}
    \item Define
    \begin{equation*}
      \varphi \coloneqq \begin{cases}
        \arccos(x / a),  &y \geq 0 \\
        -\arccos(x / a), &y < 0.
      \end{cases}
    \end{equation*}

    In both cases,
    \begin{equation*}
      a \cos\varphi = a \cdot x / a = x.
    \end{equation*}

    If \( y \geq 0 \), then, for some \( \varepsilon \in \set{ -1, 1 } \),
    \begin{equation*}
      b \sin\varphi
      =
      b \sin\arccos(x / a)
      \reloset {\eqref{eq:thm:trigonometric_identities/pythagorean_identity}} =
      \varepsilon b \sqrt{ 1 - \cos(x / a)^2 }
      =
      \varepsilon b \sqrt{ 1 - \frac {x^2} {a^2} }
      =
      \varepsilon b \cdot y / b
      =
      \varepsilon y.
    \end{equation*}

    Since \( y \geq 0 \), \( b > 0 \) and \( \sin\varphi > 0 \), it follows that \( \varepsilon = 1 \).

    \item Otherwise,
    \begin{equation*}
      b \sin\varphi
      =
      -b \sin\arccos(x / a)
      =
      \cdots
      =
      \varepsilon y.
    \end{equation*}

    Since \( y < 0 \) and \( \sin\varphi < 0 \), again it follows that \( \varepsilon = 1 \).
  \end{itemize}

  \ImplicationSubProof{def:ellipse/parametric_equation}{def:ellipse/standard_equation} Conversely, if \( (x, y) \) satisfies \eqref{eq:def:ellipse/parametric_equation}, then
  \begin{equation*}
    \frac {x^2} {a^2} + \frac {y^2} {b^2}
    =
    (\cos \varphi)^2 + (\sin \varphi)^2
    \reloset {\eqref{eq:thm:trigonometric_identities/pythagorean_identity}} =
    1.
  \end{equation*}

  \EquivalenceSubProof{def:ellipse/standard_equation}{def:ellipse/focal_equation} On the line through \( F_1 \) and \( F_2 \) with origin their midpoint, \( F_1 \) and \( F_2 \) have coordinates \( (\pm c, 0) \), where \( c \) is half of the distance between \( F_1 \) and \( F_2 \).

  We have an affine coordinate system on some line. Add an orthogonal basis vector so that this extends to a coordinate system for the entire plane. For any point \( P \) with coordinates \( (x, y) \), we then have
  \begin{equation*}
    \norm{\vect{F_1 P}} = \sqrt{(x + c)^2 + y^2}
  \end{equation*}
  and
  \begin{equation*}
    \norm{\vect{F_2 P}} = \sqrt{(x - c)^2 + y^2}.
  \end{equation*}

  \begin{equation*}
    p \coloneqq \frac {\norm{\vect{F_1 P}} + \norm{\vect{F_2 P}}} 2.
  \end{equation*}

  Then
  \begin{equation*}
    \norm{\vect{F_1 P}} = 2p - \norm{\vect{F_2 P}},
  \end{equation*}
  hence
  \begin{equation*}
    \norm{\vect{F_1 P}}^2 = 4p^2 - 4p\norm{\vect{F_2 P}} + \norm{\vect{F_2 P}}^2
  \end{equation*}
  and
  \begin{equation*}
    \norm{\vect{F_2 P}}
    =
    p + \frac {\norm{\vect{F_2 P}}^2 - \norm{\vect{F_1 P}}^2} {4p}
    =
    p + \frac {(x - c)^2 + y^2 - (x + c)^2 - y^2} {4p}
    =
    p + \frac {-2xc - 2xc} {4p}.
  \end{equation*}

  Thus,
  \begin{equation*}
    \sqrt{(x - c)^2 + y^2} = \norm{\vect{F_2 P}} = p - \frac a p x.
  \end{equation*}

  We have
  \begin{equation*}
    \norm{\vect{F_2 P}}^2
    =
    (x - c)^2 + y^2
    =
    x^2 - 2xc + c^2 + y^2.
  \end{equation*}

  Then
  \begin{equation*}
    x^2 - 2xc + c^2 + y^2
    =
    \norm{\vect{F_2 P}}^2
    =
    p^2 - 2 x c + \frac {c^2} {p^2} x^2,
  \end{equation*}
  hence
  \begin{equation*}
    x^2 + y^2 + c^2 = p^2 - \frac {c^2} {p^2} x^2.
  \end{equation*}

  This can be rewritten as
  \begin{equation*}
    \parens*{ 1 - \frac {c^2} {p^2} } x^2 + y^2 = p^2 - c^2
  \end{equation*}
  and, finally,
  \begin{equation*}
    \parens*{ \frac {\cancel{p^2 - c^2}} {p^2} \cdot \frac 1 {\cancel{p^2 - c^2}} } x^2 + \frac {y^2} {p^2 - c^2} = 1.
  \end{equation*}

  Then, in the given coordinate system, this becomes the standard equation of an ellipse \eqref{eq:def:ellipse/standard_equation} with big radii \( p \) and small radii \( \sqrt{p^2 - c^2} \).
\end{defproof}

\begin{proposition}\label{thm:ellipse_is_closed_simple_curve}
  An \hyperref[def:ellipse]{ellipse}, defined via the parametric equations \eqref{eq:def:ellipse/parametric_equation}, is a \hyperref[def:simple_curve]{closed simple curve} (technically, we must allow \( \varphi \) to be \( 2\pi \) for this to be true).
\end{proposition}
\begin{proof}
  We will show that the ellipse is a \hyperref[def:simple_curve]{closed simple curve}. It is obviously closed, hence we must show that the function
  \begin{equation*}
    \varphi \mapsto \parens[\Big]{ a \cos \varphi, b \sin \varphi }
  \end{equation*}
  is injective.

  If
  \begin{equation*}
    \cos \varphi - \cos \psi = 0,
  \end{equation*}
  then \fullref{thm:trigonometric_identities/sums} implies that
  \begin{equation*}
    0
    =
    \cos \varphi - \cos \psi
    =
    -2 \sin\parens*{ \frac{\varphi - \psi} 2 } \sin\parens*{ \frac{\varphi + \psi} 2 }.
  \end{equation*}

  Hence, either \( (\varphi - \psi) / 2 \) or \( (\varphi + \psi) / 2 \) is a multiple of \( \pi \).

  Similarly, if
  \begin{equation*}
    \sin \varphi - \sin \psi = 0,
  \end{equation*}
  then \fullref{thm:trigonometric_identities/sums} implies that
  \begin{equation*}
    0
    =
    \sin \varphi - \sin \psi
    =
    -2 \sin\parens*{ \frac{\varphi - \psi} 2 } \cos\parens*{ \frac{\varphi + \psi} 2 }.
  \end{equation*}

  Hence, either \( (\varphi - \psi) / 2 \) or \( (\varphi + \psi + \pi) / 2 \) is a multiple of \( \pi \).

  Both \( \varphi + \psi \) and \( \varphi + \psi + \pi \) cannot be multiples of \( 2\pi \),  thus it remains for \( \varphi - \psi \) to be. Since both \( \varphi \) and \( \psi \) are taken from the interval \( [0, 2\pi) \), it follows that \( \varphi = \psi \).
\end{proof}

\begin{example}\label{ex:imaginary_ellipse}
  The only pathological example of a nondegenerate quadratic curve in \( \BbbR^2 \) is the \term{imaginary ellipse} induced by the polynomial
  \begin{equation}\label{eq:ex:imaginary_ellipse}
    \frac {X^2} {a^2} + \frac {Y^2} {b^2} + 1.
  \end{equation}

  It has no real roots, yet it is irreducible in \( \BbbC[X, Y] \) as a consequence of \fullref{thm:axn_byn_czn_irreducible} and \fullref{thm:homogeneous_polynomial_constant}. Imaginary ellipses naturally occur during canonization of quadratic curves --- see \fullref{alg:canonization_of_quadratic_plane_curves/oval/imaginary_ellipse}.
\end{example}

\begin{definition}\label{def:hyperbola}
  A \term{hyperbola} is, up to a choice of \hyperref[def:affine_coordinate_system]{affine coordinate system}, the \hyperref[def:root_of_polynomial]{set of zeros} of the polynomial
  \begin{equation}\label{eq:def:hyperbola/polynomial}
    \frac {X^2} {a^2} - \frac {Y^2} {b^2} - 1
  \end{equation}
  for some positive real numbers \( a \) and \( b \), which we call the real and imaginary \term{radii}.

  The polynomial is unique in the sense that no two polynomials of the form \eqref{eq:def:hyperbola/polynomial} have the same set of zeros. In particular, the radii are well-defined.

  It is an \hyperref[def:affine_algebraic_set/curve]{algebraic curve} because, by \fullref{thm:axn_byn_czn_irreducible} and \fullref{thm:homogeneous_polynomial_constant}, the polynomial is irreducible over \( \BbbC \).

  \begin{thmenum}
    \thmitem{def:hyperbola/standard_equation} The set of zeros of \eqref{eq:def:hyperbola/polynomial} consists of all pairs \( (x, y) \) of real numbers satisfying
    \begin{equation}\label{eq:def:hyperbola/standard_equation}
      \frac {x^2} {a^2} - \frac {y^2} {b^2} = 1.
    \end{equation}

    We call \eqref{eq:def:hyperbola/standard_equation} the \term{standard equation} of the hyperbola.

    \thmitem{def:hyperbola/parametric_equation} Hyperbolas can also be described via the pair of \hyperref[def:parametric_curve]{parametric curves}
    \begin{equation}\label{eq:def:hyperbola/parametric_equation}
      \begin{cases}
        x = \pm a \cdot \cosh t, \\
        y = b \cdot \sinh t,
      \end{cases}
    \end{equation}
    where \( t \in \BbbR \).

    \thmitem{def:hyperbola/eccentricity} The \term{linear eccentricity} of the hyperbola is
    \begin{equation*}
      c \coloneq \sqrt{ a^2 + b^2 },
    \end{equation*}
    and the \term{eccentricity} is
    \begin{equation*}
      e \coloneqq \frac c a.
    \end{equation*}

    The eccentricity of a hyperbola is always strictly greater than \( 1 \).

    \thmitem{def:hyperbola/foci} The \term{foci} of the hyperbola are the points \( F_1 \coloneqq (-c, 0) \) and \( F_2 \coloneqq (c, 0) \).

    \thmitem{def:hyperbola/focal_equation} We can also describe a hyperbola as a set of points \( P \) such that
    \begin{equation}\label{eq:def:hyperbola/focal_equation}
      \abs[\Big]{ \norm{\vect{F_1 P}} - \norm{\vect{F_2 P}} } = 2a.
    \end{equation}

    We call \eqref{eq:def:hyperbola/focal_equation} the \term{focal equation} of the hyperbola.
  \end{thmenum}
\end{definition}
\begin{defproof}
  \ImplicationSubProof{def:hyperbola/standard_equation}{def:hyperbola/parametric_equation} Suppose that \( (x, y) \) satisfies the standard equation \eqref{eq:def:hyperbola/standard_equation}.

  Define
  \begin{equation*}
    t \coloneqq \begin{cases}
      \hyperref[eq:thm:hyperbolic_identities/inverse/sinh]{\sinh^{-1}}(y / b), &x \geq 0, \\
      -\sinh^{-1}(y / b),                                                      &x < 0.
    \end{cases}
  \end{equation*}

  In both cases,
  \begin{equation*}
    b \sinh t = b \cdot y / b = y.
  \end{equation*}

  \begin{itemize}
    \item If \( x \geq 0 \), then, for some \( \varepsilon \in \set{ -1, 1 } \),
    \begin{equation*}
      a \cosh\varphi
      =
      a \cosh\sinh^{-1}(y / b)
      \reloset {\eqref{eq:thm:trigonometric_identities/pythagorean_identity}} =
      =
      \varepsilon a \sqrt{ \frac {y^2} {b^2} + 1 }
      =
      \varepsilon a \cdot x / a
      =
      \varepsilon x.
    \end{equation*}

    Since \( x \geq 0 \), \( a > 0 \) and \( \cosh\varphi > 0 \), it follows that \( \varepsilon = 1 \).

    \item Otherwise,
    \begin{equation*}
      a \cosh\varphi
      =
      a \cosh\parens*{ -\sinh^{-1}(y / b) }
      =
      \cdots
      =
      \varepsilon x.
    \end{equation*}

    Since \( x < 0 \), it follows that \( \varepsilon = -1 \).
  \end{itemize}

  \ImplicationSubProof{def:hyperbola/parametric_equation}{def:hyperbola/standard_equation} Conversely, if \( (x, y) \) satisfies \eqref{eq:def:hyperbola/parametric_equation}, then
  \begin{equation*}
    \frac {x^2} {a^2} - \frac {y^2} {b^2}
    =
    \parens[\Big]{ \pm \cosh(t) }^2 - \sinh(t)^2
    \reloset {\eqref{eq:thm:hyperbolic_identities/pythagorean_identity}} =
    1.
  \end{equation*}

  \EquivalenceSubProof{def:hyperbola/standard_equation}{def:hyperbola/focal_equation} The proof is analogous to the proof for ellipses but with
  \begin{equation*}
    p \coloneqq \frac {\norm{\vect{F_1 P}} - \norm{\vect{F_2 P}}} 2.
  \end{equation*}
  rather than
  \begin{equation*}
    p \coloneqq \frac {\norm{\vect{F_1 P}} + \norm{\vect{F_2 P}}} 2.
  \end{equation*}
\end{defproof}

\begin{proposition}\label{thm:hyperbola_is_closed_simple_curve}
  A \hyperref[def:hyperbola]{hyperbola}, defined via the parametric equations \eqref{eq:def:hyperbola/parametric_equation}, is a pair of non-intersecting \hyperref[def:simple_curve]{simple curve}.
\end{proposition}
\begin{proof}
  The inverse of \( \sinh \) is defined for all real numbers; hence the parametric functions are injective and the curves are \hyperref[def:simple_curve]{simple curves}. Furthermore, since \( \cosh \) is strictly positive, the two curves do not intersect.
\end{proof}

\begin{definition}\label{def:parabola}
  A \term{parabola} is, up to a choice of \hyperref[def:affine_coordinate_system]{affine coordinate system}, the \hyperref[def:root_of_polynomial]{set of zeros} of the polynomial
  \begin{equation}\label{eq:def:parabola/polynomial}
    Y^2 - 2p X
  \end{equation}
  for some nonzero real number \( p \), called the \term{parameter} of the parabola.

  The polynomial is unique in the sense that no two polynomials of the form \eqref{eq:def:hyperbola/polynomial} have the same set of zeros. In particular, the parameter is well-defined.

  It is an \hyperref[def:affine_algebraic_set/curve]{algebraic curve} because, by \fullref{thm:axz_byy_irreducible} and \fullref{thm:homogeneous_polynomial_constant}, the polynomial is irreducible over \( \BbbC \).

  \begin{thmenum}
    \thmitem{def:parabola/standard_equation} The set of zeros of \eqref{eq:def:parabola/polynomial} consists of all pairs \( (x, y) \) of real numbers satisfying
    \begin{equation}\label{eq:def:parabola/standard_equation}
      y^2 = 2px.
    \end{equation}

    We call \eqref{eq:def:parabola/standard_equation} the \term{standard equation} of the parabola.

    \thmitem{def:parabola/parametric_equation} Parabolas can also be described via the pair of \hyperref[def:parametric_curve]{parametric curves}
    \begin{equation}\label{eq:def:parabola/parametric_equation}
      \begin{cases}
        x = t, \\
        y = \pm \sqrt{ 2pt },
      \end{cases}
    \end{equation}
    where \( t \geq 0 \).

    \thmitem{def:parabola/eccentricity} The \term{linear eccentricity} and \term{eccentricity} of parabolas are \( c = e = 1 \).

    \thmitem{def:parabola/focus} The \term{focus} of the parabola is the point \( F \coloneqq (p / 2, 0) \).

    \thmitem{def:parabola/directrix} The \term{directrix} \( d \) of the parabola is the line with \hyperref[def:plane_line_equations/general]{general equation} \( x = -p / 2 \).

    \thmitem{def:parabola/focal_equation} We can also describe a parabola as a set of points \( P \) such that
    \begin{equation}\label{eq:def:parabola/focal_equation}
      \norm{\vect{FP}} = \op{dist}(P, d),
    \end{equation}
    where \( \op{dist}(P, d) \) is the distance from \( P \) to \( d \) in the sense of \fullref{def:distance_to_subspace}.

    We call \eqref{eq:def:parabola/focal_equation} the \term{focal equation} of the parabola.
  \end{thmenum}
\end{definition}
\begin{proof}
  \EquivalenceSubProof{def:parabola/standard_equation}{def:parabola/focal_equation} The vector \( v \) with coordinates \( (0, 1) \) is directional for \( d \) as a consequence of \fullref{thm:coordinates_of_directional_vector}, and point \( O \) with coordinates \( (-p / 2, 0) \) lies on \( d \).

  The orthogonal projection of the point \( P \) with coordinates \( (x, y) \) onto \( d \) is then
  \begin{align*}
    \pi_d(P)
    &=
    O + \inprod v {\vect{OP}} \cdot v
    = \\ &=
    \parens*{ -\frac p 2, 0 } + \inprod*{ \parens*{ 0, 1 } } { \parens*{ x + \frac p 2, y } } \cdot \parens*{ 0, 1 }
    = \\ &=
    \parens*{ -\frac p 2, 0 } + y \cdot \parens*{ 0, 1 }
    = \\ &=
    \parens*{ -\frac p 2, y }.
  \end{align*}

  Then
  \begin{equation*}
    \op{dist}(P, d)
    =
    \norm{P - \pi_d(P)}
    =
    \norm*{\parens*{ x + \frac p 2, 0 }}
    =
    x + \frac p 2.
  \end{equation*}

  Then
  \begin{equation*}
    \norm{P - F}^2 - \norm{P - \pi_d(P)}^2
    =
    \parens*{ x - \frac p 2 }^2 + y^2 - \parens*{ x + \frac p 2 }^2
    =
    y^2 - 2px.
  \end{equation*}
\end{proof}

\begin{proposition}\label{thm:parabola_is_closed_simple_curve}
  A \hyperref[def:parabola]{parabola}, defined via the parametric equations \eqref{eq:def:parabola/parametric_equation}, is a pair of \hyperref[def:simple_curve]{simple curves} that intersect only at the origin of the coordinate system.
\end{proposition}
\begin{proof}
  Since \( \sqrt {\anon} \) is injective on the nonnegative real numbers, the parametric curves \eqref{eq:def:hyperbola/parametric_equation} are \hyperref[def:simple_curve]{simple curve}. Since \( \sqrt {\anon} \) is also positive on positive real numbers, it follows that the two curves only intersect at the origin.
\end{proof}

\begin{algorithm}\label{alg:canonization_of_quadratic_plane_curves}
  We can use \hyperref[def:rigid_motion]{rigid motions} to convert \hyperref[def:quadratic_plane_curve]{quadratic plane curve} into what we will call its \term{canonical form}, whose set of zeros is either
  \begin{itemize}
    \item an ellipse \eqref{eq:def:ellipse/polynomial} --- case \cref{alg:canonization_of_quadratic_plane_curves/oval/ellipse}.
    \item a hyperbola \eqref{eq:def:hyperbola/polynomial} --- case \cref{alg:canonization_of_quadratic_plane_curves/oval/hyperbola}.
    \item a parabola \eqref{eq:def:parabola/polynomial} --- cases \cref{alg:canonization_of_quadratic_plane_curves/y_parabola} and \cref{alg:canonization_of_quadratic_plane_curves/x_parabola}.
    \item an imaginary ellipse \eqref{eq:ex:imaginary_ellipse} --- case \cref{alg:canonization_of_quadratic_plane_curves/oval/imaginary_ellipse}.
    \item a degenerate quadratic curve --- cases \cref{alg:canonization_of_quadratic_plane_curves/y_parallel_lines},
  \cref{alg:canonization_of_quadratic_plane_curves/x_parallel_lines} and \cref{alg:canonization_of_quadratic_plane_curves/oval/lines}.
  \end{itemize}

  Consider the polynomial
  \begin{equation*}
    p(X, Y)
    \coloneqq
    a X^2 + b XY + c Y^2 + d X + e Y + f
    =
    \begin{pmatrix}
      X \\ Y
    \end{pmatrix}^T
    \begin{pmatrix}
      a          & b / 2 \\
      b / 2 & c
    \end{pmatrix}
    \begin{pmatrix}
      X \\ Y
    \end{pmatrix}
    +
    \begin{pmatrix}
      d \\ e
    \end{pmatrix}^T
    \begin{pmatrix}
      X \\ Y
    \end{pmatrix}
    +
    f.
  \end{equation*}

  The matrix
  \begin{equation*}
    \begin{pmatrix}
      a          & b / 2 \\
      b / 2 & c
    \end{pmatrix}
  \end{equation*}
  is symmetric, and \fullref{thm:finite_dimensional_spectral_theorem} ensures that it can be diagonalized via some \hyperref[def:unitary_matrix]{orthogonal matrix} \( P \). Let \( \alpha \) and \( \gamma \) be the eigenvalues corresponding to \( P \). Then
  \begin{equation*}
    p(X, Y)
    =
    \bracks*
      {
        P
        \begin{pmatrix}
          X \\ Y
        \end{pmatrix}
      }^T
    \begin{pmatrix}
      \alpha &       \\
             & \gamma
    \end{pmatrix}
    \bracks*
      {
        P
        \begin{pmatrix}
          X \\ Y
        \end{pmatrix}
      }
    +
    \bracks*
      {
        P
        \begin{pmatrix}
          d \\ e
        \end{pmatrix}
      }
    \bracks*
      {
        P
        \begin{pmatrix}
          X \\ Y
        \end{pmatrix}
      }
    +
    f.
  \end{equation*}

  The matrix \( P \) thus induces a \hyperref[con:change_of_basis]{change of basis}. Denote by \( \xi \) and \( \eta \) the new indeterminates, and by \( \delta \) and \( \varepsilon \) the corresponding images of \( d \) and \( e \). Then
  \begin{equation*}
    p(\xi, \eta)
    =
    \begin{pmatrix}
      \xi \\ \eta
    \end{pmatrix}^T
    \begin{pmatrix}
      \alpha &       \\
             & \gamma
    \end{pmatrix}
    \begin{pmatrix}
      \xi \\ \eta
    \end{pmatrix}
    +
    \begin{pmatrix}
      \delta \\ \varepsilon
    \end{pmatrix}^T
    \begin{pmatrix}
      \xi \\ \eta
    \end{pmatrix}
    +
    f.
  \end{equation*}

  Want to further simplify the expression via \hyperref[def:rigid_motion/translation]{translations}. There are several cases to consider. Note that \( \alpha \) and \( \gamma \) cannot simultaneously be zero.

  \begin{thmenum}
    \thmitem{alg:canonization_of_quadratic_plane_curves/y_parallel_lines} If \( \alpha = \delta = 0 \) and \( \gamma \neq 0 \), we use the translation
    \begin{balign*}
      p\parens*{ \xi, \eta - \frac \varepsilon {2 \gamma} }
      &=
      \gamma \parens*{ \eta - \frac \varepsilon {2 \gamma} }^2 + \varepsilon \parens*{ \eta - \frac \varepsilon {2 \gamma} } + f
      = \\ &=
      \gamma \eta^2 + \parens[\Big]{ 2 \gamma \cdot \frac {-\varepsilon} {2 \gamma} + \varepsilon } \eta + \parens[\Big]{ \frac {\varepsilon^2} {4 \gamma^2} + f }
      =
      \gamma \eta^2 + \parens[\Big]{ \frac {\varepsilon^2} {4 \gamma^2} + f }.
    \end{balign*}

    This polynomial is reducible over \( \BbbC \), hence its set of zeros is not an algebraic curve, but rather a pair of parallel lines in \( \BbbC^2 \) (but possibly no real roots).

    \thmitem{alg:canonization_of_quadratic_plane_curves/x_parallel_lines} If \( \alpha \neq 0 \) and \( \gamma = \varepsilon = 0 \), we reduce this to \fullref{alg:canonization_of_quadratic_plane_curves/y_parallel_lines} by swapping \( \xi \) and \( \eta \).

    \thmitem{alg:canonization_of_quadratic_plane_curves/y_parabola} If \( \alpha = 0 \), and \( \gamma \neq 0 \) and \( \delta \neq 0 \), we use the translation
    \begin{balign*}
      p\parens*{ \xi - \frac {\varepsilon^2} {4 \delta \gamma^2} - \frac f \delta, \eta - \frac \varepsilon {2 \gamma} }
      &=
      \gamma \parens*{ \eta - \frac \varepsilon {2 \gamma} }^2 + \delta \parens[\Big]{ \xi - \frac {\varepsilon^2} {4 \delta \gamma^2} - \frac f \delta } + \varepsilon \parens*{ \eta - \frac \varepsilon {2 \gamma} } + f
      = \\ &=
      \gamma \eta^2 + \delta \parens[\Big]{ \xi - \frac {\varepsilon^2} {4 \delta \gamma^2} - \frac f \delta } + \parens[\Big]{ 2 \gamma \cdot \frac {-\varepsilon} {2 \gamma} + \varepsilon } \eta + \parens[\Big]{ \frac {\varepsilon^2} {4 \gamma^2} + f }
      = \\ &=
      \gamma \eta^2 + \delta \xi.
    \end{balign*}

    We conclude that \( p(X, Y) \) induces a \hyperref[def:parabola]{parabola}.

    \thmitem{alg:canonization_of_quadratic_plane_curves/x_parabola} If \( \alpha \neq 0 \), \( \varepsilon \neq 0 \) and \( \gamma = 0 \), we reduce this to \fullref{alg:canonization_of_quadratic_plane_curves/y_parabola} by swapping \( \xi \) and \( \eta \).

    \thmitem{alg:canonization_of_quadratic_plane_curves/oval} If \( \alpha \neq 0 \) and \( \gamma \neq 0 \), we proceed as follows:
    \begin{balign*}
      &\phantom{{}={}}
      p\parens*{ \xi - \frac \delta {2 \alpha}, \eta - \frac \varepsilon {2 \gamma} }
      = \\ &=
      \alpha \parens*{ \xi - \frac \delta {2 \alpha} }^2 + \gamma \parens*{ \eta - \frac \varepsilon {2 \gamma} }^2 + \delta \parens*{ \xi - \frac \delta {2 \alpha} } + \varepsilon \parens*{ \eta - \frac \varepsilon {2 \gamma} } + f
      = \\ &=
      \alpha \xi^2 + \gamma \eta^2 + \parens[\Big]{ 2 \alpha \cdot \frac {-\delta} {2 \alpha} + \delta } \xi + \parens[\Big]{ 2 \gamma \cdot \frac {-\varepsilon} {2 \gamma} + \varepsilon } \eta + \parens[\Big]{ \frac {\delta^2} {4 \alpha^2} + \frac {\varepsilon^2} {4 \gamma^2} + f }
      = \\ &=
      \alpha \xi^2 + \gamma \eta^2 + \parens[\Big]{ \frac {\delta^2} {4 \alpha^2} + \frac {\varepsilon^2} {4 \gamma^2} + f }.
    \end{balign*}

    Define
    \begin{equation*}
      \varphi \coloneqq \frac {\delta^2} {4 \alpha^2} + \frac {\varepsilon^2} {4 \gamma^2} + f.
    \end{equation*}

    Finally,
    \begin{equation*}
      \alpha \xi^2 + \gamma \eta^2 + \varphi.
    \end{equation*}

    \begin{thmenum}
      \thmitem{alg:canonization_of_quadratic_plane_curves/oval/lines} If \( \varphi = 0 \), then \( p(X, Y) \) is reducible over \( \BbbC \), and its set of zeros is a pair of orthogonal lines in \( \BbbC^2 \) (but possibly no real roots).

      \thmitem{alg:canonization_of_quadratic_plane_curves/oval/imaginary_ellipse} If all coefficients have the same sign, then \( p(X, Y) \) is irreducible, but has no roots --- it is an \hyperref[ex:imaginary_ellipse]{imaginary ellipse}.

      \thmitem{alg:canonization_of_quadratic_plane_curves/oval/hyperbola} If \( \varphi \neq 0 \) and \( \alpha \gamma < 0 \), then \( p(X, Y) \) induces a \hyperref[def:hyperbola]{hyperbola}.

      \thmitem{alg:canonization_of_quadratic_plane_curves/oval/ellipse} If \( \alpha \gamma > 0 \) and \( \alpha \varphi < 0 \), then \( p(X, Y) \) induces an \hyperref[def:ellipse]{ellipse}.
    \end{thmenum}
  \end{thmenum}
\end{algorithm}

\begin{definition}\label{def:disk}
  A \term[bg=кръг,ru=круг]{disk} with \term{center} \( O = (x_0, y_0) \) and \term{radius} \( r > 0 \) is a geometric figure that can be defined equivalently as:

  \begin{thmenum}
    \thmitem{def:disk/sphere} The \hyperref[def:metric_space/ball]{closed ball} \( \cl B(O, r) \) in the metric space \( \BbbR^2 \).

    Consistently with the terminology for balls in metric spaces, the term \enquote{unit disk} refers to the case \( r = 1 \).

    \thmitem{def:disk/standard_equation} The set of points \( (x, y) \) satisfying
    \begin{equation}\label{eq:def:disk/standard_equation}
      (x - x_0)^2 + (y - y_0)^2 \leq r^2.
    \end{equation}
  \end{thmenum}
\end{definition}
\begin{proof}
  Note that the left side of \eqref{eq:def:disk/standard_equation} is the squared norm in \( \BbbR^2 \).
\end{proof}

\begin{definition}\label{def:circle}\mimprovised
  A \term[bg=окръжност,ru=окружность]{circle} with \term{center} \( O = (x_0, y_0) \) and \term{radius} \( r > 0 \) is a geometric figure that can be defined equivalently as:

  \begin{thmenum}[series=def:circle]
    \thmitem{def:circle/disk} The \hyperref[def:topological_boundary_operator]{topological boundary} of the \hyperref[def:disk]{disk} with center \( O \) and radius \( r \).

    \thmitem{def:circle/sphere} The \hyperref[def:metric_space/sphere]{sphere} \( S(O, r) \) in the metric space \( \BbbR^2 \).

    Consistently with the terminology for spheres in metric spaces, the term \enquote{unit circle} refers to the case \( r = 1 \).

    \thmitem{def:circle/ellipse} The \hyperref[def:ellipse]{ellipse} with \hyperref[def:ellipse]{radii} \( a = b = r \) and \hyperref[def:ellipse/foci]{foci} \( F_1 = F_2 = O \).

    The standard equation, also called the \term{central equation}, is conventionally written as
    \begin{equation}\label{eq:def:circle/ellipse/central_equation}
      (x - x_0)^2 + (y - y_0)^2 = r^2.
    \end{equation}

    As a consequence, ellipses have eccentricity \( 0 \).
  \end{thmenum}
\end{definition}
\begin{defproof}
  \EquivalenceSubProof{def:circle/disk}{def:circle/sphere} Spheres are the boundaries of the corresponding balls.
  \EquivalenceSubProof{def:circle/sphere}{def:circle/ellipse} The equivalence of definitions follows from the equivalence of the \hyperref[def:ellipse/focal_equation]{standard equation} and the \hyperref[def:ellipse/focal_equation]{focal equation} of an ellipse.
\end{defproof}

\begin{proposition}\label{thm:circle_diameter}
  The \hyperref[def:metric_space/diameter]{diameter} of a \hyperref[def:circle]{circle} or \hyperref[def:disk]{disk} with radius \( r \) is \( 2r \).
\end{proposition}
\begin{proof}
  The \hyperref[def:metric_space/diameter]{diameter} is twice the radius of the smallest closed ball containing the disk. This ball is the disk itself, hence the diameter of is \( 2r \).
\end{proof}

\begin{definition}\label{def:circumference}
  The term \term{circumference} is used for the perimeter of a disk or the \hyperref[def:arc_length]{arc length} of a circle.
\end{definition}
\begin{defproof}
  The two values are equivalent since circles are boundaries of disks.
\end{defproof}

\begin{proposition}\label{thm:circumference}
  The \hyperref[def:circumference]{circumference} of a \hyperref[def:circle]{circle} or \hyperref[def:disk]{disk} with radius \( r \) is \( 2\pi r \).
\end{proposition}
\begin{proof}
  Follows from \fullref{thm:circle_diameter} by noting that \( \pi \) is defined in \fullref{def:pi/circle} as the ratio of the circumference by the diameter.
\end{proof}

\begin{proposition}\label{thm:area_of_circle}
  The \hyperref[def:figure_area]{area} of a \hyperref[def:disk]{disk} with radius \( r \) is \( \pi r^2 \).
\end{proposition}
\begin{proof}
  We want to determine the Lebesgue measure of the disk \( D \). This requires integrating
  \begin{equation*}
    \int_D \dl x \dl y.
  \end{equation*}

  The parametric equation \eqref{eq:def:ellipse/parametric_equation} suggests expressing \( D \) via \hyperref[con:polar_coordinate_system]{polar coordinates}. Let \( O = (x_0, y_0) \) be the center of \( D \). Then the Cartesian coordinates of \( D \) are
  \begin{equation*}
    \begin{cases}
      x = x_0 + \rho \cdot \cos \varphi, \\
      y = y_0 + \rho \cdot \sin \varphi,
    \end{cases}
  \end{equation*}
  where \( 0 \leq \rho \leq r \) and \( 0 \leq \varphi < 2\pi \). The Jacobian of this change of variables is
  \begin{equation*}
    \det
    \begin{pmatrix}
      \cos \varphi & -\rho \sin \varphi \\
      \sin \varphi & \rho \cos \varphi
    \end{pmatrix}
    =
    \rho (\cos \varphi)^2 + \rho (\sin \varphi)^2
    =
    \rho.
  \end{equation*}

  Then
  \begin{equation*}
    \int_D \dl x \dl y
    =
    \int_0^{2\pi} \int_0^r \rho \dl \rho \dl \varphi
    =
    \frac {r^2} 2 \int_0^{2\pi} \dl \varphi
    =
    \pi r^2.
  \end{equation*}
\end{proof}


  \section{Group theory}\label{sec:group_theory}

Modern algebra takes its roots in abstracting \hyperref[def:integers]{integers} and \hyperref[def:real_numbers]{real numbers} and their addition and multiplication. Both of these operations are \hyperref[def:magma/commutative]{commutative} and, if we want to generalize their properties, it is sensible to study commutative operations.

Another type of objects that usually fits in the same algebraic framework are \hyperref[def:function]{functions} and their \hyperref[def:multi_valued_function/composition]{composition}. Functions from a set to itself can be composed to form another function of the same type, similarly to how two integers can be added to obtain another integer. The main difference is in the non-commutativity of function composition.

This suggests that we use the same algebraic structures to study both generalizations of numbers and generalizations of functions over a set. The first case is commutative, the second is not. This is why commutative and non-commutative structures, even though they are similarly defined, can have very different properties and applications.

We shall not attempt to give a precise definition for an \term{algebraic structure}. There are very general frameworks for doing so, however their complexity is unjustified for us. We will instead build standard algebraic structures from \enquote{base building blocks}, although we will utilize very general definitions like categorical kernels and images defined in \fullref{def:zero_morphisms}.

The simplest algebraic structures that will be of interest to us are \hyperref[def:group]{groups} and their less well-behaved generalizations, \hyperref[def:magma/associative]{semigroups} and \hyperref[def:monoid]{monoids}.

Except as a building block for more complicated algebraic structures, groups arise whenever some mathematical structure exhibits symmetries, and this concept is formalized via \hyperref[def:automorphism_group]{automorphism groups} and \hyperref[def:group_action]{group actions}. If, instead of symmetries we have non-invertible but nonetheless well-behaved transformations, we can instead study \hyperref[def:endomorphism_monoid]{endomorphism monoids} and \hyperref[def:monoid_action]{monoid actions}.

  \subsection{Monoids}\label{subsec:monoids}

We list here several basic algebraic structures that we will use mostly as building blocks for more complicated structures.

As discussed in \fullref{rem:first_order_model_notation}, listing all operations explicitly is cumbersome, and we will usually avoid it.

\begin{figure}[!ht]
  \caption{Generalizations of groups}\label{fig:monoid_hierarchy}
  \smallskip
  \hfill
  \begin{forest}
    for tree={grow=north}
    [
      {\hyperref[def:group]{Group}}
        [
          {\hyperref[def:set_with_involution]{Set with involution}}, edge label={node[midway, right=1em]{\( x^{-1} \) (unary)}}
        ]
        [
          {\hyperref[def:monoid]{Monoid}},
            [{\hyperref[def:semigroup]{Semigroup}}, edge label={node[midway, right=1em]{\( x \cdot y \) (binary)}}]
            [{\hyperref[def:pointed_set]{Pointed set}}, edge label={node[midway, left=1em]{\( e \) (nullary)}}]
        ]
    ]
  \end{forest}
  \hfill\hfill
\end{figure}

\begin{definition}\label{def:operation_on_set}\mcite[12]{ЦаленкоШульеейфер1974}
  An \( n \)-ary \term[ru=операция (\cite[12]{ЦаленкоШульеейфер1974})]{operation} on a \hyperref[def:set]{plain set} \( A \) is a \hyperref[def:function]{function} from \( A^n \) to \( A \).
\end{definition}
\begin{comments}
  \item Function arguments are discussed in full generality in \fullref{rem:function_arguments}. For operations, the arity is unambiguous, and so we use the terms \enquote{unary}, \enquote{binary}, \enquote{ternary}, etc. more freely.
\end{comments}

\paragraph{Pointed sets}

\begin{definition}\label{def:pointed_set}\mcite[26]{MacLane1998}
  A \term[ru=множество с отмеченной точкой (\cite[16]{ЦаленкоШульеейфер1974})]{pointed set} is a simple algebraic structure --- a nonempty set \( X \) equipped with a distinguished element \( e \). It is an algebraic structure because \( e \) can be regarded as the sole value of a nullary \hyperref[def:operation_on_set]{operation} \( \ast: X^0 \to X \).

  We will call \( e \) the \term{origin} of \( X \) based on the terminology for \hyperref[def:affine_coordinate_system]{affine coordinate systems}.

  Pointed sets have the following metamathematical properties:
  \begin{thmenum}
    \thmitem{def:pointed_set/theory} Pointed sets can also be viewed as \hyperref[def:first_order_model]{models} of an empty \hyperref[def:first_order_theory]{theory} for a \hyperref[def:first_order_language]{first-order logic language} with a constant symbol, i.e. a nullary \hyperref[def:first_order_language/fun]{functional symbol}.

    \thmitem{def:pointed_set/homomorphism} A \hyperref[def:first_order_homomorphism]{first-order homomorphism} between the pointed sets \( (X, e_{X}) \) and \( (Y, e_{Y}) \) is, explicitly, a function \( \varphi: X \to Y \) that satisfies
    \begin{equation}\label{eq:def:pointed_set/homomorphism}
      \varphi(e_{X}) = e_{Y}.
    \end{equation}

    \thmitem{def:pointed_set/submodel} The set \( S \subseteq X \) is a \hyperref[def:first_order_submodel]{first-order submodel} of \( X \) if \( e \in S \).

    In particular, as a consequence of \fullref{thm:positive_formulas_preserved_under_homomorphism}, the image of a pointed set homomorphism is a submodel of its codomain.

    \thmitem{def:pointed_set/category} We denote the \hyperref[def:category_of_small_first_order_models]{category of \( \mscrU \)-small models} for this theory by \( \ucat{Set}_* \).
  \end{thmenum}
\end{definition}

\paragraph{Trivial objects}

\begin{remark}\label{rem:trivial_object}
  Some algebraic theories have an established \term{trivial structure} --- for example the \term{trivial group} \( \set{ e } \) defined by \incite[146]{Knapp2016BasicAlgebra} and \incite[42]{Aluffi2009}, also called a \term{zero group} by \incite[25]{MacLane1998} or the \term{trivial module} \( \set{ 0 } \), used by \incite[175]{Aluffi2009}, also called a \term{zero module} by \incite[146]{Knapp2016BasicAlgebra}, \incite[25]{MacLane1998} and even Paolo Aluffi himself in \cite[341]{Aluffi2009}. Dimension theory for modules allows characterizing the trivial module as a zero-dimensional module or vector space.

  All these are single-element models of the corresponding theories. We may also define trivial \hyperref[def:partially_ordered_set]{partially ordered sets}, which would be empty sets, trivial \hyperref[def:semilattice/lattice]{lattices}, which could either be one-element or two-element sets depending on whether \( \top = \bot \), and trivial \hyperref[def:directed_graph]{directed graphs}, which could be any of the things discussed in \fullref{rem:trivial_graph}. The trouble is that these are not well-established concepts, and hence it could not be clear from the context what we mean.

  For the few purposes that we will need them for, it is possible to give a general definition for trivial objects --- see \fullref{def:trivial_object}.
\end{remark}

\begin{definition}\label{def:trivial_object}
  Let \( \cat{C} \) be a \hyperref[def:category]{category} that is \hyperref[def:concrete_category]{concrete} over the category \hyperref[def:pointed_set/category]{\( \cat{Set}_* \)} of pointed sets, and let \( U: \cat{C} \to \cat{Set}_* \) be the corresponding (\hyperref[def:functor_invertibility/faithful]{faithful}) forgetful functor.

  Then any preimage under \( U \) of any single-element pointed set is a \hyperref[def:universal_objects/zero]{zero object} in \( \cat{C} \). We will call such objects \term{trivial}.
\end{definition}
\begin{comments}
  \item Even if the category is concrete over \( \cat{Set}_* \), it may not have a zero object. As an example, we can remove the trivial pointed set and all related morphisms from \( \cat{Set}_* \) itself.
  \item We decided to use the term \enquote{trivial} rather than \enquote{zero} since it is better established in the context of algebra.
  \item In particular, any single-element pointed set is a zero object in the category \( \cat{Set}_* \).
  \item Other examples for which we will use this definition include \hyperref[def:group]{groups}, \hyperref[def:ring]{nonunital ring}, \hyperref[def:semimodule]{semimodules} and \hyperref[def:module]{modules} --- in the aforementioned cases, all single-element structures are isomorphic.
\end{comments}
\begin{defproof}
  Suppose that \( U(Z) = \set{ e } \).

  \SubProof{Proof that \( Z \) is an \hyperref[def:universal_objects/initial]{initial object}} For any morphism \( f: Z \to B \), the function \( U(f) \) \enquote{picks} an element out of \( U(B) \). The only choice of element that makes \( U(f) \) a pointed set homomorphism is the origin of \( U(B) \), and hence the image of the morphism set \( \cat{C}(Z, B) \) under \( U \) consists of only one function.

  Since \( U \) is a faithful functor, this means that \( \cat{C}(Z, B) \) also has only one element.

  The object \( B \) was chosen arbitrarily, so we conclude that \( Z \) is an initial object.

  \SubProof{Proof that \( Z \) is a \hyperref[def:universal_objects/initial]{terminal object}} If \( U(A) = \set{ e } \), then, for any morphism \( f: A \to Z \), the function \( U(f) \) must take the only possible value \( e \) at any element of \( U(Z) \). Again, since \( U \) is faithful, we conclude that \( Z \) is also a terminal object.
\end{defproof}

\paragraph{Kernels and cokernels}

\begin{definition}\label{def:zero_morphisms}
  Suppose that the category \( \cat{C} \) has a \hyperref[def:universal_objects/zero]{zero object} \( 0 \).

  \begin{thmenum}
    \thmitem{def:zero_morphisms/morphism}\mcite[20]{MacLane1998} For every pair of objects \( A \) and \( B \) in \( \cat{C} \), there exists a unique morphism, called the \term[ru=нулевой морфизм (\cite[75]{ЦаленкоШульеейфер1974})]{zero morphism}, that \hyperref[def:factors_through]{uniquely factors through} \( 0 \):
    \begin{equation}\label{eq:def:zero_morphisms/morphism}
      \begin{aligned}
        \includegraphics[page=1]{output/def__zero_morphisms}
      \end{aligned}
    \end{equation}

    We denote this zero morphism by \( 0_{A,B} \) or even \( 0 \).

    \thmitem{def:zero_morphisms/kernel}\mcite[191]{MacLane1998} We define the \term[ru=ядро (\cite[36]{ЦаленкоШульеейфер1974})]{kernel} cone of a morphism \( f: A \to B \) as the \hyperref[eq:def:equalizers/equalizer]{equalizer cone} of \( f \) and \( 0_{A,B} \).

    \Fullref{thm:equalizer_invertibility} implies that a kernel morphism is necessarily a monomorphism. The converse may not hold.

    \thmitem{def:zero_morphisms/cokernel}\mcite[64]{MacLane1998} \hyperref[thm:categorical_principle_of_duality]{Dually}, we define the \term[ru=коядро (\cite[37]{ЦаленкоШульеейфер1974})]{cokernel} cocone of a morphism \( f: A \to B \) as the \hyperref[eq:def:equalizers/coequalizer]{coequalizer cocone} of \( f \) and \( 0_{A,B} \).

    \Fullref{thm:equalizer_invertibility} implies that a cokernel morphism is necessarily an epimorphism. The converse may not hold.

    \thmitem{def:zero_morphisms/image}\mcite[def. IX.1.15]{Aluffi2009} We define the \term{image} of a morphism \( f: A \to B \) as the kernel of the cokernel morphism.

    \thmitem{def:zero_morphisms/coimage}\mcite[def. IX.1.15]{Aluffi2009} Dually, we define the \term{coimage} of a morphism \( f: A \to B \) as the cokernel of the kernel morphism.
  \end{thmenum}
\end{definition}
\begin{comments}
  \item Neither the kernel, cokernel, image nor coimage of a morphism is guaranteed to exist.
\end{comments}

\begin{lemma}\label{thm:zero_morphism_composition}
  The left or right composition of a \hyperref[def:zero_morphisms/morphism]{zero morphism} with any other morphism is also a zero morphism.
\end{lemma}
\begin{proof}
  Consider the composition \( 0_{B,C} \bincirc f \), where \( f: A \to B \). The following diagram commutes because \( 0 \) is a terminal object:
  \begin{equation*}
    \begin{aligned}
      \includegraphics[page=1]{output/thm__zero_morphism_composition}
    \end{aligned}
  \end{equation*}

  The morphism \( 0_{A,C} \) uniquely factors through \( 0 \) by definition, and such a factorization is provided by \( 0_{0,C} \bincirc 0_{A,0} \). Then
  \begin{equation*}
    0_{A,C} = 0_{0,C} \bincirc 0_{A,0} = \underbrace{0_{0,C} \bincirc 0_{B,0}}_{0_{B,C}} \bincirc f.
  \end{equation*}

  Therefore, \( 0_{B,C} \bincirc f \) is also a zero morphism.

  An analogous proof can be given when composing \( f \) with a zero morphism on the right.
\end{proof}

\begin{theorem}[Image-coimage factorization]\label{thm:image_coimage_factorization}
  In a category with a zero object, for every morphism \( f: A \to B \), if both its \hyperref[def:zero_morphisms/image]{image} and \hyperref[def:zero_morphisms/coimage]{coimage} exist, there exists a unique morphism \( \widehat{f}: \co\img f \to \img f \) such that the following diagram commutes:
  \begin{equation*}
    \begin{aligned}
      \includegraphics[page=1]{output/thm__image_coimage_factorization}
    \end{aligned}
  \end{equation*}
\end{theorem}
\begin{proof}
  Consider the extended diagram
  \begin{equation*}
    \begin{aligned}
      \includegraphics[page=2]{output/thm__image_coimage_factorization}
    \end{aligned}
  \end{equation*}

  \SubProof{Construction of \( f': \co\img f \to B \)}

  The projection \( \pi: A \to \co\img f \) is a cokernel morphism of \( \iota' \) above, which is in turn a kernel morphism of \( \ker f \). Another coequalizer cocone for \( \iota' \) is \( (B, f) \). Indeed, \( \iota' \) itself satisfies
  \begin{equation}\label{eq:thm:image_coimage_factorization/f_cocone/1}
    f \bincirc \iota' = 0_{A, B} \bincirc \iota'.
  \end{equation}

  \Fullref{thm:zero_morphism_composition} implies that
  \begin{equation}\label{eq:thm:image_coimage_factorization/f_cocone/2}
    0_{A, B} \bincirc \iota' = 0_{\ker f, B}
  \end{equation}
  and
  \begin{equation}\label{eq:thm:image_coimage_factorization/f_cocone/3}
    0_{\ker f, B} = f \bincirc 0_{\ker f, A}.
  \end{equation}

  We can combine \eqref{eq:thm:image_coimage_factorization/f_cocone/1}, \eqref{eq:thm:image_coimage_factorization/f_cocone/2} and \eqref{eq:thm:image_coimage_factorization/f_cocone/3} to obtain
  \begin{equation*}
    f \bincirc \iota' = f \bincirc 0_{\ker f, A}.
  \end{equation*}

  Thus, \( (B, f) \) is also a coequalizer cocone for \( \iota' \), and hence there exists a unique morphism \( f': \co\img f \to B \) such that \( f = f' \bincirc \pi \).

  \SubProof{Construction of \( \widetilde{f}: \co\img f \to \img f \)} The embedding \( \iota: \img f \to B \) is a kernel morphism of \( \pi' \), which is a cokernel morphism of \( f \). As a cokernel morphism of \( f \), \( \pi' \) satisfies the following:
  \begin{equation*}
    \pi' \bincirc f = \underbrace{\pi' \bincirc 0_{A,B}}_{0_{A,\co\ker f}}.
  \end{equation*}

  But \( f = f' \bincirc \pi \), thus
  \begin{equation*}
    \pi' \bincirc f' \bincirc \pi = 0_{A,\co\ker f} = 0_{A,B} \bincirc \pi.
  \end{equation*}

  Since \( \pi \) is a cokernel morphism, \fullref{thm:equalizer_invertibility} implies that it is an epimorphism, and thus we can cancel it:
  \begin{equation*}
    \pi' \bincirc f' = \underbrace{0_{A,B}}_{\mathclap{0_{B,\co\ker f} \bincirc f'}}.
  \end{equation*}

  Hence, \( (\co\img f, f') \) is another equalizer cone for \( \pi' \), and there exists a unique homomorphism \( \widetilde{f}: \co\img f \to \img f \) such that \( f' = \pi \bincirc \widetilde{f} \).

  This is the desired homomorphism.
\end{proof}

\begin{proposition}\label{thm:zero_morphisms_pointed}
  Let \( \mscrL \) be a \hyperref[def:first_order_language]{first-order language} with no predicate symbols and a fixed nullary functional symbol \( \ast \). Let \( \Gamma \) be a theory over \( \mscrL \) that consists of \hyperref[def:positive_formula]{positive formulas} without existential quantifiers. Finally, let \( \cat{C} \) be the \hyperref[def:category_of_small_first_order_models]{category of \( \mscrU \)-small models} of \( \Gamma \), and suppose that \( \cat{C} \) has a \hyperref[def:trivial_object]{trivial object}.\footnote{All these requirements seem restrictive, yet the categories we will encounter will satisfy them.}

  We will denote the interpretation of \( \ast \) in the structure \( A \) via \( e_A \) rather than \( I_A(\ast) \).

  \begin{thmenum}
    \thmitem{thm:zero_morphisms_pointed/morphism} The unique \hyperref[def:zero_morphisms/morphism]{zero morphism} between the structures \( A \) and \( B \) is the function that sends each element of \( A \) to the origin of \( B \):
    \begin{equation*}
      \begin{aligned}
        &0_{A,B}: A \to B, \\
        &0_{A,B}(x) \coloneqq e_B.
      \end{aligned}
    \end{equation*}

    \thmitem{thm:zero_morphisms_pointed/kernel} The \hyperref[def:zero_morphisms/kernel]{categorical kernel} of a morphism \( f: A \to B \) is the \hyperref[def:set_valued_map/inverse]{preimage}
    \begin{equation*}
      f^{-1}(e_B) = \set{ x \in A \given f(x) = e_B }.
    \end{equation*}

    We denote this set via \( \ker f \). The kernel pair is \( (\ker f, \iota) \), where \( \iota: \ker f \to A \) is the canonical embedding.

    \thmitem{thm:zero_morphisms_pointed/cokernel} Dually, the \hyperref[def:zero_morphisms/cokernel]{categorical cokernel} of a morphism \( f: A \to B \) is the \hyperref[def:first_order_quotient]{quotient} of \( B \) by the \hyperref[def:first_order_congruence]{congruence} \hyperref[def:first_order_generated_congruence]{generated} by the relation \( {\sim} \) defined as
    \begin{equation*}
      y \sim e_B \T{if and only if} y \in f[A].
    \end{equation*}

    We denote this quotient via \( \co\ker f \). The cokernel pair is \( (\co\ker f, \pi) \), where \( \pi: B \to \co\ker f \) is the canonical projection.

    \thmitem{thm:zero_morphisms_pointed/image} The \hyperref[def:zero_morphisms/image]{categorical image} of a morphism \( f: A \to B \) is the \hyperref[def:set_valued_map/image]{set-theoretic image} \( f[A] \).

    \thmitem{thm:zero_morphisms_pointed/coimage} Dually, the \hyperref[def:zero_morphisms/coimage]{categorical coimage} of a morphism \( f: A \to B \) is the \hyperref[def:first_order_quotient]{quotient} of \( A \) by the \hyperref[def:first_order_congruence]{congruence} \hyperref[def:first_order_generated_congruence]{generated} by the relation \( {\sim} \) defined as
    \begin{equation*}
      x \sim e_A \T{if and only if} f(x) = e_B.
    \end{equation*}
  \end{thmenum}
\end{proposition}
\begin{proof}
  \SubProofOf{thm:zero_morphisms_pointed/morphism} Trivial.

  \SubProofOf{thm:zero_morphisms_pointed/kernel}

  \SubProof*{Proof that \( \ker f \) is a submodel of \( A \)} As the preimage of the submodel \( \set{ e_B } \) of \( B \), \( \ker f \) is a substructure of \( A \) as a consequence of \fullref{thm:def:first_order_homomorphism/preimage_is_substructure}. As a substructure, \( \ker f \) is a model of \( \Gamma \) as a consequence of \fullref{thm:substructure_is_model}.

  \SubProof*{Proof that \( (\ker f, \iota) \) is an equalizer cone} The following diagram must commute:
  \begin{equation*}
    \begin{aligned}
      \includegraphics[page=1]{output/thm__zero_morphisms_pointed}
    \end{aligned}
  \end{equation*}

  Indeed, if \( a \in \ker f \), then by definition \( f(\iota(x)) = f(a) = e_B \).

  \SubProof*{Proof that \( (\ker f, \iota) \) is universal} If \( (X, g) \) is another cone, there must exist a homomorphism \( \widetilde g: X \to \ker f \) such that the following diagram commutes:
  \begin{equation*}
    \begin{aligned}
      \includegraphics[page=2]{output/thm__zero_morphisms_pointed}
    \end{aligned}
  \end{equation*}

  In order for \( (X, g) \) to be a cone, for every element \( a \) of \( X \) we need \( f(g(a)) \) to be \( e_B \), which in turn requires \( g(a) \) to be in the preimage \( f^{-1}(e_B) \).

  Therefore, we can restrict the codomain of \( g \) to \( \ker f \) to obtain the desired homomorphism \( \widetilde g: X \to \ker f \).

  \SubProofOf{thm:zero_morphisms_pointed/cokernel}

  \SubProof*{Proof that \( \co\ker f \) is a model of \( \Gamma \)} Follows from \fullref{thm:quotient_preserves_positive_formulas}.

  \SubProof*{Proof that \( (\co\ker f, \pi) \) is a coequalizer cocone} The following diagram must commute:
  \begin{equation*}
    \begin{aligned}
      \includegraphics[page=3]{output/thm__zero_morphisms_pointed}
    \end{aligned}
  \end{equation*}

  Indeed, for every \( b \) in \( f[A] \), we must have \( \pi(b) = \pi(e_B) \), which is guaranteed by \( f(x) \cong e_B \).

  \SubProof*{Proof that \( (\co\ker f, \pi) \) is universal} If \( (X, g) \) is another cocone, there must exist a homomorphism \( \widetilde g: \co\img f \to Y \) such that the following diagram commutes:
  \begin{equation*}
    \begin{aligned}
      \includegraphics[page=4]{output/thm__zero_morphisms_pointed}
    \end{aligned}
  \end{equation*}

  In order for \( (\co\ker f, \pi) \) to be a cokernel, for every \( b \) in \( B \), even outside \( f[A] \), \( \widetilde g \) must satisfy
  \begin{equation}\label{eq:thm:zero_morphisms_pointed/cokernel/homomorphism}
    \widetilde{g}(\pi(b)) = g(b).
  \end{equation}

  We will now show that \( b \cong b' \) implies \( g(b) = g(b') \), which would allow us to use \eqref{eq:thm:zero_morphisms_pointed/cokernel/homomorphism} as a definition\footnote{We could also use \fullref{thm:quotient_structure_universal_property} to prove the statement, but it would not make the proof simpler.}.

  \Fullref{thm:homomorphism_induces_congruence} implies that the relation \( b \congdot b' \) defined to hold when \( g(b) = g(b') \) is a \hyperref[def:first_order_congruence]{congruence}. Furthermore, since \( (X, g) \) is a cocone, it must satisfy \( g(b) = e_Y \) whenever \( b \) is in \( f[A] \), which in turn implies that \( b \congdot b' \) for all pairs \( b \) and \( b' \) in \( f[A] \).

  But \( {\cong} \) is defined as the intersection of all congruences satisfying the latter condition. Therefore, \( {\cong} \) must be a subset of \( {\congdot} \).

  Then \( b \cong b' \) implies \( b \congdot b' \). Restated in terms of the respective homomorphisms, this means that \( \pi(b) = \pi(b') \) implies \( g(b) = g(b') \). Thus, the function \( \widetilde{g} \) is well-defined by \eqref{eq:thm:zero_morphisms_pointed/cokernel/homomorphism}.

  This shows that \( (\co\ker f, \pi) \) is a cokernel pair of \( f \).

  \SubProofOf{thm:zero_morphisms_pointed/image} Based on \fullref{thm:zero_morphisms_pointed/cokernel}, the cokernel of \( f \) is \( B / {\cong} \) based on the relation \( {\cong} \) generated by \( f(x) \sim e_B \) for all \( x \in A \), and the corresponding projection \( \pi \) is the cokernel morphism.

  Based on \fullref{thm:zero_morphisms_pointed/kernel}, the category-theoretic image of \( f \), the kernel of \( \pi \), is the preimage of the origin \( \pi(e_B) \) of \( B / {\cong} \) under \( \pi \), i.e. the equivalence class of \( e_B \) under \( {\cong} \).

  Therefore, the set-theoretic image \( f[A] \) necessarily belongs to the category-theoretic image \( \img f \).

  \Fullref{thm:def:first_order_homomorphism/image_is_substructure} implies that \( f[A] \) is a substructure of \( B \) and \fullref{thm:substructure_is_model} implies that it is a model of \( \Gamma \). Hence, \( f[A] \) is an object in the category \( \cat{C} \).

  As a substructure of \( \mscrY \), it is closed under function applications in \( \mscrY \), hence it contains the equivalence class of \( e_B \) under \( {\cong} \).

  Therefore, it \( f[A] \) is both a subset and a superset of \( \img f \), which means they are equal.

  \SubProofOf{thm:zero_morphisms_pointed/coimage} Based on \fullref{thm:zero_morphisms_pointed/kernel}, the categorical kernel of \( f \) is the preimage of \( e_B \) under \( f \), and the corresponding embedding \( \iota \) is the kernel morphism.

  Based on \fullref{thm:zero_morphisms_pointed/cokernel}, the category-theoretic coimage of \( f \), the cokernel of \( \iota \), is \( A / {\cong} \) based on the relation \( {\cong} \) generated by \( x \sim e_A \) for every \( x \) in \( \ker f = f^{-1}(e_B) \).
\end{proof}

\paragraph{Sets with involutions}

\begin{definition}\label{def:involution}\mcite[4]{Birkhoff1948}
  We say that an \hyperref[def:function/endofunction]{endofunction} \( f: X \to X \) on a \hyperref[def:set]{plain set} is an \term[ru=инволюция (\cite[333]{Арнольд2012})]{involution} if it is its own \hyperref[def:morphism_invertibility/isomorphism]{two-sided inverse}, that is,
  \begin{equation*}
    f \circ f^{-1} = f^{-1} \circ f = \id_X
  \end{equation*}
\end{definition}

\begin{definition}\label{def:set_with_involution}\mimprovised
  A \term{set with an involution} is a \hyperref[def:set]{set} \( X \) with a unary \hyperref[def:involution]{involution} \( (\anon)^{-1} \).

  Sets with involutions have the following metamathematical properties:
  \begin{thmenum}
    \thmitem{def:set_with_involution/theory} We define the theory of sets with involution as a theory over the language consisting of a single unary functional symbol \( \anon^{-1} \) with the sole axiom
    \begin{equation}\label{eq:def:set_with_involution/theory/axiom}
      (\xi^{-1})^{-1} \doteq \xi.
    \end{equation}

    \thmitem{def:set_with_involution/homomorphism} A \hyperref[def:first_order_homomorphism]{first-order homomorphism} between sets with involutions \( X \) and \( Y \) is a function \( \varphi: X \to Y \) satisfying
    \begin{equation}\label{eq:def:set_with_involution/homomorphism}
      \varphi(x^{-1})
      =
      \varphi(x)^{-1}.
    \end{equation}

    \thmitem{def:set_with_involution/submodel} Any subset of a set with involution is again a set with involution.

    In particular, as a consequence of \fullref{thm:positive_formulas_preserved_under_homomorphism}, the \hyperref[def:set_valued_map/image]{image} of a homomorphism \( \varphi: X \to Y \) is a submodel of \( Y \).

    The \hyperref[thm:substructures_form_complete_lattice/bottom]{bottom substructure} of any set with involution is the empty set.

    See \fullref{rem:empty_first_order_structures} regarding allowing empty sets as first-order structures.

    \thmitem{def:set_with_involution/category} We introduce no special notation for the corresponding \hyperref[def:category_of_small_first_order_models]{category of \( \mscrU \)-small models}.
  \end{thmenum}
\end{definition}

\paragraph{Semigroups}

\begin{definition}\label{def:binary_operation}
  An important special case of \hyperref[def:operation_on_set]{operations} are \term{binary operations}. Given a set \( A \), a binary operation is simply a function \( \cdot: A^2 \to A \), usually denoted via \hyperref[rem:first_order_formula_conventions/infix]{infix notation}.

  The following \hyperref[def:first_order_syntax/formula]{formulas}, or rather their \hyperref[def:universal_closure]{universal closures}, are common \hyperref[def:first_order_theory/axiomatized]{axioms}:
  \begin{thmenum}
    \thmitem{def:binary_operation/associative}\mcite[\( \S 1.18(\alpha) \)]{Ляпин1960} \term[bg=асоциативност (\cite[prop. 1.13]{Соскова2015}), ru=ассоциативность (\cite[21]{Яблонский1986})]{Associativity}:
    \begin{equation}\label{eq:def:binary_operation/associative}
      (\xi \cdot \eta) \cdot \zeta = \xi \cdot (\eta \cdot \zeta).
    \end{equation}

    \thmitem{def:binary_operation/commutative}\mcite[\( \S 1.18(\beta) \)]{Ляпин1960} \term[bg=комутативност (\cite[prop. 1.13]{Соскова2015}), ru=коммутативность (\cite[21]{Яблонский1986})]{Commutativity}:
    \begin{equation}\label{eq:def:binary_operation/commutative}
      \xi \cdot \eta \doteq \eta \cdot \xi.
    \end{equation}

    \thmitem{def:binary_operation/idempotent}\mcite[\( \S 3.1 \)]{Мальцев1970} \term[bg=идемпотентност (\cite[prop. 1.13]{Соскова2015}), ru=идемпотентность (\cite[90]{Мальцев1970})]{Idempotence}:
    \begin{equation}\label{eq:def:binary_operation/idempotent}
      \xi \cdot \xi \doteq \xi.
    \end{equation}

    \thmitem{def:binary_operation/cancellative}\mcite[19]{MacLane1998} Left and right \term[bg=съкращаване (\cite[4]{КоцевСидеров2016}), ru=сокращение (\cite[\( 1.18(\varepsilon) \)]{Ляпин1960})]{cancellation}:
    \begin{align}
      \qforall \zeta (\zeta \cdot \xi \doteq \zeta \cdot \eta) \rightarrow \xi = \eta
      \label{eq:def:binary_operation/cancellative/left}
      \\
      \qforall \zeta (\xi \cdot \zeta \doteq \eta \cdot \zeta) \rightarrow \xi = \eta
      \label{eq:def:binary_operation/cancellative/right}
    \end{align}
  \end{thmenum}
\end{definition}

\begin{example}\label{ex:def:binary_operation}
  We list several examples of \hyperref[def:semigroup]{semigroups} satisfying different properties.

  \begin{thmenum}
    \thmitem{ex:def:binary_operation/algebraic} \hyperref[eq:def:binary_operation/associative]{Associative} \hyperref[def:binary_operation]{binary operations} on a set are abundant and are part of the definition of essential algebraic structures like \hyperref[def:group]{groups}, \hyperref[def:ring]{rings}, \hyperref[def:vector_space]{vector spaces} and \hyperref[def:semilattice]{(semi)lattices}.

    These operations are \term{homogeneous} in the sense that their signature only contains a single set, unlike \hyperref[def:group_action]{group actions} and scalar products in \hyperref[def:vector_space]{vector spaces}.

    \thmitem{ex:def:binary_operation/composition} The quintessential example of a non-\hyperref[def:binary_operation/commutative]{commutative} operation is \hyperref[def:set_valued_map/composition]{composition} in any set of functions or, more generally, \hyperref[def:category/composition]{morphism composition} in any \hyperref[def:category]{category}.

    Composition is \hyperref[def:binary_operation/associative]{associative}. \hyperref[def:binary_operation/cancellative]{Cancellation} with respect to composition is discussed in \fullref{def:morphism_invertibility} and, for function composition, in \fullref{thm:function_invertibility_categorical}.

    \thmitem{ex:def:binary_operation/midpoint} The midpoint operation
    \begin{equation*}
      (x, y) \mapsto \dfrac {x + y} 2
    \end{equation*}
    on \( \BbbR \) is commutative and cancellative semigroup, but not associative.
  \end{thmenum}
\end{example}

\begin{remark}\label{rem:binary_operation_syntax_trees}
  \hyperref[def:binary_operation]{Binary operations} are easily extended to higher arities. Given a binary operation \( +: A \times A \to A \), we can extend it via \hyperref[rem:natural_number_recursion]{natural number recursion} to arbitrary \( n \)-tuples \( x_1, \ldots, x_n \) via parentheses:
  \begin{equation}\label{eq:rem:binary_operation_syntax_trees/right}
    (x_1 + (x_2 + \cdots + (x_{n-1} + x_n) \cdots )).
  \end{equation}

  This expression corresponds to the \hyperref[rem:abstract_syntax_tree]{abstract syntax tree}
  \begin{equation}\label{eq:rem:binary_operation_syntax_trees/tree/basic}
    \begin{aligned}
      \includegraphics[page=1]{output/rem__binary_operation_syntax_trees}
    \end{aligned}
  \end{equation}

  Several things can be noted here.
  \begin{thmenum}
    \thmitem{rem:binary_operation_syntax_trees/associativity} When exchanging the order of the parentheses in the expression \eqref{eq:rem:binary_operation_syntax_trees/right}, only the root is changed in the syntax tree \eqref{eq:rem:binary_operation_syntax_trees/tree/basic}.

    Thus, for an \hyperref[def:binary_operation/associative]{associative} binary operation, abstract syntax trees can instead be represented as finite ordered rooted trees like
    \begin{equation}\label{eq:rem:binary_operation_syntax_trees/tree/associative}
      \begin{aligned}
        \includegraphics[page=2]{output/rem__binary_operation_syntax_trees}
      \end{aligned}
    \end{equation}

    Of course, \( n \)-ary trees can still be used for non-associative binary operations, as long as we have selected a strategy for \hyperref[rem:evaluation]{evaluation}. If we evaluate \eqref{eq:rem:binary_operation_syntax_trees/tree/associative} as we would evaluate \eqref{eq:rem:binary_operation_syntax_trees/right}, we say that the operation is \term{right associative}. Dually, if we evaluate \eqref{eq:rem:binary_operation_syntax_trees/tree/associative} as we would evaluate
    \begin{equation}\label{eq:rem:binary_operation_syntax_trees/left}
      (( \cdots (x_1 + x_2) + \cdots + x_{n-1}) + x_n),
    \end{equation}
    we say that the operation is \term{left associative}.

    Note that, unlike associativity, left and right associativity are not properties of the operation, but rather conventions on how to evaluate expressions without explicit parentheses. For example, in a \hyperref[def:heyting_algebra]{Heyting algebra}, the \hyperref[eq:def:heyting_algebra/conditional]{conditional} \( \rightarrow \) is not associative, but it is often taken to be right associative so that
    \begin{equation*}
      x \rightarrow y \rightarrow z
    \end{equation*}
    is evaluated as
    \begin{equation*}
      (x \rightarrow (y \rightarrow z)).
    \end{equation*}

    \thmitem{rem:binary_operation_syntax_trees/commutativity} If, additionally, the operation is \hyperref[def:binary_operation/commutative]{commutative}, we can regard the syntax tree as if the tree was a plain unordered (but still node-labeled) tree.

    Commutativity allows us to swap summands inside parentheses, and associativity is needed to \enquote{remove} the parentheses. Thus, it makes sense to always place parentheses in operations that are commutative but not associative.

    An an example, consider the real number midpoint operation
    \begin{equation*}
      x \oplus y \coloneqq \frac {x + y} 2
    \end{equation*}
    from \fullref{ex:def:binary_operation/midpoint}. It is commutative and not associative, and in the expression \( x \oplus y \oplus z \), we can swap \( x \) with \( y \) but not with \( z \). This can be very unintuitive. We aim to always write parentheses for non-associative operations.

    \thmitem{rem:binary_operation_syntax_trees/infinite} Suppose that \( + \) is associative and commutative. Suppose also that we are given an \hyperref[def:cartesian_product/indexed_family]{indexed family} \( \seq{ x_k }_{k \in \mscrK} \) of elements of \( M \). We can construct an infinite tree with root \( + \) and children \( \seq{ x_k }_{k \in \mscrK} \).

    This is not strictly a syntax tree. Syntax trees are necessarily finite, and our family \( \mscrK \) may even be uncountable. Nevertheless, we can sometimes evaluate this tree to obtain a member of the monoid.
    \begin{thmenum}
      \thmitem{rem:binary_operation_syntax_trees/infinite/lattice} The operations of a \hyperref[def:semilattice/lattice]{lattice} arise by specializing \hyperref[def:extremal_points/supremum_and_infimum]{suprema and infima} to binary sets. Conversely, as discussed in \fullref{thm:binary_lattice_operations/new_lattice}, the binary lattice operations induce a \hyperref[def:partially_ordered_set]{partial order}, and we can define arbitrary suprema and infima. If the lattice happens to be \hyperref[def:semilattice/complete]{complete}, instead of the binary operations, we can use
      \begin{equation*}
        \bigvee_{k \in \mscrK} x_k = \sup\set{ x_k \given k \in \mscrK }.
      \end{equation*}

      \thmitem{rem:binary_operation_syntax_trees/infinite/convergence} If \( M \) is a \hyperref[rem:topological_first_order_structures]{topological monoid} and if \( \mscrK = \set{ 1, 2, 3, \ldots } \), the family is a \hyperref[def:sequence]{sequence}, and we can define the sequence of partial sums
      \begin{equation*}
        \seq*{ \sum_{k=1}^n x_k }_{n=1}^\infty
      \end{equation*}

      This gives rise to \hyperref[def:convergent_series]{series} discussed in \fullref{subsec:series} and \fullref{subsec:real_series}. A limit may not exist for the net, unfortunately, and if it does, it may not be unique (if the topology is not \hyperref[def:separation_axioms/T2]{Hausdorff}).

      \thmitem{rem:binary_operation_syntax_trees/infinite/direct_sum} Suppose that \( M \) has an \hyperref[def:monoid]{neutral element} \( 0_M \). If only finitely many elements of the family are different from \( 0_M \), we regard the ordinary summation operation as well-defined on the whole family and write
      \begin{equation*}
        s \coloneqq \sum_{k \in \mscrK} x_k.
      \end{equation*}

      Technically, this involves selecting a \hyperref[def:well_ordered_set]{well-ordering} \( x_{1_n}, \ldots, x_{k_n} \) on the set
      \begin{equation*}
        \set{ x_k \given k \in \mscrK \T{and} x_k \neq 0 }
      \end{equation*}
      and assigning to \( s \) the result of the iterated binary operation
      \begin{equation*}
        (x_{1_n} + (x_{2_n} + \cdots + (x_{k_{n-1}} + x_{k_n}) \cdots)).
      \end{equation*}

      Commutativity ensures that the sum \( s \) does not depend on the order of summands (and hence on the well-order we have chosen). Adding any member of the family would not change the sum, which justifies this shorthand definition. Furthermore, since we only sum finitely many summands, we can construct a well-ordering using \hyperref[rem:natural_number_recursion]{natural number recursion} without relying on the \hyperref[def:zfc/choice]{axiom of choice}.

      This is fundamental for the definition of \hyperref[def:first_order_direct_product]{direct sums}, which in turn are used to define \hyperref[rem:linear_combinations]{linear combinations} and \hyperref[def:polynomial_algebra]{polynomials}.
    \end{thmenum}
  \end{thmenum}
\end{remark}

\begin{remark}\label{rem:magma_terminology}
  The term \enquote{magma} for a set with a binary operation was used by Serre in his 1964 lectures. Decades later, however, it is still not widely established. \incite[example 21.3]{Golan2010} uses the term \enquote{groupoid} instead, but the latter term seems to be overtaken by groupoids in category theory --- see \fullref{def:groupoid}. \incite{Ляпин1960} calls them \enquote{multiplicative sets}, noting that the terminology is not established and that some authors use the term \enquote{groupoid}.

  Semigroups, being a special case when the operation is \hyperref[def:binary_operation/associative]{associative}, are well-established concepts, unlike magmas. The term is defined by \incite[144]{MacLane1998}, \incite[1]{Golan2010} and \incite[28]{Ляпин1960}, and used without definition by many others, including \incite[162]{Birkhoff1948} and \incite[402]{Rockafellar1997}.

  Since non-associative operation rarely arise, we prefer defining semigroups rather than magmas.
\end{remark}

\begin{definition}\label{def:semigroup}\mcite[144]{MacLane1998}
  A \term[ru=полугруппа (\cite[28]{Ляпин1960})]{semigroup} is a nonempty set \( G \) equipped with a \hyperref[rem:function_arguments]{binary function} \( \cdot: G \times G \to G \), called the \term{semigroup operation}. We often denote this operation by juxtaposition as \( xy \) instead of \( x \cdot y \).

  We often call the operation \term{multiplication} or, in the case of \hyperref[def:endomorphism_monoid]{endomorphism monoids} --- \term{composition}. See also the notes in \fullref{rem:additive_semigroup} regarding additive semigroups and in \fullref{def:monoid_delooping} regarding the order of operands.

  \begin{thmenum}[series=def:semigroup]
    \thmitem{def:semigroup/theory} In analogy to the \hyperref[def:pointed_set/theory]{theory of pointed sets}, we can define the theory of semigroups as an empty theory over a language with a single \hyperref[rem:first_order_formula_conventions/infix]{infix} binary functional symbol.

    \thmitem{def:semigroup/homomorphism} A \hyperref[def:first_order_homomorphism]{first-order homomorphism} between the semigroups \( (G, \cdot_{G}) \) and \( (H, \cdot_{H}) \) is, explicitly, a function \( \varphi: G \to H \) such that
    \begin{equation}\label{eq:def:semigroup/homomorphism}
      \varphi(x \cdot_{G} y) = \varphi(x) \cdot_{H} \varphi(y)
    \end{equation}
    for all \( x, y \in G \).

    \thmitem{def:semigroup/submodel} The set \( A \subseteq G \) is a \hyperref[def:first_order_substructure]{first-order submodel} of \( G \) if it is closed under the semigroup operation, that is, if \( x, y \in A \) implies \( xy \in A \).

    We call \( A \) a \term{sub-semigroup} of \( G \).

    As a consequence of \fullref{thm:positive_formulas_preserved_under_homomorphism}, the image of a semigroup homomorphism is a sub-semigroup of its codomain.

    \thmitem{def:semigroup/category} We introduce no special notation for the \hyperref[def:category_of_small_first_order_models]{category of \( \mscrU \)-small models} for the theory of semigroups.

    \thmitem{def:semigroup/exponentiation} We define an additional \term{exponentiation} operation for positive integers \( n \) \hyperref[rem:natural_number_recursion]{recursively} as
    \begin{equation}\label{eq:def:semigroup/exponentiation}
      x^n \coloneqq \begin{cases}
        x,               &n = 1 \\
        x^{n-1} \cdot x, &n > 1
      \end{cases}
    \end{equation}

    \thmitem{def:semigroup/opposite} The \term{opposite semigroup} of \( (G, \cdot) \) is the semigroup \( (G, \star) \) with multiplication reversed:
    \begin{equation*}
      x \star y \coloneqq y \cdot x.
    \end{equation*}

    We denote the opposite semigroup by \( G^{\opcat} \). This is justified in \fullref{def:monoid/opposite}.
  \end{thmenum}
\end{definition}

\begin{definition}\label{def:power_semigroup}\mcite[1]{McCarthyHayes1973}
  Given a \hyperref[def:semigroup]{semigroup} \( (G, \cdot) \), its \hyperref[def:basic_set_operations/power_set]{power set} \( \pow(G) \) is also a semigroup with the operation
  \begin{equation*}
    A \star B \coloneqq \set{ a \cdot b \given a \in A \T{and} b \in B }.
  \end{equation*}

  We will call this the \term{power semigroup} of \( G \). There is an obvious injection
  \begin{equation*}
    \begin{aligned}
      &\iota: G \to \pow(G), \\
      &\iota(g) \coloneqq \set{ g }.
    \end{aligned}
  \end{equation*}
\end{definition}
\begin{comments}
  \item Operations on sets are used for \hyperref[def:topological_vector_space]{topological vector spaces} by \incite[6]{Rudin1991Functional}, \incite[4]{Rockafellar1997}, \incite[4]{Clarke2013}, Georgii Magaril-Ilyaev, \incite[25]{МагарилИльяевТихомиров2002ВыпуклыйАнализ}.

  \item Note that this concept is distinct from \hyperref[def:semiring_ideal/product]{product ideals}, which use the same notation.

  \item Power semigroups are ordered --- see \fullref{ex:def:ordered_semigroup/power}
\end{comments}

\begin{example}\label{ex:def:power_semigroup}
  We list some examples of \hyperref[def:power_semigroup]{power semigroups}:
  \begin{thmenum}
    \thmitem{ex:def:power_semigroup/cancellative} The cancellative property does not extend to the power semigroup.

    Consider the additive group \hyperref[def:group_of_integers_modulo]{\( \BbbZ_2 \)}. It is a cancellative semigroup as a consequence of \fullref{thm:def:group/cancellative}. Define the sets \( A \coloneqq \set{ 0, 1 } \) and \( B \coloneqq \set{ 0 } \). Then
    \begin{equation*}
      A + A = A = A + B,
    \end{equation*}
    however we cannot cancel \( A \) from the left because \( A \neq B \).

    \thmitem{ex:def:power_semigroup/semiring} If \( R \) is a \hyperref[def:semiring]{semiring}, then \( \pow(R) \) is a semigroup with respect to both operations, however \( \pow(R) \) is not a semiring because the operations do not \hyperref[def:semiring/left_distributivity]{distribute}.

    Consider the ring of integers and let
    \begin{align*}
      A = \set{ -1, 1 },
      &&
      B = \set{ -1 },
      &&
      C = \set{ 1 }.
    \end{align*}

    Then
    \begin{equation*}
      A(\underbrace{B + C}_{\set{ 0 }})
      =
      \set{ 0 },
    \end{equation*}
    however
    \begin{equation*}
      \underbrace{AB}_{A} + \underbrace{AC}_{A}
      =
      \set{ -2, 0, 0, 2 }.
    \end{equation*}
  \end{thmenum}
\end{example}

\begin{proposition}\label{thm:semigroup_exponentiation_properties}
  Fix a semigroup \( G \). \hyperref[def:semigroup/exponentiation]{Exponentiation} in \( G \) has the following basic properties:

  \begin{thmenum}
    \thmitem{thm:semigroup_exponentiation_properties/commutativity} We have the following \hyperref[def:binary_operation/commutative]{commutativity}-like property: for any \( x \) in \( G \) and any positive integer \( n \), we have
    \begin{equation}\label{eq:thm:semigroup_exponentiation_properties/commutativity}
      x^n = x x^{n-1} = x^{n-1} x.
    \end{equation}

    \thmitem{thm:semigroup_exponentiation_properties/distributivity} Exponentiation distributes over multiplication: for any member \( x \) of \( G \) and any two positive integers \( n \) and \( m \),
    \begin{equation}\label{eq:thm:semigroup_exponentiation_properties/multiplication}
      x^{n + m} = x^n x^m.
    \end{equation}

    \thmitem{thm:semigroup_exponentiation_properties/repeated} For any member \( x \) of \( M \) and any two positive integers \( n \) and \( m \),
    \begin{equation}\label{eq:thm:semigroup_exponentiation_properties/repeated}
      (x^n)^m = x^{nm}.
    \end{equation}
  \end{thmenum}
\end{proposition}
\begin{proof}
  \SubProofOf{thm:semigroup_exponentiation_properties/commutativity} We use induction on \( n \). The cases \( n = 1 \) and \( n = 2 \) are obvious. For \( n > 2 \), we have
  \begin{equation*}
    x^n
    \reloset {\eqref{eq:def:semigroup/exponentiation}} =
    x x^{n-1}
    \reloset {\T{ind.}} =
    x x^{n-2} x
    \reloset {\eqref{eq:def:semigroup/exponentiation}} =
    x^{n-1} x.
  \end{equation*}

  \SubProofOf{thm:semigroup_exponentiation_properties/distributivity} We use induction on \( n \). The case \( n = 1 \) follows directly from \eqref{eq:def:semigroup/exponentiation}. The case \( n > 1 \) follows from
  \begin{equation*}
    x^{n + m}
    \reloset {\eqref{eq:def:semigroup/exponentiation}} =
    x x^{n + (m - 1)}
    \reloset {\T{ind.}} =
    x x^{n-1} x^m
    \reloset {\eqref{eq:def:semigroup/exponentiation}} =
    x^n x^m.
  \end{equation*}

  \SubProofOf{thm:semigroup_exponentiation_properties/repeated} We use induction on \( n \). The case \( n = 1 \) is obvious and the rest follows from
  \begin{equation*}
    (x^n)^m
    \reloset {\eqref{eq:def:semigroup/exponentiation}} =
    x^n (x^n)^{m-1}
    \reloset {\T{ind.}} =
    x^n x^{n (m - 1)}
    \reloset {\eqref{eq:thm:semigroup_exponentiation_properties/multiplication}} =
    =
    x^{nm}.
  \end{equation*}
\end{proof}

\begin{definition}\label{def:ordered_semigroup}\mcite[535]{Ляпин1960}
  A (partially) \term[ru=частично упорядоченная полугруппа (\cite[535]{Ляпин1960})]{ordered semigroup} is a \hyperref[def:semigroup]{semigroup} \( M \) equipped with a \hyperref[def:partially_ordered_set]{partial order} \( \leq \) such that \( x \leq y \) implies \( xz \leq yz \) and \( zx \leq zy \) for every \( z \) in \( M \).
\end{definition}
\begin{comments}
  \item The category of small ordered semigroups is a \hyperref[def:concrete_category]{concrete category} over both the \hyperref[def:semigroup/category]{category of semigroups} and the \hyperref[def:partially_ordered_set]{category of partially ordered sets}.
\end{comments}

\begin{example}\label{ex:def:ordered_semigroup}
  We list several examples of \hyperref[def:ordered_semigroup]{ordered semigroups}:

  \begin{thmenum}
    \thmitem{ex:def:ordered_semigroup/natural_numbers} The \hyperref[def:natural_numbers]{natural numbers} with addition form an ordered semigroup as a consequence of \fullref{thm:natural_numbers_are_well_ordered}; and so do \( \BbbZ \), \( \BbbQ \), \( \BbbR \) and \( \BbbC \).

    \thmitem{ex:def:ordered_semigroup/ordinal} More generally, every \hyperref[def:successor_and_limit_ordinal]{limit ordinal} as the set of all smaller ordinals is an ordered semigroup under \hyperref[def:ordinal_arithmetic/addition]{ordinal addition}.

    Unlike addition in natural numbers, however, ordinal addition is not commutative as shown in \fullref{ex:ordinal_addition}.

    \thmitem{ex:def:ordered_semigroup/semilattice} Every \hyperref[def:semilattice/join]{join-semilattice} \( (X, \vee) \) is an ordered semigroup with the lattice order. Indeed, if \( x \leq y \), then
    \begin{itemize}
      \item If \( z \leq x \), then
      \begin{equation*}
        x = x \vee z \leq y \vee z = y.
      \end{equation*}

      \item If \( x \leq z \leq y \), then
      \begin{equation*}
        z = x \vee z \leq y \vee z = y.
      \end{equation*}

      \item If \( z \geq y \), then
      \begin{equation*}
        z = x \vee z \leq y \vee z = z.
      \end{equation*}
    \end{itemize}

    Every \hyperref[def:semilattice/meet]{meet-semilattice} is also an ordered semigroup.

    \thmitem{ex:def:ordered_semigroup/power} Any \hyperref[def:power_semigroup]{power semigroup} is a Boolean algebra as discussed in \fullref{thm:boolean_algebra_of_subsets}, and hence also an ordered semigroup as a consequence of \fullref{ex:def:ordered_semigroup/semilattice}.
  \end{thmenum}
\end{example}

\paragraph{Monoids}

\begin{definition}\label{def:monoid}\mcite[vii]{MacLane1998}
  A \term[ru=моноид (\cite[94]{Мальцев1970})]{monoid} is an \hyperref[eq:def:binary_operation/associative]{associative} \hyperref[def:semigroup]{semigroup} with a distinguished element \( e \) such that \( ex = x = xe \) for every \( x \). Such an element is obviously unique, and we call it the \term{neutral element} of the monoid. This makes monoids \hyperref[def:pointed_set]{pointed sets}.

  \begin{thmenum}
    \thmitem{def:monoid/theory} The theory of monoids consists of \hyperref[eq:def:binary_operation/associative]{associativity} and the axiom
    \begin{equation}\label{eq:def:monoid/theory/neutral}
      \qforall \xi (e \cdot \xi \doteq \xi \wedge \xi \cdot e \doteq \xi)
    \end{equation}
    over the combined language of \hyperref[def:pointed_set/theory]{pointed sets} and \hyperref[def:semigroup/theory]{semigroups}.

    \thmitem{def:monoid/homomorphism} A \hyperref[def:first_order_homomorphism]{first-order homomorphism} between monoids is a function that is both a \hyperref[def:pointed_set/homomorphism]{pointed set homomorphism} and a \hyperref[def:semigroup/homomorphism]{semigroup homomorphism}, that is, it satisfies both \eqref{eq:def:pointed_set/homomorphism} and \eqref{eq:def:semigroup/homomorphism}.

    \thmitem{def:monoid/submodel} The set \( A \subseteq M \) is a \hyperref[def:first_order_submodel]{first-order submodel} of the monoid \( M \) if it contains \( e \). This is equivalent to \( A \) being a pointed subset.

    We say that \( A \) is a \term{submonoid}.

    As a consequence of \fullref{thm:positive_formulas_preserved_under_homomorphism}, the image of a monoid homomorphism is a submonoid of its codomain.

    \thmitem{def:monoid/category} The \hyperref[def:category_of_small_first_order_models]{category of \( \mscrU \)-small models} \( \ucat{Mon} \) of monoids is \hyperref[def:concrete_category]{concrete} with respect to both the \hyperref[def:pointed_set/category]{category of \( \mscrU \)-small pointed sets} and \hyperref[def:semigroup/category]{category of \( \mscrU \)-small semigroups}.

    As such, it has a \hyperref[def:universal_objects/zero]{zero object} --- any one-element monoid --- and it satisfies \fullref{thm:zero_morphisms_pointed}.

    We will discuss the free monoid functor in \fullref{thm:free_monoid_universal_property}. Then from \fullref{thm:first_order_categorical_invertibility} it will follow that monoid monomorphisms are injective functions.

    For epimorphisms a similar statement does not hold, unfortunately. The embedding \( \iota: \BbbN \to \BbbZ \) is an epimorphism by \fullref{thm:grothendieck_monoid_completion_universal_property}, but it is clearly not surjective.

    \thmitem{def:monoid/trivial} Any single-element monoid, i.e. every zero object in \( \ucat{Mon} \), is as trivial object in the sense of \fullref{def:trivial_object}. We often call it \enquote{the} trivial monoid.

    \thmitem{def:monoid/exponentiation} We extend \hyperref[def:semigroup/exponentiation]{semigroup exponentiation} to all nonnegative integers by additionally defining
    \begin{equation*}
      x^0 \coloneqq e.
    \end{equation*}

    \thmitem{def:monoid/commutative} We usually write \hyperref[def:binary_operation/commutative]{commutative} monoids additively as explained in \fullref{rem:additive_semigroup}. We denote the subcategory of commutative monoids by \( \ucat{CMon} \).

    \thmitem{def:monoid/opposite} The \hyperref[def:monoid_delooping]{delooping} of the \hyperref[def:semigroup/opposite]{opposite semigroup} for a monoid \( M \) is the \hyperref[def:opposite_category]{opposite category} of the delooping of \( M \). This justifies the notation \( M^{\opcat} \)
  \end{thmenum}
\end{definition}
\begin{comments}
  \item The requirement of associativity is conventional but not strictly necessary. Non-associative monoids will not be useful to us, however.
\end{comments}

\begin{example}\label{ex:def:monoid}
  We list several important examples \hyperref[def:monoid]{monoids}.

  \begin{thmenum}
    \thmitem{ex:def:monoid/natural_numbers} The \hyperref[def:natural_numbers]{nonnegative natural numbers} with addition form a quintessential example of a monoid. We prove in \fullref{thm:natural_number_addition_properties} that they are a monoid.

    \thmitem{ex:def:monoid/kleene_star} Another important example of a monoid is the \hyperref[def:formal_language/kleene_star]{Kleene star} \( \mscrA \) over some \hyperref[def:formal_language]{alphabet} \( \mscrA \).

    The importance for monoid theory comes from the free monoid universal property described in \fullref{thm:free_monoid_universal_property}.

    \thmitem{ex:def:monoid/semilattice} Every \hyperref[def:semilattice/bounded]{bounded} \hyperref[def:semilattice/join]{join-semilattice} is a monoid as a consequence of \eqref{eq:thm:binary_lattice_operations/neutral/meet}, and similarly for \hyperref[def:semilattice/meet]{meet-semilattice}.

    \thmitem{ex:def:monoid/power} The \hyperref[def:power_semigroup]{power semigroup} \( \pow(M) \) of a monoid \( M \) is also a monoid --- \( \iota(1_M) = \set{ 1_M } \) is an neutral element.
  \end{thmenum}
\end{example}

\begin{example}\label{ex:monoid_cancellation_not_preserved_by_homomorphism}\mcite{MathSE:semigroup_cancellation_not_preserved}
  \hyperref[def:monoid/homomorphism]{Monoid homomorphisms} may not preserve the \hyperref[def:binary_operation/cancellative]{cancellation property}. For example, the \hyperref[def:natural_numbers]{natural numbers} \( \BbbN \) are a cancellative monoid under addition, as shown in \fullref{thm:natural_number_addition_properties}, but the homomorphism
  \begin{equation*}
    \begin{aligned}
      &h: (\BbbN, +) \to (\BbbF_2, \max) \\
      &h(n) \coloneqq \begin{cases}
        0, &n = 0 \\
        1, &n > 0
      \end{cases}
    \end{aligned}
  \end{equation*}
  does not preserve the cancellation property.

  Indeed, \( \max\set{ 0, 1 } = \max\set{ 1, 1 } \), but \( 0 \neq 1 \).
\end{example}

\paragraph{Direct sums}

\begin{definition}\label{def:direct_sum}\mimprovised
  Let \( \Gamma \) be an extension of the \hyperref[def:pointed_set/theory]{first-order theory of pointed sets} with \hyperref[def:positive_formula]{positive formulas} \hi{without disjunctions or existential quantifiers}.

  The \term{external direct sum} or simply \term{direct sum} \( \bigoplus_{k \in \mscrK} \mscrX_k \) of the family of models \( \seq{ \mscrX_k }_{k \in \mscrK} \) of \( \Gamma \) is the \hyperref[def:first_order_submodel]{submodel} of their \hyperref[def:first_order_direct_product]{direct product} \( \prod_{k \in \mscrK} \mscrX_k \) where only finitely many members of each tuple \( \seq{ x_k }_{k \in \mscrK} \) differ from the origin.

  In case all summands are equal to \( \mscrX \), we denote the direct sum by \( \mscrX^{\oplus \mscrK} \).

  If all summands are submodels of \( \mscrX \) and if the sum \( \bigoplus_{k \in \mscrK} \mscrX_k \) is isomorphic to \( \mscrX \), we call it an \term{internal direct sum} and treat the tuple \( \seq{ x_k }_{k \in \mscrK} \) as the product \( x_{k_1} x_{k_2} \cdots x_{k_n} \) of the elements of \( \mscrX \) distinct from the origin.
\end{definition}
\begin{defproof}
  \Fullref{thm:direct_product_preserves_positive_formulas} ensures that the direct product is a model of \( \Gamma \) and \fullref{thm:substructure_is_model} ensures that the direct sum is a model of \( \Gamma \) as a substructure of the product.
\end{defproof}

\begin{proposition}\label{thm:monoid_categorical_limits}
  \hfill
  \begin{thmenum}
    \thmitem{thm:monoid_categorical_limits/product} The \hyperref[def:discrete_category_limits]{categorical product} of the family \( \seq{ M_k }_{k \in \mscrK} \) in the category \hyperref[def:monoid/category]{\( \cat{Mon} \)} of monoids is their \hyperref[def:first_order_direct_product]{direct product} \( \prod_{k \in \mscrK} M_k \).

    \thmitem{thm:monoid_categorical_limits/coproduct} The \hyperref[def:discrete_category_limits]{categorical coproduct} of the family \( \seq{ M_k }_{k \in \mscrK} \) in the category \hyperref[def:monoid/category]{\( \cat{CMon} \)} of \hi{commutative} monoids is their \hyperref[def:direct_sum]{direct sum} \( \bigoplus_{k \in \mscrK} M_k \).
  \end{thmenum}
\end{proposition}
\begin{comments}
  \item Compare \fullref{thm:monoid_categorical_limits/coproduct} to the non-commutative case for groups discussed in \fullref{thm:group_categorical_limits/coproduct}.
\end{comments}
\begin{proof}
  \SubProofOf{thm:monoid_categorical_limits/product} Let \( (A, \alpha) \) be a \hyperref[def:category_of_cones/cone]{cone} for the \hyperref[def:discrete_category]{discrete} \hyperref[def:categorical_diagram]{diagram} \( \seq{ M_k }_{k \in \mscrK} \). We want to define a monoid homomorphism \( l_A: A \to \prod_{k \in \mscrK} M_k \) such that, for every \( m \in \mscrK \) and \( a \in A \),
  \begin{equation*}
    \alpha_m(a) = \pi_m(l_A(a)).
  \end{equation*}

  This suggests the definition
  \begin{equation*}
    l_A(a) \coloneqq \seq{ \alpha_k(a) }_{k \in \mscrK}.
  \end{equation*}

  \SubProofOf{thm:monoid_categorical_limits/coproduct}  Let \( (A, \alpha) \) be a \hyperref[def:category_of_cones/cocone]{cocone} for the discrete diagram \( \seq{ M_k }_{k \in \mscrK} \). We want to define a monoid homomorphism \( l_A: \bigoplus_{k \in \mscrK} M_k \to A \) such that, for every \( m \in \mscrK \) and \( x \in M_m \),
  \begin{equation*}
    \alpha_m(x) = l_A(\iota_m(x)).
  \end{equation*}

  This suggests the definition
  \begin{equation*}
    l_A(\seq{ x_k }_{k \in \mscrK}) \coloneqq \prod_{k \in \mscrK}^n \alpha_k(x_k).
  \end{equation*}

  We discuss well-definedness of infinitary operations in direct sums in \fullref{rem:binary_operation_syntax_trees/infinite/direct_sum}.
\end{proof}

  \subsection{Groups}\label{subsec:groups}

\begin{definition}\label{def:monoid_inverse}
  Let \( M \) be a \hyperref[def:monoid]{monoid}. We say that \( y \) is the \term{left inverse} (resp. \term{right inverse}) of \( x \) if \( yx = e \) (resp. \( xy = e \)).

  If \( y \) is simultaneously a left and right inverse of \( x \), we call a \term{two-sided inverse} or simply an \term{inverse} of \( x \) and denote it by \( x^{-1} \). It is unique by \fullref{thm:monoid_inverse_unique}. This notation is consistent with monoid exponentiation defined in \fullref{def:monoid/exponentiation}.

  We call \( x \) \term{invertible} if it has a two-sided inverse.
\end{definition}

\begin{proposition}\label{thm:monoid_inverse_unique}
  For every element \( x \) of any monoid, the (two-sided) \hyperref[def:monoid_inverse]{inverse} \( x^{-1} \) of \( x \), if it exists, is unique.
\end{proposition}
\begin{proof}
  If \( y \) and \( z \) are both inverses of \( x \), then
  \begin{equation*}
    y = ey = zxy = ze = z.
  \end{equation*}
\end{proof}

\begin{definition}\label{def:zero_morphisms}
  Let \( \cat{C} \) be a \hyperref[def:universal_objects/zero]{pointed category} with a fixed \hyperref[def:universal_objects/zero]{zero object} \( Z \).

  \begin{thmenum}
    \thmitem{def:zero_morphisms/morphism}\mcite{nLab:zero_morphism} For every pair of objects \( A \) and \( B \) in \( \cat{C} \), there exists unique morphism, called the \term{zero morphism}, that \hyperref[def:factors_through]{uniquely factors through} \( Z \):
    \begin{equation}\label{eq:def:zero_morphisms/morphism}
      \begin{aligned}
        \includegraphics[page=1]{output/def__zero_morphisms.pdf}
      \end{aligned}
    \end{equation}

    We denote this zero morphism by \( 0_{A,B} \).

    \thmitem{def:zero_morphisms/kernel} The \term{kernel} cone of a morphism \( f: A \to B \) is the \hyperref[eq:def:equalizers/equalizer]{equalizer} cone of \( f \) and \( 0_{A,B} \).

    By \fullref{thm:equalizer_invertibility}, a kernel morphism is necessarily a monomorphism. A monomorphism is \term{normal} if it is a kernel.

    \thmitem{def:zero_morphisms/cokernel} \hyperref[thm:categorical_principle_of_duality]{Dually}, the \term{cokernel} cocone of a morphism \( f: A \to B \) is the \hyperref[eq:def:equalizers/coequalizer]{coequalizer} cocone of \( f \) and \( 0_{A,B} \).

    By \fullref{thm:equalizer_invertibility}, a cokernel morphism is necessarily an epimorphism. An epimorphism is \term{normal} if it is a kernel.
  \end{thmenum}
\end{definition}

\begin{definition}\label{def:group}
  A \term{group} is a \hyperref[def:monoid]{monoid} in which every element has an \hyperref[def:monoid_inverse]{inverse}. Groups are the most well-studied and most well-behaved magmas. Many useful properties like \hyperref[thm:def:group/cancellative]{cancellation} rely on associativity, so we do not consider non-associative groups.

  Groups have the following metamathematical properties:
  \begin{thmenum}
    \thmitem{def:group/theory} We can construct a \hyperref[def:first_order_theory]{first-order theory} for groups by adding a unary \hyperref[def:first_order_language/func]{functional symbol} \( (\anon)^{-1} \) to the language and the axiom
    \begin{equation}\label{eq:def:group/theory/inverse_axiom}
      \qforall \xi (\xi \cdot \xi^{-1} = e \wedge \xi^{-1} \cdot \xi = e)
    \end{equation}
    to the \hyperref[def:monoid/theory]{theory of monoids}.

    \thmitem{def:group/function_parity} A \hyperref[def:function]{function} \( \varphi: G \to H \) between two groups is called \term{even} if, for every \( x \in G \), we have
    \begin{equation}\label{eq:def:group/function_parity/even}
      \varphi(x^{-1}) = \varphi(x)
    \end{equation}
    and \term{odd} if
    \begin{equation}\label{eq:def:group/function_parity/odd}
      \varphi(x^{-1}) = \varphi(x)^{-1}.
    \end{equation}

    \thmitem{def:group/homomorphism} A \hyperref[def:first_order_homomorphism]{first-order homomorphism} between the groups \( G \) and \( H \) is an odd \hyperref[def:monoid/homomorphism]{monoid homomorphism}.

    As shown in \fullref{thm:group_homomorphism_single_condition}, however, the conditions \eqref{eq:rem:pointed_set/homomorphism} and \eqref{eq:def:group/function_parity/odd} are redundant.

    \thmitem{def:group/submodel} The set \( A \subseteq G \) is a \hyperref[thm:substructure_is_model]{submodel} of \( G \) if it is a \hyperref[def:monoid/submodel]{submonoid} and if \( x \in A \) implies \( x^{-1} \in A \). We say that \( A \) is a \term{subgroup} of \( G \).

    As a consequence of \fullref{thm:positive_formulas_preserved_under_homomorphism}, the image of a group homomorphism is a subgroup of its range.

    For an arbitrary subset \( A \) of \( G \), we denote the \hyperref[def:first_order_generated_substructure]{generated submodel} by \( \braket{ A } \). In addition to the elements of \( A \), \( \braket{ A } \) contains their products and inverses, the products of their products and inverses, etc...

    The \hyperref[def:free_group]{free group} builds a group out of a plain set; furthermore, as a consequence of \fullref{thm:group_presentation_existence}, every group is a \hyperref[def:group/quotient]{quotient} of a free group. Compare this to \hyperref[def:free_semimodule]{free semimodules} and \hyperref[def:polynomial_algebra]{polynomial semirings}.

    \thmitem{def:group/trivial} The \hyperref[def:trivial_structure]{trivial} group is the \hyperref[rem:pointed_set/trivial]{trivial pointed set} \( \set{ e } \). It is isomorphic to \hyperref[thm:substructures_form_complete_lattice/bottom]{initial substructure} of any group.

    \thmitem{def:group/exponentiation} We extend \hyperref[def:monoid/exponentiation]{monoid exponentiation} to all integers by setting
    \begin{equation*}
      x^{-n} \coloneqq (x^n)^{-1}.
    \end{equation*}

    This operation behaves well as shown in \fullref{thm:def:group/negative_power}.

    \thmitem{def:group/category} The \hyperref[def:category_of_small_first_order_models]{category of \( \mscrU \)-small models} of groups \( \ucat{Grp} \) is \hyperref[def:concrete_category]{concrete} over \hyperref[def:monoid]{\( \ucat{Mon} \)}.

    By \fullref{thm:def:group/involution}, \( \ucat{Grp} \) is also a concrete category over \( \ucat{Inv} \).

    The unique up to an isomorphism zero object in this category is the trivial group \( \set{ e } \). The \hyperref[def:zero_morphisms/morphism]{zero morphism} from \( G \) to \( H \) is
    \begin{equation*}
      \begin{aligned}
        &0_{G,H}: G \to H \\
        &0_{G,H}(x) \coloneqq e_H.
      \end{aligned}
    \end{equation*}

    We will define the \hyperref[def:free_group]{free group} functor in \fullref{subsec:free_groups}. Then from \fullref{thm:first_order_categorical_invertibility} it will follow that monomorphisms are precisely the injective homomorphisms, and that the \hyperref[def:subobject_and_quotient]{categorical subobjects} correspond to subgroups.

    Unlike in the category \hyperref[def:monoid/category]{\( \cat{Mon} \)} of monoids, in \( \cat{Grp} \) every epimorphism is surjective. We will prove this in \fullref{thm:group_epimorphisms_are_surjective}. Along with \fullref{thm:group_epimorphisms_are_normal}, this shows that the \hyperref[def:subobject_and_quotient]{categorical quotient objects} correspond to \hyperref[def:group/quotient]{quotient groups}, which we will define shortly.

    To avoid circularity, in this section, we will avoid using that monomorphisms are injective and epimorphism are surjective.

    \thmitem{def:group/kernel} The \term{kernel} of a group homomorphism \( \varphi: G \to H \) is the subgroup
    \begin{equation*}
      \ker \varphi \coloneqq \varphi^{-1}(e_H) = \set{ x \in G \given \varphi(x) = e_H }.
    \end{equation*}

    This coincides with the notion of a categorical kernel defined in \fullref{def:zero_morphisms/kernel}. Similarly to \fullref{ex:equalizers_in_set/equalizer}, \( \ker \varphi \) is the equalizer of \( \varphi \) and the zero morphism \( 0_{G,H} \).

    This equivalence holds much more generally, even for \hyperref[rem:pointed_set]{pointed sets}, however speaking of kernels is only established when we have an appropriate notion of cokernels. As we will see in \fullref{def:group/quotient}, cokernels are very well-behaved for groups, but not in general. Some related problems are highlighted in \cite[ch. 8]{Golan2010}. \Fullref{thm:def:group/kernel_cokernel_compatibility} expresses the compatibility between group kernels and cokernels.

    \thmitem{def:group/quotient} Consider the group homomorphism \( \varphi: G \to H \). We will find its \hyperref[def:zero_morphisms/cokernel]{cokernel}. This will highlight several very fundamental facts about groups, especially quotient groups. In practice, quotients are conveniently characterized by \fullref{thm:quotient_group_universal_property}.

    Similarly to \fullref{ex:equalizers_in_set/coequalizer}, the cokernel is an \hyperref[thm:equivalence_partition]{equivalence partition} of \( H \). The partitioning relation is different, however. An equivalence relation that is compatible with the operations of an algebraic structure is called a \term{congruence}. For the group \( H \), the equivalence relation \( \cong \) is a congruence if:
    \begin{itemize}
      \item It is compatible with the group operation: \( x \cong x' \) and \( y \cong y' \) imply \( x y \cong x' y' \).
      \item It is compatible with identities: \( e_H \cong x \) implies \( y \cong xy \) for all \( y \in H \). This easily follows from the first condition.
      \item It is compatible with inverses: \( x \cong x' \) implies \( x^{-1} \cong x'^{-1} \). This also follows from the first condition: \( x^{-1} \cong x^{-1} \) implies \( e \cong x^{-1} x' \), and thus \( x'^{-1} \cong x^{-1} \).
    \end{itemize}

    We need congruences since we are working with groups and group homomorphisms rather than sets and functions. We define \( \cong \) to be the smallest congruence relation containing
    \begin{equation*}
      \set{ (s(g), e_H) \given g \in G }.
    \end{equation*}

    Denote the partition \( H / {\cong} \) by \( Q \). Define a group operation on \( Q \) as \( [x] \cdot [y] = [xy] \).
    \begin{itemize}
      \item This operation is well-defined since group congruences are compatible with the group operation. We are thus free to denote it via juxtaposition.
      \item The coset \( [e_H] \) is the identity of \( Q \) since congruences are compatible with identities.
      \item The coset \( [x]^{-1} \) is the inverse of \( [x] \) since congruences are compatible with inverses.
    \end{itemize}

    Therefore, \( Q \) is a group and \( \pi(x) \coloneqq [x] \) is a group homomorphism. The pair \( (Q, \pi) \) is thus a categorical cokernel of \( \varphi \) by the same argument as in \fullref{ex:equalizers_in_set/coequalizer}.

    Denote the identity \( [e_H] \) by \( N \). It is a subgroup of \( H \):
    \begin{itemize}
      \item It contains the identity \( e_H \).
      \item It is closed under the group operation. Indeed, if \( [x] = [y] = N \), then
      \begin{equation*}
        [xy] = [x][y] = NN = N.
      \end{equation*}

      \item It is closed under the group inverse. Indeed, \( [x^{-1} x] = N \) for every \( x \in H \). If \( [x] = N \), then \( [x^{-1}] N = N \), and hence \( [x^{-1}] = N \).

      \item It possesses one additional important property. If \( [x] = N \), then not only \( x \in N \), but also \( y^{-1} x y \in N \) for every \( y \in H \). This holds because
      \begin{equation*}
        [y^{-1} x y]
        =
        [y^{-1}] [x] [y]
        =
        [y^{-1}] [y]
        =
        [y]^{-1} [y]
        =
        N.
      \end{equation*}
    \end{itemize}

    This last property distinguishes \( N \) from the \hyperref[def:multi_valued_function/image]{image} of \( \varphi \). A subgroup satisfying this property is called a \term{normal subgroup}. See \fullref{thm:normal_subgroup_equivalences} for equivalent conditions. If the image \( \img \varphi \) is a normal subgroup of \( H \), \( \varphi \) is a normal epimorphism in the sense of \fullref{def:zero_morphisms/cokernel}.

    Obviously \( \img \varphi \subseteq N \). Since \( Q \) is a colimit, \( N \) must be the smallest normal subgroup containing \( \img \varphi \).

    It is more intriguing that \( [x] = xN \) for every \( x \in H \). This can be shown as follows:
    \begin{itemize}
      \item Suppose first that \( y \in xN \), i.e. \( y = xn \) for some \( n \in N \). Then
      \begin{equation*}
        y \in [y] = [xn] = [x] N = [x].
      \end{equation*}

      Generalizing on \( y \), we obtain that \( xN \subseteq [x] \).

      \item Conversely, let \( y \in [x] \). Obviously \( x = y (y^{-1} x) \). Then
      \begin{equation*}
        [x^{-1} y] = [x^{-1}] [y] = [x]^{-1} [y] = [x]^{-1} [x] = N.
      \end{equation*}

      Hence, \( x^{-1} y \in N \) and \( y \in xN \). Generalizing on \( y \), we obtain that \( [x] \subseteq xN \).
    \end{itemize}

    Therefore, all cosets in the quotient group \( Q = H / {\cong} \) are translations of the identity \( N \). In particular, in the notation of \hyperref[def:magma/power_set]{power set operations}, it follows that
    \begin{equation*}
      xyN = xN yN.
    \end{equation*}

    Finally, given a normal subgroup \( N \) of an arbitrary group \( G \), we can define the \term{quotient group} \( G / N \) as the cokernel of the inclusion \( \iota: N \to G \). That is, \( G / N \) consists of the cosets \( xN \) for \( x \in G \) with the group operation \( xN yN = xyN \).

    \thmitem{def:group/simple} If the only proper \hyperref[thm:normal_subgroup_equivalences]{normal subgroup} of \( G \) is the \hyperref[def:group/trivial]{trivial subgroup} \( \set{ e_G } \), we say that \( G \) is a \term{simple group}.

    The trivial group itself is not simple, because it has no proper subgroups.
  \end{thmenum}
\end{definition}

\begin{example}\label{ex:power_set_is_not_a_group}
  The \hyperref[def:magma/power_set]{power set magma} \( \pow(G) \) of a group \( G \) is a monoid, but it is not a group unless \( G \) is trivial.
\end{example}

\begin{proposition}\label{thm:def:group}
  Every \hyperref[def:group]{group} \( G \) has the following basic properties:
  \begin{thmenum}
    \thmitem{thm:def:group/cancellative} The (binary) group operation is \hyperref[def:magma/cancellative]{cancellative}.
    \thmitem{thm:def:group/identity_inverse} The identity \( e \) is its own inverse.
    \thmitem{thm:def:group/inverse_composition} \( (xy)^{-1} = y^{-1} x^{-1} \).
    \thmitem{thm:def:group/involution} \( x = (x^{-1})^{-1} \)
    \thmitem{thm:def:group/negative_power} For any positive integer \( n \), \( (x^n)^{-1} = (x^{-1})^n \)
    \thmitem{thm:def:group/inverse_isomorphism} The map \( x \mapsto x^{-1} \) is a group isomorphism.

    \thmitem{thm:def:group/zero_kernel} The \hyperref[def:group/kernel]{kernel} of a group homomorphism \( \varphi: G \to H \) is trivial if and only if \( \varphi \) is an \hyperref[def:first_order_homomorphism_invertibility/embedding]{embedding} (injective homomorphism).

    \thmitem{thm:def:group/kernel_cokernel_compatibility} For a \hyperref[def:group/quotient]{quotient group} \( G / N \) with canonical projection \( \pi(x) \coloneqq xN \), the \hyperref[def:group/kernel]{kernel} of \( \pi \) is \( N \).

    \thmitem{thm:def:group/kernel_is_normal_subgroup} The kernel of a group homomorphism is a \hyperref[thm:normal_subgroup_equivalences]{normal subgroup}.
  \end{thmenum}
\end{proposition}
\begin{proof}
  \SubProofOf{thm:def:group/cancellative} If \( x = y \), obviously \( xz = yz \) and \( zx = zy \). Now if \( xz = yz \), we have
  \begin{equation*}
    x = x(zz^{-1}) = (xz)z^{-1} = (yz)z^{-1} = y(zz^{-1}) = y.
  \end{equation*}

  The case \( zx = zy \) is analogous.

  \SubProofOf{thm:def:group/identity_inverse} \( ee = e \).
  \SubProofOf{thm:def:group/inverse_composition}
  \begin{equation*}
    (xy) (y^{-1} x^{-1})
    =
    x (y y^{-1}) x^{-1}
    =
    e
    =
    y^{-1} (x^{-1} x) y
    =
    (y^{-1} x^{-1}) (xy).
  \end{equation*}

  \SubProofOf{thm:def:group/involution}
  \begin{equation*}
    (x^{-1})^{-1}
    =
    x x^{-1} (x^{-1})^{-1}
    =
    x.
  \end{equation*}

  \SubProofOf{thm:def:group/negative_power} Using \fullref{thm:def:group/involution},
  \begin{equation*}
    x^{-n}
    =
    (x^n)^{-1}
    =
    x^{-1} \cdots x^{-1}
    =
    (x^{-1})^n.
  \end{equation*}

  \SubProofOf{thm:def:group/inverse_isomorphism} Trivial.

  \SubProofOf{thm:def:group/zero_kernel}
  \SufficiencySubProof* Suppose that \( \ker \varphi = \set{ e_H } \) and \( \varphi(x) = \varphi(y) \). Then
  \begin{equation*}
    e_H = \varphi(x) \varphi(y)^{-1} = \varphi(x y^{-1}).
  \end{equation*}

  Thus, \( x y^{-1} \in \ker \varphi \), and hence \( x = y \).

  Therefore, \( \varphi \) is injective.

  \NecessitySubProof* Suppose that \( \varphi \) is injective. Since \( \varphi(e_G) = e_H \), \( \varphi(x) = e_H \) implies that \( x = e_G \).

  \SubProofOf{thm:def:group/kernel_cokernel_compatibility} Trivial.

  \SubProofOf{thm:def:group/kernel_is_normal_subgroup} For a homomorphism \( \varphi: G \to H \), if \( x \in \ker \varphi \), then
  \begin{equation*}
    \varphi(y^{-1} x y) = \varphi(y)^{-1} \varphi(x) \varphi(y) = \varphi(y)^{-1} \varphi(y) = e_H,
  \end{equation*}
  and thus \( y^{-1} x y \in \ker \varphi \).
\end{proof}

\begin{proposition}\label{thm:group_homomorphism_single_condition}
  A function between groups is a \hyperref[def:group/homomorphism]{group homomorphism} if and only if it satisfies \eqref{eq:def:magma/homomorphism}.
\end{proposition}
\begin{proof}
  \SufficiencySubProof \eqref{eq:def:magma/homomorphism} is required to hold by definition.

  \NecessitySubProof Let the function \( \varphi: G \to H \) satisfy \eqref{eq:def:magma/homomorphism}. Then it preserves identities, i.e. is a \hyperref[rem:pointed_set/homomorphism]{pointed set homomorphism}. Indeed, we have
  \begin{equation*}
    e_H \varphi(e_G) = \varphi(e_G) = \varphi(e_G e_G) = \varphi(e_G) \varphi(e_G).
  \end{equation*}

  By \fullref{thm:def:group/cancellative}, \( \varphi \) is cancellative, and hence \( e_H = \varphi(e_G) \).

  Inverses are preserved (i.e. \eqref{eq:def:group/function_parity/odd} holds) because
  \begin{equation*}
    \varphi(x^{-1})
    =
    \varphi(x^{-1}) e_H
    =
    \varphi(x^{-1}) \varphi(x) \varphi(x)^{-1}
    =
    \varphi(x^{-1} x) \varphi(x)^{-1}
    =
    e_H \varphi(x)^{-1}
    =
    \varphi(x)^{-1}.
  \end{equation*}

  Therefore, \( \varphi \) is indeed a group homomorphism.
\end{proof}

\begin{lemma}\label{thm:group_operation_induces_bijections}
  For each element \( x \) of a group \( G \), consider the function \( \varphi_x \coloneqq x \id_G \), i.e.
  \begin{equation*}
    \begin{aligned}
      &\varphi_x: G \to G \\
      &\varphi_x(y) \coloneqq x \cdot y.
    \end{aligned}
  \end{equation*}

  This is a bijective function (but not necessarily a group isomorphism).
\end{lemma}
\begin{proof}
  \SubProofOf[def:function_invertibility/injective]{injectivity} If \( y, y' \in G \) and \( \varphi_x(y) = \varphi_x(y') \), we have
  \begin{equation*}
    xy = \varphi_x(y) = \varphi_x(y') = xy'.
  \end{equation*}

  By \fullref{thm:def:group/cancellative}, \( y = y' \). Therefore, \( \varphi_x \) is injective.

  \SubProofOf[def:function_invertibility/surjective]{surjectivity} If \( z \in G \), then \( z = x(x^{-1} z) \). Therefore, \( z = \varphi_x(x^{-1} z) \), and thus every member of \( G \) has a preimage. Thus, \( \varphi_x \) is surjective.
\end{proof}

\begin{proposition}\label{thm:invertible_submonoid_is_group}
  The set of all \hyperref[def:monoid_inverse]{invertible} elements of a \hyperref[def:monoid]{monoid} is a \hyperref[def:group]{group}.
\end{proposition}
\begin{proof}
  Fix a monoid \( M \).

  \begin{itemize}
    \item \( e_M \) is invertible.
    \item If \( x \) and \( y \) are invertible, then \( xy \) is invertible with inverse \( y^{-1} x^{-1} \).
    \item If \( x \) is invertible with inverse \( x^{-1} \), then \( x^{-1} \) is invertible with inverse \( x \).
  \end{itemize}

  Therefore, the set of invertible elements is a submonoid and is closed under inverses.
\end{proof}

\begin{definition}\label{def:subgroup_cosets}
  Let \( H \subseteq G \) be a subgroup of \( G \). Even if \( H \) is not normal, we can define the \term{left} and \term{right cosets}
  \begin{equation*}
    x H \coloneqq \set{ xh \given h \in H }
    \quad\quad
    H x \coloneqq \set{ hx \given h \in H }.
  \end{equation*}

  The \term{index} \( [G : H] \) of \( H \) in \( G \) is \hyperref[def:cardinal]{cardinality} of the family of all left cosets.

  The discussion in \fullref{def:group/quotient} can be generalized to show that  \( \set{ xH \given x \in G } \) is a \hyperref[def:set_partition]{partition} of \( G \) into \hyperref[def:equinumerosity]{equinumerous} sets. If \( H \) is not normal, this partition is not induced by a congruence, and we cannot form a quotient group using a non-normal subgroup. Nonetheless, left and right cosets still turn out useful.
\end{definition}

\begin{proposition}\label{thm:group_coset_bijection}
  The family of all \hyperref[def:subgroup_cosets]{left cosets} of a subgroup is \hyperref[def:equinumerosity]{equinumerous} to the family of all right cosets.
\end{proposition}
\begin{proof}
  Fix a subgroup \( H \) of \( G \), and consider the function \( xH \mapsto Hx \) taking left cosets to right cosets.

  It is well-defined because, if \( x H = x' H \), then there exists \( h \), such that \( x = x' h \), and thus
  \begin{equation*}
    H x' = H h^{-1} x = H x.
  \end{equation*}

  It is injective by the same converse argument, and it is surjective by definition. Therefore, it is bijective.
\end{proof}

\begin{proposition}\label{thm:normal_subgroup_equivalences}
  For a subgroup \( N \) of \( G \), the following conditions are equivalent:
  \begin{thmenum}
    \thmitem{thm:normal_subgroup_equivalences/congruence} For every element \( x \) of \( G \), we have the set equality
    \begin{equation}\label{eq:thm:normal_subgroup_equivalences/congruence}
      x^{-1} N x = N.
    \end{equation}

    This is the definition of a normal subgroup obtained in \fullref{def:group/quotient}.

    \thmitem{thm:normal_subgroup_equivalences/cosets} The partitions induced by the \hyperref[def:subgroup_cosets]{left and rights cosets} of \( N \) coincide.

    \thmitem{thm:normal_subgroup_equivalences/kernel} \( N \) is the \hyperref[def:group/kernel]{kernel} of some group homomorphism.
  \end{thmenum}

  In particular, kernels are always normal subgroups.
\end{proposition}
\begin{proof}
  This is the group-theoretic analog to \fullref{thm:equivalence_partition}.

  \ImplicationSubProof{thm:normal_subgroup_equivalences/congruence}{thm:normal_subgroup_equivalences/cosets} For any \( x \in G \)
  \begin{equation*}
    N x = (x N x^{-1})x = x N(x^{-1}x) = x N,
  \end{equation*}
  thus every left coset is a right coset and vice versa.

  \ImplicationSubProof{thm:normal_subgroup_equivalences/cosets}{thm:normal_subgroup_equivalences/kernel} We can take the \hyperref[def:group/quotient]{canonical projection} \( \pi(x) \coloneqq x N \) as the homomorphism. By \fullref{thm:def:group/kernel_cokernel_compatibility}, \( \ker \pi = N \).

  \ImplicationSubProof{thm:normal_subgroup_equivalences/kernel}{thm:normal_subgroup_equivalences/congruence} Let \( \varphi: G \to H \) be a group homomorphism and fix any \( x \in G \). Denote \( N \coloneqq \ker(f) \). By \fullref{thm:def:group/kernel_is_normal_subgroup}, it is a normal subgroup in the sense of \fullref{def:group/quotient}, i.e. it satisfies \eqref{eq:thm:normal_subgroup_equivalences/congruence}.
\end{proof}

\begin{theorem}[Quotient group universal property]\label{thm:quotient_group_universal_property}\mcite[thm. II.7.12]{Aluffi2009}
  For every \hyperref[def:group]{group} \( G \) and \hyperref[thm:normal_subgroup_equivalences]{normal subgroup} \( N \), the \hyperref[def:group/quotient]{quotient group} \( G / N \) has the following \hyperref[rem:universal_mapping_property]{universal mapping property}:
  \begin{displayquote}
    Every group homomorphism \( \varphi: G \to H \) satisfying \( N \subseteq \ker \varphi \) \hyperref[def:factors_through]{uniquely factors through} \( G / N \). That is, there exists a unique homomorphism \( \widetilde{\varphi}: G / N \to H \), such that the following diagram commutes:
    \begin{equation}\label{eq:thm:quotient_group_universal_property/diagram}
      \begin{aligned}
        \includegraphics[page=1]{output/thm__quotient_group_universal_property.pdf}
      \end{aligned}
    \end{equation}

    In the case where \( N = \ker \varphi \), \( \widetilde{\varphi} \) is an \hyperref[def:first_order_homomorphism_invertibility/embedding]{embedding}.
  \end{displayquote}

  This extends to \fullref{thm:quotient_module_universal_property} and \fullref{thm:quotient_algebra_universal_property}.
\end{theorem}
\begin{proof}
  We want
  \begin{equation*}
    \widetilde{\varphi}(\pi(x)) = \widetilde{\varphi}(xN) = \varphi(x).
  \end{equation*}

  This suggests the definition
  \begin{equation*}
    \widetilde{\varphi}(xN) \coloneqq \varphi(x).
  \end{equation*}

  The homomorphism \( \widetilde{\varphi} \) is well-defined because, if \( x N = x' N \), since \( N \subseteq \ker \varphi \), we have
  \begin{equation*}
    \varphi(x)
    =
    \varphi(x) e_N
    =
    \varphi(x) \varphi(N)
    =
    \varphi(x N)
    =
    \varphi(x' N)
    =
    \cdots
    =
    \varphi(y).
  \end{equation*}

  If \( N = \ker \varphi \), the kernel of \( \widetilde{\varphi} \) is trivial. By \fullref{thm:def:group/zero_kernel}, it is an injective function.
\end{proof}

\begin{corollary}\label{thm:quotient_group_by_kernel}
  Every \hyperref[def:group/homomorphism]{group homomorphism} \( \varphi: G \to H \) induces an isomorphism
  \begin{equation*}
    G / \ker \varphi \cong \img \varphi.
  \end{equation*}
\end{corollary}
\begin{proof}
  Directly follows from \fullref{thm:quotient_group_universal_property} by restricting the range of \( \widetilde{\varphi} \) to its image.
\end{proof}

\begin{corollary}\label{thm:group_epimorphisms_are_normal}
  Every surjective group homomorphism is a \hyperref[def:zero_morphisms/cokernel]{normal epimorphism}.
\end{corollary}
\begin{proof}
  Fix a group homomorphism \( \varphi: G \to H \). By \fullref{thm:quotient_group_by_kernel}, \( G / \ker \varphi \cong \img \varphi = H \). Thus, \( H \) is a \hyperref[def:zero_morphisms/cokernel]{cokernel} of the canonical inclusion \( \iota: \ker \varphi \to G \).
\end{proof}

\begin{theorem}[Quotient subgroup lattice theorem]\label{thm:quotient_subgroup_lattice_theorem}\mcite[prop. II.8.9]{Aluffi2009}
  Given a \hyperref[thm:normal_subgroup_equivalences]{normal subgroup} \( N \) of \( G \), the function \( H \mapsto H / N \) is a \hyperref[def:semilattice/homomorphism]{lattice homomorphism} between the \hyperref[thm:substructures_form_complete_lattice]{lattice of subgroups} of \( G \) containing \( N \) and the lattice of subgroups of the \hyperref[def:group/quotient]{quotient} \( G / N \).

  \begin{figure}[!ht]
    \centering
    \includegraphics[page=1]{output/thm__lattice_theorem_for_groups.pdf}
    \caption{The lattice of subgroups of \( G \) and the lattice of subgroups of \( G / N \).}
    \label{fig:thm:quotient_subgroup_lattice_theorem}
  \end{figure}

  This extends to \fullref{thm:quotient_submodule_lattice_theorem} and \fullref{thm:quotient_ideal_lattice_theorem}.
\end{theorem}
\begin{proof}
  \SubProofOf[def:function_invertibility/injective/equality]{injectivity} Let \( H_1 / N = H_2 / N \). Both \( H_1 / N \) and \( H_2 / N \) consist of the same cosets, hence
  \begin{equation*}
    H_1 = \bigcup (H_1 / N) = \bigcup (H_2 / N) = H_2.
  \end{equation*}

  Therefore, the map \( H \mapsto H / N \) is injective.

  \SubProofOf[def:function_invertibility/surjective/existence]{surjectivity} Fix a subgroup \( M \) of \( G / N \) and define
  \begin{equation*}
    H \coloneqq \set{ x \in G \given xN \in M }.
  \end{equation*}

  Then clearly \( H / N = M \). Therefore, the map \( H \mapsto H / N \) is surjective.

  \SubProofOf[def:semilattice/homomorphism]{lattice compatibility} The join \( \braket{ K \cup H } \) of two subgroups of \( G \) containing \( N \) must satisfy the equality
  \begin{equation}\label{eq:thm:quotient_subgroup_lattice_theorem/join}
    \underbrace{ \braket{ K \cup H } / N }_{\set{ xN \given x \in \braket{ K \cup H } }}
    =
    \underbrace{ \braket{ (K / N) \cup (H / N) } }_{\braket{ \set{ xN \given x \in K \cup H } }}.
  \end{equation}

  Verifying this amounts to noting that \( \braket{ K \cup H } \) is obtained by adding the products and inverses of any elements of \( G \) not in \( K \cup H \). Since the projection map \( \pi: G \to G / N \) is a homomorphism, the coset \( xy N \) of the product of \( x, y \in \braket{ K \cup H } \) is the product \( (xN) (yN) \) of the cosets \( xN \) and \( yN \), and analogously for inverses. Hence, adding an element \( x \in G \) to \( K \cup H \) and then taking all cosets is the same as adding the coset \( xH \) to \( (K / N) \cup (H / N) \).

  Therefore, \eqref{eq:thm:quotient_subgroup_lattice_theorem/join} holds, and thus \( H \mapsto H / N \) preserves joins in the lattice of subgroups.

  The other verifications are simpler. For meets, we have
  \begin{equation*}
    (K \cap H) / N
    =
    \set{ xN \given x \in K \cap H }
    =
    \set{ xN \given x \in K } \cap \set{ xN \cap x \in H }
    =
    (K / N) \cap (H / N).
  \end{equation*}

  Finally, it remains to show that \( H \mapsto H / N \) preserves the \hyperref[def:partially_ordered_set_extremal_points/top_and_bottom]{top and bottom elements}. This is trivial since \( G / N \) contains all possible cosets of \( N \) and is hence the top in the lattice of subgroups of \( G / N \), and \( N / N \) is the trivial group and hence the bottom.

  Therefore, \( H \mapsto H / N \) is a lattice isomorphism.
\end{proof}

\begin{theorem}[Lagrange's theorem for groups]\label{thm:lagranges_theorem_for_groups}
  Let \( H \) be a subgroup of \( G \). We have the following equality
  \begin{equation}\label{eq:thm:lagranges_theorem_for_groups/index}
    \card(G) = \card(H) \cdot [G : H].
  \end{equation}

  If \( H \) is a \hyperref[thm:normal_subgroup_equivalences]{normal subgroup}, then \( [G : H] = \card(G / H) \) and
  \begin{equation}\label{eq:thm:lagranges_theorem_for_groups/card}
    \card(G) = \card(H) \cdot \card(G / H).
  \end{equation}

  This demonstrates that there exists a bijective function between the \hyperref[def:monoid_direct_product]{direct product} \( H \times G / H \) and \( H \), however this may not be a group homomorphism --- see \fullref{ex:lagranges_theorem_for_groups/direct_product_zn}.

  Note that we will define the order of a group in \fullref{def:group_order} as its cardinality and thus the theorem is actually a statement about the orders of \( G \) and \( H \). But we have not yet formally defined group orders, and working with cardinalities highlights that the statement also holds for groups of arbitrary infinite cardinality.
\end{theorem}
\begin{proof}
  By \fullref{def:subgroup_cosets}, every coset of \( G \) with respect to \( H \) is equinumerous with \( H \), and there is a total of \( [G : H] \) cosets. Therefore, \eqref{eq:thm:lagranges_theorem_for_groups/index} holds.
\end{proof}

\begin{example}\label{ex:subgroups_of_integers}
  Consider the group \( \BbbZ \) of integers with respect to addition.

  Let \( 2\BbbZ \) be the subgroup of all even integers. Then both \( \BbbZ \) and \( 2\BbbZ \) are countably infinite, but their quotient group \( \BbbZ / 2\BbbZ \) has two elements --- the set \( 2\BbbZ \) of all even integers and the set \( 2\BbbZ + 1 \) of all odd integers. Generalizations of this quotient group are discussed in \fullref{thm:group_of_integers_modulo}. \Fullref{thm:lagranges_theorem_for_groups} holds, but it gives no insight due to the absorbing properties of transfinite cardinal arithmetic described in \fullref{thm:simplified_cardinal_arithmetic}.

  Now consider the groups \( 4\BbbZ \subseteq 2\BbbZ \subseteq \BbbZ \). As a consequence of \fullref{thm:lagranges_theorem_for_groups}, \( 3\BbbZ \) is not a subgroup of \( 2\BbbZ \), and so we consider powers of \( 2 \).

  Since \( 2\BbbZ \) is a subgroup of \( \BbbZ \), the quotient \( 2\BbbZ / 4\BbbZ \) must a subgroup of \( \BbbZ / 4\BbbZ \) as a consequence of \fullref{thm:quotient_subgroup_lattice_theorem}. We may not know the structure of the quotient groups (although we do, see \fullref{thm:group_of_integers_modulo}), but we know how \( 4\BbbZ \), \( 2\BbbZ \) and \( \BbbZ \) relate to each other, and we are able to determine how the quotient groups relate to each other.
\end{example}

  \section{Free groups}\label{sec:free_groups}

\paragraph{Free monoids}

\begin{definition}\label{def:free_monoid}\mcite[48]{MacLane1998Categories}
  We associate with every \hyperref[def:set]{plain set} \( A \) its \term{free monoid} defined as the \hyperref[def:formal_language/kleene_star]{Kleene star} \( A^* \) with \hyperref[def:formal_language/concatenation]{concatenation} as the monoid operation.

  Denote by \( \iota_A: A \to A^* \) the canonical inclusion function, which sends every member of \( A \) into the corresponding single-symbol string in the \hyperref[def:free_monoid]{free monoid} \( A^* \).
\end{definition}
\begin{defproof}
  Concatenation is clearly associative and the empty string \( \bnfes \) is an \hyperref[def:monoid]{neutral element} under concatenation.
\end{defproof}

\begin{theorem}[Free monoid universal property]\label{thm:free_monoid_universal_property}
  The \hyperref[def:free_monoid]{free monoid} \( A^* \) is the unique up to an isomorphism monoid that satisfies the following \hyperref[rem:universal_mapping_property]{universal mapping property}:
  \begin{displayquote}
    For every monoid \( M \) and every function \( f: A \to M \), there exists a unique \hyperref[def:monoid/homomorphism]{monoid homomorphism} \( \widetilde{f}: A^* \to M \) such that the following diagram commutes:
    \begin{equation}\label{eq:thm:free_monoid_universal_property/diagram}
      \begin{aligned}
        \includegraphics[page=1]{output/thm__free_monoid_universal_property}
      \end{aligned}
    \end{equation}
  \end{displayquote}
\end{theorem}
\begin{comments}
  \item Via \fullref{rem:universal_mapping_property}, \( (\Anon*)^* \) becomes \hyperref[def:category_adjunction]{left adjoint} to the \hyperref[def:concrete_category]{forgetful functor}
  \begin{equation*}
    U: \cat{Mon} \to \cat{Set}.
  \end{equation*}
\end{comments}
\begin{proof}
  In order for \( \widetilde{f} \) to be a monoid homomorphism, it must satisfy \( \widetilde{f}(\bnfes) = e_M \) and
  \begin{equation*}
    \widetilde{f}(a_1 a_2 \ldots a_n) = \widetilde{f}(a_1) \cdot \widetilde{f}(a_2) \cdot \ldots \cdot \widetilde{f}(a_n).
  \end{equation*}

  Via induction on \( n \) we can see that every two such functions are equal because their action is determined entirely by the empty string and the action of \( f \) on the symbols from \( A \).

  This suggests the recursive definition
  \begin{equation*}
    \widetilde{f}(w) \coloneqq \begin{cases}
      e_M,                         &w = \bnfes, \\
      f(a) \cdot \widetilde{f}(w') &w = a \cdot w'.
    \end{cases}
  \end{equation*}
\end{proof}

\paragraph{Free groups}

\begin{definition}\label{def:free_group}\mimprovised
  Let \( A \) be a \hyperref[def:set]{plain set}. We will now construct the \term{free group} \( F(A) \) of \( A \).

  Let \( + \) and \( - \) be a arbitrary plain sets not in \( A \). Consider the \hyperref[def:disjoint_union]{disjoint union} \( U \coloneqq A \times \set{ +, - } \), whose members we will denote by \( a^+ \) and \( a^- \).

  Consider the \hyperref[def:formal_language/kleene_star]{Kleene star} \( U^* \). We say that a string in \( U^* \) is \term{reduced} if there exists no symbol \( a \in A \) such that either \( a^+ a^- \) or \( a^- a^+ \) is a substring of \( w \).

  Define via \hyperref[con:evaluation]{pattern matching} the reduction function
  \begin{equation*}
    \red(w) \coloneqq \begin{cases}
      p \cdot \red(s), &w = p a^+ a^- s \T{or} w = p a^- a^+ s, \T{where} p \T{is reduced} \\
      w,               &\T{otherwise.}
    \end{cases}
  \end{equation*}

  It is well-defined because \( s \) is always shorter than \( w \) and thus \( \red(w) \) recursively applies itself finitely many times. Furthermore, the condition for \( p \) to be reduced ensures that the leftmost pair always gets eliminated, making the process deterministic.

  Clearly \( w \) is reduced if and only if \( w = \red(w) \).

  We define the \term{free group} \( F(A) \) as the set of reduced strings over \( U \), with the operation \( v \star w \coloneqq \red(vw) \).

  The identity is the empty string and the inverse \( w^{-1} \) of \( w \) can be characterized recursively as
  \begin{equation*}
    w^{-1} = \begin{cases}
      \bnfes &w = \bnfes \\
      w^{-1} a^-  &w = a^+ v \\
      w^{-1} a^+  &w = a^- v
    \end{cases}
  \end{equation*}

  The canonical inclusion is
  \begin{equation*}
    \begin{aligned}
      &\iota_A: A \to F(A) \\
      &\iota_A(a) \coloneqq a^+.
    \end{aligned}
  \end{equation*}
\end{definition}
\begin{comments}
  \item Compare this definition to the much less syntactic definition of free abelian groups as \hyperref[def:free_semimodule]{free modules} over \( \BbbZ \).
\end{comments}

\begin{theorem}[Free group universal property]\label{thm:free_group_universal_property}
  The \hyperref[def:free_group]{free group} \( F(A) \) is the unique up to an isomorphism group that satisfies the following \hyperref[rem:universal_mapping_property]{universal mapping property}:
  \begin{displayquote}
    For every group \( G \) and every function \( f: A \to G \), there exists a unique \hyperref[def:group/homomorphism]{group homomorphism} \( \widetilde{f}: F(A) \to G \) such that the following diagram commutes:
    \begin{equation}\label{eq:thm:free_group_universal_property/diagram}
      \begin{aligned}
        \includegraphics[page=1]{output/thm__free_group_universal_property}
      \end{aligned}
    \end{equation}
  \end{displayquote}
\end{theorem}
\begin{comments}
  \item Via \fullref{rem:universal_mapping_property}, \( F \) becomes \hyperref[def:category_adjunction]{left adjoint} to the \hyperref[def:concrete_category]{forgetful functor}
  \begin{equation*}
    U: \cat{Grp} \to \cat{Set}.
  \end{equation*}
\end{comments}
\begin{proof}
  A group homomorphism is a monoid homomorphism, hence we can utilize our reasoning for free monoids to extend the definition from \fullref{thm:free_monoid_universal_property} to
  \begin{equation*}
    \widetilde{f}(w) \coloneqq \begin{cases}
      \bnfes,                      &w = \bnfes, \\
      f(a) \cdot \widetilde{f}(w')      &w = a^+ \cdot w', \\
      f(a)^{-1} \cdot \widetilde{f}(w') &w = a^- \cdot w'.
    \end{cases}
  \end{equation*}
\end{proof}

\begin{corollary}\label{thm:injective_group_homomorphisms_are_monomorphisms}
  Injective group homomorphisms are \hyperref[def:morphism_invertibility/left_cancellative]{categorical monomorphisms}.
\end{corollary}
\begin{proof}
  Follows from \fullref{thm:free_group_universal_property} and \fullref{thm:first_order_categorical_invertibility/injective}.
\end{proof}

\paragraph{Free products}

\begin{remark}\label{rem:group_presentation_syntax}
  \hyperref[def:object_presentation/cardinality]{Finitely-presented} groups have a special syntax: a group with generators \( a_1, \ldots, a_n \) and relators \( a_{i_1} \sim a_{j_1}, \ldots, a_{i_m} \sim a_{j_m} \) can be written in a syntax resembling that of \hyperref[def:group/generated]{generated subgroups}:
  \begin{equation*}
    \braket{ a_1, \ldots, a_n \given a_{i_1} = a_{j_1}, \ldots, a_{i_m} = a_{j_m} }.
  \end{equation*}

  We simply list the elements we want to see and which of them we want to be equal.
\end{remark}

\begin{example}\label{ex:free_group_with_uncountably_many_subgroups}\mcite{MathSE:countable_group_uncountably_many_distinct_subgroup}
  Consider some sequence \( x_1, x_2, \ldots \), as well as the \hyperref[def:free_group]{free group} with \hyperref[rem:group_presentation_syntax]{presentation}
  \begin{equation*}
    F_\infty = \braket{ x_1, x_2, x_3, \ldots }.
  \end{equation*}

  For any set \( N \) of indices, we also have the subgroup
  \begin{equation*}
    F_N \coloneqq \braket{ x_i \given i \in N }.
  \end{equation*}

  \Fullref{thm:cantor_power_set_theorem} implies that there is an uncountable amount of such subgroups. Some of them are isomorphic (if the sets of indices are \hyperref[def:equinumerosity]{equinumerous}), but they are nonetheless distinct.

  Therefore, \( F_\infty \) is a countable group with uncountably many subgroups.
\end{example}

\begin{definition}\label{def:group_free_product}\mcite[323]{Knapp2016BasicAlgebra}
  We define the \term{free product} of a nonempty pairwise disjoint family of groups \( \seq{ \braket{S_k \mid R_k} }_{k \in \mscrK} \) as the group with presentation
  \begin{equation*}
    \Ast_{k \in \mscrK} X_k \coloneqq \braket*{ \bigcup_{k \in \mscrK} S_k \given* \bigcup_{k \in \mscrK} R_k }.
  \end{equation*}

  If the constituent groups are not disjoint, we may instead use \hyperref[def:disjoint_union]{disjoint unions} as
  \small
  \begin{equation*}
    \Ast_{k \in \mscrK} X_k \coloneqq \braket*{ \coprod_{k \in \mscrK} S_k \given* \set[\Big]{ (k, x_1) (k, x_2) \ldots (k, x_n) \given x_1 x_2 \ldots x_n \in R_k } }.
  \end{equation*}
  \normalsize

  For every index \( m \in \mscrK \), we define the canonical embedding
  \begin{equation*}
    \begin{aligned}
       &\iota_m: X_m \to \Ast_{k \in \mscrK} X_k \\
       &\iota_m(x) \coloneqq (m, x).
    \end{aligned}
  \end{equation*}
\end{definition}

\begin{proposition}\label{thm:group_coproduct}
  The \hyperref[def:discrete_category_limits]{categorical coproduct} of the family \( \seq{ G_k }_{k \in \mscrK} \) in the category \hyperref[def:group/category]{\( \cat{Grp} \)} of groups is their \hyperref[def:group_free_product]{free product} \( \Ast_{k \in \mscrK} G_k \).
\end{proposition}
\begin{proof}
  Let \( (A, \alpha) \) be a \hyperref[def:category_of_cones/cocone]{cocone} for the discrete diagram \( \seq{ G_k }_{k \in \mscrK} \). We want to define a group homomorphism \( l: \Ast_{k \in \mscrK} G_k \to A \) such that, for every \( m \in \mscrK \),
  \begin{equation*}
    \alpha_m(x) = l_A(\iota_m(x)).
  \end{equation*}

  This suggests the definition
  \begin{equation*}
    l_A\parens[\Big]{ \iota_{k_1}(x_1) \iota_{k_2}(x_2) \ldots \iota_{k_n}(x_n) } \coloneqq \alpha_{k_1}(x_1) \cdot \alpha_{k_2}(x_k) \cdot \ldots \cdot \alpha_{k_n}(x_n).
  \end{equation*}
\end{proof}

\paragraph{Cyclic groups}

\begin{definition}\label{def:cyclic_group}\mcite[43]{Jacobson1985AlgebraPart1}
  We say that a \hyperref[def:group]{group} is \term[ru=циклическая группа (\cite[97]{Тыртышников2017Алгебра})]{cyclic} if it can be \hyperref[def:object_presentation]{generated} by a single element.

  \Fullref{thm:cyclic_group_classification} implies that, up to an isomorphism, there is only one cyclic group for every possible order. When referring to an abstract cyclic group, we will use the notation \( C_n \) for groups of finite cardinality \( n \) or \( C_\infty \) for (countably) infinite groups.
\end{definition}
\begin{comments}
  \item Given an ambient group \( G \) and some element \( g \in G \), the \term{cyclic subgroup} of \( g \) is the cyclic group isomorphic to the \hyperref[def:group/submodel]{generated subgroup} of \( G \).

  \item As shown in \fullref{thm:cyclic_group_isomorphic_to_integers_modulo_n}, cyclic groups are isomorphic to certain groups of integers, however it is still useful to have cyclic groups as a separate concept.
\end{comments}

\begin{proposition}\label{thm:cyclic_group_classification}
  Fix a \hyperref[def:cyclic_group]{cyclic group} \( G \) and a symbol \( a \) not in \( G \).

  \begin{thmenum}
    \thmitem{thm:cyclic_group_classification/finite} If \( G \) has finite cardinality \( n \), it is isomorphic to the \hyperref[def:free_group]{free group}
    \begin{equation}\label{eq:thm:cyclic_group_classification/finite}
      \braket{ a \given a^n = 1 }.
    \end{equation}

    \thmitem{thm:cyclic_group_classification/infinite} If \( G \) has infinite cardinality \( \infty \), it is isomorphic to
    \begin{equation}\label{eq:thm:cyclic_group_classification/infinite}
      \braket{ a }.
    \end{equation}
  \end{thmenum}
\end{proposition}
\begin{proof}
  Let \( G \) be a cyclic group with generator \( b \).

  \SubProofOf{thm:cyclic_group_classification/finite} If \( G \) has \( n \) elements, then \( b^n = 1 \), and thus \( a \mapsto b \) is a group isomorphism of \( \braket{ a \given a^n = 1 } \) and \( G \).

  \SubProofOf{thm:cyclic_group_classification/infinite} Suppose that \( G \) is infinite. There are only countably many powers of \( b \), one for each integer, so the cardinality of \( G \) is \( \aleph_0 \). The map \( a \mapsto b \) is an isomorphism of \( \braket{ a } \) and \( G \).
\end{proof}

\begin{proposition}\label{thm:cyclic_subgroup_classification}
  Fix a finite \hyperref[def:cyclic_group]{cyclic group} \( G \) with generator \( a \) and a subgroup \( H \) of \( G \). Let \( s \) be the smallest positive integer such that \( a^s \) is in \( H \).

  \begin{thmenum}
    \thmitem{thm:cyclic_subgroup_classification/cyclic} \( H \) is a cyclic group with generator \( a^s \).

    \thmitem{thm:cyclic_subgroup_classification/finite} If \( G \) has finite cardinality \( n \), then the cardinality \( m \) of \( H \) divides \( n \) and \( s = n / m \).

    \thmitem{thm:cyclic_subgroup_classification/infinite} If \( G \) has infinite cardinality, then so does \( H \).
  \end{thmenum}
\end{proposition}
\begin{comments}
  \item \Fullref{thm:cyclic_subgroup_characterization} provides a converse, allowing to conclude that a group is cyclic based on how many subgroups it has.
\end{comments}
\begin{proof}
  \SubProofOf{thm:cyclic_subgroup_classification/cyclic} Let \( s \) be the smallest positive integer such that \( a^s \) is in \( H \). For every member \( a^k \) of \( H \), \fullref{alg:integer_division} gives us nonnegative \( q \) and \( r \), where \( r < s \), such that
  \begin{equation*}
    k = qs + r.
  \end{equation*}

  Then
  \begin{equation*}
    a^k = (a^s)^q \cdot a^r.
  \end{equation*}

  Since \( a^s \) belongs to \( H \), so does is integer power \( a^{sq} \). Then its inverse also belongs to \( H \), as well as
  \begin{equation*}
    a^r = a^k a^{-sq}.
  \end{equation*}

  But we have assumed that \( r < s \), which contradicts the minimality of \( s \) unless \( r = 0 \).

  Therefore, every member of \( H \) is a power of \( a^s \), that is, \( H \) is cyclic with generator \( a^s \).

  \SubProofOf{thm:cyclic_subgroup_classification/finite} Suppose that \( G \) has \( n \) elements and that \( H \) has \( m \) elements.

  \Fullref{thm:lagranges_subgroup_theorem} implies that \( m \) divides \( G \).

  Since \( a^{sm} = (a^s)^m = e \), it follows that \( sm \) is a multiple of \( n \). But \( H \) is cyclic of cardinality \( m \), hence \( a^{sk} \neq e \) whenever \( 0 < k < m \). So \( sm \) is the smallest multiple of \( n \), hence \( n \) itself. Therefore, \( s = n / m \).

  \SubProofOf{thm:cyclic_subgroup_classification/infinite} Suppose that \( G \) is countable.

  Suppose also that \( H \) is finite of cardinality \( m \). Then \( a^m = e \). For every nonnegative integer \( k \), \fullref{alg:integer_division} gives us nonnegative \( q \) and \( r \) such that \( k = mq + r \) and \( 0 \leq r < m \).

  Then
  \begin{equation*}
    a^k = (a^m)^q \cdot a^r = e^q \cdot a^r = a^r.
  \end{equation*}

  Therefore, \( G \) itself must have only \( m \) elements. The obtained contradiction shows that \( H \) is infinite.
\end{proof}

\begin{proposition}\label{thm:def:cyclic_group}
  \hyperref[def:cyclic_group]{Cyclic groups} have the following basic properties:
  \begin{thmenum}
    \thmitem{thm:def:cyclic_group/direct_sum} The \hyperref[def:semimodule_direct_sum]{direct sum} \( C_n \oplus C_m \) of two cyclic groups is cyclic if and only if \( n \) and \( m \) are \hyperref[def:coprime_elements]{coprime}.

    \thmitem{thm:def:cyclic_group/generators} A member of \( C_n \) generates it if and only if it has \hyperref[def:group_element_order]{order} \( n \).

    \thmitem{thm:def:cyclic_group/generators_cardinality} The set of \hyperref[def:object_presentation]{generators} of \( C_n \) has cardinality \( \varphi(n) \), where \( \varphi \) is \hyperref[def:eulers_totient_function]{Euler's totient function}.
  \end{thmenum}
\end{proposition}
\begin{proof}
  \SubProofOf{thm:def:cyclic_group/direct_sum} Let \( a \) be a generator of \( C_n \) and \( b \) --- of \( C_m \).

  \SufficiencySubProof* Suppose that \( (a^i, b^j) \) generates \( C_n \oplus C_m \).

  Let \( d \) be a common divisor of \( n \) and \( m \). \Fullref{thm:common_divisor_to_multiple_lemma} implies that \( s \coloneqq nm / d \) is a common multiple. \Fullref{thm:def:group_element_order/neutral} implies that \( a^s = a^0 \) and \( b^s = b^0 \).

  Then
  \begin{equation*}
    (a^i, b^j)^s = (a^{is}, b^{js}) = (a^0, b^0).
  \end{equation*}

  The order of \( (a^i, b^j) \) is thus at most \( s \).

  But it is a generator of \( C_n \oplus C_m \), hence its order is \( nm \). Then \( d = 1 \) and \( n \) and \( m \) are coprime.

  \NecessitySubProof* Conversely, suppose that \( n \) and \( m \) are coprime.

  Suppose that, for some \( j < i \) we have
  \begin{equation*}
    (a, b)^i = (a, b)^j.
  \end{equation*}

  \Fullref{thm:def:group_element_order/neutral} implies that both \( n \) and \( m \) divide \( i - j \). Then their \hyperref[def:lcm]{least common multiple} \( l \) also divides \( i - j \). But since \( n \) and \( m \) are coprime, \( l = nm \), and thus \( nm \) divides \( i - j \).

  Hence, if \( 0 \leq j < i < nm \),
  \begin{equation*}
    (a, b)^i \neq (a, b)^j.
  \end{equation*}

  Then the direct sum \( C_n \oplus C_m \) has at least \( nm \) elements. But by definition it has at most \( nm \) elements. Therefore, \( C_n \oplus C_m \) is a cyclic group of order \( nm \) generated by \( (a, b) \).

  \SubProofOf{thm:def:cyclic_group/generators} By definition of \( \ord(a) \), the subgroup \( \braket{ a } \) has \( \ord(a) \) distinct elements.

  Then \( x \) is a generator of \( C_n \) if and only if \( \ord(a) \) coincides with the cardinality \( n \) of \( C_n \).

  \SubProofOf{thm:def:cyclic_group/generators_cardinality} Let \( a \) be a generator of \( C_n \). \Fullref{thm:def:group_element_order/power} implies that the order of \( a^m \) is \( n / \gcd(n, m) \). This order equals \( n \) if and only if \( n \) and \( m \) are coprime. Therefore, there are \( \varphi(n) \) generators of \( C_n \).
\end{proof}

  \subsection{Group actions}\label{subsec:group_actions}

\paragraph{Monoid actions}

\begin{definition}\label{def:endomorphism_monoid}\mimprovised
  For every object \( X \) in an arbitrary \hyperref[def:category]{category} \( \cat{C} \), the set \( \cat{C}(X) \) is a \hyperref[def:monoid]{monoid} with morphism composition as the monoid operation and the identity \( \id_X \) as the monoid neutral element.

  Outside of \hyperref[sec:category_theory]{category theory}, whenever the category \( \cat{C} \) is clear from the context, we call \( \cat{C}(X) \) the \term{endomorphism monoid} over \( X \) and denote it by \( \End(X) \).
\end{definition}
\begin{comments}
  \item The notation is based on Paolo Aluffi's from \cite[29]{Aluffi2009}, while the name is based on the special case for \hyperref[def:undirected_graph]{simple undirected graphs} from \cite[8]{GodsilRoyle2001}.
\end{comments}

\begin{definition}\label{def:monoid_action}\mimprovised
  Let \( M \) be a \hyperref[def:monoid]{monoid} and let \( X \) be an object in some \hyperref[def:concrete_category]{concrete category} \( \cat{C} \).

  A \term{left monoid action} or simply \term{monoid action} of \( M \) on \( X \) can be defined equivalently as:
  \begin{thmenum}
    \thmitem{def:monoid_action/homomorphism} A \hyperref[def:monoid/homomorphism]{monoid homomorphism} from \( M \) to the \hyperref[def:endomorphism_monoid]{endomorphism monoid} of \( X \).

    Right actions are instead homomorphisms from the \hyperref[def:monoid/opposite]{opposite monoid} \( M^\oppos \) to \( \End(X) \).

    \thmitem{def:monoid_action/functor} A \hyperref[def:functor]{functor} from the \hyperref[def:monoid_delooping]{delooping} \( \cat{B}_M \) to \( \cat{C} \).

    Right actions are instead \hyperref[rem:contravariant_functor]{contravariant functors}.

    \thmitem{def:monoid_action/family} An \hyperref[def:cartesian_product/indexed_family]{indexed family} \( \seq{ \Phi_m }_{m \in M} \) of \hyperref[def:morphism_invertibility/endomorphism]{endomorphisms} of \( X \) such that
    \begin{align}
      &\Phi_e = \id_X, \label{eq:def:monoid_action/family/identity}\tag{\logic{MA1}} \\
      &\Phi_{mn} = \Phi_m \bincirc \Phi_n. \label{eq:def:monoid_action/family/compatibility}\tag{\logic{MA2}}
    \end{align}

    This defines a function \( \Phi: M \times X \to X \).
  \end{thmenum}
\end{definition}
\begin{defproof}
  \ImplicationSubProof{def:monoid_action/homomorphism}{def:monoid_action/functor} Suppose that we have a monoid homomorphism \( \Phi: M \to \End(X) \). Define the functor
  \begin{equation*}
    \begin{aligned}
      &F: \cat{B}_M \to \cat{C} \\
      &F(\anon) \coloneqq X \\
      &F(m) \coloneqq \Phi(m).
    \end{aligned}
  \end{equation*}

  This is indeed a functor because \eqref{eq:def:functor/CF1} follows from \eqref{eq:def:pointed_set/homomorphism} and \eqref{eq:def:functor/CF2} follows from \eqref{eq:def:semigroup/homomorphism}.

  \ImplicationSubProof{def:monoid_action/functor}{def:monoid_action/family} Suppose that we have a functor \( F: \cat{B}_M \to \cat{C} \). Let \( X \coloneqq F(\anon) \) and define the \( M \)-indexed family
  \begin{equation*}
    \begin{aligned}
      &\Phi_m: X \to X \\
      &\Phi_m \coloneqq F(m).
    \end{aligned}
  \end{equation*}

  It satisfies the necessary axioms:
  \begin{itemize}
    \item \ref{eq:def:monoid_action/family/identity} holds:
    \begin{equation*}
      \Phi_e
      =
      F(e)
      \reloset {\eqref{eq:def:functor/CF1}} =
      \id_A.
    \end{equation*}

    \item \ref{eq:def:monoid_action/family/compatibility} holds: for every pair \( m, n \in M \), we have
    \begin{equation*}
      \Phi_{mn}
      =
      F(mn)
      \reloset {\eqref{eq:def:functor/CF2}} =
      F(m) \bincirc F(n)
      =
      \Phi_m \bincirc \Phi_n
    \end{equation*}
  \end{itemize}

  \ImplicationSubProof{def:monoid_action/family}{def:monoid_action/homomorphism} Suppose that we have an indexed family \( \seq{ \Phi_m }_{m \in M} \) of endomorphisms of \( A \) that satisfies the axioms for left action. Regard this indexed family as a function \( \Phi: M \to \End(X) \).

  Then \( \Phi \) is a monoid homomorphism because \ref{eq:def:monoid_action/family/identity} implies \( \Phi(e) = \id_X \) and \eqref{eq:def:monoid_action/family/compatibility} implies
  \begin{equation*}
    \Phi(mn) = \Phi(m) \bincirc \Phi(n).
  \end{equation*}
\end{defproof}

\begin{remark}\label{rem:monoid_action_notation}
  For convenience, we will sometimes denote the values of the \hyperref[def:monoid_action]{evolution function} \( \Phi: T \times X \to X \) as \( \Phi_t(x) \) rather than \( \Phi(t, x) \).
\end{remark}

\begin{proposition}\label{thm:monoid_is_action}
  Every \hyperref[def:monoid]{monoid} \( M \) \hyperref[def:monoid_action]{acts} on itself via the family of plain functions
  \begin{equation*}
    \Phi_m(h) \coloneqq m \cdot h.
  \end{equation*}
\end{proposition}
\begin{comments}
  \item These functions are not monoid homomorphisms in general.
  \item Compare this result to the case of groups in \fullref{thm:group_is_action}.
\end{comments}
\begin{proof}
  The family satisfies \fullref{def:monoid_action/family}:
  \begin{itemize}
    \item \ref{eq:def:monoid_action/family/identity} follows from \eqref{eq:def:monoid/theory/neutral}.

    \item \ref{eq:def:monoid_action/family/compatibility} follows from associativity:
    \begin{equation*}
      [\Phi_{m_1}(h)] \bincirc [\Phi_{m_2}(h)] = [\Phi_{m_1}(\Phi_{m_2}(h))] = [\Phi_{m_1 \cdot m_2}(h)].
    \end{equation*}
  \end{itemize}
\end{proof}

\begin{theorem}[Cayley's theorem for monoids]\label{thm:cayleys_theorem_for_monoids}
  Every \hyperref[def:monoid]{monoid} \( M \) \hyperref[rem:embeds_isomorphically]{embeds isomorphically} into the monoid of all functions on \( M \).
\end{theorem}
\begin{comments}
  \item Compare this to \fullref{thm:cayleys_theorem}.
\end{comments}
\begin{proof}
  The embedding is given by the group action from \fullref{thm:monoid_is_action}.
\end{proof}

\begin{proposition}\label{thm:exponentiation_monoid_action}
  The monoid of \hyperref[def:natural_numbers]{natural numbers} \( \BbbN \) (\hyperref[rem:peano_arithmetic_zero]{with zero}) act on any \hyperref[def:monoid]{monoid} by \hyperref[def:monoid/exponentiation]{exponentiation} via the family of function \( g \mapsto g^n \) indexed by \( n \in \BbbN \).
\end{proposition}
\begin{comments}
  \item A stronger result holds for groups --- see \fullref{thm:exponentiation_group_action}.
\end{comments}
\begin{proof}
  This family satisfies \fullref{def:monoid_action/family}:
  \begin{itemize}
    \item \ref{eq:def:monoid_action/family/identity} is obvious.
    \item \ref{eq:def:monoid_action/family/compatibility} follows from \fullref{thm:semigroup_exponentiation_properties/repeated}.
  \end{itemize}
\end{proof}

\paragraph{Group actions}

\begin{definition}\label{def:automorphism_group}\mcite[29]{Aluffi2009}
  For every object \( X \) in an arbitrary \hyperref[def:category]{category} \( \cat{C} \), define the \term{automorphism group} \( \aut(X) \) as the set of all invertible endomorphisms on \( X \).
\end{definition}
\begin{defproof}
  \Fullref{thm:invertible_submonoid_is_group} implies that \( \aut(X) \) is indeed a group as the subset of all invertible elements of the \hyperref[def:endomorphism_monoid]{endomorphism monoid} \( \End(X) \).
\end{defproof}

\begin{definition}\label{def:group_action}
  Let \( G \) be a \hyperref[def:group]{group} and let \( X \) be an object in some \hyperref[def:concrete_category]{concrete category} \( \cat{C} \).

  A \term{left group action} or simply \term[bg=действие (\cite[def. IV.18]{ГеновМиховскиМоллов1991}), ru=действие (\cite[def. 10.3.1]{Винберг2014})]{group action} of \( G \) on \( X \) can be defined equivalently as:
  \begin{thmenum}
    \thmitem{def:group_action/homomorphism}\mcite[108]{Aluffi2009} A \hyperref[def:group/homomorphism]{group homomorphism} from \( G \) to the \hyperref[def:automorphism_group]{automorphism group} \( \aut(X) \).

    Right actions are instead homomorphisms from the \hyperref[def:group/opposite]{opposite group} \( G^\oppos \) to \( \aut(X) \).

    \thmitem{def:group_action/functor} A \hyperref[def:functor]{functor} from the \hyperref[def:monoid_delooping]{delooping} \( \cat{B}_G \) to \( \cat{C} \).

    Right actions are instead \hyperref[rem:contravariant_functor]{contravariant functors}.

    \thmitem{def:group_action/family} An \hyperref[def:cartesian_product/indexed_family]{indexed family} \( \seq{ \Phi_x }_{x \in G} \) of \hyperref[def:morphism_invertibility/isomorphism]{isomorphisms} of \( X \) such that, for every pair \( g, h \in G \),
    \begin{equation}\label{eq:def:group_action/family/compatibility}\tag{\logic{GA}}
      \Phi_{gh} = \Phi_g \bincirc \Phi_h.
    \end{equation}

    This defines a function \( \Phi: M \times X \to X \).
  \end{thmenum}
\end{definition}

\begin{proposition}\label{thm:group_action_of_neutral_element}
  For a \hyperref[def:group_action]{group action} \( \Phi: G \times X \to X \), the morphism \( \Phi_e \) is the identity on \( X \).
\end{proposition}
\begin{comments}
  \item This is a restatement of \eqref{eq:def:monoid_action/family/identity}.
\end{comments}
\begin{proof}
  Since \( \Phi \) is, equivalently, a group homomorphism from \( G \) to \( \aut(X) \), the morphism \( \Phi_e \) is the neutral element of \( \aut(X) \), hence it is the identity on \( X \).
\end{proof}

\begin{lemma}\label{thm:group_operation_induces_bijections}
  For each element \( g \) of a group \( G \), the function \( h \mapsto g \cdot h \) is bijective.
\end{lemma}
\begin{comments}
  \item These functions are not group isomorphisms unless \( g \) is the neutral element.
\end{comments}
\begin{proof}
  Denote \( h \mapsto g \cdot h \) by \( \Phi_g \).

  \SubProofOf[def:function_invertibility/injective/equality]{injectivity} If \( \Phi_g(h) = \Phi_g(h') \), then
  \begin{equation*}
    gh = \Phi_g(h) = \Phi_g(h') = gh'.
  \end{equation*}

  By \fullref{thm:def:group/cancellative}, \( h = h' \).

  \SubProofOf[def:function_invertibility/surjective/existence]{surjectivity} If \( h \in G \), then \( h = g(g^{-1} h) \). Therefore, \( h = \Phi_g(g^{-1} h) \), and thus every member of \( G \) has a preimage.
\end{proof}

\begin{proposition}\label{thm:group_is_action}
  Every \hyperref[def:group]{group} \( G \) \hyperref[def:group_action]{acts} on itself via the family of plain (bijective) functions
  \begin{equation*}
    \Phi_g(h) \coloneqq g \cdot h.
  \end{equation*}
\end{proposition}
\begin{comments}
  \item The phrase \enquote{plain bijective functions} highlight that the functions may not be group homomorphisms.
\end{comments}
\begin{proof}
  Follows directly from \fullref{thm:monoid_is_action} and \fullref{thm:group_operation_induces_bijections}.
\end{proof}

\begin{theorem}[Cayley's theorem]\label{thm:cayleys_theorem}
  Every \hyperref[def:group]{group} \( G \) \hyperref[rem:embeds_isomorphically]{embeds isomorphically} into the corresponding \hyperref[def:symmetric_group]{symmetric group} \( S_G \).
\end{theorem}
\begin{comments}
  \item Compare this to \fullref{thm:cayleys_theorem_for_monoids}.
\end{comments}
\begin{proof}
  The embedding is given by the group action from \fullref{thm:group_is_action}.
\end{proof}

\begin{proposition}\label{thm:exponentiation_group_action}
  The group of \hyperref[def:integers]{integers} \( \BbbZ \) acts on any \hyperref[def:group]{group} by \hyperref[def:monoid/exponentiation]{exponentiation} via the family of function \( g \mapsto g^n \) indexed by \( n \in \BbbZ \).
\end{proposition}
\begin{comments}
  \item A weaker result holds for monoid --- see \fullref{thm:exponentiation_monoid_action}.
\end{comments}
\begin{proof}
  Follows from \fullref{thm:exponentiation_monoid_action}.
\end{proof}

\begin{proposition}\label{thm:group_conjugation_action}\mcite[165]{Knapp2016BasicAlgebra}
  Left (resp. right) \hyperref[def:group_conjugation]{conjugation} by group elements is a left (resp. right) \hyperref[def:group_action]{group action} of the group on itself.
\end{proposition}
\begin{proof}
  We will first consider left actions. Let
  \begin{equation*}
    \Phi_g(x) \coloneqq g \cdot x \cdot g^{-1}.
  \end{equation*}

  We must only verify \eqref{eq:def:group_action/family/compatibility}, which follows directly from \fullref{thm:def:group/inverse_composition}:
  \begin{equation*}
    \Phi_{gh}(x)
    =
    g \cdot h \cdot x \cdot h^{-1} \cdot g^{-1}
    =
    \Phi_g(\Phi_h(x)).
  \end{equation*}

  For right conjugation, we instead have
  \begin{equation*}
    \Phi_{gh}(x)
    =
    h^{-1} \cdot g^{-1} \cdot x \cdot g \cdot h
    =
    \Phi_h(\Phi_g(x)).
  \end{equation*}
\end{proof}

\begin{definition}\label{def:group_action_orbit}\mcite[163]{Knapp2016BasicAlgebra}
  The \term[bg=орбита (\cite[def. IV.20]{ГеновМиховскиМоллов1991}), ru=орбита (\cite[453]{Винберг2014})]{orbit} of \( x \) under the \hyperref[def:group_action]{group action} \( \Phi: G \times X \to X \) is the set of all members of \( X \) \enquote{reachable} from \( x \):
  \begin{equation*}
    \set{ \Phi_g(x) \given g \in G }.
  \end{equation*}
\end{definition}

\begin{proposition}\label{thm:orbit_induces_partition}
  The set of all \hyperref[def:group_action_orbit]{orbits} of a group action \( \Phi: G \times X \to X \) is a \hyperref[def:set_partition]{partition} of \( X \).
\end{proposition}
\begin{proof}
  It will be simpler for us to show that \( {\sim} \) is an equivalence relation, where we define \( {\sim} \) to hold for \( x \) and \( x' \) if they belong to the same orbit.

  \SubProofOf[def:binary_relation/reflexive]{reflexivity} \Fullref{thm:group_action_of_neutral_element} implies that, for any \( x \in X \), we have \( \Phi_e(x) = x \) and hence \( x \sim x \).

  \SubProofOf[def:binary_relation/symmetric]{symmetry} If \( x \sim x' \), then there exists a group element \( g \) such that \( x' = \Phi_g(x) \). Then \( x = \Phi_{g^{-1}}(x') \).

  \SubProofOf[def:binary_relation/transitive]{transitivity} Follows directly from \eqref{eq:def:group_action/family/compatibility}.
\end{proof}

\begin{example}\label{ex:def:group_action_orbit}
  We give examples of \hyperref[def:group_action_orbit]{orbits} of group actions:
  \begin{thmenum}
    \thmitem{ex:def:group_action_orbit/rotation} The \hyperref[def:circle_group]{circle group} represents \hyperref[def:rigid_motion/rotation]{rotation} in the \hyperref[def:euclidean_plane]{Euclidean plane} under composition and \hyperref[def:group_action_orbit]{orbits} of this action are \hyperref[def:circle]{circles}.

    \begin{figure}
      \centering
      \includegraphics{output/ex__def__group_action_orbit__rotation}
      \caption{The rotation action from \fullref{ex:def:group_action_orbit/rotation}}
      \label{fig:ex:def:group_action_orbit/rotation}
    \end{figure}

    Indeed, the orbit of the point \( (x, y) \) is the set of all rotations of \( (x, y) \). \Fullref{thm:isometry_iff_affine_orthogonal_operator} implies that rotations are isometries, hence the point \( (x, y) \) uniquely determines the norm of each point in its orbit. Therefore, this orbit is precisely the \hyperref[def:circle]{circle} centered at \( \vec 0 \) with radius \( \sqrt{x^2 + y^2} \).
  \end{thmenum}
\end{example}

\paragraph{Symmetric groups and permutations}

\begin{definition}\label{def:symmetric_group}\mcite[def. II.2.1]{Aluffi2009}
  We call the \hyperref[def:automorphism_group]{automorphism group} of a set \( A \) the \term[bg=симетрична група (\cite[80]{ГеновМиховскиМоллов1991}), ru=симметрическая группа (\cite[154]{Винберг2014})]{symmetric group}\fnote{The term is closely related to symmetric functions defined in \fullref{def:symmetric_function}} on \( A \) and denote it by \( S(A) \). The group \( S(A) \) consists of bijective functions, which we call \term[bg=субституция (\cite[80]{ГеновМиховскиМоллов1991}), ru=подстановка (\cite[154]{Винберг2014})); перестановка (\cite[sec. 4.2]{Тыртышников2007})]{permutations}.

  Rather than considering arbitrary sets, we often restrict \( A \) to be the set of the first \( n \) positive integers:
  \begin{equation*}
    S_n \coloneqq S(\set{ 1, 2, \ldots, n }).
  \end{equation*}

  We can denote the permutation \( \sigma \) in \( S_n \) as
  \begin{equation*}
    \sigma
    =
    \begin{pmatrix}
      1         & \cdots & n \\
      \sigma(1) & \cdots & \sigma(n)
    \end{pmatrix}.
  \end{equation*}
\end{definition}
\begin{comments}
  \item Some authors, for example \incite[253]{MacLane1998} and \incite[121]{Knapp2016BasicAlgebra}, call \( S_n \) the \enquote{symmetric group on \( n \) letters}.
  \item Our first complete example of a symmetric group will be given for \( S_3 \) in \fullref{ex:s3}, since we need additional machinery in order to properly study it.
\end{comments}

\begin{example}\label{ex:def:symmetric_group}
  We list examples of \hyperref[def:symmetric_group]{symmetric groups and permutations}
  \begin{thmenum}
    \thmitem{ex:def:symmetric_group/automorphism} Every \hyperref[def:morphism_invertibility/automorphism]{automorphism} is a permutation, but not vice versa.

    For example, the additive group of integers \( \BbbZ \) has an automorphism given by \( x \mapsto -x \), and this is a permutation since the only requirement for being a permutation is being a bijective function.

    But an arbitrary permutation like \( x \mapsto x + 1 \) is not an automorphism since \( 0 \mapsto 1 \).

    \thmitem{ex:def:symmetric_group/diamond} The permutation
    \begin{equation}\label{eq:ex:def:symmetric_group/diamond}
      \begin{pmatrix}
        1 & 2 & 3 & 4 & 5 & 6 \\
        3 & 4 & 5 & 6 & 1 & 2
      \end{pmatrix}
    \end{equation}
    is displayed graphically in \ref{fig:ex:def:symmetric_group/diamond}.

    The permutation sends even numbers to even numbers and odd numbers to odd numbers. In fact, we can represent it as the composition
    \begin{equation*}
      \underbrace
        {
          \begin{pmatrix}
            \mathbf{1} & 2 & \mathbf{3} & 4 & \mathbf{5} & 6 \\
            \mathbf{3} & 2 & \mathbf{5} & 4 & \mathbf{1} & 6
          \end{pmatrix}
        }
        _
        {
          \cycle{ 1, 3, 5 }
        }
      \bincirc
      \underbrace
        {
          \begin{pmatrix}
            1 & \mathbf{2} & 3 & \mathbf{4} & 5 & \mathbf{6} \\
            1 & \mathbf{4} & 3 & \mathbf{6} & 5 & \mathbf{2}
          \end{pmatrix}.
        }
        _
        {
          \cycle{ 2, 4, 6 }
        }
    \end{equation*}

    The shortened notation in the underbraces is introduced in \fullref{rem:cycle_notation} and, for these concrete permutations, discussed in \fullref{ex:def:cyclic_permutation/diamond}.

    \begin{figure}
      \centering
      \includegraphics{output/ex__def__symmetric_group}
      \caption{A visualization of the permutation from \fullref{ex:def:symmetric_group/diamond}}
      \label{fig:ex:def:symmetric_group/diamond}
    \end{figure}
  \end{thmenum}
\end{example}

\begin{lemma}\label{thm:sum_of_powers_in_composition}
  For any endomorphism \( f: A \to A \) in any \hyperref[def:category]{category}, for any integers \( n \) and \( m \) we have
  \begin{equation*}
    f^{n + m} = f^n \bincirc f^m.
  \end{equation*}
\end{lemma}
\begin{proof}
  Follows via simple induction on \( m \).
\end{proof}

\begin{proposition}\label{thm:symmetric_group_action}
  Given any \hyperref[def:symmetric_group]{permutation} \( \sigma \) from \( S_A \), the additive group of integers \( \BbbZ \) \hyperref[def:group_action]{acts} on \( A \) via
  \begin{equation*}
    \Phi_e(a) \coloneqq \sigma^e(a).
  \end{equation*}
\end{proposition}
\begin{proof}
  \Fullref{thm:sum_of_powers_in_composition} implies that the condition \eqref{eq:def:group_action/family/compatibility} holds.
\end{proof}

\begin{remark}\label{rem:symmetric_group_actions}
  Every symmetric group \( S_A \) acts on \( A \) via
  \begin{equation*}
    \Phi_\sigma(a) \coloneqq \sigma(a).
  \end{equation*}

  This should not be confused with the action \fullref{thm:symmetric_group_action}, with respect to which we consider orbits in \fullref{def:cyclic_permutation}.
\end{remark}

\paragraph{Cyclic permutations}

\begin{definition}\label{def:cyclic_permutation}\mcite[def. II.4.1]{Aluffi2009}
  We say that the \hyperref[def:symmetric_group]{permutation} \( \sigma \) in \( S_n \) is \term{cyclic} or a \term[ru=цикл (\cite[sec. 4.3]{Тыртышников2007})]{cycle} if the action \( \Phi_e \) from \fullref{thm:symmetric_group_action} has exactly one \hyperref[def:group_action_orbit]{orbit} that is nontrivial, that is, it has more than one element. We call the cardinality of this orbit the \term{length} of the cycle.
\end{definition}
\begin{comments}
  \item Cyclic permutations are defined only for the finite symmetric groups \( S_n \).
  \item We elaborate on the definition in \fullref{thm:cyclic_permutation_characterization} and introduce a special cycle notation in \fullref{rem:cycle_notation}.
\end{comments}

\begin{proposition}\label{thm:cyclic_permutation_characterization}
  \hyperref[def:cyclic_permutation]{Cyclic permutations} in \( S_n \) can be characterized as follows:
  \begin{thmenum}
    \thmitem{thm:cyclic_permutation_characterization/cycle} Pick one element from the nontrivial orbit and denote it by \( s_1 \). Recursively define
    \begin{equation*}
      s_{k+1} \coloneqq \sigma(s_k).
    \end{equation*}
    for all \( k < m \). Then
    \begin{equation}\label{eq:thm:def:cyclic_permutation/cycle}
      \sigma(q) = \begin{cases}
        s_1,       &q = s_m, \\
        s_{i + 1}, &q = s_i \T{for} i < m, \\
        q,         &\T{otherwise}
      \end{cases}
    \end{equation}

    \begin{figure}
      \centering
      \includegraphics{output/thm__cyclic_permutation_characterization}
      \caption{Visualization of a cycle with five elements}
      \label{fig:thm:cyclic_permutation_characterization}
    \end{figure}

    \thmitem{thm:cyclic_permutation_characterization/generating} If \( s_1, \ldots, s_m \) are distinct numbers between \( 1 \) and \( n \) and if \( m > 1 \), then \fullref{eq:thm:def:cyclic_permutation/cycle} defines a cycle of length \( m \).
  \end{thmenum}
\end{proposition}
\begin{comments}
  \item Some authors like \incite[15]{Knapp2016BasicAlgebra} and \incite[sec. 4.3]{Тыртышников2007} define cycles as permutations satisfying \cref{eq:thm:def:cyclic_permutation/cycle}.
\end{comments}
\begin{proof}
  \SubProofOf{thm:cyclic_permutation_characterization/cycle} We will prove the individual cases in \eqref{eq:thm:def:cyclic_permutation/cycle}:
  \begin{itemize}
    \item \Fullref{def:pigeonhole_principle} implies that the value \( \sigma(s_m) \) must be among \( s_1, \ldots, s_m \).

    If \( \sigma(s_m) = s_i \) for \( i > 1 \), then \( \sigma(s_m) = s_i = \sigma(s_{i-1}) \), contradicting the injectivity of \( \sigma \).

    Hence, it follows that \( \sigma(s_m) = s_1 \).

    \item For any \( i < m \), by definition \( \sigma(s_i) = s_{i + 1} \).

    \item If \( q \) is not in the nontrivial orbit of \( \sigma \), it is the only element of its own orbit, and hence \( \sigma(q) = q \).
  \end{itemize}

  \SubProofOf{thm:cyclic_permutation_characterization/generating} Trivial.
\end{proof}

\begin{remark}\label{rem:cycle_notation}
  Having \fullref{thm:cyclic_permutation_characterization} in mind, we can introduce a simplified notation for \hyperref[def:cyclic_permutation]{cyclic permutations}.

  Rather than the elaborate definition \eqref{eq:thm:def:cyclic_permutation/cycle}, we can write the corresponding cycle as
  \begin{equation*}
    \sigma = \cycle{ s_1, \ldots, s_m }.
  \end{equation*}
\end{remark}

\begin{example}\label{ex:def:cyclic_permutation}
  We list examples of \hyperref[def:cyclic_permutation]{cyclic permutations}:
  \begin{thmenum}
    \thmitem{ex:def:cyclic_permutation/id} The identity does not satisfy our definition of permutation because every orbit is trivial.

    \thmitem{ex:def:cyclic_permutation/123} Consider the permutation \( \cycle{ 1, 2, 3 } \) from \( S_3 \). Writing it via the permutation notation from \fullref{def:symmetric_group}, we obtain
    \begin{equation*}
      \begin{pmatrix}
        1 & 2 & 3 \\
        2 & 3 & 1
      \end{pmatrix}.
    \end{equation*}

    Furthermore,
    \begin{equation*}
      \cycle{ 1, 2, 3 } = \cycle{ 1, 3 } \bincirc \cycle{ 1, 2 },
    \end{equation*}
    which generalizes to \fullref{thm:cycle_transposition_decomposition}.

    We will discuss this group in detail in \fullref{ex:s3}.

    \thmitem{ex:def:cyclic_permutation/diamond} Consider the permutation \eqref{eq:ex:def:symmetric_group/diamond} from \fullref{ex:def:symmetric_group/diamond}. It has two orbits --- \( \set{ 1, 3, 5 } \) and \( \set{ 2, 4, 6 } \). The two orbits correspond to the cycles \( \cycle{ 1, 3, 5 } \) and \( \cycle{ 2, 4, 6 } \). Composing these cycles in any order gives \eqref{eq:ex:def:symmetric_group/diamond}.

    This is generalized in \fullref{thm:permutation_decomposition_into_disjoint_cycles}.

    Furthermore, from \cref{fig:ex:def:symmetric_group/diamond} it becomes clear that
    \begin{equation*}
      \cycle{ 1, 3, 5 } = \cycle{ 3, 5, 1 } = \cycle{ 5, 1, 3 }.
    \end{equation*}

    This is generalized in \fullref{thm:cyclic_permutation_cyclic_shift}.
  \end{thmenum}
\end{example}

\begin{definition}\label{def:cyclic_shift}\mimprovised
  The \term{cyclic shift} of a finite sequence
  \begin{equation*}
    a_1, a_2, \ldots, a_n
  \end{equation*}
  by an integer \( c \) is the sequence
  \begin{equation*}
    a_{k_1}, \ldots, a_{k_n},
  \end{equation*}
  where \( k_i \coloneqq \rem(i + c, m) \) for \( i = 1, \ldots, n \).
\end{definition}

\begin{proposition}\label{thm:cyclic_permutation_cyclic_shift}
  \hyperref[def:cyclic_permutation]{Cyclic permutations} are invariant under \hyperref[def:cyclic_shift]{cyclic shifts} of the elements in their representations.

  More concretely, consider a cycle \( \cycle{ s_1, \cdots, s_m } \) from \( S_n \) and let \( k_1, \ldots, k_m \) be the coefficients of some cyclic shift of \( s_1, \ldots, s_n \). Then
  \begin{equation*}
    \cycle{ s_1, \cdots, s_m } = \cycle{ s_{k_1}, \cdots, s_{k_m} }.
  \end{equation*}
\end{proposition}
\begin{comments}
  \item If we visualize the cycle as in \cref{fig:thm:cyclic_permutation_characterization}, cyclic shifting corresponds to \hyperref[def:rigid_motion/rotation]{plane rotation}.
\end{comments}
\begin{proof}
  Obvious from the definition of cycle.
\end{proof}

\begin{definition}\label{def:disjoint_cycle}\mcite[214]{Aluffi2009}
  We say that two \hyperref[def:cyclic_permutation]{cycles} on \( S_n \) are \term[ru=независимые (\cite[sec. 4.3]{Тыртышников2007})]{disjoint} if their nontrivial \hyperref[def:group_action_orbit]{orbits} are disjoint.
\end{definition}
\begin{comments}
  \item More simply, the cycles \( \cycle{ s_1, \ldots, s_m } \) and \( \cycle{ r_1, \ldots, r_l } \) are disjoint if the sets \( \set{ s_1, \ldots, s_m } \) and \( \set{ r_1, \ldots, r_l } \) are disjoint.
\end{comments}

\begin{proposition}\label{thm:disjoint_cycles_commute}
  \hyperref[def:disjoint_cycle]{Disjoint cycles} commute under composition.
\end{proposition}
\begin{proof}
  Trivial.
\end{proof}

\begin{proposition}\label{thm:permutation_decomposition_into_disjoint_cycles}
  Let \( \sigma \) be an arbitrary permutation in \( S_n \) and let \( O_1, \ldots, O_p \) be the nontrivial orbits of \( \sigma \).

  To each nontrivial orbit \( O_i \) there corresponds some cycle \( \cycle{ s_{i,1}, \ldots, s_{i,m_i} } \). Then
  \begin{equation*}
    \sigma = \cycle{ s_{1,1}, \ldots, s_{1,m_1} } \bincirc \cdots \bincirc \cycle{ s_{p,1}, \ldots, s_{p,m_p} }.
  \end{equation*}

  Furthermore, the cycles are pairwise \hyperref[def:disjoint_cycle]{disjoint}, and \fullref{thm:disjoint_cycles_commute} implies that we can rearrange them.
\end{proposition}
\begin{proof}
  Trivial.
\end{proof}

\paragraph{Alternating groups}

\begin{definition}\label{def:transposition}\mcite[219]{Aluffi2009}
  A \term{transposition} is a \hyperref[def:cyclic_permutation]{cyclic permutation} of length \( 2 \).
\end{definition}

\begin{proposition}\label{thm:transpositions_are_involutions}
  Every \hyperref[def:transposition]{transposition} is an \hyperref[def:involution]{involution}.
\end{proposition}
\begin{proof}
  Trivial.
\end{proof}

\begin{proposition}\label{thm:cycle_transposition_decomposition}
  Every nontrivial cycle \( \cycle{ s_1, \cdots, s_m } \) can be decomposed into the product of \hyperref[def:transposition]{transpositions}
  \begin{equation*}
    \cycle{ s_1, \cdots, s_m } = \cycle{ s_1, s_m } \bincirc \cycle{ s_1, s_{m-1} } \bincirc \cdots \bincirc \cycle{ s_1, s_2 }.
  \end{equation*}
\end{proposition}
\begin{proof}
  We will use induction by \( m \). The proposition is trivial for transpositions (\( m = 2 \)), hence suppose that it holds for \( m - 1 \) and consider the transposition \( \cycle{ s_1, \cdots, s_m } \).

  We have, by the inductive hypothesis,
  \begin{equation*}
    \cycle{ s_1, \cdots, s_{m - 1} }
    =
    \cycle{ s_1, s_{m - 1} } \bincirc \cycle{ s_1, s_{m-1} } \bincirc \cdots \bincirc \cycle{ s_1, s_2 }.
  \end{equation*}

  By definition of cycle, \( s_{m-1} \) and \( s_1 \) get swapped, therefore
  \begin{equation*}
    \cycle{ s_1, \cdots, s_m } = \cycle{ s_1, s_m } \bincirc \cycle{ s_1, \cdots, s_{m - 1} }.
  \end{equation*}

  This completes the induction.
\end{proof}

\begin{lemma}\label{thm:permutation_parity_correctness}
  If a \hyperref[def:symmetric_group]{permutation} \( \sigma \in S_n \) can be decomposed into \hyperref[def:transposition]{transpositions} as both
  \begin{equation}\label{eq:thm:permutation_parity_correctness/s}
    \sigma = \underbrace{\cycle{ s_1, s_2 } \bincirc \cycle{ s_3, s_4 } \bincirc \cdots \bincirc \cycle{ s_{2m-1}, s_{2m} }}_{m \T*{transpositions}}
  \end{equation}
  and
  \begin{equation}\label{eq:thm:permutation_parity_correctness/r}
    \sigma = \underbrace{\cycle{ r_1, r_2 } \bincirc \cycle{ r_3, r_4 } \bincirc \cdots \bincirc \cycle{ r_{2l-1}, r_{2l} }}_{l \T*{transpositions}},
  \end{equation}
  then \( m - l \) is an even number.
\end{lemma}
\begin{comments}
  \item The gist of the proposition is discussed in \fullref{ex:thm:permutation_parity_correctness}.
\end{comments}
\begin{proof}
  We will use induction on \( m \). First consider the base case \( m = 0 \). Then \( \sigma \) is the identity. Hence, every transposition in \eqref{eq:thm:permutation_parity_correctness/r} should be present twice so that its action cancels out. Therefore, \( l \) is an even number.

  Now suppose that the statement holds for \( m - 1 \). Add (compose on the right) the last transposition \( \cycle{ s_{2m-1}, s_{2m} } \) of \eqref{eq:thm:permutation_parity_correctness/s} to both \eqref{eq:thm:permutation_parity_correctness/s} and \eqref{eq:thm:permutation_parity_correctness/r}. The obtained permutations are obviously equal. Furthermore, since \( \cycle{ s_{2m-1}, s_{2m} } \) is its own inverse, we can just as well remove \( \cycle{ s_{2m-1}, s_{2m} } \) from \eqref{eq:thm:permutation_parity_correctness/s} to obtain a decomposition of \( \sigma \bincirc \cycle{ s_{2m-1}, s_{2m} } \) into \( m - 1 \) (rather than \( m + 1 \)) transpositions.

  By the inductive hypothesis, \( (m - 1) - (l + 1) = m - l - 2 \) is an even number. Therefore, \( m - l \) is also an even number.
\end{proof}

\begin{example}\label{ex:thm:permutation_parity_correctness}
  We will demonstrate \fullref{thm:permutation_parity_correctness}. Consider the permutation \( \sigma \) from \fullref{ex:def:symmetric_group/diamond}. We have shown in \fullref{ex:def:cyclic_permutation/diamond} that
  \begin{equation*}
    \sigma = \cycle{ 1, 3, 5 } \bincirc \cycle{ 2, 4, 6 }.
  \end{equation*}

  \Fullref{thm:cycle_transposition_decomposition} implies that
  \begin{equation*}
    \sigma = \cycle{ 1, 5 } \bincirc \cycle{ 1, 3 } \bincirc \cycle{ 2, 4 } \bincirc \cycle{ 2, 6 }.
  \end{equation*}

  We can add any transposition to this decomposition, but we must also reverse its action to obtain \( \eqref{eq:ex:def:symmetric_group/diamond} \) again. For example, \( \sigma \bincirc \cycle{ 1, 2 } \) is not equal to \( \sigma \), but
  \begin{equation*}
    \sigma \bincirc \cycle{ 1, 2 } \bincirc \cycle{ 1, 2 } = \sigma.
  \end{equation*}

  Let us try to find another decomposition of \( \sigma \). First note that
  \begin{equation*}
    \sigma \bincirc \cycle{ 1, 2 }
    =
    \begin{pmatrix}
      1 & 2 & 3 & 4 & 5 & 6 \\
      3 & 4 & 5 & 6 & 2 & 1
    \end{pmatrix}
    =
    \cycle{ 1, 3, 5, 2, 4, 6 }
    \reloset {\ref{thm:cyclic_permutation_cyclic_shift}} =
    \cycle{ 3, 5, 2, 4, 6, 1 }.
  \end{equation*}

  We choose to start with \( 3 \) so that no two transpositions in our final representation are equal. \Fullref{thm:cycle_transposition_decomposition} implies that
  \begin{equation*}
    \sigma \bincirc \cycle{ 1, 2 }
    =
    \underbrace{\cycle{ 3, 1 }}_{\cycle{ 1, 3 }} \bincirc \cycle{ 3, 6 } \bincirc \cycle{ 3, 4 } \bincirc \underbrace{\cycle{ 3, 2 }}_{\cycle{ 2, 3 }} \bincirc \cycle{ 3, 5 }.
  \end{equation*}

  Then
  \begin{equation*}
    \sigma
    \reloset {\ref{thm:transpositions_are_involutions}} =
    \sigma \bincirc \cycle{ 1, 2 } \bincirc \cycle{ 1, 2 }
    =
    \cycle{ 1, 3 } \bincirc \cycle{ 3, 6 } \bincirc \cycle{ 3, 4 } \bincirc \cycle{ 2, 3 } \bincirc \cycle{ 3, 5 } \bincirc \cycle{ 1, 2 }.
  \end{equation*}

  These transpositions are not disjoint, unlike those from our initial decomposition.
\end{example}

\begin{proposition}\label{thm:permutation_decomposition_existence}
  Every \hyperref[def:symmetric_group]{permutation} from \( S_n \) can be decomposed into a product of \hyperref[def:cyclic_permutation]{transpositions}.
\end{proposition}
\begin{comments}
  \item It follows from the discussion in \fullref{ex:thm:permutation_parity_correctness} that this decomposition is not unique.
\end{comments}
\begin{proof}
  By \fullref{thm:permutation_decomposition_into_disjoint_cycles}, the permutation can be decomposed into a (perhaps nullary) product of cycles. By \fullref{thm:permutation_parity_correctness}, each of those cycles can be decomposed into a product of transpositions. Hence, at least one decomposition into transpositions exists for every permutation.
\end{proof}

\begin{definition}\label{def:permutation_parity}\mimprovised
  We say that a \hyperref[def:symmetric_group]{permutation} from \( S_n \) is \term{even} if it can be decomposed into an even number of \hyperref[def:cyclic_permutation]{transpositions}. Otherwise, we call the permutation \term{odd}.

  We correspondingly define the \term{sign} of a permutation as \( 1 \) for even permutations and as \( -1 \) for odd permutation.
\end{definition}
\begin{defproof}
  \Fullref{thm:permutation_decomposition_existence} implies that at least one such decomposition exists.

  \Fullref{thm:permutation_parity_correctness} implies that if one such decomposition has an even (resp. odd) number of transpositions, every other decomposition also has an even (resp. odd) number of transpositions.
\end{defproof}
\begin{proposition}\label{thm:symmetric_group_cardinality}
  The \hyperref[def:symmetric_group]{symmetric group} \( S_n \) has \( n! \) elements.
\end{proposition}
\begin{proof}
  We use induction on \( n \). The case \( n = 1 \) is trivial. Suppose that \( S_{n-1} \) has \( (n-1)! \) elements. Then \( S_n \) is obtained by permuting \( n \) with each element of \( S_{n-1} \). That is,
  \begin{equation*}
    S_n = \set{ \cycle{ k, n } \bincirc \sigma \given \sigma \in S_{n-1} \T{and} 1 \leq k \leq n }.
  \end{equation*}

  It follows that
  \begin{equation*}
    \card(S_n) = n \cdot \card(S_{n-1}) = n (n-1)! = n!.
  \end{equation*}
\end{proof}

\begin{example}\label{ex:s3}
  The \hyperref[def:symmetric_group]{symmetric group} \( S_3 \) contains the following \hyperref[def:symmetric_group]{permutations}:
  \begin{equation*}
    S_3
    \coloneqq
    \set[\vast]
    {
      \underbrace
        {
          \begin{pmatrix}
            1 & 2 & 3 \\
            1 & 2 & 3
          \end{pmatrix}
        }_{
          \id
        },
      \underbrace
        {
          \begin{pmatrix}
            1 & 2 & 3 \\
            2 & 1 & 3
          \end{pmatrix}
        }_{
          \cycle{ 1, 2 }
        },
      \underbrace
        {
          \begin{pmatrix}
            1 & 2 & 3 \\
            2 & 3 & 1
          \end{pmatrix}
        }_{
          \cycle{ 1, 2, 3 }
        },
      \underbrace
        {
          \begin{pmatrix}
            1 & 2 & 3 \\
            3 & 2 & 1
          \end{pmatrix}
        }_{
          \cycle{ 1, 3 }
        },
      \underbrace
        {
          \begin{pmatrix}
            1 & 2 & 3 \\
            3 & 1 & 2
          \end{pmatrix}
        }_{
          \cycle{ 1, 3, 2 }
        },
      \underbrace
        {
          \begin{pmatrix}
            1 & 2 & 3 \\
            1 & 3 & 2
          \end{pmatrix}
        }_{
          \cycle{ 2, 3 }
        }
    }
  \end{equation*}

  We will now construct a multiplication table to show that \( S_3 \) is not commutative. This will also help us build examples based on \( S_3 \) later on.

  The entire table can be filled using the following techniques:
  \begin{itemize}
    \item The identity permutation \( \id \) leaves the other multiplicand unchanged.
    \item The cycles \( \cycle{ 1, 2, 3 } \) and \( \cycle{ 1, 3, 2 } \) are inverses of each other, and \fullref{thm:transpositions_are_involutions} implies that all others are \hyperref[def:involution]{involutions}.
    \item \Fullref{thm:cycle_transposition_decomposition} implies that \( \cycle{ 1, 2, 3 } = \cycle{ 1, 3 } \bincirc \cycle{ 1, 2 } \) and \( \cycle{ 1, 3, 2 } = \cycle{ 1, 2 } \bincirc \cycle{ 1, 3 } \).
    \item The last two points together imply
    \begin{equation*}
      \cycle{ 1, 2, 3 } \bincirc \cycle{ 1, 2 }
      =
      \cycle{ 1, 3 } \bincirc \cycle{ 1, 2 } \bincirc \cycle{ 1, 2 }
      =
      \cycle{ 1, 3 }.
    \end{equation*}

    \item We can also utilize cyclic shifts as justified by \fullref{thm:cyclic_permutation_cyclic_shift}:
    \begin{equation*}
      \cycle{ 1, 2 } \bincirc \cycle{ 2, 3 }
      =
      \cycle{ 2, 1 } \bincirc \cycle{ 2, 3 }
      =
      \cycle{ 2, 3, 1 }
      =
      \cycle{ 1, 2, 3 }
    \end{equation*}

    Other cases can be handled similarly.
  \end{itemize}

  All in all, here is a table for \( \tau \bincirc \sigma \):
  \begin{center}
    \begin{tabular}{r | c c c c c c}
      \diagbox[height=2em]{\( \tau \)}{\( \sigma \)} & \( \id \)               & \( \cycle{ 1, 2 } \)    & \( \cycle{ 1, 3 } \)    & \( \cycle{ 2, 3 } \)    & \( \cycle{ 1, 2, 3 } \) & \( \cycle{ 1, 3, 2 } \) \\
      \hline
      \( \id \)                                      & \( \id \)               & \( \cycle{ 1, 2 } \)    & \( \cycle{ 1, 3 } \)    & \( \cycle{ 2, 3 } \)    & \( \cycle{ 1, 2, 3 } \) & \( \cycle{ 1, 3, 2 } \) \\
      \( \cycle{ 1, 2 } \)                           & \( \cycle{ 1, 2 } \)    & \( \id \)               & \( \cycle{ 1, 3, 2 } \) & \( \cycle{ 1, 2, 3 } \) & \( \cycle{ 2, 3 } \)    & \( \cycle{ 1, 3 } \)    \\
      \( \cycle{ 1, 3 } \)                           & \( \cycle{ 1, 3 } \)    & \( \cycle{ 1, 2, 3 } \) & \( \id \)               & \( \cycle{ 1, 3, 2 } \) & \( \cycle{ 1, 2 } \)    & \( \cycle{ 2, 3 } \)    \\
      \( \cycle{ 2, 3 } \)                           & \( \cycle{ 2, 3 } \)    & \( \cycle{ 1, 3, 2 } \) & \( \cycle{ 1, 2, 3 } \) & \( \id \)               & \( \cycle{ 1, 3 } \)    & \( \cycle{ 1, 2 } \)    \\
      \( \cycle{ 1, 2, 3 } \)                        & \( \cycle{ 1, 2, 3 } \) & \( \cycle{ 1, 3 } \)    & \( \cycle{ 2, 3 } \)    & \( \cycle{ 1, 2 } \)    & \( \cycle{ 1, 3, 2 } \) & \( \id \)               \\
      \( \cycle{ 1, 3, 2 } \)                        & \( \cycle{ 1, 3, 2 } \) & \( \cycle{ 2, 3 } \)    & \( \cycle{ 1, 2 } \)    & \( \cycle{ 1, 3 } \)    & \( \id \)               & \( \cycle{ 1, 2, 3 } \) \\
    \end{tabular}
  \end{center}
\end{example}

\begin{definition}\label{def:alternating_group}
  The \term{alternating group} \( A_n \) on \( n \) letters is the subgroup of all \hyperref[def:permutation_parity]{even permutation} in the \hyperref[def:symmetric_group]{symmetric group} \( S_n \).
\end{definition}

\begin{proposition}\label{thm:alternating_group_cardinality}
  The \hyperref[def:alternating_group]{alternating group} \( A_n \) has \( n! / 2 \) elements.
\end{proposition}
\begin{proof}
  We use induction on \( n \). The case \( n = 1 \) is trivial. Suppose that \( A_{n-1} \) has \( (n-1)! / 2 \) elements. Then
  \begin{equation*}
    A_n = \set{ \cycle{ k, n } \bincirc \sigma \given \sigma \in S_{n-1} \setminus A_{n-1} \T{and} 1 \leq k \leq n }.
  \end{equation*}

  We obtain \( A_n \) by taking all the odd permutations in \( S_{n-1} \) and composing them with one new transposition. It follows that
  \begin{equation*}
    \card(A_n) = n \cdot \card(S_{n-1} \setminus A_{n-1}) = n \frac {(n-1)!} 2 = \frac {n!} 2.
  \end{equation*}
\end{proof}

\begin{example}\label{ex:s3_and_a3}
  In the \hyperref[def:symmetric_group]{symmetric group} \( S_3 \) discussed in \fullref{ex:s3}, the \hyperref[def:alternating_group]{alternating group} \( A_3 \) consists of:
  \begin{itemize}
    \item The identity, which is a product of zero transpositions.
    \item The odd-length cycle \( \cycle{ 1, 2, 3 } = \cycle{ 1, 3 } \bincirc \cycle{ 1, 2 } \).
    \item The odd-length cycle \( \cycle{ 1, 3, 2 } = \cycle{ 1, 2 } \bincirc \cycle{ 1, 3 } \).
  \end{itemize}
\end{example}

\begin{proposition}\label{thm:group_epimorphisms_are_surjective}\mcite[exer. I.5.5]{MacLane1998}
  Every \hyperref[def:morphism_invertibility/right_cancellative]{epimorphism} in \hyperref[def:group/category]{\( \cat{Grp} \)} is \hyperref[def:function_invertibility/surjective]{surjective}.
\end{proposition}
\begin{proof}
  Let \( \varphi: G \to H \) be an epimorphism and suppose that it is not surjective. Let \( M \) be the \hyperref[def:normal_closure]{normal closure} \( \img \varphi \).

  Aiming at a contradiction, suppose that \( M \) has \hyperref[def:subgroup_index]{index} \( 2 \) in \( H \), consider the quotient map \( \pi: H \to H / M \) and the constant map \( c(h) \coloneqq M \). Then
  \begin{equation*}
    \pi \bincirc \varphi = c \bincirc \varphi.
  \end{equation*}

  Since \( \varphi \) is an epimorphism, we have \( \pi = c \). But we have deliberately taken \( \pi \) and \( c \) so that \( \pi \neq c \). The obtained contradiction shows that \( M \) must have an index greater than \( 2 \).

  Thus, the index of \( M \) in \( G \) is greater than \( 2 \). Let \( M \), \( uM \) and \( vM \) be different cosets. Define \( \sigma: H \to H \) as the \hyperref[def:symmetric_group]{permutation} on \( H \) that exchanges \( xu \) with \( xv \) for every \( x \in M \). \Fullref{thm:group_conjugation_action} implies that the following is a homomorphism:
  \begin{equation*}
    \begin{aligned}
      &\theta: H \to S(H) \\
      &\theta(h) \coloneqq \sigma^{-1} \bincirc \psi(h) \bincirc \sigma.
    \end{aligned}
  \end{equation*}

  Consider also another homomorphism,
  \begin{equation*}
    \begin{aligned}
      &\psi: H \to S(H) \\
      &\psi(h) \coloneqq (x \mapsto hx),
    \end{aligned}
  \end{equation*}
  where \( S(H) \) is the \hyperref[def:symmetric_group]{symmetric group}. This is indeed a homomorphism by \fullref{thm:cayleys_theorem}.

  Since \( \sigma \) fixes the members of \( M \) in-place, we have \( \theta(h)\restr_M = \psi(h)\restr_M \). Since \( M \) contains the image of \( \varphi \), this implies
  \begin{equation*}
    \psi \bincirc \varphi = \theta \bincirc \varphi.
  \end{equation*}

  Since \( \varphi \) is an epimorphism, we have \( \psi = \theta \). But we have deliberately constructed \( \psi \) and \( \theta \) such that \( \psi \neq \theta \). The obtained contradiction shows that \( \img \varphi \) cannot be a strict subgroup of \( G \). Therefore, \( \varphi \) must be surjective.
\end{proof}

\paragraph{Dynamical systems}

\begin{definition}\label{def:dynamical_system}\mimprovised
  Fix an object \( X \) in some \hyperref[def:concrete_category]{concrete category} and an \hyperref[rem:additive_semigroup]{additive} \hyperref[def:monoid]{monoid} (resp. \hyperref[def:group]{group}) \( T \).

  A \term{dynamical system} with \term{phase space} \( X \) and \term{time system} \( T \) is a \hyperref[def:monoid_action]{monoid action} (resp. \hyperref[def:group_action]{group action}) \( \Phi: T \times X \to X \). We refer to \( \Phi \) itself as the \term{evolution function} of the system.
\end{definition}
\begin{comments}
  \item The purpose of this definition is to make concrete any discussion of dynamical systems. Formally, it is the same concept as that of a monoid or group action.
\end{comments}

\begin{example}\label{ex:def:dynamical_system}
  We list examples of \hyperref[def:dynamical_system]{dynamical systems}:
  \begin{thmenum}
    \thmitem{ex:def:dynamical_system/rotation} Let our phase space be the \hyperref[def:euclidean_plane]{Euclidean plane}. A dynamical system then corresponds to movement of points in the plane.

    For example, in \fullref{ex:def:group_action_orbit/rotation} we discussed that rotation is a group action on the plane, and hence we can view rotation as a dynamical system whose time is given by the \hyperref[def:circle_group]{circle group}.
  \end{thmenum}
\end{example}

\begin{definition}\label{def:dynamical_system_time_classification}\mimprovised
  We say that a \hyperref[def:dynamical_system]{dynamical system} has \term{discrete time} if \( T \) is a \hyperref[def:monoid/submodel]{submonoid} of the additive group of \hyperref[def:integers]{integers} \( \BbbZ \) and \term{continuous time} if \( T \) is a submonoid of the additive group of the \hyperref[def:real_numbers]{real numbers} \( \BbbR \).
\end{definition}
\begin{comments}
  \item We choose whether the system is a monoid action or a group action based on the context. Since subgroups are submonoids, the definition holds generally.
\end{comments}

\begin{proposition}\label{thm:discrete_dynamical_system}
  \hyperref[def:dynamical_system_time_classification]{Discrete-time} \hyperref[def:dynamical_system]{dynamical systems} are particularly simple.

  The evolution function \( \Phi: T \times X \to X \) is entirely determined by \( \Phi_1 \) in the following sense: for any integer \( n \), we have
  \begin{equation}\label{eq:thm:discrete_dynamical_system}
    \Phi_n(x) = \Phi_1^n(x).
  \end{equation}

  Therefore, the entire evolution function is determined by a single endofunction on \( X \).
\end{proposition}
\begin{proof}
  We will first show \eqref{eq:thm:discrete_dynamical_system} for nonnegative integers via \hyperref[rem:induction/peano_arithmetic]{natural number induction}:
  \begin{itemize}
    \item For the base case, \eqref{eq:def:monoid_action/family/identity} implies that \( \Phi_0(x) = x = \Phi_1^0(x) \).

    \item Suppose that \eqref{eq:thm:discrete_dynamical_system} holds. Then
    \begin{equation*}
      \Phi_{n + 1}(x)
      \reloset {\eqref{eq:def:monoid_action/family/compatibility}} =
      \Phi_n(x) \bincirc \Phi_1(x)
      \reloset {\T{ind.}} =
      \Phi^n(x) \bincirc \Phi_1(x)
      =
      \Phi^{n+1}(x).
    \end{equation*}
  \end{itemize}

  For negative numbers (if the time system is a group) we can also use induction, with \eqref{eq:def:group_action/family/compatibility} instead of \eqref{eq:def:monoid_action/family/compatibility}, but with a different hypothesis: for every nonnegative integer \( n \), we have
  \begin{equation}\label{eq:thm:discrete_dynamical_system/nonpositive}
    \Phi_{-n}(x) = \Phi_1^{-n}(x)
  \end{equation}

  In order for \eqref{eq:thm:discrete_dynamical_system/nonpositive} to make sense, \( \Phi_1(x) \) must be invertible. This follows from \eqref{eq:def:group_action/family/compatibility} by noting that
  \begin{equation*}
    \id(x) = \Phi_0(x) = \Phi_1(x) \bincirc \Phi_{-1}(x),
  \end{equation*}
  implying that \( \Phi_1(x)^{-1} = \Phi_{-1}(x) \).
\end{proof}

\begin{definition}\label{def:dynamical_system_trajectory}\mimprovised
  Fix a \hyperref[def:dynamical_system]{dynamical system} with evolution function \( \Phi: T \times X \to X \).

  A \term{trajectory} starting at the \term{initial state} \( x_0 \in X \) is an \hyperref[def:cartesian_product/indexed_family]{indexed family} \( \seq{ x_t }_{t \in T} \) obtained as
  \begin{equation*}
    x_t \coloneqq \Phi_t(x_0).
  \end{equation*}
\end{definition}
\begin{comments}
  \item This notation is consistent because \eqref{eq:def:monoid_action/family/identity} implies that \( \Phi_0(x_0) = x_0 \).
\end{comments}

\begin{proposition}\label{thm:def:dynamical_system_trajectory}
  \hyperref[def:dynamical_system_trajectory]{Dynamic system trajectories} have the following basic properties:
  \begin{thmenum}
    \thmitem{thm:def:dynamical_system_trajectory/composition} For a dynamical system with evolution function \( \Phi: T \times X \to X \), the trajectory of \( x_0 \) satisfies, for every \( t \in X \),
    \begin{equation*}
      x_{t+s} = \Phi_s(x_t).
    \end{equation*}

    The order of \( s \) and \( t \) is important unless \( T \) is commutative.

    \thmitem{thm:def:dynamical_system_trajectory/discrete} For a \hyperref[def:dynamical_system_time_classification]{discrete-time} dynamical system with evolution function \( \Phi: T \times X \to X \), the trajectory of \( x_0 \) is a sequence, perhaps two-sided, such that, for any integer \( n \) in \( T \), we have
    \begin{equation*}
      x_n = \Phi_1^n(x_0).
    \end{equation*}
  \end{thmenum}
\end{proposition}
\begin{proof}
  \SubProofOf{thm:def:dynamical_system_trajectory/composition} Follows from \eqref{eq:def:monoid_action/family/compatibility}.

  \SubProofOf{thm:def:dynamical_system_trajectory/discrete} Follows from \fullref{thm:def:dynamical_system_trajectory/composition} and \fullref{thm:discrete_dynamical_system}.
\end{proof}

  \subsection{Abelian groups}\label{subsec:abelian_groups}

\paragraph{Abelian groups}

\begin{definition}\label{def:abelian_group}\mcite[119]{Knapp2016BasicAlgebra}
  \hyperref[def:binary_operation/commutative]{Commutative} \hyperref[def:group]{groups} are often called \term[bg=абелева группа (\cite[29]{КоцевСидеров2016}), ru=абелева группа (\cite[sec. 2.6]{Тыртышников2007})]{abelian groups}.
\end{definition}

\begin{proposition}\label{thm:category_of_abelian_groups}
  The \hyperref[def:category]{category} of \hyperref[def:abelian_group]{abelian groups}, which we will denote by \( \cat{Ab} \), is \hyperref[rem:category_similarity/isomorphism]{isomorphic} to the \hyperref[def:module_over_ring/category]{category of modules} over \( \BbbZ \).
\end{proposition}
\begin{comments}
  \item Similar results include \fullref{thm:commutative_monoid_is_semimodule} for commutative monoids and \fullref{thm:ring_is_integer_algebra} for rings.
\end{comments}
\begin{proof}
  Follows from the similar proposition for commutative monoids \fullref{thm:commutative_monoid_is_semimodule}.
\end{proof}

\begin{remark}\label{rem:additive_semigroup}
  General groups often arise as \hyperref[def:automorphism_group]{automorphism groups}, which are, for the most part, non-commutative, while abelian groups often arise as the main building block for \hyperref[def:ring]{rings} and \hyperref[def:module]{modules}. The same holds for \hyperref[def:semigroup]{semigroups} and \hyperref[def:monoid]{monoids}.

  To make a further distinction, if the operation is denoted by \( \cdot \) or juxtaposition, we say that the semigroup is \term{multiplicative}, and if the operation is denoted by \( + \), we say that the group is \term{additive}. This terminology usually, but not necessarily, coincides with the semigroup being \hyperref[def:binary_operation/commutative]{commutative}.

  To make things explicit, a \term{multiplicative semigroup} is any semigroup as defined in \fullref{def:semigroup}. Compare this to \term{additive semigroups}, where
  \begin{thmenum}
    \thmitem{rem:additive_semigroup/addition} The operation is denoted by \( + \) and called \term{addition}.

    \thmitem{rem:additive_semigroup/multiplication} The \hyperref[def:semigroup/exponentiation]{exponentiation operation} \( x^n \) is denoted by \( n \cdot x \) or juxtaposition and called \term{multiplication}. Thus, multiplication is not defined for two elements of the semigroup, but defined for a nonnegative positive integer and an element of the semigroup. That is,
    \begin{equation}\label{eq:rem:additive_semigroup/multiplication}
      \begin{aligned}
        &\cdot: \cdot: \BbbN \times R \to R \\
        &n \cdot x \coloneqq \begin{cases}
          0_M,           &n = 0, \T{initial condition if} M \T{is a monoid} \\
          x,             &n = 1, \T{initial condition if} M \T{is not a monoid} \\
          n \cdot x + x, &n > 1 \\
          -(n \cdot x),  &n < 0, \\
        \end{cases}
      \end{aligned}
    \end{equation}

    In the case of a \hyperref[def:binary_operation/commutative]{commutative} \hyperref[def:monoid]{monoid}, if multiplication is extended to two elements of the monoid, we instead talk about \hyperref[def:semiring]{semirings}.

    \thmitem{rem:additive_semigroup/neutral} The \hyperref[def:monoid]{neutral element} is usually denoted by \( 0 \).

    \thmitem{rem:additive_semigroup/inverse} If an \hyperref[def:monoid_inverse]{inverse} of \( x \) exists, it is denoted by \( -x \) rather than \( x^{-1} \).
  \end{thmenum}
\end{remark}

\begin{proposition}\label{thm:abelian_normal_subgroups}
  All subgroups of an abelian group are \hyperref[def:normal_subgroup]{normal}.
\end{proposition}
\begin{proof}
  Let \( G \) be abelian and \( H \) be a subgroup of \( G \). Then \( g h g^{-1} = gg^{-1} h = h \) for any \( g \in G \) and \( h \in H \), and thus \( H \) is normal.
\end{proof}

\paragraph{Integers modulo \( n \)}

\begin{definition}\label{def:group_of_integers_modulo}
  Consider the abelian group of \hyperref[def:integers]{integers} \( \BbbZ \) under addition. For every positive integer \( n \), we define the group
  \begin{equation*}
    \BbbZ_n \coloneqq \set{ 0, 1, \ldots, n - 1 }
  \end{equation*}
  with the operation
  \begin{equation*}
    x \oplus y \coloneqq \rem(x + y, n)
  \end{equation*}
  so that
  \begin{equation*}
    x \oplus y \cong x + y \pmod n.
  \end{equation*}

  The group \( \BbbZ_n \) is called the \term{group of integers modulo} \( n \).
\end{definition}
\begin{comments}
  \item This result extends to rings --- see \fullref{thm:ring_of_integers_modulo}.
\end{comments}
\begin{defproof}
  We will prove that \( \BbbZ_n \) is an abelian group.

  \SubProofOf[def:binary_operation/associative]{associativity} Addition in \( \BbbZ_n \) is associative since
  \begin{balign*}
    (x \oplus y) \oplus z
    &=
    \rem((x \oplus y) + z, n)
    = \\ &=
    \rem(\rem(x + y, n) + z, n)
    = \\ &=
    \rem(x + y - n \cdot \quot(x + y, n) + z, n)
    = \\ &=
    \rem(x + y + z, n)
    = \\ &=
    \ldots
    = \\ &=
    x \oplus (y \oplus z).
  \end{balign*}

  \SubProof{Proof that \( 0 \) is the \hyperref[def:monoid]{neutral element}} Trivial.

  \SubProof{Proof that \( n - x \) is the \hyperref[def:monoid_inverse]{inverse}} Fix \( x \in \BbbZ_n \). If \( x = 0 \), its inverse is \( 0 \). If \( x > 0 \), its inverse is \( n - x \) since \( n - x \in \BbbZ_n \) and
  \begin{equation*}
    x \oplus (n - x) = x + (n - x) - n = 0.
  \end{equation*}

  \SubProofOf[def:binary_operation/commutative]{commutativity} Follows from
  \begin{equation*}
    x \oplus y
    =
    \rem(x + y, n)
    =
    \rem(y + x, n)
    =
    y \oplus x.
  \end{equation*}
\end{defproof}

\begin{proposition}\label{thm:integers_modulo_isomorphic_to_quotient_group}
  The group \hyperref[def:group_of_integers_modulo]{\( \BbbZ_n \)} of integers modulo \( n \) is isomorphic to the quotient of \( \BbbZ \) by the subgroup \( n\BbbZ = \set{ nz \given z \in \BbbZ } \), that is,
  \begin{equation*}
    \BbbZ_n \cong \BbbZ / n\BbbZ.
  \end{equation*}
\end{proposition}
\begin{proof}
  Define the function
  \begin{align*}
    &\varphi: \BbbZ_n \to \BbbZ / n\BbbZ  \\
    &\varphi(x) \coloneqq x + n\BbbZ.
  \end{align*}

  It is a homomorphism because
  \begin{balign*}
    \varphi(x \oplus y)
    &=
    \varphi(\rem(x + y, n))
    = \\ &=
    \varphi(x + y - n \cdot \quot(x + y, n))
    = \\ &=
    x + y - n \cdot \quot(x + y, n) + n\BbbZ
    = \\ &=
    x + y + n\BbbZ
    = \\ &=
    (x + n\BbbZ) + (y + n\BbbZ)
    = \\ &=
    \varphi(x) + \varphi(y).
  \end{balign*}

  Furthermore, this shows that \( \varphi \) is also an isomorphism.
\end{proof}

\begin{example}\label{ex:lagranges_theorem_for_groups/direct_product_zn}
  \Fullref{thm:lagranges_subgroup_theorem} and \fullref{thm:integers_modulo_isomorphic_to_quotient_group} imply that, for any positive integer \( n \), \( (nm, k) \mapsto nm + k \) is a bijection between \( n \BbbZ \times \BbbZ_n \) and \( \BbbZ \). This bijection, however, is not necessarily a group isomorphism because \eqref{eq:def:semigroup/homomorphism} may not hold.

  Consider the tuples \( (nm_1, k_1) \) and \( (nm_2, k_2) \)  in \( n \BbbZ \times \BbbZ_n \). We have
  \begin{equation*}
    (nm_1, k_1) + (nm_2, k_2) = (nm_1 + nm_2, \rem(k_1 + k_2, n)).
  \end{equation*}

  Therefore, if \( k_1 + k_2 \geq n \),
  \begin{equation*}
    nm_1 + nm_2 + \rem(k_1 + k_2, n) < (nm_1 + k_1) + (nm_2 + k_2).
  \end{equation*}
\end{example}

\begin{proposition}\label{thm:cyclic_group_isomorphic_to_integers_modulo_n}
  The \hyperref[def:cyclic_group]{cyclic group} \( C_n \) is isomorphic to the group \hyperref[def:group_of_integers_modulo]{\( \BbbZ_n \)} of integers modulo \( n \).
\end{proposition}
\begin{proof}
  The homomorphism
  \begin{equation*}
    \begin{aligned}
      &\varphi: \BbbZ_n \to C_n \\
      &\varphi(k) \coloneqq a^k,
    \end{aligned}
  \end{equation*}
  and the analogous homomorphism for the infinite group, are isomorphisms.
\end{proof}

\paragraph{Grothendieck completion}

\begin{definition}\label{def:monoid_grothendieck_completion}\mcite[sec. 2.1]{LimaFilho1993}
  Let \( M \) be a \hyperref[def:binary_operation/commutative]{commutative} \hyperref[def:monoid]{monoid}. Define the \hyperref[def:first_order_congruence]{congruence} \( \cong \) on tuples of members of \( M \) to hold for \( (a, b) \) and \( (a', b') \) if there exists an element \( u \) of \( M \) such that
  \begin{equation*}
    a + b' + u = a' + b + u.
  \end{equation*}

  The \hyperref[def:first_order_quotient]{quotient} \( \overline M \coloneqq M^2 / {\cong} \) is then an \hyperref[def:abelian_group]{abelian group}, called the \term{Grothendieck completion} of \( M \).

  As a canonical embedding, we choose
  \begin{equation*}
    \begin{aligned}
      &\iota_M: M \to \overline M \\
      &\iota_M(m) \coloneqq [(m, 0)].
    \end{aligned}
  \end{equation*}
\end{definition}
\begin{comments}
  \item The congruence \( {\cong} \) on \( M^2 \) is a submonoid of \( M^4 \).
\end{comments}
\begin{defproof}
  \SubProof{Proof that \( \cong \) is a congruence}
  \SubProofOf*[def:binary_relation/reflexive]{reflexivity}
  \begin{equation*}
    (a, b) \cong (a, b) \T{if and only if} a + b + 0 = a + b + 0
  \end{equation*}

  \SubProofOf*[def:binary_relation/symmetric]{symmetry} By commutativity, if \( (a, b) \cong (a', b') \), then there exists \( u \) such that
  \begin{equation*}
    a + b' + u = a' + b + u
    =
    a' + b + u = a + b' + u,
  \end{equation*}
  hence \( (a', b') \cong (a, b) \).

  \SubProofOf*[def:binary_relation/transitive]{transitivity} Suppose that \( (a, b) \cong (a', b') \) and \( (a', b') \cong (a^\dprime, b^\dprime) \). Thus, there exist elements \( u \) and \( v \) of \( M \) such that
  \begin{align*}
    a + b' + u         &= a' + b + u, \\
    a' + b^\dprime + v &= a^\dprime + b' + v.
  \end{align*}

  Summing both sides, we obtain
  \begin{equation*}
    (a + b' + u) + (a' + b^\dprime + v) = (a' + b + u) + (a^\dprime + b' + v)
  \end{equation*}

  We reorder both sides to obtain
  \begin{equation*}
    (a + b^\dprime) + (a' + b' + u + v) = (a^\dprime + b) + (a' + b' + u + v),
  \end{equation*}
  which implies \( (a, a^\dprime) \cong (b, b^\dprime) \).

  \SubProofOf*[def:first_order_congruence/direct]{compatibility} Let \( (a, b) \cong (a', b') \) and \( (c, d) \cong (c', d') \). There exist elements \( u \) and \( v \) of \( M \) such that
  \begin{align*}
    a + b' + u &= a' + b + u, \\
    c' + d + v &= c + d' + v.
  \end{align*}

  Then
  \begin{equation*}
    (a + c) + (b' + d') + (u + v) = (a' + c') + (b + d) + (u + v),
  \end{equation*}
  therefore
  \begin{equation*}
    (a + c, b + d) \cong (a' + c', b' + d').
  \end{equation*}

  \SubProof{Proof that \( \overline M \) is a group} In order for \( \overline M \) to be a group, every element \( [(a, b)] \) must be invertible.

  In the direct product \( M^2 \), the sum of \( (a, b) \) and \( (b, a) \) is \( (a + b, b + a) \), which is equivalent via \( {\cong} \) to \( (e_M, e_M) \) because, due to commutativity in \( M \),
  \begin{equation*}
    (a + b) + e_M = e_M + (b + a).
  \end{equation*}

  Therefore, \( [(b, a)] \) is the two-sided inverse of \( [(a, b)] \).

  \SubProof{Proof that \( \overline M \) is \hyperref[def:binary_operation/commutative]{commutative}} Follows from the commutativity of \( M \).
\end{defproof}

\begin{theorem}[Grothendieck monoid completion universal property]\label{thm:grothendieck_monoid_completion_universal_property}\mcite[sec. 2.1]{LimaFilho1993}
  The \hyperref[def:monoid_grothendieck_completion]{Grothendieck completion} \( \overline{M} \) of a commutative monoid \( M \) satisfies the following \hyperref[rem:universal_mapping_property]{universal mapping property}:
  \begin{displayquote}
    For every abelian group \( G \) and every monoid homomorphism \( \varphi: M \to G \), there exists a unique \hyperref[def:group/homomorphism]{group homomorphism} \( \widetilde{\varphi}: \overline{M} \to G \) such that the following diagram commutes:
    \begin{equation}\label{eq:thm:grothendieck_monoid_completion_universal_property/diagram}
      \begin{aligned}
        \includegraphics[page=1]{output/thm__grothendieck_monoid_completion_universal_property}
      \end{aligned}
    \end{equation}
  \end{displayquote}

  Via \fullref{rem:universal_mapping_property}, \( \overline{\anon} \) becomes \hyperref[def:category_adjunction]{left adjoint} to the \hyperref[def:concrete_category]{forgetful functor}
  \begin{equation*}
    U: \cat{Ab} \to \cat{CMon}.
  \end{equation*}
\end{theorem}
\begin{comments}
  \item Compare this result to \fullref{thm:grothendieck_semiring_completion_universal_property}.
\end{comments}
\begin{proof}
  Let \( \varphi: M \to G \) be a monoid homomorphism into an abelian group \( G \). We want to define a homomorphism \( \overline{\varphi} \) such that
  \begin{equation*}
    \overline{\varphi}(\iota_M(a)) = \overline{\varphi}([(a, 0)]) = \varphi(a).
  \end{equation*}

  Each equivalence class \( C \) in \( G \) has a unique member \( a \) such that \( (a, 0) \in C \), hence the above condition is well-posed.

  Fix pairs \( (a, b) \) and \( (a', b') \) from \( M^2 \). Suppose that \( (a, b) \cong (a', b') \). Then there exists \( u \in M \) such that
  \begin{equation*}
    a + b' + u = a' + b + u.
  \end{equation*}

  An additional restriction on \( \overline{\varphi} \) is then
  \begin{equation*}
    \overline{\varphi}([(a, b)])
    =
    \overline{\varphi}([(a', b')]).
  \end{equation*}

  We need to cancel out \( u \). This uniquely determines \( \overline{\varphi} \) as
  \begin{equation*}
    \overline{\varphi}([(a, b)]) \coloneqq \varphi(a) - \varphi(b).
  \end{equation*}
\end{proof}

\paragraph{Group abelianization}

\begin{definition}\label{def:group_commutator}\mcite[313]{Knapp2016BasicAlgebra}
  Let \( G \) be an arbitrary group. We define the \term[ru=коммутант (\cite[104]{Мальцев1970})]{commutator} of the elements \( x \) and \( y \) as
  \begin{equation*}
    [x, y] \coloneqq \underbrace{xy(yx)^{-1}}_{xyx^{-1}y^{-1}}.
  \end{equation*}

  The \term{commutator subgroup} \( [G, G] \) is the subgroup \hyperref[def:group/submodel]{generated} by all the commutators in \( G \).
\end{definition}

\begin{proposition}\label{thm:commutator_subgroup_is_normal}
  The \hyperref[def:group_commutator]{commutator subgroup} of any group is \hyperref[def:normal_subgroup]{normal}.
\end{proposition}
\begin{proof}
  Let \( G \) be an arbitrary group. Fix \( g \) from \( G \) and \( n \) from \( [G, G] \). Via \fullref{thm:induction_on_generated_substructures}, we will show that the conjugate \( g n g^{-1} \) belongs to \( [G, G] \).

  \begin{itemize}
    \item If \( n = [x, y] \), then
    \begin{equation*}
      g n g^{-1}
      =
      g(xyx^{-1}y^{-1})g^{-1}
      =
      (gxg^{-1}) (gyg^{-1}) (\underbrace{gx^{-1}g^{-1}}_{(gxg^{-1})^{-1}}) (\underbrace{gy^{-1}g^{-1}}_{(gxg^{-1})^{-1}})
      =
      [gxg^{-1}, gyg^{-1}].
    \end{equation*}

    Hence, \( gng^{-1} \) is a commutator and thus also belongs to \( [G, G] \).

    \item Suppose that \( n = n_1 \cdot n_2 \) for some elements \( n_1 \) and \( n_2 \) of \( [G, G] \) and suppose that the inductive hypothesis holds for \( n_1 \) and \( n_2 \).

    Since \( n_1 \) and \( n_2 \) belong to \( N \), the inductive hypothesis implies that \( g n_1 g^{-1} \) and \( g n_2 g^{-1} \) also do. Then so does
    \begin{equation*}
      gng^{-1}
      =
      g n_1 n_2 g^{-1}
      =
      g n_1 g^{-1} \cdot g n_2 g^{-1}.
    \end{equation*}
  \end{itemize}
\end{proof}

\begin{proposition}\label{thm:quotient_is_abelian_iff_subgroup_contains_commutator}
  The \hyperref[def:group/quotient]{quotient} \( G / N \) of a \hyperref[def:group]{group} is \hyperref[def:abelian_group]{abelian} if and only if \( N \) contains the \hyperref[def:group_commutator]{commutator subgroup} \( [G, G] \) of \( G \).
\end{proposition}
\begin{proof}
  \SufficiencySubProof Suppose that \( G / N \) is abelian. Fix two elements \( x \) and \( y \) of \( G \).

  Due to commutativity, we have
  \begin{equation*}
    xyN = xN \cdot yN = yN \cdot xN = yxN.
  \end{equation*}

  Then
  \begin{equation*}
    N = xyN \cdot (yxN)^{-1} = (xy(yx)^{-1})N,
  \end{equation*}
  which implies that the commutator \( xy(yx)^{-1} \) belongs to \( N \).

  Since \( x \) and \( y \) were arbitrary, we conclude that the commutator subgroup \( [G, G] \) belongs to \( N \).

  \NecessitySubProof Suppose that \( [G, G] \subseteq N \). Fix two elements \( x \) and \( y \) of \( G \).

  We have
  \begin{equation*}
    xy = \underbrace{xy (yx)^{-1}}_{[x, y]} yx,
  \end{equation*}
  hence \( xy \) belongs to
  \begin{equation*}
    Nyx
    \reloset {\ref{thm:normal_subgroup_left_right_cosets}} =
    yxN
    =
    yN \cdot xN.
  \end{equation*}

  But
  \begin{equation*}
    xy \in xyN = xN \cdot yN.
  \end{equation*}

  The cosets \( xyN \) and \( yxN \) intersect, and, as equivalence classes, they can only be equal.

  Therefore,
  \begin{equation*}
    xN \cdot yN = yN \cdot xN,
  \end{equation*}
  which shows that \( G / N \) is abelian.
\end{proof}

\begin{definition}\label{def:group_abelianization}\mcite[example 2.1.3(b)]{Leinster2016Basic}
   We define the \term{abelianization} of a group \( G \) as its \hyperref[def:group/quotient]{quotient} \( G / [G, G] \) by its \hyperref[def:group_commutator]{commutator group} \( [G, G] \).
\end{definition}
\begin{comments}
  \item \Fullref{thm:quotient_is_abelian_iff_subgroup_contains_commutator} implies that \( G / [G, G] \) is abelian, which justifies the name.
\end{comments}
\begin{defproof}
  \Fullref{thm:commutator_subgroup_is_normal} implies that \( [G, G] \) is normal, hence we are allowed to take quotients.
\end{defproof}

\begin{theorem}[Group abelianization universal property]\label{thm:group_abelianization_universal_property}\mcite[prop. 7.4]{Knapp2016BasicAlgebra}
  The \hyperref[def:group_abelianization]{abelianization} \( G / [G, G] \) of any group satisfies the following \hyperref[rem:universal_mapping_property]{universal mapping property}:
  \begin{displayquote}
    For every abelian group \( H \), every \hyperref[def:group/homomorphism]{group homomorphism} \( \varphi: G \to H \) \hyperref[def:factors_through]{uniquely factors through} \( G / [G, G] \). More precisely, there exists a unique group homomorphism \( \widetilde{\varphi}: G / [G, G] \to H \) such that the following diagram commutes:
    \begin{equation}\label{eq:thm:group_abelianization_universal_property/diagram}
      \begin{aligned}
        \includegraphics[page=1]{output/thm__group_abelianization_universal_property}
      \end{aligned}
    \end{equation}
  \end{displayquote}

  Via \fullref{rem:universal_mapping_property}, the abelianization functor becomes \hyperref[def:category_adjunction]{left adjoint} to the \hyperref[def:concrete_category]{forgetful functor}
  \begin{equation*}
    U: \cat{Ab} \to \cat{Grp}.
  \end{equation*}
\end{theorem}
\begin{comments}
  \item This result extends to \fullref{thm:ring_abelianization_universal_property}.
\end{comments}
\begin{proof}
  Let \( H \) be an abelian group and let \( \varphi: G \to H \) be a group homomorphism.

  We want \( \overline{\varphi}: G / [G, G] \to H \) to satisfy
  \begin{equation}
    \overline{\varphi}(\pi_G(x)) = \varphi(x).
  \end{equation}

  In order to use \eqref{thm:group_abelianization_universal_property} as a definition, we must prove that \( x \cong y \pmod {[G, G]} \) implies that \( \varphi(x) = \varphi(y) \), that is, that \( xy^{-1} \in [G, G] \) implies \( \varphi(xy^{-1}) = e_H \). We will show via \fullref{thm:induction_on_generated_substructures} on \( n \in [G, G] \) that \( \varphi(n) = e_H \).
  \begin{itemize}
    \item If \( n = [x, y] \), then, by the commutativity of \( H \),
    \begin{equation*}
      \varphi(n)
      =
      \varphi(x) \varphi(y) \cdot \varphi(x)^{-1} \cdot \varphi(y)^{-1}
      =
      \parens[\Big]{ \varphi(x) \cdot \varphi(x)^{-1} } \cdot \parens[\Big]{ \varphi(y) \cdot \varphi(y)^{-1} }
      =
      e_H.
    \end{equation*}

    \item If \( n = n_1 \cdot n_2 \) for \( n_1 \) and \( n_2 \) from \( [G, G] \) and if the inductive hypothesis holds for \( n_1 \) and \( n_2 \), then
    \begin{equation*}
      \varphi(n)
      =
      \varphi(n_1) \cdot \varphi(n_2)
      =
      e_H \cdot e_H
      =
      e_H.
    \end{equation*}
  \end{itemize}

  Therefore, the function \( \overline{\varphi} \) is well-defined via \eqref{thm:group_abelianization_universal_property}.
\end{proof}


  \chapter{Ring theory}\label{ch:ring_theory}

As discussed in \fullref{con:additive_semigroup}, commutative and non-commutative groups are quite different despite having similar definitions. Rings are extensions of \hyperref[def:abelian_group]{abelian groups}, which allow multiplication with more than members of \( \BbbZ \).

For commutative rings, this second operation is often truly an extension of \fullref{def:semigroup/exponentiation} to arbitrary ring elements. For noncommutative ring, this second operation is usually given by function composition.

This section also describes \hyperref[def:module]{modules}, which are important both as generalizations of \hyperref[def:vector_space]{vector spaces} and as a tool to study rings. \hyperref[def:semiring_ideal]{Ring ideals} are instances of submodules, for example.

In an attempt to encompass the \hyperref[def:natural_numbers]{natural numbers}, \hyperref[def:lattice_ideal]{lattice ideals} and \hyperref[def:polynomial_algebra]{polynomials} and \hyperref[def:array/matrix]{matrices} over \hyperref[def:tropical_semiring]{tropical semirings}, we have chosen to use \hyperref[def:semiring]{semirings}, \hyperref[def:semimodule]{semimodules} and \hyperref[def:semiring_ideal]{semiring ideals} as fundamental notions.

\begin{figure}[!ht]
  \caption{Some important kinds of semirings}\label{fig:ring_hierarchy}
  \smallskip
  \hfill
  \begin{forest}
    [
      {\hyperref[def:semiring]{semiring}}
        [{\hyperref[def:noetherian_semiring]{noetherian}}, name=noetherian]
        [
          {\hyperref[def:zerosumfree]{zerosumfree}}
            [{\hyperref[def:distributive_lattice]{distributive lattice}}, name=lattice]
        ]
        [
          {\hyperref[def:semiring/commutative]{commutative}}, name=commutative
            [
              {\hyperref[def:integral_domain]{integral domain}}, name=domain
                [
                  {\hyperref[def:gcd_domain]{greatest common divisor domain}}
                    [
                      {\hyperref[def:factorial_domain]{factorial domain}}, name=factorial
                        [
                          {\hyperref[def:principal_ideal_domain]{principal ideal domain}}, name=pid
                            [
                              {\hyperref[def:euclidean_domain]{Euclidean domain}}
                                [{\hyperref[def:field]{field}}, name=field]
                            ]
                        ]
                    ]
                    [{\hyperref[def:bezout_domain]{Bezout domain}}, name=bezout]
                ]
            ]
        ]
        [{\hyperref[def:entire_semiring]{entire}}, name=entire]
        [
          {\hyperref[def:ring]{ring}}, name=ring
            [
              {\hyperref[def:simple_object]{simple ring}}, name=simple
                [{\hyperref[def:division_ring]{division ring}}, name=division]
            ]
        ]
    ]
    \draw[-] (ring) to (domain);
    \draw[-] (entire) to (domain);
    \draw[-] (commutative) to (lattice);
    \draw[-] (bezout) to (pid);
    \draw[-] (division) to[out=south, in=east] (field);
    \draw[-] (noetherian) to[out=south, in=west] (pid);
  \end{forest}
  \hfill\hfill
\end{figure}

  \subsection{Semirings}\label{subsec:semirings}

\paragraph{Distributivity in monoids}

We will start by defining semirings, and to do that we will first motivate distributivity.

\begin{proposition}\label{thm:monoid_distributivity}
  Fix an \hyperref[rem:additive_semigroup/multiplication]{additive} \hyperref[def:monoid]{monoid} \( (R, +, \cdot) \), where \( +: R \times R \to R \) is the monoid operation and \( \cdot: \BbbN \times R \to R \) is defined via \eqref{eq:rem:additive_semigroup/multiplication}.

  We have the following property, which we call \term{distributivity} of \( \cdot \) over \( + \):
  \begin{equation}\label{eq:thm:monoid_distributivity}
    n \cdot (x + y) = n \cdot x + n \cdot y.
  \end{equation}
\end{proposition}
\begin{proof}
  We use induction on \( n \). The case \( n = 0 \) is trivial. Suppose that \eqref{eq:thm:monoid_distributivity} holds. Then
  \begin{equation*}
    (n + 1) \cdot (x + y)
    \reloset {\eqref{eq:def:semigroup/exponentiation}} =
    n \cdot (x + y) + (x + y)
    \reloset {\T{ind.}} =
    n \cdot x + n \cdot y + (x + y)
    \reloset {\eqref{eq:def:semigroup/exponentiation}} =
    (n + 1) \cdot x + (n + 1) \cdot y.
  \end{equation*}
\end{proof}

\paragraph{Semirings}

\begin{definition}\label{def:semiring}\mcite[1]{Golan2010}
  A \term[ru=полупръстен (\cite[372]{ГеновМиховскиМоллов1991}), ru=полукольцо (\cite[4]{ВечтомовПетров2022})]{semiring} is a \hyperref[def:binary_operation/commutative]{commutative} \hyperref[def:monoid]{monoid} \( (R, +) \) with a second \hyperref[def:binary_operation/associative]{associative} operation \( \cdot: R \times R \to R \) called \term{multiplication}, which extends multiplication with natural numbers. The precise compatibility axioms are listed in \fullref{def:semiring/theory} because they fit nicely into first-order logic (unlike the \hyperref[def:semimodule/theory]{theory of semimodules}, for example, for which we prefer expressing these conditions in the metalogic).

  Although not strictly necessary, it will be convenient for us to assume that multiplication has an identity. If a multiplicative identity does not exist, we say that \( (R, +, \cdot) \) is \term{nonunital}. A canonical example of a nonunital semiring is a \hyperref[def:semiring_ideal]{semiring ideal}. We will not use nonunital semirings, but it is important to acknowledge their existence. In this context, if an identity exists, we will say that \( (R, + \cdot) \) is \term{unital}.

  We call \( (R, +) \) the \term{additive monoid} and \( (R, \cdot) \) the \term{multiplicative monoid} of the semiring. We also consider the \term{additive group} and the \term{multiplicative group} as the subsets of \hyperref[def:monoid_inverse]{invertible} elements. Both are instances of \fullref{thm:invertible_submonoid_is_group}. The multiplicative group is denoted by \( R^\times \); it is discussed further in \hyperref[def:divisibility/unit]{units}.

  Semirings have the following metamathematical properties:
  \begin{thmenum}
    \thmitem{def:semiring/theory} The \hyperref[def:first_order_theory]{first-order theory} for semirings extends the \hyperref[def:monoid/theory]{theory of monoids}.

    First, we add another \hyperref[rem:first_order_formula_conventions/infix]{infix} binary functional symbol \( \cdot \) and a constant \( 1 \). The notation for the constant is justified by \fullref{thm:semiring_characteristic_homomorphism}.

    We then extend the theory of monoids with \hyperref[def:binary_operation/commutative]{commutativity} for \( + \), \hyperref[def:binary_operation/associative]{associativity} for \( \cdot \), and the following axioms:
    \begin{thmenum}
      \thmitem{def:semiring/left_distributivity} Multiplication on the left distributes over addition:
      \begin{equation}\label{eq:def:semiring/left_distributivity}
        \xi \cdot (\eta + \zeta) \doteq \xi \cdot \eta + \xi \cdot \zeta.
      \end{equation}

      \thmitem{def:semiring/right_distributivity} Multiplication on the right also distributes over addition:
      \begin{equation}\label{eq:def:semiring/right_distributivity}
        (\xi + \eta) \cdot \zeta \doteq \xi \cdot \zeta + \eta \cdot \zeta.
      \end{equation}

      If multiplication is commutative, right distributivity follows from left distributivity.

      \thmitem{def:semiring/absorption} Zero is an absorbing element:
      \begin{equation}\label{eq:def:semiring/absorption}
        \xi \cdot 0 \doteq 0 \wedge 0 \cdot \xi \doteq 0.
      \end{equation}
    \end{thmenum}

    \thmitem{def:semiring/homomorphism} A \hyperref[def:first_order_homomorphism]{first-order homomorphism} from the semiring \( R \) to \( T \) is a function \( \varphi: R \to T \) that is a \hyperref[def:monoid/homomorphism]{monoid homomorphism} both for their additive monoids also for their multiplicative monoids.

    \thmitem{def:semiring/submodel} The set \( A \subseteq R \) is a \hyperref[def:first_order_submodel]{first-order submodel} of \( R \) if it is a both \hyperref[def:monoid/submodel]{submonoid} of the additive monoid and also of the multiplicative monoid. We call \( A \) a \term{sub-semiring}.

    As a consequence of \fullref{thm:positive_formulas_preserved_under_homomorphism}, the \hyperref[def:set_valued_map/image]{image} of a homomorphism \( \varphi: R \to T \) is a sub-semiring of \( A \).

    \thmitem{def:semiring/generated} For an arbitrary set \( A \), we denote the \hyperref[def:first_order_generated_substructure]{generated submodel} by \( \braket{ A } \).

    \thmitem{def:semiring/exponentiation} As we shall see in \fullref{thm:semiring_characteristic_homomorphism}, multiplication in \( \cdot \) extends left multiplication with natural numbers in the monoid \( (R, +) \). We do have a third operation, however --- \hyperref[def:monoid/exponentiation]{monoid exponentiation} in \( (R, \cdot) \).

    For any integer \( n \), we have the fundamental property \( 1^n = 1 \).

    \thmitem{def:semiring/category} We denote the corresponding \hyperref[def:category_of_small_first_order_models]{category of \( \mscrU \)-small models} for unital semirings by \( \ucat{SRing} \) and for non-unital semirings by \( \ucat{SRng} \).

    Both are \hyperref[def:concrete_category]{concrete} over \hyperref[def:monoid/category]{\( \ucat{Mon} \)} with the forgetful functor taking the additive monoids.

    \thmitem{def:semiring/opposite}\mcite[555]{Knapp2016BasicAlgebra} The \term{opposite semiring} of \( (R, +, \cdot) \) is the semiring \( (R, +, \star) \), with multiplication defined as \( x \star y = y \cdot x \).

    \thmitem{def:semiring/commutative} If multiplication is commutative, we call the semiring itself \enquote{commutative}. Unless multiplication corresponds to function composition, most semirings we will encounter will be commutative\footnote{
    Notable exceptions to this rule are \hyperref[def:ordinal]{ordinals}. A \hyperref[def:successor_and_limit_ordinal]{limit ordinal} \( \alpha \), regarded as the set of all smaller ordinals, is a semiring. It is not commutative, however, as shown in \fullref{ex:ordinal_addition}.}.

    We denote the category of commutative semirings by \( \ucat{CSRng} \).

    \thmitem{def:semiring/trivial} Any single-element semiring is as trivial object in \( \ucat{SRng} \) in the sense of \fullref{def:trivial_object}. It is not a zero object in \( \ucat{SRing} \), although we may refer to \( \set{ 0 } \) as \enquote{the trivial semiring}.
  \end{thmenum}
\end{definition}
\begin{comments}
  \item We can also restate the identity axiom \eqref{eq:def:monoid/theory/neutral} for the multiplicative unit \( 1 \) to highlight its connection with \eqref{eq:def:semiring/absorption}:
  \begin{equation}\label{eq:def:semiring/identity}
    \xi \cdot 1 \doteq \xi \wedge 1 \cdot \xi \doteq \xi.
  \end{equation}
\end{comments}

\begin{remark}\label{rem:semiring_etymology}
  In \fullref{def:semiring}, we require semirings to have both an additive identity and a multiplicative identity. This is not consistent with semigroups defined in \fullref{def:binary_operation/associative}, which in general do not have identities.

  \incite[ch. 3]{GondranMinoux1984Graphs} suggest using \enquote{dioid} (short for \enquote{double monoid}) instead of \enquote{semiring}. \incite[xi]{Golan2010} describes how the term \enquote{dioid} may refer to semirings with idempotent addition, i.e. a general form of the tropical semirings defined in \fullref{def:tropical_semiring}.

  We thus prefer using the term \enquote{semiring} as we have defined it in \fullref{def:semiring}.
\end{remark}

\begin{example}\label{ex:def:semiring}
  We list several examples of \hyperref[def:semiring]{semirings} that are not \hyperref[def:ring]{rings}.

  \begin{thmenum}
    \thmitem{ex:def:semiring/trivial} A semiring is \hyperref[def:semiring/trivial]{trivial} if and only if \( 0_R = 1_R \). This follows from \eqref{eq:def:semiring/absorption} and \eqref{eq:def:semiring/identity}.

    As a consequence, if \( \varphi: \set{ 0 } \to R \) is a \hyperref[def:semiring/homomorphism]{homomorphism of unital semirings}, \( R \) is trivial. This is further strengthened by \fullref{thm:ring_embedding_preserves_characteristic}.

    \thmitem{ex:def:semiring/natural_numbers} The \hyperref[def:natural_numbers]{natural numbers} are the quintessential example of a semiring. We prove in \fullref{thm:natural_number_multiplication_properties} that \( \BbbN \) is a semiring.

    \thmitem{ex:def:semiring/weak_limit_cardinal} More generally, for every \hyperref[def:successor_and_limit_cardinal/weak_limit]{weak limit cardinal}, the set of smaller cardinals is a commutative semiring under \hyperref[def:cardinal_arithmetic/addition]{cardinal addition} and \hyperref[def:cardinal_arithmetic/multiplication]{cardinal multiplication}.

    \thmitem{ex:def:semiring/lattice} We discussed in \fullref{ex:def:monoid/semilattice} that in a \hyperref[def:semilattice/bounded]{bounded lattice} \( (X, \vee, \wedge, \top, \bot) \), both \( (X, \vee, \bot) \) and \( (X, \wedge, \top) \) are monoids.

    As a consequence of \fullref{thm:bounded_lattice_absorbing}, \( \bot \) is absorbing with respect to \( \wedge \) and \( \top \) with respect to \( \vee \). Therefore, if the lattice is \hyperref[def:semilattice/distributive_lattice]{distributive}, as a consequence of \fullref{thm:bounded_lattice_absorbing}, both \( (X, \vee, \wedge) \) and \( (X, \wedge, \vee) \) are semirings.

    We refer to these semirings are the \term{join-meet} semiring and the \term{meet-join} semiring of the lattice.

    \thmitem{def:def:semiring/power} We may presume that the power set of a semiring is also a semiring under pointwise operations, analogously to the power semigroups defined in \fullref{def:power_semigroup}.

    If \( R \) is a \hyperref[def:semiring]{semiring}, then \( \pow(R) \) is a semigroup with respect to both operations, however \( \pow(R) \) is not a semiring because the operations do not \hyperref[def:semiring/left_distributivity]{distribute}.

    Consider the ring of integers and let
    \begin{align*}
      A = \set{ -1, 1 },
      &&
      B = \set{ -1 },
      &&
      C = \set{ 1 }.
    \end{align*}

    Then
    \begin{equation*}
      A(\underbrace{B + C}_{\set{ 0 }})
      =
      \set{ 0 },
    \end{equation*}
    however
    \begin{equation*}
      \underbrace{AB}_{A} + \underbrace{AC}_{A}
      =
      \set{ -2, 0, 0, 2 }.
    \end{equation*}
  \end{thmenum}
\end{example}

\begin{definition}\label{def:tropical_semiring}\mcite[example 1.12]{Golan2010}
  Consider the additive monoid \( (\BbbN, +) \) of natural numbers or, more generally, an \hyperref[def:ordered_semigroup]{ordered} \hyperref[def:binary_operation/commutative]{commutative} \hyperref[def:monoid]{monoid} \( (M, +, \leq) \).

  We adjoin a \hyperref[def:extremal_points/top_and_bottom]{top element} \( \infty \) to \( M \) that is absorbing with respect to addition. That is, \( x + \infty = \infty \) for every \( x \in M \).

  The \( \min \)-plus semiring over \( M \) is the triple \( (M \cup \set{ \infty }, \min, +) \). The \hyperref[def:extremal_points/maximum_and_minimum]{minimum} as a binary operation plays the role of semiring addition, with \( \infty \) as the absorbing element. The usual addition in \( M \) extended with \( \infty \) plays the role of semiring multiplication, with \( 0 \) as the multiplicative identity.

  We analogously define the \( \max \)-plus semiring, adjoining a \hyperref[def:extremal_points/top_and_bottom]{bottom element} \( -\infty \) rather than a top element \( \infty \).

  We will collectively call these the \term{tropical semirings}.
\end{definition}
\begin{comments}
  \item According to \incite{Pin1994}, the name \enquote{tropical semiring} is a dedication to the Brazilian Imre Simon. The paper also introduces the terms \enquote{tropical integers}, \enquote{tropical reals}, etc. \incite[3]{Golan2010} refers to the more general notion of additively-idempotent semirings. Both reserve the term \enquote{tropical semiring} for the case where \( M = \BbbN \). \incite[ch. 3]{GondranMinoux1984Graphs} does not explicitly use the word \enquote{tropical}, but instead refers to semirings as \enquote{dioids}, and the latter term sometimes refers to additively-idempotent semirings.
\end{comments}
\begin{defproof}
  We will only show \hyperref[def:semiring/left_distributivity]{distributivity}. If \( x \leq y \), since \( \leq \) is compatible with \( + \), we have
  \begin{equation*}
    \underbrace{\min\set{ x, y }}_{x} + z = x + z \leq y + z.
  \end{equation*}

  Therefore,
  \begin{equation*}
    \min\set{ x, y } + z = \min\set{ x + z , y + z }.
  \end{equation*}
\end{defproof}

\begin{proposition}\label{thm:semiring_characteristic_homomorphism}
  For every \hyperref[def:semiring]{semiring}, multiplication extends the abelian group multiplication.

  More precisely, denote the additive identity by \( 0_R \) and the multiplicative identity by \( 1_R \). Define the following semiring homomorphism:
  \begin{equation}\label{eq:thm:semiring_characteristic_homomorphism}
    \begin{aligned}
      &\iota: \BbbN \to R \\
      &\iota(n) \coloneqq \begin{cases}
        0_R                &n = 0, \\
        \iota(n - 1) + 1_R &n > 0.
      \end{cases}
    \end{aligned}
  \end{equation}

  This is the unique homomorphism from \( \BbbN \) to \( R \). Furthermore, we have the following analogue to \eqref{eq:def:semigroup/exponentiation}:
  \begin{equation}\label{eq:thm:semiring_characteristic_homomorphism/multiplication}
    \iota(n) \cdot x \coloneqq \begin{cases}
      0_R,                      &n = 0, \\
      \iota(n - 1) \cdot x + x, &n > 1.
    \end{cases}
  \end{equation}
\end{proposition}
\begin{proof}
  First note that \eqref{eq:thm:semiring_characteristic_homomorphism/multiplication} follows from \eqref{eq:thm:semiring_characteristic_homomorphism} via \hyperref[def:semiring/right_distributivity]{right distributivity}.

  It remains to show that \( \iota \) is a monoid homomorphism, and that it is unique. Clearly \( \iota(0) = 0_R \) and \( \iota(1) = 1_R \). Proving \( \iota(n + m) = \iota(n) + \iota(m) \) and \( \iota(nm) = \iota(n) \cdot \iota(m) \) can be done via nested induction.

  Now suppose \( \varphi: \BbbN \to R \) is a homomorphism. It is clear that \( \varphi(0) = 0_R \) and \( \varphi(1) = 1_R \), and also
  \begin{equation*}
    \varphi(n + 1) = \varphi(n) + \varphi(1) = \varphi(n) + 1_R.
  \end{equation*}

  This implies \( \iota = \varphi \).
\end{proof}

\begin{proposition}\label{thm:category_of_semirings_properties}
  The \hyperref[def:semiring/category]{category of unital semirings} has the following basic properties:
  \begin{thmenum}
    \thmitem{thm:category_of_semirings_properties/initial} The \hyperref[def:integers]{ring of integers} \( \BbbZ \) is an \hyperref[def:universal_objects/initial]{initial object}.

    \thmitem{thm:category_of_semirings_properties/terminal} The \hyperref[def:semiring/trivial]{trivial semiring} \( \set{ 0 } \) is a \hyperref[def:universal_objects/terminal]{terminal object}.
  \end{thmenum}
\end{proposition}
\begin{proof}
  \SubProofOf{thm:category_of_semirings_properties/initial} Follows from \fullref{thm:semiring_characteristic_homomorphism}.
  \SubProofOf{thm:category_of_semirings_properties/terminal} Follows from \fullref{ex:def:semiring/trivial}.
\end{proof}

\paragraph{Divisibility}

\begin{definition}\label{def:divisibility}\mimprovised
  Fix an arbitrary element \( x \) in a \hyperref[def:semiring]{semiring}. If there exist elements \( l \) and \( r \) such that \( x = lr \), we say that \( l \) is a \term{left divisor} of \( x \), and that \( r \) is a \term{right divisor}.

  In a \hyperref[def:semiring/commutative]{commutative semiring}, the two notions coincide, and we simply use the term \enquote{divisor}. If \( x \) is a divisor of \( y \), we say that \( x \) \term{divides} \( y \) and write \( x \mid y \). We also say that \( y \) is a \term{multiple} of \( x \). Most rings we will encounter will be commutative, but it is useful to have the weaker notions of left and right divisors.

  \begin{thmenum}
    \thmitem{def:divisibility/zero}\mcite[4]{Golan2010} Divisors of \( 0 \) are called \term{zero divisors}. Due to \hyperref[def:semiring/absorption]{absorption}, every semiring element is a zero divisor. If \( lr = 0 \) for nonzero \( l \) and \( r \), we say that \( l \) (resp. \( r \)) is a \term{nontrivial} left (resp. right) zero divisor.

    \thmitem{def:divisibility/unit} Divisors of \( 1 \) are called \term{invertible}, since they are precisely the \hyperref[def:monoid_inverse]{monoid inverses} under multiplication. They are also sometimes called \term{units}.
  \end{thmenum}
\end{definition}
\begin{comments}
  \item The set of all two-sided units of \( R \) is precisely the \hyperref[def:semiring]{multiplicative group} \( R^\times \).
  \item Divisibility is extensively studied in \fullref{subsec:integral_domains} and, more generally, in \fullref{subsec:semiring_ideals}.
\end{comments}

\begin{example}\label{ex:def:divisibility}
  We list examples of \hyperref[def:divisibility]{semiring divisibility}:
  \begin{thmenum}
    \thmitem{ex:def:divisibility/integers} The positive integers are commutative and their left and right divisors coincide. They have no \hyperref[def:divisibility/zero]{nontrivial zero divisors} as a consequence of \fullref{thm:natural_number_multiplication_properties}.

    \thmitem{ex:def:divisibility/matrix_zero_divisors} A simple example of nontrivial zero divisors is given by the \hyperref[thm:matrix_algebra]{matrix algebra} \( \BbbZ^{2 \times 2} \). We have
    \begin{equation*}
      \underbrace
      {
        \begin{pmatrix}
          0 & 1 \\
          0 & 0
        \end{pmatrix}
      }_{L}
      \underbrace
      {
        \begin{pmatrix}
          0 & 0 \\
          0 & 1
        \end{pmatrix}
      }_{R}
      =
      \begin{pmatrix}
        0 & 0 \\
        0 & 0
      \end{pmatrix}.
    \end{equation*}

    Therefore, \( L \) is a left zero divisor and \( R \) is a right zero divisor. The two do not commute because
    \begin{equation*}
      \underbrace
      {
        \begin{pmatrix}
          0 & 0 \\
          0 & 1
        \end{pmatrix}
      }_{R}
      \underbrace
      {
        \begin{pmatrix}
          0 & 1 \\
          0 & 0
        \end{pmatrix}
      }_{L}
      =
      \begin{pmatrix}
        0 & 0 \\
        1 & 0
      \end{pmatrix}.
    \end{equation*}

    Nevertheless, \( RLRL \) is the zero matrix, so \( R \) is a left zero divisor and \( L \) is a right zero divisor.

    \thmitem{ex:def:divisibility/i_sqrt5}\mcite[388]{Knapp2016BasicAlgebra} Consider the semiring \( \BbbN[\sqrt{-5}] \) obtained by \hyperref[thm:adjoining_elements_to_semiring]{adjoining} the complex number \( \sqrt{-5} \) to \( \BbbN \).

    Consider the (complex) absolute value
    \begin{equation*}
      \abs[\Big]{a + b \sqrt{-5}}
      =
      \sqrt{\parens[\Big]{ a + b \sqrt{-5} }\parens[\Big]{ a - b \sqrt{-5}}}
      =
      \sqrt{a^2 + 5b^2}.
    \end{equation*}

    It preserves products, i.e.
    \begin{equation*}
      \abs[\Big]{\parens[\Big]{a + b \sqrt{-5}} \cdot \parens[\Big]{c + d \sqrt{-5}}}
      =
      \abs[\Big]{a + b \sqrt{-5}} \cdot \abs[\Big]{c + d \sqrt{-5}}.
    \end{equation*}

    In order for \( a + b\sqrt{-5} \) to be a unit, we must have \( \abs{a + b\sqrt{-5}} = 1 \), which is equivalent to
    \begin{equation*}
      \abs[\Big]{a + b\sqrt{-5}}^2 = a^2 + 5b^2 = 1.
    \end{equation*}

    Since both \( a \) and \( b \) are integers, we have \( a \geq 1 \) and \( b \geq 1 \). Thus, \( b \) must be \( 0 \) and \( a \) must be \( 1 \).

    Therefore, the unit of \( \BbbN[\sqrt{-5}] \) is \( 1 \), just like in \( \BbbN \).
  \end{thmenum}
\end{example}

\begin{proposition}\label{thm:divisibility_and_isomorphisms}
  Suppose that \( R \) and \( S \) are \hyperref[def:semiring/commutative]{commutative semirings}.

  \begin{thmenum}
    \thmitem{thm:divisibility_and_isomorphisms/divisibility} If \( \varphi: R \to S \) is any homomorphism, then \( x \mid y \) implies \( \varphi(x) \mid \varphi(y) \). The converse holds if \( \varphi \) is an isomorphism.

    \thmitem{thm:divisibility_and_isomorphisms/zero} If \( R \) and \( S \) are isomorphic, the \hyperref[def:divisibility/zero]{zero divisors} of \( R \) are precisely the zero divisors of \( S \).

    \thmitem{thm:divisibility_and_isomorphisms/unit} If \( R \) and \( S \) are isomorphic, the \hyperref[def:divisibility/unit]{units} of \( R \) are precisely the units of \( S \).
  \end{thmenum}
\end{proposition}
\begin{proof}
  \SubProofOf{thm:divisibility_and_isomorphisms/divisibility} If \( x \mid y \), then \( xr = y \) for some \( r \in R \). Then \( \varphi(x) \varphi(r) = \varphi(y) \), hence \( \varphi(x) \mid \varphi(y) \). If \( \varphi \) is an isomorphism, the converse follows by using \( \varphi^{-1}: S \to R \).

  \SubProofOf{thm:divisibility_and_isomorphisms/zero} Follows from \fullref{thm:divisibility_and_isomorphisms/divisibility} by noting that homomorphisms preserve zeros.

  \SubProofOf{thm:divisibility_and_isomorphisms/unit} Follows from \fullref{thm:divisibility_and_isomorphisms/divisibility} by noting that homomorphisms preserve ones.
\end{proof}

\begin{proposition}\label{thm:semiring_divisibility_order}
  In a \hyperref[def:semiring/commutative]{commutative semiring}, the \hyperref[def:divisibility]{divisibility} relation is a \hyperref[def:preordered_set]{preorder}.
\end{proposition}
\begin{comments}
  \item Divisibility is not a partial order in general. To avoid the nonuniqueness problems described in \fullref{ex:preorder_nonuniqueness}, we sometimes instead prefer working with ideals.
\end{comments}
\begin{proof}
  Fix a semiring \( R \).

  \SubProofOf[def:binary_relation/reflexive]{reflexivity} Clearly every element of \( R \) divides itself.

  \SubProofOf[def:binary_relation/transitive]{transitivity} Let \( x \mid y \mid z \). Then there exist elements \( a \) and \( b \) such that \( y = a x \) and \( z = b y \). Hence, \( z = (ba) x \), and hence \( x \mid z \).
\end{proof}

\paragraph{Entire semirings}

\begin{definition}\label{def:entire_semiring}\mcite[4]{Golan2010}
  We say that a \hyperref[def:semiring]{semiring} is \term[ru=целостное (\cite[def. 3.5.1]{Винберг2014})]{entire} if it has no \hyperref[def:divisibility]{nontrivial zero divisors}, neither left nor right.
\end{definition}

\begin{example}\label{ex:def:entire_semiring}
  We list examples of (non-)\hyperref[def:entire_semiring]{entire} semirings:
  \begin{thmenum}
    \thmitem{ex:def:entire_semiring/domain} \hyperref[def:integral_domain]{Integral domains}, which is the dominating type of rings studied, are, by definition, entire.

    \thmitem{ex:def:entire_semiring/trivial} The \hyperref[def:semiring/trivial]{trivial semiring} is not entire because \( 1 \cdot 1 = 1 = 0 \).

    \thmitem{ex:def:entire_semiring/matrix} For \( n > 1 \), the \hyperref[thm:matrix_algebra]{matrix algebra} \( R^{n \times n} \) over any semiring is not entire.

    Indeed, consider the product
    \begin{equation*}
      \begin{pmatrix}
        1      & 0      & \cdots & 0 & 0 \\
        0      & 1      & \cdots & 0 & 0 \\
        \vdots & \vdots & \ddots & 0 & 0 \\
        0      & 0      & 0      & 1 & 0 \\
        0      & 0      & 0      & 0 & 0
      \end{pmatrix}
      \begin{pmatrix}
        0      & 0      & \cdots & 0 & 0 \\
        0      & 0      & \cdots & 0 & 0 \\
        \vdots & \vdots & \ddots & 0 & 0 \\
        0      & 0      & 0      & 0 & 0 \\
        0      & 0      & 0      & 0 & 1
      \end{pmatrix}
      \begin{pmatrix}
        0      & 0      & \cdots & 0 & 0 \\
        0      & 0      & \cdots & 0 & 0 \\
        \vdots & \vdots & \ddots & 0 & 0 \\
        0      & 0      & 0      & 0 & 0 \\
        0      & 0      & 0      & 0 & 0
      \end{pmatrix}
    \end{equation*}
  \end{thmenum}
\end{example}

\paragraph{Ordered semirings}

\begin{definition}\label{def:ordered_semiring}\mcite[223]{Golan2010}
  A (partially) \term{ordered semiring} is a semiring \( R \) with a \hyperref[def:partially_ordered_set]{partial order} \( \leq \) such that \( (R, +) \) is an \hyperref[def:ordered_semigroup]{ordered semigroup} and, additionally, \( x \leq y \) and \( 0 \leq z \) together imply \( xz \leq yz \) and \( zx \leq zy \).

  We will also use the following terminology for \( x \) itself:
  \begin{center}
    \begin{tabular}{l | l || l | l}
      Zero        & \( x = 0 \) & Nonzero     & \( x \neq 0 \) \\
      Positive    & \( x > 0 \) & Nonpositive & \( x \not> 0 \) \\
      Negative    & \( x < 0 \) & Nonnegative & \( x \not< 0 \)
    \end{tabular}
  \end{center}
\end{definition}

\begin{proposition}\label{thm:def:ordered_semiring}
  \hyperref[def:ordered_semiring]{Ordered semirings} have the following basic properties:
  \begin{thmenum}
    \thmitem{thm:def:ordered_semiring/sum} If two elements are simultaneously both positive or negative, then so is their sum.
    \thmitem{thm:def:ordered_semiring/strict_sum} If addition is \hyperref[def:binary_operation/cancellative]{cancellative}, for any element \( z \), the strict inequality \( x < y \) implies \( x + z < y + z \).
    \thmitem{thm:def:ordered_semiring/positive_prod} The product of positive elements is positive in \hyperref[def:entire_semiring]{entire} semirings and nonnegative in general.
    \thmitem{thm:def:ordered_semiring/alternating_prod} The product of a positive and negative element is negative in \hyperref[def:entire_semiring]{entire} semirings and nonpositive in general.
  \end{thmenum}
\end{proposition}
\begin{proof}
  \SubProofOf{thm:def:ordered_semiring/sum} If \( 0 < x \), then \( y \leq x + y \). If additionally \( 0 < y \), then \( 0 < y \leq x + y \).

  Similarly, if \( x < 0 \), then \( y + x \leq y \). If additionally \( y < 0 \), then \( y + x \leq y < 0 \).

  \SubProofOf{thm:def:ordered_semiring/strict_sum} Clearly \( x < y \) implies \( x + z \leq y + z \). If \( x + z = y + z \), we can cancel \( z \) to obtain \( x = y \), which is a contradiction. Hence, \( x + z < y + z \).

  \SubProofOf{thm:def:ordered_semiring/positive_prod} If \( 0 < x \) and \( 0 < y \), then \( 0 = 0 \cdot y \leq xy \).

  In an entire semiring we have \( 0 < xy \) because otherwise \( xy = 0 \) implies that \( x = 0 \) or \( y = 0 \).

  \SubProofOf{thm:def:ordered_semiring/alternating_prod} If \( x < 0 \) and \( 0 < y \), then \( xy \leq 0 \cdot y = 0 \). Again, in an entire ring the inequality would be strict.
\end{proof}

\begin{example}\label{ex:def:ordered_semiring}
  We list several examples of \hyperref[def:ordered_semiring]{ordered semirings}.

  \begin{thmenum}
    \thmitem{ex:def:ordered_semiring/natural_numbers} The \hyperref[def:natural_numbers]{natural numbers} form an ordered semiring as shown in \fullref{thm:natural_numbers_are_well_ordered}.

    \thmitem{ex:def:ordered_semiring/lattice} We discussed in \fullref{ex:def:semiring/lattice} that a \hyperref[def:semilattice/bounded]{bounded} \hyperref[def:semilattice/distributive_lattice]{distributive} \hyperref[def:semilattice/lattice]{lattice} \( (X, \vee, \wedge) \) can be regarded as a semiring, and so can its opposite lattice.

    We discussed in \fullref{ex:def:ordered_semigroup/semilattice} that both \( (X, \vee) \) and \( (X, \wedge) \) are \hyperref[def:ordered_semigroup]{ordered semigroups}. Both \( (X, \vee, \wedge) \) and \( (X, \wedge, \vee) \) vacuously satisfy the condition from \fullref{def:ordered_semiring}, which makes them ordered semirings.

    All elements of the ordered semiring \( (X, \vee, \wedge) \) are nonnegative and all elements of \( (X, \wedge, \vee) \) are nonpositive. With a slight abuse of terminology, we refer to them as the \term{positive} and \term{negative} semirings of the lattice.
  \end{thmenum}
\end{example}

\begin{definition}\label{def:zerosumfree}\mcite[4]{Golan2010}
  We say that an \hyperref[rem:additive_semigroup]{additive} \hyperref[def:monoid]{monoid} is \term{zerosumfree} if the \hyperref[thm:invertible_submonoid_is_group]{additive group} is trivial. That is, if \( x + y = 0 \) implies \( x = y = 0 \).
\end{definition}

\begin{example}\label{ex:def:zerosumfree}
  We list several examples of \hyperref[def:zerosumfree]{zerosumfree} semirings:
  \begin{thmenum}
    \thmitem{ex:def:zerosumfree/natural_numbers} By \fullref{thm:natural_number_addition_properties}, the natural numbers are zerosumfree.

    \thmitem{ex:def:zerosumfree/lattice} We discussed in \fullref{ex:def:semiring/lattice} that every bounded distributive lattice \( (X, \vee, \wedge) \) has two associated semirings.

    We will show that the join-meet semiring \( (X, \vee, \wedge) \) is zerosumfree. The proof only relies on \( \vee \) being idempotent. Suppose that \( x \vee y = \bot \). Then
    \begin{equation*}
      \bot
      =
      x \vee y
      \reloset {\eqref{eq:def:binary_operation/idempotent}} =
      (x \vee x) \vee y
      \reloset {\eqref{eq:def:binary_operation/associative}} =
      x \vee (x \vee y)
      =
      x \vee \bot
      \reloset {\eqref{eq:thm:binary_lattice_operations/neutral/join}} =
      x.
    \end{equation*}

    Therefore, \( x = \bot \). But \( \bot \vee y = y \), hence \( x \vee y = \bot \) implies \( y = \bot \).

    This demonstrates that the join-meet semiring is zerosumfree.

    \thmitem{ex:def:zerosumfree/tropical} The \hyperref[def:tropical_semiring]{\( \min \)-plus semiring} \( (\BbbN \cup \set{ \infty }, \min, +) \) discussed in \fullref{def:tropical_semiring} is also zerosumfree. Indeed, \( \min \) is idempotent, and the proof is analogous to the one for lattices in \fullref{ex:def:zerosumfree/lattice}.
  \end{thmenum}
\end{example}

  \section{Semimodules}\label{sec:semimodules}

\paragraph{Semimodules over semirings}

Semimodules are generalizations of monoid actions. Notation and terminology-wise, semimodules are somewhat special in that they are very much influenced by linear algebra and analysis, where vector spaces are crucial.

\begin{definition}\label{def:endomorphism_semiring}\mimprovised
  Let \( X \) be a monoid or, more generally, an object in a \hyperref[def:category]{category} that is \hyperref[def:concrete_category]{concrete} over \hyperref[def:monoid/category]{\( \cat{Mon} \)}.

  Let \( \End(X) \) be the \hyperref[def:endomorphism_monoid]{endomorphism monoid} over \( X \). These are necessarily monoid endomorphisms, however they may carry additional structure like being \hyperref[def:group/homomorphism]{group homomorphisms}, \hyperref[def:semimodule/homomorphism]{semimodule homomorphisms}, \hyperref[def:lattice/homomorphism]{(semi)lattice homomorphisms} or their \hyperref[rem:topological_first_order_structures]{continuous counterparts}.

  Define addition in \( \End(X) \) pointwise as \( [f + g](x) \coloneqq f(x) + g(x) \). Then \( \End(X) \) with pointwise addition and composition is a \hyperref[def:semiring]{semiring}, which we call the \term{endomorphism semiring} over \( X \).
\end{definition}

\begin{definition}\label{def:semimodule}\mcite[149]{Golan1999Semirings}
  Fix a \hyperref[def:semiring]{semiring} \( R \), whose elements we will call \term{scalars}, and an \hyperref[con:additive_semigroup]{additive} \hyperref[def:binary_operation/commutative]{commutative} \hyperref[def:monoid]{monoid} \( M \), whose elements we will call \term{vectors}. See \fullref{rem:vector_etymology} for a discussion of the term \enquote{vector}.

  We say that \( M \) is a \term[ru=полумодуль (\cite[99]{ВечтомовПетров2022Полукольца})]{semimodule} over \( R \) if they are compatible in any of the equivalent ways listed below. Analogously to \hyperref[def:monoid_action]{monoid actions}, if \( R \) is not commutative, we distinguish between left and right semimodules. Rather than \enquote{\( M \) is a semimodule over \( R \)}, it is often more convenient to say \enquote{\( M \) is an \( R \)-semimodule}.

  \begin{thmenum}[series=def:semimodule]
    \thmitem{def:semimodule/action}\mimprovised A left semimodule is a \hyperref[def:semiring/homomorphism]{homomorphism} from \( R \) to the \hyperref[def:endomorphism_semiring]{endomorphism semiring} \( \End(M) \). A right semimodule is a homomorphism from the \hyperref[def:semiring/opposite]{dual semiring} \( R^{-1} \) to \( \End(M) \).

    This definition is concise and natural, but unfortunately not very useful.

    \thmitem{def:semimodule/operation}\mcite[149]{Golan1999Semirings} The explicit way to define a left semimodule is via a binary operation \( \cdot: R \times M \to M \) called \term{scalar multiplication} that satisfies the following conditions:
    \begin{thmenum}
      \thmitem{def:semimodule/operation/scalar_multiplication_action} Scalar multiplication is a \hyperref[def:monoid_action]{monoid action} of the multiplicative monoid \( (R, \cdot_R) \) on \( M \). The following conditions correspond to \eqref{eq:def:monoid_action/family/identity} and \eqref{eq:def:monoid_action/family/compatibility}:
      \begin{align}
        &1_R \cdot x = x, \label{eq:def:semimodule/operation/scalar_multiplication_action/identity} \\
        &(r \cdot_R s) \cdot x = r \cdot (s \cdot x). \label{eq:def:semimodule/operation/scalar_multiplication_action/compatibility}
      \end{align}

      The second condition can be regarded as a form of associativity.

      \thmitem{def:semimodule/operation/scalar_addition_distributivity} Scalar addition distributes over scalar multiplication:
      \begin{equation}\label{eq:def:semimodule/operation/scalar_addition_distributivity}
        (r +_R s) \cdot x = r \cdot x + s \cdot x.
      \end{equation}

      \thmitem{def:semimodule/operation/vector_addition_distributivity} Vector addition distributes over scalar multiplication:
      \begin{equation}\label{eq:def:semimodule/operation/vector_addition_distributivity}
        r \cdot (x + y) = r \cdot x + r \cdot y.
      \end{equation}

      \thmitem{def:semimodule/operation/absorption} The scalar and vector zeros are compatible:
      \begin{equation}\label{eq:def:semimodule/operation/absorption}
        0_R \cdot x = 0_M = r \cdot 0_M.
      \end{equation}
    \end{thmenum}

    In practice, we use the same symbol for both scalar and vector addition, and we denote both scalar and vector multiplication via juxtaposition.
  \end{thmenum}

  Semimodules have the following metamathematical properties:
  \begin{thmenum}[resume=def:semimodule]
    \thmitem{def:semimodule/theory}\mimprovised In order to fit the heterogeneous operation \( \cdot \) into the framework of \hyperref[def:first_order_model]{first-order logic models}, we can extend the \hyperref[def:monoid/theory]{theory of monoids} by adding, for every semiring element \( r \), a unary \hyperref[def:first_order_language/fun]{functional symbol} \( m_r \).

    We have placed a restriction that the number of functional symbols must be finite, as discussed in \fullref{rem:uncountable_first_order_language}, hence this method is only available for finite semirings and is simply a conceptual sketch for infinite semirings.

    All conditions can then be reformulated via this operation. For example, \eqref{eq:def:semimodule/operation/scalar_multiplication_action/compatibility} corresponds to the axiom schema
    \begin{equation*}
      m_{rs}(\synx) = m_r(m_s(\synx)).
    \end{equation*}

    \thmitem{def:semimodule/homomorphism}\mcite[156]{Golan1999Semirings} A \hyperref[def:first_order_homomorphism]{first-order homomorphism} between two \( R \)-semimodules \( M \) and \( N \) is a function \( \varphi: M \to N \) that is a \hyperref[def:monoid/homomorphism]{monoid homomorphism} and satisfies
    \begin{equation}\label{eq:def:semimodule/homomorphism/compatibility}
      \varphi \bincirc m_r^M = m_r^N \bincirc \varphi.
    \end{equation}

    We will see in \fullref{thm:semimodule_homomorphism_iff_linear} that the homomorphisms are precisely the \hyperref[def:linear_function]{linear functions}.

    \thmitem{def:semimodule/submodel}\mcite[150]{Golan1999Semirings} The set \( A \subseteq M \) is a \hyperref[def:first_order_submodel]{submodel} of \( M \) if it is a \hyperref[def:monoid/submodel]{submonoid} of \( M \) that is closed under scalar multiplication, i.e. \( rM = m_r[M] \subseteq M \) for every \( r \in R \). We say that \( A \) is an \( R \)-\term{sub-semimodule} of \( M \).

    If \( M \) is a semimodule over some semiring extension \( T \) of \( R \), \( A \) may not be a \( T \)-sub-semimodule. For this reason, we should only use the term \enquote{sub-semimodule} (without specifying the semiring) if the underlying ring is clear from the context.

    As a consequence of \fullref{thm:positive_formulas_preserved_under_homomorphism}, the \hyperref[def:set_valued_map/image]{image} of an \( R \)-semimodule homomorphism \( \varphi: M \to N \) is an \( R \)-sub-semimodule of \( M \).

    \thmitem{def:semimodule/generated}\mimprovised For an arbitrary set \( A \), we denote the \hyperref[def:first_order_generated_substructure]{generated submodel} by \( \linspan{ A } \) and call it the \term[ru=линейная оболочка (\cite[sec. 3.2]{Тыртышников2007ЛинейнаяАлгебра})]{linear span} of \( A \).

    \Fullref{rem:span_over_different_semirings} shows how it is important to be unambiguous about over which semiring we take the span of \( A \). In case of possible ambiguity, we will use subscripts like \( \linspan_R A \).

    The linear span can be characterized via \hyperref[def:linear_combination]{linear combinations} --- see \fullref{ex:def:first_order_substructure/vector_space}.

    \thmitem{def:semimodule/category}\mimprovised For a fixed semiring \( R \), the \hyperref[def:category_of_small_first_order_models]{category of \( \mscrU \)-small models} \( \ucat{SMod}_R \) of left semimodules is \hyperref[def:concrete_category]{concrete} over \hyperref[def:monoid]{\( \ucat{Mon} \)}.

    Other notations are in use, for example \( R-\cat{Mod} \) for \( R \)-modules by \incite[158]{Aluffi2009Algebra}, that better highlight whether we are considering left or right (semi)modules. We will prefer \( \cat{SMod}_R^\oppos \) for the category of right semimodules.

    \thmitem{def:semimodule/bisemimodule}\mcite[149]{Golan1999Semirings} An \( (R, T) \)-\term{bisemimodule} is a triple \( (M, R, T) \), where \( M \) is an abelian group that is both a left \( R \)-semimodule and a right \( T \)-semimodule, and the following associativity condition holds for \( m \in M \), \( r \in R \) and \( t \in T \):
    \begin{equation}\label{eq:def:semimodule/bisemimodule/associativity}
      (r \cdot_A m) \cdot_B t = r \cdot_A (m \cdot_B t).
    \end{equation}
  \end{thmenum}
\end{definition}
\begin{defproof}
  \ImplicationSubProof{def:semimodule/action}{def:semimodule/operation} Fix a semiring homomorphism \( \varphi: R \to \End(M) \) and define the operation \( r \cdot x \coloneqq \varphi(r)(x) \).

  We will verify that all conditions from \fullref{def:semimodule/operation} hold for this operation.

  \begin{itemize}
    \item By definition, \( \varphi \) is a monoid action of \( (R, \cdot) \) on \( (M, \bincirc) \).

    \item Distributivity of scalar addition holds because \( \varphi \) is a \hyperref[def:semigroup/homomorphism]{semigroup homomorphism} from \( (R, +) \) to \( (M, +) \).

    \item Distributivity of vector addition holds because, for each \( r \), \( \varphi(r) \) is a semigroup endomorphism of \( (M, +) \).

    \item Since \( \varphi \) is a monoid homomorphism from \( (R,  +) \) to \( (R, \cdot) \), it preserves identities and hence
    \begin{equation*}
      0_R \cdot x = \varphi(0_R)(x) = [y \mapsto 0_M](x) = 0_M.
    \end{equation*}

    This proves half of \eqref{eq:def:semimodule/operation/absorption}.

    \item Since, for each \( r \), \( \varphi(r) \) is a monoid endomorphism of \( (M, +) \), we have
    \begin{equation*}
      r \cdot 0_M = \varphi(r)(0_M) = 0_M.
    \end{equation*}

    This proves the other half of \eqref{eq:def:semimodule/operation/absorption}.
  \end{itemize}

  \ImplicationSubProof{def:semimodule/operation}{def:semimodule/action} Let \( \cdot: R \times M \to M \) be an operation satisfying all conditions from \fullref{def:semimodule/operation}. Define the function \( \varphi(r) \coloneqq (x \mapsto r \cdot x) \). We will show that this is a semiring homomorphism.

  The operation preserves both identities because
  \begin{equation*}
    \varphi(0_R) = (x \mapsto 0) = 0_{\End(M)}
  \end{equation*}
  and
  \begin{equation*}
    \varphi(1_R) = (x \mapsto x) = \id_M.
  \end{equation*}

  We must also show that it preserves both binary operations. Clearly
  \begin{equation*}
    \varphi(r + s)
    =
    (x \mapsto (r + s) x)
    \reloset {\eqref{eq:def:semiring/right_distributivity}} =
    (x \mapsto r x + s x)
    =
    (x \mapsto r x) + (x \mapsto s x)
    =
    \varphi(r) + \varphi(s).
  \end{equation*}

  For multiplication, we have
  \begin{equation*}
    \varphi(rs)
    =
    (x \mapsto (rs)x)
    \reloset {\eqref{eq:def:binary_operation/associative}} =
    (x \mapsto r(sx))
    =
    \parens[\Big]{ x \mapsto \varphi(r)\parens[\Big]{ \varphi(s)(x) } }
    =
    \varphi(r) \bincirc \varphi(s).
  \end{equation*}
\end{defproof}

\begin{proposition}\label{thm:def:semimodule}
  \hyperref[def:semimodule]{Semimodules} have the following basic properties:
  \begin{thmenum}
    \thmitem{thm:def:semimodule/union} The union of a \hyperref[def:order_function/ascending]{ascending sequence} \( N_1 \subseteq N_2 \subseteq \cdots \) of \( R \)-\hyperref[def:semimodule/submodel]{sub-semimodules} of \( M \) is also an \( R \)-sub-semimodule of \( M \).
  \end{thmenum}
\end{proposition}
\begin{proof}
  \SubProofOf{thm:def:semimodule/union} Trivial.
\end{proof}

\begin{proposition}\label{thm:bisemimodule_over_submonoid}
  If \( S \) is (isomorphic to) a sub-semiring of \( R \), then \( R \) is an \( S \)-bisemimodule with scalar multiplication given by multiplication in \( R \).
\end{proposition}
\begin{comments}
  \item This result specializes to algebras over semirings in \fullref{thm:algebra_over_subring}.
\end{comments}
\begin{proof}
  We must show that \( \cdot \) satisfied the conditions in \fullref{def:semimodule/operation}.
  \begin{itemize}
    \item The identity law \eqref{eq:def:semimodule/operation/scalar_multiplication_action/identity} holds because \( 1 \) is a multiplicative identity of \( M \).
    \item The associativity-like law \eqref{eq:def:semimodule/operation/scalar_multiplication_action/compatibility} follows from associativity of multiplication.
    \item The two distributivity laws \eqref{eq:def:semimodule/operation/scalar_addition_distributivity} and \eqref{eq:def:semimodule/operation/vector_addition_distributivity} follow from left and right distributivity on \( R \).
    \item The absorption law \eqref{eq:def:semimodule/operation/absorption} follows from absorption on semirings.
  \end{itemize}

  All the above also hold for right semimodules rather than left.
\end{proof}

\begin{proposition}\label{thm:commutative_monoid_is_semimodule}
  The categories \( \hyperref[def:monoid/category]{\cat{CMon}} \) of commutative monoids and \( \hyperref[def:semimodule/category]{\cat{SMod}_\BbbN} \) of natural number semimodules are \hyperref[rem:category_similarity/isomorphism]{isomorphic}.

  More concretely, every commutative monoid \( M \) is a left semimodule over \( \BbbN \) with scalar multiplication given by \hyperref[con:additive_semigroup/multiplication]{recursively defined multiplication}
  \begin{equation}\label{eq:thm:commutative_monoid_is_semimodule/operation}
    \begin{aligned}
      &\cdot: \BbbN \times M \to M \\
      &n \cdot x \coloneqq \begin{cases}
        0_M,           &n = 0, \\
        n \cdot x + x, &n > 1.
      \end{cases}
    \end{aligned}
  \end{equation}

  Conversely, in every semimodule over \( \BbbN \), scalar multiplication matches the recursively defined multiplication.
\end{proposition}
\begin{comments}
  \item This result specializes to algebras over semirings in \fullref{thm:semiring_is_natural_number_algebra} and modules over rings in \fullref{thm:abelian_group_is_module}.
\end{comments}
\begin{proof}
  \SufficiencySubProof Let \( M \) be a commutative monoid. The operation \( \cdot: \BbbN \times M \to M \) defined in \fullref{thm:semiring_characteristic_homomorphism} satisfies the conditions in \fullref{def:semimodule/operation} as either a direct consequence of the definition or as a consequence of \fullref{thm:monoid_distributivity}.

  The homomorphisms are thus also compatible.

  \NecessitySubProof Let \( M \) be a semimodule over \( \BbbN \). We will use induction to show that \eqref{eq:thm:commutative_monoid_is_semimodule/operation} holds.
  \begin{itemize}
    \item For \( n = 0 \), this follows from the absorption law \eqref{eq:def:semimodule/operation/absorption}.
    \item If \( n \cdot x = n \cdot x + x \), then by scalar distributivity, \( (n + 1) \cdot x = n \cdot x + 1 \cdot x \). The multiplicative identity law \eqref{eq:def:semimodule/operation/scalar_multiplication_action/identity} then shows that \( 1 \cdot x = x \), which concludes our proof.
  \end{itemize}

  The homomorphisms are thus also compatible.
\end{proof}

\paragraph{Linear functions}

\begin{definition}\label{def:homogeneous_function}\mimprovised
  We say that a function \( f: M \to N \) between \( R \)-\hyperref[def:semimodule]{semimodules} is \term[ru=однородная (\cite[416]{Зорич2019АнализТом1}), en=homogeneous (\cite[def. 2.3.1]{HillePhillips1996FunctionalAnalysis})]{homogeneous} of degree \( d \) if, for every scalar \( r \) and ever vector \( x \), we have
  \begin{equation}\label{eq:def:homogeneous_function}
    f(rx) = r^d \cdot f(x).
  \end{equation}

  We will shorten \enquote{homogeneous of degree \( 1 \)} to simply \enquote{homogeneous}.
\end{definition}
\begin{comments}
  \item We generalize this definition from \incite[def. 2.3.1]{HillePhillips1996FunctionalAnalysis}, who define homogeneous functions between \hyperref[def:topological_vector_space]{topological vector spaces} without degrees, and \incite[416]{Зорич2019АнализТом1}, who defines homogeneous real-valued functions on \hyperref[def:euclidean_space]{Euclidean spaces} with positive real-valued degrees.
\end{comments}

\begin{definition}\label{def:linear_function}\mimprovised
  We say that a function between \( R \)-\hyperref[def:semimodule]{semimodules} is \term[bg=линейно преобразувание (\cite[101]{Обрешков1962ВисшаАлгебра}), ru=линейный оператор (\cite[236]{Тыртышников2007ЛинейнаяАлгебра}), en=linear transformation (\cite[def. 2.3.2]{HillePhillips1996FunctionalAnalysis})]{linear} if it is \hyperref[def:additive_function]{additive} and \hyperref[def:homogeneous_function]{homogeneous}.
\end{definition}
\begin{comments}
  \item As a consequence of \fullref{thm:semimodule_homomorphism_iff_linear}, the linear functions between semimodules are precisely the \hyperref[def:semimodule/homomorphism]{homomorphisms} between them.

  \item \incite*[def. 2.3.2]{HillePhillips1996FunctionalAnalysis} state
  \begin{displayquote}
    An additive and homogeneous transformation is said to be linear.
  \end{displayquote}

  Our underlying definitions of \enquote{additive} and \enquote{homogeneous} differ in their generality, however --- see the comments to \fullref{def:additive_function} and \fullref{def:homogeneous_function}.
\end{comments}

\begin{proposition}\label{thm:semimodule_homomorphism_iff_linear}
  A function between \( R \)-\hyperref[def:semimodule]{semimodules} is a \hyperref[def:semimodule/homomorphism]{semimodule homomorphism} if and only if it is \hyperref[def:linear_function]{linear}.
\end{proposition}
\begin{proof}
  Fix a function \( \varphi: M \to N \).

  Homogeneity is simply a restatement of \eqref{eq:def:semimodule/homomorphism/compatibility}: indeed, for any scalar \( r \) and any vector \( x \), we have
  \begin{equation}\label{eq:thm:semimodule_homomorphism_iff_linear/proof/homogeneity}
    \varphi(rx)
    =
    \varphi(m_r^M(x))
    \reloset{\eqref{eq:def:semimodule/homomorphism/compatibility}} =
     m_r^N(\varphi(x))
     =
     r \cdot \varphi(x).
  \end{equation}

  Thus, it remains to show that \( \varphi \) is a homomorphism of the additive monoids if and only if it is additive. This is false in general since monoid homomorphisms do not automatically preserve neutral elements. Due to homogeneity, however, for any element \( x \) of \( M \) we have
  \begin{equation*}
    \varphi(0_M)
    \reloset {\eqref{eq:def:semimodule/operation/absorption}} =
    \varphi(0 \cdot x)
    =
    0 \cdot \varphi(x)
    \reloset {\eqref{eq:def:semimodule/operation/absorption}} =
    0_N.
  \end{equation*}
\end{proof}

\paragraph{Semimodule direct sums}

\begin{definition}\label{def:function_support}\mcite[19]{Golan1999Semirings}
  We define the \term[bg=носител (\cite[58]{Боянов2008ЧислениМетоди}), ru=носитель (\cite[135]{КанторовичАкилов1984ФункАнализ})]{support} of a function \( f: S \to R \) from any set \( S \) to a semiring \( R \) as the set
  \begin{equation*}
    \supp(f) \coloneqq \set{ x \in S \given f(x) \neq 0_R }.
  \end{equation*}
\end{definition}

\begin{definition}\label{def:semimodule_direct_sum}\mimprovised
  We define the \term{external direct sum} or simply \term{direct sum} \( \bigoplus_{k \in \mscrK} M_k \) of the family of \( R \)-\hyperref[def:semimodule]{semimodules} \( \seq{ M_k }_{k \in \mscrK} \) of \( \Gamma \) as the subset of their \hyperref[def:first_order_direct_product]{direct product} \( \prod_{k \in \mscrK} M_k \) consisting of tuples with finite \hyperref[def:function_support]{support}.

  The sums and scalar products of tuples with finite support also have finite support, hence the direct sum is an \( R \)-\hyperref[def:semimodule/submodel]{sub-semimodule} of the direct product.

  \begin{thmenum}
    \thmitem{def:semimodule_direct_sum/inclusion} For every \( m \in \mscrK \), we define the following \term{canonical inclusion} \hyperref[def:semimodule/homomorphism]{homomorphism}:
    \begin{equation*}
      \begin{aligned}
        &\iota_m: M_m \to \bigoplus_{k \in \mscrK} M_k, \\
        &\iota_m(x) \coloneqq \begin{rcases}
          \begin{cases}
            x, &k = m \\
            0, &k \neq m
          \end{cases}
        \end{rcases}_{k \in \mscrK}
      \end{aligned}
    \end{equation*}

    \thmitem{def:semimodule_direct_sum/power} In case all summands are equal to \( M \), we denote the direct sum by \( M^{\oplus \mscrK} \).
    \thmitem{def:semimodule_direct_sum/internal} If all summands are submodels of \( M \) and if the sum \( \bigoplus_{k \in \mscrK} M_k \) is isomorphic to \( M \), we call it an \term{internal direct sum} and treat the tuple \( \seq{ x_k }_{k \in \mscrK} \) as the product \( x_{k_1} x_{k_2} \cdots x_{k_n} \) of the elements of \( M \) distinct from zero.
  \end{thmenum}
\end{definition}

\begin{proposition}\label{thm:semimodule_coproduct}
  The \hyperref[def:discrete_category_limits]{categorical coproduct} of the family \( \seq{ M_k }_{k \in \mscrK} \) in the \hyperref[def:semimodule/category]{category of \hi{commutative} semimodules} is their \hyperref[def:semimodule_direct_sum]{direct sum} \( \bigoplus_{k \in \mscrK} M_k \).
\end{proposition}
\begin{proof}
  First note that the sum \( A \coloneqq \bigoplus_{k \in \mscrK} A_k \) with the \hyperref[def:semimodule_direct_sum/inclusion]{inclusions} \( \iota \coloneqq \seq{ \iota_k }_{k \in \mscrK} \) are a \hyperref[def:category_of_cones/cocone]{cocone} for the \hyperref[def:discrete_category]{discrete} \hyperref[def:categorical_diagram]{diagram} \( \seq{ A_k }_{k \in \mscrK} \).

  Let \( (C, \alpha) \) also be a cocone. We want to define a semimodule homomorphism
  \begin{equation*}
    l_C: \bigoplus_{k \in \mscrK} A_k \to A
  \end{equation*}
  such that, for every \( m \in \mscrK \) and \( x \in M_m \),
  \begin{equation*}
    \alpha_m(x) = l_A(\iota_m(x)).
  \end{equation*}

  Thus, the value of \( l_A(c) \) on members of the inclusion \( \iota_m[M_m] \) is entirely determined by \( \alpha_m \). This suggests the definition
  \begin{equation*}
    l_C(\seq{ x_k }_{k \in \mscrK}) \coloneqq \prod_{k \in \mscrK}^n \alpha_k(x_k).
  \end{equation*}

  We discuss well-definedness of infinitary operations in direct sums in \fullref{rem:binary_operation_syntax_trees/infinite/direct_sum}.
\end{proof}

\paragraph{Free semimodules}

\begin{definition}\label{def:free_semimodule}\mimprovised
  Fix a \hyperref[def:semiring]{semiring} \( R \). We associate with every \hyperref[def:set]{plain set} \( A \) its \term{free \( R \)-semimodule} \( R^{\oplus A} \) over \( R \) defined as the set
  \begin{equation*}
    R^{\oplus A} \coloneqq \bigoplus_{x \in A} R = \set{ t: A \to R \given t \T{has finite \hyperref[def:function_support]{support}} }.
  \end{equation*}

  \begin{thmenum}
    \thmitem{def:free_semimodule/operations} \Fullref{thm:functions_over_model_form_model} implies that \( R^{\oplus A} \) is a semiring with the elementwise addition and multiplication inherited from \( R \). Scalar multiplication by \( r \in R \) can be defined as multiplication by the \hyperref[def:constant_function]{constant function} at \( r \).

    \thmitem{def:free_semimodule/inclusion} We start with the canonical inclusion
    \begin{equation*}
      \begin{aligned}
        &\iota_A: A \to R^{\oplus A}, \\
        &\iota_A(x) \coloneqq \parens[\Bigg]
          {
            y \mapsto \begin{rcases}
              \begin{cases}
                1_R, &y = x \\
                0_R, &y \neq x
              \end{cases}
            \end{rcases}
          }
      \end{aligned}
    \end{equation*}
  \end{thmenum}
\end{definition}
\begin{comments}
  \item We will use the notation \fullref{rem:free_semimodule_notation} for semimodules.

  \item In the case when \( R \) is the semiring \( \BbbN \) of natural numbers, \( \BbbN^{\oplus A} \) is the set of finite \hyperref[def:multiset]{multisets} over \( S \).
\end{comments}

\begin{remark}\label{rem:free_semimodule_notation}
  We can use the inclusion \fullref{def:free_semimodule/inclusion} of \( A \) into its \hyperref[def:free_semimodule]{free semimodule} \( R^{\oplus A} \) to introduce a simpler notation.

  For any \( t: A \to R^{\oplus A} \) and any \( a \in A \), denote by \( t_a \) the constant function taking the value \( t(a) \)\fnote{In \fullref{def:free_semimodule/operations} we defined scalar multiplication by \( t(a) \) as multiplication with \( t_a \). Having a clear distinction between functions and scalars makes the derivation of \eqref{eq:def:free_semimodule/sum} clearer}.

  Then
  \begin{equation*}
    t_a(x) \cdot \iota_A(a)(x)
    =
    \begin{cases}
      t(a) \cdot_R 0_R, &a \neq x, \\
      t(a) \cdot_R 1_R, &a = x
    \end{cases}
  \end{equation*}

  On other words, the product \( t_a(x) \cdot \iota_A(a)(x) \) is zero for every \( x \neq a \) and coincides with \( t(x) \) when \( x = a \).  As discussed in \fullref{rem:binary_operation_syntax_trees/infinite/direct_sum}, an infinite sum is well-defined if only finitely many summands are nonzero, thus
  \begin{equation*}
    t(x) = \sum_{a \in A} t_a(x) \cdot_R \iota_A(a)(x),
  \end{equation*}
  which can also be written more succinctly as
  \begin{equation*}
    t = \sum_{a \in A} t_a \cdot \iota_A(a).
  \end{equation*}

  To make the notation simpler, we can write \( a \) instead of \( \iota_A(a) \):
  \begin{equation}\label{eq:def:free_semimodule/sum}
    t = \sum_{a \in A} t_a \cdot a.
  \end{equation}

  Outside this remark, we will regard \( t_a \) as a scalar and \( a \) as a vector.
\end{remark}

\begin{theorem}[Free semimodule universal property]\label{thm:free_semimodule_universal_property}
  Fix a semiring \( R \) and a set \( A \). The \hyperref[def:free_semimodule]{free \( R \)-semimodule} \( R^{\oplus A} \) over \( R \) is the unique up to a unique isomorphism semimodule that satisfies the following \hyperref[rem:universal_mapping_property]{universal mapping property}:
  \begin{displayquote}
    For every semimodule \( M \) over \( R \) and every function \( e: A \to M \), there exists a unique \( R \)-semimodule homomorphism \( \Phi_e: R^{\oplus A} \to M \) such that the following diagram commutes:
    \begin{equation}\label{eq:thm:free_semimodule_universal_property/diagram}
      \begin{aligned}
        \includegraphics[page=1]{output/thm__free_semimodule_universal_property}
      \end{aligned}
    \end{equation}
  \end{displayquote}
\end{theorem}
\begin{comments}
  \item Via \fullref{rem:universal_mapping_property}, \( A \mapsto R^{\oplus A} \) becomes \hyperref[def:category_adjunction]{left adjoint} to the \hyperref[def:concrete_category]{forgetful functor}
  \begin{equation*}
    U: \cat{SMod}_R \to \cat{Set}.
  \end{equation*}

  \item \hyperref[con:evaluation_homomorphism]{Evaluation homomorphisms} for polynomials are actually defined via the map \( \Phi_e \) discussed here. We thus refer to \( \Phi_e \) as an evaluation homomorphism.
\end{comments}
\begin{proof}
  \ExistenceSubProof For every function \( e: A \to M \), we want
  \begin{equation*}
    \Phi_e(\iota(x)) = e(x).
  \end{equation*}

  This suggests the definition
  \begin{equation*}
    \Phi_e(\sum_{a \in A} t_a \cdot a) \coloneqq \sum_{x \in A} \sum_{a \in A} t_a \cdot e(a).
  \end{equation*}

  \UniquenessSubProof Fix a linear map \( \Psi_e: R^{\oplus A} \to M \) such that
  \begin{equation*}
    \Psi_e(\iota(x)) = e(x).
  \end{equation*}

  We will use induction on the number of nonzero entries in \( \seq{ t_x }_{x \in A} \) to show that \( \Phi_e \) and \( \Psi_e \) coincide. The base case where all entries are zero is obvious since
  \begin{equation*}
    \Psi_e(\iota(0_R)) = e(0_r) = \Phi_e(\iota(0_R)).
  \end{equation*}

  Now suppose that
  \begin{equation*}
    \Psi_e(\iota(x)) = e(x),
  \end{equation*}
  and that \( \Psi_e \) coincides with \( \Phi_e \) for every element of \( R^{\oplus A} \) with \( n - 1 \) entries. Given \( \seq{ t_x }_{x \in A} \) with \( n \) nonzero entries, fix a nonzero \( t_{x_0} \) and define
  \begin{equation}
    t'_x \coloneqq \begin{cases}
      0,   &x = x_0, \\
      t_x, &\T{otherwise.}
    \end{cases}
  \end{equation}

  Then
  \begin{balign*}
    \Psi_e(\seq{ t_x }_{x \in A})
    &=
    \sum_{x \neq x_0} t_x \cdot e(x) + t_{x_0} \cdot e(x_0)
    = \\ &=
    \Psi_e(\seq{ t'_x }_{x \in A}) + t_{x_0} \cdot e(x_0)
    \reloset {\T{ind.}} = \\ &=
    \Phi_e(\seq{ t'_x }_{x \in A})
    = \\ &=
    \Phi_e(\seq{ t_x }_{x \in A}).
  \end{balign*}

  This concludes the proof.
\end{proof}

\begin{definition}\label{def:linear_combination}\mimprovised
  Fix some list of \hyperref[def:formal_language/symbol]{symbols} \( X_1, \ldots, X_n \), which we will call \term{indeterminates}.

  A \term[ru=линейная комбинация (\cite[\S 3.2]{Тыртышников2007ЛинейнаяАлгебра}), en=linear combination (\cite[6]{Treil2017LinearAlgebra})]{linear combination} in \( X_1, \ldots, X_n \) is simply an element of the \hyperref[def:free_semimodule]{free semimodule} over the set of indeterminates.

  For every linear combination \( t \), we can adapt the notation from \fullref{rem:free_semimodule_notation} to write
  \begin{equation}\label{eq:def:linear_combination}
    \sum_{k=1}^n t_k \cdot X_k.
  \end{equation}

  We refer to the scalars \( t_k \) in \eqref{eq:def:linear_combination} as \term[ru=коэффициенты (линейной комбинации) (\cite[\S 3.2]{Тыртышников2007ЛинейнаяАлгебра})]{coefficients} and to the products \( t_k \cdot X_k \) as \term{terms}.

  \begin{thmenum}
    \thmitem{def:linear_combination/trivial} If all coefficients are zero, we say that the linear combination is \term[ru=тривиальная (линейная комбинация) (\cite[\S 3.3]{Тыртышников2007ЛинейнаяАлгебра}), en=trivial (linear combination) (\cite[8]{Treil2017LinearAlgebra})]{trivial}.

    \thmitem{def:linear_combination/sum} For every \( R \)-semimodule \( M \) and every list \( m_1, \ldots, m_n \) of vectors in \( M \), \fullref{thm:free_semimodule_universal_property} allows us to evaluate \eqref{eq:def:linear_combination} to obtain a vector in \( M \):
    \begin{equation}\label{eq:def:linear_combination/sum}
      \sum_{k=1}^n t_k \cdot m_k.
    \end{equation}

    We will refer to \eqref{eq:def:linear_combination/sum} as a \enquote{linear combination \hi{of the vectors}} \( m_1, \ldots, m_n \).
  \end{thmenum}
\end{definition}
\begin{comments}
  \item Linear combinations and their sums are thus an instance of the \hyperref[con:syntax_semantics_duality]{syntax-semantics duality}. We base our presentation on \hyperref[def:polynomial_algebra]{polynomials}, where this distinction is made clear.

  Unfortunately, it is not established practice to distinguish between a linear combination and its sum, which formally renders the discussions of coefficients and terms nonsensical. Linear combinations are assumed knowledge and even informal definitions can rarely be found. Even those authors like \incite[\S 3.2]{Тыртышников2007ЛинейнаяАлгебра} and \incite[6]{Treil2017LinearAlgebra} that give definitions do it informally, without delving into the syntax-semantics distinction.
\end{comments}

\begin{proposition}\label{thm:span_via_linear_combinations}
  For a set \( A \) in an \( R \)-\hyperref[def:semimodule]{semimodule} \( M \), the \hyperref[def:semimodule/generated]{linear span} of \( A \) equals the set of all \hyperref[def:linear_combination]{linear combinations} over \( A \).
\end{proposition}
\begin{comments}
  \item Compare this result to \fullref{thm:generators_via_polynomials} for algebras and polynomials.
\end{comments}
\begin{proof}
  \Cref{fig:thm:span_via_linear_combinations} shows an \hyperref[con:abstract_syntax_tree]{abstract syntax tree} for a given linear combination, which can be traversed and evaluated to obtain a vector in \( M \). This vector must be a member of the span of \( S \) since the latter is closed under vector addition and scalar multiplication with members of \( S \). Hence, the set \( L \) of all linear combinations over \( S \) is a subset of the span.

  Generalizing the syntax tree construction from \cref{fig:thm:span_via_linear_combinations}, we see that \( L \) satisfies \fullref{def:first_order_substructure/universe/inductive}, and is thus a submodule of \( M \). Since the span is the smallest module containing \( S \), we have \( L = \linspan S \).

  \begin{figure}[!ht]
    \hfill
    \includegraphics[page=1]{output/thm__span_via_linear_combinations}
    \hfill\hfill
    \caption{A linear combination is simply a \hyperref[con:function_superposition]{superposition} of scalar multiplication and binary addition.}
    \label{fig:thm:span_via_linear_combinations}
  \end{figure}
\end{proof}

\begin{remark}\label{rem:span_over_different_semirings}
  If \( M \) is both an \( R \)-semimodule and a \( T \)-semimodule, \fullref{thm:span_via_linear_combinations} highlights a fundamental difference between the generated \( R \)-sub-semimodule and the generated \( T \)-sub-semimodule.

  For example, the \( \BbbN \)-sub-semimodule generated by \( 2 \) is the semiring \( 2\BbbN \) of even natural numbers, while the \( \BbbR_{\geq 0} \)-sub-semimodule generated by \( 2 \) is \( \BbbR_{\geq 0} \) itself.
\end{remark}

\paragraph{Free commutative monoids}

\begin{remark}\label{rem:free_commutative_monoid_as_quotient}
  Consider the \hyperref[def:free_monoid]{free monoid} \( A^* \) over some \hyperref[def:set]{plain set} \( A \) and also the \hyperref[def:free_semimodule]{free \( \BbbN \)-semimodule} \( \BbbN^{\oplus A} \).

  Define the homomorphism
  \begin{equation*}
    \begin{aligned}
      &\varphi: A^* \to \BbbN^{\oplus A}, \\
      &\varphi(x_1 \cdots x_n) \coloneqq \seq[\Big]{ \underbrace{\sum_{k=1}^n 1_{a = x_k}}_{\mathclap{\T*{repetitions of} a \T*{among} x_1 \cdots x_n} }}_{a \in A}
    \end{aligned}
  \end{equation*}

  By \fullref{thm:homomorphism_induces_congruence}, \( \varphi \) induces a \hyperref[def:first_order_congruence]{monoid congruence} \( \cong \) on \( A^* \) where two strings are congruent if they have the same amount of each symbol from \( A \).

  This ensures that the strings \enquote{\( abc \)}, \enquote{\( bac \)}, \enquote{\( bca \)}, \enquote{\( cba \)}, \enquote{\( cab \)} and \enquote{\( acb \)} are congruent --- these are precisely all variations of \enquote{\( abc \)} that can be obtained via \hyperref[def:transposition]{transpositions} due to commutativity.

  The \hyperref[def:first_order_quotient]{quotient} \( A^* / \cong \) is isomorphic to \( \BbbN^{\oplus A} \) as a monoid. This motivates the definition of free commutative monoids in \fullref{def:free_commutative_monoid}.
\end{remark}

\begin{definition}\label{def:free_commutative_monoid}\mimprovised
  We associate with every \hyperref[def:set]{plain set} \( A \) its \term{free commutative monoid} defined as the \hyperref[def:free_semimodule]{free \( \BbbN \)-module} \( \BbbN^{\oplus A} \).
\end{definition}
\begin{comments}
  \item We regard \( \BbbN^{\oplus A} \) only as a monoid, without generally considering scalars.
  \item The relation to \hyperref[def:free_monoid]{free monoids} is given in \fullref{rem:free_commutative_monoid_as_quotient}.
\end{comments}

\begin{theorem}[Free commutative monoid universal property]\label{thm:free_commutative_monoid_universal_property}
  Given a set \( A \), the \hyperref[def:free_commutative_monoid]{free commutative monoid} \( \BbbN^{\oplus A} \) is the unique up to a unique isomorphism commutative monoid that satisfies the following \hyperref[rem:universal_mapping_property]{universal mapping property}:
  \begin{displayquote}
    For every commutative monoid \( M \) and every function \( e: A \to M \), there exists a unique monoid homomorphism \( \Phi_e: \BbbN^{\oplus A} \to M \) such that the following diagram commutes:
    \begin{equation}\label{eq:thm:free_commutative_monoid_universal_property/diagram}
      \begin{aligned}
        \includegraphics[page=1]{output/thm__free_commutative_monoid_universal_property}
      \end{aligned}
    \end{equation}
  \end{displayquote}
\end{theorem}
\begin{comments}
  \item Via \fullref{rem:universal_mapping_property}, \( A \mapsto \BbbN^{\oplus A} \) becomes \hyperref[def:category_adjunction]{left adjoint} to the \hyperref[def:concrete_category]{forgetful functor}
  \begin{equation*}
    U: \cat{CMon} \to \cat{Set}.
  \end{equation*}
\end{comments}
\begin{proof}
  Follows from \fullref{thm:free_semimodule_universal_property} by noting that, as shown in \fullref{thm:commutative_monoid_is_semimodule}, commutative monoids are semimodules over \( \BbbN \).
\end{proof}

  \subsection{Semiring ideals}\label{subsec:semiring_ideals}

\paragraph{Ideals}

When regarding \hyperref[def:semiring]{(semi)rings} as \hyperref[def:semimodule]{(semi)modules} over themselves, as per \fullref{thm:commutative_monoid_is_semimodule}, we obtain the important notion of ideals.

\begin{definition}\label{def:semiring_ideal}\mimprovised
  We say that the subset \( I \) of a semiring \( R \) is a \term[bg=(ляв) идеал (\cite[3]{КоцевСидеров2016}), ru=(левый) идеал (\cite[def. 1.1.3]{ВечтомовПетров2022})]{left ideal} (resp. \term{right ideal}) if any of the following equivalent conditions hold:
  \begin{thmenum}[series=def:semiring_ideal]
    \thmitem{def:semiring_ideal/submodule} It is a \hyperref[def:semimodule/submodel]{sub-semimodule} of \( R \) when regarded as a left (resp. right) semimodule over itself.

    \thmitem{def:semiring_ideal/direct}\mcite[70]{Golan2010} The set \( I \) is closed under addition with elements of itself, as well as left multiplication with elements of \( R \). Explicitly,
    \begin{align}
      &i \in I \T{and} j \in I \T{imply that} i + j \in I, \\
      &i \in I \T{and} r \in R \T{imply that} r \cdot i \in I \quad (\T{resp.} i \cdot r \in I).
    \end{align}
  \end{thmenum}

  If \( I \) is both a left and right ideal of \( R \), we say that it is a \term{two-sided ideal}. They are useful for \hyperref[def:ring/quotient]{quotient rings}. If multiplication is commutative, every left ideal is a right ideal and there is no distinction between the two. Generally, right ideals are left ideals in the \hyperref[def:semiring/opposite]{opposite semiring}.

  \begin{thmenum}[resume=def:semiring_ideal]
    \thmitem{def:semiring_ideal/generated} For an arbitrary subset \( A \) of \( R \), we call the (left) \hyperref[def:semimodule/submodel]{linear span} of \( A \) the left ideal \term{generated} by \( A \). Explicitly, this is the set
    \begin{equation*}
      \sum_{a \in A} A a = \set*{ \sum_{k=1}^n t_k a_k \given* n > 0 \T{and, for} k < n, t_k \in R \T{and} a_k \in A }.
    \end{equation*}

    If \( A = \set{ a_1, \ldots, a_n } \), we say that the ideal is \term{finitely generated} and write
    \begin{equation*}
      A a_1 + \cdots + A a_n.
    \end{equation*}

    For right ideals, this becomes
    \begin{equation*}
      a_1 A + \cdots + a_n A.
    \end{equation*}

    In commutative rings, we use the notation \( \braket{ A } \). In general rings, we are more explicit for the sake of avoiding possible confusion.

    \thmitem{def:semiring_ideal/principal} If an ideal is generated by a single element, we call it a \term[bg=главен идеал (\cite[4]{КоцевСидеров2016}), ru=главный идеал (\cite[def. 20.5]{ГлуховЕлизаровНечаев2015})]{principal ideal}. In a general ring, there can be left, right and two-sided principal ideals.

    \thmitem{def:semiring_ideal/product} We define the \term{product ideal} \( IJ \) of \( I \) and \( J \) as
    \begin{equation*}
      \set*{ \sum_{k=1}^n i_k j_k \given* n > 0 \T{and, for} k < n, i_k \in I \T{and} j_k \in J }.
    \end{equation*}

    This notation is unfortunately inconsistent with the pointwise product
    \begin{equation*}
      \set{ ij \mid i \in I, j \in J }
    \end{equation*}
    from \fullref{def:power_semigroup}; it is actually the ideal generated by the pointwise product.

    \begin{figure}[!ht]
      \caption{Some important kinds of ideals}\label{fig:def:semiring_ideal/hierarchy}
      \smallskip
      \hfill
      \begin{forest}
        for tree=
          {
            s sep=2.25cm
          }
        [
          {\hyperref[def:semiring_ideal]{ideal}}, name=ideal
            [{\hyperref[def:semiring_ideal/principal]{principal}}, name=principal]
            [
              {\hyperref[def:semiring_ideal/prime]{prime}}, name=prime
                [{\hyperref[def:semiring_ideal/maximal]{maximal}}, name=maximal]
            ]
            [{\hyperref[def:radical_ideal]{radical}}, name=radical]
        ]
        \draw[->, dashed] (prime) to node[below] {\hyperref[def:semiring/commutative]{commutative}} (radical);
        \draw[->, dashed] (prime) to[out=west, in=west] node[left] {\hyperref[def:principal_ideal_domain]{PID}} (maximal);
        \draw[->, dashed] (ideal) to[out=west, in=north] node[above] {\hyperref[def:principal_ideal_domain]{PID}} (principal);
      \end{forest}
      \hfill\hfill
    \end{figure}

    \thmitem{def:semiring_ideal/prime}\mcite[85]{Golan2010} If \( P \) is a proper ideal and if from \( IJ \subseteq P \) it follows that \( I \subseteq P \) or \( J \subseteq P \) (or both), we say that \( P \) is a \term[bg=прост (\cite[7]{КоцевСидеров2016}), ru=простой (\cite[14]{ВечтомовПетров2022})]{prime ideal}.

    When working with commutative semirings, \fullref{thm:def:semiring_ideal/prime_pointwise} is instead sometimes taken as the definition of a prime ideal.

    \thmitem{def:semiring_ideal/coprime}\mcite[18]{КоцевСидеров2016} If \( I + J = R \) for proper ideals \( I \) and \( J \), we say that \( I \) and \( J \) are \term[bg=взаимно прости (\cite[18]{КоцевСидеров2016})]{coprime}. Equivalently, \( I \) and \( J \) are coprime if their sum contains a unit.

    This is further refined in \hyperref[def:coprime_elements]{coprime elements}.

    \thmitem{def:semiring_ideal/maximal} A (left) \term[bg=максимален (\cite[7]{КоцевСидеров2016}), ru=максимальный (\cite[13]{ВечтомовПетров2022})]{maximal ideal} is a proper (left) ideal that is maximal with respect to set inclusion. The maximal ideals are the predecessors of \( R \) in the lattice of (left) ideals described in \fullref{thm:semiring_of_ideals}.
  \end{thmenum}
\end{definition}

\begin{proposition}\label{thm:def:semiring_ideal}
  The \hyperref[def:semiring_ideal]{left ideals} of a semiring \( R \) have the following basic properties:
  \begin{thmenum}
    \thmitem{thm:def:semiring_ideal/ideal_containing_unit} An ideal contains a \hyperref[def:divisibility/unit]{unit} if and only if it is not proper. In particular, \( R = \braket{ 1 } \).

    \thmitem{thm:def:semiring_ideal/units} A semiring element is a \hyperref[def:divisibility/unit]{unit} if and only if it does not belong to any proper ideal.

    \thmitem{thm:def:semiring_ideal/division} For two-sided ideals and \hyperref[def:divisibility]{two-sided divisors}, we have \( \braket{ x } \subseteq \braket{ y } \) if and only if \( y \mid x \) .

    More generally, we have \( Rx \subseteq Ry \) if and only if \( y \) is a right divisor of \( x \). Note how \( Rx \) and \( Ry \) are \hi{left} principal ideals but \( y \) is a \hi{right} divisor.

    \thmitem{thm:def:semiring_ideal/union} The union of an \hyperref[def:stabilizing_chain]{ascending chain} \( I_1 \subseteq I_2 \subseteq \cdots \) of ideals is again an ideal.

    \thmitem{thm:def:semiring_ideal/maximal_is_prime} Every (left or right) \hyperref[def:semiring_ideal/maximal]{maximal ideal} is \hyperref[def:semiring_ideal/prime]{prime}.

    \Fullref{thm:def:principal_ideal_domain/prime_ideal_is_maximal} is a converse that holds for \hyperref[def:principal_ideal_domain]{principal ideal domains}.

    \thmitem{thm:def:semiring_ideal/coprime_product} We have \( IJ \subseteq I \cap J \). The converse inclusion holds if \( R \) is \hyperref[def:semiring/commutative]{commutative} and if \( I \) and \( J \) are \hyperref[def:semiring_ideal/coprime]{coprime}.

    \thmitem{thm:def:semiring_ideal/product_of_principal_ideals} In a \hyperref[def:semiring/commutative]{commutative} semiring, the product of the principal ideals \( \braket{x} \) and \( \braket{y} \) is \( \braket{xy} \).

    \thmitem{thm:def:semiring_ideal/prime_pointwise} In a commutative semiring, an equivalent condition to \( P \) being \hyperref[def:semiring_ideal/prime]{prime} is that \( xy \in P \) implies \( x \in P \) or \( y \in P \) (or both).
  \end{thmenum}
\end{proposition}
\begin{proof}
  \SubProofOf{thm:def:semiring_ideal/ideal_containing_unit} We will prove that there exists a unit \( u \in I \) if and only if \( I = R \).

  \SufficiencySubProof* Let \( u \in I \) be a unit. Then \( 1 = u^{-1} u \in I \). It follows that \( 1 \cdot x = x \) for any \( x \in R \), thus \( IR = R \). But \( I \) is an ideal, hence we have that \( IR = I \). Therefore, \( I = IR = R \).

  \NecessitySubProof* If \( I = R \), then obviously \( 1 \in I \).

  An analogous proof follows for the case when \( I \) is a right ideal.

  \SubProofOf{thm:def:semiring_ideal/units}

  \SufficiencySubProof* Suppose that \( x \) is a unit and that \( x \) belongs to some proper ideal \( I \). Then \( Rx = R \), implying that \( I = R \), which is a contradiction.

  \NecessitySubProof* Suppose that \( x \) does not belong to any proper ideal. Then \( Rx \) is not a proper ideal, implying that \( R = Rx \). There exists some \( y \) such that \( yx = 1 \). Hence, \( x \) is a unit.

  \SubProofOf{thm:def:semiring_ideal/division} We will show the general statement.

  \SufficiencySubProof* Suppose that \( Rx \subseteq Ry \). Then \( x \in Ry \), and hence there exists an element \( l \) of \( R \) such that \( x = ly \). So \( y \) is a right divisor of \( x \).

  \NecessitySubProof* Suppose that \( y \) is a right divisor of \( x \). Then there exists an element \( l \) of \( R \) such that \( x = ly \). Thus, \( x \in Ry \), and hence \( Rx \subseteq Ry \).

  \SubProofOf{thm:def:semiring_ideal/union} Follows from \fullref{thm:def:semimodule/union}.

  \SubProofOf{thm:def:semiring_ideal/maximal_is_prime} Let \( M \) be a maximal (left) ideal and let \( IJ \subseteq M \). Aiming at a contradiction, suppose that both \( M \setminus I \) and \( M \setminus J \) are nonempty.

  Then there exist elements \( i \in I \), \( j \in J \), \( m_i \in M \) and \( m_j \in M \) such that
  \begin{equation*}
    i + m_i = j + m_j = 1.
  \end{equation*}

  Then
  \begin{equation*}
    1 = (i + m_i) (j + m_j) = \underbrace{ij}_{IJ} + \overbrace{m_i j}^M + \underbrace{m_j i}_M + \overbrace{m_i m_j}^M.
  \end{equation*}

  Hence,
  \begin{equation*}
    1 = (i + m_i) (j + m_j) \in M,
  \end{equation*}
  which contradicts our assumption that \( M \) is maximal.

  Therefore, \( M \setminus I \) and \( M \setminus J \) cannot both be nonempty.

  \SubProofOf{thm:def:semiring_ideal/coprime_product}
  \SufficiencySubProof* We will first show that \( IJ \subseteq I \cap J \). Let
  \begin{equation*}
    \sum_{k=1}^n x_k y_k \in IJ.
  \end{equation*}

  For each \( k \), \( x_k y_k \) belongs to both \( I \) and to \( J \). Hence, the sum over \( k \) also belongs to the intersection. Therefore,
  \begin{equation*}
    IJ \subseteq I \cap J.
  \end{equation*}

  \NecessitySubProof* Conversely, in a commutative semiring, if \( I \) and \( J \) are coprime, let \( x \in I \cap J \). There exist some \( i \in I \) and \( j \in J \) such that \( 1 = i + j \). Thus,
  \begin{equation*}
    x = x \cdot 1 = x \cdot (i + j) = \underbrace{x \cdot i}_{i \cdot x \in IJ} + \underbrace{x \cdot j}_{IJ}.
  \end{equation*}

  Generalizing on \( x \), we conclude that \( IJ \subseteq I \cap J \).

  \SubProofOf{thm:def:semiring_ideal/product_of_principal_ideals} Suppose that \( R \) is commutative.
  \SufficiencySubProof* Let \( z \in \braket{x} \braket{y} \). Then there exist elements \( x_z \) of \( \braket{x} \) and \( y_z \) of \( \braket{y} \) such that \( z = x_z y_z \), and elements \( r_x \) and \( r_y \) of \( R \) such that \( x r_x = x_z \) and \( y r_y = y_z \).

  Therefore,
  \begin{equation*}
    z = \underbrace{(x r_x) (y r_y)}_{(xy) (r_x r_y)} \in \braket{xy}.
  \end{equation*}

  \NecessitySubProof* Let \( z \in \braket{xy} \). Then there exists an element \( r \) of \( R \) such that \( z = rxy = (rx)(y) \), hence \( z \in \braket{x} \braket{y} \).

  \SubProofOf{thm:def:semiring_ideal/prime_pointwise} Suppose that \( R \) is commutative.

  \SufficiencySubProof* Let \( P \) be prime and let \( xy \in P \). Then \( \braket{xy} \subseteq P \). By \fullref{thm:def:semiring_ideal/product_of_principal_ideals}, \( \braket{x} \braket{y} \subseteq P \), and hence \( \braket{x} \subseteq P \) or \( \braket{y} \subseteq P \). Therefore, \( x \in P \) or \( y \in P \).

  \NecessitySubProof* Let \( P \) be an ideal such that \( xy \in P \) implies \( x \in P \) or \( y \in P \). Let \( IJ \subseteq P \) and suppose that there exist \( i \in I \setminus P \) and \( j \in J \setminus P \).

  Obviously \( ij \in I \). But since \( P \) is prime, it follows that \( i \in P \) or \( j \in P \).

  The obtained contradiction shows that \( I \) or \( J \) must be a subset of \( P \). Therefore, \( P \) is prime.
\end{proof}

\begin{remark}\label{rem:semiring_ideal_as_sub_semiring}
  A proper semiring ideal is a canonical example of a nonunital sub-semiring. As a consequence of \fullref{thm:def:semiring_ideal/ideal_containing_unit}, a proper ideal cannot contain the multiplicative identity \( 1 \), and is thus not a sub-semiring unless we allow sub-semirings to not contain \( 1 \).
\end{remark}

\begin{example}\label{ex:def:semiring_ideal}
  We list several examples of \hyperref[def:semiring_ideal]{semiring ideals}.
  \begin{thmenum}
    \thmitem{ex:def:semiring_ideal/not_principal} The simplest example of an ideal that is not principal is the ideal \( \braket{ 2, 3 } \) in \( \BbbN \).

    To see that it is not principal, suppose that \( \braket{ n } = \braket{ 2, 3 } \) for some natural number \( n \). This implies that there exist nonzero numbers \( a \) and \( b \) such that \( n = 2a + 3b \). Hence, \( n > 2a > a \) and \( n > 3b > b \). But then neither \( 2 \) nor \( 3 \) belongs to \( \braket{ n } \), contradicting our assumption.

    \thmitem{ex:def:semiring_ideal/prime_not_maximal} The zero ideal \( \braket{ 0 } \) in \( \BbbN \) is \hyperref[def:semiring_ideal/prime]{prime} but not \hyperref[def:semiring_ideal/maximal]{maximal}.

    Indeed, since \( \BbbN \) is \hyperref[def:entire_semiring]{entire}, \( \braket{ 0 } = \set{ 0 } \) and thus \( \braket{ 0 } \) is prime. But it is not maximal since it is contained in every other ideal.

    \thmitem{ex:def:semiring_ideal/natural_numbers_principal_ideals} For natural numbers, \( \braket{ n } = \braket{ m } \) implies \( n = m \).

    Indeed, by \fullref{thm:def:semiring_ideal/division}, \( n \mid m \) and \( m \mid n \). Thus, there exist numbers \( a \) and \( b \) such that \( n = am \) and \( m = bn \), hence \( n = abn \). Since the semiring \( \BbbN \) is \hyperref[def:entire_semiring]{entire}, we can cancel \( n \) to obtain \( ab = 1 \). Then \( a = b = 1 \), and hence \( n = m \).

    \thmitem{ex:def:semiring_ideal/prime_numbers} A natural number \( n \) is \hyperref[def:prime_number]{prime} if and only if \( \braket{n} \) is a \hyperref[def:semiring_ideal/prime]{prime ideal} in \( \BbbN \).

    Suppose that \( n \) is prime and let \( n \mid mk \). From \fullref{thm:euclids_lemma} it follows that either \( n \mid k \) or \( n \mid m \), hence \fullref{thm:def:semiring_ideal/prime_pointwise} is satisfied and \( \braket{ n } \) is a prime ideal.

    In the other direction, suppose that \( \braket{ n } \) is a prime ideal and let \( n = ab \). By \fullref{thm:def:semiring_ideal/product_of_principal_ideals}, \( \braket{ n } = \braket{ a } \braket{ b } \). Since \( \braket{ n } \) is a prime ideal, \( \braket{ a } \subseteq \braket{ n } \) or \( \braket{ b } \subseteq \braket{ n } \).

    Therefore, \( \braket{ n } = \braket{ a } \) or \( \braket{ n } = \braket{ b } \). By \fullref{ex:def:semiring_ideal/natural_numbers_principal_ideals}, \( n = a \) or \( n = b \), which in turn implies that the other is a unit.

    Therefore, \( n \) is a prime number.
    \thmitem{ex:def:semiring_ideal/matrices} Consider the matrix algebra \( \BbbZ^{2 \times 2} \). The set
    \begin{equation*}
      \set[\Bigg]
      {
        \begin{pmatrix}
          0 & a \\
          0 & b
        \end{pmatrix}
        \given*
        a, b \in \BbbZ
      }.
    \end{equation*}
    is a left ideal. It is not a right ideal, however, because
    \begin{equation*}
      \begin{pmatrix}
        1 & 0 \\
        1 & 0
      \end{pmatrix}
      \begin{pmatrix}
        0 & 1 \\
        0 & 1
      \end{pmatrix}
      =
      \begin{pmatrix}
        0 & 1 \\
        0 & 1
      \end{pmatrix}.
    \end{equation*}

    \thmitem{ex:def:semiring_ideal/polynomial_ideals} Consider the bivariate \hyperref[def:polynomial_algebra]{polynomial semiring} \( \BbbN[X, Y] \) over natural numbers. Since \( (X + Y)^2 = X^2 + 2XY + Y^2 \), we have
    \begin{equation*}
      \braket{ X^2 + 2XY + Y^2 } \subseteq \braket{ X + Y }.
    \end{equation*}

    \thmitem{ex:def:semiring_ideal/ideal_polynomials} Ideals in polynomial semirings are often studied, but we can also study polynomials in ideal semirings, i.e. polynomials over the semiring \( \mscrI \) of ideals of a semiring \( R \). For example,
    \begin{equation*}
      I^2 J + K
    \end{equation*}
    is a trivariate polynomial function over \( \mscrI \).

    \thmitem{ex:def:semiring_ideal/maximal_induced_coprime} If \( M \) is a maximal ideal and \( x \in R \setminus M \), then \( M \) and \( \braket{ x } \) are \hyperref[def:semiring_ideal/coprime]{coprime}.
  \end{thmenum}
\end{example}

\paragraph{Lattices of ideals}

\begin{proposition}\label{thm:semiring_of_ideals}
  \hfill
  \begin{thmenum}
    \thmitem{thm:semiring_of_ideals/semiring} The set \( \mscrI \) of all ideals of a semiring \( R \) is itself a semiring with the addition and multiplication defined pointwise as for \hyperref[def:power_semigroup]{power semigroups}.

    \thmitem{thm:semiring_of_ideals/order} Furthermore, \( \mscrI \) is an \hyperref[def:ordered_semiring]{ordered semiring} with respect to set inclusion.

    \thmitem{thm:semiring_of_ideals/lattice} The \hyperref[def:extremal_points/supremum_and_infimum]{supremum} of \( I \) and \( J \) is their sum \( I + J \) and their \hyperref[def:extremal_points/supremum_and_infimum]{infimum} is their intersection \( I \cap J \). With this, \( \mscrI \) becomes a lattice.
  \end{thmenum}
\end{proposition}
\begin{proof}
  \SubProofOf{thm:semiring_of_ideals/semiring} Associativity and commutativity in \( \mscrI \) are inherited from \( R \), as well as both left and right distributivity. Distributivity ensures that \( I + J \) is an ideal, while associativity of multiplication ensures that \( IJ \) is an ideal.

  \SubProofOf{thm:semiring_of_ideals/order} We must now prove that the partial order \( \subseteq \) is compatible with addition and multiplication. Suppose that \( I \subseteq J \) and let \( H \) be any ideal in \( \mscrI \). Then
  \begin{equation*}
    I + H \subseteq J + H
  \end{equation*}
  and
  \begin{equation*}
    IH \subseteq JH.
  \end{equation*}

  Therefore, \( \mscrI \) is an ordered semiring.

  \SubProofOf{thm:semiring_of_ideals/lattice} Since \( 0 \in I \), obviously \( I \subseteq I + J \), and thus \( I + J \) is an upper bound of \( I \) and \( J \). If \( H \) is any other upper bound, it must contain the sums of all elements of \( I \) and all elements of \( J \), hence \( I + J \subseteq H \). Therefore, \( \sup\set{I, J} = I + J \).

  For \( I \cap J \), it is an infimum of \( I \) and \( J \) as a consequence of \fullref{thm:boolean_algebra_of_subsets/meet}.
\end{proof}

\begin{theorem}[Maximal ideal theorem]\label{thm:maximal_ideal_theorem}\mcite[prop. 6.59]{Golan2010}
  Every proper \hyperref[def:semiring_ideal]{semiring ideal} is contained in a \hyperref[def:semiring_ideal/maximal]{maximal ideal}.
\end{theorem}
\begin{comments}
  \item Within \hyperref[def:zfc]{\logic{ZF}}, this theorem is equivalent to the \hyperref[def:zfc/choice]{axiom of choice} --- see \fullref{thm:axiom_of_choice_equivalences/maximal_ideal}.

  \item \Fullref{thm:def:semiring_ideal/maximal_is_prime} implies that every maximal ideal is \hyperref[def:semiring_ideal/prime]{prime}, hence every proper ideal is contained in some prime ideal. For this reason, this result is referred to as the \enquote{prime ideal theorem} by \incite{Johnstone1983}.
\end{comments}
\begin{proof}
  We will discuss equivalence with \fullref{thm:zorns_lemma}.

  \ImplicationSubProof[thm:zorns_lemma]{Zorn's lemma}[thm:maximal_ideal_theorem]{maximal ideal theorem} Let \( I \) be a proper ideal in the semiring \( R \). Denote by \( \mscrH \) the set of all proper ideals in \( R \) that contain \( I \). The union of every chain in \( \mscrH \) is again an ideal, and by Zorn's lemma, \( \mscrH \) has a maximal element. More precisely, there exists a maximal ideal in \( \mscrH \) that contains \( I \).

  \ImplicationSubProof[thm:maximal_ideal_theorem]{maximal ideal theorem}[thm:zorns_lemma]{Zorn's lemma} In \cite{Hodges1979}, Hodges proves that the statement \enquote{every \hyperref[def:unique_factorization_domain]{unique factorization domain} has a maximal ideal} implies Zorn's lemma. We have an even stronger antecedent.
\end{proof}

\paragraph{Radical ideals}

\begin{definition}\label{def:radical_of_ideal}\mcite[15]{КоцевСидеров2016}
  In a \hyperref[def:ring/commutative]{commutative (semi)ring}, we define the \term[bg=радикал (\cite[15]{КоцевСидеров2016})]{radical} of an ideal \( I \) as the ideal
  \begin{equation}\label{eq:def:radical_ideal}
    \sqrt I \coloneqq \set{ x \in R \given \qexists {n \geq 1} x^n \in I }.
  \end{equation}

  It is a \hyperref[def:moore_closure_operator]{Moore closure operator} for ideals.
\end{definition}
\begin{defproof}
  \SubProof{Proof that \( \sqrt I \) is an ideal}

  \SubProof*{Proof of multiplicative closure} If \( x \) belongs to \( \sqrt I \), then there exists a power \( x^n \) that belongs to \( I \). Let \( r \) be any member of \( R \). Then \( rx = rx^n \in I \) since \( I \) is closed with respect to multiplication.

  \SubProof*{Proof of additive closure} If \( x \) and \( y \) both belong to \( \sqrt I \), then there exist powers \( n \) and \( m \) such that \( x^n \in I \) and \( y^m \in I \). Let \( u \coloneqq n + m - 1 \). By \fullref{thm:binomial_theorem},
  \begin{equation*}
    (x + y)^u = \sum_{k=0}^u \binom u k x^k y^{u-k}.
  \end{equation*}

  \begin{itemize}
    \item If \( k < n \), then \( x^k y^{u-k} = (x^k y^{n - k - 1}) y^m \) and, since \( y^m \in I \), we have \( x^k y^{u-k} \in I \).
    \item If \( k \geq n \), then \( x^k y^{u-k} = x^n (x^{k-n} y^{u-k}) \) and, since \( x^n \in I \), we have \( x^k y^{u-k} \in I \).
  \end{itemize}

  Since \( I \) is closed under addition, \( (x + y)^u \in I \).

  \SubProof{Proof that \( \sqrt I \) is a closure of \( I \)}

  \SubProofOf*[def:extensive_function]{extensiveness} Clearly \( I \subseteq \sqrt I \).

  \SubProofOf*[def:binary_operation/idempotent]{idempotence} If \( x \in \sqrt {\sqrt I} \), then there exists some positive integer \( n \) such that \( x^n \in \sqrt I \). Similarly, there exists some \( m \) such that \( (x^n)^m \in I \). Then \( x^{nm} \in I \), hence \( x \in \sqrt I \).

  \SubProofOf*[def:order_homomorphism/increasing]{monotonicity} If \( I \subseteq J \) and \( x^n \in I \), then \( x^n \in J \). Therefore, \( \sqrt I \subseteq \sqrt J \).
\end{defproof}

\begin{definition}\label{def:radical_ideal}\mimprovised
  In a \hyperref[def:ring/commutative]{commutative (semi)ring}, we say that an ideal \( I \) is \term{radical} if any of the following equivalent conditions hold:
  \begin{thmenum}
    \thmitem{def:radical_ideal/direct} If \( x^n \in I \) for some positive integer \( n \), then \( x \in I \).
    \thmitem{def:radical_ideal/radical} The \hyperref[def:radical_of_ideal]{radical} \( \sqrt I \) coincides with \( I \).
  \end{thmenum}
\end{definition}

\begin{proposition}\label{thm:prime_ideal_is_radical}
  In a \hyperref[def:ring/commutative]{commutative (semi)ring}, every \hyperref[def:semiring_ideal/prime]{prime ideal} is \hyperref[def:radical_ideal]{radical}.
\end{proposition}
\begin{proof}
  Let \( P \) be a prime ideal. Suppose that, for some positive integer \( n \) and some semiring element \( x \), we have \( x^n \in P \).

  We will use induction on \( n \) to show that \( x \) is in \( P \).
  \begin{itemize}
    \item The case \( n = 1 \) is obvious.
    \item Suppose that \( x^n \in P \) implies \( x \in P \) and suppose that \( x^{n+1} \in P \)

    Since \( P \) is prime and \( x^{n+1} = x \cdot x^n \), either \( x \in P \) or \( x^n \in P \). By the inductive hypothesis, the latter also implies that \( x \in P \).
  \end{itemize}

  Generalizing on \( x \) we conclude that \( P \) is a radical ideal.
\end{proof}

\begin{proposition}\label{thm:radical_ideal_is_intersection}
  The \hyperref[def:radical_of_ideal]{radical} \( \sqrt I \) of an ideal \( I \) in a commutative semiring coincides with the intersection of all \hyperref[def:semiring_ideal/prime]{prime ideals} that contain \( I \).
\end{proposition}
\begin{proof}
  \SubProof{Constructing the family of prime ideals} \Fullref{thm:maximal_ideal_theorem} implies that at least one maximal ideal contains \( I \), and \fullref{thm:def:semiring_ideal/maximal_is_prime} implies that this maximal ideal is prime. Then the family \( \seq{ P_k }_{k \in \mscrK} \) of all prime ideals that contain \( I \) is nonempty. We will show that \( I \) coincides with the intersection of this family.

  \SubProof{Proof that the intersection contains \( \sqrt I \)} Let \( x \) be in \( \sqrt I \). Then there exists some positive integer \( n \) such that \( x^n \in I \). Then \( x^n \) also belongs to every prime ideal \( P_k \) containing \( I \).

  \Fullref{thm:prime_ideal_is_radical} implies that, for every \( k \in \mscrK \), the prime ideal \( P_k \) is radical and thus \( x \) belongs to \( P_k \). Hence, \( x \) also belongs to their intersection.

  Generalizing on \( x \), we conclude that \( \sqrt I \subseteq \bigcap_{k \in \mscrK} P_k \).

  \SubProof{Proof that \( \sqrt I \) contains the intersection} Let \( x \) be in the intersection, that is, \( x \in P_k \) for every \( k \in \mscrK \). Aiming at a contradiction, suppose that \( x \) is not in \( \sqrt I \).

  Consider the following family of ideals:
  \begin{equation*}
    \mscrH \coloneqq \set{ J \T{is an ideal containing} I \given \qforall {n \geq 1} x^n \not\in J }.
  \end{equation*}

  It is nonempty because \( \sqrt I \in \mscrH \).

  For every chain of ideals in \( \mscrH \), their join is also an ideal in \( \mscrH \). By \fullref{thm:zorns_lemma}, \( \mscrH \) has a maximal element \( H \).

  \SubProof*{Proof that \( H \) is prime} We will show that, if neither \( a \) nor \( b \) belong to \( H \), neither does \( ab \).

  Clearly \( H \) is a strict subset of both \( H + \braket{ a } \) and \( H + \braket{ b } \), hence the latter two ideals are not in \( \mscrH \) --- otherwise this would contradict the maximality of \( H \). Then there exist some positive integers \( n \) and \( m \) such that \( x^n \in H + \braket{ a } \) and \( x^m \in H + \braket{ b } \).

  Then \( x^{n+m} \) is in
  \begin{equation*}
    (H + \braket{ a })(H + \braket{ b })
    \reloset {\ref{thm:semiring_of_ideals/semiring}} =
    H \cdot H + H \cdot \braket{ a } + H \cdot \braket{ b } + \braket{ a } \cdot \braket{ b }
    \reloset {\ref{thm:def:semiring_ideal/product_of_principal_ideals}} \subseteq
    H + \braket{ ab }.
  \end{equation*}

  Thus, \( x^{n+m} \) is in \( H + \braket{ ab } \) but not in \( H \). Then \( ab \) is also not in \( H \) --- if \( ab \) was in \( H \), then \( H \) would coincide with \( H + \braket{ ab } \).

  Therefore, we have shown that \( H \) is prime.

  \SubProof*{Proof of contradiction} We have shown that \( H \) is a prime ideal containing \( I \) but not \( x \), and we have assumed that every prime ideal containing \( I \) must contain \( x \). The obtained contradiction shows the following chain of conclusions:
  \begin{itemize}
    \item The ideal \( H \) does not exist.
    \item The family \( \mscrH \) must be empty.
    \item \( \sqrt I \) is not in \( \mscrH \).
    \item \( x \) belongs to \( \sqrt I \).
  \end{itemize}

  Generalizing on \( x \), we conclude that \( \bigcap_{k \in \mscrK} P_k \subseteq \sqrt I \).
\end{proof}

\begin{definition}\label{def:nilradical}\mcite[13]{КоцевСидеров2016}
  We define the \term{nilradical} of a \hyperref[def:ring/commutative]{commutative (semi)ring} as the \hyperref[def:radical_of_ideal]{radical} \( \sqrt {\braket{ 0 }} \) of the zero ideal, that is, the set of all elements \( x \) such that \( x^n = 0 \) for some positive integer \( n \).

  We call such elements \term{nilpotent}.
\end{definition}

\begin{example}\label{ex:def:radical_ideal}
  We list examples of \hyperref[def:radical_ideal]{radical ideal}.

  \begin{thmenum}
    \thmitem{ex:def:radical_ideal/natural_numbers} Suppose that the natural number \( m \) has a prime decomposition
    \begin{equation*}
      m = p_1^{k_1} \cdots p_n^{k_n}.
    \end{equation*}

    Then the radical of its principal ideal is
    \begin{equation*}
      \sqrt{ \braket{ p_1^{k_1} \cdots p_n^{k_n} } } = \braket{ p_1 \cdots p_n }.
    \end{equation*}

    Indeed, note that
    \begin{equation*}
      m = p_1^{k_1} \cdots p_n^{k_n} = (p_1^{k_1-1} \cdots p_n^{k_1-1}) (p_1 \cdots p_n).
    \end{equation*}

    Particular examples of this are
    \begin{itemize}
      \item The ideal \( { \braket{ 6 } } \) is radical. It is not prime since \( 2 \cdot 3 \) is in \( \braket{ 6 } \), but neither \( 2 \) nor \( 3 \) are.

      \item For any prime \( p \), \( \braket{ p } \) is radical

      \item For any prime power \( p^n \), \( \sqrt{\braket{ p^n }} = \braket{ p } \). For example, \( \sqrt{\braket{ 4 }} = \braket{ 2 } \).
    \end{itemize}

    \thmitem{ex:def:radical_ideal/matrices} Consider the matrix ring \( \BbbN^{2 \times 2} \). The matrix
    \begin{equation*}
      A \coloneqq
      \begin{pmatrix}
        0 & 1 \\
        0 & 0 \\
      \end{pmatrix}
    \end{equation*}
    is a \hyperref[def:radical_ideal]{nilpotent element} of \( \BbbN^{n \times n} \) because \( A^2 \) is the zero matrix.

    The transposed matrix \( A^T \) is also nilpotent. Their linear combinations are also nilpotent.
  \end{thmenum}
\end{example}

  \subsection{Algebras over semirings}\label{subsec:algebras_over_semirings}

Algebras are usually defined for fields or at least commutative rings. We extend this to semirings for the purposes of polynomial semirings.

\begin{definition}\label{def:multilinear_function}\mimprovised
  Generalizing \hyperref[def:semimodule/homomorphism]{linear maps}, if \( M_1, \ldots, M_n \) and \( N \) are \( R \)-modules, we say that the function
  \begin{equation*}
    f: M_1 \times \ldots \times M_n \to N
  \end{equation*}
  is \term{multilinear} (\term{bilinear} for \( n = 2 \)) if it is linear in each component. That is, for every tuple
  \begin{equation*}
    (x_1, \ldots, x_n) \in M_1 \times \cdots \times M_n,
  \end{equation*}
  and for every index \( k = 1, \ldots, n \), the following function is linear:
  \begin{equation*}
    y \mapsto f(x_1, \ldots, x_{k-1}, y, x_{k+1}, \ldots, x_n)
  \end{equation*}
\end{definition}

\begin{definition}\label{def:algebra_over_semiring}\mimprovised
  An \term{algebra} over a \hyperref[def:semiring/commutative]{commutative semiring} \( R \) is an \( R \)-\hyperref[def:semimodule]{semimodule} \( M \) with an \hyperref[def:magma/associative]{associative} \hyperref[def:multilinear_function]{bilinear} vector multiplication operation. This makes \( M \) a nonunital ring. By default, we will also assume that \( M \) has a multiplicative unit, although nonunital algebras as just as valid as nonunital rings.

  As in the case of general rings, by \enquote{\( M \) is commutative}, we will mean that vector multiplication is commutative. Furthermore, although we assume it by default, if needed, we will distinguish between associative and non-associative algebra.

  We identify every element \( t \) of \( R \) with its canonical embedding \( t \cdot 1_M \) in \( M \), and thus we can also regard \( R \) as a sub-semiring of \( M \).

  Algebras have the following metamathematical properties:
  \begin{thmenum}
    \thmitem{def:algebra_over_semiring/theory} The \hyperref[def:first_order_theory]{first-order theory} for algebras extends the \hyperref[def:semimodule/theory]{theory of commutative semimodules}. We add a new \hyperref[rem:first_order_formula_conventions/infix]{infix} binary function symbol \( \odot \) to the language, and add to the theory all semiring axioms from \fullref{def:semiring/theory} for \( + \) and \( \odot \). We must also add axioms ensuring that \( \odot \) is bilinear. Additivity follows from distributivity, hence it remains to account for homogeneity. Using the notation of \fullref{def:semimodule/theory}, this amounts to the following axiom schemas:
    \begin{align*}
      m_r(x) \odot y &= m_r(x \odot y), \\
      x \odot m_r(y) &= m_r(x \odot y).
    \end{align*}

    \thmitem{def:algebra_over_semiring/homomorphism} A \hyperref[def:first_order_homomorphism]{first-order homomorphism} between two \( R \)-algebras \( M \) and \( N \) is a linear map that also preserves vector multiplication.

    \thmitem{def:algebra_over_semiring/submodel} The set \( A \subseteq M \) is a \hyperref[thm:substructure_is_model]{submodel} of \( M \) if it is a \hyperref[def:monoid/submodel]{submodule} of \( M \) that is closed under algebra multiplication. We say that \( A \) is a \term{subalgebra}.

    As for general submodules, \fullref{rem:span_over_different_semirings} shows how it is important to be unambiguous about over which semiring we consider the subalgebra.

    As a consequence of \fullref{thm:positive_formulas_preserved_under_homomorphism}, the image of an \( R \)-algebra homomorphism is a subalgebra of its range.

    \thmitem{def:algebra_over_semiring/trivial} The \hyperref[rem:trivial_structure]{trivial} semimodule is the \hyperref[rem:pointed_set/trivial]{trivial pointed set} \( \set{ 0 } \).

    \thmitem{def:algebra_over_semiring/initial} The \hyperref[thm:substructures_form_complete_lattice/bottom]{initial substructure} of any \( R \)-algebra is isomorphic to the trivial \( R \)-algebra.

    \thmitem{def:algebra_over_semiring/category} We denote the category of algebras over \( R \) by \( \cat{Alg}_R \) and the subcategory of commutative algebras by \( \cat{CAlg}_R \).
  \end{thmenum}
\end{definition}

\begin{proposition}\label{thm:semiring_is_algebra}
  Every \hyperref[def:semiring]{semiring} \( R \) is an \( R \)-\hyperref[def:algebra_over_semiring]{algebra} with both scalar and vector multiplication given by the multiplication in \( R \).

  This extends \fullref{thm:commutative_monoid_is_semimodule}.
\end{proposition}
\begin{proof}
  Follows from \fullref{thm:commutative_monoid_is_semimodule} by noting that bilinearity follows from left distributivity in \( R \).
\end{proof}

\begin{proposition}\label{thm:semiring_is_natural_number_algebra}
  The categories \( \hyperref[def:semiring/category]{\cat{SRing}} \) of semirings and \( \hyperref[def:algebra_over_semiring/category]{\cat{Alg}_\BbbN} \) of natural number algebras are \hyperref[rem:category_similarity/isomorphism]{isomorphic}.

  Compare this result to \fullref{thm:commutative_monoid_is_semimodule} and \fullref{thm:ring_is_integer_algebra}.
\end{proposition}
\begin{proof}
  Follows from \fullref{thm:commutative_monoid_is_semimodule} by noting that, as in our proof of \fullref{thm:semiring_is_algebra}, distributivity implies bilinearity.
\end{proof}

\begin{proposition}\label{thm:functions_over_algebra}
  For a set \( A \) and an \( R \)-\hyperref[def:algebra_over_semiring]{algebra} \( N \), the set of all functions from \( A \) to \( N \) is itself an \( R \)-algebra with the following operations:
  \begin{thmenum}
    \thmitem{thm:functions_over_algebra/addition} Pointwise addition
    \begin{equation*}
      [f + g](x) \coloneqq f(x) + g(x)
    \end{equation*}

    \thmitem{thm:functions_over_algebra/scalar_multiplication} Pointwise scalar multiplication
    \begin{equation*}
      [t \cdot f](x) \coloneqq t \cdot f(x)
    \end{equation*}

    \thmitem{thm:functions_over_algebra/vector_multiplication} Pointwise vector multiplication
    \begin{equation*}
      [f \odot g](x) \coloneqq f(x) \cdot g(x)
    \end{equation*}

    In practice, we use juxtaposition \( fg \) or \( f \cdot g \) instead of \( f \odot g \).
  \end{thmenum}

  If \( A \) is also an \( R \)-algebra, we denote the set of all \( R \)-\hyperref[def:algebra_over_semiring/homomorphism]{algebra homomorphisms} by \( \hom(A, N) \).

  This result extends \fullref{thm:functions_over_semimodule}.
\end{proposition}
\begin{proof}
  By \fullref{thm:functions_over_model_form_model}, \( N \) is both an \( R \)-semiring and an \( R \)-semimodule. Compatibility comes from left distributivity in \( N \).
\end{proof}

\begin{definition}\label{def:multi_index}\mimprovised
  A \term{multi-index} over the \hyperref[def:set]{plain set} \( \mscrK \) is a member of the \hyperref[def:free_semimodule]{free \( \BbbN \)-semimodule} \( \mscrK^{\oplus \BbbN} \) over \( \mscrK \). We endow \( \mscrK^{\oplus \BbbN} \) with the \hyperref[def:norm]{norm}
  \begin{equation*}
    \norm{ \alpha } \coloneqq \sum_{k \in \mscrK} \alpha_k
  \end{equation*}
  and the \hyperref[def:partially_ordered_set]{partial order}
  \begin{equation*}
    \alpha \leq \beta \T{if and only if} \qforall {k \in \mscrK} \alpha_k \leq \beta_k.
  \end{equation*}

  Multi-indices are \hyperref[def:labeled_set/multiset]{multisets} with extra structure.
\end{definition}

\begin{definition}\label{def:polynomial_algebra}
  Fix a \hyperref[def:semiring/commutative]{commutative semiring} \( R \) and a set \( \mscrX \) of \hyperref[def:formal_language/symbol]{symbols}, which we will call \term{indeterminates}.

  Let \( \mscrM \) be the \hyperref[def:free_semimodule]{free \( R \)-semimodule} over \( \mscrX \), written \hyperref[rem:additive_magma]{multiplicatively}. We will call the members of \( \mscrM \) \term{monomials}. Using a \hyperref[def:multi_index]{multi-index} \( \gamma \) over \( \mscrX \), every monomial can be written as
  \begin{equation*}
    \prod_{X \in \mscrX} X^{\gamma_X},
  \end{equation*}
  where \( \gamma_X \) are the coefficients in \( R^{\oplus \mscrX} \) of the monomial.

  The \term{polynomial algebra} or \term{polynomial semiring} \( R[\mscrX] \) for the given indeterminates is the \hyperref[def:free_semimodule]{free \( R \)-semimodule} over \( \mscrM \). That is, a polynomial \( p \in R[\mscrX] \) is an \( R \)-linear combination of monomials, and we denote polynomials by
  \begin{equation}\label{eq:def:polynomial_algebra/p}
    p(\mscrX) = \sum_\gamma a_\gamma \prod_{X \in \mscrX} X^{\gamma_X}.
  \end{equation}

  We call \( a_\gamma \) the \term{coefficients} of the polynomial. We use the components of the multi-index as powers in the monomials, but we use \( \gamma \) itself as an index for the coefficient \( a_\gamma \). Unfortunately, multi-indices are sometimes confusing, but often their brevity outweighs the possible confusion.

  We do not ignore the structure of \( \mscrM \). We conflate exponentiation in \( \mscrM \) in the sense of \fullref{def:monoid/exponentiation} with exponentiation in \( R[\mscrX] \) in the sense of \fullref{def:semiring/exponentiation}. Multiplication in \( \mscrM \) motivates us to define multiplication in \( R[\mscrX] \) via a convolution of the coefficients. We define the product of \( p(X) \) from \eqref{eq:def:polynomial_algebra/p} with
  \begin{equation}\label{eq:def:polynomial_algebra/q}
    q(\mscrX) = \sum_\gamma b_\gamma \prod_{X \in \mscrX} X^{\gamma_X}
  \end{equation}
  as
  \begin{equation}\label{eq:def:polynomial_algebra/pq}
    [pq](\mscrX) \coloneqq \sum_\gamma \parens*{ \sum_{\delta + \eta = \gamma} a_\delta b_\eta } \prod_{X \in \mscrX} X^{\gamma_X}.
  \end{equation}

  We simultaneously use multi-indices as vectors with pointwise summation (i.e. \( \delta + \eta = \gamma \)) and as indices of coefficients (i.e. \( a_\delta \) and \( b_\eta \)).

  We avoid writing the embedding \( \iota: \mscrX \to R[\mscrX] \), but it is sometimes beneficial to denote it explicitly, for example in \fullref{thm:polynomial_algebra_universal_property}.
\end{definition}

\begin{theorem}[Polynomial algebra universal property]\label{thm:polynomial_algebra_universal_property}
  Fix a \hyperref[def:semiring/commutative]{commutative semiring} \( R \) and a set \( \mscrX \) of indeterminates. The \hyperref[def:polynomial_algebra]{polynomial algebra} \( R[\mscrX] \) is the unique up to a unique isomorphism commutative \hyperref[def:algebra_over_semiring]{algebra} that satisfies the following \hyperref[rem:universal_mapping_property]{universal mapping property}:
  \begin{displayquote}
    For every commutative \( R \)-algebra \( M \) and every function \( e: \mscrX \to M \), there exists a unique \( R \)-algebra homomorphism \( \Phi_e: R[\mscrX] \to M \) such that the following diagram commutes:
    \begin{equation}\label{eq:thm:polynomial_algebra_universal_property/diagram}
      \begin{aligned}
        \includegraphics[page=1]{output/thm__polynomial_algebra_universal_property.pdf}
      \end{aligned}
    \end{equation}
  \end{displayquote}

  The function \( e \) evaluates each indeterminate in \( M \), while \( \Phi_e \) substitutes this value in every polynomial. We call \( \Phi_e \) the \term{substitution homomorphism} corresponding to the \term{variable assignment} \( e \). We can parameterize this by the evaluation functions to obtain the functional evaluation homomorphism
  \begin{equation*}
    \begin{aligned}
      &\Phi: R[\mscrX] \to \fun(M^\mscrX, M) \\
      &\Phi(p) \coloneqq (e \mapsto \Phi_e(p))
    \end{aligned}
  \end{equation*}

  We call the values of \( \Phi \) \term{polynomial functions}. Given elements \( x_1, \ldots, x_n \) of \( M \), we write
  \begin{equation*}
    p(x_1, \ldots, x_n)
  \end{equation*}
  rather than
  \begin{equation*}
    \Phi(p)(x_1, \ldots, x_n).
  \end{equation*}

  Via \fullref{rem:universal_mapping_property}, \( R[\anon*] \) becomes \hyperref[def:category_adjunction]{left adjoint} to the \hyperref[def:concrete_category]{forgetful functor}
  \begin{equation*}
    U: \cat{CAlg}_R \to \cat{Set}.
  \end{equation*}

  The action of \( R[\anon*] \) on morphisms is given by \( \Phi \).
\end{theorem}
\begin{proof}
  For every indeterminate \( X \), we want
  \begin{equation*}
    \Phi_e(\iota(X)) = f(X).
  \end{equation*}

  This suggests defining \( \Phi_e \) for the polynomial
  \begin{equation*}
    p(\mscrX) = \sum_\gamma a_\gamma \prod_{X \in \mscrX} \iota(X)^{\gamma_X}
  \end{equation*}
  as the evaluation
  \begin{equation*}
    \Phi_e(p) \coloneqq \sum_\gamma a_\gamma \prod_{X \in \mscrX} f(X)^{\gamma_X}.
  \end{equation*}

  We discuss well-definedness of infinitary operations in direct sums in \fullref{rem:binary_operation_syntax_trees/infinite/direct_sum}.
\end{proof}

\begin{remark}\label{rem:polynomials_over_infinitely_many_indeterminates}
  As we saw in \fullref{def:polynomial_algebra} and \fullref{thm:polynomial_algebra_universal_property}, there is no formal problem in defining polynomial algebras over infinitely many indeterminates.

  There is a problem, however. Polynomials in one indeterminate, which we will call univariate in accordance to \fullref{def:multi_valued_function/arguments}, have a \hyperref[def:well_ordered_set]{well-ordering} on their monomials, induced by the degree of their monomials. This is defined and discussed in \fullref{subsec:univariate_polynomials}.

  Polynomials in more than one variable do not have a well-ordering by default. If the indeterminates themselves are well-ordered, as is the case for finitely many indeterminates, we may introduce, for example, a \hyperref[def:lexicographic_order]{reverse lexicographic order} on the monomials. Furthermore, for finitely many variables, \fullref{thm:def:polynomial_algebra/iterated} allows us to use \hyperref[rem:induction/peano_arithmetic]{natural number induction} on the number of variables in order to prove statements about multivariate polynomial rings.

  For infinitely many, especially uncountably many variables, however, the theory is seriously crippled by the lack of the tools described above. For this reason, only polynomials in finitely many variables are often considered.
\end{remark}

\begin{proposition}\label{thm:def:polynomial_algebra}
  The following are basic properties of \hyperref[def:polynomial_algebra]{polynomial semirings}:
  \begin{thmenum}
    \thmitem{thm:def:polynomial_algebra/empty} If \( \mscrX \) is empty, \( R[\mscrX] \cong R \).

    \thmitem{thm:def:polynomial_algebra/iterated} The polynomial algebras \( R[\mscrX] \) and \( R[\mscrX \setminus \set{ X_0 }][X_0] \) are isomorphic for any \( X_0 \in \mscrX \) (in case \( \mscrX \) has more than one member).

    In particular,
    \begin{equation*}
      R[X_1, \ldots, X_{n-1}][X_n] \cong R[X_1, \ldots, X_n].
    \end{equation*}

    \thmitem{thm:def:polynomial_algebra/entire} The univariate \hyperref[def:polynomial_algebra]{polynomial semiring} \( R[X] \) is \hyperref[def:divisibility/zero]{entire} if and only if \( R \) is entire.

    \thmitem{thm:def:polynomial_algebra/units} If \( R \) is entire, the \hyperref[def:divisibility/unit]{units} in \( R[X_1, \ldots, X_n] \) are precisely the (embeddings of) the units of \( R \).
  \end{thmenum}
\end{proposition}
\begin{proof}
  \SubProofOf{thm:def:polynomial_algebra/empty} Trivial.

  \SubProofOf{thm:def:polynomial_algebra/iterated} Polynomials in \( R[\mscrX] \) have the form
  \begin{equation*}
    p(\mscrX) = \sum_{k=0}^\infty \sum_\gamma \parens*{ a_{(k,\gamma)} \prod_{\mathclap{X \in \mscrX \setminus \set{ X_0 }}} X^{\gamma_X} } X_0^k,
  \end{equation*}
  where \( \gamma \) is a \hyperref[def:multi_index]{multi-index} on \( \mscrX \).

  Due to associativity, commutativity and distributivity, this can be rewritten as
  \begin{equation*}
    p(\mscrX) = \sum_{k=0}^\infty \parens*{ \sum_\gamma a_{(k,\gamma)} \prod_{\mathclap{X \in \mscrX \setminus \set{ X_0 }}} X^{\gamma_X} } X_0^k.
  \end{equation*}

  This shows how \( R[\mscrX] \) can be embedded into \( R[\mscrX \setminus \set{ X_0 }][X_0] \). This embedding is surjective because the coefficients \( a_\gamma \) range through \( R \). Therefore, the embedding is an isomorphism.

  \SubProofOf{thm:def:polynomial_algebra/entire}

  \SufficiencySubProof Since \( R \) is an \( R \)-subalgebra of \( R[X] \), if the latter is entire, so is the former.

  \NecessitySubProof Suppose that \( R \) is entire and that \( R[X] \) isn't. Then there exist nonzero polynomials \( p(X) \) and \( q(X) \) such that \( p(X) q(X) = 0 \). If \( a_n \) is the leading coefficient of \( p(X) \) and \( b_m \) --- of \( q(X) \), the leading coefficient of \( p(X) q(X) \) is \( a_n b_m \). Since \( p(X) q(X) \) is the zero polynomial, \( a_n b_m = 0 \), which contradicts the assumption that \( R \) is entire.

  Therefore, \( R[X] \) is entire.

  \SubProofOf{thm:def:polynomial_algebra/units} As in \fullref{thm:def:polynomial_algebra/entire}, it is sufficient to prove the statement for one indeterminate.

  Clearly every constant is invertible as a constant polynomial.

  Now suppose that \( p(X) q(X) = 1 \). By definition of multiplication, the product has only one nonzero coefficient. Since \( R \) is entire, it follows that both \( p(X) \) and \( q(X) \) have only one nonzero coefficient, and are hence constants.
\end{proof}

\begin{example}\label{ex:def:polynomial_algebra}
  We list several examples of \hyperref[def:polynomial_algebra]{polynomials} over semirings.
  \begin{thmenum}
    \thmitem{ex:def:polynomial_algebra/natural_numbers} Consider the polynomial \( p(X) \coloneqq aX^2 + bX + c \) in \( \BbbN[X] \). A function from the set \( \set{ X } \) to \( \BbbN \) corresponds to an element of \( \BbbN \), and hence evaluating the polynomial is done by simply replacing \( X \) symbolically in \( p \) and then evaluating the obtained \hyperref[rem:binary_operation_syntax_trees]{syntax tree}.

    We seek the roots of \( p(X) \). We will only formally define roots in \fullref{def:polynomial_root}; for the purposes of the example, a root is a natural number \( n \) such that \( \Phi_n(p) = 0_R \).

    By \fullref{thm:fundamental_theorem_of_algebra} and \fullref{def:algebraically_closed_field/exactly_n_roots}, \( p \) has two roots in the \hyperref[def:complex_numbers]{complex plane}. That is, we regard \( \BbbC \) as an algebra over \( \BbbN \) and use \fullref{thm:polynomial_algebra_universal_property} to obtain a polynomial function on \( \BbbC \). Furthermore, over the complex numbers the roots can be explicitly found using
    \begin{equation*}
      \frac {-b \pm \sqrt{b^2 - 4ac}} {2a}.
    \end{equation*}

    Finding a root of \( p \) over the natural numbers cannot be done in general, however. If \( p(n) = 0 \), by the ordering of the natural numbers we have
    \begin{equation*}
      p(n) = an^2 + bn + c \geq c,
    \end{equation*}
    and hence \( c \) must necessarily be \( 0 \). If \( c = 0 \), then zero is a root of the polynomial \( p(X) = aX^2 + bX \).

    Now let \( n \) be any root of \( p \). We have
    \begin{equation*}
      an^2 + bn \geq bn,
    \end{equation*}
    and hence \( bn \) must also be \( 0 \). Thus, either \( b = 0 \) or \( n = 0 \). If we want a root other than \( n \), both \( a \) and \( b \) must be \( 0 \).

    Therefore, the only natural number solution to the quadratic equation is \( 0 \), and it is only a solution if \( c = 0 \).

    \thmitem{ex:def:polynomial_algebra/tropical} Consider again the polynomial \( p(X) \coloneqq aX^2 + bX + c \) over \( \BbbN \), but this time evaluated over the \hyperref[def:tropical_semiring]{\( \min \)-plus semiring} \( (\BbbN \cup \set{ \infty }, \min, +) \).

    Expressed via the standard natural number operations, this polynomial becomes
    \begin{equation*}
      \min\set{ 2X + a, X + b, c }.
    \end{equation*}

    This allows us to express certain optimization problems via polynomials.

    This polynomial has a root if and only if \( a = b = c \). Roots in the tropical semiring are not very interesting, however.
  \end{thmenum}
\end{example}

\begin{proposition}\label{thm:generators_via_polynomials}
  For a set \( A \) in an \( R \)-\hyperref[def:algebra_over_semiring]{algebra} \( M \), the \hyperref[def:algebra_over_semiring/submodel]{generated subalgebra} of \( A \), defined as the \( R \)-subalgebra generated by \( A \) in the sense of \fullref{def:first_order_generated_substructure}, equals the set
  \begin{equation*}
    \bigcup \set[\Big]{ R[a_1, \ldots, a_n] \given* a_1, \ldots, a_n \in A }
  \end{equation*}
  obtained by evaluating all multivariate polynomials over \( R \) with elements of \( A \).

  The \( \BbbN \)-subalgebras of \( M \) correspond to \hyperref[def:semiring/submodel]{sub-semirings} and the \( M \)-subalgebras correspond to \hyperref[def:semiring_ideal/generated]{ideals}.

  Compare this result to \fullref{thm:span_via_linear_combinations} for modules.
\end{proposition}
\begin{proof}
  Similar to \fullref{thm:span_via_linear_combinations}.
\end{proof}

\begin{proposition}\label{thm:adjoining_elements_to_semiring}
  Let \( R \subseteq S \) be \hyperref[def:semiring/commutative]{commutative semirings} and let \( A \subseteq S \) be an arbitrary subset.

  Fix a set \( \mscrX \) of indeterminates and a bijective function \( e: \mscrX \to A \) and consider the \hyperref[thm:polynomial_algebra_universal_property]{evaluation homomorphism}
  \begin{equation*}
    \Phi_e: R[\mscrX] \to S.
  \end{equation*}

  The image \( R[A] \) of \( \Phi_e \) is the smallest super-semiring of \( R \) that contains \( A \).

  We say that \( R[A] \) is obtained by \term{adjoining} the elements of \( A \) to \( R \).
\end{proposition}
\begin{proof}
  Follows from \fullref{thm:generators_via_polynomials}.
\end{proof}

\begin{example}\label{ex:adjoining_root}
  Continuing \fullref{ex:def:polynomial_algebra/natural_numbers}, consider the polynomial equation
  \begin{equation*}
    X + 1 = 0.
  \end{equation*}

  It has no natural number root as a consequence of \eqref{eq:def:peano_arithmetic/PA2}.

  It does have an integer root, however, \( -1 \). We can \hyperref[thm:adjoining_elements_to_semiring]{adjoin} \( -1 \) to the semiring \( \BbbN \) to obtain the semiring \( \BbbN[-1] \). But this latter semiring is (isomorphic to) \( \BbbZ \).

  Therefore, \( \BbbZ \) is the smallest extension of \( \BbbN \) that contains a root to the polynomial \( X + 1 \).

  This example extends to the theory of \hyperref[def:transcendental_element]{transcendental and algebraic} elements of fields.
\end{example}

\begin{definition}\label{def:formal_power_series}\mimprovised
  If we extend the concept of \hyperref[def:polynomial_algebra]{polynomials} to allow countably many nonzero terms, we obtain a set \( R\Bracks{\mscrX} \) which we call the \term{formal power series} over \( R \) with indeterminates from the set \( \mscrX \).

  The evaluation homomorphism defined in \fullref{thm:polynomial_algebra_universal_property} is problematic, however, since algebraic operations are finitary by nature. This is discussed in \fullref{rem:binary_operation_syntax_trees/infinite}, along with how sometimes we can make sense of infinitary algebraic operations.
\end{definition}

  \subsection{Rings}\label{subsec:rings}

\begin{definition}\label{def:ring}
  A \term{ring} is a \hyperref[def:semiring]{semiring} with additive inverses. More precisely, this means that the additive monoid is a group.

  As for semirings, rings can also be nonunital, with \hyperref[def:semiring_ideal]{ring ideals} being the main example.

  Rings have the following metamathematical properties:
  \begin{thmenum}
    \thmitem{def:ring/theory} We can construct a \hyperref[def:first_order_theory]{first-order theory} for rings by adding a unary functional symbol \( - \) and the involution axiom \eqref{eq:def:group/theory/inverse_axiom} to the \hyperref[def:semiring/theory]{theory of semirings}.

    \thmitem{def:ring/homomorphism} A \hyperref[def:first_order_homomorphism]{first-order homomorphism} between the rings \( R \) and \( T \) is a \hyperref[def:semiring/homomorphism]{semiring homomorphism} \( \varphi: R \to T \) that additionally preserves additive inverses.

    As shown in \fullref{thm:group_homomorphism_single_condition}, this condition is not only redundant, but the structure of a ring rather than semiring also automatically implies that \( \varphi(0_R) = 0_S \).

    \thmitem{def:ring/submodel} The set \( A \subseteq R \) is a \hyperref[def:first_order_submodel]{submodel} of \( R \) if it is both a \hyperref[def:semiring]{sub-semiring} of \( R \) and an additive submonoid of \( R \).

    As a consequence of \fullref{thm:positive_formulas_preserved_under_homomorphism}, the image of a ring homomorphism is a subring of its codomain.

    \thmitem{def:ring/trivial} Any single-element ring, that is, every zero object in \( \ucat{Ring}_R \), is as trivial object in the sense of \fullref{def:trivial_object}. We often call it \enquote{the} trivial \( R \)-algebra.

    \thmitem{def:ring/commutative} If multiplication is commutative, we call the ring itself \term{commutative}. Unless multiplication corresponds to function composition, most rings we will encounter will be commutative.

    \thmitem{def:ring/category} The corresponding \hyperref[def:category_of_small_first_order_models]{category of \( \mscrU \)-small models} \( \ucat{Ring} \) is \hyperref[def:concrete_category]{concrete} over \hyperref[def:monoid]{\( \ucat{SRing} \)}. We denote the category of commutative rings by \( \cat{CRing} \).

    Unlike the category \hyperref[def:group/category]{\( \cat{Grp} \)} of groups, \( \cat{Ring} \) is not as well-behaved. Nevertheless, kernels and quotients of rings are commonly established concepts.

    The category of unital rings does not have a zero object, but the category of nonunital rings does, and we will sometimes consider nonunital ring homomorphisms between unital rings. That is, ring homomorphisms that may not preserve the multiplicative identity.

    \thmitem{def:ring/kernel} The \hyperref[def:zero_morphisms/kernel]{kernel} of a ring homomorphism \( \varphi: R \to T \) is simply its \hyperref[def:zero_locus]{zero locus} \( \varphi^{-1}(0_S) \). This is precisely the kernel of the additive group in the sense of \fullref{def:group/kernel}, and the \hyperref[def:zero_morphisms/cokernel]{categorical kernel} in the category of nonunital rings.

    Furthermore, \( \ker \varphi \) is an ideal of \( R \) because, if \( x \in \ker \varphi \),
    \begin{equation*}
      \varphi(xy)
      =
      \varphi(x) \varphi(y)
      =
      0_S \varphi(y)
      =
      0_S,
    \end{equation*}
    and thus \( xy \in \ker \varphi \).

    Despite being categorical kernels only in the category of nonunital rings, the kernel is defined and used mainly for unital ring homomorphism.

    \thmitem{def:ring/quotient} The \hyperref[def:zero_morphisms/cokernel]{categorical cokernel} of homomorphism \( \varphi: R \to T \) in the category of nonunital rings is, similarly to the case for groups in \fullref{def:group/quotient}, a partition of \( T \) induced by the image of \( \varphi \).

    This is not merely the cokernel \( T / \img \varphi \) of the additive group, however. Multiplication induces an additional restriction on congruences: \( x \cong x' \) and \( y \cong y' \) together imply \( x y \cong x' y' \). Hence, \( [x][y] = [xy] \).  Denote the coset \( [0_S] \) by \( I \). We have \( I[x] = [0x] = I \), therefore the cokernel inherits absorption from \( T \).

    Additive subgroups of \( T \) that absorb multiplication are precisely the \hyperref[def:semiring_ideal]{two-sided ideals} of \( T \). Hence, \( I \) is the ideal generated by \( \img \varphi \). From the general case for groups it follows that quotient ring cosets have the form \( x + I \).

    Finally, given an ideal \( I \) of an arbitrary ring \( R \), we can define the \term{quotient ring} \( R / I \) as the cokernel of the inclusion \( \iota: I \to R \). That is, \( R / I \) consists of the cosets \( x + I \) for \( x \in R \). In practice, quotients are conveniently characterized by \fullref{thm:quotient_algebra_universal_property}.

    Somewhat similarly to groups, the kernel \( \ker \pi \) of the canonical projection \( \pi(x) \coloneqq x + I \) is the ideal \( I \) itself.

    Despite being \hyperref[def:zero_morphisms/cokernel]{categorical cokernel} only in the category of nonunital rings, the quotient \( R / I \) is defined and used mainly for unital rings.

    Fortunately, for unital ring \( R \), the quotient \( R / I \) is also unital. The projection morphism is an epimorphism by \fullref{thm:equalizer_invertibility}, and hence \( R / I \) is a \hyperref[def:subobject_and_quotient]{categorical quotient object}.
  \end{thmenum}
\end{definition}

\begin{proposition}\label{thm:quotient_equality_via_difference}
  Given an \hyperref[def:semiring_ideal]{ideal} \( I \) in a \hyperref[def:ring]{ring} \( R \), we have \( x + I = y + I \) if and only if \( x - y \in I \).
\end{proposition}
\begin{proof}
  Trivial.
\end{proof}

\begin{proposition}\label{thm:semiring_cancellative_iff_no_zero_divisors}
  An element of a \hyperref[def:semiring]{ring} is left (resp. right) \hyperref[def:binary_operation/cancellative]{cancellable} if and only if it is not a left (resp. right) \hyperref[def:divisibility/zero]{nontrivial zero divisor}. That is, \( xy = xz \) implies \( y = z \) if and only if \( xa = 0 \) implies \( a = 0 \).
\end{proposition}
\begin{proof}
  \SufficiencySubProof Suppose that \( xy = xz \) implies \( y = z \). Suppose also that \( xa = 0 \). Then \( xa = 0 = x0 \), implying that \( a = 0 \).

  \NecessitySubProof Suppose that \( xa = 0 \) implies \( a = 0 \). Suppose also that \( xy = xz \). Then \( x(y - z) = 0 \), implying that \( y = z \).
\end{proof}

\begin{definition}\label{def:entire_semiring}\mcite[4]{Golan2010}
  We say that the \hyperref[def:semiring]{semiring} \( R \) is \term{entire} if it has no nontrivial \hyperref[def:divisibility]{zero divisors}, neither left nor right.
\end{definition}

\begin{proposition}\label{thm:ring_entire_iff_cancellative}
  In an \hyperref[def:entire_semiring]{entire} \hyperref[def:ring]{ring} \( R \), the set \( R \setminus \set{ 0_R } \) is a \hyperref[def:binary_operation/cancellative]{cancellative} \hyperref[def:monoid]{monoid} with respect to multiplication.
\end{proposition}
\begin{proof}
  Follows from \fullref{thm:semiring_cancellative_iff_no_zero_divisors}.
\end{proof}

\begin{proposition}\label{thm:simple_ring_homomorphism_is_injective}
  If \( R \) is a \hyperref[def:simple_object]{simple ring}, every \hyperref[def:ring/homomorphism]{unital ring homomorphism} \( \varphi: R \to T \) is an injective function.
\end{proposition}
\begin{proof}
  Let \( \varphi: R \to T \) be a unital ring homomorphism. Its kernel is an ideal of \( R \), and the only ideals are the trivial ring or \( R \) itself. Since \( \varphi(1_R) = 1_T \), the kernel cannot be \( R \), so it can only be the trivial ring. \Fullref{thm:group_homomorphism_zero_kernel} then implies that \( \varphi \) is injective.
\end{proof}

\begin{proposition}\label{thm:ring_of_integers_modulo}
  For a positive integer \( n > 1 \), we extend the group \hyperref[def:group_of_integers_modulo]{\( \BbbZ_n \)} of integers modulo \( n \) with the operation
  \begin{equation*}
    x \odot y \coloneqq \rem(xy, n).
  \end{equation*}

  Then \( \BbbZ_n \) is a \hyperref[def:ring/commutative]{commutative ring} called the \term{ring of integers modulo} \( n \).
\end{proposition}
\begin{proof}
  Note that
  \begin{balign*}
    &\phantom{{}\cong{}} \rem(x, n) \rem(y, n)
    &\cong \pmod n \\ &\cong
    (x - n \quot(x, n)) (y - n \quot(y, n))
    &\cong \pmod n \\ &\cong
    xy - n \quot(x, n) - n \quot(y, n) + n^2 \quot(x, n) \quot(y, n)
    &\cong \pmod n \\ &\cong
    xy.
  \end{balign*}

  The proof that multiplication in \( \BbbZ_n \) is associative, unital and commutative becomes trivial.

  We will prove that multiplication distributes over addition. Fix \( x, y, z \in \BbbZ_n \). We have
  \begin{balign*}
    (x \oplus y) \odot z
     & =
    \rem((x \oplus y) z, n)
    =    \\ &=
    \rem(\rem(x + y, n) z, n)
    =    \\ &=
    \rem((x + y - n \quot(x + y, n)) z, n)
    =    \\ &=
    \rem((x + y)z, n).
  \end{balign*}
  and
  \begin{balign*}
    (x \odot z) \oplus (y \odot z)
     & =
    \rem([(x \odot z) + (y \odot z)], n)
    =    \\ &=
    \rem([xz - n \quot(xz, n) + yz - n \quot(yz, n)], n)
    =    \\ &=
    \rem(xz + yz, n)
    =    \\ &=
    \rem((x + y)z, n).
  \end{balign*}

  Hence,
  \begin{equation*}
    (x \oplus y) \odot z = (x \odot z) \oplus (y \odot z).
  \end{equation*}
\end{proof}

\begin{proposition}\label{thm:ring_characteristic_homomorphism}
  Similarly to how \( \BbbN \) is an \hyperref[def:universal_objects/initial]{initial object} in the category \hyperref[def:semiring/category]{\( \cat{SRing} \)} of semirings, \( \BbbZ \) is an initial object in the category \hyperref[def:ring/category]{\( \cat{Ring} \)} of rings.
\end{proposition}
\begin{proof}
  Follows from \fullref{thm:semiring_characteristic_homomorphism} with the addition of \( (-n)x = -nx \).
\end{proof}

\begin{definition}\label{def:ring_characteristic}\mimprovised
  We define the \term{characteristic} \( \op{char}(R) \) of a ring \( R \) via the following equivalent definitions:
  \begin{thmenum}
    \thmitem{def:ring_characteristic/embedding} \( \op{char}(R) \) is the unique nonnegative integer \( n \) for which \( \BbbZ_n \) can be embedded into \( R \). That is,
    \begin{equation*}
      \BbbZ_n \cong \BbbZ / \ker \iota,
    \end{equation*}
    where \( \iota \) is the homomorphism from the integers defined via \eqref{eq:thm:semiring_characteristic_homomorphism}.

    We use here that \( \BbbZ_0 \) is the \hyperref[def:ring/trivial]{trivial ring}.

    \thmitem{def:ring_characteristic/direct} \( \op{char}(R) \) is the \hyperref[def:group_order]{order} of the additive group of \( R \) and, if it exists, and \( 0 \) otherwise.

    That is, \( \op{char}(R) \) is the smallest positive integer \( n \) such that \( n \cdot 1_R = 0_R \) and \( \op{char}(R) = 0 \) if \( 0_R \) cannot be obtained in this way.
  \end{thmenum}
\end{definition}
\begin{defproof}
  \EquivalenceSubProof{def:ring_characteristic/embedding}{def:ring_characteristic/direct} Let \( n \) be such that
  \begin{equation*}
    n \BbbZ = \ker\iota.
  \end{equation*}

  In particular, \( \iota(0) = \iota(n) \).

  If \( n = 0 \), \( \ker\iota \) is a trivial group and \( \iota \) is an embedding. Then there cannot exist a positive integer \( n \) such that
  \begin{equation*}
    n \cdot 1_R = 0_R.
  \end{equation*}

  Otherwise, \( n \) is the smallest positive integer such that
  \begin{equation*}
    n \cdot 1_R = 0 \cdot 1_R = 0_R.
  \end{equation*}
\end{defproof}

\begin{proposition}\label{thm:ring_embedding_preserves_characteristic}
  If \( \varphi: R \to S \) is a \hyperref[def:ring/homomorphism]{ring embedding}, then \( S \) inherits its \hyperref[def:ring_characteristic]{characteristics} from \( R \).
\end{proposition}
\begin{proof}
  First suppose that \( R \) has positive characteristic \( n \). Then \( n \cdot 1_R = 0_R \), which implies \( n \cdot \varphi(1_R) = \varphi(0_R) \), hence \( \op{char}(T) \leq n \). But \( \varphi \) is an embedding, hence if \( k \cdot 1_R \neq 0_R \), then
  \begin{equation*}
    k \cdot \varphi(1_R) \neq \varphi(0_R).
  \end{equation*}

  This implies that \( \op{char}(S) \geq \op{char}(R) \), which in turn shows that \( \op{char}(S) = \op{char}(R) \).

  If \( R \) has characteristic zero, then \( \iota: \BbbN \to R \) is an embedding and thus \( \varphi \bincirc \iota: \BbbN \to S \) is also an embedding. It is unique as shown in \fullref{thm:ring_characteristic_homomorphism}. Therefore, \( S \) also has characteristic zero.
\end{proof}

\begin{example}\label{ex:def:ring_characteristic}
  The following are examples of \hyperref[def:ring_characteristic]{ring characteristics}:
  \begin{thmenum}
    \thmitem{ex:def:ring_characteristic/natural_numbers} The \hyperref[def:integers]{integers} \( \BbbZ \) have characteristic \( \op{char}(\BbbZ) = 0 \) because \( \iota \) is an isomorphism. Consequently, by \fullref{thm:ring_embedding_preserves_characteristic}, any superring of \( \BbbZ \) has characteristic zero, most notably the fields \( \BbbQ \), \( \BbbR \) and \( \BbbC \).

    \thmitem{ex:def:ring_characteristic/integers_modulo} The ring \hyperref[thm:ring_of_integers_modulo]{\( \BbbZ_n \)} of integers modulo \( n \) has characteristic \( \op{char}(\BbbZ_n) = n \) because of \fullref{thm:integers_modulo_isomorphic_to_quotient_group}.

    \thmitem{ex:def:ring_characteristic/polynomial_ring} An \hyperref[def:algebra_over_semiring]{algebra} \( M \) over a nontrivial commutative unital ring \( R \) has the same characteristic as \( R \) because of the canonical embedding of \( R \) in \( M \). In particular, the \hyperref[def:polynomial_algebra]{polynomial ring} \( R[X] \) has the same characteristic as their ring.
  \end{thmenum}
\end{example}

\begin{proposition}\label{thm:grothendieck_semiring_completion}\mcite[95]{Enderton1977Sets}
  The \hyperref[def:monoid_grothendieck_completion]{Grothendieck completion} \( \overline{R} \) of the additive monoid of a \hyperref[def:semiring]{semiring} \( R \) becomes a \hyperref[def:ring]{ring} with the operation
  \begin{equation*}
    [(a, b)] \odot [(c, d)] \coloneqq [(ac + bd, ad + bc)].
  \end{equation*}

  This definition is motivated in our proof of \fullref{thm:grothendieck_semiring_completion_universal_property}.
\end{proposition}
\begin{proof}
  Multiplication on \( R \) does not depend on the representative of the equivalence class. Indeed, let \( (a, b) \sim (a', b') \) and \( (c, d) \sim (c', d') \). Then there exist \( u \) and \( v \) such that
  \begin{align*}
    a + b' + u &= a' + b + u, \\
    c + d' + v &= c' + d + v.
  \end{align*}

  Then
  \begin{align*}
    &\phantom{{}={}}
    \hi{ac} + b'c + uc + a'd + \hi{bd} + ud + a'c + \hi{a'd'} + a'v + \hi{b'c'} + b'd + b'v
    = \\ &=
    (a + b' + u)c + (a' + b + u)d + a'(c + d' + v) + b'(c' + d + v)
    = \\ &=
    (a' + b + u)c + (a + b' + u)d + a'(c' + d + v) + b'(c + d' + v)
    = \\ &=
    a'c + \hi{bc} + uc + \hi{ad} + b'd + ud + \hi{a'c'} + a'd + a'v + b'c + \hi{b'd'} + b'v.
  \end{align*}

  Therefore,
  \begin{equation*}
    (a \cdot c + b \cdot d, a \cdot d + b \cdot c) \sim (a' \cdot c' + b' \cdot d', a' \cdot d' + b' \cdot c').
  \end{equation*}

  Associativity and distributivity in \( \overline{R} \) are inherited from \( R \).
\end{proof}

\begin{theorem}[Grothendieck semiring completion universal property]\label{thm:grothendieck_semiring_completion_universal_property}
  The \hyperref[thm:grothendieck_semiring_completion]{Grothendieck completion} \( \overline{R} \) of a semiring \( R \) satisfies the following \hyperref[rem:universal_mapping_property]{universal mapping property}:
  \begin{displayquote}
    For every ring \( T \) and every semiring homomorphism \( \varphi: R \to T \), there exists a unique ring homomorphism \( \widetilde{\varphi}: \overline{R} \to T \) such that the following diagram commutes:
    \begin{equation}\label{eq:thm:grothendieck_semiring_completion_universal_property/diagram}
      \begin{aligned}
        \includegraphics[page=1]{output/thm__grothendieck_semiring_completion_universal_property}
      \end{aligned}
    \end{equation}
  \end{displayquote}

  Via \fullref{rem:universal_mapping_property}, \( \overline{\anon} \) becomes \hyperref[def:category_adjunction]{left adjoint} to the \hyperref[def:concrete_category]{forgetful functor}
  \begin{equation*}
    U: \cat{CRing} \to \cat{CSRing}.
  \end{equation*}

  Compare this result to \fullref{thm:grothendieck_monoid_completion_universal_property}.
\end{theorem}
\begin{proof}
  \Fullref{thm:grothendieck_monoid_completion_universal_property} suggests the definition
  \begin{equation*}
    \overline{\varphi}([(a, b)]) \coloneqq \varphi(a) - \varphi(b).
  \end{equation*}

  We must only show that \( \overline{\varphi} \) is a ring homomorphism. Clearly
  \begin{equation*}
    \overline{\varphi}([(1, 0)]) = \varphi(1) - \varphi(0),
  \end{equation*}
  which implies that \( \varphi \) preserves multiplicative identities. Also,
  \begin{balign*}
    \overline{\varphi}\parens[\Big]{ [(a, b)] \odot [(c, d)] }
    &=
    \overline{\varphi}\parens[\Big]{ [(a \cdot b + c \cdot d, a \cdot d + b \cdot c)] }
    = \\ &=
    \varphi(a \cdot b + c \cdot d) - \varphi(a \cdot d + b \cdot c)
    = \\ &=
    \varphi(c) \parens[\Big]{ \varphi(d) - \varphi(b) } - \varphi(a) \parens[\Big]{ \varphi(d) - \varphi(b) }
    = \\ &=
    \parens[\Big]{ \varphi(c) - \varphi(a) } \parens[\Big]{ \varphi(d) - \varphi(b) }
    = \\ &=
    \overline{\varphi}\parens[\Big]{ [(a, c)] } \overline{\varphi}\parens[\Big]{ [(b, d)] }.
  \end{balign*}
\end{proof}

\begin{proposition}\label{thm:def:grothendieck_semiring_completion}
  The \hyperref[thm:grothendieck_semiring_completion]{Grothendieck completion} \( \overline{R} \) of a semiring \( R \) satisfies the following basic properties:
  \begin{thmenum}
    \thmitem{thm:def:grothendieck_semiring_completion/commutative} If \( R \) is commutative, so is \( \overline{R} \).
    \thmitem{thm:def:grothendieck_semiring_completion/entire} If \( R \) is \hyperref[def:divisibility/zero]{entire}, so is \( \overline{R} \).
  \end{thmenum}
\end{proposition}
\begin{proof}
  \SubProofOf{thm:def:grothendieck_semiring_completion/commutative} This is clear from the definition of multiplication.
  \SubProofOf{thm:def:grothendieck_semiring_completion/entire} Suppose that
  \begin{equation*}
    \underbrace{[(a, b)] \cdot [(c, d)]}_{[(ac + bd, ad + bc)]} = [(0, 0)]
  \end{equation*}

  Then there exists an element \( u \) in \( R \) such that
  \begin{equation*}
    (ac + bd) + 0 + u = 0 + (ad + bc) + u.
  \end{equation*}

  Suppose that \( d = c + e \). Then
  \begin{equation*}
    ac + b(c + e) = a(c + e) + bc
  \end{equation*}
  and
  \begin{equation*}
    (ac + bc) + be = (ac + bc) + ae.
  \end{equation*}

  Cancelling \( e \), we obtain that \( a = b \). But \( [(a, b)] = [(0, 0)] \).
\end{proof}

\begin{definition}\label{def:ring_commutator}
  Let \( R \) be an arbitrary ring. We define the \term{commutator} of the elements \( x \) and \( y \) as
  \begin{equation*}
    [x, y] \coloneqq xy - yx.
  \end{equation*}

  The \term{commutator ideal} \( [R, R] \) of \( R \) is the two-sided ideal \hyperref[def:semiring_ideal/generated]{generated} by all the commutators in \( G \).
\end{definition}

\begin{theorem}[Ring abelianization universal property]\label{thm:ring_abelianization_universal_property}\mcite[prop. 7.4]{Knapp2016BasicAlgebra}
  The quotient \( R / [R, R] \) of a ring \( R \) by its commutator ideal \( [R, R] \) is a commutative ring, which we call the \term{abelianization} of \( R \), and satisfies the following \hyperref[rem:universal_mapping_property]{universal mapping property}:
  \begin{displayquote}
    For every commutative ring \( T \) and every ring homomorphism \( \varphi: R \to T \), \( \varphi \) \hyperref[def:factors_through]{uniquely factors through} \( R / [R, R] \). More precisely, there exists a unique ring homomorphism \( \widetilde{\varphi}: R / [R, R] \to T \) such that the following diagram commutes:
    \begin{equation}\label{eq:thm:ring_abelianization_universal_property/diagram}
      \begin{aligned}
        \includegraphics[page=1]{output/thm__ring_abelianization_universal_property}
      \end{aligned}
    \end{equation}
  \end{displayquote}

  Via \fullref{rem:universal_mapping_property}, the abelianization functor becomes \hyperref[def:category_adjunction]{left adjoint} to the \hyperref[def:concrete_category]{forgetful functor}
  \begin{equation*}
    U: \cat{CRing} \to \cat{Ring}.
  \end{equation*}

  Compare this result to \fullref{thm:group_abelianization_universal_property}.
\end{theorem}
\begin{proof}
  This is a refinement of \fullref{thm:group_abelianization_universal_property}, and we only need to show that \( R / [R, R] \) is a commutative ring. For \( x \) and \( y \) in \( R \), since \( yx - xy \in I \), we have
  \begin{equation*}
    (x + I) (y + I)
    =
    (xy + I)
    =
    (xy + yx - xy + I)
    =
    (yx + I)
    =
    (y + I) (x + I).
  \end{equation*}
\end{proof}

\begin{definition}\label{def:multiplicative_set_in_semiring}\mcite[428]{Knapp2016BasicAlgebra}
  We call the subset of the semiring \( R \) a \term{multiplicative set} if it contains \( 1_R \) and, furthermore, it is closed under multiplication.
\end{definition}

\begin{proposition}\label{thm:complement_of_prime_ideal}
  The \hyperref[def:semiring_ideal]{ideal} \( P \) in the \hyperref[def:semiring/commutative]{commutative semiring} \( R \) is \hyperref[def:semiring_ideal/prime]{prime} if and only if \( R \setminus P \) is a \hyperref[def:multiplicative_set_in_semiring]{multiplicative set}.

  Not all multiplicative sets are obtained as complements of prime ideals --- see \fullref{ex:def:ring_localization/powers_of_two}.
\end{proposition}
\begin{proof}
  By \fullref{thm:def:semiring_ideal/ideal_containing_unit}, \( P \) is a proper ideal if and only if \( 1_R \in R \setminus P \).

  By \fullref{thm:def:semiring_ideal/prime_pointwise}, \( P \) is prime if and only if \( x, y \in R \setminus P \) implies \( xy \in R \setminus P \).
\end{proof}

\begin{definition}\label{def:ring_localization}\mcite[428]{Knapp2016BasicAlgebra}
  Let \( R \) be a \hyperref[def:ring/commutative]{commutative ring} and let \( S \subseteq R \) be a \hyperref[def:multiplicative_set_in_semiring]{multiplicative set}.

  Define the equivalence relation \( (r, s) \sim (r', s') \) on \( R \times S \) to hold if and only if there exists some \( u \in S \) such that \( u r s' = u r' s \).

  Consider the set
  \begin{equation*}
    S^{-1} R \coloneqq R \times S / \sim,
  \end{equation*}
  whose cosets we will denote by \( \ifrac r s \) rather than \( [(r, s)] \).

  Define on \( S^{-1} R \) the operations
  \begin{align*}
    \frac a b + \frac c d     &\coloneqq \frac {a d + b c} {b d}, \\
    \frac a b \cdot \frac c d &\coloneqq \frac {a c} {b d},
  \end{align*}
  and the canonical inclusion
  \begin{equation*}
    \begin{aligned}
      &\iota: R \to S^{-1} R \\
      &\iota(r) \coloneqq \frac r {1_R}.
    \end{aligned}
  \end{equation*}

  This ring is called the \term{localization} of \( R \) with respect to \( A \); we denote it by \( S^{-1} R \). In case \( S \) is the \hyperref[thm:boolean_algebra_of_subsets/complement]{complement} of a \hyperref[def:semiring_ideal/prime]{prime ideal}, we may denote the localization by \( R_P \) (or \( R_p \) if \( P = \braket{ p } \)).

  The image under \( \iota \) of every element \( s \) of \( S \) is invertible in \( S^{-1} R \), and we call the inverse \( \ifrac {1_R} s \) the \term{reciprocal} of \( s \).

  This construction is very similar to the \hyperref[def:monoid_grothendieck_completion]{Grothendieck completion} of a monoid or semiring, although with notable differences --- the set \( S \) may be a strict subset of \( R \), and addition in the Grothendieck completion corresponds to multiplication in the localization, while addition in the completion has no analogy.
\end{definition}
\begin{defproof}
  The proof that \( {\sim} \) is an equivalence relation is the same as in \fullref{def:monoid_grothendieck_completion}. The result is then a ring if the operations are well-defined.

  We will show that both operations are well-defined. Let \( u ab' = u a'b \), meaning that \( (a, b) \sim (a', b') \) and hence \( \ifrac a b = \ifrac {a'} {b'} \), and let \( v cd' = v c'd \).

  For addition, we have
  \begin{align*}
    u v (ad + bc) b' d'
    &=
    v dd' (u ab') + u bb' (v cd')
    = \\ &=
    v dd' (u a'b) + u bb' (v c'd)
    = \\ &=
    u v (a'd' + b'c') b d,
  \end{align*}
  hence \( (ad + bc, bd) \sim (a'd' + b'c', b'd') \).

  The proof for correctness of multiplication is the same as our proof of correctness of addition in \fullref{def:monoid_grothendieck_completion}.
\end{defproof}

\begin{theorem}[Ring localization universal property]\label{thm:ring_localization_universal_property}\mcite[431]{Knapp2016BasicAlgebra}
  The \hyperref[def:ring_localization]{localization} of \( R \) by \( S \) satisfies the following \hyperref[rem:universal_mapping_property]{universal mapping property}:
  \begin{displayquote}
    For every commutative ring \( T \) and every ring homomorphism \( \varphi: R \to T \) such that \( \varphi(s) \) is invertible in \( T \) for every \( s \in S \), \( \varphi \) \hyperref[def:factors_through]{uniquely factors through} \( S^{-1} R \). More precisely, there exists a unique ring homomorphism \( \widetilde{\varphi}: S^{-1} R \to T \) such that the following diagram commutes:
    \begin{equation}\label{eq:thm:ring_localization_universal_property/diagram}
      \begin{aligned}
        \includegraphics[page=1]{output/thm__ring_localization_universal_property}
      \end{aligned}
    \end{equation}
  \end{displayquote}
\end{theorem}
\begin{proof}
  The condition suggests the definition
  \begin{equation*}
    \widetilde{\varphi}\parens*{ \frac r s } \coloneqq \varphi(r) \varphi(s)^{-1}.
  \end{equation*}
\end{proof}

\begin{example}\label{ex:def:ring_localization}
  We list several examples of \hyperref[def:ring/commutative]{commutative ring} \hyperref[def:ring_localization]{localization}.

  \begin{thmenum}
    \thmitem{ex:def:ring_localization/zero} If \( S \) contains \( 0_R \), then \( S^{-1} R \) is the trivial ring.

    \thmitem{ex:def:ring_localization/powers_of_two} The localization \( S^{-1} \BbbZ \) by the set \( S \coloneqq \set{ 2^n \given n \geq 0 } \) is (a ring isomorphic to) the rational numbers with denominators that are powers of two. This is an example of a multiplicative set that is not the complement of a prime ideal.

    This ring is isomorphic to the ring \( \BbbZ[\ifrac 1 2] \) obtained by \hyperref[thm:adjoining_elements_to_semiring]{adjoining} the rational number \( \ifrac 1 2 \) to \( \BbbZ \).

    \thmitem{ex:def:ring_localization/prime_number} Let \( p \) be a \hyperref[def:prime_number]{prime number}. The localization \( S^{-1} \BbbZ \) by \( S \coloneqq \BbbZ \setminus \braket{ p } \) is (a ring isomorphic to) the rational numbers with denominators coprime to \( p \).

    For \( p = 2 \), this localization consists of rational numbers whose denominator is an odd number.
  \end{thmenum}
\end{example}

\begin{proposition}\label{thm:def:ring_localization}
  \hyperref[def:ring_localization]{Ring localization} has the following basic properties:

  \begin{thmenum}
    \thmitem{thm:def:ring_localization/image_of_ideal}\mcite[432]{Knapp2016BasicAlgebra} Localization preserves \hyperref[def:semiring_ideal]{ideals}. More precisely, given a commutative ring \( R \), a multiplicative set \( S \) and an ideal \( I \), the set
    \begin{equation*}
      S^{-1} I \coloneqq \set*{ \frac r s \given* r \in I \T{and} s \in S }
    \end{equation*}
    is an ideal of the localization \( S^{-1} R \).

    \thmitem{thm:def:ring_localization/prime_ideals}\mcite[exer. 4.3]{КоцевСидеров2016} The map \( I \mapsto S^{-1} I \) is a \hyperref[def:order_homomorphism/isomorphism]{strict order isomorphism} between the set of \hyperref[def:ring/submodel]{prime ideals} of \( R \) not intersecting \( S \) and the set of all prime ideals of \( S^{-1} R \).

    \thmitem{thm:def:ring_localization/by_prime_ideal}\mcite[exer. 4.2a)]{КоцевСидеров2016} The localization \( R_P \) by a \hyperref[def:semiring_ideal/prime]{prime ideal} \( P \) has a unique maximal ideal \( S^{-1} P \) (here \( S \coloneqq R \setminus P \)).

    \thmitem{thm:def:ring_localization/injective_inclusion} The canonical inclusion \( \iota: R \to S^{-1} R \) is injective if and only if \( S \) contains no zero divisors.
  \end{thmenum}
\end{proposition}
\begin{proof}
  \SubProofOf{thm:def:ring_localization/image_of_ideal} Trivial since \( S \) is closed under multiplication.

  \SubProofOf{thm:def:ring_localization/prime_ideals} Let \( P \) be a prime ideal in \( R \) disjoint from \( S \). By \fullref{thm:def:ring_localization/image_of_ideal}, \( S^{-1} P \) is an ideal of \( S^{-1} R \). If the product \( \ifrac {ac} {bd} \) belong to \( S^{-1} P \), then \( ac \in P \) and \( bd \in S \). Since \( P \) is prime, \( a \in P \) or \( c \in P \). If \( a \in P \), then \( ba \in P \) and \( \ifrac a d = \ifrac {ba} {bd} \in S^{-1} P \); if \( c \in P \), we proceed analogously. Thus, \( S^{-1} P \) is a prime ideal, i.e. the image under \( I \mapsto S^{-1} I \) of a prime ideal is a prime ideal.

  \SubProofOf[def:function_invertibility/injective/equality]{injectivity} Let \( S^{-1} P = S^{-1} Q \) for prime ideals \( P \) and \( Q \) disjoint from \( S \). Suppose that \( P \setminus Q \) contains at least one element, say \( p \). Then \( \iota(p) = \ifrac p 1 \) belongs to both \( S^{-1} P \) and \( S^{-1} Q \); hence, \( Q \) contains an element \( q \) such that, for some \( s \in S \) and \( u \in S \),
  \begin{equation*}
    p \cdot s \cdot u = 1 \cdot q \cdot u.
  \end{equation*}

  Since \( Q \) is an ideal, \( qu \in Q \), and hence \( psu \in Q \). But neither \( p \), \( s \) nor \( u \) belong to \( Q \), which contradicts the assumption that \( Q \) is prime. Therefore, \( P \setminus Q \) is empty. Generalizing, we obtain that \( I \mapsto S^{-1} I \) is injective on prime ideals.

  \SubProofOf[def:function_invertibility/surjective/existence]{surjectivity} Fix a prime ideal \( T \) in \( S^{-1} R \) and let \( P \) be the set of numerators in \( T \), i.e. if \( \ifrac p s \in T \), then \( p \in P \). We will show that \( P \) is a prime ideal; clearly \( T = S^{-1} P \).

  Clearly \( 0_R \in P \). Let \( a, c \in P \). Then there exist \( b, d \in S \) such that \( \ifrac a b \) and \( \ifrac c d \) belong to \( T \). But \( T \) is closed under multiplication with members of \( R \), hence \( \ifrac a {1_R} = b (\ifrac a b) \) and \( \ifrac c {1_R} = d (\ifrac c d) \) also belong to \( T \). Then their sum \( \ifrac {a + c} 1 \) belongs to \( T \), and hence also to \( P \). Thus, \( P \) is closed under addition. We analogously obtain that it is closed under multiplication.

  We have shown that \( P \) is an ideal in \( R \). We must show that it is a prime ideal. Let \( ac \in P \). Then
  \begin{equation*}
    \frac a b \cdot \frac c d \in T
  \end{equation*}
  for some \( b, d \in S \). Hence, \( \ifrac a b \) or \( \ifrac c d \) belongs to \( T \), implying that \( a \in P \) or \( c \in P \).

  \SubProofOf[def:order_homomorphism]{monotonicity} Follows from \fullref{thm:order_embedding_is_strict}.

  \SubProofOf{thm:def:ring_localization/by_prime_ideal} In the localization \( R_P \) be a prime ideal, all members of \( P \) become invertible. Hence, a maximal ideal cannot contain members of \( P \). By \fullref{thm:def:ring_localization/image_of_ideal}, \( S^{-1} P \) is an ideal, therefore it must be the largest proper ideal.

  \SubProofOf{thm:def:ring_localization/injective_inclusion} Let \( sr = 0 \) for \( s \in S \). Then \( \iota(s) = \ifrac s {1_R} \) is invertible in \( S^{-1} R \) and hence
  \begin{equation*}
    \frac {0_R} {1_R}
    =
    \frac {sr} {1_R}
    =
    \frac {1_R} s \cdot \frac {sr} {1_R}
    =
    \frac r {1_R}.
  \end{equation*}

  Hence, \( \iota(r) = \iota(0_R) \).

  It follows that \( \iota \) is injective if and only if \( S \) contains no zero divisors.
\end{proof}

\begin{definition}\label{def:division_ring}\mcite[144]{Knapp2016BasicAlgebra}
  If every nonzero element of a ring is \hyperref[def:divisibility/unit]{invertible}, we call it a \term{division ring}.
\end{definition}

\begin{proposition}\label{thm:division_ring_is_entire}
  A nontrivial \hyperref[def:division_ring]{division ring} is \hyperref[def:entire_semiring]{entire}.
\end{proposition}
\begin{proof}
  Let \( xy = 0 \). If \( x \) is nonzero, multiplying both sides by \( x^{-1} \), we obtain \( y = 0 \). Analogously, \( y \neq 0 \) implies that \( x = 0 \). In all cases, either \( x \) or \( y \) is necessarily zero.

  Therefore, the ring has no nontrivial zero divisors.
\end{proof}

\begin{definition}\label{def:field}
  We will call the \hyperref[def:ring/trivial]{nontrivial} \hyperref[def:ring]{ring} \( \BbbK \) a \term{field} if any of the following equivalent conditions hold:
  \begin{thmenum}
    \thmitem{def:field/simple} \( \BbbK \) is \hyperref[def:ring/commutative]{commutative} and \hyperref[def:simple_object]{simple}.
    \thmitem{def:field/division_ring} \( \BbbK \) is a \hyperref[def:ring/commutative]{commutative} \hyperref[def:division_ring]{division ring}.
  \end{thmenum}

  Fields have the following metamathematical properties:
  \begin{thmenum}
    \thmitem{def:field/theory} We can construct a \hyperref[def:first_order_theory]{first-order theory} for fields by adding to the \hyperref[def:semiring/theory]{theory of rings} the axioms \( \neg (0 \doteq 1) \) and
    \begin{equation}\label{eq:def:field/theory/invertibility}
      (\xi \doteq 0) \vee \qexists \eta (\xi \cdot \eta \doteq 1).
    \end{equation}

    These axioms are not \hyperref[def:positive_formula]{positive formulas}, hence fields automatically get worse metamathematical properties than rings, for example.

    \thmitem{def:field/homomorphism}\mcite[453]{Knapp2016BasicAlgebra} A \hyperref[def:first_order_homomorphism]{first-order homomorphism} between fields is simply a \hyperref[def:ring/homomorphism]{unital ring homomorphism}.

    \thmitem{def:field/submodel} If for two fields \( \Bbbk \) and \( \BbbK \) are have \( \Bbbk \subseteq \BbbK \), we say that \( \BbbK \) is a \term{field extension} of \( \Bbbk \) and that \( \Bbbk \) is a \term{subfield} of \( \BbbK \). In particular, if \( \BbbK = \Bbbk \), we say that the extension is trivial.

    \thmitem{def:field/category} The category of \hyperref[def:large_and_small_sets]{\( \mscrU \)-small} fields \( \ucat{Field} \) is a full subcategory of \hyperref[def:ring/category]{\( \ucat{CRing} \)} with objects restricted to fields.
  \end{thmenum}
\end{definition}
\begin{defproof}
  The equivalence of definitions follows from \fullref{thm:def:semiring_ideal/units}.
\end{defproof}

\begin{proposition}\label{thm:field_of_fractions}
  Let \( D \) be an \hyperref[def:integral_domain]{integral domain}. The \hyperref[def:ring_localization]{localization} of \( D \) at the zero ideal \( \set{ 0_R } \) is a \hyperref[def:field]{field}, which we call the \term{field of fractions} of \( D \).
\end{proposition}
\begin{proof}
  By \fullref{thm:def:ring_localization/prime_ideals}, the localization by the prime ideal \( \set{ 0_R } \) has only one maximal ideal --- \( S^{-1} \set{ 0_R } \). Since \( 0_R \) is absorbing, \( S^{-1} \set{ 0_R } \) is again the zero ideal. Therefore, it is the only proper ideal of the localization \( S^{-1} D \), and hence the localization is a \hyperref[def:simple_object]{simple ring}.

  Since \( D \) is an integral domain, by \fullref{thm:def:ring_localization/injective_inclusion}, \( S^{-1} D \) is a superring of \( D \). It is therefore a nontrivial commutative simple ring, and thus it satisfies \fullref{def:field/simple}.
\end{proof}

\begin{theorem}[Field of fractions universal property]\label{thm:field_of_fractions_universal_property}
  The \hyperref[thm:field_of_fractions]{field of fractions} \( \BbbK \) of the integral domain \( D \) satisfies the following \hyperref[rem:universal_mapping_property]{universal mapping property}:
  \begin{displayquote}
    For every field \( \BbbL \) and every ring homomorphism \( \varphi: D \to \BbbL \), \( \varphi \) \hyperref[def:factors_through]{uniquely factors through} \( \BbbK \). More precisely, there exists a unique field homomorphism \( \widetilde{\varphi}: \BbbK \to \BbbL \) such that the following diagram commutes:
    \begin{equation}\label{eq:thm:field_of_fractions_universal_property/diagram}
      \begin{aligned}
        \includegraphics[page=1]{output/thm__field_of_fractions_universal_property}
      \end{aligned}
    \end{equation}
  \end{displayquote}
\end{theorem}
\begin{proof}
  This is simply a special case of \fullref{thm:ring_localization_universal_property}.
\end{proof}

\begin{definition}\label{def:rational_function_field}
  The field of \term{rational algebraic functions} \( D(\mscrX) \) for the set of indeterminates \( \mscrX \) over the \hyperref[def:integral_domain]{integral domain} \( D \) is the \hyperref[thm:field_of_fractions]{field of fractions} of the \hyperref[def:polynomial_algebra]{polynomial ring} \( D[\mscrX] \).

  Despite the name, elements of the field of fractions are not actually functions, but merely formal expressions. In particular, an analog of \fullref{thm:polynomial_algebra_universal_property} does not really make sense.
\end{definition}

\begin{proposition}\label{thm:adjoining_elements_to_field}
  Let \( \Bbbk \subseteq \BbbK \) be \hyperref[def:field]{fields} and let \( A \) be an arbitrary subset of \( \BbbK \).

  Let \( \Bbbk[A] \) be the ring obtained by adjoining the elements of \( A \) to \( \Bbbk \) as described in \fullref{thm:adjoining_elements_to_semiring}. The \hyperref[thm:field_of_fractions]{field of fractions} of \( \Bbbk[A] \) is the smallest field extension of \( \Bbbk \) containing \( A \).

  We denote this extension by \( \Bbbk(A) \). It should not be confused with the image of the evaluation homomorphism on the \hyperref[def:rational_function_field]{field of rational functions}; the rational functions do not actually have an evaluation homomorphism.
\end{proposition}
\begin{proof}
  It follows from \fullref{thm:adjoining_elements_to_semiring} that \( \Bbbk[A] \) is the smallest superring of \( \Bbbk \) containing \( A \). By \fullref{thm:field_of_fractions_universal_property}, \( \Bbbk(A) \) is the smallest field containing \( \Bbbk[A] \).
\end{proof}

  \subsection{Modules}\label{subsec:modules}

\begin{definition}\label{def:module}
  A \term{module} is a \hyperref[def:semimodule]{semimodule} over a \hyperref[def:ring]{ring} rather than a \hyperref[def:semiring]{semiring}.

  This makes the identity law \eqref{eq:def:semimodule/operation/scalar_multiplication_action/identity} redundant.

  Modules have the following metamathematical properties:
  \begin{thmenum}
    \thmitem{def:module/theory} The first-order theory is identical to the \hyperref[def:semimodule/theory]{theory of semimodules}.

    \thmitem{def:module/homomorphism} A \hyperref[def:first_order_homomorphism]{first-order homomorphism} between two \( R \)-modules \( M \) and \( N \) is simply a \hyperref[def:semimodule/homomorphism]{linear map}.

    \thmitem{def:module/submodel} The set \( A \subseteq M \) is a \hyperref[thm:substructure_is_model]{submodel} of \( M \) if it is a sub-semimodule of \( M \), i.e. a subgroup of \( M \) that is closed under scalar multiplication. We say that \( A \) is a \term{submodule} of \( M \).

    As a consequence of \fullref{thm:positive_formulas_preserved_under_homomorphism}, the image of a module homomorphism is a submodule of its range.

    \thmitem{def:module/trivial} The \hyperref[rem:trivial_structure]{trivial} \( R \)-module is the \hyperref[rem:pointed_set/trivial]{trivial pointed set} \( \set{ 0 } \).

    We sometimes denote the zero of the module via \( \vect 0 \).

    \thmitem{def:module/initial} The \hyperref[thm:substructures_form_complete_lattice/bottom]{initial substructure} of any \( R \)-module is isomorphic to the trivial \( R \)-module \( \set{ \vect 0 } \).

    \thmitem{def:module/bimodule} A \term{bimodule} is simply a \hyperref[def:semimodule/bisemimodule]{bisemimodule} over a ring.

    \thmitem{def:module/category} For a fixed ring \( R \), we denote the \hyperref[def:category_of_small_first_order_models]{category of \( \mscrU \)-small models} by \( \ucat{Mod}_R \).

    It is a very well-behaved category, even more than the category \hyperref[def:group/category]{\( \ucat{Grp} \)} of \( \mscrU \)-small groups.
    \begin{itemize}
      \item The trivial module \( \set{ 0 } \) is a zero object. Therefore, we can define kernels and cokernels, and cokernels for modules are particularly simple.

      \item The \hyperref[def:free_semimodule]{free semimodules} over a ring are modules, and \fullref{thm:free_semimodule_universal_property} ensures that this is left adjoint to the forgetful functor \( U: \ucat{Mod}_R \to \ucat{Set} \). Therefore, by \fullref{thm:first_order_categorical_invertibility}, the monomorphisms are exactly the injective homomorphisms, and that the \hyperref[def:subobject_and_quotient]{categorical subobjects} correspond to submodules.

      \item Every epimorphism in \( \ucat{Mod}_R \) is surjective. This will be proved in \fullref{thm:module_epimorphisms_are_surjective}. Along with \fullref{thm:group_epimorphisms_are_normal}, this shows that the \hyperref[def:subobject_and_quotient]{categorical quotient objects} correspond to \hyperref[def:module/quotient]{quotient modules}, which we will define shortly.
    \end{itemize}

    \thmitem{def:module/kernel} The \term{kernel} of an \( R \)-module homomorphism \( \varphi: M \to N \) is its \hyperref[def:zero_locus]{zero locus} \( \varphi^{-1}(0_N) \). This is a submodule of \( M \). It is precisely the kernel of the underlying group in the sense of \fullref{def:group/kernel}, and the \hyperref[def:zero_morphisms/cokernel]{categorical kernel} in the category of modules.

    \thmitem{def:module/quotient} The \hyperref[def:zero_morphisms/cokernel]{categorical cokernel} of an \( R \)-homomorphism \( \varphi: M \to N \) in the category \( \cat{Mod}_R \) is simply the additive \hyperref[def:group/quotient]{quotient group} \( N / \img \varphi \). The quotient group is again a module over \( R \) because \( N \) is closed under scalar multiplication and, for every coset \( x + N \),
    \begin{equation*}
      r(x + N) = rx + rN = rx + N
    \end{equation*}
    is again a coset in \( N / \img \varphi \).

    In particular, given a submodule \( N \) of \( M \), we can form the \term{quotient module} \( M / N \). In practice, quotients are conveniently characterized by \fullref{thm:quotient_module_universal_property}.

    \thmitem{def:module/simple} Analogously to \hyperref[def:group/simple]{simple groups}, if the only proper \hyperref[def:module/submodel]{submodule} of \( R \) is the \hyperref[def:module/trivial]{trivial module} \( \set{ 0_M } \), we say that \( M \) is a \term{simple module}.

    The trivial module itself is not simple, because it has no proper ideals.
  \end{thmenum}
\end{definition}

\begin{proposition}\label{thm:abelian_group_is_module}
  We have an \hyperref[rem:category_similarity/isomorphism]{isomorphism of categories} \( \hyperref[def:abelian_group]{\cat{Ab}} \cong \hyperref[def:module]{\cat{Mod}_\BbbZ} \).

  More concretely, every abelian group \( G \) is a left module over \( \BbbZ \) with scalar multiplication given by \hyperref[rem:additive_magma/multiplication]{recursively defined multiplication}
  \begin{equation}\label{eq:thm:abelian_group_is_module/operation}
    \begin{aligned}
      &\cdot: \BbbZ \times G \to G \\
      &n \cdot x \coloneqq \begin{cases}
        0_G,           &n = 0, \\
        n \cdot x + x, &n > 1, \\
        -(n \cdot x),  &n < 1.
      \end{cases}
    \end{aligned}
  \end{equation}

  Conversely, in every module over \( \BbbZ \), scalar multiplication matches the recursively defined multiplication.

  Compare this result to \fullref{thm:commutative_monoid_is_semimodule}.
\end{proposition}
\begin{proof}
  Simple refinement of \fullref{thm:commutative_monoid_is_semimodule}.
\end{proof}

\begin{theorem}[Quotient module universal property]\label{thm:quotient_module_universal_property}
  For every \( R \)-\hyperref[def:module]{module} \( M \) and every submodule \( N \) of \( M \), the \hyperref[def:module/quotient]{quotient module} \( R / I \) has the following \hyperref[rem:universal_mapping_property]{universal mapping property}:
  \begin{displayquote}
    Every \( R \)-module homomorphism \( \varphi: M \to K \) satisfying \( N \subseteq \ker \varphi \) \hyperref[def:factors_through]{uniquely factors through} \( M / N \). That is, there exists a unique \( R \)-module homomorphism \( \widetilde{\varphi}: M / N \to K \), such that the following diagram commutes:
    \begin{equation}\label{eq:thm:quotient_module_universal_property/diagram}
      \begin{aligned}
        \includegraphics[page=1]{output/thm__quotient_module_universal_property.pdf}
      \end{aligned}
    \end{equation}

    In the case where \( N = \ker \varphi \), \( \widetilde{\varphi} \) is an \hyperref[def:first_order_embedding]{embedding}.
  \end{displayquote}

  Compare this result to \fullref{thm:quotient_group_universal_property} and \fullref{thm:quotient_algebra_universal_property}.
\end{theorem}
\begin{proof}
  Simple refinement of \fullref{thm:quotient_group_universal_property}.
\end{proof}

\begin{theorem}[Quotient submodule lattice theorem]\label{thm:quotient_submodule_lattice_theorem}
  Given a \hyperref[def:module/submodel]{submodule} \( N \) of \( M \), the function \( K \mapsto K / N \) is a \hyperref[def:semilattice/homomorphism]{lattice isomorphism} between the \hyperref[thm:substructures_form_complete_lattice]{lattice of submodules} of \( M \) containing \( N \) and the lattice of submodules of the \hyperref[def:module/quotient]{quotient} \( M / N \).

  Compare this result to \fullref{thm:quotient_subgroup_lattice_theorem} and \fullref{thm:quotient_ideal_lattice_theorem}.
\end{theorem}
\begin{proof}
  Simple refinement of \fullref{thm:quotient_subgroup_lattice_theorem}.
\end{proof}

\begin{proposition}\label{thm:module_epimorphisms_are_surjective}
  Every \hyperref[def:morphism_invertibility/right_cancellative]{epimorphism} in \hyperref[def:group/category]{\( \cat{Mod}_R \)} is \hyperref[def:function_invertibility/surjective]{surjective}.
\end{proposition}
\begin{proof}
  Let \( \varphi: M \to N \) be an \( R \)-module epimorphism. Consider the canonical projection \( \pi: N \to N / \img \varphi \) and the zero morphism \( z: N \to N / \img \varphi \). Clearly
  \begin{equation*}
    \pi \bincirc \varphi = z \bincirc \varphi,
  \end{equation*}
  and thus \( \pi = z \) is the zero morphism.

  By, \fullref{thm:def:group/kernel_cokernel_compatibility}, \( \ker \pi = \img \varphi \), and since \( \ker \pi = N \), it follows that \( \varphi \) is a surjective function.
\end{proof}

\begin{definition}\label{def:module_presentation}
  A \term{presentation} of the \( R \)-\hyperref[def:module]{module} \( M \) is an \hyperref[def:module/homomorphism]{epimorphism} \( \varphi: R^{\oplus A} \to M \), where \( R^{\oplus A} \) is a \hyperref[def:free_semimodule]{free semimodule}.

  By \fullref{thm:quotient_module_universal_property},
  \begin{equation*}
    M = \img \varphi \cong R^{\oplus A} / \ker \varphi.
  \end{equation*}

  Analogously to \hyperref[def:group_presentation]{group presentations}, we say that \( M \) is finitely generated/related/presented if there exists an appropriate presentation.
\end{definition}

\begin{proposition}\label{thm:module_presentation_existence}
  Every module has at least one \hyperref[def:module_presentation]{presentation}.

  Compare this to \fullref{thm:group_presentation_existence} and \fullref{thm:algebra_presentation_existence}.
\end{proposition}
\begin{proof}
  This can be proven analogously to \fullref{thm:group_presentation_existence}.
\end{proof}

\begin{definition}\label{def:linear_dependence}\mimprovised
  Let \( M \) be an \( R \)-\hyperref[def:module]{module} and fix a subset \( E \subseteq M \). We say that the elements of \( E \) are \term{linearly independent} if any of the following conditions hold:

  \begin{thmenum}
    \thmitem{def:linear_dependence/direct} A \hyperref[rem:linear_combinations]{linear combination} in \( E \) sums to zero if and only if it is \hyperref[def:free_semimodule]{trivial}.

    \thmitem{def:linear_dependence/evaluation} The \hyperref[thm:free_semimodule_universal_property]{linear combination evaluation map}
    \begin{equation*}
      \begin{aligned}
        &\Phi_E: R^{\oplus E} \to M \\
        &\Phi_E( \seq{ t_e }_{e \in E} ) \coloneqq \sum_{e \in E} t_e e
      \end{aligned}
    \end{equation*}
    is \hyperref[def:function_invertibility/injective]{injective}.
  \end{thmenum}

  Unsurprisingly, if the elements of \( E \) are not \term{linearly independent}, we say that they are \term{linearly dependent}.

  Compare this concept to \hyperref[def:algebraic_dependence]{algebraic dependence}.
\end{definition}
\begin{defproof}
  \ImplicationSubProof{def:linear_dependence/direct}{def:linear_dependence/evaluation} Suppose that only the trivial linear combination of \( E \) sums to zero. If \( \sum_{e \in E} t_e e = \sum_{e \in E} r_e e \), then
  \begin{equation*}
    \sum_{e \in E} t_e e - \sum_{e \in E} r_e e
    \reloset {\eqref{eq:def:semiring/left_distributivity}} =
    \sum_{e \in E} (t_e - r_e) e
    =
    0,
  \end{equation*}
  implying that \( t_e = r_e \) for every \( e \in E \).

  Hence, \( \Phi_E \) is injective.

  \ImplicationSubProof{def:linear_dependence/evaluation}{def:linear_dependence/direct} Trivial.
\end{defproof}

\begin{remark}\label{rem:linear_dependence_ease_ring}
  Like all concepts related to \hyperref[rem:linear_combinations]{linear combinations}, linear dependence may behave differently depending on the underlying ring. For example, every irrational number is a linear combination of itself, but it is not a linear combination of rational numbers.

  For simplicity, we will not specify the ring explicitly unless this may cause confusion.
\end{remark}

\begin{proposition}\label{thm:def:linear_dependence}
  \hyperref[def:linear_dependence]{Linear (in)dependence} in the \( R \)-module \( M \) has the following basic properties:
  \begin{thmenum}
    \thmitem{thm:def:linear_dependence/zero} The zero vector \( 0_M \) is by itself linearly dependent.

    \thmitem{thm:def:linear_dependence/monotonicity} If \( E \) is a linearly \hi{dependent} set and \( E \subseteq F \), then \( F \) is also a linearly dependent set.

    \thmitem{thm:def:linear_dependence/antimonotonicity} If \( F \) is a linearly \hi{independent} set and \( E \subseteq F \), then \( E \) is also a linearly independent set.

    \thmitem{thm:def:linear_dependence/dependent_combination} The set \( E \cup \set{ x } \) is linearly dependent if and only if \( x \in \linspan E \).

    This may not hold for more general modules.

    \thmitem{thm:def:linear_dependence/span_dependent} For any set of vectors \( E \), if \( x \in \linspan E \setminus E \), then \( E \cup \set{ x } \) is a linearly dependent set.

    A partial converse is stated in \fullref{thm:def:vector_space/span_independent}.
  \end{thmenum}
\end{proposition}
\begin{proof}
  \SubProofOf{thm:def:linear_dependence/zero} Trivial.

  \SubProofOf{thm:def:linear_dependence/monotonicity} Trivial.

  \SubProofOf{thm:def:linear_dependence/antimonotonicity} Trivial.

  \SubProofOf{thm:def:linear_dependence/span_dependent} By \fullref{thm:span_via_linear_combinations}, there exists a linear combination of members of \( E \) such that
  \begin{equation*}
    x = \sum_{k=1}^n t_k x_k.
  \end{equation*}

  If \( x = 0_M \), then \( E \cup \set{ x } \) is linearly dependent by \fullref{thm:def:linear_dependence/zero} and \fullref{thm:def:linear_dependence/monotonicity}.

  If \( x \neq 0_M \), then \( E \cup \set{ x } \) is linearly dependent because \( x \) is a nontrivial linear combination of other vectors of \( E \).
\end{proof}

\begin{example}\label{ex:def:linear_dependence}
  We list several (counter)examples for \hyperref[def:linear_dependence]{linear dependence}:
  \begin{thmenum}
    \thmitem{ex:def:linear_dependence/not_in_span} The columns
    \begin{equation*}
      \begin{pmatrix}
        0 \\ 1 \\ 0
      \end{pmatrix}
      \begin{pmatrix}
        1 \\ 1 \\ 0
      \end{pmatrix}
      \begin{pmatrix}
        0 \\ 1 \\ 1
      \end{pmatrix}
    \end{equation*}
  \end{thmenum}
\end{example}

\begin{definition}\label{def:hamel_basis}\mimprovised
  Let \( M \) be a left \( R \)-\hyperref[def:module]{module} and fix a subset \( E \subseteq M \). We say that \( E \) is a \term{Hamel basis} or simply \term{basis} of \( M \) if any of the following equivalent conditions hold:

  \begin{thmenum}
    \thmitem{def:hamel_basis/independent} It is a \hyperref[thm:span_via_linear_combinations]{spanning set} of \hyperref[def:linear_dependence]{linearly independent} elements.

    \thmitem{def:hamel_basis/evaluation} The \hyperref[thm:free_semimodule_universal_property]{linear combination evaluation map}
    \begin{equation*}
      \begin{aligned}
        &\Phi_E: R^{\oplus E} \to M \\
        &\Phi_E( \seq{ t_e }_{e \in E} ) \coloneqq \sum_{e \in E} t_e e
      \end{aligned}
    \end{equation*}
    is \hyperref[def:function_invertibility/bijective]{bijective}.

    \thmitem{def:hamel_basis/free} \( M \) is \hyperref[def:semimodule/homomorphism]{linearly isomorphic} to the \hyperref[def:free_semimodule]{free module} \( R^{\oplus E} \).

    It is established terminology to say that \( M \) is a free module.
  \end{thmenum}
\end{definition}
\begin{defproof}
  \ImplicationSubProof{def:hamel_basis/independent}{def:hamel_basis/evaluation} Suppose that \( E \) is a spanning set of linearly independent elements.

  \Fullref{def:linear_dependence/evaluation} is satisfied, hence \( \Phi_E \) is injective.

  Furthermore, \( E \) is spanning, meaning that \( \linspan E = M \). By \fullref{def:linear_dependence/evaluation}, it is surjective.

  \ImplicationSubProof{def:hamel_basis/evaluation}{def:hamel_basis/free} If \( \Phi_E \) is bijective, then it is a linear isomorphism.

  \ImplicationSubProof{def:hamel_basis/free}{def:hamel_basis/independent} Suppose that \( \Psi: R^{\oplus E} \to M \) is a linear isomorphism. By \fullref{def:linear_dependence/evaluation}, \( \Psi \) is surjective, and hence a spanning set of \( M \). Furthermore, it satisfies \fullref{def:linear_dependence/evaluation}, meaning that \( E \) is a linearly independent set.
\end{defproof}

\begin{example}\label{ex:module_without_basis}
  As a \hyperref[thm:abelian_group_is_module]{\( \BbbZ \)-module}, the additive group of \( \BbbQ \) has no basis.

  Indeed, given any two rational numbers \( \ifrac a b \) and \( \ifrac c d \), we have
  \begin{equation*}
    cb \cdot \frac a b + da \cdot \frac c d = 0.
  \end{equation*}

  Therefore, only singleton sets of rational numbers are linearly independent with respect to \( \BbbZ \). But no single integer generates \( \BbbQ \).
\end{example}

\begin{proposition}\label{thm:def:hamel_basis}
  \hyperref[def:hamel_basis]{Bases} of the \( R \)-module \( M \) have the following basic properties:
  \begin{thmenum}
    \thmitem{thm:def:hamel_basis/span} Every \hyperref[def:linear_dependence]{linearly independent} subset of \( M \) is a basis for its \hyperref[def:semimodule/submodel]{linear span}.

    \thmitem{thm:def:hamel_basis/maximal_independent} Every basis of \( M \) is maximal among \hyperref[thm:def:linear_dependence]{linearly independent} sets.

    The converse holds for vector spaces --- see \fullref{thm:def:vector_space/minimal_spanning}.

    \thmitem{thm:def:hamel_basis/minimal_spanning} Every basis of \( M \) is minimal among spanning sets.

    The converse holds for vector spaces --- see \fullref{thm:def:vector_space/minimal_spanning}.
  \end{thmenum}
\end{proposition}
\begin{proof}
  \SubProofOf{thm:def:hamel_basis/span} Trivial.

  \SubProofOf{thm:def:hamel_basis/maximal_independent} Follows from \fullref{thm:def:linear_dependence/span_dependent}.

  \SubProofOf{thm:def:hamel_basis/minimal_spanning} Let \( E \) be a basis of \( M \) and suppose that \( F \subsetneq E \) is also a spanning set of \( M \).

  Then there exists some vector \( x \in E \setminus F \). Since both sets are spanning, \( \linspan F = \linspan E \). By \fullref{thm:def:linear_dependence/span_dependent}, \( F \cup \set{ x } \) is linearly dependent, and by \fullref{thm:def:linear_dependence/antimonotonicity}, \( E \) is linearly dependent. This contradicts the assumption that \( E \) has a basis.

  Therefore, no proper subset of \( E \) is spanning for \( M \).
\end{proof}

\begin{proposition}\label{thm:basis_of_direct_sum}
  Consider the \( R \)-modules \( \seq{ M_k }_{k \in \mscrK} \). Suppose that \( \seq{ E_k }_{k \in \mscrK} \) are bases of the corresponding modules. Then the set
  \begin{equation*}
    E \coloneqq \bigcup_{k \in \mscrK} E_k
  \end{equation*}
  is a basis for the \hyperref[def:semimodule_direct_product]{direct sum}
  \begin{equation*}
    M \coloneqq \bigoplus_{k \in \mscrK} M_k.
  \end{equation*}
\end{proposition}
\begin{proof}
  Regard \( M_k \) as a subspace of the direct sum \( M \). The subspaces \( M_k \) and \( M_n \) are disjoint for \( k \neq n \). Thus, every vector from \( M \) is a unique sum of vectors from the subspaces.

  Since, for every \( k \in \mscrK \), every vector from \( M_k \) can further be uniquely represented as a linear combination of vectors from \( E_k \), we conclude that every vector from \( M \) is a linear combination of vectors from \( E \).
\end{proof}

\begin{definition}\label{def:basis_decomposition}\mimprovised
  If \( E \) is a basis of the \( R \)-\hyperref[def:module]{module} \( M \), the inverse of the linear isomorphism from \fullref{def:hamel_basis/evaluation} is
  \begin{equation*}
    \begin{aligned}
      &\pi_E: M \to R^{\oplus E} \\
      &\pi_E\parens[\Big]{ \sum_{e \in E} t_e \cdot e } \coloneqq \seq{ t_e }_{e \in E}.
    \end{aligned}
  \end{equation*}

  We denote the \( e \)-th component of this function by \( \pi_e: M \to R \). This is also a linear map, which we call the \term{coordinate projection} for \( e \).

  For every vector \( x \) in \( M \), we thus have
  \begin{equation*}
    x = \sum_{e \in E} \pi_e(x) \cdot e.
  \end{equation*}

  This linear combination is unique, and we call it the \term{decomposition} of \( x \) along \( E \).

  As in \fullref{rem:linear_combinations}, we sometimes take only the basis vectors with nonzero coefficients as \enquote{the} decomposition.
\end{definition}
\begin{defproof}
  The decomposition is indeed unique --- if
  \begin{equation*}
    x = \sum_{e \in E} t_e e = \sum_{e \in E} r_e e,
  \end{equation*}
  then
  \begin{equation*}
    \sum_{e \in E} (t_e - r_e) e = 0_M.
  \end{equation*}

  Since \( E \) is a basis, it is linearly independent and hence \( t_e = r_e \) for all \( e \) in \( E \).
\end{defproof}

\begin{definition}\label{def:semimodule_torsion}\mimprovised
  We say that the semimodule element \( x \) is a \term{torsion element} if there exists some nonzero scalar \( t \) such that \( tx \) is the zero vector. A semimodule without torsion elements is called \term{torsion-free}.
\end{definition}

\begin{example}\label{ex:def:semimodule_torsion}
  We list examples of \hyperref[def:semimodule_torsion]{torsion elements}:
  \begin{thmenum}
    \thmitem{ex:def:semimodule_torsion/scalars} When regarded as a module over itself, every zero divisor of a ring is a torsion element. For example, \( 2 \) and \( 3 \) in \( \BbbZ_6 \).

    \thmitem{ex:def:semimodule_torsion/matrices} The matrix
    \begin{equation*}
      \begin{pmatrix}
        2 & 0 \\
        0 & 3
      \end{pmatrix}
    \end{equation*}
    is a torsion element of the \hyperref[thm:matrix_algebra]{matrix algebra} \( \BbbZ_6^{2 \times 2} \) because \( 2 \cdot 3 = 3 \cdot 4 = 0 \).
  \end{thmenum}
\end{example}

\begin{proposition}\label{thm:basis_implies_torsion_free}
  If the \( R \)-module \( M \) has a \hyperref[def:hamel_basis]{basis}, it is \hyperref[def:semimodule_torsion]{torsion-free}.
\end{proposition}
\begin{proof}
  Suppose that \( tx \) for some nonzero scalar \( t \) and nonzero vector \( x \). Let \( E \) be a basis of \( M \) and let \( x = \sum_{e \in E} \pi_e(x) \cdot e \) be the basis decomposition of \( x \). Then this is a nontrivial linear combination that sums to zero, which contradicts the assumption that \( E \) is a basis.
\end{proof}

\begin{proposition}\label{thm:basis_projection_orthonormal}
  If the \( R \)-module \( M \) has a \hyperref[def:hamel_basis]{basis} \( E \), then the \hyperref[def:basis_decomposition]{projection functionals} satisfy
  \begin{equation}\label{eq:thm:basis_projection_orthonormal}
    \pi_e(f) = \begin{cases}
      1, &e = f, \\
      0, &f \in E \setminus \set{ e }.
    \end{cases}
  \end{equation}
\end{proposition}
\begin{proof}
  We have
  \begin{equation*}
    f = \sum_{e \in B} \pi_e(f) \cdot e.
  \end{equation*}

  Thus,
  \begin{equation*}
    0 = f - \sum_{e \in B} \pi_e(f) \cdot e = (1 - \pi_f(f)) \cdot f - \sum_{e \in E \setminus \set{ f }} \pi_e(f).
  \end{equation*}

  Since the vectors in \( E \) are linearly independent, \eqref{eq:thm:basis_projection_orthonormal} follows.
\end{proof}

\begin{definition}\label{def:sequence_space}
  Let \( R \) be a \hyperref[def:ring/commutative]{commutative ring} and let \( \alpha \) be either a finite \hyperref[def:ordinal]{ordinal} or the \hyperref[thm:omega_is_an_ordinal]{smallest infinite ordinal} \( \omega \). We call the \hyperref[def:free_semimodule]{free module} \( R^\alpha \) a \term{sequence space} or, in case \( \alpha \) is finite, also a \term{coordinate space}.

  The \term{standard basis} of the \( R^\alpha \) is the \hyperref[def:transfinite_sequence]{transfinite sequence} \( \seq{ e_i }_{j < \alpha} \) whose \hyperref[def:basis_decomposition]{projection maps} satisfy
  \begin{equation*}
    \pi_i(e_j) = \begin{cases}
      0, &i = j, \\
      1, &i \neq j.
    \end{cases}
  \end{equation*}

  When \( \alpha \) is finite, the vectors in \( R^\alpha \) are \hyperref[def:sequence]{finite sequences}, and we often conflate them with \hyperref[def:array/column_vector]{column vectors} or, less often, \hyperref[def:array/row_vector]{row vectors}. See \fullref{rem:vector_etymology} for a more detailed discussion of terminology. Similarly, when \( \alpha = \omega \), we conflate the vectors with (infinite) \hyperref[def:sequence]{sequences}.
\end{definition}

\begin{proposition}\label{thm:basis_of_polynomial_ring}
  For a nontrivial commutative ring \( R \), the set
  \begin{equation*}
    \set*{ \prod_{X \in \mscrX} X^{\gamma_X} \given \gamma \T{is a \hyperref[def:multi_index]{multi-index}} }
  \end{equation*}
  of all monomials is a \hyperref[def:hamel_basis]{basis} for \hyperref[def:polynomial_algebra]{polynomial ring} \( R[\mscrX] \).
\end{proposition}
\begin{proof}
  In \fullref{def:polynomial_algebra}, we have defined a polynomial as a \hyperref[def:free_semimodule]{free \( R \)-module} over the set of all monomials.
\end{proof}

\begin{definition}\label{def:vector_space}
  A \term{vector space} is a \hyperref[def:module]{left module} over a \hyperref[def:field]{field}.

  We denote the category of vector spaces over \( \BbbK \) by \( \cat{Vect}_{\BbbK} \).
\end{definition}

\begin{proposition}\label{thm:def:vector_space}
  The \hyperref[def:vector_space]{vector space} \( V \) over \( \BbbK \) has the following basic properties:
  \begin{thmenum}
    \thmitem{thm:def:vector_space/span_independent} For a linearly independent set \( A \), if \( x \in V \setminus \linspan A \), then \( A \cup \set{ x } \) is also a linearly independent set.

    A more general converse holds in \fullref{thm:def:linear_dependence/span_dependent}.

    \thmitem{thm:def:vector_space/maximal_independent} Every maximally \hyperref[thm:def:linear_dependence]{linearly independent} subset of \( V \) is a basis.

    The converse holds more generally --- see \fullref{thm:def:hamel_basis/maximal_independent}.

    \thmitem{thm:def:vector_space/minimal_spanning} Every minimal spanning subset of \( V \) is a basis.

    The converse holds more generally --- see \fullref{thm:def:hamel_basis/minimal_spanning}.

    \thmitem{thm:def:vector_space/expansion} Given a subspace \( U \) of \( V \), if \( A \) is a finite basis of \( U \) and if \( V \) has a finite basis, then we can extend \( A \) to a finite basis of \( V \). That is, there exists a finite basis \( E \) of \( V \) such that \( A \subseteq E \).

    \thmitem{thm:def:vector_space/dimension_lemma} If \( \linspan\set{ a_1, \ldots, a_n } \subseteq \linspan\set{ b_1, \ldots, b_m } \), then \( n \leq m \).
  \end{thmenum}
\end{proposition}
\begin{proof}
  \SubProofOf{thm:def:vector_space/span_independent} Suppose that \( x \in V \setminus \linspan A \) and that \( A \cup \set{ x } \) is linearly dependent, Then there exist coefficients \( t_0, t_1, \ldots, t_n \) such that
  \begin{equation*}
    t_0 x + \sum_{k=1}^n t_k a_k = 0_M.
  \end{equation*}

  If \( t_0 = 0_\BbbK \), then \( \sum_{k=1}^n t_k a_k \in \linspan A \) and thus \( t_1 = \cdots = t_n = 0_\BbbK \).

  Otherwise, we can divide by \( t_0 \) to obtain
  \begin{equation*}
    x = -\sum_{k=1}^n \frac {t_k} {t_0} a_k,
  \end{equation*}
  which implies that \( x \in \linspan A \), contradicting our choice of \( x \).

  Therefore, \( A \cup \set{ x } \) satisfies \fullref{def:linear_dependence}.

  \SubProofOf{thm:def:vector_space/maximal_independent} Let \( A \) be a maximally linearly independent set.

  Suppose that it is not a spanning set and let \( x \in V \setminus \linspan A \). By \fullref{thm:def:vector_space/span_independent}, the set \( A \cup \set{ x } \) is linearly independent, contradicting the maximality of \( A \).

  Therefore, \( A \) is a spanning set for \( V \).

  \SubProofOf{thm:def:vector_space/minimal_spanning} Let \( A \) be a minimal spanning set. Suppose that it is linearly dependent. Then there exist distinct vectors \( x_1, \ldots, x_n \) in \( A \) and scalars \( t_1, \ldots, t_n \), at least one of which is nonzero, such that
  \begin{equation*}
    \sum_{k=1}^n t_k x_k = 0_M.
  \end{equation*}

  Let \( k_0 \) be the smallest index such that \( t_{k_0} \neq 0_R \). Then
  \begin{equation*}
    x_{k_0} = -\sum_{k \neq k_0} \frac {t_k} {t_{k_0}} a_k.
  \end{equation*}

  Then \( \linspan A = \linspan A \setminus \set{ x_{k_0} } \), contradicting the minimality of \( A \).

  \SubProofOf{thm:def:vector_space/expansion} Let \( a_1, \ldots, a_n \) be a basis of \( U \) and let \( b_1, \ldots, b_m \) be a basis of \( V \).

  We use recursion on \( k \leq n \) to build linearly independent sets of the form
  \begin{equation*}
    \set{ a_1, \ldots, a_n, b_{i_1}, \ldots, b_{i_k} },
  \end{equation*}
  at least one of which will be a basis.

  The base case \( k = 0 \) is vacuous. Now suppose that the vectors \( a_1, \ldots, a_n, b_{i_1}, \ldots, b_{i_k} \) are linearly independent. If there exists an index \( i_{k+1} \) distinct from \( i_1, \ldots, i_k \) such that \( b_{i_{k+1}} \) is not in
  \begin{equation*}
    L_k \coloneqq \linspan\set{ a_1, \ldots, a_n, b_{i_1}, \ldots, b_{i_k} },
  \end{equation*}
  then, by \fullref{thm:def:vector_space/span_independent}, the vectors \( a_1, \ldots, a_n, b_{i_1}, \ldots, b_{i_k} \) are linearly independent. Otherwise, every basis vector \( b_1, \ldots, b_n \) belongs to \( L_k \), hence it is a spanning set of \( V \) of linearly independent vectors.

  Since the basis \( b_1, \ldots, b_n \) of \( V \) necessarily belongs to \( L_n \), it follows that \( L_k \) will be a basis of \( V \) for some \( k \leq n \).

  \SubProofOf{thm:def:vector_space/dimension_lemma} We will use induction on \( n \). The base case \( n = 0 \) holds because \( \set{ 0_V } \) is the only submodule of the linear span of zero vectors.

  Suppose that the statement holds for \( n - 1 \) and note that \( b_m \) can be \hyperref[def:basis_decomposition]{decomposed} as
  \begin{equation*}
    b_m = \sum_{k=1}^n t_k a_k.
  \end{equation*}

  Then
  \begin{equation*}
    a_k = \frac 1 {t_{k_0}} b_m - \sum_{k \neq k_0} \frac {t_k} {t_{k_0}} a_{k_0}.
  \end{equation*}

  Thus,
  \begin{equation*}
    \linspan\set{ a_1, \ldots, a_n } = \linspan\set{ a_1, \ldots, a_{k_0 - 1}, a_{k_0 + 1}, \ldots, a_n, b_m }.
  \end{equation*}

  We can thus remove \( b_m \) to obtain the inclusion
  \begin{equation*}
    \linspan\set{ b_1, \ldots, b_{m-1} } \subseteq \linspan\set{ a_1, \ldots, a_{k_0 - 1}, a_{k_0 + 1}, a_n }.
  \end{equation*}

  From the inductive hypothesis, we conclude that \( m - 1 \leq n - 1 \), and hence \( m \leq n \).
\end{proof}

\begin{theorem}[Vector space basis existence]\label{thm:vector_space_basis_existence}
  Every \hyperref[def:vector_space]{vector space} has a \hyperref[def:hamel_basis]{basis}.

  Within \hyperref[def:zfc]{\logic{ZF}}, this theorem is equivalent to the \hyperref[def:zfc/choice]{axiom of choice} --- see \fullref{thm:axiom_of_choice_equivalences/vector_space_bases}.
\end{theorem}
\begin{proof}
  \ImplicationSubProof[thm:zorns_lemma]{Zorn's lemma}[thm:vector_space_basis_existence]{vector space existence} Let \( V \) be a vector space over \( \BbbK \). Let \( \mathcal{B} \) be the family of all linearly independent \hyperref[rem:linear_combinations]{subsets} of \( V \).

  The family \( \mathcal{B} \) is nonempty since any \hyperref[rem:singleton_sets]{singleton} from \( V \) belongs to \( \mathcal{B} \). The union of any chain \( \mathcal{B}' \subseteq \mathcal{B} \) can then contain only linearly independent elements since otherwise we would have that some set in \( \mathcal{B}' \) is not linearly independent. Thus, \fullref{thm:zorns_lemma} shows the existence of a maximal linearly independent set \( E \). By \fullref{thm:def:vector_space/maximal_independent}, \( E \) is a basis.

  \ImplicationSubProof[thm:vector_space_basis_existence]{vector space existence}[def:zfc/choice]{the axiom of choice} Shown in \cite{Blass1984}.
\end{proof}

\begin{proposition}\label{thm:vector_space_dimension}
  All \hyperref[def:hamel_basis]{bases} of a \hyperref[def:vector_space]{vector space} are \hyperref[def:equinumerosity]{equinumerous}.

  We define the \term{dimension} \( \dim V \) of the vector space \( V \) as the \hyperref[thm:cardinality_existence]{cardinality} of any of its bases. By \fullref{thm:vector_space_basis_existence}, every vector space has a dimension.
\end{proposition}
\begin{proof}
  Let \( A \) and \( E \) be bases of \( V \).

  \SubProof{Proof for finite bases} Suppose that \( a_1, \ldots, a_n \) are the vectors of \( A \) and, aiming at a contradiction, suppose that \( E \) is infinite.

  Every vector \( a_i \) of \( A \) can be \hyperref[def:basis_decomposition]{decomposed} along \( E \) as
  \begin{equation*}
    a_i = \sum_{j=1}^{m_i} t^{(i)}_j e^{(i)}_j.
  \end{equation*}
  for appropriate scalars from \( \BbbK \) and vectors from \( E \). Since this can be done for every \( a_i \) in \( A \), we conclude that
  \begin{equation*}
    V = \linspan\set{ a_1, \ldots, a_n } = \linspan\set{ e^{(1)}_1, \ldots, e^{(1)}_{m_1}, \ldots, e^{(n)}_1, \ldots, e^{(n)}_{m_n} }.
  \end{equation*}

  Hence, a finite subset of \( E \) spans \( V \), contradicting \fullref{thm:def:hamel_basis/minimal_spanning}. Therefore, \( E \) must be a finite set.

  Let \( b_1, \ldots, b_m \) be the vectors of \( E \). By applying \fullref{thm:def:vector_space/dimension_lemma} twice, we conclude that \( n \leq m \) and \( m \leq n \), hence \( n = m \).

  \SubProof{Proof for infinite bases} Suppose that both \( A \) and \( E \) are infinite. Let \( S_x \) be the set of vectors in \( E \) with nonzero coefficients in the decomposition of \( x \in V \).
  \begin{itemize}
    \item \( S_x \) is necessarily a finite set.
    \item Every vector \( x \) in \( V \) belongs to \( S_a \) for some \( a \in A \).

    Indeed, \( x \) can be decomposed along \( A \) as
    \begin{equation*}
      x = \sum_{a \in A} \pi_a(x) \cdot a,
    \end{equation*}
    and \( x \in S_{a_k} \) whenever \( \pi(a) \neq 0_\BbbK \).

    \item For every basis vector \( e \) in \( E \), \( S_e = \set{ e } \).

    \item We have
    \begin{equation*}
      E \subseteq \bigcup_{a \in A} S_a
    \end{equation*}
  \end{itemize}

  Therefore,
  \begin{equation*}
    \card(B)
    \leq
    \card\parens[\Bigg]{ \bigcup_{a \in A} S_a }
    \leq
    \card\parens[\Bigg]{ \coprod_{a \in A} S_a }
    \leq
    \card(A \times \omega)
    \leq
    \card(A) \cdot \aleph_0
    \reloset {\ref{thm:simplified_cardinal_arithmetic/infinite}} \leq
    \card(A).
  \end{equation*}

  Since \( A \) and \( E \) were arbitrary bases, we can exchange them to obtain the converse inequality, and thus \( \card(A) = \card(B) \).
\end{proof}

\begin{proposition}\label{thm:commutative_module_rank}
  All \hyperref[def:hamel_basis]{bases} of a \hyperref[def:module]{module} over a nontrivial \hyperref[def:ring/commutative]{commutative ring} are \hyperref[def:equinumerosity]{equinumerous}.

  We define the \term{rank} of a module as the \hyperref[thm:cardinality_existence]{cardinality} of any of its bases. This is a generalization of \hyperref[thm:vector_space_dimension]{vector space dimensions}. Unlike vector space dimensions, however, module ranks may not exist --- see \fullref{ex:module_without_basis}. Modules with at least one basis are often called \term{free}, however we will prefer to use the term for modules with a concrete bases, as in \fullref{def:free_semimodule}.
\end{proposition}
\begin{proof}
  Suppose that \( A \) and \( E \) are bases of the \( R \)-module \( M \). Let \( I \) be a \hyperref[def:semiring_ideal/maximal]{maximal ideal} of \( R \).

  By \fullref{thm:quotient_by_maximal_ideal}, \( \BbbK = R / I \) is a field (this is a forward reference to \fullref{thm:quotient_by_maximal_ideal}). Given an isomorphism \( \Phi: R^{\oplus A} \to R^{\oplus E} \) with components \( \Phi_e: R^{\oplus A} \to R \), define
  \begin{equation*}
    \begin{aligned}
      &\Psi: \BbbK^{\oplus A} \to \BbbK^{\oplus E} \\
      &\Psi\parens[\Big]{ \seq{ t_a + I }_{a \in A} } \coloneqq \seq[\Big]{ \Phi_e(\seq{ t_a }_{a \in A}) + I }_{e \in E}
    \end{aligned}
  \end{equation*}

  This is clearly an isomorphism if it is well-defined. Indeed, suppose we are given \( t_a + I = t_a' + I \) for every \( a \in A \), i.e. \( t_a - t_a' \in I \). Then
  \small
  \begin{equation*}
    \Phi_e(\seq{ t_a }_{a \in A}) - \Phi_e(\seq{ t_a' }_{a \in A})
    =
    \Phi_e(\seq{ t_a }_{a \in A} - \seq{ t_a' }_{a \in A})
    =
    \Phi_e(\seq{ t_a - t_a' }_{a \in A})
    =
    \sum_{a \in A} (\underbrace{ t_a - t_a' }_I) \Phi_e(a)
    \in
    I.
  \end{equation*}
  \normalsize

  Thus,
  \begin{equation*}
    \Phi_e(\seq{ t_a }_{a \in A}) + I = \Phi_e(\seq{ t_a' }_{a \in A}) + I.
  \end{equation*}

  Therefore, \( \Psi \) is a well-defined linear isomorphism. Applying \fullref{thm:vector_space_dimension}, we conclude that \( A \) and \( E \) are equinumerous.
\end{proof}

\begin{corollary}\label{thm:finitely_generated_module_basis}
  For a nontrivial commutative ring \( R \), if a \hyperref[def:module_presentation]{finitely-generated} \( R \)-module has a \hyperref[def:hamel_basis]{basis}, the basis is finite.
\end{corollary}
\begin{proof}
  Follows from \fullref{thm:commutative_module_rank}.
\end{proof}

\begin{example}\label{ex:field_submodules}
  For a nontrivial \hyperref[def:ring/commutative]{commutative ring} \( R \), every ideal of \( R \) is a submodule of the \hyperref[thm:commutative_module_rank]{rank-one} \hyperref[def:module]{module} \( R \). For example, both \( \BbbZ \) and \( 2\BbbZ \) are \( \BbbZ \)-modules of rank one.

  For a \hyperref[def:field]{field} \( \BbbK \), every ideal of \( \BbbK \) is a submodule of the \hyperref[thm:vector_space_dimension]{unidimensional} \hyperref[def:vector_space]{vector space} \( \BbbK \). The only ideals of \( \BbbK \) are the zero ideal, whose dimension is zero, and the field itself, whose dimension is one.
\end{example}

\begin{proposition}\label{thm:modules_with_same_rank_are_isomorphic}
  Two \( R \)-modules having the same \hyperref[thm:commutative_module_rank]{rank} are \hyperref[def:semimodule/homomorphism]{linearly isomorphic}.
\end{proposition}
\begin{proof}
  Suppose that \( A \) is a \hyperref[def:hamel_basis]{basis} of \( M \) and \( E \) is a basis of \( N \). Also suppose that both \( M \) and \( N \) have the same rank.

  Then \( A \) is equinumerous with \( E \) and there exists a bijective function \( f: A \to E \). We can define the map \( \varphi: A \to R^{\oplus E} \) by sending \( a \) to the embedding of \( f(a) \) in \( R^{\oplus E} \). We can then \hyperref[thm:free_semimodule_universal_property]{extend} \( \varphi \) to a linear isomorphism \( \Phi: R^{\oplus A} \to R^{\oplus E} \).

  Therefore, there exists an isomorphism between \( M \) and \( R^{\oplus A} \), between \( R^{\oplus A} \) and \( R^{\oplus E} \) and between \( R^{\oplus E} \) and \( N \). Hence, \( M \) and \( N \) are isomorphic.
\end{proof}

\begin{lemma}\label{thm:vector_space_dimension_monotonicity}
  For vector spaces \( U \subseteq V \) we have \( \rank U \leq \rank V \).
\end{lemma}
\begin{proof}
  Let \( A \) be a basis of \( U \). By \fullref{thm:def:vector_space/expansion}, there exists a basis \( E \) of \( V \) such that \( A \subseteq E \). Then clearly \( \rank U \leq \rank V \).
\end{proof}

\begin{proposition}\label{thm:rank_nullity_via_subspaces}
  Let \( U \) be a subspace of the \hyperref[thm:vector_space_dimension]{finite-dimensional} \hyperref[def:vector_space]{vector space} \( V \). Then
  \begin{equation*}
    V \cong U \oplus (V / U).
  \end{equation*}
\end{proposition}
\begin{proof}
  By \fullref{thm:vector_space_dimension_monotonicity}, \( \dim U \leq \dim V \). Let \( a_1, \ldots, a_m \) be an ordered basis for \( U \). By \fullref{thm:def:vector_space/expansion}, there exist vectors \( a_{m+1}, \ldots, a_n \) such that \( a_1, \ldots, a_n \) is a basis of \( V \). We will show that \( a_{m+1} + U, \ldots, a_n + U \) is a basis of \( V / U \), thus demonstrating the isomorphism.

  First, we must show that it is a spanning set. Fix a vector \( x + U \) from \( V / U \). We know that \( x \) can be decomposed as
  \begin{equation*}
    x = \sum_{k=1}^n \pi_k(x) \cdot a_k.
  \end{equation*}

  Since \( a_1, \ldots, a_m \) belong to \( U \),
  \begin{equation*}
    x + U = \sum_{k={m+1}}^n \pi_k(x) \cdot a_k + U.
  \end{equation*}

  Hence, \( x + U \) can be represented as a linear combination of the vectors \( a_{m+1} + U, \ldots, a_n + U \).

  To show uniqueness, suppose that
  \begin{equation*}
    x + U = \sum_{k={m+1}}^n t_k (a_k + U) = \sum_{k={m+1}}^n r_k (a_k + U).
  \end{equation*}

  Then
  \begin{equation*}
    U = \sum_{k={m+1}}^n (t_k - r_k) (a_k + U).
  \end{equation*}

  Since neither of \( a_{m+1}, \ldots, a_n \) are in \( U \), it follows that \( t_k = r_k \) for \( k = m + 1, \ldots, n \).

  Therefore, \( a_{m+1} + U, \ldots, a_n + U \) is a basis of \( V / U \).
\end{proof}

\begin{proposition}\label{thm:rank_of_direct_sum}
  Assuming that the \( R \)-modules \( M_1, \ldots, M_n \) have bases, the \hyperref[thm:commutative_module_rank]{rank} of their direct sum
  \begin{equation*}
    M_1 \oplus \cdots \oplus M_n
  \end{equation*}
  is the \hyperref[def:cardinal_arithmetic/addition]{sum of cardinals}
  \begin{equation*}
    \rank M_1 + \cdots + \rank M_n.
  \end{equation*}
\end{proposition}
\begin{proof}
  Follows from \fullref{thm:basis_of_direct_sum}.
\end{proof}

\begin{theorem}[Rank-nullity theorem]\label{thm:rank_nullity_theorem}
  For every \hyperref[def:semimodule/homomorphism]{linear map} \( \varphi: U \to V \) between \hyperref[thm:vector_space_dimension]{finite-dimensional} \hyperref[def:vector_space]{vector spaces}, we have
  \begin{equation}\label{eq:thm:rank_nullity_theorem}
    U \cong \ker \varphi \oplus \img \varphi.
  \end{equation}

  In particular,
  \begin{equation}\label{eq:thm:rank_nullity_theorem/dim}
    \dim U = \dim \ker \varphi + \dim \img \varphi.
  \end{equation}

  The dimension of the kernel of \( \varphi \) is often called the \term{nullity} of \( \varphi \) and the dimension of the image - the \term{rank} of \( \varphi \)
\end{theorem}
\begin{proof}
  By \fullref{thm:quotient_module_universal_property},
  \begin{equation*}
    \img \varphi \cong U / \ker \varphi.
  \end{equation*}

  By \fullref{thm:rank_nullity_via_subspaces},
  \begin{equation*}
    U \cong \ker \varphi \oplus (U / \ker \varphi) \cong \ker \varphi \oplus \img \varphi.
  \end{equation*}

  The equality \eqref{thm:rank_nullity_theorem} then follows from \fullref{thm:rank_of_direct_sum}.
\end{proof}

  \section{Univariate polynomials}\label{sec:univariate_polynomials}

We will discuss here the \hyperref[def:polynomial_algebra]{polynomial algebra} \( R[X] \) in one indeterminate over the \hyperref[def:ring/trivial]{nontrivial} \hyperref[def:ring/commutative]{commutative unital ring} \( R \). We call them \term{univariate polynomials} based on \fullref{def:operation_arity}, although we acknowledge the polynomials are not functions. \Fullref{rem:polynomials_over_infinitely_many_indeterminates} discusses why we often focus only on finitely many indeterminates, and why the theory of univariate polynomials is often sufficient.

Polynomials are not functions in general, and the exact relationship between polynomials and polynomial functions is discussed in \fullref{thm:polynomial_algebra_universal_property} and \fullref{thm:functions_over_finite_fields}.

\paragraph{Univariate polynomials}

\begin{definition}\label{def:monic_polynomial}
  We say that the nonzero univariate polynomial \( p(X) \) is \term[bg=нормиран (\cite[409]{Обрешков1962ВисшаАлгебра}), ru=нормированный/приведенный (\cite[102]{Винберг2014Алгебра})]{monic} if its leading coefficient is \( 1 \).
\end{definition}

\begin{proposition}\label{thm:polynomial_degree_arithmetic}
  The \hyperref[def:polynomial_degree]{polynomial degree} has the following basic properties:
  \begin{thmenum}
    \thmitem{thm:polynomial_degree_arithmetic/sum} For any two nonzero polynomials satisfying \( p(X) \neq -q(X) \), we have
    \begin{equation}\label{eq:thm:polynomial_degree_arithmetic/sum}
      \deg (p + q) \leq \max \set{ \deg p, \deg q }.
    \end{equation}

    \thmitem{thm:polynomial_degree_arithmetic/product} For any two nonzero polynomials \( p(X) \) and \( q(X) \) whose leading coefficients do not multiply to zero, we have
    \begin{equation}\label{eq:thm:polynomial_degree_arithmetic/product}
      \deg (pq) = \deg p + \deg q.
    \end{equation}
  \end{thmenum}
\end{proposition}
\begin{comments}
  \item An easy sufficient condition for \eqref{eq:thm:polynomial_degree_arithmetic/product} is for the ring to be \hyperref[def:entire_semiring]{entire}, although it is also sufficient for the ring to be nontrivial (so that \( 0_R \neq 1_R \)) and either \( p(X) \) or \( q(X) \) to be \hyperref[def:monic_polynomial]{monic}.

  \item We generalize a weaker statement for multivariate polynomials in \fullref{thm:degree_of_multivariate_polynomial_product}.
\end{comments}
\begin{proof}
  Fix nonzero polynomials
  \begin{align*}
    p(X) = \sum_{k=0}^n a_k X^k, &&
    q(X) = \sum_{k=0}^m b_k X^k.
  \end{align*}

  \SubProofOf{thm:polynomial_degree_arithmetic/sum} Additionally assume that \( p(X) \neq -q(X) \) since otherwise \( p(X) + q(X) = 0 \) and \( \deg(p + q) \) is undefined.

  Since the polynomials are not equal, there exists at least one index \( k = 1, 2, \ldots \) such that \( a_k \neq b_k \). Denote by \( k_0 \) the largest such index (only finitely many are nonzero). Then
  \begin{equation*}
    a_k - b_k = 0 \T{for} k > k_0.
  \end{equation*}

  Therefore, \( \deg(p + q) = k_0 \). Note that \( k_0 \) cannot exceed both \( \deg p \) and \( \deg q \) because it corresponds to a nonzero coefficient in both. Thus, \( k_0 \leq \max\set{ \deg p, \deg q } \).

  \SubProofOf{thm:polynomial_degree_arithmetic/product} The coefficient \( c_{n + m} \) of the product \( p(X) q(X) \) is \( a_n b_m \) by definition. By assumption, it is nonzero. Then, since \( c_{n+m+1} = 0 \), we have
  \begin{equation*}
    \deg (pq) = \deg p + \deg q.
  \end{equation*}
\end{proof}

\begin{corollary}\label{thm:leading_coefficient_of_product}
  If the leading coefficients of two univariate polynomials do not multiply to zero, the product of their leading coefficients is the leading coefficient of their product.
\end{corollary}
\begin{proof}
  Consider the polynomials
  \begin{align*}
    p(X) = \sum_{k=0}^n a_k X^k, &&
    q(X) = \sum_{k=0}^m b_k X^k.
  \end{align*}

  By definition of convolution product, the \( (n + m) \)-th coefficient of \( p(X) q(X) \) is \( a_n b_m \). \Fullref{thm:polynomial_degree_arithmetic/product} implies that this is their leading coefficient.
\end{proof}

\begin{algorithm}[Euclidean division of polynomials]\label{alg:euclidean_division_of_polynomials}\mcite[prop. 1.12]{Knapp2016BasicAlgebra}
  Fix two univariate polynomials \( f(X) \) and \( g(X) \), and assume that \( g(X) \) is \hyperref[def:monic_polynomial]{monic}.

  We will build polynomials \( q(X) \) and \( r(X) \), where \( r(X) \) is either zero or \( \deg r < \deg g \), such that
  \begin{equation*}
    f(X) = g(X) \cdot q(X) + r(X).
  \end{equation*}

  The algorithm only demonstrates existence; we will prove uniqueness right after it.

  \begin{thmenum}
    \thmitem{alg:euclidean_division_of_polynomials/zero_degree} If \( \deg f = \deg g = 0 \), necessarily \( g(X) = 1_R \), and in this case we define
    \begin{align*}
      q(X) &\coloneqq f(X), \\
      r(X) &\coloneqq 0_R.
    \end{align*}

    \thmitem{alg:euclidean_division_of_polynomials/no_division} If \( f(X) \) is the zero polynomial or \( \deg f < \deg g \), define
    \begin{align*}
      q(X) &\coloneqq 0_R, \\
      r(X) &\coloneqq f(X).
    \end{align*}

    In this case, \( r(X) \) is either zero or \( \deg r = \deg f < \deg b \).

    \thmitem{alg:euclidean_division_of_polynomials/positive_degree} Suppose that
    \begin{align*}
      f(X) = a_n X^n + \hat f(X), \\
      g(X) = X^m + \hat g(X),
    \end{align*}
    where \( n \) and \( m \) are positive, \( \hat f(X) \) is either zero or \( \deg \hat f < \deg f \), and similarly for \( \deg \hat g \).

    Then
    \begin{align*}
    f(X) - g(X) a_n X^{n-m}
    &=
    a_n X^n + \hat f(X) - (b_m X^m + \hat g(X)) a_n X^{n-m}
    = \\ &=
    a_n X^n + \hat f(X) - a_n X^n - \hat g(X) a_n X^{n-m}
    = \\ &=
    \underbrace{\hat f(X) - \hat g(X) a_n X^{n-m}}_{\hat r(X)}.
    \end{align*}

    The polynomial \( \hat r(X) \) is either zero, in which case we define \( r(X) \coloneqq \hat r(X) \), or \( \deg \hat r \leq n - 1 \).

    We use the algorithm recursively to divide \( \hat r(X) \) by \( g(X) \), and obtain \( \hat q(X) \) and \( r(X) \) such that
    \begin{equation*}
      \hat r(X) \coloneqq g(X) \hat q(X) + r(X),
    \end{equation*}
    where \( r(X) \) is either zero or \( \deg r < \deg g \).

    Then
    \begin{align*}
      \hat r(X)                                         &= f(X) - g(X) a_n X^{n-m} \\
      g(X) \hat q(X) + r(X)                             &= f(X) - g(X) a_n X^{n-m} \\
      g(X) \left(\hat q(X) + a_n X^{n-m} \right) + r(X) &= f(X).
    \end{align*}

    Define
    \begin{equation*}
      q(X) \coloneqq \hat q(X) + a_n X^{n-m}.
    \end{equation*}

    We have obtained polynomials \( r(X) \) and \( q(X) \) where \( r(X) \) is either zero or \( \deg r < \deg g \).
  \end{thmenum}
\end{algorithm}
\begin{comments}
  \item This algorithm can be found as \identifier{polynomials.univariate.euclidean_divmod} in \cite{notebook:code}.
\end{comments}
\begin{defproof}
  \UniquenessSubProof Suppose that
  \begin{equation*}
    a(X) = g(X)q(X) + r(X) = g(X) \widetilde{q}(X) + \widetilde{r}(X),
  \end{equation*}
  where \( r(X) \) and \( \widetilde{r}(X) \) are either zero or have degree less than \( g(X) \).

  Assume that \( r(X) \neq \widetilde{r}(X) \).

  \begin{itemize}
    \item If both \( r(X) \) and \( \widetilde{r}(X) \) are nonzero, we have
    \begin{equation*}
      g(X) \parens[\Big]{ q(X) - \widetilde{q}(X) } = -\parens[\Big]{ r(X) - \widetilde{r}(X) }.
    \end{equation*}

    Since \( g(X) \) is monic and its leading coefficient \( 1_R \) is not a zero divisor, \fullref{thm:polynomial_degree_arithmetic/product} holds, and thus
    \begin{equation*}
      \deg g + \deg(q - \widetilde{q})
      \reloset {\eqref{eq:thm:polynomial_degree_arithmetic/product}} =
      \deg(g (q - \widetilde{q}))
      =
      \deg(r - \widetilde{r})
      \reloset {\eqref{eq:thm:polynomial_degree_arithmetic/sum}} =
      \leq \max\set{ \deg r, \deg \widetilde{r} }
      <
      \deg g,
    \end{equation*}
    which is a contradiction.

    \item If \( r(X) \) is zero but \( \widetilde{r}(X) \) is not, then
    \begin{equation*}
      g(X) q(X) = g(X) \widetilde{q}(X) + \widetilde{r}(X),
    \end{equation*}
    implying that
    \begin{equation*}
      \widetilde{r}(X) = g(X) \parens[\Big]{ q(X) - \widetilde{q}(X) }.
    \end{equation*}

    By \eqref{thm:polynomial_degree_arithmetic/product}, \( \deg g \leq \widetilde{r} \), which contradicts our choice of \( \widetilde{r}(X) \).
  \end{itemize}
\end{defproof}

\begin{algorithm}[Horner's rule]\label{alg:horners_rule}
  Consider the polynomials
  \begin{equation*}
    f(X) = \sum_{k=0}^n a_k X^k
  \end{equation*}
  and
  \begin{equation*}
    g(X) = X + b.
  \end{equation*}

  The coefficients of the quotient of \( f(X) \) and \( g(X) \) with respect to \fullref{alg:euclidean_division_of_polynomials} can be computed recursively as follows:
  \begin{equation}\label{eq:alg:horners_rule/quot}
    c_{n-k} \coloneqq \begin{cases}
      a_n,                              &k = 1 \\
      a_{n-(k-1)} - b \cdot c_{n-(k-1)} &k > 1.
    \end{cases}
  \end{equation}

  Furthermore, \( r(X) \) is a constant that can be obtained from the above as
  \begin{equation}\label{eq:alg:horners_rule/rem}
    c_{-1} = a_0 - b \cdot c_0.
  \end{equation}
\end{algorithm}
\begin{comments}
  \item This algorithm can be found as \identifier{polynomials.division.horner_divmod} in \cite{notebook:code}.
\end{comments}
\begin{proof}
  Denote the quotient by
  \begin{equation*}
    q(X) = \sum_{k=0}^{n-1} c_k X^k.
  \end{equation*}

  We have
  \begin{equation*}
    f(X) = g(X) \cdot q(X) + r(X),
  \end{equation*}
  from where, for every \( k > 0 \),
  \begin{equation*}
    a_k = c_{k-1} + b \cdot c_k,
  \end{equation*}
  from which \eqref{eq:alg:horners_rule/quot} follows.

  Furthermore,
  \begin{equation*}
    r(X) = a_0 - b \cdot c_0,
  \end{equation*}
  which demonstrates \eqref{eq:alg:horners_rule/rem}.
\end{proof}

\paragraph{\( n \)-th roots}

\begin{definition}\label{def:nth_root}\mimprovised
  In a \hyperref[def:ring/commutative]{commutative ring}, for \( n \geq 2 \), we say that an element is an \term{\( n \)-th root} of \( a \) if it is a \hyperref[def:root_of_polynomial]{polynomial root} of \( X^n - a \).

  We use the prefixes from the polynomial degree terminology described in \fullref{def:polynomial_degree_terminology}, with one modification --- instead of \enquote{quadratic root}, we say \enquote{square root}.

  If \( b \) is a square root (resp. cubic root) of \( a \), we say that \( a \) is a \term{square} (resp. \term{cube}) of \( b \).
\end{definition}
\begin{comments}
  \item In the case of positive \hyperref[def:real_numbers]{real numbers}, we have a canonical choice of \( n \)-th roots given by \fullref{def:principal_nonnegative_nth_root}, where we use the notation \( \sqrt[n]{ x } \). We also have such a canonical choice for square roots of negative real numbers --- see \fullref{def:principal_real_square_root}.
\end{comments}

\begin{definition}\label{def:algebraic_equation}\mimprovised
  \hyperref[def:first_order_equation]{Equations} in the \hyperref[def:ring/theory]{theory of rings} are called \term{algebraic}. Such equations are equalities of polynomials in finitely many indeterminates. It is sufficient to only consider equations whose right term is zero (since we can cancel it otherwise), thus a general algebraic equation has the form
  \begin{equation*}\label{eq:def:algebraic_equation}
    f(X_1, \ldots, X_n) = 0.
  \end{equation*}

  We use the prefixes from the polynomial degree terminology described in \fullref{def:polynomial_degree_terminology}, e.g. we call \eqref{eq:def:algebraic_equation} a \enquote{quadratic equation} if \( f(X_1, \ldots, X_n) \) is a quadratic polynomial.

  \begin{thmenum}
    \thmitem{def:algebraic_equation/trivial_solution} If evaluating all indeterminates to \( 0 \) yields a solution, we call it the \term{trivial solution}.

    \thmitem{def:algebraic_equation/diophantine}\mcite[5]{Apostol1976AnalyticNumberTheory} Algebraic equations over the \hyperref[def:ring_of_integers]{ring of integers} are called \term[ru=Диофантово уравнение (\cite[\S 1.9]{ШеньВерещагин2017Вычислимость})]{Diophantine equations}.
  \end{thmenum}
\end{definition}

\begin{remark}\label{rem:root_terminology}
  The term \enquote{root} has several distinct (but related) meanings:
  \begin{itemize}
    \item The \( n \)-th root in the sense of \fullref{def:nth_root}. The discussion in \cite{HSM:radical_symbol_history} suggests that this is the origin of the terms \enquote{root} and \enquote{radical}, based on the Latin \enquote{radix}.

    There is inherent ambiguity in pick \enquote{the} root, since there may be many. For positive real numbers we have an established canonical choice of \( n \)-th roots, which we call \enquote{principal roots}, and for negative real numbers --- of square roots.

    \item The \hyperref[def:first_order_equation]{solutions} of an \hyperref[def:algebraic_equation]{algebraic equation} are also called \enquote{roots}. An explicit definition of this usage can be found in \incite[2]{Обрешков1962ВисшаАлгебра}.

    As noted in \fullref{def:algebraic_equation}, polynomials naturally arise from equations on rings, thus it makes sense for this usage to predate polynomials.

    \item From a modern perspective, both are encompassed by roots of polynomials as defined in \fullref{def:root_of_polynomial}. Roots are zeros in the sense of \fullref{def:zero_of_function} of a fixed polynomial function. For \hyperref[def:univariate_polynomial]{univariate polynomials} \fullref{def:multiple_root} provides a characterization in terms of divisibility, as well as a concept of \enquote{multiplicity} of a root.

    A \enquote{root} is defined as a zero of a univariate polynomial by
    \incite[282]{Aluffi2009Algebra},
    \incite[11]{Knapp2016BasicAlgebra} and
    \incite[119]{Тыртышников2017Алгебра}.
  \end{itemize}
\end{remark}

\begin{definition}\label{def:square_free_element}\mimprovised
   In a \hyperref[def:ring/commutative]{commutative ring}, we say that \( x \) of is \term[ru=свободное от квадратов (число) (\cite[def. 93]{Бухштаб1966ТеорияЧисел}), en=square-free (element) (\cite[79]{JedrzejewiczEtAl2017SquareFree})]{square-free} if no \hi{non-invertible} divisor of \( x \) has a \hyperref[def:nth_root]{square root}.
\end{definition}
\begin{comments}
  \item This concept is established for the ring of integers. An explicit definition can be found in \cite[def. 93]{Бухштаб1966ТеорияЧисел}) and an inline definition can be found in \cite[176]{Birkhoff1967Lattices}, both requiring all prime factors of square-free integers to be distinct.

  \incite[79]{JedrzejewiczEtAl2017SquareFree} generalize this concept to commutative rings --- they require \( y \mid x \) to imply that \( y^2 \not\mid x \). We give an equivalent definition in terms of square roots.
\end{comments}

\paragraph{Multiple roots}

\begin{definition}\label{def:algebraic_derivative}\mcite[461]{Knapp2016BasicAlgebra}
  Generalizing \fullref{def:differentiability} from analysis, we define the \term[ru=производная (\cite[163]{Тыртышников2017Алгебра})]{algebraic derivative} of a univariate polynomial
  \begin{equation*}
    p(X) = \sum_{k=0}^n a_k X^k = a_n X^n + a_{n-1} X^{n-1} + \cdots + a_2 X^2 + a_1 X + a_0
  \end{equation*}
  as
  \begin{equation*}
    p'(X) \coloneqq \sum_{k=1}^n k a_k X^{k-1} = n a_n X^{n-1} + (n-1) a_{n-1} X^{n-2} + \cdots + a_2 X + a_1.
  \end{equation*}

  Via \hyperref[rem:natural_number_recursion]{natural number recursion}, we can define algebraic derivatives of order \( m \) as
  \begin{equation*}
    p^{(m)}(X) \coloneqq \begin{cases}
      p(X)              &m = 0 \\
      \parens[\Big]{ p^{(m - 1)} }'(X) &m > 0
    \end{cases}
  \end{equation*}
\end{definition}

\begin{proposition}\label{thm:def:algebraic_derivative}
  \hyperref[def:algebraic_derivative]{Algebraic derivatives} have the following basic properties:
  \begin{thmenum}
    \thmitem{thm:def:algebraic_derivative/linear} The derivative operator \( p(X) \mapsto p'(X) \) is linear.
    \thmitem{thm:def:algebraic_derivative/degree} \( p^{(n)}(X) \) is either zero or has degree \( \deg p - n \).

    \thmitem{thm:def:algebraic_derivative/product} The product rule holds:
    \begin{equation}\label{eq:thm:def:algebraic_derivative/product}
      (pq)' = p'q + pq'.
    \end{equation}

    \thmitem{thm:def:algebraic_derivative/leibniz} \hyperref[thm:leibniz_rule]{Leibniz' rule} holds:
    \begin{equation}\label{eq:thm:def:algebraic_derivative/leibniz}
      (pq)^{(n)} = \sum_{k=0}^n \binom n k p^{(k)} q^{(n-k)}
    \end{equation}

    \thmitem{thm:def:algebraic_derivative/affine_power} If \( m \leq n \), the \( m \)-th derivative of \( (X - \alpha)^n \) is
    \begin{equation*}
      \frac {n!} {(n-m)!} (X - \alpha)^{n-m}.
    \end{equation*}
  \end{thmenum}
\end{proposition}
\begin{proof}
  \SubProofOf{thm:def:algebraic_derivative/linear} Trivial.

  \SubProofOf{thm:def:algebraic_derivative/degree} Trivial.

  \SubProofOf{thm:def:algebraic_derivative/product} By \fullref{thm:def:algebraic_derivative/linear}, it is enough to consider the case where both \( p(X) \) and \( q(X) \) are monomials.

  \begin{align*}
    p'(X) q(X) + p(X) q'(X)
    &=
    n a_n X^{n-1} \cdot b_m X^m + a_n X^n \cdot m b_m X^{m-1}
    = \\ &=
    (n + m) a_n b_m X^{n+m-1}
    = \\ &=
    (a_n b_m X^{n+m})'
    = \\ &=
    (pq)'(X).
  \end{align*}

  \SubProofOf{thm:def:algebraic_derivative/leibniz} The proof in \fullref{thm:leibniz_rule} relies only on the product rule, hence it holds here as well.

  \SubProofOf{thm:def:algebraic_derivative/affine_power} We use outer induction on \( m \) and inner induction on \( n \).

  The case \( m = n = 1 \) is obvious. Assume that the statement holds for \( m = 1 \) and \( n - 1 \). Then
  \begin{equation*}
    \parens[\Big]{ (X - \alpha)^n }'
    =
    \parens[\Big]{ (X - \alpha)^{n-1} \cdot (X - \alpha) }'
    \reloset {\eqref{eq:thm:def:algebraic_derivative/product}} =
    \parens[\Big]{ (X - \alpha)^{n-1} }' (X - \alpha) + (X - \alpha)^{n-1}
    \reloset {\T{ind.}} =
    n (X - \alpha)^{n-1}.
  \end{equation*}

  Now suppose that the statement holds for derivatives of order less than \( m \) and for every \( n \geq m \). Then,
  \begin{equation*}
    \parens[\Big]{ (X - \alpha)^n }^{(m)}
    =
    \parens*{\parens[\Big]{ (X - \alpha)^n }^{(m-1)}}'
    \reloset {\T{ind.}} =
    \parens*{ \frac {n!} {(n - m + 1)!} (X - \alpha)^{n - m + 1} }'
    \reloset {\T{ind.}} =
    \frac {n!} {(n - m)!} (X - \alpha)^{n - m}.
  \end{equation*}
\end{proof}

\begin{definition}\label{def:multiple_root}\mimprovised
  Let \( R \) be a nontrivial commutative ring.

  We say that the \hyperref[def:root_of_polynomial]{root} \( \alpha \) of the univariate polynomial \( p(X) \) of degree \( n \) has \term[bg=кратност (\cite[171]{Обрешков1962ВисшаАлгебра}), ru=кратность (\cite[163]{Тыртышников2017Алгебра}), en=multiplicity (\cite[229]{Jacobson1985AlgebraPart1})]{multiplicity} \( m \) if any of the following equivalent conditions hold:
  \begin{thmenum}
    \thmitem{def:multiple_root/division}\mcite[163]{Тыртышников2017Алгебра} The polynomial \( (X - \alpha)^m \) divides \( p(X) \).

    \thmitem{def:multiple_root/derivative_roots} The value \( \alpha \) is a \hyperref[def:root_of_polynomial]{zero} of the \hyperref[def:algebraic_derivative]{algebraic derivatives} \( p^{(0)}(X), p^{(1)}(X), \ldots, p^{(m-1)}(X) \).
  \end{thmenum}

  Every polynomial \( p(X) \) has a \hyperref[def:multiset]{multiset} of roots. If at least one of them has multiplicity greater than one, we say that it is a \term[bg=многократен (корен) (\cite[171]{Обрешков1962ВисшаАлгебра}), ru=кратный (корень) (\cite[163]{Тыртышников2017Алгебра}), en=multiple root (\cite[229]{Jacobson1985AlgebraPart1})]{multiple root}, otherwise we say that it is a \term[bg=прост корен (\cite[171]{Обрешков1962ВисшаАлгебра}), ru=простой корень (\cite[163]{Тыртышников2017Алгебра}), en=simple root (\cite[229]{Jacobson1985AlgebraPart1})]{simple root}.
\end{definition}
\begin{comments}
  \item This is an extension of the general concept of polynomial roots discussed in \fullref{def:root_of_polynomial}.

  \item The equivalence between \( (X - \alpha) \) dividing \( p(X) \) and \( p(\alpha) \) being zero is called the \enquote{factor theorem} by \incite[corr. 1.13]{Knapp2016BasicAlgebra}.
\end{comments}
\begin{defproof}
  \ImplicationSubProof{def:multiple_root/division}{def:multiple_root/derivative_roots} Suppose that \( (X - \alpha)^m \) divides \( p(X) \). Then there exists a polynomial \( q(X) \) such that
  \begin{equation*}
    p(X) = (X - \alpha)^m q(X).
  \end{equation*}

  For the \( n < m \)-th derivative of \( p(X) \), by \fullref{thm:def:algebraic_derivative/leibniz}, we have
  \begin{equation*}
    p^{(n)}(X) = \sum_{k=0}^n \binom n k \underbrace{\parens[\Big]{ (X - \alpha)^m }^{(k)}}_{\mathclap{\frac {m!} {(m-k)!} (X - \alpha)^{m-k} \T*{by} \ref{thm:def:algebraic_derivative/affine_power}}} q^{(n-k)}(X).
  \end{equation*}

  Let \( \Phi_\alpha: R[X] \to R \) be the \hyperref[con:evaluation_homomorphism]{evaluation homomorphism} at \( \alpha \). Then
  \begin{equation*}
    \Phi_\alpha(p^{(n)}) = \sum_{k=0}^n \binom n k \frac {m!} {(m-k)!} 0_R^{m-k} \Phi_\alpha(q^{(n-k)})
  \end{equation*}

  For \( n < m \), clearly \( \Phi_\alpha(p^{(n)}) = 0_R \).

  \ImplicationSubProof{def:multiple_root/derivative_roots}{def:multiple_root/division} Suppose that \( \alpha \) is a root of \( p^{(0)}(X), \ldots, p^{(m-1)}(X) \). We will use induction on \( m \) to show that \( (X - \alpha)^m \mid p(X) \).

  The case \( m = 0 \) is trivial. Suppose that \( \alpha \) is root of \( p^{(0)}(X), \ldots, p^{(m-1)}(X) \) and that \( (X - \alpha)^{m-1} \) divides \( p(X) \). Additionally suppose that \( \alpha \) is a root of \( p^{(m)}(X) \).

  By the inductive hypothesis, there exists a polynomial \( q(X) \) such that
  \begin{equation*}
    p(X) = (X - \alpha)^{m-1} q(X).
  \end{equation*}

  By \fullref{thm:def:algebraic_derivative/leibniz},
  \begin{equation*}
    p^{(m-1)}(X) = \sum_{k=0}^{m-1} \binom {m-1} k \underbrace{\parens[\Big]{ (X - \alpha)^{m - 1} }^{(k)}}_{\mathclap{\frac {(m - 1)!} {(m - 1 - k)!} (X - \alpha)^{m - 1 - k} \T*{by} \ref{thm:def:algebraic_derivative/affine_power}}} q^{(m - 1 - k)}(X).
  \end{equation*}

  Then
  \begin{equation*}
    \Phi_\alpha(p^{(m-1)}) = \sum_{k=0}^{m-1} \binom {m-1} k \frac {(m - 1)!} {(m - 1 - k)!} 0_R^{m - 1 - k} \Phi_\alpha(q^{(m - 1 - k)}).
  \end{equation*}

  All terms on the right are zero except for the case where \( k = m - 1 \), where \( q^{(0)} = q \) and the entire expression reduces to \( \Phi_\alpha(q) \). The scalar \( \alpha \) is a root of \( p^{(m-1)} \), implying that it is also a root of \( q \).

  We use \fullref{alg:euclidean_division_of_polynomials} to obtain a polynomial \( s(X) \) and a constant polynomial \( r(X) = r_0 \) so that
  \begin{equation*}
    q(X) = (X - \alpha) s(X) + r_0.
  \end{equation*}

  Since \( \Phi_\alpha(q) = 0_R \), then necessarily \( r_0 = 0_R \). Therefore,
  \begin{equation*}
    p(X) = (X - \alpha)^{m-1} q(X) = (X - \alpha)^m s(X).
  \end{equation*}
\end{defproof}

  \section{Algebras over rings}\label{sec:algebras_over_rings}

\paragraph{Algebras over rings}

\begin{definition}\label{def:algebra_over_ring}\mcite[15]{Kaplansky1974CommutativeRings}
  An \term[bg=алгебра (\cite[4]{КоцевСидеров2016КомутативнаАлгебра}), ru=алгебра (\cite[def. 1.7.1]{Винберг2014КурсАлгебры}), en=commutative algebra (\cite[28]{Eisenbud1995CommutativeAlgebra})]{algebra} over a commutative ring \( R \) rather than over a \hyperref[def:algebra_over_semiring]{semiring} exhibits some more interesting metamathematical properties.

  \begin{thmenum}
    \thmitem{def:algebra_over_ring/theory}\mimprovised The first-order theory is identical to the \hyperref[def:algebra_over_semiring/theory]{theory of algebras over semimodules}.

    \thmitem{def:algebra_over_ring/homomorphism}\mcite[28]{Eisenbud1995CommutativeAlgebra} A \hyperref[def:first_order_homomorphism]{first-order homomorphism} between two \( R \)-algebras is a \hyperref[def:linear_function]{linear map} that preserves multiplication. This is the same as for semirings.

    \thmitem{def:algebra_over_ring/submodel}\mcite[28]{Eisenbud1995CommutativeAlgebra}) The set \( A \subseteq M \) is a \hyperref[def:first_order_submodel]{submodel} of \( M \) if it is a \hyperref[def:monoid/submodel]{submodule} of \( M \) that contains \( 1 \) and is closed under algebra multiplication. We say that \( A \) is an \( R \)-\term{subalgebra} of \( M \).

    If \( A \) does not contain \( 1 \), we may instead refer to nonunital \( R \)-subalgebras. We will use them for quotients.

    As a consequence of \fullref{thm:positive_formulas_preserved_under_homomorphism}, the image of an \( R \)-algebra homomorphism is an \( R \)-subalgebra of its codomain.

    \thmitem{def:algebra_over_ring/category}\mimprovised For a fixed ring \( R \), we denote the \hyperref[def:category_of_small_first_order_models]{category of \( \mscrU \)-small models} by \( \ucat{Alg}_R \). It is concrete with respect to both \( \ucat{CRing} \) and \( \ucat{Mod}_R \).

    Unfortunately, these categories are not as well-behaved as categories of modules. Similarly to rings, unital and nonunital algebras behave differently.

    \thmitem{def:algebra_over_ring/trivial}\mimprovised Similarly to rings, \enquote{the} \hyperref[def:trivial_object]{trivial object} is the one-element algebra \( \set{ 0 } \).

    \thmitem{def:algebra_over_ring/kernel}\mimprovised \Fullref{thm:ring_zero_morphisms/kernel} implies that the \hyperref[def:zero_morphisms/kernel]{categorical kernel} of a homomorphism \( \varphi: M \to N \) between \hi{nonunital} \( R \)-algebras is the additive group kernel
    \begin{equation*}
      \ker \varphi \coloneqq \varphi^{-1}(0_N) = \set{ x \in M \given \varphi(x) = 0_N }.
    \end{equation*}

    The kernel is a both a two-sided ideal of \( M \) as a consequence of \fullref{thm:kernel_is_ideal} and a submodule of \( M \) as a consequence of \fullref{thm:kernel_is_submodule}.

    \thmitem{def:algebra_over_ring/quotient}\mimprovised Similarly to rings, we define quotient \( R \)-algebras of \( M \) by nonunital \( R \)-subalgebras. In particular, \fullref{thm:algebra_ideal_is_subalgebra} implies that we can take the quotient by any ideal of \( M \).

    \thmitem{def:algebra_over_ring/commutative}\mimprovised As in the case of algebras over semirings, by \enquote{\( M \) is commutative}, we will mean that vector multiplication is commutative.

    We denote the subcategory of commutative algebras by \( \cat{CAlg}_R \).
  \end{thmenum}
\end{definition}
\begin{comments}
  \item We adopt Eisenbud's definitions for submodules and homomorphisms from \cite[28]{Eisenbud1995CommutativeAlgebra}, however for the main definition we stick to Kaplansky since he allows vector products that are not commutative.
\end{comments}

\begin{proposition}\label{thm:algebra_ideal_is_subalgebra}
  Every \hyperref[def:semiring_ideal]{left ideal} of an \( R \)-\hyperref[def:algebra_over_ring]{algebra} is a nonunital \( R \)-\hyperref[def:algebra_over_ring/submodel]{subalgebra}.
\end{proposition}
\begin{proof}
  Let \( I \) be a left ideal of the \( R \)-algebra \( M \). By definition of left ideal, \( I \) is a left \( M \)-submodule of \( M \), and, because \( R \) is a subring of \( M \), \( I \) is a left \( R \)-submodule of \( M \).

  Furthermore, \eqref{eq:def:semiring_ideal/direct/multiplicative} implies that \( I \) is closed under vector multiplication with arbitrary elements of \( M \), hence \( I \) is also an \( R \)-subalgebra.
\end{proof}

\begin{proposition}\label{thm:ring_is_integer_algebra}
  The categories \( \hyperref[def:ring/category]{\cat{Ring}} \) of rings and \( \hyperref[def:algebra_over_ring/category]{\cat{Alg}_\BbbZ} \) of integer algebras are \hyperref[rem:category_similarity/isomorphism]{isomorphic}.
\end{proposition}
\begin{comments}
  \item Compare this result to \fullref{thm:abelian_group_is_module} for modules and \fullref{thm:semiring_is_natural_number_algebra} for algebras over semirings.
\end{comments}
\begin{proof}
  Follows from \fullref{thm:semiring_is_natural_number_algebra}.
\end{proof}

\paragraph{Quotients of polynomial algebras by principal ideals}

\begin{proposition}\label{thm:representatives_in_univariate_polynomial_quotient_set}
  Fix a \hyperref[def:monic_polynomial]{monic polynomial} \( f(X) \) in a nontrivial commutative ring \( R \).

  Every coset in \( R[X] / \braket{ f(X) } \) has a unique representative that is either zero or has degree less than that of \( f(X) \).
\end{proposition}
\begin{proof}
  Let \( g(X) \) be an arbitrary polynomial. \Fullref{alg:euclidean_division_of_polynomials} gives us polynomials \( q(X) \) and \( r(X) \) such that
  \begin{equation*}
    g(X) = f(X) q(X) + r(X),
  \end{equation*}
  where \( r(X) \) is either zero or has degree less than that of \( f(X) \).

  Multiples of \( q(X) \) are congruent to \( 0_R \) modulo the ideal \( \braket{ q(X) } \), hence \( g(X) \) is congruent to \( r(X) \).

  By the uniqueness of \( r(X) \), the statement of the corollary follows.
\end{proof}

\begin{corollary}\label{thm:polynomial_quotient_modules_vs_algebras}
  For two nonzero monic polynomials \( f(X) \) and \( g(X) \), the degrees of \( f(X) \) and \( g(X) \) coincide if and only if the \hyperref[def:algebra_over_ring/quotient]{quotient algebras} \( R[X] / \braket{ f(X) } \) and \( R[X] / \braket{ g(X) } \) are isomorphic as \( R \)-modules.
\end{corollary}
\begin{comments}
  \item As shown in \fullref{ex:gaussian_integers} and \fullref{ex:integers_with_sqrt2}, the vector multiplication operation on the quotients may differ --- the quotients may be isomorphic as \( R \)-modules, but not as \( R \)-algebras.
\end{comments}
\begin{proof}
  By \fullref{thm:representatives_in_univariate_polynomial_quotient_set}, for every coset in the quotient, \fullref{alg:euclidean_division_of_polynomials} gives us a unique representative of the corresponding degree. Addition and scalar multiplication must be the same in both.
\end{proof}

\begin{corollary}\label{thm:polynomial_quotient_module_dimension}
  The quotient of the polynomial algebra \( R[X] \) by the principal ideal \( \braket{ f(X) } \), where \( f(X) \) is monic, has as \hyperref[def:module_rank]{module rank} the degree of \( f \).
\end{corollary}
\begin{proof}
  Follows from \fullref{thm:polynomial_quotient_module_dimension} by noting that \( 1, X, X^2, \ldots, X^{\deg f-1} \) is a basis.
\end{proof}

\begin{proposition}\label{thm:adjoint_roots_and_quotients}
  Fix arbitrary commutative rings \( R \subseteq S \) and some element \( u \) of \( S \). Consider the \hyperref[con:evaluation_homomorphism]{evaluation homomorphism} \( \Phi_u: R[X] \to S \) sending \( X \) to \( u \). Additionally suppose that the kernel of \( \Phi_u \) is principal with generator \( f(X) \).

  Then the quotient ring \( R[X] / \braket{ f(X) } \) is isomorphic to the ring \( R[u] \) obtained by \hyperref[def:semiring_adjunction]{adjoining} \( u \) to \( R \).
\end{proposition}
\begin{comments}
  \item This is useful when \( R \) is a \hyperref[def:field]{field} and thus \( R[X] \) is a \hyperref[def:principal_ideal_domain]{principal ideal domain}. See the equivalences in \fullref{def:algebraic_element}.
\end{comments}
\begin{proof}
  \Fullref{thm:ring_zero_morphisms/isomorphism} implies that
  \begin{equation*}
    R[X] / \underbrace{\ker \Phi_u}_{\braket{ f(X) }} \cong \underbrace{\img \Phi_u}_{R[u]}.
  \end{equation*}
\end{proof}

\begin{definition}\label{def:gaussian_integers}\mcite[\S V.6.2]{Aluffi2009Algebra}
  A \term[ru=целые гауссовые числа (\cite[example 3.5.1]{Винберг2014КурсАлгебры})]{Gaussian integer} is a \hyperref[def:complex_numbers]{complex number} whose real and imaginary part are \hyperref[def:integers]{integers}.
\end{definition}

\begin{example}\label{ex:gaussian_integers}
  We can define several isomorphic rings for the \hyperref[def:gaussian_integers]{Gaussian integers}, demonstrating \fullref{thm:adjoint_roots_and_quotients}.

  We assume that the field of complex numbers is available to us.

  \begin{thmenum}
    \thmitem{ex:gaussian_integers/quotient} Analogous to our discussion in \fullref{def:complex_numbers}, we can take the \hyperref[def:algebra_over_ring/quotient]{quotient algebra} \( \BbbZ[X] / \braket{X^2 + 1} \).

    \thmitem{ex:gaussian_integers/evaluation} We can also \hyperref[def:semiring_adjunction]{adjoin} \( i \) to \( \BbbZ \) to obtain the ring \( \BbbZ[i] \).

    Given a Gaussian integer \( z = a + bi \), it corresponds to the polynomial
    \begin{equation*}
      f_z(X) \coloneqq a + bX.
    \end{equation*}

    Conversely, consider the \hyperref[con:evaluation_homomorphism]{evaluation homomorphism} \( \Phi_i: \BbbZ[X] \to \BbbC \) for the imaginary unit. Let \( f(X) \in \BbbZ[X] \). Then
    \begin{equation*}
      f(i)
      =
      \Phi_i(f)
      =
      \sum_{k=0}^n a_k i^n
      =
      \thickspace \sum_{\scriptscriptstyle{\mathclap{\rem(k, 4) = 0}}}^n a_k - \sum_{\scriptscriptstyle{\mathclap{\rem(k, 4) = 2}}}^n a_k + i \parens[\Bigg]{ \quad \sum_{\scriptscriptstyle{\mathclap{\rem(k, 4) = 1}}}^n a_k - \sum_{\scriptscriptstyle{\mathclap{\rem(k, 4) = 3}}}^n a_k }.
    \end{equation*}

    This is clearly again a Gaussian integer.

    Thus, although we skipped proving that \( \braket{X^2 + 1} \) is the kernel of \( \Phi_i: \BbbZ[X] \to \BbbC \), we have shown that the algebras \( \BbbZ[X] / \braket{ X^2 + 1 } \) and \( \BbbZ[i] \) behave identically.
  \end{thmenum}
\end{example}

\begin{example}\label{ex:integers_mod_xx_minus_1}
  In \fullref{ex:gaussian_integers}, we discussed how the ring of \hyperref[def:gaussian_integers]{Gaussian integers} equals the quotient
  \begin{equation*}
    \BbbZ[X] / \braket{ X^2 + 1 }.
  \end{equation*}

  If we instead take
  \begin{equation*}
    \BbbZ[X] / \braket{ X^2 - 1 },
  \end{equation*}
  multiplication of \( a + bX \) and \( c + dX \) would behave as
  \begin{equation*}
    (a + bX) (c + dX) = (ac + bd) + (bc + ad)X.
  \end{equation*}

  This is less useful since \( -1 \) and \( 1 \) are already roots of \( X^2 - 1 \), and we adjoin a new root. Nevertheless, the example shows that polynomial quotient algebras are constructed similarly when the polynomial is not irreducible.
\end{example}

\begin{example}\label{ex:integers_with_sqrt2}
  Similarly to \fullref{ex:gaussian_integers}, we have
  \begin{equation*}
    \BbbZ[X] / \braket{X^2 - 2} \cong \BbbZ[\sqrt 2].
  \end{equation*}

  The gist of this example is that, even though \( \BbbZ[\sqrt 2] \) and \( \BbbZ[i] \) are isomorphic as modules, their vector multiplication operation is different. Indeed, since \( X^2 = 2 \), we have
  \begin{align*}
    (a + bX) (c + dX) = (ac + 2bd) + (bc + ad)X
  \end{align*}
  which is different compared to complex multiplication.
\end{example}

\paragraph{Noetherian algebras}

\begin{definition}\label{def:noetherian_semimodule}\mcite[prop. 6.16]{Golan1999Semirings}
  We say that an \( R \)-\hyperref[def:semimodule]{semimodule} is \term[bg=ньотеров (\cite[41]{КоцевСидеров2016КомутативнаАлгебра}), ru=нётеровый (\cite[def. 9.4.1]{Винберг2014КурсАлгебры})]{noetherian} if any of the following equivalent conditions hold:
  \begin{thmenum}
    \thmitem{def:noetherian_semimodule/acc} It satisfies the \hyperref[def:chain_condition]{ascending chain condition} on \( R \)-sub-semimodules.

    \thmitem{def:noetherian_semimodule/generated} Every \( R \)-sub-semimodule is \hyperref[def:semimodule/generated]{finitely generated}, i.e. is the \hyperref[def:semimodule/submodel]{linear span} of finitely many elements.
  \end{thmenum}
\end{definition}
\begin{defproof}
  Fix an \( R \)-semimodule \( M \).

  \ImplicationSubProof{def:noetherian_semimodule/acc}{def:noetherian_semimodule/generated} Suppose that \( M \) satisfies \fullref{def:chain_condition/maximal}, i.e. every nonempty family of \( R \)-sub-semimodules of \( M \) has a maximal element.

  Let \( K \coloneqq \linspan{ x_1, \ldots, x_n } \) be maximal in the family of all finitely generated \( R \)-sub-semimodules. Let \( N \) be any sub-semimodule. Adding a particular element from \( N \) does not change \( K \), because otherwise it would not be maximal. Thus, \( N \subseteq K \).

  \ImplicationSubProof{def:noetherian_semimodule/generated}{def:noetherian_semimodule/acc} Suppose that every \( R \)-sub-semimodule is finitely generated.

  Consider the ascending sequence of \( R \)-sub-semimodules
  \begin{equation}\label{eq:def:noetherian_semimodule/chain}
    N_1 \subseteq N_2 \subseteq N_3 \subseteq \cdots
  \end{equation}

  By \fullref{thm:def:semimodule/union}, their union \( \bigcup_{k \in \mscrK} N_k \) is also an \( R \)-sub-semimodule.

  Let \( x_1, \ldots, x_n \) be the set of generators for the union. Let \( k_m \) be the index of the first sub-semimodule that contains \( x_m \). Every next sub-semimodule in the sequence contains the previous, hence \( k \leq k_m \) implies that \( x_k \in N_{k_m} \).

  Let \( k_{m_0} \) be a maximal index. Then \( N_{k_{m_0}} \) contains all the generators, and hence it coincides with the union \( \bigcup_{k \in \mscrK} N_k \). Every sub-semimodule with a greater index is simply equal to the previous.

  Therefore, the sequence \eqref{eq:def:noetherian_semimodule/chain} stabilizes.
\end{defproof}

\begin{proposition}\label{thm:def:noetherian_semimodule}
  \hyperref[def:noetherian_semimodule]{Noetherian modules} over an arbitrary ring \( R \) have the following basic properties:
  \begin{thmenum}
    \thmitem{thm:def:noetherian_semimodule/submodule} If \( M \) is noetherian, then every \( R \)-submodule of \( M \) also is.
    \thmitem{thm:def:noetherian_semimodule/quotient}\mcite[prop. 6.3b)]{КоцевСидеров2016КомутативнаАлгебра} Let \( N \) be an \( R \)-\hyperref[def:module/submodel]{submodule} of \( M \). Then \( M \) is noetherian if and only if both \( N \) and their \hyperref[def:module/quotient]{quotient} \( M / N \) are.
  \end{thmenum}
\end{proposition}
\begin{proof}
  \SubProofOf{thm:def:noetherian_semimodule/submodule} Trivial.
  \SubProofOf{thm:def:noetherian_semimodule/quotient} By \fullref{thm:lattice_theorem_for_submodules}, every sequence of \( R \)-submodules of \( M / N \) corresponds to a sequence of \( R \)-submodules in \( M \). Thus, if \( M \) is noetherian, clearly \( M / N \) also is.

  Conversely, suppose that both \( N \) and \( M / N \) are noetherian. Let
  \begin{equation}\label{eq:thm:def:noetherian_semimodule/quotient/chain}
    K_1 \subseteq K_2 \subseteq K_3 \subseteq \cdots
  \end{equation}
  be an ascending sequence of \( R \)-submodule of \( M \). Then
  \begin{equation}\label{eq:thm:def:noetherian_semimodule/quotient/chain/intersection}
    K_1 \cap N \subseteq K_2 \cap N \subseteq K_3 \cap N \subseteq \cdots
  \end{equation}
  is an ascending sequence of \( R \)-submodules of \( N \) and
  \begin{equation}\label{eq:thm:def:noetherian_semimodule/quotient/chain/quotient}
    (K_1 + N) / N \subseteq (K_2 + N) / N \subseteq (K_3 + N) / N \subseteq \cdots
  \end{equation}
  is an ascending sequence of \( R \)-submodule of \( M / N \).

  Both \eqref{eq:thm:def:noetherian_semimodule/quotient/chain/intersection} and \eqref{eq:thm:def:noetherian_semimodule/quotient/chain/quotient} stabilize. Let \( n \) be an index such that, for every positive integer \( k \), \( K_n \cap N = K_{n + k} \cap N \) and \( (K_n + N) / N = (K_{n + k} + N) / N \). For a fixed \( k \), we will show that \( K_n = K_{n + k} \).

  Let \( x \in K_{n + k} \). If \( x \in N \), then \( x \in K_n \) since \( K_n \cap N = K_{n + k} \cap N \). Suppose that \( x \in K_{n + k} \setminus N \). For any \( n \in N \), we have \( x + n \in K_{n + k} + N \), and hence
  \begin{equation*}
    x + n + N = x + N \in (K_{n + k} + N) / N = (K_n + N) / N.
  \end{equation*}

  Then there exists some \( y \in K_n \) such that \( x - y \in N \). Actually
  \begin{equation*}
    x - y \in K_{n + k} \cap N = K_n \cap N.
  \end{equation*}

  Since both \( y \) and \( x - y \) are in \( K_n \), so is their sum \( x \). Generalizing on \( x \), we conclude that \( K_n = K_{n + k} \).

  Therefore, the sequence \eqref{eq:thm:def:noetherian_semimodule/quotient/chain} stabilizes, implying that \( M \) is noetherian.
\end{proof}

\begin{definition}\label{def:noetherian_semiring}\mcite[prop. 6.16]{Golan1999Semirings}
  We say that a (not necessarily commutative) \hyperref[def:semiring]{semiring} is \term{left noetherian} (resp. right noetherian) if it is a left (resp. right) \hyperref[def:noetherian_semimodule]{noetherian semimodule} over itself.

  Explicitly, any of the following equivalent conditions characterize a left noetherian semiring:
  \begin{thmenum}
    \thmitem{def:noetherian_semiring/acc} It satisfies the \hyperref[def:chain_condition]{ascending chain condition} on left (resp. right) ideals.
    \thmitem{def:noetherian_semiring/generated} Every left (resp. right) ideal is \hyperref[def:semiring_ideal/generated]{finitely generated}.
  \end{thmenum}
\end{definition}

\begin{proposition}\label{thm:noetherian_free_module}
  For a \hyperref[def:noetherian_semiring]{noetherian ring} \( R \), the \hyperref[def:coordinate_space]{coordinate space} \( R^n \) is a \hyperref[def:noetherian_semimodule]{noetherian module}.
\end{proposition}
\begin{proof}
  We will use induction on \( n \). The cases \( n = 0 \) and \( n = 1 \) are trivial.

  Suppose that \( R^{n-1} \) is noetherian. We can identify \( R \) with the submodule of \( R^n \) generated by the vector \( (0, \ldots, 0, 1) \). Two vectors \( \seq{ x_k }_{i=1}^n \) and  \( \seq{ y_k }_{i=1}^n \) in \( R^n \) belong to this submodule if and only if \( x_k = y_k \) for \( k = 1, \ldots, n - 1 \).

  By \fullref{thm:def:ring/quotient_equality_via_difference}, these vectors get mapped to the same vector in the quotient \( R^n / R \). Then \( R^n / R \cong R^{n-1} \), which is noetherian by the inductive hypothesis. By \fullref{thm:def:noetherian_semimodule/quotient}, \( R^{n-1} \) is noetherian if and only if \( R^n \) is noetherian.

  Therefore, \( R^n \) is noetherian.
\end{proof}

\begin{lemma}\label{thm:surjective_endomorphism_over_noetherian_module}
  Every surjective endomorphism \( f: M \to M \) of a noetherian \( R \)-module \( M \) is an isomorphism.
\end{lemma}
\begin{proof}
  Consider the equation
  \begin{equation*}
    f(f(x)) = 0_M.
  \end{equation*}

  It is obviously satisfied for \( x \in \ker f \), but it is also possible that \( f(x) \neq 0_M \) while \( f(f(x)) = 0_M \). Therefore,
  \begin{equation*}
    \ker f \subseteq \ker f^2 \subseteq \ker f^3 \subseteq \cdots,
  \end{equation*}
  where \( f^k \) is \( k \)-fold iterated composition.

  Since \( M \) is noetherian, this sequence stabilizes. Let \( n \) be an index such that \( \ker f^n = \ker f^{n + k} \) for every positive integer \( k \).

  Let \( y \in \ker f^n \). Since \( f \) is surjective, so is \( f^n \), and hence there exists some \( x \) be such that \( f^n(x) = y \). Then \( f^n(y) = f^n(f^n(x)) = 0_M \). But \( \ker f^n = \ker f^{2n} \), hence \( x \in \ker f^n \). Therefore, \( y = f^n(x) = 0 \).

  It follows that \( f^n \) has a trivial kernel. Then so does \( f \). By \fullref{thm:group_homomorphism_trivial_kernel}, this implies that \( f \) is injective, and hence an isomorphism.
\end{proof}

\begin{proposition}\label{thm:surjective_endomorphism_in_free_module}
  Consider the \hyperref[def:coordinate_space]{coordinate space} \( R^n \) for a \hyperref[def:noetherian_semiring]{noetherian ring} \( R \). If the endomorphism \( \varphi: R^n \to R^n \) is surjective, then it is also injective and hence an automorphism.
\end{proposition}
\begin{proof}
  Follows from \fullref{thm:noetherian_free_module} and \fullref{thm:surjective_endomorphism_over_noetherian_module}.
\end{proof}

\begin{theorem}[Hilbert's basis theorem]\label{thm:hilberts_basis_theorem}\mcite[thm. 7.4]{КоцевСидеров2016КомутативнаАлгебра}
  If \( R \) is a \hyperref[def:noetherian_semiring]{noetherian} commutative ring, then so is \( R[X] \).
\end{theorem}
\begin{proof}
  Let \( I \subseteq R[X] \) be an arbitrary ideal. We will prove that \( I \) is finitely generated.

  Denote by \( L \) the set of all leading coefficients of polynomials in \( I \). \Fullref{thm:leading_coefficient_of_product} implies that the leading coefficient of the product of univariate polynomials is the product of their leading coefficients, hence \( L \) is an ideal as a consequence of \( I \) being an ideal.

  As a consequence of \( R \) being noetherian, \( L \) is finitely generated. Suppose that \( L = \set{ l_1, \ldots, l_n } \).

  For every generator \( l_k \), there exists a polynomial \( f_k(X) \) in \( I \) whose leading coefficient is \( l_k \). Denote by \( d_k \) the degree of \( f_k \) and let \( d \) be the maximum of the degrees. We will show that \( I \) itself is equal to the sum of the finitely generated ideals
  \begin{equation*}
    J \coloneqq \underbrace{ \braket{ f_1, \ldots, f_n } + \braket{ X, X^2, \ldots, X^d } }_{ \braket{ f_1, \ldots, f_n, X, X^2, \ldots, X^d } }.
  \end{equation*}

  Let \( g(X) \) be some polynomial from \( I \). Denote by \( m \) its degree and by \( l \) its leading coefficient. We proceed by induction on \( m \) to show that \( g(X) \) belongs to \( J \).
  \begin{itemize}
    \item If \( m \leq d \), then \( g(X) \) belongs to the second ideal \( \braket{ X, X^2, \ldots, X^d } \).

    \item Suppose that \( m > d \) and that every polynomial in \( I \) of degree less than \( m \) belongs to \( J \).

    Since \( l \in L \), it is a linear combination \( l = \sum_{k=1}^n t_k l_k \) with coefficients in \( R \). Consider the polynomial
    \begin{equation*}
      f(X) \coloneqq \sum_{k=1}^n t_k X^{m - d_k} \cdot f_k(X).
    \end{equation*}

    Define \( r(X) \coloneqq g(X) - f(X) \). Since \( f(X) \) belongs to \( I \), \( r(X) \) does too. The difference \( r(X) \) is a polynomial in \( I \) of degree less than \( m \), hence it also belongs to \( J \). Then
    \begin{equation*}
      g(X) = \underbrace{ f(X) }_{\mathclap{\braket{ f_1(X) \cdots, f_n(X) }}} + \overbrace{ r(X) }^J.
    \end{equation*}

    Hence, \( g(X) \) belongs to \( J \).
  \end{itemize}

  Generalizing on \( g(X) \), we conclude that
  \begin{equation*}
    I \subseteq J = \braket{ f_1, \ldots, f_n } + \braket{ X, X^2, \ldots, X^d }.
  \end{equation*}

  Therefore, \( I \) is finitely generated.
\end{proof}

\begin{example}\label{ex:countable_indeterminates_non_noetherian}
  \Fullref{thm:hilberts_basis_theorem} implies that if \( R \) is a noetherian commutative ring, so is \( R[X_1, \ldots, X_n] \). If we instead consider the polynomial algebra \( R[X_1, X_2, X_3, \ldots] \) in countably many indeterminates, it will not be noetherian because we have the following non-stabilizing sequence of ideals:
  \begin{equation*}
    \braket{ X_1 } \subseteq \braket{ X_1, X_2 } \subseteq \braket{ X_1, X_2, X_3 } \subseteq \cdots.
  \end{equation*}
\end{example}

\paragraph{Algebraic dependence}

\begin{definition}\label{def:algebraic_dependence}\mimprovised
  Let \( M \) be an \hyperref[def:algebra_over_ring]{algebra} over a \hyperref[def:ring/commutative]{commutative ring} \( R \). Fix some indexed set \( \seq{ u_e }_{e \in E} \) from \( M \) and consider the \hyperref[def:polynomial_algebra]{polynomial algebra} \( R[X_e \given e \in E] \) over some fixed indeterminates.

  We say that the elements of \( E \) are \term[ru=алгебрически независимые (елементы) (\cite[408]{Винберг2014КурсАлгебры})]{algebraically independent} if any of the following conditions hold:

  \begin{thmenum}
    \thmitem{def:algebraic_dependence/direct} If \( \seq{ u_e }_{e \in E} \) is a root of some polynomial \( f(X_e \given e \in E) \) with coefficients in \( R \), then \( f \) is the zero polynomial.

    \thmitem{def:algebraic_dependence/evaluation} The \hyperref[thm:polynomial_algebra_universal_property]{evaluation map} \( \Phi_u: R[X_e \given e \in E] \to M \) sending \( X_e \) to \( u_e \) is injective.
  \end{thmenum}
\end{definition}
\begin{comments}
  \item Unsurprisingly, if the elements of \( E \) are not \term{algebraically independent}, we say that they are \term{algebraically dependent}.
  \item Compare this concept to linear dependence defined in \fullref{def:linear_dependence}.
\end{comments}
\begin{defproof}
  \ImplicationSubProof{def:algebraic_dependence/direct}{def:algebraic_dependence/evaluation} Suppose that \( \Phi_e \) is injective and that there exists a polynomial \( f(X_e \given e \in E) \) such that \( \Phi_e(f) = 0_M \).

  For any other polynomial \( g(X_e \given e \in E) \), we have \( \Phi_e(f g) = 0_M \), and hence either \( f \) is the zero polynomial or the evaluation map is not injective. We have assumed that it is injective, hence \( f \) is the zero polynomial.

  \ImplicationSubProof{def:algebraic_dependence/evaluation}{def:algebraic_dependence/direct} Conversely, suppose that \( E \) is a root only of the zero polynomial. Let \( \Phi_e(f) = \Phi_e(g) \). Then \( \seq{ u_e }_{e \in E} \) is a root of \( f - g \) and hence the latter is the zero polynomial. But this implies that \( f = g \). Hence, the evaluation map is injective.
\end{defproof}

\begin{proposition}\label{thm:def:algebraic_dependence}
  \hyperref[def:algebraic_dependence]{Algebraic (in)dependence} for the \hyperref[def:ring/commutative]{commutative ring} \( R \) has the following basic properties:
  \begin{thmenum}
    \thmitem{thm:def:algebraic_dependence/element} Every nonzero element of \( R \) is algebraically dependent over \( R \).

    \thmitem{thm:def:algebraic_dependence/n_independent} Different indeterminates are algebraically independent over \( R \).

    \thmitem{thm:def:algebraic_dependence/two_univariate_dependent}\mcite{MathOF:univariate_polynomials_algebraically_dependent} Every two univariate polynomials in \( R \) are algebraically dependent over \( R \).

    \thmitem{thm:def:algebraic_dependence/n_plus_one_dependent} Every \( n + 1 \) polynomials in \( R[X_1, \ldots, X_n] \) are algebraically dependent over \( R \).
  \end{thmenum}
\end{proposition}
\begin{proof}
  \SubProofOf{thm:def:algebraic_dependence/element} Every nonzero element \( x \) is a root of the univariate polynomial \( X - x \).

  \SubProofOf{thm:def:algebraic_dependence/n_independent} Fix some indeterminates \( X_1, \ldots, X_n \). For a nonzero polynomial \( f(Y_1, \ldots, Y_n) \), the evaluation \( \Phi_{X_1, \dots, X_n}(f) \) is zero if and only if \( f \) is zero because the evaluation simply renames the variables.

  \SubProofOf{thm:def:algebraic_dependence/two_univariate_dependent} Fix polynomials \( p(X) \) and \( q(X) \) over \( R \). We will construct a polynomial \( f(Y, Z) \) over \( R \) such that \( \Phi_{p,q}(f) = 0 \).

  If \( p(X) \) is zero, simply define \( f(Y, Z) \coloneqq Z \). If \( q(X) \) is zero, put \( f(Y, Z) \coloneqq Y \).

  Suppose that both are nonzero; denote by \( n \) be the degree of \( p(X) \) and by \( m \) the degree of \( q(X) \). We will consider polynomials of the form \( p^l q^k \).

  Fix a positive integer \( d \). We want the degree of \( p^l q^k \) to be at most \( d \). If
  \begin{align*}
    l < \frac d {2n} && k < \frac d {2m},
  \end{align*}
  then, by \fullref{thm:polynomial_degree_arithmetic/product}, either \( p^l q^k \) is the zero polynomial or
  \begin{equation*}
    \deg(p^l q^k) = nl + km < \frac d 2 + \frac d 2 = d.
  \end{equation*}

  These polynomials are all in
  \begin{equation*}
    L_d \coloneqq \linspan\set{ 1, X, X^2, X^3, \ldots, X^{d-1} }.
  \end{equation*}

  This is a module of \hyperref[def:module_rank]{rank} \( d \).

  Furthermore, there are \( d^2 / 4nm \) such polynomials. If \( d > 4nm \), there are more polynomials of the form \( p^l q^k \) than the \hyperref[def:module_rank]{rank} of \( L_d \). Hence, every \( d + 1 \) such polynomials are linearly dependent, and hence there exists some linear combination
  \begin{equation*}
    a_1 p^{l_1} q^{k_1} + \cdots + a_{d+1} p^{l_{d+1}} q^{k_{d+1}} = 0.
  \end{equation*}

  We can thus define the following polynomial in \( R[Y, Z] \):
  \begin{equation*}
    f(Y, Z) \coloneqq a_1 Y^{l_1} Z^{k_1} + \cdots + a_{d+1} Y^{l_{d+1}} Z^{k_{d+1}}.
  \end{equation*}

  Then clearly \( \Phi_{p,q}(f) = 0 \), so \( p \) and \( q \) are algebraically dependent over \( R \).

  \SubProofOf{thm:def:algebraic_dependence/n_plus_one_dependent} Let \( p_1, \ldots, p_{n+1} \) be polynomials in \( R[X_1, \ldots, X_{n-1}][X_n] \). By \fullref{thm:def:algebraic_dependence/two_univariate_dependent}, the polynomials \( p_n \) and \( p_{n+1} \) are algebraically dependent over \( R[X_1, \ldots, X_{n-1}] \).

  Let \( f(Y_n, Y_{n+1}) \) be a polynomial in \( R[X_1, \ldots, X_{n-1}][Y_n, Y_{n+1}] \) such that \( \Phi_{p_n,p_{n+1}}(f) = 0 \). The coefficients of \( f \) are themselves polynomials. Let
  \begin{equation*}
    \widehat{f}(Y_1, \ldots, Y_{n-1}, Y_n, Y_{n+1})
  \end{equation*}
  be the polynomial obtained from
  \begin{align*}
    f(X_1, \ldots, X_{n-1}, Y_n, Y_{n+1})
  \end{align*}
  by renaming the corresponding variables.

  Then \( \Phi_{p_1,\ldots,p_{n+1}}(\widehat{f}) = 0 \). Therefore, \( p_1, \ldots, p_{n+1} \) are algebraically dependent over \( R \).
\end{proof}

\begin{proposition}\label{thm:change_of_polynomial_basis}
  Let \( R \) be a \hyperref[def:ring/commutative]{commutative ring}, let \( X_1, \ldots, X_n \) be arbitrary symbols and consider some polynomials \( Y_k(X_1, \ldots, X_n) \), \( k = 1, \ldots, n \) from \( R[X_1, \ldots, X_n] \) that are \hyperref[def:algebraic_dependence]{algebraically independent}.

  Then the \hyperref[thm:polynomial_algebra_universal_property]{evaluation map} \( \Phi: R[X_1, \ldots, X_n] \to R[X_1, \ldots, X_n][Y_1, \ldots, Y_n] \), given by
  \begin{equation*}
    X_k \mapsto Y_k(X_1, \ldots, X_n),
  \end{equation*}
  is an isomorphic embedding of \( R \)-algebras.
\end{proposition}
\begin{comments}
  \item A polynomial in the image of \( \Phi \) does not explicitly contain any of \( X_1, \ldots, X_n \), hence we can regard it as a polynomial in the indeterminates \( Y_1, \ldots, Y_n \) and then regard \( \Phi \) as an isomorphism between \( R[X_1, \ldots, X_n] \) and \( R[Y_1, \ldots, Y_n] \).

  \item This is a generalization of the \hyperref[con:change_of_basis]{change of basis} of linear operators to polynomials.
\end{comments}
\begin{proof}
  The evaluation map is a homomorphism of \( R \)-algebras, hence it remains only to show that it is injective. Suppose that \( \Phi(f) = \Phi(g) \) for some polynomials \( f \) and \( g \) from \( R[X_1, \ldots, X_n] \).

  Then \( \Phi(f - g) \) is the zero polynomial. Since \( Y_1, \ldots, Y_n \) are algebraically independent, it follows that \( f - g \) is the zero polynomial, that is, \( f = g \).
\end{proof}

\paragraph{Quaternions}

We have defined the field \( \BbbC \) of \hyperref[def:complex_numbers]{complex numbers} as the \hyperref[def:algebra_over_ring/quotient]{quotient algebra} \( \BbbR[X] / \braket{ X^2 - 1 } \). Furthermore, as discussed in \fullref{rem:real_field_extensions}, there are no nontrivial finite-dimensional field extensions of \( \BbbC \).

Nevertheless, we can generalize the construction by instead considering \hyperref[def:noncommutative_polynomial_algebra]{noncommutative polynomials}.

\begin{definition}\label{def:quaternion_algebra}\mimprovised
  We define the \term[bg=кватерниони (\cite[454]{Обрешков1962ВисшаАлгебра}), ru=алгебра кватернионов (\cite[41]{Винберг2014КурсАлгебры}), en=quaternion algebra (\cite[exerc. III.1.12]{Aluffi2009Algebra})]{quaternion algebra} \( \BbbH \) as the \hyperref[def:algebra_over_ring/quotient]{quotient} of the \hyperref[def:noncommutative_polynomial_algebra]{noncommutative polynomial algebra} \( \BbbR\braket{ i, j, k } \) by the congruence \hyperref[def:first_order_generated_congruence]{generated} by the following relation:
  \begin{align*}
    i^2 \sim -1, && jk \sim i, && (kj \sim -i), \\
    j^2 \sim -1, && ki \sim j, && (ik \sim -j), \\
    k^2 \sim -1, && ij \sim k, && (ji \sim -k).
  \end{align*}

  \begin{figure}[!ht]
    \centering
    \includegraphics[page=1]{output/def__quaternions}
    \caption{An illustration of the congruence in \fullref{def:quaternion_algebra} --- by multiplying the head of an arc by its tail, we obtain the third vertex; by multiplying the tail by the head, we instead obtain the negation.}\label{fig:def:quaternion_algebra}
  \end{figure}

  Since every possible pair of symbols can now be reduced to a single one, we are left with only four monomials, and we can denote a quaternion as a (real) linear combination:
  \begin{equation*}
    a + bi + cj + dk.
  \end{equation*}
\end{definition}
\begin{comments}
  \item The last column defining the congruence is redundant because
  \begin{equation*}
    kj \sim (-1)^2 kj \sim (-1) j^2 (kj) \sim -j (jk) j \sim -j (ij) \sim -jk \sim -i
  \end{equation*}
  and analogously for the others.

  \item The letter \( \BbbH \) is chosen in honor of William Hamilton, who introduced them in a multi-part paper over the course of several years, from 1846 to 1850. The compiled paper can be found in \cite{Hamilton2000Quaternions}.
\end{comments}

\begin{proposition}\label{thm:def:quaternion_algebra}
  \hyperref[def:quaternion_algebra]{Quaternions} have the following basic properties:
  \begin{thmenum}
    \thmitem{thm:def:quaternion_algebra/commutative} Multiplication of quaternions is not commutative.
    \thmitem{thm:def:quaternion_algebra/inverse} \( \BbbH \) is a \hyperref[def:division_ring]{division ring} --- every nonzero quaternion \( a + bi + cj + dk \) has a (two-sided) multiplicative inverse:
    \begin{equation}\label{eq:thm:def:quaternion_algebra/inverse}
      \frac {a - bi - cj - dk} {a^2 + b^2 + c^2 + d^2}
    \end{equation}
  \end{thmenum}
\end{proposition}
\begin{proof}
  \SubProofOf{thm:def:quaternion_algebra/commutative} We have \( ij = k = -ji \).
  \SubProofOf{thm:def:quaternion_algebra/inverse} We have
  \begin{align*}
    &\phantom{{}={}}
    (a + bi + cj + dk) (a - bi - cj - dk)
    = \\ &=
    [\hi{a^2} - (ab)i - (ac)j - (ad)k]
    +
    [(ab)i + \hi{b^2} - (bc)k + (bd)j]
    + \\ &+
    [(ac)j + (bc)k + \hi{c^2} - (cd)i]
    +
    [(ad)k - (bd)j + (cd)i + \hi{d^2}]
    = \\ &=
    a^2 + b^2 + c^2 + d^2.
  \end{align*}

  Furthermore,
  \begin{align*}
    &\phantom{{}={}}
    (a - bi - cj - dk) (a + bi + cj + dk)
    = \\ &=
    [a + (-b) i + (-c) j + (-d) k] [a - (-b) i - (-c) j - (-d) k]
    = \\ &=
    a^2 + (-b)^2 + (-c)^2 + (-d)^2.
  \end{align*}
\end{proof}

\begin{definition}\label{def:quaternionic_group}\mcite[128]{Aluffi2009Algebra}
  We also define the \term{quaternionic group} \( Q_8 \) as the multiplicative subgroup \( \braket{ 1, i, j, k } \) of the \hyperref[def:quaternion_algebra]{quaternion algebra} \( \BbbH \) consisting of the four basis elements and their inverses.
\end{definition}
\begin{comments}
  \item It is tempting to define the quaternion algebra as the group algebra of \( Q_8 \), but that would make the algebra eight-dimensional over \( \BbbR \), with distinct coordinates for \( i \) and \( -i \), \( j \) and \( -j \) and so forth.
\end{comments}

\begin{example}\label{ex:quaternion_polynomial_with_infinitely_many_roots}
  Consider the \hyperref[def:noncommutative_polynomial_algebra]{noncommutative polynomial} \( f(X) = X^2 + 1 \) over the \hyperref[def:quaternion_algebra]{quaternion algebra} \( \BbbH \).

  The square of the quaternion \( x \coloneqq bi + cj + dk \) is
  \begin{align*}
    &\phantom{{}={}}
    (bi + cj + dk) (bi + cj + dk)
    = \\ &=
    [-\hi{b^2} + (bc)k - (bd)j]
    + \\ &+
    [-(bc)k - \hi{c^2} + (cd)i]
    +
    [(bd)j - (cd)i - \hi{d^2}]
    = \\ &=
    -b^2 - c^2 - d^2.
  \end{align*}

  Therefore, if \( b^2 + c^2 + d^2 = 1 \), then \( x^2 = -1 \), and \( f(x) = 0 \). Therefore, if we extend \fullref{def:root_of_polynomial} of polynomial roots to noncommutative polynomials, we obtain that \( f(X) \) has infinitely many roots.

  This contrasts with the commutative case, where \fullref{thm:def:integral_domain/root_limit} implies that \hyperref[def:entire_semiring]{entire} rings have at most \( n \) roots for a polynomial of degree \( n \).
\end{example}

  \subsection{Integral domains}\label{subsec:integral_domains}

\paragraph{Integral domains}

\begin{definition}\label{def:integral_domain}\mcite[def. III.1.10]{Aluffi2009Algebra}
  An \term[bg=област на цялостност (\cite[def. V.4]{ГеновИПр1991Алгебра}), ru=область целостности (\cite[def. 3.5.1]{Винберг2014Алгебра})]{integral domain} is an \hyperref[def:ring/trivial]{nontrivial} \hyperref[def:entire_semiring]{entire} \hyperref[def:ring/commutative]{commutative (unital) ring}.
\end{definition}

\begin{proposition}\label{thm:def:integral_domain}
  \hyperref[def:integral_domain]{Integral domains} have the following basic properties:
  \begin{thmenum}
    \thmitem{thm:def:integral_domain/subring} Any nontrivial \hyperref[def:ring/submodel]{subring} of an integral domain is also an integral domain.

    \thmitem{thm:def:integral_domain/polynomial_ring} A commutative ring \( R \) is an integral domain if and only if its \hyperref[def:polynomial_algebra]{polynomial ring} \( R[X] \) is.

    \thmitem{thm:def:integral_domain/root_limit} In an integral domain, the \hyperref[def:polynomial_root]{multiset of roots} of a univariate nonzero polynomial of \hyperref[def:polynomial_degree]{degree} \( n \) has \hyperref[def:multiset]{multiset cardinality} at most \( n \).

    In other words, a polynomial of degree cannot have more that \( n \) roots, counting multiple roots.

    \thmitem{thm:def:integral_domain/polynomial_divides} The polynomial \( p(X) \) divides \( X^n \) if and only if \( p(X) = aX^m \), where \( a \) is invertible and \( m \leq n \).
  \end{thmenum}
\end{proposition}
\begin{proof}
  \SubProofOf{thm:def:integral_domain/subring} Trivial.

  \SubProofOf{thm:def:integral_domain/polynomial_ring}

  \NecessitySubProof* If \( R[X] \) is an integral domain, by \fullref{thm:def:integral_domain/subring}, so is \( R \).

  \SufficiencySubProof* It is sufficient to prove the statement for one indeterminate. If \( p(X) \) and \( q(X) \) are nonzero polynomials, then so is \( p(X) q(X) \) by \fullref{thm:def:polynomial_degree/product}.

  \SubProofOf{thm:def:integral_domain/root_limit} We will use induction on the degree. Zero-degree polynomials clearly have zero roots. Suppose that the statement holds for polynomials of degree \( n - 1 \), and let \( p(X) \) have degree \( n \).

  If \( p(X) \) has a root \( u \), by the equivalence in \fullref{def:polynomial_root}, \( (X - u) \) divides \( p(X) \). Then \( p(X) / (X - u) \) has degree \( n - 1 \) by \fullref{thm:def:polynomial_degree/product}. Let \( M \) be the multiset of roots of \( p(X) / (X - u) \). After adding \( u \) to \( M \), we have incremented its total cardinality by \( 1 \), thus making it at most \( n \).

  \SubProofOf{thm:def:integral_domain/polynomial_divides} Suppose that \( X^n = p(X) \cdot q(X) \).

  We have
  \begin{equation*}
    X^n
    =
    p(X) \cdot q(X)
    =
    \sum_{k=0}^\infty (\sum_{m+l=k} a_m b_l) X^k.
  \end{equation*}

  Then
  \begin{equation*}
    a_m b_l = \begin{cases}
      1, &m + l = n, \\
      0, &\T{otherwise}
    \end{cases}
  \end{equation*}

  Since we are working over a domain and there are no zero divisors, there exists only one pair of nonnegative integers \( m \) and \( l \) such that \( a_m b_l = 1 \). It follows that \( p(X) = a_m X^m \) and \( q(X) = a_l X^l \).
\end{proof}

\begin{proposition}\label{thm:quotient_by_prime_ideal}
  The ideal \( P \) of the \hyperref[def:ring/commutative]{commutative ring} \( R \) is \hyperref[def:semiring_ideal/prime]{prime} if and only if the \hyperref[def:ring/quotient]{quotient ring} \( R / P \) is an \hyperref[def:integral_domain]{integral domain}.
\end{proposition}
\begin{comments}
  \item See \fullref{thm:quotient_by_maximal_ideal} for the corresponding statement for \hyperref[def:semiring_ideal/maximal]{maximal ideals} in possibly noncommutative rings.
\end{comments}
\begin{proof}
  \SufficiencySubProof Suppose that \( P \) is a prime ideal. Clearly \( R / P \) is a commutative ring. Since \( P \) is a proper ideal, \( R / P \) must be nontrivial. We will show that it is an \hyperref[def:entire_semiring]{entire ring}.

  Let \( [x] [y] = [0] = P \) (where \( [x] = x + P \) is the coset of \( x \) in \( R / P \)). By definition,
  \begin{equation*}
    [x] [y] = (x + P) (y + P) = (xy + P),
  \end{equation*}
  which implies \( xy + P = P \) and hence \( xy \in P \). Since \( P \) is prime, by \fullref{thm:def:semiring_ideal/prime_pointwise}, we have \( x \in P \) or \( y \in P \).

  Therefore, \( [x] = [0] \) or \( [y] = [0] \). Generalizing on \( x \) and \( y \), we can conclude that \( R / P \) is entire, and thus an integral domain.

  \NecessitySubProof Suppose that \( R / P \) is an integral domain. Since \( R / P \) is nontrivial, \( P \) must be a proper ideal. We will show that it satisfies \fullref{thm:def:semiring_ideal/prime_pointwise}.

  Let \( xy \in P \). We have
  \begin{equation*}
    P = [0] = [xy] = [x] [y],
  \end{equation*}
  hence \( [x] \) and \( [y] \) are zero divisors in \( R / P \). But \( R / P \) is entire, hence either \( [x] \) or \( [y] \) must be zero. That is, either \( x \in P \) or \( y \in P \).

  Generalizing on \( x \) and \( y \), we can conclude that \( P \) is a prime ideal.
\end{proof}

\paragraph{Divisibility in domains}

\begin{proposition}\label{thm:ring_entire_iff_unique_quotient}
  A \hyperref[def:ring/commutative]{commutative ring} is \hyperref[def:entire_semiring]{entire} if and only if, whenever \( y \) \hyperref[def:divisibility]{divides} \( x \), there exists a \hi{unique} element \( z \) such that \( x = yz \).
\end{proposition}
\begin{proof}
  \SufficiencySubProof Suppose that \( R \) is entire. Let \( x = yz = yz' \). \Fullref{thm:def:ring/cancellable_iff_not_zero_divisor} implies that we can cancel \( y \) to obtain \( z = z' \). This demonstrates uniqueness.

  \NecessitySubProof Suppose instead that uniqueness holds and let \( 0 = yz \). By absorption, we have \( 0 = y0 \), and thus \( z = 0 \), showing that \( y \) has no nontrivial zero divisors.

  Since \( y \) was arbitrary, we conclude that the ring is entire.
\end{proof}

\begin{example}\label{ex:nonunique_divisor}
  The requirement in \fullref{thm:ring_entire_iff_unique_quotient} for the ring to be entire is essential --- \fullref{thm:idempotent_division} implies that (nontrivial) multiplicatively idempotent elements are nontrivial zero divisor.

  For example, if \( R \) is nontrivial, consider the \hyperref[thm:matrix_algebra]{matrix algebra} \( R^{2 \times 2} \) and the idempotent matrix
  \begin{equation*}
    \begin{pNiceMatrix}
      1 & 0 \\
      0 & 0
    \end{pNiceMatrix}
  \end{equation*}

  Because it is idempotent, it divides itself:
  \begin{equation*}
    \begin{pNiceMatrix}
      1 & 0 \\
      0 & 0
    \end{pNiceMatrix}^2
    =
    \begin{pNiceMatrix}
      1 & 0 \\
      0 & 0
    \end{pNiceMatrix}.
  \end{equation*}

  But another possible quotient is the identity:
  \begin{equation*}
    \begin{pNiceMatrix}
      1 & 0 \\
      0 & 0
    \end{pNiceMatrix}
    \begin{pNiceMatrix}
      1 & 0 \\
      0 & 1
    \end{pNiceMatrix}
    =
    \begin{pNiceMatrix}
      1 & 0 \\
      0 & 1
    \end{pNiceMatrix}
    \begin{pNiceMatrix}
      1 & 0 \\
      0 & 0
    \end{pNiceMatrix}
    =
    \begin{pNiceMatrix}
      1 & 0 \\
      0 & 0
    \end{pNiceMatrix}.
  \end{equation*}
\end{example}

\begin{definition}\label{def:domain_quotient}\mimprovised
  By \fullref{thm:ring_entire_iff_unique_quotient}, in an \hyperref[def:integral_domain]{integral domain}, if \( y \) divides \( x \), then there exists a unique element \( z \) such that \( x = yz \). \Fullref{thm:def:ring_localization/divisibility} motivates the notation \( x / y \).

  We call \( x / y \) the \term[ru=частное/отношение (\cite[16]{Винберг2014Алгебра})]{quotient} of \( x \) by \( y \).
\end{definition}

\begin{proposition}\label{thm:domain_quotient_inverse}
  If the \hyperref[def:domain_quotient]{quotient} \( x / y \) is invertible, then \( y \) divides \( x \).
\end{proposition}
\begin{comments}
  \item Using the terminology from \fullref{def:domain_divisibility/associates/direct}, \( x \) and \( y \) are thus associates.
\end{comments}
\begin{proof}
  If \( x = yz \) and \( z \) is invertible, then \( x z^{-1} = y \) and thus \( y \) is invertible.
\end{proof}

\begin{definition}\label{def:domain_divisibility}
  We will introduce several notions related to \hyperref[def:divisibility]{divisibility} in \hyperref[def:integral_domain]{integral domains}.

  \begin{thmenum}
    \thmitem{def:domain_divisibility/associates} We say that \( x \) and \( y \) are \term[bg=асоциирани (\cite[142]{ГеновИПр1991Алгебра}), ru=ассоциированные (\cite[118]{Винберг2014Алгебра})]{associated} if any of the following conditions hold:
    \begin{thmenum}
      \thmitem{def:domain_divisibility/associates/direct}\mcite[246]{Aluffi2009Algebra} Both \( x \mid y \) and \( y \mid x \).

      \thmitem{def:domain_divisibility/associates/invertible}\mcite[393]{Knapp2016BasicAlgebra} There exists an \hyperref[def:divisibility/invertible]{invertible element} \( u \) such that \( x = uy \).

      \thmitem{def:domain_divisibility/associates/ideals}\mcite[246]{Aluffi2009Algebra} The \hyperref[def:semiring_ideal/principal]{principal ideals} \( \braket{ x } \) and \( \braket{ y } \) are equal.
    \end{thmenum}

    \thmitem{def:domain_divisibility/irreducible} We say that the nonzero non-invertible element \( x \) is \term[bg=неразложим (\ref{rem:prime_and_irreducible_terminology}), ru=неприводимый (\ref{rem:prime_and_irreducible_terminology})]{irreducible} if any of the following conditions hold:
    \begin{thmenum}
      \thmitem{def:domain_divisibility/irreducible/direct}\mcite[388]{Knapp2016BasicAlgebra} Whenever \( x = yz \), at least one of \( y \) or \( z \) is invertible.
      \thmitem{def:domain_divisibility/irreducible/ideals} \( \braket{ x } \) is maximal among all proper principal ideals. Maximality means that, if \( \braket{ x } \subseteq \braket{ y } \) for some nonzero non-invertible \( y \), then \( \braket{ x } = \braket{ y } \).
    \end{thmenum}

    \thmitem{def:domain_divisibility/prime} We say that the nonzero element \( x \) is \term[bg=прост (\cref{rem:prime_and_irreducible_terminology}), ru=простой (\cref{rem:prime_and_irreducible_terminology})]{prime} if any of the following equivalent conditions hold:
    \begin{thmenum}
      \thmitem{def:domain_divisibility/prime/direct}\mcite[388]{Knapp2016BasicAlgebra} If \( x \mid yz \), then \( x \mid y \) or \( x \mid z \) (or both).
      \thmitem{def:domain_divisibility/prime/ideals}\incite[113]{Lang2002Algebra} The ideal \( \braket{ x } \) is \hyperref[def:semiring_ideal/prime]{prime}.
    \end{thmenum}

    This definition is motivated by \fullref{thm:euclids_lemma}. It also applies more generally for \hyperref[def:entire_semiring]{entire semirings}.
  \end{thmenum}
\end{definition}
\begin{defproof}
  \SubProofOf{def:domain_divisibility/associates}
  \ImplicationSubProof*{def:domain_divisibility/associates/direct}{def:domain_divisibility/associates/invertible} If \( x \mid y \) and \( y \mid x \), then there exist \( a \) and \( b \) such that \( x = ay \) and \( y = bx \). Hence, \( x = abx \). Since we are working in an integral domain, we can cancel \( x \) to obtain \( ab = 1 \). Therefore, both \( a \) and \( b \) are \hyperref[def:divisibility/invertible]{units}.

  \ImplicationSubProof*{def:domain_divisibility/associates/invertible}{def:domain_divisibility/associates/ideals} Suppose that \( x = uy \) for some invertible element \( u \). If \( z \) is in \( \braket{ x } \), then \( x = uy \) divides \( z \) and hence \( y \) also divides \( z \), implying that \( \braket{ x } \subseteq \braket{ y } \). We obtain the converse inclusion by noting that \( y = u^{-1} x \).

  \ImplicationSubProof*{def:domain_divisibility/associates/ideals}{def:domain_divisibility/associates/direct} If \( \braket{ x } = \braket{ y } \), then, by \fullref{thm:def:semiring_ideal/division}, \( x \mid y \) and \( y \mid x \).

  \SubProofOf{def:domain_divisibility/irreducible}
  \ImplicationSubProof*{def:domain_divisibility/irreducible/direct}{def:domain_divisibility/irreducible/ideals} Suppose that \( x \) is not invertible and that \( x = yz \) implies that \( y \) or \( z \) is invertible. Since we are working in an integral domain, \( x \) is necessarily nonzero.

  Let \( \braket{ x } \subseteq \braket{ w } \) for some non-invertible \( w \). By \fullref{thm:def:semiring_ideal/division}, \( w \mid x \). Then there exists some element \( a \) such that \( x = aw \). Since \( w \) is not invertible by assumption, \( a \) must be invertible. By the equivalent definitions of associates in a domain, \( \braket{ x } = \braket{ w } \).

  \ImplicationSubProof*{def:domain_divisibility/irreducible/ideals}{def:domain_divisibility/irreducible/direct} Suppose that \( \braket{ x } \) is maximal among nonzero proper principal ideals.

  Let \( x = yz \). If, without loss of generality, \( \braket{ x } \subseteq \braket{ y } \), then \( \braket{ x } = \braket{ y } \) and, again by the equivalent conditions for associates, there exists some invertible element \( u \) such that \( x = uy \). Cancelling \( y \) in \( yu = yz \), we obtain \( u = z \). Hence, \( z \) is invertible.

  \SubProofOf{def:domain_divisibility/prime} Trivial.
\end{defproof}

\begin{remark}\label{rem:prime_and_irreducible_terminology}
  There is a certain discrepancy in the literature regarding prime and irreducible elements. It does not matter much because, in \hyperref[def:factorial_domain]{factorial domains}, an element is prime if and only if it is irreducible.

  \begin{itemize}
    \item \incite[111]{Lang2002Algebra} uses the term \enquote{irreducible} for what we call \enquote{prime}, and later notes that, in factorial domains, the principal ideal of an irreducible element is prime, and thus it makes sense to call irreducible elements prime in this context. In the Russian translation of the book, in \cite[89]{Ленг1968Алгебра}, irreducible elements are called \enquote{неприводимые элементы}, and prime elements are called \enquote{простые елементы}.

    Later Russian authors, for example \incite[def. 3.5]{Винберг2014Алгебра} and \incite[30]{Шафаревич1999Алгебра}, call irreducible elements \enquote{простые} (which is used as a translation for \enquote{prime} elsewhere in the book, e.g. for prime ideals). This also transfers to Bulgarian books --- for example, \enquote{прост елемент} is used by \incite[def. VI.5]{ГеновИПр1991Алгебра}.

    In the context of polynomials over fields, however, \incite[121]{Винберг2014Алгебра} and \incite[19]{Шафаревич1999Алгебра} use \enquote{неприводимый многочлен} and \incite[def. VI.3]{ГеновИПр1991Алгебра} use \enquote{неразложим полином} for what we call an irreducible polynomial, and avoid mentioning prime polynomials.

    \item On the other hand, a distinction between \enquote{prime} and \enquote{irreducible} elements in general integral domains is made in modern anglophone literature --- for example by \incite[343]{Jacobson1985AlgebraPart1}, \incite[107]{Rotman2010Algebra}, \incite[388,389]{Knapp2016BasicAlgebra} and \incite[def. V.1.6]{Aluffi2009Algebra}.
  \end{itemize}
\end{remark}

\begin{proposition}\label{thm:def:domain_divisibility}
  The divisibility notions from \fullref{def:domain_divisibility} have the following basic properties:
  \begin{thmenum}
    \thmitem{thm:def:domain_divisibility/prime_is_irreducible} Every \hyperref[def:domain_divisibility/prime]{prime element} is \hyperref[def:domain_divisibility/irreducible]{irreducible}.

    The converse is true in \hyperref[def:factorial_domain]{factorial domains}.

    \thmitem{thm:def:domain_divisibility/irreducible_in_polynomial_ring} An element of the domain \( D \) is \hyperref[def:domain_divisibility/irreducible]{irreducible} in \( D \) if and only if it is irreducible in \( D[X] \).

    \thmitem{thm:def:domain_divisibility/irreducible_polynomial_coefficients} If a polynomial is irreducible in \( D[X] \), its nonzero non-invertible coefficients are irreducible in \( D \).

    \thmitem{thm:def:domain_divisibility/associates_and_isomorphisms} If \( \varphi: D \to E \) is an isomorphism, then \( x \) and \( y \) are \hyperref[def:domain_divisibility/associates]{associates} in \( D \) if and only if \( \varphi(x) \) and \( \varphi(y) \) are associates in \( E \).

    \thmitem{thm:def:domain_divisibility/primes_and_isomorphisms} If \( \varphi: D \to E \) is an isomorphism, then \( x \) is \hyperref[def:domain_divisibility/prime]{prime} (resp. \hyperref[def:domain_divisibility/irreducible]{irreducible}) in \( D \) if and only if \( \varphi(x) \) is prime (resp. irreducible) in \( E \).
  \end{thmenum}
\end{proposition}
\begin{proof}
  \SubProofOf{thm:def:domain_divisibility/prime_is_irreducible} Let \( x \) be a prime element and suppose that \( x = yz \). Then \( x \) divides \( y \) or \( z \). If, without loss of generality, \( x \) divides \( y \), then \( x \) and \( y \) are \hyperref[def:domain_divisibility/associates]{associates}, and, by the equivalence of conditions in \fullref{def:domain_divisibility/associates}, \( z \) must be invertible.

  \SubProofOf{thm:def:domain_divisibility/irreducible_in_polynomial_ring}

  \SufficiencySubProof* Suppose that \( x \) is irreducible in \( D \) and let \( x = y(X) z(X) \). By \fullref{thm:def:polynomial_degree/product}, both \( y(X) \) and \( z(X) \) must be constant polynomials. Therefore, they are scalars, and since \( x \) is irreducible, \( y \) or \( z \) is invertible. By \fullref{thm:def:polynomial_algebra/invertible}, if \( y \) is invertible in \( D \), it is invertible in \( D[X] \).

  Generalizing on \( x \), it follows that every irreducible element in \( D \) is also irreducible in \( D[X] \).

  \NecessitySubProof* Suppose that \( x \) is irreducible in \( D[X] \) and let \( x = yz \). Then \( y \) or \( z \) is invertible in \( D[X] \), and thus again by \fullref{thm:def:polynomial_algebra/invertible}, it is invertible in \( D \).

  Generalizing on \( x \), it follows that every element of \( D \) that is irreducible in \( D[X] \) is also irreducible in \( D \).

  \SubProofOf{thm:def:domain_divisibility/irreducible_polynomial_coefficients} Suppose that the polynomial
  \begin{equation*}
    p(X) = \sum_{k=0}^n a_k x^k
  \end{equation*}
  is irreducible in \( D[X] \).

  Suppose that the non-invertible element \( b \) divides \( a_0, a_1, \ldots, a_n \). Then \( b \) also divides \( p(X) \), and \fullref{thm:def:domain_divisibility/irreducible_in_polynomial_ring} implies that \( b \) is not invertible in \( D[X] \). But this contradicts that \( p(X) \) is irreducible.

  \SubProofOf{thm:def:domain_divisibility/associates_and_isomorphisms} Follows from \fullref{thm:divisibility_and_isomorphisms}.

  \SubProofOf{thm:def:domain_divisibility/primes_and_isomorphisms} Follows from \fullref{thm:divisibility_and_isomorphisms}.
\end{proof}

\begin{example}\label{ex:def:domain_divisibility}
  We list some examples of the divisibility notions from \fullref{def:domain_divisibility}:
  \begin{thmenum}
    \thmitem{ex:def:domain_divisibility/integers} \hyperref[def:prime_number]{Prime numbers} are irreducible integers by definition. By \fullref{thm:euclids_lemma}, they are also prime elements.

    The inverse \( -p \) of the prime number \( p \) is also irreducible and thus a prime element in \( \BbbZ \), but convention requires \enquote{prime numbers} to be positive.

    So, we make a distinction between \enquote{prime numbers} and \enquote{prime elements of \( \BbbZ \)}

    \thmitem{ex:def:domain_divisibility/irreducible_not_prime}\mcite[388]{Knapp2016BasicAlgebra} Consider the ring \( \BbbZ[\sqrt{-5}] \) obtained by \hyperref[def:semiring_adjunction]{adjoining} the complex number \( \sqrt{-5} \) to \( \BbbZ \). We will examine irreducible elements in this ring and show that irreducible elements need not be prime.

    Based on our discussion in \fullref{ex:def:divisibility/i_sqrt5}, we can conclude that \( 1 \) and \( -1 \) are the units of \( \BbbZ \).

    Let \( n \) be a positive integer strictly less than \( 5 \). Suppose that
    \begin{equation}\label{eq:ex:def:domain_divisibility/irreducible_not_prime/decomposition}
      n = \parens[\Big]{ a + b \sqrt{-5} }\parens[\Big]{ c + d \sqrt{-5}}.
    \end{equation}

    Then
    \begin{equation}\label{eq:ex:def:domain_divisibility/irreducible_not_prime/abs}
      \abs{n}^2
      =
      \abs[\Big]{a + b \sqrt{-5}}^2 \cdot \abs[\Big]{c + d \sqrt{-5}}^2
      =
      (a^2 + 5b^2) (c^2 + 5d^2)
      =
      a^2 c^2 + 5a^2 d^2 + 5b^2 c^2 + 25 b^2 d^2.
    \end{equation}

    Since \( n < 5 \), the product \( bd \) must be zero, and since \( \BbbZ \) is entire, it follows that either \( b \) or \( d \) or both must be zero. Furthermore, \( ad \) and \( bc \) must also be zero, hence \( b = d = 0 \). Then \( n = \abs{ac} \). Without loss of generality, suppose that both \( a \) and \( c \) are positive so that \( n = ac \).

    If \( n \) is \( 2 \) or \( 3 \), it is a prime number and is thus irreducible in \( \BbbZ \), hence also in \( \BbbZ[\sqrt{-5}] \).

    Now consider \( 1 \pm \sqrt{-5} \), whose (complex) absolute value is \( \sqrt 6 \). Suppose that it factors as \fullref{eq:ex:def:domain_divisibility/irreducible_not_prime/decomposition}. \Fullref{eq:ex:def:domain_divisibility/irreducible_not_prime/abs} implies that either \( b = 0 \) or \( d = 0 \), but both cannot be zero because the condition \( a^2 c^2 = 6 \) cannot be satisfied.
    \begin{itemize}
      \item If \( b = 0 \), then
      \begin{equation*}
        6 = a^2 (c^2 + 5 d^2).
      \end{equation*}

      We have \( c^2 + 5 d^2 \geq 6 \), so \( a^2 \) must be \( 1 \).

      \item If \( d = 0 \), then similarly
      \begin{equation*}
        6 = c^2 (a^2 + 5b^2),
      \end{equation*}
      from which we conclude that \( c^2 \) must be \( 1 \).
    \end{itemize}

    It follows that \( 1 + \sqrt{5} \) and \( 1 - \sqrt{5} \) are also irreducible.

    Therefore, we have the following ways of representing \( 6 \) as a product of irreducible factors:
    \begin{equation*}
      6 = 2 \cdot 3 = \parens[\Big]{ 1 + \sqrt{-5} } \cdot \parens[\Big]{ 1 - \sqrt{-5}}.
    \end{equation*}

    Furthermore, \( 2 \) and \( 3 \) are irreducible, \( 1 + \sqrt{5} \) doesn't divide neither \( 2 \) nor \( 3 \) but divides their product. Hence, both \( 2 \) and \( 3 \) are irreducible element that are not prime. The same holds for \( 1 + \sqrt{5} \) and \( 1 - \sqrt{5} \).

    \thmitem{ex:def:domain_divisibility/x2_plus_y2} The polynomial \( p(X, Y) \coloneqq a X^2 + b Y^2 \), where \( a \) and \( b \) are positive real numbers, is irreducible in \( \BbbR[X, Y] \).

    Indeed, fix some decomposition \( p(X, Y) = q(X, Y) \cdot r(X, Y) \). From \eqref{eq:thm:def:polynomial_degree/product} it follows that \( \deg q + \deg r = 2 \). If \( \deg q = 2 \), then \( \deg r = 0 \), and hence \( p(X, Y) \) and \( q(X, Y) \) differ by a scalar factor, i.e. it is invertible. Similarly, if \( \deg r = 2 \), then \( q \) is invertible.

    In order for \( p(X, Y) \) to be reducible, both \( q(X, Y) \) and \( r(X, Y) \) must be linear polynomials. Suppose that
    \begin{align*}
      q(X, Y) &= c X + d Y + e, \\
      r(X, Y) &= f X + g Y + h.
    \end{align*}

    Then
    \begin{equation*}
      q(X, Y) \cdot r(X, Y) = c f X^2 + c g X Y + c h X + d f X Y + d g Y^2 + e h Y + e f X + e g Y + e h.
    \end{equation*}

    In order for there to be no mixed monomials, we must have \( c g = - d f \). Furthermore, \( c \), \( d \), \( f \) and \( g \) are nonzero because otherwise either \( a = c f \) or \( b = d g \) would be zero. Thus,
    \begin{equation*}
      a = c f = - \frac {df} g \cdot f = -f^2 \frac d g.
    \end{equation*}

    Since \( a \) and \( f^2 \) are both positive, \( d / g \) must be negative, i.e. \( d \) and \( g \) must have different signs. But then \( b = d g \) would be negative, and this contradicts out initial assumption that \( b \) is positive.

    The obtained contradiction demonstrates that \( p(X, Y) = a X^2 + b Y^2 \) is irreducible over \( \BbbR \).

    \thmitem{ex:def:domain_divisibility/x2_plus_y2_plus_z2} The polynomial \( p(X, Y, Z) \coloneqq a X^2 + b Y^2 + c Z^2 \), where \( a \), \( b \) and \( c \) are nonzero scalars from an arbitrary \hyperref[def:field]{field} \( \BbbK \), is irreducible in \( \BbbK[X, Y, Z] \).

    As in \fullref{ex:def:domain_divisibility/x2_plus_y2}, suppose that \( p(X, Y, Z) \) is a product of the linear polynomials
    \begin{align*}
      q(X, Y, Z) &= d X + e Y + f Z + g, \\
      r(X, Y, Z) &= h X + i Y + j Z + k.
    \end{align*}

    Then \( q(X, Y, Z) \cdot r(X, Y, Z) \) is
    \begin{align*}
      &\phantom{{}+{}}
      d h X^2 + d i X Y + d j X Z + d k X
      + \\ &+
      e h X Y + e i Y^2 + e j Y Z + e k Y
      + \\ &+
      f h X Z + f i Y Z + f j Z^2 + f k Z
      + \\ &+
      g h X + g i Y + g j Z + g k.
    \end{align*}

    Since \( a = dh \), \( b = ei \) and \( c = fj \) are nonzero, it follows that the corresponding scalars are nonzero. Furthermore, we must have
    \begin{align*}
      (d i + e h) X Y &= 0, \\
      (d j + f h) X Z &= 0, \\
      (e j + f i) Y Z &= 0,
    \end{align*}
    that is,
    \begin{align*}
      d i &= - e h, \\
      d j &= - f h, \\
      e j &= - f i.
    \end{align*}

    We can divide the first two equalities to obtain \( i / j = e / f \), i.e. \( ej = fi \). But the third equality states that \( ej = -fi \). Hence, \( ej \) and \( fi \) can both only be zero. But we know that \( e \), \( f \), \( i \) and \( j \) are all nonzero, hence \( ej \) and \( fi \) must also be nonzero.

    The obtained contradiction shows that the polynomial \( p(X, Y, Z) = a X^2 + b Y^2 + c Z^2 \) is irreducible over any field.
  \end{thmenum}
\end{example}

\begin{remark}\label{rem:choice_of_associates}
  If \( x \) and \( y \) are \hyperref[def:domain_divisibility/associates]{associates}, we generally have no reason to prefer \( x \) to \( y \). This leads to a non-uniqueness in certain contexts, e.g. choosing a \hyperref[def:gcd]{greatest common divisor} or, more generally, a generator for a principal ideal. In such cases, we often prefer working with ideals.

  Fortunately, in the majority of cases, we have good candidates for uniqueness:
  \begin{itemize}
    \item In the domain \( \BbbZ \) of integers, there are two units, \( 1 \) and \( -1 \). It is convenient to choose the positive greatest common divisor.

    \item In a polynomial ring over the integers \( \BbbZ \), by \fullref{thm:def:polynomial_algebra/invertible}, the units are again \( 1 \) and \( -1 \), and we can choose the leading coefficient to be positive.

    \item If \( \BbbK \) is any \hyperref[def:field]{field}, any polynomial is associated with a unique \hyperref[def:monic_polynomial]{monic polynomial}.
  \end{itemize}
\end{remark}

\paragraph{Common divisors and multiples}

\begin{example}\label{ex:common_polynomial_divisors}\mcite{MathSE:polynomials_without_gcd}
  Consider two polynomials \( p(X) \) and \( q(X) \) over some domain that we wish to find the common roots of.

  The element \( u \) is a common root if and only if the polynomial \( (X - u) \) divides both \( p(X) \) and \( q(X) \). Thus, if \( r(X) \) is a common divisor of \( p(X) \) and \( q(X) \), every root of \( r(X) \) is a common root for \( p(X) \) and \( q(X) \).

  Let \( C \) be the set of all common divisors of \( p(X) \) and \( q(X) \). Every invertible element of the domain is itself a common divisor, so \( C \) is necessarily nonempty. Consider the \hyperref[thm:semiring_divisibility_order]{divisibility (pre)order} in \( C \). A \hyperref[def:extremal_points/greatest_and_least]{greatest element} with respect to divisibility must contain all common roots of \( p(X) \) and \( q(X) \), and thus it makes sense to search for greatest common divisors.

  \begin{thmenum}
    \thmitem{ex:common_polynomial_divisors/field} If \( p(X) \) and \( q(X) \) are polynomials over a \hyperref[def:field]{field} like \( \BbbR \), \fullref{alg:euclidean_algorithm} explicitly constructs a greatest common divisor.

    \thmitem{ex:common_polynomial_divisors/factorial} More generally, polynomials over \hyperref[def:factorial_domain]{factorial domains} always have a greatest common divisor, but the aforementioned algorithm may fail.

    \thmitem{ex:common_polynomial_divisors/distinct} As a simple concrete example, consider the polynomials \( X^5 \) and \( X^6 \) over any ring. Clearly \( X^5 \) is a common divisor and, furthermore, any polynomial dividing both \( X^5 \) and \( X^6 \) vacuously divides \( X^5 \).

    Thus, \( X^5 \) is a greatest common divisor of \( X^5 \) and \( X^6 \), but it is not unique - for any invertible ring element \( a \), the polynomial \( a X^5 \) is also a greatest common divisor.

    This general problem comes from the lack of antisymmetry in preorders --- see \fullref{ex:preorder_nonuniqueness}.

    \thmitem{ex:common_polynomial_divisors/incomparable} Consider the polynomial algebra \( \BbbZ[X^2, X^3] \) ordered by divisibility. Generally there may be distinct maximal elements on a preordered set. Based on our discussion in \fullref{ex:adjoining_polynomial}, we conclude that these polynomials have the form
    \begin{equation*}
      \sum_{k \neq 1} a_k X^k.
    \end{equation*}

    Because the ring features no monomial \( X \), the monomial \( X^2 \) does not divide \( X^3 \). This has some interesting consequences.

    Consider the polynomials \( X^5 \) and \( X^6 \) in this domain.

    \Fullref{thm:def:integral_domain/polynomial_divides} implies that \( p(X) \) divides \( X^6 \) if and only if \( p(X) = aX^s \) for some invertible \( a \) and \( s \leq 6 \). The quotient of \( X^6 \) by \( p(X) \) is \( a^{-1} X^{6-s} \). Furthermore, neither \( s \) nor \( 6 - s \) must be \( 1 \), thus \( s \) must be among \( 0 \), \( 2 \), \( 3 \), \( 4 \) and \( 6 \).

    Similarly, we can conclude that \( p(X) = aX^s \) divides \( X^5 \) if and only if \( s \) is among \( 0 \), \( 2 \), \( 3 \) and \( 5 \).

    Hence, \( p(X) \) is a common divisor of \( X^5 \) and \( X^6 \) if and only if \( s \) is among \( 0 \), \( 2 \) and \( 3 \).

    \begin{figure}[!ht]
      \centering
      \includegraphics[page=1]{output/ex__common_polynomial_divisors__incomparable}
      \caption{A fragment of the \hyperref[def:hasse_diagram]{Hasse diagram} of the divisibility relation in \( \BbbZ[X^2, X^3] \).}
      \label{fig:ex:common_polynomial_divisors/incomparable}
    \end{figure}

    Thus, up to a choice of invertible element \( a \), the common divisors of \( X^5 \) and \( X^6 \) are \( a \) itself, \( aX^2 \) and \( aX^3 \). But \( aX^2 \) and \( aX^3 \) are not comparable because neither divides the other in \( \BbbZ[X^2, X^3] \).

    Therefore, the monomials \( X^5 \) and \( X^6 \) have two monic maximal divisors but no greatest common divisor in \( \BbbZ[X^2, X^3] \).
  \end{thmenum}
\end{example}

\begin{definition}\label{def:gcd}
  We say that, in a \hyperref[def:integral_domain]{integral domain}, \( g \) is a \term[bg=най-голям общ делител (\cite[def. II.2]{ГеновИПр1991Алгебра}), ru=наибольший общий делитель (\cite[def. 3.5.3]{Винберг2014Алгебра})]{greatest common divisor} of \( x \) and \( y \) if it satisfies the following equivalent conditions:

  \begin{thmenum}
    \thmitem{def:gcd/direct}\mcite[2]{Knapp2016BasicAlgebra} \( g \) divides both \( x \) and \( y \) and, whenever \( d \) is also a common divisor, \( d \) divides \( g \).

    \thmitem{def:gcd/infimum} \( g \) is an \hyperref[def:extremal_points/supremum_and_infimum]{infimum} of \( x \) and \( y \) with respect to the \hyperref[thm:semiring_divisibility_order]{divisibility order}.

    \thmitem{def:gcd/ideals}\mcite[def. V.2.2]{Aluffi2009Algebra} The ideal \( \braket{ g } \) is the smallest principal ideal that contains the join \( \braket{ x } + \braket{ y } = \braket{ x, y } \) of \( \braket{ x } \) and \( \braket{ y } \).
  \end{thmenum}
\end{definition}
\begin{comments}
  \item \Fullref{def:gcd/ideals} can be simplified in \hyperref[def:bezout_domain]{Bezout domains}, where \( \braket{ x, y } \) is principal and must thus coincide with \( \braket{ g } \). Note that the case of least common multiples in \fullref{def:lcm/ideals} always satisfies the corresponding analog to this stronger condition.

  \item There may be distinct greatest common divisors that divide each other --- see \fullref{ex:common_polynomial_divisors/distinct} for a simple example or \fullref{ex:preorder_nonuniqueness} for a discussion of this problem for general preordered sets. For the \hyperref[def:natural_numbers]{natural numbers}, where the GCD of \( x \) and \( y \) is unique, we denote it via \( \gcd(x, y) \).

  \item Some authors like \incite[111]{Lang2002Algebra}, \incite[144]{Jacobson1985AlgebraPart1}, \incite[304]{Rotman2010Algebra} and \incite[2]{Knapp2016BasicAlgebra} leave greatest common divisors of \( 0 \) and \( 0 \) undefined, while others like \incite[def. V.2.2]{Aluffi2009Algebra} and \incite[119]{Винберг2014Алгебра} and \incite[143]{ГеновИПр1991Алгебра} do not handle them as a special case.
\end{comments}
\begin{defproof}
  \ImplicationSubProof{def:gcd/direct}{def:gcd/infimum} Suppose that \( g \) is a common divisor of \( x \) and \( y \) and any other common divisor divides \( g \).

  Then \( g \) is a lower bound of \( x \) and \( y \) with respect to divisibility. Furthermore, if \( d \) is a common divisor, then \( d \) divides \( g \), and thus is smaller than \( g \) with respect to divisibility.

  Therefore, \( g \) is a greatest lower bound of \( x \) and \( y \).

  \ImplicationSubProof{def:gcd/infimum}{def:gcd/ideals} Suppose that \( g \) is an infimum of \( x \) and \( y \) with respect to divisibility.

  Since \( g \) divides both \( x \) and \( y \), any linear combination \( \alpha x + \beta y \) from \( \braket{ x, y } \) is a multiple of \( g \):
  \begin{equation*}
    \alpha x + \beta y
    =
    g \parens*{ \alpha \frac x g + \beta \frac y g }.
  \end{equation*}

  Hence, \( \braket{ x, y } \subseteq \braket{ g } \).

  Now let \( d \) be such that
  \begin{equation*}
    \braket{ x, y } \subseteq \braket{ d } \subseteq \braket{ g }.
  \end{equation*}

  Since \( g \) a greatest lower bound of \( x \) and \( y \) and since \( d \) divides \( g \), it follows that \( g \) divides \( d \) and thus \( \braket{ d } = \braket{ g } \).

  Therefore, \( \braket{ g } \) is the smallest principal ideal containing \( \braket{ x, y } \).

  \ImplicationSubProof{def:gcd/ideals}{def:gcd/direct} Suppose that \( \braket{ g } \) is the smallest principal ideal containing \( \braket{ x, y } \).

  Then \( \braket{ x } \subseteq \braket{ g } \) and thus \( g \mid x \), and similarly \( g \mid y \). It is thus a common divisor for \( x \) and \( y \).

  Furthermore, if \( d \) is also a common divisor, then, by the proof of the previous implication, \( \braket{ x, y } \subseteq \braket{ d } \). The minimality of \( \braket{ g } \) then ensures that \( \braket{ g } \subseteq \braket{ d } \) and, by \fullref{thm:def:semiring_ideal/division}, \( d \mid g \).
\end{defproof}

\begin{proposition}\label{thm:def:gcd}
  \hyperref[def:gcd]{Greatest common divisors} have the following basic properties:
  \begin{thmenum}
    \thmitem{thm:def:gcd/divides} \( x \) divides \( y \) if and only if \( x \) is a GCD of \( x \) and \( y \).

    \thmitem{thm:def:gcd/associates} Let \( g \) be a \hyperref[def:gcd]{GCD} of \( x \) and \( y \). Then an element \( g' \) is also a greatest common divisor if and only if \( g \) and \( g' \) are \hyperref[def:domain_divisibility/associates]{associates}.

    \thmitem{thm:def:gcd/bezouts_identity_converse} If \( d \) is a common divisor for \( x \) and \( y \), and if there exist elements \( a \) and \( b \) such that \( ax + by = d \), then \( d \) is a \hyperref[def:gcd]{GCD} of \( x \) and \( y \).

    The premise here always holds in \hyperref[def:bezout_domain]{Bezout domains}.
  \end{thmenum}
\end{proposition}
\begin{proof}
  \SubProofOf{thm:def:gcd/divides} Trivial.

  \SubProofOf{thm:def:gcd/associates} The element \( g' \) satisfies \fullref{def:gcd/ideals} if and only if it has the same principal ideal as \( g \).

  \SubProofOf{thm:def:gcd/bezouts_identity_converse} Let \( e \) be a common divisor of \( x \) and \( y \). Then \( e \) divides both \( ax \) and \( by \), hence also \( ax + by = d \). Therefore, \( e \) divides \( d \), implying that \( d \) is a greatest common divisor of \( x \) and \( y \).
\end{proof}

\begin{definition}\label{def:lcm}
  \hyperref[thm:lattice_duality]{Dually} to \fullref{def:gcd}, we say that, in a \hyperref[def:integral_domain]{integral domain}, \( l \) is a \term[bg=най-малко общо кратно (\cite[381]{ГеновИПр1991Алгебра}), ru=наименьшее общее кратное (\cite[exer. 3.6.3]{Винберг2014Алгебра})]{least common multiple} (LCM) of \( x \) and \( y \) if it satisfies the following equivalent conditions:

  \begin{thmenum}
    \thmitem{def:lcm/direct}\mcite[32]{Knapp2016BasicAlgebra} Both \( x \) and \( y \) divide \( l \) and, whenever \( m \) is also a common multiple, \( l \) divides \( m \).

    \thmitem{def:lcm/supremum} \( l \) is an \hyperref[def:extremal_points/supremum_and_infimum]{supremum} of \( x \) and \( y \) with respect to the \hyperref[thm:semiring_divisibility_order]{divisibility order}.

    \thmitem{def:lcm/ideals} The ideal \( \braket{ l } \) coincides with the meet \( \braket{ x } \cap \braket{ y } \) of \( \braket{ x } \) and \( \braket{ y } \).
  \end{thmenum}
\end{definition}
\begin{comments}
  \item Note how \fullref{def:lcm/ideals} differs from \fullref{def:gcd/ideals} --- the existence of least common multiples in general domains is indeed stronger, as we shall see in \fullref{rem:gcd_but_no_lcm} and \fullref{thm:gcd_and_lcm_existence}.

  \item A result similar to \fullref{thm:def:gcd/associates} holds -- every pair of least common multiples are associates.

  \item As in the case of GCDs, for the \hyperref[def:natural_numbers]{natural numbers}, where the LCM of \( x \) and \( y \) is unique, we denote it via \( \lcm(x, y) \).
\end{comments}
\begin{proof}
  \ImplicationSubProof{def:lcm/direct}{def:lcm/supremum} Same as in the case of GCDs in \fullref{def:gcd}.

  \ImplicationSubProof{def:lcm/supremum}{def:lcm/ideals} We can show that \( \braket{ l } \) is the largest principal ideal contained in \( \braket{ x } \cap \braket{ y } \) similar to the case of GCDs in \fullref{def:gcd}.

  Now let \( z = ax = by \) be a member of the intersection. Then it is a common multiple of \( x \) and \( y \), hence \( l \) must divide \( z \).

  Generalizing on \( z \), we conclude that \( \braket{ x } \cap \braket{ y } \subseteq \braket{ l } \). Since we already have the converse inclusion, it follows that the two ideals are equal.

  \ImplicationSubProof{def:lcm/ideals}{def:lcm/direct} Same as in the case of GCDs in \fullref{def:gcd}.
\end{proof}

\begin{proposition}\label{thm:gcd_and_lcm}
  If \( g \) is a \hyperref[def:gcd]{greatest common divisor} of \( x \) and \( y \) and \( l \) is a \hyperref[def:lcm]{least common multiple}, then \( xy \) and \( gl \) are \hyperref[def:domain_divisibility/associates]{associates}.
\end{proposition}

\begin{remark}\label{rem:gcd_but_no_lcm}
  From \fullref{thm:gcd_and_lcm} it may seem that we are always able to recover a GCD from a LCM, and it is indeed so, but the existence of a LCM does not necessarily follow from the existence of a GCD. The precise existence conditions are described in \fullref{thm:gcd_and_lcm_existence}.

  \incite[thm. 4]{Khurana2003GCD} provides counterexamples --- in \( \BbbZ[\sqrt{-3}] \), the elements \( x = 2 \) and \( y = 1 + \sqrt{-3} \) have a GCD, but not a LCM.

  Fortunately, if \hi{every} pair of elements has a GCD, then every pair also has a LCM.
\end{remark}

\begin{lemma}\label{thm:gcd_of_multiple}\mcite[lemma 1]{Khurana2003GCD}
  Let \( x \), \( y \) and \( r \) be nonzero elements of some integral domain. If \( g \) is a \hyperref[def:gcd]{greatest common divisor} for \( rx \) and \( ry \), then \( r \) divides \( g \) and their \hyperref[def:domain_quotient]{quotient} is a greatest common divisor of \( x \) and \( y \).
\end{lemma}
\begin{proof}
  Since \( r \) is a common divisor of \( rx \) and \( ry \), it follows that \( r \) divides \( g \). Thus, \( g = r \cdot g / r \) divides both \( rx \) and \( ry \). We can cancel \( r \) to obtain that \( g / r \) divides both \( x \) and \( y \).

  Furthermore, if \( d \) is also a common divisor for \( x \) and \( y \), then \( rd \) divides both \( rx \) and \( ry \) and hence also \( g \). Again, cancelling \( r \), we obtain that \( d \) divides \( g / r \).
\end{proof}

\begin{proposition}\label{thm:gcd_and_lcm_existence}\mcite[thm. 2]{Khurana2003GCD}
  For an arbitrary \hyperref[def:integral_domain]{integral domain}, two elements \( x \) and \( y \) have a \hyperref[def:lcm]{least common multiple} if and only if, for every nonzero \( r \), the elements \( rx \) and \( ry \) have a \hyperref[def:gcd]{greatest common divisor}.
\end{proposition}
\begin{proof}
  \SufficiencySubProof

  \SubProof*{Proof that \( x \) and \( y \) have a GCD} Suppose that \( l \) is a LCM of \( x \) and \( y \). Since \( xy \) is also a common multiple, \( l \) divides \( xy \). Let \( g \) be the \hyperref[def:domain_quotient]{quotient} of \( xy \) by \( l \).

  Then
  \begin{equation*}
    xy = gl = g \parens*{ \frac l y \cdot y}
  \end{equation*}
  and we can cancel \( y \) to obtain
  \begin{equation*}
    x = g \frac l y.
  \end{equation*}

  Thus, \( g \) is a divisor of \( x \), and we can similarly conclude that it is a divisor of \( y \). We will show that it is a greatest common divisor.

  For any common divisor \( d \), we have
  \begin{equation*}
    d \cdot \frac x d \cdot y
    =
    xy
    =
    d \cdot \frac {xy} d.
  \end{equation*}

  We can cancel \( d \) to obtain
  \begin{equation*}
    \frac x d \cdot y = \frac {xy} d.
  \end{equation*}

  Thus, \( y \) divides \( xy / d \), and we can similarly conclude that \( x \) divides \( xy / d \). Then it is a common multiple, hence \( l \) must divide it. Thus,
  \begin{equation*}
    gl = xy = \frac {xy} d \cdot d = \frac {xy / d} l \cdot l \cdot d.
  \end{equation*}

  We can cancel \( l \) to obtain
  \begin{equation*}
    g = \frac {xy / d} l \cdot d.
  \end{equation*}

  We have obtained that \( d \) divides \( g \), which makes \( g \) a greatest common divisor of \( x \) and \( y \).

  \SubProof*{Proof that \( rx \) and \( ry \) have a GCD} Since \( l \) is a LCM of \( x \) and \( y \), it is natural to suppose that \( rl \) will be an LCM of \( rx \) and \( ry \).

  It is clearly a common multiple. Furthermore, if \( m \) is also a common multiple, \( m / r \) is a common multiple for \( x \) and \( y \) and thus \( l \) divides \( m / r \). Hence, \( rl \) divides \( m / r \), making \( rl \) a least common multiple of \( rx \) and \( ry \).

  Therefore, by what we have already shown, the following is a GCD for \( rx \) and \( ry \):
  \begin{equation*}
    \frac {r^2 xy} {rl} = \frac {rxy} l.
  \end{equation*}

  \NecessitySubProof Suppose that \( rx \) and \( ry \) have a GCD for every nonzero \( r \) and let \( g \) be a GCD for \( x \) and \( y \) themselves.

  Define
  \begin{equation*}
    l \coloneqq g \cdot \frac x g \cdot \frac y g = x \cdot \frac y g = \frac x g \cdot y.
  \end{equation*}

  Thus, \( l \) is a common multiple for \( x \) and for \( y \). Let \( m \) also be a common multiple of \( x \) and \( y \). We will show that \( l \) divides \( m \).

  Let \( g' \) a greatest common divisor for \( mx \) and \( my \). The product \( xy \) is also a common divisor for \( mx \) and \( my \), hence \( xy \) divides \( g' \).

  \Fullref{thm:gcd_of_multiple} implies that \( g' / m \) is a greatest common divisor for \( x \) and \( y \). \Fullref{thm:def:gcd/associates} implies that \( g \) and \( g' / m \) are associates. Then \( gm \) and \( g' \) are also associates, from which it follows that \( gm \) divides \( xy = gl \). Therefore, \( m \) divides \( l \).

  Generalizing on \( m \), we conclude that \( l \) is a least common multiple.
\end{proof}

\paragraph{Greatest common divisor domains}

\begin{definition}\label{def:gcd_domain}\mcite[32]{Kaplansky1974Rings}
  We say that an \hyperref[def:integral_domain]{integral domain} is a \term{greatest common divisor domain} if any two elements have a \hyperref[def:gcd]{greatest common divisor}.

  \Fullref{thm:gcd_and_lcm_existence} shows that it is equivalent for the \hyperref[def:lcm]{least common multiple} to exist.
\end{definition}

\begin{proposition}\label{thm:def:gcd_domain}
  \hyperref[def:gcd_domain]{Greatest common divisor domains} have the following basic properties:
  \begin{thmenum}
    \thmitem{thm:def:gcd_domain/polynomial_ring} If the \hyperref[def:polynomial_algebra]{polynomial ring} \( R[X] \) over a commutative ring \( R \) is a GCD domain, then \( R \) also is.

    The converse to this is true, but it is more difficult to prove. See \fullref{thm:polynomial_ring_over_gcd_domain}.

    \thmitem{thm:def:gcd_domain/irreducible_is_prime} Every \hyperref[def:domain_divisibility/irreducible]{irreducible element} in a GCD domain is \hyperref[def:domain_divisibility/prime]{prime}.
  \end{thmenum}
\end{proposition}
\begin{proof}
  \SubProofOf{thm:def:gcd_domain/polynomial_ring} By \fullref{thm:def:integral_domain/subring}, \( R \) is an integral domain.

  Let \( a(X) \) and \( b(X) \) be embeddings in \( R[X] \) of the elements \( a \) and \( b \) from \( R \) and let \( g(X) \) be their greatest common divisor.

  \begin{itemize}
    \item If \( a(X) \) and \( b(X) \) are zero, so is \( g(X) \).
    \item If both are nonzero, \fullref{thm:def:polynomial_degree/product} implies that \( g(X) \) has degree one.
  \end{itemize}

  In both cases, \( g(X) \) corresponds to a constant in \( R \).

  Generalizing, we conclude that \( R \) is a GCD domain.

  \SubProofOf{thm:def:gcd_domain/irreducible_is_prime} Suppose that \( x \) is an irreducible element and that \( x \) divides \( yz \).

  Let \( g \) be a GCD of \( xz \) and \( yz \). Both \( x \) and \( z \) are also common divisors, hence they divide \( g \). Thus,
  \begin{equation*}
    g = z \cdot \frac g z.
  \end{equation*}

  Then
  \begin{equation*}
    xz = g \cdot \frac {xz} g = z \cdot \frac g z \cdot \frac {xz} g.
  \end{equation*}

  We can cancel \( z \) to obtain
  \begin{equation*}
    x = \frac g z \cdot \frac {xz} g.
  \end{equation*}

  Because \( x \) is irreducible, (at least) one of the multiplicands must be invertible.
  \begin{itemize}
    \item If \( g / z \) is invertible, by \fullref{thm:domain_quotient_inverse}, \( g \) and \( z \) are associated. Since \( x \) divides \( g \), it follows that \( x \) divides \( z \).

    \item If \( xz / g \) is invertible, again by \fullref{thm:domain_quotient_inverse}, \( xz \) divides \( g \), which in turn divides \( yz \). Then
    \begin{equation*}
      yz = xz \cdot \frac {yz} {xz}.
    \end{equation*}

    After cancelling \( z \), we obtain that \( x \) divides \( y \).
  \end{itemize}
\end{proof}

\begin{example}\label{ex:def:gcd_domain}
  We list examples of \hyperref[def:gcd_domain]{GCD domains}:
  \begin{thmenum}
    \thmitem{ex:def:gcd_domain/euclidean} \Fullref{alg:euclidean_algorithm} allows computing GCDs in arbitrary \hyperref[def:euclidean_domain]{Euclidean domains}.

    \thmitem{ex:def:gcd_domain/countable_indeterminates} Consider the polynomial algebra \( D[X_1, X_2, \ldots] \) in countably many indeterminates.

    Every polynomial only has finitely many indeterminates, and so two polynomials belong to a subdomain with finitely many indeterminates, along with all their divisors. \Fullref{thm:polynomial_ring_over_gcd_domain} then implies that the polynomials have a greatest common divisor in the subdomain and hence also in \( D[X_1, X_2, \ldots] \).
  \end{thmenum}
\end{example}

\paragraph{Bezout domains}

\begin{definition}\label{def:bezout_domain}
  We say that an \hyperref[def:integral_domain]{integral domain} is a \term{Bezout domain} if any of the following equivalent conditions hold:
  \begin{thmenum}
    \thmitem{def:bezout_domain/identity} For any two elements \( x \) and \( y \), there exists a \hyperref[def:gcd]{greatest common divisor} \( g \) and elements \( a \) and \( b \) such that
    \begin{equation}\label{eq:def:bezout_domain/identity}
      g = ax + by.
    \end{equation}

    We call \eqref{eq:def:bezout_domain/identity} \term[en=Bezout's identity (\cite[3]{Knapp2016BasicAlgebra})]{Bezout's identity}.

    \thmitem{def:bezout_domain/ideals}\mcite[38]{Kaplansky1974Rings} Every finitely-generated ideal is \hyperref[def:semiring_ideal/principal]{principal}.
  \end{thmenum}
\end{definition}
\begin{comments}
  \item Every though the condition is not explicitly requires for \fullref{def:bezout_domain/ideals} to hold, Bezout domains are, by \fullref{def:bezout_domain/identity}, \hyperref[def:gcd_domain]{GCD domains}.
  \item \Fullref{alg:extended_euclidean_algorithm} gives us an explicit construction for \( a \) and \( b \) in the case of \hyperref[def:euclidean_domain]{Euclidean domains}.
\end{comments}
\begin{defproof}
  \ImplicationSubProof{def:bezout_domain/identity}{def:bezout_domain/ideals} Suppose that \eqref{eq:def:bezout_domain/identity} holds.

  Let \( I = \braket{ x_1, \ldots, x_n } \) be a (finitely-generated) ideal. We will use induction on \( n \) to show that \( I \) is principal.

  The base case \( n = 1 \) is trivial. Suppose that \( \braket{ x_1, \ldots, x_{n-1} } \) is principal with generator \( d \). Then \eqref{eq:def:bezout_domain/identity} gives us a GCD \( g \) of \( d \) and \( x_n \), along with coefficients \( a \) and \( b \) such that
  \begin{equation*}
    g = ad + bx_n.
  \end{equation*}

  Clearly \( g \) is a member of \( I \) because
  \begin{equation*}
    I = \braket{ x_1, \ldots, x_{n-1} } + \braket{ x_n }.
  \end{equation*}

  To see that \( I = \braket{ g } \), consider some member \( y \) of \( I \). It is, by definition, a linear combination
  \begin{equation*}
    y = \alpha d + \beta x_n.
  \end{equation*}

  Then it is a common divisor of \( d \) and \( x_n \), hence it also divides \( g \). Hence, every member of \( I \) belongs to \( \braket{ g } \).

  We conclude that \( I = \braket{ g } \).

  \ImplicationSubProof{def:bezout_domain/ideals}{def:bezout_domain/identity} Suppose that all finitely-generated ideals are principal. Then the join \( \braket{ x, y } =\braket{ x } + \braket{ y } \) of the principal ideals of any two elements \( x \) and \( y \) is principal. Let \( g \) be a generator of \( \braket{ x, y } \). Then, by definition of ideal, there exist some coefficients from the domain such that
  \begin{equation*}
    g = ax + by.
  \end{equation*}

  Furthermore, \( g \) is a greatest common divisor of \( x \) and \( y \) because \( \braket{ g } \) is vacuously the smallest principal ideal containing \( \braket{ x, y } \).

  Then \eqref{eq:def:bezout_domain/identity} holds.
\end{defproof}

\begin{proposition}\label{thm:def:bezout_domain}
  \hyperref[def:bezout_domain]{Bezout domains} have the following basic properties:
  \begin{thmenum}
    \thmitem{thm:def:bezout_domain/gcd} Equality holds in \fullref{def:gcd/ideals}: If \( g \) is a \hyperref[def:gcd]{greatest common divisor} of \( x \) and \( y \), then
    \begin{equation*}
      \braket{ g } = \braket{ x } + \braket{ y } = \braket{ x, y }.
    \end{equation*}

    \thmitem{thm:def:bezout_domain/ideal_gcd_closed} Ideals are closed under GCD: If \( g \) is a GCD of some elements in an ideal \( I \), then \( g \) itself belongs to \( I \).
  \end{thmenum}
\end{proposition}
\begin{proof}
  \SubProofOf{thm:def:bezout_domain/gcd} Follows from \eqref{eq:def:bezout_domain/identity}.
  \SubProofOf{thm:def:bezout_domain/ideal_gcd_closed} Follows from \fullref{thm:def:bezout_domain/gcd}.
\end{proof}

\begin{example}\label{ex:def:bezout_domain}
  We list examples of \hyperref[def:bezout_domain]{Bezout domains}:
  \begin{thmenum}
    \thmitem{ex:def:bezout_domain/integers} By \fullref{thm:bezout_lemma}, the ring \( \BbbZ \) of integers is a Bezout domain.

    \thmitem{ex:def:bezout_domain/noetherian} More generally, every \hyperref[def:noetherian_semiring]{noetherian} domain is a Bezout domain.

    \thmitem{ex:def:bezout_domain/integer_polynomials} The algebra \( \BbbZ[X] \) of univariate integer polynomials is not a Bezout domain --- the ideal \( \braket{ 2, X } \) is finitely generated, but not principal.

    This is a simple example of a GCD domain which is not Bezout.

    \thmitem{ex:def:bezout_domain/multivariate_polynomials} For any Bezout domain \( D \), the algebra \( D[X, Y] \) of bivariate polynomials is not a Bezout domain --- the ideal \( \braket{ X, Y } \) is finitely generated, but not principal.
  \end{thmenum}
\end{example}

\paragraph{Coprime elements}

\begin{definition}\label{def:coprime_elements}\mimprovised
  We say that two nonzero elements of a \hyperref[def:bezout_domain]{Bezout domain} are \term[ru=взаимно простые (идеалы) (\cite[120]{Винберг2014Алгебра}), en=relatively prime (\cite[113]{Lang2002Algebra})]{coprime} if any the following equivalent conditions hold:
  \begin{thmenum}
    \thmitem{def:coprime_elements/divisors} Every common divisor is invertible.
    \thmitem{def:coprime_elements/greatest} Every GCD is invertible.
    \thmitem{def:coprime_elements/ideals} Their principal ideals are \hyperref[def:semiring_ideal/coprime]{coprime}.
  \end{thmenum}
\end{definition}
\begin{defproof}
  \ImplicationSubProof{def:coprime_elements/divisors}{def:coprime_elements/greatest} Special case.

  \ImplicationSubProof{def:coprime_elements/greatest}{def:coprime_elements/ideals} Let \( g \) be a GCD of \( x \) and \( y \) and suppose that it is invertible.

  We have \( \braket{ x } + \braket{ y } = \braket{ x, y } = \braket{ g } \). Since \( g \) is invertible, \( \braket{ g } \) is the entire domain. Hence, the ideals \( \braket{ x } \) and \( \braket{ y } \) are coprime.

  \ImplicationSubProof{def:coprime_elements/ideals}{def:coprime_elements/divisors} Suppose that \( \braket{ x } \) and \( \braket{ y } \) are coprime ideals. Let \( d \) be a common divisor of \( x \) and \( y \). Then
  \begin{equation*}
    \braket{ x, y } \subseteq \braket{ d }.
  \end{equation*}

  But \( \braket{ x, y } \) is the entire domain, hence \( \braket{ d } \) also is. It is thus invertible as a consequence of \fullref{thm:def:semiring_ideal/ideal_containing_unit}.
\end{defproof}

\begin{definition}\label{def:lowest_terms}\mimprovised
  Let \( \BbbK \) be the \hyperref[thm:field_of_fractions]{field of fractions} of the Bezout domain \( D \). We say that the concrete representative \( a / b \) of its class in \( \BbbK \) is \term{in lowest terms} if \( a \) and \( b \) are \hyperref[def:coprime_elements]{coprime} elements of \( D \).
\end{definition}
\begin{comments}
  \item This definition is based on one given by \incite[119]{Rotman2010Algebra} for rational algebraic functions.
\end{comments}

\begin{proposition}\label{thm:def:coprime_elements}
  \hyperref[def:coprime_elements]{Coprime elements} have the following basic properties:
  \begin{thmenum}
    \thmitem{thm:def:coprime_elements/divisors} If \( x \) and \( y \) are coprime, then \( x \) is coprime to any divisor of \( y \).

    \thmitem{thm:def:coprime_elements/gcd_quotients} If \( g \) is a GCD of \( x \) and \( y \), then \( x / g \) is coprime to \( y / g \).

    \thmitem{thm:def:coprime_elements/irreducible} Two \hyperref[def:domain_divisibility/irreducible]{irreducible} elements in a Bezout domain are either \hyperref[def:domain_divisibility/associates]{associated} or coprime.

    \thmitem{thm:def:coprime_elements/lowest_terms} Let \( \BbbK \) be the \hyperref[thm:field_of_fractions]{field of fractions} of the Bezout domain \( D \). For any element of \( \BbbK \), there exists an equal one \hyperref[def:lowest_terms]{in lowest terms}.
  \end{thmenum}
\end{proposition}
\begin{proof}
  \SubProofOf{thm:def:coprime_elements/divisors} Suppose that \( x \) and \( y \) are coprime and let \( z \) be a divisor of \( y \).

  If \( d \) is a common divisor of \( x \) and \( z \), then \( d \) is also a common divisor of \( x \) and \( y \), and so it follows that \( d \) is invertible. So, all common divisors of \( x \) and \( z \) are invertible, and hence, they are coprime.

  Generalizing on \( z \), we conclude that \( x \) is coprime to any divisor of \( y \).

  \SubProofOf{thm:def:coprime_elements/gcd_quotients} Let \( g \) be a GCD of \( x \) and \( y \).

  Let \( d \) be a common divisor of \( x / g \) and \( y / g \). Then
  \begin{equation*}
    x = g \cdot \frac x g = g \cdot d \cdot \frac {x / g} d
  \end{equation*}
  and
  \begin{equation*}
    y = g \cdot \frac y g = g \cdot d \cdot \frac {y / g} d.
  \end{equation*}

  Hence, \( gd \) is a common divisor of \( x \) and \( y \), and thus \( gd \) must divide \( g \). We have
  \begin{equation*}
    g = gd \cdot \frac g {gd}.
  \end{equation*}

  By cancelling \( g \), we obtain
  \begin{equation*}
    1 = d \cdot \frac g {gd}.
  \end{equation*}

  Therefore, \( d \) is invertible.

  Generalizing on \( d \), we conclude that \( x / g \) and \( y / g \) are coprime because all their common divisors are invertible.

  \SubProofOf{thm:def:coprime_elements/irreducible} Suppose that \( x \) and \( y \) are irreducible. Let \( g \) be a GCD of \( x \) and \( y \). Then there exist elements \( x' \) and \( y' \) such that \( x = x'g \) and \( y = y'g \).

  \begin{itemize}
    \item If \( g \) is invertible, then \( x \) and \( y \) satisfy the definition of coprimality.
    \item Otherwise, \( x' \) and \( y' \) are both invertible, and thus \( x \), \( g \) and \( y \) are associates.
  \end{itemize}

  \SubProofOf{thm:def:coprime_elements/lowest_terms} Let \( a \) and \( b \) be elements of \( D \). We want to find coprime elements \( c \) and \( d \) such that \( a / b = c / d \). By definition of localization, this holds if, for some \( u \), we have \( adu = bcu \). Since we are working in an integral domain and multiplication is cancellative, the condition reduces to \( ad = bc \).

  Let \( g \) be a GCD of \( a \) and \( b \). \Fullref{thm:def:coprime_elements/gcd_quotients} implies that \( a / g \) and \( b / g \) are coprime. Furthermore,
  \begin{equation*}
    a \cdot \frac b g = b \cdot \frac a g,
  \end{equation*}
  hence \( c = a / g \) and \( d = b / g \) are the desired coprime elements.
\end{proof}

\paragraph{Irreducible factorizations}

\begin{definition}\label{def:irreducible_factorization}\mcite[def. 1.8]{Aluffi2009Algebra}
  An \term{irreducible factorization} or simply \term{factorization} of a nonzero element \( x \) in an arbitrary \hyperref[def:integral_domain]{integral domain} is a finite sequence \( p_1, \ldots, p_n \) of \hyperref[def:domain_divisibility/irreducible]{irreducible elements} such that, for some \hyperref[def:divisibility/invertible]{invertible element} \( u \),
  \begin{equation*}
    x = u p_1 \cdots p_n.
  \end{equation*}

  The invertible element \( u \) is uniquely determined by the irreducible factors\fnote{If \( x \) is itself invertible, then it is its own factorization.}.

  \begin{thmenum}
    \thmitem{def:irreducible_factorization/equivalent} We say that two factorizations
    \begin{equation*}
      x = u p_1 \cdots p_n = v q_1 \cdots q_m
    \end{equation*}
    are equivalent if \( n = m \) and if there exists a \hyperref[def:symmetric_group]{permutation} \( \pi \in S_n \) such that \( p_k \) and \( q_{\pi(k)} \) are \hyperref[def:domain_divisibility/associates]{associated} for every \( k = 1, \ldots, n \).

    \thmitem{def:irreducible_factorization/unique} Finally, if any two factorizations of \( x \) are equivalent, we say that \( x \) \term{factors uniquely} into a product of irreducible factors.
  \end{thmenum}
\end{definition}

\begin{example}\label{ex:def:irreducible_factorization}
  We list some examples of \hyperref[def:irreducible_factorization]{irreducible factorization}:
  \begin{thmenum}
    \thmitem{ex:def:irreducible_factorization/integers} By \fullref{thm:fundamental_theorem_of_arithmetic}, every integer has a unique factorization.

    \thmitem{ex:def:irreducible_factorization/nonunique} Consider the ring \( \BbbZ[\sqrt{-5}] \) from \fullref{ex:def:domain_divisibility/irreducible_not_prime}. We have obtained there two distinct irreducible factorizations:
    \begin{equation*}
      6 = 2 \cdot 3 = \parens[\Big]{ 1 + \sqrt{-5} } \cdot \parens[\Big]{ 1 - \sqrt{-5}}
    \end{equation*}
  \end{thmenum}
\end{example}

\begin{proposition}\label{thm:def:irreducible_factorization}
  \hyperref[def:irreducible_factorization]{Irreducible factorizations} in integral domains have the following basic properties:
  \begin{thmenum}
    \thmitem{thm:def:irreducible_factorization/existence} If every ascending sequence of \hi{principal} ideals \hyperref[def:stabilizing_sequence]{stabilizes}, then every element has at least one irreducible \hi{factorization}.

    \thmitem{thm:def:irreducible_factorization/uniqueness} If every \hyperref[def:domain_divisibility/irreducible]{irreducible element} is \hyperref[def:domain_divisibility/prime]{prime}, then all factorizations of an element are \hyperref[def:irreducible_factorization/equivalent]{equivalent}\fnote{But there may be elements with no factorization.}.

    \thmitem{thm:def:irreducible_factorization/polynomial_ring} For any domain \( D \), \( x = u p_1 \cdots p_n \) is an irreducible factorization in \( D \) if and only if it is an irreducible factorization of \( x \) in \( D[X] \).
  \end{thmenum}
\end{proposition}
\begin{proof}
  \SubProofOf{thm:def:irreducible_factorization/existence} Suppose that every ascending sequence of principal ideals stabilizes. We will show that every element has an irreducible factorization, but first we will need an auxiliary result.

  \SubProof*{Every non-invertible element has at least one irreducible \hi{factor}} Fix some element \( x \). Let \( x_1 \coloneqq x \). Via recursion on \( k = 1, 2, \ldots \), define a sequence as follows:
  \begin{itemize}
    \item If all divisors of \( x_k \) are invertible, then \( x_k \) is irreducible. Let \( x_{k+1} \coloneqq x_k \).
    \item Otherwise, let \( d \) be a non-invertible divisor of \( x_k \), define \( x_{k+1} \) to be the \hyperref[def:domain_quotient]{quotient} \( x_k / d \).
  \end{itemize}

  The corresponding ascending sequence of principal ideals
  \begin{equation*}
    \braket{ x_1 } \subseteq \braket{ x_2 } \subseteq \braket{ x_3 } \subseteq \cdots
  \end{equation*}
  must stabilize at some index \( n \). Then, by construction, \( x_n \) must be irreducible.

  \SubProof*{Every element has at least one irreducible \hi{factorization}} Fix again some element \( x \) and define \( x_1 \coloneqq x \). Via recursion on \( k = 1, 2, \ldots \), define the following sequence:
  \begin{itemize}
    \item If \( x_k \) is either invertible or irreducible, let \( x_{k+1} \coloneqq x_k \).
    \item Otherwise, let \( p \) be an irreducible divisor of \( x_k \) and define \( x_{k+1} \) to be \( x_k / p \).
  \end{itemize}

  Similarly, the corresponding sequence of principal ideals must stabilize at some index \( n \).

  Define \( p_k \coloneqq x_{k+1} / x_k \) for every \( k \). If \( n > 1 \), then the elements \( p_1, \ldots, p_{n-1} \) must be irreducible. By construction, \( x_n \) is either invertible or is itself irreducible.
  \begin{itemize}
    \item If \( x_n \) is invertible, let \( u \coloneqq x_n \). Then \( x = u p_1 \cdots p_{n-1} \) is an irreducible factorization of \( x \).
    \item If \( x_n = p_n \) is irreducible, then \( x = p_1 \cdots p_n \) is an irreducible factorization of \( x \).
  \end{itemize}

  \SubProofOf{thm:def:irreducible_factorization/uniqueness} Suppose that every irreducible element is prime.

  Fix an element \( x \). If it has no irreducible factorization, then all its irreducible factorizations are vacuously equivalent. Otherwise, suppose that \( x \) has a factorization of length \( n \). We will prove by induction on \( n \) that any other factorization is equivalent.

  If \( n = 0 \), then \( x = u \) is invertible, and hence it has no irreducible divisors, that is, all factorizations have length \( 0 \).

  Otherwise, suppose that factorizations of length \( n - 1 \) are unique and that we are given the factorizations
  \begin{equation}\label{eq:def:irreducible_factorization/uniqueness/proof/assumption}
    x = u p_1 \cdots p_n = v q_1 \cdots q_m.
  \end{equation}

  Since \( p_n \) is prime, there exists an index \( k_0 \) among \( 1, \ldots, m \) such that \( p_n \) divides \( q_{k_0} \). Both are irreducible, thus, \( p_n = w q_k \) for some invertible element \( w \). We can cancel \( p_1 \) to obtain
  \begin{equation*}
    u p_1 \cdots p_{n-1} = (vw) q_1 \cdots q_{k_0-1} q_{k_0+1} \cdots q_m.
  \end{equation*}

  By the inductive hypothesis, this factorization is unique. Hence, \( n = m \), and there exists a permutation \( \pi \in S_{n-1} \) such that \( p_k = q_{\pi(k)} \) for \( k = 1, \ldots, n - 1 \). Then
  \begin{equation*}
    \widehat{\pi}(k) \coloneqq \begin{cases}
      k_0,    & k = n \\
      \pi(i), & \T{otherwise.}
    \end{cases}
  \end{equation*}
  is a permutation witnessing the equivalence of the factorizations in \eqref{eq:def:irreducible_factorization/uniqueness/proof/assumption}.

  \SubProofOf{thm:def:irreducible_factorization/polynomial_ring}

  \SufficiencySubProof* Suppose that
  \begin{equation}\label{eq:thm:def:irreducible_factorization/polynomial_ring/sufficiency_assumption}
    x = u(X) p_1(X) \cdots p_n(X)
  \end{equation}
  be an irreducible factorization of \( x \in D \) in \( D[X] \). By \fullref{thm:def:polynomial_degree/product}, all polynomials in this factorization are constants. By \fullref{thm:def:domain_divisibility/irreducible_in_polynomial_ring}, since they are irreducible in \( D[X] \), they are also irreducible in \( D \).

  Therefore, \eqref{eq:thm:def:irreducible_factorization/polynomial_ring/sufficiency_assumption} is an irreducible factorization of \( x \) in \( D \).

  \NecessitySubProof* Due to \fullref{thm:def:polynomial_algebra/invertible} and \fullref{thm:def:domain_divisibility/irreducible_in_polynomial_ring}, irreducible elements and units in \( D \) are also irreducibles and units in \( D[X] \). Hence, every irreducible factorization in \( D \) is also an irreducible factorization in \( D[X] \).
\end{proof}

\paragraph{Factorial domains}

\begin{definition}\label{def:factorial_domain}
  We say that an \hyperref[def:integral_domain]{integral domain} is a \term[ru=факториальное кольцо (\cite[def. 9.7.1]{Винберг2014Алгебра})]{factorial domain} or \term{unique factorization domain} (UFD) if any of the following equivalent conditions hold:
  \begin{thmenum}
    \thmitem{def:factorial_domain/factorization}\mcite[def. V.1.10]{Aluffi2009Algebra} Every element \hyperref[def:irreducible_factorization/unique]{factors uniquely} into a product of \hyperref[def:domain_divisibility/irreducible]{irreducible} elements.

    \thmitem{def:factorial_domain/gcd} It is a \hyperref[def:gcd_domain]{GCD domain} and every ascending sequence of principal ideals \hyperref[def:stabilizing_sequence]{stabilizes}.
  \end{thmenum}
\end{definition}
\begin{defproof}
  \ImplicationSubProof{def:factorial_domain/factorization}{def:factorial_domain/gcd} Suppose that every element has a unique factorization.

  \SubProof*{Proof that it is a GCD domain} Fix arbitrary elements \( x \) and \( y \). We will show that they have a GCD.

  Let \( x = u p_1 \cdots p_n \) be a decomposition of \( x \). Define \( y_0, y_1, \ldots, y_n \) as follows:
  \begin{equation*}
    y_k \coloneqq \begin{cases}
      y,             &k = 0, \\
      y_{k-1} / p_k, &k > 0 \T{and} p_k \mid y_{k-1}, \\
      y_{k-1},       &k > 0 \T{and} p_k \not\mid y_{k-1}.
    \end{cases}
  \end{equation*}

  Finally, let \( q_k \coloneqq y_k / y_{k-1} \) for \( k = 1, \ldots, n \) so that \( q_k \) is either \( p_k \) if \( p_k \) divides both \( x \) and \( y \) the same amount of times and \( 1 \) otherwise. Then \( q_1 \cdots q_n \) is a greatest common divisor of \( x \) and \( y \).

  \SubProof*{Proof of ascending chain condition} Fix an ascending sequence of principal ideals
  \begin{equation*}
    \braket{ x_1 } \subseteq \braket{ x_2 } \subseteq \braket{ x_3 } \cdots.
  \end{equation*}

  \Fullref{thm:def:semiring_ideal/division} implies that \( x_{k+1} \) divides \( x_k \) for \( k = 1, 2, \ldots \).

  Let \( x_1 = y x_2 \) and fix \hyperref[def:irreducible_factorization]{irreducible factorizations}
  \begin{align*}
    x_1 &= u p_1 \cdots p_n \\
    x_2 &= v q_1 \cdots q_m \\
    y   &= w r_1 \cdots r_k.
  \end{align*}

  Since the factorizations are unique, we have \( n = m + k \). If \( k = 0 \), then \( x_1 \) and \( x_2 \) are associated and \( \braket{ x_1 } = \braket{ x_2 } \). If \( k > 0 \), then \( \braket{ x_1 } \subsetneq \braket{ x_2 } \), and \( x_2 \) has a strictly shorter irreducible factorization.

  Proceeding by induction on the length of the factorization, we conclude that there are at most \( n \) strict inclusions in the sequence of ideals.

  \ImplicationSubProof{def:factorial_domain/gcd}{def:factorial_domain/factorization} Let \( D \) be a GCD domain in which every ascending sequence of principal ideals stabilizes.

  The latter condition via \fullref{thm:def:irreducible_factorization/existence} ensures that every element of \( D \) has at least one irreducible factorization.

  Furthermore, since \( D \) is a GCD domain, \fullref{thm:def:gcd_domain/irreducible_is_prime} implies that every irreducible element is prime, so the assumptions of \fullref{thm:def:irreducible_factorization/uniqueness} are satisfied and irreducible factorization are unique.
\end{defproof}

\begin{example}\label{ex:def:factorial_domain}
  We list examples of \hyperref[def:factorial_domain]{factorial domains}:
  \begin{thmenum}
    \thmitem{ex:def:factorial_domain/integers} By \fullref{thm:fundamental_theorem_of_arithmetic}, the natural numbers are a factorial domain.

    \thmitem{ex:def:factorial_domain/pid} More generally, by \fullref{thm:def:principal_ideal_domain/factorial}, every principal ideal domain is factorial.

    \thmitem{ex:def:factorial_domain/infinitely_descending_divisor_chain} Consider the subring \( D \) of the polynomial ring \( \BbbR[X] \) with an integer free term. The following are elements of \( D \):
    \begin{equation*}
      \frac X {2^1}, \frac X {2^2}, \frac X {2^3}, \cdots, \frac X {2^n}, \cdots
    \end{equation*}The Contortionist - Language
    but their denominators are not.

    Hence, we have the following strictly descending sequence of proper divisors:
    \begin{equation*}
      \cdots \mid \frac X {2^n} \mid \frac X {2^{n-1}} \mid \cdots \frac X {2^2} \mid \frac X {2^1} \mid X,
    \end{equation*}
    which in turn corresponds to an strictly ascending sequence of principal ideals.

    Therefore, \( D \) is not a factorial domain.
  \end{thmenum}
\end{example}

\begin{proposition}\label{thm:def:factorial_domain}
  \hyperref[def:factorial_domain]{Factorial domains} have the following basic properties:
  \begin{thmenum}
    \thmitem{thm:def:factorial_domain/polynomial_ring} If the \hyperref[def:polynomial_algebra]{polynomial ring} \( R[X] \) over a commutative ring \( R \) is a factorial domain, then \( R \) also is.

    The converse to this is true, but it is more difficult to prove. See \fullref{thm:polynomial_ring_over_factorial}.
  \end{thmenum}
\end{proposition}
\begin{proof}
  \SubProofOf{thm:def:factorial_domain/polynomial_ring} Suppose that \( D[X] \) is a factorial domain. \Fullref{thm:def:irreducible_factorization/polynomial_ring} implies that every irreducible factorization of \( x \in D \) in \( D[X] \) is also an irreducible factorization in \( D[X] \). This in turn implies both existence and uniqueness.
\end{proof}

\paragraph{Principal ideal domains}

\begin{definition}\label{def:principal_ideal_domain}\mcite[def. III.4.2]{Aluffi2009Algebra}
  We say that an \hyperref[def:integral_domain]{integral domain} is a \term[bg=област на главни идеали (\cite[def. VI.3]{ГеновИПр1991Алгебра}), ru=\cite[def. 9.3]{Винберг2014Алгебра}]{principal ideal domain} if any of the following equivalent conditions hold:
  \begin{thmenum}
    \thmitem{def:principal_ideal_domain/direct} Every \hyperref[def:semiring_ideal]{ideal} is \hyperref[def:semiring_ideal/principal]{principal}, i.e. has a single generator.

    \thmitem{def:principal_ideal_domain/bezout} It is a \hyperref[def:bezout_domain]{Bezout domain} and every ascending sequence of principal ideals \hyperref[def:stabilizing_sequence]{stabilizes}.
  \end{thmenum}
\end{definition}
\begin{proof}
  \ImplicationSubProof{def:principal_ideal_domain/direct}{def:principal_ideal_domain/bezout} It is immediate that, if all ideals are principal, the domain is both noetherian and is a Bezout domain.

  \ImplicationSubProof{def:principal_ideal_domain/bezout}{def:principal_ideal_domain/direct} Let \( D \) be a Bezout domain and suppose that every ascending sequence of principal ideals stabilizes.

  Fix an arbitrary ideal \( I \) and an arbitrary member \( x_1 \) of \( I \). Define the following sequence via recursion on \( k \):
  \begin{displayquote}
    If there exists members of \( I \) not in \( \braket{ x_k } \), let \( a \) be such a member, and define \( x_{k+1} \) to be a GCD of \( x_k \) and \( a \).

    Otherwise, define \( x_{k+1} \) to be \( x_k \).
  \end{displayquote}

  Since \( D \) is a Bezout domain, \( x_{k+1} \) is a linear combination of \( x_1, \ldots, x_k \), and thus \( \braket{ x_{k+1} } = \braket{ x_1, \ldots, x_k } \). Furthermore, \fullref{thm:def:bezout_domain/ideal_gcd_closed} implies that \( x_k \) belongs to \( I \) for every index \( k \).

  By construction, for each index \( k \), \( x_{k+1} \) divides \( x_k \). The corresponding sequence of principal ideals stabilizes at some index \( n \). By construction, \( \braket{ x_n } \) must be \( I \) because otherwise we would have found some member \( a \) from \( I \setminus \braket{ x_n } \).

  Therefore, \( I = \braket{ x_n } \).
\end{proof}

\begin{example}\label{ex:def:principal_ideal_domain}
  We list examples of \hyperref[def:principal_ideal_domain]{principal ideal domains}:
  \begin{thmenum}
    \thmitem{ex:def:principal_ideal_domain/integers} Every \hyperref[def:euclidean_domain]{Euclidean domain} is a principal ideal domain as a consequence of \fullref{thm:def:euclidean_domain/pid}.

    \thmitem{ex:def:principal_ideal_domain/integer_polynomials} The algebra \( \BbbZ[X] \) of univariate integer polynomials is not a principal ideal domain because it is not a Bezout domain --- as discussed in \fullref{ex:def:bezout_domain/integer_polynomials}, the ideal \( \braket{ 2, X } \) is not principal.

    \thmitem{ex:def:principal_ideal_domain/multivariate_polynomials} Again, as discussed in, \fullref{ex:def:bezout_domain/multivariate_polynomials}, multivariate polynomial rings are not Bezout domains, and hence not principal ideal domains.

    A necessary condition for a polynomial ring to be a PID is given in \fullref{thm:def:principal_ideal_domain/field_polynomials} --- its underlying ring must be an Euclidean domain.
  \end{thmenum}
\end{example}

\begin{proposition}\label{thm:def:principal_ideal_domain}
  \hyperref[def:principal_ideal_domain]{Principal ideal domains} have the following basic properties:
  \begin{thmenum}
    \thmitem{thm:def:principal_ideal_domain/noetherian} Every principal ideal domain is \hyperref[def:noetherian_semiring]{noetherian}.

    \thmitem{thm:def:principal_ideal_domain/factorial} Every principal ideal domain is \hyperref[def:factorial_domain]{factorial}.

    \thmitem{thm:def:principal_ideal_domain/prime_ideal_is_maximal} \hyperref[def:semiring_ideal/prime]{Prime ideals} in a principal ideal domains are \hyperref[def:semiring_ideal/maximal]{maximal}.

    \thmitem{thm:def:principal_ideal_domain/field_polynomials} A commutative ring \( R \) is a \hyperref[def:field]{field} if and only if its \hyperref[def:polynomial_algebra]{polynomial algebra} \( R[X] \) is a principal ideal domain.
  \end{thmenum}
\end{proposition}
\begin{proof}
  \SubProofOf{thm:def:principal_ideal_domain/noetherian} Every principal ideal is finitely-generated.

  \SubProofOf{thm:def:principal_ideal_domain/factorial} \Fullref{def:factorial_domain/gcd} follows directly from \fullref{def:principal_ideal_domain/bezout}.

  \SubProofOf{thm:def:principal_ideal_domain/prime_ideal_is_maximal} Let \( P \) be a prime ideal in a principal ideal domain. Then \( P = \braket{ p } \) for some prime element \( p \). By \fullref{thm:def:domain_divisibility/prime_is_irreducible}, \( p \) is irreducible, and hence \( \braket{ p } \) is a maximal ideal.

  \SubProofOf{thm:def:principal_ideal_domain/field_polynomials} Let \( R \) be a commutative ring.

  \SufficiencySubProof* Suppose that \( R \) is a field. \Fullref{thm:def:integral_domain/polynomial_ring} implies that \( R[X] \) is a domain. Let \( I \) be an ideal in \( R[X] \).

  \Fullref{alg:euclidean_division_of_polynomials} allows us to divide any polynomial \( f(X) \) by a monic polynomial \( g(X) \), such that \( f(X) = g(X) q(X) + r(X) \) and \( r(X) \) is either zero or \( \deg r < \deg g \).

  More generally, if \( b_m \) is the leading coefficient of \( g(X) \), then \( g(X) / b_m \) is a monic polynomial and
  \begin{equation*}
    f(X) = \frac {g(X)} {b_m} q(X) + r(X).
  \end{equation*}

  If both \( f(X) \) and \( g(X) \) are in \( R[X] \), then so it \( r(X) = f(X) - \frac {g(X)} {b_m} q(X) \).

  Now let \( g(X) \) be a polynomial from \( I \) of minimal degree. Then \( r(X) \) cannot have a degree less than \( g(X) \), and thus the remaining option for \( r(X) \) in this case is to be zero. But then
  \begin{equation*}
    f(X) = \frac {q(X)} {b_m} g(X).
  \end{equation*}

  Since \( f(X) \) was arbitrary, we conclude that \( g(X) \) generates \( I \). Generalizing on \( I \), we conclude that \( R[X] \) is a principal ideal domain.

  \NecessitySubProof* Suppose that \( R[X] \) is a principal ideal domain.
  \begin{itemize}
    \item \Fullref{thm:def:integral_domain/subring} implies that \( R \) is an integral domain.
    \item \Fullref{thm:quotient_by_prime_ideal} implies that \( \braket{ X } \) is a prime ideal in \( R[X] \).
    \item Since \( R \) is a PID, \Fullref{thm:def:principal_ideal_domain/prime_ideal_is_maximal} then implies that \( \braket{ X } \) is a maximal ideal.
    \item \Fullref{thm:quotient_structure_universal_property} implies that \( R \) is isomorphic to \( R[X] / \braket{ X } \).
    \item \Fullref{thm:quotient_by_maximal_ideal} then implies \( R[X] / \braket{ X } \cong R \) is a field.
  \end{itemize}
\end{proof}

\begin{proposition}\label{thm:multiplicative_group_of_integers_modulo}
  The \hyperref[def:semiring]{multiplicative group} \( \BbbZ_n^\times \) of the ring \hyperref[def:ring_of_integers_modulo]{\( \BbbZ_n \)} of integers modulo \( n > 1 \) is the set of all positive integers \hyperref[def:coprime_elements]{coprime} to \( n \).
\end{proposition}
\begin{proof}
  Note that \( x < n \) is invertible modulo \( n \) if and only if there exists an integer \( a \) such that \( ax = 1 \pmod n \). That is, if there exist integers \( a \) and \( b \) such that \( ax + bn = 1 \).

  The rest of the theorem follows from \eqref{eq:def:bezout_domain/identity} in one direction and \fullref{thm:def:gcd/bezouts_identity_converse} in the other direction.
\end{proof}

\begin{corollary}\label{thm:zp_is_field}
  The ring \( \BbbZ_p \) is a \hyperref[def:field]{field} if and only if \( p \) is a \hyperref[def:prime_number]{prime number}.
\end{corollary}
\begin{proof}
  Follows from \fullref{thm:multiplicative_group_of_integers_modulo}.
\end{proof}

\paragraph{Euclidean domains}

\begin{definition}\label{def:euclidean_domain}\mcite[124]{Rotman2010Algebra}
  An \term[bg=евклидов (пръстен) (\cite[def. VI.2]{ГеновИПр1991Алгебра}), ru=евклидово (кольцо) (\cite[def. 3.5.2]{Винберг2014Алгебра})]{Euclidean domain} is an \hyperref[def:integral_domain]{integral domain} \( D \) endowed with a function
  \begin{equation*}
    \delta: D \setminus \set{ 0 } \to \BbbZ_{\geq 0},
  \end{equation*}
  which we call the \term{Euclidean degree}, such that the following conditions hold:
  \begin{thmenum}
    \thmitem{def:euclidean_domain/multiplication}  For every nonzero pair \( x \) and \( y \) we have
    \begin{equation}\label{eq:def:euclidean_domain/multiplication}
      \deg(x) \leq \deg(xy).
    \end{equation}

    \thmitem{def:euclidean_domain/division} For every pair \( x \) and \( y \) of elements of \( D \) in which \( y \) is nonzero, there exists a pair \( q \) and \( r \), where \( r \) is either zero or \( \delta(r) < \delta(y) \), such that
    \begin{equation}\label{eq:def:euclidean_domain/division}
      x = yq + r
    \end{equation}
  \end{thmenum}

  We say that \( y \) \term{divides} \( x \) with \term{quotient} \( q \) and \term{remainder} \( r \). If the quotient and remainder are unique, as they usually are, we use the special notation
  \begin{align*}
    &\quot(x, y) \coloneqq q, \\
    &\rem(x, y) \coloneqq r = x - y \cdot \quot(x, y).
  \end{align*}
\end{definition}
\begin{comments}
  \item The concept is inconsistent across authors.
  \begin{itemize}
    \item \incite[124]{Rotman2010Algebra} and \incite[def. 3.5.2]{Винберг2014Алгебра} both use the term \enquote{Euclidean ring} for our definition.
    \item \incite[36]{Шафаревич1999Алгебра} and \incite[def. VI.2]{ГеновИПр1991Алгебра} also use \enquote{Euclidean ring}, however neither state \fullref{def:euclidean_domain/multiplication}, and the latter even requires degrees to be defined for zero elements.
    \item \incite[392]{Knapp2016BasicAlgebra} and \incite[def 2.5]{Jacobson1985AlgebraPart1} use the term \enquote{Euclidean domain}, however they both require \( \delta \) to be defined for zero elements and neither states \fullref{def:euclidean_domain/multiplication}.
  \end{itemize}
\end{comments}

\begin{example}\label{ex:def:euclidean_domain}
  We list examples of \hyperref[def:euclidean_domain]{Euclidean domains}:
  \begin{thmenum}
    \thmitem{ex:def:euclidean_domain/integers} \Fullref{thm:integers_are_euclidean_domain} states that the ring of integers is an Euclidean domain.

    \thmitem{ex:def:euclidean_domain/polynomials} \Fullref{thm:def:principal_ideal_domain/field_polynomials} characterizes Euclidean polynomial rings as univariate polynomial rings over fields.

    \thmitem{ex:def:euclidean_domain/not_euclidean} Principal ideal domain that are not euclidean are discussed in \cite{Anderson1988NonEuclideanPID}.
  \end{thmenum}
\end{example}

\begin{proposition}\label{thm:def:euclidean_domain}
  \hyperref[def:euclidean_domain]{Euclidean domains} have the following basic properties:
  \begin{thmenum}
    \thmitem{thm:def:euclidean_domain/pid} Every Euclidean domain is a \hyperref[def:principal_ideal_domain]{principal ideal domain}.

    \thmitem{thm:def:euclidean_domain/field} Every field is an \hyperref[def:euclidean_domain]{Euclidean domain}.

    \thmitem{thm:def:euclidean_domain/polynomials} For any given field \( \BbbK \), the \hyperref[def:polynomial_algebra]{polynomial algebra} \( \BbbK[X] \) via \fullref{alg:euclidean_division_of_polynomials} becomes an Euclidean domain when using the\hyperref[def:polynomial_degree]{polynomial degree} as a degree function.

    \thmitem{thm:def:euclidean_domain/nested_quot} We have
    \begin{equation}\label{eq:thm:def:euclidean_domain/nested_quot}
      \quot(\quot(a, b), c) = \quot(a, bc).
    \end{equation}
  \end{thmenum}
\end{proposition}
\begin{proof}
  \SubProofOf{thm:def:euclidean_domain/pid} This proof generalizes one direction of \fullref{thm:def:principal_ideal_domain/field_polynomials} by considering an element of minimal Euclidean degree rather than a polynomial of minimal degree.

  \SubProofOf{thm:def:euclidean_domain/field} By \fullref{thm:division_ring_is_entire}, a field is an integral domain. Since every pair from \( \BbbK \) is divisible without remainder as long as the denominator is nonzero, the Euclidean function can be arbitrary; for definiteness, we take it to be canonically zero.

  \SubProofOf{thm:def:euclidean_domain/polynomials} Trivial.
\end{proof}

\begin{algorithm}[Euclidean algorithm]\label{alg:euclidean_algorithm}
  In an \hyperref[def:euclidean_domain]{Euclidean domain}, we can explicitly construct a \hyperref[def:gcd]{greatest common divisor} \( g \) of arbitrary elements \( x \) and \( y \) as follows:
  \begin{thmenum}
    \thmitem{alg:euclidean_algorithm/base} Define \( r_0 \coloneqq x \) and \( r_1 \coloneqq y \).
    \thmitem{alg:euclidean_algorithm/step} Starting with \( k = 2 \), if \( r_{k-1} \) is zero, halt the algorithm with \( g \coloneqq r_{k-1} \).

    Otherwise, obtain a quotient \( q_k \) and remainder \( r_k \) so that
    \begin{equation*}
      r_{k-2} = r_{k-1} q_k + r_k.
    \end{equation*}

    Repeat \fullref{alg:euclidean_algorithm/step} with \( k + 1 \) instead of \( k \).
  \end{thmenum}
\end{algorithm}
\begin{comments}
  \item This algorithm can be found as \identifier{arithmetic.primes.gcd} in \cite{notebook:code}.
\end{comments}
\begin{defproof}
  Euclidean division ensures that, at the \( k \)-th step, either \( r_{k-1} \) is zero or \( r_k \) is zero or \( \delta(r_k) < \delta(r_{k-1}) \). Thus, there can only be finitely many steps where \( r_{k-1} \) is nonzero. So the algorithm halts.

  Let \( n \) be the index of the last nonzero remainder or \( 0 \) if \( x = 0 \). We show by induction on \( 0 \leq k < n \) that \( r_n \) divides \( r_{n-k} \).

  \begin{itemize}
    \item The base case \( k = 0 \) is obvious since \( r_n \) divides itself.
    \item If \( r_n \) divides \( r_{n-i} \) for \( 0 \leq i < k \), since
    \begin{equation*}
      r_{n-k} = r_{n-(k-1)} q_{n-(k-2)} + r_{n-(k-2)}
    \end{equation*}
    and both of the terms on the right-hand side are multiples of \( r_n \), the left-hand side \( r_{n-k} \) is also a multiple.
  \end{itemize}

  So, we conclude that \( r_n \) divides both \( r_{n-(n-1)} = r_1 = y \) and \( r_{n-n} = r_0 = x \).

  Finally, we must show that \( r_n \) is greatest among all common divisors of \( x \) and \( y \). Let \( d \) be a common divisor.

  We will use induction on \( k < n \) to show that \( d \mid r_k \).
  \begin{itemize}
    \item We have \( d \mid r_0 \) and \( d \mid r_1 \) by assumption.
    \item Fix \( 2 \leq k \leq n \) and suppose that, for every \( i < k \), \( d \mid r_i \). Then
    \begin{equation*}
      r_{k-1} = r_k q_{k+1} + r_{k+1},
    \end{equation*}
    and since \( i \) divides both \( r_{k-2} \) and \( r_{k-1} \), \( d \) also their linear combination \( r_{k+1} \).
  \end{itemize}

  Hence, \( d \mid r_n \). Since our choice of common divisor \( d \) was arbitrary, we conclude that \( r_n \) is a greatest common divisor.
\end{defproof}

\begin{algorithm}[Extended Euclidean algorithm]\label{alg:extended_euclidean_algorithm}
  In an \hyperref[def:euclidean_domain]{Euclidean domain}, given a pair \( x \) and \( y \), \fullref{alg:euclidean_algorithm} allows us to construct a greatest common divisor \( g \). We can also explicitly construct elements \( a \) and \( b \) so that Bezout's identity \eqref{eq:def:bezout_domain/identity} holds, i.e.
  \begin{equation*}
    g = ax + by.
  \end{equation*}

  The algorithm is as simple as defining the following in the inductive step \fullref{alg:euclidean_algorithm/step}:
  \begin{equation*}
    a_k \coloneqq \begin{cases}
      1,                     &k = 0, \\
      0,                     &k = 1, \\
      a_{k-2} - q_k a_{k-1}, &k > 1,
    \end{cases}
  \end{equation*}
  and
  \begin{equation*}
    b_k \coloneqq \begin{cases}
      0,                     &k = 0, \\
      1,                     &k = 1, \\
      b_{k-2} - q_k b_{k-1}, &k > 1.
    \end{cases}
  \end{equation*}

  If \( r_n \) is the latest nonzero remainder, halt the algorithm with \( a \coloneqq a_n \) and \( b \coloneqq b_n \).
\end{algorithm}
\begin{comments}
  \item This algorithm can be found as \identifier{arithmetic.primes.extended_gcd} in \cite{notebook:code}.
\end{comments}
\begin{defproof}
  We will prove with induction on \( k \leq n \) that
  \begin{equation*}
    r_k = a_k x + b_k y.
  \end{equation*}

  \begin{itemize}
    \item We have
    \begin{equation*}
      r_0 = x = 1 \cdot x + 0 \cdot y = a_0 x + b_0 y
    \end{equation*}
    and
    \begin{equation*}
      r_1 = y = 0 \cdot x + 1 \cdot y = a_1 x + b_1 y.
    \end{equation*}

    \item Fix \( 2 \leq k \leq n \) and suppose that \( r_i = a_i x + b_i y \) whenever \( i < n \). Then
    \begin{equation*}
      r_{k-2} = r_{k-1} q_k + r_k
    \end{equation*}
    becomes
    \begin{equation*}
      a_{k-2} x + b_{k-2} y = (a_{k-1} x + b_{k-1} y) q_k + r_k.
    \end{equation*}

    Therefore,
    \begin{equation*}
      r_k = (\underbrace{a_{k-2} - q_k a_{k-1}}_{a_k}) x + (\underbrace{b_{k-2} - q_k b_{k-1}}_{b_k}) y
    \end{equation*}
  \end{itemize}
\end{defproof}

\paragraph{Fields of fractions}

\begin{proposition}\label{thm:field_of_fractions}\mcite[110]{Lang2002Algebra}
  Let \( D \) be an \hyperref[def:integral_domain]{integral domain}. The \hyperref[def:ring_localization]{localization} of \( D \) at the zero ideal \( \set{ 0 } \) is a \hyperref[def:field]{field}, which we call the \term[bg=поле от частни (\cite[def. V.16]{ГеновИПр1991Алгебра}), ru=поле частных (\cite[26]{Шафаревич1999Алгебра})]{field of fractions} of \( D \).
\end{proposition}
\begin{defproof}
  Denote \( D \setminus \set{ 0 } \) by \( S \). We will show that \( S^{-1} D \) is indeed a field.

  By \fullref{thm:def:ring_localization/prime_ideals}, the localization by the prime ideal \( \set{ 0 } \) has only one maximal ideal --- \( S^{-1} \set{ 0 } \). Since \( 0 \) is absorbing, \( S^{-1} \set{ 0 } \) is again the zero ideal. Therefore, it is the only proper ideal of the localization \( S^{-1} D \), and hence the localization is a \hyperref[def:simple_object]{simple ring}.

  Since \( D \) is an integral domain, by \fullref{thm:def:ring_localization/injective_inclusion}, \( S^{-1} D \) is a superring of \( D \). It is therefore a nontrivial commutative simple ring, and thus it satisfies \fullref{def:field/simple}.
\end{defproof}

\begin{theorem}[Field of fractions universal property]\label{thm:field_of_fractions_universal_property}
  The \hyperref[thm:field_of_fractions]{field of fractions} \( \BbbK \) of the integral domain \( D \) satisfies the following \hyperref[rem:universal_mapping_property]{universal mapping property}:
  \begin{displayquote}
    For every field \( \BbbL \) and every ring homomorphism \( \varphi: D \to \BbbL \), \( \varphi \) \hyperref[def:factors_through]{uniquely factors through} \( \BbbK \). More precisely, there exists a unique field homomorphism \( \widetilde{\varphi}: \BbbK \to \BbbL \) such that the following diagram commutes:
    \begin{equation}\label{eq:thm:field_of_fractions_universal_property/diagram}
      \begin{aligned}
        \includegraphics[page=1]{output/thm__field_of_fractions_universal_property}
      \end{aligned}
    \end{equation}
  \end{displayquote}
\end{theorem}
\begin{proof}
  This is simply a special case of \fullref{thm:ring_localization_universal_property}.
\end{proof}

\begin{definition}\label{def:rational_function_field}\mcite[def. V.4.13]{Aluffi2009Algebra}
  The \term[bg=поле на рационалните функции (\cite[360]{ГеновИПр1991Алгебра}), ru=поле рациональных функций (\cite[18]{Шафаревич1999Алгебра})]{field of rational (algebraic) functions} \( D(\mscrX) \) for the set of indeterminates \( \mscrX \) over the \hyperref[def:integral_domain]{integral domain} \( D \) is the \hyperref[thm:field_of_fractions]{field of fractions} of the corresponding \hyperref[def:polynomial_algebra]{polynomial algebra} \( D[\mscrX] \).
\end{definition}
\begin{comments}
  \item Despite the name, elements of the field of fractions are not actually functions, but merely formal expressions. In particular, an analog of \fullref{thm:polynomial_algebra_universal_property} does not really make sense.
\end{comments}

\paragraph{Polynomial rings over domains}

\begin{proposition}\label{thm:polynomial_ring_over_gcd_domain}
  If the \hyperref[def:integral_domain]{integral domain} \( D \) is a \hyperref[def:gcd_domain]{greatest common divisor domain}, so is \( D[X] \).
\end{proposition}
\begin{proof}
  Let \( D \) be a GCD domain and let \( \BbbK \) be its \hyperref[thm:field_of_fractions]{field of fractions}.

  Let \( p(X) \) and \( q(X) \) be arbitrary polynomials in \( D[X] \). We will show that they have a greatest common divisor.

  By \fullref{thm:def:principal_ideal_domain/field_polynomials}, \( \BbbK[X] \) is an Euclidean domain, and thus \( p(X) \) and \( q(X) \) have a GCD in \( \BbbK[X] \), which is unique up to multiplication by an invertible element in \( \BbbK[X] \). Thus, taking an arbitrary GCD
  \begin{equation*}
    r(X) = \sum_{k=0}^n \frac {a_k} {b_k} X^k,
  \end{equation*}
  the polynomial \( b_0 \cdots b_n r(X) \) is also a GCD. Furthermore, the latter is actually a polynomial in \( D[X] \).

  Therefore, \( p(X) \) and \( q(X) \) have a GCD in \( D[X] \).
\end{proof}

\begin{definition}\label{def:primitive_polynomial}\mcite[394]{Knapp2016BasicAlgebra}
  We say that a polynomial \( p(X) \) in a \hyperref[def:gcd_domain]{GCD domain} is \term[bg=примитивен (полином) (\cite[43]{ГеновИПр1991Алгебра}), ru=примитивный (многочлен) (\cite[124]{Винберг2014Алгебра})]{primitive} if its coefficients have an invertible \hyperref[def:gcd]{greatest common divisor}.
\end{definition}
\begin{comments}
  \item For a \hyperref[def:bezout_domain]{Bezout domain}, this condition is equivalent to the coefficients being \hyperref[def:coprime_elements]{coprime}.
\end{comments}

\begin{lemma}[Gauss' lemma]\label{thm:gauss_lemma}
  If \( p(X) \) and \( q(X) \) are \hyperref[def:primitive_polynomial]{primitive polynomials}, then so is \( p(X) q(X) \).
\end{lemma}
\begin{proof}
  Fix two primitive polynomials
  \begin{align*}
    p(X) = \sum_{k=0}^n a_k X^k,
    &&
    q(X) = \sum_{k=0}^m b_k X^k.
  \end{align*}

  Let \( d \) be a GCD of the coefficients of \( p(X) q(X) \). It divides
  \begin{equation*}
    \sum_{i+j=k} a_i b_j,
  \end{equation*}
  the \( k \)-th coefficient of \( p(X) q(X) \), and hence also \( a_i b_j \) for every particular pair of indices \( i < n \) and \( j < m \).

  In particular, for a fixed index \( i \), \( d \) simultaneously divides \( a_i b_1, a_i b_2, \ldots, a_i b_m \).

  Suppose that \( d \) is not invertible. Then there exists a prime element \( p \) that divides \( d \), and hence also the aforementioned products. Since it is prime, \( p \) must divide \( a_i \) or \( b_1 \), \( a_i \) or \( b_2 \), etc.

  But \( p \) cannot divide each one of \( b_1, \ldots, b_m \) because that would imply that \( g(X) \) is not primitive. It then follows that \( p \) divides \( a_i \) for a fixed value of \( i \). Generalizing on \( i \), we conclude that \( p \) divides \( a_1, \ldots, a_n \), which implies that \( f(X) \) is not primitive.

  But both \( f(X) \) and \( g(X) \) are primitive by assumption. Then we have reached a contradiction with the assumption that \( d \) is not invertible.

  Therefore, \( d \) is an invertible GCD of the coefficients of \( p(X) q(X) \), meaning that \( p(X) q(X) \) is a primitive polynomial.
\end{proof}

\begin{lemma}\label{thm:irreducible_primitive_polynomial_in_field_of_fractions}
  Let \( D \) be a \hyperref[def:gcd_domain]{GCD domain} and let \( \BbbK \) be its \hyperref[thm:field_of_fractions]{field of fractions}. If the polynomial \( p(X) \) from \( \BbbK[X] \) is \hyperref[def:domain_divisibility/irreducible]{irreducible} (in \( \BbbK[X] \)), then there exists a scalar \( c \) from \( \BbbK \) and an irreducible in \( D[X] \) primitive polynomial \( q(X) \) such that \( p(X) = c \cdot q(X) \).
\end{lemma}
\begin{proof}
  Let \( p(X) \) be irreducible in \( \BbbK[X] \). It has the form
  \begin{equation*}
    p(X) = \sum_{k=1}^n \frac {a_k} {b_k} X^k.
  \end{equation*}

  Clearly \( b_1 \cdots b_n p(X) \) is in \( D[X] \). Let \( d \) be a GCD (in \( D \)) of the coefficients of \( b_1 \cdots b_n p(X) \). Let \( c \coloneqq d / b_1 \cdots b_n \) and define the following polynomial:
  \begin{equation*}
    q(X) \coloneqq \frac {p(X)} c.
  \end{equation*}

  By construction, \( q(X) \) is a primitive polynomial with coefficients in \( D \). Furthermore, as a scalar multiple of the irreducible in \( \BbbK[X] \) polynomial \( p(X) \), \( q(X) \) is also irreducible in \( \BbbK[X] \).

  We will now show that \( q(X) \) is irreducible in \( D[X] \). Let \( q(X) = f(X) g(X) \), where both multiplicands are from \( D[X] \).

  Since \( q(X) \) is irreducible in \( \BbbK[X] \), at least one of \( f(X) \) or \( g(X) \) is invertible in \( \BbbK[X] \), hence it is a nonzero constant polynomial. Without loss of generality, suppose that this is \( f(X) \) and let \( f(X) = f_0 \).

  Then \( f_0 \) is a common divisor of the coefficients of \( q(X) \). Since \( q(X) \) is primitive in \( D[X] \), it follows that \( f_0 \) is invertible in \( D \).

  Therefore, \( q(X) \) is irreducible in \( D[X] \) and
  \begin{equation*}
    p(X) = c \cdot p(X).
  \end{equation*}
\end{proof}

\begin{proposition}\label{thm:polynomial_ring_over_factorial}
  If the \hyperref[def:integral_domain]{integral domain} \( D \) is a \hyperref[def:factorial_domain]{factorial domain}, so is \( D[X] \).
\end{proposition}
\begin{proof}
  Let \( D \) be a factorial domain and let \( \BbbK \) be its \hyperref[thm:field_of_fractions]{field of fractions}.

  By \fullref{thm:polynomial_ring_over_gcd_domain}, \( D[X] \) is a GCD domain. Then every irreducible element is prime as a shown in \fullref{thm:def:gcd_domain/irreducible_is_prime}, and \fullref{thm:def:irreducible_factorization/uniqueness} implies that, if an element has at least one \hyperref[def:irreducible_factorization]{irreducible factorization}, all others are equivalent to it.

  We will now show existence of irreducible factorizations.

  Let \( p(X) \) be a polynomial from \( D[X] \). It has an irreducible factorization in \( \BbbK[X] \):
  \begin{equation*}
    p(X) = u q_1(X) \cdots q_n(X).
  \end{equation*}

  For every \( q_i(X) \), \fullref{thm:irreducible_primitive_polynomial_in_field_of_fractions} gives us a constant \( c_i \) and an irreducible in \( D[X] \) primitive polynomial \( r_i(X) \) such that
  \begin{equation*}
    q_i(X) = c_i \cdot r_i(X).
  \end{equation*}

  Then
  \begin{equation*}
    p(X)
    =
    u \cdot q_1(X) \cdots q_n(X)
    =
    u c_1 \cdots c_n \cdot r_1(X) \cdots r_n(X).
  \end{equation*}

  Let \( \alpha \) be a GCD of the coefficients of \( p(X) \) and denote the product \( u c_1 \cdots c_n \) by \( \beta \). We have
  \begin{equation*}
    p(X) = \alpha \cdot \frac {p(X)} \alpha = \beta \cdot r_1(X) \cdots r_n(X).
  \end{equation*}

  The polynomial \( p(X) / \alpha \) is primitive by construction, and \( r_1(X) \cdots r_n(X) \) is primitive by \fullref{thm:gauss_lemma}. Then \( \alpha \) necessarily divides \( \beta \) and vice versa, hence they are associated, and so \( \beta = v \alpha \), where \( v \) is invertible.

  Finally, let \( \alpha = w t_1 \cdots t_m \) be an irreducible decomposition of \( \alpha \). \Fullref{thm:def:domain_divisibility/irreducible_in_polynomial_ring} implies that \( t_1 \cdots t_m \) are irreducible in \( D[X] \).

  Then
  \begin{equation*}
    p(X) = \underbrace{(v w) \cdot t_1 \cdots t_m}_{\beta} \cdot r_1(X) \cdots r_n(X)
  \end{equation*}
  is an irreducible factorization of \( p(X) \) in \( D[X] \).
\end{proof}

  \subsection{Field extensions}\label{subsec:field_extensions}

\paragraph{Algebraic and transcendental elements}

\begin{lemma}\label{thm:quotient_by_irreducible_polynomial}
  Let \( p(X) \) be an \hyperref[def:domain_divisibility/irreducible]{irreducible polynomial} over some \hyperref[def:field]{field} \( \Bbbk \). The \hyperref[def:algebra_over_ring/quotient]{quotient} of the \hyperref[def:polynomial_algebra]{polynomial algebra} \( \Bbbk[X] \) by the principal ideal \( \braket{ p(X) } \) is then a \hyperref[def:field/submodel]{field extension} of \( \Bbbk \).
\end{lemma}
\begin{proof}
  Since \( p(X) \) is irreducible, \fullref{thm:def:gcd_domain/irreducible_is_prime} implies that it is prime, and since \( \Bbbk[X] \) is a principal ideal domain, \fullref{thm:def:principal_ideal_domain/prime_ideal_is_maximal} implies that the ideal \( \braket{ p(X) } \) is maximal.

  By \fullref{thm:quotient_by_maximal_ideal}, \( \Bbbk[X] / \braket{ p(X) } \) is a field.
\end{proof}

\begin{definition}\label{def:algebraic_element}
  Let \( \BbbK \) be an \hyperref[def:field/submodel]{extension} of the \hyperref[def:field]{field} \( \Bbbk \). We say that an element \( u \) of \( \BbbK \) is \term[ru=алгебрический (елемент) (\cite[407]{Винберг2014})]{algebraic} over \( \Bbbk \) if any of the following equivalent conditions hold:
  \begin{thmenum}
    \thmitem{def:algebraic_element/direct}\mcite[124]{Jacobson1985Vol1} \( u \) is a \hyperref[def:polynomial_root]{root} of some nonzero polynomial from \( \Bbbk[X] \).

    \thmitem{def:algebraic_element/embedding} The \hyperref[rem:substitution_homomorphism]{evaluation map} \( \Phi_u: \Bbbk[X] \to \BbbK \), sending the monomial \( X \) to \( u \), is not \hyperref[def:function_invertibility/injective]{injective}.

    \thmitem{def:algebraic_element/quotient}\mcite[391]{Aluffi2009} There exists a \hyperref[def:monic_polynomial]{monic} \hyperref[def:domain_divisibility/irreducible]{irreducible polynomial} \( m(X) \) such that the quotient \( \Bbbk[X] / \braket{ m(X) } \) is isomorphic to the \hyperref[def:semiring_adjunction]{ring adjunction} \( \Bbbk[u] \).

    \thmitem{def:algebraic_element/field} The ring \( \Bbbk[u] \) is a \hyperref[def:field]{field}.

    \thmitem{def:algebraic_element/dimensions} The ring \( \Bbbk[u] \) has finite rank over \( \Bbbk \).
  \end{thmenum}
\end{definition}
\begin{defproof}
  \ImplicationSubProof{def:algebraic_element/direct}{def:algebraic_element/embedding} Let \( p(X) \) be a polynomial over \( \Bbbk \) such that \( \Phi_u(p) = 0 \). Then \( \Phi_u(pq) = 0 \) for any polynomial \( q \), hence \( \Phi_u \) is not injective.

  \ImplicationSubProof{def:algebraic_element/embedding}{def:algebraic_element/quotient} Suppose that \( \Phi_u \) is not injective. Since \( \Bbbk \) is a field, \fullref{thm:def:principal_ideal_domain/field_polynomials} implies that \( \Bbbk[X] \) is a principal ideal domain. So the kernel of \( \Phi_u \) is generated by a single polynomial. Let \( m(X) \) be a monic generator of \( \ker \Phi_u \). By \fullref{thm:ring_zero_morphisms/isomorphism},
  \begin{equation*}
    \Bbbk[X] / \braket{ m(X) } \cong \Bbbk[u].
  \end{equation*}

  It remains to show that \( m(X) \) is irreducible. Let \( m(X) = a(X) b(X) \). Since \( \Phi_u(m) = 0 \), either \( \Phi_u(a) = 0 \) or \( \Phi_u(b) = 0 \) or both. Then at least one of the factors must belong to \( \ker \Phi_u \). But \( m(X) \) is a generator, hence either \( a(X) \) or \( b(X) \) has the same degree (and thus the other is a constant polynomial). It follows that \( m(X) \) is irreducible.

  \ImplicationSubProof{def:algebraic_element/quotient}{def:algebraic_element/field} Suppose that, for some monic irreducible polynomial \( m(X) \), the quotient \( \Bbbk[X] / \braket{ m(X) } \) is isomorphic to \( \Bbbk[u] \). \Fullref{thm:quotient_by_irreducible_polynomial} then implies that \( \Bbbk[X] / \braket{ m(X) } \) is a field, and hence so is \( \Bbbk[u] \).

  \ImplicationSubProof{def:algebraic_element/field}{def:algebraic_element/dimensions} Suppose that \( \Bbbk[u] \) is a field.

  Since \( \Bbbk[X] \) is not a field, the evaluation map \( \Phi_u: \Bbbk[X] \to \Bbbk[u] \) cannot be injective. \Fullref{thm:group_homomorphism_trivial_kernel} implies that the kernel of \( \Phi_u \) is not trivial, and hence there is a monic polynomial \( m(X) \) of positive degree that generates it.

  \Fullref{thm:polynomial_quotient_module_dimension} then implies that \( \Bbbk[X] / \braket{ m(X) } \) has finite rank, and thus so does \( \Bbbk[u] \).

  \ImplicationSubProof{def:algebraic_element/dimensions}{def:algebraic_element/direct} Suppose that \( \Bbbk[u] \) has finite rank over \( \Bbbk \).

  Since \( \Bbbk[X] \) has infinite rank, the evaluation map \( \Phi_u: \Bbbk[X] \to \Bbbk[u] \) must not be injective and thus the kernel of \( \Phi_u \) has a nonzero element \( p(X) \). Then \( \Phi_u(p) = 0 \), so \( u \) is a root of \( p(X) \).
\end{defproof}

\begin{definition}\label{def:algebraic_element_minimal_polynomial}\mcite[388]{Aluffi2009}
  Let \( u \) be an \hyperref[def:algebraic_element]{algebraic element} over \( \Bbbk \). It thus satisfies \fullref{def:algebraic_element/quotient}, and there exists an irreducible monic element \( m(X) \) such that \( \Bbbk[u] \cong \Bbbk[X] / \braket{ m(X) } \). Furthermore, \( m(X) \) is unique, up to a scalar multiple, as a generator of the kernel of \( \Phi_u: \Bbbk[X] \to \Bbbk[u] \).

  We call \( m(X) \) the \term[bg=минимален полином (\cite[def. VI.2]{ГеновМиховскиМоллов1991}), ru=минимальный многочлен (\cite[410]{Винберг2014})]{minimal polynomial} of \( u \).
\end{definition}

\begin{example}\label{ex:def:algebraic_element}
  We list examples of \hyperref[def:algebraic_element]{algebraic elements}:
  \begin{thmenum}
    \thmitem{ex:def:algebraic_element/member} Any member of the field \( \Bbbk \) itself is algebraic over \( \Bbbk \).

    \thmitem{ex:def:algebraic_element/sqrt} Consider the \hyperref[def:nth_root]{\( n \)-th root} \( \sqrt[n]{ p } \) of (the real embedding of) an arbitrary \hyperref[def:prime_number]{prime number}.

    \Fullref{thm:power_of_nth_root} implies that \( \sqrt[n]{ p } \) is a root of the integer polynomial \( X^n - p \). Hence, it is algebraic over \( \BbbQ \). But \fullref{thm:nth_root_is_not_rational} implies that \( \sqrt[n]{ p } \) is not rational.

    Its \hyperref[def:algebraic_element_minimal_polynomial]{minimal polynomial} is \( X^n - p \) and hence the dimension of \( \BbbQ[\sqrt[n]{ p }] \) over \( \BbbQ \) is \( n \).

    \thmitem{ex:def:algebraic_element/complex} Similarly to the above example, the \hyperref[def:complex_numbers]{imaginary unit} \( i \) is, by definition, algebraic over \( \BbbR \) with minimal polynomial \( X^2 + 1 \). It is also algebraic over \( \BbbQ \) with the same polynomial, but we obtain the field \( \BbbQ[i] \) instead.

    \thmitem{ex:def:algebraic_element/multiple_minimal}\mcite{MathSE:non_unique_minimal_polynomial} If \( \Bbbk \) is not a field, there may be multiple minimal polynomials.

    Consider the ring \( \BbbZ[X^2, X^3] \) discussed in \fullref{ex:common_polynomial_divisors/incomparable} of all polynomials without linear terms. The monomial \( X \) from \( \BbbZ[X] \) satisfies both \( Y^2 - X^2 \) and \( Y^2 + X^2Y - X^2 - X^3 \) from \( \BbbZ[X^2, X^3][Y] \), and the two are not associates.
  \end{thmenum}
\end{example}

\begin{definition}\label{def:transcendental_element}\mcite[124]{Jacobson1985Vol1}
  We say that an element of a field extension is \term[bg=трансцендентен (елемент) (\cite[135]{ГеновМиховскиМоллов1991}), ru=трансцендентный (елемент) (\cite[407]{Винберг2014})]{transcendental} if it is not \hyperref[def:algebraic_element]{algebraic}.
\end{definition}

\begin{theorem}[Lindemann-Weierstrass theorem]\label{thm:lindemann_weierstrass}\mcite[277]{Jacobson1985Vol1}
  Let \( u_1, \ldots, u_n \) be \hyperref[def:complex_numbers]{complex numbers} that are \hyperref[def:algebraic_element]{algebraic} over \( \BbbQ \). If they are \hyperref[def:linear_dependence]{linearly independent} over \( \BbbQ \), their complex exponentials \( e^{u_1}, \ldots, e^{u_n} \) are \hyperref[def:algebraic_dependence]{algebraically independent} over \( \BbbQ \).
\end{theorem}

\begin{corollary}\label{thm:eulers_constant_is_transcendental}
  \hyperref[def:exponential_function]{Euler's constant} \( e \) is \hyperref[def:transcendental_element]{transcendental} over \( \BbbQ \).
\end{corollary}
\begin{proof}
  Since \( 1 \) is by itself linearly independent over \( \BbbQ \), \fullref{thm:lindemann_weierstrass} implies that \( e \) is by algebraically independent over \( \BbbQ \). Hence, there exists no rational polynomial whose root is \( e \), and thus \( e \) satisfies \fullref{def:transcendental_element}.
\end{proof}

\begin{corollary}\label{thm:pi_is_transcendental}\mcite[454]{Knapp2016BasicAlgebra}
  The number \hyperref[def:pi]{\( \pi \)} is \hyperref[def:transcendental_element]{transcendental} over \( \BbbQ \).
\end{corollary}
\begin{proof}
  Suppose that \( \pi \) is algebraic over \( \BbbQ \).

  The complex unit \( i \) is algebraic by construction --- it has minimal polynomial \( X^2 + 1 \). Then \( i\pi \) is also algebraic because it belongs to \( \BbbQ[\pi][i] \), which must be finite-dimensional.

  Furthermore, \( i\pi \) is linearly independent over \( \BbbQ \) --- \fullref{thm:basis_implies_torsion_free} implies that vector spaces are torsion-free, so \fullref{thm:lindemann_weierstrass} implies that \( e^{i\pi} \) is transcendental. But \( e^{i\pi} \) equals \( -1 \), which is a root of the integer polynomial \( X + 1 \).

  The obtained contradiction shows that our initial assumption of \( \pi \) being algebraic over \( \BbbQ \) is false, that is, \( \pi \) is transcendental.
\end{proof}

\begin{example}\label{ex:polynomials_over_pi}
  \Fullref{thm:pi_is_transcendental} implies that the polynomials \( \BbbQ[X] \) can be embedded into \( \BbbR \) via \( \Phi_\pi: \BbbQ[X] \to \BbbR \). We can thus identify a rational polynomial
  \begin{equation*}
    p(X) = \sum_{i=0}^n a_k X^k
  \end{equation*}
  with the number
  \begin{equation*}
    p(\pi) = \sum_{i=0}^n a_k \pi^k.
  \end{equation*}
\end{example}

\paragraph{Algebraic extensions}

\begin{definition}\label{def:algebraic_extension}\mcite[216]{Jacobson1985Vol1}
  We say that a field extension is \term[bg=алгебрично разширение (\cite[201]{ГеновМиховскиМоллов1991}), ru=алгебраическое расширение (\cite[408]{Винберг2014})]{algebraic} if every element is \hyperref[def:algebraic_element]{algebraic} over the base field.
\end{definition}

\begin{proposition}\label{thm:field_is_algebraic_over_itself}
  Every field is an \hyperref[def:algebraic_extension]{algebraic extension} of itself.
\end{proposition}
\begin{proof}
  Every field element \( a \) is a root of the polynomial \( X - a \).
\end{proof}

\begin{proposition}\label{thm:finite_field_extensions_are_algebraic}
  Every \hyperref[def:field_extension_degree]{finite field extension} is \hyperref[def:algebraic_extension]{algebraic}.
\end{proposition}
\begin{proof}
  If \( \BbbK \) be a field extension of \( \Bbbk \), for every element \( u \), \( \Bbbk(u) \) is also a finite extension, and thus every element is algebraic.
\end{proof}

\paragraph{Simple extensions}

\begin{definition}\label{def:field_adjunction}\mcite[213]{Jacobson1985Vol1}
  Let \( \BbbK \) be an \hyperref[def:field/submodel]{extension} of the \hyperref[def:field]{field} \( \Bbbk \) and let \( A \) be a subset of \( \Bbbk \). Denote by \( \Bbbk(A) \) the (unique) smallest extension of \( \Bbbk \) that contains all elements of \( A \). We say that \( \Bbbk(A) \) is generated by \term{adjoining} \( A \).

  As with \hyperref[def:semiring_adjunction]{(semi)ring adjunctions}, in case \( A \) is finite, we can list the individual elements as \( \Bbbk(a_1, \ldots, a_n) \).
\end{definition}
\begin{defproof}
  We must show the existence of \( \Bbbk(A) \).

  The intersection of any family of subfields of \( \BbbK \) is clearly also a subfield. \Fullref{thm:closure_operator_from_set_semilattice} implies that the intersection of all subfields containing an arbitrary subset of \( \BbbK \) is itself a subfield containing that subset.

  \Fullref{thm:closure_operator_minimality} then implies that the (unique) smallest subfield containing \( \Bbbk \cup A \) is the intersection of all subfields containing \( \Bbbk \cup A \). Since \( \BbbK \) is a subfield of itself, this intersection is nonempty, and \( \Bbbk(A) \) is well-defined.
\end{defproof}

\begin{proposition}\label{thm:field_adjunction_tower}
  Let \( \BbbK \) be an extension of \( \Bbbk \) and let \( A \) and \( B \) be subsets of \( \BbbK \). For the corresponding \hyperref[def:field_adjunction]{field adjunctions} we have
  \begin{equation*}
    \Bbbk(A \cup B) = \Bbbk(A)(B).
  \end{equation*}
\end{proposition}
\begin{comments}
  \item This also holds for \hyperref[def:semiring_adjunction]{(semi)ring adjunctions} --- see \fullref{thm:semiring_adjunction_tower}.
\end{comments}
\begin{proof}
  Clearly \( \Bbbk(A) \subseteq \Bbbk(A \cup B) \) and \( B \subseteq \Bbbk(A \cup B) \), hence \( \Bbbk(A)(B) \subseteq \Bbbk(A \cup B) \).

  Conversely, since \( \Bbbk(A)(B) \) contains \( A \cup B \), it follows that \( \Bbbk(A \cup B) \subseteq \Bbbk(A)(B) \).
\end{proof}

\begin{definition}\label{def:simple_field_extension}\mcite[214]{Jacobson1985Vol1}
  We say that the field \( \BbbK \) is a \term[bg=просто разширение (\cite[def. VI.3]{ГеновМиховскиМоллов1991}), ru=простое расширение (\cite[409]{Винберг2014})]{simple extension} of \( \Bbbk \) if \( \BbbK \) is obtained by \hyperref[def:field_adjunction]{adjoining} a single element to \( \Bbbk \). We call the element itself a \term{primitive element} of the extension.
\end{definition}

\begin{proposition}\label{thm:simple_field_extension_characterization}
  Let \( \Bbbk(u) \) be a \hyperref[def:simple_field_extension]{simple field extension}.
  \begin{thmenum}
    \thmitem{thm:simple_field_extension_characterization/algebraic} \( u \) is \hyperref[def:algebraic_element]{algebraic} over \( \Bbbk \) if and only if \( \Bbbk(u) \) equals the ring \( \Bbbk[u] \), obtained by \hyperref[def:semiring_adjunction]{adjoining} \( u \) to \( \Bbbk \).

    \thmitem{thm:simple_field_extension_characterization/transcendental} \( u \) is \hyperref[def:transcendental_element]{transcendental} over \( \Bbbk \) if and only if \( \Bbbk(u) \) is isomorphic to the \hyperref[def:rational_function_field]{rational function field} \( \Bbbk(X) \).
  \end{thmenum}
\end{proposition}
\begin{proof}
  \SubProofOf{thm:simple_field_extension_characterization/algebraic} One of the equivalent definitions in \fullref{def:algebraic_element} for \( u \) being algebraic is \fullref{def:algebraic_element/field}, which requires \( \Bbbk[u] \) to be a field. Since \( \Bbbk(u) \) is by definition the smallest extension of \( \Bbbk \) containing \( u \), we conclude that \( \Bbbk(u) \) equals \( \Bbbk[u] \), the smallest ring containing \( u \).

  \SubProofOf{thm:simple_field_extension_characterization/transcendental}

  \SufficiencySubProof* Suppose that \( u \) is transcendental.

  Consider the map
  \begin{equation*}
    \begin{aligned}
      &\Psi: \Bbbk(X) \to \Bbbk(u), \\
      &\Psi\parens*{ \frac p q } \coloneqq \frac {\Phi_u(p)} {\Phi_u(q)}.
    \end{aligned}
  \end{equation*}

  It is well-defined because \( u \) is, by definition, not he root of any nonzero polynomial, and hence \( \Phi_u(q) \) is always nonzero. We will show that \( \Psi \) is an isomorphism.

  It is injective --- if \( \Psi(p / q) = \Psi(f / g) \), then
  \begin{equation*}
    \Phi_u(pg) = \Phi_u(p) \Phi_u(g) = \Phi_u(f) \Phi_u(q) = \Phi_u(fq).
  \end{equation*}

  Since \( u \) is transcendental, \( \Phi_u: \Bbbk[X] \to \Bbbk[u] \) is injective, hence \( p(X) g(X) = f(X) q(X) \), and thus
  \begin{equation*}
    \frac {p(X)} {q(X)} = \frac {f(X)} {g(X)}.
  \end{equation*}

  The map \( \Psi \) is also surjective. The image of \( \Psi \) is a field extension of \( \Bbbk \) containing \( u \). Since \( \Bbbk(u) \) is the minimal such extension, it follows that the image of \( \Psi \) is \( \Bbbk(u) \).

  \NecessitySubProof* Suppose that \( \Bbbk[u] \) is isomorphic to the rational function field \( \Bbbk(X) \). Then \( \Bbbk[X] \) is infinite-dimensional as a vector space over \( \Bbbk \), hence \( u \) is transcendental because it satisfies the negation of \fullref{def:algebraic_element/dimensions}.
\end{proof}

\begin{definition}\label{def:field_extension_degree}\mcite[def. VII.1.1]{Aluffi2009}
  We define the \term[ru=степень (def. \cite[9.5.1]{Винберг2014})]{degree} of the \hyperref[def:field/submodel]{field extension} \( \BbbK \) of \( \Bbbk \) as the \hyperref[thm:vector_space_dimension]{vector space dimension} of \( \BbbK \) over \( \Bbbk \).

  We use \enquote{finite extension} rather than \enquote{extension of finite degree}.
\end{definition}

\begin{proposition}\label{thm:simple_extension_dimension}
  The \hyperref[def:simple_field_extension]{simple field extension} \( \Bbbk(u) \) is \hyperref[def:field_extension_degree]{finite} if and only if \( u \) is \hyperref[def:algebraic_element]{algebraic}, in which case the degree of \( \Bbbk(u) \) is the degree of the minimal polynomial.
\end{proposition}
\begin{proof}
  \SufficiencySubProof Suppose that \( \Bbbk(u) \) is finite-dimensional. Then \fullref{thm:simple_field_extension_characterization/transcendental} implies that \( u \) cannot be transcendental because \( \Bbbk(X) \) is infinite-dimensional. It remains for \( u \) to be algebraic.

  \NecessitySubProof Suppose that \( u \) is algebraic over \( \Bbbk \).

  \Fullref{thm:simple_field_extension_characterization/algebraic} implies that \( \Bbbk(u) \) equals \( \Bbbk[u] \), and the latter is by definition isomorphic to \( \Bbbk[X] / \braket{ m(X) } \), where \( m(X) \) is the zero polynomial.

  \Fullref{thm:polynomial_quotient_module_dimension} implies that the dimension of \( \Bbbk[X] \) is the degree of \( m(X) \).
\end{proof}

\begin{proposition}\label{thm:finite_adjunction_finite_extension}
  The \hyperref[def:field_adjunction]{adjunction} \( \Bbbk(u_1, \ldots, u_n) \) of finitely many \hyperref[def:algebraic_element]{algebraic} elements is a \hyperref[def:field_extension_degree]{finite extension} of \( \Bbbk \).
\end{proposition}
\begin{proof}
  We will use induction on \( n \). The base case \( n = 0 \) is vacuous. Furthermore, if \( \Bbbk(u_1, \ldots, u_k) \) is a finite extension of \( \Bbbk \), then \fullref{thm:simple_extension_dimension} implies that \( \Bbbk(u_1, \ldots, u_k)(u_{k+1}) \) is a finite extension of \( \Bbbk(u_1, \ldots, u_k) \) and, by \fullref{thm:lagranges_subgroup_theorem}, also a finite extension of \( \Bbbk \).
\end{proof}

\paragraph{Splitting fields}

\begin{definition}\label{def:splitting_field}\mcite[def. VII.4.1]{Aluffi2009}
  Fix a field \( \Bbbk \). We say that a polynomial \( f(X) \) over \( \Bbbk \) \term{splits into linear factors} in the extension \( \BbbK \) if there exist elements \( u_1, \ldots, u_n \) in \( \BbbK \) such that
  \begin{equation*}
    f(X) = a \cdot \prod_{k=1}^n (X - u_k).
  \end{equation*}
  where \( a \) is invertible in \( \Bbbk \).

  We say that \( \BbbK \) is a \term[ru=поле разложения (def. \cite[9.5.2]{Винберг2014})]{splitting field} of \( p(X) \) if it splits into linear factors in \( \BbbK \) and, furthermore,
  \begin{equation*}
    \BbbK \cong \Bbbk(u_1, \ldots, u_n).
  \end{equation*}
\end{definition}
\begin{comments}
  \item \Fullref{thm:splitting_field_existence} implies that splitting fields always exist and are unique up to an isomorphism.
\end{comments}

\begin{lemma}\label{thm:splitting_field_uniqueness_step}
  Let \( \varphi: \Bbbk \to \Bbbl \) be an isomorphism of fields. Let \( u \) be an \hyperref[def:algebraic_element]{algebraic element} over \( \Bbbk \) with \hyperref[def:algebraic_element_minimal_polynomial]{minimal polynomial}
  \begin{equation*}
     m(X) = \sum_{k=1}^n a_k X^k.
  \end{equation*}

  Consider the following polynomial over \( \Bbbl \):
  \begin{equation*}
    \widehat{m}(X) \coloneqq \sum_{k=1}^n \varphi(a_k) X^k.
  \end{equation*}

  Let \( \BbbL \) be an extension of \( \Bbbl \) in which \( \widehat{m}(X) \) has a root, say \( v \). Then there exists a field isomorphism \( \phi: \Bbbk(u) \to \Bbbk(v) \) extending \( \varphi \) and sending \( u \) to \( v \).
\end{lemma}
\begin{proof}
  Note that the map \( p(X) \mapsto \widehat{p}(X) \) is an isomorphism from the polynomial algebra \( \Bbbk[X] \) to \( \Bbbl[X] \). It induces the following isomorphism of fields:
  \begin{equation*}
    \Bbbk(u) = \Bbbk[u] \cong \Bbbk[X] / \braket{ m(X) } \cong \Bbbl[X] / \braket{ \widehat{m}(X) } \cong \Bbbl[v] = \Bbbl(v).
  \end{equation*}
\end{proof}

\begin{proposition}\label{thm:splitting_field_existence}
  Every polynomial of positive degree over a field has a unique up to an isomorphism \hyperref[def:splitting_field]{splitting field}.
\end{proposition}
\begin{comments}
  \item Note that, unlike for \hyperref[rem:universal_mapping_property]{universal mapping properties}, we do not claim that this isomorphism is unique.
\end{comments}
\begin{proof}
  \ExistenceSubProof\mcite[thm. 4.3]{Jacobson1985Vol1} Fix a field \( \Bbbk \) and consider the polynomial \( p(X) \) of positive degree \( n \). Suppose that \( p(X) \) has \( m \) irreducible factors over \( \Bbbk \).

  Clearly \( m \leq n \). We will use induction on \( s = n - m \) to show that \( p(X) \) has a splitting field.

  The base case \( n = m \) is trivial because every factor is then linear in \( \Bbbk \), hence it is its own splitting field.

  Now suppose that splitting fields exist when the difference is less than \( s \) and suppose that \( n - m = s \). Then there exists a nonlinear irreducible factor, say \( f(X) \). Let
  \begin{equation*}
    f(X) = \sum_{i=1}^l a_i X^i.
  \end{equation*}

  \Fullref{thm:quotient_by_irreducible_polynomial} implies that \( \BbbL \coloneqq \Bbbk[X] / \braket{ f(X) } \) is a field. Denote by \( \pi: \Bbbk[X] \to \BbbL \) the canonical projection.

  Consider the evaluation homomorphism \( \Phi_{\pi(X)}: \Bbbk[X] \to \BbbL \) sending \( X \) to \( \pi(X) \). We have
  \begin{equation*}
    \Phi_{\pi(X)}(f) = \sum_{k=1}^l a_i \pi(X)^i = \pi\parens*{ \sum_{k=1}^l a_i X^i } = \pi(f(X)) = 0_\BbbL.
  \end{equation*}

  Therefore, \( u \coloneqq \pi(X) \) is a root of \( f(X) \) in \( \BbbL \), and the latter splits as \( f(X) = (X - u) q(X) \).

  Thus, in \( \BbbL \), \( p(X) \) has at least \( m + 1 \) irreducible factors. We can now apply the inductive hypothesis to obtain a splitting field for \( p(X) \) over \( \BbbL \), where
  \begin{equation*}
    p(X) = b \cdot \prod_{j=1}^n (X - u_j).
  \end{equation*}

  Since \( X - u \) divides \( p(X) \), we conclude that there exists some index \( k_0 \) such that \( u = u_{k_0} \). Then \( \Bbbk(u_1, \ldots, u_n) \) is a splitting field for \( p(X) \).

  \UniquenessSubProof Suppose that \( \BbbK \) and \( \BbbL \) are both splitting fields for the polynomial \( p(X) \) of degree \( n \) over \( \Bbbk \). Suppose that
  \begin{equation*}
    \BbbK = \Bbbk(u_1, \ldots, u_n).
  \end{equation*}

  Via recursion on \( k = 0, \ldots, n \), we can construct a sequence of monomorphisms
  \begin{equation*}
    \varphi_k: \Bbbk(u_1, \ldots, u_k) \to \BbbL
  \end{equation*}
  such that \( \varphi_{k+1} \) extends \( \varphi_k \).

  The base case is trivial: define \( \varphi_0 \) to be the identity on \( \Bbbk \).

  For any monomorphism \( \varphi_k \), \fullref{thm:splitting_field_uniqueness_step} gives us a monomorphism \( \varphi_{k+1} \) via the algebraic element \( u_{k+1} \) over \( \Bbbk(u_1, \ldots, u_k) \). Then \fullref{thm:field_adjunction_tower} implies that
  \begin{equation*}
    \Bbbk(u_1, \ldots, u_k)(u_{k+1})
    =
    \Bbbk(u_1, \ldots, u_k, u_{k+1})
  \end{equation*}

  The last element of the sequence is
  \begin{equation*}
    \varphi_n: \BbbK \to \BbbL.
  \end{equation*}

  Since for every \( k = 1, \ldots, n \), the element \( \varphi_n(u_k) \) is a root of \( p(X) \) in \( \BbbL \), we conclude that the image of \( \varphi_n \) contains \( n \) roots of \( p(X) \), and must thus coincide with \( \BbbL \) itself.

  Thus, \( \varphi_n \) is the desired isomorphism between \( \BbbK \) and \( \BbbL \).
\end{proof}

\begin{proposition}\label{thm:splitting_field_is_finite_extension}
  Every \hyperref[def:splitting_field]{splitting field} is a \hyperref[def:field_extension_degree]{finite extension}.
\end{proposition}
\begin{proof}
  Follows from \fullref{thm:finite_adjunction_finite_extension}.
\end{proof}

\paragraph{Finite fields}

\begin{definition}\label{def:finite_field}\mimprovised
  Unsurprisingly, if a \hyperref[def:field]{field} has finite \hyperref[thm:cardinality_existence]{cardinality}, we call it a \term{finite field}.

  \Fullref{thm:finite_fields/uniqueness} implies that finite fields of the same cardinality are isomorphic, and we will not distinguish between them. Thus, if a finite field has \( q \) elements, we denote it by \( \BbbF_q \).
\end{definition}
\begin{comments}
  \item \Fullref{thm:zp_is_field} implies that, if \( p \) is a prime number, the ring of integers modulo \( p \) is a field. \Fullref{thm:zp_is_field} further classifies finite fields.
\end{comments}

\begin{theorem}[Classification of finite fields]\label{thm:finite_fields}
  \hyperref[def:finite_field]{Finite fields} can be classified as follows:
  \begin{thmenum}
    \thmitem{thm:finite_fields/characteristic} The \hyperref[def:ring_characteristic]{characteristic} of a finite field is a \hyperref[def:prime_number]{prime number}.

    \thmitem{thm:finite_fields/cardinality} The \hyperref[thm:cardinality_existence]{cardinality} of a finite field of characteristic \( p \) is a positive power of \( p \).

    \thmitem{thm:finite_fields/polynomial} Every element of a field with \( q \) elements is a root of the polynomial
    \begin{equation}\label{eq:thm:finite_fields/polynomial}
      X^q - X.
    \end{equation}

    \thmitem{thm:finite_fields/splitting} A field with \( q = p^n \) elements is a \hyperref[def:splitting_field]{splitting field} over \( \BbbZ_p \) for the polynomial \eqref{eq:thm:finite_fields/polynomial}.

    \thmitem{thm:finite_fields/uniqueness} All finite fields having the same cardinality are isomorphic.
  \end{thmenum}
\end{theorem}
\begin{proof}
  \SubProofOf{thm:finite_fields/characteristic} Let \( \BbbK \) be a field with \( q \) elements and let \( n \) be the characteristic of \( \BbbK \).

  Let \( n = mk \). Then
  \begin{equation*}
    0 = n \cdot 1 = mk \cdot 1 = (m \cdot 1) (k \cdot 1).
  \end{equation*}

  Since \( \BbbK \) is a field, it has no zero divisors, and hence either \( m \cdot 1 \) or \( k \cdot 1 \) must be zero. Thus, either \( m \) or \( k \) is the characteristic of \( \BbbK \), and we have assumed that the characteristic is \( n \).

  It follows that \( n \) equals \( m \) or \( k \), that is, \( n \) is a prime number.

  \SubProofOf{thm:finite_fields/cardinality} Let \( \BbbK \) be a field with \( q \) elements and let \( p \) be the characteristic of \( \BbbK \). \Fullref{thm:finite_fields/characteristic} implies that \( p \) is prime.

  By definition of characteristic, \( \BbbZ_p \) is a subring of \( \BbbK \). \Fullref{thm:zp_is_field} implies that \( \BbbZ_p \) is a field.

  Then \( \BbbK \) is a vector space over \( \BbbZ_p \). \Fullref{thm:vector_space_basis_existence} shows that \( \BbbK \) has a basis, and since \( \BbbK \) has finitely many elements, this basis must be finite.

  Therefore, if \( n \) is the dimension of \( \BbbK \) over \( \BbbZ_p \), then \( q = p^n \).

  \SubProofOf{thm:finite_fields/polynomial} Let \( \BbbK \) be a field with \( q = p^n \) elements with characteristic \( p \).

  The multiplicative group of \( \BbbK \) has order \( q - 1 \). \Fullref{thm:lagranges_subgroup_theorem} implies that the order of a non-zero element \( a \) of \( \BbbK \) divides \( q - 1 \), hence \( a^{q - 1} = 1 \). We also have \( 0^q = 0 \). Therefore, for every element \( a \) of \( \BbbF_q \), we have \( a^q = a \).

  Therefore, every element of \( \BbbK \) is a root of \( X^q - X \).

  \SubProofOf{thm:finite_fields/splitting} Clearly
  \begin{equation*}
    X^q - X = \prod_{u \in \BbbK} (X - u).
  \end{equation*}

  Clearly adjoining all elements of \( \BbbK \) to \( \BbbZ_p \) will give \( \BbbK \), thus the latter is indeed a splitting field.

  \SubProofOf{thm:finite_fields/uniqueness} Follows from \fullref{thm:finite_fields/splitting} and \fullref{thm:splitting_field_existence}.
\end{proof}

\begin{proposition}\label{thm:functions_over_prime_fields}
  For every \hyperref[def:finite_field]{finite field} \( \BbbF_q \) and every \hyperref[def:polynomial_algebra]{polynomial ring} \( \BbbF_q[X_1, \ldots, X_n] \) in finitely many indeterminates, there exists an \( \BbbF_q \)-\hyperref[def:algebra_over_ring]{algebra} isomorphism
  \begin{equation*}
    \frac {\BbbF_q[X_1, \ldots, X_n]} {\braket{ X_i^q - X_i \given i = 1, \ldots, n }} \cong \fun(\BbbF_q^n, \BbbF_q),
  \end{equation*}
  where \( \fun(\BbbF_q^n, \BbbF_q) \) is the \hyperref[thm:functions_over_algebra]{\( \BbbF_q \)-algebra of all functions} from \( \BbbF
  _q^n \) to \( \BbbF_q \).
\end{proposition}
\begin{comments}
  \item Every coset of polynomials has a unique representative of minimal degree as discussed in \fullref{thm:representatives_in_univariate_polynomial_quotient_set}.
\end{comments}
\begin{proof}
  Consider the \hyperref[rem:substitution_homomorphism]{functional evaluation homomorphism}
  \begin{equation*}
    \Phi: \BbbF_q[X_1, \ldots, X_m] \to \fun(\BbbF_q^m, \BbbF_q).
  \end{equation*}

  By \fullref{thm:finite_field_lagrange_interpolation}, \( \Phi \) is surjective. Then, by \fullref{thm:quotient_structure_universal_property},
  \begin{equation*}
    \BbbF_q[X_1, \ldots, X_n] / \ker \Phi \cong \fun(\BbbF_q^m, \BbbF_q).
  \end{equation*}

  We will now prove that \( \ker \Phi \) equals
  \begin{equation*}
    I \coloneqq \braket{ X_i^q - X_i \given i = 1, \ldots, n }.
  \end{equation*}

  First, let \( e: \mscrX \to \BbbF_q \) be the variable assignment that assigns \( u_1, \ldots, u_n \) to the corresponding indeterminates. By \fullref{thm:finite_fields/polynomial}, every member of \( \BbbF_q \) is a root of \( X_i^q - X_i \). Then, for any indeterminate \( X_i \),
  \begin{equation*}
    \Phi_e(X_i^q - X_i) = u_i^q - u_i = 0 \pmod q.
  \end{equation*}

  Hence, the polynomial function \( \Phi(X_i^q - X_i) \) is the zero constant function. It follows that any linear combination of the polynomials \( X_i^q - X_i \) for \( i = 1, \ldots, n \) is also the zero function. Therefore, \( I \subseteq \ker \Phi \).

  We will prove the converse inclusion via induction on \( n \).

  In the case of a single indeterminate \( X \), for every polynomial \( f(X) \in \ker \Phi \), we know that the entirety of \( \BbbF_q \) are roots of \( f(X) \). By \fullref{thm:def:integral_domain/root_limit}, \( f(X) \) has at most \( q \) roots, counting multiplicities. Hence, \( X - u \) divides \( f(X) \) for every \( u \in \BbbF_q \). We have
  \begin{equation*}
    \underbrace{\prod_{u \in \BbbF_q} (X - u)}_{\mathclap{ X^q - X \T*{by \fullref{thm:finite_fields/splitting}}}} \mid f(X),
  \end{equation*}
  and hence \( f(X) \in \braket{ X^q - X } \).

  We have, up until now, shown that the entire proposition holds for the case of one indeterminate. Suppose that the proposition holds for \( n - 1 \) indeterminates and let \( f \in \BbbF_q[X_1, \ldots, X_n] \) be a nonconstant polynomial such that \( \Phi(f) \) is the zero function. Due to \fullref{thm:def:polynomial_algebra/union}, we can regard \( f \) as a univariate polynomial in \( X_n \)over \( \BbbF_q[X_1, \ldots, X_{n-1}] \). Thus,
  \begin{equation*}
    f(X_1, \ldots, X_n) = \sum_{k =0}^\infty \underbrace{\parens*{ \sum_\gamma a_{(k,\gamma)} X_1^{\gamma_1} X_2^{\gamma_1} \cdots X_{n-1}^{\gamma_{n-1}} }}_{s_k(X_1, \ldots, X_{n-1})} {X_n}^k,
  \end{equation*}
  where \( \gamma \) is a multi-index over the first \( n - 1 \) indeterminates.

  As a polynomial in \( X_n \), \( f \) has \( m \coloneqq (n-1)p \) roots \( s_1, \ldots, s_m \), which are themselves polynomials from \( \BbbF_q[X_1, \ldots, X_{n-1}] \). For some \( c \), we have
  \begin{equation*}
    f(X_1, \ldots, X_n) = c(X_1, \ldots, X_{n-1}) \prod_{j=1}^m (X_n - s_j(X_1, \ldots, X_{n-1}))
  \end{equation*}
  and
  \begin{equation*}
    0 = \Phi(f) = \Phi(c) \cdot \prod_{j=1}^m \parens[\Big]{ \Phi(X_n) - \Phi(s_j) }.
  \end{equation*}

  Since \( \BbbF_q[X_1, \ldots, X_{n-1}] \) is \hyperref[def:entire_semiring]{entire}, we conclude that either \( \Phi(c) \) is the zero function or \( \Phi(X_n) = \Phi(s_j) \) for at least one index \( 1 \leq j \leq m \). The latter is impossible, because \( \Phi(X_n) \) is linearly independent from polynomials in the first \( n - 1 \) variables.

  The inductive hypothesis holds for the polynomial \( c \), and \( \Phi(c) \) being the zero function implies
  \begin{equation*}
    c \in \braket{ X_i^q - X_i \given i = 1, \ldots, n - 1 } \subsetneq I.
  \end{equation*}

  Therefore, \( f \in I \) since \( f \) divides \( c \). We have chosen \( f \) to be an arbitrary member of \( \ker \Phi \), which implies \( \ker \Phi \subseteq I \).

  We have already shown that \( I \subseteq \ker \Phi \). We thus conclude that \( I = \ker \Phi \) and
  \begin{equation*}
    \BbbF_q[X_1, \ldots, X_m] / I \cong \fun(\BbbF_q^m, \BbbF_q).
  \end{equation*}
\end{proof}

\paragraph{Algebraically closed fields}

\begin{definition}\label{def:algebraically_closed_field}
  We say that the field \( \BbbK \) is \term[bg=алгебрически затворено (поле) (\cite[217]{ГеновМиховскиМоллов1991}), ru=алгебраически замкнутое (поле) (\cite[106]{Винберг2014})]{algebraically closed} if any of the equivalent conditions are satisfied:
  \begin{thmenum}
    \thmitem{def:algebraically_closed_field/trivial_algebraic_extensions}\mcite[prop. 9.20(a)]{Knapp2016BasicAlgebra} \( \BbbK \) has no nontrivial \hyperref[def:algebraic_extension]{algebraic extensions}.
    \thmitem{def:algebraically_closed_field/linear_irreducible_polynomials}\mcite[prop. 9.20(b)]{Knapp2016BasicAlgebra} Every irreducible polynomial in \( \BbbK[X] \) is linear.
    \thmitem{def:algebraically_closed_field/at_least_one_root}\mcite[224]{Jacobson1985Vol1} Every nonconstant polynomial in \( \BbbK[X] \) has at least one root in \( \BbbK \).
    \thmitem{def:algebraically_closed_field/factorization} Every polynomial in \( \BbbK[X] \) \hyperref[def:splitting_field]{splits into linear factors}.
    \thmitem{def:algebraically_closed_field/exactly_n_roots} Every polynomial in \( \BbbK[X] \) of degree \( n \) has exactly \( n \) roots in \( \BbbK \), counting the root multiplicities.
  \end{thmenum}
\end{definition}
\begin{defproof}
  \ImplicationSubProof{def:algebraically_closed_field/trivial_algebraic_extensions}{def:algebraically_closed_field/linear_irreducible_polynomials} Suppose that \( \BbbK \) has no nontrivial algebraic extensions. Let \( p(X) \) be an irreducible polynomial over \( \BbbK \) of degree \( n \). We will show that \( p(X) \) is linear.

  \Fullref{thm:quotient_by_irreducible_polynomial} implies that \( \BbbK[X] / \braket{ p(X) } \) is a field extension of \( \BbbK \). By our assumption, it must be isomorphic to \( \BbbK \) itself, that is, has dimension \( 1 \) over \( \BbbK \).

  \Fullref{thm:polynomial_quotient_module_dimension} then implies that \( p(X) \) has degree \( 1 \).

  \ImplicationSubProof{def:algebraically_closed_field/linear_irreducible_polynomials}{def:algebraically_closed_field/at_least_one_root} Suppose that every irreducible polynomial is linear.

  Any nonconstant polynomial \( p(X) \) has an \hyperref[def:irreducible_factorization]{irreducible factorization} and hence at least one irreducible factor. Each irreducible factor has exactly one root, therefore \( p(X) \) also has at least one root.

  \ImplicationSubProof{def:algebraically_closed_field/at_least_one_root}{def:algebraically_closed_field/factorization} Suppose that every nonconstant polynomial has at least one root.

  Let \( u_1 \) be a root of \( p(X) \). Then \( p(X) \) is divisible by \( (X - u_1) \). Using induction on the degree \( n \) of \( p(X) \), we can factor \( p(X) \) into
  \begin{equation*}
    p(X) = a (X - u_1) (X - u_2) \cdots (X - u_n).
  \end{equation*}

  This is the desired factorization.

  \ImplicationSubProof{def:algebraically_closed_field/factorization}{def:algebraically_closed_field/exactly_n_roots} Suppose that every nonconstant polynomial splits into linear factors.

  For a polynomial of degree \( n \), there can be at most \( n \) linear factors, and each one represents a root. Then the polynomial itself has at least \( n \) roots, counting multiplicities. \Fullref{thm:def:integral_domain/root_limit} on the other hand implies that the number of roots is bounded from above \( n \), and is thus exactly \( n \).

  \ImplicationSubProof{def:algebraically_closed_field/exactly_n_roots}{def:algebraically_closed_field/trivial_algebraic_extensions} Suppose that every nonconstant polynomial of degree \( n \) has exactly \( n \) roots in \( \Bbbk \).

  Let \( \BbbK \) be an algebraic extension of \( \Bbbk \). Let \( u \) be an element of \( \BbbK \). It is necessarily algebraic over \( \Bbbk \), and thus has a minimal polynomial \( m(X) \) of degree \( n \).

  If \( n > 1 \), then \( m(X) \) can be further split into linear factors, which in turn implies that \( m(X) \) is not irreducible. The obtained contradiction show that \( m(X) \) itself is linear.

  Since \( u \) is a root of the linear polynomial \( m(X) \), it follows that \( u \) itself belongs to \( \Bbbk \). Generalizing on \( u \), we conclude that \( \BbbK \) belongs to \( \Bbbk \). The converse inclusion holds by assumption, thus \( \BbbK = \Bbbk \).
\end{defproof}

\begin{definition}\label{def:algebraic_closure}\mcite[465]{Jacobson1985Vol2}
  We say that the \hyperref[def:algebraic_extension]{algebraic extension} \( \BbbK \) of \( \Bbbk \) is an \term[ru=алгебраическое замыкание (\cite[412]{Винберг2014})]{algebraic closure} if it is \hyperref[def:algebraically_closed_field]{algebraically closed}.
\end{definition}
\begin{comments}
  \item \Fullref{thm:algebraic_closure_existence} implies that algebraic closures always exist and are unique up to an isomorphism.
\end{comments}

\begin{proposition}\label{thm:algebraic_closure_existence}
  Every \hyperref[def:field]{field} has a unique up to a isomorphism \hyperref[def:algebraic_extension]{algebraic extension} that is \hyperref[def:algebraically_closed_field]{algebraically closed}.
\end{proposition}
\begin{comments}
  \item Like in \fullref{thm:splitting_field_existence}, we do not claim that this isomorphism is unique.
\end{comments}
\begin{proof}
  \ExistenceSubProof\mcite{Jelonek1991} Let \( \Bbbk \) be a field. Let \( U \) be the \hyperref[def:grothendieck_universe]{Grothendieck universe} containing \( \Bbbk \). Let \( \mscrL \) be the family of all algebraic extensions of \( \Bbbk \) contained in \( U \), ordered by set inclusion. \( \mscrL \) is nonempty since any nonconstant polynomial induces a \hyperref[def:splitting_field]{splitting field}, as shown in \fullref{thm:splitting_field_existence}, and this field is algebraic.

  The supremum of an ascending chain in \( \mscrL \) is their union, which is again an algebraic extension --- every element comes from an algebraic extension, where it is the root of a polynomial over \( \Bbbk \). \Fullref{thm:zorns_lemma} then implies that \( \mscrL \) has a maximal element \( \BbbK \).

  Then \( \BbbK \) is algebraically closed because it has no nontrivial algebraic field extensions. It is thus the desired algebraic closure.

  \UniquenessSubProof Let \( \BbbK \) and \( \BbbL \) be two algebraically closed algebraic extensions of \( \Bbbk \).

  Let \( \mscrF \) be the family of all field homomorphisms \( \varphi: \Bbbl \to \BbbK \), where \( \Bbbl \) is a subfield of \( \BbbL \), ordered such that \( \varphi_1 \leq \varphi_2 \) if \( \varphi_2 \) is an extension of \( \varphi_1 \). The set is nonempty because of the canonical inclusion \( \iota: \Bbbk \to \BbbK \).

  The supremum of an ascending chain in \( \mscrF \) if the (set-theoretic) union of the homomorphisms. \Fullref{thm:zorns_lemma} then implies that \( \mscrF \) has a maximal element \( \Phi \). The domain of \( \Phi \) is \( \BbbL \), because otherwise \( \Phi \) would not be maximal.

  Therefore, we have a field homomorphism \( \Phi: \BbbL \to \BbbK \). It is also surjective because, if some element \( y \) of \( \BbbK \) has an empty preimage under \( \Phi \), and if \( y \) is a root of \( p(X) \in \Bbbk \), then \( \BbbL \) has a nontrivial splitting field that contains the roots of \( p(X) \). But the latter contradicts the assumption that \( \BbbL \) is algebraically closed, because splitting fields are algebraic, as shown in \fullref{thm:splitting_field_is_finite_extension} and \fullref{thm:finite_field_extensions_are_algebraic}.

  The obtained contradiction shows that \( \Phi \) must be surjective. It is also injective as a consequence of \fullref{thm:def:ring/simple_ring_homomorphism_is_injective}.

  Thus, \( \Phi \) is the desired isomorphism.
\end{proof}

\begin{proposition}\label{thm:no_finite_extensions_of_closed_fields}
  An \hyperref[def:algebraically_closed_field]{algebraically closed field} has no nontrivial finite field extensions.
\end{proposition}
\begin{proof}
  Follows from \fullref{thm:finite_field_extensions_are_algebraic} applied to \fullref{def:algebraically_closed_field/trivial_algebraic_extensions}.
\end{proof}


  \chapter{Linear algebra}\label{ch:linear_algebra}

Linear algebra is a branch of mathematics that is both very accessible and enormously useful. It studies \hyperref[def:vector_space]{vector spaces}, mostly over \hyperref[def:real_numbers]{real} or \hyperref[def:complex_numbers]{complex} numbers, and \hyperref[def:linear_function]{linear maps} between them. For \hyperref[thm:vector_space_dimension]{finite-dimensional} vector spaces, this reduces to studying \hyperref[def:array/matrix]{matrices}, which also leads to a very rich computational theory.

We have discussed the basics of vector spaces in \fullref{sec:modules} in the context of general \hyperref[def:module]{modules over rings}. The key takeaways are:
\begin{itemize}
  \item Vector spaces, defined incrementally in \cref{def:semimodule}, \cref{def:module} and \cref{def:vector_space}, with some properties listed in \cref{thm:def:vector_space}.
  \item Linear combinations, defined in \cref{def:free_semimodule} and characterized by \fullref{thm:free_semimodule_universal_property}, with important remarks in \cref{def:linear_combination}.
  \item Linear spans, defined in \cref{def:semimodule/submodel} and characterized via \cref{thm:span_via_linear_combinations}.
  \item Linear maps, defined in \cref{def:linear_function}, and multilinear maps, defined in \cref{def:multilinear_function}.
  \item Quotient spaces, discussed in \cref{def:module/quotient} for general modules.
  \item \Fullref{thm:quotient_structure_universal_property} and \Fullref{thm:lattice_theorem_for_submodules}.
  \item Linear (in)dependence, defined in \cref{def:linear_dependence} with some properties listed in \cref{thm:def:linear_dependence}.
  \item Hamel bases, defined in \cref{def:hamel_basis} and \cref{def:ordered_hamel_basis}, with some properties listed in \cref{thm:def:hamel_basis}.
  \item Basis decomposition defined in \cref{def:basis_decomposition}.
  \item Coordinate spaces, defined in \cref{def:coordinate_space}.
  \item Vector space dimensions, whose existence and uniqueness is shown in \cref{thm:vector_space_dimension}.
  \item \Fullref{thm:rank_nullity_theorem}.
\end{itemize}

  \subsection{Matrices over rings}\label{subsec:matrices_over_rings}

We will define and prove some fundamental notions about matrices. We will start with matrices over plain sets and end up with matrices over nontrivial noetherian commutative rings. This is about as general as we want to go without the underlying ring being a field. The main benefit is being able to work with the \hyperref[def:integers]{ring of integers} or more general semirings, like the \hyperref[def:tropical_semiring]{tropical semirings}.

\begin{definition}\label{def:array}\mimprovised
  Let \( S \) be any nonempty \hyperref[def:set]{set} and \( n_1, \ldots, n_k \) be \hyperref[def:integer_signum]{positive integers}. An \term[ru=массив (\cite[sec. 40.1]{Тыртышников2007})]{array} of shape \( n_1 \times \cdots \times n_k \) is a \hyperref[def:function]{function} with signature
  \begin{equation*}
    A: \set{ 1, 2, \ldots, n_1 } \times \ldots \times \set{ 1, 2, \ldots, n_k } \to S.
  \end{equation*}

  \enquote{Multi-dimensional array} is also used as a term, but we will avoid it because the terminology conflicts with \hyperref[thm:vector_space_dimension]{vector space dimensions}.

  We can regard \enquote{\( n_1 \times \cdots \times n_k \)} simultaneously as a convenient notation and as a \hyperref[def:cartesian_product]{Cartesian product} of finite \hyperref[def:ordinal]{ordinals} (modulo the fact that finite ordinals are zero-based).

  In particular:
  \begin{thmenum}
    \thmitem{def:array/matrix} A two-dimensional array of shape \( m \times n \) is usually called a \term[ru=матрица (\cite[\textnumero 1.1.2]{ПетровЗяпков2010}), ru=матрица (\cite[sec. 1.1]{Тыртышников2007})]{matrix}. Let \( A \) be an \( m \times n \)-matrix. We will denote \( A \) as
    \begin{equation*}
      A = \seq{ a_{i,j} }_{i,j=1}^{m,n}
    \end{equation*}
    or graphically as the table
    \begin{equation*}
      \begin{pmatrix}
        a_{1,1} & a_{1,2} & \cdots & a_{1,n} \\
        a_{2,1} & a_{2,2} & \cdots & a_{2,n} \\
        \vdots  & \vdots  & \ddots & \vdots  \\
        a_{m,1} & a_{m,2} & \cdots & a_{m,n}
      \end{pmatrix}.
    \end{equation*}

    \thmitem{def:array/square_matrix} A \term{square matrix} of order \( n \) is simply an \( n \times n \) matrix.

    \thmitem{def:array/column_vector} A \term{column vector} of dimension \( m \) is simply a \( m \times 1 \) matrix
    \begin{equation*}
      \begin{pmatrix}
        a_{1,1} \\
        \vdots  \\
        a_{m,1}
      \end{pmatrix}.
    \end{equation*}

    When \( S \) is a \hyperref[def:semiring]{semiring} \( R \), we often identify the set of all \( m \)-dimensional column vectors with the \hyperref[def:sequence_space]{coordinate space} \( R^m \).

    \thmitem{def:array/row_vector} A \term{row vector} of dimension \( n \) is simply an \( 1 \times n \) matrix
    \begin{equation*}
      \begin{pmatrix}
        a_{1,1} & \cdots & a_{1,n}
      \end{pmatrix}.
    \end{equation*}
  \end{thmenum}
\end{definition}

\begin{remark}\label{rem:vector_etymology}
  In practice, the terms \enquote{vector}, \enquote{tuple} and \enquote{finite sequence} are used interchangeably. Formally, the concepts differ slightly:

  \begin{itemize}
    \item \enquote{Vector} refers to an element of a \hyperref[def:vector_space]{vector space} or, more generally, a \hyperref[def:semimodule]{semimodule}. \hyperref[def:array/column_vector]{Column vectors} and \hyperref[def:array/row_vector]{row vectors} are important special cases.

    \item Tuples are defined and discussed in \fullref{def:cartesian_product/tuple}. Tuples are technically \hyperref[def:cartesian_product/indexed_family]{indexed families} and the latter are defined without reference to functions, which we use to define both arrays and sequences.

    \item Sequences are defined and discussed in \fullref{def:sequence}. Formally, a finite sequence of length \( n \) is the same as an array of shape \( n \).
  \end{itemize}
\end{remark}

\begin{definition}\label{def:block_matrix}
  A \term{block matrix} is a \enquote{matrix of matrices}. That is, a matrix of the form
  \begin{equation*}
    \begin{pmatrix}
      A_{1,1} & A_{1,2} & \cdots & A_{1,n} \\
      A_{2,1} & A_{2,2} & \cdots & A_{2,n} \\
      \vdots  & \vdots  & \ddots & \vdots  \\
      A_{m,1} & A_{m,2} & \cdots & A_{m,n}
    \end{pmatrix},
  \end{equation*}
  where all \( A_{i,j} \) are matrices of compatible dimensions.

  We can write the block matrix
  \begin{equation*}
    \begin{pmatrix}
      A      & \cdots & B      \\
      \vdots & \ddots & \vdots \\
      C      & \cdots & D
    \end{pmatrix}
  \end{equation*}
  as
  \begin{equation*}
    \parens*
      {
        \begin{array}{ccc|c|ccc}
          a_{1,1}   & \cdots & a_{1,n_A}   & \cdots & b_{1,1}   & \cdots & b_{1,n_B} \\
          \vdots    & \ddots & \vdots      & \cdots & \vdots    & \ddots & \vdots \\
          a_{m_A,1} & \cdots & a_{m_A,n_A} & \cdots & b_{m_B,1} & \cdots & b_{m_B,n_B} \\
          \hline
          \vdots    & \vdots & \vdots      & \ddots & \vdots    & \vdots & \vdots \\
          \hline
          c_{1,1}   & \cdots & c_{1,n_C}   & \cdots & d_{1,1}   & \cdots & d_{1,n_D} \\
          \vdots    & \ddots & \vdots      & \cdots & \vdots    & \ddots & \vdots \\
          c_{m_C,1} & \cdots & c_{m_C,n_C} & \cdots & d_{m_D,1} & \cdots & d_{m_D,n_D} \\
        \end{array}
      }.
  \end{equation*}

  Given any matrix \( A = \seq{ a_{i,j} }_{i,j=1}^{n,m} \), we can represent it via its block matrix of rows
  \begin{equation*}
    \parens*
      {
        \begin{array}{c}
          a_{1,\anon*} \\
          \hline
          a_{2,\anon*} \\
          \hline
          \vdots \\
          \hline
          a_{n,\anon*}
        \end{array}
      },
  \end{equation*}
  and its block matrix of columns
  \begin{equation*}
    \parens*
      {
        \begin{array}{c|c|c|c}
          a_{\anon*,1} & a_{\anon*,2} & \cdots & a_{\anon*,m}
        \end{array}
      }
  \end{equation*}
\end{definition}

\begin{definition}\label{def:matrix_diagonal}
  The \term{main diagonal} of the matrix \( A = \seq{ a_{i,j} }_{i,j=1}^{m,n} \) is the sequence \( a_{1,1}, \ldots, a_{i,i}, \ldots, a_{k,k} \), where \( k \coloneqq \min\set{ m, n } \). The \term{antidiagonal} is instead \( a_{1,k}, \ldots, a_{i,k-i}, \ldots, a_{k,n-k} \). These can be visualized as follows:
  \begin{equation*}
    \begin{pmatrix}
      \fbox{\( a_{1,1} \)} & a_{1,2}                & \cdots & a_{1,k-1}                & \fbox{\( a_{k,k} \)} & \cdots \\
      a_{2,1}              & \fbox{\( a_{2,2} \)}   &        & \fbox{\( a_{2,k-1} \)}   & a_{2,k}              &        \\
      \vdots               &                        & \ddots &                          & \vdots               &        \\
      a_{k-1,1}            & \fbox{\( a_{k-1,2} \)} &        & \fbox{\( a_{k-1,k-1} \)} & a_{k-1,k}            & \cdots \\
      \fbox{\( a_{k,1} \)} & a_{k,2}                & \cdots & a_{k,k-1}                & \fbox{\( a_{k,k} \)} &        \\
      \vdots               &                        &        & \vdots                   &                      & \ddots
    \end{pmatrix}
  \end{equation*}

  Over a \hyperref[def:semiring]{semiring}, we say that a square matrix \term{diagonal} if all entries outside the main diagonal are zero. For brevity, we write
  \begin{equation*}
    \op{diag}(a_1, \ldots, a_n)
    \coloneqq
    \begin{pmatrix}
      a_1    & 0      & \cdots & 0      \\
      0      & a_2    & \cdots & 0      \\
      \vdots & \vdots & \ddots & \vdots \\
      0      & 0      & \cdots & a_n
    \end{pmatrix}
  \end{equation*}

  The notation \( \op{diag}(A) \) is also used to denote the sequence of diagonal entries of \( A \).
\end{definition}

\begin{proposition}\label{thm:matrix_algebra}
  Denote by \( R^{m \times n} \) the set of \( m \times n \) \hyperref[def:array/matrix]{matrices} over the \hyperref[def:semiring]{semiring} \( R \). We define three operations on matrices:
  \begin{thmenum}
    \thmitem{thm:matrix_algebra/addition} We define \term{matrix addition} \( +: R^{m \times n} \times R^{m \times n} \to R^{m \times n} \) componentwise as
    \begin{equation*}
      \begin{pmatrix}
        a_{1,1} & \cdots & a_{1,n} \\
        \vdots  & \ddots & \vdots  \\
        a_{m,1} & \cdots & a_{m,n}
      \end{pmatrix}
      +
      \begin{pmatrix}
        b_{1,1} & \cdots & b_{1,n} \\
        \vdots  & \ddots & \vdots  \\
        b_{m,1} & \cdots & b_{m,n}
      \end{pmatrix}
      \coloneqq
      \begin{pmatrix}
        a_{1,1} + b_{1,1} & \cdots & a_{1,n} + b_{1,n} \\
        \vdots            & \ddots & \vdots            \\
        a_{m,1} + b_{m,1} & \cdots & a_{m,n} + b_{m,n}
      \end{pmatrix}.
    \end{equation*}

    With addition, \( R^{m \times n} \) becomes an \hyperref[def:binary_operation/commutative]{commutative} \hyperref[def:monoid]{monoid} with neutral element the \term{zero matrix}
    \begin{equation}\label{eq:thm:matrix_algebra/matrix_multiplication/zero}
      \begin{pmatrix}
        0       & 0      & \cdots & 0      \\
        0       & 0      & \cdots & 0      \\
        \vdots  & \cdots & \ddots & \vdots \\
        0       &        & \cdots & 0
      \end{pmatrix}.
    \end{equation}

    \thmitem{thm:matrix_algebra/scalar_multiplication} We define \term{scalar multiplication} \( \cdot: R \times R^{m \times n} \to R^{m \times n} \) as
    \begin{equation*}
       \lambda \cdot \begin{pmatrix}
        a_{1,1} & \cdots & a_{1,n} \\
        \vdots  & \ddots & \vdots  \\
        a_{m,1} & \cdots & a_{m,n}
      \end{pmatrix}
      \coloneqq
      \begin{pmatrix}
        \lambda a_{1,1} & \cdots & \lambda a_{1,n} \\
        \vdots          & \ddots & \vdots          \\
        \lambda a_{m,1} & \cdots & \lambda a_{m,n}
      \end{pmatrix}.
    \end{equation*}

    Under \hyperref[thm:matrix_algebra/addition]{addition} and \hyperref[thm:matrix_algebra/scalar_multiplication]{scalar multiplication}, \( R^{m \times n} \) becomes an \( R \)-\hyperref[def:semimodule]{semimodule}.

    \thmitem{thm:matrix_algebra/matrix_multiplication} We define \term{matrix multiplication} in two steps. The definition is justified by \fullref{thm:matrix_and_linear_function_algebras}. First, if \( \seq{ a_{1,j} }_{j=1}^n \) is a \hyperref[def:array/row_vector]{row vector} and \( \seq{ b_{i,1} }_{i=1}^m \) is a \hyperref[def:array/column_vector]{column vector}, we define their \term{inner product} as
    \begin{equation*}
      a \cdot b \coloneqq \sum_{i=1}^n a_i b_i.
    \end{equation*}

    We can now define matrix multiplication \( \cdot: R^{m \times k} \times R^{k \times n} \to R^{m \times n} \) as
    \begin{equation*}
     \parens*
       {
         \begin{array}{c}
            a_{1,-} \\
            \hline
            a_{2,-} \\
            \hline
            \vdots \\
            \hline
            a_{m,-}
          \end{array}
        }
      \cdot
      \parens*
        {
          \begin{array}{c|c|c|c}
            b_{-,1} & b_{-,2} & \cdots & b_{-,n}
          \end{array}
        }
      \coloneqq
      \begin{pmatrix}
        a_{1,-} \cdot b_{-,1} & a_{1,-} \cdot b_{-,2} & \vdots & a_{1,-} \cdot b_{-,n} \\
        a_{2,-} \cdot b_{-,1} & a_{2,-} \cdot b_{-,2} & \vdots & a_{2,-} \cdot b_{-,n} \\
        \vdots                & \vdots                & \ddots & \vdots                \\
        a_{m,-} \cdot b_{-,1} & a_{m,-} \cdot b_{-,2} & \cdots & a_{m,-} \cdot b_{-,n}
      \end{pmatrix}.
    \end{equation*}

    If \( n \) and \( m \) are equal, \( R^{n \times n} \) becomes an \( R \)-\hyperref[def:algebra_over_semiring]{algebra} under \hyperref[thm:matrix_algebra/matrix_multiplication]{matrix multiplication} with multiplicative identity the \term{identity matrix} of order \( n \)
    \begin{equation}\label{eq:thm:matrix_algebra/matrix_multiplication/identity}
      \op{diag}(\underbrace{ 1, \cdots, 1 }_{n \T*{ones}})
      =
      \begin{pmatrix}
        1       & 0      & \cdots & 0      \\
        0       & 1      & \cdots & 0      \\
        \vdots  & \ddots & \ddots & \vdots \\
        0       &        & \cdots & 1
      \end{pmatrix}.
    \end{equation}
  \end{thmenum}
\end{proposition}
\begin{proof}
  The semimodule structure is inherited by the \hyperref[thm:commutative_monoid_is_semimodule]{semimodule structure} on \( R \). We will show that, if \( n = n \), matrix multiplication is associative and bilinear. Fix matrices
  \begin{equation*}
    \begin{aligned}
      A = \seq{ a_{i,j} }_{i,j=1}^{m,k} && B = \seq{ b_{i,j} }_{i,j=1}^{k,l} && C = \seq{ c_{i,j} }_{i,j=1}^{l,n}.
    \end{aligned}
  \end{equation*}

  \SubProofOf[def:binary_operation/associative]{associativity} The \( (i, j) \)-th entry in \( D \coloneqq (AB)C \) is
  \begin{equation*}
    d_{i,j} = \sum_{s=1}^n \parens*{ \sum_{r=1}^n a_{i,r} \cdot b_{r,s} } \cdot c_{s,j}.
  \end{equation*}

  Due to distributivity,
  \begin{equation*}
    d_{i,j}
    =
    \sum_{s=1}^n \sum_{r=1}^n a_{i,r} \cdot b_{r,s} \cdot c_{s,j}
    =
    \sum_{r=1}^n a_{i,r} \cdot \parens*{ \sum_{s=1}^n b_{r,s} \cdot c_{s,j} },
  \end{equation*}
  which is the \( (i, j) \)-th entry in \( A(BC) \).

  Therefore, \( (AB)C = A(BC) \).

  \SubProofOf[eq:def:semimodule/homomorphism/additive]{additivity} Again due to distributivity,
  \begin{equation*}
    \sum_{r=1}^n \parens*{ a_{i,r} + b_{i,r} } \cdot c_{r,j}
    =
    \sum_{r=1}^n a_{i,r} \cdot c_{r,j} + \sum_{r=1}^n b_{i,r} \cdot c_{r,j}.
  \end{equation*}

  Therefore, \( (A + B)C = AC + BC \). The proof that \( A(B + C) = AB + AC \) is analogous.

  \SubProofOf[eq:def:semimodule/homomorphism/homogeneity]{homogeneity} Again due to distributivity,
  \begin{equation*}
    t \cdot \sum_{r=1}^n a_{i,r} \cdot b_{r,j}
    =
    \sum_{r=1}^n (t \cdot a_{i,r}) \cdot b_{r,j}
    =
    \sum_{r=1}^n a_{i,r} \cdot (t \cdot b_{r,j}).
  \end{equation*}

  Therefore, \( t(AB) = (tA)B = A(tB) \).
\end{proof}

\begin{example}\label{ex:matrix_multiplication_is_noncommutative}
  For \( n > 1 \), the \hyperref[thm:matrix_algebra]{matrix algebra} \( R^{n \times n} \) is a noncommutative ring. Consider the following example:
  \begin{align*}
    \begin{pmatrix}
      0 & 0 \\
      0 & 1
    \end{pmatrix}
    \begin{pmatrix}
      1 & 0 \\
      1 & 0
    \end{pmatrix}
    &=
    \begin{pmatrix}
      0 & 0 \\
      1 & 0
    \end{pmatrix},
    \\
    \begin{pmatrix}
      1 & 0 \\
      1 & 0
    \end{pmatrix}
    \begin{pmatrix}
      0 & 0 \\
      0 & 1
    \end{pmatrix}
    &=
    \begin{pmatrix}
      0 & 0 \\
      0 & 0
    \end{pmatrix}.
  \end{align*}
\end{example}

\begin{remark}\label{rem:matrices_as_functions}
  Let \( R \) be a \hyperref[def:ring/commutative]{commutative ring} and let \( e_1, \ldots, e_n \) be the \hyperref[def:sequence_space]{standard basis} of \( R^n \). The \hyperref[def:basis_decomposition]{coordinate projections} \( \pi_{e_1}, \ldots, \pi_{e_n} \) allow us to identify \( R^n \) with the module \( R^{n \times 1} \) of \hyperref[def:array/column_vector]{column vectors} by regarding the vector \( x \) from \( R^n \) as the column vector
  \begin{equation*}
    \begin{pmatrix}
      \pi_{e_1}(x) \\
      \vdots \\
      \pi_{e_n}(x)
    \end{pmatrix}.
  \end{equation*}

  Under this identification, the columns on the identity matrix \eqref{eq:thm:matrix_algebra/matrix_multiplication/identity} are precisely the column vectors of the standard basis.

  Let \( A \) be an \( m \times n \) matrix over \( R \). If we regard \( R^n \) as a set of column vectors, then \hyperref[thm:matrix_algebra/matrix_multiplication]{matrix multiplication} allows us to regard \( A \) as the function \( x \mapsto Ax \), which maps column vectors from \( R^n \) to column vectors in \( R^m \).

  This justifies using juxtaposition for application of linear maps, e.g. \( Lx \) rather than \( L(x) \).

  Conversely, let \( e_1, \ldots, e_n \) be the standard basis of \( R^n \) and \( f_1, \ldots, f_m \) --- of \( R^m \). The linear map \( L: R^n \to R^m \) corresponds to the following matrix:
  \begin{equation*}
    \begin{pmatrix}
      \pi_{e_1}(L f_1) & \cdots & \pi_{e_1}(L f_1) \\
      \vdots           & \ddots & \vdots       \\
      \pi_{e_n}(L f_m) & \cdots & \pi_{e_n}(L f_m)
    \end{pmatrix}.
  \end{equation*}
\end{remark}

\begin{proposition}\label{thm:matrix_and_linear_function_algebras}
  For a \hyperref[def:ring/commutative]{commutative ring} \( R \), the \hyperref[thm:matrix_algebra]{matrix algebra} \( R^{m \times n} \) is \hyperref[def:algebra_over_semiring/homomorphism]{isomorphic} to the \hyperref[thm:functions_over_algebra]{linear function algebra} \( \hom(R^n, R^m) \)\fnote{Note that the maps are from \( R^n \) to \( R^m \) and not vice versa}.
\end{proposition}
\begin{proof}
  Follows from our discussion in \fullref{rem:matrices_as_functions} due to linearity.
\end{proof}

\begin{remark}\label{rem:double_index_maps}
  We want to be able to map single indices to double indices and vice versa, for example for the purpose of \fullref{thm:matrix_spaces_are_free_modules}. As an example, we want to be able to \enquote{linearize} an \( m \times n \) matrix such as the \( 2 \times 3 \) matrix
  \begin{equation}\label{eq:rem:double_index_maps/example/matrix}
    \begin{pmatrix}
      1 & 2 & 3 \\
      4 & 5 & 6
    \end{pmatrix}
  \end{equation}
  into the tuple
  \begin{equation}\label{eq:rem:double_index_maps/example/row_major}
    (1, 2, 3, 4, 5, 6)
  \end{equation}
  and vice versa. This is called \term{row-major order} of the elements of a matrix. The \term{column-major order} would instead be
  \begin{equation}\label{eq:rem:double_index_maps/example/column_major}
    (1, 4, 2, 5, 3, 6).
  \end{equation}

  Let \( m \) and \( n \) be \hyperref[def:integer_signum]{positive integers}. We will explicitly define functions for linearizing a matrix like \eqref{eq:rem:double_index_maps/example/matrix} into its row-major order \eqref{eq:rem:double_index_maps/example/row_major}. Consider the sets
  \begin{align*}
    S &\coloneqq \overbrace{ \set{ 1, \ldots, mn - 1, mn } }^{\T{single indices}}
    \\
    D &\coloneqq \underbrace{ \set{ 1, \ldots, m } \times \set{ 1, \ldots, n } }_{\T{double indices}}
  \end{align*}
  and the mutually inverse operations
  \begin{align}
    &\begin{aligned}\label{eq:rem:double_index_maps/sharp}
      &\sharp: S \to D \\
      &\sharp(k) \coloneqq \parens[\Big]{ \quot(k - 1, m) + 1, \rem(k - 1, m) + 1 } \\
    \end{aligned}
    \\[0.5\baselineskip]
    &\begin{aligned}\label{eq:rem:double_index_maps/flat}
      &\flat: D \to S \\
      &\flat(i, j) \coloneqq (i - 1) \cdot m + (j - 1) + 1.
    \end{aligned}
  \end{align}

  The operation \( \sharp \) encodes the matrix \eqref{eq:rem:double_index_maps/example/matrix} into its row-major order \eqref{eq:rem:double_index_maps/example/row_major} and \( \flat \) does the opposite. Both operations are trivial except for the shifting needed in to allow us to use \hyperref[def:euclidean_domain]{remainders and quotients}.

  We can easily verify that \( \sharp \) is a \hyperref[def:morphism_invertibility/left_invertible]{left inverse} of \( \flat \) (note that \( j < m \)):
  \begin{align*}
    \sharp(\flat(i, j))
    &=
    \sharp\parens[\Big]{ (i - 1) \cdot m + (j - 1) + 1 }
    = \\ &=
    \parens[\Big]{ \quot(\cdots, m) + 1, \rem(\cdots, m) + 1 }
    = \\ &=
    \parens[\Big]{ (i - 1) + 1, (j - 1) + 1 }
    = \\ &=
    (i, j).
  \end{align*}

  We can just as easily verify that \( \flat \) is a \hyperref[def:morphism_invertibility/right_invertible]{right inverse} of \( \sharp \):
  \begin{align*}
    \flat(\sharp(k))
    &=
    \flat\parens[\Big]{ \quot(k, m) + 1, \rem(k, m) + 1 }
    = \\ &=
    \quot(k, m) \cdot m + \rem(k, m)
    = \\ &=
    k.
  \end{align*}

  Hence, \( \sharp \) is fully invertible with inverse \( \flat \). By \fullref{thm:function_invertibility_categorical/fully_invertible}, it is bijective.
\end{remark}

\begin{proposition}\label{thm:matrix_spaces_are_free_modules}
  The \hyperref[thm:matrix_algebra]{matrix algebra} \( R^{m \times n} \) is isomorphic as a \hyperref[def:semimodule]{semimodule} to \( R^{mn} \).
\end{proposition}
\begin{proof}
  \Fullref{rem:double_index_maps} gives us a semimodule isomorphism between \( m \times n \) matrices and \( mn \)-dimensional column vectors when extended to linear maps via \fullref{thm:free_semimodule_universal_property}.
\end{proof}

\begin{definition}\label{def:matrix_determinant}\mcite[215]{Knapp2016BasicAlgebra}
  The \term{determinant} for the \hyperref[thm:matrix_algebra]{matrix algebra} \( R^{n \times n} \) over the \hyperref[def:semiring/commutative]{commutative semiring} \( R \) is the function
  \begin{equation}\label{eq:def:matrix_determinant}
    \begin{aligned}
      &\det: R^{n \times n} \to R \\
      &\det(\seq{ a_{i,j} }_{i,j=1}^n) \coloneqq \sum_{\sigma \in S_n} \sgn(\sigma) \prod_{i=1}^n a_{i,\sigma(i)},
    \end{aligned}
  \end{equation}
  where \( S_n \) is the \hyperref[def:symmetric_group]{symmetric group} and \( \sgn \) is the \hyperref[def:permutation_parity]{sign} of the permutation \( \sigma \).

  \Fullref{thm:similar_matrices_and_determinants} allows us to define determinants for linear endomorphisms rather than square matrices.

  See our proof of \fullref{thm:determinant_on_columns} for a justification of the definition.
\end{definition}

\begin{definition}\label{def:symmetric_function}\mcite[def. 2.11.1]{Savage1998}
  Given a function \( f: X^n \to Y \), where \( X \) and \( Y \) are \hyperref[def:set]{plain sets}, we say that \( f \) is \term{symmetric} if, for any \hyperref[def:symmetric_group]{permutation} \( \sigma \in S_n \), we have
  \begin{equation*}
    f(x_1, \ldots, x_n) = f(x_{\sigma(1)}, \ldots, x_{\sigma(n)}).
  \end{equation*}

  A permutation can be decomposed into \hyperref[def:cyclic_permutation]{transpositions} due to \fullref{thm:permutation_decomposition_existence}. Hence, the above condition reduces to the simpler condition of \( f \) being invariant with respect to swapping any two arguments. That is,
  \begin{equation*}
    f(\ldots, x_{i-1}, \fbox{\( x_i \)}, x_{i+1}, \cdots, x_{j-1}, \fbox{\( x_j \)}, x_{j+1}, \ldots)
    =
    f(\ldots, x_{i-1}, \fbox{\( x_j \)}, x_{i+1}, \cdots, x_{j-1}, \fbox{\( x_i \)}, x_{j+1}, \ldots).
  \end{equation*}

  In the case where \( n = 2 \), this reduces to the simple condition
  \begin{equation*}
    f(x, y) = f(y, x).
  \end{equation*}

  Symmetric functions should not be confused with symmetric binary relations defined in \fullref{def:binary_relation/symmetric}.
\end{definition}

\begin{definition}\label{def:antisymmetric_function}\mimprovised
  Given a function \( f: X^n \to Y \), where \( X \) is a \hyperref[def:set]{plain set} and \( Y \) is an \hyperref[rem:additive_semigroup]{additive group}, we say that \( f \) is \term{antisymmetric} if, for any \hyperref[def:symmetric_group]{permutation} \( \sigma \in S_n \), we have
  \begin{equation*}
    f(x_1, \ldots, x_n) = \sgn(\sigma) \cdot f(x_{\sigma(1)}, \ldots, x_{\sigma(n)}).
  \end{equation*}

  A permutation can be decomposed into \hyperref[def:cyclic_permutation]{transpositions} due to \fullref{thm:permutation_decomposition_existence}. Hence, the above condition reduces to the simpler condition of \( f \) changing sign when swapping any two arguments. That is,
  \begin{equation*}
    f(\ldots, x_{i-1}, \fbox{\( x_i \)}, x_{i+1}, \cdots, x_{j-1}, \fbox{\( x_j \)}, x_{j+1}, \ldots)
    =
    -f(\ldots, x_{i-1}, \fbox{\( x_j \)}, x_{i+1}, \cdots, x_{j-1}, \fbox{\( x_i \)}, x_{j+1}, \ldots).
  \end{equation*}

  In the case where \( n = 2 \), this reduces to the simple condition
  \begin{equation*}
    f(x, y) = -f(y, x).
  \end{equation*}

  Antisymmetric functions should not be confused with antisymmetric binary relations defined in \fullref{def:binary_relation/antisymmetric}.
\end{definition}

\begin{definition}\label{def:alternating_function}\mimprovised
  Given a commutative ring \( R \), and \( R \)-module \( M \) and a \hyperref[def:multilinear_function]{multilinear function} \( f: M \to R \), we say that \( f \) is \term{alternating} if, \( x_i = x_j \) implies that
  \begin{equation*}
    f(x_1, \ldots, x_i, \ldots, x_j, \ldots x_n) = 0.
  \end{equation*}
\end{definition}

\begin{proposition}\label{thm:alternating_multilinear_is_antisymmetric}
  If a \hyperref[def:multilinear_function]{multilinear map} is \hyperref[def:alternating_function]{alternating}, it is \hyperref[def:antisymmetric_function]{antisymmetric}. The converse holds if \( 2 \) is invertible.
\end{proposition}
\begin{proof}
  \SufficiencySubProof If \( f \) is an alternating multilinear map, then
  \begin{equation*}
    0
    =
    f(\cdots, x_i + x_j, \cdots, x_i + x_j, \cdots)
    =
    f(\cdots, x_i, \cdots, x_j, \cdots)
    +
    f(\cdots, x_j, \cdots, x_i, \cdots).
  \end{equation*}

  Therefore,
  \begin{equation*}
    f(\cdots, x_i, \cdots, x_j, \cdots)
    =
    -f(\cdots, x_j, \cdots, x_i, \cdots).
  \end{equation*}

  \NecessitySubProof If \( f \) is an antisymmetric multilinear map, then
  \begin{equation*}
    0
    =
    f(\cdots, x_i + x_i, x_i, \cdots)
    =
    2 f(\cdots, x_i, x_i, \cdots).
  \end{equation*}

  If \( 2 \) is invertible, this implies
  \begin{equation*}
    f(\cdots, x_i, x_i, \cdots) = 0.
  \end{equation*}
\end{proof}

\begin{proposition}\label{thm:determinant_on_columns}
  In the \hyperref[thm:matrix_algebra]{matrix algebra} \( R^{n \times n} \) over the commutative ring \( R \), the determinant function \( \det: R^{n \times n} \to R \) can be regarded as a function that maps \( n \) column vectors from \( R^n \) to \( R \). That is,
  \begin{equation}\label{eq:thm:determinant_on_columns}
    \det(v_1, \cdots, v_n) = \sum_{\sigma \in S_n} \sgn(\sigma) \prod_{i=1}^n \pi_{\sigma(i)} (v_i).
  \end{equation}

  The determinant is an \hyperref[def:alternating_function]{alternating} \hyperref[def:multilinear_function]{multilinear function} on columns. Furthermore, it is the unique alternating multilinear function \( f(v_1, \ldots, v_n) \) such that \( f(e_1, \ldots, e_n) = 1 \), where \( e_1, \ldots, e_n \) are vectors of the \hyperref[def:sequence_space]{standard basis} in \( R^n \).
\end{proposition}
\begin{proof}
  Let \( \pi_1, \ldots, \pi_n \) be the \hyperref[def:basis_decomposition]{projection functionals} corresponding to the \hyperref[def:sequence_space]{standard basis} \( e_1, \ldots, e_n \).

  \SubProof{Proof of multilinearity} Due to linearity of the coordinate projection functionals \( \pi_i \) and due to distributivity in \( R \), for every \( j \) we have
  \begin{align*}
    &\phantom{{}={}}
    \det(\cdots, v_{j-1}, ty + rz, v_{j+1}, \cdots)
    = \\ &=
    \sum_{\sigma \in S_n} \sgn(\sigma) \cdot \pi_{\sigma(j)} (ty + rz) \prod_{i \neq j} \pi_{\sigma(i)} (v_i)
    = \\ &=
    t \sum_{\sigma \in S_n} \sgn(\sigma) \cdot \pi_{\sigma(j)} (y) \prod_{i \neq j} \pi_{\sigma(i)} (v_i) + r \sum_{\sigma \in S_n} \sgn(\sigma) \cdot \pi_{\sigma(j)} (z) \prod_{i \neq j} \pi_{\sigma(i)} (v_i)
    = \\ &=
    t \cdot \det(\cdots, y, \cdots) + r \cdot \det(\cdots, z, \cdots).
  \end{align*}

  \SubProof{Proof of alternation} If \( v_i = v_j \), then for every even (resp. odd) permutation \( \sigma \), the permutation \( \cycle{i, j} \bincirc \sigma \) is odd (resp. even), and hence they cancel out in the sum \eqref{eq:thm:determinant_on_columns}. This holds for every permutation, hence it remains for the determinant to be zero.

  \SubProof{Proof of \( \det(I_n) = 1 \)} Note that
  \begin{equation*}
    \prod_{i=1}^n \pi_i (e_{\sigma(i)}) \neq 0
  \end{equation*}
  if and only if \( i = \sigma(i) \) for every \( i = 1, \ldots, n \). This only holds for the identity permutation, hence
  \begin{equation*}
    \det(e_1, \ldots, e_n) = \prod_{i=1}^n \pi_i(e_i) = \prod_{i=1}^n 1 = 1.
  \end{equation*}

  \UniquenessSubProof Suppose that \( f(v_1, \ldots, v_n) \) is an alternating multilinear function such that \( f(e_1, \ldots, e_n) = 1 \).

  For an arbitrary column vector \( v_j \) in \( R^n \), we have
  \begin{equation*}
    v_j = \sum_{i=1}^n \pi_i(v_j) \cdot e_i.
  \end{equation*}

  Then
  \begin{align*}
    f(v_1, \ldots, v_n)
    &=
    f\parens*{ \sum_{i_1=1}^n \pi_{i_1}(v_1) \cdot e_{i_1}, \ldots, \sum_{i_n=1}^n \pi_{i_n}(v_n) \cdot e_{i_n} }
    = \\ &=
    \sum_{i_1=1}^n \pi_{i_1}(v_1) \cdots \sum_{i_n=1}^n \pi_{i_n}(v_n) f(e_{i_1}, \ldots, e_{i_n})
    = \\ &=
    \sum_{\sigma \in S_n} \pi_{\sigma(i)}(v_i) f(e_{\sigma(1)}, \ldots, e_{\sigma(n)}).
  \end{align*}

  The last step is valid because \( f \) is \hyperref[def:alternating_function]{alternating} and thus \( f(e_{i_1}, \cdots, e_{i_n}) \) is zero when not all of \( i_1, \ldots, i_n \) are distinct, and they are necessarily distinct if the indices are given by a permutation from \( S_n \).

  Finally, since \( f \) is \hyperref[def:antisymmetric_function]{antisymmetric} due to \fullref{thm:alternating_multilinear_is_antisymmetric},
  \begin{equation*}
    f(e_{\sigma(1)}, \ldots, e_{\sigma(n)}) = \sgn(\sigma) \underbrace{f(e_1, \ldots, e_n)}_{1 \T*{by assumption}} = \sgn(\sigma).
  \end{equation*}

  Therefore,
  \begin{equation*}
    f(v_1, \ldots, v_n) = \det(v_1, \ldots, v_n).
  \end{equation*}
\end{proof}

\begin{definition}\label{def:transpose_matrix}\mimprovised
  The \term{transpose matrix} of
  \begin{equation*}
    A = \begin{pmatrix}
      a_{1,1} & a_{1,2} & \cdots & a_{1,n} \\
      a_{2,1} & a_{2,2} & \cdots & a_{2,n} \\
      \vdots  & \vdots  & \ddots & \vdots  \\
      a_{m,1} & a_{m,2} & \cdots & a_{m,n}
    \end{pmatrix}
  \end{equation*}
  is defined as
  \begin{equation*}
    A^T = \begin{pmatrix}
      a_{1,1} & a_{1,2} & \cdots & a_{n,1} \\
      a_{2,1} & a_{2,2} & \cdots & a_{n,2} \\
      \vdots  & \vdots  & \ddots & \vdots  \\
      a_{1,m} & a_{2,m} & \cdots & a_{n,m}
    \end{pmatrix}.
  \end{equation*}

  A matrix that is equal to its transpose is called \term{symmetric}.
\end{definition}

\begin{proposition}\label{thm:def:matrix_determinant}
  In the \hyperref[thm:matrix_algebra]{matrix algebra} \( R^{n \times n} \) over the commutative ring \( R \), the \hyperref[def:matrix_determinant]{determinant} as function on matrices has the following basic properties:
  \begin{thmenum}
    \thmitem{thm:def:matrix_determinant/transpose} \( \det(A^T) = \det(A) \).
    \thmitem{thm:def:matrix_determinant/homogeneous} \( \det(tA) = t^n \cdot \det(A) \).
    \thmitem{thm:def:matrix_determinant/homomorphism}\mcite[sec. 6.7]{Тыртышников2007} \( \det(AB) = \det(A) \cdot \det(B) \).

    That is, \( \det: R^{n \times n} \to R \) is a \hyperref[def:monoid/homomorphism]{monoid homomorphism} from the \hyperref[def:semiring]{multiplicative monoid} of the ring \( R^{n \times n} \) to the multiplicative monoid of \( R \).
  \end{thmenum}
\end{proposition}
\begin{proof}
  \SubProofOf{thm:def:matrix_determinant/transpose} The inverse of any permutation in \( S_n \) is also a permutation in \( S_n \), hence
  \begin{equation*}
    \det(A^T)
    =
    \sum_{\sigma \in S_n} \sgn(\sigma) \prod_{i=1}^n a_{\sigma(i),i}
    =
    \sum_{\sigma \in S_n} \sgn(\sigma^{-1}) \prod_{i=1}^n a_{i,\sigma^{-1}(i)}
    =
    \det(A).
  \end{equation*}

  \SubProofOf{thm:def:matrix_determinant/homogeneous} Follows from \fullref{thm:determinant_on_columns}.

  \SubProofOf{thm:def:matrix_determinant/homomorphism} The \( j \)-th column of the product \( C = AB \) is
  \begin{equation*}
    c_{\anon*,j}
    =
    \sum_{i=1}^n b_{i,j} a_{\anon*,i}
    =
    \begin{pmatrix}
      \sum_{i=1}^n a_{1,i} b_{i,j} \\
      \vdots \\
      \sum_{i=1}^n a_{n,i} b_{i,j} \\
    \end{pmatrix}.
  \end{equation*}

  Since the determinant is a multilinear function on columns,
  \begin{balign*}
    \det(c_{\anon*,1}, \cdots, c_{\anon*,n})
    &=
    \det\parens*{ \sum_{i_1=1}^n b_{i_1,1} a_{\anon*,i_1}, \cdots, \sum_{i_n=1}^n b_{i_n,n} a_{\anon*,i_n} }
    = \\ &=
    \sum_{i_1=1}^n b_{i_1,1} \det\parens*{ a_{\anon*,i_1}, \cdots, \sum_{i_n=1}^n b_{i_n,n} a_{\anon*,i_n} }
    = \\ &=
    \sum_{i_1=1}^n b_{i_1,1} \cdots \sum_{i_n=1}^n b_{i_n,n} \det(a_{\anon*,i_1}, \cdots, a_{\anon*,i_n})
    = \\ &=
    \sum_{i_1=1}^n \cdots \sum_{i_n=1}^n b_{i_1,1} \cdots b_{i_n,n} \det(a_{\anon*,i_1}, \cdots, a_{\anon*,i_n}).
  \end{balign*}

  Since the determinant is \hyperref[def:alternating_function]{alternating} on columns, \( \det(a_{\anon*,i_1}, \cdots, a_{\anon*,i_n}) \) is zero when not all of \( i_1, \ldots, i_n \) are distinct. They are necessarily distinct if the indices are given by a permutation from \( S_n \). Therefore,
  \begin{balign*}
    \det(AB)
    &=
    \sum_{\sigma \in S_n} \prod_{i=1}^n a_{i,\sigma(i)} \cdot \sigma(a_{\anon*, \sigma(1)}, \ldots, a_{\anon*, \sigma(n)})
    = \\ &=
    \sum_{\sigma \in S_n} \prod_{i=1}^n a_{i,\sigma(i)} \cdot \sgn(\sigma) \cdot \sigma(a_{\anon*, 1}, \ldots, a_{\anon*, n})
    = \\ &=
    \det(B) \det(A).
  \end{balign*}
\end{proof}

\begin{definition}\label{def:submatrix}\mimprovised
  If for the matrices \( A = \seq{ a_{i,j} }_{i,j=1}^{m,n} \) and \( B = \seq{ b_{i,j} }_{i,j=1}^{k,k} \) over a commutative ring there exist \hyperref[def:order_function]{monotone functions}
  \begin{align*}
    &h: \set{ 1, \ldots, k } \to \set{ 1, \ldots, m }, \\
    &w: \set{ 1, \ldots, l } \to \set{ 1, \ldots, n },
  \end{align*}
  such that, for every \( i = 1, \ldots, k \) and \( j = 1, \ldots, l \) we have
  \begin{equation*}
    b_{i,j} = a_{h(i),w(j)}.
  \end{equation*}
\end{definition}

\begin{definition}\label{def:matrix_minor}\mimprovised
  A \term{minor} of a matrix is a \hyperref[def:matrix_determinant]{determinant} of a square \hyperref[def:submatrix]{submatrix}.
\end{definition}

\medskip

\begin{theorem}[Laplace expansion]\label{thm:laplace_expansion}\mcite[prop. 2.36]{Knapp2016BasicAlgebra}
  For a square matrix \( A = \seq{ a_{i,j} }_{i,j=1}^{n,n} \) over a commutative ring and a row index \( i \), we have
  \begin{equation*}
    \det A = \sum_{j=1}^n (-1)^{i + j} a_{i,j} \det A_{i,j},
  \end{equation*}
  where \( A_{i,j} \) is the \hyperref[def:submatrix]{submatrix} of \( A \) obtained by removing the \( i \)-th row and the \( j \)-th column.

  By \fullref{thm:def:matrix_determinant/transpose}, we can also expand along a column rather than a row.
\end{theorem}
\begin{proof}
  Denote the ring by \( R \). We will show that, for the \( i \)-th row,
  \begin{equation*}
    \begin{aligned}
      &\Phi: R^{n \times n} \to R, \\
      &\Phi(A) \coloneqq \sum_{j=1}^n (-1)^{i + j} a_{i,j} \det A_{i,j}.
    \end{aligned}
  \end{equation*}
  is an \hyperref[def:alternating_function]{alternating} \hyperref[def:multilinear_function]{multilinear function} on columns.

  Multilinearity follows from the multilinearity of determinants. For proving alternation, suppose that the \( k \)-th and \( l \)-th columns are equal. Then
  \begin{equation*}
    \Phi(A) = (-1)^{i + k} a_{i,k} \det A_{i,k} + (-1)^{i + l} a_{i,l} \det A_{i,l}.
  \end{equation*}

  The matrix \( A_{i,l} \) can be obtained from \( A_{i,k} \) by swapping \( \abs{k - l} \) columns. Since determinants are antisymmetric, it follows that
  \begin{equation*}
    \det A_{i,l} = (-1)^{k - l} \det A_{i,k}.
  \end{equation*}

  Furthermore, \( a_{i,k} = a_{i,l} \). Therefore,
  \begin{equation*}
    \Phi(A) = (-1)^{i + k} a_{i,k} \det A_{i,k} + (-1)^{(i + l) + (k - l)} a_{i,k} \det A_{i,k} = 0.
  \end{equation*}
\end{proof}

\begin{definition}\label{def:adjugate_matrix}\mimprovised
  The \term{cofactor matrix} of the \( m \times n \) matrix \( A \) is
  \begin{equation*}
    \seq{ (-1)^{i + j} \det A_{i,j} }_{i,j=1}^{m,n},
  \end{equation*}
  where \( A_{i,j} \) is the \hyperref[def:submatrix]{submatrix} of \( A \) obtained by removing the \( i \)-th row and the \( j \)-th column.

  The \term{adjugate matrix} \( A^{\op{adj}} \), also called the \term{classical adjoint matrix}, is the transpose of the cofactor matrix.
\end{definition}

\begin{proposition}\label{thm:inverse_via_adjunction}
  The \hyperref[def:adjugate_matrix]{adjugate matrix} of the square \( n \times n \) matrix \( A \) satisfies
  \begin{equation*}
    A \cdot A^{\op{adj}} = \det A \cdot I_n.
  \end{equation*}
\end{proposition}
\begin{proof}
  From \fullref{thm:laplace_expansion} it follows that the \( (i, i) \)-th entry of the matrix \( A \cdot A^{\op{adj}} \) is
  \begin{equation*}
    \sum_{k=1}^n (-1)^{i + k} a_{i,k} \det A_{i,k}
    =
    \det A.
  \end{equation*}

  For \( i \neq j \), the \( (i, j) \)-th entry is
  \begin{equation*}
    \sum_{k=1}^n (-1)^{i + j} a_{i,k} \det A_{j,k}
    =
    \det \widehat A_{j \mapsto i},
  \end{equation*}
  where \( \widehat A_{j \mapsto i} \) is the matrix obtained by replacing the \( j \)-th column in \( A \) with the \( i \)-th. The determinant is then zero because it is an alternating function on the columns.

  Thus, the proposition follows.
\end{proof}

\begin{definition}\label{def:inverse_matrix}
  We say that \( B \in R^{n \times m} \) is a \term{left inverse matrix} of \( A \in R^{m \times n} \) if \( BA \) is the identity matrix \( I_n \) and a \term{right inverse matrix} if \( AB \) is \( I_m \). These are precisely the left and right inverse linear maps in the correspondence described in \fullref{thm:matrix_and_linear_function_algebras}.

  Due to \fullref{thm:square_matrix_left_invertible_iff_right_invertible}, for square matrices, the two notions coincide, and we say that \( B \) is simply an \term{inverse} of \( A \). An inverse matrix, if it exists, is unique. We denote this inverse of \( A \) by \( A^{-1} \).

  We say that \( A \) is \term{invertible} if an inverse exists, and \term{singular} otherwise.
\end{definition}
\begin{defproof}
  The inverse is unique by \fullref{thm:monoid_inverse_unique}.
\end{defproof}

\begin{proposition}\label{thm:square_matrix_left_invertible_iff_right_invertible}
  Over a nontrivial \hyperref[def:noetherian_semiring]{noetherian} commutative ring \( R \), a square matrix is \hyperref[def:inverse_matrix]{left invertible} if and only if it is \hyperref[def:inverse_matrix]{right invertible}.
\end{proposition}
\begin{proof}
  \NecessitySubProof Suppose that \( A \) is a right invertible matrix. When regarding \( A \) as a linear map via the identification from \fullref{rem:matrices_as_functions}, this implies that \( A \), as a linear map from \( R^n \) to \( R^n \), is right invertible. Then it is surjective and, by \fullref{thm:surjective_endomorphism_in_free_module}, an isomorphism. Therefore, \( A \) is a fully invertible matrix.

  \SufficiencySubProof Now suppose that \( A \) is a left inverse of \( B \). Then \( B \) is a right inverse of \( A \), and, by the other direction of the proposition, a two-sided inverse of \( A \).
\end{proof}

\begin{proposition}\label{thm:matrix_invertibility}
  In the \hyperref[def:inverse_matrix]{matrix algebra} \( R^{n \times n} \) over a nontrivial \hyperref[def:noetherian_semiring]{noetherian} \hyperref[def:ring/commutative]{commutative ring} \( R \), the following are equivalent:
  \begin{thmenum}
    \thmitem{thm:matrix_invertibility/invertible} The matrix \( A \) is \hyperref[def:inverse_matrix]{invertible}.
    \thmitem{thm:matrix_invertibility/determinant} The \hyperref[def:matrix_determinant]{determinant} of \( A \) is \hyperref[def:divisibility/invertible]{invertible}.
    \thmitem{thm:matrix_invertibility/columns} The \hyperref[def:block_matrix]{columns} of \( A \) are \hyperref[def:linear_dependence]{linearly independent}.
  \end{thmenum}
\end{proposition}
\begin{proof}
  \ImplicationSubProof{thm:matrix_invertibility/invertible}{thm:matrix_invertibility/determinant} Suppose that \( A \) is invertible.

  By \fullref{thm:def:matrix_determinant/homomorphism},
  \begin{equation*}
    \det(A^{-1}) \det(A) = \det(A^{-1} A) = \det(I_n) = 1,
  \end{equation*}
  hence \( \det(A) \) has a multiplicative inverse.

  \ImplicationSubProof{thm:matrix_invertibility/determinant}{thm:matrix_invertibility/columns} As in \fullref{thm:determinant_on_columns}, regard \( \det(v_1, \ldots, v_n) \) are an alternating multilinear function on the columns of a matrix.

  Suppose that \( \det(v_1, \ldots, v_n) \) is invertible in \( R \) and, aiming at a contradiction, suppose that the column vectors \( v_1, \ldots, v_n \) are linearly dependent. Then there exists a nontrivial linear combination that sums to zero:
  \begin{equation*}
    \sum_{i=1}^n t_i v_i = 0.
  \end{equation*}

  Suppose that \( t_k \) is nonzero. Then
  \begin{align*}
    0
    &=
    \det\parens*{ v_1, \ldots, v_{k-1}, 0, v_{k+1}, \ldots, v_n }
    = \\ &=
    \det\parens*{ v_1, \ldots, v_{k-1}, \sum_{i=1}^n t_i v_i, v_{k+1}, \ldots, v_n }
    = \\ &=
    \sum_{i=1}^n t_i \det( v_1, \ldots, v_{k-1}, v_i, v_{k+1}, \ldots, v_n )
    = \\ &=
    t_k \det( v_1, \ldots, v_{k-1}, v_k, v_{k+1}, \ldots, v_n ).
  \end{align*}

  But we have assumed that \( \det( v_1, \ldots, v_{k-1}, v_k, v_{k+1}, \ldots, v_n ) \) is invertible and that \( t_k \) is nonzero. Hence, the determinant can only be a zero divisor if the ring is trivial, which we have assumed it is not. The obtained contradiction shows that \( v_1, \ldots, v_n \) are linearly independent.

  \ImplicationSubProof{thm:matrix_invertibility/columns}{thm:matrix_invertibility/invertible} Suppose that the columns of \( A \) are linearly independent. Consider the matrix equation
  \begin{equation*}
    Ax
    =
    \parens*
    {
      \begin{array}{c|c|c}
        a_{\anon*,1} & \cdots & a_{\anon*,n}
      \end{array}
    }
    \begin{pmatrix}
      x_1 \\ \vdots \\ x_n
    \end{pmatrix}
    =
    \sum_{k=1}^n x_k a_{\anon*,k}
    =
    \vect 0.
  \end{equation*}

  Since the columns are linearly independent, only \( x_1 = \cdots = x_n \) is a solution to this equation. Thus, when regarding \( A \) as the linear map \( x \mapsto Ax \), the \hyperref[def:module/kernel]{kernel} of \( A \) becomes trivial. By \fullref{thm:group_homomorphism_trivial_kernel}, this map is injective. As discussed in \fullref{def:module/category}, the injective linear maps are exactly the left invertible linear maps. Hence, there exists a left inverse of \( A \). Since \( A \) is a square matrix, by \fullref{thm:square_matrix_left_invertible_iff_right_invertible}, this implies that \( A \) is invertible.
\end{proof}

\begin{proposition}\label{thm:inverse_of_2x2_matrix}
  Assuming that the matrix is invertible,
  \begin{equation*}
    \begin{pmatrix}
      a & b \\
      c & d
    \end{pmatrix}^{-1}
    =
    \frac 1 {ad - bc}
    \begin{pmatrix}
      d  & -b \\
      -c & a
    \end{pmatrix}
  \end{equation*}
\end{proposition}
\begin{proof}
  Follows from \fullref{thm:inverse_via_adjunction}.
\end{proof}

\begin{proposition}\label{thm:determinant_of_inverse}
  The determinant of \( A^{-1} \) is \( \det(A)^{-1} \).
\end{proposition}
\begin{proof}
  Follows from \fullref{thm:inverse_via_adjunction}.
\end{proof}

\begin{definition}\label{def:linear_groups}\mimprovised
  The \hyperref[def:semiring]{multiplicative group} of the \hyperref[thm:matrix_algebra]{matrix algebra} \( R^{n \times n} \) is called the \term{general linear group} \( \grp{GL}_R(n) \). These are the \hyperref[def:inverse_matrix]{invertible} \( n \times n \) matrices over \( R \).

  The subgroup of matrices with determinant \( 1 \) is called the \term{special linear group} \( \grp{SL}_R(n) \).
\end{definition}

\begin{remark}\label{rem:change_of_basis}
  Let \( R \) be a \hyperref[def:ring/commutative]{commutative ring} and \( V \) be an \( R \)-module of finite \hyperref[thm:commutative_module_rank]{rank}. Suppose that \( e_1, \ldots, e_n \) and \( f_1, \ldots, f_n \) are both bases of \( V \). Any vector \( v \) in \( V \) can be decomposed along \( e_1, \ldots, e_n \) to form the tuple \( x_1, \ldots, x_n \) and along \( f_1, \ldots, f_n \) to form \( y_1, \ldots, y_n \).

  \begin{equation*}
    f_j = \sum_{k=1}^n \pi_{e_k}(f_j) \cdot e_k.
  \end{equation*}

  Thus, for \( j = 1, \ldots, n \),
  \begin{equation*}
    y_j
    =
    \pi_{f_j}(y)
    =
    \pi_{f_j}\parens*{ \sum_{k=1}^n \pi_{e_k}(y) \cdot e_k }
    =
    \sum_{k=1}^n \pi_{e_k}(y) \cdot \pi_{f_j}(e_k)
    =
    \sum_{k=1}^n x_k \cdot \pi_{f_j}(e_k).
  \end{equation*}

  This can be alternatively written as
  \begin{equation*}
    \begin{pmatrix}
      y_1 \\ \vdots \\ y_n
    \end{pmatrix}
    =
    \begin{pmatrix}
      \pi_{f_1}(e_1) & \cdots & \pi_{f_1}(e_n) \\
      \vdots         & \ddots & \vdots         \\
      \pi_{f_n}(e_1) & \cdots & \pi_{f_n}(e_n)
    \end{pmatrix}
    \begin{pmatrix}
      x_1 \\ \vdots \\ x_n
    \end{pmatrix}.
  \end{equation*}

  The above matrix allows us to transform coordinates with respect to one basis to coordinates with respect to another. For this reason, we call it the \term{change of basis} matrix from \( e_1, \ldots, e_n \) to \( f_1, \ldots, f_n \).
\end{remark}

\begin{definition}\label{def:similar_matrices}\mcite[48]{Knapp2016BasicAlgebra}
  We say that the \( n \times n \) matrices \( A \) and \( B \) over the commutative ring \( R \) are \term{similar} if they are \hyperref[thm:group_conjugation_action]{conjugates} in the \hyperref[def:linear_groups]{general linear group} \( \grp{GL}_R(n) \). That is, if there exists an \hyperref[def:inverse_matrix]{invertible matrix} \( P \) such that \( A = P^{-1} B P \).
\end{definition}

\begin{proposition}\label{thm:similar_matrices_and_determinants}
  \hyperref[def:matrix_determinant]{Determinants} are invariant under \hyperref[def:similar_matrices]{matrix similarity}.

  This allows us to consider determinants of operators rather than matrices.
\end{proposition}
\begin{proof}
  By \fullref{thm:def:matrix_determinant/homomorphism} and \fullref{thm:determinant_of_inverse},
  \begin{equation*}
    \det(A) = \det(P^{-1} B P) = \det(P)^{-1} \det(B) \det(P) = \det(B).
  \end{equation*}
\end{proof}

\begin{remark}\label{rem:linear_operators_and_matrices}
  Let \( R \) be a \hyperref[def:ring/commutative]{commutative ring}. Unlike in the \( R \)-module \( R^n \) of tuples, in a general \( R \)-module of finite \hyperref[thm:commutative_module_rank]{rank}, we have no concept of a \hyperref[def:sequence_space]{standard basis}. Instead, a linear operator \( T: U \to V \) corresponds to multiple matrix.

  If \( U \) has rank \( n \), a choice of basis for \( U \) is simply a choice of isomorphism with \( R^n \). Hence, given isomorphisms \( \varphi: U \to R^n \) and via \( \psi: V \to R^m \), the operator \( \psi^{-1} \bincirc T \bincirc \varphi \) becomes a function from \( R^n \) to \( R^m \). We can now use the identification with matrices discussed in \fullref{rem:matrices_as_functions}.

  This requires a choice of bases for \( U \) and for \( V \). If we have a matrix representing an operator and if we wish to use different bases, we must multiply it with the corresponding change of basis matrices discussed in \fullref{rem:matrices_as_functions}.
\end{remark}

\begin{proposition}\label{thm:matrices_of_operator_are_similar}
  Let \( V \) be a module of finite rank over a commutative ring and let \( T: V \to V \) be a linear endomorphism. Then two matrices \( A \) and \( B \) represent \( T \) if and only if they are \hyperref[def:similar_matrices]{similar}.
\end{proposition}
\begin{proof}
  \SufficiencySubProof Suppose that \( A \) represents \( T \) with respect to \( e_1, \ldots, e_n \) and \( B \) represents \( T \) with respect to \( f_1, \ldots, f_n \).

  Define \( P \) to be the \hyperref[rem:change_of_basis]{change of basis} matrix from \( e_1, \ldots, e_n \) to \( f_1, \ldots, f_n \). Then
  \begin{equation*}
    A = P^{-1} B P.
  \end{equation*}

  \NecessitySubProof Suppose that \( A = P^{-1} B P \) and that \( B \) represents \( T \) with respect to \( e_1, \ldots, e_n \). Then \( A \) represents \( T \) with respect to the basis \( P e_1, \ldots, P e_n \).
\end{proof}

  \section{Matrices over fields}\label{sec:matrices_over_fields}

We will assume that all matrices have entries from some fixed \hyperref[def:field]{field} \( \BbbK \). We will later on need to distinguish between real and complex matrices, but the theory built here holds more generally than that, and we choose to postulate it for arbitrary fields.

The definitions of \hyperref[def:triangular_matrix]{triangular} and \hyperref[def:triangular_matrix]{elementary matrices} make sense over more general rings, however we introduce them because of \fullref{alg:plu_decomposition}, which has no direct generalization.

\begin{definition}\label{def:triangular_matrix}
  An \term{upper triangular matrix} is one with zeros below its \hyperref[def:matrix_diagonal]{main diagonal}. More precisely, \( U = \seq{ u_{i,j} }_{i,j=1}^{m,n} \) is an upper triangular matrix if \( u_{i,j} = 0 \) when \( i > j \). We call it \term{unitriangular} if the diagonal entries are ones.

  Similarly, a \term{lower triangular matrix} is one with zeros above its main diagonal.

  A matrix that is either upper or lower triangular is simply referred to as \enquote{triangular}.
\end{definition}

\begin{proposition}\label{thm:def:triangular_matrix}
  \hyperref[def:triangular_matrix]{Triangular matrices} have the following basic properties:
  \begin{thmenum}
    \thmitem{thm:def:triangular_matrix/diagonal} A matrix that is both upper and lower triangular is a \hyperref[def:matrix_diagonal]{diagonal matrix}.

    \thmitem{thm:def:triangular_matrix/product} The \hyperref[thm:matrix_algebra/matrix_multiplication]{product} of upper (resp. lower) triangular matrices is upper (resp. lower).

    Consequently, the product of diagonal matrices is a diagonal matrix.

    \thmitem{thm:def:triangular_matrix/determinant} The \hyperref[thm:def:triangular_matrix/determinant]{determinant} of a triangular matrix is the product of (the entries on) its main diagonal.

    \thmitem{thm:def:triangular_matrix/invertible} A triangular matrix is \hyperref[def:inverse_matrix]{invertible} if and only if its main diagonal has no zero entries.

    Here, the assumption that \( \BbbK \) is a field is essential.
  \end{thmenum}
\end{proposition}
\begin{proof}
  \SubProofOf{thm:def:triangular_matrix/diagonal} Trivial.

  \SubProofOf{thm:def:triangular_matrix/product} Let \( A \) be an \( m \times k \) upper triangular matrix and \( B \) be a \( k \times n \) upper triangular matrix. The \( (i, j) \)-th element of \( C = AB \) is
  \begin{equation*}
    \sum_{l=1}^k a_{i,l} b_{l,j}.
  \end{equation*}

  Since \( A \) and \( B \) are upper triangular, we have \( b_{l,j} = 0 \) whenever \( l > j \) and \( a_{i,l} = 0 \) whenever \( l < i \). Thus, \( a_{i,l} b_{l,j} = 0 \) if either condition holds. If \( i > j \), then either \( l > j \) or \( l < j < i \), implying that \( a_{i,l} b_{l,j} = 0 \). Therefore, \( AB \) is also upper triangular.

  The proof for lower triangular matrices is analogous.

  \SubProofOf{thm:def:triangular_matrix/determinant} Let \( A \) be an \( n \times n \) upper triangular matrix. Let \( \sigma \in S_n \) be any permutation. Then \( a_{i,\sigma(i)} = 0 \) when \( i > \sigma(i) \). Hence, the only permutation for which the product \( \prod_{i=1}^n a_{i,\sigma(i)} \) is nonzero is the identity permutation. Therefore,
  \begin{equation*}
    \det(A) = \prod_{i=1}^n a_{i,i}.
  \end{equation*}

  \SubProofOf{thm:def:triangular_matrix/invertible} Follows from \fullref{thm:def:triangular_matrix/determinant} and \fullref{thm:matrix_invertibility_via_determinants} by noting that \( 0 \) is the only non-invertible element in a field.
\end{proof}

\begin{definition}\label{def:elementary_matrix}
  We introduce the following three types of \hyperref[def:inverse_matrix]{invertible} \( n \times n \) matrices, collectively known as \term{elementary matrices}:
  \begin{thmenum}
    \thmitem{def:elementary_matrix/permutation} For a \hyperref[def:symmetric_group]{permutation} \( \sigma \in S_n \), the \term{permutation matrix} \( P_\sigma \) is obtained by permuting the columns \( e_1, \ldots, e_n \) of the identity matrix \( I_n \) in accordance with \( \sigma \). The permutation matrix
    \begin{equation*}
      P_\sigma = \parens*
      {
        \begin{array}{c|c|c}
          e_{\sigma(1)} & \cdots & e_{\sigma(n)}
        \end{array}
      }
    \end{equation*}
    acts on the \( n \times m \) matrix \( B = \seq{ b_{i,j} }_{i,j=1}^{m,n} \) by permuting the \hi{rows} of \( B \), i.e.
    \begin{equation*}
      P_\sigma B
      =
      \parens*
      {
        \begin{array}{c|c|c}
          e_{\sigma(1)} & \cdots & e_{\sigma(n)}
        \end{array}
      }
      \cdot
      \begin{pmatrix}
        b_{1,1} & b_{1,2} & \cdots & b_{1,m} \\
        \vdots  & \vdots  & \ddots & \vdots \\
        b_{n,1} & b_{n,2} & \cdots & b_{n,m}
      \end{pmatrix}
      =
      \begin{pmatrix}
        b_{\sigma(1),1} & b_{\sigma(1),2} & \cdots & b_{\sigma(1),m} \\
        \vdots          & \vdots          & \ddots & \vdots \\
        b_{\sigma(n),1} & b_{\sigma(n),2} & \cdots & b_{\sigma(n),m}
      \end{pmatrix}.
    \end{equation*}

    The inverse is the matrix corresponding to its inverse permutation.

    \thmitem{def:elementary_matrix/scaling} For a nonzero element \( a \) and index \( i \), the \( n \times n \) \term{scaling matrix} \( S_{i \mapsto a} \) is a \hyperref[def:matrix_diagonal]{diagonal matrix} that differs from the identity by replacing \( 1 \) with \( a \) instead of \( 1 \) in the \( (i, i) \)-th place. The scaling matrix
    \begin{equation*}
      S_{i \mapsto a}
      \coloneqq
      \begin{blockarray}{*{7}{c} c}
        1      & \cdots & {i-1}   & i      & {i+1}  & \cdots & n      &        \\
      \begin{block}{(*{7}{c}) c}
        1      & \cdots & 0       & 0      & 0      & \cdots & 0      & 1      \\
        \vdots & \ddots &         & \vdots &        &        & \vdots & \vdots \\
        0      &        & 1       & 0      &        &        & 0      & {i-1}  \\
        0      & \cdots & 0       & a      & 0      & \cdots & 0      & i      \\
        0      &        &         & 0      & 1      &        & 0      & {i+1}  \\
        \vdots &        &         & \vdots &        & \ddots & \vdots & \vdots \\
        0      & \cdots & 0       & 0      & 0      & \cdots & 1      & n      \\
      \end{block}
      \end{blockarray}
    \end{equation*}
    acts on the \( n \times m \) matrix \( B \) by scaling the \( i \)-th row of \( B \) by \( a \).

    The inverse is the same matrix with \( a \) replaced by its multiplicative inverse \( a^{-1} \).

    \thmitem{def:elementary_matrix/transvection} For any element \( a \) and indices \( i \) and \( j \), the \term{transvection matrix} \( T_{i \reloset a \to j} \) is obtained from the identity matrix \( I_n \) by placing \( a \) on the \( (j, i) \)-th place. The transvection matrix
    \begin{equation*}
      T_{i \reloset a \to j}
      \coloneqq
      \begin{blockarray}{*{7}{c} c}
        1      & \cdots & i       &        &        &        & n      &        \\
      \begin{block}{(*{7}{c}) c}
        1      & \cdots & 0       & \cdots & 0      & \cdots & 0      & 1      \\
        \vdots & \ddots &         &        &        &        & \vdots &        \\
        0      &        & 1       & 0      &        & \cdots & 0      &        \\
        \vdots &        &         & \ddots & 0      &        & \vdots &        \\
        0      & \cdots & a       &        & 1      & \cdots & 0      & j      \\
        \vdots &        & \vdots  &        &        & \ddots & \vdots & \vdots \\
        0      & \cdots & 0       & \cdots & 0      & \cdots & 1      & n      \\
      \end{block}
      \end{blockarray}
    \end{equation*}
    acts on the \( n \times m \) matrix \( B \) by adding the \( i \)-th row of \( B \) scaled by \( a \) to the \( j \)-th row.

    The order of indices is important --- if the \( (j, i) \)-th entry is nonzero, the scaled \( i \)-th row gets added to the \( j \)-th.

    The inverse is the same matrix with \( a \) replaced by its additive inverse \( -a \).
  \end{thmenum}
\end{definition}

\begin{proposition}\label{thm:def:elementary_matrix}
  \hyperref[def:elementary_matrix]{Elementary matrices} have the following basic properties:
  \begin{thmenum}
    \thmitem{thm:def:elementary_matrix/permutation_product} The product of \hyperref[def:elementary_matrix/permutation]{permutation matrices} is a permutation matrix.

    \thmitem{thm:def:elementary_matrix/permutation_determinant} The \hyperref[def:matrix_determinant]{determinant} of a permutation matrix is the \hyperref[def:permutation_parity]{sign} of the \hyperref[def:symmetric_group]{permutation}.

    \thmitem{thm:def:elementary_matrix/transvection_product_same} The product of the \hyperref[def:elementary_matrix/transvection]{transvection matrices} \( T_{i \reloset \alpha \to j} \) and \( T_{i \reloset \beta \to j} \) is the transvection matrix \( T_{i \reloset {\alpha + \beta} \to j} \).

    \thmitem{thm:def:elementary_matrix/transvection_product} The product of the \hyperref[def:elementary_matrix/transvection]{transvection matrices} \( A = T_{i_A \reloset \alpha \to j_A} \) and \( B = T_{i_B \reloset \beta \to j_B} \) with \( i_A \neq i_B \) or \( j_A \neq j_B \) is the identity matrix \( I_n \) modified with \( \alpha \) in the \( (j_A, i_A) \)-th place and \( \beta \) in the \( (j_B, i_B) \)-th.
  \end{thmenum}
\end{proposition}
\begin{proof}
  \SubProofOf{thm:def:elementary_matrix/permutation_product} Trivial.

  \SubProofOf{thm:def:elementary_matrix/permutation_determinant} As discussed in our proof of \fullref{thm:determinant_on_columns}, the determinant of any permutation of the vectors of the standard basis is the sign of the permutation.

  \SubProofOf{thm:def:elementary_matrix/transvection_product} The \( (i, j) \)-th entry of the product \( C = AB \) is
  \begin{equation*}
    c_{i,j} = \sum_{k=1}^n a_{i,k} b_{k,j}.
  \end{equation*}

  \begin{itemize}
    \item If \( i = j \), \( c_{i,j} \) is clearly \( 1 \).
    \item If \( i = i_A \) and \( j = j_A \), then
    \begin{equation*}
      a_{i_A,k} b_{k,j_A} = \begin{cases}
        a_{i_A,j_A}, &i_A = j_A \\
        0,           &i_A \neq j_A
      \end{cases}.
    \end{equation*}

    Thus, \( c_{i_A,j_A} = a_{i_A,j_A} \).

    \item Analogously, \( c_{i_B,j_B} = b_{i_B,j_B} \).
    \item Otherwise, for \( k = 1, \ldots, n \), either \( a_{i,k} \) or \( b_{k,j} \) is zero, hence \( c_{i,j} \) also is.
  \end{itemize}

  \SubProofOf{thm:def:elementary_matrix/transvection_product_same} This proof only requires a slight modification to our proof of \fullref{thm:def:elementary_matrix/transvection_product}.
\end{proof}

\begin{algorithm}[PLU decomposition]\label{alg:plu_decomposition}
  Fix an \( n \times n \) matrix \( A \). We will build a \hyperref[def:triangular_matrix]{lower \hi{unitriangular} matrix} \( L \), \hyperref[def:upper_row_echelon_form]{upper row echelon form} \( U \) and a \hyperref[def:elementary_matrix/permutation]{permutation matrix} \( P \) such that \( A = PLU \).

  We will proceed via \hyperref[thm:bounded_transfinite_recursion]{bounded recursion} on \( n \). After the \( k \)-th step, for \( k = 1, \ldots, n - 1 \), we will have built a lower triangular matrix \( L_k \) and a permutation matrix \( P_k \) such that for \( i > k \), \( (i, k) \)-th entry of \( L_k P_k A \) is zero.

  Furthermore, we will obtain \( L_k \) as a product of \hyperref[def:elementary_matrix/transvection]{transvection} and permutation matrices. Therefore, at each step, both \( P_k \) and \( L_k \) will be \hyperref[def:inverse_matrix]{invertible} as products of invertible matrices.

  The matrix \( U \coloneqq L_{n-1} P_{n-1} A \) will be upper triangular, and hence, putting \( P \coloneqq P_{n-1}^{-1} \) and \( L \coloneqq L_{n-1}^{-1} \), we obtain
  \begin{equation*}
    A = PLU.
  \end{equation*}

  \begin{thmenum}
    \thmitem{alg:plu_decomposition/init} As an initial condition, put \( L_0 \coloneqq I_n \) and \( P_0 \coloneqq I_n \) as identity matrices.

    \thmitem{alg:plu_decomposition/step} Suppose that we have already built \( L_{k-1} \) and \( P_{k-1} \). Let \( U_{k-1} \coloneqq L_{k-1} P_{k-1} A \). We will describe step \( k \) of the algorithm.

    If the \( (k, j) \)-th entry of \( U_{k-1} \) is zero for all \( j > k \), put \( P_k = P_{k-1} \) and \( L_k = L_{k-1} \).

    Otherwise, let \( j_0 \) be the first row index of \( L_{k-1} P_{k-1} A \) for which the \( k \)-th entry is nonzero. Let \( P_{k \to j_0} \) be the permutation matrix exchanging the \( k \)-th and \( j_0 \)-th column of the identity and put
    \begin{equation*}
      P_k \coloneqq P_{k \to j_0} P_{k-1}.
    \end{equation*}

    Then, since \( P_{k \to j_0} \) is its own inverse,
    \begin{equation*}
      \widehat{U}_{k-1} \coloneqq P_{k \to j_0} U_{k-1} = P_{k \to j_0} L_{k-1} (\smash{ \overbrace{P_{k \to j_0} P_{k \to j_0}}^{I_n} P_{k-1}) A = (P_{k \to j_0} L_{k-1} P_{k \to j_0}) } P_k A.
    \end{equation*}

    Denote by \( u_{i,j} \) the entries of \( \widehat{U}_{k-1} \).

    Also put \( \widehat{L}_{k-1} \coloneqq P_{k \to j_0} L_{k-1} P_{k \to j_0} \). This is again a lower triangular matrix since we only swap columns below the main diagonal.

    For each row \( j > k \), define \( \upsilon_j \coloneqq - u_{k,j} / u_{k,k} \) consider the transvection matrix \( T_{k \reloset {\upsilon_j} \to j} \). When multiplied by \( \widehat{U}_{k-1} \) from the right, it adds the \( k \)-th row of \( \widehat{U}_{k-1} \) to the \( j \)-th after multiplying it by \( \upsilon_j \). Hence, \( T_{k \reloset {\upsilon_j} \to j} \widehat{U}_{k-1} \) has zero at as its \( (i, j) \)-th entry.

    Finally, put
    \begin{equation*}
      L_k \coloneqq \parens*{ \prod_{j=k}^n T_{k \reloset {\upsilon_j} \to j} } \widehat{L}_{k-1}.
    \end{equation*}

    By \fullref{thm:def:elementary_matrix/transvection_product}, \( L_k \) adds nonzero entries to \( \widehat{L}_{k-1} \) only below the main diagonal. Since \( \widehat{L}_{k-1} \) is lower triangular, so is \( L_k \). Furthermore, for \( j > k \), the coefficient \( \upsilon_j \) is chosen so that the \( (k, j) \)-th entry of \( L_k P_k A \) of zero, which ensures that the latter matrix will be upper triangular when \( k = n - 1 \).
  \end{thmenum}
\end{algorithm}
\begin{comments}
  \item \todo{Compare with Gaussian elimination}
  \item \todo{Handle non-square matrices}
  \item \todo{Ensure that \( U \) is in row echelon form}.
\end{comments}

\begin{proposition}\label{thm:alg:plu_decomposition}
  Let \( A = PLU \) be the decomposition of some matrix \( A \) obtained via \fullref{alg:plu_decomposition}.

  \begin{thmenum}
    \thmitem{thm:alg:plu_decomposition/upper_triangular} If \( A \) is upper triangular, then \( P = L = I_n \) and \( A = U \).

    \thmitem{thm:alg:plu_decomposition/nonsingular} \( A \) is \hyperref[def:inverse_matrix]{nonsingular} if and only if \( U \) is nonsingular.
  \end{thmenum}
\end{proposition}
\begin{proof}
  \SubProofOf{thm:alg:plu_decomposition/upper_triangular} Suppose that \( A \) is upper triangular. Then, at step \( k \) of the algorithm:
  \begin{itemize}
     \item If \( u_{k,k} \) is zero, then all entries below it are also zero, and hence we directly continue to the next step.
     \item If \( u_{k,k} \) is not zero, then the transvection matrices \( T_{k \reloset 0 \to j} \) for \( j > k \) are all identity matrices, and hence \( L_k \) is also the identity.
  \end{itemize}

  In both cases, \( L_k = P_k = I_n \).

  \SubProofOf{thm:alg:plu_decomposition/nonsingular} By \fullref{thm:def:matrix_determinant/homomorphism},
  \begin{equation*}
    \det(A) = \det(P) \det(L) \det(U).
  \end{equation*}

  Since \( P \) and \( L \) are products of permutation and transvection matrices, by \fullref{thm:def:elementary_matrix/permutation_determinant} and \fullref{thm:def:elementary_matrix/transvection_product}, their determinants are either \( 1 \) or \( -1 \). Hence,
  \begin{equation*}
    \abs{\det(A)} = \abs{\det(U)}.
  \end{equation*}

  It follows from \fullref{thm:def:elementary_matrix/transvection_product} that \( A \) is nonsingular if and only if \( U \) is.
\end{proof}

\begin{algorithm}[Elementary matrix decomposition]\label{alg:elementary_matrix_decomposition}
  Fix a \hyperref[def:inverse_matrix]{nonsingular} \( n \times n \) matrix \( A \). Let \( A = PLU \) be the decomposition obtained via \fullref{alg:plu_decomposition}. By \fullref{thm:alg:plu_decomposition/nonsingular}, \( U \) is a nonsingular matrix. Both \( P \) and \( L \) are products of elementary matrices, hence it suffices to show that \( U \) is a product of elementary matrices in order to show that \( A \) is a product of elementary matrices.

  The algorithm is complementary to \fullref{alg:plu_decomposition}, although with noticeable differences. We will assume that \( k = 2, \ldots, n \). At each step, we will build a matrix \( U_k \) whose first \( k \) columns match those of \( U \). Then \( U_n \) must equal \( U \).

  Denote by \( u_{i,j} \) the entries of \( U \).

  \begin{thmenum}
    \thmitem{alg:elementary_matrix_decomposition/initialization} We will define the initial condition \( U_1 \) to be he diagonal matrix whose main diagonal matches that of \( U \). This can be achieved via \hyperref[def:elementary_matrix/scaling]{scaling matrices}:
    \begin{equation*}
      U_1 \coloneqq \prod_{i=1}^n S_{i \mapsto u_{i,i}}.
    \end{equation*}

    \thmitem{alg:elementary_matrix_decomposition/step} At step \( k \), given \( U_{k-1} \), for \( j < k \) define \( \upsilon_j \coloneqq u_{k,j} / u_{k,k} \). It is important that here, unlike in \fullref{alg:plu_decomposition}, we put no minus sign in \( \upsilon_j \) since we are building the matrix \( U \) directly rather than building an intermediate matrix that we will later invert. Put
    \begin{equation*}
      U_k \coloneqq \parens*{ \prod_{j=k}^n T_{k \reloset {\upsilon_j} \to j} } U_{k-1}.
    \end{equation*}

    Since \( U \) is nonsingular, by \fullref{thm:def:triangular_matrix/determinant}, \( u_{k,k} \) must be nonzero. Hence, we can divide by it.

    As a product of scaling and \hyperref[def:elementary_matrix/transvection]{transvection matrices} with nonzero entries above the main diagonal, \( U_k \) is an upper diagonal matrix. Furthermore, the scaling matrices neutralize the division done by the transvection matrices, so the \( k \)-th column of \( U_k \) and \( U \) must match.
  \end{thmenum}
\end{algorithm}

\begin{proposition}\label{thm:product_of_elementary_matrices_iff_invertible}
  The square matrix over a field is \hyperref[def:inverse_matrix]{invertible} if and only if it is a product of \hyperref[def:elementary_matrix]{elementary matrices}.
\end{proposition}
\begin{proof}
  \SufficiencySubProof Follows from \fullref{alg:elementary_matrix_decomposition}.
  \NecessitySubProof Follows from \fullref{thm:def:matrix_determinant/homomorphism}.
\end{proof}

\begin{example}\label{ex:vandermonde_matrix}\mcite[corr. 2.37]{Knapp2016BasicAlgebra}
  Given elements \( r_0, r_1, \ldots, r_n \) of some commutative ring, we define their \term{Vandermonde matrix} as
  \begin{equation*}
    \RenewDocumentCommand \arraystretch {} {1.3}
    V_n(r_0, r_1, \ldots, r_n)
    \coloneqq
    \begin{pmatrix}
      r_0^0  & r_0^1  & r_0^2  & \cdots & r_0^n  \\
      r_1^0  & r_1^1  & r_1^2  & \cdots & r_1^n  \\
      \vdots & \vdots & \vdots & \ddots & \vdots \\
      r_n^0  & r_n^1  & r_n^2  & \cdots & r_n^n
    \end{pmatrix}.
  \end{equation*}

  Having in mind that, by \fullref{thm:def:triangular_matrix/determinant}, transvection matrices have determinant \( 1 \), subtracting the \( k \)-th row multiplied by \( r_0 \) from the \( (k + 1) \)-th does not change the determinant. We thus have
  \begin{balign*}
    \det V_n
    =
    \det V_n^T
    &=
    \det
    \begin{pmatrix}
      1         & 1                     & 1                     & \cdots & 1                     \\
      0         & r_1 - r_0             & r_2 - r_0             & \cdots & r_n - r_0             \\
      0         & r_1 (r_1 - r_0)       & r_2 (r_2 - r_0)       & \cdots & r_n (r_n - r_0)       \\
      \vdots    & \vdots                & \vdots                & \ddots & \vdots                \\
      0         & r_1^{k-1} (r_1 - r_0) & r_2^{k-1} (r_2 - r_0) & \cdots & r_n^{k-1} (r_n - r_0) \\
      \vdots    & \vdots                & \vdots                & \ddots & \vdots                \\
      0         & r_1^{n-1} (r_1 - r_0) & r_2^{n-1} (r_2 - r_0) & \cdots & r_n^{n-1} (r_n - r_0) \\
    \end{pmatrix}
    \reloset {\ref{thm:laplace_expansion}} = \\ &=
    \det
    \begin{pmatrix}
      r_1 - r_0             & r_2 - r_0             & \cdots & r_n - r_0             \\
      r_1 (r_1 - r_0)       & r_2 (r_2 - r_0)       & \cdots & r_n (r_n - r_0)       \\
      \vdots                & \vdots                & \ddots & \vdots                \\
      r_1^{k-1} (r_1 - r_0) & r_2^{k-1} (r_2 - r_0) & \cdots & r_n^{k-1} (r_n - r_0) \\
      \vdots                & \vdots                & \ddots & \vdots                \\
      r_1^{n-1} (r_1 - r_0) & r_2^{n-1} (r_2 - r_0) & \cdots & r_n^{n-1} (r_n - r_0) \\
    \end{pmatrix}
    = \\ &=
    (r_1 - r_0) (r_2 - r_0) \cdots (r_n - r_0)
    \det
    \begin{pmatrix}
      1         & 1         & \cdots & 1         \\
      r_1       & r_2       & \cdots & r_n       \\
      \vdots    & \vdots    & \ddots & \vdots    \\
      r_1^{k-1} & r_2^{k-1} & \cdots & r_n^{k-1} \\
      \vdots    & \vdots    & \ddots & \vdots    \\
      r_1^{n-1} & r_2^{n-1} & \cdots & r_n^{n-1}
    \end{pmatrix}.
  \end{balign*}

  Proceeding by induction, we conclude that
  \begin{equation}\label{eq:ex:vandermonde_matrix/determinant}
    \det V_n = \prod_{i < j} (r_j - r_i).
  \end{equation}

  Hence, the determinant is nonzero if and only if all of \( r_0, \ldots, r_n \) are distinct.
\end{example}

\paragraph{Row and column spaces}

\begin{definition}\label{def:column_and_row_spaces}\mcite[39; 41]{Knapp2016BasicAlgebra}
  The \term{column space} of the \( m \times n \) matrix \( A \) is the \hyperref[def:module/submodel]{linear span} of the columns of \( A \), regarded as a subspace of \( R^n \).

  Analogously, the \term{row space} is the span of the rows, regarded as a subspace of \( R^m \).

  We define the \term{column rank} (resp. \term{row rank}) of \( A \) as the dimension of the column space (resp. row space).
\end{definition}
\begin{comments}
  \item The column space is the image of \( A \) when regarded as a linear operator, while the row space is the image of \( A^T \).

  \item In accordance with \fullref{def:rank_and_nullity}, we should refer to the column rank as \enquote{the rank}. We make a distinction between column and row ranks because the two are not in general equal. \Fullref{thm:matrix_ranks_over_field_coincide} implies that over field they coincide, however, and in that case it makes sense to conflate them.
\end{comments}

\begin{lemma}\label{thm:matrix_product_column_space_subspace}
  The column space of \( AB \) is a subspace of the column space of \( A \). If \( B \) is invertible, the column spaces coincide.
\end{lemma}
\begin{proof}
  Let
  \begin{align*}
    A = \seq{ a_{i,j=1}^{m,n} }
    &&
    B = \seq{ a_{i,j=1}^{n,k} }
    &&
    AB = \seq{ c_{i,j=1}^{m,k} }.
  \end{align*}

  Then, by definition of multiplication,
  \begin{equation*}
    c_{i,j} = \sum_{s=1}^n a_{i,s} \cdot b_{s,j}.
  \end{equation*}

  The \( j \)-th column of \( AB \) is thus
  \begin{equation*}
    \begin{pmatrix}
      c_{1,j} \\
      \vdots \\
      c_{m,j}
    \end{pmatrix}
    =
    \begin{pmatrix}
      \sum_{s=1}^n a_{1,s} \cdot b_{s,j} \\
      \vdots \\
      \sum_{s=1}^n a_{m,s} \cdot b_{s,j}
    \end{pmatrix}
    =
    \sum_{s=1}^n b_{s,j}
    \begin{pmatrix}
      a_{1,s} \\
      \vdots \\
      a_{m,s}
    \end{pmatrix}
    =
    \sum_{s=1}^n b_{s,j} A_{\Anon*,s}.
  \end{equation*}

  It is thus a linear combination of the columns of \( A \), hence it belongs to the column space of \( A \). Therefore, the entire column space of \( AB \) is a subspace of the column space of \( A \).

  If \( B \) is invertible, we can apply the proposition to \( AB \) and \( B^{-1} \) to conclude that the column space of \( ABB^{-1} = A \) is a subspace of that of \( AB \).
\end{proof}

\begin{proposition}\label{thm:matrix_ranks_over_field_coincide}
  The row and column ranks of a matrix over a field coincide.
\end{proposition}
\begin{proof}
  \todo{Prove}.
\end{proof}

\paragraph{Systems of linear equations}

\begin{definition}\label{def:matrix_equation}\mimprovised
  A fundamental aspect of linear algebra are \term{matrix equations} --- \hyperref[def:equation]{equation} whose sides are \hyperref[def:array/matrix]{matrices} of identical dimensions.
\end{definition}

\begin{definition}\label{def:system_of_linear_equations}\mimprovised
  \hyperref[def:equation/system]{Systems} of \hyperref[def:polynomial_degree_terminology]{linear} \hyperref[def:algebraic_equation]{algebraic equation} have an intimate connection to a kind of \hyperref[def:matrix_equation]{matrix equations}.

  Consider the system
  \begin{equation}\label{eq:def:system_of_linear_equations/scalar_form}
    \begin{cases}
      \begin{array}{ccccccc}
        a_{1,1} x_1 & +      & \cdots & +      & a_{1,n} x_n & =      & b_1 \\
                    & \vdots &        & \vdots &             & \vdots &     \\
        a_{m,1} x_1 & +      & \cdots & +      & a_{m,n} x_n & =      & b_m
      \end{array}
    \end{cases}
  \end{equation}
  with coefficients over the \hyperref[def:semiring]{(semi)ring} \( r \).

  We can arrange the coefficients into matrices:
  \begin{align*}
    A \coloneqq \begin{pmatrix}
      a_{1,1} & \cdots & a_{1,n} \\
      \vdots  & \ddots & \vdots  \\
      a_{m,1} & \cdots & a_{m,n}
    \end{pmatrix}
    &&
    b \coloneqq \begin{pmatrix}
      b_1 \\ \cdots \\ b_m
    \end{pmatrix}
  \end{align*}

  Despite both containing coefficients of algebraic equations, we call \( A \) the \term[ru=матрица коеффициентов (\cite[\S 3.1]{Тыртышников2007ЛинейнаяАлгебра})]{coefficient matrix} and refer to \( b \) simply as the \term[ru=правая часть (\cite[\S 3.1]{Тыртышников2007ЛинейнаяАлгебра})]{right side}. These matrices allow us to express \eqref{eq:def:system_of_linear_equations/scalar_form} succinctly in matrix form:
  \begin{equation}\label{eq:def:system_of_linear_equations/matrix_form}
    Ax = b,
  \end{equation}
  where \( x \) is an indeterminate \hyperref[def:array/column_vector]{column vector} form \( R^m \).

  \begin{thmenum}
    \thmitem{def:system_of_linear_equations/homogeneous} If all linear equations are \hyperref[def:homogeneous_equation]{homogeneous}, we refer to the system itself as \term[ru=однородная (система алгебраических уравнений) (\cite[\S 7.7]{Тыртышников2007ЛинейнаяАлгебра})]{homogeneous}.

    \Fullref{thm:homogeneous_linear_equation} implies that, equivalently, a system is homogeneous if \( b \) is the zero vector.

    To every inhomogeneous system there corresponds a homogeneous system --- this is used when characterizing solutions in \fullref{thm:linear_system_solutions}.

    \thmitem{def:system_of_linear_equations/solution_space} We call the \hyperref[def:equation/solution]{solution set} of a linear system a \term{solution space} --- as we shall see in \fullref{thm:linear_system_solution_space}, it is in fact an \hyperref[def:affine_subspace]{affine subspace} of \( R^m \).
  \end{thmenum}
\end{definition}

\begin{definition}\label{def:minkowski_sum}\mcite[\S 11.1.3]{Berger1987GeometryI}
  The \term{Minkowski sum} of two sets \( A \) and \( B \) in a \hyperref[def:module]{module} is simply their sum in the additive \hyperref[def:power_semigroup]{power group}:
  \begin{equation}\label{eq:def:minkowski_sum}
    A + B \coloneqq \set{ a + b \given a \in A \T{and} b \in B }.
  \end{equation}
\end{definition}

\begin{proposition}\label{thm:linear_system_solutions}
  Let \( x_0 \) be a \hyperref[def:equation/solution]{solution} of the \hyperref[def:system_of_linear_equations]{system of linear equations} \( Ax = b \) over a ring \( R \).

  Then the entire set of solutions is the \hyperref[def:minkowski_sum]{Minkowski sum} of \( x_0 \) with the \hyperref[def:module/kernel]{null space} of \( A \).
\end{proposition}
\begin{comments}
  \item The theorem can be restated as follows: every solution of an inhomogeneous system is a sum of one fixed solution and some solution of the corresponding homogeneous system.

  \item In particular, if the system is homogeneous, the null space is the solution set.
\end{comments}
\begin{proof}
  All elements of \( x_0 + \ker A \) are solutions to \( Ax = b \). Indeed, if \( x = x_0 + y \), where \( Ay = \vect 0 \), then \( Ax = Ax_0 + Ay = Ax_0 \).

  Conversely, if \( x \) is a solution, then \( A(x - x_0) = \vect 0 \), and thus \( x - x_0 \) is in the null space. Then \( x \) is the sum of \( x_0 \) and the element \( x - x_0 \) of \( \ker A \).
\end{proof}

\begin{corollary}\label{thm:homogeneous_linear_equations_solutions}
  The \( n \times n \) square \hyperref[def:system_of_linear_equations/homogeneous]{homogeneous linear system}
  \begin{equation*}
    Ax = \vect 0
  \end{equation*}
  has a nontrivial solution if and only if \( A \) is \hi{not} \hyperref[def:inverse_matrix]{invertible}.
\end{corollary}
\begin{proof}
  Due to \fullref{thm:group_homomorphism_trivial_kernel}, \( A \) is invertible if and only if the null space of \( A \) is trivial.
\end{proof}

\begin{corollary}\label{thm:system_of_equations_unique_solution}
  The \hyperref[def:system_of_linear_equations]{system of linear equations} \( Ax = b \) has a unique solution if and only if \( A \) is invertible.
\end{corollary}
\begin{proof}
  \SufficiencySubProof If \( Ax = b \) has a unique solution, then the solution space has dimension zero, the kernel of \( A \) is empty, and, by \fullref{thm:group_homomorphism_trivial_kernel}, \( A \) is invertible.

  \NecessitySubProof If \( A \) is invertible, then \( x = A^{-1} b \) is a solution.
\end{proof}

\begin{corollary}\label{thm:linear_system_solution_space}
  The \hyperref[def:system_of_linear_equations/solution_space]{solution space} of the \( m \times n \) \hyperref[def:system_of_linear_equations]{system of linear equations} \( Ax = b \) is either empty or an \hyperref[def:affine_subspace]{affine subspace} of \( R^m \) of dimension \( m - \rank A \).

  If the system is homogeneous, the solution set is a \hyperref[def:module/submodel]{linear subspace}.
\end{corollary}
\begin{proof}
  Fix a solution \( x_0 \). From \fullref{thm:linear_system_solutions} it follows that the solution space is \( x_0 + \ker A \). It thus satisfies the definition of affine subspace in \fullref{def:affine_subspace}.

  Finally, \fullref{thm:rank_nullity_theorem} implies that
  \begin{equation*}
    m = \dim \ker \varphi + \underbrace{ \rank A }_{ \dim \img A }.
  \end{equation*}
\end{proof}

\begin{theorem}[Kroneker-Capelli theorem]\label{thm:kroneker_capelli}
  The \( m \times n \) \hyperref[def:system_of_linear_equations]{linear system}
  \begin{equation*}
    Ax = b
  \end{equation*}
  has a solution if and only if the \hyperref[def:rank_and_nullity]{rank} of its coefficient matrix \( A \) is equal to the rank of the augmented matrix
  \begin{equation*}
    \parens*
    {
      \begin{array}{c|c}
        A & b
      \end{array}
    }.
  \end{equation*}
\end{theorem}
\begin{comments}
  \item The names of Kroneker and Capelli are associated with this theorem in Russophone literature --- for example \incite[\S 7.8]{Тыртышников2007ЛинейнаяАлгебра}, \incite[thm. 2.3.1]{Винберг2014Алгебра}, \incite[thm. IV.4.4]{Фаддеев1984Алгебра} and \incite[thm. 2.2.2]{Кострикин2000АлгебраЧасть1}.
\end{comments}
\begin{proof}
  If both matrices have the same rank, \( b \) belongs to the column space of \( A \). Then there exists some linear combination of the columns of \( A \) that equals \( b \). The coefficients of every such linear combination form a solution to the system.
\end{proof}

  \subsection{Bilinear forms}\label{subsec:bilinear_forms}

In this subsection, we restrict ourselves to fields rather than arbitrary rings.

We define \hyperref[def:bilinear_form]{bilinear forms} over arbitrary fields, although they are almost exclusively used over the field of real numbers. In the latter case, they are a special case of \hyperref[def:sesquilinear_form]{sesquilinear forms} over complex numbers. This relationship is made precise via \hyperref[def:complexification]{complexification}. This is discussed further in \fullref{rem:real_field_extensions}.

\begin{definition}\label{def:bilinear_form}\mcite[249]{Knapp2016BasicAlgebra}
  A \term{bilinear form} over the vector space \( V \) over \( \BbbK \) is a bilinear form is a \hyperref[def:multilinear_function]{multilinear function} with signature \( L: V \times V \to \BbbK \).
\end{definition}

\begin{remark}\label{rem:matrices_as_bilinear_forms}
  Similarly to what we discussed in \fullref{rem:matrices_as_functions}, matrices correspond to linear functions and vice versa. Square matrices correspond to \hyperref[def:bilinear_form]{bilinear forms}.

  Let \( e_1, \ldots, e_n \) be the \hyperref[def:sequence_space]{standard basis} of \( \BbbK^n \). This basis allows us to identify vectors of \( \BbbK^n \) with column vectors. To every \( n \times n \) matrix \( A \), there corresponds a bilinear form
  \begin{equation*}
    L_A(x, y) \coloneqq x^T A y.
  \end{equation*}

  Similarly, given a bilinear form \( L \), we can build the following matrix:
  \begin{equation*}
    A_L \coloneqq
    \begin{pmatrix}
      L(e_1, e_1) & \cdots & L(e_1, e_n) \\
      \vdots      & \ddots & \vdots      \\
      L(e_n, e_1) & \cdots & L(e_n, e_n)
    \end{pmatrix}.
  \end{equation*}

  This matrix is called the generalized \term{Gram matrix} corresponding to \( L \).
\end{remark}

\begin{proposition}\label{thm:symmetric_bilinear_form_matrix}
  The \hyperref[rem:matrices_as_bilinear_forms]{Gram matrix} of a \hyperref[def:symmetric_function]{symmetric} \hyperref[def:bilinear_form]{bilinear form} is \hyperref[def:transpose_matrix]{symmetric}.
\end{proposition}
\begin{proof}
  Trivial.
\end{proof}

\begin{definition}\label{def:bilinear_form_radicals}\mcite[250]{Knapp2016BasicAlgebra}
  Let \( L: V \times V \to \BbbK \) be a bilinear form. We define its \term{left radical}
  \begin{equation*}
    \set{ x \in V \given \qforall {y \in V} L(x, y) = 0 }
  \end{equation*}
  and its \term{right radical}
  \begin{equation*}
    \set{ y \in V \given \qforall {x \in V} L(x, y) = 0 }
  \end{equation*}

  Note that if \( L \) is symmetric or skew-symmetric, the two radicals are identical, and we speak simply of the \term{radical} \( \sqrt L \).
\end{definition}

\begin{definition}\label{def:degenerate_bilinear_form}\mcite[249]{Knapp2016BasicAlgebra}
  We say that a \hyperref[def:bilinear_form]{bilinear form} is \term{degenerate} if either its left or right \hyperref[def:bilinear_form_radicals]{radical} is not trivial.
\end{definition}

\begin{example}\label{ex:def:bilinear_form}\hfill
  \begin{thmenum}
    \thmitem{ex:def:bilinear_form/asymmetric_degenerate} The matrix
    \begin{equation*}
      \begin{pmatrix}
        0 & 1 \\
        0 & 0
      \end{pmatrix}
    \end{equation*}
    corresponds to a \hyperref[def:degenerate_bilinear_form]{degenerate} \hyperref[def:bilinear_form]{bilinear form}. Its \hyperref[def:bilinear_form_radicals]{left radical} is
    \begin{equation*}
      \set*{ \begin{pmatrix} 0 \\ r \end{pmatrix} \given* r \in \BbbK }.
    \end{equation*}

    Its right radical is
    \begin{equation*}
      \set*{ \begin{pmatrix} r \\ 0 \end{pmatrix} \given* r \in \BbbK }.
    \end{equation*}

    \thmitem{ex:def:bilinear_form/symmetric_degenerate} The matrix
    \begin{equation*}
      \begin{pmatrix}
        1 & 0 \\
        0 & 0
      \end{pmatrix}
    \end{equation*}
    is also degenerate. It is symmetric, however, and its left and right radicals coincide with
    \begin{equation*}
      \set*{ \begin{pmatrix} 0 \\ r \end{pmatrix} \given* r \in \BbbK }.
    \end{equation*}

    \thmitem{ex:def:bilinear_form/euclidean} The identity matrix induces the nondegenerate bilinear form \( (x, y) \mapsto x^T y \). It is called the \term{Euclidean product}.
  \end{thmenum}
\end{example}

\begin{definition}\label{def:homogeneous_polynomial}\mimprovised
  We say that a \hyperref[def:polynomial_algebra/polynomials]{polynomial} is \term{homogeneous} of degree \( n \) if all of its monomials have degree \( n \).
\end{definition}

\begin{definition}\label{def:homogenous_function}\mimprovised
  We say that the function \( f: V \to \BbbK \) is \term{homogeneous} of degree \( n \) if
  \begin{equation*}
    f(t x) = t^n f(x).
  \end{equation*}

  This is a generalization of \hyperref[eq:def:semimodule/homomorphism/homogeneity]{homogeneity} used in the definition of linear maps.
\end{definition}

\begin{proposition}\label{thm:homogeneous_polynomial_is_homogeneous_function}
  A \hyperref[def:homogeneous_polynomial]{homogeneous polynomial} is a \hyperref[def:homogenous_function]{homogeneous function} of the same degree.
\end{proposition}
\begin{proof}
  Trivial.
\end{proof}

\begin{proposition}\label{thm:polarization_identity}\mcite[91]{Knapp2016BasicAlgebra}
  Let \( L: V \times V \to \BbbK \) be a \hyperref[def:symmetric_function]{symmetric} \hyperref[def:bilinear_form]{bilinear form} and define \( Q(x) \coloneqq L(x, x) \). Then the \term{polarization identity} holds:
  \begin{equation}\label{thm:polarization_identity/polarization_identity}
    Q(x + y) - Q(x - y) = 2 L(x, y)
  \end{equation}

  The similar looking, but slightly less useful parallelogram law also holds:
  \begin{equation}\label{thm:polarization_identity/parallelogram_law}
    Q(x + y) + Q(x - y) = 2 Q(x) + 2 Q(y)
  \end{equation}

  We can also \enquote{recover} \( L \) from \( Q \):
  \begin{equation}\label{thm:polarization_identity/definition}
    L(x, y) \coloneqq \frac 1 2 \bracks{ Q(x + y) - Q(x) - Q(y) }.
  \end{equation}
\end{proposition}
\begin{proof}
  The identities all follow from the bilinearity of \( L \):
  \begin{equation*}
    Q(x \pm y)
    =
    L(x, x) \pm L(x, y) \pm L(y, x) + L(y, y)
    =
    [Q(x) + Q(y)] \pm [L(x, y) + L(y, x)].
  \end{equation*}
\end{proof}

\begin{proposition}\label{thm:quadratic_forms}
  There is a bijective correspondence between \hyperref[def:symmetric_function]{symmetric} \hyperref[def:bilinear_form]{bilinear forms} and \hyperref[def:homogeneous_polynomial]{homogeneous} \hyperref[def:polynomial_degree]{quadratic polynomials}.

  A \term{quadratic form} \( Q: V \to \BbbK \) is defined as either \( Q(x) \coloneqq L(x, x) \) for a symmetric bilinear form \( L \), or as the \hyperref[thm:polynomial_algebra_universal_property]{polynomial function} of a homogeneous quadratic polynomial.

  \hi{Real} quadratic forms are simply quadratic forms over \( \BbbK = \BbbR \). \hi{Complex} quadratic forms, by contrast, differ from the general theory --- see \fullref{rem:complex_quadratic_form}.
\end{proposition}
\begin{proof}
  First, let \( L: V \times V \to \BbbK \) be a bilinear form. Let \( e_k, k \in \mscrK, \) be a \hyperref[def:hamel_basis]{basis} of \( V \). Define the polynomial
  \begin{equation*}
    p_L(X_k \given k \in \mscrK) \coloneqq \sum_{i \in \mscrK} \sum_{j \in \mscrK} L(e_i, e_j) X_i X_j.
  \end{equation*}

  Conversely, let \( p(X_k \given k \in \mscrK) \) be a homogeneous quadratic polynomial over the set of indeterminates \( \mscrX \). Via \eqref{thm:polarization_identity/definition}, we can define
  \begin{equation*}
    L_p(x, y) = \frac 1 2 \bracks{ p(x + y) - p(x) - p(y) },
  \end{equation*}
  where \( x \) and \( y \) are vectors from \( V \) (i.e. \( \mscrX \)-indexed tuples).

  Given a symmetric bilinear form \( L \), we have
  \begin{balign*}
    &\phantom{{}={}}
    L_{p_L}(x, y)
    = \\ &=
    \frac 1 2 \parens*{ \sum_{i \in \mscrK} \sum_{j \in \mscrK} L(e_i, e_j) (x_i + y_i) (x_j + y_j) - \sum_{i \in \mscrK} \sum_{j \in \mscrK} L(e_i, e_j) x_i x_j - \sum_{i \in \mscrK} \sum_{j \in \mscrK} L(e_i, e_j) y_i y_j }
    = \\ &=
    \frac 1 2 \sum_{i \in \mscrK} \sum_{j \in \mscrK} L(e_i, e_j) x_i y_j + \frac 1 2 \sum_{i \in \mscrK} \sum_{j \in \mscrK} L(e_i, e_j) y_i x_j
    = \\ &=
    \sum_{j \in \mscrK} L\parens*{ \sum_{i \in \mscrK} x_i e_i, \sum_{j \in \mscrK} y_j e_j }
    = \\ &=
    L(x, y).
  \end{balign*}

  Conversely, given a homogeneous quadratic polynomial \( p \), we have
  \begin{balign*}
    &\phantom{{}={}}
    p_{L_p}(X_k \given k \in \mscrK)
    = \\ &=
    \sum_{i \in \mscrK} \sum_{j \in \mscrK} L_p(e_i, e_j) X_i X_j
    = \\ &=
    \sum_{i \in \mscrK} \sum_{j \in \mscrK} L_p(e_i, e_j) X_i X_j
    = \\ &=
    \frac 1 2 \sum_{i \in \mscrK} \sum_{j \in \mscrK} \bracks{ p(e_i + e_j) - p(e_i) - p(e_j) } X_i X_j.
  \end{balign*}

  By definition, all \hyperref[def:basis_decomposition]{coordinate projections} of \( e_i \) are zero except for the \( i \)-th coordinate, which is one. Hence, the value \( p(e_i) \) is the coefficient before \( X_i^2 \) in \( p \), and similarly for \( p(e_j) \). The value \( p(e_i + e_j) \) is the sum of coefficients before \( X_i^2 \), \( X_i X_j \) and \( X_j^2 \). Therefore,
  \begin{equation*}
    p(e_i + e_j) - p(e_i) - p(e_j)
  \end{equation*}
  is the coefficient before \( X_i X_j \). Furthermore, \( X_i X_j \) is equal to \( X_j X_i \), and so are their coefficients, which allows us to cancel \( 1 / 2 \) above. Thus,
  \begin{equation*}
    p_{L_p}(X_k \given k \in \mscrK) = p(X_k \given k \in \mscrK).
  \end{equation*}
\end{proof}

\begin{definition}\label{def:complexification}\mcite[def. 2.1]{Conrad2020Complexification}
  Let \( V \) be a real vector space. The \term{complexification} \( V^\BbbC \) of \( V \) is the \hyperref[def:semimodule_direct_sum]{direct sum} \( V \oplus V \) equipped with the canonical inclusion
  \begin{equation*}
    \begin{aligned}
      &\iota_V: V \to V \oplus V, \\
      &\iota_V(x) \coloneqq (x, 0_V).
    \end{aligned}
  \end{equation*}

  We regard it as a complex vector space by defining scalar multiplication as
  \begin{equation*}
    (a + bi) \cdot (x, y) = (ax - by, bx + ay).
  \end{equation*}

  To avoid working with ordered pairs, we use that
  \begin{equation*}
    (x, y) = \iota(x) + i \cdot \iota(y).
  \end{equation*}

  Conversely, given a complex vector space \( W \), we define its \term{decomplexification} \( W^\BbbR \) as the same underlying Abelian group but with scalars restricted to real numbers.

  Therefore, \( \iota \) embeds \( V \) into \( (V^\BbbC)^\BbbR = V \oplus V \).
\end{definition}

\begin{proposition}\label{thm:basis_of_complexification}
  If \( B \) is a basis of the real vector space \( V \), then
  \begin{equation*}
    B^\BbbC \coloneqq \set{ \iota(b) \given b \in B }
  \end{equation*}
  is a basis of its \hyperref[def:complexification]{complexification} \( V^\BbbC \).
\end{proposition}
\begin{proof}
  It is obvious that the vectors in \( B^\BbbC \) are linearly independent. We have to show that they span \( V^\BbbC \).

  Let \( (x, y) \) be a vector of \( V^\BbbC = V \oplus V \). Then
  \begin{equation*}
    (x, y)
    =
    \iota(x) + i \cdot \iota(y)
    =
    \iota\parens*{ \sum_{b \in B} \pi_b(x) \cdot b } + i \cdot \iota\parens*{ \sum_{b \in B} \pi_b(y) \cdot b }
    =
    \sum_{b \in B} \parens[\Big]{ \pi_b(x) + i \cdot \pi_b(y) } \cdot \iota(b).
  \end{equation*}
\end{proof}

\begin{theorem}[Complexification universal property]\label{thm:complexification_universal_property}
  The \hyperref[def:complexification]{complexification} \( V^\BbbC \) of a real vector space \( V \) satisfies the following \hyperref[rem:universal_mapping_property]{universal mapping property}:
  \begin{displayquote}
    For every complex vector space \( W \) and every real linear map \( T: V \to W^\BbbR \), there exists a unique complex linear map \( T^\BbbC: V^\BbbC \to W \) such that the following diagram commutes:
    \begin{equation}\label{eq:thm:complexification_universal_property/diagram}
      \begin{aligned}
        \includegraphics[page=1]{output/thm__complexification_universal_property}
      \end{aligned}
    \end{equation}
  \end{displayquote}

  Note that \( W^\BbbR \) and \( W \) have the same underlying abelian group but the scalar multiplication in \( W^\BbbR \) is restricted to real numbers.

  Via \fullref{rem:universal_mapping_property}, \( (\anon*)^\BbbC \) becomes \hyperref[def:category_adjunction]{left adjoint} to the \hyperref[def:complexification]{decomplexification} \hyperref[def:concrete_category]{forgetful functor}
  \begin{equation*}
    (\anon*)^\BbbR: \cat{Vect}_\BbbC \to \cat{Vect}_\BbbR.
  \end{equation*}
\end{theorem}
\begin{proof}
  Given \( T: V \to W^\BbbR \) and \( x \in V \), the map \( T^\BbbC \) must satisfy
  \begin{equation*}
    T^\BbbC(x) = Tx.
  \end{equation*}

  This suggests the only possible definition
  \begin{equation*}
    \begin{aligned}
      &T^\BbbC: V^\BbbC \to W, \\
      &T^\BbbC(x + i \cdot y) \coloneqq Tx + i \cdot Ty.
    \end{aligned}
  \end{equation*}
\end{proof}

\begin{definition}\label{def:antilinear_function}\mimprovised
  We say that the function \( L: V \to W \) between complex vector spaces is \term{antilinear} if it satisfies the additivity condition \eqref{eq:def:semimodule/homomorphism/additive} and if
  \begin{equation}\label{eq:def:antilinear_function}
    L(tx) = \overline t L(x).
  \end{equation}

  That is, we enhance the homogeneity condition \eqref{eq:def:semimodule/homomorphism/homogeneity} from \fullref{def:semimodule/homomorphism} with \hyperref[def:complex_numbers]{complex conjugation}.
\end{definition}

\begin{definition}\label{def:sesquilinear_form}\mcite[258]{Knapp2016BasicAlgebra}
  A \term{sesquilinear\fnote{\enquote{sesqui} is a Latin prefix meaning \enquote{one and a half}} form} over the \hyperref[def:complex_numbers]{complex} vector space \( V \) is a function \( L: V \times V \to \BbbC \) that is \hyperref[def:semimodule/homomorphism]{linear} in the first argument and \hyperref[def:antilinear_function]{antilinear} in the second.

  Unlike bilinear forms, we have \( L(x, ty) = \overline t L(x, y) \) rather than \( L(x, ty) = t L(x, y) \). Sesquilinear forms coincide with bilinear forms when restricted to real numbers.
\end{definition}

\begin{definition}\label{def:hermitian_form}\mcite[258]{Knapp2016BasicAlgebra}
  A \hyperref[def:sesquilinear_form]{sesquilinear form} \( L: V \times V \to \BbbC \) is called \term{Hermitian} if
  \begin{equation*}
    L(x, y) = \overline{L(y, x)}.
  \end{equation*}

  Hermitian forms coincide with symmetric forms when restricted to real numbers.

  This should not be confused with \hyperref[def:adjoint_operator]{Hermitian linear operators}.
\end{definition}

\begin{proposition}\label{thm:complexification_of_symmetric_bilinear_form}
  If \( L: V \times V \to \BbbR \) is a real \hyperref[def:symmetric_function]{symmetric} \hyperref[def:bilinear_form]{bilinear form}, then its \hyperref[def:complexification]{complexification}
  \begin{equation*}
    \begin{aligned}
      &L^\BbbC: V^\BbbC \times V^\BbbC \to \BbbC, \\
      &L^\BbbC\parens[\Big]{ \iota(x) + i \cdot \iota(y), \iota(u) + i \cdot \iota(v) } \coloneqq L(x, u) + L(y, v) - i \cdot L(x, v) + i \cdot L(u, y).
    \end{aligned}
  \end{equation*}
  is a \hyperref[def:hermitian_form]{Hermitian form}.

  Furthermore,
  \begin{equation*}
    L^\BbbC(\iota(x), \iota(y)) = L(x, y).
  \end{equation*}
\end{proposition}
\begin{proof}
  The additivity of \( L^\BbbC = L \oplus L \) follows from the additivity of \( L \).

  Note that
  \begin{equation*}
    (a + bi) (\iota(x) + i \cdot \iota(y)) = \iota(ax - by) + i \cdot \iota(bx + ay).
  \end{equation*}

  By the homogeneity and symmetry of \( L \), we have linearity in the first argument:
  \begin{balign*}
    &\phantom{{}={}}
    L^\BbbC \parens[\Big]{ (a + bi) \parens[\Big]{ \iota(x) + i \cdot \iota(y) }, \iota(u) + i \cdot \iota(v) }
    = \\ &=
    L(ax - by, u) + L(bx + ay, v) - i \cdot L(ax - by, v) + i \cdot L(u, bx + ay)
    = \\ &=
    a L(x, u) - b L(y, u) + b L(x, v) + a L(y, v) - ai L(x, v) + bi L(y, v) + bi L(u, x) + ai L(u, y)
    = \\ &=
    (a + bi) L(x, u) + i(a + bi) L(u, y) - i(a + bi) L(x, v) + (a + bi) L(y, v)
    = \\ &=
    (a + bi) \cdot L^\BbbC \parens[\Big]{ \iota(x) + i \cdot \iota(y), \iota(u) + i \cdot \iota(v) }
  \end{balign*}
  and antilinearity in the second:
  \begin{balign*}
    &\phantom{{}={}}
    L^\BbbC \parens[\Big]{ \iota(x) + i \cdot \iota(y), (a + bi) \parens[\Big]{ \iota(u) + i \cdot \iota(v) } }
    = \\ &=
    L(x, au - bv) + L(y, bu + av) - i \cdot L(x, bu + av) + i \cdot L(au - bv, y)
    = \\ &=
    a L(x, u) - b L(x, v) + b L(y, u) + a L(y, v) - bi L(x, u) - ai L(x, v) + ai L(u, y) - bi L(v, y)
    = \\ &=
    (a - bi) L(x, u) - i (a - bi) L(x, v) + i (a - bi) L(u, y) + (a - bi) L(y, v)
    = \\ &=
    (a - bi) \cdot L^\BbbC \parens[\Big]{ \iota(x) + i \cdot \iota(y), \iota(u) + i \cdot \iota(v) }.
  \end{balign*}

  The Hermitian property follows easily.
\end{proof}

\begin{definition}\label{def:conjugate_transpose}
  The \term{conjugate transpose} \( A^* \) of the matrix \( A \) over the complex numbers is the \hyperref[def:transpose_matrix]{transpose matrix} in which we take the complex conjugate of every entry.

  If \( A = A^* \), we say that the matrix is \term{Hermitian}.
\end{definition}

\begin{remark}\label{rem:complex_quadratic_form}
  \hyperref[thm:quadratic_forms]{Quadratic forms} can be defined for complex numbers in multiple ways. The obvious definition is to take the base field to be \( \BbbC \). The more popular definition is presented below.

  Let \( L: V \times V \to \BbbC \) be a \hyperref[def:hermitian_form]{Hermitian} \hyperref[def:sesquilinear_form]{sesquilinear form}. Exchanging its parameters yields
  \begin{equation*}
    L(x, x) = \overline {L(x, x)},
  \end{equation*}
  which ensures that \( Q(x) \coloneqq L(x, x) \) is always a real number. Thus, we regard \( Q \) as a function from \( V \) to \( \BbbR \). This is not a quadratic form in the sense of \fullref{thm:quadratic_forms}, but we will call it a \term{complex quadratic form}.

  The definition via homogeneous quadratic polynomials is no longer compatible.
\end{remark}

\begin{definition}\label{thm:quadratic_forms_definiteness}\mimprovised
  We say that the real or \hyperref[rem:complex_quadratic_form]{complex quadratic form} \( Q: V \to \BbbR \) is
  \begin{thmenum}
    \thmitem{thm:quadratic_forms_definiteness/positive_semidefinite} \term{positive semidefinite} if \( Q(x) \geq 0 \) for all \( x \in V \).
    \thmitem{thm:quadratic_forms_definiteness/negative_semidefinite} \term{negative semidefinite} if \( Q(x) \leq 0 \) for all \( x \in V \).
    \thmitem{thm:quadratic_forms_definiteness/positive_definite} \term{positive definite} if \( Q(x) > 0 \) for all \( x \neq 0_V \).
    \thmitem{thm:quadratic_forms_definiteness/negative_definite} \term{negative definite} if \( Q(x) < 0 \) for all \( x \neq 0_V \).
    \thmitem{thm:quadratic_forms_definiteness/indefinite} \term{indefinite} otherwise.
  \end{thmenum}

  The above terminology also applies to symmetric bilinear or Hermitian sesquilinear forms, since they can be used to obtain a quadratic form.
\end{definition}

\begin{proposition}\label{thm:quadratic_forms_are_nondegenerate}
  \hyperref[thm:quadratic_forms_definiteness]{Definite} \hyperref[thm:quadratic_forms]{quadratic forms} are \hyperref[def:degenerate_bilinear_form]{nondegenerate}.
\end{proposition}
\begin{proof}
  In the real case, the quadratic form is induced by some symmetric bilinear form
  \begin{equation*}
    L: V \times V \to \BbbR.
  \end{equation*}

  Suppose also that \( L \) is positive definite. Due to positive (resp. negative) definiteness, for every \( x \neq 0 \), \( L(x, y) \) is positive (resp. negative) when \( y = x \). Hence, \( L(x, y) = 0 \) for every \( y \) if and only if \( x = 0 \). Hence, \( L \) is nondegenerate.

  In the complex case, the quadratic form is induced by some Hermitian sesquilinear form
  \begin{equation*}
    L: V \times V \to \BbbC.
  \end{equation*}

  Again, \( L(x, y) \) is zero for all \( y \) when \( x = 0 \), hence \( L \) is nondegenerate.
\end{proof}

\begin{definition}\label{def:inner_product_space}\mimprovised
  A \term{real inner product space} is a vector space \( V \) over \( \BbbR \) equipped with a \hyperref[thm:quadratic_forms_definiteness/positive_definite]{positive definite} \hyperref[def:symmetric_function]{symmetric} \hyperref[def:bilinear_form]{bilinear form}
  \begin{equation*}
    \inprod \anon \anon: V \times V \to \BbbR.
  \end{equation*}

  A \term{complex inner product space} is a vector space \( V \) over \( \BbbC \) equipped with a \hyperref[thm:quadratic_forms_definiteness/positive_definite]{positive definite} \hyperref[def:hermitian_form]{Hermitian} \hyperref[def:sesquilinear_form]{sesquilinear form}
  \begin{equation*}
    \inprod \anon \anon: V \times V \to \BbbC.
  \end{equation*}

  This notation is generalized for application of linear functionals --- see \fullref{rem:dual_space_bilinear_form}.

  By default, for \( \BbbK^n \) we assume that the inner product is the \term{dot product} \( \inprod x y \coloneqq x^T y \) or, when working over complex inner product spaces, \( \inprod x y \coloneqq y^* x \).
\end{definition}

\begin{remark}\label{rem:structure_hierarchy}
  The hierarchy in \cref{fig:rem:structure_hierarchy} describes how, given any of the structures there, we automatically also have available all the ones to the bottom of it. For example, every real or complex inner product space is also a normed space, metric space and so forth. We will use this implicitly.

  \begin{figure}[!ht]
    \caption{Hierarchy of important mathematical structures}\label{fig:rem:structure_hierarchy}
    \smallskip
    \hfill
    \begin{forest}
      [
        {\hyperref[def:inner_product_space]{inner product}}, name=product
        [
          {\hyperref[def:norm]{norm}}, name=norm, edge label={node[midway,left]{\hyperref[def:bilinear_form_induced_norm]{induced norm}}}
          [
            {\hyperref[def:metric_space]{metric}}, name=metric, edge label={node[midway,left]{\hyperref[def:norm_induced_metric]{induced metric}}}
            [
              {\hyperref[def:uniform_space]{uniformity}}, name=uniformity, edge label={node[midway,left]{\hyperref[def:metric_uniformity]{metric uniformity}}}
              [
                {\hyperref[def:topological_space]{topology}}, name=topology, edge label={node[midway,left]{\hyperref[def:uniform_topology]{uniform topology}}}
              ]
            ]
          ]
        ]
      ]
    \end{forest}
    \hfill\hfill
  \end{figure}
\end{remark}

\begin{definition}\label{def:orthogonality}\mimprovised
  We say that the vectors \( x \) and \( y \) are \term{orthogonal} and write \( x \perp y \) if \( \inprod x y = 0 \). If, in addition, both vectors have unit norm, we say that they are \term{orthonormal}.
\end{definition}

\begin{remark}\label{rem:inner_product_basis_decomposition}
  Let \( V \) be an \hyperref[def:inner_product_space]{inner product space} over \( \BbbC \) and let \( e_1, \ldots, e_n \) be an orthonormal basis of \( V \). By \fullref{thm:basis_projection_orthonormal},
  \begin{equation*}
    \pi_{e_i}(e_j) = \begin{cases}
      1, &i = j \\
      0, &i \neq j
    \end{cases}.
  \end{equation*}

  By \fullref{thm:free_semimodule_universal_property}, there exist unique linear maps that satisfy the above conditions. Therefore,
  \begin{equation*}
    \pi_{e_i}(e_j) = \inprod {e_i} {e_j}
  \end{equation*}
  and hence the \hyperref[def:basis_decomposition]{basis decomposition} of \( x \) becomes
  \begin{equation}\label{eq:rem:inner_product_basis_decomposition/decomposition}
    x = \sum_{k=1}^n \inprod {x} {e_k} e_k.
  \end{equation}

  If \( V \) has no inner product defined and \( e_1, \ldots, e_n \) is an arbitrary basis, we can define the inner product
  \begin{equation}\label{eq:rem:inner_product_basis_decomposition/product}
    \inprod x y = \sum_{k=1}^n \pi_{e_k}(x) \cdot \overline{\pi_{e_k}(y)}.
  \end{equation}

  It is obviously bilinear, Hermitian and positive definite. Furthermore, \( e_1, \ldots, e_n \) is an \hyperref[def:orthogonality]{orthonormal} basis with respect to this inner product.

  This reduces in an obvious way to real vector spaces.
\end{remark}

\begin{theorem}[Pythagoras' theorem]\label{thm:pythagoras_theorem}
  In an \hyperref[def:inner_product_space]{inner product space}, if \( x \) and \( y \) are \hyperref[def:orthogonality]{orthogonal}, then
  \begin{equation*}
    \norm{x + y}^2 = \norm{x}^2 + \norm{y}^2.
  \end{equation*}
\end{theorem}
\begin{proof}
  Since \( \inprod x y = \inprod y x = 0 \), then
  \begin{equation*}
    \norm{x + y}^2
    =
    \inprod {x + y} {x + y}
    =
    \inprod x x + \inprod x y + \inprod y x + \inprod y y
    =
    \norm{x}^2 + \norm{y}^2.
  \end{equation*}
\end{proof}

\begin{proposition}\label{thm:orthogonal_implies_linearly_independent}
  Two nonzero orthogonal vectors are necessarily \hyperref[def:linear_dependence]{linearly independent}.
\end{proposition}
\begin{proof}
  We will prove the converse. Suppose that \( x = ty \) and that both vectors are nonzero. In particular, \( t \neq 0 \). Then
  \begin{equation*}
    \inprod x y = t \inprod y y.
  \end{equation*}

  Since inner products are positive definite, \( \inprod y y > 0 \) and hence \( \inprod x y > 0 \). Therefore, \( x \) and \( y \) are not orthogonal.
\end{proof}

\begin{definition}\label{def:orthogonal_complement}\mimprovised
  The \term{orthogonal complement} of the subspace \( U \) of \( V \) is the subspace
  \begin{equation*}
    U^\perp \coloneqq \set{ y \in U \given \qforall {x \in U} \inprod x y = 0 }
  \end{equation*}
  of \hyperref[def:orthogonality]{orthogonal} to all members of \( U \) subspaces.

  The (bi)linearity of the inner product ensures that \( U^\perp \) is indeed a subspace.

  This concept generalizes to \hyperref[def:orthogonal_complement]{annihilators}.
\end{definition}

\begin{theorem}[Gram–Schmidt orthogonalization]\label{thm:gramm_schmidt_orthogonalization}
  Let \( e_1, \ldots, e_n \) be a \hyperref[def:orthogonality]{basis} for the vector space \( V \). For \( i = 1, \ldots, n \) define
  \begin{equation*}
    f_k \coloneqq e_k - \sum_{i=1}^{k-1} \frac { \inprod {e_k} {f_i} } { \inprod {f_i} {f_i} } f_i.
  \end{equation*}

  Then \( f_1, \ldots, f_n \) is an \hyperref[def:orthogonality]{orthogonal} basis. We can then additionally rescale these vectors to obtain an \hyperref[def:orthogonality]{orthonormal} basis.
\end{theorem}
\begin{proof}
  Since for \( k \leq n \), the vector \( f_k \) is a linear combination of \( e_1, \ldots, e_k \), we have
  \begin{equation*}
    \linspan\set{ f_1, \ldots, f_n } = \linspan\set{ e_1, \ldots, e_n }.
  \end{equation*}

  We will use induction on \( n \) to show that \( f_1, \ldots, f_{n-1} \) are orthogonal (and, in particular, linearly independent). Since \( f_1 = e_1 \), the case \( n = 1 \) is trivial. Suppose that the first \( n - 1 \) vectors are orthogonal. Then, for \( k < n \),
  \small
  \begin{equation*}
    \inprod{f_n} {f_k}
    =
    \inprod*{ e_n - \sum_{i=1}^{n-1} \frac { \inprod {e_n} {f_i} } { \inprod {f_i} {f_i} } f_i} {f_k}
    =
    \inprod {e_n} {f_k} - \sum_{i=1}^{n-1} \frac { \inprod {e_n} {f_i} } { \inprod {f_i} {f_i} } \underbrace{ \inprod{f_i } {f_k} }_{ 0 \T*{if} i \neq j }
    =
    \inprod {e_n} {f_k} - \frac { \inprod {e_n} {f_k} } { \inprod {f_k} {f_k} } \inprod {f_k} {f_k}
    =
    0.
  \end{equation*}
  \normalsize

  Therefore, \( f_1, \ldots, f_n \) are pairwise orthogonal.
\end{proof}

\begin{corollary}\label{thm:finite_dimensional_orthonormal_basis_existence}
  Any finite-dimensional vector space has an orthonormal basis.
\end{corollary}
\begin{proof}
  We can simply apply \fullref{thm:gramm_schmidt_orthogonalization} to any basis.
\end{proof}

\begin{proposition}\label{thm:direct_sum_with_orthogonal_complement}
  Let \( U \) be a subspace of the finite-dimensional inner product space \( V \). Then
  \begin{equation*}
    V \cong U \oplus U^\perp,
  \end{equation*}
  where \( U^\perp \) is the \hyperref[def:orthogonal_complement]{orthogonal complement} of \( U \).
\end{proposition}
\begin{proof}
  \Fullref{thm:finite_dimensional_orthonormal_basis_existence} ensures that an orthonormal basis \( e_1, \ldots, e_n \) of \( U \) exists. \Fullref{thm:def:vector_space/expansion} allows us to expand this to a basis of \( V \) and \fullref{thm:gramm_schmidt_orthogonalization} allows us to orthogonalize the entire basis of \( V \). Let \( e_{n+1}, \ldots, e_m \) be the remaining vectors of the orthogonalized basis. Then
  \begin{equation*}
    U = \linspan{ e_1, \ldots, e_n }.
  \end{equation*}

  Furthermore, \( e_k \) is orthogonal to each of \( e_1, \ldots, e_n \) for \( k > n \), hence also to their linear combinations. Therefore,
  \begin{equation*}
    \linspan{ e_{n+1}, \ldots, e_m } \subsetneq U^\perp.
  \end{equation*}

  If \( e_{m+1} \) is an additional basis vector of \( U^\perp \), then it is a basis vector of \( V \). But that would mean that both \( e_1, \ldots, e_m \) and \( e_1, \ldots, e_{m+1} \) are bases of \( V \), which contradicts \fullref{thm:vector_space_dimension}.

  Hence,
  \begin{equation*}
    U^\perp = \linspan{ e_{n+1}, \ldots, e_m }.
  \end{equation*}

  Therefore,
  \begin{equation*}
    V \cong U \oplus U^\perp.
  \end{equation*}
\end{proof}

\begin{theorem}[Cauchy-Bunyakovsky-Schwarz inequality]\label{thm:cauchy_bunyakovsky_schwarz_inequality}
  In a real or complex \hyperref[def:inner_product_space]{inner product space}, the following \hyperref[def:inequality]{inequality} holds:
  \begin{equation}\label{thm:cauchy_bunyakovsky_schwarz_inequality/inequality}
    \abs{ \inprod x y} \leq \norm x \cdot \norm y.
  \end{equation}

  Furthermore, equality holds if and only if \( x \) and \( y \) are linearly dependent.
\end{theorem}
\begin{proof}
  \SubProof{Proof of inequality} Fix \( x, y \in V \) and \( t \in \BbbC \). If either \( x \) or \( y \) is the zero vector, the statement is trivially true. Suppose that both are nonzero.

  We have
  \begin{balign*}
    Q(x + y)
     & =
    \inprod {x + y} {x + y}
    =    \\ &=
    Q(x) + \overline t \inprod x y +  \inprod y x + \abs{t}^2 Q(y)
    =    \\ &=
    Q(x) + 2\real \parens*{ t \overline{ \inprod x y} } + \abs{t}^2 Q(y).
  \end{balign*}

  Take \( t \coloneqq - { \inprod x y} / {Q(y)} \), so that
  \begin{equation*}
    Q(x + y)
    =
    Q(x) - 2 \frac {\abs{ \inprod x y}^2} {Q(y)} + \frac {\abs{ \inprod x y}^2} {Q(y)}
    =
    Q(x) - \frac {\abs{ \inprod x y}^2} {Q(y)}.
  \end{equation*}

  Since \( Q(x + y) \geq 0 \), it follows that
  \begin{balign*}
    Q(x) - \frac {\abs{ \inprod x y}^2} {Q(y)} &\geq 0                  \\
    Q(x) Q(y)                                 &\geq \abs{ \inprod x y}^2.
  \end{balign*}

  \SubProof{Proof of equality} If \( x \) and \( y \) are linearly dependent, equality obviously holds. Conversely, suppose that equality holds. This implies that
  \begin{equation*}
    Q(x + y) = 0,
  \end{equation*}
  which by the positive definiteness of \( Q \) means that \( x = -ty \). Thus, \( x \) and \( y \) are linearly dependent.
\end{proof}

  \section{Diagonalization}\label{sec:diagonalization}

In this subsection, we restrict ourselves to fields rather than arbitrary rings.

\begin{definition}\label{def:eigenpair}\mcite[def. VI.6.13]{Aluffi2009Algebra}
  Let \( T: V \to V \) be a \hyperref[def:linear_function]{linear endomorphism} over the \hyperref[def:vector_space]{vector space} \( V \) over \( \BbbK \).

  An \term[ru=собственная пара (\cite[\S 29.2]{Тыртышников2007ЛинейнаяАлгебра}), en=(right) eigenpair (\cite[def. 1.1]{Stewart2001MatrixAlgorithmsVol2})]{eigenpair} \( (\lambda, e) \) of \( T \) consists of an \term[bg=собствена стойност (\cite[256]{ГеновМиховскиМоллов1991Алгебра}), ru=собственное значение (\cite[\S 29.2]{Тыртышников2007ЛинейнаяАлгебра})]{eigenvalue} \( \lambda \in \BbbK \) and a \hi{nonzero} \term[bg=собствен вектор (\cite[256]{ГеновМиховскиМоллов1991Алгебра}), ru=собственный вектор (\cite[\S 29.2]{Тыртышников2007ЛинейнаяАлгебра})]{eigenvector} \( e \in V \) such that
  \begin{equation}\label{eq:def:eigenpair}
    Te = \lambda e.
  \end{equation}

  We say that \( \lambda \) is an eigenvalue of \( T \) if it is part of at least one eigenpair; analogously, we say that \( e \) is an eigenvector if it is part of at least one eigenpair.

  In function spaces, the term \term[en=eigenfunction (\cite[169]{Evans2010PDE})]{eigenfunction} is sometimes used instead of \enquote{eigenvector}.
\end{definition}
\begin{comments}
  \item We only consider \enquote{right eigenvectors} satisfying \eqref{eq:def:eigenpair}, however it is also possible to consider \enquote{left eigenvectors} that satisfy \( x T = \lambda T \).

  \item If \( e \) is an eigenvector, so is \( te \) for any nonzero scalar \( t \). To ensure uniqueness, we often restrict ourselves to \hyperref[def:normed_vector]{normed} eigenvectors.
\end{comments}

\begin{example}\label{ex:def:eigenpair}
  We list some examples of \hyperref[def:eigenpair]{eigenpairs}:
  \begin{thmenum}
    \thmitem{ex:def:eigenpair/zero} The zero matrix over \( \BbbK \) of degree \( n \) has the entire vector space \( \BbbK^n \), without the zero vector, as its eigenvectors corresponding to the eigenvalue \( 0 \).

    \thmitem{ex:def:eigenpair/identity} The identity matrix \( I_n \) over \( \BbbK \) of degree \( n \) also has \( \BbbK^n \setminus \set{ \vect 0 } \) as its eigenvectors, however they correspond to the eigenvalue \( 1 \) rather than \( 0 \).

    \thmitem{ex:def:eigenpair/2112} Consider the matrix
    \begin{equation*}
      \begin{pmatrix}
        2 & 1 \\
        1 & 2
      \end{pmatrix}.
    \end{equation*}

    The following two are eigenpairs:
    \begin{align*}
      \begin{pmatrix}
        2 & 1 \\
        1 & 2
      \end{pmatrix}
      \begin{pmatrix}
        1 \\ 1
      \end{pmatrix}
      =
      3
      \begin{pmatrix}
        1 \\ 1
      \end{pmatrix}
      &&
      \begin{pmatrix}
        2 & 1 \\
        1 & 2
      \end{pmatrix}
      \begin{pmatrix}
        1 \\ -1
      \end{pmatrix}
      =
      1
      \begin{pmatrix}
        1 \\ -1
      \end{pmatrix}
    \end{align*}

    As we shall show in \fullref{ex:def:eigenspace/2112}, these are not the only eigenpairs for \( A \).

    \thmitem{ex:def:eigenpair/exponent} In suitable function spaces like \( C^\infty(\BbbR) \), the function \( e^{\lambda x} \) is en eigenvector of the \hyperref[def:differentiability]{differentiation} operator \( D_x \) corresponding to the eigenvalue \( \lambda \) because
    \begin{equation*}
      D_x e^{\lambda x} = \lambda e^{\lambda x}.
    \end{equation*}

    In particular, the case \( \lambda = 0 \) corresponds to the constant function \( e^0 = 1 \), hence
    \begin{equation*}
      D_x 1 = 0 \cdot 1.
    \end{equation*}
  \end{thmenum}
\end{example}

\begin{proposition}\label{thm:def:eigenpair}
  \hyperref[def:eigenpair]{Eigenpairs} have the following basic properties:
  \begin{thmenum}
    \thmitem{thm:def:eigenpair/independent} The eigenvectors corresponding to different eigenvalues are linearly independent.
    \thmitem{thm:def:eigenpair/power} If \( (\lambda, e) \) is an eigenpair for \( A \), then \( (\lambda^k, e) \) is an eigenpair for \( A^k \).
  \end{thmenum}
\end{proposition}
\begin{proof}
  \SubProofOf{thm:def:eigenpair/independent}\mcite{MathSE:eigenvectors_are_linearly_independent} Let \( e_1, \ldots, e_n \) be eigenvectors of \( T \) corresponding to \( \lambda_1, \ldots, \lambda_n \).

  We will use induction on \( n \) to show that \( e_1, \ldots, e_n \) are linearly independent.

  \Fullref{thm:def:linear_dependence/empty} implies that the empty family of vectors is linearly independent, which covers the base case \( n = 0 \).

  Suppose that the inductive hypothesis holds for families of \( n - 1 \) vectors. Suppose that there exist coefficients \( a_1, \ldots, a_n \) such that
  \begin{equation*}
    \sum_{k=1}^n a_k e_k = \vect 0.
  \end{equation*}

  Clearly we can multiply with \( \lambda_n \):
  \begin{equation*}
    \sum_{k=1}^n a_k \lambda_n e_k = \vect 0.
  \end{equation*}

  Furthermore, by linearity of \( T \), we have
  \begin{equation*}
    \vect 0
    =
    T(\vect 0)
    =
    T\parens[\Big]{ \sum_{k=1}^n a_k e_k }
    =
    \sum_{k=1}^n a_k T e_k
    =
    \sum_{k=1}^n a_k \lambda_k e_k.
  \end{equation*}

  Taking the difference of the above, we obtain
  \begin{equation*}
    \sum_{k=1}^{n-1} a_k (\lambda_n - \lambda_k) e_k = \vect 0.
  \end{equation*}

  This is a linear combination of \( n - 1 \) vectors that sums to zero. By the inductive hypothesis, the corresponding vectors are linearly independent, so the combination must be trivial. Thus, for every \( k < n \), since \( \lambda_k \neq \lambda_n \), it remains for \( a_k \) to be zero.

  Then \( a_1 = \ldots = a_{n-1} = 0 \) and
  \begin{equation*}
    \sum_{k=1}^n a_k e_k = \underbrace{\sum_{k=1}^{n-1} a_k e_k}_{\vect 0} + a_n e_n = \vect 0,
  \end{equation*}
  which implies that \( a_n = 0 \) also.

  Since the linear combination was arbitrary, we conclude that \( e_1, \ldots, e_n \) are linearly independent.

  \SubProofOf{thm:def:eigenpair/power} We can use induction to show that \( Ae = \lambda e \) implies \( A^k e = \lambda ^k e \).

  The base case \( k = 0 \) is vacuous, and the inductive case is simple:
  \begin{equation*}
    A^{k+1} e
    =
    A (A^k e)
    \reloset {\T{ind.}} =
    =
    A (\lambda^k e)
    =
    \lambda^k (Ae)
    =
    \lambda^{k+1} e.
  \end{equation*}
\end{proof}

\begin{remark}\label{rem:eigenpairs_via_invertibility}
  The \hyperref[def:eigenpair]{eigenpair equation}
  \begin{equation*}
    Tx = \lambda x
  \end{equation*}
  can be rewritten as
  \begin{equation*}
    (T - \lambda \cdot \id) x = \vect 0.
  \end{equation*}

  We can regard the above as a \hyperref[def:system_of_linear_equations]{system of equations}. By \fullref{thm:homogeneous_linear_equations_solutions}, there exists a nonzero solution, i.e. an eigenvector corresponding to \( \lambda \), if and only if the map
  \begin{equation*}
    T - \lambda \cdot \id
  \end{equation*}
  is an isomorphism.

  If \( V \) is finite-dimensional, by \fullref{thm:square_matrix_left_invertible_iff_right_invertible}, this map is injective if and only if it is surjective. Otherwise, it may fail to be either injective or surjective.
\end{remark}

\begin{definition}\label{def:eigenspace}\mcite[\S 30.2]{Тыртышников2007ЛинейнаяАлгебра}
  The \term[ru=собственное подпространство, en=eigenspace (\cite[265]{FriedbergInselSpence2018LinearAlgebra})]{eigenspace} of an \hyperref[def:eigenpair]{eigenvalue} is the set of all corresponding \hyperref[def:eigenpair]{eigenvectors}, along with the zero vector.
\end{definition}
\begin{comments}
  \item It is a vector (sub)space as a consequence of \fullref{thm:linear_system_solution_space}.

  \item In Russian, the phrase \enquote{собственное подпространство} (\enquote{proper space}) is used for eigenspaces, for example by \incite[\S 30.2]{Тыртышников2007ЛинейнаяАлгебра} and \incite[def. II.3]{Курош1968КурсВысшейАлгебры}. The phrase is however ambiguous because \enquote{собственный} (\enquote{proper}) is generally used for proper substructures the sense of \fullref{thm:substructures_form_complete_lattice/top}.

  In Bulgarian, \incite[101]{Обрешков1962ВисшаАлгебра} uses \enquote{собствено подпространство} (\enquote{proper subspace}) in the latter sense.
\end{comments}

\begin{example}\label{ex:def:eigenspace}
  We list some examples of \hyperref[def:eigenspace]{eigenspaces}:
  \begin{thmenum}
    \thmitem{ex:def:eigenspace/zero} We discussed in \fullref{ex:def:eigenpair/zero} how, with respect to the zero matrix, the entire space is the eigenspace of zero. Hence, the geometric multiplicity of zero is \( n \).

    \thmitem{ex:def:eigenspace/identity} Similarly, with respect to the identity matrix, the entire space is the eigenspace of \( 1 \).

    \thmitem{ex:def:eigenspace/2112} We discussed in \fullref{ex:def:eigenpair/2112} how the vector \( (1, 1) \) corresponds to the eigenvalue \( 3 \). The entire eigenspace is
    \begin{equation*}
      \set{ (t, t) \given t \in \BbbK }.
    \end{equation*}

    The eigenspace of \( 1 \) is
    \begin{equation*}
      \set{ (-t, t) \given t \in \BbbK }.
    \end{equation*}

    \thmitem{ex:def:eigenspace/1101} The eigenspace of \( 1 \) with respect to
    \begin{equation*}
      \begin{pmatrix}
        1 & 1 \\
        0 & 1
      \end{pmatrix}
    \end{equation*}
    is
    \begin{equation*}
      \set{ (t, 0) \given t \in \BbbK }.
    \end{equation*}

    We will show in \fullref{ex:def:linear_operator_characteristic_polynomial/1101} that this is the only eigenvalue of the matrix.

    \thmitem{ex:def:eigenspace/exponent} The eigenspace of \( \lambda \) with respect to differentiation is
    \begin{equation*}
      \set{ t e^\lambda \given t \in \BbbK }.
    \end{equation*}

    Hence, every scalar is an eigenvalue with geometric multiplicity \( 1 \).
  \end{thmenum}
\end{example}

\begin{definition}\label{def:linear_operator_characteristic_polynomial}\mcite[def. VI.6.8]{Aluffi2009Algebra}
  Let \( T: V \to V \) be a linear endomorphism over the \( n \)-dimensional vector space \( V \) over \( \BbbK \). Regard \( T \) as an endomorphism over the module \( \BbbK[\Lambda]^n \) over the \hyperref[def:polynomial_algebra]{polynomial ring} \( \BbbK[\Lambda] \). The \term[bg=характеристичен полином (\cite[77]{Обрешков1962ВисшаАлгебра}), ru=характеристический многочлен (\cite[def. 6.2.2]{Винберг2014КурсАлгебры})]{characteristic polynomial} of \( T \) is defined as
  \begin{equation*}
    \chi(\Lambda) \coloneqq \det(\Lambda \id - T).
  \end{equation*}
\end{definition}
\begin{comments}
  \item We could define the characteristic polynomial as \( \det(T - \Lambda \id) \) instead. See \fullref{rem:linear_operator_characteristic_polynomial_convensions} for a broader discussion.
\end{comments}

\begin{remark}\label{rem:linear_operator_characteristic_polynomial_convensions}
  There are two conventions for \hyperref[def:linear_operator_characteristic_polynomial]{characteristic polynomials}: \( \det(\Lambda \id - T) \) and \( \det(T - \Lambda \id) \). Since we are interested mostly in the roots of this polynomial, the choice is not significant; yet, we prefer the former because it makes the polynomial monic.

  \begin{itemize}
    \item \( \det(\Lambda \id - T) \) is used by
    \incite[74]{Knapp2016BasicAlgebra}
    \incite[def. VI.6.8]{Aluffi2009Algebra}
    \incite[196]{Jacobson1985BasicAlgebraI}
    \incite[561]{Lang2002Algebra}
    \incite[390]{Rotman2015AdvancedModernAlgebraPart1}
    \incite[499]{Knuth1997ArtVol1}
    \incite[def. 6.2.2]{Винберг2014КурсАлгебры}
    \incite[77]{Обрешков1962ВисшаАлгебра}

    \item \( \det(T - \Lambda \id) \) is used by
    \incite[101]{Halmos1974VectorSpaces}
    \incite[\S 1.2]{Treil2017LinearAlgebraDoneWrong}
    \incite[248]{FriedbergInselSpence2018LinearAlgebra}
    \incite[\S 29.4]{Тыртышников2007ЛинейнаяАлгебра}
    \incite[207]{Курош1968КурсВысшейАлгебры}
    \incite[78]{Кострикин2000АлгебраЧасть2}
    \incite[255]{ГеновМиховскиМоллов1991Алгебра}
  \end{itemize}
\end{remark}

\begin{definition}\label{def:eigenpair_multiplicity}\mcite[\S 29.5; \S 30.2]{Тыртышников2007ЛинейнаяАлгебра}
  We define the \term[ru=алгебраическая кратность, en=algebraic multiplicity (\cite[def. VI.6.17]{Aluffi2009Algebra})]{algebraic multiplicity} of an \hyperref[def:eigenpair]{eigenvalue} as its \hyperref[def:multiple_root]{root multiplicity} in the \hyperref[def:linear_operator_characteristic_polynomial]{characteristic polynomial}.

  We define the \term[ru=геометрическая кратность, en=geometric multiplicity (\cite[def. VI.6.20]{Aluffi2009Algebra})]{geometric multiplicity} of an eigenvalue as the dimension of its \hyperref[def:eigenspace]{eigenspace}.
\end{definition}

\begin{proposition}\label{thm:eigenvalues_and_characteristic_polynomials}
  The \hyperref[def:eigenpair]{eigenvalues} of a square matrix are precisely the \hyperref[def:multiple_root]{roots} of its \hyperref[def:linear_operator_characteristic_polynomial]{characteristic polynomial}.
\end{proposition}
\begin{proof}
  Follows from \fullref{rem:eigenpairs_via_invertibility}.
\end{proof}

\begin{example}\label{ex:def:linear_operator_characteristic_polynomial}
  We list some examples of \hyperref[def:linear_operator_characteristic_polynomial]{characteristic polynomials}:
  \begin{thmenum}
    \thmitem{ex:def:linear_operator_characteristic_polynomial/zero} The characteristic polynomial of the \( n \times n \) zero matrix is \( \Lambda^n \). Its only root is \( 0 \). Both the geometric and the algebraic multiplicities of \( 0 \) are \( n \).

    \thmitem{ex:def:linear_operator_characteristic_polynomial/identity} Similarly, the characteristic polynomial of the identity matrix \( I_n \) is \( (\Lambda - 1)^n \). Hence, the only eigenvalue of \( I_n \) is \( 1 \) and both of its multiplicities are \( n \).

    \thmitem{ex:def:linear_operator_characteristic_polynomial/2112} We continue \fullref{ex:def:eigenspace/2112}. The matrix
    \begin{equation*}
      \begin{pmatrix}
        2 & 1 \\
        1 & 2
      \end{pmatrix}
    \end{equation*}
    has characteristic polynomial
    \begin{equation*}
      (\Lambda - 2)^2 - 1 = \Lambda^2 - 4\Lambda + 3.
    \end{equation*}

    Its roots are
    \begin{equation*}
      \frac {4 \pm \sqrt{16 - 12}} 2 = 2 \pm 1.
    \end{equation*}

    Both roots have geometric and algebraic multiplicities \( 1 \).

    \thmitem{ex:def:linear_operator_characteristic_polynomial/1101} The matrix
    \begin{equation*}
      \begin{pmatrix}
        1 & 1 \\
        0 & 1
      \end{pmatrix}
    \end{equation*}
    has characteristic polynomial
    \begin{equation*}
      (\Lambda - 1)^2.
    \end{equation*}

    The only eigenvalue is hence \( 1 \), and it has algebraic multiplicity \( 2 \). We discussed in \fullref{ex:def:eigenspace/1101} that the corresponding geometric multiplicity is \( 1 \).
  \end{thmenum}
\end{example}

\begin{proposition}\label{thm:geometric_vs_algebraic_multiplicity}\mcite{MathSE:geometric_multiplicity_is_bounded_by_algebraic}
  The \hyperref[def:eigenpair_multiplicity]{geometric multiplicity} of an eigenvalue does not exceed its \hyperref[def:eigenpair_multiplicity]{algebraic multiplicity}.
\end{proposition}
\begin{proof}
  Suppose that the geometric multiplicity of \( \lambda \) with respect to \( T \) is \( m \). Then there exist linearly independent eigenvalues \( x_1, \ldots, x_m \) such that, for \( k = 1, \ldots, m \),
  \begin{equation*}
    T x_k = \lambda x_k.
  \end{equation*}

  \Fullref{thm:def:vector_space/expansion} allows us to expand this to a basis of \( \BbbK^n \) via some vectors \( x_{m+1}, \ldots, x_n \). With respect to this basis, the operator \( T \) has the matrix
  \begin{equation*}
    A = \begin{pmatrix}
      \lambda I_k & B \\
      0           & C
    \end{pmatrix},
  \end{equation*}
  for suitable matrices \( B \) and \( C \).

  The characteristic polynomial of this matrix is
  \begin{equation*}
    \chi_A(\Lambda) = (\Lambda - \lambda)^k \chi_C(\Lambda).
  \end{equation*}

  Therefore, the algebraic multiplicity of \( A \) is at least \( k \).
\end{proof}

\begin{definition}\label{def:point_spectrum}\mcite[def. 10.32]{Rudin1991FunctionalAnalysis}
  The set of all \hyperref[def:eigenpair]{eigenvalues} of a linear endomorphism is called its \term{point spectrum}.
\end{definition}

\begin{lemma}\label{thm:eigenvectors_are_linearly_independent}
  Eigenvectors corresponding to different eigenvalues are linearly independent.
\end{lemma}
\begin{proof}
  Suppose that for some endomorphism \( T: V \to V \) we have distinct eigenvalues \( \lambda \) and \( \mu \), let \( e \) be an eigenvector corresponding to \( \lambda \) and \( f \) --- to \( \mu \).

  Let \( t e + r f = 0_V \) for some scalars \( t \) and \( r \). Then
  \begin{equation*}
    0_V = A(t e + r f) = t A e + r A f = t \lambda e + r \mu f.
  \end{equation*}

  Also, we can multiply \( t e + r f = 0_V \) by \( \lambda \) to obtain
  \begin{equation*}
    0_V = t \lambda e + r \lambda f.
  \end{equation*}

  We can now subtract the two to obtain
  \begin{equation*}
    0_V = r (\mu - \lambda) f.
  \end{equation*}

  Since \( \mu \neq \lambda \) and since \( f \) is by definition nonzero, we have \( r = 0 \). We similarly conclude that \( t = 0 \).

  Therefore, \( e \) and \( f \) are linearly independent.
\end{proof}

\begin{proposition}\label{thm:operator_diagonalizability}
  Let \( T: V \to V \) be a linear endomorphism over the vector space \( V \) over \( \BbbK \) of dimension \( n \). The following are equivalent:
  \begin{thmenum}
    \thmitem{thm:operator_diagonalizability/basis} The space \( V \) has a basis \( e_1, \ldots, e_n \) of eigenvectors of \( T \).

    \thmitem{thm:operator_diagonalizability/diagonal} There exists a basis \( e_1, \ldots, e_n \) of \( V \) such that \( T \) has a diagonal matrix with respect to this basis.

    The \( k \)-th diagonal entry of this matrix is then an eigenvalue of \( T \) for which \( e_k \) is an eigenvector.

    \thmitem{thm:operator_diagonalizability/multiplicities} There exist eigenvalues \( \lambda_1, \ldots, \lambda_m \) whose algebraic and geometric multiplicities coincide and sum to \( n \).

    \thmitem{thm:operator_diagonalizability/direct_sum} There exist eigenvalues \( \lambda_1, \ldots, \lambda_m \) such that
    \begin{equation*}
      V \cong \ker(\lambda_1 \id - T) \oplus \cdots \oplus \ker(\lambda_m \id - T).
    \end{equation*}
  \end{thmenum}
\end{proposition}
\begin{proof}
  \ImplicationSubProof{thm:operator_diagonalizability/basis}{thm:operator_diagonalizability/diagonal} Suppose that there exists a basis \( e_1, \ldots, e_n \) of \( V \) such that \( T e_k = \lambda_k e_k \) for some eigenvalues \( \lambda_1, \ldots, \lambda_n \). Let
  \begin{equation*}
    A \coloneqq \op{diag}(\lambda_1, \ldots, \lambda_n).
  \end{equation*}

  Given an arbitrary vector \( x \) from \( V \), we have
  \begin{equation*}
    x = \sum_{k=1}^n \pi_{e_k}(x) \cdot e_k.
  \end{equation*}

  Define
  \begin{equation*}
    \begin{aligned}
      &\Phi: V \to \BbbK^n \\
      &\Phi(x) \coloneqq \begin{pmatrix}
        \pi_{e_1}(x) \\
        \vdots \\
        \pi_{e_n}(x) \\
      \end{pmatrix}.
    \end{aligned}
  \end{equation*}

  Then
  \begin{multline*}
    \Phi(T x)
    =
    \Phi\parens*{ T \parens*{ \sum_{k=1}^n \pi_{e_k}(x) \cdot e_k } }
    =
    \Phi\parens*{ \sum_{k=1}^n \pi_{e_k}(x) \cdot T e_k }
    = \\ =
    \Phi\parens*{ \sum_{k=1}^n \lambda_k \cdot \pi_{e_k}(x) \cdot e_k }
    =
    \begin{pmatrix}
      \lambda_1 \cdot \pi_{e_1}(x) \\
      \vdots \\
      \lambda_n \cdot \pi_{e_n}(x)
    \end{pmatrix}
    =
    A \cdot \Phi(x)
  \end{multline*}

  \ImplicationSubProof{thm:operator_diagonalizability/diagonal}{thm:operator_diagonalizability/basis} Conversely, suppose that
  \begin{equation*}
    A \coloneqq \op{diag}(\lambda_1, \ldots, \lambda_n)
  \end{equation*}
  is the matrix of \( T \) with respect to the basis \( e_1, \ldots, e_n \), i.e.
  \begin{equation*}
    \Phi(T x) = A \cdot \Phi(x).
  \end{equation*}

  Then
  \begin{equation*}
    A \cdot \Phi(e_k) = a_{k,\anon*} = \begin{pmatrix} 0 & \lambda_k & \end{pmatrix} = \lambda_k \Phi(e_k)
  \end{equation*}
  and
  \begin{equation*}
    T e_k = \Phi^{-1}(A \cdot \Phi(e_k)) = \lambda_k e_k.
  \end{equation*}

  Therefore, \( e_k \) is an eigenvector of \( T \) corresponding to \( \lambda_k \).

  \ImplicationSubProof{thm:operator_diagonalizability/basis}{thm:operator_diagonalizability/multiplicities} Suppose that there exists a basis \( e_1, \ldots, e_n \) of \( V \) such that \( Te_k = \lambda_k e_k \) for some eigenvalues \( \lambda_1, \ldots, \lambda_n \).

  If \( \lambda_{i_1} = \cdots = \lambda_{i_m} \) and all other eigenvalues are distinct from \( \lambda_{i_1} \), then
  \begin{equation*}
    \ker(T - \lambda_{i_1} \id) = \linspan\set{ e_{i_1}, \cdots, e_{i_m} }.
  \end{equation*}

  Thus, the algebraic and geometric multiplicities coincide. The sum of multiplicities over all distinct eigenvalues is obviously \( n \).

  \ImplicationSubProof{thm:operator_diagonalizability/multiplicities}{thm:operator_diagonalizability/direct_sum} Suppose that \( \lambda_1, \ldots, \lambda_m \) are the distinct eigenvalues of \( T \) and that their algebraic and geometric multiplicities coincide and sum to \( n \).

  For \( i \neq j \), the eigenvectors corresponding to \( \lambda_i \) are linearly independent from those corresponding to \( \lambda_j \) by \fullref{thm:eigenvectors_are_linearly_independent}. Thus, the union of some bases of \( \ker(\lambda_k \id - T) \) for all \( k = 1, \ldots, m \) is a basis of \( V \).

  By \fullref{thm:basis_of_direct_sum}, the union is also a basis of the direct sum
  \begin{equation*}
    \ker(\lambda_1 \id - T) \oplus \cdots \oplus \ker(\lambda_m \id - T).
  \end{equation*}

  Therefore, this is an \hyperref[def:semimodule_direct_sum]{internal direct sum} giving \( V \).

  \ImplicationSubProof{thm:operator_diagonalizability/direct_sum}{thm:operator_diagonalizability/basis} Suppose that
  \begin{equation*}
    V \cong \ker(\lambda_1 \id - T) \oplus \cdots \oplus \ker(\lambda_m \id - T).
  \end{equation*}

  Construct a basis of \( V \) by taking bases of the eigenspaces \( \ker(\lambda_k \id - T) \) together. This will be a basis of eigenvectors by \fullref{thm:basis_of_direct_sum}.
\end{proof}

\begin{definition}\label{def:diagonalizable_matrix}\mimprovised
  A square matrix is called \term{diagonalizable} if, as a linear endomorphism, it satisfies any of the conditions from \fullref{thm:operator_diagonalizability}.

  The condition \fullref{thm:operator_diagonalizability/diagonal} for the \( n \times n \) matrix \( A \) is then equivalent to the existence of a \hyperref[con:change_of_basis]{change of basis} matrix \( P \) and a \hyperref[def:matrix_diagonal]{diagonal matrix} \( D \) such that
  \begin{equation*}
    A = P^{-1} D P.
  \end{equation*}

  It is thus necessary and sufficient for \( A \) to be \hyperref[def:similar_matrices]{similar} to a diagonal matrix. This is often taken as \enquote{the} definition of diagonalizability, however it hides the relationship with the theory of eigenvalues and eigenspaces presented in \fullref{thm:operator_diagonalizability}. That is, the \( k \)-th diagonal entry of \( D \) is an eigenvalue of \( A \) for which the \( k \)-th column of \( P \) is an eigenvector.
\end{definition}

\begin{example}\label{ex:def:diagonalizable_matrix}
  We list examples of \hyperref[def:diagonalizable_matrix]{matrix diagonalization}:
  \begin{thmenum}
    \thmitem{ex:def:diagonalizable_matrix/01-12} The matrix
    \begin{equation*}
      A = \begin{pmatrix}
        0  & 1 \\
        -1 & 2.
      \end{pmatrix}
    \end{equation*}
    is not diagonalizable.

    \todo{Prove}.
  \end{thmenum}
\end{example}

\begin{definition}\label{def:adjoint_operator}\mcite[sec. 38.1]{Тыртышников2007ЛинейнаяАлгебра}
  We say that the linear operator \( T^*: V \to U \) between \hyperref[def:inner_product_space]{inner product spaces} is \term{adjoint} to \( T: U \to V \) if for every \( x \in U \) and \( y \in V \) we have
  \begin{equation*}
    \inprod {Tx} y = \inprod x {T^* y}.
  \end{equation*}

  If an adjoint exists, it is unique, and in particular \( (T^*)^* = T \). But an adjoint may not exist.

  If \( U = V \) and \( T = T^* \), we say that the operator is \term{self-adjoint} or \term{symmetric}. In the case of a complex inner product space, we say that \( T \) is \enquote{\term{Hermitian}} instead of \enquote{symmetric}.

  Note that this definition holds for linear operators and the concept differs from the similar concepts for bilinear forms. The inner product itself is symmetric or Hermitian by definition.
\end{definition}
\begin{defproof}
  We will show uniqueness. Suppose that, for some linear maps \( R \) and \( S \), for any \( x \in U \) and \( y \in V \) we have
  \begin{equation*}
    \inprod {T x} y = \inprod x {R y} = \inprod x {S y}.
  \end{equation*}

  Then, for a fixed nonzero \( x_0 \) and for any \( y \), we have
  \begin{equation*}
    \inprod {x_0} {(R - S) y} = 0.
  \end{equation*}

  Since the inner product is nondegenerate, this is possible if \( R - S \) is the zero operator. Hence, \( R = S \).
\end{defproof}

\begin{proposition}\label{thm:image_of_adjoint}
  A linear operator \( T: U \to V \) and its \hyperref[def:adjoint_operator]{adjoint} \( T^*: V \to U \) have isomorphic images.
\end{proposition}
\begin{proof}
  By \fullref{thm:quotient_structure_universal_property}, \( T \) \hyperref[def:factors_through]{factors through}
  \begin{equation*}
    U / \ker T \cong \img T.
  \end{equation*}

  More precisely, there exists an operator \( R: \img T \to V \) such that \( T = R \bincirc \pi \), where \( \pi \) is the projection into \( \img T \).

  We have \( T^* = \pi^* \bincirc R^* \). Hence, \( T^* \) also factors through \( \img T \), and thus
  \begin{equation*}
    \dim \img T^* \leq \dim \img T.
  \end{equation*}

  Conversely,
  \begin{equation*}
    \dim \img T = \dim \img (T^*)^* \leq \dim \img T^*.
  \end{equation*}

  Therefore, the dimensions of the images of \( T \) and \( T^* \) coincide, and hence the images are isomorphic.
\end{proof}

\begin{proposition}\label{thm:conjugate_transpose}
  The \hyperref[def:conjugate_transpose]{conjugate transpose} \( A^* \) of the matrix \( A \) corresponds to the \hyperref[def:adjoint_operator]{adjoint operator} of \( A \) (with respect to the \hyperref[def:inner_product_space]{dot product}).
\end{proposition}
\begin{proof}
  \begin{equation*}
    \inprod {Ax} {y}
    =
    (Ax)^* y
    =
    (x^* A^*) y
    =
    x^* (A^* y)
    =
    \inprod {x} {A^* y}.
  \end{equation*}
\end{proof}

\begin{definition}\label{def:invariant_subset}\mcite[23]{Зорич2019АнализЧасть1}
  We say that the subset \( B \) of any \hyperref[def:set]{plain set} \( A \) is \term[bg=инвариантно (подпространство) (\cite[def. X.1]{ГеновМиховскиМоллов1991Алгебра}), ru=инвариантное (множество), en=invariant (\cite[108]{DaveyPriestley2002LatticeTheory})]{invariant} under the endofunction \( f: A \to A \) if \( f[B] \subseteq B \).
\end{definition}
\begin{comments}
  \item Invariant subsets allow us to restrict \( f \) to an endofunction on \( B \).
  \item \enquote{Invariant} in this sense often refers to linear subspaces invariant under a linear map. Examples of such usage include \incite[218]{Knapp2016BasicAlgebra} and \incite[sec. 28.5]{Тыртышников2007ЛинейнаяАлгебра}.
\end{comments}

\begin{proposition}\label{thm:invariant_subspace}
  Let \( T: V \to V \) be a linear operator. \hyperref[def:invariant_subset]{Invariant} under \( T \) subspaces of \( V \) have the following basic properties:
  \begin{thmenum}
    \thmitem{thm:invariant_subspace/kernel} The kernel of \( T \) is invariant.
    \thmitem{thm:invariant_subspace/image} The image of \( T \) is invariant.
    \thmitem{thm:invariant_subspace/complement} If \( T \) is \hyperref[def:adjoint_operator]{self-adjoint} and \( U \) is invariant under \( T \), so is its \hyperref[def:orthogonal_complement]{orthogonal complement} \( U^\perp \).
  \end{thmenum}
\end{proposition}
\begin{proof}
  \SubProofOf{thm:invariant_subspace/kernel} If \( x \in \ker T \), then \( Tx = 0 \in \ker T \).

  \SubProofOf{thm:invariant_subspace/image} If \( y \in \img T \), then \( Ty \in \img T \) because \( \img T \) is a subspace of \( V \).

  \SubProofOf{thm:invariant_subspace/complement} Let \( x \in U \) and \( y \in U^\perp \). We know that \( Tx \in U \) and thus \( \inprod {Tx} y = 0 \). If \( T \) is self-adjoint, then
  \begin{equation*}
    0 = \inprod {Tx} y = \inprod x {Ty},
  \end{equation*}
  and thus \( Ty \in U^\perp \).
\end{proof}

\begin{definition}\label{def:unitary_matrix}
  We say that a square real (resp. complex) matrix is \term{orthogonal} (resp. \term{unitary}) if any of the following conditions hold:
  \begin{thmenum}
    \thmitem{def:unitary_matrix/transpose} The \hyperref[def:transpose_matrix]{transpose matrix} (resp. \hyperref[def:conjugate_transpose]{conjugate transpose}) is also the \hyperref[def:inverse_matrix]{inverse matrix}.
    \thmitem{def:unitary_matrix/columns} The columns of the matrix are pairwise \hyperref[def:orthogonality]{orthonormal} with respect to the \hyperref[def:inner_product_space]{dot product}.
  \end{thmenum}
\end{definition}
\begin{defproof}
  \EquivalenceSubProof{def:unitary_matrix/transpose}{def:unitary_matrix/columns} Let
  \begin{equation*}
    A = \begin{pmatrix} a_1 \\ \vdots \\ a_n \end{pmatrix}.
  \end{equation*}

  Then
  \begin{equation*}
    A A^*
    =
    \begin{pmatrix} a_1 \\ \vdots \\ a_n \end{pmatrix}
    \begin{pmatrix} a_1^* & \cdots & a_n^* \end{pmatrix}
    =
    \begin{pmatrix}
      a_1 a_1^*  & \cdots & a_1 a_n^* \\
      \vdots     & \ddots & \vdots \\
      a_n a_1^*  & \cdots & a_n a_n^*
    \end{pmatrix}.
  \end{equation*}

  Hence, \( A A^* = I_n \) if and only if
  \begin{equation*}
    a_i a_j^* = \begin{cases}
      1, &i = j \\
      0, &i \neq j
    \end{cases}.
  \end{equation*}
\end{defproof}

\begin{definition}\label{def:unitary_groups}\mimprovised
  The \hyperref[def:unitary_matrix]{orthogonal} (resp. unitary) matrices form a subgroup of the \hyperref[def:linear_groups]{general linear group} \( \grp{GL}(n) \), which we denote by \( \grp{O}(n) \) (resp \( \op{U}(n) \)).

  The subgroup of orthogonal (resp. unitary) matrices with determinant \( 1 \) is called the \term{special orthogonal (resp. unitary) group} \( \grp{SO}(n) \) (resp. \( \grp{SU}(n) \)).
\end{definition}

\begin{definition}\label{def:unitary_operator}\mimprovised
  We say that a linear automorphism \( T: V \to V \) over the real (resp. complex) inner product space \( V \) is \term{orthogonal} (resp. \term{unitary}) if any of the following conditions hold:
  \begin{thmenum}
    \thmitem{def:unitary_operator/inverse} The \hyperref[def:adjoint_operator]{adjoint operator} \( T^*: V \to V \) exists and is the inverse of \( T \).
    \thmitem{def:unitary_operator/inner_product} We have
    \begin{equation*}
      \inprod {Tx} {Ty} = \inprod x y.
    \end{equation*}
  \end{thmenum}
\end{definition}
\begin{defproof}
  \ImplicationSubProof{def:unitary_operator/inverse}{def:unitary_operator/inner_product} If \( T^* = T^{-1} \), then
  \begin{equation*}
    \inprod {Tx} {Ty} = \inprod x {T^* T y} = \inprod x y.
  \end{equation*}

  \ImplicationSubProof{def:unitary_operator/inner_product}{def:unitary_operator/inverse} Conversely, suppose that
  \begin{equation*}
    \inprod {Tx} {Ty} = \inprod x y.
  \end{equation*}

  Then \( T^{-1} \) is the unique adjoint of \( T \) because
  \begin{equation*}
    \inprod x {T^{-1}y}
    =
    \inprod {Tx} {TT^{-1}y}
    =
    \inprod {Tx} y
    =
    \inprod x {T^* y}
    =
    \inprod x {T^* y}.
  \end{equation*}
\end{defproof}

\begin{definition}\label{def:circle_group}\mimprovised
  The \term{circle group} \( \BbbT \) is any of the following isomorphic groups:
  \begin{thmenum}
    \thmitem{def:circle_group/complex} The set of all complex numbers with unit norm under multiplication.
    \thmitem{def:circle_group/real} The \hyperref[def:group/quotient]{quotient} \( \BbbR / \BbbZ \) under addition.
    \thmitem{def:circle_group/rotations} The set of all \hyperref[def:rigid_motion/rotation]{rotation} in the \hyperref[def:euclidean_plane]{Euclidean plane} under composition.
    \thmitem{def:circle_group/so2} The \hyperref[def:unitary_groups]{special orthogonal group} \( \grp{SO}(2) \).
    \thmitem{def:circle_group/su1} The \hyperref[def:unitary_groups]{special unitary group} \( \grp{SU}(1) \).
  \end{thmenum}
\end{definition}
\begin{defproof}
  \EquivalenceSubProof{def:circle_group/complex}{def:circle_group/real} Define the map
  \begin{equation*}
    \begin{aligned}
      &\varphi: \BbbR / \BbbZ \to \BbbC \\
      &\varphi(t) \coloneqq \frac {e^{it}} {2\pi}.
    \end{aligned}
  \end{equation*}

  It follows from \fullref{thm:def:exponential_function/homomorphism} that it is a homomorphism and from \fullref{thm:def:exponential_function/unit_circle} that it is injective and the values of \( \varphi \) are precisely the complex numbers with norm \( 1 \). Therefore, it is an isomorphism.

  \EquivalenceSubProof{def:circle_group/real}{def:circle_group/so2} Follows from \fullref{thm:plane_rotation_matrix}.

  \EquivalenceSubProof{def:circle_group/rotations}{def:circle_group/so2} This is the definition of rotation in \fullref{def:rigid_motion/rotation}.

  \EquivalenceSubProof{def:circle_group/so2}{def:circle_group/su1} Follows from \fullref{thm:complex_numbers_as_matrices} by noting that the norm of a complex number in matrix form is the determinant of the corresponding matrix.
\end{defproof}

\begin{theorem}[Finite-dimensional spectral theorem]\label{thm:finite_dimensional_spectral_theorem}
  Let \( V \) be a real (resp. complex) vector space of dimension \( n \).

  \begin{thmenum}
    \thmitem{thm:finite_dimensional_spectral_theorem/basis} Every \hyperref[def:transpose_matrix]{symmetric} (resp. \hyperref[def:conjugate_transpose]{Hermitian}) endomorphism on \( V \) induces an \hyperref[def:orthogonality]{orthonormal} basis of eigenvectors.

    \thmitem{thm:finite_dimensional_spectral_theorem/eigenvalues} Every symmetric (resp. Hermitian) endomorphism on \( V \) has \( n \) real eigenvalues (counting multiplicity).

    \thmitem{thm:finite_dimensional_spectral_theorem/matrix} Every symmetric (resp. Hermitian) \( n \times n \) matrix \( A \) is \hyperref[def:diagonalizable_matrix]{diagonalizable} via an \hyperref[def:unitary_matrix]{orthogonal} (resp. \hyperref[def:unitary_matrix]{unitary}) matrix. More precisely, we can decompose \( A \) as
    \begin{equation*}
      A = P^{-1} D P,
    \end{equation*}
    where \( D \) is a real diagonal matrix of eigenvalues and \( P \) is a unitary matrix whose columns are eigenvalues of \( A \).

    Furthermore, the \( k \)-th diagonal entry of \( D \) is an eigenvalue of \( A \) for which the \( k \)-th column of \( P \) is an eigenvector.
  \end{thmenum}
\end{theorem}
\begin{proof}
  We will only consider Hermitian matrices since the proof for symmetric matrices is identical.

  \SubProofOf{thm:finite_dimensional_spectral_theorem/basis} We will use induction on \( n \) to show that every Hermitian endomorphism on an \( n \)-dimensional space induces an orthonormal basis of eigenvectors. The case \( n = 1 \) is obvious.

  Suppose that the statement holds for \( n - 1 \), let \( V \) be an \( n \)-dimensional space and let \( T \) be an endomorphism on \( V \). By \fullref{thm:fundamental_theorem_of_algebra}, the characteristic polynomial has at least one eigenvalue. Let \( (\lambda, x) \) be an eigenpair, let \( e_1 \coloneqq x / \norm x \) and let
  \begin{equation*}
    U \coloneqq \linspan\set{ e_1 }.
  \end{equation*}

  The space \( U \) is invariant under \( T \) because
  \begin{equation*}
    T(t e_1) = t(T e_1) = (t\lambda) e_1.
  \end{equation*}

  By \fullref{thm:invariant_subspace/complement}, its \hyperref[def:orthogonality]{orthogonal complement} \( U^\perp \) is also invariant under \( T \). We can thus restrict \( T \) to \( U^\perp \). Furthermore, \( \dim U^\perp = n - 1 \) by \fullref{thm:direct_sum_with_orthogonal_complement} and \fullref{thm:rank_of_direct_sum}. Hence, by the inductive hypothesis, there exists an orthonormal basis \( e_2, \ldots, e_n \) of \( U^\perp \) of eigenvectors of \( T\restr_{U^\perp} \).

  Therefore, \( e_1, \ldots, e_n \) is an orthonormal basis of \( \BbbC^n = U \oplus U^\perp \) consisting of eigenvectors of \( T \).

  \SubProofOf{thm:finite_dimensional_spectral_theorem/eigenvalues}
  \SubProof*{Proof for complex numbers} Every characteristic polynomial of degree \( n \) over \( \BbbC \) has \( n \) roots, counting multiplicity. For every eigenpair \( (\lambda, x) \) of \( T: V \to V \) we have
  \begin{equation*}
    \lambda \inprod x x
    =
    \inprod {\lambda x} x
    =
    \inprod {T x} x
    =
    \inprod x {T x}
    =
    \inprod x {\lambda x}
    =
    \oline \lambda \inprod x x.
  \end{equation*}

  Since inner products are positive definite, \( \inprod x x \) is a positive real number and hence we can cancel it to obtain the equality \( \lambda = \oline \lambda \). Hence, if \( \lambda \) is an eigenvalue, it is a real number.

  \SubProof*{Proof for real numbers} If \( V \) is instead a real vector space, the characteristic polynomial \( \chi(\Lambda) \) of \( T \) belongs to \( \BbbR[\Lambda] \). It can be \hyperref[thm:polynomial_algebra_universal_property]{evaluated} over \( \BbbC \), which will give us \( n \) roots over \( \BbbC \). But these roots are real, hence \( \chi(\Lambda) \) has \( n \) roots over \( \BbbR \). These are the eigenvalues of \( T \).

  \SubProofOf{thm:finite_dimensional_spectral_theorem/matrix} Follows from \fullref{thm:finite_dimensional_spectral_theorem/basis}, \fullref{thm:finite_dimensional_spectral_theorem/eigenvalues} and \fullref{thm:operator_diagonalizability}.
\end{proof}

\begin{definition}\label{def:singular_value}
  We say that \( \sigma \) is a \term{singular value} of the operator \( T: U \to V \) if \( \sigma^2 \) is an eigenvalue of \( T^* \bincirc T \). \Fullref{thm:finite_dimensional_spectral_theorem/eigenvalues} implies that singular values are always real.

  Note that this definition implicitly assumes the existence of an adjoint operator.
\end{definition}

\begin{theorem}[Singular value decomposition]\label{thm:singular_value_decomposition}
  \hfill
  \begin{thmenum}
    \thmitem{thm:singular_value_decomposition/basis} For every linear operator \( T: U \to V \) between finite-dimensional real or complex vector spaces, there exist orthonormal bases of \( U \) and \( V \) such that the matrix of \( T \) with respect to them is diagonal with the \hyperref[def:singular_value]{singular values} of \( T \) on its diagonal.

    If \( m \leq n \), this matrix has the form
    \begin{equation}\label{eq:thm:singular_value_decomposition/matrix}
      \Sigma
      =
      \begin{pmatrix}
        \sigma_1 & 0        & \cdots & 0        & 0 & \cdots & 0 \\
        0        & \sigma_2 & \cdots & 0        & 0 & \cdots & 0 \\
        \vdots   & \vdots   & \ddots & 0        & 0 & \cdots & 0 \\
        0        & 0        & 0      & \sigma_m & 0 & \cdots & 0
      \end{pmatrix}.
    \end{equation}

    \thmitem{thm:singular_value_decomposition/matrix} For every \( m \times n \) real or complex matrix \( A \), there exists an \( m \times m \) \hyperref[def:unitary_matrix]{unitary matrix} \( U \) and an \( n \times n \) unitary matrix \( V \) such that
    \begin{equation*}
      A = U \Sigma V^*,
    \end{equation*}
    where \( \Sigma \) is a diagonal matrix with the singular values of \( A \) on its diagonal.
  \end{thmenum}
\end{theorem}
\begin{proof}
  \SubProofOf{thm:singular_value_decomposition/basis} \Fullref{thm:finite_dimensional_spectral_theorem} gives us a basis \( e_1, \ldots, e_n \) of \( U \) of eigenvectors of \( T^* \bincirc T \). Suppose that only \( e_1, \ldots, e_r \) have nonzero eigenvalues and let \( \sigma_1, \ldots, \sigma_r \) be the corresponding singular values, i.e.
  \begin{equation*}
    [T^* \bincirc T] e_k = \sigma_k^2 e_k.
  \end{equation*}

  We are interested in the kernel of \( T \). Suppose that \( Tx = 0_V \). Then
  \begin{equation*}
    0_U
    =
    [T^* \bincirc T] x
    =
    \sum_{k=1}^n \inprod { [T^* \bincirc T]x } { e_k }_U e_k
    =
    \sum_{k=1}^n \sigma_k^2 \inprod { x } { e_k }_U e_k
  \end{equation*}

  We have \( \sigma_k^2 \inprod { x } { e_k }_U = 0 \), which implies \( \inprod x { e_k }_U = 0 \) for \( k \leq r \). Thus, \( x \) is a linear combination of \( e_{r+1}, \ldots, e_n \), which implies that \( e_{k+1}, \ldots, e_n \) is a basis of \( \ker T \).

  We will now construct a basis. For \( k = 1, \ldots, r \) define
  \begin{equation*}
    f_k \coloneqq \frac 1 {\sigma_k} T e_k.
  \end{equation*}

  Since
  \begin{equation*}
    T x = \sum_{k=1}^n \inprod x { e_k } T e_k
  \end{equation*}
  and since \( T e_{r+1} = \cdots = T e_n = 0_V \), we conclude that the vectors \( f_1, \ldots, f_r \) span \( \img T \). They are also pairwise orthogonal:
  \begin{equation*}
    \inprod { f_i } { f_j }_V
    =
    \frac 1 {\sigma_j^2} \inprod { T e_i } { T e_j }_V
    =
    \frac 1 {\sigma_j^2} \inprod { e_i } { [T^* \bincirc T] e_j }_U
    =
    \frac 1 {\sigma_j^2} {\sigma_j^2} \inprod { e_i } { e_j }_U
    =
    \begin{cases}
      1, &i = j \\
      0, &\T{otherwise.}
    \end{cases}
  \end{equation*}

  Therefore, \( f_1, \ldots, f_r \) is a basis of \( \img T \). \Fullref{thm:image_of_adjoint} implies that \( \img T \cong \img T^* \) and \fullref{thm:rank_nullity_theorem} and \fullref{thm:basis_of_direct_sum} imply that it is sufficient to take an orthonormal basis \( f_{r+1}, \ldots, f_m \) of \( \ker T^* \) in order for \( f_1, \ldots, f_m \) to be an orthonormal basis of \( V \).

  The \( (i, j) \)-th entry of the matrix of \( T \) with respect to the bases \( e_1, \ldots, e_n \) and \( f_1, \ldots, f_m \) is \( \inprod { T e_i } { f_j }_V \).
  \begin{itemize}
    \item If \( i \leq r \), then
    \begin{equation*}
      \inprod { T e_i } { f_j }_V
      =
      \sigma_i \inprod { f_i } { f_j }_V
      =
      \begin{cases}
        \sigma_i, &i = j \\
        0,        &i \neq j \\
      \end{cases}
    \end{equation*}

    \item If \( i > r \), then \( e_i \) is in the kernel of \( T \) and hence
    \begin{equation*}
      \inprod { T e_i } { f_j }_V
      =
      \inprod { 0_V } { f_j }_V
      =
      0.
    \end{equation*}
  \end{itemize}

  Therefore, the desired matrix has the form \eqref{eq:thm:singular_value_decomposition/matrix}.

  \SubProofOf{thm:singular_value_decomposition/matrix} Follows from \fullref{thm:singular_value_decomposition/basis} by defining
  \begin{equation*}
    U \coloneqq \parens*
    {
      \begin{array}{c|c|c}
        f_1 & \cdots & f_m
      \end{array}
    }
  \end{equation*}
  and
  \begin{equation*}
    V \coloneqq \parens*
    {
      \begin{array}{c|c|c}
        e_1 & \cdots & e_n
      \end{array}
    }.
  \end{equation*}

  Denote by \( u_{i,j} \), \( v_{i,j} \) and \( s_{i,j} \) the \( (i, j) \)-th entries of \( U \), \( V \) and \( \Sigma \). Note that
  \begin{equation*}
    e_k = \begin{pmatrix} v_{k,1} \\ \cdots \\ v_{k,n} \end{pmatrix}
  \end{equation*}
  and
  \begin{equation*}
    f_k = \begin{pmatrix} u_{1,k} \\ \cdots \\ u_{m,k} \end{pmatrix}.
  \end{equation*}

  Since for \( k \leq r \) we have \( \sigma_k f_k = A e_k \), for the \( i \)-th entry of \( \sigma_k f_k \) we have
  \begin{equation}\label{eq:thm:singular_value_decomposition/matrix/scalars}
    \sigma_k u_{i,k} = \sum_{l=1}^n a_{i,l} v_{k,l}.
  \end{equation}

  We have
  \begin{equation*}
    s_{i,j} = \begin{cases}
      \sigma_i, &i = j \leq r \\
      0,        &\T{otherwise.}
    \end{cases}
  \end{equation*}

  Then \( U \Sigma \) is an \( m \times n \) matrix whose \( (i, j) \)-th entry is
  \begin{equation*}
    \sum_{k=1}^m u_{i,k} s_{k,j}
    =
    \begin{cases}
      u_{i,j} \sigma_j, &j \leq r \\
      0,                &\T{otherwise.}
    \end{cases}
  \end{equation*}

  It is only nonzero if \( j \leq r \). Then the \( (i, j) \)-th entry of \( U \Sigma V^* \) is
  \begin{equation*}
    \sum_{k=1}^r u_{i,k} \sigma_k v_{k,j}
    \reloset {\eqref{eq:thm:singular_value_decomposition/matrix/scalars}} =
    \sum_{k=1}^r \sum_{l=1}^n a_{i,l} v_{k,l} v_{k,l}
    =
    \sum_{j=1}^n a_{i,j} \underbrace{ \sum_{k=1}^r v_{k,i} v_{k,j} }_{ v_i^* v_j }
    =
    a_{i,j}.
  \end{equation*}

  Therefore, \( A = U \Sigma V^* \).
\end{proof}

  \section{Algebraic dual spaces}\label{sec:algebraic_dual_spaces}

In this subsection, we restrict ourselves to fields rather than arbitrary ring.

\begin{definition}\label{def:dual_vector_space}\mcite[50]{Knapp2016BasicAlgebra}
  Let \( V \) be a \hyperref[def:vector_space]{vector space} over the \hyperref[def:field]{field} \( \BbbK \). By \fullref{thm:functions_over_algebra}, the set \( \hom(V, \BbbK) \) of all \hyperref[def:linear_function]{linear maps} from \( V \) to the underlying field \( \BbbK \) also form a vector space over \( \BbbK \).

  We will call this space the \term{algebraic dual space} of \( V \) and denote it by \( V^* \). We will call the functions in \( V^* \) \term{linear functionals}. The prefix \enquote{algebraic} is important when confusion is possible with \hyperref[def:continuous_dual_space]{continuous linear functionals}.
\end{definition}

\begin{remark}\label{rem:dual_space_bilinear_form}\mcite[16]{ИоффеТихомиров1974ЭкстремальныеЗадачи}
  If \( l \) is a \hyperref[def:dual_vector_space]{linear functional} over \( V \), we often use the notation \( \inprod l x \) rather than the function notation \( l(x) \). This is an extension of the notation for \hyperref[def:inner_product_space]{inner product spaces}.

  Moreover, \( \inprod \anon \anon \) is a \hyperref[def:multilinear_function]{bilinear function} from the Cartesian product \( V^* \times V \) to \( \BbbK \). Hence, if \( V \) is isomorphic to \( V^* \), then this is precisely an inner product.
\end{remark}

\begin{concept}\label{con:functional}
  The term \enquote{functional} as a noun has no definite meaning.

  \begin{itemize}
    \item In the context of linear algebra, and in particular \fullref{def:dual_vector_space}, the term \enquote{functional} refers to \enquote{linear functional}, i.e. a \hyperref[def:linear_function]{linear map} from a \hyperref[def:vector_space]{vector space} to its base field.

    This terminology can be found, for example, in \cite[50]{Knapp2016BasicAlgebra} and \cite[sec. 26.1]{Тыртышников2007ЛинейнаяАлгебра}.

    \item In the context of functional analysis, \enquote{linear functional} may refer to either \hyperref[def:continuous_dual_space]{continuous linear functionals} from some \hyperref[def:topological_vector_space]{topological vector space} to its base field, or to arbitrary linear functionals.

    The former terminology can be found, for example, in \cite[def. 3.1]{Rudin1991FunctionalAnalysis} and \cite[sec. 1.3]{Clarke2013OptimalControl}.

    An arbitrary map from a topological vector space to its field may also be called a functional. For example, \cite[223]{Deimling1985NonlinearFA} refers to \enquote{nonlinear functionals}. \hyperref[def:minkowski_functional]{Minkowski functionals} are notoriously nonlinear.

    \item In the context of recursive functions, for example in \cite{StanfordPlato:recursive_functions}, functionals are defined as \enquote{operations which map one or more functions of type \( \BbbN^k \to \BbbN \) (possibly of different arities) to other functions}.
  \end{itemize}

  The commonality between linear algebra and functional analysis is that \enquote{functional} refers to a map from a vector space to its base field. The commonality between functional analysis and logic is that \enquote{functional} refers to a map acting on a set of functions.
\end{concept}

\begin{remark}\label{rem:vector_space_and_dual_space}
  A vector space \( V \) over \( \BbbK \) with \hyperref[def:hamel_basis]{basis} \( E \) is, by definition, isomorphic to the \hyperref[def:free_semimodule]{free module} \( \BbbK^{\oplus E} \). We can thus regard \( V \) as the set of all \hyperref[def:set_finiteness]{finitely}-\hyperref[def:function_support]{supported} functions from \( E \) to \( \BbbK \).

  By \fullref{thm:free_semimodule_universal_property}, the linear functions from \( V \) to \( \BbbK \) are precisely the linear extensions of the functions from \( E \) to \( \BbbK \).

  It is now clear that \( V \) can be embedded in \( V^* \). This is explicitly given by the map \( e \mapsto \pi_e \), where \( \pi_e \) is the \hyperref[def:basis_decomposition]{projection} onto the basis vector \( e \).

  The space \( V \) is thus finite-dimensional if and only if \( V \) and \( V^* \) are isomorphic. We often restrict ourselves to \hyperref[def:continuous_dual_space]{continuous linear functionals}, in which case even infinite-dimensional vector spaces can be isomorphic to their duals --- see \fullref{sec:hilbert_spaces}.
\end{remark}

\begin{remark}\label{rem:finite_dimensional_dual_space_isomorphism}
  As discussed in \fullref{rem:vector_space_and_dual_space}, the vector space \( \BbbK^n \) is isomorphic to its dual.

  We discussed in \fullref{rem:matrices_as_functions} that vectors in \( \BbbK^n \) can be regarded as \hyperref[def:array/column_vector]{column vectors}. Depending on the situation, we regard linear functionals as either:
  \begin{itemize}
    \item Functions acting on vectors.
    \item Row vectors, which can be multiplied with column vectors from \( \BbbK^n \).
    \item Given the \hyperref[def:inner_product_space]{inner product} \( \inprod l x \coloneqq l^T x \), we can identify functionals with column vectors so that the functional \( l \) can be identified in \( x \mapsto \inprod l x \).
  \end{itemize}

  For example, given the real \hyperref[def:differentiability]{differentiable} function \( f(x, y) = xy \), we can regard its gradient at the point \( (x_0, y_0) \) as the row vector
  \begin{balign*}
    f'(x_0, y_0) =
    \begin{pmatrix}
      y_0 & x_0
    \end{pmatrix}.
  \end{balign*}

  This is a linear functional that acts on vectors from \( \BbbR^2 \) by multiplying them from the left.
\end{remark}

\begin{remark}\label{rem:complex_linear_functional}
  A complex-valued linear function \( l: V \to \BbbC \) is entirely determined by its real part \( \real l: V \to \BbbR \). More precisely,
  \begin{equation*}
    \imag \inprod l x = -\real i \inprod l x
  \end{equation*}
  because
  \begin{equation*}
    -\real i \inprod l x
    =
    -\real (i \real \inprod l x + i^2 \imag \inprod l x)
    =
    \imag \inprod l x.
  \end{equation*}
\end{remark}

\begin{definition}\label{def:vector_space_annihilator}\mcite[52]{Knapp2016BasicAlgebra}
  Fix a subset \( S \subseteq V \) of the vector space \( V \) over \( \BbbK \). We define the \term{annihilator} of \( S \) as the vector space of functionals
  \begin{equation*}
    S^\perp \coloneqq \set{ l \in V^* \given \qforall {x \in S} l(x) = 0_\BbbK }.
  \end{equation*}
\end{definition}

\begin{example}\label{ex:def:vector_space_annihilator}
  We list several examples of \hyperref[def:vector_space_annihilator]{vector space annihilators}:
  \begin{thmenum}
    \thmitem{ex:def:vector_space_annihilator/whole} The annihilator of the entire space \( V \) is the zero subspace
    \begin{equation*}
      V^\perp = \set{ 0_{V^*} }.
    \end{equation*}

    \thmitem{ex:def:vector_space_annihilator/zero} The annihilator of the zero subspace \( \set{ 0_V } \) is the entire space
    \begin{equation*}
      \set{ 0_V }^\perp = V^*.
    \end{equation*}

    \thmitem{ex:def:vector_space_annihilator/complement} Consider the space \( \BbbR^2 \) with basis \( \set{ x, y } \). The annihilator of the subspace
    \begin{equation*}
      \set{ tx \given t \in \BbbR }
    \end{equation*}
    is
    \begin{equation*}
      \set{ ty \given t \in \BbbR }.
    \end{equation*}
  \end{thmenum}
\end{example}

\begin{remark}\label{rem:double_dual}
  We discussed in \fullref{rem:vector_space_and_dual_space} that any vector space \( V \) can be embedded into its dual \( V^* \). The dual can, in turn, be embedded into the double dual \( V^{**} \).

  What is more remarkable is that \( V \) can be directly embedded into \( V^{**} \) via by identifying the vector \( x \) with the map \( l \mapsto l(x) \).

  This is an isomorphism if and only if \( V \) is finite dimensional. When restricted to only \hyperref[def:continuous_dual_space]{continuous functionals}, it is possible that \( V \) is isomorphic to \( V^{**} \) --- see \fullref{sec:reflexive_spaces}.
\end{remark}

\begin{theorem}\label{thm:linear_functionals_over_c}
  Let \( X \) be a \hyperref[def:vector_space]{vector space} over \( \BbbC \). There is a bijection between the real-valued and the complex-valued linear functionals on \( X \).
\end{theorem}
\begin{proof}
  Let \( c: X \to \BbbC \) be a complex-valued linear functional. Denote \( a(x) \coloneqq \real c(x) \) and \( b(x) \coloneqq \imag c(x) \). Then \( a: X \to \BbbR \) and \( b: X \to \BbbR \) are linear functionals. We will show that \( a(x) \) uniquely determines \( b(x) \) and hence \( c(x) \).

  Note that \( c(ix) = a(ix) + i b(ix) = i a(x) - b(x) \). Therefore, \( b(x) = a(ix) - c(ix) \) and
  \begin{equation*}
    c(x) = a(x) + i (a(ix) - c(ix)) = a(x) - a(x) + c(x) = c(x).
  \end{equation*}
\end{proof}

\begin{remark}\label{rem:linear_functionals_over_c}
  \Fullref{thm:linear_functionals_over_c} allows us to identify the dual space \( X* \) of a complex vector space \( X \) with \( \hom(X, \BbbR) \) in the case of an algebraic \hyperref[def:dual_vector_space]{dual} or with the corresponding subspace in the case of a \hyperref[def:continuous_dual_space]{continuous dual space}.

  This allows us to reuse some theory for real vector spaces, for example hyperplane \hyperref[def:hyperplane_separation]{separation}.
\end{remark}


  \chapter{Formal language theory}\label{ch:formal_language_theory}

Our purpose here is to study artificial languages, including ways to generate and recognize them. Chomsky's hierarchy, defined in \fullref{def:chomsky_hierarchy}, is a fundamental classification of languages by which \hyperref[def:formal_grammar]{formal grammars} can be used to generate them. We dedicate \fullref{sec:syntax_trees} to studying \hyperref[def:chomsky_hierarchy/context_free]{context-free languages} via \hyperref[def:parse_tree]{parse trees} and \fullref{sec:regular_languages} to studying \hyperref[def:chomsky_hierarchy/regular]{regular languages} via \hyperref[def:finite_automaton]{finite automata}.

\begin{concept}\label{con:metalanguage}
  We dedicate this entire chapter to studying rigidly structured languages, which we will call our \term[ru=предметный язык (\cite[35]{Герасимов2011Вычислимость}), en=object language (\cite[3]{Kleene2002Logic})]{object languages}. The monograph itself is written in a language with looser rules, which we call our \term[ru=метаязык (\cite[35]{Герасимов2011Вычислимость}), en=metalanguage (\cite[3]{Kleene2002Logic})]{metalanguage}. This distinction is important and leads to conventions like those in \fullref{rem:object_language_dots} that allow us to more easily disambiguate between the object language and metalanguage.

  In relation to logic, \fullref{con:metalogic} introduces more related notions like object logic and object theories, as well as their metalingual counterparts. \Cref{fig:con:metalogic} allows us to hierarchically visualize these concepts.
\end{concept}

\begin{concept}\label{con:syntax_semantics_duality}
  As long as the object language allows specifying numbers in \hyperref[def:positional_number_system/decimal]{decimal notation}, in the \hyperref[con:metalanguage]{metalanguage} we distinguish between the following:
  \begin{itemize}
    \item The numeral \enquote{\( 1 \)} as a single-symbol string in the object language with no inherent meaning. In many programming languages this corresponds to the three-symbol string literal expression \texttt{"1"}.

    \item The value \( 1 \) as a metalingual object, which we can formally define as the set \( \set{ \varnothing } \) (see \fullref{thm:omega_is_model_of_pa} for a broader discussion). In many programming languages this corresponds to the single-symbol number literal expression \texttt{1}.
  \end{itemize}

  Every numeral can be interpreted \enquote{within the metalanguage} as the corresponding numeric value in decimal notation, and every numeric value can be expressed as a decimal string within the object language. Distinguishing between the two in the metalanguage leads to the dot conventions from \fullref{rem:object_language_dots}.

  Of course, more complicated expressions in the object language often have intermediate forms like \hyperref[con:abstract_syntax_tree]{abstract syntax trees}. We call these strings and intermediate forms the \term[en=syntax (\cite[8]{Hinman2005Logic})]{syntax} of the object language. We call the systematic assignment of values to these objects the \term[ru=семантика (\cite[54]{КолмогоровДрагалин2006Логика}), en=semantics (\cite[8]{Hinman2005Logic})]{semantics} of the language, and we call the assignment itself \term[ru=интерпретация (\cite[17]{Герасимов2011Вычислимость}), en=interpretation (\cite[10]{Smullyan1995FOL})]{interpretation} or \term[ru=оценка (\cite[77]{ШеньВерещагин2017ЯзыкиИИсчисления})]{evaluation} (see \fullref{con:evaluation} regarding the latter). These concepts are thoroughly studied in \fullref{ch:mathematical_logic}, as well as, more abstractly, in other places like \fullref{sec:free_groups}.

  We will refer to the interaction between syntactic objects and their semantic counterparts as the \enquote{syntax-semantics duality}.
\end{concept}

\begin{remark}\label{rem:object_language_dots}
  The object language is part of the metalanguage, hence we may expect a clash of notation. Numeric strings and numbers were briefly discussed in \fullref{con:syntax_semantics_duality}, but the distinctions are often more subtle. For example, when defining propositional valuations in \fullref{def:propositional_valuation}, the symbol \( \wedge \) refers both to a logical connective and to a metalogical operation. Furthermore, their role is reversed in \fullref{def:lattice/theory}, where we study the first-order theory of lattices.

  For this reason, we introduce the following conventions:
  \begin{thmenum}
    \thmitem{rem:object_language_dots/terminals} Whenever the object language features some kind of alphabetic symbols, such as the variable identifiers as defined in \fullref{def:variable_identifier}, these symbols may coincide with our metalingual variable identifiers. In the case of such an ambiguity, we put a dot on top of the symbols in the object language.

    For example, \enquote{\( \syn u \syn v \)} is a definite two-symbol string in the object language, while \enquote{\( uv \)} refers to an expression where both are metalingual variables whose values are unspecified. This makes it theoretically possible to feature both dotted and un-dotted variables within a single string, but we will not find this useful.

    This becomes helpful because, though the entire monograph, we use the same letters for metalingual variables, and this allows highlighting variables in the object language --- see, for example, the axioms for semirings in \fullref{def:semiring/theory}, or the \hyperref[def:lambda_term]{\( \muplambda \)-terms} in \fullref{ex:def:beta_eta_reduction}.

    Another convention that can be found in \cite[\S 51]{Andrews2002Logic} is to use bold letters for metalingual variables and normal-weight letters for variables in the object language. We find Andrews' convention more inconvenient since it heavily depends on the font.

    \thmitem{rem:object_language_dots/connectives} We also put dots over all the propositional connectives in \fullref{def:propositional_alphabet}, as well as the formal equality in higher-order logic in \fullref{def:simply_typed_hol} (hence also first-order equality \fullref{def:first_order_language}). This is motivated by the examples from the beginning of this remark.

    \thmitem{rem:object_language_dots/ambiguity} In other cases like the numerical arithmetic in \fullref{ex:natural_number_arithmetic_grammar/evaluation}, where ambiguity is possible, we also place dots over symbols in the object language.
  \end{thmenum}

  \incite[rem. 2.1.3]{Hinman2005Logic} follows similar conventions, but uses the dots over function and predicate symbols mostly and avoids placing them over variables and connectives. We chose different conventions because most of the monograph is not concerned with syntax.
\end{remark}

  \section{Formal languages}\label{sec:formal_languages}

\paragraph{Languages}

Languages are used to define formulas for expressing the \hyperref[def:zfc]{axioms of set theory}. Here, sets are used to formally define languages. A simple way out of this vicious cycle is via the theory-metatheory relationship discussed in \cref{con:metalogic} and \cref{rem:set_definition_recursion}. In short, we define languages within the metatheory using the already available concept of set, and we later define formulas, again in the metatheory, which allows us to subsequently formally define sets via axioms within the object logic.

\begin{definition}\label{def:formal_language}\mcite[9]{Savage2008ModelsOfComputation}
  \hfill
  \begin{thmenum}
    \thmitem{def:formal_language/alphabet} Fix a nonempty set \( \mscrA \), which we will call an \term[ru=алфавит (\cite[19]{Гладкий1973ГрамматикиИЯзыки})]{alphabet}. Unless explicitly noted otherwise, like in \fullref{sec:free_groups}, we will assume that \( \mscrA \) is finite.

    \thmitem{def:formal_language/symbol} We call each element of \( \mscrA \) a \term[ru=символ (\cite[19]{Гладкий1973ГрамматикиИЯзыки})]{symbol}.

    \thmitem{def:formal_language/string} We call a \hyperref[def:ordered_tuple]{tuple} of symbols a \term[ru=слово (\cite[19]{Гладкий1973ГрамматикиИЯзыки}), en=word (\cite[3]{Salomaa1973FormalLanguages})]{word} or \term[ru=цепочка (\cite[19]{Гладкий1973ГрамматикиИЯзыки})]{string}. If \( (a, b, c) \) is a string, for convenience we use the notation \( abc \). This notation only makes sense if each symbol of the language is actually represented by one typographic symbol.

    \thmitem{def:formal_language/empty_string} We denote the empty string via \( \bnfves \)\fnote{This notation is used, for example, by \incite[9]{Savage2008ModelsOfComputation}. \incite[3]{Salomaa1973FormalLanguages} uses \( \lambda \) instead, while \incite[19]{Гладкий1973ГрамматикиИЯзыки} uses \( \Lambda \).}.

    \thmitem{def:formal_language/string_length} We define the \term[ru=длина (\cite[19]{Гладкий1973ГрамматикиИЯзыки})]{length} \( \len(w) \) of a string \( w \) as the length of the corresponding tuple.

    \thmitem{def:formal_language/concatenation} We define \term[ru=конкатенация (\cite[19]{Гладкий1973ГрамматикиИЯзыки})]{concatenation} of the strings \( v = (v_1, \ldots, v_n) \) and \( w = (w_1, \ldots, w_m) \) as the string
    \begin{equation*}
      v \cdot w \coloneqq (v_1, \ldots, v_n, w_1, \ldots, w_m).
    \end{equation*}

    We abbreviate \( \smash{\overbrace{w \ldots w}^{k \T*{times}}} \) as \( w^k \).\fnote{This is only a notational shortcut within the metalogic. We do not distinguish, formally, between the strings \( aaabbaa \) and \( a^3 b^2 a^2 \), nor between \( a \varepsilon b \) and \( ab \).}

    \thmitem{def:formal_language/reverse}\mimprovised We define the \term{reverse string} of \( w = (w_1, \ldots, w_n) \) as
    \begin{equation*}
      \op{rev}(w) \coloneqq (w_n, \ldots, w_1).
    \end{equation*}

    \thmitem{def:formal_language/prefix}\mimprovised We say that the string \( p = (p_1, \ldots, p_m) \) is a \term[ru=начало (\cite[20]{Гладкий1973ГрамматикиИЯзыки})]{prefix} of \( w = (w_1, \ldots, w_n) \) if
    \begin{equation*}
      w = (\underbrace{p_1, \ldots, p_m}_p, w_{m+1}, \ldots, w_n).
    \end{equation*}

    \thmitem{def:formal_language/suffix}\mimprovised We say that the string \( s \) is a \term[ru=конец (\cite[20]{Гладкий1973ГрамматикиИЯзыки})]{suffix} of \( w \) if \( \op{rev}(s) \) is a prefix of \( \op{rev}(w) \).

    \thmitem{def:formal_language/substring}\mimprovised We say that the string \( v \) is a \term{substring} of \( w \) if there exists a prefix \( p \) and a suffix \( s \) of \( v \) such that
    \begin{equation*}
      w = p v s.
    \end{equation*}

    \thmitem{def:formal_language/kleene_star}\mcite[158]{Savage2008ModelsOfComputation} We define the \term{Kleene star}\fnote{The Kleene star is a monoid --- see \cref{def:free_monoid}.} \( \mscrA^* \) of \( \mscrA \) as the set of all strings over \( \mscrA \). If we wish to exclude the empty string, like we often do, we instead write \( \mscrA^+ \) for the set of all non-empty strings over \( \mscrA \).

    \thmitem{def:formal_language/language} We call any subset of \( \mscrA^* \) a \term{language} over \( \mscrA \).
  \end{thmenum}
\end{definition}

\begin{example}\label{ex:def:formal_language}
  We list several examples of \hyperref[def:formal_language]{formal languages}:
  \begin{thmenum}
    \thmitem{ex:def:formal_language/full} The simplest examples are the empty language and the Kleene star itself.

    \thmitem{ex:def:formal_language/an} For any alphabet, we can pick one letter \( a \) and form the language
    \begin{equation*}
      \set{ a^n \given n \geq 0 }
    \end{equation*}
    consisting of all finite repetitions of the symbol \( a \).

    We can redefine all operations from \fullref{sec:natural_numbers} to hold for strings in \( \mscrL \). For example, addition of \( a^n \) and \( a^m \) is their concatenation \( a^{n + m} \), while the exponentiation \( (a^n)^m \) corresponds to \( m \) repetitions of the string \( a^n \), that is, to \( a^{nm} \).

    We can identify the language with the \hyperref[def:labeled_set]{edge-labeled} \hyperref[def:directed_graph]{directed graph}
    \begin{equation}\label{eq:def:formal_language/an}
      \begin{aligned}
        \includegraphics[page=1]{output/ex__def__formal_language}
      \end{aligned}
    \end{equation}

    Each string in \( \mscrL \) corresponds to a \hyperref[def:graph_walk/directed]{walk} in \eqref{eq:def:formal_language/an} and vice versa.

    \thmitem{ex:def:formal_language/anbn} A slightly more complicated language is
    \begin{equation*}
      \set{ a^n b^n \given n \geq 0 }.
    \end{equation*}

    It can encode natural number operations just as well. It cannot, however, be represented via a directed graph like \eqref{eq:def:formal_language/an}. This will be made precise and proved in \cref{ex:def:finite_automaton/anbn}.

    \thmitem{ex:def:formal_language/even} Even numbers in binary notation are described by the language
    \begin{equation*}
      \set[\Big]{ w \syn0 \given w \in \set{ \syn0, \syn1 }^* }.
    \end{equation*}

    \thmitem{ex:def:formal_language/leucine}\mcite[ch. 1]{Waterman1995ComputationalBiology} DNA and RNA molecules are composed of smaller molecules that are linked together. These smaller molecules are called nucleotides.

    There are four DNA nucleotides:
    \begin{itemize}
      \item Adenine (\texttt{A}).
      \item Cytosine (\texttt{C}).
      \item Guanine (\texttt{G}).
      \item Thymine (\texttt{T}).
    \end{itemize}

    We can regard DNA molecules as strings over the alphabet
    \begin{equation*}
      \set{ \texttt{A}, \texttt{C}, \texttt{G}, \texttt{T} }.
    \end{equation*}

    During a process called replication, when RNA is produced from DNA, thymine is replaced with uracil (\texttt{U}). We can regard RNA molecules are strings over the alphabet
    \begin{equation*}
      \set{ \texttt{A}, \texttt{C}, \texttt{G}, \texttt{U} }
    \end{equation*}
    corresponding exactly to the DNA strings.

    A DNA or RNA \term{codon} is a triplet of nucleotides.

    Codons describe how amino acids should be linked together. Proteins are produced from amino acids based on RNA. Out of \( 64 \) codons, \( 61 \) map to amino acids. The following DNA codons map to the corresponding RNA codons, which in turn map to the amino acid leucine:
    \begin{equation*}
       \set{ \texttt{TTA}, \texttt{TTG}, \texttt{CTT}, \texttt{CTC}, \texttt{CTA}, \texttt{CTG} }.
    \end{equation*}
  \end{thmenum}
\end{example}

\paragraph{String distance}

\begin{definition}\label{def:hamming_distance}\mcite[24]{Golan1999Semirings}
  The \term[ru=расстояние Хэмминга (\cite[\S 6.3.3]{Новиков2013ДискретнаяМатематика}), en=Hamming distance]{Hamming distance} between two \hyperref[def:formal_language/string]{strings} \( a_1 \cdots a_n \) and \( b_1 \cdots b_n \) of the same length is simply the number of indices for which \( a_k \neq b_k \).
\end{definition}

\begin{definition}\label{def:levenshtein_distance}\mcite[4]{Левенштейн1965ДвоичныеКоды}
  The \term{Levenshtein distance} from the \hyperref[def:formal_language/string]{string} \( v \) to \( w \) is the minimum number of single-symbol insertions, deletions and substitutions needed to obtain \( w \) from \( v \).

  In this context, following \incite[168]{WagnerFischer1974LevenshteinDistance}, we will refer to these operations as \enquote{edit operations}.
\end{definition}
\begin{comments}
  \item We give here Levenshtein's definition, but we will conflate it with the recursively-defined function from \cref{thm:levenshtein_distance_characterization}.

  \item It will follow from \cref{thm:levenshtein_distance_characterization}, but even here we can easily deduce that the distance from \( v \) to \( w \) coincides with the distance from \( w \) to \( v \) because every edit operation is invertible.

  \item \incite[2]{Левенштейн1965ДвоичныеКоды} also discusses a similar distance function, but without allowing substitutions. We use here the more established version, called \enquote{edit distance} by \incite[168]{WagnerFischer1974LevenshteinDistance}.
\end{comments}

\begin{proposition}\label{thm:levenshtein_distance_characterization}
  The following function describes the \hyperref[def:levenshtein_distance]{Levenshtein distance} between two strings:
  \begin{equation}\label{eq:thm:levenshtein_distance_characterization}
    l(v, w) \coloneqq \begin{cases}
      \len(w),                                             &v = \varepsilon, \\
      \len(v),                                             &w = \varepsilon, \\
      l(v', w'),                                           &v = av' \T{and} w = bw' \T{and} a = b, \\
      1 + \min\set[\Big]{ l(v, w'), l(v', w), l(v', w') }, &v = av' \T{and} w = bw' \T{and} a \neq b.
    \end{cases}
  \end{equation}
\end{proposition}
\begin{comments}
  \item In order for this definition to be precise, we must define \( l \) as a function over the Kleene star of some alphabet.
  \item If we wish to disallow substitution, following Levenshtein's original discussion in \cite[2]{Левенштейн1965ДвоичныеКоды}, we must replace the last case with
  \begin{equation*}
    \min\set[\Big]{ 1 + l(v, w'), 1 + l(v', w), 2 + l(v', w') }.
  \end{equation*}

  See the proof for the corresponding reasoning.
\end{comments}
\begin{proof}
  We use natural number induction on the minimum of the lengths of \( v \) and \( w \):
  \begin{itemize}
    \item If \( v = \varepsilon \), then \( w \) can be obtained from \( v \) only by appending every symbol of \( w \), hence the Levenshtein distance is the length of \( w \).

    \item If \( w = \varepsilon \), then \( w \) can be obtained from \( v \) only by removing every symbol of \( v \), hence the Levenshtein distance is the length of \( v \).

    \item If \( v = av' \) and \( w = bw' \) and \( a = b \), then the Levenshtein distance from \( v \) to \( w \) is the distance of \( v' \) to \( w' \). Assuming the inductive hypothesis holds for \( \min\set{ \len(v'), \len(w') } \), we conclude that \( l(v', w') \) is the corresponding Levenshtein distance.

    \item If again \( v = av' \) and \( w = bw' \) but \( a \neq b \), assuming the inductive hypothesis holds for pairs whose minimum length is less than that of \( v \) and \( w \), we have several possibilities:
    \begin{itemize}
      \item We can delete the first symbol of \( w \) and then obtain \( w' \) from \( v \). The inductive hypothesis holds for \( v \) and \( w' \), hence this requires \( 1 + l(v, w') \) operations.

      \item We can delete the first symbol from \( v \) and then obtain \( w \) from \( v' \). This requires \( 1 + l(v', w) \) operations.

      \item We can substitute the first symbol of \( v \) with that of \( w \) and then obtain \( w' \) from \( v' \). This requires \( 1 + l(v', w') \) operations.

      \item If we wish to avoid substitution and only work with addition and deletion, the last case would require \( 2 + \len(v', w') \) operations because we must first remove the initial symbol from \( w \) and add \( a \) instead.
    \end{itemize}
  \end{itemize}

  Summarizing the above, we obtain the piecewise recursive definition \eqref{eq:thm:levenshtein_distance_characterization}.
\end{proof}

\begin{proposition}\label{thm:levenshtein_distance_metric}
  As an operator on the Kleene star of some alphabet, the \hyperref[def:levenshtein_distance]{Levenshtein distance} is a \hyperref[def:metric_space]{metric}.
\end{proposition}
\begin{proof}
  \SubProofOf{def:metric_space/M1} No edit operations are needed if and only if the two strings match, that is, \( l(v, w) = 0 \) if and only if \( v = w \).

  \SubProofOf{def:metric_space/M2} It is clear from \eqref{eq:thm:levenshtein_distance_characterization} that Levenshtein distance is symmetric.

  \SubProofOf{def:metric_space/M3} For any three strings \( u \), \( v \) and \( w \), the minimum number of edit operations needed to convert \( u \) to \( w \) is bounded by the number of operations needed for \( u \) to \( v \) and then \( v \) to \( w \).
\end{proof}

\begin{example}\label{ex:def:levenshtein_distance}
  We list examples of \hyperref[def:levenshtein_distance]{Levenshtein distances}:
  \begin{thmenum}
    \thmitem{ex:def:levenshtein_distance/shift} We only need two edit operations to convert \( a b^n c \) to \( b^n c a \) --- one for deleting \( a \) from the start and another for inserting it at the end. The Levenshtein distance is thus \( 2 \) for any nonnegative integer \( n \).

    The \hyperref[def:hamming_distance]{Hamming distance} is \( 2 \) if \( n = 0 \) and otherwise it is
    \begin{equation*}
      l(a b^n c, b^n c a) = l(a b^{n-1} b c, b b^{n-1} c a) = 3.
    \end{equation*}

    \thmitem{ex:def:levenshtein_distance/spelling} A useful application of the Levenshtein metric is to perform \enquote{fuzzy search}, that is, to find the closest matching word from a list.

    For example, a spelling correction program that works on the dictionary
    \begin{equation*}
      \set{ \texttt{junction}, \texttt{conjunction}, \texttt{disjunction} }
    \end{equation*}
    will be able to correct the string \( \texttt{conunction} \) to \( \texttt{conjunction} \) because
    \begin{equation*}
      l(\texttt{conunction}, \texttt{con}\oline{\texttt{j}}\texttt{unction}) = 1,
    \end{equation*}
    while
    \begin{equation*}
      l(\hi{\texttt{con}}\texttt{unction}, \oline{\texttt{j}}\texttt{unction}) = 1 + l(\hi{\texttt{co}}\oline{\texttt{j}}\texttt{unction}, \oline{\texttt{j}}\texttt{unction}) = 3
    \end{equation*}
    and
    \begin{equation*}
      l(\hi{\texttt{con}}\texttt{unction}, \hi{\texttt{dis}}\oline{\texttt{j}}\texttt{unction}) = 1 + l(\hi{\texttt{con}}\oline{\texttt{j}}\texttt{unction}, \hi{\texttt{dis}}\oline{\texttt{j}}\texttt{unction}) = 4.
    \end{equation*}

    On the other hand, \( \texttt{subjunction} \) is at a distance of \( 3 \) from all three strings.
  \end{thmenum}
\end{example}

\begin{algorithm}[Wagner-Fisher algorithm for Levenshtein distance]\label{alg:wagner_fisher}\mcite[171; 172]{WagnerFischer1974LevenshteinDistance}
  Fix two strings
  \begin{align*}
    v = a_1 \cdots a_n && w = b_1 \cdots b_m.
  \end{align*}

  The \hyperref[def:levenshtein_distance]{Levenshtein distance} from the prefix \( a_1 \cdots a_j \) to the prefix \( b_1 \cdots b_i \) can be computed recursively as follows:
  \begin{equation*}
    d_{i,j} = \begin{cases}
      j,                                                         &i = 0, \\
      i,                                                         &j = 0, \\
      \min\set{ 1 + d_{i-1,j}, 1 + d_{i,j-1}, 0 + d_{i-1,j-1} }, &i > 0 \T{and} j > 0 \T{and} a_i = b_i, \\
      \min\set{ 1 + d_{i-1,j}, 1 + d_{i,j-1}, 1 + d_{i-1,j-1} }, &i > 0 \T{and} j > 0 \T{and} a_i \neq b_i.
    \end{cases}
  \end{equation*}

  The algorithm itself consists of arranging these values into a \( (m + 1) \times (n + 1) \) matrix.
\end{algorithm}
\begin{comments}
  \item Unlike in \fullref{sec:matrices_over_rings}, both the row and the column index here must start at zero.
  \item This algorithm can be found as \identifier{lang.distance.wagner_fisher} in \cite{notebook:code}.
\end{comments}

\paragraph{Formal grammars}

\begin{definition}\label{def:rewriting_system}\mcite[6]{Salomaa1973FormalLanguages}
  An \term[en=rewriting system (\cite[6]{Salomaa1973FormalLanguages})]{abstract rewriting system} over the \hyperref[def:formal_language/language]{language} \( \mscrL \) is simply a \hyperref[def:binary_relation]{binary relation} over \( \mscrL \). It is conventional to denote this relation via an arrow, for example \( {\to} \). When regarded as a set of \hyperref[def:ordered_tuple]{ordered pairs}, we refer to elements of the relation \( {\to} \) as \term{rewriting rules} or \term{production rules}.
\end{definition}

\begin{definition}\label{def:formal_grammar}\mcite[def. 4.9.1]{Savage2008ModelsOfComputation}
  Let \( \Sigma \) and \( V \) be disjoint nonempty subsets of some \hyperref[def:formal_language]{alphabet}, whose members we call \term[ru=основные (символы) (\cite[27]{Гладкий1973ГрамматикиИЯзыки})]{terminals} and \term[ru=вспомогательные (символы) (\cite[27]{Гладкий1973ГрамматикиИЯзыки})]{nonterminals}, respectively. Fix some \term{starting nonterminal} \( S \in V \).

  Let \( {\to} \) be some \hyperref[def:binary_relation]{binary relation} over \( (V \cup \Sigma)^* \), whose members we call \term[ru=правила (\cite[27]{Гладкий1973ГрамматикиИЯзыки})]{production rules}. We impose the restriction that, for every rule \( v \to w \), the string \( v \) contains at least one nonterminal.

  We call the quadruple \( (V, \Sigma, \to, S) \) a \term[ru=(формальная) генеративная грамматика (\cite[10]{Гладкий1973ГрамматикиИЯзыки})]{formal generative grammar}.

  \begin{thmenum}
    \thmitem{def:formal_grammar/derivation} We define the binary relation \( \Rightarrow \) on the Kleene star \( (V \cup \Sigma)^* \) by declaring that, for every two \hyperref[def:formal_language/string]{strings} \( p \) and \( s \) over \( V \cup \Sigma \) and every production rule \( v \to w \), we have \( pvs \Rightarrow pws \). We call this an \term[ru=непосредственный (вывод) (\cite[28]{Гладкий1973ГрамматикиИЯзыки})]{immediate derivation}.

    A \term[ru=вывод (\cite[27]{Гладкий1973ГрамматикиИЯзыки})]{derivation} of length \( m \) of the string \( w_m \) from \( w_0 \) is a list of strings such that
    \begin{equation}\label{eq:def:formal_grammar/derivation}
      w_0
      \Rightarrow
      w_1
      \Rightarrow
      \cdots
      \Rightarrow
      w_{n-1}
      \Rightarrow
      w_m.
    \end{equation}

    We say that \( w_m \) is (immediately) \term[ru=выводимая (цепочка) (\cite[27]{Гладкий1973ГрамматикиИЯзыки})]{derivable} from \( w_0 \) if there exists a corresponding derivation

    We denote the \hyperref[def:relation_closures/transitive]{transitive closure} of \( \Rightarrow \) by \( \reloset + \Rightarrow \) and the \hyperref[def:relation_closures/reflexive]{reflexive} closure of \( \reloset + \Rightarrow \) by \( \reloset {*} \Rightarrow \). Clearly \( w_1 \) is derivable from \( w_n \) if and only if \( w_1 \reloset {*} \Rightarrow w_m \).

    \thmitem{def:formal_grammar/language} We associate the following language with the grammar \( G \):
    \begin{equation*}
      \mscrL(G) \coloneqq \set{ w \in \Sigma^* \given S \reloset + \Rightarrow w }.
    \end{equation*}

    It consists of all strings that are derivable from \( S \) and contain only terminals.

    We say that strings in \( \mscrL(G) \) are \term{generated} by \( G \).

    \thmitem{def:formal_grammar/equivalent} We say that two grammars are \term{equivalent} if they generate the same language.

    \thmitem{def:formal_grammar/graph}\mimprovised We can regard the derivation relation \( \Rightarrow \) as a set of \hyperref[def:directed_graph/arcs]{arcs} over the grammar language \( \mscrL(G) \). Thus, \( (\mscrL(G), \Rightarrow) \) is a \hyperref[def:directed_graph]{directed graph}\fnote{It is possible for multiple rules to produce the same immediate derivation, however \( \Rightarrow \) is only concerted that at least one exists.}, whose nonempty walks are precisely the derivations in \( G \). Furthermore, we can \hyperref[def:labeled_set]{label} each arc with the rule applied.

    We call it the \term{derivation graph} of \( G \).

    \thmitem{def:formal_grammar/schema}\mcite[27]{Гладкий1973ГрамматикиИЯзыки} There are different grammars sharing the same alphabet and rules, but having different starting nonterminals. We will call the set of rules a \term[ru=схема]{grammar schema}.

    The schema rather that the grammar itself is a \hyperref[def:rewriting_system]{rewriting system}.
  \end{thmenum}
\end{definition}

\begin{proposition}\label{thm:derivation_graph_connected}
  Every \hyperref[def:formal_grammar/graph]{derivation graph} is \hyperref[def:graph_connectedness/weak]{weakly connected}.
\end{proposition}
\begin{proof}
  A \hyperref[def:formal_grammar/language]{grammar's language} is defined as the set of all strings that can be derived from the starting nonterminal.
\end{proof}

\begin{example}\label{ex:natural_number_arithmetic_grammar/schema}
  We define a \hyperref[def:formal_grammar/schema]{grammar schema} for arithmetic of \hyperref[def:natural_numbers]{natural numbers}. We will use binary notation for simplicity.

  Let \( V \coloneqq \set{ N, O, E } \) and \( \Sigma \coloneqq \set{ \syn0, \syn1, \synplus, \syntimes, (,) } \). Consider the derivation rules
  \begin{equation}\label{eq:ex:natural_number_arithmetic_grammar/schema/simple}
    \begin{aligned}
      N &\to \syn0   & \quad B &\to \syn0   & \quad O &\to \synplus  & \quad E &\to N       \\
      N &\to \syn1   & \quad B &\to \syn0 B & \quad O &\to \syntimes & \quad E &\to (E O E) \\
      N &\to \syn1 B & \quad B &\to \syn1   & \quad   &              &         &            \\
        &            & \quad B &\to \syn1 B & \quad   &              &         &
    \end{aligned}
  \end{equation}

  Following the convention from \cref{rem:object_language_dots}, we have placed dots above the terminal symbols to avoid ambiguity, especially later in \cref{ex:natural_number_arithmetic_grammar/evaluation} when we will discuss evaluation of these symbols.

  It is convenient to use the following shorthands:
  \begin{equation}\label{eq:ex:natural_number_arithmetic_grammar/schema/shorthand}
    \begin{aligned}
      N &\to \syn0 \mid \syn1   \mid \syn1 B \\
      B &\to \syn0 \mid B \syn0 \mid \syn1 \mid B \syn1 \\
      O &\to \synplus \mid \times \\
      E &\to N     \mid (E O E)
    \end{aligned}
  \end{equation}

  Choosing a different starting nonterminal generates different languages. The symbol \( N \) corresponds to numbers, \( O \) corresponds to operations, and \( E \) can be either a number or a binary expression.

  \begin{figure}[!ht]
    \centering
    \includegraphics[page=1]{output/ex__natural_number_arithmetic_grammar__rules}
    \caption{A fragment of the \hyperref[def:formal_grammar/graph]{derivation graph} of the binary natural number arithmetic grammar from \cref{ex:natural_number_arithmetic_grammar/schema}.}
    \label{fig:ex:natural_number_arithmetic_grammar/schema}
  \end{figure}
\end{example}

\paragraph{Length-increasing grammars}

\begin{remark}\label{rem:length_increasing_grammar}
  Given a finite set of grammar rules, there may be \hyperref[def:formal_grammar/derivation]{derivations} of arbitrary length. This can happen even if we remove cycles from the \hyperref[def:formal_grammar/graph]{derivation graph} --- see \cref{ex:unboudned_grammar_derivation_length}. We will discuss a way to restrict this behavior and ensure that it is possible to determine whether a derivation exists in finitely many steps.

  In general, given the grammar rule \( v \to w \), it is possible that \( \len(v) \geq \len(w) \). This is always true for \hyperref[def:epsilon_free_grammar]{\( \varepsilon \) rules}, for example.

  \incite*[361]{Chomsky1963Grammars} defines \enquote{type 1} grammars as those satisfying the inequality \( \len(v) \leq \len(w) \) for every rule \( v \to w \). These grammars still have cycles like \( A \to B \to A \), but they avoid more convoluted cases and lead to \Fullref{alg:context_sensitive_string_membership}.

  This restriction excludes the empty string from the grammar's language. \incite[15]{Salomaa1973FormalLanguages} additionally allows the rule \( S \to \varepsilon \), but only if \( S \) does not occur on the right side of any derivation. \Cref{ex:unboudned_grammar_derivation_length} highlights the importance of the latter assumption.

  Salomaa calls these grammars \enquote{length-increasing}, preferring to use \enquote{type 1} for what we call \hyperref[def:chomsky_hierarchy/context_sensitive]{context-sensitive} grammars. The latter are, up to nuances of \( \varepsilon \) rules, called \enquote{type 1} grammars by Chomsky earlier in \cite[142]{Chomsky1959Grammars}. John Savage in \cite[def. 4.9.2]{Savage2008ModelsOfComputation} defines \enquote{context-sensitive grammars} as what Salomaa calls length-increasing. We introduce the term \enquote{essentially length-increasing} in \cref{def:length_increasing_grammar} and generally avoid \enquote{type 1} when referring to grammars or languages to circumvent ambiguity.

  (What we call) context-sensitive \hi{grammars} is a strict subset of the essentially length-increasing grammars, but \cref{thm:context_sensitive_languages} shows that context-sensitive \hi{languages} and length-increasing languages coincide. Hence, the term \enquote{type 1 language} makes sense, although we prefer \enquote{context-sensitive language}.
\end{remark}

\begin{definition}\label{def:epsilon_free_grammar}\mcite[54]{Salomaa1973FormalLanguages}
  Fix a formal grammar \( G = (V, \Sigma, \to, S) \). Rules of the form \( A \to \varepsilon \), where \( A \) is a nonterminal, play a special role. We call them \term{\( \varepsilon \) rules}.

  We say that \( G \) is \term{\( \mathbf{\varepsilon} \)-free} if there are no \( \varepsilon \) rules.

  This condition sometimes turns out to be too restrictive because it disallows the grammar to generate empty strings. Thus, we introduce another concept. We say that \( G \) is \term{essentially \( \mathbf{\varepsilon} \)-free} if \( S \to \varepsilon \) is the only \( \varepsilon \) rule allowed, and if it is present, \( S \) must not be on the right side of any rule.
\end{definition}
\begin{comments}
  \item This definition for essentially \( \varepsilon \)-free grammars is our own generalization of the concept of essentially length-increasing grammars defined in \cref{def:length_increasing_grammar}.
\end{comments}

\begin{definition}\label{def:length_increasing_grammar}\mcite[15]{Salomaa1973FormalLanguages}
  We say that the grammar \( G = (V, \Sigma, \to, S) \) is \term[ru=неукорачивающая (граматика) (\cite[83]{Гладкий1973ГрамматикиИЯзыки})]{length-increasing} if \( \len(v) \leq \len(w) \) for every rule \( v \to w \) and \term{essentially length-increasing} if it is \hyperref[def:epsilon_free_grammar]{essentially \( \varepsilon \)-free} and if \( \len(v) \leq \len(w) \) for any non-\( \varepsilon \) rule.
\end{definition}

\begin{lemma}\label{thm:length_increasing_grammar}
  Fix an \hyperref[def:length_increasing_grammar]{essentially length-increasing} grammar \( G = (V, \Sigma, \to, S) \). Let \( m \) be the cardinality of \( \Sigma \cup V \).

  If the string \( w \) over \( \Sigma \) of length \( n \) is derivable, there exists a derivation of length at most
  \begin{equation*}
    \sum_{k=0}^n m^k.
  \end{equation*}
\end{lemma}
\begin{comments}
  \item If \( m > 1 \), we have
  \begin{equation*}
    \sum_{k=0}^n m^k
    \reloset {\ref{thm:def:geometric_series/finite_sum}} =
    \frac {1 - m^{n+1}} {1 - m}.
  \end{equation*}

  \item Chomsky introduced length-increasing grammars in \cite[360]{Chomsky1963Grammars} and on the same page hinted at this property of theirs, which we will use in \fullref{alg:context_sensitive_string_membership}.
\end{comments}
\begin{proof}
  Suppose that the string \( w \) of length \( n \) is derivable in \( G \) and consider the derivation
  \begin{equation*}
    S = w_0 \Rightarrow w_1 \Rightarrow \cdots \Rightarrow w_r = w.
  \end{equation*}

  The case where \( w \) is empty should be handled separately. Since \( G \) is essentially \( \varepsilon \)-free, the only possible derivation of \( \varepsilon \) follows the rule \( S \to \varepsilon \) and thus has length
  \begin{equation*}
    1 = m^0 = \sum_{k=0}^0 m^k.
  \end{equation*}

  Now suppose that \( n > 0 \). Let \( i_0 \coloneqq 0 \) and, for every \( s = 1, \ldots, n \), let \( i_s \) be the index of the last string of length \( s \) in the derivation (if no string of length \( s \) exists, let \( i_s \) match \( i_{s-1} \)). Note that there are exactly \( m^s \) possible strings of length \( s \), and thus there must exist a derivation of \( w_{i_s} \) from \( w_{i_{s-1} + 1} \) in at most \( m^s - 1 \) steps. Any derivation longer than that necessarily follows a \hyperref[def:graph_cycle]{cycle} in the \hyperref[def:formal_grammar/graph]{derivation graph}.

  Therefore, \( w_{i_s} \) can be derived from \( w_{i_0} = S \) in at most
  \begin{equation*}
    \sum_{k=0}^s (m^k - 1) \leq \sum_{k=0}^s m^k
  \end{equation*}
  steps.
\end{proof}

\begin{example}\label{ex:unboudned_grammar_derivation_length}
  Consider the grammar
  \begin{equation}\label{eq:ex:unboudned_grammar_derivation_length/bad}
    \begin{aligned}
      S &\to \varepsilon \\
      S &\to a, \\
      S &\to SS.
    \end{aligned}
  \end{equation}

  It is not \hyperref[def:epsilon_free_grammar]{essentially epsilon-free}, hence also not \hyperref[def:length_increasing_grammar]{essentially length-increasing}, and thus \cref{thm:length_increasing_grammar} does not apply. The lemma would imply that the acyclic derivation length should be bounded by \( 2 \), while there exist countably many acyclic derivations of the string \( a \):
  \begin{equation*}
    \begin{aligned}
      S &\Rightarrow a \\
      S &\Rightarrow SS \Rightarrow Sa \Rightarrow a \\
      S &\Rightarrow SS \Rightarrow SSS \Rightarrow SSa \Rightarrow Sa \Rightarrow a \\
        &\vdots
    \end{aligned}
  \end{equation*}

  This is problematic for parsing algorithms because we cannot check in finitely many steps whether a grammar generates a string. \Fullref{alg:epsilon_rule_removal} suggests instead the essentially length-increasing grammar
  \begin{equation*}
    \begin{aligned}
      S &\to \varepsilon \\\
      S &\to A, \\
      A &\to a, \\
      A &\to AA,
    \end{aligned}
  \end{equation*}
  which disallows the aforementioned derivations.
\end{example}

\begin{algorithm}[String membership in context-sensitive languages]\label{alg:context_sensitive_string_membership}
  Let \( G = (V, \Sigma, \to, S) \) be an \hyperref[def:length_increasing_grammar]{essentially length-increasing} grammar for \( \mscrL \). Denote by \( m \) the cardinality of \( V \cup \Sigma \). Fix some upper bound \( n \) on lengths of strings. We can construct a set \( L_n \) that contains all strings in \( \mscrL(G) \) of length at most \( n \) (it may also contain longer strings).

  \begin{thmenum}
    \thmitem{alg:context_sensitive_string_membership/start} Start with a set \( W_0 \coloneqq \set{ S } \) of strings derivable in zero steps.

    \thmitem{alg:context_sensitive_string_membership/step} Given \( W_k \), define the set of strings derivable in \( k + 1 \) steps:
    \begin{equation*}
      W_{k+1} \coloneqq \set{ p v s \given p u s \in W_k \T{and} u \to v }.
    \end{equation*}

    \thmitem{alg:context_sensitive_string_membership/union} \Cref{thm:length_increasing_grammar} gives us an upper bound on the derivation length of strings of length \( n \). Denote this bound by \( u \). We can thus take the union of all strings derivable in at most \( u \) steps and ignore those that contain nonterminals:
    \begin{equation*}
      L_n \coloneqq \bigcup_{k=1}^u \set{ w \in W_k \given w \T{has only terminals} }.
    \end{equation*}

    \thmitem{alg:context_sensitive_string_membership/membership} If we are interested in whether a particular string \( w \) of length at most \( n \) is in \( \mscrL(G) \), we can simply check if it is in \( L_n \), or, even better, at every step of the algorithm check if it is in \( W_k \).
  \end{thmenum}
\end{algorithm}
\begin{comments}
  \item This algorithm can be used to test whether a string belongs to a \hyperref[def:chomsky_hierarchy/context_sensitive]{context-sensitive language}, hence the name.

  \item Compare this to the more complicated \fullref{alg:brute_force_parsing} that is intended for more restricted cases, but also handles \( \varepsilon \) rules.
\end{comments}

\paragraph{Hierarchy of grammars}

\begin{definition}\label{def:chomsky_hierarchy}\mcite[15]{Salomaa1973FormalLanguages}
  We can classify \hyperref[def:formal_grammar]{formal grammars} and \hyperref[def:formal_language]{languages} to form the \term{Chomsky hierarchy}. Chomsky himself in \cite[def. 6]{Chomsky1959Grammars} defined a hierarchy of grammars consisting of four types --- \enquote{type 0} through \enquote{type 3}. He also defined a parallel hierarchy of languages, in which \( \mscrL \) is a \enquote{type \( i \) language} if there exists a type \( i \) grammar \hyperref[def:formal_grammar/language]{generating it}.

  Unfortunately, these definitions later evolved to be inconsistent across authors, and even among different publications by Chomsky. We thus entirely avoid numeric grammar and language types, and use more descriptive names instead. The grammars no longer form a hierarchy, but the corresponding languages do.

  \begin{thmenum}
    \thmitem{def:chomsky_hierarchy/unrestricted} When no additional restrictions are imposed on the rules of the grammar, we call it \term{unrestricted}. We call the corresponding languages \term{recursively enumerable} following \cite[thm 5.4.1; thm 5.4.2]{Savage2008ModelsOfComputation}.

    \thmitem{def:chomsky_hierarchy/context_sensitive} We say that the grammar is \term[ru=грамматика составляющих (\cite[29]{Гладкий1973ГрамматикиИЯзыки})]{context-sensitive} if it is \hyperref[def:epsilon_free_grammar]{essentially \( \varepsilon \)-free} and if every non-\( \varepsilon \) rule has the form
    \begin{equation*}
      p A s \to p w s
    \end{equation*}
    for a nonterminal \( A \), arbitrary strings \( p \) and \( s \) and a nonempty\footnote{Since \( w \) is nonempty, a context-sensitive grammar is \hyperref[def:length_increasing_grammar]{essentially length-increasing}.} string \( w \). Of course, there may be multiple such representations for a single rule.

    We call the corresponding languages \term{context-sensitive}. The kerfuffle surrounding the term \enquote{context-sensitive} is discussed in \cref{rem:length_increasing_grammar}. \Cref{thm:context_sensitive_languages} better characterizes these languages.

    \thmitem{def:chomsky_hierarchy/context_free} We say that the grammar is \term[ru=безконтекстная / контекстно-свободная (грамматика) (\cite[29]{Гладкий1973ГрамматикиИЯзыки})]{context-free} if every rule has the form
    \begin{equation*}
      A \to w,
    \end{equation*}
    where \( A \) is a nonterminal and \( w \) is an arbitrary string.

    We call the corresponding languages \term{context-free}. While it is possible for a context-free \hi{grammar} to not be \hyperref[def:epsilon_free_grammar]{essentially \( \varepsilon \)-free} and hence not context-sensitive, a context-free \hi{language} is context-sensitive. This is shown in \cref{thm:context_free_languages_are_context_sensitive} and discussed in \cref{rem:chomsky_hierarchy_failure}.

    \thmitem{def:chomsky_hierarchy/regular}\mcite[44]{Salomaa1973FormalLanguages} Finally, we call the grammar \term{left linear} if every rule has one of the forms
    \begin{itemize}
      \item \( A \to w \),
      \item \( A \to B w \),
    \end{itemize}
    where \( w \) is an arbitrary (possibly empty) string consisting entirely of terminals. Similarly, the rule is \term{right linear} if every rule has one of the forms
    \begin{itemize}
      \item \( A \to w \),
      \item \( A \to w B \).
    \end{itemize}

    We refer to the two types of grammars collectively as \term[ru=автоматная (грамматика) (\cite[29]{Гладкий1973ГрамматикиИЯзыки})]{regular grammars}.

    We call the language \term{regular} if it can be generated by a regular grammar. \Cref{thm:regular_languages} better characterizes these languages.
  \end{thmenum}
\end{definition}
\begin{comments}
  \item In \cite[142]{Chomsky1959Grammars}, Chomsky calls context-free languages \enquote{type 2} and regular languages \enquote{type 3}, while in \cite[366]{Chomsky1963Grammars} he calls context-free languages \enquote{type 4} and gives no number for regular languages.
\end{comments}

\begin{proposition}\label{thm:non_recursively_enumerable_language}\mcite[thm. 5.7.4]{Savage2008ModelsOfComputation}
  There exists a formal language that cannot be generated by a grammar.
\end{proposition}

\begin{example}\label{ex:def:chomsky_hierarchy}
  We give several examples of grammars in the \hyperref[def:chomsky_hierarchy]{Chomsky hierarchy}.

  \begin{thmenum}
    \thmitem{ex:def:chomsky_hierarchy/non_enumerable} While every grammar is an unrestricted grammar, \cref{thm:non_recursively_enumerable_language} shows that not every language is recursively enumerable.

    \thmitem{ex:def:chomsky_hierarchy/an} The \hyperref[def:chomsky_hierarchy/regular]{right linear grammar}
    \begin{equation*}
      \begin{aligned}
        S &\to aS \mid \varepsilon
      \end{aligned}
    \end{equation*}
    describes the language \( \mscrL = \set{ a^n \given n \geq 0 } \) discussed in \cref{ex:def:formal_language/an}.

    It can also be described via the left linear grammar
    \begin{equation*}
      \begin{aligned}
        S &\to Sa \mid \varepsilon.
      \end{aligned}
    \end{equation*}

    Neither of these grammars is \hyperref[def:epsilon_free_grammar]{essentially \( \varepsilon \)-free}. \Cref{ex:alg:epsilon_rule_removal/an} proposes using \Fullref{alg:epsilon_rule_removal} to obtain the essentially length-increasing grammar
    \begin{equation*}
      \begin{aligned}
        S &\to A \mid \varepsilon, \\
        A &\to aA \mid a.
      \end{aligned}
    \end{equation*}

    \thmitem{ex:def:chomsky_hierarchy/anbn} Consider the \hyperref[def:chomsky_hierarchy/context_free]{context-free grammar}
    \begin{equation*}
      \begin{aligned}
        S &\to A \mid \varepsilon, \\
        A &\to aAb \mid ab
      \end{aligned}
    \end{equation*}
    describing \( \mscrL = \set{ a^n b^n \given n \geq 0 } \) from \cref{ex:def:formal_language/anbn}.

    We have shown in \cref{ex:def:finite_automaton/anbn} that this language cannot be recognized by a finite automaton. In \fullref{sec:regular_languages} we will prove \cref{thm:regular_languages}, which suggests that \( \mscrL \) is not \hyperref[def:chomsky_hierarchy/regular]{regular}.

    Hence, \( \mscrL \) is a context-free language that is not regular.

    \thmitem{ex:def:chomsky_hierarchy/length_increasing_not_context_sensitive} In a \hyperref[def:chomsky_hierarchy/context_sensitive]{context-sensitive grammar}, the only possible production rules replace a nonterminal with some nonempty string. In general, it is possible to replace terminal symbols with arbitrary strings. For example, consider the grammar
    \begin{equation*}
      \begin{aligned}
         S &\to aA, \\
        aA &\to BB \\
         B &\to b \\
      \end{aligned}
    \end{equation*}

    The language generated by this grammar is \( \mscrL = \set{ bb } \). The entirety of \( aA \) gets replaced by a new string not starting with \( a \), while in a context-sensitive grammar the second rule would need to include the terminal \( a \) as the first symbol --- only the nonterminal \( A \) would get replaced, and only when it is preceded by \( a \). Of course, the grammar can be vastly simplified:
    \begin{equation*}
      S \to bb.
    \end{equation*}

    It is reasonable to expect that grammars are context-sensitive, i.e. that terminals should never get replaced. Unfortunately, as we just saw, this is not so general \hyperref[def:length_increasing_grammar]{length-increasing} grammars. \Fullref{alg:length_increasing_to_context_sensitive} however allows us to convert general (essentially) length-increasing grammars to equivalent context-sensitive grammars.
  \end{thmenum}
\end{example}

\paragraph{Context-sensitive languages}

\begin{algorithm}[Length-increasing to context-sensitive grammar]\label{alg:length_increasing_to_context_sensitive}\mcite[thm. 9.2]{Salomaa1973FormalLanguages}
  Fix an \hyperref[def:length_increasing_grammar]{essentially length-increasing} grammar \( G = (V, \Sigma, \to, S) \). We will build an \hyperref[def:formal_grammar/equivalent]{equivalent} (essentially length-increasing) \hyperref[def:chomsky_hierarchy/context_sensitive]{context-sensitive} grammar \( G' = (V', \Sigma, \to', S) \) with the same terminals.

  Enumerate all non-\( \varepsilon \) rules of \( \to \) from \( 1 \) to \( n \).

  \begin{thmenum}
    \thmitem{alg:length_increasing_to_context_sensitive/init} For every terminal \( a \) in \( \Sigma \), let \( a' \) be a new nonterminal not in \( V \). Let
    \begin{equation*}
      V_0 \coloneqq V \cup \set{ a' \given a \in \Sigma }.
    \end{equation*}

    For every nonterminal \( A \) in \( V_0 \), let \( A' \) refer to \( A \) itself.

    Let \( {\to_0} \) consist only of the rules \( a' \mapsto a \) for every terminal \( a \) in \( \Sigma \).

    \thmitem{alg:length_increasing_to_contexs_sensitive/step} At step \( k \), we consider the rule
    \begin{equation*}
      r_1 \cdots r_{m_k} \to s_1 \cdots s_{l_k},
    \end{equation*}
    where \( r_1 \cdots r_{m_k} \) and \( s_1 \cdots s_{l_k} \) are either terminal or nonterminal symbols.

    Let \( C_1, \ldots, C_{m_k} \) be new nonterminal symbols, and let
    \begin{equation*}
      V_k \coloneqq V_{k-1} \cup \set{ C_1, \ldots, C_{m_k} }.
    \end{equation*}

    Define \( \to'_k \) as \( \to'_{k-1} \) with the following additional rules:
    \begin{align*}
      r_1' r_2' \cdots r_{m_k}'                                      &\to'_k C_1 r_2' \cdots r_{m_k}', \\
      C_1 r_2' \cdots r_{m_k}'                                       &\to'_k C_1 C_2 \cdots r_{m_k}', \\
                                                                     &\vdots, \\
      C_1 C_2 \cdots C_{m_k-1} r_{m_k}'                              &\to'_k C_1 C_2 \cdots C_{m_k-1} C_{m_k} s_{m_k+1}' \cdots s_{l_k}', \\
      C_1 C_2 \cdots C_{m_k-1} C_{m_k} s_{m_k+1}' \cdots s_{l_k}'    &\to'_k s_1' C_2 \cdots C_{m_k-1} C_{m_k} s_{m_k+1}' \cdots s_{l_k}', \\
                                                                     &\vdots, \\
      s_1' s_2' \cdots s_{m_k-1}' C_{m_k} s_{m_k+1}' \cdots s_{l_k}' &\to'_k s_1' s_2' \cdots s_{m_k-1}' B_{m_k} s_{m_k+1}' \cdots s_{l_k}'.
    \end{align*}

    Each of these rules replaces exactly one nonterminal with a nonempty string.

    \thmitem{alg:length_increasing_to_context_sensitive/finish} Finally, let \( V' \coloneqq V'_n \) and \( {\to'} \coloneqq {\to'_n} \). In the case where \( S \to \varepsilon \), let \( S \to' \varepsilon \). Then \( G' = (V', \Sigma, \to', S) \) is the desired grammar.
  \end{thmenum}
\end{algorithm}

\begin{proposition}\label{thm:context_sensitive_languages}
  The class of languages generated by \hyperref[def:chomsky_hierarchy/context_sensitive]{context-sensitive grammars} and by \hyperref[def:length_increasing_grammar]{essentially length-increasing} grammars coincide.
\end{proposition}
\begin{comments}
  \item As discussed in \cref{rem:length_increasing_grammar} and \cref{def:chomsky_hierarchy/context_sensitive}, we call this class of languages \enquote{context-sensitive languages}.
\end{comments}
\begin{proof}
  Every context-sensitive grammar is, by definition, essentially length-increasing. \Fullref{alg:length_increasing_to_context_sensitive} shows that every essentially length-increasing grammar has an equivalent context-sensitive grammar.
\end{proof}

\begin{remark}\label{rem:chomsky_hierarchy_failure}
  Every \hyperref[def:chomsky_hierarchy/regular]{regular grammar} is \hyperref[def:chomsky_hierarchy/context_free]{context-free}, but not every context-free grammar is \hyperref[def:epsilon_free_grammar]{essentially \( \varepsilon \)-free} and \hyperref[def:chomsky_hierarchy/context_sensitive]{context-sensitive}. Chomsky disallowed \( \varepsilon \) rules in \cite[def. 6]{Chomsky1959Grammars}, and this led to a tidy hierarchy because it made context-free grammars necessarily context-sensitive.

  Fortunately, context-free \hi{languages} are context-sensitive as a consequence of \cref{thm:context_free_languages_are_context_sensitive}.

  Unfortunately, for an arbitrary context-free grammar, \cref{thm:length_increasing_grammar} does not hold, and neither does \fullref{alg:context_sensitive_string_membership}. This is not a problem because:
  \begin{itemize}
    \item For theoretical purposes, we can use \fullref{alg:epsilon_rule_removal} to adapt context-free languages to \fullref{alg:context_sensitive_string_membership}.

    \item For practical purposes, parsing context-free languages is a topic in itself. These algorithms are mostly restricted to certain classes of context-free grammars; see e.g. \cite[ch. 6]{Salomaa1973FormalLanguages}. General algorithms for parsing all context-free grammars are scarcer; several are discussed in \cite{Economopoulos2006ParsingAlgorithms}. We present \fullref{alg:brute_force_parsing}, which handles arbitrary context-free grammars, but is too inefficient to use in practice.
  \end{itemize}
\end{remark}

\paragraph{Context-free languages}

\begin{algorithm}[Epsilon rule removal]\label{alg:epsilon_rule_removal}\mcite[thm. 6.2]{Salomaa1973FormalLanguages}
  Fix a \hyperref[def:chomsky_hierarchy/context_free]{context-free} grammar \( G = (V, \Sigma, \to, S) \). We will build an \hyperref[def:formal_grammar/equivalent]{equivalent} context-free grammar \( G' = (V', \Sigma, \to', S') \) with the same terminals, such that \( G' \) is \hyperref[def:epsilon_free_grammar]{essentially \( \varepsilon \)-free}, and thus context-sensitive.

  \begin{thmenum}
    \thmitem{alg:epsilon_rule_removal/init} We will recursively construct a set \( U \) so that \( A \reloset {*} \Rightarrow \varepsilon \) if and only if \( A \in U \).

    First, define
    \begin{equation*}
      U_k \coloneqq \begin{cases}
        \set[\Big]{ A \in V \given A \to \varepsilon },                               &k = 0, \\
        U_{k-1} \cup \set[\Big]{ A \in V \given \qexists*{w \in U_{k-1}^*} A \to w }, &k > 0.
      \end{cases}
    \end{equation*}

    At each step, we \( U_k \) is a subset of \( V \). Since \( V \) has only finitely many nonterminals, the sequence \hyperref[def:stabilizing_sequence]{stabilizes} --- there exists some index \( m \) such that \( U_m = U_k \) for any \( k > m \). Denote \( U_m \) via \( U \).

    Then \( A \reloset {*} \Rightarrow \varepsilon \) if and only if \( A \in U \).

    \thmitem{alg:epsilon_rule_removal/rules} Now our goal is to define the rules of \( G' \) based on the rules of \( G \), but with zero or more \enquote{nullable} nonterminals removed from the right side of any rule.

    Let \( S' \) be an entirely new start symbol and let \( V' \coloneqq V \cup \set{ S' } \). Define the production relation \( A \to' w \) to hold if \( w \) is \hi{nonempty} and if there exist strings \( w_0, \ldots, w_n \) over \( V \cup \Sigma \) and nonterminals \( B_1, \ldots, B_n \) from \( U \) such that
    \begin{equation*}
      A \to w_0 B_1 w_1 B_2 w_2 \cdots B_n w_n.
    \end{equation*}
    and
    \begin{equation*}
      w = w_0 w_1 w_2 \cdots w_n.
    \end{equation*}

    \thmitem{alg:epsilon_rule_removal/finish} Define \( S' \to S \). If \( S \reloset {*} \Rightarrow \varepsilon \), also add the rule \( S' \to \varepsilon \). Then \( G' = (V', \Sigma, \to', S') \) is the desired grammar.
  \end{thmenum}
\end{algorithm}

\begin{example}\label{ex:alg:epsilon_rule_removal}
  We list several examples demonstrating the operation of \fullref{alg:epsilon_rule_removal}:
  \begin{thmenum}
    \thmitem{ex:alg:epsilon_rule_removal/an} We discussed the grammar
    \begin{equation*}
      S \to aS \mid \varepsilon
    \end{equation*}
    in \cref{ex:def:chomsky_hierarchy/an}.

    Using \fullref{alg:epsilon_rule_removal}, we conclude that the only nonterminal \( S \) belongs to \( U_0 \). Thus, \fullref{alg:epsilon_rule_removal/rules} suggests instead the rules
    \begin{equation*}
      S \to' aS \mid a
    \end{equation*}
    and \fullref{alg:epsilon_rule_removal/finish} suggests
    \begin{equation*}
      S' \to' S \mid \varepsilon,
    \end{equation*}
    where \( S' \) is the new starting nonterminal.

    The obtained grammar is essentially \( \varepsilon \)-free.

    \thmitem{ex:alg:epsilon_rule_removal/natural} Given the rules
    \begin{equation*}
      \begin{aligned}
        N &\to \syn0 \mid \syn1 B, \\
        B &\to \varepsilon \mid \syn0 B \mid \syn1 B
      \end{aligned}
    \end{equation*}
    and starting nonterminal \( N \), the algorithm suggests instead
    \begin{equation*}
      \begin{aligned}
        S' &\to' N, \\
        N  &\to' \syn0 \mid \syn1 \mid \syn1 B, \\
        B  &\to' \syn0 \mid \syn0 B \mid \syn1 \mid \syn1 B.
      \end{aligned}
    \end{equation*}

    The only member of \( U \) is \( B \), and we add a new rule of every instance of \( B \) where it is removed.

    This motivated our choice for rules in \cref{ex:natural_number_arithmetic_grammar/schema}. The obtained grammar is \( \varepsilon \)-free, not merely essentially \( \varepsilon \)-free.

    \thmitem{ex:alg:epsilon_rule_removal/dead} It is possible to obtain \enquote{dead} rules via \fullref{alg:epsilon_rule_removal}. For example, for
    \begin{equation*}
      \begin{aligned}
        S &\to A B, \\
        A &\to \varepsilon \mid a, \\
        B &\to \varepsilon
      \end{aligned}
    \end{equation*}
    with starting nonterminal \( S \), the algorithm produces
    \begin{equation*}
      \begin{aligned}
        S' &\to S \mid \varepsilon, \\
        S &\to A B \mid A \mid B, \\
        A &\to a. \\
      \end{aligned}
    \end{equation*}

    The derivations
    \begin{equation*}
      S' \Rightarrow S \Rightarrow B
    \end{equation*}
    and
    \begin{equation*}
      S' \Rightarrow S \Rightarrow A B \Rightarrow a B
    \end{equation*}
    cannot be expanded further in order to reach a string consisting entirely of terminals.

    Thus, the rules \( S \to B \) and \( S \to A B \) are essentially useless, but do no harm unless the number of rules matters as in \fullref{alg:brute_force_parsing}.
  \end{thmenum}
\end{example}

\begin{proposition}\label{thm:context_free_languages_are_context_sensitive}
  \hyperref[def:chomsky_hierarchy/context_free]{Context-free} \hi{languages} are \hyperref[def:chomsky_hierarchy/context_sensitive]{context-sensitive}.
\end{proposition}
\begin{proof}
  \Fullref{alg:epsilon_rule_removal} shows that every context-free grammar can be converted to another context-free grammar that is also context-sensitive.
\end{proof}

\begin{definition}\label{def:renaming_rule}\mimprovised
  A \term{renaming rule} in a formal grammar is a production rule \( A \to B \), where both \( A \) and \( B \) are nonterminals.
\end{definition}

\begin{algorithm}[Renaming rule collapse]\label{alg:renaming_rule_collapse}
  Fix an \hyperref[def:epsilon_free_grammar]{essentially \( \varepsilon \)-free} \hyperref[def:chomsky_hierarchy/context_free]{context-free} grammar \( G = (V, \Sigma, \to, S) \). We will build an \hyperref[def:formal_grammar/equivalent]{equivalent} context-free grammar \( G' = (V, \Sigma, \to', S) \) without \hyperref[def:renaming_rule]{renaming rules}.

  \begin{thmenum}
    \thmitem{alg:renaming_rule_collapse/rules} For every rule \( A \to w \), we define a \hyperref[def:function]{set-valued map} \( C(A \to w, U) \) whose values are strings \( v \) over \( V \cup \Sigma \) for which \( A \to v \) is a non-renaming rule. The parameter \( U \) is a set of nonterminals \enquote{already traversed} during recursion, we use it to avoid unbounded recursion like in the case of \( A \to B \to A \). Let
    \begin{equation*}
      C(A \to w, U) \coloneqq \begin{cases}
        \varnothing,                                    &A \in U, \\
        \bigcup_{B \to v} C(B \to v, U \cup \set{ A }), &w = B, \\
        \set{ w },                                      &\T{otherwise.}
      \end{cases}
    \end{equation*}

    \thmitem{alg:renaming_rule_collapse/new_grammar} Let \( A \to' v \) hold if and only if \( v \in C(A \to w, \varnothing) \) for some rule \( A \to w \). The obtained quadruple \( G' = (V', \Sigma, \to', S) \) is the desired grammar without renaming rules.
  \end{thmenum}
\end{algorithm}
\begin{comments}
  \item This algorithm can be found as \identifier{grammars.renaming_rules.collapse_renaming_rules} in \cite{notebook:code}.
  \item Our goal is to \enquote{collapse} derivations like \( A \to B \to C \to w \) into \( A \to' w \).
\end{comments}

\begin{remark}\label{con:backus_normal_form}\mcite[5]{Backus1959Syntax}
  Practice requires introducing a more convenient metasyntax (a syntax for describing language syntax).

  For \hyperref[def:chomsky_hierarchy/context_free]{context-free grammars}, we can use the \term{Backus normal form} (BNF; used by \incite[61]{Salomaa1973FormalLanguages}), sometimes also called \term{Backus-Naur form} (also BNF; used by \incite[40]{AhoEtAl2006Compilers}). Backus himself in \cite[5]{Backus1959Syntax} described it via examples.

  For \cref{ex:natural_number_arithmetic_grammar/schema}, one possible BNF is
  \begin{bnf*}
    \bnfprod{expression}     {\bnfpn{number} \bnfor \bnftsq{(} \bnfsp \bnfpn{expression} \bnfsp \bnfpn{operation} \bnfsp \bnfpn{expression} \bnfsp \bnftsq{)}} \\
    \bnfprod{operation}      {\bnftsq{\( \syntimes \)} \bnfor \bnftsq{\( \syntimes \)}} \\
    \bnfprod{natural number} {\bnftsq{\( \syn0 \)} \bnfor \bnftsq{\( \syn1 \)} \bnfor \bnftsq{\( \syn1 \)} \bnfsp \bnfpn{digit string}} \\
    \bnfprod{digit string}   {\bnftsq{\( \syn0 \)} \bnfor \bnftsq{\( \syn0 \)} \bnfsp \bnfpn{digit string} \bnfor \bnftsq{\( \syn1 \)} \bnfor \bnftsq{\( \syn1 \)} \bnfsp \bnfpn{digit string}}
  \end{bnf*}

  Compared to \eqref{eq:ex:natural_number_arithmetic_grammar/schema/simple}, some differences are:
  \begin{itemize}
    \item Variables are denoted by \( \bnfpn{strings enclosed in angle brackets} \), so that we can name variables more descriptively using more than one symbol.
    \item Terminals are put in \enquote{quotes}.
    \item The symbol \( \Coloneqq \) is used instead of \( \to \) for specifying transition rules.
    \item Different rules with the same source are combined as in \eqref{eq:ex:natural_number_arithmetic_grammar/schema/shorthand}.
    \item In order to fully describe a context-free grammar, we must only specify its Backus-Naur form and its starting variable.
  \end{itemize}
\end{remark}
\begin{comments}
  \item We will use Backus normal forms in more complicated grammars like \cref{def:propositional_syntax} and \cref{def:first_order_syntax/grammar_schema}, while preferring more the primitive notation \eqref{eq:ex:natural_number_arithmetic_grammar/schema/simple} for generic examples relating to grammars.
  \item Note that we have placed the (only) \hyperref[con:abstract_syntax_tree/syntactic]{syntactic rule} on top.
\end{comments}

\begin{remark}\label{rem:decimal_notation_grammar}
  Our series of examples ending with \cref{con:backus_normal_form} describes natural number arithmetic in binary notation. We made the restriction because decimal notation requires more complicated rules. One such set of rules is
  \begin{bnf*}
    \bnfprod{nonzero digit}  {\bnftsq{\( \syn1 \)} \bnfor \bnftsq{\( \syn2 \)} \bnfor \cdots \bnfor \bnftsq{\( \syn9 \)}} \\
    \bnfprod{digit}          {\bnftsq{\( \syn0 \)} \bnfor \bnfpn{nonzero digit}} \\
    \bnfprod{digit string}   {\bnfpn{digit} \bnfor \bnfpn{digit} \bnfsp \bnfpn{digit string}} \\
    \bnfprod{natural number} {\bnfpn{digit} \bnfor \bnfpn{nonzero digit} \bnfsp \bnfpn{digit string}}
  \end{bnf*}
\end{remark}

  \section{Regular languages}\label{sec:regular_languages}

\paragraph{Finite automata}

\begin{definition}\label{def:finite_automaton}\mcite[27]{Salomaa1973FormalLanguages}
  Fix an \hyperref[def:formal_language]{alphabet} \( \Sigma \). Let \( Q \) be a finite nonempty set, whose members we will call \term{states}. Let \( \delta: Q \times \Sigma \multto Q \) be a \hyperref[def:set_valued_map]{set-valued map}, which we will call a \term{transition function} because, depending on a state in \( Q \) and a symbol in \( \Sigma \), \( \delta \) gives us the possible states towards we can transition.

  Finally, let \( S \) and \( T \) be nonempty sets of states, which we call an \term{initial} and \term{terminal states}, correspondingly.

  We call this entire contraption \( (\Sigma, Q, \delta, I, T) \) a \term[ru=конечный автомат (\cite[159]{Гладкий1973ГрамматикиИЯзыки})]{finite automaton}. It models a real-world device that starts its work from some initial state and, via a sequence of state transitions, reaches some terminal state.

  \begin{figure}[!ht]
    \hfill
    \includegraphics[page=1]{output/def__finite_automaton}
    \hfill
    \includegraphics[page=2]{output/def__finite_automaton}
    \hfill\hfill
    \caption{A \hyperref[def:finite_automaton/determinism]{nondeterministic finite automaton} and its \hyperref[alg:determinization_of_finite_automata]{determinization}, both accepting the language \( \set{ a } \cup \set{ b^n \given n > 0 } \cup \set{ aa b^n \given n > 0 } \).}
    \label{fig:def:finite_automaton}
  \end{figure}

  \begin{thmenum}
    \thmitem{def:finite_automaton/graph}\mimprovised Regard \( \delta \) as a set of triples \( (h_0, l_0, t_0) \). Denote by \( h: \delta \to Q \), \( l: \delta \to \Sigma \) and \( t: \delta \to Q \) the functions that take the corresponding entry out of each triple.

    Then the quadruple \( (Q, \delta, h, t) \) is a \hyperref[def:directed_multigraph]{directed multigraph}, whose arcs are \hyperref[def:labeled_set]{labeled} by \( l \).

    We can identify the automaton with its (multi)graph. When drawing this graph, for example in \cref{fig:def:finite_automaton}, we denote initial states via inward arrows without a source and terminal states via double circles.

    \thmitem{def:finite_automaton/determinism} If there is only one initial state and if \( \delta \) is a \hyperref[def:set_valued_map/partial]{single-valued partial function}, we say that the automata is \term{deterministic}.

    Determinism ensures that there is at most one possible state to transition to, given a current state and a symbol.

    \thmitem{def:finite_automaton/recognition}\mcite[def. 4.1.1]{Savage2008ModelsOfComputation} We say that the automaton \term{accepts} or \term{recognizes} the \hyperref[def:formal_language/string]{string} \( a_1 \cdots a_n \) over \( \mscrA \) if there exists a \hyperref[def:graph_walk/directed]{walk}
    \begin{equation*}
      s \reloset {e_1} \to \anon \reloset {e_2} \to \cdots \reloset {e_{n-1}} \to \anon \reloset {e_n} \to \anon.
    \end{equation*}
    such that
    \begin{itemize}
      \item The label \( l(e_k) \) is \( a_k \) for every \( k = 1, \ldots, n \).
      \item The tail \( t(e_n) \) of the walk is a terminal state from \( T \).
    \end{itemize}

    \thmitem{def:finite_automaton/language} The set of all strings recognized by the automaton is called the language \term{accepted} or \term[ru=(язык) распознается (автоматом) (\cite[45]{Гладкий1973ГрамматикиИЯзыки})]{recognized} by the automaton. We denote this language via \( \mscrL(F) \).

    \thmitem{def:finite_automaton/equivalent}\mcite[152]{Savage2008ModelsOfComputation} We say that two finite automata are \term{equivalent} if they recognize the same language.
  \end{thmenum}
\end{definition}

\begin{definition}\label{def:reverse_language}\mimprovised
  We define the \term{reverse language} of \( \mscrL \) as
  \begin{equation*}
    \op{rev}(\mscrL) \coloneqq \set{ \op{rev}(w) \given w \in \mscrL }.
  \end{equation*}
\end{definition}

\begin{proposition}\label{thm:reverse_language_involution}
  The \hyperref[def:reverse_language]{reverse} of the reverse of a language is the original language.
\end{proposition}
\begin{proof}
  Trivial.
\end{proof}

\begin{definition}\label{def:reverse_finite_automaton}\mimprovised
  We define the \term{reverse automaton} of \( F = (\Sigma, Q, \delta, I, T) \) as
  \begin{equation*}
    \op{rev}(F) \coloneqq (\Sigma, Q, \op{rev}(\delta), T, I),
  \end{equation*}
  where
  \begin{equation*}
    \op{rev}(\delta) \coloneqq \set{ (q, a, p) \given (p, a, q) \in \delta }.
  \end{equation*}
\end{definition}
\begin{comments}
  \item We have
  \begin{equation*}
    p \in \op{rev}(\delta)(q, a) \T{if and only if} q \in \delta(p, a).
  \end{equation*}
\end{comments}

\begin{proposition}\label{thm:reverse_finite_automaton_graph}
  The \hyperref[def:finite_automaton/graph]{multigraph} of a \hyperref[def:reverse_finite_automaton]{reverse automaton} is its \hyperref[def:opposite_directed_multigraph]{opposite multigraph}.
\end{proposition}
\begin{proof}
  Trivial.
\end{proof}

\begin{proposition}\label{thm:reverse_finite_automaton_language}
  For a given \hyperref[def:finite_automaton]{finite automaton} \( F = (\Sigma, Q, \delta, I, T) \), we have
  \begin{equation*}
    \mscrL(\op{rev}(F)) = \op{rev}(\mscrL(F)).
  \end{equation*}
\end{proposition}
\begin{proof}
  Trivial.
\end{proof}

\begin{example}\label{ex:def:finite_automaton}
  We list several examples of \hyperref[def:finite_automaton]{finite automata}:
  \begin{thmenum}
    \thmitem{ex:def:finite_automaton/even} Consider the language form \fullref{ex:def:formal_language/leucine} describing even binary numbers. The language can be recognized by the automaton with multigraph
    \begin{equation*}
      \includegraphics[page=2]{output/ex__def__finite_automaton}
    \end{equation*}

    \thmitem{ex:def:finite_automaton/leucine} Consider the language form \fullref{ex:def:formal_language/leucine} describing Leucine. It can be recognized by the nondeterministic finite automaton
    \begin{equation*}
      \includegraphics[page=1]{output/ex__def__finite_automaton}
    \end{equation*}

    \thmitem{ex:def:finite_automaton/anbn} Consider the language
    \begin{equation*}
      \mscrL \coloneqq \set{ a^n b^n \given n \geq 0 }
    \end{equation*}
    from \fullref{ex:def:formal_language/anbn}. We will shown in \fullref{ex:thm:regular_pumping_lemma/anbn} that no finite automaton recognizes it, but we will give a direct proof here.

    As we will see in \fullref{alg:determinization_of_finite_automata}, a deterministic automaton exists accepting a language if and only if a nondeterministic one exists. Aiming at a contradiction, suppose that there exists some deterministic finite automaton \( F = (\Sigma, Q, \delta, \set{ s }, T) \) whose language is \( \mscrL \). Let \( G \) be its multigraph.

    Since \( s \) is the only initial state, and since \( \mscrL \) contains the empty string, then \( s \) must also be a terminal state.

    Furthermore, \( \mscrL \) contains the string \( ab \), hence there must exist a terminal state \( t_1 \) and an intermediate state \( q_1 \) such that
    \begin{equation*}
      \includegraphics[page=3]{output/ex__def__finite_automaton}
    \end{equation*}

    Furthermore, the above is a \hyperref[def:induced_subgraph]{induced subgraph} of \( G \) --- none of the states above are interconnected by any additional arcs.

    Then, in order for \( F \) to accept \( aabb \), it must have more states. Since the automaton is deterministic, there cannot be another arc with label \enquote{\( a \)} starting at \( s \). Hence, \( q_1 \) is the only node where it is possible to have another arc with label \enquote{\( a \)}.

    Hence, \( G \) either has as a subgraph either
    \begin{equation*}
      \includegraphics[page=4]{output/ex__def__finite_automaton}
    \end{equation*}
    or
    \begin{equation*}
      \includegraphics[page=5]{output/ex__def__finite_automaton}
    \end{equation*}

    In particular, \( F \) must have at least five states in order to recognize \( a^2 b^2 \).

    Continuing by induction, we conclude that in order for \( F \) to recognize \( a^n b^n \), it must have at least \( 2n + 1 \) states. For example, the following automaton recognizes \( a^n b^n \):
    \begin{equation*}
      \includegraphics[page=6]{output/ex__def__finite_automaton}
    \end{equation*}

    But \( \mscrL \) contains strings of arbitrary length. Therefore, no finite automaton is able to recognize \( \mscrL \).
  \end{thmenum}
\end{example}

\begin{algorithm}[Determinization of finite automata]\label{alg:determinization_of_finite_automata}
  Let \( N = (\Sigma, Q, \delta, I, T) \) be a \hyperref[def:finite_automaton]{finite automaton}. We will build an \hyperref[def:finite_automaton/equivalent]{equivalent} \hyperref[def:finite_automaton/determinism]{deterministic automaton} \( \det(N) \).

  This can be achieved by grouping states that would otherwise make the automaton nondeterministic. We will first recursively construct the operator \( \mscrD(P, V) \), which, given a set of states \( P \) and a family of \enquote{visited} sets of states \( V \), produces a family of triples describing the transitions of the deterministic automaton. The family \( V \) helps us avoid cycles when traversing \( N \).

  \begin{thmenum}
    \thmitem{alg:determinization_of_finite_automata/step} Suppose we are given a subset \( P \) of \( P \) and a subset \( V \) of \( \pow(Q) \).

    For every symbol \( a \) in \( \Sigma \), consider the set of all states that we can transition to via \( a \) from some state in \( P \):
    \begin{equation*}
      \delta(P, a) = \bigcup_{p \in P} \delta(p, a) = \set{ q \in Q \given \qexists* {p \in P} q \in \delta(p, a) }.
    \end{equation*}

    Define the set of triples that would become part the graph of the new transition function:
    \begin{equation*}
      E_P \coloneqq \set{ (P, a, \delta(P, a)) \given \delta(P, a) \neq \varnothing }.
    \end{equation*}

    Finally, define
    \begin{equation*}
      \mscrD(P, V) \coloneqq E_P \cup \bigcup \set[\Big]{ \mscrD\parens[\Big]{ \delta(P, a), V \cup \set{ P } } \given* \delta(P, a) \T{is nonempty and is not in} V }.
    \end{equation*}

    Note how we used \( V \) to filter out only those sets of states that have not yet been visited.

    \thmitem{alg:determinization_of_finite_automata/run} Let \( \delta' \coloneqq \mscrD(I, \varnothing) \). Define the new set of states
    \begin{equation*}
      Q' \coloneqq \set[\Big]{ P \given* (P, a, O) \in \delta' } \cup \set[\Big]{ O \given* (P, a, O) \in \delta' }.
    \end{equation*}

    Define also the set of initial states \( I' \coloneqq \set{ I } \) and of terminal states
    \begin{equation*}
      T' \coloneqq \set{ P \in Q' \given P \cap T \neq \varnothing }.
    \end{equation*}

    Then \( \det(N) \coloneqq (\Sigma, Q', \delta', I', T') \) is the desired finite automaton.
  \end{thmenum}
\end{algorithm}
\begin{comments}
  \item This algorithm can be found as \texttt{automata.finite.determinize} in \cite{notebook:code}.
\end{comments}
\begin{defproof}
  We have explicitly made \( I \) the only initial state, and we have grouped arcs with identical labels. Hence, \( F \) is indeed deterministic.

  We must show that \( \mscrL(N) = \mscrL(F) \). We have introduced a special state for every occurrence of several vertices that have incoming arcs with identical labels. Hence, we replace every such group of arcs with a single arc. The two automata should then accept identical languages.
\end{defproof}

\paragraph{Finite automata and regular grammars}

\begin{definition}\label{def:reverse_grammar}\mcite[17]{Salomaa1973FormalLanguages}
  We define the \term{reverse grammar} of the \hyperref[def:chomsky_hierarchy/context_free]{context-free} \( G = (V, \Sigma, \to, S) \) as
  \begin{equation*}
    \op{rev}(G) \coloneqq (V, \Sigma, \to_{\op{rev}}, S),
  \end{equation*}
  where
  \begin{equation*}
    A \to_{\op{rev}} \op{rev}(w) \T{if} A \to w.
  \end{equation*}
\end{definition}
\begin{comments}
  \item For example, the rule \( A \to Ba_1 \cdots a_n \) becomes \( A \to_{\op{rev}} a_n \cdots a_1 B \).
\end{comments}

\begin{proposition}\label{thm:reverse_linear_grammar}
  The \hyperref[def:reverse_grammar]{reverse} of a \hyperref[def:chomsky_hierarchy/regular]{left linear grammar} is \hyperref[def:chomsky_hierarchy/regular]{right linear} and vice versa.
\end{proposition}
\begin{proof}
  Trivial.
\end{proof}

\begin{proposition}\label{thm:reverse_grammar_language}
  For a given \hyperref[def:chomsky_hierarchy/context_free]{context-free} grammar \( G = (V, \Sigma, \to, S) \), we have
  \begin{equation*}
    \mscrL(\op{rev}(G)) = \op{rev}(\mscrL(G)).
  \end{equation*}
\end{proposition}
\begin{proof}
  Trivial.
\end{proof}

\begin{algorithm}[Regular grammar to finite automaton]\label{alg:regular_grammar_to_automaton}
  Let \( G = (V, \Sigma, \to, S) \) be a \hyperref[def:chomsky_hierarchy/regular]{regular grammar}. We will construct a \hyperref[def:finite_automaton]{finite automaton} that \hyperref[def:finite_automaton/language]{accepts} \( \mscrL(G) \).

  \begin{thmenum}
    \thmitem{alg:regular_grammar_to_automaton/init} Let \( G_1 = (V, \Sigma, \to_1, S) \) be \( G_1 \) if it is right linear and \( \op{rev}(G) \) if it is left linear. Then \( G_1 \) is necessarily right linear.

    \thmitem{alg:regular_grammar_to_automaton/epsilon} Let \( G_2 = (V_2, \Sigma, \to_2, S_2) \) be the grammar obtained from \( G_1 \) by removing \( \bnfes \) rules via \Fullref{alg:epsilon_rule_removal}.

    \thmitem{alg:regular_grammar_to_automaton/collapse} Let \( G_3 = (V_2, \Sigma, \to_3, S_2) \) be the grammar obtained from \( G_2 \) by collapsing renaming rules via \fullref{alg:renaming_rule_collapse}.

    \thmitem{alg:regular_grammar_to_automaton/intermediate} Build another intermediate grammar \( G_4 = (V_4, \Sigma, \to_4, S_2) \) as follows:
    \begin{itemize}
      \item Add each \( \bnfes \) rule as-is. There should be at most one \( \bnfes \) rule after \fullref{alg:regular_grammar_to_automaton/epsilon}.
      \item For each rule \( A \to_3 w \), \( w = a_1 \cdots a_n \), consider the sequence of rules
      \begin{align*}
        A       &\to_4 a_1 A_1, \\
        A_1     &\to_4 a_2 A_2, \\
                &\vdots \\
        A_{n-1} &\to_4 a_n,
      \end{align*}
      where \( A_1, \ldots, A_{n-1} \) are new nonterminals.

      \item For each rule \( A \to_3 wB \) with \( \len(w) > 0 \), consider a similar sequence, but the last rule being
      \begin{equation*}
        A_{n-1} \to_4 a_n B.
      \end{equation*}
    \end{itemize}

    Thus, every rule in \( G_4 \) has one of the forms \( A \to_4 a \) or \( A \to_4 a B \) or \( A \to_4 B \).

    \thmitem{alg:regular_grammar_to_automaton/automaton}\mcite[thm. 4.10.1]{Savage2008ModelsOfComputation} Let \( F \) be some new nonterminal symbol. Then the following is a finite automaton that accepts \( \mscrL(G_4) \), and hence also \( G_2 \) and \( G_1 \):
    \begin{itemize}
      \item \( \Sigma \) is the alphabet.
      \item \( V_4 \cup \set{ F } \) is the set of states.
      \item \( S_2 \) the only starting state.
      \item \( F \) is a final state. \( S_2 \) is also a final state if \( \bnfes \in \mscrL(G) \).
      \item Add the following transitions:
      \begin{itemize}
        \item \( \delta(A, a) \coloneqq F \) if \( A \to_4 a \).
        \item \( \delta(A, a) \coloneqq B \) if \( A \to_4 aB \).
      \end{itemize}
    \end{itemize}

    \thmitem{alg:regular_grammar_to_automaton/reverse} If \( G \) is right linear, then \( F \) is the desired automaton because
    \begin{equation*}
      \mscrL(F) = \mscrL(G) = \mscrL(G_4).
    \end{equation*}

    Otherwise, we take the \hyperref[def:reverse_finite_automaton]{reverse automaton} \( \op{rev}(F) \) because
    \begin{equation*}
      \mscrL(\op{rev}(F))
      \reloset {\ref{thm:reverse_finite_automaton_language}} =
      \op{rev}(\mscrL(F))
      =
      \op{rev}(\mscrL(G_4))
      =
      \op{rev}(\mscrL(\op{rev}(G)))
      \reloset {\ref{thm:reverse_grammar_language}} =
      \op{rev}(\op{rev}(\mscrL(G)))
      \reloset {\ref{thm:reverse_language_involution}} =
      \mscrL(G).
    \end{equation*}
  \end{thmenum}
\end{algorithm}
\begin{comments}
  \item This algorithm can be found as \identifier{grammars.regular.to_finite_automaton} in \cite{notebook:code}.
\end{comments}

\begin{algorithm}[Finite automaton to right-linear grammar]\label{alg:finite_automaton_to_right_linear_grammar}\mcite[thm. 4.10.1]{Savage2008ModelsOfComputation}
  Let \( F = (\Sigma, Q, \delta, I, T) \) be a \hyperref[def:finite_automaton]{finite automaton}. We will build a \hyperref[def:chomsky_hierarchy/regular]{right linear grammar} \( G = (V, \Sigma, \to, S) \) that \hyperref[def:formal_grammar/language]{generates} \( \mscrL(F) \).

  \begin{thmenum}
    \thmitem{alg:finite_automaton_to_right_linear_grammar/determinize} Use \fullref{alg:determinization_of_finite_automata} to obtain a deterministic automaton \( \det(F) = (\Sigma, Q', \delta', \set{ I }, T') \) equivalent to \( F \).

    \thmitem{alg:finite_automaton_to_right_linear_grammar/grammar} The following describes the desired grammar:
    \begin{itemize}
      \item \( \Sigma \) is the set of terminals.
      \item \( Q' \) is the set of nonterminals.
      \item \( I \) is the starting nonterminal.
      \item The following are rules:
      \begin{itemize}
        \item \( A \to aB \) if \( \delta'(A, a) = B \).
        \item \( A \to \bnfes \) for each terminal state \( A \in T' \).
      \end{itemize}
    \end{itemize}
  \end{thmenum}
\end{algorithm}
\begin{comments}
  \item This algorithm can be found as \identifier{grammars.regular.from_finite_automaton} in \cite{notebook:code}.
\end{comments}

\paragraph{Regular language characterization}

\begin{proposition}\label{thm:regular_languages}
  The following are equivalent for a given \hyperref[def:formal_language/language]{language}:
  \begin{thmenum}
    \thmitem{thm:regular_languages/right} It is \hyperref[def:formal_grammar/language]{generated} by a \hyperref[def:chomsky_hierarchy/regular]{right linear grammar}.
    \thmitem{thm:regular_languages/left} It is \hyperref[def:formal_grammar/language]{generated} by a \hyperref[def:chomsky_hierarchy/regular]{left linear grammar}.
    \thmitem{thm:regular_languages/nfa} It is \hyperref[def:finite_automaton/language]{recognized} by a (possibly nondeterministic) \hyperref[def:finite_automaton]{finite automaton}.
    \thmitem{thm:regular_languages/dfa} It is \hyperref[def:finite_automaton/language]{recognized} by a \hyperref[def:finite_automaton/determinism]{deterministic finite automaton}.
  \end{thmenum}
\end{proposition}
\begin{proof}
  \ImplicationSubProof{thm:regular_languages/right}{thm:regular_languages/left} Let \( G = (V, \Sigma, \to, S) \) be a right-regular grammar. We will describe a procedure for obtaining an equivalent left-regular grammar.

  \begin{itemize}
    \item \Fullref{alg:regular_grammar_to_automaton} gives us a finite automaton \( F = (\Sigma, Q, \delta, \set{ S }, T) \) such that \( \mscrL(F) = \mscrL(G) \).

    \item Take the \hyperref[def:reverse_finite_automaton]{reverse automaton} \( \op{rev}(F) \) of \( F \).

    \item Determinize \( \op{rev}(F) \) via \fullref{alg:determinization_of_finite_automata}.

    \item Use \fullref{alg:finite_automaton_to_right_linear_grammar} to convert \( \det(\op{rev}(F)) \) to a right-regular grammar \( G' \). At this point, we have
    \begin{equation*}
      \mscrL(G')
      =
      \mscrL(\det(\op{rev}(F)))
      =
      \mscrL(\op{rev}(F))
      \reloset {\ref{thm:reverse_finite_automaton_language}} =
      \op{rev}(\mscrL(F)).
    \end{equation*}

    \item Finally, take the \hyperref[def:reverse_grammar]{reverse grammar} \( \op{rev}(G') \). It is a left-regular grammar as a consequence of \fullref{thm:reverse_linear_grammar}. Furthermore,
    \begin{equation*}
      \mscrL(\op{rev}(G'))
      \reloset {\ref{thm:reverse_grammar_language}} =
      \op{rev}(\mscrL(G'))
      =
      \op{rev}(\op{rev}(\mscrL(F)))
      \reloset {\ref{thm:reverse_language_involution}} =
      \mscrL(F).
    \end{equation*}

    Hence, \( \op{rev}(G') \) is the desired left linear grammar.
  \end{itemize}

  \ImplicationSubProof{thm:regular_languages/left}{thm:regular_languages/nfa} \Fullref{alg:regular_grammar_to_automaton} allows us to convert a left linear grammar to a finite automaton.

  \ImplicationSubProof{thm:regular_languages/nfa}{thm:regular_languages/dfa} \Fullref{alg:determinization_of_finite_automata} allows us to convert a general finite automata to a deterministic one.

  \ImplicationSubProof{thm:regular_languages/dfa}{thm:regular_languages/right} \Fullref{alg:finite_automaton_to_right_linear_grammar} allows us to convert a finite automaton to a right linear grammar.
\end{proof}

\begin{lemma}[Pumping lemma for regular languages]\label{thm:regular_pumping_lemma}\mcite[lemma 4.5.1]{Savage2008ModelsOfComputation}
  For every \hyperref[def:chomsky_hierarchy/regular]{regular language} \( \mscrL \) there exists a constant \( p \) such that any string \( w \) in \( \mscrL \) with \( \len(w) \geq p \) can be decomposed as \( w = x y z \), where \( y \) is nonempty, \( \len(xy) \leq p \) and, for any \hi{nonnegative} integer \( n \), the string \( x y^n z \) belongs to \( \mscrL \).
\end{lemma}
\begin{comments}
  \item In simpler terms, in any regular language some part of a sufficiently long string can be repeated indefinitely.
  \item If the language \( \mscrL \) is finite, the lemma is vacuous because we may simply take \( p \) to be longer than the longest string of \( \mscrL \).
\end{comments}
\begin{proof}
  Fix a deterministic finite automaton \( F = (\Sigma, Q, \delta, \set{ S }, T) \) recognizing \( \mscrL \) and denote by \( p \) the number of states in \( Q \).

  Let \( w \) be a string in \( \mscrL \) of length at least \( p \), and let
  \begin{equation*}
    S = q_0 \reloset {a_1} \to q_1 \reloset {a_2} \to \cdots \reloset {a_n} \to q_n
  \end{equation*}
  be a \hyperref[def:graph_walk/directed]{walk} through the graph of \( F \) witnessing \( w \), that is,
  \begin{equation*}
    w = a_1 \cdots a_n.
  \end{equation*}

  \Fullref{thm:pigeonhole_principle/simple} implies that, since \( n \geq p \), at least one state is visited twice. Let \( i \) be the smallest index such that \( q_i \) is repeated, and let \( j > i \) be the smallest index such that \( q_j = q_i \). Then the states \( q_0, \ldots, q_{j-1} \) are distinct by construction, implying \( j - 1 \leq p \).

  Let
  \begin{equation*}
    \underbrace{a_1 \cdots a_{i-1}}_x \underbrace{a_i \cdots a_j}_y \underbrace{a_{j+1} \cdots a_n}_z.
  \end{equation*}

  Since \( i < j \), the string \( y \) is nonempty. Furthermore, \( \len(xy) = j - 1 \leq p \). To complete the proof of the lemma, we must show that, for any nonnegative integer \( n \), the string \( xy^nz \) is in \( \mscrL \).

  \begin{equation*}
    \begin{aligned}
      \includegraphics[page=1]{output/thm__regular_pumping_lemma}
    \end{aligned}
  \end{equation*}

  The \hyperref[def:graph_walk]{walk}
  \begin{equation*}
    q_i \reloset {a_{i+1}} \to q_{i+1} \reloset {a_{i+2}} \to \cdots \reloset {a_j} \to q_j
  \end{equation*}
  is then \hyperref[def:graph_walk/closed]{closed}, and so it can be traversed as many times as desired (including zero).

  Therefore, for any nonnegative integer \( n \), the string \( xy^nz \) is recognized by \( F \), and thus it is in \( \mscrL \).
\end{proof}

\begin{example}\label{ex:thm:regular_pumping_lemma}
  We list some examples related to \fullref{thm:regular_pumping_lemma}:
  \begin{thmenum}
    \thmitem{ex:thm:regular_pumping_lemma/anbn} Consider the language
    \begin{equation*}
      \mscrL \coloneqq \set{ a^n b^n \given n \geq 0 }
    \end{equation*}
    from \fullref{ex:def:formal_language/anbn}. We have shown in \fullref{ex:def:finite_automaton/anbn} that no finite automaton recognizes it, and thus \fullref{thm:regular_languages} implies that it is not regular. We will give another proof here, based on \fullref{thm:regular_pumping_lemma}.

    Fix any positive integer \( p \) and let \( w = a^n b^n \) be a string of length at least \( p \). Let \( w = xyz \) be an arbitrary decomposition such that \( y \) is nonempty and \( \len(xy) \leq p \).

    \begin{itemize}
      \item If \( y = a^m \) for some positive integer \( m \), then \( x y^2 z = a^{n + m} b^n \), which is not in \( \mscrL \).
      \item If instead \( y = b^m \), then \( x y^2 z = a^n b^{n + m} \), which is again not in \( \mscrL \).
      \item If instead \( y = a^k b^m \), then \( x y^2 z = a^{n-k} a^k b^m a^k b^m b^{n - m} \), which is again not in \( \mscrL \).
    \end{itemize}

    Therefore, the conclusion of \fullref{thm:regular_pumping_lemma} does not hold, and thus \( \mscrL \) is not regular.

    \thmitem{ex:thm:regular_pumping_lemma/even} Consider the language
    \begin{equation*}
      \mscrL \coloneqq \set[\Big]{ w \syn0 \given w \in \set{ \syn0, \syn1 }^* }.
    \end{equation*}
    of even numbers in binary notation from \fullref{ex:def:formal_language/even}. This language is obviously regular. We will use it to validate \fullref{thm:regular_pumping_lemma}.

    For any integer \( p > 2 \) and any string \( w = a_1 \cdots a_n \) of length at least \( p \), we can take \( x = \bnfes \), \( y = a_1 \cdots a_{p-1} \) and \( z = a_p \cdots a_n \). Since \( p \leq n \), \( z \) always contains a trailing \( \syn0 \), and thus, for any positive integer \( k \), the string \( xy^kz \) belongs to \( \mscrL \).

    \thmitem{ex:thm:regular_pumping_lemma/balanced_parentheses} \hyperref[def:chomsky_hierarchy/regular]{Regular grammars} cannot express languages with arbitrarily nested \hyperref[def:paired_delimiters]{balanced delimiters} such as the language of propositional logic described in \fullref{def:propositional_syntax/language} or the language of untyped lambda calculus described in \fullref{def:lambda_term}.

    This will be shown in \fullref{thm:paired_delimiters_not_regular}.
  \end{thmenum}
\end{example}

  \subsection{Syntax trees}\label{subsec:syntax_trees}

\paragraph{Parse trees}

\begin{definition}\label{def:parse_tree}\mcite[\S 2.2.3]{AhoEtAl2006Compilers}
  A \term[ru=дерево вывода (\cite[81]{Гладкий1973Языки})]{parse tree} or \term{concrete syntax tree} for the \hyperref[def:chomsky_hierarchy/context_free]{context-free grammar} \( G = (V, \Sigma, \to, S) \) is a \hyperref[def:labeled_tree]{labeled tree} \( T \) with labels from \( V \cup \Sigma \) such that:
  \begin{thmenum}
    \thmitem{def:parse_tree/root} The root has label \( S \).
    \thmitem{def:parse_tree/leaves} If a leaf is labeled by a nonterminal \( N \), there exists a rule \( N \to \bnfes \).
    \thmitem{def:parse_tree/children} There is a rule
    \begin{equation*}
      x \to y_1 \cdots y_n,
    \end{equation*}
    where \( x \) is the label of the root of \( T \) and \( y_1, \ldots, y_n \) --- of the immediate children of \( T \).
  \end{thmenum}
\end{definition}
\begin{comments}
  \item We associate with each parse tree a string as shown in \fullref{def:parse_tree_string}.
  \item Although we generally follow the definition from \cite[\S 2.2.3]{AhoEtAl2006Compilers}, we disallow parse trees for strings containing nonterminals, and thus we do not need to specially handle \( \bnfes \)-labeled nodes.
\end{comments}

\begin{remark}\label{rem:parse_tree_roots}
  In syntactic definitions and proofs, for example in relation to \hyperref[con:evaluation]{evaluation}, we usually consider \hyperref[def:formal_grammar/schema]{grammar schemas} and each subtree has a root with a different label. In this case, we work with parse trees for different grammars over the same grammar schema (differing by the start symbol). We also allow singleton trees consisting of a single terminal or \( \bnfes \), acknowledging that these are not technically parse trees.
\end{remark}

\begin{definition}\label{def:parse_tree_string}\mcite[\S 2.2.3]{AhoEtAl2006Compilers}
  We say that the \hyperref[def:parse_tree]{parse tree} \( T \) \term{yields} the string \( x_1 \cdots x_n \) if, for \( k = 1, \ldots, n \), the \( k \)-th member in the \hyperref[def:ordered_tree_enumeration]{pre-order enumeration} of the terminal-labeled nodes of \( T \) has label \( x_k \).
\end{definition}
\begin{comments}
  \item We implicitly associate this string with the parse tree.
\end{comments}

\begin{example}\label{ex:def:parse_tree}
  We give several examples of \hyperref[def:parse_tree]{parse trees}.

  \begin{thmenum}
    \thmitem{ex:def:parse_tree/an} Consider the grammar
    \begin{equation*}
      \begin{aligned}
        S &\to A \mid \bnfes, \\
        A &\to aA \mid a.
      \end{aligned}
    \end{equation*}
    from \fullref{ex:def:chomsky_hierarchy/an}.

    It describes the language \( \mscrL = \set{ a^n \given n \geq 0 } \) discussed in \fullref{ex:def:formal_language/an}. The only possible parse tree for the string \( aaa \) is
    \begin{equation*}\label{eq:ex:def:parse_tree/an}
      \begin{aligned}
        \includegraphics[page=1]{output/ex__def__parse_tree}
      \end{aligned}
    \end{equation*}

    If we instead use the \enquote{simpler} grammar
    \begin{equation*}
      S \to aS \mid \bnfes,
    \end{equation*}
    then we would have an \( \bnfes \)-labeled node in the parse tree:
    \begin{equation*}
      \includegraphics[page=2]{output/ex__def__parse_tree}
    \end{equation*}

    \thmitem{ex:def:parse_tree/anbn} Consider the grammar
    \begin{equation*}
      \begin{aligned}
        S &\to A \mid \bnfes, \\
        A &\to aAb \mid ab
      \end{aligned}
    \end{equation*}
    from \fullref{ex:def:chomsky_hierarchy/anbn} describing \( \mscrL = \set{ a^n b^n \given n \geq 0 } \) from \fullref{ex:def:formal_language/anbn}.

    The only possible parse tree for the string \( aaabbb \) is
    \begin{equation*}
      \includegraphics[page=3]{output/ex__def__parse_tree}
    \end{equation*}

    \thmitem{ex:def:parse_tree/arithmetic} We continue the binary natural number grammar example from \fullref{ex:natural_number_arithmetic_grammar/schema} with the grammar \eqref{eq:ex:natural_number_arithmetic_grammar/schema/shorthand}.

    The only possible parse tree for the expression \( (\syn1 \syn0 \syntimes (\syn1 \synplus \syn1 \syn0)) \) is
    \begin{equation*}
      \includegraphics[page=4]{output/ex__def__parse_tree}
    \end{equation*}

    We will discuss abstract syntax trees in \fullref{con:abstract_syntax_tree}, which are notably tidier.

    We can \hyperref[con:evaluation]{evaluate} this parse tree by substituting every binary number string (such as \( \syn1 \syn0 \)) with its numeric value in \( \BbbN \) and then recursively applying the corresponding expressions. Evaluating the above parse tree gives the same result, the decimal number \( 6 \), as evaluating \( \syn1 \syn 1 \syn0 \). More formal details on this evaluation are discussed in \fullref{ex:natural_number_arithmetic_grammar/evaluation}.

    Now consider instead the same grammar without parentheses, that is, replace the expression expansion rule
    \begin{equation*}
      E \to (E O E)
    \end{equation*}
    with
    \begin{equation*}
      E \to E O E.
    \end{equation*}

    The corresponding expression \( \syn1 \syn0 \synplus \syn1 \syntimes \syn1 \syn0 \) has more than one parse tree:
    \begin{equation*}
      \begin{aligned}
        \includegraphics[page=5]{output/ex__def__parse_tree}
        \qquad\qquad
        \includegraphics[page=6]{output/ex__def__parse_tree}
      \end{aligned}
    \end{equation*}

    The first tree corresponds to \( (10 \syntimes (1 \synplus 10)) \), which evaluates to \( 6 \), while the second one corresponds to \( ((\syn1 \syn0 \syntimes \syn1) \synplus 10) \), which evaluates to \( 4 \).\fnote{\( 10 \) in binary notation is \( 2 \) is decimal notation, \( 100 \) in binary is \( 4 \) in decimal and \( 110 \) in binary is \( 6 \) in decimal.}

    Thus, removing parentheses makes the grammar ambiguous in the sense of \fullref{def:grammar_ambiguity}. We will show in \fullref{ex:natural_number_arithmetic_grammar/unambiguous} that the parenthesized grammar is unambiguous.
  \end{thmenum}
\end{example}

\begin{algorithm}[Brute force parsing]\label{alg:brute_force_parsing}
  Fix a \hyperref[def:chomsky_hierarchy/context_free]{context-free grammar} \( G = (V, \Sigma, \to, S) \). For each nonterminal \( A \), we will introduce a \hyperref[def:function]{set-valued map} \( P_A(w) \), whose values are parse trees of \( w \).

  To avoid unbounded recursion, we will also define the auxiliary set-valued map \( P'_A(w, U) \), where \( U \) is a set of nonterminal-string pairs \( (B, u) \) \enquote{already traversed} during the recursion. This will allow us to define
  \begin{equation*}
    P_A(w)
    \coloneqq
    P'_A(w, \varnothing)
  \end{equation*}

  Suppose that we are given the rule \( A \to v_1 \cdots v_m \), where each \( v_k \) is a single symbol (either a terminal or nonterminal). The algorithm relies on partitioning \( w \) into \( m \) parts. Denote by \( S(w, m) \) the set of all partitions of \( w \) into \( m \) parts, that is, all \( m \)-tuples \( \vect w = (w_1, \ldots, w_m) \) such that
  \begin{equation*}
    w = \underbrace{ r_1 \cdots r_{l_1} }_{w_1} \underbrace{ r_{l_1 + 1} \cdots r_{l_1 + l_2} }_{w_2} \cdots \underbrace{ r_{l_1 + \cdots + l_{m-1} + 1} \cdots r_{l_1 + \cdots + l_m} }_{w_m},
  \end{equation*}
  where \( l_k \coloneqq \len(w_k) \).

  We will introduce a second auxiliary set-valued map
  \begin{equation*}
    P^\dprime_A(A \to v_1 \cdots v_m, \vect w, U),
  \end{equation*}
  which gives the parse trees of \( w \) corresponding to a concrete rule and partition. This will allow us to define
  \begin{equation*}
    P'_A(w, U)
    \coloneqq
    \bigcup\set[\Big]{ P^\dprime_A(A \to v_1 \cdots v_m, \vect w, U) \given A \to v_1 \cdots v_m \T{and} \vect w \in S(w, m) }.
  \end{equation*}

  \begin{thmenum}
    \thmitem{alg:brute_force_parsing/tree_set} For each \( v_k \) in the rule \( A \to v_1 \cdots v_m \), construct a set \( \mscrT_k \) of parse trees as follows:
    \begin{itemize}
      \item If \( v_k \) is a terminal, let \( \mscrT_k \) be a set with one element --- \hyperref[def:canonical_singleton_tree]{canonical singleton tree} with label \( w_k \).

      \item If \( v_k \) is a nonterminal, let
      \begin{equation*}
        \mscrT_k \coloneqq \begin{cases}
          \varnothing,                                         &(A, w_k) \in U, \\
          P'_{v_k}\parens[\Big]{ w_k, U \cup \set{ (A, w) } }, &\T{otherwise.}
        \end{cases}
      \end{equation*}
    \end{itemize}

    \thmitem{alg:brute_force_parsing/combine} Combine the above sets \( \mscrT_1, \cdots, \mscrT_m \) as follows:
    \begin{itemize}
      \item If \( m = n = 0 \), i.e. if \( A \to \bnfes \) and \( w = \bnfes \), let
      \begin{equation*}
        P^\dprime_A\parens[\Big]{ A \to v_1 \cdots v_m, \vect w, U }
      \end{equation*}
      consist of the \hyperref[def:canonical_singleton_tree]{canonical singleton tree} with label \( A \).

      \item Otherwise, let
      \begin{equation*}
        P^\dprime_A\parens[\Big]{ A \to v_1 \cdots v_m, \vect w, U }
      \end{equation*}
      consist of all \hyperref[def:ordered_tree_grafting_product]{grafted trees} of the form
      \begin{equation*}
        \begin{aligned}
          \includegraphics[page=1]{output/alg__brute_force_parsing}
        \end{aligned}
      \end{equation*}
      where \( (T_1, \ldots, T_m) \) is a tuple in the Cartesian product \( \mscrT_1 \times \cdots \times \mscrT_m \).
    \end{itemize}
  \end{thmenum}
\end{algorithm}
\begin{comments}
  \item This is a variation of Stephen Unger's algorithm described in \cite{Unger1968Parser}.
  \item This algorithm can be found as \identifier{grammars.brute_force_parse.parse} in \cite{notebook:code}.
  \item We can avoid the auxiliary parameter \( U \) if the grammar has no \hyperref[def:epsilon_free_grammar]{\( \bnfes \) rules} and no \hyperref[def:renaming_rule]{renaming rules}, but otherwise we can easily get stuck in a loop like \( A \Rightarrow B \Rightarrow A \) (if \( A \to B \) and \( B \to A \)) or \( A \Rightarrow AB \Rightarrow A \) (if \( B \to \bnfes \)).
\end{comments}

\paragraph{Leftmost and rightmost derivations}

\begin{definition}\label{def:leftmost_derivation}\mcite[53]{Salomaa1973Languages}
  In a context-free grammar, we say that the \hyperref[def:formal_grammar/derivation]{derivation}
  \begin{equation*}
    S \Rightarrow w_1 \Rightarrow \cdots \Rightarrow w_m = w
  \end{equation*}
  is \term{leftmost} if, given any step \( w_{k-1} = p_k A_k s_k \Rightarrow p_k v_k s_k = w_k \), the prefix \( p_k \) contains only terminals.

  We define \term{rightmost} by instead requiring that the suffix \( s_k \) contains only terminals.
\end{definition}

\begin{example}\label{ex:natural_number_arithmetic_grammar/derivation}
  We will consider derivations in the binary natural number grammar \eqref{eq:ex:natural_number_arithmetic_grammar/schema/shorthand} from \fullref{ex:natural_number_arithmetic_grammar/schema}.

  We have shown in \fullref{ex:def:parse_tree/arithmetic} that removing parentheses results in multiple parse trees yielding the same string.

  Similarly, using parentheses results in a single leftmost derivation:
  \begin{equation*}
    \begin{aligned}
      \includegraphics[page=1]{output/ex__natural_number_arithmetic_grammar__derivation}
    \end{aligned}
  \end{equation*}

  Removing parentheses allows for multiple leftmost derivations:
  \begin{equation*}
    \begin{aligned}
      \includegraphics[page=2]{output/ex__natural_number_arithmetic_grammar__derivation}
    \end{aligned}
  \end{equation*}
\end{example}

\begin{proposition}\label{thm:leftmost_derivation_existence}
  Every string in a context-free grammar's generated language has at least one \hyperref[def:leftmost_derivation]{leftmost derivation}.
\end{proposition}
\begin{proof}
  A leftmost derivation can be achieved via reordering of the derivation steps.
\end{proof}

\begin{algorithm}[Derivation to parse tree]\label{alg:derivation_to_parse_tree}
  Fix a context-free grammar \( G = (V, \Sigma, \to, S) \) and a derivation
  \begin{equation*}
    S = w_0 \Rightarrow w_1 \Rightarrow \cdots \Rightarrow w_m = w
  \end{equation*}
  of the terminal-only string \( w = r_1 \cdots r_n \).

  We will construct a parse tree recursively. At step \( k \), where \( k = 0, \ldots, m \), the leaves of the tree \( T_k \), enumerated by \hyperref[def:ordered_tree_enumeration]{pre-ordering}, should be the symbols of the string \( w_k \). Proceed as follows:
  \begin{thmenum}
    \thmitem{alg:derivation_to_parse_tree/initial} Let \( T_0 \) be the singleton tree with label \( S \).

    \thmitem{alg:derivation_to_parse_tree/step} At step \( k > 0 \), having already built \( T_{k-1} \) for \( w_{k-1} \), let \( v_0 \) be the first (with respect to pre-order enumeration) node labeled by \( A \) in \( T_{k-1} \).

    Then the tree \( T_{k-1} \) has the form
    \begin{equation*}
      \includegraphics[page=1]{output/alg__parse_tree_from_derivation}
    \end{equation*}
    where \( w_{k-1} = r_1 \cdots r_l \) are the symbols of \( w_{k-1} \), and \( j_0 \) is the index of the first instance of \( A \) in \( w_{k-1} \).

    \begin{itemize}
      \item If \( l = 0 \), i.e. if \( w_i = \bnfes \), define \( T_k \coloneqq T_{k-1} \).

      \item Otherwise, there exists a rule \( A \to s_1 \cdots s_p \) producing \( w_i \) from \( w_{k-1} \), and we instead define \( T_k \) by \hyperref[def:ordered_tree_grafting]{grafting} \( s_1, \ldots, s_p \) onto \( v_0 \):
      \begin{equation*}
        \begin{aligned}
          \includegraphics[page=2]{output/alg__parse_tree_from_derivation}
        \end{aligned}
      \end{equation*}
    \end{itemize}
  \end{thmenum}

  \thmitem{alg:derivation_to_parse_tree/finish} The tree \( T_m \) is the desired parse tree for \( w \).
\end{algorithm}
\begin{comments}
  \item This algorithm can be found as \identifier{grammars.parse_tree.parse_tree_to_derivation} in \cite{notebook:code}.
\end{comments}

\begin{algorithm}[Parse tree to leftmost derivation]\label{alg:parse_tree_to_leftmost_derivation}
  Let \( T \) be a parse tree for \( w = r_1 \cdots r_n \). We will explicitly construct a \hyperref[def:leftmost_derivation]{leftmost derivation} for \( w \).

  We proceed as follows:
  \begin{thmenum}
    \thmitem{alg:parse_tree_to_leftmost_derivation/initial} Let \( v_0, v_1, \ldots, v_m \) be the \hyperref[def:ordered_tree_enumeration]{pre-order enumeration} of the nodes of \( T \) labeled by nonterminals. Denote by \( l_k \) the label of \( v_k \).

    Define \( w_0 \coloneqq l_0 \). By assumption, the root label \( l_0 \) of \( T \) is \( S \).

    \thmitem{alg:parse_tree_to_leftmost_derivation/step} At step \( 0 < k \leq m \), the nonterminal \( l_{k-1} \) must occur in \( w_{k-1} \).

    Let \( s_1, \ldots, s_p \) be the labels of the children of \( v_k \). \Fullref{def:parse_tree/children} ensures that there exists a production rule
    \begin{equation*}
      l_{k-1} \to s_1 \cdots s_p.
    \end{equation*}

    Given \( w_{k-1} = a l_{k-1} b \), where \( a \) and \( b \) are strings and \( a \) does not contain nonterminals, we define
    \begin{equation*}
      w_k \coloneqq a s_1 \cdots s_p b.
    \end{equation*}

    \thmitem{alg:parse_tree_to_leftmost_derivation/finish} At step \( k = m + 1 \), there are no more terminals in \( w_m \). Furthermore, the symbols of \( w_m \) are by construction a pre-order enumeration of \( T \), thus \( w_m = w \).

    Therefore, we have the derivation
    \begin{equation*}
      S = w_0 \Rightarrow w_1 \Rightarrow \cdots \Rightarrow w_m = w.
    \end{equation*}
  \end{thmenum}
\end{algorithm}
\begin{comments}
  \item This algorithm can be found as \identifier{grammars.parse_tree.parse_tree_to_derivation} in \cite{notebook:code}.
\end{comments}

\begin{proposition}\label{thm:derivations_and_parse_trees}
  In a \hyperref[def:chomsky_hierarchy/context_free]{context-free grammar}, there is a bijective correspondence between \hyperref[def:parse_tree]{parse trees} and \hyperref[def:leftmost_derivation]{leftmost derivations} of terminal-only strings.
\end{proposition}
\begin{proof}
  \Fullref{alg:derivation_to_parse_tree} allows us to construct a parse tree from any derivation. Multiple derivations can lead to the same parse tree, but our choice of nonterminal in \fullref{alg:derivation_to_parse_tree/step} ensures that a leftmost derivation leads to a unique parse tree.

  Conversely, \fullref{alg:parse_tree_to_leftmost_derivation} allows us to construct a leftmost derivation from any parse tree. Uniqueness is ensured by choosing at \fullref{alg:parse_tree_to_leftmost_derivation/step} the smallest possible index.
\end{proof}

\begin{corollary}\label{thm:parse_tree_existence}
  A string over a context-free grammar's terminal symbols is derivable if and only if there exists a parse tree for it.
\end{corollary}
\begin{proof}
  Follows from \fullref{thm:derivations_and_parse_trees} and \fullref{thm:leftmost_derivation_existence}.
\end{proof}

\paragraph{Grammar ambiguity}

\begin{definition}\label{def:grammar_ambiguity}\mcite[54]{Salomaa1973Languages}
  We say that a string generated by a \hyperref[def:chomsky_hierarchy/context_free]{context-free grammar} is \term{ambiguous} if any of the following equivalent conditions hold:
  \begin{thmenum}
    \thmitem{def:grammar_ambiguity/tree} It has multiple \hyperref[def:parse_tree]{parse trees}.
    \thmitem{def:grammar_ambiguity/leftmost} It has multiple \hyperref[def:leftmost_derivation]{leftmost derivations}.
    \thmitem{def:grammar_ambiguity/rightmost} It has multiple \hyperref[def:leftmost_derivation]{right derivations}.
  \end{thmenum}

  We say that the grammar itself is \term{ambiguous} if it generates at least one ambiguous string, and \term{unambiguous} otherwise.
\end{definition}
\begin{comments}
  \item \Fullref{thm:derivations_and_parse_trees} ensures that every generated string has at least one parse tree.
  \item \Fullref{thm:leftmost_derivation_existence} ensures that every generated string has at least one leftmost derivation.
\end{comments}
\begin{proof}
  \EquivalenceSubProof{def:grammar_ambiguity/tree}{def:grammar_ambiguity/leftmost} Follows from \fullref{thm:derivations_and_parse_trees}.
  \EquivalenceSubProof{def:grammar_ambiguity/leftmost}{def:grammar_ambiguity/rightmost} Trivial.
\end{proof}

\begin{example}\label{ex:natural_number_arithmetic_grammar/unambiguous}
  We will show that the binary natural number grammar \eqref{eq:ex:natural_number_arithmetic_grammar/schema/shorthand} from \fullref{ex:natural_number_arithmetic_grammar/schema} is \hyperref[ex:natural_number_arithmetic_grammar/unambiguous]{unambiguous}.

  Let \( w \) be a string in \( \mscrL(G) \). We will build a derivation tree for \( w \) via \hyperref[rem:natural_number_recursion]{natural number recursion} on \( \len(w) \):

  \begin{itemize}
    \item If \( \len(w) = 1 \), then \( w \) is either \enquote{\( \syn0 \)} or \enquote{\( \syn1 \)}, and the possible parse trees are
    \begin{equation*}
      \begin{aligned}
        \includegraphics[page=1]{output/ex__natural_number_arithmetic_grammar__unambiguous}
        \qquad\qquad
        \includegraphics[page=2]{output/ex__natural_number_arithmetic_grammar__unambiguous}
      \end{aligned}
    \end{equation*}

    \item Let \( w = r_1 \cdots r_n \) be a string of length \( n \) and suppose that every string shorter than \( w \) is unambiguous.

    \begin{itemize}
      \item If \( r_1 \) is \enquote{\( ( \)}, then \( w \) is an expression. Hence, \( r_n \) is \enquote{\( ) \)} and there exists some index \( k \) such that \( r_k \) is either \enquote{\( \synplus \)} or \enquote{\( \syntimes \)}. The inductive hypothesis applies to \( r_2 \cdots r_{k-1} \) and  \( r_{k\synplus1} \cdots r_{n-1} \), and gives us parse trees \( T_1 \) and \( T_2 \) whose roots are labeled with \( E \). Then the following is the unique parse tree of \( w \):
      \begin{equation*}
        \begin{aligned}
          \includegraphics[page=3]{output/ex__natural_number_arithmetic_grammar__unambiguous}
        \end{aligned}
      \end{equation*}

      \item Otherwise, \( w \) is a binary numerical string. Given a parse tree of \( r_1 \cdots r_{n-2} \), since the grammar with start symbol \( N \) is right linear, the parse tree for \( w \) requires simply adding the dotted edges
      \begin{equation*}
        \begin{aligned}
          \includegraphics[page=4]{output/ex__natural_number_arithmetic_grammar__unambiguous}
        \end{aligned}
      \end{equation*}
    \end{itemize}
  \end{itemize}
\end{example}

\begin{proposition}\label{thm:regular_grammars_are_unambiguous}
  All \hyperref[def:chomsky_hierarchy/regular]{regular grammars} are \hyperref[def:grammar_ambiguity]{unambiguous}.
\end{proposition}
\begin{proof}
  The right side of any rule contains at most one nonterminal, so there is only one possible derivation for every string.
\end{proof}

\paragraph{Evaluation}

\begin{concept}\label{con:evaluation}
  We have briefly discussed \enquote{evaluation} in \fullref{con:syntax_semantics_duality}. We will now describe it in more detail.

  Fix a grammar \( G = (V, \Sigma, \to, S) \) with nonterminals \( A_1, \cdots, A_n \). Then an example evaluation function has the form
  \begin{equation*}
    E(w) \coloneqq \begin{cases}
      \cdots, &w \T{is generated from} S, \\
      \cdots, &w \T{is generated from} A_1, \\
              &\vdots,
    \end{cases}
  \end{equation*}

  \term{Pattern matching} is a term used for partial definition of \( E \) based on the structure of \( w \). Pattern matching depends on knowing the possible parse trees for \( w \). We use parse trees implicitly without explicitly mentioning it. \Fullref{ex:natural_number_arithmetic_grammar/evaluation} provides justification for not using parse trees directly.

  We use the convention from \fullref{rem:parse_tree_roots} regarding parse tree for different starting nodes over the same grammar schema. Of course, more granular rules are possible based on the structure of strings --- see \fullref{ex:natural_number_arithmetic_grammar/evaluation} for an example.
\end{concept}
\begin{comments}
  \item If the grammar is \hyperref[def:grammar_ambiguity]{ambiguous}, \( E \) may not be \hyperref[def:function]{single-valued} if we are not careful.
  \item We allow different starting nonterminals because, if \( F \) applies itself recursively, it must be able to process strings generated in different ways.
\end{comments}

\begin{example}\label{ex:natural_number_arithmetic_grammar/evaluation}
  We will define \hyperref[con:evaluation]{evaluation} for the binary natural number grammar \eqref{eq:ex:natural_number_arithmetic_grammar/schema/shorthand} from \fullref{ex:natural_number_arithmetic_grammar/schema}. Let
  \begin{equation*}
    F(w) \coloneqq \begin{cases}
      0,                  &w = \syn0, \\
      1,                  &w = \syn1, \\
      2 \times F(w'),     &w = w' \syn0, \\
      2 \times F(w') + 1, &w = w' \syn1, \\
      F(u) + F(v),        &w = (u \synplus v), \\
      F(u) \times F(v),   &w = (u \syntimes v),
    \end{cases}
  \end{equation*}
  where we have utilized the dot convention from \fullref{rem:object_language_dots/terminals} to unambiguously utilize metalingual numbers and operations.

  In this example, we do pattern matching on the string \( w \). This pattern matching depends on knowing the possible parse trees and knowing that the grammar is unambiguous. We use parse trees implicitly without recognizing it.

  Using parse trees directly would be much more tedious to describe --- for example, defining summation, we would need to state that trees of the form
  \begin{equation*}
    \begin{aligned}
      \includegraphics[page=1]{output/ex__natural_number_arithmetic_grammar__evaluation}
    \end{aligned}
  \end{equation*}
  evaluate as \( F(T_1) + F(T_2) \).
\end{example}

\paragraph{Abstract syntax trees}

\begin{concept}\label{con:abstract_syntax_tree}
  \hyperref[def:parse_tree]{Parse trees} focus on how strings are built from symbols. Take the following parse tree from \fullref{ex:def:parse_tree/arithmetic}:
  \begin{equation}\label{eq:con:abstract_syntax_tree/base}
    \begin{aligned}
      \includegraphics[page=1]{output/rem__abstract_syntax_tree}
    \end{aligned}
  \end{equation}

  Most of the information in this tree is not necessary for \hyperref[con:evaluation]{evaluating} it. An \term[en=abstract syntax tree (\cite[41]{AhoEtAl2006Compilers})]{abstract syntax tree} (AST) is instead an \hyperref[def:labeled_tree]{labeled tree} that contains information that is necessary for evaluation, but not for yielding the source string. The possible abstract syntax trees for a language differ depending on their use, as we will show, and are often produced from parse trees (which we called \enquote{concrete syntax trees} in \fullref{def:parse_tree}). The only restriction we put on abstract syntax trees is that they should be finite labeled trees in a bijective correspondence with parse trees.

  First, we can vastly simplify the tree by splitting the grammar \eqref{eq:ex:natural_number_arithmetic_grammar/schema/shorthand} into two parts:
  \begin{thmenum}
    \thmitem{con:abstract_syntax_tree/lexical} The \term{lexical part}
    \begin{equation*}
      \begin{aligned}
        N &\to \syn0 \mid \syn1 \mid \syn1 B \\
        B &\to \syn0 \mid B \syn0 \mid \syn1 \mid B \syn1 \\
        O &\to \synplus \mid \syntimes
      \end{aligned}
    \end{equation*}

    Any strings produced via the lexical rules are called \term[ru=лексемы (\cite[329]{Гладкий1973Языки})]{lexemes}. The goal is to isolate atoms of a language from arbitrarily complicated recursive structures. This distinction is important for evaluation and becomes apparent in more complicated use cases like first-order formula substitution in \fullref{def:first_order_substitution}.

    We rarely need parse trees for lexical rules. The lexical grammar is often \hyperref[def:chomsky_hierarchy/regular]{regular}.

    \thmitem{con:abstract_syntax_tree/syntactic} The \term{syntactic part}
    \begin{equation*}
      E \to N \mid (E O E)
    \end{equation*}

    Syntactic rules regard lexemes as base symbols, although this is often not possible formally. In this example, \( N \) produces infinitely many strings, and by definition a grammar's terminals are only finitely many.

    Compared to lexical rules, syntactic rules benefit from parse trees because they highlight the recursive structure of a string.
  \end{thmenum}

  One simplification we can do to \eqref{eq:con:abstract_syntax_tree/base} is to collapse lexical rules and regard lexemes as strings:
  \begin{equation}\label{eq:con:abstract_syntax_tree/syntactic}
    \begin{aligned}
      \includegraphics[page=2]{output/rem__abstract_syntax_tree}
    \end{aligned}
  \end{equation}

  Another thing we can simplify is to remove auxiliary symbols like parentheses\fnote{Whitespace and comments in programming languages are also often useless in parse trees during evaluation, but they are useful for code refactoring tools, error reporting and metaprogramming.}. They are only necessary in order to the grammar to be unambiguous, but once we already have a parse tree, they become meaningless. Removing the parentheses in \eqref{eq:con:abstract_syntax_tree/syntactic} leads to
  \begin{equation}\label{eq:con:abstract_syntax_tree/no_parens}
    \begin{aligned}
      \includegraphics[page=3]{output/rem__abstract_syntax_tree}
    \end{aligned}
  \end{equation}

  We can also remove unnecessarily long chains of nonterminals from \eqref{eq:con:abstract_syntax_tree/no_parens} where that would not be ambiguous (i.e. collapse \( E \to E \to \syn1 \syn0 \) to \( E \to \syn1 \syn0 \)):
  \begin{equation}\label{eq:con:abstract_syntax_tree/collapsed}
    \begin{aligned}
      \includegraphics[page=4]{output/rem__abstract_syntax_tree}
    \end{aligned}
  \end{equation}

  We can now do a simplification based on our intended semantics. Every node in \eqref{eq:con:abstract_syntax_tree/collapsed} is labeled via either a number lexeme, operation symbol or \( E \). We know the operations are binary, and we can use the symbol of the operation instead of \( E \) as a label, removing the need to keep a separate node for the operation symbol. This leads to
  \begin{equation}\label{eq:con:abstract_syntax_tree/final}
    \begin{aligned}
      \includegraphics[page=5]{output/rem__abstract_syntax_tree}
    \end{aligned}
  \end{equation}

  This last tree has only 5 nodes, compared to the original tree \eqref{eq:con:abstract_syntax_tree/base} with 21 nodes. Yet, we can recover \eqref{eq:con:abstract_syntax_tree/base} from \eqref{eq:con:abstract_syntax_tree/final}.
\end{concept}

\begin{example}\label{ex:natural_number_arithmetic_grammar/induction}
  We will show how to use \fullref{thm:induction_on_rooted_trees} in the context of the binary natural number grammar \eqref{eq:ex:natural_number_arithmetic_grammar/schema/shorthand} from \fullref{ex:natural_number_arithmetic_grammar/schema}.

  We want to prove that an expression without ones \hyperref[con:evaluation]{evaluates} to zero.

  \begin{itemize}
    \item The simplest expressions we have are number \hyperref[con:abstract_syntax_tree/lexical]{lexemes}. The only one that does not contain ones is \( \syn0 \), which evaluates to zero.

    \item Consider the expression \( (u \synplus v) \).

    We implicitly assume that existence of a unique parse tree
    \begin{equation*}
      \includegraphics[page=1]{output/ex__natural_number_arithmetic_grammar__induction}
    \end{equation*}
    where \( T_u \) and \( T_v \) are the (unique) parse trees for \( u \) and \( v \).

    Having assumed that both \( u \) and \( v \) don't contain ones and thus evaluate to zero, we conclude that their sum \( (u \synplus v) \) should also evaluate to zero due to \ref{eq:def:peano_arithmetic/PA4}.

    It is easier for us to rely on \hyperref[con:evaluation]{pattern matching} than to explicitly construct the tree, and we have only done it for demonstrational purposes.

    \item Analogously, the expression \( (u \syntimes v) \) evaluates to zero due to \ref{eq:def:peano_arithmetic/PA6}.
  \end{itemize}
\end{example}


  \section{Mathematical logic}\label{sec:mathematical_logic}

Mathematical logic uses mathematics to study logic and vice versa.

We start with objects that are purely logical in nature --- formulas --- which are strings of symbols representing truth values. Formal definitions for formulas are given here using \hyperref[def:formal_grammar]{grammars}, which in turn depend on \hyperref[def:formal_language]{languages}. Formal definitions for truth values are given using \hyperref[def:heyting_algebra]{Heyting} and \hyperref[def:boolean_algebra]{Boolean algebras}

These concepts help us define the theory necessary to study following two important intertwined topics:
\begin{itemize}
  \item We are interested in establishing whether the formula \( \varphi \) logically entails the formula \( \psi \). This is done using \hyperref[def:deductive_system]{deductive systems} which specify precisely how we can manipulate strings of symbols. This aspect is called \term{syntactic} or \term{logical} and is the basis or \hyperref[def:proof_derivability]{proof theory}. Formulas allow us to express statements about mathematics and proof theory allows us to systematically study the relationships between them. In practice, we manipulate \hyperref[rem:abstract_syntax_tree]{syntax trees} rather than strings.

  \item Given a formula, we are interested in assigning a meaning to it. Different logical systems provide different syntax that is useful for different purposes - \hyperref[def:propositional_syntax/formula]{propositional formulas} allow us to express complex relationships between propositions via \hyperref[subsec:boolean_operators]{Boolean operators} while \hyperref[def:first_order_syntax/formula]{first-order logic formulas} allow us to go one level lower and give a precise meaning to these propositions via \hyperref[def:first_order_structure]{structures}. This aspect of logic is called \term{semantical} and is the basis of \hyperref[subsec:first_order_models]{model theory}. Model theory allows us to study logical formulas using pre-existing mathematics.
\end{itemize}

There are two aspects in which logical systems are categorized:
\begin{itemize}
  \item \hyperref[subsec:propositional_logic]{Propositional} and \hyperref[subsec:first_order_logic]{first-order logic} (among others) differ in what their syntax allows us to express. This also means that they differ in what their semantics can express, but, just as the syntax of first-order logic is a superset of the syntax of propositional logic, \hyperref[subsec:boolean_operators]{Boolean operators} can express relations between quantifierless atomic formulas in any \hyperref[def:first_order_structure]{structures}. In other words, semantics are identical in places where the syntax is the same.

  \item \hyperref[rem:classical_logic]{Classical} and \hyperref[rem:intuitionistic_logic]{intuitionistic logic} (among others) differ in their semantics and their logical \hyperref[def:judgment/inference_rule]{inference rules}. This has several implications:
  \begin{itemize}
     \item Boolean operators describe \hyperref[rem:classical_logic]{classical logic}, however they fail to describe \hyperref[rem:intuitionistic_logic]{intuitionistic logic} because double negation elimination \eqref{eq:thm:minimal_propositional_negation_laws/dne} no longer holds and neither do other similar statements. So, while retaining the same syntax, we must resort to much more complicated semantical frameworks like \hyperref[def:propositional_heyting_algebra_semantics]{Heyting} or \hyperref[def:propositional_topological_semantics]{topological semantics}.

     \item The proof theory that describes classical logic no longer matches the semantics, hence we must resort to other proof systems. This turns out not to be trivial because we need a clear understanding of which logical axioms imply the others. \Fullref{subsec:deductive_systems} lists different proof systems and their corresponding semantics.
  \end{itemize}
\end{itemize}

\begin{remark}\label{rem:metalogic}
  The statements of mathematical logic can themselves be studied logically. We distinguish between the \term{object logic} which we study and the \term{metalogic} which we use to study it. It is possible, for example, to study intuitionistic propositional logic using classical first-order logic. The metalogic is usually less formal and its statements are written in prose for the sake of easier understanding.

  It is an exercise in futility to try and completely formalize the language, syntax and theory of the metalogic --- or, as they are sometimes called, the \term{metalanguage}, \term{metasyntax} and \term{metatheory}. We must take a given metalogical framework for granted and then study a certain object logical framework. This is not to say that the principles and rules that hold in the metalogic are immaterial --- see for example the discussion of the differences between \hyperref[rem:intuitionistic_logic]{intuitionistic logic} and \hyperref[rem:classical_logic]{classical logic}. Still, it makes little sense to attempt to study the metalogic because at that point it becomes the object logic and the still more abstract conceptual framework in which we reason about the metalogic now becomes the new metalogic. We can thus form a hierarchy that is unbounded in both directions --- we can study a more concrete object logic within our object logic, and we can jump from one metalogical level to the next.

  An important connection between the logic and metalogic is given in \fullref{rem:set_definition_recursion}.
\end{remark}

\begin{remark}\label{rem:classical_logic}
  Classical logic is a term used to describe, among others:
  \begin{itemize}
    \item A semantical framework for propositional logic defined in \fullref{def:propositional_semantics}.
    \item A matching \hyperref[def:deductive_system]{deductive system}, defined in \fullref{def:classical_propositional_deductive_systems}.
    \item A semantical framework for \hyperref[subsec:first_order_logic]{first-order logic} defined in \fullref{def:first_order_semantics}.
    \item A matching deductive system, defined in \fullref{def:first_order_natural_deduction_system}.
  \end{itemize}

  Classical logic is characterized by the ability to use the law of double negation elimination \eqref{eq:thm:minimal_propositional_negation_laws/dne}. A more popular (but less accurate due to \fullref{thm:minimal_propositional_negation_laws}) characterization is that the law of the excluded middle \eqref{eq:thm:minimal_propositional_negation_laws/lem} holds.

  Within the metalogic, the law of the excluded middle is called the \term{principle of bivalence} and states that either a statement holds or its negation holds.
\end{remark}

\begin{remark}\label{rem:intuitionistic_logic}\mcite[35]{TroelstraSchwichtenberg2000}
  Intuitionistic logic is a generalization of \hyperref[rem:classical_logic]{classical logic}. It is also called \term{constructive logic} due to the \hyperref[rem:brouwer_heyting_kolmogorov_interpretation]{Brouwer-Heyting-Kolmogorov interpretation}. See \fullref{rem:brouwer_heyting_kolmogorov_interpretation_compatibility} for further discussion of the topic.

  Instead of the law of the excluded middle \eqref{eq:thm:minimal_propositional_negation_laws/lem}, we have the strictly weaker principle of explosion \eqref{eq:thm:minimal_propositional_negation_laws/efq} stating that everything can be proved given a contradiction.

  To these ideas there correspond \hyperref[def:propositional_heyting_algebra_semantics]{Heyting algebra semantics} and \hyperref[def:propositional_topological_semantics]{topological semantics} and a matching deductive system, \fullref{def:intuitionistic_propositional_deductive_systems}, for \hyperref[subsec:propositional_logic]{propositional logic}.
\end{remark}

\begin{remark}\label{rem:minimal_logic}\mcite[35]{TroelstraSchwichtenberg2000}
  Minimal logic is a further generalization of \hyperref[rem:intuitionistic_logic]{intuitionistic logic}.

  Instead of the law of the excluded middle \eqref{eq:thm:minimal_propositional_negation_laws/lem} or the strictly weaker principle of explosion \eqref{eq:thm:minimal_propositional_negation_laws/efq}, we have the even weaker law of non-contradiction \eqref{eq:thm:minimal_propositional_negation_laws/lnc}.

  Metalogically speaking, we can only conclude that there is no statement such that both the statement and its negation are true. If the statement instead does not hold, we cannot automatically conclude that its negation holds.

  \Fullref{def:minimal_propositional_hilbert_system} provides a deductive system for \hyperref[subsec:propositional_logic]{propositional logic}. We avoid studying semantics of minimal logic.
\end{remark}

\begin{remark}\label{rem:mathematical_logic_conventions}
  Several conventions related to logic are followed through the document.

  We only work within \hyperref[def:classical_propositional_deductive_systems]{classical metalogic}. Outside the section on logic, we use formulas and, more generally, use object logic only in dedicated places like \fullref{def:group/theory} describing the \hyperref[def:first_order_theory]{logical theory} of \hyperref[def:group]{groups}. Most axioms, like \ref{def:norm/N1}-\ref{def:norm/N3} for norms, are formulated entirely within the metalanguage under the assumption that we are working within a model of set theory --- \hyperref[def:axiom_of_universes]{\logic{ZFC+U}}, to be more precise. Using logical theories like \fullref{def:semiring/theory} allows us to implicitly obtain a plethora of definitions and theorems from \fullref{subsec:first_order_models} and \fullref{subsec:deductive_systems}. To keep a clear distinction between logical formulas and non-logical axioms and, more generally, to distinguishing between logic and metalogic, we use the following conventions:

  \begin{thmenum}
    \thmitem{rem:mathematical_logic_conventions/variable_symbols} Variables in the object language are denoted by the small Greek letters, usually \( \xi, \eta, \zeta \), while variables in the metalanguage are denoted by small Latin letters, usually \( x, y, z \). If needed, we add subscripts with indices.

    \thmitem{rem:mathematical_logic_conventions/formula_term_symbols} Formulas, which we only consider in the object language, are also denoted by small Greek letters --- \( \varphi, \psi, \theta, \chi \) --- and so are terms --- \( \tau, \sigma, \rho, \kappa, \mu, \nu \).

    \thmitem{rem:mathematical_logic_conventions/propositional_constants} The propositional constants denoting truth and falsity are denoted by \( \top \) and \( \bot \) in the object language and by \( T \) and \( F \) in the metalanguage. This is only for the sake of following an established convention, and we still use \( \top \) and \( \bot \) in general \hyperref[def:semilattice/lattice]{lattices}.

    \thmitem{rem:mathematical_logic_conventions/connective_symbols} We usually prefer prose to symbolic quantifiers and connectives in the metalanguage. The longer double arrows \( \implies \) and \( \iff \) are occasionally used within the metalogic outside this section.

    \thmitem{rem:mathematical_logic_conventions/structure_pairs} We conflate structures in the metalogic (i.e. sets with functions and/or relations defined on them) with their domain --- see \fullref{rem:first_order_model_notation} for a discussion.

    \thmitem{rem:mathematical_logic_conventions/shorthands} We additionally use syntactic shorthands like \fullref{rem:propositional_formula_parentheses} and \fullref{rem:first_order_formula_conventions} when writing formulas.

    \thmitem{rem:mathematical_logic_conventions/quantification} We avoid writing excessive universal quantification and instead rely on implicit quantification as described in \fullref{thm:implicit_universal_quantification}. If we need the formulas to be closed, such as in the case of \hyperref[def:first_order_theory]{first-order theories} for example, we assume all formulas are closed and if they are not, we add explicit universal quantifiers in front of them.
  \end{thmenum}
\end{remark}

\begin{remark}\label{rem:higher_order_logic}
  Since we describe first-order logic, it may be helpful to clarify why it is named so. It is merely a shorthand for \enquote{first-order predicate logic}. There are other predicate logical frameworks, namely second-order predicate logic and higher-order predicate logic, discussed in \cite[sec. 3.6]{Hinman2005}.

  Second-order logic allows us to quantify over relations between variables. In that case, we refer to the variables of first-order logic as \enquote{individuals} and to the relations as \enquote{relation variables}. This allows us, for example, to avoid axiom schemas like the \hyperref[def:zfc/specification]{axiom schema of specification} by instead replacing them with a single axiom that quantifies over unary relations. A downside of second-order logic is that it has worse properties --- it is incomplete in the sense that there exists no \hyperref[def:deductive_system]{deductive system} that is both sound and complete, and it is not compact in the sense that the analogue to \fullref{thm:first_order_compactness_theorem} does not hold. These defects are attributed to the expressive power of second-order logic --- while a first-order axiom schema may have only a countable number of axioms, a second-order quantifier may range over uncountably many relations.

  Clearly anything that extends second-order logic must suffer from the same problems, however higher-order logic is still useful because it allows us to utilize some very powerful concepts. Rather than quantifying over relations over the relations over individuals that would happen in third-order logic, we instead consider the more abstract frameworks of type theory. Type theory itself comes in many flavors, but simple type theory can be viewed as a generalization of first-order logic --- see \cite[thm. 2]{Farmer2008}. The rough idea is that rather than having individual variables, relation variables, etc., we have \term{base types} and \term{type constructors}. The individual variables have a dedicated base type, and the types of functions and predicates are easily constructed using the type constructors. For example, if \( i \) is the type of individuals, \( i \rightarrow i \) is the type of unary functions and \( i \rightarrow (i \rightarrow i) \) is the type of binary functions. The syntax of simple type theory is inspired by \( \lambda \)-calculus, which is a huge topic in itself and one of the frameworks for studying computability theory. The semantics of simple type theory are merely an extension of first-order semantics with different universes for different types. Like second-order logic, however, type theories have worse properties than those of first-order logic.

  Another benefit of type theories is that they allow for multiple base types. For example, in the definition of a \hyperref[def:vector_space]{vector space}, we have scalars and vectors, and we introduce an axiom schema parameterized by the scalars. In contrast, we could have a type for scalars and a type for vectors. This is also easily achievable in first-order logic via the so-called \enquote{many-sorted first-order logic}, where the types are called \enquote{sorts}. Both many-sorted first-order logic languages and simple type theory languages can be reformulated as first-order logic languages --- see \cite[ch. 8]{Farmer2008}.

  We circumvent the need for any of these higher-order logical frameworks by relying on set theory. This is further discussed in \fullref{rem:first_order_theories_in_zfc}.
\end{remark}

  \subsection{Boolean functions}\label{subsec:boolean_functions}

\paragraph{Boolean values and functions}

\begin{proposition}\label{thm:two_element_lattice}
  Any two-element \hyperref[def:lattice]{lattice} is \hyperref[def:lattice/homomorphism]{isomorphic} to the \hyperref[def:finite_field]{finite field} \( \BbbF_2 = \set{ 0, 1 } \).
\end{proposition}
\begin{comments}
  \item \Fullref{ex:def:boolean_algebra/f2} shows how \( \BbbF_2 \) can be regarded as a \hyperref[def:boolean_algebra]{Boolean algebra}.
\end{comments}
\begin{proof}
  There is only one possible lattice homomorphism from \( \set{ \top, \bot } \) to \( \BbbF_2 = \set{ 0, 1 } \). Furthermore, it is invertible, and its inverse is also a lattice homomorphism. Hence, it is a lattice isomorphism.
\end{proof}

\begin{concept}\label{con:boolean_value}\mimprovised
  There is a natural \hyperref[def:boolean_algebra]{Boolean algebra} structure on any two-element set \( \set{ T, F } \) in which one of the elements is larger than the other. We can regard \( T \) as a value denoting truth and \( F \) as denoting falsity, in which case \( F < T \). We call \( T \) and \( F \) in this context \term{Boolean values}.
\end{concept}
\begin{comments}
  \item See \fullref{rem:mathematical_logic_conventions/propositional_constants} for more discussions regarding related notation conventions.

  \item For certain purposes, for example \hyperref[def:zhegalkin_polynomial]{Zhegalkin polynomials}, it makes sense to identify \( F \) with \( 0 \) and \( T \) with \( 1 \) in the \hyperref[def:finite_field]{finite field} \( \BbbF_2 \) (this is technically the isomorphism of Boolean algebras from \fullref{thm:two_element_lattice}).
\end{comments}

\begin{definition}\label{def:boolean_function}\mcite[9, 120]{Яблонский2003ДискретнаяМатематика}
  We call functions from any set to \( \set{ T, F } \) (Boolean-valued) \term[ru=предикаты, en=predicates (\cite[15]{Savage1998Computability})]{predicates} and functions from \( \set{ T, F }^n \) to \( \set{ T, F } \) \( n \)-ary \term[ru=булевые функции, en=Boolean functions (\cite[847]{Rosen2019DiscreteMathematics})]{Boolean functions}.
\end{definition}

\begin{remark}\label{rem:boolean_valued_functions_and_predicates}
  \hyperref[def:boolean_function]{Boolean-valued functions} and \hyperref[def:relation]{relations} represent the same concept. In particular, if we fix some sets \( X_1, \ldots, X_n \), any relation \( R \subseteq X_1 \times \cdots \times X_n \) corresponds to a unique Boolean-valued function
  \begin{equation*}
    \begin{aligned}
      &f: X_1 \times \cdots \times X_n \to \set{ T, F } \\
      &f(x_1, \ldots, x_n) = \begin{cases}
        T, &(x_1, \ldots, x_n) \in R, \\
        F, &\T{otherwise}
      \end{cases}
    \end{aligned}
  \end{equation*}
  and vice versa.
\end{remark}

\begin{definition}\label{def:boolean_closure}\mcite[30; 33]{Яблонский2003ДискретнаяМатематика}
  Consider the set of all Boolean functions
  \begin{equation*}
    \mscrB \coloneqq \set[\Big]{ f \given* f \colon \set{ T, F }^n \to \set{ T, F } \T{for some nonnegative integer} n }.
  \end{equation*}

  \begin{thmenum}
    \thmitem{def:boolean_closure/closed} We say that a subset \( B \) of \( \mscrB \) is \term{closed} if, whenever \( g(x_1, \ldots, x_m) \) is in \( B \) and \( f_k(x_1, \ldots, x_n) \) is in \( B \) for \( k = 1, \ldots, m \), then their \hyperref[con:function_superposition]{superposition}
    \begin{equation*}
      h(x_1, \ldots, x_n) \coloneqq g(f_1(x_1, \ldots, x_n), \ldots, f_m(x_1, \ldots, x_n))
    \end{equation*}
    is also in \( B \).

    \thmitem{def:boolean_closure/closure} We define the \hyperref[def:moore_closure_operator]{Moore closure operator} \( \cl \) on \( \pow(\mscrB) \) by assigning to every set \( B \) the smallest closed set containing \( B \).

    \thmitem{def:boolean_closure/complete} If the closure \( \cl{B} \) is the entire set \( \mscrB \), we say that \( B \) is \term[ru=полная система, en=functionally complete (\cite[857]{Rosen2019DiscreteMathematics})]{complete}.
  \end{thmenum}
\end{definition}
\begin{comments}
  \item If \( B \) is complete, then from \fullref{thm:functions_over_model_form_model} it follows that \( B \) is a Boolean algebra. This is used in \fullref{thm:propositional_formulas_and_boolean_functions/bijection}.
\end{comments}
\begin{defproof}
  Let \( B \) be an arbitrary set of Boolean functions (i.e. an arbitrary subset of \( \mscrB \)). If a function belongs to \( \cl{B} \), it must belong to any closed set containing \( B \), and vice versa - if a function belongs to every closed superset of \( B \), it must belong to \( \cl{B} \). Hence, \( \cl{B} \) is the intersection of all closed superset of \( B \). At least one such superset exists - \( \mscrB \) itself - hence \( \cl{B} \) is well-defined.

  The conditions for closure operator from \fullref{def:moore_closure_operator} are trivial to verify.
\end{defproof}

\paragraph{Zhegalkin polynomials}

\begin{definition}\label{def:square_free_element}\mcite[79]{JedrzejewiczEtAl2017SquareFree}
  We say that an element \( x \) if a \hyperref[def:semiring]{semiring} is \term{square-free} if there exists no element \( y \) such that \( y^2 \) divides \( x \).
\end{definition}

\begin{definition}\label{def:zhegalkin_polynomial}\mcite[32]{Яблонский2003ДискретнаяМатематика}
  A \term[ru=полином Жегалкина]{Zhegalkin polynomial} is a \hyperref[def:polynomial_algebra/polynomials]{polynomial} in the \hyperref[def:finite_field]{finite field} \( \BbbF_2 \).
\end{definition}
\begin{comments}
  \item \hyperref[def:square_free_element]{Square-free} Zhegalkin polynomials are unique --- see \fullref{thm:zhegalkin_polynomial_uniqueness}.
\end{comments}

\begin{algorithm}\label{alg:infer_zhegalkin_polynomial}
  Given a \hyperref[def:boolean_function]{Boolean function} \( f(x_1, \ldots, x_n) \), we can recursively build a square-free \hyperref[def:zhegalkin_polynomial]{Zhegalkin polynomial} \( p(X_1, \ldots, X_n) \) that \hyperref[con:substitution_homomorphism]{evaluates} to \( f(x_1, \ldots, x_n) \).

  \begin{thmenum}
    \thmitem{alg:infer_zhegalkin_polynomial/base} If \( n = 0 \), then \( f \) is a constant; Let \( p \coloneqq f \).
    \thmitem{alg:infer_zhegalkin_polynomial/step} If \( n > 0 \), we can recursively apply the algorithm to obtain polynomials \( p_T(X_2, \ldots, X_n) \) and \( p_F(X_2, \ldots, X_n) \) corresponding to \( f(T, X_2, \ldots, X_n) \) and \( f(F, X_2, \ldots, X_n) \).

    We then define
    \begin{equation}\label{eq:alg:infer_zhegalkin_polynomial/step}
      p(X_1, \ldots, X_n) = X_1 \cdot \parens[\Big]{ p_T(X_2, \ldots, X_n) \oplus p_F(X_2, \ldots, X_n) } \oplus p_F(X_2, \ldots, X_n).
    \end{equation}
  \end{thmenum}
\end{algorithm}
\begin{comments}
  \item This algorithm can be found as \identifier{rings.zhegalkin.infer_zhegalkin} in \cite{notebook:code}.
\end{comments}
\begin{defproof}
  We will prove correctness by induction. The base case \( n = 0 \) is vacuous, so suppose that \( n > 0 \) and that the algorithm is correct for less than \( n \) variables.

  Fix a Boolean function \( f(X_1, \ldots, X_n) \) and let \( p_T \) and \( p_F \) be as in \fullref{alg:infer_zhegalkin_polynomial/step}.

  Both are elements of the ring \( \BbbF_2[X_2, \ldots, X_n] \). The polynomial \( p \) as defined by \eqref{eq:alg:infer_zhegalkin_polynomial/step} is a linear polynomial in \( X_1 \) over this ring.
  \begin{itemize}
    \item Evaluating \( p \) with \( X_1 \mapsto F \) yields the constant coefficient, which by definition is \( p_F \).
    \item Evaluating \( p \) with \( X_1 \mapsto T \) yields a value that is the sum of the linear and constant coefficient. Thus, the linear coefficient itself is \( p_T \ominus p_F = p_T \oplus p_F \).
  \end{itemize}

  By the inductive hypothesis, evaluating \( p \) over \( \BbbF_2 \) gives \( f \).
\end{defproof}

\begin{proposition}\label{thm:zhegalkin_polynomial_uniqueness}
  To every Boolean function in \( n \) variables there corresponds a unique square-free Zhegalkin polynomial in \( n \) indeterminates.
\end{proposition}
\begin{proof}
  Existence is provided by \fullref{alg:infer_zhegalkin_polynomial}, while uniqueness is provided by \fullref{thm:functions_over_prime_fields}.
\end{proof}

\begin{lemma}\label{thm:f2_sum_parity}
  In the \hyperref[def:finite_field]{finite field} \( \BbbF_2 \), the sum \( a_1 \oplus \cdots \oplus a_n \) is \( T \) if and only if there is an odd amount of summands with value \( T \).
\end{lemma}
\begin{proof}
  Trivial.
\end{proof}

\begin{proposition}\label{thm:unary_boolean_function_zhegalkin_polynomial}
  For a unary \hyperref[def:boolean_function]{Boolean function} \( f(x) \), the unique square-free Zhegalkin polynomial
  \begin{equation}\label{eq:thm:unary_boolean_function_zhegalkin_polynomial}
    p(x) = ax \oplus b
  \end{equation}
  has coefficients
  \begin{align*}
    b &\coloneqq f(F) \\
    a &\coloneqq f(T) \oplus f(F).
  \end{align*}
\end{proposition}
\begin{proof}
  This is simply a restatement of \eqref{eq:alg:infer_zhegalkin_polynomial/step}.
\end{proof}

\begin{proposition}\label{thm:binary_boolean_function_zhegalkin_polynomial}
  For a binary \hyperref[def:boolean_function]{Boolean function} \( f(x, y) \), the unique square-free Zhegalkin polynomial
  \begin{equation}\label{eq:thm:binary_boolean_function_zhegalkin_polynomial}
    p(x, y) = axy \oplus bx \oplus cy \oplus d
  \end{equation}
  has coefficients
  \begin{align*}
    d &\coloneqq f(F, F) \\
    c &\coloneqq f(F, T) \oplus f(F, F) \\
    b &\coloneqq f(T, F) \oplus f(F, F) \\
    a &\coloneqq \underbrace{f(T, T) \oplus f(T, F) \oplus f(T, F) \oplus f(F, F)}_{T \T*{if} 1 \T*{or} 3 \T*{of the values are} T}.
  \end{align*}
\end{proposition}
\begin{proof}
  \Fullref{alg:infer_zhegalkin_polynomial}
  \begin{equation*}
    f(x, y) = x(ay \oplus b) \oplus (cy \oplus d).
  \end{equation*}

  The expressions for the concrete values for the coefficients follow from \fullref{thm:unary_boolean_function_zhegalkin_polynomial} and \eqref{eq:alg:infer_zhegalkin_polynomial/step}.
\end{proof}

\paragraph{Concrete Boolean functions}

\begin{proposition}\label{thm:standard_boolean_functions}\mcite[sec. 1.1]{Rosen2019DiscreteMathematics}
  The following \hyperref[def:boolean_function]{Boolean functions} are well-established:
  \columnratio{0.15}
  \small
  \begin{paracol}{2}
    \begin{leftcolumn}
      \begin{equation*}
        \begin{array}{*{2}{c}}
          \toprule
          x & \oline{x}  \\
          \midrule
            & \T{not} x  \\
          \midrule
          F & T          \\
          T & F          \\
          \midrule
            & x \oplus T \\
          \bottomrule
        \end{array}
      \end{equation*}
    \end{leftcolumn}

    \begin{rightcolumn}
      \begin{equation*}
        \begin{array}{*{8}{c}}
          \toprule
          x & y & x \vee y             & x \oplus y  & x \wedge y  & x \uparrow y & x \rightarrow y      & x \leftrightarrow y \\
          \midrule
            &   & x \T{or} y           & x \T{xor} y & x \T{and} y & x \T{nand} y & x \T{implies} y      & x \T{iff} y         \\
          \midrule
          F & F & F                    & F           & F           & T            & T                    & T                   \\
          T & F & T                    & T           & F           & T            & F                    & F                   \\
          F & T & T                    & T           & F           & T            & T                    & F                   \\
          T & T & T                    & F           & T           & F            & T                    & T                   \\
          \midrule
            &   & xy \oplus x \oplus y & x \oplus y  & xy          & xy \oplus T  & xy \oplus x \oplus T & x \oplus y \oplus T \\
          \bottomrule
        \end{array}
      \end{equation*}
    \end{rightcolumn}
  \end{paracol}
  \normalsize
  \columnratio{}

  The following abbreviations have been used:
  \begin{itemize}
    \item \term{xor} for \enquote{exclusive or}.
    \item \term[en=nand (\cite[40]{Hinman2005Logic})]{nand} for \enquote{not and}, also known as \term[ru=штрих Шеффера (\cite[29]{Эдельман1975Логика}), en=Sheffer's stroke (\cite[40]{Hinman2005Logic})]{Sheffer's stroke}.
    \item \term{iff} for \enquote{if and only if}.
  \end{itemize}
\end{proposition}
\begin{comments}
  \item The last rows of the two tables contain \hyperref[def:zhegalkin_polynomial]{Zhegalkin polynomials} of the corresponding functions. \Fullref{thm:unary_boolean_function_zhegalkin_polynomial} and \fullref{thm:binary_boolean_function_zhegalkin_polynomial} were used to determine the coefficients.

  \item See \fullref{ex:def:heyting_algebra/three_valued} for how \( {\rightarrow} \) becomes more complicated in three-valued logic.

  \item These tables can be used to prove \fullref{thm:classical_equivalences} and \fullref{thm:classical_tautologies}.
\end{comments}

\begin{proposition}\label{thm:complete_sets_of_boolean_functions}
  The following sets of Boolean functions are \hyperref[def:boolean_closure/complete]{complete}:
  \begin{thmenum}
    \thmitem{thm:complete_sets_of_boolean_functions/zhegalkin} \( \set{ \oplus, \wedge, F, T } \).
    \thmitem{thm:complete_sets_of_boolean_functions/or_not} \( \set{ \vee, \oline{\anon} } \).
    \thmitem{thm:complete_sets_of_boolean_functions/and_not} \( \set{ \wedge, \oline{\anon} } \).
    \thmitem{thm:complete_sets_of_boolean_functions/nand} \( \set{ \uparrow } \).
    \thmitem{thm:complete_sets_of_boolean_functions/conditional_falsum} \( \set{ \rightarrow, F } \).
  \end{thmenum}
\end{proposition}
\begin{proof}
  \SubProofOf{thm:complete_sets_of_boolean_functions/zhegalkin} \Fullref{alg:infer_zhegalkin_polynomial} gives an explicit Zhegalkin polynomial for every Boolean function. The polynomial's constant coefficient is either \( F \) or \( T \); \( \wedge \) is used when evaluating monomials, and \( \oplus \) is used when summing the values of the monomials.

  \SubProofOf{thm:complete_sets_of_boolean_functions/or_not} By inspecting \fullref{thm:standard_boolean_functions}, we can conclude that we can express the operators from \fullref{thm:complete_sets_of_boolean_functions/zhegalkin} via \( \vee \) and \( \oline{\anon} \):
  \begin{itemize}
    \item \( T = x \vee \oline{x} \),
    \item \( F = \oline{T} \),
    \item \( x \wedge y = \oline*{\oline{x} \vee \oline{y}} \),
    \item \( x \oplus y = (x \vee y) \wedge (\oline{x} \vee \oline{y}) \).
  \end{itemize}

  Since the former operators are a complete set, the latter are also a complete set.

  \SubProofOf{thm:complete_sets_of_boolean_functions/and_not} Follows from \fullref{thm:complete_sets_of_boolean_functions/or_not} by noting that
  \begin{equation*}
    x \vee y = \oline*{\oline x \wedge \oline y}.
  \end{equation*}

  \SubProofOf{thm:complete_sets_of_boolean_functions/nand} Note that
  \begin{itemize}
    \item \( \oline{x} = x \uparrow T \),
    \item \( x \wedge y = \oline{x \uparrow y} \).
  \end{itemize}

  \Fullref{thm:complete_sets_of_boolean_functions/and_not} implies that \( \set{ \wedge, \oline{\anon} } \) is complete. Then so is \( \set{ \uparrow } \).

  \SubProofOf{thm:complete_sets_of_boolean_functions/conditional_falsum} Follows from \fullref{thm:complete_sets_of_boolean_functions/nand} by noting that
  \begin{equation*}
    x \uparrow y = x \rightarrow (y \rightarrow F).
  \end{equation*}
\end{proof}

  \subsection{Propositional logic}\label{subsec:propositional_logic}

Propositional logic allows us to express basic relations between atomic propositions --- variables that can be true or false or, in the case of \hyperref[def:propositional_semantics]{non-classical semantics}, have some intermediate value.

\paragraph{Syntax of propositional logic}\hfill

There are different approaches to formalizing the syntax of propositional logic. We use the theory of formal grammars that we have developed in \fullref{subsec:formal_languages} and \fullref{subsec:syntax_trees}. This requires specific adjustments related to other treatments of propositional logic --- e.g. \incite[ch. 1]{Hinman2005}, \incite[pt. I]{Smullyan1995} and \incite[ch. 1]{КолмогоровДрагалин2006} --- which rely on a more informal treatment of syntax more akin to what can be achieved via \fullref{thm:knaster_tarski_iteration}. Our approach allows us to use powerful tools while treating the object logic in complete formality. \incite[45]{Mimram2020} also describes syntax via formal grammars, but does not utilize the theory of formal languages.

\begin{definition}\label{def:propositional_alphabet}\mcite[4; 5]{Smullyan1995}
  The \hyperref[def:formal_language/alphabet]{alphabet} of \term[ru=логика высказываний (\cite[43]{КолмогоровДрагалин2006})]{propositional logic} consists of:

  \begin{thmenum}
    \thmitem{def:propositional_alphabet/constants} Two \term{propositional constants}:
    \begin{thmenum}
      \thmitem{def:propositional_alphabet/constants/verum}\mcite[70]{CitkinMuravitsky2021} \term{verum} \enquote{\( \syntop \)}.
      \thmitem{def:propositional_alphabet/constants/falsum}\mcite[70]{CitkinMuravitsky2021} \term{falsum} \enquote{\( \synbot \)}.
    \end{thmenum}

    \thmitem{def:propositional_alphabet/negation} \term[ru=отрицание (\cite[17]{КолмогоровДрагалин2006})]{Negation} \enquote{\( \synneg \)}.
    \thmitem{def:propositional_alphabet/connectives} The set \( \op*{Conn} \) of \term{binary propositional connectives}, namely
    \begin{thmenum}
      \thmitem{def:propositional_alphabet/connectives/disjunction} \term[ru=дизъюнкция (\cite[17]{КолмогоровДрагалин2006})]{Disjunction} \enquote{\( \synvee \)}, also known as \hyperref[thm:standard_boolean_functions]{\term{or}}.
      \thmitem{def:propositional_alphabet/connectives/conjunction} \term[ru=конъюнкция (\cite[17]{КолмогоровДрагалин2006})]{Conjunction} \enquote{\( \synwedge \)}, also known as \hyperref[thm:standard_boolean_functions]{\term{and}}.
      \thmitem{def:propositional_alphabet/connectives/conditional} \term{Conditional}\fnote{Note that \enquote{conditional} and \enquote{biconditional} are used nouns in this context, although we prefer the phrase \enquote{conditional formula}. We use these terms to avoid confusion with the same connectives in the metalogic, for example \enquote{A biconditional formula is equivalent \dots} could otherwise become \enquote{An equivalence is equivalent \dots}.} \( \synimplies \), also known as \term{if\ldots then} and \hyperref[thm:standard_boolean_functions]{\term[ru=импликация (\cite[17]{КолмогоровДрагалин2006})]{implication}}.
      \thmitem{def:propositional_alphabet/connectives/biconditional} \term{Biconditional} \enquote{\( \syniff \)}, also known as \term{if and only if} (\hyperref[thm:standard_boolean_functions]{\term{iff}}) and \term[ru=эквиваленция (\cite[17]{КолмогоровДрагалин2006})]{equivalence}.
    \end{thmenum}

    \thmitem{def:propositional_alphabet/parentheses} Parentheses \enquote{\( ( \)} and \enquote{\( ) \)} for defining the order of operations unambiguously.
  \end{thmenum}
\end{definition}
\begin{comments}
  \item If desired, we can utilize a smaller propositional language without losing its semantical properties. Important example are the \hyperref[def:cnf_and_dnf]{conjunctive normal forms} in \fullref{alg:perfect_cnf_and_dnf}, although similar constructions hold for other \hyperref[def:boolean_closure/complete]{complete sets of Boolean functions} like those from \fullref{thm:complete_sets_of_boolean_functions}.

  \item We place dots over the various symbols in order to highlight that they are merely symbols without semantics --- see \fullref{rem:mathematical_logic_conventions/metavariable_syntax} for a general discussion.

  \item \enquote{Conjunctio} and \enquote{disjunctio} are Latin words for \enquote{union} and \enquote{separation}, respectively. \enquote{Implicatio} is Latin for \enquote{entangled}.

  \item A popular alternative for the conditional symbol \enquote{\( \synimplies \)} is \enquote{\( \rightimply \)}. It is used by \incite[5]{Kleene2002Logic}, \incite[4]{Smullyan1995}, \incite[44]{КолмогоровДрагалин2006}, \incite[14]{Герасимов2011} and \incite[15]{Эдельман1975}.
\end{comments}

\begin{definition}\label{def:propositional_syntax}\mimprovised
  We will introduce a \hyperref[def:formal_grammar/schema]{grammar schema} whose rules and generated languages we will collectively call the \enquote{\hyperref[con:syntax_and_semantics]{syntax} of propositional logic}:
  \begin{thmenum}
    \thmitem{def:propositional_syntax/schema} Consider the following \hyperref[def:formal_grammar/schema]{grammar schema}:
    \begin{bnf*}
      \bnfprod{variable}    {\bnfpn{Small Latin identifier}} \\
      \bnfprod{connective}  {\bnftsq{\( \synvee \)} \bnfor \bnftsq{\( \synwedge \)} \bnfor \bnftsq{\( \synimplies \)}\bnfor \bnftsq{\( \syniff \)}} \\
      \bnfprod{formula}     {\bnftsq{\( \syntop \)} \bnfor \bnftsq{\( \synbot \)} \bnfor} \\
      \bnfmore              {\bnfpn{variable} \bnfor} \\
      \bnfmore              {\bnftsq{\( \synneg \)} \bnfsp \bnfpn{formula} \bnfor} \\
      \bnfmore              {\bnftsq{(} \bnfsp \bnfpn{formula} \bnfsp \bnfpn{connective} \bnfsp \bnfpn{formula} \bnfsp \bnftsq{)}}
    \end{bnf*}
    where we have used variable identifier rules from \fullref{def:variable_identifier}.

    \thmitem{def:propositional_syntax/prop} We will denote by \( \op*{Prop} \) the set of all \term[ru=пропозициональные переменные (\cite[43]{КолмогоровДрагалин2006})]{propositional variables}, that is, all strings generated by the above grammar schema with starting nonterminal \( \bnfpn{variable} \).

    \thmitem{def:propositional_syntax/formula} Similarly, we will denote by \( \op*{Form} \) the set of \term[ru=формула (\cite[43]{КолмогоровДрагалин2006})]{formulas}, that is, strings generated with starting nonterminal \( \bnfpn{formula} \).

    Within the metalanguage, we will denote abstract formulas via \( \varphi \), \( \psi \), \( \theta \) and other letters as discussed in \fullref{rem:mathematical_logic_conventions/greek_alphabet}. This convention will later lead us to a formal definition of formula schemas in \fullref{def:propositional_formula_schema}.

    We will also call \term{sentences} in relation to abstract \hyperref[def:logical_framework]{logical frameworks}. This contrasts with first-order logic, where only specific formulas are called sentences --- see \fullref{def:first_order_syntax/closed_formula}.

    \thmitem{def:propositional_syntax/language} By \enquote{language of propositional logic} we will mean the set \( \op*{Form} \) of formulas (and hence also the variables).

    \thmitem{def:propositional_syntax/fragment} By a \term[en=fragment (\cite[49]{Mimram2020})]{fragment} of propositional logic we will mean a subset of \( \op*{Form} \), most easily obtained by restricting which rules from the full schema are used. See \fullref{def:implicational_propositional_fragment}.
  \end{thmenum}
\end{definition}
\begin{comments}
  \item Various authors refer to the \enquote{propositional language} --- for example \incite[13]{Hinman2005} --- in relation to various syntactic constructions. These authors do not formally define this language, however. Since we use the mechanism of \hyperref[def:formal_language]{formal languages}, we prefer being more concrete about which syntactic notions we refer to.

  \item We implicitly associate with each propositional formula an \hyperref[con:abstract_syntax_tree]{abstract syntax tree} --- see \fullref{def:propositional_formula_ast}. The grammar of propositional formulas is unambiguous as shown via \fullref{thm:propositional_formulas_are_unambiguous}, which makes it possible to perform proofs via \fullref{thm:induction_on_rooted_trees}.

  \item If the root of the tree is a conjunction, we refer to the formula itself as a conjunction, and similarly for other propositional connectives.
\end{comments}

\begin{proposition}\label{thm:propositional_formulas_are_unambiguous}
  The \hyperref[def:formal_grammar]{grammar} of \hyperref[def:propositional_syntax/formula]{propositional formulas} is \hyperref[def:grammar_ambiguity]{unambiguous}.
\end{proposition}
\begin{proof}
  It is straightforward to adapt the proof from \fullref{ex:natural_number_arithmetic_grammar/unambiguous}.
\end{proof}

\begin{definition}\label{def:conditional_formula}
  For a given \hyperref[def:propositional_syntax/formula]{conditional formula} \( \varphi \synimplies \psi \), we introduce the following terminology:
  \begin{thmenum}
    \thmitem{def:conditional_formula/sufficient_condition}\mcite[def. I.4.1]{Эдельман1975} \( \varphi \) is a \term[ru=достаточное условие]{sufficient condition} for \( \psi \).

    \thmitem{def:conditional_formula/necessary_condition}\mcite[def. I.4.1]{Эдельман1975} \( \psi \) is a \term[ru=необходимое условие]{necessary condition} for \( \varphi \).

    \thmitem{def:conditional_formula/antecedent}\mcite[16]{Эдельман1975} \( \varphi \) is the \term[ru=посылка, en=antecedent (\cite[35]{Rosen1999})]{antecedent} of the conditional \( \varphi \synimplies \psi \).

    \thmitem{def:conditional_formula/consequent}\mcite[16]{Эдельман1975} \( \psi \) is the \term[ru=следствие, en=consequent (\cite[37]{Rosen1999})]{consequent} of \( \varphi \synimplies \psi \).

    \thmitem{def:conditional_formula/inverse}\mcite[def. I.4.3]{Эдельман1975} We call the formula \( \synneg \varphi \synimplies \synneg \psi \) the \term[ru=противоположная (теорема), en=inverse (\cite[49]{Rosen1999})]{inverse} of \( \varphi \synimplies \psi \).

    \thmitem{def:conditional_formula/converse}\mcite[13]{Kleene2002Logic} We call the formula \( \psi \synimplies \varphi \) the \term[ru=обратная (теорема) (\cite[def. I.4.2]{Эдельман1975})]{converse} of \( \varphi \synimplies \psi \).

    \thmitem{def:conditional_formula/contrapositive}\mcite[13]{Kleene2002Logic} The formula \( \synneg \psi \synimplies \synneg \varphi \) is the \term[ru=контрапозиция (\cite[26]{Эдельман1975})]{contrapositive} of \( \varphi \synimplies \psi \).
  \end{thmenum}
\end{definition}
\begin{comments}
  \item In \hyperref[def:classical_logic]{classical logic}, the contrapositive is \hyperref[def:semantic_equivalence]{equivalent} to the original formula due to \fullref{thm:classical_equivalences/contrapositive}.
\end{comments}

\begin{definition}\label{def:propositional_formula_ast}\mimprovised
  We implicitly associate with each propositional formula \( \varphi \) an \hyperref[con:abstract_syntax_tree]{abstract syntax tree} \( T(\varphi) \) as follows:
  \begin{thmenum}
    \thmitem{def:propositional_formula_ast/atomic} If \( \varphi \) is a propositional constant or variable, we define \( T(\varphi) \) as the \hyperref[def:canonical_singleton_tree]{canonical singleton tree} with label \( \varphi \).

    \thmitem{def:propositional_formula_ast/negation} If \( \varphi = \synneg \psi \), assuming we have already built \( T(\psi) \), we define \( T(\varphi) \) by \hyperref[def:ordered_tree_grafting_product]{grafting} \( T(\psi) \) to a new root labeled with \( \synneg \):
    \begin{equation*}
      \includegraphics[page=1]{output/def__propositional_formula_ast}
    \end{equation*}

    \thmitem{def:propositional_formula_ast/connective} If \( \varphi = \psi \syncirc \theta \), assuming we have built \( T(\psi) \) and \( T(\theta) \), we define \( T(\varphi) \) by \hyperref[def:ordered_tree_grafting_product]{grafting} \( T(\psi) \) and \( T(\theta) \) to a new root labeled with \( \syncirc \):
    \begin{equation*}
      \includegraphics[page=2]{output/def__propositional_formula_ast}
    \end{equation*}
  \end{thmenum}
\end{definition}
\begin{comments}
  \item This is similar to \enquote{formation trees} used by \incite[9]{Smullyan1995}, but drawn in reverse and having no excess information present.

  \item The \hyperref[def:rooted_tree/leaf]{leaves} of the tree are variables and constants, while every other node is either a binary connective or negation.
\end{comments}

\begin{example}\label{ex:def:propositional_formula_ast}
  We list examples of \hyperref[def:propositional_formula_ast]{abstract syntax trees for propositional formulas}:
  \begin{thmenum}
    \thmitem{ex:def:propositional_formula_ast/lnc} The law of non-contradiction \eqref{eq:thm:intuitionistic_tautologies/lnc} has the following AST:
    \begin{equation*}
      \includegraphics[page=1]{output/ex__def__propositional_formula_ast}
    \end{equation*}

    \thmitem{ex:def:propositional_formula_ast/associative_conjunction} We will define semantics for conjunction and disjunction so that they become associative Boolean operators. Following our general discussion of such binary operations in \fullref{rem:binary_operation_syntax_trees}, we will generally conflate the formula
    \begin{equation*}
      ((\synp \synwedge \synq) \synwedge \syn r)
    \end{equation*}
    with AST
    \begin{equation*}
      \includegraphics[page=2]{output/ex__def__propositional_formula_ast}
    \end{equation*}
    and the formula
    \begin{equation*}
      ((\synp \synwedge \synq) \synwedge \syn r)
    \end{equation*}
    with AST
    \begin{equation*}
      \includegraphics[page=3]{output/ex__def__propositional_formula_ast}
    \end{equation*}
  \end{thmenum}
\end{example}

\begin{proposition}\label{thm:propositional_formula_balanced_parentheses}
  Propositional formulas have \hyperref[ex:thm:regular_pumping_lemma/balanced_parentheses]{balanced parentheses}, that is, for any formula there are as many left parentheses as there are right parentheses.
\end{proposition}
\begin{proof}
  Denote by \( l_\varphi \) and \( r_\varphi \) the number of left and right parentheses in the formula \( \varphi \).

  We will use \fullref{thm:induction_on_rooted_trees} on \( \varphi \) to prove that \( l_\varphi = r_\varphi \).
  \begin{itemize}
    \item If \( \varphi \) is a constant or variable, there are no parentheses, and \( l_\varphi = r_\varphi = 0 \).
    \item If \( \varphi = \synneg \psi \) and if the inductive hypothesis holds for \( \psi \), then
    \begin{equation*}
      l_\varphi = l_\psi \reloset{\T{ind.}} = r_\psi = r_\varphi
    \end{equation*}

    \item If \( \varphi = (\psi \syncirc \theta) \) and if the hypothesis holds for \( \psi \) and \( \theta \), then
    \begin{equation*}
      l_\varphi = l_\psi + l_\theta + 1 \reloset{\T{ind.}} = r_\psi + r_\theta + 1 = r_\varphi.
    \end{equation*}
  \end{itemize}
\end{proof}

\begin{remark}\label{rem:propositional_formula_parentheses}
  We use several following \enquote{abuse-of-notation} conventions regarding parentheses. These are only notations shortcuts in the \hyperref[con:metalogic]{metalanguage} and the formulas themselves (as abstract mathematical objects) are still assumed to contain parentheses that help them avoid syntactic ambiguity.

  \begin{thmenum}
    \thmitem{rem:propositional_formula_parentheses/outermost} We may skip the outermost parentheses in formulas with top-level \hyperref[def:propositional_alphabet/connectives]{connectives}, e.g. we may write \( \varphi \synwedge \psi \) rather than \( (\varphi \synwedge \psi) \).

    \thmitem{rem:propositional_formula_parentheses/associative} Because of the associativity of \( \synwedge \) and \( \synvee \), which is implied by \fullref{def:propositional_valuation/valuation_function} and \fullref{thm:standard_boolean_functions}, we may skip the parentheses in chains like
    \begin{equation*}
      ( \ldots ((\varphi_1 \synwedge \varphi_2) \synwedge \varphi_3) \synwedge \ldots \synwedge \varphi_{n-1} ) \synwedge \varphi_n.
    \end{equation*}
    and instead write
    \begin{equation*}
      \varphi_1 \synwedge \varphi_2 \synwedge \ldots \synwedge \varphi_{n-1} \synwedge \varphi_n.
    \end{equation*}

    \thmitem{rem:propositional_formula_parentheses/additional} Although not formally necessary, for the sake of readability we may choose to add parentheses around certain formulas like
    \begin{equation*}
      \synneg \varphi \synvee \synneg \psi.
    \end{equation*}
    and instead write
    \begin{equation*}
      (\synneg \varphi) \synvee (\synneg \psi).
    \end{equation*}

    This latter convention is more useful for quantifiers in \hyperref[def:first_order_syntax/formula]{first-order formulas}.
  \end{thmenum}
\end{remark}

\paragraph{Subformulas}

\begin{definition}\label{def:propositional_subformula}\incite[8]{Smullyan1995}
  For every \hyperref[def:propositional_syntax/formula]{propositional formula}, we define the set of \term{subformulas} as follows:
  \begin{equation*}
    \op*{Subform}(\varphi) \coloneqq \begin{cases}
      \set{ \varphi },                                                        &\varphi \in \set{ \syntop, \synbot } \T{or} \varphi \in \op*{Prop}, \\
      \set{ \varphi } \bigcup \op*{Subform}(\psi),                            &\varphi = \synneg \psi, \\
      \set{ \varphi } \bigcup \op*{Subform}(\psi) \cup \op*{Subform}(\theta), &\varphi = \psi \syncirc \theta, {\syncirc} \in \op*{Conn}.
    \end{cases}
  \end{equation*}
\end{definition}

\begin{lemma}\label{thm:propositional_subformula_lemma}
  If the formula \( \psi \) is a \hyperref[def:formal_language/subword]{substring} of the formula \( \varphi \), we have the following possibilities:
  \begin{thmenum}
    \thmitem{thm:propositional_subformula/variable} \( \varphi \) is a variable or constant and \( \psi = \varphi \).
    \thmitem{thm:propositional_subformula/negation_self} \( \varphi = \synneg \theta \) and \( \psi \) coincides with \( \varphi \).
    \thmitem{thm:propositional_subformula/negation} \( \varphi = \synneg \theta \) and \( \psi \) is a subformula of \( \theta \).
    \thmitem{thm:propositional_subformula/connective_self} \( \varphi = (\theta \syncirc \chi) \) and \( \psi \) coincides with \( \varphi \).
    \thmitem{thm:propositional_subformula/connective_left} \( \varphi = (\theta \syncirc \chi) \) and \( \psi \) is a subformula of \( \theta \).
    \thmitem{thm:propositional_subformula/connective_right} \( \varphi = (\theta \syncirc \chi) \) and \( \psi \) is a subformula of \( \chi \) but not \( \theta \).
  \end{thmenum}
\end{lemma}
\begin{proof}
  We use \fullref{thm:induction_on_rooted_trees} on \( \varphi \):
  \begin{itemize}
    \item If \( \varphi \) is a variable or constant, it is a single lexeme, and the only possible substring that is a formula is \( \varphi \) itself. This corresponds to \fullref{thm:propositional_subformula/variable}.
    \item If \( \varphi = \synneg \theta \) and if the inductive hypothesis holds for \( \theta \), we have the following possibilities:
    \begin{itemize}
      \item If \( \psi = \varphi \), then \fullref{thm:propositional_subformula/negation_self} holds.
      \item If \( \psi = \synneg \), it is a substring of \( \varphi \), but not itself a formula.
      \item If \( \psi \) is a substring of \( \theta \), we apply the inductive hypothesis --- then \fullref{thm:propositional_subformula/negation} holds.
    \end{itemize}

    \item If \( \varphi = (\theta \syncirc \chi) \), where the inductive hypothesis holds for \( \theta \) and \( \chi \), we have the following possibilities:
    \begin{itemize}
      \item If \( \psi = \varphi \), then \fullref{thm:propositional_subformula/connective_self} holds.
      \item If \( \psi \) is a substring of \( \theta \), then \fullref{thm:propositional_subformula/connective_left} holds.
      \item If \( \psi \) is a substring of \( \chi \) but not of \( \theta \), then \fullref{thm:propositional_subformula/connective_right} holds.
      \item If \( \psi = (\theta \syncirc \psi_\chi \), where \( \psi_\chi \) is a prefix of \( \chi \), then \( \psi \) has unbalanced parentheses, which contradicts \fullref{thm:propositional_formula_balanced_parentheses}, and thus \( \psi \) is not a formula.
      \item Similarly, if \( \psi = \psi_\theta \syncirc \chi) \), where \( \psi_\theta \) is a suffix of \( \theta \), then again \( \psi \) has unbalanced parentheses.
      \item If \( \psi = \psi_\theta \syncirc \psi_\chi \), where \( \psi_\theta \) is a suffix of \( \theta \) and \( \psi_\chi \) is a prefix of \( \chi \), then \( \psi \) is again not a formula because it is not wrapped in parentheses.
    \end{itemize}
  \end{itemize}
\end{proof}

\begin{proposition}\label{thm:propositional_formula_characterization}
  The \hyperref[def:formal_language/subword]{substring} \( \psi \) of the formula \( \varphi \) is a \hyperref[def:propositional_subformula]{subformula} of \( \varphi \) if and only if \( \psi \) is itself a formula.
\end{proposition}
\begin{proof}
  \SufficiencySubProof Straightforward.
  \NecessitySubProof Suppose that \( \psi \) is itself a formula. Then \fullref{thm:propositional_subformula_lemma} implies that it falls into one of the cases of \fullref{def:propositional_subformula}, and is thus a subformula of \( \varphi \).
\end{proof}

\begin{proposition}\label{thm:propositional_ast_subformula}
  The formula \( \psi \) is a \hyperref[def:propositional_subformula]{subformula} of \( \varphi \) if and only if the \hyperref[def:propositional_formula_ast]{abstract syntax tree} \( T(\varphi) \) has a \hyperref[def:tree/subtree]{subtree} \hyperref[def:labeled_tree/homomorphism]{isomorphic} to \( T(\psi) \).
\end{proposition}
\begin{proof}
  Trivial.
\end{proof}

\paragraph{Intuitionistic propositional semantics}

\begin{definition}\label{def:truth_value_algebra}
  In order to determine whether a sentence holds under given circumstances, we must have a set of \term[ru=истинностное значение (\cite[17]{Герасимов2011}), en=truth value (\cite[9]{Smullyan1995})]{truth values}, for example the Boolean values \( T \) and \( F \) from \fullref{con:boolean_value}. This set may need some additional structure depending on the syntax of the sentences. The aforementioned set is naturally a \hyperref[def:boolean_algebra]{Boolean algebra}, and the most general setting we will consider are \hyperref[def:heyting_algebra]{Heyting algebras}.

  When working with \hyperref[def:institution/models]{models} and notions related to them, we will presume that an underlying Heyting algebra \( \BbbH \) is fixed. In accordance with \fullref{rem:mathematical_logic_conventions/propositional_constants}, we will denote the top and bottom element of \( \BbbH \) by \( T \) and \( F \).
\end{definition}
\begin{comments}
  \item In \fullref{def:propositional_semantics} we will introduce new terminology based on the particular choice of \( \BbbH \).
\end{comments}

\begin{definition}\label{def:propositional_formula_variables}\mimprovised
  For each formula \( \varphi \), we \hyperref[con:evaluation]{recursively define} the set of variables occurring in the \hyperref[def:propositional_syntax/formula]{propositional formulas}:
  \begin{equation*}
    \op*{Var}(\varphi) \coloneqq \begin{cases}
      \varnothing,                            &\varphi \in \set{ \syntop, \synbot }, \\
      \set{ \varphi },                        &\varphi \in \op*{Prop}, \\
      \op*{Var}(\psi),                        &\varphi = \synneg \psi, \\
      \op*{Var}(\psi) \cup \op*{Var}(\theta), &\varphi = \psi \syncirc \theta, {\syncirc} \in \op*{Conn}.
    \end{cases}
  \end{equation*}
\end{definition}

\begin{definition}\label{def:propositional_valuation}\mimprovised
  We will define valuations for propositional formulas in a fixed \hyperref[def:truth_value_algebra]{truth value Heyting algebra} \( \BbbH \).

  \begin{thmenum}
    \thmitem{def:propositional_valuation/interpretation} A \term[ru=интерпретация (\cite[17]{Герасимов2011}), en=interpretation (\cite[10]{Smullyan1995})]{propositional interpretation} is a function with signature \( I: \op*{Prop} \to \BbbH \).

    We may call an interpretation a \enquote{propositional model} in accordance with \fullref{rem:institutional_model_terminology}. This is further discussed in \fullref{rem:classical_propositional_interpretations}.

    \thmitem{def:propositional_valuation/formula_valuation} Given an interpretation \( I \), we define the corresponding \term[ru=значение истинности (формулы) (\cite[8]{Эдельман1975}), en=valuation (\cite[10]{Smullyan1995})]{valuation} of a formula \( \varphi \) inductively as follows:
    \begin{equation}\label{eq:def:propositional_valuation/formula_valuation}
      \Bracks{\varphi}_I \coloneqq \begin{cases}
        T,                                         &\varphi = \syntop \\
        F,                                         &\varphi = \synbot \\
        I(\varphi),                                &\varphi \in \op*{Prop} \\
        \oline{\Bracks{\psi}_I},                   &\varphi = \synneg \psi \\
        \Bracks{\psi}_I \relcirc \Bracks{\theta}_I &\varphi = \psi \syncirc \theta, {\syncirc} \in \op*{Conn},
      \end{cases}
    \end{equation}
    where \( \relcirc \) denotes \hyperref[def:heyting_algebra]{Heyting algebra} operation corresponding to the connective \( \syncirc \).

    \thmitem{def:propositional_valuation/valuation_function} If \( \op*{Var}(\varphi) \subseteq \set{ p_1, \ldots, p_n } \), the valuation \( \Bracks{\varphi}_I \) only depends on the particular values \( I(p_1), \ldots, I(p_n) \) of \( I \). Hence, if the variables are clear from the context, we obtain a Boolean function
    \begin{equation*}
      \begin{aligned}
        &\Bracks{\varphi}: \BbbH^n \to \BbbH, \\
        &\Bracks{\varphi}(x_1, \ldots, x_n) \coloneqq \Bracks{\varphi}_I,
      \end{aligned}
    \end{equation*}
    where \( I \) is any interpretation such that \( I(p_k) = x_k \) for \( k = 1, \ldots, n \).

    Unless otherwise noted, we assume that \( p_1, \ldots, p_n \) are precisely the variables of \( \varphi \), ordered lexicographically as discussed in \fullref{def:variable_identifier}. We will call this the \term{valuation function} of \( \varphi \).
  \end{thmenum}
\end{definition}
\begin{comments}
  \item Different authors use different terminology for the concepts in this definition:
  \begin{itemize}
    \item In a very general setting where \hyperref[def:operation_on_set]{algebraic operations} are assigned to abstract logical connectives, \incite[def. 3.2.3]{CitkinMuravitsky2021} use \enquote{valuation} for what we call \enquote{interpretation}. The aforementioned authors explain how valuations can be extended for all formulas without introducing additional terminology.

    \item When restricted to the case where \( \BbbH = \set{ T, F } \), which in accordance to \hyperref{def:propositional_semantics} we will call \enquote{classical semantics}, \incite[10]{Smullyan1995} uses terminology analogous to ours, but instead allows valuations to be arbitrary functions satisfying \eqref{eq:def:propositional_valuation/formula_valuation}. \incite[80]{Mimram2020} proceeds similarly for interpretations, but introduces no special term for valuations. \incite[def. 1.1.6]{Hinman2005} follows a similar approach, but uses the term \enquote{atomic truth assignment} for what we call \enquote{propositional interpretation}, and \enquote{truth assignment} for what we call \enquote{valuation}.
  \end{itemize}
\end{comments}

\begin{definition}\label{def:propositional_institution}\mimprovised
  For a fixed \hyperref[def:truth_value_algebra]{truth value Heyting algebra} \( \BbbH \), \hyperref[def:propositional_valuation/interpretation]{propositional interpretations} naturally give rise to an \hyperref[def:institution]{institution} as follows:
  \begin{thmenum}
    \thmitem{thm:propositional_institution/signatures} For the category of \hyperref[def:institution/signatures]{signatures}, fix some symbol \( \anon \) and let \( \cat{Sign} \) be the \hyperref[def:discrete_category]{discrete category} on \( \set{ \anon } \).

    \thmitem{thm:propositional_institution/sentences} The \hyperref[def:institution/sentences]{sentence functor} is \( \anon \mapsto \op*{Form} \).

    \thmitem{thm:propositional_institution/models} Let \( \BbbI \) be the discrete category on the function set \( \fun(\op*{Prop}, \BbbH) \). The \hyperref[def:institution/models]{model functor} can then be described as \( \anon \mapsto \BbbI \).

    \thmitem{thm:propositional_institution/satisfaction} Finally, let the \hyperref[def:institution/satisfaction]{satisfaction} relation \( I \vDash_{\anon} \varphi \) hold if \( \Bracks{\varphi}_I = T \).
  \end{thmenum}
\end{definition}
\begin{defproof}
  We must show that this is indeed an institution, which requires verifying \eqref{eq:def:institution/satisfaction}. But this is vacuous because there are no nontrivial signature morphisms.
\end{defproof}

\begin{definition}\label{def:propositional_semantics}\mimprovised
  By the \enquote{\hyperref[con:syntax_and_semantics]{semantics} of propositional logic} we mean the \hyperref[def:propositional_institution]{propositional institution} and all related notions like \hyperref[def:institution/models]{models}, \hyperref[def:institution/satisfaction]{satisfaction}, \hyperref[def:institutional_entailment]{semantic entailment} and \hyperref[def:semantic_equivalence]{semantic equivalence}. We will say that the semantics are \term{\hyperref[def:intuitionistic_logic]{intuitionistic}}, while in the special case where the \hyperref[def:truth_value_algebra]{truth value Heyting algebra} is the two-element Boolean algebra \( \set{ T, F } \), we will call them \term{\hyperref[def:classical_logic]{classical}} or \term{Boolean}.
\end{definition}

\begin{remark}\label{rem:classical_propositional_interpretations}
  Note that the same propositional interpretation may be used to define different valuations, and the notion of \enquote{classical} or \enquote{intuitionistic} semantics from \fullref{def:propositional_semantics} only applies when \eqref{eq:def:propositional_valuation/formula_valuation} is used. For example, a slight variation of the aforementioned valuation results in \fullref{def:minimal_propositional_semantics}. Thus, an interpretation cannot, by itself, be classical.

  On the other hand, (institutional) models can be classical, and propositional models are interpretations, so we are free to use the term \enquote{classical model} when referring to interpretations in an appropriate context.
\end{remark}

\begin{definition}\label{def:propositional_entailment}\mimprovised
  For a fixed \hyperref[def:truth_value_algebra]{truth value Heyting algebra}, we will now consider the \hyperref[def:institutional_entailment]{institutional entailment} relation obtained via \fullref{def:propositional_institution}, according to which that the set \( \Gamma \) of \hyperref[def:propositional_syntax/formula]{propositional formulas} entails \( \psi \) if, whenever some \hyperref[def:propositional_valuation/interpretation]{interpretation} \hyperref[thm:propositional_institution/satisfaction]{satisfies} every formula of \( \Gamma \), it also satisfies \( \psi \).

  We denote the corresponding relation via \( {\vDash} \). Similarly to other \hyperref[def:entailment_system/entailment]{entailment relations}, we use the sequent notation discussed in \fullref{rem:sequent_notation}.
\end{definition}

\begin{example}\label{ex:trivial_heyting_semantics}
  Within the one-element Heyting algebra, where \( T = F \), every formula is satisfied because there is simply no \enquote{non-true} truth value. \Fullref{thm:inconsistent_lindenbaum_tarski_algebra} implies that this precisely is the \hyperref[def:lindenbaum_tarski_algebra]{Lindenbaum-Tarski algebra} of any \hyperref[def:consistent_set_of_sentences]{inconsistent} set of formulas.
\end{example}

\begin{example}\label{ex:heyting_semantics_lem_counterexample}
  Consider the three-valued Heyting algebra from \fullref{ex:def:heyting_algebra/three_valued}, where \( F < N < T \).

  Let \( I \) be a \hyperref[def:propositional_valuation]{propositional interpretation} such that \( I(\synp) = N \). Then the valuation of \eqref{eq:thm:classical_tautologies/lem} is
  \begin{equation*}
    \Bracks{\synp \synvee \synneg \synp}_I
    =
    \Bracks{\synp}_I \vee \oline{\Bracks{\synp}_I}
    =
    N \vee F
    =
    N.
  \end{equation*}

  Therefore, \eqref{eq:thm:classical_tautologies/lem} does not hold in general.
\end{example}

\begin{example}\label{ex:topological_semantics_lem_counterexample}
  Consider the standard topology in \( \BbbR \) and the corresponding topological Heyting algebra from \fullref{ex:def:heyting_algebra/topology}.

  Let \( U \) be an open set. We will examine \eqref{eq:thm:classical_tautologies/lem}. Given any \hyperref[def:propositional_valuation]{propositional interpretation} \( I \) such that \( I(\synp) = U \), we have
  \begin{equation*}
    \Bracks{\synp \synvee \synneg \synp}_I
    =
    \Bracks{\synp}_I \cup \oline{\Bracks{\synp}_I}
    =
    U \cup \oline{U}
    =
    U \cup \Int(\BbbR \setminus U).
  \end{equation*}

  If \( U \) is empty, then \( \Bracks{\synp \synvee \synneg \synp}_I = \BbbR \) and \eqref{eq:thm:classical_tautologies/lem} holds in this case. If \( U \) is the open unit interval \( (0, 1) \), then \( \Bracks{\synp \synvee \synneg \synp}_I = \BbbR \setminus \set{ 0, 1 } \) and \eqref{eq:thm:classical_tautologies/lem} does not hold.
\end{example}

\begin{proposition}\label{thm:intuitionistic_equivalences}
  We will list some \hyperref[def:propositional_semantics]{intuitionistic} propositional \hyperref[def:semantic_equivalence]{semantic equivalences}:
  \begin{thmenum}
    \thmitem{thm:intuitionistic_equivalences/negation_bottom} Negation can be expressed as a conditional formula whose \hyperref[def:conditional_formula/consequent]{consequent} is the falsum:
    \begin{equation}\label{eq:thm:intuitionistic_equivalences/negation_bottom}
      \mathllap{\synneg \varphi} \gleichstark \mathrlap{\varphi \synimplies \synbot.}
    \end{equation}

    \thmitem{thm:intuitionistic_equivalences/top_elim} We can eliminate verum from the antecedent of conditional formulas:
    \begin{equation}\label{eq:thm:intuitionistic_equivalences/top_elim}
      \mathllap{\syntop \synimplies \varphi} \gleichstark \mathrlap{\varphi.}
    \end{equation}

    \thmitem{thm:intuitionistic_equivalences/contradiction} Falsum is equivalent to a conjunction of a formula and its negation:
    \begin{equation}\label{eq:thm:intuitionistic_equivalences/contradiction}
      \mathllap{\varphi \synwedge \synneg \varphi} \gleichstark \mathrlap{\synbot.}
    \end{equation}
  \end{thmenum}
\end{proposition}
\begin{comments}
  \item For every formula \( \varphi \), \eqref{eq:thm:intuitionistic_equivalences/negation_bottom} is a different axiom, and we refer to \eqref{eq:thm:intuitionistic_equivalences/negation_bottom} itself as an \enquote{axiom schema}. We will formalize schemas via \hyperref[def:propositional_formula_schema]{formula schemas} in \fullref{subsec:axiomatic_derivations}.
  \item Both \eqref{eq:thm:intuitionistic_equivalences/negation_bottom} and \eqref{eq:thm:intuitionistic_equivalences/top_elim} hold in \hyperref[def:minimal_propositional_semantics]{minimal semantics}, and their proofs do not use any semantic properties of the falsum.
\end{comments}
\begin{proof}
  \SubProofOf{thm:intuitionistic_equivalences/negation_bottom} Follows from \fullref{def:heyting_algebra/pseudocomplement}.

  \SubProofOf{thm:intuitionistic_equivalences/top_elim} Follows from \fullref{thm:def:heyting_algebra/top_left}.

  \SubProofOf{thm:intuitionistic_equivalences/contradiction} For any interpretation \( I \), we have
  \begin{equation*}
    \Bracks{ \varphi \synwedge \synneg \varphi }_I
    =
    \Bracks{ \varphi }_I \synwedge (\Bracks{ \varphi }_I \rightarrow F)
    \reloset {\eqref{eq:def:heyting_algebra/axioms/modus_ponens}} =
    \Bracks{ \varphi }_I \wedge F
    =
    F
    =
    \Bracks{\synbot}_I.
  \end{equation*}
\end{proof}

\begin{definition}\label{def:propositional_tautology}
  We say that \( \varphi \) is a \term[ru=пропозициональная тавтология (\cite[44]{КолмогоровДрагалин2006})]{propositional tautology} if any of the following equivalent conditions hold:
  \begin{thmenum}
    \thmitem{def:propositional_tautology/interpretations}\mcite[11]{Smullyan1995} We have \( \Bracks{\varphi}_I = T \) for every interpretation \( I \).
    \thmitem{def:propositional_tautology/entailment} We have the entailment \( \vDash \varphi \).
    \thmitem{def:propositional_tautology/equivalence} We have the equivalence \( \varphi \gleichstark \syntop \).
  \end{thmenum}
\end{definition}

\begin{definition}\label{def:propositional_contradiction}
  Dually, we say that \( \varphi \) is a \term[en=contradictory (formula) (\cite[28]{Kleene2002Logic})]{propositional contradiction} if any of the following equivalent conditions hold:
  \begin{thmenum}
    \thmitem{def:propositional_contradiction/interpretations}\mcite[def. 1.4.1(i)]{Hinman2005} We have \( \Bracks{\varphi}_I = F \) for every interpretation \( I \).
    \thmitem{def:propositional_contradiction/entailment} We have the entailment \( \varphi \vDash \synbot \).
    \thmitem{def:propositional_contradiction/equivalence} We have the equivalence \( \varphi \gleichstark \synbot \).
  \end{thmenum}
\end{definition}

\begin{proposition}\label{thm:intuitionistic_tautologies}
  We will list some \hyperref[def:propositional_semantics]{intuitionistic} propositional \hyperref[def:propositional_tautology]{tautologies}:
  \begin{thmenum}
    \thmitem{thm:intuitionistic_tautologies/self} Every formula implies itself:
    \begin{equation}\label{eq:thm:intuitionistic_tautologies/self}
      \varphi \synimplies \varphi.
    \end{equation}

    \thmitem{thm:intuitionistic_tautologies/dni} Any formula implies its double negation:
    \begin{equation}\label{eq:thm:intuitionistic_tautologies/dni}
      \varphi \synimplies \neg \neg \varphi \tag{\( \logic{DNI}_A \)}
    \end{equation}

    \enquote{DNI} stands for \enquote{double negation introduction}.

    \thmitem{thm:intuitionistic_tautologies/efq} Falsum implies anything:
    \begin{equation}\label{eq:thm:intuitionistic_tautologies/efq}
      \synbot \synimplies \varphi \tag{\( \logic{EFQ}_A \)}
    \end{equation}

    \enquote{EFQ} stands for \enquote{ex falso quodlibet}, which is Latin for \enquote{from falsehood, anything}. It is also known as the \enquote{the principle of explosion} because it allows us to show that intuitionistic semantics are \hyperref[def:paraconsistent_consequence_operator]{explosive} --- see \fullref{thm:intuitionistic_semantics_are_explosive}. Both terms are used by \incite[47]{Mimram2020}.

    \thmitem{thm:intuitionistic_tautologies/ecq} A \hyperref[def:propositional_contradiction]{contradiction} implies anything:
    \begin{equation}\label{eq:thm:intuitionistic_tautologies/ecq}
      (\varphi \synwedge \neg \varphi) \synimplies \psi \tag{\( \logic{ECQ}_A \)}
    \end{equation}

    \enquote{EFQ} stands for \enquote{ex contradictione quodlibet}, which is Latin for \enquote{from a contradiction, anything}. The term is used by \incite[2]{DienerMcKubreJordens2016}.

    \thmitem{thm:intuitionistic_tautologies/lnc} A formula and its negation cannot both hold:
    \begin{equation}\label{eq:thm:intuitionistic_tautologies/lnc}
      \synneg (\varphi \synwedge \synneg \varphi). \tag{\( \logic{LNC}_A \)}
    \end{equation}

    \enquote{LCN} stands for \enquote{law of non-contradiction}.
  \end{thmenum}
\end{proposition}
\begin{comments}
  \item Some tautologies that hold in \hyperref[def:propositional_semantics]{classical semantics} are listed in \fullref{thm:classical_tautologies}.
  \item Both \eqref{eq:thm:intuitionistic_tautologies/dni} and \eqref{eq:thm:intuitionistic_tautologies/lnc} hold under \hyperref[def:minimal_propositional_semantics]{minimal semantics}, but our semantic proofs for them are not valid in minimal logic. We will demonstrate simple syntactic proofs in \fullref{thm:syntactic_minimal_tautologies}.
\end{comments}
\begin{proof}
  Fix an interpretation \( I \).

  \SubProofOf{thm:intuitionistic_tautologies/self}
  \begin{equation*}
    \Bracks{\varphi \synimplies \varphi}_I
    =
    \Bracks{\varphi}_I \synimplies \Bracks{\varphi}_I
    \reloset {\ref{thm:def:heyting_algebra/leq}} =
    T.
  \end{equation*}

  \SubProofOf{thm:intuitionistic_tautologies/dni}
  \begin{equation*}
    \Bracks{\neg \neg \varphi}_I
    =
    \Bracks{\varphi}_I \rightarrow \oline{\oline{\Bracks{\varphi}_I}}
    \reloset {\ref{thm:def:heyting_algebra/dni}} =
    T.
  \end{equation*}

  \SubProofOf{thm:intuitionistic_tautologies/efq}
  \begin{equation*}
    \Bracks{\synbot \synimplies \varphi}_I
    =
    \Bracks{\synbot}_I \implies \Bracks{\varphi}_I
    \reloset {\ref{thm:def:heyting_algebra/leq}} =
    T.
  \end{equation*}

  \SubProofOf{thm:intuitionistic_tautologies/ecq}
  \begin{equation*}
    \Bracks{(\varphi \synwedge \neg \varphi) \synimplies \psi}_I
    \reloset {\eqref{eq:thm:intuitionistic_equivalences/contradiction}} =
    \Bracks{\synbot \synimplies \psi}_I
    \reloset {\eqref{eq:thm:intuitionistic_tautologies/efq}} =
    T.
  \end{equation*}

  \SubProofOf{thm:intuitionistic_tautologies/lnc}
  \begin{equation*}
    \Bracks{\synneg (\varphi \synwedge \synneg \varphi)}_I
    =
    \oline{\Bracks{\varphi}_I \wedge \oline{\Bracks{\varphi}_I}}
    \reloset {\eqref{eq:thm:intuitionistic_equivalences/contradiction}} =
    \oline{F}
    =
    F \rightarrow F
    \reloset {\eqref{eq:thm:intuitionistic_tautologies/self}} =
    T.
  \end{equation*}
\end{proof}

\paragraph{Brouwer-Heyting-Kolmogorov interpretation}

\begin{concept}\label{con:brouwer_heyting_kolmogorov_interpretation}\mcite[sec. 1.3.1]{TroelstraSchwichtenberg2000}
  \hyperref[def:propositional_semantics]{Intuitionistic semantics} correspond to the less formal \term{Brouwer-Heyting-Kolmogorov interpretation}. This interpretation is based on the notion of a \enquote{construction}, which is also why we will refer to intuitionistic logic as \term{constructive logic}.

  \begin{thmenum}
    \thmitem{con:brouwer_heyting_kolmogorov_interpretation/verum} We suppose that \( \syntop \) is evident without a construction.

    \thmitem{con:brouwer_heyting_kolmogorov_interpretation/atomic} We suppose that we know what constitutes a construction proving a propositional variable.

    \thmitem{con:brouwer_heyting_kolmogorov_interpretation/falsum} There is no construction that proves \( \synbot \).

    \thmitem{con:brouwer_heyting_kolmogorov_interpretation/disjunction} A construction proving \( \varphi \synvee \psi \) is a construction proving \( \varphi \), or a construction proving \( \psi \).

    \thmitem{con:brouwer_heyting_kolmogorov_interpretation/conjunction} A construction proving \( \varphi \synwedge \psi \) is a pair of constructions, one proving \( \varphi \) and one proving \( \psi \).

    \thmitem{con:brouwer_heyting_kolmogorov_interpretation/conditional} A construction proving \( \varphi \synimplies \psi \) is a transformation of constructions proving \( \varphi \) into constructions proving \( \psi \).

    \thmitem{con:brouwer_heyting_kolmogorov_interpretation/biconditional} Based on \fullref{con:brouwer_heyting_kolmogorov_interpretation/conditional}, the biconditional \( \varphi \syniff \varphi \) then corresponds to a pair of transformations (not necessarily inverses).

    \thmitem{con:brouwer_heyting_kolmogorov_interpretation/negation} A construction of \( \neg \varphi \) demonstrates the impossibility of a construction of \( \varphi \). Based on \fullref{con:brouwer_heyting_kolmogorov_interpretation/falsum} and \fullref{con:brouwer_heyting_kolmogorov_interpretation/conditional}, a construction of \( \neg \varphi \) should correspond to a transformation of constructions proving \( \varphi \) into elements of the empty set, and such a transformation is only possible if there exist no constructions proving \( \varphi \).
  \end{thmenum}
\end{concept}

\begin{example}\label{ex:con:brouwer_heyting_kolmogorov_interpretation/well_ordering_principle_zfc}
  \Fullref{thm:well_ordering_theorem} in \hyperref[def:zfc]{\( \logic{ZFC} \)} does not provide a way to well-order an arbitrary set. The theorem relies on the axiom of choice, whose consequence \fullref{thm:diaconescu_goodman_myhill_theorem} proves the law of the excluded middle \eqref{eq:thm:classical_tautologies/lem} from the axioms of \logic{ZF}.

  Since \logic{LEM} may not hold in intuitionistic logic, it follows that both \fullref{thm:well_ordering_theorem} and the axiom of choice itself should not in general hold under the Brouwer-Heyting-Kolmogorov interpretation, hence by the terminology in \fullref{con:brouwer_heyting_kolmogorov_interpretation}, \fullref{thm:well_ordering_theorem} is a non-constructive theorem.
\end{example}

\paragraph{Minimal propositional semantics}

\begin{definition}\label{def:minimal_propositional_semantics}\mimprovised
  In addition to classical and intuitionistic semantics discussed in \fullref{def:propositional_semantics}, we will also occasionally consider \term{minimal semantics}, obtained by interpreting \( \synbot \) as a propositional variable. Thus, a propositional interpretation must now provide a value for \( \synbot \).

  Formula valuations are thus defined as follows:
  \begin{equation}\label{eq:def:minimal_propositional_semantics/formula_valuation}
    \Bracks{\varphi}_I \coloneqq \begin{cases}
      I(\synbot),                             &\varphi = \synbot \\
      \Bracks{\psi}_I \rightarrow I(\synbot), &\varphi = \synneg \psi \\
      \vdots, &
    \end{cases}
  \end{equation}
  the other cases being the same as in \eqref{eq:def:propositional_valuation/formula_valuation}.
\end{definition}
\begin{comments}
  \item \Fullref{thm:lindenbaum_tarski_algebras} provides a concrete motivation for this precise definition.
  \item Unless we are specifically interested in \( \synbot \) and \( \synneg \), a more reasonable approach may be to only consider the \hyperref[def:propositional_syntax/fragment]{fragment} of propositional logic without them. This is approach to minimal logic mentioned in \mcite[49]{Mimram2020}, and this is the approach used in \fullref{def:minimal_implication_logic}. In this case we lose the distinction between minimal and intuitionistic logic. Perhaps this is the entire point of working in minimal logic?
  \item Minimal logic was introduced by Ingebrigt Johansson as an attempt to further refine intuitionistic logic. An analysis of his works and his interactions with Gerhard Gentzen is given by \incite{VanDerMolen2016}. No semantics are discussed there.
  \item Our definition for minimal semantics is based on the general principles outlined by \incite[3]{VanDerMolen2016}, \incite[35]{TroelstraSchwichtenberg2000} and \incite[1]{DienerMcKubreJordens2016}. None of the aforementioned discuss semantics.
\end{comments}

\begin{definition}\label{def:paraconsistent_consequence_operator}\mcite{StanfordPlato:paraconsistent_logic}
  We say that the \hyperref[def:consequence_operator]{consequence operator} \( \vdash \) for propositional logic is \term{explosive} if the following generalization of \eqref{eq:thm:intuitionistic_tautologies/ecq} holds:
  \begin{equation*}
    \varphi, \synneg \varphi \vdash \psi.
  \end{equation*}

  If \( \vdash \) is not explosive, we say that it is \term{paraconsistent}.
\end{definition}

\begin{proposition}\label{thm:minimal_semantics_are_paraconsistent}
  \hyperref[def:minimal_propositional_semantics]{Minimal semantics} are \hyperref[def:paraconsistent_consequence_operator]{paraconsistent}.
\end{proposition}
\begin{proof}
  Fix an interpretation \( I \) such that \( I(\synbot) = T \). Then
  \begin{equation*}
    \Bracks{\synneg \syntop}_I
    =
    \Bracks{\syntop}_I \rightarrow I(\synbot)
    =
    T \rightarrow T
    \reloset {\ref{thm:intuitionistic_tautologies/self}} =
    T
  \end{equation*}

  Then \( I \vDash \syntop \) and \( I \vDash \synneg \syntop \). Yet, if \( I(P) = F \) for some variable \( P \), we cannot conclude that \( I \vDash P \).
\end{proof}

\begin{proposition}\label{thm:intuitionistic_semantics_are_explosive}
  \hyperref[def:propositional_semantics]{Intuitionistic semantics} are \hyperref[def:paraconsistent_consequence_operator]{explosive}.
\end{proposition}
\begin{proof}
  Fix two formulas, \( \varphi \) and \( \psi \), and an interpretation \( I \) such that \( I \vDash \varphi \) and \( I \vDash \psi \). Then \( I \vDash \varphi \synwedge \synneg \varphi \) because
  \begin{equation*}
    \Bracks{\varphi \synwedge \synneg \varphi}_I
    =
    \Bracks{\varphi}_I \synwedge \Bracks{\synneg \varphi}_I
    =
    T \wedge T
    =
    T.
  \end{equation*}

  But also
  \begin{equation*}
    \Bracks{\varphi \synwedge \synneg \varphi}_I
    =
    \Bracks{\varphi}_I \synwedge (\Bracks{\varphi}_I \rightarrow F)
    \reloset {\eqref{eq:def:heyting_algebra/axioms/modus_ponens}} =
    \Bracks{\varphi}_I \synwedge F
    =
    F.
  \end{equation*}

  The obtained contradiction shows that such an interpretation \( I \) does not exist.

  Therefore, for all zero interpretations \( I \) for which both \( I \vDash \varphi \) and \( I \vDash \psi \), we can conclude that \( I \vDash \psi \).
\end{proof}

\begin{proposition}\label{thm:semantic_propositional_conjunction_of_premises}
  With respect to \hyperref[def:minimal_propositional_semantics]{minimal semantics}, we have \( \varphi, \psi \vDash \theta \) if and only if \( (\varphi \synwedge \psi) \vDash \theta \).
\end{proposition}
\begin{proof}
  Trivial.
\end{proof}

\begin{theorem}[Propositional semantic deduction theorem]\label{thm:propositional_semantic_deduction_theorem}
  With respect to \hyperref[def:minimal_propositional_semantics]{minimal semantics}, for an arbitrary \hyperref[def:truth_value_algebra]{truth value Heyting algebra} and arbitrary propositional formulas, we have
  \begin{equation*}
    \Gamma, \varphi \vDash \psi \T{if and only if} \Gamma \vDash \varphi \synimplies \psi.
  \end{equation*}
\end{theorem}
\begin{comments}
  \item See \fullref{rem:deduction_theorem_list} for a list of similar theorems.
\end{comments}
\begin{proof}
  Fix an interpretation \( I \) satisfying \( \Gamma \).

  \SufficiencySubProof Suppose that from \( I \vDash \varphi \) it follows that \( I \vDash \psi \). If \( I \vDash \varphi \), then
  \begin{equation*}
    \Bracks{\varphi \synimplies \psi}_I
    =
    \underbrace{\Bracks{\varphi}_I}_T \rightarrow \underbrace{\Bracks{\psi}_I}_T
    \reloset {\ref{thm:def:heyting_algebra/top_left}} =
    \underbrace{\Bracks{\psi}_I}_T,
  \end{equation*}
  thus \( I \vDash (\varphi \synimplies \psi) \).

  \NecessitySubProof Suppose that \( I \vDash (\varphi \synimplies \psi) \). If \( I \vDash \varphi \), then
  \begin{equation*}
    \underbrace{\Bracks{\varphi \synimplies \psi}_I}_{T}
    =
    \underbrace{\Bracks{\varphi}_I}_T \rightarrow \Bracks{\psi}_I
    \reloset {\ref{thm:def:heyting_algebra/top_left}} =
    \Bracks{\psi}_I,
  \end{equation*}
  thus \( I \vDash \psi \).
\end{proof}

\begin{corollary}\label{thm:intuitionistic_deduction_consequences}
  We list some consequences of \Fullref{thm:propositional_semantic_deduction_theorem}:
  \begin{thmenum}
    \thmitem{thm:intuitionistic_deduction_consequences/top} Under \hyperref[def:minimal_propositional_semantics]{minimal semantics}, \( \varphi \vDash \syntop \) for every formula \( \varphi \).

    \thmitem{thm:intuitionistic_deduction_consequences/bot} Under \hyperref[def:propositional_semantics]{intuitionistic semantics}, \( \synbot \vDash \varphi \) for every formula \( \varphi \).
  \end{thmenum}
\end{corollary}
\begin{proof}
  \SubProofOf{thm:intuitionistic_deduction_consequences/top} Follows from \fullref{thm:def:heyting_algebra/top_right}.

  \SubProofOf{thm:intuitionistic_deduction_consequences/bot} Follows from \fullref{thm:def:heyting_algebra/leq}.
\end{proof}

\begin{remark}\label{rem:deduction_theorem_list}
  The following is a list of different flavors of the \enquote{deductive theorem} that allows us to use \hyperref[def:propositional_alphabet/connectives/conditional]{implication} and \hyperref[def:entailment_system/entailment]{entailment} interchangeably:
  \begin{thmenum}
    \thmitem{rem:deduction_theorem_list/propositional_semantic} \Fullref{thm:propositional_semantic_deduction_theorem}
    \thmitem{rem:deduction_theorem_list/implicational_syntactic} \Fullref{thm:implicational_syntactic_deduction_theorem}
    \thmitem{rem:deduction_theorem_list/propositional_syntactic} \Fullref{thm:propositional_syntactic_deduction_theorem}
  \end{thmenum}
\end{remark}

\paragraph{Classical propositional semantics}

\begin{proposition}\label{thm:classical_equivalences}
  We will list some \hyperref[def:propositional_semantics]{classical} propositional \hyperref[def:semantic_equivalence]{equivalences}\fnote{These are actually statements about \hyperref[thm:standard_boolean_functions]{standard Boolean functions}, but nevertheless we find propositional logic more convenient for stating them.}:
  \begin{thmenum}
    \thmitem{thm:classical_equivalences/double_negation} Negation is an \hyperref[def:involution]{involution}:
    \begin{equation}\label{eq:thm:classical_equivalences/double_negation}
      \mathllap{\synneg \synneg \varphi} \gleichstark \mathrlap{\varphi.}
    \end{equation}

    This equivalence is a combination of \eqref{eq:thm:intuitionistic_tautologies/dni} and \eqref{eq:thm:classical_tautologies/dne}.

    \thmitem{thm:classical_equivalences/conditional_as_disjunction} A conditional formula is a disjunction with the \hyperref[def:conditional_formula/antecedent]{antecedent} negated:
    \begin{equation}\label{eq:thm:classical_equivalences/conditional_as_disjunction}
      \mathllap{\varphi \synimplies \psi} \gleichstark \mathrlap{ \synneg \varphi \synvee \psi. }
    \end{equation}

    This equivalence is known under the name \enquote{material implication} and is discussed in \fullref{con:material_implication}.

    \thmitem{thm:classical_equivalences/contrapositive} A conditional formula is equivalent to its \hyperref[def:conditional_formula/contrapositive]{contrapositive}:
    \begin{equation}\label{eq:thm:classical_equivalences/contrapositive}
      \mathllap{\varphi \synimplies \psi} \gleichstark \mathrlap{\synneg \psi \synimplies \synneg \varphi.}
    \end{equation}

    \thmitem{thm:classical_equivalences/distributivity} Disjunctions and conjunctions distribute over each other:
    \begin{subequations}
      \begin{align}
        \mathllap{\varphi \synvee (\psi \synwedge \theta)} &\gleichstark \mathrlap{(\varphi \synvee \psi) \synwedge (\varphi \synvee \theta),}   \label{eq:thm:classical_equivalences/distributivity/join_over_meet} \\
        \mathllap{\varphi \synwedge (\psi \synvee \theta)} &\gleichstark \mathrlap{(\varphi \synwedge \psi) \synvee (\varphi \synwedge \theta).} \label{eq:thm:classical_equivalences/distributivity/meet_over_join}
      \end{align}
    \end{subequations}

    This equivalence motivates axioms \eqref{eq:def:distributive_lattice/join_over_meet} and \eqref{eq:def:distributive_lattice/meet_over_join} for \hyperref[def:distributive_lattice]{distributive lattices}.

    \thmitem{thm:classical_equivalences/de_morgan} Conjunctions and disjunctions are obtained from each other via negation:
    \begin{subequations}
      \begin{align}
        \mathllap{\synneg (\varphi \synvee \psi)}   &\gleichstark \mathrlap{\synneg \varphi \synwedge \synneg \psi,} \label{eq:thm:classical_equivalences/de_morgan/complement_of_join} \\
        \mathllap{\synneg (\varphi \synwedge \psi)} &\gleichstark \mathrlap{\synneg \varphi \synvee \synneg \psi.}   \label{eq:thm:classical_equivalences/de_morgan/complement_of_meet}
      \end{align}
    \end{subequations}

    See \fullref{rem:de_morgans_laws} for a list of related theorems.

    \thmitem{thm:classical_equivalences/biconditional_member_negation} A biconditional formula is equivalent to its termwise negation:
    \begin{equation}\label{eq:thm:classical_equivalences/biconditional_member_negation}
      \mathllap{\varphi \syniff \psi} \gleichstark \mathrlap{\synneg \varphi \syniff \synneg \psi.}
    \end{equation}

    \thmitem{thm:classical_equivalences/biconditional_negation} A negation of a biconditional formula is again a biconditional with one of the terms negated:
    \begin{equation}\label{eq:thm:classical_equivalences/biconditional_negation}
      \begin{aligned}
        \mathllap{\synneg \parens{\varphi \syniff \psi}}
        &\gleichstark
        \mathrlap{\synneg \varphi \syniff \psi \gleichstark}
        \\ &\gleichstark
        \mathrlap{\varphi \syniff \synneg \psi.}
      \end{aligned}
    \end{equation}
  \end{thmenum}
\end{proposition}
\begin{comments}
  \item These equivalences fail more generally in \hyperref[def:intuitionistic_logic]{intuitionistic logic}.
\end{comments}
\begin{proof}
  The proofs follow directly from the table in \fullref{thm:standard_boolean_functions}.
\end{proof}

\begin{proposition}\label{thm:classical_tautologies}
  We will list some \hyperref[def:propositional_semantics]{classical} propositional \hyperref[def:propositional_tautology]{tautologies}:
  \begin{thmenum}
    \thmitem{thm:classical_tautologies/dne} The double negation of a formula implies it:
    \begin{equation}\label{eq:thm:classical_tautologies/dne}
      \synneg \synneg \varphi \synimplies \varphi \tag{\( \logic{DNE}_A \)}
    \end{equation}

    \enquote{DNE} stands for \enquote{double negation elimination}.

    \thmitem{thm:classical_tautologies/pierce} The following, named \enquote{Pierce's law}, holds:
    \begin{equation}\label{eq:thm:classical_tautologies/pierce}
      ((\varphi \synimplies \psi) \synimplies \varphi) \synimplies \varphi \tag{\( \logic{Pierce}_A \)}
    \end{equation}

    \thmitem{thm:classical_tautologies/lem} Either a formula holds or its negation does:
    \begin{equation}\label{eq:thm:classical_tautologies/lem}
      \varphi \synvee \synneg \varphi. \tag{\( \logic{LEM}_A \)}
    \end{equation}

    \enquote{LEM} stands for \enquote{law of the excluded middle}.
  \end{thmenum}
\end{proposition}
\begin{comments}
  \item Some tautologies that hold in \hyperref[def:propositional_semantics]{intuitionistic semantics} are listed in \fullref{thm:intuitionistic_tautologies}.
\end{comments}
\begin{proof}
  The proofs follow directly from the table in \fullref{thm:standard_boolean_functions}.
\end{proof}

\begin{concept}\label{con:material_implication}
  Consider the conditional formula \( \varphi \synimplies \psi \). Via \fullref{thm:propositional_semantic_deduction_theorem}, the equivalence \eqref{eq:thm:classical_equivalences/conditional_as_disjunction} can be restated as follows:
  \begin{displayquote}
    The formula \( \varphi \synimplies \psi \) holds unless the antecedent \( \varphi \) holds and the consequent \( \psi \) doesn't.
  \end{displayquote}

  This equivalence goes back to Philo of Megara, who lived in the third century B.C. We refer to it as \term[ru=материальная импликация (\cite[74]{КолмогоровДрагалин2006}), en=material implication (\cite[9]{Kleene2002Logic})]{material implication} to distinguish it from \enquote{strict} and \enquote{relevant} implications, as well as \enquote{assertions} and other possibilities that try to avoid the so-called \enquote{paradoxes of material implication}. A survey of the aforementioned concepts has been written by \incite{StanfordPlato:logic_of_conditionals}.
\end{concept}

  \section{Simple type theory}\label{sec:simple_type_theory}

  \subsection{First-order logic}\label{subsec:first_order_logic}

The idea of first-order predicate logic (we will omit \enquote{predicate} and only refer to \enquote{first-order logic}) is to create a formal language whose semantics (given by structures) supports boolean operations and can quantify over all elements of an ambient universe. Unlike in \hyperref[subsec:propositional_logic]{propositional logic}, there are different first-order logic languages.

\begin{definition}\label{def:first_order_language}\mcite[def. 2.1.2]{Hinman2005}
  A \term{first-order language}\fnote{As in propositional logic, a first-order language is an \hyperref[def:formal_language]{alphabet} rather than a \hyperref[def:formal_language/language]{formal language}.} \( \mscrL \) extends the language of \hyperref[subsec:propositional_logic]{propositional logic} and consists of two types of symbols.

  \begin{description}
    \item[Logical symbols]
    \hfill
    \begin{thmenum}[series=def:first_order_language]
      \thmitem{def:first_order_language/propositional} The entirety of the \hyperref[subsec:propositional_logic]{propositional logic language}.

      \thmitem{def:first_order_language/quantifiers} The set \( \op*{Quan} \) consisting of the \term[ru=квантор общости (\cite[61]{Эдельман1975})]{universal quantifier} \enquote{\( \synforall \)} and the \term[ru=квантор существования (\cite[61]{Эдельман1975})]{existential quantifier} \enquote{\( \synexists \)}.

      \thmitem{def:first_order_language/dot} A dot \enquote{\( . \)} for separating a quantifier from its formula.

      \thmitem{def:first_order_language/equality} A symbol for \term{formal equality}\fnote{Equality is sometimes omitted by logicians, but examples of first-order languages without formal equality are obscure.} \enquote{\( \syneq \)}.
    \end{thmenum}

    \item[Non-logical symbols]
    \hfill
    \begin{thmenum}[resume=def:first_order_language]
      \thmitem{def:first_order_language/fun} A possibly empty \hi{finite} set \( \op*{Fun} \) of symbols for denoting functions.

      Note that, despite using notation like \( f_n \), we regard \( f_n \) as a single symbol, and it is usually a single symbol like \( + \) or \( / \).

      Each functional symbol has an associated natural number called its \term{arity}, which we denote by \( \# f \). Functional symbols of zero arity are called \term{constants}.

      Of course, none of the functional symbols are allowed to clash with the logical symbols.

      \thmitem{def:first_order_language/pred} A possibly empty \hi{finite} set \( \op*{Pred} \) of symbols for denoting predicates.

      Predicate symbols also have an associated arity. Predicate symbols of zero arity act as \hyperref[def:propositional_syntax/prop]{propositional variables}.

      None of the predicate symbols are allowed to clash with either the functional symbols or with the logical symbols.
    \end{thmenum}
  \end{description}

  The logical symbols are common for all first-order languages. Thus, first-order languages differ by their non-logical symbols. The collection of functional and predicate symbols of a language are sometimes called its \term{signature}.
\end{definition}
\begin{comments}
  \item The dot is not itself a quantifier and is not strictly necessary --- we use it only for readability.

  \item We can avoid functional symbols altogether in favor of predicate symbols because functions can be represented via relations. This however introduces certain complications --- see \fullref{ex:replacing_functional_symbols_via_relations}.
\end{comments}

\begin{remark}\label{rem:uncountable_first_order_language}
  We require the sets of \hyperref[def:first_order_language/fun]{functional symbols} and \hyperref[def:first_order_language/fun]{predicate symbols} to be finite. This restriction is natural in several ways:
  \begin{itemize}
    \item This allows us to stick to the conventional theory of \hyperref[def:formal_language/language]{formal languages} over finite alphabets and \hyperref[def:formal_grammar]{formal grammars} over finite alphabets with finitely many rules.

    \item The above carets to the philosophical implication of languages being at most countable.

    \item The above also allows us to implement these concepts using conventional programming languages, for example in \cite{code}.
  \end{itemize}

  Unfortunately, this decision also limits us in certain definitions like \fullref{def:semimodule/theory}, where we may need uncountably many functional symbols. This particular example may be considered pathological, because more general logical frameworks provide a more elegant solution.

  Thus, we restrict ourselves to what Peter Hinman in \incite[def. 2.1.27]{Hinman2005} calls \term{finite languages}.

  An alternative is to use \fullref{thm:knaster_tarski_theorem} rather than formal grammars, however this has the downside of allowing languages which cannot be parsed.
\end{remark}

\begin{definition}\label{def:first_order_syntax}
  Similarly to the \hyperref[def:propositional_syntax]{syntax of propositional logic}, we define the \term{syntax} of a fixed \hyperref[def:first_order_language]{first-order language} \( \mscrL \).

  \begin{thmenum}
    \thmitem{def:first_order_syntax/grammar_schema} Consider the following \hyperref[def:formal_grammar/schema]{grammar schema}:
    \begin{bnf*}
      \bnfprod{connective}      {\bnftsq{\( \synvee \)} \bnfor \bnftsq{\( \synwedge \)} \bnfor \bnftsq{\( \synimplies \)} \bnfor \bnftsq{\( \syniff \)}} \\
      \bnfprod{quantifier}      {\bnftsq{\( \synforall \)} \bnfor \bnftsq{\( \synexists \)}} \\
      \bnfprod{variable}        {\bnfpn{Small Latin identifier}} \\
      \bnfprod{term}            {\bnfpn{variable} \bnfor} \\
      \bnfmore                  {f \thickspace \} \thickspace \T*{Standalone rule for each} f \in \op*{Fun} \T*{where} \# f = 0} \\
      \bnfmore                  {\underbrace{f \bnfsp \bnftsq{(} \bnfsp \overbrace{\bnfpn{term} \bnfsp \bnftsq{,} \bnfsp \bnfsk \bnfsp \bnftsq{,} \bnfsp \bnfpn{term}}^{\# f \T*{terms separated by commas}} \bnfsp \bnftsq{)}}_{\T*{Standalone rule for each} f \in \op*{Fun} \T*{where} \# f > 0}} \\
      \bnfprod{atomic formula}  {\bnftsq{\( \syntop \)} \bnfor \bnftsq{\( \synbot \)} \bnfor} \\
      \bnfmore                  {\bnftsq{(} \bnfsp \bnfpn{term} \bnfsp \bnftsq{\( \syneq \)} \bnfsp \bnfpn{term} \bnfsp \bnftsq{)} \bnfor} \\
      \bnfmore                  {\underbrace{p \bnfsp \bnftsq{(} \bnfsp \overbrace{\bnfpn{term} \bnfsp \bnftsq{,} \bnfsp \bnfsk \bnfsp \bnftsq{,} \bnfsp \bnfpn{term}}^{\# p \T*{terms separated by commas}} \bnfsp \bnftsq{)}}_{\T*{Standalone rule for each} p \in \op*{Prod}}} \\
      \bnfprod{formula}         {\bnfpn{atomic formula} \bnfor} \\
      \bnfmore                  {\bnftsq{\( \syntop \)} \bnfor \bnftsq{\( \synbot \)} \bnfor} \\
      \bnfmore                  {\bnftsq{\( \synneg \)} \bnfpn{formula} \bnfor} \\
      \bnfmore                  {\bnftsq{(} \bnfsp \bnfpn{formula} \bnfsp \bnfpn{connective} \bnfsp \bnfpn{formula} \bnfsp \bnftsq{)} \bnfor} \\
      \bnfmore                  {\bnfpn{quantifier} \bnfsp \bnfpn{variable} \bnfsp \bnftsq{.} \bnfsp \bnfpn{formula}}
    \end{bnf*}

    \thmitem{def:first_order_syntax/var} We denote by \( \op*{Var} \) the set of strings generated by the above grammar schema with starting nonterminal \( \bnfpn{variable} \).

    \thmitem{def:first_order_syntax/term} Similarly, we denote by \( \op*{Term} \) the set of strings generated with starting nonterminal \( \bnfpn{term} \).

    We implicitly associate with each first-order term a syntax tree. The grammar of terms is unambiguous as shown via \fullref{thm:propositional_formulas_are_unambiguous}, which makes it possible to perform proofs via \fullref{thm:induction_on_rooted_trees}.

    \thmitem{def:first_order_syntax/subterm} If \( \tau \) and \( \kappa \) are terms and \( \kappa \) is a \hyperref[def:formal_language/subword]{substring} of \( \tau \), we say that \( \kappa \) is a \term{subterm} of \( \tau \).

    \thmitem{def:first_order_syntax/term_variables} For each term \( \tau \), we define the set of variables occurring in the term:
    \begin{equation}\label{eq:def:first_order_syntax/term_variables}
      \op*{Var}(\tau) \coloneqq \begin{cases}
        \synx,                                                            &\tau = \synx \in \op*{Var}, \\
        \op*{Var}(\kappa_1) \cup \ldots \cup \op*{Var}(\kappa_n), &\tau = f(\kappa_1, \ldots, \kappa_n).
      \end{cases}
    \end{equation}

    \thmitem{def:first_order_syntax/closed_term} A term \( \tau \) is called a \term{closed term} if \( \op*{Var}(\tau) = \varnothing \).

    Peter Hinman in \incite[def. 2.6.10]{Hinman2005} calls them \enquote{constant terms}, but we prefer \enquote{closed term} because of similarity with closed formulas defined in \fullref{def:first_order_syntax/closed_formula}.

    \thmitem{def:first_order_syntax/atomic_formula} We denote by \( \op*{Var} \) the set of strings generated by the above grammar schema with starting nonterminal \( \bnfpn{atomic formula} \).

    \thmitem{def:first_order_syntax/formula} Similarly, we denote by \( \op*{Formula} \) the set of strings generated with starting nonterminal \( \bnfpn{formula} \).

    The grammar of first-order formulas is unambiguous as shown in \fullref{thm:first_order_terms_and_formulas_are_unambiguous}.

    \Fullref{ex:first_order_substitution} discussed the structure of some first-order formulas, while arbitrarily complicated formulas can be found in \fullref{sec:set_theory}, for example the axiom of universes \eqref{eq:def:axiom_of_universes}.

    \thmitem{def:first_order_syntax/subformula} If \( \varphi \) and \( \psi \) are formulas and \( \psi \) is a \hyperref[def:formal_language/subword]{substring} of \( \varphi \), we say that \( \psi \) is a \term{subformula} of \( \varphi \).

    \thmitem{def:first_order_syntax/formula_terms} If \( \varphi \) is a formula, if \( \tau \) is a term and if \( \tau \) is a \hyperref[def:formal_language/subword]{substring} of \( \varphi \), we say that \( \tau \) is a \term{term} of \( \varphi \).

    \thmitem{def:first_order_syntax/formula_bound_variables} For each formula \( \varphi \), we define its set of \term{bound variables} as
    \begin{equation*}
      \op*{Bound}(\varphi) \coloneqq \begin{cases}
        \varnothing,                                        &\varphi \T{is atomic,} \\
        \op*{Bound}(\psi),                               &\varphi = \synneg \psi, \\
        \op*{Bound}(\psi_1) \cup \op*{Bound}(\psi_2), &\varphi = \psi_1 \syncirc \psi_2, {\syncirc} \in \op*{Conn}, \\
        \op*{Bound}(\psi) \cup \set{ \synx },              &\varphi = \quantifier{q}{\synx} \psi, q \in \op*{Quan}.
      \end{cases}
    \end{equation*}

    \thmitem{def:first_order_syntax/formula_free_variables} We also define the set of \term{free variables} as
    \begin{equation*}
      \op*{Free}(\varphi) \coloneqq \begin{cases}
        \varnothing,                                                &\varphi \in \set{ \syntop, \synbot }, \\
        \op*{Var}(\tau_1) \cup \ldots \cup \op*{Var}(\tau_n), &\varphi = p(\tau_1, \ldots, \tau_n), \\
        \op*{Var}(\tau_1) \cup \op*{Var}(\tau_2),             &\varphi = \tau_1 \syneq \tau_2, \\
        \op*{Free}(\psi),                                        &\varphi = \synneg \psi, \\
        \op*{Free}(\psi_1) \cup \op*{Free}(\psi_2),           &\varphi = \psi_1 \syncirc \psi_2, {\syncirc} \in \op*{Conn}, \\
        \op*{Free}(\psi) \setminus \set{ \synx },                  &\varphi = \quantifier{q}{\synx} \psi, q \in \op*{Quan}
      \end{cases}
    \end{equation*}

    \thmitem{def:first_order_syntax/closed_formula}\mcite[def. 2.2.7]{Hinman2005} A formula \( \varphi \) is called a \term{closed formula}.

    Peter Hinman in \incite[def. 2.2.7]{Hinman2005} calls them \term{sentences}. This contrasts with propositional logic, where all formulas are called sentences --- see \fullref{def:propositional_syntax/formula}. We prefer \enquote{closed formula} because of similarity with closed terms defined in \fullref{def:first_order_syntax/closed_term}.

    \thmitem{def:first_order_syntax/formula_variables} Finally, the set of all variables of a formula \( \varphi \) is
    \begin{equation*}
      \op*{Var}(\varphi) \coloneqq \op*{Free}(\varphi) \cup \op*{Bound}(\varphi).
    \end{equation*}
  \end{thmenum}
\end{definition}
\begin{comments}
  \item Certain important theorems like \fullref{thm:semantic_deduction_theorem} and \fullref{thm:syntactic_deduction_theorem} require some formulas to be closed. In general, wherever we need closed formulas, we will rely on implicit quantification as mentioned in \fullref{rem:mathematical_logic_conventions/quantification}.
\end{comments}

\begin{proposition}\label{thm:first_order_terms_and_formulas_are_unambiguous}
  The grammars of \hyperref[def:first_order_syntax/term]{first-order terms} and of \hyperref[def:first_order_syntax/formula]{first-order formulas} are \hyperref[def:grammar_ambiguity]{unambiguous}.
\end{proposition}
\begin{proof}
  The proof is more complicated, but similar to \fullref{thm:propositional_formulas_are_unambiguous}.
\end{proof}

\begin{remark}\label{rem:first_order_formula_conventions}
  In order to simplify exposition, we use several conventions. As in the case of \fullref{rem:propositional_formula_parentheses}, both of these conventions exist only in the metalanguage and the formulas themselves are assumed to have the former form within the object language.
  \begin{thmenum}
    \thmitem{rem:first_order_formula_conventions/parentheses} We use the parentheses conventions from \fullref{rem:propositional_formula_parentheses}.

    \thmitem{rem:first_order_formula_conventions/infix} Binary functional symbols are often written using \term{infix notation}, i.e.
    \begin{equation*}
      \synz \syneq \synx + \syny
    \end{equation*}
    rather than the \term{prefix notation}
    \begin{equation*}
      \synz \syneq +(\synx, \syny).
    \end{equation*}

    This also applies to predicates --- we write \( \synx \sim \syny \) rather than \( \sim(\synx, \syny) \).

    \thmitem{rem:first_order_formula_conventions/negation} Negation of an infix binary predicate symbol \( \sim \) can be written more simply as
    \begin{equation*}
      \synx \not\sim \syny
    \end{equation*}
    rather than
    \begin{equation*}
      \synneg(\synx \sim \syny).
    \end{equation*}

    \thmitem{rem:first_order_formula_conventions/relativization}\mcite[def. 2.6.24]{Hinman2005} For any predicate \( p \), to further shorten notation, we write
    \begin{equation*}
      \qforall {p(\synx, \cdots)} \varphi
    \end{equation*}
    as a shorthand for
    \begin{equation*}
      \mathbf{\qforall \synx (p(\synx, \cdots) \synimplies \varphi)}
    \end{equation*}
    \begin{equation*}
      \mathbf{\qforall x (p(x, \cdots) \synimplies \varphi)}
    \end{equation*}
    \begin{equation*}
      \qforall \dot x (\dot p(\dot x, \cdots) \synimplies \varphi)
    \end{equation*}
    and
    \begin{equation*}
      \qexists {p(\synx, \cdots)} \varphi
    \end{equation*}
    as a shorthand for
    \begin{equation*}
      \qexists \synx (p(\synx, \cdots) \synwedge \varphi).
    \end{equation*}

    \Fullref{thm:relativized_first_order_quantifiers_are_dual} justifies this duality.

    This is called \term{relativization} of the quantifier and is useful when working with heterogeneous objects or even in \hyperref[sec:set_theory]{set theory}.

    \thmitem{rem:first_order_formula_conventions/exists_unique}\mcite[177]{Hinman2005} We sometimes want to specify not only existence, but also uniqueness. This is the case in \eqref{eq:def:zfc/choice}, for example. It is conventional to write
    \begin{equation*}
      \qExists \synx \varphi
    \end{equation*}
    as a shorthand for
    \begin{equation*}
      \qexists \synx \parens[\Big]{ \varphi \synwedge \parens[\Big]{ \qforall \syny \varphi[\synx \mapsto \syny] \synimplies (\synx \syneq \syny) } },
    \end{equation*}
    where \( \varphi[\synx \mapsto \syny] \) is substitution defined in \fullref{def:first_order_substitution/term_in_formula}.

    \thmitem{rem:first_order_formula_conventions/necessary_signature} We only add to the language itself the functional and predicate symbols that are necessary for our desired axioms --- see \fullref{def:first_order_theory}. We can define additional functions and predicates in terms of these, but we avoid using them as much as possible when writing formulas in the object language. For example, we avoid adding the functional symbols \hyperref[thm:well_ordered_order_type_existence]{\( \ord(A) \)} and \hyperref[def:cardinal]{\( \card(A) \)} or even \hyperref[def:basic_set_operations/union]{\( \cup \)} and \hyperref[def:basic_set_operations/intersection]{\( \cap \)} to \hyperref[def:zfc]{\logic{ZFC}}.

    If needed, we can consider these new functions and predicates to be abbreviations for more verbose terms and formulas as described in \fullref{con:predicate_formula}.
  \end{thmenum}
\end{remark}

\begin{definition}\label{def:first_order_structure}\mcite[def. 2.1.15]{Hinman2005}
  Fix a first-order logic language \( \mscrL \). A \term{structure} for \( \mscrL \) is a pair \( \mscrX = (X, I) \), where
  \begin{thmenum}
    \thmitem{def:first_order_structure/set}\mcite[def. 2.1.15(i)]{Hinman2005} \( X \) is a nonempty\fnote{See \fullref{rem:empty_first_order_structures} regarding special cases where we allow \( X \) to be empty.} set called the \term{domain} or \term{universe} of the structure \( \mscrX \).

    \thmitem{def:first_order_structure/interpretation} The \term{interpretation} \( I \) of the structure \( \mscrX \) is a \hyperref[def:function]{function} that is defined on the signature of \( \mscrL \) and satisfies the following conditions:
    \begin{thmenum}
      \thmitem{def:first_order_structure/interpretation/function}\mcite[def. 2.1.15(ii)]{Hinman2005} For every \( n \)-ary function symbol \( f \), its interpretation is a \hyperref[def:function]{function} with signature \( I(f): X^n \to X \).

      \thmitem{def:first_order_structure/interpretation/predicate}\mcite[def. 2.1.15(iii)]{Hinman2005} For every \( n \)-ary predicate \( p \), its interpretation is an n-ary \hyperref[def:boolean_function]{Boolean-valued function} with signature \( I(p): X^n \to \set{ T, F } \). A tuple \( (x_1, \ldots, x_n) \) satisfies \( p \) if \( p(x_1, \ldots, x_n) = T \).
    \end{thmenum}
  \end{thmenum}
\end{definition}
\begin{comments}
  \item Some authors like Peter Hinman in \incite[def. 2.1.15(iii)]{Hinman2005} define \( I(p) \) to be a \hyperref[def:relation]{relation} \( I(p) \subseteq X^n \), however it is more convenient for us to work with Boolean-valued functions. The two approaches are equivalent as explained in \fullref{rem:boolean_valued_functions_and_predicates}.
  \item Unlike in the rest of this monograph, when dealing with first-order structures directly, it is important to distinguish between the structure \( \mscrX \) as a pair and its domain \( X \) as a set. See \fullref{rem:first_order_model_notation}.
\end{comments}

\begin{remark}\label{rem:empty_first_order_structures}
   If we allow the domain of a structure to be empty, we would have to reformulate a lot of important theorems (e.g. our proof of \fullref{thm:renaming_assignment_compatibility/formulas}), which would complicate compatibility between semantics and \hyperref[def:deduction_system]{deduction systems}.

   See \fullref{thm:intersection_substructure} and \fullref{thm:substructures_form_complete_lattice/bottom} for contexts where empty sets are justified as domains of first-order structures. We also use this in \fullref{rem:propositional_logic_as_first_order_logic}, although in the latter case the domain itself is irrelevant.
\end{remark}

\begin{definition}\label{def:first_order_valuation}
  Fix a structure \( \mscrX = (X, I) \) for a first-order logic language \( \mscrL \).

  \begin{thmenum}
    \thmitem{def:first_order_valuation/variable_assignment}\mcite[def. 2.1.16]{Hinman2005} A \term[ru=оценка (\cite[77]{ШеньВерещагин2017Языки})]{variable assignment} for the variables of \( \mscrL \) is any function \( v: \op*{Var} \to X \).

    These are loosely similar to \hyperref[def:propositional_valuation/interpretation]{propositional interpretations}.

    \thmitem{def:first_order_valuation/modified_assignment}\mcite[92]{Hinman2005} For every variable \( \synx \) and every domain element \( x \in X \) we also define the \term{modified assignment} at \( \synx \) with \( x \):
    \begin{equation*}
      v_{\synx \mapsto x}(\synz) \coloneqq \begin{cases}
        x,        &\synz = \synx, \\
        v(\synz), &\synz \neq \synx.
      \end{cases}
    \end{equation*}

    We can also modify the value at \( \synx \) with another variable, e.g.
    \begin{equation*}
      v_{\synx \mapsto \syny}(\synz) \coloneqq \begin{cases}
        v(\syny),  &\synz = \synx, \\
        v(\synz), &\synz \neq \synx.
      \end{cases}
    \end{equation*}

    Inductively,
    \begin{equation*}
      v_{\synx_1 \mapsto x_1, \ldots, \synx_n \mapsto x_n}(\syny) \coloneqq ((\ldots(v_{\synx_1 \mapsto x_1})\ldots)_{\synx_n \mapsto x_n})(\syny).
    \end{equation*}

    \thmitem{def:first_order_valuation/term_valuation}\mcite[def. 2.1.17]{Hinman2005} The \term{valuation} of a term \( \tau \) is a value in the domain \( X \) given by
    \begin{equation}\label{eq:def:first_order_valuation/term_valuation}
      \Bracks{\tau}_v \coloneqq \begin{cases}
        v(\synx),                                                 &\tau = \synx \in \op*{Var}, \\
        I(f)(\Bracks{\kappa_1}_v, \ldots, \Bracks{\kappa_n}_v), &\tau = f(\kappa_1, \ldots, \kappa_n).
      \end{cases}
    \end{equation}

    \thmitem{def:propositional_valuation/term_valuation_function} Analogously to how we defined Boolean functions for propositional formulas in \fullref{def:propositional_valuation/valuation_function}, if \( \op*{Var}(\tau) \subseteq \set{ \synx_1, \ldots, \synx_n } \), the valuation \( \Bracks{\tau}_v \) only depends on the particular values \( v(\synx_1), \ldots, v(\synx_n) \). Hence, if the variables are clear from the context, we obtain a function
    \begin{equation*}
      \begin{aligned}
        &\Bracks{\tau}_\mscrX: X^n \to X, \\
        &\Bracks{\tau}_\mscrX(x_1, \ldots, x_n) \coloneqq \Bracks{\tau}_v,
      \end{aligned}
    \end{equation*}
    where \( v \) is any variable assignment such that \( v(\synx_k) = x_k \) for \( k = 1, \ldots, n \).

    \thmitem{def:first_order_valuation/formula_valuation}\mcite[def. 2.1.19, def. 2.1.22]{Hinman2005} We extend the classical propositional valuations from \fullref{def:propositional_valuation}. The (classical) \term{valuation} of a formula \( \varphi \) is a \hyperref[con:boolean_value]{Boolean value} given by
    \begin{equation}\label{eq:def:first_order_valuation/formula_valuation}
      \Bracks{\varphi}_v \coloneqq \begin{cases}
        T,                                                                        &\varphi = \syntop, \\
        F,                                                                        &\varphi = \synbot, \\
        \Bracks{\tau_1}_v = \Bracks{\tau_2}_v,                                    &\varphi = \tau_1 \syneq \tau_2, \\
        I(p)(\Bracks{\tau_1}_v, \ldots, \Bracks{\tau_n}_v),                       &\varphi = p(\tau_1, \ldots, \tau_n), \\
        \oline{\Bracks{\psi}_v},                                               &\varphi = \synneg \psi, \\
        \Bracks{\psi_1}_v \relcirc \Bracks{\psi_2}_v,                    &\varphi = \psi_1 \syncirc \psi_2, {\syncirc} \in \op*{Conn}, \\
        \bigwedge\set[\Big]{ \Bracks{\psi}_{v_{\synx \mapsto x}} \given* x \in X }, &\varphi = \qforall \synx \psi, \\
        \bigvee\set[\Big]{ \Bracks{\psi}_{v_{\synx \mapsto x}} \given* x \in X },   &\varphi = \qexists \synx \psi.
      \end{cases}
    \end{equation}

    \thmitem{def:propositional_valuation/formula_valuation_function} We also define a function corresponding to a first-order formula, but we are only interested in the \hyperref[def:first_order_syntax/formula_free_variables]{free variables} rather than \hyperref[def:first_order_syntax/formula_variables]{all variables}:
    \begin{equation*}
      \begin{aligned}
        &\Bracks{\varphi}_\mscrX: X^n \to \set{ T, F }, \\
        &\Bracks{\varphi}_\mscrX(\synx_1, \ldots, \synx_n) \coloneqq \Bracks{\varphi}_v,
      \end{aligned}
    \end{equation*}
    where \( v \) is any variable assignment such that \( v(\synx_k) = x_k \) for \( k = 1, \ldots, n \).
  \end{thmenum}
\end{definition}
\begin{comments}
  \item The rules for evaluating constants, negations and connectives are a direct extension of the rules for propositional logic from \fullref{def:propositional_valuation/formula_valuation}.

  \item It is important that the domain is nonempty because \( \bigwedge\varnothing = T \), which directly contradicts our intent of defining \( \synexists \) as a quantifier for existence.

  \item Except for semantics of quantification, modified assignments are also used in other places like \fullref{thm:renaming_assignment_compatibility}.
\end{comments}

\begin{remark}\label{rem:propositional_logic_as_first_order_logic}
  We will show in details how, with certain limitations on the number of variables, propositional logic can be embedded into first-order logic.

  Fix a set \( \Gamma \) of \hyperref[def:propositional_syntax/formula]{propositional formulas}. Suppose that there is only a finite set of variables in \( \Gamma \); otherwise we would have to generalize first-order syntax as discussed in \fullref{rem:uncountable_first_order_language}. Consider the \hyperref[def:first_order_language]{first-order language} \( \mscrL \) with no functional symbols and a nullary predicate symbol for every propositional variable in \( \Gamma \). Denote by \( \widehat{P} \) the predicate symbol corresponding to the variable \( P \).

  Given the propositional formula \( \varphi \) from \( \Gamma \), let \( \widehat \varphi \) be the formula over \( \mscrL \) in which the variables are replaced with the predicate symbols from \( \mscrL \). Then \( \widehat \varphi \) is closed, and thus its interpretation does not depend on the variable assignment.

  Given an arbitrary propositional interpretation \( I \), let \( \widehat{I} \) be a first-order interpretation over the empty domain\fnote{See \fullref{rem:empty_first_order_structures} regarding empty domains.} \( \varnothing \) such that \( \widehat{I}(\widehat{P}) = I(P) \). Let \( \mscrX = (\varnothing, \widehat{I}) \). Then
  \begin{equation*}
    \Bracks{\widehat{P}}_\mscrX = \Bracks{\widehat{P}}_I.
  \end{equation*}
\end{remark}

\begin{definition}\label{def:first_order_equation}\mimprovised
  A \term{first-order equation} over some fixed \hyperref[def:first_order_language]{language} with \term{left side} the \hyperref[def:first_order_syntax/term]{term} \( \tau \) and \term{right side} \( \sigma \) is the \hyperref[def:first_order_syntax/formula]{formula} \( (\tau \syneq \sigma) \).

  Suppose that the \hyperref[def:first_order_syntax/formula_free_variables]{free variables} (which are actually all the variables) of this formula are among \( \synx_1, \ldots, \synx_n \). Then our goal is to find values \( x_1, \ldots, x_n \) in the domain of some \hyperref[def:first_order_structure]{structure} \( \mscrX \) such that
  \begin{equation*}
    \Bracks{\tau}_\mscrX(x_1, \ldots, x_n) = \Bracks{\sigma}_\mscrX(x_1, \ldots, x_n).
  \end{equation*}

  We call any such tuple \( (x_1, \ldots, x_n) \) in any structure \( \mscrX \) a \term{solution} of the equation.
\end{definition}

\begin{example}\label{ex:equations}
  A remarkable portion of mathematics concerns the study of different types of equations, and many of them are more general than \hyperref[def:first_order_equation]{equations in first-order logic}. The reason for this interest is that equations provide a simple way to specify rich semantic structure using simple syntactic objects.

  \begin{itemize}
    \item Matrix theory can be regarded as the study of linear equations. See \fullref{subsec:matrices_over_rings} and \fullref{subsec:matrices_over_fields}.
    \item Differential equations is aptly named since it studies equations in functional spaces concerning functions and their derivatives.
    \item Roots of generalized derivatives are studied in optimization. See \fullref{subsec:nonsmooth_derivatives}.
    \item Fixed points of functions are studied in different branches of mathematics. See \fullref{thm:banach_fixed_point_theorem} or \fullref{thm:knaster_tarski_theorem}.
    \item Affine varieties, which are sets of roots of polynomials, are studied in algebraic geometry. See \fullref{subsec:quadratic_plane_curves}.
  \end{itemize}
\end{example}

\begin{remark}\label{rem:first_order_entailment_without_closed_formulas}
  \todo{Commutativity counterexample}
\end{remark}

\begin{definition}\label{def:first_order_semantics}
  Fix a first-order logic language \( \mscrL \). We introduce notions analogous to \hyperref[def:propositional_entailment]{propositional semantics}:
  \begin{thmenum}
    \thmitem{def:first_order_semantics/assignment_entailment}\mcite[def. 2.1.23(i)]{Hinman2005} We say that the set of \hyperref[def:first_order_syntax/formula]{first-order formulas} \enquote{\( \Gamma \) \term{entails} \( \psi \) in the \hyperref[def:first_order_structure]{structure} \( \mscrX = (X, I) \) under the \hyperref[def:first_order_valuation/variable_assignment]{variable assignment} \( v \)} if, whenever \( \Bracks{\varphi}_v = T \) for all \( \varphi \in \Gamma \), it follows that \( \Bracks{\psi}_v = T \).

    \thmitem{def:first_order_semantics/entailment}\mcite[def. 2.2.1(iv)]{Hinman2005} If \( \Gamma \) entails \( \psi \) under every variable assignment \( v \) in every structure \( \mscrX \), we simply say that \enquote{\( \Gamma \) entails \( \psi \)} and write \( \Gamma \vDash \psi \).

    As in the case of propositional formulas, we prefer the notation \( \gamma_1, \ldots, \gamma_n \vDash \psi \) to \( \set{ \gamma_1, \ldots, \gamma_n } \vDash \psi \).

    If \( \Gamma \) does not entail \( \varphi \) under at least one variable assignment, we write \( \Gamma \not\vDash \varphi \).

    As in the case of propositional formulas, we prefer the notation \( \gamma_1, \ldots, \gamma_n \vDash \psi \) to \( \set{ \gamma_1, \ldots, \gamma_n } \vDash \psi \).

    \thmitem{def:first_order_semantics/equivalence}\mcite[def. 2.2.1(ii)]{Hinman2005} As in the case with \hyperref[def:semantic_equivalence]{propositional semantical equivalence}, we say that \( \varphi \) and \( \psi \) are \term{semantically equivalent} and write \( \varphi \gleichstark \psi \) if both \( \varphi \vDash \psi \) and \( \psi \vDash \varphi \).

    \thmitem{def:first_order_semantics/validity}\mcite[def. 2.2.1(i)]{Hinman2005} We say that the formula \( \varphi \) is \term{universally valid} if \( \vDash \varphi \).
  \end{thmenum}
\end{definition}
\begin{comments}
  \item It is actually important that we haven't defined entailment within a structure, because it can only be well-defined for \hyperref[def:first_order_syntax/closed_formula]{closed formulas} for reasons outlined in \fullref{rem:first_order_satisfiability_bivalence}.

  \item The concept of universal validity is more general than \hyperref[def:propositional_tautology]{propositional tautologies} in the sense that we define validity for arbitrary formulas that may or may not be closed.

  \item \Fullref{thm:intuitionistic_equivalences} hold for first-order entailment as well.
\end{comments}

\begin{definition}\label{def:first_order_substitution}
  As in \hyperref[subsec:propositional_logic]{propositional logic}, we sometimes want to perform substitution, however we have different types of syntactic objects (terms and formulas) which have different substitution rules. The notion of free and bound variables further complicates us --- see for example the problems outlined in \fullref{rem:first_order_substitution_renaming_justification}. It is not only difficult, but also it is of no practical use to define substitution of a first-order subformula inside another formula as it is done in \fullref{def:propositional_substitution}. Instead, we concern ourselves with substituting variables --- propositional variables with first-order formulas and first-order variables with first-order terms. Furthermore, since this does not complicate us, we allow substituting arbitrary terms rather than only first-order variables.

  While substituting a propositional variable is the syntactic analog to applying \hyperref[def:boolean_function]{Boolean functions} to different variables or propositional constants, substituting a first-order variable can express applying \hyperref[def:function]{arbitrary functions} to different first-order variables or arbitrary constants. For example, in a suitable language, we can apply \( \log(x) \) to the constant \( e \) by substituting \( x \) with \( e \) to obtain the closed term \( \log(e) \).

  Where applicable, \term{simultaneous substitution} is defined via the same trick as in \fullref{def:propositional_substitution}.

  \begin{thmenum}
    \thmitem{def:first_order_substitution/propositional} Let \( \varphi \) be a \hyperref[def:propositional_syntax/formula]{propositional formula} with variables \( \op*{Var}(\varphi) = \set{ P_1, \ldots, P_n } \). For brevity, denote \( V \coloneqq \op*{Var}(\varphi) \). Let \( \Theta = \set{ \theta_1, \ldots, \theta_n } \) be a set of \hyperref[def:first_order_syntax/formula]{first-order formulas}.

    It does not make sense to replace a single propositional variable with a single formula. Furthermore, a first-order formula \( \theta_k \) cannot possibly contain any of the propositional variables \( P_1, \ldots, P_n \). This allows us to introduce a simplification of the simultaneous substitution based on \eqref{eq:def:propositional_substitution/single} as follows:
    \begin{equation}\label{eq:def:first_order_substitution/propositional}
      \varphi[V \mapsto \Theta] \coloneqq \begin{cases}
        \varphi,                                                    &\varphi \in \set{ \syntop, \synbot } \\
        \theta_k,                                                   &\varphi = \theta_k \T{for some} k = 1, \ldots, n \\
        \synneg \psi[V \mapsto \Theta],                                &\varphi = \synneg \psi \\
        \psi_1[V \mapsto \Theta] \syncirc \psi_2[V \mapsto \Theta], &\varphi = \psi_1 \syncirc \psi_2, {\syncirc} \in \op*{Conn}.
      \end{cases}
    \end{equation}

    As in \fullref{def:propositional_substitution}, it is not strictly necessary for any of the variables to belong to \( \op*{Var}(\varphi) \).

    \thmitem{def:first_order_substitution/term_in_term}\mcite[def. 2.2.19]{Hinman2005} We define the substitution of the \hyperref[def:first_order_syntax/term]{first-order term} \( \kappa \) with \( \mu \) in the term \( \tau \) as follows:
    \begin{equation}\label{eq:def:first_order_substitution/term_in_term}
      \tau[\kappa \mapsto \mu] \coloneqq \begin{cases}
        \mu,                                                               &\tau = \kappa, \\
        \tau,                                                              &\tau \neq \kappa \T{and} \tau \in \op*{Var}, \\
        f(\rho_1[\kappa \mapsto \mu], \ldots, \rho_n[\kappa \mapsto \mu]), &\tau \neq \kappa \T{and} \tau = f(\rho_1, \ldots, \rho_n).
      \end{cases}
    \end{equation}

    It is not strictly necessary for \( \kappa \) to be a \hyperref[def:first_order_syntax/subterm]{subterm} of \( \tau \).

    \thmitem{def:first_order_substitution/term_in_formula}\mimprovised This last case is more complicated. We define the substitution of the term \( \kappa \) with the term \( \mu \) in the first-order formula \( \varphi \) as follows:
    \begin{equation}\label{eq:def:first_order_substitution/term_in_formula}
      \varphi[\kappa \mapsto \mu] \coloneqq \begin{cases}
        \varphi,                                                           &\varphi \in \set{ \syntop, \synbot }, \\
        p(\tau_1[\kappa \mapsto \mu], \ldots, \tau_n[\kappa \mapsto \mu]), &\varphi = p(\tau_1, \ldots, \tau_n), \\
        \tau_1[\kappa \mapsto \mu] \syneq \tau_2[\kappa \mapsto \mu],      &\varphi = \tau_1 \syneq \tau_2, \\
        \synneg \psi[\kappa \mapsto \mu],                                     &\varphi = \synneg \psi, \\
        \psi_1[\kappa \mapsto \mu] \syncirc \psi_2[\synx \mapsto \mu],       &\varphi = \psi_1 \syncirc \psi_2, {\syncirc} \in \op*{Conn}, \\
        (\dagger),                                                         &\varphi = \quantifier Q {\synx} \psi, Q \in \op*{Quan},
      \end{cases}
    \end{equation}
    where
    \begin{subequations}
      \begin{empheq}[left=(\dagger) \coloneqq \empheqlbrace]{align}
        &\varphi,                                                                        && \synx \in \op*{Var}(\kappa), \label{eq:def:first_order_substitution/term_in_formula/quantifiers/trivial} \\
        &\quantifier Q {\synx} \parens[\Big]{\psi[\kappa \mapsto \mu]},                    && \synx \not\in \op*{Var}(\kappa) \cup \op*{Var}(\mu), \label{eq:def:first_order_substitution/term_in_formula/quantifiers/direct} \\
        &\quantifier Q {\syny} \parens[\Big]{\psi[\synx \mapsto \syny][\kappa \mapsto \mu]}, && \synx \not\in \op*{Var}(\kappa) \T{and} \synx \in \op*{Var}(\mu) \T{and} &\label{eq:def:first_order_substitution/term_in_formula/quantifiers/renaming} \\
                                                                                        &&& \syny \not\in \op*{Var}(\kappa) \cup \op*{Var}(\mu) \cup \op*{Free}(\psi). \nonumber
      \end{empheq}
    \end{subequations}
  \end{thmenum}
\end{definition}
\begin{comments}
  \item In \eqref{eq:def:first_order_substitution/term_in_formula/quantifiers/renaming} we chose a fresh variable \( \syny \). As discussed in \fullref{def:variable_identifier}, we can use the well-ordering on the set \( \op*{Var} \) of variables and deterministically choose \( \syny \) to be the smallest variable not present in \( \varphi \). This rule is called \term{renaming of the bound variables} \( \synx \) to \( \syny \) and is done to mitigate capturing as described in \fullref{rem:first_order_substitution_renaming_justification/capturing}.

  \item We could avoid the rule for renaming, however renaming both free and bound variables is natural and is often done in practice (i.e. outside formal logic). For example, consider the \hyperref[def:peano_arithmetic]{Peano arithmetic} formula \enquote{there exists \( n \) such that \( nm \) is even}. Note that the bound variable \( n \) is renamed to \( k \) and the free variable \( m \) to \( n \) in the larger formula \enquote{for every \( n \) there exists \( k \) such that \( kn \) is even}.

  \item The rule \eqref{eq:def:first_order_substitution/term_in_formula/quantifiers/trivial} may seem redundant, but when doing inductive proofs (e.g. our proof of \fullref{thm:renaming_assignment_compatibility}), we usually need to separately consider the cases where \( \synx \in \op*{Var}(\kappa) \) and \( \synx \not\in \op*{Var}(\kappa) \setminus \op*{Var}(\mu) \) and the rule \eqref{eq:def:first_order_substitution/term_in_formula/quantifiers/direct} being trivial simplifies the proofs.

  \item See \fullref{rem:first_order_substitution_parentheses} regarding the additional parentheses in \eqref{eq:def:first_order_substitution/term_in_formula/quantifiers/renaming}.

  \item See \fullref{ex:first_order_substitution} for examples of applying the different quantifier rules.

  \item We use the shorthand convention from \fullref{rem:uniform_substitution_notation}.

  \item These algorithms can be found as \identifier{fol.substitution.substitute_in_term} and \identifier{fol.substitution.substitute_in_formula} in \cite{code}.
\end{comments}

\begin{remark}\label{rem:first_order_substitution_renaming_justification}
  The renaming rule \eqref{eq:def:first_order_substitution/term_in_formula/quantifiers/renaming} is designed to mitigate the following two problems when compared to \eqref{eq:def:first_order_substitution/term_in_formula/quantifiers/direct}:

  \begin{thmenum}
    \thmitem{rem:first_order_substitution_renaming_justification/capturing} Renaming mitigates \enquote{capturing} free variables as in
    \begin{equation*}
      \parens[\Big]{ \qforall \syny p(\synx, \syny) }[\synx \mapsto \syny] = \qforall \syny p(\syny, \syny)
    \end{equation*}
    by instead producing, up to a choice of fresh variables, the formula
    \begin{equation*}
      \parens[\Big]{ \qforall \syny p(\synx, \syny) }[\synx \mapsto \syny] = \qforall \synz p(\syny, \synz).
    \end{equation*}

    \thmitem{rem:first_order_substitution_renaming_justification/colliding} Renaming mitigates \enquote{colliding} multiple bound variables as in
    \begin{equation*}
      \parens[\Big]{ \qforall \synx \qforall \syny p(\synx, \syny) }[\synx \mapsto \syny] = \qforall \synx \qforall \syny p(\syny, \syny)
    \end{equation*}
    by instead producing, up to a choice of fresh variables, the formula
    \begin{equation*}
      \parens[\Big]{ \qforall \synx \qforall \syny p(\synx, \syny) }[\synx \mapsto \syny] = \qforall \synz \qforall \sigma p(\synz, \sigma).
    \end{equation*}
  \end{thmenum}
\end{remark}

\begin{remark}\label{rem:first_order_substitution_parentheses}
  When performing \hyperref[def:first_order_substitution]{substitution}, it is sometimes convenient to add additional parentheses to avoid ambiguity. For example, while parentheses around quantifier expressions are not necessary by the syntax of first-order logic, adding such parentheses helps avoid the ambiguity in
  \begin{equation*}
    \qforall \synx p(\synx, \syny) [\syny \mapsto \synz].
  \end{equation*}

  Instead, we either write
  \begin{equation*}
    \parens[\Big]{ \qforall \synx p(\synx, \syny) } [\syny \mapsto \synz]
  \end{equation*}
  or
  \begin{equation*}
    \qforall \synx \parens[\Big]{ p(\synx, \syny)[\syny \mapsto \synz] }.
  \end{equation*}

  This convention is only part of the metasyntax and the parentheses are not part of the syntax of the formulas themselves.
\end{remark}

\begin{example}\label{ex:first_order_substitution}
  The following term substitutions should justify the distinct cases in \eqref{eq:def:first_order_substitution/term_in_formula}:
  \begin{thmenum}
    \thmitem{ex:first_order_substitution/1} The trivial case without actual substitution:
    \begin{equation*}
      \parens[\Big]{\qforall \synx p(\synx, \syny)}[\synx \mapsto \syny]
      \reloset {\eqref{eq:def:first_order_substitution/term_in_formula/quantifiers/trivial}} =
      \qforall \synx p(\synx, \syny).
    \end{equation*}

    \Fullref{ex:first_order_substitution/5} demonstrates that this does not work for nested substitution.

    \thmitem{ex:first_order_substitution/2} A simple substitution without renaming:
    \begin{align*}
      &\phantom{{}={}}
      \parens[\Big]{\qforall \synx p(\synx, \syny)}[\syny \mapsto \synz]
      \reloset {\eqref{eq:def:first_order_substitution/term_in_formula/quantifiers/direct}} = \\ &=
      \qforall \synx \parens[\Big]{p(\synx, \syny)[\syny \mapsto \synz]}
      \reloset {\eqref{eq:def:first_order_substitution/term_in_formula}} = \\ &=
      \qforall \synx p(\synx, \synz).
    \end{align*}

    \thmitem{ex:first_order_substitution/3} A simple renaming without actual substitution:
    \begin{align*}
      &\phantom{{}={}}
      \parens[\Big]{\qforall \synx p(\synx, \syny)}[\syny \mapsto \synx]
      \reloset {\eqref{eq:def:first_order_substitution/term_in_formula/quantifiers/renaming}} = \\ &=
      \qforall \synz \parens[\Big]{p(\synx, \syny)[\synx \mapsto \synz][\syny \mapsto \synx]}
      \reloset {\eqref{eq:def:first_order_substitution/term_in_formula}} = \\ &=
      \qforall \synz p(\synz, \synx).
    \end{align*}

    \thmitem{ex:first_order_substitution/4} \Fullref{ex:first_order_substitution/3}, but with \( \mu \) in \eqref{eq:def:first_order_substitution/term_in_formula} containing \( \synx \) indirectly:
    \begin{align*}
      &\phantom{{}={}}
      \parens[\Big]{\qforall \synx p(\synx, \syny)}[\syny \mapsto f(\synx)]
      \reloset {\eqref{eq:def:first_order_substitution/term_in_formula/quantifiers/renaming}} = \\ &=
      \qforall \synz \parens[\Big]{p(\synx, \syny)[\synx \mapsto \synz][\syny \mapsto f(\synx)]}
      \reloset {\eqref{eq:def:first_order_substitution/term_in_formula}} = \\ &=
      \qforall \synz p(\synz, f(\synx)).
    \end{align*}

    \thmitem{ex:first_order_substitution/5} Only renaming with multiple quantifiers which shows the limitations of \eqref{eq:def:first_order_substitution/term_in_formula/quantifiers/trivial} as a way to avoid unnecessary renaming:
    \begin{align*}
      &\phantom{{}={}}
      \parens[\Big]{\qforall \syny \qexists \synx p(\synx, \syny)}[\synx \mapsto \syny]
      \reloset {\eqref{eq:def:first_order_substitution/term_in_formula/quantifiers/renaming}} = \\ &=
      \qforall \synz \parens*{ \parens[\Big]{ \qforall \synx p(\synx, \syny) }[\syny \mapsto \synz][\synx \mapsto \syny]}
      \reloset {\eqref{eq:def:first_order_substitution/term_in_formula}} = \\ &=
      \qforall \synz \parens*{ \parens[\Big]{ \qforall \synx p(\synx, \synz) }[\synx \mapsto \syny]}
      \reloset {\eqref{eq:def:first_order_substitution/term_in_formula/quantifiers/trivial}} = \\ &=
      \qforall \synz \qforall \synx p(\synx, \synz).
    \end{align*}

    \thmitem{ex:first_order_substitution/6} Both renaming and substitution with multiple quantifiers:
    \begin{align*}
      &\phantom{{}={}}
      \parens[\Big]{ \qforall \syny (p(\synx, \syny) \synvee \qforall \synx p(\synx, \syny)) }[\synx \mapsto \syny]
      \reloset {\eqref{eq:def:first_order_substitution/term_in_formula/quantifiers/renaming}} = \\ &=
      \qforall \synz \parens*{ \parens[\Big]{ p(\synx, \syny) \synvee \qexists \synx p(\synx, \syny) }[\syny \mapsto \synz][\synx \mapsto \syny] }
      \reloset {\eqref{eq:def:first_order_substitution/term_in_formula}} = \\ &=
      \qforall \synz \parens*{ p(\syny, \synz) \synvee \parens[\Big]{ \qexists \synx p(\synx, \syny) }[\syny \mapsto \synz][\synx \mapsto \syny] }
      \reloset {\eqref{eq:def:first_order_substitution/term_in_formula/quantifiers/direct}} = \\ &=
      \qforall \synz \parens*{ p(\syny, \synz) \synvee \parens[\Big]{ \qexists \synx p(\synx, \synz) }[\synx \mapsto \syny] }
      \reloset {\eqref{eq:def:first_order_substitution/term_in_formula/quantifiers/trivial}} = \\ &=
      \parens[\Big]{ \qforall \synz p(\syny, \synz) } \synvee \parens[\Big]{ \qexists \synx p(\synx, \synz) }.
    \end{align*}

    \thmitem{ex:first_order_substitution/7} Substitution of more general terms than variables with renaming of term's variables:
    \begin{align*}
      &\phantom{{}={}}
      \parens*{\qforall \synx p(\synx, \syny, f(\syny))}[f(\syny) \mapsto g(\syny, \synx)]
      \reloset {\eqref{eq:def:first_order_substitution/term_in_formula/quantifiers/renaming}} = \\ &=
      \qforall \synz \parens[\Big]{p(\synx, \syny, f(\syny))[\synx \mapsto \synz][f(\syny) \mapsto g(\syny, \synx)]}
      \reloset {\eqref{eq:def:first_order_substitution/term_in_formula}} = \\ &=
      \qforall \synz p(\synz, \syny, g(\syny, \synx)).
    \end{align*}
  \end{thmenum}
\end{example}

\begin{proposition}\label{thm:renaming_assignment_compatibility}
  We will show how \hyperref[rem:first_order_substitution_renaming_justification]{syntactic renaming} is compatible with a certain \enquote{semantic renaming}.

  Fix a \hyperref[def:first_order_language]{first-order language} \( \mscrL \), a \hyperref[def:first_order_structure]{structure} \( \mscrX = (X, I) \) on \( \mscrL \) and a \hyperref[def:first_order_valuation/variable_assignment]{variable assignment} \( v \) in \( \mscrX \).

  \begin{thmenum}
    \thmitem{thm:renaming_assignment_compatibility/terms} For any term \( \tau \) and any two variables \( \synx \) and \( \syny \), we have
    \begin{equation}\label{eq:thm:renaming_assignment_compatibility/terms}
      \Bracks{\tau}_{v_{\synx \mapsto \syny}}
      =
      \Bracks[\Big]{\tau[\synx \mapsto \syny]}_v.
    \end{equation}

    \thmitem{thm:renaming_assignment_compatibility/formulas} For any formula \( \varphi \), any variable \( \synx \) and any other variable \( \syny \) not in \( \op*{Var}(\varphi) \) we have
    \begin{equation}\label{eq:thm:renaming_assignment_compatibility/formulas}
      \Bracks{\varphi}_{v_{\synx \mapsto \syny}}
      =
      \Bracks[\Big]{\varphi[\synx \mapsto \syny]}_v.
    \end{equation}
  \end{thmenum}
\end{proposition}
\begin{proof}
  In both cases, we use \fullref{thm:induction_on_rooted_trees} on the definition of the substitution.

  \SubProofOf{thm:renaming_assignment_compatibility/terms}

  \begin{itemize}
    \item If \( \tau = \synx \), then
    \begin{equation*}
      \tau[\synx \mapsto \syny] = \synx[\synx \mapsto \syny] = \syny
    \end{equation*}
    and \eqref{eq:thm:renaming_assignment_compatibility/terms} follows directly.

    \item If \( \tau \) is a variable and \( \tau \neq \synx \), then
    \begin{equation*}
      \tau[\synx \mapsto \syny] = \tau
    \end{equation*}
    and \eqref{eq:thm:renaming_assignment_compatibility/terms} again holds trivially.

    \item If \( \tau = f(\kappa_1, \ldots, \kappa_n) \) and if the inductive hypothesis holds for \( \kappa_1, \ldots, \kappa_n \), then
    \begin{balign*}
      \Bracks[\Big]{ \tau[\synx \mapsto \syny] }_v
      &=
      \Bracks[\Big]{ f(\kappa_1[\synx \mapsto \syny], \ldots, \kappa_n[\synx \mapsto \syny]) }_v
      = \\ &=
      I(f) \parens[\Bigg]{ \Bracks[\Big]{ \kappa_1[\synx \mapsto \syny] }_v, \ldots, \Bracks[\Big]{ \kappa_1[\synx \mapsto \syny] }_v }
      \reloset {\T{ind.}} = \\ &=
      I(f) \parens[\Big]{ \Bracks{\kappa_1}_{v_{\synx \mapsto \syny}}, \ldots, \Bracks{\kappa_n}_{v_{\synx \mapsto \syny}} }
      = \\ &=
      \Bracks[\Big]{ f(\kappa_1, \ldots, \kappa_n) }_{\synx \mapsto \syny}
      = \\ &=
      \Bracks{\tau}_{\synx \mapsto \syny}.
    \end{balign*}
  \end{itemize}

  In all cases, \eqref{eq:thm:renaming_assignment_compatibility/terms} holds.

  \SubProofOf{thm:renaming_assignment_compatibility/formulas}
  \hfill
  \begin{itemize}
    \item If \( \varphi \in \set{ \syntop, \synbot } \), then \( \varphi \) has no subterms and thus \eqref{eq:thm:renaming_assignment_compatibility/formulas} holds vacuously.

    \item If \( \varphi = p(\tau_1, \ldots, \tau_n) \), then since \eqref{eq:thm:renaming_assignment_compatibility/terms} holds for all \( \tau_k \), we have
    \begin{equation*}
      \Bracks[\Big]{ \tau_k[\synx \mapsto \syny] }_v = \Bracks{\tau_k}_{v_{\synx \mapsto \syny}}
    \end{equation*}
    and thus
    \begin{equation*}
      I(p)\parens[\Big]{ \Bracks[\Big]{ \tau_1[\synx \mapsto \syny] }_v, \ldots, \Bracks[\Big]{ \tau_1[\synx \mapsto \syny] }_v }
      \reloset {\T{ind.}} =
      I(p)\parens[\Big]{ \Bracks{\tau_1}_{v_{\synx \mapsto \syny}}, \ldots, \Bracks{\tau_n}_{v_{\synx \mapsto \syny}} }.
    \end{equation*}

    Therefore,
    \begin{balign*}
      \Bracks[\Big]{ \varphi[\synx \mapsto \syny] }_v
      &=
      \Bracks[\Big]{ p(\tau_1[\synx \mapsto \syny], \ldots, \tau_n[\synx \mapsto \syny]) }_v
      = \\ &=
      \Bracks{ p(\tau_1, \ldots, \tau_n) }_{v_{\synx \mapsto \syny}}
      = \\ &=
      \Bracks{\varphi}_{v_{\synx \mapsto \syny}}.
    \end{balign*}

    \item The case \( \varphi = \tau_1 \syneq \tau_2 \) is proved analogously.

    \item The cases \( \varphi = \synneg \psi \) and \( \varphi = \psi_1 \syncirc \psi_2 \) are proved in an straightforward manner.

    \item Let \( \varphi = \qforall \synz \psi \), where the inductive hypothesis holds for \( \psi \). We consider three cases
    \begin{itemize}
      \item Suppose that \( \synz = \synx \). By definition, we have
      \begin{equation*}
        \varphi[\synx \mapsto \syny]
        =
        \varphi,
      \end{equation*}
      hence \eqref{eq:thm:renaming_assignment_compatibility/formulas} holds trivially.

      \item Suppose that \( \synz \neq \synx \). It follows that
      \begin{equation*}
        \varphi[\synx \mapsto \syny]
        =
        \qforall \synz \parens[\Big]{ \psi[\synx \mapsto \syny] }.
      \end{equation*}

      \begin{itemize}
        \item If \( \Bracks[\Big]{\varphi[\synx \mapsto \syny]}_v = T \), by definition of \hyperref[def:first_order_valuation/formula_valuation]{quantifier formula valuation}, for any \( x \in X \) we have
        \begin{equation}\label{eq:thm:renaming_assignment_compatibility/formulas/true_modified_assignment}
          \underbrace{\Bracks[\Big]{\qforall \synz \parens[\Big]{ \psi[\synx \mapsto \syny] } }_v}_{\varphi[\synx \mapsto \syny]}
          =
          \Bracks[\Big]{\psi[\synx \mapsto \syny]}_{v_{\synz \mapsto x}}
          =
          T.
        \end{equation}

        On the other hand, by the inductive hypothesis,
        \begin{equation*}
          \Bracks[\Big]{ \psi[\synx \mapsto \syny] }_v = \Bracks{\psi}_{v_{\synx \mapsto \syny}}
        \end{equation*}
        and, as a special case, for any \( x \in X \),
        \begin{equation}\label{eq:thm:renaming_assignment_compatibility/formulas/ind_hyp_modified_assignment}
          \Bracks[\Big]{ \psi[\synx \mapsto \syny] }_{v_{\synz \mapsto x}} = \Bracks{\psi}_{v_{\synx \mapsto \syny, \synz \mapsto x}}.
        \end{equation}

        Combining \eqref{eq:thm:renaming_assignment_compatibility/formulas/true_modified_assignment} and \eqref{eq:thm:renaming_assignment_compatibility/formulas/ind_hyp_modified_assignment}, we obtain
        \begin{equation*}
          \Bracks[\Big]{ \varphi[\synx \mapsto \syny] }_v
          =
          \Bracks[\Big]{ \psi[\synx \mapsto \syny] }_{v_{\synz \mapsto x}}
          =
          \underbrace{\Bracks{\psi}_{v_{\synx \mapsto \syny, \synz \mapsto x}}}_{T \T*{for all} x \in X}
          =
          \Bracks{\varphi}_{v_{\synx \mapsto \syny}},
        \end{equation*}
        which proves the case.

        \item If \( \Bracks[\Big]{\varphi[\synx \mapsto \syny]}_v = F \), then there exists \( x \in X \) such that
        \begin{equation*}
          \Bracks[\Big]{\psi[\synx \mapsto \syny]}_{v_{\synz \mapsto x}} = F.
        \end{equation*}

        Since \eqref{eq:thm:renaming_assignment_compatibility/formulas/ind_hyp_modified_assignment} holds by the inductive hypothesis, we have
        \begin{equation*}
          \Bracks{\psi}_{v_{\synx \mapsto \syny, \synz \mapsto x}} = F
        \end{equation*}
        for the same \( x \).

        It follows that \( \Bracks{\varphi}_{v_{\synx \mapsto \syny}} = F \), which proves the case.
      \end{itemize}
    \end{itemize}

    \item We can prove the case \( \varphi = \qexists \synz \psi \) using double negation on the previous case.
  \end{itemize}

  In all cases, \eqref{eq:thm:renaming_assignment_compatibility/formulas} holds.
\end{proof}

\begin{proposition}\label{thm:first_order_substitution_equivalence}
  Analogously to \fullref{thm:propositional_substitution_equivalence}, we will show that all substitutions defined in \fullref{def:first_order_substitution} types of substitution preserve the corresponding \hyperref[def:first_order_semantics]{semantics}.

  By induction, this proposition also holds for \hyperref[def:propositional_substitution/simultaneous]{simultaneous substitution}.

  Fix a \hyperref[def:first_order_structure]{structure} \( \mscrX = (X, I) \) and a \hyperref[def:first_order_valuation/variable_assignment]{variable assignment} \( v \).

  \begin{thmenum}
    \thmitem{thm:first_order_substitution_equivalence/propositional} As in \fullref{def:first_order_substitution/propositional}, let \( \varphi \) be a \hyperref[def:propositional_syntax/formula]{propositional formula} with variables \( {V = \set{ P_1, \ldots, P_n }} \) and let \( \Theta = \set{ \theta_1, \ldots, \theta_n } \) be a set of \hyperref[def:first_order_syntax/formula]{first-order formulas}.

    Furthermore, let \( J \) be a \hyperref[def:propositional_valuation/interpretation]{propositional interpretation} such that, for all \( k = 1, \ldots, n \),
    \begin{equation}\label{eq:thm:first_order_substitution_equivalence/propositional/compatibility}
      \Bracks{P_k}_J = \Bracks{\theta_k}_v.
    \end{equation}

    Then
    \begin{equation}\label{eq:thm:first_order_substitution_equivalence/propositional}
      \Bracks[\Big]{ \varphi[V \mapsto \Theta] }_v = \Bracks{\varphi}_J.
    \end{equation}

    In particular, \( \vDash \varphi \) in the sense of \fullref{def:propositional_tautology} implies \( \vDash \varphi[V \mapsto \Theta] \) in the sense of \fullref{def:first_order_semantics/validity}.

    \thmitem{thm:first_order_substitution_equivalence/term_in_term} Let \( \tau \) be a \hyperref[def:first_order_syntax/term]{first-order term} and let \( \kappa \) be a \hyperref[def:first_order_syntax/subterm]{subterm} of \( \tau \). Let \( \mu \) be another term such that
    \begin{equation}\label{eq:thm:first_order_substitution_equivalence/term_in_term/compatibility}
      \Bracks{\mu}_v = \Bracks{\kappa}_v.
    \end{equation}

    Then
    \begin{equation}\label{eq:thm:first_order_substitution_equivalence/term_in_term}
      \Bracks[\Big]{\tau[\kappa \mapsto \mu]}_v = \Bracks{\tau}_v.
    \end{equation}

    \thmitem{thm:first_order_substitution_equivalence/term_in_formula} Let \( \varphi \) be a \hyperref[def:first_order_syntax/formula]{first-order formula} and let \( \kappa \) be a \hyperref[def:first_order_syntax/formula_terms]{term} of \( \varphi \). Let \( \mu \) be another term such that
    \begin{equation}\label{eq:thm:first_order_substitution_equivalence/term_in_formula/compatibility}
      \Bracks{\mu}_v = \Bracks{\kappa}_v.
    \end{equation}

    Then
    \begin{equation}\label{eq:thm:first_order_substitution_equivalence/term_in_formula}
      \Bracks{\varphi[\kappa \mapsto \mu]}_v = \Bracks{\varphi}_v.
    \end{equation}
  \end{thmenum}
\end{proposition}
\begin{proof}
  In all cases, we use \fullref{thm:induction_on_rooted_trees} by the definition of the substitution. The inductive hypothesis for a formula is that the proposition holds for arbitrary substitutions and valuations.

  \SubProofOf{thm:first_order_substitution_equivalence/propositional} Let \( \varphi \) be a propositional formula.
  \begin{itemize}
    \item If \( \varphi \in \set{ \syntop, \synbot } \), no substitution is performed and thus \eqref{eq:thm:first_order_substitution_equivalence/propositional} holds trivially.

    \item If \( \varphi = P_k \) for some \( k = 1, \ldots, n \), then follows \eqref{eq:thm:first_order_substitution_equivalence/propositional} from \eqref{eq:thm:first_order_substitution_equivalence/propositional/compatibility}.

    \item If \( \varphi = \synneg \psi \) and if the inductive hypothesis holds for \( \psi \), then
    \begin{equation*}
      \Bracks[\Big]{ \psi[V \mapsto \Theta] }_v
      =
      \oline{\Bracks[\Big]{ \psi[V \mapsto \Theta] }_v}
      \reloset {\T{ind.}} =
      \oline{\Bracks{\psi}_J}
      =
      \Bracks{\varphi}_J.
    \end{equation*}

    \item If \( \varphi = \psi_1 \syncirc \psi_2, {\syncirc} \in \op*{Conn} \) and if the inductive hypothesis holds for both \( \psi_1 \) and \( \psi_2 \), then
    \begin{equation*}
      \Bracks[\Big]{ \psi[V \mapsto \Theta] }_v
      =
      \Bracks[\Big]{ \psi_1[V \mapsto \Theta] }_v \relcirc \Bracks[\Big]{ \psi_2[V \mapsto \Theta] }_v
      \reloset {\T{ind.}} =
      \Bracks{\psi_1}_J \relcirc \Bracks{\psi_2}_J
      =
      \Bracks{\varphi}_J.
    \end{equation*}
  \end{itemize}

  In all cases, \eqref{eq:thm:first_order_substitution_equivalence/propositional} holds.

  \SubProofOf{thm:first_order_substitution_equivalence/term_in_term} The proof is identical to that of \fullref{thm:renaming_assignment_compatibility/terms}.

  \SubProofOf{thm:first_order_substitution_equivalence/term_in_formula} The proof is identical to that of \fullref{thm:renaming_assignment_compatibility/formulas} except for the special cases where \hyperref[rem:first_order_substitution_renaming_justification]{renaming} occurs, i.e. \( \varphi = \qforall \synx \psi \) and \( \varphi = \qexists \synx \psi \), where
  \begin{itemize}
    \item \( \synx \in \op*{Free}(\mu) \).
    \item \( \syny \not\in \op*{Var}(\kappa) \cup \op*{Var}(\mu) \cup \op*{Var}(\psi) \).
    \item The inductive hypothesis holds for \( \psi \).
  \end{itemize}

  We will only show the case \( \varphi = \qforall \synx \psi \) since the existential case is handled similarly. Since \( \synx \in \op*{Free}(\mu) \), we have
  \begin{equation*}
    \varphi[\kappa \mapsto \mu]
    =
    \qforall \syny \parens[\Big]{ \psi[\synx \mapsto \syny][\kappa \mapsto \mu] },
  \end{equation*}
  which does not allow us to use the inductive hypothesis directly.

  We proceed to prove the statement by nested induction on the number of quantifiers. We have already shown the case of \( 0 \) quantifiers. Suppose that the statement holds for all formulas with strictly less than \( n \) quantifiers and suppose that \( \varphi \) has exactly \( n \) quantifiers.

  Furthermore, for formulas with \( n \) quantifiers with \( \synforall \) as the outermost one, the non-renaming cases \eqref{eq:def:first_order_substitution/term_in_formula/quantifiers/trivial} and \eqref{eq:def:first_order_substitution/term_in_formula/quantifiers/direct} hold. Therefore, since \( \syny \not\in \op*{Free}(\mu) \),
  \begin{equation}\label{eq:thm:first_order_substitution_equivalence/term_in_formula/nested_induction}
    \begin{aligned}
      &\phantom{{}={}}
      \Bracks{\varphi[\kappa \mapsto \mu]}_v
      = \\ &=
      \Bracks[\Big]{\qforall \syny \parens[\Big]{ \psi[\synx \mapsto \syny][\kappa \mapsto \mu] }}_v
      \reloset {\eqref{eq:def:first_order_substitution/term_in_formula/quantifiers/direct}} = \\ &=
      \Bracks[\Big]{ \parens*{ \qforall \syny \parens[\Big]{ \psi[\synx \mapsto \syny] } }[\kappa \mapsto \mu] }_v
      \reloset {\T{ind.}} = \\ &=
      \Bracks[\Big]{\qforall \syny \parens[\Big]{ \psi[\synx \mapsto \syny] } }_v,
    \end{aligned}
  \end{equation}
  where we have implicitly used that \( \psi \) has \( n - 1 \) quantifiers.

  On the other hand, due to \fullref{thm:renaming_assignment_compatibility/formulas},
  \begin{equation*}
    \Bracks[\Big]{ \psi[\synx \mapsto \syny] }_v = \Bracks{\psi}_{v_{\synx \mapsto \syny}}
  \end{equation*}
  and, in particular, for any \( x \in X \),
  \begin{equation*}
    \Bracks[\Big]{ \psi[\synx \mapsto \syny] }_{v_{\syny \mapsto x}}
    =
    \Bracks{\psi}_{v_{\synx \mapsto \syny,\syny \mapsto x}}
    =
    \Bracks{\psi}_{v_{\synx \mapsto x}},
  \end{equation*}
  where the last equality holds because \( \syny \not\in \op*{Var}(\psi) \).

  Hence,
  \begin{equation*}
    \underbrace{ \Bracks[\Big]{\qforall \syny \parens[\Big]{ \psi[\synx \mapsto \syny] } }_v }_{\Bracks{\varphi[\synx \mapsto \syny]}_v}
    =
    \underbrace{ \Bracks[\Big]{\qforall \synx \psi }_{v_{\synx \mapsto \syny}} }_{\Bracks{\varphi}_{v_{\synx \mapsto \syny}}}.
  \end{equation*}

  This proves \eqref{eq:thm:first_order_substitution_equivalence/term_in_formula}.
\end{proof}

\begin{concept}\label{con:predicate_formula}
  As explained in \fullref{rem:first_order_formula_conventions/necessary_signature}, we avoid adding more predicates than necessary to a language. For this reason, we sometimes use \term{predicate formulas}. For example, if \( \leq \) is a \hyperref[def:partially_ordered_set]{partial order} symbol and we want to have a predicate for whether \( \synx \) is the \hyperref[def:extremal_points/top_and_bottom]{bottom element}, we can define the formula
  \begin{equation}\taglabel[\op{IsBottom}]{con:predicate_formula/bottom}
    \ref{con:predicate_formula/bottom}[\synx] \coloneqq \qforall \syny (\synx \leq \syny).
  \end{equation}

  Note that \( [\synx] \) is only a notational convenience for highlighting which variables are free, the actual formula is named \( \op{IsBottom} \). This is consistent with \fullref{rem:uniform_substitution_notation} which allows us to write \( \op{IsBottom}[\syny] \) rather than \( \op{IsBottom}[\synx \mapsto \syny] \) to verify if \( \syny \) is a bottom element
\end{concept}

\begin{proposition}\label{thm:first_order_quantifiers_are_dual}
  For any formula \( \varphi \) and any variable \( \synx \) over \( \mscrL \), we have the following equivalences:
  \begin{align}
    \synneg \qforall \synx \varphi &\gleichstark \qexists \synx \synneg \varphi \label{eq:thm:first_order_quantifiers_are_dual/negation_of_universal} \\
    \synneg \qexists \synx \varphi &\gleichstark \qforall \synx \synneg \varphi \label{eq:thm:first_order_quantifiers_are_dual/negation_of_existential}
  \end{align}
\end{proposition}
\begin{proof}
  The two equivalences are connected using \hyperref[thm:classical_equivalences/double_negation]{double negation}. We will only prove \eqref{eq:thm:first_order_quantifiers_are_dual/negation_of_universal}.

  Let \( \mscrX = (X, I) \) be a structure over \( \mscrL \) and let \( v \) be a variable assignment. Then
  \begin{align*}
    \Bracks{ \synneg \qforall \synx \varphi }_v
    &=
    \oline{\Bracks{ \qforall \synx \varphi }_v}
    = \\ &=
    \oline{\bigwedge\set{ \Bracks{\varphi}_{v_{\synx \mapsto x}} \given x \in X }}
    \reloset {\eqref{eq:thm:de_morgans_laws/complement_of_meet}} = \\ &=
    \bigvee\set*{ \oline{\Bracks{\varphi}_{v_{\synx \mapsto x}}} \given x \in X }
    = \\ &=
    \bigvee\set{ \Bracks{\synneg \varphi}_{v_{\synx \mapsto x}} \given x \in X }
    = \\ &=
    \Bracks{\qexists \synx \synneg \varphi}_v.
  \end{align*}
\end{proof}

\begin{corollary}\label{thm:relativized_first_order_quantifiers_are_dual}
  \Fullref{thm:first_order_quantifiers_are_dual} also holds for \hyperref[rem:first_order_formula_conventions/relativization]{relativized} formulas:
  \begin{align}
    \synneg \qforall {p(\synx, \cdots)} \varphi &\gleichstark \qexists {p(\synx, \cdots)} \synneg \varphi \label{eq:thm:relativized_first_order_quantifiers_are_dual/negation_of_universal} \\
    \synneg \qexists {p(\synx, \cdots)} \varphi &\gleichstark \qforall {p(\synx, \cdots)} \synneg \varphi \label{eq:thm:relativized_first_order_quantifiers_are_dual/negation_of_existential}
  \end{align}
\end{corollary}
\begin{proof}
  We will only show \eqref{eq:thm:relativized_first_order_quantifiers_are_dual/negation_of_universal}:
  \begin{align*}
    \synneg \qforall {p(\synx, \cdots)} \varphi
    &=
    \synneg \qforall \synx \parens[\Big]{ p(\synx, \cdots) \synimplies \varphi }
    \gleichstark \\ \reloset {\eqref{eq:thm:classical_equivalences/conditional_as_disjunction}} \gleichstark
    \synneg \qforall \synx \parens[\Big]{ \synneg p(\synx, \cdots) \synvee \varphi }
    \gleichstark \\ \reloset {\eqref{eq:thm:first_order_quantifiers_are_dual/negation_of_universal}} \gleichstark
    \qexists \synx \parens[\Big]{ p(\synx, \cdots) \synwedge \synneg \varphi }
  \end{align*}
\end{proof}

\begin{proposition}\label{thm:pulling_quantifiers_out}
  For \( Q \in \op*{Quan} \) and \( \syncirc \in \set{ \synvee, \synwedge } \), if \( \syny \) is not free in \( \varphi \) nor \( \psi \), then
  \begin{equation}\label{thm:pulling_quantifiers_out/first}
    \parens[\Big]{ \quantifier Q \synx \varphi } \syncirc \psi \gleichstark \quantifier Q \syny \parens[\Big]{ \varphi[\synx \mapsto \syny] \syncirc \psi }\phantom{.}
  \end{equation}
  and similarly
  \begin{equation}\label{thm:pulling_quantifiers_out/second}
    \varphi \syncirc \parens[\Big]{ \quantifier Q \synx \psi } \gleichstark \quantifier Q \syny \parens[\Big]{ \varphi \syncirc \psi[\synx \mapsto \syny] }.
  \end{equation}
\end{proposition}
\begin{proof}
  We will only consider one special case of \eqref{thm:pulling_quantifiers_out/first}:
  \begin{equation}\label{thm:pulling_quantifiers_out/first_proof}
    \parens[\Big]{ \qforall \synx \varphi } \synwedge \psi \gleichstark \qforall \synx \parens[\Big]{ \varphi[\synx \mapsto \syny] \synwedge \psi }
  \end{equation}

  The other cases have similar proofs.

  Fix some \hyperref[def:first_order_structure]{structure} \( \mscrX = (X, I) \) and some \hyperref[def:first_order_valuation/variable_assignment]{variable assignment} \( v: \op*{Var} \to X \). Since \( \syny \) is not free in \( \psi \),
  \begin{balign*}
    \Bracks[\Big]{ \qforall \syny \parens[\Big]{ \varphi[\synx \mapsto \syny] \synwedge \psi } }_v
    &=
    \bigvee\set[\Big]{ \Bracks[\Big]{\varphi[\synx \mapsto \syny]}_{v_{\syny \mapsto x}} \synwedge \Bracks{\psi}_{v_{\syny \mapsto x}} \given* x \in X }
    \reloset {\eqref{eq:thm:renaming_assignment_compatibility/formulas}} = \\ &=
    \bigvee\set[\Big]{ \Bracks{\varphi}_{v_{\synx \mapsto \syny,\syny \mapsto x}} \synwedge \Bracks{\psi}_{v_{\syny \mapsto x}} \given* x \in X }
    = \\ &=
    \bigvee\set[\Big]{ \Bracks{\varphi}_{v_{\synx \mapsto x}} \synwedge \Bracks{\psi}_{v_{\syny \mapsto x}} \given* x \in X }
    = \\ &=
    \bigvee\set[\Big]{ \Bracks{\varphi}_{v_{\synx \mapsto x}} \synwedge \Bracks{\psi}_v \given* x \in X }
    \reloset {\eqref{eq:thm:def:boolean_algebra/distributive/meet_over_join}} = \\ &=
    \bigvee\set[\Big]{ \Bracks{\varphi}_{v_{\synx \mapsto x}} \given* x \in X } \synwedge \Bracks{\psi}_v
    = \\ &=
    \Bracks[\Big]{ \qforall \synx \varphi }_v \synwedge \Bracks{\psi}_v
    = \\ &=
    \Bracks[\Big]{ \parens[\Big]{ \qforall \synx \varphi } \synwedge \psi }_v.
  \end{balign*}

  Generalizing on variable assignments and structures, we conclude that \eqref{thm:pulling_quantifiers_out/first_proof} holds.
\end{proof}

\begin{definition}\label{def:prenex_normal_form}\mcite[def. 2.2.33]{Hinman2005}
  We say that the formula \( \varphi \) over \( \mscrL \) is in \term{prenex normal form} if it can be split as \( \varphi = \psi \theta \), where \( \psi \) contains only quantifiers and their variables (and dots) and \( \theta \) is a formula without quantifiers.

  We call \( \psi \) the \term{quantifier prefix} of \( \varphi \) and \( \theta \) --- the \term{matrix} of \( \varphi \).
\end{definition}

\begin{algorithm}[Prenex normal form conversion]\label{alg:prenex_normal_form_conversion}
  We will describe an algorithm that takes a formula \( \varphi \) over \( \mscrL \) and outputs another formula \( \theta \) over \( \mscrL \) that is \hyperref[def:first_order_semantics/equivalence]{semantically equivalent}, but is in \hyperref[def:prenex_normal_form]{prenex normal form}.

  \begin{thmenum}
    \thmitem{alg:prenex_normal_form/conditionals} First, we remove the conditional \( \synimplies \) and biconditional \( \syniff \). Define
    \begin{equation*}
      P_1(\varphi) \coloneqq \begin{cases}
        \synneg P_1(\psi) \synvee P_1(\theta),                                                                          &\varphi = \psi \synimplies \theta, \\
        \parens[\Big]{ \synneg P_1(\psi) \synvee P_1(\theta) } \synwedge \parens[\Big]{ P_1(\psi) \synvee \synneg P_1(\theta) }, &\varphi = \psi \syniff \theta, \\
        \hyperref[rem:straightforward_traversal]{\T{straightforward traversal}},                                  &\T{otherwise.}
      \end{cases}
    \end{equation*}

    \Fullref{thm:classical_equivalences/conditional_as_disjunction}, \fullref{thm:classical_equivalences/biconditional_as_conjunction} and \fullref{thm:first_order_substitution_equivalence/propositional} together ensure that \( P_1(\varphi) \gleichstark \varphi \) for any formula \( \varphi \).

    \thmitem{alg:prenex_normal_form/negation} Second, we \enquote{push} negations inwards. Define
    \begin{equation*}
      P_2(\varphi) \coloneqq \begin{cases}
        \qexists \synx P_2(\synneg \psi),                                             &\varphi = \synneg \qforall \synx \psi, \\
        \qforall \synx P_2(\synneg \psi),                                             &\varphi = \synneg \qexists \synx \psi, \\
        P_2(\synneg \psi) \synwedge P_2(\synneg \theta),                                  &\varphi = \synneg (\psi \synvee \theta), \\
        P_2(\synneg \psi) \synvee P_2(\synneg \theta),                                    &\varphi = \synneg (\psi \synwedge \theta), \\
        P_2(\psi),                                                               &\varphi = \synneg \synneg \psi, \\
        \synneg P_2(\psi),                                                          &\T{otherwise if} \varphi = \synneg \psi, \\
        \hyperref[rem:straightforward_traversal]{\T{straightforward traversal}}, &\T{otherwise.}
      \end{cases}
    \end{equation*}

    \Fullref{thm:de_morgans_laws}, \fullref{thm:classical_equivalences/double_negation}, \fullref{thm:first_order_substitution_equivalence/propositional} and \fullref{thm:first_order_quantifiers_are_dual} ensure that \( P_2(\varphi) \gleichstark \varphi \) for any formula \( \varphi \)  not containing \( \synimplies \) or \( \syniff \).

    \thmitem{alg:prenex_normal_form/quantifiers} Finally, we \enquote{pull} quantifiers outwards. Define
    \begin{equation*}
      P_3(\varphi) \coloneqq \begin{cases}
        \varphi,                                                                           &\op*{Bound}(\varphi) = \varnothing, \\
        P_3\parens[\Big]{ P_3(\psi) \syncirc P_3(\theta) },                                &\varphi = \psi \syncirc \theta \T{and} \op*{Bound}(\varphi) \neq \varnothing, \\
        \quantifier Q \syny P_3 \parens[\Big]{ \varphi[\synx \mapsto \syny] \syncirc \theta }, &\varphi = \parens[\Big]{ \quantifier Q \synx \psi } \syncirc \theta, \\
                                                                                           &Q \in \op*{Quan}, \quad \syncirc \in \set{ \synvee, \synwedge }, \\
                                                                                           &\syny \not\in \op*{Free}(\psi) \cup \op*{Free}(\theta), \\
        \quantifier Q \syny P_3 \parens[\Big]{ \varphi \syncirc \theta[\synx \mapsto \syny] }, &\varphi = \psi \syncirc \parens[\Big]{ \quantifier Q \synx \theta }, \\
                                                                                           &Q \in \op*{Quan}, \quad \syncirc \in \set{ \synvee, \synwedge }, \\
                                                                                           &\syny \not\in \op*{Free}(\psi) \cup \op*{Free}(\theta), \\
        \hyperref[rem:straightforward_traversal]{\T{straightforward traversal}},           &\T{otherwise.}
      \end{cases}
    \end{equation*}

    \Fullref{thm:pulling_quantifiers_out} ensures that \( P_3(\varphi) \gleichstark \varphi \) for any formula \( \varphi \) not containing \( \synimplies \) or \( \syniff \). It is not necessarily in prenex normal form because it does nothing to \( \varphi = \synneg \qforall \synx \psi \).
  \end{thmenum}

  It remains to compose the operators. As per our comments above, the formula \( P_3(P_2(P_1(\varphi))) \) is in prenex normal form and is semantically equivalent to \( \varphi \).
\end{algorithm}
\begin{comments}
  \item The first rule in the definition of \( P_3 \) ensures that the recursion is well-founded. Checking whether the formula has bound variables is a way to verify that it contains no quantifiers.
  \item This algorithm can be found as \identifier{fol.pnf.to_pnf} in \cite{code}.
\end{comments}

\begin{proposition}\label{thm:quantifier_satisfiability}
  Let \( \mscrL \) be a first-order language, let \( \varphi \) be a formula, let \( \synx \) be any variable and let \( \tau \) be any term in \( \mscrL \). The following hold:

  \begin{thmenum}
    \thmitem{thm:quantifier_satisfiability/universal} \( \qforall \synx \varphi \vDash \varphi[\synx \mapsto \tau] \),

    \thmitem{thm:quantifier_satisfiability/existential} \( \varphi[\synx \mapsto \tau] \vDash \qexists \synx \varphi \).
  \end{thmenum}
\end{proposition}
\begin{comments}
  \item See also \fullref{def:first_order_natural_deduction_system/terms} for inference rules corresponding to this proposition.
\end{comments}
\begin{proof}
  The proof is very straightforward, but the technical details make it look a bit more complicated.

  First note that if the formulas on the left are unsatisfiable, the proof is trivial. Hence, we will assume that they are satisfiable.

  \SubProofOf{thm:quantifier_satisfiability/universal} Suppose that \( \Bracks{\qforall \synx \varphi}_v = T \) for some variable assignment in \( \mscrX \), and let \( t \coloneqq \Bracks{\tau}_v \).

  To avoid the case where \( \synx \in \op*{Var}(\varphi) \) for the purposes for \fullref{thm:renaming_assignment_compatibility}, we substitute it with \( \syny \) that is not a variable in neither \( \tau \) nor \( \varphi \). Then
  \begin{align*}
    \Bracks[\Big]{\varphi[\synx \mapsto \tau]}_v
    &=
    \Bracks[\Big]{\varphi[\synx \mapsto \syny][\syny \mapsto \tau]}_v
    = \\ &=
    \Bracks[\Big]{\varphi[\synx \mapsto \syny][\syny \mapsto \tau]}_{v_{\syny \mapsto t}}
    \reloset {\ref{thm:first_order_substitution_equivalence/term_in_formula}} = \\ &=
    \Bracks[\Big]{\varphi[\synx \mapsto \syny]}_{v_{\syny \mapsto t}}
    \reloset {\ref{thm:renaming_assignment_compatibility/formulas}} = \\ &=
    \Bracks{\varphi}_{v_{\synx \mapsto \syny, \syny \mapsto t}}
    = \\ &=
    \Bracks{\varphi}_{v_{\synx \mapsto t}}.
  \end{align*}

  Since \( \Bracks{\qforall \synx \varphi} = T \), by definition of valuation of \( \synforall \) we have
  \begin{equation*}
    \Bracks{\varphi}_{v_{\synx \mapsto t}} = T.
  \end{equation*}
  and thus
  \begin{equation*}
    \Bracks[\Big]{\varphi[\synx \mapsto \tau]}_v = T.
  \end{equation*}

  Therefore,
  \begin{equation*}
    \qforall \synx \varphi \T{entails} \varphi[\synx \mapsto \tau] \T{under} v.
  \end{equation*}

  Both \( v \) and \( \mscrX \) were chosen arbitrarily, implying
  \begin{equation*}
    \qforall \synx \varphi \vDash \varphi[\synx \mapsto \tau].
  \end{equation*}

  \SubProofOf{thm:quantifier_satisfiability/existential} If \( \Bracks{\varphi[\synx \mapsto \tau]} = T \), then, as shown previously,
  \begin{equation*}
    \Bracks[\Big]{\varphi[\synx \mapsto \tau]}_v = \Bracks{\varphi}_{v_{\synx \mapsto t}}.
  \end{equation*}

  Hence,
  \begin{equation*}
    \Bracks{\qexists \synx \varphi} = \Bracks{\varphi}_{v_{\synx \mapsto t}} = T.
  \end{equation*}

  Therefore,
  \begin{equation*}
    \varphi[\synx \mapsto \tau] \T{entails} \varphi \T{under} v,
  \end{equation*}
  and, generalizing,
  \begin{equation*}
    \varphi[\synx \mapsto \tau] \vDash \varphi.
  \end{equation*}
\end{proof}

\begin{proposition}\label{thm:existential_quantifier_removal}
  Let \( \varphi \) be a formula in the first-order language \( \mscrL \), and let \( \widetilde \mscrL \) be an extension of \( \mscrL \). Clearly \( \varphi \) is a formula in \( \widetilde \mscrL \).

  Suppose that \( c \) is some constant in \( \widetilde \mscrL \) that does not occur in \( \varphi \). Then there exists a structure \( \mscrX = (X, I_X) \) for \( \mscrL \) and an assignment \( v \) in \( \mscrX \) such that
  \begin{equation*}
    \Bracks{\qexists \synx \varphi}_v = T
  \end{equation*}
  if and only if there exists a structure \( \mscrY = (Y, I_Y) \) for \( \widetilde \mscrL \) and an assignment \( w \) in \( \mscrY \) such that
  \begin{equation*}
    \Bracks[\Big]{\varphi[\synx \mapsto c]}_w = T
  \end{equation*}
\end{proposition}
\begin{comments}
  \item The sufficiency statement is a weaker converse to \fullref{thm:quantifier_satisfiability/existential}.
  \item We can always adjoin a single constant \( c \) to \( \mscrL \) to obtain \( \widetilde \mscrL \) (this is actually our proof of sufficiency). This allows us to eliminate existential quantifiers from formulas at the cost of a more general first-order language.
\end{comments}
\begin{proof}
  \SufficiencySubProof Suppose that, for some assignment \( v \) in \( \mscrX \), \( \Bracks{\qexists \synx \varphi}_v = T \). Let \( x_0 \) be a value in \( X \) such that \( \Bracks{\varphi}_{v_{\synx \to x_0}} = T \).

  Construct a language \( \widetilde \mscrL \) with all predicate and functional symbols of \( \mscrL \), along with a new constant symbol \( c \).

  Define a structure \( \mscrY = (X, I_Y) \), where
  \begin{equation*}
    \widetilde I_Y(a) \coloneqq \begin{cases}
      x_0,    &a = c, \\
      I_X(a), &a \in \op*{Func}, \\
      I_X(a), &a \in \op*{Pred}
    \end{cases}
  \end{equation*}

  This structure has the same domain as \( \mscrX \), hence \( v \) is also a variable assignment in \( \mscrY \).

  Since \( \Bracks{c}_{v_{\synx \mapsto x_0}} = \Bracks{\synx}_{v_{\synx \mapsto x_0}} \), \fullref{thm:first_order_substitution_equivalence/term_in_formula} implies that
  \begin{equation*}
    \Bracks[\Big]{\varphi[\synx \mapsto c]}_{v_{\synx \mapsto x_0}} = \Bracks{\varphi}_{v_{\synx \mapsto x_0}} = T.
  \end{equation*}

  But, since \( \varphi[\synx \mapsto c] \) contains no free instances of \( \synx \),
  \begin{equation*}
    \Bracks[\Big]{\varphi[\synx \mapsto c]}_v = \Bracks[\Big]{\varphi[\synx \mapsto c]}_{v_{\synx \mapsto x_0}}.
  \end{equation*}

  Therefore, \( \Bracks{\qexists \synx \varphi}_v = T \) in \( \mscrX \) implies \( \Bracks{\varphi[\synx \mapsto c]}_v = T \) in \( \mscrY \).

  \NecessitySubProof Follows from \fullref{thm:quantifier_satisfiability/existential}.
\end{proof}

\begin{example}\label{ex:existential_quantifier_removal_monoids}
  The \hyperref[def:monoid/theory]{theory of monoids} features a simple demonstration of \fullref{thm:existential_quantifier_removal}.

  One of the axioms for monoids states, in prose, that there exists a distinguished element that is a two-sided identity. This can be written as follows:
  \begin{equation*}
    \qexists \syny \qforall \synx (\syny \cdot \synx \syneq \synx \synwedge \synx \cdot \syny \syneq \synx).
  \end{equation*}

  Instead, for \fullref{def:monoid/theory} we chose to include a constant symbol \( e \) in the language and use the axiom
  \begin{equation*}
    \qforall \synx (e \cdot \synx \syneq \synx \synwedge \synx \cdot e \syneq \synx).
  \end{equation*}
\end{example}

\begin{theorem}[First-order semantic deduction theorem]\label{thm:semantic_deduction_theorem}
  For arbitrary first-order formulas, the entailment \( \Gamma, \varphi \vDash \psi \) holds if and only if \( \Gamma \vDash \varphi \to \psi \) holds.
\end{theorem}
\begin{comments}
  \item This theorem also holds for propositional formulas with the adaptation outlined in \fullref{rem:propositional_logic_as_first_order_logic}.
  \item Compare this result with \fullref{thm:syntactic_deduction_theorem}.
\end{comments}
\begin{proof}
  \SufficiencySubProof Suppose that \( \Gamma, \varphi \vDash \psi \). Let \( \mscrX = (X, I) \) be some structure and let \( v \) be a variable assignment in \( \mscrX = (X, I) \) such that \( \Bracks{\theta}_v = T \) for every \( \theta \in \Gamma \).

  \begin{itemize}
    \item If \( \Bracks{\varphi}_v = T \), by assumption it follows that \( \Bracks{\psi}_v = T \), and
    \begin{equation*}
      \Bracks{\varphi \synimplies \psi}_v
      =
      \Bracks{\varphi}_v \rightarrow \Bracks{\psi}_v
      =
      T \rightarrow T
      =
      T.
    \end{equation*}

    \item If \( \Bracks{\varphi}_v = F \), then
    \begin{equation*}
      \Bracks{\varphi \synimplies \psi}_v
      =
      F \rightarrow \Bracks{\psi}_v
      =
      T.
    \end{equation*}
  \end{itemize}

  In both cases we conclude that \( \Gamma \) entails \( \varphi \synimplies \psi \) under \( v \). Generalizing on \( v \) and \( \mscrX \), we conclude that \( \Gamma \vDash \varphi \synimplies \psi \).

  \NecessitySubProof Conversely, suppose that \( \Gamma \vDash \varphi \synimplies \psi \). Let \( v \) again be a variable assignment in \( \mscrX = (X, I) \) such that \( \Bracks{\theta}_v = T \) for every \( \theta \in \Gamma \). Then
  \begin{equation*}
    \Bracks{\varphi \synimplies \psi}_v = T.
  \end{equation*}

  If we additionally suppose that \( \Bracks{\varphi}_v = T \), it follows from \fullref{thm:standard_boolean_functions} that \( \Bracks{\psi}_v = T \).

  Generalizing on \( v \) and \( \mscrX \), we conclude that \( \Gamma, \varphi \vDash \psi \).
\end{proof}

\paragraph{Natural deduction}

\begin{definition}\label{def:first_order_natural_deduction_system}\mcite[def. 2.1.1]{TroelstraSchwichtenberg2000}
  If we wish to work with first-order logic rather than merely propositional logic, we must extend the \hyperref[def:propositional_natural_deduction_systemss]{classical propositional natural deduction system}. We call this, very simply, the (classical) \term{first-order natural deduction system}.

  \begin{thmenum}
    \thmitem{def:first_order_natural_deduction_system/eigenvariables} We first add the following two \hyperref[con:judgment/inference_rule]{inference rules} for quantification:

    \begin{minipage}{0.45\textwidth}
      \begin{equation*}\taglabel[\( \synforall^+ \)]{eq:def:first_order_natural_deduction_system/forall/intro}
        \begin{prooftree}
          \hypo{ \varphi }
          \infer1[\ref{eq:def:first_order_natural_deduction_system/forall/intro}]{ \qforall \synx \varphi }
        \end{prooftree}
      \end{equation*}
    \end{minipage}
    \hfill
    \begin{minipage}{0.45\textwidth}
      \begin{equation*}\taglabel[\( \synexists^- \)]{eq:def:first_order_natural_deduction_system/exists/elim}
        \begin{prooftree}
          \hypo{ \qexists \synx \varphi }
          \hypo{ [\varphi]^n }
          \ellipsis {} { \psi }
          \infer[left label=\( n \)]2[\ref{eq:def:first_order_natural_deduction_system/exists/elim}]{ \psi }
        \end{prooftree}
      \end{equation*}
    \end{minipage}

    Here \( \synx \) is a variable that is not free in any undischarged assumption in our proof of \( \varphi \) (it may be free in \( \varphi \) as long as \( \varphi \) itself is discharged). A variable \( \synx \) satisfying these conditions is called an \term{eigenvariable} of the rule.

    These rules are the primary motivation for inference rules accepting proof trees rather than only formulas --- see \fullref{con:judgment/inference_rule} and \fullref{def:deduction_system/rule}. See \fullref{ex:def:first_order_natural_deduction_system/eigenvariables/invalid_universal} for why this condition is important.

    \thmitem{def:first_order_natural_deduction_system/terms} We add two \hyperref[con:judgment/inference_rule]{inference rules}, where \( \tau \) is an arbitrary term:

    \begin{minipage}{0.45\textwidth}
      \begin{equation*}\taglabel[\( \synforall^- \)]{eq:def:first_order_natural_deduction_system/forall/elim}
        \begin{prooftree}
          \hypo{ \qforall \synx \varphi }
          \infer1[\ref{eq:def:first_order_natural_deduction_system/forall/elim}]{ \varphi[\synx \mapsto \tau] }
        \end{prooftree}
      \end{equation*}
    \end{minipage}
    \hfill
    \begin{minipage}{0.45\textwidth}
      \begin{equation*}\taglabel[\( \synexists^+ \)]{eq:def:first_order_natural_deduction_system/exists/intro}
        \begin{prooftree}
          \hypo{ \varphi[\synx \mapsto \tau] }
          \infer1[\ref{eq:def:first_order_natural_deduction_system/exists/intro}]{ \qexists \synx \varphi }
        \end{prooftree}
      \end{equation*}
    \end{minipage}

    Compare this to \fullref{thm:quantifier_satisfiability}.

    \thmitem{def:first_order_natural_deduction_system/equality} Finally, we also add three rules for formal equality:

    \begin{minipage}{0.3\textwidth}
      \begin{equation*}\taglabel[\( \syneq^+ \)]{eq:def:first_order_natural_deduction_system/equality/intro}
        \begin{prooftree}
          \infer0[\ref{eq:def:first_order_natural_deduction_system/equality/intro}]{ \tau \syneq \tau }
        \end{prooftree}
      \end{equation*}
    \end{minipage}
    \hfill
    \begin{minipage}{0.3\textwidth}
      \begin{equation*}\taglabel[\( \syneq_L^- \)]{eq:def:first_order_natural_deduction_system/equality/elim_left}
        \begin{prooftree}
          \hypo{ \tau \syneq \sigma }
          \hypo{ \varphi[\synx \mapsto \tau] }
          \infer2[\ref{eq:def:first_order_natural_deduction_system/equality/elim_left}]{ \varphi[\synx \mapsto \sigma] }
        \end{prooftree}
      \end{equation*}
    \end{minipage}
    \hfill
    \begin{minipage}{0.3\textwidth}
      \begin{equation*}\taglabel[\( \syneq_L^+ \)]{eq:def:first_order_natural_deduction_system/equality/elim_right}
        \begin{prooftree}
          \hypo{ \tau \syneq \sigma }
          \hypo{ \varphi[\synx \mapsto \sigma] }
          \infer2[\ref{eq:def:first_order_natural_deduction_system/equality/elim_right}]{ \varphi[\synx \mapsto \tau] }
        \end{prooftree}
      \end{equation*}
    \end{minipage}
  \end{thmenum}
\end{definition}

\begin{example}\label{ex:def:first_order_natural_deduction_system/eigenvariables}
  \hfill
  \begin{thmenum}
    \thmitem{ex:def:first_order_natural_deduction_system/eigenvariables/invalid_universal_closure} We explicitly forbid the syntactic equivalent of \fullref{thm:implicit_universal_quantification} in order to avoid invalid proofs like \fullref{ex:def:first_order_natural_deduction_system/eigenvariables/invalid_universal}. Consider the proof
    \begin{equation*}
      \begin{prooftree}
        \hypo{ \varphi }
        \infer1[\ref{eq:def:first_order_natural_deduction_system/forall/intro}]{ \qforall \synx \varphi }
      \end{prooftree}
    \end{equation*}

    The problem here is that \( \varphi \) is itself an undischarged assumption, hence \eqref{eq:def:first_order_natural_deduction_system/forall/intro} is actually inapplicable here, and the proof is invalid.

    \thmitem{ex:def:first_order_natural_deduction_system/eigenvariables/invalid_universal} To see why the eigenvariable conditions in \fullref{def:first_order_natural_deduction_system/eigenvariables} are essential, consider the following proof of \( \qforall \synx \varphi \) from \( \qexists \synx \varphi \):
    \begin{equation*}
      \begin{prooftree}
        \hypo{ \qexists \synx \varphi }

        \hypo{ [\varphi]^1 }
        \infer1[\ref{eq:def:first_order_natural_deduction_system/forall/intro}]{ \qforall \synx \varphi }

        \infer[left label=\( 1 \)]2[\ref{eq:def:first_order_natural_deduction_system/exists/elim}]{ \qforall \synx \varphi }
      \end{prooftree}
    \end{equation*}

    This proof relies on \fullref{ex:def:first_order_natural_deduction_system/eigenvariables/invalid_universal_closure}, which we have already demonstrated to be invalid.

    \thmitem{ex:def:first_order_natural_deduction_system/eigenvariables/invalid_existence} Another invalid proof, in case \( \synx \in \op*{Free}(\varphi) \), is
    \begin{equation*}
      \begin{prooftree}
        \hypo{ \qexists \synx \varphi }

        \hypo{ [\varphi]^1 }
        \infer1{ \varphi }

        \infer[left label=\( 1 \)]2[\ref{eq:def:first_order_natural_deduction_system/exists/elim}]{ \varphi }
      \end{prooftree}
    \end{equation*}

    \thmitem{ex:def:first_order_natural_deduction_system/eigenvariables/universal_implies_existence} On the other hand, \( \qexists \synx \varphi \) can easily be derived from \( \qforall \synx \varphi \):
    \begin{equation*}
      \begin{prooftree}
        \hypo{ \qforall \synx \varphi }
        \infer1[\ref{eq:def:first_order_natural_deduction_system/forall/elim}]{ \varphi = \varphi[\synx \mapsto \synx] }
        \infer1[\ref{eq:def:first_order_natural_deduction_system/exists/intro}]{ \qexists \synx \varphi }
      \end{prooftree}
    \end{equation*}

    \thmitem{ex:def:first_order_natural_deduction_system/eigenvariables/universal_implies_universal} It is also valid to perform the completely meaningless derivation:
    \begin{equation*}
      \begin{prooftree}
        \hypo{ \qforall \synx \varphi }
        \infer1[\ref{eq:def:first_order_natural_deduction_system/forall/elim}]{ \varphi = \varphi[\synx \mapsto \synx] }
        \infer1[\ref{eq:def:first_order_natural_deduction_system/forall/intro}]{ \qforall \synx \varphi }
      \end{prooftree}
    \end{equation*}
  \end{thmenum}
\end{example}

\begin{proposition}\label{thm:syntactic_first_order_quantifiers_are_dual}
  For any formula \( \varphi \) and any variable \( \synx \) over \( \mscrL \), we have the following interderivable pairs:
  \begin{align}
    \synneg \qforall \synx \varphi &\T{and} \qexists \synx \synneg \varphi \label{thm:syntactic_first_order_quantifiers_are_dual/negation_of_universal} \\
    \synneg \qexists \synx \varphi &\T{and} \qforall \synx \synneg \varphi \label{thm:syntactic_first_order_quantifiers_are_dual/negation_of_existential}
  \end{align}
\end{proposition}
\begin{proof}
  We will only show \eqref{thm:syntactic_first_order_quantifiers_are_dual/negation_of_universal}. First,

  \begin{equation*}
    \begin{prooftree}
      \hypo{ \synneg \qforall \synx \varphi }
      \hypo{ [\qforall \synx \varphi]^1 }
      \infer2[\ref{inf:def:propositional_natural_deduction_systems/neg/elim}]{ \synbot }
      \infer[left label=\( 1 \)]1[\eqref{eq:def:propositional_natural_deduction_systemss/rules/dne}]{ \qforall \synx \varphi }
      \infer1[\eqref{eq:def:first_order_natural_deduction_system/forall/elim}]{ \varphi }

      \hypo{ [\synneg \varphi]^2 }
      \infer2[\ref{inf:def:propositional_natural_deduction_systems/neg/elim}]{ \synbot }

      \infer[left label=\( 2 \)]1[\ref{inf:def:propositional_natural_deduction_systems/neg/intro}]{ \synneg \varphi }
      \infer1[\eqref{eq:def:first_order_natural_deduction_system/exists/intro}]{ \qexists \synx \synneg \varphi }
    \end{prooftree}
  \end{equation*}

  Conversely,
  \begin{equation*}
    \begin{prooftree}
      \hypo{ \qexists \synx \synneg \varphi }

      \hypo{ [\qforall \synx \varphi]^1 }
      \infer1[\eqref{eq:def:first_order_natural_deduction_system/forall/elim}]{ \varphi }

      \hypo{ [\synneg \varphi]^2 }
      \infer2[\ref{inf:def:propositional_natural_deduction_systems/neg/elim}]{ \synbot }
      \infer[left label=\( 1 \)]1[\ref{inf:def:propositional_natural_deduction_systems/neg/intro}]{ \synneg \qforall \synx \varphi }

      \infer[left label=\( 2 \)]2[\eqref{eq:def:first_order_natural_deduction_system/exists/elim}]{ \synneg \qforall \synx \varphi }
    \end{prooftree}
  \end{equation*}
\end{proof}

\begin{theorem}[Soundness and completeness of first-order logic]\label{thm:classical_first_order_logic_soundness_and_complete}\mcite[thm. 3.4.1]{Hinman2005}
  The \hyperref[def:first_order_natural_deduction_system]{classical first-order natural deduction system} is \hyperref[def:logical_framework/soundness]{sound} and \hyperref[def:logical_framework/completeness]{complete} with respect to \hyperref[def:first_order_semantics]{classical semantics}.
\end{theorem}
\begin{comments}
  \item The completeness part is known as \enquote{G\"odel's completeness theorem} and requires an elaborate proof.
  \item The theorem also applies, with adaptations based on \fullref{rem:propositional_logic_as_first_order_logic}, to the \hyperref[def:propositional_natural_deduction_systemss]{classical propositional natural deduction system}.
\end{comments}

  \subsection{First-order models}\label{subsec:first_order_models}

\paragraph{First-order models}

\begin{definition}\label{def:propositional_model}\mcite[def. 1.4.1(i)]{Hinman2005}
  We say that the \hyperref[def:propositional_valuation/interpretation]{propositional interpretation} \( I \) is a \term{propositional model} of the set of \hyperref[def:propositional_syntax/formula]{propositional formulas} \( \Gamma \) if \( \Bracks{\varphi}_I = T \) for every formula \( \varphi \in \Gamma \).

  We also say that \( \Gamma \) is \term{valid} under \( I \), that \( I \) \term{satisfies} \( \Gamma \) and that the set \( \Gamma \) is \term{satisfiable} if there exists at least one model for it.
\end{definition}

\begin{remark}\label{rem:first_order_satisfiability_bivalence}
  We will now show why \fullref{def:propositional_model} requires some adjustments for first-order formulas.

  Given a \hyperref[def:propositional_syntax/formula]{propositional formula} \( \varphi \) and an \hyperref[def:propositional_valuation/interpretation]{interpretation} \( I \), the latter \hyperref[def:propositional_model]{satisfies} either \( \varphi \) or \( \neg \varphi \). This is the principle of bivalence discussed in \fullref{rem:classical_logic}.

  Similarly, given a \hyperref[def:first_order_valuation/variable_assignment]{first-order variable assignment} \( v \) in the \hyperref[def:first_order_structure]{structure} \( \mscrX \) and a \hyperref[def:first_order_syntax/formula]{first-order formula} \( \varphi \), either \( \Bracks{\varphi} = T \) or \( \Bracks{\neg \varphi}_v = T \).

  It is obvious how we should define satisfiability for a concrete variable assignment, but the question remains how to define it for a structure.

  Fix an arbitrary structure \( \mscrX = (X, I) \) with two elements \( X = \set{ a, b } \) and consider the formula \( \xi \doteq \eta \). Let \( v \) be any variable assignment such that \( v(\xi) = v(\eta) = a \) and let \( w \) be an assignment such that \( w(\xi) = a \) and \( w(\eta) = b \). Clearly \( \Bracks{\xi \doteq \eta}_v = T \) and \( \Bracks{\xi \doteq \eta}_w = F \). Thus, some assignments satisfy the formula and some do not.

  It makes sense to define
  \begin{equation*}
    \Bracks{\varphi}_\mscrX
    \coloneqq
    \begin{cases}
      T, \Bracks{\varphi}_v = T \T{for every assignment} v, \\
      F, \T{otherwise}.
    \end{cases}
  \end{equation*}

  This is what Peter Hinman in \incite[rem. 2.2.17]{Hinman2005} calls \term{universally valid} formulas in \( \mscrX \).

  We just saw that \( \xi \doteq \eta \) is not universally valid. But neither is \( \xi \doteq \eta \) because
  \begin{equation*}
    \Bracks{\neg (\xi \doteq \eta)}_v = \overline{\Bracks{\xi \doteq \eta}_v} = F
  \end{equation*}

  Therefore, the principle of bivalence does not hold for universal validity. In order for it to hold, what we will do in \fullref{def:first_order_model} is restrict to \hyperref[def:first_order_syntax/closed_formula]{closed formulas}, which depend only on the underlying structure and not on the variable assignments.
\end{remark}

\begin{definition}\label{def:universal_closure}\mcite[def. 2.2.28(i)]{Hinman2005}
  Given a formula \( \varphi \) with free variables \( \xi_1, \ldots, \xi_n \), we call
  \begin{equation*}
    \qforall {\xi_1} \cdots \qforall {\xi_n} \varphi
  \end{equation*}
  its \term{universal closure} and say that \( \varphi \) itself is \term{implicitly universally quantified}. Universal closures of quantifierless formulas are called \term{universal formulas}.
\end{definition}
\begin{comments}
  \item This allows us to skip quantifiers when writing formulas without changing their validity. This is useful for notational brevity and is used all across the document whenever we want \hi{closed} formulas, for example in \fullref{def:semiring/theory}, \fullref{def:semiring/theory} or \fullref{def:binary_relation}.
\end{comments}

\begin{proposition}\label{thm:implicit_universal_quantification}
  Every first-order formula is semantically equivalent to its \hyperref[def:universal_closure]{universal closure}.
\end{proposition}
\begin{comments}
  \item See \fullref{ex:def:first_order_natural_deduction_system/eigenvariables/invalid_universal_closure} for how this fails for derivability rather than entailment.
\end{comments}
\begin{proof}
  \SufficiencySubProof Let \( \mscrX = (X, I) \) be a structure that satisfies \( \varphi \). Let \( v \) be a variable assignment in \( \mscrX \). Then for any \( x \in X \), the modified variable assignment \( v_{\xi \mapsto x} \) also satisfies \( \varphi \), i.e.
  \begin{equation*}
    \Bracks{\varphi}_v = \Bracks{\varphi}_{v_{\xi \mapsto x}} = T.
  \end{equation*}

  Thus, \( \mscrX \) is also a model for \( \qforall \xi \varphi \).

  \NecessitySubProof Conversely, suppose that \( \mscrX \) satisfies \( \qforall \xi \varphi \) and \( v \) is any variable assignment. Then
  \begin{equation*}
    \Bracks{\varphi}_{v_{\xi \mapsto x}} = T
  \end{equation*}
  for any \( x \), including \( x = v(\xi) \). Thus,
  \begin{equation*}
    \Bracks{\varphi}_{v_{\xi \mapsto v(\xi)}} = \Bracks{\varphi}_v = T.
  \end{equation*}

  Therefore, \( \mscrX \) is also a model for \( \varphi \).
\end{proof}

\begin{definition}\label{def:first_order_model}\mcite[def. 2.2.13(ii)]{Hinman2005}
  We say that the \hyperref[def:first_order_structure]{first-order structure} \( \mscrX = (X, I) \) is a \term{first-order model} of the set of \hyperref[def:first_order_syntax/closed_formula]{closed first-order formulas} \( \Gamma \) if, for an arbitrary variable assignment \( v \) in \( \mscrX \), \( \Bracks{\varphi}_v = T \) for every formula \( \varphi \in \Gamma \).

  We also say that \( \Gamma \) is \term{valid} in \( \mscrX \), that \( \mscrX \) \term{satisfies} \( \Gamma \) and that the set \( \Gamma \) is \term{satisfiable} if there exists at least one model for it.
\end{definition}
\begin{comments}
  \item Since the formulas are closed, the result does not depend on the assignment. We only assignments as a technical tool because we avoid defining entailment within a structure. The problems of doing the latter are discussed in \fullref{rem:first_order_satisfiability_bivalence}.
\end{comments}

\begin{remark}\label{rem:first_order_model_notation}
  In first-order logic, \hyperref[def:first_order_structure]{structures} are defined as pairs \( \mscrX = (X, I) \). Each area of mathematics has its own conventions and structures are usually specified as simply as possible without being unambiguous (and sometimes even beyond non-ambiguity).

  A popular convention is to use compatible letters like we did with \( X \) and \( \mscrX \) or \( G \) and \( \mscrG \), where the structure itself is named using calligraphic letters while the domain is named using the corresponding capital letter in normal font. This only works very simple cases where we can say \enquote{Let \( \mscrP = (P, \leq) \) be a \hyperref[def:partially_ordered_set]{partially ordered set}}.

  The language of the \hyperref[def:group/theory]{theory of groups} has a signature consisting of three functional symbols and no predicate symbols. Specifying a structure for this language is thus the same as specifying a quadruple \( \mscrG = (G, e, (\anon)^{-1}, \cdot) \). We usually specify only the domain \( G \) and the basic structure needed to avoid ambiguity, for example \enquote{Let \( (G, \cdot) \) be a group}. This is technically wrong, but it is both convenient and conventional. The rest of the definition of the group can easily be inferred. In case of ambiguity, the simplest disambiguation is to use lower indices with the name of the structure, e.g. \( +_G \) and \( +_H \) may be the addition operation in different abelian groups.

  Furthermore, stating that \( (G, \cdot, \leq, \mscrT) \) is a totally ordered topological group is cumbersome and can even raise questions; for example, is \( \mscrT \) the \hyperref[def:order_topology]{order topology} or just an arbitrary \hyperref[rem:topological_first_order_structures]{group topology}?
\end{remark}

\begin{remark}\label{rem:questions_regarding_structures}
  Within this section, we are interested in the following questions regarding \hyperref[def:first_order_structure]{first-order structure} and \hyperref[def:first_order_model]{models}:

  \begin{itemize}
    \item Which subsets of a structure form a \hyperref[def:first_order_substructure]{substructure}?

    This is answered by \fullref{def:first_order_substructure} and by \fullref{def:first_order_generated_substructure}. Vacuously, if the language contains no functional symbols, every subset of (the domain of) a structure is a substructure. Such is the case with \hyperref[def:set]{sets} themselves, with \hyperref[def:partially_ordered_set]{partially ordered sets} or with \hyperref[def:metric_space]{metric} and \hyperref[def:topological_space]{topological spaces}.

    \Fullref{thm:substructures_form_complete_lattice} shows that the family of all substructures of a structure is worth studying in itself.

    \item Given a model of some set \( \Gamma \) of formulas, which substructures and \hyperref[def:first_order_homomorphism]{homomorphic} images of the model are again models of \( \Gamma \)?

    This is answered by \fullref{thm:positive_formulas_preserved_under_homomorphism}, \fullref{thm:arbitrary_formulas_preserved_under_isomorphisms} and \fullref{thm:functions_over_model_form_model}.
  \end{itemize}
\end{remark}

\paragraph{First-order substructures}

\begin{definition}\label{def:first_order_substructure}\mcite[def. 2.3.12]{Hinman2005}
  Let \( \mscrX = (X, I) \) be a structure for the language \( \mscrL \) and let \( Y \subseteq X \). We say that \( \mscrY = (Y, J) \) is a \term{substructure} of \( \mscrX \) if:

  \begin{thmenum}
    \thmitem{def:first_order_substructure/universe} The universe \( Y \) satisfies either of the equivalent conditions:

    \begin{thmenum}
      \thmitem{def:first_order_substructure/universe/deductive} It is closed under function application, that is, for any functional symbol \( f \) in \( \mscrL \) with arity \( n \), we have \( I(f)(Y^n) \subseteq Y \).

      \thmitem{def:first_order_substructure/universe/inductive} It is a \hyperref[def:fixed_point]{fixed point} of the operator
      \begin{equation*}
        \begin{aligned}
          &T: \pow(X) \to \pow(X), \\
          &T(A) \coloneqq A \cup \set[\Big]{ I(f)(x_1, \ldots, x_{\#f}) \given f \in \boldop{Fun} \T{and} x_1, \ldots, x_{\#f} \in A},
        \end{aligned}
      \end{equation*}
      which enlarges \( A \) with the union of all image of \( A \) under functions of the language \( \mscrL \).
    \end{thmenum}

    \thmitem{def:first_order_substructure/functional} For every functional symbol \( f \in \boldop{Fun}_\mscrL \), the interpretation \( J(f) \) is a restriction of \( I(f) \) to \( Y \).

    \thmitem{def:first_order_substructure/predicate} For every predicate symbol \( p \in \boldop{Pred}_\mscrL \), the interpretation \( J(p) \) is a restriction of \( I(p) \) to \( Y \).
  \end{thmenum}
\end{definition}
\begin{comments}
  \item We will say \enquote{\( (Y, I) \)} when we mean that \( Y \) is the domain of a substructure of \( \mscrX = (X, I) \) to avoid defining an interpretation \( J \) that merely restricts the domains and codomains from \( X \) to \( Y \).
\end{comments}
\begin{defproof}
  \ImplicationSubProof{def:first_order_substructure/universe/deductive}{def:first_order_substructure/universe/inductive} By definition of \( T \), \( Y \) if a fixed point if and only if
  \begin{equation*}
    \set[\Big]{ I(f)(x_1, \ldots, x_{\#f}) \given f \in \boldop{Fun} \T{and} x_1, \ldots, x_{\#f} \in A} \subseteq Y.
  \end{equation*}

  This condition is clearly satisfied if \( B \) satisfies \fullref{def:first_order_substructure/universe/deductive}.

  \ImplicationSubProof{def:first_order_substructure/universe/inductive}{def:first_order_substructure/universe/deductive} If, instead \( Y \) is a fixed point of \( T \), for the \( n \)-ary functional symbol \( f \in \boldop{Fun} \) and for any tuple \( x_1, \ldots, x_n \), the value \( I(f)(x_1, \ldots, x_n) \) belongs to \( Y \). Therefore, \fullref{def:first_order_substructure/universe/deductive} is satisfied.
\end{defproof}

\begin{proposition}\label{thm:substructure_relation_is_transitive}
  The \hyperref[def:first_order_substructure]{first-order substructure} relation is \hyperref[def:binary_relation/transitive]{transitive}.

  More precisely, let \( \mscrY = (Y, I) \) be a substructure of the first-order structure \( \mscrX = (X, I) \) and let \( Z \) be a subset of \( Y \).

  Then \( (Z, I) \) is a substructure of \( (Y, I) \) if and only if it is a substructure of \( (X, I) \).
\end{proposition}
\begin{proof}
  Trivial in both directions.
\end{proof}

\begin{example}\label{ex:def:first_order_substructure/vector_space}
  The classic definition for a subset \( U \) of a \hyperref[def:vector_space]{vector space} \( \mscrV \) being a vector subspace is that \( U \) is closed under \hyperref[rem:linear_combinations]{linear combinations}. Linear combinations are simply finite \hyperref[rem:function_superposition]{superpositions} of addition and scalar multiplication in \( \mscrV \). So this condition ensures that \( U \) is closed under application of the functional symbols corresponding to addition and scalar multiplication.

  See \fullref{thm:span_via_linear_combinations} for a further discussion.
\end{example}

\begin{remark}\label{rem:topological_first_order_structures}
  Let \( \mscrX = (X, I) \) be a structure over some language \( \mscrL \) without predicate symbols.

  If, for every functional symbol \( f \), the interpretation \( I(f) \) is a \hyperref[def:global_continuity]{continuous function}, we call \( \mscrX \) a \term{topological structure}.

  For every algebraic structure defined in \fullref{sec:group_theory} and \fullref{sec:ring_theory}, there exists a topological equivalent. We discuss \hyperref[def:topological_group]{topological groups} and \hyperref[def:topological_vector_space]{topological vector spaces} through the document, especially in \fullref{sec:functional_analysis}.

  Naturally, every substructure of a topological structure is again a topological structure.
\end{remark}

\begin{proposition}\label{thm:intersection_substructure}
  Fix a first-order structure \( \mscrX = (X, I) \). Let \( \seq{ (X_k, I) }_{k \in \mscrK} \) be a family of substructures of \( \mscrX \) and suppose that they are not disjoint. Then their \term{intersection substructure} \( \parens*{\bigcap_{k \in \mscrK} X_k, I} \) is again a substructure of \( \mscrX \).
\end{proposition}
\begin{comments}
  \item In case the substructures are disjoint, their intersection is empty, which is technically not the domain of a structure. If we allow empty domains, as per \fullref{rem:empty_first_order_structures}, then the condition for the substructures not to be disjoint is unnecessary.
\end{comments}
\begin{proof}
  For any functional symbol \( f \) in \( \mscrL \) with arity \( n \), we have
  \begin{equation*}
    I(f)\parens*{\parens*{\bigcap_{\smash{k \in \mscrK}} X_k}^n}
    \reloset {\ref{thm:function_image_properties/intersection}} \subseteq
    \bigcap_{k \in \mscrK} I(f)(X_k^n).
  \end{equation*}

  Therefore, \( \parens*{\bigcap_{k \in \mscrK} X_k, \bigcap_{k \in \mscrK} I_k} \) is indeed a substructure of \( \mscrX \).
\end{proof}

\begin{definition}\label{def:first_order_generated_substructure}\mimprovised
  Let \( \mscrX = (X, I) \) be a first-order structure. Given a subset \( A \) of \( X \), we define the substructure \term[bg=породена (\cite[110]{ПетровЗяпков2010}), ru=порожденная (\cite[55]{Мальцев1970})]{generated} by \( A \) as the \hyperref[thm:intersection_substructure]{intersection substructure} of all substructures whose domains contain \( A \).

  We denote the domain of this generated substructure by \( \braket{ A } \). \Fullref{thm:closure_operator_from_set_semilattice} implies that it is a \hyperref[def:moore_closure_operator]{Moore closure operator} on \( \pow(X) \).
\end{definition}
\begin{comments}
  \item In particular, as a consequence of \fullref{thm:closure_operator_minimality}, \( \braket{ A } \) is the minimum among all substructure domains containing \( A \).
\end{comments}

\begin{theorem}[Induction on generated substructures]\label{thm:induction_on_generated_substructures}
  Fix a \hyperref[def:first_order_structure]{first-order structure} \( \mscrX = (X, I) \) over \( \mscrL \) and a subset \( A \) of \( X \). Let \( \varphi \) be a formula over \( \mscrL \) with a single free variable \( \xi \).

  Then, in order for \( \Bracks{\varphi}(x) = T \) to hold for every \( x \) in the substructure \hyperref[def:first_order_generated_substructure]{generated} by \( A \), the following conditions are sufficient:
  \begin{thmenum}
    \thmitem{thm:induction_on_generated_substructures/base} Prove that \( \Bracks{\varphi}(x) = T \) for every \( x \) in \( A \).
    \thmitem{thm:induction_on_generated_substructures/step} For every functional symbol \( f \) in \( \mscrL \), prove that, given that \( n \) is the arity of \( f \), if there exists a tuple \( x_1, \ldots, x_n \) such that \( \Bracks{\varphi}(x_k) = T \) for every \( k = 1, \ldots, n \), then
    \begin{equation*}
      \Bracks{\varphi}\parens[\Big]{ I(f)(x_1, \ldots, x_n) } = T.
    \end{equation*}
  \end{thmenum}
\end{theorem}
\begin{proof}
  Suppose that both \fullref{thm:induction_on_generated_substructures/base} and \fullref{thm:induction_on_generated_substructures/step} hold. Let \( x \) be an arbitrary member of \( \braket{ A } \).

  \begin{itemize}
    \item If \( x \in A \), then \fullref{thm:induction_on_generated_substructures/base} ensures that \( \Bracks{\varphi}(x) = T \).
    \item Otherwise, there exists a functional symbol \( f \) and a tuple \( x_1, \ldots, x_n \) such that \( x = f(x_1, \ldots, x_n) \).

    Indeed, aiming at a contradiction, suppose not. Then \( \braket{ A } \setminus \set{ x } \) is also closed under application of all functions in \( \mscrL \), and hence is itself the domain of a substructure of \( \mscrX \). This contradicts the minimality of \( \braket{ A } \), however.

    Now that we have shown the existence of \( f \) and \( x_1, \ldots, x_n \), \fullref{thm:induction_on_generated_substructures/step} implies that
    \begin{equation*}
      \Bracks{\varphi}\parens[\Big]{ \underbrace{ I(f)(x_1, \ldots, x_n) }_{x} } = T.
    \end{equation*}
  \end{itemize}

  We have shown that \fullref{thm:induction_on_generated_substructures/base} and \fullref{thm:induction_on_generated_substructures/step} together imply that \( \Bracks{\varphi}(x) = T \) for every member \( x \) of \( \braket{ A } \).
\end{proof}

\begin{example}\label{ex:def:first_order_generated_substructure}
  Common examples of generated substructures are the \hyperref[def:semimodule/submodel]{linear span} discussed in \fullref{thm:span_via_linear_combinations} and the \hyperref[def:semiring_ideal/generated]{generated ring ideals}.
\end{example}

\begin{proposition}\label{thm:substructures_form_complete_lattice}
  Fix a structure \( \mscrX = (X, I) \) for the language \( \mscrL \).

  With respect to set inclusion of domains, the family of all substructures of a \( \mscrX \) forms a complete lattice. Explicitly:
  \begin{thmenum}
    \thmitem{thm:substructures_form_complete_lattice/join} The \hyperref[def:semilattice/join]{join} of the family of substructures \( \seq{ (X_k, I) }_{k \in \mscrK} \) is the \hyperref[def:first_order_generated_substructure]{generated substructure} of the set \( \bigcup_{k \in \mscrK} X_k \).

    \thmitem{thm:substructures_form_complete_lattice/top} The \hyperref[def:extremal_points/top_and_bottom]{top element} is the structure \( \mscrX \) itself. Any substructures that are different from \( \mscrX \) are called \term{proper}.

    \thmitem{thm:substructures_form_complete_lattice/meet} The \hyperref[def:semilattice/meet]{meet} of the family of substructures \( \seq{ (X_k, I) }_{k \in \mscrK} \) is simply the \hyperref[thm:intersection_substructure]{intersection substructure} \( \parens*{\bigcap_{k \in \mscrK} X_k, I} \).

    \thmitem{thm:substructures_form_complete_lattice/bottom} The \hyperref[def:extremal_points/top_and_bottom]{bottom element} of this lattice is the intersection of all substructures.
  \end{thmenum}
\end{proposition}
\begin{comments}
  \item The initial substructure may or may not be isomorphic to the \hyperref[rem:trivial_object]{trivial structure}.

  \item As discussed in \fullref{rem:empty_first_order_structures}, the empty set is not allowed to be the domain of a structure by definition. Nevertheless, for the sake of having a bottom element we allow structures with empty domains in this lattice.
\end{comments}
\begin{proof}
  \SubProofOf{thm:substructures_form_complete_lattice/join} Let \( (Y, I) \) be the generated substructure of the set \( A \coloneqq \bigcup_{k \in \mscrK} X_k \). From \fullref{thm:closure_operator_minimality} it follows that out of the domains of all substructures of \( \mscrX \), \( Y \) is the smallest that contains \( A \) and hence the smallest that contains \( X_k \) for all \( k \in \mscrK \). Therefore, it is indeed the supremum of the family \( \seq{ (X_k, I) }_{k \in \mscrK} \) with respect to set inclusion of domains.

  \SubProofOf{thm:substructures_form_complete_lattice/top} Since \( \mscrX \) is a substructure of itself, it is not only the supremum of the entire lattice, but actually the maximum.

  \SubProofOf{thm:substructures_form_complete_lattice/meet} The domain of the intersection substructure of the family \( \seq{ (X_k, I_k) }_{k \in \mscrK} \) of substructures of \( \mscrX \) is the infimum of the family as a consequence of the equivalence in \fullref{def:first_order_generated_substructure}.

  \SubProofOf{thm:substructures_form_complete_lattice/bottom} It follows from \fullref{thm:substructures_form_complete_lattice/meet} that the bottom element is the intersection of all substructures of \( \mscrX \).
\end{proof}

\paragraph{Positive formulas}

\begin{definition}\label{def:positive_formula}\mcite[def. 3.5.67]{Hinman2005}
  The following grammar rule, extending the \hyperref[def:first_order_syntax/grammar_schema]{grammar schema of first-order logic}, describes what we call \term{positive formulas}:
  \begin{bnf*}
    \bnfprod{positive formula} {\bnftsq{\( \top \)} \bnfor} \\
    \bnfmore                   {\bnfpn{atomic formula} \bnfor} \\
    \bnfmore                   {\bnftsq{(} \bnfsp \bnfpn{formula} \bnfsp \bnftsq{\( \vee \)} \bnfsp \bnfpn{formula} \bnfsp \bnftsq{)} \bnfor} \\
    \bnfmore                   {\bnftsq{(} \bnfsp \bnfpn{formula} \bnfsp \bnftsq{\( \wedge \)} \bnfsp \bnfpn{formula} \bnfsp \bnftsq{)} \bnfor} \\
    \bnfmore                   {\bnfpn{quantifier} \bnfsp \bnfpn{variable} \bnfsp \bnftsq{.} \bnfsp \bnfpn{formula}}
  \end{bnf*}
\end{definition}
\begin{comments}
  \item The point of positive formulas is to avoid \hyperref[def:propositional_language/negation]{negation \( \neg \)}. We also avoid \( \rightarrow \) because, assuming classical logic, \fullref{thm:boolean_equivalences/conditional_as_disjunction} would then allow us to introduce negation.

  \item Positive formulas are used in \fullref{thm:positive_formulas_preserved_under_homomorphism}, which fails to hold for some non-positive formulas --- see \fullref{ex:monoid_cancellation_not_preserved_by_homomorphism}.

  \item When dealing with first-order logic, we simply use \hyperref[thm:first_order_substitution_equivalence/propositional]{substitution} to replace propositional variables with atomic formulas. This way we obtain positive first-order formulas with \hyperref[thm:implicit_universal_quantification]{implicit universal quantification}. Of course, we can always add explicit universal quantifiers, but we avoid existential quantifiers because of \fullref{thm:first_order_quantifiers_are_dual}.
\end{comments}

\begin{definition}\label{def:first_order_submodel}
  Given a \hyperref[def:first_order_model]{model} \( \mscrX \) of the set of \hyperref[def:first_order_syntax/closed_formula]{closed formulas} \( \Gamma \), we say that a \hyperref[def:first_order_substructure]{substructure} \( \mscrY \) of \( \mscrX \) is a \term{submodel} (with respect to \( \Gamma \)) if \( \mscrY \) is also a model of \( \Gamma \).
\end{definition}
\begin{comments}
  \item In this document, \( \Gamma \) is usually unambiguously clear from the context.
\end{comments}

\begin{proposition}\label{thm:substructure_is_model}
  If \( \Gamma \) is a set of closed \hyperref[def:positive_formula]{positive formulas} \hi{without existential quantifiers}, any \hyperref[def:first_order_substructure]{substructure} of a model of \( \Gamma \) is again a model of \( \Gamma \), i.e. a \hyperref[def:first_order_submodel]{first-order submodel}.
\end{proposition}
\begin{comments}
  \item See \fullref{ex:replacing_functional_symbols_via_relations} for an example of how this may fail if there are existential quantifiers.
\end{comments}
\begin{proof}
  This proposition can be proven by a straightforward application of \fullref{thm:induction_on_syntax_trees}. In each case of the induction, we do not require any elements in addition to those already present in the substructure.
\end{proof}

\begin{proposition}\label{thm:functions_over_model_form_model}
  Let \( \Gamma \) to be a set of closed \hyperref[def:positive_formula]{positive formulas} \hi{without existential quantifiers}. Let \( \mscrX = (X, I) \) be a model of \( \Gamma \) and let \( A \) be a nonempty \hyperref[def:set]{plain set}, possibly unrelated to \( \mscrX \). Consider the set \( Y \coloneqq \fun(A, \mscrX) \) of \hyperref[def:function]{all set functions} from \( A \) to \( X \).

  Define \( \iota: X \mapsto Y \) by sending each \( x \in X \) to the corresponding constant function in \( Y \).

  Define the interpretation \( J \) as follows:
  \begin{itemize}
    \item For each \( n \)-ary functional symbol \( f \) in \( \mscrL \), define the interpretation of the functions \( y_1, \ldots, y_n \) componentwise as
    \begin{equation*}
      \begin{aligned}
        &J(f): Y^n \to Y \\
        &J(f) \parens[\Big]{ y_1, \ldots, y_n } \coloneqq \parens[\Big]{ s \mapsto I(f) \parens[\Big]{ y_1(s), \ldots, y_n(s) } }.
      \end{aligned}
    \end{equation*}

    \item For each \( n \)-ary predicate symbol \( p \) in \( \mscrL \), define \( J(p) \subseteq Y^n \) via
    \begin{equation*}
      \begin{aligned}
        &J(p): Y^n \to \set{ T, F } \\
        &J(p) \parens[\Big]{ y_1, \ldots, y_n } \coloneqq \bigwedge_{s \in S} I(p) \parens[\Big]{ y_1(s), \ldots, y_n(s) }.
      \end{aligned}
    \end{equation*}

    This way \( J(p) (y_1, \ldots, y_n) = T \) if and only if \( I(p) (y_1(s), \ldots, y_n(s)) = T \) simultaneously for all \( s \in S \).
  \end{itemize}

  Then the structure \( \mscrY = (Y, J) \) is also a model of \( \Gamma \).
\end{proposition}
\begin{proof}
  It is obvious that \( \mscrY \) is a structure.

  Let \( \varphi \) be a positive formula and suppose that it is valid in \( \mscrX \). We will use \fullref{thm:induction_on_syntax_trees} to show that \( \Bracks{\varphi}_w = T \) for a fixed variable assignment \( w \) in \( \mscrY \).
  \begin{itemize}
    \item If \( \varphi = \top \), its valuation does not depend on \( w \) and thus \( \Bracks{\varphi}_w = T \).

    \item If \( \varphi = \tau_1 \doteq \tau_2 \), then \( \Bracks{\tau_1}_v = \Bracks{\tau_2}_v \) for every assignment \( v \) in \( \mscrX \), hence for any \( s \in S \) we have \( \Bracks{\tau_1}_w(s) = \Bracks{\tau_2}_w(s) \) since both sides of the latter equality here are elements of \( \mscrX \).

    \item Similarly, if \( \varphi = p(\tau_1, \ldots, \tau_n) \), then
    \begin{equation*}
      J(p) \parens[\Big]{ y_1, \ldots, y_n }
      =
      \bigwedge \set[\Big]{ I(p) \parens[\Big]{ y_1(s), \ldots, y_n(s) } \given* s \in S }
      =
      \bigwedge \set{ T \given s \in S }
      =
      T.
    \end{equation*}

    \item Analogous to our proof of \fullref{thm:positive_formulas_preserved_under_homomorphism}, conjunction and disjunction formulas that are valid in \( \mscrX \) are valid in \( \mscrY \).

    \item If \( \varphi = \qforall \xi \psi \) and the inductive hypothesis holds for \( \psi \), then
    \begin{equation*}
      \Bracks{\varphi}_w
      =
      \bigwedge_{y \in Y} \Bracks{\psi}_{w_{\xi \to y}}
      \geq
      \bigwedge_{\substack{\T{const.} \\ \T{func.} y}} \Bracks{\psi}_{w_{\xi \to y}}
      =
      \bigwedge_{x \in X} \Bracks{\psi}_{v_{\xi \to x}}
      =
      \bigwedge_{x \in X} T
      =
      T.
    \end{equation*}
  \end{itemize}
\end{proof}

\begin{example}\label{ex:thm:functions_over_model_of_positive_formulas_form_model}
  While the statement of \fullref{thm:functions_over_model_form_model} may be a little cryptic, a few examples show that it is actually obvious.
  \begin{itemize}
    \item \hyperref[def:boolean_operator]{Boolean operators} have their values in the Boolean algebra \( \set{ T, F } \). Let \( S \) be the set of all tuples of values in \( \set{ T, F }^n \) for arbitrary \( n \). That is,
    \begin{equation*}
      S \coloneqq \bigcup_{n \geq 1} \set{ T, F }^n.
    \end{equation*}

    Then from \fullref{thm:functions_over_model_form_model} it follows that the set \( B = \fun(S, \set{ T, F }) \) of all Boolean operators of arbitrary arities is again a Boolean algebra. See \fullref{thm:lindenmaum_tarski_algebra_of_full_propositional_logic/bijection} for further discussion.

    \item If \( R \) is a \hyperref[def:ring]{ring} and \( A \) is any set, then \( \fun(A, R) \) is again a ring with componentwise operations --- see \fullref{thm:functions_over_algebra}.

    This is useful in functional analysis where we study real-valued and complex-valued functions over arbitrary sets.

    \item If \( \BbbK \) is a \hyperref[def:field]{field}, then in general \( \fun(A, \BbbK) \) is not a field. The simplest example are the real-valued real functions --- \( \sin(x) \) has no multiplicative inverse since \( 1 / \sin(x) \) is not defined for \( x = 2k\pi, k = 1, 2, \ldots \). We can form a \hyperref[thm:field_of_fractions]{field of fractions}, but in general fields of fractions over function rings no longer correspond to functions --- they are purely algebraic constructions, just like \hyperref[def:formal_power_series]{formal power series}.

    This happens because the definition of a field and has an axiom with an existential quantifier --- it requires every nonzero element to have a multiplicative inverse, which can be described formally as
    \begin{equation*}
      \qforall \xi \parens[\Big]{ (\xi \doteq 0) \vee \qexists \eta (\xi \cdot \eta \doteq 1) }.
    \end{equation*}
  \end{itemize}
\end{example}

\paragraph{First-order homomorphisms}

\begin{definition}\label{def:first_order_homomorphism}\mcite[def. 2.3.26(i); rem. 2.3.27]{Hinman2005}
  Let \( \mscrX = (X, I) \) and \( \mscrY = (Y, J) \) be structures over \( \mscrL \). We say that the \hyperref[def:function]{set-theoretic function} \( h: X \to Y \) is a \term{homomorphism} between \( \mscrX \) and \( \mscrY \) if it preserves all logical functions and predicates of \( \mscrL \). Explicitly:
  \begin{thmenum}
    \thmitem{def:first_order_homomorphism/functions} For any functional symbol \( f \) of arity \( n \) and any tuple \( x_1, \ldots, x_n \in X \) we have
    \begin{equation}\label{eq:def:first_order_homomorphism/functions}
      h\parens[\Big]{ I(f)(x_1, \ldots, x_n) } = J(f) \parens[\Big]{ h(x_1), \ldots, h(x_n) }
    \end{equation}

    \thmitem{def:first_order_homomorphism/predicates} For any predicate symbol \( p \) of arity \( n \) and any \( x_1, \ldots, x_n \in X \),
    \begin{equation}\label{eq:def:first_order_homomorphism/predicates}
      I(p) (x_1, \ldots, x_n) = T \T{implies} J(p) \parens[\Big]{ h(x_1), \ldots, h(x_n) } = T.
    \end{equation}
  \end{thmenum}
\end{definition}
\begin{comments}
  \item We sometimes use the notation \( h: \mscrX \to \mscrY \).
  \item \Fullref{thm:first_order_homomorphism_as_substructure} provides a condition equivalent to \fullref{def:first_order_homomorphism/functions}.
  \item Note that the condition \fullref{def:first_order_homomorphism/predicates} for predicates is strictly weaker than
  \begin{equation*}
    I(p) (x_1, \ldots, x_n) = J(p) \parens[\Big]{ h(x_1), \ldots, h(x_n) },
  \end{equation*}
  which is used in \fullref{def:first_order_embedding} to define embeddings.

  Peter Hinman in \incite[def. 2.3.26(i)]{Hinman2005} requires the latter condition, and then in \cite[rem. 2.3.27]{Hinman2005} calls \term{positive homomorphism} out definition.

  We see this as unnecessarily complicated since most of our use cases like \hyperref[def:partially_ordered_set]{partially ordered sets} and \hyperref[def:directed_graph]{simple graphs} are covered precisely by (what we just called) positive homomorphisms.
\end{comments}

\begin{proposition}\label{thm:def:first_order_homomorphism}
  \hyperref[def:first_order_homomorphism]{First-order homomorphisms} have the following basic properties:
  \begin{thmenum}
    \thmitem{thm:def:first_order_homomorphism/inclusion} If \( \mscrX = (X, I) \) is a structure and \( \mscrY = (Y, I) \) is a \hyperref[def:first_order_substructure]{substructure} of \( \mscrX \), then the \term{canonical inclusion} function
    \begin{equation}\label{thm:def:first_order_homomorphism/inclusion/canonical_inclusion}
      \begin{aligned}
        &\iota: Y \to X \\
        &\iota(y) \coloneqq y
      \end{aligned}
    \end{equation}
    is indeed a homomorphism (and thus an embedding in the sense of \fullref{def:first_order_embedding}).

    \thmitem{thm:def:first_order_homomorphism/image_is_substructure} Given a homomorphism \( h: \mscrX \to \mscrY \), the \hyperref[def:set_valued_map/image]{image} of \( \mscrX \) under \( h \) is a substructure of \( \mscrY \).

    \thmitem{thm:def:first_order_homomorphism/preimage_is_substructure} Given a homomorphism \( h: \mscrX \to \mscrY \) and a substructure \( \mscrZ \) of \( \mscrY \), the \hyperref[def:set_valued_map/inverse]{preimage} of \( \mscrZ \) under \( h \) is a substructure of \( \mscrX \).

    \thmitem{thm:def:first_order_homomorphism/composition} The \hyperref[def:set_valued_map/composition]{composition} of two homomorphisms is again a homomorphism.

    \thmitem{thm:def:first_order_homomorphism/term_valuation} Fix a homomorphism \( h: X \to Y \) and a term \( \tau \). For any variable assignments \( v \) and \( w \) such that \( w(\xi) = h(v(\xi)) \) for all \( \xi \in \boldop{Var}(\tau) \), we have
    \begin{equation*}
      h(\Bracks{\tau}_v) = \Bracks{\tau}_w.
    \end{equation*}
  \end{thmenum}
\end{proposition}
\begin{proof}
  \SubProofOf{thm:def:first_order_homomorphism/inclusion} The interpretation in the substructure \( \mscrY \) is the \hyperref[def:set_valued_map/restriction]{restriction} \( J \) of \( I \) to \( Y \). Thus, \( \mscrY = (Y, J) \) is a structure. Conditions \fullref{def:first_order_homomorphism/functions} and \fullref{def:first_order_homomorphism/predicates} are thus satisfied. Hence, \( \iota \) is a homomorphism.

  \SubProofOf{thm:def:first_order_homomorphism/image_is_substructure} We must show that the image \( h[X] \) satisfies \fullref{def:first_order_substructure/universe/deductive} and thus the domain of a substructure of \( \mscrY \).

  Indeed, due to \fullref{def:first_order_homomorphism/functions}, for any \( n \)-ary functional symbol and any \( n \)-tuple \( x_1, \ldots, x_n \) from \( X \), we have that
  \begin{equation*}
    J(f) \parens[\Big]{ h(x_1), \ldots, h(x_n) }
    \reloset {\ref{def:first_order_homomorphism/functions}} =
    h\parens[\Big]{ I(f)(x_1, \ldots, x_n) }
    \reloset {\ref{def:first_order_substructure/universe/deductive}} \in
    h[X].
  \end{equation*}

  \SubProofOf{thm:def:first_order_homomorphism/preimage_is_substructure} We must show that \( h^{-1}[Z] \) satisfies \fullref{def:first_order_substructure/universe/deductive}.

  For any \( n \)-ary functional symbol and any \( n \)-tuple \( x_1, \ldots, x_n \) from \( h^{-1}[Z] \), the value
  \begin{equation*}
    J(f) \parens[\Big]{ h(x_1), \ldots, h(x_n) }
    \reloset {\ref{def:first_order_homomorphism/functions}} =
    h\parens[\Big]{ I(f) (x_1, \ldots, x_n) }
  \end{equation*}
  is in \( Z \), hence \( I(f) (x_1, \ldots, x_n) \) is in \( h^{-1}[Z] \).

  \SubProofOf{thm:def:first_order_homomorphism/composition} Let \( h: \mscrX \mapsto \mscrY \) and \( l: \mscrY \mapsto \mscrZ \) both be homomorphisms. Then \( l \bincirc h: \mscrX \to \mscrY \) is a homomorphism:

  \begin{itemize}
    \item \Fullref{def:first_order_homomorphism/functions} is satisfied because for any \( n \)-ary functional symbol \( f \) and any tuple \( x_1, \ldots, x_n \in X \),
    \small
    \begin{equation*}
      (l \bincirc h) \parens[\Big]{ I(f)(x_1, \ldots, x_n) }
      \reloset {\ref{def:first_order_homomorphism/functions}} =
      l\parens[\Big]{ J(f) \parens[\Big]{ h(x_1), \ldots, h(x_n) } }
      \reloset {\ref{def:first_order_homomorphism/functions}} =
      I_{\mscrZ}(f) \parens[\Big]{ (l \bincirc h)(x_1), \ldots, (l \bincirc h)(x_n) }.
    \end{equation*}
    \normalsize

    \item \Fullref{def:first_order_homomorphism/predicates} is satisfied because for any \( n \)-ary predicate symbol \( p \) and any tuple \( x_1, \ldots, x_n \in X \),
    \begin{equation*}
      I(p) (x_1, \ldots, x_n)
      \reloset {\ref{def:first_order_homomorphism/predicates}} =
      J(p) \parens[\Big]{ h(x_1), \ldots, h(x_n) }
      \reloset {\ref{def:first_order_homomorphism/predicates}} =
      I_{\mscrZ}(p) \parens[\Big]{ (l \bincirc h)(x_1), \ldots, (l \bincirc h)(x_n) }.
    \end{equation*}
  \end{itemize}

  \SubProofOf{thm:def:first_order_homomorphism/term_valuation} We use induction on the structure of \( \tau \).
  \begin{itemize}
    \item If \( \tau \) is a variable, the statement is obvious from the compatibility condition for \( v \) and \( w \).
    \item If \( \tau = f(\kappa_1, \ldots, \kappa_m) \), then
    \begin{align*}
      \Bracks{\tau}_w
      &=
      I(f) \parens[\Big]{ \Bracks{\kappa_1}_w, \ldots, \Bracks{\kappa_m}_w }
      = \\ &=
      I(f) \parens[\Big]{ h(\Bracks{\kappa_1}_v), \ldots, h(\Bracks{\kappa_m}_v) }
      \reloset {\ref{def:first_order_homomorphism/functions}} = \\ &=
      h\parens[\Big]{ I(f) \parens[\Big]{ \Bracks{\kappa_1}_v, \ldots, \Bracks{\kappa_m}_v } }
      = \\ &=
      h(\Bracks{\tau}_v).
    \end{align*}
  \end{itemize}
\end{proof}

\begin{proposition}\label{thm:positive_formulas_preserved_under_homomorphism}
  A \hyperref[def:first_order_homomorphism]{first-order homomorphism} preserves models of closed \hyperref[def:positive_formula]{positive formulas}.

  More concretely, given some language \( \mscrL \), if \( \Gamma \) is a set of closed positive formulas and \( h: \mscrX \to \mscrY \) is a homomorphism between \hyperref[def:first_order_model]{models} of \( \Gamma \), then the image \( h[X] \) is also (the domain of) a model of \( \Gamma \).
\end{proposition}
\begin{comments}
  \item Compare this to \fullref{thm:arbitrary_formulas_preserved_under_isomorphisms}, which places restrictions on the homomorphism rather than the formulas.
\end{comments}
\begin{proof}
  Since the formulas in \( \Gamma \) are closed, we are free to choose any variable assignment we like. Via \fullref{thm:induction_on_syntax_trees} on an arbitrary positive formula \( \varphi \), we will show that, for every variable assignment \( w \) in \( \mscrY \), there exists an assignment \( v \) in \( \mscrX \) such that \( \Bracks{\varphi}_v = T \) implies \( \Bracks{\varphi}_w = T \).

  Let \( w \) a variable assignment in \( \mscrY \). Let \( v \) be an assignment in \( \mscrX \) such that, for any variable \( \xi \), we have
  \begin{equation*}
    h(v(\xi)) = w(\xi).
  \end{equation*}

  The \hyperref[def:zfc/choice]{axiom of choice} guarantees the existence of such an assignment. Furthermore, if \( h \) is injective, this assignment is unique.

  The inductive hypothesis is more general than \enquote{\( \Bracks{\varphi}_v = T \) implies \( \Bracks{\varphi}_w = T \)} because of the more complicated case of quantified formulas. Nonetheless, given our fixed assignments \( v \) and \( w \), the inductive conclusion in turn implies that the image \( h[X] \) satisfies any closed positive formula if \( \mscrX \) does, in particular the formulas from \( \Gamma \).

  \begin{itemize}
    \item The constant \( \top \) is vacuously preserved by homomorphisms because it does not depend on the interpretation or variable assignment.

    \item Suppose that \( \varphi = \tau_1 \doteq \tau_2 \). We have \( \Bracks{\tau_1}_v = \Bracks{\tau_2}_v \) and hence
    \begin{equation*}
      \Bracks{\tau_1}_w
      \reloset {\ref{thm:def:first_order_homomorphism/term_valuation}} =
      h(\Bracks{\tau_1}_v)
      =
      h(\Bracks{\tau_2}_v)
      \reloset {\ref{thm:def:first_order_homomorphism/term_valuation}} =
      \Bracks{\tau_2}_w.
    \end{equation*}

    \item Suppose that \( \varphi \) is the predicate formula \( p(\tau_1, \ldots, \tau_n) \). By assumption, \( \Bracks{p(\tau_1, \ldots, \tau_n)}_v = T \). Then
    \begin{equation}\label{eq:thm:positive_formulas_preserved_under_homomorphism/predicates/x}
      I(p) \parens[\Big]{ \Bracks{\tau_1}_v, \ldots, \Bracks{\tau_n}_v } = T.
    \end{equation}

    By the definition of homomorphism, this implies
    \begin{equation}\label{eq:thm:positive_formulas_preserved_under_homomorphism/predicates/y}
      J(p) \parens[\Big]{ \underbrace{h(\Bracks{\tau_1}_v)}_{\Bracks{\tau_1}_w}, \ldots, \underbrace{h(\Bracks{\tau_n}_v)}_{\Bracks{\tau_n}_w} } = T.
    \end{equation}

    Now \( \Bracks{p(\tau_1, \ldots, \tau_n)}_w = T \) follows from \fullref{thm:def:first_order_homomorphism/term_valuation}.

    \item Suppose that \( \varphi = \psi_1 \wedge \psi_2 \), where \( \psi_1 \) and \( \psi_2 \) are positive formulas, and that the inductive hypothesis holds for \( \psi_1 \) and \( \psi_2 \).

    Since \( \Bracks{\varphi}_v = T \) by assumption, by definition of valuation of conjunction we have
    \begin{equation*}
      \Bracks{\psi_1}_v
      =
      \Bracks{\psi_2}_v
      =
      T.
    \end{equation*}

    This allows us to apply the inductive hypothesis to obtain
    \begin{equation*}
      \Bracks{\psi_1}_w
      =
      \Bracks{\psi_2}_w
      =
      T.
    \end{equation*}
    and conclude that
    \begin{equation*}
      \Bracks{\varphi}_w
      =
      \Bracks{\psi_1}_w \wedge \Bracks{\psi_2}_w
      =
      T \wedge T
      =
      T.
    \end{equation*}

    \item Suppose that \( \varphi = \psi_1 \vee \psi_2 \), where \( \psi_1 \) and \( \psi_2 \) are positive formulas, and that the inductive hypothesis holds for \( \psi_1 \) and \( \psi_2 \).

    Since the formula \( \varphi \) is valid in \( \mscrX \), at least one of \( \psi_1 \) or \( \psi_2 \) is valid under \( v \), hence
    \begin{equation}\label{eq:thm:positive_formulas_preserved_under_homomorphism/sup_of_disjunction}
      \Bracks{\varphi}_v = \sup\set{ \Bracks{\psi_1}_v, \Bracks{\psi_2}_v } = T.
    \end{equation}

    The values \( \Bracks{\psi_1}_v \) and \( \Bracks{\psi_2}_v \) may differ between \( \psi_1 \) and \( \psi_2 \), but \eqref{eq:thm:positive_formulas_preserved_under_homomorphism/sup_of_disjunction} always holds.

    The inductive hypothesis holds for both \( \psi_1 \) and \( \psi_2 \), therefore
    \begin{equation*}
      \Bracks{\varphi}_w = \sup\set{ \Bracks{\psi_1}_w, \Bracks{\psi_2}_w } = T.
    \end{equation*}

    \item Suppose that \( \varphi = \qforall \xi \psi \), where \( \psi \) is a positive formula for which the inductive hypothesis holds.

    Fix some \( x \in X \). Since \( \Bracks{\varphi}_v = T \), then \( \Bracks{\psi}_{v_{\xi \to x}} = T \), hence the inductive hypothesis implies \( \Bracks{\psi}_{w_{\xi \to x}} = T \). This happens for all \( x \in X \), therefore \( \Bracks{\varphi}_w = T \).

    \item Finally, suppose that \( \varphi = \qexists \xi \psi \), where again \( \psi \) is a positive formula for which the inductive hypothesis holds.

    Since \( \Bracks{\varphi}_v = T \), there exists some \( x \in X \) such that \( \Bracks{\psi}_{v_{\xi \to x}} = T \), hence the inductive hypothesis implies \( \Bracks{\psi}_{w_{\xi \to x}} = T \). Therefore, \( \Bracks{\varphi}_w = T \).
  \end{itemize}
\end{proof}

\begin{example}\label{ex:thm:positive_formulas_preserved_under_homomorphism/proof_failure}
  To see how our proof of \fullref{thm:positive_formulas_preserved_under_homomorphism} fails for conditionals, consider the formula \( \varphi = \psi_1 \rightarrow \psi_2 \), where the inductive hypothesis holds for \( \psi_1 \) and \( \psi_2 \).

  If \( \Bracks{\varphi}_v = T \), then one of the following holds:
  \begin{itemize}
    \item \( \Bracks{\psi_1}_v = \Bracks{\psi_2}_v = T \), which via the inductive hypothesis implies \( \Bracks{\psi_1}_w = \Bracks{\psi_2}_w = T \) and hence \( \Bracks{\varphi}_w = T \).
    \item \( \Bracks{\psi_1}_v = F \), which doesn't imply \( \Bracks{\psi_1}_w = F \). In this case it is possible to have \( \Bracks{\psi_1}_w = T \) and \( \Bracks{\psi_2}_w = F \), which implies \( \Bracks{\varphi}_w = F \). This is the negation of what we need to prove.
  \end{itemize}

  What if we change our inductive hypothesis to \( \Bracks{\varphi}_v = \Bracks{\varphi}_w \)\footnote{This is precisely what we will do in the case of \hyperref[def:first_order_embedding]{embeddings} in \fullref{thm:arbitrary_formulas_preserved_under_isomorphisms}}? This fails for atomic formulas because \( \Bracks{\tau_1 \doteq \tau_2}_v = F \) doesn't imply \( \Bracks{\tau_1 \doteq \tau_2}_w = F \) --- consider the case of a group homomorphism from \( (\BbbR, +) \) to \( (\set{ 0 }, +) \). We can this of this alternative inductive hypothesis as a semantical way of introducing negation, which turns our to cause problems.

  See \fullref{ex:monoid_cancellation_not_preserved_by_homomorphism} for an example where a conditional is not preserved by a homomorphism.
\end{example}

\begin{example}\label{ex:replacing_functional_symbols_via_relations}
  Consider the \hyperref[def:semigroup/theory]{theory of semigroups}. We have a functional symbol \( \cdot \), which we can also represent via the ternary predicate \( p(\xi, \eta, \zeta) \), which holds for \( (x, y, z) \) in some model \( \mscrX \) if and only if \( x \cdot y = z \).

  If we choose to work only with the relation, the language would not have any functional symbols, and every subset of (the domain of) \( \mscrX \) would be a substructure.

  This also introduces a complication, however. We must ensure that the relation represents a function, and this can be done via the axiom
  \begin{equation*}
    \qforall \xi \qforall \eta \qExists \zeta p(\xi, \eta, \zeta),
  \end{equation*}
  where we have used the unique existence shorthand from \fullref{ex:replacing_functional_symbols_via_relations}.

  In this setting, a model of the theory of semigroups must satisfy this axiom, and thus it is possible for a substructure not to be a model.

  For example, the negative real numbers are not a semigroup under multiplication because the product of two negative numbers is positive. A functional symbol encodes this requirement into the definition of a substructure. But otherwise we must encode this via formulas, which makes the definition of substructure trivial, but now it is possible for a substructure not to be a model.
\end{example}

\paragraph{First-order direct products}

\begin{definition}\label{def:first_order_direct_product}\mcite[def. 2.3.54]{Hinman2005}
  Fix a family of \hyperref[def:first_order_structure]{first-order structures} \( \seq{ \mscrX_k }_{k \in \mscrK} \) over a common language \( \mscrL \), where \( \mscrX_k = (X_k, I_k) \).

  We call the \term{direct product} of this family the structure \( (X, I) \) denoted by \( \prod_{k \in \mscrK} \mscrX_k \) (or \( \mscrX^\mscrK \) if all structures are equal to \( \mscrX \)) and defined as follows:
  \begin{thmenum}
    \thmitem{def:first_order_direct_product/domain} The domain \( X \) is the \hyperref[def:cartesian_product]{Cartesian product} \( \prod_{k \in \mscrK} X_k \) of the corresponding domains.

    \thmitem{def:first_order_direct_product/functions} For every \( n \)-ary functional symbol \( f \) in \( \mscrL \), we define its interpretation componentwise:
    \begin{equation*}
      \begin{aligned}
        &I(f): \prod_{k \in \mscrK} \mscrX_k \to X, \\
        &I(f)\parens[\Big]{ \seq{ x_{1,k} }_{k \in \mscrK}, \ldots, \seq{ x_{n,k} }_{k \in \mscrK} } = \seq[\Big]{ I_k(f)(x_{1,k}, \ldots, x_{n,k}) }_{k \in \mscrK}. \\
      \end{aligned}
    \end{equation*}

    \thmitem{def:first_order_direct_product/predicates} For every \( n \)-ary predicate symbol \( p \) in \( \mscrL \), we define its interpretation componentwise:
    \begin{equation*}
      \begin{aligned}
        &I(p): \prod_{k \in \mscrK} \mscrX_k \to \set{ T, F }, \\
        &I(p)\parens[\Big]{ \seq{ x_{1,k} }_{k \in \mscrK}, \ldots, \seq{ x_{n,k} }_{k \in \mscrK} } = \bigwedge_{k \in \mscrK} I_k(p)(x_{1,k}, \ldots, x_{n,k}). \\
      \end{aligned}
    \end{equation*}
  \end{thmenum}
\end{definition}
\begin{comments}
  \item The direct product is surprisingly useful --- we use it to characterize homomorphisms and congruences as substructures. See \fullref{thm:first_order_homomorphism_as_substructure}.
\end{comments}

\begin{proposition}\label{thm:direct_product_projections}
  Given the \hyperref[def:first_order_direct_product]{first-order direct product} \( \prod_{k \in \mscrK} \mscrX_k \), for every \( m \in \mscrK \), the following \term{canonical projection} function is a \hyperref[def:first_order_homomorphism]{first-order homomorphism}:
  \begin{equation*}
    \begin{aligned}
      &\pi_m: \prod_{k \in \mscrK} \mscrX_k \to \mscrX_m, \\
      &\pi_m(\seq{ x_k }_{k \in \mscrK}) \coloneqq x_m.
    \end{aligned}
  \end{equation*}
\end{proposition}
\begin{proof}
  Let \( I \) be the interpretation of the product and \( I_k \) --- of \( \mscrX_k \). For every \( n \)-ary functional symbol \( f \) in \( \mscrL \), we have
  \begin{align*}
    &\phantom{{}={}}
    \pi_m\parens[\Big]{ I(f)\parens[\Big]{ \seq{ x_{1,k} }_{k \in \mscrK}, \ldots, \seq{ x_{n,k} }_{k \in \mscrK} } }
    = \\ &=
    \pi_m\parens[\Big]{ \seq[\Big]{ I_k(f)\parens{ x_{1,k}, \ldots, x_{n,k} } }_{k \in \mscrK} }
    = \\ &=
    I_m(f)\parens{ x_{1,m}, \ldots, x_{n,m} }
    = \\ &=
    I_m(f)\parens[\Big]{ \pi_m\parens[\Big]{ \seq{ x_{1,k} }_{k \in \mscrK} }, \ldots, \pi_m\parens[\Big]{ \seq{ x_{n,k} }_{k \in \mscrK} } }.
  \end{align*}

  This shows \fullref{def:first_order_homomorphism/functions}.

  Now fix an \( n \)-ary predicate symbol \( p \) in \( \mscrL \) and suppose that
  \begin{equation*}
    I(p)\parens[\Big]{ \seq{ x_{1,k} }_{k \in \mscrK}, \ldots, \seq{ x_{n,k} }_{k \in \mscrK} } = T.
  \end{equation*}

  Then \fullref{def:first_order_direct_product/predicates} implies that
  \begin{equation*}
    I_{k_0}(p)(x_{1,m}, \ldots, x_{n,m}) = T.
  \end{equation*}

  Then
  \begin{equation*}
    I_{k_0}(p)\parens[\Big]{ \pi_m\parens[\Big]{ \seq{ x_{1,k} }_{k \in \mscrK} }, \ldots, \pi_m\parens[\Big]{ \seq{ x_{n,k} }_{k \in \mscrK} } }
    =
    I_{k_0}(p)(x_{1,m}, \ldots, x_{n,m})
    =
    T.
  \end{equation*}

  This shows \fullref{def:first_order_homomorphism/predicates}.

  Therefore, \( \pi_m \) is a first-order homomorphism.
\end{proof}

\begin{proposition}\label{thm:first_order_homomorphism_as_substructure}
  Let \( \mscrX = (X, I) \) and \( \mscrY = (Y, J) \) be structures over \( \mscrL \). The \hyperref[def:function]{set-theoretic function} \( h: X \to Y \) preserves logical functions, i.e. satisfies \fullref{def:first_order_homomorphism/functions}, if and only if it is the domain of a \hyperref[def:first_order_substructure]{substructure} of the \hyperref[def:first_order_direct_product]{direct product} \( \mscrX \times \mscrY \).
\end{proposition}
\begin{proof}
  \SufficiencySubProof Suppose that \( h: X \to Y \) is a function that satisfies \fullref{def:first_order_homomorphism/functions}. We must show that \( h \) is closed under function application in \( \mscrX \times \mscrY \). Denote by \( K \) the interpretation in the structure \( \mscrX \times \mscrY \).

  First note that \( (x, y) \in h \) if and only if \( y = h(x) \).

  Let \( f \) be an \( n \)-ary function symbol from \( \mscrL \). Then, for any \( n \)-tuple of pairs \( (x_1, y_1), \ldots, (x_n, y_n) \) from \( h \), we have
  \begin{align*}
    K(f)\parens[\Big]{ (x_1, y_1), \ldots, (x_n, y_n) }
    &=
    \parens[\Big]{ I(f)\parens[\Big]{ x_1, \ldots, x_n }, J(f)\parens[\Big]{ y_1, \ldots, y_n } }
    = \\ &=
    \parens[\Big]{ I(f)\parens[\Big]{ x_1, \ldots, x_n }, J(f)\parens[\Big]{ h(x_1), \ldots, h(x_n) } }.
  \end{align*}

  Since \( h \) satisfies \fullref{def:first_order_homomorphism/functions}, it follows that
  \begin{equation*}
    h\parens[\Big]{ I(f)(x_1, \ldots, x_n) } = J(f)\parens[\Big]{ h(x_1), \ldots, h(x_n) },
  \end{equation*}
  and
  \begin{equation*}
    K(f)\parens[\Big]{ (x_1, y_1), \ldots, (x_n, y_n) } \in h.
  \end{equation*}

  Therefore, \( h \) is a substructure of \( \mscrX \times \mscrY \).

  \NecessitySubProof Suppose that \( h \) is a substructure of \( \mscrX \times \mscrY \). Again, denote by \( K \) the interpretation in the structure \( \mscrX \times \mscrY \).

  Fix an \( n \)-ary functional symbol \( f \) and an \( n \)-tuple \( x_1, \ldots, x_n \) from \( X \). The pairs \( (x_1, h(x_1)), \ldots, (x_n, h(x_n)) \) belong to \( h \). Since \( h \) is closed under logical function application, it must contain
  \begin{equation*}
    K(f)\parens[\Big]{ (x_1, h(x_1)), \ldots, (x_n, h(h_n)) }
    =
    \parens[\Big]{ I(f)\parens[\Big]{ x_1, \ldots, x_n }, J(f)\parens[\Big]{ h(x_1), \ldots, h(x_n) } }.
  \end{equation*}

  Therefore,
  \begin{equation*}
    h\parens[\Big]{ I(f)(x_1, \ldots, x_n) } = J(f)\parens[\Big]{ h(x_1), \ldots, h(x_n) }
  \end{equation*}
  and \( h \) satisfies \fullref{def:first_order_homomorphism/functions}.
\end{proof}

\begin{proposition}\label{thm:direct_product_preserves_positive_formulas}
  Let \( \Gamma \) be a closed set of \hyperref[def:positive_formula]{positive formulas} \hi{without disjunctions} over \( \mscrL \) and let \( \seq{ \mscrX_k }_{k \in \mscrK} \) be a family of structures over \( \mscrL \), where \( \mscrX_k = (X_k, I_k) \) for every \( k \in \mscrK \).

  If, for every \( k \in \mscrK \), the structure \( \mscrX_k \) is a \hyperref[def:first_order_model]{model} of \( \Gamma \), then so is the \hyperref[def:first_order_direct_product]{direct product} \( \prod_{k \in \mscrK} \mscrX_k \).
\end{proposition}
\begin{comments}
  \item We can see in \eqref{eq:thm:direct_product_preserves_positive_formulas/proof_conjunction} why the proof fails for formulas with disjunction.
\end{comments}
\begin{proof}
  We will denote the domain and interpretation of the direct product by \( X \) and \( I \) correspondingly.

  First, we will use \fullref{thm:induction_on_syntax_trees} on a term \( \tau \) to show that the following extension of \fullref{def:first_order_direct_product/functions} holds:
  \begin{equation}\label{eq:thm:direct_product_preserves_positive_formulas/hypothesis_terms}
    \Bracks{\tau}_v = \seq[\Big]{ \Bracks{\tau}_{v^k} }_{k \in \mscrK},
  \end{equation}
  where \( v \) is any \hyperref[def:first_order_valuation/variable_assignment]{variable assignment} in \( \prod_{k \in \mscrK} \mscrX_k \) and \( v^k \) be the \( k \)-th component of \( v \).

  \begin{itemize}
    \item The case when \( \tau \) is a variable follows from our definition of \( v^k \).
    \item If \( \tau = f(\kappa_1, \ldots, \kappa_n) \) and the inductive hypothesis holds for \( \kappa_1, \ldots, \kappa_n \), then, for any assignment \( v \),
    \begin{align*}
      \Bracks{\tau}_v
      &=
      I(f) \parens[\Big]{ \Bracks{\kappa_1}_v, \ldots, \Bracks{\kappa_n}_v }
      \reloset {\T{ind.}} = \\ &=
      I(f) \parens[\Big]{ \seq[\Big]{ \Bracks{\kappa_1}_{v^k} }_{k \in \mscrK}, \ldots, \seq[\Big]{ \Bracks{\kappa_n}_{v^k} }_{k \in \mscrK} }
      = \\ &=
      \seq[\Big]{ I_k(f) \parens[\Big]{ \Bracks{\kappa_1}_{v^k}, \ldots, \Bracks{\kappa_n}_{v^k} } }_{k \in \mscrK}
      = \\ &=
      \seq[\Big]{ \Bracks{\tau}_{v^k} }_{k \in \mscrK}.
    \end{align*}
  \end{itemize}

  Now, we will use \fullref{thm:induction_on_syntax_trees} on a positive formula \( \varphi \) to show that the following extension of \fullref{def:first_order_direct_product/predicates} holds:
  \begin{equation}\label{eq:thm:direct_product_preserves_positive_formulas/hypothesis_formulas}
    \Bracks{\varphi}_v = \bigwedge_{k \in \mscrK} \Bracks{\varphi}_{v^k},
  \end{equation}
  where again \( v \) is taken to be arbitrary.

  \begin{itemize}
    \item The case \( \varphi = \top \) is vacuous.
    \item If \( \varphi = \tau_1 \doteq \tau_2 \), for a fixed assignment \( v \) we have two possibilities
    \begin{itemize}
      \item If \( \Bracks{\varphi}_{v^k} = T \) for every \( k \in \mscrK \), then \( \Bracks{\tau_1}_{v^k} = \Bracks{\tau_2}_{v^k} \) and thus
      \begin{equation}\label{eq:thm:direct_product_preserves_positive_formulas/proof_equality}
        \Bracks{\tau_1}_v
        \reloset {\eqref{eq:thm:direct_product_preserves_positive_formulas/hypothesis_terms}} =
        \seq[\Big]{ \Bracks{\tau_1}_{v^k} }_{k \in \mscrK}
        =
        \seq[\Big]{ \Bracks{\tau_2}_{v^k} }_{k \in \mscrK}
        \reloset {\eqref{eq:thm:direct_product_preserves_positive_formulas/hypothesis_terms}} =
        \Bracks{\tau_2}_v.
      \end{equation}

      Therefore,
      \begin{equation*}
        \Bracks{\varphi}_v
        =
        T
        =
        \bigwedge_{k \in \mscrK} \Bracks{\varphi}_{v^k}.
      \end{equation*}

      \item If there exists some index \( k_0 \) for which \( \Bracks{\varphi}_{v^{k_0}} = F \), then \eqref{eq:thm:direct_product_preserves_positive_formulas/proof_equality} does not hold, hence
      \begin{equation*}
        \Bracks{\varphi}_v
        =
        F
        =
        \bigwedge_{k \in \mscrK} \Bracks{\varphi}_{v^k}.
      \end{equation*}
    \end{itemize}

    \item If \( \varphi = p(\tau_1, \ldots, \tau_n) \), then, by what we have shown for terms, for any variable assignment \( v \):
    \begin{equation*}
      \Bracks{\varphi}_v
      =
      I(p)\parens[\Big]{ \Bracks{\tau_1}_v, \ldots, \Bracks{\tau_n}_v }
      \reloset {\eqref{eq:thm:direct_product_preserves_positive_formulas/hypothesis_terms}} =
      \bigwedge_{k \in \mscrK} I_k(p)\parens[\Big]{ \Bracks{\tau_1}_{v^k}, \ldots, \Bracks{\tau_n}_{v^k} }
      =
      \bigwedge_{k \in \mscrK} \Bracks{\varphi}_{v^k}.
    \end{equation*}

    \item Suppose that \( \varphi = \psi_1 \wedge \psi_2 \) and that the inductive hypothesis holds for \( \psi_1 \) and \( \psi_2 \). For each \( k \in \mscrK \), commutativity of \( \Bracks{\wedge} \) implies
    \begin{equation}\label{eq:thm:direct_product_preserves_positive_formulas/proof_conjunction}
      \begin{aligned}
        \Bracks{\varphi}_v
        &=
        \Bracks{\psi_1}_v \Bracks{\wedge} \Bracks{\psi_2}_v
        = \\ &=
        \parens[\Bigg]{ \bigwedge_{k \in \mscrK} \Bracks{\psi_1}_{v^k} } \Bracks{\wedge} \parens[\Bigg]{ \bigwedge_{k \in \mscrK} \Bracks{\psi_2}_{v^k} }
        = \\ &=
        \bigwedge_{k \in \mscrK} \parens[\Big]{ \Bracks{\psi_1}_{v^k} \Bracks{\wedge} \Bracks{\psi_2}_{v^k} }
        = \\ &=
        \bigwedge_{k \in \mscrK} \Bracks{\varphi}_{v^k}.
      \end{aligned}
    \end{equation}

    \item Suppose that \( \varphi = \qforall \xi \psi \) and that the inductive hypothesis holds for \( \psi \). Fix an assignment \( v \). Then, again because of the commutativity of meets in \( \set{ T, F } \),
    \begin{align*}
      \Bracks{\qforall \xi \psi}_v
      &=
      \bigwedge \set[\Big]{ \Bracks{\psi}_{v_{\xi \mapsto \seq{ x_k }_{k \in \mscrK}}} \given* \seq{ x_m }_{m \in \mscrK} \in \prod_{m \in \mscrK} X_m }
      = \\ &=
      \bigwedge \set[\Big]{ \bigwedge_{k \in \mscrK} \Bracks{\psi}_{v^k_{\xi \mapsto x_k}} \given* \seq{ x_m }_{m \in \mscrK} \in \prod_{m \in \mscrK} X_m }
      = \\ &=
      \bigwedge_{k \in \mscrK} \bigwedge \set[\Big]{ \Bracks{\psi}_{v^k_{\xi \mapsto x_k}} \given* \seq{ x_m }_{m \in \mscrK} \in \prod_{m \in \mscrK} X_m }
      = \\ &=
      \bigwedge_{k \in \mscrK} \bigwedge \set[\Big]{ \Bracks{\psi}_{v^k_{\xi \mapsto x_k}} \given* x_k \in X_k }
      = \\ &=
      \bigwedge_{k \in \mscrK} \Bracks{\qforall \xi \psi}_{v_k}.
    \end{align*}

    \item Suppose that \( \varphi = \qexists \xi \psi \) and that the inductive hypothesis holds for \( \psi \). Fix an assignment \( v \).
    \begin{itemize}
      \item If \( \Bracks{\qexists \xi \psi}_v = T \), then there exists some tuple \( \seq{ a_k }_{k \in \mscrK} \) from \( \prod_{k \in \mscrK} X_k \) such that
      \begin{equation*}
        \Bracks{\psi}_{v_{\xi \to \seq{ a_k }_{k \in \mscrK}}} = T.
      \end{equation*}

      By the inductive hypothesis, the above is equal to
      \begin{equation*}
        \bigwedge_{k \in \mscrK} \Bracks{\psi}_{v^k_{\xi \to a_k}},
      \end{equation*}
      hence \( \Bracks{\psi}_{v^k_{\xi \mapsto a_k}} = T \) for every \( k \in \mscrK \).

      Thus, for every \( k \in \mscrK \),
      \begin{equation*}
        \Bracks{\qexists \xi \psi}_{v^k}
        =
        \bigvee_{x \in X_k} \Bracks{\psi}_{v^k_{\xi \mapsto x}}
        =
        T,
      \end{equation*}
      therefore
      \begin{equation*}
        \bigwedge_{k \in \mscrK} \Bracks{\qexists \xi \psi}_{v^k}
        =
        T
        =
        \Bracks{\qexists \xi \psi}_v.
      \end{equation*}

      \item If \( \Bracks{\qexists \xi \psi}_v = F \), then, for any tuple \( \seq{ x_k }_{k \in \mscrK} \) from \( \prod_{k \in \mscrK} X_k \) we have
      \begin{equation*}
        \Bracks{\psi}_{v_{\xi \to \seq{ x_k }_{k \in \mscrK}}} = F.
      \end{equation*}

      The inductive hypothesis implies that, for any tuple,
      \begin{equation*}
        \Bracks{\psi}_{v_{\xi \to \seq{ x_k }_{k \in \mscrK}}} = \bigwedge_{k \in \mscrK} \Bracks{\psi}_{v^k_{\xi \to x_k}}.
      \end{equation*}

      Hence, for each tuple \( \seq{ x_k }_{k \in \mscrK} \), there exists some index \( k_0 \) such that
      \begin{equation}\label{eq:thm:direct_product_preserves_positive_formulas/proof_existential}
        \Bracks{\psi}_{v^{k_0}_{\xi \to x_{k_0}}} = F.
      \end{equation}

      There must exist some index \( k_0 \) for which \eqref{eq:thm:direct_product_preserves_positive_formulas/proof_existential} holds whenever \( x_{k_0} \in X_{k_0} \). Otherwise, we could assemble a tuple contradicting \eqref{eq:thm:direct_product_preserves_positive_formulas/proof_existential}.

      Then
      \begin{equation*}
        \Bracks{\qexists \xi \psi}_{v^{k_0}}
        =
        \bigvee_{x \in X_{k_0}} \Bracks{\psi}_{v^{k_0}_{\xi \mapsto x}}
        =
        F,
      \end{equation*}
      therefore
      \begin{equation*}
        \bigwedge_{k \in \mscrK} \Bracks{\qexists \xi \psi}_{v^k}
        =
        F
        =
        \Bracks{\qexists \xi \psi}_v.
      \end{equation*}
    \end{itemize}
  \end{itemize}

  In all cases, for any variable assignment \( v \), \eqref{eq:thm:direct_product_preserves_positive_formulas/hypothesis_formulas} holds. Hence, if \( \Bracks{\varphi}_{v^k} = T \) for all \( k \in \mscrK \), then \( \Bracks{\varphi}_v = T \).

  Generalizing on \( v \), we conclude that \( \prod_{k \in \mscrK} \mscrX_k \) is a model of (the universal closure of) \( \varphi \) if, for every \( k \in \mscrK \), \( \mscrX_k \) is a model.
\end{proof}

\paragraph{First-order embeddings}

\begin{definition}\label{def:first_order_embedding}\mcite[def. 2.3.26(ii)]{Hinman2005}
  Let \( \mscrX = (X, I) \) and \( \mscrY = (Y, J) \) be structures over \( \mscrL \).

  We say that an \hyperref[def:function_invertibility/injective]{injective} \hyperref[def:first_order_homomorphism]{homomorphism} \( h: \mscrX \to \mscrY \) is an \term{embedding} if any of the following equivalent conditions hold:

  \begin{thmenum}
    \thmitem{def:first_order_embedding/inverse} The inverse of \( h \) on its image is also a homomorphism.
    \thmitem{def:first_order_embedding/predicates} For any predicate symbol \( p \) in \( \mscrL \),
    \begin{equation}\label{eq:def:first_order_embedding/predicates}
      I(p) (x_1, \ldots, x_n) = J(p) \parens[\Big]{ h(x_1), \ldots, h(x_n) }.
    \end{equation}
  \end{thmenum}
\end{definition}
\begin{comments}
  \item In particular, if there are no predicate symbols in the language, an embedding is simply an injective homomorphism.

  \item This condition can sometimes be relaxed --- see \fullref{thm:totally_ordered_strict_isomorphisms}.

  \item Peter Hinman in \incite[def. 2.3.26(ii)]{Hinman2005} actually defines embeddings as injective homomorphisms, but he requires homomorphisms to satisfy \eqref{eq:def:first_order_embedding/predicates} in \cite[def. 2.3.26(i)]{Hinman2005} --- a condition that is too restrictive for \hyperref[def:order_homomorphism]{order-preserving maps} or \hyperref[def:directed_graph/homomorphism]{graph homomorphisms}.

  \item \Fullref{ex:bijective_order_homomorphism_not_isomorphism} demonstrates how a bijective homomorphism may fail to be an isomorphism.
\end{comments}
\begin{defproof}
  Let \( h: \mscrX \to \mscrY \) be an injective homomorphism. We will denote by \( h^{-1} \) the \hyperref[def:set_valued_map/partial]{partial} \hyperref[def:set_valued_map/inverse]{inverse} of \( h \).

  Requiring \( h^{-1} \) to satisfy the condition \eqref{eq:def:first_order_embedding/predicates} on homomorphisms for predicate symbols is identical to requiring \( h \) (or \( h^{-1} \)) to satisfy the condition \eqref{eq:def:first_order_homomorphism/predicates} on embeddings.

  The condition on functional symbols is simpler to handle. We will show that \( h^{-1} \) always satisfies the condition \eqref{eq:def:first_order_homomorphism/functions} on homomorphisms for functional symbols.

  Fix an \( n \)-ary functional symbol \( f \) and a tuple \( y_1, \ldots, y_n \) of members of \( h(X) \). Then
  \begin{align*}
    h^{-1}\parens[\Big]{ J(f)(y_1, \ldots, y_n) }
    &=
    h^{-1}\parens[\Big]{ J(f)(h(h^{-1}(y_1)), \ldots, h(h^{-1}(y_n))) }
    \reloset {\eqref{eq:def:first_order_homomorphism/functions}} = \\ &=
    h^{-1}\parens[\Big]{ g^{-1}\parens[\Big]{ I(f) \parens[\Big]{ h^{-1}(y_1), \ldots, h^{-1}(y_n) } } }
    = \\ &=
    I(f) \parens[\Big]{ h^{-1}(y_1), \ldots, h^{-1}(y_n) }.
  \end{align*}
\end{defproof}

\begin{definition}\label{def:first_order_isomorphism}\mcite[def. 2.3.1(i)]{Hinman2005}
  If a \hyperref[def:first_order_embedding]{first-order embedding} is \hyperref[def:function_invertibility/surjective]{surjective}, we call it an \term{isomorphism} and say that the structures are \term{isomorphic}. In accordance with \fullref{def:morphism_invertibility/automorphism}, we call an isomorphism from a structure to itself an \term{automorphism}.
\end{definition}

\begin{remark}\label{rem:embeds_isomorphically}
  We sometimes refer to \hyperref[def:first_order_embedding]{first-order embeddings} as \enquote{isomorphic embeddings} to highlight that they are not merely an injective function.

  For example, we can say that \( \mscrY \) embeds isomorphically into \( \mscrX \).

  Similar terminology is used, among others, by \incite[100]{Enderton1977Sets}.
\end{remark}

\begin{example}\label{ex:bijective_order_homomorphism_not_isomorphism}
  Consider the \hyperref[def:integers]{set of integers} \( \BbbZ \) endowed with two different \hyperref[def:partially_ordered_set]{partial orders}:
  \begin{itemize}
    \item The standard total order \( \leq \) where \( n \leq m \) if there exists a nonnegative integer \( k \) such that \( n + k = m \).
    \item The equality relation \( = \).
  \end{itemize}

  The identity \( \id(x) = x \) is an \hyperref[def:order_homomorphism]{order homomorphisms} from \( (\BbbZ, =) \) to \( (\BbbZ, \leq) \). Indeed, for any integers \( n \) and \( m \), \( n = m \) implies \( n \leq m \).

  Furthermore, the identity function is bijective. The inverse of \( \id \), which is again \( \id \), is not however a homomorphism from \( (\BbbZ, \leq) \) to \( (\BbbZ, =) \) since, for example, \( 1 \leq 2 \), but \( 1 \neq 2 \).

  Hence, \( \id: (\BbbZ, =) \to (\BbbZ, \leq) \) is a bijective homomorphism, but not an isomorphism.
\end{example}

\begin{proposition}\label{thm:arbitrary_formulas_preserved_under_isomorphisms}
  A \hyperref[def:first_order_embedding]{first-order embeddings} preserves models of arbitrary closed formulas.

  More concretely, given some language \( \mscrL \), if \( \Gamma \) is a set of closed formulas and \( h: \mscrX \to \mscrY \) is an embedding between \hyperref[def:first_order_model]{models} of \( \Gamma \), then the image \( h[X] \) is also (the domain of) a model of \( \Gamma \).
\end{proposition}
\begin{comments}
  \item Compare this to \fullref{thm:positive_formulas_preserved_under_homomorphism}, which places restrictions on the formulas rather than the homomorphism.
  \item We say that embeddings preserve the validity arbitrary formulas.
\end{comments}
\begin{proof}
  The embedding condition allows us to extend the induction in our proof of \fullref{thm:arbitrary_formulas_preserved_under_isomorphisms} to the case of arbitrary formulas by instead requiring that \( \Bracks{\varphi}_v = \Bracks{\varphi}_w \) with the fixed variable assignments \( v \) and \( w \) from our proof of \fullref{thm:arbitrary_formulas_preserved_under_isomorphisms}.
\end{proof}

\begin{proposition}\label{thm:isomorphism_preserves_validity}\mcite[prop. 2.3.39]{Hinman2005}
  Let \( \mscrX = (X, I) \) and \( \mscrY = (Y, J) \) be structures over \( \mscrL \). Let \( h: \mscrX \to \mscrY \) be a \hyperref[def:first_order_isomorphism]{first-order isomorphism}. Fix a formula \( \varphi \) in \( \mscrL \) with (at most) \( n \) free variables. Then
  \begin{equation*}
    \Bracks{\varphi}_\mscrX(x_1, \ldots, x_n) = \Bracks{\varphi}_\mscrY\parens[\Big]{ h(x_1), \ldots, h(x_n) },
  \end{equation*}
  where \( \Bracks{\varphi} \) denotes the valuation function defined in \fullref{def:propositional_valuation/valuation_function}.
\end{proposition}
\begin{proof}
  We will show via \fullref{thm:induction_on_syntax_trees} that
  \begin{equation}\label{eq:thm:isomorphism_preserves_validity/hypothesis_formulas}
    \Bracks{\varphi}_v = \Bracks{\varphi}_w,
  \end{equation}
  where \( \varphi \) is a formula over \( \mscrL \), \( v \) is a variable assignment in \( \mscrX \) and \( w(\xi) \coloneqq h(v(\xi)) \) is an assignment in \( \mscrY \).

  Then it will follow that, if \( \xi_1, \ldots, \xi_n \) are all the free variables of \( \varphi \), the following restatement of the proposition holds:
  \begin{equation*}
    \Bracks{\varphi}_{v_{\xi_1 \mapsto x_1, \ldots, \xi_n \mapsto x_n}}
    =
    \Bracks{\varphi}_{w_{\xi_1 \mapsto h(x_1), \ldots, \xi_n \mapsto h(x_n)}}.
  \end{equation*}

  First, we will need to prove the following auxiliary result: for any term \( \tau \) over \( \mscrL \), we have
  \begin{equation}\label{eq:thm:isomorphism_preserves_validity/hypothesis_terms}
    \Bracks{\tau}_w = h(\Bracks{\tau}_v).
  \end{equation}

  \begin{itemize}
    \item If \( \tau \) is a variable, \eqref{eq:thm:isomorphism_preserves_validity/hypothesis_terms} is trivial.
    \item If \( \tau = f(\kappa_1, \ldots, \kappa_n) \) and if the inductive hypothesis holds for \( \kappa_1, \ldots, \kappa_n \), then
    \begin{align*}
      \Bracks{\tau}_w
      &=
      J(f)\parens[\Big]{ \Bracks{\kappa_1}_w, \ldots, \Bracks{\kappa_n}_w }
      \reloset {\T{ind.}} = \\ &=
      J(f)\parens[\Big]{ h(\Bracks{\kappa_1}_v), \ldots, h(\Bracks{\kappa_n}_v) }
      = \\ &=
      h\parens[\Big]{ I(f)\parens[\Big]{ \Bracks{\kappa_1}_v, \ldots, \Bracks{\kappa_n}_v } }
      = \\ &=
      h(\Bracks{\tau}_v).
    \end{align*}
  \end{itemize}

  Now we will perform induction on \( \varphi \):
  \begin{itemize}
    \item The case where \( \varphi \) is a constant is vacuous.
    \item If \( \varphi = \tau_1 \doteq \tau_2 \), then, since \( h \) is injective,
    \begin{equation*}
      \Bracks{\tau_1 \doteq \tau_2}_w = T
      \T{if and only if}
      \underbrace{\Bracks{\tau_1}_w}_{h(\Bracks{\tau_1}_v)} = \underbrace{\Bracks{\tau_2}_w}_{h(\Bracks{\tau_2}_v)}
      \T{if and only if}
      \Bracks{\tau_1 \doteq \tau_2}_v = T.
    \end{equation*}

    \item Suppose that \( \varphi = p(\tau_1, \ldots, \tau_n) \).
    \begin{itemize}
      \item If \( \Bracks{\varphi}_v = T \), we have
      \begin{equation*}
        \Bracks{\varphi}_w
        =
        J(p)\parens[\Big]{ \Bracks{\tau_n}_w, \ldots, \Bracks{\tau_n}_w }
        \reloset {\T{ind.}} =
        J(p)\parens[\Big]{ h(\Bracks{\tau_n}_v), \ldots, h(\Bracks{\tau_n}_v) }
        \reloset {\eqref{eq:def:first_order_homomorphism/predicates}} =
        T.
      \end{equation*}

      \item If \( \Bracks{\varphi}_v = F \), then \( \Bracks{\varphi}_w = F \) because \( h^{-1} \) is also a homomorphism and, by what we have already shown, \( \Bracks{\varphi}_w = T \) would imply \( \Bracks{\varphi}_v = T \).
    \end{itemize}

    \item If \( \varphi = \neg \psi \) and if the inductive hypothesis holds for \( \psi \), then
    \begin{equation*}
      \Bracks{\neg \psi}_v
      =
      \overline{\Bracks{\psi}_v}
      \reloset {\T{ind.}} =
      \overline{\Bracks{\psi}_w}
      =
      \Bracks{\neg \psi}_w.
    \end{equation*}

    \item If \( \varphi = \psi_1 \bincirc \psi_2 \), where \( \bincirc \in \set{ \vee, \wedge, \rightarrow, \leftrightarrow } \), and if the inductive hypothesis holds for \( \psi_1 \) and \( \psi_2 \), then
    \begin{equation*}
      \Bracks{\varphi}_v
      =
      \Bracks{\psi_1}_v \Bracks{\bincirc} \Bracks{\psi_2}_v
      \reloset {\T{ind.}} =
      \Bracks{\psi_1}_w \Bracks{\bincirc} \Bracks{\psi_2}_w
      =
      \Bracks{\varphi}_w
    \end{equation*}

    \item If \( \varphi = \qforall \xi \psi \) and if the inductive hypothesis holds for \( \psi \), then
    \begin{equation*}
      \Bracks{\varphi}_v
      =
      \bigwedge_{x \in X} \Bracks{\psi}_{v_{\xi \mapsto x}}
      \reloset {\T{ind.}} =
      \bigwedge_{x \in X} \Bracks{\psi}_{w_{\xi \mapsto h(x)}}
      =
      \bigwedge_{y \in Y} \Bracks{\psi}_{w_{\xi \mapsto y}}
      =
      \Bracks{\varphi}_w.
    \end{equation*}

    \item The case where \( \varphi = \qexists \xi \psi \) is similar.
  \end{itemize}
\end{proof}

\paragraph{First-order congruences}

\begin{definition}\label{def:first_order_congruence}\mcite[46]{Мальцев1970}
  Let \( \mscrX = (X, I) \) be a \hyperref[def:first_order_structure]{first-order structure} over some language.

  We say that the \hyperref[def:equivalence_relation]{equivalence relation} \( \cong \) on \( X \) is a (first-order) \term[ru=конгруэнция (\cite[46]{Мальцев1970})]{congruence} on \( \mscrX \) if any of the following equivalent conditions hold:
  \begin{thmenum}
    \thmitem{def:first_order_congruence/direct} For any \( n \)-ary functional symbol \( f \), from \( x_1 \cong x_1', \ldots, x_n \cong x_n' \) it follows that
    \begin{equation}\label{eq:def:first_order_congruence/direct}
      I(f)(x_1, \ldots, x_n) \cong I(f)(x_1', \ldots, x_n').
    \end{equation}

    \thmitem{def:first_order_congruence/substructure} The relation \( \cong \) is itself the domain of a substructure of the \hyperref[def:first_order_direct_product]{direct product} \( \mscrX^2 \).
  \end{thmenum}
\end{definition}
\begin{comments}
  \item Anatoly Maltsev gives this definition in \cite[46]{Мальцев1970}, but calls structures \enquote{algebraic systems} and studies them independently of first-order logic.
\end{comments}
\begin{defproof}
  \EquivalenceSubProof{def:first_order_congruence/direct}{def:first_order_congruence/substructure} Denote by \( K \) the interpretation in \( \mscrX^2 \). Note that
  \begin{equation*}
    K(f)\parens[\Big]{ (x_1, x_1'), \ldots, (x_n, x_n') }
    =
    \parens[\Big]{ I(f)(x_1, \ldots, x_n), I(f)(x_1', \ldots, x_n') }.
  \end{equation*}

  Therefore, if \( x_1 \cong x_1', \ldots, x_n \cong x_n' \), then
  \begin{equation*}
    K(f)\parens[\Big]{ (x_1, x_1'), \ldots, (x_n, x_n') } \in {\cong}
  \end{equation*}
  if and only if
  \begin{equation*}
    I(f)(x_1, \ldots, x_n) \cong I(f)(x_1', \ldots, x_n').
  \end{equation*}
\end{defproof}

\begin{proposition}\label{thm:homomorphism_induces_congruence}
  Every first-order homomorphism \( \varphi: \mscrX \to \mscrY \) induces a \hyperref[def:first_order_congruence]{congruence} \( {\cong} \) on its domain \( \mscrX \) defined by putting \( x \cong x' \) if \( \varphi(x) = \varphi(x') \).
\end{proposition}
\begin{proof}
  The relation defined is obviously an equivalence relation. We will verify that it satisfies \eqref{eq:def:first_order_congruence/direct}.

  Fix an \( n \)-ary functional symbol \( f \) and \( n \) pairs of congruence elements \( x_1 \cong x_1', \ldots, x_n \cong x_n' \). Then
  \small
  \begin{equation*}
    \varphi\parens[\Big]{ I(f)(x_1, \ldots, x_n) }
    \reloset {\eqref{eq:def:first_order_homomorphism/functions}} =
    J(f)\parens[\Big]{ \underbrace{\varphi(x_1)}_{=\varphi(x_1')}, \ldots, \underbrace{\varphi(x_n)}_{=\varphi(x_n')} }
    =
    J(f)\parens[\Big]{ \varphi(x_1'), \ldots, \varphi(x_n') }
    \reloset {\eqref{eq:def:first_order_homomorphism/functions}} =
    \varphi\parens[\Big]{ I(f)(x_1', \ldots, x_n') }.
  \end{equation*}
  \normalsize

  Generalizing on \( f \), we conclude that \( {\cong} \) is a congruence relation.
\end{proof}

\begin{definition}\label{def:first_order_generated_congruence}\mimprovised
  Let \( \mscrX = (X, I) \) be a first-order structure. Given a binary relation \( {\sim} \) on \( X \), we define the \hyperref[def:first_order_congruence]{congruence} on \( \mscrX \) \term{generated} by \( {\sim} \) as the intersection of all congruences containing \( {\sim} \).

  \Fullref{thm:closure_operator_from_set_semilattice} implies that it is a \hyperref[def:moore_closure_operator]{Moore closure operator} on \( \pow(X^2) \).
\end{definition}
\begin{comments}
  \item In particular, as a consequence of \fullref{thm:closure_operator_minimality}, the generated congruence is the smallest congruence containing \( {\sim} \).
\end{comments}
\begin{defproof}
  Let \( \Omega \) be the family of all congruences on \( \mscrX \) containing \( {\sim} \).
  \begin{itemize}
    \item The intersection \( \bigcap \Omega \) is an equivalence relation as an intersection of equivalence relations.
    \item It is also (the domain of) a substructure of \( \mscrX^2 \) as implied by \fullref{thm:intersection_substructure}.
  \end{itemize}

  Hence, \( \bigcap \Omega \) is a congruence on \( \mscrX \).
\end{defproof}

\begin{proposition}\label{thm:congruences_form_complete_lattice}
  Fix a structure \( \mscrX = (X, I) \) for the language \( \mscrL \).

  With respect to set inclusion, the family of all \hyperref[def:first_order_congruence]{congruences} on \( \mscrX \) forms a complete lattice. Explicitly:
  \begin{thmenum}
    \thmitem{thm:congruences_form_complete_lattice/join} The \hyperref[def:semilattice/join]{join} of the family of congruences \( \seq{ {\cong_k} }_{k \in \mscrK} \) is the \hyperref[def:first_order_generated_congruence]{generated congruence} of their union \( \bigcup_{k \in \mscrK} {\cong_k} \).

    \thmitem{thm:congruences_form_complete_lattice/top} The \hyperref[def:extremal_points/top_and_bottom]{top element} is the relation under which all elements of \( X \) are equivalent.

    \thmitem{thm:congruences_form_complete_lattice/meet} The \hyperref[def:semilattice/meet]{meet} of the family of congruences \( \seq{ {\cong_k} }_{k \in \mscrK} \) is simply their intersection \( \bigcap_{k \in \mscrK} {\cong_k} \).

    \thmitem{thm:congruences_form_complete_lattice/bottom} The \hyperref[def:extremal_points/top_and_bottom]{bottom element} of this lattice is the equality relation.
  \end{thmenum}
\end{proposition}
\begin{proof}
  Trivial.
\end{proof}

\begin{definition}\label{def:first_order_quotient}\mcite[62]{Мальцев1970}
  Consider a \hyperref[def:first_order_congruence]{first-order congruence} \( \cong \) on \( \mscrX = (X, I) \) and the \hyperref[def:equivalence_relation/projection]{quotient map}
  \begin{equation*}
    \begin{aligned}
      &\pi: X \to X / {\cong}, \\
      &\pi(x) = [x].
    \end{aligned}
  \end{equation*}

  We define the following interpretation on the \hyperref[def:equivalence_relation/quotient]{quotient set} \( X / {\cong} \):
  \begin{thmenum}
    \thmitem{def:first_order_quotient/functions} For every functional symbol \( f \) of arity \( n \):
    \begin{equation}\label{eq:def:first_order_quotient/functions}
      I_\cong(f)\parens[\Big]{ [x_1], \ldots, [x_n] } \coloneqq [I(f)(x_1, \ldots, x_n)]
    \end{equation}

    \thmitem{def:first_order_quotient/predicates} For every predicate symbol \( p \) of arity \( n \):
    \begin{equation}\label{eq:def:first_order_quotient/predicates}
      I_\cong(p)\parens[\Big]{ [x_1], \ldots, [x_n] } \coloneqq \bigvee\set[\Big]{ I(p)(x_1', \ldots, x_n') \given* x_1' \in [x_1], \ldots, x_n' \in [x_n] }
    \end{equation}
  \end{thmenum}

  Then \( \mscrX / {\cong} \coloneqq (X / {\cong}, I_\cong) \) is a structure over the same language and, furthermore, the projection \( \pi \) is a \hyperref[def:first_order_homomorphism]{homomorphism}.
\end{definition}

\begin{theorem}[Quotient structure universal property]\label{thm:quotient_structure_universal_property}
  For every first-order structure \( \mscrX \) and every \hyperref[def:first_order_congruence]{congruence} \( \cong \) on \( \mscrX \), the \hyperref[def:first_order_quotient]{quotient structure} \( \mscrX / {\cong} \) has the following \hyperref[rem:universal_mapping_property]{universal mapping property}:
  \begin{displayquote}
    Every homomorphism \( h: \mscrX \to \mscrY \) for which \( x \cong x' \) implies \( h(x) = h(x') \) \hyperref[def:factors_through]{uniquely factors through} \( \mscrX / {\cong} \).

    More precisely, there exists a unique homomorphism \( \widetilde{h}: \mscrX / {\cong} \to \mscrY \), such that the following diagram commutes:
    \begin{equation}\label{eq:thm:quotient_structure_universal_property/diagram}
      \begin{aligned}
        \includegraphics[page=1]{output/thm__quotient_group_universal_property}
      \end{aligned}
    \end{equation}
  \end{displayquote}
\end{theorem}
\begin{comments}
  \item For \hyperref[def:group]{groups}, this theorem can be restated via \hyperref[def:group/kernel]{kernels} and \hyperref[def:normal_subgroup]{normal subgroups}:
  \begin{displayquote}
    Given a normal subgroup \( N \) of \( G \), every homomorphism \( \varphi: G \to H \) satisfying \( N \subseteq \ker \varphi \) uniquely factors through \( G / N \).
  \end{displayquote}

  Similarly, when stated for \hyperref[def:ring]{rings} and \hyperref[def:algebra_over_ring]{algebras over rings}, the theorem uses \hyperref[def:ring/kernel]{kernels} and \hyperref[def:semiring_ideal]{ideals}:
  \begin{displayquote}
    Given an ideal \( I \) of \( R \), every homomorphism \( \varphi: R \to T \) satisfying \( I \subseteq \ker \varphi \) uniquely factors through \( R / I \).
  \end{displayquote}

  For \hyperref[def:module]{modules over rings}, it becomes particularly simple:
  \begin{displayquote}
    Given an \( R \)-submodule \( N \) of \( M \), every homomorphism \( \varphi: M \to K \) satisfying \( N \subseteq \ker \varphi \) uniquely factors through \( M / N \).
  \end{displayquote}
\end{comments}
\begin{proof}
  For any element \( x \) of \( \mscrX \), we want
  \begin{equation*}
    \widetilde{h}(\pi(x)) = h(x).
  \end{equation*}

  This condition can be used as a definition, but only in the case where \( h \) only depends on the equivalence class \( \pi(x) \), but not the representative. This is the reason we have the additional restriction that \( x \cong x' \) must imply \( h(x) = h(x') \).

  Uniqueness follows by construction.
\end{proof}

\begin{proposition}\label{thm:quotient_preserves_positive_formulas}
  Fix a set \( \Gamma \) of closed \hyperref[def:positive_formula]{positive formulas} over a language without predicate symbols.

  If \( \mscrX = (X, I) \) is a \hyperref[def:first_order_model]{model} of \( \Gamma \) and \( \cong \) is a \hyperref[def:first_order_congruence]{first-order congruence} on \( \mscrX \), then the \hyperref[def:first_order_quotient]{quotient structure} \( \mscrX / {\cong} \) defined as \( (X / {\cong}, I_\cong) \) is also a model of \( \Gamma \).
\end{proposition}
\begin{defproof}
  Fix a positive formula \( \varphi \). Via \fullref{thm:induction_on_syntax_trees}, we will show that if any variable assignment \( v \) satisfies \( \varphi \) in \( \mscrX \), the following variable assignment satisfies \( \varphi \) in \( \mscrX / {\cong} \):
  \begin{equation*}
    \begin{aligned}
      &v^{\cong}: \boldop{Var} \to X / {\cong}, \\
      &v^{\cong}(\xi) \coloneqq [v(\xi)].
    \end{aligned}
  \end{equation*}

  We start by noting that, for any term \( \tau \), we have
  \begin{equation}\label{eq:thm:quotient_preserves_positive_formulas/term_equality}
    \Bracks{\tau}_{v^{\cong}} = \bracks[\Big]{ \Bracks{\tau}_v }.
  \end{equation}

  This is obvious if \( \tau \) is a variable, and it follows directly from \eqref{eq:def:first_order_quotient/functions} if \( \tau \) is a function application.

  Now we will use induction on \( \varphi \) to show that, if \( v \) satisfies \( \varphi \), then so does \( w \).
  \begin{itemize}
    \item The validity of \( \varphi = \top \) is clear.
    \item If \( \varphi = \tau_1 \doteq \tau_2 \) and \( v \) satisfies \( \varphi \), then \( \Bracks{\tau_1}_v = \Bracks{\tau_2}_v \), implying that \( \Bracks{\tau_1}_{v^{\cong}} = \Bracks{\tau_2}_{v^{\cong}} \) and \( \Bracks{\varphi}_{v^{\cong}} = T \).

    \item If \( \varphi = \psi_1 \wedge \psi_2 \), if the inductive hypothesis holds for \( \psi_1 \) and \( \psi_2 \) and if \( v \) satisfies \( \varphi \), then
    \begin{equation*}
      T
      =
      \Bracks{\psi_1 \wedge \psi_2}_v
      =
      \Bracks{\psi_1}_v \Bracks{\wedge} \Bracks{\psi_2}_v.
    \end{equation*}

    Thus, we have
    \begin{equation*}
      \Bracks{\psi_1}_v = \Bracks{\psi_2}_v = T,
    \end{equation*}
    and the inductive hypothesis allows us to conclude that
    \begin{equation*}
      \Bracks{\psi_1}_{v^{\cong}} = \Bracks{\psi_2}_{v^{\cong}} = T.
    \end{equation*}

    Therefore,
    \begin{equation*}
      \Bracks{\psi_1 \wedge \psi_2}_{v^{\cong}}
      =
      \Bracks{\psi_1}_{v^{\cong}} \Bracks{\wedge} \Bracks{\psi_2}_{v^{\cong}}
      =
      T \Bracks{\wedge} T
      =
      T.
    \end{equation*}

    \item For the case \( \varphi = \psi_1 \vee \psi_2 \), we have two subcases:
    \begin{itemize}
      \item If \( \Bracks{\psi_1}_v = T \), then \( \Bracks{\psi_1}_{v^{\cong}} = T \) and thus
      \begin{equation*}
        \Bracks{\psi_1 \vee \psi_2}_{v^{\cong}}
        =
        \Bracks{\psi_1}_{v^{\cong}} \Bracks{\vee} \Bracks{\psi_2}_{v^{\cong}}
        =
        T \Bracks{\vee} \Bracks{\psi_2}_{v^{\cong}}
        =
        T.
      \end{equation*}

      \item If \( \Bracks{\psi_1}_v = F \), then necessarily \( \Bracks{\psi_2}_v = T \) because otherwise we would obtain
      \begin{equation*}
        T
        =
        \Bracks{\psi_1 \vee \psi_2}_v
        =
        \Bracks{\psi_1}_v \Bracks{\vee} \Bracks{\psi_2}_v
        =
        F \Bracks{\vee} F
        =
        F,
      \end{equation*}
      which is a contradiction.

      Thus, we have \( \Bracks{\psi_2}_{v^{\cong}} = T \), and we can conclude that
      \begin{equation*}
        \Bracks{\psi_1 \vee \psi_2}_{v^{\cong}}
        =
        \Bracks{\psi_1}_{v^{\cong}} \Bracks{\vee} \Bracks{\psi_2}_{v^{\cong}}
        =
        F \Bracks{\vee} T
        =
        T.
      \end{equation*}
    \end{itemize}

    \item Suppose that \( \varphi = \qexists \xi \psi \), that \( \psi \) satisfies the inductive hypothesis and that \( \Bracks{\varphi}_v = T \).

    We have
    \begin{equation*}
      T
      =
      \Bracks{ \qexists \xi \psi }_v
      =
      \bigvee\set[\Big]{ \Bracks{\psi}_{v_{\xi \to x}} \given* x \in X },
    \end{equation*}
    thus there exists some value \( x_0 \) such that \( \Bracks{\psi}_{\xi \to x_0} = T \).

    The inductive hypothesis implies that
    \begin{equation*}
      \Bracks{\psi}_{v_{\xi \to x_0}^{\cong}}
      =
      \Bracks{\psi}_{v_{\xi \to x_0}}
      =
      T.
    \end{equation*}

    But
    \begin{equation*}
      v_{\xi \to x_0}^{\cong}
      =
      v^{\cong}_{\xi \to [x_0]}(\eta)
      =
      \begin{cases}
        [x_0],           &\eta = \xi, \\
        v^{\cong}(\eta), &\T{otherwise,}
      \end{cases}
    \end{equation*}
    hence
    \begin{equation*}
      \Bracks{\psi}_{v^{\cong}_{\xi \to [x_0]}}
      =
      T
    \end{equation*}
    and
    \begin{equation*}
      \Bracks{ \qexists \xi \psi }_{v^{\cong}}
      =
      \bigvee\set[\Big]{ \Bracks{\psi}_{v^{\cong}_{\xi \to [x]}} \given*{} [x] \in X / {\cong} }
      =
      \Bracks{\psi}_{v^{\cong}_{\xi \to [x_0]}}
      =
      T.
    \end{equation*}

    \item If \( \varphi = \qforall \xi \psi \), we instead take an arbitrary value \( [x] \) of \( X / {\cong} \) and use that
    \begin{equation*}
      \Bracks{\psi}_{v^{\cong}_{\xi \to [x]}}
      =
      \Bracks{\psi}_{v_{\xi \to x}^{\cong}}
      =
      T.
    \end{equation*}

    Generalizing on \( [x] \), we obtain
    \begin{equation*}
      \Bracks{ \qforall \xi \psi }_{v^{\cong}}
      =
      \bigwedge\set[\Big]{ \Bracks{\psi}_{v^{\cong}_{\xi \to [x]}} \given*{} [x] \in X }
      =
      \bigwedge\set[\Big]{ T \given*{} [x] \in X }
      =
      T.
    \end{equation*}
  \end{itemize}

  In all cases of the induction, we have shown that variable assignments in \( \mscrX \) that satisfy \( \varphi \) correspond to variable assignments in \( \mscrX / {\cong} \) that satisfy \( \varphi \). Generalizing on both \( \varphi \) and \( v \), we conclude that the proposition holds.
\end{defproof}

\begin{theorem}[Lattice theorem for substructures]\label{thm:lattice_theorem_for_substructures}
  Let \( \mscrX = (X, I) \) be a \hyperref[def:first_order_structure]{structure} over \hyperref[def:first_order_language]{first-order language} \( \mscrL \) and let \( {\cong} \) be a \hyperref[def:first_order_congruence]{congruence} on \( \mscrX \).

  We will consider only \hyperref[def:first_order_substructure]{substructures} \( (S, I) \) of \( (X, I) \) satisfying the following compatibility condition:
  \begin{equation}\label{eq:thm:lattice_theorem_for_substructures/compatibility}
    \T{If} x \in S \T{and} x \cong x', \T{then} x' \in S.
  \end{equation}

  This condition ensures that every equivalence classes of the quotient set \( S / {\cong} \) is also an element of \( X / {\cong} \).

  We will give a verbose formulation as a buildup for \fullref{thm:lattice_theorem_for_substructures/isomorphism}; the theorem is summarized in \ref{fig:thm:lattice_theorem_for_substructures}.

  \begin{figure}[!ht]
    \centering
    \includegraphics[page=1]{output/thm__lattice_theorem_for_substructures}
    \caption{The lattices from \fullref{thm:lattice_theorem_for_substructures}.}
    \label{fig:thm:lattice_theorem_for_substructures}
  \end{figure}

  \begin{thmenum}
    \thmitem{thm:lattice_theorem_for_substructures/direct} If \( (S, I) \) is a substructure of \( (X, I) \) compatible with \( {\cong} \), then \( (S / {\cong}, I_\cong) \) is a substructure of \( (X / {\cong}, I_\cong) \), that is, a structure for \( \mscrL \) satisfying \( S / {\cong} \subseteq X / {\cong} \).

    We know that \( (S, I) \) has a quotient structure with respect to \( {\cong} \), but the interpretation is in general different from \( I_\cong \).

    \thmitem{thm:lattice_theorem_for_substructures/reverse} If \( (Q, I_\cong) \) is a substructure of \( (X / {\cong}, I_\cong) \), then \( (\bigcup Q, I) \) is a substructure of \( (X, I) \) compatible with \( {\cong} \), that is, a structure over \( \mscrL \) satisfying \( \bigcup Q \subseteq X \) and \eqref{eq:thm:lattice_theorem_for_substructures/compatibility}.

    \thmitem{thm:lattice_theorem_for_substructures/left_invertible} If \( (S, I) \) is a substructure of \( (X, I) \) compatible with \( {\cong} \), then the sets \( S \) and \( \bigcup (S / {\cong}) \) coincide.

    \thmitem{thm:lattice_theorem_for_substructures/right_invertible} If \( (Q, I_\cong) \) is a substructure of \( (X / {\cong}, I_\cong) \), then the sets \( Q \) and \( (\bigcup Q) / {\cong} \) coincide.

    \thmitem{thm:lattice_theorem_for_substructures/direct_order} Taking congruence is \hyperref[def:order_homomorphism/increasing]{order-preserving}: if \( S \subseteq T \subseteq X \) and if both \( (S, I) \) and \( (T, I) \) are substructures of \( (X, I) \) compatible with \( {\cong} \), then \( (T / {\cong}, I_\cong) \) is a substructure of \( (S / {\cong}, I_\cong) \).

    \thmitem{thm:lattice_theorem_for_substructures/reverse_order} Similarly, if \( R \subseteq Q \subseteq X / {\cong} \) and if both \( (R, I_\cong) \) and \( (Q, I_\cong) \) are substructures of \( (X / {\cong}, I_\cong) \), then \( (\bigcup R, I) \) is a substructure of \( (\bigcup Q, I_\cong) \).

    \thmitem{thm:lattice_theorem_for_substructures/src_complete_lattice} The family of all substructures of \( (X, I) \) compatible with \( {\cong} \) is a \hyperref[def:semilattice/lattice]{complete lattice} with respect to the substructure relation.

    In general, the join operation is incompatible with the one from the \hyperref[thm:substructures_form_complete_lattice]{substructure lattice}.

    \thmitem{thm:lattice_theorem_for_substructures/dest_complete_lattice} The family of all substructures of \( (X / {\cong}, I_\cong) \) is a \hyperref[def:semilattice/lattice]{complete lattice} with respect to the substructure relation.

    This is precisely the \hyperref[thm:substructures_form_complete_lattice]{substructure lattice} of \( (X / {\cong}, I_\cong) \).

    \thmitem{thm:lattice_theorem_for_substructures/isomorphism} The map \( (S, I) \mapsto (S / {\cong}, I_\cong) \) is an isomorphism between the (complete) lattice of all substructures of \( (X, I) \) compatible with \( {\cong} \) and the lattice of all substructures of \( (X / {\cong}, I_\cong) \).
  \end{thmenum}
\end{theorem}
\begin{comments}
  \item Simpler forms of this theorem hold in some special cases --- see \fullref{thm:lattice_theorem_for_subgroups}, \fullref{thm:lattice_theorem_for_ideals} and especially \fullref{thm:lattice_theorem_for_submodules}.
\end{comments}
\begin{proof}
  \SubProofOf{thm:lattice_theorem_for_substructures/direct} Let \( (S, I) \) be a substructure of \( (X, I) \). In order to show that \( (S / {\cong}, I_\cong) \) is a substructure of \( (X / {\cong}, I_\cong) \), we must show that the set \( S / {\cong} \) is closed under function application.

  Let \( f \) be an \( n \)-ary functional symbol from \( \mscrL \). If \( [x_1], \ldots, [x_n] \) are members of \( S / {\cong} \), then \( x_1, \ldots, x_n \) are members of \( S \). Since \( S \) is closed under function application, it contains
  \begin{equation*}
    I(f)(x_1, \ldots, x_n).
  \end{equation*}

  Thus, \( S / {\cong} \) contains
  \begin{equation*}
    I_\cong(f)\parens{ [x_1], \ldots, [x_n] }
    =
    [I(f)(x_1, \ldots, x_n)].
  \end{equation*}

  Therefore, \( S / {\cong} \) is closed under function application.

  \SubProofOf{thm:lattice_theorem_for_substructures/reverse} Let \( (Q, I_\cong) \) be a substructure of \( (X / {\cong}, I_\cong) \). We must show that the set \( \bigcup Q \) is closed under function application.

  Again, let \( f \) be an \( n \)-ary functional symbol from \( \mscrL \). If \( x_1, \ldots, x_n \) are members of \( \bigcup Q \), then the cosets \( [x_1], \ldots, [x_n] \) belong to \( Q \) and since \( Q \) is closed under application of \( I_\cong(f) \), it contains
  \begin{equation*}
    I_\cong(f)\parens{ [x_1], \ldots, [x_n] }
    =
    [I(f)(x_1, \ldots, x_n)].
  \end{equation*}

  The union \( \bigcup Q \) satisfies the compatibility condition \eqref{eq:thm:lattice_theorem_for_substructures/compatibility} because it contains the entire equivalence classes. In particular, \( \bigcup Q \) contains \( I(f)(x_1, \ldots, x_n) \) itself.

  Therefore, \( \bigcup Q \) is closed under function application and \( (\bigcup Q, I) \) is a substructure of \( (X, I) \) compatible with \( {\cong} \).

  \SubProofOf{thm:lattice_theorem_for_substructures/left_invertible} Let \( (S, I) \) be a substructure of \( (X, I) \) compatible with \( {\cong} \). We will show that the sets \( S \) and \( \bigcup(S / {\cong}) \) coincide.

  In one direction, if \( s \in S \), then \( [s] \in S / {\cong} \) and thus \( s \in [s] \subseteq \bigcup (S / {\cong}) \).

  In the other direction, if \( s \in \bigcup (S / {\cong}) \), then \( [s] \in S / {\cong} \) and there exists some \( s' \) congruent to \( s \) that belongs to \( S \). The compatibility condition \eqref{eq:thm:lattice_theorem_for_substructures/compatibility} ensures that \( s \) itself belongs to \( S \).

  Therefore, \( S = \bigcup(S / {\cong}) \).

  \SubProofOf{thm:lattice_theorem_for_substructures/right_invertible} Let \( (Q, I_\cong) \) be a substructure of \( (X / {\cong}, I_\cong) \). We will show that the sets \( Q \) and \( (\bigcup Q) / {\cong} \) coincide.

  In one direction, if \( [q] \in Q \), then \( q \in \bigcup Q \) and, since we have shown in \fullref{thm:lattice_theorem_for_substructures/reverse} that \( (\bigcup Q, I) \) is a structure satisfying the compatibility condition \eqref{eq:thm:lattice_theorem_for_substructures/compatibility}, we conclude that \( [q] \in (\bigcup Q) / {\cong} \).

  In the other direction, if \( [q] \in (\bigcup Q) / {\cong} \), then \( q \in \bigcup Q \) and thus its coset \( [q] \) belongs to \( Q \).

  \SubProofOf{thm:lattice_theorem_for_substructures/direct_order} Let \( (S, I) \) and \( (T, I) \) be substructures of \( (X, I) \) compatible with \( {\cong} \) and suppose that \( T \subseteq S \).

  Every member \( t \) of \( T \) is also a member of \( S \), hence compatibility ensures that \( [t] \) is a member of \( S / {\cong} \).

  Therefore, \( T / {\cong} \) is a subset \( S / {\cong} \). \Fullref{thm:lattice_theorem_for_substructures/direct} ensures that both are substructures of \( (X, I) \) and \fullref{thm:substructure_relation_is_transitive} ensures that \( (T / {\cong}, I_\cong) \) is a substructure of \( (S / {\cong}, I_\cong) \).

  \SubProofOf{thm:lattice_theorem_for_substructures/reverse_order} Analogous.

  \SubProofOf{thm:lattice_theorem_for_substructures/src_complete_lattice} Let \( \seq{ (S_k, I) }_{k \in \mscrK} \) be a nonempty family of substructures of \( (X, I) \) compatible with \( {\cong} \).

  It is clear that their \hyperref[thm:intersection_substructure]{intersection structure} \( (\bigcap_{k \in \mscrK} S_k, I) \) is also a substructure of \( (X, I) \) compatible with \( {\cong} \). Then this is their infimum, that is, meet.

  Furthermore, \fullref{thm:closure_operator_from_set_semilattice} shows that the following is a \hyperref[def:moore_closure_operator]{Moore closure operator}:
  \begin{equation*}
    \begin{aligned}
      &\cl^\cong: \pow(X) \to \pow(X), \\
      &\cl^\cong(A) \coloneqq \bigcap \set{ S \subseteq X \given (S, I) \T{is a substructure of} (X, I) \T{compatible with} {\cong} }.
    \end{aligned}
  \end{equation*}

  Then we can define the supremum of the family \( \seq{ (S_k, I) }_{k \in \mscrK} \) as the structure
  \begin{equation*}
    \parens[\Bigg]{ \cl^\cong \parens[\Bigg]{ \bigcup_{k \in \mscrK} S_k }, I }.
  \end{equation*}

  From \fullref{thm:closure_operator_minimality} it follows that this is indeed the smallest suitable substructure.

  Therefore, arbitrary families of substructures compatible with \( {\cong} \) have both joins and meets, implying that the family of all such substructures is a complete lattice.

  \SubProofOf{thm:lattice_theorem_for_substructures/dest_complete_lattice} Trivial.

  \SubProofOf{thm:lattice_theorem_for_substructures/isomorphism} The map \( (S, I) \mapsto (S / {\cong}, I_\cong) \) is left invertible as per \fullref{thm:lattice_theorem_for_substructures/left_invertible} and right invertible as per \fullref{thm:lattice_theorem_for_substructures/right_invertible}. It is also order-preserving as per \fullref{thm:lattice_theorem_for_substructures/direct_order}, and is order-reflecting as per \fullref{thm:lattice_theorem_for_substructures/reverse_order}.

  Then \fullref{thm:lattice_isomorphism_characterization} is satisfied, implying that the aforementioned map is an isomorphism of complete lattices.
\end{proof}

\paragraph{First-order definability}

\begin{definition}\label{def:first_order_definability}\mcite[def. 2.3.37]{Hinman2005}
  Fix a \hyperref[def:first_order_language]{first-order language} \( \mscrL \) and a \hyperref[def:first_order_structure]{structure} \( \mscrX = (X, I) \) for \( \mscrL \).

  To every \hyperref[def:first_order_syntax/formula]{formula} \( \varphi \) whose \hyperref[def:first_order_syntax/formula_free_variables]{free variables} are among \( \xi_1, \ldots, \xi_n, \eta_1, \ldots, \eta_m \), and to every \( m \)-tuple \( u_1, \ldots, u_m \) of members of \( A \), which we call \term{parameters}, there corresponds a set \( A \subseteq X^n \) such that
  \begin{equation*}
    (x_1, \ldots, x_n) \in A \T{if and only if} \Bracks{\varphi}(x_1, \ldots, x_n, u_1, \ldots, u_m) = T,
  \end{equation*}
  where \( \Bracks{\varphi} \) denotes the valuation function defined in \fullref{def:propositional_valuation/valuation_function}.

  We say that \( \varphi \) \term{defines} \( A \) with parameters \( u_1, \ldots, u_m \). An arbitrary set \( A \subseteq X^n \) is \term{definable} with parameters \( u_1, \ldots, u_m \) if there exists a formula \( \varphi \) that defines \( A \), or simply \enquote{definable} if it is definable without parameters.
\end{definition}
\begin{comments}
  \item See \fullref{def:set_builder_notation} and \fullref{thm:cumulative_hierarchy_model_of_zfc} for how this concept deeply relates to set theory.
\end{comments}

\begin{proposition}\label{thm:automorphism_of_definable_set}
  Let \( \mscrX = (X, I) \) be a structure over some language \( \mscrL \).

  \begin{thmenum}
    \thmitem{thm:automorphism_of_definable_set/direct}\mcite[corr. 2.3.40]{Hinman2005} If the set \( A \subseteq X \) is \hyperref[def:first_order_definability]{definable} (without parameters) and if \( h: \mscrX \to \mscrX \) is an automorphism, then \( h(A) = A \).

    \thmitem{thm:automorphism_of_definable_set/contrapositive} If for some automorphism \( h: X \to X \) we have \( h(A) \neq A \), then the set \( A \subseteq X \) is not definable.
  \end{thmenum}
\end{proposition}
\begin{proof}
  \SubProofOf{thm:automorphism_of_definable_set/direct} If \( A \) is definable via \( \varphi \), then \fullref{thm:isomorphism_preserves_validity} implies that,if \( (x_1, \ldots, x_n) \in A \), then
  \begin{equation*}
    \Bracks{\varphi}_\mscrX\parens[\Big]{ h(x_1), \ldots, h(x_n) } = \Bracks{\varphi}_\mscrX(x_1, \ldots, x_n) = T,
  \end{equation*}
  which in turn implies
  \begin{equation*}
    \parens[\Big]{ h(x_1), \ldots, h(x_n) } \in A.
  \end{equation*}

  Thus, \( A = h(A) \).

  \SubProofOf{thm:automorphism_of_definable_set/contrapositive} This is the contrapositive of \fullref{thm:automorphism_of_definable_set/direct}.
\end{proof}

  \subsection{Deductive systems}\label{subsec:deductive_systems}

Without a clear context, by \enquote{logical formula} we will mean a \hyperref[def:formal_language/word]{word} over some \hyperref[def:formal_language/alphabet]{alphabet}. In practice, these will be either \hyperref[def:propositional_syntax/formula]{propositional formulas} or \hyperref[def:first_order_syntax/formula]{first-order formulas}.

It is challenging to formally define a deductive system in a way that reflects reality. We will do this via some auxiliary definitions that rely heavily on the interaction between the object logic and the \hyperref[rem:metalogic]{metalogic}.

\begin{definition}\label{def:judgment}\mimprovised
  A \term{judgment} is a logical formula of the metalogic, which is usually used for assertion. Using judgments allows us to quantify over metalogical properties. We will be interested in the following kinds of judgments:

  \begin{thmenum}
    \thmitem{def:judgment/sequent} A \term{sequent} is a judgment with at least two free variables. We denote a sequent \( \vdash \) with sets of free variables \( \Gamma \) and \( \Delta \) by
    \begin{equation*}
      \Gamma \vdash \Delta.
    \end{equation*}

    The intended interpretation for both \( \Gamma \) and \( \Delta \) is that of sets of formulas in the object logic, in which case a sequent expresses the metalogical statement that \enquote{the formulas in \( \Gamma \) collectively entail via \( \vdash \) any formula in \( \Delta \)}.

    We could have defined a sequent as a predicate, however we may not have appropriate predicate symbols on the metalanguage. This issue is discussed in \fullref{rem:predicate_formula}.

    \thmitem{def:judgment/inference_rule} An \term{inference rule} is a judgment with at least one free variable. We denote an inference rule \( R \) with free variables \( \psi \) and \( \varphi_1, \ldots, \varphi_n \) by
    \begin{equation*}
      \begin{prooftree}
        \hypo{ \varphi_1 }
        \hypo{ \cdots }
        \hypo{ \varphi_n }
        \infer3[R]{ \psi }
      \end{prooftree}
    \end{equation*}

    We allow the possibility that \( n = 0 \), in which case the rule becomes
    \begin{equation*}
      \begin{prooftree}
        \infer0[R]{ \psi }
      \end{prooftree}
    \end{equation*}

    The intended interpretation for the listed variables is that of formulas in the object logic, in which case an inference rule expresses the metalogical statement that \enquote{the premises \( \varphi_1, \ldots, \varphi_n \) collectively entail the consequence \( \psi \), as justified by the rule \( R \)}. When building complicated proofs, however, the applicability of the rule may depend on some context, and for this reason \( \varphi_1, \ldots, \varphi_n \) are often interpreted as \hyperref[def:proof_tree]{proof trees} rather than single formulas. See \fullref{def:first_order_natural_deduction_system/eigenvariables} for such an example.
  \end{thmenum}
\end{definition}

\begin{remark}\label{rem:sequents_inference_rules}
  Inference rules are both special cases and generalizations of sequents, depending on how we view them. Using an interpretation where \( \Gamma = \varphi_1, \ldots, \varphi_n \) and \( \Delta = \psi \), inference rules simply allow an alternative syntax for sequents.

  It is sometimes convenient, however, to interpret the formulas \( \varphi_1, \ldots, \varphi_n \) as sequents in some auxiliary logic between the object and metalogic, in which case we are able to express more complicated inference rules of the sort
  \begin{equation*}
    \begin{prooftree}
      \hypo{ \varphi, \Gamma \vdash \Delta, \psi }
      \infer1{ \Gamma \vdash \Delta, \varphi \to \psi }
    \end{prooftree}
  \end{equation*}

  This can be very useful when inductively defining the metalogical relation \( \vdash \).
\end{remark}

\begin{definition}\label{def:proof_tree}\mimprovised
  A \term{proof tree} is a \hyperref[def:arborescence/undirected]{rooted tree} of logical formulas. The validity of a proof tree is handled by \hyperref[def:deductive_system]{deductive system} and is irrelevant for this definition.

  \begin{thmenum}
    \thmitem{def:proof_tree/subproof} A \term{subproof} is simply a subtree of a proof. If a subproof was obtained using an \hyperref[def:judgment/inference_rule]{inference rule}, the subproof \hyperref[def:weighted_set]{labeled} with the name of the rule by the deductive systems.

    \thmitem{def:proof_tree/premises} The root of the tree is called the \term{conclusion} of the proof and the \hyperref[def:arborescence/ancestry]{leaves} are called \term{premises} or, in the context of \hyperref[def:natural_deduction_system]{natural deduction}, \term{assumptions}. We sometimes add a \term{non-premise} label to a leaf that prevents it from being added to the list of premises.

    \thmitem{def:proof_tree/drawing} We will draw graphically proof trees with no edges and with the root at the bottom in a style inspired by inference rules. See \fullref{ex:def:positive_implicational_deductive_system/identity} for an example.
  \end{thmenum}
\end{definition}

\begin{definition}\label{def:deductive_system}\mimprovised
  A \term{deductive system} for a set \( \mscrF \) of formulas in the object logic is a metalogical collection of \hyperref[def:judgment/inference_rule]{inference rules}, which are used to generate \hyperref[def:proof_tree]{proof trees} in the object logic.

  We will not attempt to encode inference rules themselves into the object theory and instead regard a deductive system as a pair \( (\mscrF, \mscrP) \), where \( \mscrF \) is a set of logical formulas and \( \mscrP \) is a set of proofs over \( \mscrF \).

  We define the set \( \mscrP \) of proofs via \hyperref[thm:structural_recursion]{structural recursion}.
  \begin{thmenum}
    \thmitem{def:deductive_system/base} For every formula \( \varphi \) in \( \mscrF \), the tree with root \( \varphi \) and no children is a proof \( \mscrP \).

    \thmitem{def:deductive_system/rule} Let \( \varphi_1, \ldots, \varphi_n \) and \( \psi \) be formulas in \( \mscrF \) and \( P_1, \ldots, P_n \) be proofs of \( \varphi_1, \ldots, \varphi_n \).

    Suppose that the deductive system has an inference rule
    \begin{equation*}
      \begin{prooftree}
        \hypo{ \Phi_1 }
        \hypo{ \cdots }
        \hypo{ \Phi_n }
        \infer3[R]{ \Psi }
      \end{prooftree}
    \end{equation*}
    such that
    \begin{equation*}
      R\Bracks{ \Phi_1 \mapsto P_1, \ldots, \Phi_n \mapsto P_n, \Psi \mapsto \psi } = T
    \end{equation*}
    in the metalogic.

    Then the tree with root \( \psi \), subtrees \( P_1, \ldots, P_n \) of the root, and label \( R \), is a proof in \( \mscrP \). In the case of rules with no premises like \eqref{eq:def:minimal_propositional_natural_deduction_system/top/intro}, we add a \hyperref[def:proof_tree/premises]{non-premise label} to the proof in order to exclude \( \psi \) from the list of premises of the proof.

    We say that the resulting proof
    \begin{equation*}
      \begin{prooftree}
        \hypo{ P_1 }
        \hypo{ \cdots }
        \hypo{ P_n }
        \infer3[r]{ \psi }
      \end{prooftree}
    \end{equation*}
    is an \term{application} of the rule \( R \).
  \end{thmenum}
\end{definition}

\begin{definition}\label{def:proof_derivability}\mimprovised
  We are often interested not in the proofs of a \hyperref[def:deductive_system]{deductive system}, but in \term{provability}, which we express via \hyperref[def:judgment/sequent]{sequents}.

  Fix a deductive system \( (\mscrF, \mscrP) \). If \( \mscrP \) contains a proof of \( \varphi \), whose premises are all members of \( \Gamma \), the following sequent is valid:
  \begin{equation*}
    \Gamma \vdash \varphi.
  \end{equation*}

  We say that \( \varphi \) is a \term{theorem} of \( \Gamma \). If \( \Gamma \) is empty, we say that \( \varphi \) is a \term{logical theorem}.

  Furthermore, \( \vdash \) is a reflexive and transitive relation, which makes \( (\pow(\mscrF), \vdash) \) a \hyperref[def:preordered_set]{preordered set}.
\end{definition}

\begin{definition}\label{def:axiomatic_deductive_system}\mimprovised
  \term{Axiomatic deductive systems}, also called \term{Hilbert-style systems}, are \hyperref[def:deductive_system]{deductive systems} for propositional formulas consist of the single \hyperref[def:judgment/inference_rule]{inference rule} \term[en=mode that by affirming affirms]{modus ponens}:
  \begin{equation*}\taglabel[\textrm{MP}]{eq:def:def:axiomatic_deductive_system/mp}
    \begin{prooftree}
      \hypo{ \varphi }
      \hypo{ \varphi \rightarrow \psi }
      \infer2[\ref{eq:def:def:axiomatic_deductive_system/mp}]{ \psi }
    \end{prooftree}
  \end{equation*}

  Fix a set \( \mscrF \) of logical formulas. Let \( \mscrA \subseteq \mscrF \) be a predefined subset of formulas, which we will call \term{logical axioms} of \( \mscrF \). The axiomatic deductive system itself is the pair \( (\mscrF, \mscrA) \).

  Given a proof of the deductive system, we split its premises into \term{logical axioms} and \term{non-logical axioms} depending on whether they belong to \( \mscrA \) or not.

  Note that we cannot have a complete axiomatic deductive system for first-order logic because of the eigenvariable condition in the rules \eqref{eq:def:first_order_natural_deduction_system/forall/intro} and \eqref{eq:def:first_order_natural_deduction_system/exists/elim}.
\end{definition}

\begin{proposition}\label{thm:deductive_system_transitivity}
  Given a deductive system, if \( \Gamma \vdash \varphi \), then \( \Gamma, \Delta \vdash \varphi \) for any formula \( \varphi \) and any sets \( \Gamma \) and \( \Delta \).

  If every formula in \( \Delta \) is derivable from \( \Gamma \), then the converse also holds: \( \Gamma, \Delta \vdash \varphi \) implies \( \Gamma \vdash \varphi \).
\end{proposition}
\begin{proof}
  If there exists a proof \( \varphi \) from \( \Gamma \), then adding additional axioms does not change anything.

  The second part of the theorem has a tad more complicated proof. Assume that every formula in \( \Delta \) is derivable from \( \Gamma \) and that \( \Gamma, \Delta \vdash \varphi \).

  For every \( \delta \in \Delta \), let \( P_\delta \) be a proof of \( \delta \) from members of \( \Gamma \) and let \( P_\varphi \) be a proof of \( \varphi \) from \( \Gamma \cup \Delta \).

  Then, for every \( \delta \in \Delta \), we can replace the subtree of \( \delta \) with \( P_\delta \) to obtain a proof of \( \varphi \) from \( \Gamma \).

  Therefore, \( \Gamma \vdash \varphi \).
\end{proof}

\begin{definition}\mcite[180]{Wasilewska2018}\label{def:positive_implicational_deductive_system}
  The \term{positive implicational propositional deductive system} is an extraordinarily simple \hyperref[def:axiomatic_deductive_system]{axiomatic deductive system}.

  It is based on the \hyperref[def:propositional_language]{language of propositional logic}, but limited to formulas containing only the \hyperref[def:propositional_language/connectives/conditional]{conditional connective} \( \rightarrow \), without any \hyperref[def:propositional_language/constants]{propositional constants} or \hyperref[def:propositional_language/negation]{negation}. Note that this is only a special case of \hyperref[def:positive_formula]{positive formulas}.

  The adjective \enquote{positive} in the name of the system refers to the impossibility to negate a formula. \enquote{Implicational} refers to the fact that all formulas are \hyperref[def:material_implication]{material implications} and the \hyperref[eq:def:def:axiomatic_deductive_system/mp]{sole inference rule} is based on eliminating the connective.

  The system has the following logical axiom schemas:
  \begin{thmenum}
    \thmitem{def:positive_implicational_deductive_system/intro} For every formula \( \varphi \), we can \enquote{introduce} an \hyperref[def:material_implication]{implication} whose consequent is \( \varphi \) and whose antecedent is any other formula \( \psi \):
    \begin{equation}\label{eq:def:positive_implicational_deductive_system/intro}
      \varphi \rightarrow (\psi \rightarrow \varphi) \tag{\textrm{A} \( \rightarrow^+ \)}.
    \end{equation}

    \thmitem{def:positive_implicational_deductive_system/trans} Implication is transitive:
    \begin{equation}\label{eq:def:positive_implicational_deductive_system/trans}
      \parens[\Big]{ \varphi \rightarrow (\psi \rightarrow \theta) } \rightarrow \parens[\Big]{ (\varphi \rightarrow \psi) \rightarrow (\varphi \rightarrow \theta)} \tag{\textrm{A} \( \twoheadrightarrow \)}.
    \end{equation}
  \end{thmenum}
\end{definition}

\begin{example}\mcite[180]{Wasilewska2018}\label{ex:def:positive_implicational_deductive_system/identity}
  Fix any \hyperref[def:positive_implicational_deductive_system]{positive implicational formula} \( \varphi \). We will construct a derivation of the implication
  \begin{equation}\label{eq:ex:def:positive_implicational_deductive_system/identity}
    \varphi \rightarrow \varphi.
  \end{equation}

  We derive the proof from the two logical axioms:
  \begin{equation}\label{eq:ex:def:positive_implicational_deductive_system/identity/proof}
    \begin{prooftree}[separation=3em]
      \hypo
        {
          \eqref{eq:def:positive_implicational_deductive_system/intro}
        }

      \ellipsis
        {
          \( \begin{array}{l}
            \psi \mapsto (\varphi \rightarrow \varphi)
            \\
            \mbox{}
          \end{array} \)
        }
        {
          \eqref{eq:ex:propositional_positive_implicational_logic/dagger}
        }

      \hypo
        {
          \eqref{eq:def:positive_implicational_deductive_system/trans}
        }

      \ellipsis
        {
          \( \begin{array}{l}
            \psi \mapsto (\varphi \rightarrow \varphi)
            \\
            \theta \mapsto \varphi
          \end{array} \)
        }
        {
          \eqref{eq:ex:propositional_positive_implicational_logic/dagger}
          \rightarrow ((\varphi \rightarrow (\varphi \rightarrow \varphi)) \rightarrow (\varphi \rightarrow \varphi))
        }

      \infer2[\ref{eq:def:def:axiomatic_deductive_system/mp}]{(\varphi \rightarrow (\varphi \rightarrow \varphi)) \rightarrow (\varphi \rightarrow \varphi)}

      \hypo
        {
          \eqref{eq:def:positive_implicational_deductive_system/intro}
        }

      \ellipsis
        {
          \( \psi \mapsto \varphi \)
        }
        {
          \varphi \rightarrow (\varphi \rightarrow \varphi)
        }

      \infer2[\ref{eq:def:def:axiomatic_deductive_system/mp}]{\varphi \rightarrow \varphi}
    \end{prooftree}
  \end{equation}
  where
  \begin{equation}\label{eq:ex:propositional_positive_implicational_logic/dagger}
    \varphi \rightarrow ((\varphi \rightarrow \varphi) \rightarrow \varphi).
  \end{equation}

  The only assumptions used in the derivation were logical axioms, hence \eqref{eq:ex:def:positive_implicational_deductive_system/identity} is a logical theorem.
\end{example}

\begin{definition}\label{def:derivability_and_satisfiability}\mcite[146]{Wasilewska2018}
  We introduce two notions connecting \hyperref[def:proof_derivability]{derivability} and \hyperref[def:first_order_semantics/satisfiability]{satisfiability}:
  \begin{thmenum}
    \thmitem{def:derivability_and_satisfiability/soundness} If, for any closed formula \( \varphi \), derivability \( \vdash \varphi \) implies satisfiability \( \vDash \varphi \), we say that the deductive system is \term{sound} with respect to the semantical framework.

    \thmitem{def:derivability_and_satisfiability/completeness} Dually, if satisfiability \( \vDash \varphi \) implies derivability \( \vdash \varphi \) for any closed formula \( \varphi \), we say that the deductive system is \term{complete} with respect to the semantical framework.
  \end{thmenum}
\end{definition}
\begin{thmcomment}
  We restrict our attention to closed formulas because we wish to avoid the problems described in \fullref{rem:deduction_with_free_variables}. If we have a formula with free variables, we may simply take its \hyperref[thm:implicit_universal_quantification]{universal closure}.
\end{thmcomment}

\begin{proposition}\label{thm:soundness_of_positive_implicational_propositional_deductive_system}\mcite[thm. 5.1]{Wasilewska2018}
  The \hyperref[def:positive_implicational_deductive_system]{positive implicational propositional deductive system} is \hyperref[def:derivability_and_satisfiability/soundness]{sound} with respect to \hyperref[def:propositional_semantics]{classical semantics}.
\end{proposition}

\begin{theorem}[Syntactic deduction theorem]\label{thm:syntactic_deduction_theorem}\mcite[thm. 5.1]{Wasilewska2018}
  In the \hyperref[def:positive_implicational_deductive_system]{positive implicational deductive system}, \( \Gamma, \psi \vdash \varphi \) holds if and only if \( \Gamma \vdash \psi \rightarrow \varphi \) holds.
\end{theorem}
\begin{thmcomment}
  This theorem also holds for propositional deductive systems which extend the positive implication system with compatible rules, as in the case of \fullref{def:minimal_propositional_natural_deduction_system} or \fullref{def:first_order_natural_deduction_system}.
\end{thmcomment}
\begin{proof}
  \SufficiencySubProof Suppose that \( \Gamma, \psi \vdash \varphi \) and let \( P \) be a proof of \( \varphi \) from \( \Gamma \cup \set{ \psi } \). We will use \fullref{thm:structural_induction_on_unambiguous_grammars} on \( \varphi \).

  \begin{itemize}
    \item First suppose that \( \varphi \) is either a logical or a nonlogical axiom; in the latter case either \( \varphi \in \Gamma \) or \( \varphi = \psi \). In all three cases, the logical axiom \eqref{eq:def:positive_implicational_deductive_system/intro} allows us to derive \( \psi \rightarrow \varphi \) from \( \Gamma \) using \eqref{eq:def:def:axiomatic_deductive_system/mp}.

    \item Otherwise, since the only rule is \eqref{eq:def:def:axiomatic_deductive_system/mp}, there exists some formula \( \theta \) derivable from \( \Gamma \cup \set{ \psi } \) such that \( P \) contains the formulas \( \theta \) and for \( \theta \rightarrow \varphi \). The inductive hypothesis holds for both, and hence \( \Gamma \vdash \psi \to \theta \) and \( \Gamma \vdash \psi \to (\theta \rightarrow \varphi) \). Let \( P_1 \) and \( P_2 \) be proofs corresponding to these two sequents.

    We can now build the following proof of \( \psi \to \varphi \) from \( \Gamma \):
    \begin{equation*}
      \begin{prooftree}
        \hypo{ P_2 }

        \hypo{ \eqref{eq:def:positive_implicational_deductive_system/trans} }
        \ellipsis
          {}
          {
            \parens[\Big]{ \psi \rightarrow (\theta \rightarrow \varphi) } \rightarrow \parens[\Big]{ (\psi \rightarrow \theta) \rightarrow (\psi \rightarrow \varphi)}
          }

        \infer2[\ref{eq:def:def:axiomatic_deductive_system/mp}]{ (\psi \rightarrow \theta) \rightarrow (\psi \rightarrow \varphi) }

        \hypo{ P_1 }
        \infer2[\ref{eq:def:def:axiomatic_deductive_system/mp}]{ \psi \rightarrow \varphi }
      \end{prooftree}
    \end{equation*}
  \end{itemize}

  \NecessitySubProof Now suppose that \( \Gamma \vdash \psi \rightarrow \varphi \). Then we can apply \eqref{eq:def:def:axiomatic_deductive_system/mp} to obtain \( \varphi \) from \( \Gamma \cup \set{ \psi } \).
\end{proof}

\begin{definition}\label{def:minimal_propositional_axiomatic_deductive_system}\mcite[308]{Wasilewska2018}
  While the \hyperref[def:positive_implicational_deductive_system]{positive implicational propositional deductive system} is simple, it is of more practical use to have all propositional connectives available. As it turns out, we cannot utilize \hyperref[ex:thm:posts_completeness_theorem]{complete families of Boolean functions} unless we are dealing with \hyperref[def:propositional_semantics]{classical semantics} --- see for example \fullref{ex:heyting_semantics_lem_counterexample} and \fullref{ex:topological_semantics_lem_counterexample} for how \fullref{thm:boolean_equivalences/conditional_as_disjunction} fails to hold.

  Our goal is to define the (axiomatic) \term{minimal propositional axiomatic deductive system}, which would correspond to \hyperref[def:minimal_logic]{minimal logic}. It is axiomatic in the sense that we do not use new rules to express the rest of the propositional syntax, but instead we need axiom schemas for each connective. The only exception is \hyperref[def:propositional_language/constants/verum]{\( \bot \)}, the axioms for which tend to change semantics by a lot --- see \fullref{thm:minimal_propositional_negation_laws}.

  Axioms with \( + \) in the superscript are called \term{introduction axioms} and axioms with \( - \) are called \term{elimination axioms}.

  The following axioms are essential in the sense that they cannot be defined in terms of others:
  \begin{thmenum}[series=def:minimal_propositional_axiomatic_deductive_system]
    \thmitem{def:minimal_propositional_axiomatic_deductive_system/top} The simplest axiom states that the constant \hyperref[def:propositional_language/constants/verum]{\( \top \)} is itself an axiom:
    \begin{equation}\label{eq:def:minimal_propositional_axiomatic_deductive_system/top/intro}
      \top \tag{\textrm{A} \( \top^+ \)}
    \end{equation}

    \thmitem{def:minimal_propositional_axiomatic_deductive_system/and} Axioms for \hyperref[def:propositional_language/connectives/conjunction]{conjunction}:
    \begin{align}
      \mathllap{ (\varphi \wedge \psi) } &\rightarrow \mathrlap{ \psi } \tag{\textrm{A} \( \wedge_L^- \)} \label{eq:def:minimal_propositional_axiomatic_deductive_system/and/elim_left} \\
      \mathllap{ (\varphi \wedge \psi) } &\rightarrow \mathrlap{ \varphi } \tag{\textrm{A} \( \wedge_R^- \)} \label{eq:def:minimal_propositional_axiomatic_deductive_system/and/elim_right} \\
      \mathllap{ \varphi }               &\rightarrow \mathrlap{ \parens[\Big]{ \psi \rightarrow (\varphi \wedge \psi) } } \tag{\textrm{A} \( \wedge^+ \)} \label{eq:def:minimal_propositional_axiomatic_deductive_system/and/intro}
    \end{align}

    \thmitem{def:minimal_propositional_axiomatic_deductive_system/or} Axioms for \hyperref[def:propositional_language/connectives/disjunction]{disjunction}:
    \begin{align}
      \mathllap{ \varphi }                      &\rightarrow \mathrlap{ (\varphi \vee \psi) } \tag{\textrm{A} \( \vee_L^+ \)} \label{eq:def:minimal_propositional_axiomatic_deductive_system/or/intro_left} \\
      \mathllap{ \psi }                      &\rightarrow \mathrlap{ (\varphi \vee \psi) } \tag{\textrm{A} \( \vee_R^+ \)} \label{eq:def:minimal_propositional_axiomatic_deductive_system/or/intro_right} \\
      \mathllap{ (\varphi \rightarrow \theta) } &\rightarrow \mathrlap{ \parens[\Big]{ (\psi \rightarrow \theta) \rightarrow ((\varphi \vee \psi) \rightarrow \theta) } } \tag{\textrm{A} \( \vee^- \)} \label{eq:def:minimal_propositional_axiomatic_deductive_system/or/elim}
    \end{align}
  \end{thmenum}

  The following axioms and are said to be \enquote{abbreviations} and do not affect semantics:
  \begin{thmenum}[resume=def:minimal_propositional_axiomatic_deductive_system]
    \thmitem{def:minimal_propositional_axiomatic_deductive_system/iff} The axioms for the biconditional are motivated by \fullref{thm:boolean_equivalences/biconditional_via_conditionals}:
    \begin{align}
      \mathllap{ (\varphi \rightarrow \psi)     } &\rightarrow \mathrlap{ \parens[\Big]{ (\psi \rightarrow \varphi) \rightarrow (\varphi \leftrightarrow \psi) } } \tag{\textrm{A} \( \leftrightarrow^+ \)} \label{def:minimal_propositional_axiomatic_deductive_system/iff/intro} \\
      \mathllap{ (\varphi \leftrightarrow \psi)  }&\rightarrow \mathrlap{ (\varphi \rightarrow \psi) } \tag{\textrm{A} \( \leftrightarrow_L^- \)} \label{eq:def:minimal_propositional_axiomatic_deductive_system/iff/elim_left} \\
      \mathllap{ (\varphi \leftrightarrow \psi) } &\rightarrow \mathrlap{ (\psi \rightarrow \varphi) } \tag{\textrm{A} \( \leftrightarrow_R^- \)} \label{eq:def:minimal_propositional_axiomatic_deductive_system/iff/elim_right}
    \end{align}

    \thmitem{def:minimal_propositional_axiomatic_deductive_system/negation} The axioms for negation are motivated by \fullref{thm:boolean_equivalences/negation_bottom}:
    \begin{align}
      \mathllap{ \neg \varphi }               &\rightarrow \mathrlap{ (\varphi \rightarrow \bot) } \tag{\textrm{A} \( \neg^- \)} \label{eq:def:minimal_propositional_axiomatic_deductive_system/neg/elim} \\
      \mathllap{ (\varphi \rightarrow \bot) } &\rightarrow \mathrlap{ \neg \varphi } \tag{\textrm{A} \( \neg^+ \)} \label{eq:def:minimal_propositional_axiomatic_deductive_system/neg/intro}
    \end{align}
  \end{thmenum}
\end{definition}

\begin{definition}\label{def:natural_deduction_system}\mimprovised
  \term{Natural deductive systems} are \hyperref[def:deductive_system]{deductive systems} whose set of rules allows \term{discharging} certain assumptions of the proof tree. These rules correspond to \enquote{bringing in} the sequent \( \varphi \vdash \psi \) as a formula \( \varphi \to \psi \), thus eliminating \( \varphi \) as an assumption as justified by \fullref{thm:syntactic_deduction_theorem}.

  In a natural deduction system, all assumptions in a \hyperref[def:proof_tree]{proof trees} are labeled differently. When applying a rule that supports discharging assumptions, we add an additional \hyperref[def:weighted_set]{label} to the subproof that matches the label of the assumption which we discharge.

  This allows us to distinguish between \term{discharged assumptions} and \term{undischarged assumptions}. We add a \hyperref[def:proof_tree/premises]{non-premise label} to every discharged assumption so that it does not affect \hyperref[def:proof_derivability]{derivability}.
\end{definition}

\begin{definition}\label{def:minimal_propositional_natural_deduction_system}\mcite[sec. 1.4]{VanDalen2004}
  We define the \term{minimal propositional natural deduction system}, which is the \hyperref[def:natural_deduction_system]{natural deduction} equivalent of the \hyperref[def:minimal_propositional_axiomatic_deductive_system]{minimal propositional axiomatic deductive system}.

  \begin{thmenum}
    \thmitem{def:minimal_propositional_natural_deduction_system/imp} The following rules corresponds to the conditional axiom schemas in \fullref{def:positive_implicational_deductive_system}:

    \begin{minipage}[t]{0.45\textwidth}
      This rule is inspired by \eqref{eq:def:positive_implicational_deductive_system/intro}:
      \begin{equation*}\taglabel[\( \rightarrow^+ \)]{eq:def:minimal_propositional_natural_deduction_system/imp/intro}
        \begin{prooftree}
          \hypo{ [\psi]^n }
          \ellipsis {} { \varphi }
          \infer[left label=\( n \)]1[\ref{eq:def:minimal_propositional_natural_deduction_system/imp/intro}]{ \psi \rightarrow \varphi }
        \end{prooftree}
      \end{equation*}
    \end{minipage}
    \hfill
    \begin{minipage}[t]{0.45\textwidth}
      This rule is merely a renaming of \eqref{eq:def:def:axiomatic_deductive_system/mp}:
      \begin{equation*}\taglabel[\( \rightarrow^- \)]{eq:def:minimal_propositional_natural_deduction_system/imp/elim}
        \begin{prooftree}
          \hypo{ \varphi \rightarrow \psi }
          \hypo{ \varphi }
          \infer2[\ref{eq:def:minimal_propositional_natural_deduction_system/imp/elim}]{ \psi }
        \end{prooftree}
      \end{equation*}
    \end{minipage}

    The additional notation in \eqref{eq:def:minimal_propositional_natural_deduction_system/imp/intro} means that the premise labeled with \( n \), if any, can be discharged.

    Note that there is no rule corresponding to \eqref{eq:def:positive_implicational_deductive_system/trans} because this axiom schema follows from \eqref{eq:def:minimal_propositional_natural_deduction_system/imp/intro} and \eqref{eq:def:minimal_propositional_natural_deduction_system/imp/elim}. Unlike in the axiomatic deductive system where \eqref{eq:def:positive_implicational_deductive_system/trans} is used to prove \fullref{thm:syntactic_deduction_theorem}, here we have a stronger connection between \( \rightarrow \) in the object language and \( \vdash \) in the metalanguage given by \eqref{eq:def:minimal_propositional_natural_deduction_system/imp/intro}.

    \thmitem{def:minimal_propositional_natural_deduction_system/top} The following rule corresponds to the axiom \eqref{eq:def:minimal_propositional_axiomatic_deductive_system/top/intro}:
    \begin{equation*}\taglabel[\( \top^+ \)]{eq:def:minimal_propositional_natural_deduction_system/top/intro}
      \begin{prooftree}
        \infer0[\ref{eq:def:minimal_propositional_natural_deduction_system/top/intro}]{ \top }
      \end{prooftree}
    \end{equation*}

    As discussed in \fullref{def:deductive_system/rule}, applications of this rule have a \hyperref[def:proof_tree/premises]{non-premise label} in order to prevent \( \top \) as an undischarged assumption.

    \thmitem{def:minimal_propositional_natural_deduction_system/and} The following rules corresponds to the conjunction axiom schemas in \fullref{def:minimal_propositional_axiomatic_deductive_system/and}:

    \begin{minipage}{0.3\textwidth}
      \begin{equation*}\taglabel[\( \wedge^+ \)]{eq:def:minimal_propositional_natural_deduction_system/and/intro}
        \begin{prooftree}
          \hypo{ \varphi }
          \hypo{ \psi }
          \infer2[\ref{eq:def:minimal_propositional_natural_deduction_system/and/intro}]{ \varphi \wedge \psi }
        \end{prooftree}
      \end{equation*}
    \end{minipage}
    \hfill
    \begin{minipage}{0.3\textwidth}
      \begin{equation*}\taglabel[\( \wedge_L^- \)]{eq:def:minimal_propositional_natural_deduction_system/and/elim_left}
        \begin{prooftree}
          \hypo{ \varphi \wedge \psi }
          \infer1[\ref{eq:def:minimal_propositional_natural_deduction_system/and/elim_left}]{ \psi }
        \end{prooftree}
      \end{equation*}
    \end{minipage}
    \hfill
    \begin{minipage}{0.3\textwidth}
      \begin{equation*}\taglabel[\( \wedge_R^- \)]{eq:def:minimal_propositional_natural_deduction_system/and/elim_right}
        \begin{prooftree}
          \hypo{ \varphi \wedge \psi }
          \infer1[\ref{eq:def:minimal_propositional_natural_deduction_system/and/elim_right}]{ \varphi }
        \end{prooftree}
      \end{equation*}
    \end{minipage}

    \thmitem{def:minimal_propositional_natural_deduction_system/or} The following rules corresponds to the disjunction axiom schemas in \fullref{def:minimal_propositional_axiomatic_deductive_system/or}:

    \begin{minipage}{0.3\textwidth}
      \begin{equation*}\taglabel[\( \vee_L^+ \)]{eq:def:minimal_propositional_natural_deduction_system/or/intro_left}
        \begin{prooftree}
          \hypo{ \varphi }
          \infer1[\ref{eq:def:minimal_propositional_natural_deduction_system/or/intro_left}]{ \varphi \vee \psi }
        \end{prooftree}
      \end{equation*}
    \end{minipage}
    \hfill
    \begin{minipage}{0.3\textwidth}
      \begin{equation*}\taglabel[\( \vee_R^+ \)]{eq:def:minimal_propositional_natural_deduction_system/or/intro_right}
        \begin{prooftree}
          \hypo{ \psi }
          \infer1[\ref{eq:def:minimal_propositional_natural_deduction_system/or/intro_right}]{ \varphi \vee \psi }
        \end{prooftree}
      \end{equation*}
    \end{minipage}
    \hfill
    \begin{minipage}{0.3\textwidth}
      \begin{equation*}\taglabel[\( \vee^- \)]{eq:def:minimal_propositional_natural_deduction_system/or/elim}
        \begin{prooftree}
          \hypo{ \varphi \vee \psi }
          \hypo{ [\varphi]^n }
          \ellipsis {} { \theta }
          \hypo{ [\psi]^n }
          \ellipsis {} { \theta }
          \infer[left label=\( n \)]3[\ref{eq:def:minimal_propositional_natural_deduction_system/or/elim}]{ \theta }
        \end{prooftree}
      \end{equation*}
    \end{minipage}

    \thmitem{def:minimal_propositional_natural_deduction_system/iff} The following rules corresponds to the biconditional axiom schemas in \fullref{def:minimal_propositional_axiomatic_deductive_system/iff}:

    \begin{minipage}{0.3\textwidth}
      \begin{equation*}\taglabel[\( \leftrightarrow^+ \)]{eq:def:minimal_propositional_natural_deduction_system/iff/intro}
        \begin{prooftree}
          \hypo{ [\varphi]^n }
          \ellipsis {} { \psi }
          \hypo{ [\psi]^n }
          \ellipsis {} { \varphi }
          \infer[left label=\( n \)]2[\ref{eq:def:minimal_propositional_natural_deduction_system/iff/intro}]{ \varphi \leftrightarrow \psi }
        \end{prooftree}
      \end{equation*}
    \end{minipage}
    \hfill
    \begin{minipage}{0.3\textwidth}
      \begin{equation*}\taglabel[\( \leftrightarrow_L^- \)]{eq:def:minimal_propositional_natural_deduction_system/iff/elim_left}
        \begin{prooftree}
          \hypo{ \varphi \leftrightarrow \psi }
          \hypo{ \psi }
          \infer2[\ref{eq:def:minimal_propositional_natural_deduction_system/iff/elim_left}]{ \varphi }
        \end{prooftree}
      \end{equation*}
    \end{minipage}
    \hfill
    \begin{minipage}{0.3\textwidth}
      \begin{equation*}\taglabel[\( \leftrightarrow_R^- \)]{eq:def:minimal_propositional_natural_deduction_system/iff/elim_right}
        \begin{prooftree}
          \hypo{ \varphi \leftrightarrow \psi }
          \hypo{ \varphi }
          \infer2[\ref{eq:def:minimal_propositional_natural_deduction_system/iff/elim_right}]{ \psi }
        \end{prooftree}
      \end{equation*}
    \end{minipage}

    \thmitem{def:minimal_propositional_natural_deduction_system/negation} The following rules corresponds to the negation axiom schemas in \fullref{def:minimal_propositional_axiomatic_deductive_system/negation}:

    \begin{minipage}{0.45\textwidth}
      \begin{equation*}\taglabel[\( \neg^+ \)]{eq:def:minimal_propositional_natural_deduction_system/neg/intro}
        \begin{prooftree}
          \hypo{ [\varphi]^n }
          \ellipsis {} { \bot }
          \infer[left label=\( n \)]1[\ref{eq:def:minimal_propositional_natural_deduction_system/neg/intro}]{ \neg \varphi }
        \end{prooftree}
      \end{equation*}
    \end{minipage}
    \hfill
    \begin{minipage}{0.45\textwidth}
      \begin{equation*}\taglabel[\( \neg^- \)]{eq:def:minimal_propositional_natural_deduction_system/neg/elim}
        \begin{prooftree}
          \hypo{ \varphi }
          \hypo{ \neg \varphi }
          \infer2[\ref{eq:def:minimal_propositional_natural_deduction_system/neg/elim}]{ \bot }
        \end{prooftree}
      \end{equation*}
    \end{minipage}
  \end{thmenum}
\end{definition}
\begin{defproof}
  We will prove that the axiomatic \hyperref[def:minimal_propositional_axiomatic_deductive_system]{minimal propositional axiomatic deductive system} is equivalent to the rules of natural deduction described in this proposition.

  \SubProofOf{def:minimal_propositional_natural_deduction_system/imp} Consider first the axiom \eqref{eq:def:positive_implicational_deductive_system/intro}. Fix two formulas \( \varphi \) and \( \psi \). Then \( \varphi \rightarrow (\psi \rightarrow \varphi) \) is an instance of \eqref{eq:def:positive_implicational_deductive_system/intro}. Thus, we obtain \( \varphi \vdash \psi \rightarrow \varphi \) by applying \eqref{eq:def:def:axiomatic_deductive_system/mp}, which in turn shows the validity of the rule \eqref{eq:def:minimal_propositional_natural_deduction_system/imp/intro}.

  The labeled assumption here is essential for showing that \eqref{eq:def:minimal_propositional_natural_deduction_system/imp/intro} implies \eqref{eq:def:positive_implicational_deductive_system/intro}. Without it we would have the rule
  \begin{equation*}
    \begin{prooftree}
      \hypo{ \psi }
      \infer1{ \varphi \rightarrow \psi }
    \end{prooftree}
  \end{equation*}
  which would not allow us to discharge the assumption \( \varphi \) when it is in fact immaterial for the validity of \( \psi \).

  Now we will show that \eqref{eq:def:positive_implicational_deductive_system/trans} can be derived using only the rules \eqref{eq:def:minimal_propositional_natural_deduction_system/imp/intro} and \eqref{eq:def:minimal_propositional_natural_deduction_system/imp/elim}:
  \begin{equation}\label{eq:def:minimal_propositional_natural_deduction_system/imp/trans_derivation}
    \begin{prooftree}
      \hypo{ [\varphi \rightarrow (\psi \rightarrow \theta)]^1 }
      \hypo{ [\varphi]^2 }
      \infer2[\ref{eq:def:minimal_propositional_natural_deduction_system/imp/elim}]{ \psi \rightarrow \theta }

      \hypo{ [\varphi \rightarrow \psi]^3 }
      \hypo{ [\varphi]^2 }
      \infer2[\ref{eq:def:minimal_propositional_natural_deduction_system/imp/elim}]{ \psi }

      \infer2[\ref{eq:def:minimal_propositional_natural_deduction_system/imp/elim}]{ \theta }

      \infer[left label=\( 2 \)]1[\ref{eq:def:minimal_propositional_natural_deduction_system/imp/intro}]{ \varphi \rightarrow \theta }
      \infer[left label=\( 3 \)]1[\ref{eq:def:minimal_propositional_natural_deduction_system/imp/intro}]{ (\varphi \rightarrow \psi) \rightarrow (\varphi \rightarrow \theta) }
      \infer[left label=\( 1 \)]1[\ref{eq:def:minimal_propositional_natural_deduction_system/imp/intro}]{ \eqref{eq:def:positive_implicational_deductive_system/trans} }
    \end{prooftree}
  \end{equation}

  \SubProofOf{def:minimal_propositional_natural_deduction_system/top} Obvious.

  \SubProofOf{def:minimal_propositional_natural_deduction_system/and} The rule \eqref{eq:def:minimal_propositional_axiomatic_deductive_system/and/intro} is equivalent by more readable than proving \( \set{ \varphi, \psi } \vdash \varphi \wedge \psi \) directly. Indeed, compare it to
  \begin{equation*}
    \begin{prooftree}
      \hypo{ \varphi }
      \hypo{ \eqref{eq:def:minimal_propositional_axiomatic_deductive_system/and/intro} }
      \infer2[\ref{eq:def:def:axiomatic_deductive_system/mp}]{ \psi \rightarrow (\varphi \wedge \psi) }

      \hypo{ \psi }
      \infer2[\ref{eq:def:def:axiomatic_deductive_system/mp}]{ \varphi \wedge \psi },
    \end{prooftree}
  \end{equation*}
  which is a derivation of \( \varphi \wedge \psi \) from \( \set{ \varphi, \psi } \) using the axiomatic system. The other direction is also simple:
  \begin{equation}\label{eq:def:minimal_propositional_natural_deduction_system/and_intro_axiom_derivation}
    \begin{prooftree}
      \hypo{ [\varphi]^1 }
      \hypo{ [\psi]^2 }
      \infer2[\ref{eq:def:minimal_propositional_natural_deduction_system/and/intro}]{ \varphi \wedge \psi }
      \infer[left label=\( 2 \)]1[\ref{eq:def:minimal_propositional_natural_deduction_system/imp/intro}]{ \psi \rightarrow (\varphi \wedge \psi) },
      \infer[left label=\( 1 \)]1[\ref{eq:def:minimal_propositional_natural_deduction_system/imp/intro}]{ \eqref{eq:def:minimal_propositional_axiomatic_deductive_system/and/intro} },
    \end{prooftree}
  \end{equation}

  The other two rules are trivially connected to the corresponding axioms using a single application of \eqref{eq:def:def:axiomatic_deductive_system/mp}.

  \SubProofOf{def:minimal_propositional_natural_deduction_system/or} For a more complicated example, consider \eqref{eq:def:minimal_propositional_axiomatic_deductive_system/or/elim}. We have
  \begin{equation*}
    \begin{prooftree}
      \hypo{ \eqref{eq:def:minimal_propositional_axiomatic_deductive_system/or/elim} }
      \hypo{ \varphi \rightarrow \theta }
      \infer2[\ref{eq:def:def:axiomatic_deductive_system/mp}]{ (\psi \rightarrow \theta) \rightarrow ((\varphi \vee \psi) \rightarrow \theta) },

      \hypo{ \psi \rightarrow \theta }
      \infer2[\ref{eq:def:def:axiomatic_deductive_system/mp}]{ (\varphi \vee \psi) \rightarrow \theta }.

      \hypo{ \varphi \vee \psi }
      \infer2[\ref{eq:def:def:axiomatic_deductive_system/mp}]{ \theta }.
    \end{prooftree}
  \end{equation*}

  The assumptions of this derivations are \( \varphi \rightarrow \theta \), \( \psi \rightarrow \theta \) and \( \varphi \vee \psi \). Instead of adding them directly as premises of the inference rule \eqref{eq:def:minimal_propositional_natural_deduction_system/or/elim}, we replace the conditional \( \rightarrow \) with labeled assumptions that correspond to \( \varphi \vdash \theta \) and \( \psi \vdash \theta \).

  We can prove that \eqref{eq:def:minimal_propositional_natural_deduction_system/or/elim} implies \eqref{eq:def:minimal_propositional_axiomatic_deductive_system/or/elim} analogously to \eqref{eq:def:minimal_propositional_natural_deduction_system/and_intro_axiom_derivation}.

  The other two rules are again trivial to obtain from the corresponding axioms and vice versa.

  \SubProofOf{def:minimal_propositional_natural_deduction_system/iff} Analogous to what we have already shown.

  \SubProofOf{def:minimal_propositional_natural_deduction_system/negation} \eqref{eq:def:minimal_propositional_natural_deduction_system/neg/intro} is obtained from \eqref{eq:def:minimal_propositional_axiomatic_deductive_system/neg/intro} by applying \eqref{eq:def:def:axiomatic_deductive_system/mp} once and \eqref{eq:def:minimal_propositional_natural_deduction_system/neg/elim} is obtained from \eqref{eq:def:minimal_propositional_axiomatic_deductive_system/neg/intro} by applying \eqref{eq:def:def:axiomatic_deductive_system/mp} twice. Using the rules to derive the axioms is similar to \eqref{eq:def:minimal_propositional_natural_deduction_system/and_intro_axiom_derivation}.
\end{defproof}

\begin{proposition}\label{thm:conjunction_of_premises}
  In deductive systems that extend the \hyperref[def:minimal_propositional_natural_deduction_system]{minimal propositional natural deduction system}, we have \( \psi_1, \psi_1 \vdash \varphi \) if and only if \( (\psi_1 \wedge \psi_2) \vdash \varphi \).
\end{proposition}
\begin{proof}
  \SufficiencySubProof If \( \psi_1, \psi_2 \vdash \varphi \), then
  \begin{equation*}
    \begin{prooftree}
      \hypo{ \psi_1 \wedge \psi_2 }
      \infer1[\eqref{eq:def:minimal_propositional_natural_deduction_system/and/elim_right}]{ \psi_1 }

      \hypo{ \psi_1 \wedge \psi_2 }
      \infer1[\eqref{eq:def:minimal_propositional_natural_deduction_system/and/elim_right}]{ \psi_2 }

      \infer2{}

      \ellipsis{}{ \varphi }
    \end{prooftree}
  \end{equation*}

  \NecessitySubProof If \( (\psi_1 \wedge \psi_2) \vdash \varphi \), then
  \begin{equation*}
    \begin{prooftree}
      \hypo{ \psi_1 }
      \hypo{ \psi_2 }
      \infer2[\eqref{eq:def:minimal_propositional_natural_deduction_system/and/intro}]{ \psi_1 \wedge \psi_2 }
      \ellipsis{}{ \varphi }
    \end{prooftree}
  \end{equation*}
\end{proof}

\begin{theorem}\label{thm:minimal_propositional_negation_laws}
  Consider the following propositional formula schemas:
  \begin{thmenum}
    \thmitem{thm:minimal_propositional_negation_laws/dne} Double negation elimination:
    \begin{equation}\label{eq:thm:minimal_propositional_negation_laws/dne}
      \neg \neg \varphi \rightarrow \varphi \tag{A DNE}.
    \end{equation}

    The semantic counterpart to this law is \fullref{thm:boolean_equivalences/double_negation}.

    \thmitem{thm:minimal_propositional_negation_laws/efq} \term[en=from falsity everything follows]{Ex falso quodlibet}, also known as the \term{principle of explosion}:
    \begin{equation}\label{eq:thm:minimal_propositional_negation_laws/efq}
      \bot \rightarrow \varphi \tag{A EFQ}
    \end{equation}

    \thmitem{thm:minimal_propositional_negation_laws/pierce} \term{Pierce's law}:
    \begin{equation}\label{eq:thm:minimal_propositional_negation_laws/pierce}
      ((\varphi \rightarrow \psi) \rightarrow \varphi) \rightarrow \varphi \tag{A Pierce}
    \end{equation}

    \thmitem{thm:minimal_propositional_negation_laws/lem} The \term{law of the excluded middle}:
    \begin{equation}\label{eq:thm:minimal_propositional_negation_laws/lem}
      \varphi \vee \neg \varphi \tag{A LEM}
    \end{equation}

    \thmitem{thm:minimal_propositional_negation_laws/lnc} The \term{law of non-contradiction}:
    \begin{equation}\label{eq:thm:minimal_propositional_negation_laws/lnc}
      \neg (\varphi \wedge \neg \varphi). \tag{A LNC}
    \end{equation}
  \end{thmenum}

  Assuming the \hyperref[def:minimal_propositional_axiomatic_deductive_system]{minimal propositional axiomatic deductive system}, we have the following derivations:
  \begin{center}
    \begin{forest}
      [
        {\eqref{eq:thm:minimal_propositional_negation_laws/dne}}
          [
            {\eqref{eq:thm:minimal_propositional_negation_laws/pierce}}
              [{\eqref{eq:thm:minimal_propositional_negation_laws/lem}}]
          ]
          [
            {\eqref{eq:thm:minimal_propositional_negation_laws/efq}}
              [{\eqref{eq:thm:minimal_propositional_negation_laws/lnc}}]
          ]
      ]
    \end{forest}
  \end{center}

  As it turns out, \eqref{eq:thm:minimal_propositional_negation_laws/lnc}, which is often associated with intuitionistic logic, is a theorem of \hyperref[def:minimal_logic]{minimal logic}.

  Conversely, \eqref{eq:thm:minimal_propositional_negation_laws/efq} and \eqref{eq:thm:minimal_propositional_negation_laws/lem} together can be used to derive \eqref{eq:thm:minimal_propositional_negation_laws/dne}.
\end{theorem}
\begin{proof}
  Most proofs are given in \cite[prop. 3]{DienerMcKubreJordens2016} and \cite[prop. 13]{DienerMcKubreJordens2016}. We will only show that \eqref{eq:thm:minimal_propositional_negation_laws/lnc} is strictly weaker than \eqref{eq:thm:minimal_propositional_negation_laws/efq}.

  For any formula \( \varphi \), we have the \hyperref[def:minimal_propositional_natural_deduction_system]{natural deduction} proof that \( \eqref{eq:thm:minimal_propositional_negation_laws/lnc} \) is a tautology:
  \begin{equation*}
    \begin{prooftree}[separation=3em]
      \hypo{ [\varphi \wedge \neg \varphi]^1 }
      \infer1[\ref{eq:def:minimal_propositional_natural_deduction_system/and/elim_left}]{ \varphi }

      \hypo{ [\varphi \wedge \neg \varphi]^1 }
      \infer1[\ref{eq:def:minimal_propositional_natural_deduction_system/and/elim_right}]{ \neg \varphi }

      \infer2[\ref{eq:def:minimal_propositional_natural_deduction_system/neg/elim}]{ \bot }

      \infer[left label=\( 1 \)]1[\ref{eq:def:minimal_propositional_natural_deduction_system/neg/intro}]{ \neg (\varphi \wedge \neg \varphi) }
    \end{prooftree}
  \end{equation*}

  Hence, \eqref{eq:thm:minimal_propositional_negation_laws/lnc} is a theorem of \hyperref[def:minimal_logic]{minimal logic}. If it were to imply \eqref{eq:thm:minimal_propositional_negation_laws/efq}, then minimal and intuitionistic logic would be equivalent, which would contradict \cite[prop. 3]{DienerMcKubreJordens2016}. Therefore, \eqref{eq:thm:minimal_propositional_negation_laws/lnc} is indeed strictly weaker than \eqref{eq:thm:minimal_propositional_negation_laws/efq}.
\end{proof}

\begin{proposition}\label{thm:syntactic_contraposition}
  In the \hyperref[def:minimal_propositional_natural_deduction_system]{minimal propositional natural deduction system}, we have
  \begin{align}
    \varphi \rightarrow \psi &\vdash \neg \psi \rightarrow \neg \varphi \label{eq:thm:syntactic_contraposition/straight} \\
    \eqref{eq:thm:minimal_propositional_negation_laws/dne}, \neg \varphi \rightarrow \neg \psi &\vdash \psi \rightarrow \varphi \label{eq:thm:syntactic_contraposition/reverse}
  \end{align}
\end{proposition}
\begin{proof}
  We will only prove \eqref{eq:thm:syntactic_contraposition/straight}. The derivability \eqref{eq:thm:syntactic_contraposition/reverse} can be proved in the same way except that we would use \eqref{eq:def:minimal_propositional_natural_deduction_system/neg/intro} rather than \eqref{eq:def:classical_propositional_deductive_systems/rules/dne}.

  \begin{equation*}
    \begin{prooftree}
      \hypo{ \varphi \rightarrow \psi }
      \hypo{ [\varphi]^1 }
      \infer2[\eqref{eq:def:minimal_propositional_natural_deduction_system/imp/elim}]{ \psi }

      \hypo{ [\neg \psi]^2 }
      \infer2[\eqref{eq:def:minimal_propositional_natural_deduction_system/neg/elim}]{ \bot }

      \infer[left label=\( 1 \)]1[\eqref{eq:def:minimal_propositional_natural_deduction_system/neg/intro}]{ \neg \varphi }
      \infer[left label=\( 2 \)]1[\eqref{eq:def:minimal_propositional_natural_deduction_system/imp/intro}]{ \neg \psi \rightarrow \neg \varphi }
    \end{prooftree}
  \end{equation*}
\end{proof}

\begin{definition}\label{def:intuitionistic_propositional_deductive_systems}\mcite[sec. 1.4 \\ 158]{VanDalen2004}
  The \term{intuitionistic propositional natural deduction system} extends the \hyperref[def:minimal_propositional_natural_deduction_system]{minimal propositional natural deduction system} with the rule
  \begin{equation*}\taglabel[\textrm{EFQ}]{eq:def:intuitionistic_propositional_deductive_systems/rules/efq}
    \begin{prooftree}
      \hypo{ \bot }
      \infer1[\ref{eq:def:intuitionistic_propositional_deductive_systems/rules/efq}]{ \varphi }
    \end{prooftree}
  \end{equation*}

  This corresponds to the axiom \eqref{eq:thm:minimal_propositional_negation_laws/efq}, which we can add to the \hyperref[def:minimal_propositional_axiomatic_deductive_system]{minimal propositional axiomatic deductive system}.

  The corresponding semantics are defined in \fullref{def:propositional_heyting_algebra_semantics} and their link with the deductive system is given in \fullref{thm:intuitionistic_propositional_logic_is_sound_and_complete}.
\end{definition}

\begin{definition}\label{def:propositional_heyting_algebra_semantics}\mcite[def. 7.1]{Wasilewska2018}
  We define \term{Heyting semantics} for propositional formulas similarly to how it is done with classical Boolean semantics in \fullref{def:propositional_semantics}, except that instead of using a \hyperref[def:boolean_algebra]{Boolean algebra} we use a more general \hyperref[def:heyting_algebra]{Heyting algebra}.

  Logical negations depend on complements in Boolean algebras. Since Heyting algebras do not have complements, we instead use \hyperref[def:heyting_algebra/pseudocomplement]{pseudocomplements}.

  Fix a Heyting algebra \( \mscrH = (H, \sup, \inf, T, F, \rightarrow) \). \hyperref[def:propositional_valuation/interpretation]{Propositional interpretations} in Heyting semantics may take any value in \( X \), as can \hyperref[def:propositional_valuation/formula_valuation]{formula valuations}.

  Given an interpretation \( I \) and a formula \( \varphi \), we define \( \varphi\Bracks{I} \) via \eqref{eq:def:propositional_valuation/formula_interpretation}, the sole difference being that negation valuation is defined via the pseudocomplement:
  \begin{equation*}
    (\neg \psi)\Bracks{I} \coloneqq \widetilde{\varphi\Bracks{I}}.
  \end{equation*}

  We say that \( I \) satisfies \( \varphi \) if \( \varphi\Bracks{I} = T \). Thus, if the valuation of \( \varphi \) takes any value in \( H \setminus \set{ T } \), then \( I \) does not satisfy \( \varphi \), but that does not necessarily mean that \( I \) satisfies \( \neg \varphi \).

  Then \( \Gamma \) entails \( \varphi \) if, for every \( \psi \in \Gamma \) and every interpretation \( I \) in every Heyting algebra, we have \( \varphi\Bracks{I} = \psi\Bracks{I} \).

  It is important that different Heyting algebras may provide different semantics --- see \fullref{ex:heyting_semantics_lem_counterexample} for an example of what is impossible in a Boolean algebra.
\end{definition}

\begin{example}\label{ex:heyting_semantics_lem_counterexample}
  Let \( \mscrX \) be an extension of the trivial Boolean algebra \( \set{ T, F } \) with the \enquote{indeterminate} symbol \( N \). That is, the domain of \( \mscrX \) is \( \set{ F, N, T } \) and the order is \( F \leq N \leq T \).

  The pseudocomplement of \( N \) is
  \begin{equation*}
    \widetilde{N}
    \reloset {\eqref{eq:def:heyting_algebra/pseudocomplement}} =
    \sup\set{ a \in X \given a \wedge N = \bot }
    =
    F.
  \end{equation*}

  Consider any \hyperref[def:propositional_valuation]{propositional interpretation} \( I \) such that \( I(P) = N \).

  Then the valuation of \eqref{eq:thm:minimal_propositional_negation_laws/lem} is
  \begin{equation*}
    (P \vee \neg P)\Bracks{I}
    =
    \sup\set{ P\Bracks{I}, \widetilde{P\Bracks{I}} }
    =
    \sup\set{ N, \widetilde{N} }
    =
    \sup\set{ N, F }
    =
    N.
  \end{equation*}

  Therefore, \eqref{eq:thm:minimal_propositional_negation_laws/lem} does not hold.
\end{example}

\begin{definition}\label{def:propositional_topological_semantics}\mcite[]{Wasilewska2018}
  Since arbitrary \hyperref[def:heyting_algebra]{Heyting algebras} can be cumbersome to come up with when used for \hyperref[def:propositional_heyting_algebra_semantics]{propositional Heyting semantics}, we can instead utilize \fullref{ex:topological_space_is_heyting_algebra} and define \term{topological semantics} for some nonempty \hyperref[def:topological_space]{topological space}.

  The truth values of interpretations and valuations are then open sets in some topological space and a formula is said to be valid if its valuation is the whole space.
\end{definition}

\begin{theorem}[Intuitionistic propositional logic is sound and complete]\label{thm:intuitionistic_propositional_logic_is_sound_and_complete}\mcite[thm. 7.3]{Wasilewska2018}
  The \hyperref[def:intuitionistic_propositional_deductive_systems]{intuitionistic propositional deductive system} is \hyperref[def:derivability_and_satisfiability/soundness]{sound} and \hyperref[def:derivability_and_satisfiability/completeness]{complete} with respect to both \hyperref[def:propositional_heyting_algebra_semantics]{Heyting semantics} and \hyperref[def:propositional_topological_semantics]{topological semantics}. To elaborate,
  \begin{thmenum}
    \thmitem{thm:intuitionistic_propositional_logic_is_sound_and_complete/sound} If \( \vdash \varphi \), then \( \vDash \varphi \) for every Heyting algebra (resp. topological space).
    \thmitem{thm:intuitionistic_propositional_logic_is_sound_and_complete/complete} If \( \vDash \varphi \) in every Heyting algebra (resp. topological space), then \( \vdash \varphi \).
  \end{thmenum}
\end{theorem}

\begin{example}\label{ex:topological_semantics_lem_counterexample}
  Let \( U \) be an open set in the standard topology in \( \BbbR \). We will examine \eqref{eq:thm:minimal_propositional_negation_laws/lem} with respect to \hyperref[def:propositional_topological_semantics]{topological semantics} for \( \BbbR \). Due to \fullref{ex:topological_space_is_heyting_algebra}, given any \hyperref[def:propositional_valuation]{propositional interpretation} \( I \) such that \( I(P) = U \), we have
  \begin{equation*}
    (P \vee \neg P)\Bracks{I}
    =
    P\Bracks{I} \cup \widetilde{P\Bracks{I}}
    =
    U \cup \widetilde{U}
    =
    U \cup \Int(\BbbR \setminus U).
  \end{equation*}

  If \( U = \varnothing \), then \( (P \vee \neg P)\Bracks{I} = \BbbR \) and \eqref{eq:thm:minimal_propositional_negation_laws/lem} holds. If \( U = (0, 1) \), then \( (P \vee \neg P)\Bracks{I} = \BbbR \setminus \set{ 0, 1 } \) and \eqref{eq:thm:minimal_propositional_negation_laws/lem} does not hold.

  Compare this result with \fullref{ex:heyting_semantics_lem_counterexample}.
\end{example}

\begin{remark}\label{rem:brouwer_heyting_kolmogorov_interpretation}\mcite[156]{VanDalen2004}
  Another semantical framework for the \hyperref[def:intuitionistic_propositional_deductive_systems]{intuitionistic propositional deductive system} is the \term{Brouwer-Heyting-Kolmogorov interpretation}.

  It uses a less formal approach than \hyperref[def:propositional_heyting_algebra_semantics]{Heyting algebra semantics} that is based on the notion of a \enquote{construction}, which is also why it is sometimes called \term{constructive logic}.

  \begin{thmenum}
    \thmitem{rem:brouwer_heyting_kolmogorov_interpretation/atomic} We assume that we know what constitutes a construction of propositional variables.
    \thmitem{rem:brouwer_heyting_kolmogorov_interpretation/constant} There is no construction of \( \bot \) and no construction of \( \top \) is needed.
    \thmitem{rem:brouwer_heyting_kolmogorov_interpretation/disjunction} A construction of \( \psi_1 \vee \psi_2 \) is a pair \( (k, M) \), where \( k = 1, 2 \) and \( M \) is a construction of \( \psi_m \) if and only if \( k = m \). The notion of a pair here is informal.
    \thmitem{rem:brouwer_heyting_kolmogorov_interpretation/conjunction} A construction of \( \psi_1 \wedge \psi_2 \) is a pair \( (M_1, M_2) \), where \( M_k \) is a construction of \( \psi_k \) for \( k = 1, 2 \).
    \thmitem{rem:brouwer_heyting_kolmogorov_interpretation/conditional} A construction of \( \psi_1 \rightarrow \psi_2 \) is a function that converts a construction of \( \psi_1 \) into a construction of \( \psi_2 \). The notion of a function here is informal.
  \end{thmenum}

  The negation \( \neg\psi \) that corresponds to pseudocomplements in Heyting algebra semantics corresponds to the metastatement \enquote{a construction of \( \psi \) is impossible} under the Heyting-Brouwer-Kolmogorov interpretation.

  If the set \( \Gamma \) of formulas does not derive \( \varphi \), we say that \( \varphi \) is non-constructive under the axioms \( \Gamma \).
\end{remark}

\begin{remark}\label{rem:brouwer_heyting_kolmogorov_interpretation_compatibility}
  Since the \hyperref[rem:brouwer_heyting_kolmogorov_interpretation]{Heyting-Brouwer-Kolmogorov interpretation} is not very formal, we cannot properly prove its, soundness or completeness with respect to the \hyperref[def:intuitionistic_propositional_deductive_systems]{intuitionistic propositional deductive system}.

  Nevertheless, we generally accept the interpretation and conflate \enquote{constructive} and \enquote{intuitionistic} statements.
\end{remark}

\begin{example}\label{ex:rem:brouwer_heyting_kolmogorov_interpretation/well_ordering_principle_zfc}
  \Fullref{thm:well_ordering_theorem} in \hyperref[def:set]{\logic{ZFC}} does not provide a way to well-order an arbitrary set. The theorem relies on the axiom of choice, whose consequence \fullref{thm:diaconescu_goodman_myhill_theorem} implies the law of the excluded middle (LEM) assuming the nonlogical axioms of \logic{ZFC}.

  Since LEM may not hold in intuitionistic logic, it follows that both \fullref{thm:well_ordering_theorem} and the axiom of choice itself should not in general hold under the Heyting-Brouwer-Kolmogorov interpretation, hence by the terminology in \fullref{rem:brouwer_heyting_kolmogorov_interpretation}, \fullref{thm:well_ordering_theorem} is a non-constructive theorem.
\end{example}

\begin{definition}\label{def:classical_propositional_deductive_systems}
  In order to obtain a deductive system that matches \hyperref[def:propositional_semantics]{classical propositional semantics}, we may extend the \hyperref[def:minimal_propositional_natural_deduction_system]{minimal propositional natural deduction system} with the rule
  \begin{equation*}\taglabel[\textrm{DNE}]{eq:def:classical_propositional_deductive_systems/rules/dne}
    \begin{prooftree}
      \hypo{ [\neg \varphi]^n }
      \ellipsis {} { \bot }
      \infer[left label=\( n \)]1[\ref{eq:def:classical_propositional_deductive_systems/rules/dne}]{ \varphi }
    \end{prooftree}
  \end{equation*}

  This corresponds to the axiom \eqref{eq:thm:minimal_propositional_negation_laws/dne}, which we can add to the \hyperref[def:minimal_propositional_axiomatic_deductive_system]{minimal propositional axiomatic deductive system}. As per \fullref{thm:minimal_propositional_negation_laws}, we can instead add \eqref{eq:thm:minimal_propositional_negation_laws/lem} to the \hyperref[def:intuitionistic_propositional_deductive_systems]{intuitionistic propositional axiomatic deductive system}, since
  \begin{equation*}
    \eqref{eq:thm:minimal_propositional_negation_laws/lem}, \eqref{eq:thm:minimal_propositional_negation_laws/efq} \vdash \eqref{eq:thm:minimal_propositional_negation_laws/dne}
  \end{equation*}

  We call this, very simply, the (classical) \term{propositional deductive system}.
\end{definition}

\begin{theorem}[Glivenko's double negation theorem]\label{thm:glivenkos_double_negation_theorem}\mcite[thm. 7.7]{Wasilewska2018}
  A formula \( \varphi \) is derivable in the \hyperref[def:classical_propositional_deductive_systems]{classical propositional natural deduction system} if and only if it's double negation \( \neg \neg \varphi \) is derivable in the \hyperref[def:intuitionistic_propositional_deductive_systems]{intuitionistic propositional natural deduction system}.
\end{theorem}

\begin{theorem}\label{thm:classical_propositional_logic_is_sound_and_complete}\mcite[thm. 1.5.13]{VanDalen2004}
  The \hyperref[def:classical_propositional_deductive_systems]{classical propositional natural deduction system} is \hyperref[def:derivability_and_satisfiability/soundness]{sound} and \hyperref[def:derivability_and_satisfiability/completeness]{complete} with respect to \hyperref[def:propositional_semantics]{classical semantics}.
\end{theorem}

\medskip

\begin{definition}\label{def:first_order_natural_deduction_system}\mcite[sec. 2.8]{VanDalen2004}
  If we wish to work with first-order logic rather than merely propositional logic, we must extend the \hyperref[def:classical_propositional_deductive_systems]{classical propositional natural deduction system}. We call this, very simply, the (classical) \term{first-order natural deduction system}.

  \begin{thmenum}
    \thmitem{def:first_order_natural_deduction_system/eigenvariables} We first add the following two \hyperref[def:judgment/inference_rule]{inference rules} for quantification:

    \begin{minipage}{0.45\textwidth}
      \begin{equation*}\taglabel[\( \forall^+ \)]{eq:def:first_order_natural_deduction_system/forall/intro}
        \begin{prooftree}
          \hypo{ \varphi }
          \infer1[\ref{eq:def:first_order_natural_deduction_system/forall/intro}]{ \qforall \xi \varphi }
        \end{prooftree}
      \end{equation*}
    \end{minipage}
    \hfill
    \begin{minipage}{0.45\textwidth}
      \begin{equation*}\taglabel[\( \exists^- \)]{eq:def:first_order_natural_deduction_system/exists/elim}
        \begin{prooftree}
          \hypo{ \qexists \xi \varphi }
          \hypo{ [\varphi]^n }
          \ellipsis {} { \psi }
          \infer[left label=\( n \)]2[\ref{eq:def:first_order_natural_deduction_system/exists/elim}]{ \psi }
        \end{prooftree}
      \end{equation*}
    \end{minipage}

    Here \( \xi \) is a variable that is not free in any undischarged assumption in the proof of \( \varphi \) (it may be free in \( \varphi \) as long as \( \varphi \) is not itself an undischarged assumption). A variable \( \xi \) satisfying these conditions is called an \term{eigenvariable} of the rule.

    These rules are the primary motivation for inference rules accepting proof trees rather than only formulas --- see \fullref{def:judgment/inference_rule} and \fullref{def:deductive_system/rule}. See \fullref{ex:def:first_order_natural_deduction_system/eigenvariables/invalid_universal} for why this condition is important.

    \thmitem{def:first_order_natural_deduction_system/terms} We add two \hyperref[def:judgment/inference_rule]{inference rules}, where \( \tau \) is an arbitrary term:

    \begin{minipage}{0.45\textwidth}
      \begin{equation*}\taglabel[\( \forall^- \)]{eq:def:first_order_natural_deduction_system/forall/elim}
        \begin{prooftree}
          \hypo{ \qforall \xi \varphi }
          \infer1[\ref{eq:def:first_order_natural_deduction_system/forall/elim}]{ \varphi[\xi \mapsto \tau] }
        \end{prooftree}
      \end{equation*}
    \end{minipage}
    \hfill
    \begin{minipage}{0.45\textwidth}
      \begin{equation*}\taglabel[\( \exists^+ \)]{eq:def:first_order_natural_deduction_system/exists/intro}
        \begin{prooftree}
          \hypo{ \varphi[\xi \mapsto \tau] }
          \infer1[\ref{eq:def:first_order_natural_deduction_system/exists/intro}]{ \qexists \xi \varphi }
        \end{prooftree}
      \end{equation*}
    \end{minipage}

    Compare this to \fullref{thm:quantifier_satisfiability}.

    \thmitem{def:first_order_natural_deduction_system/equality} Finally, we also add three rules for formal equality:

    \begin{minipage}{0.3\textwidth}
      \begin{equation*}\taglabel[\( \doteq^+ \)]{eq:def:first_order_natural_deduction_system/equality/intro}
        \begin{prooftree}
          \infer0[\ref{eq:def:first_order_natural_deduction_system/equality/intro}]{ \tau \doteq \tau }
        \end{prooftree}
      \end{equation*}
    \end{minipage}
    \hfill
    \begin{minipage}{0.3\textwidth}
      \begin{equation*}\taglabel[\( \doteq_L^- \)]{eq:def:first_order_natural_deduction_system/equality/elim_left}
        \begin{prooftree}
          \hypo{ \tau \doteq \sigma }
          \hypo{ \varphi[\xi \mapsto \tau] }
          \infer2[\ref{eq:def:first_order_natural_deduction_system/equality/elim_left}]{ \varphi[\xi \mapsto \sigma] }
        \end{prooftree}
      \end{equation*}
    \end{minipage}
    \hfill
    \begin{minipage}{0.3\textwidth}
      \begin{equation*}\taglabel[\( \doteq_L^+ \)]{eq:def:first_order_natural_deduction_system/equality/elim_right}
        \begin{prooftree}
          \hypo{ \tau \doteq \sigma }
          \hypo{ \varphi[\xi \mapsto \sigma] }
          \infer2[\ref{eq:def:first_order_natural_deduction_system/equality/elim_right}]{ \varphi[\xi \mapsto \tau] }
        \end{prooftree}
      \end{equation*}
    \end{minipage}
  \end{thmenum}
\end{definition}

\begin{example}\label{ex:def:first_order_natural_deduction_system/eigenvariables}
  \hfill
  \begin{thmenum}
    \thmitem{ex:def:first_order_natural_deduction_system/eigenvariables/invalid_universal_closure} We explicitly forbid the syntactic equivalent of \fullref{thm:implicit_universal_quantification} in order to avoid invalid proofs like \fullref{ex:def:first_order_natural_deduction_system/eigenvariables/invalid_universal}. Consider the proof
    \begin{equation*}
      \begin{prooftree}
        \hypo{ [\varphi]^1 }
        \infer1[\ref{eq:def:first_order_natural_deduction_system/forall/intro}]{ \qforall \xi \varphi }
      \end{prooftree}
    \end{equation*}

    The problem here is that \( \varphi \) is itself an undischarged assumption, hence \eqref{eq:def:first_order_natural_deduction_system/forall/intro} is actually inapplicable here, and the proof is invalid.

    \thmitem{ex:def:first_order_natural_deduction_system/eigenvariables/invalid_universal} To see why the eigenvariable conditions in \fullref{def:first_order_natural_deduction_system/eigenvariables} are essential, consider the following proof of \( \qforall \xi \varphi \) from \( \qexists \xi \varphi \):
    \begin{equation*}
      \begin{prooftree}
        \hypo{ \qexists \xi \varphi }

        \hypo{ [\varphi]^1 }
        \infer1[\ref{eq:def:first_order_natural_deduction_system/forall/intro}]{ \qforall \xi \varphi }

        \infer[left label=\( 1 \)]2[\ref{eq:def:first_order_natural_deduction_system/exists/elim}]{ \qforall \xi \varphi }
      \end{prooftree}
    \end{equation*}

    This proof relies on \fullref{ex:def:first_order_natural_deduction_system/eigenvariables/invalid_universal_closure}, which we have already demonstrated to be invalid.

    \thmitem{ex:def:first_order_natural_deduction_system/eigenvariables/invalid_existence} Another invalid proof, in case \( \xi \in \boldop{Free}(\varphi) \), is
    \begin{equation*}
      \begin{prooftree}
        \hypo{ \qexists \xi \varphi }

        \hypo{ [\varphi]^1 }
        \infer1{ \varphi }

        \infer[left label=\( 1 \)]2[\ref{eq:def:first_order_natural_deduction_system/exists/elim}]{ \varphi }
      \end{prooftree}
    \end{equation*}

    \thmitem{ex:def:first_order_natural_deduction_system/eigenvariables/universal_implies_existence} On the other hand, \( \qexists \xi \varphi \) can easily be derived from \( \qforall \xi \varphi \):
    \begin{equation*}
      \begin{prooftree}
        \hypo{ \qforall \xi \varphi }
        \infer1[\ref{eq:def:first_order_natural_deduction_system/forall/elim}]{ \varphi = \varphi[\xi \mapsto \xi] }
        \infer1[\ref{eq:def:first_order_natural_deduction_system/exists/intro}]{ \qexists \xi \varphi }
      \end{prooftree}
    \end{equation*}

    \thmitem{ex:def:first_order_natural_deduction_system/eigenvariables/universal_implies_universal} It is also valid to perform the completely meaningless derivation:
    \begin{equation*}
      \begin{prooftree}
        \hypo{ \qforall \xi \varphi }
        \infer1[\ref{eq:def:first_order_natural_deduction_system/forall/elim}]{ \varphi = \varphi[\xi \mapsto \xi] }
        \infer1[\ref{eq:def:first_order_natural_deduction_system/forall/intro}]{ \qforall \xi \varphi }
      \end{prooftree}
    \end{equation*}
  \end{thmenum}
\end{example}

\begin{proposition}\label{thm:syntactic_first_order_quantifiers_are_dual}
  For any formula \( \varphi \) and any variable \( \xi \) over \( \mscrL \), we have the following interderivable pairs:
  \begin{align}
    \neg \qforall \xi \varphi &\T{and} \qexists \xi \neg \varphi \label{thm:syntactic_first_order_quantifiers_are_dual/negation_of_universal} \\
    \neg \qexists \xi \varphi &\T{and} \qforall \xi \neg \varphi \label{thm:syntactic_first_order_quantifiers_are_dual/negation_of_existential}
  \end{align}
\end{proposition}
\begin{proof}
  We will only show \eqref{thm:syntactic_first_order_quantifiers_are_dual/negation_of_universal}. First,

  \begin{equation*}
    \begin{prooftree}
      \hypo{ \neg \qforall \xi \varphi }
      \hypo{ [\qforall \xi \varphi]^1 }
      \infer2[\eqref{eq:def:minimal_propositional_natural_deduction_system/neg/elim}]{ \bot }
      \infer[left label=\( 1 \)]1[\eqref{eq:def:classical_propositional_deductive_systems/rules/dne}]{ \qforall \xi \varphi }
      \infer1[\eqref{eq:def:first_order_natural_deduction_system/forall/elim}]{ \varphi }

      \hypo{ [\neg \varphi]^2 }
      \infer2[\eqref{eq:def:minimal_propositional_natural_deduction_system/neg/elim}]{ \bot }

      \infer[left label=\( 2 \)]1[\eqref{eq:def:minimal_propositional_natural_deduction_system/neg/intro}]{ \neg \varphi }
      \infer1[\eqref{eq:def:first_order_natural_deduction_system/exists/intro}]{ \qexists \xi \neg \varphi }
    \end{prooftree}
  \end{equation*}

  Conversely,
  \begin{equation*}
    \begin{prooftree}
      \hypo{ \qexists \xi \neg \varphi }

      \hypo{ [\qforall \xi \varphi]^1 }
      \infer1[\eqref{eq:def:first_order_natural_deduction_system/forall/elim}]{ \varphi }

      \hypo{ [\neg \varphi]^2 }
      \infer2[\eqref{eq:def:minimal_propositional_natural_deduction_system/neg/elim}]{ \bot }
      \infer[left label=\( 1 \)]1[\eqref{eq:def:minimal_propositional_natural_deduction_system/neg/intro}]{ \neg \qforall \xi \varphi }

      \infer[left label=\( 2 \)]2[\eqref{eq:def:first_order_natural_deduction_system/exists/elim}]{ \neg \qforall \xi \varphi }
    \end{prooftree}
  \end{equation*}
\end{proof}

\begin{theorem}\label{thm:classical_first_order_logic_is_sound_and_complete}\mcite[thm. 3.1.13]{VanDalen2004}
  The \hyperref[def:first_order_natural_deduction_system]{classical first-order natural deduction system} is \hyperref[def:derivability_and_satisfiability/soundness]{sound} and \hyperref[def:derivability_and_satisfiability/completeness]{complete} with respect to \hyperref[def:first_order_semantics]{classical semantics}.

  The completeness part is known as \enquote{G\"odel's completeness theorem} and requires an elaborate proof.
\end{theorem}

  \subsection{Logical theories}\label{subsec:logical_theories}

As in \fullref{subsec:first_order_models}, we will only be interested in (sets of) closed first-order formulas.

\paragraph{First-order theories}

\begin{definition}\label{def:first_order_theory}\mcite[def. 1.4.5]{Hinman2005}
  The \term{closure} of the set \( \Gamma \) of closed formulas in the \hyperref[def:first_order_language]{first-order language} \( \mscrL \) is the set
  \begin{equation*}
    \cl(\Gamma) \coloneqq \set{ \varphi \in \boldop{Form} \given \Gamma \vDash \varphi }.
  \end{equation*}

  If necessary, we may distinguish between the syntactic closure \( \cl^\vdash(\Gamma) \) and the semantic closure \( \cl^\vDash(\Gamma) \) of \( \Gamma \).

  \begin{thmenum}
    \thmitem{def:first_order_theory/axiomatized}\mcite[def. 1.4.37]{Hinman2005} We say that \( \Gamma \) is \term{axiomatized} by \( \Delta \) if \( \Gamma = \cl(\Delta) \). If \( \Delta \) is clear from the context, we call its members (nonlogical) \term{axioms}.

    \medskip

    \thmitem{def:first_order_theory/complete}\mcite[def. 1.4.8]{Hinman2005} The set \( \Gamma \) of closed formulas is said to be \term{complete} if every for every formula in \( \varphi \), either \( \Gamma \vDash \varphi \) or \( \Gamma \vDash \neg \varphi \).

    \thmitem{def:first_order_theory/consistent} Given a deductive system, the set \( \Gamma \) of formulas is said to be \term[bg=противоречива,ru=противоречивая]{inconsistent} if \( \Gamma \vdash \bot \) and \term{consistent} otherwise.
  \end{thmenum}
\end{definition}
\begin{comments}
  \item Closedness of a set of closed formulas should not be confused with \fullref{def:derivability_and_satisfiability/completeness}, which defines completeness of a deductive system via how it relates to semantics.

  \item Peter Hinman in \incite[def. 1.4.5]{Hinman2005} calls \( \cl(\Gamma) \) the \enquote{theory generated by \( \Gamma \)}, but we avoid the term \enquote{theory} altogether as much as possible.

  \item This definition straightforwardly adapts to propositional formulas.
\end{comments}

\begin{proposition}\label{thm:formulas_unsatisfiable_iff_inconsistent}
   Assuming \hyperref[rem:classical_logic]{classical logic}, a theory \( \Gamma \) is \hyperref[def:propositional_model]{unsatisfiable} if and only if it is semantically \hyperref[def:first_order_theory/consistent]{inconsistent}.
\end{proposition}
\begin{proof}
  \SufficiencySubProof Assume first that \( \Gamma \) is unsatisfiable. Then for all zero models of \( \Gamma \) any formula is satisfied vacuously, in particular that any model of \( \Gamma \) satisfies \( \bot \). Thus, \( \Gamma \vDash \bot \) and, by \fullref{thm:classical_first_order_logic_is_sound_and_complete}, \( \Gamma \vdash \bot \). Therefore, \( \Gamma \) is inconsistent.

  \NecessitySubProof Let \( \Gamma \) be inconsistent and suppose that \( \mscrX = (X, I) \) is a model of \( \Gamma \). Fix any valuation \( v \) in \( \mscrX \). Since \( \Gamma \vdash \bot \), \fullref{thm:classical_first_order_logic_is_sound_and_complete} implies that \( \Bracks{\bot}_v = T \).

  By the \hyperref[def:first_order_valuation/formula_valuation]{definition of formula valuation}, however, we have \( \Bracks{\bot}_v = F \). The obtained contradiction shows that \( \mscrX \) cannot be a model of \( \Gamma \) and since this structure was chosen arbitrarily, we conclude that \( \Gamma \) is unsatisfiable.
\end{proof}

\paragraph{The L\"owenheim-Skolem theorems}

\begin{theorem}[First-order compactness theorem]\label{thm:first_order_compactness_theorem}\mcite[thm. 3.1.2]{Hinman2005}
  If every finite subset of the set \( \Gamma \) of closed formulas is satisfiable, then \( \Gamma \) itself is satisfiable.
\end{theorem}
\begin{comments}
  \item The theorem also applies, with adaptations based on \fullref{rem:propositional_logic_as_first_order_logic}, to propositional formulas.
\end{comments}

\begin{theorem}[Downward L\"owenheim-Skolem theorem]\label{thm:downward_lowenheim_skolem_theorem}\mcite[thm. 2.3.35]{Hinman2005}
  Let \( \Gamma \) be a set of closed first-order formulas over some arbitrary language. Suppose that \( \Gamma \) has a \hyperref[def:first_order_model]{model} of \hyperref[def:set_finiteness]{infinite} \hyperref[thm:cardinality_existence]{cardinality}.

  Then \( \Gamma \) also has a \hyperref[def:set_countability]{countable} model.
\end{theorem}

\begin{example}[Skolem's paradox]\label{ex:skolems_paradox}
  From \fullref{thm:downward_lowenheim_skolem_theorem} it follows that \hyperref[def:zfc]{\logic{ZFC}}, if it is consistent, has a model at is at most countably infinite. The \hyperref[def:zfc/infinity]{axiom of infinity} states that no model of \logic{ZFC} is finite. Therefore, there exists a model of \logic{ZFC} that is countably infinite. But we can construct uncountable sets in \logic{ZFC}.

  Therefore, either:
  \begin{itemize}
    \item Uncountable sets within the metatheory are fundamentally different from those within the object theory.
    \item Uncountable sets are paradoxical and \logic{ZFC} must disallow them in order to be consistent.
    \item \logic{ZFC} is inconsistent even when restricted to countable sets.
  \end{itemize}
\end{example}

\begin{theorem}[Upward L\"owenheim-Skolem theorem]\label{thm:upward_lowenheim_skolem_theorem}\mcite[thm. 3.4.9]{Hinman2005}
  Let \( \Gamma \) be a set of closed first-order formulas over some arbitrary language. Suppose that \( \Gamma \) has a \hyperref[def:first_order_model]{model} of \hyperref[def:set_finiteness]{infinite} \hyperref[thm:cardinality_existence]{cardinality} \( \kappa \).

  Then for every \hyperref[def:set_finiteness]{infinite} \hyperref[def:cardinal]{cardinal} \( \mu \), the theory \( \Gamma \) also has a model of cardinality \( \mu \).
\end{theorem}

\paragraph{Categories of first-order models}

\begin{definition}\label{def:category_of_small_first_order_models}\mimprovised
  Let \( \mscrL \) be a first-order language and let \( \Gamma \) be a nonempty set of formulas. Suppose that we are given a \hyperref[def:grothendieck_universe]{Grothendieck universe} \( \mscrU \), which is safe to assume to be the smallest suitable one as explained in \fullref{def:large_and_small_sets}. We define the \term{category of \( \mscrU \)-small models} for \( \Gamma \) as the following \hyperref[rem:concrete_categories]{concrete category}:

  \begin{itemize}
    \item The \hyperref[def:category/objects]{objects} are the \( \mscrU \)-small models of \( \Gamma \).

    \item The \hyperref[def:category/morphisms]{morphisms} between two models are the \hyperref[def:first_order_homomorphism]{structure homomorphisms} between them.
  \end{itemize}
\end{definition}

\begin{remark}\label{rem:positive_formulas_in_category_of_models}
  We usually want \( \Gamma \) to be a set of \hyperref[def:positive_formula]{positive formulas} because, in general, homomorphisms are not injective and the homomorphic image of a model may fail to be a model. \Fullref{thm:positive_formulas_preserved_under_homomorphism} shows that if all formulas in \( \Gamma \) are positive, the image of any homomorphism is again a model and thus an object in the category.
\end{remark}

\begin{example}\label{ex:def:category_of_small_first_order_models}
  This is an incomplete list of categories of small models of \hyperref[def:first_order_theory]{first-order theories} that can be found in this document:
  \begin{itemize}
    \item The categories \hyperref[def:pointed_set/category]{\( \cat{Set_*} \)}, \hyperref[def:set_with_involution/category]{\( \cat{Inv} \)}, \hyperref[def:semigroup/category]{\( \cat{Mag} \)}, \hyperref[def:monoid/category]{\( \cat{Mag}_* \)}, \hyperref[def:monoid]{\( \cat{Mon} \)}, \hyperref[def:group/category]{\( \cat{Grp} \)} and \hyperref[def:abelian_group]{\( \cat{Ab} \)}, whose relations are crucial for the definition and properties of groups.

    \item The category \hyperref[def:semiring/category]{\( \cat{SRing} \)} of semirings, \hyperref[def:semimodule]{\( \cat{SMod}_R \)} of semimodules, \hyperref[def:ring/category]{\( \cat{Ring} \)} of rings and \hyperref[def:module/category]{\( \cat{Mod}_R \)} of modules, which are all based on commutative monoids.

    \item The categories \hyperref[def:partially_ordered_set]{\( \cat{Pos} \)} in order theory and the related \hyperref[def:preordered_set/category]{\( \cat{PreOrd} \)}, \hyperref[def:totally_ordered_set]{\( \cat{Tos} \)}, as well as \hyperref[def:lattice/category]{\( \cat{Lat} \)}, \hyperref[def:heyting_algebra/category]{\( \cat{Heyt} \)} and \hyperref[def:boolean_algebra/category]{\( \cat{Bool} \)} in lattice theory.
  \end{itemize}

  In contrast:
  \begin{itemize}
    \item We define the category \hyperref[def:category_of_small_topological_spaces]{\( \cat{Top} \)}  of topological spaces and all of its related categories within set theory without a corresponding first-order theory, and similarly for the \hyperref[def:category_of_small_affine_spaces]{category of affine spaces}.

    \item The category \hyperref[def:category_of_small_sets]{\( \cat{Set} \)} of sets with respect to either \logic{ZFC}, \logic{ZFC+U} or na\"ive set theory is not the same as the category of \( \mscrU \)-small models of set theory. Instead, it is a category within a fixed set theory. Within the metatheory of this document, we work within a fixed model of \logic{ZFC+U} with respect to \hyperref[rem:classical_logic]{classical logic}.

    \item Similarly, we do not care about models of \hyperref[def:peano_arithmetic]{Peano arithmetic} enough to study its category of \( \mscrU \)-small models. Instead, we only use a single model and denote it by \( \BbbN \).
  \end{itemize}
\end{example}

\begin{definition}\label{def:subobject_and_quotient}\mcite[126]{MacLane1998}
  For an object \( X \) in some \hyperref[def:category]{category}, define an equivalence relation on \hyperref[def:morphism_invertibility/left_cancellative]{monomorphisms} \( f: A \to X \) and \( g: B \to X \) with codomain \( X \) to hold if there exists an isomorphism \( h: A \to B \) such that \( f = g \bincirc h \). We call the equivalence classes of this relation \term{subobjects} of \( X \).

  This definition seems vague, and we will indeed say \enquote{\( A \) is a subobject} rather than \enquote{The equivalence class of \( f \) is a subobject} or even \enquote{\( f \) is a subobject}. In \hyperref[rem:concrete_categories]{concrete categories}, as long as the set-theoretic image of a function is an object, we can simply take this image as \enquote{the} canonical subobject.

  \hyperref[thm:categorical_principle_of_duality]{Dually}, for the relation where for two epimorphisms \( f: X \to A \) and \( g: X \to B \) there exists an isomorphism \( h: A \to B \) such that \( f = g \bincirc h \), we say that the equivalence classes are \term{quotient objects} of \( X \).
\end{definition}

\begin{proposition}\label{thm:def:subobject_and_quotient}
  \hyperref[def:subobject_and_quotient]{Subobjects and quotients} have the following basic properties:
  \begin{thmenum}
    \thmitem{thm:def:subobject_and_quotient/self} Every object is both a subobject and a quotient of itself.
    \thmitem{thm:def:subobject_and_quotient/zero} A \hyperref[def:universal_objects/zero]{zero object} is both a subobject and a quotient object of every object.
  \end{thmenum}
\end{proposition}
\begin{proof}
  \SubProofOf{thm:def:subobject_and_quotient/self} The identity isomorphism \( \id: A \to A \) is both a monomorphism into \( A \) and an epimorphism from \( A \).

  \SubProofOf{thm:def:subobject_and_quotient/zero} Let \( Z \) be a zero object and \( A \) be an arbitrary object.

  There exists a unique morphism \( f: Z \to A \). It is a monomorphism because, given a pair of parallel morphisms \( g: B \to Z \) and \( h: B \to Z \) satisfying \( f \bincirc g = f \bincirc h \), we have \( g = h \) simply because \( Z \) is a terminal object. The condition \( f \bincirc g = f \bincirc h \) is not every necessary.

  Thus, \( Z \) is a subobject of \( A \). We can similarly show that it is a quotient object.
\end{proof}

\begin{definition}\label{def:simple_object}\mimprovised
  In a \hyperref[def:category]{category} with a \hyperref[def:universal_objects/zero]{zero object} \( Z \), we say that an object \( A \) is \term[bg=прост (\cite[307]{ГеновМиховскиМоллов1991}), ru=простой (\cite[99]{Шафаревич1999}), en=simple (\cite[643]{Lang2002})]{simple} if its only \hyperref[def:subobject_and_quotient]{quotient objects} are \( Z \) and \( A \) itself.
\end{definition}

\begin{proposition}\label{thm:first_order_categorical_invertibility}
  Fix a \hyperref[def:category_of_small_first_order_models]{category of small models} of a \hyperref[def:first_order_theory]{first-order theory}.

  \begin{thmenum}
    \thmitem{thm:first_order_categorical_invertibility/injective} Every \hyperref[def:set_valued_map/empty]{\hi{nonempty}} injective homomorphism is a \hyperref[def:morphism_invertibility/left_cancellative]{categorical monomorphism}.

    This holds for any \hyperref[def:first_order_embedding]{structure embedding}. In particular, via the canonical inclusion, every \hyperref[def:first_order_submodel]{submodel} of \( \mscrX \) is a \hyperref[def:subobject_and_quotient]{categorical subobject} of \( \mscrX \).

    A partial converse always holds --- every \hyperref[def:morphism_invertibility/left_cancellative]{split monomorphism} is injective.

    If the \hyperref[def:concrete_category]{forgetful functor} \( U: \cat{C} \to \cat{Set} \) has a \hyperref[def:category_adjunction]{left adjoint}, as it often does, then every monomorphism is injective. Otherwise, a homomorphism may be left invertible with respect to homomorphisms but not general functions.

    \thmitem{thm:first_order_categorical_invertibility/surjective} \hyperref[thm:categorical_principle_of_duality]{Dually}, every surjective homomorphism is a \hyperref[def:morphism_invertibility/right_cancellative]{categorical epimorphism}.

    In particular, via the canonical projection, every \hyperref[def:first_order_quotient]{quotient structure} of \( \mscrX \) is a \hyperref[def:subobject_and_quotient]{categorical quotient object} of \( \mscrX \).

    A partial converse always holds --- every \hyperref[def:morphism_invertibility/right_cancellative]{split epimorphism} is surjective.

    If the \hyperref[def:concrete_category]{forgetful functor} \( U: \cat{C} \to \cat{Set} \) has a \hyperref[def:category_adjunction]{right adjoint}, then every epimorphism is surjective.

    \thmitem{thm:first_order_categorical_invertibility/bijective} Every \hyperref[def:first_order_embedding]{structure isomorphism} is a \hyperref[def:morphism_invertibility/isomorphism]{categorical isomorphism} and vice versa.
  \end{thmenum}
\end{proposition}
\begin{proof}
  All follow from \fullref{thm:concrete_category_function_invertibility}.
\end{proof}

\begin{proposition}\label{thm:first_order_direct_product_is_categorical_product}
  Consider the \hyperref[def:category_of_small_first_order_models]{category of small models} of a \hyperref[def:first_order_theory]{first-order theory}. Suppose that \hyperref[def:first_order_direct_product]{direct products} exist, for example if the theory consists of \hyperref[def:positive_formula]{positive formulas} \hi{without disjunctions}.

  Then the \hyperref[def:discrete_category_limits]{categorical product} of a family of models is their \hyperref[def:first_order_direct_product]{direct product}.
\end{proposition}
\begin{comments}
  \item \Fullref{thm:direct_product_preserves_positive_formulas} ensures that positive formulas without disjunctions have direct products.
\end{comments}
\begin{proof}
  First note that the product \( A \coloneqq \prod_{k \in \mscrK} A_k \) with the \hyperref[thm:direct_product_projections]{projections} \( \pi \coloneqq \seq{ \pi_k }_{k \in \mscrK} \) are a \hyperref[def:category_of_cones/cone]{cone} for the \hyperref[def:discrete_category]{discrete} \hyperref[def:categorical_diagram]{diagram} \( \seq{ A_k }_{k \in \mscrK} \).

  Let \( (C, \alpha) \) also be a cone. We want to define a first-order homomorphism \( l_C: C \to A \) such that, for every \( m \in \mscrK \) and \( c \in C \),
  \begin{equation*}
    \alpha_m(c) = \pi_m(l_C(c)).
  \end{equation*}

  The value of \( l_C(c) \) is determined entirely by the action on each individual coordinate. This suggests the definition
  \begin{equation*}
    l_C(c) \coloneqq \seq{ \alpha_k(c) }_{k \in \mscrK}.
  \end{equation*}
\end{proof}

\begin{definition}\label{def:object_presentation}\mimprovised
  Consider the \hyperref[def:category_of_small_first_order_models]{category of small models} \( \cat{C} \) of a \hyperref[def:first_order_theory]{first-order theory} consisting of \hyperref[def:positive_formula]{positive formulas} over a language without predicates. Suppose that the \hyperref[def:concrete_category]{forgetful functor} \( U: \cat{C} \to \cat{Set} \) has a \hyperref[def:category_adjunction]{left adjoint} \( F: \cat{Set} \to \cat{C} \).

  Fix a \hyperref[def:set]{plain set} \( S \) and a \hyperref[def:binary_relation]{binary relation} \( {\sim} \) on \( S \). Let \( {\cong} \) be the \hyperref[def:first_order_congruence]{congruence} in \( F(S) \) \hyperref[def:first_order_generated_congruence]{generated} by \( {\sim} \). Then we can form the \hyperref[def:first_order_quotient]{quotient} \( F(S) / {\cong} \).

  If the model \( X \) is isomorphic to \( F(S) / {\cong} \), we say that \( (S, {\sim}) \) is a \term{presentation} of \( X \) with a set \( S \) of \term{generators} and \( {\sim} \) of \term{relators}\footnote{We regard \( {\sim} \) as a set of ordered pairs}.

  \begin{thmenum}
    \thmitem{def:object_presentation/generator_cardinality} We say that \( X \) is \term{finitely generated} if there exists a presentation with finitely many generators. We can introduce similar terminology for countably many generators, as well as other cardinalities.

    \thmitem{def:object_presentation/relator_cardinality} Analogously, we say that \( X \) is \term{finitely related} if there exists a presentation with finitely many relators, and similarly for other cardinalities.

    \thmitem{def:object_presentation/cardinality} Finally, we say that \( X \) is \term{finitely presented} if there exists a presentation where both the set of generators and set of relators are finite, and similarly for other cardinalities.
  \end{thmenum}
\end{definition}

\begin{proposition}\label{thm:object_presentation_existence}
  In the conditions of \fullref{def:object_presentation}, every object in \( \cat{C} \) has at least one \hyperref[def:object_presentation]{presentation}.
\end{proposition}
\begin{proof}
  Fix an object \( X \) in \( \cat{C} \) and denote its underlying set \( U(X) \) by \( S \).

  The unit \( \eta \) and counit \( \varepsilon \) of the adjunction exist have the following instances: morphisms \( \eta_S: S \to U(F(S)) \) and \( \varepsilon_X: F(S) \to X \) such that
  \begin{equation*}
    U(\varepsilon_X) \bincirc \eta_S = \id_S.
  \end{equation*}

  \Fullref{thm:homomorphism_induces_congruence} implies that the relation \( {\cong} \) such that \( a \cong a' \) if and only if \( \varepsilon_X(a) = \varepsilon_X(a') \) is a congruence. \Fullref{thm:quotient_structure_universal_property} then implies that there exists a unique homomorphism \( h: F(S) / {\cong} \to X \) such that
  \begin{equation*}
    h \bincirc \pi = \varepsilon_X.
  \end{equation*}

  But
  \begin{equation*}
    U(h \bincirc \pi) \bincirc \eta_S = U(\varepsilon_X) \bincirc \eta_S = \id_S.
  \end{equation*}

  This implies that \( U(h) \) is right invertible, and hence surjective.

  Note that \( h \) is injective by construction, and since the language contains no predicate symbols, injective homomorphisms are automatically embeddings. As a surjective embedding, \( h \) is an isomorphism between \( F(S) / {\cong} \) and \( X \).

  Define \( {\sim} \) as the restriction of \( {\cong} \) to members of \( S \). Then \( (S, \cong) \) is a presentation of \( X \).
\end{proof}

\paragraph{Lindenbaum-Tarski algebras}

\begin{definition}\label{def:lindenbaum_tarski_algebra}\mcite[def. 1.7.4]{Hinman2005}
  Assume some fixed \hyperref[def:deductive_system]{deductive system} in propositional or first-order logic. Let \( \Gamma \) be a set of \hyperref[def:first_order_syntax/closed_formula]{closed formulas} within the corresponding logic, and suppose that \( \Gamma \) is closed in the sense of \fullref{def:first_order_theory}.

  Then \( (\Gamma, \vdash) \) is a \hyperref[def:preordered_set]{preordered set}. We define the \term{Lindenbaum-Tarski algebra} of the theory \( \Gamma \) is the partially ordered set obtained from \( (\Gamma, \vdash) \) using \fullref{thm:preorder_to_partial_order}.

  More concretely, the Lindenbaum-Tarski algebra of \( \Gamma \) is a quotient set of \( \Gamma \) by the equivalence relation \hyperref[thm:equivalence_closure]{induced by} \( \vdash \) and endowed with the partial order
  \begin{equation}\label{eq:def:lindenbaum_tarski_algebra/order}
    [\varphi] \leq [\psi] \T{if and only if} \varphi \vdash \psi.
  \end{equation}
\end{definition}
\begin{comments}
  \item Of course, we can define the algebra using entailment rather than derivability, but in the cases we consider, the two are equivalent and derivability is simpler to work with.
\end{comments}
\begin{proof}
  The correctness of \eqref{eq:def:lindenbaum_tarski_algebra/order}, i.e. the fact that the relation \( \leq \) does not depend on the choice of representatives from the quotient sets, follows from \fullref{thm:preorder_to_partial_order}.

  We must only demonstrate that \( (\Gamma, \vdash) \) is indeed a preordered set. Reflexivity of \( \vdash \) follows from \fullref{def:axiomatic_deductive_system} and transitivity follows from \fullref{thm:deductive_system_transitivity}.
\end{proof}

\begin{proposition}\label{thm:intuitionistic_lindenbaum_tarski_algebra}
  Assume that we are working in the \hyperref[def:intuitionistic_propositional_deductive_systems]{intuitionistic propositional natural deduction system}. A \hyperref[def:lindenbaum_tarski_algebra]{Lindenbaum-Tarski algebra} is then is a \hyperref[def:heyting_algebra]{Heyting algebra}.

  In the \hyperref[def:classical_propositional_deductive_systems]{classical deductive system}, the algebra is instead a \hyperref[def:boolean_algebra]{Boolean algebra}.

  In the \hyperref[def:minimal_propositional_hilbert_system]{minimal deductive system}, we have an unbounded lattice with only a top element, but no bottom. Consequently, relative pseudocomplements and pseudocomplements may fail to exist.

  Explicitly:
  \begin{thmenum}
    \thmitem{thm:intuitionistic_lindenbaum_tarski_algebra/join} The \hyperref[def:lattice/join]{join} of the equivalence classes \( [\psi_1] \) and \( [\psi_2] \) is the class \( [\psi_1 \vee \psi_2] \) of their \hyperref[def:propositional_language/connectives/disjunction]{disjunction}.

    \thmitem{thm:intuitionistic_lindenbaum_tarski_algebra/bottom} The class of \hyperref[def:propositional_semantics/contradiction]{contradictions} \( [\bot] \) is the \hyperref[def:extremal_points/top_and_bottom]{bottom element}.

    \thmitem{thm:intuitionistic_lindenbaum_tarski_algebra/meet} Similarly to joins, the \hyperref[def:lattice/meet]{meet} of \( [\psi_1] \) and \( [\psi_2] \) is the equivalence class \( [\psi_1 \wedge \psi_2] \) of their \hyperref[def:propositional_language/connectives/conjunction]{conjunction}.

    \thmitem{thm:intuitionistic_lindenbaum_tarski_algebra/top} The class of \hyperref[def:propositional_semantics/tautology]{tautologies} \( [\top] \) is the \hyperref[def:extremal_points/top_and_bottom]{top element}.

    \thmitem{thm:intuitionistic_lindenbaum_tarski_algebra/relative_pseudocomplement} The \hyperref[def:heyting_algebra]{relative pseudocomplement} of \( [\psi_1] \) and \( [\psi_2] \) is the equivalence class \( [\psi_1 \rightarrow \psi_2] \).

    \thmitem{thm:intuitionistic_lindenbaum_tarski_algebra/complement} The \hyperref[eq:def:heyting_algebra/pseudocomplement]{pseudocomplement} \( \widetilde{[\psi]} \) of \( [\psi] \) equals \( [\neg \psi] \).

    In the classical deductive system this pseudocomplement is a complement, i.e. it satisfies \eqref{eq:def:boolean_algebra/join} and \eqref{eq:def:boolean_algebra/meet}.
  \end{thmenum}
\end{proposition}
\begin{proof}
  \SubProofOf{thm:intuitionistic_lindenbaum_tarski_algebra/join} We will show that \( [\psi_1 \vee \psi_2] \) is the supremum of \( [\psi_1] \) and \( [\psi_2] \).

  The inference rule \eqref{eq:def:minimal_propositional_natural_deduction_system/or/intro_left} implies that \( \psi_1 \vdash \psi_1 \vee \psi_2 \) and \eqref{eq:def:minimal_propositional_natural_deduction_system/or/intro_right} implies that \( \psi_2 \vdash \psi_1 \vee \psi_2 \). Thus, \( \psi_1 \vee \psi_2 \) is an upper bound for both \( \psi_1 \) and \( \psi_2 \) under the ordering \( \vdash \).

  Let \( \varphi \) be any formula in \( \Gamma \) such that \( \psi_1 \vdash \varphi \), \( \psi_2 \vdash \varphi \) and \( \varphi \vdash (\psi_1 \vee \psi_2) \). Then the following instance of \eqref{eq:def:minimal_propositional_natural_deduction_system/or/elim}
  \begin{equation*}
    \begin{prooftree}
      \hypo{ [\psi_1 \vee \psi_2] }
      \hypo{ [\psi_1]^1 }
      \ellipsis {} { \varphi }
      \hypo{ [\psi_2]^1 }
      \ellipsis {} { \varphi }
      \infer[left label=\( 1 \)]3[\ref{eq:def:minimal_propositional_natural_deduction_system/or/elim}]{ \varphi }
    \end{prooftree}
  \end{equation*}
  demonstrates that \( \psi_1, \psi_2, (\psi_1 \vee \psi_2) \vdash \varphi \). Hence, from \fullref{thm:deductive_system_transitivity} it follows that
  \begin{equation*}
    (\psi_1 \vee \psi_2) \vdash \varphi.
  \end{equation*}

  Our choice of representatives from \( [\psi_1] \) and \( [\psi_2] \) does not matter for derivability, hence \( [\varphi] = [\psi_1 \vee \psi_2] \) and this is indeed the supremum of \( [\psi_1] \) and \( [\psi_2] \).

  \SubProofOf{thm:intuitionistic_lindenbaum_tarski_algebra/top} The inference rule \eqref{eq:def:minimal_propositional_natural_deduction_system/top/intro} shows that \( [\varphi] \leq [\top] \) for any formula \( \varphi \) and \( [\top] \) is the top element.

  \SubProofOf{thm:intuitionistic_lindenbaum_tarski_algebra/meet} Let \( \psi_1 \) and \( \psi_2 \) be any formulas in \( \Gamma \). The inference rule \eqref{eq:def:minimal_propositional_natural_deduction_system/and/elim_left} implies that \( \psi_1 \wedge \psi_2 \vdash \psi_2 \) and \eqref{eq:def:minimal_propositional_natural_deduction_system/and/elim_right} implies that \( \psi_1 \wedge \psi_2 \vdash \psi_1 \). Thus, \( \psi_1 \wedge \psi_2 \) is a lower bound for both \( \psi_1 \) and \( \psi_2 \) under the ordering \( \vdash \).

  We must show that \( \psi_1 \wedge \psi_2 \) is derives the greatest lower bound and vice versa. Let \( \varphi \) be a formula in \( \Gamma \) such that \( \varphi \vdash \psi_1 \), \( \varphi \vdash \psi_2 \) and \( (\psi_1 \wedge \psi_2) \vdash \varphi \). We will show that \( \varphi \vdash (\psi_1 \wedge \psi_2) \).

  The rule \eqref{eq:def:minimal_propositional_natural_deduction_system/and/intro} implies that
  \begin{equation*}
    \psi_1, \psi_2 \vdash \psi_1 \wedge \psi_2.
  \end{equation*}

  But \( \varphi \) derives both \( \psi_1 \) and \( \psi_2 \), hence due to \fullref{thm:deductive_system_transitivity},
  \begin{equation*}
    \varphi \vdash (\psi_1 \wedge \psi_2).
  \end{equation*}

  Analogously to our proof of \fullref{thm:intuitionistic_lindenbaum_tarski_algebra/join}, we conclude that the choice of representatives of the equivalence classes is irrelevant and that \( [\varphi] = [\psi_1 \wedge \psi_2] \) is the infimum of \( [\psi_1] \) and \( [\psi_2] \).

  \SubProofOf{thm:intuitionistic_lindenbaum_tarski_algebra/bottom} That \( [\bot] \) is the bottom is a restatement of \eqref{eq:thm:minimal_propositional_negation_laws/efq}.

  \SubProofOf{thm:intuitionistic_lindenbaum_tarski_algebra/relative_pseudocomplement} We must prove that \( [\psi_1 \rightarrow \psi_2] \) equals
  \begin{equation*}
    ([\psi_1] \rightarrow [\psi_2]) = \underbrace{\sup\set[\Big]{ [\varphi] \given* \varphi \in \Gamma \T{and} (\varphi \wedge \psi_1) \vdash \psi_2 }}_{\Phi}.
  \end{equation*}

  An equivalent condition for \( \varphi \) to be in \( \Phi \) is, due to \fullref{thm:conjunction_of_premises},
  \begin{equation*}
    \varphi, \psi_1 \vdash \psi_2.
  \end{equation*}

  \Fullref{thm:syntactic_deduction_theorem} implies that
  \begin{equation*}
    \varphi \vdash (\psi_1 \rightarrow \psi_2).
  \end{equation*}

  Thus, \( [\psi_1 \rightarrow \psi_2] \) is an upper bound of \( \Psi \).

  It remains to show that \( (\psi_1 \rightarrow \psi_2) \in \Psi \). Both \( \psi_1 \) and \( (\psi_1 \rightarrow \psi_2) \) follow from the formula \( \parens[\Big]{ (\psi_1 \rightarrow \psi_2) \wedge \psi_1 } \) and by applying \eqref{eq:def:def:axiomatic_deductive_system/mp}, we obtain
  \begin{equation*}
    \psi_1, (\psi_1 \rightarrow \psi_2) \vdash \psi_2.
  \end{equation*}

  From \fullref{thm:conjunction_of_premises},
  \begin{equation*}
    \parens[\Big]{ (\psi_1 \rightarrow \psi_2) \wedge \psi_1 } \vdash \psi_2.
  \end{equation*}

  Therefore, \( [\psi_1 \rightarrow \psi_2] \in \Phi \) and it is indeed the supremum of \( \Psi \).

  \SubProofOf{thm:intuitionistic_lindenbaum_tarski_algebra/complement} The pseudocomplement \( \widetilde{[\psi]} \) is, by definition,
  \begin{equation*}
    \widetilde{[\psi]}
    =
    [\psi] \rightarrow [\bot].
  \end{equation*}

  From what we have already proved, we can conclude that \( \widetilde{[\psi]} = [\psi \rightarrow \bot] \). From \fullref{def:minimal_propositional_hilbert_system/negation} it follows that the formula \( \psi \rightarrow \bot \) derives \( \neg \psi \) and vice versa, thus \( \widetilde{[\psi]} = [\neg \psi] \).

  If we are working in classical logic where \eqref{eq:thm:minimal_propositional_negation_laws/lem} holds, then
  \begin{equation*}
    \sup\set{ [\psi], \widetilde{[\psi]} }
    \reloset {\ref{thm:intuitionistic_lindenbaum_tarski_algebra/join}} =
    [\psi \vee \neg \psi]
    \reloset {\eqref{eq:thm:minimal_propositional_negation_laws/lem}} =
    [\top],
  \end{equation*}
  which proves \eqref{eq:def:boolean_algebra/join}.

  The dual law shows \eqref{eq:def:boolean_algebra/meet}:
  \begin{equation*}
    \inf\set{ [\psi], \widetilde{[\psi]} }
    \reloset {\ref{thm:intuitionistic_lindenbaum_tarski_algebra/meet}} =
    [\psi \wedge \neg \psi]
    \reloset {\eqref{eq:thm:minimal_propositional_negation_laws/lnc}} =
    [\neg \top]
    =
    \widetilde{[\top]}
    \reloset {\eqref{eq:def:heyting_algebra/pseudocomplement}} =
    [\bot].
  \end{equation*}
\end{proof}

\begin{remark}\label{rem:thm:intuitionistic_lindenbaum_tarski_algebra/syntactic_proof}
  Notice that our proof of \fullref{thm:intuitionistic_lindenbaum_tarski_algebra} relies entirely on the deductive system. On the other hand, we define \hyperref[def:propositional_heyting_algebra_semantics]{Heyting semantics} for intuitionistic formulas.

  Thus, we have shown that Heyting algebras arise naturally from the intuitionistic natural deduction system and that their role as a semantical framework is justified.
\end{remark}

\begin{proposition}\label{thm:filters_intuitionistic_lindenbaum_tarski_algebra}
  Fix some first-order language \( \mscrL \) and let \( L \) be the \hyperref[thm:filters_intuitionistic_lindenbaum_tarski_algebra]{classical Lindenbaum-Tarski algebra} over all formulas of \( \mscrL \).

  Given a \hyperref[def:first_order_theory/complete]{complete set of closed formulas} \( \Gamma \), the set of (equivalence classes of) all formulas derivable from \( \Gamma \) is an \hyperref[def:ultrafilter]{ultrafilter} in \( L \).
\end{proposition}
\begin{proof}
  Simply a reformulation of the definition of completeness.
\end{proof}


  \chapter{Set theory}\label{ch:set_theory}

Sets are ubiquitous in mathematics, yet set theory itself is quite complicated. Attention needs to be put to define a \hyperref[def:first_order_theory]{logical theory} of sets that is both useful and \hyperref[def:first_order_theory/consistent]{consistent}.

We first use the simplicity of \hyperref[def:naive_set_theory]{na\"ive set theory} to introduce some fundamental definitions. This theory turns out to be inconsistent due to \fullref{thm:russels_paradox}.

We later introduce the more sophisticated \hyperref[def:zfc]{Zermelo-Fraenkel set theory} (\logic{ZFC}) and enhance it with Grothendieck's \hyperref[def:axiom_of_universes]{axiom of universes} to obtain \logic{ZFC+U}. The consistency of the latter theory is discussed in \fullref{rem:set_definition_recursion}.

Depending on the context, by \enquote{set theory} we mean either na\"ive set theory, \logic{ZF}, \logic{ZFC}, \logic{ZFC+U} or further variations.

\begin{remark}\label{rem:set_definition_recursion}
  The relation between \hyperref[sec:first_order_logic]{first-order logic} and set theory is remarkably circular.

  \begin{itemize}
    \item We define set theory as a \hyperref[def:first_order_theory]{theory} of first-order logic.

    \item We define first-order logic itself via sets --- see even the basic definitions from \fullref{sec:first_order_logic}.
  \end{itemize}

  In order to resolve this circularity, we utilize the concept of \hyperref[con:metalogic]{metalogic}:
  \begin{itemize}
    \item We start with some intuitive understanding of sets and build first-order logic as in \fullref{ch:mathematical_logic}. This is our initial metatheory, where we assume the availability of first-order logic. We construct formulas specifying certain special sets that can be used as models of \logic{ZFC} --- see \fullref{sec:grothendieck_universes}

    \item We now transition to work within \logic{ZFC+U} as a metatheory and \logic{ZFC} as an object theory. We use Grothendieck's \hyperref[def:axiom_of_universes]{axiom of universes} in the metatheory because it provides us with \hyperref[def:set_countability/uncountable]{uncountable} \hyperref[def:grothendieck_universe]{Grothendieck universes}\fnote{We can assume the existence of only one such universe, but that would complicate our discussion of \hyperref[ch:category_theory]{category theory}}, which are models of \logic{ZFC} as a consequence of \fullref{thm:grothendieck_universe_is_model_of_zfc}. Hence, within this metatheory, the object theory \logic{ZFC} is consistent.
  \end{itemize}

  There are, however, caveats, among which:
  \begin{itemize}
    \item We restrict ourselves to \hyperref[rem:standard_model_of_set_theory]{standard} \hyperref[rem:transitive_model_of_set_theory]{transitive} models in order to avoid very counterintuitive results.

    \item It is possible that the set theory which we use within the metatheory in order to provide models of \logic{ZFC} is itself inconsistent. In that case, due to \eqref{eq:thm:intuitionistic_tautologies/efq}, every theorem can be derived and a proof of consistency of \logic{ZFC} is insubstantial.
  \end{itemize}
\end{remark}

\begin{remark}\label{rem:first_order_theories_in_zfc}
  Instead of discussing first-order theories like the \hyperref[def:group/theory]{theory of groups}, we can instead reformulate the definition within set theory and add a \hyperref[con:formula_defined_predicate]{formula-defined predicate} \( \op{IsGroup}[\synx] \), which is valid only for groups.

  This is a natural approach, and we will use it implicitly. Furthermore, it makes no sense to speak about concepts like the \hyperref[thm:substructures_form_complete_lattice]{lattice of subgroups} or the \hyperref[def:cardinal]{cardinality} of a group otherwise.

  This is also a natural framework for defining \hyperref[def:topological_space]{topological spaces} and \hyperref[def:category]{categories} via \hyperref[def:directed_multigraph]{directed multigraphs}. Some theories like the \hyperref[def:partially_ordered_set]{partially ordered sets} are first-order theories, but \hyperref[def:well_ordered_set]{well-ordered sets} is an extension that requires an ambient set theory.

  Thus, roughly, set theory allows us to use \hyperref[def:higher_order_logic]{higher-order relations and types} within first-order logic.

  We will end this remark with a quote from \cite[17]{UnivalentProject2024OctoberHoTT}:
  \begin{displayquote}
    We note that a set-theoretic foundation has two \enquote{layers}: the deductive system of first-order logic, and, formulated inside this system, the axioms of a particular theory, such as ZFC. Thus, set theory is not only about sets, but rather about the interplay between sets (the objects of the second layer) and propositions (the objects of the first layer).
  \end{displayquote}
\end{remark}

  \subsection{Na\"ive set theory}\label{subsec:naive_set_theory}

Na\"ive set theory is traditionally defined informally by only specifying that a set is an unordered collection of objects without repetition. It turns out that this can easily be formalized as a \hyperref[def:first_order_theory]{first-order theory}, albeit an inconsistent one. Still, this theory is useful for introducing important concepts that can ease the introduction of more elaborate theories like \hyperref[def:zfc]{\logic{ZFC}}. The definitions we introduce and the proofs we provide will turn out to be valid in \logic{ZFC} also. In other words, we will transparently utilize \hyperref[def:naive_set_theory/unrestricted_comprehension]{unrestricted comprehension} for constructing sets and later in \fullref{thm:zfc_existence_theorems} we will prove that they exist not only in na\"ive set theory, but also in \logic{ZFC}.

\begin{remark}\label{rem:pure_set_theory}
  What we lose in this formalization are objects which are not sets, usually called \term{atoms} or \term{urelements} (because of the German prefix \enquote{ur}, meaning primordial). It is not necessary for us to add such elements since we can encode everything via sets. Theories without atoms, like our versions of nai\"ve set theory and \hyperref[def:axiom_of_universes]{\logic{ZFC+U}}, are called \term{pure set theories}.
\end{remark}

\begin{definition}\label{def:naive_set_theory}
  The \term{language of na\"ive set theory} is a \hyperref[def:first_order_syntax]{first-order language} \( \mscrL \) with only a single \hyperref[rem:first_order_formula_conventions/infix]{infix} binary relation \( \in \) called \term{set membership}. If \( \xi \in \eta \), we say that \( \xi \) is a \term{member} or \term{element} of \( \eta \) and that \( \eta \) \term{contains} \( \xi \).

  For the sake of simplicity, we will not introduce into the language any other functional or predicate symbols, but will use \hyperref[rem:predicate_formula]{predicate formulas} when needed mostly for formulating axioms. See the \( \ref{eq:def:grothendieck_universe/predicate}[\tau] \) predicate for an extreme example.

  \term{Na\"ive set theory} is a \hyperref[def:first_order_theory]{first-order theory} axiomatized by the following:
  \begin{thmenum}
    \thmitem{def:naive_set_theory/extensionality}\mcite[6]{Jech2003} The \term{axiom of extensionality}, which states that two sets are equal if and only if they have the same members. Symbolically,
    \begin{equation}\label{eq:def:naive_set_theory/extensionality}
      \parens[\Big]{ \qforall \xi (\xi \in \tau \leftrightarrow \xi \in \sigma) } \rightarrow \parens[\Big]{ \tau \doteq \sigma }.
    \end{equation}

    As a consequence, a set is only distinguished by what it contains and thus the ordering and repetition of members of a set play no role. This axiom is also important in \logic{ZFC} --- see \fullref{def:zfc/extensionality}.

    It is very common when dealing with sets, as in \eqref{eq:def:naive_set_theory/extensionality}, to use \hyperref[rem:first_order_formula_conventions/relativization]{relativization of quantifiers} with \( \in \).

    As explained in \fullref{rem:mathematical_logic_conventions/quantification}, we avoid excessive universal quantification. We actually add as an axiom of the theory the \hyperref[def:universal_closure]{universal closure} of \eqref{eq:def:naive_set_theory/extensionality}:
    \begin{equation}\label{eq:def:naive_set_theory/extensionality_quantified}
      \qforall \tau \qforall \sigma \parens[\Bigg]{ \parens[\Big]{ \qforall \xi (\xi \in \tau \leftrightarrow \xi \in \sigma) } \rightarrow \parens[\Big]{ \tau \doteq \sigma } }.
    \end{equation}

    The \hyperref[def:material_implication/converse]{converse} of \eqref{eq:def:naive_set_theory/extensionality} obvious.

    \thmitem{def:naive_set_theory/unrestricted_comprehension} The \term{axiom schema of unrestricted comprehension} states that any formula defines a set. For each formula \( \varphi \) not containing \( \tau \) as a free variable, the following is an axiom:
    \begin{equation}\label{eq:def:naive_set_theory/unrestricted_comprehension}
      \qexists \tau \qforall \xi (\xi \in \tau \leftrightarrow \varphi).
    \end{equation}

    It is important to highlight that \( \varphi \) may have any number of free variables as long as \( \tau \) is not among them. Of course, this axiom is only interesting if \( \xi \in \boldop{Free}(\varphi) \). If \( \eta_1, \ldots, \eta_n \) are all the other free variables of \( \varphi \), then the \hyperref[def:universal_closure]{universal closure} of the corresponding axiom is
    \begin{equation}\label{eq:def:naive_set_theory/unrestricted_comprehension_quantified}
      \qforall {\eta_1} \cdots \qforall {\eta_n} \qexists \tau \qforall \xi (\xi \in \tau \leftrightarrow \varphi).
    \end{equation}

    In other words, the set \( \tau \) is not unique in general, but depends on the free variables \( \eta_1, \ldots, \eta_n \). For this reason, they are called \term{parameters} of the axiom.

    Compare this axiom schema to \hyperref[def:zfc/specification]{restricted comprehension}. In the context of na\"ive set theory they are equivalent because each is a special case of the other one.

    Because our goal is for all our constructions to be valid in \hyperref[def:zfc]{\logic{ZFC}}, we only use unrestricted comprehension where the existence of the set is justified by other axioms of \logic{ZFC}.
  \end{thmenum}
\end{definition}

\begin{remark}\label{rem:epsilon_and_set_membership}
  The symbol \( \in \) is derived from \( \varepsilon \). Some older books like \cite{Kelley1975} even use \( \varepsilon \) for set membership. \Fullref{thm:epsilon_induction} is named after set membership.
\end{remark}

\begin{definition}\label{def:set}
  Assume that we have a fixed \hyperref[rem:standard_model_of_set_theory]{standard} \hyperref[rem:transitive_model_of_set_theory]{transitive} model \( \mscrV = (V, I) \) of \hyperref[def:naive_set_theory]{na\"ive set theory} or \hyperref[def:zfc]{\logic{ZFC}}, with or without the \hyperref[def:axiom_of_universes]{axiom of universes}. We will assume \logic{ZFC+U} by default.

  We say that any member of \( V \) is a \term{set}. If \( A \) is a set and \( x \in A \), we say that \( x \) is a \term{member} or \term{element} of \( A \) or, in a geometric context, a \term{point} in \( A \).

  We usually refer to \( V \) as our \term{universe} or \term{universal set}. When working with \hyperref[def:grothendieck_universe]{Grothendieck universes}, we may wish to further distinguish between the universal set and some Grothendieck universe. Fortunately, we very rarely refer to \( V \) itself.
\end{definition}

\begin{remark}\label{rem:standard_model_of_set_theory}
  We will say that a model \( \mscrV = (V, I) \) of set theory is \term{standard} if the interpretation \( I(\in) \) of the membership predicate symbol is precisely the membership relation in the metatheory. We will only consider standard models of set theory. This is immensely important for the following reasons:

  \begin{itemize}
    \thmitem{rem:standard_model_of_set_theory/set_builder_notation} Set-builder notation relies on constructing sets in the metatheory and then using them in the object theory. If the model is not standard, then, for any variable assignment \( v \) in the model, it does not hold that \( \Bracks{\xi \in \eta}_{v_{\xi \mapsto x, \eta \mapsto y}} = T \) if and only if \( x \in y \).

    \thmitem{rem:standard_model_of_set_theory/skolems_paradox} It is possible that cardinality is incompatible between the object theory and metatheory --- see \fullref{ex:skolems_paradox}. We want to avoid countable sets in the metatheory to be uncountable in the object theory, for example.
  \end{itemize}

  Therefore, it is reasonable to assume that all our models of set theory are standard. We also want their domains to be transitive sets -- see \fullref{rem:transitive_model_of_set_theory}.
\end{remark}

\begin{definition}\label{def:set_builder_notation}
  As mentioned in \fullref{rem:set_definition_recursion}, set theory somewhat blurs the line between logic and metalogic. In particular some \hyperref[def:first_order_definability]{definable} subsets of the universe \( U \) of the fixed model \( \mscrU \) are themselves sets within the object logic.

  Fix a formula \( \varphi \) whose free variables are \( \xi \) and \( \eta_1, \ldots, \eta_n \). In the simplest case, \( n = 0 \) and \( \xi \) is the only free variable of \( \varphi \).

  Also fix an \( n \)-tuple \( u_1, \ldots, u_n \) of members of \( U \), which we will call \term{parameters}. Denote by \( A \) the subset of \( U \) consisting of members \( x \) of \( U \) such that \( \varphi\Bracks{x, u_1, \ldots, u_n} = T \).

  We introduce a special convenience notation for \( A \) called \term{set-builder notation}:
  \begin{equation*}
    A \coloneqq \set{ x \given \varphi\Bracks{x, u_1, \ldots, u_n} }.
  \end{equation*}

  Since set-builder notation is metalogical, we do not impose strict syntax rules and use prose where it is straightforward to translate it into a logical formula.

  For example, the \hyperref[def:basic_set_operations/intersection]{intersection} of the sets \( B \) and \( C \) is given by the formula \( \xi \in \eta \wedge \xi \in \zeta \), where \( B \) is a value for the parameter \( \eta \) and \( C \) is a parameter for \( \zeta \). The intersection can thus be written as
  \begin{equation*}
    A \coloneqq \set{ x \given x \in B \T{and} x \in C }.
  \end{equation*}

  Note that at this point \( A \) is a set within the metatheory and its members are sets within the object logic, however \( A \) may not be a set within the object logic and its members may not be sets within the metatheory.

  Nonetheless, within na\"ive set theory, as a consequence of the \hyperref[def:naive_set_theory/unrestricted_comprehension]{axiom schema of unrestricted comprehension}, \( A \) is also a set within the object logic. More precisely, given our choice of parameters \( u_1, \ldots, u_n \), the axiom schema instance \eqref{eq:def:naive_set_theory/unrestricted_comprehension_quantified} guarantees the existence of a member \( a \) of \( U \), such that
  \begin{equation*}
    \parens{ \xi \in \tau }\Bracks{ \tau \mapsto a, \xi \mapsto x } = T
    \T{if and only if}
    x \isinE A,
  \end{equation*}
  where we have denoted set membership within the object logic by \( \in \) and within the metatheory by \( \isinE \). We will not further use this symbol and the two membership relations will be used interchangeably.

  This is where the line between logic and metalogic blurs --- we can speak about roughly the same sets within the object logic and the metatheory.

  Examples such as \fullref{thm:russels_paradox} show that unrestricted comprehension can easily lead to an inconsistent object logic. In more elaborate set theories like \hyperref[def:zfc]{\logic{ZFC}}, we only allow restricted comprehension via the \hyperref[def:zfc/specification]{axiom schema of specification}. Instead of defining \( A \) as a set of all members of \( U \) satisfying a certain property, restricted comprehension allows us to define \( A \) as a subset not of the universe \( U \), but of some well-behaved subset \( B \) of \( U \). The corresponding notation is
  \begin{equation*}
    \set{ x \in B \given \varphi\Bracks{x, u_1, \ldots, u_n} }.
  \end{equation*}

  Of course, we may still use unrestricted comprehension of the result is guaranteed to be a set within the object logic.

  Within \logic{ZFC}, subsets of \( U \) which are not sets in the object logic are called \term{proper classes}. Sets and proper classes are collectively called \term{classes}. We avoid referencing proper classes because that can easily lead us to an inconsistent theory. See \fullref{def:large_and_small_sets} for a clever workaround.

  A bigger problem that may happen is described in \fullref{rem:transitive_model_of_set_theory}.

  Other liberties regarding set-builder notation include the following:
  \begin{itemize}
    \item We often place arbitrary terms on the left side rather than only sets. This is simply a convenient metalogical notation; the symbols that are used in these terms are often not part of the object language. For example, we write the odd integers as
    \begin{equation*}
      \set{ 2n + 1 \given n \in \BbbZ }.
    \end{equation*}

    \item Instead of using the delimiter \( \given \), we sometimes also use \( : \), especially when dealing with absolute values and divisibility:
    \begin{equation*}
      \set{ \abs{n} : n \mid 125 }
    \end{equation*}
    can be more readable than
    \begin{equation*}
      \set{ \abs{n} \given n \mid 125 }
    \end{equation*}

    \item If a set has only a small finite amount of members, we usually prefer to enumerate them as
    \begin{equation*}
      \set{ 1, 3, 9, 27 }.
    \end{equation*}

    Because of the \hyperref[def:naive_set_theory/extensionality]{axiom of extensionality}, the order and repetition of the objects inside the curly braces are irrelevant. Nevertheless, using any unconventional order does not benefit us in any way.

    \item We can also place an ellipsis if a certain pattern is obvious:
    \begin{equation*}
      \set{ 1, 3, 9, 27, \ldots }.
    \end{equation*}

    This works specifically for defining countable sets.
  \end{itemize}

  Note that we have used certain numbers, but this was only for illustrative purposes because even the \hyperref[def:natural_numbers]{natural numbers} are not yet defined in terms of sets.
\end{definition}

\begin{remark}\label{rem:multile_set_membership_shorthand}
  Within the metatheory, we often use the notation \( x_1, \ldots, x_n \in A \) to mean that \( x_k \in A \) for \( k \in 1, \ldots, n \).
\end{remark}

\begin{remark}\label{rem:singleton_sets}
  Sets with a single elements are usually called \term{singletons}. It is sometimes convenient, especially with connection to geometry or \hyperref[def:multi_valued_function]{multi-valued functions} (e.g. when dealing with \hyperref[def:net_convergence/limit]{limits of nets} or \hyperref[def:subdifferentials]{subdifferentials}), to not distinguish between singleton sets and their corresponding element.
\end{remark}

\begin{definition}\label{def:empty_set}
  A very important set is the \term{empty set}
  \begin{equation*}
    \varnothing \coloneqq \set{ x \given \bot },
  \end{equation*}
  which contains no elements. We will also find useful the \hyperref[rem:predicate_formula]{predicate formula}
  \begin{equation*}\taglabel[\op{IsEmpty}]{eq:def:empty_set/predicate}
    \ref{eq:def:empty_set/predicate}[\tau] \coloneqq \qforall \eta \neg \eta \in \tau.
  \end{equation*}

  We will often refer to \term{nonempty sets}, which are exactly what they sound --- sets that are not the empty set.
\end{definition}

\begin{theorem}[Russell's paradox]\label{thm:russels_paradox}
  \hyperref[def:naive_set_theory]{Na\"ive set theory} is \hyperref[def:first_order_theory_consistency]{inconsistent}. More precisely, the instance of the \hyperref[def:naive_set_theory/unrestricted_comprehension]{schema of unrestricted comprehension} with
  \begin{equation}\label{eq:thm:russels_paradox_comprehension_formula}
    \varphi = (\xi \not\in \xi)
  \end{equation}
  allows us to derive \( \bot \) in \hyperref[rem:classical_logic]{classical logic}.

  Thus, the set
  \begin{equation}\label{eq:thm:russels_paradox_set}
    R \coloneqq \set{ x \given x \not\in x }
  \end{equation}
  of all sets that do not contain themselves is not well-defined. Indeed, from \( R \not\in R \) it follows that \( R \in R \) and from \( R \in R \) it follows that \( R \not\in R \).
\end{theorem}
\begin{proof}
  After substituting \eqref{eq:thm:russels_paradox_comprehension_formula} in \eqref{eq:def:naive_set_theory/unrestricted_comprehension}, we obtain the following axiom of na\"ive set theory:
  \begin{equation}\label{eq:thm:russels_paradox_comprehension_axiom}
    \psi \coloneqq \qexists \tau \qforall \xi (\xi \in \tau \leftrightarrow \neg (\xi \in \xi)).
  \end{equation}

  We will show that the negation \( \neg\psi \) of \( \psi \) is also derivable in this theory. An explicit form of the negation can be obtained by utilizing the equivalences \fullref{thm:first_order_quantifiers_are_dual} and \fullref{thm:boolean_equivalences/biconditional_negation}:
  \begin{equation*}
    \neg\psi = \qforall \tau \qexists \xi (\xi \in \tau \leftrightarrow \xi \in \xi).
  \end{equation*}

  This holds when \( \xi \) and \( \tau \) take on the same value, hence it is satisfiable in na\"ive set theory. By \fullref{thm:classical_first_order_logic_is_sound_and_complete}, \( \neg\psi \) is derivable in the theory.

  Thus, \( \psi \) and \( \neg\psi \) are both derivable in the same theory, and we can use \eqref{eq:def:minimal_propositional_natural_deduction_system/neg/elim} to also derive \( \bot \), which shows that na\"ive set theory is inconsistent.
\end{proof}

\begin{definition}\label{def:subset}
  We say that \( A \) is a \term{subset} of \( B \) and write \( A \subseteq B \) if every member of \( A \) is a member of \( B \). If \( A \) is a subset of \( B \), we say that B is a \term{superset} of \( A \).

  If \( A \subseteq B \) and \( A \neq B \), we say that \( A \) is a \term{proper subset} of \( B \) and write \( A \subsetneq B \).

  The relation \( \subseteq \) is called the inclusion relation, and it gives a partial ordering between sets. See \fullref{thm:boolean_algebra_of_subsets}. If an entire family of sets are not pairwise comparable, we say that they are \term{disjoint}.

  The following \hyperref[rem:predicate_formula]{predicate formula}
  \begin{equation*}\taglabel[\op{IsSubset}]{eq:def:subset/predicate}
    \ref{eq:def:subset/predicate}[\rho, \tau] \coloneqq \qforall \xi (\xi \in \rho \rightarrow \xi \in \tau),
  \end{equation*}
  which is valid when \( \rho \) is a subset of \( \tau \), will occasionally be useful for us. We have chosen the letter \( \rho \) because it is the result that would be on the right-hand side if we had a corresponding infix relation in the language.
\end{definition}

\begin{remark}\label{rem:subset_notation}
  Some authors, such as \cite{Kelley1975}, use the notation \( A \subset B \) to mean \enquote{all elements of \( A \) belong to \( B \)}, even in the case when \( A = B \). To avoid confusion, we use the notations \( A \subseteq B \) and \( A \subsetneq B \).
\end{remark}

\begin{remark}\label{rem:family_of_sets}
  In a \hyperref[rem:pure_set_theory]{pure set theory}, everything is encoded as a set. However, it is often the case that we are not interested in how a set's elements are encoded as sets and only in how they behave, e.g. when working with \hyperref[def:natural_numbers]{natural numbers}, we are interested in the elements of \( \BbbN \) and not in the way every element of \( \BbbN \) is encoded as a set.

  In order to reduce repetitiveness, sets whose elements we consider to be other sets, are often called \term{families of sets}. In particular, if all (different) sets are \hyperref[def:subset]{disjoint}, we say that the family is a \term{disjoint family}. It is usually assumed that the sets are nonempty.

  We often consider \hyperref[def:cartesian_product/indexed_family]{indexed families}, i.e. sets which depend on a parameter, which further highlight our intention to distinguish between a point in a set, the set itself and some family of sets to which the latter belongs.
\end{remark}

\begin{definition}\label{def:basic_set_operations}
  We define the following operations:

  \begin{thmenum}
    \thmitem{def:basic_set_operations/union} Dually to \hyperref[def:basic_set_operations/intersection]{intersections}, the \term{union} of an arbitrary set \( \mscrA \) is defined as
    \begin{equation*}
      \bigcup A \coloneqq \set{ x \given \qexists {A \in \mscrA} x \in A }.
    \end{equation*}

    We define the \hyperref[rem:predicate_formula]{predicate formula}
    \begin{equation*}\taglabel[\op{IsUnion}]{eq:def:basic_set_operations/union/predicate}
      \ref{eq:def:basic_set_operations/union/predicate}[\rho, \tau] \coloneqq \qforall \xi \parens[\Big]{ \xi \in \rho \leftrightarrow \qexists {\eta \in \tau} \xi \in \rho }.
    \end{equation*}

    In particular, \( \bigcup \varnothing = \varnothing \).

    For two sets \( A \) and \( B \), we define the \term{binary union} as
    \begin{equation*}
      A \cup B \coloneqq \bigcup \set{ A, B } = \set{ x \given x \in A \T{or} x \in B }.
    \end{equation*}

    \thmitem{def:basic_set_operations/intersection} The \term{intersection} of a nonempty set \( \mscrA \) is
    \begin{equation*}
      \bigcap \mscrA \coloneqq \set{ x \given \qforall {A \in \mscrA} x \in A }.
    \end{equation*}

    We also introduce the \hyperref[rem:predicate_formula]{predicate formula}
    \begin{equation*}\taglabel[\op{IsIntersection}]{eq:def:basic_set_operations/intersection/predicate}
      \ref{eq:def:basic_set_operations/intersection/predicate}[\rho, \tau] \coloneqq \qforall \xi \parens[\Big]{ \xi \in \rho \leftrightarrow \qforall {\eta \in \tau} \xi \in \eta }.
    \end{equation*}

    We leave \( \bigcap \varnothing \) undefined because it should be a \hyperref[def:extremal_points/top_and_bottom]{top element} in the \hyperref[thm:boolean_algebra_of_subsets]{Boolean algebra of all sets}, but the latter object is an ambiguous object and does not even exist in \logic{ZFC} --- see \fullref{thm:zfc_existence_theorems/universe}. It does nonetheless satisfy \( \ref{eq:def:basic_set_operations/intersection/predicate}[\rho, \tau] \).

    For two sets \( A \) and \( B \), we define the \term{binary intersection} as
    \begin{equation*}
      A \cap B \coloneqq \bigcap \set{ A, B } = \set{ x \given x \in A \T{and} x \in B }.
    \end{equation*}

    \thmitem{def:basic_set_operations/difference} The \term{difference} of the sets \( A \) and \( B \) is
    \begin{equation*}
      A \setminus B \coloneqq \set{ x \in A \given x \not\in B }.
    \end{equation*}

    We define the \hyperref[rem:predicate_formula]{predicate formula}
    \begin{equation*}\taglabel[\op{IsDifference}]{eq:def:basic_set_operations/difference/predicate}
      \ref{eq:def:basic_set_operations/difference/predicate}[\rho, \tau, \sigma] \coloneqq \qforall \xi \parens[\Big]{ \xi \in \rho \leftrightarrow \xi \in \tau \wedge \neg(\xi \in \sigma) }.
    \end{equation*}

    \thmitem{def:basic_set_operations/power_set} The \term{power set} \( \pow(A) \) of \( A \) is the family of all subsets of \( A \). Symbolically,
    \begin{equation*}
      \pow(A) \coloneqq \set{ B \given B \subseteq A }.
    \end{equation*}

    The operation \( \pow \) is not technically a function since its domain is supposed to be the set of all sets, whose existence contradicts \fullref{thm:russels_paradox}. Nevertheless, this notation makes sense and is justified by \fullref{rem:unbounded_transfinite_recursion} and \fullref{ex:unary_functors_in_set}.

    We define the \hyperref[rem:predicate_formula]{predicate formula}
    \begin{equation*}\taglabel[\op{IsPowerSet}]{eq:def:basic_set_operations/power_set/predicate}
      \ref{eq:def:basic_set_operations/power_set/predicate}[\rho, \tau] \coloneqq \qforall \xi \parens[\Big]{ \xi \in \rho \leftrightarrow \ref{eq:def:subset/predicate}[\xi, \tau] }.
    \end{equation*}

    See \fullref{thm:power_set_via_subsets} for a characterization of the power set.
  \end{thmenum}
\end{definition}

\begin{proposition}\label{thm:union_intersection_distributivity}
  \hyperref[def:basic_set_operations/union]{Set unions} and \hyperref[def:basic_set_operations/intersection]{set intersections} distribute over themselves and over each other. That is,
  \begin{align}
    X \cup \bigcup \mscrA &= \bigcup \set{ A \cup X \given A \in \mscrA }, \label{eq:thm:union_intersection_distributivity/union_over_union} \\
    X \cap \bigcup \mscrA &= \bigcup \set{ A \cap X \given A \in \mscrA }, \label{eq:thm:union_intersection_distributivity/intersection_over_union} \\
    X \cup \bigcap \mscrA &= \bigcap \set{ A \cup X \given A \in \mscrA }, \label{eq:thm:union_intersection_distributivity/union_over_intersection} \\
    X \cap \bigcap \mscrA &= \bigcap \set{ A \cap X \given A \in \mscrA }. \label{eq:thm:union_intersection_distributivity/intersection_over_intersection}
  \end{align}
\end{proposition}
\begin{proof}
  \SubProofOf{eq:thm:union_intersection_distributivity/union_over_union} Trivial.

  \SubProofOf{eq:thm:union_intersection_distributivity/union_over_intersection}
  \begin{align*}
    X \cap \bigcup \mscrA
    &=
    \set{ x \given x \in X \T{and} \qexists{A \in \mscrA} x \in A }
    = \\ &=
    \set{ x \given \qexists{A \in \mscrA} (x \in A \T{and} x \in X) }
    = \\ &=
    \bigcup \set{ A \cup X \given A \in \mscrA }.
  \end{align*}

  \SubProofOf{eq:thm:union_intersection_distributivity/intersection_over_union} Analogous.

  \SubProofOf{eq:thm:union_intersection_distributivity/intersection_over_intersection} Trivial.
\end{proof}

\begin{proposition}\label{thm:set_difference}
  \hyperref[def:basic_set_operations/difference]{Set difference} has the following basic properties:
  \begin{thmenum}
    \thmitem{thm:set_difference/intersection} If \( A \) and \( B \) are subsets of \( C \), then \( A \setminus B = A \cap (C \setminus B) \).

    \thmitem{thm:set_difference/superset} For any sets \( A \) and \( B \), we have \( A \setminus B = A \setminus (A \cap B) \).

    \thmitem{thm:set_difference/double_difference} If \( A \subseteq B \), then \( B \setminus (B \setminus A) = A \)
  \end{thmenum}
\end{proposition}
\begin{proof}
  \SubProofOf{thm:set_difference/intersection} Since \( a \in A \) implies \( a \in C \), we have
  \begin{align*}
    A \setminus B
    &=
    \set{ x \in A \given x \not\in B }
    = \\ &=
    \set{ x \in A \given x \in C \T{and} x \not\in B }
    = \\ &=
    A \cap (C \setminus B).
  \end{align*}

  \SubProofOf{thm:set_difference/superset} Follows from \fullref{thm:set_difference/intersection}.

  \SubProofOf{thm:set_difference/double_difference} By \hyperref[thm:minimal_propositional_negation_laws/dne]{double negation elimination},
  \begin{align*}
    B \setminus (B \setminus A)
    &=
    \set[\Big]{ x \in B \given x \not\in \set{ x \in B \given x \not\in A } }
    = \\ &=
    \set{ x \in B \given x \in A }
    = \\ &=
    A.
  \end{align*}
\end{proof}

\begin{proposition}\label{thm:boolean_algebra_of_subsets}
  Let \( X \) be an arbitrary set. Then the \hyperref[def:basic_set_operations/power_set]{power set} \( \pow(X) \) endowed with the \hyperref[def:subset]{inclusion} partial order \( \subseteq \) is a \hyperref[def:semilattice/complete]{complete} \hyperref[def:boolean_algebra]{Boolean algebra}.

  It is \hyperref[def:boolean_algebra/trivial]{trivial} if and only if \( X \) is the empty set.

  We will not make use here of the full apparatus of Boolean algebras. Instead, we regard this statement as a shorthand that would otherwise require enumerating a lot of properties.

  \begin{thmenum}
    \thmitem{thm:boolean_algebra_of_subsets/join} The \hyperref[def:semilattice/join]{join} of an arbitrary family \( \mscrA \) of subsets of \( X \) is simply the \hyperref[def:basic_set_operations/union]{union} \( \bigcap \mscrA \).

    \thmitem{thm:boolean_algebra_of_subsets/top} The \hyperref[def:extremal_points/top_and_bottom]{top element} is the set \( X \) itself.

    \thmitem{thm:boolean_algebra_of_subsets/meet} The \hyperref[def:semilattice/meet]{meet} of an arbitrary family \( \mscrA \) of sets is simply the \hyperref[def:basic_set_operations/intersection]{intersection} \( \bigcup \mscrA \). Unlike for a general family of sets, we have no problem defining the intersection of an empty set to be the top element \( X \).

    \thmitem{thm:boolean_algebra_of_subsets/bottom} The \hyperref[def:extremal_points/top_and_bottom]{bottom element} is the empty set.

    \thmitem{thm:boolean_algebra_of_subsets/complement} The \hyperref[def:boolean_algebra]{complement} \( A^\complement \) of the subset \( A \) is the \hyperref[def:basic_set_operations/difference]{difference} \( X \setminus A \).
  \end{thmenum}

  \begin{figure}[!ht]
    \hfill
    \includegraphics[page=1]{output/thm__boolean_algebra_of_subsets.pdf}
    \hfill\hfill
    \caption{The \hyperref[def:hasse_diagram]{Hasse diagram} of \( \pow(\set{ A, B }) \) with respect to \hyperref[def:subset]{set inclusion}}
    \label{fig:thm:boolean_algebra_of_subsets}
  \end{figure}
\end{proposition}
\begin{proof}
  \SubProofOf{thm:boolean_algebra_of_subsets/join} The union of \( \mscrA \) exists by \fullref{thm:zfc_existence_theorems/arbitrary_union}, and it is itself a subset of \( \mscrA \). Every set in \( \mscrA \) is contained in \( \bigcup \mscrA \), hence it is indeed a join.

  \SubProofOf{thm:boolean_algebra_of_subsets/top} Clearly \( X \) contains every subset of \( X \).

  \SubProofOf{thm:boolean_algebra_of_subsets/join} The intersection of \( \mscrA \) exists by \fullref{thm:zfc_existence_theorems/arbitrary_union}, and it is itself a subset of \( \mscrA \). Every set in \( \mscrA \) contains \( \bigcap \mscrA \), hence it is indeed a meet.

  \SubProofOf{thm:boolean_algebra_of_subsets/bottom} The empty set is contained in every set, in particular in every subset of \( A \).

  \SubProofOf[def:semilattice/distributive_lattice]{distributivity} Follows from \fullref{thm:union_intersection_distributivity}.

  \SubProofOf{thm:boolean_algebra_of_subsets/complement} The operation \( A^\complement \) is well-defined for each subset \( A \) of \( X \) due to \fullref{thm:zfc_existence_theorems/difference}.

  By definition
  \begin{equation*}
    A \vee A^\complement
    =
    A \cup (X \setminus A)
    =
    X
  \end{equation*}
  and
  \begin{equation*}
    A \vee A^\complement
    =
    A \cup (X \setminus A)
    =
    X,
  \end{equation*}
  hence \( A^\complement \) is indeed the complement of \( A \).

  Therefore, \( \pow(X) \) is a Boolean algebra.
\end{proof}

\begin{theorem}[De Morgan's laws for sets]\label{thm:de_morgans_laws_for_sets}
  The following hold for any sets:
  \begin{align}
    X \setminus \bigcup \mscrA = \bigcap \set{ X \setminus A \given A \in \mscrA } \label{eq:thm:de_morgans_laws_for_sets/complement_of_union} \\
    X \setminus \bigcap \mscrA = \bigcup \set{ X \setminus A \given A \in \mscrA } \label{eq:thm:de_morgans_laws_for_sets/complement_of_intersection}
  \end{align}
\end{theorem}
\begin{proof}
  \SubProof{Direct proof} We have
  \begin{align*}
    X \setminus \bigcup \mscrA
    &=
    \set{ x \given x \in X \T{and} \neg \qexists{A \in \mscrA} x \in A }
    \reloset {\eqref{eq:thm:first_order_quantifiers_are_dual/negation_of_universal}} = \\ &=
    \set{ x \given x \in X \T{and} \qforall{A \in \mscrA} {x \not\in A} }
    = \\ &=
    \set{ x \given \qforall{A \in \mscrA} (x \in X \T{and} x \not\in A) }
    = \\ &=
    \bigcap \set{ X \setminus A \given A \in \mscrA }
  \end{align*}
  and
  \begin{align*}
    X \setminus \bigcap \mscrA
    &=
    \set{ x \given x \in X \T{and} \neg \qforall{A \in \mscrA} x \in A }
    \reloset {\eqref{eq:thm:first_order_quantifiers_are_dual/negation_of_universal}} = \\ &=
    \set{ x \given x \in X \T{and} \qexists{A \in \mscrA} {x \not\in A} }
    = \\ &=
    \set{ x \given \qforall{A \in \mscrA} (x \in X \T{and} x \not\in A) }
    = \\ &=
    \bigcup \set{ X \setminus A \given A \in \mscrA }.
  \end{align*}

  \SubProof{Consequence of \fullref{thm:de_morgans_laws}}
  All subsets of \( X \) form a Boolean algebra as a consequence of \fullref{thm:boolean_algebra_of_subsets}. Then
  \begin{equation*}
    \bigcup_{k \in \mscrK} (X \setminus A_k)
    \reloset {\ref{thm:set_difference/superset}} =
    \bigcup_{k \in \mscrK} (X \cap A_k)^\complement
    \reloset {\eqref{eq:thm:de_morgans_laws/complement_of_join}} =
    \parens*{ \bigcap_{k \in \mscrK} (X \cap A_k) }^\complement
    =
    \parens*{ X \cap \bigcap_{k \in \mscrK} A_k }^\complement
    \reloset {\ref{thm:set_difference/superset}} =
    X \setminus \parens*{ \bigcap_{k \in \mscrK} A_k }.
  \end{equation*}

  The other identity requires distributivity:
  \begin{equation*}
    \bigcap_{k \in \mscrK} (X \setminus A_k)
    \reloset {\ref{thm:set_difference/superset}} =
    \bigcap_{k \in \mscrK} (X \cap A_k)^\complement
    \reloset {\eqref{eq:thm:de_morgans_laws/complement_of_join}} =
    \parens*{ \bigcup_{k \in \mscrK} (X \cap A_k) }^\complement
    \reloset {\eqref{eq:thm:union_intersection_distributivity/intersection_over_union}} =
    \parens*{ X \cap \bigcup_{k \in \mscrK} A_k }^\complement
    \reloset {\ref{thm:set_difference/superset}} =
    X \setminus \parens*{ \bigcup_{k \in \mscrK} A_k }.
  \end{equation*}
\end{proof}

\begin{remark}\label{rem:inductive_sets}
  Induction is an important proof technique that is discussed in detail in the proof of \fullref{thm:nonzero_natural_numbers_have_predecessors}. There are more general forms of induction than \eqref{eq:def:peano_arithmetic/PA3} like \fullref{thm:well_founded_induction} and \fullref{thm:well_founded_induction}. They do, however, require concepts which in turn depend on the existence of natural numbers within set theory. As a consequence, we cannot prove \eqref{eq:def:peano_arithmetic/PA3} via \fullref{thm:well_founded_induction}.

  We will introduce the concept of inductive sets in \fullref{def:inductive_set} and prove in \fullref{thm:omega_induction} that a special inductive set \hyperref[thm:smallest_inductive_set_existence]{\( \omega \)}, which will be the domain of our model of \( \BbbN \), allows performing inductive proofs. The technique that allows us to perform inductive proofs on \( \omega \) can be seen in the proof of \fullref{thm:omega_is_transitive}. \Fullref{thm:omega_induction} will allow us to define natural numbers without relying on metalogical induction along the way. See the proof of \fullref{thm:omega_induction} for a description of natural number induction within set theory and \fullref{rem:standard_models_of_arithmetic} for a further discussion of the use of natural numbers in the metatheory and in the object logic.

  We also introduce \term{recursion} in parallel as a technique for constructing objects. See \fullref{thm:omega_recursion}.
\end{remark}

\begin{definition}\label{def:ordinal_successor}
  The \term{successor} \( \op{succ}(A) \) of a set \( A \) is the set
  \begin{equation*}
    \op{succ}(A) \coloneqq A \cup \set{ A }.
  \end{equation*}

  It is also called the \term{ordinal successor} operation since it is an important concept in the \hyperref[subsec:ordinals]{theory of ordinals}. See \fullref{rem:def:ordinal_successor} for an example of how it naturally arises. It should be distinguished from \term{successor cardinals} defined in \fullref{def:successor_and_limit_cardinal}.

  The following \hyperref[rem:predicate_formula]{predicate formula}
  \begin{equation*}\taglabel[\op{IsSucc}]{eq:def:ordinal_successor/predicate}
    \ref{eq:def:ordinal_successor/predicate}[\rho, \tau] \coloneqq \qforall \xi \parens[\Big]{ \xi \in \rho \leftrightarrow (\xi \in \tau \vee \xi = \tau) },
  \end{equation*}
  which states that \( \rho \) is the successor of \( \tau \), will be useful for us when working with \hyperref[def:inductive_set]{inductive sets}.
\end{definition}

\begin{definition}\label{def:inductive_set}
  A set is called \term{inductive} if contains the empty set and is closed under the \hyperref[def:ordinal_successor]{successor operator}.

  We introduce the following \hyperref[rem:predicate_formula]{predicate formula}
  \begin{equation*}\taglabel[\op{IsInductive}]{eq:def:inductive_set/predicate}
    \ref{eq:def:inductive_set/predicate}[\tau] \coloneqq
      \parens[\Big]{ \qexists {\xi \in \tau} \ref{eq:def:empty_set/predicate}[\xi] }
      \wedge
      \parens[\Big]{ \qforall {\xi \in \tau} \qexists {\eta \in \tau} \ref{eq:def:ordinal_successor/predicate}[\eta, \xi] }.
  \end{equation*}
\end{definition}

\begin{proposition}\label{thm:smallest_inductive_set_existence}
  There is a smallest (with respect to set inclusion) \hyperref[def:inductive_set]{inductive set}, which we denote by \( \omega \).
\end{proposition}
\begin{proof}
  We cannot directly define \( \omega \) as the intersection of all inductive sets since we want to avoid unrestricted comprehension. Fortunately, the existence of at least one inductive set \( A \) is justified by the \hyperref[def:zfc/infinity]{axiom of infinity} in \logic{ZFC} or by taking the entire universe in na\"ive set theory.

  Hence, we use restricted comprehension:
  \begin{equation*}
    \omega \coloneqq \set{ x \in A \given x \T{belongs to every inductive set} }.
  \end{equation*}

  To see that \( \omega \) is itself inductive, note that \( \varnothing \in \omega \) and that if \( x \in \omega \), then it also belongs to all inductive sets and hence \( \op{succ}(x) \) also belongs to all inductive sets, proving \( \op{succ}(x) \in \omega \).
\end{proof}

\begin{theorem}[Induction via inductive sets]\label{thm:omega_induction}
  We can perform induction on the \hyperref[thm:smallest_inductive_set_existence]{smallest inductive set} \( \omega \). That is, we can prove that some property holds for every element of \( \omega \) by proving the following:
  \begin{itemize}
    \item The property holds for \( \varnothing \)
    \item We can prove that is holds for \( \op{succ}(n) \) by assuming that it holds for some set \( n \in \omega \).
  \end{itemize}

  This is an analog of \eqref{eq:def:peano_arithmetic/PA3} and is actually used in \fullref{thm:omega_is_model_of_pa} to prove that \( \omega \) is a model of \hyperref[def:peano_arithmetic]{\logic{PA}}. Instead of an entire theorem schema, however, for this theorem it is sufficient to use one single formula. The more general induction principles that use theorem schemas cannot be proved without natural numbers, which are a model of \logic{PA} by virtue of this theorem.

  More formally, the following is a theorem of both na\"ive set theory and \hyperref[def:zfc]{\logic{ZF}}:
  \footnotesize
  \begin{equation}\label{eq:thm:omega_induction}
    \qexists \sigma
    \qforall \tau
    \parens[\Bigg]
      {
        \parens[\Bigg]
          {
            \parens[\Big]
            {
              \underbrace{ \qexists {\xi \in \tau} \ref{eq:def:empty_set/predicate}[\xi] }_{\mathclap{\T{base case}}}
            }
            \wedge
            \qforall \xi \parens[\Big]
              {
                \overbrace
                  {
                    \underbrace{ \xi \in \tau }_{\mathclap{\substack{\T{inductive} \\ \T{hypothesis}}}}
                    \rightarrow
                    \underbrace
                      {
                        \qexists {\eta \in \tau} \ref{eq:def:ordinal_successor/predicate}[\eta, \xi]
                      }_{\mathclap{\substack{\T{inductive step} \\ \T{conclusion}}}}
                  }^{\T{inductive step}}
              }
          }
        \rightarrow
        \underbrace{ \ref{eq:def:subset/predicate}[\sigma, \tau] }_{\T{conclusion}}
      }
  \end{equation}
  \normalsize
\end{theorem}
\begin{proof}
  The antecedent of (the inner formula in) \eqref{eq:thm:omega_induction} is a restatement of the predicate formula \( \ref{eq:def:inductive_set/predicate}[\tau] \). The situation resembles the \hyperref[eq:def:zfc/infinity]{axiom of infinity}, but, instead of existence of an inductive set \( \tau \), it states the existence of a set \( \sigma \) such that if \( \tau \) is an inductive set, then \( \sigma \) is a subset of \( \tau \) (if we restrict \( \xi \) to range only over members of \( \sigma \), then we would obtain equality of \( \tau \) and \( \sigma \) instead). In other words, we have reduced the verification of \eqref{eq:thm:omega_induction} to showing that there exists a minimal inductive set in both na\"ive set theory and \logic{ZF}.

  We have already proved in \fullref{thm:smallest_inductive_set_existence} that our fixed model \( \mscrV = (V, I) \) of set theory has a minimal inductive set \( \omega \). Thus, for any variable assignment \( v: \boldop{Var} \to V \), the modified assignment \( v_{\sigma \mapsto \omega} \) satisfies \eqref{eq:thm:omega_induction} with the outer existential quantifier removed. Hence, by \fullref{def:first_order_valuation/formula_valuation}, it follows that the entire formula \eqref{eq:thm:omega_induction} is satisfied by the assignment \( v \).

  Both the assignment \( v \) and the model \( \mscrV \) were arbitrary, therefore we can conclude that \eqref{eq:thm:omega_induction} is a theorem of both na\"ive set theory and \logic{ZF}.
\end{proof}

\begin{definition}\label{def:transitive_set}
  A set \( A \) is \term{transitive} if from \( B \in A \) it follows that \( B \subseteq A \).

  See \fullref{rem:ordinal_definition} for a discussion of the motivation and terminology of transitive sets and \fullref{rem:transitive_model_of_set_theory} for their importance.

  We introduce the following \hyperref[rem:predicate_formula]{predicate formula}:
  \begin{equation*}\taglabel[\op{IsSetTransitive}]{eq:def:transitive_set/predicate}
    \ref{eq:def:transitive_set/predicate}[\tau] \coloneqq \qforall {\xi \in \tau} \qforall {\eta \in \xi} {\eta \in \tau}
  \end{equation*}
\end{definition}

\begin{proposition}\label{thm:omega_is_transitive}
  The set \( \omega \) is transitive and every member of \( \omega \) is transitive.

  This proof demonstrates usage of \fullref{thm:omega_induction}.
\end{proposition}
\begin{proof}
  \SubProof{Proof that all members of \( \omega \) are transitive} In order to demonstrate how \fullref{thm:omega_induction} works in practice, we will use inductive sets directly. Let \( T \subseteq \omega \) be the subset of all transitive members of \( \omega \). We will show that \( T \) is inductive. Since \( T \subseteq \omega \) and \( \omega \) is the smallest inductive set, we can conclude that \( T = \omega \).

  Clearly \( \varnothing \in T \)  because every member of \( \varnothing \) vacuously is a subset of \( \varnothing \).

  Now suppose that \( n \in T \) and let \( m \in \op{succ}(n) = n \cup \set{ n } \). If \( m = n \), then \( m \in \op{succ}(n) \) by definition of the successor operation. If \( m \in n \), then \( m \subseteq n \) by the inductive hypothesis and hence also \( m \subseteq n \cup \set{ n } = \op{succ}(n) \). Thus, \( \op{succ}(n) \) is also transitive.

  We have shown that \( T \) is inductive. Therefore, \( \omega = T \) and every member of \( \omega \) is transitive.

  From now on, we will not be as explicit about the use of induction on \( \omega \).

  \SubProof{Proof that \( \omega \) is transitive} We will show that for all members \( n \) of \( \omega \) we have \( n \subseteq \omega \).

  The case \( n = \varnothing \) is again trivial.

  Now suppose that \( n \subseteq \omega \) and let \( m \in \op{succ}(n) \). If \( m = n \), clearly \( m \subseteq \omega \). If \( m \in n \), then \( m \subseteq n \) and, since \( n \subseteq \omega \), we have \( m \subseteq \omega \) by transitivity of \( \subseteq \).

  Therefore, \( \omega \) is transitive.
\end{proof}

\begin{remark}\label{rem:transitive_model_of_set_theory}
  As discussed in \fullref{def:set_builder_notation}, within the \hyperref[def:naive_set_theory/unrestricted_comprehension]{axiom schema of unrestricted comprehension} it may happen that \( U \subseteq V \) is not a set within the object logic.

  But there is a bigger problem that may happen even for \hyperref[rem:standard_model_of_set_theory]{standard models}. If \( A \in V \) and \( x \in A \) (in the metatheory), it is possible that \( x \) is not in \( V \). Therefore, if we have shown that \( A \) is a set within the object logic, it is possible that its members within the metatheory are not members in the object logic. On other words, it is possible for set membership itself to be incompatible between the metatheory and object logic.

  If \( V \) is a transitive set, however, we would not have such a problem. That is, if we construct a set \( A \) in the metatheory and show that it belongs to some set \( B \) in the object logic, then \( A \) itself would also be a set in the object logic.

  For this reason, it is very important to consider only transitive models of set theory.
\end{remark}

\begin{lemma}\label{thm:members_of_omega_do_not_contain_themselves}
  No element of \( \omega \) is a member of itself.
\end{lemma}
\begin{proof}
  We will again use \fullref{thm:omega_induction}. By definition, \( \omega \not\in \omega \). Now suppose that \( n \not\in n \) for some \( n \in \omega \).

  Aiming at a contradiction, suppose that \( \op{succ}(n) \in \op{succ}(n) \). The assumption that \( n = \op{succ}(n) = n \cup \set{ n } \) implies that \( n = \set{ n } \). The assumption that \( \op{succ}(n) \in n \) implies that \( n \in n \) since \( \op{succ}(n) \) is transitive by \fullref{thm:omega_is_transitive}. In both cases we have \( n \in n \), which contradicts our inductive hypothesis. This contradiction shows that \( \op{succ}(n) \not\in \op{succ}(n) \).

  \Fullref{thm:omega_induction} allows us to conclude that no member of \( \omega \) contains itself.
\end{proof}

\begin{theorem}\label{thm:omega_is_model_of_pa_without_operations}
  The \hyperref[thm:smallest_inductive_set_existence]{smallest inductive set} \( \omega \) satisfies the axioms \eqref{eq:def:peano_arithmetic/PA1}-\eqref{eq:def:peano_arithmetic/PA3} from \hyperref[def:peano_arithmetic]{Peano arithmetic} with the following interpretation:
  \begin{thmenum}
    \thmitem{thm:omega_is_model_of_pa_without_operations/zero} \hyperref[def:peano_arithmetic/zero]{Zero} is interpreted as \( \varnothing \).

    \thmitem{thm:omega_is_model_of_pa_without_operations/succ} The \hyperref[def:peano_arithmetic/succ]{successor} operation \( s \) is interpreted as \( \op{succ} \).
  \end{thmenum}

  We will generalize this theorem to \fullref{thm:omega_is_model_of_pa} after we are able to define the arithmetic operations in \( \omega \).
\end{theorem}
\begin{proof}
  \SubProofOf{eq:def:peano_arithmetic/PA1} Let \( n, m \in \omega \) and suppose that \( \op{succ}(n) = \op{succ}(m) \). If \( n = m \), there is nothing to prove.

  Suppose that \( n \neq m \). Thus, since
  \begin{equation*}
    n \cup \set{ n } = \op{succ}(n) = \op{succ}(m) = m \cup \set{ m },
  \end{equation*}
  we have both \( n \in m \) and \( m \in n \).

  \Fullref{thm:omega_is_transitive} implies that \( n \) is transitive and hence \( n \in n \), which contradicts \fullref{thm:members_of_omega_do_not_contain_themselves}.

  The obtained contradiction shows that \( n = m \).

  \SubProofOf{eq:def:peano_arithmetic/PA2} Suppose that \( \varnothing \) has a predecessor \( n \in \omega \). Then
  \begin{equation*}
    \varnothing = \op{succ}(n) = n \cup \set{ n },
  \end{equation*}
  which implies that \( n \in \varnothing \). But this contradicts the \hyperref[def:empty_set]{definition of \( \varnothing \)}.

  Therefore, \( \varnothing \) has no predecessor.

  \SubProofOf{eq:def:peano_arithmetic/PA3} It follows from \fullref{thm:omega_induction} that \eqref{eq:thm:omega_induction} is a theorem of \logic{ZF}. Let \( \mscrV = (V, I) \) be our ambient \hyperref[rem:standard_model_of_set_theory]{standard} \hyperref[rem:transitive_model_of_set_theory]{transitive} model of \logic{ZFC}.

  Fix any variable assignment \( v: \boldop{Var} \to V \). As in the proof of \fullref{thm:omega_induction}, we consider the modified assignment \( v_{\sigma \mapsto \omega} \) that \enquote{eliminates} the outer existential quantifier in \eqref{eq:thm:omega_induction}.

  To show that \eqref{eq:thm:omega_induction} really corresponds to \eqref{eq:def:peano_arithmetic/PA3} (and hence that \( \omega \) satisfies \eqref{eq:def:peano_arithmetic/PA3}), fix some formula \( \varphi \) of \hyperref[def:peano_arithmetic]{Peano arithmetic} (\hi{not \logic{ZFC}!}) and suppose that \( \xi, \zeta_1, \ldots, \zeta_n \) are all of its free variables. Fix also some parameter values \( u_1, \ldots, u_n \in \omega \) and, as in \fullref{def:set_builder_notation}, define the set
  \begin{equation*}
    A \coloneqq \set{ x \in \omega \given \varphi\Bracks{x, u_1, \ldots, x_n} }.
  \end{equation*}

  Since \eqref{eq:thm:omega_induction} is satisfied by \( v \), the inner formula in \eqref{eq:thm:omega_induction} (without the quantifiers over \( \sigma \) and \( \tau \)) is satisfied by \( v_{\sigma \mapsto \omega, \tau \mapsto A} \).

  Since our choice of parameters \( u_1, \ldots, u_n \) was arbitrary, we can conclude that the universal closure \eqref{eq:def:peano_arithmetic/PA3_quantified} of \eqref{eq:def:peano_arithmetic/PA3} is satisfied by \( \omega \) for every formula \( \varphi \) of \logic{PA}.
\end{proof}

\begin{remark}\label{rem:set_theory_natural_numbers_without_operations}
  Due to \fullref{thm:omega_is_model_of_pa_without_operations}, we will henceforth identify the \hyperref[thm:smallest_inductive_set_existence]{smallest inductive set} \( \omega \) with the set \( \BbbN \) of \hyperref[def:natural_numbers]{natural numbers}.

  We are not yet able to add or multiply natural numbers, nor rely on their well-foundedness, however for all other purposes we are able to utilize them.

  Since the ordering in \fullref{def:natural_numbers_ordering} is defined via addition, we must define some other ordering. Luckily, as we shall see in \fullref{subsec:ordinals}, \( n < m \) corresponds to \( n \in m \). In particular, the members of \( m \) are ordered.

  As a consequence, every natural number \( n \) equals the set of all smaller natural numbers by the \hyperref[def:naive_set_theory/extensionality]{axiom of extensionality}. It is conventional to write \( \set{ 0, 1, \ldots, n } \) rather than \( \set{ m \given m \in n } \). The former notation will be fully justified in \fullref{subsec:ordinals}.

  This is useful, for example, in \fullref{def:sequence}.
\end{remark}

  \subsection{Relations}\label{subsec:relations}

\begin{definition}\label{def:cartesian_product}\mcite[def. 1.23]{OpenLogicFull}
  We are now in a vicious cycle where we need binary Cartesian products in order to define arbitrary Cartesian products. We will do this as quickly as possible, without introducing relations and functions. The latter two will be discussed in detail in \fullref{subsec:relations} and \fullref{subsec:functions}, respectively.

  \begin{thmenum}
    \thmitem{def:cartesian_product/kuratowski_pair} The \term{Kuratowski pair} or simply \term{ordered pair} \( \braket{ x, y } \) of the sets \( x \) and \( y \) is defined as
    \begin{equation*}
      \braket{ x, y } \coloneqq \set{ \set{ x }, \set{ x, y } }.
    \end{equation*}

    This is a simple and widespread definition that encodes the order of \( x \) and \( y \), unlike the set \( \set{ x, y } \) for example.

    We will later use the notation \( (x, y) \), but until \fullref{rem:kuratowski_pairs_and_tuples}, we want to distinguish between Kuratowski pairs and \( 2 \)-tuples.

    We will use the following \hyperref[rem:predicate_formula]{predicate formula} in \( \ref{eq:def:function/predicate}[\rho, \tau, \sigma] \):
    \begin{equation*}\taglabel[\op{IsPair}]{eq:def:cartesian_product/kuratowski_pair_predicate}
      \ref{eq:def:cartesian_product/kuratowski_pair_predicate}[\rho, \tau, \sigma] \coloneqq \qforall \xi \parens[\Bigg]{ \xi \in \rho \leftrightarrow \parens[\Big]{ \parens[\Big]{ \qforall {\eta \in \xi} \eta \doteq \tau } \vee \parens[\Big]{ \qforall {\eta \in \xi} (\eta \doteq \tau \vee \eta \doteq \sigma) } } }
    \end{equation*}

    \thmitem{def:cartesian_product/indexed_family} A set \( i \) of Kuratowski pairs is called an \term{indexed family} if whenever \( \braket{ k, A } \in i \) and \( \braket{ k, B } \in i \), we have \( A = B \). It is conventional to denote this unique set corresponding to \( k \) as \( A_k \) without an explicit reference to \( i \).

    The \term{index set} of the family is
    \begin{equation*}
      \mscrK \coloneqq \set{ k \given \qexists A \braket{ k, A } \in i }.
    \end{equation*}

    The family itself is then denoted as
    \begin{equation*}
      \seq{ A_k }_{k \in \mscrK}.
    \end{equation*}

    If \( A_k \in \mscrA \) for every \( k \in \mscrK \), we sometimes write
    \begin{equation*}
      \seq{ A_k }_{k \in \mscrK} \subseteq \mscrA,
    \end{equation*}
    although the latter is an embedding rather than set inclusion.

    \thmitem{def:cartesian_product/tuple} A \term{tuple} from the indexed family \( \seq{ A_k }_{k \in \mscrK} \) is another indexed family \( \seq{ x_k }_{k \in \mscrK} \) with the same index set satisfying the condition that for every \( k \in \mscrK \), the value \( x_k \) belongs to \( A_k \).

    We will see later that this is precisely a \hyperref[def:choice_function]{choice function}.

    \thmitem{def:cartesian_product/product} The \term{Cartesian product} of an indexed family \( \seq{ A_k }_{k \in \mscrK} \) is the set of all tuples from this family. We denote the Cartesian product by
    \begin{equation*}
      \prod_{k \in \mscrK} A_k.
    \end{equation*}

    With the availability of \hyperref[def:function]{functions} in \fullref{subsec:functions}, we can define the Cartesian product as the set
    \begin{equation*}
      \prod_{k \in \mscrK} A_k = \set{ f: A \to \bigcup_{k \in \mscrK} A_k \given \qforall {k \in \mscrK} k \in A_k }.
    \end{equation*}
  \end{thmenum}
\end{definition}

\begin{definition}\label{def:sequence}
  Families indexed by \( \omega \) are called \term{infinite sequences} or simply \term{sequences}. We will use several notations, depending on the context.
  \begin{itemize}
    \item \( \seq{ A_k }_{k \in \BbbN} \), which is the conventional notation for indexed families.
    \item \( \seq{ A_k }_{k=0}^\infty \), which easily extends to cases such as \( \seq{ A_k }_{k=m}^n \) when the index set is \( \set{ m, \ldots, n } \).
    \item \( (A_0, A_1, \ldots) \), which is used when explicitly enumerating members of the sequence.
  \end{itemize}

  It is conventional to write a family \( \seq{ A_k }_{k \in n} \) indexed by a natural number \( n \) using the notation \( (A_1, \cdots, A_n) \), with or without the outer parentheses. Families indexed by natural numbers are called \term{finite sequences}, although \enquote{tuple} is often used.

  For \( n = 2 \), finite sequences are called \term{pairs}, for \( n = 3 \) --- \term{triples} and for \( n = 4 \) --- \term{quadruples}.

  We say that \( A_{k_1}, A_{k_2}, \ldots \) is a \term{subsequence} of \( A_1, A_2, \ldots \) if \( k_1 < k_2 < \ldots \), i.e. if the sequence of indices is monotone in the sense of \eqref{eq:def:partially_ordered_set/homomorphism/sequence}.
\end{definition}

\begin{remark}\label{rem:kuratowski_pairs_and_tuples}
  Note that the tuple \( (A, B) \)
  \begin{equation*}
    (A, B) = \set[\Bigg]{ \set[\Big]{ \set{ 0 }, \set{ 0, A } }, \set[\Big]{ \set{ 1 }, \set{ 1, B } } }
  \end{equation*}
  is formally different from the \hyperref[def:cartesian_product/kuratowski_pair]{Kuratowski pair}
  \begin{equation*}
    \braket{ A, B } = \set[\Big]{ \set{ A }, \set{ A, B } }.
  \end{equation*}

  This is one reason we hurried to define general Cartesian products --- we wanted to avoid working with tuples defined in terms of Kuratowski pairs. We even introduced a special notation for them, just so we can avoid any confusion. Nevertheless, it is conventional to conflate Kuratowski pairs with \( \set{ 0, 1 } \)-indexed tuples.

  We also conflate the tuples \( (A, (B, C)) \), \( ((A, B), C) \) and \( (A, B, C) \).
\end{remark}

\begin{definition}\label{def:relation}
  Let \( A_1, \ldots, A_n \) be a \hyperref[def:sequence]{finite sequence} of sets and let
  \begin{equation*}
    R \subseteq A_1 \times \cdots \times A_n
  \end{equation*}
  be a subset of their \hyperref[def:cartesian_product/product]{Cartesian product}.

  The sequence \( (R, A_1, \ldots, A_n) \) is called an \( n \)-ary \term{relation}. We say that the tuple \( (x_1, \ldots, x_n) \in A_1 \times \cdots \times A_n \) is related with respect to \( R \) if \( (x_1, \ldots, x_n) \in R \).

  Relations are the semantical counterpart to \hyperref[def:first_order_structure/interpretation/predicate]{first-order predicates} and are equivalent to Boolean-valued functions, as discussed in \fullref{rem:boolean_valued_functions_and_predicates}.

  We generalize only the following notions from \hyperref[def:binary_relation]{binary relations}:
  \begin{thmenum}[series=def:relation]
    \thmitem{def:relation/graph} The set \( R \) of tuples is called the \term{graph} of the relation. In case the sequence \( A_1, \ldots, A_n \) is clear from the context, we can identify the relation \( (R, A_1, \ldots, A_n) \) with its graph \( R \). We occasionally use the notation \( \gph(R) \) for explicitly denoting the graph.

    \thmitem{def:relation/signature} The \term{signature} of the relation is the sequence \( (A_1, \ldots, A_n) \). Obviously this definition only makes sense if we know what the signature is, either from the context or from the definition of the relation as the sequence \( (R, A_1, \ldots, A_n) \) rather than only via its graph \( R \).

    As a matter of fact, it is common to ignore the signature when defining relations --- see e.g. \cite[7]{Kelley1975} or \cite[def. 2.1]{OpenLogicFull}. If we do identify a relation only with its graph, however some notions like \hyperref[def:multi_valued_function/range]{ranges} and \hyperref[def:multi_valued_function/image]{images} coincide despite being different and other notions like \hyperref[def:function_invertibility/surjective]{function surjectivity} make no sense.

    Furthermore, two relations whose graphs are equal may have different signatures, which further highlights how important it is to distinguish between a relation and its graph.

    \thmitem{def:relation/arity} For some small values of \( n \), \( n \)-ary relations have established names:
    \begin{itemize}
      \item \term{Nullary} if \( n = 0 \).
      \item \term{Unary} if \( n = 1 \).
      \item \term{Binary} if \( n = 2 \).
      \item \term{Ternary} if \( n = 3 \).
    \end{itemize}

    This is not to be confused with \hyperref[def:multi_valued_function/arity]{function arity} --- functions are always binary relations.

    \thmitem{def:relation/single_set} If all \( A_k \) for \( k = 1, \ldots, n \) are equal to the set \( A \), we say that \( R \subseteq A^n \) is a relation \hi{on \( A \)}.
  \end{thmenum}
\end{definition}

\begin{definition}\label{def:binary_relation}
  An important special case of \hyperref[def:relation]{relations} are \term{binary relations}. Given two sets \( A \) and \( B \), a binary relation between them is a triple \( (R, A, B) \).

  In addition to the terminology for \fullref{def:relation}, we also introduce the following terms:
  \begin{thmenum}[series=def:binary_relation]
    \thmitem{def:binary_relation/empty} The relation is \term{empty} if its graph is the empty set, i.e. if no two elements are related.

    It is important to highlight that the graphs of all empty relations are equal, but two empty relations are only equal if their signatures are.

    \thmitem{def:binary_relation/converse} The \term{converse relation} of \( R \) is
    \begin{equation*}
      R^{-1} \coloneqq \set{ (y, x) \given (x, y) \in R }.
    \end{equation*}

    \thmitem{def:binary_relation/restriction} If \( A = B \), the \term{restriction} of \( R \) to \( X \subseteq A \) is the binary relation \( (R\restr_{X}, X, X) \) is
    \begin{equation*}
      R\restr_{X} \coloneqq R \cap (X \times X) = \set{ (x, y) \in R \given x \in X \T{and} y \in X }.
    \end{equation*}

    We say that \( R \) is an \term{extension} of \( R\restr_{X} \).

    \thmitem{def:binary_relation/diagonal} A special relation is the \term{diagonal relation} on a set \( A \):
    \begin{equation*}
      \increment_A \coloneqq \set{ (x, x) \given x \in A }.
    \end{equation*}

    \thmitem{def:binary_relation/composition} Given two binary relations \( R \subseteq A \times B \) and \( T \subseteq B \times C \), we define their composition as
    \begin{equation*}
      T \bincirc R \coloneqq \set*{ (x, z) \in A \times C \given* \qexists {y \in B} \parens[\Big]{ (x, y) \in R \T{and} (y, z) \in T } }.
    \end{equation*}
  \end{thmenum}

  Whenever \( A = B \) and \( R \) is simply a binary relation over \( A \), the following are commonly used conditions that are often as axioms to some theory:
  \begin{thmenum}[resume=def:binary_relation]
    \thmitem{def:binary_relation/reflexive} \( R \) is \term{reflexive} if \( \Delta_A \subseteq R \), i.e. if every element of \( A \) is related with itself.

    The following formula is used as an axiom for \hyperref[def:partially_ordered_set/nonstrict]{nonstrict partial orders} and \hyperref[def:entourage]{entourages}:
    \begin{equation}\label{eq:def:binary_relation/reflexive}
      \qforall \xi (\xi \mathbin{R} \xi).
    \end{equation}

    Note that we use \fullref{rem:first_order_formula_conventions/infix} notation in the latter case. Using either infix or prefix notation is actually a necessity since we do not actually have a concept of an ordered tuple in general (not set-based) first-order theories --- see \fullref{rem:first_order_formula_conventions/infix}.

    \thmitem{def:binary_relation/irreflexive} \( R \) is \term{irreflexive} if \( \Delta_A \cap R = \varnothing \), i.e. if no element of \( A \) is related with itself.

    The following formula is used as an axiom for \hyperref[def:partially_ordered_set/strict]{strict partial orders}:
    \begin{equation}\label{eq:def:binary_relation/irreflexive}
      \neg \qexists \xi (\xi \mathbin{R} \xi).
    \end{equation}

    \thmitem{def:binary_relation/symmetric} \( R \) is \term{symmetric} if \( R = R^{-1} \).

    The following formula is used as an axiom for \hyperref[def:equivalence_relation]{equivalence relations}, \hyperref[def:undirected_multigraph]{undirected graphs} and \hyperref[def:entourage]{entourages}:
    \begin{equation}\label{eq:def:binary_relation/symmetric}
      \xi \mathbin{R} \eta \rightarrow \eta \mathbin{R} \xi.
    \end{equation}

    \thmitem{def:binary_relation/antisymmetric} \( R \) is \term{antisymmetric} if \( R \cap R^{-1} = \Delta_X \).

    The following formula is used as an axiom for \hyperref[def:partially_ordered_set]{partial orders}:
    \begin{equation}\label{eq:def:binary_relation/antisymmetric}
      (\xi \mathbin{R} \eta \wedge \eta \mathbin{R} \xi) \rightarrow \xi \doteq \eta.
    \end{equation}

    \thmitem{def:binary_relation/transitive} \( R \) is \term{transitive} if \( R = R \bincirc R \).

    The following formula is used as an axiom for \hyperref[def:partially_ordered_set]{preorders}:
    \begin{equation}\label{eq:def:binary_relation/transitive}
      (\xi \mathbin{R} \eta \wedge \eta \mathbin{R} \zeta) \rightarrow \xi \mathbin{R} \zeta.
    \end{equation}

    \thmitem{def:binary_relation/total} \( R \) is \term{total} if any two member of \( A \) are related.

    The following formula is used as an axiom for \hyperref[def:totally_ordered_set]{nonstrict total orders}:
    \begin{equation}\label{eq:def:binary_relation/total}
      \qforall \xi \qforall \eta (\xi \mathbin{R} \eta \vee \eta \mathbin{R} \xi).
    \end{equation}

    Total relations are also called \term{connected}.

    \thmitem{def:binary_relation/trichotomic} \( R \) is \term{trichotomic} if every two elements of \( A \) are either related or equal.

    The following formula is used as an axiom for \hyperref[def:totally_ordered_set]{strict total orders}:
    \begin{equation}\label{eq:def:binary_relation/trichotomic}
      \qforall \xi \qforall \eta (\xi \mathbin{R} \eta \vee \eta \mathbin{R} \xi \vee \eta = \xi).
    \end{equation}
  \end{thmenum}
\end{definition}

\begin{example}\label{ex:def:binary_relation}
  \hyperref[def:binary_relation]{Binary relations} are used in vastly different contexts:
  \begin{itemize}
    \item \hyperref[def:function]{Functions} are special binary relations.
    \item \hyperref[sec:order_theory]{Orders} are also special binary relations.
    \item \hyperref[def:quiver/simple]{Directed graphs} are commonly defined as binary relations.
    \item \hyperref[def:entourage]{Entourages} are binary relations in \hyperref[def:uniform_space]{uniform spaces}.
    \item Relations are equivalent to Boolean-valued functions as shown in \fullref{rem:boolean_valued_functions_and_predicates}, and are often used for defining semantics of predicate symbols in \hyperref[subsec:first_order_logic]{first-order logic}.
  \end{itemize}
\end{example}

\begin{definition}\label{def:equivalence_relation}
  A binary relation on the set \( A \) that is \hyperref[def:binary_relation/reflexive]{reflexive}, \hyperref[def:binary_relation/symmetric]{symmetric} and \hyperref[def:binary_relation/transitive]{transitive} is called an \term{equivalence relation}. In other words, an equivalence relation is a symmetric \hyperref[def:preordered_set]{preorder}.

  We often denote equivalence relations via the symbol \( \sim \).

  \begin{thmenum}
    \thmitem{def:equivalence_relation/coset} The \term{equivalence class} of \( x \in A \), also called its \term{coset}, is the set
    \begin{equation*}
      [x] \coloneqq \set{ y \in A \given x \sim y }
    \end{equation*}
    of all elements of \( A \) that are related to \( x \).

    \thmitem{def:equivalence_relation/quotient} The \term{quotient set} of \( A \) by \( \sim \) is the set
    \begin{equation*}
      A / {\sim} \ \coloneqq \set{ [x] \given x \in A }.
    \end{equation*}

     If we have an easy way to choose a representative from each coset, then \( A / {\sim} \) may be regarded as a subset of \( A \). In general, this is not a subset relation by only an \hyperref[def:first_order_homomorphism_invertibility/embedding]{embedding}.

    \thmitem{def:equivalence_relation/projection} Using forward references to \fullref{subsec:functions}, we define the \term{canonical projection} as the function
    \begin{equation*}
      \begin{aligned}
        &\pi: A \to A / {\sim}  \\
        &\pi(x) \coloneqq [x].
      \end{aligned}
    \end{equation*}

    If we have a fixed \hyperref[def:choice_function]{choice function} \( c: A / {\sim} \to A \), we also define the canonical embedding
    \begin{equation*}
      \begin{aligned}
        &\iota: A / {\sim} \to A \\
        &\iota([x]) \coloneqq c(x).
      \end{aligned}
    \end{equation*}

    We sometimes have an obvious choice function, for example in \fullref{thm:representatives_in_univariate_polynomial_quotient_set}. In this case, the canonical projection may be regarded as a function from \( A \) to the subset \( c(A / {\sim}) \) of \( A \). Otherwise, the function \( \pi \) can be regarded as a \hyperref[def:multi_valued_function]{multi-valued function} from \( A \) to \( A \).
  \end{thmenum}
\end{definition}

\begin{remark}\label{rem:congruence_modulo_relation}
  If \( x \sim y \) for some \hyperref[def:equivalence_relation]{equivalence relation}, we say that they are \term{congruent modulo} \( \sim \). This concept specializes to congruence modulo normal subgroups defined in \fullref{def:congruence_modulo_normal_subgroup}.
\end{remark}

\begin{proposition}\label{thm:equality_is_smallest_equivalence_relation}
  The equality \hyperref[def:binary_relation]{relation} \( = \) is the intersection of all equivalence relations.
\end{proposition}
\begin{proof}
  It is equivalent to the \hyperref[def:binary_relation/diagonal]{diagonal relation} \( \Delta_X \). It is the smallest reflexive (resp. symmetric and transitive) relation on \( A \), i.e. the intersection of all reflexive (resp. symmetric and transitive) relations.
\end{proof}

\begin{definition}\label{def:set_partition}
  Let \( A \) be a set. A \term{cover} of \( A \) is a \hyperref[rem:family_of_sets]{family} \( \mscrA \subseteq \pow(A) \) of nonempty subsets of \( A \) such that \( A = \bigcup \mscrA \). We sometimes use the term more loosely and say that an arbitrary family of sets \( \mscrA \) is a cover of \( A \) if \( A \subseteq \bigcup \mscrA \). The two definitions are identical if we intersect each set in \( \mscrA \) with \( A \) and exclude the empty sets.

  A \term{partition} of \( A \) is a pairwise \hyperref[def:subset]{disjoint} cover. In other words, the cover \( \mscrA \) is a partition if and only if each element of \( A \) belong to exactly one set in \( \mscrA \).
\end{definition}

\begin{proposition}\label{thm:equivalence_partition}
  Fix a set \( A \). Let \( {\sim} \) be a binary relation of \( A \). The following are equivalent:
  \begin{thmenum}
    \thmitem{thm:equivalence_partition/equivalence} \( {\sim} \) is an \hyperref[def:equivalence_relation]{equivalence relation}.

    \thmitem{thm:equivalence_partition/partition} There exists a \hyperref[def:set_partition]{partition} \( \mscrA \) of \( A \) such that \( x \sim y \) if and only if they belong to the same set in the partition \( \mscrA \).
  \end{thmenum}
\end{proposition}
\begin{proof}
  \ImplicationSubProof{thm:equivalence_partition/equivalence}{thm:equivalence_partition/partition} Let \( {\sim} \) be an equivalence relation on \( A \). The quotient set \( A / {\sim} \) is a partition. Indeed:
  \begin{itemize}
    \item Every element \( x \in A \) belongs exactly one equivalence class \( [x] \) by definition.

    \item The equivalence classes are disjoint. Indeed, assume the contrary. Then there exist \( x \) and \( y \) such that \( [x] \cap [y] \neq \varnothing \) and yet \( x \not\sim y \).

    Let \( z \in [x] \cap [y] \). Then \( z \sim x \) and \( z \sim y \), thus from transitivity of \( {\sim} \) we have \( x \sim z \sim y \) and hence \( x \sim y \), which contradicts our assumption that \( x \not\sim y \).

    Hence, either \( [x] = [x] \) or \( [x] \cap [y] = \varnothing \). That is, different equivalence classes are disjoint.
  \end{itemize}

  \ImplicationSubProof{thm:equivalence_partition/partition}{thm:equivalence_partition/equivalence} Let \( \mscrA \) be a partition of \( A \) such that \( x \sim y \) if and only if they both belong to the same set in \( \mscrA \).

  Given \( x \in A \), denote by \( A_x \) the set in \( \mscrA \) which contains \( x \). The family \( \seq{ A_x }_{x \in A} \) is well-defined since \( \mscrA \) is a partition, which means that \( x \) belongs to exactly one set in \( \mscrA \).

  \SubProofOf*[def:binary_relation/reflexive]{reflexivity} Clearly \( A_x = A_x \), hence \( x \sim x \).

  \SubProofOf*[def:binary_relation/symmetric]{symmetry} If \( x \sim y \), then \( A_x = A_y \), which implies \( A_y = A_x \) and thus \( x \sim y \).

  \SubProofOf*[def:binary_relation/transitive]{transitivity} If \( x \sim y \) and \( y \in z \), then \( A_x = A_y = A_z \) and thus \( x \sim z \).
\end{proof}

\begin{definition}\label{def:relation_closures}
  Let \( R \subseteq A^2 \) be a binary relation on the set \( A \). We define several \hyperref[def:abstract_closure_operator]{closure operators}:
  \begin{thmenum}
    \thmitem{def:relation_closures/reflexive} The \term{reflexive closure} of \( R \) is
    \begin{equation*}
      \cl^R(R) \coloneqq R \cup \Delta_X.
    \end{equation*}

    \thmitem{def:relation_closures/symmetric} The \term{symmetric closure} of \( R \) is
    \begin{equation*}
      \cl^S(R) \coloneqq R \cup R^{-1}.
    \end{equation*}

    \thmitem{def:relation_closures/transitive} The \term{transitive closure} \( \cl^T(R) \) of \( R \) is
    \begin{equation*}
      \cl^T(R) \coloneqq \bigcup \set{ R^k \given k = 1, 2, \ldots },
    \end{equation*}
    where \( R^k \) is iterated \hyperref[def:binary_relation/composition]{composition} of \( R \).

    Note that this is very different from the transitive closure of a set defined in \fullref{def:transitive_closure_of_a_set}.

    A \term{transitive reduction} of \( R \) is a \hyperref[def:partially_ordered_set_extremal_points/maximal_and_minimal_element]{minimal} relation \( Q \subseteq R \) such that \( \cl^T(Q) = \cl^T(R) \). If there exists a smallest such relation, it is the unique transitive reduction, and we denote it by \( \red^T(R) \).
  \end{thmenum}
\end{definition}

\begin{proposition}\label{thm:def:relation_closures}
  \hyperref[def:relation_closures]{Binary relation closures} have the following basic properties:
  \begin{thmenum}
    \thmitem{thm:def:relation_closures/reflexive_relation} The symmetric and transitive closures of a reflexive relation are symmetric.
    \thmitem{thm:def:relation_closures/symmetric_relation} The reflexive and transitive closures of a symmetric relation are symmetric.
    \thmitem{thm:def:relation_closures/transitive_relation} The reflexive closure of a transitive relation is transitive. The symmetric closure of a transitive relation may not be transitive --- see \fullref{ex:thm:def:relation_closures/symmetric_and_transitive}.

    \thmitem{thm:def:relation_closures/reflexive_and_symmetric} The reflexive and symmetric closures commute:
    \begin{equation}\label{eq:thm:def:relation_closures/reflexive_and_symmetric}
      \cl^S \cl^R(R) = \cl^R \cl^S(R).
    \end{equation}

    \thmitem{thm:def:relation_closures/reflexive_and_transitive} The transitive and reflexive closures commute:
    \begin{equation}\label{eq:thm:def:relation_closures/reflexive_and_transitive}
      \cl^R \cl^T(R) = \cl^T \cl^R(R).
    \end{equation}

    \thmitem{thm:def:relation_closures/symmetric_and_transitive} For the transitive and symmetric closures of \( R \) we have
    \begin{equation}\label{eq:thm:def:relation_closures/symmetric_and_transitive}
      \cl^S \cl^T(R) \subseteq \cl^T \cl^S(R).
    \end{equation}

    The converse holds if \( R \) is symmetric but not in general --- see \fullref{ex:thm:def:relation_closures/symmetric_and_transitive}.
  \end{thmenum}
\end{proposition}
\begin{proof}
  \SubProofOf{thm:def:relation_closures/reflexive_relation} Trivial.
  \SubProofOf{thm:def:relation_closures/symmetric_relation} Trivial.
  \SubProofOf{thm:def:relation_closures/transitive_relation} Trivial.
  \SubProofOf{thm:def:relation_closures/reflexive_and_symmetric} Trivial.
  \SubProofOf{thm:def:relation_closures/reflexive_and_transitive} The reflexive closure only adds pairs of the form \( (x, x) \). Thus, if \( (x, y) \in \cl^T(\cl^R(R)) \) for \( x \neq y \), then \( (x, y) \in \cl^T(R) \subseteq \cl^R(\cl^T(R)) \).

  Conversely, if \( (x, y) \in \cl^R(\cl^T(R)) \) for \( x \neq y \), then \( (x, y) \in \cl^T(R) \subseteq \cl^R(\cl^T(R)) \).

  \SubProofOf{thm:def:relation_closures/symmetric_and_transitive} If \( (x, y) \in \cl^S(\cl^T(R)) \), then we have the following possibilities:
  \begin{itemize}
    \item If \( (x, y) \in R \), obviously \( (x, y) \in \cl^T(\cl^S(R)) \).
    \item If \( (x, y) \in \cl^T(R) \setminus R \), then there exists some natural number \( k > 1 \) such that \( (x, y) \in R^k \).

    Since \( R^k \subseteq [\cl^S(R)]^k \), as can be shown by induction, we have \( (x, y) \in [\cl^S(R)]^k \). We thus conclude that \( (x, y) \in \cl^T(\cl^S(R)) \).

    \item Finally, if \( (x, y) \not\in \cl^T(R) \), then \( (y, x) \in \cl^T(R) \). As in the previous step, we can show that \( (y, x) \in \cl^T(\cl^S(R))) \). The latter set is symmetric, hence \( (x, y) \in \cl^T(\cl^S(R)) \).
  \end{itemize}

  Since \( (x, y) \) was arbitrary, we conclude that \eqref{eq:thm:def:relation_closures/symmetric_and_transitive} holds.

  Furthermore, if \( R \) is symmetric, then
  \begin{equation*}
    \cl^T \cl^S(R)
    =
    \cl^T(R)
    \reloset {\ref{thm:def:relation_closures/reflexive_relation}} =
    \cl^S \cl^T(R).
  \end{equation*}
\end{proof}

\begin{example}\label{ex:thm:def:relation_closures/symmetric_and_transitive}
  We will show that symmetric and transitive closures of relations do not commute. This is also a consequence of the difference between \hyperref[def:quiver_connectedness/weak]{weak} and \hyperref[def:quiver_connectedness/strong]{strong} connectedness of quivers.

  Consider the set \( A = \set{ a, b, c } \) and the relation \( R = \set{ (a, b), (c, b) } \).

  It should be noted that \( R \) is \hyperref[def:binary_relation/transitive]{transitive}. Thus,
  \begin{equation*}
    \cl^S(\cl^T(R)) = \cl^S(R) = R \cup \set{ (b, a), (b, c) }.
  \end{equation*}

  The latter set is not transitive because \( (a, b) \) and \( (b, a) \) both belong to \( \cl^S(R) \) and neither \( (a, a) \) nor \( (b, b) \) do not.

  This shows that the converse of \eqref{eq:thm:def:relation_closures/symmetric_and_transitive} does not hold in general.
\end{example}

\begin{proposition}\label{thm:equivalence_closure}
  The \hyperref[def:relation_closures/reflexive]{reflexive}, \hyperref[def:relation_closures/symmetric]{symmetric} and \hyperref[def:relation_closures/transitive]{transitive} closure \( \cl^T \cl^S \cl^R (R) \) of any relation \( R \) is an \hyperref[def:equivalence_relation]{equivalence relation}.

  This holds for any permutation of the closures as long as \( \cl^T \) is applied \hi{after} \( \cl^S \). This latter restriction is due to \fullref{thm:def:relation_closures/symmetric_and_transitive}.
\end{proposition}
\begin{proof}
  Let \( R \subseteq A \times B \) be an arbitrary relation. By \fullref{thm:def:relation_closures/reflexive_relation}, \( \cl^S \cl^R (R) \) is reflexive. It is also symmetric as the symmetric closure of \( \cl^R(R) \).

  Then the transitive closure \( \cl^T \cl^S \cl^R (R) \) is also symmetric and reflexive by \fullref{thm:def:relation_closures/reflexive_relation} and \fullref{thm:def:relation_closures/symmetric_relation}.

  Therefore, \( \cl^T \cl^S \cl^R (R) \) is an equivalence relation.
\end{proof}

  \subsection{Functions}\label{subsec:functions}

\begin{remark}\label{rem:function_definition}
  Consider the formula of first-order set theory
  \begin{equation}\label{eq:rem:function_definition}
    \qforall \xi \qexists \eta (\xi \in \eta)
  \end{equation}
  stating that every set belongs to some set.

  Fix an arbitrary set \( A \). We are free to choose any set, for which reason we call \( \xi \) an \term{independent variable}. Then we must find a set \( B \) of \( V \), \hi{dependent} on \( A \), such that \( A \in B \). Possible values for \( B \) are the singleton set \( \set{ A } \), as well as all its supersets, for example \( \pow(A) \). We call \( \eta \) a \term{dependent variable} because it depends on \( \xi \)\fnote{This terminology can be slightly inappropriate if \( \eta \) does not actually depend on \( \xi \), for example in the formula \( \qforall \xi \qexists \eta (\eta \syneq \eta) \)}.

  We want an abstraction that would capture such dependence. The approach that we took in \fullref{def:basic_set_operations/power_set} is to describe how to obtain the power set \( \pow(A) \) of any set. Thus, for each set \( A \), there exists a unique set \( \pow(A) \) whose members are the subsets of \( A \). We say that the \term[bg=изображение,ru=изображение]{mapping} \( \pow \) \term{maps} a set to its power set.

  What if we wanted to study properties of a mapping and relations between different mappings? In order to do this via set theory, we must represent a mapping as a set. Many authors like \incite[40]{Enderton1977Sets}, \incite[7]{Kelley1975} and \incite[1]{Engelking1989} propose using a \hyperref[def:binary_relation]{binary relation} \( R \), called the \term{graph} of the mapping, such that the sets \( A \) and \( B \) are related under \( R \) if \( B \) is the unique value corresponding to \( A \).

  In the case of the power set mapping, it would make sense for the graph to be a subset of \( V \times V \), where \( V \) is a set containing all sets. But if we want our definition to be compatible with \hyperref[def:zfc]{\logic{ZFC}}, then the existence of \( V \) would contradict \fullref{thm:zfc_existence_theorems/set_containing_all}.

  Instead, fix some set \( V \) closed under the power set operation\fnote{Such as \( V_\lambda \) for limit ordinals --- see \fullref{thm:def:cumulative_hierarchy/power_set}.} and call it our \term{domain}. Then we can define the relation\fnote{The \hyperref[def:zfc/replacement]{axiom of replacement} ensures that the set \( \mscrP \) exists.}
  \begin{equation*}
    \mscrP \coloneqq \set[\Big]{ (A, \pow(A)) \given* A \in V }.
  \end{equation*}

  There is a nuance here that turns out to be important. We know that \( \mscrP \) maps members of the domain into members of some other set, the \term{codomain}. A natural candidate is the following set, which we will call the \term{image}:
  \begin{equation*}
    W \coloneqq \set[\Big]{ B \given* (A, B) \in \mscrP }.
  \end{equation*}

  We thus have the ordered triple \( (\mscrP, V, W) \). This definition captures the graph of the mapping, as well as its domain and codomain. We will call this triple a \term{function}. In this more narrow context, we will synonymously use \enquote{function}, \enquote{transformation}, \enquote{operator}, \enquote{map} and \enquote{mapping}.

  We have chosen \( V \) so that if \( A \in V \), then also \( \pow(A) \in V \), thus \( V \) is also a candidate for a codomain. The triple \( (\mscrP, V, V) \) is another function with the same domain and graph, yet with a different codomain.

  The difference between the two leads to different \hyperref[def:function/set_of_functions]{sets of functions}, either those from \( V \) to \( W \) or those from \( V \) to \( V \). Sets of functions are fundamental for \fullref{sec:functional_analysis} and \fullref{sec:category_theory}, as well as algebraic disciplines like \fullref{sec:group_theory}, \fullref{sec:ring_theory} and \fullref{sec:linear_algebra}.

  Another function from \( V \) to \( V \) that we discussed earlier relates \( \set{ A } \) to \( A \). Both of the aforementioned functions have different ranges, but we chose them to have the same domain and codomain, and thus belong to the set of functions from \( V \) to itself.

  Rather than focusing on only one solution to \eqref{eq:rem:function_definition}, we may instead define the relation
  \begin{equation*}
    \mscrR \coloneqq \set[\Big]{ (A, B) \given* A, B \in V \T{and} A \in B }.
  \end{equation*}

  Then \( A \) is related under \( \mscrR \) with both \( \set{ A } \) and \( \pow(A) \). This is technically not a graph of a function because we lack uniqueness. It is a generalization, which \incite[def. 1.3.1]{AubinFrankowska1990}, \incite[1]{DontchevRockafellar2014} and \incite[VII]{Phelps1993} call \term{set-valued maps}\fnote{Every mapping is technically set-valued, but the values of single-valued functions are treated as atoms, while the values of set-valued maps are treated as sets of atoms}.

  Prime examples of set-valued maps used in practice are the subdifferentials defined in \fullref{def:subdifferentials}, as well as solution mappings which provide sets of solutions to some equation that depends on a parameter.
\end{remark}

\begin{definition}\label{def:set_valued_map}\mcite[def. 1.3.1]{AubinFrankowska1990}
  A \term{set-valued map} from \( A \) to \( B \) is an ordered triple \( F = (R, A, B) \), where \( R \subseteq A \times B \) is a \hyperref[def:binary_relation]{binary relation}.

  We use the notation \( F: A \multto B \) instead of \( F = (R, A, B) \), where we call the string of symbols \( A \multto B \) the \term{signature} of the map.

  \begin{thmenum}[series=def:set_valued_map]
    \thmitem{def:set_valued_map/graph} We call the relation \( R \) the \term{graph}\fnote{The term \enquote{graph} in this context is unrelated to the different kinds of graphs defined in \fullref{subsec:graphs}.} of \( F \) and denote it by \( \gph(F) \).

    \thmitem{def:set_valued_map/value} We define the \term{value} of \( F \) at \( x \) as
    \begin{equation*}
      F(x) \coloneqq \set{ y \in B \given (x, y) \in \gph(F) }.
    \end{equation*}

    The notation \( F(x) \) refers to a member of \( B \) if \( x \) is a \hyperref[def:first_order_syntax/formula_bound_variables]{bound variable} and to the map \( F \) itself if \( x \) is a \hyperref[def:first_order_syntax/formula_free_variables]{free variable}.

    \incite[9]{Aluffi2009} also uses the terminology \enquote{\term{action} of \( F \) on \( x \)} or \enquote{\term{image} of \( F \) on \( x \)}.

    We say that \( F \) is \term{single-valued} at \( x \) if \( F(x) \) has exactly one member and \term{multivalued} if it has more than one member.

    \thmitem{def:set_valued_map/partial}\mcite[41]{Rosen1999} We say that \( F \) is \term{partial} if \( F(x) = \varnothing \) for some \( x \in A \) and \term{total} otherwise.

    \thmitem{def:set_valued_map/image} We also define the \term{image} of a set \( X \subseteq A \) under \( F \) as
    \begin{equation*}
      F[X] \coloneqq \bigcup \set{ F(x) \given x \in X }.
    \end{equation*}

    If \( X = A \), we call the set \( F[A] \) the image of \( F \) and denote it by \( \img(F) \).

    \thmitem{def:set_valued_map/domain} We define the \term{domain} of \( F \) as the set
    \begin{equation*}
      \dom(F) \coloneqq \set{ x \in A \given \qexists{F(x)} \neq \varnothing }.
    \end{equation*}

    \thmitem{def:set_valued_map/codomain} Contrastingly, we define the \term{codomain} \( \co\dom(F) \) of \( F \) as the entire set \( B \).

    The reason for the inconsistency with the term \enquote{domain} is due to the terminology coming from \hyperref[def:function]{single-valued functions}, where the domain is required to be the entirety of \( A \).

    \thmitem{def:set_valued_map/restriction} For each subset \( X \subseteq A \), we can define the \term{restriction} \( F_X: X \to B \) of \( F \) to \( X \) as the set-valued map such that
    \begin{equation*}
      \gph(F_X) \coloneqq \gph(F) \cap X \times B.
    \end{equation*}
  \end{thmenum}

  The following terminology is consistent with relations:
  \begin{thmenum}[resume=def:set_valued_map]
    \thmitem{def:set_valued_map/empty} We say that the function \( F: A \multto B \) is \term{empty} if \( \gph(F) = \varnothing \), i.e. if the graph is empty in the sense of \fullref{def:relation/empty}.

    \thmitem{def:set_valued_map/identity}\mcite[2]{Engelking1989} We define the \term{identity function} \( \id_A \) on the set \( A \) as the function corresponding to the identity relation \( \increment_A \) from \fullref{def:binary_relation/diagonal}.

    \thmitem{def:set_valued_map/inverse} We define the \term{inverse} \( F^{-1}: B \multto A \) of a set-valued map \( F: A \multto B \) as the function whose graph is the inverse of \( \gph(F) \) in the sense of \fullref{def:binary_relation/inverse}.

    For each set \( Y \subseteq B \), we call \( F^{-1}(Y) \) the \term{preimage} of \( Y \).

    \thmitem{def:set_valued_map/composition} We define the \term{composition} \( G \bincirc F \) of two set-valued maps \( F: A \multto B \) and \( G: B \multto C \) as the function
    \begin{equation*}
      [G \bincirc F](x) \coloneqq G(F(x)).
    \end{equation*}

    The square brackets around \( G \bincirc F \) are not a special notation, but rather another pair of delimiters that looks different from parentheses for the sake of reducing visual clutter.

    This definition is consistent with binary relation composition from \fullref{def:binary_relation/composition}.
  \end{thmenum}
\end{definition}
\begin{comments}
  \item Rather than introducing these concepts for set-valued functions, \incite[8]{Kelley1975} does it for relations.

  \item Some authors like \incite[1]{Engelking1989} and \incite[596]{Knapp2016BasicAlgebra} use \enquote{range} for what we call the codomain of a function, while others like \incite[11]{Kelley1975} and \incite[43]{Enderton1977Sets} use \enquote{range} for what we call the image.

  To avoid ambiguity, we avoid the term \enquote{range} altogether.
\end{comments}

\begin{remark}\label{rem:notation_for_function_image}
  The notation \( f[X] \) for the image of a set \( A \) is mostly used within set theory, for example by \incite[44]{Enderton1977Sets}. The notation \( f(X) \) is used in other context like topology, for example by \incite[1]{Engelking1989}, and analysis, for example by \incite[def. 1.3.1]{AubinFrankowska1990}. We generally prefer \( f(X) \) because it reflects the established convention in fields more relevant to us, but occasionally use \( F[X] \) in order to avoid ambiguities. For example, if \( A \) is a \hyperref[def:transitive_set]{transitive set}, then \( x \in A \) implies that \( x \subseteq A \), yet
  \begin{equation*}
    f[\set{ \varnothing }] = \set{ f(\varnothing) } \neq f(\set{ \varnothing }).
  \end{equation*}
\end{remark}

\begin{proposition}\label{thm:def:set_valued_map}
  \hyperref[def:set_valued_map]{Multi-valued functions} have the following basic properties:

  \begin{thmenum}
    \thmitem{thm:def:set_valued_map/associative} \hyperref[def:set_valued_map/composition]{Composition} is associative. That is, for any three functions \( F: A \to B \), \( G: B \to C \) and \( H: C \to D \) we have
    \begin{equation*}
      H \bincirc [G \bincirc F] = [H \bincirc G] \bincirc F.
    \end{equation*}

    We will henceforth simply write \( H \bincirc G \bincirc F \).

    \thmitem{thm:def:set_valued_map/composition_inverse} If \( F: A \to B \) and \( G: B \to C \) are \hyperref[def:function]{set-valued maps}, then
    \begin{equation*}
      [G \bincirc F]^{-1} = F^{-1} \bincirc G^{-1}.
    \end{equation*}
  \end{thmenum}
\end{proposition}
\begin{proof}
  \SubProofOf{thm:def:set_valued_map/associative} Let \( a \in A \). Then in order for \( d \in D \) to belong to \( [[H \bincirc G] \bincirc F](a) \), there must exist values \( b \in B \) and \( c \in C \) such that \( b \in F(a) \) and \( c \in G(b) \) and \( d \in H(c) \). Clearly this is also the condition for \( d \) to belong to \( [H \bincirc [G \bincirc F]](a) \).

  \SubProofOf{thm:def:set_valued_map/composition_inverse} Since \( G \bincirc F \) has signature \( A \to C \), clearly \( [G \bincirc F]^{-1} \) has signature \( C \to A \). Let \( c \in C \).

  \begin{itemize}
    \item If \( [G \bincirc F]^{-1}(c) \) is empty, \( c \not\in \img(G \bincirc F) \), hence either \( G^{-1}(c) \) is empty or is nonempty, but disjoint from \( \img(F) \). Hence, \( [F^{-1} \bincirc G^{-1}](c) \) is also empty.

    \item Suppose that \( [G \bincirc F]^{-1}(c) \) is not empty and let \( a \in [G \bincirc F]^{-1}(c) \).

    By definition, there exists \( b \in B \) and such that \( b \in F(a) \) and \( c \in G(b) \). Hence, \( b \in G^{-1}(c) \) and \( a \in F^{-1}(b) \), which implies that the image of \( c \) under the composition \( F^{-1} \bincirc G^{-1} \) also contains \( a \).
  \end{itemize}

  In both cases, for every \( c \in C \) we have
  \begin{equation*}
    [G \bincirc F]^{-1}(c) = [F^{-1} \bincirc G^{-1}](c).
  \end{equation*}

  Hence, the two set-valued maps are equal.
\end{proof}

\begin{concept}\label{con:function_arguments}
  As mentioned in \fullref{def:set_valued_map/value}, given a map \( F: A \multto B \) and a free variable \( x \), we sometimes use the notation \( F(x) \).

  If \( A = A_1 \times \cdots \times A_n \) is a \hyperref[def:cartesian_product/product]{finite Cartesian product}, we can instead use the notation \( F(x_1, \ldots, x_n) \) and regard \( x_1, \ldots, x_n \) as free variables that have no assigned value.

  The variables are called \term{arguments} or sometimes \term{parameters}. This notion is somewhat informal and depends on the context since \( A \) can usually be represented as a Cartesian product in different ways and with different arities. For example, if \( A = B \times C \), we can write both \( f(a) \) and \( f(b, c) \) and the function has a different number of parameters in each case. In practice, the number of arguments is usually clear from the context. We sometimes use \( \vect{a} \) to highlight that we regard \( a \) as a tuple.

  We may use prefixes from \fullref{def:operation_arity} like \enquote{unary}, \enquote{binary} and so forth on arbitrary functions if there is no possibility of confusion about the number of parameters.
\end{concept}

\begin{definition}\label{def:function}
  We say that the \hyperref[def:function]{set-valued map} \( F: A \multto B \) is a \term{function} if \( F(x) \) is \hyperref[def:set_valued_map/value]{single-valued} for each \( x \in A \). In this case, we write \( F: A \to B \) rather than \( F: A \multto B \). If disambiguation is necessary, we will use the term \term[bg=еднозначна функция, ru=однозначная функция]{single-valued function}.

  Strictly speaking, the value \( f(x) \) of a function \( f: A \to B \) is a \hyperref[rem:singleton_sets]{singleton set}. It is thus prevalent to define the value of a function to be an element of \( B \) rather than a subset of \( B \).

  Single-valued functions satisfy the following \hyperref[con:predicate_formula]{predicate formula}, which states that the free variable \( \rho \) is the graph of a function:
  \begin{equation*}\taglabel[\op{IsFun}]{eq:def:function/predicate}
    \ref{eq:def:function/predicate}[\rho] \coloneqq \qforall \xi \parens[\Big]{ \xi \in \rho \synimplies \qexists \eta \qexists \zeta \ref{eq:def:cartesian_product/kuratowski_pair_predicate}[\xi, \eta, \zeta] }
  \end{equation*}

  We will also use the following to verify that the domain of a function is a given set:
  \begin{equation*}\taglabel[\op{IsDom}]{eq:def:function/dom_predicate}
    \ref{eq:def:function/dom_predicate}[\rho, \sigma] \coloneqq \T{\todo{}}
  \end{equation*}

  Similarly:
  \begin{equation*}\taglabel[\op{IsRan}]{eq:def:function/ran_predicate}
    \ref{eq:def:function/ran_predicate}[\rho, \sigma] \coloneqq \T{\todo{}}
  \end{equation*}

  \begin{thmenum}
    \thmitem{def:function/endofunction}\mcite[183]{DaveyPriestley2002} We call functions from a set to itself \term{endofunctions}.

    \medskip

    \thmitem{def:function/selection}\mcite[52]{DontchevRockafellar2014} If \( f: A \to B \) is a single-valued function, \( F: A \multto B \) is a set-valued map and \( \gph(f) \subseteq \gph(F) \), we say that \( f \) is a \term{selection} of \( F \).

    \thmitem{def:function/set_of_functions}\mimprovised We denote the set of all total single-valued functions from \( A \) to \( B \) by \( \fun(A, B) \) and abbreviate \( \fun(A, A) \) as \( \fun(A) \).
  \end{thmenum}
\end{definition}
\begin{comments}
  \item All definitions from \fullref{def:function} holds for single-valued functions.

  \item We can simplify \fullref{def:set_valued_map/image} to
  \begin{equation*}
    f[A]
    =
    \bigcup \set[\Big]{ \set{ f(x) } \given x \in A }
    =
    \set{ f(x) \given x \in A }.
  \end{equation*}

  As mentioned in \fullref{def:set_valued_map/image}, we usually prefer the notation \( f[A] \) outside \fullref{sec:set_theory} where we are less prone to ambiguity.

  \item Accepted notations for \( \fun(A, B) \) are:
  \begin{itemize}
    \item \( \cat{Set}(A, B) \), used in relation to the category \hyperref[def:category_of_small_sets]{\( \cat{Set} \)}.
    \item \( B^A \), which is consistent with \hyperref[def:cardinal_arithmetic/exponentiation]{cardinal exponentiation} (used by \incite[example 3.2]{Aluffi2009}).
    \item \( {}^A B \) (used by \incite[52]{Enderton1977Sets}).
  \end{itemize}

  \item By convention, when both single-valued and set-valued maps are involved, the former are denoted using lowercase letters and the latter using uppercase letters.
\end{comments}

\begin{proposition}\label{thm:def:function}
  \hyperref[def:function]{Single-valued functions} have the following properties when regarded as \hyperref[def:function]{set-valued maps}:
  \begin{thmenum}
    \thmitem{thm:def:function/total} Single-valued functions are \hyperref[def:set_valued_map/partial]{total} as set-valued maps.

    \thmitem{thm:def:function/section_total} If a set-valued map has a \hyperref[def:function/selection]{selection}, it is \hyperref[def:set_valued_map/partial]{total}.

    \thmitem{thm:def:set_valued_map/composition} The \hyperref[def:set_valued_map/composition]{composition} of single-valued functions is a single-valued function.
  \end{thmenum}
\end{proposition}
\begin{proof}
  Trivial.
\end{proof}

\begin{definition}\label{def:constant_function}\mimprovised
  If the \hyperref[def:set_valued_map/image]{image} of a \hyperref[def:function]{single-valued function} \( f: A \to B \) is some \hyperref[rem:singleton_sets]{singleton set} \( \set{ b } \), we call it a \term{constant function} with value \( b \).
\end{definition}

\begin{remark}\label{rem:constant_function}\mcite{Social:empty_constant_function}
  \hyperref[def:constant_function]{Our definition} for constant functions avoids a common pitfall.

  Consider the following definition, based on Walter Rudin's one in \cite[def. 4.3]{Rudin1976Principles}:
  \begin{displayquote}
    If a \hyperref[def:function]{single-valued function} \( f: A \to B \) satisfies \( f(a) = b \) for every \( a \in A \), we call it a \term{constant function} with value \( b \).
  \end{displayquote}

  Under the latter definition, any function \( f: \varnothing \to B \) from an empty set is vacuously a constant function taking every possible value in \( B \). This is discussed in

  On the other hand, under the former definition, constant functions are precisely those that that \hyperref[def:factors_through]{factor through} \enquote{the} \hyperref[def:universal_objects/terminal]{terminal object} in \hyperref[def:category_of_small_sets]{\( \cat{Set} \)}, the set \( \set{ \varnothing } \).

  Hence, our definition also gives a good categorical characterization.
\end{remark}

\begin{remark}\label{rem:set_valued_map_as_single_valued}
  Rather than defining a \hyperref[def:function]{single-valued function} to be a special case of \hyperref[def:set_valued_map]{set-valued maps}, \incite[def. 2.3]{Phelps1993} defines set-valued maps from \( A \) to \( B \) as single-valued functions from \( A \) to \( \pow(B) \).

  The downside of the latter approach is that notions such as the \hyperref[def:set_valued_map/image]{image}, \hyperref[def:set_valued_map/codomain]{codomain} and \hyperref[def:set_valued_map/inverse]{inverse} of the set-valued map have a different and less useful for us meaning.
\end{remark}

\begin{remark}\label{rem:implicit_function_notation}
  It is sometimes useful to utilize \term{anonymous functions} such as \( x \mapsto x^2 \), which generalize \hyperref[def:first_order_substitution/term_in_term]{term substitution}. The goal is to avoid naming functions whose name we will not use. Some examples where we utilize this:
  \begin{itemize}
    \item Quotient maps such as \( H \mapsto H / N \) from \fullref{thm:lattice_theorem_for_subgroups}.
    \item Highlighting dependent and independent variables, such as Euclidean plane rotations in \fullref{thm:plane_rotation_matrix_angle}:
    \begin{equation*}
      \varphi
      \mapsto
      \begin{pmatrix}
        \cos \varphi & -\sin \varphi \\
        \sin \varphi & \cos \varphi
      \end{pmatrix}
    \end{equation*}

    \item Defining functors such as the power set functor from \fullref{ex:unary_functors_in_set/power}:
    \begin{equation*}
      \begin{aligned}
        &\pow: \cat{Set} \to \cat{Set}, \\
        &\pow(A) \coloneqq \set{ S \given S \subseteq A }, \\
        &\pow(f: A \to B) \coloneqq (S \mapsto f[S]). \\
      \end{aligned}
    \end{equation*}
  \end{itemize}
\end{remark}

\begin{definition}\label{def:function_invertibility}\mimprovised
  We introduce the following terminology for invertibility of a (single-valued) function \( f: A \to B \):
  \begin{thmenum}
    \begin{minipage}[t]{0.43\textwidth}
      \thmitem{def:function_invertibility/injective} We say that \( f \) is \term{injective} if any of the following equivalent conditions hold:
      \begin{thmenum}
        \thmitem{def:function_invertibility/injective/existence} For any \( y \in B \) there exists \hi{at most} one \( x \in A \) such that \( f(x) = y \).

        Each point in \( B \) is the image of at most one point in \( A \).
        \newline

        \thmitem{def:function_invertibility/injective/equality} For all \( x_1, x_2 \in A \), the equality \( f(x_1) = f(x_2) \) implies \( x_1 = x_2 \).

        The contrapositive of this statement is that different points in \( A \) have different images under \( f \).

        \thmitem{def:function_invertibility/injective/inverse} The inverse is a partial single-valued function.
      \end{thmenum}
    \end{minipage}
    \hfill
    \begin{minipage}[t]{0.44\textwidth}
      \thmitem{def:function_invertibility/surjective} We say that \( f \) is \term{surjective} if any of the following equivalent conditions hold:
      \newline
      \begin{thmenum}[leftmargin=0.9cm]
        \thmitem{def:function_invertibility/surjective/existence} For any \( y \in B \) there exists \hi{at least} one \( x \in A \) such that \( f(x) = y \).

        Each point in \( B \) is the image of at least one point in \( A \). Hence, the image of \( f \) is the entire codomain \( B \).

        \thmitem{def:function_invertibility/surjective/equality} For all \( y_1, y_2 \in B \), the equality \( f^{-1}(y_1) = f^{-1}(y_2) \) implies \( y_1 = y_2 \).

        Without surjectivity, the above holds only for the points in the image of \( f \).

        \thmitem{def:function_invertibility/surjective/inverse} The inverse is a total set-valued map.
      \end{thmenum}
    \end{minipage}

    \thmitem{def:function_invertibility/bijective} Finally, we say that \( f \) is \term{bijective} if any of the following equivalent conditions hold:
    \begin{thmenum}
      \thmitem{def:function_invertibility/bijective/direct} It is both injective and surjective.
      \thmitem{def:function_invertibility/bijective/existence} For any \( y \in B \) there exists exactly one \( x \in A \) such that \( f(x) = y \).
      \thmitem{def:function_invertibility/bijective/inverse} The inverse is a total single-valued function.
    \end{thmenum}
  \end{thmenum}
\end{definition}
\begin{defproof}
  The equivalences are trivial to verify.
\end{defproof}

\begin{proposition}\label{thm:function_composition_invertibility}
  The composition of injective (resp. surjective or bijective) functions is injective (resp. surjective or bijective).
\end{proposition}
\begin{comments}
  \item Compare this result to \fullref{thm:def:morphism_invertibility/invertible_composition} and \fullref{thm:function_superposition_invertibility}.
\end{comments}
\begin{proof}
  Let \( f: A \to B \) and \( g: B \to C \) be arbitrary functions and define \( h \coloneqq g \circ f: A \to C \).

  \SubProofOf[def:function_invertibility/injective/equality]{injectivity} Suppose that \( f \) and \( g \) are injective and satisfy \fullref{def:function_invertibility/injective/equality}.

  Let \( x_1, x_2 \in A \) and suppose that \( h(x_1) = h(x_2) \), that is, \( g(f(x_1)) = g(f(x_2)) \). Then \( f(x_1) = f(x_2) \) since \( g \) is injective and \( x_1 = x_2 \) since \( f \) is injective.

  Since \( x_1 \) and \( x_2 \) were arbitrary, we conclude that \( h \) is also injective.

  \SubProofOf[def:function_invertibility/surjective/existence]{surjectivity} Suppose that \( f \) and \( g \) are surjective and satify \fullref{def:function_invertibility/injective/existence}.

  Let \( z \in C \). Then there exists some \( y \in B \) such that \( g(y) = z \) because \( g \) is surjective and similarly there exists some \( x \in B \) such that \( f(x) = y \) because \( f \) is surjective. Thus, \( h(x) = g(f(x)) = z \).

  Since \( z \) was arbitrary, we conclude that \( h \) is also surjective.

  \SubProofOf[def:function_invertibility/bijective/direct]{bijectivity} We have shown that if \( f \) and \( g \) are both injective or surjective, so is \( h \). Hence, if \( f \) and \( g \) are bijective, so is \( h \).
\end{proof}

\begin{lemma}\label{thm:diagonal_product_injectivity}
  Let \( \seq{ f_k: A \to B_k }_{k \in \mscrK} \) be an indexed family of functions, where \( f_k: A \to B_k \) in \( k \in \mscrK \). If at least one of the functions is injective, the \hyperref[def:topological_product]{diagonal product}
  \begin{equation*}
    \begin{aligned}
      f: A \to \prod B_k \\
      f(x) \coloneqq \seq{ f_k(x) }_{k \in \mscrK}
    \end{aligned}
  \end{equation*}
  is also injective.
\end{lemma}
\begin{proof}
  Suppose that \( f_{k_0} \) is injective. Let \( f(x_1) = f(x_2) \). Then \( \seq{ f_k(x_1) }_{k \in \mscrK} = \seq{ f_k(x_2) }_{k \in \mscrK} \) and thus \( f_{k_0}(x) = f_{k_0}(y) \). Since \( f_{k_0} \) is injective, we conclude that \( x_1 = x_2 \). Since \( x_1 \) and \( x_2 \) were arbitrary, we conclude that \( f \) is also injective.
\end{proof}

\begin{concept}\label{con:function_superposition}
  Although the terms \enquote{composition} and \enquote{superposition} are used interchangeably, for example by \incite[\textnumero 25]{ФихтенгольцОсновыТом1}, the term \enquote{superposition} often refers to a certain generalization of \hyperref[def:set_valued_map/composition]{function composition}. The latter is used by \incite[16]{Яблонский2003}.

  If we are given the family of functions (or set-valued maps) \( f_k: A \multto B_k \) for \( k = 1, \ldots, n \) and \( G: B_1 \times \cdots \times B_n \multto C \), we define their \term{superposition} as
  \begin{equation*}
    \begin{aligned}
      &H: A \to C, \\
      &H(x) \coloneqq G(F_1(x), \ldots, F_n(x)).
    \end{aligned}
  \end{equation*}
\end{concept}

\begin{proposition}\label{thm:function_superposition_invertibility}
  The superposition of injective functions is injective.
\end{proposition}
\begin{comments}
  \item Compare this result to \fullref{thm:function_composition_invertibility}.
\end{comments}
\begin{proof}
  Suppose that we are given injective functions \( f_k: A \multto B_k \) for \( k = 1, \ldots, n \) and \( g: B_1 \times \cdots \times B_n \multto C \). From \fullref{thm:diagonal_product_injectivity} it follows that the function
  \begin{equation*}
    \begin{aligned}
      &d: A \to B_1 \times \cdots \times B_n, \\
      &d(x) \coloneqq (f_1(x), \ldots, f_n(x))
    \end{aligned}
  \end{equation*}
  is injective.

  Then from \fullref{thm:function_composition_invertibility} it follows that the desired superposition
  \begin{equation*}
    \begin{aligned}
      &h: A \to C, \\
      &h(x) \coloneqq g(f_1(x), \ldots, f_n(x)) = g(d(x)).
    \end{aligned}
  \end{equation*}
  is injective.
\end{proof}

\begin{proposition}\label{thm:function_image_preimage_composition}
  For any function \( f: A \to B \), we have
  \begin{thmenum}
    \thmitem{thm:function_image_preimage_composition/preimage_of_image} If \( X \subseteq A \), then \( X \subseteq f^{-1}[f[X]] \). Equality holds if \( f \) is injective.
    \thmitem{thm:function_image_preimage_composition/image_of_preimage} If \( Y \subseteq B \), then \( f[f^{-1}[Y]] \subseteq Y \). Equality holds if \( f \) is surjective.
  \end{thmenum}
\end{proposition}
\begin{proof}
  \SubProofOf{thm:function_image_preimage_composition/preimage_of_image} If \( x \in X \), clearly \( x \in f^{-1}(f(x)) \). Thus,
  \begin{equation*}
    X \subseteq f^{-1}[f[X]].
  \end{equation*}

  Now suppose that \( f \) is injective and let \( x \in f^{-1}[f[X]] \). There exists some \( y \in f[X] \) such that \( x \in f^{-1}(y) \) and some \( z \in X \) such that \( y = f(z) \). Since \( f \) is injective and \( f(x) = y = f(z) \), it follows that \( x = z \) and thus \( x \in X \). Since \( x \) was chosen arbitrarily from \( f^{-1}[f[X]] \), we conclude that
  \begin{equation*}
    f^{-1}[f[X]] \subseteq X.
  \end{equation*}

  \SubProofOf{thm:function_image_preimage_composition/image_of_preimage} If \( y \in f[f^{-1}[Y]] \), there exists some \( x \in f^{-1}[Y] \) such that \( f(x) = y \). Furthermore, there also exists some \( t \in Y \) such that \( x \in f^{-1}(t) \). Hence, \( y = f(x) = t \) and \( y \in Y \). Therefore,
  \begin{equation*}
    f[f^{-1}[Y]] \subseteq Y.
  \end{equation*}

  Now suppose that \( f \) is surjective and let \( y \in Y \). Then from surjectivity if follows that there exists some \( x \in X \) such that \( f(x) = y \). Hence, \( x \in f^{-1}(y) \) and \( y = f(x) \in f[f^{-1}(y)] \). Since \( y \) was chosen arbitrarily from \( Y \), we conclude that
  \begin{equation*}
    Y \subseteq f[f^{-1}[Y]].
  \end{equation*}
\end{proof}

\begin{proposition}\label{thm:function_image_properties}
  \hyperref[def:set_valued_map/image]{Images of sets} under \( f: A \to B \) have the following basic properties:
  \begin{thmenum}
    \thmitem{thm:function_image_properties/monotonicity} If \( A_1 \subseteq A_2 \), then \( f[A_1] \subseteq f[A_2] \).

    \thmitem{thm:function_image_properties/union} For any \hyperref[def:cartesian_product/indexed_family]{indexed family} \( \seq{ A_k }_{k \in \mscrK} \subseteq A \) of subsets of \( A \), we have the equality
    \begin{equation}\label{eq:thm:function_image_properties/union}
      f\bracks*{ \bigcup_{k \in \mscrK} A_k } = \bigcup_{k \in \mscrK} f[A_k].
    \end{equation}

    \thmitem{thm:function_image_properties/intersection} For any indexed family \( \seq{ A_k }_{k \in \mscrK} \) of subsets of \( A \), we have the inclusion
    \begin{equation}\label{eq:thm:function_image_properties/intersection}
      f\bracks*{ \bigcap_{k \in \mscrK} A_k } \subseteq \bigcap_{k \in \mscrK} f[A_k].
    \end{equation}

    Equality in \eqref{eq:thm:function_image_properties/intersection} holds if \( f \) is \hyperref[def:function_invertibility/injective]{injective}. If \( f \) is not injective, for example if both \( A \) and \( B \) are nonempty, \( A_1 \) and \( A_2 \) are disjoint subsets of \( A \) and \( f[A_1] = f[A_2] = B \), then
    \begin{equation*}
      f[A_1 \cap A_2] = f[\varnothing] = \varnothing \subsetneq f[A_1] \cap f[A_2] = B.
    \end{equation*}

    \thmitem{thm:function_image_properties/difference} For any two subsets \( A_1 \) and \( A_2 \) of \( A \), we have the inclusion
    \begin{equation}\label{eq:thm:function_image_properties/difference}
      f[A_1] \setminus f[A_2] \subseteq f[A_1 \setminus A_2].
    \end{equation}
  \end{thmenum}
\end{proposition}
\begin{comments}
  \item Compare this result to the more well-behaved properties of \hyperref[def:set_valued_map/inverse]{preimages} described in \fullref{thm:function_preimage_properties}.

  \item Equality in \eqref{eq:thm:function_image_properties/difference} holds if \( f \) is injective. If \( f \) is not injective, for example if \( A_1 \subsetneq A_2 \), but \( f[A_1] = f[A_2] \), then
  \begin{equation*}
    f[A_1] \setminus f[A_2] = \varnothing \subsetneq f[A_1 \setminus A_2].
  \end{equation*}
\end{comments}
\begin{proof}
  \SubProofOf{thm:function_image_properties/monotonicity} If \( x \in A_1 \), then \( x \in A_2 \) and hence \( f(x) \in f[A_2] \). Therefore, \( f[A_1] \subseteq f[A_2] \).

  \SubProofOf{thm:function_image_properties/union} If \( x_0 \in A_{k_0} \) for some \( k_0 \in \mscrK \), clearly
  \begin{equation*}
    f(x_0) \in f[A_{k_0}] \subseteq \bigcup_{k \in \mscrK} f[A_k].
  \end{equation*}

  Therefore,
  \begin{equation*}
    f\bracks*{ \bigcup_{k \in \mscrK} A_k } \subseteq \bigcup_{k \in \mscrK} f[A_k].
  \end{equation*}

  Conversely, if \( y_0 \in f[A_{k_0}] \) for some \( k_0 \in \mscrK \), by \fullref{thm:function_image_properties/monotonicity} obviously
  \begin{equation*}
    y_0 \in f\bracks*{ \bigcup_{k \in \mscrK} A_k }.
  \end{equation*}

  Therefore,
  \begin{equation*}
    f\bracks*{ \bigcup_{k \in \mscrK} A_k } \supseteq \bigcup_{k \in \mscrK} f[A_k].
  \end{equation*}

  Hence, \eqref{eq:thm:function_image_properties/union} holds.

  \SubProofOf{thm:function_image_properties/intersection} If \( x_0 \) belongs to \( \bigcap_{k \in \mscrK} A_k \), then \( x_0 \) belongs to \( A_k \) for all \( k \in \mscrK \). It follows that \( f(x_0) \) belongs to \( f[A_k] \) for all \( k \in \mscrK \) and hence to their intersection. Therefore, the inclusion \eqref{eq:thm:function_image_properties/intersection} holds.

  Now suppose that \( f \) is injective. Let \( y_0 \) be a point in the intersection \( \bigcap_{k \in \mscrK} f[A_k] \). We thus have \( y_0 \in f[A_k] \) for all \( k \in \mscrK \). Since \( f \) is injective, for each \( k \in \mscrK \) there exists a unique \( x_k \in A_k \) such that \( f(x_k) = y_0 \). Again because of injectivity of \( f \), all these elements are equal because \( f(x_k) = f(x_m) = y_0 \) for \( k, m \in \mscrK \). Hence,
  \begin{equation*}
    y_0 \in f\bracks*{ \bigcap_{k \in \mscrK} A_k }.
  \end{equation*}

  Therefore, the reverse inclusion in \eqref{eq:thm:function_image_properties/intersection} holds if \( f \) is injective.

  \SubProofOf{thm:function_image_properties/difference} If \( f[A_1] \setminus f[A_2] \) is empty, \eqref{eq:thm:function_image_properties/difference} obviously holds. Suppose that it is nonempty and let \( y_0 \in f[A_1] \setminus f[A_2] \).

  Then there exists a point \( x_0 \in A_1 \) such that \( f(x_0) = y_0 \). It cannot be that \( x_0 \in A_2 \) because otherwise \( y_0 = f(x_0) \in f[A_2] \), which would contradict our choice of \( y_0 \). Hence, \( x_0 \in A_1 \setminus A_2 \) and \( y_0 \in f(A_1 \setminus A_2) \).

  Since \( y_0 \) was chosen arbitrarily, we conclude that the inclusion \eqref{eq:thm:function_image_properties/difference} holds.

  Conversely, suppose that \( f \) is injective. If \( f(A_1 \setminus A_2) \) is empty, by \eqref{eq:thm:function_image_properties/difference} the set \( f[A_1] \setminus f[A_2] \) is also empty and the converse holds.

  Now suppose that it is nonempty and let \( y_0 \in f(A_1 \setminus A_2) \). Then there exists a point \( x_0 \in A_1 \setminus A_2 \) such that \( f(x_0) = y_0 \). Furthermore, since \( f \) is injective, \( x_0 \) is the only preimage of \( y_0 \) and hence \( f(x_0) \in f[A_1] \setminus f[A_2] \), which proves the reverse inclusion in \eqref{eq:thm:function_image_properties/difference}.
\end{proof}

\begin{proposition}\label{thm:function_preimage_properties}
  Function \hyperref[def:set_valued_map/inverse]{preimages} have the following basic properties:
  \begin{thmenum}
    \thmitem{thm:function_preimage_properties/monotonicity} If \( B_1 \subseteq B_2 \), then \( f^{-1}[B_1] \subseteq f^{-1}[B_2] \).

    \thmitem{thm:function_preimage_properties/union} For any \hyperref[def:cartesian_product/indexed_family]{indexed family} \( \seq{ B_k }_{k \in \mscrK} \subseteq B \) of subsets of \( B \), we have the equality
    \begin{equation}\label{eq:thm:function_preimage_properties/union}
      f^{-1}\bracks*{ \bigcup_{k \in \mscrK} B_k } = \bigcup_{k \in \mscrK} f^{-1}[B_k].
    \end{equation}

    \thmitem{thm:function_preimage_properties/intersection} For any indexed family \( \seq{ B_k }_{k \in \mscrK} \) of subsets of \( B \), we have the equality
    \begin{equation}\label{eq:thm:function_preimage_properties/intersection}
      f^{-1}\bracks*{ \bigcap_{k \in \mscrK} B_k } = \bigcap_{k \in \mscrK} f^{-1}[B_k].
    \end{equation}

    \thmitem{thm:function_preimage_properties/difference} For any two subsets \( B_1 \) and \( B_2 \) of \( B \), we have the equality
    \begin{equation}\label{eq:thm:function_preimage_properties/difference}
      f^{-1}[B_1] \setminus f^{-1}[B_2] = f^{-1}[B_1 \setminus B_2].
    \end{equation}
  \end{thmenum}
\end{proposition}
\begin{comments}
  \item Compare this result to the less well-behaved properties of images described in \fullref{thm:function_image_properties}.
\end{comments}
\begin{proof}
  \SubProofOf{thm:function_image_properties/monotonicity} Analogous to \fullref{thm:function_image_properties/monotonicity}.

  \SubProofOf{thm:function_image_properties/union} Analogous to \fullref{thm:function_image_properties/union}.

  \SubProofOf{thm:function_image_properties/intersection} If \( y_0 \) belongs to \( \bigcap_{k \in \mscrK} B_k \), then \( y_0 \) belongs to \( B_k \) for all \( k \in \mscrK \). It follows that \( f(y_0) \subseteq f^{-1}[B_k] \) for all \( k \in \mscrK \) and hence it is also a subset of their intersection. Therefore,
  \begin{equation*}
    f^{-1} \bracks*{ \bigcap_{k \in \mscrK} B_k } \subseteq \bigcap_{k \in \mscrK} f^{-1}[B_k].
  \end{equation*}

  Conversely, if \( x_0 \in \bigcap_{k \in \mscrK} f^{-1}[B_k] \), it belongs to \( f^{-1}[B_k] \) for all \( k \in \mscrK \). Clearly then \( f(x_0) \in B_k \) for all \( k \in \mscrK \) and thus \( f(x_0) \in \bigcap_{k \in \mscrK} B_k \). Hence, by \fullref{thm:function_preimage_properties/monotonicity} we have
  \begin{equation*}
    f^{-1}(f(x_0))
    \subseteq
    f^{-1}\bracks*{ \bigcap_{k \in \mscrK} B_k }.
  \end{equation*}

  Since \( x_0 \in f^{-1}(f(x_0)) \),
  \begin{equation*}
    x_0 \in f^{-1}\bracks*{ \bigcap_{k \in \mscrK} B_k }.
  \end{equation*}

  Since \( x_0 \) was chosen arbitrarily from \( \bigcap_{k \in \mscrK} f^{-1}[B_k] \), we can conclude that
  \begin{equation*}
    \bigcap_{k \in \mscrK} f^{-1}[B_k] \in f^{-1}\bracks*{ \bigcap_{k \in \mscrK} B_k }.
  \end{equation*}

  Hence, \eqref{eq:thm:function_preimage_properties/intersection} holds.

  \SubProofOf{thm:function_preimage_properties/difference} If \( y_0 \in B_1 \setminus B_2 \), there exists a point \( x_1 \in B_1 \) such that \( f(x_1) = y_0 \). Aiming at a contradiction, suppose that there exists a point \( x_2 \in f^{-1}[B_2] \) such that \( f(x_2) = y_0 \). Then \( y_0 = f(x_1) = f(x_2) \) implies that \( f^{-1}(y_0) \subseteq f^{-1}[B_1] \cap f^{-1}[B_2] \). \Fullref{thm:function_preimage_properties/intersection} then in turn implies that \( f^{-1}(y_0) \subseteq f^{-1}[B_1 \cap B_2] \) and hence by \fullref{thm:function_image_properties/monotonicity}
  \begin{equation*}
    y_0 = f(f^{-1}(y_0)) \in f[f^{-1}[B_1 \cap B_2]] = B_1 \cap B_2,
  \end{equation*}
  which contradicts our choice of \( y_0 \). Since the choice of \( y_0 \in B_1 \setminus B_2 \), \( x_1 \in f^{-1}(y_0) \cap B_1 \) and \( x_2 \in f^{-1}(y_0) \cap B_2 \) was arbitrary, the obtained contradiction shows that
  \begin{equation*}
    f^{-1}(B_1 \setminus B_2) \subseteq f^{-1}[B_1] \setminus f^{-1}[B_2].
  \end{equation*}

  Conversely, we have
  \begin{equation*}
    f(f^{-1}[B_1] \setminus f^{-1}[B_2])
    \reloset {\eqref{eq:thm:function_image_properties/difference}} \subseteq
    f(f^{-1}(B_1 \setminus B_2))
    \reloset {\ref{thm:function_image_preimage_composition/image_of_preimage}} \subseteq
    B_1 \setminus B_2.
  \end{equation*}

  Hence,
  \begin{equation*}
    f^{-1}[B_1] \setminus f^{-1}[B_2]
    \reloset {\ref{thm:function_image_preimage_composition/preimage_of_image}} \subseteq
    f^{-1}\parens[\Big]{ f\parens[\Big]{ f^{-1}[B_1] \setminus f^{-1}[B_2] } }
    \reloset {\eqref{thm:function_preimage_properties/monotonicity}} \subseteq
    f^{-1}(B_1 \setminus B_2).
  \end{equation*}
\end{proof}

\begin{theorem}[Recursion theorem]\label{thm:omega_recursion}\mcite[73]{Enderton1977Sets}
  Let \( A \) be a nonempty set. Suppose that we are given some member \( a_0 \) of \( A \) and some transformation \( T: A \to A \). Then there exists a \hyperref[def:sequence]{sequence} \( f: \omega \to A \) such that
  \begin{itemize}
    \item \( f(n) = a_0 \).
    \item For every \( n \in \omega \), we have \( f(\op{succ}(n)) = T(f(n)) \).
  \end{itemize}
\end{theorem}
\begin{comments}
  \item Note that we do not yet use the notation \( n + 1 \) because we will use this theorem to define addition.
  \item See \fullref{rem:natural_number_recursion} for a simpler and more conventional notation for recursion on \( \omega \).
\end{comments}
\begin{proof}
  Let \( G \subseteq \pow(\omega \times A) \) be the set of all \hyperref[def:set_valued_map/partial]{partial single-valued functions} \( g: \omega \to A \) such that
  \begin{itemize}
    \item If \( g \) is defined at \( \varnothing \), then \( g(\varnothing) = a_0 \).
    \item For every \( n \in \omega \), if \( g \) is defined at \( \op{succ}(n) \), then \( g \) is also defined at \( n \) and
    \begin{equation*}
      g(\op{succ}(n)) = T(f(n)).
    \end{equation*}
  \end{itemize}

  Clearly \( G \) is nonempty because the function \( \set{ (\varnothing, a_0) } \) belongs to \( G \).

  The conditions imposed on the functions in \( G \) ensure that every function is defined in some \hyperref[def:order_interval/unbounded]{initial segment} of the natural numbers. A more obvious approach is to require \( g \) to be defined at \( \op{succ}(n) \) if it is defined at \( n \), however that would hinder our induction schema.

  Define \( f \coloneqq \bigcup G \). At this point \( f \) is simply some \hyperref[def:function]{set-valued map}. We must now show that \( f \) has all the properties that we want.

  \SubProofOf[def:set_valued_map/partial]{totality} First, we will use \fullref{thm:omega_induction} to show that \( f \) is total. Clearly \( \varnothing \in \dom f \).

  Now fix \( n \in \dom f \). Then there exists a function \( g \in G \) defined at \( n \).

  \begin{itemize}
    \item If \( g \) is also defined at \( \op{succ}(n) \), this directly proves that \( \op{succ}(n) \in \dom f \).
    \item If \( g \) is not defined at \( \op{succ}(n) \), consider
    \begin{equation*}
      \widehat g \coloneqq g \cup \set{ (\op{succ}(n), T(g(n)) }.
    \end{equation*}

    The function \( \widehat g \) is again a single-valued partial function, and thus it belongs to \( G \), hence \( \op{succ}(n) \in \dom f \).
  \end{itemize}

  Therefore, \fullref{thm:omega_induction} allows us to conclude \( f: \omega \multto A \) is a total set-valued map.

  \SubProofOf[def:function]{single-valuedness} Now that we know that \( f \) is total, we will prove that it is single-valued and thus is a function in the usual sense of the term.

  Clearly \( f \) is single-valued at \( \varnothing \).

  Now suppose that \( f \) is single-valued at \( n \). Since \( f \) is total, there exist at least one partial function in \( G \) that is defined at \( \op{succ}(n) \), from which it follows that it is also defined at \( n \).  Let \( g \) and \( h \) both be such (single-valued partial) functions.

  Then
  \begin{equation*}
    g(\op{succ}(n)) = T(g(n)) = T(f(n)) = T(h(n)) = h(\op{succ}(n)),
  \end{equation*}
  hence \( g \) and \( h \) coincide at \( \op{succ}(n) \), which in turn implies that \( f \) is single-valued at \( \op{succ}(n) \).

  Therefore, \fullref{thm:omega_induction} allows us to conclude that \( f \) is a single-valued total function.

  \SubProofOf[def:function]{uniqueness} Now that it is clear that \( f \) satisfies the theorem, we must verify that it is unique.

  Suppose that \( f_1 \) and \( f_2 \) both satisfy the theorem. Clearly \( \varnothing \in H \). Fix some \( n \neq \varnothing \) and suppose that \( f_1(n) = f_2(n) \). Then
  \begin{equation*}
    f_2(\op{succ}(n)) = T(f_1(n)) = T(f_2(n)) = f_2(\op{succ}(n)).
  \end{equation*}

  Therefore, \fullref{thm:omega_induction} allows us to conclude that \( f_1 = f_2 \). So there is at most one function that satisfies the theorem, and we have already shown that \( f \) is such a function.
\end{proof}

\begin{definition}\label{def:omega_operations}\mimprovised
  We will use \fullref{thm:omega_recursion} for defining arithmetic operations for natural numbers. These constructions will be more elaborate than the basic recursive sequences like those defined in \fullref{thm:banach_fixed_point_theorem}.

  \begin{thmenum}
    \thmitem{def:omega_operations/addition} We will represent the addition operation \( \oplus \) as follows:

    \begin{itemize}
      \item Fix the first summand \( n \). We will now construct a function \( \oplus_n: \omega \to \omega \) such that \( \oplus_n(m) = n \oplus m \). This is a particular instance of \hyperref[def:function_currying]{currying}.

      In the context of \fullref{thm:omega_recursion}, let \( a_0^{(n)} = n \) and
      \begin{equation*}
        \begin{aligned}
          &T^{(n)}: \omega \to \omega, \\
          &T^{(n)}(k) \coloneqq \op{succ}(k).
        \end{aligned}
      \end{equation*}

      Recursion then gives us a function \( \oplus_n: \omega \to \omega \).

      \item We can now define the full binary addition function \( \oplus: \omega \times \omega \to \omega \) via its graph
      \begin{equation*}
        \set[\Big]{ \parens[\Big]{ (n, m), \oplus_n(m) } \given* n, m \in \omega }.
      \end{equation*}
    \end{itemize}

    \thmitem{def:omega_operations/multiplication} We define natural number multiplication analogously. For each \( n \in \omega \), recursively define \( \odot_n \) via
    \begin{equation*}
      \begin{aligned}
        &T^{(n)}: \omega \to \omega, \\
        &T^{(n)}(k) \coloneqq k \oplus n.
      \end{aligned}
    \end{equation*}
    and \( a_0^{(n)} = 0 \) and then define \( \odot: \omega \times \omega \to \omega \) via its graph
    \begin{equation*}
      \set[\Big]{ \parens[\Big]{ (n, m), \odot_n(m) } \given* n, m \in \omega }.
    \end{equation*}
  \end{thmenum}
\end{definition}

\begin{theorem}\label{thm:omega_is_model_of_pa}
  The \hyperref[thm:smallest_inductive_set_existence]{smallest inductive set} \( \omega \) is a model of \hyperref[def:peano_arithmetic]{Peano arithmetic} with the following interpretation:
  \begin{thmenum}
    \thmitem{thm:omega_is_model_of_pa/zero} \hyperref[def:peano_arithmetic/zero]{Zero} is interpreted as \( \varnothing \).

    \thmitem{thm:omega_is_model_of_pa/succ} The \hyperref[def:peano_arithmetic/succ]{successor} operation \( s \) is interpreted as \( \op{succ} \).

    \thmitem{thm:omega_is_model_of_pa/plus} \hyperref[def:peano_arithmetic/plus]{Addition} is interpreted by the \( \oplus \) function given in \fullref{def:omega_operations/addition}.

    \thmitem{thm:omega_is_model_of_pa/mult} Similarly, \hyperref[def:peano_arithmetic/mult]{multiplication} is interpreted by \( \odot \) from \fullref{def:omega_operations/multiplication}.
  \end{thmenum}
\end{theorem}
\begin{comments}
  \item This is an extension of \fullref{thm:omega_is_model_of_pa_without_operations}.
\end{comments}
\begin{proof}
  We have already shown in \fullref{thm:omega_is_model_of_pa_without_operations} that \( \omega \) satisfies the axioms \eqref{eq:def:peano_arithmetic/PA1}-\eqref{eq:def:peano_arithmetic/PA3}. It remains to show that is satisfies \eqref{eq:def:peano_arithmetic/PA4}-\eqref{eq:def:peano_arithmetic/PA7}.

  \SubProofOf{eq:def:peano_arithmetic/PA4} For each \( n \in \omega \) the starting condition (i.e. \( m = \varnothing \)) in \fullref{def:omega_operations/addition} implies that \( n \oplus \varnothing = n \).

  \SubProofOf{eq:def:peano_arithmetic/PA5} For each \( n \in \omega \) the transformation \( T_n \) in \fullref{def:omega_operations/addition} is defined so that \( \oplus_n(m) = k \) implies
  \begin{equation*}
    \oplus_n(\op{succ}(m)) = \op{succ}(k).
  \end{equation*}

  It follows that, for all \( n, m \in \omega \),
  \begin{equation*}
    n \oplus \op{succ}(m) = \op{succ}(n \oplus m).
  \end{equation*}

  \SubProofOf{eq:def:peano_arithmetic/PA6} For each \( n \in \omega \), the starting condition in \fullref{def:omega_operations/multiplication} implies that \( n \odot \varnothing = \varnothing \).

  \SubProofOf{eq:def:peano_arithmetic/PA7} For each \( n \in \omega \), the transformation \( T_n \) in \fullref{def:omega_operations/multiplication} is defined so that \( \odot_n(m) = k \) implies
  \begin{equation*}
    \odot_n(\op{succ}(m)) = k \oplus n
  \end{equation*}

  It follows that, for all \( n, m \in \omega \),
  \begin{equation*}
    n \odot \op{succ}(m) = \op{succ}(n \oplus m).
  \end{equation*}
\end{proof}

\begin{remark}\label{rem:natural_number_recursion}
  With the availability of natural numbers, instead of the tedious constructions in \fullref{def:omega_operations/addition}, we can use a more conventional notation when applying \fullref{thm:omega_recursion}.

  As an example, we can define the Fibonacci sequence. The sequence is motivated by the problem in \fullref{ex:fibonacci_rabbits}. In the notation of \fullref{thm:omega_recursion}, we can define the sequence as follows:
  \begin{equation*}
    \begin{aligned}
      &T: \BbbN \times \BbbN \to \BbbN \times \BbbN \\
      &T(a, b) \coloneqq (b, a + b)
    \end{aligned}
  \end{equation*}
  and
  \begin{equation*}
    a_0 \coloneqq (0, 1).
  \end{equation*}

  The recursion theorem gives us a sequence of pairs
  \begin{equation*}
    \underbrace{(0, 1)}_{a_0}, \underbrace{(1, 1)}_{a_1}, \underbrace{(1, 2)}_{a_2}, \underbrace{(2, 3)}_{a_3}, \underbrace{(3, 5)}_{a_4}, \underbrace{(5, 8)}_{a_5}, \underbrace{(8, 13)}_{a_6}, \ldots
  \end{equation*}

  The pairs are only a technicality because otherwise we would not be able to define the sequence \( \seq{ a_k }_{k=1}^\infty \).

  By taking the second element of each pair, we obtain the sequence
  \begin{equation*}
    \underbrace{1}_{b_0}, \underbrace{1}_{b_1}, \underbrace{2}_{b_2}, \underbrace{3}_{b_3}, \underbrace{5}_{b_4}, \underbrace{8}_{b_5}, \underbrace{13}_{b_6}, \ldots
  \end{equation*}

  In order to obtain the Fibonacci sequence, we must prefix the sequence \( \seq{ b_k }_{k=0}^\infty \) with \( 0 \).

  This is undoubtedly much more complicated than writing
  \begin{equation*}
    b_k \coloneqq \begin{cases}
      0,                &k = 0, \\
      1,                &k = 1, \\
      b_{k-1} + b_{k-2} &k > 1.
    \end{cases}
  \end{equation*}

  To see that the latter notation is merely syntax sugar, note that the other sequence can be written as
  \begin{equation*}
    a_k \coloneqq \begin{cases}
      (0, 1),     &k = 0 \\
      T(a_{k-1}), &k > 1.
    \end{cases}
  \end{equation*}
\end{remark}

  \section{Zermelo-Fraenkel set theory}\label{sec:zermelo_fraenkel_set_theory}

\begin{definition}\label{def:zfc}\mcite[def. 6.1.6]{Hinman2005Logic}
  The \hyperref[def:first_order_theory]{first-order theory} commonly abbreviated as \term{\logic{ZFC}} is based on the same language as \hyperref[def:naive_set_theory]{na\"ive set theory}, but with different axioms. The three letters refer to:
  \begin{itemize}
    \item \textbf{Z}ermelo, who formulated the entire theory except for the \hyperref[def:zfc/replacement]{axiom schema of replacement} and the \hyperref[def:zfc/foundation]{axiom of foundation}.
    \item \textbf{F}raenkel, who simultaneously with Skolem reformulated the theory within first-order logic while also introducing the axiom schema of replacement.
    \item The \hyperref[def:zfc/choice]{axiom of \textbf{c}hoice}, which is part of Zermelo's original theory, but is controversial enough to attract special attention --- see \fullref{thm:axiom_of_choice_equivalences}.
  \end{itemize}

  We are usually only interested in the entire theory. If we wish to avoid the axiom of choice --- for example when proving the equivalences in \fullref{thm:axiom_of_choice_equivalences} --- we can instead use \logic{ZF}, which excludes the axiom of choice. The abbreviation of the latter theory is inaccurate historically, but is nevertheless established.

  If we wish to instead exclude the axiom schema of replacement, we obtain the theory \logic{Z}, however without context it is unclear whether the axiom of choice is included in \logic{Z} or not.

  The full list of axioms is:
  \begin{thmenum}
    \thmitem{def:zfc/extensionality} The \term{axiom of extensionality}, as defined in \fullref{def:naive_set_theory/extensionality}. This is also the only axiom of the theory that does not deal with existence.

    \thmitem{def:zfc/specification} The \term{axiom schema of specification}, also known as the axiom schema of \term{separation} or of \term{restricted comprehension}, states that, given a set \( A \), any formula defines a subset of \( A \).

    More precisely, for each formula \( \varphi \) containing neither \( \sigma \) nor \( \rho \) as free variables, but possibly containing \( \synx \), the following is an axiom:
    \begin{equation}\label{eq:def:zfc/specification}
      \qforall \sigma \qexists \rho \qforall \synx \parens[\Big]{ \synx \in \rho \syniff (\varphi \synwedge \synx \in \sigma) }.
    \end{equation}

    As explained in \fullref{def:naive_set_theory/unrestricted_comprehension} and \fullref{def:set_builder_notation}, this set may depend on parameters, which are other sets. We must formally take the \hyperref[def:universal_closure]{universal closure} of this formula to quantify over all possible values for the parameters.

    Compare this axiom to \hyperref[def:naive_set_theory/unrestricted_comprehension]{unrestricted comprehension}. Informally, restricted comprehension axiom can be obtained by taking the result of unrestricted comprehension and intersecting it with some set \( A \). As mentioned in \fullref{def:set_builder_notation}, in set-builder notation such a set can be denoted via
    \begin{equation*}
      \set[\Big]{ x \in B \given* \Bracks{\varphi}(x, u_1, \ldots, u_n) },
    \end{equation*}
    where \( \Bracks{\varphi}(\cdots) \) is the formula valuation function from \fullref{def:propositional_valuation/formula_valuation_function}.

    Unlike for unrestricted comprehension, some sets definable in the metatheory no longer have a corresponding set within the object logic.

    \thmitem{def:zfc/power_set} The \term{axiom of power sets} states that every set has a corresponding \hyperref[def:basic_set_operations/power_set]{power set}. In symbols:
    \begin{equation}\label{eq:def:zfc/power_set}
      \qforall \sigma \qexists \rho \ref{eq:def:basic_set_operations/power_set/predicate}[\rho, \sigma].
    \end{equation}

    \thmitem{def:zfc/union} The \term{axiom of unions} states that for every set there exists another set that is its \hyperref[def:basic_set_operations/union]{union}. In symbols:
    \begin{equation}\label{eq:def:zfc/union}
      \qforall \sigma \qexists \rho \ref{eq:def:basic_set_operations/union/predicate}[\rho, \sigma].
    \end{equation}

    \thmitem{def:zfc/pairing} The \term{axiom of pairing} states that for any sets \( A \) and \( B \) there exists an unordered pair containing them, i.e. another set that contains exactly \( A \) and \( B \). This is \( \set{ A, B } \) in set-builder notation. In symbols:
    \begin{equation}\label{eq:def:zfc/pairing}
      \qforall \sigma \qforall \tau \qexists \rho \qforall \synx \parens[\Big]{ \synx \in \rho \syniff ( \synx \syneq \sigma \synvee \synx \syneq \tau) }.
    \end{equation}

    \thmitem{def:zfc/infinity} The \term{axiom of infinity} states that an \hyperref[def:inductive_set]{inductive set} exists. In symbols:
    \begin{equation}\label{eq:def:zfc/infinity}
      \qexists \sigma \ref{eq:def:inductive_set/predicate}[\sigma].
    \end{equation}

    As will be shown in \fullref{thm:cumulative_hierarchy_model_of_zfc_without_infinity}, no other axioms of \logic{ZFC} ensure the existence of at least one infinite set.

    \thmitem{def:zfc/choice}\mcite[def. 6.4.1]{Hinman2005Logic} The \term{axiom of choice} states that a \hyperref[def:choice_function]{choice function} exists for any family of nonempty sets. In symbols:
    \begin{equation}\label{eq:def:zfc/choice}
      \qforall \sigma \qexists \rho \parens[\Bigg]
        {
          \ref{eq:def:indexed_family/predicate}[\rho]
          \wedge
          \ref{eq:def:indexed_family/predicate_dom}[\sigma]
          \wedge
          \qforall {\synx \in \sigma} \parens[\Big]{ \neg\ref{eq:def:empty_set/predicate}[\synx] \synimplies \qexists \syny (\syny \in \rho \synimplies \qExists \synz (\synz \in \synx \synwedge \ref{eq:def:kuratowski_pair/predicate}[\syny, \synx, \synz]) }
        }
    \end{equation}

    \Fullref{thm:carterian_product_universal_property_lemma} shows that, as long as one choice function exists, we can find another one that suits our needs.

    See \fullref{thm:axiom_of_choice_equivalences} for more statements equivalent to this axiom.

    \thmitem{def:zfc/replacement} The \term{axiom schema of replacement} roughly states that every \hyperref[rem:function_definition]{mapping} that is definable via a formula of \logic{ZFC} is a function. As we have done for the \hyperref[def:zfc/choice]{axiom of choice}, we only formulate the axiom via the image of the function. More concretely, given a formula \( \varphi \) not containing \( \sigma \) nor \( \rho \) as free variables, but possibly containing \( \synx \) and \( \syny \), the following is an axiom:
    \begin{equation}\label{eq:def:zfc/replacement}
      \qforall \sigma \parens[\Bigg]
        {
          \parens[\Big]{ \qforall {\synx \in \sigma} \qExists \syny \varphi }
          \synimplies
          \parens[\Big]{ \qexists \rho \qforall \syny \parens[\Big]{ \syny \in \rho \syniff \qexists {\synx \in \sigma} \varphi } }
        }.
    \end{equation}

    As is the case with the \hyperref[def:zfc/specification]{axiom schema of specification}, the formula \( \varphi \) may depend on parameters, in which case we use its \hyperref[def:universal_closure]{universal closure}.

    This axiom is useful in cases where a function can be defined via formulas, but its existence cannot be proven --- for example in \fullref{thm:hartogs_lemma}.

    This is the axiom that makes \logic{ZFC} require large models --- see \fullref{thm:cumulative_hierarchy_model_of_zfc}.

    \thmitem{def:zfc/foundation} The \term{axiom of foundation} states that every nonempty set contains a member disjoint from the set itself. In symbols:
    \begin{equation}\label{eq:def:zfc/foundation}
      \qforall \sigma \parens[\Big]
        {
          \synneg \ref{eq:def:empty_set/predicate}[\sigma]
          \synimplies
          \qexists {\rho \in \sigma} \synneg \qexists \synx \parens{ \synx \in \sigma \synwedge \synx \in \rho }
        }.
    \end{equation}

    This is a powerful axiom because it shows that set membership in \logic{ZFC} is well-founded --- see \fullref{thm:set_membership_is_well_founded}. It is equivalent to \fullref{thm:axiom_of_regularity} and is often itself called the \term{axiom of regularity}.
  \end{thmenum}
\end{definition}

\begin{proposition}\label{thm:zfc_existence_theorems}
  We will now prove that all sets we have considered up until now in \fullref{ch:set_theory} are sets in \hyperref[def:zfc]{\logic{ZFC}}. The uniqueness in all cases follows from the \hyperref[def:zfc/extensionality]{axiom of extensionality}.

  A very fundamental existence result is provided by the fact that we are assuming \hyperref[rem:standard_model_of_set_theory]{standard} and \hyperref[rem:transitive_model_of_set_theory]{transitive} models of \logic{ZFC}. Let \( \mscrV = (V, I) \) be such a model. Then, if \( v \in V \) and \( u \in v \), transitivity implies that \( u \in V \). Since the model is also standard, this shows that both \( u \) and \( v \) are sets within the object theory. Thus, if \( A \) is a set within the object theory and if \( B \in A \) within the metatheory, then necessarily \( B \) itself is a set within the object theory.

  For example, \fullref{thm:zfc_existence_theorems/set_of_functions} shows that the set \( \fun(A, B) \) of functions exists within the object theory for any two sets \( A \) and \( B \) in the object theory. Therefore, every single function between \( A \) and \( B \) is a set within the object theory because it is a member of \( \fun(A, B) \).

  With that in mind, we will show the following:

  \begin{thmenum}
    \thmitem{thm:zfc_existence_theorems/set_containing_all} There is no set that contains all sets.

    \thmitem{thm:zfc_existence_theorems/subset} If \( A \) is a set, then for any formula \( \varphi \) the set
    \begin{equation*}
      \set{ x \in A \given \Bracks{\varphi}(x) }.
    \end{equation*}
    exists and, furthermore, it is a \hyperref[def:subset]{subset} of \( A \).

    Only \hyperref[def:first_order_definability]{definable subsets} of \( A \) can be described in this way, however. See \fullref{thm:zfc_existence_theorems/power_set}.

    \thmitem{thm:zfc_existence_theorems/empty_set} There exists a unique \hyperref[def:empty_set]{empty set}.

    \thmitem{thm:zfc_existence_theorems/singleton} For every set \( A \), there exists a \hyperref[def:singleton_set]{singleton set} \( \set{ A } \) that contains only \( A \).

    \thmitem{thm:zfc_existence_theorems/arbitrary_intersection} For any \hi{nonempty} family \( \mscrA \), the \hyperref[def:basic_set_operations/intersection]{intersection} \( \bigcap \mscrA \) exists.

    \thmitem{thm:zfc_existence_theorems/binary_intersection} For any two sets \( A \) and \( B \), their \hyperref[def:basic_set_operations/intersection]{intersection} \( A \cap B \) exists.

    \thmitem{thm:zfc_existence_theorems/arbitrary_union} For any family \( \mscrA \), the \hyperref[def:basic_set_operations/union]{union} \( \bigcup \mscrA \) exists.

    \thmitem{thm:zfc_existence_theorems/binary_union} For any two sets \( A \) and \( B \), their \hyperref[def:basic_set_operations/union]{union} \( A \cup B \) exists.

    \thmitem{thm:zfc_existence_theorems/difference} For any two sets \( A \) and \( B \), their \hyperref[def:basic_set_operations/difference]{difference} \( A \setminus B \) exists.

    \thmitem{thm:zfc_existence_theorems/power_set} For any set \( A \), its \hyperref[def:basic_set_operations/power_set]{power set} \( \pow(A) \) exists.

    As a consequence, even subsets of \( A \) which are not \hyperref[def:first_order_definability]{definable} exist.

    \thmitem{thm:zfc_existence_theorems/successor} For any set \( A \), its \hyperref[def:ordinal_successor]{successor} \( \op{sc}(A) \) exists.

    \thmitem{thm:zfc_existence_theorems/indexed_family} For any index set \( \mscrK \), any \( \mscrK \)-\hyperref[def:indexed_family]{indexed family} \( \seq{ A_k }_{k \in \mscrK} \) exists.

    \thmitem{thm:zfc_existence_theorems/cartesian_product} For any indexed family \( \set{ A_k }_{k \in \mscrK} \), its \hyperref[def:cartesian_product]{Cartesian product} \( \prod_{k \in \mscrK} A_k \) exists.

    \thmitem{thm:zfc_existence_theorems/set_of_relations} For any two sets \( A \) and \( B \), the set of all relations between \( A \) and \( B \) exists.

    \thmitem{thm:zfc_existence_theorems/set_of_functions} For any two sets \( A \) and \( B \), the set \hyperref[def:function/set_of_functions]{\( \fun(A, B) \)} of functions between them exists.

    \thmitem{thm:zfc_existence_theorems/quotient_set} For any set \( A \) and any \hyperref[def:equivalence_relation]{equivalence relation} \( \cong \), the \hyperref[def:equivalence_relation/quotient]{quotient set} \( A / {\cong} \) exists.

    \thmitem{thm:zfc_existence_theorems/function_evaluation} Fix some sets \( A \) and \( B \) and some indexed family of functions \( \seq{ f_k }_{k \in \mscrK} \) where \( f_k: A \to B_k \) for \( k \in \mscrK \).

    For any \( x \in A \) the corresponding tuple \( \seq{ f_k(x) }_{k \in \mscrK} \) exists.
  \end{thmenum}
\end{proposition}
\begin{proof}
  \SubProofOf{thm:zfc_existence_theorems/set_containing_all} Suppose that \( V \) is a set containing all sets. Take the subset of \( V \) of all sets that do not belong to themselves. We restrict via the \hyperref[def:zfc/specification]{axiom schema of specification} our comprehension to a set \( V \) we know exists, but this nonetheless this leads to \fullref{thm:russels_paradox}. We can thus conclude that \( V \) does not exist.

  \SubProofOf{thm:zfc_existence_theorems/subset} This is a trivial consequence of the \hyperref[def:zfc/specification]{axiom schema of specification}.

  \SubProofOf{thm:zfc_existence_theorems/empty_set} As a consequence of the \hyperref[def:zfc/infinity]{axiom of infinity}, there exists at least one inductive set. Let \( A \) be an inductive set. Then from the \hyperref[def:zfc/specification]{axiom schema of specification} it follows that
  \begin{equation*}
    \set{ x \in A \given \synbot }
  \end{equation*}
  is a set. Furthermore, \( x \) belongs to this set if and only if \( x \) satisfies \( \synbot \), which is impossible, hence the set is empty.

  As a consequence of the \term{axiom of extensionality}, this empty set is unique. As discussed in \fullref{def:empty_set}, we denote this unique empty set by \( \varnothing \).

  \SubProofOf{thm:zfc_existence_theorems/singleton} Fix a set \( A \). The set \( \set{A} \), if it exists, is equal to \( \set{A} = \set{A, A} \), by the \hyperref[def:zfc/extensionality]{axiom of extensionality}.

  Thus, by the \hyperref[def:zfc/pairing]{axiom of pairing}, the singleton set \( \set{A} = \set{A, A} \) actually exists.

  \SubProofOf{thm:zfc_existence_theorems/arbitrary_intersection} Let \( \mscrA \) be a nonempty family of sets. Their intersection \( \bigcap \mscrA \), if it exists, is a subset of every set \( A \in \mscrA \).

  Therefore, since the family \( \mscrA \) is nonempty, the \hyperref[def:zfc/specification]{axiom schema of specification} applied to any set in \( A \in \mscrA \) guarantees the existence of the intersection \( \bigcap \mscrA \). More precisely, for any \( A_0 \in \mscrA \), we can define the intersection of \( A \) as
  \begin{equation*}
    \bigcap \mscrA = \set{ x \in A_0 \given \qexists {A \in \mscrA} x \in A }.
  \end{equation*}

  \SubProofOf{thm:zfc_existence_theorems/binary_intersection} For sets \( A \) and \( B \), by the \hyperref[def:zfc/pairing]{axiom of pairing} the set \( \set{ A, B } \) exists. Then by \fullref{thm:zfc_existence_theorems/arbitrary_intersection}, the binary intersection
  \begin{equation*}
    A \cap B = \bigcap \set{ A, B }
  \end{equation*}
  also exists.

  \SubProofOf{thm:zfc_existence_theorems/arbitrary_union} The existence of arbitrary unions is merely a restatement of the \hyperref[def:zfc/union]{axiom of unions}.

  \SubProofOf{thm:zfc_existence_theorems/binary_union} Similarly to \fullref{thm:zfc_existence_theorems/binary_intersection}, for sets \( A \) and \( B \), by the \hyperref[def:zfc/pairing]{axiom of pairing} the set \( \set{ A, B } \) exists and by the \hyperref[def:zfc/union]{axiom of unions}, the binary union
  \begin{equation*}
    A \cup B = \bigcup \set{ A, B }
  \end{equation*}
  exists.

  \SubProofOf{thm:zfc_existence_theorems/difference} The difference \( A \setminus B \) is guaranteed to exist by \hyperref[def:zfc/specification]{restricted comprehension}:
  \begin{equation*}
    A \setminus B = \set{ x \in A \given x \not\in B }.
  \end{equation*}

  \SubProofOf{thm:zfc_existence_theorems/power_set} The existence of power sets is a restatement of the \hyperref[def:zfc/power_set]{axiom of power sets}.

  \SubProofOf{thm:zfc_existence_theorems/successor} The successor of \( A \) is
  \begin{equation*}
    \op{sc}(A) = \set{ A } \cup A.
  \end{equation*}

  Its existence follows from \fullref{thm:zfc_existence_theorems/singleton} and \fullref{thm:zfc_existence_theorems/binary_union}.

  \SubProofOf{thm:zfc_existence_theorems/indexed_family} Given an indexed family \( \seq{ A_k }_{k \in \mscrK} \), consider the set
  \begin{equation*}
    \mscrA \coloneqq \set{ A_k \given k \in \mscrK }.
  \end{equation*}

  The family \( \seq{ A_k }_{k \in \mscrK} \) is, formally, a set of \hyperref[def:kuratowski_pair]{Kuratowski pairs}. Every pair \( \braket{ k, A_k } \) is itself a subset of \( \pow(\mscrK \bigcup \mscrA) \). The family is then a subset of \( \pow(\pow(\mscrK \bigcup \mscrA)) \). Applying the \hyperref[def:zfc/power_set]{axiom of power sets} again, we obtain that the family exits in \( \mscrV \).

  \SubProofOf{thm:zfc_existence_theorems/cartesian_product} By definition, the Cartesian product \( \prod_{k \in \mscrK} A_k \) is a set of families indexed by \( \mscrK \) whose terms are in the union \( \bigcup\set{ A_k \given k \in \mscrK } \).

  We can apply the \hyperref[def:zfc/power_set]{axiom of power sets} one more time in addition to those in \fullref{thm:zfc_existence_theorems/indexed_family} to obtain the set of all indexed by \( \mscrK \) families of members of this union. Then we can apply the \hyperref[def:zfc/specification]{axiom schema of specification} to restrict only to those families that satisfy the condition of \fullref{def:cartesian_product}.

  \SubProofOf{thm:zfc_existence_theorems/set_of_relations} All relations between \( A \) and \( B \) are subsets of \( A \times B \), hence elements of \( \pow(A \times B) \). The latter exists by \fullref{thm:zfc_existence_theorems/cartesian_product} and \fullref{thm:zfc_existence_theorems/power_set}.

  \SubProofOf{thm:zfc_existence_theorems/set_of_functions} The set of single-valued functions from \( A \) to \( B \) is a subset of \( \pow(A \times B) \), hence it exists by \fullref{thm:zfc_existence_theorems/set_of_relations} and \fullref{thm:zfc_existence_theorems/power_set}.

  \SubProofOf{thm:zfc_existence_theorems/quotient_set} Let \( A \) be an arbitrary set and \( \cong \) be a binary relation over \( A \). Then \( A / {\cong} \) is a subset of \( \pow(A) \) and hence it exists as a consequence of the \hyperref[def:zfc/power_set]{axiom of power sets} and the \hyperref[def:zfc/specification]{axiom schema of specification} --- see \fullref{thm:equivalence_partition/partition}.

  \SubProofOf{thm:zfc_existence_theorems/function_evaluation} The set \( \seq{ f_k }_{k \in \mscrK} \) exists because it is a member of \( \fun(\mscrK, \fun(A, B)) \), which set exists by \fullref{thm:zfc_existence_theorems/set_of_functions}.

   Then \( \seq{ f_k(x) }_{k \in \mscrK} \) is the function
   \begin{equation*}
     \begin{aligned}
       &g_x: \mscrK \to B, \\
       &g_x(k) \coloneqq f_k(x).
     \end{aligned}
   \end{equation*}
\end{proof}

\paragraph{Function invertibility}

\begin{theorem}[Existence of single-valued selections]\label{thm:existence_of_single_valued_selections}
  Every \hyperref[def:set_valued_map/partial]{total set-valued map} has a \hyperref[def:function/selection]{single-valued selection}.
\end{theorem}
\begin{comments}
  \item Within \hyperref[def:zfc]{\logic{ZF}}, this theorem is equivalent to the \hyperref[def:zfc/choice]{axiom of choice} --- see \fullref{thm:axiom_of_choice_equivalences/selection}.
\end{comments}
\begin{proof}
  \ImplicationSubProof[def:zfc/choice]{the axiom of choice}[thm:existence_of_single_valued_selections]{selection existence} Let \( F: A \multto B \) be a total set-valued map. Consider the indexed family \( \set{ F(a) }_{a \in A} \). It is a (potentially empty) family of nonempty sets. Thus, we can apply the axiom of choice to obtain some \hyperref[def:choice_function]{choice function}
  \begin{equation*}
    c: \set{ F(a) \given a \in A } \to B.
  \end{equation*}

  Define
  \begin{equation*}
    \begin{aligned}
      &f: A \to B, \\
      &f(a) \coloneqq c(F(a)).
    \end{aligned}
  \end{equation*}

  Then \( f \) is a selection of \( F \).

  \ImplicationSubProof[thm:existence_of_single_valued_selections]{selection existence}[def:zfc/choice]{axiom of choice} Suppose that every total set-valued map has a single-valued selection.

  Fix a family \( \mscrA \) of nonempty sets and define the function
  \begin{equation*}
    \begin{aligned}
      &F: \mscrA \to \bigcup \mscrA, \\
      &F(A) \coloneqq A
    \end{aligned}
  \end{equation*}
  that sends each set in \( \mscrA \) to the corresponding subset of \( \bigcup \mscrA \). In terms of relations, we have \( (A, x) \in \gph(F) \) if and only if \( x \in A \). This is a total set-valued map because every set in \( \mscrA \) is nonempty.

  Then every selection of \( F \) is a choice function for \( \mscrA \).
\end{proof}

\begin{theorem}[Surjective functions are right-invertible]\label{thm:surjective_functions_are_right_invertible}
  Every \hyperref[def:function_invertibility/surjective]{surjective function} is \hyperref[def:morphism_invertibility/right_invertible]{right-invertible}.
\end{theorem}
\begin{comments}
  \item Within \hyperref[def:zfc]{\logic{ZF}}, this theorem is equivalent to the \hyperref[def:zfc/choice]{axiom of choice} --- see \fullref{thm:axiom_of_choice_equivalences/surjective}.
\end{comments}
\begin{proof}
  Let \( f: A \to B \) be any function. Its \hyperref[def:set_valued_map/inverse]{inverse} \( f^{-1}: B \multto A \) is, by definition, a partial set-valued map.

  \ImplicationSubProof[def:zfc/choice]{the axiom of choice}[thm:epimorphisms_split_in_set]{right-invertibility} Suppose that \( f \) is surjective and thus satisfies \fullref{def:function_invertibility/surjective/inverse}, stating that its inverse is total.

  Then the axiom of choice via \fullref{thm:existence_of_single_valued_selections} gives us a \hyperref[def:function/selection]{single-valued selection} \( g \) of \( f^{-1} \).

  Since the value of \( f \) is \( y \) for all members of \( f^{-1}(y) \), and since \( g(y) \in f^{-1}(y) \), we have
  \begin{equation*}
    [f \bincirc g](y) = f(g(y)) = y.
  \end{equation*}

  Therefore, \( g \) is a right inverse of \( f \).

  \ImplicationSubProof[thm:epimorphisms_split_in_set]{right-invertibility}[def:zfc/choice]{the axiom of choice} Suppose that every surjective function is invertible.

  Let \( \mscrA \) be an arbitrary family of nonempty sets. Define the function
  \begin{equation*}
    \begin{aligned}
      &f: \coprod_{A \in \mscrA} A \to \mscrA, \\
      &f(A, x) \coloneqq A,
    \end{aligned}
  \end{equation*}
  where \( \coprod \) denotes the \hyperref[def:disjoint_union]{disjoint union}.

  This function is surjective, thus then there exists some right-inverse
  \begin{equation*}
    g: \mscrA \to \coprod_{A \in \mscrA} A.
  \end{equation*}

  Given a set \( A \) from the family \( \mscrA \), \( g(A) \) is a pair \( (A, x) \) with \( x \in A \). This allows us to define the choice function
  \begin{equation*}
    \begin{aligned}
      &c: \mscrA \to \bigcup \mscrA, \\
      &c(A) \coloneqq x \T{where} (A, x) = g(A).
    \end{aligned}
  \end{equation*}

  Since the family \( \mscrA \) was arbitrary, we can conclude that the axiom of choice holds.
\end{proof}

\begin{proposition}\label{thm:function_invertibility_categorical}
  In relation to \hyperref[def:morphism_invertibility]{morphism invertibility} in the category \hyperref[def:category_of_small_sets]{\( \cat{Set} \)}, we have the following:
  \begin{thmenum}
    \thmitem{thm:function_invertibility_categorical/empty} An \hyperref[def:set_valued_map/empty]{empty function} is always \hyperref[def:function_invertibility/injective]{injective} and \hyperref[def:morphism_invertibility/left_invertible]{left-invertible}, however only if its codomain is empty it is \hyperref[def:morphism_invertibility/left_cancellative]{left-cancellative}, \hyperref[def:morphism_invertibility/right_cancellative]{right-cancellative}, \hyperref[def:morphism_invertibility/right_invertible]{right-invertible} and \hyperref[def:function_invertibility/surjective]{surjective}.

    \thmitem{thm:function_invertibility_categorical/nonempty_left_invertible} A nonempty function is \hyperref[def:morphism_invertibility/left_invertible]{left-invertible} if and only if it is \hyperref[def:function_invertibility/injective]{injective}.

    \thmitem{thm:function_invertibility_categorical/left_cancellative} A function is \hyperref[def:morphism_invertibility/left_cancellative]{left-cancellative} if and only if it is \hyperref[def:function_invertibility/injective]{injective}.

    \thmitem{thm:function_invertibility_categorical/right_invertible} A function is \hyperref[def:morphism_invertibility/right_invertible]{right-invertible} if and only if it is \hyperref[def:function_invertibility/surjective]{surjective}.

    \thmitem{thm:function_invertibility_categorical/right_cancellative} A function is \hyperref[def:morphism_invertibility/right_cancellative]{right-cancellative} if and only if it is \hyperref[def:function_invertibility/surjective]{surjective}.

    \thmitem{thm:function_invertibility_categorical/fully_invertible} A function is \hyperref[def:function_invertibility/bijective]{bijective} if and only if it is \hyperref[def:morphism_invertibility/isomorphism]{fully invertible}.
  \end{thmenum}
\end{proposition}
\begin{comments}
  \item We prove this result here rather than in \fullref{sec:functions} because it requires \fullref{thm:surjective_functions_are_right_invertible}, which requires the axiom of choice that we have introduced in this section.
\end{comments}
\begin{proof}
  \SubProofOf{thm:function_invertibility_categorical/empty} Let \( g: \varnothing \to C \) be the empty function to \( C \).

  \begin{itemize}
    \item It is vacuously injective.
    \item It is also left-invertible because the only function that can be composed on the right with is the unique function from \( \varnothing \) to \( \varnothing \).
    \item Clearly \( g: \varnothing \to C \) is surjective if and only if \( C = \varnothing \).
    \item For left-invertibility, note that \( g: \varnothing \to C \) composed with the function \( h: C \to D \) on the left is another empty function \( h \bincirc g: \varnothing \to D \). The latter is the identity \( \id_\varnothing \) if and only if \( C = D = \varnothing \).
    \item For right-invertibility, note that \( g: \varnothing \to C \) composed with the function \( f: A \to \varnothing \) on the right is the function \( g \bincirc f: A \to C \). But \( A = \varnothing \) since otherwise \( f \) would be nonempty, hence \( g \bincirc f: \varnothing \to C \). The latter is the identity \( \id_\varnothing \) if and only if \( C = \varnothing \).
    \item For right-cancellation, note that \( g \bincirc f_1 = g \bincirc f_2 \) implies \( f_1 = f_2 \) if and only if \( B \) is empty.
  \end{itemize}

  \SubProofOf{thm:function_invertibility_categorical/nonempty_left_invertible}

  \SufficiencySubProof Let \( f: A \to B \) be a nonempty injective function. It thus satisfies \fullref{def:function_invertibility/injective/inverse}, hence the \hyperref[def:set_valued_map/inverse]{inverse} \( f^{-1}: B \to A \) is a partial single-valued function.

  Fix some value \( a \in A \) and define
  \begin{equation*}
    \begin{aligned}
      &g: B \to A \\
      &g(y) \coloneqq \begin{cases}
        f^{-1}(y), &y \in f(A) \\
        a,         &\T{otherwise.}
      \end{cases}
    \end{aligned}
  \end{equation*}

  This function \( g \) is a left inverse of \( f \) because, for any \( x \in A \),
  \begin{equation*}
    [g \bincirc f](x)
    =
    g(f(x))
    =
    f^{-1}(f(x))
    =
    x.
  \end{equation*}

  We can see that \( g \) is not unique because of our choice of \( a \). We may even define \( g \) to take different values in \( A \) outside \( f(A) \). Thus, \( g \) is non-unique in general. But at least one such function exists.

  \NecessitySubProof Conversely, suppose that \( f: A \to B \) is not necessarily injective and let \( g: B \to A \) be a left inverse of \( f \). Let \( x_1 \) and \( x_2 \) be two different points in \( A \). Since \( g \bincirc f = \id_A \), clearly \( g(f(x_1)) \neq g(f(x_2)) \). If we suppose that \( f(x_1) = f(x_2) \), we would obtain a contradiction since then \( g(f(x_1)) \) would equal \( g(f(x_2)) \). Hence, \( f(x_1) \neq f(x_2) \). This shows that \( f \) is injective.

  \SubProofOf{thm:function_invertibility_categorical/left_cancellative}

  \SufficiencySubProof* The case with an empty function is handled in \fullref{thm:function_invertibility_categorical/empty}, and we assume that it is nonempty.

  Suppose that \( g: B \to C \) is a nonempty left-cancellative function. Let \( y_1 \) and \( y_2 \) be some members of \( B \) such that \( g(y_1) = g(y_2) \).

  Suppose that \( y_1 \neq y_2 \) and define the function
  \begin{equation*}
    \begin{aligned}
      &f: B \to B \\
      &f(y) \coloneqq \begin{cases}
        y_2, &y = y_1 \\
        y_1, &y = y_2 \\
        y,   &y \neq y_1 \T{and} y \neq y_2
      \end{cases}
    \end{aligned}
  \end{equation*}

  Then
  \begin{equation*}
    g(f(y_2)) = g(y_1) = g(y_2) = g(f(y_1))
  \end{equation*}

  For all \( y \in B \) different from \( y_1 \) and \( y_2 \), we have \( y = f(y) \).

  Since \( g \) is left-cancellative, from \( g \bincirc \id_B = g \bincirc f \) it follows that \( \id_B = f \), which is a contradiction.

  It remains for \( y_1 \) to be equal to \( y_2 \). Since these were arbitrary points in \( B \) satisfying \( g(y_1) = g(y_2) \), we conclude that \( g \) is injective.

  \NecessitySubProof* Conversely, if \( f \) is injective, it is left-invertible by \fullref{thm:function_invertibility_categorical/nonempty_left_invertible} and left-cancellative by \fullref{thm:def:morphism_invertibility/split_monomorphism}.

  \SubProofOf{thm:function_invertibility_categorical/right_invertible}

  \SufficiencySubProof* In one direction, we have \fullref{thm:surjective_functions_are_right_invertible}.

  \NecessitySubProof* Conversely, suppose that \( g: B \to A \) is a right inverse of \( f: A \to B \). Let \( y \in B \). We have that \( g(y) \) is in the preimage of \( y \) under \( f \) because \( f(g(y)) = y \). Thus, the preimage is not empty for an arbitrary point in \( B \). We conclude that \( f \) is surjective.

  \SubProofOf{thm:function_invertibility_categorical/right_cancellative} The case with an empty function is handled in \fullref{thm:function_invertibility_categorical/empty}, and we assume that it is nonempty.

  \SufficiencySubProof* Let \( f: A \to B \) be a nonempty right-cancellative function. Suppose that it is not surjective and let \( y_0 \in B \setminus \img f \). Let \( z \) be some set not belonging to \( B \). Define the function
  \begin{equation*}
    \begin{aligned}
      &g: B \to B \cup \set{ z } \\
      &g(y) \coloneqq \begin{cases}
        z, &y = y_0 \\
        y, &y \neq y_0
      \end{cases}
    \end{aligned}
  \end{equation*}

  Since \( f \) is right-cancellative, from \( \id_B \bincirc f = g \bincirc f \) it follows that \( \id_B = g \), which is a contradiction. Therefore, \( f \) is surjective.

  \NecessitySubProof* We can prove the converse using \fullref{thm:function_invertibility_categorical/right_invertible} like we did for \fullref{thm:function_invertibility_categorical/left_cancellative}, however we prefer a direct proof that does not rely on the axiom of choice. As a bonus, this would allow us to prove \fullref{thm:epimorphisms_split_in_set}.

  Conversely, suppose that \( f: A \to B \) is surjective and that for some functions \( g_1, g_2: B \to C \) we have
  \begin{equation*}
    g_1 \bincirc f = g_2 \bincirc f.
  \end{equation*}

  Fix some \( y \in B \). Because \( f \) is surjective, there exists some \( x \in A \) such that \( f(x) = y \). Then
  \begin{equation*}
    g_1(y) = [g_1 \bincirc f](x) = [g_2 \bincirc f](x) = g_2(y).
  \end{equation*}

  Since \( y \in B \) was arbitrary, we conclude that \( f \) is right-cancellative.

  it is right-invertible by \fullref{thm:surjective_functions_are_right_invertible} and right-cancellative by \fullref{thm:def:morphism_invertibility/split_epimorphism}.

  \SubProofOf{thm:function_invertibility_categorical/fully_invertible}

  \SufficiencySubProof* If \( f: \varnothing \to B \) is a bijective empty function, then it is surjective and, by \fullref{thm:function_invertibility_categorical/right_invertible}, it is right-invertible. By \fullref{thm:function_invertibility_categorical/empty}, it is also left-invertible. Thus, it is fully invertible.

  \NecessitySubProof* Conversely, if an empty function \( f: \varnothing \to B \) is fully invertible, by \fullref{thm:function_invertibility_categorical/empty} we have \( A = B = \varnothing \) and hence it is bijective.

  Finally, if \( f: A \to B \) is \hi{nonempty} function, then by \fullref{thm:function_invertibility_categorical/nonempty_left_invertible} and \fullref{thm:function_invertibility_categorical/right_invertible} it is bijective if and only if it is fully invertible.

  In the bijective case, we can avoid the axiom of choice via \fullref{thm:surjective_functions_are_right_invertible} by noting that if \( f \) is bijective, its inverse is single-valued, and thus it is not necessary to do a selection of \( f^{-1} \).
\end{proof}

\begin{theorem}[Axiom of choice equivalences]\label{thm:axiom_of_choice_equivalences}
  The following statements are commonly referred to as \enquote{the} \hyperref[def:zfc/choice]{axiom of choice}:
  \begin{thmenum}[series=thm:axiom_of_choice_equivalences]
    \thmitem{thm:axiom_of_choice_equivalences/choice_sets} For every family of nonempty sets, \( \mscrA \) there exists a set \( B \) such that \( A \cap B \) is a singleton set for every \( A \in \mscrA \).

    \thmitem{thm:axiom_of_choice_equivalences/choice_function} Every family of nonempty sets has a corresponding \hyperref[def:choice_function]{choice function}.

    \thmitem{thm:axiom_of_choice_equivalences/choice_product} The \hyperref[def:cartesian_product]{Cartesian product} of a family of nonempty sets is nonempty.
  \end{thmenum}

  The following statements are equivalent to the axiom of choice, but are not conflated with it:
  \begin{thmenum}[resume=thm:axiom_of_choice_equivalences]
    \thmitem{thm:axiom_of_choice_equivalences/selection} \Fullref{thm:existence_of_single_valued_selections}: Every \hyperref[def:set_valued_map/partial]{total set-valued map} has a \hyperref[def:function/selection]{single-valued selection}.

    \thmitem{thm:axiom_of_choice_equivalences/surjective} \Fullref{thm:surjective_functions_are_right_invertible}: Every \hyperref[def:function_invertibility/surjective]{surjective function} is \hyperref[def:morphism_invertibility/right_invertible]{right-invertible}.

    \thmitem{thm:axiom_of_choice_equivalences/epimorphisms} \Fullref{thm:epimorphisms_split_in_set}: Every \hyperref[def:morphism_invertibility/right_cancellative]{epimorphism} in \hyperref[def:category_of_small_sets]{\( \cat{Set} \)} splits.

    \thmitem{thm:axiom_of_choice_equivalences/fully_faithful_essentially_surjective} \Fullref{thm:fully_faithful_and_essentially_surjective_functor_induces_equivalence}: Every \hyperref[def:functor_invertibility/fully_faithful]{fully faithful} and \hyperref[def:functor_invertibility/surjective_on_objects]{essentially surjective on objects} functor induces a \hyperref[def:category_equivalence]{category equivalence}.

    \thmitem{thm:axiom_of_choice_equivalences/skeletons} \Fullref{thm:category_skeleton_existence}: Every \hyperref[def:category]{category} has a \hyperref[def:skeletal_category]{skeleton}.

    \thmitem{thm:axiom_of_choice_equivalences/well_ordering} \Fullref{thm:well_ordering_theorem}: Every \hyperref[def:set]{set} can be \hyperref[def:well_ordered_set]{well-ordered}.

    \thmitem{thm:axiom_of_choice_equivalences/zorns_lemma} \Fullref{thm:zorns_lemma}: If every \hyperref[def:partial_order_chain]{chain} in a \hyperref[def:partially_ordered_set]{partially ordered set} has an \hyperref[def:extremal_points/bounds]{upper bound}, then the entire set has a \hyperref[def:extremal_points/maximal_and_minimal_element]{maximal element}.

    \thmitem{thm:axiom_of_choice_equivalences/maximal_ideal} \Fullref{thm:maximal_ideal_theorem}: Every proper \hyperref[def:semiring_ideal]{semiring ideal} is contained in a \hyperref[def:semiring_ideal/maximal]{maximal ideal}.

    \thmitem{thm:axiom_of_choice_equivalences/vector_space_bases} \Fullref{thm:vector_space_basis_existence}: Every \hyperref[def:vector_space]{vector space} has a \hyperref[def:hamel_basis]{basis}.

    \thmitem{thm:axiom_of_choice_equivalences/tychonoff} \Fullref{thm:tychonoffs_product_theorem}: The \hyperref[def:topological_product]{topological product} of \hyperref[def:compact_space]{compact spaces} is compact.
  \end{thmenum}
\end{theorem}
\begin{proof}
  The equivalence proofs can be found in the linked theorems since that is usually the most appropriate place to put them.
\end{proof}

\begin{theorem}[Diaconescu-Goodman-Myhill theorem]\label{thm:diaconescu_goodman_myhill_theorem}\mcite{GoodmanMyhill1978AOC}
  In \hyperref[def:zfc]{\logic{ZF}}, the \hyperref[def:zfc/choice]{axiom of choice} entails the law of the excluded middle \eqref{eq:thm:classical_tautologies/lem}.
\end{theorem}
\begin{comments}
  \item \incite[corr. 2]{Diaconescu1975AOC} formulated a more abstract result, which \incite{GoodmanMyhill1978AOC} later vastly simplified for the case of intuitionistic set theory. Our proof is based on the one given by \incite[thm. 5.3.3.6]{Mimram2020Types}, which in turn is based on the later.
\end{comments}
\begin{proof}
  Fix any closed formula \( \varphi \) and consider the sets
  \begin{align*}
    A \coloneqq \set[\Big]{ x \in \set{ T, F } \given* \Bracks{(\synx \doteq \syny) \synvee \varphi}(x, T) }
    &&\T{and}&&
    B \coloneqq \set[\Big]{ x \in \set{ T, F } \given* \Bracks{(\synx \doteq \syny) \synvee \varphi}(x, F) },
  \end{align*}
  where \( T \) and \( F \) are the Boolean values discussed in \fullref{con:boolean_value}, which we may, for simplicity, regard as \( \set{ \varnothing } \) and \( \varnothing \).

  Both are not empty since \( T \in A \) and \( F \in B \). Since the axiom of choice holds, there exists a choice function \( c: \set{ A, B } \to \set{ T, F } \).

  We have the following possibilities:
  \begin{itemize}
    \item If \( c(A) = F \), then
    \begin{equation*}
      T = \Bracks{(\synx \doteq \syny) \synvee \varphi}(c(A), T) = \underbrace{\Bracks{\synx \doteq \syny}(c(A), T)}_{F} \vee \Bracks{\varphi},
    \end{equation*}
    thus \( \Bracks{\varphi} = T \).

    \item If \( c(B) = T \), then similarly \( \Bracks{\varphi} = T \).

    \item If \( c(A) = T \) and \( c(B) = F \), then \( A \neq B \), and we have the following possibilities:
    \begin{itemize}
      \item If \( A = \set{ T } \), then
      \begin{equation*}
        F = \Bracks{(\synx \doteq \syny) \synvee \varphi}(c(A), F) = \underbrace{\Bracks{\synx \doteq \syny}(c(A), F)}_{F} \vee \Bracks{\varphi},
      \end{equation*}
      hence \( \Bracks{\varphi} = F \).

      \item If \( B = \set{ F } \), we similarly conclude that \( \Bracks{\varphi} = F \).
    \end{itemize}
  \end{itemize}

  In all cases, either \( \Bracks{\varphi} = T \) or \( \Bracks{\neg \varphi} = \oline{\Bracks{\varphi}} = T \), hence
  \begin{equation*}
    \Bracks{\varphi \synvee \neg \varphi} = T.
  \end{equation*}

  This is precisely \eqref{eq:thm:classical_tautologies/lem}.
\end{proof}

\begin{definition}\label{def:well_founded_relation}
  If \( \prec \) is simply a binary relation without any additional conditions imposed, we can still make sense of the chain conditions from \fullref{def:chain_condition}. We will restrict ourselves to the descending chain condition --- we call the relation \( \prec \) \term{well-founded} if any of the following equivalent conditions hold:

  \begin{thmenum}
    \thmitem{def:well_founded_relation/minimal}\mcite[241]{Enderton1977Sets} \Fullref{def:chain_condition/maximal} can be restated as follows: every nonempty subset of \( P \) has a minimal element \( m \) in the sense that there does not exist any other element \( x \) from \( P \) such that \( x \prec m \).

    \thmitem{def:well_founded_relation/stabilization} \Fullref{def:chain_condition/stabilization} can be restated as follows: every nonstrict descending sequence
    \begin{equation}\label{eq:def:well_founded_relation/stabilization}
      \cdots \prec x_3 \prec x_2 \prec x_1
    \end{equation}
    stabilizes in the sense that there exists an index \( n \) such that \( x_k = x_n \) for \( k > n \).

    \thmitem{def:well_founded_relation/infinite} \Fullref{def:chain_condition/infinite} can be restarted as follows: there exists no strictly infinitely ascending descending sequence in \( P \), i.e. a sequence of the form \eqref{eq:def:well_founded_relation/stabilization} such that no two members are equal.
  \end{thmenum}
\end{definition}

\begin{proposition}\label{thm:set_membership_is_well_founded}
  Set membership is \hyperref[def:well_founded_relation]{well-founded} in \logic{ZFC}. More precisely, given a set \( A \), if we regard \( \in \) as a binary relation between members of \( A \), then we would obtain that \( \in \) is a well-founded relation.
\end{proposition}
\begin{proof}
  The empty set is vacuously well-founded, so suppose that \( A \) is nonempty.

  Aiming at a contradiction, suppose also that \( \in \) is not well-founded on \( A \). Then there exists an strictly descending sequence \( \seq{ x_k }_{k=1}^\infty \subseteq A \) such that
  \begin{equation*}
    \cdots \in x_3 \in x_2 \in x_1.
  \end{equation*}

  Denote by \( B \) the set \( \set{ x_k \given k = 1, 2, \ldots } \). By the \hyperref[def:zfc/foundation]{axiom of foundation}, \( B \) contains a set \( C \) which is disjoint from \( B \).

  Clearly \( C \) coincides with at least one of \( x_1, x_2, \ldots \). Let \( C = x_{k_0} \). Since \( x_{k_0} \cap B = \varnothing \), then \( x_{k_0 + 1} \) is either not a member of \( x_{k_0} \) or of \( B \). But it is a member of both by our assumption that the sequence is infinitely descending.

  The obtained contradiction proves that \( \in \) is well-founded on \( A \).
\end{proof}

\begin{corollary}\label{thm:simple_foundation_theorems}
  We list several statements, which are consequences of the \hyperref[def:zfc/foundation]{axiom of foundation} via \fullref{thm:set_membership_is_well_founded}:

  \begin{thmenum}
    \thmitem{thm:simple_foundation_theorems/member_of_itself} No set is a member of itself.

    \thmitem{thm:simple_foundation_theorems/mutual_membership} For any two sets \( A \) and \( B \), it is not possible for both \( A \in B \) and \( B \in A \) to hold.

    \thmitem{thm:simple_foundation_theorems/ordinal_successor_is_injective} The \hyperref[def:ordinal_successor]{ordinal successor operation} is injective. That is, for every two set \( A \) and \( B \) from \( \op{sc}(A) = \op{sc}(B) \) it follows that \( A = B \).
  \end{thmenum}
\end{corollary}
\begin{proof}
  \SubProofOf{thm:simple_foundation_theorems/member_of_itself} The sequence
  \begin{equation*}
    \cdots A \in A \in A
  \end{equation*}
  is infinitely descending, hence \( A \) cannot be a member of itself.

  \SubProofOf{thm:simple_foundation_theorems/mutual_membership} The sequence
  \begin{equation*}
    \cdots A \in B \in A \in B \in A
  \end{equation*}
  is infinitely descending, hence \( A \in B \) and \( B \in A \) cannot simultaneously hold.

  \SubProofOf{thm:simple_foundation_theorems/ordinal_successor_is_injective} Note that \( A \in \op{sc}(A) = \op{sc}(B) \), hence either \( A = B \) or \( A \in B \). We can analogously conclude that either \( A = B \) or \( B \in A \).

  But \( A \in B \) and \( B \in A \) cannot simultaneously hold due to \fullref{thm:simple_foundation_theorems/mutual_membership}. Hence, it remains for \( A = B \) to hold.
\end{proof}

\begin{theorem}[Well-founded induction]\label{thm:well_founded_induction}
  Let \( \mscrL \) be the \hyperref[def:first_order_language]{first-order language} with no functional symbols and a single predicate symbol \( \prec \). We have already mentioned that we cannot formalize the concept of well-foundedness in first-order logic alone, so we will work with \hyperref[def:first_order_structure]{structures} directly.

  Every \hyperref[def:well_founded_relation]{well-founded} structure \( \mscrX = (X, I) \) over \( \mscrL \) satisfies an inductive axiom schema. For every formula \( \varphi \) over \( \mscrL \) not containing \( \syny \) nor \( \synz \) as free variables, \( X \) satisfies
  \begin{equation}\label{eq:thm:well_founded_induction}
    \qforall \syny
    \parens[\Big]
      {
        \overbrace
          {
            \underbrace{ \parens[\Big]{ \qforall {\synz \prec \syny} \varphi[\synx \mapsto \synz] } }_{\mathclap{\substack{\T{inductive} \\ \T{hypothesis}}}}
            \synimplies
            \underbrace{ \varphi[\synx \mapsto \syny] }_{\mathclap{\substack{\T{inductive step} \\ \T{conclusion}}}}
          }^{\T{inductive step}}
      }
    \synimplies
    \underbrace{ \qforall \syny \varphi[\synx \mapsto \syny] }_{\T{conclusion}}.
  \end{equation}
\end{theorem}
\begin{comments}
  \item See the comments in \fullref{def:peano_arithmetic/PA3} regarding variables and quantification in axiom schemas and \fullref{con:induction} for a general discussion of induction.
  \item In the special case where \( X = \BbbN \), this is called \term{strong induction} compared to the usual natural number induction \eqref{eq:def:peano_arithmetic/PA3}. This is discussed in \fullref{con:induction/well_founded}.
\end{comments}
\begin{proof}
  Fix a formula \( \varphi \) in \( \mscrL \). Fix a variable assignment \( v \) in \( X \). We will show that the \hyperref[def:conditional_formula/contrapositive]{contraposition} of \eqref{eq:thm:well_founded_induction} holds in this model.

  Suppose that there exists a value \( x \in X \) such that \( \Bracks{\varphi}_{v_{\synx \mapsto x}} = F \), that is,
  \begin{equation*}
    \qexists \syny \synneg \varphi[\synx \mapsto \syny].
  \end{equation*}
  holds.

  Let \( x_0 \) be the smallest such value (which is guaranteed to exist because \( X \) is well-founded by \( \prec \)). Thus, the inductive hypothesis \( \qforall {\synz < x_0} \varphi[\synx \mapsto \synz] \) holds, but the inductive step conclusion \( \varphi[\synx \mapsto x_0] \) does not.

  Since \( v \) was chosen arbitrarily, this is true for all variable assignments in \( X \). Then the following formula is \hyperref[def:first_order_model]{valid} in \( \mscrX \):
  \begin{equation}\label{eq:thm:well_founded_induction/contraposition}
    \qexists \syny \synneg \varphi[\synx \mapsto \syny]
    \synimplies
    \qexists \syny
    \parens[\Big]
      {
        \qforall {\synz \prec \syny} \varphi[\synx \mapsto \synz] \synwedge \synneg \varphi[\synx \mapsto \syny]
      }
  \end{equation}

  This is precisely the contrapositive of \eqref{eq:thm:well_founded_induction}. Since we are working in classical logic, the contrapositive is semantically equivalent to its original implication, hence \( \eqref{eq:thm:well_founded_induction} \) is also valid in \( \mscrX \).
\end{proof}

\begin{theorem}[Epsilon induction]\label{thm:epsilon_induction}
  For every formula \( \varphi \) in the language of set theory not containing \( \syny \) nor \( \synz \) as free variables, the following is a theorem of \logic{ZFC}:
  \begin{equation*}
    \qforall \syny
    \parens[\Big]
      {
        \overbrace
          {
            \underbrace{ \parens[\Big]{ \qforall {\synz \in \syny} \varphi[\synx \mapsto \synz] } }_{\mathclap{\substack{\T{inductive} \\ \T{hypothesis}}}}
            \synimplies
            \underbrace{ \varphi[\synx \mapsto \syny] }_{\mathclap{\substack{\T{inductive step} \\ \T{conclusion}}}}
          }^{\T{inductive step}}
      }
    \synimplies
    \underbrace{ \qforall \syny \varphi[\synx \mapsto \syny] }_{\T{conclusion}}.
  \end{equation*}
\end{theorem}
\begin{comments}
  \item This induction schema is called \enquote{\( \varepsilon \)-induction} because the set membership symbol \( \in \) is derived from \( \varepsilon \) as explained in \fullref{rem:epsilon_and_set_membership}.

  \item See the comments in \fullref{def:peano_arithmetic/PA3} regarding variables and quantification in axiom schemas and \fullref{con:induction} for a general discussion of induction.
\end{comments}
\begin{proof}
  Every model of \logic{ZFC} is well-founded by \( \in \) as a consequence of \fullref{thm:set_membership_is_well_founded}. The corollary then follows from \fullref{thm:well_founded_induction}.
\end{proof}

  \section{Ordinals}\label{sec:ordinals}

\paragraph{Infinity}

\begin{concept}\label{con:transfinum}\mcite[ch. I.5]{Кантор1985Труды}
  \incite*{Cantor1883Mannigfaltigkeitslehre}\fnote{We cite the Russian translation \cite[73]{Кантор1985Труды}.} attacks the maxima \enquote{infinitum actu non datur} (Latin for \enquote{infinity does not actually exist}), which he himself traces back to Aristotle. Cantor claims that Aristotle only acknowledged the existence of finite numbers because it was not known at his time how to operate on infinite sets.

  Aristotle's distinction between \term[en=actual infinity (\cite[196]{Kleene2002Logic})]{actual} and \term[ru=потенциальная бесконечность, en=potential infinite (\cite[196]{Kleene2002Logic})]{potential infinity} could be reworded as the existence of sets which are not finite. We could, for example, starting with zero, generate new \hyperref[def:natural_number]{natural numbers} on demand --- if desired --- potentially forever. But no set of natural numbers could exist as it would be an instance of actual infinity. The latter was rejected as essentially undefinable.

  Cantor suggested, based on the idea of \hyperref[def:dedekind_cut]{Dedekind cuts}, to introduce \enquote{number classes}, which would be sets of natural numbers and smaller number classes. These number classes are now known more concretely as \hyperref[def:ordinal]{ordinals}. Thus, Cantor did not aim to formalize actual infinity; he introduced sets strictly between those that are finite and those that would be actually infinite.

  He called this concept \term{transfinitum} with \enquote{suprafinitum} suggested as an alternative. For the first number class, he introduced the symbol \( \omega \) to distinguish is from the \hyperref[rem:lemniscate_symbol]{lemniscate} used to denote actual infinity. Both the word \enquote{transfinitum} and the symbol \( \omega \) are in wide use today.
\end{concept}

\begin{remark}\label{rem:lemniscate_symbol}
  We use the \term{lemniscate} symbol \( \infty \) is used to denote \hyperref[con:infinity]{infinity} in its different manifestations. The symbol is attributed to John Wallis by \incite{MartinLöf1990Infinity}, where the latter also remarks that
  \begin{displayquote}
    \textellipsis we take great pains to explain that \( \infty \) makes no sense by itself, that is, is no detachable part of the notation for a limit, an infinite sum, product or the like \textellipsis
  \end{displayquote}

  \incite{Cantor1883Mannigfaltigkeitslehre}\fnote{We cite the Russian translation \cite[92]{Кантор1985Труды}.} argues that this usage refers to \hyperref[con:transfinite]{potential infinity}, which explains why \( \infty \) has no well-defined meaning in general. For reasons of compatibility with existing notation, we use the symbol for the use cases Martin-L\"of described, for example
  \begin{equation*}
    \sum_{k=1}^\infty a_k.
  \end{equation*}

  Cantor prefers \( \omega \) for denoting actual infinity; we now regard \( \omega \) as the \hyperref[thm:omega_is_an_ordinal]{smallest infinite ordinal} \( \omega \) (which is the initial ordinal of the \hyperref[def:aleph_hierarchy]{smallest infinite cardinal} \( \aleph_0 \)).

  Even though we try to be concrete regarding infinity, we still want to retain usage of familiar notation, for which reason we sometimes identify, notation-wise, \( \infty \) with \( \omega \). For example, we denote \hyperref[def:sequence_space]{sequence spaces} by \( R^\infty \) rather than \( R^\omega \).

  As a meaningful standalone symbol, we will use \( \infty \) for positive and negative infinity within the \hyperref[def:extended_real_numbers]{extended real numbers}.
\end{remark}

\paragraph{Ordinals}

\begin{remark}\label{rem:ordinal_definition}
  Ordinals are generalizations of \hyperref[def:natural_numbers]{natural numbers}. We will find characterizing properties of the natural numbers (defined as members of \hyperref[thm:smallest_inductive_set_existence]{\( \omega \)}), so that it is clear what we want to generalize.

  Every natural number is defined as a set of other natural numbers:
  \begin{align*}
    &0 = \varnothing \\
    &1 = \set{ \varnothing } = \set{ 0 } \\
    &2 = \set{ \varnothing, \set{ \varnothing } } = \set{ 0, 1 } \\
    &3 = \set{ \varnothing, \set{ \varnothing }, \set{ \varnothing, \set{ \varnothing } } } = \set{ 0, 1, 2 }
  \end{align*}

  It just, so happens that each natural number \( n \) is the set of natural numbers that are smaller with respect to the strict order relation \( < \) defined in \eqref{eq:def:natural_numbers_ordering/strict_predicate}.

  Therefore, \( \in \) and \( < \) are equivalent on the set \( \omega \). It follows from \fullref{thm:natural_numbers_are_well_ordered} that \( \in \) is a \hyperref[def:totally_ordered_set]{strict total order} on \( \omega \). It is even \hyperref[def:well_ordered_set]{well-ordered} by \( \in \), but the latter condition is redundant due to \fullref{thm:set_membership_is_well_founded}.

  For an arbitrary set \( A \), set membership is not even a \hyperref[def:strict_partial_order]{strict partial order} --- irreflexivity is implied by \fullref{thm:set_membership_is_well_founded}, but transitivity of \( \in \) as a binary relation on \( A \) fails to hold in general, not speaking about trichotomy.

  A very simple counterexample for transitivity of \( \in \) is the set \( \set{ \varnothing, \set{ \varnothing }, \set{ \set{ \varnothing } } } \). Clearly \( \varnothing \in \set{ \varnothing } \) and \( \set{ \varnothing } \in \set{ \set{ \varnothing } } \), but \( \varnothing \not\in \set{ \set{ \varnothing } } \).

  In order for a set \( A \) to be a member of \( \omega \), it is not sufficient for \( \in \) to be a strict total ordered on \( A \). Except for the members of \( \omega \), another set that is totally ordered by \( \in \) is \( A = \set{ 0, 2, 4 } \).

  If we require \( A = \set{ 0, 2, 4 } \) to be a \hyperref[def:transitive_set]{transitive set}, however, it will be a natural number. Indeed, since \( 4 \) is a member of \( A \) and \( 1 \) and \( 3 \) are members of \( 4 \), then by adding \( 1 \) and \( 3 \) to \( A \) we obtain the set \( \set{ 0, 1, 2, 3, 4 } \), which by our definition of natural numbers is \( 5 \).

  note that transitivity of the relation \( \in \) on \( A \) and transitivity of the set \( A \) itself are entirely different concepts, although we will use both. Every member of \( \omega \) is a transitive set by \fullref{thm:omega_is_transitive} and the relation \( \in \) is a strict total order by \fullref{thm:natural_numbers_are_well_ordered}.

  This is the reasoning behind our definition of an ordinal --- \fullref{def:ordinal}. From this definition it will follow that the ordinals are unique representatives of order-isomorphisms classes of well-ordered sets.

  As a final note, the above two conditions are not sufficient for \( A \) to be a member of \( \omega \) (they are too general), but if we additionally require \( A \) to be a \hyperref[def:set_finiteness]{finite set}, then \( A \) will be a member of \( \omega \). We have yet to define finiteness, however.
\end{remark}

\begin{definition}\label{def:ordinal}\mcite[subsec. 6.3.1(i)]{Hinman2005Logic}
  An \term{ordinal number} or simply \term{ordinal} is a \hyperref[def:transitive_set]{transitive set} \( A \) such that set membership (as a binary relation on \( A \)) \hyperref[def:well_ordered_set]{well-orders} \( A \). By tradition, ordinals are denoted by initial small Greek letters like \( \alpha \) and \( \beta \).

  Because of \fullref{thm:set_membership_is_well_founded}, it is sufficient for set membership to be a \hyperref[def:totally_ordered_set]{strict total order} on \( A \). Since well-foundedness also implies \hyperref[def:binary_relation/irreflexive]{irreflexivity}, it follows that set membership must only be \hyperref[def:binary_relation/transitive]{transitive} and satisfy \hyperref[def:binary_relation/trichotomy]{trichotomy} on \( A \).

  In the absence of the \hyperref[def:zfc/foundation]{axiom of foundation}, we additionally require set membership to be a \hyperref[def:well_founded_relation]{well-founded relation} on \( A \), so that \( A \) is well-ordered.

  See \fullref{rem:ordinal_definition} for a further discussion of the definition, especially the different notions of transitivity.

  We introduce the notation \( \alpha < \beta \) for \( \alpha \in \beta \) in analogy with natural numbers. This is not a binary relation since there is no set of all ordinals by \fullref{thm:burali_forti_paradox}, however it does satisfy the properties of a well-order due to \fullref{thm:ordinals_are_well_ordered/trichotomy}.

  Finally, we introduce the following \hyperref[con:formula_defined_predicate]{formula-defined predicate}
  \begin{equation*}\taglabel[\op{IsOrdinal}]{eq:def:ordinal/predicate}
    \begin{aligned}
      \ref{eq:def:ordinal/predicate}[\tau] \coloneqq
        &\ref{eq:def:transitive_set/predicate}[\tau]
        \synwedge \\ \synwedge&
        \parens[\Big]
        {
          \qforall {\synx \in \tau}
          \qforall {\syny \in \tau}
          \parens[\Big]
            {
              \syny \in \synx \synvee \syny \syneq \synx \synvee \synx \in \syny
            }
        }
        \synwedge \\ \synwedge&
        \parens[\Big]
        {
          \qforall {\synx \in \tau}
          \qforall {\syny \in \tau}
          \qforall {\synz \in \tau}
          \parens[\Big]
          {
            (\synx \in \syny \synwedge \syny \in \synz) \synimplies \synx \in \synz
          }
        }.
    \end{aligned}
  \end{equation*}
\end{definition}

\begin{proposition}\label{thm:omega_is_an_ordinal}
  The \hyperref[thm:smallest_inductive_set_existence]{smallest inductive set} \( \omega \) is an \hyperref[def:ordinal]{ordinal}.
\end{proposition}
\begin{proof}
  From \fullref{thm:omega_is_transitive} it follows that \( \omega \) is a transitive set.

  Also, as discussed in \fullref{rem:ordinal_definition}, from \fullref{thm:natural_numbers_are_well_ordered} it follows that set membership is a strict total order on \( \omega \).

  Therefore, \( \omega \) is an \hyperref[def:ordinal]{ordinal}.
\end{proof}

\begin{proposition}\label{thm:member_of_ordinal_is_ordinal}
  Every member of an ordinal is an ordinal.
\end{proposition}
\begin{proof}
  Let \( \alpha \) be an ordinal and let \( \beta \in \alpha \). We will show that \( \beta \) is an ordinal.

  By transitivity of \( \alpha \), we have \( \beta \subseteq \alpha \), and thus \( (\beta, \in) \) is a (strictly) totally ordered set as a \hyperref[def:first_order_substructure]{substructure} of \( (\alpha, \in) \).

  It remains to show that \( \beta \) is itself transitive. Let \( x \in \beta \). We have that \( \beta \subseteq \alpha \) since \( \alpha \) is transitive, hence \( x \in \alpha \).

  Fix \( y \in x \). Again from the transitivity of \( \alpha \) it follows that \( y \in \alpha \). Also, \( \in \) is a total order on \( \alpha \) and hence from \( y \in x \) and \( x \in \beta \) it follows that \( y \in \beta \).

  Since \( y \in x \) was chosen arbitrarily, it follows that \( x \subseteq \beta \). Since \( x \) was chosen arbitrarily, it follows that \( \beta \) is transitive.
\end{proof}

\begin{proposition}\label{thm:initial_segment_of_ordinal}
  Let \( \alpha \) be an ordinal. For any \( \beta \in \alpha \), the \hyperref[def:order_interval/unbounded]{open initial segment} \( \alpha_{<\beta} \) equals \( \beta \).

  This is the bounded version of \fullref{thm:ordinal_is_set_of_smaller_ordinals}.
\end{proposition}
\begin{proof}
  Let \( \beta \in \alpha \). Consider the initial segment
  \begin{equation*}
    \alpha_{<\beta} = \set{ \gamma \in \alpha \given \gamma \in \beta }.
  \end{equation*}

  Clearly \( \alpha_{<\beta} = \alpha \cap \beta \). Given that \( \alpha \) is a transitive set, however, we have \( \beta \subseteq \alpha \) and thus \( \alpha \cap \beta = \beta \).

  Therefore, \( \alpha_{<\beta} = \beta \).
\end{proof}

\begin{corollary}\label{thm:natural_numbers_are_ordinals}
  The natural numbers (as members of \hyperref[thm:smallest_inductive_set_existence]{\( \omega \)}) are ordinals.
\end{corollary}
\begin{proof}
  Follows from \fullref{thm:omega_is_an_ordinal} and \fullref{thm:member_of_ordinal_is_ordinal}.
\end{proof}

\begin{definition}\label{def:transfinite_sequence}
  For any ordinal \( \alpha \) we call any function with \( \alpha \) as its domain a \( \alpha \)-indexed \term{transfinite sequence}.

  In particular, the case \( \alpha = \omega \) corresponds to the standard \hyperref[def:sequence]{natural number sequences}.
\end{definition}

\begin{definition}\label{def:truncated_sequence}\mimprovised
  Given an \hyperref[def:ordinal]{ordinal} \( \alpha \), for every \( \beta < \alpha \), we say that the \hyperref[def:transfinite_sequence]{transfinite sequence} \( \seq{ x_\beta }_{\gamma < \alpha} \) is a \term{truncation} of \( \seq{ x_\gamma }_{\gamma < \alpha} \).
\end{definition}

\begin{theorem}[Bounded transfinite induction]\label{thm:bounded_transfinite_induction}
  For every formula \( \varphi \) in the language of set theory not containing \( \tau \), \( \syny \) nor \( \synz \) as free variables, the following is a theorem of \logic{ZFC}:
  \small
  \begin{equation*}
    \qforall \tau
    \parens[\Bigg]
    {
      \ref{eq:def:ordinal/predicate}[\tau]
      \synimplies
      \parens[\Big]
        {
          \qforall {\syny \in \tau}
          \overbrace
            {
              \underbrace{ \parens[\Big]{ \qforall {\synz \in \syny} \varphi[\synx \mapsto \synz] } }_{\mathclap{\substack{\T{inductive} \\ \T{hypothesis}}}}
              \synimplies
              \underbrace{ \varphi[\synx \mapsto \syny] }_{\mathclap{\substack{\T{inductive step} \\ \T{conclusion}}}}
            }^{\T{inductive step}}
        }
      \synimplies
      \underbrace{ \qforall {\syny \in \tau} \varphi[\synx \mapsto \syny] }_{\T{conclusion}}
    }.
  \end{equation*}
  \normalsize

  See the comments in \fullref{def:peano_arithmetic/PA3} regarding variables and quantification in axiom schemas and \fullref{con:induction} for a general discussion of induction.

  See \fullref{rem:transfinite_induction} about a reformulation that is often useful in practice.
\end{theorem}
\begin{proof}
  This theorem is a special case of \fullref{thm:epsilon_induction} with the formula \( \synx \in \tau \synimplies \varphi \) that is explicitly universally quantified by the parameter \( \tau \) which ranges over all ordinals.

  Note that it is unnecessary to verify that \( \syny \) and \( \synz \) are ordinals because \fullref{thm:member_of_ordinal_is_ordinal} ensures that \( \syny \) is only quantified over ordinals.
\end{proof}

\begin{theorem}[Bounded transfinite recursion]\label{thm:bounded_transfinite_recursion}\mcite[177]{Enderton1977SetTheory}
  Fix an \hyperref[def:ordinal]{ordinal} \( \alpha \) and a nonempty set \( A \). Suppose that we are given some transformation \( T: \pow(\alpha \times A) \to A \), which sends every \hyperref[def:function]{set-valued map} with signature \( \alpha \multto A \) to an element of \( A \). Then there exists a unique \( \alpha \)-indexed \hyperref[def:transfinite_sequence]{transfinite sequence} \( f: \alpha \to A \) such that for any \( \beta \in \alpha \) we have \( f(\beta) = T(f\restr_\beta) \).
\end{theorem}
\begin{comments}
  \item This is a vast generalization of \fullref{thm:omega_recursion} from \hyperref[def:sequence]{natural number sequences} to \hyperref[def:transfinite_sequence]{transfinite sequences}.

  \item See \fullref{rem:transfinite_induction} about a reformulation that is often useful in practice.
\end{comments}
\begin{proof}
  The proof is analogous to that of \fullref{thm:omega_recursion}, but we will give it anyway to highlight the difference between using \fullref{thm:omega_induction} and \fullref{thm:bounded_transfinite_induction}.

  Let \( G \subseteq \pow(\alpha \times A) \) be the set of all \hyperref[def:set_valued_map/partial]{partial single-valued functions} \( g: \alpha \to A \) such that
  \begin{itemize}
    \item There exists some \( \beta_g \in \alpha \) such that \( g \) is defined only in the \hyperref[def:order_interval/unbounded]{open initial segment} \( \alpha_{< \beta_g} \). That is, \( g \) is defined for all \( \beta \) up to not including \( \beta_g \).

    \item \( g(\beta) = T(g\restr_\beta) \) for all \( \beta < \beta_g \).
  \end{itemize}

  Clearly \( G \) is nonempty because the function \( \set{ (\varnothing, T(\varnothing)) } \) belongs to \( G \).

  Define \( f \coloneqq \bigcup G \). At this point \( f \) is a \hyperref[def:function]{set-valued map}. We must now show that \( f \) has all the properties that we want.

  \SubProofOf[def:set_valued_map/partial]{totality} First, we will use \fullref{thm:bounded_transfinite_induction} to show that \( f \) is total.

  Fix \( \beta \in \dom f \). Then there exists a function \( g \in G \) defined for all \( \gamma < \beta \).

  \begin{itemize}
    \item If \( g \) is also defined at \( \beta \) also, this directly proves that \( \beta \in \dom f \).
    \item If \( g \) is not defined at \( \beta \), consider
    \begin{equation*}
      \widehat g \coloneqq g \cup \set{ (\beta, T(g\restr_\beta) }.
    \end{equation*}

    The function \( \widehat g \) is again a single-valued partial function and thus it belongs to \( G \), hence \( \beta \in \dom f \).
  \end{itemize}

  Therefore, \fullref{thm:bounded_transfinite_induction} allows us to conclude that \( f: \alpha \multto A \) is a total set-valued map.

  \SubProofOf[def:function]{single-valuedness} Now that we know that \( f \) is total, we will prove that it is single-valued and thus is a function in the usual sense of the term.

  Fix \( \beta \in \alpha \). Suppose that \( f \) is single-valued for all \( \gamma < \beta \). Since \( f \) is total, there exist at least one partial function \( g \) in \( G \) that is defined at \( \beta \). Let \( g \) and \( h \) both be such (single-valued partial) functions.

  Then
  \begin{equation*}
    g(\beta) = T(g\restr_\beta) = T(f\restr_\beta) = T(h\restr_\beta) = h(\beta),
  \end{equation*}
  hence \( g \) and \( h \) coincide at \( \beta \), which in turn implies that \( f \) is single-valued at \( \beta \).

  Therefore, \fullref{thm:omega_induction} allows us to conclude that \( f \) is a single-valued total function.

  \SubProofOf[def:function]{uniqueness} Now that it is clear that \( f \) satisfies the theorem, we must verify that it is unique.

  Suppose that \( f_1 \) and \( f_2 \) both satisfy the theorem. Fix some \( \beta \in \alpha \) and suppose that for every \( \gamma < \beta \), we have \( f_1(\gamma) = f_2(\gamma) \). Then
  \begin{equation*}
    f_2(\beta) = T(f_1\restr_\beta)) = T(f_2\restr_\beta) = f_2(\beta).
  \end{equation*}

  Therefore, \fullref{thm:omega_induction} allows us to conclude that \( f_1 = f_2 \). So there is at most one function that satisfies the theorem, and we have already shown that \( f \) is such a function.
\end{proof}

\begin{theorem}[Transfinite induction]\label{thm:transfinite_induction}
  It turns out that \fullref{thm:bounded_transfinite_recursion} is valid for all ordinals simultaneously.

  For every formula \( \varphi \) in the language of set theory not containing \( \syny \) nor \( \synz \) as free variables, the following is a theorem of \logic{ZFC}:
  \small
  \begin{equation*}
    \qforall \syny
    \parens[\Bigg]
      {
        \ref{eq:def:ordinal/predicate}[\syny]
        \synimplies
        \parens[\Big]
        {
          \overbrace
            {
              \underbrace{ \parens[\Big] { \qforall {\synz \in \syny} \varphi[\synx \mapsto \synz] } }_{\mathclap{\substack{\T{inductive} \\ \T{hypothesis}}}}
              \synimplies
              \underbrace{ \varphi[\synx \mapsto \syny] }_{\mathclap{\substack{\T{inductive step} \\ \T{conclusion}}}}
            }^{\T{inductive step}}
        }
      }
    \synimplies
    \qforall \syny \parens[\Bigg]
    {
      \ref{eq:def:ordinal/predicate}[\syny]
      \synimplies
      \underbrace{ \varphi[\synx \mapsto \syny] }_{\T{conclusion}}
    }.
  \end{equation*}
  \normalsize

  This theorem could be a special case of \fullref{thm:bounded_transfinite_recursion}, but there exists no set of all ordinals due to \fullref{thm:burali_forti_paradox}.

  See the comments in \fullref{def:peano_arithmetic/PA3} regarding variables and quantification in axiom schemas and \fullref{con:induction} for a general discussion of induction.

  See \fullref{rem:transfinite_induction} about a reformulation that is often useful in practice.
\end{theorem}
\begin{proof}
  The proof is similar to our proof of \fullref{thm:well_founded_induction}.
\end{proof}

\begin{proposition}\label{thm:ordinals_are_well_ordered}
  The ordinals are \hyperref[def:well_ordered_set]{well-ordered}. Since there exists no set of all ordinals due to \fullref{thm:burali_forti_paradox}, we cannot say that the ordinals form a well-ordered set. We will instead state a more concrete result.

  \begin{thmenum}
    \thmitem{thm:ordinals_are_well_ordered/transitivity} For any three ordinals \( \alpha \), \( \beta \) and \( \gamma \) such that \( \alpha < \beta < \gamma \) we have \( \alpha < \gamma \).

    \thmitem{thm:ordinals_are_well_ordered/trichotomy} For any two ordinals \( \alpha \) and \( \beta \), exactly one of \( \alpha = \beta \), \( \alpha < \beta \) or \( \alpha > \beta \) holds.
  \end{thmenum}

  As discussed in \fullref{def:ordinal}, irreflexivity and well-foundedness hold by \fullref{thm:set_membership_is_well_founded}.
\end{proposition}
\begin{proof}
  \SubProofOf{thm:ordinals_are_well_ordered/transitivity} Let \( \alpha \), \( \beta \) and \( \gamma \) be ordinals and let \( \alpha \in \beta \in \gamma \). Since \( \gamma \) is a transitive set, it follows that \( \beta \subseteq \gamma \) and thus \( \alpha \in \gamma \).

  Therefore, we have used the fact that \( \gamma \) is a transitive set to prove that set membership is a transitive relation, thus obtaining a connection between two distinct concepts both named \enquote{transitivity}.

  \SubProofOf{thm:ordinals_are_well_ordered/trichotomy} Let \( \alpha \) and \( \beta \) be ordinals.

  Due to \fullref{thm:simple_foundation_theorems/member_of_itself}, it is not possible for both \( \alpha \in \beta \) and \( \alpha = \beta \) to hold simultaneously. Due to \fullref{thm:simple_foundation_theorems/mutual_membership}, it is not possible for both \( \alpha \in \beta \) and \( \beta \in \alpha \) to hold simultaneously.

  Thus, at most one of \( \alpha = \beta \), \( \alpha \in \beta \) or \( \beta \in \alpha \) holds.

  We will use \fullref{thm:transfinite_induction} on \( \beta \) to show that at least one of them holds. Fix an ordinal \( \beta_0 \). Our inductive hypothesis is that for every ordinal \( \alpha \) and every \( \gamma \in \beta_0 \) at least one of \( \alpha = \gamma \), \( \alpha \in \gamma \) or \( \gamma \in \alpha \) holds.

  Fix some ordinal \( \alpha_0 \). We will show that at least one of \( \alpha_0 = \beta_0 \), \( \alpha_0 \in \beta_0 \) or \( \beta_0 \in \alpha_0 \) holds. Since the case \( \beta_0 = \alpha_0 \) is trivial, we exclude it from consideration.

  \begin{itemize}
    \item If there exists \( \gamma \in \beta_0 \) such that \( \gamma = \alpha_0 \), clearly \( \alpha_0 \in \beta_0 \).
    \item If there exists \( \gamma \in \beta_0 \) such that \( \alpha_0 \in \gamma \), then by transitivity \( \alpha_0 \in \beta_0 \).
    \item If for every \( \gamma \in \beta_0 \) we have \( \gamma \in \alpha_0 \), then \( \beta_0 \subsetneq \alpha_0 \). Let \( \gamma_0 \) be the smallest member of \( \alpha_0 \setminus \beta_0 \). We will show that \( \gamma_0 = \beta_0 \).

    Our first goal is to show that \( \gamma_0 \subseteq \beta_0 \). Aiming at a contradiction, suppose that there exists some \( \lambda \in \gamma_0 \setminus \beta \). Since \( \gamma_0 \) is a transitive set, we have \( \lambda \in \alpha \). Thus, \( \lambda \in \alpha \setminus \beta \) and \( \lambda \in \gamma_0 \), contradicting the minimality of \( \gamma_0 \). Therefore, \( \gamma_0 \subseteq \beta \).

    Now we will use the existing inductive hypothesis for \( \alpha = \gamma_0 \) to show that \( \beta_0 \subseteq \gamma_0 \).

    \begin{itemize}
      \item If there exists \( \lambda \in \beta_0 \) such that \( \lambda = \gamma_0 \), clearly \( \gamma_0 \in \beta_0 \). But that contradicts our choice of \( \gamma_0 \) as a member of \( \alpha_0 \setminus \beta_0 \).

      \item If there exists \( \lambda \in \beta_0 \) such that \( \gamma_0 \in \lambda \), then by transitivity \( \gamma_0 \in \beta_0 \), which again contradicts our choice of \( \gamma_0 \).

      \item Finally, if for every \( \gamma \in \beta_0 \) we have \( \gamma \in \gamma_0 \), then \( \beta_0 \subseteq \gamma_0 \).
    \end{itemize}

    Thus, both \( \gamma_0 \subseteq \beta_0 \) and \( \beta_0 \subseteq \gamma_0 \), giving us the equality \( \beta_0 = \gamma_0 \). This allows us to conclude that \( \beta_0 \in \alpha_0 \).
  \end{itemize}

  We have shown by transfinite induction that for a fixed ordinal \( \beta_0 \), for every other ordinal \( \alpha \) at least one of \( \beta_0 = \alpha \), \( \beta_0 \in \alpha \) or \( \alpha \in \beta_0 \) holds. We have already shown that at most one of the three holds, hence exactly one of the three holds.

  Since \( \beta_0 \) is itself arbitrary, we can conclude that trichotomy holds for any two ordinals \( \alpha \) and \( \beta \).
\end{proof}

\begin{proposition}\label{thm:ordinal_ordering_via_subsets}
  For any two ordinals \( \alpha \) and \( \beta \) we have \( \beta \in \alpha \) if and only if \( \beta \subsetneq \alpha \).
\end{proposition}
\begin{proof}
  \NecessitySubProof Since \( \alpha \) is a transitive set, from \( \beta \in \alpha \) it follows that \( \beta \subseteq \alpha \).

  We cannot have \( \beta = \alpha \) due to \fullref{thm:simple_foundation_theorems/member_of_itself}, hence \( \beta \subsetneq \alpha \).

  \SufficiencySubProof Suppose that \( \beta \subsetneq \alpha \).

  By \fullref{thm:ordinals_are_well_ordered/trichotomy}, the ordinals \( \beta \) and \( \alpha \) must be related by set membership.
  \begin{itemize}
    \item If \( \alpha = \beta \), this directly contradicts our assumption that \( \beta \subsetneq \alpha \).
    \item If \( \alpha \in \beta \), then \( \alpha \in \alpha \) which contradicts \fullref{thm:simple_foundation_theorems/member_of_itself}.
    \item It remains for \( \beta \in \alpha \) to hold.
  \end{itemize}
\end{proof}

\begin{proposition}\label{thm:ordinal_successor_strictly_monotone_on_ordinals}
  The ordinal successor operation is strictly monotone on ordinals. That is, if \( \alpha < \beta \), then \( \op{sc}(\alpha) < \op{sc}(\beta) \).
\end{proposition}
\begin{proof}
  Let \( \alpha \in \beta \) and let \( \gamma \in \op{sc}(\alpha) \).
  \begin{itemize}
    \item If \( \gamma \in \alpha \), clearly \( \gamma \in \beta \) because \( \beta \) is a transitive set.

    \item If \( \gamma = \alpha \), then \( \gamma = \alpha \in \beta \).
  \end{itemize}

  We have shown that \( \op{sc}(\alpha) \subseteq \beta \). Thus, we either have \( \op{sc}(\alpha) = \beta \in \op{sc}(\beta) \) or else by \fullref{thm:ordinal_ordering_via_subsets} we have \( \op{sc}(\alpha) \in \beta \in \op{sc}(\beta) \).
\end{proof}

\begin{proposition}\label{thm:transitive_set_of_transitive_sets}\mcite{MathSE:transitive_set_of_transitive_sets}
  A \hyperref[def:transitive_set]{transitive set} whose members are transitive sets is an ordinal.

  In particular, a transitive set of ordinals is an ordinal. If a set of ordinals is not transitive, we can instead take its \hyperref[def:transitive_closure_of_a_set]{transitive closure}.
\end{proposition}
\begin{proof}
  Let \( A \) be a set whose members are themselves transitive sets.

  We will first show that set membership is transitive on \( A \). If \( x \), \( y \) and \( z \) are members of \( A \) such that \( x \in y \in z \), then since \( z \) is transitive we have \( y \subseteq z \) and thus \( x \in z \).

  Therefore, we can conclude that set membership is a \hyperref[def:strict_partial_order]{strict partial order} on \( A \). Now define the set
  \begin{equation*}
    B \coloneqq \set{ x \in A \given \qexists {y \in A} x \not\in y \synwedge y \not\in x }
  \end{equation*}
  of all members of \( A \) which are not related to at least one other member. If \( B \) is empty, then set membership is trichotomous on \( A \).

  If \( B \) is nonempty, let \( b \) be a minimal element of \( B \). A minimal element must exist because set membership is a well-founded partial order on \( A \). We have chosen \( b \), so that every member of \( b \) is related to every other member of \( A \).

  Define the set
  \begin{equation*}
    C \coloneqq \set{ x \in A \given x \neq b \synwedge x \not\in b \synwedge b \not\in x }
  \end{equation*}
  of all members of \( A \) which are not related to \( b \) and let \( c \) be a minimal element of \( C \). We will now show that \( b = c \), which is a contradictions with our choice of \( c \).

  Let \( x \in b \). As we have already mentioned, \( x \) is related to every other member of \( A \), including \( c \).
  \begin{itemize}
    \item If \( c = x \), then \( c \in b \), which contradicts our choice of \( c \).

    \item Suppose that \( c \in x \). We have chosen \( x \) to be a member of \( b \) and we thus have \( c \in x \in b \). The set \( b \) is transitive as a member of \( A \), hence \( x \) is also a member of \( A \). Since set membership is a transitive relation on \( A \), it follows that \( c \in b \), which contradicts our choice of \( c \).

    \item It remains for \( x \in c \) to hold.
  \end{itemize}

  Therefore, \( b \subseteq c \). The converse inclusion \( c \subseteq b \) can be obtained analogously by noting that \( c \) is a minimal element of \( C \) and hence every \( x \in c \) is related to \( b \). Thus, we obtain \( b = c \), which contradicts our choice of \( c \) as a member of \( C \).

  The obtained contradiction shows that the set \( B \) is empty and thus every member of \( A \) is related to every other member, proving trichotomy of set membership on \( A \). This allows us to conclude that \( A \) is an ordinal because it is a transitive set and we have already shown in the beginning of the proof that set membership is a transitive relation on \( A \).
\end{proof}

\begin{definition}\label{def:transitive_closure_of_a_set}\mcite[491]{Hinman2005Logic}
  To show that every set has a rank we must introduce additional definitions. We use \fullref{thm:omega_recursion} to define the \term{transitive closure} of a set \( A \) as
  \begin{equation*}
    \cl^T(A) \coloneqq \bigcup \set{ \cl_n^T(A) \given n \in \omega },
  \end{equation*}
  where
  \begin{equation*}
    \cl_n^T(A) \coloneqq \begin{cases}
      A,                      &n = 0 \\
      \bigcup \cl_{n-1}^T(A), &n > 0
    \end{cases}
  \end{equation*}

  Note that this is different from the transitive closure of a relation defined in \fullref{def:relation_closures/transitive}.
\end{definition}

\begin{proposition}\label{thm:transitive_closure_of_a_set}\mcite[corr. 6.2.43]{Hinman2005Logic}
  The \hyperref[def:transitive_closure_of_a_set]{transitive closure} \( \cl^T(A) \) of the set \( A \) is the smallest transitive set containing \( A \).
\end{proposition}
\begin{proof}
  It is clear that \( A = \cl_0^T(A) \) is a subset of \( \cl^T(A) \).

  If \( x \in \cl^T(A) \), then there exists some natural number \( n \) for which \( x \in \cl_n^T(A) \). Therefore, \( x \subseteq \bigcup \cl_n^T(A) = \cl_{n+1}^T(A) \) and thus \( x \subseteq \cl^T(A) \).

  Now suppose that \( B \) is a transitive subset of \( \cl^T(A) \) which contains \( A \). Let \( x_0 \in \cl^T(A) \).

  Suppose that \( x_0 \not\in B \). Then there must exist a smallest nonzero number \( n \) such that \( x_0 \in \cl_n^T(A) \). Then \( x_0 \) belongs to some member \( x_1 \) of \( \cl_{n-1}^T(A) \). If \( x_1 \in A \), then \( x_0 \) must belong to \( B \) since it is transitive. But this contradicts our choice of \( x_0 \). Then \( x_1 \not\in A \), in which case there exists some \( x_2 \in \cl_{n-2}^T(A) \) such that \( x_1 \in x_2 \). If \( x_2 \in A \), then again \( x_0 \in B \), which contradicts our choice of \( x_0 \). We can thus recursively construct a sequence \( \set{ x_k }_{k=0}^\infty \) such that for every \( k \geq 0 \) both \( x_k \in x_{k+1} \) and \( x_k \not\in A \) hold. The existence of such a sequence contradicts \fullref{thm:set_membership_is_well_founded}.

  Therefore, \( \cl^T(A) \subseteq B \), hence \( \cl^T(A) \) is the smallest transitive set containing \( A \).
\end{proof}

\begin{proposition}\label{thm:ordinal_is_set_of_smaller_ordinals}
  Every ordinal equals the set of all smaller ordinals.

  This is the unbounded version of \fullref{thm:initial_segment_of_ordinal}.
\end{proposition}
\begin{proof}
  Let \( \alpha \) be an ordinal and let \( A \) be the set of all ordinals smaller than \( \alpha \). We will show that \( A = \alpha \). We will first show that \( A \) is a transitive set. Let \( \beta \in A \) and \( \gamma \in \beta \). Since \( \alpha \) is a transitive set that contains \( \beta \), we have \( \gamma \in \alpha \). Thus, \( \gamma \) is smaller than \( \alpha \) and hence it belongs to \( A \). Therefore, \( A \) is a transitive set of ordinals and by \fullref{thm:transitive_set_of_transitive_sets}, it is itself an ordinal.

  \Fullref{thm:ordinals_are_well_ordered} implies that \( \alpha \) and \( A \) are either equal or related by set membership.
  \begin{itemize}
    \item If \( \alpha \in A \), then \( \alpha \) is smaller than itself, which contradicts \fullref{thm:simple_foundation_theorems/member_of_itself}.

    \item If \( A \in \alpha \), then \( A \) is smaller than itself, which again contradicts \fullref{thm:simple_foundation_theorems/member_of_itself}.

    \item It remains for \( A \) to be equal to \( \alpha \).
  \end{itemize}
\end{proof}

\begin{theorem}[Burali-Forti paradox]\label{thm:burali_forti_paradox}\mcite[rem. 6.3.3]{Hinman2005Logic}
  Assuming \logic{ZFC}, there is no set of all ordinals.
\end{theorem}
\begin{proof}
  Aiming at a contradiction, suppose that \( A \) is a containing all ordinals. If \( \alpha \in A \) and \( \beta \in \alpha \), transitivity \( \beta \) implies \( \beta \in A \). Thus, \( A \) is a transitive set of ordinals, which \fullref{thm:transitive_set_of_transitive_sets} is itself an ordinal. Hence, \( A \in A \).

  But this contradicts \fullref{thm:simple_foundation_theorems/member_of_itself}. Hence, there is no set of all ordinals.
\end{proof}

\begin{proposition}\label{thm:union_of_set_of_ordinals}\mcite{MathSE:union_of_set_of_ordinals}
  Let \( A \) be a set of ordinals and denote \( \alpha \coloneqq \bigcup A \).

  \begin{thmenum}
    \thmitem{thm:union_of_set_of_ordinals/is_ordinal} The union \( \alpha \) is itself an ordinal.

    \thmitem{thm:union_of_set_of_ordinals/supremum} The union \( \alpha \), which is the \hyperref[def:extremal_points/supremum_and_infimum]{supremum} of \( A \) with respect to set inclusion, is also the supremum of \( A \) with respect to ordinal ordering.

    That is, either \( \alpha = A \) or \( \alpha \) is the smallest ordinal that is larger than every member of \( A \).

    \thmitem{thm:union_of_set_of_ordinals/ordinal} If \( A \) is an ordinal, then \( \alpha \leq A \). Furthermore, in the case \( \alpha < A \), there is no ordinal between \( A \) and \( \alpha \). That is, if \( \alpha < A \), then \( A \) is the smallest ordinal strictly larger than \( \alpha \).

    See \fullref{def:successor_and_limit_ordinal/union} for a further distinction between \( \alpha = A \) and \( \alpha < A \).
  \end{thmenum}

  For recursive definitions like \fullref{def:ordinal_arithmetic/addition} this proposition justifies using \( \sup A \) instead of the more confusing \( \bigcup A \).

  Compare this result to \fullref{thm:union_of_set_of_cardinals}.
\end{proposition}
\begin{proof}
  \SubProofOf{thm:union_of_set_of_ordinals/is_ordinal} Let \( A \) be a set of ordinals. Denote its union by \( \alpha \coloneqq \bigcup A \).

  We will show that \( \alpha \) is a transitive set. Due to \fullref{thm:transitive_set_of_transitive_sets}, this is sufficient for \( \alpha \) to be an ordinal.

  Let \( \beta \in \alpha \). Then there exists an ordinal \( \gamma \) in \( A \) such that \( \beta \in \gamma \). Since \( \gamma \) is itself a transitive set, we have \( \beta \subseteq \gamma \). But \( \gamma \subseteq \alpha \), hence \( \beta \subseteq \alpha \).

  Therefore, \( \alpha = \bigcup A \) is a transitive set and thus an ordinal.

  \SubProofOf{thm:union_of_set_of_ordinals/supremum} For every \( \beta \in A \) we have \( \beta \subseteq \alpha \), which by \fullref{thm:ordinal_ordering_via_subsets} corresponds to \( \beta \leq \alpha \) with respect to ordinal ordering.

  We will show that \( \alpha \) is the smallest ordinal with this property. Let \( \lambda_0 \) be any other ordinal such that \( \gamma \leq \lambda_0 \) for any \( \gamma \in A \). Fix some \( \gamma_0 \in \alpha \). Then there exists an ordinal \( \beta_0 \in A \) such that \( \gamma_0 \in \beta_0 \). Since \( \beta_0 \in A \) and \( A \subseteq \lambda_0 \), since \( \lambda_0 \) itself is a transitive set we have \( \gamma_0 \in \lambda_0 \).

  Therefore, \( \alpha \) is the least upper bound of \( A \) with respect to ordinal ordering.

  \SubProofOf{thm:union_of_set_of_ordinals/ordinal} Assume that \( A \) is an ordinal.

  If \( \gamma \in \alpha \), by transitivity of \( \alpha \) we have \( \gamma \subseteq \alpha \). Thus, \( \gamma \in A \). Hence, \( \alpha \subseteq A \).

  Now suppose that there exists some \( \gamma \in A \) such that \( \alpha \in \gamma \). Then
  \begin{equation*}
    \gamma \subseteq \bigcup A = \alpha \in \gamma,
  \end{equation*}
  which contradicts \fullref{thm:simple_foundation_theorems/member_of_itself}.

  Hence, there is not ordinal between \( \alpha = \bigcup A \) and \( A \).
\end{proof}

\begin{proposition}\label{thm:successor_of_ordinal}
  The \hyperref[def:ordinal_successor]{successor} \( \alpha \coloneqq \op{sc}(\beta) \) of an ordinal \( \beta \) is the smallest ordinal larger than \( \alpha \).
\end{proposition}
\begin{proof}
  We must show that \( \alpha \) is a transitive set and thus by \fullref{thm:transitive_set_of_transitive_sets} an ordinal. Note that \( \alpha = \op{sc}(\beta) = \beta \cup \set{ \beta } \).

  Let \( \gamma \in \alpha \).
  \begin{itemize}
    \item If \( \gamma \in \beta \), then since \( \beta \) is a transitive set, we have \( \gamma \subseteq \beta \). Furthermore, since \( \beta \subseteq \alpha \), by transitivity of set inclusion \( \gamma \subseteq \alpha \).

    \item If \( \gamma = \beta \), then \( \gamma \subseteq \alpha \) by definition of successor.
  \end{itemize}

  Therefore, \( \alpha \) is a transitive set and thus an ordinal.

  Now suppose that there is another ordinal \( \gamma \) such that \( \beta \in \gamma \in \alpha = \beta \cup \set{ \beta } \).
  \begin{itemize}
    \item If \( \gamma \in \beta \), this would contradict \fullref{thm:simple_foundation_theorems/mutual_membership}.

    \item If \( \gamma = \beta \), this would contradict \fullref{thm:simple_foundation_theorems/member_of_itself}.
  \end{itemize}

  The obtained contradictions show that there is no ordinal between \( \alpha \) and \( \beta \).
\end{proof}

\begin{remark}\label{rem:def:ordinal_successor}
  It follows from \fullref{thm:ordinal_is_set_of_smaller_ordinals} and \fullref{thm:successor_of_ordinal} and that for any ordinal \( \beta \) we have
  \begin{align*}
    \beta            &= \set{ \gamma \given \gamma \T{is an ordinal and} \gamma < \beta }, \\
    \op{sc}(\beta) &= \set{ \gamma \given \gamma \T{is an ordinal and} \gamma \leq \beta }.
  \end{align*}

  This shows that the \hyperref[def:ordinal_successor]{ordinal successor} operation is very natural in the context of ordinals.
\end{remark}

\begin{definition}\label{def:successor_and_limit_ordinal}\mcite[def. 6.3.10]{Hinman2005Logic}
  We say that the ordinal \( \alpha \) is a \term{successor ordinal} if any of the following equivalent conditions hold:

  \begin{thmenum}
    \thmitem{def:successor_and_limit_ordinal/successor} The ordinal \( \alpha \) is the successor of another ordinal. More precisely, there exists another ordinal \( \beta \) such that \( \alpha = \op{sc}(\beta) \).

    \thmitem{def:successor_and_limit_ordinal/smaller_successor} There exists some \( \beta \in \alpha \) such that \( \op{sc}(\beta) \) does not belong to \( \alpha \).

    \thmitem{def:successor_and_limit_ordinal/union} We have \( \bigcup \alpha \in \alpha \).
  \end{thmenum}

  If \( \alpha \) is neither zero nor a successor ordinal, we call it a \term{limit ordinal}. See \fullref{thm:limit_ordinal_order_topology} for a more involved equivalent condition.

  These notions should not be confused with \hyperref[def:successor_and_limit_cardinal]{successor and weak/strong limit cardinals}.
\end{definition}
\begin{comments}
  \item A useful characterization of limit ordinals will be given in \fullref{thm:limit_ordinal_division_characterization}.
\end{comments}
\begin{proof}
  \ImplicationSubProof{def:successor_and_limit_ordinal/successor}{def:successor_and_limit_ordinal/smaller_successor} If \( \alpha = \op{sc}(\beta) \), then \( \beta \) satisfies \fullref{def:successor_and_limit_ordinal/smaller_successor}.

  \ImplicationSubProof{def:successor_and_limit_ordinal/smaller_successor}{def:successor_and_limit_ordinal/successor} Let \( \beta \in \alpha \) be such that \( \op{sc}(\beta) \not\in \alpha \). Then by trichotomy we have that either \( \op{sc}(\beta) = \alpha \) or \( \op{sc}(\beta) > \alpha \).

  If \( \op{sc}(\beta) > \alpha \), then either \( \alpha = \beta \), which would contradict \fullref{thm:simple_foundation_theorems/member_of_itself}, or \( \alpha \in \beta \), which would contradict \fullref{thm:simple_foundation_theorems/mutual_membership}.

  Thus, it remains for \( \op{sc}(\beta) \) to be equal to \( \alpha \).

  \ImplicationSubProof{def:successor_and_limit_ordinal/successor}{def:successor_and_limit_ordinal/union} Suppose that \( \alpha = \op{sc}(\beta) \).

  We have
  \begin{balign*}
    \bigcup \alpha
    &=
    \bigcup (\beta \cup \set{ \beta })
    = \\ &=
    \set{ \gamma \given \qexists \delta (\delta \in \beta \T{or} \delta = \beta) \T{and} \gamma \in \delta }
    \reloset {\eqref{eq:def:distributive_lattice/meet_over_join}} = \\ &=
    \set{ \gamma \given \qexists \delta (\delta \in \beta \T{and} \gamma \in \delta) \T{or} (\delta = \beta \T{and} \gamma \in \delta }
    = \\ &=
    \set*{ \gamma \given* \parens[\Big]{ \qexists {\delta \in \beta} \gamma \in \delta } \T{or} \gamma \in \beta }
    = \\ &=
    \parens*{ \bigcup \beta } \cup \beta
    \reloset{ \bigcup \beta \subseteq \beta } = \\ &=
    \beta.
  \end{balign*}

  Thus, \( \beta = \bigcup \alpha \in \alpha \).

  \ImplicationSubProof{def:successor_and_limit_ordinal/union}{def:successor_and_limit_ordinal/successor} Let \( \bigcup \alpha \in \alpha \). From \fullref{rem:def:ordinal_successor} we have that
  \begin{equation*}
    \op{sc}\parens*{ \bigcup \alpha }
    =
    \set*{ \gamma \given* \gamma \subseteq \bigcup \alpha }
    =
    \set{ \gamma \given \gamma \subsetneq \alpha }
    =
    \alpha.
  \end{equation*}
\end{proof}

\begin{remark}\label{rem:transfinite_induction}
  It is sometimes efficient to reformulate transfinite induction and recursion. Although analogous principles hold for \fullref{thm:bounded_transfinite_induction} and \fullref{thm:bounded_transfinite_recursion}, we will only demonstrate them for \fullref{thm:transfinite_induction}. The original statement is that in order to prove that some formula is satisfied for all sets, it is sufficient to only prove one inductive step.

  More precisely, let \( \varphi \) be a formula in the language of set theory, let \( \mscrV = (V, I) \) be a \hyperref[rem:standard_model_of_set_theory]{standard} \hyperref[rem:transitive_model_of_set_theory]{transitive} model of set theory and let \( v: \op*{Var} \to V \) be some variable assignment. This allows us to fix any parameters that would otherwise be present in the induction schema. We will say that the set \( A \) satisfies \( \varphi \) if \( \Bracks{\varphi}_{v_{\synx \mapsto A}} = T \).

  \Fullref{thm:transfinite_induction} states that in order to prove that \( \varphi \) holds for any ordinal, the following is sufficient:
  \begin{thmenum}[series=rem:transfinite_induction]
    \thmitem{rem:transfinite_induction/single_inductive_step} For every ordinal \( \alpha \), by assuming that every smaller ordinal satisfies \( \varphi \), we must prove that \( \alpha \) does.
  \end{thmenum}

  We have just defined in \fullref{def:successor_and_limit_ordinal} three mutually exclusive types of ordinals. We can now restate the principles of transfinite induction as follows:
  \begin{thmenum}[resume=rem:transfinite_induction]
    \thmitem{rem:transfinite_induction/base_case} In the \term{base case}, we must prove that \( 0 \) satisfies \( \varphi \).

    \thmitem{rem:transfinite_induction/successor_case} In the \term{successor case}, by assuming that \( \alpha \) satisfies \( \varphi \), we must prove that \( \op{sc}(\alpha) \) does. It is in line with \fullref{thm:transfinite_induction} assume that every ordinal smaller that or equal to \( \alpha \) satisfies \( \varphi \), however it is often enough to do, so only for \( \alpha \) itself.

    \thmitem{rem:transfinite_induction/limit_case} In the \term{limit case}, if \( \lambda \) is a limit ordinal, by assuming that every smaller ordinal satisfies \( \varphi \) we must prove that \( \lambda \) satisfies \( \varphi \).
  \end{thmenum}

  We must note that \fullref{rem:transfinite_induction/limit_case} is essentially the same as the single inductive step \fullref{rem:transfinite_induction/single_inductive_step} except that it is restricted to limit ordinals. The reasoning for this is that the proofs for zero and every successor ordinal can be different from those for limit ordinals.

  See \fullref{thm:ordinal_bijection_and_choice} and \fullref{def:cumulative_hierarchy} for concrete examples.
\end{remark}

\begin{concept}\label{con:induction}
  \term{Mathematical induction} is a very valuable proof technique for universal statements. The proof of \fullref{thm:nonzero_natural_numbers_have_predecessors} contains remarks regarding its usage and its difference from deduction principles that are formalized via \hyperref[def:deduction_system]{deduction systems}. Although it is a logical tool, this remark belongs to this section because it contains several induction and recursion principles.

  More generally, given a \hyperref[def:first_order_syntax/formula]{first-order formula} \( \varphi \) over some \hyperref[def:first_order_language]{first-order language}, certain \hyperref[def:first_order_theory]{logical theories} allow us to prove indirectly \( \qforall \synx \varphi[\synx \mapsto \syny] \) by proving simpler statements (\fullref{def:peano_arithmetic/PA3} contains very important remarks regarding the free variables of \( \varphi \)). This can be done in cases where every \hyperref[def:first_order_model]{model} \( X = (X, I) \) of the theory allows us to exhaust its universe \( X \) in a small finite number of steps. We can sometimes use the same steps to instead build objects. The latter principles is called \term{recursion}.

  It should be noted that induction and recursion are used interchangeably in the literature.

  Not much more can be said at this level of generality, so we list several induction principles and give examples of their usage:
  \begin{thmenum}
    \thmitem{con:induction/peano_arithmetic} The most basic induction principles is the (weak) natural number induction. It is best described via the axiom schema \eqref{eq:def:peano_arithmetic/PA3}. \Fullref{thm:nonzero_natural_numbers_have_predecessors} contains detailed commentary regarding its usage and most of the proofs in \fullref{sec:natural_numbers} are performed inductively.

    It its set-theoretic form \fullref{thm:omega_induction} it is important as a tool for introducing a model of Peano arithmetic. It is used directly for proving \fullref{thm:omega_is_transitive} and \fullref{thm:omega_recursion}. It is essentially the same as \fullref{rem:transfinite_induction} without \fullref{rem:transfinite_induction/limit_case}.

    \Fullref{thm:omega_recursion} is an important standalone tool that allows us to perform recursive definitions for natural numbers. The latter is used, often implicitly, in a great variety of places, from the ability to define natural number operations in \fullref{def:omega_operations} to the definition of semigroup exponentiation in \fullref{def:semigroup/exponentiation}.

    \thmitem{con:induction/well_founded} A vast generalization of natural number induction is \fullref{thm:well_founded_induction}. It is stated in a very general setting, but is not frequently used. It can be used to prove \fullref{thm:epsilon_induction}, which however is even less frequently used. We do not use neither in practice, however the special case where \( X = \BbbN \) is called strong induction on natural numbers.

    The usual (weak) natural number induction which is performed by proving the statement for \( 0 \) and then proceeding to prove it for \( n + 1 \) by assuming that it holds for \( n \). Strong induction instead has no base cases and is performed by proving a statement for \( n \) by assuming that it holds for all natural numbers strictly smaller than \( n \).

    Well-founded induction and epsilon-induction have no corresponding recursion principle.

    \thmitem{con:induction/transfinite} Another vast generalization of natural number induction is \fullref{thm:bounded_transfinite_induction}, which is further generalized by \fullref{thm:transfinite_induction}.

    Both principles are used to prove fundamental properties of the ordinals. Outside of set theory, transfinite induction is immensely useful, but it is rarely used directly. Instead, it is usually combined with the \hyperref[def:zfc/choice]{axiom of choice} via \fullref{thm:zorns_lemma}. \Fullref{rem:transfinite_induction} contains notes on how it is used directly.

    Only bounded transfinite induction has a corresponding recursion principle --- \fullref{thm:bounded_transfinite_recursion}. Unbounded transfinite induction cannot define such a principle because that would easily lead to \fullref{thm:burali_forti_paradox}. \Fullref{rem:unbounded_transfinite_recursion} shows how to circumvent this, however.

    Transfinite recursion is used to construct the cumulative hierarchy in \fullref{def:cumulative_hierarchy}.

    \Fullref{rem:cardinal_transfinite_recursion_and_induction} provides alternative transfinite recursion and induction principles for cardinals rather than for ordinals.

    \thmitem{con:induction/grammars} Within this monograph, we use \fullref{thm:induction_on_rooted_trees}, for example \hyperref[def:propositional_syntax/formula]{propositional} and \hyperref[def:first_order_syntax/formula]{first-order formulas}. It is not an established principle like the others.

    Rather than having a corresponding recursion principle, we construct words via \hyperref[def:formal_grammar]{formal grammars}.

    \thmitem{con:induction/lattices} \Fullref{thm:knaster_tarski_theorem} and \fullref{thm:knaster_tarski_iteration} hold in \hyperref[def:complete_lattice]{complete lattices}, and enable more esoteric principles like \fullref{thm:induction_on_recursively_defined_relations}.
  \end{thmenum}
\end{concept}

\begin{proposition}\label{thm:ordinal_isomorphism}
  Two ordinals are equal if and only if they are order-isomorphic.
\end{proposition}
\begin{proof}
  \SufficiencySubProof Trivial.

  \NecessitySubProof Let \( \alpha \) and \( \beta \) be two ordinals. The case \( \alpha = \beta \) is clear. Without loss of generality, suppose that \( \beta < \alpha \). \Fullref{thm:ordinal_ordering_via_subsets} implies that \( \beta \subsetneq \alpha \).

  Let \( f: \alpha \to \beta \) be an order isomorphism. Let \( \gamma_0 \) be the smallest value in \( \alpha \setminus \beta \).

  From \fullref{thm:def:well_ordered_set/embedding_extensional} it follows that \( \gamma_0 \leq f(\gamma_0) \). But \( f(\gamma_0) \in \beta \), hence \( \gamma_0 < \beta \). But this contradicts our choice of \( \gamma_0 \) as a member of \( \alpha \setminus \beta \).

  Therefore, \( \gamma_0 \in \beta \). Since \( \gamma_0 \) was arbitrary, we conclude that \( \beta < \alpha \) leads to a contradiction.

  We can obtain an analogous contradiction for \( \alpha < \beta \), hence it only remains possible for \( \alpha \) and \( \beta \) to be equal.
\end{proof}

\begin{theorem}[Hartogs' lemma]\label{thm:hartogs_lemma}
  For every set \( A \) there exists a smallest ordinal \( \alpha \) such that no function from \( \alpha \) to \( A \) is injective.
\end{theorem}
\begin{proof}
  Define the set
  \begin{equation*}
    W \coloneqq \set{ (P, \leq) \given P \subseteq A \T*{and} \leq \T*{well-orders} P }.
  \end{equation*}

  Let \( \beta \) be an ordinal for which there exists an injective function \( f: \beta \to A \). At least one such pair of a ordinal and function exists because \( \varnothing: 0 \to A \) is an injective function.

  The image of \( f \) can be well-ordered by
  \begin{equation*}
    f(\delta) < f(\gamma) \T{if and only if} \delta \in \gamma,
  \end{equation*}
  where \( \delta \) and \( \gamma \) are members of \( \beta \). Thus, if we restrict the codomain of \( f \) to its image, it would be an explicit order isomorphism of \( (\beta, \in) \) and \( (\img f, <) \).

  We have shown that every ordinal \( \beta \) and every injective function \( f: \beta \to A \) induces a well-ordered set which belongs to \( W \). Furthermore, if \( f_1: \beta_1 \to A \) and \( f_2: \beta_2 \to A \) are two such injective functions and if the induced well-ordered sets \( (\img f_1, \leq_1) \) and \( (\img f_2, \leq_2) \) are order-isomorphic, then \( \beta_1 \) and \( \beta_2 \) are themselves order-isomorphic and thus \( \beta_1 = \beta_2 \) by \fullref{thm:ordinal_isomorphism}.

  Therefore, for any well-ordered set in \( W \) there exists at most one ordinal that induces it via some function. Let \( W' \) be the subset of all well-ordered sets in \( W \) induced by exactly one ordinal.

  We have that \( W' \) is a set and for each member of \( W' \) there corresponds exactly one ordinal. It follows from the \hyperref[def:zfc/replacement]{axiom schema of replacement} that these ordinals form a set. Denote this set by \( B \).

  There is no set of all ordinals by \fullref{thm:burali_forti_paradox}, hence there must exist an ordinal not in \( B \). But every ordinal that has an injective function into \( A \) belongs to \( B \). Hence, there exists some ordinal \( \alpha \) such that no function from \( \alpha \) to \( A \) is injective.

  If \( \alpha \) is not the smallest ordinal with this property, we can now easily take the smallest member of \( \alpha \) with this property.
\end{proof}

\begin{lemma}\label{thm:ordinal_bijection_and_choice}\mcite{MathSE:order_type_existence}
  Let \( A \) be a nonempty set and let \( c \) be a \hyperref[def:choice_function]{choice function} on \( A \). Then there exists an ordinal \( \alpha \) and a bijection between \( A \) and \( \alpha \).
\end{lemma}
\begin{proof}
  We will explicitly build the desired ordinal. We will use \fullref{thm:bounded_transfinite_recursion} in its structured form presented in \fullref{rem:transfinite_induction} to build a \hyperref[def:transfinite_sequence]{transfinite sequence} of injective maps from ordinals into \( A \).

  Let \( \delta \) be the smallest ordinal such that no function from \( \delta \) to \( A \) is injective. Such an ordinal exists by \fullref{thm:hartogs_lemma}. Note that \( \delta \) cannot be zero because the empty function is always injective.

  Any smaller than \( \delta \) ordinal has at least one injective function into \( A \). We will use recursion bounded by \( \delta \) to construct a \hyperref[def:transfinite_sequence]{transfinite sequence} \( \seq{ \iota_\beta }_{\beta < \delta} \) satisfying the invariant that for any \( \beta \in \delta \), the function \( \iota_\beta \) is injective and strictly contains \( \iota_\gamma \) for any \( \gamma < \beta \).

  \begin{itemize}
    \item The zero ordinal has only one possible injective function into \( A \), namely the empty function \( \varnothing: 0 \to P \).

    \item Now let \( \iota_\beta: \beta \to A \) be an injective function.

    If \( \iota_\beta \) is surjective, then it is the desired bijection and the rest of the construction is irrelevant.

    If \( \iota_\beta \) is not surjective, define
    \begin{equation*}
      \begin{aligned}
        &\iota_{\op{sc}(\beta)}: \op{sc}(\beta) \to P \\
        &\iota_{\op{sc}(\beta)}(\gamma) \coloneqq \begin{cases}
          \iota_\beta(\gamma),         &\gamma < \beta \\
          c(A \setminus \img(\iota_\beta)), &\gamma = \beta.
        \end{cases}
      \end{aligned}
    \end{equation*}

    This function is clearly injective.

    \item Let \( \lambda \) be a limit ordinal and let \( \iota_\beta \) be an injective function for any \( \beta < \lambda \). Note that the domain of \( \iota_\beta \) is not necessarily \( \beta \) --- it may be a smaller ordinal \( \gamma \) in case \( \iota_\gamma \) is surjective (and thus the desired bijection).

    In any case, we have a family \( \set{ \iota_\beta }_{\beta < \lambda} \) of functions such that \( \iota_\gamma \subseteq \iota_\beta \) if and only if \( \gamma < \beta \). We simply take their union
    \begin{equation*}
      \iota_\lambda \coloneqq \bigcup \set{ \iota_\beta \given \beta < \lambda }.
    \end{equation*}

    This function is injective because each of the constituent functions it injective.
  \end{itemize}

  We will now thoroughly build the order isomorphism itself.

  Let \( \alpha \leq \delta \) be the (nonstrict) initial segment of \( \delta \) such that \( \iota_\beta \) is fully defined on \( \beta \) for every \( \beta \in \alpha \). More precisely, let
  \begin{equation*}
    \alpha \coloneqq \set{ \beta \in \delta \given \dom(\iota_\beta) = \beta }.
  \end{equation*}

  We will show that \( \alpha \) is a successor ordinal. Note that \( \alpha \) cannot be zero because \( \dom(\iota_0) = \varnothing = 0 \). Aiming at a contradiction, assume that \( \alpha \) is a limit ordinal.

  Since for every \( \beta \in \alpha \) the value \( \iota_{\op{sc}(\beta)}(\beta) \) is defined, we can conclude that the union
  \begin{equation*}
    \bigcup\set{ \iota_\beta \given \beta \in \alpha }
  \end{equation*}
  is an injective function from \( \alpha \) to \( A \).

  If \( \alpha = \delta \), this contradicts our choice of \( \delta \). If \( \alpha \in \delta \), this means that the function \( \iota_\alpha \) is equal to \( \iota_\gamma \) for some \( \gamma < \alpha \). But then \( \iota_{\op{sc}(\gamma)} \) is not defined on the entirely of \( \gamma \). This implies that \( \gamma \geq \alpha \), which contradicts our assumption that \( \gamma < \alpha \).

  Therefore, \( \alpha \) cannot be a limit ordinal. It remains for \( \alpha \) to be a successor ordinal. Then there exists \( \beta \) such that \( \alpha = \op{sc}(\beta) \).

  Suppose that \( \iota_\beta \) is not surjective. We can then construct \( \iota_{\op{sc}(\beta)} \) as in the corresponding recursion step. This will be an injective function from \( \alpha = \op{sc}(\beta) \) to \( A \). But then we would have \( \alpha \in \alpha \), which contradicts \fullref{thm:simple_foundation_theorems/member_of_itself}.

  Therefore, \( \iota_\beta \) is a surjective function.
\end{proof}

\begin{theorem}\label{thm:well_ordered_order_type_existence}
  Any \hyperref[def:well_ordered_set]{well-ordered set} \( (P, \leq) \) is \hyperref[def:preordered_set/homomorphism]{order-isomorphic} to a unique ordinal. This ordinal is called the \term{order type} of \( (P, \leq) \) and is denoted by \( \ord(P, \leq) \) or simply \( \ord(P) \).

  Furthermore, this isomorphism is unique.
\end{theorem}
\begin{proof}
  For \( P = \varnothing \), then the empty function \( \iota: 0 \to P \) is the desired isomorphism.

  We use \fullref{thm:ordinal_bijection_and_choice} on \( P \) with the choice function
  \begin{equation*}
    \begin{aligned}
      &c: \pow(P) \setminus \set{ \varnothing } \to P \\
      &c(B) \coloneqq \min B
    \end{aligned}
  \end{equation*}
  to obtain an ordinal \( \alpha \) and a bijection \( \iota: \alpha \to P \).

  The function \( \iota \) is directly verified to be an order embedding in each of the recursion cases. Therefore, \( \iota \) is a surjective order embedding, that is, an order isomorphism.

  Uniqueness of \( \iota \) follows from \fullref{thm:def:well_ordered_set/unique_isomorphism}.
\end{proof}

\begin{theorem}[Zermelo's well-ordering theorem]\label{thm:well_ordering_theorem}\mcite[thm. 6.4.18(i)]{Hinman2005Logic}
  Any \hyperref[def:set]{set} can be \hyperref[def:well_ordered_set]{well-ordered}.
\end{theorem}
\begin{comments}
  \item Within \hyperref[def:zfc]{\logic{ZF}}, this theorem is equivalent to the \hyperref[def:zfc/choice]{axiom of choice} --- see \fullref{thm:axiom_of_choice_equivalences/well_ordering}.
\end{comments}
\begin{proof}
  \ImplicationSubProof[def:zfc/choice]{the axiom of choice}[thm:well_ordering_theorem]{well-ordering theorem} The empty set is trivially well-ordered.

  Let \( A \) be any nonempty set. By the axiom of choice, there exists a \hyperref[def:choice_function]{choice function} \( c \) for \( A \). We use \fullref{thm:ordinal_bijection_and_choice} on \( A \) and \( c \) to obtain an ordinal \( \alpha \) and a bijection \( \iota: \alpha \to A \). Since \( \alpha \) itself is well-ordered, then the relation
  \begin{equation*}
    x <_A y \T{if and only if} \iota^{-1}(x) <_\alpha \iota^{-1}(y)
  \end{equation*}
  defines a well-order on \( A \).

  \ImplicationSubProof[thm:well_ordering_theorem]{well-ordering theorem}[def:zfc/choice]{axiom of choice} Let \( A \) be any set and suppose that \( < \) well-orders \( A \).

  Define the function
  \begin{equation*}
    \begin{aligned}
      &c: \pow(A) \setminus \set{ \varnothing } \to A \\
      &c(B) \coloneqq \min B.
    \end{aligned}
  \end{equation*}

  In the case where \( A \) is the empty set, \( c \) is the empty function.

  It is clear that \( c(B) \in B \) for every subset \( B \) of \( A \). Therefore, \( c \) is a choice function for \( A \).
\end{proof}

  \subsection{Cardinals}\label{subsec:cardinals}

\begin{definition}\label{def:equinumerosity}\mcite[129 \\ 145]{Enderton1977Sets}
  We say that two sets are \term[bg=равномощни,ru=равномощные]{equinumerous} if there exists a \hyperref[def:function_invertibility/bijective]{bijective function} between them.

  If there exists an \hyperref[def:function_invertibility/injective]{injective} function from \( A \) to \( B \) that is not necessarily \hyperref[def:function_invertibility/surjective]{surjective}, we say that \( A \) is \term{dominated by} \( B \) or that \( B \) \term{dominates} \( A \). If \( B \) dominates \( A \), and they are not equinumerous, we say that \( B \) \term{strictly dominates} \( A \).

  Equinumerosity arises naturally outside the theory of cardinal numbers, unlike set dominance. We are usually instead interested only in injective functions that preserve some structure, i.e. \hyperref[def:first_order_embedding]{embeddings}.
\end{definition}

\begin{lemma}\label{thm:three_equinumerous_sets_lemma}
  If \( A \subseteq B \subseteq C \) and \( A \) is \hyperref[def:equinumerosity]{equinumerous} with \( C \), then \( B \) is equinumerous with \( C \).
\end{lemma}
\begin{proof}
  If \( B = C \), the lemma is trivial since the \hyperref[def:set_valued_map/identity]{identity function} \( \id_B: B \to B \) is bijective. If \( B \subsetneq C \), however, the identity \( \id_B \) must be extended in order to be a bijective function between \( B \) and \( C \). It will actually be simpler for us to define a function from \( C \) to \( B \).

  Let \( f: C \to A \) be a bijective function (such a function exists by the statement of the lemma). Define the set
  \begin{equation*}
    I \coloneqq \bigcap\set{ X \subseteq C \given (C \setminus B) \subseteq X \T{and} f(X) \subseteq X }
  \end{equation*}
  of all intermediate sets between \( C \setminus B \) and \( C \) that are invariant under \( f \).

  Use \hyperref[rem:natural_number_recursion]{natural number recursion} to build the function
  \begin{equation*}
    \begin{aligned}
      &g: C \to B \\
      &g(x) \coloneqq \begin{cases}
        x,    &x \in C \setminus I \\
        f(x), &x \in I
      \end{cases}
    \end{aligned}
  \end{equation*}

  By construction, \( C \setminus B \subseteq I \) and thus \( C \setminus I \subseteq C \setminus (C \setminus B) = B \). Therefore, the image of \( g \) really is \( B \). We must show that \( g \) is injective and surjective.

  Let \( g(x_1) = g(x_2) \) for some members \( x_1 \) and \( x_2 \) of \( C \). If \( x_1 \) and \( x_2 \) both belong to either \( I \) or \( C \setminus I \), it is trivial to see that \( x_1 = x_2 \). It turns out that these are two only possible scenarios. Indeed, without loss of generality, suppose that \( x_1 \in I \) and \( x_2 \in C \setminus I \). Then \( f(x_2) = g(x_2) = g(x_1) = x_1 \). Since \( I \) is invariant under \( f \) and \( x_1 \in I \), we have \( x_2 = f(x_1) \in I \), which contradicts our choice of \( x_2 \). Therefore, \( g \) is an injective function.

  To see that \( g \) is also surjective, suppose that there exists some \( y \in B \setminus g[B] \). If \( y \in I \), then by the invariance of \( f \) we have \( g(y) = f(y) \in I \) and thus \( g(y) \not\in B \), which contradicts our definition of \( g \). If instead \( y \in C \setminus I \), then \( g(y) = y \) and thus \( y \in g[B] \), which contradicts our choice of \( y \). The obtained contradictions show that \( g \) is surjective.
\end{proof}

\begin{theorem}[Cantor-Schr\"oder-Bernstein theorem]\label{thm:cantor_schroder_bernstein_theorem}\mcite[thm. 6.4.4]{Hinman2005}
  If two sets \hyperref[def:equinumerosity]{dominate} each other, they are \hyperref[def:equinumerosity]{equinumerous}.
\end{theorem}
\begin{proof}
  Let \( f: A \to B \) and \( g: B \to A \) be injective functions. From \fullref{thm:function_composition_invertibility} it follows that \( g \bincirc f: A \to A \) is also an injective function. If we restrict its codomain to its image \( g[f[A]] \), it becomes bijective. Hence, \( A \) is equinumerous with \( g[f[A]] \). Since \( g[f[A]] \subseteq g[B] \subseteq A \), from \fullref{thm:three_equinumerous_sets_lemma} it follows that \( A \) is equinumerous with \( g[B] \), which is the desired result.
\end{proof}

\begin{remark}\label{rem:cardinal_definition}
  \hyperref[def:equinumerosity]{Set domination} generalizes the \hyperref[def:subset]{subset relation} between sets.

  If we take a family \( \mscrA \) of sets, then domination is a \hyperref[def:preordered_set]{preorder} rather than a true \hyperref[def:partially_ordered_set]{partial order}.
  \begin{itemize}
    \item Reflexivity follows because the \hyperref[def:set_valued_map/identity]{identity function} for any set is injective.
    \item Transitivity is a consequence of \fullref{thm:function_composition_invertibility}.
    \item Antisymmetry fails if \( A \) dominates \( B \) and \( B \) dominates \( A \), but \( A \neq B \). For example, the map \( n \mapsto 2n \) from all natural numbers \( \BbbN \) to the even natural numbers \( 2\BbbN \) is injective and the identity map on \( 2\BbbN \) is an injective function from \( 2\BbbN \) to \( \BbbN \), however \( \BbbN \neq 2\BbbN \)
  \end{itemize}

  As a matter of fact, \hyperref[def:equinumerosity]{equinumerosity} is also a preorder and the proof for that is identical.

  If we partition \( \mscrA \) using the \hyperref[def:equinumerosity]{equinumerosity relation} if follows from \fullref{thm:preorder_to_partial_order} that the result will be a partial ordered set. The equivalence classes of this partition are then subfamilies of \( \mscrA \) such that every two sets in a single subfamily are equinumerous. For example, if \( \mscrA = \set{ \set{ A, B }, \set{ C, I }, \set{ A }, \set{ C } } \), then the corresponding equivalence classes are \( \set{ \set{ A, B }, \set{ C, I } } \) and \( \set{ \set{ A }, \set{ C } } \).

  Each of these equivalence classes consists of sets that are identical in \enquote{size} (not to be confused with \enquote{large} and \enquote{small} sets as defined in \fullref{def:large_and_small_sets}). In the above example, the corresponding equivalence classes correspond to sets of sizes \( 1 \) and \( 2 \). If we want to extend this notion of \enquote{size} to infinite sets, we must introduce a hierarchy of \enquote{sizes}. A natural candidate for such a hierarchy are the equivalence classes themselves. Unfortunately, this would mean that every family of sets has a different hierarchy. Since the entire universe is only available within the metatheory, we cannot partition the universe itself and must instead resort to finding a concrete representative of each possible equivalence class. We will call these representatives \term{cardinal numbers}.

  As explained in our proof of \fullref{thm:cardinality_existence}, it will be convenient for us to define cardinal numbers as certain \hyperref[def:ordinal]{ordinal numbers} --- see \fullref{def:cardinal}.
\end{remark}

\begin{definition}\label{def:cardinal}
  A \term{cardinal number} or simply \term{cardinal} is an \hyperref[def:ordinal]{ordinal} that is not \hyperref[def:equinumerosity]{equinumerous} with any smaller ordinal. We usually denote them using the small Greek letters \( \kappa \), \( \mu \) and \( \nu \).

  A cardinal is by definition an ordinal and this is useful. For example, the cardinals are well-ordered in the sense of \fullref{thm:ordinals_are_well_ordered}.

  We often regard cardinal numbers as abstract entities, however. It is thus accepted to call the ordinal itself the \term{initial ordinal} of the cardinal.
\end{definition}

\begin{lemma}\label{thm:natural_number_is_not_equinumerous_to_proper_subset}
  No natural number (as a member of \hyperref[thm:smallest_inductive_set_existence]{\( \omega \)}) is equinumerous to a proper subset of itself.
\end{lemma}
\begin{proof}
  We will use \fullref{thm:omega_induction} on \( n \in \omega \). The lemma holds vacuously for \( n = 0 \).

  Now suppose that \( n \) is not equinumerous to a proper subset of itself and, aiming at a contradiction, suppose that there exists a subset \( E \subseteq \op{succ}(n) \) and a bijective function \( {f: \op{succ}(n) \to E} \).
  \begin{itemize}
    \item If \( n \in E \), then \( E \setminus \set{ n } \) is a subset of \( n \) and thus the restriction \( f\restr_n: n \to (E \setminus \set{ n }) \) is a bijective function.
    \item If \( n \not\in E \), then \( E \) is a subset of \( n \) and thus \( f\restr_n: n \to E \) is a bijective function.
  \end{itemize}

  In both cases we obtain that \( n \) is equinumerous with a proper subset of itself, which is a contradiction. Hence, this also holds for \( \op{succ}(n) \).

  The induction principle allows us to conclude that the lemma holds for all natural numbers.
\end{proof}

\begin{proposition}\label{thm:natural_numbers_are_cardinals}
  The natural numbers (as members of \hyperref[thm:smallest_inductive_set_existence]{\( \omega \)}) are \hyperref[def:cardinal]{cardinals}.
\end{proposition}
\begin{proof}
  Fix a natural number \( n \in \omega \). Note that for every \( m < n \), \( m \) is a proper subset of \( n \) by \fullref{thm:ordinal_ordering_via_subsets}. From \fullref{thm:natural_number_is_not_equinumerous_to_proper_subset} it follows that no ordinal strictly smaller than \( n \) is equinumerous with \( n \) and hence \( n \) is an initial ordinal.
\end{proof}

\begin{proposition}\label{thm:omega_is_a_cardinal}
  The \hyperref[thm:smallest_inductive_set_existence]{smallest inductive set} \( \omega \) is a \hyperref[def:cardinal]{cardinal}.

  When regarded as a cardinal, we denote it by \( \aleph_0 \). This is consistent with \fullref{def:aleph_hierarchy}.
\end{proposition}
\begin{proof}
  We will use induction on \( n < \omega \) to show that no function \( f: \omega \to n \) is surjective. This is trivial for \( 0 \). Suppose that it holds for some fixed \( n \). Let \( f: \omega \to \op{succ}(n) \) be any function. Define
  \begin{equation*}
    \begin{aligned}
      &g: \omega \to n \\
      &g(m) \coloneqq \begin{cases}
        f(m), &f(m) < n \\
        0,    &f(m) = n
      \end{cases}
    \end{aligned}
  \end{equation*}

  Our inductive hypothesis states that \( g \) cannot be surjective. Hence, \( f \) also cannot be surjective.

  Therefore, \( \omega \) is an initial ordinal.
\end{proof}

\begin{proposition}\label{thm:cardinality_existence}
  Every set \( A \) is equinumerous with a unique \hyperref[def:cardinal]{cardinal}. We denote this cardinal by \( \card(A) \) and call it the \term{cardinality} of \( A \).
\end{proposition}
\begin{proof}
  By \fullref{thm:well_ordering_theorem} there exists a relation \( \prec \) that well-orders \( A \). The \hyperref[thm:well_ordered_order_type_existence]{order type} \( \ord(A, \prec) \) is an ordinal that is equinumerous with \( A \), however it may not be the smallest one. Fortunately, we can define
  \begin{equation*}
    \card(A) \coloneqq \min\set{ \beta \leq \ord(A, \prec) \given \beta \T{is equinumerous with} A }.
  \end{equation*}
\end{proof}

\begin{proposition}\label{thm:cardinality_order_compatibility}
  The set \( A \) is dominated by \( B \) if and only if \( \card(A) \leq \card(B) \).
\end{proposition}
\begin{proof}
  \SufficiencySubProof First suppose that \( \card(A) \leq \card(B) \). By \fullref{thm:ordinal_ordering_via_subsets}, we have \( \card(A) \subseteq \card(B) \) and thus the identity function \( \id_{\card(A)} \) is an injective function from \( \card(A) \) to \( \card(B) \). Since \( A \) is equinumerous with \( \card(A) \) and \( B \) is equinumerous with \( \card(B) \), by \fullref{thm:function_composition_invertibility} we obtain that there is an injective function from \( A \) to \( B \) and hence \( B \) dominates \( A \).

  \NecessitySubProof Conversely, let \( f: A \to B \) be an injective function. We again use \fullref{thm:function_composition_invertibility} to conclude that \( \card(B) \) dominates \( \card(A) \).

  We will show that \( \card(A) > \card(B) \) leads to a contradiction, which by the trichotomy of cardinals will entail that \( \card(A) \leq \card(B) \). If we suppose that \( \card(A) > \card(B) \), then \( \card(B) \subseteq \card(A) \) and hence the identity on \( \card(B) \) is an injective function. Thus, \( \card(A) \) dominates \( \card(B) \) and vice versa, which by \fullref{thm:cantor_schroder_bernstein_theorem} implies that \( \card(A) \) is equinumerous with \( \card(B) \). It follows that \( \card(A) = \card(B) \), which contradicts our assumption that \( \card(A) > \card(B) \).

  Therefore, \( \card(A) \leq \card(B) \).
\end{proof}

\begin{corollary}\label{thm:set_domination_relation_trichotomy}
  Any two sets are either equinumerous or one strictly dominates the other.
\end{corollary}
\begin{comments}
  \item See also \fullref{def:pigeonhole_principle}.
\end{comments}
\begin{proof}
  Follows from cardinal trichotomy and \fullref{thm:cardinality_order_compatibility}.
\end{proof}

\begin{theorem}[Cantor's power set theorem]\label{thm:cantor_power_set_theorem}\mcite[thm. 6.4.10]{Hinman2005}
  The power set of any set \( A \) \hyperref[def:equinumerosity]{strictly dominates} \( A \). That is,
  \begin{equation*}
    \card(A) < \card(\pow(A)).
  \end{equation*}
\end{theorem}
\begin{proof}
  The function \( x \mapsto \set{ x } \) is clearly an injective function from \( A \) to \( \pow(A) \), therefore \( \pow(A) \) dominates \( A \). The converse is not true, however.

  Indeed, fix some function \( f: A \to \pow(A) \) and define the set
  \begin{equation*}
    B \coloneqq \set{ x \in A \colon x \not\in f(x) }.
  \end{equation*}

  Note that \( B \subseteq A \) and thus \( B \in \pow(A) \), however \( B \) is not in the \hyperref[def:set_valued_map/image]{image} of \( f \) and thus \( f \) is not \hyperref[def:function_invertibility/surjective]{surjective}.

  Since \( f \) was arbitrary, we conclude that no function from \( A \) to \( \pow(A) \) is surjective.
\end{proof}

\begin{definition}\label{def:set_finiteness}
  We say that the set \( A \) is \term{finite} if any of the following equivalent conditions hold:
  \begin{thmenum}
    \thmitem{def:set_finiteness/cardinality}\mcite[def. 6.2.32]{Hinman2005} The \hyperref[thm:cardinality_existence]{cardinality} of \( A \) is a natural number. That is, we have \( \card(A) < \aleph_0 \).

    \medskip

    \thmitem{def:set_finiteness/dedekind}\mcite[def. II.64]{Beman1901Dedekind} The set \( A \) is not \hyperref[def:equinumerosity]{equinumerous} all of its proper subsets. That is, if \( B \) is a proper subset of \( A \), then no function from \( A \) to \( B \) is injective.
  \end{thmenum}

  If a set is not finite, we say that it is \term{infinite}. If a set does not satisfy \fullref{def:set_finiteness/dedekind}, we say that it is \term{Dedekind infinite}.
\end{definition}
\begin{proof}
  \ImplicationSubProof{def:set_finiteness/cardinality}{def:set_finiteness/dedekind} We will use \fullref{thm:omega_induction} on \( n < \aleph_0 \) to prove that all sets of cardinality \( n \) are not Dedekind infinite. The case \( n = 0 \) is vacuous. Suppose that all sets of cardinality \( n \) are not Dedekind infinite and suppose that \( A \) of cardinality \( \op{succ}(n) \) is Dedekind infinite.

  Then there exists a proper subset \( B \) of \( A \) that is equinumerous with \( \op{succ}(n) \). Since \( \op{succ}(n) \) is the cardinality of \( A \), there exists a bijective function \( f: A \to \op{succ}(n) \). Then \( f[B] \subseteq \op{succ}(n) \). Furthermore, the inequality is strict because otherwise any member of \( A \setminus B \) would contradict \fullref{def:pigeonhole_principle}. Therefore, \( f[B] \) is a proper subset of \( \op{succ}(n) \) that is equinumerous with it. But this contradicts \fullref{thm:natural_number_is_not_equinumerous_to_proper_subset}.

  The obtained contradiction shows that \( A \) is also not Dedekind infinite.

  \ImplicationSubProof{def:set_finiteness/dedekind}{def:set_finiteness/cardinality} Let \( A \) be a Dedekind infinite set and let \( f: A \to B \) be a bijective function into some proper subset \( B \) of \( A \). We will construct an injective function from \( \omega \) to \( A \). Fix some member \( x_0 \in A \setminus B \) and recursively define
  \begin{equation*}
    \begin{aligned}
      &g: \omega \to A \\
      &g(n) \coloneqq \begin{cases}
        x_0,         &n = 0 \\
        f(g(n - 1)), &n > 0
      \end{cases}
    \end{aligned}
  \end{equation*}

  We now use induction on \( m \) to prove that \( g(n) = g(m) \) implies \( n = m \). Since \( g(0) \not\in B \) and \( g(n) \in B \) for any \( n > 0 \), the base case holds. Suppose that the inductive hypothesis holds for \( m \) and that for some \( n \) we have \( g(n) = g(m + 1) \). It is clear that \( g(0) \neq g(m + 1) \), so necessarily \( n > 0 \). We have
  \begin{equation*}
    f(g(n - 1))
    =
    g(n)
    =
    g(m + 1)
    =
    f(g(m)),
  \end{equation*}
  which by the inductive hypothesis implies that \( m = n - 1 \). Thus, \( g(m + 1) = g(n) \) and the inductive step is proved.

  Therefore, \( g \) is injective and thus \( A \) dominates \( \omega \). From \fullref{thm:cardinality_order_compatibility} it follows that \( \card(A) \geq \aleph_0 \) and thus \( \card(A) \) is not a natural number.
\end{proof}

\begin{proposition}\label{thm:cardinal_is_finite_iff_successor_ordinal}
  A nonzero cardinal is \hyperref[def:set_finiteness]{finite} if and only if it is a \hyperref[def:successor_and_limit_ordinal]{successor ordinal}.
\end{proposition}
\begin{proof}
  Finite cardinals are natural numbers by definition and all nonzero natural numbers are successor ordinals.

  Conversely, suppose that \( \kappa = \op{succ}(\alpha) \) is a successor ordinal that is a cardinal. That is, \( \kappa \) is not equinumerous with any smaller ordinal and in particular with \( \alpha \). From \fullref{thm:cardinality_order_compatibility} it follows that \( \card(\alpha) < \kappa \).

  Let \( A \subseteq \kappa \) be a proper subset of \( \kappa \). We want to show that \( \card(A) < \kappa \).

  If \( \alpha \not\in A \), define \( B \coloneqq A \). Otherwise, pick some member \( x_0 \in \kappa \setminus A \) and define
  \begin{equation*}
    B \coloneqq (A \cup \set{ x_0 }) \setminus \set{ \alpha }.
  \end{equation*}

  In both cases we have \( \card(A) = \card(B) \), but unlike \( A \), \( B \) is always a subset of \( \alpha \) since \( \kappa = \alpha \cup \set{ \alpha } \).

  Therefore,
  \begin{equation*}
    \card(A) = \card(B) \leq \card(\alpha) < \kappa.
  \end{equation*}

  Hence, \( \kappa \) dominates every proper subset, which by \fullref{def:set_finiteness/dedekind} means that \( \kappa \) is a finite cardinal.
\end{proof}

\begin{corollary}\label{thm:cardinal_is_infinite_iff_limit_ordinal}
  A cardinal is \hyperref[def:set_finiteness]{infinite} if and only if it is a \hyperref[def:successor_and_limit_ordinal]{limit ordinal}.
\end{corollary}
\begin{proof}
  This is the contraposition to \fullref{thm:cardinal_is_finite_iff_successor_ordinal} excluding the zero cardinal.
\end{proof}

\begin{proposition}\label{thm:power_set_finiteness}
  A set is finite if and only if its \hyperref[def:basic_set_operations/power_set]{power set} is finite.
\end{proposition}

\begin{proposition}\label{thm:finite_unions_and_products_are_finite}
  All finite \hyperref[def:basic_set_operations/union]{unions} and \hyperref[def:cartesian_product]{Cartesian products} of finite sets are finite.
\end{proposition}

\begin{definition}\label{def:successor_and_limit_cardinal}
  \begin{thmenum}
    \thmitem{def:successor_and_limit_cardinal/successor} If \( \kappa \) is the smallest cardinal such that \( \mu < \kappa \) for some other cardinal \( \mu \), we say that \( \kappa \) is the \term{successor} of \( \mu \) and that \( \kappa \) is itself a \term{successor cardinal}.

    The existence of \( \kappa \) is guaranteed by \fullref{thm:successor_cardinal_existence}, but it is natural to ask whether \( \kappa \) can be constructed from \( \mu \) similarly to how the \hyperref[def:ordinal_successor]{ordinal successor operator} gives us a successor ordinal. This turns out to be a deep question --- see \fullref{hyp:generalized_continuum_hypothesis}.

    \thmitem{def:successor_and_limit_cardinal/weak_limit} If \( \kappa > 0 \) is not the successor cardinal of any other cardinal, we say that it is a \term{weak limit cardinal}.

    See \fullref{thm:weak_limit_cardinal_equivalences} for some equivalent conditions.

    \thmitem{def:successor_and_limit_cardinal/strong_limit} We say that \( \kappa \) is a \term{strong limit cardinal} if \( \mu < \kappa \) implies that \( \card(\pow(\mu)) < \kappa \).

    We can benefit from using forward references to \fullref{subsec:transfinite_arithmetic}, more precisely \fullref{thm:cardinal_exponentiation_power_set}, which justifies using \hyperref[def:cardinal_arithmetic/exponentiation]{cardinal exponentiation} to rewrite the condition for \( \kappa \) being a strong limit cardinal as
    \begin{equation*}
      \mu < \kappa \T{implies} 2^\mu < \kappa.
    \end{equation*}

    Every strong limit cardinal is a weak limit cardinal as shown in \fullref{thm:strong_limit_cardinal_is_weak_limit}, however the converse is only true assuming \fullref{hyp:generalized_continuum_hypothesis} --- see \fullref{thm:limit_cardinals_and_gch}.

    Strong limit cardinals are further motivated by the usage of \hyperref[rem:strongly_inaccessible_cardinal]{regular strong limit cardinals} in \fullref{thm:strong_regular_cardinal_stages}.
  \end{thmenum}

  These notions should not be confused with \hyperref[def:successor_and_limit_ordinal]{successor and limit ordinals}.
\end{definition}

\begin{proposition}\label{thm:successor_cardinal_existence}
  For any cardinal there exists a successor cardinal.
\end{proposition}
\begin{proof}
  Fix a cardinal \( \kappa \). By \fullref{thm:hartogs_lemma}, there exists a smallest ordinal \( \alpha \) such that \( \kappa \) does not dominate \( \alpha \). Thus, \( \alpha \) is the initial ordinal of a cardinal \( \mu \) because it is not equinumerous with any smaller ordinal.

  \Fullref{thm:set_domination_relation_trichotomy} implies that \( \kappa < \mu \).

  Furthermore, every cardinal smaller than \( \mu \) does not dominate \( \kappa \), i.e. if \( \nu < \mu \), then \( \nu \geq \kappa \).

  Therefore, \( \mu \) is the successor cardinal of \( \kappa \).
\end{proof}

\begin{proposition}\label{thm:union_of_set_of_cardinals}\mcite[lemma 8B]{Enderton1977Sets}
  If \( A \) is a set of cardinals, then \( \bigcup A \) is a cardinal. Furthermore, \( \bigcup A \) is the supremum of \( A \) with respect to cardinal ordering.

  See a more thorough discussion of a similar issue in \fullref{thm:union_of_set_of_ordinals}.
\end{proposition}
\begin{proof}
  From \fullref{thm:union_of_set_of_ordinals} it follows that \( \bigcup A \) is an ordinal. Then there exists some cardinal \( \kappa \in A \) such that \( \alpha \in \kappa \).

  We have \( \kappa \subseteq \bigcup A \). Thus, with regard to ordinal ordering, \( \alpha < \kappa \leq \bigcup A \). But since \( \kappa \) is a cardinal, it is not equinumerous with \( \alpha \) and hence \( \bigcup A \) is also not equinumerous with \( \alpha \).

  Therefore, \( \bigcup A \) is a cardinal. It follows from \fullref{thm:ordinal_ordering_via_subsets} that it is also the supremum of \( A \).
\end{proof}

\begin{definition}\label{def:aleph_hierarchy}\mcite[def. 6.4.27]{Hinman2005}
  We use transfinite recursion to define, for each ordinal \( \alpha \), the cardinal
  \begin{equation}\label{eq:def:aleph_hierarchy}
    \aleph_\alpha \coloneqq \begin{cases}
      \omega,                                        &\alpha = 0 \\
      \T{successor cardinal of} \beta,               &\alpha = \op{succ}(\beta) \\
      \sup\set{ \aleph_\beta \given \beta < \alpha } &\alpha \T{is a limit ordinal}.
    \end{cases}
  \end{equation}

  We denote the initial ordinal of \( \aleph_\alpha \) by \( \omega_\alpha \). In particular, \( \omega_0 = \omega \) and \( \omega_1 \) is the first \hyperref[def:set_countability/uncountable]{uncountable ordinal}.
\end{definition}
\begin{comments}
  \item Note that \( \aleph_\lambda \) exists and is a cardinal for every limit ordinal \( \lambda \) as a consequence of \fullref{thm:union_of_set_of_cardinals}.

  \item See \fullref{rem:unbounded_transfinite_recursion} for some technical details.

  \item This hierarchy is important because it describes all infinite cardinals as shown in \fullref{thm:infinite_cardinal_is_aleph}. It is intimately connected to the simpler \hyperref[def:beth_hierarchy]{\( \beth \) hierarchy} via \fullref{hyp:generalized_continuum_hypothesis}.
\end{comments}

\begin{remark}[Unbounded transfinite recursion]\label{rem:unbounded_transfinite_recursion}
  Although we cannot formally do unbounded transfinite recursion, there is an easy way to circumvent this.

  Formally, in \fullref{def:aleph_hierarchy}, for every ordinal \( \alpha \) we use \fullref{thm:bounded_transfinite_recursion} define a \( \alpha \)-indexed transfinite sequence \( \aleph_0, \aleph_1, \ldots, \aleph_\omega, \ldots \) and then use the sequence to define \( \aleph_\alpha \). The definition does not depend on any particular ordinal \( \alpha \), however, and thus all ways to obtain \( \aleph_\alpha \) are equivalent.
\end{remark}

\begin{proposition}\label{thm:aleph_hierarchy_is_strictly_monotone}
  If \( \alpha < \beta \), then \( \aleph_\alpha < \aleph_\beta \).
\end{proposition}
\begin{proof}
  We will use \fullref{rem:transfinite_induction} on \( \beta \).
  \begin{itemize}
    \item The condition \( \alpha < \beta \) is vacuously false for the base case \( \beta = 0 \), hence by \eqref{eq:def:intuitionistic_propositional_deductive_systems/rules/efq} the statement vacuously holds.

    \item Suppose that \( \alpha < \beta \) and \( \aleph_\alpha < \aleph_\beta \). We then have \( \alpha < \op{succ}(\beta) \) and, since, \( \aleph_{\op{succ}(\beta)} > \aleph_\beta \), also \( \aleph_\alpha < \aleph_{\op{succ}(\beta)} \).

    \item Let \( \lambda \) be a limit ordinal and suppose that the proposition holds for all \( \beta < \lambda \) and for arbitrary \( \alpha \). Then \( \aleph_\beta \subseteq \aleph_\lambda \) for every \( \beta < \lambda \), hence \( \aleph_\beta \leq \aleph_\lambda \) by \fullref{thm:ordinal_ordering_via_subsets}.

    Suppose that \( \alpha < \lambda \).
    \begin{itemize}
      \item If there exists some \( \beta_0 < \lambda \) such that \( \alpha < \beta_0 \), clearly \( \aleph_\alpha < \aleph_{\beta_0} \leq \aleph_\lambda \).
      \item If \( \alpha > \beta \) for all \( \beta < \lambda \), then \( \alpha \) is an upper bound of the set \( \lambda = \set{ \beta \given \beta < \lambda } \). Hence, \( \alpha \geq \lambda \), which contradicts our choice of \( \alpha \).
    \end{itemize}

    Therefore, \( \aleph_\alpha < \aleph_\lambda \).
  \end{itemize}
\end{proof}

\begin{remark}[Cardinal recursion and induction]\label{rem:cardinal_transfinite_recursion_and_induction}
  Just like we have (bounded and unbounded) transfinite recursion and induction on ordinals, we also have transfinite recursion and induction on cardinals.
  \begin{itemize}
    \item We only consider cardinals rather than arbitrary ordinals.
    \item In its structured form presented in \fullref{rem:transfinite_induction}, rather than considering successor ordinals and limit ordinals, we consider successor cardinals and weak limit cardinals.
  \end{itemize}

  Thus, recursion and induction on cardinals is formally quite different from the equivalent statements for ordinals. The usage of the two is analogous, however.

  See \fullref{thm:infinite_cardinal_is_aleph} for how this principles is used.
\end{remark}

\begin{proposition}\label{thm:infinite_cardinal_is_aleph}\mcite[thm. 8A]{Enderton1977Sets}
  For every \hyperref[def:set_finiteness]{infinite cardinal} \( \kappa \) there exists an ordinal \( \alpha \) such that \( \kappa = \aleph_\alpha \).
\end{proposition}
\begin{proof}
  We will use \fullref{rem:cardinal_transfinite_recursion_and_induction} on \( \kappa \).

  \begin{itemize}
    \item The base case \( \kappa = 0 \) vacuously holds because \( 0 \) is not an infinite cardinal. The actual base case is \( \kappa = \omega = \aleph_0 \), which holds by definition. This case may not seem formally necessary, however we need to consider it separately from the limit case and calling it the \enquote{base case} seems most appropriate.

    \item If \( \kappa = \aleph_\alpha \) and \( \mu \) is the successor cardinal of \( \kappa \), then by definition \( \kappa = \aleph_{\op{succ}(\alpha)} \).

    \item Finally, let \( \kappa \) be a limit cardinal and let \( \mu = \aleph_{\alpha_\mu} \) for every infinite cardinal \( \mu < \kappa \). Define
    \begin{equation*}
      \alpha \coloneqq \bigcup\set{ \alpha_\mu \given \mu < \kappa }.
    \end{equation*}

    We have
    \begin{align*}
      \kappa
      &\reloset {\eqref{eq:def:aleph_hierarchy}} =
      \bigcup\set{ \aleph_{\alpha_\mu} \given \mu < \kappa }
      \reloset {\eqref{eq:thm:ordinal_addition_is_monotone/right}} \leq \\ &\leq
      \bigcup\set[\Big]{ \aleph_\beta \given \beta < \sup\set{ \alpha_\mu \given \mu < \kappa } }
      = \\ &=
      \bigcup\set{ \aleph_\beta \given \beta < \alpha }
      \reloset {\ref{thm:ordinal_is_set_of_smaller_ordinals}} = \\ &=
      \aleph_\alpha.
    \end{align*}

    If we suppose that \( \kappa < \aleph_\alpha \), then similarly to \fullref{thm:ordinal_ordering_via_addition} there exists some ordinal \( \beta_0 < \alpha \) such that \( \aleph_{\beta_0} > \aleph_{\alpha_\mu} \) for every \( \mu < \kappa \). In particular, \fullref{thm:aleph_hierarchy_is_strictly_monotone} implies that \( \beta_0 > \alpha_\mu \) for every \( \mu < \kappa \). Thus,
    \begin{equation*}
      \underbrace{\bigcup\set{ \alpha_\mu \given \mu < \kappa }}_{\alpha} \leq \beta_0 < \alpha,
    \end{equation*}
    which is a contradiction. Therefore, \( \kappa = \aleph_\alpha \).
  \end{itemize}
\end{proof}

\begin{corollary}\label{thm:weak_limit_cardinal_equivalences}.
  The cardinal \( \kappa = \aleph_\alpha \) is a weak limit cardinal if and only if \( \alpha \) is a limit ordinal.

  In particular, \( \kappa > 0 \) is a weak limit cardinal if and only if \( \mu < \kappa \) implies that \( \nu < \kappa \), where \( \nu \) is the successor cardinal of \( \mu \).
\end{corollary}
\begin{proof}
  Clear from \fullref{def:aleph_hierarchy}.
\end{proof}

\begin{definition}\label{def:set_countability}\mcite[def. 6.2.32]{Hinman2005}
  We will introduce the notion of \term{countability}, which generalizes \hyperref[def:set_finiteness]{finiteness}.

  \begin{thmenum}
    \thmitem{def:set_countability/countably_infinite} The smallest infinite cardinal is \hyperref[def:aleph_hierarchy]{\( \aleph_0 \)}. Every set with cardinality \( \aleph_0 \) is called \term{countably infinite}. The countably infinite sets are precisely those that can be ordered into a \hyperref[def:sequence]{sequence}.

    \thmitem{def:set_countability/at_most_countable} A set that is either finite or countably infinite is called \term{at most countable}.

    \thmitem{def:set_countability/uncountable} Any set that \hyperref[def:equinumerosity]{strictly dominates} \( \aleph_0 \) is called \term{uncountable}. The smallest uncountable cardinal is the successor cardinal \( \aleph_1 \) of \( \aleph_0 \).

    \thmitem{def:set_countability/continuum} The cardinality of \( \pow(\aleph_0) \) has a special name --- the \term{cardinality of the continuum}. It is sometimes denoted by \( c \). See \fullref{hyp:continuum_hypothesis} for its relation to \( \aleph_1 \).
  \end{thmenum}

  See \fullref{rem:countability_etymology} for additional terminology that is potentially more ambiguous.
\end{definition}

\begin{remark}\label{rem:countability_etymology}
  Some authors, for example \incite[159]{Enderton1977Sets}, Peter Hinman in \incite[def. 6.2.32(v)]{Hinman2005} and \incite[4]{Engelking1989}, use the shortened term \enquote{countable} to mean \enquote{at most countable}, while other authors like \incite[def. 2.4(c)]{Rudin1976Principles} use \enquote{countable} to mean \enquote{countably infinite}. The term \enquote{denumerable} is also used for \enquote{countably infinite}, for example by Peter Hinman in \incite[def. 6.2.32(iv)]{Hinman2005}, \incite[123]{MacLane1998} and \incite[def. 2.4]{Rudin1976Principles}.
\end{remark}

\begin{conjecture}[Continuum hypothesis]\label{hyp:continuum_hypothesis}\mcite[165]{Enderton1977Sets}
  The \hyperref[def:set_countability/continuum]{cardinality of the continuum} \( c \) is the \hyperref[def:set_countability/uncountable]{first uncountable cardinal} \( \aleph_1 \).
\end{conjecture}
\begin{comments}
  \item Compare this to \fullref{hyp:generalized_continuum_hypothesis}.
  \item The hypothesis has been shown by G\"odel not to be disprovable in \hyperref[def:set]{\logic{ZFC}} and by Cohen not to be provable in \logic{ZFC}.
\end{comments}

\begin{proposition}\label{thm:omega_equinumerous_with_omega_squared}
  The smallest inductive set \( \omega \) is equinumerous with \( \omega \times \omega \).
\end{proposition}
\begin{proof}
  We can give a short proof using \fullref{thm:cantor_schroder_bernstein_theorem} using the injective functions \( (n, m) \mapsto 2^n 3^m \) in one direction and \( n \mapsto (n, 0) \) in the other direction. Proving the injectivity of \( f \), however, requires \fullref{thm:fundamental_theorem_of_arithmetic}, and we prove the latter using machinery from \fullref{sec:ring_theory}. We will instead give a direct proof with an explicit construction. We will construct a bijective function from \( \omega \times \omega \) to \( \omega \) --- the function visualized in \cref{fig:thm:omega_equinumerous_with_omega_squared}.

  We begin by defining the diagonal in \cref{fig:thm:omega_equinumerous_with_omega_squared}. For each natural number \( k \), define the set of pairs that sum to \( k \):
  \begin{equation*}
    A_k \coloneqq \set{ (n, m) \in \omega \times \omega \given n + m = k }.
  \end{equation*}

  We can use induction to show that \( \card(A_k) = k + 1 \). That is, \( A_k \) can \enquote{fit} \( k + 1 \) numbers. We can now define the function
  \begin{equation*}
    \begin{aligned}
      &d: \omega \to \omega \\
      &d(n) \coloneqq \sum_{k=0}^n \card(A_n)
    \end{aligned}
  \end{equation*}
  that gives us how many numbers we have already \enquote{fit} in the first \( n \) diagonals.

  It is clear that the point \( (n, m) \) lies in \( A_{n + m} \). We want to know how many numbers we have \enquote{fit} in the diagonal prior to that, for which we can use \( d(n + m - 1) \). This leads us to the definition
  \begin{equation*}
    \begin{aligned}
      &f: \omega \times \omega \to \omega \\
      &f(n, m) \coloneqq \begin{cases}
        0,                &n + m = 0 \\
        d(n + m - 1) + n, &n + m > 0. \\
      \end{cases}
    \end{aligned}
  \end{equation*}

  We will first show that \( f \) is injective using induction on \( n + m \) (that is, on the diagonals). Suppose that \( f(n_1, m_1) = f(n_2, m_2) \) implies \( n_1 = n_2 \) and \( m_1 = m_2 \) for all pairs with a sum less than \( l \). Let \( (n_1, m_1) \) and \( (n_2, m_2) \) be two points in \( A_l \) such that \( f(n_1, m_1) = f(n_2, m_2) \). The cases \( l <= 1 \) are trivial, so suppose that \( l > 1 \). Then
  \begin{equation*}
    f(n_1, m_1)
    =
    d(n_1 + m_1 - 1) + n_1
    =
    l - 1 + d(n_1 + m_1 - 2) + n_1
    =
    l + f(n_1 - 1, m_1).
  \end{equation*}

  We can now apply the inductive hypothesis and obtain that \( n_1 = n_2 \) and \( m_1 = m_2 \). This proves injectivity.

  To see that \( f \) is surjective, we will use induction on \( k \in \omega \). The base case is again trivial. Now suppose that \( n + m > 0 \) and
  \begin{equation*}
    f(n, m) = d(n + m - 1) + n = k.
  \end{equation*}

  We have two cases:
  \begin{itemize}
    \item If \( m = 0 \), then
    \begin{equation*}
      f(0, n + m + 1) = d(n + m) + 0 = d(n + m - 1) + n + m = f(n, m) + 1.
    \end{equation*}

    \item If \( m > 0 \), then
    \begin{equation*}
      f(n + 1, m - 1) = d(n + m - 1) + n + 1 = f(n, m) + 1.
    \end{equation*}
  \end{itemize}

  In both cases we have shown that \( k + 1 = f(n, m) + 1 \) is in the image of \( f \), which concludes our proof of surjectivity (and hence bijectivity).

  Lastly, although it is not necessary for the proof, we can expand the definition of \( d \) to see that \( f \) is actually a \hyperref[def:polynomial_algebra/polynomial]{polynomial}:
  \begin{equation*}
    f(n, m)
    =
    \sum_{k=0}^{n + m - 1} (k + 1) + n
    =
    \sum_{k=1}^{n + m} k + n
    \reloset {\eqref{eq:thm:arithmetic_progression_partial_sums}} =
    \frac {(n + m) (n + m + 1)} 2 + n.
  \end{equation*}

  As an added benefit, this polynomial also handles the case \( n = m = 0 \).

  \begin{figure}[!ht]
    \hfill
    \includegraphics[page=1]{output/thm__omega_equinumerous_with_omega_squared}
    \hfill
    \includegraphics[page=2]{output/thm__omega_equinumerous_with_omega_squared}
    \hfill\hfill
    \caption{Visualization on an integer coordinate grid of the diagonal sets \( A_k \) and of the bijective function defined in \fullref{thm:omega_equinumerous_with_omega_squared}.}\label{fig:thm:omega_equinumerous_with_omega_squared}
  \end{figure}
\end{proof}

\begin{corollary}\label{thm:countable_product_of_countable_sets}
  A finite \hyperref[def:cartesian_product]{Cartesian product} of at most countable sets is at most countable.
\end{corollary}
\begin{proof}
  Let \( A_1, \ldots, A_n \) be a finite family of at most countable sets.

  Suppose that the inductive hypothesis holds for \( n \). Countability ensures that there exist injective functions \( g: A_1 \times \cdots \times A_n \times \omega \) and \( h: A_{n+1} \to \omega \). Denote by \( f \) the bijective function from \( \omega \) to \( \omega \times \omega \) obtained in \fullref{thm:omega_equinumerous_with_omega_squared} and define
  \begin{equation*}
    \begin{aligned}
      &F: A_1 \times \cdots \times A_n \times A_{n+1} \to \omega \\
      &F(a_1, \ldots, a_n, a_{n+1}) \coloneqq f(g(a_1, \ldots, a_n), h(a_{n+1}))
    \end{aligned}
  \end{equation*}

  By \fullref{thm:function_superposition_invertibility}, the function \( F \) is injective as a \hyperref[rem:function_superposition]{superposition} of injective functions.

  Therefore, \( \omega \) dominates the product \( A_1 \times \cdots \times A_n \), i.e. the product is countable.
\end{proof}

\begin{proposition}\label{thm:countably_infinite_union_of_countably_infinite_sets}\mcite[thm. 6Q]{Enderton1977Sets}
  A \hyperref[def:set_countability/countably_infinite]{countably infinite} union of countably infinite sets is countably infinite.
\end{proposition}
\begin{proof}
  Let \( \seq{ A_k }_{k \in \omega} \) be a countably infinite family of countably infinite sets. Define instead the disjoint family
  \begin{equation*}
    B_k \coloneqq \set{ (k, a) \given a \in A_k }.
  \end{equation*}

  Denote the union of the former family by \( A \) and of the latter family by \( B \). Since each \( A_k \) is countably infinite, so is \( A \). Furthermore, there exists an obvious injective function from \( B \) to \( A \), thus
  \begin{equation}\label{eq:thm:countably_infinite_union_of_countably_infinite_sets/union_card_inequality}
    \aleph_0 \leq \card(A) \leq \card(B).
  \end{equation}

  Define the set-valued mapping
  \begin{equation*}
    \begin{aligned}
      &G: \omega \to \pow(\fun(\omega, B)) \\
      &G(k) \coloneqq \set{ g: \omega \to B_k \given g \T{is bijective} }.
    \end{aligned}
  \end{equation*}

  The value \( G(k) \) is nonempty because we have assumed that \( B_k \) is countable for every \( k \in \mscrK \). \Fullref{thm:existence_of_single_valued_selections} gives us a single-valued function \( G: \omega \to \fun(\omega, B) \). Since the family \( \seq{ B_k }_{k \in \omega} \) is disjoint, \( G \) is injective. We can thus \hyperref[def:function_currying]{uncurry} \( G \) to obtain a function \( g \) from \( \omega \times \omega \) to \( B \).

  To prove that \( g \) is injective, suppose that \( g(n_1, m_1) = g(n_2, m_2) \). Then
  \begin{equation*}
    G(n_1)(m_1) = g(n_1, m_1) = g(n_2, m_2) = G(n_2)(m_2).
  \end{equation*}

  Note that \( n_1 \neq n_2 \) would lead to a contradiction because  \( \seq{ B_k }_{k \in \omega} \) is a disjoint family. So \( n_1 = n_2 \) and we obtain \( m_1 = m_2 \) since the function \( G(n_1): \omega \to B_k \) is injective. Therefore, \( g \) itself is also injective.

  It is also surjective because for every \( a \in B \) there exists some \( k \in \omega \) such that
  \begin{equation*}
    a \in B_k = \img(G(k)).
  \end{equation*}

  Denote by \( f \) the bijective function from \( \omega \) to \( \omega \times \omega \) obtained in \fullref{thm:omega_equinumerous_with_omega_squared}. Then the function \( f \bincirc g: \omega \to B \) is bijective by \fullref{thm:function_composition_invertibility}. Therefore, the union \( B \) is countably infinite. From \eqref{eq:thm:countably_infinite_union_of_countably_infinite_sets/union_card_inequality} it follows that \( A \) is also countably infinite.
\end{proof}

\begin{corollary}\label{thm:at_most_countable_union_of_at_most_countable_sets}
  An \hyperref[def:set_countability/at_most_countable]{at most countable} union of at most countable sets is at most countable.
\end{corollary}
\begin{proof}
  Let \( \seq{ A_k }_{k \in \mscrK} \) be an at most countable family of at most countable sets. Denote their union by \( A \). For every \( A_k \) let \( g_k: A_k \to \omega \) be an injective function and define the \hyperref[def:disjoint_union]{disjoint union}
  \begin{equation*}
    B_k \coloneqq A_k \amalg \set{ n \in \omega \given n \not\in \img(g_k) }
  \end{equation*}
  and the bijective function
  \begin{equation*}
    \begin{aligned}
      &h_k: B_k \to \omega \\
      &h_k(x) \coloneqq \begin{cases}
        (0, g_k(x)), &x \in A_k \\
        (1, x),      &\T{otherwise.}
      \end{cases}
    \end{aligned}
  \end{equation*}

  For every \( k \in \omega \setminus \mscrK \) instead define
  \begin{equation*}
    B_k \coloneqq \set{ (k, n) \given n \in \omega }
  \end{equation*}
  and let \( h_k: B_k \to \omega \) be the obvious bijective function.

  Then
  \begin{equation*}
    A
    =
    \bigcup_{k \in \mscrK} A_k
    \subseteq
    \bigcup_{k \in \mscrK} B_k
    \subseteq
    \bigcup_{k \in \omega} B_k
  \end{equation*}
  and the latter is countably infinite by \fullref{thm:countably_infinite_union_of_countably_infinite_sets}. Therefore, \( A \) is at most countable.
\end{proof}

  \subsection{Transfinite arithmetic}\label{subsec:transfinite_arithmetic}

Our purpose is to extend natural number arithmetic to ordinals and cardinals. It turns out that the two are rather different. We will first introduce some additional concepts, however.

\begin{definition}
  Let \( A \) and \( B \) be sets of \hyperref[def:ordinal]{ordinals}. We say
\end{definition}

\begin{definition}\label{def:ordinal_arithmetic}
  We recursively define arithmetic operations for arbitrary \hyperref[def:ordinal]{ordinals} as extensions of the corresponding operations of \hyperref[def:peano_arithmetic]{Peano arithmetic}.

  \begin{thmenum}
    \thmitem{def:ordinal_arithmetic/addition}\mcite[def. 2.18]{Jech2003} The \term{sum} of \( \alpha \) and \( \beta \) extends \eqref{eq:def:peano_arithmetic/PA4} and \eqref{eq:def:peano_arithmetic/PA5} with a case for limit ordinals:
    \begin{equation}\label{eq:def:ordinal_arithmetic/addition}
      \alpha + \beta \coloneqq \begin{cases}
        \alpha,                                            &\beta = 0 \\
        \op{succ}(\alpha + \gamma),                        &\beta = \op{succ}(\gamma) \\
        \sup\set{ \alpha + \gamma \given \gamma < \beta }, &\beta \T{is a limit ordinal} \\
      \end{cases}
    \end{equation}

    From \fullref{thm:union_of_set_of_ordinals} it follows that the in limit case \( \alpha + \beta \) is the smallest ordinal strictly larger than \( \alpha + \gamma \) for any \( \gamma < \beta \).

    \thmitem{def:ordinal_arithmetic/multiplication}\mcite[def. 2.19]{Jech2003} Analogously, the \term{product} of \( \alpha \) and \( \beta \) extends \eqref{eq:def:peano_arithmetic/PA6} and \eqref{eq:def:peano_arithmetic/PA7}:
    \begin{equation}\label{eq:def:ordinal_arithmetic/multiplication}
      \alpha \cdot \beta \coloneqq \begin{cases}
        0,                                                     &\beta = 0 \\
        \alpha \cdot \gamma + \alpha,                          &\beta = \op{succ}(\gamma) \\
        \sup\set{ \alpha \cdot \gamma \given \gamma < \beta }, &\beta \T{is a limit ordinal} \\
      \end{cases}
    \end{equation}

    \thmitem{def:ordinal_arithmetic/exponentiation}\mcite[def. 2.20]{Jech2003} Exponentiation extends \fullref{def:monoid/exponentiation}:
    \begin{equation}\label{eq:def:ordinal_arithmetic/exponentiation}
      \alpha^\beta \coloneqq \begin{cases}
        1,                                               &\beta = 0 \\
        \alpha^\gamma \cdot \alpha,                      &\beta = \op{succ}(\gamma) \\
        \sup\set{ \alpha^\gamma \given \gamma < \beta }, &\beta \T{is a limit ordinal} \\
      \end{cases}
    \end{equation}
  \end{thmenum}
\end{definition}

\begin{remark}\label{rem:ordinal_successor_via_addition}
  For any ordinal \( \alpha \) we have
  \begin{equation*}
    \op{succ}(\alpha)
    \reloset {\ref{eq:def:ordinal_arithmetic/addition}} =
    \op{succ}(\alpha + 0)
    \reloset {\ref{eq:def:ordinal_arithmetic/addition}} =
    \alpha + \op{succ}(0)
    =
    \alpha + 1.
  \end{equation*}

  We will occasionally use the later notation.

  Note that for infinite ordinals \( \op{succ}(\alpha) = 1 + \alpha \) as discussed in \fullref{ex:ordinal_addition}.

  This is an extension of \fullref{rem:natural_number_successor_via_addition}.
\end{remark}

\begin{proposition}\label{thm:ordinal_addition_is_monotone}
  \hyperref[def:ordinal_arithmetic/addition]{Ordinal addition} has the following monotonicity properties:
  \begin{thmenum}
    \thmitem{thm:ordinal_addition_is_monotone/left} Left addition is \hyperref[eq:def:order_homomorphism/increasing/strict]{strictly order-preserving}:
    \begin{equation}\label{eq:thm:ordinal_addition_is_monotone/left}
      \alpha < \beta \T{implies} \gamma + \alpha < \gamma + \beta.
    \end{equation}

    \thmitem{thm:ordinal_addition_is_monotone/right} Right addition is \hyperref[def:order_homomorphism/increasing]{order-preserving}:
    \begin{equation}\label{eq:thm:ordinal_addition_is_monotone/right}
      \alpha < \beta \T{implies} \alpha + \gamma \leq \beta + \gamma.
    \end{equation}

    See \fullref{ex:ordinal_addition} for examples where the strict inequality fails.
  \end{thmenum}
\end{proposition}
\begin{proof}
  \SubProofOf{thm:ordinal_addition_is_monotone/left} We proceed by induction on \( \beta \).
  \begin{itemize}
    \item The condition \( \alpha < \beta \) is vacuously false for the base case \( \beta = 0 \), hence by \eqref{eq:def:intuitionistic_propositional_deductive_systems/rules/efq} the statement vacuously holds.

    \item Fix some nonzero \( \beta \) and some \( \alpha < \beta \). If \( \gamma + \alpha < \gamma + \beta \), then
    \begin{equation*}
      \gamma + \alpha < \gamma + \beta < \op{succ}(\gamma + \beta) = \gamma + \op{succ}(\beta).
    \end{equation*}

    Since \( \beta < \op{succ}(\beta) \), we have used the inductive hypothesis to conclude that
    \begin{equation*}
      \alpha < \op{succ}(\beta) \T{implies} \gamma + \alpha < \gamma + \op{succ}(\beta).
    \end{equation*}

    \item Let \( \lambda \) be a limit ordinal and suppose that \eqref{eq:thm:ordinal_addition_is_monotone/left} holds for all \( \beta < \lambda \). Let \( \alpha < \lambda \). Then \( \op{succ}(\alpha) < \lambda \) since \( \lambda \) is a limit ordinal and thus
    \begin{equation*}
      \gamma + \alpha
      <
      \gamma + \op{succ}(\alpha)
      \leq
      \sup\set{ \gamma + \beta \given \beta < \lambda }
      =
      \gamma + \lambda.
    \end{equation*}
  \end{itemize}

  \SubProofOf{thm:ordinal_addition_is_monotone/right} We proceed by induction on \( \gamma \).
  \begin{itemize}
    \item The base case \( \gamma = 0 \) is vacuous.
    \item If \( \alpha + \gamma < \beta + \gamma \), then
    \begin{equation*}
      \alpha + \op{succ}(\gamma)
      \reloset {\eqref{eq:def:ordinal_arithmetic/addition}} =
      \op{succ}(\alpha + \gamma)
      \reloset {\ref{thm:ordinal_successor_strictly_monotone_on_ordinals}} <
      \op{succ}(\beta + \gamma)
      \reloset {\eqref{eq:def:ordinal_arithmetic/addition}} =
      \beta + \op{succ}(\gamma).
    \end{equation*}

    \item Let \( \lambda \) be a limit ordinal and suppose that the lemma holds for every \( \gamma < \lambda \). That is, for every \( \gamma < \lambda \) we have
    \begin{equation*}
      \alpha + \gamma < \beta + \gamma.
    \end{equation*}

    Thus,
    \begin{equation*}
      \alpha + \lambda
      =
      \sup\set{ \alpha + \gamma \given \gamma < \lambda }
      \leq
      \sup\set{ \beta + \gamma \given \gamma < \lambda }
      =
      \beta + \lambda.
    \end{equation*}

    We cannot make a stronger conclusion here --- see \fullref{ex:ordinal_addition} for a counterexample.
  \end{itemize}
\end{proof}

\begin{proposition}\label{thm:ordinal_ordering_via_addition}
  For any two ordinals \( \alpha \) and \( \beta \) it holds that \( \alpha \leq \beta \) if and only if there exists an ordinal \( \gamma \) such that \( \alpha + \gamma = \beta \). This ordinal is unique and satisfies \( \gamma \leq \beta \).

  The strict inequality \( \alpha < \beta \) holds if and only if \( \gamma \neq 0 \).
\end{proposition}
\begin{proof}
  \SufficiencySubProof By definition \( \beta + 0 = \beta \), hence we are not interested in the case \( \alpha = \beta \). That is, we will only consider the case \( \alpha < \beta \).

  We will first show uniqueness of \( \gamma \). Suppose that \( \alpha + \gamma_1 = \beta = \alpha + \gamma_2 \). From \eqref{thm:ordinal_addition_is_monotone/left} it follows that if either \( \gamma_1 < \gamma_2 \) or \( \gamma_1 > \gamma_2 \), we would have a strict inequality. Hence, it only remains for \( \gamma_1 = \gamma_2 \) to hold.

  We now use induction on \( \beta \) on prove the existence of \( \gamma \).
  \begin{itemize}
    \item The condition \( \alpha < \beta \) is vacuously false for the base case \( \beta = 0 \), hence by \eqref{eq:def:intuitionistic_propositional_deductive_systems/rules/efq} the statement vacuously holds.

    \item Suppose that \( \alpha < \beta \) and that there exists a unique \( \gamma \leq \beta \) such that \( \alpha + \gamma = \beta \). Then
    \begin{equation*}
      \alpha + \op{succ}(\gamma)
      \reloset {\eqref{eq:def:ordinal_arithmetic/addition}} =
      \op{succ}(\alpha + \gamma)
      =
      \op{succ}(\beta).
    \end{equation*}

    Since \( \alpha < \beta \) and \( \beta < \op{succ}(\beta) \), we have used the inductive hypothesis to conclude that
    \begin{equation*}
      \alpha < \op{succ}(\beta) \T{implies} \qexists {\underbrace{\delta}_{\mathclap{\op{succ}(\gamma)}}} \alpha + \delta = \op{succ}(\beta).
    \end{equation*}

    Furthermore, since \( \gamma \leq \beta \), then also \( \op{succ}(\gamma) \leq \op{succ}(\beta) \).

    \item Suppose that \( \lambda \) is a limit ordinal, \( \alpha < \lambda \) and for each \( \beta < \lambda \) there exists some \( \gamma_\beta \leq \beta \) such that \( \alpha + \gamma_\beta = \beta \). Define
    \begin{equation*}
      \gamma \coloneqq \sup\set{ \gamma_\beta \given \beta < \lambda }.
    \end{equation*}

    By \fullref{thm:union_of_set_of_ordinals} we have that \( \gamma \) is an ordinal and that \( \gamma_\beta \leq \gamma \) for every \( \beta < \lambda \). Thus,
    \begin{align*}
      \lambda
      &\reloset {\ref{thm:ordinal_is_set_of_smaller_ordinals}} =
      \sup\set{ \beta \given \beta < \lambda }
      \reloset {\T{ind.}} = \\ &=
      \sup\set{ \alpha + \gamma_\beta \given \beta < \lambda }
      \reloset {\eqref{eq:thm:ordinal_addition_is_monotone/right}} \leq \\ &\leq
      \sup\set[\Big]{ \alpha + \delta \given \delta < \sup\set{ \gamma_\beta \given \beta < \lambda } }
      = \\ &=
      \sup\set{ \alpha + \delta \given \delta < \gamma }
      \reloset {\eqref{eq:def:ordinal_arithmetic/addition}} = \\ &=
      \alpha + \gamma.
    \end{align*}

    Aiming at a contradiction, suppose that the strict inequality holds. That is, suppose that \( \lambda < \alpha + \gamma \). Then there exists some \( \delta_0 < \gamma \) such that \( \alpha + \delta_0 > \alpha + \gamma_\beta \) for any \( \beta < \lambda \). It follows from \fullref{thm:comparables_reflect_inequalities} that \( \gamma_\beta < \delta_0 \) for any \( \beta < \lambda \) and thus
    \begin{equation*}
      \underbrace{\sup\set{ \gamma_\beta \given \beta < \lambda }}_{\gamma} \leq \delta_0 < \gamma.
    \end{equation*}

    The obtained contradiction shows that such an ordinal \( \delta_0 \) cannot exist and hence \( \lambda = \alpha + \gamma \).

    Furthermore, since \( \gamma_\beta \leq \beta \) for each \( \beta < \lambda \), we have
    \begin{equation*}
      \gamma
      =
      \sup\set{ \gamma_\beta \given \beta < \lambda }
      \leq
      \sup\set{ \beta \given \beta < \lambda }
      =
      \lambda.
    \end{equation*}
  \end{itemize}

  \NecessitySubProof Suppose that \( \alpha \), \( \beta \) and \( \gamma \leq \beta \) are ordinals and that \( \alpha + \gamma = \beta \). Obviously \( \gamma = 0 \) implies that \( \alpha = \beta \). If \( \gamma > 0 \), then from \eqref{eq:thm:ordinal_addition_is_monotone/left} it follows that
  \begin{equation*}
     \beta = \alpha + \gamma > \alpha + 0 = 0.
  \end{equation*}
\end{proof}

\begin{proposition}\label{thm:ordinal_addition_algebraic_properties}
  Ordinal number addition is \hyperref[def:magma/associative]{associative} and \hyperref[def:magma/cancellative]{left cancellative}.

  As in \fullref{thm:ordinals_are_well_ordered}, we adapt the corresponding axioms due to \fullref{thm:burali_forti_paradox}. The more concrete result is:
  \begin{thmenum}
    \thmitem{thm:ordinal_addition_algebraic_properties/associative} For any three ordinals \( \alpha \), \( \beta \) and \( \gamma \) we have
    \begin{equation*}
      (\alpha + \beta) + \gamma = \alpha + (\beta + \gamma).
    \end{equation*}

    \thmitem{thm:ordinal_addition_algebraic_properties/left_cancellative} For any three ordinals \( \alpha \), \( \beta \) and \( \gamma \) such that \( \gamma + \alpha = \gamma + \beta \), we have \( \alpha = \beta \).
  \end{thmenum}

   See \fullref{ex:ordinal_addition} for counterexamples to \hyperref[def:magma/commutative]{commutativity}.

   Compare this with \fullref{thm:natural_number_addition_properties} and \fullref{thm:cardinal_addition_algebraic_properties}.
\end{proposition}
\begin{proof}
  \SubProofOf{thm:ordinal_addition_algebraic_properties/associative} We will use induction on \( \gamma \). \Fullref{thm:natural_number_addition_properties} already proves the base and successor cases.

  Fix some ordinals \( \alpha \) and \( \beta \). Let \( \lambda \) be a limit ordinal and suppose that
  \begin{equation*}
    (\alpha + \beta) + \gamma = \alpha + (\beta + \gamma)
  \end{equation*}
  holds for all \( \gamma < \lambda \). Then
  \begin{align*}
    (\alpha + \beta) + \lambda
    &\reloset {\eqref{eq:def:ordinal_arithmetic/addition}} =
    \sup\set{ (\alpha + \beta) + \gamma \given \gamma < \lambda }
    \reloset {\T{ind.}} = \\ &=
    \sup\set{ \alpha + (\beta + \gamma) \given \gamma < \lambda }
    \reloset {\eqref{eq:def:order_homomorphism/increasing/strict}} = \\ &=
    \sup\set{ \alpha + \delta \given \delta < \beta + \lambda }
    =
    \alpha + (\beta + \lambda).
  \end{align*}

  \SubProofOf{thm:ordinal_addition_algebraic_properties/left_cancellative} Follows from \fullref{thm:comparables_reflect_inequalities} and \eqref{eq:thm:ordinal_addition_is_monotone/left}.
\end{proof}

\begin{proposition}\label{thm:ordinal_addition_disjoin_union}
  For any two ordinals \( \alpha \) and \( \beta \), their \hyperref[def:ordinal_arithmetic/addition]{sum} satisfies
  \begin{equation*}
    \alpha + \beta = \ord(\alpha \amalg \beta, \prec),
  \end{equation*}
  where \( \prec \) is the \hyperref[def:lexicographic_order]{lexicographic order} on the \hyperref[def:disjoint_union]{disjoint union} \( \alpha \amalg \beta \).
\end{proposition}
\begin{proof}
  We will explicitly build an \hyperref[def:order_homomorphism/isomorphism]{order isomorphism} between \( (\alpha + \beta, \in) \) and \( (\alpha \amalg \beta, \prec) \). Define
  \begin{equation*}
    \begin{aligned}
      &f: (\alpha + \beta) \to (\alpha \amalg \beta) \\
      &f(\gamma) \coloneqq \begin{cases}
        (\gamma, 0), &\gamma < \alpha \\
        (\delta, 1), &\qexists \delta (\gamma = \alpha + \delta).
      \end{cases}
    \end{aligned}
  \end{equation*}

  From \fullref{thm:ordinal_ordering_via_addition} it follows that the existence of \( \delta \) such that \( \gamma = \alpha + \delta \) is equivalent to the condition \( \gamma \geq \alpha \). Since \( \gamma < \alpha + \beta \), we have \( \alpha + \delta < \alpha + \beta \) and from \fullref{thm:ordinal_addition_algebraic_properties/left_cancellative} we have \( \delta < \beta \). Therefore, \( f \) is a total function. Furthermore, it is single-valued because of the uniqueness of \( \delta \).

  We will first show that \( f \) is a strict order homomorphism. Let \( \gamma_1 < \gamma_2 \). We have the following possibilities:
  \begin{itemize}
    \item If \( \gamma_2 < \alpha \), then \( f(\gamma_1) = (\gamma_1, 0) < (\gamma_2, 0) = f(\gamma_2) \).
    \item If \( \gamma_1 \geq \alpha \), then \( f(\gamma_1) = (\gamma_1, 1) < (\gamma_2, 1) = f(\gamma_2) \).
    \item If \( \gamma_1 < \alpha \leq \gamma_2 \), then \( f(\gamma_1) = (\gamma_1, 0) < (\gamma_2, 1) = f(\gamma_2) \).
  \end{itemize}

  Therefore, \( f \) is a strict order homomorphism and from \fullref{thm:total_order_embedding_iff_strict} it follows that \( f \) is an order embedding. Due to \fullref{thm:totally_ordered_strict_isomorphisms}, in order to show that \( f \) is an order isomorphism it only remains to show that it is a surjective function.

  Let \( (\gamma, k) \in \alpha \amalg \beta \).
  \begin{itemize}
    \item If \( k = 0 \), then \( f(\gamma) = (\gamma, k) \) since \( \gamma \in \alpha \).
    \item If \( k = 1 \), then \( \gamma \in \beta \) and by \eqref{eq:thm:ordinal_addition_is_monotone/left} we have \( \alpha + \gamma < \alpha + \beta \), so \( \alpha + \gamma \) is within the domain of \( f \). Furthermore, as shown in \fullref{thm:ordinal_ordering_via_addition}, if \( \alpha + \delta = \alpha + \gamma \), then \( \delta = \gamma \), Thus, \( f(\alpha + \gamma) = (\gamma, 1) \) .
  \end{itemize}

  Therefore, \( f \) is an order isomorphism between \( (\alpha + \beta, \in) \) and \( (\alpha \amalg \beta, \prec) \) and hence
  \begin{equation*}
    \alpha + \beta = \ord(\alpha \amalg \beta, \prec).
  \end{equation*}
\end{proof}

\begin{example}\label{ex:ordinal_addition}
  The distinction between \( \alpha + \beta \) and \( \beta + \alpha \) is important. A simple example is provided by any limit ordinal \( \lambda \), in particular by \( \omega \). The examples are inconvenient to demonstrate with the recursive definition, however \fullref{thm:ordinal_addition_disjoin_union} eases us.

  In particular, \fullref{thm:ordinal_addition_disjoin_union} highlights that adding one ordinal to another, in an informal sense, \enquote{appending} a copy of the second to a copy the first.

  It is clear that
  \begin{equation*}
    0 + \lambda = \ord(0 \sqcap \lambda) = \ord(\lambda) = \lambda.
  \end{equation*}

  That is, we \enquote{append} \( \lambda \) to an empty well-ordered set only to obtain \( \lambda \) again.

  This operation seems different from \( 1 + \lambda \), which \enquote{appends} \( \lambda \) to a well-ordered singleton set. But this operation only \enquote{shifts} \( \lambda \) --- the function
  \begin{equation*}
    \begin{aligned}
      &f: \ord(1 \sqcap \lambda) \to \ord(0 \sqcap \lambda) \\
      &f(k, \gamma) \coloneqq \begin{cases}
        (0, 0),          &k = 0 \\
        (0, \gamma + 1), &k = 1.
      \end{cases}
    \end{aligned}
  \end{equation*}
  is an order isomorphism and thus
  \begin{equation*}
    1 + \lambda = \ord(1 \sqcap \lambda) = \ord(\lambda) = \lambda.
  \end{equation*}

  What \fullref{thm:comparables_reflect_inequalities} gives us is that
  \begin{equation*}
    \lambda \leq 1 + \lambda = \lambda.
  \end{equation*}

  This inequality is, of course, strict when dealing with finite ordinals exclusively, but for limit ordinals its results may be counterintuitive.

  What is more interesting is that, as a consequence of \fullref{thm:comparables_reflect_inequalities}, we have \( \lambda < \lambda + 1 \). This can be explained as follows. Instead of \enquote{appending} an infinite set to a finite one, we append a finite set to an infinite one. This way \( \lambda \) cannot \enquote{absorb} \( 1 \) like it does in \( 1 + \lambda \).

  As a consequence of this example, addition of ordinals is not commutative and also not right-cancellative.

  As discussed in our proof of \fullref{thm:cardinal_addition_algebraic_properties}, this is only a restriction of well-orders and not of the resulting sets themselves.
\end{example}

\begin{proposition}\label{thm:ordinal_multiplication_cartesian_product}
  For any two ordinals \( \alpha \) and \( \beta \), their \hyperref[def:ordinal_arithmetic/multiplication]{product} satisfies
  \begin{equation*}
    \alpha \cdot \beta = \ord(\alpha \times \beta, \prec),
  \end{equation*}
  where \( \prec \) is the \hyperref[def:lexicographic_order]{lexicographic order} on the \hyperref[def:cartesian_product]{Cartesian product} \( \alpha \times \beta \).
\end{proposition}
\begin{proof}
  We will build an order isomorphism between \( (\alpha \cdot \beta, \in) \) and \( (\alpha \times \beta, \prec) \) using recursion on \( \beta \).
  \begin{itemize}
    \item Both sets \( \alpha \cdot 0 \) and \( \alpha \cdot 0 \) are empty and the empty function is an order isomorphism.
    \item Suppose that \( f: \alpha \cdot \beta \to \alpha \times \beta \) is an order isomorphism. We construct the function
    \begin{equation*}
      \begin{aligned}
        &\widehat f: \alpha \cdot (\beta + 1) \to \alpha \times (\beta + 1) \\
        &\widehat f \coloneqq \begin{cases}
          f(\gamma),       &\gamma < \alpha \cdot \beta \\
          (\delta, \beta), &\qexists \delta (\gamma = \alpha \cdot \beta + \delta).
        \end{cases}
      \end{aligned}
    \end{equation*}

    In complete analogy with \fullref{thm:ordinal_ordering_via_addition} we can prove that \( \widehat f \) is an order isomorphism.

    \item Let \( \lambda \) be a limit ordinal and let \( f_\beta: \alpha \cdot \beta \to \alpha \times \beta \) be an order isomorphism for every \( \beta < \lambda \). Take their union
    \begin{equation*}
      f \coloneqq \bigcup\set{ f_\beta \given \beta < \lambda }.
    \end{equation*}

    The uniqueness of each \( f_\beta \) from \fullref{thm:well_ordered_order_type_existence} shows that \( f_{\beta_1} \subseteq f_{\beta_2} \) for each pair \( \beta_1 < \beta_2 \). Therefore, the union \( f \) is a single-valued partial function. It is also total because every ordinal \( \gamma < \alpha \cdot \lambda \) is contains in the image of the function \( f_{\gamma + 1} \).

    The function \( f \) is an order embedding by construction. It is also surjective because, if \( (\delta, \beta) \in \alpha \times \lambda \), then from the successor step we can conclude that \( f(\alpha \cdot \beta + \delta) = (\delta, \beta) \).

    Therefore, \( f \) is an order isomorphism.
  \end{itemize}
\end{proof}

\begin{proposition}\label{thm:ordinal_multiplication_algebraic_properties}
  Similarly to \fullref{thm:ordinal_addition_algebraic_properties} for ordinal number addition, multiplication is also \hyperref[def:magma/associative]{associative} and \hyperref[def:magma/cancellative]{left cancellative}:
  \begin{thmenum}
    \thmitem{thm:ordinal_multiplication_algebraic_properties/associative} For any three ordinals \( \alpha \), \( \beta \) and \( \gamma \) we have
    \begin{equation*}
      (\alpha \cdot \beta) \cdot \gamma = \alpha \cdot (\beta \cdot \gamma).
    \end{equation*}

    \thmitem{thm:ordinal_multiplication_algebraic_properties/left_cancellative} For any three ordinals \( \alpha \), \( \beta \) and \( \gamma \) such that \( \gamma \cdot \alpha = \gamma \cdot \beta \), we have \( \alpha = \beta \).
  \end{thmenum}

  Compare this with \fullref{thm:natural_number_multiplication_properties} and \fullref{thm:cardinal_multiplication_algebraic_properties}.
\end{proposition}
\begin{proof}
  \SubProofOf{thm:ordinal_multiplication_algebraic_properties/associative} Associativity follows easily from the obvious isomorphisms between \( (\alpha \times \beta) \times \gamma \) and \( \alpha \times (\beta \times \gamma) \).

  \SubProofOf{thm:ordinal_multiplication_algebraic_properties/left_cancellative} Now suppose that \( \gamma \cdot \alpha = \gamma \cdot \beta \). Let \( f \) the unique order isomorphism between \( \gamma \times \alpha \) and \( \gamma \times \beta \).

  Suppose that \( \alpha \neq \beta \). Without loss of generality, suppose that \( \beta \subsetneq \alpha \). Then \( f\restr_{\gamma \times \alpha} \) is the identity mapping and hence any set from \( \gamma \times (\beta \setminus \alpha) \) would make \( f \) not injective.

  The obtained contradiction shows that \( \alpha = \beta \).
\end{proof}

\begin{example}\label{ex:countable_limit_ordinals}
  We already know that \( \omega \) is a limit ordinal. It is clear from \fullref{ex:ordinal_addition} that \( \omega + n \) is a successor ordinal for every natural number \( n \).

  What about \( \omega + \omega \)? This corresponds to \enquote{placing} two copies of the natural numbers one after another.

  Suppose that \( \omega + \omega = \omega \cdot 2 \) is the successor of \( \alpha \). Then \( \alpha < \omega + \omega \) and we can show by induction on the natural numbers that \( \alpha + n < \omega + \omega \). But \( \alpha + 1 = \omega \) by assumption, which contradicts trichotomy of ordinals.

  Therefore, \( \omega + \omega \) is a limit ordinal. Furthermore, \( \omega + \omega \) is the second smallest limit ordinal since only ordinals of the form \( \omega + n \) for nonzero finite \( n \) satisfy \( \omega < \omega + n < \omega + \omega \).

  Both ordinals \( \omega \) and \( \omega + \omega \) are countable by \fullref{thm:omega_equinumerous_with_omega_squared}.

  Another limit ordinal is \( \omega \cdot \omega = \omega^2 \). It is also countable by \fullref{thm:countable_product_of_countable_sets}. Actually \( \omega^n \) for any natural number \( n \) is countable by the same theorem.

  Therefore, any \enquote{\hyperref[def:polynomial_algebra]{polynomial}} of the form
  \begin{equation*}
    \alpha_n \omega^n + \alpha_{n-1} \omega^{n-1} + \cdots + \alpha_1 \omega + \alpha_0
  \end{equation*}
  with countable coefficients is also countable.
\end{example}

\begin{proposition}\label{thm:square_of_infinite_ordinal}\mcite[30]{Jech2003}
  If \( \alpha \) is an infinite ordinal, then \( \alpha = \alpha^2 \).
\end{proposition}
\begin{proof}
  We will show that \( \alpha \) is order isomorphic to its \hyperref[def:lexicographic_order]{lexicographically ordered} square
  \begin{equation*}
    S \coloneqq \alpha \times \alpha.
  \end{equation*}

  Aiming at a contradiction, suppose that \( \alpha \) is the smallest ordinal for which this does not hold. \Fullref{thm:omega_equinumerous_with_omega_squared} implies that \( \alpha \geq \omega \).

  Since \( \alpha < \ord(\alpha \times \alpha) \), there \fullref{thm:initial_segment_of_ordinal} implies the existence of some segment \( S_{<(\beta_1, \beta_2)} \) such that
  \begin{equation*}
    \alpha = \ord(S_{<(\beta_1, \beta_2)}),
  \end{equation*}
  where
  \begin{equation*}
    S_{<(\beta_1, \beta_2)} = \set[\Big]{ (\gamma_1, \gamma_2) \in \alpha \times \alpha \given* \gamma_1 < \beta_1 \T{or} (\gamma_1 = \beta_1 \T{and} \gamma_2 < \beta_2) }
  \end{equation*}
  is the corresponding \hyperref[def:order_interval/ray]{initial segment}.

  Let \( \beta \coloneqq \max\set{ \beta_1, \beta_2 } \). Then
  \begin{equation*}
    S_{<(\beta_1, \beta_2)} \subsetneq \beta \times \beta.
  \end{equation*}

  Since \( \beta < \alpha \), the lemma holds for \( \beta \), and thus
  \begin{equation*}
    \alpha
    =
    \ord(S_{<(\beta_1, \beta_2)})
    \leq
    \ord(\beta \times \beta)
    =
    \beta
    <
    \alpha.
  \end{equation*}

  The obtained contradiction shows that \( \alpha = \ord(\alpha \times \alpha) \) for every infinite ordinal \( \alpha \).
\end{proof}

\begin{definition}\label{def:cardinal_arithmetic}\mcite[28]{Jech2003}
  We will define arithmetic operations for them. Unlike in \fullref{def:ordinal_arithmetic}, we will directly define the operations as \hyperref[thm:cardinality_existence]{cardinal numbers} of some sets rather than via some form of recursion.

  Fix two ordinals \( \kappa \) and \( \mu \).
  \begin{thmenum}
    \thmitem{def:cardinal_arithmetic/addition} Based on \fullref{thm:ordinal_addition_disjoin_union}, we define their \term{sum} as
    \begin{equation*}
      \kappa + \mu \coloneqq \card(\kappa \amalg \mu),
    \end{equation*}
    where \( \kappa \amalg \mu \) is their \hyperref[def:disjoint_union]{disjoint union}.

    \thmitem{def:cardinal_arithmetic/multiplication} Based on \fullref{thm:ordinal_multiplication_cartesian_product}, we define their \term{product} as
    \begin{equation*}
      \kappa \cdot \mu \coloneqq \card(\kappa \times \mu).
    \end{equation*}

    \thmitem{def:cardinal_arithmetic/exponentiation} We define \term{exponentiation} as
    \begin{equation*}
      \kappa^\mu \coloneqq \card(\fun(\kappa, \mu)).
    \end{equation*}
  \end{thmenum}
\end{definition}

\begin{proposition}\label{thm:cardinal_addition_algebraic_properties}
  Cardinal number addition is \hyperref[def:magma/associative]{associative}, \hyperref[def:magma/associative]{commutative} and \hyperref[def:magma/cancellative]{cancellative}.

  Compare this with \fullref{thm:natural_number_addition_properties} and \fullref{thm:ordinal_addition_algebraic_properties}.
\end{proposition}
\begin{proof}
  Associativity and left cancellation is inherited from the ordinals. Commutativity and right cancellation hold because we are considering arbitrary bijective functions rather than the more restrictive order isomorphisms. Indeed, \( \kappa \amalg \mu \) and \( \mu \amalg \kappa \) may have different order types as demonstrated in \fullref{ex:ordinal_addition}, however there is an obvious bijective function between them.
\end{proof}

\begin{proposition}\label{thm:cardinal_multiplication_algebraic_properties}
  Cardinal number multiplication is \hyperref[def:magma/associative]{associative}, \hyperref[def:magma/associative]{commutative} and \hyperref[def:magma/cancellative]{cancellative}.

  Compare this with \fullref{thm:natural_number_multiplication_properties} and \fullref{thm:ordinal_addition_algebraic_properties}.
\end{proposition}
\begin{proof}
  The result follows from the same considerations as in \fullref{thm:cardinal_addition_algebraic_properties}.
\end{proof}

\begin{proposition}\label{thm:double_and_square_of_cardinal}
  For every cardinal \( \kappa \) we have
  \begin{thmenum}
    \thmitem{thm:double_and_square_of_cardinal/double} \( \kappa + \kappa = 2\kappa \).
    \thmitem{thm:double_and_square_of_cardinal/square} \( \kappa \cdot \kappa = \kappa^2 \).
  \end{thmenum}
\end{proposition}
\begin{proof}
  \SubProofOf{thm:double_and_square_of_cardinal/double} Obviously \( \kappa \amalg \kappa = 2 \times \kappa \).
  \SubProofOf{thm:double_and_square_of_cardinal/square} The function
  \begin{equation*}
    \begin{aligned}
      &T: \fun(2, \kappa) \to \kappa \times \kappa \\
      &T(f) \coloneqq (f(0), f(1))
    \end{aligned}
  \end{equation*}
  is clearly injective. It is also surjective because for any ordered pair \( (\gamma, \delta) \in \kappa \times \kappa \) we can define the function
  \begin{equation*}
    \begin{aligned}
      &f: 2 \to \kappa \\
      &f(k) \coloneqq \begin{cases}
        \gamma, k = 0 \\
        \delta, k = 1
      \end{cases}
    \end{aligned}
  \end{equation*}

  Then \( T(f) = (\gamma, \delta) \).
\end{proof}

\begin{proposition}\label{thm:simplified_cardinal_arithmetic}
  Unlike \hyperref[def:ordinal_arithmetic]{ordinal arithmetic} with its intricacies like \fullref{ex:ordinal_addition}, \hyperref[def:cardinal_arithmetic]{cardinal arithmetic} has a simpler behavior:
  \begin{thmenum}
    \thmitem{thm:simplified_cardinal_arithmetic/finite} If \( \kappa \) and \( \mu \) are finite cardinals, then \( \kappa + \mu \) and \( \kappa \cdot \mu \) are the familiar operations on \hyperref[def:natural_numbers]{natural numbers}.

    \thmitem{thm:simplified_cardinal_arithmetic/infinite} If either \( \kappa \) or \( \mu \) is infinite, then
    \begin{equation*}
      \kappa + \mu = \max\set{ \kappa, \mu }.
    \end{equation*}

    If, additionally, both are nonzero, then
    \begin{equation*}
      \kappa \cdot \mu = \max\set{ \kappa, \mu }.
    \end{equation*}
  \end{thmenum}
\end{proposition}
\begin{proof}
  \SubProofOf{thm:simplified_cardinal_arithmetic/finite} The addition of ordinals defined in \fullref{def:ordinal_arithmetic/addition} is an extension of addition of natural numbers, hence the two are equivalent for finite ordinals. The equivalence with cardinal addition defined in \fullref{def:cardinal_arithmetic/addition} comes from \fullref{thm:ordinal_addition_disjoin_union} and the fact that every finite ordinal is a cardinal as demonstrated in \fullref{thm:natural_numbers_are_cardinals}.

  Analogously, equivalence of cardinal and ordinal multiplication follows from \fullref{thm:ordinal_multiplication_cartesian_product}.

  \SubProofOf{thm:simplified_cardinal_arithmetic/infinite} Suppose that either \( \kappa \) or \( \mu \) is infinite and let \( \nu \coloneqq \max\set{ \kappa, \mu } \). The cases where either of them is zero are trivial, hence suppose that both are nonzero.

  We have
  \begin{equation*}
    \kappa \amalg \mu \subseteq \nu \amalg \nu = 2 \amalg \nu \subseteq \nu \times \nu,
  \end{equation*}
  hence \( \kappa + \mu \leq \nu \cdot \nu = \nu^2 \).

  Furthermore there exists an obvious injective function from \( \nu \) to \( \kappa \amalg \mu \) (which is different depending on whether \( \nu = \kappa \) or \( \nu = \mu \)). Therefore,
  \begin{equation*}
    \nu \leq \kappa + \mu \leq \nu^2.
  \end{equation*}

  For multiplication we have \( \kappa \times \mu \subseteq \nu \times \nu \), hence
  \begin{equation*}
    \nu \leq \kappa \cdot \mu \leq \nu^2.
  \end{equation*}

  The rest follows from \fullref{thm:square_of_infinite_ordinal}.
\end{proof}

\begin{corollary}\label{thm:aleph_zero_is_strong_limit}
  The first infinite cardinal \( \aleph_0 \) is a \hyperref[def:successor_and_limit_cardinal/strong_limit]{strong limit cardinal}.

  See also \fullref{thm:aleph_zero_is_regular}
\end{corollary}
\begin{proof}
  \Fullref{thm:simplified_cardinal_arithmetic/finite} states that cardinal exponentiation extends natural number exponentiation. Hence, we can conclude that \( 2^n < \aleph_0 \) for any \( n < \aleph_0 \) since the former are finite and the latter is not.
\end{proof}

\begin{lemma}\label{thm:power_set_via_subsets}
  Fix a set \( A \). The power set \( \pow(A) \) is equinumerous with the set of \hyperref[def:boolean_operator]{Boolean-valued functions} \( \pow(A, \set{ T, F }) \).

  More precisely, then the operator
  \begin{equation*}
    \begin{aligned}
      &T: \fun(A, \set{ T, F }) \to \pow(A) \\
      &T(f) \coloneqq \set{ f(x) = T \given x \in A }.
    \end{aligned}
  \end{equation*}
  is bijective.
\end{lemma}
\begin{proof}
  Injectivity is clear. To see surjectivity, fix some subset \( B \subset A \) and define
  \begin{equation*}
    \begin{aligned}
      &f: A \to \set{ T, F } \\
      &f(x) \coloneqq \begin{cases}
        T, &x \in B \\
        F, \T{otherwise}.
      \end{cases}
    \end{aligned}
  \end{equation*}

  Clearly \( f \in \fun(A, \set{ T, F }) \) and \( T(f) = B \).
\end{proof}

\begin{proposition}\label{thm:cardinal_exponentiation_power_set}
  For every set \( A \) we have
  \begin{equation*}
    \card(\pow(A)) = 2^{\card(A)}.
  \end{equation*}
\end{proposition}
\begin{proof}
  The proof is similar to \fullref{thm:power_set_via_subsets}, but much more convoluted.

  Let \( \varphi: A \to \card(A) \) be \( \psi: \pow(A) \to \card(\pow(A)) \) bijective functions. Note that
  \begin{equation*}
    2^{\card(A)} = \card(\fun(\card(A), \set{ 0, 1 }))
  \end{equation*}
  by definition. Let \( \theta: \fun(\card(A), \set{ 0, 1 }) \to 2^{\card(A)} \) be a bijective function.

  Define the operator
  \begin{equation*}
    \begin{aligned}
      &T: 2^{\card(A)} \to \card(\pow(A)) \\
      &T(p) \coloneqq \psi\parens{ \set{ \varphi^{-1}(\gamma) \given \gamma \in \card(A) \T{and} \theta^{-1}(p)(\gamma) = 1 } }.
    \end{aligned}
  \end{equation*}

  This operator is bijective since for any \( \delta \in \card(\pow(A)) \) we can define
  \begin{equation*}
    \begin{aligned}
      &f: \card(A) \to \set{ 0, 1 } \\
      &f(\gamma) \coloneqq \begin{cases}
        1, &\varphi^{-1}(\gamma) \in \psi^{-1}(\delta) \\
        0, &\T{otherwise.}
      \end{cases}
    \end{aligned}
  \end{equation*}
  so that \( T(\theta(f)) = \delta \).

  Therefore, \( \card(A) \) and \( \card(\pow(A)) \) are equinumerous.
\end{proof}

\begin{corollary}\label{thm:cardinal_exponentiation_comparison}
  We have \( \card A < \card B \) if and only if \( 2^{\card A} < 2^{\card B} \).
\end{corollary}
\begin{proof}
  \SufficiencySubProof Suppose that \( \card A < \card B \).

  Let \( f: A \to B \) be an injective function. It is not surjective by assumption. Then \( f(A) \subsetneq B \).

  Clearly \( \pow(f(A)) \subsetneq \pow(B) \). \Fullref{thm:cardinal_exponentiation_power_set} then implies that \( 2^{\card A} < 2^{\card B} \).

  \NecessitySubProof Suppose that \( 2^{\card A} < 2^{\card B} \).
  \begin{itemize}
    \item If \( \card A = \card B \), then \( 2^{\card A} = 2^{\card B} \), which is a contradiction.
    \item If \( \card A > \card B \), then we have shown that \( 2^{\card A} > 2^{\card B} \), which is again a contradiction.
    \item It remains for \( \card A \) to be strictly less than \( \card B \).
  \end{itemize}
\end{proof}

\begin{proposition}\label{thm:strong_limit_cardinal_is_weak_limit}
  If \( \mu \) is the successor cardinal of \( \kappa \), then \( \mu \leq 2^\kappa \).

  In particular, every \hyperref[def:successor_and_limit_cardinal/strong_limit]{strong limit cardinal} is a \hyperref[def:successor_and_limit_cardinal/strong_limit]{weak limit cardinal}.

  Furthermore, if \( \kappa \) is infinite, then \fullref{hyp:generalized_continuum_hypothesis} implies that \( \mu = 2^\kappa \).
\end{proposition}
\begin{proof}
  It is clear from \fullref{thm:cantor_power_set_theorem} and \fullref{thm:cardinal_exponentiation_power_set} that \( \kappa < 2^\kappa \). By definition, \( \mu \) is the smallest cardinal such that \( \kappa < \mu \). Therefore, \( \mu \leq 2^\kappa \).
\end{proof}

\begin{conjecture}[Generalized continuum hypothesis]\label{hyp:generalized_continuum_hypothesis}
  For every ordinal \( \alpha \) we have
  \begin{equation*}
    \aleph_{\alpha + 1} = 2^{\aleph_\alpha},
  \end{equation*}

  For the definition and the properties of the \( \aleph \) hierarchy, see \fullref{def:aleph_hierarchy}. For the related \( \beth \) hierarchy, see \fullref{def:beth_hierarchy}.

  This is a vast generalization of \fullref{hyp:continuum_hypothesis} from the case \( \alpha = 0 \) to arbitrary ordinals.
\end{conjecture}

\begin{corollary}\label{thm:limit_cardinals_and_gch}
  \Fullref{hyp:generalized_continuum_hypothesis} implies that every \hyperref[def:successor_and_limit_cardinal/weak_limit]{weak limit cardinal} is a \hyperref[def:successor_and_limit_cardinal/strong_limit]{strong limit cardinal}.

  The converse holds in \logic{ZFC} as shown in \fullref{thm:strong_limit_cardinal_is_weak_limit}.
\end{corollary}
\begin{proof}
  Follows from \fullref{thm:infinite_cardinal_is_aleph} and \fullref{hyp:generalized_continuum_hypothesis}.
\end{proof}

\begin{definition}\label{def:beth_hierarchy}\mcite[55]{Jech2003}
  Similarly to \fullref{def:aleph_hierarchy}, we use transfinite recursion to define, for each ordinal \( \alpha \), the cardinal
  \begin{equation}\label{eq:def:beth_hierarchy}
    \beth_\alpha \coloneqq \begin{cases}
      \omega,                                       &\alpha = 0 \\
      2^\beta,                                      &\alpha = \op{succ}(\beta) \\
      \sup\set{ \beth_\beta \given \beta < \alpha } &\alpha \T{is a limit ordinal}.
    \end{cases}
  \end{equation}

  Unlike \fullref{def:aleph_hierarchy}, it is able to explicitly build the successor of any member of the hierarchy. \Fullref{hyp:generalized_continuum_hypothesis} states that \( \aleph_\alpha = \beta_\alpha \) for every ordinal \( \alpha \), however in general it is only provable that \( \aleph_\alpha \leq \beta_\alpha \) --- see \fullref{thm:strong_limit_cardinal_is_weak_limit}.
\end{definition}

  \subsection{Von Neumann's cumulative hierarchy}\label{subsec:von_neumanns_cumulative_hierarchy}

We will now investigate how we can use set theory itself to build models of set theory. \Fullref{thm:cumulative_hierarchy_model_of_zfc} contains the important results.

\begin{definition}\label{def:cumulative_hierarchy}\mcite[63]{Jech2003}
  For every ordinal \( \alpha \) we use \fullref{rem:unbounded_transfinite_recursion} to define
  \begin{equation}\label{eq:def:cumulative_hierarchy}
    V_\alpha \coloneqq \begin{cases}
      \varnothing,                                  &\alpha = 0 \\
      \pow(V_\beta),                                &\alpha = \beta + 1 \\
      \bigcup\set{ V_\beta \given \beta < \alpha }, &\alpha \T{is a limit ordinal.}
    \end{cases}
  \end{equation}

  Each \( V_\alpha \) is called a \term{stage} and the index \( \alpha \) of a stage is called its \term{rank}. The entire proper class of stages is called the \term{cumulative hierarchy}.

  If some set \( A \) is a subset of \( V_\alpha \), but not of \( V_\beta \) for any \( \beta < \alpha \), we say that \( \alpha \) is the \term{rank} of the set \( A \) and denote it by \( \rank(A) \). We will see in \fullref{thm:axiom_of_regularity} that every set has a rank.
\end{definition}

\begin{proposition}\label{thm:def:cumulative_hierarchy}
  Without relying on the \hyperref[def:zfc/foundation]{axiom of foundation} and by assuming that ordinals are well-founded by definition, we can prove the following basic properties for \hyperref[def:cumulative_hierarchy]{Von Neumann's cumulative hierarchy}:
  \begin{thmenum}
    \thmitem{thm:def:cumulative_hierarchy/transitive} Each stage \( V_\alpha \) is a transitive set.
    \thmitem{thm:def:cumulative_hierarchy/membership} For any two ordinals \( \alpha < \beta \) we have \( V_\alpha \in V_\beta \).
    \thmitem{thm:def:cumulative_hierarchy/rank_inequality} If \( A \in B \) and both sets have ranks, then \( \rank(A) \in \rank(B) \).
    \thmitem{thm:def:cumulative_hierarchy/well_founded} Each stage \( V_\alpha \) is well-founded by set membership.
    \thmitem{thm:def:cumulative_hierarchy/subsets} We have \( \alpha < \beta \) if and only if \( V_\alpha \subsetneq V_\beta \).
    \thmitem{thm:def:cumulative_hierarchy/ordinals} For every ordinal \( \alpha \) we have \( \rank(\alpha) = \alpha \).
    \thmitem{thm:def:cumulative_hierarchy/stage_rank} For every stage \( V_\alpha \) we have \( \rank(V_\alpha) = \alpha \).
  \end{thmenum}
\end{proposition}
\begin{proof}
  \SubProofOf{thm:def:cumulative_hierarchy/transitive} The statement is vacuous for \( \alpha = 0 \). Suppose that \( \alpha > 0 \), let \( A \in V_\alpha \) and \( B \in A \). We will show that \( B \in V_\alpha \).
  \begin{itemize}
    \item Suppose that \( \alpha = \beta + 1 \) and that \( V_\beta \) is a transitive set. Then \( A \in V_\alpha \) implies that \( A \subseteq V_\beta \). Thus, \( B \in V_\beta \) and, since \( V_\beta \) is a transitive set, \( B \subseteq V_\beta \).

    Therefore, \( B \in V_\alpha = \pow(V_\beta) \).

    \item Suppose that \( \alpha \) is a limit ordinal and that \( V_\beta \) are transitive sets for every \( \beta < \alpha \). Then \( A \in V_\alpha \) implies that \( A \) belongs to \( V_{\beta_0} \) for some \( \beta_0 < \alpha \). The inductive hypothesis implies that \( A \subseteq V_{\beta_0} \). Therefore, \( A \subseteq V_\alpha \).
  \end{itemize}

  \SubProofOf{thm:def:cumulative_hierarchy/membership} Let \( \alpha < \beta \) be some ordinals. We will show that \( V_\alpha \in V_\beta \) using induction on \( \beta \).
  \begin{itemize}
    \item Suppose that \( \beta \) is a successor ordinal, i.e. \( \beta = \mu + 1 \) for some \( \mu \), and suppose that for every \( \alpha < \mu \) we have \( V_\alpha \in V_\mu \). Clearly \( V_\mu \in V_\beta \) because \( V_\mu \) is a subset of itself.

    If \( \alpha < \mu \), then \( V_\alpha \in V_\mu \) by the inductive hypothesis and, since \( V_\beta \) is a transitive set by \fullref{thm:def:cumulative_hierarchy/transitive}, \( V_\alpha \in V_\beta \).

    \item Suppose that \( \beta \) is a limit ordinal. For some fixed \( \alpha_0 \in \beta \) we have \( V_{\alpha_0} \in V_{\alpha_0 + 1} \) by what we have already proved. We also have
    \begin{equation*}
      V_\beta
      =
      \bigcup_{\alpha < \beta} V_\alpha,
    \end{equation*}
    hence \( V_{\alpha_0} \in V_\beta \).
  \end{itemize}

  \SubProofOf{thm:def:cumulative_hierarchy/rank_inequality} Let \( A \in B \) be arbitrary sets for which ranks are defined. Trichotomy holds for ordinals, so we have to show that \( \rank(A) \geq \rank(B) \) leads to a contradiction. Denote the ranks by \( \alpha \) and \( \beta \) for brevity.

  We have \( A \subseteq V_\alpha \) by definition and \( A \subseteq V_\beta \) since \( V_\alpha \in V_\beta \) and \( V_\beta \) is transitive. If we suppose that \( \rank(B) < \rank(A) \), this would mean that \( A \) belongs to a stage below \( V_\alpha \), which contradicts the minimality of \( \alpha = \rank(A) \).

  Now suppose that \( \beta = \rank(B) = \rank(A) = \alpha \).
  \begin{itemize}
    \item If \( \beta = 0 \), then \( A \in B \) is impossible.

    \item If \( \beta \) is a successor ordinal of \( \alpha \), then \( V_\beta = \pow(V_\alpha) \). Since \( B \subseteq V_\beta \) is a set of subsets of \( V_\alpha \), \( A \in B \) is a subset of \( V_\alpha \). Thus, we have \( \rank(A) \leq \alpha < \rank(A) \), which contradicts the well-foundedness of the ordinal ordering.

    \item If \( \beta \) is a limit ordinal, then \( V_\beta = \bigcup\set{ V_\alpha \given \alpha < \beta } \). As a member of \( B \), the set \( A \) belongs to some lower stage \( V_{\alpha_0} \), which again leads to \( \rank(A) < \rank(A) \).
  \end{itemize}

  Thus, it remains for \( \rank(A) < \rank(B) \).

  \SubProofOf{thm:def:cumulative_hierarchy/well_founded} We will use induction on \( \alpha \). Suppose that \( V_\beta \) is well-founded for every \( \beta < \alpha \). Aiming at a contradiction, suppose that there exists an infinitely descending sequence \( \seq{ x_k }_{k=1}^\infty \subseteq V_\alpha \). Then the sequence \( \seq{ \rank(x_k) }_{k=0}^\infty \) of ranks is an infinitely descending set of ordinals. The \hyperref[def:transitive_closure_of_a_set]{transitive closure} of the underlying set is then an ordinal by \fullref{thm:transitive_set_of_transitive_sets}. But this contradicts the well-foundedness of ordinals.

  Therefore, \( V_\alpha \) must be well-founded.

  \SubProofOf{thm:def:cumulative_hierarchy/subsets} If \( \alpha < \beta \), then from \fullref{thm:def:cumulative_hierarchy/membership} and \fullref{thm:def:cumulative_hierarchy/transitive} it follows that \( V_\alpha \subseteq V_\beta \). We will show that \( V_\alpha \neq V_\beta \). Fix some \( \mu < \beta \) and suppose that \( V_\alpha \subsetneq V_\mu \) holds for all \( \alpha < \mu \). If \( V_\alpha = V_\beta \), this would mean that \( V_\alpha \subsetneq V_\mu \subseteq V_\beta = V_\alpha \), which is a contradiction. Therefore, \( V_\alpha \subsetneq V_\beta \).

  Conversely, suppose that \( V_\alpha \subsetneq V_\beta \). Since trichotomy holds for ordinals and since \( V_\alpha \neq V_\beta \), it is sufficient to show that \( \alpha > \beta \) leads to a contradiction. If \( \alpha > \beta \), from \fullref{thm:def:cumulative_hierarchy/membership} it follows that \( V_\beta \in V_\alpha \), which implies that \( V_\beta \subsetneq V_\beta \). The obtained contradiction shows that \( \alpha < \beta \).

  \SubProofOf{thm:def:cumulative_hierarchy/ordinals} We will use transfinite induction to show that \( \rank(\alpha) = \alpha \) for every ordinal.
  \begin{itemize}
    \item The case \( \alpha = 0 \) is trivial because \( \alpha = \varnothing \subseteq \varnothing = V_0 \).

    \item Suppose that \( \alpha = \beta + 1 \) is a successor ordinal and \( \rank(\beta) = \beta \). Clearly \( \beta \in \pow(V_\beta) = V_\alpha \) and \( \set{ \beta } \in V_\alpha \). Since \( V_\alpha \) is a transitive set, we also have \( \beta \subseteq V_\alpha \). Thus,
    \begin{equation*}
      \alpha = \beta + 1 = \beta \cup \set{ \beta } \subseteq V_\alpha.
    \end{equation*}

    \item Suppose that \( \alpha \) is a limit ordinal and that for \( \rank(\beta) = \beta \) for all \( \beta < \alpha \). Since \( \beta \in V_{\beta + 1} \) for any \( \beta < \alpha \), we have
    \begin{equation*}
      \alpha
      =
      \set{ \beta \T{is an ordinal} \given \beta < \alpha }
      \subseteq
      \bigcup\set{ V_{\beta + 1} \given \beta < \alpha }
      =
      V_\alpha.
    \end{equation*}
  \end{itemize}

  \SubProofOf{thm:def:cumulative_hierarchy/stage_rank} Clearly \( V_\alpha \) is a subset of itself, hence \( \rank(V_\alpha) \leq \alpha \). In particular, the rank of \( V_\alpha \) exists.

  We will use induction on \( \alpha \) to show that \( \rank(V_\alpha) \geq \alpha \).
  \begin{itemize}
    \item For \( \alpha = 0 \) this is obvious.
    \item If \( \rank(V_\alpha) = \alpha \), then
    \begin{equation*}
      \rank(V_{\alpha + 1})
      =
      \rank(\pow(V_\alpha))
      \reloset {\ref{thm:def:cumulative_hierarchy/rank_inequality}} >
      \rank(V_\alpha)
      =
      \alpha.
    \end{equation*}

    \item Let \( \lambda \) be a limit ordinal and suppose that \( \rank(V_\alpha) = \alpha \) for any \( \alpha < \lambda \). Then
    \begin{equation*}
      \rank(V_\lambda)
      \reloset {\ref{thm:def:cumulative_hierarchy/rank_inequality}} \geq
      \sup\set{ \rank(V_\alpha) \given \alpha < \lambda }
      =
      \sup\set{ \alpha \given \alpha < \lambda }
      =
      \lambda.
    \end{equation*}
  \end{itemize}

  Therefore, \( \rank(V_\alpha) = \alpha \).
\end{proof}

\begin{theorem}[Axiom of regularity]\label{thm:axiom_of_regularity}\mcite[lemma 6.3]{Jech2003}
  Every set belongs to a stage in \hyperref[def:cumulative_hierarchy]{von Neumann's cumulative hierarchy}.

  This statement is called the \term{axiom of regularity} and in the presence of the other axioms of \logic{ZF}, it is equivalent to the \hyperref[def:zfc/foundation]{axiom of foundation}. It is much more difficult to state in the language of set theory, however.
\end{theorem}
\begin{proof}
  \ImplicationSubProof[def:zfc/foundation]{axiom of foundation}[thm:axiom_of_regularity]{axiom of regularity} Let \( A \) be a set. By \fullref{thm:transitive_closure_of_a_set}, its \hyperref[def:transitive_closure_of_a_set]{transitive closure} \( \cl^T(A) \) is a transitive set. Define
  \begin{equation*}
    D \coloneqq \set{ B \in \cl^T(A) \given B \T{does not belong to the cumulative hierarchy} }.
  \end{equation*}

  We will show that \( D \) is empty. Assume the contrary. Then by the \hyperref[def:zfc/foundation]{axiom of foundation} there exists \( B_0 \in D \) such that \( B_0 \cap D = \varnothing \). Since \( \cl^T(A) \) is a transitive set, \( B_0 \subseteq \cl^T(A) \). Thus, \( B_0 \) consists of members \( x \) of \( \cl^T(A) \) that themselves have ranks, i.e. a minimal ordinal \( \beta \) such that \( x \subseteq V_\beta \). It follows from the \hyperref[def:zfc/replacement]{axiom schema of replacement} that these ordinals form a set. Denote this set by \( C \).

  If \( x \in B_0 \), then \( x \subseteq V_\beta \) for some \( \beta \in C \) and \( x \in V_{\beta + 1} \). Thus,
  \begin{equation*}
    B_0 \subseteq \bigcup\set{ V_{\beta + 1} \given \beta \in C }.
  \end{equation*}

  Denote the union on the right by \( \alpha \). From \fullref{thm:union_of_set_of_ordinals} it follows that \( \alpha \) is an ordinal strictly larger than the ordinals in \( C \). By \fullref{thm:def:cumulative_hierarchy/membership} we have that \( V_{\beta + 1} \in V_\alpha \) for every \( \beta \in C \). Thus,
  \begin{equation*}
    B_0 \subseteq \bigcup\set{ V_{\beta + 1} \given \beta \in C } \subseteq V_\alpha.
  \end{equation*}

  This contradicts our assumption that \( B_0 \) does not belong to the cumulative hierarchy. Therefore, \( D = \varnothing \) and every member of \( \cl^T(A) \) also belongs to the cumulative hierarchy. In particular, every member of \( A \) belongs to the cumulative hierarchy.

  Define
  \begin{equation*}
    \mu \coloneqq \bigcup\set{ \rank(B) + 1 \given B \in A }.
  \end{equation*}

  Since \( B \) is a member of the stage with rank \( \rank(B) + 1 \) for every \( B \in A \), with the same reasoning as above it follows that \( A \subseteq V_\mu \).

  \ImplicationSubProof[thm:axiom_of_regularity]{axiom of regularity}[def:zfc/foundation]{axiom of foundation} Let \( A \) be any nonempty set. We will show that there exists a subset of \( A \) that is disjoint from \( A \).

  The \hyperref[thm:axiom_of_regularity]{axiom of regularity} ensures that \( A \) belongs to the von Neumann cumulative hierarchy. Let \( B \in A \) be a set with minimal rank.

  Suppose that \( B \cap A \) is not empty. Then there exists some set \( C \in A \setminus B \). From \fullref{thm:def:cumulative_hierarchy/rank_inequality} it follows that \( \rank(C) < \rank(B) < \rank(A) \). But \( C \) belongs to \( A \) and has a rank strictly smaller than \( \rank(B) \), which contradicts the minimality of \( \rank(B) \).

  The obtained contradiction shows that \( B \cap A = \varnothing \).
\end{proof}

\begin{theorem}[Model of Z]\label{thm:cumulative_hierarchy_model_of_z}\mcite[exer. 12.7 \\ exer. 12.8]{Jech2003}
  The stage \( V_{\omega + \omega} \) of the \hyperref[def:cumulative_hierarchy]{von Neumann's cumulative hierarchy} is a standard model of \logic{Z}, i.e. \logic{ZFC} without the \hyperref[def:zfc/replacement]{axiom schema of replacement}.

  More generally, a necessary and sufficient condition for \( V_\alpha \) to be a model of \logic{Z} is for \( \alpha \) to be a limit ordinal larger than \( \omega \).
\end{theorem}
\begin{proof}
  The following axioms are automatically satisfied for any stage \( V_\alpha \):
  \begin{itemize}
    \item The validity of the \hyperref[def:zfc/extensionality]{axiom of extensionality} is inherited from the metatheory.

    \item The \hyperref[def:zfc/specification]{axiom schema of specification} is satisfied because each axiom in the schema \hyperref[def:first_order_definability]{defines} a subset of \( V_\alpha \) and because \( V_\alpha \) is a transitive set. This does not necessarily require the axiom schema of specification in the metatheory --- we only need the subsets of \( V_\alpha \) defined in \fullref{def:first_order_definability}.

    \item The \hyperref[def:zfc/union]{axiom of unions} is satisfied because if \( A \in V_\alpha \) and \( C \in \bigcup A \), then there exists some \( B \in A \) such that \( C \in B \in A \). Since \( V_\alpha \) is a transitive set, it follows that \( C \in V_\alpha \).

    \item The \hyperref[def:zfc/foundation]{axiom of foundation} is satisfied for any \( V_\alpha \) due to \fullref{thm:def:cumulative_hierarchy/well_founded}. Its validity is also inherited from the metatheory, but we do not actually need the \hyperref[def:zfc/foundation]{axiom of foundation} in the metatheory.

    \item The validity of the \hyperref[def:zfc/choice]{axiom of choice} is inherited from the metatheory.

    Indeed, for any family of sets \( \mscrA \in V_\alpha \) there exists a \hyperref[def:choice_function]{choice function} \( c: \mscrA \to \bigcup \mscrA \). The set \( \set{ c(A) \given A \in \mscrA } \) then has a lower rank by \fullref{thm:def:cumulative_hierarchy/rank_inequality} and thus by \fullref{thm:def:cumulative_hierarchy/subsets} it belongs to \( V_\alpha \).
  \end{itemize}

  The rest of the axioms are satisfied whenever some easy restrictions are imposed on \( \alpha \):
  \begin{itemize}
    \item The \hyperref[def:zfc/power_set]{axiom of power sets} is satisfied by \( V_\lambda \) for a limit ordinal \( \lambda \) if the axiom of power sets holds in the metatheory.

    Indeed, if \( A \in V_\lambda \) and \( A \) has rank \( \beta \), then necessarily \( \beta < \lambda \). Since \( A \subseteq V_\beta \), it follows that \( \pow(A) \subseteq \pow(V_\beta) = V_{\beta + 1} \). But \( \beta + 1 < \lambda \) since \( \lambda \) is a limit ordinal. Therefore, \( \rank(\pow(A)) = \beta + 1 < \lambda \). From \fullref{thm:def:cumulative_hierarchy/subsets} it follows that \( V_{\beta + 1} \subsetneq V_\lambda \) and thus \( \pow(A) \subsetneq V_\lambda \).

    \item The \hyperref[def:zfc/pairing]{axiom of pairing} is also satisfied by \( V_\lambda \) for a limit ordinal \( \lambda \) if the axiom of pairing holds in the theory.

    Let \( A \) and \( B \) be members of \( V_\lambda \). Let \( \beta \) be the larger of their ranks. Then \( A \) and \( B \) are subsets of \( V_\beta \), hence members of \( \pow(V_\beta) = V_{\beta + 1} \) and thus the set \( \set{ A, B } \) has rank \( \beta + 2 \).

    Since \( \lambda \) is a limit ordinal, we have \( \beta + 2 < \lambda \) and hence \( \set{ A, B } \in V_\lambda \) by \fullref{thm:def:cumulative_hierarchy/subsets}.

    \item The \hyperref[def:zfc/infinity]{axiom of infinity} is satisfied by any ordinal \( \alpha > \omega \).

    Indeed, by \fullref{thm:def:cumulative_hierarchy/ordinals} we have \( \omega \subseteq V_\omega \) and by \fullref{thm:def:cumulative_hierarchy/subsets} we have \( \omega \in V_\alpha \) for \( \alpha \geq \omega + 1 \).
  \end{itemize}

  As discussed in \fullref{ex:countable_limit_ordinals}, \( \omega + \omega \) is the smallest ordinal that is both strictly larger than \( \omega \) and is a limit ordinal. Therefore, \( \omega + \omega \) or any larger limit ordinal is a model of \( \logic{ZFC} \) without the axiom of replacement.
\end{proof}

\begin{definition}\label{def:cofinality}
  The \term{cofinality} of a \hyperref[def:preordered_set]{preordered set} \( (P, \leq) \) is defined as
  \begin{equation*}
    \op{cf}(P, \leq) \coloneqq \min\set{ \card(A) \given A \T{is a cofinal subset of} P }.
  \end{equation*}
\end{definition}

\begin{proposition}\label{thm:cardinal_cofinality}
  The \hyperref[def:cofinality]{cofinality} of an infinite cardinal \( \kappa \) is
  \begin{equation*}
    \op{cf}(\kappa) = \min\set{ \card(A) \given A \T{is an unbounded subset of} \kappa }.
  \end{equation*}
\end{proposition}
\begin{proof}
  Well-foundedness of \( \kappa \) ensures that it is bounded from below, hence a subset \( A \) of \( \kappa \) is bounded from above if and only if it is bounded.

  Since \( \kappa \) itself as a limit ordinal by \fullref{thm:cardinal_is_infinite_iff_limit_ordinal}, it has no maximum. Hence, it is unbounded.

  The above reflections along with \fullref{thm:totally_ordered_cofinal_equivalences} imply that a subset \( A \) of \( \kappa \) is cofinal if and only if it is unbounded.
\end{proof}

\begin{definition}\label{def:regular_cardinal}
  We say that the infinite cardinal \( \kappa \) is \term{regular} if any of the following equivalent conditions hold:
  \begin{thmenum}
    \thmitem{def:regular_cardinal/cofinality} \( \kappa \) it is equal to its own \hyperref[def:cofinality]{cofinality}.
    \thmitem{def:regular_cardinal/unbounded_subsets} Every unbounded subset of \( \kappa \) has cardinality \( \kappa \).
  \end{thmenum}

  Note that the term \enquote{regular cardinal} is unrelated to \fullref{thm:axiom_of_regularity}.

  If \( \kappa \) is not regular, we say that it is \term{singular}. Finite ordinals are neither regular nor singular.
\end{definition}
\begin{proof}
  \ImplicationSubProof{def:regular_cardinal/cofinality}{def:regular_cardinal/unbounded_subsets} Note that \( \card(A) \leq \kappa \) for \( A \subseteq \kappa \). \Fullref{thm:cardinal_cofinality} implies that if \( A \) is unbounded, we have \( \card(A) \geq \op{cf}(\kappa) \). But \( \op{cf}(\kappa) = \kappa \), hence the result follows.

  \ImplicationSubProof{def:regular_cardinal/unbounded_subsets}{def:regular_cardinal/cofinality} If every unbounded subset of \( \kappa \) has cardinality \( \kappa \), the minimum of all such cardinalities is \( \kappa \) and hence \( \op{cf}(\kappa) = \kappa \).
\end{proof}

\begin{remark}\label{rem:strongly_inaccessible_cardinal}
  If \( \kappa \) is an \hi{uncountable} \hyperref[def:regular_cardinal]{regular} (weak or strong) \hyperref[def:successor_and_limit_cardinal]{limit cardinal}, it is commonly called a (weakly or strongly) \term{inaccessible cardinal}.

  The assumption of uncountability is sometimes dropped, however. We avoid this ambiguity by being explicit and using \enquote{uncountable regular strong limit cardinal} rather than \enquote{strongly inaccessible cardinal}.

  Furthermore, we often do not need to restrict ourselves to uncountable regular strong limit cardinals. An added benefit to this is that we can utilize the universe of hereditary finite sets \hyperref[def:universe_of_hereditary_finite_sets]{\( V_\omega \)} as a \hyperref[def:grothendieck_universe]{Grothendieck universe} in category theory.
\end{remark}

\begin{proposition}\label{thm:aleph_zero_is_regular}
  The first infinite cardinal \( \aleph_0 \) is a \hyperref[def:regular_cardinal]{regular}.
\end{proposition}
\begin{proof}
  The only strict subsets of \( \aleph_0 \) are either countable or finite and the finite sets are bounded. The only unbounded subsets of \( \aleph_0 \) have cardinality \( \aleph_0 \). Hence, \( \aleph_0 \) equals its own cofinality and is thus regular.
\end{proof}

\begin{lemma}\label{thm:regular_cardinal_stage_supremum_lemma}
  Let \( \kappa \) be a \hyperref[def:regular_cardinal]{regular cardinal} and let \( R \subseteq \kappa \). If \( \card(R) < \kappa \), then \( \sup R < \kappa \).
\end{lemma}
\begin{proof}
  We shall prove that the set \( R \) is bounded from above with respect to the membership ordering of \( \kappa \). Indeed, suppose that it is unbounded. Then \( \card(R) = \kappa \) since \( \kappa \) is regular. But this contradicts our assumption that \( \card(R) < \kappa \).

  The obtained contradiction shows that \( R \) is bounded from above (with respect to membership in \( \kappa \)). Hence, there exists some ordinal \( \rho < \kappa \) that is an upper bound of \( R \). The supremum \( \sup R \) then satisfies \( \sup R \leq \rho < \kappa \).
\end{proof}

\begin{proposition}\label{thm:regular_cardinal_stage_inverse_transitivity}
  For any regular cardinal \( \kappa \), we have
  \begin{equation*}
    A \subseteq V_\kappa \T{if and only if} (A \in V_\kappa \T{and} \card(A) < \kappa).
  \end{equation*}
\end{proposition}
\begin{proof}
  \SufficiencySubProof Let \( A \subseteq V_\kappa \) and \( \card(A) < \kappa \). Define
  \begin{equation*}
    R \coloneqq \set{ r(x) \given x \in A }.
  \end{equation*}

  Then from \fullref{thm:regular_cardinal_stage_supremum_lemma} it follows that \( \sup R < \kappa \). Denote \( \sup R \) by \( \rho \). Then
  \begin{equation*}
    x \subseteq V_\rho \T{for every} x \in A.
  \end{equation*}

  Therefore, \( A \in V_{\rho + 2} \). Since \( \kappa \) is an infinite cardinal, from \fullref{thm:cardinal_is_infinite_iff_limit_ordinal} it follows that it is a limit ordinal and \( \rho + 1 < \kappa \); then from \fullref{thm:def:cumulative_hierarchy/subsets} it follows that \( A \in V_\kappa \).

  \NecessitySubProof Follows from \fullref{thm:def:cumulative_hierarchy/transitive}.
\end{proof}

\begin{proposition}\label{thm:strong_regular_cardinal_stages}
  If \( \kappa \) is a \hyperref[rem:strongly_inaccessible_cardinal]{regular strong limit cardinal}, then \( \card(V_\alpha) < \kappa \) for every \( \alpha < \kappa \).
\end{proposition}
\begin{proof}
  We use \fullref{thm:bounded_transfinite_induction} on \( \alpha \).
  \begin{itemize}
    \item If \( \alpha = 0 \), then \( V_\alpha = \varnothing \) and hence \( \card(V_\alpha) = 0 < \kappa \).
    \item If \( \alpha < \kappa \) and \( \card(V_\alpha) < \kappa \), then
    \begin{equation*}
      \card(V_{\alpha + 1})
      \reloset {\eqref{eq:def:cumulative_hierarchy}} =
      \card(\pow(V_\alpha))
      \reloset {\ref{thm:cardinal_exponentiation_power_set}} =
      2^{\card(V_\alpha)}.
    \end{equation*}

    Since \( \kappa \) is a strong limit, we have \( \card(V_{\alpha + 1}) < \kappa \).

    \item Suppose that \( \lambda < \kappa \) is a limit ordinal and that \( \card(V_\alpha) < \kappa \) for every \( \alpha < \lambda \).

    From \fullref{thm:def:cumulative_hierarchy/subsets} we have \( V_\alpha \subsetneq V_{\alpha + 1} \) for any \( \alpha < \lambda \), hence
    \begin{equation}\label{eq:thm:def:cumulative_hierarchy/subsets/limit_inclusion}
      \card(V_\alpha) \leq \card(V_{\alpha + 1}).
    \end{equation}

    Define the set
    \begin{equation*}
      C \coloneqq \set{ \card(V_\alpha) \given \alpha < \lambda }.
    \end{equation*}

    Then
    \begin{equation*}
      \card(V_\lambda)
      \reloset {\eqref{eq:def:cumulative_hierarchy}} =
      \card\parens*{ \bigcup \set{ V_\alpha \given \alpha < \lambda } }
      \reloset {\eqref{eq:thm:def:cumulative_hierarchy/subsets/limit_inclusion}} =
      \sup\set{ \card(V_\alpha) \given \alpha < \lambda }
      =
      \sup C.
    \end{equation*}

    From \fullref{thm:regular_cardinal_stage_supremum_lemma} it follows that \( \sup C < \kappa \). Therefore, \( \card(V_\lambda) = \sup C < \kappa \).
  \end{itemize}
\end{proof}

\begin{corollary}\label{thm:strong_regular_cardinal_stage_cardinality}
  If \( \kappa \) is a \hyperref[rem:strongly_inaccessible_cardinal]{regular strong limit cardinal}, then \( \card(A) < \kappa \) for every \( A \in V_\kappa \).
\end{corollary}
\begin{proof}
  Denote by \( \alpha \) the rank of \( A \). Then
  \begin{equation*}
    \card(A) \leq \card(V_\alpha) < \kappa,
  \end{equation*}
  where the last inequality follows from \fullref{thm:strong_regular_cardinal_stages}.
\end{proof}

\begin{theorem}[Model of ZFC]\label{thm:cumulative_hierarchy_model_of_zfc}\mcite[lemma 12.13]{Jech2003}
  The stage \( V_\kappa \) of the \hyperref[def:cumulative_hierarchy]{von Neumann's cumulative hierarchy} is a standard model of \logic{ZFC} for every \hyperref[rem:strongly_inaccessible_cardinal]{uncountable regular strong limit cardinal} \( \kappa \).
\end{theorem}
\begin{thmcomment}
  \begin{itemize}
    \item This theorem is an extension of \fullref{thm:cumulative_hierarchy_model_of_z}.
    \item Assuming that at least one strongly inaccessible cardinal exists, \logic{ZFC} is consistent since it has a model.
  \end{itemize}
\end{thmcomment}
\begin{proof}
  First note that \( \kappa \) is necessarily a limit ordinal by \fullref{thm:cardinal_is_infinite_iff_limit_ordinal}. It is also larger than \( \omega \) since it is uncountable. Hence, \fullref{thm:cumulative_hierarchy_model_of_z} is satisfied. We must only show that the \hyperref[def:zfc/replacement]{axiom schema of replacement} holds in \( V_\kappa \).

  Let \( A \in V_\kappa \). The \hyperref[def:zfc/replacement]{axiom schema of replacement} requires the image of every definable function from \( A \) to \( V_\kappa \) to be a member of \( V_\kappa \).

  Let \( \varphi \) be a \hyperref[def:first_order_syntax/formula]{formula} of \hyperref[def:zfc]{\logic{ZFC}} not containing \( \tau \) nor \( \sigma \) as free variables. Suppose additionally that for every \( x \in V_\kappa \) there exists a unique \( y \in V_\kappa \) such that \( \varphi\Bracks{\xi \to x, \eta \to y} = T \). These conditions ensure that \( \varphi \) can be plugged into the \hyperref[def:zfc/replacement]{axiom schema of replacement}. Define the relation
  \begin{equation*}
    f \coloneqq \set{ (x, y) \in V_\kappa^2 \given \varphi\Bracks{\xi \to x, \eta \to y} = T }.
  \end{equation*}

  Then \( f \) is a function because of our earlier restrictions on \( \varphi \). Since \( V_\kappa \) is transitive and \( A \in V_\kappa \), it is sufficient to consider the restriction of \( f\restr_A \) of \( f \). It will not be necessary to do even that since we will explicitly restrict ourselves to values of \( f \) on \( A \).

  Clearly \( f[A] \) is a subset of \( V_\kappa \) and hence \( \rank(f[A]) \leq \kappa \). In order to show that the instance of the axiom schema of replacement with \( \varphi \) holds, is sufficient to show that \( \rank(f[A]) < \kappa \).

  It is clear that
  \begin{equation*}
    \card(f[A]) \leq \card(\dom(f\restr_A)) = \card(A).
  \end{equation*}

  From \fullref{thm:strong_regular_cardinal_stage_cardinality} it follows that \( \card(A) < \kappa \) and since \( \card(f[A]) < \kappa \), from \fullref{thm:regular_cardinal_stage_inverse_transitivity} it follows that \( f[A] \in V_\kappa \).
\end{proof}

\begin{definition}\label{def:universe_of_hereditary_finite_sets}
  For this reason, \( V_\omega \) is known as the \term{universe of hereditary finite sets}.

  From \fullref{thm:aleph_zero_is_strong_limit} and \fullref{thm:aleph_zero_is_regular} it follows that \( \omega = \aleph_0 \) is a regular strong limit cardinal. Then by \fullref{thm:strong_regular_cardinal_stage_cardinality}, every member of \( V_\omega \) is finite. Because \( V_\omega \) is a transitive set, every member of every member of \( V_\omega \) is also finite. So is every member of every member of every member.
\end{definition}

\begin{proposition}\label{thm:cumulative_hierarchy_model_of_zfc_without_infinity}
  The universe of hereditary finite sets \hyperref[def:universe_of_hereditary_finite_sets]{\( V_\omega \)} is a standard model of \logic{ZFC} without the \hyperref[def:zfc/infinity]{axiom of infinity}.
\end{proposition}
\begin{proof}
  From the proof of \fullref{thm:cumulative_hierarchy_model_of_z} is follows that \( V_\omega \) is a model of \logic{ZFC} without the axiom of infinity and the axiom schema of replacement.

  \Fullref{thm:cumulative_hierarchy_model_of_zfc} shows that being a regular strong limit cardinal is sufficient for \( V_\omega \) to satisfy the axiom of replacement.
\end{proof}

  \subsection{Grothendieck universes}\label{subsec:grothendieck_universes}

Instead of having one single universe, we can have multiple universes where each is contained in another one. The upside of this is that we can do category theory formally within set theory --- see the discussions in \fullref{def:category_size}. The downside of this is that, unlike models of \hyperref[def:zfc]{\logic{ZFC}}, models of \hyperref[def:axiom_of_universes]{\logic{ZFC+U}} (\logic{ZFC} with the axiom that every set is contained in some \hyperref[def:grothendieck_universe]{Grothendieck universe}) are much less studied. In particular, this axiom requires the existence of an unbounded hierarchy of \hyperref[rem:strongly_inaccessible_cardinal]{regular strong limit cardinal}, unlike \( \logic{ZFC} \) for which only one such cardinal is sufficient.

\begin{definition}\label{def:grothendieck_universe}\mcite[22]{MacLane1998}
  We say that the set \( \mscrU \) is a \term{Grothendieck universe} if it satisfied the following conditions:
  \begin{thmenum}
    \thmitem[def:grothendieck_universe/nonempty]{GU1} It is \hyperref[def:empty_set]{nonempty}.

    \thmitem[def:grothendieck_universe/transitive]{GU2} It is a \hyperref[def:transitive_set]{transitive set}.

    \thmitem[def:grothendieck_universe/power_set]{GU3} For any \( A \in \mscrU \), the \hyperref[def:basic_set_operations/power_set]{power set} \( \pow(A) \) also belongs to \( \mscrU \).

    \thmitem[def:grothendieck_universe/union]{GU4} For any member \( A \in \mscrU \) and any \( A \)-indexed family \( \set{ B_a }_{a \in A} \subseteq \mscrU \), the union
    \begin{equation*}
      \bigcup\set{ B_a \given a \in A }
    \end{equation*}
    belongs to \( \mscrU \). This is a restriction from unions over completely arbitrary families of sets to those families that can be indexed by members of \( A \).
  \end{thmenum}

  We formalize the entire concept via the following monstrous formula:
  \small
  \begin{equation*}\taglabel[\op{IsUniverse}]{eq:def:grothendieck_universe/predicate}
    \begin{aligned}
      \ref{eq:def:grothendieck_universe/predicate}[\upsilon] \coloneqq
        &\neg \ref{eq:def:empty_set/predicate}[\upsilon]
        \wedge
        \qforall {\tau \in \upsilon}
        \vast(
          \ref{eq:def:subset/predicate}[\tau, \upsilon]
          \wedge
          \parens[\Big]
            {
              \qexists {\xi \in \upsilon}
              \ref{eq:def:basic_set_operations/power_set/predicate}[\xi, \tau]
            }
          \wedge \\ &\wedge
          \qforall \xi
          \qforall \eta
          \parens[\Bigg]
          {
            \parens[\Big]
              {
                \underbrace
                  {
                    \ref{eq:def:function/predicate}[\xi, \tau, \upsilon]
                  }_{\mathclap{\xi \T*{is a function} \tau \to \upsilon}}
                \wedge
                \underbrace
                  {
                    \ref{eq:def:grothendieck_universe/predicate_isimage}[\eta, \xi]
                  }_{ \mathclap{\eta \T*{is the image of} \xi} }
              }
            \rightarrow
            \underbrace
              {
                \qexists {\zeta \in \upsilon} \ref{eq:def:basic_set_operations/union/predicate}[\zeta, \xi]
              }_{ \bigcup \img(\xi) \in \upsilon}
        }
      \vast),
    \end{aligned}
  \end{equation*}
  \normalsize
  where
  \begin{equation*}\taglabel[\op{IsImage}]{eq:def:grothendieck_universe/predicate_isimage}
    \ref{eq:def:grothendieck_universe/predicate_isimage}[\rho, \tau]
    \coloneqq
    \qforall \xi
    \parens[\Big]
    {
      \xi \in \rho
      \leftrightarrow
      \underbrace
      {
        \qexists {\eta \in \tau}
        \qexists \zeta
        \ref{eq:def:cartesian_product/kuratowski_pair_predicate}[\eta, \zeta, \xi]
      }_{\tau(\zeta) = \xi \T*{for some} \xi \in \dom(\tau) }
    }.
  \end{equation*}
\end{definition}

\begin{lemma}\label{thm:grothendieck_universe_contains_finite_sets}
  Every Grothendieck universe is a superset of universe of hereditary finite sets \hyperref[def:universe_of_hereditary_finite_sets]{\( V_\omega \)}.
\end{lemma}
\begin{proof}
  Let \( \mscrU \) be a Grothendieck universe.

  \ref{def:grothendieck_universe/nonempty} ensures that it is nonempty. Then there exists some set \( A \in \mscrU \). By \ref{def:grothendieck_universe/power_set}, \( \pow(A) \in \mscrU \). By \ref{def:grothendieck_universe/transitive}, \( \varnothing \in \pow(A) \in \mscrU \) implies \( \varnothing \in \mscrU \).

  Finally, from \ref{def:grothendieck_universe/power_set} by \fullref{thm:bounded_transfinite_induction} if follows that \( V_{n+1} = \pow(V_n) \) is a member of \( \mscrU \) for every \( n \in \omega \).

  Therefore, \( V_\omega = \bigcup \set{ V_n \given n \in \omega } \) is a subset of \( \mscrU \).
\end{proof}

\begin{definition}\label{def:axiom_of_universes}\mcite[24]{MacLane1998}
  The \term{axiom of universes} states that any set is contained in a \hyperref[def:grothendieck_universe]{Grothendieck universe}. Symbolically,
  \begin{equation}\label{eq:def:axiom_of_universes}
    \begin{aligned}
      \qforall \tau \qexists \upsilon \parens[\Big]{ \ref{eq:def:grothendieck_universe/predicate}[\upsilon] \wedge \tau \in \upsilon }.
    \end{aligned}
  \end{equation}

  We usually add this theorem to \hyperref[def:zfc]{ZFC} and call the resulting \hyperref[def:first_order_theory]{logical theory} \logic{ZFC+U}.
\end{definition}

\begin{example}\label{ex:def:axiom_of_universes}
  From \fullref{thm:grothendieck_universe_iff_regular_strong_limit} it follows that the universe of hereditary finite sets \hyperref[def:universe_of_hereditary_finite_sets]{\( V_\omega \)} is a Grothendieck universe.

  The existence of other universes cannot be proven in \logic{ZFC}. For this reason, we use the \hyperref[def:axiom_of_universes]{axiom of universes}.
\end{example}

\begin{proposition}\label{thm:smallest_grothendieck_universe_existence}
  Suppose that we are working in \logic{ZFC+U}. Then for any set \( A \), there exists a smallest Grothendieck universe containing \( A \).

  More generally, fix a set \( A \). Then there exists a smallest Grothendieck universe containing \( A \).
\end{proposition}
\begin{proof}
  If no set \( A \) is given, we simply take \( A = \varnothing \) since it must belong to every universe by definition.

  We use a trick analogous to \fullref{thm:smallest_inductive_set_existence}.

  The \hyperref[def:axiom_of_universes]{axiom of universes} states that there exists at least one universe \( \mscrU \) that contains \( A \). Define
  \begin{equation*}
    \widehat \mscrU \coloneqq \set{ x \in \mscrU \given x \T{belongs to every Grothendieck universe} }.
  \end{equation*}

  Now that we have defined \( \widehat \mscrU \), it remains to verify that it is itself a universe. To show \ref{def:grothendieck_universe/nonempty}, note that \( V_\omega \in \mscrU \) by \fullref{thm:grothendieck_universe_contains_finite_sets} and hence, by \ref{def:grothendieck_universe/transitive}, \( \omega \in \mscrU \).

  The rest of the verification is trivial.
\end{proof}

\begin{definition}\label{def:large_and_small_sets}\mcite[22]{MacLane1998}
  Suppose \( \mscrV = (V, I) \) is a \hyperref[def:first_order_semantics/satisfiability]{model} of \logic{ZFC+U}. Let \( \mscrU \) be a fixed Grothendieck universe.

  We say that a set \( A \) is \( \mscrU \)-\term{small} if \( A \in \mscrU \) and \( \mscrU \)-\term{moderate} if \( A \subseteq \mscrU \). This situation resembles the difference between sets and proper classes described in \fullref{def:set_builder_notation}.

  A set that is not \( \mscrU \)-\term{small} is called \( \mscrU \)-\term{large}. Note that any strict superset of \( \mscrU \) is \( \mscrU \)-large, but not \( \mscrU \)-moderate.

  Without further context (i.e. in \enquote{ordinary mathematics}), we assume that \( \mscrU \) refers to the \hyperref[thm:smallest_grothendieck_universe_existence]{smallest Grothendieck universe} that contains all sets of interest and instead of the terms \( \mscrU \)-large and \( \mscrU \)-small, we simply use the terms \term{large} and \term{small}.

  In category theory, however, if there is nothing to guarantee the existence of a larger Grothendieck universe, we cannot construct the \hyperref[def:functor_category]{functor category} of \( \mscrU \)-large categories, as discussed in \fullref{rem:functor_category_size}. This is the main motivation for the \hyperref[def:axiom_of_universes]{axiom of universes}.
\end{definition}

\begin{example}\label{ex:large_and_small_sets}
  \hfill
  \begin{itemize}
    \item A set is finite if and only if it is \( V_\omega \)-small. A set is \( V_\omega \)-moderate if and only if it is an infinite family of finite sets.

    For finite mathematics such as most of combinatorics, we rarely need to work outside of \( V_\omega \).

    \item If \( \kappa < \mu \) are regular strong limit cardinals, the stage \( V_\kappa \) of von Neumann's hierarchy is \( V_\mu \)-small by \fullref{thm:def:cumulative_hierarchy/membership}.
  \end{itemize}
\end{example}

\begin{theorem}\label{thm:grothendieck_universe_iff_regular_strong_limit}\mcite{Kruse1966}
  The stage \( V_\kappa \) of the \hyperref[def:cumulative_hierarchy]{von Neumann's cumulative hierarchy} is a \hyperref[def:grothendieck_universe]{Grothendieck universe} for every \hyperref[rem:strongly_inaccessible_cardinal]{regular strong limit cardinal} \( \kappa \).

  Conversely, for every Grothendieck universe \( \mscrU \), there exists a regular strong limit cardinal \( \kappa \) such that \( \mscrU = V_\kappa \).
\end{theorem}
\begin{proof}
  \SufficiencySubProof Let \( \kappa \) be a regular strong limit cardinal.

  \SubProofOf*{def:grothendieck_universe/nonempty} Since \( \kappa \) is infinite and thus \( 0 < \kappa \), from \fullref{thm:def:cumulative_hierarchy/membership} it follows that \( V_0 \in V_\kappa \).

  \SubProofOf*{def:grothendieck_universe/transitive} The set \( V_\kappa \) is transitive as shown in \fullref{thm:def:cumulative_hierarchy/transitive}.

  \SubProofOf*{def:grothendieck_universe/power_set} Since \( \kappa \) is a limit ordinal, \( V_\kappa \) satisfies the axiom of power sets as shown in \fullref{thm:cumulative_hierarchy_model_of_z} and thus if \( A \in V_\kappa \), then \( \pow(A) \in V_\kappa \).

  \SubProofOf*{def:grothendieck_universe/union} Fix some member \( A \in V_\kappa \) and some \( A \)-indexed family \( \set{ B_a }_{a \in A} \subseteq V_\kappa \). From \fullref{thm:strong_regular_cardinal_stage_cardinality} it follows that \( \card(A) < \kappa \) and \( \card(B_a) < \kappa \) for every \( a \in A \). Thus,
  \begin{equation*}
    \card(\set{ B_a \given a \in A }) \leq \card(A) < \kappa
  \end{equation*}
  and from \fullref{thm:regular_cardinal_stage_inverse_transitivity} it follows that
  \begin{equation*}
    \set{ B_a \given a \in A } \in V_\kappa.
  \end{equation*}

  The union
  \begin{equation*}
    \bigcap \set{ B_a \given a \in A }
  \end{equation*}
  is then a member of a lower stage, hence it also belongs to \( V_\kappa \).

  \NecessitySubProof Let \( \mscrU \) be a Grothendieck universe and let \( \alpha \) be the smallest \hi{ordinal} not in \( \mscrU \). We will first show that \( \mscrU = V_\alpha \) and gradually prove that \( \alpha \) is actually an inaccessible cardinal.

  \SubProof*{Proof that \( V_\beta \in \mscrU \) for ordinals \( \beta \in \mscrU \)} We will use \fullref{thm:bounded_transfinite_induction} on \( \beta < \alpha \).
  \begin{itemize}
    \item From \fullref{thm:grothendieck_universe_contains_finite_sets} it follows that \( \varnothing \in \mscrU \).

    \item If \( \beta < \alpha \) and \( V_\beta \in \mscrU \), then by \ref{def:grothendieck_universe/power_set} we have \( V_{\beta + 1} = \pow(V_\beta) \in \mscrU \).

    We will come back to this step a bit later, but for now note that \( V_{\beta + 1} \in \mscrU \) regardless of whether \( \beta + 1 \in \mscrU \).

    \item If \( \lambda < \alpha \) is a limit ordinal and \( V_\beta \in \mscrU \) for every \( \beta < \lambda \), we have
    \begin{equation*}
      V_\lambda
      \reloset {\eqref{eq:def:cumulative_hierarchy}} =
      \bigcup\set{ V_\beta \given \beta < \lambda },
    \end{equation*}
    which is a \( \lambda \)-indexed union of members of \( \mscrU \). Since \( \lambda \in \mscrU \), by \ref{def:grothendieck_universe/power_set} we have \( V_\lambda \in \mscrU \).
  \end{itemize}

  \SubProof*{Proof that \( \alpha \) is a limit ordinal} In the successor case we noted that \( V_{\beta + 1} \in \mscrU \) for every \( \beta < \alpha \) regardless of whether \( \beta + 1 < \alpha \). Since \( \rank(\beta + 1) = \beta + 1 \), it follows that \( \beta + 1 \in V_{\beta + 2} \in \mscrU \) and thus by \ref{def:grothendieck_universe/transitive}, \( \beta + 1 \in \mscrU \). Therefore, \( \alpha \) cannot be a successor ordinal --- if \( \alpha = \beta + 1 \), then \( \beta \in \mscrU \) by definition of \( \alpha \) and thus \( \beta + 1 = \alpha \in \mscrU \), which is a contradiction.

  Since \( \alpha > 0 \), it remains for \( \alpha \) to be a limit ordinal.

  \SubProof*{Proof that \( V_\alpha \subseteq \mscrU \)} By \ref{def:grothendieck_universe/transitive}, \( V_\beta \subseteq \mscrU \) for every \( \beta < \alpha \). We can conclude that
  \begin{equation*}
    V_\alpha
    \reloset {\eqref{eq:def:cumulative_hierarchy}} =
    \bigcup\set{ V_\beta \given \beta < \alpha }
    \subseteq
    U.
  \end{equation*}

  In order to show that equality holds, we must first prove that \( \alpha \) is a strongly inaccessible cardinal. But this requires some auxiliary results.

  \SubProof*{Proof that \( \set{ B } \in \mscrU \) for every \( B \in \mscrU \)} By \ref{def:grothendieck_universe/power_set} we have that \( \pow(\pow(B)) \in \mscrU \). But \( \set{ B } \subseteq \pow(B) \) and hence \( \set{ B } \in \pow(\pow(B)) \). By \ref{def:grothendieck_universe/transitive}, \( \set{ B } \in \mscrU \).

  \SubProof*{Proof that \( \kappa = \alpha \) is a cardinal} Suppose that \( \alpha \) is not a cardinal. Indeed, suppose that there exists some \( \beta < \alpha \) such that there exists a bijective function \( f: \beta \to \alpha \). Then
  \begin{equation*}
    \alpha = \bigcup\set{ \set{ f(\gamma) } \given \gamma < \beta }
  \end{equation*}
  is a \( \beta \)-indexed union of members of \( \alpha \) and hence \( \alpha \in \alpha \). But this contradicts \fullref{thm:simple_foundation_theorems/member_of_itself}.

  Therefore, \( \alpha \) is a cardinal. We will henceforth denote it by \( \kappa \) to highlight that it is a cardinal.

  \SubProof*{Proof that \( \card(B) \in \mscrU \) for every \( B \in \mscrU \)} Let \( B \in \mscrU \) and let \( f: B \to \card(B) \) be a bijective function. Then
  \begin{equation*}
    \card(B)
    =
    f[B]
    =
    \set{ f(x) \given x \in B }
    =
    \bigcup\set[\Big]{ \set{ f(x) } \given x \in B }
  \end{equation*}
   is a \( B \)-indexed union of members of \( \mscrU \) and, by \ref{def:grothendieck_universe/union}, \( \card(B) \in \mscrU \).

  \SubProof*{Proof that \( \kappa \) is a strong limit} For every \( \beta < \kappa \) by \ref{def:grothendieck_universe/power_set} we have \( \pow(\beta) \in \mscrU \). We have already shown that \( \card(\pow(\beta)) \in \mscrU \) and, by \fullref{thm:cardinal_exponentiation_power_set}, we have
  \begin{equation*}
    \card(\pow(\beta)) = 2^{\card(\beta)} = 2^\beta.
  \end{equation*}

  Hence, \( 2^\beta < \kappa \) and \( \kappa \) is a strong limit.

  \SubProof*{Proof that \( \kappa \) is regular} Let \( C \subseteq \kappa \) be an unbounded set. We will show that \( \card(C) = \kappa \).

  Suppose that \( \card(C) < \kappa \). Then \( \card(C) \in \mscrU \) since \( \kappa = \alpha \) is the smallest ordinal not contained in \( \mscrU \). Let \( f: \card(C) \to C \) be a bijective function. Then
  \begin{equation*}
    C = \bigcup\set[\Big]{ \set{ f(\gamma) } \given \gamma < \card(C) }
  \end{equation*}
  and by \ref{def:grothendieck_universe/union}, \( C \in \mscrU \).

  Since \( C \) is unbounded, we have \( \sup C \geq \kappa \). But from \fullref{thm:union_of_set_of_ordinals} is follows that \( \bigcup C = \sup C \) and by \ref{def:grothendieck_universe/union}, \( \bigcup C \in \mscrU \), which implies that \( \sup C < \kappa \).

  The obtained contradiction shows that \( \card(C) = \kappa \). Since \( C \) was an arbitrary unbounded set, it follows that \( \kappa \) satisfies \fullref{def:regular_cardinal/unbounded_subsets} and is thus regular.

  \SubProof*{Proof that \( V_\kappa = U \)} Finally, now that we know that \( \kappa \) is a strongly inaccessible cardinal, we can show that equality holds in \( V_\kappa \subseteq \mscrU \).

  Aiming at a contradiction, suppose that \( U \setminus V_\kappa \) is nonempty. By the \hyperref[def:zfc/foundation]{axiom of foundation}, there exists a set \( C \in \mscrU \setminus V_\kappa \) such that
  \begin{equation*}
    C \cap (U \setminus V_\kappa) = \varnothing,
  \end{equation*}
  thus \( C \subseteq V_\kappa \). From \fullref{thm:strong_regular_cardinal_stage_cardinality} if follows that \( \card(C) < \kappa \) and from \fullref{thm:regular_cardinal_stage_inverse_transitivity} it follows that \( C \in V_\kappa \), which contradicts our choice of \( C \) as a member of \( U \setminus V_\kappa \).

  Therefore, \( V_\kappa = U \).
\end{proof}

\begin{corollary}\label{thm:grothendieck_universe_is_model_of_zfc}
  Every uncountable Grothendieck universe is a standard model of \hyperref[def:zfc]{\logic{ZFC}}.
\end{corollary}
\begin{proof}
  Follows from \fullref{thm:grothendieck_universe_iff_regular_strong_limit} and \fullref{thm:cumulative_hierarchy_model_of_zfc}.
\end{proof}


  \section{Category theory}\label{sec:category_theory}

Category theory studies objects via how they relate to other objects. It shifts the focus from how individual members behave and even has no concept of membership, upon which set theory is based.

This shift is evident from the following diagram, which is actually half of the proof of \fullref{thm:functor_adjoint_uniqueness}:
\begin{equation*}
  \begin{aligned}
    \includegraphics[page=1]{output/thm__functor_adjoint_uniqueness.pdf}
  \end{aligned}
\end{equation*}

We do still have individual objects, precisely the nodes of the diagram above, however we are only interested in how the nodes are related to each other. Chasing the relations in this diagram individually would require a lot more effort with little gain.

Categories can be defined \enquote{from the ground up} so that they may be used without an underlying set theory or logic. For our purposes, it will be more appropriate to define categories via \hyperref[def:quiver]{quivers}, a.k.a. directed multigraphs. This latter approach will be much more convenient for us, since we are working in \hyperref[def:axiom_of_universes]{\logic{ZFC+U}} and are only interested in categories insomuch as they are helpful to us.

Furthermore, categories are actually the primary motivation for us include the \hyperref[def:axiom_of_universes]{axiom of universes} in our metatheory that would otherwise include only the axioms of \hyperref[def:zfc]{\logic{ZFC}}. This is discussed further in \fullref{rem:functor_size} and \fullref{rem:functor_category_size}.

  \subsection{Categories}\label{subsec:categories}

\begin{definition}\label{def:category}\mcite[def. 1.1.1]{Leinster2016Basic}
  A \term{category} is a \hyperref[def:quiver]{quiver} \( \cat{C} \) equipped with a \hyperref[def:partial_function]{partial operation} \( \bincirc \) on the arrows of \( \cat{C} \) and another operation \( \id \) that selects a distinguished arrow for each vertex.

  In tradition regarding \hyperref[def:concrete_category]{forgetful functors}, we denote the underlying quiver of \( \cat{C} \) by \( U(\cat{C}) \).

  \begin{thmenum}[series=def:category]
    \thmitem{def:category/objects} We call the vertices of the quiver \term{objects} and denote the set of all objects by \( \obj(\cat{C}) \). We will often write \( A \in \cat{C} \) as a shorthand for \( A \in \obj(\cat{C}) \).

    \thmitem{def:category/morphisms} We call the arrows of the quiver \term{morphisms} or sometimes \term{maps}. If \( f \) is a morphism, we call its head its \term{domain} \( \dom(f) \) and its tail its \term{codomain} \( \co\dom(f) \). We denote a morphism from \( A \) to \( B \) by \( f: A \to B \) or \( A \reloset f \to B \).

    We call the set \( \cat{C}(A, B) \) of all morphisms from \( A \) to \( B \) a \term{morphism set} or \term{\( \hom \)-set}. We use the shorthand \( \cat{C}(A) \) for \( \cat{C}(A, A) \). Another established notation is \( \op{hom}(A, B) \) instead of \( \cat{C}(A, B) \).

    Both of these notations highlight that \( \cat{C}(A, B) \), when parameterized by \( A \) and \( B \), is a \hyperref[def:functor]{functor}, as discussed in \fullref{def:hom_functor}.

    \thmitem{def:category/composition} We require the \term{composition} \( \bincirc \) of the arrows \( f \) and \( g \) to be defined only if \( \co\dom(f) = \dom(g) \). In this case, we require \( g \bincirc f \) to be a morphism from \( \dom(f) \) to \( \co\dom(g) \).

    Note how the order of \( f \) and \( g \) may seem confusing: we write the composition of \( f: A \to B \) and \( g: B \to C \) as \( g \bincirc f: A \to C \). This is set up so that it matches \hyperref[def:multi_valued_function/composition]{function composition}. The order may seem different compared to multiplication in \hyperref[def:group]{groups}, for example, however \fullref{def:monoid_delooping} shows that this is actually a generalization of multiplication.

    This order of composition is used in \cite[7]{MacLane1994}, \cite[def. 1.1.1]{Leinster2016Basic} and \cite[def. I.3.1]{Aluffi2009}.

    \thmitem{def:category/identity} We denote the \term{identity morphism} of an object \( A \) by \( \id_A \).
  \end{thmenum}

  The definition of a category additionally requires the following conditions to hold:
  \begin{thmenum}[resume=def:category]
    \thmitem[def:category/C1]{C1} For any morphism \( f: A \to B \), the identities \( \id_A \) and \( \id_B \) must satisfy
    \begin{equation}\label{eq:def:category/C1}\tag{\logic{C1}}
      f \bincirc \id_A = \id_B \bincirc f = f.
    \end{equation}

    \thmitem[def:category/C2]{C2} Composition must be associative. That is, for each triple of morphism \( f: A \to B \), \( g: B \to C \) and \( h: C \to D \), the following must hold:
    \begin{equation}\label{eq:def:category/C2}\tag{\logic{C2}}
      (h \bincirc g) \bincirc f = h \bincirc (g \bincirc f).
    \end{equation}
  \end{thmenum}
\end{definition}

\begin{example}\label{ex:def:category}
  Examples of categories include:

  \begin{itemize}
    \item The category \( \cat{Set} \) of \hyperref[def:large_and_small_sets]{small} \hyperref[def:set]{sets} and \hyperref[def:function]{functions} defined in \fullref{def:category_of_small_sets}.

    \item The category \( \cat{Cat} \) of small categories defined in \fullref{def:category_of_small_categories}.

    \item All the \hyperref[def:category_of_small_first_order_models]{categories of small first-order models} listed in \fullref{ex:def:category_of_small_first_order_models}

    \item The category \( \cat{Top} \) of small \hyperref[def:topological_space]{topological spaces} and \hyperref[def:global_continuity]{continuous functions} defined in \fullref{def:category_of_small_topological_spaces}.

    \item For every topological space, the fundamental groupoid defined in \fullref{def:fundamental_groupoid}.

    \item The category \( \cat{Quiv} \) of small \hyperref[def:quiver]{quivers} defined in \fullref{def:category_of_small_quivers}.

    \item For every quiver, the free category defined in \fullref{def:quiver_free_category}.

    \item For every \hyperref[def:preordered_set]{preordered set}, the induced category defined in \fullref{thm:order_category_isomorphism}.
  \end{itemize}
\end{example}

\begin{definition}\label{def:category_size}
  As can be seen from \fullref{ex:def:category}, some of the categories we are working with, like \( \cat{Set} \), contain as objects all \hyperref[def:large_and_small_sets]{small sets}. As mentioned in \fullref{def:large_and_small_sets}, the concept of a small set is defined relative to the smallest Grothendieck universe that suits our needs.

  \Fullref{thm:russels_paradox} demonstrates that the set of all sets easily leads to a paradox, which is the reason we restrict our attention only to sets within some Grothendieck universe. This universe is implicit by default, however we will occasionally need to make it explicit.

  We will say that the category \( \cat{C} \) is \term{locally \( \mscrU \)-small} if the morphism set \( \cat{C}(A, B) \) is \( \mscrU \)-small for every pair of objects \( A \) and \( B \). If, in addition, the set \( \obj(\cat{C}) \) of objects is also \( \mscrU \)-small, we will say that the category \( \cat{C} \) is \term{\( \mscrU \)-small}. If a category is not \( \mscrU \)-small, we say that it is \term{\( \mscrU \)-large}.

  In particular, \term{finite} and \term{locally finite} categories are ones who are \( V_\omega \)-small and \( V_\omega \)-locally small for the universe of hereditary finite sets \hyperref[def:universe_of_hereditary_finite_sets]{\( V_\omega \)}. This notion of local finiteness is unrelated to local finiteness of graphs defined in \fullref{def:hypergraph/degree}.

  Universes are crucial to be able to do a lot of categorical constructions within set theory, most importantly \( \mscrU \)-large \hyperref[def:functor_category]{functor categories} but also \hyperref[def:product_category]{product categories} and, as discussed in \fullref{rem:functor_size}, even the \hyperref[def:functor]{functors} themselves.

  Note that, even if a category is \( \mscrU \)-small, the category itself as the tuple \( (Q, \bincirc, \id) \) from \fullref{def:category} may not be a \( \mscrU \)-small set.

  Also note that, in a locally small category, it is possible for the set of all morphisms to be \( \mscrU \)-large. This is impossible for small categories due to \ref{def:grothendieck_universe/union}.

  We sometimes skip the prefix \enquote{\( \mscrU \)-} if it is unimportant, and simply speak of \enquote{large categories} or \enquote{locally small categories}.
\end{definition}

\begin{definition}\label{def:category_of_small_sets}
  Suppose that we are given a \hyperref[def:grothendieck_universe]{Grothendieck universe} \( \mscrU \), which is safe to assume to be the smallest suitable one as explained in \fullref{def:large_and_small_sets}.

  We denote the \hyperref[def:category]{category} of \( \mscrU \)-small \hyperref[def:set]{sets} by \( \ucat{Set} \) or, if the universe is clear from the context, simply by \( \cat{Set} \). See \fullref{def:category_size} for a further discussion of universes and categories.

  \begin{itemize}
    \item The \hyperref[def:category/objects]{set of objects} \( \obj(\cat{Set}) \) is \( \mscrU \) itself.

    \item The \hyperref[def:category/morphisms]{set of morphisms} \( \cat{Set}(A, B) \) from \( A \) to \( B \) is the set \hyperref[def:function/set_of_functions]{\( \fun(A, B) \)} of all total single-valued functions from \( A \) to \( B \).

    \item The \hyperref[def:category/composition]{composition of morphisms} is the usual \hyperref[def:multi_valued_function/composition]{function composition}.

    \item The \hyperref[def:category/identity]{identity morphism} on the set \( A \) is the \hyperref[def:multi_valued_function/identity]{identity function}
    \begin{equation*}
      \begin{aligned}
        &\id_A: A \to A \\
        &\id_A(x) \coloneqq A.
      \end{aligned}
    \end{equation*}
  \end{itemize}
\end{definition}
\begin{defproof}
  To see that \( \ucat{Set} \) is indeed a category, we verify the conditions \ref{def:category/C1} and \ref{def:category/C2}.

  \SubProofOf{def:category/C1} For every two sets \( A, B \in \mscrU \) and every function \( f: A \to B \), for all \( x \in A \) we have
  \begin{equation*}
    [\id_B \bincirc f](x)
    =
    \id_B(f(x))
    =
    f(x)
    =
    f(\id_A(x))
    =
    [f \bincirc \id_A](x).
  \end{equation*}

  Therefore, \( \id_A \) and \( \id_B \) satisfy \eqref{eq:def:category/C1}.

  \SubProofOf{def:category/C2} Associativity of function composition is proved in \fullref{thm:def:multivalued_function/associative}.
\end{defproof}

\begin{proposition}\label{thm:def:category_of_small_sets}
  We collect here important properties of the category \hyperref[def:category_of_small_sets]{\( \ucat{Set} \)} of \( \mscrU \)-small sets. Most of them require forward references.

  \begin{thmenum}
    \thmitem{thm:def:category_of_small_sets/large} It is a \( \mscrU \)-large category in the sense of \fullref{def:category_size} because \( \mscrU \) itself is the set of objects and, defined as a \hyperref[def:quiver]{quiver} with additional operations, the category is a \( \mscrU \)-large set in the sense of \fullref{def:large_and_small_sets}.

    \thmitem{thm:def:category_of_small_sets/locally_small} It is a \hyperref[def:category_size]{\( \mscrU \)-locally small category} because \( \mscrU \) is a model of \hyperref[def:zfc]{\( \logic{ZFC} \)} and \fullref{thm:zfc_existence_theorems/set_of_functions} holds.

    \thmitem{thm:def:category_of_small_sets/morphism_invertibility} All \hyperref[def:morphism_invertibility/right_cancellative]{epimorphisms} and \hyperref[def:multi_valued_function/empty]{nonempty} \hyperref[def:morphism_invertibility/left_cancellative]{monomorphisms} \hyperref[def:morphism_invertibility/left_invertible]{split} and are precisely the \hyperref[def:function_invertibility/surjective]{surjective} and nonempty \hyperref[def:function_invertibility/injective]{injective functions}, respectively.

    This is stated in \fullref{thm:function_invertibility_categorical}. See also \fullref{thm:epimorphisms_split_in_set}.

    \thmitem{thm:def:category_of_small_sets/universal_objects} The empty set \( \varnothing \) is an \hyperref[def:universal_objects/initial]{initial object} and the singleton set \( \set{ A } \) is a \hyperref[def:universal_objects/terminal]{terminal object} for every \( A \in \ucat{Set} \). No \hyperref[def:universal_objects/zero]{zero objects} exist in \( \ucat{Set} \) by \fullref{thm:def:universal_objects/no_zero}.

    This is discussed in \fullref{ex:def:universal_objects}.

    \thmitem{thm:def:category_of_small_sets/discrete_category} The \hyperref[def:discrete_category]{discrete category} functor \( D: \ucat{Set} \to \ucat{Cat} \) is left adjoint to the forgetful functor \( U: \ucat{Cat} \to \ucat{Set} \)

    This is discussed in \fullref{ex:def:category_adjunction/set_cat}.

    \thmitem{thm:def:category_of_small_sets/limits} The \hyperref[def:discrete_category_limits]{products} and \hyperref[def:discrete_category_limits]{coproducts} are the \hyperref[def:cartesian_product/product]{Cartesian products} and the \hyperref[def:disjoint_union]{disjoint unions}, respectively.

    This is stated in \fullref{thm:discrete_category_limits_in_set}.
  \end{thmenum}
\end{proposition}

\begin{definition}\label{def:opposite_category}\mcite[def. 1.1.9]{Leinster2016Basic}
  The \term{opposite} category of \( \cat{C} \) is obtained by \enquote{reversing} all arrows. This reversing is merely a relabeling of the domain and codomain --- the underlying morphisms are the same. This concept is quite powerful because it allows performing constructions and proofs by duality --- see \fullref{thm:categorical_principle_of_duality}.

  Formally, the category \( \cat{C}^{\opcat} \) is defined as follows:
  \begin{itemize}
    \item The \hyperref[def:category/objects]{set of objects} \( \obj(\cat{C}^{\opcat}) \) is the set of objects \( \obj(\cat{C}) \) of \( \cat{C} \).

    \item The \hyperref[def:category/morphisms]{set of morphisms} \( \cat{C}^{\opcat}(A, B) \) is the set \( \cat{C}(B, A) \). Thus, any morphism \( f^{\opcat}: A \to B \) in the opposite category \( \cat{C}^{\opcat} \) is a morphism \( f: B \to A \) in \( \cat{C}^{\opcat} \).

    The superscript here is used solely to distinguish between \( f \) being regarded as a morphism of \( \cat{C} \) and of \( \cat{C}^{\opcat} \) --- the morphisms in \( \cat{C} \) are exactly those of \( \cat{C}^{\opcat} \), simply relabeled.

    \item The \hyperref[def:category/composition]{composition of the morphisms}
    \begin{align*}
      f^{\opcat} &\in \cat{C}^{\opcat}(A, B) = \cat{C}(B, A) \\
      g^{\opcat} &\in \cat{C}^{\opcat}(B, C) = \cat{C}(C, B)
    \end{align*}
    is the morphism
    \begin{equation*}
      \underbrace{g^{\opcat} \bincirc f^{\opcat}}_{\cat{C}^{\opcat}(A, C)} \coloneqq \underbrace{f \bincirc g}_{\cat{C}(C, A)}.
    \end{equation*}

    \item The \hyperref[def:category/identity]{identity morphism} on the object \( A \in \cat{C} \) is again \( \id_A \).
  \end{itemize}
\end{definition}

\begin{remark}\label{rem:double_opposite_category}
  The double-opposite of a category or morphism is obviously the original. This is made precise with the oppositization functor defined in \fullref{def:opposite_functor}.
\end{remark}

\begin{example}\label{ex:def:opposite_category}
  A morphism \( f^{\opcat}: A \to B \) in the category \( \cat{Set}^{\opcat} \) is a function from the set \( B \) to the set \( A \). We cannot apply \( f \) to a point in \( B \) unless \( B \subseteq A \). Thus, we cannot regard, in general, the morphism \( f^{\opcat} \) as a function, although only the \hyperref[def:multi_valued_function]{signature} of \( f \) is different from that of \( f^{\opcat} \) --- their \hyperref[def:multi_valued_function/graph]{graphs} are the same.
\end{example}

\begin{proposition}\label{thm:categorical_principle_of_duality}
  We can extend the principle of duality for preordered sets discussed in \fullref{def:preordered_set/duality} to categories. Since we have defined categories in \hyperref[def:axiom_of_universes]{\logic{ZFC+U}} rather than as a first-order theory, we will state this principle informally:
  \begin{displayquote}
    If a statement holds for every category, its dual statement obtained by \enquote{reversing} all morphisms as in \fullref{def:opposite_category}, also holds for every category.
  \end{displayquote}

  See \fullref{thm:def:morphism_invertibility/split_epimorphism} for how this principle can be utilized easily.

  We list here results that heavily utilize this principle. Note that it is now always obvious what exactly needs to reversed in order for this principle to hold. For example, as discussed in \fullref{def:opposite_functor}, for opposite functors we have
  \begin{equation*}
    [F \bincirc G]^{\opcat} = F^{\opcat} \bincirc G^{\opcat},
  \end{equation*}
  which is somewhat unexpected.

  \begin{thmenum}
    \thmitem{thm:categorical_principle_of_duality/morphism_invertibility} \Fullref{thm:morphism_invertibility_duality}: A morphism \( f: A \to B \) in \( \cat{C} \) is a \hyperref[def:morphism_invertibility/left_invertible]{(split)} \hyperref[def:morphism_invertibility/left_cancellative]{monomorphism} if and only if \( f^{\opcat}: B \to A \) in the opposite category \( \cat{C}^{\opcat} \) is a \hyperref[def:morphism_invertibility/right_invertible]{(split)} \hyperref[def:morphism_invertibility/right_cancellative]{epimorphism}.

    In particular, \( f \) is an \hyperref[def:morphism_invertibility/isomorphism]{isomorphism} in \( \cat{C} \) if and only if \( f^{\opcat} \) is an isomorphism in \( \cat{C}^{\opcat} \).

    \thmitem{thm:categorical_principle_of_duality/universal_objects} \Fullref{thm:universal_object_duality}: An object is \hyperref[def:universal_objects/initial]{initial} if and only if it is a \hyperref[def:universal_objects/terminal]{terminal object} the \hyperref[def:opposite_category]{opposite category}.

    \thmitem{thm:categorical_principle_of_duality/functor_categories} \Fullref{thm:opposite_of_functor_category}: For the \hyperref[def:opposite_category]{opposite} of the \hyperref[def:functor_category]{functor category} \( [\cat{C}, \cat{D}] \) we have
    \begin{equation*}
      [\cat{C}, \cat{D}]^{\opcat} = [\cat{C}^{\opcat}, \cat{D}^{\opcat}].
    \end{equation*}

    \thmitem{thm:categorical_principle_of_duality/equivalences} \Fullref{thm:opposite_of_category_equivalence}: The \hyperref[def:opposite_category]{duals} of \hyperref[def:category_equivalence]{equivalent categories} are equivalent.

    \thmitem{thm:categorical_principle_of_duality/adjunctions} \Fullref{thm:category_adjunction_duality}: The functor \( F \) is \hyperref[def:category_adjunction]{left adjoint} to \( G \) if and only if the \hyperref[def:opposite_functor]{dual functor} \( F^{\opcat} \) is right adjoint to \( G^{\opcat} \).

    \thmitem{thm:categorical_principle_of_duality/limits} \Fullref{thm:categorical_limit_duality}: For every \hyperref[def:category_of_cones/cone]{cone} \( (A, \alpha) \) of the \hyperref[def:categorical_diagram]{diagram} \( D \) in \( \cat{C} \), \( (A, \alpha^{\opcat}) \) is a \hyperref[def:category_of_cones/cone]{cocone} of \( D^{\opcat} \) in \( \cat{C}^{\opcat} \).

    Even more, for every \hyperref[def:category_of_cones/limit]{limit} \( (L, \pi) \) of \( D \) in \( \cat{C} \), \( (L, \pi^{\opcat}) \) is a \hyperref[def:category_of_cones/colimit]{colimit} of \( D^{\opcat} \) in \( \cat{C}^{\opcat} \).
  \end{thmenum}
\end{proposition}

\begin{definition}\label{def:morphism_invertibility}
  In connection with \fullref{def:function_invertibility} and \fullref{def:first_order_homomorphism_invertibility}, we introduce the following terminology:
  \begin{thmenum}
    \thmitem{def:morphism_invertibility/left_cancellative} The morphism \( g: B \to C \) is \term{left-cancellative} if, for any pair of morphisms \( f_1, f_2: A \to B \), the equality \( g \bincirc f_1 = g \bincirc f_2 \) implies \( f_1 = f_2 \).

    Left-cancellative morphisms are also called \term{monic morphisms} or \term{monomorphisms}.

    \thmitem{def:morphism_invertibility/left_invertible} The morphism \( f: A \to B \) is \term{left-invertible} if there exists a morphism \( g: B \to A \) such that \( g \bincirc f = \id_A \). We call \( g \) a \term{left inverse} of \( f \).

    Using forward references to \fullref{def:categorical_diagram}, we can restate this condition by saying that the following diagram commutes:
    \begin{equation}\label{eq:def:morphism_invertibility/left_invertible}
      \begin{aligned}
        \includegraphics[page=1]{output/def__morphism_invertibility.pdf}
      \end{aligned}
    \end{equation}

    Left-invertible morphisms are sometimes called \term{split monomorphisms} because they \enquote{split} the identity \( \id_A \) into a composition of \( f \) and \( g \).

    \thmitem{def:morphism_invertibility/right_cancellative} \hyperref[thm:categorical_principle_of_duality]{Dually}, the morphism \( f: A \to B \) is \term{right-cancellative} if, for any pair of morphisms \( g_1, g_2: B \to C \), the equality \( g_1 \bincirc f = g_2 \bincirc f \) implies \( g_1 = g_2 \).

    Right-cancellative morphisms are also called \term{epic morphisms} or \term{epimorphisms}.

    \thmitem{def:morphism_invertibility/right_invertible} The morphism \( g: B \to A \) is \term{right-invertible} if there exists a morphism \( f: A \to B \) such that \( f \bincirc g = \id_B \). We call \( g \) a \term{right inverse} of \( f \).

    Using forward references to \fullref{def:categorical_diagram}, we can restate this condition by saying that the following diagram commutes:
    \begin{equation}\label{eq:def:morphism_invertibility/right_invertible}
      \begin{aligned}
        \includegraphics[page=2]{output/def__morphism_invertibility.pdf}
      \end{aligned}
    \end{equation}

    Right-invertible morphisms are sometimes called \term{split epimorphisms} because they \enquote{split} the identity \( \id_B \) into a composition of \( g \) and \( f \).

    \thmitem{def:morphism_invertibility/isomorphism} The morphism \( f: A \to B \) is \term{fully invertible} it is both left-invertible and right-invertible. By \fullref{thm:def:morphism_invertibility/left_and_right}, in this case, there exists a unique morphism \( f^{-1}: B \to A \) that is a \term{two-sided inverse}, i.e. it is both a left inverse and a right inverse.

    A fully invertible morphism is usually called an \term{isomorphism}. If there exists an isomorphism between \( A \) and \( B \), we say that they are \term{isomorphic} and write \( A \cong B \).

    \thmitem{def:morphism_invertibility/endomorphism} A morphism \( f: A \to A \) from an object to itself is called an \term{endomorphism}.

    \thmitem{def:morphism_invertibility/automorphism} A morphism that is both an endomorphism and an isomorphism is called an \term{automorphism}.
  \end{thmenum}
\end{definition}

\begin{example}\label{ex:def:morphism_invertibility}
  \Fullref{thm:function_invertibility_categorical} characterizes the cancellative and invertible morphisms defined in \fullref{def:morphism_invertibility} for \hyperref[def:category_of_small_sets]{\( \cat{Set} \)} in terms of \hyperref[def:function_invertibility/injective]{injectivity} and \hyperref[def:function_invertibility/injective]{surjectivity}.

  A very simple example of a monomorphism which does not split is the empty function with nonempty domain. These are discussed in \fullref{thm:function_invertibility_categorical/empty}.

  \Fullref{thm:surjective_functions_are_right_invertible} is important enough to have a categorical interpretation via \fullref{thm:epimorphisms_split_in_set}, where its relation to the \hyperref[def:zfc/choice]{axiom of choice} is also discussed.
\end{example}

\begin{proposition}\label{thm:morphism_invertibility_duality}
  A morphism \( f: A \to B \) in \( \cat{C} \) is a \hyperref[def:morphism_invertibility/left_invertible]{(split)} \hyperref[def:morphism_invertibility/left_cancellative]{monomorphism} if and only if \( f^{\opcat}: B \to A \) in the opposite category \( \cat{C}^{\opcat} \) is a \hyperref[def:morphism_invertibility/right_invertible]{(split)} \hyperref[def:morphism_invertibility/right_cancellative]{epimorphism}.

  In particular, \( f \) is an \hyperref[def:morphism_invertibility/isomorphism]{isomorphism} in \( \cat{C} \) if and only if \( f^{\opcat} \) is an isomorphism in \( \cat{C}^{\opcat} \).

  This is part of the duality principles listed in \fullref{thm:categorical_principle_of_duality}.
\end{proposition}
\begin{proof}
  Trivial.
\end{proof}

\begin{proposition}\label{thm:def:morphism_invertibility}
  Morphisms have the following basic properties regarding their \hyperref[def:morphism_invertibility]{invertibility} (compare to \fullref{thm:function_composition_invertibility}):

  \begin{thmenum}
    \thmitem{thm:def:morphism_invertibility/split_monomorphism} Any \hyperref[def:morphism_invertibility/left_invertible]{left-invertible morphism} is \hyperref[def:morphism_invertibility/left_cancellative]{left-cancellative}.

    In more categorical terms, every split monomorphism is a monomorphism.

    \thmitem{thm:def:morphism_invertibility/split_epimorphism} Any \hyperref[def:morphism_invertibility/right_invertible]{right-invertible morphism} is \hyperref[def:morphism_invertibility/right_cancellative]{right-cancellative}.

    In more categorical terms, every split epimorphism is an epimorphism.

    \thmitem{thm:def:morphism_invertibility/at_most_one_inverse}\mcite[exer. 1.1.13]{Leinster2016Basic} Any morphism has at most one two-sided inverse.

    \thmitem{thm:def:morphism_invertibility/left_and_right} If a morphism is both left-invertible and right-invertible, the two inverses are equal, and the morphism is fully invertible.

    \thmitem{thm:def:morphism_invertibility/inverse_interchanges} The morphism \( f: A \to B \) is a right inverse of \( g: B \to A \) if and only if \( g \) is a left inverse of \( f \).

    \thmitem{thm:def:morphism_invertibility/monomorphism_and_split_epimorphism} If a morphism left-cancellative and right-invertible, it is an isomorphism.

    \thmitem{thm:def:morphism_invertibility/split_monomorphism_and_epimorphism} If a morphism left-invertible and right-cancellative, it is an isomorphism.

    \thmitem{thm:def:morphism_invertibility/cancellative_composition} The composition of two monomorphisms (resp. epimorphisms) is again a monomorphism (resp. epimorphism).

    \thmitem{thm:def:morphism_invertibility/invertible_composition} The composition of two split monomorphisms (resp. epimorphisms) is again a split monomorphism (resp. epimorphism).
  \end{thmenum}
\end{proposition}
\begin{proof}
  \SubProofOf{thm:def:morphism_invertibility/split_monomorphism} Suppose that \( g: B \to C \) is left-invertible with inverse \( h: C \to B \). Suppose that \( f_1, f_2: A \to B \) are morphisms such that
  \begin{equation*}
    g \bincirc f_1 = g \bincirc f_2.
  \end{equation*}

  Then
  \begin{equation*}
    f_1
    \reloset {\eqref{eq:def:category/C1}} =
    \id_B \bincirc f_1
    =
    (h \bincirc g) \bincirc f_1
    \reloset {\eqref{eq:def:category/C2}} =
    h \bincirc (g \bincirc f_1)
    =
    h \bincirc (g \bincirc f_2)
    =
    \cdots
    =
    f_2.
  \end{equation*}

  \SubProofOf{thm:def:morphism_invertibility/split_epimorphism} This is an exemplar proof using duality. By \fullref{thm:morphism_invertibility_duality}, every split epimorphism \( f: A \to B \) in \( \cat{C} \) is a split monomorphism in \( \cat{C}^{\opcat} \). By \fullref{thm:def:morphism_invertibility/split_monomorphism}, \( f^{\opcat} \) is a monomorphism. Then gain by \fullref{thm:morphism_invertibility_duality}, \( f \) is an epimorphism.

  \SubProofOf{thm:def:morphism_invertibility/at_most_one_inverse} If \( f: A \to B \) has no inverse, it vacuously has at most one inverse.

  Now assume that \( f: A \to B \) has two inverses \( g_1: B \to A \) and \( g_2: B \to A \):
  \begin{align*}
    g_1 \bincirc f = \id_A &&& f \bincirc g_1 = \id_B, \\
    g_2 \bincirc f = \id_A &&& f \bincirc g_2 = \id_B.
  \end{align*}

  Then
  \begin{equation*}
    g_1
    \reloset {\eqref{eq:def:category/C1}} =
    g_1 \bincirc \id_B
    =
    g_1 \bincirc (f \bincirc g_2)
    \reloset {\eqref{eq:def:category/C2}} =
    (g_1 \bincirc f) \bincirc g_2
    =
    \id_A \bincirc g_2
    \reloset {\eqref{eq:def:category/C1}} =
    g_2.
  \end{equation*}

  \SubProofOf{thm:def:morphism_invertibility/left_and_right} Suppose that \( f: A \to B \) has a left-inverse \( l: B \to A \) and a right-inverse \( r: B \to A \). Then

  \SubProofOf{thm:def:morphism_invertibility/inverse_interchanges} Trivial.

  \SubProofOf{thm:def:morphism_invertibility/monomorphism_and_split_epimorphism} Let \( g: B \to A \) be left-cancellative and right-invertible. Let \( f: A \to B \) be a right inverse of \( g \). Then
  \begin{equation*}
    f
    =
    \reloset {\eqref{eq:def:category/C1}} =
    f \bincirc \id_A
    =
    f \bincirc (g \bincirc f)
    =
    (f \bincirc g) \bincirc f.
  \end{equation*}

  Because \( g \) is a left inverse of \( f \), from \fullref{thm:def:morphism_invertibility/split_monomorphism} it follows that \( f \) is left-cancellative. Since we have
  \begin{equation*}
    \id_B \bincirc f
    =
    (f \bincirc g) \bincirc f,
  \end{equation*}
  it follows that \( f \bincirc g = \id_B \).

  Therefore, \( f \) is a left inverse of \( g \) and hence an isomorphism.

  \SubProofOf{thm:def:morphism_invertibility/split_monomorphism_and_epimorphism} The proof is analogous to \fullref{thm:def:morphism_invertibility/monomorphism_and_split_epimorphism}.

  \SubProofOf{thm:def:morphism_invertibility/cancellative_composition} Let \( g: B \to C \) and \( h: C \to D \) be monomorphisms (left-cancellative).

  Let \( f_1, f_2: A \to B \) be two arbitrary morphisms with codomain \( B \). Suppose that
  \begin{equation*}
    (h \bincirc g) \bincirc f_1 = (h \bincirc g) \bincirc f_2.
  \end{equation*}

  Then, by \ref{def:category/C2},
  \begin{equation*}
    h \bincirc (g \bincirc f_1) = h \bincirc (g \bincirc f_2).
  \end{equation*}

  Since \( h \) is left-cancellative, it follows that
  \begin{equation*}
    g \bincirc f_1 = g \bincirc f_2.
  \end{equation*}

  Since \( g \) is also left-cancellative, \( f_1 = f_2 \).

  Therefore, \( h \bincirc g \) is a monomorphism.

  The proof for composition of epimorphisms is identical.

  \SubProofOf{thm:def:morphism_invertibility/invertible_composition} Let \( f: A \to B \) and \( g: B \to C \) be split monomorphisms (left-invertible).

  Then there exist left inverses \( l_f: B \to A \) and \( l_g: C \to B \) of \( f \) and \( g \), respectively. We have
  \begin{equation*}
    (l_f \bincirc l_g) \bincirc (g \bincirc f)
    \reloset {\eqref{eq:def:category/C2}} =
    l_f \bincirc (l_g \bincirc g) \bincirc f
    =
    l_f \bincirc \id_B \bincirc f
    \reloset {\eqref{eq:def:category/C1}} =
    l_f \bincirc f
    =
    \id_A.
  \end{equation*}

  Therefore, \( g \bincirc f \) is also left-invertible.

  The proof for composition of split epimorphisms is identical.
\end{proof}

\begin{theorem}[Epimorphisms split in Set]\label{thm:epimorphisms_split_in_set}
  Every \hyperref[def:morphism_invertibility/right_cancellative]{epimorphism} in \hyperref[def:category_of_small_sets]{\( \cat{Set} \)} splits. That is, all epimorphisms in \( \cat{Set} \) are \hyperref[def:morphism_invertibility/right_invertible]{split epimorphisms}.

  Assuming the existence of the \hyperref[def:grothendieck_universe]{Grothendieck universe} containing \( \cat{Set} \), in \hyperref[def:zfc]{\logic{ZF}} this theorem is equivalent to the \hyperref[def:zfc/choice]{axiom of choice} --- see \fullref{thm:axiom_of_choice_equivalences/epimorphisms}.

  Since not every epimorphism splits in a general category, this theorem is sometimes considered to be a categorical statement of the axiom of choice, which holds in some categories but not in others.
\end{theorem}
\begin{proof}
  By \fullref{thm:function_invertibility_categorical/right_cancellative}, a function is an epimorphism if and only if it is surjective. Thus, the theorem is equivalent to \fullref{thm:surjective_functions_are_right_invertible}.
\end{proof}

\begin{definition}\label{def:universal_objects}\mcite[def. 2.1.7]{Leinster2016Basic}
  Fix a category \( \cat{C} \).

  \begin{thmenum}
    \thmitem{def:universal_objects/initial} We call the object \( I \in \cat{C} \) an \term{initial object} if for any other object \( A \in \cat{C} \) there exists a unique morphism \( f: I \to A \),

    \thmitem{def:universal_objects/terminal} \hyperref[thm:categorical_principle_of_duality]{Dually}, we call the object \( T \in \cat{C} \) a \term{terminal object} or \term{final object} if for any other object \( A \in \cat{C} \) there exists a unique morphism \( f: A \to T \).

    The initial and terminal objects are collectively called \term{universal objects}.

    \thmitem{def:universal_objects/zero}\mcite[20]{MacLane1994} If \( Z \) is both an initial and a terminal object, we say that \( Z \) is a \term{zero object}.
  \end{thmenum}
\end{definition}

\begin{example}\label{ex:def:universal_objects}
  \begin{thmenum}
    \thmitem{ex:def:universal_objects/set} In the category \hyperref[def:category_of_small_sets]{\( \cat{Set} \)} of small sets, for any set \( A \) there is a unique \hyperref[def:multi_valued_function/empty]{empty function} from \( \varnothing \) to \( A \). Therefore, \( \varnothing \) is an \hyperref[def:universal_objects/initial]{initial object} in \( \cat{Set} \).

    For any set \( A \), there is a unique function that contracts any set \( B \) to \( \set{ A } \). Therefore, every singleton set is a \hyperref[def:universal_objects/terminal]{final object} in \( \cat{Set} \).

    We often denote the initial and terminal objects in \( \cat{Set} \) by \( 0 \) and \( 1 \) respectively, which corresponds to their definition as \hyperref[def:ordinal]{ordinals}.

    By \fullref{thm:def:universal_objects/no_zero}, \( \cat{Set} \) has no zero object.

    \thmitem{ex:def:universal_objects/grp} In the category \hyperref[def:group/category]{\( \cat{Grp} \)} of small groups, the \hyperref[def:group/trivial]{trivial group} is a \hyperref[def:universal_objects/zero]{zero object}. This holds more generally for pointed sets rather than groups.

    Indeed, it can be embedded into any other group and any group can be contracted into the corresponding trivial group. Furthermore, all trivial groups are isomorphic.
  \end{thmenum}
\end{example}

\begin{proposition}\label{thm:universal_object_duality}
  An object is \hyperref[def:universal_objects/initial]{initial} if and only if it is a \hyperref[def:universal_objects/terminal]{terminal object} the opposite category.

  This is part of the duality principles listed in \fullref{thm:categorical_principle_of_duality}.
\end{proposition}
\begin{proof}
  Trivial.
\end{proof}

\begin{proposition}\label{thm:def:universal_objects}
  \hfill
  \begin{thmenum}
    \thmitem{thm:def:universal_objects/initial} An \hyperref[def:universal_objects/initial]{initial object} is unique up to an isomorphism.
    \thmitem{thm:def:universal_objects/terminal} \hyperref[thm:categorical_principle_of_duality]{Dually}, a \hyperref[def:universal_objects/initial]{terminal object} is also unique up to an isomorphism.
    \thmitem{thm:def:universal_objects/zero} If a category has an initial and a terminal object and if they are isomorphic, then both are zero objects.

    In particular, a zero object is unique up to an isomorphism.

    \thmitem{thm:def:universal_objects/no_zero} If an initial and a terminal object exists and are not isomorphic, then there exist no zero objects.
  \end{thmenum}
\end{proposition}
\begin{proof}
  \SubProofOf{thm:def:universal_objects/initial} Suppose that \( A \) and \( B \) are both initial objects in \( \cat{C} \). Then there exist morphisms \( f: A \to B \) and \( g: B \to A \). Their composition \( g \bincirc f \) is an \hyperref[def:morphism_invertibility/endomorphism]{endomorphism} on \( A \).

  But there exists a unique \hyperref[def:morphism_invertibility/endomorphism]{endomorphism} on \( A \), which must be the identity \( \id_A \). Thus, \( g \bincirc f = \id_A \) and \( g \) is a left inverse of \( f \).

  We can analogously show that \( g \) is a right inverse of \( f \). Therefore, \( f \) is fully invertible, and \( A \) and \( B \) are isomorphic.

  \SubProofOf{thm:def:universal_objects/terminal} If \( T' \) and \( T^\dprime \) are terminal objects in \( \cat{C} \), by \fullref{thm:universal_object_duality}, they are initial objects in \( \cat{C}^{\opcat} \). By \fullref{thm:def:universal_objects/initial}, they are isomorphic in \( \cat{C}^{\opcat} \) and by \fullref{thm:morphism_invertibility_duality}, they are isomorphic in \( \cat{C} \).

  \SubProofOf{thm:def:universal_objects/zero} Suppose that \( A \) is an initial object and that \( B \) is a final object in \( \cat{C} \). Let \( f: A \to B \) be an isomorphism between them.

  Let \( C \in \cat{C} \) be any other object and let \( g: C \to B \) be the unique morphism to \( B \). Then \( f^{-1} \bincirc g: C \to A \) is a morphism from \( C \) to \( A \). The inverse \( f^{-1}: B \to A \) is unique by \fullref{thm:def:morphism_invertibility/at_most_one_inverse}, therefore its composition with \( g: C \to B \) is also unique. Hence, any object has a unique morphism to \( A \). This makes \( A \) a terminal object and thus a zero object.

  We can analogously show that \( B \) is a zero object.

  \SubProofOf{thm:def:universal_objects/no_zero} By \fullref{thm:def:universal_objects/zero}, all zero objects are isomorphic. By \fullref{thm:def:universal_objects/initial}, all initial objects are isomorphic and analogously for terminal objects. Hence, if a zero object exists, all initial objects are isomorphic to all terminal objects.

  If some initial object is not isomorphic to some terminal object, then by contraposition it follows that no zero object exists.
\end{proof}

  \subsection{Functors}\label{subsec:functors}

\begin{definition}\label{def:functor}\mcite[def. 1.2.1; def. 1.2.10]{Leinster2014BasicCategories}
  Fix some \hyperref[def:category]{categories} \( \cat{C} \) and \( \cat{D} \). A \term{functor} \( F: \cat{C} \to \cat{D} \) is a \hyperref[def:directed_multigraph/homomorphism]{directed multigraph homomorphism} between the underlying multigraphs that is compatible with composition and identities.

  Explicitly, a functor is a family of functions
  \begin{equation}\label{eq:def:functor_as_family_of_function}
    \begin{aligned}
      F_{\obj}:       &\obj(\cat{C}) \to \obj(\cat{D}) \\
      F_{\hom(A, B)}: &\cat{C}(A, B) \to \cat{D}(F_{\obj}(A), F_{\obj}(B)),
    \end{aligned}
  \end{equation}
  where \( F_{\hom(A, B)} \) is a distinct function for every pair of objects \( A \) and \( B \).

  In practice, we usually define the functor as the set
  \begin{equation}\label{eq:def:functor_as_single_function}
    F \coloneqq F_{\obj} \cup \bigcup\set{ F_{\hom(A, B)} \given A, B \in \obj(\cat{C}) }
  \end{equation}

  Since the domains of all constituent functions are disjoint, \( F \) is again a total single-valued function. This allows us to justify the notation \( F(A) \) for objects and \( F(f) \) for morphisms.

  \begin{thmenum}[resume=def:functor]
    \thmitem{def:functor/domain_and_codomain} We say that the category \( \cat{C} \) is the \term{domain} and \( \cat{D} \) --- the \term{codomain} of the functor \( F \). These are technically not the domain and codomain of \( F \) when regarded as a function, however it is consistent with \fullref{def:category_of_small_categories}.

    \thmitem{def:functor/endofunctor} Similarly to \fullref{def:function/endofunction} for functions, if the domain \( \cat{C} \) and codomain \( \cat{D} \) of a functor coincide, we say that it is an \term{endofunctor}.
  \end{thmenum}

  The definition of a functor additionally requires the following compatibility conditions to hold:
  \begin{thmenum}[series=def:functor]
    \thmitem[def:functor/CF1]{CF1} Functors must preserve identities, meaning that for any object \( A \in \cat{C} \) the following equality must hold:
    \begin{equation}\label{eq:def:functor/CF1}\tag{\logic{CF1}}
      F(\id_A) = \id_{F(A)}.
    \end{equation}

    \thmitem[def:functor/CF2]{CF2} Functors must preserve composition, meaning that for any pair of morphism \( f: A \to B \) and \( g: B \to C \) in \( \cat{C} \),
    \begin{equation}\label{eq:def:functor/CF2}\tag{\logic{CF2}}
      F(g \bincirc f) = F(g) \bincirc F(f).
    \end{equation}
  \end{thmenum}
\end{definition}
\begin{defproof}
  The definition \eqref{eq:def:functor_as_family_of_function} ensures that the directed multigraph homomorphism conditions \eqref{eq:def:directed_multigraph/homomorphism/head} and \eqref{eq:def:directed_multigraph/homomorphism/tail} hold.

  Indeed, for any morphism \( f: A \to B \) in \( \cat{C} \) we have
  \begin{equation*}
    F(\dom(f)) = F(A) = \dom(F(f)),
  \end{equation*}
  which implies \eqref{eq:def:directed_multigraph/homomorphism/head}. We also have
  \begin{equation*}
    F(\co\dom(f)) = F(B) = \co\dom(F(f)),
  \end{equation*}
  which implies \eqref{eq:def:directed_multigraph/homomorphism/tail}.
\end{defproof}

\begin{remark}\label{rem:functor_size}
  It is possible that \( \cat{C} \) is \( \mscrU \)-small in the sense of \fullref{def:category_size}, but the \hyperref[def:functor]{functor} \( F \), as the set \eqref{eq:def:functor_as_single_function}, is not \( \mscrU \)-small in the sense of \fullref{def:large_and_small_sets}. Without using universes, we cannot prove the existence of any functor from the category of smalls sets to itself, for example.
\end{remark}

\begin{example}\label{ex:unary_functors_in_set}
  In \fullref{def:basic_set_operations}, we defined some operations on the category \hyperref[def:category_of_small_sets]{\( \cat{Set} \)} of small sets.

  \begin{thmenum}
    \thmitem{ex:unary_functors_in_set/power} The \hyperref[def:basic_set_operations/power_set]{power set} \( \pow: \cat{Set} \to \cat{Set} \) is a canonical example of an \hyperref[def:functor/endofunctor]{endofunctor}. Explicitly:
    \begin{equation*}
      \begin{aligned}
        &\pow: \cat{Set} \to \cat{Set}, \\
        &\pow(A) \coloneqq \set{ S \given S \subseteq A }, \\
        &\pow(f: A \to B) \coloneqq (S \mapsto f[S]). \\
      \end{aligned}
    \end{equation*}

    We must verify that it is indeed a functor. \ref{def:functor/CF2} is satisfied because
    \begin{equation*}
      \pow(g) \bincirc \pow(f) = (S \mapsto g[f[S]]) = \pow(g \bincirc f).
    \end{equation*}

    The condition \ref{def:functor/CF2} is also obviously satisfied.

    The nuance here is that we send every function \( f: A \to B \) to its \hyperref[def:set_valued_map/image]{set value} \( f[S] \) of some subset \( S \) of \( A \).

    \thmitem{ex:unary_functors_in_set/union} The \hyperref[def:basic_set_operations/union]{union} \( \bigcup \) and \hyperref[def:basic_set_operations/intersection]{intersection} \( \bigcap \) may seem to be good examples of endofunctors in \( \cat{Set} \). Unfortunately, there is no natural way to extend a morphism (function) \( f: A \to B \) to a morphism from \( \bigcup A \) to \( \bigcup B \) or \( \bigcap A \) to \( \bigcap B \).
  \end{thmenum}
\end{example}

\begin{definition}\label{def:subcategory}\mcite[def. 1.2.18]{Leinster2014BasicCategories}
  We call the category \( \cat{D} \) a \term{subcategory} of \( \cat{C} \) if the following hold:
  \begin{itemize}
    \item The underlying multigraph \( U(\cat{D}) \) is a \hyperref[def:directed_multigraph/subgraph]{subgraph} of \( U(\cat{C}) \). That is, every object in \( \cat{D} \) is an object in \( \cat{C} \) and every morphism in \( \cat{D} \) is a morphism in \( \cat{C} \).
    \item Composition and identity in \( \cat{D} \) are \hyperref[def:set_valued_map/restriction]{restrictions} of composition and identity in \( \cat{C} \).
  \end{itemize}

  \begin{thmenum}
    \thmitem{def:subcategory/inclusion} For every subcategory there exists an \term{inclusion functor} \( \Iota: \cat{D} \to \cat{C} \), which sends every object and morphism of \( \cat{D} \) to itself in \( \cat{C} \).

    \thmitem{def:subcategory/full} We say that \( \cat{D} \) is a \term{full subcategory} if the underlying multigraph \( U(\cat{C}) \) is a \hyperref[def:induced_subgraph]{induced subgraph}. That is, in case \( \cat{D}(A, B) = \cat{C}(A, B) \) for every pair of objects \( A \) and \( B \) of \( \cat{D} \).

    By \fullref{thm:def:functor_invertibility/full_subcategory}, this is equivalent to the inclusion functor being \hyperref[def:functor_invertibility/full]{full}.

    \thmitem{def:subcategory/induced} Every \hyperref[rem:family_of_sets]{family} \( \mscrD \) of objects in \( \cat{C} \) induces a full subcategory \( \cat{D} \) of \( \cat{C} \), whose objects are those of \( \mscrD \) and whose morphisms are restricted to those whose domain and codomain are both in \( \mscrD \).
  \end{thmenum}
\end{definition}

\begin{remark}\label{rem:contravariant_functor}\mcite[def. 1.2.10]{Leinster2014BasicCategories}
  We can invert the order of composition in \ref{def:functor/CF2} in the definition of a functor given in \fullref{def:functor}.
  \begin{thmenum}
    \thmitem[rem:contravariant_functor/CF2]{CF2\Textprime} We can replace \ref{def:functor/CF2} with
    \begin{equation}\label{eq:rem:contravariant_functor/CF2}\tag{\logic{CF2\Textprime}}
      F(g \bincirc f) = F(f) \bincirc F(g).
    \end{equation}
  \end{thmenum}

  This also requires some other straightforward modifications to the definition of a functor.

  A functor that satisfies \ref{rem:contravariant_functor/CF2} rather than \ref{def:functor/CF2} is called \term{contravariant}. In this context, a functor satisfying \ref{def:functor/CF2} is called \term{covariant}.

  Fortunately, a contravariant functor from \( \cat{C}^\oppos \) to \( \cat{D} \) is identical to a covariant functor from \( \cat{C} \) to \( \cat{D} \). Therefore, there is no formal difference between the two concepts.

  The usage of the terms are entirely dictated by context. Unless necessary, we will avoid speaking about contravariant functors to avoid confusion. Some examples where this terminology may be useful are \fullref{def:opposite_functor}, \fullref{def:hom_functor/unary} and \fullref{ex:dual_space_contravariant_functor}.
\end{remark}

\begin{example}\label{ex:dual_space_contravariant_functor}
  We can try to na\"ively define a functor that assigns to a \hyperref[def:vector_space]{vector space} its \hyperref[def:dual_vector_space]{algebraic dual}:
  \begin{equation*}
    \begin{aligned}
      &F: \cat{Vect_\BbbK} \to \cat{Vect_\BbbK}, \\
      &F(V) \coloneqq V^*, \\
      &F(f: V \to W) \coloneqq (\varphi: W \to \BbbK \mapsto \varphi \bincirc f).
    \end{aligned}
  \end{equation*}

  Unfortunately, \( F(f) \) is supposed to be a morphism from \( V^* \) to \( W^* \), but is actually a morphism from \( W^* \) to \( V^* \). This makes \( F \) a \hyperref[rem:contravariant_functor]{contravariant functor} or, equivalently, a functor from \( \cat{Vect_\BbbK}^\oppos \) to \( \cat{Vect_\BbbK} \).
\end{example}

\begin{definition}\label{def:discrete_category}
  A \term{discrete category} is a category with no morphisms except for the identities. Clearly to any set there corresponds exactly one discrete category and vice versa.
\end{definition}

\begin{example}\label{ex:discrete_category_adjunction}
  Denote by
  \begin{equation*}
    U: \cat{Cat} \to \cat{Set}
  \end{equation*}
  the forgetful functor that for any small category \( \cat{C} \) gives us its set of objects \( \obj(\cat{C}) \). There is also a functor
  \begin{equation*}
    D: \cat{Set} \to \cat{Cat}
  \end{equation*}
  that for any small set \( A \) gives us the \hyperref[def:discrete_category]{discrete category} whose set of objects is \( A \).

  This is actually an \hyperref[def:category_adjunction]{adjunction} --- see \fullref{ex:def:category_adjunction/set_cat}.
\end{example}

\begin{proposition}\label{thm:def:functor}
  \hyperref[def:functor]{Functors} have the following basic properties:
  \begin{thmenum}
    \thmitem{thm:def:functor/half_inverses} Functors preserve inverses. For every functor \( F: \cat{C} \to \cat{D} \) and every morphism \( f: A \to B \) in \( \cat{C} \) with a right inverse \( g: B \to A \), \( F(f) \) is a right inverse of \( F(g) \). Similarly, if \( g \) is a left inverse of \( f \), then \( F(g) \) is a left inverse of \( F(f) \).

    \thmitem{thm:def:functor/inverses} For every functor \( F: \cat{C} \to \cat{D} \) and every isomorphism \( f: A \to B \) in \( \cat{C} \),
    \begin{equation}\label{eq:thm:def:functor/inverses}
      [F(f)]^{-1} = F(f^{-1}).
    \end{equation}

    \thmitem{thm:def:functor/isomorphisms} Functors preserve \hyperref[def:morphism_invertibility/isomorphism]{isomorphisms}. That is, for every functor \( F: \cat{C} \to \cat{D} \), if \( f: A \to B \) is an isomorphism in \( \cat{C} \), \( F(f) \) is an isomorphism in \( \cat{D} \).

    Consequently, for every pair of objects \( A \) and \( B \) in \( \cat{C} \), from \( A \cong B \) it follows that \( F(A) \cong F(B) \).

    The converse sometimes also holds --- see \fullref{thm:def:functor_invertibility/fully_faithful_reflects_invertible}.
  \end{thmenum}
\end{proposition}
\begin{proof}
  \SubProofOf{thm:def:functor/half_inverses} Let \( f: A \to B \) be a right inverse of \( g: B \to A \) in \( \cat{C} \). Then
  \begin{equation*}
    F(g) \bincirc F(f)
    \reloset {\eqref{eq:def:functor/CF2}} =
    F(g \bincirc f)
    =
    F(\id_A)
    \reloset {\eqref{eq:def:functor/CF1}} =
    \id_{F(A)}.
  \end{equation*}

  Thus, \( F(f) \) is a right inverse of \( F(g) \). Since \( g \) is a left inverse of \( f \), automatically \( F(g) \) is a left inverse of \( F(f) \).

  \SubProofOf{thm:def:functor/inverses} If \( f^{-1} \) is a left inverse of \( f \), by \fullref{thm:def:functor/half_inverses} we have that \( F(f^{-1}) \) is a left inverse of \( F(f) \). But \( F(f^{-1}) \) is also a right inverse, and again by \fullref{thm:def:functor/half_inverses} \( F(g) \) is a right inverse of \( F(f^{-1}) \).

  Therefore, \( F(f^{-1}) \) is a two-sided inverse of \( F(f) \). By \fullref{thm:def:morphism_invertibility/at_most_one_inverse}, it is the only two-sided inverse, hence
  \begin{equation*}
    [F(f)]^{-1} = F(f^{-1}).
  \end{equation*}

  \SubProofOf{thm:def:functor/isomorphisms} Follows from \fullref{thm:def:functor/inverses}.
\end{proof}

\begin{definition}\label{def:category_of_small_categories}
  Suppose that we are given a \hyperref[def:grothendieck_universe]{Grothendieck universe} \( \mscrU \), which is safe to assume to be the smallest suitable one as explained in \fullref{def:large_and_small_sets}.

  We denote the \hyperref[def:category]{category} of \( \mscrU \)-small \hyperref[def:category]{categories} by \( \ucat{Cat} \) or, if the universe is clear from the context, simply by \( \cat{Cat} \). See \fullref{def:category_size} for a further discussion of universes and categories.

  \begin{itemize}
    \item The \hyperref[def:category/objects]{set of objects} \( \obj(\cat{Cat}) \) is the set of all \( \mscrU \)-small categories.

    \item The \hyperref[def:category/morphisms]{set of morphisms} \( \cat{Cat}(A, B) \) from \( A \) to \( B \) is the set of all \hyperref[def:functor]{functors} from \( A \) to \( B \).

    \item The \hyperref[def:category/composition]{composition of morphisms} is the \hyperref[def:set_valued_map/composition]{function composition} of the functors regarded as the functions \eqref{eq:def:functor_as_single_function}. That is, the composition of \( F: \cat{C} \to \cat{D} \) and \( G: \cat{D} \to \cat{E} \) is the functor
    \begin{equation}\label{eq:def:category_of_small_categories/composition}
      \begin{aligned}
        &[G \bincirc F]: \cat{C} \to \cat{E}, \\
        &[G \bincirc F](A) \coloneqq G(F(A)), \\
        &[G \bincirc F](f) \coloneqq G(F(f)).
      \end{aligned}
    \end{equation}

    \item The \hyperref[def:category/identity]{identity morphism} on the category \( \cat{C} \) is the \term{identity functor}
    \begin{equation}\label{eq:def:category_of_small_categories/identity}
      \begin{aligned}
        &\id_{\cat{C}}: \cat{C} \to \cat{C}, \\
        &\id_{\cat{C}}(A) \coloneqq A, \\
        &\id_{\cat{C}}(f) \coloneqq f.
      \end{aligned}
    \end{equation}
  \end{itemize}
\end{definition}
\begin{defproof}
  To see that \( \ucat{Cat} \) is indeed a category, we verify the conditions \ref{def:category/C1} and \ref{def:category/C2}.

  \SubProofOf{def:category/C1} For every two \( \mscrU \)-small categories \( \cat{C} \) and \( \cat{D} \) and every functor \( F: \cat{C} \to \cat{D} \), for every object \( A \in \cat{C} \) we have
  \begin{equation*}
    [\id_{\cat{D}} \bincirc F](A)
    =
    \id_{\cat{D}}(F(A))
    =
    F(A)
    =
    F(\id_{\cat{C}}(A))
    =
    [F \bincirc \id_{\cat{C}}](A)
  \end{equation*}
  and analogously for morphisms.

  Therefore, \( \id_{\cat{C}} \) and \( \id_{\cat{D}} \) satisfy \eqref{eq:def:category/C1}.

  \SubProofOf{def:category/C2} Associativity of functor composition is inherited from the associativity of function composition.
\end{defproof}

\begin{definition}\label{def:universal_categories}
  For any \hyperref[def:grothendieck_universe]{Grothendieck universe} \( \mscrU \), the \hyperref[def:category_of_small_categories]{category of \( \mscrU \)-small categories} \( \ucat{Cat} \) has an initial and a terminal object.

  Similarly to how we use the \hyperref[def:ordinal]{ordinals} \( 0 \) and \( 1 \) to denote the initial and terminal object in the category of sets, we denote the initial category by \( \cat{0} \) and the final category by \( \cat{1} \). Note that the final category is only unique up to an isomorphism. They are identical, however, for all universes \( \mscrU \).

  These categories are precisely the \hyperref[def:discrete_category]{discrete categories} induced by the ordinals \( 0 \) and \( 1 \) as described in \fullref{thm:order_category_isomorphism}.
\end{definition}

\begin{definition}\label{def:opposite_functor}\mcite[33]{MacLane1998Categories}
  The \term{opposite} functor of \( F: \cat{C} \to \cat{D} \) is the functor
  \begin{equation*}
    \begin{aligned}
      &F^\oppos: \cat{C}^\oppos \to \cat{D}^\oppos \\
      &F^\oppos(A) \coloneqq A \\
      &F^\oppos(f^\oppos: B \to A) \coloneqq [F(f: A \to B)]^\oppos.
    \end{aligned}
  \end{equation*}

  For the composition of functors, we then have
  \begin{equation}\label{eq:def:opposite_functor/composition}
    [G \bincirc F]^\oppos = G^\oppos \bincirc F^\oppos.
  \end{equation}

  This is somewhat in contrast to the general practice of inverting morphisms when taking opposites. Thus, for any \hyperref[def:grothendieck_universe]{Grothendieck universe} \( \mscrU \), we have the \hyperref[rem:contravariant_functor]{contravariant} \term{oppositization functor}
  \begin{equation*}
    (\anon*)^\oppos: \ucat{Cat}^\oppos \to \ucat{Cat}.
  \end{equation*}

  As an \hyperref[def:function/endofunction]{endofunction} on \( \obj(\ucat{Cat}) \), the oppositization functor is clearly an \hyperref[def:morphism_invertibility/involution]{involution}.

  Dual functors also arise naturally in \fullref{thm:opposite_of_functor_category}.
\end{definition}

\begin{definition}\label{def:functor_image}
  The \term{image} of a functor \( F: \cat{C} \to \cat{D} \) is the \hyperref[def:directed_multigraph]{directed multigraph} whose vertex set is
  \begin{equation*}
    V \coloneqq \set{ F(A) \given A \in \cat{C} }
  \end{equation*}
  and whose arc set is
  \begin{equation*}
    A \coloneqq \set{ F(f) \given A, B \in \cat{C} \T{and} f \in \cat{C}(A, B) }.
  \end{equation*}

  This graph has no categorical structure --- it is merely a directed multigraph. As shown in \fullref{ex:functor_image_not_a_category}, imposing a categorical structure na\"ively may fail.
\end{definition}

\begin{example}\label{ex:functor_image_not_a_category}\mcite{MathSE:image_of_functor_is_not_a_category}
  \begin{figure}[!ht]
    \hfill
    \includegraphics[page=1]{output/ex__functor_image_not_a_category}
    \hfill
    \hfill
    \caption{A functor whose image is not a category.}\label{fig:ex:functor_image_not_a_category}
  \end{figure}

  Consider the functor \( F: \cat{C} \to \cat{D} \) from \cref{fig:ex:functor_image_not_a_category}.

  \begin{itemize}
    \item The solid arrows are the morphisms in \( \cat{C} \) and their images in \( F(\cat{C}) \).
    \item The dashed arrows denote the action of the functor \( F \).
    \item The dotted arrow exists in \( \cat{D} \) as the composition of the other two arrows, however it is missing in the image \( F(\cat{C}) \). Thus, composition is not fully defined in \( F(\cat{C}) \), and \( F(\cat{C}) \) fails to be a category.
  \end{itemize}
\end{example}

\begin{definition}\label{def:categorical_diagram}
  Fix a category \( \cat{I} \), called an \term{index category}. A \term{diagram} in \( \cat{C} \) of shape \( \cat{I} \) is simply a functor \( D: \cat{I} \to \cat{C} \), whose domain is \( \cat{I} \). We sometimes identify a diagram functor with its image \( D(\cat{I}) \).

  It is often convenient to draw graphically the \hyperref[def:graph_geometric_realization]{geometric realizations} of the \hyperref[def:directed_multigraph]{multigraph} \( D(\cat{I}) \). An established convention is to allow multiple vertices representing the same object, which can be achieved formally by actually adjoining new vertices to the graph and labeling them as in \fullref{def:labeled_set}. Other established conventions for drawing diagrams include not drawing identity morphisms and adding various visual aids. In this regard, categorical diagrams correspond to the everyday sense of the word \enquote{diagram}.

  We say that the diagram is \( \mscrU \)-\term{small} if its index category is \( \mscrU \)-\hyperref[def:category_size]{small}.

  We say that the diagram \( D \) over \( \cat{C} \) \term{commutes} if, whenever
  \begin{equation*}
    A \reloset {f_1} \to \anon \reloset {f_2} \to \cdots \reloset {f_{n-1}} \to \anon \reloset {f_n} \to B
  \end{equation*}
  and
  \begin{equation*}
    A \reloset {g_1} \to \anon \reloset {g_2} \to \cdots \reloset {g_{n-1}} \to \anon \reloset {g_m} \to B
  \end{equation*}
  are two \hyperref[def:graph_walk/directed]{directed walks} in the graph \( D(\cat{I}) \) with identical endpoints and either \( n > 1 \) or \( m > 1 \), then
  \begin{equation*}
    f_n \bincirc f_{n-1} \bincirc \cdots \bincirc f_2 \bincirc f_1
    =
    g_m \bincirc g_{m-1} \bincirc \cdots \bincirc g_2 \bincirc g_1.
  \end{equation*}

  We do not really care about how the objects and morphisms in \( \cat{I} \) are labeled, hence we often use placeholder dots like in \eqref{eq:ex:directed_multigraphs_as_functors/index/dots}.

  The requirement that the one of the paths is nontrivial, however, is crucial in \fullref{def:equalizers}.
\end{definition}

\begin{remark}\label{rem:inverting_isomorphisms_may_preserve_commutativity}
  Inverting isomorphisms in a \hyperref[def:categorical_diagram]{commutative diagram} may or may not preserve commutativity.

  If \( p = (f_1, \ldots, f_n) \) and \( q = (g_1, \ldots, g_m) \) are two paths in a commutative diagram, and if \( f_1 \) is invertible, then obviously
  \begin{equation*}
    f_n \bincirc \cdots \bincirc f_1 = g_m \bincirc \cdots \bincirc g_1
  \end{equation*}
  if and only if
  \begin{equation*}
    f_n \bincirc \cdots \bincirc f_2 = g_m \bincirc \cdots \bincirc g_1 \bincirc f_1
  \end{equation*}
  and similarly if \( f_n \) is invertible.

  On the other hand, consider \hyperref[def:ordinal]{ordinals} in \( \cat{Set} \). Denote by \( \iota \) the inclusion maps and by \( f: \omega^2 \to \omega \) the bijective map from \fullref{thm:omega_equinumerous_with_omega_squared}. Then the following diagram commutes:
  \begin{equation}\label{eq:rem:inverting_isomorphisms_may_preserve_commutativity/ordinals_commuting}
    \begin{aligned}
      \includegraphics[page=1]{output/rem__inverting_isomorphisms_may_preserve_commutativity}
    \end{aligned}
  \end{equation}
  but the following does not:
  \begin{equation}\label{eq:rem:inverting_isomorphisms_may_preserve_commutativity/ordinals_not_commuting}
    \begin{aligned}
      \includegraphics[page=2]{output/rem__inverting_isomorphisms_may_preserve_commutativity}
    \end{aligned}
  \end{equation}
\end{remark}

\begin{definition}\label{def:functor_invertibility}
  In connection with \fullref{def:morphism_invertibility} and \fullref{def:function_invertibility}, we introduce the following terminology:
  \begin{thmenum}
    \thmitem{def:functor_invertibility/injective_on_objects} The \hyperref[def:functor]{functor} \( F: \cat{C} \to \cat{D} \) is \term{injective on objects} if the \hyperref[def:set_valued_map/restriction]{restriction}
    \begin{equation*}
      F\restr_{\obj(C)}: \obj(C) \to \obj(D)
    \end{equation*}
    is \hyperref[def:function_invertibility/injective]{injective}.

    That is, for every pair of objects \( A \) and \( B \) in \( \cat{C} \), from \( F(A) = F(B) \) it follows that \( A = B \).

    If, instead, from \( F(A) \cong F(B) \) it follows that \( A \cong B \), we say that \( F \) if \term{essentially injective on objects}.

    \thmitem{def:functor_invertibility/injective_on_morphisms} The \hyperref[def:functor]{functor} \( F: \cat{C} \to \cat{D} \) is \term{injective on morphisms} if its restriction to the set
    \begin{equation*}
      \bigcup\set{ \cat{C}(A, B) \given A, B \in \obj(\cat{C}) }
    \end{equation*}
    of all morphisms is injective.

    That is, for every pair of morphisms \( f \) and \( g \) in \( \cat{C} \), from \( F(f) = F(g) \) it follows that \( f = g \). Note that if the morphisms are not parallel, we assume that they are not equal.

    \thmitem{def:functor_invertibility/faithful}\mcite[def. 1.2.16]{Leinster2014BasicCategories} The functor \( F: \cat{C} \to \cat{D} \) is \term{faithful} if it is \hyperref[def:function_invertibility/injective]{injective} on \( \hom \)-sets, i.e. for all pairs of objects \( A \) and \( B \) in \( \cat{C} \), the restriction of \( F \) to \( \cat{C}(A, B) \) is an injective function.

    That is, for every pair of objects \( A \) and \( B \) in \( \cat{C} \) and every pair of morphisms \( f \) and \( g \) in \( \cat{C}(A, B) \), from \( F(f) = F(g) \) it follows that \( f = g \).

    See \fullref{thm:def:functor_invertibility/injective} for how faithful functors relate to functors injective on objects or on morphisms.

    \thmitem{def:functor_invertibility/surjective_on_objects}\mcite[def. 1.3.17]{Leinster2014BasicCategories} The functor \( F: \cat{C} \to \cat{D} \) is \term{surjective on objects} if the restriction
    \begin{equation*}
      F\restr_{\obj(C)}: \obj(C) \to \obj(D)
    \end{equation*}
    is \hyperref[def:function_invertibility/surjective]{surjective}.

    That is, for every object \( B \) in \( \cat{D} \), there exists at least one object \( A \) in \( \cat{C} \) such that \( F(A) = B \).

    If, instead, there exists at least one object \( A \in \cat{C} \) such that \( F(A) \cong B \), we say that \( F \) is \term{essentially surjective on objects}.

    \thmitem{def:functor_invertibility/surjective_on_morphisms} Similarly, \( F: \cat{C} \to \cat{D} \) is \term{surjective on morphisms} if its restriction to the set of all morphisms is surjective.

    That is, for every morphism \( g \) in \( \cat{D} \), there exists at least one morphism \( f \) in \( \cat{C} \) such that \( F(f) = g \).

    \thmitem{def:functor_invertibility/full}\mcite[def. 1.2.16]{Leinster2014BasicCategories} The functor \( F: \cat{C} \to \cat{D} \) is \term{full} if it is surjective on \( \hom \)-sets, i.e. for all pairs of objects \( A \) and \( B \) in \( \cat{C} \), the restriction of \( F \) to \( \cat{C}(A, B) \) is a surjective function.

    That is, for every pair of objects \( A \) and \( B \) in \( \cat{C} \) and every morphism \( g: F(A) \to F(B) \) in \( \cat{D} \), there exists at least one morphism in \( f: A \to B \) in \( \cat{C} \) such that \( F(f) = g \).

    \thmitem{def:functor_invertibility/fully_faithful} Finally, \( F: \cat{C} \to \cat{D} \) is \term{fully faithful} if it is both full and faithful.
  \end{thmenum}
\end{definition}

\begin{proposition}\label{thm:commutative_diagrams_preserved_and_reflected}
  Functors preserve commutative diagrams and faithful functors also reflect commutative diagrams.

  More precisely, let \( \cat{C} \) be an arbitrary category, let \( D \) be a diagram in \( \cat{C} \), and let
  \begin{equation*}
    A \reloset {f_1} \to \anon \reloset {f_2} \to \cdots \reloset {f_{n-1}} \to \anon \reloset {f_n} \to B
  \end{equation*}
  and
  \begin{equation*}
    A \reloset {g_1} \to \anon \reloset {g_2} \to \cdots \reloset {g_{n-1}} \to \anon \reloset {g_m} \to B
  \end{equation*}
  be two \hyperref[def:graph_walk/directed]{directed walks} with the same endpoints in \( D \).

  For any functor \( F: \cat{C} \to \cat{D} \), if
  \begin{equation}\label{eq:thm:commutative_diagrams_preserved_and_reflected/source}
    f_n \bincirc \cdots \bincirc \bincirc f_1 = g_m \bincirc \cdots \bincirc \bincirc g_1,
  \end{equation}
  then
  \begin{equation}\label{eq:thm:commutative_diagrams_preserved_and_reflected/image}
    F(f_n) \bincirc \cdots \bincirc \bincirc F(f_1) = F(g_m) \bincirc \cdots \bincirc \bincirc F(g_1),
  \end{equation}

  Conversely, if \( F \) is faithful, then \eqref{eq:thm:commutative_diagrams_preserved_and_reflected/image} implies \eqref{eq:thm:commutative_diagrams_preserved_and_reflected/source}.
\end{proposition}
\begin{proof}
  Functors preserve composition by \ref{eq:def:functor/CF2}, hence \eqref{eq:thm:commutative_diagrams_preserved_and_reflected/image} follows from \eqref{eq:thm:commutative_diagrams_preserved_and_reflected/source} directly.

  Now suppose that \eqref{eq:thm:commutative_diagrams_preserved_and_reflected/image} holds for a faithful functor \( F \). \ref{eq:def:functor/CF2} allows us to reduce \eqref{eq:thm:commutative_diagrams_preserved_and_reflected/image} to
  \begin{equation*}
    F(f_n \bincirc \cdots \bincirc \bincirc f_1) = F(g_m \bincirc \cdots \bincirc \bincirc g_1).
  \end{equation*}

  Then, by injectivity of \( F \) on the morphism set \( \cat{C}(\dom(f_1), \co\dom(f_1)) \), \eqref{eq:thm:commutative_diagrams_preserved_and_reflected/source} holds.
\end{proof}

\begin{proposition}\label{thm:def:functor_invertibility}
  \hyperref[def:functor]{Functors} have the following basic properties regarding their \hyperref[def:functor_invertibility]{invertibility}:

  \begin{thmenum}
    \thmitem{thm:def:functor_invertibility/injective} A functor is \hyperref[def:functor_invertibility/injective_on_morphisms]{injective on morphisms} if and only if it is both \hyperref[def:functor_invertibility/injective_on_objects]{injective on objects} and \hyperref[def:functor_invertibility/faithful]{faithful}.

    \thmitem{thm:def:functor_invertibility/surjective} A functor is \hyperref[def:functor_invertibility/surjective_on_morphisms]{surjective on morphisms} if and only if it is both \hyperref[def:functor_invertibility/surjective_on_objects]{surjective on objects} and \hyperref[def:functor_invertibility/full]{full}.

    \thmitem{thm:def:functor_invertibility/full_subcategory} A \hyperref[def:subcategory]{subcategory} \( \cat{D} \) of \( \cat{C} \) is full in the sense of \fullref{def:subcategory} if and only if the \hyperref[def:subcategory]{inclusion functor} \( \Iota: \cat{D} \to \cat{C} \) is full in the sense of \fullref{def:functor_invertibility/full}.

    \thmitem{thm:def:functor_invertibility/preserves_inverses} Any functor preserves \hyperref[def:morphism_invertibility/left_invertible]{left}, \hyperref[def:morphism_invertibility/right_invertible]{right inverses} and \hyperref[def:morphism_invertibility/isomorphism]{two-sided inverses}.

    \thmitem{thm:def:functor_invertibility/faithful_reflects_composition} \hyperref[def:functor_invertibility/faithful]{Faithful} functors reflect composition. That is, for every functor \( F: \cat{C} \to \cat{D} \), if the following diagram commutes:
    \begin{equation}\label{eq:thm:def:functor_invertibility/faithful_reflects_composition/image}
      \begin{aligned}
        \includegraphics[page=1]{output/thm__def__functor_invertibility__properties}
      \end{aligned}
    \end{equation}
    then the following diagram are identities:
    \begin{equation}\label{eq:thm:def:functor_invertibility/faithful_reflects_composition/source}
      \begin{aligned}
        \includegraphics[page=2]{output/thm__def__functor_invertibility__properties}
      \end{aligned}
    \end{equation}

    \thmitem{thm:def:functor_invertibility/faithful_reflects_cancellative} A \hyperref[def:functor_invertibility/faithful]{faithful} functor reflects monomorphisms and epimorphisms. That is, for every functor \( F: \cat{C} \to \cat{D} \) and morphism \( f: A \to B \) in \( \cat{C} \), if \( F(f) \) is a monomorphism (resp. epimorphism), so is \( f \).

    \thmitem{thm:def:functor_invertibility/fully_faithful_reflects_identities} A \hyperref[def:functor_invertibility/fully_faithful]{fully faithful} functor reflects identities. That is, for every functor \( F: \cat{C} \to \cat{D} \) and endomorphism \( f: A \to A \) in \( \cat{C} \), if \( F(f) = \id_{F(A)} \), then \( f = \id_A \).

    \thmitem{thm:def:functor_invertibility/fully_faithful_reflects_invertible} A \hyperref[def:functor_invertibility/fully_faithful]{fully faithful} functor reflects split monomorphisms and split epimorphisms.

    That is, for every functor \( F: \cat{C} \to \cat{D} \) and morphism \( f: A \to B \) in \( \cat{C} \), if \( F(f) \) is a split monomorphism (resp. split epimorphism or isomorphism), so is \( f \).

    \thmitem{thm:def:functor_invertibility/isomorphism} A functor between \( \mscrU \)-small categories that is both injective and surjective on morphisms is itself an isomorphism in \( \ucat{Cat} \).
  \end{thmenum}
\end{proposition}
\begin{proof}
  \SubProofOf{thm:def:functor_invertibility/injective}
  \SufficiencySubProof* Let \( F: \cat{C} \to \cat{D} \) be injective on morphisms. It is trivially faithful since faithfulness is a more restrictive condition.

  To see that \( F \) is injective on objects, let \( A, B \in \cat{C} \) and suppose that \( F(A) = F(B) \). Then \( \id_{F(A)} = \id_{F(B)} \) and
  \begin{equation*}
    F(\id_A)
    \reloset {\eqref{eq:def:functor/CF2}} =
    \id_{F(A)}
    =
    \id_{F(B)}
    \reloset {\eqref{eq:def:functor/CF2}} =
    F(\id_B).
  \end{equation*}

  Since \( F \) is injective on morphisms, it follows that \( \id_A = \id_B \), hence \( A = B \). Thus, \( F \) is injective on objects.

  \NecessitySubProof* Let \( F: \cat{C} \to \cat{D} \) be faithful and injective on objects. Let \( f: A \to B \) and \( g: C \to D \) be morphisms in \( \cat{C} \) such that \( F(f) = F(g) \).

  Then both \( F(f) \) and \( F(g) \) have the same domain \( F(A) = F(C) \) and codomain \( F(B) = F(D) \). Hence, since \( F \) is injective on objects, we have \( A = C \) and \( B = D \).

  Thus, \( f \) and \( g \) are both morphisms from \( A \) to \( B \). Since \( F \) is also faithful, from \( F(f) = F(g) \) it follows that \( f = g \).

  Therefore, \( F \) is injective on morphisms.

  \SubProofOf{thm:def:functor_invertibility/surjective}
  \SufficiencySubProof* Let \( F: \cat{C} \to \cat{D} \) be surjective on morphisms. It is trivially full since fullness is a more restrictive condition.

  To see that \( F \) is surjective on objects, let \( C \in \cat{D} \). Then there exists some morphism \( f: A \to B \) in \( \cat{C} \) such that \( F(f) = \id_Z \). We thus necessarily have \( F(A) = C \) and \( F(B) = C \).

  \NecessitySubProof* Let \( F: \cat{C} \to \cat{D} \) be full and injective on objects. Let \( g: C \to D \) be a morphism in \( \cat{D} \).

  Since \( F \) is surjective on objects, there exists preimages \( A \) of \( C \) and \( B \) of \( D \) under \( F \). Thus, \( g \in \cat{D}(F(A), F(B)) \).

  Since \( F \) is also full, there exists some morphism \( f: A \to B \) such that \( F(f) = g \).

  Therefore, \( F \) is surjective on morphisms.

  \SubProofOf{thm:def:functor_invertibility/full_subcategory} Trivial.

  \SubProofOf{thm:def:functor_invertibility/preserves_inverses} Let \( g: B \to A \) be a left inverse of \( f: A \to B \). Then
  \begin{equation*}
    F(g) \bincirc F(f)
    \reloset {\eqref{eq:def:functor/CF2}} =
    F(g \bincirc f)
    =
    F(\id_A)
    \reloset {\eqref{eq:def:functor/CF1}} =
    \id_{F(A)}.
  \end{equation*}

  The case of right inverses is similar.

  \SubProofOf{thm:def:functor_invertibility/faithful_reflects_composition} Suppose that \eqref{eq:thm:def:functor_invertibility/faithful_reflects_composition/image} commutes. Then, since \( F \) is faithful and thus injective on the morphism set \( \cat{C}(A, C) \), the equality \( F(g \bincirc f) = F(g) \bincirc F(f) = F(h) \) implies that \( g \bincirc f = h \). Hence, \eqref{eq:thm:def:functor_invertibility/faithful_reflects_composition/source} also commutes.

  \SubProofOf{thm:def:functor_invertibility/faithful_reflects_cancellative} Let \( F(g) \) be a monomorphism and let \( f_1, f_2: A \to B \) be parallel morphisms such that
  \begin{equation*}
    g \bincirc f_1 = g \bincirc f_2.
  \end{equation*}

  Then, since \( F(g) \) is a monomorphism, we have that \( F(f_1) = F(f_2) \). Since \( F \) is faithful, the restriction \( F\restr_{C(A, B)} \) is injective, and \( f_1 = f_2 \).

  The proof when \( F(g) \) is an epimorphism is analogous.

  \SubProofOf{thm:def:functor_invertibility/fully_faithful_reflects_identities} If \( F: \cat{C} \to \cat{D} \) is fully faithful, for every object \( A \) in \( \cat{C} \), the identity morphism \( \id_{F(A)} \) has a unique preimage under \( F \). By \ref{def:functor/CF1}, this preimage can only be \( \id_A \).

  \SubProofOf{thm:def:functor_invertibility/fully_faithful_reflects_invertible} Let \( q \) be a left inverse of \( F(f) \). Since \( F \) is fully faithful, there exists a unique morphism \( g: B \to A \) such that \( F(g) = q \).

  Since
  \begin{equation*}
    F(g) \bincirc F(f) = \id_{F(A)},
  \end{equation*}
  by \fullref{thm:def:functor_invertibility/fully_faithful_reflects_identities} we have
  \begin{equation*}
    g \bincirc f = \id_A.
  \end{equation*}

  Therefore, \( g \) is a left inverse of \( F(f) \).

  The proof for right inverses follows from \fullref{thm:def:morphism_invertibility/inverse_interchanges}.

  From \fullref{thm:def:morphism_invertibility/left_and_right} it follows that if \( F(f) \) is an isomorphism, so is \( f \).

  \SubProofOf{thm:def:functor_invertibility/isomorphism} If \( F \) is both injective and surjective on morphisms, it is also injective and surjective on objects and hence, as a function, is bijective. Therefore, it is both left and right invertible as a consequence of \fullref{thm:function_invertibility_categorical/fully_invertible}.
\end{proof}

\begin{example}\label{ex:def:functor_invertibility}
  \hfill
  \begin{thmenum}
    \thmitem{ex:def:functor_invertibility/power} The power set functor described in \fullref{ex:unary_functors_in_set} is clearly \hyperref[def:functor_invertibility/injective_on_morphisms]{injective on morphisms}, hence by \fullref{thm:def:functor_invertibility/injective}, it is also \hyperref[def:functor_invertibility/injective_on_objects]{injective on objects} and \hyperref[def:functor_invertibility/faithful]{faithful}.

    It is not full, nor surjective on objects.

    \thmitem{ex:def:functor_invertibility/cat_to_set} The forgetful functor \( D: \ucat{Cat} \to \ucat{Set} \) discussed in \fullref{def:discrete_category} is \hyperref[def:functor_invertibility/surjective_on_morphisms]{surjective on morphisms}, hence by \fullref{thm:def:functor_invertibility/surjective}, it is also \hyperref[def:functor_invertibility/surjective_on_objects]{surjective on objects} and \hyperref[def:functor_invertibility/full]{full}.

    It is not faithful, nor injective on objects.
  \end{thmenum}
\end{example}

\begin{definition}\label{def:natural_transformation}\mcite[def. 1.3.1]{Leinster2014BasicCategories}
  Let \( F \) and \( G \) be parallel \hyperref[def:functor]{functors} from the category \( \cat{C} \) to \( \cat{D} \).

  A \term{natural transformation} \( \alpha \) from \( F \) to \( G \) is an \hyperref[def:cartesian_product/indexed_family]{indexed family} of
  \begin{equation}\label{eq:def:natural_transformation/family}
    \seq{ \alpha_A: F(A) \to G(A) }_{A \in \cat{C}}
  \end{equation}
  of morphisms in \( \cat{D} \) such that, for every morphism \( f: A \to B \) in \( \cat{C} \), the following \hyperref[def:categorical_diagram]{diagram commutes}:
  \begin{equation}\label{eq:def:natural_transformation/diagram}
    \begin{aligned}
      \includegraphics[page=1]{output/def__natural_transformation}
    \end{aligned}
  \end{equation}

  The morphisms \( \alpha_A \) are called the components of \( \alpha \). We denote natural transformations by \( \alpha: F \Rightarrow G \) and, when used in diagrams, by
  \begin{equation}\label{eq:def:natural_transformation/notation}
    \begin{aligned}
      \includegraphics[page=2]{output/def__natural_transformation}
    \end{aligned}
  \end{equation}
\end{definition}

\begin{example}\label{ex:directed_multigraphs_as_functors}\mcite[example 1.3.46]{Perrone2021Categories}
  In \fullref{def:directed_multigraph}, we have defined a directed multigraph as a set \( V \) of vertices, a set \( A \) of arcs and two functions --- the initial \( i: A \to V \) and terminal \( t: A \to V \) vertices of an arc.

  Now consider the following \hyperref[def:categorical_diagram]{index category} \( \cat{I}: \)
  \begin{equation}\label{eq:ex:directed_multigraphs_as_functors/index/dots}
    \begin{aligned}
      \includegraphics[page=1]{output/ex__directed_multigraphs_as_functors}
    \end{aligned}
  \end{equation}

  For the sake of readability, we will give the following explicit labels in this category:
  \begin{equation}\label{eq:ex:directed_multigraphs_as_functors/index/annotated}
    \begin{aligned}
      \includegraphics[page=2]{output/ex__directed_multigraphs_as_functors}
    \end{aligned}
  \end{equation}

  A directed multigraph can then be defined as a functor \( G: \cat{I} \to \ucat{Set} \) to the category \hyperref[def:category_of_small_sets]{\( \ucat{Set} \)} of \( \mscrU \)-small sets (for a \hyperref[def:category_size]{fixed Grothendieck universe} \( \mscrU \)).

  A \hyperref[def:natural_transformation]{natural transformation} from the directed multigraph \( G: \cat{I} \to \ucat{Set} \) to \( H: \cat{I} \to \ucat{Set} \) is then a pair of functions \( f_V: G(V) \to H(V) \) and \( f_A: G(A) \to H(A) \) such that the following \hyperref[def:categorical_diagram]{diagrams commute}:
  \begin{equation}\label{eq:ex:directed_multigraphs_as_functors/index/diagram}
    \begin{aligned}
      \includegraphics[page=3]{output/ex__directed_multigraphs_as_functors}
      \quad\quad\quad\quad
      \includegraphics[page=4]{output/ex__directed_multigraphs_as_functors}
    \end{aligned}
  \end{equation}

  See \fullref{ex:isomorphism_of_directed_multigraph_categories} for how these functors related to directed multigraphs as defined in \fullref{def:directed_multigraph}.
\end{example}

\begin{remark}\label{rem:natural_transformations_into_set}
  Let \( \cat{C} \) be an arbitrary \( \mscrU \)-small category. A \hyperref[def:natural_transformation]{natural transformation} \( \alpha \) from \( F: \cat{C} \to \ucat{Set} \) to \( G: \cat{C} \to \ucat{Set} \) is then a family of functions
  \begin{equation*}
    \seq{ \alpha_A: F(A) \to G(A) }_{A \in \cat{C}}.
  \end{equation*}

  Suppose that for every two objects \( A \) and \( B \) in \( \cat{C} \), the functions \( \alpha_A \) and \( \alpha_B \) agree on \( F(A) \cap F(B) \). This is automatically satisfied in \( F(A) \) and \( F(B) \) are disjoint whenever \( A \neq B \).

  We can then take the set-theoretic union of \( \alpha \) to obtain the function
  \begin{equation*}
    \bigcup_{A \in \cat{C}} \alpha_A: \bigcup\set{ F(A) \given A \in \cat{C} } \to \bigcup\set{ G(A) \given A \in \cat{C} }.
  \end{equation*}

  Both the domain and codomain are sets as a consequence of \ref{def:grothendieck_universe/union}, therefore the function is well-defined in the universe \( \mscrU \). Denote it on \( \Alpha \) for brevity.

  An advantage of this is that we can define a natural transformation to be a function on a general enough set and then prove that its restrictions satisfy \eqref{eq:def:natural_transformation/diagram}.

  For example, consider the power set functor \( \pow: \ucat{Set} \to \ucat{Set} \) discussed in \fullref{ex:unary_functors_in_set}. The \hyperref[def:set_valued_map/identity]{identity function} \( \id_\mscrU \) is then a natural transformation from the identity functor \( \id_{\ucat{Set}} \) to \( \pow \).

  Another natural transformation between the same functors is the singleton set operation \( \Sigma \) on sets defined as \( A \mapsto \set{ A } \). Note that, in this context, \( \Sigma \) operates not on the sets \( \id_{\ucat{Set}}(A) \) and \( \pow(A) \), but on their members. The diagram \eqref{eq:def:natural_transformation/diagram} becomes
  \begin{equation}\label{eq:rem:natural_transformations_into_set}
    \begin{aligned}
      \includegraphics[page=1]{output/rem__natural_transformations_into_set}
    \end{aligned}
  \end{equation}

  This diagram commutes because, for every function \( f: A \to B \) and every \( x \in A \), we have
  \begin{equation*}
    f[\set{ x }] = \set{ f(x) }.
  \end{equation*}
\end{remark}

\begin{definition}\label{def:functor_category}
  Let \( \cat{C} \) and \( \cat{D} \) be arbitrary \hyperref[def:category]{categories}. The \term{functor category} \( [\cat{C}, \cat{D}] \), also denoted as \( \cat{D}^{\cat{C}} \), is defined as follows:

  \begin{itemize}
    \item The \hyperref[def:category/objects]{set of objects} \( \obj([\cat{C}, \cat{D}]) \) is the set of all functors from \( \cat{C} \) to \( \cat{D} \).

    \item The \hyperref[def:category/morphisms]{set of morphisms} \( [\cat{C}, \cat{D}](F, G) \) from \( F \) to \( G \) is the set of all \hyperref[def:natural_transformation]{natural transformations} from \( F \) to \( G \).

    \item The \hyperref[def:category/composition]{composition of the morphisms} \( \alpha: F \Rightarrow G \) and \( \beta: G \Rightarrow H \) is the natural transformation \( \beta \bincirc \alpha: F \Rightarrow H \) defined in terms of componentwise morphism composition, i.e.
    \begin{equation}\label{eq:def:functor_category/composition}
      (\beta \bincirc \alpha)_A \coloneqq \beta_A \bincirc \alpha_A.
    \end{equation}

    \item The \hyperref[def:category/identity]{identity morphism} on the functor \( F: \cat{C} \to \cat{D} \) is the \term{identity natural transformation} \( \id_F: F \Rightarrow F \) with components
    \begin{equation}\label{eq:def:functor_category/identity}
      (\id_F)_A \coloneqq \underbrace{\id_{F(A)}}_{F(\id_A)}
    \end{equation}
  \end{itemize}
\end{definition}
\begin{defproof}
  Just to verify that the composition \( \beta \bincirc \alpha \) defined in \eqref{eq:def:functor_category/composition} is indeed a natural transformation from \( F \) to \( H \), note that the following diagram trivially commutes:
  \begin{equation}\label{def:functor_category/composition}
    \begin{aligned}
      \includegraphics[page=1]{output/def__functor_category}
    \end{aligned}
  \end{equation}

  Now, to see that \( [\cat{C}, \cat{D}] \) is indeed a category, we verify the conditions \ref{def:category/C1} and \ref{def:category/C2}, which are in turn inherited from the same conditions on the categories \( \cat{C} \) and \( \cat{D} \).

  \SubProofOf{def:category/C1} For every two functors \( F, G: \cat{C} \to \cat{D} \) and natural transformation \( \alpha: F \Rightarrow G \), for every object \( A \in \cat{C} \) we have
  \begin{equation*}
    \id_{G(A)} \bincirc \alpha_A
    \reloset{\eqref{def:category/C1}} =
    \alpha_A
    \reloset{\eqref{def:category/C1}} =
    \alpha_A \bincirc \id_{F(A)}
  \end{equation*}

  Therefore,
  \begin{equation*}
    \id_G \bincirc \alpha = \alpha = \alpha \bincirc \id_F
  \end{equation*}
  and, after generalizing, we obtain that \eqref{eq:def:category/C1} holds in \( [\cat{C}, \cat{D}] \).

  \SubProofOf{def:category/C2} For any quadruple \( F \), \( G \), \( H \) and \( T \) of functors from \( \cat{C} \) to \( \cat{D} \) and every combination of natural transformations \( \alpha: F \Rightarrow G \), \( \beta: G \Rightarrow H \) and \( \gamma: H \Rightarrow T \), for every object \( A \in \cat{C} \) we have
  \begin{equation*}
    (\gamma_A \bincirc \beta_A) \bincirc \alpha_A
    \reloset{\eqref{def:category/C2}} =
    \gamma_A \bincirc (\beta_A \bincirc \alpha_A).
  \end{equation*}

  Therefore, after generalizing, we obtain that \eqref{eq:def:category/C2} holds in \( [\cat{C}, \cat{D}] \).
\end{defproof}

\begin{remark}\label{rem:functor_category_size}
  If \( \cat{C} \) and \( \cat{D} \) are \( \mscrU \)-large categories in the sense of \fullref{def:category_size}, we cannot construct the \hyperref[def:functor_category]{functor category} \( [\cat{C}, \cat{D}] \). This is the main motivation for the \hyperref[def:axiom_of_universes]{axiom of universes}, which is discussed in \fullref{def:large_and_small_sets} and, in relation to category theory, in \fullref{def:category_size}.
\end{remark}

\begin{example}\label{ex:isomorphism_of_directed_multigraph_categories}
  In \fullref{ex:directed_multigraphs_as_functors}, we defined \hyperref[def:directed_multigraph]{directed multigraphs} as functors from a certain index category \( \cat{I} \) to \( \ucat{Set} \) (for a \hyperref[def:category_size]{fixed Grothendieck universe} \( \mscrU \)).

  There is then an obvious correspondence between directed multigraphs as objects of the category \( \ucat{DM} \) defined in \fullref{def:directed_multigraph/category}, and directed multigraphs as objects in the \hyperref[def:functor_category]{functor category} \( [\cat{I}, \ucat{Set}] \), defined in \fullref{ex:directed_multigraphs_as_functors}. Indeed, given any functor \( G: \cat{I} \to \ucat{Set} \), the quadruple
  \begin{equation*}
    \parens[\Big]{ G(V), G(A), G(h), G(t) }
  \end{equation*}
  is a directed multigraph in the sense of \fullref{def:directed_multigraph}.

  No object in \( \ucat{DM} \) is formally equal to any object in \( [\cat{I}, \ucat{Set}] \) in the sense of \hyperref[def:zfc]{\logic{ZFC}}. They are, however, equivalent, as shown above, and this can be formalized by stating that the two categories are isomorphic, in the sense of \fullref{def:morphism_invertibility/isomorphism}, as objects of the category \( \ucat[\mscrV]{Cat} \), where \( \mscrV \) is a Grothendieck universe that strictly contains \( \mscrU \). We have already defined this isomorphism explicitly.

  This is an example of \term{isomorphism of categories}. In practice, if two categories are not so obviously identical, we are usually better served by \term{equivalences of categories} defined in \fullref{def:category_equivalence}.
\end{example}

\begin{definition}\label{def:opposite_natural_transformation}\mimprovised
  The \term{opposite} natural transformation of \( \alpha: F \Rightarrow G \), where \( F \) and \( G \) are functors from \( \cat{C} \) to \( \cat{D} \), is the natural transformation \( \alpha^\oppos: G^\oppos \Rightarrow F^\oppos \), in which we take the opposite of each component in \( \alpha \).

  Dual natural transformation arise naturally in \fullref{thm:opposite_of_functor_category}.
\end{definition}
\begin{defproof}
  The naturality diagram \eqref{eq:def:natural_transformation/diagram} commutes for \( \alpha^\oppos \) because all morphisms are simply reversed.
\end{defproof}

\begin{proposition}\label{thm:opposite_of_functor_category}
  For the \hyperref[def:opposite_category]{opposite} of the \hyperref[def:functor_category]{functor category} \( [\cat{C}, \cat{D}] \) we have
  \begin{equation*}
    [\cat{C}, \cat{D}]^\oppos = [\cat{C}^\oppos, \cat{D}^\oppos].
  \end{equation*}

  This is part of the duality principles listed in \fullref{thm:categorical_principle_of_duality}.
\end{proposition}
\begin{proof}
  In \fullref{def:opposite_functor}, we have defined the opposite functor \( F^\oppos: \cat{C}^\oppos \to \cat{D}^\oppos \) of \( F: \cat{C} \to \cat{D} \) in a way that allows us to regard it as an object of \( [\cat{C}^\oppos, \cat{D}^\oppos] \).

  In \fullref{def:opposite_natural_transformation}, we have defined the opposite natural transformation \( \alpha^\oppos: G^\oppos \to F^\oppos \) of \( \alpha: F \Rightarrow G \) in a way that allows us to regard it as a morphism of \( [\cat{C}^\oppos, \cat{D}^\oppos] \).

  Furthermore, \( \alpha^\oppos: G^\oppos \to F^\oppos \) reverses the direction of its morphisms, and hence it is the dual to \( \alpha \) in the category \( [\cat{C}, \cat{D}]^\oppos \).
\end{proof}

\begin{definition}\label{def:diagonal_functor}\mcite[exercise 3.1.1]{Leinster2014BasicCategories}
  Given an \term{index category} \( \cat{I} \) and an arbitrary category \( \cat{C} \), for any object \( A \) in \( \cat{C} \), we can define the \term{constant functor}
  \begin{equation*}
    \begin{aligned}
      &\Delta_A^{\cat{I}}: \cat{I} \to \cat{C}, \\
      &\Delta_A^{\cat{I}}(X) \coloneqq X, \\
      &\Delta_A^{\cat{I}}(g: X \to Y) \coloneqq \id_A.
    \end{aligned}
  \end{equation*}

  Given two objects \( A \) and \( B \) in \( \cat{C} \), a natural transformation \( \alpha: \Delta_A^{\cat{I}} \Rightarrow \Delta_B^{\cat{I}} \) is an \hyperref[def:cartesian_product/indexed_family]{indexed family} that gives the same morphism for every object of the index category \( \cat{I} \).

  Indeed, the diagram \eqref{eq:def:natural_transformation/diagram} in this case becomes
  \begin{equation}\label{eq:def:diagonal_functor/nat}
    \begin{aligned}
      \includegraphics[page=1]{output/def__diagonal_functor}
    \end{aligned}
  \end{equation}

  This diagram implies that \( \alpha_A = \alpha_B \) for any two objects \( A \) and \( B \) in \( \cat{I} \). Therefore, all components of \( \alpha \) are equal to some morphism in \( \cat{C}(A, B) \).

  We can now define the \( \cat{I} \)-shaped \term{diagonal functor} on \( \cat{C} \)
  \begin{equation*}
    \begin{aligned}
      &\Delta^{\cat{I}}: \cat{C} \to [\cat{I}, \cat{C}], \\
      &\Delta^{\cat{I}}(A) \coloneqq \Delta_A^{\cat{I}}, \\
      &\Delta^{\cat{I}}(f: A \to B) \coloneqq \seq{ f: A \to B }_{k \in \cat{I}}.
    \end{aligned}
  \end{equation*}

  It is called a diagonal functor because, if \( \cat{I} \) is a discrete category of two objects, then \( \Delta_{\cat{I}} \) gives the diagonal of the \hyperref[def:product_category]{product category} \( \cat{C}^2 \) by providing, for each object \( A \) of \( \cat{C} \), the ordered pair \( (A, A) \) (and similarly for morphisms).
\end{definition}

\begin{proposition}\label{thm:natural_isomorphism}\mcite{math3ma:natural_transformations}
  Let \( F \) and \( G \) be parallel \hyperref[def:functor]{functors} from the category \( \cat{C} \) to \( \cat{D} \). The family \eqref{eq:def:natural_transformation/family} is an isomorphism in the corresponding \hyperref[def:functor_category]{functor category} \( [\cat{C}, \cat{D}] \) if and only if all of its components are isomorphisms and, for any morphism \( f: A \to B \) in \( \cat{C} \), the following diagram commutes:
  \begin{equation}\label{eq:thm:natural_isomorphism/diagram}
    \begin{aligned}
      \includegraphics[page=1]{output/thm__natural_isomorphism}
    \end{aligned}
  \end{equation}

  We say that \( \alpha \) is a \term{natural isomorphism}.
\end{proposition}
\begin{proof}
  If all components of \( \alpha \) are isomorphisms, the condition
  \begin{equation*}
    \alpha_B \bincirc F(f) = G(f) \bincirc \alpha_A
  \end{equation*}
  is equivalent to
  \begin{equation*}
    F(f) = \alpha_B^{-1} \bincirc G(f) \bincirc \alpha_A.
  \end{equation*}

  We must now show that, if \( \alpha \) is an isomorphism in \( [\cat{C}, \cat{D}] \), all of its components are isomorphisms.

  If \( \alpha: F \Rightarrow G \) is an isomorphism in \( [\cat{C}, \cat{D}] \). Then there exists some natural transformation \( \beta: G \Rightarrow F \) such that
  \begin{equation*}
    \beta \bincirc \alpha = \id_F \quad\T{and}\quad \alpha \bincirc \beta = \id_G.
  \end{equation*}

  For every object \( A \) in \( \cat{C} \), the morphism \( \alpha_A: F(A) \to G(A) \) composed with \( \beta_A: G(A) \to F(A) \) is
  \begin{equation*}
    \beta_A \bincirc \alpha_A = \id_{F(A)}.
  \end{equation*}

  Therefore, \( \alpha_A \) is left-invertible. Analogously,
  \begin{equation*}
    \alpha_A \bincirc \beta_A = \id_{F(A)}
  \end{equation*}
  and hence \( \alpha_A \) is right-invertible.

  Therefore, for every object \( A \) in \( \cat{C} \), the morphism \( \alpha_A \) is fully invertible, i.e. an isomorphism.
\end{proof}

\begin{definition}\label{def:product_category}\mcite[const. 1.1.11]{Leinster2014BasicCategories}
  We define the \term{product category} \( \cat{C} \times \cat{D} \) of \( \cat{C} \) and \( \cat{D} \) as follows:

  \begin{itemize}
    \item The \hyperref[def:category/objects]{set of objects} is the \hyperref[def:cartesian_product]{Cartesian product}
    \begin{equation}\label{eq:def:product_category/objects}
      \obj(\cat{C} \times \cat{D}) \coloneqq \obj(\cat{C}) \times \obj(\cat{D}).
    \end{equation}

    \item The \hyperref[def:category/morphisms]{set of morphisms} from the pair of objects \( (A, X) \) to \( (B, Y) \) is the product
    \begin{equation}\label{eq:def:product_category/morphisms}
      (\cat{C} \times \cat{D})\parens[\Big]{ (A, X), (B, Y) } \coloneqq \cat{C}(A, B) \times \cat{D}(X, Y).
    \end{equation}

    \item The \hyperref[def:category/composition]{composition of the morphisms}
    \begin{align*}
      (f, r)&: (A, X) \to (B, Y) \\
      (g, s)&: (B, Y) \to (C, Z)
    \end{align*}
    is the pairwise composition
    \begin{equation}\label{eq:def:product_category/composition}
      (g, s) \bincirc (f, r) \coloneqq \underbrace{(g \bincirc f, s \bincirc r)}_{(A, X) \to (C, Z)}.
    \end{equation}

    \item The \hyperref[def:category/identity]{identity morphism} of the pair \( (A, X) \) is simply the pair of identity morphisms \( (\id_A, \id_X) \).
  \end{itemize}
\end{definition}

\begin{definition}\label{def:hom_functor}
  Let \( \cat{C} \) be a \hyperref[def:category_size]{locally \( \mscrU \)-small} category. We can regard the morphism sets \( \cat{C}(A, B) \) as a functor parameterized by objects of \( \cat{C} \).

  \begin{thmenum}
    \thmitem{def:hom_functor/binary} For any pair of morphisms \( f: B \to A \) and \( g: X \to Y \) in \( \cat{C} \), define the operator
    \begin{equation}\label{eq:def:hom_functor/t}
      \begin{aligned}
        &T_{f,g}: \cat{C}(A, X) \to \cat{C}(B, Y) \\
        &T_{f,g}(s) \mapsto g \bincirc s \bincirc f.
      \end{aligned}
    \end{equation}

    The action of \( T_{f,g} \) can be expressed graphically as
    \begin{equation}\label{eq:def:hom_functor/t_diagram}
      \begin{aligned}
        \includegraphics[page=1]{output/def__hom_functor}
      \end{aligned}
    \end{equation}

    We can now define the following \term{binary hom-functor}:
    \begin{equation}\label{eq:def:hom_functor/binary}
      \begin{aligned}
        &\cat{C}(\anon*, \anon*): \cat{C}^\oppos \times \cat{C} \to \ucat{Set} \\
        &\cat{C}(A, X) \coloneqq \set{ s: A \to X } \\
        &\cat{C}(f, g) \coloneqq T_{f,g}
      \end{aligned}
    \end{equation}

    \thmitem{def:hom_functor/unary} Fixing the first argument \( A \) in \eqref{eq:def:hom_functor/binary}, we instead obtain a covariant unary hom-functor:
    \begin{equation}\label{eq:def:hom_functor/unary/covariant}
      \cat{C}(A, \anon*): \cat{C} \to \ucat{Set}
    \end{equation}

    Analogously, fixing the second argument \( X \), we obtain a \hyperref[def:hom_functor/unary]{contravariant} unary hom-functor:
    \begin{equation}\label{eq:def:hom_functor/unary/contravariant}
      \cat{C}(\anon*, X): \cat{C}^\oppos \to \ucat{Set}
    \end{equation}
  \end{thmenum}
\end{definition}
\begin{defproof}
  It is sufficient to verify that \eqref{eq:def:hom_functor/binary} defines a functor. \ref{def:functor/CF2} can be seen to hold by inspecting the diagram:
  \begin{equation}\label{eq:def:hom_functor/inv_composition}
    \begin{aligned}
      \includegraphics[page=2]{output/def__hom_functor}
    \end{aligned}
  \end{equation}

  The other functor condition \ref{def:functor/CF1} is straightforward to prove.
\end{defproof}

\begin{definition}\label{def:function_currying}\mcite[259]{TroelstraSchwichtenberg2000Proofs}
  Given the \hyperref[con:function_arguments]{binary} function \( f: A \times B \to C \), we can define another function \( g: A \to \fun(B, C) \) as
  \begin{equation*}
    g(x) \coloneqq (y \mapsto f(x, y)).
  \end{equation*}

  We call this process \term{currying} after Haskell Curry. Obtaining \( f \) from \( g \) can be called \term{uncurrying}.
\end{definition}
\begin{comments}
  \item Currying is useful if we have somehow fixed a value \( x_0 \in A \), in which case we can \enquote{get rid} of one argument by introducing some shortcut for the function \( g(x_0): B \to C \) for the sake of reducing notational clutter. See \fullref{def:differentiability/first_variation} and our proof of \fullref{thm:countably_infinite_union_of_countably_infinite_sets} for example of how this is useful in the wild.
\end{comments}

\begin{proposition}\label{thm:currying_is_natural_isomorphism}
  \hyperref[def:function_currying]{Function currying} is a natural isomorphism between the functors
  \begin{align*}
    &\cat{Set}(A \times B, C)
    &\cat{Set}(A, \cat{Set}(B, C))
  \end{align*}

  More concretely, consider the following functors, which are loosely based on \fullref{def:hom_functor}:
  \begin{equation*}
    \begin{aligned}
      &V: \cat{Set}^3 \to \cat{Set} \\
      &V(A, B, C) \coloneqq \cat{Set}(A \times B, C) \\
      \Big[& V(f: X \to A, g: Y \to B, h: C \to Z) \Big](s: A \times B \to C) \coloneqq \smash{ \overbrace{ (x, y) \mapsto h\parens[\Bigg]{ s\parens[\Big]{ \underbrace{ f(x), g(y) }_{A \times B} } } }^{X \times Y \to Z} } \\
    \end{aligned}
  \end{equation*}
  and
  \begin{equation*}
    \begin{aligned}
      &W: \cat{Set}^3 \to \cat{Set} \\
      &W(A, B, C) \coloneqq \cat{Set}(A, \cat{Set}(B, C)) \\
      \Big[& W(f: X \to A, g: Y \to B, h: C \to Z) \Big](t: A \to \cat{Set}(B, C)) \coloneqq \smash{ \underbrace{ x \mapsto \overbrace{y \mapsto h\parens[\Bigg]{ \overbrace{t(f(x))}^{B \to C}\parens[\Big]{ g(y) } } }^{Y \to Z} }_{X \to \cat{Set}(Y, Z)} } \\
    \end{aligned}
  \end{equation*}

  Then the family of functions
  \begin{equation*}
    \begin{aligned}
      &\alpha: V \Rightarrow W \\
      &\alpha_{A,B,C}(s: A \times B \to C) \coloneqq a \mapsto b \mapsto s(a, b)
    \end{aligned}
  \end{equation*}
  is a \hyperref[thm:natural_isomorphism]{natural isomorphism}.
\end{proposition}
\begin{proof}
  The function \( \varphi \) is clearly invertible. Fix a triple of functions \( f: X \to A \), \( g: Y \to B \) and \( h: C \to Z \). For every \( s: A \times B \to C \) we have
  \begin{align*}
    [W(f, g, h)](\alpha_{A,B,C}(s))
    &=
    x \mapsto y \mapsto \parens[\Big]{ h\parens[\Big]{ [a \mapsto b \mapsto s(a, b)](f(x))(g(y)) } }
    = \\ &=
    x \mapsto y \mapsto h\parens[\Big]{ s(f(x), g(y)) }
    = \\ &=
    \alpha_{X,Y,Z}(V(f, g, h)),
  \end{align*}
  which proves that the following diagram commutes:
  \begin{equation}\label{eq:thm:currying_is_natural_isomorphism/diagram}
    \begin{aligned}
      \includegraphics[page=1]{output/thm__currying_is_natural_isomorphism}
    \end{aligned}
  \end{equation}
\end{proof}

  \subsection{Category equivalences}\label{subsec:category_equivalences}

\begin{remark}\label{rem:category_similarity}
  We have the following notions for expressing that two categories \( \cat{C} \) and \( \cat{D} \) are similar:

  \begin{thmenum}
    \thmitem{rem:category_similarity/equality} Obviously, if \( \cat{C} \) and \( \cat{D} \) are equal, they are similar.

    \thmitem{rem:category_similarity/isomorphism} A slightly less obvious notion is \term{isomorphism of categories}. This is an isomorphism, in the sense of \fullref{def:morphism_invertibility/isomorphism}, in the category \hyperref[def:category_of_small_categories]{\( \ucat{Cat} \)} of small categories for a suitable \hyperref[def:grothendieck_universe]{Grothendieck universe} \( \mscrU \). That is, \( \cat{C} \) and \( \cat{D} \) are isomorphic if there exists an invertible functor between them.

    We rarely distinguish between objects and arrows of isomorphic categories, even if we do not have strict equality in the sense of the \hyperref[def:zfc/extensionality]{axiom of extensionality} in \hyperref[def:zfc]{\logic{ZFC}}.

    Examples of isomorphic categories include \fullref{thm:order_category_isomorphism} and \fullref{ex:isomorphism_of_directed_multigraph_categories}.

    \thmitem{rem:category_similarity/equivalence} A weaker but very useful notion is \term{category equivalence} defined in \fullref{def:category_equivalence}.
  \end{thmenum}
\end{remark}

\begin{example}\label{ex:set_discr_cat_isomorphism}
  There is an isomorphism between the category \( \ucat{Set} \) of small sets and \( \ucat{DiscrCat} \) of all small \hyperref[def:discrete_category]{discrete categories}.

  Consider the pair of functors
  \begin{equation*}
    U: \ucat{DiscrCat} \to \ucat{Set},
  \end{equation*}
  which for any small category \( \cat{C} \) gives us its set of objects \( \obj(\cat{C}) \) and
  \begin{equation*}
    D: \ucat{Set} \to \ucat{DiscrCat},
  \end{equation*}
  which for any small set \( A \) gives us the \hyperref[def:discrete_category]{discrete category} whose set of objects is \( A \).

  These were discussed in \fullref{ex:discrete_category_adjunction}, although with \( \cat{Cat} \) rather than \( \ucat{DiscrCat} \).

  It is obvious that for any small category \( \cat{C} \), the functor \( D \bincirc U \) is bijective on objects. We need to verify that it is bijective on morphisms, however, in order to prove that \( D \bincirc U \) is the identity functor on \( \ucat{DiscrCat} \).

  But any functor \( F: \cat{C} \to \cat{D} \) is completely determined by how \( F \) acts on the objects of \( \cat{C} \). Indeed, the only morphisms in \( \cat{C} \) are the identity morphisms and, by \ref{def:functor/CF1}, every object \( X \) in \( \cat{C} \) determines how the identity \( \id_X \in \cat{C}(X) \) is mapped by \( F \).

  Therefore, \( D \) is a left inverse of \( U \). It is also a right inverse --- for any function \( f: A \to B \) between small sets,
  \begin{equation*}
    [U \bincirc D](f) = D(f)\restr_{A} = f.
  \end{equation*}

  Therefore, the forgetful functor \( U \) is invertible, and its inverse is \( D \).
\end{example}

\begin{definition}\label{def:category_equivalence}\mcite[def. 1.3.15]{Leinster2016Basic}
  An \term{equivalence} between the \hyperref[def:category]{categories} \( \cat{C} \) and \( \cat{D} \) is a quadruple
  \begin{equation}\label{eq:def:category_equivalence/signature}
    \begin{aligned}
                F &: \cat{C} \to \cat{D}, \\
                G &: \cat{D} \to \cat{C}, \\
             \eta &: \id_{\cat{C}} \Rightarrow G \bincirc F, \\
      \varepsilon &: F \bincirc G \Rightarrow \id_{\cat{D}},
    \end{aligned}
  \end{equation}
  where \( \eta \) and \( \varepsilon \) are \hyperref[thm:natural_isomorphism]{natural isomorphisms}.

  We call \( \eta \) the \term{unit} of the equivalence and \( \varepsilon \) the \term{counit}.

  If \( (F, G, \eta, \varepsilon) \) is an equivalence, we say that \( \cat{C} \) and \( \cat{D} \) are equivalent categories. This is justified because equivalence of categories is an equivalence relation --- see \fullref{thm:category_equivalence_is_equivalence_relation}.

  Note that an equivalence is not necessarily an \hyperref[def:category_adjunction]{adjunction}, they simply have a common setup. This is discussed in \fullref{thm:adjoint_equivalence}.
\end{definition}

\begin{proposition}\label{thm:category_equivalence_is_equivalence_relation}
  For every \hyperref[def:grothendieck_universe]{Grothendieck universe} \( \mscrU \), \hyperref[def:category_equivalence]{category equivalence} is an \hyperref[def:equivalence_relation]{equivalence relation} on the set \( \obj(\ucat{Cat}) \)
\end{proposition}
\begin{proof}
  \SubProofOf[def:binary_relation/reflexive]{reflexivity} Clearly \( (\id_{\cat{C}}, \id_{\cat{C}}, \id_{\id_{\cat{C}}}, \id_{\id_{\cat{C}}}]) \) is a self-equivalence for the category \( \cat{C} \).

  \SubProofOf[def:binary_relation/symmetric]{symmetry} If \( (F, G, \eta, \varepsilon) \) is an \hyperref[def:category_equivalence]{equivalence} between the categories \( \cat{C} \) and \( \cat{D} \), then \( (G, F, \varepsilon^{-1}, \eta^{-1}) \) is an equivalence between \( \cat{D} \) and \( \cat{C} \).

  \SubProofOf[def:binary_relation/transitive]{transitivity} By \fullref{thm:def:morphism_invertibility/invertible_composition}, the composition of invertible natural isomorphisms is again a natural isomorphism, hence equivalence in \( \obj(\ucat{Cat}) \) is a transitive relation.
\end{proof}

\begin{proposition}\label{thm:discrete_category_equivalence}
  Let \( \cat{C} \) and \( \cat{D} \) be \hyperref[def:discrete_category]{discrete categories}. Then \( \cat{C} \) and \( \cat{D} \) are \hyperref[def:category_equivalence]{equivalent} if and only if the underlying sets \( \obj(\cat{C}) \) and \( \obj(\cat{D}) \) are \hyperref[def:equinumerosity]{equinumerous}.
\end{proposition}
\begin{proof}
  \SufficiencySubProof Suppose that \( (F, G, \eta, \varepsilon) \) be a category equivalence.

  The unit natural transformation \( \eta: \id_{\cat{C}} \Rightarrow G \bincirc F \) consists of a morphism
  \begin{equation*}
    \eta_A: A \to [G \bincirc F](A)
  \end{equation*}
  for every object \( A \) of \( \cat{C} \). Since the only morphisms in \( \cat{C} \) are the identities, it follows that \( \eta_A = \id_A \) and hence \( [G \bincirc F](A) = A \). In particular, this implies that \( \eta \) is the \hyperref[eq:def:functor_category/identity]{identity natural transformation} on \( \id_{\cat{C}} \) and that the restriction \( G\restr_{\obj(D)} \) is a left inverse of \( F\restr_{\obj(\cat{C})} \).

  Similarly, for the counit \( \varepsilon: F \bincirc G \Rightarrow \id_{\cat{D}} \), for every object \( X \) in \( \cat{D} \) we have \( \varepsilon_X = \id_X \) and hence \( [F \bincirc G](X) = X \). Thus, \( \eta \) is the identity natural transformation on \( \id_{\cat{D}} \) and \( G\restr_{\obj(D)} \) is a right inverse of \( F\restr_{\obj(\cat{C})} \).

  Therefore, the sets \( \obj(\cat{C}) \) and \( \obj(D) \) are equinumerous.

  \NecessitySubProof Suppose that \( F: \obj(\cat{C}) \to \obj(D) \) is a bijective function. Then it is an isomorphism in the category \( \ucat{Cat} \) for an appropriate universe \( \mscrU \), hence it induces an equivalence between \( \cat{C} \) and \( \cat{D} \).
\end{proof}

\begin{proposition}\label{thm:opposite_of_category_equivalence}
  The \hyperref[def:opposite_category]{opposite} of \hyperref[def:category_equivalence]{equivalent categories} are equivalent.

  More precisely, if \( (F, G, \eta, \varepsilon) \) is an \hyperref[def:category_equivalence]{equivalence} between the categories \( \cat{C} \) and \( \cat{D} \), then
  \begin{equation*}
    \begin{aligned}
                G^{\opcat} &: \cat{D}^{\opcat} \to \cat{C}^{\opcat}, \\
                F^{\opcat} &: \cat{C}^{\opcat} \to \cat{D}^{\opcat}, \\
      \varepsilon^{\opcat} &: \id_{\cat{D}} \Rightarrow [F \bincirc G]^{\opcat}, \\
             \eta^{\opcat} &: \underbrace{[G \bincirc F]^{\opcat}}_{G^{\opcat} \bincirc F^{\opcat}} \Rightarrow \id_{\cat{C}^{\opcat}},
    \end{aligned}
  \end{equation*}
  is an equivalence between \( \cat{C}^{\opcat} \) and \( \cat{D}^{\opcat} \).

  This is part of the duality principles listed in \fullref{thm:categorical_principle_of_duality}.
\end{proposition}
\begin{proof}
  Trivial.
\end{proof}

\begin{proposition}\label{thm:equivalence_induces_fully_faithful_and_essentially_surjective_functor}
  In any \hyperref[def:category_equivalence]{category equivalence} \( (F, G, \eta, \varepsilon) \), the functor \( F \) is \hyperref[def:functor_invertibility/fully_faithful]{fully faithful} and \hyperref[def:functor_invertibility/surjective_on_objects]{essentially surjective on objects}.

  The converse of this statement is \fullref{thm:fully_faithful_and_essentially_surjective_functor_induces_equivalence}.
\end{proposition}
\begin{proof}
  \SubProofOf[def:functor_invertibility/surjective_on_objects]{essential surjectivity} For any object \( X \) in \( \cat{D} \), \( A \coloneqq G(X) \) is an object in \( \cat{C} \).

  By definition of category equivalence, the morphism
  \begin{equation*}
    \varepsilon_X: \underbrace{[F \bincirc G](X)}_{F(A)} \to X
  \end{equation*}
  is an isomorphism.

  Therefore, for every object \( X \) in \( \cat{D} \), there exists some object \( A \) in \( \cat{C} \) such that \( F(A) \cong X \). Thus, \( F \) is essentially surjective.

  \SubProofOf[def:functor_invertibility/faithful]{faithfulness} Fix some objects \( A \) and \( B \) in \( \cat{C} \). Let \( f_1: A \to B \) and \( f_2: A \to B \) be morphisms such that \( F(f_1) = F(f_2) \).

  From the naturality of \( \eta \) it follows that the following diagram commutes:
  \begin{equation}\label{eq:thm:equivalence_induces_fully_faithful_and_essentially_surjective_functor/faithfullness}
    \begin{aligned}
      \includegraphics[page=1]{output/thm__equivalence_induces_fully_faithful_and_essentially_surjective_functor}
    \end{aligned}
  \end{equation}

  Therefore, \( \eta_B \bincirc f_1 = \eta_B \bincirc f_2 \) and, since \( \eta_B \) is left-cancellative, \( f_1 = f_2 \).

  \SubProofOf[def:functor_invertibility/full]{fullness} Fix some objects \( A \) and \( B \) in \( \cat{C} \). Let \( g: F(A) \to F(B) \) be an arbitrary morphism.

  We can define a morphism \( f: A \to B \) via the composition
  \begin{equation}\label{eq:thm:equivalence_induces_fully_faithful_and_essentially_surjective_functor/fullness/def}
    \begin{aligned}
      \includegraphics[page=2]{output/thm__equivalence_induces_fully_faithful_and_essentially_surjective_functor}
    \end{aligned}
  \end{equation}

  By naturality of \( \eta \), the following diagram commutes:
  \begin{equation}\label{eq:thm:equivalence_induces_fully_faithful_and_essentially_surjective_functor/fullness/eta_nat}
    \begin{aligned}
      \includegraphics[page=3]{output/thm__equivalence_induces_fully_faithful_and_essentially_surjective_functor}
    \end{aligned}
  \end{equation}

  Therefore,
  \begin{equation*}
     G(g) = [G \bincirc F](f).
  \end{equation*}

  We have already shown that \( F \) is faithful and, by \fullref{thm:category_equivalence_is_equivalence_relation}, \( G \) is also faithful. Since \( g \) and \( F(f) \) are parallel, it follows that they are equal.

  Therefore, \( F \) is full.
\end{proof}

\begin{remark}\label{rem:adjoint_equivalence_induces_fully_faithful_and_essentially_surjective_functor}
  In the fullness proof of \fullref{thm:equivalence_induces_fully_faithful_and_essentially_surjective_functor}, we can use another argument if \( (F, G, \eta, \varepsilon) \) is an \hyperref[def:adjoint_equivalence]{adjoint equivalence}.

  If the triangle diagram \eqref{eq:def:category_adjunction/d_triangle} commutes, the dashed lines in the following diagram also commute:
  \begin{equation}\label{eq:thm:equivalence_induces_fully_faithful_and_essentially_surjective_functor/fullness/triangles}
    \begin{aligned}
      \includegraphics[page=1]{output/rem__adjoint_equivalence_induces_fully_faithful_and_essentially_surjective_functor}
    \end{aligned}
  \end{equation}
\end{remark}

\begin{theorem}[Fully faithful and essentially surjective functor induces equivalence]\label{thm:fully_faithful_and_essentially_surjective_functor_induces_equivalence}
  Every \hyperref[def:functor_invertibility/fully_faithful]{fully faithful} and \hyperref[def:functor_invertibility/surjective_on_objects]{essentially surjective on objects} functor induces a \hyperref[def:category_equivalence]{category equivalence}.

  More precisely, given a functor \( F: \cat{C} \to \cat{D} \) that is fully faithful and essentially surjective on objects, there exists a functor \( G: \cat{D} \to \cat{C} \) and \hyperref[def:natural_transformation]{natural transformations}
  \begin{align*}
    \eta        &: \id_{\cat{C}} \Rightarrow G \bincirc F, \\
    \varepsilon &: F \bincirc G \Rightarrow \id_{\cat{D}},
  \end{align*}
  such that the quadruple \( (F, G, \eta, \varepsilon) \) is a \hyperref[def:category_equivalence]{category equivalence}.

  In \hyperref[def:zfc]{\logic{ZF}}, this theorem is equivalent to the \hyperref[def:zfc/choice]{axiom of choice} --- see \fullref{thm:axiom_of_choice_equivalences/fully_faithful_essentially_surjective}.

  We prove the converse of this statement separately in \fullref{thm:equivalence_induces_fully_faithful_and_essentially_surjective_functor}.
\end{theorem}
\begin{proof}
  \ImplicationSubProof[def:zfc/choice]{the axiom of choice}[thm:fully_faithful_and_essentially_surjective_functor_induces_equivalence]{functors induce equivalences} Suppose that the axiom of choice holds and let \( F \) be a fully faithful functor that is surjective on objects.

  From essential surjectivity of \( F \), it follows that for every object \( X \) in \( \cat{D} \), the preimage of \( X \) under \( F \) is nonempty. The preimage of \( X \) is the set \( \mscrA_X \) of objects in \( \cat{C} \) such that \( A \in \mscrA_X \) if and only if \( F(A) \cong X \). We use the axiom of choice on the family \( \set{ \mscrA_X }_{X \in \cat{D}} \) to select a single preimage for every \( X \), which we denote by \( G(X) \).

  Again using the axiom of choice, we pick an isomorphism \( \varepsilon_X: F(G(X)) \to X \) for every \( X \).

  We have defined a function \( G \) from \( \obj(\cat{D}) \) to \( \obj(\cat{C}) \). In order to \( G \) to become a functor, we must extend it to morphisms. Let \( X \) and \( Y \) be objects in \( \cat{D} \) and \( g: X \to Y \) be any morphism.

  Consider the morphism
  \begin{equation*}
    \varepsilon_Z^{-1} \bincirc f \bincirc \varepsilon_X: [F \bincirc G](X) \to [F \bincirc G](Y).
  \end{equation*}

  Since \( F \) is fully faithful, there exists a unique morphism \( g \) in \( \cat{D}(G(X), G(Y)) \) such that the following diagram commutes:
  \begin{equation}\label{eq:thm:fully_faithful_and_essentially_surjective_functor_induces_equivalence/inverse_morphism_definition}
    \begin{aligned}
      \includegraphics[page=1]{output/thm__fully_faithful_and_essentially_surjective_functor_induces_equivalence}
    \end{aligned}
  \end{equation}

  We define \( G(g) \coloneqq g \).

  In order to prove that \( G \) is a functor, we need to show that \ref{def:functor/CF1} and \ref{def:functor/CF2} hold.

  For \ref{def:functor/CF1}, note that the following diagram commutes for any object \( X \) in \( \cat{D} \):
  \begin{equation}\label{eq:thm:fully_faithful_and_essentially_surjective_functor_induces_equivalence/identity}
    \begin{aligned}
      \includegraphics[page=2]{output/thm__fully_faithful_and_essentially_surjective_functor_induces_equivalence}
    \end{aligned}
  \end{equation}

  Note that \eqref{eq:thm:fully_faithful_and_essentially_surjective_functor_induces_equivalence/inverse_morphism_definition} also commutes if we replace \( [F \bincirc G](\id_X) \) with \( F(\id_{G(X)}) \). Since \( F \) is fully faithful, this morphism is unique and it follows that
  \begin{equation*}
    [F \bincirc G](\id_X) = F(\id_{G(X)}).
  \end{equation*}

  For \ref{def:functor/CF2}, analogously, given morphisms \( g: X \to Y \) and \( f: Y \to Z \), the following diagram commutes:
  \begin{equation}\label{eq:thm:fully_faithful_and_essentially_surjective_functor_induces_equivalence/composition}
    \begin{aligned}
      \includegraphics[page=3]{output/thm__fully_faithful_and_essentially_surjective_functor_induces_equivalence}
    \end{aligned}
  \end{equation}

  We have implicitly used \fullref{rem:inverting_isomorphisms_may_preserve_commutativity} above.

  By the same uniqueness argument used for \ref{def:functor/CF1}, we conclude that
  \begin{equation*}
    G(f \bincirc q) = G(f) \bincirc G(g).
  \end{equation*}

  We have shown that \( G \) is a functor. Furthermore, \( \varepsilon \) is a natural transformation since, for any morphism \( g: X \to Y \) in \( \cat{D} \), the following diagram commutes:
  \begin{equation}\label{eq:thm:fully_faithful_and_essentially_surjective_functor_induces_equivalence/varepsilon}
    \begin{aligned}
      \includegraphics[page=4]{output/thm__fully_faithful_and_essentially_surjective_functor_induces_equivalence}
    \end{aligned}
  \end{equation}

  To show that \( F \) induces an equivalence, it now only remains to define a unit natural transformation \( \eta: \id_{\cat{C}} \to G \bincirc F \). For every object \( A \) in \( \cat{C} \) we have an isomorphism
  \begin{equation*}
    \varepsilon_{F(A)}^{-1}: F(A) \to [F \bincirc G \bincirc F](A).
  \end{equation*}

  Using \( G(\varepsilon_{F(A)}^{-1}) \) will get us nowhere. Fortunately, \( F \) is fully faithful, so there is a bijective function
  \begin{equation*}
    \varphi: \cat{D}(F(A), [F \bincirc G \bincirc F](A)) \to \cat{C}(A, [F \bincirc G](A)).
  \end{equation*}

  Hence, we can define
  \begin{equation*}
    \begin{aligned}
      &\eta: \id_{\cat{C}} \Rightarrow F \bincirc G \\
      &\eta_A \coloneqq \varphi(\varepsilon_{F(A)}^{-1})
    \end{aligned}
  \end{equation*}
  so that \( F(\eta_A) = \varepsilon_{F(A)}^{-1} \). By \fullref{thm:def:functor_invertibility/fully_faithful_reflects_invertible}, since \( \varepsilon_{F(A)} \) is an isomorphism, \( \eta_A \) is also an isomorphism.

  By naturality of \( F(\eta_A) \), the following diagram commutes:
  \begin{equation}\label{eq:thm:fully_faithful_and_essentially_surjective_functor_induces_equivalence/varepsilon_image_nat}
    \begin{aligned}
      \includegraphics[page=5]{output/thm__fully_faithful_and_essentially_surjective_functor_induces_equivalence}
    \end{aligned}
  \end{equation}

  Hence, by \fullref{thm:commutative_diagrams_preserved_and_reflected}, the following diagram also commutes:
  \begin{equation}\label{eq:thm:fully_faithful_and_essentially_surjective_functor_induces_equivalence/varepsilon_source_nat}
    \begin{aligned}
      \includegraphics[page=6]{output/thm__fully_faithful_and_essentially_surjective_functor_induces_equivalence}
    \end{aligned}
  \end{equation}

  Therefore, the quadruple \( (F, G, \eta, \varepsilon) \) is an equivalence of categories.

  \ImplicationSubProof[thm:fully_faithful_and_essentially_surjective_functor_induces_equivalence]{functors induce equivalences}[def:zfc/choice]{the axiom of choice} Let \( \mscrA \) be a family of nonempty sets. Let \( \cat{D} \) be the \hyperref[def:discrete_category]{discrete category} induced by \( \mscrA \).

  Define the category \( \cat{C} \) as follows:
  \begin{itemize}
    \item The \hyperref[def:category/objects]{set of objects} \( \obj(\cat{C}) \) is the \hyperref[def:disjoint_union]{disjoint union} \( \coprod_{A \in \mscrA} A \).

    \item The \hyperref[def:category/morphisms]{set of morphisms} \( \cat{C}((A, x), (B, y)) \) has a single morphism if \( A = B \) and no morphisms otherwise. This single morphism can be encoded as the triple \( (A, x, y) \).

    \item The \hyperref[def:category/composition]{composition of the morphisms} \( (A, x, y) \) and \( (A, y, z) \) is the morphism \( (A, x, z) \).

    \item The \hyperref[def:category/identity]{identity morphism} on the object \( (A, x) \in \cat{C} \) is \( (A, x, x) \).
  \end{itemize}

  Define the functor
  \begin{equation*}
    \begin{aligned}
      &F: \cat{C} \to \cat{D} \\
      &F(A, x) \coloneqq A \\
      &F(A, x, y) \coloneqq \id_A
    \end{aligned}
  \end{equation*}
  that maps each point \( x \in A \in \mscrA \) into the set \( A \) it belongs to. We have taken the disjoint union of \( \mscrA \) since otherwise there may not be a canonical choice of set \( A \) for \( F \) to send \( x \) to. Thus, the functor is surjective on objects (not essentially surjective but actually surjective).

  Note that \( \cat{D}(F(A, x), F(B, y)) \) has a single morphism if \( A = B \) and is empty otherwise. From this it follows that \( F \) is fully faithful.

  Therefore, \( F \) induces a \hyperref[def:category_equivalence]{category equivalence} \( (F, G, \eta, \varepsilon) \). The functor \( G \) chooses an object \( (A, x) \) of \( \cat{C} \) for each object \( A \) of \( \cat{D} \). This induces a \hyperref[def:choice_function]{choice function} on \( \mscrA \).

  We have shown that the axiom of choice holds.
\end{proof}

\begin{definition}\label{def:skeletal_category}\mcite[93]{MacLane1998}
  We say that a category is \term{skeletal} if wherever two objects are isomorphic, they are equal.

  If \( \cat{S} \) is a subcategory of \( \cat{C} \) and if they are \hyperref[def:category_equivalence]{equivalent}, we say that \( \cat{S} \) is a \term{skeleton} of \( \cat{C} \).
\end{definition}

\begin{example}\label{ex:skeleton_of_set}
  For a Grothendieck universe \( \mscrU \), the \hyperref[def:cardinal]{cardinal numbers} in \( \mscrU \) are a \hyperref[def:skeletal_category]{skeletal subcategory} of \( \ucat{Set} \).
\end{example}

\begin{theorem}[Category skeleton existence]\label{thm:category_skeleton_existence}
  Every \hyperref[def:category]{category} has a \hyperref[def:skeletal_category]{skeleton}.

  In \hyperref[def:zfc]{\logic{ZF}}, this theorem is equivalent to the \hyperref[def:zfc/choice]{axiom of choice} --- see \fullref{thm:axiom_of_choice_equivalences/skeletons}.
\end{theorem}
\begin{proof}
  \ImplicationSubProof[def:zfc/choice]{the axiom of choice}[thm:category_skeleton_existence]{skeleton existence} Suppose that the axiom of choice holds and let \( F \) be a fully faithful functor that is surjective on objects.

  Fix a category \( \cat{C} \). We will build a subcategory \( \cat{S} \) of \( \cat{C} \) whose inclusion functor \( \Iota: \cat{S} \to \cat{C} \) is essentially surjective and fully faithful. By \fullref{thm:fully_faithful_and_essentially_surjective_functor_induces_equivalence}, this is sufficient for \( \cat{S} \) and \( \cat{C} \) to be equivalent.

  In order for \( \Iota \) to be a full functor, \( \cat{S} \) must be a full subcategory. Therefore, when building \( \cat{S} \), we can only remove objects and must preserve the morphism sets for the remaining objects.

  Denote by \( \obj(\cat{C}) / \cong \) the quotient of \( \cat{C} \) by the isomorphism relation. Using the axiom of choice, we can obtain a \hyperref[def:choice_function]{choice function} \( c: (\obj(\cat{C}) / \cong) \to \obj(\cat{C}) \).

  Define \( \cat{S} \) as the subcategory induced by the image \( c[\obj(\cat{C}) / \cong] \).

  Now consider the \hyperref[def:subcategory]{inclusion functor} \( \Iota: \cat{S} \to \cat{C} \). For every pair \( X \) and \( Y \) of objects in \( \cat{S} \), clearly
  \begin{equation*}
    \cat{S}(X, Y) = \cat{C}(\Iota(X), \Iota(Y)).
  \end{equation*}

  Hence, \( \Iota \) is fully faithful.

  Now let \( X \) and \( Y \) be objects of \( \cat{C} \). Since the objects of \( \cat{S} \) were chosen from isomorphism classes of \( \cat{C} \), there exist objects \( X' \) and \( Y' \) in \( \cat{S} \) that are isomorphic to \( X \) and \( Y \), correspondingly. Hence, \( \Iota \) is essentially surjective.

  Therefore, \( F \) satisfies \fullref{thm:fully_faithful_and_essentially_surjective_functor_induces_equivalence}, from which is follows that \( \cat{C} \) and \( \cat{S} \) are equivalent.

  \ImplicationSubProof[thm:category_skeleton_existence]{skeleton existence}[def:zfc/choice]{the axiom of choice} Suppose that every category has a skeleton.

  Let \( \mscrA \) be a family of nonempty sets. Construct a category \( \cat{C} \)from the \hyperref[def:disjoint_union]{disjoint union} \( \coprod_{A \in \mscrA} A \), where a morphism exists only between members of the same set. This construction is performed in detail in our proof of \fullref{thm:fully_faithful_and_essentially_surjective_functor_induces_equivalence}.

  Then \( \cat{C} \) has a skeleton \( \cat{S} \). All morphisms in \( \cat{C} \) are isomorphisms, hence the set \( \obj(\cat{S}) \) contains exactly one representative for each set in the family \( \mscrA \).

  More precisely, define the set
  \begin{equation*}
    S \coloneqq \set{ x \given (A, x) \in \obj(\cat{S}) }.
  \end{equation*}

  Then \( S \) satisfies \fullref{thm:axiom_of_choice_equivalences/choice_sets}.

  Since the family \( \mscrA \) is arbitrary, we conclude that the axiom of choice holds.
\end{proof}

\begin{remark}\label{rem:skeletons_and_preorder_categories}
  Rather than defining representatives of equivalence classes, as in \fullref{thm:category_skeleton_existence}, we can define morphisms between the equivalence classes themselves, as in \fullref{thm:preorder_to_partial_order}.

  This does not require the \hyperref[def:zfc/choice]{axiom of choice}, but is rarely applicable, unfortunately. One case where it is applicable is in \hyperref[def:preorder_category]{preorder categories} --- see \fullref{thm:order_category_isomorphism}.
\end{remark}

\begin{definition}\label{def:groupoid}
  A \term{groupoid} is a category whose only morphisms are \hyperref[def:morphism_invertibility/isomorphism]{isomorphisms}.
\end{definition}

\begin{definition}\label{def:monoid_delooping}\mcite[def. 1.1.7]{Perrone2019}
  Let \( (M, \cdot, e) \) be a \hyperref[def:monoid]{monoid}. The \term{delooping} \( \cat{B}_M \) of \( M \) is the following category:
  \begin{itemize}
    \item The \hyperref[def:category/objects]{set of objects} \( \obj(\cat{B}_M) \) is the singleton set \( \set{ \Anon } \), where \( \Anon \) is any set not in the set of all morphisms \( M \).

    \item The only \hyperref[def:category/morphisms]{set of morphisms} \( \cat{B}_M(\Anon) \) is the underlying set \( M \) of the monoid.

    \item The \hyperref[def:category/composition]{composition of the morphisms} \( x \) and \( y \) is the multiplication:
    \begin{equation*}
      y \bincirc x \coloneqq y \cdot x.
    \end{equation*}

    Note how we write composition in the same order as multiplication. This may seem to contradict the general convention, however it is consistent with groups being regarded as sets of invertible transformations.

    \item The \hyperref[def:category/identity]{identity morphism} is \( e \).
  \end{itemize}
\end{definition}

\begin{proposition}\label{thm:delooping_of_group}
  The \hyperref[def:monoid_delooping]{delooping} of a \hyperref[def:group]{group} \( G \) is a \hyperref[def:groupoid]{groupoid}.
\end{proposition}
\begin{comments}
  \item There is a restricted form of a converse --- see \fullref{thm:connected_delooping}.
\end{comments}
\begin{proof}
  Trivial.
\end{proof}

\begin{definition}\label{def:connected_category}\mcite[88]{MacLane1998}
  We say that a \hyperref[def:category]{category} is \term{connected} its \hyperref[def:category]{underlying multigraph} is \hyperref[def:graph_connectedness/weak]{weakly connected}.
\end{definition}

\begin{proposition}\label{thm:connected_delooping}
  For every \hyperref[def:connected_category]{connected} \hyperref[def:groupoid]{groupoid} \( \cat{G} \) there exists a \hyperref[def:group]{group} \( G \) such that \( \cat{G} \) is \hyperref[def:category_equivalence]{equivalent} to the \hyperref[def:monoid_delooping]{delooping} \( \cat{B}_G \). Furthermore, if \( \cat{G} \) has only one object, then this equivalence is an \hyperref[rem:category_similarity/isomorphism]{isomorphism}.
\end{proposition}
\begin{comments}
  \item See \fullref{thm:delooping_of_group} for a much simpler converse.
\end{comments}
\begin{proof}
  \SubProof{Proof that all objects are isomorphic} Fix two objects \( A \) and \( B \) in \( \cat{G} \). We will show that they are isomorphic.

  Since \( \cat{G} \) is connected, there exists an undirected path connecting \( A \) with \( B \). Since \( \cat{G} \) is a groupoid, all morphisms in the path are invertible, and hence there also exists a directed path from \( A \) to \( B \). This directed path can be composed into an isomorphism from \( A \) to \( B \).

  Therefore, all objects in \( \cat{G} \) are isomorphic.

  \SubProof{Construction of group} \Fullref{thm:category_skeleton_existence} implies that there exists a \hyperref[def:skeletal_category]{skeleton} \( \cat{S} \) of \( \cat{G} \). Let \( (F, G, \eta, \varepsilon) \) be the equivalence between \( \cat{G} \) and \( \cat{S} \).

  Since all objects in \( \cat{G} \) are isomorphic, \( \cat{S} \) has a single object \( \anon \). Hence, it is possible to compose any two morphisms in \( \cat{S} \).

  Define the group \( G \) as follows:
  \begin{itemize}
    \item Let the underlying set of \( G \) be the set of all morphisms of \( \cat{S} \).
    \item Define the group operation as
    \begin{equation*}
      x \cdot y \coloneqq x \bincirc y.
    \end{equation*}
  \end{itemize}

  It follows that the identity element on \( G \) is then the identity morphism on \( \anon \) and that the inverse of \( x \) is the inverse morphism \( x^{-1} \).

  \SubProof{Proof of equivalence} Note that the skeleton \( \cat{S} \) and the delooping \( \cat{B}_G \) are equal. Since \( \cat{G} \) and \( \cat{S} \) are equivalent, so are \( \cat{G} \) and \( \cat{B}_G \)
\end{proof}

\begin{definition}\label{def:preorder_category}\mcite[11]{MacLane1998}
  We will say that a \hyperref[def:category]{category} is a \term{preorder category} if any two parallel morphisms are equal.

  This is equivalent to saying that the function for every two objects \( A \) and \( B \) in \( \cat{P} \), whenever the set \( \cat{P}(A, B) \) is at most a singleton.
\end{definition}
\begin{comments}
  \item Preorder categories are often conflated with preordered sets due to \fullref{thm:order_category_isomorphism}.

  \item As shown in \fullref{ex:preorder_nonuniqueness} and discussed in \fullref{thm:order_category_isomorphism}, a preorder category may not be \hyperref[def:skeletal_category]{skeletal}.
\end{comments}

\begin{theorem}[Ordered sets as categories]\label{thm:order_category_isomorphism}
  The category \( \cat{PreOrd} \) of small \hyperref[def:preordered_set]{preordered sets} and (nonstrict) \hyperref[def:preordered_set/homomorphism]{monotone maps} is \hyperref[rem:category_similarity/isomorphism]{isomorphic} to the category of small \hyperref[def:preorder_category]{preorder categories} and their functors.

  Furthermore, \hyperref[def:partially_ordered_set]{partially ordered sets} correspond to \hyperref[def:skeletal_category]{skeletal} preorder categories.
\end{theorem}
\begin{proof}
  \SubProof{Proof that preordered sets induce preorder categories} Let \( (P, \leq) \) be a small preordered set. Let \( \cat{P} \) be the \hyperref[def:directed_multigraph_free_category]{free category} obtained by regarding \( (P, \leq) \) as a \hyperref[def:directed_multigraph]{directed multigraph}. Explicitly, the category \( \cat{P} \) is built as follows:
  \begin{itemize}
    \item The \hyperref[def:category/objects]{set of objects} \( \obj(\cat{P}) \) is simply \( P \).

    \item The \hyperref[def:category/morphisms]{set of morphisms} \( \cat{P}(x, y) \) consists of the tuple \( (x, y) \) if \( x \leq y \) and is empty otherwise.

    \item The \hyperref[def:category/composition]{composition of the morphisms} \( (x, y) \) and \( (y, z) \) is simply \( (x, y) \). This is well-defined because of the \hyperref[def:binary_relation/transitive]{transitivity} of \( \leq \).

    \item The \hyperref[def:category/identity]{identity morphism} on the object \( x \in \cat{C} \) is \( (x, x) \). This is well-defined because of the \hyperref[def:binary_relation/reflexive]{reflexivity} of \( \leq \).
  \end{itemize}

  This is a preorder category because \( \leq \) is a binary relation and ordered tuples with the same elements are equal.

  Now let \( f: P \to Q \) be a nonstrict monotone map. It induces the functor
  \begin{equation*}
    \begin{aligned}
      &F: \cat{P} \to \cat{Q} \\
      &F(x) \coloneqq f(x) \\
      &F(x, y) \coloneqq (f(x), f(y)).
    \end{aligned}
  \end{equation*}

  \ref{def:functor/CF1} is immediate and \ref{def:functor/CF2} follows from \eqref{eq:def:order_homomorphism/increasing}, hence \( F \) is indeed a functor.

  \SubProof{Proof that preorder categories induce preordered sets} Let \( \cat{P} \) be a small category. Define the binary relation
  \begin{equation*}
    X \leq Y \T{if and only if} \cat{P}(X, Y) \neq \varnothing.
  \end{equation*}

  This is a binary relation over the set \( P \coloneqq \obj(\cat{P}) \). It is reflexive because of the existence of identity morphisms in \( \cat{P} \) and transitive because of the requirement that the composition of compatible morphisms exists.

  Therefore, \( (P, \leq) \) is a preordered set.

  Given a functor \( F: \cat{P} \to \cat{Q} \), the restriction \( F\restr_{\obj(\cat{P})} \) is a monotone map from \( (P, \leq_P) \) to \( (Q, \leq_Q) \).

  Indeed, if \( X \leq Y \) for \( X, Y \in P \), then \( \cat{P}(X, Y) \neq \varnothing \). Hence, \( \cat{P}(F(X), F(Y)) \neq \varnothing \) and \( F(X) \leq F(Y) \).

  \SubProof{Proof of category isomorphism} We have implicitly defined a functor between the categories of preordered sets and of preorder categories. Isomorphism requires that these functors are mutually inverse. The verification of this is trivial and we only mention it to highlight that we have not ignored this issue.

  \SubProof{Proof for partial orders} For a preorder category, there is an isomorphism between \( x \) and \( y \) if and only if there is both a morphism from \( x \) to \( y \) and one from \( y \) to \( x \). This isomorphism may not be unique as a consequence of \fullref{ex:preorder_nonuniqueness}. Uniqueness requires \( x = y \) to hold in this case, which is in turn equivalent to partial order \hyperref[def:binary_relation/antisymmetric]{antisymmetry}.
\end{proof}

\begin{proposition}\label{thm:order_category_isomorphism_properties}
  Let \( (P, \leq) \) be a \hyperref[def:partially_ordered_set]{partially ordered set} and let \( \cat{P} \) be its corresponding category induced by \fullref{thm:order_category_isomorphism}.

  \begin{thmenum}
    \thmitem{thm:order_category_isomorphism_properties/opposite} The \hyperref[def:opposite_category]{opposite category} \( \cat{P}^{\opcat} \) corresponds to the \hyperref[thm:preorder_duality]{opposite partially ordered set} \( (P, \geq) \).

    \thmitem{thm:order_category_isomorphism_properties/universal} If \( \cat{P} \) has an \hyperref[def:universal_objects/initial]{initial object}, since it is a skeletal category, this initial object is unique.

    An object \( I \) is an initial object if and only if it is the \hyperref[def:extremal_points/top_and_bottom]{bottom} of \( P \).

    \hyperref[thm:categorical_principle_of_duality]{Dually}, \( T \) is a \hyperref[def:universal_objects/terminal]{terminal object} if and only if it is the \hyperref[def:extremal_points/top_and_bottom]{top} of \( P \).
  \end{thmenum}
\end{proposition}
\begin{proof}
  Trivial.
\end{proof}

  \section{Category adjunctions}\label{sec:category_adjunctions}

\begin{concept}\label{con:free_construction}
  \todo{Free constructions}
\end{concept}

\begin{remark}\label{rem:adjoint_functors}
  Suppose we have the functors \( F: \cat{C} \to \cat{D} \) and \( G: \cat{D} \to \cat{C} \).

  \begin{itemize}
    \item If \( F \) is a left inverse to \( G \), given only \( G(X) \), \( F \) can restore \( X \).

    \item If \( F \) is instead a left adjoint to \( G \), given \( G(X) \) and some object \( A \) in \( \cat{C} \), \( F \) can give us an object \( F(A) \) such that the morphisms from \( F(A) \) to \( X \) are uniquely defined by those from \( A \) to \( G(X) \). Thus, \( F \) cannot restore \( X \), but it can give us objects in \( \cat{D} \) that act on \( X \) as their images under \( G \) would act on \( G(X) \) in \( \cat{C} \). This is described in \cite{StanfordPlato:category_theory} as \( F \) being a \enquote{conceptual inverse} of \( G \).

    \item If \( F \) is a right adjoint to \( G \), the morphisms from \( X \) to \( F(A) \) are determined by those from \( G(X) \) to \( A \).
  \end{itemize}

  \Fullref{def:category_adjunction} contains two equivalent definition of an adjunction, and \fullref{rem:universal_mapping_property} describes how they can be characterized via universal mapping properties.
\end{remark}

\begin{definition}\label{def:category_adjunction}\mcite[ch. 2]{Leinster2014BasicCategories}
  \todo{Change notation as per \fullref{thm:object_presentation_generators} --- swap \( \cat{C} \) and \( \cat{D} \)}.

  An \term{adjunction} between the \hyperref[def:category]{categories} \( \cat{C} \) and \( \cat{D} \) can be defined in several equivalent ways. Let \( F: \cat{C} \to \cat{D} \) and \( G: \cat{D} \to \cat{C} \) be arbitrary functors.

  In both cases below, if there exists an adjunction between \( F \) and \( G \), we say that \( F \) is \term{left adjoint} to \( G \) and, correspondingly, that \( G \) is \term{right adjoint} to \( F \). A conventional notation for adjoint functors is \( F \dashv G \).

  \begin{thmenum}
    \thmitem{def:category_adjunction/hom} A \term{hom-adjunction} is a triple \( (F, G, \varphi) \), where \( \varphi \) is \hyperref[thm:natural_isomorphism]{natural isomorphism}
    \begin{equation}\label{eq:def:category_adjunction/hom}
      \varphi: \cat{D}(F(\anon*), \anon*) \Rightarrow \cat{C}(\anon*, G(\anon*)).
    \end{equation}

    The functors
    \begin{align*}
      &\cat{D}(F(\anon*), \anon*): \cat{C}^\oppos \times \cat{D} \to \cat{Set}, \\
      &\cat{C}(\anon*, G(\anon*)): \cat{C}^\oppos \times \cat{D} \to \cat{Set}
    \end{align*}
    are straightforward modifications of the \hyperref[eq:def:hom_functor/binary]{binary hom-functor} on \( \cat{C} \).

    Naturality of \( \varphi \) means that, for every two morphisms \( f: B \to A \) in \( \cat{C} \) and \( g: X \to Y \) in \( \cat{D} \), the following diagram commutes:
    \begin{equation}\label{eq:def:category_adjunction/varphi_nat_diagram}
      \begin{aligned}
        \includegraphics[page=1]{output/def__category_adjunction}
      \end{aligned}
    \end{equation}

    The above diagram reduces to the simpler to verify condition that, for every triple of morphisms \( f: B \to A \), \( s: F(A) \to X \) and \( g: X \to Y \), we must have the equality
    \begin{equation}\label{eq:def:category_adjunction/varphi_nat}
      G(g) \bincirc \varphi_{A,X}(s) \bincirc f = \varphi_{B,Y}(g \bincirc s \bincirc F(f)).
    \end{equation}

    \thmitem{def:category_adjunction/unit_counit} A \term{unit-counit adjunction} is a quadruple \( (F, G, \eta, \varepsilon) \), where
    \begin{equation}\label{eq:def:category_adjunction/unit_counit/signature}
      \begin{aligned}
               \eta &: \id_{\cat{C}} \Rightarrow G \bincirc F, \\
        \varepsilon &: F \bincirc G \Rightarrow \id_{\cat{D}}
      \end{aligned}
    \end{equation}
    are natural transformations satisfying the condition that, for any pair of objects \( A \) in \( \cat{C} \) and \( X \) in \( \cat{D} \), the following triangle diagrams commute:

    \begin{minipage}{0.43\textwidth}
      \begin{equation}\label{eq:def:category_adjunction/d_triangle}
        \begin{aligned}
          \includegraphics[page=2]{output/def__category_adjunction}
        \end{aligned}
      \end{equation}
    \end{minipage}
    \hfill
    \begin{minipage}{0.44\textwidth}
      \raggedright
      \begin{equation}\label{eq:def:category_adjunction/c_triangle}
        \begin{aligned}
          \includegraphics[page=3]{output/def__category_adjunction}
        \end{aligned}
      \end{equation}
    \end{minipage}
    \smallskip

    Note that an adjunction is not an \hyperref[def:category_equivalence]{equivalence}, they simply have a common setup. Similarly to \hyperref[def:category_equivalence]{equivalence}, we call the \hyperref[def:natural_transformation]{natural transformation} \( \eta \) the \term{unit} of the adjunction and \( \varepsilon \) the \term{counit}.
  \end{thmenum}
\end{definition}
\begin{defproof}
  \ImplicationSubProof{def:category_adjunction/hom}{def:category_adjunction/unit_counit} Let \( (F, G, \varphi) \) be a hom-adjunction.

  For every morphism \( f: B \to A \) in \( \cat{C} \), from the naturality of \( \varphi \) we have
  \begin{equation}\label{eq:def:category_adjunction/varphi_eta}
    \begin{aligned}
      \includegraphics[page=4]{output/def__category_adjunction}
    \end{aligned}
  \end{equation}

  Since \( \varphi_{A,F(B)} \) is a morphism in \( \cat{Set} \), it is a function, and we can apply it in order to define the family
  \begin{equation*}
    \begin{aligned}
      &\eta: \id_{\cat{C}} \Rightarrow G \bincirc F, \\
      &\eta_A \coloneqq \varphi_{A,F(A)}(\id_{F(A)}).
    \end{aligned}
  \end{equation*}

  We must show that \( \eta \) is a natural transformation. On the diagram \eqref{eq:def:category_adjunction/varphi_eta}, we can start in the top left corner with \( F(\id_A) \) and top right corner with \( F(\id_B) \) and reach the middle.

  We obtain that,
  \begin{equation*}
    \cat{C}(f, [G \bincirc F](\id_A))\parens[\Big]{ \underbrace{\varphi_{A, F(A)}(\id_A)}_{\eta_A} }
    =
    \eta_A \bincirc f
  \end{equation*}
  and
  \begin{equation*}
    \cat{C}(\id_B, [G \bincirc F](f))\parens[\Big]{ \underbrace{\varphi_{B, F(B)}(\id_B)}_{\eta_B} }
    =
    [G \bincirc F](f) \bincirc \eta_B
  \end{equation*}
  are equal. That is, the following diagram commutes:
  \begin{equation}\label{eq:def:category_adjunction/eta_nat}
    \begin{aligned}
      \includegraphics[page=5]{output/def__category_adjunction}
    \end{aligned}
  \end{equation}

  In order to define the natural transformation \( \varepsilon: F \bincirc G \Rightarrow \id_{\cat{D}} \), we use the inverse transformation \( \varphi^{-1} \). For every morphism \( g: X \to Y \) in \( \cat{D} \), we have
  \begin{equation}\label{eq:def:category_adjunction/varphi_varepsilon}
    \begin{aligned}
      \includegraphics[page=6]{output/def__category_adjunction}
    \end{aligned}
  \end{equation}

  Define the family
  \begin{equation*}
    \begin{aligned}
      &\varepsilon: F \bincirc G \Rightarrow \id_{\cat{D}}, \\
      &\varepsilon_X \coloneqq \varphi_{G(X),X}^{-1}(\id_{G(X)}).
    \end{aligned}
  \end{equation*}

  We can prove that \( \varepsilon \) is a natural transformation analogously to how we proved it for \( \eta \), and we will skip the details.

  We will now show that the triangle diagram \eqref{eq:def:category_adjunction/d_triangle} commutes. Consider the morphism \( (\eta_A, F(\id_A)) \) in \( \cat{C}^\oppos \times \cat{D} \). Applying the functors \( \cat{D}(F(\anon*), \anon*) \) and \( \cat{D}(\anon*, G(\anon*)) \) to this morphism and using the naturality of \( \varphi \), we obtain
  \begin{equation}\label{eq:def:category_adjunction/d_triangle_proof}
    \begin{aligned}
      \includegraphics[page=7]{output/def__category_adjunction}
    \end{aligned}
  \end{equation}

  Note that \( \varepsilon_{F(A)} \) is a member of \( \cat{D}([F \bincirc G \bincirc F](A), F(A)) \).

  Composing the functions in \eqref{eq:def:category_adjunction/d_triangle_proof} in one direction, we obtain
  \begin{balign*}
    &\phantom{{}={}}
    \varphi_{A,F(A)}^{-1} \parens[\Bigg]{ \cat{C}\parens[\Big]{ \eta_A, [G \bincirc F](\id_A) } \parens[\Big]{ \varphi_{[G \bincirc F](A),F(A)} (\varepsilon_{F(A)}) } }
    = \\ &=
    \varphi_{A,F(A)}^{-1} \parens[\Bigg]{ \parens[\Big]{ \varphi_{[G \bincirc F](A),F(A)} (\varepsilon_{F(A)}) } \bincirc \eta_A }
    = \\ &=
    \varphi_{A,F(A)}^{-1} \parens[\Big]{ \id_{[G \bincirc F](A)} \bincirc \eta_A }
    = \\ &=
    \id_{F(A)}.
  \end{balign*}

  Composing the functions in \eqref{eq:def:category_adjunction/d_triangle_proof} in the other direction, we obtain
  \begin{equation*}
    \cat{D}\parens[\Big]{ F(\eta_A), F(\id_A) } (\varepsilon_{F(A)})
    =
    \varepsilon_{F(A)} \bincirc F(\eta_A).
  \end{equation*}

  Therefore,
  \begin{equation*}
    \id_{F(A)} = \varepsilon_{F(A)} \bincirc F(\eta_A).
  \end{equation*}
  and thus \eqref{eq:def:category_adjunction/d_triangle} commutes.

  We can similarly prove that \eqref{eq:def:category_adjunction/c_triangle} commutes.

  Therefore, \( (F, G, \eta, \varepsilon) \) is a unit-counit adjunction.

  \ImplicationSubProof{def:category_adjunction/unit_counit}{def:category_adjunction/hom} Let \( (F, G, \eta, \varepsilon) \) be a unit-counit adjunction.

  For every pair of objects \( A \in \cat{C} \) and \( X \in \cat{D} \), define the functions
  \begin{equation*}
    \begin{aligned}
      &\varphi_{A,X}: \cat{D}(F(A), X) \to \cat{C}(A, G(X)) \\
      &\varphi_{A,X}(g) \coloneqq G(g) \bincirc \eta_A.
    \end{aligned}
  \end{equation*}
  and
  \begin{equation*}
    \begin{aligned}
      &\psi_{A,X}^{-1}: \cat{C}(A, G(X)) \to \cat{D}(F(A), X) \\
      &\psi_{A,X}^{-1}(f) \coloneqq \varepsilon_X \bincirc F(f),
    \end{aligned}
  \end{equation*}

  From the naturality of \( \varepsilon \) and from \eqref{eq:def:category_adjunction/d_triangle}, it follows that the following diagram commutes:
  \begin{equation}\label{eq:def:category_adjunction/varphi_inverse_def}
    \begin{aligned}
      \includegraphics[page=8]{output/def__category_adjunction}
    \end{aligned}
  \end{equation}

  Therefore,
  \begin{equation*}
    g = \varepsilon_X \bincirc \underbrace{[F \bincirc G](f) \bincirc F(\eta_A)}_{F(\varphi_{A,X}(g))} = [\varphi_{A,X} \bincirc \varphi_{A,X}](g)
  \end{equation*}
  and thus \( \psi_{A,X} \) is a left inverse of \( \varphi_{A,X} \).

  We can analogously show that \( \psi_{A,X} \) is a right inverse, and hence that \( \varphi_{A,X} \) is invertible.

  Since we have already shown that \( \varphi \) is a bijective function, it remains to verify the naturality of \( \varphi \) in order to show that it is a natural isomorphism. Let \( f: B \to A \) be a morphism in \( \cat{C} \) and \( g: X \to Y \) be a morphism in \( \cat{D} \). Fix some morphism \( s: F(A) \to X \).

  Composing the functions of \eqref{eq:def:category_adjunction/varphi_nat_diagram} in one direction, we obtain
  \begin{equation}\label{eq:def:category_adjunction/varphi_nat_diagram_chase_right}
    \varphi_{B, Y}\parens[\Big]{ \cat{D}(F(f), g)(s) }
    =
    \varphi_{B, Y}\parens[\Big]{ g \bincirc s \bincirc F(f) }
    =
    G(g) \bincirc G(s) \bincirc [G \bincirc F](f) \bincirc \eta_B.
  \end{equation}

  In the other direction, we have
  \begin{equation}\label{eq:def:category_adjunction/varphi_nat_diagram_chase_down}
    \cat{C}(f, G(g))\parens[\Big]{ \varphi_{A, X}(s) }
    =
    \cat{C}(f, G(g))\parens[\Big]{ G(s) \bincirc \eta_A }
    =
    G(g) \bincirc G(s) \bincirc \eta_A \bincirc f.
  \end{equation}

  From the naturality of \( \eta \), we have that \eqref{eq:def:category_adjunction/eta_nat} commutes and hence
  \begin{equation*}
    \eta_A \bincirc f
    =
    [G \bincirc F](s) \bincirc \eta_B.
  \end{equation*}

  Therefore, \eqref{eq:def:category_adjunction/varphi_nat_diagram_chase_right} and \eqref{eq:def:category_adjunction/varphi_nat_diagram_chase_down} are equal and, thus, \eqref{eq:def:category_adjunction/varphi_nat_diagram} also commutes.

  This proves the naturality of \( \varphi \).
\end{defproof}

\begin{proposition}\label{thm:category_adjunction_duality}
  The functor \( F: \cat{C} \to \cat{D} \) is \hyperref[def:category_adjunction]{left adjoint} to \( G: \cat{D} \to \cat{C} \) if and only if the \hyperref[def:opposite_functor]{dual functor} \( F^\oppos \) is right adjoint to \( G^\oppos \).

  This is part of the duality principles listed in \fullref{thm:categorical_principle_of_duality}.
\end{proposition}
\begin{proof}
  \begin{equation*}
    \cat{C^\oppos}(G^\oppos(X), A) = \cat{C}(A, G(X)) \cong \cat{D}(F(A), X) = \cat{D^\oppos}(X, F^\oppos(A)).
  \end{equation*}
\end{proof}

\begin{definition}\label{def:concrete_category}\mcite[26]{MacLane1998Categories}
  A \term{concrete category} is a pair \( (\cat{C}, U) \), where \( \cat{C} \) is a category and \( U: \cat{C} \to \cat{Set} \) is a \hyperref[def:functor_invertibility/faithful]{faithful functor} that gives us a set for any object of \( \cat{C} \). More generally, a pair \( (\cat{C}, U) \), where \( U: \cat{C} \to \cat{D} \), is a concrete category over \( \cat{D} \).

  In the case of a concrete category over \( \cat{Set} \), we can regard any morphism \( f \) in \( \cat{C} \) as a function. Thus, for example, we can say that \( f \) is an \hyperref[def:function_invertibility/injective]{injective} morphism if \( U(f) \) is an injective function. This is discussed further in \fullref{thm:concrete_category_function_invertibility}.

  In the context of a concrete category, we call \( U \) a \term{forgetful functor} and any \hyperref[def:category_adjunction]{left adjoint} to \( U \) functor a \term{free functor}. According to Jean-Pierre Marquis in \cite{StanfordPlato:category_theory}, the motivation for this terminology is that free functors build objects that are free from additional restrictions.

  We list several examples in \fullref{ex:def:category_adjunction}. The forgetful functor is usually clear from the context, and we identify a concrete category \( (\cat{C}, U) \) with its underlying set \( \cat{C} \). The corresponding free functor, however, often requires a nontrivial but straightforward construction.
\end{definition}

\begin{remark}\label{rem:concrete_categories}
  Unless it is important, in \hyperref[def:concrete_category]{concrete categories} over \( \cat{Set} \), we will ignore writing the forgetful functor and always assume that morphism composition is function composition. This automatically implies that the identity morphism is the identity function.
\end{remark}

\begin{example}\label{ex:def:category_adjunction}
  We list some examples of \hyperref[def:category_adjunction]{category adjunctions}. Note that only some of them are commonly referred to as \enquote{free}.

  \begin{thmenum}
    \thmitem{ex:def:category_adjunction/set_top} Perhaps the simplest meaningful example of an adjunction is the \hyperref[def:discrete_topology]{discrete topology} functor \( D: \cat{Set} \to \cat{Top} \), which is left adjoint to the forgetful functor \( U: \cat{Top} \to \cat{Set} \), which maps a small \hyperref[def:topological_space]{topological space} \( (X, \mscrT) \) into its underlying set \( X \).

    Given a set \( A \) and a topological space \( (X, \mscrT) \), every function \( s: A \to X \) is \hyperref[def:global_continuity]{continuous} when \( A \) is endowed with the discrete topology. Conversely, every continuous function is obviously a \hyperref[def:function]{function}. It follows that there is an equality
    \begin{equation*}
      \cat{Top}\parens[\Big]{ \underbrace{(A, \pow(A))}_{D(A)}, (X, \mscrT) } = \cat{Set}\parens[\Big]{ A, X }.
    \end{equation*}

    Therefore, \( (D, U, \id) \) is a hom-adjunction. Furthermore, \( (D, U, \id, \id) \) is a unit-counit adjunction.

    \thmitem{ex:def:category_adjunction/top_set} The \hyperref[def:indiscrete_topology]{indiscrete topology} functor \( I: \cat{Set} \to \cat{Top} \) is right-adjoint to the same forgetful functor \( U: \cat{Top} \to \cat{Set} \), again with identities for all natural transformations of the adjunction.

    Therefore, we have
    \begin{equation*}
      D \dashv U \dashv I.
    \end{equation*}

    \thmitem{ex:def:category_adjunction/set_cat} We discussed in \fullref{ex:discrete_category_adjunction} the \hyperref[def:discrete_category]{discrete category} functor \( D: \cat{Set} \to \cat{Cat} \). We showed in \fullref{ex:set_discr_cat_isomorphism} that, when restricted to the subcategory \( \cat{DiscrCat} \) rather than \( \cat{Cat} \), \( D \) it is an inverse to the forgetful functor \( U \). In the general case, however, this is an adjunction rather than an isomorphism. More precisely, \( D \) is left adjoint to \( U \).

    Note that for any functor \( F: \cat{C} \to \cat{D} \), we have \( U(F) \coloneqq F\restr_{\obj(C)} \). Thus, \( U \) is not only a functor in \( [\cat{Cat}, \cat{Set}] \); it also induces a natural isomorphism between the functors \( \cat{Cat}(D(\anon*), \cat{\anon*}) \) and \( \cat{Set}(\anon*, U(\anon*)) \).

    Indeed, fix a small category \( \cat{C} \) and a set \( A \). From our discussion in \fullref{ex:set_discr_cat_isomorphism} it is obvious that the restriction
    \begin{equation*}
      U: \cat{Cat}(D(A), \cat{C}) \to \cat{Set}(A, U(\cat{C}))
    \end{equation*}
    is a bijective function.

    In order to verify the naturality of the transformation induced by \( U \), we must show that for any function \( f: B \to A \) and functor \( F: \cat{C} \to \cat{D} \), the following diagram commutes
    \begin{equation}\label{eq:ex:def:category_adjunction/set_cat/u_nat}
      \begin{aligned}
        \includegraphics[page=1]{output/ex__def__category_adjunction}
      \end{aligned}
    \end{equation}

    The commutativity of \eqref{eq:ex:def:category_adjunction/set_cat/u_nat} follows from the following: for every functor \( S: D(A) \to \cat{C} \) we have
    \begin{equation*}
      U(F \bincirc S \bincirc D(f))
      =
      U(F) \bincirc U(S) \bincirc U(D(f))
      =
      U(F) \bincirc U(S) \bincirc f.
    \end{equation*}

    Therefore, \( (D, U, U) \) is a \hyperref[def:category_adjunction/hom]{hom-adjunction}.

    We can also explicitly define a unit-counit adjunction. The unit \( \eta: \id_{\cat{Set}} \Rightarrow U \bincirc D \) is simply the identity.

    The counit is slightly more interesting. Given a small category \( \cat{C} \), applying \( D \bincirc U \) gives us the subcategory consisting only of the objects and identity morphisms of \( \cat{C} \). Then the counit \( \varepsilon: D \bincirc U \Rightarrow \id_{\cat{Cat}} \) is simply the inclusion functor \( \Iota \) from this subcategory to \( \cat{C} \).

    The triangle
    \begin{equation}\label{eq:ex:def:category_adjunction/set_cat/triangles}
      \begin{aligned}
        \includegraphics[page=2]{output/ex__def__category_adjunction}
        \quad\quad
        \includegraphics[page=3]{output/ex__def__category_adjunction}
      \end{aligned}
    \end{equation}
    corresponding to \eqref{eq:def:category_adjunction/d_triangle} and \eqref{eq:def:category_adjunction/c_triangle}, obviously commute.

    The quadruple \( (D, U, \eta, \varepsilon) \) is a \hyperref[def:category_adjunction/unit_counit]{unit-counit adjunction}.

    \thmitem{ex:def:category_adjunction/dm_cat} The left adjoint of the forgetful functor \( U \) from \( \cat{Cat} \) to the \hyperref[def:directed_multigraph/category]{category of directed multigraphs}, which we will denote by \( \cat{DMGph} \) is the free category functor defined in \fullref{def:directed_multigraph_free_category}. We denote this functor by \( F \).

    We can define the family of functions
    \begin{equation}\label{eq:ex:def:category_adjunction/dm_cat/varphi_family}
      \begin{aligned}
        &\varphi: \cat{Cat}(F(\anon*), \cat{\anon*}) \Rightarrow \cat{DMGph}(\anon*, U(\cat{\anon*})), \\
        &\varphi_{Q, \cat{C}}(S) \coloneqq \parens[\Big]{ v \mapsto S(v), a \mapsto S(\iota(a)) }.
      \end{aligned}
    \end{equation}

    For every functor \( S: F(Q) \to \cat{C} \), define \( \varphi_{Q, \cat{C}} \) so that it sends \( S \) to a \hyperref[def:graph_walk/directed]{walk} containing only one arc.

    Define the canonical embedding \( \iota \):
    \begin{equation*}
      \begin{aligned}
        &\iota: Q \to [U \bincirc F](Q) \\
        &\iota_V(v) \coloneqq v, \\
        &\iota_A(e) \coloneqq h(e) \overset e \to t(e).
      \end{aligned}
    \end{equation*}

    We will later see that \( \iota \) is the unit of a unit-counit adjunction.

    Now, from \eqref{eq:def:directed_multigraph_free_category/functor_from_homomorphism}, it is clear that the free category functor \( F \), when restricted to the set of directed multigraph homomorphisms \( \cat{DMGph}(Q, U(\cat{C})) \), is the two-sided inverse of \( \varphi_{Q, \cat{C}} \).

    We will show that \( \varphi \) is a natural transformation. Fix a functor \( G: \cat{C} \to \cat{D} \) and a homomorphism \( (g_V, g_A): Q \to R \). We must show that the following diagram commutes:
    \begin{equation}\label{eq:ex:def:category_adjunction/dm_cat/varphi_nat}
      \begin{aligned}
        \includegraphics[page=4]{output/ex__def__category_adjunction}
      \end{aligned}
    \end{equation}

    That is, for every functor \( S: F(A) \to \cat{C} \), we must show
    \begin{equation*}
      \varphi_{R, \cat{D}}(G \bincirc S \bincirc F(g_V, g_A))
      =
      U(G) \bincirc \varphi_{R, \cat{D}}(S) \bincirc (g_V, g_A).
    \end{equation*}

    This is also clear from \eqref{eq:def:directed_multigraph_free_category/functor_from_homomorphism}.

    Therefore, \( (F, U, \varphi) \) is a \hyperref[def:category_adjunction/hom]{hom-adjunction}.

    Furthermore, the canonical embedding \( \iota \) defined above, when parameterized by \( Q \), is a unit of adjunction.

    We can define a counit \( \varepsilon: F \bincirc U \Rightarrow \id_{\cat{Cat}} \) as follows. As discussed in \fullref{def:directed_multigraph_free_category}, for every path
    \begin{equation*}
      p = \anon \overset {e_1} \to \anon \overset \cdots \to \anon \overset {e_n} \to \anon
    \end{equation*}
    in the directed multigraph \( U(\cat{C}) \), the functor \( F \bincirc U \) \enquote{evaluates} \( p \) as
    \begin{equation*}
      e_n \bincirc e_{n-1} \bincirc \cdots \bincirc e_1.
    \end{equation*}

    Since the embedding only produces paths with a single arc, the adjunction triangles commute:
    \begin{equation}\label{eq:ex:def:category_adjunction/dm_cat/triangles}
      \begin{aligned}
        \includegraphics[page=5]{output/ex__def__category_adjunction}
        \quad\quad
        \includegraphics[page=6]{output/ex__def__category_adjunction}
      \end{aligned}
    \end{equation}

    \thmitem{ex:def:category_adjunction/us_ds}\mcite{MathOF:free_digraph} Denote the \hyperref[def:directed_graph/category]{category of simple directed graphs} by \( \cat{DGph} \) and the \hyperref[def:directed_graph/category]{category of simple undirected graphs} by \( \cat{UGph} \).

    The doubling functor \( D_S: \cat{UGph} \to \cat{DGph} \) from \fullref{def:graph_functors/simple_doubling} is a natural candidate for left adjoint to the \hyperref[def:concrete_category]{forgetful functor} \hyperref[def:graph_functors/simple_forgetful]{\( U_S: \cat{DGph} \to \cat{UGph} \)}. This is not the case, however, as there are \hyperref[def:hypergraph/homomorphism]{undirected graph homomorphisms} that fail to be \hyperref[def:directed_multigraph/homomorphism]{directed graph homomorphisms}. An example is shown in \cref{fig:ex:def:category_adjunction/us_ds}.

    \begin{figure}[!ht]
      \hfill
      \includegraphics[page=7]{output/ex__def__category_adjunction}
      \hfill
      \includegraphics[page=8]{output/ex__def__category_adjunction}
      \hfill\hfill
      \caption{Two undirected homomorphisms from \( G \) to \( U_S(Q) \), denoted using dashed lines, only one of which is a directed multigraph homomorphism from \( D_S(G) \) to \( Q \).}
      \label{fig:ex:def:category_adjunction/us_ds}
    \end{figure}

    If we consider instead a directed graph homomorphisms \( f: Q \to D_S(G) \), it is clear that it is also an undirected graph homomorphism from \( U_S(Q) \) to \( G \). Thus, \( D_S \) is actually \hi{right adjoint} to \( U_S \).
  \end{thmenum}
\end{example}

\begin{definition}\label{def:adjoint_equivalence}
  We call the quadruple \( (F, G, \eta, \varepsilon) \) with signature \eqref{eq:def:category_equivalence/signature} an \term{adjoint equivalence} if it is both an \hyperref[def:category_adjunction]{adjunction} and \hyperref[def:category_equivalence]{equivalence}.
\end{definition}

\begin{proposition}\label{thm:adjoint_equivalence}
  Let \( (F, G, \eta, \varepsilon) \) be a \hyperref[def:category_equivalence]{category equivalence} between \( \cat{C} \) and \( \cat{D} \).

  There exists a natural isomorphism \( \zeta: \id_{\cat{C}} \Rightarrow G \bincirc F \) such that \( (F, G, \zeta, \varepsilon) \) is an \hyperref[def:adjoint_equivalence]{adjoint equivalence}.
\end{proposition}
\begin{proof}
  From \fullref{thm:equivalence_induces_fully_faithful_and_essentially_surjective_functor} it follows that \( F \) is fully faithful and essentially surjective.

  We will now use the same trick as in the end of our proof of \fullref{thm:fully_faithful_and_essentially_surjective_functor_induces_equivalence} to define \( \zeta \).

  Since \( F \) is fully faithful, there is a bijective function
  \begin{equation*}
    \varphi: \cat{D}\parens[\Big]{ F(A), [F \bincirc G \bincirc F](A) } \to \cat{C}\parens[\Big]{ A, [F \bincirc G](A) }.
  \end{equation*}

  Hence, we can define
  \begin{equation*}
    \begin{aligned}
      &\zeta: \id_{\cat{C}} \to G \bincirc F, \\
      &\zeta_A \coloneqq \varphi(\varepsilon_{F(A)}^{-1})
    \end{aligned}
  \end{equation*}
  so that \( F(\zeta_A) = \varepsilon_{F(A)}^{-1} \). By \fullref{thm:def:functor_invertibility/fully_faithful_reflects_invertible}, \( \zeta_A \) is also an isomorphism.

  As in \fullref{thm:fully_faithful_and_essentially_surjective_functor_induces_equivalence}, we use \fullref{thm:commutative_diagrams_preserved_and_reflected} and the naturality of \( \varepsilon \) to prove that \eqref{eq:thm:fully_faithful_and_essentially_surjective_functor_induces_equivalence/varepsilon_source_nat} implies \eqref{eq:thm:fully_faithful_and_essentially_surjective_functor_induces_equivalence/varepsilon_image_nat} (with \( \eta \) replaced by \( \zeta \)).

  Therefore, \( \zeta \) is a natural isomorphism and the quadruple \( (F, G, \zeta, \varepsilon) \) is an equivalence of categories.
\end{proof}

\begin{proposition}\label{thm:functor_adjoint_uniqueness}
  If a functor has two \hyperref[def:category_adjunction]{left adjoints} (resp. right adjoints), then there exists a unique natural isomorphism between them.

  We say that left adjoints (resp. right adjoints) are unique up to a unique natural isomorphism.
\end{proposition}
\begin{proof}
  We will first prove the statement for left adjoints. Suppose that \( (F', G, \eta', \varepsilon') \) and \( (F', G, \eta^\dprime, \varepsilon^\dprime) \) are two unit-counit adjunctions.

  \SubProof{Proof of existence of isomorphism}\mcite{MathSE:left_adjoint_uniqueness} We can utilize the naturality of the units \( \eta' \) and \( \eta^\dprime \) and counits \( \varepsilon' \) and \( \varepsilon^\dprime \) to show that the following diagram commutes:
  \begin{equation}\label{eq:thm:functor_adjoint_uniqueness/existence}
    \begin{aligned}
      \includegraphics[page=1]{output/thm__functor_adjoint_uniqueness}
    \end{aligned}
  \end{equation}

  By the commuting triangle \eqref{eq:def:category_adjunction/d_triangle}, all paths from \( F'(A) \) to \( F'(A) \) above are identities.

  The bottom-most path in \eqref{eq:thm:functor_adjoint_uniqueness/existence} justifies defining the natural transformation
  \begin{equation*}
    \begin{aligned}
      &\alpha: F' \Rightarrow F^\dprime \\
      &\alpha_A \coloneqq \varepsilon'_{F^\dprime(A)} \bincirc F'(\eta_A^\dprime).
    \end{aligned}
  \end{equation*}

  Then \( \alpha_A \) is an isomorphism for every object \( A \) in \( \cat{C} \) with inverse \( \varepsilon_{F'(A)}^\dprime \bincirc F^\dprime(\eta_A') \). Therefore, it is a natural isomorphism from \( F' \) to \( F^\dprime \).

  \SubProof{Proof of uniqueness of isomorphism} Suppose that \( \beta: F' \Rightarrow F^\dprime \) is another natural isomorphism. Then, by the commuting triangle \eqref{eq:def:category_adjunction/d_triangle}, the following diagram also commutes:
  \begin{equation}\label{eq:thm:functor_adjoint_uniqueness/uniqueness}
    \begin{aligned}
      \includegraphics[page=2]{output/thm__functor_adjoint_uniqueness}
    \end{aligned}
  \end{equation}

  Therefore,
  \begin{equation*}
    \beta_A
    =
    \varepsilon^\dprime_{F^\dprime(A)} \bincirc F^\dprime(\eta^\dprime_A) \bincirc \alpha_A
    \reloset {\eqref{eq:def:category_adjunction/d_triangle}} =
    \alpha_A.
  \end{equation*}

  This finishes the proof for left adjoints. The other direction is \hyperref[thm:categorical_principle_of_duality]{dual}. If \( G' \) and \( G^\dprime \) are two right adjoints to \( F \), then by \fullref{thm:category_adjunction_duality}, \( G'^\oppos \) and \( {G^\dprime}^\oppos \) are left adjoints and are thus isomorphic. Then by \fullref{thm:morphism_invertibility_duality}, \( G' \) and \( G^\dprime \) are also isomorphic.
\end{proof}

\begin{proposition}\label{thm:universal_objects_as_adjunctions}
  Fix a category \( \cat{C} \). We can characterize the universal objects in \( \cat{C} \) from \fullref{def:universal_objects} via adjunctions with the \hyperref[def:universal_categories]{terminal category} \( \cat{1} \).

  Let \( \Delta^{\cat{1}}: \cat{C} \to \cat{1} \) be the \hyperref[def:diagonal_functor]{constant functor} into \( \cat{1} \).

  \begin{thmenum}
    \thmitem{thm:universal_objects_as_adjunctions/initial} The object \( I \) of \( \cat{C} \) is \hyperref[def:universal_objects]{initial} if and only if it is (the unique value of) a left adjoint to \( \Delta^{\cat{1}} \) functor.

    In particular, the uniqueness proved in \fullref{thm:def:universal_objects/initial} follows from \fullref{thm:functor_adjoint_uniqueness}.

    \thmitem{thm:universal_objects_as_adjunctions/terminal} \hyperref[thm:categorical_principle_of_duality]{Dually}, the \hyperref[def:universal_objects]{terminal objects} are exactly the right adjoint to \( \Delta_I^{\cat{1}} \) functors.
  \end{thmenum}
\end{proposition}
\begin{proof}
  We will only prove \fullref{thm:universal_objects_as_adjunctions/initial} since the other direction is \hyperref[thm:categorical_principle_of_duality]{dual}.

  \SufficiencySubProof Let \( I \) be an initial object in \( \cat{C} \). We can then regard it as a functor \( F: \cat{1} \to \cat{C} \).

  We define the natural transformations
  \begin{equation*}
    \begin{aligned}
      &\eta: \id_{\cat{1}} \Rightarrow \Delta_I^{\cat{1}} \bincirc F \\
      &\eta_{\cat{0}} \coloneqq \id_{\cat{0}}
    \end{aligned}
  \end{equation*}
  and
  \begin{equation*}
    \begin{aligned}
      &\varepsilon: F \bincirc \Delta_I^{\cat{1}} \Rightarrow \id_{\cat{C}} \\
      &\varepsilon_A \T{is the unique morphism} I \to A
    \end{aligned}
  \end{equation*}

  Since \( I \) has a unique morphism into any other object of \( \cat{C} \), for every morphism \( f: A \to B \), the following diagram commutes:
  \begin{equation}\label{eq:thm:universal_objects_as_adjunctions/sufficiency_nat}
    \begin{aligned}
      \includegraphics[page=1]{output/thm__universal_objects_as_adjunctions}
    \end{aligned}
  \end{equation}

  It follows that both \( \eta \) and \( \varepsilon \) are natural transformations. Furthermore, they trivially satisfy the triangle diagrams triangle \eqref{eq:def:category_adjunction/d_triangle} and \eqref{eq:def:category_adjunction/c_triangle}.

  Hence, \( (F, \Delta_I^{\cat{1}}, \eta, \varepsilon) \) is a \hyperref[def:category_adjunction/unit_counit]{unit-counit adjunction}.

  \NecessitySubProof Conversely, suppose that \( (F, \Delta_I^{\cat{1}}, \eta, \varepsilon) \) is a unit-counit adjunction.

  Let \( I \coloneqq F(\cat{0}) \). Then \( \varepsilon_A \) is a morphism from \( I \) to \( A \). By \eqref{eq:def:category_adjunction/d_triangle}, \( \varepsilon_I = \id_I \) since the following diagram commutes:
  \begin{equation}\label{eq:thm:universal_objects_as_adjunctions/d_triangle}
    \begin{aligned}
      \includegraphics[page=2]{output/thm__universal_objects_as_adjunctions}
    \end{aligned}
  \end{equation}

  Suppose that \( \zeta \) is another morphism from \( I \) to \( A \). The naturality of \( \varepsilon \) implies that, for the morphism \( \id_A: A \to A \), the following diagram commutes:
  \begin{equation}\label{eq:thm:universal_objects_as_adjunctions/necessity_nat}
    \begin{aligned}
      \includegraphics[page=3]{output/thm__universal_objects_as_adjunctions}
    \end{aligned}
  \end{equation}

  The upper left triangle in \eqref{eq:thm:universal_objects_as_adjunctions/necessity_nat} is \eqref{eq:thm:universal_objects_as_adjunctions/d_triangle}.

  We conclude that \( \zeta = \varepsilon_A \) and, generalizing on \( A \), that every morphism from \( I \) is unique.
\end{proof}

\begin{remark}\label{rem:left_and_right_adjoint_not_equivalence}
  We discussed in \fullref{ex:def:universal_objects/grp} that \enquote{the} trivial group \( \set{ e } \) is a \hyperref[def:universal_objects/zero]{zero object} of \( \cat{Grp} \). By \fullref{thm:universal_objects_as_adjunctions}, this object induces a functor that is both left adjoint and right adjoint of \( \Delta_I^{\cat{1}} \). Nevertheless, the categories \( \cat{Grp} \) and \( \cat{1} \) are not \hyperref[def:category_equivalence]{equivalent}.
\end{remark}

\begin{remark}\label{rem:universal_mapping_property}
  We will now regard adjoint functors as a way to \enquote{construct} new objects.

  Let \( (F, G, \iota, \pi) \) be a \hyperref[def:category_adjunction/unit_counit]{unit-counit adjunction} between the categories \( \cat{C} \) and \( \cat{D} \). In the current context, especially in connection with \hyperref[def:category_of_cones/limit]{limits} and \hyperref[def:category_of_cones/colimit]{colimits}, we will call the components of the counit \( \pi: F \bincirc G \Rightarrow \id_{\cat{D}} \) --- \term{projections}, and the components of the unit \( \iota: \id_{\cat{C}} \Rightarrow G \bincirc F \) --- \term{coprojections}.

  Take objects \( A \) in \( \cat{C} \) and \( X \) in \( \cat{D} \) and a morphism \( f: A \to G(X) \). We want to obtain a morphism \( \widetilde{f}: F(A) \to X \), for which the following diagram commutes:
  \begin{equation}\label{eq:rem:universal_mapping_property/c_triangle}
    \begin{aligned}
      \includegraphics[page=1]{output/rem__universal_mapping_property}
    \end{aligned}
  \end{equation}

  From the naturality of \( \iota \) and from the triangle diagram \eqref{eq:def:category_adjunction/c_triangle} it follows that the following diagram commutes:
  \begin{equation}\label{eq:rem:universal_mapping_property/f_tilde_existence}
    \begin{aligned}
      \includegraphics[page=2]{output/rem__universal_mapping_property}
    \end{aligned}
  \end{equation}

  It is clear from \eqref{eq:rem:universal_mapping_property/f_tilde_existence} that
  \begin{equation*}
    G(\widetilde{f}) = G(\pi_X) \bincirc [G \bincirc F](f) = G(\pi_X \bincirc F(f)).
  \end{equation*}

  Furthermore, this value is unique. From the naturality of \( \pi \) and the triangle diagram \eqref{eq:def:category_adjunction/d_triangle} it follows that the following diagram commutes:
  \begin{equation}\label{eq:rem:universal_mapping_property/f_tilde_uniquness}
    \begin{aligned}
      \includegraphics[page=3]{output/rem__universal_mapping_property}
    \end{aligned}
  \end{equation}

  Therefore,
  \begin{equation*}
    \widetilde{f} = \pi_X \bincirc F(f)
  \end{equation*}

  Taking into account that the functor \( F \) itself is unique up to a unique isomorphism, as per \fullref{thm:functor_adjoint_uniqueness}, we have proved the following statement:
  \begin{displayquote}
    For every object \( A \) in \( \cat{C} \), there exist unique up to a unique isomorphism object \( F(A) \) in \( \cat{D} \) and canonical coprojection map \( \iota_A: A \to [G \bincirc F](A) \) satisfying the following property, called a \term{universal mapping property}:
    \begin{displayquote}
      For every object \( X \) in \( \cat{D} \) and every map \( f: A \to G(X) \) in \( \cat{C} \), there exists a unique map \( \widetilde{f}: F(A) \to X \) in \( \cat{D} \) such that the diagram \eqref{eq:rem:universal_mapping_property/c_triangle} commutes.
    \end{displayquote}
  \end{displayquote}

  Intuitively, this universal mapping property states that any map (morphism) with domain \( A \) in \( \cat{C} \) can be transformed into a map with domain \( F(A) \) in \( \cat{D} \).

  The statement becomes more meaningful when we regard \( G: \cat{D} \to \cat{C} \) as a \hyperref[def:concrete_category]{forgetful functor}. In this case, every object of \( \cat{D} \) is regarded as an object of \( \cat{C} \), and we write \( X \) rather than \( G(X) \). The universal mapping property then becomes:
  \begin{displayquote}
    For every object \( A \) in \( \cat{C} \), there exist unique up to a unique isomorphism object \( F(A) \) in \( \cat{D} \) and canonical coprojection map \( \iota_A: A \to F(A) \) satisfying the following universal mapping property:
    \begin{displayquote}
      For every object \( X \) in \( \cat{D} \) and every map \( f: A \to X \) in \( \cat{C} \), there exists a unique map \( \widetilde{f}: F(A) \to X \) in \( \cat{D} \) such that the following diagram commutes:
      \begin{equation}\label{eq:rem:universal_mapping_property/c_triangle_forgetful}
        \begin{aligned}
          \includegraphics[page=4]{output/rem__universal_mapping_property}
        \end{aligned}
      \end{equation}
    \end{displayquote}
  \end{displayquote}

  In \fullref{def:concrete_category} we mentioned that we will call the left adjoint of a forgetful functor a free functor. Universal mapping properties allow characterizing certain \enquote{free constructions}, such as the free groups defined in \fullref{def:free_group}, without explicitly building a free functor and proving that it is left adjoint. Indeed, for every suitable object and map, we explicitly build the natural isomorphism \( \varphi \) of a hom-adjunction, and the commutative triangle \eqref{eq:rem:universal_mapping_property/c_triangle} ensures that this \( \varphi \) is a natural transformation.

  Universal mapping properties of this form are used for \hyperref[def:category_of_cones/colimit]{colimits} --- see \fullref{rem:limit_universal_mapping_property}.

  Of course, there is a \hyperref[thm:categorical_principle_of_duality]{dual} universal mapping property:
  \begin{displayquote}
    For every object \( X \) in \( \cat{D} \) there exist unique up to a unique isomorphism object \( G(X) \) in \( \cat{C} \) and canonical projection map \( \pi_X: [F \bincirc G](X) \to X \) satisfying the following property, called a \term{universal mapping property}:
    \begin{displayquote}
      For every object \( A \) in \( \cat{C} \) and every map \( g: F(A) \to X \) in \( \cat{D} \), there exists a unique map \( \widetilde{g}: A \to G(X) \) in \( \cat{C} \) such that the following diagram commutes:
      \begin{equation}\label{eq:rem:universal_mapping_property/d_triangle}
        \begin{aligned}
          \includegraphics[page=5]{output/rem__universal_mapping_property}
        \end{aligned}
      \end{equation}
    \end{displayquote}
  \end{displayquote}

  In this case, we can regard \( F \) as a forgetful functor and \( G \) as a free functor to obtain the following:
  \begin{displayquote}
    For every object \( X \) in \( \cat{D} \) there exist unique up to a unique isomorphism object \( G(X) \) in \( \cat{C} \) and canonical projection map \( \pi_X: G(X) \to X \) satisfying the following property, called a \term{universal mapping property}:
    \begin{displayquote}
      For every object \( A \) in \( \cat{C} \) and every map \( g: A \to X \) in \( \cat{D} \), there exists a unique map \( \widetilde{g}: A \to G(X) \) in \( \cat{C} \) such that the following diagram commutes:
      \begin{equation}\label{eq:rem:universal_mapping_property/d_triangle_forgetful}
        \begin{aligned}
          \includegraphics[page=6]{output/rem__universal_mapping_property}
        \end{aligned}
      \end{equation}
    \end{displayquote}
  \end{displayquote}

  Universal mapping properties of this form are used for \hyperref[def:category_of_cones/limit]{limits} --- see \fullref{rem:limit_universal_mapping_property}.
\end{remark}

\begin{remark}\label{rem:definition_via_characterization}
  It is often the case that we do not care about how an object is constructed, but only about certain properties that it satisfies. It is thus tempting to define objects via \hyperref[rem:universal_mapping_property]{universal mapping properties} and merely prove existence via some particular construction.

  This is applicable, for example, to \hyperref[def:disjoint_union]{disjoint unions}, as we explicitly avoid relying on the precise construction \( A \amalg B \) and only work the inclusions \( \iota_A \) and \( \iota_B \) --- we might just as well use any other construction satisfying \fullref{thm:disjoint_union_universal_property}.

  In other cases, however, this restricts us, since we can no longer rely on desirable intrinsic properties. For example, if we only use the \hyperref[def:free_monoid]{free monoid} \( F(A) \) via \fullref{thm:free_monoid_universal_property}, we will no longer be able to regard \( F(A) \) as the \hyperref[def:formal_language/kleene_star]{Kleene star} \( A^* \) --- a set of strings. Instead, we will only be able to use tools from \fullref{ch:category_theory} to work with free monoids.

  Our proof of \fullref{thm:group_coproduct} illustrates the latter point. It is based on several applications of universal properties. The proof is almost reduced to formal verification, with little justification beyond that.

  As a compromise, we give explicit constructions, however we also avoid relying on them unless that benefits us.
\end{remark}

\begin{proposition}\label{thm:concrete_category_function_invertibility}
  If \( \cat{C} \) is a \hyperref[def:concrete_category]{concrete category} with forgetful functor \( U: \cat{C} \to \cat{Set} \).

  \begin{thmenum}
    \thmitem{thm:concrete_category_function_invertibility/injection_is_mono} Every \hyperref[def:set_valued_map/empty]{\hi{nonempty}} \hyperref[def:function_invertibility/injective]{injective} morphism in \( \cat{C} \) is a \hyperref[def:morphism_invertibility/left_cancellative]{monomorphism}.

    That is, for every morphism \( f \) in \( \cat{C} \), if \( U(f) \) is a \hyperref[def:set_valued_map/empty]{\hi{nonempty}} \hyperref[def:function_invertibility/injective]{injective function}, \( f \) is a \hyperref[def:morphism_invertibility/left_cancellative]{monomorphism}.

    \thmitem{thm:concrete_category_function_invertibility/surjection_is_epi} \hyperref[thm:categorical_principle_of_duality]{Dually}, every \hyperref[def:function_invertibility/surjective]{surjective} morphism in \( \cat{C} \) is an \hyperref[def:morphism_invertibility/right_cancellative]{epimorphism}.

    \thmitem{thm:concrete_category_function_invertibility/invertible} Every \hyperref[def:morphism_invertibility/left_invertible]{split monomorphism} is injective, every \hyperref[def:morphism_invertibility/right_cancellative]{split epimorphism} is surjective and every categorical isomorphism is a bijective function.

    \thmitem{thm:concrete_category_function_invertibility/mono_is_injection} If \( U \) has a \hyperref[def:category_adjunction]{left adjoint}, every monomorphism is injective.

    \thmitem{thm:concrete_category_function_invertibility/epi_is_surjection} If \( U \) has a \hyperref[def:category_adjunction]{right adjoint}, every epimorphism is surjective.
  \end{thmenum}
\end{proposition}
\begin{proof}
  \SubProofOf{thm:concrete_category_function_invertibility/injection_is_mono} By \fullref{thm:function_invertibility_categorical}, a nonempty injective function is a monomorphism in \( \cat{Set} \). This translates to morphisms in \( \cat{C} \) since they are special cases of functions. Formally, this follows from \fullref{thm:def:functor_invertibility/faithful_reflects_cancellative}.

  \SubProofOf{thm:concrete_category_function_invertibility/surjection_is_epi} Again by \fullref{thm:function_invertibility_categorical}, a surjective function is an epimorphism in \( \cat{Set} \). Again from \fullref{thm:def:functor_invertibility/faithful_reflects_cancellative}, this implies that every surjective morphism in \( \cat{C} \) is an epimorphism.

  \SubProofOf{thm:concrete_category_function_invertibility/invertible} Follows from \fullref{thm:def:functor_invertibility/preserves_inverses}.

  \SubProofOf{thm:concrete_category_function_invertibility/mono_is_injection} Let \( F: \cat{Set} \to \cat{C} \) be a left adjoint to \( U \) and fix a set \( A \). As discussed in \fullref{rem:universal_mapping_property}, this leads to the following \hyperref[rem:universal_mapping_property]{universal mapping property}:
  \begin{displayquote}
    For every object \( X \) in \( \cat{D} \) and every morphism \( g: A \to U(X) \) in \( \cat{C} \), there exists a unique morphism \( \widetilde{g}: F(A) \to X \) in \( \cat{D} \) such that the following diagram commutes:
    \begin{equation}\label{eq:thm:concrete_category_function_invertibility/left_universal_diagram}
      \begin{aligned}
        \includegraphics[page=1]{output/thm__concrete_category_function_invertibility}
      \end{aligned}
    \end{equation}
  \end{displayquote}

  First, suppose that \( f: X \to Y \) is a monomorphism in \( \cat{C} \) and let \( [U(f)](x_1) = [U(f)](x_2) \) (by \fullref{thm:function_invertibility_categorical/empty}, \( U(f) \) is nonempty). Aiming at a contradiction, suppose that \( x_1 \neq x_2 \). Define the function \( g: U(X) \to U(X) \) as the identity \( U(\id_X) \) for which \( x_1 \) and \( x_2 \) are swapped as in our proof of \fullref{thm:function_invertibility_categorical/left_cancellative}. By the property \eqref{eq:thm:concrete_category_function_invertibility/left_universal_diagram}, the following diagram commutes:
  \begin{equation}\label{eq:thm:concrete_category_function_invertibility/left_diagram}
    \begin{aligned}
      \includegraphics[page=2]{output/thm__concrete_category_function_invertibility}
    \end{aligned}
  \end{equation}

  In particular,
  \begin{equation*}
    U(f) \bincirc U(\widetilde{\id_X}) = U(f) \bincirc U(\widetilde{g}),
  \end{equation*}
  and thus \( U(\widetilde{\id_X}) = U(\widetilde{g}) \) and \( U(\id_X) = g \). But this contradicts our assumption that \( g \) differs from the identity by construction. Therefore, \( U(f) \) must be an injective function.

  \SubProofOf{thm:concrete_category_function_invertibility/epi_is_surjection} Dually, let \( F \) be a right adjoint to \( U \). It follows from \fullref{thm:category_adjunction_duality}, \fullref{thm:categorical_principle_of_duality/morphism_invertibility} and \fullref{thm:concrete_category_function_invertibility/mono_is_injection} that every epimorphism is a surjection.

  We will also give a direct proof. Since \( F \) is right adjoint, we have the following \hyperref[rem:universal_mapping_property]{universal mapping property}:
  \begin{displayquote}
    For every object \( X \) in \( \cat{C} \) and every function \( h: A \to U(X) \) in \( \cat{Set} \), there exists a unique morphism \( \widetilde{h}: F(A) \to X \) in \( \cat{C} \) such that the following diagram commutes:
    \begin{equation}\label{eq:thm:concrete_category_function_invertibility/right_universal_diagram}
      \begin{aligned}
        \includegraphics[page=3]{output/thm__concrete_category_function_invertibility}
      \end{aligned}
    \end{equation}
  \end{displayquote}

  Suppose that \( f \) is an epimorphism and that there exists some value \( y_0 \in U(Y) \) not in the image of \( f \). As in our proof of \fullref{thm:function_invertibility_categorical/right_cancellative}, fix some set \( z \) not in \( U(Y) \), define \( S \coloneqq U(Y) \cup \set{ z } \) and define \( g: U(Y) \to S \) by swapping \( y_0 \) and \( z \) in the identity \( \id_Z \). Let \( s: U(Y) \to S \) be the restriction of \( \id_Z \) to \( U(Y) \).

  By the property \eqref{eq:thm:concrete_category_function_invertibility/right_universal_diagram}, the following diagram commutes:
  \begin{equation}\label{eq:thm:concrete_category_function_invertibility/right_diagram}
    \begin{aligned}
      \includegraphics[page=4]{output/thm__concrete_category_function_invertibility}
    \end{aligned}
  \end{equation}

  In particular,
  \begin{equation*}
  U(\widetilde{\id_Z}) \bincirc U(f) = U(\widetilde{h}) \bincirc U(f),
  \end{equation*}
  and thus \( U(\widetilde{\id_Z}) = U(\widetilde{h}) \) and \( \id_Z = h \). But this contradicts our construction of \( h \). Therefore, \( U(f) \) must be a surjective function.
\end{proof}

  \subsection{Categorical limits}\label{subsec:categorical_limits}

\begin{definition}\label{def:comma_category}\mcite[def. 2.3.1]{Leinster2016Basic}
  Comma categories allow us to define morphisms between morphisms, which becomes useful in, for example, the concise definition of a limit in \fullref{def:category_of_cones/limit}.

  \begin{thmenum}
    \thmitem{def:comma_category/variable} We first prove the most general construction. Let \( F: \cat{A} \to \cat{C} \) and \( G: \cat{B} \to \cat{C} \) be any two functors with a common codomain. We define their \term{comma category} \( (F \downarrow G) \) as follows:

    \begin{itemize}
      \item The \hyperref[def:category/objects]{objects} are the triples \( (A, s, B) \), where \( A \in \cat{A} \), \( B \in \cat{B} \) and \( s: F(A) \to G(B) \).

      \item The \hyperref[def:category/morphisms]{morphisms} from \( (A, s, B) \) to \( (A', s', B')) \) are the pairs
      \begin{equation*}
        (f: A \to A', g: B \to B'),
      \end{equation*}
      such that the following diagram commutes:
      \begin{equation}\label{eq:def:comma_category/variable}
        \begin{aligned}
          \includegraphics[page=1]{output/def__comma_category}
        \end{aligned}
      \end{equation}

      \item The \hyperref[def:category/composition]{composition of morphisms} is their pairwise composition, analogically to \hyperref[def:product_category]{product categories}.

      \item The \hyperref[def:category/identity]{identity morphism} on the object \( (A, s, B) \) is the identity pair \( (\id_A, \id_B) \).
    \end{itemize}

    \thmitem{def:comma_category/fixed} It is often the case where either \( F \) or \( G \) are constant. If, instead of \( G \), we are given an object \( X \) in \( \cat{C} \), we use the constant functor \( \Delta_X^{\cat{1}}: \cat{1} \to \cat{C} \) with domain the \hyperref[def:universal_categories]{terminal category} \( \cat{1} \), in order to utilize the comma category \( (F \downarrow \Delta_X^{\cat{1}}) \).

    We can thus simplify \fullref{def:comma_category/variable} as follows:

    \begin{minipage}[t]{0.43\textwidth}
      \begin{equation*}
        (F \downarrow X) \coloneqq (F \downarrow \Delta_X^{\cat{1}}).
      \end{equation*}

      \begin{itemize}
        \item The \hyperref[def:category/objects]{set of objects} \( \obj(F \downarrow X) \) is
        \begin{equation*}
          \set{ (A, s) \given s: F(A) \to X }
        \end{equation*}

        \item The \hyperref[def:category/morphisms]{hom-set}
        \begin{equation*}
          [F \downarrow X]\parens[\Big]{ (A, s), (A', s') }
        \end{equation*}
        is the set of \( f: A \to A' \), such that
        \begin{equation}\label{eq:def:comma_category/fixed/right}
          \begin{aligned}
            \includegraphics[page=2]{output/def__comma_category}
          \end{aligned}
        \end{equation}
      \end{itemize}
    \end{minipage}
    \begin{minipage}[t]{0.43\textwidth}
      \begin{equation*}
        (X \downarrow G) \coloneqq (\Delta_X^{\cat{1}} \downarrow G)
      \end{equation*}

      \begin{itemize}
        \item The \hyperref[def:category/objects]{set of objects} \( \obj(X \downarrow G) \) is
        \begin{equation*}
          \set{ (s, B) \given s: X \to F(B) }
        \end{equation*}

        \item The \hyperref[def:category/morphisms]{hom-set}
        \begin{equation*}
          [X \downarrow G]\parens[\Big]{ (s, B), (s', B') }
        \end{equation*}
        is the set of \( g: B \to B' \), such that
        \begin{equation}\label{eq:def:comma_category/fixed/left}
          \begin{aligned}
            \includegraphics[page=3]{output/def__comma_category}
          \end{aligned}
        \end{equation}
      \end{itemize}
    \end{minipage}
  \end{thmenum}
\end{definition}

\begin{definition}\label{def:factors_through}\mcite[133]{Knapp2016BasicAlgebra}
  Consider a morphism \( f: A \to B \). We say that \( f \) \term{factors through} the object \( X \) if there exist morphisms \( g: A \to X \) and \( h: X \to B \) such that the following diagram commutes:
  \begin{equation}\label{eq:def:factors_through}
    \begin{aligned}
      \includegraphics[page=1]{output/def__factors_through}
    \end{aligned}
  \end{equation}

  If, given \( g \), \( h \) is unique or vice versa, we say that \( f \) \term{uniquely factors through} \( X \). In some cases, for example in \fullref{def:zero_morphisms/morphism}, \( f \) there exists unique \( g \) and \( h \) and neither of them must be given beforehand.
\end{definition}

\begin{definition}\label{def:category_of_cones}
  Let \( D: \cat{I} \to \cat{C} \) be a \hyperref[def:categorical_diagram]{diagram}.

  \begin{thmenum}
    \thmitem{def:category_of_cones/category} We define \term{category of cones} to \( D \) as the \hyperref[def:comma_category/fixed]{constant-functor comma category}
    \begin{equation*}
      \cat{Cone}(D) \coloneqq \underbrace{ (\Delta^{\cat{I}} \downarrow D) }_{(\Delta^{\cat{I}} \downarrow \Delta^{\cat{1}}_{D})},
    \end{equation*}
    where \( \Delta^{\cat{I}}: \cat{C} \to [\cat{I}, \cat{C}] \) is the \( \cat{I} \)-shaped \hyperref[def:diagonal_functor]{diagonal functor} on \( \cat{C} \).

    \thmitem{def:category_of_cones/cone} A \( D \)-\term{cone} is simply a member of \( \cat{Cone}(D) \).

    Explicitly, a cone with \term{vertex} \( A \in \cat{C} \) is a pair \( (\Delta^{\cat{I}}_A, \alpha) \), where \( \Delta^{\cat{I}}_A \) is the constant functor at \( A \) and \( \alpha \) is a natural transformation from \( \Delta^{\cat{I}}_A \) to \( D \).

    Even more explicitly, a cone is a family of morphisms
    \begin{equation}\label{eq:def:category_of_cones/cone}
      \seq{ \alpha_k: A \to D(k) }_{k \in \cat{I}}.
    \end{equation}
    satisfying a simplified naturality condition (compared to \eqref{eq:def:natural_transformation/diagram}). For every morphism \( u: k \to m \) in \( \cat{I} \), the following diagram must commute:
    \begin{equation}\label{eq:def:category_of_cones/cone_nat}
      \begin{aligned}
        \includegraphics[page=1]{output/def__category_of_cones}
      \end{aligned}
    \end{equation}

    Note that this diagram is very different from \eqref{eq:def:comma_category/fixed/right}, they merely look similar.

    \thmitem{def:category_of_cones/limit} A \term{limit cone} of the diagram \( D \) is a \hyperref[def:universal_objects/terminal]{terminal object} of the cone category \( \cat{Cone}(D) \).

    Explicitly, \( (L, \lambda) \) is limit cone if, for every cone \( (A, \alpha) \), there exists a unique \hyperref[eq:def:comma_category/fixed/right]{cone morphism} \( l_A: (A, \alpha) \to (L, \lambda) \).

    Even more explicitly, \( (L, \lambda) \) is a limit cone if it satisfies the following \hyperref[rem:limit_universal_mapping_property]{universal mapping property}:
    \begin{displayquote}
      For every cone \( (A, \alpha) \), there exists a unique morphism \( l_A: A \to L \) such that following diagram commutes for every index morphism \( u: k \to m \):
      \begin{equation}\label{eq:def:category_of_cones/limit}
        \begin{aligned}
          \includegraphics[page=2]{output/def__category_of_cones}
        \end{aligned}
      \end{equation}
    \end{displayquote}

    Thus, \( \alpha_k \) \hyperref[def:factors_through]{uniquely factors through} \( L \) for every \( k \in \cat{I} \). Furthermore, \( l_A \) is compatible with morphisms in \( \cat{I} \) and does not depend on \( k \).

    From \fullref{thm:def:universal_objects/terminal} and \fullref{thm:universal_objects_as_adjunctions/terminal} it follows that a limit, if it exists, is unique up to a unique isomorphism.

    Without further context, we usually refer to \( L \) as the limit vertex and \( (L, \lambda) \) as the limit cone. By \enquote{the limit}, we usually mean the vertex \( L \).

    \thmitem{def:category_of_cones/cocone} \hyperref[thm:categorical_principle_of_duality]{Dually}, a \( D \)-\term{cocone} with vertex \( A \) is a family of morphisms
    \begin{equation}\label{eq:def:category_of_cones/cocone}
      \seq{ \alpha_k: D(k) \to A }_{k \in \cat{I}}.
    \end{equation}
    satisfying the naturality condition that for every morphism \( u: k \to m \) in \( \cat{I} \), the following diagram must commute:
    \begin{equation}\label{eq:def:category_of_cones/cocone_nat}
      \begin{aligned}
        \includegraphics[page=3]{output/def__category_of_cones}
      \end{aligned}
    \end{equation}

    \thmitem{def:category_of_cones/colimit} A \term{colimit cocone} of the diagram \( D \) is an \hyperref[def:universal_objects/initial]{initial object} of the cocone category \( (D \downarrow \Delta) \).

    There are two major differences compared to limits: a colimit cocone is an initial object, not a terminal object, and its underlying comma category is \( (D \downarrow \Delta) \), not \( (\Delta^{\cat{I}} \downarrow D) \).

    The analogous diagram to \eqref{eq:def:category_of_cones/limit} is exactly its \hyperref[thm:categorical_principle_of_duality]{opposite}:
    \begin{equation}\label{eq:def:category_of_cones/colimit}
      \begin{aligned}
        \includegraphics[page=4]{output/def__category_of_cones}
      \end{aligned}
    \end{equation}

    In particular, \( \alpha_k \) \hyperref[def:factors_through]{uniquely factors through} \( L \) for every \( k \in \cat{I} \).

    Without further context, we usually refer to \( L \) as the colimit vertex and \( (L, \lambda) \) as the colimit cocone. By \enquote{the colimit}, we usually mean the vertex \( L \).
  \end{thmenum}
\end{definition}

\begin{proposition}\label{thm:categorical_limit_duality}
  For every \hyperref[def:category_of_cones/cone]{cone} \( (A, \alpha) \) of the \hyperref[def:categorical_diagram]{diagram} \( D \) in \( \cat{C} \), \( (A, \alpha^{\opcat}) \) is a \hyperref[def:category_of_cones/cone]{cocone} of \( D^{\opcat} \) in the \hyperref[def:opposite_category]{opposite category} \( \cat{C}^{\opcat} \).

  Even more, for every \hyperref[def:category_of_cones/limit]{limit} \( (L, \lambda) \) of \( D \) in \( \cat{C} \), \( (L, \lambda^{\opcat}) \) is a \hyperref[def:category_of_cones/colimit]{colimit} of \( D^{\opcat} \) in \( \cat{C}^{\opcat} \).

  This is part of the duality principles listed in \fullref{thm:categorical_principle_of_duality}.
\end{proposition}
\begin{proof}
  Note that the defining diagrams \eqref{eq:def:category_of_cones/cone}, \eqref{eq:def:category_of_cones/cone_nat} and \eqref{eq:def:category_of_cones/limit} are dual to \eqref{eq:def:category_of_cones/cocone}, \eqref{eq:def:category_of_cones/cocone_nat} and \eqref{eq:def:category_of_cones/colimit}.
\end{proof}

\begin{lemma}\label{thm:categorical_limit_uniqueness_lemma}
  Any two limits (resp. colimits) of a diagram are isomorphic.

  We prove a stronger result in \fullref{thm:categorical_limit_uniqueness}.
\end{lemma}
\begin{proof}
  Let \( (L', \lambda') \) and \( (L^\dprime, \lambda^\dprime) \) be two limit cones over the diagram \( D: \cat{I} \to \cat{C} \). The definition \eqref{eq:def:category_of_cones/limit} of a limit implies that
  \begin{equation}\label{eq:thm:categorical_limit_uniqueness}
    \begin{aligned}
      \includegraphics[page=1]{output/thm__categorical_limit_uniqueness}
    \end{aligned}
  \end{equation}
  commutes.

  Therefore, the limits \( L' \) and \( L^\dprime \) are isomorphic.

  Now let \( (L', \lambda') \) and \( (L^\dprime, \lambda^\dprime) \) be colimit cocones. By \fullref{thm:categorical_limit_duality}, \( (L', \lambda'^{\opcat}) \) and \( (L^\dprime, {\lambda^\dprime}^{\opcat}) \) are limits in the opposite category and are thus isomorphic. By \fullref{thm:morphism_invertibility_duality}, the colimits are isomorphic.
\end{proof}

\begin{proposition}\label{thm:categorical_limit_is_adjoint}
  Suppose that, for a given category \( \cat{C} \), the limits over all \( \cat{I} \)-shaped diagrams exist. Denote by \( \lim(D) \) the vertex of the limiting cone of the diagram \( D \).

  Given a natural transformation \( \alpha: D \to E \) between diagrams, the diagram \eqref{eq:thm:categorical_limit_is_adjoint/c_triangle} defining a limit uniquely determines a morphism \( \lim(D) \) to \( \lim(E) \). Denote this morphism by \( \lim(\alpha) \).

  We have defined a functor
  \begin{equation*}
    \lim: [\cat{I}, \cat{C}] \to \cat{C}.
  \end{equation*}

  This functor is \hyperref[def:category_adjunction]{right adjoint} to the diagonal functor
  \begin{equation*}
    \Delta: \cat{C} \to [\cat{I}, \cat{C}]
  \end{equation*}

  \hyperref[thm:categorical_principle_of_duality]{Dually}, the colimit functor
  \begin{equation*}
    \co\lim: [\cat{I}, \cat{C}] \to \cat{C}
  \end{equation*}
  is left adjoint to \( \Delta \).
\end{proposition}
\begin{proof}
  It is sufficient to prove this for limits since the statement for colimits follows from the duality principles \fullref{thm:category_adjunction_duality} and \fullref{thm:categorical_limit_duality}. The unit \( \eta: \id_{\cat{C}} \to [{\lim} \bincirc \Delta] \) of the adjunction
  \begin{equation*}
    \Delta \dashv \lim
  \end{equation*}
  is the unique morphism from an object \( A \) of \( \cat{C} \) to the limit of its constant diagram \( \Delta_A: \cat{I} \to \cat{C} \) such that \eqref{eq:def:category_of_cones/limit} commutes. The counit is more complicated because its components are themselves natural transformations:
  \begin{equation*}
    \begin{aligned}
      \varepsilon:       &[\Delta \bincirc \lim] \Rightarrow \id_{[\cat{I}, \cat{C}]} \\
      \varepsilon_D:     &\Delta(\lim D) \Rightarrow D \\
      \varepsilon_{D,k}: &\lim(D) \to D(k),
    \end{aligned}
  \end{equation*}
  where \( \varepsilon_{D,k} \) are the projections of the limit.

  For any diagram \( D: \cat{I} \to \cat{C} \), the following triangle commutes:
  \begin{equation}\label{eq:thm:categorical_limit_is_adjoint/ic_triangle}
    \begin{aligned}
      \includegraphics[page=1]{output/thm__categorical_limit_is_adjoint}
    \end{aligned}
  \end{equation}

  Indeed,
  \begin{equation*}
    \eta_{\lim(D)}: \lim(D) \to \smash{ \overbrace{\lim(\Delta_{\lim(D)})}^{[{\lim} \bincirc \Delta \bincirc {\lim}](D)} }
  \end{equation*}
  is the unique morphism such that \eqref{eq:def:category_of_cones/limit} commutes, and (somewhat) similarly for
  \begin{equation*}
    \lim(\varepsilon_D): [{\lim} \bincirc \Delta \bincirc {\lim}](D) \to \lim(D).
  \end{equation*}

  It follows that both \( \lim(D) \) and \( [{\lim} \bincirc \Delta \bincirc {\lim}](D) \) are limits over the same diagram. By \fullref{thm:categorical_limit_is_adjoint}, they are isomorphic, and hence \eqref{eq:thm:categorical_limit_is_adjoint/ic_triangle} commutes.

  Also, for any object \( A \) in \( \cat{C} \), the following triangle also commutes:
  \begin{equation}\label{eq:thm:categorical_limit_is_adjoint/c_triangle}
    \begin{aligned}
      \includegraphics[page=2]{output/thm__categorical_limit_is_adjoint}
    \end{aligned}
  \end{equation}

  Indeed,
  \begin{equation*}
    \Delta(\eta_A) = \seq{ \eta_A: A \to \lim(\Delta_A) }_{k \in \cat{I}}
  \end{equation*}
  is the constant family consisting of \( \eta_A \) and
  \begin{equation*}
    \varepsilon_{\Delta(A)} = \seq{ \lambda_k^{\Delta_A}: \lim(\Delta_A) \to \underbrace{\Delta_A(k)}_{A} }_{k \in \cat{I}}.
  \end{equation*}
  is a constant family of the single projection of the limit to \( A \). From the commutativity of \eqref{eq:def:category_of_cones/limit} it follows that \eqref{eq:thm:categorical_limit_is_adjoint/c_triangle} also commutes.
\end{proof}

\begin{corollary}\label{thm:categorical_limit_uniqueness}
  A limit (resp. colimit) of a diagram, if it exists, is unique up to a unique isomorphism.

  This statement strengthens \fullref{thm:categorical_limit_uniqueness_lemma}.
\end{corollary}
\begin{proof}
  Follows from \fullref{thm:functor_adjoint_uniqueness} and \fullref{thm:categorical_limit_is_adjoint}.
\end{proof}

\begin{remark}\label{rem:limit_universal_mapping_property}
  The limit diagram \eqref{eq:def:category_of_cones/limit} may seem unrelated to the universal mapping properties discussed in \fullref{rem:universal_mapping_property}, however it is actually a special case.

  Suppose that, for a given category \( \cat{C} \), the limits over all \( \cat{I} \)-shaped diagrams exist and fix a diagram \( D: \cat{I} \to \cat{C} \). Consider the functors
  \begin{align*}
    \lim:   &[\cat{I}, \cat{C}] \to \cat{C} \\
    \Delta: &\cat{C} \to [\cat{I}, \cat{C}]
  \end{align*}
  discussed in \fullref{thm:categorical_limit_is_adjoint}. We have established that \( \Delta \dashv \lim \).

  For every diagram \( D: \cat{I} \to \cat{C} \), there exist unique up to a unique isomorphism object \( \lim(D) \) in \( \cat{C} \) and canonical projection map \( \lambda: [\Delta \bincirc {\lim}(D)] \to D \) satisfying the following universal mapping property:
  \begin{displayquote}
    For every object \( A \) in \( \cat{C} \) and every natural transformation \( \alpha: \Delta(A) \Rightarrow D \), there exists a unique morphism \( l: A \to \lim(D) \) such that the following diagram commutes:
    \begin{equation}\label{eq:rem:limit_universal_mapping_property/ic_triangle}
      \begin{aligned}
        \includegraphics[page=1]{output/rem__limit_universal_mapping_property}
      \end{aligned}
    \end{equation}
  \end{displayquote}

  For a fixed index object \( k \in \cat{I} \), this becomes:
  \begin{equation}\label{eq:rem:limit_universal_mapping_property/c_triangle_basic}
    \begin{aligned}
      \includegraphics[page=2]{output/rem__limit_universal_mapping_property}
    \end{aligned}
  \end{equation}

  The defining diagram \eqref{eq:def:category_of_cones/limit} of a limit simply encodes the cone naturality condition \eqref{eq:def:category_of_cones/cone_nat} into \eqref{eq:rem:limit_universal_mapping_property/c_triangle_basic}.

  Except for being simpler to check, limits defined via universal mapping properties have the advantage (compared to adjoint functors) that limits can exist for some diagrams and not for others.

  The construction for colimits is dual.
\end{remark}

\begin{proposition}\label{thm:limits_of_empty_diagram}
  The \hyperref[def:category_of_cones/limit]{limits} of an empty diagram over \( \cat{C} \) are the \hyperref[def:universal_objects/terminal]{terminal objects} and the \hyperref[def:category_of_cones/colimit]{colimits} are the \hyperref[def:universal_objects/initial]{initial object}.

  Compare this result with \fullref{thm:limits_of_identity_functor}.
\end{proposition}
\begin{proof}
  A cone of the empty diagram is simply an object of \( \cat{C} \). A limit cone is then an object \( L \) such that every other object \( A \) has a unique morphism \( l_A: A \to L \). This is precisely the definition of a terminal object.

  The statement for colimits follows by \hyperref[thm:categorical_principle_of_duality]{duality}.
\end{proof}

\begin{proposition}\label{thm:limits_of_identity_functor}
  Fix an arbitrary category \( \cat{C} \).

  \begin{thmenum}
    \thmitem{thm:limits_of_identity_functor/initial_object_ks_limit} If \( I \) is an \hyperref[def:universal_objects/initial]{initial object} of \( \cat{C} \), \( (I, \xi) \) is a limit cone of the \hyperref[eq:def:category_of_small_categories/identity]{identity functor} \( \id_{\cat{C}} \), where
    \begin{equation*}
      \xi \coloneqq \seq{ \xi_A: I \to A }_{A \in \cat{C}}
    \end{equation*}
    is the family of unique morphisms with domain \( I \).

    \thmitem{thm:limits_of_identity_functor/limit_ks_knitial_object} Conversely, if \( (L, \lambda) \) is a limit cone of the identity, then \( L \) is an initial object.
  \end{thmenum}

  \hyperref[thm:categorical_principle_of_duality]{Dually}, by \fullref{thm:universal_object_duality} and \fullref{thm:categorical_limit_duality}, the cocones (and colimits) of the identity functor are the terminal objects.

  Compare this result with \fullref{thm:limits_of_empty_diagram}.
\end{proposition}
\begin{proof}
  \SubProofOf{thm:limits_of_identity_functor/initial_object_ks_limit} For an initial object \( I \) with morphisms \( \xi \), and for any morphism \( f: B \to C \), from the uniqueness of the arrows in \( \xi \) it follows that \( \xi_C = f \bincirc \xi_B \). Thus, the following naturality diagram commutes:
  \begin{equation}\label{eq:def:thm:limits_of_identity_functor/initial_object_ks_limit/nat}
    \begin{aligned}
      \includegraphics[page=1]{output/thm__limits_of_identity_functor}
    \end{aligned}
  \end{equation}

  Therefore, \( (I, \xi) \) is a cone.

  Now let \( (A, \alpha) \) be another cone. The naturality of \( \alpha \) implies that the following diagram commutes for every object \( B \):
  \begin{equation}\label{eq:def:thm:limits_of_identity_functor/initial_object_ks_limit/half_limit}
    \begin{aligned}
      \includegraphics[page=2]{output/thm__limits_of_identity_functor}
    \end{aligned}
  \end{equation}

  In particular, since \( \xi_I = \id_I \), for every morphism \( g: A \to I \) we have
  \begin{equation}\label{eq:def:thm:limits_of_identity_functor/initial_object_ks_limit/cone_morphism_uniqueness}
    \begin{aligned}
      \includegraphics[page=3]{output/thm__limits_of_identity_functor}
    \end{aligned}
  \end{equation}
  which shows that \( \alpha_I = g \) and thus \( \alpha_I \) is the unique morphism from \( A \) to \( I \).

  Therefore, for any morphism \( f: B \to C \), the following diagram commutes:
  \begin{equation}\label{eq:def:thm:limits_of_identity_functor/initial_object_ks_limit/limit}
    \begin{aligned}
      \includegraphics[page=4]{output/thm__limits_of_identity_functor}
    \end{aligned}
  \end{equation}

  This is precisely the defining diagram for a limit cone. Therefore, \( (I, \xi) \) is a limit cone.

  \SubProofOf{thm:limits_of_identity_functor/limit_ks_knitial_object} Let \( (L, \lambda) \) be a limit cone of the identity. The component \( \lambda_B \) of the natural transformation \( \lambda \) is a morphism from \( L \) to \( B \). In order for \( L \) to be an initial object, these morphisms must be unique.

  Since \( (L, \lambda) \) is a limit cone, for any cone \( (A, \alpha) \) and any morphism \( f: B \to C \), there exists a unique morphism \( l_A: A \to L \) such that the following diagram commutes:
  \begin{equation}\label{eq:def:thm:limits_of_identity_functor/limit_ks_knitial_object/limit}
    \begin{aligned}
      \includegraphics[page=5]{output/thm__limits_of_identity_functor}
    \end{aligned}
  \end{equation}

  In particular, there exists a unique map \( l_L: L \to L \) for the cone \( (L, \lambda) \) such that the following diagram commutes:
  \begin{equation}\label{eq:def:thm:limits_of_identity_functor/limit_ks_knitial_object/endomorphism_uniqueness}
    \begin{aligned}
      \includegraphics[page=6]{output/thm__limits_of_identity_functor}
    \end{aligned}
  \end{equation}

  Since \eqref{eq:def:thm:limits_of_identity_functor/limit_ks_knitial_object/endomorphism_uniqueness} commutes with \( \id_L \) instead of \( l_L \), by uniqueness it follows that \( l_L = \id_L \). Since, by the naturality of \( \lambda \), \( \lambda_L \) also satisfies this condition, \( \lambda_L = \id_L \).

  From the naturality of \( \lambda \), for any map \( f: L \to C \) it follows that the following diagram commutes:
  \begin{equation}\label{eq:def:thm:limits_of_identity_functor/limit_ks_knitial_object/morphism_uniqueness}
    \begin{aligned}
      \includegraphics[page=7]{output/thm__limits_of_identity_functor}
    \end{aligned}
  \end{equation}

  Therefore, \( f = \lambda_B \). Since \( f \) was an arbitrary morphism with domain \( L \), we conclude that \( L \) has a unique morphism to every object in \( \cat{C} \). Hence, \( L \) is an initial object.
\end{proof}

\begin{example}\label{ex:limits_of_partially_ordered_set}
  We will show that limits and colimits correspond to suprema and infima of partially ordered sets.

  Let \( (P, \leq) \) be a \hyperref[def:partially_ordered_set]{partially ordered set} and \( \cat{P} \) be its corresponding \hyperref[def:skeletal_category]{skeletal} \hyperref[def:preorder_category]{preorder category}. The correspondence is discussed in \fullref{thm:order_category_isomorphism}.

  The image of a diagram \( D: \cat{I} \to \cat{P} \) is a pair \( (A, R) \), where \( A \) is the set of objects \( D(\obj(\cat{I})) \) in the image \( D(\cat{I}) \) and \( R \) is a subrelation of \( \leq \). This relation is reflexive, however it may not even be a preorder as shown in \fullref{ex:functor_image_not_a_category}.

  Conversely, every subset \( A \subseteq P \) with a corresponding category \( \cat{A} \) is given by the diagram \( \Iota_A \), where the \hyperref[def:subcategory]{inclusion functor} \( \Iota_A: \cat{A} \to \cat{P} \).

  A cone of a diagram \( D: \cat{I} \to \cat{P} \) is a \hyperref[def:extremal_points/upper_and_lower_bounds]{lower bound} of the set \( D(\obj(\cat{I})) \). The relation induced by \( D \) does not matter here.

  Indeed, the cone \( (x, R) \) in \( \cat{P} \) consists of morphisms with domain \( x \). That is, \( R \) is a subrelation of \( \leq \) whose first component is \( x \). Clearly then \( x \) is a lower bound of the set \( D(\obj(\cat{I})) \).

  For the limiting cone \( (y, S) \), it holds that \( x \leq y \) for every other cone \( (x, R) \). Both \( R \) and \( S \) are subrelations of \( \leq \) with the same second components. Thus, \( y \) is the greatest lower bound of \( D(\obj(\cat{I})) \).

  \hyperref[thm:categorical_principle_of_duality]{Dually}, cocones are upper bounds and colimits are suprema.

  As mentioned, the relation induced by the diagram \( D \) does not actually matter. Therefore, we may choose, without loss of generality, \( D \) to be a \hyperref[def:discrete_category]{discrete category}. It then follows that infima and suprema correspond to \hyperref[def:discrete_category_limits]{products} and \hyperref[def:discrete_category_limits]{coproduct}, however what we have shown here is more general.
\end{example}

\begin{definition}\label{def:direct_and_inverse_limits}
  We will define limits and colimits of diagrams over infinite \hyperref[def:partial_order_chain]{chains} of integers. These notions predate limits and colimits, which explains why their names may seem inconsistent with other limits and colimits.

  \begin{thmenum}
    \thmitem{def:direct_and_inverse_limits/direct}\mcite[example 5.2.15]{Leinster2016Basic} Consider the category \hyperref[thm:order_category_isomorphism]{induced} by the positive integers
    \begin{equation}\label{eq:def:direct_and_inverse_limits/direct}
      \begin{aligned}
        \includegraphics[page=1]{output/def__direct_and_inverse_limits}
      \end{aligned}
    \end{equation}

    A \hyperref[def:category_of_cones/colimit]{colimit} over a diagram of this shape is called a \term{direct limit}. In this context, the diagram itself is sometimes called a \term{direct system}.

    \thmitem{def:direct_and_inverse_limits/inverse}\mcite[example 5.1.21(d)]{Leinster2016Basic} Now consider the opposite category
    \begin{equation}\label{eq:def:direct_and_inverse_limits/inverse}
      \begin{aligned}
        \includegraphics[page=2]{output/def__direct_and_inverse_limits}
      \end{aligned}
    \end{equation}

    Somewhat confusingly, a \hyperref[def:category_of_cones/limit]{limit} of a diagram of this shape is called an \term{inverse limit}. In this context, the diagram itself is sometimes called an \term{inverse system}.
  \end{thmenum}
\end{definition}

\begin{example}\label{ex:def:direct_and_inverse_limits}
  We will list several examples of \hyperref[def:direct_and_inverse_limits/direct]{direct} and \hyperref[def:direct_and_inverse_limits/inverse]{inverse} limits.

  \begin{thmenum}
    \thmitem{ex:def:direct_and_inverse_limits/vector_space_direct} Consider the chain of \hyperref[def:vector_space]{vector spaces}
    \begin{equation}\label{eq:ex:def:direct_and_inverse_limits/vector_space_direct/diagram}
      \begin{aligned}
        \includegraphics[page=1]{output/ex__def__direct_and_inverse_limits}
      \end{aligned}
    \end{equation}
    where \( \iota_n^m: \BbbK^n \to \BbbK^m \) is simply the canonical inclusion map.

    Let \( \BbbK_0^\infty \) be the vector space of all sequences in \( \BbbK \) with only finitely many nonzero elements. For each positive integer \( n \), denote by \( \iota_n^\infty: \BbbK^n \to \BbbK_0^\infty \) the canonical inclusion.

    Then \( (\BbbK_0^\infty, \iota) \) is a \hyperref[def:category_of_cones/cocone]{cocone} of \eqref{eq:ex:def:direct_and_inverse_limits/vector_space_direct/diagram}. We will show that it is a colimit cocone, i.e. that \( \BbbK_0^\infty \) is a \hyperref[def:direct_and_inverse_limits]{direct limit} of \eqref{eq:ex:def:direct_and_inverse_limits/vector_space_direct/diagram}.

    Let \( (A, \alpha) \) be another cocone. We want to define a linear map \( l_A: \BbbK_0^\infty \to A \) so that the following diagram commutes:
    \begin{equation}\label{eq:ex:def:direct_and_inverse_limits/vector_space_direct/limit}
      \begin{aligned}
        \includegraphics[page=2]{output/ex__def__direct_and_inverse_limits}
      \end{aligned}
    \end{equation}

    For every vector \( x \in \BbbK^m \), we must have
    \begin{equation*}
      \alpha_n(x) = l_A(\iota_n^\infty(x)).
    \end{equation*}

    This implies the obvious definition where, given a vector \( x \in \BbbK_0^\infty \) whose greatest nonzero element has index \( m \), we define
    \begin{equation*}
      l_A(x) \coloneqq \alpha_m(x).
    \end{equation*}

    This map is well-defined because the compatibility with inclusion maps guarantees that \( \alpha_{m+1}(x) = \iota_{m+1}(\alpha^{m+1}_m(x)) \), i.e. the result obtained by using \( \alpha_m \) and \( \alpha_{m+1} \) is the same.

    Therefore, \( (\BbbK_0^\infty, \iota) \) is a direct limit of the diagram \eqref{eq:ex:def:direct_and_inverse_limits/vector_space_direct/diagram}.

    \thmitem{ex:def:direct_and_inverse_limits/vector_space_inverse} \hyperref[thm:categorical_principle_of_duality]{Dually}, consider the chain
    \begin{equation}\label{eq:ex:def:direct_and_inverse_limits/vector_space_inverse/diagram}
      \begin{aligned}
        \includegraphics[page=3]{output/ex__def__direct_and_inverse_limits}
      \end{aligned}
    \end{equation}
    where
    \begin{equation*}
      \begin{aligned}
        &\pi_n^m: \BbbK^m \to \BbbK^n \\
        &\pi_n^m(x_1, \ldots, x_n, x_{k+1}, \ldots, x_m) \coloneqq (x_1, \ldots, x_n).
      \end{aligned}
    \end{equation*}

    Let \( \BbbK^\infty \) be the vector space of all sequences in \( \BbbK \) and, for each positive integer \( n \), define \( \pi_n^\infty \) as a truncation in the obvious way.

    Then \( (\BbbK^\infty, \pi^\infty) \) is a \hyperref[def:category_of_cones/cone]{cone} of \eqref{eq:ex:def:direct_and_inverse_limits/vector_space_direct/diagram}. We will show that it is a limit cone, i.e. that \( \BbbK^\infty \) is a \hyperref[def:direct_and_inverse_limits]{direct limit} of \eqref{eq:ex:def:direct_and_inverse_limits/vector_space_direct/diagram}.

    Let \( (A, \alpha) \) be another cone. We want to define \( l_A: A \to \BbbK^\infty \) so that the following diagram commutes:
    \begin{equation}\label{eq:ex:def:direct_and_inverse_limits/vector_space_inverse/limit}
      \begin{aligned}
        \includegraphics[page=4]{output/ex__def__direct_and_inverse_limits}
      \end{aligned}
    \end{equation}

    For every \( a \in A \), we must have
    \begin{equation*}
      \alpha_n(a) = \pi_n^\infty(l_A(a)),
    \end{equation*}
    which implies the obvious definition where the \( n \)-th coordinate of the vector \( l_A(a) \) is
    \begin{equation*}
      [l_A(a)]_n \coloneqq [\alpha_n(a)]_n.
    \end{equation*}

    That is, \( l_A(a) \) is a sequence whose \( n \)-th coordinate is the \( n \)-th coordinate \( \alpha_n(a) \). This is well-defined because, for \( m > n \), we have
    \begin{equation*}
      \pi_m^n \bincirc \alpha_m = \alpha_n.
    \end{equation*}

    \thmitem{ex:def:direct_and_inverse_limits/generalized_intersection} In \hyperref[def:concrete_category]{concrete categories}, \hyperref[def:direct_and_inverse_limits/inverse]{inverse limits} are generalizations of \hyperref[thm:zfc_existence_theorems/arbitrary_intersection]{set intersections}.

    As an example, consider the chain of sets
    \begin{equation}\label{eq:ex:inverse_limit_as_intersection/diagram}
      \begin{aligned}
        \includegraphics[page=5]{output/ex__def__direct_and_inverse_limits}
      \end{aligned}
    \end{equation}
    where \( A_{k+1} \subseteq A_k \) for every positive integer \( k \) and \( \iota_k^m: A_m \to A_k \) is simply the canonical inclusion map.

    Then the intersection \( \bigcap_{k=1}^\infty A_k \) along with its inclusion maps is a limit of \eqref{eq:ex:inverse_limit_as_intersection/diagram}. This is a consequence of the discussion in \fullref{ex:limits_of_partially_ordered_set}.

    We can replace the inclusion maps \( \iota_k^m \) with other injective functions, or even non-injective functions. In this case, we would obtain a \enquote{generalized intersection}.
  \end{thmenum}
\end{example}

\begin{definition}\label{def:discrete_category_limits}
  Fix an arbitrary category \( \cat{C} \). A \term{product} in \( \cat{C} \) is a \hyperref[def:category_of_cones/limit]{limit} over a diagram \( D: \cat{I} \to \cat{C} \), whose domain \( \cat{I} \) is a \hyperref[def:discrete_category]{discrete category}. \hyperref[thm:categorical_principle_of_duality]{Dually}, a \term{coproduct} or \term{sum} in \( \cat{C} \) is a colimit of \( D \).

  It will be convenient for us to speak about the product of an \hyperref[def:cartesian_product/indexed_family]{indexed family} \( \seq{ X_k }_{k \in \mscrK} \) of objects in \( \cat{C} \).

  \begin{minipage}[t]{0.47\textwidth}
    A cone with vertex \( A \) consists of morphisms with signatures
    \begin{equation*}
      \alpha = \seq{ \alpha_k: A \to X_k }.
    \end{equation*}
  \end{minipage}
  \hfill
  \begin{minipage}[t]{0.47\textwidth}
    A cocone with vertex \( A \) consists of morphisms with signatures
    \begin{equation*}
      \alpha = \seq{ \alpha_k: X_k \to A }.
    \end{equation*}
  \end{minipage}

  \begin{minipage}[t]{0.47\textwidth}
    A product \hyperref[def:category_of_cones/cone]{cone} \( (\prod_{k \in \mscrK} X_k, \pi) \) satisfies the following \hyperref[rem:limit_universal_mapping_property]{universal mapping property}:
    \begin{displayquote}
      For every cone \( (A, \alpha) \), there exists a unique morphism
      \begin{equation*}
        l_A: A \to \prod_{k \in \mscrK} X_k,
      \end{equation*}
      such that the following diagram commutes:
    \end{displayquote}
    \begin{equation}\label{eq:def:discrete_category_limits/product}
      \begin{aligned}
        \includegraphics[page=1]{output/def__discrete_category_limits}
      \end{aligned}
    \end{equation}
  \end{minipage}
  \hfill
  \begin{minipage}[t]{0.47\textwidth}
    A coproduct \hyperref[def:category_of_cones/cocone]{cocone} \( (\coprod_{k \in \mscrK} X_k, \iota) \) satisfies the following \hyperref[rem:limit_universal_mapping_property]{universal mapping property}:
    \begin{displayquote}
      For every cone \( (A, \alpha) \), there exists a unique morphism
      \begin{equation*}
        l_A: A \to \coprod_{k \in \mscrK} X_k,
      \end{equation*}
      such that the following diagram commutes:
    \end{displayquote}
    \begin{equation}\label{eq:def:discrete_category_limits/coproduct}
      \begin{aligned}
        \includegraphics[page=3]{output/def__discrete_category_limits}
      \end{aligned}
    \end{equation}
  \end{minipage}

  \begin{minipage}[t]{0.47\textwidth}
    We call the morphism \( \pi_k \) the \term{canonical projection} of the product onto \( X_k \), even though it may not be a surjective function.
  \end{minipage}
  \hfill
  \begin{minipage}[t]{0.47\textwidth}
    We call the morphism \( \iota_k \) the \term{canonical inclusion} of \( X_k \) into the coproduct, even though it may not be an injective function.
  \end{minipage}
  \medskip

  From \fullref{thm:limits_of_empty_diagram} it follows that the product (resp. coproduct) of an empty family is a terminal (resp. initial) object of \( \cat{C} \).

  As in the case of general limits, we call \( \prod_{k \in \mscrK} X_k \) \hi{the} product of the family \( X \).

  In the case of only two objects, their product is given by the following diagram:
  \begin{equation}\label{eq:def:discrete_category_limits/product/binary}
    \begin{aligned}
      \includegraphics[page=2]{output/def__discrete_category_limits}
    \end{aligned}
  \end{equation}
  and their coproduct by
  \begin{equation}\label{eq:def:discrete_category_limits/coproduct/binary}
    \begin{aligned}
      \includegraphics[page=4]{output/def__discrete_category_limits}
    \end{aligned}
  \end{equation}
\end{definition}

\begin{proposition}\label{thm:discrete_category_limits_in_set}
  The \hyperref[def:discrete_category_limits]{product} in the category \hyperref[def:category_of_small_sets]{\( \ucat{Set} \)} of \( \mscrU \)-small sets of a family \( \mscrA = \seq{ A_k }_{k \in \mscrK} \) is their \hyperref[def:cartesian_product/product]{Cartesian product} \( \prod_{k \in \mscrK} A_k \) and the \hyperref[def:discrete_category_limits]{coproduct} is their \hyperref[def:disjoint_union]{disjoint union} \( \coprod_{k \in \mscrK} A_k \).
\end{proposition}
\begin{proof}
  \SubProof{Proof for products} Consider the Cartesian product
  \begin{equation*}
    L \coloneqq \prod_{k \in \mscrK} A_k = \set*{ f: A \to \bigcup_{k \in \mscrK} A_k \given* \qforall {k \in \mscrK} k \in A_k }.
  \end{equation*}

  Since \( \mscrU \) is a model of \logic{ZFC} (with or without the axiom of infinity), hence the product is an object of \( \ucat{Set} \). Define the projection morphisms
  \begin{equation*}
    \pi_k(f) \coloneqq f(k).
  \end{equation*}

  Then \( (L, \pi) \) is a cone for (some diagram forming) \( \mscrA \). Let \( (B, \beta) \) also be a cone and consider the following diagram:
  \begin{equation}\label{eq:thm:discrete_category_limits_in_set/limit}
    \begin{aligned}
      \includegraphics[page=1]{output/thm__discrete_category_limits_in_set}
    \end{aligned}
  \end{equation}

  We want to define a function \( l_B: B \to L \) such that
  \begin{equation*}
    \underbrace{ \pi_k(l_B(b)) }_{[l_B(b)](k)} = \beta_k(b)
  \end{equation*}
  for every \( b \in B \) and \( k \in \mscrK \). This is uniquely determined by \( b \) and \( k \), hence
  \begin{equation*}
    l_B(b) \coloneqq \seq{ \beta_k(b) }_{k \in \mscrK}.
  \end{equation*}

  The cone \( (B, \beta) \) was chosen randomly, and we showed that there exists a unique morphism \( l_B: B \to L \) such that \eqref{eq:thm:discrete_category_limits_in_set/limit} commutes. Therefore, \( (L, \pi) \) is a categorical product of the family \( \mscrA \).

  \SubProof{Proof for coproducts} Now consider the disjoint union
  \begin{equation*}
    L \coloneqq \coprod_{k \in \mscrK} A_k = \set{ (k, a) \given k \in \mscrK \T{and} a \in A_k }
  \end{equation*}
  and the inclusions
  \begin{equation*}
    \iota_k\parens[\Big]{ (k, a) } = a.
  \end{equation*}

  We must show that, for any other cocone \( (B, \beta) \), the following diagram commutes:
  \begin{equation}\label{eq:thm:discrete_category_limits_in_set/colimit}
    \begin{aligned}
      \includegraphics[page=2]{output/thm__discrete_category_limits_in_set}
    \end{aligned}
  \end{equation}

  Analogously to products, the condition
  \begin{equation*}
    \underbrace{ l_B(\iota_k(a)) }_{l_B(k, a)} = \beta_k(a)
  \end{equation*}
  for every \( b \in B \) and \( k \in \mscrK \) uniquely identifies the function
  \begin{equation*}
    l_B(k, a) \coloneqq \beta_k(a).
  \end{equation*}
\end{proof}

\begin{definition}\label{def:equalizers}\mcite[def. 5.1.11]{Leinster2016Basic}
  Consider the index category
  \begin{equation}\label{eq:def:equalizers/index}
    \begin{aligned}
      \includegraphics[page=1]{output/def__equalizers}
    \end{aligned}
  \end{equation}

  A diagram indexed by \eqref{eq:def:equalizers/index} is sometimes called a \term{fork}. A \hyperref[def:category_of_cones/limit]{limit} of a fork is called an \term[ru=уравнитель \cite[36]{ЦаленкоШульеейфер1974}]{equalizer} and a \hyperref[def:category_of_cones/colimit]{colimit} --- a \term[ru=коуравнитель \cite[37]{ЦаленкоШульеейфер1974}]{coequalizer}.

  Note that in \fullref{def:categorical_diagram} we defined commutativity only when at least one path is nontrivial. That is, \eqref{eq:def:equalizers/index} cannot commute by definition since all of its paths have length either \( 0 \) or \( 1 \).

  We will describe equalizers in more detail. Fix a fork in \( \cat{C} \) which we will denote by
  \begin{equation}\label{eq:def:equalizers/raw_diagram}
    \begin{aligned}
      \includegraphics[page=2]{output/def__equalizers}
    \end{aligned}
  \end{equation}

  \begin{minipage}[t]{0.47\textwidth}
    A \hyperref[def:category_of_cones/cone]{cone} with vertex \( A \) over this diagram has a single morphism \( f: A \to X \) such that the following diagram commutes:
    \begin{equation}\label{eq:def:equalizers/cone}
      \begin{aligned}
        \includegraphics[page=3]{output/def__equalizers}
      \end{aligned}
    \end{equation}

    This is equivalent to requiring
    \begin{equation*}
      s \bincirc f = t \bincirc f.
    \end{equation*}

    An equalizer limit cone \( (L, \iota) \) satisfies the following \hyperref[rem:limit_universal_mapping_property]{universal mapping property}:
    \begin{displayquote}
      For every cone \( (A, f) \), \( f \) \hyperref[def:factors_through]{uniquely factors through} \( L \). That is, there exists a unique morphism
      \begin{equation*}
         l_A: A \to L
      \end{equation*}
      such that the following diagram commutes:
    \end{displayquote}
    \begin{equation}\label{eq:def:equalizers/equalizer}
      \begin{aligned}
        \includegraphics[page=4]{output/def__equalizers}
      \end{aligned}
    \end{equation}
  \end{minipage}
  \hfill
  \begin{minipage}[t]{0.47\textwidth}
    A \hyperref[def:category_of_cones/cocone]{cocone} with vertex \( A \) over this diagram has a single morphism \( f: Y \to A \) such that the following diagram commutes:
    \begin{equation}\label{eq:def:equalizers/cocone}
      \begin{aligned}
        \includegraphics[page=5]{output/def__equalizers}
      \end{aligned}
    \end{equation}

    This is equivalent to requiring
    \begin{equation*}
      f \bincirc s = f \bincirc t.
    \end{equation*}

    A coequalizer cocone \( (L, \pi) \) satisfies the following \hyperref[rem:limit_universal_mapping_property]{universal mapping property}:
    \begin{displayquote}
      For every cocone \( (A, f) \), \( f \) \hyperref[def:factors_through]{uniquely factors through} \( L \). That is, there exists a unique morphism
      \begin{equation*}
        l_A: L \to A
      \end{equation*}
      such that the following diagram commutes:
    \end{displayquote}
    \begin{equation}\label{eq:def:equalizers/coequalizer}
      \begin{aligned}
        \includegraphics[page=6]{output/def__equalizers}
      \end{aligned}
    \end{equation}
  \end{minipage}

  The requirement that \eqref{eq:def:equalizers/equalizer} commutes does not mean that \( p = q \). As discussed in \fullref{def:categorical_diagram}, for commutative diagrams, we only consider a pair of paths if at least one of them is nontrivial. We made this requirement in order to allow parallel morphisms.

  Note how we interchanged the notation for the projections and inclusions compared to \fullref{def:discrete_category_limits} --- \( (L, \iota) \) is a \hi{limit} of a fork, while \( (L, \pi) \) is a \hi{colimit}. The reason for this is that equalizers are usually canonical inclusions, while coequalizers are projections.
\end{definition}

\begin{proposition}\label{thm:equalizer_invertibility}
  If \( (L, \iota) \) is an \hyperref[def:equalizers]{equalizer} in the category \( \cat{C} \), then \( \iota \) is a \hyperref[def:morphism_invertibility/left_cancellative]{monomorphism}.

  \hyperref[thm:categorical_principle_of_duality]{Dually}, if \( (L, \pi) \) is a \hyperref[def:equalizers]{coequalizer}, then \( \pi \) is an \hyperref[def:morphism_invertibility/right_cancellative]{epimorphism}.
\end{proposition}
\begin{proof}
  Fix an equalizer cone \( (L, \iota) \) of the fork \eqref{eq:def:equalizers/raw_diagram}. Fix any object \( A \) in \( \cat{C} \) and any two parallel morphisms \( a_1, a_2: A \to X \) such that
  \begin{equation*}
    \iota \bincirc a_1 = \iota \bincirc a_2.
  \end{equation*}

  Then
  \begin{equation*}
    g \bincirc \iota \bincirc a_1 = h \bincirc \iota \bincirc a_2,
  \end{equation*}
  and thus \( \iota \bincirc a_1 = \iota \bincirc a_2 \) is the morphism of a cone. Hence, both \( (A, \iota \bincirc a_1) \) and \( (A, \iota \bincirc a_2) \) are cones and thus \( a_1 \) and \( a_2 \) are the unique maps such that the following diagram commutes:
  \begin{equation}\label{eq:thm:equalizer_invertibility/monomorphism}
    \begin{aligned}
      \includegraphics[page=1]{output/thm__equalizer_invertibility}
    \end{aligned}
  \end{equation}

  Therefore, \( a_1 = a_2 \) and, since \( a_1 \) and \( a_2 \) were arbitrary, it follows that \( \iota \) is a monomorphism.

  Dually, if \( (L, \pi) \) is a coequalizer colimit cocone in \( \cat{C} \), then by \fullref{thm:categorical_limit_duality}, \( (L, \pi^{\opcat}) \) is an equalizer in \( \cat{C}^{\opcat} \). Hence, \( \pi^{\opcat} \) is a monomorphism, and by \fullref{thm:morphism_invertibility_duality}, \( \pi \) is an epimorphism.
\end{proof}

\begin{example}\label{ex:equalizers_in_set}
  \hfill
  \begin{thmenum}
    \thmitem{ex:equalizers_in_set/equalizer} For a pair of functions \( s, t: X \to Y \) in \( \cat{Set} \), define the set
    \begin{equation*}
      E = \set{ x \in X \given s(x) = t(x) }.
    \end{equation*}

    With the inclusion map \( \iota: E \to X \), this is an \hyperref[def:equalizers]{equalizer} cone for \( s \) and \( t \).

    We will now prove that it is a limit cone. The pair \( (E, \iota) \) is obviously a cone. For any other cone \( (A, f) \), we must find a map \( l_A: A \to E \) such that the following equalizer diagram commutes:
    \begin{equation}\label{eq:ex:equalizers_in_set/equalizer}
      \begin{aligned}
        \includegraphics[page=1]{output/ex__equalizers_in_set}
      \end{aligned}
    \end{equation}

    In order for \( (A, f) \) to be a cone, the image \( f[A] \) must be a subset of \( E \). In order for \eqref{eq:ex:equalizers_in_set/equalizer} to commute, it only makes sense to take \( l_A \) to be \( f \) with its codomain restricted to \( E \).

    It follows that \( (E, \iota) \) is a limit cone.

    \thmitem{ex:equalizers_in_set/coequalizer} The coequalizer of \( s, t: X \to Y \) is more nuanced. Suppose that \( (L, \pi) \) is a colimit cocone and \( (A, f) \) is any cocone for the coequalizer diagram
    \begin{equation}\label{eq:ex:equalizers_in_set/coequalizer}
      \begin{aligned}
        \includegraphics[page=2]{output/ex__equalizers_in_set}
      \end{aligned}
    \end{equation}

    In order for \( (A, f) \) to be a cocone, for every \( x \in X \), we must have \( f(s(x)) = f(t(x)) \). Thus, \( A \) must be a partition of \( Y \) in a way such that \( s(x) \) and \( t(x) \) belong to the same coset if and only if \( f(s(x)) = f(t(x)) \). Outside the images of \( s \) and \( t \), \( f \) is free to take any value.

    Let \( {\sim} \) be the smallest equivalence relation on \( Y \) such that \( s(x) \sim t(x) \) for every \( x \in X \). Explicitly, this is the \hyperref[thm:equivalence_closure]{equivalence closure} of the relation
    \begin{equation*}
      \set{ (s(x), t(x)) \given x \in X }.
    \end{equation*}

    Consider the partition \( Y / {\sim} \) with the projection map
    \begin{equation*}
      \begin{aligned}
        &\pi: Y \to Y / {\sim} \\
        &\pi(y) \coloneqq [y]
      \end{aligned}
    \end{equation*}

    This is a cocone. Furthermore, it is a colimit cocone because, for any other cocone \( (A, f) \), we can define \( l_A([y]) \coloneqq f(y) \) so that \eqref{eq:ex:equalizers_in_set/coequalizer} commutes.

    Coequalizers mostly make sense in the context of \hyperref[def:group/quotient]{quotient groups}, where partitions are especially well-behaved and admit a much simpler description. See \fullref{def:group/quotient}.
  \end{thmenum}
\end{example}


  \chapter{Order theory}\label{ch:order_theory}

\term{Orders} are special \hyperref[def:binary_relation]{binary relations} which, surprisingly, are used to compare elements in a \hyperref[def:set]{set}. Order theory studies pairs \( X = (X, \leq) \), where \( X \) is a set and \( \leq \) is a \hyperref[def:preordered_set]{preorder}.

We denote orders using symbols rather than letters because it is customary to write orders using \hyperref[def:function_application_syntax]{infix notation}, e.g. \( a \leq b \) rather than \( (a, b) \in {\leq} \).

\cref{fig:ordered_sets_hierarchy} features a hierarchy of ordered sets we will consider here. The dashed lines indicate that the objects under consideration are discussed in other sections.

\begin{figure}[!ht]
  \caption{Some important kinds ordered sets}\label{fig:ordered_sets_hierarchy}
  \smallskip
  \hfill
  \begin{forest}
    [
      {\hyperref[def:preordered_set]{Preordered set}}
        [{\hyperref[def:directed_set]{Directed set}}]
        [
          {\hyperref[def:partially_ordered_set]{Partially ordered set}}
            [
              {\hyperref[def:totally_ordered_set]{Totally ordered set}}
                [
                  {\hyperref[def:well_ordered_set]{Well-ordered set}}, edge=dashed
                  [{\hyperref[def:ordinal]{Ordinal}}, edge=dashed]
                ]
            ]
            [
              {\hyperref[def:lattice]{Semilattice}}
                [
                  {\hyperref[def:lattice]{Lattice}}
                    [
                      {\hyperref[def:heyting_algebra]{Heyting algebra}}
                        [
                          {\hyperref[def:boolean_algebra]{Boolean algebra}}
                          [{\hyperref[def:sigma_algebra]{\( \sigma \)-algebra}}, edge=dashed]
                        ]
                        [{\hyperref[def:category_of_small_locales]{Locale}}, edge=dashed]
                        [{\hyperref[def:topological_space]{Topology}}, edge=dashed]
                    ]
                ]
            ]
        ]
        [{\hyperref[def:equivalence_relation]{Equivalence partition}}]
      ]
  \end{forest}
  \hfill\hfill
\end{figure}

General (semi)lattices also admit algebraic definitions, however these algebraic descriptions have some drawbacks:
\begin{itemize}
  \item There is no general way to extend algebraic operations from finitary to infinitary.

  \item We often implicitly rely on the order structure, for example in \hyperref[def:heyting_algebra]{the definition for Heyting algebra}.
\end{itemize}

  \subsection{Preordered sets}\label{subsec:preordered_sets}

\paragraph{Preorders}

\begin{definition}\label{def:preordered_set}\mcite[13]{Harzheim2005OrderedSets}
  We say that a \hyperref[def:binary_relation]{binary relation} \( \leq \) on \( P \) is a \term[bg=преднаредба (\cite[9]{Проданов1982ФункАнализТом1}), ru=предпорядок (\cite[def 3.1]{Гуров2013Решётки})]{preorder} if it is \hyperref[def:binary_relation/reflexive]{reflexive} and \hyperref[def:binary_relation/transitive]{transitive}. The pair \( (P, \leq) \) is then called a \term{preordered set}.

  It is conventional to use the same symbol \( \leq \) as for \hyperref[def:partially_ordered_set]{partial orders}, however the lack of \hyperref[def:binary_relation/antisymmetric]{antisymmetry} may be confusing --- see \fullref{ex:preorder_nonuniqueness}.

  Preordered sets have the following metamathematical properties:
  \begin{thmenum}[series=def:preordered_set]
    \thmitem{def:preordered_set/inverse}\mcite[14]{Harzheim2005OrderedSets} We define \( \geq \) as the \hyperref[def:binary_relation/inverse]{inverse relation} of \( \leq \).

    \thmitem{def:preordered_set/strict}\mcite[14]{Harzheim2005OrderedSets} We also define the relation \( < \) as \( \leq \) minus the \hyperref[def:binary_relation/diagonal]{diagonal relation} \( \Delta \). This condition corresponds to the following axiom:
    \begin{equation}\label{eq:def:preordered_set/compatibility_nonstrict}
      (\xi \leq \eta) \syniff \parens[\Big]{(\xi < \eta) \synvee (\xi \syneq \eta)}.
    \end{equation}

    By adding \( \anon \synwedge \synneg (x = y) \) to both sides of \eqref{eq:def:preordered_set/compatibility_nonstrict}, using \fullref{thm:de_morgans_laws} and taking irreflexivity of \( < \) into account, we obtain
    \begin{equation}\label{eq:def:preordered_set/compatibility_strict}
      (\xi < \eta) \syniff \parens[\Big]{(\xi \leq \eta) \synwedge \synneg (\xi = \eta)}.
    \end{equation}

    We call \( < \) the \term{strict preorder} associated with \( \leq \), in which context we call \( \leq \) \term{nonstrict preorder}. We also define \( > \) as the converse relation of \( < \).

    \thmitem{def:preordered_set/comparable}\mcite[29]{Harzheim2005OrderedSets} We call the elements \( x \) and \( y \) are \term1{comparable} if either \( x \leq y \) or \( y \leq x \) and \term{incomparable} otherwise.
  \end{thmenum}

  Preordered sets have the following metamathematical properties:
  \begin{thmenum}[resume=def:preordered_set]
    \thmitem{def:preordered_set/theory} Consider a \hyperref[def:first_order_language]{first-order language} \( \mscrL \) with two \hyperref[rem:first_order_formula_conventions/infix]{infix} binary predicate symbols --- \( \leq \) and \( \geq \).

    The theory of preordered sets is a \hyperref[def:first_order_theory]{first-order theory} in \( \mscrL \) consisting of the axioms \eqref{eq:def:binary_relation/reflexive} and \eqref{eq:def:binary_relation/transitive} for \( \leq \) and the compatibility axiom
    \begin{equation}\label{eq:def:preordered_set/theory}
      (\xi \leq \eta) \syniff (\eta \geq \xi).
    \end{equation}

    We purposely avoid adding \( < \) and \( > \) to this language because that would change the behavior of \hyperref[def:first_order_homomorphism]{first-order homomorphisms}. If needed, we can regard them as \hyperref[con:predicate_formula]{predicate formulas}.

    If we want to add strict orders to the language, we should also add one of the compatibility axioms \eqref{eq:def:preordered_set/compatibility_nonstrict} or \eqref{eq:def:preordered_set/compatibility_strict} to the resulting theory (it is unnecessary to add both).

    \thmitem{def:preordered_set/homomorphism} The \hyperref[def:first_order_homomorphism]{first-order homomorphisms} between two preordered sets are the \hyperref[def:order_function/preserving]{nonstrictly order-preserving maps}, which we will discuss shortly.

    \thmitem{def:preordered_set/submodel} Since the theory contains only positive formulas over a language with no functional symbols, any subset \( A \) of the domain of a preordered set \( P \) becomes a preordered set with the induced preorder \( \leq_A \) defined as the restriction of \( \leq_P \) to only elements of \( A \).

    \thmitem{def:preordered_set/opposite}\mcite[316]{PicadoPultr2012Frames} We define the \term{dual} or \term{opposite} preordered set of \( (P, \leq) \) as \( (P, \geq) \).

    \thmitem{def:preordered_set/category}  We denote the \hyperref[def:category_of_small_first_order_models]{category of \( \mscrU \)-small models} for the theory of preordered sets by \( \ucat{PreOrd} \).

    This category is isomorphic to that of \( \mscrU \)-small preorder categories --- see \fullref{thm:order_category_isomorphism}.
  \end{thmenum}
\end{definition}
\begin{comments}
  \item Following \incite[3]{Gratzer2011Lattices} and \incite[def 3.1]{Гуров2013Решётки}, we prefer the term \enquote{preorder} to \enquote{quasi-order}. Other authors like \incite[20]{Birkhoff1967Lattices} and \incite[13]{Harzheim2005OrderedSets} instead prefer \enquote{quasi-order}.
\end{comments}

\begin{theorem}[Principle of duality for preorders]\label{thm:preorder_duality}\mcite[5]{Gratzer2011Lattices}
  Consider the \hyperref[def:preordered_set/theory]{first-order theory of preordered sets}. Within it, consider the opposite formula \( \varphi^\oppos \) of the \hyperref[def:first_order_syntax/closed_formula]{closed formula} \( \varphi \), in which we swap all instances of \( \leq \) and \( \geq \).

  If every preordered set \hyperref[def:first_order_model]{satisfies} \( \varphi \), then every preordered set also satisfies \( \varphi^\oppos \).

  More generally, \( (P, \leq) \) satisfies \( \varphi \) if and only if its \hyperref[def:preordered_set/opposite]{dual} \( (P, \geq) \) satisfies \( \varphi^\oppos \).
\end{theorem}
\begin{comments}
  \item The corresponding syntactic statement is that the formula \( \varphi \) is \hyperref[def:logical_framework]{derivable} in the \hyperref[def:preordered_set/theory]{theory of preordered sets} if and only if the opposite formula \( \varphi^\oppos \) is also derivable.

  \item The opposite of the opposite formula of \( \varphi \) is obviously \( \varphi \).

  \item Another form of this duality is formalized in \fullref{thm:order_category_isomorphism}.

  \item Similar statements hold in special cases --- see \fullref{thm:lattice_duality} and \fullref{thm:boolean_algebra_duality}.

  \item The actual replacement can be formalized by performing the \hyperref[def:first_order_substitution/term_in_formula]{simultaneous substitution}
  \begin{equation*}
    \begin{aligned}
      \varphi^\oppos \coloneqq \varphi[
        &\tau_1 \leq \sigma_1 \mapsto \tau_1 \geq \sigma_1, &&\tau_1 \geq \sigma_1 \mapsto \tau_1 \leq \sigma_1, \\
        &\vdots                                       &&\vdots \\
        &\tau_n \leq \sigma_n \mapsto \tau_n \geq \sigma_n, &&\tau_n \geq \sigma_n \mapsto \tau_n \leq \sigma_n]
    \end{aligned}
  \end{equation*}
  for all pairs \( (\tau_k, \sigma_k) \) of terms in \( \varphi \).
\end{comments}
\begin{proof}
  Note that the semantics of \( \leq \) and \( \geq \) are swapped along with the swapping of the corresponding symbols in the formulas. Therefore, if \( (P, \leq) \) satisfies \( \varphi \), then \( (P, \geq) \) must satisfy \( \varphi^\oppos \).

  Now suppose that every preordered set satisfies \( \varphi \). For a given preordered set \( (P, \leq) \), its dual \( (P, \geq) \) satisfies \( \varphi \), and hence \( (P, \leq) \) satisfies \( \varphi^\oppos \). Since \( (P, \leq) \) was chosen arbitrarily, we conclude that every preordered sets satisfies \( \varphi^\oppos \).
\end{proof}

\paragraph{Inequalities}

\begin{definition}\label{def:inequality}\mimprovised
  When regarding some \hyperref[def:preordered_set]{preorder relation} \( \leq \) as a \hyperref[def:first_order_language/pred]{first-order predicate}, we call the corresponding \hyperref[def:first_order_syntax/atomic_formula]{atomic formulas} \term{inequalities}. More simply put, an inequality is a formula of the form \( \tau \leq \sigma \).

  We distinguish between \term{strict inequalities} \( \tau < \sigma \) and \term{nonstrict inequalities} \( \tau \leq \sigma \).

  As in the case of \hyperref[def:first_order_equation]{equations}, we call the set \hyperref[def:first_order_definability]{defined} by an inequality a \term{solution set}.
\end{definition}

\paragraph{Extremal points}

\begin{definition}\label{def:extremal_points}
  We introduce the following terminology for extremal elements of a preordered set. These definitions are often given only for \hyperref[def:partially_ordered_set]{partially ordered sets}, but we prefer a more general setting that allows us to cover some cases like the \hyperref[thm:semiring_divisibility_order]{divisibility preorder}. The downside of this generality is that, without \hyperref[def:binary_relation/antisymmetric]{antisymmetry}, some common notions like maxima and suprema lack uniqueness.

  The notions on the left and on the right are \hyperref[thm:preorder_duality]{dual}, but we discuss both nonetheless.

  \begin{thmenum}
    \thmitem{def:extremal_points/bounds}
    \begin{paracol}{2}
      \begin{leftcolumn}\mcite[21]{Harzheim2005OrderedSets}
        An \term[bg=горна граница (\cite[18]{Тагамлицки1971Диф}) / мажоранта (\cite[10]{Проданов1982ФункАнализТом1}), ru=верхняя грань/мажоранта (\cite[def. 3.8]{Гуров2013Решётки})]{upper bound} for the set \( A \) is an element \( m \) of the ambient preordered set such that \( x \leq m \) for every \( x \in A \).

        We say that the upper bound is \term{strict} if it does not itself belong to \( A \).

        If \( A \) has at least one upper bound, we say that \( A \) is \term{bounded from above}.
      \end{leftcolumn}

      \begin{rightcolumn}
        Dually, \( m \) is a \term[bg=долна граница (\cite[18]{Тагамлицки1971Диф}) / миноранта (\cite[10]{Проданов1982ФункАнализТом1}), ru=нижняя грань/миноранта (\cite[def. 3.8]{Гуров2013Решётки})]{lower bound} of \( A \) if \( m \leq x \) for every \( x \in A \).

        If \( A \) has a lower bound, we say that \( A \) is \term{bounded from below}.

        We say that \( A \) is \term[bg=ограничено (множество) (\cite[19]{Тагамлицки1971Диф}), ru=ограниченное (множество) (\cite[39]{Зорич2019АнализТом1})]{bounded} if it is both bounded from above and from below.
      \end{rightcolumn}
    \end{paracol}

    \thmitem{def:extremal_points/maximal_and_minimal_element}
    \begin{paracol}{2}
      \begin{leftcolumn}\mimprovised
        A \term[ru=максимальный (елемент) (\cite[def. 3.6]{Гуров2013Решётки}), en=maximal (element) (\cite[33]{Harzheim2005OrderedSets})]{maximal element} for the set \( A \subseteq P \) is a member \( m \) of \( A \) such that, whenever \( x \geq m \) for some \( x \in A \), we have \( x \leq m \).

        If antisymmetry holds, the last \enquote{\( x \leq m \)} can be simplified to \enquote{\( x = m \)}.
      \end{leftcolumn}

      \begin{rightcolumn}
        Dually, \( m \in A \) is \term[ru=минимальный (елемент) (\cite[def. 3.6]{Гуров2013Решётки}), en=minimal (element) (\cite[33]{Harzheim2005OrderedSets})]{minimal} if, whenever \( x \leq m \) for some \( x \in A \), we have \( x \geq m \).

        If antisymmetry holds, the last \enquote{\( x \geq m \)} can be simplified to \enquote{\( x = m \)}.
      \end{rightcolumn}
    \end{paracol}

    \thmitem{def:extremal_points/greatest_and_least}
    \begin{paracol}{2}
      \begin{leftcolumn}\mcite[22]{Harzheim2005OrderedSets}
        If \( m \) is an upper bound of \( A \) that belongs to \( A \), we call it a \term[ru=наибольший (елемент) (\cite[def. 3.6]{Гуров2013Решётки})]{greatest element}.

        If antisymmetry holds, greatest elements are unique.
      \end{leftcolumn}

      \begin{rightcolumn}
        Dually, if \( m \) is a lower bound of \( A \) that belongs to \( A \), we call it a \term[ru=наименьший (елемент) (\cite[def. 3.6]{Гуров2013Решётки})]{least element}.

        If antisymmetry holds, least elements are unique
      \end{rightcolumn}
    \end{paracol}

    \thmitem{def:extremal_points/maximum_and_minimum}
    \begin{paracol}{2}
      \begin{leftcolumn}\mimprovised
        In totally ordered sets, due to \fullref{thm:def:totally_ordered_set/maximal_iff_greatest} an element of \( A \) is maximal if and only if it is the greatest element of \( A \). We call such an element the \term{maximum} of \( A \) and denote it by \( \max A \).
      \end{leftcolumn}

      \begin{rightcolumn}
        Dually, in a totally ordered set an element of \( A \) is minimal if and only if it is the least element of \( A \). We call such an element the \term{minimum} of \( A \) and denote it by \( \min A \).
      \end{rightcolumn}
    \end{paracol}

    \thmitem{def:extremal_points/supremum_and_infimum}
    \begin{paracol}{2}
      \begin{leftcolumn}\mcite[22]{Harzheim2005OrderedSets}
        If \( m \) is a minimum among all upper bounds of \( A \), we call it a \term[bg=точна горна граница (\cite[19]{Тагамлицки1971Диф}), ru=точная верхняя граница (\cite[\S 5.12]{Ляпин1960Полугруппы})]{least upper bound} or \term[bg=супремум (\cite[10]{Проданов1982ФункАнализТом1}), ru=супремум (\cite[\S 5.12]{Ляпин1960Полугруппы})]{supremum} of \( A \).

        If antisymmetry holds, suprema are unique, and we use the notation \( \sup A \).
      \end{leftcolumn}

      \begin{rightcolumn}
        Dually, if \( m \) is a maximum among all lower bounds of \( A \), we call it a \term[bg=точна долна граница (\cite[19]{Тагамлицки1971Диф}), ru=точная нижняя граница (\cite[\S 5.12]{Ляпин1960Полугруппы})]{greatest lower bound} or \term[bg=инфимум (\cite[10]{Проданов1982ФункАнализТом1}), ru=супремум (\cite[\S 5.12]{Ляпин1960Полугруппы})]{infimum} of \( A \).

        If antisymmetry holds, infima are unique, and we use the notation \( \inf A \).
      \end{rightcolumn}
    \end{paracol}

    \thmitem{def:extremal_points/top_and_bottom}
    \begin{paracol}{2}
      \begin{leftcolumn}\mcite[15]{DaveyPriestley2002Lattices}
        If the ambient space itself has a maximum, we call it a \term{top element}.

        If antisymmetry holds, the top element is unique, and we denote it via \( \top \).
      \end{leftcolumn}

      \begin{rightcolumn}
        Dually, if the ambient space has a minimum, we call it a \term{bottom element}.

        If antisymmetry holds, the bottom element is unique, and we use \( \bot \).
      \end{rightcolumn}
    \end{paracol}
  \end{thmenum}
\end{definition}

\begin{proposition}\label{thm:def:extremal_points}
  Fix a \hyperref[def:preordered_set]{preordered set} \( P \) and some nonempty subset \( A \) of \( P \). The extremal point notions from \fullref{def:extremal_points} have the following basic properties:
  \begin{thmenum}
    \thmitem{thm:def:extremal_points/empty_exact_bounds} If \( P \) has a bottom element, it is a supremum of the empty set.

    Dually, if \( P \) has a top element, it is an infimum of the empty set.

    \thmitem{thm:def:extremal_points/greatest_is_maximal} The greatest element of \( A \), if it exists, is also a maximal element of \( A \).

    Dually, the least element of \( A \) is a minimal element of \( A \).

    The converse holds in \hyperref[def:totally_ordered_set]{totally ordered sets} --- see \fullref{thm:def:totally_ordered_set/maximal_iff_greatest}.

    \thmitem{thm:def:extremal_points/greatest_is_supremum} The greatest element of \( A \), if it exists, is the supremum of \( A \).

    Dually, the least element of \( A \) is the infimum of \( A \).

    \thmitem{thm:def:extremal_points/finite_set_has_maximal_element} If \( A \) is \hi{finite}, it has a maximal element and a minimal element.

    \thmitem{thm:def:extremal_points/unique_maximal_element} If \( m \) is a unique maximal element of \( A \) and if every nonempty subset of \( A \) has at least one maximal element, then \( m \) is a greatest element of \( A \).

    Dually, if \( m \) is a unique minimal element and every nonempty subset of \( A \) has a minimal element, then \( m \) is a minimum.
  \end{thmenum}
\end{proposition}
\begin{proof}
  \SubProofOf{thm:def:extremal_points/empty_exact_bounds} A supremum is a minimum among all upper bounds. All elements of \( P \) are vacuously upper bounds of \( \varnothing \) since there is nothing to compare them to, so a minimum of \( P \) is a supremum of \( \varnothing \).

  \SubProofOf{thm:def:extremal_points/greatest_is_maximal} Let \( m \) be a greatest element of \( A \).

  Let \( a \) be a member of \( A \) such that \( m \leq a \). But \( m \leq a \) since \( m \) is an upper bound of \( A \). Generalizing on \( a \), we conclude that \( m \) is maximal.

  \SubProofOf{thm:def:extremal_points/greatest_is_supremum} Let \( m \) be a greatest element of \( A \).

  Let \( u \) be any upper bound of \( A \). Then, in particular, \( m \leq u \) because \( m \) belongs to \( A \). Generalizing on \( u \), we conclude that \( m \) is a least upper bound of \( A \).

  \SubProofOf{thm:def:extremal_points/finite_set_has_maximal_element} Suppose that \( A \) is finite. Label the elements of \( A \) as follows:
  \begin{equation*}
    a_1, a_2, \ldots, a_n.
  \end{equation*}

  We will use induction on \( n \) to show that \( A \) has a maximal element.
  \begin{itemize}
    \item The base case \( n = 1 \) is trivial --- \( A \) has one element, which is maximal.
    \item Suppose that every set of size \( n - 1 \) has a maximal element. Let \( m \) be a maximal element of \( A \setminus \set{ a_n } \) . Then we have the following possibilities:
    \begin{itemize}
      \item If \( m \) is not comparable to \( a_n \), then both \( m \) and \( a_n \) are maximal for \( A \).

      \item If \( m \leq a_n \), then \( a_n \) is maximal for \( A \). Indeed, if \( a_k \geq a_n \) for \( k < n \), then \( a_k \leq m \) and, by transitivity, \( a_k \leq a_n \).

      \item If \( m \geq a_n \), the \( m \) is maximal for \( A \). Indeed, if \( a_k \geq m \) for \( k < n \), then \( a_k \leq m \) because \( m \) is maximal for \( A \setminus \set{ a_n } \), and if \( a_n \geq m \), we already have \( m \geq a_n \) by assumption.
    \end{itemize}

    In all cases, we have shown that \( A \) has at least one maximal element.
  \end{itemize}

  \SubProofOf{thm:def:extremal_points/unique_maximal_element} Suppose that every nonempty subset of \( A \) has a maximal element. Suppose also that \( m \) is a unique maximal element of \( A \) itself.

  Consider the set \( B \) of elements incomparable to \( m \). Suppose that \( M \) is nonempty. Let \( m' \) be a maximal element of \( B \). Since \( m \) and \( m' \) are incomparable, \( m' \) is then a maximal element of \( A \).

  But we have assumed that \( m \) is unique, hence \( m = m' \), which contradicts our choice of \( m' \).

  The obtained contradiction shows that \( B \) is empty. Then every element of \( A \) is comparable to \( m \). Fix an arbitrary element \( a \) of \( A \).
  \begin{itemize}
    \item If \( a \geq m \), then, since \( m \) is maximal, we have \( a \leq m \).
    \item The remaining option is \( a \leq m \).
  \end{itemize}

  Therefore, \( m \) is a greatest element of \( A \).
\end{proof}

\begin{example}\label{ex:thm:def:extremal_points}
  We list examples of for the extremal point notions from \fullref{def:extremal_points}:
  \begin{thmenum}
    \thmitem{ex:thm:def:extremal_points/empty} Fix some arbitrary preordered set \( P \) and consider the empty subset \( \varnothing \).

    Clearly every member of \( P \) is both an upper and a lower bound of \( \varnothing \). If \( P \) has a bottom element, it is a supremum of \( \varnothing \) because it is a minimum among all upper bounds of \( \varnothing \), that is, among all elements of \( P \). Similarly, if \( P \) has a top element, it is an infimum of \( \varnothing \).

    Since \( \varnothing \) is empty, it cannot have a greatest element, nor a maximal element.

    \thmitem{ex:thm:def:extremal_points/one} Consider the one-element set \( \set{ a } \). The only possible reflexive relation entails \( a \leq a \), so \( a \) is the top and bottom of \( \set{ a } \).

    \thmitem{ex:thm:def:extremal_points/two_isolated} Consider the two-element set \( \set{ a, b } \) with the \hyperref[def:binary_relation/diagonal]{diagonal relation} (every element is related only to itself).

    Then \( a \) is the only upper bound of \( \set{ a } \), hence also a maximal element, greatest element and supremum. It is similarly a lower bound, minimal element, minimum and infimum.

    But for \( \set{ a, b } \), it is neither an upper nor lower bound.

    On the other hand, both \( a \) and \( b \) are maximal elements of \( \set{ a, b } \), because there exist no elements greater than them. Similarly, they are both minimal.

    \thmitem{ex:thm:def:extremal_points/two_connected} Consider again the two-element set \( \set{ a, b } \), but this time with \( a \leq b \). \Fullref{thm:two_element_lattice} implies that all such preordered sets are isomorphic.

    The \( a \) is a bottom element and \( b \) is a top element.

    \thmitem{ex:thm:def:extremal_points/two_equivalent} Consider yet again the two-element set \( \set{ a, b } \), but this time with both \( a \leq b \) and \( b \leq a \).

    Both of them are a top element and a bottom element. Because we lack antisymmetry in general, we cannot conclude that \( a = b \), so we have two distinct top elements.

    \thmitem{ex:thm:def:extremal_points/three_incomparable} Consider the three-element set \( \set{ a, b, c } \) such that \( a \leq c \) and \( b \leq c \).

    Clearly \( c \) is a top element. But there is no bottom. Both \( a \) and \( b \) are minimal elements of \( \set{ a, b, c } \), but neither of them is a minimum because they are not comparable.

    \thmitem{ex:thm:def:extremal_points/unique_maximal_element_not_greatest}\mcite{MathSE:unique_maximal_element_that_is_not_greatest} Consider the set \( \BbbZ \) of integers with the standard ordering from \fullref{def:integer_ordering}.

    Adjoin some symbol \( u \) and define a relation to \( \BbbZ \cup \set{ u } \) extending the ordering on \( \BbbZ \) with \( u \leq u \).

    Then \( u \) is incomparable with any integer. It is the unique maximal element of \( \BbbZ \cup \set{ u } \) because no integer is greater than \( u \). But, since it is incomparable with integers, \( u \) is not a greatest element.

    \thmitem{ex:thm:def:extremal_points/nonunique_maxima} Consider again the integers \( \BbbZ \), but this time adjoin two incomparable elements --- \( \infty \) and \( \tieinfty \) --- and let both of them be greater than any integer.

    Then both \( \infty \) and \( \tieinfty \) are upper bounds of \( \BbbZ \) in \( \BbbZ \cup \set{ \infty, \tieinfty } \). Furthermore, both of them are minimal in the set \( \set{ \infty, \tieinfty } \) of upper bounds, but, since they are not comparable, neither is a least upper bound.
  \end{thmenum}
\end{example}

\paragraph{Functions between ordered sets}

\begin{definition}\label{def:order_function}
  Fix \hyperref[def:preordered_set]{preordered sets} \( (P, \leq_P) \) and \( (Q, \leq_Q) \) and consider an arbitrary function \( f: P \to Q \). \Fullref{rem:order_homomorphism_terminology} demonstrates how widely the terminology for such functions varies. We try to choose the most unambiguous terminology.

  \begin{thmenum}
    \thmitem{def:order_function/preserving}\mcite[35]{Harzheim2005OrderedSets} We call \( f \) \term{order-preserving} or an \term{order homomorphism} if
    \begin{equation}\label{eq:def:order_function/preserving}
      x \leq_P y \T{implies} f(x) \leq_Q f(y).
    \end{equation}

    If the inequalities are strict, i.e. if
    \begin{equation}\label{eq:def:order_function/preserving/strict}
      x <_P y \T{implies} f(x) <_Q f(y),
    \end{equation}
    we call \( f \) \term{strictly order-preserving}.

    To disambiguate, we sometimes call function satisfying \eqref{eq:def:order_function/preserving} \term{nonstrictly order-preserving}.

    While we avoid the term \enquote{monotone} for non-real-valued functions, we refer to this property as \term{monotonicity}.

    \thmitem{def:order_function/reflecting}\mcite[35]{Harzheim2005OrderedSets} We call \( f \) \term{order-reflecting} if it satisfies \hyperref[def:conditional_formula/converse]{converse} of \eqref{eq:def:order_function/preserving}:
    \begin{equation}\label{eq:def:order_function/reflecting}
      f(x) \leq_Q f(y) \T{implies} x \leq_Q y.
    \end{equation}

    \thmitem{def:order_function/reversing}\mcite[35]{Harzheim2005OrderedSets} \hyperref[thm:preorder_duality]{Dually}, we call \( f \) \term{order-reversing} if
    \begin{equation}\label{eq:def:order_function/reversing}
      x \leq_P y \T{implies} f(x) \geq_Q f(y).
    \end{equation}
    and \term{strictly decreasing} or \term{strictly order-reversing} if
    \begin{equation}\label{eq:def:order_function/reversing/strict}
      x <_P y \T{implies} f(x) >_Q f(y).
    \end{equation}

    \thmitem{def:order_function/increasing}\mcite[35]{Harzheim2005OrderedSets} In the case of \hyperref[def:totally_ordered_set]{totally ordered sets}, we prefer the term (strictly) \term[bg=растяща (функция) (\cite[198]{Тагамлицки1971Диф}), ru=возрастающая (функция) (\cite[132]{ИльинСадовничийСендов1985АнализТом1})]{increasing} (resp. \term[bg=намаляваща (функция) (\cite[198]{Тагамлицки1971Диф}), ru=убывающая (функция) (\cite[132]{ИльинСадовничийСендов1985АнализТом1})]{decreasing}) to (strictly) \enquote{order-preserving} (resp. \enquote{order-reversing}).

    \thmitem{def:order_function/monotone}\mcite[95]{Rudin1976AnalysisPrinciples} In the context of real-valued functions, we use the \term[bg=монотонна (функция) (\cite[199]{Тагамлицки1971Диф}), ru=монотонная (функция) (\cite[\S 47]{Фихтенгольц1968ОсновыТом1})]{monotone} as a collective term for nondecreasing and nonincreasing functions. Similarly, we use \term{strictly monotone} for increasing and decreasing functions, in which context we refer to usual monotone functions as \term{nonstrictly monotone}.

    Outside of analysis, we generally avoid the term \enquote{monotone} due to ambiguity.

    \thmitem{def:order_function/ascending}\incite[35]{Harzheim2005OrderedSets} In the context of \hyperref[def:chain_condition]{chain conditions}, we refer to increasing and decreasing sequences as \term{ascending} and \term{descending}.
  \end{thmenum}
\end{definition}
\begin{comments}
  \item Nonstrict order-preserving maps are used extensively in the theory of \hyperref[subsec:partially_ordered_sets]{partially ordered sets}, in particular in \hyperref[subsec:lattices]{lattice theory}, while strict order-preserving maps are used in the theory of \hyperref[subsec:partially_ordered_sets]{totally ordered sets}, in particular for \hyperref[subsec:ordinals]{ordinals}.
\end{comments}

\begin{remark}\label{rem:order_homomorphism_terminology}
  There is widely varying terminology for the maps defined in \fullref{def:order_function}.

  \begin{itemize}
    \item \enquote{Monotone} is used by
    \begin{itemize}
      \item \incite[30]{Gratzer2011Lattices}, \incite[23]{DaveyPriestley2002Lattices}, \incite[317]{PicadoPultr2012Frames}, \incite[3]{Kelley1975Topology} and \incite[92]{MacLane1998Categories} for what we call a nonstrictly order-preserving map.

      \item \incite[132]{ИльинСадовничийСендов1985АнализТом1}, \incite[\S 47]{Фихтенгольц1968ОсновыТом1} and \incite[def. 4.28]{Rudin1976AnalysisPrinciples} for what we call either a nonincreasing or nondecreasing real-valued function.
    \end{itemize}

    \item \enquote{Weakly monotone} is used \incite[175]{MacLane1998Categories} for what we call a nondecreasing function.

    \item \enquote{Strictly monotone} is used \incite[133]{ИльинСадовничийСендов1985АнализТом1} and \incite[def. 4.28]{Rudin1976AnalysisPrinciples} for what we call a strictly order-preserving real-valued function.

    \item \enquote{Isotone} is used \incite[2]{Birkhoff1967Lattices}, \incite[30]{Gratzer2011Lattices}, \incite[35]{Harzheim2005OrderedSets}, \incite[23]{DaveyPriestley2002Lattices}, \incite[3]{Kelley1975Topology} and \incite[def. 3.9]{Гуров2013Решётки} for what we call a nonstrictly order-preserving function.

    \item \enquote{Reverse isotone} is used \incite[def. 3.9]{Гуров2013Решётки} for what we call a nonstrictly order-reflecting map.

    \item \enquote{Antitone} is used by
    \begin{itemize}
      \item \incite[30]{Gratzer2011Lattices} for what we call an order-reversing map.
      \item \incite[3]{Birkhoff1967Lattices} for when the order-reversing condition \eqref{eq:def:order_function/reversing} is an equivalence rather than an implication
    \end{itemize}

    \item \enquote{Anti-isotone} is used \incite[35]{Harzheim2005OrderedSets} and \incite[def. 3.9]{Гуров2013Решётки} for what we call an order-reversing map.

    \item \enquote{Order-preserving} is used by
    \begin{itemize}
       \item \incite[2]{Birkhoff1967Lattices}, \incite[30]{Gratzer2011Lattices}, \incite{Harzheim2005OrderedSets}, \incite[3]{Kelley1975Topology} and \incite[95]{MacLane1998Categories} for what we call a nondecreasing function.

       \item \incite[4]{Engelking1989Topology} for what we call a strictly order-preserving function, however only for totally ordered sets.
    \end{itemize}

    \item \enquote{Order-reversing} is used by \incite[35]{Harzheim2005OrderedSets} and \incite[95]{MacLane1998Categories} for what we call analogously.

    \item \enquote{Order-reflecting} is used by \incite[35]{Harzheim2005OrderedSets} for what we call analogously.

    \item \enquote{Nondecreasing} is used by
    \begin{itemize}
      \item \incite[35]{Harzheim2005OrderedSets} for what we call analogously.
      \item \incite[8]{Engelking1989Topology} for what we call analogously, however only for totally ordered sets.
    \end{itemize}

    The term \enquote{nonincreasing} is used similarly.

    \item \enquote{Increasing} and \enquote{decreasing} is used by \incite[def. 4.28]{Rudin1976AnalysisPrinciples} for what we call analogously, however only for the real numbers.

    \item (Strictly) \enquote{ascending} (resp. \enquote{descending}) is used by \incite[35]{Harzheim2005OrderedSets} for what we call a (strictly) order-preserving (resp. order-reflecting) function.

    \item \incite[132]{ИльинСадовничийСендов1985АнализТом1}, \incite[\S 47]{Фихтенгольц1968ОсновыТом1} and \incite[def. 17]{Зорич2019АнализТом1} use \enquote{возрастающая функция} (resp. \enquote{убывающая функция}) for what we call a strictly order-preserving (resp. order-reversing) map. The terms can be translated as \enquote{growing} and \enquote{diminishing}, respectively. For non-strict maps, the authors suggest adding a \enquote{не} prefix, similarly to how \enquote{non} is added in English.
  \end{itemize}
\end{remark}

\begin{proposition}\label{thm:def:order_function}
  The functions between ordered sets discussed in \fullref{def:order_function} have the following basic properties:
  \begin{thmenum}
    \thmitem{thm:def:order_function/injective_implies_strict} Every \hyperref[def:function_invertibility/injective]{injective} \hyperref[def:order_function/preserving]{order-preserving map} is \hyperref[def:order_function/preserving]{strict}.

    The converse holds for totally ordered sets --- see \fullref{thm:def:totally_ordered_set/embedding_iff_strict}.

    \thmitem{thm:def:preordered_set/homomorphism_is_reflecting} An injective function between preordered sets is a \hyperref[def:first_order_embedding]{first-order embedding} if and only if it is both \hyperref[def:order_function/preserving]{order-preserving} and \hyperref[def:order_function/preserving]{order-reflecting}.
  \end{thmenum}
\end{proposition}
\begin{defproof}
  \SubProofOf{thm:def:order_function/injective_implies_strict} Let \( f: P \to Q \) be an injective order-preserving map.

  Let \( x <_P y \) for some members \( x \) and \( y \) of \( P \). Since \( f \) is order-preserving, we have \( f(x) \leq_Q f(y) \). Since it is also injective, \( f(x) = f(y) \) implies \( x = y \), hence it remains for \( f(x) <_Q f(y) \) to hold.

  Generalizing on \( x \) and \( y \), we conclude that \( f \) is strict.

  \SubProofOf{thm:def:preordered_set/homomorphism_is_reflecting} Let \( f: P \to Q \) be an injective function between preordered sets.

  \SufficiencySubProof* Suppose that \( f \) is an embedding, that is, an injective order-preserving map whose inverse is order-preserving.

  If \( f(x) \leq_Q f(y) \), then
  \begin{equation*}
    x = f^{-1}(f(x)) \leq_P f^{-1}(f(y)) = y.
  \end{equation*}

  We conclude that \( f \) is order-reflecting.

  \NecessitySubProof* Suppose that \( f \) is both order-preserving and order-reflecting.

  If \( f(x) \leq_Q f(y) \), then
  \begin{equation*}
    f^{-1}(f(x)) = x \leq_P y = f^{-1}(f(y)).
  \end{equation*}

  We conclude that \( f^{-1} \) is order-preserving.
\end{defproof}

  \subsection{Partially ordered sets}\label{subsec:partially_ordered_sets}

\paragraph{Partial orders}

\hyperref[def:preordered_set]{Preordered sets} are simple to define and arise naturally, for example \fullref{def:lindenbaum_tarski_algebra} or \fullref{thm:semiring_divisibility_order}. Unfortunately, they require certain uniqueness considerations, as discussed in \fullref{ex:preorder_nonuniqueness}. If we want to overcome them, we arrive at the concept of a partially ordered set.

\begin{definition}\label{def:partially_ordered_set}\mcite[1]{Birkhoff1967}
  We define a \term[ru=частично упорядоченное множество (\cite[72]{Гуров2013})]{partially ordered set} to be a \hyperref[def:preordered_set]{preordered set} that also satisfies the antisymmetry condition \eqref{eq:def:binary_relation/antisymmetric}.
\end{definition}
\begin{comments}
  \item The metamathematical properties are inherited from \fullref{def:preordered_set} with this additional axiom added. We denote the corresponding category via \( \cat{Pos} \).

  \item \incite[72]{Гуров2013}, \incite[10]{Проданов1982}, \incite[4]{Engelking1989}, \incite[2]{DaveyPriestley2002}, \incite[13]{Harzheim2005}, \incite[1]{Gratzer2011} and \incite[1]{Birkhoff1967} define partial orders like us. \incite[62]{Enderton1977Sets} defines partial orders as what we call \enquote{\hyperref[def:strict_partial_order]{strict partial orders}}, while \incite[13]{Kelley1975} defines partial orders as simply \hyperref[def:binary_relation/transitive]{transitive} relations.

  \item The category \( \cat{Pos} \) is isomorphic to that of small skeletal preorder categories --- see \fullref{thm:order_category_isomorphism}.
\end{comments}

\begin{definition}\label{def:strict_partial_order}\mcite[13]{Harzheim2005}
  We call an \hyperref[def:binary_relation/irreflexive]{irreflexive} and \hyperref[def:binary_relation/transitive]{transitive} \hyperref[def:binary_relation]{binary relation} a \term{strict partial order}.
\end{definition}

\begin{proposition}\label{thm:strict_partial_order}
  For an arbitrary set \( P \) and binary relation \( \leq \) on \( P \), the pair \( (P, \leq) \) is a \hyperref[def:partially_ordered_set]{partially ordered set} if and only the relation \( < \), defined via \eqref{eq:def:preordered_set/compatibility_strict}, is a \hyperref[def:strict_partial_order]{strict partial order}.
\end{proposition}
\begin{proof}
  \SufficiencySubProof Let \( \leq \) be a partial order. We will show that \( < \) is a strict partial order.

  \SubProofOf*[def:binary_relation/transitive]{transitivity} The relation \( < \) is \hyperref[def:binary_relation/transitive]{transitive}. To see this, let \( x < y \) and \( y < z \). In particular, \( x \leq y \) and \( y \leq z \). From transitivity, we have \( x \leq z \).

  Additionally, \( x \neq y \) and \( y \neq z \). Assume that \( x = z \). From reflexivity of \( \leq \) we have \( z \leq x \) and, since \( y \leq z \), from transitivity we obtain \( y \leq x \). But since \( x \leq y \), from the antisymmetry of \( \leq \), we have \( x = y \), which contradicts the assumption that \( x < y \).

  Therefore, \( x < z \).

  \SubProofOf*[def:binary_relation/irreflexive]{irreflexivity}. Follows directly from reflexivity of \( \leq \) and the compatibility condition.

  Since the right side is false, the left side \( x < x \) is also false.

  \NecessitySubProof Let \( < \) be a strict partial order. We will show that \( \leq \) is a partial order.

  \SubProofOf*[def:binary_relation/reflexive]{reflexivity} Fix \( x \in P \) and assume that \( x \not\leq x \). Then \( x \neq x \) which contradicts the reflexivity of equality. Hence, \( x \leq x \).

  \SubProofOf*[def:binary_relation/antisymmetric]{antisymmetry} Let \( x \leq y \) and \( y \leq x \), that is, either \( x = y \) or both \( x < y \) and \( y < x \) hold. Assume the latter. By the transitivity of \( \leq \), we have \( x < x \), which contradicts the irreflexivity of \( < \). Hence, \( x = y \).

  \SubProofOf*[def:binary_relation/transitive]{transitivity} Let \( x \leq y \) and \( y \leq z \). Then we have four cases depending on which of \( x \), \( y \) and \( z \) are equal. Since both relations \( < \) and \( = \) are transitive, it follows that in all four cases \( x \leq z \).
\end{proof}

\begin{proposition}\label{thm:def:partially_ordered_set}
  \hyperref[def:partially_ordered_set]{Partially ordered sets} have the following basic properties:
  \begin{thmenum}
    \thmitem{def:partially_ordered_set/comparables_reflect_inequalities} For any \hyperref[def:order_function/preserving]{order-preserving map} \( f: P \to Q \) between partially ordered sets, if \( x \) and \( y \) are comparable elements, \( f(x) < f(y) \) implies \( x < y \).
  \end{thmenum}
\end{proposition}
\begin{proof}
  \SubProofOf{def:partially_ordered_set/comparables_reflect_inequalities} Let \( f(x) <_Q f(y) \) and suppose that \( x \geq y \). Since \( f \) is order-preserving, we have \( f(x) \geq_Q f(y) \), which is a contradiction.
\end{proof}

\begin{definition}\label{def:hasse_diagram}\mcite[11]{DaveyPriestley2002}
  It can be easier to define small finite partially ordered sets by drawing graphs than by enumerating all relation pairs. Let \( (P, \leq) \) be a finite partially ordered set. The relation \( \leq \) may also be regarded as the set of edges of a \hyperref[def:directed_graph]{directed graph}. We will instead consider the \hyperref[def:transitive_reduction]{transitive reduction} \( \red^T(\leq) \).

  We call the graph \( (P, \red^T(\leq)) \) the \term[ru=диаграмма Хассе (\cite[78]{Гуров2013})]{Hasse diagram} of \( (P, \leq) \).
\end{definition}
\begin{comments}
  \item When drawing Hasse diagrams, we draw no arrowheads; instead, arcs point upwards. See \fullref{ex:def:hasse_diagram}.
\end{comments}

\begin{example}\label{ex:def:hasse_diagram}
  Consider the partial order over \( \set{ a, b, c, d, e } \) defined via
  \begin{equation}\label{eq:ex:def:hasse_diagram/partially ordered set}
    \hi{a \leq c},\quad \hi{a \leq d},\quad a \leq e,\quad \hi{b \leq d},\quad b \leq e,\quad \hi{d \leq e}.
  \end{equation}

  The corresponding Hasse diagram includes only the highlighted edges. The rest of the edges can be recovered via transitivity. In this case, the Hasse diagram has edges
  \begin{equation}\label{eq:ex:def:hasse_diagram/hasse_graph}
    \set{ a \to c, a \to d, b \to d, d \to e }
  \end{equation}

  \begin{figure}[!ht]
    \centering
    \includegraphics[page=1]{output/ex__def__hasse_diagram}
    \caption{A drawing of the Hasse diagram \eqref{eq:ex:def:hasse_diagram/hasse_graph}}
    \label{fig:ex:def:hasse_diagram}
  \end{figure}
\end{example}

\begin{example}\label{ex:preorder_nonuniqueness}
  Consider the preordered set \( P \) in \cref{fig:ex:preorder_nonuniqueness} in which \( b \leq c \) and \( c \leq b \), but \( b \neq c \). We cannot properly draw a \hyperref[def:hasse_diagram]{Hasse diagram} because we have the restriction that \( c \) is drawn (strictly) higher than \( b \) if \( c > b \) and that \( c \) is drawn lower than \( b \) if \( c < b \). We face a similar problem formally, for example in the definition of \hyperref[def:lindenbaum_tarski_algebra]{Lindenbaum-Tarski algebras}, where the preorder \( \vdash \) allows \( \varphi \vdash \psi \) and \( \psi \vdash \varphi \), but still \( \varphi \neq \psi \). Thus, we have nonuniqueness --- every tautology is a largest element with respect to \( \vdash \), while it is natural to prefer a unique largest element.

  \begin{figure}[!ht]
    \hfill
    \includegraphics[page=1]{output/ex__preorder_nonuniqueness}
    \hfill
    \includegraphics[page=2]{output/ex__preorder_nonuniqueness}
    \hfill\hfill
    \caption{A preordered set and its induced partially ordered set.}
    \label{fig:ex:preorder_nonuniqueness}
  \end{figure}

  This leads to \fullref{def:antisymmetric_quotient}, where we factor \( P \) by an equivalence relation and obtain a partially ordered set. In the language of graph theory, if we have \hyperref[def:graph_cycle]{cycles}, we can contract each directed cycle into a single vertex, at which point the graph becomes acyclic. This corresponds to \hyperref[def:directed_graph_condensation]{graph condensation}.
\end{example}

\begin{definition}\label{def:antisymmetric_quotient}\mimprovised
  Let \( (P, \leq) \) be a preordered set. Define the relation \( x \sim y \) to hold if \( x \leq y \) and \( y \leq x \).

  On the \hyperref[def:equivalence_relation/quotient]{quotient set} \( Q \coloneqq P / {\sim} \), define the relation \( [x] \preceq [y] \) to hold if \( x \leq y \).

  The pair \( (Q, \preceq) \) is then a \hyperref[def:partially_ordered_set]{partially ordered set}, which we will call the \term{antisymmetric quotient} of \( (P, \leq) \).
\end{definition}
\begin{comments}
  \item The relation \( \sim \) is the intersection of the relation \( \leq \) with its \hyperref[def:binary_relation/inverse]{inverse}.
  \item We may prefer, as in the case of \fullref{thm:lindenbaum_tarski_theories}, to refer to the projection operator \( \pi: P \to Q \) rather than to quotient sets.
\end{comments}
\begin{defproof}
  First, we must show that the relation \( \preceq \) is well-defined. Indeed, let \( x \sim x' \) and \( y \sim y' \). If \( x \leq y \), we have \( x' \leq x \leq y \leq y' \), hence \( x' \leq y' \) because of transitivity.

  It is then clear that \( \preceq \) is a partial order because it inherits reflexivity and transitivity from \( \leq \) and antisymmetry is imposed by taking quotient sets --- equality in \( Q \) holds precisely when \( {\sim} \) holds in \( P \).

  Thus, \( (Q, \preceq) \) is indeed a partially ordered set.
\end{defproof}

\paragraph{Intervals}

\begin{definition}\label{def:order_interval}\mcite[31]{Harzheim2005}
  Fix a \hyperref[def:partially_ordered_set]{partially ordered set} \( (P, \leq) \). For any two elements \( a \leq b \) we define the following related (ordered) subsets:

  \begin{thmenum}
    \thmitem{def:order_interval/unbounded} First, we define the \term{open initial segment} and \term{open final segment} and as
    \begin{equation*}
      \begin{aligned}
        P_{>a} \coloneqq \set{ x \in P \given b > a },
        \\
        P_{<b} \coloneqq \set{ x \in P \given x < b }.
      \end{aligned}
    \end{equation*}

    Similarly, we define the corresponding \term{closed segments} as
    \begin{equation*}
      \begin{aligned}
        P_{\geq a} \coloneqq \set{ x \in P \given x \geq a },
        \\
        P_{\leq b} \coloneqq \set{ x \in P \given x \leq b }.
      \end{aligned}
    \end{equation*}

    \thmitem{def:order_interval/closed} Next, we define the \term[bg=затворен интервал (\cite[39]{Тагамлицки1971Диф}), ru=отрезок (\cite[82]{ЭнциклопедияЕлементарнойМатематикиТом1}); отрезок/сегмент (\cite[def. 6]{Александров1977Введение})]{closed interval} with endpoints \( a \) and \( b \) as
    \begin{equation*}
      [a, b] \coloneqq \set{ x \in P \given a \leq x \leq b } = P_{\geq a} \cap P_{\leq b}.
    \end{equation*}

    \thmitem{def:order_interval/open} The \term[bg=отворен интервал (\cite[39]{Тагамлицки1971Диф}), ru=интервал (\cite[82]{ЭнциклопедияЕлементарнойМатематикиТом1}); интервал/промежуток (\cite[def. 6]{Александров1977Введение})]{open interval} with endpoints \( a \) and \( b \) is
    \begin{equation*}
      (a, b) \coloneqq \set{ x \in P \given a < x < b } = P_{> a} \cap P_{< b}.
    \end{equation*}

    \thmitem{def:order_interval/half_open} The \term[bg=полузатворен интервал (\cite[39]{Тагамлицки1971Диф}), ru=полуинтервал (\cite[82]{ЭнциклопедияЕлементарнойМатематикиТом1})]{half-open intervals} are
    \begin{equation*}
      \begin{aligned}
        (a, b] \coloneqq \set{ x \in P \given a < x \leq b } = P_{> a} \cap P_{\leq b},
        \\
        [a, b) \coloneqq \set{ x \in P \given a \leq x < b } = P_{\geq a} \cap P_{< b}.
      \end{aligned}
    \end{equation*}
  \end{thmenum}
\end{definition}
\begin{comments}
  \item We implicitly assume that \( a \leq b \), but this is not strictly necessary --- \( [a, b] \) is an empty set otherwise.

  \item In the context of \hyperref[def:extended_real_numbers]{extended real numbers}, we use \( -\infty \) and \( \infty \) as interval endpoints. If the interval is open at \( -\infty \) and \( \infty \), as in \( (-\infty, b] = \BbbR_{\leq b} \) or \( [a, \infty) = \BbbR_{\geq a} \), we obtain sets containing only real numbers.

  \item Another notation for open intervals is \enquote{\( ]a, b[ \)}. It is used by \incite[51]{Зорич2019Том1}. We prefer \enquote{\( (a, b) \)}, which is in turn used by most of our sources --- \incite[9]{Lorentz1966}, \incite[31]{Rudin1976Principles}, \incite[10]{Engelking1989}, \incite[269]{Jacobson1985Vol1}, \incite[157]{Gratzer2011}, \incite[31]{Harzheim2005}, \incite[15]{BrannanEsplenGray2011}, \incite[67]{Rosen1999}, \incite[82]{ЭнциклопедияЕлементарнойМатематикиТом1}, \incite[25]{МагарилИльяевТихомиров2002ВыпуклыйАнализ}, \incite[18]{ИльинСадовничийСендов1985Том1}, \incite[\S 22.2]{Тыртышников2007}, \incite[12]{Александров1977Введение}, \incite[112]{Винберг2014}, \incite[10]{БелоусовТкачёв2004} \incite[23]{Гуров2013}, \incite[7]{Боянов2008}, \incite[225]{ГеновМиховскиМоллов1991} and \incite[39]{Тагамлицки1971Диф}.

  \incite[36]{Knuth1997Vol1} uses \enquote{\( (a..b) \)}.
\end{comments}

\paragraph{Chains}

\begin{definition}\label{def:partial_order_chain}
  Fix a partially ordered set \( (P, \leq) \) and a subset \( C \) of \( P \).

  \begin{thmenum}
    \thmitem{def:partial_order_chain/chain}\mcite[4]{Gratzer2011} We say that \( C \) is a \term[bg=верига (\cite[10]{Проданов1982}), ru=цепь (\cite[def. 3.4]{Гуров2013})]{chain} if every two elements of \( C \) are comparable.

    \thmitem{def:partial_order_chain/length}\mimprovised We define the \term[ru=длина (\cite[83]{Гуров2013}), en=length (\cite[4]{Gratzer2011})]{length} \( \len(C) \) of a \hi{nonempty} chain \( C \) as the unique \hyperref[def:cardinal]{cardinal} satisfying \( {\len(C) + 1 = \card(A)} \), that is, \( \len(C) = \card(C) - 1 \) for finite chains and \( \len(C) = \card(C) \) otherwise.

    \thmitem{def:partial_order_chain/height}\mimprovised We define the \term[ru=высота (\cite[83]{Гуров2013}), en=height (\cite[24]{Harzheim2005})]{height} of \( P \) as the supremum among the lengths of all nonempty chains in \( P \).

    If \( P \) has no nonempty chains, the supremum is the bottom element of the natural numbers, zero.

    \thmitem{def:partial_order_chain/antichain}\mcite[4]{Gratzer2011} We say that \( C \) is an \term[ru=антицепь (\cite[78]{Гуров2013})]{antichain} if no two elements of \( C \) are comparable.

    \thmitem{def:partial_order_chain/width}\mcite[57]{Harzheim2005} We define the \term[ru=ширина (\cite[83]{Гуров2013})]{width} of \( P \) as the supremum among the cardinalities of all antichains.
  \end{thmenum}
\end{definition}
\begin{comments}
  \item The definitions for length, width and height are defined for finite sets of elements in \cite[4]{Gratzer2011}. We extend the definition to infinite partially ordered sets.
  \item Unlike for the height, for the width we do not subtract \( 1 \) from the cardinality of antichains.
  \item \Fullref{thm:union_of_set_of_cardinals} implies that the height and width are well-defined cardinals.
  \item The concepts of width and height become apparent in the context of Hasse diagrams --- see \fullref{ex:def:partial_order_chain}.
  \item \incite[5]{Birkhoff1967} and \incite[4]{Gratzer2011} use \enquote{length} for what we call \enquote{height}, while \incite[24]{Harzheim2005} and \incite[83]{Гуров2013} prefer our terminology.
  \item \incite[19]{Harzheim2005} defines the length of a chain as the cardinality itself, while \incite[5]{Birkhoff1967}, \incite[4]{Gratzer2011}, \incite[def. 2.37]{DaveyPriestley2002} and \incite[83]{Гуров2013} subtract one. We prefer the latter convention.
\end{comments}

\begin{example}\label{ex:def:partial_order_chain}
  We list some examples of \hyperref[def:partial_order_chain]{chains and antichains}:
  \begin{thmenum}
    \thmitem{ex:def:partial_order_chain/natural_numbers} The set \( \BbbN \) of \hyperref[def:natural_numbers]{natural numbers}, as well as any subset of \( \BbbN \), is a chain.

    The length of the initial segment \( \BbbN_{\leq n} = \set{ 0, 1, \ldots, n } \) is \( n \) because it has \( n + 1 \) elements. The length of \( \BbbN \) itself is \( \aleph_0 \).

    Hence, the height of \( \BbbN \) is \( \aleph_0 \) and the height of any individual number \( n \) is \( n \) itself.

    The width of \( \BbbN \) (or, more generally, any \hyperref[def:totally_ordered_set]{totally ordered set}) is \( 1 \).

    \thmitem{ex:def:partial_order_chain/binary_power_set} Consider the \hyperref[def:basic_set_operations/power_set]{power set} of the two-element set \( \set{ a, b } \) depicted in \cref{fig:ex:def:partial_order_chain/binary_power_set}.

    \begin{figure}[!ht]
      \hfill
      \includegraphics[page=1]{output/ex__def__partial_order_chain__binary_power_set}
      \hfill\hfill
      \caption{A Hasse diagram of the power set of a two-element set.}
      \label{fig:ex:def:partial_order_chain/binary_power_set}
    \end{figure}

    Only the sets \( \set{ a } \) and \( \set{ b } \) are incomparable, thus the width of the power set is \( 2 \).

    On the other hand, the maximal chain \( \varnothing \subseteq \set{ a } \subseteq \set{ a, b } \) has three elements, hence is of length two. Thus, the height of the power set is \( 2 \).

    \thmitem{ex:def:partial_order_chain/ternary_power_set} Consider the \hyperref[def:basic_set_operations/power_set]{power set} of \( \set{ a, b, c } \) depicted in \cref{fig:ex:def:partial_order_chain/ternary_power_set}.

    \begin{figure}[!ht]
      \hfill
      \includegraphics[page=1]{output/ex__def__partial_order_chain__ternary_power_set}
      \hfill\hfill
      \caption{A Hasse diagram of the power set of a three-element set.}
      \label{fig:ex:def:partial_order_chain/ternary_power_set}
    \end{figure}

    Like in \fullref{ex:def:partial_order_chain/binary_power_set}, we can find several maximal antichains:
    \begin{itemize}
      \item The singleton sets \( \set{ a } \), \( \set{ b } \) and \( \set{ c } \).
      \item The binary sets \( \set{ a, b } \), \( \set{ a, c } \) and \( \set{ b, c } \).
    \end{itemize}

    Both have cardinality three, hence the width of the power set is \( 3 \).

    Maximal chains have four elements, so the height of the power set is \( 3 \).

    \thmitem{ex:def:partial_order_chain/quad_power_set} The pattern from \fullref{ex:def:partial_order_chain/binary_power_set} and \fullref{ex:def:partial_order_chain/ternary_power_set} breaks when we consider four elements.

    Indeed, for the power set of \( \set{ a, b, c, d } \), the following is an antichain of six elements:
    \begin{equation}\label{eq:ex:def:partial_order_chain/quad_power_set/antichain}
      \set{ a, b }, \set{ a, c }, \set{ a, d }, \set{ b, c }, \set{ b, d }, \set{ c, d }.
    \end{equation}

    This generalizes to \fullref{thm:sperners_theorem}.
  \end{thmenum}
\end{example}

\begin{proposition}\label{thm:def:partial_order_chain}
  \hyperref[thm:def:partial_order_chain]{Chains and antichains} have the following basic properties:
  \begin{thmenum}
    \thmitem{thm:def:partial_order_chain/image_of_chain} Let \( f: P \to Q \) be an order-preserving map between partially ordered sets. If \( A \) is a chain in \( P \), then its image \( f[A] \) is a chain in \( Q \).

    \thmitem{thm:def:partial_order_chain/preimage_of_antichain} Again, let \( f: P \to Q \) be an order-preserving map between partially ordered sets. If the image \( f[A] \) of a subset \( A \) of \( P \) is an antichain, then \( A \) itself is an antichain.

    \thmitem{thm:def:partial_order_chain/unbounded} A chain is \hyperref[def:extremal_points/bounds]{unbounded from above} (resp. below) if, for every element, it contains a strictly larger (resp. smaller) element.

  \end{thmenum}
\end{proposition}
\begin{proof}
  \SubProofOf{thm:def:partial_order_chain/image_of_chain} Fix two elements \( x \) and \( y \) in \( f[A] \). Let \( a \) and \( b \) be elements of \( P \) such that \( f(a) = x \) and \( f(b) = y \). Then \( a \leq b \) implies \( f(a) \leq f(b) \) and \( a > b \) implies \( f(a) \geq f(b) \). Hence, \( x \) and \( y \) are comparable, and since they were arbitrary, we conclude that \( f[A] \) is a chain.

  \SubProofOf{thm:def:partial_order_chain/preimage_of_antichain} Follows from \fullref{def:partially_ordered_set/comparables_reflect_inequalities}.

  \SubProofOf{thm:def:partial_order_chain/unbounded} Fix a chain \( C \) in a partially ordered set \( (P, \leq) \).

  Expanding the definition \fullref{def:extremal_points/bounds} of upper bound, we conclude that
  \begin{displayquote}
    \( C \) is unbounded if and only if there does not exist an element \( x \in C \) such that, for every \( y \in C \), we have \( y \leq x \).
  \end{displayquote}

  The consequent of the above statement has a negation in front of its outermost quantifier. \Fullref{thm:first_order_quantifiers_are_dual} justifies moving the negation inwards to obtain
  \begin{displayquote}
    \( C \) is unbounded if and only if, for every \( x \in C \), there exists some \( y \in C \) such that \( y \leq x \) does not hold.
  \end{displayquote}

  Taking into account that every pair of elements in \( C \) is comparable, \( y \leq x \) does not hold if and only if \( y > x \) holds. Then
  \begin{displayquote}
    \( C \) is unbounded if and only if, for every \( x \in C \), there exist \( y \in C \) such that \( y > x \).
  \end{displayquote}

  The case of lower bounds is analogous.
\end{proof}

\begin{theorem}[Sperner's theorem]\label{thm:sperners_theorem}\mcite[287]{Harzheim2005}
  Let \( S \) be a set with \( n \) elements and let \( \mscrU \) be an \hyperref[def:partial_order_chain/antichain]{antichain} in the power set of \( S \).

  Then \( \mscrU \) has at most \( \binom n {\quot(n, 2)} \) elements.
\end{theorem}

\begin{definition}\label{def:partial_order_element_height}\mimprovised
  Fix a \hyperref[def:partially_ordered_set]{partially ordered set} \( (P, \leq) \) with \hyperref[def:extremal_points/top_and_bottom]{bottom element} \( \bot \).

  We define the \term[ru=высота (\cite[83]{Гуров2013}), en=height (\cite[5]{Birkhoff1967})]{height} of an element \( x \) of \( P \) as the \hyperref[def:partial_order_chain/height]{height} of the \hyperref[def:order_interval/unbounded]{initial segment} \( P_{\leq x} \).

  If \( x \) has height \( 1 \), we call it an \term[ru=атом (\cite[89]{Гуров2013}), en=atom (\cite[5]{Birkhoff1967})]{atom}.
\end{definition}
\begin{comments}
  \item The condition for \( P \) to be bounded from below, which ensures well-definedness, is due to \incite[5]{Birkhoff1967}, while using the height of the initial segment is due to \incite[4]{Gratzer2011}.

  \item The definition of atom is also due to \incite[5]{Birkhoff1967}.
\end{comments}

\begin{proposition}\label{thm:atoms_are_incomparable}
  Distinct \hyperref[def:partial_order_element_height]{atoms} are incomparable.
\end{proposition}
\begin{proof}
  If \( x \leq y \) are atoms, then in order for \( \bot \leq x \leq y \) to be a chain of length \( 1 \), \( x \) and \( y \) must be equal.
\end{proof}

\paragraph{Lexicographic order}

\begin{definition}\label{def:lexicographic_order}\mimprovised
  Let \( (P, \leq_P) \) and \( (Q, \leq_Q) \) be partially ordered sets.

  We define the \term[en=lexicographic order (\cite[18]{DaveyPriestley2002}), ru=лексикографический (порядок) (\cite[99]{Гуров2013})]{lexicographic order} on \( P \times Q \) as
  \begin{equation}\label{eq:def:lexicographic_order}
    (a, b) \prec (c, d) \thickspace \T{if and only if} \thickspace \parens[\Big]{ a <_P c \T{or} \parens[\Big]{ a = c \T{and} b <_Q d } }
  \end{equation}
  and the \term{reverse lexicographic order} as
  \begin{equation}\label{eq:def:lexicographic_order/reverse}
    (a, b) \prec (c, d) \thickspace \T{if and only if} \thickspace \parens[\Big]{ b <_Q d \T{or} \parens[\Big]{ b = d \T{and} a <_P c } }.
  \end{equation}
\end{definition}
\begin{comments}
  \item We can use natural number recursion to extend this to arbitrary \( n \)-tuples.
\end{comments}

\begin{proposition}\label{thm:lexicographic_order_is_partial_order}
  The \hyperref[eq:def:lexicographic_order]{lexicographic} and \hyperref[eq:def:lexicographic_order/reverse]{reverse lexicographic} orders are \hyperref[def:partially_ordered_set]{strict partial order} relations.
\end{proposition}
\begin{comments}
  \item An analogous result holds for total orders (\fullref{thm:def:totally_ordered_set/lexicographic}) and well-ordered sets (\fullref{thm:def:well_ordered_set/lexicographic}).
\end{comments}
\begin{proof}
  \SubProofOf[def:binary_relation/irreflexive]{irreflexivity} Trivial.

  \SubProofOf[def:binary_relation/transitive]{transitivity} Let \( \prec \) be a lexicographic order on \( P \times Q \).

  If \( (a, b) \prec (c, d) \) and \( (c, d) \prec (e, f) \), then
  \begin{itemize}
    \item If \( a < c \), then \( a < c \leq e \) and thus \( (a, b) \prec (e, f) \).

    \item If \( a = c \) and \( b < d \), then \( a \leq e \) and \( b < d \leq f \) and thus \( (a, b) \prec (e, f) \).
  \end{itemize}

   The proof for the reverse lexicographic order is analogous.
\end{proof}

\begin{example}\label{ex:def:lexicographic_order}
  We list several examples of \hyperref[def:lexicographic_order]{lexicographic ordering}:
  \begin{thmenum}
    \thmitem{ex:def:lexicographic_order/words} In a real-world dictionary like a thesaurus, the word \enquote{bright} comes after \enquote{blight}.

    Denote the English alphabet by \( A \). Then the lexicographic ordering on the Cartesian power \( A^6 \) suggests that since the first letters of \enquote{bright} and \enquote{blight} are equal, we should compare the second letters, which would lead us to the usual thesaurus order.

    \thmitem{ex:def:lexicographic_order/heterogenous_words} Real-world dictionaries list words with different numbers of letters. It is not clear how to compare \enquote{car} to \enquote{carboy}.

    We can add a sentinel symbol \( \anon* \) to the English alphabet \( A \) to denote empty spaces. Then \enquote{car} in \( A^6 \) becomes \enquote{car\( \anon* \anon* \anon* \)}. Other situations like \fullref{ex:def:lexicographic_order/natural_numbers} may require prepending \( \anon* \) rather than appending it.

    Now, if we specify how \( \anon* \) should compare to the letters, we can use lexicographic ordering to compare the two words. For example, making \( \anon* \) the bottom of the alphabet will make \enquote{car} precede \enquote{carboy}, while making it the top would do the converse.

    Note that this is only a problem if one word is a prefix of another. We can easily conclude that \enquote{apple} is smaller than \enquote{apricot} because the third letter \( p \) is smaller than \( r \).

    \thmitem{ex:def:lexicographic_order/natural_numbers} Consider the language of natural numbers in decimal notation from \fullref{def:positional_number_system/decimal}. Ordering single-digit strings is trivial, but ordering multi-digit numeric strings requires using lexicographic ordering.

    Furthermore, even though it is clear that \( \syn1 \syn0 < \syn1 \syn1 \), handling numeric strings of differing lengths requires prepending a sentinel symbol \( \anon* \) that is less than any numeral. Then we have \( \anon* \syn9 < \syn1 \syn0 \), which corresponds to the natural number ordering discussed in \fullref{def:natural_numbers_ordering}.

    \thmitem{ex:def:lexicographic_order/rectangle} A slightly more relevant example for mathematics is the lexicographic ordering
    \begin{equation*}
      AB < AD < BC < CD
    \end{equation*}
    of the names of the edges of a \hyperref[def:parallelogram/rectangle]{rectangle}. The reverse lexicographic ordering is
    \begin{equation*}
      AB < BC < AD < CD.
    \end{equation*}

    \thmitem{ex:def:lexicographic_order/triangle} For the sides of a \hyperref[def:triangle]{triangle}, we have \( AB < AC < BC \) for both orderings.

    \thmitem{ex:def:lexicographic_order/antichain} The antichain \eqref{eq:ex:def:partial_order_chain/quad_power_set/antichain} is ordered with respect to the lexicographic ordering on the Cartesian square of \( \set{ a, b, c, d } \).

    \thmitem{ex:def:lexicographic_order/graph} The edges of the graph in \eqref{eq:fig:def:directed_multigraph} are numbered in lexicographic order, which also happens to be the reverse lexicographic order.

    \thmitem{ex:def:lexicographic_order/ordinals} \Fullref{thm:ordinal_addition_disjoin_union} and \fullref{thm:ordinal_multiplication_cartesian_product} contain more interesting applications of lexicographic orders.
  \end{thmenum}
\end{example}

\paragraph{Chain conditions}

\begin{remark}\label{rem:ascending_chains}
  As discussed in \fullref{rem:order_homomorphism_terminology}, some authors refer to order-preserving maps as \enquote{ascending}. When regarding a \hyperref[def:sequence]{sequence} as a map between ordered sets, an \enquote{ascending sequence} is then simply one where every next element is larger than or equal to the previous one.

  Such sequences are used in the context of the \enquote{ascending chain condition}, which states that no strictly ascending sequences exist. This condition is named so by many authors, usually in the context of lattices of ring ideals. The term can be found in English --- see \incite[244]{Aluffi2009}, \incite[417]{Knapp2016BasicAlgebra}, \incite[102]{Jacobson1985Vol1}, \incite[112]{Lang2002}, \incite[304]{Rotman2010}, \incite[69]{Golan2010} --- as well as in Russian (\enquote{возрастающая цепь}) --- see \incite[61]{Шафаревич1999} and \incite[def. 9.4.1]{Винберг2014} --- and Bulgarian (\enquote{растяща верига}) --- see \incite[248]{ГеновМиховскиМоллов1991}.

  In the more general context of abstract ordered sets, the term is used by \incite[51]{DaveyPriestley2002}, \incite[181]{Birkhoff1967} and \incite[24]{Gratzer2011}.

  None of the aforementioned authors, however, gives a definition for an \enquote{ascending chain}, leaving open the possibility that an ascending chain is not merely an ascending sequence.

  Suppose that we define an \enquote{ascending chain} as a chain, in the sense of \fullref{def:partial_order_chain/chain}, that is not \hyperref[def:extremal_points/bounds]{bounded from above}. As we shall see in our proof equivalence in \fullref{def:chain_condition}, no ascending chains (in this sense) exist if and only if no strictly ascending sequences exist.

  Thus, if we decide to provide a definition for the term \enquote{ascending chain}, we would have to choose between at least two options --- ascending sequences, which is truer to their usage, and unbounded from above chains, which is truer to the concept of chains defined in \fullref{def:partial_order_chain}.

  We avoid defining the term \enquote{ascending chain} entirely and give several equivalent definitions for the \enquote{ascending chain conditions} in \fullref{def:chain_condition}.
\end{remark}

\begin{definition}\label{def:stabilizing_sequence}\mcite[244]{Aluffi2009}
  Consider the \hyperref[def:order_function/ascending]{ascending sequence}
  \begin{equation*}
    x_1 \leq x_2 \leq x_3 \leq \cdots
  \end{equation*}

  If there exists an index \( n \) such that \( x_k = x_n \) whenever \( k > n \), we say that the chain \term[bg=стабилизира (\cite[41]{КоцевСидеров2016}), ru=стабилизируется (\cite[52]{Яблонский2003})]{stabilizes} at \( n \).
\end{definition}
\begin{comments}
  \item Of course, descending sequences can also stabilize.
\end{comments}

\begin{definition}\label{def:chain_condition}
  We say that a \hyperref[def:partially_ordered_set]{partially ordered set} satisfies the \term{ascending chain condition} (resp. \term{descending chain condition}) if any of the following equivalent conditions hold:
  \begin{thmenum}
    \thmitem{def:chain_condition/maximal}\mcite[180]{Birkhoff1967} Every nonempty subset has a \hyperref[def:extremal_points/maximal_and_minimal_element]{maximal} (resp. minimal) element.

    \thmitem{def:chain_condition/stabilization}\mcite[51]{DaveyPriestley2002} Every nonstrict ascending (resp. descending) sequence \hyperref[def:stabilizing_sequence]{stabilizes}.

    \thmitem{def:chain_condition/infinite} There exists no strictly ascending (resp. descending) sequence.

    \thmitem{def:chain_condition/unbounded} There exists no \hyperref[def:extremal_points/bounds]{unbounded from above} (resp. from below) \hyperref[def:partial_order_chain]{chain}.
  \end{thmenum}
\end{definition}
\begin{defproof}
  We will restrict ourselves to ascending sequences, maximal elements and upper bounds.

  \ImplicationSubProof{def:chain_condition/maximal}{def:chain_condition/stabilization} Suppose that every subset has a maximal element.

  Suppose that there is a nonstrict ascending sequence
  \begin{equation*}
    x_1 \leq x_2 \leq x_3 \leq \cdots.
  \end{equation*}

  Then the set \( \set{ x_k \given k \geq 1 } \) has a maximal element, say \( x_{k_0} \). Then, since \( x_{k_0} \) is maximal, using \hyperref[con:induction/peano_arithmetic]{natural number induction} we can prove that \( x_{k_0} = x_{k_0 + i} \) for \( i \geq 0 \).

  Thus, the sequence stabilizes.

  \ImplicationSubProof{def:chain_condition/stabilization}{def:chain_condition/infinite} Suppose that every ascending sequence in \( P \) stabilizes.

  Every strictly ascending sequence is also nonstrictly ascending, hence it stabilizes, which contradicts our assumption that it is strictly ascending. The obtained contradiction shows that there are not strictly ascending sequences in \( P \).

  \ImplicationSubProof{def:chain_condition/infinite}{def:chain_condition/unbounded} Suppose that there exists no infinite strictly ascending sequence.

  Suppose also that there exists an unbounded from above chain \( C \). Let \( x_1 \) be an element of \( C \). Via \hyperref[rem:natural_number_recursion]{natural number recursion}, we can define a sequence by letting \( x_{n+1} \) be an element of \( C \) strictly larger than \( x_n \) --- \fullref{thm:def:partial_order_chain/unbounded} guarantees the existence of such an element. Note that the axiom of choice is needed here.

  We have thus constructed a strictly ascending sequence
  \begin{equation*}
    x_1 < x_2 < x_3 < \cdots
  \end{equation*}

  The existence of such a sequence contradicts our assumption, hence also the existence of \( C \).

  \ImplicationSubProof{def:chain_condition/unbounded}{def:chain_condition/maximal} Suppose that there exists no unbounded from above chain.

  Let \( A \) be some nonempty subset. Suppose that \( A \) has no maximal element. Then we can construct a strictly ascending sequence
  \begin{equation*}
    x_1 < x_2 < x_3 < \cdots
  \end{equation*}

  The set
  \begin{equation*}
    \set{ x_1, x_2, x_3, \cdots }
  \end{equation*}
  is an unbounded from above chain.

  The obtained contradiction shows that \( A \) has a maximal element.
\end{defproof}

\paragraph{Cofinal sets}

\begin{definition}\label{def:cofinal_set}\mcite[71]{Harzheim2005}
  We say that a subset \( A \) of a \hyperref[def:preordered_set]{preordered set} \( (P, \leq) \) is \term{cofinal} if, for every \( x \in P \), there exists some \( y \in A \) such that \( x \leq y \).
\end{definition}

\begin{example}\label{ex:def:cofinal_set}
  We list several examples of \hyperref[def:cofinal_set]{cofinal} and non-cofinal sets.

  \begin{thmenum}
    \thmitem{ex:def:cofinal_set/finite} In a finite set like \( \set{ 0, 1, 2 } \), the set \( \set{ 2 } \) containing the maximum is cofinal. This is generalized by \fullref{thm:def:cofinal_set/top} and \fullref{thm:def:cofinal_set/maximal}.

    \thmitem{ex:def:cofinal_set/integers} Consider the set \( \BbbZ \) of integers. Clearly the set \( 2\BbbZ \) of even integers is cofinal. This is generalized by \fullref{thm:def:totally_ordered_set/cofinal_iff_unbounded}.

    \thmitem{ex:def:cofinal_set/net_convergence} Cofinal sets are encountered in topology when discussing convergence of nets --- see \fullref{subsec:net_convergence}.

    \thmitem{ex:def:cofinal_set/regular_cardinals} \hyperref[def:regular_cardinal]{Regular cardinals} are equal to their own \hyperref[def:cofinality]{cofinality}.
  \end{thmenum}
\end{example}

\begin{proposition}\label{thm:def:cofinal_set}
  \hyperref[def:cofinal_set]{Cofinal sets} have the following basic properties:
  \begin{thmenum}
    \thmitem{thm:def:cofinal_set/top} If a preordered set is bounded from above, a subset is cofinal if and only if it contains a \hyperref[def:extremal_points/top_and_bottom]{top element}.

    \thmitem{thm:def:cofinal_set/maximal} A cofinal subset of a \hi{partially ordered set} contains all \hyperref[def:extremal_points/maximal_and_minimal_element]{maximal elements}.

    \thmitem{thm:def:cofinal_set/acc} In a partially ordered set satisfying the \hyperref[def:chain_condition]{ascending chain condition}, if a set contains all \hyperref[def:extremal_points/maximal_and_minimal_element]{maximal elements}, it is cofinal.

    \thmitem{thm:def:cofinal_set/transitive}\mcite[72]{Harzheim2005} For any preordered set \( P \), if \( A \) is cofinal in \( P \) and \( B \) --- in \( A \), then \( B \) is cofinal in \( P \).

    This is a form of \hyperref[def:binary_relation/transitive]{transitivity}.
  \end{thmenum}
\end{proposition}
\begin{proof}
  \SubProofOf{thm:def:cofinal_set/top} Let \( (P, \leq) \) be a preordered set with top element \( \top \).

  \SufficiencySubProof* Let \( A \) be a cofinal subset of \( P \). Then it must contain some element \( x \) such that \( \top \leq x \). By transitivity, for any element \( y \) of \( P \), we have \( y \leq \top \) and \( \top \leq x \), hence \( y \leq x \).

  Then \( x \) is a top element of \( P \) that belongs to \( A \).

  \NecessitySubProof* Suppose that \( \top \in A \). Fix an arbitrary element \( x \) of \( P \). Since \( \top \) is an upper bound of \( P \), it follows that \( x \leq \top \). But \( \top \) belongs to \( A \) and \( x \) is arbitrary, thus \( A \) is cofinal.

  \SubProofOf{thm:def:cofinal_set/maximal} Suppose that \( A \) is cofinal in \( (P, \leq) \), where \( \leq \) is a partial order. Let \( m \) be a maximal element of \( P \).

  There must exist some \( x \) in \( A \) such that \( m \leq x \). But \( m \) is maximal, thus \( x = m \). Therefore, \( m \) belongs to \( A \).

  Generalizing on \( m \), we conclude that \( A \) contains all maximal elements of \( P \).

  \SubProofOf{thm:def:cofinal_set/acc} Suppose that \( A \) contains all maximal elements.

  Let \( x \) be an arbitrary element of \( P \). Let \( B \) be the set of all elements strictly larger than \( x \).
  \begin{itemize}
    \item If \( B \) is empty, then \( x \) is itself maximal, and by assumption belongs to \( A \).
    \item Otherwise, since \( P \) satisfies the ascending chain condition, \( B \) has a maximal element \( m \). If \( y \) is any element of \( P \) such that \( m \leq y \), by transitivity \( y \) belongs to \( B \) and, since \( m \) is maximal, \( m = y \). Hence, \( m \) is also maximal for \( P \), and thus belongs to \( A \).

    Therefore, \( A \) has an element \( m \) larger than \( x \).
  \end{itemize}

  Generalizing on \( x \), we conclude that \( A \) is cofinal.

  \SubProofOf{thm:def:cofinal_set/transitive} Fix any \( p \in P \). Since \( A \) is cofinal in \( P \), there exists some \( a \in A \) such that \( p \leq a \), and similarly some \( b \in B \) such that \( a \leq b \). Transitivity of \( \leq \) implies that \( p \leq b \).

  Hence, \( B \) is transitive in \( P \).
\end{proof}

\paragraph{Moore closure operators}

\begin{definition}\label{def:extensive_function}\mcite[40]{Harzheim2005}
  We say that an \hyperref[def:function/endofunction]{endofunction} \( f \) on a \hyperref[def:partially_ordered_set]{partially ordered set} \( (P, \leq) \) is \term[en=extensive \cite[111]{Birkhoff1967}]{extensive} if \( x \leq f(x) \) for every \( x \) in \( P \).
\end{definition}
\begin{comments}
  \item \incite[40]{Harzheim2005} prefers the term \enquote{extensional}, while \incite[def. 2.12]{Гуров2013} uses \enquote{reflexive} (\enquote{рефлексивный (оператор)}).
\end{comments}

\begin{definition}\label{def:idempotent_function}\mcite[40]{Harzheim2005}
  We say that an \hyperref[def:function/endofunction]{endofunction} \( f \) on an arbitrary \hyperref[def:set]{set} \( A \) is \term{idempotent} if \( f(f(x)) = f(x) \) for every \( x \) in \( A \).
\end{definition}
\begin{comments}
  \item Idempotent functions are \hyperref[def:monoid_idempotent]{idempotent elements} in \hyperref[def:endomorphism_monoid]{endomorphism monoids}.
\end{comments}

\begin{definition}\label{def:moore_closure_operator}\mcite[40]{Harzheim2005}
  Let \( (P, \leq) \) be a \hyperref[def:partially_ordered_set]{partially ordered set}. We say that the function \( \cl: P \to P \) is a \term[ru=оператор замыкания (\cite[def. 4.12]{Гуров2013})]{Moore closure operator} in \( P \) if it is \hyperref[def:extensive_function]{extensive}, \hyperref[def:idempotent_function]{idempotent} and \hyperref[def:order_function/preserving]{order-preserving}.

  We say that \( x \) is \term[ru=замкнутый (\cite[def. 4.12]{Гуров2013})]{closed} with respect to \( \cl \) if \( x = \cl(x) \).
\end{definition}
\begin{comments}
  \item \Fullref{thm:closure_operator_minimality} gives an equivalent condition for an element to be closed while \fullref{thm:closure_operator_from_set_semilattice} simplifies defining closure operators on power sets.

  \item \incite[40]{Harzheim2005} and \incite[146]{DaveyPriestley2002} call such an operator simply a \enquote{closure operator}, \incite[def. 26]{Gratzer2011} uses \enquote{closure system}, \incite[111]{Birkhoff1967} uses \enquote{closure operation} but restricts the definition to lattices of subsets.

  We add the prefix \enquote{Moore} because of the related Moore families discussed in \fullref{def:moore_family}.
\end{comments}

\begin{proposition}\label{thm:closure_operator_minimality}
  For a given \hyperref[def:moore_closure_operator]{Moore closure operator}, we have \( \cl(x) = c \) if and only if \( c \) is the \hyperref[def:extremal_points/greatest_and_least]{least} of all closed elements greater than or equal to \( x \).
\end{proposition}
\begin{proof}
  Fix some element \( x \) and consider the set
  \begin{equation*}
    D \coloneqq \set{ d \geq x \given \cl(d) = d }.
  \end{equation*}

  Clearly \( \cl(x) \) belongs to \( D \) because \( \cl \) is extensive.

  \SufficiencySubProof Consider the closure \( \cl(x) \). Let \( d \geq x \) be an arbitrary closed element. We have \( x \leq d \), and since \( \cl \) preserves order, \( \cl(x) \leq \cl(d) = d \).

  Therefore, \( \cl(x) \) is a lower bound of \( D \) that belongs to \( D \), that is, the least element of \( D \).

  \NecessitySubProof Suppose that \( c \) is the least element of \( D \). Then \( x \leq c \leq \cl(x) \). But
  \begin{equation*}
    \cl(x) \leq \underbrace{\cl(c)}_{c} \leq \underbrace{\cl(\cl(x))}_{\cl(x)},
  \end{equation*}
  hence
  \begin{equation*}
    \cl(x) = c.
  \end{equation*}
\end{proof}

\begin{definition}\label{def:moore_family}\mcite[111]{Birkhoff1967}
  We say that a family of subsets of an arbitrary \hyperref[def:set]{set} is a \term{Moore family} if it is closed under arbitrary (including empty) intersections.
\end{definition}

\begin{proposition}\label{thm:closure_operator_from_set_semilattice}
  Let \( X \) be some set and \( \mscrL \) be a \hyperref[def:moore_family]{Moore family} in \( X \). Then the following function is a \hyperref[def:moore_closure_operator]{Moore closure operator} on \( X \):
  \begin{equation*}
    \begin{aligned}
      &\cl: \pow(X) \to \mscrL, \\
      &\cl(A) \coloneqq \bigcap \set{ L \in \mscrL \given A \subseteq L }. \\
    \end{aligned}
  \end{equation*}
\end{proposition}
\begin{comments}
  \item This proposition implies that \( \cl(A) \) is the intersection of all closed sets containing \( A \).
  \item \Fullref{thm:closure_operator_minimality} implies that \( \cl(A) \) is the smallest closed set containing \( A \).
  \item This allows us to introduce a closure operator on arbitrary families that are closed under intersection --- including \hyperref[def:topological_space]{topological closed sets}, \hyperref[def:affine_hull]{affine hulls}, \hyperref[def:convex_hull]{convex hulls} and \hyperref[def:first_order_generated_substructure]{generated first-order substructures} (groups, rings, \( R \)-modules and lattices, among others --- see \fullref{ex:def:category_of_small_first_order_models}).
\end{comments}
\begin{proof}
  \SubProofOf[def:extensive_function]{extensiveness} The intersection \( \cl(A) \) of \( \set{ L \in \mscrL \given A \subseteq L } \) obviously contains \( A \).

  \SubProofOf[def:idempotent_function]{idempotence} Note that \( \cl(A) \) itself belongs to \( \mscrL \), thus
  \begin{equation*}
    \cl(\cl(A)) = \bigcap \set{ L \in \mscrL \given \cl(A) \subseteq L } = \cl(A).
  \end{equation*}

  \SubProofOf[def:order_function/preserving]{monotonicity} If \( A \subseteq B \), then every set from \( \mscrL \) containing \( B \) also contains \( A \), hence \( \cl(A) \subseteq \cl(B) \).
\end{proof}

\paragraph{Zorn's lemma}

\begin{lemma}[Zorn's lemma]\label{thm:zorns_lemma}
  If every \hyperref[def:partial_order_chain]{chain} in a nonempty \hyperref[def:partially_ordered_set]{partially ordered set} has an \hyperref[def:extremal_points/bounds]{upper bound}, then the entire set has a \hyperref[def:extremal_points/maximal_and_minimal_element]{maximal element}.
\end{lemma}
\begin{comments}
  \item Within \hyperref[def:zfc]{\logic{ZF}}, this theorem is equivalent to the \hyperref[def:zfc/choice]{axiom of choice} --- see \fullref{thm:axiom_of_choice_equivalences/zorns_lemma}.

  \item Zorn's lemma is sometimes stated and used only in a \hyperref[thm:boolean_algebra_of_subsets]{lattice of sets} --- for example by \incite[117]{Gratzer2011}, \incite[317]{Rotman2010}, \incite[thm. 6.4.34]{Hinman2005}, \incite[151]{Enderton1977Sets} and \incite[16]{КанторовичАкилов1984}. A more general statement similar to this one is used by \incite[50]{Harzheim2005}, \incite[33]{Kelley1975}, \incite[8]{Engelking1989}, \incite[lemma V.3.1]{Aluffi2009}, \incite[880]{Lang2002}, \incite[117]{Гуров2013}, \incite[31]{Мальцев1970Системы} and \incite[13]{Проданов1982}. \incite[2]{Jacobson1989Vol2} states both.
\end{comments}
\begin{proof}
  \ImplicationSubProof[def:zfc/choice]{the axiom of choice}[thm:zorns_lemma]{Zorn's lemma} Let \( (P, \leq) \) be a partially ordered set in which every chain has an upper bound. Aiming at a contradiction, suppose that \( P \) has no maximal elements.

  Let \( \mscrC \) be the set of all chains in \( P \). Define the set-valued map \( F: \mscrC \multto P \) that assigns to each set in \( P \) the family of all its strict upper bounds.

  Since every chain \( C \) has an upper bound, say \( u \), and since \( P \) has no maximal element, there exists some element strictly larger than \( u \). Hence, \( F \) is a total set-valued map. By \fullref{thm:existence_of_single_valued_selections}, there exists a single-valued selection \( f: \pow(P) \to P \) of \( F \).

  By \fullref{thm:hartogs_lemma}, there exists a smallest \hyperref[def:ordinal]{ordinal} \( \alpha \) such that no function from \( \alpha \) to \( P \) is injective. Using \fullref{thm:bounded_transfinite_recursion}, we can define
  \begin{equation*}
    \begin{aligned}
      &g: \alpha \to P, \\
      &g(\beta) \coloneqq f(\set{ g(\gamma) \given \gamma < \beta }) = f(g[\beta]). \\
    \end{aligned}
  \end{equation*}

  If \( \beta_1 < \beta_2 \), then
  \begin{equation*}
    \set{ g(\gamma) \given \gamma < \beta_1 }
    \subsetneq
    \set{ g(\gamma) \given \gamma < \beta_2 },
  \end{equation*}
  and thus
  \begin{equation*}
    g(\beta_1) < g(\beta_2).
  \end{equation*}

  Then the images under \( g \) of different elements of \( \alpha \) are different, i.e. \( g \) is injective. But this contradicts our choice of \( \alpha \).

  The obtained contradiction shows that \( P \) has a maximal element.

  \ImplicationSubProof[thm:zorns_lemma]{Zorn's lemma}[def:zfc/choice]{axiom of choice} Let \( \mscrA \) be a family of nonempty sets. Let \( \mscrF \) be the set of all \hyperref[def:set_valued_map/partial]{partial single-valued functions} from \( \mscrA \) to \( \bigcup \mscrA \) with the subset ordering. That is, \( f \leq g \) if \( \dom(f) \subseteq \dom(g) \) for \( f, g \in \mscrF \).

  Clearly every chain has a greatest element - a total single-valued function. Then \( \mscrF \) itself has a maximal element by Zorn's lemma. This maximal element is necessarily a total function because otherwise it would not be maximal.

  Then this is the desired choice function for the family \( \mscrA \).
\end{proof}

  \subsection{Totally ordered sets}\label{subsec:totally_ordered_sets}

\begin{definition}\label{def:totally_ordered_set}\cite[def. 2.1.1(ii)]{Hinman2005}
  We say that a partially ordered set is \term{totally ordered} if either the nonstrict order \( \leq \) is \hyperref[def:binary_relation/connected]{connected} or if the strict order \( < \) is \hyperref[def:binary_relation/trichotomic]{trichotomic}.

  The theory, homomorphisms and category \( \cat{Tos} \) are inherited from \fullref{def:partially_ordered_set} with one additional axiom added.
\end{definition}
\begin{comments}
  \item Peter Hinman in \incite[def. 2.1.1(ii)]{Hinman2005} uses our terminology, however some authors give slightly different definitions. \incite[62]{Enderton1977Sets} and \incite[4]{Engelking1989} both define \enquote{linear orders} as what we call \enquote{strict total orders} (Engelking defines partial orders as nonstrict). \incite[14]{Kelley1975} also does this, but mentions the possibility of nonstrict total orders. \incite[def. I.6]{Birkhoff1948} uses the term \enquote{chain} for nonstrict total orders (we also use this in the context of partial orders --- see \fullref{def:partial_order_chain}).
\end{comments}
\begin{defproof}
  Equivalence between nonstrict and strict total orders follows directly from the compatibility condition \eqref{eq:def:preordered_set/compatibility_nonstrict}.
\end{defproof}

\begin{proposition}\label{thm:total_order_embedding_iff_strict}
  An \hyperref[def:order_homomorphism/increasing]{order homomorphism} between \hyperref[def:totally_ordered_set]{totally ordered sets} is an \hyperref[def:order_homomorphism/embedding]{order embedding} if and only if it is \hyperref[def:order_homomorphism/increasing]{strict}.
\end{proposition}
\begin{proof}
  \SufficiencySubProof Follows from \fullref{thm:order_embedding_is_strict}.

  \NecessitySubProof
  \SubProof*{Proof that \( f \) is injective} Let \( f: P \to Q \) be a strict order homomorphism and suppose that \( f(x_1) = f(x_2) \) for some \( x_1 \) and \( x_2 \) in \( P \). We will use the \hyperref[def:binary_relation/trichotomic]{trichotomy} of \( < \).
  \begin{itemize}
    \item If \( x_1 < x_2 \), then \( f(x_1) < f(x_2) \) since \( f \) is strictly monotone, which contradicts our assumption \( f(x_1) = f(x_2) \).

    \item If \( x_1 > x_2 \), similarly \( f(x_1) > f(x_2) \) and we again obtain a contradiction.

    \item It remains for \( x_1 \) to be equal to \( x_2 \).
  \end{itemize}

  Since \( x_1 \) and \( x_2 \) were arbitrary, we conclude that \( f \) is injective.

  \SubProof*{Proof that \( f^{-1} \) is order-preserving} Since \( f \) is injective, its inverse \( f^{-1} \) is a single-valued partial function, and is total and injective on \( f(P) \). It then follows from \fullref{thm:order_embedding_is_strict} that the restriction of \( f^{-1} \) to \( f(P) \) is order-preserving.
\end{proof}

\begin{corollary}\label{thm:totally_ordered_strict_isomorphisms}
  An \hyperref[def:order_homomorphism/increasing]{order homomorphism} between \hyperref[def:totally_ordered_set]{totally ordered sets} is an \hyperref[def:order_homomorphism/embedding]{order isomorphism} if and only if it is \hyperref[def:order_homomorphism/increasing]{strict} and \hyperref[def:function_invertibility/surjective]{surjective}.
\end{corollary}
\begin{proof}
  Follows from \fullref{thm:total_order_embedding_iff_strict}.
\end{proof}

\begin{proposition}\label{thm:totally_ordered_minimal_element_is_minimum}
  In a \hyperref[def:totally_ordered_set]{totally ordered set}, every \hyperref[def:extremal_points/maximal_and_minimal_element]{minimal element} is a \hyperref[def:extremal_points/maximum_and_minimum]{minimum}.
\end{proposition}
\begin{proof}
  Let \( x_0 \) be a minimal element of \( (P, \leq) \). By definition of total order, for any \( x \in P \) either \( x \leq x_0 \) or \( x_0 \leq x \). If \( x \leq x_0 \), then since \( x_0 \) is a minimal element, we have \( x = x_0 \).

  Therefore, for any \( x \in P \), either \( x = x_0 \) or \( x > x_0 \). That is, \( x \leq x_0 \) always holds, proving that \( x_0 \) is a minimum.
\end{proof}

\begin{proposition}\label{thm:totally_ordered_segment_isomorphism}
  Let \( (P, \leq) \) be a \hyperref[def:totally_ordered_set]{totally-ordered set}. Let \( Q \) be the set containing the \hyperref[def:order_interval/ray]{strict initial segment} \( P_{<x} \) for every member \( x \) of \( P \).

  Then \( (P, \leq) \) is \hyperref[def:order_homomorphism/isomorphism]{order-isomorphic} to \( (Q, \subseteq) \).
\end{proposition}
\begin{proof}
  Explicitly define the isomorphism
  \begin{equation*}
    \begin{aligned}
      &f: P \to Q \\
      &f(x) \coloneqq P_{<x} = \set{ y \in P \given y < x }.
    \end{aligned}
  \end{equation*}

  Note that \( f \) is \hyperref[def:order_homomorphism/increasing]{strictly order-preserving}. Indeed, if \( x < y \), then \( x \in P_{<y} \), but \( x \not\in P_{<x} \) and hence \( P_{<x} \) is a strict subset of \( P_{<y} \).

  Then \fullref{thm:total_order_embedding_iff_strict} implies that \( f \) is an embedding. Since \( f \) is also surjective by definition of \( Q \), it follows that it is a strict isomorphism between \( (P, \leq) \) and \( (Q, \subseteq) \).
\end{proof}

\begin{definition}\label{def:cofinal_set}\mcite[8]{Engelking1989}
  We say that a subset \( A \) of a preordered set \( (P, \leq) \) is \term{cofinal} if, for every \( x \in P \), there exists some \( y \in A \) such that \( x \leq y \).
\end{definition}

\begin{example}\label{ex:def:cofinal_set}
  We list several examples of \hyperref[def:cofinal_set]{cofinal} and non-cofinal sets.

  \begin{itemize}
    \thmitem{ex:def:cofinal_set/finite} In a finite set like \( \set{ 0, 1, 2 } \), the set \( \set{ 2 } \) containing the maximum is cofinal. This is generalized by \fullref{thm:partially_ordered_cofinal_equivalences}.

    \thmitem{ex:def:cofinal_set/integers} Consider the set \( \BbbZ \) of integers. Clearly the set \( 2\BbbZ \) of even integers is cofinal. This is generalized by \fullref{thm:totally_ordered_cofinal_equivalences}.

    \thmitem{ex:def:cofinal_set/net_convergence} Cofinal sets are encountered in topology when discussing convergence of nets --- see \fullref{subsec:net_convergence}.

    \thmitem{ex:def:cofinal_set/regular_cardinals} \hyperref[def:regular_cardinal]{Regular cardinals} are equal to their own \hyperref[def:cofinality]{cofinality}.
  \end{itemize}
\end{example}

\begin{proposition}\label{thm:partially_ordered_cofinal_equivalences}
  Let \( (P, \leq) \) be a \hyperref[def:extremal_points/upper_and_lower_bounds]{bounded from above} partially ordered set and let \( A \subseteq P \). Then \( A \) is \hyperref[def:cofinal_set]{cofinal} if and only if it contains the \hyperref[def:extremal_points/top_and_bottom]{top element} \( \top \).
\end{proposition}
\begin{comments}
  \item There is a somewhat similar result case of totally ordered sets in \fullref{thm:totally_ordered_cofinal_equivalences}.
\end{comments}
\begin{proof}
  \SufficiencySubProof Let \( A \) be a cofinal set. Then \( A \) must contain an element \( x \) such that \( \top \leq x \). But \( \top \) is a maximum and hence \( x = \top \) and thus \( \top \in A \).

  \NecessitySubProof Let \( A \) be a set containing \( \top \). Then for any \( x \in P \) we have \( x \leq \top \) and hence \( A \) is cofinal.
\end{proof}

\begin{lemma}\label{thm:unbounded_totally_ordered_set}
  A totally ordered set is \hyperref[def:extremal_points/upper_and_lower_bounds]{unbounded from above} if and only if, for every element, there exists some strictly larger element.
\end{lemma}
\begin{proof}
  Fix a totally ordered set \( (P, \leq) \).

  Expanding the definition of boundedness, we conclude that
  \begin{displayquote}
    \( P \) is unbounded if and only if there does not exist \( x \in P \) such that, for every \( y \in P \), \( y \leq x \).
  \end{displayquote}

  The consequent of the above statement has a negation in front of its outermost quantifier, hence we can move the negation inwards\footnote{This is justified by \fullref{thm:first_order_quantifiers_are_dual}} to obtain
  \begin{displayquote}
    \( P \) is unbounded if and only if, for every \( x \in P \), there exist \( y \in P \) such that \( y \leq x \) does not hold.
  \end{displayquote}

  Taking into account that \( P \) is totally ordered, \( y \leq x \) does not hold if and only if \( y > x \) holds. Then
  \begin{displayquote}
    \( P \) is unbounded if and only if, for every \( x \in P \), there exist \( y \in P \) such that \( y > x \).
  \end{displayquote}
\end{proof}

\begin{proposition}\label{thm:totally_ordered_cofinal_equivalences}
  Let \( (P, \leq) \) be an \hyperref[def:extremal_points/upper_and_lower_bounds]{unbounded from above} totally ordered set and let \( A \subseteq P \). Then \( A \) is \hyperref[def:cofinal_set]{cofinal} if and only if it is itself unbounded from above.
\end{proposition}
\begin{comments}
  \item There is a somewhat similar result in \fullref{thm:partially_ordered_cofinal_equivalences}.
  \item This equivalence is useful for \hyperref[def:regular_cardinal]{regular cardinals} --- for example \fullref{thm:cardinal_cofinality}.
\end{comments}
\begin{proof}
  We will prove the converse - a set is bounded if and only if it is not cofinal.

  \SufficiencySubProof Let \( A \) be bounded from above by \( x \). \Fullref{thm:unbounded_totally_ordered_set} then implies that there exists some \( y \) such that \( x < y \). But no element of \( A \) is greater than \( y \), implying that \( A \) is not cofinal.

  \NecessitySubProof Suppose that \( A \) is not cofinal. Let \( x \in P \). Then, for every \( y \in A \), \( x < y \) does not hold. That is, \( y \leq x \) for every \( y \in A \). Therefore, \( x \) is bounded.
\end{proof}

\begin{corollary}\label{thm:natural_number_cofinal_subsets}
  For a set of positive integers, the following are equivalent:
  \begin{thmenum}
    \thmitem{thm:natural_number_cofinal_subsets/cofinal} It is \hyperref[def:cofinal_set]{cofinal}
    \thmitem{thm:natural_number_cofinal_subsets/unbounded} It is unbounded.
    \thmitem{thm:natural_number_cofinal_subsets/infinite} It is infinite.
  \end{thmenum}
\end{corollary}
\begin{proof}
  A set of positive integers is infinite if and only if it is unbounded, which \fullref{thm:totally_ordered_cofinal_equivalences} implies holds if and only if it is cofinal.
\end{proof}

\begin{proposition}\label{thm:total_lexicographic_order_is_total_order}
  If \( (P, \leq_P) \) and \( (Q, \leq_Q) \) are \hyperref[def:totally_ordered_set]{totally ordered sets}, then the \hyperref[eq:def:lexicographic_order]{lexicographic} and \hyperref[eq:def:lexicographic_order/reverse]{reverse lexicographic} orders on \( P \times Q \) are \hyperref[def:totally_ordered_set]{strict total order} relations.
\end{proposition}
\begin{comments}
  \item An analogous result holds for partial orders (\fullref{thm:lexicographic_order_is_partial_order}) and well-ordered sets (\fullref{thm:well_ordered_lexicographic_order_is_well_ordered}).
\end{comments}
\begin{proof}
  We have already shown in \fullref{thm:lexicographic_order_is_partial_order} and these are partial orders. It only remains to check trichotomy.

  \SubProofOf[def:binary_relation/trichotomic]{trichotomy} Let \( \prec \) be the lexicographic order on \( P \times Q \). Let \( (a, b) \) and \( (c, d) \) be pairs in \( P \times Q \). Since \( <_P \) and \( <_Q \) are strict total orders, we only have the following possibilities:
  \begin{itemize}
    \item If \( a = c \) and \( b = d \), then \( (a, b) = (c, d) \).
    \item If \( a = c \) and \( b <_Q d \), then \( (a, b) \prec (c, d) \).
    \item If \( a = c \) and \( b >_Q d \), then \( (a, b) \succ (c, d) \).
    \item If \( a <_P c \), then \( (a, b) \prec (c, d) \).
    \item If \( a >_P c \), then \( (a, b) \succ (c, d) \).
  \end{itemize}

  The proof for the reverse lexicographic order is analogous.
\end{proof}

\begin{definition}\label{def:stabilizing_chain}\mimprovised
  We say that the \hyperref[def:sequence]{sequence} \( x_1, x_2, \ldots \) is an \term{ascending} if \( x_k \leq x_{k+1} \) for every \( k = 1, 2, \ldots \) and a \term{descending} if \( x_k \geq x_{k+1} \) instead. They are also referred to as \enquote{ascending/descending chains}.

  If there exists an index \( n \) such that \( x_k = x_n \) for \( k > n \). we say that the chain \term{stabilizes}.
\end{definition}

\begin{definition}\label{def:chain_condition}
  We say that a \hyperref[def:partially_ordered_set]{partially ordered set} \( (P, \leq) \) satisfies the \term{ascending chain condition} (resp. \term{descending chain condition}) if any of the following equivalent conditions hold:
  \begin{thmenum}
    \thmitem{def:chain_condition/maximal}\mcite[37]{Birkhoff1948} Every nonempty subset of \( P \) has a \hyperref[def:extremal_points/maximal_and_minimal_element]{maximal} (resp. minimal) element.

    \medskip

    \thmitem{def:chain_condition/stabilization}\mcite[thm. 5]{Birkhoff1948} Every nonstrict ascending (resp. descending) sequence \hyperref[def:stabilizing_chain]{stabilizes}.

    \medskip

    \thmitem{def:chain_condition/infinite}\mcite[thm. 9A]{Enderton1977Sets} There exists no strictly infinitely ascending (resp. descending) sequence in \( P \).

    \medskip
  \end{thmenum}
\end{definition}
\begin{defproof}
  We will restrict ourselves to ascending chains and maximal elements.

  \ImplicationSubProof{def:chain_condition/maximal}{def:chain_condition/stabilization} Suppose that every subset of \( P \) has a maximal element.

  Suppose that there is a nonstrict ascending sequence
  \begin{equation*}
    x_1 \leq x_2 \leq x_3 \leq \cdots.
  \end{equation*}

  Then the set \( \set{ x_k \given k \geq 1 } \) has a maximal element, say \( x_{k_0} \). Then, since \( x_{k_0} \) is maximal, using \hyperref[rem:induction/peano_arithmetic]{natural number induction} we can prove that \( x_{k_0} = x_{k_0 + i} \) for \( i \geq 0 \).

  Thus, the sequence stabilizes.

  \ImplicationSubProof{def:chain_condition/stabilization}{def:chain_condition/infinite} Suppose that every ascending chain in \( P \) stabilizes.

  Every strictly ascending sequence is also nonstrictly ascending, hence it stabilizes, which contradicts our assumption that it is strictly ascending. The obtained contradiction shows that there are not strictly ascending sequences in \( P \).

  \ImplicationSubProof{def:chain_condition/infinite}{def:chain_condition/maximal} Suppose that there exists no infinite strictly ascending sequence in \( P \).

  Let \( A \) be a nonempty subset of \( P \). Aiming at a contradiction, suppose that it has no maximal element.

  Pick \( x_1 \in A \). Since \( x_1 \) is not maximal, there must exist some element \( x_2 \in A \) such that \( x_1 < x_2 \). Using \hyperref[rem:natural_number_recursion]{natural number recursion}, we can build an infinitely strongly ascending sequence
  \begin{equation*}
    x_1 < x_2 < x_3 < \cdots.
  \end{equation*}

  But the existence of such a chain contradicts our assumption. Therefore, every nonempty set in \( P \) must have a maximal element.
\end{defproof}

\begin{definition}\label{def:well_ordered_set}\mcite[37]{Birkhoff1948}
  We say that a \hyperref[def:totally_ordered_set]{totally ordered set} \( (P, \leq) \) is \term{well-ordered} if it satisfies the \hyperref[def:chain_condition]{descending chain condition}.
\end{definition}

\begin{proposition}\label{thm:finite_totally_ordered_set_is_well_ordered}
  Every finite \hyperref[def:totally_ordered_set]{totally ordered set} is \hyperref[def:well_ordered_set]{well-ordered}.
\end{proposition}
\begin{proof}
  Let \( (P, \leq) \) be totally ordered set of finite cardinality \( n \). Let \( A \) be any nonempty subset of \( P \). We will show by induction on \( \card(A) \) that \( A \) has a minimum.

  \begin{itemize}
    \item If \( A = \set{ a } \), then \( a \) is vacuously a minimum.
    \item Suppose that all subsets of \( P \) of cardinality \( k - 1 \) have a minimum. Let \( \card(A) = k \). Pick an arbitrary element \( a \in A \). By the inductive hypothesis, \( A \setminus \set{ a } \) has a minimum, say \( b \). Then \( \min\set{ a, b } \) is a minimum of \( A \).
  \end{itemize}
\end{proof}

\begin{lemma}\label{thm:well_ordered_embedding_extensive}
  Any \hyperref[def:order_homomorphism/embedding]{order embedding} on a well-ordered set is \hyperref[def:extensive_function]{extensive}.

  That is, if \( (P, \leq) \) is a well-ordered set and \( f: P \to P \) is an order embedding, then \( x \leq f(x) \) for any \( x \in P \).
\end{lemma}
\begin{proof}
  We proceed by induction on \( P \). Fix \( x_0 \) and suppose that \( y \leq f(y) \) for all \( y < x_0 \).

  Aiming at a contradiction, suppose that \( f(x_0) < x_0 \). Thus, there exists some \( y_0 < x_0 \) such that \( f(x_0) = y_0 \). By the inductive hypothesis we have \( f(x_0) = y_0 \leq f(y_0) \). Thus, either \( f(x_0) < f(y_0) \), which contradicts that \( f \) is an order homomorphism, or \( f(x_0) = f(y_0) \), which contradicts the injectivity of \( f \).

  The obtained contradiction demonstrates that \( x \leq f(x) \) for all \( x \in P \).
\end{proof}

\begin{proposition}\label{thm:well_ordered_isomorphism_is_unique}
  There is at most one isomorphism between any pair of well-ordered sets.
\end{proposition}
\begin{proof}
  Fix two well-ordered sets \( (P, \leq_P) \) and \( (Q, \leq_Q) \) and let \( f: P \to Q \) and \( g: P \to Q \) be two order isomorphisms.

  For any member \( x \) of \( P \) we have the following possibilities:
  \begin{itemize}
    \item If \( f(x) <_Q g(x) \), then \( g^{-1}(f(x)) <_P x \) and \( f(g^{-1}(f(x))) <_Q f(x) \). Then we can use \hyperref[rem:natural_number_recursion]{natural number recursion} to build an infinitely descending sequence of members of \( Q \). But this is a contradiction because \( Q \) satisfies the descending chain condition.

    \item We can build a similar sequence if \( g(x) <_Q f(x) \).

    \item It remains for \( f(x) = g(x) \) to hold.
  \end{itemize}

  Since \( x \in P \) was arbitrary, we conclude that \( f = g \).
\end{proof}

\begin{proposition}\label{thm:well_ordered_lexicographic_order_is_well_ordered}
  If \( (P, \leq_P) \) and \( (Q, \leq_Q) \) are \hyperref[def:well_ordered_set]{well-ordered sets}, then the \hyperref[eq:def:lexicographic_order]{lexicographic} and \hyperref[eq:def:lexicographic_order/reverse]{reverse lexicographic} orders on \( P \times Q \) are well-ordering relations.
\end{proposition}
\begin{comments}
  \item An analogous result holds for partial orders (\fullref{thm:lexicographic_order_is_partial_order}) and total orders (\fullref{thm:total_lexicographic_order_is_total_order}).
\end{comments}
\begin{proof}
  We have already shown in \fullref{thm:total_lexicographic_order_is_total_order} and these total orders. It only remains to check the \hyperref[def:chain_condition]{descending chain condition}.

  Let \( \prec \) be the lexicographic order on \( P \times Q \).

  Suppose that there exists an infinitely descending sequence
  \begin{equation*}
    \cdots \prec (a_3, b_3) \prec (a_2, b_2) \prec (a_1, b_1).
  \end{equation*}

  Since \( P \) satisfies the descending chain condition, the corresponding sequence \( \seq{ a_k }_{k=1}^\infty \). Then there exists an index \( k_0 \) such that \( a_{k_0} = a_k \) for \( k \geq k_0 \). Therefore, the sequence \( \set{ b_k }_{k=k_0}^\infty \) must be infinitely descending. But this contradicts the chain condition imposed on \( Q \).

  Therefore, no infinitely descending sequence exists in the totally ordered set \( (P \times Q, \prec) \), and thus it is well-ordered.
\end{proof}

\begin{definition}\label{def:order_topology}\mcite[exer. 1.7.4]{Engelking1989}
  Let \( P \) be a \hyperref[def:partially_ordered_set]{totally ordered set} with more than one element. The \term{order topology} induced by \( \leq \) is the topology generated by the \hyperref[def:topological_subbase]{subbase} of open \hyperref[def:order_interval/ray]{rays}
  \begin{equation}\label{eq:def:order_topology/subbase}
    \mscrS \coloneqq \set[\Big]{ (a, \infty) \given a \in P } \cup \set[\Big]{ (-\infty, b) \given b \in P }.
  \end{equation}

  The \hyperref[def:topological_base]{base} corresponding to this subbase is
  \begin{equation}\label{eq:def:order_topology/base}
    \mscrB = \mscrS \cup \set[\Big]{ \varnothing } \cup \set[\Big]{ (a, b) \given a, b \in P \T{and} a < b }.
  \end{equation}

  See our proof of \hyperref[thm:topology_from_base/B1]{B1} for why \( P \) must have more than one element.
\end{definition}
\begin{defproof}
  \SubProof{Proof of compatibility of \( \mscrS \) and \( \mscrB \)} Define
  \begin{equation*}
    \mscrC = \set*{ \bigcap \mscrS \given* \mscrS \T{is a nonempty finite subset of} \mscrS }.
  \end{equation*}

  We will show that \( \mscrB = \mscrC \).

  Let \( B \in \mscrB \). The cases \( B \in \mscrS \) and \( B = \varnothing \) are trivial. Suppose that \( B \not\in \mscrS \). Then there exist points \( a < b \) such that
  \begin{equation*}
    B = (a, b) = (-\infty, b) \cap (a, \infty).
  \end{equation*}

  This is an intersection of members of \( \mscrS \), hence \( B \in \mscrC \). Therefore, \( \mscrB \subseteq \mscrC \).

  Now let \( C = S_1 \cap \cdots \cap S_n \), where \( S_1, \ldots, S_n \) are members of \( \mscrS \). We will show by induction on \( n > 0 \) that \( C \in \mscrB \). The case \( n = 1 \) is trivial. Suppose that all \( n \)-ary intersections belong to \( \mscrB \) and let
  \begin{equation*}
    C = S_1 \cap \cdots \cap S_n \cap S_{n+1}.
  \end{equation*}

  By the inductive hypothesis, we have that \( D \coloneqq S_1 \cap \cdots \cap S_n \) belongs to \( \mscrB \) and thus we have three cases:
  \begin{itemize}
    \item If either \( D = (a, \infty) \) or \( D = (-\infty, b) \), then \( D \in \mscrS \).
    \item If \( D = \varnothing \), then \( D = (-\infty, a) \cap (a, \infty) \) for some \( a \in P \).
    \item If \( D = (a, b) \), then \( D = (-\infty, b) \cap (a, \infty) \).
  \end{itemize}

  In all cases both \( D \) and \( C = D \cap S_{n+1} \) are finite intersection of members of \( \mscrS \). Therefore, \( \mscrC \subseteq \mscrB \). Since we already have the inclusion in the other direction, we conclude that \( \mscrC = \mscrB \).

  \SubProof{Proof that \( \mscrB \) is a base} We will show that the axioms in \fullref{thm:topology_from_base} hold.

  \SubProofOf*[thm:topology_from_base/B1]{B1} Let \( x \in P \).

  If \( x \) is a \hyperref[def:extremal_points/maximum_and_minimum]{maximum}, then take any other value \( y < x \) and the set \( (y, \infty) \) will contain \( x \). We use here that there is more than one element in \( P \).

  If \( x \) is not a maximum, then \( x \) belongs to any interval \( (-\infty, y) \) whenever \( y > x \).

  In both cases there exists an interval in \( S \) containing \( x \). Thus, \( \bigcup S = P \).

  \SubProofOf*[thm:topology_from_base/B2]{B2} Let \( U \) and \( V \) be members of \( \mscrB \). We consider \( 14 \) cases:
  \begin{itemize}
    \item If either \( U = \varnothing \) or \( V = \varnothing \), then \( U \cap V = \varnothing \).
    \item If \( U = (-\infty, u) \) and \( V = (v, \infty) \), then
    \begin{itemize}
      \item If \( u \leq v \), then \( U \cap V = \varnothing \).
      \item If \( v < u \), then \( U \cap V = (v, u) \).
    \end{itemize}

    \item If \( U = (-\infty, u) \) and \( V = (v_1, v_2) \), then
    \begin{itemize}
      \item If \( u \leq v_1 \), then \( U \cap V = \varnothing \).
      \item If \( v_1 < v_2 \leq u \), then \( U \cap V = V =  (v_1, v_2) \).
      \item If \( v_1 \leq u < v_2 \), then \( U \cap V = (v_1, u) \).
    \end{itemize}

    \item If \( U = (u_1, u_2) \) and \( V = (v, \infty) \), then
    \begin{itemize}
      \item If \( u_2 \leq v \), then \( U \cap V = \varnothing \).
      \item If \( u_1 \leq v < u_2 \), then \( U \cap V = (v, u_2) \).
      \item If \( v \leq u_1 < u_2 \), then \( U \cap V = U = (u_1, v_1) \).
    \end{itemize}

    \item If \( U = (u_1, u_2) \) and \( V = (v_1, v_2) \), then
    \begin{itemize}
      \item If \( u_2 < v_1 \), then \( U \cap V = \varnothing \).
      \item If \( u_1 < v_1 < u_2 < v_2 \) then \( U \cap V = (v_1, u_2) \).
      \item If \( u_1 < v_1 < v_2 < u_2 \) then \( U \cap V = V = (v_1, v_2) \).
      \item If \( v_1 < u_1 < u_2 < v_2 \) then \( U \cap V = U = (u_1, u_2) \).
      \item If \( v_1 < u_1 < v_2 < u_2 \) then \( U \cap V = (u_1, v_2) \).
    \end{itemize}
  \end{itemize}

  In all cases, the intersection \( U \cap V \) belongs to \( \mscrB \).
\end{defproof}

\begin{proposition}\label{thm:order_topology_intervals}
  Under the \hyperref[def:order_topology]{order topology} \( \mscrT \) on a \hyperref[def:totally_ordered_set]{totally ordered set} \( (P, \leq) \):
  \begin{thmenum}
    \thmitem{thm:order_topology_intervals/open} \hyperref[def:order_interval/ray]{Open rays} and \hyperref[def:order_interval/open]{open intervals} are open sets.

    \thmitem{thm:order_topology_intervals/closed} \hyperref[def:order_interval/ray]{Closed rays} and \hyperref[def:order_interval/closed]{closed intervals} are closed sets.
  \end{thmenum}
\end{proposition}
\begin{proof}
  \SubProofOf{thm:order_topology_intervals/open} Open rays are members of the subbase \eqref{eq:def:order_topology/subbase}, which makes them open.

  The open interval \( (a, b) \) is a members of the base \eqref{eq:def:order_topology/base}. Hence, it is also open.

  \SubProofOf{thm:order_topology_intervals/closed} The closed ray \( (-\infty, b] \) is the complement of the open ray \( (a, \infty) \), which makes it a closed set. Similarly, \( [a, \infty) \) is the complement of \( (-\infty, b) \), making it a closed set.

  For the closed interval \( [a, b] \), we have
  \begin{equation*}
    [a, b] = (-\infty, b] \cap [a, \infty).
  \end{equation*}

  Both \( (-\infty, b] \) and \( [a, \infty) \) are closed rays, and hence their union is also a closed set. Hence, \( [a, b] \) is closed.
\end{proof}

\begin{example}\label{ex:def:order_topology}
  Examples of \hyperref[def:order_topology]{order topologies} include:
  \begin{itemize}
    \item The order topology on \( \BbbR \), which is equivalent to the \hyperref[def:metric_topology]{metric topology} as shown in \fullref{thm:real_metric_and_order_topologies_coincide}.

    \item All \hyperref[def:ordinal]{ordinals} greater than \( 1 \) induce topological spaces called the \hyperref[def:ordinal_space]{ordinal spaces}.
  \end{itemize}
\end{example}

\begin{definition}\label{def:ordinal_space}
  Let \( \alpha \) be an \hyperref[def:ordinal]{ordinal}. When regarded as the set of smaller ordinals, as shown valid in \fullref{thm:ordinal_is_set_of_smaller_ordinals}, \( \alpha \) is a \hyperref[def:totally_ordered_set]{totally order set} and hence we can endow it with the \hyperref[def:order_topology]{order topology} \( \mscrT \) to obtain a \hyperref[def:topological_space]{topological space}. We call the space \( (\alpha, \mscrT) \) an \term{ordinal space}.
\end{definition}

\begin{proposition}\label{thm:limit_ordinal_order_topology}
  In an \hyperref[def:ordinal_space]{ordinal space} \( (\alpha, \mscrT) \), a nonzero ordinal \( \beta \in \alpha \) is a \hyperref[def:successor_and_limit_ordinal]{limit ordinal} if and only if it is the \hyperref[def:set_cluster_point]{cluster point} of \( \alpha \).
\end{proposition}
\begin{proof}
  \SufficiencySubProof Let \( \beta \) be a limit ordinal. We need to show that every neighborhood of \( \beta \) contains points distinct from \( \beta \).

  \Fullref{thm:properties_via_bases/cluster} allows us to only consider neighborhood in a local base at \( \beta \). We will use the base of open intervals containing \( \beta \).
  \begin{itemize}
    \item For \( \gamma < \beta < \delta \), the interval \( (\gamma, \delta) \) must contain \( \op{succ}(\gamma) \) because, as a limit ordinal, \( \beta \) contains the successors of all smaller ordinals.

    \item The same proof applies to the interval \( (\gamma, \infty) \).

    \item For \( \beta < \delta \), the interval \( (-\infty, \delta) \) contains all predecessors of \( \beta \). It is important here that \( \beta \) is nonzero, which we implicitly assume for limit ordinals.
  \end{itemize}

  Therefore, every neighborhood of the local base at \( \beta \) contains points distinct from \( \beta \), making \( \beta \) a cluster point.

  \NecessitySubProof Let \( \beta \) be a cluster point and let \( \gamma < \beta \). Then the set
  \begin{equation*}
    (\gamma, \op{succ}(\beta)) \setminus \set{ \beta }
    =
    (\gamma, \beta)
  \end{equation*}
  is nonempty, implying that \( \beta \) is not the successor of \( \gamma \). Since this holds for all smaller ordinals, we conclude that \( \beta \) is a limit ordinal.
\end{proof}

  \subsection{Lattices}\label{subsec:lattices}

\begin{definition}\label{def:semilattice}\mcite[16]{Birkhoff1948}
  Lattices are \hyperref[def:partially_ordered_set]{partially ordered sets} in which \hyperref[def:extremal_points/supremum_and_infimum]{suprema and infima} are taken as basic operations called \enquote{joins} and \enquote{meets}\footnote{The terms \hyperref[def:semilattice/join]{\enquote{join}} for \( \vee \) and \hyperref[def:semilattice/meet]{\enquote{meet}} for \( \wedge \) are notoriously difficult to remember. A helpful accident is the ability to write \enquote{meet} as \enquote{\( \wedge \wedge \)eet}.}. This shifts the focus from ordering to operations, i.e. from predicates to functions.

  \begin{thmenum}[series=def:semilattice]
    \thmitem{def:semilattice/join} We say that a partially ordered set is a \term{join-semilattice} if the supremum of every nonempty finite set always exists. The operation itself is denoted by \( \vee \) and referred to as \term{join} and rather than supremum. In contrast to suprema, joins are usually written in \hyperref[rem:first_order_formula_conventions/infix]{infix} notation, e.g. \( x \vee y \vee z \) rather than \( \sup\set{ x, y, z } \).

    \thmitem{def:semilattice/meet} Analogously, we say that a partially ordered set is aa \term{meet-semilattice} if the infimum of every nonempty finite set exists. The infimum is denoted by \( \wedge \) and called \term{meet}.

    \thmitem{def:semilattice/complete} A semilattice is said to be \term{complete} if the corresponding operation is defined for arbitrary sets rather than only finite ones.

    \thmitem{def:semilattice/lattice} We say that a partially ordered set is a \term{lattice} if it is both a join-semilattice and a meet-semilattice.

    We call it \term{join-complete} if the join-semilattice is complete, \term{meet-complete} if the meet-semilattice is complete, and simply \term{complete} if both semilattices are complete.

    \thmitem{def:semilattice/bounded} A \term{bounded lattice} is a semilattice that is \hyperref[def:extremal_points/top_and_bottom]{bounded} as a \hyperref[def:partially_ordered_set]{partially ordered set}.

    We purposely avoid defining bounded semilattices.

    \thmitem{def:semilattice/distributive_lattice}\mcite[133]{Birkhoff1948} A lattice is said to be \term{distributive} if the following two conditions hold:
    \begin{align}
      x \vee (y \wedge z) &= (x \vee y) \wedge (x \vee z) \label{eq:def:semilattice/distributive_lattice/finite/join_over_meet} \\
      x \wedge (y \vee z) &= (x \wedge y) \vee (x \wedge z) \label{eq:def:semilattice/distributive_lattice/finite/meet_over_join}.
    \end{align}

    If the lattice is \hyperref[def:semilattice/complete]{complete}, the above conditions are not enough. A complete lattice \( L \) it is said to be \term{distributive} if any of the following more general distributive axioms hold for every \( x \in L \) and \hyperref[def:cartesian_product/indexed_family]{family} \( \seq{ y_k }_{k \in \mscrK} \subseteq L \):
    \begin{align}
      x \vee \parens*{ \bigwedge_{k \in \mscrK} y_k } &= \bigwedge_{k \in \mscrK} \parens{ x \vee y_k } \label{eq:def:semilattice/distributive_lattice/arbitrary/join_over_meet} \\
      x \wedge \parens*{ \bigvee_{k \in \mscrK} y_k } &= \bigvee_{k \in \mscrK} \parens{ x \wedge y_k } \label{eq:def:semilattice/distributive_lattice/arbitrary/meet_over_join}
    \end{align}
  \end{thmenum}

  Lattices have the following metamathematical properties:
  \begin{thmenum}[resume=def:semilattice]
    \thmitem{def:semilattice/theory} The language of the theory of lattices consists of the language of the \hyperref[def:partially_ordered_set]{theory of partially ordered sets} with the addition of the binary infix functional symbols \( \vee \) and \( \wedge \). If we only want to restrict ourselves to semilattices, we can add only one of the two operations as functional symbols. If we wish to study \hyperref[def:semilattice/bounded]{bounded lattices}, as it is often done, we must also add the constants \( \top \) and \( \bot \).

    For meet-semilattices, we add the following axiom schema to the theory to ensure compatibility between infima and meets (we use \( \mathbin\& \) to denote \hyperref[def:propositional_language/connectives/conjunction]{logical conjunction} to avoid symbol collision with meets):
    \begin{equation}\label{eq:def:semilattice/theory/meet_compat}
      \parens[\Big]{ \xi \wedge \eta \doteq \zeta } \leftrightarrow \parens[\Big]{ \zeta \leq \xi \mathbin\& \zeta \leq \eta \mathbin\& \qforall \alpha ((\alpha \leq \xi \mathbin\& \alpha \leq \eta) \rightarrow \alpha \leq \zeta) }.
    \end{equation}

    An analogous axiom must be added for join-semilattices.

    For bounded lattices, we also add the axioms
    \begin{equation}\label{eq:def:semilattice/theory/bottom_compat}
      \qforall \xi (\bot \leq \xi)
    \end{equation}
    and
    \begin{equation}\label{eq:def:semilattice/theory/top_compat}
      \qforall \xi (\xi \leq \top).
    \end{equation}

    We cannot properly express the theory of complete (semi)lattices as an extension of this theory since we must define join and meet as unary operations on subsets of the domain rather than binary operations on members of the domain. Complete semilattices can instead be defined within \hyperref[def:zfc]{\logic{ZFC}}.

    \thmitem{def:semilattice/submodel} Unlike for preordered and partially ordered sets, whose submodels are discussed in \fullref{def:preordered_set/submodel}, not every subset of a semilattice is a sub-semilattice because \( \vee \) and \( \wedge \) are now regarded as functional symbols. A sub-(semi)lattice must be closed under joins and meets. The axiom \eqref{eq:def:semilattice/theory/meet_compat} is not a positive formula, but does not cause trouble itself as it merely specifies compatibility of \( \leq \) and \( \wedge \).

    For bounded lattices, the relevant constants should formally be present in the subset. This is sometimes inconvenient, however, and we will distinguish between \enquote{sublattices} and \enquote{bounded sublattices}. Similarly, we distinguish between \enquote{sublattices} and \enquote{complete sublattices}.

    \thmitem{def:semilattice/trivial} The \hyperref[rem:trivial_structure]{trivial} join-semilattice and the trivial meet-semilattice are the empty set.

    The trivial bounded lattice consists only of one element, which is both the top and the bottom.

    \thmitem{def:semilattice/initial} The \hyperref[thm:substructures_form_complete_lattice/bottom]{initial substructure} of a bounded lattice is instead \( \set{ \top, \bot } \). Note that \( \top = \bot \) if and only if the algebra is trivial. Unless trivial, the initial substructures of different bounded lattices are isomorphic since homomorphisms must preserve constants.

    \thmitem{def:semilattice/homomorphism} \hyperref[def:first_order_homomorphism]{First-order homomorphisms} between (semi)lattices are the \hyperref[def:order_homomorphism/increasing]{order-preserving} maps that preserve joins, meets and constants (where applicable).

    Homomorphisms of complete lattices need to preserve arbitrary joins and meets rather than only finite ones.

    \begin{figure}[!ht]
      \centering
      \includegraphics[page=1]{output/def__semilattice}
      \caption{A monotone map between lattices, which is not a lattice homomorphism}
      \label{fig:def:semilattice/homomorphism/monotone_map_not_homomorphism}
    \end{figure}

    As we shall see in \fullref{thm:lattice_homomorphism_is_monotone}, the requirement of monotonicity is redundant.

    \thmitem{def:semilattice/category} The \hyperref[def:category_of_small_first_order_models]{categories of \( \mscrU \)-small models} for (semi)lattices are subcategories of \hyperref[def:partially_ordered_set]{\( \ucat{Pos} \)}. We only give a special name for the category \( \ucat{Lat} \) of lattices.

    \Fullref{ex:limits_of_partially_ordered_set} discusses how \hyperref[def:category_of_cones/limit]{categorical limits} correspond to meets and \hyperref[def:category_of_cones/colimit]{categorical colimits} correspond to joins.

    \thmitem{def:semilattice/duality} The \hyperref[thm:preorder_duality]{principle of duality for partially ordered sets} holds for lattices if we also swap the binary operations \( \vee \) and \( \wedge \).

    If the lattice is bounded, we must additionally swap the constants \( \top \) and \( \bot \).

    If the lattice is bounded from only one side, the principle of duality does not hold unless we restrict ourselves to formulas that do not contain the constants.
  \end{thmenum}
\end{definition}

\begin{example}\label{ex:complete_sublattice_of_incomplete_lattice}
  Any finite \hyperref[def:semilattice/submodel]{sublattice} is \hyperref[def:semilattice/complete]{complete}, even if the ambient lattice is not.
\end{example}

\begin{proposition}\label{thm:binary_lattice_operations}\mcite[thm. II.1]{Birkhoff1948}
  Let \( (P, \leq) \) be a partially ordered set.

  \begin{thmenum}
    \thmitem{thm:binary_lattice_operations/semilattices} If it is a \hyperref[def:semilattice/join]{join-semilattice} (resp. \hyperref[def:semilattice/meet]{meet-semilattice}), then \( \vee \) (resp. \( \wedge \)) is \hyperref[def:magma/associative]{associative}, \hyperref[def:magma/commutative]{commutative} and \hyperref[def:magma/idempotent]{idempotent} when considered as a binary operation.

    \thmitem{thm:binary_lattice_operations/identity} If \( P \) is a bounded lattice, the constants act as \hyperref[def:monoid]{monoid identities}. That is, for each \( x \in P \),
    \begin{align}
      x \vee \bot = x \label{eq:thm:binary_lattice_operations/identity/join} \\
      x \wedge \top = x \label{eq:thm:binary_lattice_operations/identity/meet}
    \end{align}

    \thmitem{thm:binary_lattice_operations/absorption} If \( P \) is a lattice, then the following absorption laws hold:
    \begin{align}
      x \vee (x \wedge y) &= x \label{eq:thm:binary_lattice_operations/absorption/join} \\
      x \wedge (x \vee y) &= x \label{eq:thm:binary_lattice_operations/absorption/meet}.
    \end{align}

    \thmitem{thm:binary_lattice_operations/compatibility} The following conditions for compatibility with \( \leq \) hold:
    \begin{align}
      x \leq y &\T{if and only if} x \vee y = y \label{eq:thm:binary_lattice_operations/compatibility/join} \\
      x \leq y &\T{if and only if} x \wedge y = x \label{eq:thm:binary_lattice_operations/compatibility/meet}.
    \end{align}

    \thmitem{thm:binary_lattice_operations/new_semilattice} If \( A \) is an arbitrary \hyperref[def:set]{set} and if \( \vee \) is a binary operation that is associative, commutative and idempotent (the conclusion of \fullref{thm:binary_lattice_operations/semilattices}), then \( (A, \leq) \) is a join-semilattice with an ordering defined by \eqref{eq:thm:binary_lattice_operations/compatibility/join}.

    An analogous statement holds for meet-semilattices.

    \thmitem{thm:binary_lattice_operations/new_lattice} If \( (A, \leq) \) is both a join-semilattice and meet-semilattice and if \( \vee \) and \( \wedge \) satisfy the absorption conditions \eqref{eq:thm:binary_lattice_operations/absorption/join} and \eqref{eq:thm:binary_lattice_operations/absorption/meet}, then \( (A, \leq) \) is a lattice. Furthermore, proving idempotence for \( \vee \) or \( \wedge \) is unnecessary because both follow from the absorption conditions.

    If there exist distinguished elements \( \top \) and \( \bot \) such that the condition \fullref{thm:binary_lattice_operations/identity} holds, then \( (A, \leq) \) is bounded.
  \end{thmenum}
\end{proposition}
\begin{proof}
  \SubProofOf{thm:binary_lattice_operations/semilattices} Suprema and infima are obviously associative and commutative as binary operations because ordering is immaterial for pure sets and \( x \vee y \) is defined as \( \sup\set{ x, y } \).

  Idempotence is also obvious because \( x \vee x = \sup\set{ x } = x \).

  \SubProofOf{thm:binary_lattice_operations/identity} Obvious since \( \bot \leq x \leq \top \) for all \( x \in P \).

  \SubProofOf{thm:binary_lattice_operations/absorption} If we rewrite \eqref{eq:thm:binary_lattice_operations/absorption/join} using suprema and infima, we obtain
  \begin{equation*}
    \sup\set{ x, \inf\set{ y, x } } = x.
  \end{equation*}

  If \( x \leq y \), then \( \inf\set{ y, x } = x \) and \( \sup\set{ x, \inf\set{ y, x } } = \sup\set{ x, x } = x \).

  If \( x \geq y \), then \( \inf\set{ y, x } = y \) and \( \sup\set{ x, \inf\set{ y, x } } = \sup\set{ x, y } = x \).

  This proves \eqref{eq:thm:binary_lattice_operations/absorption/join}. Since \( \wedge \) is \( \vee \) in the \hyperref[thm:preorder_duality]{opposite partially ordered set}, \eqref{eq:thm:binary_lattice_operations/absorption/meet} follows automatically.

  \SubProofOf{thm:binary_lattice_operations/compatibility} We have
  \begin{equation*}
    x \vee y
    =
    \sup\set{ x, y }
    =
    \begin{cases}
      y, &x \leq y, \\
      x, &x > y,
    \end{cases}
  \end{equation*}
  and dually for \( \wedge \).

  \SubProofOf{thm:binary_lattice_operations/new_semilattice} Since the binary join and/or meet are defined for all members of the set \( A \), it is indeed a join-semilattice because all finite joins and meets exist by definition.

  \SubProofOf{thm:binary_lattice_operations/new_lattice} If \( \vee \) and \( \wedge \) define a join-semilattice or meet-semilattice, then absorption is simply compatibility for the two induced orders.

  Idempotence of \( \vee \) follows from \eqref{eq:thm:binary_lattice_operations/absorption/meet},
  \begin{equation*}
    x \vee x = x \vee (x \wedge (x \vee x)) = x,
  \end{equation*}
  and dually for \( \wedge \).

  The boundedness conditions are obvious.
\end{proof}

\begin{corollary}\label{thm:lattice_homomorphism_is_monotone}
  If a function \( f: L \to M \) between \hyperref[def:semilattice]{(semi)lattices} preserves either joins or meets, it is \hyperref[def:order_homomorphism/increasing]{order-preserving}.

  Thus, the requirement for \hyperref[def:semilattice/homomorphism]{lattice homomorphisms} to be order-preserving is redundant.
\end{corollary}
\begin{proof}
  If the function \( f \) preserves joins and if \( x < y \), by \eqref{eq:thm:binary_lattice_operations/compatibility/join} we have \( x \vee y = y \) and thus
  \begin{equation*}
    f(x) \vee f(y) = f(x \vee y) = f(y),
  \end{equation*}
  which again by \eqref{eq:thm:binary_lattice_operations/compatibility/join} implies \( f(x) \leq f(y) \).
\end{proof}

\begin{proposition}\label{thm:bounded_lattice_absorbing}
  In any \hyperref[def:semilattice/bounded]{bounded lattice}, \( \bot \) is absorbing with respect to meets and \( \top \) with respect to joins. That is, \( \bot \wedge x = \bot \) and \( \top \vee x = \top \).
\end{proposition}
\begin{proof}
  Obvious when the lattice is regarded as a partially ordered set.
\end{proof}

\begin{definition}\label{def:fixed_point}\mimprovised
  Given a \hyperref[def:function]{function} \( f: A \to A \) between \hyperref[def:set]{plain set}, we call \( x \in A \) a \term{fixed point} of \( f \) if \( x = f(x) \).
\end{definition}

\begin{theorem}[Knaster-Tarski theorem]\label{thm:knaster_tarski_theorem}\mcite[thm. 1]{Tarski1955}
  The \hyperref[def:fixed_point]{fixed points} of an \hyperref[def:order_homomorphism]{order-preserving} \hyperref[def:function/endofunction]{endofunction} in a \hyperref[def:semilattice/lattice]{complete lattice} form a complete sublattice. In particular, the function has at least one fixed point.
\end{theorem}
\begin{proof}
  \SubProof{Proof that a fixed point exists} Let \( (L, \leq) \) be a complete lattice and let \( \varphi: L \to L \) be a monotone function. Define
  \begin{equation*}
    D \coloneqq \set{ x \in L \given f(x) \leq x }.
  \end{equation*}

  We know that \( D \) is nonempty because \( \top \in D \).

  Since the lattice is complete, we can take \( l \coloneqq \inf D \). Note that \( f(l) \) is a lower bound of \( D \) because, for any \( y \in D \), we have
  \begin{equation*}
    f(l) \leq f(y) \leq y.
  \end{equation*}

  But \( l \) is the largest lower bound of \( D \), hence
  \begin{equation}\label{eq:thm:knaster_tarski/f_lower}
    f(l) \leq l.
  \end{equation}

  Therefore, \( f(f(l)) \leq f(l) \), and hence \( f(l) \in D \). But \( l \) is a lower bound for \( D \), hence
  \begin{equation}\label{eq:thm:knaster_tarski/f_upper}
    f(l) \geq l
  \end{equation}

  From \eqref{eq:thm:knaster_tarski/f_lower} and \eqref{eq:thm:knaster_tarski/f_upper} it follows that \( l = f(l) \), that is, \( l \) is a fixed point of \( f \).

  \SubProof{Proof that the fixed points form a complete lattice} Denote by \( F \) the set of all fixed points of \( L \). We just showed that \( F \) is nonempty. Let \( G \subseteq F \). We will show that the infimum and supremum of \( G \) is in \( F \).

  Denote
  \begin{equation*}
    l_G \coloneqq \inf G.
  \end{equation*}

  For any \( g \in G \) we have \( l_G \leq g \). From monotonicity of \( f \),
  \begin{equation*}
    f(l_G) \leq f(g) = g,
  \end{equation*}
  hence \( f(l_G) \) is a lower bound of \( F \).

  Then \( f(l_G) \leq l_G \) because \( l_G \) is the greatest lower bound of \( G \). But, because \( f \) is order-preserving, we have \( l_G \leq f(l_G) \). Therefore, \( f(l_G) = l_G \), and hence \( l_G \in F \).

  We can analogously show that \( \sup G \in F \) and conclude that \( (F, \leq) \) is itself a complete lattice.
\end{proof}

\begin{definition}\label{def:square_free}\mimprovised
  An element \( x \) of a \hyperref[def:semiring/commutative]{commutative semiring} is said to be \term{square-free} if \( y \mid x \) implies that \( y^2 \not\mid x \).
\end{definition}

\begin{remark}\label{rem:lattice_polynomials}
  Let \( L \) be a \hyperref[def:semilattice/bounded]{bounded} \hyperref[def:semilattice/distributive_lattice]{distributive lattice}. We will discuss polynomials over \( L \).

  As discussed in \fullref{ex:def:semiring/lattice}, \( L \) induces a positive and a negative \hyperref[def:semiring/commutative]{commutative semiring}. Given a set \( \mscrX \) of indeterminates, we can form the \hyperref[def:polynomial_algebra]{polynomial semiring} \( L[\mscrX] \) over the positive semiring. Suppose that we are given an evaluation \( f: \mscrX \to \fun(\mscrX, L) \) and consider the \hyperref[thm:polynomial_algebra_universal_property]{evaluation homomorphism} \( \Phi_f: L[\mscrX] \to \fun(\mscrX, L) \).

  Since meets are idempotent,
  \begin{equation*}
    \Phi_f(X^2) = \Phi_f(X) \wedge \Phi_f(X) = \Phi_f(x).
  \end{equation*}

  Hence, we can limit ourselves to \hyperref[def:square_free]{square-free} monomials. That is, monomials of the form \( \prod_{X \in \mscrX} X^{\gamma_X} \), where \( \gamma_X \) is either \( 0 \) or \( 1 \). More succinctly, since every monomial has finitely many indeterminates of positive power, it can be written as a finite meet
  \begin{equation*}
    X_1 \wedge \cdots \wedge X_n.
  \end{equation*}

  A polynomial is then a finite join of finite meets of indeterminates and constants.

  There is a nuance, however. In \cite[19]{Birkhoff1948}, a multivariate lattice polynomial is defined to consist only of indeterminates; for example
  \begin{equation*}
    p(X, Y, Z) = (X \wedge Y) \vee (X \wedge Z) \vee (Y \wedge Z).
  \end{equation*}

  This excludes coefficients before the monomials, hence making the definition distinct from the general notion of a polynomial over a commutative semiring. In \cite{Marichal2007}, polynomials with coefficients in front of the monomials are called \term{weighted lattice polynomials}. For example, a weighted polynomial is
  \begin{equation*}
    q(X, Y, Z) = (a \wedge X \wedge Y) \vee (b \wedge X \wedge Z) \vee (c \wedge Y \wedge Z).
  \end{equation*}

  Compare \( q(X, Y, Z) \) to \( p(X, Y, Z) \). We will refer to unweighted polynomials by default.

  We can analogously define polynomials over the negative semiring of \( L \), e.g.
  \begin{equation*}
    r(X, Y, Z) = (X \vee Y) \wedge (X \vee Z) \wedge (Y \vee Z).
  \end{equation*}

  The latter polynomials are related to but distinct from \hyperref[def:cnf_and_dnf]{conjunctive normal forms}, while the former --- to \hyperref[def:cnf_and_dnf]{disjunctive normal forms}.
\end{remark}

\begin{example}\label{ex:lattice_polynomials}
  We list several examples of \hyperref[rem:lattice_polynomials]{lattice polynomials}:
  \begin{thmenum}
    \thmitem{ex:lattice_polynomials/distributivity} The distributivity axiom \eqref{eq:def:semilattice/distributive_lattice/finite/meet_over_join} implies that the polynomial
    \begin{equation*}
      X \wedge (Y \vee Z)
    \end{equation*}
    over the positive semiring evaluates to the same trivariate function as the polynomial
    \begin{equation*}
      (X \wedge Y) \vee (X \wedge Z)
    \end{equation*}
    over the negative semiring.

    \thmitem{ex:lattice_polynomials/boolean} A lattice polynomial for the two-element Boolean algebra \( \set{ T, F } \) corresponds to a \hyperref[def:boolean_operator]{Boolean operator}. We cannot express negation only via \( \vee \) and \( \wedge \). \hyperref[def:positive_formula]{Positive formulas} in \hyperref[def:cnf_and_dnf]{conjunctive/disjunctive normal form}, however, correspond exactly to lattice polynomials with square-free monomials.

    For example, \hyperref[def:standard_boolean_operators]{exclusive or} \( \oplus \) can be expressed via the \hyperref[def:propositional_syntax/formula]{propositional formula}
    \begin{equation*}
      P \vee Q \vee (P \wedge Q).
    \end{equation*}

    This corresponds to a bivariate polynomial over the negative semiring of \( \set{ T, F } \).
  \end{thmenum}
\end{example}

\begin{proposition}\label{thm:lattice_divisibility}
  Identify a \hyperref[def:semilattice/bounded]{bounded} \hyperref[def:semilattice/distributive_lattice]{distributive lattice} with its \hyperref[ex:def:semiring/lattice]{positive semilattice}.

  \begin{thmenum}
    \thmitem{thm:lattice_divisibility/divides} We have \( x \mid y \) if and only if \( x \geq y \).

    \thmitem{thm:lattice_divisibility/prime} The element \( p \) is \hyperref[def:domain_divisibility/prime]{prime} if and only if \( x \wedge y \leq p \) implies that \( x \leq p \) or \( y \leq p \) (or both).
  \end{thmenum}
\end{proposition}
\begin{comments}
  \item This leads to \fullref{def:lattice_prime_element}.
\end{comments}
\begin{proof}
  \SubProofOf{thm:lattice_divisibility/divides}
  \SufficiencySubProof* Suppose that \( p \) divides \( x \). Then there exists some \( y \) for which \( x = p \wedge y \).
  \begin{itemize}
    \item If \( y \leq p \), then \( x = p \wedge y = y \leq p \).
    \item If \( y > p \), then \( x = p \wedge y = p \).
  \end{itemize}

  In both cases, \( p \geq x \).

  \NecessitySubProof* Suppose that \( p \geq x \). Then \eqref{eq:thm:binary_lattice_operations/compatibility/meet} implies that \( x = p \wedge x \), hence \( p \mid x \).

  \SubProofOf{thm:lattice_divisibility/prime} This is a reformulation of \fullref{def:domain_divisibility/prime/direct} with respect to \fullref{thm:lattice_divisibility/divides}.
\end{proof}

\begin{definition}\label{def:lattice_ideal}
  We will define concepts analogous to those from \fullref{subsec:semiring_ideals}, but adapted to \hyperref[def:semilattice/lattice]{lattices}. \Fullref{thm:lattice_divisibility} provides motivation. In some cases the notions coincide --- see \fullref{thm:semilattice_ideal_as_semiring_ideal}.

  Fix some lattice \( L \).

  \begin{thmenum}
    \thmitem{def:lattice_ideal/ideal}
    \begin{minipage}[t]{0.45\textwidth}
      We call the subset \( I \) a \term{lattice ideal} it is closed under joins and if \( i \in I \) and \( l \in L \) together imply \( i \wedge l \in I \).

      Since this also implies that \( I \) is closed under meets, it follows that ideals are automatically \hyperref[def:semilattice/submodel]{sublattices}.
    \end{minipage}
    \hspace{0.02\textwidth}
    \begin{minipage}[t]{0.45\textwidth}
      \hyperref[thm:preorder_duality]{Dually}, we call the subset \( F \) a \term{filter} if it is closed under meets and if \( f \in F \) and \( l \in L \) together imply \( f \vee l \in F \).

      An equivalent condition is given in \fullref{thm:def:lattice_ideal/directed_and_closed}.
    \end{minipage}

    \thmitem{def:lattice_ideal/principal}\mcite[exer. II.5.2]{Birkhoff1948}
    \begin{minipage}[t]{0.45\textwidth}
      We define the \term{principal ideal} of \( x \) as the \hyperref[def:order_interval/ray]{closed initial segment}
      \begin{equation*}
        \set{ y \in L \given y \leq x }.
      \end{equation*}
    \end{minipage}
    \hspace{0.02\textwidth}
    \begin{minipage}[t]{0.45\textwidth}
      Dually, we define the \term{principal filter} of \( x \) as the \hyperref[def:order_interval/ray]{closed final segment}
      \begin{equation*}
        \set{ y \in L \given y \geq x }.
      \end{equation*}
    \end{minipage}

    \thmitem{def:lattice_ideal/prime}\mcite[141]{Birkhoff1948}
    \begin{minipage}[t]{0.45\textwidth}
      We call a proper ideal \( I \) \term{prime} if \( {x \wedge y \in I} \) implies that \( x \in I \) or \( y \in I \) (or both) and \term{completely prime} if \( \bigwedge_{k \in \mscrK} x_k \in I \) implies that \( x_k \in I \) for at least one \( k \in \mscrK \).
    \end{minipage}
    \hspace{0.02\textwidth}
    \begin{minipage}[t]{0.45\textwidth}
      Dually, we call a proper filter \( F \) \term{prime} if \( x \vee y \in F \) implies that \( x \in F \) or \( y \in F \) and \term{completely prime} if \( \bigvee_{k \in \mscrK} x_k \in I \) implies that \( x_k \in I \) for at least one \( k \in \mscrK \).
    \end{minipage}

    \thmitem{def:lattice_ideal/maximal}
    \begin{minipage}[t]{0.45\textwidth}
      We call a proper ideal \term{maximal} if it is \hyperref[def:extremal_points/maximal_and_minimal_element]{maximal} with respect to set inclusion.
    \end{minipage}
    \hspace{0.02\textwidth}
    \begin{minipage}[t]{0.45\textwidth}
      Dually, we call a proper filter is \term{maximal} if it is maximal with respect to inclusion.

      Unlike for semirings, it is possible for a maximal ideal or filter not to be prime.
    \end{minipage}

    \thmitem{def:lattice_ideal/submodel}
    \begin{minipage}[t]{0.45\textwidth}
      We say that \( J \) is a \term{subideal} of \( I \) if both are ideals and if \( J \subseteq I \).
    \end{minipage}
    \hspace{0.02\textwidth}
    \begin{minipage}[t]{0.45\textwidth}
      Dually, we say that \( G \) is a \term{subfilter} of \( F \) if both are filters and if \( G \subseteq F \).
    \end{minipage}
  \end{thmenum}
\end{definition}

\begin{definition}\label{def:closed_ordered_subset}\mcite[113]{Enderton1977Sets}
  We say that the subset \( S \) of the \hyperref[def:partially_ordered_set]{partially ordered set} \( (P, \leq) \) is \term{downward closed} if, for every \( s \in S \), \( x \leq s \) implies \( x \in S \).

  We define \term{upward closed} sets analogously.
\end{definition}

\begin{definition}\label{def:directed_set}\mcite[8]{Engelking1989}
  A \hyperref[def:preordered_set]{preordered set} \( P \) is called an \term{upward/right directed set} if every finite subset of \( P \) has an \hyperref[def:extremal_points/upper_and_lower_bounds]{upper bound}, i.e. for all \( x, y \in P \) there must exist \( z \in P \) such that \( x \leq z \) and \( y \leq z \). We do not care how many upper bounds exist and how they are related, we simply need one upper bound to exist for every pair of elements of \( P \).

  \hyperref[thm:preorder_duality]{Dually}, \( P \) is a \term{downward/left directed set} if every two elements have a lower bound.
\end{definition}

\begin{proposition}\label{thm:def:lattice_ideal}
  The \hyperref[def:lattice_ideal]{ideals and filters} of a lattice have the following basic properties:
  \begin{thmenum}
    \thmitem{thm:def:lattice_ideal/directed_and_closed} A set is an ideal if and only if it is \hyperref[def:directed_set]{upward directed} and \hyperref[def:closed_ordered_subset]{downward closed}.

    Analogously, a set is a filter if it is \hyperref[def:directed_set]{downward directed} and \hyperref[def:closed_ordered_subset]{upward closed}.

    \thmitem{thm:def:lattice_ideal/prime} The complement of a \hyperref[def:lattice_ideal/prime]{prime ideal} (resp. \hyperref[def:lattice_ideal/prime]{prime filter}) is a filter (resp. ideal).

    \thmitem{thm:def:lattice_ideal/completely_prime} A set is a prime \hyperref[def:lattice_ideal/principal]{principal ideal} (resp. filter) if and only if its complement is \hyperref[def:lattice_ideal/prime]{completely prime filter} (resp. \hyperref[def:lattice_ideal/prime]{completely prime ideal}).
  \end{thmenum}
\end{proposition}
\begin{proof}
  We will prove the statements only for complements of ideals. The statements for complements of filters will follow by \hyperref[def:semilattice/duality]{duality}.

  \SubProofOf{thm:def:lattice_ideal/directed_and_closed}

  \SufficiencySubProof* Suppose that \( I \) is an ideal.

  \begin{itemize}
    \item It is upward directed because, if \( x \in I \) and \( y \in I \), then, since \( I \) is closed under joins, \( x \vee y \in I \).
    \item It is downward closed because, if \( i \in I \) and \( x \leq i \), then, since \( I \) is closed under meets with arbitrary elements of the lattice, \( x = i \wedge x \in I \)
  \end{itemize}

  \NecessitySubProof* Conversely, suppose that \( I \) is upward directed and downward closed.

  \begin{itemize}
    \item It is closed under joins because if \( x \in I \) and \( y \in I \), then directedness gives us an upper bound \( z \). Then \( x \vee y \leq z \), and since \( I \) is downward closed, we conclude that \( x \vee y \in I \).

    \item It is closed under meets with elements of the lattice because if \( i \in I \), then \( x \wedge i \leq i \) and downward closedness implies \( x \wedge i \in I \).
  \end{itemize}

  \SubProofOf{thm:def:lattice_ideal/prime} Let \( I \) be a prime ideal of the lattice \( L \). Its complement is a filter because:
  \begin{itemize}
    \item It is closed under meets. Let \( x, y \in L \setminus I \) and suppose that \( x \wedge y \in I \). Then, since \( I \) is a prime ideal, \( x \in I \) or \( y \in I \). The obtained contradiction shows that \( x \wedge y \in L \setminus I \).

    \item It is closed under joins with elements of \( L \). Let \( l \in L \) and \( f \in L \setminus I \) and suppose that \( l \vee f \in I \). Then \( f \leq l \vee f \leq p \), implying that \( f \in I \). The obtained contradiction shows that \( l \vee f \in L \setminus I \).
  \end{itemize}

  \SubProofOf{thm:def:lattice_ideal/completely_prime}

  \SufficiencySubProof* Suppose that \( I \) be a principal prime ideal with generator \( p \).

  Its complement is a filter as a consequence of \fullref{thm:def:lattice_ideal/prime}. We must show that it is completely prime.

  Suppose that \( \bigvee_{k \in \mscrK} x_k \in L \setminus I \) and that \( x_k \not\in F \) for every \( k \in \mscrK \). Then \( p \) as the generator for \( I \) is an upper bound for the set \( \set{ x_k \given k \in \mscrK } \). Since \( \bigvee_{k \in \mscrK} x_k \) is the least upper bound, \( x \) must dominate it. That is, \( \bigvee_{k \in \mscrK} x_k \) must belong to \( I \), which contradicts our assumption that it belongs to \( L \setminus I \).

  Therefore, \( I \) is a completely prime filter.

  \NecessitySubProof* Now suppose that \( I \) is completely prime. Define
  \begin{equation*}
    f_0 \coloneqq \bigwedge L \setminus I
  \end{equation*}
  and let
  \begin{equation*}
    F \coloneqq \set{ x \in L \given x \geq f_0 }
  \end{equation*}
  be the principal filter of \( f_0 \). Obviously \( L \setminus I \subseteq F \).

  Conversely, let \( y \in F \), i.e. \( y \geq f_0 \). Suppose that \( y \in I \). Since \( f_0 \leq y \) and \( I \) is an ideal, it follows that \( f_0 \in I \). Since \( I \) is completely prime, there exists at least one point of \( L \setminus I \) that is also a point of \( I \).

  The obtained contradiction shows that \( y \in F \) implies that \( y \in L \setminus I \). Hence, \( F \subseteq L \setminus I \).

  Therefore, \( F = L \setminus I \).
\end{proof}

\begin{proposition}\label{thm:semilattice_ideal_as_semiring_ideal}
  If we identify the \hyperref[def:semilattice/bounded]{bounded} \hyperref[def:semilattice/distributive_lattice]{distributive lattice} with its \hyperref[ex:def:semiring/lattice]{positive semilattice}, then the \hyperref[def:lattice_ideal/ideal]{lattice ideals} are precisely the \hyperref[def:semiring_ideal]{semiring ideals}.

  Moreover, \hyperref[def:lattice_ideal/prime]{prime} and \hyperref[def:lattice_ideal/maximal]{maximal} lattice ideals are \hyperref[def:semiring_ideal/prime]{prime} and \hyperref[def:semiring_ideal/maximal]{maximal} semiring ideals.

  The same holds for \hyperref[def:lattice_ideal/ideal]{lattice filters} if we instead consider the \hyperref[ex:def:semiring/lattice]{negative semilattice}.
\end{proposition}
\begin{proof}
  Straightforward.
\end{proof}

\begin{example}\label{ex:def:lattice_ideal}
  We list examples of \hyperref[def:lattice_ideal]{lattice ideals and filters}:
  \begin{thmenum}
    \thmitem{ex:def:lattice_ideal/trivial} The simplest ideal is \( \set{ \bot } \) and the simplest filter is \( \set{ \top } \).

    We call them the \term{trivial ideal} and \term{trivial filter}.

    \thmitem{ex:def:lattice_ideal/lattice} Consider the (zero-based) natural number divisibility lattice from \fullref{thm:natural_number_divisibility_lattice}. For any natural number \( n \), the semiring ideal
    \begin{equation*}
      \braket{ n } \coloneqq \set{ 0, n, 2n, 3n, \ldots }
    \end{equation*}
    is a principal filter in the divisibility lattice.

    Indeed,
    \begin{itemize}
      \item \( \braket{ n } \) contains the top \( 0 \).
      \item \( \braket{ n } \) contains the meet \( \gcd(a, b) \) of every two members of \( \braket{ n } \).
      \item If \( n \mid a \) and \( b \) is any natural number, then \( n \mid a \mid \lcm(a, b) \), meaning that the join \( \lcm(a, b) \) also belongs to \( \braket{ n } \).
    \end{itemize}

    If \( p \) is a \hyperref[def:prime_number]{prime number}, then \( \braket{ p } \) is a \hyperref[def:semiring_ideal/maximal]{maximal semiring ideal}, and hence also a \hyperref[def:semiring_ideal/prime]{prime semiring ideal}, and hence also a \hyperref[def:lattice_ideal/prime]{prime lattice ideal}.

    The corresponding filter is more complicated. For any natural number \( n \), the set \( D_n \) of all \hyperref[def:divisibility]{divisors} of \( n \) is a lattice ideal.

    \begin{figure}[!ht]
      \centering
      \includegraphics[page=1]{output/ex__lattice_ideals}
      \caption{A Hasse diagram for the divisors of \( 24 \).}
      \label{fig:ex:def:lattice_ideal/lattice}
    \end{figure}

    Indeed,
    \begin{itemize}
      \item The bottom \( 1 \) divides \( n \).
      \item If \( a \) and \( b \) divide \( n \), their product \( ab \) also does, and hence their join \( \lcm(a, b) \) also does.
      \item If \( a \mid n \) and \( b \) is any natural number, then \( \gcd(a, b) \mid a \mid n \).
    \end{itemize}

    Furthermore, \( D_n \) is a \hyperref[def:semiring_ideal]{principal ideal} since it can be obtained as \( \set{ n \wedge m \mid m \in \BbbN } \).

    \thmitem{ex:def:lattice_ideal/subgroups} Given a \hyperref[def:group]{group} \( G \) and a proper \hyperref[thm:normal_subgroup_equivalences]{normal subgroup} \( N \), consider the \hyperref[thm:substructures_form_complete_lattice]{lattice of subgroups} \( L_G \) of \( G \) and the lattice \( L_N \) of subgroups of \( N \). Note that the top \( N \) of \( L_N \) is not the top of \( L_G \), hence \( L_N \) is a lattice but not a bounded sublattice of \( L_G \).

    Then \( L_N \) is an ideal on \( L \).
    \begin{itemize}
      \item \( L_N \) contains the bottom \( \set{ e } \).
      \item \( L_N \) contains the join \( \braket{ K \cup H } \) of every two subgroups of \( N \).
      \item \( L_N \) contains the meet \( K \cap H \) of \( K \in L_N \) and \( H \in L_G \) since \( K \cap H \subseteq K \subseteq N \).
    \end{itemize}

    By \hyperref[def:semilattice/duality]{duality}, the sublattice of subgroups containing \( N \) (rather than contained in \( N \)) is a filter (rather than an ideal).
  \end{thmenum}
\end{example}

\begin{definition}\label{def:prefilter}\mcite[52]{Engelking1989}
  We say that a subset \( P \) of a filter \( F \) is a \term{prefilter} or \term{filter base} of \( F \) if, for every \( x \in F \), there exists some \( y \in P \) such that \( y \leq x \).
\end{definition}

\begin{proposition}\label{thm:prefilter_can_generate_filter}
  Every prefilter is downward directed.
\end{proposition}
\begin{proof}
  Given two members \( a \) and \( b \) of \( P \), since they are also members of \( F \) and so is their meet \( a \wedge b \), there exists some \( y \in P \) such that \( y \leq a \wedge b \).
\end{proof}

\begin{proposition}\label{thm:filter_from_prefilter}
  Let \( P \) be a \hyperref[def:directed_set]{downward directed} set in \( L \). Then the union of final segments
  \begin{equation*}
    F \coloneqq \bigcup_{x \in P} \set{ y \in L \given y \geq x } = \set{ y \in L \given \qexists{x \in P} y \geq x }
  \end{equation*}
  is the smallest \hyperref[def:lattice_ideal/ideal]{filter} containing \( P \).

  Furthermore, \( P \) is a \hyperref[def:prefilter]{prefilter} of \( F \). We say that the prefilter \( P \) \term{generates} \( F \).
\end{proposition}
\begin{proof}
  \SubProof{Proof that \( F \) is a filter}

  \begin{itemize}
    \item \( F \) is closed under meets. Fix some members \( x \) and \( y \) of \( F \). Since \( P \) is downward directed, there exists some \( z \in P \) such that \( z \leq x \wedge y \). Then \( x \wedge y \) belongs to the upward closure of \( z \), and hence \( F \) contains the meet of any two of its members.

    \item \( F \) is closed under joins with members of \( L \). Fix \( l \in L \) and \( f \in F \). Since \( f \leq f \vee l \), and since \( F \) contains the upward closure of \( f \), it follows that \( f \vee l \in F \).
  \end{itemize}

  \SubProof{Proof that \( F \) is minimal} Fix some other filter \( G \) containing \( P \).

  Since \( P \) is not guaranteed to be upward closed, \( G \) must contain the upward closure of each of its members. Thus, \( G \) must contain \( F \).

  This shows the minimality of \( F \).

  \SubProof{Proof that \( P \) is a prefilter} Since \( P \) contains a lower bound for any member of \( F \), it is a prefilter.
\end{proof}

\begin{definition}\label{def:filter_subbase}\mimprovised
  We say that a subset \( S \) of a filter \( F \) is a \term{filter subbase} of \( F \) if, for every \( x \in F \), there exists some subfamily \( y_1, \ldots, y_n \in S \) such that
  \begin{equation}\label{eq:def:filter_subbase}
    \bot < y_1 \wedge \cdots \wedge y_n \leq x.
  \end{equation}
\end{definition}
\begin{comments}
  \item The strict inequality is necessary because otherwise \( P = \set{ \bot } \) would always be a subbase.
\end{comments}

\begin{proposition}\label{thm:filter_from_subbase}
  Let \( S \) be a nonempty subset of \( L \). Then the family
  \begin{equation*}
    P \coloneqq \set*{ \bigwedge S' \given* S' \T{is a finite nonempty subfamily of} S }
  \end{equation*}
  is a \hyperref[def:prefilter]{prefilter}, and its generated filter \( F \) is the smallest one containing \( S \). Note that we make no claims about \( P \) being the smallest prefilter containing \( S \).

  The set \( S \) is a \hyperref[def:filter_subbase]{filter subbase} of \( F \). We say that the subbase \( S \) \term{generates} \( F \).
\end{proposition}
\begin{proof}
  The family \( P \) is a filter because it is clearly downward directed.

  The prefilter \( P \) may or may not be the smallest one containing \( S \) --- since \( x \wedge y \wedge z \leq x \wedge y \), including the element \( x \wedge y \) in \( P \) may be redundant. After taking the upward closure, however, this becomes irrelevant.

  Since we are taking the \hi{greatest} lower bound of a subfamily of \( S \) when generating the prefilter \( P \), it follows that every filter containing \( S \) also contains \( F \).
\end{proof}

\begin{definition}\label{def:lattice_prime_element}
  We say that the element \( p \) of the lattice \( L \) is \term{prime} if any of the following equivalent conditions hold:
  \begin{thmenum}
    \thmitem{def:lattice_prime_element/direct} If \( x \wedge y \leq p \), then \( x \leq p \) or \( y \leq p \) (or both).
    \thmitem{def:lattice_prime_element/ideals} The \hyperref[def:lattice_ideal/principal]{principal ideal} of \( p \) is \hyperref[def:lattice_ideal/prime]{prime}.
    \thmitem{def:lattice_prime_element/filters} The complement of the \hyperref[def:lattice_ideal/principal]{principal ideal} of \( p \) is a \hyperref[def:lattice_ideal/prime]{completely prime filter}.
  \end{thmenum}
\end{definition}
\begin{defproof}
  \ImplicationSubProof{def:lattice_prime_element/direct}{def:lattice_prime_element/ideals} Suppose that \( y \wedge z \leq p \) implies \( y \leq z \) or \( y \leq p \).

  The principal ideal \( I \) of \( p \) consists of all elements of \( L \) that are dominated by \( p \). Thus, if \( y \wedge z \in I \), then \( y \wedge z \leq p \) and hence \( y \leq p \), in which case \( y \in I \), or \( z \leq p \).

  Hence, \( I \) is a principal ideal.

  \EquivalenceSubProof{def:lattice_prime_element/ideals}{def:lattice_prime_element/filters} Follows from \fullref{thm:def:lattice_ideal/completely_prime}.

  \ImplicationSubProof{def:lattice_prime_element/filters}{def:lattice_prime_element/direct} Let \( I \) be the ideal generated by \( p \) and suppose that its complement \( L \setminus I \) is a completely prime filter.

  If \( x \wedge y \in I \), and if both \( x \) and \( y \) belong to \( L \setminus I \), then, since as a filter \( L \setminus I \) is closed under meets, it follows that \( x \wedge y \in L \setminus I \).

  The obtained contradiction shows that if \( x \wedge y \in I \), then \( x \in I \) or \( y \in I \).
\end{defproof}

  \subsection{Ordered set completion}\label{subsec:ordered_set_completion}

\paragraph{Dedekind-MacNeille completion}

\begin{definition}\label{def:dedekind_macnielle_closure}\mimprovised
  Let \( (P, \leq) \) be a \hyperref[def:partially_ordered_set]{partially ordered set}. For a given subset \( A \) of \( P \), define the following sets of all \hyperref[def:extremal_points/bounds]{upper bounds} and of all \hyperref[def:extremal_points/bounds]{lower bounds} of \( A \):
  \begin{align*}
    A^U \coloneqq \overbrace{\bigcap_{a \in A} P_{\geq a}}^{\mathclap{\set{ x \in L \given \qforall {a \in A} x \leq a }}}
    &&
    A^L \coloneqq \overbrace{\bigcap_{a \in A} P_{\leq a}}^{\mathclap{\set{ x \in L \given \qforall {a \in A} a \leq x }}}
  \end{align*}

  Then \( A \mapsto A^{UL} \) is a \hyperref[def:moore_closure_operator]{Moore closure operator}. We call \( A^{UL} \) the \term{Dedekind-MacNaille closure} of \( A \).

  \begin{figure}[!ht]
    \centering
    \includegraphics{output/def__dedekind_macnielle_closure}
    \caption{The \hyperref[def:dedekind_macnielle_completion]{Dedekind-MacNeille closure} of \( \set{ b, c } \) is \( \set{ a, b, c } \).}\label{fig:def:dedekind_macnielle_closure}
  \end{figure}
\end{definition}
\begin{defproof}
  \SubProofOf[def:extensive_function]{extensiveness} Fix a subset \( A \) of \( P \) and a member \( x \) of \( A \).

  For every \( u \in A^U \), by definition of upper bound we have \( x \leq u \). Then \( x \) is a lower limit of \( A^U \), hence \( x \in A^{UL} \).

  Generalizing on \( x \in A \), we conclude that \( A \subseteq A^{UL} \).

  \SubProofOf[def:idempotent_function]{idempotence} Fix again a subset \( A \) of \( P \).

  We have shown that \( B \subseteq B^{UL} \) for any subset \( B \) of \( P \), in particular for \( B = A \). Dually, we can show that \( B \subseteq B^{LU} \), in particular for \( B = A^U \). Furthermore, \( A \subseteq A^{UL} \) implies \( A^U \supseteq A^{ULU} \) because an upper bound for a set is necessarily an upper bound for a subset.

  Thus,
  \begin{equation*}
    A^U \subseteq A^{ULU} \subseteq A^U.
  \end{equation*}

  Therefore, \( A^U = A^{ULU} \) and \( A^{UL} = A^{ULUL} \).

  \SubProofOf[def:order_function/preserving]{monotonicity} Fix subsets \( A \subseteq B \) of \( P \) and a member \( a \) of \( A^{UL} \). We will show that \( a \) is also a member of \( B^{UL} \).

  First note that \( B^U \subseteq A^U \) because an upper limit of \( B \) is necessarily also an upper limit of \( A \).

  For every member \( u \) of \( A^U \), we have \( a \leq u \), in particular for those \( u \) that belong to \( B^U \). Therefore, \( a \) is a lower bound of \( B^U \), that is, a member of \( B^{UL} \).
\end{defproof}

\begin{proposition}\label{thm:def:dedekind_macnielle_closure}
  \hyperref[def:dedekind_macnielle_completion]{Dedekind-MacNeille closures} have the following basic properties:
  \begin{thmenum}
    \thmitem{thm:def:dedekind_macnielle_closure/downward_closed} The Dedekind-MacNeille closure of any set is \hyperref[def:closed_ordered_subset]{downward closed}.

    \thmitem{thm:def:dedekind_macnielle_closure/ideal} The Dedekind-MacNeille closure of any nonempty set in a lattice is a \hyperref[def:lattice_ideal]{lattice ideal}.

    \thmitem{thm:def:dedekind_macnielle_closure/point_in_closure} In a \hyperref[def:totally_ordered_set]{totally ordered set}, every point \( x \) in the closure \( A^{UL} \) of a \hyperref[def:closed_ordered_subset]{downward closed} set \( A \), either \( x \) is in \( A \) or is the supremum of \( A \).

    \thmitem{thm:def:dedekind_macnielle_closure/totally_ordered} In a totally ordered set, if a subset \( A \) is downward closed and, in case \( A \) has a supremum, it contains its supremum, then \( A \) is Dedekind-MacNeille closed.
  \end{thmenum}
\end{proposition}
\begin{proof}
  \SubProofOf{thm:def:dedekind_macnielle_closure/downward_closed} If \( x \in A^{UL} \) and \( y \leq x \), then \( y \leq u \) for every upper bound \( u \) of \( A \). Thus, \( y \) is a lower bound of \( A^U \), i.e. it belongs to \( A^{UL} \).

  Therefore, \( A^{UL} \) is downward closed.

  \SubProofOf{thm:def:dedekind_macnielle_closure/ideal} Let \( A \) be an arbitrary subset of the lattice \( X \). \Fullref{thm:def:dedekind_macnielle_closure/downward_closed} shows that \( A^{UL} \) is downward closed. We will show that \( A^{UL} \) is directed from above. Then from \fullref{thm:def:lattice_ideal/directed_and_closed} it will follow that \( A^{UL} \) is a lattice ideal.

  For any two elements \( x \) and \( y \) of \( A^{UL} \), their supremum \( x \wedge y \) is an upper bound of \( \set{ x, y } \subseteq A^{UL} \). For any upper bound \( u \) of \( A \), we have \( x \wedge y \leq u \), hence \( x \wedge y \) is a lower bound of \( A^{ULU} \). Then \( x \wedge y \) belongs to \( A^{ULUL} = A^{UL} \), demonstrating that \( A^{UL} \) is directed from above.

  \SubProofOf{thm:def:dedekind_macnielle_closure/point_in_closure} Let \( P \) be a totally ordered set. Fix some downward closed subset \( A \) of \( P \).

  The case where \( A \) is empty is vacuous --- every element of \( P \) is an upper bound and so \( A^{UL} \) is either empty or contains the bottom element of \( P \) (which is the supremum of the empty set \( A \)).

  Suppose that \( A \) is not empty and let \( x \) be a point in \( A^{UL} \). We have the following possibilities:
  \begin{itemize}
    \item There exists a point \( y \) in \( A \) such that \( x \leq y \). Then \( x \) is itself in \( A \) because the latter is downward closed by assumption.
    \item Otherwise, \( x \) is an upper bound of \( A \). Let \( u \) be any upper bound of \( A \). Then \( z \leq u \) for any element \( z \) of \( A^{UL} \), in particular \( x \leq u \). Therefore, \( x \) is the least upper bound of \( A \).
  \end{itemize}

  \SubProofOf{thm:def:dedekind_macnielle_closure/totally_ordered} Follows from \fullref{thm:def:dedekind_macnielle_closure/point_in_closure}.
\end{proof}

\begin{definition}\label{def:dedekind_macnielle_completion}\mcite[166]{DaveyPriestley2002}
  Let \( (P, \leq) \) be a \hyperref[def:partially_ordered_set]{partially ordered set}. Using the closure operator from \fullref{def:dedekind_macnielle_closure}, we define the \term{Dedekind-MacNeille completion} of \( P \) as the family
  \begin{equation*}
    D(P) \coloneqq \set{ A \subseteq P \given A = A^{UL} }
  \end{equation*}
  of \hyperref[def:dedekind_macnielle_closure]{Dedekind-MacNeille-closed sets} ordered via \hyperref[def:subset]{set inclusion}.

  The set \( P \) itself can be embedded via the \hyperref[def:order_function/preserving]{order-preserving map} which sends each member of \( P \) to its \hyperref[def:lattice_ideal/principal]{principal ideal}, that is,
  \begin{equation*}
    \begin{aligned}
      &\iota: P \to D(P), \\
      &\iota(x) \coloneqq P_{\leq x}.
    \end{aligned}
  \end{equation*}
\end{definition}
\begin{comments}
  \item Note that \( P_{\leq x} = \set{ x }^{UL} \).
  \item Joins may misbehave without the closure provided by \( A \mapsto A^{UL} \) --- see \fullref{ex:dedekind_macnielle_join_closure}.
\end{comments}

\begin{theorem}[Dedekind-MacNeilla completion]\label{thm:def:dedekind_macnielle_completion}
  The \hyperref[def:dedekind_macnielle_completion]{Dedekind-MacNeille completion} \( D(P) \) of a partially ordered set \( (P, \leq) \) is a \hyperref[def:lattice]{complete lattice} with the following operations:
  \begin{align*}
    \bigvee \mscrA   &\coloneqq \parens[\Big]{ \bigcup \mscrA }^{UL} \\
    \bigwedge \mscrA &\coloneqq \bigcap \mscrA
  \end{align*}

  Furthermore, joins and meets from \( P \) (whenever they exist) are preserved in \( D(P) \).
\end{theorem}
\begin{proof}
  It is obvious that \( D(P) \) is partially ordered by \( \subseteq \).

  \SubProof{Proof that \( D(P) \) is join-complete} Let \( \mscrA \) be a family of members of \( D(P) \). Then, since we have shown idempotence in \fullref{def:dedekind_macnielle_closure}, we have
  \begin{equation*}
    \parens[\Big]{ \bigvee \mscrA }^{UL}
    =
    \parens[\Big]{ \bigcup \mscrA }^{ULUL}
    =
    \parens[\Big]{ \bigcup \mscrA }^{UL}
    =
    \bigvee \mscrA.
  \end{equation*}

  Therefore, the join \( \bigvee \mscrA \) of members of \( P \) also belongs to \( D(P) \).

  \SubProof{Proof that \( D(P) \) is meet-complete} For any family \( \mscrB \) of members of \( D(P) \), we have
  \begin{align*}
    \parens[\Big]{ \bigcap \mscrB }^U
    &=
    \set[\Big]{ u \in P \given* \qforall {x \in \bigcap \mscrB} x \leq u }
    = \\ &=
    \set[\Big]{ u \in P \given* \qforall {B \in \mscrB} \qforall {x \in B} x \leq u }
    = \\ &=
    \bigcap_{B \in \mscrB} B^U
  \end{align*}
  and similarly
  \begin{equation*}
    \parens[\Big]{ \bigcap \mscrB }^L
    =
    \bigcap_{B \in \mscrB} B^L.
  \end{equation*}

  Therefore, for a fixed family \( \mscrA \subseteq D(P) \), we have
  \begin{equation*}
    \parens[\Big]{ \bigwedge \mscrA }^{UL}
    =
    \parens[\Big]{ \bigcap \mscrA }^{UL}
    =
    \parens[\Big]{ \bigcap \mscrA }^U
    =
    \bigcap \mscrA.
  \end{equation*}

  Therefore, the meet \( \bigwedge \mscrA \) is also a member of \( D(P) \).

  \SubProof{Proof that \( D(P) \) preserves joins} Let \( A \) be a set in \( P \) such that \( a_0 \coloneqq \bigvee^P A \) exists.

  Then
  \begin{equation*}
    \bigvee\nolimits^{D(P)} \set{ \iota(a) \given a \in A }
    =
    \bigvee\nolimits^{D(P)} \set{ P_{\leq a} \given a \in A }
    =
    \parens[\Big]{ \bigcup_{a \in A} P_{\leq a} }^{UL}
    =
    \parens[\Big]{ P_{\geq a_0} }^L
    =
    P_{\leq a_0}
    =
    \iota(a_0).
  \end{equation*}

  \SubProof{Proof that \( D(P) \) preserves meets} Let \( A \) be a set in \( P \) such that \( a_0 \coloneqq \bigwedge^P A \) exists.

  Then
  \begin{equation*}
    \bigwedge\nolimits^{D(P)} \set{ \iota(a) \given a \in A }
    =
    \bigwedge\nolimits^{D(P)} \set{ P_{\leq a} \given a \in A }
    =
    \bigcap_{a \in A} P_{\leq a}
    =
    P_{\leq a_0}
    =
    \iota(a_0).
  \end{equation*}
\end{proof}

\begin{example}\label{ex:dedekind_macnielle_join_closure}
  Taking the closure of the union in the definition of join for \hyperref[def:dedekind_macnielle_completion]{Dedekind-MacNeille completions} can be justified as follows.

  Consider the sequence of \hyperref[def:rational_numbers]{rational numbers}
  \begin{equation*}
    a_n \coloneqq -\frac 1 n.
  \end{equation*}

  The supremum of this sequence is zero --- it is clearly an upper bound and, for any negative rational number \( -r \), we have a larger negative number from the sequence:
  \begin{equation*}
    -r
    =
    -\frac 1 {1 / r}
    \leq
    -\frac 1 {\floor(1 / r)}
    <
    0.
  \end{equation*}

  Now consider the completion \( D(\BbbQ) \) and take the embeddings of \( \seq{ a_n }_{n=1}^\infty \), the sequence
  \begin{equation*}
    a_n' \coloneqq \iota(a_n) = \set*{ x \in \BbbQ \given* x \leq -\frac 1 n }.
  \end{equation*}

  Then
  \begin{equation*}
    \bigcup a_n' = \set{ x \in \BbbQ \given x < 0 },
  \end{equation*}
  but this is not an upper bound because the set does not itself belong to \( D(\BbbQ) \).

  Taking the closure, however, we obtain
  \begin{equation*}
    \parens[\Big]{ \bigcup a_n' }^{UL}
    =
    \set{ x \in \BbbQ \given x \geq 0 }^L
    =
    \set{ x \in \BbbQ \given x \leq 0 }
    =
    \bigcup a_n' \cup \set{ 0 }
    =
    \iota(0),
  \end{equation*}
  which is actually the supremum of \( \seq{ a_n' }_{n=1}^\infty \) in \( D(\BbbQ) \).
\end{example}

\begin{theorem}[Dedekind-MacNeille completion universal property]\label{thm:dedekind_macneille_completion_universal_property}
  The \hyperref[def:dedekind_macnielle_completion]{Dedekind-MacNeille completion} \( D(P) \) of a \hyperref[def:partially_ordered_set]{partially ordered set} \( P \) satisfies the following \hyperref[rem:universal_mapping_property]{universal mapping property}:
  \begin{displayquote}
    For every \hyperref[def:lattice]{complete lattice} \( L \), every \hyperref[def:order_function/preserving]{order-preserving map} \( \varphi: P \to L \) \hyperref[def:factors_through]{uniquely factors through} \( D(P) \). More precisely, there exists a unique \hyperref[def:lattice/homomorphism]{lattice homomorphism} \( \widetilde{\varphi}: D(P) \to L \) such that the following diagram commutes:
    \begin{equation}\label{eq:thm:dedekind_macneille_completion_universal_property/diagram}
      \begin{aligned}
        \includegraphics[page=1]{output/thm__dedekind_macneille_completion_universal_property}
      \end{aligned}
    \end{equation}
  \end{displayquote}
\end{theorem}
\begin{comments}
  \item Via \fullref{rem:universal_mapping_property}, this functor becomes \hyperref[def:category_adjunction]{left adjoint} to the \hyperref[def:concrete_category]{forgetful functor} from complete lattices to partially ordered sets.
\end{comments}
\begin{proof}
  In order for a homomorphism \( \widetilde{\varphi} \) to satisfy the theorem, it must satisfy
  \begin{equation*}
    \widetilde{\varphi}(\iota(x)) = \varphi(x).
  \end{equation*}

  Thus, \( \widetilde{\varphi}(\iota(x)) \) depends on \( \varphi(x) \), which suggests the definition
  \begin{equation*}
    \widetilde{\varphi}(A) \coloneqq \bigvee\nolimits^L \set{ \varphi(a) \given a \in A }.
  \end{equation*}

  Since \( L \) is a complete lattice, it has arbitrary joins and hence the homomorphism is well-defined. Furthermore,
  \begin{equation*}
    \widetilde{\varphi}(\iota(x))
    =
    \widetilde{\varphi}(P_{\leq x})
    =
    \bigvee\nolimits^L \set{ \varphi(a) \given a \in P_{\leq x} }
    \reloset {\eqref{eq:def:order_function/preserving}} =
    \varphi(x).
  \end{equation*}

  The last equality holds because \( \varphi \) is order-preserving and hence \( \varphi(x) \geq \varphi(a) \) for every \( a \leq x \).
\end{proof}

\begin{proposition}\label{thm:dedekind_macnielle_closure_is_totally_ordered}
  The Dedekind-MacNeille completion of a totally ordered set is also totally ordered.
\end{proposition}
\begin{proof}
  Fix a totally ordered set \( (P, \leq) \) and denote by \( D(P) \) its Dedekind-MacNeille completion.

  By \fullref{thm:def:dedekind_macnielle_completion}, \( D(P) \) is a complete lattice, and hence a partially ordered set. It remains only to prove trichotomy on \( D(P) \).

  Fix members \( A \) and \( A' \) of \( D(P) \). Note that both are \hyperref[def:closed_ordered_subset]{downward closed} because \( A = A^{UL} \) and similarly for \( A' \).

  Aiming at a contradiction, suppose that both \( A \setminus A' \) and \( A' \setminus A \) are nonempty, and let \( a \) and \( a' \) be members of the corresponding sets.

  \begin{itemize}
    \item If \( a = a' \), then \( a \) belongs both \( A \) and \( P \setminus A \), which is a contradiction.
    \item If \( a < a' \), then \( a \) is a member of \( A' \) because the latter is downward closed --- again a contradiction.
    \item If \( a > a' \), then \( a' \) is a member of \( A \), which is similar to the above case.
  \end{itemize}

  The obtained contradictions show that either \( A \subseteq A' \) or \( A' \subseteq A \).
\end{proof}

\paragraph{Dedekind cuts}

\begin{definition}\label{def:dedekind_cut}\mcite[sec. I.IV]{Beman1901Dedekind}
  A \term[ru=(Дедекиндово) сечение (\cite[34]{Тагамлицки1971Диф}), ru=(Дедекиндово) сечение (\cite[34]{Александров1977Введение})]{Dedekind cut} in a \hyperref[def:totally_ordered_set]{totally ordered set} \( (P, \leq) \) is pair \( (A, B) \) of \hi{nonempty sets} that \hyperref[def:set_partition]{partitions} \( P \), such that \( a < b \) for every \( a \in A \) and \( b \in B \).

  \begin{figure}[!ht]
    \centering
    \includegraphics{output/def__dedekind_cut}
    \caption{Dedekind cuts over the \hyperref[def:rational_numbers]{rational numbers}}
    \label{fig:def:dedekind_cut}
  \end{figure}
\end{definition}
\begin{comments}
  \item Dedekind's goal was to define the \hyperref[def:real_numbers]{real numbers} as a certain family of cuts of \hyperref[def:rational_numbers]{rational numbers}. We will only use the cuts for defining completeness and instead rely on the \hyperref[def:dedekind_macnielle_completion]{Dedekind-MacNeille completion} of the rationals. Hence, we will not be interested in the properties of Dedekind cuts. See \fullref{rem:dedekind_completion_through_dedekind_macneille_closures} for a discussion of the side effects of this decision.

  \item We use the original definition given by \incite[sec. I.IV]{Beman1901Dedekind} (although in greater generality). His definition has been refined, for example by \incite[113]{Enderton1977Sets} and \incite[17]{Rudin1976Principles}, which defines a cut as a \hyperref[def:closed_ordered_subset]{downward closed} nonempty set \( A \) with no maximum (which uniquely defines \( B \) as its complement). We will not have use for such refinements.
\end{comments}

\begin{definition}\label{def:dedekind_completeness}\mcite[sec. I.IV]{Beman1901Dedekind}
  We say that the \hyperref[def:totally_ordered_set]{totally ordered set} \( (P, \leq) \) is \term{Dedekind complete} if, for every \hyperref[def:dedekind_cut]{Dedekind cut} \( (A, B) \), either \( A \) has a maximum or \( B \) has a minimum.
\end{definition}

\begin{proposition}\label{thm:dedekind_completeness_unbounded_characterization}
  An \hi{\hyperref[def:extremal_points/bounds]{unbounded}} totally ordered set is \hyperref[def:dedekind_completeness]{Dedekind complete} if and only if every \hi{bounded} \hi{nonempty} subset has both a supremum and an infimum.
\end{proposition}
\begin{proof}
  Let \( (P, \leq) \) be an unbounded totally ordered set.

  \SufficiencySubProof Suppose that \( P \) is Dedekind complete. Let \( S \) be a bounded nonempty set in \( P \).

  \SubProof{Proof that \( S \) has a supremum} If \( S \) has a maximum, \fullref{tthm:def:extremal_points/greatest_is_supremum} implies that it is also a supremum.

  We can further assume that \( S \) has no maximum, and thus all upper bounds of \( S \) are strict. Consider the set \( B \coloneqq S^U \) of all upper bounds of \( S \) and also the set \( A \coloneqq B^L \setminus B \) of all strict lower bounds of \( B \).

  As already discussed, every element \( x \) of \( S \) satisfies \( x < u \) for every upper bound \( u \) of \( S \), thus \( S \subseteq A \).

  Since \( P \) is Dedekind complete, we have two possibilities:
  \begin{itemize}
    \item If \( A \) has a maximum it is an upper bound of \( S \) since \( S \subseteq A \). But we have defined \( A \) so that it contains no upper bounds of \( S \). Thus, \( A \) cannot have a maximum.

    \item If \( B \) has a minimum, it is the least upper bound of \( S \).
  \end{itemize}

  \SubProof{Proof that \( S \) has an infimum} Simply take the supremum of the set of lower bounds of \( S \).

  \NecessitySubProof Suppose that every bounded nonempty subset of \( P \) has both a supremum and an infimum.

  Let \( (A, B) \) be a Dedekind cut. Let \( a \) be a member of \( A \) that is not a maximum --- such an element exists because \( A \) is unbounded from below. Then \( P_{>a} \cap A \) is a nonempty bounded set, and our assumption implies that it has a supremum. Furthermore, this supremum is the supremum of \( A \).

  By the totality of the order, \( \sup A \) it is either the maximum of \( A \) or the minimum of \( B \).

  Generalizing on \( (A, B) \), we conclude that \( P \) is Dedekind complete.
\end{proof}

\begin{definition}\label{def:dedekind_completion}\mimprovised
  We define the \term{Dedekind completion} of an \hyperref[def:extremal_points/bounds]{unbounded} \hyperref[def:totally_ordered_set]{totally ordered set} as its \hyperref[def:dedekind_macnielle_completion]{Dedekind-MacNeille completion} with the top and bottom elements removed.
\end{definition}
\begin{comments}
  \item \Fullref{thm:dedekind_macnielle_closure_is_totally_ordered} implies that the Dedekind completion itself is totally ordered.
  \item We need to remove the top and bottom elements in order to approximate Dedekind's construction from \cite[sec. I.IV]{Beman1901Dedekind} --- we discuss this in \fullref{rem:dedekind_completion_through_dedekind_macneille_closures}. Otherwise, we could have just used the Dedekind-MacNeille completion itself, and we would obtain the \hyperref[def:extended_real_numbers]{extended real numbers} rather than the \hyperref[def:real_numbers]{real numbers}.
\end{comments}

\begin{remark}\label{rem:dedekind_completion_through_dedekind_macneille_closures}
  Consider the \hyperref[def:dedekind_completion]{Dedekind completion} of an unbounded totally ordered set. \incite[17]{Rudin1976Principles} and \incite[113]{Enderton1977Sets} instead define the completion as the set of all nonempty downward closed sets without a maximum. These authors call their definitions \enquote{Dedekind cuts}, but Dedekind's original definition that we use here is different. Within this remark, we will call them \enquote{Dedekind lower sets}.

  \Fullref{thm:def:dedekind_macnielle_closure/totally_ordered} implies that, in an unbounded totally ordered set, a subset is \hyperref[def:dedekind_macnielle_closure]{Dedekind-MacNeille closed} if and only if it is downward closed and, if it has a supremum, the supremum is a maximum.

  Here are two discrepancies between using Dedekind-MacNeille closed sets and using Dedekind lower sets.
  \begin{itemize}
    \item Both the empty set and the entire ambient set are Dedekind-MacNeille closed. Excluding them from the definition of Dedekind completion ensures compatibility with the nonemptiness condition of the Dedekind lower sets.

    \item Dedekind-MacNeille closed sets contain their supremum (if it exists), hence they correspond to closed initial segments like \( P_{\leq x} \), while Dedekind lower sets do not contain their supremum, hence they correspond to open initial segments \( P_{< x} \).

    If a supremum does not exist, both are open initial segments.
  \end{itemize}
\end{remark}

\begin{theorem}[Dedekind completion]\label{thm:def:dedekind_completion}
  The \hyperref[def:dedekind_completion]{Dedekind completion} of an \hyperref[def:extremal_points/bounds]{unbounded} \hyperref[def:totally_ordered_set]{totally ordered set} is, up to an \hyperref[def:preordered_set/homomorphism]{order isomorphism}, the smallest such set that is \hyperref[def:dedekind_completeness]{Dedekind complete}.
\end{theorem}
\begin{proof}
  \SubProof{Proof of completeness} Let \( (\mscrA, \mscrB) \) be a \hyperref[def:dedekind_cut]{cut} in the Dedekind completion. We will show that \( \mscrA \) has a maximum or \( \mscrB \) has a minimum.

  Since the Dedekind-MacNeille completion is a complete lattice, the family \( \mscrA \) has a join (supremum). Since the family is nonempty by definition of Dedekind cut, this supremum is not the bottom element, hence it belongs to the Dedekind completion.
  \begin{itemize}
    \item If \( \bigvee \mscrA \) belongs to \( \mscrA \), then it is the maximum of \( \mscrA \).
    \item If \( \bigvee \mscrA \) belongs to \( \mscrB \), then it is the minimum of \( \mscrB \).

    Indeed, if \( B \subseteq \bigvee \mscrA \) for some member \( B \) of \( \mscrB \), then this member is an upper bound of \( \mscrA \) smaller than \( \bigvee \mscrA \), which contradicts the minimality of the supremum.
  \end{itemize}

  \SubProof{Proof of minimality} Follows from \fullref{thm:def:dedekind_macnielle_completion} and \fullref{thm:dedekind_completeness_unbounded_characterization} by noting that the top is the supremum for unbounded from above sets, and similarly for the bottom.
\end{proof}

\begin{example}\label{ex:thm:def:dedekind_completion}
  We list examples of \hyperref[def:dedekind_completion]{Dedekind completion}:
  \begin{thmenum}
    \thmitem{ex:thm:def:dedekind_completion/integers} The set of integers is Dedekind complete. This follows from \fullref{thm:def:dedekind_completion} via \fullref{thm:dedekind_completeness_unbounded_characterization} by noting that every nonempty bounded set of integers has a maximum and a minimum. The latter is a direct consequence of a bounded set of integers being finite.

    \thmitem{ex:thm:def:dedekind_completion/rational_numbers} Most famously, the entire theory of Dedekind and Dedekind-MacNeille completions is derived from Dedekind's original construction of the real numbers from the rationals.
  \end{thmenum}
\end{example}

\begin{definition}\label{def:dense_total_order}\mcite[200]{Birkhoff1967}
  We say that a subset \( A \) of a \hyperref[def:totally_ordered_set]{totally ordered set} \( (P, \leq) \) is \term{dense} in \( P \) if, whenever \( a < c \) for some members \( a \) and \( c \) of \( P \), there exists some \( b \) in \( A \) such that \( a < b < c \).
\end{definition}

\begin{proposition}\label{thm:dedekind_completion_dense}
  A \hyperref[def:dense_total_order]{dense-in-itself} unbounded totally ordered set is \hyperref[def:dense_total_order]{dense} in its \hyperref[def:dedekind_completion]{Dedekind completion}.
\end{proposition}
\begin{proof}
  Let \( D(P) \) be the completion of \( (P, \leq) \). Consider the Dedekind-MacNeille closed nonempty proper subsets \( A \) and \( B \) of \( P \) and suppose that \( A \subsetneq B \).

  Let \( u \) be a value in \( B \) not in \( A \). It is an upper bound of \( A \), but it cannot be a least upper bound because that would contradict \fullref{thm:def:dedekind_macnielle_closure/totally_ordered}. Hence, there must exist a smaller upper bound \( l \) of \( A \).

  Since \( P \) is dense-in-itself, there exists a value \( r \) strictly between \( l \) and \( u \). Then
  \begin{equation*}
    A \subsetneq P_{\leq r} \subsetneq B.
  \end{equation*}
\end{proof}

  \subsection{Boolean algebras}\label{subsec:boolean_algebras}

\paragraph{Heyting algebras}

\begin{proposition}\label{thm:heyting_conditional_set_is_ideal}
  In any \hyperref[def:lattice]{lattice} \( X \), for any two elements \( x \) and \( y \), the following set is nonempty and \hyperref[def:closed_ordered_subset]{downward closed}:
  \begin{equation*}
    H_{x,y} \coloneqq \set{ c \in X \given x \wedge c \leq y }.
  \end{equation*}

  If \( X \) is \hyperref[def:distributive_lattice]{distributive}, then \( H_{x,y} \) is \hyperref[def:directed_set]{upward directed} and hence a \hyperref[def:lattice_ideal]{lattice ideal}.
\end{proposition}
\begin{proof}
  The set \( H_{x,y} \) is nonempty --- it contains \( y \) because \( x \wedge y \leq y \). Furthermore, if \( a \) belongs to \( H_{x,y} \) and \( b \leq a \), then \( x \wedge b \leq x \wedge a \leq y \), thus \( b \) also belongs to \( H_{x,y} \).

  If \( X \) is distributive and if \( a \) and \( b \) belong to \( H_{x,y} \), then
  \begin{equation*}
    x \wedge (a \vee b)
    \reloset {\eqref{eq:def:distributive_lattice/meet_over_join}} =
    (x \wedge a) \vee (x \wedge b)
    \reloset {\ref{thm:def:lattice/operations_preserve_order}} \leq
    y \vee y
    =
    y,
  \end{equation*}
  hence their least upper bound \( a \vee b \) is also in \( H_{x,y} \).
\end{proof}

\begin{example}\label{ex:heyting_conditional_ideal}
  We will list several examples related to \fullref{thm:heyting_conditional_set_is_ideal}.

  \begin{thmenum}
    \thmitem{ex:heyting_conditional_ideal/pentagon} Without the assumption of distributivity, \fullref{thm:heyting_conditional_set_is_ideal} may not hold. Indeed, consider the pentagon lattice \eqref{eq:ex:def:modular_lattice/pentagon}. We have
    \begin{equation*}
      H_{a,b} = \set{ \bot, b, c },
    \end{equation*}
    but
    \begin{equation*}
      a \wedge (b \vee c) = a \wedge \top = a
    \end{equation*}
    is not in \( H_{a,b} \).

    Nevertheless, \( H_{b,c} = \set{ \bot, c } \) is an ideal --- in fact, the principal ideal of \( c \).

    \thmitem{ex:heyting_conditional_ideal/diamond} Consider the diamond lattice \eqref{eq:ex:def:distributive_lattice/diamond}. We have
    \begin{equation*}
      H_{a,c} = \set{ \bot, b, c }
    \end{equation*}
    but, again,
    \begin{equation*}
      a \wedge (b \vee c) = a \wedge \top = a,
    \end{equation*}
    so \( H_{a,c} \) is also not an ideal.

    \thmitem{ex:heyting_conditional_ideal/nonprincipal_ideal} Adjoin to the \hyperref[def:ordinal]{ordinal} \( \omega + 1 \) an auxiliary element \( a \) that is comparable only to the bottom \( 0 \) and the top \( \omega \). The following is a fragment of the Hasse diagram of \( \omega \cup \set{ a } \):
    \begin{equation*}
      \begin{aligned}
        \includegraphics[page=2]{output/ex__heyting_conditional_ideal}
      \end{aligned}
    \end{equation*}

    It is a distributive lattice as a consequence of \fullref{thm:distributive_lattice_characterization}.

    We have \( H_{a,0} = \omega \), which is an ideal of \( \omega \cup \set{ a } \). It is not, however, a principal ideal, because it contains no greatest element.

    On the other hand, the ideal \( H_{\omega,a} = \set{ 0, a } \) is principal.
  \end{thmenum}
\end{example}

\begin{proposition}\label{thm:heyting_conditional_set}
  For every two \hyperref[def:lattice]{lattice} elements \( x \) and \( y \), the following are equivalent for a third element \( z \):
  \begin{thmenum}
    \thmitem{thm:heyting_conditional_set/iff} We have \( c \leq z \) if and only if \( x \wedge c \leq y \).
    \thmitem{thm:heyting_conditional_set/greatest} We have \( x \wedge z \leq y \) and \( z \) is the greatest among all such elements.
    \thmitem{thm:heyting_conditional_set/ideal} The set\fnote{We have not required the lattice to be distributive, hence \( H_{x,y} \) may fail to be an ideal in general.} \( H_{x,y} \) from \fullref{thm:heyting_conditional_set_is_ideal} is a \hyperref[def:lattice_ideal/principal]{principal ideal} generated by \( z \).
  \end{thmenum}
\end{proposition}
\begin{comments}
  \item In particular, if \( z \) exists, it is unique.
\end{comments}
\begin{proof}
  \ImplicationSubProof{thm:heyting_conditional_set/iff}{thm:heyting_conditional_set/greatest} If \( c \) is an element such that \( x \wedge c \leq y \), by assumption \( c \leq z \). Then \( z \) is the greatest among all such elements.

  \ImplicationSubProof{thm:heyting_conditional_set/greatest}{thm:heyting_conditional_set/ideal} Trivial.

  \ImplicationSubProof{thm:heyting_conditional_set/ideal}{thm:heyting_conditional_set/iff} If \( c \leq z \), then \( c \) is in \( H_{x,y} \) and hence \( x \wedge c \leq y \). Conversely, if \( x \wedge c \leq y \), then \( c \) is in \( H_{x,y} \) and thus \( c \leq z \).
\end{proof}

\begin{definition}\label{def:heyting_algebra}\mcite[331]{PicadoPultr2012}
  A \term{Heyting algebra} is a \hyperref[def:extremal_points/bounds]{bounded} \hyperref[def:lattice]{lattice} \( X \) with an additional \hyperref[def:binary_operation]{binary operation} \( {\rightarrow} \), which we call a \term{relative pseudocomplement} in accordance to \cite[51]{Birkhoff1967}, such that any of the following equivalent conditions hold:
  \begin{thmenum}[series=def:heyting_algebra]
    \thmitem{def:heyting_algebra/ideal} The element \( x \rightarrow y \) satisfies the equivalent conditions in \fullref{thm:heyting_conditional_set}.

    \thmitem{def:heyting_algebra/axioms} The following first-order axioms hold:
    \begin{thmenum}
      \thmitem{def:heyting_algebra/axioms/self} The following adaptation of \fullref{thm:boolean_equivalences/self_conditional}:
      \begin{equation}\label{eq:def:heyting_algebra/axioms/self}
        \mathllap{\xi \rightarrow \xi} \syneq \mathrlap{\top.}
      \end{equation}

      \thmitem{def:heyting_algebra/axioms/modus_ponens} The following variation of \eqref{eq:def:def:axiomatic_deductive_system/mp}:
      \begin{equation}\label{eq:def:heyting_algebra/axioms/modus_ponens}
        \mathllap{\xi \wedge (\xi \rightarrow \eta)} \syneq \mathrlap{\xi \wedge \eta.}
      \end{equation}

      \thmitem{def:heyting_algebra/axioms/modular} The following consequence of the modular identity \eqref{eq:def:modular_lattice}\fnote
        {
          If \( \varphi \) and \( \psi \) are \hyperref[def:propositional_syntax/formula]{propositional formulas}, then
          \begin{equation*}
            \varphi \synwedge (\psi \synimplies \varphi)
            \reloset {\eqref{eq:thm:boolean_equivalences/conditional_as_disjunction}} \gleichstark
            \varphi \synwedge (\synneg \psi \synvee \varphi)
            \reloset {\eqref{eq:def:modular_lattice}} \gleichstark
            (\varphi \synwedge \synneg \psi) \synvee \varphi
            \gleichstark
            \varphi.
          \end{equation*}
        }:
      \begin{equation}\label{eq:def:heyting_algebra/axioms/modular}
        \mathllap{\eta \wedge (\xi \rightarrow \eta)} \syneq \mathrlap{\eta.}
      \end{equation}

      \thmitem{def:heyting_algebra/axioms/distributive} Distributivity of the relative pseudocomplement over conjunction:
      \begin{equation}\label{eq:def:heyting_algebra/axioms/distributive}
        \mathllap{\xi \rightarrow (\eta \wedge \zeta)} \syneq \mathrlap{(\xi \rightarrow \eta) \wedge (\xi \rightarrow \zeta).}
      \end{equation}
    \end{thmenum}
  \end{thmenum}

  Heyting algebras also have the following additional structure:
  \begin{thmenum}[resume=def:heyting_algebra]
    \thmitem{def:heyting_algebra/pseudocomplement} A unary operation \( {\widetilde {\anon}} \), which we call the \term{pseudocomplement}, for which the following axiom holds:
    \begin{equation}\label{eq:def:heyting_algebra/pseudocomplement}
      \widetilde \xi \syneq \xi \rightarrow \bot.
    \end{equation}
  \end{thmenum}

  Heyting algebras have the following metamathematical properties:
  \begin{thmenum}[resume=def:heyting_algebra]
    \thmitem{def:heyting_algebra/theory} We extend the \hyperref[def:lattice/theory]{first-order theory of lattices} by adding:
    \begin{itemize}
      \item The nullary functional symbols \( \top \) and \( \bot \) along with the axioms \eqref{eq:thm:def:lattice/bounded_absorption/join} and \eqref{eq:thm:def:lattice/bounded_absorption/meet}.

      \item The infix binary functional symbol \( \rightarrow \) along with the axioms \eqref{eq:def:heyting_algebra/axioms/self}, \eqref{eq:def:heyting_algebra/axioms/modus_ponens}, \eqref{eq:def:heyting_algebra/axioms/modular} and \eqref{eq:def:heyting_algebra/axioms/distributive}.

      \item The unary functional symbol \( {\widetilde {\anon}} \) along with the axiom \eqref{eq:def:heyting_algebra/pseudocomplement}.
    \end{itemize}

    \thmitem{def:heyting_algebra/submodel} In addition to containing the joins and meets of all its members, a \hyperref[def:first_order_submodel]{first-order submodel} of a Heyting algebra must also contain the top, bottom and be closed with respect to the relative pseudocomplement\fnote{It follows from \eqref{eq:def:heyting_algebra/pseudocomplement} that if a Heyting subalgebra is closed under relative pseudocomplements and it contains the bottom element, then it is also closed under pseudocomplements.}. We call such submodels \term{Heyting subalgebras}.

    \thmitem{def:heyting_algebra/homomorphism} A function \( f: X \to Y \) between Heyting algebras is a \hyperref[def:first_order_homomorphism]{first-order homomorphisms} if it is a \hyperref[def:lattice/homomorphism]{lattice homomorphism} that additionally satisfies
    \begin{align}\label{eq:def:heyting_algebra/homomorphism/top_bottom}
      f(\top) = \top
      &&
      f(\bot) = \bot
    \end{align}
    and
    \begin{equation}\label{eq:def:heyting_algebra/homomorphism/operation}
      f(x_1 \rightarrow x_2) = f(x_1) \rightarrow f(x_2).
    \end{equation}

    \thmitem{def:heyting_algebra/opposite} The \hyperref[def:lattice/opposite]{dual lattice} \( X^{\opcat} \) of a Heyting algebra may not be a Heyting algebra.

    \thmitem{def:heyting_algebra/category} We denote the \hyperref[def:category_of_small_first_order_models]{category of \( \mscrU \)-small models} for Heyting algebras by \( \cat{Heyt} \). It is a subcategory of the \hyperref[def:lattice/category]{category \( \cat{Lat} \) of lattices}.
  \end{thmenum}
\end{definition}
\begin{comments}
  \item What we call a Heyting algebra is called a \enquote{Browerian lattice} by \incite[147]{Birkhoff1967} and a \enquote{browerian algebra} \incite[7]{Golan2010}. The latter uses \enquote{Heyting algebras} for what we call a \hyperref[def:complete_lattice]{complete} Heyting algebra.
\end{comments}
\begin{defproof}
  \ImplicationSubProof{def:heyting_algebra/ideal}{def:heyting_algebra/axioms} Suppose that, for any elements \( x \) and \( y \), we have
  \begin{equation}\label{eq:def:heyting_algebra/proof/iff}
    c \leq (x \rightarrow y) \T{if and only if} x \wedge c \leq y.
  \end{equation}

  \SubProofOf*{def:heyting_algebra/axioms/self} Clearly \( x \wedge \top = x \), thus \( \top \leq (x \rightarrow x) \) and hence \( x \rightarrow x \) is the top element.

  \SubProofOf*{def:heyting_algebra/axioms/modus_ponens} \eqref{eq:def:heyting_algebra/proof/iff} implies
  \begin{equation*}
    x \wedge (x \rightarrow y) \leq y.
  \end{equation*}

  Then, since \( x \wedge \anon \) preserves order,
  \begin{equation*}
    \underbrace{x \wedge (x \wedge (x \rightarrow y))}_{x \wedge (x \rightarrow y)} \leq x \wedge y.
  \end{equation*}

  Conversely, since \( x \wedge y \leq y \), \eqref{eq:def:heyting_algebra/proof/iff} implies that \( y \leq (x \rightarrow y) \). Hence,
  \begin{equation*}
    x \wedge y \leq x \wedge (x \rightarrow y).
  \end{equation*}

  Then \eqref{eq:def:heyting_algebra/axioms/modus_ponens} follows.

  \SubProofOf*{def:heyting_algebra/axioms/modular} Since \( y \leq (x \rightarrow y) \), obviously
  \begin{equation*}
    y \wedge (x \rightarrow y) = y.
  \end{equation*}

  \SubProofOf*{def:heyting_algebra/axioms/distributive} \eqref{eq:def:heyting_algebra/proof/iff} implies
  \begin{equation*}
    x \wedge (x \rightarrow y) \leq y
  \end{equation*}
  and similarly
  \begin{equation*}
    x \wedge (x \rightarrow z) \leq z.
  \end{equation*}

  Then \fullref{thm:def:lattice/operations_preserve_order} implies that
  \begin{equation*}
    x \wedge \parens[\Big]{ (x \rightarrow y) \wedge (x \rightarrow z) } \leq y \wedge z.
  \end{equation*}

  Hence,
  \begin{equation}\label{eq:def:heyting_algebra/proof/distributive_backward}
    (x \rightarrow y) \wedge (x \rightarrow z) \leq x \rightarrow (y \wedge z).
  \end{equation}

  Conversely, we have
  \begin{equation*}
    x \wedge (x \rightarrow (y \wedge z)) \leq y \wedge z \leq y,
  \end{equation*}
  thus
  \begin{equation*}
    x \rightarrow (y \wedge z) \leq (x \rightarrow y)
  \end{equation*}
  and similarly
  \begin{equation*}
    x \rightarrow (y \wedge z) \leq (x \rightarrow z).
  \end{equation*}

  Therefore,
  \begin{equation}\label{eq:def:heyting_algebra/proof/distributive_forward}
    x \rightarrow (y \wedge z) \leq (x \rightarrow y) \wedge (x \rightarrow z).
  \end{equation}

  Combining \eqref{eq:def:heyting_algebra/proof/distributive_backward} and \eqref{eq:def:heyting_algebra/proof/distributive_forward}, we obtain \eqref{eq:def:heyting_algebra/axioms/distributive}.

  \ImplicationSubProof{def:heyting_algebra/axioms}{def:heyting_algebra/ideal} Suppose that the axioms from \fullref{def:heyting_algebra/axioms} hold.

  First let \( c \leq (x \rightarrow y) \). \Fullref{thm:def:lattice/operations_preserve_order} implies that
  \begin{equation*}
    x \wedge c
    \leq
    x \wedge (x \rightarrow y)
    \reloset {\eqref{eq:def:heyting_algebra/axioms/modus_ponens}} =
    x \wedge y
    \leq
    y.
  \end{equation*}

  Conversely, if \( x \wedge c \leq y \), we have
  \begin{equation*}
    c
    \reloset {\eqref{eq:def:heyting_algebra/axioms/modular}} =
    c \wedge (x \rightarrow c)
    \leq
    \top \wedge (x \rightarrow c)
    \reloset {\eqref{eq:def:heyting_algebra/axioms/self}} =
    (x \rightarrow x) \wedge (x \rightarrow c)
    \reloset {\eqref{eq:def:heyting_algebra/axioms/distributive}} =
    x \rightarrow (\underbrace{x \wedge c}_{\leq y})
    \leq
    x \rightarrow y.
  \end{equation*}
\end{defproof}

\begin{proposition}\label{thm:def:heyting_algebra}
  \hyperref[def:heyting_algebra]{Heyting algebras} have the following basic properties:
  \begin{thmenum}
    \thmitem{thm:def:heyting_algebra/distributive} Every Heyting algebra is a \hyperref[def:distributive_lattice]{distributive lattice}.
    \thmitem{thm:def:heyting_algebra/leq} We always have \( y \leq (x \rightarrow y) \).
  \end{thmenum}
\end{proposition}
\begin{proof}
  \SubProofOf{thm:def:heyting_algebra/distributive} \Fullref{ex:heyting_conditional_ideal/pentagon} implies that the pentagon lattice \eqref{eq:ex:def:modular_lattice/pentagon} is not a Heyting algebra, and \fullref{ex:heyting_conditional_ideal/diamond} implies that the diamond lattice \eqref{eq:ex:def:distributive_lattice/diamond} is not a Heyting algebra. Therefore, no Heyting algebra contains as a sublattice either \( N_5 \) or \( M_3 \).

  Then \fullref{thm:distributive_lattice_characterization} implies that Heyting algebras are distributive.

  \SubProofOf{thm:def:heyting_algebra/leq} Clearly \( y \in H_{x,y} \) because \( x \wedge y \leq y \).
\end{proof}

\begin{proposition}\label{thm:complete_heyting_algebra}
  A \hyperref[def:complete_lattice]{complete lattice} is a \hyperref[def:heyting_algebra]{Heyting algebra} if and only if the following infinite distributive law holds:
  \begin{equation}\label{eq:thm:complete_heyting_algebra/infinite_distributive}
    x \wedge \parens[\Big]{ \bigvee_{k \in \mscrK} y_k } = \bigvee_{k \in \mscrK} (x \wedge y_k).
  \end{equation}

   Furthermore, we have
  \begin{equation}\label{eq:thm:complete_heyting_algebra/relative_pseudocomplement}
    (x \rightarrow y) \coloneqq \bigvee\set{ c \in X \given x \wedge c \leq y }.
  \end{equation}
\end{proposition}
\begin{proof}
  Fix a complete lattice \( X \).

  \SufficiencySubProof Suppose that \( X \) is a Heyting algebra.

  Fix some element \( y \) and an arbitrary indexed family \( \seq{ x_k }_{k \in \mscrK} \) from \( X \). We have, for every \( a \) in \( X \),
  \begin{equation}\label{eq:thm:complete_heyting_algebra/proof/first}
    x \wedge \parens[\Big]{ \bigvee_{k \in \mscrK} y_k } \leq a
  \end{equation}
  if and only if
  \begin{equation*}
    \bigvee_{k \in \mscrK} y_k \leq (x \rightarrow a)
  \end{equation*}
  if and only if, for every \( k \in \mscrK \),
  \begin{equation*}
    y_k \leq (x \rightarrow a)
  \end{equation*}
  if and only if, for every \( k \in \mscrK \),
  \begin{equation}\label{eq:thm:complete_heyting_algebra/proof/last}
    x \wedge y_k \leq a.
  \end{equation}

  Taking \( a \) to be \( \bigvee_{k \in \mscrK} (x \wedge y_k) \), we obtain a true statement in \eqref{eq:thm:complete_heyting_algebra/proof/last}, hence \eqref{eq:thm:complete_heyting_algebra/proof/first} holds:
  \begin{equation*}
    \parens[\Big]{ \bigvee_{k \in \mscrK} y_k } \leq \bigvee_{k \in \mscrK} (x \wedge y_k).
  \end{equation*}

  The converse follows from \fullref{thm:def:complete_lattice/distributive_inequality}.

  \NecessitySubProof Suppose that \eqref{eq:thm:complete_heyting_algebra/infinite_distributive} holds. Consider the definition \eqref{eq:thm:complete_heyting_algebra/relative_pseudocomplement}.

  Obviously \( a \wedge x \leq y \) implies \( a \leq (x \rightarrow y) \).

  Conversely, suppose that \( a \leq (x \rightarrow y) \). Then
  \begin{equation*}
    x \wedge a
    \leq
    x \wedge (x \rightarrow y)
    =
    x \wedge \bigvee\set{ c \in X \given x \wedge c \leq y }
    \reloset {\eqref{eq:thm:complete_heyting_algebra/infinite_distributive}} =
    \bigvee\set{ x \wedge c \given x \wedge c \leq y }
    =
    y.
  \end{equation*}

  Generalizing on \( x \) and \( y \), we conclude that \( {\rightarrow} \) is indeed a relative pseudocomplement and thus \( X \) is a Heyting algebra.
\end{proof}

\begin{example}\label{ex:def:heyting_algebra}
  We list examples of \hyperref[def:heyting_algebra]{Heyting algebras}:
  \begin{thmenum}
    \thmitem{ex:def:heyting_algebra/lindenbaum_tarski} As shown in \fullref{thm:intuitionistic_lindenbaum_tarski_algebra}, every \hyperref[def:lindenbaum_tarski_algebra]{Lindenbaum-Tarski algebra} for the \hyperref[def:intuitionistic_propositional_deductive_systems]{intuitionistic propositional deduction system} is a Heyting algebra.

    \thmitem{ex:def:heyting_algebra/topology} The topology \( \mscrT \) of a \hyperref[def:topological_space]{topological space} \( (X, \mscrT) \) is a complete Heyting algebra.

    Indeed,
    \begin{itemize}
      \item \hyperref[def:lattice/join]{Arbitrary joins} are given by \hyperref[def:basic_set_operations/union]{unions}.
      \item \hyperref[def:lattice/meet]{Finite meets} are given by \hyperref[def:basic_set_operations/intersection]{intersections}.
      \item The \hyperref[def:extremal_points/top_and_bottom]{top element} is the entire domain \( L \).
      \item The \hyperref[def:extremal_points/top_and_bottom]{bottom element} is the empty set.
      \item The \hyperref[def:heyting_algebra]{relative pseudocomplement} \( U \rightsquigarrow V \) is then
      \begin{equation*}
        \bigcup\set[\Big]{ A \in T \given \underbrace{A \cap U}_{A \setminus (X \setminus U)} \subseteq V }
        =
        \bigcup\set[\Big]{ A \in T \given A \subseteq V \cup (X \setminus U) }
        =
        \Int((X \setminus U) \cup V),
      \end{equation*}
      which is actually similar to \fullref{thm:boolean_equivalences/conditional_as_disjunction} despite the fact that arbitrary topologies are not Boolean algebras.

      \item As a result, the \hyperref[def:heyting_algebra/pseudocomplement]{pseudocomplement} is
      \begin{equation*}
        \widetilde U = \Int(X \setminus U).
      \end{equation*}
    \end{itemize}

    This is actually used in topological semantics --- see \fullref{def:propositional_topological_semantics}.
  \end{thmenum}
\end{example}

\paragraph{Boolean algebra}

\begin{definition}\label{def:bounded_lattice_complement}\mcite[16]{Birkhoff1967}
  In a \hyperref[def:extremal_points/bounds]{bounded} \hyperref[def:lattice]{lattice}, a \term[ru=дополнение (\cite[def. 1.1]{Гуров2013})]{complement} of an element \( x \) is another element \( y \) such that \( x \wedge y = \bot \) and \( x \vee y = \top \).
\end{definition}

\begin{proposition}\label{thm:distributive_bounded_lattice_unique_complement}
  In a \hyperref[def:extremal_points/bounds]{bounded} \hyperref[def:distributive_lattice]{distributive lattice}, each element has at most one complement.
\end{proposition}
\begin{proof}
  If \( y \) and \( z \) are both complements of \( x \), then
  \begin{balign*}
    y
    &\reloset {\eqref{eq:thm:def:lattice/bounded_absorption/meet}} =
    y \wedge \top
    = \\ &=
    y \wedge (z \vee x)
    = \\ &\reloset {\eqref{eq:def:distributive_lattice/meet_over_join}} =
    (y \wedge z) \vee (y \wedge x)
    = \\ & =
    y \wedge z
    = \\ & =
    (x \wedge z) \vee (y \wedge z)
    = \\ &\reloset {\eqref{eq:def:distributive_lattice/meet_over_join}} =
    (x \vee y) \wedge z
    = \\ & =
    z.
  \end{balign*}
\end{proof}

\begin{definition}\label{def:boolean_algebra}\mcite[18]{Birkhoff1967}
  A \term[ru=булева алгебра (\cite[def. 1.1]{Гуров2013})]{Boolean algebra} is a \hyperref[def:extremal_points/bounds]{bounded} \hyperref[def:distributive_lattice]{distributive lattice} with an additional unary operation \( {\oline \anon} \), such that \( {\oline x} \) is a \hyperref[def:bounded_lattice_complement]{complement} of \( x \).

  Existence of the complement is provided by the operation itself, while uniqueness follows from \fullref{thm:distributive_bounded_lattice_unique_complement}.

  Boolean algebras have the following metamathematical properties:
  \begin{thmenum}[resume=def:boolean_algebra]
    \thmitem{def:boolean_algebra/theory} We extend the \hyperref[def:lattice/theory]{first-order theory of lattices} by adding:
    \begin{itemize}
      \item The nullary functional symbols \( \top \) and \( \bot \) along with the axioms \eqref{eq:thm:def:lattice/bounded_absorption/join} and \eqref{eq:thm:def:lattice/bounded_absorption/meet}.

      \item The unary functional symbol \( {\oline {\anon}} \) along with the axioms
      \begin{align}
        \xi \vee \oline \xi \syneq \top, \label{eq:def:boolean_algebra/join} \\
        \xi \wedge \oline \xi \syneq \bot. \label{eq:def:boolean_algebra/meet}
      \end{align}
    \end{itemize}

    \thmitem{def:boolean_algebra/submodel}\mcite[18]{Birkhoff1967} In addition to containing the joins and meets of all its members, a \hyperref[def:first_order_submodel]{first-order submodel} of a Boolean algebra must also contain the complement of each of its members. \fnote{Since the axioms \eqref{eq:def:boolean_algebra/join} and \eqref{eq:def:boolean_algebra/meet} hold, it is redundant to require that a Boolean subalgebra contains the top and bottom elements}. We call such submodels \term{Boolean subalgebras}.

    \thmitem{def:boolean_algebra/homomorphism} A \hyperref[def:first_order_homomorphism]{first-order homomorphism} between Boolean algebras is a lattice homomorphism that preserve top and bottom elements. Complements are automatically preserved, as we shall see in \fullref{thm:distributive_bounded_lattice_unique_complement}.

    \thmitem{def:boolean_algebra/opposite} The \hyperref[def:lattice/opposite]{dual lattice} of a Boolean algebra is again a Boolean algebra. We will call it the \term{dual Boolean algebra}.

    \thmitem{def:boolean_algebra/category} We denote the \hyperref[def:category_of_small_first_order_models]{category of \( \mscrU \)-small models} for Boolean algebras via \( \cat{Bool} \).
  \end{thmenum}
\end{definition}

\begin{example}\label{ex:def:boolean_algebra}
  We list examples of \hyperref[def:boolean_algebra]{Boolean algebras}:

  \begin{thmenum}
    \thmitem{ex:def:boolean_algebra/lindenbaum_tarski} As shown in \fullref{thm:intuitionistic_lindenbaum_tarski_algebra}, every \hyperref[def:lindenbaum_tarski_algebra]{Lindenbaum-Tarski algebra} for the \hyperref[def:classical_propositional_deductive_systems]{classical propositional deductive system} is a Boolean algebra.

    \thmitem{ex:def:boolean_algebra/f2} The \hyperref[def:finite_field]{finite field} \( \BbbF_2 = \set{ 0, 1 } \) is a Boolean algebra. Indeed, it is clearly a bounded lattice, and since it doesn't contain neither the pentagon lattice \eqref{eq:ex:def:modular_lattice/pentagon} nor the diamond lattice \eqref{eq:ex:def:distributive_lattice/diamond}, \fullref{thm:distributive_lattice_characterization} implies that \( \BbbF_2 \) is distributive. Finally, the complement operation can be defined in the obvious way --- by exchanging the two elements.

    \Fullref{thm:two_element_lattice} implies that every two-element lattice is isomorphic to \( \BbbF_2 \).

    \thmitem{ex:def:boolean_algebra/power_set} The power set of any set is a \hyperref[def:complete_lattice]{complete} Boolean algebra --- see \fullref{thm:boolean_algebra_of_subsets}.
  \end{thmenum}
\end{example}

\begin{proposition}\label{thm:def:boolean_algebra}
  \hyperref[def:boolean_algebra]{Boolean algebras} have the following basic properties:
  \begin{thmenum}
    \thmitem{thm:def:boolean_algebra/involution} Complementation is an \hyperref[def:involution]{involution}.

    \thmitem{thm:def:boolean_algebra/opposite_complement} The complementation operation in a Boolean algebra coincides with complementation in its \hyperref[def:boolean_algebra/opposite]{dual}.

    \thmitem{thm:def:boolean_algebra/heyting} Every Boolean algebra is a \hyperref[def:heyting_algebra]{Heyting algebra}. Furthermore, we have the following variation of \eqref{eq:thm:boolean_equivalences/conditional_as_disjunction}:
    \begin{equation*}
      (x \rightarrow y) \coloneqq \oline {x} \vee y.
    \end{equation*}

    \thmitem{thm:def:boolean_algebra/distributive}\mcite[lemma 162]{Gratzer2011} In a \hyperref[def:complete_lattice]{complete} Boolean algebra, for any element \( x \) and any family \( \seq{ y_k }_{k \in \mscrK} \), we have
    \begin{align}
      x \vee \parens[\Big]{ \bigwedge_{k \in \mscrK} y_k } &= \bigwedge_{k \in \mscrK} (x \vee y_k), \label{eq:thm:def:boolean_algebra/distributive/join_over_meet} \\
      x \wedge \parens[\Big]{ \bigvee_{k \in \mscrK} y_k } &= \bigvee_{k \in \mscrK} (x \wedge y_k). \label{eq:thm:def:boolean_algebra/distributive/meet_over_join}
    \end{align}
  \end{thmenum}
\end{proposition}
\begin{proof}
  \SubProofOf{thm:def:boolean_algebra/involution} We have
  \begin{equation*}
    x
    \reloset {\eqref{eq:thm:lattice_operation_characterization/compatibility/meet}} =
    x \wedge \top
    \reloset {\eqref{eq:def:boolean_algebra/join}} =
    x \wedge (\oline x \vee \doline x)
    \reloset {\eqref{eq:def:distributive_lattice/meet_over_join}} =
    (x \wedge \oline x) \vee (x \wedge \doline x)
    \reloset {\eqref{eq:def:boolean_algebra/meet}} =
    \bot \vee (x \wedge \doline x)
    \reloset {\eqref{eq:thm:lattice_operation_characterization/compatibility/join}} =
    x \wedge \doline x
    \reloset {\eqref{eq:thm:lattice_operation_characterization/compatibility/meet}} \leq
    \doline x.
  \end{equation*}

  Analogously,
  \begin{equation*}
    \doline x
    \reloset {\eqref{eq:thm:lattice_operation_characterization/compatibility/meet}} =
    \doline x \wedge \top
    \reloset {\eqref{eq:def:boolean_algebra/join}} =
    \doline x \wedge (x \vee \oline x)
    \reloset {\eqref{eq:def:distributive_lattice/meet_over_join}} =
    (\doline x \wedge x) \vee (\doline x \wedge \oline x)
    \reloset {\eqref{eq:def:boolean_algebra/meet}} =
    (\doline x \wedge x) \vee \bot
    \reloset {\eqref{eq:thm:lattice_operation_characterization/compatibility/join}} =
    \doline x \wedge x
    \reloset {\eqref{eq:thm:lattice_operation_characterization/compatibility/meet}} \leq
    x.
  \end{equation*}

  Therefore,
  \begin{equation*}
    x = \doline x
  \end{equation*}

  \SubProofOf{thm:def:boolean_algebra/opposite_complement} Note that \eqref{eq:def:boolean_algebra/join} in a Boolean algebra corresponds to \eqref{eq:def:boolean_algebra/meet} in its dual.

  \SubProofOf{thm:def:boolean_algebra/heyting} We will show that
  \begin{equation*}
    c \leq \oline {x} \vee y \T{if and only if} x \wedge c \leq y.
  \end{equation*}

  First, if \( c \leq \oline x \vee y \), we have
  \begin{equation*}
    x \wedge c
    \leq
    x \wedge (\oline x \vee y)
    \reloset {\eqref{eq:def:distributive_lattice/meet_over_join}} =
    (\underbrace{x \wedge \oline x}_{\bot}) \vee (x \wedge y)
    =
    x \wedge y
    \leq
    y.
  \end{equation*}

  Conversely, if \( x \wedge c \leq y \), we have
  \begin{equation*}
    \oline x \vee (x \wedge c) \leq \oline x \vee y,
  \end{equation*}
  which again due to distributivity implies
  \begin{equation*}
    (\underbrace{\oline x \vee x}_{\top}) \wedge (\oline x \vee c) \leq \oline x \vee y
  \end{equation*}
  and
  \begin{equation*}
    c \leq \oline x \vee c \leq \oline x \vee y.
  \end{equation*}

  \SubProofOf{thm:def:boolean_algebra/distributive} Let \( X \) be a Boolean algebra. \Fullref{thm:def:boolean_algebra/heyting} implies that it is a Heyting algebra, hence \eqref{eq:thm:def:boolean_algebra/distributive/join_over_meet} holds as a restatement of \eqref{eq:thm:complete_heyting_algebra/infinite_distributive}. Furthermore, the opposite Boolean algebra \( X^{\opcat} \) is also a Heyting algebra and \eqref{eq:thm:complete_heyting_algebra/infinite_distributive} holds in \( X^{\opcat} \), hence \eqref{eq:thm:def:boolean_algebra/distributive/join_over_meet} holds in \( X \).
\end{proof}

\begin{theorem}[De Morgan's laws]\label{thm:de_morgans_laws}
  In a \hyperref[def:boolean_algebra]{Boolean algebra}, the following hold for any finite family \( \set{ x_k }_{k \in \mscrK} \):
  \begin{align}
    \oline{\bigvee_{k \in \mscrK} x_k}   &= \bigwedge_{k \in \mscrK} \oline{x_k}, \label{eq:thm:de_morgans_laws/complement_of_join} \\
    \oline{\bigwedge_{k \in \mscrK} x_k} &= \bigvee_{k \in \mscrK} \oline{x_k}.   \label{eq:thm:de_morgans_laws/complement_of_meet}
  \end{align}

  If the algebra is a \hyperref[def:complete_lattice]{complete lattice}, \( \mscrK \) can be any family, not necessarily finite.
\end{theorem}
\begin{comments}
  \item See also the syntactic analog, \fullref{thm:boolean_equivalences/de_morgan}.
\end{comments}
\begin{proof}
  We will show that \( \bigwedge_{m \in \mscrK} \oline{x_m} \) is the complement of \( \bigvee_{k \in \mscrK} x_k \). Indeed, we have
  \begin{equation*}
    \parens*{ \bigvee_{k \in \mscrK} x_k } \vee \parens*{ \bigwedge_{m \in \mscrK} \oline{x_m} }
    \reloset {\eqref{eq:thm:def:boolean_algebra/distributive/join_over_meet}} =
    \bigwedge_{m \in \mscrK} \parens*{ \bigvee_{k \in \mscrK} x_k } \vee \oline {x_m}
    =
    \bigwedge_{m \in \mscrK} \parens*{ \bigvee_{k \neq m} x_k } \vee \underbrace{x_m \vee \oline {x_m}}_{\top}
    =
    \bigwedge_{m \in \mscrK} \top
    =
    \top
  \end{equation*}
  and,
  \begin{equation*}
    \parens*{ \bigwedge_{m \in \mscrK} \oline{x_m} } \wedge \parens*{ \bigvee_{k \in \mscrK} x_k }
    \reloset {\eqref{eq:thm:def:boolean_algebra/distributive/meet_over_join}} =
    \bigvee_{k \in \mscrK} \parens*{ \bigwedge_{m \in \mscrK} x_m } \wedge \oline {x_k}
    =
    \bigvee_{k \in \mscrK} \parens*{ \bigwedge_{m \neq k} x_m } \wedge \underbrace{x_k \wedge \oline {x_k}}_{\bot}
    =
    \bigvee_{k \in \mscrK} \bot
    =
    \bot.
  \end{equation*}

  This demonstrates \eqref{eq:thm:de_morgans_laws/complement_of_join}. We can analogously prove \eqref{eq:thm:de_morgans_laws/complement_of_meet}.
\end{proof}

\begin{theorem}[Principle of duality for Boolean algebras]\label{thm:boolean_algebra_duality}\mcite{Gottschalk1953}
  Consider the \hyperref[def:lattice/theory]{first-order theory of Boolean algebras}. Within it, consider the \hyperref[def:first_order_syntax/closed_formula]{closed formula} \( \varphi \). Denote by \( \varphi^C \) the dual formula in the sense of \fullref{thm:lattice_duality}, in which we swap all connectives --- all instances \( \vee \) and \( \wedge \), as well as \( \leq \) and \( \geq \). Denote by \( \varphi^V \) the formula obtained from \( \varphi \) by swapping each variable with its complement.

  If every Boolean algebra \hyperref[def:first_order_model]{satisfies} \( \varphi \), then every Boolean algebra also satisfies \( \varphi^C \), \( \varphi^V \) and \( \varphi^{CV} \).

  More generally, the following are equivalent for a Boolean algebra \( X \):
  \begin{TwoColumns}
    \begin{itemize}
      \item \( X \) satisfies \( \varphi \).
      \item \( X \) satisfies \( \varphi^{CV} \).
    \end{itemize}
    \BeginSecondColumn
    \begin{itemize}
      \item \( X^{\opcat} \) satisfies \( \varphi^C \).
      \item \( X^{\opcat} \) satisfies \( \varphi^V \).
    \end{itemize}
  \end{TwoColumns}
\end{theorem}
\begin{comments}
  \item Similar statements hold more generally --- see \fullref{thm:preorder_duality} and \fullref{thm:lattice_duality}.

  \item This result is richer than the aforementioned ones --- \incite{Gottschalk1953} refers to it as \enquote{quaterniality} rather than \enquote{duality}.
\end{comments}
\begin{proof}
  \Fullref{thm:def:boolean_algebra/opposite_complement} implies that complementation coincides in \( X \) and \( X^{\opcat} \), thus, if \( X \) satisfies \( \varphi \), then, as in \fullref{thm:lattice_duality}, \( X^{\opcat} \) satisfies \( \varphi^C \).

  On the other hand, a \hyperref[def:first_order_valuation/variable_assignment]{variable assignment} in \( X \) corresponds to the valuation of the complemented variables in \( X^{\opcat} \), thus \( X \) satisfies \( \varphi \) if and only if \( X^{\opcat} \) satisfies \( \varphi^V \).

  Finally, \( X \) satisfies \( \varphi \) if and only if \( X^{\opcat} \) satisfies \( \varphi^C \) if and only if \( X = (X^{\opcat})^{\opcat} \) satisfies \( \varphi^{CV} \).
\end{proof}

\paragraph{Ultrafilters}

\begin{remark}\label{rem:boolean_algebra_ideal}
  \Fullref{thm:lattice_ideal_as_semiring_ideal} demonstrates that a subset of a Boolean algebra is a \hyperref[def:lattice_ideal]{lattice ideal} (resp. filter) if and only if it is a \hyperref[def:semiring_ideal]{semiring ideal} of the \hyperref[ex:def:semiring/lattice]{join-meet semiring} (resp. meet-join semiring).
\end{remark}

\begin{proposition}\label{thm:improper_boolean_ideal}
  The only \hyperref[def:lattice_ideal]{lattice ideal} or \hyperref[def:lattice_ideal]{filter} in a \hyperref[def:boolean_algebra]{Boolean algebra} that contains both an element and its complement is the algebra itself.
\end{proposition}
\begin{proof}
  Suppose that some ideal \( I \) in \( X \) contains both \( x \) and \( \oline x \). Then \( I \) must contain their join \( \top \), and since \( I \) is closed under arbitrary meets, for every element \( y \) of the \( X \), \( I \) must contain \( y \wedge \top = y \). Therefore, \( I = X \).

  The proof for filters follows via \fullref{thm:boolean_algebra_duality}.
\end{proof}

\begin{proposition}\label{thm:boolean_prime_iff_maximal}
  A \hyperref[def:lattice_ideal]{lattice ideal} or \hyperref[def:lattice_ideal]{filter} in a \hyperref[def:boolean_algebra]{Boolean algebra} is \hyperref[def:lattice_ideal/prime]{prime} if and only if it is \hyperref[def:lattice_ideal/maximal]{maximal}.
\end{proposition}
\begin{proof}
  We will consider only ideals. The proof for filters follows via \fullref{thm:boolean_algebra_duality}.

  \SufficiencySubProof Let \( P \) be a prime ideal. \Fullref{thm:maximal_ideal_theorem} gives us a maximal ideal \( M \) containing \( I \).

  Suppose that \( M \) has some element \( x \) not in \( P \). Then \( x \wedge \oline x = \bot \) is in \( P \), thus \( \oline x \) must also be in \( P \) because the latter is prime. Then \( M \) contains both \( x \) and \( \oline x \), and \fullref{thm:improper_boolean_ideal} implies that \( M \) coincides with the ambient Boolean algebra. But \( M \) must be proper, giving us a contradiction with the existence of \( x \).

  Therefore, \( M \) and \( P \) coincide, thus \( P \) is maximal.

  \NecessitySubProof Follows from \fullref{thm:lattice_ideal_as_semiring_ideal} and \fullref{thm:def:semiring_ideal/maximal_is_prime}.
\end{proof}

\begin{definition}\label{def:ultrafilter}
  We say that a proper \hyperref[def:lattice_ideal]{filter} \( F \) in a \hyperref[def:boolean_algebra]{Boolean algebra} is an \term[bg=ултрафилтер (\cite[18]{Проданов1982}), ru=ультрафильтр (\cite[182]{Гуров2013})]{ultrafilter} if any of the following equivalent conditions hold:
  \begin{thmenum}
    \thmitem{def:ultrafilter/direct}\mcite[182]{Гуров2013} For every algebra element \( x \), either\fnote{Both \( x \in F \) and \( \oline x \in F \) cannot hold because of \fullref{thm:improper_boolean_ideal}.} \( x \in F \) or \( \oline x \in F \).

    \thmitem{def:ultrafilter/prime} \( F \) is a \hyperref[def:lattice_ideal/prime]{prime filter}.

    \thmitem{def:ultrafilter/maximal}\mcite[233]{DaveyPriestley2002} \( F \) is a \hyperref[def:lattice_ideal/maximal]{maximal filter}.
  \end{thmenum}
\end{definition}
\begin{proof}
  \ImplicationSubProof{def:ultrafilter/direct}{def:ultrafilter/prime} Suppose that, for every algebra element \( x \), either \( x \in F \) or \( \oline x \in F \).

  Let \( x \vee y \in F \). If \( x \not\in F \), then \( \oline x \in F \) and hence the following is also a member of \( F \):
  \begin{equation*}
    \oline x \vee (x \vee y)
    =
    (\oline x \vee x) \vee y
    =
    \top \vee y
    =
    y.
  \end{equation*}

  Hence, if \( x \not\in F \), then \( y \in F \).

  Since \( x \) was chosen arbitrarily, we conclude that \( F \) is a prime filter.

  \ImplicationSubProof{def:ultrafilter/prime}{def:ultrafilter/maximal} Follows from \fullref{thm:boolean_prime_iff_maximal}.

  \EquivalenceSubProof{def:ultrafilter/maximal}{def:ultrafilter/direct} Let \( F \) be a maximal ideal. Fix an arbitrary algebra element \( x \). Either \( x \) is in \( F \), or is in the complement of \( F \), in which case \fullref{thm:improper_boolean_ideal} implies that \( \oline x \) is in \( F \).
\end{proof}

\begin{definition}\label{def:principal_ultrafilter}\mcite[example 1.6.11]{Hinman2005}
  Fix an arbitrary \hyperref[def:set]{set} \( A \) and consider its \hyperref[thm:boolean_algebra_of_subsets]{Boolean algebra of subsets} \( \pow(A) \).

  For every element \( x \) of \( A \), the following family is an \hyperref[def:ultrafilter]{ultrafilter} in \( \pow(A) \):
  \begin{equation*}
    \mscrF_x \coloneqq \set{ B \subseteq A \given x \in B }.
  \end{equation*}

  We call \( \mscrF_x \) the \term{principal ultrafilter} of \( x \).
\end{definition}
\begin{comments}
  \item \incite[234]{DaveyPriestley2002} define \enquote{principal ultrafilters} to be maximal \hyperref[def:lattice_ideal/principal]{principal filters}, which notion differs from our definition.
\end{comments}

\begin{lemma}[Ultrafilter lemma]\label{thm:ultrafilter_lemma}
  Every proper \hyperref[def:lattice_ideal]{filter} in a \hyperref[def:boolean_algebra]{Boolean algebra} is contained in an \hyperref[def:ultrafilter]{ultrafilter}.
\end{lemma}
\begin{proof}
  Follows from \fullref{thm:lattice_ideal_as_semiring_ideal} and \fullref{thm:maximal_ideal_theorem}.
\end{proof}


  \section{Combinatorics}\label{sec:combinatorics}

Combinatorics originated as the study of problems related to counting. This study is now referred to as \term{enumerative combinatorics} to distinguish it from more abstract subfields. A canonical example of a counting problem is given in \fullref{ex:fibonacci_rabbits}. On the other hand, \fullref{thm:gamma_function_interpolates_factorial} provides an example of how mathematical analysis can answer questions related to counting. The latter subfield is called \term{analytic combinatorics}.

Results in combinatorics are traditionally obtained for \hyperref[def:set_finiteness]{finite sets} and \hyperref[def:integer_signum]{positive integers}, however many of them can be easily generalized. For example, \fullref{def:pigeonhole_principle} is stated in terms of cardinals rather than positive integers, and we generally impose no cardinality restriction on \hyperref[def:directed_graph]{graphs}.

Some basic definitions and theorems are stated in \fullref{subsec:enumerative_combinatorics} and \fullref{subsec:progressions}, however most of the section is concerned with graphs --- see \fullref{subsec:graphs}, \fullref{subsec:trees} and \fullref{subsec:graph_embeddings}.

As usual, in this section \( \BbbK \) will refer to either the field \( \BbbR \) of \hyperref[def:real_numbers]{real numbers} or the field \( \BbbC \) of \hyperref[def:real_numbers]{complex numbers}. This restriction is justified by \fullref{rem:real_field_extensions}.

\begin{example}\label{ex:fibonacci_rabbits}
  A simple, but nontrivial counting problem is due to Fibonacci. In \cite{MacTutor:fibonacci}, the problem is formulated as follows:
  \begin{displayquote}
    A certain man put a pair of rabbits in a place surrounded on all sides by a wall. How many pairs of rabbits can be produced from that pair in a year if it is supposed that every month each pair begets a new pair which from the second month on becomes productive?
  \end{displayquote}

  The rabbit behavior described in ridiculously idealized. Pregnancy time, for example, seems to be nonexistent in this problem. We are interested in counting, however, and not in biology. Nonetheless, the problem as it is stated is still open to interpretations. To obtain Fibonacci's result, we must add some further assumptions:
  \begin{itemize}
    \item The goal is to find the cumulative rabbit count, including the first pair.
    \item No crossbreeding is assumed --- each rabbit is monogamous and is part of a predefined pair.
    \item The rabbits do not die, which is a realistic assumption for a short time period in case the rabbits are taken care of.
    \item During the first two months, no rabbit is born. That is, the original pair must also wait for two months prior to producing offspring.
  \end{itemize}

  Instead of trying to give a direct answer to the problem, the usual approach is to iteratively build a sequence \( \seq{ F_k }_{k=1}^\infty \), the \term{Fibonacci sequence}, indicating the cumulative number of pairs of rabbits each month.

  As mentioned, during the first two months no offspring is produced, so we only have one pair of rabbits. Thus, \( F_1 = F_2 = 1 \). In general, the number of rabbits on month \( n \) is the sum of:
  \begin{itemize}
    \item The number of existing pairs on month \( n - 1 \), which is \( F_{n - 1} \).
    \item The number of newborn pairs, which is precisely the number of mature pairs (those born at least two months ago). This is \( F_{n - 2} \).
  \end{itemize}

  The entire sequence can thus be built using the following recursive definition:
  \begin{equation*}
    F_k \coloneqq \begin{cases}
      1,                &k = 1 \T{or} k = 2, \\
      F_{k-1} + F_{k-2} &k > 2.
    \end{cases}
  \end{equation*}

  Just to give an answer to Fibonacci's question, we will list the first 12 Fibonacci numbers:
  \begin{equation*}
    1, 1, 2, 3, 5, 8, 13, 21, 34, 55, 89, 144, 233
  \end{equation*}

  It should also be noted that outside the rabbit problem, the Fibonacci sequence is often defined to start at the zeroth month with a value of zero --- this is how it is defined in \fullref{rem:natural_number_recursion} actually. So
  \begin{equation}\label{eq:ex:fibonacci_rabbits}
    F_k \coloneqq \begin{cases}
      0,                &k = 0, \\
      1,                &k = 1, \\
      F_{k-1} + F_{k-2} &k > 2.
    \end{cases}
  \end{equation}
  is actually a more conventional definition.
\end{example}

\begin{definition}\label{def:labeled_set}\mcite[example I.2.2]{Aluffi2009}
  Let \( A \) be an arbitrary \hyperref[def:set]{set} in the sense of \hyperref[def:zfc]{\logic{ZFC}}. Suppose that for every member \( x \in A \) we are given a \term{label} \( l_x \). This label may be any other set (in \logic{ZFC}).

  The function \( L(x) \coloneqq l_x \) is called a \term{weighted set}. It is a function in the sense of \fullref{def:function} due to the \hyperref[def:zfc/replacement]{axiom schema of replacement}.

  We call the set \( A \) the \term{universe} of \( L \). If \( x \in A \), we say that \( x \) belongs to \( L \) and denote this by \( x \in L \), although this is not actual set membership. We denote the weight of \( x \) using \( L(x) \).

  We list several important special cases:
  \begin{thmenum}
    \thmitem{def:labeled_set/weighted} If the labels are \hyperref[def:real_numbers]{real numbers}, we may call the labels \term{weights} and \( L \) itself --- a \term{weighted set}.

    \thmitem{def:labeled_set/multiset} If the labels are \hyperref[def:cardinal]{cardinal numbers}, we may call hte labels \term{multiplicities} and \( L \) itself --- a \term{multiset}. The \term{multiset cardinality} of \( L \) is the sum of all multiplicities.

    \thmitem{def:labeled_set/fuzzy} If the weights are real numbers in \( [0, 1] \), we may call the labels \term{degrees of membership} and \( L \) itself --- a \term{fuzzy set}.
  \end{thmenum}
\end{definition}

\begin{example}\label{ex:def:labeled_set}
  We list several examples of \hyperref[def:labeled_set]{weighted sets}.

  \begin{itemize}
    \thmitem{ex:def:labeled_set/weighted_graphs} If \( G = (V, E) \) is a \hyperref[def:directed_graph]{directed graph}, either \( V \) or \( E \) may be a weighted set. Both occur frequently in applications.

    \thmitem{ex:def:labeled_set/polynomial_roots} The roots of any polynomial form a multiset with multiplicities given by \fullref{def:polynomial_root}.

    \thmitem{ex:def:labeled_set/factorization} Any \hyperref[def:irreducible_factorization]{factorization} in an \hyperref[def:integral_domain]{integral domain} produces a multiset.

    \thmitem{ex:def:labeled_set/eigenvalues} The \hyperref[def:eigenpair]{point spectrum} of a linear operator produces a multiset.
  \end{itemize}
\end{example}

\begin{remark}\label{rem:multiset_notation}
  If it is clear from the context that \( M \) is a \hyperref[def:labeled_set/multiset]{multiset}, we may write
  \begin{equation*}
    M = \set{ a, a, b, b, c }
  \end{equation*}
  to denote the function
  \begin{equation*}
    M = \set{ (a, 2), (b, 2), (c, 1) }.
  \end{equation*}

  Since the ordering of elements is irrelevant, we can also regard a multiset as an equivalence class of \hyperref[def:transfinite_sequence]{transfinite sequences}.
\end{remark}

  \section{Enumerative combinatorics}\label{sec:enumerative_combinatorics}

\paragraph{Pigeonhole principle}

\begin{concept}\label{con:pigeonhole_principle}
  The \enquote{Pigeonhole principle} refers to the following observation:
  \begin{displayquote}
    If we are given more pigeons than pigeonholes, then at least one pigeonhole must contain multiple pigeons in it.
  \end{displayquote}

  \incite*[sec. 6.2]{Rosen2019DiscreteMathematics} dedicates entire sections to nontrivial applications of this principle.

  The Pigeonhole principle is attributed to Dirichlet, and variations of it are stated in the literature under different names:
  \begin{itemize}
    \item The \enquote{pigeonhole principle} is used, among others, by \incite[289]{Knuth1997ArtVol2}, \incite[sec. 6.2]{Rosen2019DiscreteMathematics}, \incite[66]{HopcroftEtAl2001Computability}, \incite[exer. A-4.1]{Rotman2015AlgebraVol1}, \incite[134]{Enderton1977Sets} and \incite[16]{Savage2008Computability}.
    \item The \enquote{drawer principle} is used by \incite[222]{Арнольд2014ОДУ} (as \enquote{принцип ящиков Дирихле}), Goncharov in \cite[156]{АлександровМаркушевичХинчин1952ЭнциклопедияТом3} (as \enquote{принцип Дирихле}), \incite[224]{Проданов1982ФункАнализТом1} (as \enquote{принцип за чекмеджетата}).
    \item \enquote{Dirichlet's principle} is used by \incite[64]{Гуров2013Решётки} (as \enquote{принцип Дирихле}), \incite[9]{Мирчев2001Графи} (as \enquote{принцип на Дирихле}).
  \end{itemize}
\end{concept}

\begin{theorem}[Dirichlet's pigeonhole principle]\label{thm:pigeonhole_principle}
  We give several statements of the \hyperref[con:pigeonhole_principle]{pigeonhole principle} in a differing levels of generality.

  \begin{thmenum}
    \thmitem{thm:pigeonhole_principle/simple} For every function \( f: A \to B \), if \( A \) has greater cardinality, at least two members of \( A \) map to the same value in \( B \).

    \thmitem{thm:pigeonhole_principle/general} For every function \( f: A \to B \) we have
    \begin{equation}\label{eq:thm:pigeonhole_principle/general}
      \sup\set[\Big]{ \card(f^{-1}(b)) \given* b \in B } \cdot \card(\img f) \geq \card(A).
    \end{equation}

    The supremum in \eqref{eq:thm:pigeonhole_principle/general} is \( 0 \) in case \( B \) is empty.

    \thmitem{thm:pigeonhole_principle/finitary} Fix nonempty \hi{finite} sets \( A \) with \( n \) elements and \( B \) with \( m \) elements. For every function \( f: A \to B \) between them, there exists some element \( b \) in \( B \) such that \( f(a) = b \) for at least \( n / m \) elements \( a \) of \( A \).

    \thmitem{thm:pigeonhole_principle/infinitary} Fix two sets \( A \) and \( B \) such that \( A \) is infinite and \( \card(A) > \card(B) \). Then, for every function \( f: A \to B \), we have
    \begin{equation}\label{eq:thm:pigeonhole_principle/infinitary}
      \sup\set[\Big]{ \card(f^{-1}(b)) \given* b \in B } = \card(A).
    \end{equation}

    In particular, if the supremum above is attained for \( b_0 \), then \( f^{-1}(b_0) \) and \( A \) are equinumerous.
  \end{thmenum}
\end{theorem}
\begin{proof}
  \SubProofOf{thm:pigeonhole_principle/simple} It is sufficient to show that \( f: A \to B \) cannot be injective if \( \card(A) > \card(B) \). This follows easily from \fullref{thm:set_domination_relation_trichotomy}.

  \SubProofOf{thm:pigeonhole_principle/general} Denote by \( \mu \) the supremum in \eqref{eq:thm:pigeonhole_principle/general}.

  For every \( b \) in \( B \), since \( \card(f^{-1}(b)) \leq \mu \) there exists an injective function
  \begin{equation*}
    m_b: f^{-1}(b) \to \mu
  \end{equation*}

  Then we can define
  \begin{equation*}
    \begin{aligned}
      &g: A \to \mu \times \img f, \\
      &g(a) \coloneqq (m_b(a), f(a)).
    \end{aligned}
  \end{equation*}

  It is injective by construction, hence
  \begin{equation*}
    \card(A) \leq \card(\mu \times \img f) = \mu \cdot \card(\img f).
  \end{equation*}

  \SubProofOf{thm:pigeonhole_principle/finitary} \Fullref{thm:pigeonhole_principle/general} implies that
  \begin{equation*}
    \max\set[\Big]{ \card(f^{-1}(b)) \given* b \in B } \geq \frac n m.
  \end{equation*}

  Since \( B \) is finite, there exists a member attaining the maximum above.

  \SubProofOf{thm:pigeonhole_principle/infinitary} Again denote by \( \mu \) the supremum in \eqref{eq:thm:pigeonhole_principle/general}. Then
  \begin{equation*}
    \mu \cdot \card(\img f) \geq \card(A).
  \end{equation*}

  \Fullref{thm:simplified_cardinal_arithmetic/infinite} implies that
  \begin{equation*}
    \mu \cdot \card(\img f) = \max\set{ \mu, \card(\img f) }.
  \end{equation*}

  We have \( \mu > \card(\img f) \) since otherwise we would obtain
  \begin{equation*}
    \card(B) \geq \card(\img f) \geq \card(A) > \card(B),
  \end{equation*}
  which is a contradiction. Thus,
  \begin{equation*}
    \mu \geq \card(A).
  \end{equation*}

  Furthermore, since \( f^{-1}(b) \subseteq A \) for every \( b \in B \), the supremum \( \mu \) of their cardinals cannot exceed \( \card(A) \). Therefore,
  \begin{equation*}
    \card(A) = \mu.
  \end{equation*}
\end{proof}

\begin{example}\label{ex:thm:pigeonhole_principle}
  We list several examples related to \fullref{thm:pigeonhole_principle}:
  \begin{thmenum}
    \thmitem{ex:thm:pigeonhole_principle/lagrange} Given a \hyperref[def:group/homomorphism]{group homomorphism} \( \varphi: G \to H \), by \fullref{thm:group_zero_morphisms/isomorphism},
    \begin{equation*}
      G / \ker \varphi \cong \img \varphi.
    \end{equation*}

    \Fullref{thm:lagranges_subgroup_theorem} then implies that
    \begin{equation*}
      \card(\ker \varphi) \cdot \card(\img \varphi) = \card(G).
    \end{equation*}

    All cosets are equinumerous by \fullref{thm:subgroup_cosets/equinumerous}, so we can restate the above as
    \begin{equation*}
      \sup\set[\Big]{ \card(\varphi^{-h}) \given* h \in H } \cdot \card(\img \varphi) = \card(G).
    \end{equation*}

    We have thus obtained a special case of \fullref{thm:pigeonhole_principle/general} in which equality holds due to group homomorphisms being well-behaved structurally.

    \thmitem{ex:thm:pigeonhole_principle/birthdays} \Fullref{thm:pigeonhole_principle/finitary} allows us to conclude that, among \( 1000 \) people,
    \begin{itemize}
      \item At least \( 3 = \ceil(1000 / 365) \) were born on the same day.
      \item At least \( 20 = \ceil(1000 / 52) \) were born in the same week.
      \item At least \( 84 = \ceil(1000 / 12) \) were born in the same month.
    \end{itemize}

    \thmitem{ex:thm:pigeonhole_principle/boolean} Again via \fullref{thm:pigeonhole_principle/finitary} we conclude that, for every \hyperref[def:boolean_function]{predicate} \( f: A \to \set{ T, F } \) on a finite set \( A \) with \( n \) elements, either \( f \) holds for at least \( n / 2 \) members of \( A \) or \( f \) doesn't hold for (at least) \( n / 2 \) members.

    \thmitem{ex:thm:pigeonhole_principle/party} As a consequence of \fullref{ex:thm:pigeonhole_principle/boolean}, in a group of \( 6 \) people, either \( 3 \) know everybody else or \( 3 \) do not.

    Compare this example to \fullref{ex:ramsey_party_problem}, where we are interested how many people know \hi{each other}. This leads to the considerably harder problem of estimating \hyperref[def:ramsey_number]{Ramsey numbers}.
  \end{thmenum}
\end{example}

\paragraph{Inclusion-exclusion principle}

\begin{theorem}[Binary inclusion-exclusion principle]\label{thm:binary_inclusion_exclusion_principle}
  For any two sets \( A \) and \( B \) we have
  \begin{equation}\label{eq:thm:binary_inclusion_exclusion_principle/general}
    \card(A \cup B) + \card(A \cap B) = \card(A) + \card(B).
  \end{equation}

  If both are finite, then
  \begin{equation}\label{eq:thm:binary_inclusion_exclusion_principle/finite}
    \card(A \cup B) = \card(A) + \card(B) - \card(A \cap B).
  \end{equation}
\end{theorem}
\begin{proof}
  Consider the disjoint union
  \begin{equation*}
    U \coloneqq (A \cup B) \amalg (A \cap B)
  \end{equation*}
  and the (injective) functions
  \begin{equation*}
    \begin{aligned}
      &f_A: A \to U, \\
      &f_A(x) \coloneqq \iota_{A \cup B}(x).
    \end{aligned}
  \end{equation*}
  and
  \begin{equation*}
    \begin{aligned}
      &f_B: B \to U, \\
      &f_B(x) \coloneqq \begin{cases}
        \iota_{A \cap B}(x), &x \in A, \\
        \iota_{A \cup B}(x), &x \not\in A.
      \end{cases}
    \end{aligned}
  \end{equation*}

  \begin{figure}[!ht]
    \centering
    \includegraphics[align=c]{output/thm__binary_inclusion_exclusion_principle}
    \caption{The functions from our proof of \fullref{thm:binary_inclusion_exclusion_principle}}\label{fig:thm:binary_inclusion_exclusion_principle}
  \end{figure}

  We can now easily combine \( f_A \) and \( f_B \) into a bijective function from \( A \amalg B \) to \( U \), thus
  \begin{equation*}
    \card(A \amalg B) = \card(U).
  \end{equation*}

  As per our definition of cardinal addition in \fullref{def:cardinal_arithmetic/addition}, the cardinality of a disjoint union is the sum of the individual cardinalities; thus, \eqref{eq:thm:binary_inclusion_exclusion_principle/general} follows.
\end{proof}

\begin{lemma}\label{thm:inclusion_exclusion_principle_lemma}
  For any two sets \( A \) and \( B \) we have
  \begin{equation}\label{eq:thm:inclusion_exclusion_principle_lemma}
    \card(A) = \card(A \setminus B) + \card(A \cap B).
  \end{equation}
\end{lemma}
\begin{proof}
  Trivial.
\end{proof}

\begin{theorem}[Inclusion-exclusion principle]\label{thm:inclusion_exclusion_principle}
  For every finite family \( A_1, \ldots, A_n \) of \hi{finite} sets we have
  \begin{equation}\label{eq:thm:inclusion_exclusion_principle}
    \card(A_1 \cup \cdots \cup A_n) = \sum_{k=1}^n (-1)^{k - 1} \sum_{\mathclap{i_1 < \cdots < i_k}} \card(A_{i_1} \cap \cdots \cap A_{i_k}).
  \end{equation}
\end{theorem}
\begin{comments}
  \item We can generally regard \fullref{thm:binary_inclusion_exclusion_principle} as a special case, however it also handles the case of infinite sets.

  \item \incite*[195]{Stanley2012CombinatoricsVol1} uses a linear-algebraic generalization of this theorem to handle integer-valued functions other than the cardinality.
\end{comments}
\begin{proof}
  We will use induction on \( n \). The base case \( n = 1 \) is trivial. Suppose \eqref{eq:thm:inclusion_exclusion_principle} holds for all families of \( n \) (finite) sets. For the family \( A_1, \ldots, A_n, A_{n+1} \) we thus have
  \small
  \begin{align*}
    &\phantom{{}={}}
    \card(A_1 \cup \cdots \cup A_n)
    = \\ &=
    \card\parens[\Big]{ (A_1 \setminus A_{n+1}) \cup \cdots \cup (A_n \setminus A_{n+1}) \cup A_{n+1} }
    = \\ &=
    \card\parens[\Big]{ (A_1 \setminus A_{n+1}) \cup \cdots \cup (A_n \setminus A_{n+1}) } + \card(A_{n+1})
    \reloset {\T{ind.}} = \\ &=
    \sum_{k=1}^n (-1)^{k - 1} \sum_{\mathclap{i_1 < \cdots < i_k \leq n}} \card( (A_{i_1} \cap \cdots \cap A_{i_k}) \setminus A_{n+1}) + \card(A_{n+1})
    \reloset {\eqref{eq:thm:inclusion_exclusion_principle_lemma}} = \\ &=
    \sum_{k=1}^n (-1)^{k - 1} \sum_{\mathclap{i_1 < \cdots < i_k \leq n}} \parens[\Big]{ \card(A_{i_1} \cap \cdots \cap A_{i_k}) - \card(A_{i_1} \cap \cdots \cap A_{i_k} \cap A_{n+1}) } + \card(A_{n+1})
    = \\ &=
    \sum_{k=1}^n (-1)^{k - 1} \sum_{\mathclap{i_1 < \cdots < i_k \leq n + 1}} \card(A_{i_1} \cap \cdots \cap A_{i_k}).
  \end{align*}
  \normalsize
\end{proof}

\begin{corollary}\label{thm:inclusion_exclusion_eratosthenes}\mcite[\S 8.6.3]{Rosen2019DiscreteMathematics}
  Denote by \( p_1, p_2, \ldots \) the sequence of all \hyperref[def:prime_number]{prime numbers}. Then the \hyperref[def:prime_counting_function]{prime counting function} can be expressed as
  \begin{equation}\label{eq:thm:inclusion_exclusion_eratosthenes}
    \pi(n) = (n - 1) + \sum_{m=1}^{\mathclap{\pi(\op{isqrt}(n))}} (-1)^m \sum_{\mathclap{i_1 < \cdots < i_m}} \floor\parens[\Big]{ \frac n {p_{i_1} \cdots p_{i_m}} } + \pi(\op{isqrt}(n)).
  \end{equation}
\end{corollary}
\begin{proof}
  We will adapt the reasoning of \fullref{alg:sieve_of_eratosthenes} so that it can be used with \fullref{thm:inclusion_exclusion_principle}.

  Fix a positive integer \( n \), let \( N \coloneqq \set{ 1, 2, \ldots, n } \) and, for a positive integer \( k \), let \( A_k \coloneqq k\BbbZ \cap N \) be the subset of multiples of \( k \). Finally, let \( P \) be the set of prime numbers less than or equal to \( \op{isqrt}(n) \).

  \Fullref{thm:prime_number_sqrt_prime} implies that \( m \in N \setminus \set{ 1 } \) is composite if and only if there exists some prime \( p \leq \op{isqrt}(m) \) such that \( m \in A_p \). We can extend \( p \) to range up to \( \op{isqrt}(n) \), however we must then exclude the possibility that \( m \) is itself prime, in which case \( m \in A_m \) without \( m \) being composite.

  Thus, the composite numbers in \( N \) are
  \begin{equation*}
    C \coloneqq \set{ 1 } \cup \parens[\Big]{ \parens[\Big]{ \bigcup_{p \in P} A_p } \setminus P }.
  \end{equation*}

  Then
  \begin{equation*}
    \pi(n) = \card(N \setminus C).
  \end{equation*}

  From \fullref{thm:inclusion_exclusion_principle_lemma} it follows that
  \begin{equation}\label{eq:thm:inclusion_exclusion_eratosthenes/sets}
    \pi(n) = \underbrace{\card N}_n - 1 - \card\parens[\Big]{ \bigcup_{p \in P} A_p } + \underbrace{\card P}_{\pi(\op{isqrt}(n))}.
  \end{equation}

  It remains to apply \fullref{thm:inclusion_exclusion_principle}. \Fullref{thm:natural_number_divisibility_lattice/join} implies that
  \begin{equation*}
    A_{k_1} \cap \cdots \cap A_{k_m} = A_{\lcm(k_1, \ldots,k_m)},
  \end{equation*}
  which for prime numbers becomes
  \begin{equation*}
    A_{p_1} \cap \cdots \cap A_{p_m} = A_{p_1 \cdots p_m}.
  \end{equation*}

  Finally, note that
  \begin{equation*}
    \card(A_k) = \floor\parens[\Big]{ \frac n k }.
  \end{equation*}

  Then
  \begin{equation*}
    \card\parens[\Big]{ \bigcup_{p \in P} A_p } = \sum_{m=1}^{\mathclap{\pi(\op{isqrt}(n))}} (-1)^m \sum_{\mathclap{i_1 < \cdots < i_m}} \floor\parens[\Big]{ \frac n {p_{i_1} \cdots p_{i_m}} }
  \end{equation*}
  and, after substituting in \eqref{eq:thm:inclusion_exclusion_eratosthenes/sets}, we obtain \eqref{eq:thm:inclusion_exclusion_eratosthenes}.
\end{proof}

\begin{corollary}\label{thm:inclusion_exclusion_totient}
  Let \( n = p_1^{a_1} \cdots p_r^{a_r} \) be the prime factorization of \( n \). Then \hyperref[def:eulers_totient_function]{Euler's totient function} can be expressed as
  \begin{equation}\label{eq:thm:inclusion_exclusion_totient}
    \varphi(n) = n \cdot \prod_{k=1}^r \parens[\Big]{ 1 - \frac 1 {p_k} }.
  \end{equation}
\end{corollary}
\begin{proof}
  Let \( N \) and \( A_k \) be defined as in \fullref{thm:inclusion_exclusion_eratosthenes}. Then
  \begin{equation*}
    \varphi(n) = \card N - \card\parens[\Big]{ \bigcup_{k=1}^r A_{p_k} }.
  \end{equation*}

  \Fullref{thm:inclusion_exclusion_principle} implies that
  \begin{equation}\label{eq:thm:inclusion_exclusion_totient/intermediate}
    \varphi(n) = n + \sum_{m=1}^r (-1)^m \sum_{i_1 < \cdots < i_m} \frac n {p_{i_1} \cdots p_{i_m}}.
  \end{equation}

  We now use induction on \( r \).
  \begin{itemize}
    \item If \( r = 1 \), then
    \begin{equation*}
      \varphi(n) = n + (-1) \frac n {p_1} = n \parens[\Big]{ 1 - \frac 1 {p_1} }.
    \end{equation*}

    \item Suppose that \eqref{eq:thm:inclusion_exclusion_totient} holds for all positive integers whose prime factorization has \( r \) terms. Let \( n = p_1^{a_1} \cdots p_r^{a_r} p_{r+1}^{a_{r+1}} \). Then
    \begin{align*}
      \varphi(n)
      &\reloset{\eqref{eq:thm:inclusion_exclusion_totient/intermediate}} =
      n + \sum_{m=1}^{r+1} (-1)^m \sum_{i_1 < \cdots < i_m} \frac n {p_{i_1} \cdots p_{i_m}}
      = \\ &=
      n + \sum_{m=1}^r \underbrace{(-1)^m} \sum_{i_1 < \cdots < i_m \leq r} \frac n {p_{i_1} \cdots p_{i_m}} + \\ &\quad\quad + \frac n {p_{r+1}} + \sum_{m=1}^r \underbrace{(-1)^{m+1}} \sum_{i_1 < \cdots < i_m \leq r} \frac n {p_{i_1} \cdots p_{i_m}} \cdot \frac 1 {p_{r+1}}
      = \\ &=
      \parens[\Big]{ 1 - \frac 1 {p_{r+1}} } \cdot \parens[\Big]{ n + \sum_{m=1}^r (-1)^m \sum_{i_1 < \cdots < i_m \leq r} \frac n {p_{i_1} \cdots p_{i_m}} }
      \reloset {\T{ind.}} = \\ &=
      \parens[\Big]{ 1 - \frac 1 {p_{r+1}} } \cdot n \cdot \prod_{k=1}^r \parens[\Big]{ 1 - \frac 1 {p_k} }.
    \end{align*}
  \end{itemize}
\end{proof}

\begin{example}\label{ex:thm:inclusion_exclusion_principle}
  We list several examples related to \fullref{thm:inclusion_exclusion_principle} and its corollaries:
  \begin{thmenum}
    \thmitem{ex:thm:inclusion_exclusion_principle/eratosthenes} As discussed in \fullref{thm:inclusion_exclusion_eratosthenes}, we can count the number of primes less than or equal to \( n \) using only our knowledge of much smaller primes.

    For example, let \( n = 25 \). The primes less than or equal to \( \op{isqrt}(n) = 5 \) are \( 2 \), \( 3 \) and \( 5 \), and thus
    \begin{align*}
      \pi(25) &= 25 - 1 - \parens[\Big]{ \floor\parens[\Big]{ \frac 2 {25} } - \floor\parens[\Big]{ \frac 3 {25} } - \floor\parens[\Big]{ \frac 5 {25} } } + \\
              &+ \parens[\Big]{ \floor\parens[\Big]{ \frac 6 {25} } - \floor\parens[\Big]{ \frac {10} {25} } - \floor\parens[\Big]{ \frac {15} {25} } } + \\
              &- \floor\parens[\Big]{ \frac {30} {25} } - 3
              = \\ &= 25 - 1 - (12 + 8 + 5) + (4 + 2 + 1) - 0 + 3
              = \\ &= 9
    \end{align*}

    Indeed,
    \begin{equation*}
      \begin{array}{r *{23}{c}}
        1  & \hi{2} & \hi{3} & 4       & \hi{5} & 6       & \hi{7} & 8  & 9  & 10      & \hi{11} & 12 & \hi{13} \\
        14 & 15     & 16     & \hi{17} & 18     & \hi{19} & 20     & 21 & 22 & \hi{23} & 24      & 25 &
      \end{array}
    \end{equation*}

    \thmitem{ex:thm:inclusion_exclusion_principle/totient} We can use \fullref{thm:inclusion_exclusion_totient} to compute the totient function of a number if we know its prime factors. For example,
    \begin{equation*}
      \varphi(25) = 25 \cdot \parens[\Big]{ 1 - \frac 1 5 } = \frac {25 \cdot 4} 5 = 20.
    \end{equation*}
  \end{thmenum}
\end{example}

\paragraph{Factorial}

\begin{definition}\label{def:factorial}\mcite[46]{Knuth1997ArtVol1}
  We define the \term[bg=факториел (\cite[129]{Тагамлицки1971Диф}), ru=факториал (\cite[145]{АлександровМаркушевичХинчин1952ЭнциклопедияТом3})]{factorial} of a \hyperref[def:integer_signum]{nonnegative integer} \( n \) recursively as follows:
  \begin{equation*}
    n! \coloneqq \begin{cases}
      1,          &n = 0, \\
      (n - 1)! n, &n > 0.
    \end{cases}
  \end{equation*}
\end{definition}

\begin{proposition}\label{thm:gamma_function_interpolates_factorial}
  For every \hyperref[def:integer_signum]{nonnegative integer} \( n \) we have
  \begin{equation*}
    \Gamma(n + 1) \coloneqq n!,
  \end{equation*}
  where \( \Gamma \) is the Gamma function defined in \fullref{def:gamma_function}.
\end{proposition}
\begin{proof}
  We use induction on \( n \).
  \begin{itemize}
    \item If \( n = 0 \), then
    \begin{equation*}
      \Gamma(1)
      =
      \int_0^\infty x^0 e^{-x} \dl x
      =
      -e^{-x}\restr_{x=0}^\infty
      =
      -\underbrace{\lim_{x \to \infty} e^{-x}}_{0} + 1
      =
      1
      =
      0!
    \end{equation*}

    \item If \( n > 0 \) and \( \Gamma(n) = (n - 1)! \), then
    \begin{balign*}
      \Gamma(n + 1)
      &=
      \int_0^\infty x^n \cdot e^{-x} \dl x
      = \\ &=
      \underbrace{(- x^n e^{-x})\restr_{x=0}^\infty}_{-(0 - 0)} + n \int_0^\infty e^{-x} x^{n-1} \dl x
      = \\ &=
      n \Gamma(n)
      = \\ &=
      n (n - 1)!
      = \\ &=
      n!
    \end{balign*}
  \end{itemize}
\end{proof}

\begin{theorem}[Stirling's factorial approximation]\label{thm:stirlings_factorial_approximation}
  For the factorial function we have
  \begin{equation*}
    n! = \sqrt{2 \pi n} \cdot \parens*{ \frac n e }^n \cdot e^{\mu(n)},
  \end{equation*}
  where \( \mu \), defined by \eqref{eq:thm:stirlings_gamma_approximation/mu}, satisfies
  \begin{equation*}
    0 < \mu(n) < \frac 1 {12n}.
  \end{equation*}
\end{theorem}
\begin{proof}
  Follows from \fullref{thm:gamma_function_interpolates_factorial} and \fullref{thm:stirlings_gamma_approximation}.
\end{proof}

\paragraph{Binomial coefficients}

\begin{definition}\label{def:binomial_coefficient}\mcite[53]{Knuth1997ArtVol1}
  We define the \term[ru=биномиальный коеффициент (\cite[100]{БелоусовТкачёв2004ДискретнаяМатематика})]{binomial coefficient} of the \hyperref[def:integer_signum]{nonnegative integers} \( n \geq k \) as the number
  \begin{equation*}
    \binom n k \coloneqq \frac {n!} {k!(n-k)!},
  \end{equation*}
\end{definition}
\begin{comments}
  \item Binomial coefficients are motivated by \fullref{thm:binomial_theorem}.
\end{comments}

\begin{theorem}[Pascal's binomial recurrence]\label{thm:pascals_binomial_recurrence}
  For nonnegative integers \( 1 < k < n \), \hyperref[def:binomial_coefficient]{binomial coefficients} have the following property:
  \begin{equation}\label{eq:thm:pascals_binomial_recurrence}
    \binom n k = \binom {n - 1} k + \binom {n - 1} {k - 1}.
  \end{equation}
\end{theorem}
\begin{comments}
  \item \incite[thm. 6.4.2]{Rosen2019DiscreteMathematics} calls this recurrence \enquote{Pascal's identity} after Pascal because of its relation to \hyperref[con:pascals_triangle]{Pascal's triangle}. \incite[56]{Knuth1997ArtVol1} calls it the \enquote{addition formula} for binomial coefficients.
\end{comments}
\begin{proof}
  \begin{balign*}
    \binom {n - 1} k + \binom {n - 1} {k - 1}
    &=
    \frac {(n - 1)!} {k! (n - 1 - k)!} + \frac {(n - 1)!} {(k - 1)! (n - k)!}
    = \\ &=
    \frac {(n - 1)!} {(k - 1)! (n - 1 - k)!} \bracks*{ \frac 1 k + \frac 1 {n - k} }
    = \\ &=
    \frac {(n - 1)!} {(k - 1)! (n - 1 - k)!} \frac n {k(n - k)}
    = \\ &=
    \frac {n!} {k! (n - k)!}
    = \\ &=
    \binom n k.
  \end{balign*}
\end{proof}

\begin{theorem}[Newton's binomial theorem]\label{thm:binomial_theorem}
  If, in some \hyperref[def:semiring]{semiring}, the members \( x \) and \( y \) commute (i.e. \( xy = yx \)), then
  \begin{equation}\label{eq:thm:binomial_theorem}
    (x + y)^n = \sum_{k=0}^n \binom n k x^k y^{n-k}.
  \end{equation}
\end{theorem}
\begin{proof}
  We use induction on \( n \). For \( n = 0 \), the theorem trivially holds. Assume that the theorem holds for \( 1, \ldots, n \). Then
  \begin{balign*}
    (x + y)^{n+1}
     & =
    x (x + y)^n + y (x + y)^n
    = \\ &=
    \sum_{k=0}^n \binom n k x^{k+1} y^{n-k} + y \sum_{k=0}^n \binom n k x^k y^{n-k}
    = \\ &=
    x^{n+1} + y \sum_{k=0}^{n-1} \binom n k x^{k+1} y^{n-(k+1)} + y \sum_{k=0}^n \binom n k x^k y^{n-k}
    = \\ &=
    x^{n+1} + y \parens*{ \sum_{k=1}^n \binom n {k-1} x^k y^{n-k} + y^n \sum_{k=1}^n \binom n k x^k y^{n-k} } + y^{n+1}
    = \\ &\reloset {\eqref{eq:thm:pascals_binomial_recurrence}} =
    x^{n+1} + y \sum_{k=1}^n \binom {n+1} k x^k y^{n-k} + y^{n+1}
    = \\ &=
    \sum_{k=0}^n \binom {n+1} k x^k y^{(n+1)-k}.
  \end{balign*}
\end{proof}

\begin{theorem}[Vandermonde convolution]\label{thm:vandermonde_convolution}\mcite[59]{Knuth1997ArtVol1}
  For nonnegative integers \( n \), \( m \) and \( k \leq n + m \), we have
  \begin{equation}\label{eq:thm:vandermonde_convolution}
    \binom {n + m} k = \sum_{\substack{i+j=k \\ i \leq n \T*{and} j \leq m}} \binom n i \cdot \binom m j.
  \end{equation}
\end{theorem}
\begin{proof}
  Consider the polynomial \( X + 1 \) over any commutative ring.

  By \fullref{thm:binomial_theorem}, we have
  \begin{equation*}
    (X + 1)^{n+m} = \sum_{k=0}^{n+m} x^k 1^{n+m-k}.
  \end{equation*}

  Using the \hyperref[def:semigroup_algebra]{convolution product} instead, we obtain
  \begin{equation*}
    (X + 1)^n \cdot (X + 1)^m = \sum_{k=0}^{n+m} \parens*{ \sum_{\substack{i+j=k \\ i \leq n \T*{and} j \leq m}} \binom n i \cdot \binom m j } x^k.
  \end{equation*}

  Since the two are equal, we obtain \eqref{eq:thm:vandermonde_convolution}.
\end{proof}

\begin{concept}\label{con:pascals_triangle}\mcite[fig. 6.4.1]{Rosen2019DiscreteMathematics}
  \term[ru=треугольник Паскаля (\cite[\S 5.3.4]{Новиков2013ДискретнаяМатематика})]{Pascal's triangle} is a colloquial name for schematic triangles a-la \cref{fig:con:pascals_triangle}, in which every number is the sum of the two numbers above, with ones on the boundary.

  \begin{figure}[!ht]
    \centering
    \includegraphics[page=1]{output/con__pascals_triangle}
    \caption{\hyperref[con:pascals_triangle]{Pascal's triangle}.}\label{fig:con:pascals_triangle}
  \end{figure}

  By \enquote{schematic triangle} we mean that every object showing the same recurrence can again be called \enquote{Pascal's triangle}. For example, \incite[54]{Knuth1997ArtVol1} and \incite[179]{Яблонский2003ДискретнаяМатематика} use the term to refer to \hyperref[def:pascal_matrix/lower]{lower triangular Pascal matrices}.
\end{concept}

\begin{definition}\label{def:pascal_matrix}\mcite[1]{EdelmanStrang2018PascalMatrices}
  We will define three families of square matrices, which we will collectively call \term{Pascal matrices}.
  \begin{thmenum}
    \thmitem{def:pascal_matrix/lower} The \hyperref[def:triangular_matrix]{lower triangular} matrix \( L_n = \seq{ l_{i,j} }_{i,j=1}^n \), where
    \begin{equation*}
      l_{i,j} \coloneqq \begin{cases}
        0,                       &j > i, \\
        1,                       &j = 1, \\
        1,                       &j = i, \\
        l_{i-1,j} + l_{i-1,j-1}, &1 < j < i.
      \end{cases}
    \end{equation*}

    \thmitem{def:pascal_matrix/upper} Its \hyperref[def:transpose_matrix]{transpose}, the \hyperref[def:triangular_matrix]{upper triangular} matrix \( U_n = \seq{ u_{i,j} }_{i,j=1}^n \).

    \thmitem{def:pascal_matrix/symmetric} The symmetric matrix \( S_n = \seq{ s_{i,j} }_{i,j=1}^n \), where
    \begin{equation*}
      s_{i,j} \coloneqq \begin{cases}
        1,                     &i = 1, \\
        1,                     &j = 1, \\
        s_{i-1,j} + s_{i,j-1}, &i > 1 \T{and} j > 1.
      \end{cases}
    \end{equation*}
  \end{thmenum}
\end{definition}
\begin{comments}
  \item These matrices in \identifier{combinatorics.binomial} in \cite{notebook:code}.
\end{comments}

\begin{example}\label{ex:con:pascals_triangle}
  We give examples of \hyperref[def:pascal_matrix]{Pascal matrices}:
  \begin{equation*}
    L_7
    =
    \begin{pmatrix}
      1        &          &           &        &        &      &   \\
      \fbox{1} & \fbox{1} &           &        &        &      &   \\
      1        & \fbox{2} & 1         &        &        &      &   \\
      1        & 3        & 3         & 1      &        &      &   \\
      1        & 4        & 6         & 4      & 1      &      &   \\
      1        & \fbox{5} & \fbox{10} & 10     & 5      & 1    &   \\
      1        & 6        & \fbox{15} & 20     & 15     & 6    & 1
    \end{pmatrix}
  \end{equation*}
  \begin{equation*}
    U_7
    =
    \begin{pmatrix}
      1        & \fbox{1} & 1         & 1      & 1      & 1         & 1         \\
               & \fbox{1} & \fbox{2}  & 3      & 4      & \fbox{5}  & 6         \\
               &          & 1         & 3      & 6      & \fbox{10} & \fbox{15} \\
               &          &           & 1      & 4      & 10        & 20        \\
               &          &           &        & 1      & 5         & 15        \\
               &          &           &        &        & 1         & 6         \\
               &          &           &        &        &           & 1
    \end{pmatrix}
  \end{equation*}
  \begin{equation*}
    S_7
    =
    \begin{pmatrix}
      1        & \fbox{1} & 1         & 1      & 1      & 1      & 1   \\
      \fbox{1} & \fbox{2} & 3         & 4      & 5      & 6      & 7   \\
      1        & 3        & 6         & 10     & 15     & 21     & 28  \\
      1        & 4        & \fbox{10} & 20     & 35     & 56     & 84  \\
      1        & \fbox{5} & \fbox{15} & 35     & 70     & 126    & 210 \\
      1        & 6        & 21        & 56     & 126    & 252    & 462 \\
      1        & 7        & 28        & 84     & 210    & 462    & 924
    \end{pmatrix}
  \end{equation*}
\end{example}

\begin{proposition}\label{thm:pascal_matrix_binomial}
  We have the following relation between \hyperref[def:binomial_coefficient]{Binomial coefficients} and elements of \hyperref[def:pascal_matrix]{Pascal matrices}:
  \begin{align*}
    l_{i,j} &= \binom i j \thickspace\T{if}\thickspace i \leq j, \\
    u_{i,j} &= \binom j i \thickspace\T{if}\thickspace i \geq j, \\
    s_{i,j} &= \binom {i + j} j = \binom {i + j} i = \frac {(i + j)!} {i! j!}. \\
  \end{align*}
\end{proposition}
\begin{proof}
  Follows from \fullref{thm:pascals_binomial_recurrence}.
\end{proof}

\begin{proposition}\label{thm:pascal_matrix_product}
  For \hyperref[def:pascal_matrix]{Pascal matrices}, we have \( S_n = L_n \cdot U_n \).
\end{proposition}
\begin{proof}
  The \( (i, j) \)-th entry of \( L_n \cdot U_n \) is
  \begin{equation*}
    \sum_{k=1}^n l_{i,k} \cdot u_{k,j}
    =
    \sum_{k=1}^n l_{i,k} \cdot l_{j,k}
    =
    \sum_{k=1}^{\min\set{ i, j }} \binom i k \cdot \binom j k
    \reloset {\eqref{eq:thm:vandermonde_convolution}} =
    \binom {i + j} i
    =
    s_{i,j}.
  \end{equation*}
\end{proof}

  \subsection{Progressions}\label{subsec:progressions}

Progressions are an elementary concept that happens to be useful quite often. There is no definition of progression, but rather the term \enquote{progression} refers to specific recursively defined \hyperref[def:sequence]{sequences}.

\begin{definition}\label{def:arithmetic_progression}
  The \term{arithmetic progression} with \term{base} \( a_0 \) and \term{difference} \( d \) is the sequence
  \begin{equation}\label{eq:def:arithmetic_progression}
    a_k \coloneqq \begin{cases}
      a_0,         & k = 0, \\
      a_{k-1} + d, & k > 0.
    \end{cases}
  \end{equation}

  Clearly every index \( k \geq 0 \) we have the closed form representation \( a_k = a_0 + kd \).
\end{definition}

\begin{proposition}\label{thm:arithmetic_progression_partial_sums}
  The \hyperref[def:convergent_series]{series} constructed from the arithmetic progression \eqref{eq:def:arithmetic_progression} has partial sums
  \begin{equation}\label{eq:thm:arithmetic_progression_partial_sums}
    \sum_{k=0}^n a_k = \frac {(n + 1) (a_n - a_0)} 2.
  \end{equation}

  In the special case where \( a_0 = 0 \) and \( d = 1 \), this reduces to
  \begin{equation}\label{eq:thm:arithmetic_progression_partial_sums/integers}
    \sum_{k=0}^n k = \sum_{k=1}^n k = \frac {n (n + 1)} 2.
  \end{equation}
\end{proposition}
\begin{proof}
  \begin{balign*}
    2 \sum_{k=0}^n a_k
     & =
    2 \sum_{k=0}^n (a_0 + kd)
    =    \\ &=
    \sum_{k=0}^n (a_0 + kd) + \sum_{k=0}^n (a_0 + (n-k)d)
    =    \\ &=
    \sum_{k=0}^n (2 a_0 + nd)
    =    \\ &=
    (n + 1) (a_0 + a_n).
  \end{balign*}
\end{proof}

\begin{definition}\label{def:geometric_progression}
  The \term{geometric progression} with \term{base} \( a_0 \) and \term{denominator} \( q \) is the sequence
  \begin{equation}\label{eq:def:geometric_progression}
    a_k \coloneqq \begin{cases}
      a_0,       & k = 0, \\
      a_{k-1} q, & k > 0.
    \end{cases}
  \end{equation}

  Clearly every index \( k \geq 0 \) we have the closed form representation \( a_k = a_0 q^k \).

  The \hyperref[def:convergent_series]{series}
  \begin{equation}\label{eq:def:geometric_progression/series}
    \sum_{k=0}^\infty a_k = a_0 \sum_{k=0}^\infty q^k.
  \end{equation}
  is called the \term{geometric series} for \( q \). Without loss of generality, we will assume \( a_0 = 1 \) when speaking about geometric series.
\end{definition}

\begin{proposition}\label{thm:geometric_series_properties}
  The geometric series \eqref{eq:def:geometric_progression/series} has the following basic properties:
  \begin{thmenum}
    \thmitem{thm:geometric_series_properties/finite_sum} For all \( q \in \BbbC \setminus \{ 1 \} \), the geometric series \eqref{eq:def:geometric_progression/series} has partial sums
    \begin{equation}\label{thm:geometric_progression/partial_sum}
      \sum_{k=0}^n q^k = \frac {1 - q^{n+1}} {1 - q}.
    \end{equation}

    Compare this to \fullref{thm:xn_minus_yn_factorization}.

    \thmitem{thm:geometric_series_properties/degenerate} In the degenerate case \( q = 1 \), the progression itself is constant, and its partial sums are instead
    \begin{equation}\label{thm:geometric_progression/degenerate}
      \sum_{k=0}^n q^k = n + 1.
    \end{equation}

    \thmitem{thm:geometric_series_properties/series_sum_exterior} For \( \abs{q} \geq 1 \), the geometric series diverges.

    \thmitem{thm:geometric_series_properties/series_sum_interior} For \( 0 < \abs{q} < 1 \), the geometric series converges absolutely with sum
    \begin{equation}\label{thm:geometric_progression/series_sum}
      \sum_{k=0}^\infty q^k = \frac 1 {1 - q}.
    \end{equation}
  \end{thmenum}
\end{proposition}
\begin{proof}
  \SubProofOf{thm:geometric_series_properties/finite_sum} Follows from \fullref{thm:xn_minus_yn_factorization}.
  \SubProofOf{thm:geometric_series_properties/degenerate} Obvious.

  \SubProofOf{thm:geometric_series_properties/series_sum_exterior} For \( q = 1 \), \fullref{thm:geometric_series_properties/degenerate} implies that the series diverges because it grows indefinitely. If \( \abs{q} = 1 \) and \( q \neq 1 \), the integer powers \( q^k \) are rotations around the complex plane unit circle, which do not tend to a limit. Hence, the series diverges again.

  When \( \abs{q} > 1 \), \( \abs{q^n} \) grows indefinitely with \( n \), and it follows that
  \begin{equation*}\label{thm:geometric_progression/cauchy_partial_sum}
    \sum_{k=m}^n q^k
    =
    q^m \sum_{k=0}^{n-m} q^k
    =
    q^m \frac {1 - q^{n-m+1}} {1 - q}
    =
    \frac {q^m - q^{n+1}} {1 - q}.
  \end{equation*}
  can get arbitrarily large. Therefore, in this case the series also diverges.

  \SubProofOf{thm:geometric_series_properties/series_sum_interior} Fix \( q \in B(0, 1) \). Since only \( q^{n + 1} \) depends on \( n \) in \eqref{thm:geometric_progression/partial_sum}, we obtain \eqref{thm:geometric_progression/series_sum} by simply noting that \( q^n \to 0 \) when \( n \to \infty \).
\end{proof}

\begin{example}\label{ex:n_ary_decomposition}
  A simple but important practical example of a \hyperref[eq:def:geometric_progression/series]{geometric series} is
  \begin{equation}\label{eq:ex:n_ary_decomposition/binary}
    \sum_{k=0}^\infty \frac 1 {2^k} = \frac 1 {1 - \sfrac 1 2} = 2.
  \end{equation}

  Note that if the series starts at \( k = 1 \) instead of \( k = 0 \), it sums to \( 1 \). This is often applied in analysis indirectly via \fullref{thm:continuous_function_series_powers_of_two}.

  Another application of \eqref{eq:ex:n_ary_decomposition/binary} is showing that \( 0.\overline{1} = 2 \) in the binary number system. More generally, for the \( n \)-ary number system we have
  \begin{equation}\label{eq:ex:n_ary_decomposition/general}
    \sum_{k=0}^\infty \parens*{ \frac {n-1} n }^k = \frac 1 {1 - \ifrac {(n-1)} n} = n.
  \end{equation}
\end{example}

\begin{remark}\label{rem:progressions_and_interest}
  In this example we exploit the equivalence between the closed form representations in \fullref{def:arithmetic_progression} and \fullref{def:geometric_progression} and the corresponding inductive definitions. The equivalences are obvious from a mathematical standpoint, however outside of mathematics they have highly nontrivial consequences. Indeed, they highlight the difference between simple interest and compound interest.

  As an example, a savings account with \( 1000\$ \) with a simple monthly interest of \( 2\% \) will earn \( 240\$ \) over a year:
  \begin{equation*}
    1000 (1 + 12 \cdot \sfrac 2 {100}) = 1240.
  \end{equation*}

  The same account with a compound interest of \( 2\% \) will earn a bit more - about \( 268\$ \):
  \begin{equation*}
    1000 (1 + \sfrac 2 {100})^{12} \approx 1268.24.
  \end{equation*}

  Over the course of ten years, however, simple interest will earn a total of \( 2400\$ \), while compound interest will earn \( \approx 9765\$ \).

  The difference between linear and exponential growth appears staggering in a real world situation even though the difference may not be very noticeable short-term.
\end{remark}

\begin{definition}\label{def:harmonic_progression}
  The \term{harmonic progression} with \term{base} \( a_0 \) and \term{difference} \( d \) is the sequence
  \begin{equation}\label{eq:def:harmonic_progression}
    a_k \coloneqq \frac 1 {a_0 + kd}.
  \end{equation}

  That is, each term is the reciprocal of the corresponding term in an \hyperref[def:arithmetic_progression]{arithmetic progression} with the same base and difference. In order for \eqref{eq:def:harmonic_progression} to be well-defined, either
  \begin{itemize}
    \item \( d = 0 \) and \( a_0 \neq 0 \), which turns \eqref{eq:def:harmonic_progression} into the constant sequence \( \seq{ \sfrac 1 {a_0} }_{k=0}^\infty \).
    \item \( d \neq 0 \), in which case
    \begin{equation*}
      a_k = \frac d {\sfrac {a_0} d + k}.
    \end{equation*}

    Thus, if \( d \neq 0 \), \( \ifrac {a_0} d \) must not be a negative integer unless we are satisfied with only the first \( -\ifrac {a_0} d \) terms of the progression existing.
  \end{itemize}

  Furthermore, the series may only start at \( k = 0 \) if \( a_0 \neq 0 \).

  For series related to harmonic progressions, see \fullref{ex:harmonic_series}
\end{definition}

\begin{remark}\label{rem:harmonic_progression_recursive_form}
  Unlike \fullref{def:arithmetic_progression} and \fullref{def:geometric_progression}, we have defined the harmonic progressions via closed-form expressions. Indeed, the equivalent inductive definition is more awkward to work with:
  \begin{equation*}
    a_k \coloneqq \begin{cases}
      \ifrac 1 {a_0},                    & k = 0, \text{ only defined if } a_0 \neq 0, \\
      \ifrac 1 {a_0 + d},                & k = 1,                                      \\
      \ifrac 1 {\sfrac 1 {a_{k-1}} + d}, & k > 0.
    \end{cases}
  \end{equation*}
\end{remark}

  \subsection{Graphs}\label{subsec:graphs}

\begin{example}\label{ex:konigsberg_bridges}
  A puzzle problem regarding the seven K\"onigsberg bridges was solved by Leonhard Euler in 1735 and published six years later as \cite{Euler1741}. In the English translation of the paper by \incite[ch. 1A]{BiggsLloydWilson1986}, the puzzle is described as follows in \S 2:
  \begin{displayquote}
    The problem, which I am told is widely known, is as follows: in K\"onigsberg in Prussia, there is an island \( A \). called the Kneiphof; the river which surrounds it is divided into two branches, as can be seen from \cref{fig:ex:konigsberg_bridges/schematic/drawing}, and these branches are crossed by seven bridges, \( a \), \( b \), \( c \), \( d \), \( e \), \( f \) and \( g \). Concerning these bridges, it was asked whether anyone could arrange a route in such a way that he would cross each bridge once and only once. I was told that some people asserted that this was impossible, while others were in doubt; but nobody would actually assert that it could be done. From this, I have formulated the general problem: whatever be the arrangement and division of the river into branches, and however many bridges there be, can one find out whether or not it is possible to cross each bridge exactly once?
  \end{displayquote}

  \begin{figure}[ht!]
    \frame{\includegraphics[width=\textwidth]{images/ex__konigsberg_bridges__illustration}}
    \caption{An illustration of K\"onigsberg by Matth\"aus Merian published in 1652 in \cite{MerianKönigsbergBridges}.}
    \label{fig:ex:konigsberg_bridges/illustration}
  \end{figure}

  This is considered to be the first paper in graph theory\fnote{The early history of graph theory can be found in \cite{BiggsLloydWilson1986}}. In \S 1 of his paper, Euler himself hints at the geometric nature of the problem\fnote{Topology did not exist at the time, so the problem could not have possibly been considered topological.}:
  \begin{displayquote}
    In addition to that branch of geometry which is concerned with magnitudes, and which has always received the greatest attention, there is another branch, previously almost unknown, which Leibniz first mentioned, calling it the geometry of position. This branch is concerned only with the determination of position and its properties; it does not involve measurements, nor calculations made with them. It has not yet been satisfactorily determined what kind of problems are relevant to this geometry of position, or what methods should be used in solving them. Hence, when a problem was recently mentioned, which seemed geometrical but was so constructed that it did not require the measurement of distances, nor did calculation help at all, I had no doubt that it was concerned with the geometry of position — especially as its solution involved only position, and no calculation was of any use. I have therefore decided to give here the method which I have found for solving this kind of problem, as an example of the geometry of position.
  \end{displayquote}

  To solve the puzzle, Euler denotes four pieces of land by \( A \), \( B \), \( C \) and \( D \) and considers sequences of adjacent pieces of land. This corresponds roughly to some modern definitions for both \enquote{walk} and \enquote{path} --- see the discussions in \fullref{rem:graph_walk_terminology}. Later, in \S 15, he even considers alternating sequences of pieces of land and bridges, which exactly corresponds to finite walks as defined in \fullref{def:graph_walk}. We will formalize the requirement of crossing each bridge exactly once by introducing Eulerian walks in \fullref{def:eulerian_walk}.

  \begin{figure}[ht!]
    \begin{subcaptionblock}{0.55\textwidth}
      \centering
      \includegraphics[width=\textwidth]{images/ex__konigsberg_bridges__schematic__drawing}
      \caption{The first figure from Euler's paper \cite{Euler1741}.}
      \label{fig:ex:konigsberg_bridges/schematic/drawing}
    \end{subcaptionblock}
    \hfill
    \begin{subcaptionblock}{0.4\textwidth}
      \centering
      \begin{equation}\label{eq:fig:ex:konigsberg_bridges/schematic/graph}
        \includegraphics{output/ex__konigsberg_bridges}
      \end{equation}
      \caption{The corresponding \hyperref[def:undirected_multigraph]{undirected multigraph}.}
      \label{fig:ex:konigsberg_bridges/schematic/graph}
    \end{subcaptionblock}

    \caption{Schematic drawings of the \hyperref[ex:konigsberg_bridges]{K\"onigsberg bridges puzzle}.}\label{fig:ex:konigsberg_bridges/schematic}
  \end{figure}

  Using sequences of letters, Euler manages to reformulate the puzzle in \S 7:
  \begin{displayquote}
    The problem is therefore reduced to finding a sequence of eight letters, formed from the four letters \( A \), \( B \), \( C \) and \( D \), in which the various pairs of letters occur the required number of times. Before I turn to the problem of finding such a sequence, it would be useful to find out whether or not it is even possible to arrange the letters in this way, for if it were possible to show that there is no such arrangement, then any work directed towards finding it would be wasted. I have therefore tried to find a rule which will be useful in this case, and in others, for determining whether or not such an arrangement can exist.
  \end{displayquote}

  Euler then deduces \fullref{thm:odd_degree_vertices} in \S 16 and \S 17 and further proceeds to deduce the following rule in \S 20:
  \begin{displayquote}
    If there are more than two areas to which an odd number of bridges lead, then such a journey is impossible.

    If, however, the number of bridges is odd for exactly two areas, then the journey is possible if it starts in either of these areas.

    If, finally, there are no areas to which an odd number of bridges leads, then the required journey can be accomplished starting from any area
  \end{displayquote}

  Therefore, the K\"onigsberg bridge puzzle is resolved negatively --- no route exists that can cross each bridge in \cref{fig:ex:konigsberg_bridges/schematic/drawing} exactly once. In modern terminology, the corresponding \hyperref[def:undirected_multigraph]{undirected multigraph} \eqref{eq:fig:ex:konigsberg_bridges/schematic/graph} has no \hyperref[def:walk/closed]{closed} \hyperref[def:eulerian_walk]{Eulerian walk} because it all of its vertices have odd \hyperref[def:graph_cardinality/undirected_degree]{degree}.

  Euler's general result is formulated with modern rigor and modern terminology and is proved in \fullref{thm:eulers_theorem_for_graphs}.
\end{example}

\paragraph{Four kinds of graphs}\hfill

The term \enquote{graph} is unfortunately very ambiguous\fnote{The graph of a relation or function is unrelated to the combinatorial graphs discussed here.} - it is a set of vertices connected by either directed arcs or undirected edges. \incite[362]{Knuth1997Vol1} states the following:
\begin{displayquote}
  Unfortunately, there will probably never be a standard terminology in this field, and so the author has followed the usual practice of contemporary books on graph theory, namely to use words that are similar but not identical to the terms used in any other books on graph theory.
\end{displayquote}

 We introduce distinct definitions for the four types of graphs from \cref{fig:def:graph_functors}, along with comments on how the definitions are used by different authors, and then in \fullref{def:graph_functors} we will define functors that help transparently transform some types of graphs into others. It is an established convention to not go through these hoops and implicitly transfer concepts between different kinds of graphs without even mentioning morphisms and categories. We will later follow this convention, but will nonetheless first describe the details.

\begin{figure}[!ht]
  \caption{Functors between the categories of different kinds of graphs.}\label{fig:def:graph_functors}
  \smallskip
  \hfill
  \begin{tikzpicture}
    \node[align=center] (dm) at (0, 0) {\hyperref[def:directed_graph/category]{Simple} \\ \hyperref[def:directed_graph/category]{directed} \\ \hyperref[def:directed_graph/category]{graphs}};
    \node[align=center] (um) at (5cm, 0) {\hyperref[def:undirected_graph/category]{Simple} \\ \hyperref[def:undirected_graph/category]{undirected} \\ \hyperref[def:undirected_graph/category]{graphs}};
    \node[align=center] (ds) at (0, -3cm) {\hyperref[def:directed_multigraph/category]{Directed} \\ \hyperref[def:directed_multigraph/category]{multigraphs}};
    \node[align=center] (us) at (5cm, -3cm) {\hyperref[def:undirected_multigraph/category]{Undirected} \\ \hyperref[def:undirected_multigraph/category]{multigraphs}};
    \draw[->] (dm) to node[midway, above] {\hyperref[def:graph_functors/multi_forgetful]{\( U_S \)}} (um);
    \draw[->, bend left] (um) to node[midway, below] {\hyperref[def:graph_functors/simple_doubling]{\( D_S \)}} (dm);
    \draw[->] (ds) to node[midway, below] {\hyperref[def:graph_functors/multi_forgetful]{\( U_M \)}} (us);
    \draw[->, bend left] (dm) to node[midway, right] {\hyperref[def:graph_functors/directed_forgetful]{\( U_D \)}} (ds);
    \draw[->, bend left] (ds) to node[midway, left] {\hyperref[def:graph_functors/directed_inclusion]{\( I_D \)}} (dm);
    \draw[->, bend left] (um) to node[midway, right] {\hyperref[def:graph_functors/undirected_forgetful]{\( U_U \)}} (us);
    \draw[->, bend left] (us) to node[midway, left] {\hyperref[def:graph_functors/undirected_inclusion]{\( I_U \)}} (um);
  \end{tikzpicture}
  \hfill\hfill
\end{figure}

\begin{definition}\label{def:directed_multigraph}\mcite[def. 1.1.1; def. 1.1.2]{Knauer2011}
  A \term[bg=ориентиран (\cite[6]{Мирчев2001}) мултиграф (\cite[7]{Мирчев2001}), ru=ориентированый мультиграф (\cite[16]{Емеличев1990})]{directed multigraph} \( G \) consists of the following:
  \begin{thmenum}[series=def:directed_multigraph]
    \thmitem{def:directed_multigraph/vertices} A set \( V_G \), whose elements we call \term[bg=върхове (\cite[6]{Мирчев2001}), ru=вершины (\cite[279]{Емеличев1990})]{vertices}.

    \thmitem{def:directed_multigraph/arcs} A disjoint from \( V_G \) set \( A_G \), whose elements we call \term[bg=дъги (\cite[6]{Мирчев2001}), ru=дуги (\cite[279]{Емеличев1990})]{arcs}.
    \thmitem{def:directed_multigraph/head} A function \( h_G: A \to V \), giving the \term[en=head (\cite[544]{Rosen1999})]{head} or \term[bg=начален връх (\cite[7]{Мирчев2001}), ru=начало (\cite[279]{Емеличев1990}), en=initial vertex (\cite[28]{Diestel2005})]{initial endpoint} of an arc.

    \thmitem{def:directed_multigraph/tail} A function \( t_G: A \to V \), giving the \term[en=tail (\cite[544]{Rosen1999})]{tail} or \term[bg=краен връх (\cite[7]{Мирчев2001}), ru=конец (\cite[279]{Емеличев1990}), en=terminal vertex (\cite[28]{Diestel2005})]{terminal endpoint} of an arc.
  \end{thmenum}

  \Cref{fig:def:directed_multigraph} illustrates this definition. The figure is not merely illustrative --- it is a \hyperref[def:graph_geometric_realization/embedding]{graph embedding}.

  \begin{figure}[!ht]
    \begin{equation}\label{eq:fig:def:directed_multigraph}
      \begin{aligned}
        \includegraphics[page=1]{output/def__directed_multigraph}
      \end{aligned}
    \end{equation}
    \caption{A \hyperref[def:directed_multigraph]{directed multigraph} with a pair of parallel arcs, a pair of oppositely directed arcs and a loop. Removing the dashed arcs makes it a \hyperref[def:directed_graph]{simple directed graph}.}\label{fig:def:directed_multigraph}
  \end{figure}

  We will need the following basic notions:
  \begin{thmenum}[resume=def:directed_multigraph]
    \thmitem{def:directed_multigraph/loop} We call the arc \( e \) a \term[bg=примка (\cite[7]{Мирчев2001}), ru=петля (\cite[279]{Емеличев1990})]{loop} if \( h(e) = t(e) \).

    \medskip

    \thmitem{def:directed_multigraph/parallel}\mcite[28]{Diestel2005} We call the arcs \( e \) and \( f \) \term[bg=паралелни (ребра) (\cite[7]{Мирчев2001}), ru=параллельные (рёбра) (\cite[279]{Емеличев1990})]{parallel} if \( h(e) = h(f) \) and \( t(e) = t(f) \) and \term{opposite} if it is not a loop and \( h(e) = t(f) \) and \( t(e) = h(f) \).

    \thmitem{def:directed_multigraph/homomorphism}\mcite[def. 1.4.1]{Knauer2011} A \term{homomorphism} between directed multigraphs \( G \) and \( H \) is a pair of functions
    \begin{align*}
      &f_V: V_G \to V_H, \\
      &f_A: A_G \to A_H,
    \end{align*}
    such that
    \begin{subequations}
      \begin{align}
        h_H \bincirc f_A &= f_V \bincirc h_G, \label{eq:def:directed_multigraph/homomorphism/head} \\
        t_H \bincirc f_A &= f_V \bincirc t_G. \label{eq:def:directed_multigraph/homomorphism/tail}
      \end{align}
    \end{subequations}

    Isomorphisms are discussed in \fullref{thm:graph_isomorphisms/multi_directed}.

    \thmitem{def:directed_multigraph/category}\mcite[48]{MacLane1998} Given a Grothendieck universe \( \mscrU \), we can define the \hyperref[def:category]{category} of \( \mscrU \)-small directed multigraphs as follows:
    \begin{itemize}
      \item The \hyperref[def:category/objects]{set of objects} is the family of directed multigraphs, where both the set of vertices and the set of arcs are \( \mscrU \)-small.

      \item The \hyperref[def:category/morphisms]{morphisms} between two directed multigraphs are their homomorphisms as defined in \fullref{def:directed_multigraph/homomorphism}.

      \item The \hyperref[def:category/composition]{composition of the morphisms} \( (f_V, f_A): G \to H \) and \( (g_V, g_A): H \to K \) is their componentwise function composition
      \begin{equation*}
        (g_V \bincirc f_V, g_A \bincirc f_A): G \to K.
      \end{equation*}

      \item The \hyperref[def:category/identity]{identity morphism} on the directed multigraph \( G \) is \( (\id_{V_G}, \id_{A_G}) \).
    \end{itemize}

    \thmitem{def:directed_multigraph/subgraph}\mcite[3]{Diestel2005} We say that \( H \) is a \term{subgraph} of \( G \) if \( V_G \subseteq V_H \), \( A_G \subseteq A_H \) and \( h_H \) and \( t_H \) are restrictions of \( h_G \) and \( t_G \) to \( A_H \).
  \end{thmenum}
\end{definition}
\begin{comments}
  \item Formally, we define a directed multigraph as a quadruple \( G = (V_G, A_G, h_G, t_G) \). We may skip indices when unnecessary, i.e. we are free to write \( G = (V, A, h, t) \) or even \( H = (W, B, j, u) \).

  \item \incite[def. 1.1.1]{Knauer2011}, \incite[10]{MacLane1998} and \incite[28]{Diestel2005} provide definitions similar to ours, but call them \enquote{directed graphs}. Knauer later distinguishes between simple and multigraphs, while Diestel uses \enquote{oriented graph} for what we call \enquote{simple directed graph}. \incite[8]{Bollobas1998} briefly mentions directed multigraphs and calls them as such.

  Diestel and Bollobas reserve the term \enquote{multigraph} for \hyperref[def:undirected_multigraph]{undirected multigraphs}. The latter also implicitly restricts graphs to finite order. These two books use \hyperref[def:undirected_graph]{simple undirected graphs} almost exclusively.
\end{comments}

\begin{remark}\label{rem:digraph}
  The term \enquote{digraph} is used as a shorthand for \enquote{directed graph}, for example by \incite[10]{Harary1969}, \incite[def. 1.1.1]{Knauer2011}, \incite[372]{Knuth1997Vol1} and \incite[559]{Rosen1999}. \incite[28]{Diestel2005} also uses this terminology, and in his usage \enquote{graph} exclusively refers to \enquote{undirected graph}.

  Similarly, in Russian, \enquote{орграф} is used as a shorthand for \enquote{ориентированный граф}, for example by \incite[\S 7.1.5]{Новиков2013} or \incite[279]{Емеличев1990}.

  We avoid the term.
\end{remark}

\begin{remark}\label{rem:subgraphs_and_subobjects}
  Subgraphs are not \hyperref[def:subobject_and_quotient]{categorical subobjects} because a graph can have distinct \hyperref[thm:graph_isomorphism]{isomorphic} subgraphs. For example, all one-vertex subgraphs are isomorphic, but nonetheless we consider them to be different subgraphs.
\end{remark}

\begin{definition}\label{def:directed_graph}\mcite[def. 1.1.2]{Knauer2011}
  A \term{simple directed graph} \( G \) consists of the following:
  \begin{thmenum}[series=def:directed_graph]
    \thmitem{def:directed_graph/vertices} A set \( V_G \), whose elements we call \term{vertices}.
    \thmitem{def:directed_graph/arcs} A disjoint from \( V_G \) set \( A_G \) of \hyperref[def:cartesian_product/kuratowski_pair]{ordered pairs} of \hi{distinct} vertices. As in the case of directed multigraphs, we call the elements of \( A_G \) \term{arcs}. If we allow vertices whose endpoints coincide, we will say that \( G \) is a \enquote{simple directed graph, possibly with loops}.
  \end{thmenum}

  Simple directed graphs have the following metamathematical properties:
  \begin{thmenum}[resume=def:directed_graph]
    \thmitem{def:directed_graph/theory}\mimprovised We can regard directed graphs as models of a \hyperref[def:first_order_theory]{first-order theory} over a \hyperref[def:first_order_signature]{first-order signature} \( \Sigma \) with a single \hyperref[rem:first_order_formula_conventions/infix]{infix} predicate symbol, possibly with the irreflexivity axiom \eqref{eq:def:binary_relation/irreflexive}.

    \thmitem{def:directed_graph/homomorphism}\mimprovised A \hyperref[def:first_order_homomorphism]{first-order homomorphism} between the \( G \) and \( H \) is a function \( f: V_G \to V_H \) such that \( (u, v) \in A_G \) implies \( (f(u), f(v)) \in A_H \).

    Isomorphisms are discussed in \fullref{thm:graph_isomorphisms/simple_directed}.

    \thmitem{def:directed_graph/category}\mimprovised Given a Grothendieck universe \( \mscrU \), we can define the \hyperref[def:category]{category} of \( \mscrU \)-small simple directed graphs, possibly with loops, and their homomorphisms. Disallowing loops leads to a full subcategory.

    \thmitem{def:directed_graph/subgraph}\mimprovised We say that \( H \) is a \term{subgraph} of \( G \) if \( V_G \subseteq V_H \) and \( A_G \subseteq A_H \).
  \end{thmenum}
\end{definition}
\begin{comments}
  \item We can regard directed graphs as directed multigraphs without loops and parallel arcs. This is made precise via the inclusion functor \hyperref[def:graph_functors/directed_inclusion]{\( I_D \)}.
\end{comments}

\begin{definition}\label{def:undirected_multigraph}\mcite[def. 1.1.2]{Knauer2011}
  An \term{undirected multigraph} \( G \) consists of the following:
  \begin{thmenum}[series=def:undirected_multigraph]
    \thmitem{def:undirected_multigraph/vertices} A set \( V_G \), whose elements we call \term{vertices}.
    \thmitem{def:undirected_multigraph/edges} A disjoint from \( V_G \) set \( E_G \), whose elements we call \term[bg=ребра (\cite[6]{Мирчев2001}), ru=рёбра (\cite[277]{БелоусовТкачёв2004})]{edges}.
    \thmitem{def:undirected_multigraph/endpoints} A map
    \begin{equation*}
      \mscrE: E \to \set[\Big]{ \set{ u, v } \given* u, v \in V },
    \end{equation*}
    giving an unordered pair of \term{endpoints} of an edge.
  \end{thmenum}

  \begin{figure}[!ht]
    \begin{equation}\label{eq:fig:def:undirected_multigraph}
      \begin{aligned}
        \includegraphics[page=1]{output/def__undirected_multigraph}
      \end{aligned}
    \end{equation}
    \caption{An undirected multigraph, which becomes simple after removing the dashed edges.}\label{fig:def:undirected_multigraph}
  \end{figure}

  We will use the following basic terminology:
  \begin{thmenum}[resume=def:undirected_multigraph]
    \thmitem{def:undirected_multigraph/loop} We call the edge \( e \) a \term{loop} if \( \mscrE(e) \) is a one-element set.

    \medskip

    \thmitem{def:undirected_multigraph/parallel} We call the edges \( e \) and \( f \) \term{parallel} if \( \mscrE(e) = \mscrE(f) \).

    \thmitem{def:undirected_multigraph/homomorphism}\mimprovised A \term{homomorphism} between the undirected multigraphs \( G \) and \( H \) is a pair of functions
    \begin{align*}
      &f_V: V_G \to V_H, \\
      &f_E: E_G \to E_H,
    \end{align*}
    such that, for each edge \( e \in E_G \),
    \begin{equation}\label{eq:def:undirected_multigraph/homomorphism}
      \mscrE_H(f_E(e)) = \set[\Big]{ f_V(v) \given* v \in \mscrE_G(e) }.
    \end{equation}

    Isomorphisms are discussed in \fullref{thm:graph_isomorphisms/multi_undirected}.

    \thmitem{def:undirected_multigraph/category}\mimprovised Given a Grothendieck universe \( \mscrU \), we can define the \hyperref[def:category]{category} of \( \mscrU \)-small undirected multigraphs and their homomorphisms.

    \thmitem{def:undirected_multigraph/subgraph}\mimprovised We say that \( H \) is a \term{subgraph} of \( G \) if \( V_G \subseteq V_H \), \( E_G \subseteq E_H \) and \( \mscrE_H = \mscrE_G\restr_{E_H} \).
  \end{thmenum}
\end{definition}
\begin{comments}
  \item \incite[def. 1.1.2]{Knauer2011} does not explicitly mention undirected multigraphs, but their existence is implied.

  \incite[10]{Harary1969}, \incite[28]{Diestel2005} and \incite[6]{Bollobas1998} define undirected multigraphs, although all omit the prefix \enquote{undirected}. Harary disallows loops, instead referring to multigraphs with loops as \enquote{pseudographs}. Bollobas explicitly defines directed multigraphs, while \incite[28]{Diestel2005} uses the term \enquote{directed graph} for what we call a \hyperref[def:directed_multigraph]{directed multigraph} and \enquote{oriented graph} for what we call \enquote{simple directed graph}.

  \incite[3]{GondranMinoux1984Graphs} define multigraphs without specifying whether they are directed or not, but later implicitly assume that they are undirected.
\end{comments}

\begin{definition}\label{def:undirected_graph}\mcite[3]{Diestel2005}
  A \term{simple undirected graph} \( G \) consists of the following:
  \begin{thmenum}[series=def:undirected_graph]
    \thmitem{def:undirected_graph/vertices} A set \( V_G \), whose elements we call \term{vertices}.
    \thmitem{def:undirected_graph/edges} A disjoint from \( V_G \) set \( E_G \) of unordered pairs of \hi{distinct} vertices, i.e. sets of the form \( \set{ u, v } \), whose elements we call \term{edges}. If we allow vertices whose endpoints coincide, we will say that \( G \) is a \enquote{simple undirected graph, possibly with loops}.
  \end{thmenum}

  We will use the following basic terminology:
  \begin{thmenum}[resume=def:undirected_graph]
    \thmitem{def:undirected_graph/homomorphism}\mcite[def. 1.4.3]{Knauer2011} A \term{homomorphism} between the simple undirected graphs \( G \) and \( H \) is a function \( f: V_G \to V_H \) such that \( \set{ u, v } \in E_G \) implies \( \set{ f(u), f(v) } \in E_H \).

    Isomorphisms are discussed in \fullref{thm:graph_isomorphisms/simple_undirected}.

    \thmitem{def:undirected_graph/category}\mcite[example 3.1.12]{Knauer2011} Given a Grothendieck universe \( \mscrU \), we can define the \hyperref[def:category]{category} of \( \mscrU \)-small simple undirected graphs, possibly with loops, and their homomorphisms. Disallowing loops leads to a full subcategory.

    By \fullref{thm:edgeless_graph_universal_property}, the monomorphisms are precisely the injective homomorphisms and, by \fullref{thm:complete_graph_universal_property}, the epimorphisms are precisely the surjective homomorphisms.

    \thmitem{def:undirected_graph/subgraph}\mcite[3]{Diestel2005} We say that \( H \) is a \term{subgraph} of \( G \) if \( V_G \subseteq V_H \) and \( E_G \subseteq E_H \).
  \end{thmenum}
\end{definition}
\begin{comments}
  \item We can regard undirected graphs as undirected multigraphs without loops and parallel arcs. This is made precise via the inclusion functor \hyperref[def:graph_functors/undirected_inclusion]{\( I_U \)}.

  \item \incite[9]{Harary1969}, \incite[2]{Diestel2005} and \incite[1]{Bollobas1998} define simple undirected graphs similarly to how we have done it. \incite[def. 1.1.2]{Knauer2011} also does, however he allows loops.

  \item \incite[def. 1.4.8]{Knauer2011} define subgraphs more generally as graphs which can be embedded via injective graph homomorphisms. We prefer the vertices and edges of subgraphs to be subsets.
\end{comments}

\begin{remark}\label{rem:theory_of_simple_undirected_graphs}
  For the purpose of fitting \hyperref[def:undirected_graph]{simple undirected graphs} in the framework of first-order logic, we can extend the \hyperref[def:directed_graph/theory]{theory of directed graphs} with the symmetry axiom \eqref{eq:def:binary_relation/symmetry}.

  This leads to an incompatible notion of homomorphisms, however --- the \hyperref[def:first_order_homomorphism]{first-order homomorphisms} of this theory are the directed graph homomorphisms defined in \fullref{def:directed_graph/homomorphism} and not the more general undirected graph homomorphisms defined in \fullref{def:undirected_graph/homomorphism}.

  We will still find this theory useful, for example when defining quotient graphs in \fullref{def:quotient_graph}.
\end{remark}

\begin{remark}\label{rem:simple_graphs}
  Whether or simple graphs are allowed have loops again depends on the authors.

  Defining \enquote{directed graphs} as pairs \( (V, A) \), where \( A \subseteq V \times V \), without further restrictions, is common. It is done by \incite[10]{Harary1969}, \incite[def. 1.1.2]{Knauer2011}, \incite[21]{GondranMinoux1984Graphs}, \incite[10]{Savage1998}, \incite[190]{Erickson2019}, \incite[279]{Емеличев1990}, \incite[277]{БелоусовТкачёв2004} and \incite[6]{Мирчев2001}. \incite[\S 7.1.5]{Новиков2013} also provides the same definition, however he forbids loops unless explicitly mentioned. \incite[39]{Diestel2005} uses the term \enquote{oriented graph} for what we call \enquote{simple directed graph}.

  For undirected graphs we instead have several conventions:
  \begin{itemize}
    \item Some of the aforementioned authors, namely \incite[21]{GondranMinoux1984Graphs}, \incite[10]{Savage1998} and \incite[277]{БелоусовТкачёв2004}, define directed graphs as \enquote{symmetric} undirected graphs, in which \( (u, v) \) is an arc whenever \( (v, u) \) is. This definition allows loops since the directed counterparts do.

    \item Other aforementioned authors, namely \incite[def. 1.1.2]{Knauer2011}, \incite[190]{Erickson2019}, and \incite[7]{Мирчев2001} define undirected graphs as pairs \( (V, E) \), where \( E \) consists of one-element or two-element subsets of \( V \). This definition again allows loops.

    \item \incite[9]{Harary1969}, \incite[2]{Diestel2005}, \incite[1]{Bollobas1998}, \incite[9]{Емеличев1990} and \incite[\S 7.1.2]{Новиков2013} define directed graphs as pairs \( (V, E) \), where \( E \) consists of two-element subsets of \( V \). This explicitly forbid loops.

    \item \incite[377]{Knuth1997Vol1} explicitly forbids loops in graphs, however he defines \enquote{graphs} as \enquote{a set of points together with a set of lines}.
  \end{itemize}

  \incite[def. 1.1.2]{Knauer2011} uses the adjective \enquote{simple} to refer to (multi)graphs without multiple edges, without referring to loops. So do \incite[2]{Diestel2005} and \incite[22]{Bollobas1998}, however their definitions of graphs forbid loops by default. \incite[22]{GondranMinoux1984Graphs}, \incite[191]{Erickson2019}, \incite[540]{Rosen1999}, \incite[17]{Емеличев1990} and \incite[12]{Мирчев2001} use \enquote{simple} to additionally forbid loops.

  We prefer our terminology to be as unambiguous as possible, for which reason we use \enquote{simple} in the latter sense and, if loops are allowed, we mention it explicitly.
\end{remark}

\begin{proposition}\label{thm:graph_isomorphisms}
  In order to discuss graph isomorphisms systematically, we study \hyperref[def:morphism_invertibility/isomorphism]{categorical isomorphisms} on their respective categories.

  \begin{thmenum}
    \thmitem{thm:graph_isomorphisms/multi_directed} Given \hyperref[def:directed_multigraph]{directed multigraphs} \( G \) and \( H \), the \hyperref[def:directed_multigraph/homomorphism]{directed multigraph homomorphism}
    \begin{align*}
      &f_V: V_G \to V_H, \\
      &f_A: A_G \to A_H,
    \end{align*}
    is a categorical isomorphism if and only if both \( f_V \) and \( f_A \) are bijective.

    Furthermore, the componentwise inverse \( (f_V^{-1}, f_A^{-1}) \) of \( (f_V, f_A) \) is also a homomorphism, and it is the two-sided inverse of \( (f_V, f_A) \) in the category of directed multigraphs.

    \thmitem{thm:graph_isomorphisms/simple_directed} Given \hyperref[def:directed_graph]{simple directed graphs} \( G \) and \( H \), possibly with loops, the \hyperref[def:directed_graph/homomorphism]{directed graph homomorphism} \( f: V_G \to V_H \) is a categorical isomorphism if and only if it is bijective and satisfies
    \begin{equation}\label{eq:thm:graph_isomorphisms/simple_directed}
      (u, v) \in A_G \T{if and only if} (f(u), f(v)) \in A_H.
    \end{equation}

    Furthermore, the inverse function \( f^{-1} \) of \( f \) is also a homomorphism, it satisfies \eqref{eq:thm:graph_isomorphisms/simple_directed}, and it is the two-sided inverse of \( f \) in the category of simple directed graphs.

    \thmitem{thm:graph_isomorphisms/multi_undirected} Given \hyperref[def:undirected_multigraph]{undirected multigraphs} \( G \) and \( H \), the \hyperref[def:undirected_multigraph/homomorphism]{undirected multigraph homomorphism}
    \begin{align*}
      &f_V: V_G \to V_H, \\
      &f_E: E_G \to E_H,
    \end{align*}
    is a categorical isomorphism if and only if both \( f_V \) and \( f_E \) are bijective.

    Furthermore, the componentwise inverse \( (f_V^{-1}, f_E^{-1}) \) of \( (f_V, f_E) \) is also a homomorphism, and it is the two-sided inverse of \( (f_V, f_E) \) in the category of directed multigraphs.

    \thmitem{thm:graph_isomorphisms/simple_undirected} Given \hyperref[def:undirected_graph]{simple undirected graphs} \( G \) and \( H \), possibly with loops, the \hyperref[def:undirected_graph/homomorphism]{undirected graph homomorphism} \( f: V_G \to V_H \) is a categorical isomorphism if and only if it is bijective and satisfies
    \begin{equation}\label{eq:thm:graph_isomorphisms/simple_undirected}
      \set{ u, v } \in E_G \T{if and only if} \set{ f(u), f(v) } \in E_H.
    \end{equation}
  \end{thmenum}
\end{proposition}
\begin{comments}
  \item \incite[583]{Rosen1999} uses our characterizations as definitions. \incite[3]{Diestel2005}, \incite[3]{Bollobas1998}, \incite[def. 4.1.3]{Knauer2011}, \incite[13]{Емеличев1990} and \incite[def. 5.14]{БелоусовТкачёв2004}, \incite[\S 7.1.6]{Новиков2013} define simple undirected graph isomorphisms via \eqref{eq:thm:graph_isomorphisms/simple_undirected}, however avoid discussing isomorphisms for other kinds of graphs.
\end{comments}
\begin{proof}
  \SubProofOf{thm:graph_isomorphisms/multi_directed}

  \SufficiencySubProof* Suppose that \( (f_V, f_A): G \to H \) is a categorical isomorphism. Clearly it is a homomorphism. We must show that both \( f_V \) and \( f_A \) are bijective.

  We have two forgetful functors into \( \cat{Set} \) --- one for vertices and functions between them, the other one for arcs and functions between them. Hence, by \fullref{thm:def:functor_invertibility/preserves_inverses}, both \( f_V \) and \( f_A \) are set isomorphisms, and, by \fullref{thm:function_invertibility_categorical/fully_invertible}, both are bijective.

  \NecessitySubProof* Suppose that \( (f_V, f_A): G \to H \) is a homomorphism and that both \( f_V \) and \( f_A \) are bijective. We will show that \( (f_V^{-1}, f_A^{-1}) \) is a homomorphism from \( H \) to \( G \).

  For any arc \( e \) in \( H \), we have
  \begin{equation*}
    f_V(h_G(f^{-1}_A(e)))
    \reloset {\eqref{eq:def:directed_multigraph/homomorphism/head}} =
    h_H(f_A(f^{-1}_A(e)))
    =
    h_H(e),
  \end{equation*}
  hence
  \begin{equation*}
    h_G(f^{-1}_A(e))
    =
    f_V^{-1}(f_V(h_G(f^{-1}_A(e))))
    =
    f_V^{-1}(h_H(e)).
  \end{equation*}

  We can analogously show that
  \begin{equation*}
    t_G(f^{-1}_A(e)) = f_V^{-1}(t_H(e)).
  \end{equation*}

  Thus, the pair \( (f_V^{-1},f_A^{-1}) \) satisfies both \eqref{eq:def:directed_multigraph/homomorphism/head} and \eqref{eq:def:directed_multigraph/homomorphism/tail}, which makes it a directed multigraph homomorphism.

  Composing \( (f_V, f_A) \) and \( (f_V^{-1}, f_A^{-1}) \) (componentwise), we obtain identity homomorphisms on either \( G \) or \( H \) depending on the order of composition. Therefore, \( (f_V, f_A) \) is fully invertible, that is, a categorical isomorphism.

  \SubProofOf{thm:graph_isomorphisms/simple_directed}

  \SufficiencySubProof* Suppose that \( f: V_G \to V_H \) is a categorical isomorphism between \( G \) and \( H \). Again, by \fullref{thm:def:functor_invertibility/preserves_inverses}, \( f \) is an isomorphism between \( V_G \) and \( V_H \) in the category of sets, that is, a bijective function.

  Let \( g: V_G \to V_H \) be the categorical inverse of \( f \). Then it is the inverse of \( f \) in \( \cat{Set} \), which due to the uniqueness of two-sided inverses implies that \( g \) is the inverse function \( f^{-1} \) of \( f \).

  Then, if \( (f(u), f(v)) \) is an arc of \( H \) for some vertices \( u \) and \( v \),
  \begin{equation*}
    \parens[\Big]{ f^{-1}(f(u)), f^{-1}(f(v)) } = (u, v)
  \end{equation*}
  is an edge of \( G \). Therefore, \eqref{eq:thm:graph_isomorphisms/simple_directed} holds.

  \NecessitySubProof* Suppose that \( f: V_G \to V_H \) is a bijective homomorphism from \( G \) to \( H \) and that \eqref{eq:thm:graph_isomorphisms/simple_directed} holds.

  For any arc \( (u, v) \) in \( H \), we have
  \begin{equation*}
    (u, v) = \parens[\Big]{ f(f^{-1}(u)), f(f^{-1}(v)) }.
  \end{equation*}
  and thus, by \eqref{eq:thm:graph_isomorphisms/simple_directed}, \( (f^{-1}(u), f^{-1}(v)) \) is an arc in \( G \).

  Therefore, \( f^{-1} \) is itself a homomorphism from \( H \) to \( G \), and it clearly satisfies \eqref{eq:thm:graph_isomorphisms/simple_directed}. When composed with \( f \), we obtain the identity on either \( V_G \) or \( V_H \) depending on the order of composition, hence \( f^{-1} \) is the two-sided inverse of \( f \).

  \SubProofOf{thm:graph_isomorphisms/multi_undirected}

  \SufficiencySubProof* If \( (f_V, f_E): G \to H \) is a categorical isomorphism, we can prove that both \( f_V \) and \( f_E \) are bijective analogously to the case for directed multigraphs.

  \NecessitySubProof* Suppose that \( (f_V, f_E): G \to H \) is a homomorphism and that both \( f_V \) and \( f_E \) are bijective.

  For any edge \( e \) of \( H \), we have
  \begin{equation*}
    \mscrE_H(e)
    =
    \mscrE_H\parens[\Big]{ f_E(f_E^{-1}(e)) }
    \reloset {\eqref{eq:def:undirected_multigraph/homomorphism}} =
    \set{ f_V(u) \given u in \mscrE_G(f_E^{-1}(e)) }.
  \end{equation*}

  Then
  \begin{equation*}
    \set{ f_V^{-1}(v) \given v \in \mscrE_H(e) }
    =
    \set{ f_V^{-1}(f_V(u)) \given u \in \mscrE_G(f_E^{-1}(e)) }
    =
    \mscrE_G(f_E^{-1}(e)).
  \end{equation*}

  Therefore, the pair \( (f_V^{-1}, f_E^{-1}) \) satisfies \eqref{eq:def:undirected_multigraph/homomorphism}, which makes it a homomorphism of undirected multigraphs.

  Analogously to the case of directed multigraphs, we conclude that \( (f_V, f_E) \) is a categorical isomorphism with inverse \( (f_V^{-1}, f_E^{-1}) \).

  \SubProofOf{thm:graph_isomorphisms/simple_undirected} Both sufficiency and necessity can be proven similarly to the case of simple directed graphs.
\end{proof}

\begin{definition}\label{def:multigraph_orientation}\mcite[def. 6.2.1]{Knauer2011}
  We call the \hyperref[def:directed_multigraph]{directed multigraph} \( D = (V, A, h, t) \) an \term[ru=ориентация (\cite[32]{Емеличев1990})]{orientation} of the \hyperref[def:undirected_multigraph]{undirected multigraph} \( G = (U, E, \mscrE) \) if \( U = V \), \( A = E \), and, for every arc \( e \in A \), we have \( \mscrE(e) = \set{ h(e), t(e) } \).
\end{definition}
\begin{comments}
  \item This definition also applies to simple graphs via the inclusion functors \hyperref[def:graph_functors/directed_inclusion]{\( I_D \)} and \hyperref[def:graph_functors/undirected_inclusion]{\( I_U \)}.
\end{comments}

\begin{definition}\label{def:graph_functors}\mimprovised
  Different kinds of graphs are related via the following functors (shown graphically in \fullref{fig:def:graph_functors}):
  \begin{thmenum}
    \thmitem{def:graph_functors/directed_forgetful} The simple \hyperref[def:concrete_category]{forgetful functor}:
    \begin{flalign*}
      &U_D: \hyperref[def:directed_multigraph/category]{\T*{Directed multigraphs}} \to \hyperref[def:directed_graph/category]{\T*{Simple directed graphs with loops}}, &&\\
      &U_D(V, A, h, t) \coloneqq \parens[\Big]{ V, \set[\Big]{ \parens[\Big]{ h(a), t(a) } \given* a \in A } }, &&\\
      &U_D(f_V, f_A) \coloneqq f_V.
    \end{flalign*}

    \thmitem{def:graph_functors/directed_inclusion} A \hyperref[def:category_adjunction]{left adjoint}:
    \begin{flalign*}
      &I_D: \hyperref[def:directed_graph/category]{\T*{Simple directed graphs}} \to \hyperref[def:directed_multigraph/category]{\T*{Directed multigraphs}}, &&\\
      &I_D(V, A) \coloneqq \parens[\Big]{ V, A, (u, v) \mapsto u, (u, v) \mapsto v }, &&\\
      &I_D(f) \coloneqq \parens[\Big]{ f, (u, v) \mapsto \parens[\Big]{ f(u), f(v) } }.
    \end{flalign*}

    \thmitem{def:graph_functors/undirected_forgetful} A similar forgetful functor for undirected graphs:
    \begin{flalign*}
      &U_U: \hyperref[def:undirected_multigraph/category]{\T*{Undirected multigraphs}} \to \hyperref[def:undirected_graph/category]{\T*{Simple undirected graphs with loops}}, &&\\
      &U_U(V, E, \mscrE) \coloneqq \parens[\Big]{ V, \set[\Big]{ \mscrE(e) \given* e \in E } }, &&\\
      &U_U(f_V, f_E) \coloneqq f_V.
    \end{flalign*}

    \thmitem{def:graph_functors/undirected_inclusion} A left adjoint:
    \begin{flalign*}
      &I_U: \hyperref[def:undirected_graph/category]{\T*{Simple undirected graphs}} \to \hyperref[def:undirected_multigraph/category]{\T*{Undirected multigraphs}}, &&\\
      &I_U(V, E) \coloneqq \parens[\Big]{ V, E, e \mapsto e }, &&\\
      &I_U(f) \coloneqq \parens[\Big]{ f, \set{ u, v } \mapsto \set{ f(u), f(v) } }.
    \end{flalign*}

    \thmitem{def:graph_functors/multi_forgetful} A functor that forgets the \hyperref[def:multigraph_orientation]{orientation} of a multigraph:
    \begin{flalign*}
      &U_M: \hyperref[def:directed_multigraph/category]{\T*{Directed multigraphs}} \to \hyperref[def:undirected_multigraph/category]{\T*{Undirected multigraphs}}, &&\\
      &U_M(V, A, h, t) \coloneqq \parens[\Big]{ V, A, e \mapsto \set{ h(e), t(e) } }, &&\\
      &U_M(f_V, f_A) \coloneqq (f_V, f_A).
    \end{flalign*}

    \thmitem{def:graph_functors/simple_forgetful} A functor that forgets the orientation of a simple graph:
    \begin{flalign*}
      &U_S: \hyperref[def:directed_graph/category]{\T*{Simple directed graphs}} \to \hyperref[def:undirected_graph/category]{\T*{Simple undirected multigraphs}}, &&\\
      &U_S(V, A) \coloneqq \parens[\Big]{ V, A, (u, v) \mapsto \set{ u, v } }, &&\\
      &U_S(f) \coloneqq f.
    \end{flalign*}

    Note that opposite arcs map to the same edge.

    \thmitem{def:graph_functors/simple_doubling} A right adjoint to \( U_S \), which doubles each edge to produce arcs in both directions:
    \begin{flalign*}
      &D_S: \hyperref[def:undirected_graph/category]{\T*{Simple undirected graphs}} \to \hyperref[def:directed_multigraph/category]{\T*{Simple directed graphs}}, &&\\
      &D_S(V, E) \coloneqq \parens[\Big]{ V, \set[\Big]{ (u, v) \in V^2 \given \set{ u, v } \in E } }, &&\\
      &D_S(f) \coloneqq f.
    \end{flalign*}
  \end{thmenum}
\end{definition}
\begin{comments}
  \item \Fullref{ex:def:category_adjunction/us_ds} discusses the doubling functor from \fullref{def:graph_functors/simple_doubling}.
  \item We can introduce a doubling functor for multigraphs similar to \hyperref[def:graph_functors/simple_doubling]{\( D_S \)} that introduces two opposite copies of each edge in an undirected multigraph.

  Unfortunately, that would require choosing one vertex as the head and the other as a tail, and thus the functor depends on a \hyperref[def:choice_function]{choice function}. This choice function is unnecessary for simple graphs because ordered pairs having a first and second element, while multigraphs have abstract objects as arcs.

  Furthermore, even given a canonical choice function, this functor would not be a right inverse to \hyperref[def:graph_functors/multi_forgetful]{\( U_M \)} --- instead, we would obtain an undirected multigraph with twice as many edges as the original.

  We avoid introducing such a functor altogether.
\end{comments}

\begin{remark}\label{rem:arbitrary_kind_graph}
  We will henceforth use the phrase \enquote{arbitrary-kind graph} to refer to any of the four kinds of graphs defined in this section.

  The inclusion functors \hyperref[def:graph_functors/directed_inclusion]{\( I_D \)} and \hyperref[def:graph_functors/undirected_inclusion]{\( I_U \)} allow all definitions for multigraphs to apply to simple graphs. The transition between directed and undirected graphs is more complicated, but we will nonetheless utilize \hyperref[def:graph_functors/multi_forgetful]{\( U_M \)}, \hyperref[def:graph_functors/simple_forgetful]{\( U_S \)} and \hyperref[def:graph_functors/simple_doubling]{\( D_S \)} when necessary.

  These functors will be used implicitly, but when increased attention is required, we use these functors explicitly. See \fullref{thm:graph_coloring_as_homomorphism} for an example.
\end{remark}

\begin{definition}\label{def:graph_incidence}\mcite[3]{Diestel2005}
  We say that the vertex \( v \) and the arc/edge \( e \) are \term[bg=инцидентни (ребра) (\cite[7]{Мирчев2001}), ru=инцидентные (рёбра) (\cite[9]{Емеличев1990})]{incident} of \( v \) is an endpoint of \( e \).
\end{definition}

\begin{definition}\label{def:graph_adjacency}\mcite[3]{Diestel2005}
  In an \hyperref[rem:arbitrary_kind_graph]{arbitrary-kind graph}, if two vertices are \hyperref[def:graph_incidence]{incident} to a common arc/edge, we say that they are \term[bg=съседни (върхове) (\cite[7]{Мирчев2001}), ru=смежные (вершины) (\cite[9]{Емеличев1990})]{adjacent}.
\end{definition}

\begin{remark}\label{rem:trivial_graph}
  Unlike the \hyperref[def:group/trivial]{trivial group} \( \set{ e } \) or \hyperref[def:module/trivial]{trivial \( R \)-module} \( \set{ 0 } \), which are unique up to an isomorphism, there is no trivial graph in the sense of \fullref{def:trivial_object}.

  Every graph has a subgraph with \hyperref[def:graph_cardinality/order]{order} zero, and hence up to an isomorphism we have an order-zero graph (for every kind of graph discussed here).

  Another unambiguous concept is that of an edgeless graph. Every \hyperref[def:complete_graph]{complete graph} has \( 2^n \) edgeless subgraphs (one for each set of edges).
\end{remark}

\paragraph{Cardinalities in graphs}

\begin{definition}\label{def:graph_cardinality}
  Graphs have the following notions of \hyperref[thm:cardinality_existence]{cardinality}:
  \begin{thmenum}
    \thmitem{def:graph_cardinality/order}\mcite[2]{Diestel2005} We define the \term[ru=порядок (\cite[9]{Емеличев1990})]{order} \( \ord(G) \) of an \hyperref[rem:arbitrary_kind_graph]{arbitrary-kind graph} \( G \) as the (cardinal) number of vertices.

    For a simple graph, finitely many vertices imply finitely many arcs/edges, which justifies terminology like \enquote{finite simple graph}. For multigraphs, however, we will prefer being more concrete and use \enquote{finite-order (multi)graph}.

    \thmitem{def:graph_cardinality/directed_degree}\mcite[def. 1.1.7]{Knauer2011} We define the \term[bg=полустепен на изхода (\incite[8]{Мирчев2001}), ru=полустепень исхода (\cite[283]{Емеличев1990})]{out-degree} \( \deg_{\op{out}}(v) \) (resp. \term[bg=полустепен на входа (\incite[8]{Мирчев2001}), ru=полустепень захода (\cite[283]{Емеличев1990})]{in-degree} \( \deg_{\op{in}}(v) \)) of a vertex \( v \) in a \hyperref[def:directed_multigraph]{directed (multi)graph} as the (cardinal) number of arcs starting (resp. ending) at \( v \).

    We define the \term[bg=степен (\cite[8]{Мирчев2001}), ru=степень (\cite[283]{Емеличев1990})]{degree} \( \deg(v) \) of \( v \) as the sum of the two.

    \thmitem{def:graph_cardinality/undirected_degree}\mimprovised If the graph is undirected, consider the \hyperref[def:multiset]{multiset} of all edges incident to \( v \), with loops having multiplicity \( 2 \) and all other edges having multiplicity \( 1 \).

    We define the \term{degree} \( \deg(v) \) as the \hyperref[def:multiset/cardinality]{cardinality} of this multiset.

    \thmitem{def:graph_cardinality/local}\mcite[196]{Diestel2005} We say that a graph is \term{locally finite} (resp. \term{locally countable}) if the degree of any vertex is finite (resp. countable).
  \end{thmenum}
\end{definition}

\begin{proposition}\label{thm:degree_of_undirected_counterpart}
  The degree, in the sense of \fullref{def:graph_cardinality/directed_degree}, of a vertex \( v \) in a \hyperref[def:directed_multigraph]{directed multigraph} \( G \), is equal to the degree, in the sense of \fullref{def:graph_cardinality/undirected_degree}, of \( v \) in the undirected counterpart \( \hyperref[def:graph_functors/multi_forgetful]{U_M}(G) \) of \( G \).
\end{proposition}
\begin{comments}
  \item When proving statements about vertex degrees, it thus makes sense to prove it for undirected multigraphs, since it will then also automatically apply to directed graphs.
\end{comments}
\begin{proof}
  We have adjusted our definitions so that this holds.
\end{proof}

\begin{remark}\label{rem:counting_loops_twice}
  Loops may contribute either \( 1 \) or \( 2 \) towards the degree of a graph. This choice simplifies \fullref{thm:sum_of_endpoint_degrees} and hence \fullref{thm:sum_of_graph_degrees}, as well as \fullref{def:graph_cycle}, among others.

  We describe here several conventions.
  \begin{itemize}
    \item \incite[27]{Емеличев1990}, who diligently distinguish between directed and undirected graphs, defines degrees in (simple) directed graphs as sums of out-degrees and in-degrees, like we do, and in undirected graphs they, like us, count loops twice.

    \item \incite[277]{БелоусовТкачёв2004}, who also give separate definitions for directed and undirected graphs, also define degrees in (simple) directed graphs as sums of out-degrees and in-degrees, but for undirected graphs they avoid counting loops twice.

    \item \incite[544]{Rosen1999} defines the degree of a vertex similarly to us, however without referring to multisets.

    \item \incite[191]{Erickson2019} and \incite[8]{Мирчев2001} define degrees without counting loops twice for both directed and undirected graphs.

    \item \incite[def. 1.1.7]{Knauer2011} avoids defining \enquote{total} degrees for directed graphs, leaving only out-degrees and in-degrees. For undirected graphs, however, in \cite[def. 1.1.8]{Knauer2011} he defines the degree of a vertex \( v \) as the number of edges incident to \( v \), which without our multiset trick only counts loops once.

    \item \incite[372]{Knuth1997Vol1} also defines out-degrees and in-degrees, but avoids defining degrees for both directed and undirected graphs.

    \item \incite[8]{Bollobas1998} hints that loops should be counted twice.
  \end{itemize}
\end{remark}

\begin{definition}\label{def:isolated_vertex}\mcite[5]{Diestel2005}
  We say that a vertex in an \hyperref[rem:arbitrary_kind_graph]{arbitrary-kind graph} is \term[bg=изолирани (върхове) (\cite[8]{Мирчев2001}), ru=изолированные (вершины) (\cite[26]{Емеличев1990})]{isolated} if its \hyperref[def:graph_cardinality/directed_degree]{degree} is zero.
\end{definition}

\begin{lemma}\label{thm:sum_of_endpoint_degrees}
  Fix an undirected multigraph \( G = (V, E, \mscrE) \). Fix an edge \( e \) and consider the graph \( G = (V, E \setminus \set{ e }, \mscrE) \). Denote by \( \deg'(v) \) the vertex degree in \( G' \).

  Then
  \begin{equation}\label{eq:thm:sum_of_endpoint_degrees}
    \sum_{v \in \mscrE(e)} \deg(v) = \sum_{v \in \mscrE(e)} \deg'(v) + 2.
  \end{equation}
\end{lemma}
\begin{proof}
  If \( e \) is a loop at \( v \), by definition we have
  \begin{equation*}
    \sum_{w \in \mscrE(e)} \deg(w) = \deg(v) = \deg'(v) + 2 = \sum_{w \in \mscrE(e)} \deg'(w) + 2.
  \end{equation*}

  If \( e \) is an edge between \( u \) and \( v \), then
  \begin{equation*}
    \sum_{w \in \mscrE(e)} \deg(w)
    =
    \deg(u) + \deg(v)
    =
    \deg'(u) + 1 + \deg'(v) + 1
    =
    \sum_{w \in \mscrE(e)} \deg'(w) + 2.
  \end{equation*}
\end{proof}

\begin{proposition}\label{thm:sum_of_graph_degrees}
  In an \hyperref[rem:arbitrary_kind_graph]{arbitrary-kind graph} with finitely many arcs/edges, the sum of the degrees of all vertices is twice the number of arcs/edges.
\end{proposition}
\begin{comments}
  \item As long as a graph only has finitely many arcs/edges, only finitely many vertices have positive degree, so summing all degrees is justified.
  \item As a consequence, the sum of degrees if even. \incite[4]{Bollobas1998} and \incite[prop. 5.1]{Емеличев1990} call this consequence the \enquote{handshaking lemma} because of the observation that the total number of hands shaken at a party is even.
\end{comments}
\begin{proof}
  As per \fullref{thm:degree_of_undirected_counterpart}, it is sufficient to only prove the statement for undirected multigraphs.

  We will use induction on the number \( n \) of arcs.

  \begin{itemize}
    \item If there are zero edges, then the degree of every vertex is zero, hence the sum of all degrees is also zero.

    \item Suppose that the statement holds for multigraphs with \( n \) arcs and consider an undirected multigraph \( G = (V, E, \mscrE) \) with \( n + 1 \) arcs.

    Fix any arc \( e \) and consider \( G' \coloneqq (V, E \setminus \set{ e }, \mscrE) \). Denote by \( \deg'(v) \) the degree of \( v \) in \( G' \). Then \fullref{thm:sum_of_endpoint_degrees} implies that
    \begin{equation*}
      \overbrace{\sum_{v \in V} \deg(v) = \sum_{v \in V \setminus \mscrE(e)} \underbrace{\deg(v)}_{\deg'(v)} + \underbrace{\sum_{v \in \mscrE(e)} \deg(v)}_{\sum_{v \in \mscrE(e)} \deg'(v) + 2} - 2}^{2n \T*{by the inductive hypothesis}} + 2
      =
      2(n + 1)
    \end{equation*}
  \end{itemize}
\end{proof}

\begin{corollary}\label{thm:odd_degree_vertices}\mcite[prop. 1.2.1]{Diestel2005}
  In an \hyperref[rem:arbitrary_kind_graph]{arbitrary-kind graph} with finitely many arcs/edges, the number of vertices of odd \hyperref[def:graph_cardinality/directed_degree]{degree} is even.
\end{corollary}
\begin{comments}
  \item This corollary, while by itself inappreciable, is essential for proving \fullref{thm:eulers_theorem_for_graphs}. It first appeared in \cite{Euler1741} by Leonhard Euler, considered the first paper on graph theory and discussed in \fullref{ex:konigsberg_bridges}.
\end{comments}
\begin{proof}
  As per \fullref{thm:degree_of_undirected_counterpart}, it is sufficient to consider some undirected multigraph \( G = (V, E, \mscrE) \).

  We have
  \begin{equation}\label{eq:thm:odd_degree_vertices/proof}
    \underbrace{\sum_{v \in V} \deg(v)}_{\T*{even by} \fullref{thm:sum_of_graph_degrees}} = \underbrace{\sum_{\deg(v) \T*{is even}} \deg(v)}_{\T{sum of even numbers}} + \underbrace{\sum_{\deg(v) \T*{is odd}} \deg(v)}_{\T{sum of odd numbers}}.
  \end{equation}

  In order for \eqref{eq:thm:odd_degree_vertices/proof} to hold, there must be an even number of vertices of odd degree.
\end{proof}

\begin{example}\label{ex:infinite_integer_graphs}
  An example of an infinite \hyperref[def:directed_graph]{simple directed graph} is the \hyperref[def:transitive_reduction]{transitive reduction} of the positive integers:
  \begin{equation}\label{eq:ex:infinite_integer_graphs/positive}
    \begin{aligned}
      \includegraphics[page=1]{output/ex__infinite_integer_graphs}
    \end{aligned}
  \end{equation}

  Since the graph is simple, we have
  \begin{equation*}
    \deg(n) = \begin{cases}
      \deg_{\op{in}}(n) + \deg_{\op{out}}(n) = 0 + 1 = 1, &n = 0, \\
      \deg_{\op{in}}(n) + \deg_{\op{out}}(n) = 2 + 1 = 2, &n > 0 \\
    \end{cases}
  \end{equation*}

  The \hyperref[def:categorical_diagram]{categorical diagram} corresponding to this graph is used to define direct limits in \fullref{def:direct_and_inverse_limits/direct}.

  Another related graph is based on the negative integers:
  \begin{equation}\label{eq:ex:infinite_integer_graphs/negative}
    \begin{aligned}
      \includegraphics[page=2]{output/ex__infinite_integer_graphs}
    \end{aligned}
  \end{equation}

  The categorical diagram corresponding to this graph is used to define inverse limits in \fullref{def:direct_and_inverse_limits/inverse}.

  Finally, the union of the two with zero added gives us the following directed graph:
  \begin{equation}\label{eq:ex:infinite_integer_graphs/two_sided}
    \begin{aligned}
      \includegraphics[page=3]{output/ex__infinite_integer_graphs}
    \end{aligned}
  \end{equation}

  All three graphs \hyperref[def:graph_cardinality/local]{locally finite} but have infinite \hyperref[def:graph_cardinality/order]{order}.
\end{example}

  \section{Trees}\label{sec:trees}

\begin{definition}\label{def:tree}\mcite[thm. 1.5.1]{Diestel2017GraphTheory}
  We call the (possibly infinite) \hyperref[def:undirected_graph]{simple undirected graph} \( T = (E, V) \) a \term[bg=дърво (\cite[21]{Мирчев2001Графи}), ru=дерево (\cite[53]{ЕмеличевИПр1990ТеорияГрафов})]{tree} if any of the following equivalent conditions hold:
  \begin{thmenum}[series=def:tree]
    \thmitem{def:tree/connected_acyclic} \( T \) is \hyperref[def:graph_connectedness/undirected]{connected} and \hyperref[def:acyclic_graph]{acyclic}.

    \thmitem{def:tree/minimally_connected} \( T \) is \term{minimally connected}, meaning that removing an edge will make the graph \hyperref[def:graph_connectedness/undirected]{disconnected}.

    \thmitem{def:tree/maximally_acyclic} \( T \) is \term{maximally acyclic}, meaning that adding an edge between existing vertices would create a \hyperref[def:graph_cycle]{cycle}.

    \thmitem{def:tree/single_path} There exists a unique \hyperref[def:graph_walk/path]{path} connecting every pair of distinct vertices in \( T \).
  \end{thmenum}

  We will refer to vertices in \hyperref[def:tree]{trees} as \term[en=node (\cite[190]{Erickson2019Algorithms})]{nodes}.

  Trees inherit metamathematical properties from undirected graphs:
  \begin{thmenum}[resume=def:tree]
    \thmitem{def:tree/homomorphism}\mimprovised A \term{homomorphism} between trees is simply an undirected graph homomorphisms.

    \thmitem{def:tree/category}\mimprovised Correspondingly, for a \hyperref[def:grothendieck_universe]{Grothendieck universe} \( \mscrU \), the category of trees is a \hyperref[def:subcategory/full]{full subcategory} of the category of undirected graphs.

    \thmitem{def:tree/subtree}\mcite[def. 5.9]{БелоусовТкачёв2004ДискретнаяМатематика} We say that the tree \( S \) is a \term{subtree} of \( T \) if it is a \hyperref[def:undirected_graph/subgraph]{subgraph} of \( T \).
  \end{thmenum}
\end{definition}
\begin{defproof}
  \ImplicationSubProof{def:tree/connected_acyclic}{def:tree/minimally_connected} Let \( T \) be a connected acyclic graph.

  Let \( T' \) be the graph obtained from \( T \) by removing the edge \( \set{ u, v } \). Then \( T' \) is not connected because, if there is a path from \( u \) to \( v \) in \( T' \), it would extend to a cycle in \( T \), contradicting the acyclicity of \( T \).

  Therefore, \( T \) is minimally connected.

  \ImplicationSubProof{def:tree/minimally_connected}{def:tree/maximally_acyclic} Let \( T \) be a minimally connected graph.

  \SubProof*{Proof that \( T \) is acyclic} Suppose that there exists a cycle
  \begin{equation*}
    v_0 \to v_1 \to \cdots \to v_n.
  \end{equation*}

  \Cref{thm:connected_graph_cycle_removal} implies that the graph obtained from \( T \) by removing the edge \( \set{ v_0, v_1 } \) is still connected, contradicting the assumption that \( T \) is minimally connected.

  The obtained contradiction shows that no cycles can exist in \( T \).

  \SubProof*{Proof that \( T \) is minimally acyclic} Let \( T' \) be the graph obtained from \( T \) by adding an edge between the vertices \( u \) and \( v \). Since \( T \) is connected, there exists a path from \( u \) to \( v \), which in \( T' \) can be extended to a cycle on \( u \). Thus, \( T' \) is not acyclic.

  Therefore, \( T \) is maximally acyclic.

  \ImplicationSubProof{def:tree/maximally_acyclic}{def:tree/single_path} Let \( T \) be a maximally acyclic graph.

  Fix two vertices \( u \) and \( v \) and suppose that no walk exists between them. Let \( T' \) be the graph obtained by adding the edge \( \set{ u, v } \) to \( T \).

  Suppose additionally that \( T' \) has a cycle, say
  \begin{equation*}
    w_0 \to w_1 \to \cdots \to w_n.
  \end{equation*}

  Since \( T \) is acyclic, \( \set{ u, v } \) must be an edge in this cycle.
  \begin{itemize}
    \item If, for some \( k = 0, \ldots, n - 1 \), we have \( u = w_k \) and \( v = w_{k+1} \), then the following is a walk in \( T \):
    \begin{equation*}
      u = w_k \to w_{k-1} \to \cdots \to w_0 \to w_{n-1} \to \cdots \to w_{k+1} = v.
    \end{equation*}

    \item Otherwise, for some \( k = 0, \ldots, n - 1 \), we must have \( u = w_{k+1} \) and \( v = w_k \). Then the following is a walk in \( T \):
    \begin{equation*}
      u = w_{k+1} \to w_{k+2} \to \cdots \to w_n \to w_1 \to \cdots \to w_k.
    \end{equation*}
  \end{itemize}

  The existence of these walks contradict our assumption that there is no walk from \( u \) to \( v \) in \( T \). Thus, \( T' \) must also be acyclic, which in turn contradicts our main assumption --- that \( T \) is maximally acyclic.

  Therefore, it remains for \( T \) to be connected. Furthermore, by \cref{thm:acyclic_graph_paths}, there exists at most one path, and hence exactly one path between any set of vertices.

  \ImplicationSubProof{def:tree/single_path}{def:tree/connected_acyclic} Let \( T \) be a graph in which every pair of vertices is connected by a unique path.

  Then \( T \) is clearly connected. Furthermore, if \( T \) has some cycle
  \begin{equation*}
    v_0 \to v_1 \to \cdots \to v_{n-1} \to v_n,
  \end{equation*}
  then
  \begin{equation*}
    v_0 \to v_1 \to \cdots \to v_{n-1}
  \end{equation*}
  is a path connecting \( v_0 \) to \( v_{n-1} \), and so is the single-edge path
  \begin{equation*}
    v_0 \to v_{n-1}.
  \end{equation*}

  We thus obtain two distinct paths from \( v_0 \) to \( v_{n-1} \), contradicting out initial assumption.

  Therefore, \( T \) must be acyclic.
\end{defproof}

\paragraph{Rooted trees}

\begin{definition}\label{def:rooted_tree}\mcite[def. 11.1.2; 783; 784]{Rosen2019DiscreteMathematics}
  A \term[ru=ориентированное дерево (\cite[323]{ЕмеличевИПр1990ТеорияГрафов}); корневое дерево (\cite[\S 9.2.1]{Новиков2013ДискретнаяМатематика})]{rooted tree} is a triple \( T = (V, E, r) \), where \( (V, E) \) is a \hyperref[def:tree]{tree} and \( r \) is a distinguished node, called the \term{root}.

  As per \cref{def:tree/single_path}, every node \( v \) has a unique path from \( r \).

  \begin{thmenum}
    \thmitem{def:rooted_tree/order}\mcite[15]{Diestel2017GraphTheory} We define the (nonstrict) \hyperref[def:partially_ordered_set]{partial order} \( u \leq v \) to hold if \( u \) lies on the path from \( r \) to \( v \). We call \( \leq \) the \term{ancestor order} or \term{vertical order} of \( T \) in order to distinguish it from the horizontal order in \hyperref[def:ordered_tree]{ordered trees}.

    \thmitem{def:rooted_tree/height}\mcite[15]{Diestel2017GraphTheory} The \term{height} of a node \( v \) is the length of the unique path from \( r \). This is precisely the height of \( v \) in the sense of \cref{def:partial_order_element_height}.

    \thmitem{def:rooted_tree/ancestor_descendant} If \( u \leq v \), we say that \( u \) is an \term[ru=предок (\cite[298]{БелоусовТкачёв2004ДискретнаяМатематика})]{ancestor} of \( v \) and that \( v \) is a \term[ru=потомок (\cite[298]{БелоусовТкачёв2004ДискретнаяМатематика})]{descendant} of \( u \).

    \thmitem{def:rooted_tree/parent_child} If \( u \) is the penultimate element in the path from \( r \) to \( v \), we say that \( u \) is a \term[ru=отец (\cite[298]{БелоусовТкачёв2004ДискретнаяМатематика})]{parent} of \( v \) and that \( v \) is a \term[ru=сын (\cite[298]{БелоусовТкачёв2004ДискретнаяМатематика})]{child} of \( u \).

    \thmitem{def:rooted_tree/leaf} If a node has no children, we call it a \term[ru=лист (\cite[298]{БелоусовТкачёв2004ДискретнаяМатематика})]{leaf node}.

    \thmitem{def:rooted_tree/internal} If a node is not a leaf, we say that it is \term{internal}.

    \thmitem{def:rooted_tree/siblings} If \( u \) and \( v \) has the same parent, we call them \term{siblings}.
  \end{thmenum}

  Rooted trees have the following metamathematical properties:
  \begin{thmenum}[resume=def:rooted_tree]
    \thmitem{def:rooted_tree/subtree} A \term{rooted subtree} with root \( s \) is the \hyperref[def:tree/subtree]{subtree} consisting of \( s \) and all its descendants. If \( s \) is a child of \( r \), we say that its subtree is \term[en=subtree of the root (\cite[308]{Stanley2012EnumerativeCombinatoricsVol1})]{immediate}.

    \thmitem{def:rooted_tree/homomorphism}\mimprovised A \term{homomorphism} between rooted trees is a \hyperref[def:tree/homomorphism]{tree homomorphism} that preserves roots.

    \thmitem{def:rooted_tree/category}\mimprovised For a \hyperref[def:grothendieck_universe]{Grothendieck universe} \( \mscrU \), we can consider the category of \( \mscrU \)-small rooted trees with rooted tree homomorphisms. It is \hyperref[def:concrete_category]{concrete} over the category of trees.
  \end{thmenum}
\end{definition}
\begin{comments}
  \item \incite[\S 9.2.1]{Новиков2013ДискретнаяМатематика} calls a \enquote{rooted tree} (\enquote{корневое дерево}) what we call an \hyperref[def:oriented_tree]{oriented tree}. The difference is inessential due to \cref{thm:oriented_and_rooted_trees}.
\end{comments}

\begin{definition}\label{def:oriented_tree}\mimprovised
  An \term[bg=ориентирано дърво (\cite[21]{Мирчев2001Графи}), ru=ориентированное дерево (\cite[def. 5.6]{БелоусовТкачёв2004ДискретнаяМатематика}), en=oriented tree (\cite[373]{Knuth1997ArtVol1})]{oriented tree} or \term[en=arborescence (\cite[\S 3.1]{GondranMinoux1984GraphsAndAlgorithms})]{arborescence} is an \hyperref[def:well_founded_graph]{well-founded} \hyperref[def:directed_graph]{simple directed graph} \( T = (V, A) \) in which one distinguished vertex, called the \term{root}, has \hyperref[def:graph_cardinality/directed_degree]{in-degree} \( 0 \), and all other vertices have in-degree \( 1 \).
\end{definition}
\begin{comments}
  \item The terminology from \cref{def:rooted_tree} applies to oriented trees because of \cref{thm:oriented_and_rooted_trees/orientation}.

  \item \incite[def. 5.6]{БелоусовТкачёв2004ДискретнаяМатематика} require oriented trees to be acyclic rather than well-founded, however that definition leads to pathological infinite oriented trees --- see \cref{ex:infinite_oriented_tree}.

  \item Another possible convention for oriented trees is to place the same restrictions on out-degrees rather than in-degrees, which essentially reverses the directions of the arcs. This is the approach taken by \incite[373]{Knuth1997ArtVol1}.
\end{comments}

\begin{example}\label{ex:infinite_oriented_tree}
  The requirement for well-foundedness in \cref{def:oriented_tree} is essential. Consider the graph
  \begin{equation}\label{eq:ex:infinite_oriented_tree}
    \begin{aligned}
      \includegraphics[page=1]{output/ex__infinite_oriented_tree}
    \end{aligned}
  \end{equation}

  If we were to follow \incite[def. 5.6]{БелоусовТкачёв2004ДискретнаяМатематика} and require oriented trees to be merely acyclic, then \eqref{eq:ex:infinite_oriented_tree} would satisfy the definition. For finite graphs acyclicity is sufficient due to \cref{thm:def:well_founded_graph/finite_acyclic}.

  Another option would be to require oriented trees to be acyclic, but forbid infinite paths.
\end{example}

\begin{proposition}\label{thm:oriented_tree_path_uniqueness}
  A \hyperref[def:well_founded_graph]{well-founded} \hyperref[def:directed_graph]{simple directed graph} is an \hyperref[def:oriented_tree]{oriented tree} if and only if there is a unique \hyperref[def:graph_walk/directed]{directed path} from the root to any node.
\end{proposition}
\begin{proof}
  \SufficiencySubProof Fix an oriented tree \( T = (V, A) \) with root \( r \). Fix any node \( v \).

  \UniquenessSubProof* Fix two paths
  \begin{equation*}
    r = u_0 \to u_1 \to \cdots \to u_n = v
  \end{equation*}
  and
  \begin{equation*}
    r = v_0 \to v_1 \to \cdots \to v_m = v.
  \end{equation*}

  Suppose that they are different. Let \( k_0 \) be the smallest index such that \( u_{n - k_0} \neq v_{m - k_0} \). Then the node \( u_{n - k_0 + 1} = v_{n - k_0 + 1} \) has in-degree at least \( 2 \), which contradicts the definition of oriented tree.

  Therefore, the two paths are equal.

  \ExistenceSubProof* We use natural number recursion to build a sequence of nodes starting at \( v \):
  \begin{equation*}
    v_k \coloneqq \begin{cases}
      v, &k = 0, \\
      u, &k > 1 \T{and} u \T{is the unique predecessor of} v_{k-1}, \\
      r, &\T{otherwise}.
    \end{cases}
  \end{equation*}

  For each index \( k \), either \( v_{k+1} \to v_k \) or \( v_{k+1} = v_k \). Since \( T \) is well-founded, the sequence eventually stabilizes, and thus there exists a largest index \( n \) such that \( v_{n+1} \to v_{n} \). Since all nodes except for \( r \) have a unique predecessor, we conclude that \( v_m = u \) for every integer \( m > n \).

  This gives rise to the path
  \begin{equation*}
    u = v_{n+1} = \to v_n \to \cdots \to v_1 \to v_0 = v.
  \end{equation*}

  \NecessitySubProof Fix a well-founded directed graph \( T = (V, A) \) such that there is a unique directed path from a fixed node \( r \) to any node. We will show that it is an oriented tree.

  \SubProof*{Proof that \( r \) has in-degree \( 0 \)} Suppose that there is an arc from some node \( v \) to \( r \). Then we can concatenate it with the unique path from \( r \) to \( v \) and obtain a cycle at \( r \). This contradicts \cref{thm:def:well_founded_graph/acyclic} since \( T \) is well-founded. The obtained contradiction shows that \( T \) has no arcs ending at \( r \).

  \SubProof*{Proof that any other node has in-degree \( 1 \)} Given any node \( u \) distinct from \( r \), suppose that there is an arc from \( v \) to \( u \) and from \( w \) to \( u \).

  We can concatenate the path from \( r \) to \( v \) with \( v \to u \) and the path from \( r \) to \( w \) with \( w \to u \) to obtain two paths from \( r \) to \( u \).

  The two paths must be equal, which implies that \( v = w \). Hence, there is exactly one arc ending at \( u \).
\end{proof}

\begin{lemma}\label{thm:tree_order_well_founded}
  The \hyperref[def:rooted_tree/order]{tree order} of a rooted tree is \hyperref[def:well_founded_relation]{well-founded} as a binary relation.
\end{lemma}
\begin{proof}
  Consider any descending sequence of nodes
  \begin{equation*}
    \cdots \leq v_n \leq v_{n-1} \leq \cdots \leq v_1 \leq v_0.
  \end{equation*}

  By definition, \( v_k \) lies on the path from \( r \) to \( v_0 \) for every \( k \geq 0 \). But the path is finite, thus, by \fullref{thm:pigeonhole_principle}, the sequence \hyperref[def:stabilizing_sequence]{stabilizes}.
\end{proof}

\begin{proposition}\label{thm:oriented_and_rooted_trees}
  \hfill
  \begin{thmenum}
    \thmitem{thm:oriented_and_rooted_trees/forgetful} For every \hyperref[def:oriented_tree]{oriented tree} \( T \), its underlying undirected graph \hyperref[def:graph_functors/simple_forgetful]{\( U_S \)}\( (T) \) is a \hyperref[def:tree]{tree}, and the root of \( T \) makes it a \hyperref[def:rooted_tree]{rooted tree}.

    \thmitem{thm:oriented_and_rooted_trees/orientation} Every rooted tree \( T \) is an oriented tree with respect to the \hyperref[def:multigraph_orientation]{orientation} in which \( u \to v \) if \( u \leq v \) with respect to the \hyperref[def:rooted_tree/order]{tree order}.
  \end{thmenum}
\end{proposition}
\begin{proof}
  \SubProofOf{thm:oriented_and_rooted_trees/forgetful} Let \( T = (V, A) \) be an oriented tree with root \( r \).

  \SubProof*{Proof of connectedness} Fix any vertex \( v \).

  Let \( v_0 \coloneqq v \). We will build a path
  \begin{equation*}
    r = v_n \to \cdots \to v_1 \to v_0 = v.
  \end{equation*}

  At step \( k \), if \( v_k = r \), we are done. Otherwise, the in-degree of \( v_k \) is \( 1 \), and thus there exists exactly one vertex \( v_{k+1} \) such that \( (v_{k+1}, v_k) \) is an arc. Furthermore, \( v_{k+1} \) cannot be among \( v_k, \ldots, v_0 \) because \( T \) is acyclic, thus at every step
  \begin{equation*}
    v_{k+1} \to v_k \to \cdots \to v_1 \to v_0 = v
  \end{equation*}
  is a path.

  Since \( T \) is well-founded, there exists an index \( n \) such that \( v_n \) has in-degree zero. But there is only one such vertex --- precisely \( r \).

  \SubProof*{Proof of acyclicity} Since \( T \) is well-founded, \cref{thm:def:well_founded_graph/acyclic} implies that it is also acyclic.

  Suppose that the underlying undirected graph \( U_S(T) \) has an (undirected) cycle
  \begin{equation*}
    v_0, e_1, \ldots, e_n, v_n.
  \end{equation*}

  Then, since \( T \) is acyclic, at least one of the arcs \( e_1, \ldots, e_n \) in this (undirected) cycle must be negatively oriented. Furthermore, at least one of the arcs is positively oriented --- otherwise
  \begin{equation*}
    v_n, e_n, \cdots, e_1, v_0
  \end{equation*}
  would be a directed cycle in \( T \).

  Let \( e_k \) be the last positively oriented arc.
  \begin{itemize}
    \item If \( k < n \), then \( e_{k+1} \) is negatively oriented. Thus, \( e_k \) connects \( v_{k-1} \) to \( v_k \), while \( e_{k+1} \) connects \( v_{k+1} \) to \( v_k \). Then \( v_k \) has in-degree at least \( 2 \), which contradicts our assumption that \( T \) is an oriented tree.

    \item If \( k = n \), we can instead consider the last negatively oriented arc and similarly reach a contradiction.
  \end{itemize}

  From the assumption that \( U_S(T) \) has an undirected cycle, we have reached a contradiction, thus it remains for \( U_S(T) \) to be acyclic.

  \SubProofOf{thm:oriented_and_rooted_trees/orientation} Let \( T = (V, E, r) \) be a rooted tree and let \( D \) be its orientation in which \( u \to v \) if \( u \leq v \).

  \SubProof*{Proof of well-foundedness} By \cref{thm:tree_order_well_founded}, the tree order is well-founded. \Cref{thm:def:well_founded_graph/simple} implies that \( D \) it well-founded as a graph.

  \SubProof*{Proof of degree conditions} The arcs ending at \( v \) are precisely those connecting the parent of \( v \) to \( v \) itself. There is one parent for every node except for the root, which has none. Thus, the root has in-degree \( 0 \), while the other vertices have in-degree \( 1 \).
\end{proof}

\begin{theorem}[Induction on rooted trees]\label{thm:induction_on_rooted_trees}\mimprovised
  In order to prove a statement for every finite \hyperref[def:rooted_tree]{rooted tree}, it is sufficient to do the following:
  \begin{displayquote}
    Given a finite rooted tree \( T \), suppose that the statement holds for all its \hyperref[def:rooted_tree/subtree]{immediate subtrees} and prove the statement for \( T \) itself.
  \end{displayquote}

  Generalizing on \( T \), we can conclude that the statement holds for all finite rooted trees.
\end{theorem}
\begin{comments}
  \item Unlike for most induction principles discussed in \cref{con:induction}, we did not formulate this one via logical formulas since that would make the statement unnecessarily convoluted with little gain.

  \item We will mostly use this for \hyperref[con:abstract_syntax_tree]{abstract syntax trees} or even rely on \hyperref[con:evaluation]{pattern matching} to avoid working with trees directly.

  \item We will usually deal with only certain kinds of rooted trees, for example \hyperref[def:propositional_formula_ast]{propositional formula ASTs}. In this case the statement should be conditional on the tree \( T \) being a propositional AST.
\end{comments}
\begin{proof}
  By \cref{thm:tree_order_well_founded}, the \hyperref[def:rooted_tree]{tree order} is well-founded. This allows us to reduce the theorem to \fullref{thm:well_founded_induction}.
\end{proof}

\paragraph{Infinite trees}

\begin{proposition}\label{thm:removing_node_from_oriented_tree}
  If we remove any node from an \hyperref[def:oriented_tree]{oriented tree} along with its incident edges, the \hyperref[def:graph_connected_component]{weak connected components} of the resulting graph are trees.

  \begin{figure}[!ht]
    \centering
    \includegraphics[page=1]{output/thm__removing_node_from_tree}
    \caption{An illustration of \cref{thm:removing_node_from_oriented_tree}.}\label{fig:thm:removing_node_from_oriented_tree}
  \end{figure}
\end{proposition}
\begin{proof}
  Let \( T = (V, A) \) an oriented tree with root \( r \). Fix a node \( v \) and let \( \mscrR \) be the family of connected components of \( T \) with \( v \) (and its arcs) removed.

  Let \( R \) be any component from \( \mscrR \). Since \( T \) is well-founded, clearly so is \( R \). We must show that it has a root.

  We have two cases --- either \( R \) contains either a parent or child of \( v \).

  Fix a node \( u \) from \( R \). In \( T \), by \cref{thm:oriented_tree_path_uniqueness}, there exists a (unique) path
  \begin{equation}\label{eq:thm:removing_node_from_oriented_tree/proof/main_path}
    r = u_0 \to \cdots \to u_n = u.
  \end{equation}

  \begin{itemize}
    \item If \( v \) belongs to this path, say if \( v = u_{k_0} \), then \( u_{k_0+1} \) is a child of \( v \) that belongs to \( R \).

    In this case, \( u_{k_0+1} \) has no predecessor in \( R \), but every descendant of \( u_{k_0+1} \) in \( T \) is also a descendant in \( R \). Thus, \( R \) is an oriented tree with root \( u_{k_0+1} \).

    \item Otherwise, consider the path
    \begin{equation}\label{eq:thm:removing_node_from_oriented_tree/proof/aux_path}
      r = v_0 \to \cdots \to v_m = v.
    \end{equation}

    We can reverse \eqref{eq:thm:removing_node_from_oriented_tree/proof/main_path} and concatenate it with \eqref{eq:thm:removing_node_from_oriented_tree/proof/aux_path} to obtain a generalized path from \( u \) to \( v \):
    \begin{equation*}
      u = u_n \leftarrow \cdots \leftarrow u_0 = r = v_0 \to \cdots \to v_{m-1} \to v_m = v.
    \end{equation*}

    After removing \( v \), we obtain a generalized path from \( u \) to \( v_{m-1} \), which must belong to \( R \), the connected component containing \( u \). Then \( v_{m-1} \) is the parent of \( v \).

    In this case \( r \), the root of \( T \), is also the root of \( R \), since no node in \( R \) changes its in-degree compared to \( T \).
  \end{itemize}
\end{proof}

\begin{lemma}\label{thm:konigs_infinity_lemma_oriented}
  Every \hyperref[def:graph_cardinality/local]{locally finite} \hyperref[def:oriented_tree]{oriented tree} of \hyperref[def:graph_cardinality/order]{infinite order} has an \hyperref[def:graph_walk/path]{infinite directed path}.
\end{lemma}
\begin{comments}
  \item We use an explicit choice function here given by the \hyperref[def:zfc/choice]{axiom of choice}, but in special cases like \hyperref[def:oriented_tree]{oriented trees}, there are natural choice functions to consider.
\end{comments}
\begin{proof}
  Let \( T = (V, A) \) be a locally finite infinite oriented tree. Fix a \hyperref[def:choice_function]{choice function} \( c \) on the nodes of \( T \).

  We will build in parallel an infinite path
  \begin{equation*}
    v_0 \to v_1 \to v_2 \cdots
  \end{equation*}
  and a sequence of oriented trees \( T_0, T_1, \ldots \), each also locally finite of infinite order, such that \( T_{k+1} \) is subtree of \( T_k \) rooted at \( v_{k+1} \).

  \begin{itemize}
    \item For the base case, let \( T_0 \) be \( T \) and let \( v_0 \) be the root of \( T \).

    \item For the inductive case, suppose that we have already built \( T_k \) and
    \begin{equation*}
      v_0 \to v_1 \to \cdots \to v_k.
    \end{equation*}

    Denote by \( W \) the set of children of \( v_k \) and by \( \mscrR \) the family of trees obtained by removing the root \( v_k \) from \( T_k \). \Cref{thm:removing_node_from_oriented_tree} implies that each member of \( \mscrR \) is an oriented tree.

    Since \( \mscrR \) partitions the infinitely many nodes of \( T_k \) (excluding \( v_k \)), \cref{thm:pigeonhole_principle/infinitary} implies that at least one of the finitely many trees in \( \mscrR \) has infinite order. Denote by \( \mscrR_\infty \) be the set of all such trees.

    Finally, let
    \begin{equation*}
      v_{k+1} \coloneqq c\parens[\Big]{ \set[\Big]{ w \in W \given \qexists* {S \in \mscrR_\infty} w \in S } }
    \end{equation*}
    and let \( T_{k+1} \) be the tree in \( \mscrR_\infty \) containing \( v_{k+1} \).
  \end{itemize}

  We have thus built an infinite path in \( T \) with the aid of a predetermined choice function.
\end{proof}

\begin{lemma}[K\"{o}nig's infinity lemma]\label{thm:konigs_infinity_lemma}
  Every \hyperref[def:graph_cardinality/local]{locally finite} \hyperref[def:tree]{tree} of \hyperref[def:graph_cardinality/order]{infinite order} has an \hyperref[def:graph_walk/path]{infinite path}.
\end{lemma}
\begin{comments}
  \item We use an explicit choice function here given by the \hyperref[def:zfc/choice]{axiom of choice}, but in special cases like \hyperref[def:ordered_tree]{ordered trees}, there are natural choice functions to consider.
\end{comments}
\begin{proof}
  Let \( T = (V, E) \) be a locally finite infinite tree. Fix a \hyperref[def:choice_function]{choice function} \( c \) on the nodes of \( T \) and let \( v_0 \coloneqq c(V) \).

  \Cref{thm:oriented_and_rooted_trees/orientation} gives an orientation \( T' = (V, A) \) of \( T \) rooted at \( v_0 \). We can then use \cref{thm:konigs_infinity_lemma_oriented} with the choice function \( c \) to obtain an infinite directed path in \( T' \).
\end{proof}

\paragraph{Ordered trees}

\begin{definition}\label{def:ordered_tree}\mimprovised
  An \term[ru=упорядоченное дерево (\cite[\S 9.3.5]{Новиков2013ДискретнаяМатематика}), en=ordered tree (\cite[573]{Stanley2012EnumerativeCombinatoricsVol1})]{ordered tree} is a quadruple \( T = (V, E, r, \leq) \), where \( (V, E, r) \) is a \hyperref[def:rooted_tree]{rooted tree} and \( {\leq} \) is a \hyperref[def:partially_ordered_set]{partial order}, called the \term{sibling order} or \term{horizontal order}, such that the children of every node are \hyperref[def:partial_order_chain/chain]{totally ordered}, while non-siblings are pairwise-incomparable.

  \begin{thmenum}[series=def:ordered_tree]
    \thmitem{def:ordered_tree/leftmost_rightmost} We say that the child \( v \) of \( u \) is \term{leftmost} (resp. \term{rightmost}) if it is minimal (resp. maximal) with respect to \( {\leq} \). If \( u \) has only two children, we call them \term{left} and \term{right}.
  \end{thmenum}

  Ordered trees have the following metamathematical properties:
  \begin{thmenum}
    \thmitem{def:ordered_tree/homomorphism} A \term{homomorphism} between ordered trees is a \hyperref[def:order_function/preserving]{strictly order-preserving} \hyperref[def:rooted_tree/homomorphism]{rooted tree homomorphism}.

    \thmitem{def:ordered_tree/category} For a \hyperref[def:grothendieck_universe]{Grothendieck universe} \( \mscrU \), we consider the category of \( \mscrU \)-small ordered trees with ordered tree homomorphisms. It is \hyperref[def:concrete_category]{concrete} over both the category of rooted trees and the \hyperref[def:partially_ordered_set]{category of partially ordered sets}.
  \end{thmenum}
\end{definition}
\begin{comments}
  \item The definition in \cite[573]{Stanley2012EnumerativeCombinatoricsVol1} is similar, however it relies on recursively built trees rather than graph-theoretic trees.

  \item \incite[\S 2.3]{Holtkamp2011RootedTrees} uses \enquote{planar rooted tree} as a synonym for \enquote{ordered tree}, while \cite[573]{Stanley2012EnumerativeCombinatoricsVol1} uses \enquote{plane tree} as a synonym.
\end{comments}

\begin{definition}\label{def:ordered_tree_grafting}\mimprovised
  Fix an \hyperref[def:ordered_tree]{ordered tree} \( T \) and a leaf node \( v \) of \( T \). Fix also a family \( \seq{ T_k }_{k \in \mscrK} \) of ordered trees, indexed by a \hyperref[def:totally_ordered_set]{totally ordered set} \( \mscrK \). We will define a tree \( B_+^v(\seq{ T_k }_{k \in \mscrK}) \) as follows:
  \begin{thmenum}
    \thmitem{def:ordered_tree_grafting/union} Consider the \hyperref[def:graph_disjoint_union]{graph disjoint union} \( \parens[\Big]{ \coprod_{k \in \mscrK} T_k } \amalg T \), with the \hyperref[thm:order_category_isomorphism_properties/coproduct]{disjoint union partial order} on its nodes.

    \thmitem{def:ordered_tree_grafting/root} For every \( k \in \mscrK \), add an edge from \( \iota(v) \) to \( \iota_k(r_k) \), where \( r_k \) is the root of \( T_k \) and \( \iota \) and \( \iota_k \) are inclusions of \( T \) and of \( T_k \) into the disjoint union.

    \thmitem{def:ordered_tree_grafting/order} Extend the sibling order so that \( \iota_k(r_k) < \iota_m(r_m) \) whenever \( k < m \).
  \end{thmenum}

  We say that \( B_+^v(\seq{ T_k }_{k \in \mscrK}) \) is obtained by \term{grafting} \( \seq{ T_k }_{k \in \mscrK} \) onto \( v \).
\end{definition}
\begin{comments}
  \item In case the trees are \hyperref[def:labeled_tree]{labeled}, we must label the vertex \( \iota_k(u) \) in \( B_+^v(\seq{ T_k }_{k \in \mscrK}) \) based on its label in \( T_k \), and similarly for the labels of \( T \) itself.

  \item We base our terminology and notation on grafting products defined in \cref{def:ordered_tree_grafting_product}.
\end{comments}

\begin{remark}\label{rem:tree_grafting_nodes}
  When building \hyperref[def:ordered_tree]{ordered trees} recursively, for example in \cref{def:propositional_formula_ast}, \cref{def:lambda_term_ast} or \cref{def:propositional_natural_deduction_proof_tree}, we start with singleton trees and then graft existing trees onto new roots.

  We choose \( 0 = \varnothing \) as a \enquote{canonical} node for singleton trees. It is a natural candidate and, since we will mostly graft lists of trees onto it, due to how define disjoint unions, each node in such a tree will be tuple of natural numbers.

  The actual nodes do not matter when the nodes are \hyperref[def:labeled_set]{labeled}, which motivates \cref{def:canonical_singleton_tree}.

  Thus, the canonical abstract syntax tree for the propositional formula \( ((\synp \synvee \synq) \rightarrow \synr) \) is
  \begin{equation*}
    \begin{aligned}
      \includegraphics[page=1]{output/rem__tree_grafting_nodes}
    \end{aligned}
  \end{equation*}
  which without labels becomes
  \begin{equation*}
    \begin{aligned}
      \includegraphics[page=2]{output/rem__tree_grafting_nodes}
    \end{aligned}
  \end{equation*}

  This leads to \cref{def:ordered_tree_grafting_product}.
\end{remark}

\begin{definition}\label{def:ordered_tree_grafting_product}\mimprovised
  For any family \( \seq{ T_k }_{k \in \mscrK} \) of \hyperref[def:ordered_tree]{ordered trees} indexed by a \hyperref[def:totally_ordered_set]{totally ordered set} \( \mscrK \), we will define their \term{grafting product} \( B_+(\seq{ T_k }_{k \in \mscrK}) \) as the tree obtained by \hyperref[def:ordered_tree_grafting]{grafting} \( \seq{ T_k }_{k \in \mscrK} \) onto a singleton tree with root \( 0 \).
\end{definition}
\begin{comments}
  \item Our choice of \( 0 \) for the new root is discussed in \cref{rem:tree_grafting_nodes}.

  \item We base our terminology on \cite[\S 3.6]{Holtkamp2011RootedTrees} and our notation on the earlier \cite[9]{Hoffman2003RootedTrees}.
\end{comments}

\begin{definition}\label{def:n_ary_tree}\mcite[def. 11.1.3]{Rosen2019DiscreteMathematics}
  We say that the \hyperref[def:ordered_tree]{ordered tree} is \term{\( n \)-ary} if each node has \hi{at most} \( n \) children and \term{full \( n \)-ary} if each internal node has \hi{exactly} \( n \) children.

  For some small values of \( n \), we use the adjectives from \cref{def:operation_arity_terminology} (e.g. unary trees, binary trees).
\end{definition}

\begin{definition}\label{def:ordered_tree_enumeration}\mimprovised
  Fix a finite \hyperref[def:ordered_tree]{ordered tree} \( T = (V, E, r, \leq) \). Let \( R_1, \ldots, R_n \) be the immediate subtrees of \( T \). We will define two \hyperref[def:enumeration]{enumeration} of \( V \) --- the \term{pre-order enumeration}\fnote{This concept is not related to \hyperref[def:preordered_set]{preordered sets}.}
  \begin{equation}\label{eq:def:ordered_tree_enumeration/pre}
    \op{pre}(T) \coloneqq \parens[\Big]{ r, \op{pre}(R_1), \cdots, \op{pre}(R_n) }.
  \end{equation}
  visits each root before the children, while the \term{post-order enumeration}
  \begin{equation}\label{eq:def:ordered_tree_enumeration/post}
    \op{post}(T) \coloneqq \parens[\Big]{ \op{post}(R_1), \cdots, \op{post}(R_n), r }.
  \end{equation}
  visits the root after them.

  Additionally, if \( T \) is a \hyperref[def:n_ary_tree]{full binary tree}, we can also define the \term{in-order enumeration}
  \begin{equation}\label{eq:def:ordered_tree_enumeration/in}
    \op{in}(T) \coloneqq \parens[\Big]{ \op{in}(R_1), r, \op{in}(R_2) }.
  \end{equation}

  \begin{figure}[!ht]
    \centering
    \includegraphics[page=1]{output/def__ordered_tree_enumeration}
    \caption{A full binary tree whose \hyperref[eq:def:ordered_tree_enumeration/pre]{pre-order enumeration} is \( abcde \), whose \hyperref[eq:def:ordered_tree_enumeration/post]{post-order enumeration} is \( cdbea \) and whose \hyperref[eq:def:ordered_tree_enumeration/in]{in-order enumeration} is \( cbdae \).}
    \label{fig:def:ordered_tree_enumeration}
  \end{figure}
\end{definition}
\begin{comments}
  \item We base our definition on \cite[228]{Erickson2019Algorithms}, however instead of \enquote{pre-order traversal}, which has temporal connotations, we use \enquote{pre-order enumeration}.
\end{comments}

\begin{algorithm}[Finite ordered tree to binary tree]\label{alg:finite_ordered_tree_to_binary_tree}
  For every finite \hyperref[def:ordered_tree]{ordered tree} \( T \), we will build a \hyperref[def:n_ary_tree]{binary tree} \( B \) that has the same \hyperref[eq:def:ordered_tree_enumeration/pre]{pre-order enumeration}.

  Let \( v_1, \ldots, v_n \) be the pre-order enumeration of \( T \). We will build via recursion on \( k = 1, 2, \ldots \) a binary tree \( B_k \) enumerating \( v_1, \ldots, v_{2k} \) if \( n = 2k \) and \( v_1, \ldots, v_{2k} v_{2k+1} \) otherwise.

  \begin{figure}[!ht]
    \hfill
    \includegraphics[page=1]{output/alg__finite_ordered_tree_to_binary_tree}
    \hfill
    \includegraphics[page=2]{output/alg__finite_ordered_tree_to_binary_tree}
    \hfill
    \hfill
    \caption{The two possibilities in \fullref{alg:finite_ordered_tree_to_binary_tree/step}.}
    \label{fig:alg:finite_ordered_tree_to_binary_tree}
  \end{figure}

  \begin{thmenum}
    \thmitem{alg:finite_ordered_tree_to_binary_tree/base} Define \( B_1 \) to be the singleton tree \( v_1 \). If \( n = 1 \), yield \( B_1 \) as the result of the algorithm.

    \thmitem{alg:finite_ordered_tree_to_binary_tree/step} At step \( k > 1 \), having already constructed \( B_{k-1} \), we have two possibilities:
    \begin{thmenum}
      \thmitem{alg:finite_ordered_tree_to_binary_tree/even} If \( n = 2k \), define \( B_k \) by \hyperref[def:ordered_tree_grafting]{grafting} \( v_n \) onto \( v_{2k-1} \) and yield \( B_k \) as the result of the algorithm.

      \thmitem{alg:finite_ordered_tree_to_binary_tree/odd} Otherwise, \( n < 2k \), define \( B_k \) by grafting \( v_{2k} \) and \( v_{2k+1} \) onto \( v_{2k-1} \).

      If \( n = 2k+1 \), yield \( B_k \) as the result of the algorithm.
    \end{thmenum}
  \end{thmenum}
\end{algorithm}

\paragraph{Labeled trees}

\begin{definition}\label{def:labeled_tree}\mimprovised
  An \( L \)-\term[en=labeled tree (\cite[exerc. 11.1.38]{Rosen2019DiscreteMathematics})]{labeled tree} is a quintuple \( T = (V, E, r, \leq, l) \), where \( (V, E, r, \leq) \) is an \hyperref[def:ordered_tree]{ordered tree} and \( l: V \to L \) is a \hyperref[def:labeled_set]{labeling} of the vertices.

  Labeled trees have the following metamathematical properties:
  \begin{thmenum}
    \thmitem{def:labeled_tree/homomorphism} A \term{homomorphism} between \( L \)-labeled trees is an \hyperref[def:ordered_tree/homomorphism]{ordered tree homomorphism} that also preserves labels.

    See \cref{thm:labeled_tree_isomorphisms} for a characterization of isomorphisms.

    \thmitem{def:labeled_tree/category} For a \hyperref[def:grothendieck_universe]{Grothendieck universe} \( \mscrU \), we can consider the category of \( \mscrU \)-small \( L \)-labeled trees with \( L \)-labeled tree homomorphisms. It is \hyperref[def:concrete_category]{concrete} over the category of ordered trees.
  \end{thmenum}
\end{definition}
\begin{comments}
  \item We introduce this definition mostly for having an explicit concept for isomorphic \hyperref[con:abstract_syntax_tree]{abstract syntax tree}.

  \item For \hyperref[rem:arbitrary_kind_graph]{general graphs} and even \hyperref[def:tree]{general trees} we allow both vertices and arcs/edges to be labeled, however in the case of ordered trees we are mostly interested in the vertices.

  We have chosen the term \enquote{labeled tree} for brevity; \incite[471]{Mimram2020ProgramEqualsProof} uses the same term for generic trees with labeled nodes, while \incite[\S 2.8]{Holtkamp2011RootedTrees} uses it for rooted trees with labeled nodes.
\end{comments}

\begin{definition}\label{def:canonical_singleton_tree}\mimprovised
  For every possible \hyperref[def:labeled_set]{label} \( l \), there corresponds a singleton tree with node \( 0 \) and label \( l \). We call this the \term{canonical singleton tree} labeled by \( l \).

  The choice of node \( 0 \) is discussed in \cref{rem:tree_grafting_nodes}.
\end{definition}

\begin{proposition}\label{thm:labeled_tree_isomorphisms}
  Fix two \( L \)-labeled trees \( T \) and \( S \). A function \( f: V_T \to V_S \) is a \hyperref[def:morphism_invertibility/isomorphism]{categorical isomorphism} from \( T \) to \( S \) if and only if it is a \hyperref[thm:graph_isomorphisms/simple_undirected]{isomorphism of simple undirected graphs} and, additionally, \( u \leq_T v \) if and only if \( f(u) \leq_S f(v) \) and \( l_T(v) = l_S(f(v)) \) for all vertices.
\end{proposition}
\begin{proof}
  The conditions are adjusted so that invertibility is clear.
\end{proof}

  \subsection{Graph embeddings}\label{subsec:graph_embeddings}

\begin{definition}\label{def:quiver_geometric_realization}
  Let \( Q = (V, A, h, t) \) be a \hyperref[def:quiver]{quiver}. Our goal is to construct a \hyperref[def:topological_space]{topological space} that translates the connectivity properties of \( Q \) into their topological equivalents.

  Consider the \hyperref[def:topological_sum]{topological sum}
  \begin{equation*}
    S \coloneqq \parens[\Bigg]{ \coprod_{a \in A} [0, 1] } \amalg \parens[\Bigg]{ \coprod_{\deg(v) = 0} \set{ v } }.
  \end{equation*}

  The space \( S \) consists of disjoint unit intervals, one for each arc, and of disjoint points, one for each \hyperref[def:hypergraph/degree]{isolated vertex}.

  We now want to glue common endpoints of arcs in \( \Sigma \). We define the function
  \begin{equation}\label{eq:def:quiver_geometric_realization/rv}
    \begin{aligned}
      &R_V: V \to \pow(S), \\
      &R_V(v) \coloneqq \begin{cases}
        \set[\Big]{ (v, v) },                                                           &\deg(v) = 0 \\
        \set[\Big]{ (0, a) \given h(a) = v } \cup \set[\Big]{ (1, a) \given t(a) = v }, &\deg(v) > 0
      \end{cases}
    \end{aligned}
  \end{equation}
  and
  \begin{equation}\label{eq:def:quiver_geometric_realization/ra}
    \begin{aligned}
      &R_A: A \to \pow(\pow(S)), \\
      &R_A(a) \coloneqq \set[\Big]{ R_V(h(a)), R_V(t(a)) } \cup \set[\Big]{ \set{ (x, a) } \given 0 < x < 1 }.
    \end{aligned}
  \end{equation}

  The family
  \begin{equation*}
    X \coloneqq \bigcup \set[\Big]{ R_A(a) \given* a \in A } \cup \set[\Big]{ R_V(v) \given* \deg(v) = 0 }.
  \end{equation*}
  is a \hyperref[def:set_partition]{partition} of \( S \). For each vertex, there is a single point in \( X \) (which is a set in \( S \)) and for each arc, the interior of the arc is a subset of \( X \).

  We can endow the partition \( X \) it with a \hyperref[def:topological_quotient]{quotient topology} \( \mscrT \). We will call the topological space \( (X, \mscrT, R_V, R_A) \) endowed with the functions \( R_V \) and \( R_A \) the \term{geometric realization} of \( G \).

  \begin{thmenum}
    \thmitem{def:quiver_geometric_realization/undirected} For an \hyperref[def:undirected_multigraph]{undirected multigraph} \( G = (V, E, \mscrE) \), the geometric realization is any of the geometric realizations of its \hyperref[def:multigraph_orientation]{orientations}. This construction is dependent on a choice function, but fortunately all the geometric realizations are homeomorphic as shown in \fullref{thm:undirected_multigraph_geometric_realizations_homeomorphic}. Hence, for a lot of purposes, we can speak of \enquote{the} geometric realization of an undirected multigraph.

    \thmitem{def:quiver_geometric_realization/drawing} We will call any \hyperref[def:global_continuity]{continuous function} with domain \( (X, \mscrT) \) a \term{graph drawing}. The term \enquote{graph drawing} is not standard terminology, but unfortunately non-injective continuous images of the realization have no established name.

    \thmitem{def:quiver_geometric_realization/embedding} An injective graph drawing is called a \term{graph embedding}.

    Every graph can be embedded into \( \BbbR^3 \) as shown in \fullref{thm:quiver_can_be_embedded_into_r3}.

    \thmitem{def:quiver_geometric_realization/planar} If a graph can be embedded into \( \BbbR^2 \), we say that it is \term{planar}.

    \thmitem{def:quiver_geometric_realization/linear} If a graph can be embedded into \( \BbbR \), we say that it is \term{linear}.
  \end{thmenum}
\end{definition}

\begin{example}\label{ex:def:quiver_geometric_realization}
  We will give a few examples of \hyperref[def:quiver_geometric_realization/undirected]{quiver geometric realizations}.

  \begin{thmenum}
    \thmitem{ex:def:quiver_geometric_realization/order_zero} The geometric realization of an \hyperref[def:hypergraph/trivial]{edgeless} quiver is the discrete topological space on its vertices.

    In particular, the geometric realization of the \hyperref[def:hypergraph/trivial]{order-zero} quiver (without any arcs and edges) is the empty topological space.

    \thmitem{ex:def:quiver_geometric_realization/positive_integers} Consider the reduced positive integer graph \eqref{eq:ex:infinite_integer_graphs/positive}. We start with \( \aleph_0 \) copies of \( [0, 1] \) and glue both ends of each of them except for the first. Thus, we obtain (a space homeomorphic to)
    \begin{equation*}
      \bigcup_{k \geq 0} [k, k + 1] = [0, \infty).
    \end{equation*}

    Therefore, \eqref{eq:ex:infinite_integer_graphs/positive} is a \hyperref[def:quiver_geometric_realization/linear]{linear graph}.

    \thmitem{ex:def:quiver_geometric_realization/k3} The graph with vertices \( V = \set{ a, b, c } \) and arcs \( \set{ \overbrace{a \to b}^{e_1}, \overbrace{b \to c}^{e_2}, \overbrace{c \to a}^{e_3} } \) is more subtle.

    We start with three copies of the interval \( [0, 1] \), depicted in \eqref{eq:ex:def:quiver_geometric_realization/k3/relization} as upward-pointing arrows, and use dashed lines to connect the endpoints that we want to glue together.
    \begin{equation}\label{eq:ex:def:quiver_geometric_realization/k3/relization}
      \begin{aligned}
        \includegraphics[page=1]{output/ex__def__graph_geometric_realization.pdf}
      \end{aligned}
    \end{equation}

    After contracting the dashed lines, we obtain a topological space that can easily be \hyperref[def:quiver_geometric_realization/embedding]{embedded} into \( \BbbR^2 \). An obvious embedding corresponds to \enquote{pulling up} \( e_2 \) and \( e_3 \):
    \begin{equation}\label{eq:ex:def:quiver_geometric_realization/k3/embedding}
      \begin{aligned}
        \includegraphics[page=2]{output/ex__def__graph_geometric_realization.pdf}
      \end{aligned}
    \end{equation}

    This is only one possible embedding of the geometric realization. It is sufficient, however, for proving that the graph is \hyperref[def:quiver_geometric_realization/planar]{planar}. The underlying undirected graph is the \hyperref[ex:complete_graph]{complete graph} \( K_3 \), hence we have shown that \( K_3 \) is also planar.

    \thmitem{ex:def:quiver_geometric_realization/k4} \Cref{fig:ex:complete_graph} shows that the complete graph \( K_4 \) is planar.

    This is not-at-all obvious from its geometric realization, however.
    \begin{equation}\label{eq:ex:def:quiver_geometric_realization/k4/realization}
      \begin{aligned}
        \includegraphics[page=3]{output/ex__def__graph_geometric_realization.pdf}
      \end{aligned}
    \end{equation}

    This example shows that constructing embeddings can be a tedious task.
  \end{thmenum}
\end{example}

\begin{proposition}\label{thm:linear_quiver_equivalence}
  If finite quiver is \hyperref[def:quiver_geometric_realization/linear]{linear}, it has degree at most \( 2 \).
\end{proposition}
\begin{proof}
  Let \( Q = (V, A, t, h) \) be a quiver and let \( (X, \mscrT, R_V, R_A) \) be its geometric realization. Let \( f: X \to \BbbR \) be an injective continuous function, i.e. a topological embedding.

  Suppose that the vertex \( v \) has degree larger than \( 2 \). It is sufficient to consider the case where \( a \), \( b \) and \( c \) are distinct arcs incident to \( v \).

  We have
  \begin{equation*}
    R_V(v) \in R_A(a) \cap R_A(b) \cap R_A(c)
  \end{equation*}
  thus
  \begin{equation*}
    f(R_V(v)) \in f(R_A(a)) \cap f(R_A(b)) \cap f(R_A(c))
  \end{equation*}

  Since \( R_A(a) \), \( R_A(b) \) and \( R_A(c) \) are \hyperref[def:connected_space]{connected}, so are their images under \( f \). If \( f(R_A(a)) \) has a point to the right of \( f(R_V(v)) \), then \( f(R_A(b)) \) must be left of \( R_V(v) \) and there remains nowhere to place \( R_A(c) \).

  Therefore, \( \deg(v) \leq 2 \) and, since \( v \) was arbitrary, \( \deg(Q) \leq 2 \).
\end{proof}

\begin{proposition}\label{thm:moment_curve}
  Consider \hyperref[def:parametric_curve]{curve}
  \begin{equation*}
    \begin{aligned}
      &\gamma: \BbbR \to \BbbR^n \\
      &\gamma(t) \coloneqq (t, t^2, \ldots, t^n).
    \end{aligned}
  \end{equation*}

  For any \( t_1 < \ldots < t_n \), the points \( \gamma(t_1), \ldots, \gamma(t_n) \) are linearly independent.

  This curve is called the \term{moment curve} of dimension \( n \).
\end{proposition}
\begin{proof}
  Follows from \fullref{ex:vandermonde_matrix}.
\end{proof}

\begin{proposition}\label{thm:quiver_can_be_embedded_into_r3}
  Every finite quiver can be embedded into \( \BbbR^3 \).
\end{proposition}
\begin{proof}
  Let \( Q = (V, A, t, h) \) be a finite quiver of order \( n \). By definition of cardinality, there exists a bijection from \( n \) to \( V \).

  Place the vertices of \( Q \) along the \hyperref[thm:moment_curve]{moment curve} by using \( \gamma(k) \) as the position for the \( k \)-th vertex of \( V \). Then by \fullref{thm:moment_curve}, no four of these points are \hyperref[def:coplanar_points]{coplanar}. Hence, if we connect their vertices using a straight line where there is an arc, no two lines would intersect.

  Therefore, this is an embedding.
\end{proof}

\begin{proposition}\label{thm:def:quiver_geometric_realization}
  Let \( Q = (V, A, h, t) \) be a \hyperref[def:quiver]{quiver} and let \( (X, \mscrT, R_V, R_A) \) be its \hyperref[def:quiver_geometric_realization]{geometric realization}.

  \begin{thmenum}
    \thmitem{thm:def:quiver_geometric_realization/isolated} A vertex \( v \in V \) is isolated in \( Q \), i.e. has degree zero, if and only if \( R_V(v) \) is an isolated point of \( X \).

    \thmitem{thm:def:quiver_geometric_realization/t1} If \( Q \) is \hyperref[def:hypergraph/degree]{locally finite}, then the space \( (X, \mscrT) \) satisfies the \ref{def:separation_axioms/T1} separation axiom.
  \end{thmenum}
\end{proposition}
\begin{proof}
  \SubProofOf{thm:def:quiver_geometric_realization/isolated} For every isolated vertex \( v \), the point \( R_V(v) = \set{ (v, v) } \) is isolated by definition.

  Now suppose that \( v \) is not an isolated vertex. Then \( R_V(v) \) is defined differently in \eqref{eq:def:quiver_geometric_realization/rv}. If there exists an arc \( a \) such that \( v = h(a) \), then \( (a, 0) \in R_V(v) \) and hence there exists no neighborhood of \( R_V(v) \) disjoin from \( R_A(a) \), hence \( R_V(v) \) is not a disjoint point of \( X \). The case \( v = t(a) \) is handled analogously.

  \SubProofOf{thm:def:quiver_geometric_realization/t1} Let \( x \in X \).

  If \( x = \set{ (t, a) } \) for some arc \( a \) and \( 0 < t < 1 \), then \( \set{ x } \) is closed because \( [0, 1] \) satisfies \hyperref[def:separation_axioms/T1]{T1} and hence \( \set{ t } \) is closed in \( [0, 1] \).

  If \( x = R_V(v) = \set{ (v, v) } \) for some isolated vertex \( v \), then \( R_V(v) \) is clopen by definition.

  If \( x = R_V(v) \) for some vertex \( v \) of positive degree, then
  \begin{equation*}
    R_V(v) = \set[\Big]{ (0, a) \given h(a) = v } \cup \set[\Big]{ (1, a) \given t(a) = v }.
  \end{equation*}

  Both \( \set{ 0 } \) and \( \set{ 1 } \) are closed in \( [0, 1] \), hence \( \set{ 0, 1 } \) is also closed in \( [0, 1] \). Since \( Q \) is locally finite, \( R_V(v) \) is the union of finitely many closed sets and is thus itself closed.
\end{proof}

\begin{proposition}\label{thm:undirected_multigraph_geometric_realizations_homeomorphic}
  Let \( G = (V, E, \mscrE) \) be an \hyperref[def:undirected_multigraph]{undirected multigraph}.

  Let \( (X_c, \mscrT_c, R_{V_c}, R_{A_c}) \) and \( (X_d, \mscrT_d, R_{V_d}, R_{A_d}) \) be geometric realizations corresponding to the \hyperref[def:multigraph_orientation]{orientations} \( O_c(G) \) and \( O_d(G) \) of \( G \).

  Then \( (X_c, \mscrT_c) \) and \( (X_c, \mscrT_d) \) are homeomorphic.
\end{proposition}
\begin{proof}
  Let \( h_c \) and \( h_d \) be the head functions from the \hyperref[def:quiver]{quivers} \( O_c(G) \) and \( O_d(G) \).

  Define the function
  \begin{equation*}
    \begin{aligned}
      &f: X_c \to X_d \\
      &f(x) \coloneqq \begin{cases}
        \set[\Big]{ (0, e) \given h_d(e) = v } \cup \set[\Big]{ (1, e) \given t_d(e) = v }, &x = R_{V_c}(v) \T{and} \deg(v) > 0 \\
        \set[\Big]{ (e, 1 - t) },                                                           &x = \set{ (e, t) } \T{and} h_c(e) \neq h_d(e) \\
        x,                                                                                  &\T{otherwise.}
      \end{cases}
    \end{aligned}
  \end{equation*}

  This function \enquote{reverses} the direction of some of the intervals in the construction of the realizations and fixes everything else in place. It is clearly bijective. It is also continuous because it satisfies \fullref{def:global_continuity/closure}. Finally, it is a homeomorphism because the inverse function is defined in the same way by interchanging \( c \) and \( d \).
\end{proof}

\begin{proposition}\label{thm:quiver_geometric_realization_paths}
  Let \( Q = (V, A, h, t) \) be a \hyperref[def:quiver]{quiver} and let \( (X, \mscrT, R_V, R_A) \) be its \hyperref[def:quiver_geometric_realization]{geometric realization}.

  \begin{thmenum}
    \thmitem{thm:quiver_geometric_realization_paths/quiver_to_realization} If there exists an \hyperref[def:quiver_path/undirected]{undirected path} \( p = (v, e_1, \ldots, e_n) \) from \( s \) to \( f \), then there exists a \hyperref[def:parametric_curve]{continuous path} \( \gamma: [0, 1] \to X \) from \( R_V(s)  \) to \( R_V(f) \).

    \thmitem{thm:quiver_geometric_realization_paths/realization_to_quiver} If \( Q \) is finite and if there exists a simple continuous path \( \gamma: [0, 1] \to X \) from \( R_V(s)  \) to \( R_V(f) \), then there exists a simple undirected path from \( s \) to \( f \).
  \end{thmenum}
\end{proposition}
\begin{proof}
  The case \( s = f \) is trivial, and we assume that \( s \neq f \).

  \SubProofOf{thm:quiver_geometric_realization_paths/quiver_to_realization} Suppose that \( p \) is an undirected path from \( s \) to \( f \). We will use \hyperref[rem:induction/well_founded]{strong induction} on the length of \( p \) to show that there exists a continuous path between the points \( R_V(s) \) and \( R_V(f)  \) of \( X \).

  Suppose that the statement holds for paths of length smaller than \( n \) and let
  \begin{equation*}
    p = (v, e_1, \ldots, e_{n-1}, e_n)
  \end{equation*}
  be a path of length \( n \) from some vertex \( s \) to \( f \). The inductive hypothesis holds for the initial segment \( (e_1, \ldots, e_{n-1}) \) of \( p \), hence there exists a continuous path \( \gamma: [0, 1] \to X \) from \( s \) to an endpoint of \( e_{n-1} \).
  \begin{itemize}
    \item If both \( e_{n-1} \) and \( e_n \) are positively oriented, then \( t(e_{n-1}) = h(e_n) \) and thus \( \gamma \) is a continuous path from \( R_V(s)  \) to \( R_V(h(e_n)) \). We can then append to \( \gamma \) the continuous path from \( R_V(h(e_n)) \) to \( R_V(t(e_n)) = R_V(f)  \) to obtain a path from \( R_V(s)  \) to \( R_V(f)  \).

    \item If \( e_{n-1} \) is positively oriented but \( e_n \) is not, then \( \gamma \) is a continuous path from \( R_V(s)  \) to \( R_V(h(e_{n-1})) \). We can then append to \( \gamma \) the paths from \( R_V(h(e_{n-1})) \) to \( R_V(t(e_{n-1})) = R_V(t(e_n)) \) and from \( R_V(t(e_n)) \) to \( R_V(h(e_n)) = R_V(f)  \).

    \item Similarly, if \( e_{n-1} \) is negatively oriented but \( e_n \) is not, then \( \gamma \) is a continuous path from \( R_V(s)  \) to \( R_V(t(e_{n-1})) \), and we can append to it the paths from \( R_V(t(e_{n-1})) \) to \( R_V(h(e_{n-1})) = R_V(h(e_n)) \) and from \( R_V(h(e_n)) \) to \( R_V(t(e_n)) \).

    \item Finally, if both \( e_{n-1} \) and \( e_n \) are negatively oriented, then \( \gamma \) is a continuous path from \( R_V(s)  \) to \( R_V(t(e_{n-1})) \), and we can append to it the paths from \( R_V(t(e_{n-1})) \) to \( R_V(h(e_{n-1})) = R_V(t(e_n)) \) and from \( R_V(t(e_n)) \) to \( R_V(h(e_n)) = R_V(f)  \).
  \end{itemize}

  We have shown that there exists a continuous path from \( R_V(s) \) to \( R_V(f)  \).

  \SubProofOf{thm:quiver_geometric_realization_paths/realization_to_quiver} Suppose that \( Q \) is finite and let \( \gamma: [0, 1] \to X \) be a continuous path from \( R_V(s) \) to \( R_V(f) \). We will show that there is an undirected path from \( s \) to \( f \).

  \SubProof*{\( \gamma \) contains no isolated vertices} We have
  \begin{equation}\label{eq:thm:quiver_geometric_realization_paths/full_preimage}
    \gamma^{-1}(X) = \bigcup_{a \in A} \gamma^{-1}(R_A(a)) \cup \bigcup_{\mathclap{\deg(v) = 0}} \gamma^{-1}(R_V(v)).
  \end{equation}

  For each arc \( a \), the set \( R_A(a) \) is closed as a homeomorphic image of the unit interval \( [0, 1] \). From \fullref{thm:def:quiver_geometric_realization/t1} it follows that \( \set{ R_V(v) } \) is a closed set for every vertex \( v \in V \). Therefore, \( \gamma^{-1}(X) \) is a union of disjoint closed sets.

  If we assume that \( \gamma \) passes through an isolated vertex \( v \), then \( \gamma^{-1}(v) \) would be a nonempty closed set. Then \( \img(\gamma) = \set{ v } \) because otherwise \( [0, 1] \) would be the union of finitely many nonempty disjoint closed sets, which would contradict \fullref{def:connected_space/closed_union} because \( [0, 1] \) is \hyperref[def:connected_space]{connected}.

  But \( \gamma \) passes through at least two vertices because \( s \neq t \), and hence it doesn't pass through any isolated vertex.

  \SubProof*{\( \gamma \) contains the entirety of each arc it intersects} Suppose that \( R_A(a) \cap \img(\gamma) = \varnothing \) for some arc \( a \).

  Let \( l \coloneqq \inf\set{ 0 < t < 1 \given \gamma(t) \in R_A(a) } \) and \( r \coloneqq \sup\set{ 0 < t < 1 \given \gamma(t) \in R_A(a) } \). The closed interval \( [l, r] \) is compact, hence \( \gamma([l, r]) \) is a continuous image of a compact set and hence is itself compact.

  Since \( \Int R_A(a) \) is a subset of \( \gamma([l, r]) \) and since \( \gamma([l, r]) \) is closed, its closure \( R_A(a) \) is also a subset of \( \gamma([l, r]) \).

  Therefore, if an internal point of an arc belongs to \( \img(\gamma) \), so does the entire arc.

  \SubProof*{\( \gamma \) contains an arc} From \eqref{eq:thm:quiver_geometric_realization_paths/full_preimage} if follows that
  \begin{equation*}
    \gamma^{-1}(X)
    =
    \bigcup_{a \in A} \gamma^{-1}(R_A(a))
    \reloset{\ref{thm:topology_from_closure_operator/CO3}} =
    \cl\parens*{ \bigcup_{a \in A} \gamma^{-1}(\Int R_A(a)) }.
  \end{equation*}

  Hence, \( \img(\gamma) \) contains at least one internal point of some arc and thus the entire arc.

  \SubProof*{\( \gamma \) induces an undirected path from \( s \) to \( f \)} We have that \( \gamma(0) = R_V(s) \). By what we have already shown, \( \img(\gamma) \) contains no isolated points, hence \( \gamma \) contains the set \( R_A(a) \), where \( a \) is some arc incident to \( s \).

  Suppose that \( h(a) = s \). Then \( p = (s, a) \) is an undirected path from \( s \) to \( t(a) \). If \( t(a) = f \), the proof is finished. Otherwise, define
  \begin{equation*}
    x_0 \coloneqq \sup\set{ x \in [0, 1] \given \gamma(x) = R_V(t(a)) }
  \end{equation*}
  and
  \begin{equation*}
    \begin{aligned}
      &\delta: [0, 1] \to X, \\
      &\delta(x) \coloneqq \gamma(x_0 + (1 - x_0) x).
    \end{aligned}
  \end{equation*}

  Then \( \delta \) is a continuous path from \( R_V(t(a)) \) to \( R_V(f) \).

  We now proceed by \fullref{thm:bounded_transfinite_induction} bounded by the number of arcs to define an undirected path \( p \) from \( s \) to \( f \). Since the path \( \gamma \) is simple, it does not intersect itself and hence the image of the arc \( a \) at each step will not be in \( \delta \).
\end{proof}

\begin{corollary}\label{thm:quiver_geometric_realization_connectedness}
  A finite \hyperref[def:quiver]{quiver} is \hyperref[def:quiver_connectedness/weak]{weakly connected} if and only if its \hyperref[def:quiver_geometric_realization/undirected]{geometric realization} is \hyperref[def:path_connected]{path connected}.
\end{corollary}
\begin{proof}
  Follows from \fullref{thm:quiver_geometric_realization_paths}.
\end{proof}

\begin{corollary}\label{thm:undirected_multigraph_geometric_realization_connectedness}
  A finite \hyperref[def:undirected_multigraph]{undirected multigraph} is \hyperref[def:undirected_multigraph_connectedness]{connected} if and only if its \hyperref[def:quiver_geometric_realization/undirected]{geometric realization} is a \hyperref[def:path_connected]{path connected}.
\end{corollary}
\begin{proof}
  Follows from \fullref{thm:quiver_geometric_realization_connectedness}.
\end{proof}


  \begin{sloppypar}
    \printbibliography
  \end{sloppypar}
\end{document}
