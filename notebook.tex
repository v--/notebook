\documentclass[
  headings=standardclasses,
  bibliography=totocnumbered
]{scrartcl}

\usepackage{packages/common_packages}
\usepackage{packages/macros}
\usepackage{packages/subproofs}
\usepackage{packages/theorem_styles}
\usepackage{packages/environments}
\usepackage{packages/metapost}
\usepackage{packages/extensible_arrows}
\usepackage{packages/colors}

% Bibliography
\addbibresource{bib/books.bib}
\addbibresource{bib/papers.bib}
\addbibresource{bib/articles.bib}
\addbibresource{bib/nlab.bib}
\addbibresource{bib/proofwiki.bib}
\addbibresource{bib/mathcounterexamples.bib}
\addbibresource{bib/mathse.bib}
\addbibresource{bib/misc.bib}
\addbibresource{bib/code.bib}

% Document
\title{Notebook}
\subtitle{\URL{https://github.com/v--/notebook}}
\author{Ianis Vasilev}
\date{Build: \input{./revision}}

\makeindex[title={Index\label{sec:index}}]
\indexsetup{level=\section}

\begin{document}

\tolerance=1000
\hfuzz=3.5pt
\reversemarginpar
\maketitle

\begin{abstract}
  This ever-expanding document started as a set of study notes and exercises and gradually outgrew itself to become a slightly more encyclopedic set of study notes. Having all these notes in one place is quite helpful for both expressing my own thoughts clearly and for later reference. It is also helpful for tracking connections between seemingly unrelated concepts --- the entire document is inter-hyperlinked. Even though I aim at understanding every concept in the way it is meant to be used, most concepts are presented insomuch as they are relevant to probability and optimization.

  Since these are study notes, they will naturally have a lot of errors, so read them with caution. The document claims no referential nor pedagogical value. Everything is written at the level of abstraction I am comfortable with. Furthermore, the overall style of the document is inconsistent because I am unexperienced and periodically change my writing style. Furthermore, it is in a constant state of refactoring. Feel free to contact me if something in this document happens to distress you.

  I tried putting citations on as much things as possible. The citations themselves are usually put in the left margin. If there is no citation on a definition or theorem, that means that I have either recalled it from memory or discovered it on my own. Many of the unoriginal definitions and theorems are restated. The simple proofs are mostly original and the difficult ones are, often loosely, based on proofs from the places cited. The omitted proofs are either too trivial or too involved for me to spend time on them.
\end{abstract}

\newpage
\addcontentsline{toc}{section}{Contents}
\tableofcontents
\newpage

% Numbers
\subsection{Natural numbers}\label{subsec:natural_numbers}

\begin{definition}\label{def:peano_arithmetic}\mcite[exmpl. 17.6]{OpenLogicFull}
  Peano arithmetic (commonly abbreviated as \logic{PA}) is a \hyperref[def:first_order_theory]{theory} in \hyperref[subsec:first_order_logic]{first-order predicate logic} for describing \hyperref[def:set_of_natural_numbers]{natural numbers} and their operations. It can also be formulated in \hyperref[rem:higher_order_logic]{second-order logic} or entirely within \hyperref[sec:set_theory]{set theory} (especially considering that we are working inside an ambient \hyperref[rem:standard_model_of_set_theory]{standard} \hyperref[rem:transitive_model_of_set_theory]{transitive} model of \hyperref[def:axiom_of_universes]{\logic{ZFC+U}}), however in this document we usually give preference to the first-order logic formulation of a theory.

  Peano's original axioms referred to sets rather than first-order logic, and we actually work  \ in this document, however we prefer to use a modern formulation of Peano arithmetic.

  The \hyperref[def:first_order_language]{language} of the theory consists of
  \begin{thmenum}[series=def:peano_arithmetic]
    \thmitem{def:peano_arithmetic/zero} A constant \( 0 \) for representing \term{zero}. We can alternatively require a constant for \( 1 \), but this would lead to worse metamathematical properties as discussed in \fullref{rem:peano_arithmetic_zero}.

    \thmitem{def:peano_arithmetic/succ} A unary \hyperref[def:first_order_language/func]{functional symbol} \( s \), called the \term{successor operation}.

    The successor function is only a technicality used for establishing basic properties and for defining addition and multiplication, both in this subsection and in \fullref{def:omega_operations}.

    We will only use the abstract successor operation prior to proving the familiar properties of addition and multiplication, although we will later use the \hyperref[def:ordinal_successor]{ordinal successor operation} for building a model of \logic{PA} --- see \fullref{thm:omega_is_model_of_pa}.

    \thmitem{def:peano_arithmetic/plus} An \hyperref[rem:first_order_formula_conventions/infix]{infix} binary functional symbol \( + \) for denoting \term{addition}.

    See \fullref{thm:natural_number_addition_properties} for the algebraic properties of natural number addition.

    \thmitem{def:peano_arithmetic/mult} Another infix binary functional symbol \( \cdot \) for denoting \term{multiplication}. Outside the object language we usually use juxtaposition instead.

    See \fullref{thm:natural_number_multiplication_properties} for the algebraic properties of natural number multiplication.

    As with all \hyperref[def:semiring]{semirings}, multiplication has higher priority than addition. In the unambiguous language defined in \fullref{ex:natural_number_arithmetic_grammar/backus_naur_form}, this means that we can use the shorthand \( \xi + \eta \cdot \zeta \) for \( ((\xi \cdot \eta) + \zeta) \). We often use juxtaposition for denoting multiplication.
  \end{thmenum}

  As usual, in order to avoid parentheses, we assume that multiplication has a higher precedence and thus the right-hand side of axiom \eqref{eq:def:peano_arithmetic/PA7} should be parenthesized as \( ((\eta \cdot \xi) + \xi) \). We avoid excessive parentheses in formulas as per our convention \fullref{rem:propositional_formula_parentheses}.

  We impose the following base \hyperref[def:first_order_theory]{axioms}:
  \begin{thmenum}[resume=def:peano_arithmetic]
    \thmitem[def:peano_arithmetic/PA1]{PA1} The successor function is \hyperref[thm:function_invertibility_categorical/nonempty_left_invertible]{injective}. This can be stated as follows:
    \begin{equation}\label{eq:def:peano_arithmetic/PA1}\tag{\logic{PA1}}
      s(\xi) \doteq s(\eta) \rightarrow \xi \doteq \eta.
    \end{equation}

    We use here the convention for implicit universal quantification described in \fullref{rem:mathematical_logic_conventions/quantification}.

    \thmitem[def:peano_arithmetic/PA2]{PA2} Zero is not the successor of any natural number. Symbolically,
    \begin{equation}\label{eq:def:peano_arithmetic/PA2}\tag{\logic{PA2}}
      \neg \qexists \xi (s(\xi) \doteq 0).
    \end{equation}

    \thmitem[def:peano_arithmetic/PA3]{PA3} The \term{axiom schema of induction} roughly states that for a property to hold for all natural numbers it is sufficient for the following two conditions to be met:
    \begin{itemize}
      \item The property holds for \( 0 \).
      \item We can prove that is holds for any number by assuming that it holds for its predecessor.
    \end{itemize}

    See the proof of \fullref{thm:nonzero_natural_numbers_have_predecessors} for a detailed discussion.

    To describe this formally, we state that for any \hyperref[def:first_order_language/var]{variables} \( \xi \) and \( \eta \) and any formula \( \varphi \) not containing \underline{\( \eta \)} as a \hyperref[def:first_order_syntax/formula_free_variables]{free variable}, the following is an axiom:
    \begin{equation}\label{eq:def:peano_arithmetic/PA3}\tag{\logic{PA3}}
      \parens[\Big]
        {
          \underbrace{\varphi[\xi \mapsto 0]}_{\T{base case}}
          \wedge
          \qforall \eta \parens[\Big]
            {
              \overbrace
                {
                  \underbrace{ \varphi[\xi \mapsto \eta] }_{\mathclap{\substack{\T{inductive} \\ \T{hypothesis}}}}
                  \rightarrow
                  \underbrace{ \varphi[\xi \mapsto s(\eta)] }_{\mathclap{\substack{\T{inductive step} \\ \T{conclusion}}}}
                }^{\T{inductive step}}
            }
        }
      \rightarrow
      \underbrace{ \qforall \eta \varphi[\xi \mapsto \eta] }_{\T{conclusion}}.
    \end{equation}

    It is important to highlight that \( \varphi \) may have any set of free variables, as long as \( \eta \) is not among them. As explained in \fullref{rem:mathematical_logic_conventions/quantification}, we avoid excessive universal quantification. Of course, the axiom is only interesting if \( \xi \in \boldop{Free}(\varphi) \). If \( \zeta_1, \ldots, \zeta_n \) are all the other free variables of \( \varphi \), then the \hyperref[thm:implicit_universal_quantification]{universal closure} of the corresponding axiom is
    \begin{equation}\label{eq:def:peano_arithmetic/PA3_quantified}\tag{PA3'}
      \qforall {\zeta_1} \cdots \qforall {\zeta_n}
      \parens[\Bigg]
      {
        \parens[\Big]
          {
            \underbrace{\varphi[\xi \mapsto 0]}_{\T{base case}}
            \wedge
            \qforall \eta \parens[\Big]
              {
                \overbrace
                  {
                    \underbrace{ \varphi[\xi \mapsto \eta] }_{\mathclap{\substack{\T{inductive} \\ \T{hypothesis}}}}
                    \rightarrow
                    \underbrace{ \varphi[\xi \mapsto s(\eta)] }_{\mathclap{\substack{\T{inductive step} \\ \T{conclusion}}}}
                  }^{\T{inductive step}}
              }
          }
        \rightarrow
        \underbrace{ \qforall \eta \varphi[\xi \mapsto \eta] }_{\T{conclusion}}
      }.
    \end{equation}

    Thus, the axiom holds for any assignment for the variables \( \zeta_1, \ldots, \zeta_n \). For this reason, we call these variables \term{parameters}. Parameters in axiom schemas are further discussed in \fullref{def:set_builder_notation} in relation to comprehension in set theory.

    See \fullref{rem:induction} for a more detailed discussion of induction in general and \fullref{thm:omega_recursion} for the corresponding recursion principle.
  \end{thmenum}

  The theory we obtain without the binary operations and with only the axioms \eqref{eq:def:peano_arithmetic/PA1}-\eqref{eq:def:peano_arithmetic/PA3} is itself sometimes called Peano arithmetic. The operations are defined inductively, however, and there is no way for us to formalize them within the object theory without adding them to the language and theory itself.

  \begin{thmenum}[resume=def:peano_arithmetic]
    \thmitem[def:peano_arithmetic/PA4+5]{PA4+5} The next two axioms inductively define how addition is supposed to work:
    \begin{align}
      \xi + 0       &\doteq \xi           \label{eq:def:peano_arithmetic/PA4}\tag{\logic{PA4}} \\
      \xi + s(\eta) &\doteq s(\xi + \eta) \label{eq:def:peano_arithmetic/PA5}\tag{\logic{PA5}}
    \end{align}

    \thmitem[def:peano_arithmetic/PA6+7]{PA6+7} The final two axioms are for multiplication:
    \begin{align}
      \xi \cdot 0       &\doteq 0                    \label{eq:def:peano_arithmetic/PA6}\tag{\logic{PA6}} \\
      \xi \cdot s(\eta) &\doteq \xi \cdot \eta + \xi \label{eq:def:peano_arithmetic/PA7}\tag{\logic{PA7}}
    \end{align}
  \end{thmenum}
\end{definition}

\begin{remark}\label{rem:peano_arithmetic_zero}
  It is common to consider the first natural numbers to be \( 0 \) (e.g. \cite[exmpl. 17.6]{OpenLogicFull}). Peano himself, however, considered \( 1 \) to be the first natural number - see \cite[1]{Peano1889}.

  Whether \( 0 \) is considered to be a natural number is a matter of convention. The operations defined via \eqref{eq:def:peano_arithmetic/PA4}-\eqref{eq:def:peano_arithmetic/PA7} can be modified to work if \( 1 \) was instead the first natural number.

  We make choose for \( \BbbN \) to start with \( 0 \), however we often avoid referring to the set \( \BbbN \) of natural numbers and instead rely on the concepts \enquote{nonnegative} and \enquote{positive} integers formally defined in \fullref{def:integer_ordering}.
\end{remark}

\begin{definition}\label{def:set_of_natural_numbers}
  We define the set of \term[bg=естествени числа,ru=натуральные числа]{natural numbers} \( \BbbN \) as the \hyperref[thm:smallest_inductive_set_existence]{smallest inductive set} \( \omega \) with the \hyperref[def:first_order_structure/interpretation]{interpretation} described in \fullref{thm:omega_is_model_of_pa}.

  We do not depend on any particular properties of \( \omega \), but we use it because our construction of it is careful and purposely does not use natural numbers to avoid circularity. We are working in an ambient \hyperref[rem:standard_model_of_set_theory]{standard} \hyperref[rem:transitive_model_of_set_theory]{transitive} model of \hyperref[def:axiom_of_universes]{\logic{ZFC+U}} and hence we will conflate \( \BbbN \) with \( \omega \) as sets, however the first is also a \hyperref[def:first_order_structure]{structure of first-order logic}.

  We use the usual notation
  \begin{align*}
    0 &\coloneqq \varnothing \\
    1 &\coloneqq \op{succ}(\varnothing) = \set{ \varnothing } \\
    2 &\coloneqq \op{succ}(\op{succ}(\varnothing)) = \set{ \varnothing, \set{ \varnothing } } \\
      &\vdots
  \end{align*}
  and continue to use the notation functional symbols from \fullref{def:peano_arithmetic}, however we now denote the corresponding interpretations in the structure \( \BbbN \).

  See \fullref{ex:natural_number_arithmetic_grammar/backus_naur_form} for a simple \hyperref[def:formal_grammar]{grammar} that produces numeric symbols in their decimal notation.
\end{definition}

\begin{remark}\label{rem:standard_models_of_arithmetic}
  At this point, we have two kinds of natural numbers:
  \begin{itemize}
    \item We have natural numbers within the metatheory. This is our mental model of the natural numbers, and it is used for distinguishing between \enquote{unary} functional symbols like \( s \) and \enquote{binary} functional symbols like \( + \). This is mostly used within logic itself.

    \item We have the set of natural numbers \( \BbbN \) defined in \fullref{def:set_of_natural_numbers}. These are the numbers which we have defined formally, whose properties we study and the numbers which we use in the entire document. The properties of \( \BbbN \) help us develop a better mental model, which in turn changes our perception of the natural numbers within the metatheory.
  \end{itemize}

  We want the two sets of natural numbers to coincide. This is important when talking about, for example, \hyperref[def:sequence]{sequences}. If a number in \( \BbbN \) is not a natural number within the metatheory, we say that it is \term{nonstandard}. The existence of nonstandard models is guaranteed by \fullref{thm:upward_lowenheim_skolem_theorem}. There cannot be numbers in the metatheory that are not in \( \BbbN \) because a model of \logic{PA} cannot have a finite domain and the natural numbers are the smallest metalogical infinite set.

  A model of \logic{PA} which contains precisely the numbers in the metatheory is called a \term{standard model}. For the purpose of this document, it is sufficient to accept the convention that \( \BbbN \) is a standard model of \logic{PA}.
\end{remark}

\begin{proposition}\label{thm:nonzero_natural_numbers_have_predecessors}
  Every nonzero natural number has a unique predecessor. More precisely, zero has no predecessor and for any nonzero number \( n \) there exists a unique number \( m \) such that \( n = s(m) \). We will denote this predecessor by \( p(n) \).
\end{proposition}
\begin{proof}
  This proof is exemplar because it clearly demonstrates both the distinction between inductive and deductive reasoning and the role of the main three axioms.

  \begin{itemize}
    \item The axiom \eqref{eq:def:peano_arithmetic/PA1} states that the function \( s \) is injective. By the equivalences in \fullref{def:function_invertibility/injective} its \hyperref[def:multi_valued_function/inverse]{inverse multi-valued function} is actually a \hyperref[def:partial_function]{single-valued partial function}. Denote this inverse by \( p \).

    \item The axiom \eqref{eq:def:peano_arithmetic/PA2} states that the function \( s \) is not surjective. By the equivalences in \fullref{def:function_invertibility/surjective}, the inverse \( p \) is not a \hyperref[def:multi_valued_function/total]{total function}.
  \end{itemize}

  What we have shown up until this point in the proof is deductive --- we have restated the first two axioms of \logic{PA} and used some equivalent conditions that allowed us to deduce properties of the inverse function \( p \) of \( s \). We did all of this by following the precise rules of \hyperref[def:classical_logic]{classical logic} described formally in \fullref{subsec:deductive_systems}. This reasoning emulates \hyperref[eq:def:def:axiomatic_deductive_system/mp]{modus ponens}.

  Now we will show that every nonzero natural number has a predecessor. That is, that the function \( p \) is not defined only at \( 0 \). To highlight the logical structure of this proof, we will use \hyperref[def:first_order_natural_deduction_system]{first-order natural deduction} rather than work with the model \( \BbbN \) of \logic{PA}.

  Denote by \( \theta \) the formula
  \begin{equation*}
    \xi \doteq 0 \vee \qexists \zeta (\xi \doteq s(\zeta)).
  \end{equation*}

  Clearly \( \xi \) is the only free variable in \( \theta \). We want to derive the formula \( \qforall \eta \theta[\xi \mapsto \eta] \) from the axioms of \logic{PA}.

  In this part of the proof we will use inductive reasoning. This will highlight that \eqref{eq:def:peano_arithmetic/PA3} is not an axiom schema about specifying properties, but rather about introducing a proof technique that does not hold for general \hyperref[def:first_order_theory]{logical theories}. We will not attempt to prove \( \qforall \eta \theta[\xi \mapsto \eta] \) directly. Instead, we will prove a more complicated formula that is easier to prove and then by one of the many induction principles, it will follow that our desired result holds.

  We can deduce the following \hyperref[def:proof_derivability]{logical theorem}:
  \begin{equation*}
    \begin{prooftree}
      \infer0[\eqref{eq:def:first_order_natural_deduction_system/equality/intro}]{ (\xi \doteq s(\zeta))[\zeta \mapsto \eta, \xi \mapsto s(\eta)] }
      \infer1[\eqref{eq:def:first_order_natural_deduction_system/exists/intro}]{ \parens[\Big]{ \qexists \zeta (\xi \doteq s(\zeta) }[\xi \mapsto s(\eta)] }
      \infer1[\eqref{eq:def:minimal_propositional_natural_deduction_system/or/intro_right}]{ \theta[\xi \mapsto s(\eta)] }
      \infer1[\eqref{eq:def:minimal_propositional_natural_deduction_system/imp/intro}]{ \theta[\xi \mapsto \eta] \rightarrow \theta[\xi \mapsto s(\eta)] }
      \infer1[\eqref{eq:def:first_order_natural_deduction_system/forall/intro}]{ \qforall \eta (\theta[\xi \mapsto \eta] \rightarrow \theta[\xi \mapsto s(\eta)]) }

      \infer0[\eqref{eq:def:first_order_natural_deduction_system/equality/intro}]{ (\xi \doteq 0)[\xi \mapsto 0] }
      \infer1[\eqref{eq:def:minimal_propositional_natural_deduction_system/or/intro_left}]{ \theta[\xi \mapsto s(\eta)] }

      \infer2[\eqref{eq:def:minimal_propositional_natural_deduction_system/and/intro}]{ \theta[\xi \mapsto 0] \wedge \qforall \eta \parens[\Big] { \theta[\xi \mapsto \eta] \rightarrow \theta[\xi \mapsto s(\eta)] } }
    \end{prooftree}
  \end{equation*}

  This is precisely the antecedent of the instance of \eqref{eq:def:peano_arithmetic/PA3} with \( \varphi = \theta \). By \fullref{thm:syntactic_deduction_theorem} we have
  \begin{equation*}
    \eqref{eq:def:peano_arithmetic/PA3} \vdash \qforall \eta \theta[\xi \mapsto \eta].
  \end{equation*}

  When interpreted in \( \BbbN \), this formula \( \qforall \eta \theta[\xi \mapsto \eta] \) simply states that every natural number is either zero or has a predecessor. The statement does not concern itself with uniqueness nor with whether \( 0 \) has a predecessor.

  But we have already shown uniqueness --- the predecessor function \( p \) is a partial single-valued function. And we have shown that \( p \) is not defined at zero. The last part of the proof shows \( p \) is defined for all nonzero values.

  We may choose to define \( p \) at zero by giving it a sentinel value. This is precisely the technique we use in \fullref{thm:function_invertibility_categorical/nonempty_left_invertible} to show that \( s \) has a left inverse if it is injective. We can use only \eqref{eq:def:peano_arithmetic/PA1} to show that \( s \) is injective and then pick \( p \) to be any of its left inverses. We also want \( p \) to be as close as possible to a right inverse, however. The latter, as we have seen, is more tricky.
\end{proof}

\begin{proposition}\label{thm:natural_number_addition_properties}
  The \hyperref[def:set_of_natural_numbers]{natural numbers} \( \BbbN \) with \hyperref[def:peano_arithmetic/plus]{addition} form a \hyperref[def:magma/cancellative]{cancellative} \hyperref[def:magma/commutative]{commutative} \hyperref[def:zerosumfree]{zerosumfree} \hyperref[def:monoid]{monoid} with \( 0 \) as the identity.

  Furthermore, the sum of two natural numbers is nonzero if and only if both numbers are nonzero, that is,
  \begin{equation}\label{eq:thm:natural_number_addition_properties/nonzero_sum}
    n + m = 0 \T{if and only if} n = 0 \T{and} m = 0.
  \end{equation}

  This generalizes to \fullref{thm:cardinal_addition_algebraic_properties} and \fullref{thm:cardinal_addition_algebraic_properties}.
\end{proposition}
\begin{proof}
  \SubProofOf[def:magma/commutative]{commutativity} Consider the sum \( n + m \). We use induction on \( m \) to prove its commutativity.
  \begin{itemize}
    \item If \( m = 0 \), nested induction by \( n \) yields:
    \begin{itemize}
      \item If \( n = m = 0 \), clearly \( n + m = 0 + 0 = m + n \).
      \item If the inductive hypothesis holds for its predecessor \( p(n) \),
      \begin{balign*}
        n + m
        &=
        n + 0
        = \\ &=
        s(p(n)) + 0
        \reloset {\eqref{eq:def:peano_arithmetic/PA4}} = \\ &=
        s(p(n))
        \reloset {\eqref{eq:def:peano_arithmetic/PA4}} = \\ &=
        s(p(n) + 0)
        = \\ &=
        s(p(n) + m)
        \reloset {\T{ind.}} = \\ &=
        s(m + p(n))
        \reloset {\eqref{eq:def:peano_arithmetic/PA5}} = \\ &=
        m + s(p(n))
        =
        m + n.
      \end{balign*}
    \end{itemize}

    \item If \( m \neq 0 \) and if the inductive hypothesis holds for \( p(m) \), \eqref{eq:def:peano_arithmetic/PA1} yields that \( n + m = m + n \) if and only if \( n + p(m) = p(m) + n \). But the last equality is satisfied because of the inductive hypothesis, hence commutativity of \( n \) and \( m \) follows.
  \end{itemize}

  \SubProofOf[def:magma/associative]{associativity} Fix natural numbers \( n \), \( m \), \( k \). We will prove associativity by induction on \( k \). If \( k = 0 \), we have
  \begin{equation*}
    (n + m) + 0
    \reloset {\eqref{eq:def:peano_arithmetic/PA4}} =
    n + m
    \reloset {\eqref{eq:def:peano_arithmetic/PA4}} =
    n + (m + 0).
  \end{equation*}

  If \( k \neq 0 \), the proof follows directly from \eqref{eq:def:peano_arithmetic/PA1} as in the proof of commutativity.

  \SubProofOf[def:monoid]{identity} We have \( n + 0 = n \) by \eqref{eq:def:peano_arithmetic/PA4} and \( 0 + n = n \) by commutativity.

  \SubProofOf[def:magma/cancellative]{cancellation} Let \( n + k = m + k \). We will prove that \( n = m \) by induction. This is obvious for \( k = 0 \). For \( k \neq 0 \) we have
  \begin{equation*}
    n + s(p(k))
    =
    n + k
    =
    m + k
    =
    m + s(p(k)),
  \end{equation*}
  which by \eqref{eq:def:peano_arithmetic/PA5} is equivalent to \( s(n + p(k))) = s(m + p(k))) \).

  By \eqref{eq:def:peano_arithmetic/PA1}, we have \( n + p(k) = m + p(k) \). The inductive hypothesis implies that \( n = m \).

  \SubProofOf[def:zerosumfree]{zerosumfree} We will use induction on \( m \) in \( n + m = 0 \).
  \begin{itemize}
    \item If \( m = 0 \), then \( n + m = n \) by \eqref{eq:def:peano_arithmetic/PA4}, and hence \( n = 0 \).

    \item If \( m > 0 \), then \( n + m = s(n + p(m)) \) by \eqref{eq:def:peano_arithmetic/PA5}, and by \eqref{eq:def:peano_arithmetic/PA2}, \( n + m \neq 0 \).
  \end{itemize}

  Therefore, \( n + m = 0 \) if and only if \( n = m = 0 \).
\end{proof}

\begin{remark}\label{rem:natural_number_multiplication}
  \hyperref[rem:additive_magma/multiplication]{Multiplication in commutative monoids} (i.e. monoid exponentiation) is defined in \fullref{def:monoid/exponentiation} for a natural number and a monoid member. It just to happens that, by \fullref{thm:natural_number_addition_properties}, the natural numbers are themselves a monoid. We cannot rely on \fullref{thm:magma_exponentiation_properties}, however, if we want to avoid circular definitions and proofs.

  Having multiplication as part of the signature of \hyperref[def:peano_arithmetic]{Peano arithmetic} allows us to avoid this circularity.
\end{remark}

\begin{proposition}\label{thm:natural_number_multiplication_properties}
  The \hyperref[def:set_of_natural_numbers]{natural numbers} \( \BbbN \) with \hyperref[def:peano_arithmetic/mult]{multiplication} form a \hyperref[def:magma/commutative]{commutative} \hyperref[def:monoid]{monoid} with \( 1 \) as the identity.

  When combined with addition, the natural numbers become an \hyperref[def:entire_semiring]{entire} \hyperref[def:semiring/commutative]{commutative semiring}.

  This generalizes to \fullref{thm:order_multiplication_algebraic_properties} and \fullref{thm:cardinal_multiplication_algebraic_properties}.
\end{proposition}
\begin{proof}
  \SubProofOf[def:monoid]{identity} Multiplication by \( 1 \) on the right preserves any natural number:
  \begin{equation*}
     n \cdot 1
     \reloset{\eqref{eq:def:peano_arithmetic/PA7}} =
     n \cdot 0 + n
     \reloset{\eqref{eq:def:peano_arithmetic/PA6}} =
     0 + n
     =
     n.
  \end{equation*}

  Multiplication from the left is handled by induction. Indeed, the case \( n = 0 \) is trivial and for nonzero \( n \) we have
  \begin{equation*}
     1 \cdot n
     \reloset{\eqref{eq:def:peano_arithmetic/PA7}} =
     1 \cdot p(n) + 1
     \reloset{\eqref{eq:def:peano_arithmetic/PA6}} =
     p(n) + 1
     =
     n.
  \end{equation*}

  \SubProofOf[def:semiring/left_distributivity]{distributivity} We will prove that \( (n + m)k = n \cdot k + n \cdot k \) with induction on \( k \).

  If \( k = 0 \),
  \begin{equation*}
    (n + m) \cdot 0
    \reloset{\eqref{eq:def:peano_arithmetic/PA6}} =
    0
    \reloset{\eqref{eq:def:peano_arithmetic/PA4}} =
    0 + 0
    \reloset{\eqref{eq:def:peano_arithmetic/PA6}} =
    n \cdot 0 + m \cdot 0.
  \end{equation*}

  For all nonzero \( k \), if the inductive hypothesis holds for \( p(k) \), then
  \begin{balign*}
    (n + m) \cdot k
    &\reloset*{\eqref{eq:def:peano_arithmetic/PA7}} =
    (n + m) + (n + m) \cdot p(k)
    \reloset {\T{ind.}} = \\ &=
    (n + m) + n \cdot p(k) + n \cdot p(k)
    = \\ &=
    (n + n \cdot p(k)) + (m + m \cdot p(k))
    \reloset{\eqref{eq:def:peano_arithmetic/PA7}} = \\ &=
    n \cdot k + m \cdot k.
  \end{balign*}

  \SubProofOf[def:magma/associative]{associativity} With distributivity proven, associativity of multiplication follows by induction. Indeed,
  \begin{equation*}
    (n \cdot m) \cdot k = n \cdot (m \cdot k)
  \end{equation*}
  is trivially satisfied for \( k = 0 \) and for all nonzero \( k \), whenever the inductive hypothesis holds for all \( n, m \in \BbbN \), it follows that
  \begin{balign*}
    (n \cdot m) \cdot k
    &\reloset*{\eqref{eq:def:peano_arithmetic/PA7}} =
    n \cdot m + (n \cdot m) \cdot p(k)
    \reloset {\T{ind.}} = \\ &=
    n \cdot m + n \cdot (m \cdot p(k))
    \reloset{\eqref{eq:def:semiring/left_distributivity}} = \\ &=
    n \cdot (m + m \cdot p(k))
    \reloset{\eqref{eq:def:peano_arithmetic/PA7}} = \\ &=
    n \cdot (m \cdot k).
  \end{balign*}

  \SubProofOf[def:magma/commutative]{commutativity} By induction on \( m \) we prove
  \begin{equation*}
    n \cdot m = m \cdot n.
  \end{equation*}

  The base case is trivial for nonzero \( m \), if \( n \cdot p(m) = p(m) \cdot n \) for all \( n \in \BbbN \), then
  \begin{equation*}
    n \cdot m
    \reloset{\eqref{eq:def:peano_arithmetic/PA7}} =
    n + n \cdot p(m)
    \reloset {\T{ind.}} =
    n + p(m) \cdot n
    \reloset{\eqref{def:semiring/left_distributivity}} =
    (1 + p(m)) \cdot n
    \reloset{\eqref{eq:def:peano_arithmetic/PA5}} =
    m \cdot n.
  \end{equation*}

  \SubProof{Proof of no zero divisors} We will now show that \( \BbbN \) has no zero divisors.

  If \( m = 0 \), then \( n \cdot m = 0 \) by \eqref{eq:def:peano_arithmetic/PA6}. If \( n = 0 \), then by induction on \( m \) we can easily show that \( n \cdot m = n \cdot p(m) + n = 0 + 0 = 0 \).

  Conversely, let \( n \cdot m = 0 \). By induction on \( m \), either \( m = 0 \) or we have \( n \cdot m = n \cdot p(m) + n \), in which case \eqref{eq:thm:natural_number_addition_properties/nonzero_sum} both \( n \cdot p(m) \) and \( n \) are zero. Thus, if we assume that \( m \neq 0 \), then we can conclude that \( n = 0 \).
\end{proof}

\begin{remark}\label{rem:natural_number_successor_via_addition}
  In \cite[1]{Peano1889}, Peano defined an \enquote{\( n \mapsto n + 1 \)} operation rather than a successor operation. It has since become common practice to instead define a \enquote{successor} operation, define addition and then show that the two are compatible:
  \begin{equation*}
    n + 1
    \reloset {\eqref{eq:def:peano_arithmetic/PA5}} =
    s(n + 0)
    \reloset {\eqref{eq:def:peano_arithmetic/PA4}} =
    s(n).
  \end{equation*}

  The predecessor operation then corresponds to integer subtraction by \( 1 \). We avoid subtraction in this subsection --- we are only interested in the fact that every nonzero natural number \( n \) has a predecessor \( m \) such that \( n = m + 1 \).

  It is dangerous to conflate \( n + 1 \) and \( s(n) \) until we have proved the familiar properties of addition. We have already done, so in \fullref{thm:natural_number_addition_properties}, however, and will further avoid mentioning directly the operations \( s(n) \) and \( p(n) \).

  See \fullref{rem:ordinal_successor_via_addition} for the more general case of ordinal addition.
\end{remark}

\begin{definition}\label{def:natural_number_ordering}
  Although this is sometimes stated as part of \logic{PA}, e.g. \cite[exmpl. 17.6]{OpenLogicFull}, we can define the familiar order \( \leq \) on the natural numbers via addition as the \hyperref[rem:predicate_formula]{predicate formula}
  \begin{equation}\label{eq:def:natural_number_ordering/predicate}
    \alpha \leq \beta \coloneqq \qexists \xi \parens[\Big]{ \alpha + \xi \doteq \beta }.
  \end{equation}

  We use the infix notation for convenience, however we do not assume that \( \leq \) is part of the language of \logic{PA} (as explained in \fullref{rem:first_order_formula_conventions/necessary_signature}).

  The following relation
  \begin{equation}\label{eq:def:natural_number_ordering/strict_predicate}
    \alpha < \beta \coloneqq \qexists {\xi \neq 0} \parens[\Big]{ \alpha + \xi \doteq \beta }.
  \end{equation}
  is then connected to \( \leq \) via \eqref{def:partially_ordered_set/compatibility_strict}.

  For the specific model of \logic{PA} based on the smallest inductive set \( \omega \), the latter relation \( < \) is equivalent to \( \in \) as discussed in \fullref{rem:ordinal_definition}.

  We will show in \fullref{thm:natural_numbers_are_well_ordered} that \( \BbbN \) is \hyperref[def:totally_ordered_set]{total ordered} (even \hyperref[def:well_ordered_set]{well-ordered}) with \( \leq \) as the nonstrict order and \( < \) as the strict order.
\end{definition}
\begin{proof}
  We will show that \( n < m \) if and only if \( n \leq m \) and \( n \neq m \).

  \SufficiencySubProof Assume that \( n < m \). Then there exists some nonzero \( a \) such that \( n + a = m \). In particular, we have \( n \leq m \). If we suppose that \( n = m \), then \( n + a = m \) and since addition is cancellative, it would follow that \( a = 0 \), contradicting the assumption that \( a \) is nonzero.

  Therefore, \( n \leq m \) and \( n \neq m \).

  \NecessitySubProof Assume that \( n \leq m \) and \( n \neq m \). Then there exists some \( a \) such that \( n + a = m \) If we suppose that \( a = 0 \), then we would obtain that \( n = m \), which would contradict our choice of \( n \) and \( m \).

  Therefore, \( n < m \).
\end{proof}

\begin{proposition}\label{thm:natural_numbers_are_well_ordered}
  The natural numbers are \hyperref[def:well_ordered_set]{well-ordered} by the relation \( < \) defined by \eqref{eq:def:natural_number_ordering/strict_predicate}.

  Furthermore, \( \BbbN \) is an \hyperref[def:ordered_semiring]{ordered semiring}. That is, the nonstrict order \( \leq \) is compatible with addition and multiplication.
\end{proposition}
\begin{proof}
  As discussed in \fullref{def:well_ordered_set}, in order to show that \( < \) well-orders \( \BbbN \), we only need to show that \( < \) is \hyperref[def:binary_relation/transitive]{transitive}, satisfies \hyperref[def:binary_relation/trichotomic]{trichotomy} and does not allow an infinitely descending chain.

  \SubProofOf[def:binary_relation/transitive]{transitivity} Let \( n < m \) and \( m < k \). Then there exist nonzero numbers \( a \) such that \( n + a = m \) and \( b \) such that \( m + b = k \). Thus, \( n + a + b = k \), which demonstrates that \( n \leq k \). Furthermore, because \( \BbbN \) is zerosumfree, it also follows that \( a + b \neq 0 \).

  Therefore, \( n < k \).

  \SubProofOf[def:binary_relation/trichotomic]{trichotomy} Let \( n \) and \( m \) be natural numbers.

  We have already shown in \fullref{def:natural_number_ordering} that due to \eqref{def:partially_ordered_set/compatibility_strict} the equality \( n = m \) holds if and only if neither \( n < m \) nor \( n > m \).

  Aiming at a contradiction, suppose that both \( n < m \) and \( n > m \) hold. There must exist nonzero numbers \( a \) and \( b \) such that \( n + a = m \) and \( n = m + b \). Then
  \begin{equation*}
    n = m + b = (n + a) + b.
  \end{equation*}

  Since addition is cancellative, we have \( a + b = 0 \). Therefore, \( n = m \), which as we have shown is incompatible with neither \( n < m \) nor \( n > m \).

  Therefore, at most one of the three conditions \( n = m \), \( n < m \) or \( n > m \) holds.

  We will use induction on \( m \) to show that at least one of the conditions hold. If \( m = 0 \), then either \( n = 0 \) and \( m = n \) or \( n \neq 0 \) and \( m < n \). Now suppose that the inductive hypothesis holds for \( m \). We will show that it also holds for \( m + 1 \).
  \begin{itemize}
    \item If \( n = m \), then clearly \( n < m + 1 \).
    \item If \( n < m \), then since \( m < m + 1 \) by transitivity of \( < \), we have \( n < m + 1 \).
    \item If \( n > m \), then there exists some nonzero \( a \) such that \( n = m + a \). If \( a = 1 \), then \( n = m + 1 \). If \( a \) is neither \( 0 \) nor \( 1 \), then \( n > m + 1 \).
  \end{itemize}

  \SubProofOf[def:well_founded_relation]{well-foundedness} We will show by induction on \( n \) that an infinitely descending chain ending at \( n \) cannot exist.

  If \( n = 0 \), by \eqref{eq:def:peano_arithmetic/PA2} \( n \) has no predecessor and thus there cannot exist a natural number \( m \) such that \( m < n \).

  Now assume that the inductive hypothesis holds for \( n \) and suppose that there exists an infinitely descending chain ending in \( n + 1 \):
  \begin{equation}\label{eq:thm:natural_numbers_are_well_ordered/descending_chain}
    \cdots < k < m < n + 1.
  \end{equation}

  If \( m = n \), it follows that
  \begin{equation*}
    \cdots < k < n
  \end{equation*}
  is an infinitely descending chain ending in \( n \).

  If \( m < n \), then
  \begin{equation*}
    \cdots < k < m < n
  \end{equation*}
  is again an infinitely descending chain ending in \( n \).

  By the inductive hypothesis, a chain ending at \( n \) cannot exist, therefore neither does \eqref{eq:thm:natural_numbers_are_well_ordered/descending_chain}.

  \SubProofOf[def:ordered_magma]{compatibility with addition} We will show that the nonstrict order is compatible with addition in \( \BbbN \). Let \( n \leq m \) and let \( k \) be an arbitrary natural number. Since \( n \leq m \), there exists a number \( a \) such that \( n + a = m \). Then
  \begin{equation*}
    m + k = (n + a) + k = (n + k) + a.
  \end{equation*}

  Therefore,
  \begin{equation*}
    n + k \leq m + k.
  \end{equation*}

  \SubProofOf[def:ordered_semiring]{compatibility with multiplication} If \( n \geq 0 \) and \( m \geq 0 \), then \( n \cdot m \geq 0 \) for the simple reason that all natural numbers are greater than or equal to zero.
\end{proof}

\begin{proposition}\label{thm:natural_number_divisibility_lattice}
  The semiring \hyperref[def:set_of_natural_numbers]{\( \BbbN \)} of natural numbers is a \hyperref[def:semilattice/bounded]{bounded lattice} with respect to \hyperref[def:divisibility]{semiring divisibility}. Explicitly:
  \begin{thmenum}
    \thmitem{thm:natural_number_divisibility_lattice/join} The \hyperref[def:semilattice/join]{join} of \( n \) and \( m \) is their \hyperref[def:gcd_and_lcm]{least common multiple} \( \lcm\set{ n, m } \) (defined via \fullref{alg:euclidean_algorithm} and \fullref{thm:gcd_and_lcm}).

    \thmitem{thm:natural_number_divisibility_lattice/bottom} The \hyperref[def:partially_ordered_set_extremal_points/top_and_bottom]{bottom element} is \( 1 \) since \( 1 \) divides every natural number.

    \thmitem{thm:natural_number_divisibility_lattice/meet} Dually, the \hyperref[def:semilattice/meet]{meet} of \( n \) and \( m \) is their \hyperref[def:gcd_and_lcm]{greatest common divisor} \( \gcd\set{ n, m } \) (defined via \fullref{alg:euclidean_algorithm}).

    \thmitem{thm:natural_number_divisibility_lattice/top} The \hyperref[def:partially_ordered_set_extremal_points/top_and_bottom]{top element} is \( 0 \) since every natural number divides \( 0 \).
  \end{thmenum}

  Furthermore, divisibility is compatible with the standard ordering in the sense that \( n \mid m \) implies \( n \leq m \).

  \begin{figure}
    \centering
    \includegraphics[page=1]{output/thm__natural_number_divisibility_order.pdf}
    \caption{A spatial \hyperref[def:hasse_diagram]{Hasse diagram} for a fragment of the \hyperref[thm:natural_number_divisibility_lattice]{natural number divisibility lattice}}
    \label{fig:thm:natural_number_divisibility_lattice/divisibility}
  \end{figure}

  \begin{figure}
    \centering
    \includegraphics[page=2]{output/thm__natural_number_divisibility_order.pdf}
    \caption{A comparison of the \hyperref[thm:natural_number_divisibility_lattice]{divisibility lattice} of \( \BbbN \) and the \hyperref[thm:semiring_of_ideals/lattice]{lattice of ideals} of \( \BbbZ \).}
    \label{fig:thm:natural_number_divisibility_lattice/ideals}
  \end{figure}
\end{proposition}
\begin{proof}
  By \fullref{thm:semiring_divisibility_order}, divisibility is a preorder.

  \SubProofOf[def:binary_relation/antisymmetric]{antisymmetry} If \( n \mid m \) and \( m \mid n \), there exist numbers \( a \) and \( b \) such that \( n = ay \) and \( m = bx \). Then \( n = abx \), and we can cancel \( n \) to obtain that \( ab = 1 \). But \( 1 \) is the only unit in \( \BbbN \), hence \( a = b = 1 \), and thus \( n = m \).

  \SubProofOf[def:semilattice/lattice]{lattice structure} By \fullref{alg:euclidean_algorithm}, every pair of integers has a positive greatest common divisor, and also a least common multiple.

  By \fullref{thm:def:semiring_ideal/division}, the lattice of principal ideals in \( \BbbN \) must be dual to it. Indeed, by \fullref{thm:bezouts_identity}, we have that
  \begin{equation*}
    \braket{n} + \braket{m} = \braket{ \gcd(n, m) },
  \end{equation*}
  and by \fullref{thm:def:semiring_ideal/division}, \( \braket{n} \cap \braket{m} \) contains the common multiples of \( n \) and \( m \), hence
  \begin{equation*}
    \braket{n} \cap \braket{m} = \braket{ \lcm(n, m) }.
  \end{equation*}

  \SubProof{Proof that the order are compatible} If \( n \mid m \), then there exists a positive natural number \( a \) such that \( an = m \). We have
  \begin{equation*}
    an
    \reloset{\eqref{eq:def:peano_arithmetic/PA7}} =
    (a - 1)n + n
    =
    m.
  \end{equation*}

  Thus, by \eqref{eq:def:natural_number_ordering/predicate}, \( n \leq m \).
\end{proof}

\subsection{Integers}\label{subsec:integers}

\begin{definition}\label{def:integers}
  The set \( \BbbZ \) of \term{integers} is defined as the Grothendieck \hyperref[thm:monoid_completion_to_abelian_group]{completion} of the commutative monoid \( (\BbbN, +) \).

  Let \( \oplus \), \( \odot \) and \( \leq_N \) be the operations in \( \BbbN \) (see \fullref{thm:natural_numbers_form_dioid}). Since either \( x \in \BbbZ \) or \( -x \in \BbbZ \) is isomorphic to a natural number, we extend the operations to \( \BbbZ \) as follows:
  \begin{thmenum}
    \item addition is defined in the completion.
    \item multiplication is defined as follows:
          \begin{equation*}
            x \cdot y \coloneqq \begin{cases}
              x \odot y,    & x \geq 0 \iff y \geq 0      \\
              (-x) \odot y, & x < 0 \text{ and } y \geq 0 \\
              x \odot (-y), & x \geq 0 \text{ and } y < 0
            \end{cases}
          \end{equation*}

    \ilabel{def:integers/order} the order is inherited
    \item the additional absolute \hyperref[def:absolute_value]{value} operation is defined as
          \begin{equation*}
            \abs{x} \coloneqq \begin{cases}
              x,  & x \geq 0, \\
              -x, & x < 0.
            \end{cases}
          \end{equation*}
  \end{thmenum}
\end{definition}
\begin{proof}
  The proof that multiplication and absolute values are well defined can be done similarly to the proof of \fullref{thm:monoid_completion_to_abelian_group}.

  Integer multiplication obviously generalizes natural number multiplication.
\end{proof}

\begin{proposition}\label{thm:integers_are_euclidean_domain}
  The \hyperref[def:semiring/integral_domain]{domain} of integers \( \BbbZ \) is \hyperref[def:semiring/euclidean_domain]{Euclidean} with \( \delta(n) \coloneqq \abs{n} \). Furthermore, the remainder and quotient are unique.
\end{proposition}
\begin{proof}
  Let \( a, b \in \BbbZ \) and \( b \neq 0 \). Suppose that \( b > 0 \). Define
  \begin{balign*}
    q & \coloneqq \max \{ q \in \BbbZ \colon bq \leq \abs{a} \} \\
    r & \coloneqq a - bq.
  \end{balign*}

  It remains to show that either \( r = 0 \) or \( \delta(r) < \delta(b) \).

  Note that \( r = 0 \) if and only if \( b \) is a \hyperref[def:commutative_ring_division]{divisor} of \( a \).

  Suppose that \( b \) is not a divisor of \( a \). Note that \( g \geq 0 \) and \( r > 0 \). If \( r \geq b \), this would imply\LEM \( r - b \geq 0 \) and
  \begin{equation*}
    a = bq + r = b(q + 1) + (r - b) \geq b(q + 1),
  \end{equation*}
  which would contradict the maximality of \( q \). Thus \( r < b \).

  It remains to show uniqueness. Suppose that \( a = bq + r = bq' + r' \). Then
  \begin{equation*}
    0 = b(q - q') + (r - r').
  \end{equation*}

  Thus \( b \mid r - r' \). But \( -b < r - r' < b \), which implies that \( r = r' \). Thus also implies that \( q = q' \) since \( b \neq 0 \).

  Now suppose that \( b < 0 \). Define
  \begin{balign*}
    q & \coloneqq \min \{ q \in \BbbZ \colon -\abs{a} \leq bq \} \\
    r & \coloneqq a - bq.
  \end{balign*}

  Suppose that \( b \) is not a divisor of \( a \). Note that \( q \geq 0 \) and \( r < 0 \). If \( r \leq b \), this would imply\LEM \( r - b \leq 0 \) and
  \begin{equation*}
    a = bq + r = b(q + 1) + (r - b) \leq b(q + 1),
  \end{equation*}
  which would contradict the minimality of \( q \). Thus \( r > b \) and, since both \( b \) and \( r \) are negative, \( \abs{r} < \abs{b} \).

  To see uniqueness, suppose that \( a = bq + r = bq' + r' \). Thus \( b \mid r - r' \). But \( -\abs{b} = b < r - r' < -b = \abs{b} \), which implies that \( r = r' \) and \( q = q' \).

  In both cases, we obtained unique integers \( q \) and \( r \) such that \( a = bq + r \) with \( \abs{r} < \abs{b} \).
\end{proof}

\begin{remark}\label{rem:units_in_rings_etymology}
  An integer \( a \) is divisible by \( b \neq 0 \) if there exists a number \( q \) such that
  \begin{equation*}
    a = qb.
  \end{equation*}

  Obviously \( q = (-q)(-1) \) so the following also holds:
  \begin{equation*}
    a = [(-q)(-1)]b = (-q)(-b),
  \end{equation*}
  hence \( a \) is also divisible by \( -b \).

  For any nonzero number \( b = 1 \cdot b \) that divides \( a \), the number \( -b = (-1) \cdot b \) also divides \( a \). Both \( 1 \) and \( -1 \) have unit norm (that is, \( \abs{1} = \abs{-1} = 1 \)) so it is reasonable to call them \enquote{units}. They are the only integers \( e \) with the property that if \( b | a \), then \( eb | a \). This is probably the reason why invertible elements in arbitrary rings are named \enquote{units}. Another reason is that invertible elements divide the multiplicative identity, commonly denoted by \( 1 \).

  Consider fields, in which all nonzero elements are units. It makes no sense to speak of divisibility whatsoever because any real number \( a \) is divisible by any nonzero real number \( b \). Putting \( q \coloneqq \frac a b \) satisfies the divisibility condition. Now if \( e \) is any unit in \( \BbbR \), we have
  \begin{equation*}
    a = qb = q(e^{-1} e) b = (qe^{-1}) (eb),
  \end{equation*}
  hence \( eb \) also divides \( a \).
\end{remark}

\begin{lemma}[Euclid's lemma]\label{thm:euclids_lemma}
  An \hyperref[def:integers]{integer} is \hyperref[def:prime_ring_ideal]{prime} if and only if it is irreducible.
\end{lemma}
\begin{proof}
  Follows from \fullref{thm:ufd_prime_iff_irreducible}.
\end{proof}

\begin{definition}\label{def:prime_number}
  Despite negative integers being prime \hyperref[thm:euclids_lemma]{elements} of the ring \( \BbbZ \), we only call positive prime integers \term{prime numbers}. That is, a positive integer is prime if it has no divisors except \( 1 \) and itself.

  Non-prime integers are called \term{composite numbers}.
\end{definition}

\begin{definition}\label{def:coprime_numbers}
  Two integers \( n, m \) are called \term{coprime} (see \fullref{def:coprime_ring_ideals}) if \( \gcd(n, m) = 1 \).
\end{definition}

\begin{theorem}[Fundamental theorem of arithmetic]\label{thm:fundamental_theorem_of_arithmetic}
  Every positive integer greater than \( 2 \) can be \hyperref[def:factorization_in_ring]{factored} into a product of \hyperref[def:prime_number]{prime} powers.
\end{theorem}
\begin{proof}
  The ring \( \BbbZ \) is an Euclidean domain by \fullref{thm:integers_are_euclidean_domain}, which is a principal ideal domain by \fullref{thm:euclidean_domain_is_pid}, which is a unique factorization domain by \fullref{thm:pid_is_ufd}.
\end{proof}

\begin{proposition}[Fermat's little theorem]\label{thm:fermats_little_theorem}
  If \( p \) is a prime \hyperref[def:prime_number]{number}, for any integer \( x \) we have
  \begin{equation*}
    x^p \cong x \pmod p.
  \end{equation*}
\end{proposition}

\subsection{Rational numbers}\label{subsec:rational_numbers}

\begin{definition}\label{def:rational_numbers}
  The \term{rational numbers} \( \BbbQ \) are the field of \hyperref[def:field_of_fractions]{fractions} of the \hyperref[def:integers]{integers}. Both operations from \( \BbbZ \) are inherited in \( \BbbQ \) and all nonzero elements in \( \BbbQ \) are now invertible, which makes \( \BbbQ \) a field.
\end{definition}

\subsection{Real numbers}\label{subsec:real_numbers}

\begin{definition}\label{def:set_of_rational_numbers}
  The \term{rational numbers} \( \BbbQ \) are the field of \hyperref[def:field_of_fractions]{fractions} of the \hyperref[def:set_of_integers]{integers}. Both operations from \( \BbbZ \) are inherited in \( \BbbQ \) and all nonzero elements in \( \BbbQ \) are now invertible, which makes \( \BbbQ \) a field.
\end{definition}

\begin{definition}\label{def:set_of_real_numbers}
  The \term{real numbers} \( \BbbR \) are the metric space \hyperref[def:complete_metric_space]{completion} of \( \BbbQ \) with respect to the absolute value. Unfortunately, real numbers are used for defining metric spaces, so we cannot rely on the theory of metric spaces. This can be circumvented by
  \begin{enumerate}
    \item Regarding \( \BbbQ \) as a \hyperref[def:uniform_space]{uniform space}.
    \item Using uniform space \hyperref[thm:uniform_space_completion]{completion} to obtain \( \BbbR \).
    \item Defining metric spaces.
    \item Showing that \( \BbbR \) is a metric space.
    \item Using \fullref{def:complete_metric_space/uniform} to automatically verify that \( \BbbR \) is complete as a metric space.
  \end{enumerate}
\end{definition}

\begin{definition}\label{def:extended_real_numbers}
  We are sometimes interested in \term{extended real numbers}. These can be any of the three sets
  \begin{itemize}
    \item \( \BbbR \cup \{ +\infty \} \),
    \item \( \BbbR \cup \{ -\infty \} \),
    \item \( \BbbR \cup \{ -\infty, +\infty \} \),
  \end{itemize}
  where \( -\infty \) and \( +\infty \) are both sentinel values that act as the \hyperref[def:partially_ordered_set_extremal_points/maximum_and_minimum]{greatest} and/or least real number.

  We generally avoid performing arithmetic operations on \( \pm \infty \), however it is sometimes convenient to define
  \begin{balign*}
    x + (+\infty)     & \coloneqq +\infty, x \in \BbbR
                      &                              &
    x \cdot (+\infty) & \coloneqq +\infty, x \in \BbbR
  \end{balign*}

  We leave the operations
  \begin{balign*}
     & (-\infty) + (+\infty)
     & (-\infty) \cdot (+\infty)
  \end{balign*}
  undefined.

  With these operations, the extended real numbers are no longer a \hyperref[def:field]{field}.
\end{definition}

\begin{definition}\label{def:floor_ceiling_functions}
  Let \( x \in \BbbR \) be a real number. In analogy with \fullref{def:divisibility}, we define its \term{floor}
  \begin{equation*}
    \floor(x) \coloneqq \max \{ n \in \BbbZ : n \leq x \},
  \end{equation*}
  its \term{ceiling}
  \begin{equation*}
    \ceil(x) \coloneqq \min \{ n \in \BbbZ : n \geq x \}
  \end{equation*}
  and its \term{fractional part}
  \begin{equation*}
    \op{frac}(x) \coloneqq x - \floor(x).
  \end{equation*}
\end{definition}

\begin{proposition}\label{thm:reals_not_algebraically_closed}
  The field \( \BbbR \) is not algebraically \hyperref[def:algebraically_closed_field]{closed}.

  In particular, the polynomial \( x^2 + 1 \) has no root.
\end{proposition}
\begin{proof}
  Assume that \( \BbbR \) is algebraically closed and that the polynomial \( x^2 + 1 \) has at least one root. Denote one of them by \( u \).

  By the \hyperref[def:binary_relation/trichotomic]{trichotomy} of the order \( < \) of \( \BbbR \), we have either \( u < 1 \) or \( u > 1 \) since \( u \neq 1 \).

  If \( u < 0 \), then \( u^2 = -1 < 0 \), which is impossible because the image of \( x \mapsto x^2 \) is the interval \( [0, \infty) \).

  If \( u > 0 \), then \( u^2 = -1 < 0 = 0 \), which is also impossible because \( x \mapsto x^2 \) is monotone on \( [0, \infty) \).

  Thus, \( u \) is not a root of \( x^2 + 1 \) and \( \BbbR \) is not algebraically closed.
\end{proof}

\begin{definition}\label{def:signum}
  We define the \term{signum} function \( \sgn: \BbbR \to \{ -1, 0, 1 \} \) as
  \begin{equation*}
    \sgn(x) \coloneqq \begin{cases}
      1,  & x > 0, \\
      0,  & x = 0, \\
      -1, & x < 0.
    \end{cases}
  \end{equation*}
\end{definition}

\subsection{Complex numbers}\label{subsec:complex_numbers}

\begin{definition}\label{def:complex_numbers}
  We give a few equivalent definition of the \hyperref[def:field]{field} \( \BC \) \Def{complex numbers}. Informally, there are numbers of the form \( a + bi \), where \( a, b \in \BR \) and \( i = \sqrt{-1} \). In order to find the multiplicative inverse of the nonzero polynomial \( a + bi \), we assume that division is well defined and proceed as follows:
  \begin{equation}\label{def:complex_numbers/inverse}
    \frac 1 {a + bi} = \frac {a - bi} {(a + bi)(a - bi)} = \frac{a - bi}{a^2 + b^2}.
  \end{equation}

  The closest to this informal definition is \fullref{def:complex_numbers/polynomials}.

  \begin{DefEnum}
    \ILabel{def:complex_numbers/polynomials} The most \enquote{algebraic} way to define complex numbers is as the \hyperref[def:polynomial]{polynomial} quotient \hyperref[thm:polynomial_quotient_rings_equinumerous_with_module_of_polynomials]{ring}
    \begin{equation*}
      \BC \coloneqq \BR[X] / \Braket{X^2 + 1}.
    \end{equation*}

    Elements of \( \BC \) can be identified with real polynomials of the form \( bX + a \). See \fullref{ex:polynomial_quotient_rings_gaussian_integers} for a broader discussion.
    Define \( i \coloneqq X \). We have
    \begin{equation*}
      i \cdot i = X^2 = -1 \pmod {X^2 + 1}.
    \end{equation*}

    Thus \( i \) is indeed the square root of \( -1 \). We will write
    \begin{equation*}
      a + bi = bX + a.
    \end{equation*}

    It is shown in \fullref{ex:polynomial_quotient_rings_gaussian_integers} that multiplication modulo \( X^2 + 1 \) gives
    \begin{equation}\label{def:complex_numbers/polynomials/multiplication}
      (bX + a) (dX + c) = (ad + bc)X + (ac - bd) \pmod {X^2 + 1}.
    \end{equation}

    The multiplicative inverse of \( a + bi \) is then indeed \fullref{def:complex_numbers/inverse}.

    The canonical embedding \( \iota: \BR \to \BC \) is then the standard polynomial embedding.

    \ILabel{def:complex_numbers/matrices} The complex numbers can also be defined as the matrix \hyperref[def:algebra_of_matrices]{ring}
    \begin{equation*}
      \BC \coloneqq \left\{
      \begin{pmatrix}
        a  & b \\
        -b & a
      \end{pmatrix}
      \colon a, b \in \BR \right\}
    \end{equation*}
    with the usual matrix multiplication. The canonical embedding \( \iota: \BR \to \BC \) is then
    \begin{equation*}
      \iota(a) \coloneqq \begin{pmatrix}
        a & 0 \\
        0 & a
      \end{pmatrix}
    \end{equation*}

    \ILabel{def:complex_numbers/tuples} Finally, we can define \( \BC \) is the \hyperref[def:algebra_over_ring]{algebra} obtained from the vector space \( \BR^2 \) with the multiplication operation emulating \fullref{def:complex_numbers/polynomials/multiplication} as
    \begin{BreakableAlign*}
       & \cdot: \BC \times \BC \to \BC                     \\
       & (a, b) \cdot (c, d) \coloneqq (ac - bd, ad + bc).
    \end{BreakableAlign*}

    The canonical embedding \( \iota: \BR \to \BC \) is then
    \begin{equation*}
      \iota(a) \coloneqq (a, b).
    \end{equation*}
  \end{DefEnum}

  We define the unary \Def{complex conjugation} operation as \( \Ol{a + bi} \coloneqq a - bi \) and the \Def{\hyperref[def:absolute_value]{absolute value}} as
  \begin{equation*}
    \Abs{a + bi} \coloneqq \sqrt{a^2 + b^2}.
  \end{equation*}

  For a complex number \( z = a + bi \) we denote
  \begin{BreakableAlign*}
    \Real z = a &  & \Imag z = b
  \end{BreakableAlign*}
  and call them the \Def{real} and \Def{imaginary} parts of \( z \).
\end{definition}

\begin{theorem}[Fundamental theorem of algebra]\label{thm:fundamental_theorem_of_algebra}
  The field \( \BC \) of complex numbers is algebraically \hyperref[def:algebraically_closed_field]{closed}.
\end{theorem}

\begin{definition}\label{def:gaussian_integers}
  The \Def{Gaussian integers} are a subring of the complex numbers. They are defined as numbers with integer real and imaginary parts and are commonly denoted as
  \begin{equation*}
    \BZ[i] \coloneqq \{ a + bi \colon a, b \in \BZ \}.
  \end{equation*}
\end{definition}

\begin{theorem}\label{thm:linear_functionals_over_c}
  Let \( X \) be a \hyperref[def:vector_space]{vector space} over \( \BC \). There is a bijection between the real-valued and the complex-valued linear functionals on \( X \).
\end{theorem}
\begin{proof}
  Let \( c: X \to \BC \) be a complex-valued linear functional. Denote \( a(x) \coloneqq \Real c(x) \) and \( b(x) \coloneqq \Imag c(x) \). Then \( a: X \to \BR \) and \( b: X \to \BR \) are linear functionals. We will show that \( a(x) \) uniquely determines \( b(x) \) and hence \( c(x) \).

  Note that \( c(ix) = a(ix) + i b(ix) = i a(x) - b(x) \). Therefore \( b(x) = a(ix) - c(ix) \) and
  \begin{equation*}
    c(x) = a(x) + i (a(ix) - c(ix)) = a(x) - a(x) + c(x) = c(x).
  \end{equation*}
\end{proof}

\begin{remark}\label{rem:linear_functionals_over_c}
  \Fullref{thm:linear_functionals_over_c} allows us to identify the dual space \( X* \) of a complex vector space \( X \) with \( \Hom(X, \BR) \) in the case of an algebraic \hyperref[def:dual_vector_space]{dual} or with the corresponding subspace in the case of a \hyperref[def:continuous_dual_space]{continuous dual space}.

  This allows us to reuse some of the theory for real vector spaces, for example hyperplane \hyperref[def:hyperplane_separation]{separation}.
\end{remark}


% Real analysis
\section{Real analysis}\label{sec:real_analysis}
\subsection{Topology of Euclidean spaces}\label{subsec:real_vector_space_geometry}

\begin{theorem}\label{thm:real_metric_and_order_topologies_coincide}
  For the real numbers, the \hyperref[def:metric_topology]{metric} and \hyperref[def:order_topology]{order topologies} coincide.
\end{theorem}
\begin{proof}
  The metric topology \( \mscrT_M \) is generated by the \hyperref[def:topological_base]{base}
  \begin{equation*}
    \mathcal{B} \coloneqq \{ B(x, r) \colon x \in \BbbR, r \in \BbbR_{>0} \}
  \end{equation*}
  and the order topology \( \mscrT_O \) is generated by the \hyperref[def:topological_subbase]{subbase}
  \begin{equation*}
    \mathcal{P} \coloneqq \{ (a, \infty) \colon a \in \BbbR \} \cup \{ (\infty, b) \colon b \in \BbbR \}.
  \end{equation*}

  The inclusion \( \mathcal{B} \subseteq FI(\mathcal{P}) \) is obvious since any ball \( B(x, r) \) is the intersection of the two rays
  \begin{equation*}
    B(x, r) = (x - r, \infty) \cap (-\infty, x + r).
  \end{equation*}

  Thus \( \mscrT_M \subseteq \mscrT_O \). We now only need to show that \( \mathcal{B} \) is a base for \( \mscrT_O \).

  Let \( U \in \mscrT_O \). Since \( FI(\mathcal{P}) \) is a base for \( \mscrT_O \), there \hyperref[def:topological_base/union]{exists} a family \( \{ U_i \}_{i \in I} \subseteq FI(\mathcal{P}) \) such that
  \begin{equation*}
    U = \bigcup_{i \in I} U_i.
  \end{equation*}

  We only need to express every \( U_i \) as a union of balls from \( \mathcal{B} \). There are several possibilities:
  \begin{itemize}
    \item if \( U_i \) is the open interval \( (a, \infty) \),
          \begin{equation*}
            (a, \infty) = \bigcup_{i=1}^\infty B(a + i, 1).
          \end{equation*}

    \item if \( U_i \) is the open interval \( (-\infty, b) \),
          \begin{equation*}
            (-\infty, b) = \bigcup_{i=1}^\infty B(b - i, 1).
          \end{equation*}

    \item if \( U_i \) is the intersection \( (a, \infty) \cap (-\infty, b), a < b \),
          \begin{equation*}
            (a, \infty) \cap (-\infty, b) = B(\tfrac {a + b} 2, \tfrac {b - a} 2)
          \end{equation*}

    \item if \( U_i \) is the empty set,
          \begin{equation*}
            \varnothing = \bigcup \varnothing \text{ (see \fullref{def:set_union})}.
          \end{equation*}
  \end{itemize}

  Thus \( U_i \) is the union of an at most countable amount of balls. The countable union of countable sets is again countable, hence by \fullref{def:topological_base/union}, \( \mathcal{B} \) is a base for \( \mscrT_O \).
\end{proof}

\begin{proposition}\label{thm:rn_bounded_iff_totally_bounded}
  A set in \( \BbbR^n \) is totally \hyperref[def:totally_bounded_set]{bounded} if and only if it is \hyperref[def:metric_space/bounded_set]{bounded}.
\end{proposition}
\begin{proof}
  \SufficiencySubProof Follows from \fullref{thm:totally_bounded_sets_are_bounded}.
  \NecessitySubProof Let \( A \) be a bounded set in \( \BbbR^n \) and let \( B(x, r) \) be a ball containing \( A \). Fix \( \varepsilon > 0 \).

  Denote by \( e_1, \ldots, e_n \) the \hyperref[def:left_module_hamel_basis]{basis} of \( \BbbR^n \). Denote by \( m \) the smallest integer such that \( m \varepsilon \geq r \).

  We can create a grid around \( B(x, r) \) as follows:

  Define the set
  \begin{equation*}
    \left\{ x + \sum_{i=1}^n [k_i \varepsilon] e_i \colon \forall i = 1, \ldots, n: k_i = 1, \ldots, m \right\}.
  \end{equation*}

  is finite. Furthermore, it is an \( \varepsilon \)-net of \( A \). Indeed, let \( y \in A \). Denote its coordinates along \( e_1, \ldots, e_n \) by \( y_1, \ldots, y_n \). Then \( y \) is contained in the ball
  \begin{equation*}
    B\left(x + \sum_{i=1}^n [\ceil(y_i) \varepsilon] e_i, \varepsilon \right).
  \end{equation*}
\end{proof}

\begin{theorem}[Heine-Borel theorem]\label{thm:heine_borel}
  A set in \( \BbbR^n \) is compact in the sense of \fullref{def:compact_space} if and only if it is closed and bounded.
\end{theorem}
\begin{proof}
  Follows from \fullref{thm:rn_bounded_iff_totally_bounded} and \fullref{thm:complete_metric_space_compact_conditions/closed_totally_bounded}.
\end{proof}

\begin{proposition}\label{thm:real_supremum_of_closure}
  The supremum (resp. infimum) of a set \( A \subseteq \BbbR \), if it exists, is equal to the supremum (resp. infimum) of \( \cl A \).
\end{proposition}
\begin{proof}
  \SufficiencySubProof Denote by \( M \) the supremum of \( A \). Assume\DNE that it is not a supremum of \( \cl A \), that is, there exists an upper bound \( M' \) of \( \cl A \) such that \( M' < M \). But this is impossible because \( A \subseteq \cl A \).

  Therefore \( M \) is a supremum of \( \cl A \).

  \NecessitySubProof Denote by \( M \) the supremum of \( \cl A \). Assume\DNE that it is not a supremum of \( A \), that is, there exists an upper bound \( M' \) of \( A \) such that \( M' < M \).

  Let \( \{ x_i \}_{i=1}^\infty \subseteq A \) be a sequence that converges to \( M \). Then
  \begin{equation*}
    x_i < M' < M.
  \end{equation*}

  By \fullref{thm:squeeze_lemma/sequences}, we have \( M' = M \), which contradicts our choice of \( M' \). Thus \( M \) is the supremum of \( A \).
\end{proof}

\begin{proposition}\label{thm:real_bounded_set_has_supremum}
  Every nonempty \hyperref[def:metric_space/bounded_set]{bounded set} in \( \BbbR \) has a supremum and infimum.
\end{proposition}
\begin{proof}
  Let \( A \subseteq \BbbR \) be a nonempty bounded set. By \fullref{thm:heine_borel}, the set \( \cl A \) is compact. By \fullref{thm:weierstrass_extreme_value_theorem}, the identity function \( \id: \BbbR \to \BbbR \) attains its minimum \( m \) and maximum \( M \) on \( \cl A \). Note that both \( m \) and \( M \) do not have to belong to \( A \), but \( m \) is a lower bound and \( M \) is an upper bound of the set \( A \).

  If we take any other upper bound \( M' \) of \( A \), then by \fullref{thm:real_supremum_of_closure},
  \begin{equation*}
    M' \geq \sup A = \sup \cl A = M.
  \end{equation*}

  Hence \( M \) is the least upper bound of \( A \).

  We can analogously prove that \( m \) is the greatest lower bound of \( A \).
\end{proof}

\subsection{Real-valued functions}\label{subsec:real_valued_functions}

\begin{definition}\label{def:epigraph}
  Let \( X \) be an arbitrary set. The \Def{epigraph} of the function \( f: X \to \BbbR \) is defined as
  \begin{equation*}
    \Epi f \coloneqq \{ (x, r) \in X \times \BbbR \colon r \geq f(x) \},
  \end{equation*}
\end{definition}

\subsection{Real convergence}\label{subsec:real_vector_space_convergence}

\begin{theorem}[Bolzano-Weierstrass]\label{def:bolzano_weierstrass}
  Every bounded sequence in \( \BbbR \) has a \hyperref[def:net_convergence/limit]{convergent} \hyperref[def:sequence]{subsequence}.
\end{theorem}
\begin{proof}
  Let \( \{ x_k \}_{k=1}^\infty \) be a bounded sequence in \( \BbbR \) and let \( a \leq b \) be a lower and upper \hyperref[def:preordered_set/upper_lower_bound]{bound}, respectively. Construct the sequence \( \{ F_k \}_{k=1}^\infty \) of closed intervals as follows: define \( \alpha_1 \coloneqq a \) and \( \beta_1 \coloneqq b \) and, at step \( k = 1, 2, \ldots \), put
  \begin{balign*}
    F_k \coloneqq \begin{cases}
      [\alpha_k, \tfrac{\alpha_k+\beta_k} 2], & [\alpha_k, \tfrac{\alpha_k+\beta_k} 2]\text{ contains infinitely many sequence members}, \\
      [\tfrac{\alpha_k+\beta_k} 2, \beta_k],  & \text{otherwise}.
    \end{cases}
  \end{balign*}

  Then put \( \alpha_{k+1} \) and \( \beta_{k+1} \) to be the endpoints of the interval \( F_k \) and repeat with \( k+1 \) instead of \( k \). Note that for any \( k = 1, 2, \ldots \), \( \diam(F_k) = \tfrac 1 2 \diam(F_{k-1}) \), thus \( \diam(F_k) \xrightarrow[k \to \infty]{} 0 \). As in \fullref{thm:cantors_nested_compact_theorem}, it follows that if we choose\AOC a sequence
  \begin{equation*}
    x_k \in F_k, k = 1, 2, \ldots,
  \end{equation*}
  it will be a fundamental sequence. Since the space is complete, this fundamental sequence necessarily converges.
\end{proof}

\begin{theorem}\label{def:real_numbers_complete_metric_space}
  The metric space \( \BbbR \) is complete.
\end{theorem}
\begin{proof}
  Let \( \{ x_k \}_{k=1}^\infty \) be a fundamental sequence of real numbers. By \fullref{thm:fundamental_sequence_is_bounded}, the sequence is bounded. By \fullref{def:bolzano_weierstrass}, it has a convergent subsequence
  \begin{equation*}
    \{ x_{k_m} \}_{m=1}^\infty \to x.
  \end{equation*}

  By \fullref{thm:fundamental_subsequence_convergence}, the sequence itself has the same limit \( \lim_{k \to \infty} x_k = x \).
\end{proof}

\begin{proposition}\label{thm:one_sided_squeeze_lemma}
  Fix two convergent sequences \( \{ x_k \}_{k=1}^\infty \) and \( \{ y_k \}_{k=1}^\infty \) of real numbers.

  If \( x_k \leq y_k \) for all \( k = 1, 2, \ldots \), then
  \begin{equation*}
    \lim_{k \to \infty} x_k \leq \lim_{k \to \infty} y_k.
  \end{equation*}
\end{proposition}
\begin{proof}
  Denote the respective limits by \( x \) and \( y \).

  Fix \( \varepsilon > 0 \). Then by \fullref{def:net_convergence/limit}, there exist indices \( k_0 \) and \( m_0 \) such that
  \begin{balign*}
     & \abs{x - x_k} < \frac \varepsilon 2 \quad\forall k \geq k_0 \\
     & \abs{y - y_m} < \frac \varepsilon 2 \quad\forall m \geq m_0
  \end{balign*}

  Take \( k \geq \max \{ k_0, m_0 \} \). Then \( y_k \geq x_k \) and
  \begin{balign*}
    y - x
     & =
    (y - y_k) + (y_k - x_k) + (x_k - x)
    \geq \\ &\geq
    (y - y_k) + (x - x_k)
    >    \\ &>
    - \frac \varepsilon 2 - \frac \varepsilon 2
    =
    - \varepsilon.
  \end{balign*}

  Since \( \varepsilon \) was chosen arbitrary, \( y - x \) cannot equal any negative number, because otherwise\LEM we could choose another \( \varepsilon \) smaller than the magnitude of the negative number and obtain a contradiction.

  Thus \( y \geq x \).
\end{proof}

\begin{lemma}[Squeeze lemma]\label{thm:squeeze_lemma}
  Let \( I \) be a closed \hyperref[def:total_order_interval/closed]{interval} in \( \BbbR \).

  \begin{thmenum}
    \labitem{thm:squeeze_lemma/sequences} Let \( \{ x_k \}_{k=1}^\infty, \{ x_k^- \}_{k=1}^\infty, \{ x_k^+ \}_{k=1}^\infty \) be three sequences in \( I \). If both \( \{ x_k^- \}_{k=1}^\infty \) and \( \{ x_k^+ \}_{k=1}^\infty \) converge to the same value \( \overline x \in I \) and if the following inequalities
    \begin{equation*}
      x_k^- \leq x_k \leq x_k^+
    \end{equation*}
    hold for all \( k = 1, 2, \ldots \), then the \enquote{squeezed in} sequence \( \{ x_k \}_{k=1}^\infty \) also converges to \( \overline x \).

    \labitem{thm:squeeze_lemma/functions} Let \( f, f_-, f_+: I \to \BbbR \) be three functions and let \( \overline x \in I \). If both limits \( \lim_{x \to \overline x} f_-(x) \) and \( \lim_{x \to \overline x} f_+(x) \) converge to the same value \( \overline y \in \BbbR \) and if the following inequalities
    \begin{equation*}
      f_-(x) \leq f(x) \leq f_+(x)
    \end{equation*}
    hold for all \( x \in I \), then the \enquote{squeezed in} function \( f \) also converges to \( \overline y \) at \( \overline x \).
  \end{thmenum}
\end{lemma}
\begin{proof}
  \SubProofOf{thm:squeeze_lemma/sequences} Fix \( \varepsilon > 0 \). Then by \fullref{def:net_convergence/limit}, there exist indices \( k^- \) and \( k^+ \) such that
  \begin{balign*}
     & \abs{\overline x - x_k^-} < \frac \varepsilon 3 \quad\forall k \geq k^- \\
     & \abs{\overline x - x_k^+} < \frac \varepsilon 3 \quad\forall k \geq k^+
  \end{balign*}

  By taking \( k \geq \max \{ k^-, k^+ \} \), we obtain
  \begin{equation*}
    \abs{x_k^+ - x_k^-} \leq \abs{x_k^+ - \overline x} + \abs{\overline x - x_k^-} < \frac 2 3 \varepsilon.
  \end{equation*}

  Since \( \abs{x_k^+ - x_k} \leq \abs{x_k^+ - x_k^-} \), it follows that \( \abs{x_k^+ - x_k} < \frac 2 3 \varepsilon \).

  Thus
  \begin{equation*}
    \abs{\overline x - x_k} \leq \abs{\overline x - x_k^+} + \abs{x_k^+ - x_k} < \varepsilon.
  \end{equation*}

  \Fullref{def:net_convergence/limit} is satisfied, hence \( \{ x_k \} \) converges to \( \overline x \).

  \SubProofOf{thm:squeeze_lemma/functions} The proof is analogous to that of \fullref{thm:squeeze_lemma/sequences}, but the machinery is different. Fix \( \varepsilon > 0 \). Then by \fullref{def:local_convergence/neighborhoods}, there exist radii \( \delta^- \) and \( \delta^+ \) such that
  \begin{balign*}
     & f_-(I \cap B(\overline x, \delta^-)) \subseteq B(\overline y, \tfrac \varepsilon 3) \\
     & f_+(I \cap B(\overline x, \delta^+)) \subseteq B(\overline y, \tfrac \varepsilon 3)
  \end{balign*}

  Take \( \delta < \min \{ \delta^-, \delta^+ \} \) and \( x \in I \cap B(\overline x, \delta) \). Analogously to the proof of \fullref{thm:squeeze_lemma/sequences}, we obtain the inequality
  \begin{equation*}
    \abs{f(x) - \overline x} \leq \abs{f(x) - f^-(x)} + \abs{f^-(x) - \overline x} < \varepsilon.
  \end{equation*}

  We conclude that
  \begin{equation*}
    f(I \cap B(\overline x, \delta)) \subseteq B(\overline y, \varepsilon)
  \end{equation*}
  holds and thus by \fullref{def:local_convergence/neighborhoods}, the function \( f \) converges to \( \overline y \) at \( \overline x \).
\end{proof}

\begin{proposition}\label{thm:real_monotone_sequence_converges_iff_bounded}
  A \hyperref[def:monotone_map]{monotone} sequence of real numbers \hyperref[def:net_convergence/limit]{converges} if and only if it is \hyperref[def:metric_space/bounded_sequence]{bounded}.
\end{proposition}
\begin{proof}
  \SufficiencySubProof Let \( \{ x_k \}_{k=1}^\infty \) be a convergent monotone sequence. Denote its limit by \( x \). Fix \( \varepsilon > 0 \). By \fullref{def:net_convergence/limit}, there exists \( k_0 \) such that
  \begin{equation*}
    \abs{x - x_k} < \varepsilon \quad\forall k \geq k_0.
  \end{equation*}

  Thus \( \{ x_k \colon k \geq k_0 \} \subseteq B(x, \varepsilon) \).

  Also note that
  \begin{equation*}
    \{ x_k \colon k < k_0 \} \subseteq B(x, \max_{i < k_0} \{ \abs{x - x_k} \}).
  \end{equation*}

  We obtained that the entire sequence
  \begin{equation*}
    \{ x_k \colon k \geq 1 \} = \{ x_k \colon k < k_0 \} \cup \{ x_k \colon k \geq k_0 \}
  \end{equation*}
  is contained in a union of two balls and is therefore bounded.

  \NecessitySubProof Now let \( \{ x_k \}_{k=1}^\infty \) be a bounded monotone sequence. Denote its supremum by \( \alpha \). Note that
  \begin{equation*}
    \abs{x_n - x_m} = x_n - x_m \leq \alpha \quad\forall n \geq m.
  \end{equation*}

  Fix \( \varepsilon > 0 \). Then there exists at least one element \( x_{m_0} > \alpha - \varepsilon \) because otherwise\LEM \( \alpha \) would not be a supremum.

  Then for any index \( n \geq m_0 \) we have
  \begin{equation*}
    \abs{x_n - x_{m_0}} = x_n - x_{m_0} < \alpha - (\alpha - \varepsilon) = \varepsilon.
  \end{equation*}

  Thus \fullref{def:net_convergence/limit} is satisfied and the sequence \( \{ x_k \}_{k=1}^\infty \) converges.
\end{proof}

\subsection{Real differentiability}\label{subsec:real_differentiability}

\begin{proposition}\label{thm:real_valued_differentiability}
  Let \( U \subseteq \BbbR^n \) be an open set. A real-valued function \( f: U \to \BbbR \) is differentiable at \( x \) in the direction \( h \) if and only if \( \varphi(t) = f(x + th) \) is right-differentiable at \( 0 \).
\end{proposition}
\begin{proof}
  \begin{equation*}
    f_+'(x)(h) \coloneqq \lim_{t \downarrow 0} \frac {f(x + th) - f(x)} t = \varphi_+'(0)(1).
  \end{equation*}
\end{proof}

\begin{example}[Weierstrass' nowhere differentiable function]\label{ex:weierstrass_nowhere_differentiable_function}\mcite[\textnumero 271]{Фихтенгольц1968Том2}
  Let \( a \in (0, 1) \) and \( b \) is a positive odd integer such that
  \begin{equation*}
    ab > 1 + \frac 3 2 \pi.
  \end{equation*}

  Define the function
  \begin{equation*}
    f(x) \coloneqq \sum_{k=0}^\infty a^k \cos(b^k \pi x).
  \end{equation*}

  \begin{figure}[!ht]
    \centering
    \includegraphics{output/ex__weierstrass_nowhere_differentiable_function.pdf}
    \caption
    {
      Plot of the third partial sum of the Weierstrass function with \( a = 0.9 \) and \( b = 7 \) from \( -\sfrac \pi 8 \) to \( \sfrac \pi 8 \).
    }
    \label{fig:ex:weierstrass_nowhere_differentiable_function/plot}
  \end{figure}

  Since \( \cos \) is bounded for real arguments and \( a \in (0, 1) \), each term is uniformly bounded by \( 1 \) and by \fullref{thm:weierstrass_series_criterion}, \( f \) is continuous. However, it is not \hyperref[def:differentiability]{differentiable} at any point. The proof of the latter is involved and will not be given here.
\end{example}

\begin{theorem}\label{thm:leibniz_rule}
  \todo{Prove Leibniz' rule}.
\end{theorem}

\begin{theorem}[Fundamental theorem of calculus]\label{thm:fundamental_theorem_of_calculus}
  \todo{Fundamental theorem of calculus}.
\end{theorem}

\subsection{Real series}\label{subsec:real_series}

\begin{proposition}\label{thm:almost_all_terms_positive_implies_absolute_convergent}
  If only finitely many coefficients in a real \hyperref[def:convergent_series]{convergent} series are negative, then the series converges absolutely.
\end{proposition}
\begin{proof}
  Let \( N \) be the index of the last negative coefficient in \eqref{def:convergent_series/series}. Then the series
  \begin{equation*}
    \sum_{k={N+1}}^\infty a_k
  \end{equation*}
  is absolutely convergent since every coefficient is positive. Then
  \begin{equation*}
    \sum_{k=0}^\infty \abs{a_k} = \sum_{k=0}^N \abs{a_k} + \sum_{k=N+1}^\infty \abs{a_k}
  \end{equation*}
  is convergent since the first term on the right side is a finite sum and the second is a convergent series. Hence the series \eqref{def:convergent_series/series} converges absolutely.
\end{proof}

\begin{corollary}\label{thm:almost_all_terms_negative_implies_absolute_convergent}
  If only finitely many coefficients in a real \hyperref[def:convergent_series]{convergent} series are positive, then the series converges absolutely.
\end{corollary}

\begin{theorem}[Riemann's series permutation theorem]\label{thm:riemanns_series_permutation_theorem}\mcite\cite[\textnumero 247]{Фихтенгольц1968Том2}
  If the real series
  \begin{equation*}
    \sum_{k=0}^\infty a_k
  \end{equation*}
  is \hyperref[def:convergent_series]{convergent} but not absolutely convergent, then for any extended real number \( x \in \BbbR \cup \{ -\infty, +\infty \} \) there exists a \hyperref[def:symmetric_group]{permutation} \( p \) of the coefficients \( a_0, a_1, a_2 \)
  such that
  \begin{equation*}
    \sum_{k=0}^\infty p(a_k) = x.
  \end{equation*}
\end{theorem}
\begin{proof}
  If the series is not absolutely convergent, then there exist both infinitely many positive and infinitely many negative coefficients.

  First, assume that \( x \) is finite.

  Define the permuted series
  \begin{equation*}
    \sum_{k=0}^\infty b_k
  \end{equation*}
  as follows:
  \begin{thmenum}
    \labitem{thm:riemanns_series_theorem/positive} Assign to \( b_n \) only nonnegative elements of the sequence \( \{ a_k \}_{k=0}^\infty \) until \( \sum_{k=0}^n b_k \geq x \). Then go to \ref{thm:riemanns_series_theorem/negative}.
    \labitem{thm:riemanns_series_theorem/negative} Assign to \( b_n \) only negative elements of the sequence \( \{ a_k \}_{k=0}^\infty \) until \( \sum_{k=0}^n b_k \geq x \). Then go to \ref{thm:riemanns_series_theorem/positive}.
  \end{thmenum}

  This mutual recursion builds a series that converges to \( x \) because the coefficients \( \{ a_k \}_{k=0}^\infty \) get arbitrarily close to each other.

  If \( x = +\infty \), we can add positive coefficients until \( \sum_{k=0}^n b_k \geq 1 \), then add a single negative coefficient, then continue adding positive coefficients until \( \sum_{k=0}^n b_k \geq 2 \) and so on.

  If \( x = -\infty \), we use the same process but with milestones of \( -1, -2, -3, \ldots \).
\end{proof}

\begin{example}\label{ex:riemanns_series_permutation_theorem/alternating_harmonic_series}\cite[\textnumero 247]{Фихтенгольц1968Том2}
  Consider the \term{alternating harmonic series}
  \begin{equation}\label{ex:riemanns_series_permutation_theorem/alternating_harmonic_series/series}
    \sum_{k=1}^\infty \frac {(-1)^k} k
    =
    \sum_{m=1}^\infty \left( \frac 1 {2m - 1} - \frac 1 {2m} \right)
    =
    1 - \frac 1 2 + \frac 1 3 - \frac 1 4 + \cdots.
  \end{equation}

  Compare the series with \fullref{def:harmonic_progression/series}. Note that by \fullref{leibniz_alternating_series_test} the series is convergent, however it is not absolutely convergent because the harmonic series \fullref{def:harmonic_progression/series} is divergent.

  Denote the sum by \( a \).

  We can rearrange this series by repeating two negative terms and a single positive term as follows:
  \begin{equation}\label{ex:riemanns_series_permutation_theorem/alternating_harmonic_series/rearranged}
    1 - \frac 1 2 - \frac 1 4 + \frac 1 3 - \frac 1 6 - \frac 1 8 + \cdots
    =
    \sum_{m=1}^\infty \left( \frac 1 {2m - 1} - \frac 1 {4m - 2} - \frac 1 {4m} \right).
  \end{equation}

  Note that \fullref{ex:riemanns_series_permutation_theorem/alternating_harmonic_series/rearranged} is equivalent to
  \begin{equation*}
    \sum_{m=1}^\infty \left( \frac 1 {2m - 1} - \frac 1 {4m - 2} - \frac 1 {4m} \right)
    =
    \sum_{m=1}^\infty \left( \frac 1 {4m - 2} - \frac 1 {4m} \right)
    =
    \frac 1 2 \sum_{m=1}^\infty \left( \frac 1 {2m - 1} - \frac 1 {42} \right)
    =
    \frac a 2.
  \end{equation*}
\end{example}

\begin{proposition}\label{thm:positive_series_comparison}\mcite\cite[\textnumero 237]{Фихтенгольц1968Том2}
  Fix two nonnegative series
  \begin{equation}\label{def:positive_series_comparison/a}
    \sum_{k=0}^\infty a_k
  \end{equation}
  and
  \begin{equation}\label{def:positive_series_comparison/b}
    \sum_{k=0}^\infty b_k
  \end{equation}
  that is, series with nonnegative real coefficients. Assume that there exists an index \( K \) such that
  \begin{equation*}
    a_k \leq b_k \quad\forall k \geq K.
  \end{equation*}

  We say that the series \fullref{def:positive_series_comparison/b} \term{dominates} the series \fullref{def:positive_series_comparison/a}.

  Then
  \begin{thmenum}
    \labitem{thm:positive_series_comparison/b_converges} If \fullref{def:positive_series_comparison/b} converges, so does \fullref{def:positive_series_comparison/a}.

    \labitem{thm:positive_series_comparison/a_diverges} If \fullref{def:positive_series_comparison/a} diverges, so does \fullref{def:positive_series_comparison/b}.
  \end{thmenum}
\end{proposition}
\begin{proof}
  \SubProofOf{thm:positive_series_comparison/b_converges} Suppose that \fullref{def:positive_series_comparison/b} converges. Then by \fullref{thm:real_monotone_sequence_converges_iff_bounded}, the sequence of partial sums is bounded. Therefore the sequence of partial sums of \fullref{def:positive_series_comparison/a} is also bounded and, by \fullref{thm:real_monotone_sequence_converges_iff_bounded} again, the series is convergent.

  \SubProofOf{thm:positive_series_comparison/a_diverges} Analogous to \fullref{thm:positive_series_comparison/b_converges}, but using the negation of \fullref{thm:real_monotone_sequence_converges_iff_bounded}.
\end{proof}

\begin{proposition}[Cauchy's root test]\label{thm:cauchys_root_test}\mcite\cite[thm. 3.33]{Rudin1976Principles}
  Consider the nonnegative series \fullref{def:positive_series_comparison/a}. Put
  \begin{equation*}
    q \coloneqq \limsup_{k \to \infty} \sqrt[k]{a_k},
  \end{equation*}
  where \( q = \infty \) if the limit does not exist. Then
  \begin{itemize}
    \item If \( q < 1 \), the series converges.
    \item If \( q > 1 \), the series diverges.
    \item If the limit does not exist (e.g. if \( a_k = k^k \)), the series diverges.
    \item If \( q = 1 \), the series may either converge or diverge.
  \end{itemize}
\end{proposition}
\begin{proof}
  The case when the limit \( q \) does not exist is obvious by the contraposition to \fullref{thm:convergent_series_terms_vanish}.

  Suppose that the limit exists. Therefore there exists an index \( K \) such that
  \begin{equation*}
    \sqrt[k]{a_k} \leq q \quad\forall k \geq K.
  \end{equation*}

  Thus we have the inequality
  \begin{equation*}
    a_k \leq q^k \quad\forall k \geq K.
  \end{equation*}

  The statement of the theorem now follows from comparison (\fullref{thm:positive_series_comparison}) with the \hyperref[def:geometric_progression/series]{geometric series}.
\end{proof}

\begin{proposition}[d'Alambert's ratio test]\label{thm:dalamberts_ratio_test}\mcite\cite[thm. 3.33]{Rudin1976Principles}
  Consider the nonnegative series \fullref{def:positive_series_comparison/a}. Put
  \begin{equation*}
    q \coloneqq \limsup_{k \to \infty} \frac {a_{k+1}} {a_k},
  \end{equation*}
  where \( q = \infty \) if the limit does not exist. Then
  \begin{itemize}
    \item If \( q < 1 \), the series converges.
    \item If there exists an index \( k_0 \) such that \( \frac {a_{k+1}} {a_k} \geq 1 \) for all \( k \geq k_0 \), the series diverges.
    \item If the limit does not exist (e.g. if \( a_k = k! \)), the series diverges.
  \end{itemize}
\end{proposition}
\begin{proof}
  All cases except for \( q < 1 \) are obvious by the contraposition to \fullref{thm:convergent_series_terms_vanish}.

  Suppose that the limit exists. Therefore there exists an index \( k_0 \) such that
  \begin{equation*}
    a_{k+1} \leq q a_k \quad\forall k \geq k_0.
  \end{equation*}

  Thus
  \begin{equation*}
    a_{k_0 + m} \leq q^m a_{k_0} \quad\forall m \geq \BbbZ^{\geq 0}.
  \end{equation*}

  Convergence now follows from comparison (\fullref{thm:positive_series_comparison}) of the \hyperref[def:geometric_progression/series]{geometric series} with the subseries of \fullref{def:positive_series_comparison/a} obtained by trimming the first \( k_0 \) elements.
\end{proof}

\begin{proposition}\label{rem:nonnegative_series_convergence_test_equivalence}
  The values of \( q \) in \fullref{thm:cauchys_root_test} and in \fullref{thm:dalamberts_ratio_test} are equal.
\end{proposition}
\begin{proof}
  If we assume\LEM that they are not equal, then the same series would have to be convergent and divergent simultaneously in some region.
\end{proof}

\begin{definition}\label{def:alternating_series}
  Series of the form
  \begin{equation}\label{def:alternating_series/series}
    \pm \sum_{k=0}^\infty (-1)^k a_k,
  \end{equation}
  where all \( a_k, k = 0, 1, \ldots \) are nonnegative, are called \term{alternating}.
\end{definition}

\begin{proposition}[Leibniz' alternating series test]\label{leibniz_alternating_series_test}
  Consider the alternating series \fullref{def:alternating_series}. If the sequence of terms \( \{ a_k \}_{k=0}^\infty \) decreases monotonically and if \( \lim_{k \to \infty} a_k = 0 \), then the series converges.
\end{proposition}

\begin{theorem}\label{thm:weierstrass_series_criterion_nessessity}\mcite\cite[\textnumero 268]{Фихтенгольц1968Том2}
  \Fullref{thm:weierstrass_series_criterion} is a necessary condition for nonnegative real functions.
\end{theorem}

\subsection{Riemann integration}\label{subsec:riemann_integration}

\begin{definition}\label{def:riemann_partition}\MarginCite[def. 1]{Gordon1991}
  The concept of a partition of a nonempty \hyperref[def:real_numbers]{real} \hyperref[def:total_order_interval/closed]{closed interval} \( [a, b] \) is the base for defining Riemann-style integrals.

  \begin{DefEnum}
    \ILabel{def:riemann_partition/partition} A \Def{Riemann partition} of \( [a, b] \) is a set
    \begin{equation*}
      \Delta \coloneqq \{ x_0, \ldots, x_n \} \subseteq [a, b]
    \end{equation*}
    that satisfies
    \begin{equation*}
      a = x_0 < x_1 < \ldots < x_n = b.
    \end{equation*}

    For brevity, we write
    \begin{equation}\label{eq:def:riemann_partition/partition}
      \Delta: a = x_0 < x_1 < \ldots < x_n = b.
    \end{equation}

    We denote the set of all partitions of \( [a, b] \) by \( \Op{part}([a, b]) \).

    \ILabel{def:riemann_partition/refinement} The partition
    \begin{equation*}
      \Gamma: a = y_0 < y_1 < \ldots < y_m = b
    \end{equation*}
    is called a \Def{refinement} of the partition \eqref{eq:def:riemann_partition/partition} if we have the \hyperref[def:subset]{set inclusion}
    \begin{equation}\label{eq:def:riemann_partition/refinement/inclusion}
      \{ x_0, x_1, \ldots, x_n \} \subseteq \{ y_0, y_1, \ldots, y_m \}.
    \end{equation}

    In this case, we \enquote{split} \( \Gamma \) into chains such that, for each \( k = 1, 2, \ldots, n \),
    \begin{equation}\label{def:riemann_partition/refinement/splitting}
      y_{k,j} \coloneqq \begin{cases}
        x_{k-1},                                                                          &j = 0, \\
        x_k,                                                                              &j = p_k, \\
        j\text{-th point of } \{ y_0, \ldots, y_m \} \cap [x_{k-1}, x_k], &0 < j < p_k.
      \end{cases}
    \end{equation}

    \ILabel{def:riemann_partition/diameter} Finally, the \Def{diameter} of the partition \eqref{eq:def:riemann_partition/partition} is defined as
    \begin{equation}\label{eq:def:riemann_partition/diameter}
      \Diam(\Delta) \coloneqq \max_{1 \leq k \leq n} (x_k - x_{k-1}).
    \end{equation}

    \ILabel{def:riemann_partition/order} We can make the set \( \Op{part}([a, b]) \) of all \hyperref[def:riemann_partition/partition]{Riemann partitions} of \( [a, b] \) into a \hyperref[def:directed_set]{directed set} using two common approaches:
    \begin{DefEnum}
      \ILabel{def:riemann_partition/order/refinement} Put \( \Delta \preceq_R \Gamma \) if and only if \( \Gamma \) is a \hyperref[def:riemann_partition/refinement]{refinement} of \( \Delta \). This actually makes \( (\Op{part}([a, b]), \preceq_R) \) a \hyperref[def:poset]{poset}.
      \ILabel{def:riemann_partition/order/diameter} Put \( \Delta \preceq_D \Gamma \) if and only if \( \Diam(\Gamma) \leq \Diam(\Delta) \).
    \end{DefEnum}

    \ILabel{def:riemann_partition/tagged} A \Def{tagged partition} of \( [a, b] \) is a partition \eqref{eq:def:riemann_partition/partition} of \( [a, b] \) along with a choice of a \Def{tag} \( \xi_k \) for each closed interval \( [x_{k-1}, x_k], k = 1, \ldots, n \). By putting \( \Xi \coloneqq \{ \xi_k \}_{k=1}^n \), we can define a tagged partition as the tuple \( (\Delta, \Xi) \). For brevity, we write
    \begin{AlignedEquation}\label{eq:def:riemann_partition/tagged}
      &\Delta: a = x_0 < x_1 < \ldots < x_n = b \\
      &\Xi: \xi_k \in [x_{k-1}, x_k], k = 1, \ldots, n.
    \end{AlignedEquation}

    We denote the set of all tagged partitions of \( [a, b] \) by \( \Op{tpart}([a, b]) \). We introduce an order on \( \Op{tpart}([a, b]) \) by putting
    \begin{equation*}
      (\Delta, \Xi) \preceq_R (\Gamma, \Eta) \T{if and only if} \Delta \preceq_R \Eta
    \end{equation*}
    and analogously for \( \preceq_D \). Note that \( \preceq_R \) is not a partial order in \( \Op{tpart}([a, b]) \) unlike in \( \Op{part}([a, b]) \).
  \end{DefEnum}
\end{definition}

\begin{remark}\label{remark:set_and_riemann_partitions}
  Note that \eqref{eq:def:riemann_partition/partition} is not a partition in the sense of \fullref{def:set_partition}, however the set of intervals
  \begin{equation*}
    \Big\{ [x_0, x_1), [x_1, x_2), \ldots, [x_{n-2}, x_{n-1}), [x_{n-1}, x_n] \Big\}
  \end{equation*}
  is a set-theoretic partition of \( [a, b] \). Conversely, every finite set-theoretic partition of \( [a, b] \) gives rise to a Riemann partition in the sense of \fullref{def:riemann_partition/partition}.
\end{remark}

\begin{definition}\label{def:riemann_integral}\MarginCite[def. 2]{Gordon1991}
  Let \( \CX \) be a real \hyperref[def:separation_axioms/T2]{Hausdorff} \hyperref[def:topological_vector_space]{topological vector space}. Fix a \hyperref[def:function/single_valued]{function} \( f: [a, b] \to \CX \).

  The \Def{Riemann sum} of \( f \) corresponding to the \hyperref[def:riemann_partition/tagged]{tagged partition} \eqref{eq:def:riemann_partition/tagged} is defined as
  \begin{equation*}
    S(f, \Delta, \Xi) \coloneqq \sum_{k=1}^n f(\xi_k) (x_k - x_{k-1}).
  \end{equation*}

  Consider the net
  \begin{equation}\label{eq:def:riemann_integral/net}
    \{ S(f, \Delta, \Xi) \}_{(\Delta, \Xi) \in \Op{tpart}([a, b])}
  \end{equation}

  Both orders \fullref{def:riemann_partition/order/refinement} and \fullref{def:riemann_partition/order/diameter} on \( \Op{tpart}([a, b]) \) provide equivalent convergence for Riemann sums. If the limit exists, \( f \) is said to be \Def{Riemann integrable} in \( [a, b] \). We call the limit the \Def{Riemann integral} of \( f \) and denote it by
  \begin{equation}\label{eq:def:riemann_integral}
    \int_a^b f(x) dx.
  \end{equation}
\end{definition}
\begin{proof}
  \SubProofImplication{def:riemann_partition/order/refinement}{def:riemann_partition/order/diameter} Let \( I \) be the limit \eqref{eq:def:riemann_integral} with respect to \( \preceq_R \). Fix a neighborhood \( U \) of \( 0 \). Since \eqref{eq:def:riemann_integral/net} is eventually in \( I + U \), there exists a tagged partition
  \begin{AlignedEquation}\label{eq:def:riemann_integral/tagged_zero}
    &\Delta_0: a = x_0^{(0)} < x_1^{(0)} < \ldots < x_n^{(0)} = b \\
    &\Xi_0: \xi_k^{(0)} \in [x_{k-1}^{(0)}, x_k^{(0)}], k = 1, \ldots, n_0.
  \end{AlignedEquation}
  such that \( S(f, \Gamma, \Eta) \in I + U \) if \( \Gamma \) is a refinement of \( \Delta_0 \).

  Note that \( f \) is \hyperref[def:bounded_function/bounded]{bounded}. Indeed, if\LEM it is unbounded on \( [a, b] \), then there exists a refinement \( (\Gamma, \Eta) \) of \( (\Delta_0, \Xi_0) \) such that
  \begin{equation*}
    S(f, \Gamma, \Eta) - I \not\in U.
  \end{equation*}

  But this contradicts our choice of \( \Delta_0 \). Therefore \( f \) is bounded and there exists a bounded neighborhood \( V_0 \) of \( 0 \) such that \( f([a, b]) \subseteq V_0 \) and hence \( f(x) - f(y) \in V \coloneqq V_0 - V_0 \) for all \( x, y \in [a, b] \).

  Let  \( v > 0 \) be such that \( V \subseteq vU \).

  Let \( (\Delta, \Xi) \) be a tagged partition such that \( \Diam(\Delta) \leq \Diam(\Delta_0) \).

  We introduce another partition \( \Gamma \coloneqq \Delta \cup \Delta_0 \). Since \( \Gamma \) is a refinement of \( \Delta_0 \), we can use a splitting similar to \eqref{def:riemann_partition/refinement/splitting} such that
  \begin{equation}\label{def:riemann_partition/subdiameter_splitting}
    S(f, \Delta_0, \Xi_0) = \sum_{k=1}^{n_0} \sum_{j=1}^{p_k} f(\xi^{(0)}_k) (y_{k,j} - y_{k,j-1}).
  \end{equation}

  Denote by \( \xi_{k,j} \) the largest tag in \( \Xi \) such that \( \xi_{k,j} \leq y_{k,j} \). Thus
  \begin{equation*}
    S(f, \Delta, \Xi) = \sum_{k=1}^{n_0} \sum_{j=1}^{p_k} f(\xi_{k,j}) (y_{k,j} - y_{k,j-1}).
  \end{equation*}

  For every \( k = 1, \ldots, n \) and every \( j = 0, \ldots, p_k \), choose\LEM an arbitrary tag
  \begin{equation*}
    \Eta: \eta_{k,j} \in [y_{k,j-1}, y_{k,j}].
  \end{equation*}

  Then we have
  \begin{BreakableAlign*}
    S(f, \Delta, \Xi) - I
    &=
    S(f, \Delta, \Xi) - S(f, \Gamma, \Eta) + \underbrace{S(f, \Gamma, \Eta) - I}_{\in U}
    \in \\ &\in
    \sum_{k=1}^{n_0} \sum_{j=1}^{p_k} [ \underbrace{f(\xi_{k,j}) - f(\eta_{k,j})}_{\in V} ] (y_{k,j} - y_{k,j-1}) + U
    \subseteq \\ &\subseteq
    V \cdot \sum_{k=1}^{n_0} \underbrace{\sum_{j=1}^{p_k} (y_{k,j} - y_{k,j-1})}_{x_k - x_{k-1}} + U
    \subseteq \\ &\subseteq
    \Diam(\Delta) \cdot n_0 \cdot V + U
    \subseteq \\ &\subseteq
    (\Diam(\Delta) \cdot n_0 \cdot v + 1) U.
  \end{BreakableAlign*}

  Let \( (\Delta_1, \Xi_1) \) be a tagged partition of \( [a, b] \) such that \( \Diam(\Delta_1) \leq \min \left\{ \Diam(\Delta_0), \frac 1 {v n_0} \right\} \). It follows that
  \begin{equation}\label{eq:def:riemann_integral/subdiameter_in_neighborhood}
    S(f, \Delta_1, \Xi_1) - I \subseteq 2U.
  \end{equation}

  Until now, \( U \) was fixed. Given any neighborhood \( W \) of \( 0 \), we need to choose a neighborhood \( U \) of \( 0 \) and a corresponding partition \( \Delta_1 \) such that \eqref{eq:def:riemann_integral/subdiameter_in_neighborhood} holds. Then, whenever \( \Diam(\Delta) \leq \Diam(\Delta_1) \), we have
  \begin{equation*}
    S(f, \Delta, \Xi) - I \subseteq 2U \subseteq W.
  \end{equation*}

  This finishes the proof.

  \SubProofImplication{def:riemann_partition/order/diameter}{def:riemann_partition/order/refinement} Note that if \( \Gamma \) is a refinement of \( \Delta \), clearly \( \Diam(\Gamma) \leq \Diam(\Delta) \). Therefore if the net \eqref{eq:def:riemann_integral/net} with respect to \( \preceq_D \) is eventually in some open set \( U \), the corresponding net with respect to \( \preceq_R \) is also eventually in \( U \). This finishes the proof.
\end{proof}

\begin{corollary}\label{thm:riemann_integrable_implies_bounded}
  A Riemann-integrable function is bounded.
\end{corollary}
\begin{proof}
  Proven in \fullref{def:riemann_integral}.
\end{proof}

\begin{definition}\label{def:darboux_integrability}\MarginCite[def. 17]{Gordon1991}
  Let \( (\CX, \rho) \) be a \hyperref[def:frechet_space]{Frechet space}. Fix a function \( f: [a, b] \to \CX \). Similarly to \fullref{def:riemann_integral}, choose any of the orderings \fullref{def:riemann_partition/order/refinement} and \fullref{def:riemann_partition/order/diameter} on the set of all untagged \hyperref[def:riemann_partition/partition]{Riemann partitions} \( \Op{part}([a, b]) \).

  For each partition \eqref{eq:def:riemann_partition/partition}, we define its \Def{oscillation} via the \hyperref[def:function_oscillation]{function oscillation} of \( f \)
  \begin{equation}\label{eq:def:darboux_integrability/oscillation}
    \omega(f, \Delta) \coloneqq \sum_{k=1}^n \omega(f, [x_{k-1}, x_k]) (x_k - x_{k-1}).
  \end{equation}

  Consider the net
  \begin{equation}\label{eq:def:darboux_integrability/net}
    \{ \omega(f, \Delta) \}_{\Delta \in \Op{part}([a, b])}
  \end{equation}

  If this net converges to zero, we say that \( f \) is \Def{Darboux integrable}.
\end{definition}

\begin{proposition}\label{thm:darboux_integrable_implies_riemann_integrable}
  In a \hyperref[def:banach_space]{Banach space}, \hyperref[def:darboux_integrability]{Darboux integrability} implies \hyperref[def:riemann_integral]{Riemann integrability}.
\end{proposition}
\begin{proof}
  We will show that the net \eqref{eq:def:riemann_integral/net} is fundamental. Fix \( \varepsilon > 0 \). Since \( f \) is Darboux integrable, there exists an untagged partition \( \Delta_0 \) such that, if \( \Delta \) is a refinement of \( \Delta_0 \), we have
  \begin{equation*}
    \omega(f, \Delta) < \varepsilon.
  \end{equation*}

  Let \( \Delta \) be a refinement of \( \Delta_0 \) and \( \Gamma \) be a refinement of \( \Delta \). Assume that the points of \( \Gamma \) are split as in \eqref{def:riemann_partition/refinement/splitting}. Choose arbitrary tags \( \Xi = \{ \xi_k \}_{k=1}^n \) for \( \Delta \) and \( \Eta = \{ \eta_{k,j} \}_{k=1,j=1}^{n,p_k} \) for \( \Gamma \). For the corresponding Riemann sums, we have
  \begin{BreakableAlign*}
    &{}\phantom{=}{}
    \Norm{S(f, \Delta, \Xi) - S(f, \Gamma, \Eta)}
    = \\ &=
    \Norm{\sum_{k=1}^n f(\xi_k) (x_k - x_{k-1}) - \sum_{k=1}^n \sum_{j=1}^{p_k} f(\eta_{k,j}) (y_{k,j} - y_{k,j-1}) }
    \leq \\ &\leq
    \sum_{k=1}^n \sum_{j=1}^{p_k} \Norm{f(\xi_k) - f(\eta_{k,j})} (y_{k,j} - y_{k,j-1})
    \leq \\ &\leq
    \sum_{k=1}^n \sum_{j=1}^{p_k} (y_{k,j} - y_{k,j-1}) \sup \{ \Norm{f(\xi) - f(\eta)} \colon \xi, \eta \in [y_{k,j-1}, y_{k,j}] \}
    \leq \\ &\leq
    \sum_{k=1}^n \sup \{ f(\xi) - f(\eta) \colon \xi, \eta \in [x_{k-1}, x_k] \} \underbrace{\sum_{j=1}^{p_k} (y_{k,j} - y_{k,j-1})}_{x_{k-1} - x_k}
    = \\ &=
    \omega(f, \Delta)
    <
    \varepsilon.
  \end{BreakableAlign*}

  Therefore the net \eqref{eq:def:riemann_integral/net} is fundamental and, since \( \CX \) is complete, the net converges to a limit.
\end{proof}

\begin{definition}\label{def:darboux_integral}
  Fix a real-valued function \( f: [a, b] \to \BR \). The \Def{upper Darboux sum} corresponding to the partition \eqref{eq:def:riemann_partition/partition} is defined as
  \begin{equation*}
    \Ol{S}(f, \Delta) \coloneqq \sum_{k=1}^n (x_{k-1} - x_k) \sup_{\xi \in [x_{k-1}, x_k]} f(\xi).
  \end{equation*}

  The \Def{lower Darboux sum} is defined as
  \begin{equation*}
    \Ul{S}(f, \Delta) \coloneqq \sum_{k=1}^n (x_{k-1} - x_k) \inf_{\xi \in [x_{k-1}, x_k]} f(\xi).
  \end{equation*}

  If the nets
  \begin{align}\label{eq:def:darboux_integral/nets}
    \{ \Ol{S}(f, \Delta) \}_{\Delta \in \Op{part}([a, b])}
    &&
    \{ \Ul{S}(f, \Delta) \}_{\Delta \in \Op{part}([a, b])}
  \end{align}
  have a common limit, we call this limit the \Def{Darboux integral} of \( f \) and, analogously to \fullref{def:riemann_integral}, we denote it by
  \begin{equation*}
    \int_a^b f(x) dx.
  \end{equation*}

  This notation is justified by \fullref{thm:darboux_integral_iff_riemann_integral}.
\end{definition}

\begin{proposition}\label{thm:darboux_integrable_iff_has_darboux_integral}
  A real-valued function \( f: [a, b] \to \BR \) is \hyperref[def:darboux_integrability]{Darboux integrable} if and only if it has a \hyperref[def:darboux_integral]{Darboux integral}.
\end{proposition}
\begin{proof}
  Note that, given the partition \eqref{eq:def:riemann_partition/partition}, we have
  \begin{align*}
    \Ol{S}(f, \Delta) - \Ul{S}(f, \Delta)
    &=
    \sum_{k=1}^n (x_k - x_{k-1}) \left[ \sup_{\xi \in [x_{k-1}, x_k]} f(\xi) - \inf_{\eta \in [x_{k-1}, x_k]} f(\eta) \right]
    = \\ &=
    \sum_{k=1}^n (x_k - x_{k-1}) \left[ \sup_{\xi \in [x_{k-1}, x_k]} f(\xi) + \sup_{\eta \in [x_{k-1}, x_k]} -f(\eta) \right]
    = \\ &=
    \sum_{k=1}^n (x_k - x_{k-1}) \sup \{ f(\xi) - f(\eta) \colon \xi, \eta \in [x_{k-1}, x_k] \}
    = \\ &=
    \sum_{k=1}^n (x_k - x_{k-1}) \sup \{ \Abs{f(\xi) - f(\eta)} \colon \xi, \eta \in [x_{k-1}, x_k] \}
    = \\ &=
    \omega(f, \Delta).
  \end{align*}

  Therefore the nets \eqref{eq:def:darboux_integral/nets} converge to a common limit if and only if \( \omega(f, \Delta) \xrightarrow[\Delta]{} 0 \). This finishes the proof.
\end{proof}

\begin{proposition}\label{thm:darboux_integral_iff_riemann_integral}
  A real-valued function \( f: [a, b] \to \BR \) has a \hyperref[def:darboux_integral]{Darboux integral} if and only if it has a \hyperref[def:riemann_integral]{Riemann integral}. Furthermore, the two integrals are equal.
\end{proposition}
\begin{proof}
  Fix \( \varepsilon > 0 \).

  \SubProofImplication{def:darboux_integral}{def:riemann_integral} Denote by \( I_D \) the Darboux integral of \( f \). Then there exists a partition \( \Delta_0 \) of \( [a, b] \) such that for any refinement \eqref{eq:def:riemann_partition/partition} of \( \Delta_0 \) we have
  \begin{equation*}
    \Ol{S}(f, \Delta) - \Ul{S}(f, \Delta) < \frac \varepsilon 2.
  \end{equation*}

  In particular, \( I_D - \Ul{S}(f, \Delta) < \tfrac \varepsilon 2 \).

  Let \( \Xi \coloneqq \{ \xi_k \}_{k=1}^n \) be tags for \( \Delta \). Then
  \begin{align*}
    \Abs{S(f, \Delta, \Xi) - I_D}
    &\leq
    \Abs{S(f, \Delta, \Xi) - \Ul{S}(f, \Delta)} - \Abs{\Ul{S}(f, \Delta) - I}
    \leq \\ &\leq
    \Abs{\Ol{S}(f, \Delta) - \Ul{S}(f, \Delta)} - \Abs{\Ul{S}(f, \Delta) - I}
    < \\ &<
    \frac \varepsilon 2 + \frac \varepsilon 2
    = \\ &=
    \varepsilon.
  \end{align*}

  Therefore \( I_D \) is also a Riemann integral for \( f \).

  \SubProofImplication{def:riemann_integral}{def:darboux_integral} Denote by \( I_R \) the Riemann integral of \( f \). Then there exists a partition \eqref{eq:def:riemann_integral/tagged_zero} such that for any partition \eqref{eq:def:riemann_partition/tagged} with \( \Diam(\Delta) \leq \Diam(\Delta_0) \), we have
  \begin{equation*}
    \Abs{S(f, \Delta, \Xi) - I_R} < \frac \varepsilon 2.
  \end{equation*}

  Since \eqref{thm:riemann_integrable_implies_bounded} is bounded, there exists a constant \( M > 0 \) such that \( \Abs{f(\xi) - f(\eta)} < M \) for any \( \xi, \eta \in [a, b] \).

  Using an analogous to \eqref{def:riemann_partition/subdiameter_splitting} splitting for the refinement \( \Gamma \coloneqq \Delta \cup \Delta_0 \) of \( \Delta_0 \), we obtain
  \begin{align*}
    \Ol{S}(f, \Gamma) - S(f, \Gamma, \Eta)
    &=
    \sum_{k=1}^{n_0} \sum_{k=1}^{p_k} [ \sup_{\xi \in [y_{k,j-1}, y_{k,j}]} f(\eta) - f(\eta_{k,j}) ] (y_{k,j} - y_{k,j-1})
    \leq \\ &\leq
    M \sum_{k=1}^{n_0} \sum_{k=1}^{p_k} (y_{k,j} - y_{k,j-1})
    \leq \\ &\leq
    M \cdot n_0 \cdot \Diam(\Gamma).
  \end{align*}

  By choosing a tagged partition \( (\Delta_1, \Xi_1) \) with \( \Diam(\Delta_1) < \min \left\{ \Diam(\Delta_0), \frac \varepsilon {2 M n_0} \right\} \), we obtain
  \begin{equation*}
    \Ol{S}(f, \Delta_1) - S(f, \Delta_1, \Xi) < \frac \varepsilon 2.
  \end{equation*}

  Therefore, whenever \( \Diam(\Delta) \leq \Diam(\Delta_1) \),
  \begin{equation*}
    \Ol{S}(f, \Delta) - I_R
    =
    \Ol{S}(f, \Delta) - S(f, \Delta, \Xi) + S(f, \Delta, \Xi) - I_R
    <
    \frac \varepsilon 2 + \frac \varepsilon 2
    =
    \varepsilon.
  \end{equation*}

  Thus the net \( \{ \Ol{S}(f, \Delta) \}_{\Delta \in \Op{part}([a, b])} \) of upper Darboux sums converges to \( I \). We can analogously show that the lower Darboux sums also converge to \( I_R \). Hence \( I_R \) is the Darboux integral of \( f \).
\end{proof}

\begin{proposition}\label{thm:countinuous_functions_integrable}
  In a \hyperref[def:frechet_space]{Frechet space} \( (\CX, \rho) \), \hyperref[def:global_continuity]{continuous functions} \( f: [a, b] \to \CX \) are \hyperref[def:darboux_integrability]{Darboux integrable}.
\end{proposition}
\begin{proof}
  Fix \( \delta > 0 \). Let \eqref{eq:def:riemann_partition/partition} be a partition of \( [a, b] \) such that \( \Diam(\Delta) < \delta \). We have
  \begin{equation*}
    \omega(f, \Delta)
    =
    \sum_{k=1}^n \omega(f, [x_{k-1}, x_k]) (x_k - x_{k-1})
    \leq
    \sum_{k=1}^n \omega(f, \Diam(\Delta)) \Diam(\Delta)
    <
    n \omega(f, \delta) \delta.
  \end{equation*}

  Now fix \( \varepsilon > 0 \). A continuous function on a compact interval is \hyperref[def:uniform_continuity]{uniformly continuous}. By \fullref{thm:modulus_of_continuity_properties/continuity_condition}, there exists \( \delta_0 > 0 \) such that \( \omega(f, \delta_0) < \varepsilon \). It is then enough to choose
  \begin{equation*}
    \delta \coloneqq \frac {\delta_0} {n \varepsilon}
  \end{equation*}
  to obtain
  \begin{equation*}
    \omega(f, \Delta)
    <
    n \delta \omega(f, \delta)
    =
    \delta_0 \frac {\omega(f, \delta)} {\varepsilon}
    \overset {\ref{thm:modulus_of_continuity_properties/monotone}} \leq
    \delta_0 \frac {\omega(f, \delta_0)} {\varepsilon}
    <
    \varepsilon.
  \end{equation*}

  Therefore the same inequality holds for all partitions with diameters smaller than \( \delta \), which implies that \( f \) is Darboux integrable.
\end{proof}

\begin{proposition}\label{thm:componentwise_integration}
  Let \( f: [a, b] \to \BR^n \) be a function and let \( f_k, k = 1, \ldots, n \) be its components. We have that \( f \) is integrable if and only if \( f_k \) is integrable for \( k = 1, \ldots, n \). Furthermore,
  \begin{equation}\label{eq:thm:componentwise_integration}
    \bigintss_a^b \begin{pmatrix} f_1(x) \\ \vdots \\ f_n(x) \end{pmatrix} dx
    =
    \begin{pmatrix} {\displaystyle \int_a^b f_1(x)} dx \\ \vdots \\ {\displaystyle \int_a^b f_n(x) dx} \end{pmatrix}.
  \end{equation}
\end{proposition}
\begin{proof}
  \Sufficiency Let \( f \) be integrable and let \( I = (I_1, \ldots, I_n)^T \) be the value of the integral. Fix \( \varepsilon > 0 \) and let \( (\Delta, \Xi) \) be a tagged partition such that
  \begin{equation*}
    \Norm{I - S(f, \Delta, \Xi)} < \varepsilon.
  \end{equation*}

  Then for any \( k = 1, \ldots, n \) we have
  \begin{equation*}
    \Norm{I - S(f, \Delta, \Xi)}^2
    =
    \sum_{m=1}^n \Abs{I_m - S(f_m, \Delta, \Xi)}^2
    \geq
    \Abs{I_k - S(f_k, \Delta, \Xi)}^2,
  \end{equation*}
  hence
  \begin{equation*}
    \Abs{I_k - S(f_k, \Delta, \Xi)} < \varepsilon.
  \end{equation*}

  Therefore \( f_k \) is integrable and
  \begin{equation*}
    \int_a^b f_k(x) dx = I_k.
  \end{equation*}

  \Necessity Let \( f_k \) be integrable for \( k = 1, \ldots, n \) with value \( I_k \). Put \( I \coloneqq (I_1, \ldots, I_n)^T \). Fix \( \delta > 0 \) and let \( (\Delta_k, \Xi_k) \) be a parition such that
  \begin{equation*}
    \Abs{I_k - S(f_k, \Delta_k, \Xi_k)} < \delta
  \end{equation*}

  Let \( \Gamma \coloneqq \bigcup_{k=1}^n \Delta_k \) and let \( \Eta \) be tags for \( \Gamma \). Since \( \Diam(\Gamma) \leq \Diam(\Delta_k) \) and since \( f_k \) is integrable, we have
  \begin{equation*}
    \Abs{I_k - S(f_k, \Gamma, \Eta)} < \delta \quad\forall k = 1, \ldots, n.
  \end{equation*}

  We have
  \begin{equation*}
    \Norm{I - S(f, \Gamma, \Eta)}
    =
    \sqrt{\sum_{m=1}^n \Abs{I_m - S(f_m, \Gamma, \Eta)}^2}
    <
    \delta \sqrt{n}.
  \end{equation*}

  Therefore, given \( \varepsilon > 0 \), it is enough to choose \( \delta \coloneqq \frac {\varepsilon} {\sqrt n} \) to obtain a tagged partition \( (\Gamma_0, \Eta_0) \) so that for \( (\Gamma, \Eta) \) with \( \Diam(\Gamma) < \Diam(\Gamma_0) \) we have
  \begin{equation*}
    \Norm{I - S(f, \Gamma, \Eta)} < \varepsilon.
  \end{equation*}

  This proves integrability of \( f \) and \eqref{eq:thm:componentwise_integration}.
\end{proof}

\subsection{Line integrals}\label{subsec:line_integrals}

\begin{definition}\label{def:length_of_parametric_curve}
  \begin{figure}
    \centering
    \todo{Add diagram}\iffalse\begin{mplibcode}
      input metapost/plotting;

      u := 4cm;

      beginfig(1)
      pair p[];

      vardef y(expr x) =
      if x < 1 / 3:
      -sqrt(1 - (2 * x * 4 / 3 - 1) ** 2) / 2
      else:
      -sqrt(1 - (x * 4 / 3 - 1 / 3) ** 2) / 2
      fi
      enddef;

      p[1] := (0, y(0)) scaled u;
      p[2] := (1 / 4, y(1 / 4)) scaled u;
      p[3] := (2 / 3, y(2 / 3)) scaled u;
      p[4] := (1, y(1)) scaled u;

      draw path_of_plot(y, 0, 1, 0.01, u);
      draw p[1] -- p[2] -- p[3] -- p[4];

      label.lft("$\gamma(0)$", p[1]);
      fill dot shifted p[1];

      label.bot("$\gamma(\tfrac 1 4)$", p[2]);
      fill dot shifted p[2];

      label.bot("$\gamma(\tfrac 2 3)$", p[3]);
      fill dot shifted p[3];

      label.rt("$\gamma(1)$", p[4]);
      fill dot shifted p[4];
      endfig;
    \end{mplibcode}\fi
    \Caption{def:length_of_parametric_curve/approximation}{A rough approximation of a curve using three line segments.}.
  \end{figure}

  Let \( X \) be a \hyperref[def:banach_space]{Banach space} and let \( \gamma: [a, b] \to X \) be a \hyperref{def:parametric_curve}{parametric curve}. In order to find the curve's length, we use the following procedure:
  \begin{itemize}
    \item Fix a nonnegative integer \( n \).

    \item Choose \( n - 1 \) points \( c_1, \ldots, c_{n-1} \) in \( [0, 1] \) and order them ascendingly. Define \( c_0 \coloneqq 0 \) and \( c_n \coloneqq 1 \). We will call the tuple \( c \coloneqq (c_0, \ldots, c_n) \) a \term{partition} of \( [0, 1] \) because it partitions \( [0, 1] \) into the subintervals \( [c_{k-1}, c_k], k = 1, \ldots, n \).

          Note that this choice does not actually require the axiom of choice since we will universally quantify all partitions.

    \item Use the partition \( c \) to build linear approximation to \( \gamma \) using the \hyperref[def:convex_set/line_segment]{line segments} \( [\gamma(c_{k-1}), \gamma(c_k)], k = 1, \ldots, n \).

    \item Find the total length of the approximation as
          \begin{equation*}
            \len_c (\gamma) \coloneqq \sum_{k=1}^n \norm{\gamma(c_k) - \gamma(c_{k-1})}.
          \end{equation*}

    \item Build a \hyperref[def:directed_set]{directed set} of all partitions by introducing an order that depends only the size of the tuples (i.e. we declare all partitions with the same size as equal).

    \item Using the defined directed set, build a \hyperref[def:topological_net]{net} that assigns to each partition the length \( \len_c(\gamma) \). The \hyperref[def:net_convergence/limit]{limit} of this net, if it exists, is called the \term{length} of the curve \( \gamma \) and is denoted by \( \len(\gamma) \).
  \end{itemize}

  If the curve \( \gamma \) has a length, it is called \term{rectifiable}.
\end{definition}

\begin{proposition}\label{thm:length_of_smooth_curves}
  For a \hyperref[def:differentiability/frechet]{differentiable} parametric curve \( \gamma: [a, b] \to X \) we have
  \begin{equation*}
    \len(\gamma) \coloneqq \int_a^b \norm{\gamma'(x)} dx.
  \end{equation*}
\end{proposition}
\begin{proof}
  By the mean value theorem, when constructing the length in \fullref{def:length_of_parametric_curve}, for each \( k = 1, \ldots, n \) there exists a point \( \xi_k \in [c_{k-1}, c_k] \) such that
  \begin{equation*}
    \gamma(c_k) - \gamma(c_{k-1}) = \gamma'(\xi_k) (c_k - c_{k-1}).
  \end{equation*}

  Therefore
  \begin{equation*}
    \len_n (\gamma)
    =
    \sum_{k=1}^{n+1} \norm{\gamma(c_k) - \gamma(c_{k-1})}
    =
    \sum_{k=1}^{n+1} \norm{\gamma'(\xi_k)} (c_k - c_{k-1})
    \to
    \int_a^b \norm{\gamma'(x)} dx.
  \end{equation*}
\end{proof}

\begin{corollary}\label{thm:length_of_function_graph}
  The \hyperref[def:length_of_parametric_curve]{length} of the \hyperref[def:function/graph]{graph} of a \hyperref[def:differentiability/frechet]{differentiable} function \( f: [a, b] \to \BbbR \), if it exists, is given by
  \begin{equation*}
    \len(\gph(f)) \coloneqq \int_a^b \sqrt{1 + f'(x)} dx.
  \end{equation*}
\end{corollary}
\begin{proof}
  Apply \fullref{thm:length_of_smooth_curves} for the parametric curve \( \gamma(x) \coloneqq = \gph(y^+(x)) = (x, f(x)) \).
\end{proof}

\subsection{Total variation}\label{subsec:total_variation}

\begin{definition}\label{def:riemann_stieltjes_integral}
  The most common generalization of the \hyperref[def:riemann_integral]{Riemann integral} is the Riemann-Stieltjes integral. It is not as well-behaved, hence we will give the most general definition and not attempt to prove equivalences.

  Let \( \mscrX \) be a real \hyperref[def:separation_axioms/T2]{Hausdorff} \hyperref[def:topological_vector_space]{topological vector space}. Fix two \hyperref[def:function]{functions} \( f, \alpha: [a, b] \to X \).

  The \term{Riemann sum} of \( f \) with respect to \( \alpha \) corresponding to the \hyperref[def:riemann_partition/tagged]{tagged partition} \eqref{eq:def:riemann_partition/tagged} is defined as
  \begin{equation*}
    S(f, \alpha, \Delta, \Xi) \coloneqq \sum_{k=1}^n f(\xi_k) (\alpha(x_k) - \alpha(x_{k-1})).
  \end{equation*}

  The limit of the net
  \begin{equation}\label{eq:def:riemann_stieltjes_integral/net}
    \{ S(f, \alpha, \Delta, \Xi) \}_{(\Delta, \Xi) \in \op{tpart}([a, b])},
  \end{equation}
  if it exists, is called the \term{Riemann-Stieltjes integral} of \( f \) with respect to \( \alpha \) and is denoted by
  \begin{equation*}
    \int_a^b f(x) d \alpha(x).
  \end{equation*}
\end{definition}


% Complex analysis
\section{Complex analysis}\label{sec:complex_analysis}

A lot of results in this section hold in both \( \BbbR \) and \( \BbbC \), so \( \BbbK \) will refer to either \( \BbbR \) or \( \BbbC \).

\subsection{Complex functions}\label{subsec:complex_functions}

\begin{definition}\label{def:sequence_spaces}
  We will define multiple Banach spaces of sequences over \( \BbbC \).

  \begin{defenum}
    \ilabel{def:sequence_spaces/c00} The simplest nontrivial sequence space is that of all sequences with only finitely many nonzero elements. It is denoted by \( c_{00} \). It can be defined as
    \begin{equation*}
      c_{00} \coloneqq \bigcup_{i=1}^\infty \BbbC^k,
    \end{equation*}
    where \( \BbbC^k \) is the corresponding \hyperref[def:left_module_of_tuples]{tuple space}.

    This space can be generalized to modules over \hyperref[def:left_module]{dioids}.
  \end{defenum}
\end{definition}

\begin{definition}\label{def:function_spaces}
  We will define multiple Banach spaces of functions over \( \BbbK \).

  \begin{defenum}
    \ilabel{def:function_spaces/c0} Define the set of functions \term{vanishing at infinity}:
    \begin{equation*}
      C_0(\BbbC) \coloneqq \{ f: \BbbC \to \BbbC \colon f(x) \xrightarrow[\abs{x} \to \infty]{} 0 \}.
    \end{equation*}

    \ilabel{def:function_spaces/c} Fix \hyperref[def:topological_space]{topological space} \( X \). The set \( C(X) = C(X, \BbbK) \) of all \( \BbbK \)-valued continuous functions on \( X \) in a Banach space over \( \BbbK \).
  \end{defenum}
\end{definition}

\begin{theorem}[Arzela-Ascoli]\label{thm:arzela_ascoli}\mcite[cor. 10.49]{Knapp2016BAlg}
  Let \( X \) be a \hyperref[def:compact_space]{compact} \hyperref[def:separation_axioms/T2]{Hausdorff} space.

  A family \( \mscrF \subseteq C(X, \BbbR) \) of continuous real-valued functions is totally \hyperref[def:totally_bounded_set]{bounded} if and only if it is pointwise \hyperref[def:bounded_function/pointwise]{bounded} and \hyperref[def:function_set_continuity/equicontinuous]{equicontinuous}.
\end{theorem}

\subsection{Series}\label{subsec:series}

Here \( (X, \norm) \) will refer to a Banach space over \( \BbbK \).

\begin{definition}\label{def:convergent_series}
  When extending addition to a countable amount of terms, we need to impose some regularity conditions to avoid contradictions. The topologies of \( \BbbR \) and \( \BbbC \) are complete and allow us to define convergent and divergent series. We define series in great generality because the theory easily allows it.

  A \term{numeric series} or simply \term{series} is an infinite sequence \( x_0, x_1, \ldots \in X \), which we call \term{terms}, usually written as
  \begin{equation}\label{def:convergent_series/series}
    \sum_{k=0}^\infty x_k.
  \end{equation}

  To each series, there corresponds its sequence of \term{partial sums}
  \begin{equation*}
    S_n \coloneqq \sum_{k=0}^n x_k, n = 0, 1, 2, \ldots
  \end{equation*}

  We can equivalently define a series as a sequence of partial sums and then recover the terms as
  \begin{equation*}
    x_k \coloneqq \begin{cases}
      S_0,           & k = 0, \\
      S_k - S_{k-1}, & k > 0
    \end{cases}
  \end{equation*}

  We say that the series \eqref{def:convergent_series/series} \term{converges} to a value \( x \) if \( \lim_{n \to \infty} S_n = x \) in the sense of \fullref{def:net_convergence/limit}. The value \( x \) is called the \term{sum} of the series.

  If a series does not converge, we say that it is \term{divergent}.

  If the related series
  \begin{equation}\label{def:convergent_series/absolute_series}
    \sum_{k=0}^\infty \norm{x_k}
  \end{equation}
  converges, we say that \eqref{def:convergent_series/series} is \term{absolutely convergent}.
\end{definition}

\begin{example}\label{ex:series}
  Several examples of series are
  \begin{itemize}
    \item An absolutely convergent series is \fullref{thm:geometric_series_properties/series_sum_interior}
    \item A divergent series is \fullref{thm:harmonic_series_properties/harmonic_series}.
    \item A convergent, but not absolutely convergent series is \fullref{thm:harmonic_series_properties/harmonic_series}.
  \end{itemize}
\end{example}

\begin{proposition}\label{thm:absolutely_convergent_series_is_convergent}
  An absolutely convergent series is convergent.
\end{proposition}
\begin{proof}
  Suppose that \eqref{def:convergent_series/absolute_series} converges.

  By the triangle inequality, for each index \( n \) we have
  \begin{equation*}
    \norm{\sum_{k=0}^n x_k} \leq \sum_{k=0}^n \norm{x_k} \leq \sum_{k=0}^\infty \norm{x_k}.
  \end{equation*}

  Thus, the sequence \( \left\{ \norm{\sum_{k=0}^{n} x_k} \right\}_{n=0}^\infty \) is a bounded (by \( \sum_{k=0}^\infty \norm{x_k} \)) monotone sequence, which by \fullref{thm:real_monotone_sequence_converges_iff_bounded} is convergent.

  Therefore, the series \eqref{def:convergent_series/series} is convergent.
\end{proof}

\begin{remark}\label{rem:establish_series_convergence_by_absolute_series}
  Convergence of the series \eqref{def:convergent_series/series} can be established using the convergence of the nonnegative series \eqref{def:convergent_series/absolute_series}.

  The convergence of the latter can be established using techniques in \fullref{subsec:real_series} like \fullref{thm:cauchys_root_test} or \fullref{thm:dalamberts_ratio_test}.
\end{remark}

\begin{proposition}\label{thm:infinitary_triangle_inequality}
  For every series \eqref{def:convergent_series/series} we have
  \begin{equation}\label{thm:infinitary_triangle_inequality/inequality}
    \norm{\sum_{k=0}^\infty x_k} \leq \sum_{k=0}^\infty \norm{x_k},
  \end{equation}
  where both limits are allowed to be infinite.
\end{proposition}
\begin{proof}
  If the series on the right diverges, the inequality is obviously true.

  Suppose that it is convergent. By \fullref{thm:absolutely_convergent_series_is_convergent}, the limit
  \eqref{def:convergent_series/series} exists.

  By the triangle inequality, for each index \( n \) we have
  \begin{equation*}
    \norm{\sum_{k=0}^n x_k} \leq \sum_{k=0}^n \norm{x_k}.
  \end{equation*}

  By \fullref{thm:one_sided_squeeze_lemma}, since both sequences are convergent, we obtain \fullref{thm:infinitary_triangle_inequality/inequality}.
\end{proof}

\begin{proposition}\label{thm:convergent_series_terms_vanish}
  The terms of the convergent series \eqref{def:convergent_series/series} vanish as \( k \to \infty \), that is,
  \begin{equation*}
    \lim_{k \to \infty} x_k = 0.
  \end{equation*}
\end{proposition}
\begin{proof}
  Since the series is convergent, its sequence of partial sums converges, i.e. the partial sums get arbitrarily close to each other. Then
  \begin{equation*}
    \norm{x_n} = \norm{S_n - S_{n-1}} \to 0.
  \end{equation*}
\end{proof}

\begin{theorem}\label{thm:product_of_series_convergence}
  Consider two convergent series
  \begin{equation}\label{thm:product_of_series_convergence/a}
    A \coloneqq \sum_{k=0}^\infty x_k
  \end{equation}
  and
  \begin{equation}\label{thm:product_of_series_convergence/b}
    B \coloneqq \sum_{k=0}^\infty y_k.
  \end{equation}

  If either \fullref{thm:product_of_series_convergence/a} or \fullref{thm:product_of_series_convergence/b} converges absolutely, then
  \begin{equation}\label{thm:product_of_series_convergence/prod}
    \sum_{k=0}^\infty \sum_{m=0}^k x_m y_{k-m} = AB.
  \end{equation}
\end{theorem}

\begin{proposition}[Cauchy's series convergence criterion]\label{thm:cauchy_series_convergence_criterion}\mcite[3.22]{Rudin1976Principles}
  The series \eqref{def:convergent_series/series} converges if and only if for every \( \varepsilon > 0 \) there exists an index \( k_0 \) such that
  \begin{equation*}
    \norm{\sum_{k=m}^n x_k} < \varepsilon \quad\forall m, n \geq k_0.
  \end{equation*}
\end{proposition}
\begin{proof}
  This is simply a restatement of \fullref{thm:cauchys_net_convergence_criterion}.
\end{proof}

\begin{proposition}[Cauchy's series continuity criterion]\label{thm:cauchy_series_continuity_criterion}\mcite[\textnumero 265]{Фихтенгольц1968Том2}
  Fix a topological space \( A \) and a set \( S \subseteq A \). Let \( \{ f_k \}_{k=0}^\infty \) be a sequence of continuous functions from \( S \) to \( X \).

  Define the function \( f: S \to X \) as
  \begin{equation}\label{thm:cauchy_series_continuity_criterion/function}
    f(x) \coloneqq \sum_{k=0}^\infty f_k(x).
  \end{equation}

  A sufficient condition for \( f \) to be continuous in \( S \) is that for every \( \varepsilon > 0 \) there exists an index \( K \) such that
  \begin{equation*}
    \norm{\sum_{k=m}^n f(x)} < \varepsilon \quad\forall m, n \geq K
  \end{equation*}
  simultaneously for all \( x \in S \).
\end{proposition}
\begin{proof}
  This is simply a restatement of \fullref{thm:uniform_limit_of_continuous_functions} in the style of \fullref{thm:cauchy_series_convergence_criterion}.
\end{proof}

\begin{corollary}[Weierstrass' series criterion]\label{thm:weierstrass_series_criterion}\mcite[\textnumero 265]{Фихтенгольц1968Том2}
  Let \( S \) be any set and \( \{ f_k \}_{k=0}^\infty \) be a sequence of functions from \( S \) to \( X \). Consider the series \fullref{thm:cauchy_series_continuity_criterion/function}. If
  \begin{equation*}
    \forall k \in \BbbZ^{>0} \ \exists M_k \in \BbbR^{>0} \ \forall x \in S : \norm{f_k(x)} < M_k
  \end{equation*}
  and if the series
  \begin{equation}\label{thm:weierstrass_series_criterion/dominating}
    \sum_{k=0}^\infty M_k
  \end{equation}
  converges, then the limit \fullref{thm:cauchy_series_continuity_criterion/function} exists for every \( x \in S \) and, furthermore, the series converges absolutely and uniformly.

  In analogy to \fullref{thm:positive_series_comparison}, we say that the series \fullref{thm:weierstrass_series_criterion/dominating} \term{dominates} the series \fullref{thm:cauchy_series_continuity_criterion/function}.

  In particular, if \( S \) has a topology and the functions \( f_k(x), k = 0, 1, \ldots \) are continuous (resp. uniformly continuous), so is \( f(x) \).
\end{corollary}
\begin{proof}
  By \fullref{thm:positive_series_comparison}, the series
  \begin{equation*}
    \sum_{k=0}^\infty \norm{f_k(x)}
  \end{equation*}
  converges for any \( x \in S \), hence \fullref{thm:cauchy_series_continuity_criterion/function} converges absolutely for any \( x \in S \).

  Furthermore, each of the functions \( f_k(x) \) is bounded by \( B(0, M_k) \) and \( M_k \) does not depend on \( x \), hence the convergence is uniform.

  The rest of the theorem follows from \fullref{thm:uniform_limit_of_continuous_functions}.
\end{proof}

\begin{corollary}\label{thm:continuous_function_series_powers_of_two}
  Let \( X \subseteq \BbbR \) be a nonempty set. Consider the series of real-valued real functions
  \begin{equation}\label{thm:continuous_function_series_powers_of_two/series}
    f(x) \coloneqq \sum_{k=0}^\infty \frac {f_k(x)} {2^k},
  \end{equation}
  where \( \{ f_k \}_{k=0}^\infty \subseteq B_{C(X)} \) is a sequence of continuous functions bounded in \( [-1, 1] \).

  Then \( f(x) \) is defined and continuous for all \( x \in X \).
\end{corollary}
\begin{proof}
  For \( \abs{x} \leq 1 \), the series is dominated by the geometric series \fullref{ex:n_ary_decomposition/series}, which sums to \( 2 \), hence by \fullref{thm:weierstrass_series_criterion} \( f(x) \) is continuous in the interval \( [-1, 1] \).

  Note that
  \begin{equation*}
    f(2x) \coloneqq \sum_{k=0}^\infty \frac x {2^{k-1}} = 2 f(x),
  \end{equation*}
  hence the series \fullref{thm:continuous_function_series_powers_of_two/series} also converges for \( \abs{x} \leq 2 \).

  By induction on \( n \), we show that \( f(2^n x) = 2^n f(x) \) and thus \( f(x) \) is continuous in \( B(0, 2^n) \), therefore also on the entire real line \( \BbbR \).
\end{proof}

\begin{example}\label{thm:weierstrass_series_criterion/counterexample}\mcite[\textnumero 266]{Фихтенгольц1968Том2}
  Consider the real series
  \begin{equation*}
    f(x) \coloneqq \sum_{k=0}^\infty x^k (1 - x).
  \end{equation*}

  It converges for \( \abs{x} < 1 \) because it is dominated by a convergent geometric series.

  For \( x \in (0, 1) \),
  \begin{equation*}
    f(x)
    =
    \sum_{k=0}^\infty x^k (1 - x)
    =
    \sum_{k=0}^\infty x^k - \sum_{k=1}^\infty x^k
    =
    1.
  \end{equation*}

  But
  \begin{equation*}
    \lim_{t \uparrow 1} f(x) = 1 \neq 0 = f(1) = f(\lim_{t \uparrow 1} t).
  \end{equation*}

  This shows that \( f(x) \) is not continuous, despite every term being continuous.

  By contraposition to \fullref{thm:weierstrass_series_criterion}, it follows that no series that dominates \( f(x) \) converges.
\end{example}

\begin{theorem}\label{thm:uniform_limit_exchange}\mcite[\textnumero 268]{Фихтенгольц1968Том2}
  Fix a uniform space \( (A, \mscrU) \) and let \( S \subseteq A \). Let \( f_k: S \to X, k = 0, 1, \ldots \) be a sequence of functions and assume that \( x_0 \in M \) is a limit point of each of these functions.

  \begin{thmenum}
    \thmitem{thm:uniform_limit_exchange/sequence} If the sequence \( \{ f_k \}_{k=0}^\infty \) converges uniformly on \( S \), we can exchange the limits
    \begin{equation*}
      \lim_{x \to x_0} \lim_{k \to \infty} f_k(x)
      =
      \lim_{k \to \infty} \lim_{x \to x_0} f_k(x).
    \end{equation*}

    \thmitem{thm:uniform_limit_exchange/series} If the series \fullref{thm:cauchy_series_continuity_criterion/function} converges uniformly on \( S \), we can exchange the limits
    \begin{equation*}
      \lim_{x \to x_0} \sum_{k=0}^\infty f_k(x)
      =
      \sum_{k=0}^\infty \lim_{x \to x_0} f_k(x).
    \end{equation*}
  \end{thmenum}
\end{theorem}

\begin{remark}\label{rem:thm:uniform_limit_exchange_continuity}
  If the functions \( f_k \) in \fullref{thm:uniform_limit_exchange/series} are continuous at \( x_0 \), we have the additional equality
  \begin{equation}\label{thm:uniform_limit_exchange/continuous_equality}
    \lim_{x \to x_0} f(x)
    =
    \lim_{x \to x_0} \sum_{k=0}^\infty f_k(x)
    =
    \sum_{k=0}^\infty \lim_{x \to x_0} f_k(x)
    \reloset * =
    \sum_{k=0}^\infty f_k\left(\lim_{x \to x_0} x \right)
    =
    f\left(\lim_{x \to x_0} x \right),
  \end{equation}
  thus \( f \) is continuous at \( x_0 \). The continuity actually follows from \fullref{thm:cauchy_series_continuity_criterion} directly.
\end{remark}

\begin{corollary}\label{thm:riemann_intergral_limit_exchange}\mcite[\textnumero 269]{Фихтенгольц1968Том2}
  Let \( \{ f_k \}_{k=0}^\infty \subseteq C([a, b], \BbbR) \).

  \begin{thmenum}
    \thmitem{thm:riemann_intergral_limit_exchange/sequence} If the sequence \( \{ f_k \}_{k=0}^\infty \) converges uniformly, then
    \begin{equation*}
      \lim_{k \to \infty} \int_a^b f_k(x) dx = \int_a^b \lim_{k \to \infty} f_k(x) dx.
    \end{equation*}

    \thmitem{thm:riemann_intergral_limit_exchange/series} If the series \fullref{thm:cauchy_series_continuity_criterion/function} converges uniformly, then
    \begin{equation*}
      \int_a^b f(x) dx = \int_a^b \sum_{k=0}^\infty f_k(x) dx = \sum_{k=0}^\infty \int_a^b f_k(x) dx.
    \end{equation*}
  \end{thmenum}
\end{corollary}
\begin{proof}
  \SubProofOf{thm:riemann_intergral_limit_exchange/sequence} Assume that the sequence \( \{ f_k \}_{k=0}^\infty \) converges uniformly to \( f \). Then by \fullref{thm:uniform_limit_exchange}, we note that for any index \( k \), the difference \( r_k(x) \coloneqq f(x) - f_k(x) \) is continuous, hence integrable, and
  \begin{equation*}
    \int_a^b f(x) dx = \int_a^b f_k(x) dx + \int_a^b r_k(x) dx.
  \end{equation*}

  Because of the uniform convergence, for any \( \delta > 0 \) and there exist an index \( k_0 \) such that
  \begin{equation*}
    \abs{f(x) - f_k(x)} = \abs{r_k(x)} < \delta \quad\forall k \geq k_0, \forall x \in [a, b].
  \end{equation*}

  Then
  \begin{equation*}
    \abs{\int_a^b f(x) dx - \int_a^b f_k(x) dx} = \abs{\int_a^b r_k(x) dx} < (b - a) \delta.
  \end{equation*}

  Given \( \varepsilon > 0 \), we define \( \delta \coloneqq \frac \varepsilon {b - a} \) to obtain an index \( k_0 \) such that
  \begin{equation*}
    \abs{\int_a^b f(x) dx - \int_a^b f_k(x) dx} = \abs{\int_a^b r_k(x) dx} < \varepsilon \quad\forall k \geq k_0.
  \end{equation*}

  Thus, \fullref{def:net_convergence/limit} is satisfied and equality holds.

  \SubProofOf{thm:riemann_intergral_limit_exchange/series} This is a special case of \fullref{thm:riemann_intergral_limit_exchange/sequence}.
\end{proof}

\begin{corollary}\label{thm:derivative_limit_exchange}\mcite[thm. 7.17]{Rudin1976Principles}
  Let \( \{ f_k \}_{k=0}^\infty \subseteq C^1([a, b], \BbbR) \). Suppose that the series \fullref{thm:cauchy_series_continuity_criterion/function} converges for at least one point \( x_0 \in [a, b] \).

  \begin{thmenum}
    \thmitem{thm:derivative_limit_exchange/sequence} If the sequence \( \{ D f_k \}_{k=0}^\infty \) of derivatives converges uniformly, then \( \{ f_k \}_{k=0}^\infty \) also converges uniformly, its limit is differentiable in \( (a, b) \) and
    \begin{equation*}
      D\left(\lim_{k \to \infty} f_k(x) \right) = \lim_{k \to \infty} D f_k(x).
    \end{equation*}

    \thmitem{thm:derivative_limit_exchange/series} If the series of derivatives
    \begin{equation}\label{thm:derivative_limit_exchange/derivative_series}
      \sum_{k=0}^\infty D f_k(x)
    \end{equation}
    converges uniformly, then \fullref{thm:cauchy_series_continuity_criterion/function} converges uniformly, is differentiable in \( (a, b) \) and
    \begin{equation*}
      D\left(\sum_{k=0}^\infty f_k(x)\right) = \sum_{k=0}^\infty D f_k(x).
    \end{equation*}
  \end{thmenum}
\end{corollary}
\begin{proof}
  \SubProofOf{thm:derivative_limit_exchange/sequence} Fix \( \varepsilon > 0 \). Since the sequence \( \{ f_k \}_{k=0}^\infty \) converges for \( x_0 \), there exists an index \( k_0 \) such that
  \todo{Prove complex case}
  \begin{equation*}
    \abs{f_m(x_0) - f_n(x_0)} < \varepsilon \quad\forall m, n \geq k_0.
  \end{equation*}

  Furthermore, there exists an index \( k_1 \) such that
  \begin{equation*}
    \abs{D f_m(x) - D f_n(x)} < \varepsilon \quad\forall x \in [a, b] \ \forall m, n \geq k_0.
  \end{equation*}

  Fix \( m, n \geq k_0 \) and \( x \in [a, b] \). Note that the function \( f_m - f_n \) is differentiable and thus by the mean value theorem, there exists \( \xi \) between \( x_0 \) and \( x \) such that
  \begin{equation*}
    \frac {[f_m(x) - f_n(x)] - [f_m(x_0) - f_n(x_0)]} {x - x_0} = D f_m(\xi) - D f_n(\xi).
  \end{equation*}

  Thus,
  \begin{balign*}
    \abs{f_m(x) - f_n(x)}
     & \leq
    \abs{f_m(x_0) - f_n(x_0)} + (x - x_0)\abs{D f_m(\xi) - D f_n(\xi)}
    <       \\ &<
    2 (x - x_0) \varepsilon
    \leq    \\ &\leq
    2 (b - a) \varepsilon.
  \end{balign*}

  Therefore, the limit \( \lim_{k\to\infty} f_k(x) \) exists. Since \( x \) was arbitrary and \( 2 (b - a) \varepsilon \) does not depend on \( x \), we conclude that
  \begin{equation*}
    f(x) \coloneqq \lim_{k\to\infty} f_k(x)
  \end{equation*}
  is uniformly convergent on \( [a, b] \).

  By the Newton-Leibniz theorem, for the sequence \( \{ D f_k \}_{k=0}^\infty \) of derivatives we have
  \begin{equation*}
    \lim_{k \to \infty} \int_a^x D f_k(t) dt
    =
    \lim_{k \to \infty} [f_k(x) - f_k(a)]
    =
    \lim_{k \to \infty} f_k(x) - \lim_{k \to \infty} f_k(a)
    =
    f(x) - f(a).
  \end{equation*}

  Differentiating both sides, we obtain
  \begin{equation*}
    D\left(\lim_{k \to \infty} \int_a^x D f_k(t) dt \right)
    =
    D\left(\lim_{k \to \infty} f_k(x) dt \right)
    =
    D f(x).
  \end{equation*}

  \Fullref{thm:riemann_intergral_limit_exchange/sequence} allows us to conclude that
  \begin{equation*}
    D f(x)
    =
    D\left(\lim_{k \to \infty} \int_a^x D f_k(t) dt \right)
    =
    D \int_a^x \lim_{k \to \infty} D f_k(t) dt
    =
    \lim_{k \to \infty} D f_k(x).
  \end{equation*}

  \SubProofOf{thm:derivative_limit_exchange/series} This is a special case of \fullref{thm:riemann_intergral_limit_exchange/sequence}.
\end{proof}

\subsection{Power series}\label{subsec:power_series}

\begin{definition}\label{def:convergent_power_series}
  Let \( \BbbK\Bracks{X} \) be the space of power series defined in \fullref{def:formal_power_series}.

  To each formal power series
  \begin{equation*}
    \sum_{k=0}^\infty a_k X^k
  \end{equation*}
  there corresponds a function, called a \term{power series}
  \begin{equation}\label{def:convergent_power_series/series}
    p(x) \coloneqq \sum_{k=0}^\infty a_k x^k.
  \end{equation}

  We sometimes slightly generalize this notion slightly by using a \enquote{shift} by \( \alpha \in \BbbK \): define the function
  \begin{equation}\label{def:convergent_power_series/shifted_series}
    p(x) \coloneqq \sum_{k=0}^\infty a_k (x - \alpha)^k.
  \end{equation}

  If the limit exists (as a \hyperref[def:convergent_series]{numeric series}) for a certain \( x \in \BbbK \), we say that the series \term{converges} at \( x \).

  The series is no longer \enquote{formal} because it is now a proper function instead of an abstract algebraic object, although a power series may only be defined in a subset of \( \BbbK \) (that is, a \hyperref[def:function/partial]{partial function}).
\end{definition}

\begin{theorem}\label{thm:power_series_radius_of_convergence}
  For every power series \eqref{def:convergent_power_series/series}, there exists a nonnegative extended real number \( r \in [0, +\infty] \), called its \term{radius of convergence}, such that \eqref{def:convergent_power_series/series} converges absolutely if \( \abs{x} < r \) and diverges if \( \abs{x} > r \).

  The behavior of the series is more complicated when \( \abs{x} = r \) (unless \( r = 0 \), in which case the power series converges if and only if \( x = 0 \)).
\end{theorem}
\begin{proof}
  Define
  \begin{equation*}
    q \coloneqq \limsup_{n \to \infty} \sqrt[n]{\abs{a_n}},
  \end{equation*}
  where we put \( q = +\infty \) if the limit does not exist. We have
  \begin{equation*}
    \limsup_{n \to \infty} \sqrt[n]{\abs{x^n a_n}} = \abs{x} q.
  \end{equation*}

  By \fullref{thm:cauchys_root_test}, \eqref{def:convergent_power_series/series} converges absolutely if \( \abs{z} q < 1 \) and diverges if \( \abs{z} q > 1 \).

  Thus \( r \coloneqq \tfrac 1 q \) is the desired radius of convergence.

  Note that we may also use \fullref{thm:dalamberts_ratio_test} for finding the same radius of convergence by \fullref{rem:nonnegative_series_convergence_test_equivalence}.
\end{proof}

\begin{proposition}\label{thm:power_series_parity}
  Power series of the form
  \begin{equation}\label{thm:power_series_parity/odd}
    f_o(z) \coloneqq \sum_{m \text{ is odd}} a_m z^m = \sum_{k=0}^\infty a_{2k+1} z^{2k+1}
  \end{equation}
  are \hyperref[def:group/function_parity]{odd functions} and power series of the form
  \begin{equation}\label{thm:power_series_parity/even}
    f_e(z) \coloneqq \sum_{m \text{ is even}} a_m z^m = \sum_{k=0}^\infty a_{2k} z^{2k}
  \end{equation}
  are even functions.
\end{proposition}
\begin{proof}
  If \eqref{thm:power_series_parity/odd} converges for \( z \in \BbbC \),
  \begin{equation*}
    f_o(-z)
    =
    \sum_{k=0}^\infty a_{2k+1} (-z)^{2k+1}
    =
    \sum_{k=0}^\infty a_{2k+1} (-1)^{2k+1} z^{2k+1}
    =
    - \sum_{k=0}^\infty a_{2k+1} z^{2k+1}
    =
    - f_o(z).
  \end{equation*}

  Analogously, since \( (-1)^{2k} = 1 \), we have \( f_e(-z) = f_e(z) \).
\end{proof}

\begin{proposition}\label{thm:power_series_are_locally_uniform_convergent}
  A power series is \hyperref[def:function_net_convergence/locally_uniform]{locally uniformly convergent} in the interior of its domain of convergence.
\end{proposition}
\begin{proof}
  Assume that the series \eqref{def:convergent_power_series/series} converges inside the ball \( B(0, R) \). Fix \( x \in B(0, R) \) and \( R_x < R - \abs{x} \). Then the geometric series
  \begin{equation*}
    \sum_{k=0}^\infty a_k R_x^k
  \end{equation*}
  converges and dominates \eqref{def:convergent_power_series/series} in the ball \( B(x, R_x) \). Thus by \fullref{thm:weierstrass_series_criterion}, \eqref{def:convergent_power_series/series} converges uniformly in \( B(x, R_x) \).

  Since the choice of \( x \in B(0, R) \) was arbitrary, we conclude that \eqref{def:convergent_power_series/series} is locally uniformly convergent.
\end{proof}

\begin{theorem}\label{thm:series_termwise_operations}
  Suppose that the power series \eqref{def:convergent_power_series/series} has a (potentially infinite) radius of convergence \( R \).

  \begin{thmenum}
    \ilabel{thm:series_termwise_operations/differentiation} \( p(x) \) is differentiable in \( B(0, R) \) and can be differentiated termwise as
    \begin{equation}\label{thm:series_termwise_operations/derivative}
      p'(x) = \sum_{k=0}^\infty a_{k+1} (k+1) x^k.
    \end{equation}

    Furthermore, \( p'(x) \) has the same radius of convergence as \( p(x) \).

    \ilabel{thm:series_termwise_operations/integration} If the series is real and \( \abs{x} < R \), \( p(x) \) is integrable in \( [0, x] \) (or \( [x, 0] \)) and can be integrated termwise as
    \begin{equation}\label{thm:series_termwise_operations/primitive}
      \int_0^x p(t) dt = \sum_{k=0}^\infty a_k \frac {x^{k+1}} {k+1}.
    \end{equation}
  \end{thmenum}
\end{theorem}
\begin{proof}
  \SubProofOf{thm:series_termwise_operations/differentiation} Note that the right-hand side of \fullref{thm:series_termwise_operations/derivative} is a power series. Furthermore, its radius of convergence is, by \fullref{thm:power_series_radius_of_convergence},
  \begin{equation*}
    \lim_{k \to \infty} \abs{\frac {a_{k+1} (k+1) x^k} {a_{k+2} (k+2) x^{k+1}}}
    =
    \abs{x} \lim_{k \to \infty} \frac {k+1} {k+2} \abs{\frac {a_{k+1}} {a_{k+2}}}
    =
    R.
  \end{equation*}

  Fix \( x \in B(0, R) \) and choose \( r \in (\abs{x}, R) \). Both series are uniformly convergent in \( B(0, r) \). By \fullref{thm:derivative_limit_exchange/sequence}, the equality \fullref{thm:series_termwise_operations/derivative} holds in \( B(0, r) \), hence it also holds for \( x \).

  \SubProofOf{thm:series_termwise_operations/integration} Analogously to \fullref{thm:series_termwise_operations/differentiation}, we conclude that the right-hand side of \fullref{thm:series_termwise_operations/primitive} is a power series with radius of convergence \( R \).

  The rest follows directly from \fullref{thm:riemann_intergral_limit_exchange}.
\end{proof}

\subsection{Trigonometric functions}\label{subsec:trigonometric_functions}

\begin{definition}\label{def:trigonometric_functions}
  We define the two basic \term{trigonometric functions}. They are also called \term{circular trigonometric functions} to distinguish them from the hyperbolic trigonometric functions defined and motivated in \fullref{def:hyperbolic_trigonometric_functions}.

  \begin{thmenum}
    \thmitem{def:trigonometric_functions/sine} The \term{sine} function, also called the \term{sinus} function, is
    \begin{equation*}
      \sin(z)
      \coloneqq
      -i \sum_{m \text{ is odd}}^\infty \frac {i^m z^m} {m!}
      =
      -i \sum_{k=0}^\infty \frac {i^{2k+1} z^{2k+1}} {(2k + 1)!}
      =
      \sum_{k=0}^\infty \frac {i^{2k} z^{2k+1}} {(2k + 1)!}
    \end{equation*}

    \thmitem{def:trigonometric_functions/cosine} The \term{cosine} function, also called the \term{cosinus} function, is
    \begin{equation*}
      \cos(z)
      \coloneqq
      \sum_{m \text{ is even}}^\infty \frac {i^m z^m} {m!}
      =
      \sum_{k=0}^\infty \frac {i^{2k} z^{2k}} {(2k)!}.
    \end{equation*}
  \end{thmenum}

  \Fullref{def:geometric_trigonometric_functions} justifies the term \enquote{angle} for the \hyperref[def:function/argument]{parameter} of the trigonometric functions.
\end{definition}
\begin{proposition}\label{thm:trigonometric_function_properties}
  The \hyperref[def:trigonometric_functions]{main trigonometric functions} have the following basic properties:
  \begin{thmenum}
    \thmitem{thm:trigonometric_function_properties/convergence} Both \( \sin(z) \) and \( \cos(z) \) converge in the entire complex plane.
    \thmitem{thm:trigonometric_function_properties/parity} \( \sin(z) \) is an odd function and \( \cos(z) \) is an even function.
    \thmitem{thm:trigonometric_function_properties/derivative} \( \sin'(z) = \cos(z) \) and \( \cos'(z) = -\sin(z) \) for all \( z \in \BbbC \).
  \end{thmenum}
\end{proposition}
\begin{proof}
  \SubProofOf{thm:trigonometric_function_properties/convergence} Note that the zero coefficients in the expansion of either \( \sin \) or \( \cos \) do not alter convergence. Therefore, by \fullref{thm:power_series_radius_of_convergence}, the radius of convergence is
  \begin{equation*}
    \limsup_{k \to \infty} \frac {\abs{i^{k-1} k!}} {\abs{i^k (k-1)!}}
    =
    \limsup_{k \to \infty} k
    =
    +\infty.
  \end{equation*}

  \SubProofOf{thm:trigonometric_function_properties/parity} Follows from \fullref{thm:power_series_parity}.

  \SubProofOf{thm:trigonometric_function_properties/derivative} Follows from \fullref{thm:power_series_are_locally_uniform_convergent} and \fullref{thm:derivative_limit_exchange}.
\end{proof}

\begin{proposition}\label{thm:trigonometric_identities}
  We have the following basic trigonometric identities:
  \begin{thmenum}
    \thmitem{thm:trigonometric_identities/pythagorean_identity} (Pythagorean identity) For any \( z \in \BbbC \),
    \begin{equation}\label{eq:thm:trigonometric_identities/pythagorean_identity}
      \sin(z)^2 + \cos(z)^2 = 1.
    \end{equation}

    \thmitem{thm:trigonometric_identities/products} (Products) For \( x, y \in \BbbC \),
    \begin{balign}
      2 \sin(x) \sin(y) & = \cos(x - y) - \cos(x + y) \label{eq:thm:trigonometric_identities/products/ss}  \\
      2 \cos(x) \cos(y) & = \cos(x - y) + \cos(x + y) \label{eq:thm:trigonometric_identities/products/cc}  \\
      2 \sin(x) \cos(y) & = \sin(x - y) + \sin(x + y) \label{eq:thm:trigonometric_identities/products/sc}  \\
      2 \cos(x) \sin(y) & = -\sin(x - y) + \sin(x + y) \label{eq:thm:trigonometric_identities/products/cs}
    \end{balign}

    \thmitem{thm:trigonometric_identities/sums} (Sums) For \( x, y \in \BbbC \),
    \begin{balign}
      \sin(x) + \sin(y) & = 2 \cos\left(\frac{x - y} 2 \right) \sin\left(\frac{x + y} 2 \right) \label{eq:thm:trigonometric_identities/sums/sin_sum}   \\
      \sin(x) - \sin(y) & = 2 \sin\left(\frac{x - y} 2 \right) \cos\left(\frac{x + y} 2 \right) \label{eq:thm:trigonometric_identities/sums/sin_diff}  \\
      \cos(x) + \cos(y) & = 2 \cos\left(\frac{x - y} 2 \right) \cos\left(\frac{x + y} 2 \right) \label{eq:thm:trigonometric_identities/sums/cos_sum}   \\
      \cos(x) - \cos(y) & = -2 \sin\left(\frac{x - y} 2 \right) \sin\left(\frac{x + y} 2 \right) \label{eq:thm:trigonometric_identities/sums/cos_diff}
    \end{balign}

    \thmitem{thm:trigonometric_identities/sum_of_angles} (Sum of angles) For \( x, y \in \BbbC \),
    \begin{balign}
      \sin(x + y) & = \cos(x) \sin(y) + \cos(x) \sin(y) \label{eq:thm:trigonometric_identities/sum_of_angles/sin} \\
      \cos(x + y) & = \cos(x) \cos(y) - \sin(x) \sin(y) \label{eq:thm:trigonometric_identities/sum_of_angles/cos}
    \end{balign}
  \end{thmenum}
\end{proposition}
\begin{proof}
  We first use \hyperref[def:algebra_of_polynomials/polynomial_multiplication]{Cauchy multiplication} for the power series \( \cos(v) \) and \( \cos(w) \):
  \begin{balign}
    \cos(v) \cos(w)
    &=
    \left( \sum_{k=0}^\infty \frac {i^{2k} v^{2k}} {(2k)!} \right) \Ast \left( \sum_{k=0}^\infty \frac {i^{2k} w^{2k}} {(2k)!} \right)
    = \nonumber \\ &=
    \sum_{k=0}^\infty \sum_{m=0}^k \frac {i^{2m} v^{2m}} {(2m)!} \frac {i^{2(k-m)} w^{2(k-m)}} {(2(k-m))!}
    = \nonumber \\ &=
    \sum_{k=0}^\infty \frac {i^{2k}} {(2k)!} \sum_{m=0}^k \binom {2k} {2m} v^{2m} w^{2(k-m)}. \label{eq:thm:trigonometric_identities/cos_product}
  \end{balign}

  Analogously,
  \begin{balign}
    \sin(v) \sin(w)
    &=
    (-i) (-i) \left( \sum_{k=0}^\infty \frac {i^{2k+1} v^{2k+1}} {(2k+1)!} \right) \Ast \left( \sum_{k=0}^\infty \frac {i^{2k+1} w^{2k+1}} {(2k+1)!} \right)
    = \nonumber \\ &=
    -\sum_{k=0}^\infty \sum_{m=0}^k \frac {i^{2m+1} v^{2m+1}} {(2m+1)!} \frac {i^{2(k-m)+1} w^{2(k-m)+1}} {(2(k-m)+1)!}
    = \nonumber \\ &=
    -\sum_{k=0}^\infty \frac {i^{2(k+1)}} {(2(k+1))!} \sum_{m=0}^k \binom {2(k+1)} {2m+1} v^{2m+1} w^{2(k-m)+1}
    = \nonumber \\ &=
    -\sum_{k=1}^\infty \frac {i^{2k}} {(2k)!} \sum_{m=0}^{k-1} \binom {2k} {2m+1} v^{2m+1} w^{2k-(2m+1)}. \label{eq:thm:trigonometric_identities/sin_product}
  \end{balign}

  \SubProofOf{thm:trigonometric_identities/pythagorean_identity} From \eqref{eq:thm:trigonometric_identities/cos_product} and \eqref{eq:thm:trigonometric_identities/sin_product} we have
  \begin{equation*}
    \sin(z)^2 + \cos(z)^2
    =
    1 + \sum_{k=1}^\infty \frac {i^{2k} z^{2k}} {(2k)!} \underbrace{\left[-\sum_{m=0}^{k-1} \binom {2k} {2m+1} + \sum_{m=0}^k \binom {2k} {2m} \right]}_{\eqqcolon a_k}.
  \end{equation*}

  It remains to show that the expression \( a_k \) equals zero for all \( k = 1, 2, \ldots \). We have
  \begin{equation*}
    a_k
    =
    \sum_{m=0}^k \binom {2k} {2m} - \sum_{m=0}^{k-1} \binom {2k} {2m+1}
    =
    \sum_{m=0}^k (-1)^m \binom {2k} m
    \overset {\ref{thm:binomial_theorem}} =
    1 - 1 = 0.
  \end{equation*}

  \Fullref{eq:thm:trigonometric_identities/pythagorean_identity} follows.

  \SubProofOf{thm:trigonometric_identities/products} We will only prove \eqref{eq:thm:trigonometric_identities/products/cc} because the other identities are proved analogously. We have
  \begin{balign*}
    \cos(v - w) + \cos(v + w)
    &=
    \sum_{k=0}^\infty \frac {i^{2k}} {(2k)!} \left[(v - w)^{2k} + (v + w)^{2k} \right]
    \overset {\ref{thm:binomial_theorem}} = \\ &=
    \sum_{k=0}^\infty \frac {i^{2k}} {(2k)!} \sum_{m=0}^{2k} \binom {2k} m v^{2k-m} w^m \left[ (-1)^m + 1 \right]
    = \\ &=
    2 \sum_{k=0}^\infty \frac {i^{2k}} {(2k)!} \sum_{m=0}^{2k} \binom {2k} {2m} v^{2(k-m)} w^{2m}
    \overset {\eqref{eq:thm:trigonometric_identities/cos_product}} = \\ &=
    2 \cos(v) \cos(w).
  \end{balign*}

  \SubProofOf{thm:trigonometric_identities/sums} Fix some \( v, w \in \BbbC \) and define
  \begin{balign*}
    x \coloneqq \frac {v + w} 2
    &&
    y \coloneqq \frac {v - w} 2
  \end{balign*}
  so that \( v = x + y \) and \( w = x - y \).

  The identity \eqref{eq:thm:trigonometric_identities/sums/sin_sum} the follows from \eqref{eq:thm:trigonometric_identities/products/sc} applied to \( x \) and \( y \). The other identities are proved analogously.

  \SubProofOf{thm:trigonometric_identities/sum_of_angles} We will only prove \eqref{eq:thm:trigonometric_identities/sum_of_angles/sin} because \eqref{eq:thm:trigonometric_identities/sum_of_angles/cos} is proved analogously. From \eqref{eq:thm:trigonometric_identities/products/cs},
  \begin{balign*}
    \sin(x + y)
     & =
    2 \cos(x) \sin(y) + \sin(x - y)
    \overset {\eqref{eq:thm:trigonometric_identities/products/sc}} = \\ &=
    2 \cos(x) \sin(y) + 2 \cos(x) \sin(y) - \sin(x + y).
  \end{balign*}

  After dividing by \( 2 \), we obtain \eqref{eq:thm:trigonometric_identities/sum_of_angles/sin}.
\end{proof}

\begin{lemma}\label{thm:trigonometric_function_basic_roots}
  We have the following important special values:
  \begin{balign}
    \sin(0) = 0,   &  & \cos(0) = 1,    \label{eq:thm:trigonometric_function_basic_roots/zero} \\
    \sin(\pi) = 0, &  & \cos(\pi) = -1. \label{eq:thm:trigonometric_function_basic_roots/pi}
  \end{balign}
\end{lemma}
\begin{proof}
\eqref{eq:thm:trigonometric_function_basic_roots/zero} follows directly from \fullref{def:trigonometric_functions}.
  Now consider the \hyperref[def:function/extension]{restriction} of \( \cos \) to the real line. Since \( \cos(0) \neq 0 \) and \( \cos \) is continuously differentiable as a power series, in some neighborhood \( U \) of \( 0 \) we have \( 0 \not\in \cos(U) \). Therefore the inverse function theorem holds and there exists a neighborhood \( V \subseteq U \) of \( 1 \) such that the continuously differentiable function \( f: V \to \BbbR \) is the inverse of \( \cos \) in \( V \) (we have not yet defined \hyperref[def:inverse_trigonometric_functions/arccos]{\( \arccos \)}). If \( y = \cos(x) \), then
  \begin{equation*}
    Df(y)
    =
    \frac 1 {D\cos(x)}
    =
    \frac 1 {-\sin(x)}
    \overset {\ref{thm:trigonometric_identities/pythagorean_identity}} =
    -\frac 1 {\sqrt{1 - y^2}},
    \quad y \in \cos(V).
  \end{equation*}

  The derivative is actually well-defined and continuous anywhere except for \( y \in \{ -1, 1 \} \). Therefore, for any \( \alpha \in (-1, 1) \),
  \begin{equation*}
    f(y) = f(\alpha) - \int_{\alpha}^y \frac 1 {\sqrt{1 - t^2}} dt, \quad y \in [\alpha, 1).
  \end{equation*}

  We already know that \( \cos(0) = 1 \), hence \( f(1) = 0 \) and, since \( f(y) \) is given by a convergent integral in \( [\alpha, 1) \), we can extend this interval to \( [\alpha, 1] \).

  By taking \( y = \alpha \), we obtain
  \begin{equation*}
    f(y) - f(-y) = -\int_{-y}^y \frac 1 {\sqrt{1 - t^2}} dt, \quad y \in [-1, 1].
  \end{equation*}

  Note that by \hyperref[def:pi]{our definition} of \( \pi \),
  \begin{equation*}
    \pi
    =
    \int_{-1}^1 \frac 1 {\sqrt{1 - t^2}} dt
    =
    -[\underbrace{f(1)}_{=0} - f(-1)]
    =
    f(-1).
  \end{equation*}

  Hence \( \cos(\pi) = -1 \). From \fullref{thm:trigonometric_identities/pythagorean_identity},
  \begin{equation*}
    \abs{\sin(\pi)} = \sqrt{1 - \cos(\pi)^2} = 0,
  \end{equation*}
  proving that \( \sin(\pi) = 0 \).

  This concludes the proof of \eqref{eq:thm:trigonometric_function_basic_roots/pi}.
\end{proof}

\begin{definition}\label{def:periodic_function}
  A function \( f: \mscrG \to \mscrH \) between \hyperref[def:abelian_group] abelian groups is called \term{periodic} with \term{period} \( p \in \mscrG \) if, for all \( x \in \mscrG \), we have \( f(x) = f(x + r) \).

  The \term{base period} of a function is the \hyperref[def:preordered_set/maximum_and_minimum]{least} of all periods, if a minimum exists. When referring to \enquote{the period}, we mean the base period.

  We can define periods for arbitrary magmas rather than abelian groups but the definition would make it difficult to talk about the base period.
\end{definition}

\begin{theorem}\label{thm:trigonometric_function_period}
  Both \( \sin(z) \) and \( \cos(z) \) are \( 2\pi \)-periodic.
\end{theorem}
\begin{proof}
  We will temporarily restrict ourselves to the real line. Since \( \cos(x) \) is continuous, \( \cos^{-1}(\{ 0 \}) \) is a closed set by \fullref{thm:weierstrass_extreme_value_theorem} there exists a minimum \( \gamma \) of \( [0, \pi] \cap \cos^{-1}(\{ 0 \}) \).

  Since, by \fullref{thm:trigonometric_function_basic_roots}, \( \cos(0) = 1 \), it follows that \( \cos(x) > 0, x \in (-\gamma, \gamma) \). Therefore its primitive function \( \sin(x) \) increases on the same interval. It is also continuous, hence by \fullref{thm:trigonometric_identities/pythagorean_identity}, \( \sin(\gamma) = 1 \) because \( \cos(\gamma) = 0 \).

  Because \( \sin \) is an odd function, \( \sin(-\gamma) = \sin(\gamma) = -1 \).

  From \hyperref[def:pi]{our definition} of \( \pi \) it follows that
  \begin{equation*}
    \pi
    =
    \int_{-1}^1 \frac 1 {1 - t^2} dt
    =
    \int_{-\gamma}^\gamma \frac {\cos(\varphi)} {1 - \sin(\varphi)^2} d\varphi
    \overset {\ref{thm:trigonometric_identities/pythagorean_identity}} =
    \int_{-\gamma}^\gamma d\varphi
    =
    2\gamma.
  \end{equation*}

  In order for a number \( p \) to be a period of \( \sin \), we need to have \( \sin(p) = \sin(0) = 0 \). But we showed that \( \sin(x) \) is increasing from \( 0 \) to \( \tfrac \pi 2 \) and cannot possibly contain zeros in that interval. Hence \( p > \tfrac \pi 2 \).

  We also have \( \cos(\tfrac \pi 2) = 0 \). By \fullref{thm:trigonometric_identities/sums},
  \begin{equation*}
    \sin(\tfrac \pi 2 + x)
    =
    \sin(\tfrac \pi 2) \cos(x) + \cos(\tfrac \pi 2) \sin(x)
    =
    \cos(x).
  \end{equation*}

  Since \( \cos \) is positive on \( [0, \tfrac \pi 2) \), \( \sin \) is positive on \( [\tfrac \pi 2, \pi) \).

  We already showed in \fullref{thm:trigonometric_function_basic_roots} that \( \sin(\pi) = 0 \).

  It follows that the minimal period of \( \sin \) is either \( \pi \) or a multiple of \( \pi \). It cannot be \( \pi \) since \( \cos(\pi) \neq \cos(0) \), therefore it must be \( 2\pi \).
\end{proof}

\begin{definition}\label{def:hyperbolic_trigonometric_functions}
  In analogy with \fullref{thm:exponential_trigonometric_identities/inverse_eulers_formula}, we define \term{hyperbolic trigonometric functions}.

  \begin{thmenum}
    \thmitem{def:hyperbolic_trigonometric_functions/sine} The \term{hyperbolic sine} function:
    \begin{equation*}
      \sinh(x) \coloneqq - \frac {e^x - e^{-x}} 2 \\
    \end{equation*}

    \thmitem{def:hyperbolic_trigonometric_functions/cosine} The \term{hyperbolic cosine} function:
    \begin{equation*}
      \cosh(x) \coloneqq \frac {e^x + e^{-x}} 2
    \end{equation*}
  \end{thmenum}

  Compare \fullref{def:quadratic_plane_curve/ellipse/parametric_equations} and \fullref{def:quadratic_plane_curve/hyperbola/parametric_equations} for a justification of the naming.
\end{definition}

\begin{definition}\label{def:derived_trigonometric_functions}
  In addition to \( \sin(z) \) and \( \cos(z) \), we define two additional functions, also called \enquote{trigonometric}.

  \begin{thmenum}
    \thmitem{def:derived_trigonometric_functions/tan} The \hyperref[def:function/partial]{partial} \term{tangent} function, also called \term{tangens}, is
    \begin{equation*}
      \tan(z) \coloneqq \frac {\sin(z)} {\cos(z)}.
    \end{equation*}

    It is defined in \( \BbbC \setminus (\tfrac \pi 2 + \pi\BbbZ) \).

    \thmitem{def:derived_trigonometric_functions/cot} The \hyperref[def:function/partial]{partial} \term{cotangent function}, also called \term{cotangens}, is
    \begin{equation*}
      \cot(z) \coloneqq \frac {\cos(z)} {\sin(z)}.
    \end{equation*}

    It is defined in \( \BbbC \setminus \pi\BbbZ \).
  \end{thmenum}
\end{definition}

\begin{definition}\label{def:inverse_trigonometric_functions}
  We can define \term{inverse trigonometric functions}. We will thus restrict ourselves only to real numbers. Fix an integer \( k \). Unless noted otherwise, we assume \( k = 0 \).

  \begin{thmenum}
    \thmitem{def:inverse_trigonometric_functions/arcsin} The \term{arcus sinus} function \( \arcsin(x) \) is defined as the \hyperref[def:function/inverse]{inverse function} of \( \sin(x) \) (see \fullref{def:trigonometric_functions/sine}) from \( [-1, 1] \) to \( \left[(k - \tfrac 1 2) \pi, (k + \tfrac 1 2) \pi \right) \).

    \thmitem{def:inverse_trigonometric_functions/arccos} The \term{arcus cosinus} function \( \arccos(x) \) is defined as the inverse of \( \cos(x) \) (see \fullref{def:trigonometric_functions/cosine}) from \( [-1, 1] \) to \( (k\pi, (k + 1)\pi) \).

    \thmitem{def:inverse_trigonometric_functions/arctan} The \term{arcus tangens} function \( \arctan(x) \) is defined as the inverse of \( \tan(x) \) (see \fullref{def:derived_trigonometric_functions/tan}) from \( \BbbR \) to \( \left((k - \tfrac 1 2) \pi, (k + \tfrac 1 2) \pi \right) \).

    \thmitem{def:inverse_trigonometric_functions/arccot} The \term{arcus cotangens} function \( \arccot(x) \) is defined as the inverse of \( \cot(x) \) (see \fullref{def:derived_trigonometric_functions/cot}) from \( \BbbR \) to \( (k\pi, (k + 1)\pi) \).

    \thmitem{def:inverse_trigonometric_functions/arctantwo} The \term{two-argument arcus tangens} function \( \arctantwo(y, x) \) is a bit special, however it is very useful in practice - see \fullref{thm:arctantwo}. It is defined as
    \begin{align*}
       &\arctantwo: \BbbR^2 \setminus \{ 0 \} \to [2k\pi, 2k\pi + 2)  \\
       &\arctantwo(y, x) \coloneqq \begin{dcases}
        \arctan \parens*{ \tfrac y x },       &x > 0                  \\
        \arctan \parens*{ \tfrac y x } + \pi, &x < 0 \T{and} y \geq 0 \\
        \arctan \parens*{ \tfrac y x } - \pi, &x < 0 \T{and} y < 0    \\
        \pi,                                 &x = 0 \T{and} y \geq 0 \\
        -\pi,                                &x = 0 \T{and} y < 0.
      \end{dcases}
    \end{align*}
  \end{thmenum}
\end{definition}

\begin{proposition}\label{thm:arctantwo}
  Fix an integer \( k \). Given \( (x_0, y_0) \in S_{\BbbR^2} \), \( t_0 \coloneqq \arctantwo(y_0, x_0) \) is the unique solution to the equation
  \begin{equation}\label{thm:arctantwo/equation}
    \begin{cases}
      x_0 = \cos(t) \\
      y_0 = \sin(t)
    \end{cases}
  \end{equation}
  in \( t \in [2k\pi, 2k\pi + 2) \).
\end{proposition}

\subsection{Exponential function}\label{subsec:exponential_function}

\begin{definition}\label{def:exponential_function}
  We define the \term{exponential function}
  \begin{equation}\label{def:exponential_function/series}
    \exp(z) \coloneqq \sum_{k=0}^\infty \frac {z^k} {k!}
  \end{equation}
  and \term{Euler's number}
  \begin{equation*}
    e \coloneqq \exp(1) = \sum_{i=0}^k \frac 1 {k!}.
  \end{equation*}

  \Fullref{thm:exponential_function_properties/interpolates_power} justifies the notation \( e^z = \exp(z) \).
\end{definition}
\begin{proof}
  We will show that \( \exp(z) \) converges everywhere. By \fullref{thm:power_series_radius_of_convergence}, the radius of convergence is
  \begin{equation*}
    \limsup_{k \to \infty} \frac {k!} {(k-1)!}
    =
    \limsup_{k \to \infty} k
    =
    +\infty
  \end{equation*}

  Hence the radius of convergence of \( \exp(x) \) is infinite.
\end{proof}

\begin{proposition}\label{thm:exponential_function_properties}
  The exponential function \( \exp(z) \) has the following basic properties (not that we do not use the notation \( e^z \) here in order to reduce confusion with yet-undefined power \hyperref[def:power_function]{functions}):

  \begin{thmenum}
    \thmitem{thm:exponential_function_properties/eulers_identity} (Euler's identity)
    \begin{equation*}
      \exp(i \pi) = -1.
    \end{equation*}

    \thmitem{thm:exponential_function_properties/derivative} \( \exp(z) \) is its own derivative.

    \thmitem{thm:exponential_function_properties/homomorphism} \( \exp(x + y) = \exp(x) \exp(y) \). Stated in another way, \( \exp \) is a homomorphism from the additive group of \( \BbbC \) to the multiplicative group.

    \thmitem{thm:exponential_function_properties/interpolates_power} The notation \( \exp(x) \) is consistent with iterated multiplication as defined in \fullref{def:semiring/dioid}, that is, \( \exp(n) = \underbrace{e \cdot \ldots \cdot e}_{n \text{times}} \) and for positive integers \( n \), \( \exp(n) =  \) and \( \exp(-n) =\tfrac 1 {\exp(n)} \).

    \thmitem{thm:exponential_function_properties/negative_power}
    \begin{equation*}
      \exp(z) = \frac 1 {\exp(-z)}.
    \end{equation*}

    \thmitem{thm:exponential_function_properties/real_positive} For real \( t \), \( e^t \) is a positive real number.

    \thmitem{thm:exponential_function_properties/conjugate} \( \overline{\exp(z)} = \exp(\overline{z}) \).

    \thmitem{thm:exponential_function_properties/unit_circle} For any \( c \in \BbbR \), the function \( t \mapsto \exp(it) \) is a bijection between any half-open interval \( [c, c + 2\pi) \) and the unit circle in \( \BbbC \).

    \thmitem{thm:exponential_function_properties/real_bijective} \( t \mapsto \exp(t) \) is a bijection from \( \BbbR \) to \( [0, \infty) \).

    \thmitem{thm:exponential_function_properties/bijective} For any \( c \in \BbbR \), \( \exp(z) \) is a bijection between the strip \( S \coloneqq \{ a + bi \colon c \leq b < c + 2\pi \} \) and the complex plane \( \BbbC \setminus \{ 0 \} \).

    \thmitem{thm:exponential_function_properties/periodic} \( \exp(z) \) is \( 2i\pi \)-\hyperref[def:periodic_function]{periodic}.

    \thmitem{thm:exponential_function_properties/compound_interest} For nonnegative real \( t \geq 0 \) we have
    \begin{equation*}
      \exp(t) = \lim_{n \to \infty} \left(1 + \frac t n \right)^n
    \end{equation*}
  \end{thmenum}
\end{proposition}
\begin{proof}
  \SubProofOf{thm:exponential_function_properties/eulers_identity} By \fullref{eq:thm:trigonometric_function_basic_roots/pi} and \fullref{thm:exponential_trigonometric_identities/eulers_formula}, we have
  \begin{equation*}
    \exp(i\pi) = \cos(\pi) + i\sin(\pi) = -1.
  \end{equation*}

  \SubProofOf{thm:exponential_function_properties/derivative} Follows from \fullref{thm:power_series_are_locally_uniform_convergent} and \fullref{thm:derivative_limit_exchange}.

  \SubProofOf{thm:exponential_function_properties/homomorphism} The Cauchy product of \( \exp(x) \) and \( \exp(y) \) is
  \begin{balign*}
    \exp(x) \exp(y)
     & =
    \left( \sum_{k=0}^\infty \frac {x^k} {k!} \right) \left( \sum_{k=0}^\infty \frac {y^k} {k!} \right)
    =                                       \\ &=
    \sum_{k=0}^\infty \sum_{m=0}^k \frac {x^m} {m!} \frac {x^{k-m}} {(k-m)!}
    =                                       \\ &=
    \sum_{k=0}^\infty \frac 1 {k!} \sum_{m=0}^k \binom{k}{m} x^m y^{k-m}
    \reloset {\ref{thm:binomial_theorem}} = \\ &=
    \sum_{k=0}^\infty \frac {(x + y)^k} {k!}
    =
    \exp(x + y).
  \end{balign*}

  \SubProofOf{thm:exponential_function_properties/interpolates_power} We use induction on \( n \) to prove \( \exp(n) = e^n \). The case \( \exp(0) = 1 \) is obvious. If we assume that \( \exp(n) = e^n \), by \fullref{thm:exponential_function_properties/homomorphism}, we have
  \begin{equation*}
    \exp(n + 1)
    =
    \exp(n) \exp(1)
    =
    e^n \cdot e
    =
    e^{n+1}.
  \end{equation*}

  Note that this works for negative \( n \) too.

  \SubProofOf{thm:exponential_function_properties/negative_power} Note that
  \begin{equation*}
    1 = \exp(0) = \exp(z - z) = \exp(z) \exp(-z),
  \end{equation*}
  hence
  \begin{equation*}
    \exp(-z) = \frac 1 {\exp(z)}.
  \end{equation*}

  \SubProofOf{thm:exponential_function_properties/real_positive} For \( t > 0 \), the following
  \begin{equation*}
    \exp(t) = \sum_{k=0}^\infty \frac {t^k} {k!}
  \end{equation*}
  is a series of positive real numbers. To see its convergence, we apply \fullref{thm:dalamberts_ratio_test}:
  \begin{equation*}
    \frac {t^k} {k!} \cdot \frac {(k-1)!} {t^{k-1}}
    =
    \frac t k
    \xrightarrow[k \to \infty]{} 0.
  \end{equation*}

  Thus \( \exp(t) \) is a nonnegative real number. Furthermore, since the sequence of partial sums is monotone, \( \exp(t) \) cannot be zero. Hence for \( t > 0 \), we have \( \exp(t) > 0 \).

  Notice that \( \exp(t) \exp(-t) > 0 \), hence if \( \exp(t) > 0 \), then \( \exp(-t) > 0 \).

  \SubProofOf{thm:exponential_function_properties/conjugate} By \fullref{thm:exponential_trigonometric_identities/eulers_formula},
  \begin{balign*}
    \overline{\exp(a + bi)}
     & \reloset {\ref{thm:exponential_function_properties/homomorphism}} =
    \overline{\exp(a) \exp(bi)}
    \reloset {\ref{thm:exponential_function_properties/real_positive}} =   \\ &=
    \exp(a) \overline{(\cos(b) + i\sin(b))}
    =                                                                      \\ &=
    \exp(a) (\cos(b) - i\sin(b))
    \reloset {\ref{thm:power_series_parity}} =                             \\ &=
    \exp(a) (\cos(-b) + i\sin(-b))
    =                                                                      \\ &=
    \exp(a) \exp(-bi)
    =                                                                      \\ &=
    \exp(a - bi)
    =                                                                      \\ &=
    \exp(\overline{a + bi}).
  \end{balign*}

  \SubProofOf{thm:exponential_function_properties/periodic} By \fullref{thm:exponential_function_properties/eulers_identity},
  \begin{equation*}
    \exp(x + 2i\pi) = \exp(x) \exp(2i\pi) = \exp(x).
  \end{equation*}

  Furthermore, this is also the minimal period. If we assume that \( \sin(x) \) has another period, say \( p \in (0, 2\pi) \), we would have \( \sin(p) = \sin(0) = 0 \) and \fullref{thm:trigonometric_identities/pythagorean_identity} would imply that \( \cos(p) \in \{ -1, 1 \} \). But then \( \cos(p) \) would be an extreme point for \( \cos \), which is not possible because \( \cos \) is convex in \( [0, 2\pi] \) and only has three extremal points --- \( 0, \pi, 2\pi \).

  \SubProofOf{thm:exponential_function_properties/unit_circle} For \( c, t \in \BbbR \) we have
  \begin{equation*}
    \abs{\exp(it)}
    =
    \abs{\cos(t) + i\sin(t)}
    =
    \sqrt{\cos(t)^2 + \sin(t)^2}
    \reloset {\eqref{eq:thm:trigonometric_identities/pythagorean_identity}} =
    1.
  \end{equation*}

  Furthermore, if \( r \) is another real number,
  \begin{equation}
    \exp(ir)
    =
    \exp(i(t + (r - t)))
    =
    \exp(it) \exp(i(r - t)).
  \end{equation}

  It follows that \( \exp(ir) \neq \exp(it) \) if and only if \( \exp(i(r - t)) \neq 0 \). If \( t, r \in [c, c + 2\pi) \) and \( t \neq r \), this is satisfied.

  Hence \( t \mapsto \exp(it) \) is indeed an injection of \( [c, c + 2\pi) \) into the unit circle of \( \BbbC \). It is also a surjection because of the intermediate value theorem.

  \SubProofOf{thm:exponential_function_properties/real_bijective} First, assume that \( e^t \) is not injective on \( \BbbR \). Then there exist \( t, r \in \BbbR \), \( t \neq r \), such that \( e^t = e^r \). By \fullref{thm:exponential_function_properties/real_positive}, both are positive real numbers. In particular, we can divide by \( e^t \) to obtain
  \begin{equation*}
    1
    =
    \frac {e^r} {e^t}
    \reloset {\ref{thm:exponential_function_properties/negative_power}} =
    =
    e^r e^{-t}
    \reloset {\ref{thm:exponential_function_properties/homomorphism}} =
    e^{r - t}.
  \end{equation*}

  We know that \( e^0 = 1 \) from \fullref{thm:exponential_function_properties/interpolates_power}. Thus it is enough to show that \( e^t = 1 \) if and only if \( t = 0 \).

  Assume that \( e^t = 1 \) holds for some \( t > 0 \). The partial sums are monotonely increasing so in order for them to converge to \( 1 \), for any fixed index \( n \) we must have
  \begin{balign*}
    0  & \leq \sum_{k=0}^n \frac {t^k} {k!} = 1 + \sum_{k=1}^n \frac {t^k} {k!} \leq 1, \\
    -1 & \leq \sum_{k=1}^n \frac {t^k} {k!} \leq 0.
  \end{balign*}

  But \( \sum_{k=1}^n \frac {t^k} {k!} > 0 \) because \( t > 0 \). The obtained contradiction proves that \( e^t \neq 1 \) for positive \( t \).

  For negative \( t \), note that
  \begin{equation*}
    e^t e^{-t} = 1.
  \end{equation*}

  Since \( -t \) is positive, \( e^{-t} \neq 1 \) and hence \( e^t \neq 1 \).

  Therefore the function \( t \mapsto e^t \) is injective on \( \BbbR \). It is also surjective onto \( \BbbR^{>0} \) because of the intermediate value theorem.

  \SubProofOf{thm:exponential_function_properties/bijective} Fix \( a + bi \in S_c \), that is, \( b \in [c, c + 2\pi) \). By \fullref{thm:exponential_function_properties/homomorphism},
  \begin{equation*}
    e^{a + bi} = e^a e^{bi}.
  \end{equation*}

  By \fullref{thm:exponential_function_properties/unit_circle}, \( b \mapsto e^{bi} \) is injective for \( b \in [c, c + 2\pi) \) and by \fullref{thm:exponential_function_properties/real_bijective}, \( a \mapsto e^a \) is injective on \( \BbbR \). It follows that their product is also injective.

  \SubProofOf{thm:exponential_function_properties/compound_interest}\mcite[3.31]{Rudin1976Principles}By \fullref{thm:binomial_theorem},
  \begin{balign*}
    \left(1 + \frac t n \right)^n
     & =
    \sum_{k=0}^n \binom{n}{k} \left(\frac t n\right)^k 1^{n-k}
    =    \\ &=
    \sum_{k=0}^n \frac {n!} {(n-k)! k!} \frac {t^k} {n^k}
    =    \\ &=
    \sum_{k=0}^n \frac {n!} {(n-k)! n^k} \frac {t^k} {k!}
    =    \\ &=
    \sum_{k=0}^n \left[ \prod_{j=1}^k \left(1 - \frac {k+j} n \right) \right] \frac {t^k} {k!}.
  \end{balign*}

  Fix an index \( m \). Since the series is nonnegative, there exists an index \( N \) such that for \( n \geq N \)
  \begin{equation*}
    \sum_{k=0}^m \frac {t^k} {k!}
    \leq
    \sum_{k=0}^n \left[ \prod_{j=1}^k \left(1 - \frac {k+j} n \right) \right] \frac {t^k} {k!}.
  \end{equation*}

  Note that
  \begin{equation*}
    \left[ \prod_{j=1}^k \left(1 - \frac {k+j} n \right) \right] \frac {t^k} {k!}
    \leq
    \frac {t^k} {k!},
  \end{equation*}
  hence
  \begin{equation*}
    \sum_{k=0}^m \frac {t^k} {k!}
    \leq
    \sum_{k=0}^n \left[ \prod_{j=1}^k \left(1 - \frac {k+j} n \right) \right] \frac {t^k} {k!}
    \leq
    \sum_{k=0}^n \frac {t^k} {k!}.
  \end{equation*}

  By \fullref{thm:squeeze_lemma},
  \begin{equation*}
    \lim_{n \to \infty} \left(1 + \frac t n \right)^n
    =
    \lim_{n \to \infty} \sum_{k=0}^n \frac {t^k} {k!}
    =
    \exp(t).
  \end{equation*}
\end{proof}

\begin{definition}\label{def:logarithm}
  Fix \( c \in \BbbR \). Unless specified otherwise, we assume \( c = 0 \).

  We define the \term{natural logarithm} \( \log(x) \) as \hyperref[def:multi_valued_function/inverse]{inverse function} of \( e^x \) from \( \BbbC \setminus \{ 0 \} \) to the strip \( S_c \coloneqq \{ a + bi \colon c \leq b < c + 2\pi \} \).

  We also define the \term{base \( b \) logarithm} \( \log_b(x) \) for \( b > 0 \) over the same domain as
  \begin{equation*}
    \log_b(x) \coloneqq \frac {\log(x)} {\log(b)}.
  \end{equation*}
\end{definition}
\begin{proof}
  The well-definedness follows from \fullref{thm:exponential_function_properties/bijective}.
\end{proof}

\begin{proposition}\label{thm:logarithm_properties}
  \hfill
  \begin{thmenum}
    \thmitem{thm:logarithm_properties/homomorphism} \( \log(xy) = \log(x) \log(y) \)
  \end{thmenum}
\end{proposition}
\begin{proof}
  \SubProofOf{thm:logarithm_properties/homomorphism} Follows from \fullref{thm:exponential_function_properties/homomorphism}.
\end{proof}

\begin{definition}\label{def:power_function}
  For each positive real number \( y > 0 \), we define the \term{power function}
  \begin{equation*}
    x^y \coloneqq e^{y \ln x}
  \end{equation*}
  as a function of \( x \).
\end{definition}

\begin{proposition}\label{thm:power_function_properties}
  \hfill
  \begin{thmenum}
    \thmitem{thm:power_function_properties/composition} \( (x^y)^z = x^{yz} \).
    \thmitem{thm:power_function_properties/derivative} \( D_x(x^y) = \log(x) x^y \).
  \end{thmenum}
\end{proposition}
\begin{proof}
  \SubProofOf{thm:power_function_properties/composition}
  \begin{equation*}
    (x^y)^z
    =
    e^{z \log(e^{y \log(x)})}
    =
    e^{z y \log(x)}
    =
    x^{yz}.
  \end{equation*}

  \SubProofOf{thm:power_function_properties/derivative} Using the chain rule for differentiation, we obtain
  \begin{equation*}
    D_x(x^y) = D_x(e^{\log(y) x}) = \log(x) e^{\log(y) x} = \log(x) x^y.
  \end{equation*}
\end{proof}
\begin{proposition}\label{thm:exponential-trigonometric_identities}
  We have the following exponential-trigonometric identities:
  \thmitem{thm:exponential_trigonometric_identities/eulers_formula} (Euler's formula) For any \( z \in \BbbC \),
  \begin{equation}\label{thm:exponential_trigonometric_identities/eulers_formula/identity}
    e^{iz} = \cos(z) + i \sin(z).
  \end{equation}

  \thmitem{thm:exponential_trigonometric_identities/inverse_eulers_formula} (Inverse Euler's identities) For any \( z \in \BbbC \),
  \begin{balign}
    \sin(z) & = \real(e^z) = \frac {e^{iz} - e^{-iz}} {2i} \label{thm:exponential_trigonometric_identities/inverse_eulers_formula/sin} \\
    \cos(z) & = \imag(e^z) = \frac {e^{iz} + e^{-iz}} 2 \label{thm:exponential_trigonometric_identities/inverse_eulers_formula/cos}
  \end{balign}

  \thmitem{thm:exponential_trigonometric_identities/de_moivre} (De Moivre's formula) For any complex number \( z \) and any nonnegative integer \( n \),
  \begin{equation}\label{thm:exponential_trigonometric_identities/de_moivre/identity}
    (\cos(z) + i \sin(z))^n = \cos(nz) + i \sin(nz).
  \end{equation}
\end{proposition}
\begin{proof}
  \SubProofOf{thm:exponential_trigonometric_identities/eulers_formula} Simply note that \fullref{def:exponential_function} is a termwise sum of \fullref{def:trigonometric_functions/sine} and \fullref{def:trigonometric_functions/cosine}, therefore \fullref{thm:exponential_trigonometric_identities/eulers_formula/identity} holds.

  \SubProofOf{thm:exponential_trigonometric_identities/inverse_eulers_formula} Follows from \fullref{thm:exponential_trigonometric_identities/eulers_formula}.

  \SubProofOf{thm:exponential_trigonometric_identities/de_moivre} From \fullref{thm:exponential_trigonometric_identities/eulers_formula},
  \begin{equation*}
    (\cos(z) + i \sin(z))^n
    =
    {e^{iz}}^n
    \reloset {\ref{thm:power_function_properties/composition}} {=}
    =
    e^{i(zn)}
    =
    \cos(nz) + i \sin(nz).
  \end{equation*}
\end{proof}

\subsection{Trigonometric polynomials}\label{subsec:trigonometric_polynomials}

\begin{definition}\label{def:trigonometric_polynomial}
  We define the \Def{trigonometric polynomials} over \( \BbbC \) as the \hyperref[def:laurent_polynomial/polynomial]{Laurent polynomials} \( \BbbC[e^{iz}] \). A trigonometric polynomial \( p \in \BbbC[e^{iz}] \) can be written as
  \begin{equation}\label{def:trigonometric_polynomial/exponential}
    p(z) = \sum_{k \in \BbbZ} c_k e^{ikz}
  \end{equation}
  or, using \hyperref[thm:exponential_trigonometric_identities/eulers_formula]{Euler's formula}, rewritten in the more conventional notation (see \cite[1]{Боянов2008} or \cite[88]{Rudin1987}):
  \begin{equation}\label{def:trigonometric_polynomial/trigonometric}
    p(z) = a_0 + \sum_{k=1}^\infty [ a_k \cos(kz) + b_k \sin(kz) ],
  \end{equation}
  where we denote \( a_k \coloneqq c_k \) and \( b_k \coloneqq ic_k \).

  In particular, when using \fullref{def:trigonometric_polynomial/trigonometric}, we may regard the coefficients \( \{ a_k \}_{k=0}^\infty \) and \( \{ b_k \}_{k=1}^\infty \) as either real or complex, which is a downside of \fullref{def:trigonometric_polynomial/exponential}.

  Denote by \( \tau_n(\BbbK) \) the vector space of all trigonometric polynomials of degree at most \( n \) with coefficients in \( \BbbK \). We also introduce the subspaces \( \tau_n^\alpha{\BbbK} \) of those polynomials which \( a_0 = 0 \).
\end{definition}

\subsection{Norms}\label{subsec:norms}

\begin{remark}\label{rem:normed_fields}
  Norms generalize distances of points in a plane, while absolute values generalize the absolute value over either \( \BbbR \) or \( \BbbC \). The axioms themselves differ minimally. Absolute values in a field are multiplicative norms over the field, however we cannot define absolute values in terms of norms since absolute values are needed for defining norms. Still, we will refer to fields with absolute values as \term{normed fields}.
\end{remark}

\begin{definition}\label{def:absolute_value}\mcite{nLab:absolute_value}
  Let \( R \) be a \hyperref[def:semiring]{semiring}. We say that the function \( \abs{\cdot}: V \to \BbbR_{>0} \) is an \term{absolute value} or a \term{semiring norm} if
  \begin{thmenum}
    \thmitem[def:absolute_value/RN1]{RN1}(identity) \( x = 0_R \) if and only if \( \abs{x} = 0 \)
    \thmitem[def:absolute_value/RN2]{RN2}(multiplicativity) For any \( x, y \in V \),
    \begin{equation*}
      \abs{xy} = \abs{x} \cdot \abs{y}
    \end{equation*}

    \thmitem[def:absolute_value/RN3]{RN3}(subadditivity) For any \( x, y \in V \),
    \begin{equation*}
      \abs{x + y} \leq \abs{x} + \abs{y}
    \end{equation*}
  \end{thmenum}
\end{definition}

\begin{definition}\label{def:norm}
  Let \( M \) be an \( R \)-module with absolute value \( \abs{\cdot} \). We say that the function \( \norm{\cdot}: M \to \BbbR_{\geq 0} \) is a \term{norm} if
  \begin{thmenum}
    \thmitem[def:norm/N1]{N1}(identity) \( x = 0_M \) if and only if \( \norm x = 0_{\BbbR} \)

    \thmitem[def:norm/N2]{N2}(absolute homogeneity)
    \begin{equation*}
      \norm{t x} = \abs{t} \norm{x} \text{ for all } t \in R \text{ and } x \in M
    \end{equation*}

    \thmitem[def:norm/N3]{N3}(subadditivity)
    \begin{equation*}
      \norm{x + y} \leq \norm{x} + \norm{y} \text{ for all } x, y \in M
    \end{equation*}
  \end{thmenum}

  If we remove \fullref{def:norm/N1}, then \( \norm{\cdot} \) is called a \term{seminorm}.

  If instead \( V \) is an \hyperref[def:algebra_over_semiring]{associative} and \( \norm{\cdot} \) satisfies the additional axiom
  \begin{thmenum}
    \thmitem{def:norm/multiplicativity}(multiplicativity)
    \begin{equation*}
      \norm{xy} = \norm{x} \cdot \norm{y} \text{ for all } x, y \in M,
    \end{equation*}
  \end{thmenum}
  we say that it is a \term{multiplicative norm}.
\end{definition}

\begin{definition}\label{def:norm_induced_metric}
  A norm \( \norm \cdot \) on a real or complex vector space \( V \) induces the \hyperref[def:vector_space]{metric}
  \begin{balign*}
     & \rho: V \times V \to \BbbR_{\geq 0}  \\
     & \rho(x, y) \coloneqq \norm{x - y}.
  \end{balign*}
\end{definition}
\begin{proof}
  The function is positive definite since \( \norm \cdot \) is positive definite; we will show that the function is a metric.

  \SubProofOf{def:metric_space/M1} Follows from \fullref{def:norm/N1}.

  \SubProofOf{def:metric_space/M2} By \fullref{def:norm/N2},
  \begin{equation*}
    \rho(x, y) = \norm{ x - y } = \norm{ (-1) (y - x) } = \abs{-1} \norm{y - x} = \rho(y, x).
  \end{equation*}

  \SubProofOf{def:metric_space/M3}
  \begin{equation*}
    \rho(x, y) + \rho(y, z) = \norm{x - y} + \norm{y - z} \geq \norm{x - z} = \rho(x, z).
  \end{equation*}
\end{proof}

\begin{definition}\label{def:duality_mapping}\mcite[exmpl. 2.26]{Phelps1993}
  We define the \term{duality mapping}
  \begin{balign*}
     & D: E \multto X^*,                                                                                              \\
     & D(x) \coloneqq \{ x^* \in X^* \colon \norm x = \norm {x^*} \text{ and } \inprod{x^*} x = \norm {x^*} \norm x \}.
  \end{balign*}

  We will usually use this mapping for unit vectors, so we may as well consider its restriction to the unit spheres, where
  \begin{balign*}
     & D': S_X \multto S_{X^*},                                       \\
     & D'(x) \coloneqq \{ x^* \in S_{X^*} \colon \inprod{x^*} x = 1 \}.
  \end{balign*}
\end{definition}

\begin{definition}\label{def:smooth_norm}\mcite[def. 2.36]{Phelps1993}
  The norm \( \norm \cdot \) on \( X \) is called \term{smooth} if any of  if for each \( x \in S_X \) the duality mapping is single-valued.
\end{definition}

\begin{definition}\label{def:rotund_norm}\mcite[def. 2.36]{Phelps1993}
  The norm \( \norm \cdot \) on \( X \) is called \term{rotund} or \term{strictly convex} if any of the following equivalent conditions hold:
  \begin{thmenum}
    \thmitem{def:rotund_norm/no_sphere_segments} There are no line segments in the unit sphere \( S_X \).
    \thmitem{def:rotund_norm/least_norm} Every convex subset of \( X \) has at most one point of least norm.
    \thmitem{def:rotund_norm/linearly_dependent}
    \begin{balign}\label{def:rotund_norm/linearly_dependent/equation}
      \norm{x + y} = \norm x + \norm y \implies x \text{ and } y \text{ are linearly dependent}.
    \end{balign}
  \end{thmenum}
\end{definition}
\begin{proof}
  \ImplicationSubProof{def:rotund_norm/no_sphere_segments}{def:rotund_norm/least_norm} Let the norm in \( E \) be rotund and let \( C \subseteq E \) be a (potentially empty) convex set. We will prove that \( C \) contains at most one point of least norm.

  If \( C \) is empty or otherwise contains no element of least norm, trivially contains at most one point of least norm.

  Now let \( C \) contain at least one element \( x \in C \) of least norm. Assume that \( y \in C \) is another element of least norm. Necessarily \( \norm x = \norm y \).

  Fix \( t \in (0, 1) \) and define \( z \coloneqq tx + (1-t)y \). Since \( C \) is convex, it contains \( z \). Since \( x \) and \( y \) are elements of least norm, we have \( \norm z \geq \norm x \). By the triangle inequality,
  \begin{balign*}
    \norm{z}
    =
    \norm{tx + (1-t)y}
    \leq
    t \norm x + (1-t) \norm y
    =
    \norm{x},
  \end{balign*}
  thus \( \norm z = \norm x \).

  This implies that the entire segment \( [x, y] \) are elements of least norm in \( C \). Hence, the segment \( [x, y] \) is contained in the sphere \( \norm x S_E \), which contradicts the rotundity of the norm \( \norm{\cdot} \).

  Hence, \( C \) contains at most one element of least norm.

  \ImplicationSubProof{def:rotund_norm/least_norm}{def:rotund_norm/no_sphere_segments} Let every convex set \( C \subseteq E \) have at most one element of least norm.

  Assume that the norm \( \norm{\cdot} \) is not rotund. Then the unit sphere \( S_E \) contains a line segment \( [x, y], x \neq y \). The set \( [x, y] \) is compact and, by the Weierstrass extreme value theorem, the norm attains its minimum on the segment in a point \( z \in [x, y] \). Since the segment is also convex and we assumed that convex sets have at most one element of least norm, it follows that this element \( z \) is unique.

  Then for any point \( s \in [x, y], s \neq z \), we have \( \norm s > \norm z = 1 \), thus \( s \) cannot be an element of the unit sphere. The obtained contradiction shows that the norm \( \norm{\cdot} \) is rotund.

  \ImplicationSubProof{def:rotund_norm/no_sphere_segments}{def:rotund_norm/linearly_dependent} Let \( E \) be rotund let \( x, y \in E \) be distinct vectors such that
  \begin{balign}\label{def:rotund_norm/linearly_dependent/assumption}
    \norm{x + y} = \norm x + \norm y.
  \end{balign}

  If either of them is the zero vector, then they are trivially linearly dependent.

  Assume that both \( x \) and \( y \) are nonzero and define
  \begin{balign*}
    \xi \coloneqq \frac x {\norm x}
     &  &
    \eta \coloneqq \frac y {\norm y}
     &  &
    t \coloneqq \frac {\norm x} {\norm{x + y}}
  \end{balign*}

  \Fullref{def:rotund_norm/linearly_dependent/assumption} implies that
  \begin{equation*}
    1 - t = 1 - \frac {\norm x} {\norm{x + y}} = \frac {\norm{x + y} - \norm x} {\norm{x + y}} = \frac {\norm y} {\norm{x+y}}.
  \end{equation*}

  Since both \( \xi \) and \( \eta \) are in \( S_E \), by rotundity, their convex combination
  \begin{equation*}
    \nu \coloneqq t \xi + (1-t)\eta
  \end{equation*}
  should not be contained in \( S_E \) unless \( \xi = \eta \).

  Calculating the norm, we obtain
  \begin{balign*}
    \norm{\nu}
     & =
    \norm{t \xi + (1-t)\eta}
    =    \\ &=
    \norm{\frac {\norm x \xi} {\norm{x + y}} + \frac {\norm y \eta} {\norm{x + y}}}
    =    \\ &=
    \norm{\frac {x + y} {\norm{x + y}}}
    = 1,
  \end{balign*}
  hence \( \nu \in S_E \). Thus, \( \xi = \eta \) and \( x = \frac {\norm x} {\norm y} y \), so \( x \) and \( y \) are linearly dependent.

  \ImplicationSubProof{def:rotund_norm/linearly_dependent}{def:rotund_norm/no_sphere_segments} Let \fullref{def:rotund_norm/linearly_dependent/equation} hold and fix \( x, y \in S_E, t \in (0, 1) \). Define \( z \coloneqq tx + (1-t)y \).
  First, assume that the vectors \( tx \) and \( (1-t)y \) satisfy the left part of \fullref{def:rotund_norm/linearly_dependent/equation}, i.e.
  \begin{equation*}
    \norm z = \norm{tx + (1-t)y} = t \norm x + (1-t) \norm y = 1.
  \end{equation*}

  This does not refute rotundity since \( x \) and \( y \) are not necessarily distinct. It follows from \fullref{def:rotund_norm/linearly_dependent/equation} that \( tx \) and \( (1-t)y \) are linearly dependent, hence \( x \) and \( y \) are also linearly dependent. Since \( x \) and \( y \) both have unit norm, either \( y = x \) or \( y = -x \).

  If we assume that \( y = -x \), then
  \begin{balign*}
    \norm z
    =
    \norm{tx + (1-t)y}
    =
    (2t - 1) \norm x
    =
    2t - 1,
  \end{balign*}
  which is only possible if \( t = 1 \) since \( \norm z = 1 \). But \( t \) is strictly less than 1.

  Hence, \( y \neq -x \) and the only remaining possibility is that \( y = x \).

  Now assume that the vectors \( tx \) and \( (1-t)y \) do not satisfy the left part of \fullref{def:rotund_norm/linearly_dependent/equation}. This implies \( \norm z < 1 \). Thus, \( x \) and \( y \) are necessarily distinct, but \( z \) is not contained in the unit sphere and the segment \( [x, y] \) is not contained in \( S_E \).

  We have shown that \( x, y \in S_E \) implies that either \( y = x \) or that the segment \( [x, y] \) is not contained in \( S_E \), thus the norm in \( E \) is rotund.
\end{proof}

\begin{theorem}\label{thm:smooth_rotund_norm_duality}\mcite[exer. 2.37(a)]{Phelps1993}
  If the norm in a Banach space \( X \) is such that its dual norm in \( X^* \) is rotund (resp. smooth), then it is itself smooth (resp. rotund).
\end{theorem}
\begin{proof}
  \begin{enumerate}
    \item First, let the dual norm \( \norm{\cdot}^* \) be rotund and assume that \( \norm{\cdot} \) is not smooth.

          Fix \( x \in S_X \). Since \( D(x) \) is nonempty (by \fullref{thm:hahn_banach_implies_duality_mapping_nonempty}) and since \( \norm{\cdot} \) is not smooth, then there exist two different functionals \( x^*, y^* \in D(x) \), such that
          \begin{balign*}
            \inprod {x^*} x
            =
            \inprod {y^*} x
            =
            1.
          \end{balign*}

          We will show that the segment \( [x^*, y^*] \) is contained in \( S_{X^*} \), i.e. that the dual norm is not rotund.

          Fix any \( t \in (0, 1) \) and define \( z^* \coloneqq t x^* + (1-t) y^* \). We only need to show that \( \norm{z^*} = 1 \).

          By the triangle inequality, we have
          \begin{balign*}
            \norm{z^*}
            =
            \norm{t x^* + (1-t) y^*}
            \leq
            t \norm{x^*} + (1-t) \norm{y^*}
            =
            t + (1-t)
            =
            1.
          \end{balign*}

          For the reverse inequality, note that
          \begin{balign*}
            \norm{z^*}
            \geq
            \inprod {z^*} x
            =
            t \inprod {x^*} x + (1-t) \inprod {y^*} x
            =
            t + (1-t)
            =
            1,
          \end{balign*}
          thus \( \norm{z^*} = 1 \). Hence, \( [x^*, y^*] \) is contained in \( S_{X^*} \) and the dual space is not smooth. The obtained contradiction proves that the norm in \( X \) is rotund.

    \item Now let the dual norm \( \norm{\cdot}^* \) be smooth and assume that \( \norm{\cdot} \) is not rotund. Then there exist points \( x, y \in S_X \) such that the while segment \( [x, y] \) is contained in \( S_X \).

          Fix \( t \in (0, 1) \) and define \( z \coloneqq tx + (1-t)y \in S_X \). Denote by \( J: X \to X^{**} \) the canonical embedding into the double-dual. By \fullref{thm:hahn_banach_implies_duality_mapping_nonempty}, there exists a functional \( z^* \in X^* \), such that
          \begin{balign*}
            \inprod {J(z)} {z^*}
            =
            \inprod{z^*} z
            =
            1.
          \end{balign*}

          Because the dual norm \( \norm{\cdot}^* \) is smooth, we cannot have \( \inprod{J(x)} {z^*} =  \inprod{z^*} x = 1 \) or \( \inprod{J(y)} {z^*} = \inprod{z^*} y = 1 \) and since \( \norm{z^*} = 1 \), necessarily
          \begin{equation*}
            \inprod{z^*} x < 1 \text{ and } \inprod{z^*} y < 1.
          \end{equation*}

          If follows that
          \begin{balign*}
            1
            =
            \inprod{z^*} z
            =
            t \inprod{z^*} x + (1-t) \inprod{z^*} y
            <
            t + (1-t)
            =
            1,
          \end{balign*}
          which is a contradiction. Hence, \( \norm{\cdot} \) is rotund.
  \end{enumerate}
\end{proof}

\begin{proposition}\label{thm:hilbert_space_smooth_rotund}\mcite[exer. 2.37(c)]{Phelps1993}
  Norms in Hilbert spaces are both smooth and rotund.
\end{proposition}
\begin{proof}
  Let \( X \) be a Hilbert space, i.e. the norm is generated by an inner product and, due to Riesz's theorem, we identify the space \( X \) with its continuous dual \( X^* \).

  To prove that \( X \) is rotund, choose \( x, y \in S_X, x \neq y \). We will show that the segment \( [x, y] \) is not contained in \( S_X \).

  If \( x \) and \( y \) are linearly dependent, necessarily \( y = -x \) and all non-trivial convex combinations of \( x \) and \( y \) are contained in the open unit ball, hence \( [x, y] \not\subseteq S_X \).

  Not let \( x \) and \( y \) be linearly independent. By the Cauchy-Bunyakovsky-Schwarz inequality, we have
  \begin{balign}\label{eq:hilbert_cauchy_inequality}
    \inprod x y \leq \abs{\inprod x y} < \norm x \norm y = 1.
  \end{balign}

  Fix \( t \in (0, 1) \) and let \( z \coloneqq tx + (1-t)y \). We will show that \( z \not\in S_X \). Indeed,
  \begin{balign*}
    \norm{z}^2
    =
    \inprod z z
     & =
    t^2 \norm x^2 + t(1-t) \inprod x y + (1-t) t \inprod y x + (1-t)^2 \norm y^2
    =    \\ &=
    t^2 + (1-t)^2 + 2 t(1-t) \inprod x y
    <    \\ &\reloset {(\ref{eq:hilbert_cauchy_inequality})} <
    t^2 + (1-t)^2 + 2 t(1-t)
    =    \\ &=
    t^2 + 1 - 2t + t^2 + 2t - t^2
    =
    1.
  \end{balign*}

  Thus, \( \norm{z}^2 < 1 \) and \( \norm z < 1 \) and \( z \not\in S_X \).

  In both cases, no interior point of the segment \( [x, y] \) is contained in \( S_X \), hence the norm in \( X \) is rotund.

  Since we identify \( X \) with its dual, the norm in \( X^* \) is also rotund and by \fullref{thm:smooth_rotund_norm_duality}, the norm in \( X \) is also smooth.
\end{proof}

\begin{example}\label{thm:c0_l1_not_smooth_rotund}\mcite[exer. 2.37(c)]{Phelps1993}
  The norms in \( c_0 \) and \( l^1 \) are neither smooth nor rotund.
\end{example}
\begin{proof}
  Consider the space \( c_0 \) of all real sequences that converge to zero equipped with the uniform norm
  \begin{equation*}
    \norm{x}_{c_0} \coloneqq \sup_i \abs{x_i}.
  \end{equation*}

  Note that the dual space of \( c_0 \) is (isometrically isomorphic to) the space \( l^1 \) of absolutely summable sequences with norm
  \begin{equation*}
    \norm{x}_{l^1} \coloneqq \sum_i \abs{x_i}.
  \end{equation*}

  Let \( \{ e_n \}_{n=1}^\infty \) be the canonical basis of \( c_0 \), i.e. the coordinates \( e^{(i)}_n \) of \( e_n \) are given by the Dirac delta function, \( e^{(i)}_n \coloneqq \delta_{i,n} \).

  For every natural \( n \geq 1 \), define \( x_n \) to be the same as \( e_n \) except that the first coordinate of \( x_n \) is always \( 1 \).

  The corresponding norms of \( e_n \) are all equal to 1 and the norms of \( x_n \) are
  \begin{balign*}
    \norm{x_n}_{c_0} = 1
     &  &
    \norm{x_n}_{l^1} = 2.
  \end{balign*}

  For every \( n \) we have
  \begin{equation*}
    \inprod {e_1} {x_n} = \inprod {e_n} {x_n} = 1,
  \end{equation*}
  hence \( J_{c_0}(x_n) \) has at least two elements \( e_1 \) and \( e_n \) and the norm in \( c_0 \) is not smooth.

  Given that \( \{ x_1, x_2, \ldots \} \subseteq S_{c_0} \), consider the convex combinations of \( x_2 \) and \( x_3 \):
  \begin{balign*}
    tx_2 + (1-t)x_3
    =
    (1, t, (1-t), 0, 0, \ldots).
  \end{balign*}

  Evidently \( tx_2 + (1-t)x_3 \in S_{c_0} \) for every \( t \in (0, 1) \), hence the norm in \( c_0 \) is not rotund.

  The contrapositions to the statements in \fullref{thm:smooth_rotund_norm_duality} say that if \( X \) is not rotund (resp. smooth), then the dual space \( X^* \) is not smooth (resp. rotund). Thus, \( l^1 \) is neither smooth or rotund as the dual of \( c_0 \).
\end{proof}

\begin{definition}\label{def:bilinear_form_induced_norm}
  Let \( V \) be a real or complex \hyperref[def:inner_product_space]{inner product space} with product \( \inprod \cdot \cdot \). We define its induced \hyperref[def:norm]{norm} as
  \begin{balign*}
     & \norm \cdot : V \to \BbbR_{\geq 0}    \\
     & \norm x \coloneqq \sqrt{\inprod x x}.
  \end{balign*}

  If \( V \) is a real inner product space, the induced norm is a square root of the induced quadratic \hyperref[def:quadratic_form]{form} of \( \inprod \cdot \cdot \).
\end{definition}
\begin{proof}
  We will only prove the complex case because the real case is identical, but slightly simpler.

  Note that \( \norm \cdot \) is well-defined (that is, positive definite) by \fullref{thm:inner_product_quadratic_form_is_positive_definite}.

  Now we will show that it is a norm.
  \SubProofOf{def:norm/N1} Follows from the positive definiteness of \( \inprod \cdot \cdot \)

  \SubProofOf{def:norm/N2} For \( t \in \BbbC \) and \( x \in V \) we have
  \begin{equation*}
    \norm{tx} = \sqrt{\inprod{tx} {tx}} = \abs{t} \sqrt{\inprod x x} = \abs t \norm x.
  \end{equation*}

  \SubProofOf{def:norm/N3} For \( x, y \in V \) we have
  \begin{balign*}
    \norm{x + y}^2
     & =
    \inprod{x + y} {x + y}
    =                                                            \\ &=
    \inprod x x + \inprod x y + \inprod y x + \inprod y y
    =                                                            \\ &=
    \norm{x}^2 + 2 \real \inprod x y + \norm{y}^2
    \leq                                                         \\ &\leq
    \norm{x}^2 + 2 \abs{\real \inprod x y} + \norm{y}^2
    \reloset {\ref{thm:cauchy_bunyakovsky_schwarz_inequality}} = \\ &=
    \norm{x}^2 + 2 \norm x \norm y + \norm{y}^2
    =
    (\norm{x} + \norm{y})^2
  \end{balign*}

  Therefore,
  \begin{equation*}
    \norm{x + y} \leq \norm x + \norm y.
  \end{equation*}
\end{proof}


% Functional analysis
\section{Functional analysis}\label{sec:functional_analysis}

In this section, \( \BbbK \) will refer to either \( \BbbR \) or \( \BbbC \). See \fullref{rem:real_field_extensions} for a justification.

\subsection{Topological groups}\label{subsec:topological_groups}

\begin{definition}\label{def:topological_group}
  Let \( G \) be any \hyperref[def:group]{group} and let \( \mscrT \) be a topology on \( G \). The tuple \( (G, \cdot, \mscrT) \) is called a \term{topological group} if the group structure and topological structure agree, that is, the operations \( \cdot: X \times X \to X \) and \( (-)^{-1}: X \to X \) are continuous with respect to \( \mscrT \).

  See \fullref{rem:hausdorff_topological_groups} and \fullref{def:category_of_topological_groups} for more nuances.
\end{definition}

\begin{remark}\label{rem:hausdorff_topological_groups}
  It is conventional to require the topology in a topological group to be \( T_1 \) (see \fullref{def:separation_axioms}). We will not do this due to our goal of not assuming more than is necessary.

  Due to \fullref{thm:topological_group_t0_iff_t3.5}, it is immaterial whether we require the topology to be \( T_0 \) or \( T_{3.5} \) or anywhere in between. It is customary to call the space \enquote{Hausdorff} (although stronger separation axioms actually hold) and require \( T_1 \) to hold (since it is simple to state).

  We will explicitly mention when we want a topological group to be Hausdorff. This is usually, so when we speak of convergence.
\end{remark}

\begin{definition}\label{def:category_of_topological_groups}
  The category \( \cat{TopGrp} \) of topological groups is a subcategory of both \( \cat{Top} \) and \( \cat{Grp} \). Its morphisms are the \hyperref[def:global_continuity]{continuous} group \hyperref[thm:group_homomorphism_single_condition]{homomorphisms}.
\end{definition}

\begin{proposition}\label{thm:neighborhood_translations_in_topological_groups}
  Fix \( x, y \in G \) in a topological group \( G \). If \( U \) is a neighborhood of \( x \), then both \( V = yx^{-1} U \) and \( W = U x^{-1}y \) are neighborhoods of \( y \).
\end{proposition}
\begin{proof}
  Since the group operations are continuous, for fixed \( x \) and \( y \), the function \( f(z) \coloneqq xy^{-1}z \) is continuous.

  Note that \( U = f(V) \), hence \( V \) is the preimage of \( U \) under \( f \) and it follows from the continuity of \( f \) that \( V \) is open.

  Since \( x \in U \), \( yx^{-1}x = ye = y \in V \). Therefore, \( V \) is a neighborhood of \( y \).

  The proof that \( W \) is a neighborhood of \( y \) is analogous.
\end{proof}

\begin{corollary}\label{thm:origin_neighborhoods_in_topological_groups}
  In a topological group \( G \), every neighborhood is a translation of e neighborhood of the origin \( e \).
\end{corollary}

\begin{remark}\label{rem:origin_neighborhoods_in_topological_groups}
  \Fullref{thm:origin_neighborhoods_in_topological_groups} provides a lot of uniformity by allowing us to only consider neighborhoods of zero when working with topological groups.
\end{remark}

\begin{proposition}\label{thm:topological_group_t0_iff_t3.5}
  If a topological group is \( T_0 \), it is automatically \( T_{3.5} \).
\end{proposition}

\begin{proposition}\label{thm:topological_group_uniform_space}
  A Hausdorff topological group \( G \) can be made into a uniform space by the families of entourages
  \begin{balign*}
     & V^l_A \coloneqq \{ (x, y) \in G \times G \colon x^{-1} y \in A \}, \\
     & V^r_A \coloneqq \{ (x, y) \in G \times G \colon x y^{-1} \in A \},
  \end{balign*}
  where \( A \) is a \hyperref[def:neighborhood_set_types/symmetric]{symmetric} neighborhood of the origin \( e \).

  If \( G \) is abelian, the two families of entourages coincide.
\end{proposition}

\begin{proposition}\label{thm:limits_are_topological_group_homomorphisms}
  If \( \{ a_\alpha \}_{\alpha \in \mscrK} \) and \( \{ b_\alpha \}_{\alpha \in \mscrK} \) are \hyperref[def:topological_net]{nets} in a Hausdorff topological group \( X \) that converge to \( a \) and \( b \), correspondingly, then \( a_\alpha b_\alpha \to a b \).
\end{proposition}
\begin{proof}
  Special case of \fullref{thm:linearity_of_sequence_limits}.
\end{proof}

\subsection{Topological vector spaces}\label{subsec:topological_vector_spaces}

\begin{definition}\label{def:topological_vector_space}
  Let \( X \) be any vector space and let \( \mscrT \) be a topology on \( X \). The space \( (X, +, \cdot, \mscrT) \) is called a \term{topological vector space} if the linear and topological structure agree, that is, the operations \( +: X \times X \to X \) and \( \cdot: X \times \BbbR \to X \) are continuous with respect to \( \mscrT \).

  Both the additive group \( (X, +) \) and the multiplicative group \( (X \setminus \{ 0 \}, \cdot) \) are \hyperref[def:topological_group]{topological groups}. We regard \( X \) as a subgroup of its additive topological group.

  See \fullref{rem:hausdorff_topological_groups}, \fullref{def:continuous_dual_space} and \fullref{def:category_of_topological_vector_spaces} for more nuances.
\end{definition}

Given that a topological vector space \( X \) has both a topological and an algebraic structure, we should adapt certain definitions.

\begin{definition}\label{def:continuous_dual_space}
  We define the \term{continuous dual space} \( X^* \) of a topological space \( X \) as the vector space of all \hyperref[def:global_continuity]{continuous} linear functionals. This differs drastically from \fullref{def:dual_vector_space} because in the general case, the continuous dual space may be trivial, i.e. only contain the zero functional. See \fullref{def:locally_convex_duality_pairing}.

  We use the same notation for both the algebraic dual spaces and the continuous dual space because the meaning is usually clear from the context. In particular, hyperplanes as defined in \fullref{def:hyperplane} are only relevant to continuous linear functionals.
\end{definition}

\begin{definition}\label{def:category_of_topological_vector_spaces}
  The category \( \cat{TopVect}_{\BbbK} \) of topological vector spaces over \( \BbbK \) is a subcategory of both \( \cat{Top} \) and \( \cat{Vect}_K \). Its morphisms are the \hyperref[def:global_continuity]{continuous} linear \hyperref[def:linear_operator]{maps}.
\end{definition}

\begin{remark}\label{rem:origin_neighborhoods_in_topological_vector_spaces}
  As in \fullref{rem:origin_neighborhoods_in_topological_groups}, we are only interested in neighborhoods of the origin \( 0 \) since any neighborhood \( U \) of \( x \) is simply a translation of the neighborhood \( U - x \) of the origin.
\end{remark}

\begin{proposition}\label{thm:topological_vector_space_is_uniform}
  A Hausdorff topological vector space \( X \) is a uniform space with the families of entourages
  \begin{balign*}
     & V_A \coloneqq \{ (x, y) \in X \times X \colon x - y \in A \},
  \end{balign*}
  where \( A \) is a \hyperref[def:neighborhood_set_types/symmetric]{symmetric} neighborhood of the origin \( 0 \).
\end{proposition}
\begin{proof}
  Follows from \fullref{thm:topological_group_uniform_space}.
\end{proof}

\begin{proposition}\label{thm:linearity_of_sequence_limits}
  If \( \{ a_\alpha \}_{\alpha \in \mscrK} \) and \( \{ b_\alpha \}_{\alpha \in \mscrK} \) are \hyperref[def:topological_net]{nets} in a Hausdorff topological vector space \( X \) that converge to \( a \) and \( b \), correspondingly, then
  \begin{thmenum}
    \thmitem{thm:linearity_of_sequence_limits/addition} \( a_\alpha + b_\alpha \to a + b \).
    \thmitem{thm:linearity_of_sequence_limits/scalar_multiplication} \( \lambda a_\alpha \to \lambda a \) for any scalar \( \lambda \in \BbbK \).
  \end{thmenum}
\end{proposition}
\begin{proof}
  Fix a neighborhood \( U \) of \( 0 \) and fix an index \( \alpha_0 \) such that for \( \alpha \geq \alpha_0 \) we have both \( a - a_\alpha \in U \) and \( b - b_\alpha \in U \).

  \SubProofOf{thm:linearity_of_sequence_limits/addition} For addition, we have
  \begin{equation*}
    (a + b) - (a_\alpha + b_\alpha) = (a - a_\alpha) + (b - b_\alpha) \in 2U.
  \end{equation*}

  \SubProofOf{thm:linearity_of_sequence_limits/scalar_multiplication} For scalar multiplication, we have
  \begin{equation*}
    \lambda a - \lambda a_\alpha \in \lambda U.
  \end{equation*}

  In both cases the containing neighborhood does not depend on \( \alpha \), hence the nets converge to their desired values.
\end{proof}

\begin{corollary}\label{thm:linearity_of_function_limits}
  If \( f, g: X \to Y \) are continuous functions between topological vector spaces, then for any point \( x_0 \in X \) we have
  \begin{equation*}
    \lim_{x \to x_0} (f(x) + g(x)) = \lim_{x \to x_0} f(x) + \lim_{x \to x_0} g(x)
  \end{equation*}
  and for any \( \lambda \in \BbbK \)
  \begin{equation*}
    \lim_{x \to x_0} \lambda f(x) = \lambda \lim_{x \to x_0} f(x).
  \end{equation*}
\end{corollary}

\begin{definition}\label{def:locally_convex_space}\mcite[1.8]{Rudin1991Functional}
  We say that a \hyperref[def:topological_vector_space]{topological vector space} is \term{locally convex} if there exists a \hyperref[def:topological_base]{topological base} of \hyperref[def:convex_set]{convex} sets.
\end{definition}

\begin{remark}\label{def:locally_convex_duality_pairing}
  Given a Hausdorff locally convex space \( X \), \fullref{thm:hahn_banach_implies_functionals_vanish_nowhere} shows that the canonical duality pairing as defined in \fullref{def:locally_convex_duality_pairing} is nondegenerate. If the space is not locally convex, we cannot guarantee that the pairing will be nondegenerate and our restriction to continuous linear functionals could interfere with our habits of working with linear functionals.
\end{remark}

\begin{definition}\label{def:sublinear_functional}
  We say that \( f: X \to \BbbR \) is a \term{sublinear functional} if it satisfies
  \begin{thmenum}
    \thmitem{def:sublinear_functional/subadditivity}(subadditivity) \( f(x + y) \leq f(x) + f(y) \) for any \( x, y \in X \).
    \thmitem{def:sublinear_functional/positive_homogeneity}(positive homogeneity) \( f(tx) \leq t f(x) \) for any \( t > 0 \) and \( x \in X \).
  \end{thmenum}

  Compare this definition to \fullref{def:linear_operator}.
\end{definition}

\subsection{The Hahn-Banach theorem}\label{subsec:hahn_banach}

The Hahn-Banach theorem is an important result that can be stated differently and in different levels of generality.

\begin{theorem}[Geometric Hahn-Banach theorem/Mazur's theorem]\label{thm:geometric_hahn_banach}\mcite\cite[24]{ИоффеТихомиров1974}
  Fix a \hyperref[def:topological_vector_space]{topological vector space} \( X \). Let \( A \subseteq X \) be an open \hyperref[def:convex_set]{convex} set and \( L \subseteq X \) be a subspace that is disjoint from \( A \). Then there exists a continuous linear functional \( x^* \in X^* \) such that
  \begin{equation*}
    \begin{array}{l}
      \real \inprod{x^*} x > 0, x \in A \\
      \real \inprod{x^*} x = 0, x \in L
    \end{array}
  \end{equation*}

  See \fullref{rem:linear_functionals_over_c} for a justification of only considering the real part of \( x^* \).
\end{theorem}

\begin{corollary}\label{thm:hahn_banach_implies_functionals_vanish_nowhere}\mcite\cite[24]{ИоффеТихомиров1974}
  The \hyperref[def:dual_vector_space]{dual} of a Hausdorff \hyperref[def:locally_convex_space]{locally convex space} \( X \) does not \hyperref[def:functions_vanish_nowhere]{vanish} at the nonzero vectors of \( X \).
\end{corollary}
\begin{proof}
  Fix a nonzero point \( x \in X \). The result follows from \fullref{thm:geometric_hahn_banach} with \( L \coloneqq \{ 0 \} \) and \( A \) -- any convex set containing \( x \) and not containing zero. Such a set \( A \) exists because the topology is Hausdorff and \( x \) has a neighborhood disjoint from any point in \( L \).
\end{proof}

\begin{corollary}\label{thm:hahn_banach_implies_annihilator_nontrivial}\mcite\cite[25]{ИоффеТихомиров1974}
  The \hyperref[def:vector_space_annihilator]{annihilator} of any proper subspace of a Hausdorff \hyperref[def:locally_convex_space]{locally convex space} contains nonzero elements.
\end{corollary}
\begin{proof}
  Denote the proper subspace by \( L \subsetneq X \). Fix \( x \in X \setminus L \) and let \( A \) be a convex neighborhood of \( x \) that is disjoint from \( L \). The result follows from \fullref{thm:geometric_hahn_banach}.
\end{proof}

\begin{corollary}\label{thm:hahn_banach_implies_duality_mapping_nonempty}\mcite\cite[25]{ИоффеТихомиров1974}
  In a \hyperref[def:norm]{normed} space \( X \), for any nonzero vector \( x \in X \) there exists a continuous functional \( x^* \in S_{X^*} \) such that \( \inprod {x^*} x = \norm x \). In other words, the duality \hyperref[def:duality_mapping]{mapping} is nonempty for any point.
\end{corollary}
\begin{proof}
  This follows from \fullref{thm:hahn_banach_implies_annihilator_nontrivial} by taking \( A \coloneqq B(x, \abs{x}) \) and \( L \coloneqq \{ 0 \} \) and then scaling the obtained functional.
\end{proof}

\begin{theorem}[Hahn-Banach hyperplane separation theorem]\label{thm:hahn_banach_hyperplane_separation}\mcite\cite[25]{ИоффеТихомиров1974}
  Fix a \hyperref[def:topological_vector_space]{topological vector space} \( X \). Let \( A, B \subseteq X \) be disjoint \hyperref[def:convex_set]{convex} sets. If \( \int{A} \neq \varnothing \), there exists a continuous linear functional \hyperref[def:hyperplane_separation]{separating} \( A \) and \( B \).
\end{theorem}

\subsection{Frechet spaces}\label{subsec:frechet_spaces}

\begin{definition}\label{def:frechet_space}\MarginCite[1.8 (f)]{Rudin1991}
  An \Def{F-space} is a \hyperref[thm:uniform_space_completion]{complete} \hyperref[def:metric_topology]{metrizable} \hyperref[def:topological_vector_space]{topological vector space}. We can assume that an F-space is a tuple \( (\CX, \rho) \), where \( \rho \) is a \hyperref[def:complete_metric_space]{complete} \hyperref[def:translation_invariant_metric]{translation-invariant} \hyperref[def:metric_space]{metric}.

  A \Def{Frechet space} is a \hyperref[def:locally_convex_space]{locally convex} F-space.
\end{definition}

\subsection{Banach spaces}\label{subsec:banach_spaces}

\begin{definition}\label{def:banach_space}
  A \term{Banach space} is a \hyperref[def:norm]{normed} \hyperref[def:vector_space]{vector space} which is also a \hyperref[def:complete_metric_space]{complete metric spaces} with the metric induced by the \hyperref[def:norm_induced_metric]{norm}.
\end{definition}

\begin{definition}\label{def:topological_duality_pairing}
  Let \( M \) and \( N \) be left \( R \)-modules. A \term{duality pairing} \( \inprod \cdot \cdot: M \times N \to R \) is a \hyperref[def:degenerate_bilinear_form]{nondegenerate} bilinear form.

  The \term{canonical duality pairing} of a vector space \( V \) over \( F \) is
  \begin{balign*}
     & \inprod \cdot \cdot: V^* \times V \to F \\
     & \inprod {x^*} x \mapsto x^*(x).
  \end{balign*}
\end{definition}

\begin{example}\label{ex:noncomplete_normed_space}\mcite{MathCounterExamples:noncomplete_normed_space}
  Consider the polynomial \hyperref[def:polynomial_algebra]{algebra} \( \BbbR[x] \) as a vector space with the supremum norm. We will show that it is not complete. Define the sequence
  \begin{equation*}
    p_n(x) \coloneqq \sum_{k=0}^n \frac{x^k} {2^k}, n = 1, 2, \ldots
  \end{equation*}

  Then the limit of the sequence in \( C([0, 1]) \) is the power series
  \begin{equation*}
    \lim_{n \to \infty} p_n(x)
    =
    \sum_{k=0}^n \frac{x^k} {2^k}
    =
    \frac 2 {2 - x}.
  \end{equation*}

  Since \( \BbbR[x] \) is a subspace of \( C([0, 1]) \), we conclude that \( \BbbR[x] \) has fundamental sequence, but we just demonstrated that its limit is not in \( \BbbR[x] \).
\end{example}

\begin{definition}\label{def:dual_norm}
  Fix two nonempty Banach spaces \( (X, \norm{\cdot}_X) \) and \( (Y, \norm{\cdot}_Y) \). We define the \term{operator norm} \( \norm{\cdot}_{\hom(X, Y)} \) on \( \hom(X, Y) \) equivalently as
  \begin{thmenum}
    \thmitem{def:dual_norm/sup_unit_sphere}
    \begin{equation*}
      \norm{L}_{\hom(X, Y)} \coloneqq \sup_{\norm{x}_X = 1} \norm{Lx}_Y.
    \end{equation*}

    \thmitem{def:dual_norm/sup_unit_ball}
    \begin{equation*}
      \norm{L}_{\hom(X, Y)} \coloneqq \sup_{\norm{x}_X < 1} \norm{Lx}_Y.
    \end{equation*}

    \thmitem{def:dual_norm/sup_nonzero}
    \begin{equation*}
      \norm{L}_{\hom(X, Y)} \coloneqq \sup_{x \neq 0_X} \frac {\norm{Lx}_Y} {\norm{x}_X}.
    \end{equation*}

    \thmitem{def:dual_norm/inf}
    \begin{equation*}
      \norm{L}_{\hom(X, Y)} \coloneqq \inf \left\{ c \geq 0 \colon \norm{Lx}_Y \leq c \norm{x}_X \right\}.
    \end{equation*}
  \end{thmenum}

  In particular, this induces a norm on \( X^* \).
\end{definition}

\begin{definition}\label{def:banach_space_support_function}\mcite[exmpl. 3.2(a)]{Phelps1993})
  Let \( X \) be a Banach space.

  We define the \term{support function \( \sigma_{A^*} \) for the set of functionals \( A^* \subseteq X^* \)} by
  \begin{balign*}
     & \sigma_{A^*}: X \to \BbbR \cup \{ \infty \}                             \\
     & \sigma_{A^*}(x) \coloneqq \sup \{ \inprod {x^*} x \colon x^* \in A^* \}
  \end{balign*}

  and the \term{weak* support function \( \sigma^*_A \) for the set of points \( A \subseteq X \)} by
  \begin{balign*}
     & \sigma^*_A: X^* \to \BbbR \cup \{ \infty \}                          \\
     & \sigma^*_A(x^*) \coloneqq \sup \{ \inprod {x^*} x \colon x \in A \}.
  \end{balign*}
\end{definition}

\begin{definition}\label{def:banach_space_slice}\mcite[def. 2.17]{Phelps1993}
  Given a linear functional \( x^* \), a nonempty subset \( A \) of \( X \) and a \term{diameter} \( \alpha > 0 \), the value \( S(x^*, A, \alpha) \) is called a \term{slice} of \( A \), where
  \begin{balign*}
     & S: X^* \times \pow(X) \times \BbbR_{>0} \mapsto \pow(A)                                      \\
     & S(x^*, A, \alpha) \coloneqq \{ x \in A \colon \inprod {x^*} x > \sigma_A^*(x^*) - \alpha \}.
  \end{balign*}

  We define a weak* slice of \( A^* \subseteq X^* \) as \( S^*(x, A^*, \alpha) \), where
  \begin{balign*}
     & S^*: X \times \pow(X) \times \BbbR_{>0} \mapsto \pow(A)                                            \\
     & S^*(x, A^*, \alpha) \coloneqq \{ x^* \in A^* \colon \inprod {x^*} x > \sigma_{A^*}(x) - \alpha \}.
  \end{balign*}

  If we need to make the underlying space explicit, we will use \( S_X(x^*, A, \alpha) \) and \( S_X^*(x, A^*, \alpha) \).
\end{definition}

\begin{proposition}
  If \( \{ a_k \}_{k=1}^\infty \) and \( \{ b_k \}_{k=1}^\infty \) are sequences a in a \hyperref[def:banach_space]{Banach} \hyperref[def:algebra_over_semiring]{algebra} \( X \), that converge to \( a \) and \( b \), correspondingly, then \( a_k b_k \to a b \).
\end{proposition}
\begin{proof}
  Let \( \delta > 0 \) and let \( k_0 \) be an index such that for \( k \geq k_0 \) we have both \( \norm{a - a_k} < \delta \) and \( \norm{b - b_k} < \delta \). Then
  \begin{balign*}
    ab - a_k b_k
     & =
    (ab - a b_k) + (a b_k - a_k b) + (a_k b - a_k b_k)
    =    \\ &=
    a (b - b_k) + (a b_k - ab + ab - a_k b) + (-a_k)(b_k - b)
    =    \\ &=
    a (b - b_k) + a \underbrace{(b_k - b)} + (a - a_k) b + (-a_k)\underbrace{(b_k - b)}
    =    \\ &=
    a \underbrace{(b - b_k)}_{\in B(0, \delta)} + \underbrace{(a - a_k)}_{\in B(0, \delta)} \underbrace{(b_k - b)}_{\in B(0, \delta)} + \underbrace{(a - a_k)}_{\in B(0, \delta)} b.
  \end{balign*}

  Therefore, \( \norm{ab - a_k b_k} < \delta^2 + \norm{a + b} \delta \). If we require \( \delta \) to be strictly less than \( 1 \), we obtain \( \delta^2 < \delta \) and \( \norm{ab - a_k b_k} < (1 + \norm{a + b}) \delta \).

  Given an arbitrary \( \varepsilon > 0 \), we can choose \( \delta = \tfrac {\min \{\varepsilon, 1 \}} {1 + \norm{a + b}} \) in order to have \( \norm{ab - a_k b_k} < \varepsilon \) for some large enough \( k \).

  Therefore, \( a_k b_k \to a b \).
\end{proof}

\subsection{Hilbert spaces}\label{subsec:hilbert_spaces}

\begin{definition}\label{def:hilbert_space}
  A \term{Hilbert space} is an \hyperref[def:inner_product_space]{inner product space} which is also a \hyperref[def:complete_metric_space]{complete metric space} with the metric induced by the inner product.
\end{definition}

\begin{definition}\label{def:orthonormal_system}
  A set of vectors \( A \) in a Hilbert space \( X \) is called an \term{orthonormal system} if \( A \)
  \begin{equation*}
    \inner x y \coloneqq \begin{cases}
      1, & x = y,    \\
      0, & x \neq y.
    \end{cases}
  \end{equation*}

  It is a special case of an \hyperref[def:orthogonality]{orthogonal system}. We are usually interested in \term{orthogonal bases}.
\end{definition}

\subsection{Asplund spaces}\label{subsec:asplund_spaces}

\begin{definition}\label{def:asplund_space}
  The Banach space \( X \) is called an Asplund (resp. weak Asplund) space if any of the following equivalent conditions hold:

  \begin{thmenum}
    \ilabel{def:asplund_space/differentiable_on_dense_subset}\mcite\cite[thm. 2.14]{Phelps1993}Every continuous convex function on a convex open subset \( D \) of \( X \) is Frechet (resp. Gateaux) differentiable at a dense \( G_\delta \) subset of \( D \).

    \ilabel{def:asplund_space/radon_nikodym}\mcite\cite[def. 5.2]{Phelps1993}The dual space \( X^* \) has the Radon-Nikodym property.

    \medskip

    \ilabel{def:asplund_space/exposed_points}\mcite\cite[thm. 5.12]{Phelps1993}Every nonempty weak* compact convex subset of \( X^* \) is the weak* closed convex hull of its weak* strongly exposed points.
  \end{thmenum}
\end{definition}

\subsection{Minkowski functionals}\label{subsec:minkowski_functionals}

\begin{definition}\label{def:minkowski_functional}
  Let \( A \) is an \hyperref[def:neighborhood_set_types/absorbing]{absorbing} \hyperref[def:convex_hull]{convex} set.

  We define the corresponding \term{Minkowski functional}
  \begin{balign*}
     & \rho_A: X \to [0, \infty),
     & \rho_A(x) = \inf \{ t > 0 \colon x \in tA \}.
  \end{balign*}
\end{definition}
\begin{proof}
  We will prove that \( \rho_A(x) \) is always a nonnegative real number. Obviously
  \begin{equation*}
    \rho_A(x) \geq 0
  \end{equation*}
  since the infimum over \( \BbbR_{>0} \) is \( 0 \).

  Now fix \( x \in X \). Since \( A \) is an absorbing set, there exists \( t_0 > 0 \) such that \( t_0 x \in A \). We need to take the infimum of all such numbers. This infimum exists since \( \BbbR \) is complete and the set over which we take the minimum is bounded.
\end{proof}

\subsection{Dentable sets}\label{subsec:dentable_sets}

\begin{definition}\label{def:dentability}\mcite\cite[def. 5.1]{Phelps1993}
  A subset \( A \) of a Banach space \( X \) is called \term{dentable} if it admits slices of arbitrarily small diameter, i.e. for every \( \varepsilon > 0 \) there exist a functional \( x^* \in X^* \) and a diameter \( \alpha > 0 \), such that \( \diam S(x^*, A, \alpha) < \varepsilon \).

  Weak* dentability is defined in an obvious way.
\end{definition}

\begin{definition}\label{def:radon-nikodym-property}\mcite\cite[def. 5.2]{Phelps1993}
  The space \( X \) is said to have the \term{Radon-Nikodym property (RNP)} if every nonempty bounded set \( A \) of \( X \) is dentable.
\end{definition}

\begin{proposition}\label{thm:weak_dentable_sets_are_dentable}
  Let \( X \) be a Banach space and \( A^* \subseteq X^* \) be a weak*-dentable set. Then \( A^* \) is dentable in \( X^* \).
\end{proposition}
\begin{proof}
  Let \( \varepsilon > 0 \) and let \( x \in X \) and \( \alpha > 0 \) be such that \( \diam S^*(x, A^*, \alpha) < \varepsilon \).
  We denote by \( J(x) \) the embedding of \( x \in X \) into the double-dual \( X^{**} \) and by \( T(J(x), A^*, \alpha) \) the slice of \( A^* \) in \( X^* \). We have that
  \begin{balign*}
    S^*(x, A^*, \alpha)
     & =
    \{ x^* \in A^* \colon \inner {x^*} x > \sigma_{A^*}(x) - \alpha \}
    =    \\ &=
    \{ x^* \in A^* \colon \inner {x^*} x > \sup \{ \inner {y^*} x \colon y^* \in A^* \} - \alpha \}
    =    \\ &=
    \{ x^* \in A^* \colon \inner {J(x)} {x^*} > \sup \{ \inner {J(x)} {y^*} \colon y^* \in A^* \} - \alpha \}
    =
    T(J(x), A^*, \alpha),
  \end{balign*}

  Since \( J \) is an isometry, this equality implies that
  \begin{equation*}
    \diam T(J(x), A^*, \alpha) = \diam S(x, A^*, \alpha) < \varepsilon.
  \end{equation*}

  Hence \( A^* \) admits arbitrarily small slices in \( X^* \), i.e. it is dentable in \( X^* \).
\end{proof}

\subsection{Differentiability}\label{subsec:differentiability}

Let \( X \) and \( Y \) be Hausdorff \hyperref[def:topological_vector_space]{topological vector spaces} and let \( U \subseteq X \) be an open set.

\begin{definition}\label{def:differentiability}
  Out goal is to study the following \hyperref[def:partial_function]{partial} operator:
  \begin{balign}\label{def:differentiability/partial_operator}
     & \partial: \cat{Set}(U, Y) \times U \times X \to Y                             \\
     & \partial(f, x, h) \coloneqq \lim_{t \downarrow 0} \frac {f(x + th) - f(x)} t.
  \end{balign}

  We implicitly assume that \( t \neq 0 \) because otherwise the definition would not make sense.

  We only use the operator \( \partial \) inside this definition. See \fullref{rem:derivative_notation} for a discussion of derivative notation.

  The quotient under the limit sign is called a \term{difference quotient}.

  For each function \( f: U \to Y \), each point \( x_0 \in X \) and each \enquote{direction} vector \( x_0 \in X \), we want to obtain a value in \( Y \), which we will call the \term{directional derivative} of \( f \) at \( x_0 \) in the direction \( h \). Note that \( h \) is allowed to range over \( X \).

  The existence of a directional derivative is already a harsh condition, however we impose even harsher restrictions

  \begin{thmenum}
    \thmitem{def:differentiability/first_variation}\mcite[sec. 0.2.1]{ИоффеТихомиров1974}If, for fixed \( f \) and \( x_0 \), the directional derivative \( \partial(f, x_0, h) \) exists for all directions \( h \), we define the \term{first variation} of \( f \) at \( x_0 \) as
    \begin{balign*}
       & \delta f(x_0): X \to Y                            \\
       & [\delta f(x_0)](h) \coloneqq \partial(f, x_0, h).
    \end{balign*}

    Within its domain of definition of \( \delta \), which is stricter than that of \( \partial \), the operator \( \delta \) is a \hyperref[def:function/currying]{currying} of \( \partial \). We are interested in how the operator \( \delta f(x) \) varies as \( x \) varies.

    Note that \( \delta f(x_0) \) is an operator from \( X \) to \( Y \) even if \( f \) is a function from \( U \subsetneq X \) to \( Y \).

    Note that, in general, the first variation operator \( \delta f(x_0) \) is not linear - for example, by \fullref{thm:convex_one_sided_derivatives_sublinear}, the first variation of a general convex functions is, at most, sublinear. See \fullref{subsec:nonsmooth_derivatives} for how \enquote{nonlinear derivatives} are handled.

    \thmitem{def:differentiability/gateaux}\mcite[sec. 0.2.1]{ИоффеТихомиров1974}If the first variation \( \delta f(x_0) \) is a continuous linear operator, we say that \( f \) is \term{Gateaux differentiable} or \term{weakly differentiable} at \( x_0 \) with \term{Gateaux derivative} \( f'_G(x_0) \coloneqq \delta f(x_0) \).

    Since \( f'(x_0) \) is linear in \( h \), we can replace \( t \downarrow 0 \) with \( t \to 0 \) in \fullref{def:differentiability/partial_operator} and reformulate this condition of Gateaux differentiability as the existence of a continuous linear operator \( \Lambda: X \to Y \) such that
    \begin{equation}\label{def:differentiability/gateaux/condition}
      \Lambda(h) = \lim_{t \to 0} \frac {f(x_0 + h) - f(x_0)} t.
    \end{equation}

    If \( \Lambda \) exists, we usually denote it by \( D_G f(x_0) \) or \( f_G'(x_0) \) and call it the \term{Gateaux derivative} of \( f \) at \( x_0 \). See \fullref{rem:derivative_notation} for a discussion of the notation.

    \thmitem{def:differentiability/frechet}\mcite[sec. 0.2.1]{ИоффеТихомиров1974}We now restrict our attention to \hyperref[def:banach_space]{Banach spaces}. We say that \( f \) is \term{Frechet differentiable} or \term{strongly differentiable} at \( x_0 \) if there exists a continuous linear operator \( \Lambda: X \to Y \) such that
    \begin{equation}\label{def:differentiability/frechet/condition}
      \lim_{h \to 0} \frac {\norm{f(x_0 + h) - f(x_0) - \Lambda(h)}_Y} {\norm{h}_X} = 0.
    \end{equation}

    If \( \Lambda \) exists, we usually denote it by \( D f(x_0) \) or \( f(x_0) \) and call it the \term{Frechet derivative} of \( f \) at \( x_0 \). See \fullref{rem:derivative_notation} for a discussion of the notation.

    Note that \fullref{def:differentiability/gateaux/condition} uses convergence in the topology of \( Y \) while \fullref{def:differentiability/frechet/condition} uses convergence in \( \BbbR \). We discuss in \fullref{rem:gateaux_vs_frechet} how Frechet differentiability is a special \enquote{uniform} case of Gateaux differentiability.

    \thmitem{def:differentiability/strict}\mcite[33]{DontchevRockafellar2014}If there exists a continuous linear operator \( \Lambda \) such that
    \begin{equation}\label{def:differentiability/strict/condition}
      \lim_{\substack{y \to x_0 \\ z \to x_0}} \frac {\norm{f(y) - f(z) - \Lambda (y - z)}_Y} {\norm{y - z}_X} = 0,
    \end{equation}
    we say that \( f \) is \term{strictly differentiable} at \( x_0 \).
  \end{thmenum}
\end{definition}

\begin{remark}\label{rem:derivative_notation}
  The following are standard notations for derivatives (some of the comments are based on \cite[146]{Фихтенгольц1968Том2}):
  \begin{thmenum}
    \thmitem{rem:derivative_notation/lagrange} We already used \term{Lagrange's notation} \( f'(x_0) \) and \( f_G'(x_0) \) in \fullref{def:differentiability}. Brevity is the only benefit of this notation. It becomes convenient when the functions have no name is a burden for directional derivatives.

    The second and third derivatives of \( f \) at \( x_0 \) are denoted as \( f^{''}(x_0) \) and \( f^{'''}(x_0) \) and the \( n \)-th derivative of is denoted as \( f^{(n)} \).

    See \fullref{def:nonsmooth_derivatives} for variations of this notation.

    \thmitem{rem:derivative_notation/newton} Newton's notation is similar to that of Leibniz, but depends on placing dots on top of \( f \), e.g. \( \ddot{f}(x_0) \coloneqq f''(x_0) \). This is used in areas like mathematical physics, however it has not become standard in more pure areas of analysis.

    \thmitem{rem:derivative_notation/euler} We use \term{Euler's notation} \( Df(x_0) \coloneqq f'(x_0) \) for more complicated expressions, e.g. \fullref{thm:derivative_limit_exchange}. The main benefit of this notation is that is allows to express differentiation as an operator, similar to what we defined in \fullref{def:differentiability}. The directional derivative of \( f \) at \( x_0 \) in the direction \( h \) is denoted as \( D_h f(x_0) \). Iterated differentiation corresponds to the standard notation for group composition: the \( n \)-th derivative at \( x_0 \) is denoted as \( D^n f(x_0) \).

    We also use other letters in the superscripts like \( D^G f(x_0) \) for Gateaux derivatives, \( D^\circ f(x_0) \) for Clarke's generalized derivatives, etc.

    \thmitem{rem:derivative_notation/phelps} Some authors like \cite{Phelps1993} use a variation of Euler's notation with \( \partial \) instead of \( D \). For example, directional derivatives are introduced as \( \partial^+ f(x_0)(h) \) in \cite[lemma 1.2]{Phelps1993}. This is consistent with the standard notation for subdifferentials - see \fullref{def:subdifferentials}, however Euler's notation appears to be more widely adoped.

    \thmitem{rem:derivative_notation/leibniz} The Leibniz notation for the derivative \( f'(x_0) \) is
    \begin{equation*}
      \diff f x (x_0) \coloneqq D f(x_0).
    \end{equation*}

    This notation is used extensively in integral calculus, however it is often confusing when manipulating derivatives. The fraction notation is unjustified in anything, but trivial cases and the partial derivative notation
    \begin{equation*}
      \diffp f x (x_0) \coloneqq D_x f(x_0)
    \end{equation*}
    is even more confusing.

    Note also that this depends on the convention of having variable names.
  \end{thmenum}
\end{remark}

\begin{remark}\label{rem:gateaux_vs_frechet}
  We will compare Gateaux \hyperref[def:differentiability/gateaux]{differentiability} with Frechet \hyperref[def:differentiability/frechet]{differentiability}. Let \( X \) and \( Y \) be Banach spaces, let \( U \subseteq X \) be an open set and let \( f: U \to Y \) be an arbitrary function. Fix a point \( x_0 \in U \).

  The continuous linear operator \( \Lambda: X \to Y \) is a Gateaux derivative if, for every \( \varepsilon > 0 \) and every direction \( h \in X \) there exists \( \delta_G^h > 0 \) such that
  \begin{equation}\label{rem:gateaux_vs_frechet/gateaux}
    \norm{\frac {f(x_0 + th) - f(x_0)} t - \inprod \Lambda h}_Y < \varepsilon \quad\forall t \in (0, \delta_G^h).
  \end{equation}

  In order for \( \Lambda \) to be a Frechet derivative, for every \( \varepsilon > 0 \) there must exist a \( \delta_F > 0 \), so that
  \begin{equation*}
    \frac{\norm{f(x_0 + h) - f(x_0) - \inprod \Lambda h}_Y} {\norm{h}_X} < \varepsilon \quad\forall h \in B(0, \delta_F) \setminus \{ 0 \},
  \end{equation*}
  which can be restated as
  \begin{equation}\label{rem:gateaux_vs_frechet/frechet}
    \norm{\frac {f(x_0 + th) - f(x_0)} t - \inprod \Lambda h}_Y < \varepsilon \quad\forall t \in (0, \delta_F) \ \forall h \in S_X.
  \end{equation}

  By comparing \fullref{rem:gateaux_vs_frechet/gateaux} to \fullref{def:differentiability/frechet}, we conclude that \( f \) is Frechet differentiable at \( x_0 \) if \( \inf_{h \in S_X} \delta^h_G > 0 \), that is, if \( f \) is Gateaux differentiable and the convergence of the Gateaux derivative is uniform on \( h \in S_X \).

  In particular, Frechet differentiability implies Gateaux differentiability.
\end{remark}

\begin{definition}\label{def:function_regular_at_point}
  We say that a function is \term{regular} at a point if its derivative at that point is nonzero.
\end{definition}

\begin{theorem}[Chain rule]\label{thm:chain_rule}
  \todo{Prove.}
\end{theorem}

\subsection{Banach space interpolation}\label{subsec:banach_space_interpolation}

\begin{definition}\label{def:interpolated_topological_vector_space}\mcite[24]{BerghLofstrom1976}
  Let \( \BbbK \) be either the \hyperref[def:set_of_real_numbers]{field \( \BbbR \) of real numbers} or the \hyperref[def:set_of_real_numbers]{field \( \BbbC \) of complex numbers}.

  \begin{thmenum}
    \thmitem{def:interpolated_topological_vector_space/compatibility} We say that two \hyperref[def:topological_vector_space]{topological vector spaces} \( X_0 \) and \( X_1 \) are \term{compatible} if they can both be \hyperref[def:morphism_invertibility/left_cancellative]{embedded} \hyperref[def:global_continuity]{continuously} into a \hyperref[def:separation_axioms/T2]{Hausdorff} topological vector space \( \mscrU \), in which case we can regard them as subspaces of \( \mscrU \).

    In particular, both \( X_0 \) and \( X_1 \) are Hausdorff. We write \( \overline{X} \coloneqq (X_0, X_1) \).

    \thmitem{def:interpolated_topological_vector_space/intersection} Denote by
    \begin{equation*}
      \Delta \overline{X} \coloneqq X_0 \cap X_1
    \end{equation*}
    the \term{intersection} of \( X_0 \) and \( X_1 \) (when regarded as subspaces of \( \mscrU \)).

    \thmitem{def:interpolated_topological_vector_space/sum} Denote by
    \begin{equation*}
      \Sigma \overline{X} \coloneqq ( X_0 + X_1 )
    \end{equation*}
    the \term{sum} of \( X_0 \) and \( X_1 \). If \( x \in \Sigma \overline{X} \), then there exist (possibly nonunique) vectors \( x_0 \in X_0 \) and \( x_1 \in X_1 \) such that \( x = x_0 + x_1 \).

    \thmitem{def:interpolated_topological_vector_space/intermediate_space} Let \( \overline{X} \) be a pair of compatible spaces. We say that the space \( X \) is an \term{intermediate} space for \( \overline{X} \) if \( \Delta \overline{X} \subseteq X \subseteq \Sigma \overline{X} \) with continuous linear inclusions.

    \thmitem{def:interpolated_topological_vector_space/morphisms} We introduce \hyperref[def:category/morphisms]{morphisms} between two compatible pairs \( \overline{X} \) and \( \overline{Y} \) that are, strictly speaking, not \hyperref[def:function]{functions} between the pairs themselves. We define an \term{operator} \( T: \overline{X} \to \overline{Y} \) between compatible pairs to be a function \( T \) from \( \Sigma \overline{X} \) to \( \Sigma \overline{Y} \) that satisfies the additional conditions
    \begin{align*}
      T(X_0) \subseteq Y_0
      &&
      T(X_1) \subseteq Y_1.
    \end{align*}

    \thmitem{def:interpolated_topological_vector_space/category} If \( \cat{C} \) is a \hyperref[def:subcategory]{subcategory} of the category \hyperref[def:category_of_topological_vector_spaces]{\( \cat{TopVect}_{\BbbK} \)} of topological vector spaces. We define the category \( \cat{Interp}_{\cat{C}} \) as the product category \( \cat{TopVect}_{\BbbK} \times \cat{TopVect}_{\BbbK} \). More explicitly:
    \begin{refenum}
      \refitem{def:category/objects} The \hyperref[def:set]{class} of objects is the class of all pairs of \hyperref[def:interpolated_topological_vector_space/compatibility]{compatible spaces}.
      \refitem{def:category/morphisms} The morphisms between two compatible pairs are the \hyperref[def:interpolated_topological_vector_space/morphisms]{continuous linear operators} \( T: \overline{X} \to \overline{Y} \) between them.
      \refitem{def:category/composition} Composition of morphisms is the usual \hyperref[def:multi_valued_function/composition]{function composition} if we regard a morphism \( T: \overline{X} \to \overline{Y} \) as a function from \( \Sigma \overline{X} \) to \( \Sigma \overline{Y} \).
    \end{refenum}

    \thmitem{def:interpolated_topological_vector_space/interpolation_space} We say that the intermediate spaces \( X \) for \( \overline{X} \) and \( Y \) for \( \overline{Y} \) are a pair of \term{interpolation spaces} with respect to \( \overline{X} \) and \( \overline{Y} \) if, for any continuous linear \hyperref[def:interpolated_topological_vector_space/morphisms]{operator} \( T: \overline{X} \to \overline{Y} \) between the compatible pairs, we have \( T(X) \subseteq Y \).
  \end{thmenum}
\end{definition}

\begin{lemma}\label{thm:preordered_magma_max_distributivity}
  In a \hyperref[def:ordered_magma]{preordered magma} \( M \),
  \begin{equation}\label{eq:thm:preordered_magma_max_distributivity}
    \max \set{a b, c d} \leq \max \set{a, c} \cdot \max \set{b, d}.
  \end{equation}
\end{lemma}
\begin{proof}
  Since \( a \leq \max \set{a, c} \), then
  \begin{equation*}
    ab
    \leq
    \max \set{a, c} \cdot b
    \leq
    \max \set{a, c} \cdot \set{b, d}
  \end{equation*}

  Analogously, \( cd \leq \max \set{a, c} \cdot \set{b, d} \) and
  \begin{equation*}
    \max \set{a b, c d} \leq \max \set{a, c} \cdot \set{b, d}.
  \end{equation*}
\end{proof}

\begin{proposition}\label{def:banach_space_sum_and_intersection_norms}\mcite[24]{BerghLofstrom1976}
  Let \( X \coloneqq (X_0, X_1) \) be a \hyperref[def:interpolated_topological_vector_space/compatibility]{compatible pair} of \hyperref[def:banach_space]{Banach spaces}.

  \begin{thmenum}
    \thmitem{def:banach_space_sum_and_intersection_norms/intersection} The intersection \( \Delta \overline{X} = X_0 \cap X_1 \) is a Banach space with norm
    \begin{equation}\label{eq:def:banach_space_sum_and_intersection_norms/intersection}
      \norm{x}_{\Delta \overline{X}} \coloneqq \max \set{ \norm{x}_{X_0}, \norm{x}_{X_1} }.
    \end{equation}

    \thmitem{def:banach_space_sum_and_intersection_norms/sum} The sum \( \Sigma \overline{X} = X_0 + X_1 \) is a Banach space with norm
    \begin{equation}\label{eq:def:banach_space_sum_and_intersection_norms/sum}
      \norm{x}_{\Delta \overline{X}} \coloneqq \inf \set{ \norm{x_0}_{X_0} + \norm{x_1}_{X_1} : x_0 + x_1 = x }.
    \end{equation}
  \end{thmenum}
\end{proposition}
\begin{proof}
  \SubProofOf{def:banach_space_sum_and_intersection_norms/intersection} We will first show that \( \norm{x}_{\Delta \overline{X}} \) is indeed a norm.
  \begin{refenum}
    \refitem{def:norm/N1} We have
    \begin{equation}\label{eq:def:banach_space_sum_and_intersection_norms/intersection/zero}
      \norm{x}_{\Delta \overline{X}} = \max \set{ \norm{x}_{X_0}, \norm{x}_{X_1} } = 0 \T{if and only if} \norm{x}_{X_0} = \norm{x}_{X_1} = 0.
    \end{equation}

    Clearly \( 0 \) belongs to both \( X_0 \) and \( X_1 \) hence to their intersection. Therefore, \eqref{eq:def:banach_space_sum_and_intersection_norms/intersection/zero} is satisfied if and only if \( x = 0 \).

    \refitem{def:norm/N2} Absolute homogeneity follows from
    \begin{equation*}
      \norm{tx}_{\Delta \overline{X}}
      =
      \max \set{ \norm{tx}_{X_0}, \norm{tx}_{X_1} }
      \reloset {\ref{def:norm/N2}} =
      \abs{t} \max \set{ \norm{x}_{X_0}, \norm{x}_{X_1} }
      =
      \abs{t} \norm{x}_{\Delta \overline{X}}.
    \end{equation*}

    \refitem{def:norm/N3} Subadditivity follows from
    \begin{balign*}
      \norm{x + y}_{\Delta \overline{X}}
      &=
      \max \set{ \norm{x + y}_{X_0}, \norm{x + y}_{X_1} }
      \reloset {\ref{def:norm/N3}} \leq \\ &\leq
      \max \set{ \norm{x}_{X_0} + \norm{y}_{X_0}, \norm{x}_{X_1} + \norm{y}_{X_1} }
      \reloset {\ref{eq:thm:preordered_magma_max_distributivity}} \leq \\ &\leq
      \max \set{ \norm{x}_{X_0}, \norm{x}_{X_1} } + \max \set{ \norm{y}_{X_0}, \norm{y}_{X_1} }
      = \\ &=
      \norm{x}_{\Delta \overline{X}} + \norm{y}_{\Delta \overline{X}}.
    \end{balign*}
  \end{refenum}

  We will now show the completeness of \( \norm{\cdot}_{\Delta \overline{X}} \) directly. Let \( \{ x_k \}_{k=1}^\infty \subseteq \Delta \overline{X} \) be a \hyperref[def:fundamental_net]{fundamental sequence}. Both \( X_0 \) and \( X_1 \) are complete, therefore \( \{ x_k \}_{k=1}^\infty \) converges to the same value. Both are subspaces of \( \mscrU \), therefore the limit of the sequence is the same in both. In particular, it belongs to the intersection \( \Delta \overline{X} \).

  Denote the limit of \( \{ x_k \}_{k=1}^\infty \) by \( x \). Let \( \varepsilon > 0 \) and let \( k_0 \) be an index such that both \( \norm{x_k - \xi_0}_{X_0} < \varepsilon \) and \( \norm{x_k - \xi_1}_{X_1} < \varepsilon \) whenever \( k \geq k_0 \). Then, for any \( k \geq k_0 \),
  \begin{equation*}
    \norm{x_k - x}_{\Delta \overline{X}}
    =
    \max\set{\norm{x_k - x}_{X_0}, \norm{x_k - x}_{X_1}}
    <
    \varepsilon.
  \end{equation*}

  Therefore, the sequence \( \{ x_k \}_{k=1}^\infty \) converges to \( x_0 \) in \( \Delta \overline{X} \).

  \SubProofOf{def:banach_space_sum_and_intersection_norms/sum} Again, we will first show that \( \norm{x}_{\Sigma \overline{X}} \) is indeed a norm.
  \begin{refenum}
    \refitem{def:norm/N1} Analogously to \ref{def:banach_space_sum_and_intersection_norms/sum},
    \begin{equation*}
      \norm{x}_{\Sigma \overline{X}} = \inf \set{ \norm{x_0}_{X_0} + \norm{x_1}_{X_1} : x_0 + x_1 = x } = 0
    \end{equation*}
    if and only if
    \begin{equation*}
      \norm{x}_{X_0} = \norm{x}_{X_1} = 0.
    \end{equation*}

    \refitem{def:norm/N2} Absolute homogeneity follows from
    \begin{equation*}
      \norm{tx}_{\Sigma \overline{X}}
      =
      \inf \set{ \norm{tx_0}_{X_0} + \norm{tx_1}_{X_1} | x_0 + x_1 = x }
      \reloset {\ref{def:norm/N2}} =
      \abs{t} \norm{x}_{\Sigma \overline{X}}.
    \end{equation*}

    \refitem{def:norm/N3} Subadditivity follows from
    \begin{align*}
      &\phantom{{}={}}
      \norm{x + y}_{\Sigma \overline{X}}
      = \\ &=
      \inf \set{ \left( \norm{x_0}_{X_0} + \norm{x_1}_{X_1} \right) + \left( \norm{y_0}_{X_0} + \norm{y_1}_{X_1} \right) | \substack{\textstyle{x_0 + x_1 = x} \\ \textstyle{y_0 + y_1 = y}} }
      \leq \\ &\leq
      \norm{x}_{\Sigma \overline{X}} + \norm{y}_{\Sigma \overline{X}}.
    \end{align*}
  \end{refenum}

  It remains to prove the completeness of \( \norm{\cdot}_{\Sigma \overline{X}} \). Let \( \{ x_0^{(k)} + x_1^{(k)} \}_{k=1}^\infty \subseteq \Sigma \overline{X} \) be a \hyperref[def:fundamental_net]{fundamental sequence}. Fix \( \varepsilon > 0 \). Then there exists an index \( m_0 \) such that \( k, m \geq k_0 \) implies
  \begin{equation*}
    \norm{x_0^{(k)} + x_1^{(k)} - x_0^{(m)} + x_1^{(m)}}_{\Sigma \overline{X}} < \varepsilon.
  \end{equation*}

  But
  \begin{equation*}
    \norm{x_0^{(k)} - x_0^{(m)}}_{X_0}
    \leq
    \norm{\left(x_0^{(k)} + x_1^{(k)} \right) - \left( x_0^{(m)} + x_1^{(m)} \right)}_{\Sigma \overline{X}}
    <
    \varepsilon,
  \end{equation*}
  hence the sequence \( \{ x_0^{(k)} \}_{k=1}^\infty \) is fundamental. Since \( X_0 \) is complete, this sequence has a limit, which we will denote by \( \xi_0 \). We define \( \xi_1 \) analogously.

  With the same \( \varepsilon \), denote by \( k_0 \) an index such that both \( \norm{x_0^{(k)} - \xi_0}_{X_0} < \tfrac \varepsilon 2 \) and \( \norm{x_1^{(k)} - \xi_1}_{X_1} < \tfrac \varepsilon 2 \) whenever \( k \geq k_0 \).

  Then
  \begin{balign*}
    &\phantom{{}={}}
    \norm{\left( x_0^{(k)} + x_1^{(k)} \right) - \left( \xi_0 + \xi_1 \right)}_{\Sigma \overline{X}}
    = \\ &=
    \inf \set{ \norm{x_0}_{X_0} + \norm{x_1}_{X_1} : x_0 + x_1 = \left( x_0^{(k)} + x_1^{(k)} \right) - \left( \xi_0 + \xi_1 \right) }
    \leq \\ &\leq
    \norm{x_0^{(k)} - \xi_0}_{X_0} + \norm{x_1^{(k)} - \xi_1}_{X_1}
    <
    \tfrac \varepsilon 2 + \tfrac \varepsilon 2
    =
    \varepsilon.
  \end{balign*}

  Therefore, \( \xi_0 + \xi_1 \) is the limit of the sequence \( \{ x_0^{(k)} + x_1^{(k)} \}_{k=1}^\infty \subseteq \Sigma \overline{X} \) in \( \Sigma \overline{X} \).
\end{proof}

\begin{example}\label{thm:lp_interpolation_spaces/definition}
  The spaces \( L^p(\BbbR) \) are interpolation spaces for the pair \( (L^1(\BbbR), L^\infty(\BbbR)) \). The pair is compatible because both are subspaces of the space \( S(\BbbR) \) of all Lebesgue-measurable real function with metric
  \begin{equation*}
    \rho(f, g) \coloneqq \int_{\BbbR} \frac {\abs{f(x) - g(x)}} {1 + \abs(f(x) - g(x))} d\lambda.
  \end{equation*}
\end{example}

\begin{definition}\label{def:lebesgue_space}\cite[6]{BerghLofstrom1976}
  Let \( \mu: U \to [0, \infty] \) be a positive measure and \( p \) be a positive real number. The \term{Lebesgue space} \( L_p \) is defined as the set of bounded functions \( f: U \mapsto \BbbK \) such that the norm
  \begin{equation*}
    \norm{f}_{L_p} \coloneqq \begin{dcases}
      \parens[\Big]{\int_U \abs{f(t)}^p dt}^{1/p}, &0 < p < \infty \\
      \ess\sup_{t \in U} \abs{f(t)} , &p = \infty
    \end{dcases}
  \end{equation*}
\end{definition}

\begin{theorem}[The Riesz-Thorin interpolation theorem]\label{thm:riesz_thorin}\mcite[24]{BerghLofstrom1976}
  Fix two measure spaces \( (U, \mu) \) and \( (V, \nu) \). Let \( T: S(U, \mu) \to S(V, \nu) \) be a continuous linear map between the corresponding spaces of measurable functions.

  Suppose that for some real numbers \( p_0, p_1, q_0, q_1 \geq 1 \) we have
  \begin{equation*}
    T(L^{p_j}(U, \mu)) \subseteq T(L^{q_j}(V, \nu)), j = 0, 1.
  \end{equation*}

  Additionally, let \( \theta \in (0, 1) \) and
  \begin{align*}
    \frac 1 p = \frac {1 - \theta} {p_0} + \frac {\theta} {p_1}
    &&
    \frac 1 q = \frac {1 - \theta} {q_0} + \frac {\theta} {q_1}.
  \end{align*}

  Then
  \begin{equation*}
    T(L^p(U, \mu)) \subseteq L^q(V, \nu)
  \end{equation*}
  and
  \begin{equation*}
    \norm{T}_{\hom(L^p, L^q)} \leq \norm{T}_{\hom(L^{p_0}, L^{q_0})}^{1 - \theta} \norm{T}_{\hom(L^{p_1}, L^{q_1})}^\theta.
  \end{equation*}
\end{theorem}

\begin{definition}\label{def:distribution_function}\cite[6]{BerghLofstrom1976}
  Let \( \mu: U \to [0, \infty] \) be a positive measure and \( p \) be a positive real number.

  \begin{thmenum}
    \thmitem{def:distribution_function/distribution_function} Given a scalar-valued function \( f: U \mapsto \BbbK \), we define its \term{distribution function} as
    \begin{align*}
      &m_f: [0, \infty] \to \BbbK \\
      &m_f(\sigma) \coloneqq \mu(\set{ x :  > \sigma }).
    \end{align*}

    \thmitem{def:distribution_function/rearrangement} We define the \term{decreasing rearrangement} of \( f \) as
    \begin{equation*}
      f^*(t) \coloneqq \inf\set{ \sigma : m_f(\sigma) \leq t }.
    \end{equation*}

    \thmitem{def:distribution_function/lorenz_space} The \( (p, q)-\)Lorenz space, for potentially infinite positive \( q > 0 \), is the set of functions \( f: U \mapsto \BbbK \) for which the quasinorm
    \begin{equation*}
      \norm{f}_{L_{p,q}} \coloneqq \begin{dcases}
        \parens[\Big]{\int_0^\infty \parens[\Big]{\frac {f^*(\tau)} {\tau^p}}^q \frac {d t} t}^{\frac 1 q}, &1 \leq q < \infty \\
        \ess\sup\parens[\Big]{\frac {f^*(t)} {t^p}}, &q = \infty
      \end{dcases}
    \end{equation*}
    is finite.

    In particular, when \( q = \infty \), we use the notation
    \begin{equation*}
      \norm{f}_{L^{p*}} \coloneqq \parens[\Big]{p \int_0^\infty \sigma^p m_f(\sigma) \frac {d \sigma} \sigma}^{\frac 1 p}
    \end{equation*}
  \end{thmenum}
\end{definition}

\begin{theorem}[The Marcinkiewicz interpolation theorem]
  Fix two measure spaces \( (U, \mu) \) and \( (V, \nu) \). Let \( T: S(U, \mu) \to S(V, \nu) \) be a continuous linear map between the corresponding spaces of measurable functions.

  Suppose that for some real numbers \( p_0, p_1, q_0, q_1 \geq 1 \) we have
  \begin{equation*}
    T(L^{p_j}(U, \mu)) \subseteq T(L^{q_j *}(V, \nu)), j = 0, 1.
  \end{equation*}

  Additionally, let \( \theta \in (0, 1) \) and
  \begin{align*}
    \frac 1 p = \frac {1 - \theta} {p_0} + \frac {\theta} {p_1}
    &&
    \frac 1 q = \frac {1 - \theta} {q_0} + \frac {\theta} {q_1}.
  \end{align*}

  Then, if \( p \leq q \),
  \begin{equation*}
    T(L^p(U, \mu)) \subseteq L^q(V, \nu)
  \end{equation*}
  and
  \begin{equation*}
    \norm{T}_{\hom(L^p, L^q)} \leq C_\theta \norm{T}_{\hom(L^{p_0}, L^{q_0*})}^{1 - \theta} \norm{T}_{\hom(L^{p_1}, L^{q_1*})}^\theta
  \end{equation*}
  for some constant \( C_\theta \).
\end{theorem}

\begin{definition}\label{def:banach_interpolation_space_exponent}\mcite[27]{BerghLofstrom1976}
  Let \( \overline{X} \coloneqq ( X_0, X_1 ) \) and \( \overline{Y} \coloneqq ( Y_0, Y_1 ) \) be compatible pairs of Banach spaces. If \( X \) and \( Y \) are a pair of interpolation spaces and, additionally, the inequality
  \begin{equation}\label{da:def:banach_interpolation_space_exponent}
    \norm{T}_{\hom(X, Y)} \leq C \norm{T}_{\hom(X_0, Y_0)}^{1-\theta} \cdot \norm{T}_{\hom(X_1, Y_1)}^{\theta}
  \end{equation}
  holds for some constant \( C > 0 \) and \( \theta \in [0, 1] \), we say that the pair \( (X, Y) \) are \term{interpolation spaces of exponent} \( \theta \).

  If, additionally, \( C = 1 \), we say that \( (X, Y) \) is an \term{exact pair} of interpolation spaces.
\end{definition}

\begin{definition}\label{def:k_functional}\mcite[38]{BerghLofstrom1976}
  Let \( \overline{X} \coloneqq ( X_0, X_1 ) \) be a compatible pair of Banach spaces. Instead of the norm \( \norm{\cdot}_{X_1} \) in \( X_1 \), we can consider \hyperref[def:equivalent_metrics]{equivalent norms} of the type \( t\norm{\cdot}_{X_1} \) for \( t \geq 0 \). Furthermore, we can also introduce equivalent norms in \( \Sigma \overline{X} \) via the \term{\( K \)-functional}
  \begin{equation}\label{eq:def:k_functional}
    \begin{aligned}
      &K: (0, \infty) \times {\Sigma \overline{X}} \\
      &K(t, x) \coloneqq \inf \set{ \norm{x_0}_{X_0} + t\norm{x_1}_{X_1} : x_0 + x_1 = x }.
    \end{aligned}
  \end{equation}

  See \fullref{def:k_functional_properties/equivalent_norm} for a proof that \( x \mapsto K(t, x) \) for a fixed \( t \geq 0 \) is an equivalent norm in the sum \( \Sigma \overline{X} \).
\end{definition}

\begin{proposition}\label{def:k_functional_properties}\mcite[38]{BerghLofstrom1976}
  The \hyperref[def:k_functional]{\( K \)-functional} has the following basic properties:

  \begin{thmenum}
    \thmitem{def:k_functional_properties/basic} For any fixed \( x \in \Sigma \overline{X} \), the function \( t \mapsto K(t, x) \) is positive, \hyperref[def:partially_ordered_set/homomorphism]{monotone} and \hyperref[def:convex_functions]{concave}.

    \thmitem{def:k_functional_properties/inequality} For positive real numbers \( t, s > 0 \), we have the following inequality:
    \begin{equation}\label{eq:def:k_functional_properties/inequality}
      K(t, x) \leq \max\set{1, \frac t s} K(s, x).
    \end{equation}

    \thmitem{def:k_functional_properties/equivalent_norm} For any fixed \( t > 0 \), the function \( x \mapsto K(t, x) \) is an \hyperref[def:equivalent_metrics]{equivalent norm} in the sum \( \Sigma \overline{X} \).
  \end{thmenum}
\end{proposition}
\begin{proof}
  \SubProofOf{def:k_functional_properties/basic} That \( t \mapsto K(t, x) \) is positive is a slight generalization of \fullref{def:norm/N1}, which can be proved as in \fullref{def:banach_space_sum_and_intersection_norms/sum}.

  Monotonicity follows from the monotonicity of the infimum.

  To see that \( t \mapsto K(t, x) \) is concave, fix \( x \), \( \lambda \in [0, 1] \) and \( t, s > 0 \). We have
  \begin{align*}
    &\phantom{{}={}}
    K(\lambda t + (1 - \lambda) s, x)
    = \\ &=
    \inf \set{ \norm{x_0}_{X_0} + (\lambda t + (1 - \lambda) s)\norm{x_1}_{X_1} | x_0 + x_1 = x }
    = \\ &=
    \inf \set{ \lambda \left(\norm{x_0}_{X_0} + t \norm{x_1}_{X_1} \right) + (1 - \lambda) \left(\norm{x_0}_{X_0} + s \norm{x_1}_{X_1} \right) | x_0 + x_1 = x }
    \geq \\ &\geq
    \lambda K(t, x) + (1 - \lambda) K(s, x).
  \end{align*}

  \SubProofOf{def:k_functional_properties/inequality} Fix positive real numbers \( t, s > 0 \).
  \begin{itemize}
    \item If \( t \leq s \), by monotonicity we have
    \begin{equation}\label{eq:def:k_functional_properties/inequality/monotonicity}
      K(t, x) \leq K(s, x)
    \end{equation}

    \item If \( t > s \), we use concavity with
    \begin{equation*}
      s = \frac s t t + \left(1 - \frac s t \right) 0
    \end{equation*}
    to obtain
    \begin{equation*}
      K(s, x) \geq \frac s t K(t, x) + \left(1 - \frac s t \right) K(0, x).
    \end{equation*}

    By positivity of \( K \), we have \( K(t, x) = 0 \) if and only if \( t = 0 \), hence
    \begin{equation}\label{eq:def:k_functional_properties/inequality/concavity}
      K(t, x) \leq \frac t s K(s, x).
    \end{equation}
  \end{itemize}

  Combining \eqref{eq:def:k_functional_properties/inequality/monotonicity} and \eqref{eq:def:k_functional_properties/inequality/concavity}, we obtain \eqref{eq:def:k_functional_properties/inequality}.

  \SubProofOf{def:k_functional_properties/equivalent_norm} That \( x \mapsto K(t, x) \) for a fixed \( t > 0 \) is a slight generalization of the proof in \fullref{def:banach_space_sum_and_intersection_norms/sum}.

  That the norms \( \norm{\cdot}_{\Sigma \overline{X}} \) and \( K(t, \cdot) \) are equivalent follows from \eqref{eq:def:k_functional_properties/inequality} with \( s = 1 \) for the upper bound and \( t = 1, s = t \) for the lower bound. That is,
  \begin{equation*}
    \min\set{1, t} \underbrace{K(1, x)}_{\norm{x}_{\Sigma \overline{X}}} \leq K(t, x) \leq \max\set{1, t} \underbrace{K(1, x)}_{\norm{x}_{\Sigma \overline{X}}}.
  \end{equation*}
\end{proof}

\begin{example}\label{thm:lp_interpolation_spaces/k_functional}
  The \hyperref[def:k_functional]{\( K \)-functional} for the pair \( (L_1(\BbbR), L_\infty(\BbbR)) \) from \fullref{thm:lp_interpolation_spaces/definition} is
  \begin{equation*}
    K(t, f) \coloneqq \int_0^t f^*(\tau) d\tau.
  \end{equation*}
\end{example}

\begin{definition}\label{def:lorenz_quasinorm}
  For \( \theta \in \BbbR \), \( q \in (0, \infty] \) and nonnegative functions \( g: [0, \infty) \to [0, \infty] \) we define
  \begin{equation}\label{eq:def:lorenz_quasinorm}
    \Phi_{\theta,q}(g) \coloneqq \begin{dcases}
      \left( \int_0^\infty \left( \frac {g(\tau)} {\tau^\theta} \right)^q \frac {d\tau} \tau \right)^{\tfrac 1 q}, &0 < q < \infty \\
      \ess\sup_{t \geq 0} \left( \frac {g(t)} {t^\theta} \right),                                                &q = \infty
    \end{dcases}
  \end{equation}
  and
  \begin{equation}\label{eq:def:lorenz_quasinorm/gamma}
    \gamma_{\theta,q} \coloneqq \Phi_{\theta,q}(\min(t, 1)).
  \end{equation}
\end{definition}

\begin{proposition}\label{thm:def:lorenz_quasinorm/properties}
  The function \hyperref[def:lorenz_quasinorm]{\( \Phi_{\theta,q} \)} has the following basic properties:

  \begin{thmenum}
    \thmitem{thm:def:lorenz_quasinorm/properties/reciprocal} For \( s > 0 \) and \( h(t) \coloneqq g(\tfrac t s) \) we have
    \begin{equation}\label{eq:thm:def:lorenz_quasinorm/properties/reciprocal}
      \Phi_{\theta,q}(h) = \frac 1 {s^{\theta}} \Phi_{\theta,q}(g).
    \end{equation}

    \thmitem{thm:def:lorenz_quasinorm/properties/gamma} For finite \( q \) we have
    \begin{equation}\label{eq:thm:def:lorenz_quasinorm/properties/gamma}
      \gamma_{\theta,q} = \left( \frac 1 {q \theta (1 - \theta)} \right)^{\tfrac 1 q}.
    \end{equation}
  \end{thmenum}
\end{proposition}
\begin{proof}
  \SubProofOf{thm:def:lorenz_quasinorm/properties/reciprocal} The case \( q = \infty \) is obvious. For \( 0 < q < \infty \), we have
  \begin{balign*}
    \Phi_{\theta,q}(h)
    &=
    \left( \int_0^\infty \left( \frac {h(\tau)} {\tau^\theta} \right)^q \frac {d\tau} \tau \right)^{\tfrac 1 q}
    = \\ &=
    \left( \frac 1 {s^{\theta q}} \int_0^\infty \left( \frac {g(\tfrac \tau s)} {\left(\tfrac \tau s \right)^\theta} \right)^q \frac {d{\tfrac \tau s}} {\tfrac \tau s} \right)^{\tfrac 1 q}
    = \\ &=
    \frac 1 {s^{\theta}} \Phi_{\theta,q}(g).
  \end{balign*}

  \SubProofOf{thm:def:lorenz_quasinorm/properties/gamma} We can raise \( \gamma_{\theta,q} \) to the \( q \)-th power for brevity of notation:
  \begin{balign*}
    \gamma_{\theta,q}^q
    &=
    \Phi_{\theta,q}(\min(t, 1))^q
    = \\ &=
    \int_0^1 \left( \frac {\tau} {\tau^\theta} \right)^q \frac {d\tau} \tau + \int_1^\infty \left( \frac {1} {\tau^\theta} \right)^q \frac {d\tau} \tau
    = \\ &=
    \int_0^1 \tau^{(1 - \theta) q - 1} d\tau + \int_1^\infty \tau^{-\theta q - 1} d\tau
    = \\ &=
    \frac {1 - 0} {(1 - \theta) q} + \frac {\lim_{\tau \to \infty} \tau^{-\theta q} - 1} {-\theta q}
    = \\ &=
    \frac 1 {(1 - \theta) q} - \frac 1 {-\theta q}
    = \\ &=
    \frac {-\theta q - (1 - \theta) q} {(1 - \theta) (-\theta) q^2}
    = \\ &=
    \frac 1 {(1 - \theta) \theta q}
  \end{balign*}
\end{proof}

\begin{definition}\label{def:k_functional_interpolation_space}\mcite[40]{BerghLofstrom1976}
  Let \( \overline{X} \coloneqq ( X_0, X_1 ) \) be a compatible pair of Banach spaces.

  For \( \theta \in (0, \infty) \), \( q \in (0, \infty] \), we introduce the following norm:
  \begin{equation}\label{eq:def:k_functional_interpolation_space/norm}
    \norm{x}_{\theta,q,K} \coloneqq \Phi_{\theta,q}(K(t, x)).
  \end{equation}

  The subspace of \( \Sigma\overline{X} \) for which this norm is finite is denoted by either
  \begin{align*}
    K_{\theta,q}(\overline{X})
    &&
    X_{\theta,q,K}.
  \end{align*}
\end{definition}

\begin{theorem}\label{thm:k_functional_interpolation}\mcite[thm. 3.1.2]{BerghLofstrom1976}
  Let \( \theta \in (0, 1) \) and \( q \in (0, \infty] \). The space \( X_{\theta,q,K} \) defined in \fullref{eq:def:k_functional_interpolation_space/norm} is an \hyperref[def:banach_interpolation_space_exponent]{exact interpolation space} of exponent \( \theta \). Furthermore,
  \begin{equation}\label{eq:thm:k_functional_interpolation/inequality}
    K(s, x) \leq (\gamma_{\theta,q})^{-1} s^\theta \norm{x}_{\theta,q,K}.
  \end{equation}
\end{theorem}
\begin{proof}
  Note that \( K(s, x) \) is a norm on \( \Sigma \overline{X} \) by \fullref{def:k_functional_properties/equivalent_norm}. Therefore, \( \norm{\cdot}_{\theta,q,K} \), the composition of \( K(s, x) \) with the \hyperref[def:lorenz_quasinorm]{Lorenz quasinorm} \( \Phi_{\theta,q} \), is a norm.

  We denote by \( X_{\theta,q,K} \) the space consisting of all vectors from \( \Sigma \overline{X} \) where the norm \eqref{eq:def:k_functional_interpolation_space/norm} is finite.

  From \eqref{eq:def:k_functional_properties/inequality} it follows that
  \begin{equation*}
    \min(1, \tfrac t s) K(s, x) \leq K(t, x)
  \end{equation*}
  and hence
  \begin{equation*}
    \underbrace{\Phi_{\theta,q}}_{\text{depends on } t} \parens[\Big]{ \min(1, \tfrac t s) K(s, x) } \leq \underbrace{\Phi_{\theta,q}(K(s, x))}_{\text{norm in } X_{\theta,q,K}}.
  \end{equation*}

  By \eqref{eq:thm:def:lorenz_quasinorm/properties/reciprocal}, we have
  \begin{equation*}
    \Phi_{\theta,q}(\min(1, \tfrac t s)) = \tfrac 1 {s^\theta} \underbrace{\Phi_{\theta,q}(\min(1, t))}_{\hyperref[eq:def:lorenz_quasinorm/gamma]{\gamma_{\theta,q}}},
  \end{equation*}
  and \eqref{eq:thm:k_functional_interpolation/inequality} follows.

  It remains to show that \( X_{\theta,q,K} \) is an exact interpolation space of exponent \( \theta \).

  Note that \( K(1, x) = \norm{x}_{\Sigma \overline{X}} \) and thus \eqref{eq:thm:k_functional_interpolation/inequality} with \( s = 1 \) implies that
  \begin{equation*}
    \gamma_{\theta,q} \norm{x}_{\Sigma \overline{X}} \leq \norm{x}_{\theta,q,K},
  \end{equation*}
  which shows that \( X_{\theta,q,K} \) can be embedded continuously into \( \Sigma \overline{X} \).

  On the other hand, for \( x \in \Delta \overline{X} \) we have
  \begin{equation*}
    K(t, x) \leq \norm{x} \leq \norm{x}_{\Delta \overline{X}} \T{since} x = x + 0
  \end{equation*}
  and
  \begin{equation*}
    K(t, x) \leq \norm{x} \leq t \norm{x}_{\Delta \overline{X}} \T{since} x = 0 + x.
  \end{equation*}

  Therefore,
  \begin{equation*}
    K(t, x) \leq \min(1, t) \norm{x}_{\Delta \overline{X}},
  \end{equation*}
  which after applying \( \Phi_{\theta,q} \) becomes
  \begin{equation*}
    \norm{x}_{\theta,q,K} \leq \gamma_{\theta,q} \norm{x}_{\Delta \overline{X}}.
  \end{equation*}

  Hence, we have the chain of continuous linear inclusions of Banach spaces
  \begin{equation*}
    \Delta \overline{X} \subseteq X \subseteq \Sigma \overline{X}.
  \end{equation*}

  Finally, to show that \( X \) is an interpolation space of exponent \( \theta \), fix a linear operator \( T: \overline{X} \mapsto \overline{CY} \) between compatible pairs and let \( Y \) be an intermediate space for \( \overline{CY} \).

  Then
  \begin{align*}
    K(t, Tx)_{\overline{CY}}
    &\leq
    \inf \set{ \norm{y_0}_{Y_0} + t \norm{y_1}_{Y_1} : y_0 + y_1 = Tx }
    \leq \\ &\leq
    \inf \set{ \norm{T}_{\hom(X_0, Y_0)} \norm{x_0}_{X_0} + t \norm{T}_{\hom(X_1, Y_1)} \norm{x_1}_{Y_1} : x_0 + x_1 = x }
    = \\ &=
    \norm{T}_{\hom(X_0, Y_0)} K\parens[\Bigg]{\frac {\norm{T}_{\hom(X_1, Y_1)}} {\norm{T}_{\hom(X_0, Y_0)}} t, x}.
  \end{align*}

  By applying \( \Phi_{\theta,q} \) to both sides and using \eqref{eq:thm:def:lorenz_quasinorm/properties/reciprocal}, we obtain
  \begin{equation*}
    \norm{Tx}_{\overline{Y}_{\theta,q,K}}
    \leq
    {\norm{T}_{\hom(X_1, Y_1)}}^{1 - \theta} {\norm{T}_{\hom(X_0, Y_0)}}^{\theta} \norm{x}_{\overline{Y}_{\theta,q,K}}.
  \end{equation*}

  Thus, \( X \) satisfies \fullref{def:banach_interpolation_space_exponent} for being an exact interpolating space with exponent \( \theta \).
\end{proof}

\begin{definition}\label{def:discrete_k_interpolation_space}
  For positive numbers \( \theta \) and \( q \), we denote by \( \lambda^{\theta,q} \) the set of all doubly-infinite real sequences \( \{ x_k \}_{k=-\infty}^\infty \) such that the norm
  \begin{equation}\label{eq:def:discrete_k_interpolation_space}
    \norm{\{ x_k \}_{k=-\infty}^\infty}_{\lambda^{\theta,q}} \coloneqq \left( \sum_{k=-\infty}^\infty \left( \frac {\abs{x_k}} {2^{k\theta}} \right)^q \right)^{\tfrac 1 q}
  \end{equation}
  is finite.
\end{definition}

\begin{theorem}\label{thm:discrete_k_interpolation}\mcite[lemma 3.1.3]{BerghLofstrom1976}
  The vector \( x \in \Sigma\overline{X} \) belongs to \hyperref[def:k_functional_interpolation_space]{\( X_{\theta,q,K} \)} if and only if the sequence \( \{ x_k \}_{k=-\infty}^\infty \) defined as
  \begin{equation}\label{eq:thm:discrete_k_interpolation/sequence}
    x_k \coloneqq K(2^k, x)
  \end{equation}
  belongs to \hyperref[def:discrete_k_interpolation_space]{\( \lambda^{\theta,q} \)}.

  Furthermore, for any integer \( k \) the following inequalities hold:
  \begin{equation}\label{eq:thm:discrete_k_interpolation/inequalities}
    \frac 1 {2^\theta} \ln 2 \norm{x_k}_{\lambda^{\theta,q}}
    \leq
    \norm{x}_{\theta,q,K}
    \leq
    2 \cdot \ln 2 \norm{x_k}_{\lambda^{\theta,q}}.
  \end{equation}
\end{theorem}
\begin{proof}
  We have
  \begin{equation*}
    \norm{x}_{\theta,q,K}^q
    =
    \int_0^\infty \left( \frac {K(\tau, x)} {\tau^\theta} \right)^q \frac {d\tau} \tau
    =
    \sum_{k=-\infty}^\infty \int_{2^k}^{2^{k+1}} \left( \frac {K(\tau, x)} {\tau^\theta} \right)^q \frac {d\tau} \tau.
  \end{equation*}

  By \fullref{def:k_functional_properties/inequality}, for each integer \( k \),
  \begin{equation*}
    K(2^k, x) \leq 2 K(2^k, x).
  \end{equation*}

  By the \hyperref[def:k_functional_properties/basic]{monotonicity} of \( K \), for \( t \in [2^k, 2^{k+1}] \) we have
  \begin{equation*}
    K(2^k, x) \leq K(t, x) \leq 2 K(2^k, x).
  \end{equation*}

  Denote \( x_k \coloneqq K(2^k, x) \). For \( 2^k \leq t \leq 2^{k+1} \) we have
  \begin{equation*}
    \frac{x_k}{2^{(k+1)\theta}} \leq \frac{K(t, x)}{t^\theta} \leq 2 \frac{x_k}{2^{k\theta}}
  \end{equation*}

  Therefore,
  \begin{align*}
    \norm{x}_{\theta,q,K}^q
    &=
    \sum_{k=-\infty}^\infty \int_{2^k}^{2^{k+1}} \left( \frac {K(\tau, x)} {\tau^\theta} \right)^q \frac {d\tau} \tau
    \leq \\ &\leq
    2^q \sum_{k=-\infty}^\infty \left(\frac{x_k}{2^{k\theta}} \right)^q \cdot \ln \tau \mid_{\tau=2^k}^{2^{k+1}}
    = \\ &=
    \ln 2 \cdot 2^q \sum_{k=-\infty}^\infty \left(\frac{x_k}{2^{k\theta}} \right)^q
    = \\ &=
    2^q \ln 2 \norm{\{ x_k \}_{k=-\infty}^\infty}_{\lambda^{\theta,q}}^q
  \end{align*}
  and similarly for the lower bound.
\end{proof}

\begin{definition}\label{def:e_functional}\mcite[174]{BerghLofstrom1976}
  Let \( \overline{X} = (X_0, X_1) \) be a compatible pair of Banach spaces. Let \( x \in \Sigma \overline{X} \). Put
  \begin{equation}\label{eq:def:e_functional}
    \begin{aligned}
      &E: (0, \infty) \times {\Sigma \overline{X}} \\
      &E(t, x) \coloneqq \inf \set{ \norm{x - x_0}_{X_1} : \norm{x_0}_{X_0} \leq t }.
    \end{aligned}
  \end{equation}
\end{definition}

\begin{proposition}\label{thm:def:e_functional/properties}\mcite[lemma 7.1.3]{BerghLofstrom1976}
  When \( \overline{X} \) are quasi-Banach spaces, the \hyperref[def:e_functional]{\( E \)-functional} has the following basic properties:

  \begin{thmenum}
    \thmitem{def:k_functional_properties/decreasing} For fixed \( x \in \Sigma\overline{X} \), the function \( t \mapsto E(t, x) \) is decreasing.

    \thmitem{def:k_functional_properties/subaditive} For \( \varepsilon \in (0, 1) \), we have
    \begin{equation*}
      E(t, x + y) \leq E(\varepsilon t, x) + E((1 + \varepsilon) t, y).
    \end{equation*}

    \thmitem{def:k_functional_properties/positive} \( x = 0 \) if and only if \( E(t, x) = 0 \) for all \( t > 0 \).

    \thmitem{def:k_functional_properties/k_functional_connection}\mcite[thm. 7.1.4]{BerghLofstrom1976}
    \begin{equation*}
      E(t, x) = \sup \set{ \frac {K(s, x) - t} s : s > 0 }.
    \end{equation*}
  \end{thmenum}
\end{proposition}

\begin{definition}\label{def:approximation_space}\mcite[def. 7.1.5]{BerghLofstrom1976}
  Let \( \overline{X} = (X_0, X_1) \) be a compatible pair of Banach spaces. We define an \term{approximation space} \( E_{\alpha,q}(\overline{X}) \) for \( x \in \Sigma\overline{X} \) as the space of all members of \( \Sigma\overline{X} \) for which the following norm
  \begin{equation}\label{eq:def:approximation_space/norm}
    \norm{x}_{\alpha,q,E} \coloneqq \Phi_{-\alpha,q}(E(t,a))
  \end{equation}
  is finite.

  Here \( \alpha \) and \( q \) are both positive real numbers and \( q \) is potentially \( \infty \).
\end{definition}

\begin{theorem}\label{thm:interpolation_space_and_approximation_space}\mcite[thm. 7.1.6]{BerghLofstrom1976}
  Let \( X \) be a compatible pair of Banach spaces. Let \( \alpha \) and \( q \) be positive real numbers and define
  \begin{align*}
    \theta \coloneqq \frac 1 {\alpha + 1},
    &&
    r \coloneqq \theta q.
  \end{align*}

  Then
  \begin{equation*}
    (E_{\alpha,\theta q}(\overline{X}))^\theta = K_{\theta,q}(\overline{X}).
  \end{equation*}
\end{theorem}

\begin{theorem}\label{thm:interpolation_space_and_approximation_space_reiteration}\mcite[thm. 7.1.8]{BerghLofstrom1976}
  Let \( X \) be a compatible pair of Banach spaces. Let \( \theta, \alpha_0, \alpha_1, r_0, r_1 \) and \( q \) be positive real numbers such that \( \alpha_0 \neq \alpha_1 \) and define \( r \coloneqq \theta q \) and
  \begin{align*}
    \alpha \coloneqq (1 - \theta) \alpha_0 + \theta \alpha_1,
    &&
    \beta \coloneqq - \frac {\alpha_1 - \alpha} {\alpha_0 - \alpha}.
  \end{align*}

  Then
  \begin{equation*}
    K_{\theta,q}(E_{\alpha_0,r_0}(\overline{X}), E_{\alpha_1,r_1}(\overline{X})) = E_{\alpha,q}(\overline{X})
  \end{equation*}
  and
  \begin{equation*}
    E_{\beta,r}(E_{\alpha_0,r_0}(\overline{X}), E_{\alpha_1,r_1}(\overline{X}))^\theta = E_{\alpha,q}(\overline{X}).
  \end{equation*}
\end{theorem}


% Nonsmooth analysis
\section{Nonsmooth analysis}\label{sec:nonsmooth_analysis}

\begin{remark}\label{rem:nonsmooth_analysis}
  Nonsmooth analysis studies generalized differentiability for functions which are not necessarily differentible. The generalized derivatives (see \fullref{subsec:nonsmooth_derivatives}) are not linear, which motivates the study of subdifferentials (see \fullref{subsec:subdifferentials}).

  Both optimization in Euclidean spaces and infinite-dimensional optimization studies (not necessarily linear) real-valued functionals. Hence we are only concerned with studying real-valued topological vector spaces.
\end{remark}

\subsection{Nonsmooth derivatives}\label{subsec:nonsmooth_derivatives}

\begin{remark}\label{rem:nonsmooth_differentiability}
  Unlike in \fullref{def:differentiability}, we do not introduce terminology for differentiability because actual differentiability refers to linear approximations of \( f: U \to Y \) with some consistency properties. We will say that \enquote{\( f \) has a Clarke derivative at \( x_0 \) in the direction \( h \)} rather than \enquote{\( f \) is Clarke differentiable at \( x_0 \) in the direction \( h \)}.
\end{remark}

\begin{definition}\label{def:nonsmooth_derivatives}
  Let \( X \) be a real Hausdorff \hyperref[def:topological_vector_space]{topological vector spaces} and let \( U \subseteq X \) be an open set.

  We fix a point \( x_0 \in U \) and a direction \( h \in X \).

  \begin{DefEnum}
    \ILabel{def:nonsmooth_derivatives/directional}\Fullref{def:differentiability} already introduced directional derivatives. Here we introduce a special notation for them:
    \begin{equation*}
      D_h^+ f(x_0) = f_+'(x_0)(h) \coloneqq \lim_{t \downarrow 0} \frac {f(x_0 + th) - f(x_0)} t.
    \end{equation*}

    \ILabel{def:nonsmooth_derivatives/dini}\cite[definition 11.18]{Clarke2013} The upper (resp. lower) \Def{Dini derivative} is defined as
    \begin{BreakableAlign*}
      \overline{D}_h f(x_0) = \overline{f'}(x_0)(h) & \coloneqq \limsup_{t \downarrow 0} \frac {f(x_0 + th) - f(x_0)} t
      \\
      \underline{D}_h f(x_0) = \underline{f'}(x_0)(h) & \coloneqq \liminf_{t \downarrow 0} \frac {f(x_0 + th) - f(x_0)} t
    \end{BreakableAlign*}

    Dini derivatives are useful when the difference quotients are bounded but do not have a limit.

    \ILabel{def:nonsmooth_derivatives/clarke}\cite[section 10.1]{Clarke2013} The \Def{generalized Clarke derivative} is defined as
    \begin{equation*}
      D_h^\circ f(x_0)
      =
      f^\circ(x_0)(h)
      \coloneqq
      \limsup_{\substack{y \to x_0 \\ t \downarrow 0}} \frac {f(y + th) - f(y)} t.
    \end{equation*}

    Refer to \fullref{subsec:clarke_gradients} for their usefulness.
  \end{DefEnum}
\end{definition}

\subsection{Convex functions}\label{subsec:convex_functions}

Let \( X \) be a Hausdorff \hyperref[def:topological_vector_space]{topological vector space} and \( D \) be a \hyperref[def:convex_set]{convex} subset of \( X \).

\begin{definition}\label{def:convex_functions}
  A function \( f: D \to \BbbR \) is called \term{convex} if any of the following equivalent conditions hold:

  \begin{thmenum}
    \thmitem{def:convex_functions/ineq} For any two points \( x, y \in D \) and any \( t \in [0, 1] \) we have
    \begin{equation*}
      f(tx + (1-t)y) \leq tf(x) + (1-t)f(y).
    \end{equation*}

    \thmitem{def:convex_functions/epi} The \hyperref[def:epigraph]{epigraph}
    \begin{equation*}
      \epi f \coloneqq \{ (x, a) \in X \times \BbbR \colon f(x) \leq a \}
    \end{equation*}
    is convex.
  \end{thmenum}

  If \( -g \) is convex for some function \( g: D \to \BbbR \), we call \( g \) \term{concave}.

  Note that definitions do not require any topological structure on \( X \). Most of their properties, however, require a topology.
\end{definition}
\begin{proof}
  Let \( x, y \in D \) and let \( t \in [0, 1] \).

  \ImplicationSubProof{def:convex_functions/ineq}{def:convex_functions/epi} Let \( \epi f \) be a convex set. Obviously \( (x, f(x)) \in D \) and \( (y, f(y)) \in D \). By the convexity of \( \epi f \), we have
  \begin{equation*}
    f(tx + (1-t)y) \leq tf(x) + (1-t)f(y).
  \end{equation*}

  Thus \( f \) is a convex function.

  \ImplicationSubProof{def:convex_functions/epi}{def:convex_functions/ineq} Let \( f \) be convex. Let \( a \geq f(x) \) and \( b \geq f(y) \) so that \( (x, a) \in \epi f \) and \( (y, b) \in \epi f \). Hence
  \begin{equation*}
    f(tx + (1-t)y) \leq tf(x) + (1-t)f(y) \leq ta + (1-t)b,
  \end{equation*}
  which implies that
  \begin{equation*}
    (tx + (1-t)y, ta + (1-t)b) \in \epi f.
  \end{equation*}

  Thus \( \epi f \) is a convex set.
\end{proof}

\begin{definition}\label{def:affine_functions_concave_and_convex}
  \hyperref[def:affine_operator]{Affine functions} \( f: X \to \BbbR \) are simultaneously convex and concave.
\end{definition}

\begin{proposition}\label{thm:convex_subdifferential_is_convex_and_weak*_closed}\mcite[exer. 1.10]{Phelps1993}
  For any convex function \( f \) and any \( x \in D \), the set \( \partial f(x) \) is convex and weak* closed.
\end{proposition}
\begin{proof}
  Fix \( x \in D \). If \( \partial f(x) \) is empty, then the theorem is trivially true.

  Suppose it is nonempty and \( y^*, z^* \in \partial f(x) \). For any \( x \in D \) we then have
  \begin{balign*}
     & \inprod{y^*} {x - x} \leq f(x) - f(x), \\
     & \inprod{z^*} {x - x} \leq f(x) - f(x).
  \end{balign*}

  Fix \( t \in [0, 1] \) and \( x \in D \). It follows that
  \begin{balign*}
    \inprod{t y^* + (1-t) z^*} {x - x}
     & =
    t \inprod{y^*} {x - x} + (1-t) \inprod{z^*} {x - x}
    \leq \\ &\leq
    t [f(x) - f(x)] + (1-t) [f(x) - f(x)]
    =    \\ &=
    f(x) - f(x),
  \end{balign*}
  thus \( t y^* + (1-t)z^* \in \partial f(x) \) and hence \( \partial f(x) \) is convex.

  To prove weak*-closedness, we consider the decomposition
  \begin{balign*}
    \partial f(x)
     & =
    \{ x^* \in E^* \colon \forall x \in D, \inprod {x^*} {x - x} \leq f(x) - f(x) \}
    =    \\ &=
    \bigcap_{x \in D} \{ x^* \in E^* \colon \inprod {x^*} {x - x} \leq f(x) - f(x) \}
    =    \\ &=
    \bigcap_{x \in D} L(x)^{-1} (-\infty, f(x) - f(x)],
  \end{balign*}
  where
  \begin{balign*}
     & L: E \to E^{**}                  \\
     & L(x)(x^*) = \inprod {x^*} {x - x}.
  \end{balign*}

  For each \( x \in E \), the functionals \( L(x) \) are weak*-to-weak continuous because the image \( L(E) \subseteq E^{**} \) is isometrically isomorphic to a translation of \( E \). Hence the preimage \( L(x)^{-1} (-\infty, f(x) - f(x)] \) is closed and \( \partial f(x) \) is weak*-closed as the intersection of weak*-closed sets.
\end{proof}

\begin{lemma}\label{thm:convex_difference_quotient_grows}
  For every point \( x \in X \) and every direction \( h \in S_X \) the difference quotient is a monotone function of \( t > 0 \), i.e. for \( 0 < s < t \)
  \begin{balign*}
    \frac {f(x + sh) - f(x)} s
    \leq
    \frac {f(x + th) - f(x)} t
  \end{balign*}
\end{lemma}
\begin{proof}
  \begin{balign*}
    \frac {f(x + sh) - f(x)} s
    =
    \frac t s \frac {f(x + \frac s t t h) - f(x)} t
    =
    \frac t s \frac {f\left(\frac s t (x + th) + (1 - \frac s t) x \right) - f(x)} t
    \leq \\ \leq
    \frac t s \frac {\frac s t f(x + t h) + (1 - \frac s t) f(x) - f(x)} t
    =
    \frac t s \frac s t \frac {f(x + th) - f(x)} t
    =
    \frac {f(x + th) - f(x)} t
  \end{balign*}
\end{proof}

\begin{proposition}\label{thm:convex_one_sided_derivatives_exist}
  For every point \( x \in X \) and every direction \( h \in S_X \) the one-sided derivative \( f_+'(x)(h) \) exists.
\end{proposition}
\begin{proof}
  We use the convexity of \( f \) to obtain
  \begin{balign*}
    f(x) = f \left(x + \frac {th} 2 - \frac {th} 2 \right) \leq \frac {f(x + th) + f(x - th)} 2,
    \\
    0 \leq [f(x - th) - f(x)] + [f(x + th) - f(x)],
    \\
    -[f(x - th) - f(x)] \leq [f(x + th) - f(x)],
    \\
    -\frac {f(x + t(-h)) - f(x)} t \leq \frac {f(x + th) - f(x)} t,
  \end{balign*}
  thus the difference quotient in \( f_+'(x)(h) \) is bounded below by the difference quotient for \( -f_+'(x)(-h) \).

  \Fullref{thm:convex_difference_quotient_grows} implies that the right difference quotient is non-increasing, thus both limits exist and
  \begin{equation*}
    -f_+'(x)(-h) \leq f_+'(x)(h).
  \end{equation*}
\end{proof}

\begin{proposition}\label{thm:convex_one_sided_derivatives_sublinear}
  For every point \( x \in X \) and every direction \( h \in S_X \) the one-sided derivative \( f_+'(x)(h) \) is a \hyperref[def:sublinear_functional]{sublinear functional}.
\end{proposition}
\begin{proof}
  \SubProofOf{def:sublinear_functional/subadditivity} It follows directly from
  \begin{balign*}
    \frac {f(x + t(a + b)) - f(x)} t
     & =
    \frac {f(\tfrac 1 2 (x + 2ta) + \tfrac 1 2 (x + 2tb)) - f(x)} t
    \leq \\ &\leq
    \frac {\tfrac 1 2 f(x + 2ta) + \tfrac 1 2 f(x + 2tb) - f(x)} t
    =    \\ &=
    \frac {f(x + 2ta) - f(x)} {2t} + \frac {f(x + 2tb) - f(x)} {2t}.
  \end{balign*}

  \SubProofOf{def:sublinear_functional/positive_homogeneity} For \( \lambda > 0 \) the equality \( f_+'(x)(\lambda h) = \lambda f_+'(x)(h) \) follows from
  \begin{balign*}
    \frac {f(x + t \lambda h) - f(x)} t
    =
    \lambda \frac {f(x + t \lambda h) - f(x)} {t \lambda}
  \end{balign*}
\end{proof}

\begin{corollary}\label{thm:convex_one_sided_derivative_negative_inequality}
  \begin{equation*}
    -f_+'(x)(-h) \leq f_+'(x)(h)
  \end{equation*}
\end{corollary}
\begin{proof}
  \begin{equation*}
    0 = f_+'(x)(h + (-h)) \leq f_+'(x)(h) + f_+'(x)(-h)
  \end{equation*}
\end{proof}

\begin{proposition}\label{thm:convex_iff_subdifferential_nonempty}
  The continuous function \( f: D \to X \) is convex if and only if its subdifferential \( \partial f(x) \) (see \fullref{def:subdifferentials/convex}) is nonempty for every \( x \) in \( D \).
\end{proposition}
\begin{proof}
  \todo{Prove}.
\end{proof}

\begin{proposition}
  \label{thm:convex_one_sided_derivative_is_max}
  For every direction \( h \in S_X \), we have that
  \begin{equation*}
    f_+'(x)(h) = \max\{ \inprod {x^*} h \colon x^* \in \partial f(x) \}.
  \end{equation*}
\end{proposition}

\begin{theorem}\label{thm:singleton_subdifferential_implies_gateaux}
  If \( f \) is continuous and if the subdifferential \( \partial f(x) \) at \( x \in X \) is a singleton with element \( x^* \), then \( f \) is Gateaux differentiable at \( x \) and \( f_G'(x) = x^* \).
\end{theorem}
\begin{proof}
  Let \( h \in S_X \) be arbitrary. \Fullref{thm:convex_one_sided_derivatives_exist} implies that the one-sided derivatives \( f_+'(x)(-h) \) and \( f_+'(x)(h) \) exist and
  \begin{equation*}
    -f_+'(x)(-h) \leq f_+'(x)(h).
  \end{equation*}

  Assume\DNE that \( f \) is not Gateaux differentiable at \( x \), i.e. for some \( h_0 \in X \), we have a strict inequality. Then by \fullref{thm:convex_one_sided_derivative_is_max}
  \begin{balign*}
    \min\{ \inprod {x^*} {h_0} \colon x^* \in \partial f(x) \}
    =
    -\max\{ \inprod {x^*} {-h_0} \colon x^* \in \partial f(x) \}
    =
    -f_+'(x)(-h_0)
    < \\ <
    f_+'(x)(h_0)
    =
    \max\{ \inprod {x^*} {h_0} \colon x^* \in \partial f(x) \},
  \end{balign*}
  which implies that there is more that one functional \( x^* \in \partial_C f(x) \). This contradicts the assumption of the theorem.

  Thus \( f \) is Gateaux differentiable at \( x \).
\end{proof}

\begin{theorem}\label{thm:rn_continuous_convex_partial_derivatives_imply_gateaux}\mcite[exer. 1.15(b]{Phelps1993})
  In \( \BbbR^n \), the existence of the partial derivatives at \( x \) for a continuous convex function \( f: D \to \BbbR \) at a point \( x \in D \) implies Gateaux differentiability.
\end{theorem}
\begin{proof}
  Let \( D \subseteq \BbbR^n \) be an open and convex set and let \( f: D \to \BbbR \) be continuous and convex. Then \( f_+'(x) \) exists everywhere by \fullref{thm:convex_one_sided_derivatives_exist} and is a subdifferential functional by \fullref{thm:convex_one_sided_derivatives_sublinear}.

  Let \( e_1, \ldots, e_n \) be the canonical basis for \( \BbbR^n \).

  The partial derivatives
  \begin{balign*}
    \frac {\partial f} {\partial x_i} (x)
    \coloneqq
    \lim_{t \to 0} \frac {f(x + t e_i) - f(x)} t
    =
    f_+'(x)(e_i)
  \end{balign*}
  exist, hence the projections of \( f_+'(x) \) along the coordinate exes are linear.

  Define line linear functional
  \begin{equation*}
    l(h) \coloneqq \sum_{i=1}^n h_i \inprod{\frac {\partial f} {\partial x_i} (x)} h,
  \end{equation*}
  where \( h_1, \ldots, h_n \) are the coordinates of \( h \) along \( e_1, \ldots, e_n \).

  We will show that \( l \cong f_+' \). Fix \( h \in S_X \). We have
  \begin{balign}\label{thm:rn_continuous_convex_partial_derivatives_imply_gateaux/diff_dominated}
    f_+'(x)(h)
     & =
    f_+'(x)\left(\sum_{i=1}^n h_i e_i \right)
    \overset {\text{sublinearity}} \leq \nonumber      \\ &\leq
    \sum_{i=1}^n f_+'(x)(h_i e_i)
    \overset {\text{linearity along } e_i} = \nonumber \\ &=
    \sum_{i=1}^n h_i f_+'(x)(e_i)
    =
    \sum_{i=1}^n h_i \inprod{\frac {\partial f} {\partial x_i} (x)} h.
  \end{balign}

  Thus
  \begin{balign*}
    \inprod l h
    =
    -\inprod l {-h}
    \overset {\ref{thm:rn_continuous_convex_partial_derivatives_imply_gateaux/diff_dominated}} \leq
    -f_+'(x)(-h)
    \overset {\text{\ref{thm:convex_one_sided_derivative_negative_inequality}}} \leq
    f_+'(x)(h)
    \overset {\ref{thm:rn_continuous_convex_partial_derivatives_imply_gateaux/diff_dominated}} \leq
    \inprod l h,
  \end{balign*}
  i.e. \( f_+'(x)(h) = \inprod l h \) for all \( h \in S_X \), hence \( f_+'(x) \) is a linear functional and \( f \) is Gateaux differentiable at \( x \).
\end{proof}

\begin{theorem}\label{thm:rn_continuous_convex_gateaux_implies_frechet}\mcite[exer. 1.15(a]{Phelps1993})
  In \( \BbbR^n \), Gateaux differentiability of a continuous convex function \( f: D \to \BbbR \) at a point \( x \in D \) implies Frechet differentiability.
\end{theorem}
\begin{proof}
  Since \( f \) is Gateaux differentiable (\fullref{def:differentiability/gateaux}) at \( x \), the derivative \( f'(x) = f_+'(x) \) is linear.

  Because \( f \) is continuous and convex, it is locally Lipschitz with constant \( L \) in some \( \delta \)-ball with center \( x \).

  Suppose\DNE that \( f \) is not Frechet differentiable at \( x \). Inverting the condition in \fullref{def:differentiability/frechet}, we obtain that there exist \( \varepsilon > 0 \) and a sequence \( \{ h_n \}_n \subseteq B(x, \delta) \setminus \{ 0 \} \) such that \( \norm{h_n} \to 0 \) and yet for all \( n \in \BbbZ_{>0} \),
  \begin{balign}\label{thm:rn_continuous_convex_gateaux_implies_frechet/frechet_assumption}
    \abs{f(x + h_n) - f(x) - \inprod{f'(x)} {h_n}} > \varepsilon \norm{h_n}.
  \end{balign}

  Define
  \begin{balign*}
    t_n \coloneqq \norm{h_n}
     &  &
    u_n \coloneqq \frac{h_n} {\norm {h_n}}.
  \end{balign*}

  Obviously \( t_{n_k} \downarrow 0 \). The vectors \( h_n \) are linearly independent since otherwise \( f \) would not be Gateaux differentiable at \( x \), hence \( u_n \) are not all equal.

  Since \( S_{\BbbR^n} \) is compact\USC, by the Bolzano-Weierstrass theorem, there exists a convergent subsequence \( \{ u_{n_k} \}_k \underset {k \to \infty} \to u_0 \) of \( \{ u_n \}_n \). We have

  \begin{balign}\label{thm:rn_continuous_convex_gateaux_implies_frechet/frechet_estimate}
     & \phantom= \abs{\frac {f(x + t_{n_k} u_{n_k}) - f(x)} {t_{n_k}} - \inprod{f'(x)} {u_{n_k}}}
    \leq \nonumber
    \abs{\frac {f(x + t_{n_k} u_{n_k}) - f(x + t_{n_k} u_0)} {t_{n_k}}} +                       \\ &+ \abs{\frac {f(x + t_{n_k} u_0) - f(x)} {t_{n_k}} - \inprod{f'(x)} {u_0}} + \abs{\inprod{f'(x)} {u_0 - u_{n_k}}}
    \leq \nonumber                                                                              \\ &\leq
    L \norm{u_{n_k} - u_0} + \abs{\frac {f(x + t_{n_k} u_0) - f(x)} {t_{n_k}} - \inprod{f'(x)} {u_0}} + \norm{f'(x)} \norm{u_0 - u_{n_k}}.
  \end{balign}

  Fix \( \delta > 0 \). Because of the Gateaux differentiable of \( f \) at \( x \), we can pick \( k_0 \) such that
  \begin{equation*}
    \abs{\frac {f(x + t_{n_{k_0}} u_0) - f(x)} {t_{n_{k_0}}} - \inprod{f'(x)} {u_0}} < \delta.
  \end{equation*}

  Because \( \{ u_{n_k} \}_k \) converges to \( u_0 \), we can choose \( k_1 \) such that
  \begin{equation*}
    \norm{u_0 - u_{n_{k_1}}} < \delta.
  \end{equation*}

  Thus for \( k > \max \{ k_0, k_1 \} \), \fullref{thm:rn_continuous_convex_gateaux_implies_frechet/frechet_estimate} is bounded by
  \begin{balign*}
    \abs{\frac {f(x + t_{n_k} u_{n_k}) - f(x)} {t_{n_k}} - \inprod{f'(x)} {u_{n_k}}}
    \leq
    (L + 1 + \norm{f'(x)}) \delta.
  \end{balign*}

  It suffices to choose \( \delta > 0 \) so that
  \begin{equation*}
    \delta < \frac 1 {L + 1 + \norm{f'(x)}}
  \end{equation*}
  in order to have, for \( k > \max \{ k_0, k_1 \} \),
  \begin{equation*}
    \abs{\frac {f(x + t_{n_k} u_{n_k}) - f(x)} {t_{n_k}} - \inprod{f'(x)} {u_{n_k}}} < \varepsilon.
  \end{equation*}

  But this contradicts \fullref{thm:rn_continuous_convex_gateaux_implies_frechet/frechet_assumption}, hence \( f \) is Frechet differentiable at \( x \).
\end{proof}

\begin{corollary}\label{thm:rn_continuous_convex_partial_derivatives_imply_frechet}
  In \( \BbbR^n \), the existence of the partial derivatives at \( x \) for a continuous convex function \( f: D \to \BbbR \) at a point \( x \in D \) is equivalent to Frechet differentiability.
\end{corollary}
\begin{proof}
  A direct consequence of and \fullref{thm:rn_continuous_convex_partial_derivatives_imply_gateaux} and \fullref{thm:rn_continuous_convex_gateaux_implies_frechet}.
\end{proof}

\begin{theorem}\label{thm:rn_continuous_convex_frechet_almost_everywhere}\mcite[exer. 1.17]{Phelps1993}
  In \( \BbbR^n \), continuous convex functions \( f: D \to \BbbR \) are differentiable almost everywhere.
\end{theorem}
\begin{proof}
  For all \( h \in S_X \) and small enough \( t > 0 \) we define
  \begin{balign*}
     & \varphi_h^t: D \to \BbbR
     & \varphi_h^t(x) \coloneqq \frac {f(x + th) - f(x)} t
  \end{balign*}
  and \( \varphi_h(x) \coloneqq f_+'(x)(h) = \lim_{t \downarrow 0} \varphi_h^t(x) \).

  Considered as functions of \( x \), \( \varphi_h^t \) are obviously continuous hence Borel measurable and so \( \varphi_h \) is also Borel measurable.

  Denote by
  \begin{balign*}
    B_h
    \coloneqq
    \{ x \in D \colon -f_+'(x)(-h) < f_+'(x)(h) \}
    =
    \{ x \in D \colon -\varphi_{-h}(x) - \varphi_h(x) < 0 \}
  \end{balign*}
  the set of points \( x \in D \) where the one-sided derivative \( f_+'(x)(h) \) is not linear, given a fixed direction \( h \in S_X \). If \( B_h \) is nonempty, \( f \) is not differentiable at \( x \).

  The sets \( B_h \) are Borel sets since they are the preimages of \( (-\infty, 0) \) under a Borel function. We will show that it is a null set for every direction \( h \).

  Fix \( h \in S_X \). Denote by \( \delta_x \coloneqq \sup \{ t > 0 \colon x + th \in D \} \).

  The function \( t \mapsto f(x + th) \) is a convex function of one variable. By \cite[theorem 1.16]{Phelps1993}, it is differentiable \( \mu_1 \)-almost everywhere in \( [0, \delta_x) \), where \( \mu_m \) is the Lebesgue \( m \)-measure.

  Denote
  \begin{balign*}
     & H \coloneqq \linspan\{ h \} \cong \BbbR^1,
    \\
     & H^\perp \cong \BbbR^{n-1} \text{ - the orthogonal complement of \( H \) in \( \BbbR^n \)},
    \\
     & L_x \coloneqq \{ x + th, 0 \leq t < \delta_x \} - half-open segments in D.
  \end{balign*}

  THe whole domain \( D \) can be represented as \( D = \cup \{ L_x \colon x \in H^\perp \} \).

  We can now use Fubini's theorem to show that \( B_h \) is a null set:
  \begin{balign*}
    \mu_n(B_h)
    =
    \int_{B_h} dz
    =
    \int_{\BbbR^n = H^\perp \oplus H} \chi_{B_h} (z) dz
    =
    \int_{H^\perp} \int_{L_x} \chi_{B_h} (y) dy dx
    = \\ =
    \int_{H^\perp} \mu_1(B_h \cap L_x) dx
    =
    \int_{H^\perp} 0 dx
    =
    0.
  \end{balign*}

  Hence for all \( h \in S_X \), \( -f_+'(x)(-h) = f_+'(x)(h) \) for almost all \( x \in D \).

  In particular, if \( e_1, \ldots, e_n \) is the canonical basis of \( \BbbR^n \), the \( i \)-th partial derivative \( \frac{\partial f} {\partial x_i} (x) \) exists only in \( D \ B_{e_i} \).

  The gradient
  \begin{equation*}
    \nabla f(x) = \left( \frac{\partial f} {\partial x_1} (x), \ldots, \frac{\partial f} {\partial x_n} (x) \right)
  \end{equation*}
  then exists in
  \begin{equation*}
    \hat D \coloneqq (D \ B_{e_1}) \cap \ldots \cap (D \ B_{e_n}) = D \setminus \left( \bigcup_{i=1}^n B_{e_i} \right).
  \end{equation*}

  \Fullref{thm:rn_continuous_convex_partial_derivatives_imply_frechet} then implies that \( f \) is Frechet differentiable in \( \hat D \), i.e. almost everywhere in \( D \).
\end{proof}

\subsection{Subdifferentials}\label{subsec:subdifferentials}

Let \( X \) be a Hausdorff \hyperref[def:topological_vector_space]{topological vector space}, let \( D \subseteq X \) be an open set and \( f: D \to \BbbR \) be any function.

\begin{definition}\label{def:subdifferentials}
  We fix a point \( x \in D \). We define different types of \term{subgradients} and \term{subdifferentials}. Subgradients are linear functionals \( x^* \in X^* \) that approximate \( f \) at the point \( x \) in a certain way, and a subdifferential is the set of all subgradients of a given type.

  \begin{thmenum}
    \thmitem{def:subdifferentials/convex}\mcite[59]{Clarke2013}We say that \( x^* \in X^* \) is a \term{subgradient of \( f \) at \( x \)} if for every \( y \in D \) we have
    \begin{equation*}
      f(y) - f(x) \geq \inprod {x^*} {y - x}.
    \end{equation*}

    The \term{subdifferential of \( f \) at \( x \)} is denoted by \( \partial f(x) \) and is also, sometimes called the \term{convex subdifferential} because of \fullref{thm:convex_iff_subdifferential_nonempty}.

    \thmitem{def:subdifferentials/clarke}\mcite[def. 10.3]{Clarke2013}We say that \( x^* \in X^* \) is a \term{Clarke (generalized) subgradient of \( f \) at \( x \)} if for every direction \( h \in X \) we have
    \begin{equation*}
      f^\circ(x)(h) \geq \inprod {x^*} h,
    \end{equation*}
    where \( f^\circ(x)(h) \) is the generalized Clarke \hyperref[def:nonsmooth_derivatives/clarke]{derivative}.

    The \term{subdifferential of \( f \) at \( x \)} is denoted by \( \partial_C f(x) \). Confusingly, the Clarke subdifferential is called the \enquote{generalized gradient} by Clarke himself with no special name for the Clarke subgradients.

    See \fullref{subsec:clarke_gradients} for properties of these subgradients.

    \thmitem{def:subdifferentials/proximal}\mcite[227]{Clarke2013}We say that \( x^* \in X^* \) is a \term{proximal subgradient of \( f \) at \( x \)} if there exist \( \sigma > 0 \) and a neighborhood \( V \subseteq X \) of \( x \) such that for every \( y \in D \cap V \) we have
    \begin{equation*}
      f(y) - f(x) + \sigma \norm{y - x}^2 \geq \inprod {x^*} {y - x}.
    \end{equation*}

    The \term{proximal subdifferential of \( f \) at \( x \)} is denoted by \( \partial_P f(x) \).

    \thmitem{def:subdifferentials/limiting}\mcite[def. 11.10]{Clarke2013}Suppose the following are satisfied:
    \begin{enumerate}
      \item \( \{ x_n \}_n \subseteq D \) is a sequence of points converging to \( x \)
      \item \( f(x_n) \to f(x) \) (redundant if \( f \) is continuous)
      \item \( x_n^* \) is a proximal subgradient for \( f \) at \( x_n \) for every \( n \in \BbbZ_{>0} \).
    \end{enumerate}

    If the limit \( x^* \coloneqq \lim_n x_n^* \) exists and is a continuous linear functional, we call \( x^* \) a \term{limiting subgradient of \( f \) at \( x \)}.

    The \term{limiting subdifferential of \( f \) at \( x \)} is denoted by \( \partial_P f(x) \).
  \end{thmenum}
\end{definition}

\subsection{Clarke generalized gradients}\label{subsec:clarke_gradients}

Let \( X \) be a Banach space and \( f: X \to \BbbR \) be locally Lipschitz.

\begin{definition}\label{def:clarke_gradient}\mcite[def. 10.3]{Clarke2013}
  Let \( x \in X \) and \( U \subseteq X \) be a neighborhood of x where \( f \) is \( L \)-Lipschitz, i.e.

  \begin{equation*}
    \forall y, z \in U, \abs{f(y) - f(z)} \leq L \norm{y - z}.
  \end{equation*}

  We use the Clarke generalized \hyperref[def:nonsmooth_derivatives/clarke]{derivative},
  \begin{equation*}
    f^\circ(x)(h) \coloneqq \limsup_{\substack{y \to x \\ t \downarrow 0}} \frac {f(y + th) - f(y)} t
  \end{equation*}

  We define the \term{generalized gradient of \( f \) at \( x \)} to be the set
  \begin{equation*}
    \partial_C f(x) \coloneqq \{ x^* \in X^* \colon \forall h \in X, f^\circ(x)(h) \geq \inprod {x^*} h \}.
  \end{equation*}

  We say that the vector \( h \) is a \term{descent direction of \( f \) at \( x \)} if
  \begin{equation*}
    \limsup_{t \downarrow 0} \frac {f(x + th) - f(x)} t < 0.
  \end{equation*}
\end{definition}

\begin{proposition}\label{thm:clarke_derivative_exists}
  The generalized derivative of a locally Lipschitz function \( f: X \to \BbbR \) exists for every \( x \in X \).
\end{proposition}
\begin{proof}
  Let \( x, h \in X \) and let \( U \) be a neighborhood of \( x \) where the Lipschitz condition holds with the constant \( L_U \). Then there exists \( \delta_0 > 0 \) such that \( B(x, \delta_0) \subseteq U \).

  Define \( \delta_1 \coloneqq \frac 1 2 \min \left\{\delta_0, \frac {\delta_0} {\norm h} \right\} < \delta_0 \), so that for \( y \in B(x, \delta_1) \) and \( t \in (0, \delta_1) \) we have
  \begin{balign*}
    \norm{(y + th) - x}
    \leq
    \norm{y - x} + t \norm h
    \leq
    \delta_1 + \delta_1 \norm h
    \leq
    \begin{cases}
      \frac {\delta_0} 2 (1 + \norm h),           & \norm h \leq 1 \\
      \frac {\delta_0} {2 \norm h} (1 + \norm h), & \norm h > 1.
    \end{cases}
  \end{balign*}

  In both cases we get that \( y + th \in B(x, \delta_0) \).

  The generalized derivative in \( x \) in the direction \( h \in X \) is then norm-bounded by
  \begin{balign*}
    \abs{f^\circ(x)(h)}
    =
    \abs{\limsup_{\substack{y \to x                             \\ t \downarrow 0}} \frac {f(y + th) - f(y)} t}
    =
    \abs{\lim_{\delta \to 0} \sup_{\substack{y \in B(x, \delta) \\ t \in (0, \delta)}} \frac {f(y + th) - f(y)} t}
    \leq                                                        \\ \leq
    \abs{\sup_{\substack{y \in B(x, \delta_1)                   \\ t \in (0, \delta_1)}} \frac {f(y + th) - f(y)} t}
    \leq
    \sup_{\substack{y \in B(x, \delta_1)                        \\ t \in (0, \delta_1)}} \frac {\abs{f(y + th) - f(y)}} t
    \leq                                                        \\ \leq
    \sup_{\substack{y \in B(x, \delta_1)                        \\ t \in (0, \delta_1)}} \frac {\norm{(y + th) - (y)}} t
    =
    \norm h.
  \end{balign*}

  The fact that \( f \) is locally Lipschitz gave us that the supremum is taken over a bounded set and thus the generalized derivative exists.
\end{proof}


% Approximation theory
\subsection{Lagrange polynomials}\label{subsec:lagrange_polynomials}

\begin{definition}\label{def:omega_polynomial}
  Given distinct elements \( x_0, \ldots, x_n \) of the field \( \BbbK \), we form the polynomial
  \begin{equation*}
    \omega(X) \coloneqq \prod_{k=0}^n (X - x_j).
  \end{equation*}
\end{definition}

\begin{proposition}\label{thm:omega_polynomial_derivative}
  For the polynomial \( \omega \) from \fullref{def:omega_polynomial}, for \( k = 0, \ldots, n \) we have
  \begin{equation*}
    \omega'(x_j) = \prod_{\substack{j = 0 \\ j \neq k}}^n (x_j - x_k),
  \end{equation*}
  where \( \omega' \) is the \hyperref[def:algebraic_derivative]{algebraic derivative} of \( \omega \).
\end{proposition}
\begin{proof}
  Fix \( k \in \{ 0, \ldots, n \} \) and denote
  \begin{equation*}
    q(X) \coloneqq \prod_{\substack{j = 0 \\ j \neq k}}^n (X - x_j).
  \end{equation*}

  Then
  \begin{equation*}
    \omega(X) = (X - x_k) q(X)
  \end{equation*}
  so
  \begin{equation*}
    \omega'(X) = [q(X) + X q'(X)] - x_k q'(X) = q(X) + (X - x_k) q'(X).
  \end{equation*}

  So for \( x_k \) we have
  \begin{equation*}
    \omega'(x_k) = q(x_k) = \prod_{\substack{j = 0 \\ j \neq k}}^n (x_k - x_j).
  \end{equation*}
\end{proof}

\begin{theorem}[Lagrange interpolation]\label{thm:lagrange_interpolation}
  Let \( x_0, x_1, \ldots, x_n \) be pairwise distinct elements of \( \BbbK \) and let \( y_0, y_1, \ldots, y_n \) be arbitrary elements of \( \BbbK \). Then there exists a unique \hyperref[def:polynomial_algebra]{polynomial} \( L(X) \) of degree at most \( n \) such that \( L(x_k) = y_k \) for \( k = 1, \ldots, n \).
\end{theorem}
\begin{proof}
  \UniquenessSubProof Suppose that \( p, q \) are polynomials of degree at most \( n \) that both satisfy \( L(x_k) = y_k \) for \( k = 1, \ldots, n \). Their difference \( p - q \) is a polynomial of degree at most \( n \) that has \( n + 1 \) roots. By \fullref{thm:def:integral_domain/root_limit}, \( p - q = 0 \).

  \ExistenceSubProof We will construct the polynomial explicitly. Define the Lagrange basis polynomial
  \begin{equation*}
    L(X) = \sum_{m=0}^n y_m \prod_{\substack{j = 0 \\ j \neq m}}^n \frac {(X - x_j)} {(x_m - x_j)}.
  \end{equation*}

  For \( k = 0, 1, \ldots, n \) we have
  \begin{equation*}
    L(x_k) = y_k \underbrace{\prod_{\substack{j = 0 \\ j \neq k}}^n \frac {(x_k - x_j)} {(x_k - x_j)}}_{1} + \sum_{\substack{m = 0 \\ m \neq k}}^n y_m \overbrace{\frac{(x_k - x_m)}{(x_k - x_m)}}^{0} \prod_{\substack{j = 0 \\ j \neq k \\ j \neq m}}^n \frac {(x_k - x_j)} {(x_m - x_j)} = y_k.
  \end{equation*}

  Therefore, \( L \) is the desired polynomial.
\end{proof}

\begin{theorem}[Finite field Lagrange interpolation]\label{thm:finite_field_lagrange_interpolation}\mcite{MathOF:functions_over_finite_fields}
  For a multivariate function \( f: \BbbF_q^n \to \BbbF_q \) over the \hyperref[thm:finite_fields]{finite field} \( \BbbF_q \), there exists a unique multivariate polynomial \( L(X_1, \ldots, X_n) \) such that
  \begin{itemize}
    \item For any sequence of values \( x_1, \ldots, x_n \),
    \begin{equation*}
      f(x_1, \ldots, x_n) = L(x_1, \ldots, x_n).
    \end{equation*}

    \item For every monomial \( X_1^{\gamma_n} \cdots X_n^{\gamma_n} \) of \( L \), \( \gamma_i < q \) for \( i = 1, \ldots, n \).
  \end{itemize}
\end{theorem}
\begin{proof}
  For any point \( (x_1, \ldots, x_n) \in \BbbZ_q^n \), the characteristic polynomial
  \begin{equation*}
    c(X_1, \ldots, X_n) \coloneqq \prod_{i=0}^n \parens*{ \prod_{\substack{m=0 \\ m \neq x_i}}^{q - 1} \frac {X_i - m} {x_i - m} }
  \end{equation*}
  satisfies
  \begin{equation*}
    c(y_1, \ldots, y_n) = \begin{cases}
      1, &x_i = y_i \T{for all} i = 1, \ldots, n \\
      0, &\T{otherwise.}
    \end{cases}
  \end{equation*}

  As in \fullref{thm:lagrange_interpolation}, give us the desired polynomial, a linear combination of these basis polynomials with coefficients corresponding to the values of \( f \) give us the desired polynomial.
\end{proof}

\subsection{Bernstein inequalities}\label{subsec:bernstein_inequalities}

\begin{definition}\label{def:real_function_space_operators}
  Consider the \hyperref[def:function/single_valued]{operator} \( T: C([a, b]) \to C([a, b]) \).

  \begin{thmenum}
    \labitem{def:real_function_space_operators/positive} If \( f([a, b]) \subseteq [0, \infty) \) implies \( T(f)([a, b]) \subseteq [0, \infty) \), we say that \( T \) is \term{positive}.

    \labitem{def:real_function_space_operators/monotone} If \( f(x) \leq g(x) \) for all \( x \in [a, b] \) implies \( T(f)(x) \leq T(g)(x) \) for all \( x \in [a, b] \), we say that \( T \) is \term{monotone}.
  \end{thmenum}
\end{definition}

\begin{definition}\label{def:periodic_function_space}\mcite[44]{Николов2020Лекции}
  We denote by \( \tilde{C}([a, b]) \) the subspace of \( C([a, b]) \) consisting of all continuous functions in \( [a, b] \) which are periodic with minimal period \( b - a \).
\end{definition}

\begin{definition}\label{def:approximation_error}\mcite[44]{Николов2020Лекции}
  We introduce two operators.

  \begin{thmenum}
    \labitem{def:approximation_error/algebraic} The \term{algebraic approximation error}
    \begin{balign*}
      E_n: C([a, b]) \to [0, \infty] \\
      E_n(f) \coloneqq \inf_{p \in \pi_n} \norm{f - p}.
    \end{balign*}

    \labitem{def:approximation_error/trigonometric} The \term{trigonometric approximation error}
    \begin{balign*}
      \tilde{E}_n: C([a, b]) \to [0, \infty] \\
      \tilde{E}_n(f) \coloneqq \inf_{p \in \tau_n} \norm{f - p}.
    \end{balign*}
  \end{thmenum}
\end{definition}

\begin{theorem}[Jackson's trigonometric theorem]\label{thm:jacksons_trigonometric_theorem}\mcite[47]{Николов2020Лекции}
  For \( f \in \tilde{C}[-\pi, \pi] \) we have
  \begin{equation*}
    \tilde{E}_n(f) \leq \frac {6^{k+1}} {n^k} \omega\left(f^{(k)}, \frac 1 n \right).
  \end{equation*}
\end{theorem}

\begin{theorem}[Szego’s inequality]\label{thm:szegos_trigonometric_inequality}\mcite[55]{Николов2020Лекции}
  For any nonnegative integer \( n \) and any \( s \in S_{\tau_n} \), we have
  \begin{equation}\label{eq:thm:szegos_trigonometric_inequality}
    [s'(\theta)]^2 + n^2 s^2 (\theta) \leq n^2 \quad\forall \theta \in [-\pi, \pi].
  \end{equation}
\end{theorem}
\begin{proof}
  Fix \( n = 1, 2, \ldots \) and \( \alpha \in [-1, 1] \).

  For brevity, denote \( c(\theta) \coloneqq \cos(n \theta) \). Let \( \theta_s \) and \( \theta_c \) be numbers in \( [-\pi, \pi] \) such that
  \begin{equation*}
    s(\theta_s) + c(\theta_c) = \alpha.
  \end{equation*}

  We will show that
  \begin{equation}\label{eq:thm:szegos_trigonometric_inequality/abs}
    \abs{s'(\theta_s)} \leq \abs{c'(\theta_c)}
  \end{equation}

  This will, in turn, imply that
  \begin{equation*}
    [s'(\theta_s)]^2
    \leq
    [c'(\theta_c)]^2
    =
    n^2 [\sin(n \theta_c)]^2
    \overset {\ref{thm:trigonometric_identities/pythagorean_identity}}
    =
    n^2 [1 - \cos(n \theta_c)^2]
    =
    n^2 [1 - s(\theta_s)]
  \end{equation*}
  which is equivalent to \eqref{eq:thm:szegos_trigonometric_inequality}.

  Now we will prove \eqref{eq:thm:szegos_trigonometric_inequality/abs}. If \( \theta_c \) is a critical point of \( c \), i.e. if \( r'(\theta_c) = 0 \), then \( c'(\theta_c) = s'(\theta_s) \) and \eqref{eq:thm:szegos_trigonometric_inequality/abs} holds. Suppose that \( \theta_c \) is not a critical point. Denote by
  \begin{equation*}
    \theta_n \coloneqq \tfrac k n \pi, k = -n, -n+1, \ldots, n-2, n-1.
  \end{equation*}
  the extrema of \( c(\theta) \) in \( [-\pi, \pi) \).

  Define the auxiliary function
  \begin{equation*}
    r(\theta) \coloneqq c(\theta) - s(\theta - \theta_c + \theta_s).
  \end{equation*}

  We now have \( r(\theta_c) = c(\theta_c) - s(\theta_s) = 0 \).

  Furthermore, since \( \norm{s} = 1 \), then \( \abs{s(\theta)} \leq 1 \) for all \( \theta \in [-\pi, \pi) \). Therefore \( r(\theta_k) \leq 0 \) for all odd \( k \) and \( r(\theta_k) \geq 0 \) for all even \( k \).

  If \( \theta_c \) coincides with any of the extrema \( \theta_k \), then \( r(\theta_c) \) holds trivially. Suppose that \( \theta_c \) is between \( \theta_{k-1} \) and \( \theta_k \). Without loss of generality, assume that \( k \) is even.

  By the intermediate value theorem, there exists a zero of \( r \) between \( r(\theta_c) \) and \( r(\theta_k) \). If \( r(\theta_c) < 0 \), then \( \theta_c \) is a local minimum and hence there exists a point \( \theta_{c'} \) between \( r(\theta_{k-1}) \) and \( \theta_c \) such that \( r(\theta_{c'}) = r(\theta_c) = 0 \). But this would imply that \( r \) has more than \( 2n \) different roots in the interval \( [-\pi, \pi) \), which is a contradiction.
\end{proof}

\begin{corollary}[Bernstein's trigonometric inequality]\label{thm:bernsteins_trigonometric_inequality}\mcite[53]{Николов2020Лекции}
  For any nonnegative integer \( n \) and any \( s \in \tau_n \) we have
  \begin{equation}\label{eq:thm:bernsteins_trigonometric_inequality}
    \abs{s'(\theta)} \leq n \norm{s} \quad\forall \theta \in [-\pi, \pi].
  \end{equation}
\end{corollary}
\begin{proof}
  The case \( n = 0 \) is trivial. If \( n > 0 \), for any \( s \in \tau_n \), we can apply \eqref{eq:thm:szegos_trigonometric_inequality} to \( \frac s {\norm s} \) to obtain
  \begin{equation}
    [s'(\theta)]^2 + n^2 s^2 (\theta) \leq \norm{s}^2 n^2 \quad\forall \theta \in [-\pi, \pi].
  \end{equation}

  \eqref{eq:thm:bernsteins_trigonometric_inequality} follows directly.
\end{proof}

\begin{theorem}[Bernstein's trigonometric theorem]\label{thm:bernsteins_trigonometric_theorem}\mcite[55]{Николов2020Лекции}
  Let \( f \in \tilde(C)[-\pi, \pi] \) and
  \begin{equation*}
    \tilde{E}_n(f) \leq \frac A {n^{k+\alpha}} \quad n = 0, 1, 2, \ldots,
  \end{equation*}
  where \( A \in \BbbR \) and \( \alpha \in (0, 1) \).

  Then \( f \in C^{(k)}[-\pi, \pi] \) and \( f^{(k)} \) is \( \alpha \)-H\"older.
\end{theorem}
\begin{proof}
  Since \( \tilde{E}_n(f) \) is bounded by \( \frac A {n^{k+\alpha}} \) on a compact interval, there exists a sequence \( \{ s_k \}_{k=0}^\infty \) such that
  \begin{equation*}
    \norm{f - s_n} \leq A {n^{k+\alpha}}.
  \end{equation*}

  Define the sequence of polynomials
  \begin{equation*}
    v_j \coloneqq \begin{cases}
      s_1,                  &j = 0, \\
      s_{2^j} - s_{2^{j-1}} &j > 0.
    \end{cases}
  \end{equation*}

  It is now clear that
  \begin{equation*}
    \norm{f - \sum_{j=0}^n v_j} \xrightarrow[]{j \to \infty} 0
  \end{equation*}
  because
  \begin{equation*}
    \abs{f(\theta) - \sum_{j=0}^n v_j (\theta)}
    =
    \abs{f(\theta) - s_{2^n}(\theta)}
    \leq
    A {n^{k+\alpha}}.
  \end{equation*}

  For each term of the series, we have
  \begin{equation*}
    \abs{v_j(\theta)}
    \leq
    \abs{f(\theta) - s_{2^j}(\theta)} + \abs{f(\theta) - s_{2^{j-1}}(\theta)}
    \leq
    \frac A {2^{j(k + \alpha)}} + \frac A {2^{(j - 1) (k + \alpha)}}.
  \end{equation*}

  By setting \( B \coloneqq A (2^{k + \alpha}) \), we obtain
  \begin{equation*}
    \abs{v_j(\theta)} \leq \frac B {2^{j(k + \alpha)}}.
  \end{equation*}

  By \fullref{thm:bernsteins_trigonometric_inequality},
  \begin{equation*}
    \norm{v_j^{(r)}} \leq 2^{jr} \norm{v_j} \leq \frac B {2^{j(k - r + \alpha)}}.
  \end{equation*}

  Then
  \begin{equation*}
    \sum_{j=0}^\infty v_j^{(r)} (\theta)
  \end{equation*}
  converges uniformly, therefore
  \begin{equation*}
    f^{(r)}(\theta) = \sum_{j=0}^\infty v_j^{(r)} (\theta).
  \end{equation*}
\end{proof}

\begin{theorem}[Bernstein's algebraic inequality]\label{thm:bernsteins_algebraic_inequality}\mcite[59]{Николов2020Лекции}
  For any nonnegative integer \( n \) and any \( p \in \pi_n \) and \( x \in (a, b) \) we have
  \begin{equation*}
    \abs{p'(x)} \leq n \frac 1 {(b - a)(b - x)} \norm{p}.
  \end{equation*}
\end{theorem}

\begin{theorem}[Bernstein's algebraic theorem]\label{thm:bernsteins_algebraic_theorem}\mcite[60]{Николов2020Лекции}
  Let \( f \in C[a, b] \) and
  \begin{equation*}
    E_n(f) \leq \frac A {n^{k+\alpha}} \quad n = 0, 1, 2, \ldots,
  \end{equation*}
  where \( A \in \BbbR \) and \( \alpha \in (0, 1) \).

  Then \( f \in C^{(k)}(a, b) \) and \( f^{(k)} \) is \( \alpha \)-H\"older in every \( [a_1, b_1] \) such that \( a_1 > a \) and \( b_1 < b \).
\end{theorem}


% Differential equations
\section{Differential equations}\label{sec:diffeq}
\subsection{Ordinary differential equations}\label{subsec:ordinary_differential_equations}

\begin{remark}\label{rem:real_ode}
  We restrict our attention to differential equations over the real numbers.
\end{remark}

\begin{definition}\label{def:ode}
  Let
  \begin{equation*}
    f: \underset {m \text{ times}} {\BbbR^n \times \cdots \times \BbbR^n} \to \BbbR^n
  \end{equation*}
  be a vector field. Fix \( x_0 \in \BbbR^n \). The \hyperref[ex:equations]{equation}
  \begin{equation*}
    \begin{cases}
       & f(x, x', \ldots, x^{(m)}) = 0 \\
       & x(0) = x_0
    \end{cases}
  \end{equation*}
  is called an (ordinary) \Def{differential equation} of order \( m \). Solutions to the equation are \( m \)-times differentiable functions \( x: \BbbR \to \symbb{R^n} \).
\end{definition}


% General topology
\section{General topology}\label{sec:general_topology}
\subsection{Topological spaces}\label{subsec:topological_spaces}

\begin{definition}\label{def:topological_space}\mcite\cite[11]{Engelking1989}
  Let \( X \) be any set and \( \mscrT \subseteq \pow(X) \) be a family of subsets of \( X \). \( \mscrT \) is called a \term{topology} on \( X \) and the tuple \( (X, \mscrT) \) is said to be a \term{topological space} if the following axioms are satisfied:
  \begin{defenum}
    \iaxiom{def:topological_space/O1}{O1} \( \varnothing, X \in \mscrT \)
    \iaxiom{def:topological_space/O2}{O2} \( U, V \in \mscrT \implies U \cap V \in \mscrT \)
    \iaxiom{def:topological_space/O3}{O3} \( \mscrT' \subseteq \mscrT \implies \bigcap \mscrT' \in \mscrT \)
  \end{defenum}

  If the topology is obvious from the context, we say that \( X \) is a topological space.

  Elements of the set \( X \) are called \term{points} of the topological space, elements of \( \mscrT \) are called \term{open sets} and set-theoretic complements of open sets are called \term{closed sets}.

  If \( x \in U \in \mscrT \), we say that \( U \) is a \term{neighborhood} of \( x \). Note that some authors (e.g. \cite[38]{Kelley1955}) alternatively define neighborhoods as arbitrary sets that contain an open set that contains \( x \). For simplicity, we define the subfamily
  \begin{equation*}
    \mscrT(x) \coloneqq \{ U \in \mscrT \colon x \in U \}.
  \end{equation*}

  We say that \( U \) is a \term{punctured neighborhood} of \( x \) if \( U \cup \{ x \} \) is an open set and, consequently, a neighborhood of \( x \).

  Dually, we can define the family \( \mscrF \) of closed sets, where
  \begin{defenum}
    \iaxiom{def:topological_space/F1}{F1} \( \varnothing, X \in \mscrF \)
    \iaxiom{def:topological_space/F2}{F2} \( U, V \in \mscrF \implies U \cup V \in \mscrF \)
    \iaxiom{def:topological_space/F3}{F3} \( \mscrF' \subseteq \mscrF \implies \bigcup \mscrF' \in \mscrF \)
  \end{defenum}

  If \( (X, \mscrT) \) is a topological space, we denote the corresponding family of closed sets by
  \begin{equation*}
    \mscrF_\mscrT \coloneqq \{ X \setminus U \colon U \in \mscrT \}.
  \end{equation*}
\end{definition}

\begin{definition}\label{def:standard_topologies}
  On a space \( X \), we can explicitly define the following standard topologies:
  \begin{defenum}
    \ilabel{def:standard_topologies/discrete} The \term{discrete topology} \( \mscrT \coloneqq \pow(X) \).
    \ilabel{def:standard_topologies/indiscrete} The \term{indiscrete topology} \( \mscrT \coloneqq \{ \varnothing, X \} \).
    \ilabel{def:standard_topologies/co_cardinal} For any \hyperref[def:cardinal]{cardinal} \( \xi \), the \term{co-\( \xi \) topology} \( \mscrT \coloneqq \{ A \subseteq X \colon \card A < \xi \} \) and, in particular, \term{cofinite} (\( \xi = \aleph_0 \)) and \term{cocountable} (\( \xi = c \)) topologies.
  \end{defenum}

  For a deeper connection between discrete and indiscrete topologies, see \fullref{ex:top_adjoint_functor}.
\end{definition}

\begin{proposition}\label{thm:set_open_iff_neighborhood_is_contained}
  A set \( A \) is open if and only if every point of \( A \) has a neighborhood \( U \) such that \( U \subseteq A \).
\end{proposition}
\begin{proof}
  This holds vacuously for empty sets. Assume that \( A \subseteq X \) is nonempty.

  \Sufficiency Assume that \( A \) is open and let \( x_0 \in A \). Then \( A \) is a neighborhood of \( x_0 \) and the theorem holds trivially.
  \Necessity Assume that every point \( x \in A \) has a neighborhood \( U_x \) such that \( U_x \subseteq A \). Take the union
  \begin{equation*}
    B \coloneqq \cup_{x \in A} U_x.
  \end{equation*}

  Obviously \( B \subseteq A \). Aiming at a contradiction, suppose\LEM that \( y_0 \in A \setminus B \). Then \( y_0 \) has a neighborhood \( U_{y_0} \) such that \( U_{y_0} \setminus B \) is nonempty. But this is impossible by the definition of \( B \). The obtained contradiction proves \( B = A \).
\end{proof}

\begin{remark}\label{rem:abritrary_family_to_topology}
  It is sometimes easier to define a topology \( \mscrT \) via a subset of \( \mscrT \). We will gradually construct a topology from a bare family of sets in \( X \). First, we will give two definitions for a base, one on which does not require an existing topology.
\end{remark}

\begin{definition}\label{def:topological_base}\mcite\cite[12]{Engelking1989}
  Fix a topological space \( (X, \mscrT) \). We say that the family \( \mscrB \subseteq \mscrT \) is a \term{base} for the topology \( \mscrT \) if \( \mscrB \) satisfies any of the equivalent conditions:
  \begin{defenum}
    \ilabel{def:topological_base/union} Every open set \( U \in \mscrT \) is the union \( U = \bigcup \mscrB' \) of some subset \( \mscrB' = \mscrB \)
    \ilabel{def:topological_base/subset} For any point \( x \in X \) and for any neighborhood \( U \) of \( x \) there exists a set \( V \in \mscrB \) in the base such that \( x \in V \subseteq U \)
  \end{defenum}
\end{definition}
\begin{proof}
  \SubProofImplication{def:topological_base/union}{def:topological_base/subset} Fix a point \( x \in X \) and a neighborhood \( U \in \mscrT \) of \( x \). Let \( \mscrB' \) be a subfamily of \( \mscrB \) such that
  \begin{equation*}
    U = \bigcup \mscrB'.
  \end{equation*}

  Then \( x \in V \) for at least one \( V \in \mscrB' \).

  \SubProofImplication{def:topological_base/subset}{def:topological_base/union} Fix an open set \( U \in \mscrT \). Then for every \( x \in U \), there exists a set \( V_x \in \mscrB \) such that \( x \in V_x \subseteq U \). We have
  \begin{equation*}
    \bigcup_{x \in U} V_x \subseteq U \subseteq \bigcup_{x \in U} V_x,
  \end{equation*}
  thus
  \begin{equation*}
    U = \bigcup_{x \in U} V_x.
  \end{equation*}
\end{proof}

\begin{proposition}\label{thm:topological_base_axioms}\mcite\cite[12]{Engelking1989}
  Let \( X \) be an arbitrary set and let \( \mscrB \) be a family of subset that satisfies
  \begin{propenum}
    \iaxiom{thm:topological_base_axioms/B1}{B1} \( \bigcup \mscrB = X \)
    \iaxiom{thm:topological_base_axioms/B2}{B2} \( \forall U, V \in \mscrB, \forall x \in U \cap V, \exists W \in \mscrB: x \in W \subseteq U \cap V \)
  \end{propenum}

  Then the family
  \begin{balign}\label{thm:topological_base_axioms/topology}
    \mscrT \coloneqq \left\{ \bigcup \mscrB' \colon \mscrB' \subseteq \mscrB \right\}
  \end{balign}
  is a topology on \( X \). Furthermore, \( \mscrB \) is a \hyperref[def:topological_base]{base} of \( \mscrT \).

  In particular, the base on any topology satisfies \fullref{thm:topological_base_axioms/B1} -- \fullref{thm:topological_base_axioms/B2}.
\end{proposition}
\begin{proof}
  We will first prove that \( \mscrT \) is indeed a topology.

  \begin{reflist}
    \iref{def:topological_space/O1} \( \varnothing = \bigcup \varnothing \in \tau \) and \( X = \bigcup \mscrB \in \mscrT \) (by \fullref{thm:topological_base_axioms/B1})

    \iref{def:topological_space/O3} Fix \( \mscrT' = \{ U_\alpha \colon \alpha \in A \} \subseteq \mscrT \). By \fullref{def:topological_base/union}, every set \( U_\alpha \) has a corresponding subfamily \( \mscrB_\alpha \) of \( \mscrB \) such that \( U_\alpha = \bigcup \mscrB_\alpha \).

    Define \( \mscrB' \coloneqq \bigcup_{\alpha \in A} \mscrB_\alpha \). Obviously \( \mscrB' \subseteq \mscrB \) and thus, by \fullref{thm:topological_base_axioms/B1}, \( \bigcup \mscrB \in \mscrT \).

    \iref{def:topological_space/O2} Fix \( U, V \in \mscrT \) and families \( \mscrB_U, \mscrB_V \subseteq \mscrB \) such that \( U = \bigcup \mscrB_U \) and \( V = \bigcup \mscrB_V \).

    Fix arbitrary \( U' \in \mscrB_U \) and \( V' \in \mscrB_V \). We will show that \( U' \cap V' \in \tau \).

    By \fullref{thm:topological_base_axioms/B2}, for every \( x \in U' \cap V' \) there exists a neighborhood \( W_x \) of \( x \) such that \( W \subseteq U' \cap V' \).

    The family \( \mscrB_{U',V'} \coloneqq \{ W_x \colon x \in U' \cap V' \} \)\AOC is a subfamily of \( \mscrB \) and thus \( U' \cap V' = \bigcup \mscrB_{U',V'} \in \mscrT \).

    Hence, by \fullref{def:topological_space/O3}, \( U \cap V \in \tau \).

    Now, for any \( U \in \mscrT \), by \fullref{thm:topological_base_axioms/topology}, there exists a subfamily \( \mscrB' \subseteq \mscrB \) such that
    \begin{equation*}
      U = \bigcup \mscrB'.
    \end{equation*}

    Hence \( \mscrB \) is a base for \( \mscrT \).
  \end{reflist}
\end{proof}

\begin{definition}\label{def:topological_space_weight}
  We define the \term{weight} of \( (X, \mscrT) \) as the cardinal
  \begin{equation*}
    w((X, \mscrT)) \coloneqq \min \{ \abs{\mscrB} \colon \mscrB \text{ is a base for } \mscrT \}.
  \end{equation*}

  We simply write \( w(X) \) when the topology is clear from the context.

  Spaces for which \( w(X) \leq \aleph_0 \) are said to be \term{second-countable}.
\end{definition}
\begin{proof}
  The definition is correct because of \fullref{thm:cardinals_well_ordered}.
\end{proof}

\begin{definition}\label{def:topological_subbase}\mcite\cite[12]{Engelking1989}
  Fix a topological space \( (X, \mscrT) \). We say that the family \( \mscrP \subseteq \mscrT \) is a \term{subbase} for the topology \( \mscrT \) if the family
  \begin{equation*}
    \mscrB \coloneqq \left\{ \bigcap P' \colon P' \text{ is a nonempty \hyperref[def:finite_set]{finite} subset of } P \right\}
  \end{equation*}
  of finite intersections of \( \mscrP \) is a \hyperref[def:topological_base]{base} of \( \mscrT \).
\end{definition}

\begin{proposition}\label{thm:subbase_from_arbitrary_family}
  Fix a set \( X \) and a family of subsets \( \mscrP \subseteq \pow(X) \). The family \( \mscrP' \coloneqq \mscrP \cup X \) is then a \hyperref[def:topological_subbase]{subbase} of some topology on \( X \).
\end{proposition}

\begin{definition}\label{def:topological_local_base}\mcite\cite[12]{Engelking1989}
  Fix a topological space \( (X, \mscrT) \) and a point \( x \in X \). We say that the family \( \mscrB(x) \subseteq \mscrT \) is a \term{local base} for \( \mscrT \) at \( x \) if every neighborhood of \( x \) contains a set from \( \mscrB(x) \).

  Given a base \( \mscrB \), unless explicitly noted, we consider the subfamily \( \mscrB(x) \) of all members of \( \mscrB \) containing \( x \).

  The indexed family of local bases \( \{ \mscrB(x) \colon x \in X \} \) is called a \term{neighborhood system} of \( \mscrT \).
\end{definition}

\begin{proposition}\label{thm:neighborhood_iff_union_in_topological_local_base}
  Analogously to \fullref{def:topological_base/union}, a set \( A \) containing \( x \) is a neighborhood of \( x \) if and only if \( A \) is a union of elements of the local \hyperref[def:topological_local_base]{base} \( \mscrB(x) \).
\end{proposition}
\begin{proof}
  Analogous to the proof of the equivalence in \fullref{def:topological_base}.
\end{proof}

\begin{proposition}\label{thm:topological_local_base_axioms}\mcite\cite[13]{Engelking1989}
  Let \( X \) be an arbitrary set and let \( \{ \mscrB(x) \subseteq \pow(X) \colon x \in X \} \) be an indexed family of families of subsets of \( X \) that satisfies
  \begin{defenum}
    \iaxiom{thm:topological_local_base_axioms/BP1}{BP1} For every \( x \in X \), \( \mscrB(x) \neq \varnothing \) and \( x \in U \) for every \( U \in \mscrB(x) \).
    \iaxiom{thm:topological_local_base_axioms/BP2}{BP2} For every \( x \in X \) and for all \( U, V \in \mscrB(x) \), \( \exists W \in \mscrB(x): W \subseteq U \cap V \).
    \iaxiom{thm:topological_local_base_axioms/BP3}{BP3} For all \( x, y \in X \), \( x \in U \in \mscrB(y) \) implies that there exists \( V \in \mscrB(x) \) such that \( U \subseteq V \).
  \end{defenum}

  Then the family
  \begin{equation*}
    \mscrB \coloneqq \bigcup_{x \in X} \mscrB(x)
  \end{equation*}
  is the \hyperref[thm:topological_base_axioms]{base} of some topology \( \mscrT \) on \( X \). Furthermore, \( \{ \mscrB(x) \subseteq \pow(X) \colon x \in X \} \) is a \hyperref[def:topological_local_base]{neighborhood system} for \( (X, \mscrT) \).

  In particular, the local base on any topology satisfies \fullref{thm:topological_local_base_axioms/BP1} -- \fullref{thm:topological_local_base_axioms/BP3}.
\end{proposition}

\begin{definition}\label{def:topological_space_character}
  We define the \term{character} of the point \( x \in X \) as the cardinal
  \begin{equation*}
    \chi(x) \coloneqq \min \{ \card \mscrB(x) \colon \mscrB(x) \text{ is a local base for } \mscrT \text{ at } x \}.
  \end{equation*}

  We define the \term{character} of of \( (X, \mscrT) \) as
  \begin{equation*}
    \chi((X, \mscrT)) \coloneqq \sup \{ \chi(x) \colon x \in X \}.
  \end{equation*}

  We simply write \( \chi(X) \) when the topology is clear from the context.

  Spaces for which \( \chi(X) \leq \aleph_0 \) are said to be \term{first-countable}.
\end{definition}
\begin{proof}
  The character of a point is well defined by \fullref{thm:cardinals_well_ordered}. The character of a topological space is also well defined since by \fullref{thm:equinumerous_ordinal_existence} there is at least one upper bound for the characters of all points and by \fullref{thm:cardinals_well_ordered} this set has a least element.
\end{proof}

\begin{definition}\label{def:topological_local_subbase}
  Combining \fullref{def:topological_subbase} and \fullref{def:topological_local_base}, we define a \term{local subbase} for \( \mscrT \) at \( x \) to be a family \( \mscrP(x) \subseteq \mscrT \) such that every neighborhood \( U \) of \( x \) contains a finite intersection of sets from \( \mscrP(x) \).

  Given a subbase \( \mscrP \), unless explicitly noted, we consider the subfamily \( \mscrP(x) \) of all members of \( \mscrP \) containing \( x \).
\end{definition}

\begin{definition}\label{def:closure_operator}\mcite\cite[33]{Engelking1989}
  Let \( (X, \mscrT) \) be a topological space. Define the \term{closure operator}
  \begin{balign*}
     & \cl: \pow(X) \to \pow(X)                                           \\
     & \cl(A) \coloneqq \bigcap \{ F : F \in \mscrF_\mscrT, A \subseteq F \}.
  \end{balign*}
\end{definition}

\begin{proposition}\label{thm:closure_operator_properties}
  The closure \hyperref[def:closure_operator]{operator} has the following basic properties
  \begin{propenum}
    \ilabel{thm:closure_operator_properties/closed} The set \( A \) is closed if and only if \( A = \cl A \).
    \ilabel{thm:closure_operator_properties/neighborhood_intersection} For any \( x \in X \), \( x \in \cl A \) if and only if every neighborhood of \( x \) intersects \( A \).
    \ilabel{thm:closure_operator_properties/monotone} \( \cl \) is \hyperref[def:monotone_map]{monotone}, i.e. if \( A \subseteq B \), then \( \cl(A) \subseteq \cl(B) \).
  \end{propenum}
\end{proposition}
\begin{proof}
  \SubProofOf{thm:closure_operator_properties/closed} The condition \( A = \cl{A} \) is equivalent to \( A \) being a closed superset of itself, which is equivalent to \( A \) being closed.

  \SubProofOf{thm:closure_operator_properties/neighborhood_intersection} Note that this proof relies on \fullref{def:topological_boundary}, however we do not use this property when defining the boundary.

  \Sufficiency Fix \( x \in \cl{A} \) and let \( U \) be a neighborhood of \( x \). If \( x \in A \), then obviously \( x \in U \cap A \neq \varnothing \). If \( x \not\in A \), then \( U \cap A \neq \varnothing \) by \fullref{def:topological_boundary/neighborhoods}. In both cases, we obtain \( U \cap A \neq \varnothing \), which proves the statement.

  \Necessity Fix \( x \in X \) and assume that every neighborhood of \( x \) intersects \( A \). Since the case \( x \in A \) is trivial, suppose that \( x \not\in A \). By \fullref{thm:set_open_iff_neighborhood_is_contained}, every neighborhood \( U \) of \( x \) does not entirely belong to \( A \). By \fullref{def:topological_boundary/neighborhoods}, \( x \in \fr A \subseteq \cl A \).

  \SubProofOf{thm:closure_operator_properties/monotone} If \( A \subseteq B \), every closed superset of \( B \) is also a closed superset of \( A \).
\end{proof}

\begin{proposition}\label{thm:closure_operator_axioms}\mcite\cite[14]{Engelking1989}
  Let \( X \) be an arbitrary set and let \( \cl: \pow(X) \to \pow(X) \) be a function that satisfies
  \begin{propenum}
    \iaxiom{thm:closure_operator_axioms/CO1}{CO1} \( \cl(\varnothing) = \varnothing \)
    \iaxiom{thm:closure_operator_axioms/CO2}{CO2} \( \forall A \in \pow(X), A \subseteq \cl(A) \)
    \iaxiom{thm:closure_operator_axioms/CO3}{CO3} \( \forall A, B \in \pow(X), \cl(A \cup B) = \cl(A) \cup \cl(B) \)
    \iaxiom{thm:closure_operator_axioms/CO4}{CO4} \( \forall A \in \pow(X), \cl(\cl(A)) = \cl(A) \)
  \end{propenum}

  Then the family
  \begin{equation*}
    \mscrT \coloneqq \{ X \setminus F \colon F = \cl(F) \}
  \end{equation*}
  is a topology on \( X \). Furthermore, \( \cl = \cl^{\mscrT} \), where \( \cl^{\mscrT} \) is the closure \hyperref[def:closure_operator]{operator} on \( (X, \mscrT) \).

  In particular, the closure operator on any topology satisfies \fullref{thm:closure_operator_axioms/CO1} -- \fullref{thm:closure_operator_axioms/CO4}.
\end{proposition}

\begin{definition}\label{def:interior_operator}\mcite\cite[15]{Engelking1989}
  Let \( (X, \mscrT) \) be a topological space. Define the \term{interior operator}
  \begin{balign*}
     & \inter: \pow(X) \to \pow(X)                                     \\
     & \inter(A) \coloneqq \bigcup \{ U : U \in \mscrT, U \subseteq A \}.
  \end{balign*}
\end{definition}

\begin{proposition}\label{thm:interior_operator_properties}
  The interior \hyperref[def:interior_operator]{operator} has the following basic properties
  \begin{propenum}
    \ilabel{thm:interior_operator_properties/open} A set \( A \) is a topological space is open if and only if \( A = \int A \).
    \ilabel{thm:interior_operator_properties/monotone} \( \int \) is \hyperref[def:monotone_map]{monotone}, i.e. if \( A \subseteq B \), then \( \inter(A) \subseteq \inter(B) \).
  \end{propenum}
\end{proposition}
\begin{proof}
  \SubProofOf{thm:interior_operator_properties/open} Follows from \fullref{thm:closure_operator_properties/closed} and \fullref{thm:interior_closure_complement}.
  \SubProofOf{thm:interior_operator_properties/monotone} Follows from \fullref{thm:closure_operator_properties/monotone} and \fullref{thm:interior_closure_complement}.
\end{proof}

\begin{proposition}\label{thm:interior_closure_complement} For every set \( A \subseteq X \) we have
  \begin{itemize}
    \item \( X \setminus \inter(A) = \cl(X \setminus A) \)
    \item \( X \setminus \cl(A) = \inter(X \setminus A) \)
  \end{itemize}
\end{proposition}
\begin{proof}
  Any open subset \( U \subseteq A \) is a closed superset of \( X \setminus A \). A point belongs to \( \inter(A) \) if it belongs to at least one open subset of \( A \), which happens if and only if it belongs to at least one closed superset of \( X \setminus A \). Therefore
  \begin{balign*}
    X \setminus \inter(A)
     & =
    X \setminus \bigcup \{ U : U \in \mscrT, U \subseteq A \}
    =                                            \\ &=
    X \setminus \bigcup \{ F : F \in \mscrF_\mscrT, X \setminus A \subseteq F \}
    \overset {X \setminus (X \setminus A) = A} = \\ &=
    \bigcup \{ F : F \in \mscrF_\mscrT, F \subseteq A \}.
    =                                            \\ &=
    \cl(A).
  \end{balign*}

  The other equality is obtained by noting that \( X \setminus \cl(A) = X \setminus (X \setminus \inter(A)) = \inter(A) \).
\end{proof}

\begin{proposition}\label{thm:interior_operator_axioms}
  Let \( X \) be an arbitrary set and let \( \int: \pow(X) \to \pow(X) \) be a function that satisfies
  \begin{propenum}
    \iaxiom{thm:interior_operator_axioms/IO1}{IO1} \( \inter(X) = X \)
    \iaxiom{thm:interior_operator_axioms/IO2}{IO2} \( \forall A \in \pow(X), \inter(A) \subseteq A \)
    \iaxiom{thm:interior_operator_axioms/IO3}{IO3} \( \forall A, B \in \pow(X), \inter(A \cap B) = \inter(A) \cap \inter(B) \)
    \iaxiom{thm:interior_operator_axioms/IO4}{IO4} \( \forall A \in \pow(X), \inter(\inter(A)) = \inter(A) \)
  \end{propenum}

  Then the family
  \begin{equation*}
    \mscrT \coloneqq \{ U \colon U = \inter(U) \}
  \end{equation*}
  is a topology on \( X \). Furthermore, \( \int = \int_\mscrT \), where \( \int_\mscrT \) is the interior \hyperref[def:interior_operator]{operator} on \( (X, \mscrT) \).

  In particular, the interior operator on any topology satisfies \fullref{thm:interior_operator_axioms/IO1} -- \fullref{thm:interior_operator_axioms/IO4}.
\end{proposition}

\begin{definition}\label{def:topological_boundary}
  For a subset \( A \) of a topological space we define its \term{boundary} \( \fr(A) \) equivalently as
  \begin{defenum}
    \ilabel{def:topological_boundary/closure} \( \fr(A) \coloneqq \cl(A) \setminus \inter(A) \)
    \ilabel{def:topological_boundary/neighborhoods} \( \fr(A) \) is the set of all points \( x \in X \) such that every neighborhood of \( x \) intersects both \( A \) and \( X \setminus A \).
  \end{defenum}
\end{definition}
\begin{proof}
  The equivalence of the definitions is trivial when \( \fr(A) = \varnothing \). We assume that \( \fr(A) \neq \varnothing \).

  \SubProofImplication{def:topological_boundary/closure}{def:topological_boundary/neighborhoods} Let \( x \in \cl(A) \setminus \inter(A) \).

  Aiming for a contradiction, suppose\LEM that there is a neighborhood \( U \) of \( x \) that does not intersect \( A \). Then \( U \subseteq X \setminus A \). Hence \( A \subseteq X \setminus U \). Since \( X \setminus U \) is closed, it follows that \( \cl(A) \subseteq X \setminus U \) as the intersection of all closed supersets of \( A \). But \( x \not\in X \setminus U \), therefore \( x \not\in \cl(A) \), which contradicts our choice of \( x \in \cl(A) \).

  This proves that every neighborhood of \( x \) intersects \( A \).

  By passing to complements, we can reuse this to prove that every neighborhood of \( x \) intersects \( X \setminus A \) using \fullref{thm:interior_closure_complement}.

  \SubProofImplication{def:topological_boundary/neighborhoods}{def:topological_boundary/closure} Suppose that every neighborhood of \( x \in \fr(A) \) intersects both \( A \) and \( X \setminus A \). Therefore no neighborhood of \( x \) is contained in neither \( A \) not \( X \setminus A \) and \( x \) belongs to neither \( \inter(A) \) nor \( \inter(X \setminus A) \). Hence
  \begin{equation*}
    x \in (X \setminus \inter(X \setminus A)) \setminus \inter(A) \overset {\ref{thm:interior_closure_complement}} = \cl(A) \setminus \inter(A).
  \end{equation*}
\end{proof}

\begin{proposition}\label{thm:topological_boundary_properties}
  The \hyperref[def:topological_boundary]{topological boundary} has the following basic properties
  \begin{propenum}
    \ilabel{thm:topological_boundary_properties/closed} \( \fr(A) \) is a closed set.
    \ilabel{thm:topological_boundary_properties/not_open} If \( \fr(A) \) is not empty, it is not an open set.
    \ilabel{thm:topological_boundary_properties/complement} \( \fr(A) = \fr(X \setminus A) \).
  \end{propenum}
\end{proposition}
\begin{proof}
  \SubProofOf{thm:topological_boundary_properties/closed} Note that
  \begin{equation*}
    \fr(A) = \cl(A) \setminus \inter(A) = \cl(A) \cap (X \setminus \inter(A)),
  \end{equation*}
  which is the intersection of two closed sets. Hence \( \fr(A) \) is itself a closed set.

  \SubProofOf{thm:topological_boundary_properties/not_open} Note that \( \fr(A) \) is either empty or is not open because \fullref{def:topological_boundary/neighborhoods} is incompatible with \fullref{thm:set_open_iff_neighborhood_is_contained}.

  \SubProofOf{thm:topological_boundary_properties/complement} By \fullref{thm:interior_closure_complement},
  \begin{balign*}
    \fr(A)
     & =
    \cl(A) \setminus \inter(A)
    =                                                  \\ &=
    \cl(A) \cap (X \setminus \inter(A))
    \overset {\ref{thm:interior_closure_complement}} = \\ &=
    (X \setminus \inter(X \setminus A)) \cap \cl(X \setminus A)
    =                                                  \\ &=
    \cl(X \setminus A) \setminus \inter(X \setminus A)
    =                                                  \\ &=
    \fr(X \setminus A).
  \end{balign*}
\end{proof}

\begin{definition}\label{def:topological_derived_set}\mcite\cite[24]{Engelking1989}
  Let \( (X, \mscrT) \) be a topological space.

  \begin{defenum}
    \ilabel{def:topological_derived_set/cluster_point} We say that the point \( x_0 \in X \) is a \term{cluster point} or an \term{accumulation point} of the set \( A \subseteq X \) if \( x \in \cl(A \setminus \{ x \}) \). It is not necessary for \( x_0 \) to belong to \( A \).

    \ilabel{def:topological_derived_set/derived_set} The set of all cluster points of \( A \) is called the \term{derived set} of \( A \) and is denoted by \( \derived(A) \).

    \ilabel{def:topological_derived_set/perfect_set} If a set equals its derived set, we call it a \term{perfect set}.

    \ilabel{def:topological_derived_set/isolated_point} Points in \( A \setminus \derived(A) \) are said to be \term{isolated points} of \( A \).

    \ilabel{def:topological_derived_set/discrete_set} If \( \derived(A) = \varnothing \), that is, if \( A \) consists of only discrete points, we say that \( A \) is a \term{discrete set}.
  \end{defenum}
\end{definition}

\begin{proposition}\label{thm:derived_set_properties}
  \hyperref[def:topological_derived_set]{Derived sets} have the following basic properties
  \begin{propenum}
    \ilabel{thm:derived_set_properties/cluster_via_neighborhoods} \( x \) is a cluster point of \( A \) if and only if every neighborhood of \( x \) intersects \( A \setminus \{ x \} \)
    \ilabel{thm:derived_set_properties/isolated_via_neighborhoods} \( x \) is an isolated point of \( A \) if and only if there exists a neighborhood of \( x \) that does not intersect \( A \setminus \{ x \} \)
    \ilabel{thm:derived_set_properties/closed} \( \derived(A) \) is a closed set.
    \ilabel{thm:derived_set_properties/closure} \( A \cup \derived(A) = \cl(A) \).
    \ilabel{thm:derived_set_properties/closed_iff_contains_all_cluster_points} A set is closed if and only if it contains all of its cluster points. Compare this result to \fullref{thm:limit_point_iff_in_closure}.
    \ilabel{thm:derived_set_properties/closed_iff_only_isolated_and_cluster_points} A set if closed if and only if every point is either a cluster point or an isolated point.
  \end{propenum}
\end{proposition}
\begin{proof}
  \SubProofOf{thm:derived_set_properties/cluster_via_neighborhoods} If every neighborhood \( U \) of \( x \in A \) intersects \( A \setminus \{ x \} \), by \fullref{thm:closure_operator_properties/neighborhood_intersection}, \( x \in \cl(A \setminus \{ x \}) \) and \( x \) is therefore a cluster point.

  \SufficiencyOf{thm:derived_set_properties/isolated_via_neighborhoods} Dual to \fullref{thm:derived_set_properties/cluster_via_neighborhoods}.

  \SubProofOf{thm:derived_set_properties/closed} Consider the complement of \( \derived(A) \). If it is empty, \( \derived(A) \) is trivially closed. Otherwise, let \( x \in X \setminus \derived(A) \).

  \begin{itemize}
    \item If \( x \) is an isolated point of \( A \), by \fullref{thm:derived_set_properties/isolated_via_neighborhoods} there exists a neighborhood of \( x \) that does not intersect \( A \setminus \{ x \} \).
    \item If \( x \) is not a point of \( A \), aiming at a contradiction, assume\LEM that every neighborhood of \( x \) intersects \( A \). Then, by \fullref{def:topological_boundary/neighborhoods}, \( x \in \fr(A) \). But \( \fr(A) \subseteq \cl(A) \) and \( \cl(A) = \cl(A \setminus \{ x \}) \) because \( x \) does not belong to \( A \). Therefore, \( x \) is a cluster point of \( A \). This contradicts our assumption that \( x \not\in \derived(A) \), hence we can conclude that there exists a neighborhood of \( X \) that does not intersect \( A = A \setminus \{ x \} \).
  \end{itemize}

  In both cases, \fullref{thm:set_open_iff_neighborhood_is_contained} allows us to conclude that \( X \setminus \derived(A) \) is open and, hence, \( \derived(A) \) is closed.

  \SubProofOf{thm:derived_set_properties/closure} Clearly \( A \subseteq \cl(A) \). Also
  \begin{equation*}
    \derived(A) \subseteq \bigcup_{x \in X} \cl(A \setminus \{ x \}) \subseteq \cl(A).
  \end{equation*}

  Now we will prove the reverse inclusion. Let \( x \in \cl(A) \). Then either \( x \in A \) or \( x \in \fr(A) \). Assume the latter. By \fullref{def:topological_boundary/neighborhoods}, every neighborhood \( U \) of \( x \) has points both in \( A \) and outside of \( A \), therefore \( U \cap (A \setminus \{ x \}) \) is nonempty. By \fullref{thm:closure_operator_properties/neighborhood_intersection}, \( x \in \cl(A \setminus \{ x \}) \), that is, \( x \in \derived(A) \).

  \SubProofOf{thm:derived_set_properties/closed_iff_contains_all_cluster_points}
  If \( A \) is closed, by \fullref{thm:derived_set_properties/closure},
  \begin{equation*}
    A \cup \derived(A) = \cl(A) = A,
  \end{equation*}
  hence \( \derived(A) \subseteq A \).

  \NecessityOf{thm:derived_set_properties/isolated_via_neighborhoods} Dual to \fullref{thm:derived_set_properties/cluster_via_neighborhoods}.

  \SubProofOf{thm:derived_set_properties/closed} Consider the complement of \( \derived(A) \). If it is empty, \( \derived(A) \) is trivially closed. Otherwise, let \( x \in X \setminus \derived(A) \).

  \begin{itemize}
    \item If \( x \) is an isolated point of \( A \), by \fullref{thm:derived_set_properties/isolated_via_neighborhoods} there exists a neighborhood of \( x \) that does not intersect \( A \setminus \{ x \} \).
    \item If \( x \) is not a point of \( A \), aiming at a contradiction, assume\LEM that every neighborhood of \( x \) intersects \( A \). Then, by \fullref{def:topological_boundary/neighborhoods}, \( x \in \fr(A) \). But \( \fr(A) \subseteq \cl(A) \) and \( \cl(A) = \cl(A \setminus \{ x \}) \) because \( x \) does not belong to \( A \). Therefore, \( x \) is a cluster point of \( A \). This contradicts our assumption that \( x \not\in \derived(A) \), hence we can conclude that there exists a neighborhood of \( X \) that does not intersect \( A = A \setminus \{ x \} \).
  \end{itemize}

  In both cases, \fullref{thm:set_open_iff_neighborhood_is_contained} allows us to conclude that \( X \setminus \derived(A) \) is open and, hence, \( \derived(A) \) is closed.

  \SubProofOf{thm:derived_set_properties/closure} Clearly \( A \subseteq \cl(A) \). Also
  \begin{equation*}
    \derived(A) \subseteq \bigcup_{x \in X} \cl(A \setminus \{ x \}) \subseteq \cl(A).
  \end{equation*}

  Now we will prove the reverse inclusion. Let \( x \in \cl(A) \). Then either \( x \in A \) or \( x \in \fr(A) \). Assume the latter. By \fullref{def:topological_boundary/neighborhoods}, every neighborhood \( U \) of \( x \) has points both in \( A \) and outside of \( A \), therefore \( U \cap (A \setminus \{ x \}) \) is nonempty. By \fullref{thm:closure_operator_properties/neighborhood_intersection}, \( x \in \cl(A \setminus \{ x \}) \), that is, \( x \in \derived(A) \).

  \SubProofOf{thm:derived_set_properties/closed_iff_contains_all_cluster_points} Assume that \( \derived(A) \subseteq A \) and, aiming at a contradiction, suppose that \( A \) is not closed. Fix a point \( x \in \cl(A) \setminus A \). By \fullref{thm:derived_set_properties/closure}, this is a cluster point. By \fullref{thm:derived_set_properties/cluster_via_neighborhoods}, every for neighborhood \( U \) of \( x \) the intersection \( U \cap (A \setminus \{ x \}) \subseteq U \cap A \) is nonempty. Since this holds for arbitrary neighborhoods, by \fullref{thm:closure_operator_properties/neighborhood_intersection}, \( A \) is closed.

  \SufficiencyOf{thm:derived_set_properties/closed_iff_only_isolated_and_cluster_points}
  Special case of \fullref{thm:derived_set_properties/closed_iff_contains_all_cluster_points}.
  \NecessityOf{thm:derived_set_properties/closed_iff_only_isolated_and_cluster_points} We already know from \fullref{thm:derived_set_properties/closed_iff_contains_all_cluster_points} that it is sufficient for \( \derived(A) \) to belong to \( A \) for \( A \) to be closed. But \( A \setminus \derived(A) \) consists of all isolated points, therefore every point in \( A \) is either a cluster point or an isolated point.
\end{proof}

\begin{definition}\label{def:topologically_dense_set}\mcite\cite[25]{Engelking1989}
  Let \( (X, \mscrT) \) be a topological space and \( A \subseteq X \) be any set. We say that \( A \) is

  \begin{defenum}
    \ilabel{def:topologically_dense_set/dense} \term{dense} in \( X \) if \( \cl{A} = X \) (if \( X \) is assumed from the context, we simply say that \( A \) is dense).

    \ilabel{def:topologically_dense_set/codense} \term{codense} in \( X \) if \( X \setminus A \) is dense, i.e. \( \cl(X \setminus A) = X \).

    \ilabel{def:topologically_dense_set/nowhere_dense} \term{nowhere dense} in \( X \) if \( \cl(A) \) is codense, i.e. \( X = \cl(X \setminus \cl A) \overset {\ref{thm:interior_closure_complement}} = \cl(\inter(X \setminus A)) \).

    \ilabel{def:topologically_dense_set/dense_in_itself} \term{dense in itself} if \( A \subseteq \derived(A) \), i.e. if \( A \) has no isolated points.
  \end{defenum}

  We define the \term{density} \( d(X) \) of \( X \) to be the minimum \hyperref[def:cardinal]{cardinality} of all dense sets. If \( d(X) \leq \aleph_0 \), we say that the space is \term{separable}.
\end{definition}

\begin{proposition}\label{thm:dense_set_properties}
  \hyperref[def:topologically_dense_set/dense]{Dense sets} have the following basic properties:
  \begin{propenum}
    \ilabel{thm:dense_set_properties/open_intersection}\mcite\cite[prop. 1.3.5]{Engelking1989} The set \( A \) is dense if and only if every nonempty open set intersects \( A \).
  \end{propenum}
\end{proposition}
\begin{proof}
  \SubProofOf{thm:dense_set_properties/open_intersection} Special case of \fullref{thm:closure_operator_properties/neighborhood_intersection}.
\end{proof}

\begin{proposition}\label{thm:nowhere_dense_properties}
  \hyperref[def:topologically_dense_set/nowhere_dense]{Nowhere dense sets} have the following basic properties:
  \begin{propenum}
    \ilabel{thm:nowhere_dense_properties/empty_interior} Nowhere dense sets have an empty interior
    \ilabel{thm:nowhere_dense_properties/contained_in_boundary} Nowhere dense sets are entirely contained in their boundaries.
    \ilabel{thm:nowhere_dense_properties/interior_of_closure} The set \( A \) is nowhere dense if and only if \( \inter(\cl(A)) = \varnothing \).
    \ilabel{thm:nowhere_dense_properties/closure_contains_no_open_set} The set is nowhere dense if and only if its closure does not contain a nonempty open set.
    \ilabel{thm:nowhere_dense_properties/open_subset}\mcite\cite[prop. 1.3.5]{Engelking1989}The set \( A \) is nowhere dense if and only if every open set contains a nonempty open subset disjoint from \( A \).
    \ilabel{thm:nowhere_dense_properties/subset} A subset of a nowhere dense set is nowhere dense.
    \ilabel{thm:nowhere_dense_properties/homeomorphism} The \hyperref[def:homeomorphism]{homeomorphic} image of a nowhere dense set is nowhere dense.
    \ilabel{thm:nowhere_dense_properties/complement_dense} A set is closed and nowhere dense if and only if its complement is open and dense.
  \end{propenum}
\end{proposition}
\begin{proof}
  \SubProofOf{thm:nowhere_dense_properties/interior_of_closure} Follows directly from \fullref{thm:interior_closure_complement}.
  \SubProofOf{thm:nowhere_dense_properties/empty_interior} Follows from \fullref{thm:nowhere_dense_properties/interior_of_closure} because \( \inter(A) \subseteq \inter(\cl(A)) = \varnothing \).

  \SubProofOf{thm:nowhere_dense_properties/contained_in_boundary} Follows from \fullref{thm:nowhere_dense_properties/empty_interior} and \fullref{def:topological_boundary/closure}.

  \SubProofOf{thm:nowhere_dense_properties/closure_contains_no_open_set} By \fullref{thm:dense_set_properties/open_intersection}, \( A \) is nowhere dense if and only if every nonempty open set intersects \( X \setminus \cl(A) \overset {\ref{thm:interior_closure_complement}} = \inter(X \setminus A) \). By \fullref{thm:set_open_iff_neighborhood_is_contained}, the last condition is equivalent to every nonempty open set having a nonempty open subset in \( \inter(X \setminus A) = X \setminus \cl(A) \), which in turn implies \fullref{thm:nowhere_dense_properties/closure_contains_no_open_set}.

  \SubProofOf{thm:nowhere_dense_properties/subset} Let \( A \) be a nowhere dense set and let \( B \subseteq A \). Then
  \begin{equation*}
    \inter(\cl(B))
    \overset {\ref{thm:interior_operator_properties/monotone}} \subseteq
    \inter(\cl(A))
    \overset {\ref{thm:nowhere_dense_properties/interior_of_closure}} =
    \varnothing,
  \end{equation*}
  therefore \( B \) is also nowhere dense.

  \SubProofOf{thm:nowhere_dense_properties/homeomorphism} Let \( f: X \to Y \) be a homeomorphic embedding (not necessarily surjective) and let \( A \subseteq X \) be a nowhere dense set. Let \( V \) be an open set in \( Y \). Then \( f^{-1}(V) \) is open in \( X \) and, by \fullref{thm:nowhere_dense_properties/open_subset}, there exists an open subset \( U \subseteq f^{-1}(V) \) that is disjoint from \( A \). Therefore \( f(U) \subseteq f(f^{-1}(V)) \overset {\ref{thm:function_image_preimage_composition/preimage_first}} \subseteq V \). Furthermore, \( f(U) \) is open and \( f(U) \cap f(A) \overset {\ref{thm:function_image_properties/intersection}} = f(U \cap A) = f(\varnothing) = \varnothing \), therefore \( f(A) \) is nowhere dense.

  \SubProofOf{thm:nowhere_dense_properties/complement_dense} If \( A \) is an open dense set, then \( X \setminus A \) is closed and
  \begin{equation*}
    \cl(X \setminus \cl(X \setminus A))
    =
    \cl(X \setminus (X \setminus A))
    =
    \cl(A)
    =
    X,
  \end{equation*}
  therefore \( X \setminus A \) is nowhere dense.
\end{proof}

\begin{definition}\label{def:borel_algebra}
  Fix a topological space \( X \) and \( \mscrF \subseteq \pow(X) \). Denote by \( \mscrF_\delta \) the family of all countable intersections of elements of \( \mscrF \) and by \( \mscrF_\sigma \) the family of all countable unions of elements of \( \mscrF \).

  The family \( F_\delta \) is the family of countable unions of closed sets and \( G_\sigma \) is the family of countable intersections of open sets.
\end{definition}

\subsection{Topological nets}\label{subsec:topological_nets}

In this section, \( X \) will denote an arbitrary nonempty topological space.

\begin{definition}\label{def:topological_net}
  A \term{net} or \term{generalized \hyperref[def:sequence]{sequence}} or \term{Moore-Smith sequence} in a nonempty set \( S \) is a family of elements of \( S \) indexed by a nonempty \hyperref[def:indexed_family]{directed set}, i.e. a function from a nonempty directed \hyperref[def:directed_set]{set} \( (\mscrK, \leq) \) to \( S \). We use the conventional notation for indexed sets (with the caveats described in \fullref{rem:indexed_family_notation}):
  \begin{equation*}
    \{ x_k \}_{k \in \mscrK},
  \end{equation*}
  because the preorder on the domain \( \mscrK \) is usually clear from the context.

  If we know that the net is a sequence, we will usually use the notation for sequences given in \fullref{def:sequence}.

  Note that this definition does not actually require a topology on \( S \). Some other important definitions also do not require topologies:
  \begin{defenum}
    \ilabel{def:topological_net/frequently_in} We say that \( \{ x_k \}_{k \in \mscrK} \) is \term{frequently in} the set \( A \subseteq S \) if for every index \( k_0 \in \mscrK \) there exists an index \( k \geq k_0 \) such that \( x_k \in A \).

    \ilabel{def:topological_net/eventually_in} We say that \( \{ x_k \}_{k \in \mscrK} \) is \term{eventually in} the set \( A \subseteq S \) if there exists an index \( k_0 \) such that \( x_k \in A \) whenever \( k \geq k_0 \). This is obviously a stronger condition.

    \ilabel{def:topological_net/subnet}\mcite\cite[50]{Engelking1989}We say that the net \( \{ y_m \}_{m \in \mscrM} \subseteq S \) is a \term{subnet} of \( \{ x_k \}_{k \in \mscrK} \subseteq S \) if there exists an embedding function \( \varphi: \mscrM \to \mscrK \) such that
    \begin{defenum}
      \ilabel{def:topological_net/subnet/directed} To every \( k \in \mscrK \) there corresponds \( m \in \mscrM \) such that \( \varphi(m) \geq k \).
      \ilabel{def:topological_net/subnet/identity} For every \( m \in \mscrM \) we have \( x_{\varphi(m)} = y_m \).
    \end{defenum}
  \end{defenum}
\end{definition}

\begin{proposition}\label{thm:topological_net_properties}
  \hyperref[def:topological_net]{Nets} have the following basic properties:

  \begin{propenum}
    \ilabel{thm:topological_net_properties/eventually_in_implies_frequently_in} \enquote{\hyperref[def:topological_net/eventually_in]{Eventually in}} implies \enquote{\hyperref[def:topological_net/frequently_in]{frequently in}}.

    \ilabel{thm:topological_net_properties/net_eventually_in_iff_not_frequently_in_complement} The net \( \{ x_k \}_{k \in \mscrK} \subseteq S \) is eventually in \( A \subseteq S \) if and only if it is not frequently in \( S \setminus A \).
  \end{propenum}
\end{proposition}
\begin{proof}
  \SubProofOf{thm:topological_net_properties/eventually_in_implies_frequently_in} Suppose that the net \( \{ x_k \}_{k \in \mscrK} \subseteq S \) is eventually in \( A \subseteq S \). Then there exists an index \( k_0 \) such that \( x_k \in A \) for all \( k \geq k_0 \).

  Given any index \( k_1 \), we choose \( k_2 \) such that \( k_2 \geq k_0 \) and \( k_2 \geq k_1 \) (this is possible by the definition of a directed set). Then \( x_{k_2} \in A \) and \( k_2 \) satisfies the existence quantifier in \fullref{def:topological_net/frequently_in}.

  \SubProofOf{thm:topological_net_properties/net_eventually_in_iff_not_frequently_in_complement} Suppose\LEM that \( \{ x_k \}_{k \in \mscrK} \) is both eventually in \( A \) and frequently in \( S \setminus A \).

  Since the net is eventually in \( A \), we can fix an index \( k_0 \) such that \( x_k \in A \) whenever \( k \geq k_0 \).

  Since the net is frequently in \( A \), we can fix an index \( k_1 \geq k_0 \) such that \( k_1 \in S \setminus A \), which is a contradiction.

  This proves that the two conditions are incompatible.
\end{proof}

\begin{definition}\label{def:net_convergence}
  Let \( X \) be a topological space and \( \{ x_k \}_{k \in \mscrK} \subseteq X \) be a net.

  \begin{defenum}
    \ilabel{def:net_convergence/cluster} If the net is frequently in every neighborhood of \( x_0 \in X \), we say that \( x_0 \) is a \term{cluster point} or an \term{accumulation point} of \( \{ x_k \}_{k \in \mscrK} \subseteq X \).

    \ilabel{def:net_convergence/limit} If the net is eventually in every neighborhood of \( x_0 \in X \), we say that \( x_0 \) is a \term{limit point} of \( \{ x_k \}_{k \in \mscrK} \subseteq X \).
  \end{defenum}

  In general, there can exist multiple limit points (see \fullref{ex:multiple_limit_points_of_net}) and even more cluster points (see \fullref{ex:cluster_points/sine}). In Hausdorff spaces, however, limits are unique by \fullref{thm:t2_iff_singleton_limits}.

  If \( \{ x_k \}_{k \in \mscrK} \subseteq X \) has a unique limit, we say that the net \term[bg=схожда,ru=сходится]{converges} to \( x_0 \) use the notation
  \begin{equation*}
    x_0 = \lim_{k \in \mscrK} x_k.
  \end{equation*}

  If the net is a \hyperref[def:sequence]{sequence}, we also use the following notations:
  \begin{itemize}
    \item \( x_0 = \lim_{k \to \infty} x_k \)
    \item \( x_0 = \lim x_k \)
    \item \( x_k \xrightarrow[k \to \infty]{} x_0 \)
    \item \( x_k \to x_0 \)
  \end{itemize}
\end{definition}

\begin{example}\label{ex:multiple_limit_points_of_net}
  Even limits of sequences need not be unique in arbitrary topological spaces. Let \( X = \{ y, z \} \) be a binary set with the indiscrete \hyperref[def:standard_topologies/indiscrete]{topology} \( \{ \varnothing, X \} \). L

  Define the following \hyperref[def:sequence]{sequence}
  \begin{balign*}
    x_k \coloneqq \begin{cases}
      y, & k \text{ is even}, \\
      z, & k \text{ is odd}.
    \end{cases}
  \end{balign*}

  The only neighborhood of \( y \), the whole space \( X \), contains all members of the sequence, therefore \( y \) is a limit point of the sequence. The same is true for \( z \), however.
\end{example}

\begin{example}\label{ex:cluster_points/sine}
  Consider the net \( \{ \sin(k) \}_{k \in \BbbR} \). It has no limit point, yet every real number in the interval \( [-1, 1] \) is a cluster point.
\end{example}

\begin{example}\label{ex:reverse_inclusion_net}
  A commonly used technique is to use a variation of a \term{reverse set inclusion net}.

  Fix an element \( x_0 \in X \) of any topological space and choose\LEM an element \( x_U \) our of every neighborhood \( U \) of \( x_0 \). Consider the directed set \( (\mscrT(x), \leq) \) consisting of all neighborhoods of \( x_0 \) ordered by \term{reverse inclusion}, i.e. \( U \leq V \iff U \supseteq V \).

  Choose\LEM an element \( x_U \) from each neighborhood \( U \) of \( x_0 \). Then, by construction, \( x_0 \) is a limit point of the net \( \{ x_U \}_{U \in \mscrT(x_0)} \).
\end{example}

\begin{proposition}\label{thm:net_convergence_properties}
  Convergence of \hyperref[def:net_convergence]{nets} has the following basic properties:

  \begin{propenum}
    \ilabel{thm:net_convergence_properties/sequence_converges_iff_almost_entirely_in_neighborhood} The point \( x_0 \in X \) is a limit point of the sequence \( \{ x_k \}_{k=1}^\infty \subseteq X \) if and only if, given a neighborhood \( U \) of \( x_0 \), only finitely many elements of the sequence are outside of \( U \).

    \ilabel{thm:net_convergence_properties/limit_point_is_cluster_point} Every limit point is a cluster point.

    \ilabel{thm:net_convergence_properties/cluster_point_iff_subnet_limit_point} A point \( x_0 \in X \) is a cluster point of the net \( \{ x_k \}_{k \in \mscrK} \subseteq X \) if and only if \( x_0 \) is a limit point of some subnet.

    \ilabel{thm:net_convergence_properties/limit_implies_no_proper_cluster_points} If a net has a limit point, all of its cluster points are limit points.

    \ilabel{thm:net_convergence_properties/unique_limit_point_iff_unique_cluster_point} A net has a unique limit point if and only if has a unique cluster point.

    \ilabel{thm:net_convergence_properties/unique_limit_point_iff_subnets_have_same_limit} A net has a unique limit point if and only if all subnets have the same limit point.
  \end{propenum}
\end{proposition}
\begin{proof}
  \SubProofOf{thm:net_convergence_properties/sequence_converges_iff_almost_entirely_in_neighborhood} This is simply a restatement of \fullref{def:net_convergence/limit} for the special case of sequences.

  \SubProofOf{thm:net_convergence_properties/limit_point_is_cluster_point} Follows from \fullref{thm:topological_net_properties/eventually_in_implies_frequently_in}.

  \SubProofOf{thm:net_convergence_properties/cluster_point_iff_subnet_limit_point}
  The definition of a cluster point (\fullref{def:net_convergence/cluster}) allows us to build a reverse inclusion net in the style of \fullref{ex:reverse_inclusion_net}.

  \SubProofOf{thm:topological_net_properties/eventually_in_implies_frequently_in} Suppose that the net \( \{ x_k \}_{k \in \mscrK} \subseteq S \) is eventually in \( A \subseteq S \). Then there exists an index \( k_0 \) such that \( x_k \in A \) for all \( k \geq k_0 \).

  Given any index \( k_1 \), we choose \( k_2 \) such that \( k_2 \geq k_0 \) and \( k_2 \geq k_1 \) (this is possible by the definition of a directed set). Then \( x_{k_2} \in A \) and \( k_2 \) satisfies the existence quantifier in \fullref{def:topological_net/frequently_in}.

  \SubProofOf{thm:topological_net_properties/net_eventually_in_iff_not_frequently_in_complement} Suppose\LEM that \( \{ x_k \}_{k \in \mscrK} \) is both eventually in \( A \) and frequently in \( S \setminus A \).

  Since the net is eventually in \( A \), we can fix an index \( k_0 \) such that \( x_k \in A \) whenever \( k \geq k_0 \).

  Since the net is frequently in \( A \), we can fix an index \( k_1 \geq k_0 \) such that \( k_1 \in S \setminus A \), which is a contradiction.

  This proves that the two conditions are incompatible.
\end{proof}

\begin{proposition}\label{thm:limit_point_iff_in_closure}\mcite\cite[prop. 1.6.3]{Engelking1989}
  Fix a set \( A \subseteq X \). A point \( x_0 \in X \) belongs to \( \cl{A} \) if and only if there exists a net \( \{ x_k \}_{k \in \mscrK} \subseteq A \) for which \( x_0 \) is a limit point.

  By \fullref{thm:net_convergence_properties/cluster_point_iff_subnet_limit_point}, we can consider cluster points of nets rather than limit points.
\end{proposition}
\begin{proof}
  The complement of the empty set is the empty set, hence the statement of the proposition holds vacuously. Assume that \( A \) is nonempty.

  \SufficiencySubProof Suppose that \( x_0 \in \cl{A} \). If \( x_0 \in A \), then the one-element net \( (x_0) \) converges to \( x_0 \).

  If \( x_0 \in \fr{A} \), by \fullref{def:topological_boundary/neighborhoods}, every neighborhood of \( x_0 \) contains points from \( A \). Therefore we can build reverse inclusion net in the style of \fullref{ex:reverse_inclusion_net} that converges to \( x_0 \).

  \NecessitySubProof Let \( x_0 \) be a limit point of \( \{ x_k \}_{k \in \mscrK} \subseteq A \). We will show that \( x_0 \) belongs every closed set that contains \( A \).

  Let \( F \supseteq A \) be a closed set. Denote \( U \coloneqq X \setminus F \). Suppose\LEM that \( x_0 \in U \). Then \( U \) is a neighborhood \( x_0 \) and, by \fullref{def:net_convergence/cluster}, the net \( \{ x_k \}_{k \in \mscrK} \subseteq A \) is eventually in \( U \). But \( U \) does not contains \( A \).

  The obtained contradiction shows that \( x_0 \) belongs to every closed set containing \( A \) and hence to their intersection, the closure \( \cl A \).
\end{proof}

\begin{proposition}\label{thm:cluster_point_of_set_iff_limit_point_of_net}
  The point \( x_0 \in X \) is a cluster \hyperref[def:topological_derived_set/cluster_point]{point} of the set \( A \) if and only if it is a limit \hyperref[def:net_convergence/cluster]{point} of some net in \( A \setminus \{ x_0 \} \) (or, equivalently, a cluster point of some net in \( A \setminus \{ x_0 \} \)).
\end{proposition}
\begin{proof}
  \SufficiencySubProof Let \( x_0 \in \derived(A) \). By \fullref{thm:derived_set_properties/cluster_via_neighborhoods}, every neighborhood \( U \) of \( x_0 \) intersects \( A \setminus \{ x_0 \} \). Choose\LEM \( x_U \in U \cap (A \setminus \{ x_0 \}) \) for every neighborhood \( U \) of \( x_0 \) and form the reverse inclusion \hyperref[ex:reverse_inclusion_net]{net} \( \{ x_U \}_{U \in \mscrT(x)} \). Then \( x_0 \) is a limit point of this net. Furthermore, the net is contained in \( A \setminus \{ x_0 \} \).

  \NecessitySubProof Conversely, if \( \{ x_k \}_{k \in \mscrK} \subseteq A \setminus \{ x_0 \} \) is a net and if \( x_0 \) is a limit point of this net, then for every neighborhood \( U \) of \( x_0 \) there exists an index \( k_U \) such that for \( k \geq k_U \) we have \(  x_k \in U \). In particular, \( U \cap A \) contains elements other than \( x_0 \). Since this is true for any neighborhood \( U \) of \( x_0 \), by \fullref{thm:derived_set_properties/cluster_via_neighborhoods} we conclude that \( x_0 \) is a cluster point of the set\( A \).
\end{proof}

\begin{corollary}\label{thm:closed_iff_contains_all_net_cluster_points}
  A set is closed if and only if it contains the limit points of all of its nets (or, equivalently, the cluster points of all of its nets).
\end{corollary}
\begin{proof}
  By \fullref{thm:derived_set_properties/closed_iff_contains_all_cluster_points}, the set \( A \) is closed if and only if it contains all of its cluster points. By \fullref{thm:cluster_point_of_set_iff_limit_point_of_net}, this is equivalent to \( A \) containing all limit points of its nets.
\end{proof}

\begin{proposition}\label{thm:net_convergence_via_subbases}
  Fix a topological space \( X \), a point \( x_0 \) and a local \hyperref[def:topological_local_subbase]{subbase} \( \mscrP(x_0) \). The point \( x_0 \) is a limit of the net \( \{ x_k \}_{k \in \mscrK} \subseteq X \) if and only if it is eventually in every element \( U_P \) of the local subbase \( \mscrP(x_0) \).
\end{proposition}
\begin{proof}
  \SufficiencySubProof Obvious consequence of the definition of local subbase.
  \NecessitySubProof Fix a neighborhood \( U \) of \( x_0 \). By \fullref{def:topological_local_subbase}, there exists a finite family \( \{ U_k \}_{k=1}^n \subseteq \mscrP(x_0) \) such that \( \bigcap_{k=1}^n U_k \subseteq U \). Since the net \( \{ x_k \}_{k \in \mscrK} \subseteq X \) is eventually in each of \( U_k, k = 1, \ldots, n \), by transitivity of inclusion it follows that the net is eventually in \( U \).
\end{proof}

\begin{definition}\label{def:sequential_closure_operator}
  In analogy to \fullref{def:closure_operator}, we define the \term{sequential closure operator}
  \begin{balign*}
     & \cl^S: \pow(X) \to \pow(X)                                                                                                          \\
     & \cl^S(A) \coloneqq \left\{ x \in X \colon x \text{ is a limit point of some sequence } \{ x_k \}_{k=1}^\infty \subseteq A \right\}.
  \end{balign*}

  If \( \cl^S(A) = A \), we say that \( A \) is \term{sequentially closed}.
\end{definition}

\begin{definition}\label{def:sequential_space}
  A topological space is called \term{sequential} if every sequentially \hyperref[def:sequential_closure_operator]{closed} set is closed.
\end{definition}

\begin{remark}\label{rem:sequential_spaces}
  By \fullref{thm:limit_point_iff_in_closure}, in a \hyperref[def:sequential_space]{sequential space}, a set is closed if and only if it is sequentially closed.

  By \fullref{thm:closed_iff_contains_all_net_cluster_points}, a set is closed if and only if it contains the limit points of all of its nets.

  Therefore a set in a sequential space is closed if and only if it contains the limit points of all of its sequences.

  Since we are able to define a topology in terms of closed sets, this means that the topology in a sequential space is completely determined by convergent sequences rather than convergent nets as in general topological spaces.

  This allows us to restrict ourselves only to sequences rather than arbitrary nets in certain spaces like \hyperref[def:metric_space]{metric spaces}.
\end{remark}

\begin{lemma}\label{thm:sequential_space_convergence}
  Let \( X \) be a sequential space and \( x_0 \) limit point of the net \( \{ x_k \}_{k \in \mscrK} \), then we can define a sequence
  \begin{equation*}
    \{ x_k \}_{k=1}^\infty \subseteq \{ x_k \colon k \in \mscrK \},
  \end{equation*}
  consisting of members of the net, for which \( x_0 \) is a limit point.
\end{lemma}
\begin{proof}
  Let \( X \) be a first-countable space and let \( x_0 \) be a limit point of the net \( \{ x_k \}_{k \in \mscrK} \).

  Since \( X \) is a first countable space, we can fix a countable local \hyperref[def:topological_local_base]{base} \( \{ U_k \}_{k=1}^\infty \) at \( x_0 \). For each \( k = 1, 2, \ldots \), define the neighborhood \( V_k \coloneqq \bigcap_{m=1}^k U_m \) so that \( V_k \subseteq V_m \) whenever \( k \geq m \).

  For each neighborhood \( V_k \), since \( \{ x_k \}_{k \in \mscrK} \) is eventually in \( V_k \), there exists an index \( k_k \) such that \( x_{k_k} \in V_k \).

  Thus we obtain a sequence \( \{ x_{k_k} \}_{k=1}^\infty \) that is eventually in every neighborhood of the local base \( \{ V_k \}_{k=1}^\infty \) of \( x_0 \), which by \fullref{thm:net_convergence_via_subbases} is sufficient for \( x_0 \) to be a limit point of the sequence.
\end{proof}

\begin{proposition}\label{thm:first_countable_spaces_are_sequential}
  Every first-countable space is sequential.
\end{proposition}
\begin{proof}
  Let \( X \) be a first-countable space and let \( A \subseteq X \) be a sequentially \hyperref[def:sequential_closure_operator]{closed} set. We must show that it is closed.

  Fix a point \( x_0 \in \cl(A) \). We will show that \( x_0 \in A \). By \fullref{thm:limit_point_iff_in_closure}, there is a net \( \{ x_k \}_{k \in \mscrK} \subseteq A \) for which \( x_0 \) is a limit point.

  By \fullref{thm:sequential_space_convergence}, we can choose a sequence \( \{ x_k \}_{k=1}^\infty \) that converges to \( x_0 \) out of elements of the net. But since \( X \) is a sequential space, the limit points of any sequence are contained in the sequentially closed set \( A \).

  We showed that \( A = \cl(A) \). Since \( A \) was an arbitrary sequentially closed set, we conclude that the space \( X \) is sequential.
\end{proof}

\subsection{Function convergence}\label{subsec:function_convergence}

\begin{definition}\label{def:local_convergence}
  Fix two topological spaces \( X \) and \( Y \). Let \( A \subseteq X \) be a nonempty set and let \( f: A \to Y \) be a function. We give two equivalent definitions for \( y_0 \in Y \) being a \term{limit point} of \( f \) at \( x_0 \in \cl(A) \). If \( y_0 \) is the unique limit point (e.g. in \hyperref[def:separation_axioms/T2]{Hausdorff spaces}), we write
  \begin{equation*}
    \lim_{x \to x_0} f(x) = y_0.
  \end{equation*}

  \begin{defenum}
    \ilabel{def:local_convergence/neighborhoods}(Cauchy-style condition) For every neighborhood \( V \) of \( y_0 \) there exists a neighborhood \( U \) of \( x_0 \) such that \( f(U \cap A) \subseteq V \).

    \ilabel{def:local_convergence/nets}(Heine-style condition) For every \hyperref[def:topological_net]{net} \( \{ x_k \}_{k \in \mscrK} \subseteq A \), for which \( x_0 \) is a limit \hyperref[def:net_convergence/limit]{point}, the corresponding net \( \{ f(x_k) \}_{k \in \mscrK} \) has \( y_0 \) as a limit point.
  \end{defenum}
\end{definition}
\begin{proof}
  \ImplicationSubProof{def:local_convergence/neighborhoods}{def:local_convergence/nets} Let \( \{ x_k \}_{k \in \mscrK} \subseteq U \) be a net  with limit point \( x_0 \). Consider the net \( \{ f(x_k) \}_{k \in \mscrK} \). Fix a neighborhood \( V \) of \( y_0 \). We need to show that \( \{ f(x_k) \}_{k \in \mscrK} \) is eventually in \( V \).

  By \fullref{def:local_convergence/neighborhoods}, there exists a neighborhood \( U \) of \( x_0 \) such that \( f(U) \subseteq V \). Since \( x_0 \) is a limit point of \( \{ x_k \}_{k \in \mscrK} \), there exists an index \( k_0 \) such that for all \( k \geq k_0 \), \( x_k \in U \) and therefore \( f(x_k) \in V \). Hence \( \{ f(x_k) \}_{k \in \mscrK} \) is eventually in \( V \).

  We conclude that \( y_0 \) is a limit point of the net \( \{ f(x_k) \}_{k \in \mscrK} \) and that the Heine-style condition is satisfied.

  \ImplicationSubProof{def:local_convergence/nets}{def:local_convergence/neighborhoods} Suppose that \fullref{def:local_convergence/nets} holds while \fullref{def:local_convergence/neighborhoods} does not\LEM. Let \( V \) be a neighborhood of \( y_0 \). Then there exists no neighborhood \( U \) of \( x_0 \) such that \( f(U) \subseteq V \).

  For any neighborhood \( U \) of \( x_0 \) and let \( y_U \in f(U) \setminus V \) and \( x_U \in f^{-1} (U) \) so that \( f(x_U) = y_U \). Consider the families
  \begin{balign*}
    \{ x_U \}_{U \in \mscrT(x_0)},
     &  &
    \{ f(x_U) \}_{U \in \mscrT(x_0)},
  \end{balign*}
  ordered by \hyperref[ex:reverse_inclusion_net]{reverse inclusion} of the neighborhoods \( \mscrT(x_0) \) of \( x_0 \).

  Note that \( x_0 \) is a limit point of \( \{ x_U \}_{U \in \mscrT(x_0)} \). By \fullref{def:local_convergence/nets}, \( y_0 \) is a limit point of \( \{ f(x_U) \}_{U \in \mscrT(x_0)} \). But this contradicts our choice of the nets because \( f(x_U) \not\in V \) for any \( U \in \mscrT(x) \).

  The obtained contradiction demonstrates that \fullref{def:local_convergence/nets} implies \fullref{def:local_convergence/neighborhoods}.
\end{proof}

\begin{proposition}\label{thm:cauchy_function_convergence_via_subbases}
  Fix two topological spaces \( X \) and \( Y \) and two points \( x_0 \in X \) and \( y_0 \in Y \). Let \( \mscrP(x_0) \) and \( \mscrP(y_0) \) be local \hyperref[def:topological_local_subbase]{subbases} for the corresponding points. Then the function \( f: X \to Y \) \hyperref[def:local_convergence]{converges} to \( y_0 \) at \( x_0 \) if and only if every \( V_P \in \mscrP(y_0) \) there exists \( U_P \in \mscrB(x_0) \) such that \( f(U_P) \subseteq V_P \).

  Compare this result to \fullref{thm:net_convergence_via_subbases}.
\end{proposition}
\begin{proof}
  \SufficiencySubProof Obvious consequence of \fullref{def:local_convergence/neighborhoods}.
  \NecessitySubProof Fix a neighborhood \( V \) of \( x \). We will show that \fullref{def:local_convergence/neighborhoods} holds.

  Let \( \{ V_k \}_{k=1}^n \subseteq \mscrP(y_0) \) be a family such that \( \bigcap_{k=1}^n V_k \subseteq V \) (such a family exists by definition of a local subbase). By the antecedent of the implication we are proving, for every \( k = 1, \ldots, n \) there exists an \( U_k \in \mscrP(x_0) \) such that \( f(U_k) \subseteq V_k \). Then \( U \coloneqq \bigcap_{k=1}^n U_k \) is a neighborhood of \( x_0 \) and, furthermore,
  \begin{equation*}
    f(U)
    =
    f\left(\bigcap_{k=1}^n U_k \right)
    \subseteq
    \bigcap_{k=1}^n f(U_k)
    \subseteq
    \bigcap_{k=1}^n V_k
    \subseteq
    V.
  \end{equation*}

  Therefore \fullref{def:local_convergence/neighborhoods} holds.
\end{proof}

\subsection{Topological continuity}\label{subsec:topological_continuity}

\begin{definition}\label{def:local_continuity}
  We say that the function \( f: X \to Y \) between topological spaces is \term{continuous} at the point \( x_0 \in X \) if \( f(x_0) \) is a limit \hyperref[def:local_convergence]{point} of \( f \) at \( x_0 \).

  If limit point is unique (e.g. in \hyperref[def:separation_axioms/T2]{Hausdorff spaces}), this condition can be formulated by \enquote{interchanging} \( \lim \) and \( f \) as follows:
  \begin{equation*}
    f(x_0) = f\left( \lim_{x \to x_0} x \right) = \lim_{x \to x_0} f(x).
  \end{equation*}
\end{definition}

\begin{definition}\label{def:global_continuity}
  We say that the function \( f: X \to Y \) between topological spaces is \term{everywhere continuous} or simply \term{continuous} if and of the following conditions hold:
  \begin{defenum}
    \ilabel{def:global_continuity/limits} \( f \) is continuous at every point of \( X \) in the sense of \fullref{def:local_continuity}.
    \ilabel{def:global_continuity/open} For every open set \( V \in \mscrT \), the \hyperref[def:function/preimage]{preimage} \( f^{-1}(V) \) is open.
    \ilabel{def:global_continuity/closed} For every closed set \( F \in \mscrF_{\mscrT_Y} \), the preimage \( f^{-1}(F) \) is closed.
    \ilabel{def:global_continuity/base} There exists a \hyperref[def:topological_base]{base} \( \mscrB_{\mscrT_Y} \subseteq \mscrT_Y \), such that for every \( V \in \mscrB_{\mscrT_Y} \), the preimage \( f^{-1}(V) \) is open.
    \ilabel{def:global_continuity/subbase} There exists a \hyperref[def:topological_subbase]{subbase} \( \mscrP_{\mscrT_Y} \subseteq \mscrT_Y \), such that for every \( V \in \mscrP_{\mscrT_Y} \), the preimage \( f^{-1}(V) \) is open.
    \ilabel{def:global_continuity/closure} For every set \( A \subseteq X \), \( f(\cl(A)) \subseteq \cl(f(A)) \).
  \end{defenum}

  We denote the set of all continuous functions from \( X \) to \( Y \) by \( C(X, Y) \).
\end{definition}
\begin{proof}
  \ImplicationSubProof{def:global_continuity/limits}{def:global_continuity/open} Follows from \fullref{def:local_convergence/neighborhoods}.
  \ImplicationSubProof{def:global_continuity/open}{def:global_continuity/closed} If \( F \in \mscrF_{\mscrT_Y} \) is a closed set, \( Y \setminus F \) is open, therefore \( f^{-1}(Y \setminus F) = X \setminus f^{-1}(F) \) is also open. Hence \( f^{-1}(F) \) is closed.
  \ImplicationSubProof{def:global_continuity/open}{def:global_continuity/base} \( \mscrT \) is a base of itself.
  \ImplicationSubProof{def:global_continuity/base}{def:global_continuity/subbase} Every base is also a subbase.
  \ImplicationSubProof{def:global_continuity/subbase}{def:global_continuity/limits} Follows from the equivalences in \fullref{def:local_convergence}.
  \ImplicationSubProof{def:global_continuity/closed}{def:global_continuity/closure} Note that
  \begin{equation*}
    A
    \overset {\ref{thm:function_image_preimage_composition/image_first}} \subseteq
    f^{-1}(f(A))
    \overset {\ref{thm:function_preimage_properties/monotonicity}} \subseteq
    f^{-1}(\cl(f(A))).
  \end{equation*}

  Apply \( f \circ \cl \) to the above chain of inclusions to obtain
  \begin{equation*}
    f(\cl(A))
    \subseteq
    f(\underbrace{\cl}_{\ref{def:global_continuity/closed}}(f^{-1}(\cl(f(A)))))
    \overset {\ref{thm:function_image_preimage_composition/preimage_first}} \subseteq
    \cl(f(A)),
  \end{equation*}
  which proves the implication.

  \ImplicationSubProof{def:global_continuity/closure}{def:global_continuity/closed} Fix a closed set \( F \subseteq Y \). Then
  \begin{equation}\label{def:global_continuity/closure_implies_closed_right}
    f(\cl(f^{-1}(F)))
    \overset {\ref{def:global_continuity/closure}} \subseteq
    \cl(f(f^{-1}(F)))
    \overset {\ref{thm:function_image_preimage_composition/preimage_first}} \subseteq
    \cl(F)
    =
    F.
  \end{equation}

  Since \( \cl \) is monotone, we have
  \begin{equation}\label{def:global_continuity/closure_implies_closed_left}
    f(\cl(f^{-1}(F)))
    \supseteq
    f(f^{-1}(F))
    \overset {\ref{thm:function_image_preimage_composition/image_first}} \supseteq
    F.
  \end{equation}

  From \eqref{def:global_continuity/closure_implies_closed_right} and \eqref{def:global_continuity/closure_implies_closed_left} it follows that
  \begin{equation*}
    F = f(\cl(f^{-1}(F))).
  \end{equation*}

  By taking the preimage, we obtain
  \begin{equation*}
    f^{-1}(F)
    =
    f^{-1}(f(\cl(f^{-1}(F))))
    \overset {\ref{thm:function_image_preimage_composition/preimage_first}} \supseteq
    \cl(f^{-1}(F)).
  \end{equation*}

  Therefore \( f^{-1}(F) \) is closed.
\end{proof}

\begin{definition}\label{def:homeomorphism}
  We say that the continuous function \( f: X \to Y \) is \term{open} (resp. \term{closed}), if the image \( f(U) \) of an open (resp. closed) in \( \mscrT_X \) set is open (resp. closed) in \( \mscrT_Y \).

  If \( f \) is an open bijection, we say that \( f \) is a \term{homeomorphism}. If \( f \) is only an open injection, we say that \( f \) is a \term{homeomorphic embedding}.
\end{definition}

\begin{definition}\label{def:parametric_curve}
  Let \( I \) be an interval (of any type) in \( \BbbR \) with endpoints \( a < b \), not necessarily finite. Depending on the use case, we define a \term{parametric curve} on \( I \) by any of the non-equivalent definitions

  \begin{defenum}
    \ilabel{def:parametric_curve/function} A continuous function \( \gamma: I \to X \) is called a parametric curve.

    \ilabel{def:parametric_curve/image} The image \( \img(\gamma) \) of a parametric curve \( \gamma \) is also called a parametric curve.

    \ilabel{def:parametric_curve/equivalence_class} The equivalence class of all continuous functions from \( I \) to \( X \) with
    \begin{equation*}
      \gamma \cong \beta \iff \img(\gamma) = \img(\beta) \text{ and the endpoints of } \gamma \text{ and } \beta \text{ coincide}
    \end{equation*}
    is also called a parametric curve.
  \end{defenum}

  The points \( \gamma(a) \) and \( \gamma(b) \) are called the \term{endpoints} of the curve, \( \gamma(a) \) is the \term{start} and \( \gamma(b) \) is the \term{end}. We say that \( \gamma \) \term{connects} \( a \) and \( b \).

  Parametric curves on \( I = [0, 1] \) are also called \term{paths}.

  We define some fundamental types of curves:
  \begin{defenum}
    \ilabel{def:parametric_curve/closed} The curve \( \gamma \) is called \term{closed} if its endpoints coincide, i.e. \( \gamma(a) = \gamma(b) \).

    \ilabel{def:parametric_curve/simple} The curve \( \gamma \) is called \term{simple} if the function \( \gamma: I \to Y \) is injective with the possible exception of the endpoints (in which case we speak of \term{simple closed curves}.
  \end{defenum}

  If \( \gamma: I \to X \) is a parametric curve, related curves are:
  \begin{defenum}
    \ilabel{def:parametric_curve/function_graph}\mcite\cite[def. 1.20]{Иванов2017}The \hyperref[def:function/graph]{graph} \( \gph(\gamma) \) of \( \gamma \) is a the image of the curve \( \overline{\gamma}(t, x) \coloneqq (t, \gamma(x)) \) in the topological space \( I \times X \).

    \ilabel{def:parametric_curve/implicit}\mcite\cite[def. 1.24]{Иванов2017}If \( M \) is a subset of \( X \) and if there exists a curve \( \gamma: I \to X \) such that \( \imag(\gamma) = M \), we call \( M \) an \term{implicit parametric curve}.
  \end{defenum}
\end{definition}

\begin{definition}\label{def:parametric_hypersurface}
  In analogy to \fullref{def:parametric_curve} (and with the caveats of \fullref{def:parametric_curve}), we define \term{parametric hypersurfaces} as follows:

  Let \( \xi \) is a potentially infinite cardinal number, let \( \card \mscrK = \xi \) and let \( \{ I_\alpha \}_{\alpha \in \mscrK} \) be a family of intervals in \( \BbbR \). We define a parametric hypersurface to be a continuous image from the \hyperref[def:topological_product]{product space} \( \prod_{\alpha \in \mscrK} I_\alpha \) to \( Y \).

  We call \( \xi \) the \term{dimension} of the hypersurface.
\end{definition}

\subsection{Initial and final topologies}\label{subsec:initial_final_topologies}

\begin{definition}\label{def:category_of_topological_spaces}
  Since the topology \( \mscrT \) of a \hyperref[def:topological_space]{topological space} \( (\mscrX, \mscrT) \) consists of subsets of \( \mscrX \), we cannot build a \hyperref[def:first_order_theory]{first-order theory} from \fullref{def:topological_space/O1}-\fullref{def:topological_space/O3}. We can, however, explicitly describe the \hyperref[def:category]{category} \( \cat{Top} \) of topological spaces as
  \begin{refenum}
    \refitem{def:category/C1} The \hyperref[def:set_zfc]{class} of objects is the class of all topological space.
    \refitem{def:category/C2} The morphisms between two topological spaces are the \hyperref[def:global_continuity]{continuous functions} between them.
    \refitem{def:category/C3} Composition of morphisms is the usual \hyperref[def:function/composition]{function composition}.
  \end{refenum}
\end{definition}

\begin{theorem}\label{thm:top_complete_cocomplete}
  The category \( \cat{Top} \) of is both \hyperref[def:categorical_limit]{complete} and \hyperref[def:categorical_colimit]{cocomplete}.
\end{theorem}

\begin{definition}\label{def:initial_topology}\mcite{nLab:top}
  Let \( \{ (X_k, \mscrT_k) \}_{k \in \mscrK} \) be a \hyperref[def:indexed_family]{family} of topological spaces. Let \( X \) be a bare set and let
  \begin{equation*}
    \{ f_k: X \to X_k \}_{k \in \mscrK}
  \end{equation*}
  be a family of functions.

  The topology on \( X \) generated by the subbase
  \begin{equation*}
    \mathcal{P} \coloneqq \{ f_k^{-1}(U) \colon k \in \mscrK, U \in \mscrT_k \}
  \end{equation*}
  is called the \term{initial} (or \term{weak}) topology on \( X \) generated by the family \( \{ f_k \}_{k \in \mscrK} \).

  It is the weakest topology that makes all functions in the family \( \{ f_k \}_{k \in \mscrK} \) continuous.
\end{definition}

\begin{definition}\label{def:final_topology}\mcite{nLab:top}
  Dually, if the family of functions is of the type
  \begin{equation*}
    \{ f_k: X_k \to X \}_{k \in \mscrK},
  \end{equation*}
  then we define the \term{final} (or \term{strong}) topology on \( X \) generated by the family \( \{ f_k \}_{k \in \mscrK} \) as the topology
  \begin{equation*}
    \mscrT \coloneqq \{ U \subseteq X \colon \forall k \in \mscrK, f_k^{-1}(U) \in \mscrT_k \}.
  \end{equation*}

  It is the strongest topology that makes all functions in the family \( \{ f_k \}_{k \in \mscrK} \) continuous.
\end{definition}

\begin{proposition}\label{thm:initial_final_topology_limit}\mcite{nLab:top}
  Let \( D: \boldop I \to \cat{Top} \) be a small \hyperref[def:categorical_diagram]{diagram}. For each space in the image \( D(\boldop I) \), denote the set corresponding by \( X_k \) and the corresponding topology by \( \mscrT_k \).

  The limit (resp. colimit) \( (X, \mscrT) \) of \( D \) can then be described as
  \begin{thmenum}
    \item \( (X, \{ f_k \}_{k \in \cat{I}}) = \varprojlim UD \) (resp. \( \varinjlim UD \)) is the limit (resp. colimit) in \( \cat{Set} \) of \( U \circ D \), where \( U: \boldop{Top} \to \cat{Set} \) is the forgetful functor.
    \item \( \mscrT \) is the \hyperref[def:initial_topology]{initial} (resp. \hyperref[def:final_topology]{final}) topology on \( X \) generated by the family of functions \( \{ f_k \}_{k \in \cat{I}} \).
  \end{thmenum}

  In particular, the functor \( U \) lifts limits and \hyperref[def:categorical_limit_preservation/lift]{colimits}.
\end{proposition}

\begin{definition}\label{def:topological_subspace}
  Let \( (X, \mscrT) \) be a topological space and let \( M \subseteq X \) be a subset of \( X \). The \term{topological subspace} \( (M, \mscrT_M) \) is obtained by endowing \( M \) with the topology
  \begin{equation*}
    \mscrT_M \coloneqq \{ U \cap M \colon U \in \mscrT \}.
  \end{equation*}

  The topology \( \mscrT_M \) is called the \term{subspace topology} or \term{induced topology}.

  It is the initial topology generated by the canonical embedding \( \iota: M \to X \).
\end{definition}

\begin{definition}\label{def:topological_product}
  The \term{topological product} or \term{Tychonoff product}
  \begin{equation*}
    \left( \prod_{k \in \mscrK} X_k, \prod_{k \in \mscrK} \mscrT_k \right)
  \end{equation*}
  of the family \( { (X_k, \mscrT_k) }_{k \in \mscrK} \) is simply the categorical product in the category \( \cat{Top} \) (see \fullref{def:categorical_product}). The underlying set \( \prod_{k \in \mscrK} X_k \) is the \hyperref[thm:set_categorical_limits/product]{Cartesian product} and the topology \( \prod_{k \in \mscrK} \mscrT_k \) is called the \term{product topology}.

  Let \( { (X_k, \mscrT_k) }_{k \in \mscrK} \) and \( { (Y_k, \mathcal{O}_k) }_{k \in \mscrK} \) be two families of topological spaces and let
  \begin{equation*}
    \{ f_k: X_k \to Y_k \}_{k \in \mscrK}
  \end{equation*}
  be a family of arbitrary functions between them.

  We define the \term{product \( \prod_{k \in \mscrK} f_k \) of \( \{ f_k \}_{k \in \mscrK} \)} as the function
  \begin{balign*}
     & \left(\prod_{k \in \mscrK} f_k \right): \prod_{k \in \mscrK} X_k \to \prod_{k \in \mscrK} Y_k              \\
     & \left(\prod_{k \in \mscrK} f_k \right)(\{ x_k \}_{k \in \mscrK}) \coloneqq \{ f_k (x_k) \}_{k \in \mscrK}.
  \end{balign*}

  If all of the spaces \( (X_k, \mscrT_k) \) are equal to some space \( (X, \mscrT) \), we call the product of \( \{ f_k \}_{k \in \mscrK} \) the \term{diagonal product} and denote it by
  \begin{equation*}
    \Delta_{k \in \mscrK} f_k: X \to \prod_{k \in \mscrK} Y_k.
  \end{equation*}
\end{definition}

\begin{definition}\label{def:topological_quotient}\mcite[90]{Engelking1989}
  Let \( X \) be a topological space and let \( \cong \) be an \hyperref[def:equivalence_relation]{equivalence relation} on \( X \). The \term{quotient space} \( (X, \mscrT) / \sim \) is obtained by endowing the quotient set \( X / \cong \) with the final \hyperref[def:final_topology]{topology} given by the canonical projection map \( x \mapsto [x] \).
\end{definition}

\begin{definition}\label{def:topological_sum}\mcite[74]{Engelking1989}
  The \term{topological direct sum}
  \begin{equation*}
    (\oplus_{k \in \mscrK} X_k, \oplus_{k \in \mscrK} \mscrT_k)
  \end{equation*}
  of the family \( { (X_k, \mscrT_k) }_{k \in \mscrK} \) is simply the categorical coproduct in the category \( \cat{Top} \) (see \fullref{def:categorical_coproduct}). The underlying set \( \oplus_{k \in \mscrK} X_k \) is the \hyperref[thm:set_categorical_limits/coproduct]{disjoint union} and the topology \( \oplus_{k \in \mscrK} \mscrT_k \) is called the \term{direct sum topology}.

  Let \( { (X_k, \mscrT_k) }_{k \in \mscrK} \) and \( { (Y_k, \mathcal{O}_k) }_{k \in \mscrK} \) be two families of topological spaces and let
  \begin{equation*}
    \{ f_k: X_k \to Y_k \}_{k \in \mscrK}
  \end{equation*}
  be a family of arbitrary functions between them. Let \( \iota_{X_k}: X_k \to \oplus_{k \in \mscrK} X_k \) and \( \iota_{Y_k}: Y_k \to \oplus_{k \in \mscrK} Y_k \) be the corresponding canonical embeddings.

  We define the \term{direct sum \( \oplus_{k \in \mscrK} f_k \) of \( \{ f_k \}_{k \in \mscrK} \)} as the function
  \begin{balign*}
     & (\oplus_{k \in \mscrK} f_k): \oplus_{k \in \mscrK} X_k \to \oplus_{k \in \mscrK} Y_k   \\
     & (\oplus_{k \in \mscrK} f_k){\rvert}_{X_k} \coloneqq \iota_{Y_k} \circ f_k.
  \end{balign*}

  Obviously \( \oplus_{k \in \mscrK} f_k \) is continuous whenever all \( f_k \) are continuous.

  If all of the spaces \( (Y_k, \mathcal{O}_k) \) are equal to some space \( (Y, \mathcal{O}) \), we call the direct sum of \( \{ f_k \}_{k \in \mscrK} \) simply a \term{sum} and denote it by
  \begin{equation*}
    \sum_{k \in \mscrK} f_k: \oplus_{k \in \mscrK} X_k \to Y.
  \end{equation*}
\end{definition}

\subsection{Separation axioms}\label{subsec:separation_axioms}

\begin{definition}\label{def:topological_space_separation}
  Two subsets \( A, B \subseteq X \) of a topological space \( (X, \mscrT) \) are called \term{separated} or \term{separated using neighborhoods} if there exist disjoint open sets \( U \supseteq A \) and \( V \supseteq B \). In particular, two points are separated if their respective singleton sets are separated.

  We say that \( A \) and \( B \) are \term{functionally separated} if there exists a continuous function \( f: X \to [0, 1] \) such that \( f(A) = 0 \) and \( f(B) = 1 \).
\end{definition}

\begin{definition}\label{def:separation_axioms}
  We can classify topological spaces using the following separation axioms. Fix a topological space \( (X, \mathcal{T}) \).

  \begin{thmenum}
    \thmitem[def:separation_axioms/T0]{T0} (Kolmogorov) \( X \) is \( T_0 \) if for every two different points \( x, y \in X \), there exists an open set \( U \in \mathcal{T} \) such that either \( x \in U \) or \( y \in U \).
    \thmitem[def:separation_axioms/T0.5]{T0.5} \( X \) is \( T_{0.5} \) if every singleton set \( \{ x \} \) is either open or closed.
    \thmitem[def:separation_axioms/T1]{T1} (Frechet) \( X \) is \( T_1 \) if every singleton set \( \{ x \} \) is closed.
    \thmitem[def:separation_axioms/T2]{T2} (Hausdorff) \( X \) is \( T_2 \) if every two different points \( x, y \in X \) can be separated using neighborhoods, i.e. there exist disjoint open sets \( U \ni x \) and \( V \ni y \).

    \thmitem[def:separation_axioms/T3]{T3} \( X \) is \term{regular} if every point and every closed set can be separated using \hyperref[def:topological_space_separation]{neighborhoods}.

    If in addition to being regular \( X \) is \ref{def:separation_axioms/T0}, we say that \( X \) is a \( T_3 \) space.

    \thmitem[def:separation_axioms/T3.5]{T3.5} (Tychonoff) \( X \) is \term{completely regular} if every point and every closed set can be functionally \hyperref[def:topological_space_separation]{separated}.

    If in addition to being completely regular \( X \) is \ref{def:separation_axioms/T0}, we say that \( X \) is a \( T_{3.5} \) space.

    \thmitem[def:separation_axioms/T4]{T4}(Urysohn) \( X \) is \term{normal} every two closed sets \( F, G \in \mathcal{F}_{\mathcal{T}} \) can be separated using neighborhoods, i.e. there exist disjoint open sets \( U \supseteq F \) and \( V \supseteq G \).

    If in addition to being normal \( X \) is \ref{def:separation_axioms/T1}, we say that \( X \) is a \( T_4 \) space.

    \thmitem[def:separation_axioms/T5]{T5} If every subspace of a \( T_4 \) space \( X \) is \ref{def:separation_axioms/T4}, we say that \( X \) is a \( T_5 \) space or a \term{completely normal space}.

    \thmitem[def:separation_axioms/T6]{T6} If every closed set in a \( T_4 \) space \( X \) is \( G_\delta \) (see \fullref{def:borel_algebra}), we say that \( X \) is a \( T_6 \) space or a \term{perfectly normal space}.
  \end{thmenum}
\end{definition}

\begin{proposition}\label{thm:separation_axioms_cascade}
  Each numbered axiom in \fullref{def:separation_axioms} implies the previous one.
\end{proposition}

\begin{proposition}\label{thm:t2_iff_singleton_limits}
  A topological space is \hyperref[def:separation_axioms/T2]{Hausdorff} if and only if every \hyperref[def:topological_net]{net} has at most one \hyperref[def:net_convergence/limit]{limit}.
\end{proposition}
\begin{proof}
  \SufficiencySubProof Let \( X \) be Hausdorff and assume that there exists a net \( \{ x_k \}_{k \in \mscrK} \) such that \( y \) and \( z \) are not necessarily distinct limit points.

  Fix neighborhoods \( U \) of \( y \) and \( V \) of \( z \). Since both are limit points, there exist indices \( k_U \) and \( k_V \) such that \( k \geq k_U \) implies \( x_k \in U \) and \( i \geq i_k \) implies \( x_k \in V \).

  Since \( \mscrK \) is a directed set, there exists an upper bound \( k_0 \) of \( k_U \) and \( k_V \). Thus,
  \begin{equation*}
    x_k \in U \cap V \quad\forall k \geq k_0.
  \end{equation*}

  In particular, the intersection \( U \cap V \) is nonempty and is a neighborhood of both \( y \) and \( z \).

  If \( y \neq z \), then we have two distinct points such that no two neighborhoods of \( y \) and \( z \), respectively, are disjoint. This contradicts the assumption that \( X \) is Hausdorff. Thus \( y = z \).

  \NecessitySubProof Conversely, if \( X \) is not Hausdorff, then for every two distinct points \( y \) and \( z \) and every two neighborhoods \( U \ni y \) and \( V \ni z \), their intersection \( U \cap V \) is nonempty.

  Let \( \mathcal{U} \) and \( \mathcal{V} \) be the sets of all neighborhoods of \( y \) and \( z \), respectively. Since they are both partially ordered by set inclusion \( \subseteq \), define the directed set \( (\mathcal{U} \times \mathcal{V}, \leq) \) with order
  \begin{equation*}
    (U, V) \leq (U', V') \iff U \supset V \T{and} U' \supset V'.
  \end{equation*}

  For each \( (U, V) \in \mathcal{U} \times \mathcal{V} \), choose a point \( x_{(U, V)} \) from \( U \cap V \).

  Thus the net \( \{ x_{(U, V)} \}_{(U, V) \in \mathcal{U} \cap \mathcal{V}} \) has both \( y \) and \( z \) as its limit points, which contradicts our initial assumption.
\end{proof}

\begin{lemma}[Urysohn's lemma]\label{thm:urysohns_lemma}\mcite[1.5.11]{Engelking1989}
  In a \hyperref[def:separation_axioms/T4]{normal space}, every pair \( A, B \) of disjoint closed sets can be functionally \hyperref[def:topological_space_separation]{separated}.
\end{lemma}

\begin{theorem}\label{thm:separation_axioms_of_product}
  Fix is an indexed family \( \{ X_k \}_{k \in \mscrK} \) of topological spaces. Denote their \hyperref[def:topological_product]{product} by \( X \).

  \begin{thmenum}
    \thmitem{thm:separation_axioms_of_product/direct}\cite[theorem 2.3.11]{Engelking1989} If each one of \( X_k \) is a \( T_i \) space for \ref{def:separation_axioms/T0}-\ref{def:separation_axioms/T3.5}, then \( X \) is also a \( T_i \) space.

    \thmitem{thm:separation_axioms_of_product/inverse}\cite[theorem 2.3.11]{Engelking1989} If \( X \) is a \( T_i \) space for \ref{def:separation_axioms/T0}-\ref{def:separation_axioms/T6}, then each component \( X_k \) is also a \( T_i \) space.
  \end{thmenum}
\end{theorem}

\subsection{Connected spaces}\label{subsec:connected_sets}

\begin{definition}\label{def:connected_space}\mcite\cite[thm. 6.1.1]{Engelking1989}
  We say that the topological space \( X \) is \term{connected} if it satisfies any of the following equivalent conditions:
  \begin{thmenum}
    \labitem{def:connected_space/open_union} If \( X = X_1 \cup X_2 \) and \( X_1, X_2 \) are disjoint open sets, either \( X_1 \) or \( X_2 \) is empty.

    \labitem{def:connected_space/closed_union} If \( X = X_1 \cup X_2 \) and \( X_1, X_2 \) are disjoint closed sets, either \( X_1 \) or \( X_2 \) is empty.

    \labitem{def:connected_space/separated_union} If \( X = X_1 \cup X_2 \) and \( X_1, X_2 \) are \hyperref[def:topological_space_separation]{separated}, either \( X_1 \) or \( X_2 \) is empty.

    \labitem{def:connected_space/clopen} The only subsets of \( X \) that are both open and closed are \( \varnothing \) and \( X \).

    \labitem{def:connected_space/discrete_mapping} Every continuous mapping \( f: X \to \{ 0, 1 \} \) into the two-point discrete space is constant.
  \end{thmenum}
\end{definition}

\begin{definition}\label{def:locally_connected}\mcite\cite[exer. 6.3.3]{Engelking1989}
  We say that \( X \) is \term{locally connected} if for every point \( x \in X \) and every neighborhood \( U \) of \( x \) there exists a connected set \( C \subseteq U \) such that \( x \in \inter(C) \).
\end{definition}

\begin{definition}\label{def:path_connected}\mcite\cite[exer. 6.3.9]{Engelking1989}
  We say that a topological space is \term{path connected} if every two points can be connected via a \hyperref[def:parametric_curve]{path}.
\end{definition}

\begin{definition}\label{def:locally_path_connected}\mcite\cite[exer. 6.3.10]{Engelking1989}
  We say that \( X \) is \term{locally path connected} if for every point \( x \in X \) and every neighborhood \( U \) of \( x \) there exists a neighborhood \( V \) of \( x \) such that for any \( y \in V \) there exists a path \( \gamma: [0, 1] \to U \) connecting \( x \) with \( y \).
\end{definition}

\begin{proposition}\label{thm:homomorphism_preserves_connectedness}
  If \( X \) is connected and \( f: X \to Y \) is a homeomorphism, then \( Y \) is also connected.
\end{proposition}
\begin{proof}
  Let \( Y = Y_1 \cup Y_2 \), where \( Y_1 \) and \( Y_2 \) are disjoint and open.

  Note that the preimages \( \gamma^{-1}(Y_1) \) and \( \gamma^{-1}(Y_2) \) are open and disjoint, hence \( X = \gamma^{-1}(Y_1) \cup \gamma^{-1}(Y_2) \). But \( X \) is connected and by \fullref{def:connected_space/open_union}, either \( \gamma^{-1}(Y_1) \) or \( \gamma^{-1}(Y_2) \) is the null set. Thus either \( Y_1 \) and \( Y_2 \) is the null set and, again, by \fullref{def:connected_space/open_union}, \( Y \) is connected.
\end{proof}

\begin{proposition}\label{thm:path_connected_implies_connected}
  Any path connected space is connected.
\end{proposition}
\begin{proof}
  Let \( X = X_1 \cup X_2 \), where \( X_1 \) and \( X_2 \) are disjoint and open.

  Assume\LEM that both are nonzero and take \( x_1 \in X_1, x_2 \in X_2 \). Then there exists a path \( \gamma: I \to X \) with endpoints \( x_1 \) and \( x_2 \). Note that the preimages \( \gamma^{-1}(X_1) \) and \( \gamma^{-1}(X_2) \) are nonempty and open, hence cannot be separated by \fullref{def:connected_space/separated_union}. But this contradicts the disjointedness of \( X_1 \) and \( X_2 \).

  The obtained contradiction proves that \( X \) is connected.
\end{proof}

\subsection{Compact spaces}\label{subsec:compact_spaces}

\begin{definition}\label{def:centered_family}\mcite[123]{Engelking1989}
  The nonempty family \( \mscrF \) of subsets of the topological space \( X \) is said to be a \term{centered family of sets} or to have the \term{finite intersection property} if the intersection \( F_1 \cap \cdots \cap F_n \) of any finite collection of sets is nonempty.
\end{definition}

\begin{definition}\label{def:compact_space}\mcite[123]{Engelking1989}
  The space \( X \) is called \term{compact} if any of the following equivalent finiteness conditions hold:
  \begin{thmenum}
    \thmitem{def:compact_space/finite_subcover} Every open cover of \( X \) has a finite subcover.
    \thmitem{def:compact_space/centered_family} Every centered \hyperref[def:centered_family]{family} \( \mscrF \) of closed subsets of \( X \) has a nonempty intersection.
    \thmitem{def:compact_space/convergent_nets} Every \hyperref[def:topological_net]{net} has a cluster point or, \hyperref[thm:def:net_convergence/cluster_point_iff_subnet_limit_point]{equivalently}, a \hyperref[def:net_convergence]{convergent} subnet. This property is also called \enquote{generalized sequential compactness} or, when restricted to sequences instead of general nets, simply \enquote{sequential compactness}.
  \end{thmenum}
\end{definition}
\begin{proof}
  \ImplicationSubProof{def:compact_space/finite_subcover}{def:compact_space/centered_family} Assume that every open cover of \( X \) has a finite subcover. Let \( \mscrF \) be a centered family of closed subsets of \( X \). Aiming at a contradiction, suppose that \( \bigcap \mscrF = \varnothing \). Then
  \begin{equation*}
    X
    =
    X \setminus \bigcap \mscrF
    =
    \bigcup_{F \in F} (X \setminus F),
  \end{equation*}
  which has a finite subcover indexed by, say, \( \mscrF' \subseteq F \). But \( \mscrF \) is a centered family and \( \bigcap \mscrF' \) is nonempty, hence
  \begin{equation*}
    X
    =
    \bigcup_{F \in F'} (X \setminus F)
    =
    X \setminus \bigcap \mscrF'
    \neq
    X.
  \end{equation*}

  The obtained contradiction shows that \( \bigcap \mscrF \) is nonempty.

  \ImplicationSubProof{def:compact_space/centered_family}{def:compact_space/finite_subcover} Assume that every centered family of closed sets has a nonempty intersection. Let \( \{ U_k \}_{k \in \mscrK} \) be an open cover of \( X \). By putting \( F_k \coloneqq U_k \) for all \( k \in \mscrK \), we obtain a family \( \{ F_k \}_{k \in \mscrK} \) of closed sets with an empty intersection. Therefore, it is not a centered family. Then there exists at least one finite subfamily \( \{ F_k \}_{k \in \mscrK'} \) with an empty intersection. The complement of this subfamily is then a finite cover of \( X \), which proves our statement.

  \ImplicationSubProof{def:compact_space/finite_subcover}{def:compact_space/convergent_nets} Assume that every open cover of \( X \) has a finite subcover. Fix a net \( \{ x_k \}_{k \in \mscrK} \subseteq X \).

  Aiming at a contradiction, suppose that the net has no cluster points. For any point \( x \in X \) and any neighborhood \( U_x \) of \( x \), the net is not frequently in \( U_x \). Obviously \( \{ U_x \}_{x \in X} \) is an infinite open cover of \( X \). Then it has a finite subcover indexed by, say, \( X' \subseteq X \).

  Therefore, every element of the net \( \{ x_k \}_{k \in \mscrK} \) if contained in one of the finitely many neighborhoods \( \{ U_x \}_{x \in X'} \) and the net itself is frequently in at least one of the neighborhoods.

  Thus, one of the finitely many points in \( X' \) is a cluster point of \( \{ x_k \}_{k \in \mscrK} \).

  \ImplicationSubProof{def:compact_space/convergent_nets}{def:compact_space/centered_family}\mcite[thm. 3.1.23]{Engelking1989} Assume that every net has a cluster point. Let \( \{ F_k \}_{k \in \mscrK} \) be a central family of closed sets.

  Denote by \( \mathcal C \) the family of all finite subsets of \( \mscrK \). For each \( C \in \mathcal C \), the set \( \bigcap_{c \in C} F_c \) is a finite intersection of members of a central family and is hence nonempty. Choose an element \( x_C \in \bigcap_{c \in C} F_c \) for each \( C \in \mathcal C \).

  If we order \( \mathcal C \) by reverse inclusion, that is, if \( C \leq C' \iff \bigcap_{c \in C'} F_c \subseteq \bigcap_{c \in C} F_c \), then the family \( \{ x_C \}_{C \in \mathcal C} \) becomes a net. Our assumption is that it has a cluster point, say \( x_0 \).

  It remains to show that \( x_0 \) belongs to the intersection of the centered family \( \{ F_k \}_{k \in \mscrK} \) itself. Fix an element \( F_0 \) of this family and denote by \( C_0 \) the singleton set \( \{ k_0 \} \in \mathcal C \). Because \( x_0 \) is a cluster point, for every neighborhood \( U \) of \( x_0 \), there exists an index \( C \in \mathcal C \) such that \( C \geq C_0 \) and \( x_C \in U \). Then \( x_C \in \bigcap_{c \in C} F_c \subseteq \bigcap_{c \in C_0} F_c = F_0 \).

  Therefore, \( F_0 \cap U \neq \varnothing \) for all neighborhoods \( U \) of \( x_0 \). By \fullref{thm:def:topological_closure_operator/neighborhood_intersection}, \( x_0 \in F_0 \). Since \( F_0 \) was an arbitrary set from the centered family \( \{ F_k \}_{k \in \mscrK} \), we conclude that the intersection of the family is not empty. This proves the theorem.
\end{proof}

\begin{remark}\label{rem:precompact_set}
  If the closure of \( A \) is compact, we call \( A \) \term{relatively compact} or \term{precompact} (although the term \enquote{precompact} is also used for totally bounded sets, see \fullref{def:totally_bounded_set}).
\end{remark}

\begin{proposition}\label{thm:def:compact_space}
  \hyperref[def:compact_space]{Compact spaces} have the following basic properties:
  \begin{thmenum}
    \thmitem{thm:def:compact_space/closed} If \( X \) is any topological space and \( A \subseteq X \) is a compact subspace, then \( A \) is closed in \( X \).
    \thmitem{thm:def:compact_space/closed_subspace} A closed subspace of a compact space is compact.
  \end{thmenum}
\end{proposition}
\begin{proof}
  \SubProofOf{thm:def:compact_space/closed} If \( A \) is compact, by \fullref{def:compact_space/convergent_nets} every net in \( A \) has a cluster point in \( A \). By \fullref{thm:def:derived_set/closed_iff_contains_all_cluster_points}, \( A \) is closed in \( X \).

  \SubProofOf{thm:def:compact_space/closed_subspace} Let \( X \) be \hyperref[def:compact_space]{compact} and let \( X' \subseteq X \) be a closed subspace.

  Fix an open cover \( \{ U_k \}_{k \in \mscrK} \) of \( X' \). By definition of the subspace \hyperref[def:topological_subspace]{topology} for each \( U_k \) there exists a set \( V_k \) that is open in \( X \) and \( U_k = V_k \cap X' \). Since \( X' \) is closed, \( X \setminus X' \) is open and hence the family \( \{ V_k \}_{k \in \mscrK} \) together with \( X \setminus X' \) is an open cover of the space \( X \).

  By \fullref{def:compact_space/finite_subcover}, there exists a finite subcover \( \{ V_k \}_{k \in \mscrK'} \) that, along with \( X \setminus X' \), still covers \( X' \). Therefore, \( X' \) is also compact.
\end{proof}

\begin{theorem}[Tychonoff's product theorem]\label{thm:tychonoffs_product_theorem}\mcite[thm. 3.2.4]{Engelking1989}
  Let \( { (X_k, \mscrT_k) }_{k \in \mscrK} \) be a family of topological spaces. Their \hyperref[def:topological_product]{product} \( (\prod_{k \in \mscrK} X_k, \prod_{k \in \mscrK} \mscrT_k) \) is \hyperref[def:compact_space]{compact} if and only if \( (X_k, \mscrT_k) \) is compact for every \( k \in \mscrK \).

  Within \hyperref[def:zfc]{\logic{ZF}}, this theorem is equivalent to the \hyperref[def:zfc/choice]{axiom of choice} --- see \fullref{thm:axiom_of_choice_equivalences/tychonoff}.
\end{theorem}

\begin{theorem}[Weierstrass' extreme value theorem]\label{thm:weierstrass_extreme_value_theorem}\mcite[cor. 3.2.9]{Engelking1989}
  Let \( X \) be a compact topological space and let \( f: X \to \BbbR \) be a continuous function into the \hyperref[def:real_numbers]{real numbers}.

  Then \( f \) is \hyperref[def:metric_space/bounded_function]{bounded} and there exist \( m, M \in X \) such that
  \begin{balign*}
    f(m) = \min_{x \in X} f(x)
     &  &
    f(M) = \max_{x \in X} f(x).
  \end{balign*}
\end{theorem}

\begin{definition}\label{def:locally_compact_space}\mcite[148]{Engelking1989}
  A topological space is called \term{locally compact} if every point has a relatively compact neighborhood.
\end{definition}

\subsection{Baire spaces}\label{subsec:baire_spaces}

\begin{remark}\label{rem:baire_categories}
  René-Louis Baire introduced the concept of \Def{Baire categories} in 1899, almost 50 years before Samuel Eilenberg and Saunders MacLane introduced \Def{categories} in \cite{MacLane1945} (see \fullref{sec:category_theory} for the latter).

  Unfortunately, topology utilizes both concepts so the word \enquote{category} should be used with caution. To circumvent this, we use alternative terminology for Baire categories.
\end{remark}

\begin{definition}\label{def:meager_set}\MarginCite[def. 2.1]{Rudin1991}
  Any countable union of \hyperref[def:topologically_dense_set/nowhere_dense]{nowhere dense sets} is called \Def{meager} or a \Def{first category set} (see \fullref{rem:baire_categories} for terminology). If a set is not meager, we call it \Def{nonmeager} or a \Def{second category set}.
\end{definition}

\begin{proposition}\label{thm:meager_set_properties}\MarginCite[43]{Rudin1991}
  \hyperref[def:meager_set]{Meager sets} have the following basic properties (compare to \fullref{thm:nowhere_dense_properties}):
  \begin{PropEnum}
    \ILabel{thm:meager_set_properties/union} A countable union of meager sets is meager.
    \ILabel{thm:meager_set_properties/subset} A subset of a meager set is meager.
    \ILabel{thm:meager_set_properties/homeomorphism} The \hyperref[def:homeomorphism]{homeomorphic} image of a set \( A \) is meager if and only if \( A \) itself is meager.
  \end{PropEnum}
\end{proposition}
\begin{proof}
  \SubProofOf{thm:meager_set_properties/union} Follows from \fullref{thm:countable_union_of_countable_sets}.
  \SubProofOf{thm:meager_set_properties/subset} Fix a meager set \( A \) and let \( B \subseteq A \). Then \( A = \bigcup_{k=1}^\infty A_k \) for some nowhere dense sets \( A_1, A_2, \ldots \). By \fullref{thm:nowhere_dense_properties/subset}, the sets \( A_1 \cap B, A_2 \cap B, \ldots \) are also nowhere dense. But
  \begin{equation*}
    B
    =
    A \cap B
    =
    \left(\bigcup_{k=1}^\infty A_k \right) \cap B
    \overset {\ref{thm:subsets_form_boolean_algebra}} =
    \bigcup_{k=1}^\infty (A_k \cap B).
  \end{equation*}

  Therefore \( B \) is also nowhere dense.

  \SubProofOf{thm:meager_set_properties/homeomorphism}\mbox{}
  \Necessity If \( A \) is meager, any homeomorphic image of \( A \) is meager by \fullref{thm:function_image_properties/union} and \fullref{thm:nowhere_dense_properties/homeomorphism}.
  \Sufficiency If \( f: X \to Y \) is a homeomorphism and \( f(A) \) is meager for some \( A \subseteq X \), then \( A \) is the homeomorphic image of the meager set \( f(A) \) under \( f^{-1} \) and is thus meager.
\end{proof}

\begin{definition}\label{def:baire_space}
  A topological space is called a \Def{Baire space} if any of the following equivalent conditions hold:
  \begin{DefEnum}
    \ILabel{def:baire_space/meager} Every nonempty open set is nonmeager.
    \ILabel{def:baire_space/dense} A countable intersection of dense sets is dense.
  \end{DefEnum}
\end{definition}
\begin{proof}
  \SubProofEquivalence{def:baire_space/meager}{def:baire_space/dense} Follows from \fullref{thm:nowhere_dense_properties/complement_dense} and \fullref{thm:de_morgans_laws}.
\end{proof}

\begin{proposition}\label{thm:open_subspace_of_baire_space_is_baire}
  Every open subspace of a \hyperref[def:baire_space]{Baire space} is a Baire space.
\end{proposition}
\begin{proof}
  Let \( (X, \CT) \) be a Baire space and let \( (X', \CT_{X'}) \) be an open \hyperref[def:topological_subspace]{subspace} with the canonical embedding \( \iota: X' \to X \). The proposition holds vacuously if \( X' = \varnothing \) so we assume that \( X' \neq \varnothing \).

  Note that \( \iota \) is continuous by definition, however it is also an open map because if \( U \in \CT_{X'} \), then \( \iota(U) = U \cap X' \) is open in \( X \) as the intersection of two open sets. Therefore it is a homeomorphic embedding and by \fullref{thm:meager_set_properties/homeomorphism}, \( U \) is meager if and only if \( \iota(U) \) is meager. Since \( X \) is a Baire space, \( \iota(U) \) is not meager and hence \( U \) is also not meager.

  We showed that every nonempty open set \( U \in \CT_{X'} \) is nonmeager, therefore \( X' \) is a Baire space.
\end{proof}

\begin{theorem}[Baire category theorem]\label{thm:baire_category_theorem}\MarginCite{Rudin1991}
  \begin{ThmEnum}
    \ILabel{thm:baire_category_theorem/metric} \hyperref[def:complete_metric_space]{Complete metric spaces} are \hyperref[def:baire_space]{Baire spaces}.
    \ILabel{thm:baire_category_theorem/compact} \hyperref[def:locally_compact_space]{Locally compact} \hyperref[def:separation_axioms/T2]{Hausdorff} spaces are Baire spaces.
  \end{ThmEnum}
\end{theorem}

\subsection{Uniform spaces}\label{subsec:uniform_spaces}

\begin{remark}\label{rem:entourage_notation}
  \hyperref[def:uniform_space]{Uniform spaces} are an extension of both \hyperref[def:metric_space]{metric spaces} and \hyperref[def:topological_group]{topological groups} (including \hyperref[def:topological_vector_space]{topological vector spaces}). They are topological spaces that are \enquote{uniform} in the sense that different parts of the space behave the same, unlike \hyperref[def:topological_manifold]{manifolds}.

  In metric spaces, we use the notation \( \mu(x, y) < \varepsilon \) to mean that \( x \) and \( y \) are close (at a distance less than \( \varepsilon \)).

  In (\hyperref[rem:additive_magma]{additive}) topological groups, we instead have linear operations and use \( x - y \in U \) to mean that \( x \) and \( y \) are close (their difference belongs to some neighborhood of \( 0 \)).

  A proper generalization needs to make both metric spaces and topological groups feel natural as special cases. Generalizing metric space \hyperref[def:metric_space/ball]{balls} or neighborhoods of \hyperref[thm:origin_neighborhoods_in_topological_groups]{zero} are nice options which unfortunately introduces some asymmetry since, for example for metric spaces, \( \mu(x, y) < \varepsilon \) can be written as either \( y \in B(x, \varepsilon) \) or \( x \in B(y, \varepsilon) \). This approach does not go far beyond what general topological spaces offer as a notation.

  \cite[section 8]{Engelking1989} uses the notation \( \abs{x - y} < V \) to mean that \( x \) and \( y \) belong to the \hyperref[def:entourage]{entourage} \( V \). This is a bit confusing because no absolute \hyperref[def:absolute_value]{value} nor subtraction are defined in uniform spaces. We find it simpler to not introduce any special notation beyond that of \hyperref[def:relation]{relations} and so we denote the same by \( (x, y) \in V \).
\end{remark}

\begin{definition}\label{def:entourage}\mcite\cite[sec. 8.1]{Engelking1989}
  Let \( X \) be a set. For two binary \hyperref[def:relation]{relations} \( V \) and \( U \) on \( X \) we define their sum as
  \begin{equation*}
    V + U \coloneqq \{ (x, z) \colon \exists y \in X: (x, y) \in U, (y, z) \in V \}
  \end{equation*}
  and \( nV \) by \( n \)-fold iterative addition.

  For any relation \( V \), we denote by \( -V \) the converse \hyperref[def:binary_relation/converse]{relation}.

  A relation \( V \) on \( X \) is called an \term{entourage} if \( V \) is \hyperref[def:binary_relation/reflexive]{reflexive} and \hyperref[def:binary_relation/symmetric]{symmetric}.

  In analogy to \hyperref[def:metric_space]{metric spaces}, we define
  \begin{thmenum}
    \ilabel{def:entourage/ball} We define the \term{open ball} or simply \term{ball} with \term{center} \( x \) and \term{radius} \( V \) to be the set
    \begin{equation*}
      B(x, V) \coloneqq \{ y \in X \colon (y, x) \in V \}.
    \end{equation*}

    \ilabel{def:entourage/bounded_set} We say that the set \( S \subseteq X \) is \term{bounded} if it is contained in some ball.
  \end{thmenum}
\end{definition}

\begin{proposition}\label{thm:entourage_simulates_metric}\mcite\cite[sec. 8.1]{Engelking1989}
  Using the notation of \fullref{def:entourage}, we obtain properties similar to those of metrics:
  \begin{thmenum}
    \iref{def:metric_space/M1} \( (x, x) \in V \)
    \iref{def:metric_space/M2} \( (x, y) \in V \) if and only if \( (y, x) \in V \)
    \iref{def:metric_space/M3} If \( (x, y) \in U \) and \( (y, z) \in V \), then \( (x, y) \in U + V \).
  \end{thmenum}
\end{proposition}

\begin{definition}\label{def:uniform_space}\mcite\cite[sec. 8.1]{Engelking1989}
  A \term{uniform space} is a set \( X \) with a nonempty family \( \mscrV \) of \hyperref[def:entourage]{entourages} on \( X \) such that
  \begin{thmenum}
    \iaxiom{def:uniform_space/U1}{U1} If \( V \in \mscrV \) and \( V \subseteq W \) for some entourage \( W \) on \( X \), then \( W \in \mscrV \).
    \iaxiom{def:uniform_space/U2}{U2} If \( V_1, V_2 \in \mscrV \), then \( V_1 \cap V_2 \in \mscrV \).
    \iaxiom{def:uniform_space/U3}{U3} For every \( V \in \mscrV \) there exists \( W \in \mscrV \) such that \( 2W \subseteq V \).
    \iaxiom{def:uniform_space/U4}{U4} \( \bigcap \mscrV = \Delta_X \), where \( \Delta_X \) is the \hyperref[def:binary_relation/diagonal]{diagonal relation}.
  \end{thmenum}

  The family \( \mscrV \) is called a \term{uniform structure} or \term{uniformity} on \( X \).
\end{definition}

\begin{definition}\label{def:uniform_topology}
  Let \( (X, \mscrV) \) be a \hyperref[def:uniform_space]{uniform space}. We define its \term{uniform topology} or \term{induced topology} as the \hyperref[def:topological_space]{topology} generated by the \hyperref[def:topological_local_base]{neighborhood system}
  \begin{equation*}
    \mathcal{B}(x) \coloneqq \{ B(x, V) \colon V \in \mscrV \}.
  \end{equation*}

  If for some topological space \( (X, \mscrT) \) there exists a uniformity such that \( \mscrT \) is its induced topology, we say that the topology \( \mscrT \) is \term{uniformizable}.
\end{definition}
\begin{proof}
  This proof of correctness does not actually rely on the uniform structure (except for \( \mscrV \) being nonempty) but rather on the properties of entourages.

  It is indeed a neighborhood system because
  \begin{reflist}
    \iref{thm:topological_local_base_axioms/BP1} Every \hyperref[def:entourage]{entourage} is reflexive, hence \( x \) is contained in every ball in \( \mathcal{B}(x) \).

    \iref{thm:topological_local_base_axioms/BP2} For \( B(x, U) \) and \( B(x, V) \) we have
    \begin{balign*}
      B(x, U \cap V)
       & =
      \{ y \in X \colon (x, y) \in U \cap V \}
      =    \\ &=
      \{ y \in X \colon (x, y) \in U \text{ and } (x, y) \in V \}
      =    \\ &=
      B(x, U) \cap B(x, V).
    \end{balign*}

    \iref{thm:topological_local_base_axioms/BP3} Fix \( x, y \in X \) and a ball \( B(y, V) \in \mathcal{B}(y) \) that contains \( x \). We will show that \( B(y, V) \subseteq B(x, 2V) \).

    Fix \( z \in B(y, V) \). We have \( (z, y) \in V \). Then \( (z, x) \in V + V = 2V \). Since \( z \in B(y, V) \) was arbitrary, we conclude that \( B(y, V) \subseteq B(x, 2V) \).
  \end{reflist}
\end{proof}

\begin{theorem}\label{thm:tychonoff_spaces_are_uniformizable}\mcite\cite[thm. 8.1.20]{Engelking1989}
  A topological space is \hyperref[def:uniform_topology]{uniformizable} if and only if it is a \hyperref[def:sequential_space]{Tychonoff space}.
\end{theorem}

\begin{definition}\label{def:uniform_space_base}
  Fix a uniform space \( (X, \mscrV) \). The subfamily \( \mscrB \subseteq \mscrV \) if entourages is called a \term{base} for \( \mscrV \) if every entourage \( V \in \mscrV \) contains a member of \( \mscrV \).
\end{definition}

\begin{definition}\label{thm:uniform_space_base_axioms}\mcite\cite[prop. 8.1.14]{Engelking1989}
  Let \( X \) be an arbitrary set and let \( \mscrB \) be a family of entourages satisfying the following axioms:
  \begin{thmenum}
    \iaxiom{thm:uniform_space_base_axioms/BU1}{BU1} If \( V_1, V_2 \in \mscrB \), there exists an entourage \( V \in \mscrB \) such that \( V \subseteq V_1 \cap V_2 \).
    \iaxiom{thm:uniform_space_base_axioms/BU2}{BU2} For every \( V \in \mscrB \) there exists \( W \in \mscrB \) such that \( 2W \subseteq V \).
    \iaxiom{thm:uniform_space_base_axioms/BU3}{BU3} \( \bigcap \mscrB = \Delta_X \)
  \end{thmenum}

  Then the family of entourages
  \begin{balign}\label{thm:uniform_space_base_axioms/uniformity}
    \mscrV \coloneqq \left\{ V \subseteq X \times X \colon \exists B \in \mscrB: B \in V \T{and} V \text{ is reflexive and symmetric} \right\}
  \end{balign}
  is a uniform structure on \( X \). Furthermore, \( \mscrB \) is a \hyperref[def:uniform_space_base]{base} of \( \mscrV \).

  In particular, the base on any topology satisfies \fullref{thm:uniform_space_base_axioms/BU1} -- \fullref{thm:uniform_space_base_axioms/BU2}.
\end{definition}

\begin{lemma}\label{thm:uniform_space_neighborhood_contains_ball}
  In a uniform space \( (X, \mscrV) \), for every neighborhood \( A \) (in the topology) of a point \( x_0 \in X \) there exists an entourage \( V \in \mscrV \) such that \( B(x_0, V) \subseteq A \).
\end{lemma}
\begin{proof}
  By \fullref{def:uniform_topology} and \fullref{def:topological_base/union}, \( A \) is a union of balls centered at \( x_0 \). For any ball \( B(x_0, V) \) of this union, we have \( B(x_0, V) \subseteq A \).
\end{proof}

\begin{proposition}\label{thm:uniform_topology_properties}
  The \hyperref[def:uniform_topology]{uniform topology} \( \mscrT \) on \( (X, \mscrV) \) the following basic properties:
  \begin{thmenum}
    \ilabel{thm:uniform_topology_properties/ball_is_open} All \hyperref[def:entourage/ball]{balls} are open sets.
    \ilabel{thm:uniform_topology_properties/neighborhood_contains_ball} Every neighborhood of every point a ball centered at that point.
  \end{thmenum}
\end{proposition}
\begin{proof}
  \SubProofOf{thm:uniform_topology_properties/ball_is_open} We defined the balls to be the base of the uniform topology, therefore they are open.
  \iref{thm:uniform_topology_properties/neighborhood_contains_ball} Fix a point \( x_0 \). It is a trivial consequence of \fullref{def:topological_base/subset} that every neighborhood of \( x_0 \) contains some ball centered at a point that is not necessarily \( x_0 \). By \fullref{thm:topological_local_base_axioms/BP3}, this ball contains another ball centered at \( x_0 \).
\end{proof}

\begin{proposition}\label{thm:uniform_space_local_convergence}
  Fix a topological space \( (X, \mscrT) \) and a uniform space \( (Y, \mscrU) \). Let \( A \subseteq X \) be a nonempty set and let \( f: A \to Y \) be a function. Then \( y_0 \) is a limit point of \( f \) at \( x_0 \in X \) in the sense of \fullref{def:local_continuity} if and only if
  \begin{equation}\label{thm:uniform_space_local_convergence/topological_source}
    \forall V \in \mscrV \ \exists A \in \mscrT(x_0) : x \in A \implies (f(x), y_0) \in V.
  \end{equation}

  If instead, \( (X, \mscrU) \) is a uniform space, then \( y_0 \) is a limit point of \( f \) at \( x_0 \in X \) if and only if
  \begin{equation}\label{thm:uniform_space_local_convergence/uniform_source}
    \forall V \in \mscrV \ \exists U \in \mscrU : (x, x_0) \in U \implies (f(x), y_0) \in V.
  \end{equation}

  Note that the limit point may not be unique because uniform spaces are not \hyperref[def:separation_axioms/T2]{Hausdorff} in general.
\end{proposition}
\begin{proof}
  We will only prove \fullref{thm:uniform_space_local_convergence/uniform_source} because the proof of \fullref{thm:uniform_space_local_convergence/topological_source} is a special case.

  \SufficiencySubProof Suppose that \( y_0 \) is a limit point of \( f \) at \( x_0 \) and fix a neighborhood \( B \) of \( y_0 \). Then there exists a neighborhood \( A \) of \( x_0 \) such that \( f(A) \subseteq B \).

  Fix an entourage \( V \in \mscrV \). Then \( B(y_0, V) \) is also a neighborhood of \( y_0 \). By \fullref{thm:uniform_space_neighborhood_contains_ball} and \fullref{def:uniform_space/U2}, there exists an entourage \( V' \subseteq \mscrV \) such that \( B(f(x), V') \subseteq B \cap B(y_0, V) \).

  Fix an entourage \( U \in \mscrU \) such that \( B(x_0, U) \subseteq A \). Then for any \( x \in X \), if \( (x, x_0) \in U \), we have \( (f(x), y_0) \in V' \). But \( V' \subseteq V \), therefore
  \begin{equation*}
    (x, x_0) \in U \implies (f(x), y_0) \in V.
  \end{equation*}

  This concludes the proof.

  \NecessitySubProof Fix a neighborhood \( B \) of \( y_0 \) and an entourage \( V \in \mscrV \) such that \( B(x_0, V) \subseteq B \) (see \fullref{thm:uniform_space_neighborhood_contains_ball} for a justification). Then there exists \( U \in \mscrU \) such that
  \begin{equation*}
    (x, x_0) \in U \implies (f(x), y_0) \in V.
  \end{equation*}

  Therefore \( A \coloneqq B(x_0, U) \) is a neighborhood of \( x_0 \) such that \( f(A) \subseteq B \).
\end{proof}

\begin{corollary}\label{thm:uniform_space_local_continuity}
  A function \( f: (X, \mscrV) \to (Y, \mscrU) \) between uniform spaces is continuous at \( x_0 \in X \) if and only if
  \begin{equation*}
    \forall V \in \mscrV \ \exists U \in \mscrU : (x, x_0) \in U \implies (f(x), f(x_0)) \in V.
  \end{equation*}
\end{corollary}

\begin{definition}\label{def:bounded_function}
  Fix a set \( X \) and a \hyperref[def:uniform_space]{uniform space} \( (Y, \mscrV) \). Fix a function \( f: X \to Y \).

  \begin{thmenum}
    \ilabel{def:bounded_function/bounded} We say that the function \( f: X \to Y \) is \term{bounded} if \( f(X) \) is a bounded set, that is, if there exists a \hyperref[def:entourage/ball]{ball} \( B(y, V) \) such that \( f(X) \subseteq B(y, V) \).

    \ilabel{def:bounded_function/bounded_family} We say that the family of functions \( \mscrF \) from \( X \) to \( Y \) is \term{bounded} at \( x_0 \) if there exists a ball \( B(y, V) \) such that the set \( \mscrF(x_0) \coloneqq \{ f(x_0) \colon f \in \mscrF \} \) is contained in \( B(y, V) \).

    \ilabel{def:bounded_function/pointwise} We say that \( \mscrF \) is \term{pointwise bounded} on the set \( S \subseteq X \) if
    \begin{equation*}
      \forall x \in S \ \exists B(y, V) : \mscrF(x) \subseteq B(y, V).
    \end{equation*}

    \ilabel{def:bounded_function/uniform} We say that \( \mscrF \) is \term{uniformly bounded} on \( S \subseteq X \) if
    \begin{equation*}
      \exists B(y, V) \ \forall x \in S : \mscrF(x) \subseteq B(y, V).
    \end{equation*}

    \ilabel{def:bounded_function/locally_bounded} If there is a topology \( \mscrT \) on \( X \), we say that the function \( f: X \to Y \) is \term{locally bounded} if there exists an entourage \( V \in \mscrV \) such that for each neighborhood \( A \in \mscrT(x) \) we have \( \diam{f(A)} < V \).
  \end{thmenum}
\end{definition}

\begin{proposition}\label{thm:continuous_implies_locally_bounded}
  Let \( (X, \mscrT) \) be a topological space and \( (Y, \mscrV) \) be a uniform space. Any \hyperref[thm:uniform_space_local_convergence/topological_source]{continuous function} from \( X \) to \( Y \) is locally \hyperref[def:bounded_function/locally_bounded]{bounded}.
\end{proposition}
\begin{proof}
  Trivial.
\end{proof}

\begin{definition}\label{def:function_net_convergence}
  Fix a set \( X \) and a \hyperref[def:uniform_space]{uniform space} \( (Y, \mscrV) \). Let \( \{ f_k \}_{k \in \mscrK} \) be a \hyperref[def:topological_net]{net} of functions from \( X \) to \( Y \). We say that \( \{ f_k \}_{k \in \mscrK} \) \term{converges pointwise} to the function \( f \) and write \( f_k \to f \) if
  \begin{equation}\label{def:function_net_convergence/pointwise}
    \forall V \in \mscrV \ \underbrace{\forall x \in X \ \exists k_0 \in \mscrK} : k \geq k_0 \implies (f_k(x), f(x)) \in V
  \end{equation}
  and that \( \{ f_k \}_{k \in \mscrK} \) \term{converges uniformly} to \( f \) and write \( f_k \rightrightarrows f \) if
  \begin{equation}\label{def:function_net_convergence/uniform}
    \forall V \in \mscrV \ \underbrace{\exists k_0 \in \mscrK \ \forall x \in X} : k \geq k_0 \implies (f_k(x), f(x)) \in V
  \end{equation}

  In the special case where \( X \) is a topological space with topology \( \mscrT \), we call the sequence \( \{ f_k \}_{k \in \mscrK} \) \term{locally uniformly convergent} (see \cite{ProofWiki:locally_uniform_convergence}) if every point in \( S \) has a neighborhood in which the sequence converges uniformly. Symbolically,
  \begin{equation}\label{def:function_net_convergence/locally_uniform}
    \forall V \in \mscrV \ \forall x_0 \in S \ \exists A \in \mscrT(x_0) \ \exists k_0 \in \mscrK \ \forall x \in A : k \geq k_0 \implies (f_k(x), f(x)) \in V.
  \end{equation}

  If the index \( k_0 \) does not depend on the neighborhood \( A \) and the point \( x_0 \), then this is equivalent to uniform convergence. It is still more powerful than pointwise convergence. For example, \hyperref[def:convergent_power_series]{power series} are locally uniformly convergent on the interior of their domain of convergence - see \fullref{thm:power_series_are_locally_uniform_convergent}.

  A slightly weaker notion is that of \term{compact convergence} (see \cite{ProofWiki:compact_convergence}), which is defined as uniform convergence on any compact subset. Symbolically,
  \begin{equation}\label{def:function_net_convergence/compact}
    \forall V \in \mscrV \ \forall \text{ compact } C \subseteq S \ \exists k_0 \in \mscrK \ \forall x \in C : k \geq k_0 \implies (f_k(x), f(x)) \in V.
  \end{equation}
\end{definition}

\begin{definition}\label{def:uniform_continuity}\mcite\cite[435]{Engelking1989}
  Fix two \hyperref[def:uniform_space]{uniform spaces} \( (X, \mscrU) \) and \( (Y, \mscrV) \). We say that the function \( f: X \to Y \) if is \term{uniformly continuous} on the set \( S \subseteq X \) if
  \begin{equation}\label{def:uniform_continuity/uniform}
    \forall V \in \mscrV \ \underbrace{\exists U \in \mscrU \ \forall x_1, x_2 \in S} : (x_1, x_2) \in U \implies (f(x_1), f(x_2)) \in V.
  \end{equation}

  Compare this to \term{pointwise continuity} on \( S \), which is defined by \fullref{thm:uniform_space_local_convergence/uniform_source} as convergence for any \( x_1 \in X \):
  \begin{equation}\label{def:uniform_continuity/pointwise}
    \forall V \in \mscrV \ \underbrace{\forall x_1, x_2 \in S \ \exists U \in \mscrU} : (x_1, x_2) \in U \implies (f(x_1), f(x_2)) \in V.
  \end{equation}
\end{definition}

\begin{definition}\label{def:function_set_continuity}\mcite\cite[285]{BouziadTroallic2004}
  Fix a topological space \( (X, \mscrT) \) and a \hyperref[def:uniform_space]{uniform space} \( (Y, \mscrV) \). We say that the family \( \mscrF \) of functions from \( X \) to \( Y \) is \term{functionwise continuous} at \( x_0 \in X \) if
  \begin{equation}\label{def:function_set_continuity/functionwise}
    \forall V \in \mscrV \ \underbrace{\forall f \in \mscrF \ \exists A \in \mscrT(x_0)} : f(A) \subseteq B(f(x_0), V),
  \end{equation}
  and \term{equicontinuous} at \( x_0 \in X \) if
  \begin{equation}\label{def:function_set_continuity/equicontinuous}
    \forall V \in \mscrV \ \underbrace{\exists A \in \mscrT(x_0) \ \forall f \in \mscrF} : f(A) \subseteq B(f(x_0), V).
  \end{equation}

  In the special case where \( (X, \mscrU) \) is a uniform space, then we can define \term{uniform equicontinuity} of the family \( \mscrF \) on the set \( S \subseteq X \) as
  \begin{equation}\label{def:function_set_continuity/uniform_equicontinuous}
    \forall V \in \mscrV \ \underbrace{\exists U \in \mscrU \ \forall f \in \mscrF \ \forall x_1, x_2 \in S} : (x_1, x_2) \in U \implies (f(x_1), f(x_2)) \in V
  \end{equation}

  Compare this to \term{pointwise equicontinuity} of \( \mscrF \) on \( S \), as defined by \fullref{def:function_set_continuity/equicontinuous} for all \( x_1, x_2 \in S \),
  \begin{equation}\label{def:function_set_continuity/pointwise_equicontinuous}
    \forall V \in \mscrV \ \underbrace{\forall x_1, x_2 \in S \ \exists U \in \mscrU \ \forall f \in \mscrF} : (x_1, x_2) \in U \implies (f(x_1), f(x_2)) \in V
  \end{equation}
  to \term{functionwise uniform continuity} of \( \mscrF \) on \( S \), which is defined by \fullref{def:uniform_continuity/uniform} for all \( f \in \mscrF \),
  \begin{equation}\label{def:function_set_continuity/uniform_functionwise}
    \forall V \in \mscrV \ \underbrace{\forall f \in \mscrF \ \exists U \in \mscrU \ \forall x_1, x_2 \in S} : (x_1, x_2) \in U \implies (f(x_1), f(x_2)) \in V
  \end{equation}
  and to \term{functionwise pointwise continuity} of \( \mscrF \) on \( S \), i.e. regular continuity as defined by \fullref{thm:uniform_space_local_continuity} for all \( x_1, x_2 \in S \) and all \( f \in \mscrF \),
  \begin{equation}\label{def:function_set_continuity/functionwise_pointwise}
    \forall V \in \mscrV \ \underbrace{\forall f \in \mscrF \ \forall x_1, x_2 \in S \ \exists U \in \mscrU} : (x_1, x_2) \in U \implies (f(x_1), f(x_2)) \in V
  \end{equation}
\end{definition}

\begin{proposition}\label{thm:uniform_limit_of_continuous_functions}
  \hfill
  \begin{thmenum}
    \ilabel{thm:uniform_limit_of_continuous_functions/continuous} A locally \hyperref[def:function_net_convergence]{uniform limit} of functions continuous at a \hyperref[thm:uniform_space_local_continuity]{point} is continuous at that point.
    \ilabel{thm:uniform_limit_of_continuous_functions/uniform} A uniform limit of functions uniformly \hyperref[def:uniform_continuity]{continuous} on a set is uniformly continuous on the set.
  \end{thmenum}
\end{proposition}
\begin{proof}
  The two proofs are similar but have a lot of subtle differences.

  Fix uniform spaces \( (X, \mscrU) \) and \( (Y, \mscrV) \). Let \( \{ f_k \}_{k \in \mscrK} \) be a \hyperref[def:topological_net]{net} of functions from \( S \subseteq X \) to \( (Y, \mscrV) \).

  \SubProofOf{thm:uniform_limit_of_continuous_functions/continuous} Assume that the functions \( f_k, k \in \mscrK \) are continuous and that they converge to the function \( f \) locally \hyperref[def:function_net_convergence/locally_uniform]{uniformly}.

  Fix an entourage \( W \in \mscrV \) and use \fullref{def:uniform_space/U3} to obtain \( V \subseteq W \) such that \( 3V \subseteq W \).

  . and a point \( x_0 \in S \). Let \( A \) be a neighborhood of \( x_0 \). From locally \hyperref[def:function_net_convergence]{uniform convergence}, there exists an index \( k_0 \in \mscrK \) such that
  \begin{equation*}
    \forall k > k_0 \ \forall x \in A : (f_k(x), f(x)) \in V.
  \end{equation*}

  Fix \( k > k_0 \). From \hyperref[def:function_net_convergence/locally_uniform]{uniform continuity}, there exists an entourage \( U \in \mscrU \) such that
  \begin{equation*}
    \forall x \in S : (x, x_0) \in U \implies (f_k(x_0), f_k(x)) \in V.
  \end{equation*}

  Combining the last two inequalities, we note that for any \( x \in A \),
  \begin{itemize}
    \item \( (f(x_0), f(x)) \in V \),
    \item \( (f_k(x_0), f(x_0)) \in V \),
    \item \( (f_k(x), f(x)) \in V \),
  \end{itemize}
  thus by applying the triangle inequality in \fullref{thm:entourage_simulates_metric} twice, we obtain
  \begin{equation*}
    (f(x_0), f(x)) \in 3V \subseteq W \quad\forall x \in A \cap B(x_0, U).
  \end{equation*}

  Given an entourage \( W \in \mscrV \), we found a neighborhood \( A \cap B(x_0, U) \) of \( x_0 \) such that \fullref{thm:uniform_space_local_convergence/topological_source} is satisfied. Thus \( f \) is continuous at \( x_0 \).

  \SubProofOf{thm:uniform_limit_of_continuous_functions/continuous} Assume that the functions \( f_k, k \in \mscrK \) are uniformly continuous and that they converge to \( f \) \hyperref[def:function_net_convergence/locally_uniform]{uniformly}.

  As in \fullref{thm:uniform_limit_of_continuous_functions/continuous}, fix entourages \( V, W \in \mscrV \) such that \( 3V \subseteq W \). From \hyperref[def:uniform_continuity]{uniform continuity},
  \begin{equation*}
    \forall k \in \mscrK \ \exists U \in \mscrU \ \forall x_1, x_2 \in S : (x_1, x_2) \in U \implies (f_k(x_1), f_k(x_2)) \in V.
  \end{equation*}

  From \hyperref[def:function_net_convergence]{uniform convergence}, there exists an index \( k_0 \in \mscrK \) such that
  \begin{equation*}
    \forall k > k_0 \ \forall x \in S : (f_k(x), f(x)) \in V.
  \end{equation*}

  Fix an index \( k > k_0 \) and let \( U \in \mscrU \) be such that
  \begin{equation}\label{thm:uniform_limit_of_continuous_functions/uniform/continuity}
    \forall x_1, x_2 \in S : (x_1, x_2) \in U \implies (f_k(x_1), f_k(x_2)) \in V.
  \end{equation}

  For any two points \( x_1, x_2 \in S \), we also have that
  \begin{equation}\label{thm:uniform_limit_of_continuous_functions/uniform/convergence}
    (f(x_i), f_k(x_i)) \in V, i = 1, 2.
  \end{equation}

  Analogously to \fullref{thm:uniform_limit_of_continuous_functions/continuous}, from \eqref{thm:uniform_limit_of_continuous_functions/uniform/continuity} and \eqref{thm:uniform_limit_of_continuous_functions/uniform/convergence}, we obtain
  \begin{equation*}
    \forall x_1, x_2 \in S : (x_1, x_2) \in U \implies (f(x_1), f(x_2)) \in 3V \subseteq W.
  \end{equation*}

  Thus the entourage \( U \) depends on \( W \) and not on \( x_1 \) and \( x_2 \). Technically, it also depends on \( k_0 \), however we are only concerned with existence and not uniqueness. Hence \( f \) is uniformly continuous.
\end{proof}

\begin{definition}\label{def:category_of_uniform_spaces}
  Uniform spaces and \hyperref[def:uniform_continuity]{uniformly continuous functions} form a subcategory of \( \cat{Top} \) (see \fullref{def:category_of_topological_spaces}). We denote this category by \( \cat{Met} \).
\end{definition}

\begin{definition}\label{def:fundamental_net}
  A \hyperref[def:topological_net]{net} \( \{ x_k \}_{k \in \mscrK} \) in a uniform space \( (X, \mscrV) \) is called a \term{fundamental net} or \term{Cauchy net} if
  \begin{equation*}
    \forall V \in \mscrV \ \exists k_0 \in \mscrK \ \forall k, m \geq k_0 : (x_k, x_m) \in V.
  \end{equation*}
\end{definition}

\begin{lemma}\label{thm:convergent_net_is_fundamental}
  A net in a uniform space that has a limit \hyperref[def:net_convergence/limit]{point} is \hyperref[def:fundamental_net]{fundamental}.
\end{lemma}
\begin{proof}
  If \( x_0 \) is a limit point of the net \( \{ x_k \}_{k \in \mscrK} \), the net is eventually in every ball \( B(x_0, V) \), which implies \fullref{def:fundamental_net}.
\end{proof}

\begin{definition}\label{def:complete_uniform_space}\mcite\cite[446]{Engelking1989}
  A uniform space is called \term{complete} if it is \hyperref[def:separation_axioms/T2]{Hausdorff} and if every \hyperref[def:fundamental_net]{fundamental net} \hyperref[def:net_convergence/limit]{converges}.

  The \term{completion} of uniform space \( (X, \mscrV) \) is a (\hyperref[def:uniform_continuity]{uniformly continuous}) \hyperref[def:morphism_invertibility/monomorphism]{embedding} \( f: X \to Y \) into a complete uniform space \( (Y, \mscrU) \) such that \( \img(X) \) is \hyperref[def:topologically_dense_set/dense]{dense} in \( Y \).
\end{definition}

\begin{theorem}[Uniform space completion]\label{thm:uniform_space_completion}\mcite\cite[thm. 8.3.12]{Engelking1989}
  Every uniform space has a unique (up to an isomorphism) \hyperref[def:complete_uniform_space]{completion}.

  See also \fullref{thm:metric_space_completion}.
\end{theorem}

\begin{theorem}[Cauchy's net convergence criterion]\label{thm:cauchys_net_convergence_criterion}
  A net in a complete \hyperref[def:complete_uniform_space]{uniform space} is convergent if and only if it \hyperref[def:fundamental_net]{fundamental}.

  Explicitly, a net \( \{ x_k \}_{k \in \mscrK} \) in a complete uniform space \( (X, \mscrV) \) is \hyperref[def:net_convergence/limit]{convergent} if and only if for every entourage \( V \in \mscrV \) there exists an index \( k_0 \) such that
  \begin{equation*}
    (x_k, x_m) \in V \quad\forall k, m \geq k_0.
  \end{equation*}
\end{theorem}
\begin{proof}
  \SufficiencySubProof Given by \fullref{thm:convergent_net_is_fundamental}
  \NecessitySubProof Given by \fullref{def:complete_uniform_space}
\end{proof}


% Metric spaces
\subsection{Metric topology}\label{subsec:metric_topology}

\begin{definition}\label{def:metric_space}\mcite[248]{Engelking1989}
  A \term{metric space} is a set \( X \) along with a nonnegative real-valued function \( \rho: X \times X \to [0, \infty) \), called a \term{metric}, also called the \term{distance function}, such that
  \begin{thmenum}[series=def:metric_space]
    \thmitem[def:metric_space/M1]{M1} \( \rho(x, y) = 0 \iff x = y \)
    \thmitem[def:metric_space/M2]{M2}(symmetry) \( \rho(x, y) = \rho(y, x) \)
    \thmitem[def:metric_space/M3]{M3}(triangle inequality) \( \rho(x, y) \leq \rho(x, z) + \rho(z, y) \)
  \end{thmenum}

  If instead of \ref{def:metric_space/M1} we have the weaker condition
  \begin{thmenum}[resume=def:metric_space]
    \thmitem[def:metric_space/pseudometric_identity]{M1'} \( \forall x \in X, \rho(x, x) = 0 \),
  \end{thmenum}
  we call \( \rho \) a \term{pseudometric} and \( (X, \rho) \) a \term{pseudometric space}.

  \begin{thmenum}
    \thmitem{def:metric_space/subspace} If \( A \subseteq X \) is a set, then the restriction \( (A, \rho{\rvert_A}) \) is a metric space and it is called a \term{subspace} of \( X \).

    \thmitem{def:metric_space/ball} Define the function
    \begin{balign*}
       & B: X \times (0, \infty) \to \pow(X),                   \\
       & B(x, r) \coloneqq \{ y \in X \colon \rho(x, y) < r \}.
    \end{balign*}

    The set \( B(x, r) \) is called an \term{open ball} or simply \term{ball} with \term{center} \( x \) and \term{radius} \( r \).

    The ball \( B = B(0, 1) \) is called the \term{unit ball}.

    \thmitem{def:metric_space/closed_ball} The set
    \begin{equation*}
      \overline{B(x, r)} \coloneqq \cl(B(x, r))
    \end{equation*}
    is called the \term{closed ball} with center \( x \) and radius \( r \).

    \thmitem{def:metric_space/sphere} The set
    \begin{equation*}
      S(x, r) \coloneqq \fr{B(x, r)}
    \end{equation*}
    is called the \term{sphere} with center \( x \) and radius \( r \).

    \thmitem{def:metric_space/bounded_set} A set \( A \subseteq X \) is called \term{bounded} if it is contained in some ball \( B(x, r) \).

    \thmitem{def:metric_space/bounded_sequence} A \hyperref[def:sequence]{sequence} \( \{ x_k \}_{k=1}^\infty \subseteq X \) is called \term{bounded} if the corresponding set \( \{ x_k \colon k = 1, 2, \ldots \} \) is \hyperref[def:metric_space/bounded_set]{bounded}.

    \thmitem{def:metric_space/bounded_metric} If every set is bounded, we say that the metric itself is bounded.

    \thmitem{def:metric_space/bounded_function} We say that a function \( f: S \to X \) from a set \( S \) to a metric space \( (X, \rho) \) is \term{bounded} if its image \( f(S) \) is a bounded set in \( (X, \rho) \).

    \thmitem{def:metric_space/diameter} Define the function
    \begin{balign*}
       & \diam: \pow(X) \to [0, \infty],                             \\
       & \diam(A) \coloneqq \sup \{ \rho(x, y) \colon x, y \in A \},
    \end{balign*}
    whose values include the nonnegative extended real \hyperref[def:extended_real_numbers]{numbers}.

    If it exists, we call the number \( \diam(A) \) the \term{diameter of \( A \)}.

    \thmitem{def:metric_space/distance} Define the function
    \begin{balign*}
       & \op{dist}: X \times \pow(X) \to [0, \infty),                    \\
       & \op{dist}(x, A) \coloneqq \inf \{ \rho(x, a) \colon a \in A \}.
    \end{balign*}

    We call the number \( \op{dist}(x, A) \) the \term{distance from the point \( x \) to the set \( A \)}. We use the convention that the infimum of an empty set of real numbers is \( +\infty \), hence \( \op{dist}(x, \varnothing) = \infty \).
  \end{thmenum}
\end{definition}

\begin{proposition}\label{thm:pseudometric_to_metric}
  Let \( (X, \rho) \) be a \hyperref[def:metric_space]{pseudometric space}. Define the equivalence relation
  \begin{equation*}
    x \cong y \iff \rho(x, y) = 0.
  \end{equation*}

  Then the following metric on the \hyperref[thm:equivalence_partition]{quotient set} \( M \coloneqq X / \cong \)
  \begin{balign*}
     & \rho: M \times M \to [0, \infty)    \\
     & \rho([x], [y]) \coloneqq \rho(x, y)
  \end{balign*}
  is well-defined.
\end{proposition}
\begin{proof}
  The function \( \rho \) is well-defined since, if \( x \) and \( y \) both belong to the same equivalence class \( [x] \), then \( \rho(x) = \rho(y) \). Thus, \( \rho \) does not depend on the choice of representatives.

  Additionally, \( \rho \) is a metric since \( \rho([x], [y]) = 0 \) implies that \( [x] = [y] \), that is, \( \rho(x, y) = 0 \).
\end{proof}

\begin{proposition}\label{rem:bounded_set_metric_order_equivalence}
  A set \( A \) in a metric space \( (X, \rho) \) is \hyperref[def:metric_space/bounded_set]{bounded} if and only if the set \( \{ \rho(a, b) \colon a, b \in A \} \) is bounded as a \hyperref[def:partially_ordered_set_extremal_points/upper_and_lower_bounds]{partially ordered set}.
\end{proposition}

\begin{definition}\label{def:metric_topology}\mcite[249]{Engelking1989}
  Let \( (X, \rho) \) be a metric space. We define the \term{metric topology}\( \mscrT \) , also called the \term{induced topology}, as the \hyperref[def:topological_space]{topology} generated by the \hyperref[def:topological_local_base]{neighborhood system}
  \begin{equation}\label{def:metric_topology/integer_base}
    \mathcal{B}(x) \coloneqq \{ B(x, \tfrac 1 n) \colon n = 1, 2, 3, \ldots \}.
  \end{equation}

  If for some topological space \( (X, \mscrT) \) there exists a metric such that \( \mscrT \) is its induced topology, we say that the topology \( \mscrT \) is \term{metrizable}.

  It is often conventional to consider the alternative (larger) base
  \begin{equation}\label{def:metric_topology/real_base}
    \mathcal{B}'(x) \coloneqq \{ B(x, \varepsilon) \colon \varepsilon > 0 \}.
  \end{equation}
\end{definition}
\begin{proof}
  This is indeed a neighborhood system as it satisfies \ref{thm:topological_local_base_axioms/BP1}-\ref{thm:topological_local_base_axioms/BP3}:

  \begin{refenum}
    \refitem{thm:topological_local_base_axioms/BP1} Every point \( x \) belongs to any ball centered at \( x \).

    \refitem{thm:topological_local_base_axioms/BP2} Fix \( x \in X \) and two balls \( B(x, \tfrac 1 n) \) and \( B(x, \tfrac 1 m) \). Then
    \begin{equation*}
      B(x, \tfrac 1 {\max\{ n, m \}}) \subseteq B(x, n) \cap B(x, m).
    \end{equation*}

    \refitem{thm:topological_local_base_axioms/BP3} Fix \( x, y \in X \) and let \( x \in B(y, \tfrac 1 n) \), i.e. \( \rho(x, y) < \tfrac 1 n \).

    \begin{figure}
      \centering
      \text{\todo{Add diagram}}\iffalse\begin{mplibcode}
        u := 1cm;
        r := sqrt(2) / 2;

        pair v;
        v := (1, 1);

        beginfig(1);
        draw fullcircle scaled 3u;
        dotlabel.bot("$y$", origin);

        draw fullcircle scaled 1u shifted (-r * u * v);
        dotlabel.bot("$x_1$", -r * u * v);

        draw fullcircle scaled 1u shifted (r/2 * u * v);
        dotlabel.bot("$x_2$", r/2 * u * v);
        endfig;
      \end{mplibcode}\fi
      \caption{There is a nested ball around every point in an open ball}\label{def:metric_topology/nested_balls}
    \end{figure}

    Define \( m \) to be the the smallest positive integer such that
    \begin{equation*}
      \tfrac 1 m \leq \min\{ \rho(x, \tfrac 1 n), \tfrac 1 n - \rho(x, \tfrac 1 n) \}.
    \end{equation*}

    Note that \( m \) exists since the positive integers are \hyperref[def:well_founded_relation]{well-founded}.

    Let \( z \in B(x, \tfrac 1 m) \). There are two cases:
    \begin{itemize}
      \item If \( \rho(x, y) \leq \tfrac 1 {2n} \), then
            \begin{balign*}
              \rho(z, y)
              \leq
              \rho(z, x) + \rho(x, y)
              <
              \tfrac 1 m + \rho(x, y)
              \leq
              \tfrac 1 n + \tfrac 1 n
              \leq
              2 \tfrac 1 {2n}
              =
              \tfrac 1 n.
            \end{balign*}

      \item If \( \rho(x, y) > \tfrac 1 {2n} \), then
            \begin{balign*}
              \rho(z, y)
              \leq
              \rho(z, x) + \rho(x, y)
              <
              \tfrac 1 m + \rho(x, y)
              \leq
              (\tfrac 1 n - \rho(x, y)) + \rho(x, y)
              =
              \tfrac 1 n.
            \end{balign*}
    \end{itemize}

    In both cases, \( B(x, \tfrac 1 m) \subseteq B(y, \tfrac 1 n) \).
  \end{refenum}
\end{proof}

\begin{proposition}\label{thm:def:metric_topology/properties}
  The metric topology \( \mscrT \) on \( X \) induced by \( \rho \) has the following properties:
  \begin{thmenum}
    \thmitem{thm:def:metric_topology/ball_is_open} For every point \( x \in X \) and any radius \( r > 0 \), the ball \( B(x, r) \) is an open set and, hence, a neighborhood of \( x \).
    \thmitem{thm:def:metric_topology/first_countable} \( \mscrT \) is first-countable.
    \thmitem{thm:def:metric_topology/sequential} \( \mscrT \) is sequential.
    \thmitem{thm:def:metric_topology/hausdorff} \( \mscrT \) is Hausdorff.
  \end{thmenum}
\end{proposition}
\begin{proof}
  \SubProofOf{thm:def:metric_topology/ball_is_open} Obvious from \fullref{def:metric_topology}.

  \SubProofOf{thm:def:metric_topology/first_countable} Since \fullref{def:metric_topology} involves generating a topology using a neighborhood system of countable local neighborhoods, \( \mscrT \) is first-countable.

  \SubProofOf{thm:def:metric_topology/sequential} Follows from \fullref{thm:def:metric_topology/first_countable} and \fullref{thm:first_countable_spaces_are_sequential}.

  \SubProofOf{thm:def:metric_topology/hausdorff} Let \( x, y \in X \) be distinct points. Define
  \begin{equation*}
    r \coloneqq \dfrac 1 2 \rho(x, y),
  \end{equation*}
  so that
  \begin{equation*}
    B(x, r) \cap B(y, r) = \varnothing.
  \end{equation*}
\end{proof}

\begin{definition}\label{def:metric_uniformity}
  Let \( (X, \rho) \) be a metric space.

  We define the \term{metric uniformity} \( \mscrV \), also called the \term{induced uniformity}, as the \hyperref[def:uniform_space]{uniformity} generated by the countable \hyperref[thm:uniform_space_base_axioms]{base}
  \begin{equation}\label{def:metric_uniformity/integer_base}
    \mathcal{B} \coloneqq \{ V_n \colon n = 1, 2, \ldots \},
  \end{equation}
  where
  \begin{equation*}
    V_n \coloneqq \rho^{-1}([0, \tfrac 1 n)).
  \end{equation*}

  As for the \hyperref[def:metric_topology]{metric topology}, we can instead consider the base
  \begin{equation}\label{def:metric_uniformity/real_base}
    \mathcal{B}' \coloneqq \{ \rho^{-1}([0, \varepsilon)) \colon \varepsilon > 0 \}.
  \end{equation}
\end{definition}
\begin{proof}
  Each relation \( V_r \) is obviously an entourage by \ref{def:metric_space/M1} and \ref{def:metric_space/M2}. We will prove that \( \mathcal{B} \) is indeed a uniform space base.

  \begin{refenum}
    \refitem{thm:uniform_space_base_axioms/BU1} For nonnegative integers \( n, m \) we have
    \begin{equation*}
      V_n \cap V_m
      =
      \{ (x, y) \in X \times X \colon \rho(x, y) < \tfrac 1 n \T{and} \rho(x, y) < \tfrac 1 m \}
      =
      V_{\max\{ n, m \}}.
    \end{equation*}

    Pick any integer \( k \geq \max\{ n, m \} \), so that
    \begin{equation*}
      V_k \subseteq V_n \cap V_m.
    \end{equation*}

    \refitem{thm:uniform_space_base_axioms/BU2} Fix \( V_n \in V \) and \( m \coloneqq 2n \). By the triangle inequality, we have that if \( \rho(x, y) < m \) and \( \rho(y, z) < m \), then
    \begin{equation*}
      \rho(x, z) \leq \rho(x, y) + \rho(y, z) < \tfrac 1 m + \tfrac 1 m = \tfrac 1 n.
    \end{equation*}

    Thus,
    \begin{equation*}
      V_m + V_m
      =
      \left\{ (x, z) \colon \exists y \in X: \rho(x, y) < \tfrac 1 m \T{and} \rho(y, z) < \tfrac 1 m \right\}
      \subseteq
      V_n
    \end{equation*}

    \refitem{thm:uniform_space_base_axioms/BU3} \ref{def:metric_space/M1} implies that
    \begin{equation*}
      \bigcap \mscrB = \lim_{n \to \infty} \rho^{-1}([0, \tfrac 1 n)) = \Delta_X.
    \end{equation*}
  \end{refenum}
\end{proof}

\begin{proposition}\label{thm:metric_topology_coincides_with_uniform_topology}
  The \hyperref[def:metric_topology]{metric topology} and the \hyperref[def:uniform_topology]{uniform topology} from the \hyperref[def:metric_uniformity]{metric uniformity} coincide.
\end{proposition}

\begin{theorem}\label{thm:countable_uniform_base_implies_metrizable}\mcite[thm. 8.1.21]{Engelking1989}
  A uniform space \( X \) is metrizable if and only \( w(X) \leq \aleph_0 \).
\end{theorem}

\medskip

\begin{definition}\label{def:isometry}\mcite[253]{Engelking1989}
  Let \( (X, \rho) \) and \( (Y, \nu) \) be two \hyperref[def:metric_space]{metric spaces}. We say that the function \( f: X \to Y \) is a \term{distance preserving map} or \term{isometry} or \term{isometric embedding} if
  \begin{equation*}
    \forall x, y \in X, \rho(x, y) = \nu(f(x), f(y)).
  \end{equation*}

  If \( f \) is bijective, we say that \( X \) and \( Y \) are \term{isometric}.
\end{definition}

\begin{proposition}\label{thm:isometry_is_injective}
  An \hyperref[def:isometry]{isometry} \( f: (X, \rho) \to (Y, \nu) \) is always injective.
\end{proposition}
\begin{proof}
  If \( f(x) = f(x') \), then by \fullref{def:metric_space/M1}, \( x = x' \).
\end{proof}

\begin{definition}\label{def:category_of_metric_spaces}
  Metric spaces and monotone maps form a subcategory of \( \cat{Unif} \) (see \fullref{def:category_of_uniform_spaces}). We denote this category by \( \cat{Met} \).
\end{definition}

\begin{definition}\label{def:equivalent_metrics}
  Two metrics \( \rho \) and \( \nu \) on the set \( X \) are said to be \term{equivalent} if \( \rho \) and \( \nu \) have the same \hyperref[def:metric_topology]{metric topology}. They are said to be \term{strongly equivalent} if there exist constants \( \alpha, \beta \in \BbbR \) such that for every \( x, y \in X \) we have
  \begin{equation*}
    \alpha \nu(x, y) \leq \rho(x, y) \leq \beta \nu(x, y).
  \end{equation*}
\end{definition}

\begin{remark}\label{rem:metric_space_convergence}
  All types of convergence from \fullref{subsec:topological_nets}, \fullref{subsec:topological_continuity} and \fullref{subsec:uniform_spaces} hold in metric spaces using the \hyperref[def:metric_topology]{metric topology} and \hyperref[def:metric_uniformity]{metric uniformity} structure.

  It is conventional to prefer the bases \fullref{def:metric_topology/real_base} and \fullref{def:metric_uniformity/real_base} to the bases \fullref{def:metric_topology/integer_base} and \fullref{def:metric_topology/integer_base}.

  For example, given two metric spaces \( X \) and \( Y \), continuity of \( f: X \to Y \) at \( x_0 \in X \) (see \fullref{def:local_continuity}) is usually written using the \enquote{epsilon-delta notation} as
  \begin{equation*}
    \forall \varepsilon > 0 \ \exists \delta > 0 : \rho_X(x, x_0) < \delta \implies \rho_Y(f(x), f(x_0)) < \varepsilon
  \end{equation*}
  for any \( x \in X \).
\end{remark}

\begin{definition}\label{def:translation_invariant_metric}
  A \hyperref[def:metric_space]{metric} \( \rho \) on a \hyperref[def:magma]{magma} \( G \) is said to be \term{left translation-invariant} if
  \begin{equation*}
    \rho(ax, ay) = \rho(x, y) \quad\forall a, x, y \in G
  \end{equation*}
  and \term{right translation-invariant} if
  \begin{equation*}
    \rho(xa, ya) = \rho(x, y) \quad\forall a, x, y \in G.
  \end{equation*}

  If \( \rho \) is both left and right translation invariant (e.g. for commutative magmas), we simply say that \( \rho \) is \term{translation invariant}.
\end{definition}

\subsection{Complete metric spaces}\label{subsec:metric_convergence}

\begin{definition}\label{def:complete_metric_space}
  A metric space is said to be \term{complete} if
  \begin{thmenum}
    \thmitem{def:complete_metric_space/sequences} Every fundamental sequence converges.
    \thmitem{def:complete_metric_space/uniform} It is complete as a uniform space in the sense of \fullref{def:complete_uniform_space}
  \end{thmenum}
\end{definition}
\begin{proof}
  The equivalence is due to \fullref{thm:def:metric_topology/sequential} and \fullref{thm:def:metric_topology/hausdorff}.
\end{proof}

\begin{proposition}\label{thm:fundamental_sequence_is_bounded}
  In a metric space, any \hyperref[def:fundamental_net]{fundamental sequence} \( \{ x_k \}_{i=1}^n \) is \hyperref[def:metric_space/bounded_sequence]{bounded}.
\end{proposition}
\begin{proof}
  Since the set
  \begin{equation*}
    I \coloneqq \{ x_k \colon k \leq k_0 \}
  \end{equation*}
  is finite, it has a finite \hyperref[def:metric_space/diameter]{diameter}.

  Fix \( \varepsilon > 0 \). Since the sequence is fundamental, there exists an index \( k_0 \) such that
  \begin{equation*}
    \rho(x_k, x_m) < \varepsilon \quad\forall k, m \geq k_0.
  \end{equation*}

  We are only interested in the case \( \rho(x_{k_0}, x_m) < \varepsilon \).

  Let \( k < k_0 \) and \( m \geq k_0 \). Then
  \begin{equation*}
    \rho(x_k, x_m) \leq \rho(x_k, x_{k_0}) + \rho(x_{k_0}, x_m) < \diam(I) + \varepsilon,
  \end{equation*}
  which is a finite number.

  Thus, the distance between any two elements of the sequence is finite and the sequence is bounded.
\end{proof}

\begin{proposition}\label{thm:fundamental_subsequence_convergence}
  In any \hyperref[def:complete_metric_space]{metric space}, a \hyperref[def:fundamental_net]{fundamental sequence} converges to a value if and only if it has a subsequence that converges to the same value.
\end{proposition}
\begin{proof}
  Let \( (X, \rho) \) be a metric space and let \( \{ x_k \}_{k=1}^\infty \) be a fundamental sequence.

  \SufficiencySubProof Obvious
  \NecessitySubProof Assume that the subsequence \( \{ x_{k_n} \}_{n=1}^\infty \) converges to \( x \). Fix \( \varepsilon > 0 \). There exist \( k_0 \) and \( n_0 \) such that
  \begin{balign*}
     & \rho(x_k, x_m) < \tfrac \varepsilon 2 \quad\forall k, m \geq k_0
     & \rho(x, x_{k_n}) < \tfrac \varepsilon 2 \quad\forall n \geq n_0.
  \end{balign*}

  Fix \( k \geq k_0 \) and let \( n \geq n_0 \) be such that \( k_n \geq k_0 \). Then
  \begin{equation*}
    \rho(x, x_k) \leq \rho(x, x_{k_n}) + \rho(x_{k_n}, x_k) < \varepsilon.
  \end{equation*}

  Since \( \varepsilon \) was arbitrary, we conclude that \( \lim_{k \to \infty} x_k = \lim_{n \to \infty} x_{k_n} = x \).
\end{proof}

\begin{lemma}\label{thm:metric_space_completion_uniqueness}
  Let \( X \) be a metric space. If both \( f: X \to Y \) and \( g: X \to Z \) are \hyperref[def:complete_metric_space]{completions} of \( X \), then \( Y \) and \( Z \) are isometric.
\end{lemma}
\begin{proof}
  Let \( y \in Y \) and let \( \{ x_k \}_{k \to \infty} \subseteq X \) be a sequence such that
  \begin{equation*}
    f(x_k) \xrightarrow[k \to \infty]{} y.
  \end{equation*}

  Such a sequence exists since \( f(X) \) is dense in \( Y \).

  Define \( z \coloneqq \lim_{k \to \infty} g(x_k) \). Since both \( f \) and \( g \) are isometries, \( z \) does not depend on the choice of sequence \( \{ x_k \}_{k \to \infty} \) such that \( f(x_k) \to y \). Furthermore, if \( z \in Z \) is given rather than \( y \in Y \), an analogous process allows us to determine \( y \) uniquely based on \( z \).

  Thus, we have a bijective isometry between \( Y \) and \( Z \).
\end{proof}

\begin{theorem}[Metric space completion]\label{thm:metric_space_completion}
  Every metric space has a unique (up to an isometry) \hyperref[def:complete_metric_space]{completion}.

  This is a special case of \fullref{thm:uniform_space_completion} that we prove fully.
\end{theorem}
\begin{proof}
  Let \( (X, \rho) \) be a metric space. Uniqueness of the completion follows from \fullref{thm:metric_space_completion_uniqueness}. We will only show existence.

  \begin{thmenum}
    \thmitem{thm:metric_space_completion/part_a} First, we build the pseudometric space \( (F, \rho) \). We deal with fundamental sequences and isometries in pseudometric spaces, where the definitions, however, does not change.

    Define \( F \) to be the set of all fundamental \hyperref[def:fundamental_net]{sequences} in X. Define the pseudometric
    \begin{balign*}
       & \rho: F \times F \to \BbbR_{\geq 0}                                                                               \\
       & \rho\left( \{ x_k \}_{k=1}^\infty, \{ y_k \}_{k=1}^\infty \right) \coloneqq \lim_{k \to \infty} \rho(x_k, y_k).
    \end{balign*}

    We first show that is well-defined as a \hyperref[def:function]{function}. Let \( \{ x_k \}_{k=1}^\infty \) and \( \{ y_k \}_{k=1}^\infty \) be two sequences. Fix \( \varepsilon > 0 \). Then there exists an \( k_0 \) such that
    \begin{equation*}
      \rho(x_k, x_k) < \tfrac \varepsilon 2 \text{ and } \rho(y_k, y_m) <  \quad\forall k, m \geq k_0.tfrac \varepsilon 2.
    \end{equation*}

    Fix \( k, m \geq k_0 \). Then
    \begin{equation*}
      \rho(x_k, y_k) \leq \rho(x_k, x_m) + \rho(x_m, y_m) + \rho(y_m, y_k) < \rho(x_m, y_m) + \varepsilon,
    \end{equation*}
    hence
    \begin{equation*}
      \abs{\rho(x_k, y_k) - \rho(x_m, y_m)} < \varepsilon.
    \end{equation*}

    Thus, the sequence \( \{ \rho(x_k, y_k) \}_{k=1}^\infty \) is fundamental and, by \fullref{def:real_numbers_complete_metric_space}, it is convergent.

    Now we check that \( \rho \) is indeed a pseudometric:
    \SubProofOf{def:metric_space/pseudometric_identity} For every sequence \( x \in F \),
    \begin{equation*}
      \rho(x, x) = \lim_{k \to \infty} \rho(x_k, x_k) = 0.
    \end{equation*}
    \SubProofOf{def:metric_space/M2} For all sequences \( x, y \in F \),
    \begin{equation*}
      \rho(x, y) = \lim_{k \to \infty} \rho(x_k, y_k) = \lim_{k \to \infty} \rho(y_k, x_k) = \rho(y, x).
    \end{equation*}

    \SubProofOf{def:metric_space/M3} For all sequences \( x, y, z \in F \),
    \begin{equation*}
      \rho(x, z) = \lim_{k \to \infty} \rho(x_k, z_i) \leq \lim_{k \to \infty} \rho(x_k, y_k) + \lim_{k \to \infty} \rho(y_k, z_i) = \rho(x, y) + \rho(y, z).
    \end{equation*}

    \thmitem{thm:metric_space_completion/part_b} We prove that every fundamental sequence in \( (F, \rho) \) is convergent.

    Let \( \{ c^{(k)} \}_{k=1}^\infty \) be a fundamental sequence (of sequences) in \( (F, \rho) \). Thus, for every \( k = 1, 2, \ldots \), there exists an index \( n_k \) such that
    \begin{equation*}
      \rho(c_m^{(k)}, c_{n_k}^{(k)}) < \tfrac 1 k \quad\forall m \geq n_k.
    \end{equation*}

    Define the sequence
    \begin{equation*}
      d_k \coloneqq c_{n_k}^{(k)}, k = 1, 2, \ldots
    \end{equation*}

    To see that it is fundamental, fix \( \varepsilon > 0 \). Now since the sequence \( \{ c^{(k)} \} \) in \( F \) is fundamental, there exists \( k_0 \) such that
    \begin{equation*}
      \rho(c^{(k)}, c^{(m)}) = \lim_{k \to \infty} \rho(c_k^{(k)}, c_k^{(m)}) < \frac \varepsilon 2 \quad\forall k, m \geq k_0.
    \end{equation*}

    Let \( m_0 \geq k_0 \) be an index such that
    \begin{equation*}
      \frac 2 {m_0} < \frac \varepsilon 2.
    \end{equation*}

    Fix \( k \geq m \geq m_0 \). Let \( l \geq \max \{ n_k, n_m \} \) be such that
    \begin{equation*}
      \rho(c_l^{(k)}, c_l^{(m)}) < \frac \varepsilon 2.
    \end{equation*}

    Then
    \begin{balign*}
      \rho(d_k, d_m)
       & =
      \rho(c_{n_k}^{(k)}, c_{n_m}^{(m)})
      \leq \\ &\leq
      \rho(c_{n_k}^{(k)}, c_l^{(k)}) + \rho(c_l^{(k)}, c_l^{(m)}) + \rho(c_l^{(m)}, c_{n_m}^{(m)})
      \leq \\ &\leq
      \frac 1 k + \frac \varepsilon 2 + \frac 1 m
      \leq
      \frac 2 m + \frac \varepsilon 2
      <
      \varepsilon.
    \end{balign*}

    Thus, we have
    \begin{equation*}
      \rho(d_k, d_m) < \varepsilon \quad\forall k \geq m \geq m_0,
    \end{equation*}
    which proves that the sequence \( \{ d_k \}_{k=1}^\infty \) is fundamental in \( (X, \rho) \).

    Now it remains to show that \( c^{(k)} \xrightarrow[k \to \infty]{} d \) in \( (F, \rho) \).

    Fix \( \varepsilon > 0 \) and let \( k_0 \) be such that
    \begin{equation*}
      \frac 1 {k_0} \leq \frac \varepsilon 2.
    \end{equation*}
    and
    \begin{equation*}
      \rho(d_k, d_m) < \frac \varepsilon 2 \quad\forall k, m \geq k_0.
    \end{equation*}

    Now fix \( i \geq k_0 \). We have, for all \( k \geq i \),
    \begin{balign*}
      \rho(c_{n_k}^{(k)}, d_k)
       & =
      \rho(c_{n_k}^{(k)}, c_{n_k}^{(k)})
      \leq \\ &\leq
      \rho(c_{n_k}^{(k)}, c_{n_k}^{(k)}) + \rho(c_{n_k}^{(k)}, c_{n_k}^{(k)})
      =    \\ &=
      \rho(c_{n_k}^{(k)}, c_{n_k}^{(k)}) + \rho(d_k, d_k)
      <    \\ &<
      \frac 1 k + \frac \varepsilon 2
      <
      \varepsilon.
    \end{balign*}

    Hence,
    \begin{balign*}
      \rho(c^{(k)}, d)
      =
      \lim_{k \to \infty} \rho(c_k^{(k)}, d_k)
      =
      \lim_{k \to \infty} \rho(c_k^{(k)}, c_{n_k}^{(k)})
      <
      \varepsilon.
    \end{balign*}

    Thus, given \( \varepsilon > 0 \), we found an index \( k_0 \) such that
    \begin{equation*}
      \rho(c^{(k)}, d) < \varepsilon \quad\forall g \geq k_0.
    \end{equation*}

    Thus, \( d = \lim_{k \to \infty} c^{(k)} \) and \( (F, \rho) \) is a complete pseudometric space.

    \thmitem{thm:metric_space_completion/part_c} We construct an isometry of \( (X, \rho) \) into \( (F, \rho) \).

    Define the function
    \begin{balign*}
       & \iota: X \to F                        \\
       & \iota(x) \coloneqq (x, x, x, \ldots),
    \end{balign*}
    which sends each element of \( X \) into the corresponding constant sequence in \( F \).

    It is an \hyperref[def:isometry]{isometry} since
    \begin{equation*}
      \rho(\iota(x),\iota(y)) = \lim_{k \to \infty} \rho(x, y) = \rho(x, y).
    \end{equation*}

    \thmitem{thm:metric_space_completion/part_d} We show that the image \( \iota(X) \) is dense in \( (F, \rho) \).

    Fix the fundamental sequence \( y \coloneqq \{ y_k \}_{k=1}^\infty \). Define the sequence \( x \) of sequences
    \begin{equation*}
      x^{(k)} \coloneqq \iota(y_k), i = 1, 2, \ldots
    \end{equation*}

    It is fundamental in \( (F, \rho) \) since \( e \) is an isometry and since \( y \) is fundamental in \( (X, \rho) \).

    Fix \( \varepsilon > 0 \). Let \( k_0 \) be such that
    \begin{equation*}
      \rho(y_k, y_m) < \varepsilon \quad\forall k, m \geq k_0.
    \end{equation*}

    For \( i, k \geq k_0 \), we have
    \begin{balign*}
      \rho(x_k^{(k)}, y_k)
      \leq
      \rho(x_k^{(k)}, y_k) + \rho(y_k, y_k)
      =
      0 + \rho(y_k, y_k)
      <
      \varepsilon,
    \end{balign*}
    hence
    \begin{equation*}
      \rho(x^{(k)}, y) = \lim_{k \to \infty} \rho(x_k^{(k)}, y_k) < \varepsilon.
    \end{equation*}

    We conclude that \( x^{(k)} \xrightarrow[k \to \infty]{} y \) in \( (F, \rho) \), which implies that \( e(X) \) is dense in \( (F, \rho) \).

    \thmitem{thm:metric_space_completion/part_e} We build a complete metric space \( (C, \nu) \) from \( (F, \rho) \).

    We use \fullref{thm:pseudometric_to_metric} to construct a complete metric space \( (C, \nu) \) from the complete pseudometric space \( (F, \rho) \).

    We adapt \( \iota \) to the equivalence classes on \( C \):
    \begin{balign*}
       & \hat\iota: X \to C                 \\
       & \hat\iota(x) \coloneqq [\iota(x)].
    \end{balign*}

    Thus, \( \hat\iota \) embeds \( X \) into the complete metric space \( C \).
  \end{thmenum}
\end{proof}

\begin{proposition}\label{thm:metric_space_is_dense_in_completion}
  Every \hyperref[def:metric_space]{metric space} is dense in its \hyperref[thm:metric_space_completion]{completion}.
\end{proposition}

\begin{theorem}[Cantor's nested compact theorem]\label{thm:cantors_nested_compact_theorem}
  A descending sequence of nonempty compact sets \( F_1 \supseteq F_2 \supseteq \ldots \) in a complete metric space such that \( \diam(F_i) \to 0 \) intersects at exactly one point (compare with \fullref{thm:noncompact_kuratowskis_lemma}).
\end{theorem}
\begin{proof}
  Choose an element \( x_k \in F_k \) for any \( i = 1, 2, \ldots \). Then the sequence \( \{ x_k \}_{k=1}^\infty \) is fundamental. To see this, let \( \varepsilon > 0 \) and let \( k_0 \) be an index such that \( \diam(F_{k_0}) < \varepsilon \). Then if \( j \geq i \geq k_0 \), \( x_m \) is contained in \( F_k \) and \( \rho(x_k, x_m) < \varepsilon \). Thus, the sequence is indeed fundamental and, since the space is complete, it has a limit point \( x \).

  The point \( x \) is contained in every set \( F_k, i = 1, 2, \ldots \) since all of the sets \( F_k \) are closed (by \fullref{thm:complete_metric_space_compact_conditions}) and contain their limit \hyperref[thm:limit_point_iff_in_closure]{points}. Thus,
  \begin{equation*}
    x \in \bigcap_{k=1}^\infty F_k.
  \end{equation*}

  Furthermore,
  \begin{equation*}
    \diam\left( \bigcap_{k=1}^\infty F_k \right) = 0,
  \end{equation*}
  hence \( x \) is the only point in the intersection.
\end{proof}

\subsection{Hausdorff distance}\label{subsec:hausdorff_distance}

Let \( (X, \mu) \) be a \hyperref[def:complete_metric_space]{complete metric space}.

\begin{definition}\label{def:hausdorff_distance}\mcite[144]{DontchevRockafellar2014}
  Fix two sets \( E \subseteq X \) and \( F \subseteq X \).

  The \term{excess} of \( E \) beyond \( F \) is defined as
  \begin{balign*}
     & e: \pow X \times \pow X \to \BbbR \cup \{ \infty \} \\
     & e(E, F) \coloneqq \begin{cases}
      +\infty,                                                                                    & E = \varnothing, D = \varnothing                      \\
      0,                                                                                          & E = \varnothing, D \neq \varnothing                   \\
      \sup_{x \in E} \op{dist}(x, F) \reloset{*}{=} \inf \{\delta > 0 \colon E \subseteq F_\delta \}, & E \neq \varnothing \nonumber\refstepcounter{equation}
    \end{cases}
  \end{balign*}
  where \( F_\delta \coloneqq \{ y \in X \colon \op{dist}(y, F) \leq \delta \} \).

  The \term{Pompeiu-Hausdorff distance} or simply \term{Hausdorff} distance between them is then defined as
  \begin{equation*}
    h(E, F) \coloneqq \max\{ e(E, F), e(F, E) \} = \inf \{\delta > 0 \colon E \subseteq F_\delta, F \subseteq E_\delta \}.
  \end{equation*}
\end{definition}
\begin{proof}(of the equality \( * \))
  Note that the set
  \begin{equation*}
    F_{e(E, F)} = \{ x \in X \colon \op{dist}(x, F) \leq \sup_{x \in E} \op{dist}(x, F) \}
  \end{equation*}
  obviously includes \( E \).

  Now let \( \delta > 0 \) be any real number that satisfies \( E \subseteq F_\delta \), i.e.
  \begin{equation*}
    E \subseteq F_\delta = \{ x \in X \colon \op{dist}(x, F) \leq \delta \},
  \end{equation*}
  which implies that
  \begin{equation*}
    e(E, F) = \sup_{x \in E} \op{dist}(x, F) \leq \delta.
  \end{equation*}
\end{proof}

\begin{proposition}\label{thm:hausdorff_distance_is_metric}
  The Hausdorff distance is a metric on the nonempty compact subsets of \( X \).
\end{proposition}
\begin{proof}
  Let \( E \), \( F \) and \( G \) be nonempty compact subsets of \( X \).

  The function \( h \) is nonnegative. Since we exclude empty and unbounded sets, We do not care about infinite values.

  \SubProofOf{def:metric_space/M1} Obviously \( h(E, E) = 0 \). If \( h(E, F) = 0 \), then there exists no point of \( E \) outside of \( F \) and vice versa, hence \( E = F \).

  \SubProofOf{def:metric_space/M2} This follows from the symmetry of the \( \max \) function.

  \SubProofOf{def:metric_space/M3} For any point \( y \in X \), we have
  \begin{balign*}
    \op{dist}(x, G)
    =
    \inf_{z \in G} \mu(x, z)
    \leq
    \mu(x, y) + \inf_{y \in G} \mu(y, z)
    =
    \mu(x, y) + \op{dist}(y, G).
  \end{balign*}

  Select \( y \in F \) that minimizes the distance \( \mu(x, y) \) over \( F \) (compactness allows us), so that \todo{Prove \hyperref[thm:weierstrass_extreme_value_theorem]{Weierstrass' theorem}}
  \begin{balign*}
    \op{dist}(x, G)
    \leq
    \mu(x, y) + \op{dist}(y, G)
    =
    \op{dist}(x, F) + \op{dist}(y, G)
    \leq
    \op{dist}(x, F) + e(F, G).
  \end{balign*}

  It now follows that
  \begin{balign*}
    e(E, G)
     & =
    \inf \{\delta > 0 \colon E \subseteq G_\delta \}
    =    \\ &=
    \inf \{\delta > 0 \colon E \subseteq \{ x \in X \colon \op{dist}(x, G) \leq \delta \}
    \leq \\ &\leq
    \inf \{\delta > 0 \colon E \subseteq \{ x \in X \colon \op{dist}(x, F) + e(F, G) \leq \delta, y \in X \}
    =    \\ &=
    e(F, G) + \inf \{\delta > 0 \colon E \subseteq F_\delta \}
    =    \\ &=
    e(F, G) + e(E, F).
  \end{balign*}
\end{proof}

\subsection{Totally bounded sets}\label{subsec:totally_bounded_sets}

Let \( (X, \rho) \) be a \hyperref[def:metric_space]{metric space}.

\begin{definition}\label{def:epsilon_net}
  We say that \( E \subseteq X \) is an \( \varepsilon \)-\term{net} for the set \( A \subseteq X \) if
  \begin{equation}
    A \subseteq \bigcup_{x \in E} B(x, \varepsilon).
  \end{equation}
\end{definition}

\begin{definition}\label{def:totally_bounded_set}
  The space \( A \subseteq X \) is called \term{totally bounded} if any of the following equivalent conditions hold:

  \begin{thmenum}
    \thmitem{def:totally_bounded_set/sets} For every \( \varepsilon > 0 \) there exists a finite cover of \( A \) with sets with diameter at most \( \varepsilon \).
    \thmitem{def:totally_bounded_set/epsilon_net} For every \( \varepsilon > 0 \) there exists a finite \hyperref[def:epsilon_net]{\( \varepsilon \)-net} of \( A \).
    \thmitem{def:totally_bounded_set/zero_noncompactness/sets} \hyperref[def:noncompactness_measures/sets]{Kuratowski's noncompactness measure} \( \alpha(A) \) is zero.
    \thmitem{def:totally_bounded_set/zero_noncompactness/balls} The \hyperref[def:noncompactness_measures/balls]{ball noncompactness measure} \( \beta(A) \) is zero.
    \thmitem{def:totally_bounded_set/fundamental_subsequences} Every sequence in \( A \) admits a \hyperref[def:fundamental_net]{fundamental subsequence}.
  \end{thmenum}

  Totally bounded sets are sometimes called \term{\hyperref[def:compact_space]{precompact}} because of \fullref{thm:metric_compact_iff_sequentially_compact}. This equivalence requires the metric space to be complete, however.
\end{definition}
\begin{proof}
  \EquivalenceSubProof{def:totally_bounded_set/sets}{def:totally_bounded_set/zero_noncompactness/sets} Straightforward.

  \EquivalenceSubProof{def:totally_bounded_set/epsilon_net}{def:totally_bounded_set/zero_noncompactness/balls} Straightforward.

  \ImplicationSubProof{def:totally_bounded_set/epsilon_net}{def:totally_bounded_set/sets} Given \( \varepsilon > 0 \), any cover of \( A \) with balls of radius \( \frac \varepsilon 2 \) is a cover with sets of diameter \( \varepsilon \).

  \ImplicationSubProof{def:totally_bounded_set/sets}{def:totally_bounded_set/epsilon_net} Fix \( \varepsilon > 0 \) and \( \rho \in (0, \varepsilon) \) and let \( A_1, \ldots, A_n \subseteq \pow X \) be a finite cover of \( A \) with sets of diameter at most \( \rho \).

  Choose a point \( x_k \) from every \( A_k \), \( k = 1, \ldots, n \). We then have that for every \( k = 1, \ldots, n \),
  \begin{balign*}
    A_k \subseteq \cl B(x_k, \rho) \subsetneq B(x_k, \varepsilon)
    \\
    \implies A \subseteq \bigcup_{k=1}^n A_k \subseteq \bigcup_{k=1}^n B(x_k, \rho) \subsetneq \bigcup_{k=1}^n B(x_k, \varepsilon),
  \end{balign*}
  hence \( x_1, \ldots, x_n \) are centers of \( \varepsilon \)-balls that cover \( A \).

  \ImplicationSubProof{def:totally_bounded_set/epsilon_net}{def:totally_bounded_set/fundamental_subsequences} Let \( \{ x_n \} \subseteq A \) be any sequence.

  If we assume that \( \{ x_n \} \) has no fundamental subsequence, then there exists \( \varepsilon_0 > 0 \) such that \( \rho(x_k, x_m) > \varepsilon_0 \) for any \( n, m \in \BbbZ_{>0} \).

  Consider a finite cover of \( A \) with \( \varepsilon_0 \)-balls. By the pigeonhole principle, at least one of the balls contains more than one element of the sequence, which contradicts the assumption that all elements of the sequence have a distance of at least \( \varepsilon_0 \).

  Hence an arbitrary sequence in \( A \) has a fundamental subsequence.

  \ImplicationSubProof{def:totally_bounded_set/fundamental_subsequences}{def:totally_bounded_set/epsilon_net} Assume that there exists \( \varepsilon_0 > 0 \), such that \( A \) admits no finite cover by \( \varepsilon_0 \)-balls.

  Define \( x_1 \in X, x_2 \in X \setminus B(x_1, \varepsilon_0), \ldots \), so that every two elements of the sequence \( \{ x_n \} \) have a distance of at least \( \varepsilon_0 \). But then the sequence is does not admit a fundamental subsequence, which contradicts our assumption.

  This contradiction proves that \( A \) admits a finite cover by \( \varepsilon \)-balls for every \( \varepsilon > 0 \).
\end{proof}

\begin{corollary}\label{thm:metric_space_compact_iff_closed_totally_bounded}
  Assume that \( X \) is complete. The set \( A \subseteq X \) is sequentially compact if and only if it is closed and totally bounded.
\end{corollary}
\begin{proof}
  The property that every sequence has a fundamental subsequence is equivalent to sequential compactness for a closed set in a complete metric space.
\end{proof}

\begin{proposition}\label{thm:totally_bounded_sets_are_bounded}
  Totally bounded sets are bounded.
\end{proposition}
\begin{proof}
  Fix a totally bounded set \( A \subseteq X \). Let \( \varepsilon > 0 \) and let \( x_1, x_2, \ldots, x_n \) be a finite \hyperref[def:totally_bounded_set/epsilon_net]{\( \varepsilon \)-net} of \( A \). The distance between two points of the \( \varepsilon \)-net is at most \( 2\varepsilon \). Then
  \begin{equation*}
    A \subseteq \bigcup_{i=1}^n B(x_i, \varepsilon) \subseteq B(x_i, 2 n \varepsilon).
  \end{equation*}

  Hence \( A \) is \hyperref[def:metric_space/bounded_set]{bounded}.
\end{proof}

\begin{proposition}\label{thm:closure_of_totally_bounded_is_totally_bounded}
  If a set \( A \subseteq X \) is totally bounded, then so is its closure \( \cl A \).
\end{proposition}
\begin{proof}
  Let \( \varepsilon > 0 \) and \( \rho \in (0, \varepsilon) \) and let \( x_1, \ldots, x_n \in X \) be the centers of a cover of \( A \) with \( \rho \)-balls.

  If \( y \) is a point in \( \cl A \setminus A \), there exists a point \( z \in A \) with \( \rho(y, z) < \varepsilon - \rho \). Let \( x_k \in A \) be one of the centers whose \( \rho \)-balls contain \( z \). We then have that \( y \in B(x_k, \varepsilon) \) since
  \begin{equation*}
    \rho(x_k, z) \leq \rho(x_k, y) + \rho(y, z) < \rho + \varepsilon - \rho = \varepsilon.
  \end{equation*}

  Hence the balls \( \cl B(x_k, \varepsilon) \) cover \( \cl A \), i.e.
  \begin{equation*}
    \cl A \subseteq \bigcup_{k=1}^n B(x_k, \varepsilon).
  \end{equation*}
\end{proof}

\begin{lemma}[Lebesgue's covering lemma]\label{thm:lebesgues_covering_lemma}
  Assume that \( X \) is complete. Let \( A \subseteq X \) be sequentially compact. Given an open cover \( \mathcal{F} \subseteq \pow A \), there exists a number \( \delta > 0 \) such that every \( \delta \)-ball with a center in \( A \) is contained in some set of the cover \( \mathcal{F} \).
\end{lemma}
\begin{proof}
  Assume that no such number \( \delta > 0 \) exists. Then for any natural number \( n \in \BbbZ_{>0} \), there exists an element \( x_n \in A \) such that the ball \( B(x_n, \frac 1 n) \) is not contained in any set of the cover \( \mathcal{F} \). Since \( A \) is sequentially compact, the sequence \( \{ x_n \}_n \) contains a convergent subsequence \( \{ x_{n_k} \}_k \).

  Define
  \begin{equation*}
    x \coloneqq \lim_{k \to \infty} x_{n_k}.
  \end{equation*}

  Let \( E \) be a set in \( \mathcal{F} \) that contains \( x \). Since \( E \) is open, there exists some radius \( r > 0 \) such that \( B(x, r) \subseteq E \).

  Choose any \( k_0 > \frac 2 r \) such that \( \rho(x_{n_{k_0}}, x) < \frac r 2 \). By the triangle inequality,
  \begin{equation*}
    B \left(x_{n_k}, \frac 1 k \right) \subsetneq B \left(x_k, \frac r 2 \right) \subseteq B(x, r) \subseteq E,
  \end{equation*}
  which contradicts the choice of the sequence \( \{ x_n \}_n \).

  Hence there exists a \( \delta > 0 \) such that for every \( x \in A \), the ball \( B(x, \delta) \) is contained in some element \( E \) of the cover \( \mathcal{F} \).
\end{proof}

\begin{theorem}\label{thm:metric_compact_iff_sequentially_compact}
  Assume that \( X \) is complete. The set \( A \subseteq X \) is compact if and only if it is sequentially compact.
\end{theorem}
\begin{proof}
  \SufficiencySubProof Let \( \mathcal{F} \subseteq \pow X \) be an open cover of \( A \).

  By \fullref{thm:lebesgues_covering_lemma}, there exists \( \delta > 0 \) such that for every \( x \in A \), the ball \( B(x, \delta) \) is contained in some set of the cover \( \mathcal{F} \). Let \( x_1, \ldots, x_n \) be a cover of \( A \) with \( \delta \)-balls.

  For each \( k = 1, \ldots, n \) we have that the ball \( B(x_k, \delta) \) is contained in some set \( E_k \in \mathcal{F} \). Hence \( E_1, \ldots, E_n \) is a finite subcover of \( A \), because
  \begin{equation*}
    A \subseteq \bigcup_{k=1}^\infty B(x_k, \delta) \subseteq \bigcup_{k=1}^\infty E_k.
  \end{equation*}

  Thus \( A \) is compact.

  \NecessitySubProof Let \( A \) be compact. Fix \( \varepsilon > 0 \) and take the cover
  \begin{equation*}
    \mathcal{F} \coloneqq \{ B(a, \varepsilon) \colon a \in A \}.
  \end{equation*}

  By compactness of \( A \), there exists a finite subcover. Thus a finite cover of \( A \) with \( \varepsilon \)-balls exists for every \( \varepsilon > 0 \). \Fullref{def:totally_bounded_set} then implies that total boundedness is equivalent to sequential compactness because \( X \) is complete and \( A \) is closed.
\end{proof}

\begin{corollary}\label{thm:complete_metric_space_compact_conditions}
  The following are equivalent for a set \( A \) in complete metric space:
  \begin{thmenum}
    \thmitem{thm:complete_metric_space_compact_conditions/compact} \( A \) is \hyperref[def:compact_space]{compact}
    \thmitem{thm:complete_metric_space_compact_conditions/sequentially_compact} \( A \) is sequentially \hyperref[def:compact_space/convergent_nets]{compact}.
    \thmitem{thm:complete_metric_space_compact_conditions/closed_totally_bounded} \( A \) is closed and totally \hyperref[def:totally_bounded_set]{bounded}.
  \end{thmenum}
\end{corollary}
\begin{proof}
  \EquivalenceSubProof{thm:complete_metric_space_compact_conditions/compact}{thm:complete_metric_space_compact_conditions/sequentially_compact} The equivalence is given by \fullref{thm:metric_compact_iff_sequentially_compact}.

  \ImplicationSubProof{thm:complete_metric_space_compact_conditions/sequentially_compact}{thm:complete_metric_space_compact_conditions/closed_totally_bounded} The equivalence is given by \fullref{thm:metric_space_compact_iff_closed_totally_bounded}.
\end{proof}

\subsection{Noncompactness measures}\label{subsec:noncompactness_measures}

\begin{definition}\label{def:noncompactness_measures}\mcite[def. 7.1]{Deimling1985}
  Let \( (X, \rho) \) be a \hyperref[def:metric_space]{metric space} and let \( \mscrB \) be the family of \hyperref[def:metric_space/bounded_set]{bounded sets} in \( X \). We define the following functions
  \begin{thmenum}
    \labitem{def:noncompactness_measures/sets} The \term{Kuratowski measure of noncompactness},
    \begin{balign*}
       & \alpha: \mscrB \to [0, \infty)                                                                                                            \\
       & \alpha(A) \coloneqq \inf \{d > 0 \colon \exists U_1, \ldots, U_n \subseteq X: \diam {U_k} < d \T{and} A \subseteq \bigcup_{k=1}^n U_k \}
    \end{balign*}

    \labitem{def:noncompactness_measures/balls} The \term{ball measure of noncompactness},
    \begin{balign*}
       & \beta: \mscrB \to [0, \infty)                                                                                   \\
       & \beta(A) \coloneqq \inf \{r > 0 \colon \exists x_1, \ldots, x_2 \in X: A \subseteq \cup_{k=1}^n B(x_k, r) \}
    \end{balign*}
  \end{thmenum}
\end{definition}

\begin{example}\label{ex:noncompactness_measures}\mcite[exer. 7.3]{Deimling1985}
  Consider the subsets \( B_2 \subseteq B_3 \subseteq B_1 \subseteq C([0, 1]) \), defined by
  \begin{balign*}
    B_1 & \coloneqq \left\{
    x \in C([0, 1]) \colon \begin{aligned}
      0 \leq t \leq 1 \implies 0 \leq x(t) \leq 1 \\
      x(0) = 0, x(1) = 1                          \\
    \end{aligned}
    \right\}
    \\
    B_2 & \coloneqq \left\{
    x \in B_1 \colon \begin{aligned}
      0 \leq t \leq \frac 1 2 \implies 0 \leq x(t) \leq \frac 1 2 \\
      \frac 1 2 \leq t \leq 1 \implies \frac 1 2 \leq x(t) \leq 1
    \end{aligned}
    \right\}
    \\
    B_3 & \coloneqq \left\{
    x \in B_1 \colon \begin{aligned}
      0 \leq t \leq \frac 1 2 \implies 0 \leq x(t) \leq \frac 2 3 \\
      \frac 1 2 \leq t \leq 1 \implies \frac 1 3 \leq x(t) \leq 1
    \end{aligned}
    \right\}
  \end{balign*}

  Then \( \alpha(B_1) = 1, \alpha(B_2) = \frac 1 2, \alpha(B_3) = \frac 1 3 \) and \( \beta(B_1) = \beta(B_2) = \beta(B_3) = \frac 1 2 \).
\end{example}
\begin{proof}
  Since the distance between any two functions from \( B_1 \) is at most 1, we have that \( \diam B_1 = 1 \) and \( \alpha(B_1) \leq 1 \).

  Fix \( \varepsilon > 0 \). For any function \( f \in B_1 \), continuity of \( f \) gives us a radius \( \delta_f > 0 \) such that
  \begin{equation*}
    x < 2 \delta_f \implies f(x) < \varepsilon.
  \end{equation*}

  \begin{figure}
    \centering
    \text{\todo{Add diagram}}\iffalse\begin{mplibcode}
      input metapost/plotting;
      u := 4cm;
      e := 0.16; % epsilon
      d := sqrt(e); % delta

      vardef f(expr x) =
      x ** 2
      enddef;

      vardef tf(expr x) =
      if x < d / 2:
      (2 / d) * x
      elseif x < d:
      f(d) + (1 - f(d)) * ((2 / d) * (d - x))
      else:
      f(x)
      fi
      enddef;

      beginfig(1)
      drawarrow (0, 0) -- (u,  0);
      drawarrow (0, 0) -- (0, u);

      drawoptions(dashed withdots scaled 0.3);
      draw (0, e) scaled u -- (1, e) scaled u;
      label.lft("$\varepsilon$", (0, e) scaled u);

      draw (d, 0) scaled u -- (d, e) scaled u;
      label.bot("$\delta$", (d, 0) scaled u);
      drawoptions();

      draw path_of_plot(f, 0, 1, 0.01, u);
      label.rt("$f$", (0.75, 0.5) scaled u);

      drawoptions(dashed evenly);
      draw path_of_plot(tf, 0, d, 0.01, u);
      drawoptions();
      label.rt("$T_\varepsilon(f)$", (0.3, 0.7) scaled u);
      endfig;
    \end{mplibcode}\fi
    \caption{The operator \( T_\varepsilon \) adds \enquote{spikes} to functions.}\label{ex:noncompactness_measures/spike_plot}
  \end{figure}

  Define
  \begin{balign*}
    T_\varepsilon(f)(x) \coloneqq \begin{cases}
      \frac x {\delta_f},                                       & 0 \leq x < \delta_f          \\
      f(\delta_f) + [1 - f(\delta_f)] (2 - \frac x {\delta_f}), & \delta_f \leq x < 2 \delta_f \\
      f(x),                                                     & x \geq 2 \delta_f,
    \end{cases}
  \end{balign*}
  so that
  \begin{balign*}
    \norm{T_\varepsilon(f) - f}
    \geq
    T_\varepsilon(f) (\delta_f) - f(\delta_f)
    =
    1 - f(\delta_f)
    >
    1 - \varepsilon.
  \end{balign*}

  Additionally, because \( \delta_{T_\varepsilon(f)} < \delta_f \), we have that \( f(\delta_{T_\varepsilon(f)}) < \varepsilon \) and
  \begin{balign*}
    \norm{T_\varepsilon(T_\varepsilon(f)) - f}
    \geq
    T_\varepsilon(T_\varepsilon(f)) (\delta_{T_\varepsilon(f)}) - f(\delta_{T_\varepsilon(f)})
    =
    1 - f(\delta_{T_\varepsilon(f)})
    >
    1 - \varepsilon.
  \end{balign*}

  Thus, proceeding by induction\IND, we see that for any \( m = 1, 2, \ldots \)
  \begin{equation*}
    \norm{T_\varepsilon^m(f) - f} > 1 - \varepsilon,
  \end{equation*}
  where \( T_\varepsilon^m \) denotes repeated application of \( T_\varepsilon \).

  Consider the sequence
  \begin{equation*}
    \{ T_\varepsilon^k(f) \}_{k=0}^\infty = \{ f, T_\varepsilon(f), T_\varepsilon(T_\varepsilon(f)), \ldots \}.
  \end{equation*}

  We can easily see that the distance between any two elements of the sequence, say \( T_\varepsilon^k(f) \) and \( T_\varepsilon^{k+m}(f) \), is strictly greater that \( 1 - \varepsilon \), i.e.
  \begin{balign*}
    \norm{T_\varepsilon^k(f) - T_\varepsilon^{k+m}(f)}
    =
    \norm{T_\varepsilon^k(f) - T_\varepsilon^m(T_\varepsilon^k(f))}
    >
    1 - \varepsilon.
  \end{balign*}

  Hence \( B_1 \) cannot be covered by a finite \( (1-\varepsilon) \)-net and \( \alpha(B_1) \geq 1 - \varepsilon \). Since \( \varepsilon > 0 \) can be made arbitrarily small, this implies that \( \alpha(B_1) \geq 1 \) and, because we already have the reverse inequality, \( \alpha(B_1) = 1 \).

  In the set \( B_2 \), the maximum distance between two functions is \( \frac 1 2 \), thus \( \diam(B_2) = \frac 1 2 \) and \( \alpha(B_2) \leq \frac 1 2 \). We can then define an operator similar to \( T_\varepsilon \) that creates \enquote{spikes} of height \( \frac 1 2 \) to prove the reverse inequality, obtaining
  \begin{equation*}
    \alpha(B_2) = \frac 1 2.
  \end{equation*}

  Finally, the set \( B_3 \) has diameter \( \frac 2 3 \) and hence \( \alpha(B_3) = \frac 2 3 \).

  The ball measure for \( B_1 \) satisfies the inequalities
  \begin{equation*}
    \frac 1 2 \leq \beta(B_1) \leq 1.
  \end{equation*}

  Additionally, \( B_1 \) is strictly contained in the ball centered in the constant function \( \frac 1 2 \) with radius \( \frac 1 2 \), which implies that \( \beta(B_1) \leq \frac 1 2 \), hence \( \beta(B_1) = \frac 1 2 \).

  For \( B_2 \) we have
  \begin{equation*}
    \frac 1 4 \leq \beta(B_2) \leq \frac 1 2.
  \end{equation*}

  Assume\LEM that for some \( \varepsilon > 0 \) the set \( B_2 \) can be covered by a finite set of balls with centers \( \{ f_1, \ldots, f_n \} \subsetneq C([0, 1]) \) and radius \( \frac 1 2 - \varepsilon \).

  Because of continuity, we can find a radius \( \delta > 0 \) such that for all \( f_k, k = 1, \ldots, n \) we have
  \begin{equation*}
    x \in \left[\tfrac {1 - \delta} 2, \tfrac {1 + \delta} 2 \right] \implies \abs{f_k(x) - f_k(\tfrac 1 2)} < \varepsilon.
  \end{equation*}

  Consider the function
  \begin{balign*}
    g(x) \coloneqq \begin{cases}
      0,                                & 0 \leq x < \frac {1 - \delta} 2,                       \\
      \frac{2x + \delta - 1} {2\delta}, & \frac {1 - \delta} 2 \leq x \leq \frac {1 + \delta} 2, \\
      1,                                & \frac {1 + \delta} 2 < x \leq 1.
    \end{cases}
  \end{balign*}

  \begin{figure}
    \centering
    \text{\todo{Add diagram}}\iffalse\begin{mplibcode}
      input metapost/plotting;

      u := 6cm;
      e := 0.1; % epsilon

      vardef f_k(expr x) =
      1 - 1 / (1 + exp(2 - 3 * x))
      enddef;

      vardef g(expr x) =
      if x < (1 - d) / 2:
      0
      elseif x < (1 + d) / 2:
      (2x + d - 1) / 2d
      else:
      1
      fi
      enddef;

      f_minus := f_k(0.5) - e;
      f_plus := f_k(0.5) + e;
      d := (ln(1 / (1 - f_plus) - 1) - ln(1 / (1 - f_minus) - 1)) / 3; % delta
      d_minus := (1 - d) / 2;
      d_plus := (1 + d) / 2;

      beginfig(1)
      drawarrow (-0.1, 0) -- (u,  0);
      drawarrow (0, 0) -- (0, u);

      drawoptions(dashed withdots scaled 0.3);
      draw (0, f_minus) scaled u -- (1, f_minus) scaled u;
      label.lft("$f_k(\frac 1 2) - \varepsilon$", (0, f_minus) scaled u);

      draw (0, f_plus) scaled u -- (1, f_plus) scaled u;
      label.lft("$f_k(\frac 1 2) + \varepsilon$", (0, f_plus) scaled u);

      draw (0.5, 0) scaled u -- (0.5, 1) scaled u;
      label.bot("$\frac 1 2$", (0.5, 0) scaled u);

      draw (d_minus, 0) scaled u -- (d_minus, 1) scaled u;
      label.bot("$\frac {1 - \delta} 2$", (d_minus, 0) scaled u);

      draw (d_plus, 0) scaled u -- (d_plus, 1) scaled u;
      label.bot("$\frac {1 + \delta} 2$", (d_plus, 0) scaled u);
      drawoptions();

      drawoptions(dashed evenly);
      draw path_of_plot(f_k, -0.1, 1, 0.01, u);
      label.rt("$f_k$", (1, f_k(1)) scaled u);
      drawoptions();

      draw path_of_plot(g, 0, 1, 0.005, u);
      label.lft("$g$", (d_minus, 0.1) scaled u);
      endfig;
    \end{mplibcode}\fi
    \caption{The function \( g \) always has points that are far enough from all \( f_k, k = 1, \ldots, n \).}\label{ex:noncompactness_measures/sigmoid_plot}
  \end{figure}

  If \( f_k(\tfrac 1 2) \geq \frac 1 2 \), then \( f_k(\tfrac {1 - \delta} 2) > \tfrac 1 2 - \varepsilon \) and
  \begin{equation*}
    \norm{f_k - g} \geq f_k(\tfrac {1 - \delta} 2) - g(\tfrac {1 - \delta} 2) = f_k(\tfrac {1 - \delta} 2) > \tfrac 1 2 - \varepsilon.
  \end{equation*}

  Analogously, if \( f_k(\tfrac 1 2) < \frac 1 2 \), then \( f_k(\tfrac {1 + \delta} 2) < \tfrac 1 2 + \varepsilon \) and
  \begin{equation*}
    \norm{g - f_k} \geq g(\tfrac {1 + \delta} 2) - f_k(\tfrac {1 + \delta} 2) = 1 - f_k(\tfrac {1 + \delta} 2) > \tfrac 1 2 - \varepsilon.
  \end{equation*}

  Thus, for every \( k = 1, \ldots, n \) we have
  \begin{equation*}
    \norm{g - f_k} > \frac 1 2 - \varepsilon,
  \end{equation*}
  i.e. \( g \) in not contained in a ball of radius \( \frac 1 2 - \varepsilon \) around any of the centers \( f_1, \ldots, f_n \).

  Hence \( \beta(B_2) \geq \frac 1 2 \), which implies \( \beta(B_2) = \frac 1 2 \). Because of the inclusion \( B_2 \subsetneq B_3 \subsetneq B_1 \), we have
  \begin{equation*}
    \frac 1 2 = \beta(B_2) \leq \beta(B_3) \leq \beta(B_1) = \frac 1 2,
  \end{equation*}
  hence \( \beta(B_3) = \frac 1 2 \).
\end{proof}

\begin{theorem}[Kuratowski's noncompactness lemma]\label{thm:noncompact_kuratowskis_lemma}\mcite[exer. 7.4]{Deimling1985}
  Let \( X \) be a Banach space and \( \{ A_n \}_n \) be a decreasing sequence of nonempty closed subsets such that \( \alpha(A_n) \to 0 \). Then \( A \coloneqq \bigcap_n A_n \) is nonempty and compact.
\end{theorem}
\begin{proof}
  The set \( A \) is compact because it is closed as the intersection of closed sets and \( \alpha(A) \leq \alpha(A_n) \to 0 \), hence \( \alpha(A) = 0 \).

  It remains to show that \( A \) is nonempty.
  Choose\AOC any sequence \( \{ x_n \}_n \) where \( x_n \in A_n \). Since any finite set is compact, we have that for any \( k \geq 1 \)
  \begin{balign*}
    \alpha(\{ x_n \}_{n \geq 1})
    =
    \max\{ \alpha(\{ x_n \}_{n < k}), \alpha(\{ x_n \}_{n \geq k}) \}
    =
    \alpha(\{ x_n \}_{n \geq k})
    \leq
    \alpha(A_k) \to 0,
  \end{balign*}
  hence the set \( \{ x_n \colon n \geq 1 \} \) is compact and thus sequentially compact. We can choose a convergent subsequence \( \{ x_{n_k} \}_k \) of \( \{ x_n \}_n \) whose limit lies in every \( A_n \) (since they are closed) and, consequently, in their intersection \( A \). So \( A \) is nonempty.
\end{proof}

\subsection{Lipschitz continuity}\label{subsec:lipschitz_continuity}

\begin{definition}\label{def:lipschitz_continuity}
  Let \( f: X \to Y \) be a function between metric spaces.

  \begin{thmenum}
    \labitem{def:lipschitz_continuity/holder} We say that \( f: X \to Y \) is \term{H\"older continuous} at \( x \in X \) with constant \( L \geq 0 \) and exponent \( \alpha > 0 \) if
    \begin{equation*}
      \rho_Y(f(x_1), f(x_2)) \leq L \rho_X(x_1, x_2)^\alpha \quad\forall x_1, x_2 \in X.
    \end{equation*}

    We refer to the smallest such constant, if any, as \enquote{the} H\"older constant.

    \labitem{def:lipschitz_continuity/locally_holder} We say that \( f \) is \term{locally H\"older continuous} if every point has a neighborhood where \( f \) is H\"older continuous with the same exponent, but possibly with with a different constant.

    \labitem{def:lipschitz_continuity/lipschitz} If \( \alpha = 1 \), we say that \( f \) is \term{Lipschitz continuous}.

    \labitem{def:lipschitz_continuity/contraction} If \( X = Y \) and if \( f \) is Lipschitz with constant \( L < 1 \), we call \( f \) a \term{contraction mapping}.

    \labitem{def:lipschitz_continuity/calm}\cite[53]{DontchevRockafellar2014} We say that \( f \) is \term{calm} at \( x \) if it satisfies the Lipschitz condition with one of the points fixed:
    \begin{equation*}
      \rho_Y(f(x), f(x')) \leq L \rho_X(x, x') \quad\forall x' \in X.
    \end{equation*}
  \end{thmenum}
\end{definition}

\begin{proposition}\label{thm:holder_map_is_uniformly_continuous}
  A H\"older map is uniformly continuous.
\end{proposition}
\begin{proof}
  Let \( f: X \to Y \) be a H\"older map with constant \( L \) and exponent \( \alpha \).

  Fix \( \varepsilon > 0 \). Then is enough to choose \( \delta < \sqrt[\alpha]{\frac \varepsilon L} \) so that
  \begin{equation*}
    \rho_X(x_1, x_2) < \delta \implies \rho_Y(f(x_1), f(x_2)) \leq L \rho_X(x_1, x_2)^\alpha < L \delta^\alpha < \varepsilon.
  \end{equation*}

  This implies uniform continuity.
\end{proof}

\begin{corollary}\label{thm:locally_holder_map_is_continuous}
  A locally H\"older map is continuous.
\end{corollary}

\begin{theorem}[Banach's fixed point theorem]\label{thm:banach_fixed_point_theorem}\mcite[exer. 4.3.J]{Engelking1989}
  A contraction \hyperref[def:lipschitz_continuity/contraction]{mapping} in a \hyperref[def:complete_metric_space]{complete metric space} has a unique fixed \hyperref[def:fixed_point]{point}.
\end{theorem}
\begin{proof}
  Let \( f: X \to X \) be a contraction mapping. Fix any point \( x_0 \in X \) and inductively define the sequence
  \begin{equation*}
    x_{k+1} \coloneqq f(x_k), k = 1, 2, \ldots.
  \end{equation*}

  Fix \( \varepsilon > 0 \). Since \( L < 1 \), there exists an index \( k_0 > \log_L(\varepsilon) \) such that for positive integers \( m \) and \( k > k_0 \),
  \begin{balign*}
    \rho(x_k, x_{k+m})
     & =
    \rho(f^k(x_0), f^{k+m}(x_0))
    \leq \\ &\leq
    L^k \rho(x_0, x_m)
    <    \\ &<
    \varepsilon \rho(x_0, x_m).
  \end{balign*}

  Note that
  \begin{balign*}
    \rho(x_0, x_m)
     & \leq
    \sum_{i=1}^m \rho(x_{i-1}, x_i)
    \leq    \\ &\leq
    \rho(x_0, x_1) \sum_{i=1}^m L^{i-1}
    =       \\ &=
    \rho(x_0, x_1) \frac {1 - L^m} {1 - L}
    \leq    \\ &\leq
    \rho(x_0, x_1) \frac 1 {1 - L}.
  \end{balign*}

  Thus
  \begin{equation*}
    \rho(x_k, x_{k+m}) < \frac {\varepsilon \rho(x_0, x_1)} {1 - L}.
  \end{equation*}

  The constant on the right is linear in \( \varepsilon \) and does not depend on \( k \) or \( m \), hence \( \{ x_k \}_{k=0}^\infty \) is a fundamental sequence. Since \( X \) is complete, the sequence has a limit \( x \).

  Because of the continuity of \( f \) (see \fullref{thm:holder_map_is_uniformly_continuous}),
  \begin{equation*}
    f(x) = f(\lim_{k \to \infty} x_k) = \lim_{k \to \infty} f(x_k) = \lim_{k \to \infty} x_{k+1} = x.
  \end{equation*}
\end{proof}

\subsection{Function oscillation}\label{subsec:function_oscillation}

\begin{definition}\label{def:function_oscillation}
  Let \( X \) be a nonempty set and \( (Y, \rho_{Y}) \) be a metric space. We define the \term{oscillation} of a function on a set as
  \begin{balign*}
     &\omega: \fun(X, Y) \times \pow(X) \to [0, \infty] \\
     &\omega(f, A) \coloneqq \sup \Big\{ \rho_{Y}(f(x), f(y)) \colon (x, y) \in A \Big\}.
  \end{balign*}

  In particular, if \( X \) is itself a metric space, we define its \term{modulus of continuity} \( \omega(f, \delta) \) as the oscillation of \( f \) on the ball \( B(0, \delta) \).
\end{definition}

\begin{proposition}\label{thm:def:function_oscillation/properties}
  The \hyperref[def:function_oscillation]{modulus of continuity} has the following basic properties:
  \begin{thmenum}
    \thmitem{thm:def:function_oscillation/properties/continuity_condition} \( f \) is globally \hyperref[def:uniform_continuity]{uniformly continuous} if and only if for every \( \varepsilon > 0 \) there exists \( \delta > 0 \) such that \( \omega(f, \delta) < \varepsilon \).

    \thmitem{thm:def:function_oscillation/properties/monotone} \( \omega(f, \delta) \) is monotone in \( \delta \).

    \thmitem{thm:def:function_oscillation/properties/cauchy_inequality}\mcite[28]{Николов2020Лекции}For all \( \lambda, \delta > 0 \), we have the following analog of \fullref{thm:cauchy_bunyakovsky_schwarz_inequality}
    \begin{equation}\label{thm:def:function_oscillation/properties/cauchy_inequality/inequality}
      \omega(f, \lambda \delta) \leq \omega(f, \lambda^2) + \omega(f, \delta^2).
    \end{equation}

    \thmitem{thm:def:function_oscillation/properties/single_inequality}\mcite[28]{Николов2020Лекции}For all \( \lambda, \delta > 0 \),
    \begin{equation}\label{thm:def:function_oscillation/properties/single_inequality/inequality}
      \omega(f, \lambda \delta) \leq (\lambda + 1) \omega(f, \delta).
    \end{equation}
  \end{thmenum}
\end{proposition}
\begin{proof}
  \SubProofOf{thm:def:function_oscillation/properties/continuity_condition} Follows directly from \fullref{def:uniform_continuity}.

  \SubProofOf{thm:def:function_oscillation/properties/monotone} A supremum on a larger set is larger.

  \SubProofOf{thm:def:function_oscillation/properties/cauchy_inequality} If \( \lambda \leq \delta \), clearly \( \lambda \delta \leq \delta^2 \). Otherwise, \( \lambda \delta < \lambda^2 \).

  Combining the two inequalities with \fullref{thm:def:function_oscillation/properties/monotone}, we obtain \fullref{thm:def:function_oscillation/properties/cauchy_inequality/inequality}.

  \SubProofOf{thm:def:function_oscillation/properties/single_inequality} Note that
  \begin{equation*}
    \rho_{X}(x, y) < \delta \T{implies} \rho_{Y}(f(x), f(y)) < \omega(f, \delta).
  \end{equation*}

  We can multiply this by \( \lambda \) to obtain
  \begin{equation*}
    \lambda \rho_{X}(x, y) < \lambda \delta \T{implies} \lambda \rho_{Y}(f(x), f(y)) < \lambda \omega(f, \delta).
  \end{equation*}

  If \( \lambda \geq 1 \), then \( \rho_{X}(x, y) \leq \lambda \rho_{X}(x, y) \) and \( \rho_{Y}(f(x), f(y)) \leq \lambda \rho_{Y}(f(x), f(y)) \) and hence
  \begin{equation*}
    \omega(f, \lambda \delta) \leq \lambda \omega(f, \delta).
  \end{equation*}

  Otherwise, \( \lambda < 1 \) and clearly \( \lambda \delta < \delta \), which by \fullref{thm:def:function_oscillation/properties/monotone} implies
  \begin{equation*}
    \omega(f, \lambda \delta) \leq \omega(f, \delta).
  \end{equation*}

  Combining the two cases, we obtain
  \begin{equation*}
    \omega(f, \lambda \delta) \leq \lambda \omega(f, \delta) + \omega(f, \delta),
  \end{equation*}
  which we wanted to prove.
\end{proof}


% Geometry
\section{Geometry}\label{sec:geometry}

\begin{remark}\label{def:coordinates_in_geometry}
  Geometry is the multi-millennium evolution of attempts to measure parts of the earth. Ironically, it may be the main historical justification for the gradual axiomatization of mathematics. Completely abstract results about shapes date at least as early as in Ancient Greece. The important distinction between ancient geometry and modern geometry is the introduction of coordinates in the 17th century.

  An axiomatic approach for a theory of plane and, solid figures was developed by Euclid in the third century BC. Later, Hilbert, Tarski and others independently proposed axioms that fit the requirements of modern logic systems. This is known today as \term{synthetic Euclidean geometry} and is mostly of theoretical interest because modern tools are easier to work with.

  Descartes' idea of coordinates connects problems of algebra and geometry in such a way that most of today's mathematics seamlessly switches between algebraic and geometric interpretations of the same problem. The study of classical Greek geometry in terms of coordinates is known as \term{analytic geometry}.
\end{remark}

\subsection{Affine coordinate systems}\label{subsec:affine_coordinate_system}

\begin{remark}\label{rem:affine_coordinate_system_concept}
  Most humans possess a strong intuition for visual information like drawings or diagrams. A paper or a painting is only a medium for communicating information and emotions. \Fullref{def:euclidean_plane/figures} contains, some highlighted curves that our mind maps to abstract geometric figures, without considering the size limitations of the page, the precision of the drawings or the thickness of the lines.

  \begin{figure}[b]
    \centering
    \text{\todo{Add diagram}}\iffalse\begin{mplibcode}
      u := 1cm;

      beginfig(1);
      draw (0, -1) * u -- (3, 0) * u;
      draw (-1, 2) * u -- (3, 1) * u -- (1, 3) * u -- cycle;
      draw fullcircle scaled 1.5u shifted ((0, 0.5) * u);
      endfig;
    \end{mplibcode}\fi
    \caption{A triangle, a circle and a line in the Euclidean plane.}\label{def:euclidean_plane/figures}
  \end{figure}

  Our goal is to map these visualizations to the concept of vector spaces. Formalisms at the level of formal \hyperref[def:first_order_language]{logic} will not be stated because we only want to sketch, some high-level concepts. We only give definitions that are strictly necessary, plane geometry itself is described in \fullref{subsec:analytic_geometry_in_the_plane}. We will proceed as follows:

  \begin{itemize}
    \item Define an affine plane in \fullref{def:affine_plane} with auxiliary definitions.
    \item Describe the Euclidean plane \( A_2 \) in \fullref{def:euclidean_plane} as a very special affine plane.
    \item Give additional definitions for the Euclidean plane in \fullref{def:euclidean_plane_auxiliary_definitions}.
    \item Define the set \( F_2 \) of free vectors over \( A_2 \) in \fullref{def:euclidean_plane_free_vector}.
    \item Show that \( F_2 \) is a two-dimensional vector space over \( \BbbR \) in \fullref{thm:euclidean_plane_factorization}.
    \item Define coordinate systems that give explicit isomorphisms between \( A_2 \), \( F_2 \) and \( \BbbR^2 \) in \fullref{def:euclidean_plane_coordinate_system}.
    \item Generalize these notions in \fullref{rem:coordinate_systems}
  \end{itemize}
\end{remark}

\begin{definition}\label{def:affine_plane}\mcite[1]{Hartshorne1967}
  An \term{affine plane} consists of
  \begin{itemize}
    \item a set \( X \), whose elements are called \term{points},
    \item a family of subsets of \( X \), whose members are called \term{lines}
  \end{itemize}
  with the additional relations
  \begin{itemize}
    \item a \term{parallel} relation \( l \parallel g \) for lines that holds if either \( l = g \) or if they have no points in common,
    \item a \term{collinearity} relation for a set \( B \) of points that holds if \( B \) is a subset of, some line,
  \end{itemize}
  such that
  \begin{thmenum}
    \thmitem[def:affine_plane/A1]{A1} Given two distinct points, there exists only one line that contains both.
    \thmitem[def:affine_plane/A2]{A2} Given a line \( l \) and a point \( P \not\in l \), there exists exactly one line \( g \parallel l \) that contains \( P \).
    \thmitem[def:affine_plane/A3]{A3} There exist three non-collinear points.
  \end{thmenum}
\end{definition}

\begin{definition}\label{def:euclidean_plane}
  The \term{Euclidean plane} \( A_2 \) is a formalization of a straight infinite surface. An axiomatic definition can be found in \cite{nLab:euclidean_geometry}. We will use that
  \begin{itemize}
    \item The Euclidean plane \( A_2 \) is an \hyperref[def:affine_plane]{affine plane}
    \item \( A_2 \) is a \hyperref[def:complete_metric_space]{complete metric space} with distance \( \op{dist} \).
    \item There is a \term{betweenness} relation for points that says if the point \( R \) is \term{between} \( P \) and \( Q \).
  \end{itemize}

  \begin{figure}
    \centering
    \text{\todo{Add diagram}}\iffalse\begin{mplibcode}
      input metapost/plotting;

      u := 1.5cm;

      beginfig(1);
      path l, g, h, P, Q, R;
      l = (0, -1) * u -- (3, 0) * u;
      draw l;
      label.top("$l$", midpoint of l);

      g = (0, -2) * u -- (3, -1) * u;
      draw g;
      label.bot("$g$", midpoint of g);

      h = (0, 0) * u -- (3, -2) * u;
      draw h;
      label.urt("$h$", midpoint of h);

      P = dot shifted point 0.2 of h;
      fill P;
      label.llft("$P$", midpoint of P);

      Q = dot shifted point 0.8 of h;
      fill Q;
      label.llft("$Q$", midpoint of Q);

      R = dot shifted point 0.4 of h;
      fill R;
      label.llft("$R$", midpoint of R);
      endfig;
    \end{mplibcode}\fi
    \caption{Three lines and three points in the Euclidean plane. The lines \( l \) and \( g \) are collinear, while the point \( R \) is between \( P \) and \( Q \)}\label{def:affine_plane/figure}
  \end{figure}
\end{definition}

\begin{definition}\label{def:euclidean_plane_auxiliary_definitions}
  We will also need the following definitions:
  \begin{thmenum}
    \thmitem{def:affine_plane/half_plane} Every line \( l \) gives rise to two (closed) \term{half-planes} \( H^+ \) and \( H^- \) as follows:
    \begin{itemize}
      \item \( H^+ \cap H^- = l \)
      \item \( H^+ \cup H^- = A_2 \)
      \item If \( P \in H^+ \setminus l \) and \( Q \in H^- \setminus l \), then there is a point \( R \in l \) between \( P \) and \( Q \)
    \end{itemize}

    Note that the superscripts \( + \) and \( - \) are only for distinguishing between the two half-planes and are not assigned based on, some property of the half-planes. See \fullref{def:half_space} for a definition of a half-plane that actually has a concept of signs.

    \begin{figure}
      \centering
      \text{\todo{Add diagram}}\iffalse\begin{mplibcode}
        input metapost/plotting;

        u := 1cm;

        beginfig(1);
        input hatching;

        path l, Hp, Hm;
        l = (0, -1) * u -- (3, 0) * u;
        draw l;

        Hp = l -- (3, 0.5) * u -- (0, 0.5) * u -- cycle;
        hatchfill Hp withcolor (45, 1mm, -.5bp);
        label.ulft("$H^+$", startpoint of l);

        Hm = l -- (3, -1.5) * u -- (0, -1.5) * u -- cycle;
        hatchfill Hm withcolor (135, 1mm, -.5bp);
        label.lrt("$H^-$", endpoint of l);
        endfig;
      \end{mplibcode}\fi

      \caption{Differently hatched half-planes in the Euclidean plane.}\label{def:affine_plane/bound_vector/half_plane}
    \end{figure}

    \thmitem{def:affine_plane/ray} Every line \( l \) and every point \( R \) give rise to two (closed) \term{rays} \( l^+ \) and \( l^- \) as follows:
    \begin{itemize}
      \item \( l^+ \cap l^- = \{ R \} \) are disjoint
      \item \( l^+ \cup l^- = l \)
      \item If \( P \in l^+ \setminus \{ R \} \) and \( Q \in l^- \setminus \{ R \} \), then \( R \) is between \( P \) and \( Q \)
    \end{itemize}

    The rays \( l^+ \) and \( l^- \) are called \term{opposite} of each other.

    We say that \( R \) is the \term{vertex} of \( l^+ \) and \( l^- \).

    See \fullref{def:geometric_ray} for a definition of a ray that actually has a concept of signs.

    \begin{figure}
      \centering
      \text{\todo{Add diagram}}\iffalse\begin{mplibcode}
        input metapost/plotting;

        u := 1cm;

        beginfig(1);
        path l, R;

        l = (0, -1) * u -- (3, 0) * u;
        drawdblarrow l;
        label.lft("$l^-$", startpoint of l);
        label.rt("$l^+$", endpoint of l);

        R = dot shifted midpoint of l;
        fill R;
        label.bot("$R$", midpoint of R);
        endfig;
      \end{mplibcode}\fi

      \caption{Opposite rays in the Euclidean plane.}\label{def:affine_plane/day/figure}
    \end{figure}

    \thmitem{def:affine_plane/rays_unidirectional} Two rays are said to be \term{unidirectional} if there exists a line distinct from the lines containing the rays, such that both rays are contained in the same half-plane with respect to the line.

    \thmitem{def:affine_plane/bound_vector} An ordered pair \( \overrightarrow{PQ} \) of points is called a \term{bound vector}. The point \( P \) is called the \term{beginning} of \( \overrightarrow{PQ} \) and \( Q \) is called the \term{end} of \( \overrightarrow{PQ} \).

    \begin{figure}
      \centering
      \text{\todo{Add diagram}}\iffalse\begin{mplibcode}
        input metapost/plotting;

        u := 0.75cm;

        beginfig(1);
        path P, Q, R, PQ, PR;

        PQ = (0, -1) * u -- (3, 0) * u;
        drawarrow PQ;
        label.bot("$\overrightarrow{PQ}$", midpoint of PQ);

        P = dot shifted startpoint of PQ;
        fill P;
        label.bot("$P$", midpoint of P);

        Q = dot shifted endpoint of PQ;
        label.bot("$Q$", midpoint of Q);

        PR = (0, -1) * u -- (-2, 0.5) * u;
        drawarrow PR;
        label.llft("$\overrightarrow{PR}$", midpoint of PR);

        R = dot shifted endpoint of PR;
        label.llft("$R$", midpoint of R);
        endfig;
      \end{mplibcode}\fi

      \caption{Bound vectors in the Euclidean plane can be regarded as oriented line segment.}\label{def:affine_plane/bound_vector/figure}
    \end{figure}
  \end{thmenum}
\end{definition}

\begin{definition}\label{def:euclidean_plane_free_vector}
  We say that the bound vectors \( \overrightarrow{P_1 Q_1} \) and \( \overrightarrow{P_2 Q_2} \) in \( A_2 \) are \term{congruent} if \( \op{dist}(P_1, Q_1) = \op{dist}(P_2, Q_2) \) and if the rays \( r_i, i = 1, 2 \) beginning at \( P_i \) and containing \( Q_i \), are unidirectional.

  We define \term{free vectors} as \hyperref[thm:equivalence_partition]{equivalence classes} of bound vectors by this congruence relation. We denote the corresponding equivalence partition by \( F_2 \).
\end{definition}

\begin{theorem}\label{thm:euclidean_plane_factorization}
  The set \( F_2 \) of free vectors over \( A_2 \) is a two-dimensional \hyperref[def:vector_space]{vector space} over \( \BbbR \) with the following operations:
  \begin{thmenum}
    \thmitem{thm:euclidean_plane_factorization/sum} We define the \term{sum} of the cosets \( [\overrightarrow{PQ}] \) and \( [\overrightarrow{QR}] \) as the coset \( [\overrightarrow{PR}] \).

    \thmitem{thm:euclidean_plane_factorization/scalar_product} We define the \term{scalar multiplication} of \( \lambda \in \BbbR \) with the coset \( [\overrightarrow{PQ}] \) to be the coset \( [\overrightarrow{PR}] \), where \( \overrightarrow{PR} \) is the unique vector that is unidirectional with \( \overrightarrow{PQ} \) and \( \op{dist}(P, R) = \lambda \op{dist}(P, Q) \).
  \end{thmenum}
\end{theorem}
\begin{proof}
  Proving the well-definedness of the operations and verifying that \( F_2 \) is a two-dimensional vector space requires a lot of work and the proof is skipped.
\end{proof}

\begin{definition}\label{def:euclidean_plane_coordinate_system}
  Just because \fullref{thm:euclidean_plane_factorization} states that the set \( F_2 \) of free vectors is a vector space does not mean that we can work with it as with \( \BbbR^2 \). \Fullref{thm:finite_dimensional_spaces_are_isomorphic} says that \( F_2 \) is isomorphic to \( \BbbR^2 \), however the proof requires the \hyperref[def:zfc/choice]{axiom of choice}. The concrete way to select a basis in \( F_2 \) is through coordinate systems.

  Somewhat confusingly, we define coordinate systems over \( A_2 \) rather than over \( F_2 \), but this will, soon be justified.

  A \term{coordinate system} \( Oxy \) in \( A_2 \) is a choice of
  \begin{thmenum}
    \thmitem{def:euclidean_plane_coordinate_system/origin} A point \( O \in A_2 \), called the \term{origin} of the coordinate system.
    \thmitem{def:euclidean_plane_coordinate_system/basis} An \hyperref[def:partially_ordered_set]{ordered} \hyperref[def:left_module_hamel_basis]{basis} \( (x, y) \) of \( F_2 \), called the \term{basis} of \( Oxy \).
  \end{thmenum}

  What we achieve through the choice of \( O \) is that, for each point \( P \in A_2 \), we select the bound vector \( \overrightarrow{OP} \in V_2 \), called the \term{radius vector} of \( P \). This injects \( A_2 \) into \( V_2 \), however if we take the free vector \( [\overrightarrow{OP}] \), we instead obtain a bijection between \( A_2 \) and \( F_2 \).

  Now that we have a correspondence between \( A_2 \) and \( F_2 \), coordinates for the point \( P \) are defined simply as the \hyperref[def:left_module_basis_projection]{coordinates} of \( [\overrightarrow{OP}] \) with respect to the basis \( (x, y) \).

  Thus, the pair \( (A_2, Oxy) \) has an explicit isomorphism with \( \BbbR^2 \).

  The \term{coordinate axis} of \( x \) is the unique \hyperref[def:affine_plane/ray]{ray} starting at \( O \) and containing the end of \( x \). It is called the \term{abscissa}. The coordinate axis of \( y \) is called the \term{ordinate}.
\end{definition}

\begin{remark}\label{rem:coordinate_systems}
  We sketched how to embed mental images of planes into \( \BbbR^2 \), however in mathematics we are often interested in the opposite: given a set of points in \( \BbbR^2 \), visualize them on a screen or paper and then absorb the the resulting image in our brain.

  This is one of the most powerful constructions in mathematics, yet it is, so intuitive that it is not really given a lot of attention, at least until generalizations are required. Given any vector space \( V \) in the sense of \fullref{def:vector_space}, we want a way to assign a pair of numbers to each vector in \( V \). This is only possible if \( \dim V = 2 \), however we can generalize this to tuples of coordinates via bases - see \fullref{def:left_module_hamel_basis}. This well for finitely dimensional vector spaces, however we need to generalize these notion for infinitely dimensional vector spaces and general modules over \hyperref[def:left_module]{rings}. This allows us to generalize coordinates further to manifolds - see \fullref{def:topological_manifold}.

  See \fullref{subsec:vector_space_geometry} for immediate generalizations of the concepts introduced here.
\end{remark}

\subsection{Vector space geometry}\label{subsec:vector_space_geometry}

\begin{remark}\label{rem:real_field_extensions}
  When speaking about vector spaces, we usually restrict ourselves to vector spaces over \( \BbbR \) or, at most, \( \BbbC \). This restriction may seem arbitrary, however important concepts like \hyperref[def:geometric_ray]{rays} or \hyperref[def:convex_set]{convexity} requires the field to be an extension of \( \BbbR \) and it just so happens that, by \fullref{thm:fundamental_theorem_of_algebra} and \fullref{thm:no_finite_extensions_of_closed_fields}, the only finite extension \hyperref[def:field_extension]{fields} of \( \BbbR \) are \( \BbbR \) and \( \BbbC \). It is technically possible to work with infinite extension fields, however in practice vector spaces over \( \BbbC \) are isoteric enough. A benefit of considering only \( \BbbC \) is given in \fullref{rem:linear_functionals_over_c}.
\end{remark}

\begin{definition}\label{def:geometric_shape}
  A \term{geometric shape} is an informal notion that refers to certain special subsets of a vector space, usually defined in a coordinate-independent manner. Shapes in two-dimensional spaces are called \term{figures} and shapes in three dimensions are called \term{surfaces}.

  When two geometric shapes shapes intersect, we say that they are \term{incident}.
\end{definition}

\begin{definition}\label{def:point}
  A \term{point} is a simple geometric \hyperref[def:geometric_shape]{shape} comprising a singleton subset of any set (usually a vector space or a topological space). We use the convention \fullref{rem:singleton_sets} and, unless the distinction is important, we do not distinguish between singleton sets and their only element - e.g. in \fullref{def:simplex/point}.

  Points are also called vectors, which is justified by \fullref{def:euclidean_plane_coordinate_system}.
\end{definition}

\begin{definition}\label{def:euclidean_transformation}
  The following bijections from \( \BbbR^n \) to \( \BbbR^n \) are collectively called \term{Euclidean transformations} or \term{rigid motions} in \( \BbbR^n \):

  \begin{thmenum}
    \labitem{def:euclidean_transformation/translation} For any vector \( v \in \BbbR^n \), the function
    \begin{equation*}
      \op{t}_v(x) \coloneqq x + v
    \end{equation*}
    is called a \term{translation} along the \term{direction} \( v \).

    This easily generalizes to an arbitrary \hyperref[def:magma]{magma} \( (M, \cdot) \) by setting
    \begin{equation*}
      \op{t}_v(x) \coloneqq v \cdot x
    \end{equation*}
    for some \( v \in M \).

    \labitem{def:euclidean_transformation/dilation} For any scalar \( \lambda \in \BbbR^n \), the function
    \begin{equation*}
      \op{d}_\lambda(x) \coloneqq \lambda x
    \end{equation*}
    is called a \term{dilation} or \term{scaling} by \( \lambda \).

    This easily generalizes to an arbitrary (left) \hyperref[def:magma]{module} \( (M, +, \cdot) \).

    \labitem{def:euclidean_transformation/homothety} A composition of a \hyperref[def:euclidean_transformation/dilation]{dilation} with a \hyperref[def:euclidean_transformation/translation]{translation} is called a \term{homothety}.

    \labitem{def:euclidean_transformation/rotation} For any point \( p \in \BbbR^n \) and any special (i.e. \hyperref[def:matrix_determinant]{determinant} one) \hyperref[def:orthogonal_matrix]{orthogonal matrix} \( A \in \op{O}_n(\BbbR) \), the function
    \begin{equation*}
      \op{rot}_{A,p}(x) \coloneqq p + A (x - p)
    \end{equation*}
    is called a \term{rotation} by \( A \) around \( p \).

    \labitem{def:euclidean_transformation/reflection} For any point \( p \in \BbbR^n \), the function
    \begin{equation*}
      \op{ref}_p(x) \coloneqq 2p - x
    \end{equation*}
    is called a \term{point reflection} or \term{inversion} with respect to the point \( p \).
  \end{thmenum}
\end{definition}

\begin{definition}\label{def:zero_locus}
  Let \( S \) be an arbitrary set and let \( M \) be a \hyperref[def:unital_magma]{unital} \hyperref[def:magma]{magma} with identity \( e \).

  The \term{zero locus} or \term{set of zeros} of a function \( f: S \to M \) is the preimage
  \begin{equation*}
    f^{-1}(e) = \{ x \in X \colon f(x) = e \}.
  \end{equation*}

  In practice, \( M \) is usually a \hyperref[def:semiring/ring]{ring} or a \hyperref[def:left_module]{module}, in which case the zero locus is defined for the additive group, i.e.
  \begin{equation*}
    f^{-1}(0_M) = \{ x \in X \colon f(x) = 0_M \}.
  \end{equation*}

  See also \fullref{def:unital_magma_kernel}, \fullref{def:semiring_kernel} and \fullref{def:left_module_kernel}.
\end{definition}

\begin{definition}\label{def:hypersurface}
  A \term{hypersurfaces} can have different meanings depending on the context. We are interested in

  \begin{thmenum}
    \labitem{def:hypersurface/parametric} A parametric hypersurface (\fullref{def:parametric_hypersurface}) is a purely topological definition.
    \labitem{def:hypersurface/algebraic} An affine variety (\fullref{def:affine_variety}) is a purely algebraic definition.
    \labitem{def:hypersurface/geometric} A manifold (\fullref{def:topological_manifold}) can be regarded as a geometric definition.
  \end{thmenum}

  Note that all of the enumerated hypersurfaces have a concept of dimension. Hypersurfaces of dimension \( 2 \) are simply called \term{surfaces} (see \fullref{def:affine_variety/algebraic_surface}) and hypersurfaces of dimension \( 1 \) are called \term{curves} (see also \fullref{def:affine_variety/algebraic_curve} and \fullref{def:affine_variety/algebraic_curve}).
\end{definition}

\begin{definition}\label{def:geometric_line}
  A particularly important \hyperref[def:hypersurface]{curve} is a \term{line} in a vector space \( X \) over any field \( \BbbK \), which can be defined equivalently as

  \begin{thmenum}
    \labitem{def:geometric_line/subspace} A subspace of \( X \) of \hyperref[def:vector_space_dimension]{dimension} one. Note that this is not consistent with the other definitions because this defines only lines through the origin \( 0_X \). Hence if \( L \subseteq X \) is a line (a subspace of dimension one) and if \( a \in X \) is any point, we define a line with origin \( a \) to be the translation \( a + L \).

    \labitem{def:geometric_line/algebraic} An \hyperref[def:affine_variety/algebraic_curve]{algebraic curve} in \( X \) given by a polynomial of degree one.

    \labitem{def:geometric_line/parametric} If the field \( \BbbK \) is ordered (usually when \( \BbbK = \BbbR \)), we can define a line with \term{directional vector} \( x \) and \term{origin} \( a \) as the parametric curve
    \begin{balign*}
       & l: \BbbK \to X   \\
       & l(t) = tx + a.
    \end{balign*}
  \end{thmenum}
\end{definition}

\begin{definition}\label{def:geometric_ray}
  If \( X \) is a vector space over \( \BbbK \in \{ \BbbR, \BbbC \} \), we define \term{closed rays} with a vertex \( a \) as the parametric curves
  \begin{balign*}
     & l^+: [0, \infty) \to X \\
     & l^+(t) = tx + a
  \end{balign*}
  and
  \begin{balign*}
     & l^-: (-\infty, 0] \to X \\
     & l^-(t) = tx + a
  \end{balign*}

  If the inequalities are strict, we instead obtain \term{open rays}.

  Unless explicitly noted otherwise, we assume that the vertex of the ray is \( 0 \) because every ray is a translation of a ray centered at \( 0 \).
\end{definition}

\begin{definition}\label{def:geometric_cone}
  An open (resp. closed) \term{cone} in a vector space over \( \BbbK \in \{ \BbbR, \BbbC \} \) is a union of open (resp. closed) \hyperref[def:geometric_ray]{rays} with a common vertex, called the \term{vertex} of the cone.
\end{definition}

\begin{definition}\label{def:hyperplane}
  Dually to \hyperref[def:geometric_line]{lines}, another particularly important \hyperref[def:hypersurface]{hypersurface} is a \term{hyperplane} in a vector space \( X \) over any field.

  \begin{thmenum}
    \labitem{def:hyperplane/subspace} A \term{linear hyperplane} is simply a subspace of \( X \) of \hyperref[def:vector_space_dimension]{codimension} one. As in \fullref{def:geometric_line/subspace}, we define a \term{affine hyperplane} to be a \hyperref[def:euclidean_transformation/translation]{translation} of a linear hyperplane.

    \labitem{def:hyperplane/kernel} Linear hyperplanes (as defined in \fullref{def:hyperplane/subspace}) are simply zero \hyperref[def:zero_locus]{loci} (\hyperref[def:semiring_kernel]{kernels}) of linear \hyperref[def:linear_operator]{functionals}. and affine hyperplanes are zero loci of \hyperref[def:affine_operator]{affine functionals}.
  \end{thmenum}
\end{definition}

\begin{example}\label{ex:hyperplanes}
  Affine \hyperref[def:hyperplane]{hyperplanes} in \( \BbbR^2 \) are \hyperref[def:geometric_line]{lines} and affine hyperplanes in \( \BbbR^3 \) are planes.

  Linear \hyperref[def:hyperplane]{hyperplanes} in \( \BbbR^2 \) are the lines passing through the origin \( (0, 0) \) and linear hyperplanes in \( \BbbR^3 \) are the planes incident to \( (0, 0, 0) \).
\end{example}

\begin{definition}\label{def:half_space}
  Vector spaces over \( \BbbR \) have the concept of \term{half-spaces}. Given a \hyperref[def:hyperplane]{hyperplane} \( H \) of the real vector space \( X \), defined by the affine functional \( l(x) = \inprod {x^*} x + a \), its closed half-spaces are defined as
  \begin{equation*}
    H^+ \coloneqq \{ l(x) \geq 0 \} = \{ \inprod {x^*} x \geq -a \}
  \end{equation*}
  and
  \begin{equation*}
    H^- \coloneqq \{ l(x) \leq 0 \} = \{ \inprod {x^*} x \leq -a \}.
  \end{equation*}

  If the inequalities are strict, we instead obtain \term{open half-spaces}.
\end{definition}

\begin{definition}\label{def:polyhedron}
  A \term{polyhedron} in a real vector space is an intersection of \hyperref[def:half_space]{half-spaces}.
\end{definition}

\begin{definition}\label{def:hyperplane_separation}
  Although \hyperref[def:hyperplane]{hyperplanes} are defined for vector spaces over an arbitrary field \( \BbbK \), we define \term{hyperplane separation} only for \( \BbbK \in \{ \BbbR, \BbbC \} \) (see \fullref{rem:real_field_extensions}).

  We say that the sets \( A, B \subseteq X \) are \term{separated} by the linear functional \( l \in X^* \) if there exists a real number \( c \in \BbbR \) such that
  \begin{equation}\label{def:hyperplane_separation/normal}
    \real l(x) < c \leq \real l(y) \quad\forall x \in A, y \in B.
  \end{equation}

  See \fullref{rem:linear_functionals_over_c} for a justification of only considering the real part of \( l \).

  The asymmetry in the inequalities \fullref{def:hyperplane_separation/normal} can be inverted by considering \( -l(x) \) and \( -c \).

  We say that \( A \) and \( B \) are \term{strongly separated} by \( l \) if both inequalities in \fullref{def:hyperplane_separation/normal} are strict:
  \begin{equation}\label{def:hyperplane_separation/strong}
    \real l(x) < c < \real l(y) \quad\forall x \in A, y \in B.
  \end{equation}

  We can regard \( l \) as a hyperplane as in \fullref{def:hyperplane/kernel}, which justifies the terminology \enquote{hyperplane separation}. It is more correct, however, especially if \( \BbbK = \BbbR \), to say that they are separated by the affine hyperplane \( l(x) + c \).

  If \( \BbbK = \BbbR \) \fullref{def:hyperplane_separation/normal} is equivalent to requiring that \( A \) is contained in an open \hyperref[def:half_space]{half-space} relative to \( l(x) + c \) and that \( B \) is contained in the complementing closed half-space (or vice-versa). \Fullref{def:hyperplane_separation/strong} then states that both \( A \) and \( B \) are contained in opposite open half-spaces.
\end{definition}

\begin{definition}\label{def:convex_set}
  \hfill
  \begin{thmenum}
    \labitem{def:convex_set/line_segment} Given two points \( x, y \in X \) in a Banach space \( X \), we define the \term{line segment} between \( x \) and \( y \) as the parametric curve \( t \mapsto tx + (1-t)y, t \in [0, 1] \). The image
    \begin{equation*}
      [x, y] \coloneqq \{ tx + (1-t)y \colon t \in [0, 1] \}
    \end{equation*}
    of this parametric curve is called the \term{convex hull} of \( x \) and \( y \). We usually use the term \enquote{line segment} to refer to the convex hull itself.

    The length \( \len([x, y]) \) of a line segment is defined as \( \norm{x - y} \).

    \labitem{def:convex_set/hull} We define the convex hull \( \conv A \) of a set \( A \subseteq X \) as the union of all line segments with endpoints in \( A \).

    \labitem{def:convex_set/set} We call a set \term{convex} if it coincides with its convex hull, that is, if it contains the line segment between any two of its points.
  \end{thmenum}
\end{definition}

\begin{proposition}\label{thm:convex_set_properties}
  \hyperref[def:convex_set]{Convex sets} have the following basic properties:

  \begin{thmenum}
    \labitem{thm:convex_set_properties/closed_under_combinations} A convex set is closed under convex \hyperref[def:linear_combination/convex]{combinations}.
    \labitem{thm:convex_set_properties/cone_closed_under_combinations} A closed \hyperref[def:convex_set]{convex} \hyperref[def:geometric_cone]{cone} is closed under conic \hyperref[def:linear_combination/conic]{combinations}.
    \labitem{thm:convex_set_properties/closed_under_intersections} Any intersection of convex sets is convex.
  \end{thmenum}
\end{proposition}
\begin{proof}
  \SubProofOf{thm:convex_set_properties/closed_under_combinations} Fix a convex set \( C \). Let \( \sum_{k=1}^n t_k x_k \) be a convex combination of elements of \( C \).

  We will use induction\IND on \( n \). If \( n = 1 \), this is obvious. If \( n = 2 \), this is given by definition. Assume that it is true for \( n - 1 \). Denote \( s \coloneqq \sum_{k=1}^{n-1} t_k \). If \( s = 0 \), take another convex combination in order to handle all the possible cases of the induction\IND. Suppose \( s \neq 0 \). Then
  \begin{equation*}
    \sum_{k=1}^n t_k x_k
    =
    s \sum_{k=1}^n \frac {t_k} s x_k
    =
    s \underbrace{\sum_{k=1}^{n-1} \frac {t_k} s x_k}_{\eqqcolon y} + t_n x_n.
  \end{equation*}

  By the inductive conjecture, \( y \in C \). Note that \( s \in [0, 1] \) and that \( s + t_n = 1 \) by definitions of \( s \). Then \( s y + t_n x_n \) is a binary convex combination that we know is contained in \( C \) by definition.

  \SubProofOf{thm:convex_set_properties/cone_closed_under_combinations} Fix a cone \( C \). Let \( \sum_{k=1}^n t_k x_k \) be a conic combination of elements of \( C \). Each vector \( x_k \) lies on a closed ray, say \( r_k \), thus \( t_k x_k \) also lies on \( r_k \).

  Therefore we only need to show that the sum of two elements \( x_1, x_2 \in C \) is again in \( C \). This is true because \( x_1 + x_2 \) is a convex combination of \( 2x_1 \in r_1 \) and \( 2x_2 \in r_2 \).

  \SubProofOf{thm:convex_set_properties/closed_under_intersections} Let \( X = \cap_{\alpha \in \mscrK} X_\alpha \) be an intersection of convex sets. Take \( x, y \in X \) and \( t \in [0, 1] \). Then \( tx + (1-t)y \in X_\alpha \) for all \( \alpha \in \mscrK \), hence \( tx + (1-t)y \in X \). Therefore \( X \) is convex.
\end{proof}

\begin{definition}\label{def:simplex}
  A \( k \)-\term{simplex} is the convex \hyperref[def:convex_set/hull]{hull} of \( k + 1 \) affinely \hyperref[affine_independence]{independent} vectors called the \term{vertices} of the simplex. The convex hull of any subset of the vertices is called a \term{face} of the simplex.

  \begin{thmenum}
    \labitem{def:simplex/point} A \( 0 \)-simplex is a \hyperref[def:point]{point}.
    \labitem{def:simplex/line_segment} A \( 1 \)-simplex is a line segment as defined in \fullref{def:convex_set/line_segment}.
    \labitem{def:simplex/triangle} A \( 2 \)-simplex is a triangle as defined in \fullref{def:triangle}.
    \labitem{def:simplex/tetrahedron} A \( 3 \)-simplex is called a \term{tetrahedron}.
  \end{thmenum}
\end{definition}

\begin{definition}\label{def:k_cell}
  A \( k \)-cell is a \hyperref[def:cartesian_product]{Cartesian product} of \( k \) nonempty \hyperref[def:total_order_interval/closed]{closed intervals} of real numbers.

  \begin{thmenum}
    \labitem{def:k_cell/point} A \( 0 \)-cell is a \hyperref[def:point]{point}.
    \labitem{def:k_cell/interval} A \( 1 \)-cell is a closed interval.
    \labitem{def:k_cell/rectangle} A \( 2 \)-cell is called a \term{rectangle}. If a rectangle \( R \) is a product of two copies of the same interval, i.e. if \( R = [a, b]^2 \), we say that \( R \) is a \term{square} with side \( b - a \).
    \labitem{def:k_cell/parallelepiped} A \( 3 \)-cell is called a \term{parallelepiped}. If \( R = [a, b]^3 \), we say that \( R \) is a \term{cube} with side \( b - a \).
  \end{thmenum}
\end{definition}

\begin{definition}\label{def:neighborhood_set_types}
  The following topology-independent definitions are often used for neighborhoods in a topological vector space \( X \):

  \begin{thmenum}
    \labitem{def:neighborhood_set_types/absorbing} \( A \) is \term{absorbing} if \( \bigcup_{k=0}^\infty kA = X \).
    \labitem{def:neighborhood_set_types/symmetric} \( A \) is \term{symmetric} if \( -A = A \).
    \labitem{def:neighborhood_set_types/balanced} \( A \) is \term{balanced} if \( tA \subseteq A \) for any \( t \in [0, 1] \).
  \end{thmenum}
\end{definition}

\begin{definition}\label{def:collinear_complanar}
  The geometric version of \hyperref[def:left_module_linear_dependence]{linear independence} has two special names: we say that the set \( A \subseteq X \) of any vector space \( X \) is \term{collinear} (on the same line) if \( \dim(\linspan(A)) \leq 1 \) and \term{complanar} (on the same plane) if \( \dim(\linspan(A)) \leq 2 \).
\end{definition}

\subsection{Analytic geometry in the plane}\label{subsec:analytic_geometry_in_the_plane}

\begin{remark}\label{rem:analytic_geometry}
  Analytic geometry is a XVII-century branch of mathematics that studies geometric figures using coordinate \hyperref[rem:coordinate_systems]{systems}. The term \enquote{analytic geometry} may refer to a modern subbranch of algebraic geometry, however we refrain from using \enquote{analytic geometry} in that sense. Historically, most of these definitions were given either for the Euclidean \hyperref[def:euclidean_plane]{plane} or for the three-dimensional Euclidean space.

  Most of the definitions from \fullref{subsec:vector_space_geometry} are generalizations of concepts from analytic geometry. We will state definitions in the language of linear algebra and refrain from using synthetic (axiomatic) geometry. When working in the plane (resp. three-dimensional space), we will assume that we have fixed an \hyperref[def:orthonormal_system]{orthonormal} coordinate \hyperref[def:euclidean_plane_coordinate_system]{system} \( Oxy \) (resp. \( Oxyz \)), which allows us to visualize geometric figures.
\end{remark}

\begin{definition}\label{def:plane_line_equations}
  \hyperref[def:geometric_line]{Lines} in \( \BR^2 \) are so ubiquitous that they can be represented by a lot of standard \hyperref[ex:equations]{equations}.

  \begin{DefEnum}
    \ILabel{def:plane_line_equations/vector_parametric} When regarding a line as a parametric curve as in \fullref{def:geometric_line/parametric}, the \hyperref[def:first_order_formula]{formula}
    \begin{equation}\label{def:plane_line_equations/parametric_equation}
      l(t) = tx + a
    \end{equation}
    is called a \Def{vector parametric equation}.

    \ILabel{def:plane_line_equations/scalar_parametric} Given \fullref{def:plane_line_equations/parametric_equation}, the \Def{scalar parametric equations} of the line are
    \begin{equation}\label{def:plane_line_equations/scalar_parametric_equations}
      \begin{cases}
         & l_1(t) = t x_1 + a_1  \\
         & l_2(t) = t x_2 + a_2.
      \end{cases}
    \end{equation}

    \ILabel{def:plane_line_equations/general} When regarding a line as an algebraic curve as in \fullref{def:geometric_line/algebraic}, the equation
    \begin{equation}\label{def:plane_line_equations/general_equation}
      p(x, y) \coloneqq Ax + By + C = 0
    \end{equation}
    is called the \Def{general equation} or simply \Def{equation} of a line in a plane. Either \( A \) or \( B \) must be nonzero so that \( \deg(p) = 1 \).

    Note that multiple general equations can have the same locus (e.g. the entire polynomial ideal \( \Braket{p} \)).

    \ILabel{def:plane_line_equations/normal} If \( A^2 + B^2 = 1 \), we call \fullref{def:plane_line_equations/general_equation} a \Def{normal equation}. This leaves us with only two representatives of \( \Braket{p} \).

    \ILabel{def:plane_line_equations/cartesian} Given \( k, m \in \BR \) and  \( k \neq 0 \), we define the \Def{Cartesian equation} of a line:
    \begin{equation}\label{def:plane_line_equations/cartesian_equation}
      y = kx + m.
    \end{equation}

    We call \( k \) the \Def{slope} of the line.

    This is a special case of \fullref{def:plane_line_equations/general} with \( A = -k \), \( B = -1 \) and \( C = m \). Unlike the general equation, the Cartesian equation of a line is unique.

    Conversely, if \( B \neq 0 \) in \fullref{def:plane_line_equations/general_equation}, we can define \( k = -\tfrac A B \) and \( m = -\tfrac C B \) to form a Cartesian equation.

    \ILabel{def:plane_line_equations/intercept} Given nonzero \( a, b \in \BR \), we define the \Def{intercept equation} of a line:
    \begin{equation}\label{def:plane_line_equations/intercept_equation}
      \frac x a + \frac y b = 1,
    \end{equation}

    This is a special case of \fullref{def:plane_line_equations/general} with \( A = \frac 1 a \), \( B = \frac 1 b \) and \( C = -1 \). The intercept equation of a line is also unique.

    If \( A, B, C \neq 0 \) in \fullref{def:plane_line_equations/general}, we can define an \Def{intercept equation} as \( a = -\tfrac C A \) and \( b = -\tfrac C B \)).
  \end{DefEnum}
\end{definition}

\begin{figure}
  \begin{minipage}[b]{0.40\textwidth}
    \centering
    \begin{mplibcode}
      input metapost/plotting;

      u := 1.5cm;

      beginfig(1);
      path l, x_axis, y_axis;

      x_axis = (-1, 0) scaled u -- (1, 0) scaled u;
      y_axis = (0, -1) scaled u -- (0, 1) scaled u;
      l = (-1 / 2, -1) * u -- (1, 3 / 4) * u;

      drawarrow x_axis;
      label.bot("$x$", point 0.9 of x_axis);
      drawarrow y_axis;
      label.lft("$y$", point 0.9 of y_axis);
      draw l;
      label.bot("$y = kx + m$", startpoint of l);
      endfig;
    \end{mplibcode}
    \Caption{def:plane_line_equations/cartesian_equation_drawing}{A \hyperref[def:geometric_line]{line} in \( \BR^2 \) defined using its \hyperref[def:plane_line_equations/cartesian]{Cartesian equation}.}
  \end{minipage}
  \hspace{0.05\textwidth}
  \begin{minipage}[b]{0.40\textwidth}
    \centering
    \begin{mplibcode}
      input metapost/plotting;
      u := 1.5cm;

      beginfig(1)
      drawarrow (-1 / 2, 0) scaled u -- (2, 0) scaled u;
      drawarrow (0, -1 / 2) scaled u -- (0, 2) scaled u;

      z0 = (1 / 2, 1 / 6) scaled u;
      z1 = (2, 11 / 12) scaled u;
      z2 = (1, 13 / 6) scaled u;

      draw z0 -- (x0, max(y1, y2)) dashed withdots;

      drawarrow z0 -- z1;
      draw (x0, y1) -- z1 dashed evenly;
      label.top("$x_1$", midpoint of ((x0, y1) -- z1));

      drawarrow z0 -- z2;
      draw (x0, y2) -- z2 dashed evenly;
      label.bot("$x_2$", midpoint of ((x0, y2) -- z2));
      endfig;
    \end{mplibcode}
    \Caption{def:angle/figure}{An \hyperref[def:angle/acute]{acute angle} with its measurement segments dashed.}
  \end{minipage}
\end{figure}

\begin{definition}\label{def:angle}
  A \Def{directed angle} is a tuple of two closed \hyperref[def:geometric_ray]{rays} with a common vertex. It is a closed cone. Given two rays \( r_1, r_2 \) with a common vertex, we denote their corresponding directed angle by \( \angle(r_1, r_2) \).

  Suppose that \( r_1 \) and \( r_2 \) have scalar parametric equations
  \begin{equation*}
    r_i: t \mapsto
    \begin{cases}
      tx_i + a_i \\
      ty_i + b_i,
    \end{cases}
    i = 1, 2.
  \end{equation*}

  We write

  The condition of the rays having a common vertex is equivalent to \( a_1 = a_2 \) and \( b_1 = b_2 \). If not specified otherwise, we assume that \( a_1 = a_2 = b_1 = b_2 = 0 \).

  The \Def{measure in radians} of a directed angle, often called the angle itself, is defined as the number (see \fullref{def:geometric_trigonometric_functions})
  \begin{equation*}
    \alpha \coloneqq \Rem(\ATanTwo(y_2, x_2) - \ATanTwo(y_1, x_1), 2\pi).
  \end{equation*}

  We can classify angles based on their measure as
  \begin{DefEnum}
    \ILabel{def:angle/zero} \Def{zero} if \( \alpha = 0 \),
    \ILabel{def:angle/acute} \Def{acute} if \( \alpha \in (0, \tfrac \pi 2) \),
    \ILabel{def:angle/right} \Def{right} if \( \alpha = \tfrac \pi 2 \),
    \ILabel{def:angle/obtuse} \Def{obtuse} if \( \alpha \in (\tfrac \pi 2, \pi) \),
    \ILabel{def:angle/straight} \Def{straight} if \( \alpha = \pi \), in which case the angle is actually a line,
    \ILabel{def:angle/reflex} \Def{reflex} if \( \alpha > \pi \).
  \end{DefEnum}

  We often do not care about the order of the two rays and speak of an \Def{undirected angle}. In this case, the measure of the undirected angle is the smaller of the measures of the two oriented angles. Thus we cannot speak of straight and reflex undirected angles.
\end{definition}

\begin{definition}\label{def:triangle}
  \begin{figure}
    \centering
    \begin{mplibcode}
      input metapost/plotting;

      beginfig(1)
      pair A, B, C;
      path alpha, beta, gamma;

      A := origin;
      B := (3, 0) scaled u;
      C := (2, 2) scaled u;

      draw A -- B -- C -- cycle;

      alpha = fullcircle scaled (u / 2) shifted A cutbefore (A -- B) cutafter (A -- C);
      draw alpha;
      label.urt("$\alpha$", point 0.4 of alpha);

      beta = fullcircle scaled (u / 2) shifted B cutbefore (B -- C) cutafter (B -- A);
      draw beta;
      label.ulft("$\beta$", point 1.4 of beta);

      gamma = fullcircle scaled (u / 4) shifted C cutbefore (A -- C) cutafter (B -- C);
      draw gamma;
      label.bot("$\gamma$", point 0.6 of gamma);

      fill dot shifted A;
      fill dot shifted B;
      fill dot shifted C;

      label.llft("$A$", A);
      label.lrt("$B$", B);
      label.top("$C$", C);

      label.rt("$a$", midpoint of (B -- C));
      label.ulft("$b$", midpoint of (A -- C));
      label.bot("$c$", midpoint of (A -- B));
      endfig;
    \end{mplibcode}
    \Caption{def:triangle/figure}{An \hyperref[def:triangle/acute]{acute triangle}.}
  \end{figure}

  A \Def{triangle} is a triple \( (A, B, C) \) of \hyperref[def:point]{points}, no two of which are \hyperref[def:collinear_complanar]{collinear} (see \fullref{def:simplex/triangle} for a more general definition). The three points are called the \Def{vertices} of the triangle.

  Define the associated \hyperref[def:convex_set/line_segment]{line segments}, called the \Def{sides} of the triangle, and its (undirected) \hyperref[def:angle]{angles} as
  \begin{BreakableAlign*}
    a \coloneqq [B, C], &  & \alpha \coloneqq \angle(b, c), \\
    b \coloneqq [A, C], &  & \beta \coloneqq \angle(a, c),  \\
    c \coloneqq [A, B], &  & \gamma \coloneqq \angle(a, b).
  \end{BreakableAlign*}

  Note that we defined the angles using segments rather than rays but this is immaterial because each to each segment \( [p, q] \) there corresponds exactly one closed ray \( t \mapsto p + t q \).

  We can classify triangles based on their sides as
  \begin{DefEnum}
    \ILabel{def:triangle/isosceles} \Def{isosceles} if at least two of its sides have equal length
    \ILabel{def:triangle/equilateral} \Def{equilateral} if all of its sides have equal length
  \end{DefEnum}
  or based on their angles as
  \begin{DefEnum}
    \ILabel{def:triangle/acute} \Def{acute} if all of its angles are \hyperref[def:angle/acute]{acute}.
    \ILabel{def:triangle/right} \Def{right} if at least one of the angles is \hyperref[def:angle/straight]{straight}.
    \ILabel{def:triangle/obtuse} \Def{obtuse} if at least one of its angles is \hyperref[def:angle/obtuse]{obtuse}.
  \end{DefEnum}
\end{definition}

\begin{definition}\label{def:quadratic_plane_curve}
  The \Def{quadratic plane curves} are algebraic \hyperref[def:hypersurface/algebraic]{curves} given by a bivariate polynomial of degree \( 2 \). The \Def{general equation} of a quadratic plane curve is
  \begin{equation}\label{def:quadratic_plane_curve/general_equation}
    c(x, y) \coloneqq A x^2 + B xy + C y^2 + Dx + Ey + F = 0.
  \end{equation}

  Multiple equation can correspond to the same curve. Not all general equations, however, define algebraic curves. We will not concern ourselves with the details. See \fullref{ex:affine_varieties} for a proof that the unit circle is an algebraic curve. It turns out that the algebraic curves given \fullref{def:quadratic_plane_curve/general_equation} are precisely the ones listed here, collectively known as \Def{conic sections}. We give only canonical forms of the equations; any linear transformation of the corresponding loci is described by another general equation.

  \begin{figure}
    \begin{minipage}{0.3\textwidth}
      \centering
      \begin{mplibcode}
        input metapost/plotting;

        u := 1.25cm;

        vardef scaled_sin(expr x) =
        5 / 6 * sin(x)
        enddef;

        beginfig(1)
        fill dot shifted (u, 0);

        drawarrow (-pi / 2, 0) scaled u -- (pi / 2, 0) scaled u;
        drawarrow (0, -pi / 2) scaled u -- (0, pi / 2) scaled u;

        drawarrow path_of_curve(cos, scaled_sin, -1 / 4 * pi, 3 / 4 * pi, 0.01, u);
        drawarrow path_of_curve(cos, scaled_sin, 3 / 4 * pi, 7 / 4 * pi, 0.01, u);
        endfig;
      \end{mplibcode}
    \end{minipage}
    \hspace{0.02\textwidth}
    \begin{minipage}{0.3\textwidth}
      \centering
      \begin{mplibcode}
        input metapost/plotting;

        u := 1.25cm;

        vardef minus_cosh(expr x) =
        -cosh(x)
        enddef;

        beginfig(1)
        drawarrow (-pi / 2, 0) scaled u -- (pi / 2, 0) scaled u;
        drawarrow (0, -pi / 2) scaled u -- (0, pi / 2) scaled u;

        drawarrow path_of_curve(cosh, sinh, -pi / 3, 0, 0.01, u);
        drawarrow path_of_curve(cosh, sinh, 0, pi / 3, 0.01, u);

        drawarrow path_of_curve(minus_cosh, sinh, -pi / 3, 0, 0.01, u);
        drawarrow path_of_curve(minus_cosh, sinh, 0, pi / 3, 0.01, u);
        endfig;
      \end{mplibcode}
    \end{minipage}
    \hspace{0.02\textwidth}
    \begin{minipage}{0.3\textwidth}
      \centering
      \begin{mplibcode}
        input metapost/plotting;

        u := 1.25cm;

        beginfig(1)
        fill dot;

        drawarrow (-pi / 2, 0) scaled u -- (pi / 2, 0) scaled u;
        drawarrow (0, -pi / 2) scaled u -- (0, pi / 2) scaled u;

        vardef y_upper(expr x) =
        sqrt(x)
        enddef;

        vardef y_lower(expr x) =
        -sqrt(x)
        enddef;

        drawarrow path_of_plot(y_upper, 0, pi / 3, 0.01, u);
        drawarrow path_of_plot(y_lower, 0, pi / 3, 0.01, u);
        endfig;
      \end{mplibcode}
    \end{minipage}
    \Caption{def:quadratic_plane_curve/figure}{An \hyperref[def:quadratic_plane_curve/ellipse]{ellipse}, \hyperref[def:quadratic_plane_curve/hyperbola]{hyperbola} and \hyperref[def:quadratic_plane_curve/parabola]{parabola} defined via their parametric equations. The starting point is highlighted and the direction of the parametric curves is shown.}
  \end{figure}

  \begin{DefEnum}
    \ILabel{def:quadratic_plane_curve/ellipse} An \Def{ellipse} is a quadratic curve whose canonical equation has the form
    \begin{equation}\label{def:quadratic_plane_curve/ellipse/canonical_equation}
      c(x, y) \coloneqq \frac {x^2} {a^2} + \frac {y^2} {b^2} - 1 = 0,
    \end{equation}
    where \( a, b > 0 \).

    If \( a = b \), we say that the ellipse is a \Def{circle} and we call \( a \) the circle's \Def{radius}. The \Def{unit circle} is defined by \( a = b = 1 \). Circles generalize to \hyperref[def:metric_space/sphere]{spheres} in metric spaces.

    \Fullref{def:pi} and \fullref{def:geometric_trigonometric_functions} logically belong here but are extracted separately for brevity.

    We are often interested in defining ellipses via \Def{scalar parametric equations} using \hyperref[def:trigonometric_functions]{trigonometric functions} as follows:
    \begin{equation}\label{def:quadratic_plane_curve/ellipse/parametric_equations}
      \begin{cases}
        x = a \cos(t) \\
        y = b \sin(t),
      \end{cases}
    \end{equation}
    where \( t \in [0, 2\pi) \).

    We will now demonstrate that \fullref{def:quadratic_plane_curve/ellipse/canonical_equation} and \fullref{def:quadratic_plane_curve/ellipse/parametric_equations} describe the same curve. First, suppose that the pair \( (x_0, y_0) \) satisfies \fullref{def:quadratic_plane_curve/ellipse/canonical_equation}. It follows from \fullref{thm:arctantwo} that \( t_0 \coloneqq \ATanTwo\left(\tfrac {y_0} b, \tfrac {x_0} a \right) \) is a solution to the \hyperref[def:quadratic_plane_curve/ellipse/parametric_equations]{parametric equations}. Conversely, if \( x_0 = a \cos(t_0) \) and \( y_0 = b \sin(t_0) \) for some \( t_0 \in [0, 2\pi) \), by \fullref{thm:trigonometric_identities/pythagorean_identity} it follows that the pair \( (x_0, y_0) \) is a root of \fullref{def:quadratic_plane_curve/ellipse/canonical_equation} and, by \fullref{thm:arctantwo}, \( t_0 \) can be restored given \( \cos(t_0) \) and \( \sin(t_0) \).

    Therefore every point of the parametric equation \fullref{def:quadratic_plane_curve/ellipse/parametric_equations} corresponds uniquely to a solution of the canonical equation \fullref{def:quadratic_plane_curve/ellipse/canonical_equation} and vice versa, which makes the two approaches to defining ellipses equivalent.

    \ILabel{def:quadratic_plane_curve/hyperbola} A \Def{hyperbola} is a quadratic curve whose canonical equation has the form
    \begin{equation}\label{def:quadratic_plane_curve/hyperbola/canonical_equation}
      c(x, y) \coloneqq \frac {x^2} {a^2} - \frac {y^2} {b^2} - 1 = 0,
    \end{equation}
    where \( a, b > 0 \).

    Similarly to ellipses, we are can define hyperbolas via \Def{scalar parametric equations} using \hyperref[def:hyperbolic_trigonometric_functions]{hyperbolic trigonometric functions} as follows:
    \begin{equation}\label{def:quadratic_plane_curve/hyperbola/parametric_equations}
      \begin{cases}
        x = a \cosh(t) \\
        y = b \sinh(t),
      \end{cases}
    \end{equation}
    where \( t \in \BR \). This only defines the \Def{right part} of the hyperbola. The left part is defined by replacing \( a \) with \( -a \).

    \ILabel{def:quadratic_plane_curve/parabola} A \Def{parabola} is a quadratic curve whose canonical equation has the form
    \begin{equation}\label{def:quadratic_plane_curve/parabola/canonical_equation}
      c(x, y) \coloneqq y^2 - 2px = 0,
    \end{equation}
    where \( p \neq 0 \).

    Unlike ellipses and hyperbolas, we do not define parametric equations. Instead, we define \( y \) as a function of \( x \) separately for the lower half-plane and upper half-plane:
    \begin{equation}\label{def:quadratic_plane_curve/parabola/cartesian_equation}
      y(x) = \pm \sqrt{2px}.
    \end{equation}
  \end{DefEnum}

  Ellipses, hyperbolas and parabolas are collectively called \Def{conic sections}.
\end{definition}

\begin{definition}\label{def:pi}
  \begin{figure}
    \centering
    \begin{mplibcode}
      input metapost/plotting;

      beginfig(1)
      drawarrow (-pi / 2, 0) scaled u -- (pi / 2, 0) scaled u;
      drawarrow (0, -1 / 2) scaled u -- (0, pi / 2) scaled u;

      vardef y(expr x) =
      sqrt(1 - x ** 2)
      enddef;

      drawarrow path_of_plot(y, -1, 1, 0.01, u);
      endfig;
    \end{mplibcode}
    \Caption{def:pi/upper_half_circle}{\( \Gph(y^+) \) as a parametric curve in \fullref{def:pi}.}
  \end{figure}

  The definition of a circle of unit radius as the zero-locus of the polynomial \( x^2 + y^2 - 1 \) allows us to solve a chicken-and-egg problem regarding the definitions of the number \( \pi \). It is conventional to define it as the ratio of a circle's circumference to its diameter. For a unit circle, this diameter is \( 2 \). It will be simpler for us, however, to define \( \pi \) as the radius of a half-circle's circumference since we can represent \( y \) as a function of \( x \) in the upper \hyperref[def:half_space]{half-plane} (see \fullref{def:pi/upper_half_circle}). Define the parametric curve
  \begin{BreakableAlign*}
     & y^+: [-1, 1] \to [0, 1]          \\
     & y^+(x) \coloneqq \sqrt{1 - x^2}.
  \end{BreakableAlign*}

  We use \fullref{thm:length_of_function_graph} to find the length of the graph \( \Gph(y^+(x)) \). The derivative of \( y^+(x) \) is
  \begin{equation*}
    D_x[y^+(x)] = \frac{-2x}{2 \sqrt{1 - x^2}} = - \frac x {\sqrt{1 - x^2}} dx.
  \end{equation*}

  The length of the curve \( \Gph(y^+) \) is thus
  \begin{equation*}
    \Len(\Gph(y^+)) = \int_{-1}^1 \sqrt{1 + \frac{x^2}{1 - x^2}} dx = \int_{-1}^1 \frac 1 {\sqrt{1 - x^2}} dx.
  \end{equation*}

  This justifies the definition
  \begin{equation}\label{def:pi/weierstrass_integral}
    \pi \coloneqq \int_{-1}^1 \frac 1 {\sqrt{1 - x^2}} dx.
  \end{equation}

  See \fullref{thm:trigonometric_function_basic_roots} for a proof of how this relates to the trigonometric functions and \fullref{thm:exponential_function_properties/eulers_identity} as a consequence.
\end{definition}

\begin{definition}\label{def:geometric_trigonometric_functions}
  After defining the \hyperref[def:trigonometric_functions]{trigonometric functions} \( \cos(z) \) and \( \sin(z) \) analytically via power series, we will define their geometric counterparts \( \cos_G(z) \) and \( \sin_G(z) \) and show the connection between them. The actual geometric definition relies on formalisms that are far beyond our interest (see the notes in \fullref{def:euclidean_plane}).

  Fix a point \( (x_0, y_0) \) on the unit circle (that is, \( x_0^2 + y_0^2 = 1 \)) and define the points
  \begin{equation}\label{def:geometric_trigonometric_functions/vertices}
    \begin{array}{l}
      A \coloneqq (x_0, y_0), \\
      B \coloneqq (0, 0),     \\
      C \coloneqq (x_0, 0).
    \end{array}
  \end{equation}

  Consider the \hyperref[def:triangle]{triangle} formed by these vertices. \Fullref{def:geometric_trigonometric_functions/triangle} illustrates the situation.
  \begin{figure}
    \begin{minipage}[b]{0.4\textwidth}
      \centering
      \begin{mplibcode}
        input metapost/plotting;

        u := 3.5cm;

        beginfig(1)
        pair A, B, C;
        path alpha, beta;

        t := 1;
        A := origin;
        B := (cos(t), sin(t)) scaled u;
        C := (cos(t), 0) scaled u;

        draw A -- B -- C -- cycle;

        alpha = fullcircle scaled (u / 4) shifted A cutbefore (A -- C) cutafter (A -- B);
        draw alpha;
        label.urt("$\alpha$", midpoint of alpha);
        label.lft("$\begin{rcases} \sin_G(\alpha) = \tfrac {\Len b} {\Len c} \\ \cos_G(\alpha) = \tfrac {\Len a} {\Len c} \end{rcases}$", A);

        beta = fullcircle scaled (u / 4) shifted B cutbefore (B -- A) cutafter (B -- C);
        draw beta;
        label.llft("$\beta$", point 0.7 of beta);
        label.lft("$\begin{rcases} \sin_G(\beta) = \tfrac {\Len a} {\Len c} \\ \cos_G(\beta) = \tfrac {\Len b} {\Len c} \end{rcases}$", B - (0.05, 0) * u);

        fill dot shifted A;
        fill dot shifted B;
        fill dot shifted C;

        label.bot("$A$", A);
        label.top("$B$", B);
        label.lrt("$C$", C);

        label.rt("$a$", midpoint of (B -- C));
        label.bot("$b$", midpoint of (A -- C));
        label.ulft("$c$", midpoint of (A -- B));
        endfig;
      \end{mplibcode}
    \end{minipage}
    \hspace{0.05\textwidth}
    \begin{minipage}[b]{0.4\textwidth}
      \centering
      \begin{mplibcode}
        input metapost/plotting;

        u := 3.5cm;

        beginfig(1)
        pair A, B, C;
        path alpha, beta;

        t := 1;
        A := origin;
        B := (cos(t), sin(t)) scaled u;
        C := (cos(t), 0) scaled u;

        drawarrow (-sin(pi/16), 0) scaled u -- (5/4, 0) scaled u;
        drawarrow (0, -sin(pi/16)) scaled u -- (0, 5/4) scaled u;

        draw path_of_curve(cos, sin, -pi/16, 9/16 * pi, 0.01, u);
        draw A -- B -- C -- cycle;

        alpha = fullcircle scaled (u / 4) shifted A cutbefore (A -- C) cutafter (A -- B);
        draw alpha;
        label.urt("$\alpha$", midpoint of alpha);

        beta = fullcircle scaled (u / 4) shifted B cutbefore (B -- A) cutafter (B -- C);
        draw beta;
        label.llft("$\beta$", point 0.7 of beta);

        fill dot shifted A;
        fill dot shifted B;
        fill dot shifted C;

        label.llft("$(0, 0)$", A);
        label.urt("$(x_0, y_0)$", B);
        label.lrt("$(x_0, 0)$", C);

        label.rt("$a$", midpoint of (B -- C));
        label.bot("$b$", midpoint of (A -- C));
        label.ulft("$c$", midpoint of (A -- B));
        endfig;
      \end{mplibcode}
    \end{minipage}
    \Caption{def:geometric_trigonometric_functions/triangle}{An \enquote{abstract} right triangle in the \hyperref[def:euclidean_plane]{Euclidean plane} with legends for geometric sines and cosines and the same triangle in \( \BR^2 \) connecting the origin to a point \( (x_0, y_0) \) on the unit circle.}
  \end{figure}

  The original \enquote{geometric definition} of \( \sin_G \) and \( \cos_G \) regards them as functions of an angle rather than numeric functions. \( \sin_G \) and \( \cos_G \) are only defined for two of the angles in a right triangle. The geometric definition is
  \begin{BreakableAlign*}
    \sin_G(\alpha) \coloneqq \frac{\Len(b)} {\Len(c)}, &  & \cos_G(\alpha) \coloneqq \frac{\Len(a)} {\Len(c)},
    \\
    \sin_G(\beta) \coloneqq \frac{\Len(b)} {\Len(c)},  &  & \cos_G(\beta) \coloneqq \frac{\Len(a)} {\Len(c)}.
  \end{BreakableAlign*}

  In our case, \( \Len(a) = y_0 \), \( \Len(b) = x_0 \) and \( \Len(c) = 1 \). Furthermore, \( \sin_G(\beta) \) nor \( \cos_G(
  \beta) \) are immaterial to our subsequent arguments and we only introduced them for the sake of having a full definition.

  Therefore we conclude that
  \begin{BreakableAlign*}
    \sin_G(\alpha) = x_0,
     &  &
    \cos_G(\alpha) = y_0.
  \end{BreakableAlign*}

  To see that \( \sin_G \) and \( \cos_G \) are somewhat analogous to \( \sin \) and \( \cos \), notice that by \fullref{thm:arctantwo}, there exists a unique \( t_0 \coloneqq \ATanTwo(y_0, x_0) \) such that
  \begin{BreakableAlign*}
    \sin(t_0) = x_0,
     &  &
    \cos(t_0) = y_0.
  \end{BreakableAlign*}

  Therefore our \hyperref[def:trigonometric_functions]{analytic definition} of the trigonometric functions as numeric functions correspond to the classical geometric definition in the special case where we consider the angle near the origin in the triangle formed by the vertices \fullref{def:geometric_trigonometric_functions/vertices}. This motivates \enquote{measuring} angles using the obtained correspondence. This unit of measurement is called a \Def{radian}. We say that the angle \( \alpha \) is \( t_0 \) \Def{radians}. Outside of mathematics, it is more conventional to use \Def{degrees}, which are obtained from radians by scaling with \( \tfrac {180} {\pi} \). That is, \( \alpha \) is \( \tfrac {180} {\pi} t_0 \) degrees.
\end{definition}

\subsection{Manifolds}\label{subsec:manifolds}

\begin{definition}\label{def:atlas}\MarginCite[def. 12.1]{Иванов2017}
  Let \( X \) be a \hyperref[def:topological_space]{topological space} and \( Y \) be a \hyperref[def:topological_vector_space]{topological vector space}.

  A \Def{coordinate chart} on \( X \) over \( Y \) is a pair \( (U_\alpha, \varphi_\alpha) \), where
  \begin{itemize}
    \item \( U_\alpha \subseteq X \) is a \hyperref[def:connected_space]{connected} open set.
    \item \( \varphi_\alpha: U_\alpha \to Y \) is a homeomorphic \hyperref[def:homeomorphism]{embedding}, called a \Def{coordinate homeomorphism}.
  \end{itemize}

  An \Def{atlas} on \( X \) over \( Y \) is an indexed \hyperref[def:indexed_family]{family} \( \{ (U_\alpha, \varphi_\alpha) \}_{\alpha \in \mscrK} \) of charts such that the family \( \{ U_\alpha \}_{\alpha \in \mscrK} \) is a \hyperref[def:set_partition]{cover} of \( X \). If \( Y = \BbbK^n \) for \( \BbbK \in \{ \BbbR, \BbbC \} \), we say that \( X \) is a real (resp. complex) manifold of dimension \( n \).

  For any two coordinate homeomorphisms \( \varphi_\alpha \) and \( \varphi_\beta \) in an atlas, the function restriction of the composition \( \varphi_\alpha \circ \varphi_\alpha^{-1} \) to \( U_\alpha \cap U_\beta \) is a homeomorphism from \( \varphi_\alpha(U_\alpha \cap U_\beta) \) to \( \varphi_\beta(U_\alpha \cap U_\beta) \), called a \Def{transition map}.
\end{definition}

\begin{definition}\label{def:topological_manifold}\MarginCite[def. 12.4]{Иванов2017}
  We call the topological space \( X \) a \Def{topological manifold} if it has a countable \hyperref[def:atlas]{atlas}. If \( Y = \BbbR^n \), we say that \( X \) is a manifold of dimension \( n \).
\end{definition}

\begin{definition}\label{def:differentiable_manifold}\MarginCite[def. 12.6]{Иванов2017}
  We call the \hyperref[def:topological_manifold]{topological manifold} \( X \) a \Def{smooth manifold} of type \( C^k \) if the transition maps are \( k \)-times continuously \hyperref[def:differentiability/frechet]{Frechet} differentiable.

  We also allow \( k = \infty \) for infinitely differentiable transition maps \( k = \omega \) for analytic transition maps.
\end{definition}

\subsection{Affine varieties}\label{subsec:affine_varieties}

As in \fullref{sec:commutative_algebra}, by \( R \) will denote a nontrivial commutative unital \hyperref[def:semiring/commutative_unital_ring]{ring}.

\begin{definition}\label{def:affine_variety}\MarginCite[69]{Коцев2016}
  For each ideal \( I \) of the polynomial \hyperref[def:multivariate_polynomial]{ring} \( \BK[X_1, \ldots, X_n] \) over a \hyperref[def:field]{field} \( \BK \), we define its \Def{affine variety} as the locus
  \begin{equation*}
    \CV(I) \coloneqq \{ (x_1, \ldots, x_n) \in \BK^n \colon \forall p \in I, p(x_1, \ldots, x_n) = 0 \}
  \end{equation*}
  of the simultaneous roots of all polynomials in \( I \).

  \begin{DefEnum}
    \ILabel{def:affine_variety/dimension} The \Def{dimension} \( \dim(\CV(I)) \) of an affine variety is defined as the Krull \hyperref[def:krull_dimension]{dimension} of the quotient \( \BK[X_1, \ldots, X_k] / I \).

    \ILabel{def:affine_variety/algebraic_curve} An \Def{algebraic curve} over \( \BK^n \) is an affine variety of dimension one.

    \ILabel{def:affine_variety/algebraic_surface} An \Def{algebraic surface} over \( \BK^n \) is an affine variety of dimension two.
  \end{DefEnum}
\end{definition}

\begin{proposition}\label{thm:dimension_of_variety_of_prime_ideal}
  If \( P \) is a prime ideal of \( \BK[X_1, \ldots, X_n] \), the dimension of \( \CV(P) \) is the number of prime ideals strictly containing \( P \).

  In particular, \( \dim \CV(P) \leq n - 1 \) and, if \( P \) is not maximal, \( \dim \CV(P) \geq 1 \).
\end{proposition}
\begin{proof}
  By \fullref{thm:prime_ideal_iff_prime_quotient_ideal}, \( \dim \CV(P) = \dim(\BK[X_1, \ldots, X_n] / P) \) is the number of prime ideals of \( \BK[X_1, \ldots, X_n] \) strictly containing \( P \).

  Since \( P \) itself is a prime ideal, the number of prime ideals strictly containing \( P \) is at least \( 1 \) less than the Krull dimension of \( \BK[X_1, \ldots, X_n] \). By \fullref{thm:krull_dimension_properties/polynomials_over_field}, \( \dim \CV(I) \leq n - 1 \).

  By \fullref{thm:field_maximal_ideal_representation}, \( \Gen{X_1, \ldots, X_n} \) is a maximal ideal of \( \BK[X_1, \ldots, X_n] \) (which is also prime), therefore \( \dim \CV(I) \geq 1 \).
\end{proof}

\begin{example}\label{ex:affine_varieties}
  We will work in the ring \( \BR[X, Y] \) of real polynomials in two indeterminates. By \fullref{thm:dimension_of_variety_of_prime_ideal}, any variety generated by a nonconstant irreducible polynomial is an \hyperref[def:affine_variety/algebraic_curve]{algebraic curve}.

  \begin{itemize}
    \item The variety of the ideal \( I \coloneqq \Gen{X^2 + Y^2 - 1} \) is the unit \hyperref[def:quadratic_plane_curve/ellipse]{circle}
          \begin{equation*}
            \CV(I) = \{ (x, y) \in \BR^2 \colon x^2 + y^2 = 1 \}.
          \end{equation*}

          Note that \( p(X, Y) \coloneqq X^2 + Y^2 - 1 \) is an \hyperref[def:irreducible_ring_element]{irreducible} polynomial. Indeed, if \( p(X, Y) = p_1(X, Y) p_1(X, Y) \), then by \fullref{thm:polynomial_degree_properties/product}, \( \deg(p_1) + \deg(p_2) = \deg(p) \). If\LEM \( \deg(p_1) = \deg(p_2) = 1 \), the variety of \( p \) would be a union of two lines, which cannot possibly be the unit circle \( \CV(I) \). Thus at least one of \( p_1 \) or \( p_2 \) is invertible and hence \( p \) is irreducible.

          Therefore \( \CV(I) \) is an algebraic curve.

    \item Another example is
          \begin{equation*}
            I \coloneqq \Gen{X + Y, X - Y - 1},
          \end{equation*}
          whose variety is
          \begin{equation*}
            \CV(I) = \{ (x, y) \in \BR^2 \colon x = -y \text{ and } x = y + 1 \} = \{ (x, y) \in \BR^2 \colon 2y = -1 \}.
          \end{equation*}

          This can also be shown algebraically, since the ideal \( I \) is principal and generated by
          \begin{equation*}
            \gcd(X + Y, X - Y - 1) = 2Y + 1.
          \end{equation*}

          Since \( 2Y + 1 \) is irreducible, \( \CV(I) \) is an algebraic curve.

    \item The ideal
          \begin{equation*}
            I \coloneqq \{ p(X, Y) \in R[X, Y] \colon p(X) = 0 \text{ or } \deg p \neq 0 \}
          \end{equation*}
          contains all polynomials except for the units - the nonzero constants. It is not principal because the only common divisors for all of \( I \) are the units.

          The variety \( \CV(I) \) for \( I \) is the empty set since it contains polynomials with no common roots - for example, \( X - Y \) and \( X - Y - 1 \).
  \end{itemize}
\end{example}

\begin{definition}\label{def:ideal_of_affine_variety}\MarginCite[70]{Коцев2016}
  Dually to \fullref{def:affine_variety}, for each subset \( V \subseteq R^n \), we define its \Def{ideal} as
  \begin{equation*}
    \Cal{I}(V) \coloneqq \{ p \in R[X_1, \ldots, X_n] \colon \forall (x_1, \ldots, x_n) \in V, p(x_1, \ldots, x_n) = 0 \}.
  \end{equation*}
\end{definition}

\begin{remark}\label{rem:nullstelletsatz_etymology}
  The word \enquote{nullstellensatz} is German for \enquote{\hyperref[def:zero_locus]{zero locus} theorem}.
\end{remark}

\begin{theorem}[Algebraic nullstellensatz]\label{thm:algebraic_nullstellensatz}\MarginCite[64]{Коцев2016}
  Let \( \BK \) be a field, \( A \) be a finitely-generated \( \BK \)-\hyperref[def:algebra_over_ring]{algebra} and \( M \) be a maximal ideal of \( A \). Then the field \( A / M \) is a finite extension of \( \BK \).

  In the special case that \( \BK \) is algebraically \hyperref[def:algebraically_closed_field]{closed}, we have
  \begin{equation*}
    \BK = A / M.
  \end{equation*}
\end{theorem}

\begin{example}\label{ex:algebraic_nullstellensatz_real_over_complex}
  For \( \BK = \BR \) and \( A = \BR[X] \), by \fullref{def:complex_numbers/polynomials} \( \BC = \BR[X] / \Gen{X^2 + 1} \) is a field so the ideal \( \Gen{X^2 + 1} \) is maximal.

  By \fullref{thm:algebraic_nullstellensatz}, \( \BC \) is a finite degree extension of \( \BR \).
\end{example}

\begin{corollary}\label{thm:closed_field_maximal_ideal_representation}\MarginCite[exer. 8.1]{Коцев2016}
  If \( \BK \) is algebraically closed, an ideal \( I \) of the polynomial ring \( \BK[X_1, \ldots, X_n] \) is maximal if and only if \( I = \Gen{X_1 - r_1, \ldots, X_n - r_n} \) for some \( r_1, \ldots, r_n \in \BK \).
\end{corollary}

\begin{theorem}[Geometric nullstellensatz]\label{thm:geometric_nullstellensatz}\MarginCite[70]{Коцев2016}
  If \( \BK \) is an algebraically \hyperref[def:algebraically_closed_field]{closed} field, then for each ideal \( I \subseteq \BK[X_1, \ldots, X_n] \) we have the equality
  \begin{equation*}
    \Cal{I}(\CV(I)) = \sqrt I,
  \end{equation*}
  where \( \sqrt I \) is the \hyperref[def:radical_ideal]{radical} of \( I \).
\end{theorem}

\begin{example}\label{ex:geometric_nullstellensatz_does_not_hold_for_reals}
  By \fullref{thm:reals_not_algebraically_closed}, the field \( \BR \) of \hyperref[def:real_numbers]{real numbers} is not algebraically closed since \( x^2 + 1 \) has no root. Denote \( I \coloneqq \Gen{x^2 + 1} \)

  Then \( \CV(I) = \varnothing \) so \( \Cal{I}(\CV)(I) = \BR[X] \).

  But \( \sqrt{I} = I \) since \( x^2 + 1 \) is irreducible and thus forms a a prime ideal by \fullref{thm:ufd_prime_iff_irreducible}.

  Thus \( \CV(I) \neq \sqrt{I} \) and \fullref{thm:geometric_nullstellensatz} does not hold.
\end{example}

\begin{corollary}\label{thm:weak_nullstellensatz}\MarginCite{Tao:nullstellensatz}
  If \( \BK \) is an algebraically \hyperref[def:algebraically_closed_field]{closed} field, then for each finite collection
  \begin{equation*}
    p_i(X_1, \ldots, X_n), i = 1, \ldots, k
  \end{equation*}
  of polynomials over \( n \) variables either
  \begin{CorEnum}
    \ILabel{thm:weak_nullstellensatz/roots} the system of equations
    \begin{equation}\label{thm:weak_nullstellensatz/system}
      \begin{cases}
        p_1(x_1, \ldots, x_n) = 0 \\
        p_2(x_1, \ldots, x_n) = 0 \\
        \vdots                    \\
        p_k(x_1, \ldots, x_n) = 0
      \end{cases}
    \end{equation}
    has a solution.

    \ILabel{thm:weak_nullstellensatz/bezout} there exist polynomials
    \begin{equation*}
      q_1(X_1, \ldots, X_n), i = 1, \ldots, k
    \end{equation*}
    such that
    \begin{equation*}
      p_1 q_1 + \cdots + p_k q_k = 1.
    \end{equation*}
  \end{CorEnum}
\end{corollary}
\begin{proof}
  Define the ideal
  \begin{equation*}
    I \coloneqq \Gen{p_1, \ldots, p_k}.
  \end{equation*}

  The following are equivalent:
  \begin{itemize}
    \item The variety \( \CV(I) \) is not empty.
    \item The system \fullref{thm:weak_nullstellensatz/system} has a solution.
    \item The ideal \( \Cal{I}(\CV(I)) \) is not the whole space \( \BK[X_1, \ldots, X_n] \).
    \item By \fullref{thm:geometric_nullstellensatz}, the radical \( \sqrt I \) is not the whole space.
    \item The units of \( \BK[X_1, \ldots, X_n] \) are not contained in \( \sqrt I \), hence also not contained in \( I \).
    \item There exists a set of polynomials satisfying \fullref{thm:weak_nullstellensatz/bezout}.
  \end{itemize}
\end{proof}

\begin{corollary}\label{thm:polynomial_over_closed_field_is_either_invertible_or_has_root}
  A multivariate polynomial over an algebraically closed field either has a root or is invertible.
\end{corollary}
\begin{proof}
  \Fullref{thm:weak_nullstellensatz} with \( k = 1 \).
\end{proof}


% Group theory
\subsection{Pointed sets}\label{subsec:pointed_sets}

\begin{definition}\label{def:pointed_set}
  The simplest algebraic structure is a \term{pointed set}. It is simply a nonempty set \( \mscrX \) equipped with an distinguished element \( e \in X \). It is an algebraic structure because \( e \) can be regarded as the, sole value of a nullary function \( \odot: X^0 \to X \).

  We will call \( e \) the \term{origin} of \( \mscrX \) based on the terminology for \hyperref[def:euclidean_plane_coordinate_system/origin]{affine coordinate systems}.

  \begin{thmenum}
    \thmitem{def:pointed_set/theory} Pointed sets can also be viewed as \hyperref[def:first_order_semantics/satisfiability]{models} of an \hyperref[def:first_order_theory]{empty theory} for a \hyperref[def:first_order_language]{first-order logic language} consisting of:
    \begin{thmenum}
      \thmitem{def:theory_of_pointed_sets/eq} A formal equality symbol \( \doteq \).
      \thmitem{def:theory_of_pointed_sets/point} A constant, i.e. a nullary \hyperref[def:first_order_language/func]{functional symbol}.
    \end{thmenum}

    This theory is called the theory of pointed sets.

    \thmitem{def:pointed_set/homomorphism} A \hyperref[def:first_order_homomorphism]{homomorphism} (based on \fullref{def:pointed_set/theory}) between the pointed sets \( (\mscrX, e_{\mscrX}) \) and \( (\mscrY, e_{\mscrY}) \) is, explicitly, a function \( \varphi: \mscrX \to \mscrY \) that satisfies
    \begin{equation}\label{eq:def:pointed_set/homomorphism}
      \varphi(e_{\mscrX}) = e_{\mscrY}.
    \end{equation}

    \thmitem{def:pointed_set/submodel} The set \( S \subseteq X \) is a \hyperref[def:first_order_substructure]{substructure} if \( \mscrX \) if \( e \in S \).

    \thmitem{def:pointed_set/category} We denote the \hyperref[def:category_of_small_first_order_models]{category of \( \mscrU \)-small models} for the theory of pointed sets by \( \cat{Set}_* \). We denote the \hyperref[def:category_of_small_first_order_models]{category of \( \mscrU \)-small models} for the theory of pointed sets by \( \cat{Set}_* \).
  \end{thmenum}
\end{definition}

\subsection{Magmas}\label{subsec:magmas}

\begin{definition}\label{def:magma}
  A \Def{magma} is a set \( \CM \) equipped with a \hyperref[def:function/arity]{binary function} \( \cdot: \CM \times \CM \to \CM \), called the \Def{magma operation}. Unless specified otherwise, we denote this operation by juxtaposition as \( xy \) instead of \( x \cdot y \). We often call the operation \Def{multiplication} or \Def{composition} (especially in \hyperref[def:endomorphism_monoid]{endomorphism monoids}), which contrasts which \hyperref[rem:additive_magma]{additive magmas}.

  \begin{DefEnum}
    \ILabel{def:magma/theory} In analogy to the \hyperref[def:pointed_set/theory]{theory of pointed sets}, we can define the \Def{theory of magmas} as an empty theory over a language with a single binary function.

    \ILabel{def:magma/homomorphism} A \hyperref[def:first_order_homomorphism]{homomorphism} between the magmas \( (\CM, \cdot_{\CM}) \) and \( (\CN, \cdot_{\CN}) \) is, explicitly, a function \( \varphi: \CM \to \CN \) such that
    \begin{equation}\label{eq:def:magma/homomorphism}
      \varphi(x \cdot_{\CM} y) = \varphi(x) \cdot_{\CN} \varphi(y) \T{for all} x, y \in \CM.
    \end{equation}

    \ILabel{def:magma/substructure} The set \( S \subseteq \CM \) is a \hyperref[def:first_order_substructure]{submagma} of \( \CX \) if it is closed under the magma operation, i.e. if \( x, y \in S \) imply \( xy \in S \).

    \ILabel{def:magma/category} We denote the \hyperref[def:first_order_model_category]{model category} for the theory of magmas by \( \Cat{Mag} \).

    \ILabel{def:magma/trivial} The \Def{trivial magma} is the empty set with an empty operation. It is the unique \hyperref[def:zero_objects/initial]{initial object} in \( \Cat{Mag} \).

    \ILabel{def:magma/opposite} The \Def{opposite magma} of \( (\CM, \cdot) \), also called the \Def{dual magma}, is the magma \( (\CM, \odot) \) with multiplication reversed:
    \begin{equation*}
      x \odot y \coloneqq y \cdot x.
    \end{equation*}

    We denote the opposite magma by \( \CM^{-1} \).

    \ILabel{def:magma/associative} We can add the following axiom to the theory:
    \begin{equation}\label{eq:def:magma/associative}
      (x \cdot y) \cdot z = x \cdot (y \cdot z).
    \end{equation}

    If \eqref{eq:def:magma/associative} is satisfied, we say that the operation \( \cdot \) and, by extension, the magma itself, are \Def{associative}. Associative magmas are usually called \Def{semigroups}. Associativity imposes no additional restrictions on the homomorphisms, hence semigroups are a \hyperref[def:subcategory]{full subcategory} of \( \Cat{Mag} \).

    \ILabel{def:magma/commutative} Another common axiom is \Def{commutativity}:
    \begin{equation}\label{eq:def:magma/commutative}
      x \cdot y = y \cdot x.
    \end{equation}

    Commutative magmas also form a full subcategory. Obviously \( \CM = \CM^{-1} \) in a commutative magma.

    \ILabel{def:magma/cancellative} We say that the operation \( \cdot \) is \Def{left-cancellative} (resp. \Def{right-cancellative}) if, for all \( x, y \in \CM \), we have
    \begin{align}\label{eq:def:magma/cancellative}
      x = y \T{whenever} z \cdot x = z \cdot y \T{for all} z \in \CM
      &&
      (\text{resp. } x \cdot z = y \cdot z \T{for all} z \in \CM).
    \end{align}

    The operation is \Def{cancellative} if it is both left and right cancellative. Cancellative magmas also form a full subcategory.

    \ILabel{def:magma/exponentiation} We define an additional \Def{exponentiation} operation for positive integers \( n \) inductively\IND as
    \begin{equation}\label{eq:def:magma/exponentiation}
      x^n \coloneqq \begin{cases}
        x,               & n = 1 \\
        x \cdot x^{n-1}, & n > 1
      \end{cases}
    \end{equation}

    \ILabel{def:magma/power_set} It is customary to perform magma operations with sets. That is, if \( A \) and \( B \) are sets in the magma \( \CM \), it is customary to write
    \begin{equation*}
      A \cdot B \coloneqq \{ a \cdot b \colon a \in A, b \in B \}.
    \end{equation*}

    This actually turns the power set \( \Pow(\CM) \) into a magma, which we will call the \Def{power set magma} of \( \CM \). This is especially useful with the convention \fullref{rem:singleton_sets} since it allows us to write \( aB \) for \( a \in \CM \) and \( B \subseteq \CM \).

    See \fullref{thm:power_set_magma_preservation}.
  \end{DefEnum}
\end{definition}

\begin{remark}\label{rem:additive_magma}
  General groups often arise as \hyperref[def:automorphism_group]{automorphism groups}, which are, for the most part, non-commutative, while abelian groups are usually used as the main building block for \hyperref[def:semiring/ring]{rings} and \hyperref[def:left_module]{modules}.

  To make a further distinction, if the operation is denoted by \( \cdot \) or juxtaposition, we say that the group is a \Def{multiplicative group}, and if the operation is denoted by \( + \), we say that the group is an \Def{additive group}. This terminology usually, but not necessarily, coincides with the group (or, more generally, any \hyperref[def:magma]{magma}) being \hyperref[def:magma/commutative]{commutative}.

  To make things explicit, a \Def{multiplicative magma} is any magma as defined in \fullref{def:magma}. Compare this to \Def{additive magmas}, where
  \begin{RemEnum}
    \ILabel{rem:additive_magma/addition} The magma operation is denoted by \( + \) and called \Def{addition}.
    \ILabel{rem:additive_magma/multiplication} The magma \hyperref[def:magma/exponentiation]{exponentiation operation} is denoted by \( \cdot \) or juxtaposition called \Def{multiplication}. Thus multiplication is not defined for two elements of the magma but defined for a positive integer and an element of the magma. In the case of a \hyperref[def:magma/commutative]{commutative} \hyperref[def:unital_magma/associative]{monoid}, if multiplication is extended to two elements of the monoid, we instead talk about \hyperref[def:semiring]{semirings}.

    \ILabel{rem:additive_magma/identity} The \hyperref[def:magma_identity]{identity} is usually denoted by \( 0 \).
    \ILabel{rem:additive_magma/inverse} If an \hyperref[def:unital_magma_inverse_element]{inverse} of \( x \) exists, it is denoted by \( -x \) rather than \( x^{-1} \).
  \end{RemEnum}
\end{remark}

\begin{proposition}\label{thm:power_set_magma_preservation}
  \hyperref[def:magma/associative]{Associativity} and \hyperref[def:magma/commutative]{commutativity} from a magma \( \CM \) are preserved in \( \Pow(\CM) \), unlike \hyperref[def:magma/cancellative]{cancellation}.
\end{proposition}
\begin{proof}
  Associativity and commutativity are obviously preserved.

  To show that cancellation is not, consider the group \hyperref[def:group_of_integers_modulo]{\( \BZ_2 \)}. It is a cancellative magma by \fullref{thm:group_properties/cancellative}. Define the sets \( A \coloneqq \{ 0, 1 \} \) and \( B \coloneqq \{ 0 \} \). Then
  \begin{equation*}
    A + A = A = A + B,
  \end{equation*}
  however we cannot\LEM cancel \( A \) from the left because \( A \neq B \).
\end{proof}

\begin{proposition}\label{thm:magma_exponentiation_properties}
  Fix a magma \( \CM \). \hyperref[def:magma/exponentiation]{Magma exponentiation} in \( \CM \) has the following basic properties:

  \begin{PropEnum}
    \ILabel{thm:magma_exponentiation_properties/commutativity} We have the following \hyperref[def:magma/commutative]{commutativity}-like property: for \( x \in \CM \) and \( n = 1, 2, \ldots \),
    \begin{equation}\label{eq:thm:magma_exponentiation_properties/commutativity}
      x^n = x x^{n-1} = x^{n-1} x.
    \end{equation}

    \ILabel{thm:magma_exponentiation_properties/distributivity} Exponentiation distributes over multiplication: for any member \( x \in \CM \) and any two positive integers \( n \) and \( m \),
    \begin{equation}\label{eq:thm:magma_exponentiation_properties/multiplication}
      x^{n + m} = x^n x^m.
    \end{equation}

    \ILabel{thm:magma_exponentiation_properties/repeated} For any member \( x \in \CM \) and any two positive integers \( n \) and \( m \),
    \begin{equation}\label{eq:thm:magma_exponentiation_properties/repeated}
      (x^n)^m = x^{nm}.
    \end{equation}
  \end{PropEnum}
\end{proposition}
\begin{proof}
  \SubProofOf{thm:magma_exponentiation_properties/commutativity} We use induction\IND on \( n \). The cases \( n = 1 \) and \( n = 2 \) are obvious. For \( n > 2 \), we have
  \begin{equation*}
    x^n
    \overset {\eqref{eq:def:magma/exponentiation}} =
    x x^{n-1}
    \overset {\IndHyp} =
    x x^{n-2} x
    \overset {\eqref{eq:def:magma/exponentiation}} =
    x^{n-1} x.
  \end{equation*}

  \SubProofOf{thm:magma_exponentiation_properties/distributivity} We use induction\LEM on \( n \). The case \( n = 1 \) follows directly from \eqref{eq:def:magma/exponentiation}. The case \( n > 1 \) follows from
  \begin{equation*}
    x^{n + m}
    \overset {\eqref{eq:def:magma/exponentiation}} =
    x x^{n + (m - 1)}
    \overset {\IndHyp} =
    x x^{n-1} x^m
    \overset {\eqref{eq:def:magma/exponentiation}} =
    x^n x^m.
  \end{equation*}

  \SubProofOf{thm:magma_exponentiation_properties/repeated} We use induction\LEM on \( n \). The case \( n = 1 \) is obvious and the rest follows from
  \begin{equation*}
    (x^n)^m
    \overset {\eqref{eq:def:magma/exponentiation}} =
    x^n (x^n)^{m-1}
    \overset {\IndHyp} =
    x^n x^{n (m - 1)}
    \overset {\eqref{eq:thm:magma_exponentiation_properties/multiplication}} =
    =
    x^{nm}.
  \end{equation*}
\end{proof}

\begin{definition}\label{def:preordered_magma}
  A \Def{preordered magma} is a magma \( \CM \) equipped with a \hyperref[def:preordered_set]{preorder} \( \leq \) such that
  \begin{equation}\label{eq:def:preordered_magma/compatibility}
    x \leq y \T{implies} xz \leq yz \T{and} zx \leq zy \T{for all} z \in \CM.
  \end{equation}

  The category of preordered magmas is \hyperref[def:concrete_category]{concrete} with respect to both \( \Cat{Mag} \) and the \hyperref[def:preordered_magma]{category of preordered sets}.
\end{definition}

\begin{proposition}\label{thm:preordered_magma_max_distributivity}
  In an \hyperref[def:preordered_magma]{preordered magma} \( \CM \),
  \begin{equation}\label{eq:thm:preordered_magma_max_distributivity}
    \max \Set{a b, c d} \leq \max \Set{a, c} \cdot \max \Set{b, d}.
  \end{equation}
\end{proposition}
\begin{proof}
  Since \( a \leq \max \Set{a, c} \), then
  \begin{equation*}
    ab
    \overset \leq {\eqref{eq:def:preordered_magma/compatibility}}
    \max \Set{a, c} \cdot b
    \leq
    \max \Set{a, c} \cdot \Set{b, d}
  \end{equation*}

  Analogously, \( cd \leq \max \Set{a, c} \cdot \Set{b, d} \).

  Therefore
  \begin{equation}
    \max \Set{a b, c d} \leq \max \Set{a, c} \cdot \Set{b, d}.
  \end{equation}
\end{proof}

\begin{definition}\label{def:topological_magma}
  A \Def{topological magma} is a magma equipped with a \hyperref[def:topological_space]{topology} such that the magma operation is continuous.

  The category of topological magmas is \hyperref[def:concrete_category]{concrete} with respect to both \( \Cat{Mag} \) and \hyperref[def:category_of_topological_spaces]{\( \Cat{Top} \)}.
\end{definition}

\subsection{Unital magmas}\label{subsec:unital_magmas}

\begin{definition}\label{def:magma_identity}
  An element \( e \) of a magma \( \CM \) is called a \Def{left identity} (resp. \Def{right identity}) if
  \begin{align}\label{eq:def:magma_identity}
    ex = x \T{for all} x \in \CM
    &&
    (\text{resp. } xe = x \T{for all} x \in \CM)
  \end{align}

  If \( e \) is simultaneously a left and right identity, we call a \Def{two-sided identity} or simply \Def{identity} of \( \CM \).
\end{definition}

\begin{proposition}\label{thm:magma_identity_unique}
  The two-sided \hyperref[def:magma_identity]{magma identity} \( e \) in a magma is unique.
\end{proposition}
\begin{proof}
  If \( f \) is another identity, then \( e = ef = f \).
\end{proof}

\begin{definition}\label{def:unital_magma}
  A \hyperref[def:magma]{magma} with an \hyperref[def:magma_identity]{identity} is called \Def{unital}. This makes it a \hyperref[def:pointed_set]{pointed set}. We can consider it as a pair \( (\CM, \cdot) \) rather than a triple \( (\CM, \cdot, e) \) because, by \fullref{thm:magma_identity_unique}, a two-sided identity is uniquely determined by the magma operation.

  \begin{DefEnum}
    \ILabel{def:unital_magma/theory} The theory of unital magmas consists of the axioms \eqref{eq:def:magma_identity} over the intersection of the languages of \hyperref[def:pointed_set/theory]{pointed sets} and \hyperref[def:magma/theory]{magmas}

    \ILabel{def:unital_magma/homomorphism} A \hyperref[def:first_order_homomorphism]{homomorphism} between unital magmas is a function that satisfies both \eqref{eq:def:pointed_set/homomorphism} and \eqref{eq:def:magma/homomorphism}.

    \ILabel{def:unital_magma/category} The \hyperref[def:first_order_model_category]{model category} \( \Cat{Mag}_* \) of unital magmas is \hyperref[def:concrete_category]{concrete} with respect to both \hyperref[def:pointed_set/category]{\( \Cat{Set}_* \)} and \hyperref[def:magma/category]{\( \Cat{Mag} \)}.

    \ILabel{def:unital_magma/substructure} The set \( S \subseteq \CM \) is a \hyperref[def:first_order_substructure]{unital submagma} of \( \CX \) if it is a \hyperref[def:magma/substructure]{submagma} and if \( e \in S \).

    \ILabel{def:unital_magma/trivial} A \Def{trivial unital magma} is a set \( \{ e \} \) with the \hyperref[def:function/diagonal]{identity} operation. Since it is unique up to an isomorphism, we usually speak of \enquote{the} trivial unital magma. It is the \hyperref[def:zero_objects/initial]{initial object} in \( \Cat{Mag}_* \).

    \ILabel{def:unital_magma/associative} An \hyperref[eq:def:magma/associative]{associative} unital magma is called a \Def{monoid}. The category \( \Cat{Mon} \) of monoids is a full subcategory of \( \Cat{Mag}_* \).

    \ILabel{def:unital_magma/exponentiation} We extend \hyperref[def:magma/exponentiation]{magma exponentiation} to all nonnegative integers by setting
    \begin{equation*}
      x^0 \coloneqq e.
    \end{equation*}

    \ILabel{def:unital_magma/power_set} The \hyperref[def:magma/power_set]{power set magma} \( \Pow(\CM) \) of a unital magma \( \CM \) with identity \( e \) is again a unital magma with identity \( \{ e \} \).
  \end{DefEnum}
\end{definition}

\begin{definition}\label{def:unital_magma_kernel}
  The \Def{kernel} \( \ker(\varphi) \) of a unital magma homomorphism \( \varphi: \CM \to \CN \) is the zero \hyperref[def:zero_locus]{locus} of \( \varphi \), that is, \hyperref[def:function/preimage]{preimage} \( \varphi^{-1}(e_{\CN}) \).

  It is an instance of \hyperref[def:categorical_kernel]{categorical kernels} in \hyperref[def:category_of_sets]{\( \Cat{Set} \)}. Formally, it is the \hyperref[thm:set_categorical_limits/equalizer]{equalizer} of \( \varphi \) and the constant homomorphism \( \psi(x) \coloneqq e_{\CN} \).
\end{definition}

\begin{proposition}\label{thm:unital_magma_kernel_is_submagma}
  The \hyperref[def:unital_magma_kernel]{kernel} of a unital magma homomorphism \( \varphi: \CM \to \CN \) is a \hyperref[def:first_order_substructure]{unital submagma} of \( \CM \).
\end{proposition}
\begin{proof}
  By \eqref{eq:def:pointed_set/homomorphism}, \( e_{\CM} \in \ker(\varphi) \), therefore \( \ker(\varphi) \) inherits its unital magma structure from \( \CM \). It remains only to show that it is closed under the magma operation. But this is trivial since, if \( x, y \in \ker(\varphi) \), then
  \begin{equation*}
    \varphi(xy) = \varphi(x) \varphi(y) = e_{\CN} e_{\CN} = e_{\CN}.
  \end{equation*}
\end{proof}

\subsection{Magma ideals}\label{sec:magma_ideals}

\begin{definition}\label{def:magma_ideal}
  Let \( \mscrM \) be a \hyperref[def:magma]{magma} and \( I \) be a subset of \( \mscrM \). We say that \( I \) is a \term{left ideal} of \( \mscrM \) if the inclusion \( I\mscrM \subseteq I \) holds, where we use the convention in \fullref{def:magma/power_set}, that is,
  \begin{equation*}
    I \mscrM = \set{ x \cdot y | x \in M, y \in I }.
  \end{equation*}

  Right ideals are defined analogously. If \( I \) is both a left ideal and a right ideal, we say that it is a \term{two-sided ideal}.
\end{definition}

\begin{proposition}\label{thm:magma_ideal_is_submagma}
  Every two-sided magma ideal is a submagma.
\end{proposition}
\begin{proof}
  Let \( I \) be a two-sided ideal for the magma \( \mscrM \). For \( x, y \in I \), since \( I \) is a left ideal, we have \( xy \in I \) and similarly \( yx \in I \) since \( I \) is a right ideal. Thus, \( II = I \) and \( I \) is a submagma of \( \mscrM \).
\end{proof}

\begin{example}\label{ex:subgroup_is_not_ideal}
  We explicitly give a counterexample to the converse of \fullref{thm:magma_ideal_is_submagma}. Define \( \mscrG \coloneqq \BbbZ \times \BbbZ \) to be the \hyperref[def:group_direct_sum]{direct sum} of two copies of the \hyperref[def:set_of_integers]{integers}. Define
  \begin{equation*}
    \mscrH \coloneqq \{ (n, n) \colon n \in \BbbZ \}.
  \end{equation*}

  The set \( \mscrH \) is a subgroup of \( \mscrG \) since it is closed under addition and it contains the identity element \( (0, 0) \). It is not an ideal, however, since
  \begin{equation*}
    (n, n) + (n, 0) = (2n, n) \not\in H.
  \end{equation*}
\end{example}

\begin{proposition}\label{thm:proper_ideals_containing_identity}
  A left or right ideal of a \hyperref[def:unital_magma]{unital magma} contains the identity if and only if it is not proper.
\end{proposition}
\begin{proof}
  Let \( \mscrM \) be a unital magma and \( I \) be a left ideal of \( \mscrM \). We will prove that \( e \in I \iff I = \mscrM \).

  \SufficiencySubProof Let \( e \in I \). Then \( ex = x \) for any \( x \in M \), thus \( I\mscrM = \mscrM \). But \( I \) is an ideal, hence we have that \( I\mscrM = I \), thus \( I = I\mscrM = \mscrM \).

  \NecessitySubProof If \( I = \mscrM \), then obviously \( e \in I \).

  An analogous proof follows for the case when \( I \) is a right ideal.
\end{proof}

\begin{corollary}\label{thm:unital_magma_ideal_is_submagma_iff_contains_identity}
  A two-sided ideal of a unital magma is a unital submagma if and only if it contains the identity.
\end{corollary}
\begin{proof}
  Follows from \fullref{thm:magma_ideal_is_submagma} and \fullref{thm:proper_ideals_containing_identity}.
\end{proof}

\begin{proposition}\label{thm:commutative_magma_ideals}
  In a \hyperref[def:magma/commutative]{commutative magma} \( \mscrM \), a subset \( I \subseteq M \) is a left ideal if and only if it is a right ideal. That is, in commutative magmas, it makes no sense to distinguish between left, right and two-sided ideals.
\end{proposition}
\begin{proof}
  For \( x \in M \) and \( y \in I \), by commutativity we have \( yx = xy \), thus \( \mscrM I = I \mscrM \) and \( I \) is a left ideal if and only if it is a right ideal.
\end{proof}

\begin{proposition}\label{thm:product_of_semigroup_ideals_is_in_intersection}
  Fix a \hyperref[def:magma/associative]{semigroup} \( \mscrM \). If \( I \) and \( J \) are two-sided ideals, so are \( IJ \) and \( I \cap J \) and, furthermore,
  \begin{equation*}
    IJ \subseteq I \cap J.
  \end{equation*}
\end{proposition}
\begin{proof}
  We first show that \( IJ \) is an ideal.

  Take \( x \in I \), \( y \in J \). If \( z \in M \), then associativity gives us
  \begin{equation*}
    z(xy) = (zx)y \in (zx)J \subseteq IJ
  \end{equation*}
  and
  \begin{equation*}
    (xy)z = x(yz) \in I(yz) \subseteq IJ.
  \end{equation*}

  Hence, \( IJ \) is closed under the semigroup operation. This makes \( IJ \) a two-sided ideal.

  If \( x \in I \cap J \) and \( z \in M \), obviously \( xz \in I \) and \( xz \in J \), hence \( xz \in I \cap J \). Then \( I \cap J \) is also a two-sided ideal.

  To obtain the inclusion
  \begin{equation*}
    IJ \subseteq I \cap J,
  \end{equation*}
  observe that \( xy \in IJ \) means that \( xy \in xJ = J \) and \( xy \in Iy = I \), thus \( xy \in I \cap J \).
\end{proof}

\subsection{Monoid actions}\label{subsec:monoid_actions}

\begin{definition}\label{def:endomorphism_monoid}
  Given a \hyperref[def:category_cardinality]{locally small} \hyperref[def:category]{category} \( \Cat{C} \), we call \( \Cat{C}(A) \) the \Def{endomorphism monoid} over \( A \) and denote it by \( \End(A) \). If \( A \) is the only object in \( \Cat{C} \), we can identify the entire category \( \Cat{C} \) with the monoid \( \End(A) \).
\end{definition}

\begin{remark}\label{rem:monoid_action_endomorphisms}
  It is tempting to define a monoid action over an object \( A \) of an arbitrary locally small category rather than over a set rather than specifying the properties of an action (e.g. \enquote{linear action}). Unfortunately, even the simplest examples of monoid actions may fail to hold nice properties. See, e.g. \fullref{thm:monoid_is_action} or \fullref{thm:cayleys_theorem}.
\end{remark}

\begin{definition}\label{def:left_monoid_action}\MarginCite[159]{Knapp2016BAlg}
  Let \( \mscrM \) be a \hyperref[def:unital_magma/associative]{monoid} and let \( A \) be a set. A \Def{left monoid action} of \( \mscrM \) on \( A \) can be defined equivalently as:
  \begin{DefEnum}
    \ILabel{def:left_monoid_action/homomorphism} A homomorphism from \( \mscrM \) to the \hyperref[def:endofunction]{endofunction} monoid \( \End(A) \).
    \ILabel{def:left_monoid_action/indirect_homomorphism} An indexed family of endofunctions \( \{ \tau_x \}_{x \in \mscrM} \) such that
    \begin{equation}\label{eq:def:left_monoid_action/indirect_homomorphism}
      \tau_{xy} = \tau_x \circ \tau_y \T{for all} x, y \in \mscrM.
    \end{equation}

    \ILabel{def:left_monoid_action/operation} A heterogeneous \hyperref[def:magma]{algebraic operation} \( \circ: \mscrM \times A \to A \), written using juxtaposition, such that
    \begin{DefEnum}
      \ILabel{def:left_monoid_action/operation/identity} \( e \odot a = a \) holds for any \( a \in A \).
      \ILabel{def:left_monoid_action/operation/compatibility} \( (xy) \odot a = x \odot (y \odot a) \) holds whenever \( x, y \in \mscrM \) and \( a \in A \).
    \end{DefEnum}

    See \fullref{rem:theory_of_left_monoid_actions} for the \hyperref[def:first_order_theory]{first order logic theory} behind this operation.
  \end{DefEnum}

  We say that \( \mscrM \) acts on \( A \) and optionally add adjectives, e.g. \enquote{\( \mscrM \) acts linearly/smoothly on \( A \)}.
\end{definition}
\begin{proof}
  \SubProofImplication{def:left_monoid_action/homomorphism}{def:left_monoid_action/indirect_homomorphism} Let \( \tau: \mscrM \to \End(A) \) be a homomorphism. Thus we assign a morphism \( \tau(x) \) for each member \( x \in \mscrM \). Furthermore, since monoid operation on \( \End(A) \) is function composition and since \( \tau \) is a homomorphism, we have
  \begin{equation*}
    \tau(xy) = \tau(x) \circ \tau(y).
  \end{equation*}

  \SubProofImplication{def:left_monoid_action/indirect_homomorphism}{def:left_monoid_action/operation} Assume that we have a morphism \( \tau_x: A \to A \) for each \( x \in \mscrM \) that satisfies \eqref{eq:def:left_monoid_action/indirect_homomorphism}. Define the operation
  \begin{align*}
    {}&\odot{}: \mscrM \times A \to A \\
    x &\odot a \coloneqq \tau_x(a).
  \end{align*}

  It satisfies the necessary axioms:
  \begin{RefList}
    \IRef{def:left_monoid_action/operation/identity} If \( a \in A \), we have
    \begin{equation*}
      e \odot a
      =
      \tau_e(a)
      =
      \Id(a)
      =
      a.
    \end{equation*}

    \IRef{def:left_monoid_action/operation/compatibility} If \( x, y \in \mscrM \) and \( a \in A \), we have
    \begin{equation*}
      (x y) \odot a
      =
      \tau_{x y}(a)
      =
      \tau_{x}(\tau{y}(a))
      =
      g \odot (h \odot a).
    \end{equation*}
  \end{RefList}

  \SubProofImplication{def:left_monoid_action/operation}{def:left_monoid_action/homomorphism} Assume that we have an operation \( \odot: \mscrM \times A \to A \) that satisfies the axioms for left action. Define the function
  \begin{align*}
    &\tau: \mscrM \to \End(A) \\
    &\tau(x) \coloneqq x \Id.
  \end{align*}

  Then \( \tau \) is a monoid homomorphism because \( \tau(\varepsilon) = \Id \) and
  \begin{equation}\label{def:left_monoid_action/homomorphism/proof}
    \varphi(xy)
    =
    xy \Id
    =
    x (y \Id)
    =
    x \Id (y \Id)
    =
    (x \Id) (y \Id)
    =
    \varphi(x) \varphi(y),
  \end{equation}
\end{proof}

\begin{remark}\label{rem:theory_of_left_monoid_actions}
  In order to fit the heterogeneous operation of \hyperref[def:left_monoid_action]{left monoid actions} into the framework of \hyperref[def:first_order_semantics/satisfiability]{first order logic models}, we need the category \( \Cat{C} \) to be a \hyperref[def:first_order_model_category]{model category}. A monoid action is then obtained, by extending the theory of \( \Cat{C} \).

  \begin{RemEnum}
    \ILabel{rem:theory_of_left_monoid_actions/functions} For each \( x \in \mscrM \), add a unary \hyperref[def:first_order_language/func]{functional symbol} \( \varphi_x \).

    \ILabel{rem:theory_of_left_monoid_actions/axiom} For each \( x, y \in \mscrM \), add the axiom
    \begin{equation}\label{eq:rem:theory_of_left_monoid_actions/axiom_schema}
      \forall a (\tau_{xy}(a) = \tau_x(\tau_y(a))).
    \end{equation}
  \end{RemEnum}
\end{remark}

\begin{definition}\label{def:right_monoid_action}
  We say that \( \tau: \mscrM \to \End(A) \) is a \Def{right monoid action} of \( \mscrM \) on \( A \) if the same function is a \hyperref[def:left_monoid_action]{left monoid action} of the \hyperref[def:magma/opposite]{opposite monoid} \( \mscrM^{-1} \) on \( A \).
\end{definition}

\begin{definition}\label{def:faithful_left_monoid_action}
  A left monoid action is said to be \Def{faithful} if the corresponding homomorphism is injective.
\end{definition}

\begin{proposition}\label{thm:monoid_is_action}
  Any \hyperref[def:unital_magma/associative]{monoid} \hyperref[def:left_monoid_action]{acts} on itself by \hyperref[def:endofunction]{endofunctions}.

  Compare this result to \fullref{thm:cayleys_theorem}.
\end{proposition}
\begin{proof}
  For completeness, we will verify all three definitions:

  \SubProofOf{def:left_monoid_action/homomorphism} The identity function \( \Id: \mscrM \to \mscrM \) is the identity element of \( \Fun(\mscrM) \). Define
  \begin{align*}
    &\tau: \mscrM \to \Fun(\mscrM) \\
    &\tau(x) \coloneqq x \Id
  \end{align*}

  It is a monoid homomorphism because both \eqref{eq:def:pointed_set/homomorphism} and \eqref{def:left_monoid_action/homomorphism/proof} hold.

  \SubProofOf{def:left_monoid_action/indirect_homomorphism} The proof is the same as above.

  \SubProofOf{def:left_monoid_action/operation} Define the operation
  \begin{BreakableAlign*}
    {}&\odot{}: \mscrM \times \mscrM \to \mscrM \\
    x &\odot y \coloneqq xy
  \end{BreakableAlign*}

  It immediately follows that
  \begin{RefList}
    \IRef{def:left_monoid_action/operation/identity} \( e \circ x = ex = x \) for all \( x \in \mscrM \).
    \IRef{def:left_monoid_action/operation/compatibility} \( (x y) \circ z = xyz = x \circ (y \circ z) \) for all \( x, y, z \in \mscrM \).
  \end{RefList}
\end{proof}

\begin{proposition}\label{thm:natural_numbers_monoid_action}
  The \hyperref[def:natural_numbers]{natural numbers} \( \BbbN \) act on any \hyperref[def:unital_magma/associative]{monoid} by \hyperref[def:unital_magma/exponentiation]{exponentiation}.

  Compare this result to \fullref{thm:integers_group_action}.
\end{proposition}
\begin{proof}
  The action itself is given by \( n \mapsto (x \mapsto x^n) \). We must only verify that it is a homomorphism.

  We verify the explicit axioms from \fullref{def:left_monoid_action/operation}:
  \begin{RefList}
    \IRef{def:left_monoid_action/operation/identity} \( x^1 = x \) for all \( x \in \mscrM \) by definition.
    \IRef{def:left_monoid_action/operation/compatibility} \( (x^n)^m = x^{nm} \) is the literal statement of \eqref{eq:thm:magma_exponentiation_properties/repeated}.
  \end{RefList}
\end{proof}

\subsection{Groups}\label{subsec:groups}

\begin{definition}\label{def:unital_magma_inverse_element}
  Let \( \mscrM \) be a \hyperref[def:unital_magma]{unital magma}. We say that \( y \) is the \term{left inverse} (resp. \term{right inverse}) of \( x \) if
  \begin{align}\label{eq:def:unital_magma_inverse_element}
    yx = e
    &&
    (\text{resp. } xy = e).
  \end{align}

  If \( y \) is simultaneously a left and right inverse of \( x \), we call a \term{two-sided inverse}, a \term{neutral element} or simply \term{inverse} of \( x \) and denote it by \( x^{-1} \) since it is unique by \fullref{thm:magma_identity_unique}. This notation is consistent with \fullref{def:unital_magma/exponentiation}
\end{definition}

\begin{proposition}\label{def:unital_magma_inverse_element_unique}
  The two-sided \hyperref[def:unital_magma_inverse_element]{inverse} \( x^{-1} \) of \( x \) is unique.
\end{proposition}
\begin{proof}
  If \( y \) and \( z \) are both inverses of \( x \), then \( y = ey = zxy = ze = z \).
\end{proof}

\begin{definition}\label{def:group}
  A \term{group} is a \hyperref[def:unital_magma/monoid]{monoid} in which every element has an \hyperref[def:unital_magma_inverse_element]{inverse}. Groups are the most well-studied and most well-behaved magmas. Many useful properties like \hyperref[thm:def:group/properties/cancellative]{cancellation} rely on associativity, so we do not consider non-associative groups.

  \begin{thmenum}
    \thmitem{def:group/theory} We can construct the \hyperref[def:first_order_theory]{theory of groups} by adding a unary \hyperref[def:first_order_language/func]{functional symbol} \( (\anon)^{-1} \) and the axiom
    \begin{equation}\label{eq:def:group/theory/inverse_axiom}
      \forall x (xx^{-1} = e \wedge x^{-1}x = e)
    \end{equation}
    to the language of \hyperref[def:unital_magma/monoid]{monoids}.

    \thmitem{def:group/function_parity} A \hyperref[def:function]{function} \( f: \mscrG \to \mscrH \) between two groups is called \term{even} if
    \begin{equation}\label{eq:def:group/function_parity/even}
      f(x^{-1}) = f(x) \quad\forall x \in G
    \end{equation}
    and \term{odd} if
    \begin{equation}\label{eq:def:group/function_parity/odd}
      f(x^{-1}) = f(x)^{-1} \quad\forall x \in G.
    \end{equation}

    \thmitem{def:group/homomorphism} A \hyperref[def:first_order_homomorphism]{homomorphism} between the groups \( \mscrG \) and \( \mscrH \) is an odd \hyperref[def:unital_magma/homomorphism]{unital magma homomorphism}.

    As shown in \fullref{thm:group_homomorphism_single_condition}, however, the conditions \eqref{eq:def:pointed_set/homomorphism} and \eqref{eq:def:group/function_parity/odd} are redundant.

    \thmitem{def:group/submodel} The set \( S \subseteq G \) is a \hyperref[def:first_order_substructure]{subgroup} of \( \mscrG \) if it is a \hyperref[def:unital_magma/submodel]{unital submagma} and if \( S^{-1} = S \), where \( S^{-1} = \{ s^{-1} \colon s \in S \} \).

    \thmitem{def:group/trivial} The trivial group consists only of the identity \( e \).

    \thmitem{def:group/category} The \hyperref[def:category_of_small_first_order_models]{category of \( \mscrU \)-small models} \( \cat{Grp} \) of groups is \hyperref[def:concrete_category]{concrete} with respect to \hyperref[def:unital_magma/monoid]{\( \cat{Mon} \)}.

    \thmitem{def:group/exponentiation} We extend \hyperref[def:unital_magma/exponentiation]{unital magma exponentiation} to all integers by setting
    \begin{equation*}
      x^{-n} \coloneqq (x^n)^{-1}.
    \end{equation*}

    See \fullref{thm:def:group/properties/negative_power}.
  \end{thmenum}
\end{definition}

\begin{proposition}\label{thm:def:group/properties}
  Any \hyperref[def:group]{group} \( \mscrG \) has the following basic properties:
  \begin{thmenum}
    \thmitem{thm:def:group/properties/cancellative} The operation is \hyperref[def:magma/cancellative]{cancellative}.
    \thmitem{thm:def:group/properties/identity_inverse} The identity \( e \) is its own inverse.
    \thmitem{thm:def:group/properties/inverse_composition} \( (xy)^{-1} = y^{-1} x^{-1} \).
    \thmitem{thm:def:group/properties/double_inverse} For any \( x \in G \), \( x = (x^{-1})^{-1} \)
    \thmitem{thm:def:group/properties/negative_power} For any \( x \in G \) and positive integer \( n \), \( (x^n)^{-1} = (x^{-1})^n \)
    \thmitem{thm:def:group/properties/subgroup_intersection} The intersection of any two subgroups of \( \mscrG \) is again a subgroup of \( \mscrH \).
    \thmitem{thm:def:group/properties/power_set} The \hyperref[def:magma/power_set]{power set monoid} \( \pow(\mscrG) \) of a group \( \mscrG \) is not a group.
  \end{thmenum}
\end{proposition}
\begin{proof}
  \SubProofOf{thm:def:group/properties/cancellative} If \( x = y \), obviously \( xz = yz \) and \( zx = zy \). Now if \( xz = yz \), we have
  \begin{equation*}
    x = x(zz^{-1}) = (xz)z^{-1} = (yz)z^{-1} = y(zz^{-1}) = y.
  \end{equation*}

  The case \( zx = zy \) is analogous.

  \SubProofOf{thm:def:group/properties/identity_inverse} \( ee = e \).
  \SubProofOf{thm:def:group/properties/inverse_composition}
  \begin{equation*}
    (xy) (y^{-1} x^{-1})
    =
    x (y y^{-1}) x^{-1}
    =
    e
    =
    y^{-1} (x^{-1} x) y
    =
    (y^{-1} x^{-1}) (xy).
  \end{equation*}

  \SubProofOf{thm:def:group/properties/double_inverse}
  \begin{equation*}
    (x^{-1})^{-1}
    =
    x x^{-1} (x^{-1})^{-1}
    =
    x.
  \end{equation*}

  \SubProofOf{thm:def:group/properties/negative_power} Using \fullref{thm:def:group/properties/double_inverse},
  \begin{equation*}
    x^{-n}
    =
    (x^n)^{-1}
    =
    x^{-1} \cdots x^{-1}
    =
    (x^{-1})^n.
  \end{equation*}

  \SubProofOf{thm:def:group/properties/power_set} By \fullref{thm:power_set_magma_preservation}, \( \pow(\mscrG) \) is not necessarily cancellative and, by the contraposition to \fullref{thm:def:group/properties/cancellative}, it is not a group.
\end{proof}

\begin{proposition}\label{thm:group_homomorphism_single_condition}
  A function \( \varphi: \mscrG \to \mscrH \) between the groups \( \mscrG \) and \( \mscrH \) is a \hyperref[def:group/homomorphism]{homomorphism} if and only if it satisfies \eqref{eq:def:magma/homomorphism}.
\end{proposition}
\begin{proof}
  \SufficiencySubProof \eqref{eq:def:magma/homomorphism} is required to hold by definition.

  \NecessitySubProof Let the function \( \varphi \) satisfy \eqref{eq:def:magma/homomorphism}. Then it preserves identities (i.e. is a \hyperref[def:pointed_set/homomorphism]{pointed set homomorphism}) since
  \begin{equation*}
    e_{\mscrH} \varphi(e_{\mscrG}) = \varphi(e_{\mscrG}) = \varphi(e_{\mscrG} e_{\mscrG}) = \varphi(e_{\mscrG}) \varphi(e_{\mscrG})
  \end{equation*}
  and by \fullref{thm:def:group/properties/cancellative}, the operation is cancellative.

  Inverses are preserved (i.e. \eqref{eq:def:group/function_parity/odd} holds) because
  \begin{equation*}
    \varphi(x^{-1})
    =
    \varphi(x^{-1}) e_{\mscrH}
    =
    \varphi(x^{-1}) \varphi(x) \varphi(x)^{-1}
    =
    \varphi(x^{-1} x) \varphi(x)^{-1}
    =
    e_{\mscrH} \varphi(x)^{-1}
    =
    \varphi(x)^{-1}.
  \end{equation*}

  Therefore, \( \varphi \) is indeed a group homomorphism.
\end{proof}

\begin{definition}\label{def:group_cosets}
  Let \( \mscrH \subseteq G \) be a subgroup of \( \mscrG \) and let \( x \in G \). The sets
  \begin{align*}
    x \mscrH \coloneqq \{ xh \colon h \in H \}
    &&
    \mscrH x \coloneqq \{ hx \colon h \in H \}
  \end{align*}
  are called the left and right \term{cosets} of \( \mscrH \) with respect to \( x \). The name is justified by \fullref{thm:group_coset_partition}.

  The \hyperref[def:cardinal]{cardinality} of the set of all left cosets is called the \term{index} of \( \mscrH \) and is denoted by \( [\mscrG : \mscrH] \). By \fullref{thm:lagranges_theorem_for_groups}, the index can analogously be defined as the cardinality of all right cosets.
\end{definition}

\begin{lemma}\label{thm:group_coset_partition}
  The \hyperref[def:group_cosets]{left cosets} of a subgroup of \( \mscrG \) \hyperref[def:set_partition]{partition} \( \mscrG \). The same holds for right cosets.
\end{lemma}
\begin{proof}
  To each element \( x \in G \) there corresponds a coset \( x \in x\mscrH \) (since \( \mscrH \) contains the identity as a subgroup).

  Two cosets \( x\mscrH \) and \( y\mscrH \) are either disjoint or equal. Indeed, if they are not disjoint, then there exists \( g \in x\mscrH \cap y\mscrH \) and thus \( g = xa = yb \) for some \( a, b \in H \). Thus, \( x = x a a^{-1} = y b a^{-1} \) and since \( b a^{-1} \in H \), we have that \( x \in y\mscrH \). Furthermore, for any \( c \in H \), we have \( xc = y(b a^{-1} c) \in y\mscrH \), hence \( x\mscrH \subseteq y\mscrH \). After obtaining the converse inclusion, we conclude \( x\mscrH = y\mscrH \).
\end{proof}

\begin{lemma}\label{thm:group_coset_bijection}
  Any two left cosets in a group are \hyperref[def:equinumerosity]{equinumerous}. The same holds for right cosets.
\end{lemma}
\begin{proof}
  Let \( \mscrH \) be a subgroup of \( \mscrG \) and let \( x, y \in G \). Then \( z \mapsto y x^{-1} z \) sends \( x\mscrH \) into \( y\mscrH \). By \fullref{thm:group_multiplication_is_bijection}, this function is a bijection.
\end{proof}

\begin{theorem}[Lagrange's theorem for groups]\label{thm:lagranges_theorem_for_groups}
  Let \( \mscrH \) be a subgroup of \( \mscrG \). We have the following equality
  \begin{equation}\label{eq:thm:lagranges_theorem_for_groups/index}
    \card(\mscrG) = \card(\mscrH) \cdot [\mscrG : \mscrH].
  \end{equation}

  If \( \mscrH \) is a \hyperref[def:normal_subgroup]{normal subgroup}, then \( [\mscrG : \mscrH] = \card(\mscrG / \mscrH) \) and
  \begin{equation}\label{eq:thm:lagranges_theorem_for_groups/card}
    \card(\mscrG) = \card(\mscrH) \cdot \card(\mscrG / \mscrH).
  \end{equation}

  See \fullref{ex:lagranges_theorem_for_groups/direct_product_zn}.
\end{theorem}
\begin{proof}
  Follows from \fullref{thm:group_coset_partition} and \fullref{thm:group_coset_bijection}
\end{proof}

\begin{definition}\label{def:normal_subgroup}
  Let \( \mscrN \) be a subgroup of \( \mscrG \). We say that \( \mscrN \) is a normal subgroup if any of the following equivalent conditions hold:
  \begin{thmenum}
    \thmitem{def:normal_subgroup/direct} For any \( x \in G \), we have the set equality
    \begin{equation}\label{eq:def:normal_subgroup/direct}
      x \mscrN x^{-1} = \mscrN.
    \end{equation}

    \thmitem{def:normal_subgroup/cosets} The partitions induced by the left and rights cosets of \( \mscrN \) coincide.
    \thmitem{def:normal_subgroup/kernel} \( \mscrN \) is the \hyperref[def:unital_magma_kernel]{kernel} of some group homomorphism.
  \end{thmenum}

  In particular, kernels are always normal subgroups.
\end{definition}
\begin{proof}
  This is the group-theoretic analog to \fullref{thm:equivalence_partition}.

  \ImplicationSubProof{def:normal_subgroup/direct}{def:normal_subgroup/cosets} For any \( x \in G \)
  \begin{equation*}
    \mscrN x = (x \mscrN x^{-1})x = x \mscrN(x^{-1}x) = x \mscrN,
  \end{equation*}
  thus every left coset is a right coset and vice versa.

  \ImplicationSubProof{def:normal_subgroup/cosets}{def:normal_subgroup/kernel} We can take the \hyperref[def:quotient_group]{canonical projection} \( \pi(x) \coloneqq x \mscrN \) as the homomorphism. The proof of correctness in \fullref{def:quotient_group} only uses \fullref{def:normal_subgroup/cosets} and therefore does not cause circular references.

  \ImplicationSubProof{def:normal_subgroup/kernel}{def:normal_subgroup/direct} Let \( \varphi: \mscrG \to \mscrH \) be a group homomorphism and fix any \( x \in G \). Denote \( \mscrN \coloneqq \ker(f) \). Then \( x \mscrN = \mscrN x \) since
  \begin{equation*}
    \varphi(x \mscrN)
    =
    \varphi(x) \varphi(\mscrN)
    =
    \varphi(x) \varphi(e_{\mscrG})
    =
    \varphi(x)
    =
    \varphi(\mscrN) \varphi(x)
    =
    \varphi(\mscrN x)
    =
    e_{\mscrH}.
  \end{equation*}

  Thus,
  \begin{equation*}
    \varphi^{-1}(e_{\mscrH}) = \mscrN = xx^{-1}\mscrN = x \mscrN x^{-1}.
  \end{equation*}
\end{proof}

\begin{definition}\label{def:quotient_group}
  Let \( \mscrG \) be a group and \( \mscrN \) be a normal subgroup of \( \mscrG \). Define the \term{quotient group}
  \begin{equation*}
    \mscrG / \mscrN \coloneqq \{ x \mscrN \colon x \in G \}
  \end{equation*}
  with the group operation
  \begin{equation*}
    x \mscrN \odot y \mscrN \coloneqq xy \mscrN.
  \end{equation*}

  Define the canonical projection homomorphism
  \begin{align*}
    &\pi: \mscrG \to \mscrG / \mscrN \\
    &\pi(x) \coloneqq x \mscrN.
  \end{align*}

  The kernel of \( \pi \) is precisely \( \mscrN \).
\end{definition}
\begin{proof}
  This definition is used in the proof of equivalence in \fullref{def:normal_subgroup}. This is why it is important to only use \fullref{def:normal_subgroup/cosets} as the definition for a normal subgroup.

  We first check that the group operations is well-defined, that is, does not depend on the choice of coset representatives. Fix \( x_1, x_2 \in G \) and \( y_1, y_2 \in G \), so that
  \begin{equation*}
    x_1 \mscrN = y_2 \mscrN
  \end{equation*}
  and
  \begin{equation*}
    x_2 \mscrN = y_2 \mscrN.
  \end{equation*}

  Since the left and right cosets coincide, we have
  \begin{equation*}
    x_1 x_2 \mscrN = x_1 \mscrN x_2 = x_1 \mscrN y_2 = y_1 \mscrN y_2 = y_1 y_2 \mscrN.
  \end{equation*}

  Thus, the operation is well-defined.

  It follows from the definition that the identity is \( e \mscrN = \mscrN \) and the inverse of \( x \mscrN \) is \( x^{-1} \mscrN \). Therefore, \( \mscrG / \mscrN \) is indeed a group. The fact that \( \pi \) is a homomorphism is also part of the definition of \( \odot \).

  It remains to prove that \( \mscrN = \ker(\pi) \). Obviously \( \pi(\mscrN) = \mscrN \), so \( \mscrN \subseteq \ker(\pi) \). To see that the converse holds, assume that there exists \( x \in \ker(\pi) \setminus \mscrN \), that is, \( \pi(x) = x\mscrN = \mscrN \), but \( x \not\in N \). Then there exists \( y \in N \) such that \( xy \in N \). But \( \mscrN \) is closed under multiplication and inverses, hence \( x = xyy^{-1} \in N \). This contradicts our assumption that \( x \not\in N \). Therefore, \( \mscrN = \ker \pi \).
\end{proof}

\begin{theorem}[Homomorphism theorem for groups]\label{thm:homomorphism_theorem_for_groups}
  For any \hyperref[def:group/homomorphism]{group homomorphism} \( \varphi: \mscrG \to \mscrH \), we have the isomorphism
  \begin{equation*}
    \mscrG / \ker \varphi \cong \img \varphi.
  \end{equation*}
\end{theorem}
\begin{proof}
  Denote \( \mscrN \coloneqq \ker \varphi \). Define the function
  \begin{align*}
    &\psi: \img \varphi \to \pow(\mscrG) \\
    &\psi(y) \coloneqq \varphi^{-1}(y) \mscrN.
  \end{align*}

  We will show that \( \psi \) is the desired isomorphism.

  Fix any \( y \in \img \varphi \) and \( x_1, x_2 \in \varphi^{-1}(y) \). We will first show that \( x_1 \mscrN = x_2 \mscrN \). Note that
  \begin{equation*}
    \varphi(x_1^{-1} x_2)
    =
    \varphi(x_1)^{-1} \varphi(x_2)
    =
    \varphi(x_2)^{-1} \varphi(x_2)
    =
    e_{\mscrH},
  \end{equation*}
  therefore \( x_1^{-1} x_2 \in N \). Thus,
  \begin{equation*}
    x_2 \mscrN = x_1 x_1^{-1} x_2 \mscrN = x_1 \cdot \mscrN \cdot \mscrN = x_1 \mscrN.
  \end{equation*}

  Hence, \( \varphi^{-1}(y) \mscrN \) is a coset in \( \mscrG / \mscrN \) formed by any of the elements of \( \varphi^{-1}(y) \).

  Furthermore, if \( x_1 \in x_2 \mscrN \), then there exists \( n \in N \) such that
  \begin{equation*}
    x_1 = x_2 n.
  \end{equation*}

  But \( \mscrN \) is closed under taking inverses, hence
  \begin{equation*}
    x_2 = x_1 n^{-1} \in x_1 \mscrN,
  \end{equation*}
  that is,
  \begin{equation*}
    x_1 \mscrN = x_2 \mscrN.
  \end{equation*}

  This shows that \( \psi \) is injective. It is obviously surjective because if \( x \mscrN \) is a coset, then \( \varphi(x) \in \img \varphi \). Therefore, \( \varphi \) is bijective.

  It remains to show that \( \psi \) is a homomorphism. Indeed, if \( y_1, y_2 \in \img \varphi \) and
  \begin{equation*}
    x_k \in \varphi^{-1}(y_k), k = 1, 2,
  \end{equation*}
  we have
  \begin{balign*}
    \psi(y_1) \psi(y_2)
    &=
    \varphi^{-1}(y_1) \mscrN \varphi^{-1}(y_2) \mscrN
    = \\ &=
    x_1 \mscrN x_2 \mscrN
    \reloset {\eqref{eq:def:normal_subgroup/direct}} = \\ &=
    (x_1 x_2) \mscrN
    = \\ &=
    \varphi^{-1}(y_1 y_2) \mscrN
    = \\ &=
    \psi(y_1 y_2).
  \end{balign*}
\end{proof}

\subsection{Group presentations}\label{subsec:group_presentations}

\begin{definition}\label{def:free_monoid}
  Let \( \CS \) be an arbitrary set. We associate with \( \CS \) its \Def{free monoid} \( F(\CS) \coloneqq (\CS^{\ast}, \cdot) \), where \( \CS^{\ast} \) is the \hyperref[def:language/kleene_star]{Kleene star} and \( \cdot \) is \hyperref[def:language/concatenation]{concatenation}. It is a monoid due to \fullref{thm:kleene_star_is_monoid}.
\end{definition}

\begin{proposition}\label{thm:free_monoid_is_free_functor}
  The functor \( F: \Cat{Set} \to \Cat{Mon} \), defined pointwise in \fullref{def:free_monoid}, is \hyperref[def:free_functor]{free}.
\end{proposition}
\begin{proof}
  Let \( U: \Cat{Mon} \to \Cat{Set} \) be the corresponding forgetful functor and let \( \CM \in \Cat{Mon} \), \( \CS \in \Cat{Set} \). We will first show that
  \begin{equation*}
    \Cat{Mon}(F(\CS), \CM) = \Cat{Set}(\CS, U(\CM)),
  \end{equation*}
  where equality means that all the underlying functions are equal.

  Every monoid homomorphism is a function so obviously
  \begin{equation*}
    \Cat{Mon}(F(\CS), \CM) \subseteq \Cat{Set}(\CS, U(\CM)).
  \end{equation*}

  Now consider a function \( f: \CS \to U(\CM) \). Define the function
  \begin{BreakableAlign*}
     &\varphi: F(\CS) \to \CM \\
     &\varphi\left( \{ x_k \}_{k \in \CK} \right) \coloneqq \prod_{k \in \CK} x_k.
  \end{BreakableAlign*}

  Obviously \( \varphi \) is a homomorphism from \( F(\CS) \) to \( \CM \). Hence
  \begin{equation*}
    \Cat{Set}(\CS, U(\CM)) \subseteq \Cat{Mon}(F(\CS), \CM).
  \end{equation*}
\end{proof}

\begin{definition}\label{def:free_group}
  Let \( \CS \) be an arbitrary set. We will now construct the \Def{free group} \( F(\CS) \) of \( \CS \). The construction is similar to that of \hyperref[def:free_monoid]{free monoids} but it is much more complicated because of special reduction rules for \hyperref[def:unital_magma_inverse_element]{inverse elements}. Refer to \cite{code:free_group_grammar_verification} for a software implementation of the construction.

  Let \( \star \) be a \hyperref[def:language/symbol]{symbol} not in \( \CS \). Our goal is, for each \( a \in \CS \), to make the word \( a{\star} \) behave like the inverse of \( a \) in a group. Rather than considering the \hyperref[def:language/kleene_star]{Kleene star} \( (S \cup \{ \star \})^* \) and removing elements via \enquote{reductions} as in \cite{code:free_group_reduction_verification} and \cite[306]{Knapp2016BAlg}, we directly build a language of \Def{reduced words} using the mutually recursive \hyperref[def:grammar]{grammar}
  \begin{AlignedEquation}\label{eq:def:free_group/grammar}
    &I \to \varepsilon,           &&                        && \text{\( I \) is the initial state} \\
    &I \to S_a \mid D_a,             && a \in \CS              && \\
    &S_a \to a \mid a S_a,           && a \in \CS              && S_a \text{ does not produce words beginning with } a\star \\
    &S_a \to a D_b,               && a, b \in \CS, a \neq b && \\
    &D_a \to a\star S_b,          && a, b \in \CS, a \neq b && D_a \text{ does not produce words beginning with } a \\
    &D_a \to a\star \mid a\star D_a, && a \in \CS              && \\
  \end{AlignedEquation}

  The \Def{free group} \( F(\CS) \) is defined to be the language of \eqref{eq:def:free_group/grammar} equipped with the inductively\IND defined operation
  \begin{equation}\label{eq:def:free_group/operation}
    w_1 \odot w_2 \coloneqq \begin{cases}
     p \odot s, &w_1 = p a \T{and} w_2 = a\star s \text{ for some } a \in \CS, \\
     p \odot s, &w_1 = p a\star \T{and} w_2 = as \T{and} s \neq \star t \text{ for some } a \in \CS, \\
     ps,        &\text{otherwise}.
   \end{cases}
  \end{equation}

  The inverse of the word \( w = a_1 \ldots a_n \) is \( w^{-1} \coloneqq b_1 \ldots b_n \), where
   \begin{equation}\label{eq:def:free_group/inverse}
     b_{n-k+1} \coloneqq \begin{cases}
       \varnothing, &a_k = {\star} \\
       a_k{\star},  &a_k \neq {\star} \T{and} k = n \\
       a_k{\star},  &a_k \neq {\star} \T{and} a_{k+1} \neq {\star} \\
       a_k,         &a_k \neq {\star} \T{and} k \neq n \T{and} a_{k+1} = {\star} \\
     \end{cases}
   \end{equation}
   for \( k = 1, \ldots, n \).

  The group \( (F(\CS), \odot) \) is called the \Def{free group} generated by \( \CS \).
\end{definition}
\begin{proof}
  The proof of the well-definedness of the group structure of \( F(\CS) \) is a straightforward (but tedious) application of induction\IND.
\end{proof}

\begin{proposition}\label{thm:free_group_is_free_functor}
  The functor \( F: \Cat{Set} \to \Cat{Grp} \), defined pointwise in \fullref{def:free_group}, is \hyperref[def:free_functor]{free}.
\end{proposition}
\begin{proof}
  The outline of the proof is similar to the proof of \fullref{thm:free_monoid_is_free_functor}.
\end{proof}

\begin{definition}\label{def:group_presentation}\MarginCite[314]{Knapp2016BAlg}
  Let \( \CS \) be a set, \( F(\CS) \) be the \hyperref[def:free_group]{free group} and \( \CR \subseteq F(\CS) \) be a subset. Denote by \( \CN(\CR) \) the smallest normal subgroup of \( F(\CS) \) that includes \( \CR \) as a subset.

  We define the group
  \begin{equation}\label{eq:def:group_presentation/presentation}
    \CG = \Gen{ \CS \mid \CR} \coloneqq F(\CS) / \CN(\CR)
  \end{equation}
  called the group with \Def{generators} \( \CS \) and \Def{relators} \( \CR \). The expression \fullref{def:group_presentation/presentation} is called a \Def{presentation} of \( \CG \).

  If there exists a presentation for \( \CG \) such that \( \CS \) is finite, it is called a \Def{finitely generated} group. If there exists a presentation such that both \( \CS \) and \( \CR \) are finite, it is called \Def{finitely presented}.

  If \( \CR = \varnothing \), there are no restrictions and we use the notation
  \begin{equation}\label{eq:def:group_presentation/free}
    \CG = \Gen{ \CS } \coloneqq F(\CS)
  \end{equation}
  for the free group.
\end{definition}

\begin{theorem}\label{thm:every_group_is_representable}\MarginCite[prop. 7.7]{Knapp2016BAlg}
  Every group \( \CG \) has at least one \hyperref[def:group_presentation]{presentation}.
\end{theorem}
\begin{proof}
  Let \( \CG \) be an arbitrary group and let \( \CS \coloneqq U(\CG) \) be the underlying set. Let \( F(\CS) \) be the corresponding free group with \( \iota: \CS \to F(\CS) \) sending elements of \( \CS \) to singleton words in \( F(\CS) \). By \fullref{thm:free_group_is_free_functor}, there exists a unique homomorphism \( \varphi: F(\CS) \to \CG \) such that
  \begin{equation*}
    \begin{mplibcode}
      beginfig(1);
      input metapost/graphs;

      v1 := thelabel("$\CS$", origin);
      v2 := thelabel("$U(F(\CS))$", (-1, -1) scaled u);
      v3 := thelabel("$U(G)$", (1, -1) scaled u);

      a1 := straight_arc(v1, v2);
      a2 := straight_arc(v1, v3);

      d1 := straight_arc(v2, v3);

      draw_vertices(v);
      draw_arcs(a);

      drawarrow d1 dotted;

      label.ulft("$\iota$", straight_arc_midpoint of a1);
      label.urt("$\Id$", straight_arc_midpoint of a2);
      label.top("$U(\varphi)$", straight_arc_midpoint of d1);
      endfig;
    \end{mplibcode}
  \end{equation*}
  that is, \( U(\varphi) \circ \iota = \Id \). Thus \( G = \CS \subseteq \ker \varphi \). Define \( \CR \coloneqq \ker \varphi \). By \fullref{def:normal_subgroup}, \( \CR \) is a normal subgroup of \( F(\CS) \), thus
  \begin{equation*}
    G = \varphi(F(\CS)) \cong F(\CS) / \ker \varphi = \Gen{ \CS \mid \CR }.
  \end{equation*}
\end{proof}

\begin{definition}\label{def:cyclic_group}
  For a singleton alphabet \( \{ a \} \), we define the \Def{infinite cyclic group}
  \begin{equation*}
    C \coloneqq \Gen{a}
  \end{equation*}
  and, for positive integers \( n \), the \Def{finite cyclic group} of \Def{order} \( n \) as
  \begin{equation*}
    C_n \coloneqq \Gen{a \mid a^n}.
  \end{equation*}

  We use the same notation independent of \( a \) because all cyclic groups of the same order are obviously \hyperref[def:group/homomorphism]{isomorphic}.

  See \fullref{thm:cyclic_group_isomorphic_to_integers_modulo_n}.
\end{definition}

\begin{definition}\label{def:group_free_product}\MarginCite[323]{Knapp2016BAlg}
  The \Def{free product} of a nonempty family of groups \( \{ \CX_k \}_{k \in \CK} \) with presentations \( \Gen{\CS_k \mid \CR_k}, k \in \CK \) is the group
  \begin{equation*}
    \Ast_{k \in \CK} \CX_k \coloneqq \Gen{ \coprod_{k \in \CK} \CS_k | \coprod_{k \in \CK} \CR_k },
  \end{equation*}
  where \( \coprod \) is the \hyperref[def:disjoint_union]{disjoint union}.
\end{definition}

\begin{definition}\label{def:free_abelian_group}
  A \Def{free abelian group} is a \hyperref[def:free_left_module]{free} \hyperref[thm:abelian_group_iff_z_module]{\( \BZ \)-module}. This definition of a free abelian group is different from the definition of a \hyperref[def:free_group]{free group}.
\end{definition}

\subsection{Group actions}\label{subsec:group_actions}

\begin{definition}\label{def:endomorphism_monoid}
  For every object \( X \) in an arbitrary \hyperref[def:category]{category} \( \cat{C} \), the set \( \cat{C}(X) \) is a \hyperref[def:monoid]{monoid} with morphism composition as the monoid operation and \( \id_X \) as the monoid identity.

  Outside of \hyperref[sec:category_theory]{category theory}, whenever the category \( \cat{C} \) is clear from the context, we call \( \cat{C}(X) \) the \term{endomorphism monoid} over \( X \) and denote it by \( \End(X) \).
\end{definition}

\begin{definition}\label{def:monoid_action}
  Let \( M \) be a \hyperref[def:monoid]{monoid} and let \( X \) be an object in some \hyperref[def:category]{category} \( \cat{C} \).

  We will define monoid actions of \( M \) on \( X \), which we will sometimes call \term{left monoid actions}. There are also \term{right monoid actions}, which are only briefly mentioned in \fullref{def:monoid_action/functor}.

  A \term{monoid action} can be defined equivalently as:
  \begin{thmenum}

    \thmitem{def:monoid_action/homomorphism} A \hyperref[def:monoid/homomorphism]{homomorphism} from \( M \) to the \hyperref[def:endomorphism_monoid]{endomorphism monoid} \( \End(X) \).

    \thmitem{def:monoid_action/functor} A \hyperref[def:functor]{functor} from the \hyperref[def:monoid_delooping]{delooping} \( \cat{B}_M \) to \( \cat{C} \).

    Right actions are \hyperref[rem:contravariant_functor]{contravariant functors}.

    \thmitem{def:monoid_action/family} An \hyperref[def:cartesian_product/indexed_family]{indexed family} \( \seq{ \Phi_m }_{m \in M} \) of \hyperref[def:morphism_invertibility/endomorphism]{endomorphisms} of \( X \) such that
    \begin{align}
      &\Phi_e = \id_X, \label{eq:def:monoid_action/family/identity}\tag{\logic{MA1}} \\
      &\Phi_{mn} = \Phi_m \bincirc \Phi_n. \label{eq:def:monoid_action/family/compatibility}\tag{\logic{MA2}}
    \end{align}

    This defines a function \( \Phi: M \times A \to A \).
  \end{thmenum}
\end{definition}
\begin{proof}
  \ImplicationSubProof{def:monoid_action/homomorphism}{def:monoid_action/functor} Suppose that we have a monoid homomorphism \( \Phi: M \to \End(X) \). Define the functor
  \begin{equation*}
    \begin{aligned}
      &F: \cat{B}_M \to \cat{C} \\
      &F(\anon) \coloneqq X \\
      &F(m) \coloneqq \Phi(m).
    \end{aligned}
  \end{equation*}

  This is indeed a functor because \eqref{eq:def:functor/CF1} follows from \eqref{eq:def:pointed_set/homomorphism} and \eqref{eq:def:functor/CF2} follows from \eqref{eq:def:magma/homomorphism}.

  \ImplicationSubProof{def:monoid_action/functor}{def:monoid_action/family} Suppose that we have a functor \( F: \cat{B}_M \to \cat{C} \). Let \( X \coloneqq F(\anon) \) and define the \( M \)-indexed family
  \begin{equation*}
    \begin{aligned}
      &\Phi_m: X \to X \\
      &\Phi_m \coloneqq F(m).
    \end{aligned}
  \end{equation*}

  It satisfies the necessary axioms:
  \begin{itemize}
    \item \ref{eq:def:monoid_action/family/identity} holds:
    \begin{equation*}
      \Phi_e
      =
      F(e)
      \reloset {\eqref{eq:def:functor/CF1}} =
      \id_A.
    \end{equation*}

    \item \ref{eq:def:monoid_action/family/compatibility} holds: for every pair \( m, n \in M \), we have
    \begin{equation*}
      \Phi_{mn}
      =
      F(mn)
      \reloset {\eqref{eq:def:functor/CF2}} =
      F(m) \bincirc F(n)
      =
      \Phi_m \bincirc \Phi_n
    \end{equation*}
  \end{itemize}

  \ImplicationSubProof{def:monoid_action/family}{def:monoid_action/homomorphism} Suppose that we have an indexed family \( \seq{ \Phi_m }_{m \in M} \) of endomorphisms of \( A \) that satisfies the axioms for left action. Regard this indexed family as a function \( \Phi: M \to \End(X) \).

  Then \( \Phi \) is a monoid homomorphism because \ref{eq:def:monoid_action/family/identity} implies \( \Phi(e) = \id_X \) and \eqref{eq:def:monoid_action/family/compatibility} implies
  \begin{equation*}
    \Phi(mn) = \Phi(m) \bincirc \Phi(n).
  \end{equation*}
\end{proof}

\begin{proposition}\label{thm:monoid_is_action}
  Every \hyperref[def:monoid]{monoid} \hyperref[def:monoid_action]{acts} on itself via the family of functions \( h \mapsto g \cdot h \) indexed by \( g \). These functions are not monoid homomorphisms in general.

  Compare this result to \fullref{thm:cayleys_theorem}.
\end{proposition}
\begin{proof}
  The family satisfies \fullref{def:monoid_action/family}:
  \begin{itemize}
    \item \ref{eq:def:monoid_action/family/identity} follows from \eqref{eq:def:monoid/theory/identity}.

    \item \ref{eq:def:monoid_action/family/compatibility} follows from associativity:
    \begin{equation*}
      [h \mapsto g_1 \cdot h] \bincirc [h \mapsto g_2 \cdot h] = [h \mapsto g_1 \cdot (g_2 \cdot h)] = [h \mapsto (g_1 \cdot g_2) \cdot h].
    \end{equation*}
  \end{itemize}
\end{proof}

\begin{proposition}\label{thm:exponentiation_monoid_action}
  The \hyperref[def:set_of_natural_numbers]{natural numbers} \( \BbbN \) (\hyperref[rem:peano_arithmetic_zero]{with zero}) act on any \hyperref[def:monoid]{monoid} by \hyperref[def:monoid/exponentiation]{exponentiation} via the family of function \( g \mapsto g^n \) indexed by \( n \in \BbbN \).

  Compare this result to \fullref{thm:exponentiation_group_action}.
\end{proposition}
\begin{proof}
  This family satisfies \fullref{def:monoid_action/family}:
  \begin{itemize}
    \item \ref{eq:def:monoid_action/family/identity} is obvious.
    \item \ref{eq:def:monoid_action/family/compatibility} follows from \fullref{thm:magma_exponentiation_properties/repeated}.
  \end{itemize}
\end{proof}

\begin{definition}\label{def:automorphism_group}
  For every object \( X \) in a \hyperref[def:groupoid]{groupoid} \( \cat{G} \), the set \( \cat{G}(X) \) is a \hyperref[def:group]{group} with morphism composition as the group operation.

  Similarly to \hyperref[def:endomorphism_monoid]{endomorphism monoids}, whenever the groupoid \( \cat{G} \) is clear from the context, we call \( \cat{G}(X) \) the \term{automorphism group} over \( X \) and denote it by \( \aut(X) \).
\end{definition}

\begin{definition}\label{def:symmetric_group}
  We call the \hyperref[def:automorphism_group]{automorphism group} of a set \( A \) the \term{symmetric group} on \( A \) and denote it by \( S(A) \). The group \( S(A) \) consists of bijective functions, which we call \term{permutations}.

  Rather than for arbitrary sets, we often consider symmetric group on \( n \) letters
  \begin{equation*}
    S_n \coloneqq S(\set{ 1, 2, \ldots, n }).
  \end{equation*}

  \begin{thmenum}
    \thmitem{def:symmetric_group/permutation} It is common to write a permutation \( p \) in \( S_n \) as
    \begin{equation*}
      \begin{pmatrix}
        1    & \cdots & n \\
        p(1) & \cdots & p(n)
      \end{pmatrix}
    \end{equation*}

    \thmitem{def:symmetric_group/cycle} If there exists a finite sequence \( (k_1, \ldots, k_m) \) of distinct numbers such that \( p(k_m) = k_1 \) and \( p(k_{i+1}) = p(k_i) \) for each \( i < m \), we say that the permutation is \term{cyclic} or a \term{cycle} of \term{length} \( m \). For brevity, we denote this cycle by \( \cycle{k_1, \ldots, k_m} \).

    Cycles of length \( 2 \) are also called \term{transpositions}. Every \hyperref[def:symmetric_group/cycle]{transposition} is an \hyperref[def:set_with_involution]{involution}, and thus \( t = t^{-1} \) (as permutations).

    \thmitem{def:symmetric_group/disjoint_cycle} If \( \cycle{k_1, \ldots, k_m} \) and \( \cycle{s_1, \ldots, s_l} \) are two cycles and if the sets \( \set{ k_1, \ldots, k_m } \) and \( \set{ s_1, \ldots, s_l } \) are disjoint, we say that the cycles themselves are \term{disjoint}.

    Two disjoint cycles commute. That is,
    \begin{equation*}
      \cycle{k_1, \ldots, k_m} \bincirc \cycle{s_1, \ldots, s_l} = \cycle{s_1, \ldots, s_l} \bincirc \cycle{k_1, \ldots, k_m}.
    \end{equation*}
  \end{thmenum}
\end{definition}

\begin{proposition}\label{thm:cycle_transposition_decomposition}
  Every cycle \( \cycle{ k_1, \ldots, k_m } \) can be decomposed into the product of transpositions
  \begin{equation*}
    \cycle{ k_1, \ldots, k_m } = \cycle{ k_1, k_m } \bincirc \cycle{ k_1, k_{m-1} } \bincirc \cdots \bincirc \cycle{ k_1, k_2 }.
  \end{equation*}
\end{proposition}
\begin{proof}
  Trivial.
\end{proof}

\begin{proposition}\label{thm:symmetric_group_cardinality}
  The \hyperref[def:symmetric_group]{symmetric group} \( S_n \) has \( n! \) elements.
\end{proposition}
\begin{proof}
  We use induction on \( n \). The case \( n = 1 \) is trivial. Suppose that \( S_{n-1} \) has \( (n-1)! \) elements. Then \( S_n \) is obtained by permuting \( n \) with each element of \( S_{n-1} \). That is,
  \begin{equation*}
    S_n = \set{ \cycle{ k, n } \bincirc p \given p \in S_{n-1} \T{and} 1 \leq k \leq n }.
  \end{equation*}

  It follows that
  \begin{equation*}
    \card(S_n) = n \cdot \card(S_{n-1}) = n (n-1)! = n!.
  \end{equation*}
\end{proof}

\begin{definition}\label{def:group_action}
  Let \( G \) be a \hyperref[def:group]{group} and let \( X \) be an object in some \hyperref[def:concrete_category]{concrete category} \( \cat{C} \).

  We will define group actions as a special case of \hyperref[def:monoid_action]{monoid actions}, with the same remarks regarding left and right group actions.

  A \term{group action} can be defined equivalently as:
  \begin{thmenum}
    \thmitem{def:group_action/homomorphism} A \hyperref[def:group/homomorphism]{homomorphism} from \( G \) to the \hyperref[def:automorphism_group]{automorphism group} \( \aut(X) \).

    Right actions are homomorphisms from the \hyperref[def:monoid/opposite]{opposite} group \( G^{\opcat} \) to \( \aut(X) \).

    \thmitem{def:group_action/functor} A \hyperref[def:functor]{functor} from the \hyperref[def:monoid_delooping]{delooping} \( \cat{B}_G \) to \( \cat{C} \).

    Right actions are \hyperref[rem:contravariant_functor]{contravariant functors}.

    \thmitem{def:group_action/family} An \hyperref[def:cartesian_product/indexed_family]{indexed family} \( \seq{ \Phi_x }_{x \in G} \) of \hyperref[def:morphism_invertibility/isomorphism]{isomorphisms} of \( X \) such that, for every pair \( g, h \in G \),
    \begin{equation}\label{eq:def:group_action/family/compatibility}\tag{\logic{GA}}
      \Phi_{gh} = \Phi_g \bincirc \Phi_h.
    \end{equation}

    This defines a function \( \Phi: G \times A \to A \).
  \end{thmenum}
\end{definition}
\begin{proof}
  The proof of equivalence is simple; it is similar to \fullref{def:monoid_action}.
\end{proof}

\begin{theorem}[Cayley's theorem]\label{thm:cayleys_theorem}
  Every \hyperref[def:group]{group} \hyperref[def:group_action]{acts} on itself via the family of functions \( y \mapsto x \cdot y \) indexed by \( x \). These functions are not group homomorphisms in general.

  Compare this result to \fullref{thm:monoid_is_action}.
\end{theorem}
\begin{proof}
  Follows directly from \fullref{thm:monoid_is_action} and \fullref{thm:group_operation_induces_bijections}.
\end{proof}

\begin{proposition}\label{thm:exponentiation_group_action}
  The \hyperref[def:set_of_integers]{integers} \( \BbbZ \) act on any \hyperref[def:group]{group} by \hyperref[def:monoid/exponentiation]{exponentiation} via the family of function \( g \mapsto g^n \) indexed by \( n \in \BbbZ \).

  Compare this result to \fullref{thm:exponentiation_monoid_action}.
\end{proposition}
\begin{proof}
  Follows from \fullref{thm:exponentiation_monoid_action}.
\end{proof}

\begin{definition}\label{def:group_action_orbit}\mcite[163]{Knapp2016BasicAlgebra}
  The \term{orbit} of \( x \) under the \hyperref[def:group_action]{group action} \( \Phi: G \to \End(X) \) is the set
  \begin{equation*}
    \set{ \Phi_g(x) \given g \in G }.
  \end{equation*}

  This is the set of all members of \( X \) \enquote{reachable} from \( x \) via the action.

  The relation \( g \sim h \) on \( G \), defined to hold if \( g \) and \( h \) have the same \hyperref[def:group_action_orbit]{orbit}, is an \hyperref[def:equivalence_relation]{equivalence relation}. The quotient set \( G / {\sim} \) is a partition of \( G \) into sets called \term{orbits}.
\end{definition}

\begin{example}\label{ex:plane_rotation_action_orbits}
  Consider the action of the additive group of \( \BbbR \) on \( \BbbR^2 \) given by the \hyperref[def:euclidean_transformation/rotation]{rotation} \hyperref[def:array/matrix]{matrices}
  \begin{equation*}
    \Phi_r \coloneqq \begin{pmatrix}
      \cos(r)  & \sin(r) \\
      -\sin(r) & \cos(r) \\
    \end{pmatrix}
  \end{equation*}

  Fix a nonzero vector \( (x, y)^T \) in \( \BbbR^2 \) with norm \( l \). Since rotation matrices are orthogonal, they preserve norms. Furthermore, given a vector of norm \( l \), with the angle \( r \) defined via \eqref{eq:def:angle/measure}, \( \Phi_r^{-1} \) sends the vector to \( (x, y)^T \).

  The \hyperref[def:group_action_orbit]{orbit} of \( (x, y)^T \) is thus a \hyperref[def:quadratic_plane_curve/ellipse]{circle} at the origin with radius \( l \).
\end{example}

\begin{proposition}\label{thm:group_conjugation_action}\mcite[165]{Knapp2016BasicAlgebra}
  Every \hyperref[def:group]{group} \hyperref[def:group_action]{acts} on itself via the \term{conjugation automorphisms} defined as
  \begin{equation*}
    \begin{aligned}
      &\Phi_g: G \to G, \\
      &\Phi_g(h) \coloneqq g h g^{-1}.
    \end{aligned}
  \end{equation*}
\end{proposition}
\begin{proof}
  Trivial.
\end{proof}

\begin{definition}\label{def:inner_and_outer_automorphisms}\mcite[exer. 6.12.15]{Knapp2016BasicAlgebra}
  Denote by \( \Phi: G \to \aut(G) \) the \hyperref[thm:group_conjugation_action]{conjugation action} on the group \( G \).

  The \term{inner automorphisms group} of \( G \) is
  \begin{equation*}
    \op{inn}(G) \coloneqq \set{ \Phi_g \given g \in G }.
  \end{equation*}

  The \term{outer automorphism group} is the \hyperref[def:group/quotient]{quotient group}
  \begin{equation*}
    \op{out}(G) \coloneqq \aut(G) / \op{inn}(G).
  \end{equation*}
\end{definition}
\begin{defproof}
  We will show that \( \op{inn}(G) \) is a \hyperref[thm:normal_subgroup_equivalences]{normal subgroup} of \( \aut(G) \).

  Fix a member \( g \) of \( G \) and define the inner automorphism
  \begin{equation*}
    \varphi(h) \coloneqq gh^{-1}g.
  \end{equation*}

  Let \( \psi \) be an arbitrary automorphism of \( G \). Then
  \begin{equation*}
    [\varphi \bincirc \psi \bincirc \varphi^{-1}](h)
    =
    \varphi(g \varphi^{-1}(h) g^{-1})
    =
    \varphi(g) h \varphi(g^{-1}),
  \end{equation*}
  and thus \( \varphi \bincirc \psi \bincirc \varphi^{-1} \) is again an inner automorphism.
\end{defproof}

\begin{definition}\label{def:permutation_cycle_decomposition}\mimprovised
  Let \( S_n \) be a \hyperref[def:symmetric_group]{symmetric group} and let \( p \) be a permutation in \( S_n \). Consider the \hyperref[def:group_action]{group action}
  \begin{equation*}
    \begin{aligned}
      &\Phi^{(p)}: \braket{ p } \times \set{ 1, \ldots, n } \to \set{ 1, \ldots, n } \\
      &\Phi^{(p)}_p(k) \coloneqq p(k)
    \end{aligned}
  \end{equation*}
  of the \hyperref[def:first_order_generated_substructure]{generated subgroup} \( \braket{ p } \).

  The \hyperref[def:group_action_orbit]{orbit} of \( k \) under \( \Phi^{(p)} \) is the set
  \begin{equation*}
    O_p(k) \coloneqq \set{ p^m(k) \given m \in \BbbZ }
  \end{equation*}
  of all numbers reachable from \( k \) via iterated application of \( p \) or \( p^{-1} \). The family
  \begin{equation*}
    \set{ O_p(k) \given k = 1, \ldots, n }
  \end{equation*}
  of orbits partitions \( 1, \ldots, n \) into disjoint subsets.

  Each orbit \( O \) has a smallest element \( o \); and \( O_p(o) = O \). This smallest element uniquely identifies a cycle
  \begin{equation*}
    \cycle{o, p(o), p^2(o), \ldots, p^{c-1}(o)},
  \end{equation*}
  where \( c \) is the cardinality of \( O_p(o) \).

  Cycles of length \( 1 \) are not useful to us since they represent fixed points of \( p \); we ignore these cycles. We obtain a unique set of disjoint cycles for every permutation \( p \) in \( S_n \), which we call the \term{cycle decomposition} of \( p \). We denote this set by \( C_p \).
\end{definition}
\begin{defproof}
  We must prove that \( C_p \) is family of disjoint sets. Let \( O_p(k_1) \) and \( O_p(k_2) \) be two orbits.

  Suppose that there exists a number \( k \in O_p(k_2) \cap O_p(k_1) \). Then there exist \( m_1, m_2 \leq n \) such that \( k = p^{m_1}(k_1) \) and \( k = p^{m_2}(k_2) \). Then
  \begin{equation*}
    k_1 = p^{-m_1}(k) = p^{-m_1}(p^{m_2}(k_2)) = p^{m_2 - m_1}(k_2),
  \end{equation*}
  and thus \( O_p(k_2) \subseteq O_p(k_1) \). We can analogously obtain the converse inclusion.

  Therefore, if two orbits have a nonempty intersection, they are equal.
\end{defproof}

\begin{proposition}\label{thm:permutation_decomposition}
  Given a permutation \( p \), for every ordering \( c_1, \ldots, c_m \) of the \hyperref[def:permutation_cycle_decomposition]{cycle decomposition} of \( p \), we have
  \begin{equation*}
    p = c_1 \bincirc \cdots \bincirc c_m.
  \end{equation*}
\end{proposition}
\begin{proof}
  We will use induction on \( m \). The case \( m = 0 \) is trivial. For the inductive hypothesis, note that \( c_1^{-1} p \) has as cycles \( c_2, \ldots, c_m \) because all the numbers in \( c_1 \) are fixed points of \( c_1^{-1} p \). Thus, the inductive hypothesis holds for \( c_1^{-1} p \).

  From
  \begin{equation*}
    c_1^{-1} p = c_2 \bincirc \cdots \bincirc c_m
  \end{equation*}
  it follows that
  \begin{equation*}
    p = c_1 \bincirc \cdots \bincirc c_m.
  \end{equation*}
\end{proof}

\begin{definition}\label{def:permutation_parity}
  Let \( C_p \) be the \hyperref[def:permutation_cycle_decomposition]{cycle decomposition} of \( p \). We say that \( p \) is an \term{even permutation} if the sum
  \begin{equation*}
    \sum_{c \in C_p} (\len(c) - 1)
  \end{equation*}
  is a an even number; and likewise for \term{odd permutations}. In particular, a non-identity cycle \( p \) is an even permutation if and only if \( \len(p) \) is \hi{odd}.

  The decomposition of individual cycles from \fullref{thm:cycle_transposition_decomposition} implies that the above is precisely the total number of transpositions in the decomposition of all the cycles.

  We correspondingly define the \term{sign} of a permutation as
  \begin{equation*}
    \begin{aligned}
       &\sgn: S_n \to \set{ -1, 1 }, \\
       &\sgn(p) \coloneqq \begin{cases}
        1,  &p \T{is even} \\
        -1, &p \T{is odd}
      \end{cases}
    \end{aligned}
  \end{equation*}
\end{definition}

\begin{definition}\label{def:alternating_group}
  The \term{alternating group} \( A_n \) on \( n \) letters is the subgroup of all \hyperref[def:permutation_parity]{even permutation} in the \hyperref[def:symmetric_group]{symmetric group} \( S_n \).
\end{definition}

\begin{proposition}\label{thm:alternating_group_cardinality}
  The \hyperref[def:alternating_group]{alternating group} \( A_n \) has \( \ifrac {n!} 2 \) elements.
\end{proposition}
\begin{proof}
  The proof is similar to that of \fullref{thm:symmetric_group_cardinality}, but is a little different.

  We use induction on \( n \). The case \( n = 1 \) is trivial. Suppose that \( A_{n-1} \) has \( \ifrac {(n-1)!} 2 \) elements. Then
  \begin{equation*}
    A_n = \set{ \cycle{ k, n } \bincirc p \given p \in S_{n-1} \setminus A_{n-1} \T{and} 1 \leq k \leq n }.
  \end{equation*}

  We obtain \( A_n \) by taking all the odd permutations in \( S_{n-1} \) and composing them with one new transposition. It follows that
  \begin{equation*}
    \card(A_n) = n \cdot \card(S_{n-1} \setminus A_{n-1}) = n \frac {(n-1)!} 2 = \frac {n!} 2.
  \end{equation*}
\end{proof}

\begin{example}\label{ex:s3_and_a3}
  The \hyperref[def:symmetric_group]{symmetric group} \( S_3 \) contains the following \hyperref[def:symmetric_group/permutation]{permutations}:
  \begin{equation*}
    S_3
    \coloneqq
    \set[\vast]
    {
      \begin{pmatrix}
        1 & 2 & 3 \\
        1 & 2 & 3
      \end{pmatrix},
      \underbrace
        {
          \begin{pmatrix}
            1 & 2 & 3 \\
            2 & 1 & 3
          \end{pmatrix}
        }_{
          \cycle{ 1, 2 }
        },
      \underbrace
        {
          \begin{pmatrix}
            1 & 2 & 3 \\
            2 & 3 & 1
          \end{pmatrix}
        }_{
          \cycle{ 1, 2, 3 }
        },
      \underbrace
        {
          \begin{pmatrix}
            1 & 2 & 3 \\
            3 & 2 & 1
          \end{pmatrix}
        }_{
          \cycle{ 1, 3 }
        },
      \underbrace
        {
          \begin{pmatrix}
            1 & 2 & 3 \\
            3 & 1 & 2
          \end{pmatrix}
        }_{
          \cycle{ 1, 3, 2 }
        },
      \underbrace
        {
          \begin{pmatrix}
            1 & 2 & 3 \\
            1 & 3 & 2
          \end{pmatrix}
        }_{
          \cycle{ 2, 3 }
        }
    }
  \end{equation*}

  We can observe the following:
  \begin{itemize}
    \item The permutations \( \cycle{ 1, 2, 3 } \) and \( \cycle{ 1, 3, 2 } \) are inverses of each other and all other permutations are involutions.

    \item Every conjugation automorphism is unique. This can be verified explicitly. Therefore, the \hyperref[def:inner_and_outer_automorphisms]{inner automorphism group} \( \op{inn}(S_3) \) is isomorphic to \( S_3 \).

    \item The \hyperref[def:alternating_group]{alternating group} \( A_3 \) consists of the identity and the odd-length cycles \( (1, 2, 3) \) and \( (1, 3, 2) \).

    \item When restricted to \( A_3 \), all conjugation automorphisms are trivial. This can be verified explicitly. Therefore, the inner automorphism group \( \op{inn}(A_3) \) is trivial, and hence
    \begin{equation*}
      \aut(A_3) \cong \op{out}(A_3).
    \end{equation*}

    \item The map \( p \mapsto p^{-1} \), which fixes the identity and exchanges the two other permutations, is an automorphism of \( A_3 \). It is distinct from the identity, hence it is an outer automorphism.

    This map is given by the restriction of the conjugation \( p \mapsto \cycle{1, 2} p \cycle{1, 2} \) to \( A_3 \). It is an inner automorphism of \( S_3 \), but an outer automorphism of \( A_3 \).
  \end{itemize}
\end{example}

\begin{proposition}\label{thm:group_epimorphisms_are_surjective}\mcite[exer. I.5.5]{MacLane1994}
  Every \hyperref[def:morphism_invertibility/right_cancellative]{epimorphism} in \hyperref[def:group/category]{\( \cat{Grp} \)} is \hyperref[def:function_invertibility/surjective]{surjective}.
\end{proposition}
\begin{proof}
  Let \( \varphi: G \to H \) be an epimorphism and suppose that it is not surjective. Let \( M \) be the smallest normal subgroup of \( H \) containing \( \img \varphi \).

  If \( M \) has index \( 2 \) in \( H \), consider the quotient map \( \pi: H \to H / M \) and the constant map \( c(h) \coloneqq M \). Then
  \begin{equation*}
    \pi \bincirc \varphi = c \bincirc \varphi.
  \end{equation*}

  Since \( \varphi \) is an epimorphism, we have \( \pi = c \). But we have deliberately taken \( \pi \) and \( c \) so that \( \pi \neq c \). The obtained contradiction shows that \( M \) must have an index greater than \( 2 \).

  Let \( M \), \( uM \) and \( vM \) be different cosets. Define \( \sigma: H \to H \) as the \hyperref[def:symmetric_group/permutation]{permutation} on \( H \) that exchanges \( xu \) with \( xv \) for every \( x \in M \). Define the homomorphism
  \begin{equation*}
    \begin{aligned}
      &\psi: H \to S(H) \\
      &\psi(h) \coloneqq (x \mapsto hx),
    \end{aligned}
  \end{equation*}
  where \( S(H) \) is the \hyperref[def:symmetric_group]{symmetric group}.

  This is indeed a homomorphism by \fullref{thm:cayleys_theorem}. By \fullref{thm:group_conjugation_action}, another homomorphism is
  \begin{equation*}
    \begin{aligned}
      &\theta: H \to S(H) \\
      &\theta(h) \coloneqq \sigma^{-1} \bincirc \psi(h) \bincirc \sigma.
    \end{aligned}
  \end{equation*}

  Since \( \sigma \) fixes the members of \( M \) in-place, we have \( \theta(h)\restr_M = \psi(h)\restr_M \). Since \( M \) contains the image of \( \varphi \), this implies
  \begin{equation*}
    \psi \bincirc \varphi = \theta \bincirc \varphi.
  \end{equation*}

  Since \( \varphi \) is an epimorphism, we have \( \psi = \theta \). But we have deliberately constructed \( \psi \) and \( \theta \) such that \( \psi \neq \theta \). The obtained contradiction shows that \( \img \varphi \) cannot be a strict subgroup of \( G \). Therefore, \( \varphi \) must be surjective.
\end{proof}

\begin{definition}\label{def:dynamical_system}\mimprovised
  Suppose that \( \cat{C} \) is a \hyperref[def:concrete_category]{concrete category} and let \( X \) be an object of \( \cat{C} \).

  A \term{dynamical system} is a \hyperref[def:monoid_action]{monoid action} \( \Phi: T \times X \to X \). We call \( X \) the \term{phase space} of the system. In applications, we interpret the monoid \( T \) as \term{time} and consider it to be \hyperref[rem:additive_magma]{additive}. We call \( \Phi \) the \term{evolution function} of the system.

  \begin{thmenum}
    \thmitem{def:dynamical_system/ifs} If \( T \) is either the additive monoid of the \hyperref[def:set_of_natural_numbers]{natural numbers} or the additive group of the \hyperref[def:set_of_integers]{integers}, we say that the dynamical system has \term{discrete time}.

    Due to \fullref{eq:def:monoid_action/family/compatibility}, \( \Phi_{n+1} = \Phi_n \circ \Phi_1 \) for any integer \( n \). Using \hyperref[rem:induction/peano_arithmetic]{natural number induction} and \fullref{thm:def:group/negative_power}, we can show that \( \Phi_n = \Phi_1^n \) for every integer \( n \).

    Therefore, the entire evolution function of a discrete-time dynamical system is determined by a single function \( \varphi: X \to X \). For this reason, we also refer to discrete-time dynamical systems as \term{iterated function systems}.

    \thmitem{def:dynamical_system/semiflow} If \( T \) is the additive monoid of \hyperref[def:set_of_real_numbers]{real numbers}, with or without \hyperref[def:extended_real_numbers]{an infinite element}, we say that the system is a \term{semiflow}.

    \thmitem{def:dynamical_system/flow} If \( T \) is the additive group of \hyperref[def:set_of_real_numbers]{real numbers}, we say that the system is a \term{flow}.
  \end{thmenum}

  We will call the system \term{discrete} if \( T \) is the monoid of zero-based \hyperref[def:set_of_natural_numbers]{natural numbers} and \term{continuous} if \( T \) is the monoid of nonnegative \hyperref[def:set_of_real_numbers]{real numbers}, with or without \hyperref[def:extended_real_numbers]{an infinite element}.

  The monoid \( T \) can theoretically be a \hyperref[def:group]{group}, in which case we consider \hyperref[def:group_action]{group actions}, however negative time is not as often needed in practice.
\end{definition}

\begin{definition}\label{def:dynamical_system_trajectory}
  Fix a \hyperref[def:dynamical_system]{dynamical system} with evolution function \( \Phi: T \times X \to X \).

  A \term{trajectory} in a starting at the \term{initial state} \( x_0 \in X \) is an \hyperref[def:cartesian_product/indexed_family]{indexed family} \( \seq{ x_t }_{t \in T} \) obtained as
  \begin{equation*}
    x_t \coloneqq \Phi_t(x_0).
  \end{equation*}

  The condition \ref{eq:def:monoid_action/family/identity} ensures that \( \Phi_0(x_0) = x_0 \), and \ref{eq:def:monoid_action/family/compatibility} ensures that
  \begin{equation*}
    x_{t + s}
    =
    \Phi_{t + s}(x_0)
    =
    \Phi_t(x_s).
  \end{equation*}

  For discrete dynamical systems, trajectories are sequences.
\end{definition}

\subsection{Category of groups}\label{subsec:category_of_groups}

\begin{definition}\label{def:group_direct_product}
  The \term{direct product} of a nonempty family of groups \( \{ \mscrX_k \}_{k \in \mscrK} \) is their \hyperref[def:cartesian_product]{Cartesian product} \( \prod_{k \in \mscrK} \mscrX_k \) with the componentwise group operation
  \begin{equation*}
    \{ x_k \}_{k \in \mscrK} \cdot \{ y_k \}_{k \in \mscrK}
    \coloneqq
    \{ x_k \cdot y_k \}_{k \in \mscrK}.
  \end{equation*}
\end{definition}

\begin{proposition}\label{thm:product_of_cyclic_groups}
  The \hyperref[def:group_direct_product]{direct product} \( C_n \times C_m \) of two \hyperref[def:cyclic_group]{cyclic groups} is cyclic if and only if \( n \) and \( m \) are \hyperref[def:coprime_numbers]{coprime}.
\end{proposition}
\begin{proof}
  Take \( (a^i, a^j) \in C_n \times C_m \).
\end{proof}

\begin{definition}\label{def:group_direct_sum}
  The \term{direct sum} \( \bigoplus_{k \in \mscrK} \mscrX_k \) of a nonempty family of groups \( \{ \mscrX_k \}_{k \in \mscrK} \) is a subgroup of their \hyperref[def:group_direct_sum]{direct product} where, for any group element \( \{ x_k \}_{k \in \mscrK} \), only finitely many components are different from zero.

  \begin{defenum}
    \ilabel{def:group_direct_sum/internal}\mcite[126]{Knapp2016BAlg}If all \( \{ \mscrX_k \}_{k \in \mscrK} \) are subgroups of a group \( \mscrX \), we say that \( \mscrX \) is their \term{internal direct sum} if the homomorphism
    \begin{align*}
       &\varphi: \bigoplus_{k \in \mscrK} \mscrX_k \to \mscrX \\
       &\varphi(\{ x_k \}_{k \in \mscrK}) \coloneqq \cdot_{k \in \mscrK} x_k
    \end{align*}
    is an isomorphism.

    The sum is well-defined since, by definition, there are only finitely many non-identity summands.

    \ilabel{def:group_direct_sum/external} To distinguish \( \bigoplus_{k \in \mscrK} \mscrX_k \) from \( X \), we sometimes call it the \term{external direct product}.
  \end{defenum}
\end{definition}

\begin{proposition}\label{thm:group_categorical_limits}
  We are interested in \hyperref[def:categorical_limit]{categorical limits} and \hyperref[def:categorical_colimit]{colimits} in \( \cat{Grp} \). Fix an indexed family  \( \{ \mscrX_k \}_{k \in \mscrK} \) of groups.

  \begin{defenum}
    \ilabel{thm:group_categorical_limits/product} Their \hyperref[def:categorical_product]{categorical product} is their \hyperref[def:group_direct_product]{direct product} \( \prod_{k \in \mscrK} \mscrX_k \), the projection morphisms being inherited from \fullref{thm:set_categorical_limits/product}.

    \ilabel{thm:group_categorical_limits/coproduct} Their \hyperref[def:categorical_coproduct]{categorical coproduct} is their \hyperref[def:group_free_product]{free product} \( \Ast_{k \in \mscrK} \mscrX_k \), the embedding morphisms being
    \begin{balign*}
       &\iota_m: \mscrX_m \to \Ast_{k \in \mscrK} \mscrX_k \\
       &\iota_m(x_m) \coloneqq x_m.
    \end{balign*}
  \end{defenum}
\end{proposition}

\subsection{Abelian groups}\label{subsec:abelian_groups}

\begin{definition}\label{def:abelian_group}
  A \hyperref[def:magma/commutative]{commutative} \hyperref[def:group]{group} is usually called an \term{abelian group}. It is conventional to use \hyperref[rem:additive_magma]{additive notation} for abelian groups.

  We denote by \( \cat{Ab} \) the category of abelian groups.
\end{definition}

\begin{proposition}\label{thm:abelian_normal_subgroups}
  All subgroups of an abelian group are \hyperref[def:normal_subgroup]{normal}.
\end{proposition}
\begin{proof}
  Let \( \mscrG \) be abelian and \( \mscrH \) be a subgroup of \( \mscrG \). Then \( x \mscrG x^{-1} = xx^{-1} \mscrH = \mscrH \) for any \( x \in G \) and thus \( \mscrH \) is normal.
\end{proof}

\begin{definition}\label{def:group_of_integers_modulo}
  The \hyperref[def:set_of_integers]{integers} \( \BbbZ \) notoriously form an abelian group under addition. Fix a positive integer \( n \). We define the group
  \begin{equation*}
    \BbbZ_n \coloneqq \{ 0, 1, \ldots, n - 1 \}
  \end{equation*}
  with the operation
  \begin{equation*}
    x \oplus y \coloneqq \rem(x + y, n)
  \end{equation*}
  so that
  \begin{equation*}
    x \oplus y \cong x + y \pmod n.
  \end{equation*}

  The group \( \BbbZ_n \) is called the \term{group of integers modulo} \( n \).
\end{definition}
\begin{proof}
  We will prove that \( \BbbZ_n \) is an abelian group.

  \SubProofOf{def:magma/associative} Addition in \( \BbbZ_n \) is associative since
  \begin{balign*}
    (x \oplus y) \oplus z
    &=
    \rem((x \oplus y) + z, n)
    = \\ &=
    \rem(\rem(x + y, n) + z, n)
    = \\ &=
    \rem(x + y - n \quot(x + y, n) + z, n)
    = \\ &=
    \rem(x + y + z, n)
    = \\ &=
    \ldots
    = \\ &=
    x \oplus (y \oplus z).
  \end{balign*}

  \SubProofOf{def:unital_magma} The zero is obviously the identity.

  \SubProofOf{def:unital_magma_inverse_element} Fix \( x \in \BbbZ_n \). If \( x = 0 \), its inverse is \( 0 \). If \( x > 0 \), its inverse is \( n - x \) since \( n - x \in \BbbZ_n \) and
  \begin{equation*}
    x \oplus (n - x) = x + (n - x) - n = 0.
  \end{equation*}

  \SubProofOf{def:magma/commutative} Commutativity follows from
  \begin{equation*}
    x \oplus y
    =
    \rem(x + y, n)
    =
    \rem(y + x, n)
    =
    y \oplus x.
  \end{equation*}
\end{proof}

\begin{proposition}\label{thm:integers_modulo_isomorphic_to_quotient_group}
  The group \( \BbbZ_n \) of \hyperref[def:group_of_integers_modulo]{integers modulo \( n \)} is isomorphic to the quotient of \( \BbbZ \) by \( n\BbbZ = \{ nz : z \in \BbbZ \} \), i.e.
  \begin{equation*}
    \BbbZ_n \cong \BbbZ / n\BbbZ.
  \end{equation*}
\end{proposition}
\begin{proof}
  Define the function
  \begin{align*}
    &\varphi: \BbbZ_n \to \BbbZ / n\BbbZ  \\
    &\varphi(x) \coloneqq x + n\BbbZ.
  \end{align*}

  It is a homomorphism because
  \begin{balign*}
    \varphi(x \oplus y)
    &=
    \varphi(\rem(x + y, n))
    = \\ &=
    \varphi(x + y - n \quot(x + y, n))
    = \\ &=
    x + y - n \quot(x + y, n) + n\BbbZ
    = \\ &=
    x + y + n\BbbZ
    = \\ &=
    (x + n\BbbZ) + (y + n\BbbZ)
    = \\ &=
    \varphi(x) + \varphi(y).
  \end{balign*}

  Furthermore, this shows that \( \varphi \) is also an isomorphism.
\end{proof}

\begin{example}\label{ex:lagranges_theorem_for_groups/direct_product_zn}
  \Fullref{thm:lagranges_theorem_for_groups} and \fullref{thm:integers_modulo_isomorphic_to_quotient_group} imply that, for any positive integer \( n \), there exists a bijection between \( n \BbbZ \times \BbbZ_n \) and \( \BbbZ \).

  This bijection, however, is not necessarily a group isomorphism because \eqref{thm:integers_modulo_isomorphic_to_quotient_group} may not hold.

  Let \( f: \BbbZ \to n \BbbZ \times \BbbZ_n \) be a bijection. Then, for every integer \( k \) there exist integers \( m_k \in \BbbZ \) and \( r_k \in \BbbZ_n \) such that
  \begin{equation*}
    f(k) = (n m_k, r_k).
  \end{equation*}

  For \( (mn, p) \in  \)
\end{example}

\begin{proposition}\label{thm:cyclic_group_isomorphic_to_integers_modulo_n}
  Let \( C \) be a cyclic \hyperref[def:cyclic_group]{group}. If \( C \) is finite of order \( n \), it is isomorphic to the group \( \BbbZ_n \) of integers modulo \( n \) (see \fullref{def:group_of_integers_modulo}).
\end{proposition}
\begin{proof}
  The homomorphism
  \begin{balign*}
     &\varphi: \BbbZ_n \to C_n \\
     &\varphi(k) \coloneqq a^k
  \end{balign*}
  and the analogous homomorphism for the infinite group, is an isomorphism.
\end{proof}

\begin{proposition}\label{thm:abelian_group_categorical_limits}
  We are interested in \hyperref[def:category_of_cones/limit]{categorical limits} and \hyperref[def:category_of_cones/colimit]{colimits} in the category \( \cat{Ab} \). Fix an indexed family  \( \{ \mscrX_k \}_{k \in \mscrK} \) of abelian groups.
  \begin{thmenum}
    \thmitem{thm:abelian_group_categorical_limits/product} Their \hyperref[def:discrete_category_limits]{categorical product} is the direct product as inherited from \fullref{thm:group_categorical_limits}.

    \thmitem{thm:abelian_group_categorical_limits/coproduct} Their \hyperref[def:discrete_category_limits]{categorical coproduct} is the \hyperref[def:group_direct_product]{direct sum} \( \oplus_{k \in \mscrK} \mscrX_k \), the embedding morphisms being
    \begin{align*}
       &\iota_m: \mscrX_m \to \oplus_{k \in \mscrK} \mscrX_k \\
       &\iota_m(x_m) \coloneqq \begin{dcases}
        \begin{drcases}
          x_m, &k = m \\
          0_k, &k \neq m
        \end{drcases}
      \end{dcases}_{k \in \mscrK}.
    \end{align*}
  \end{thmenum}
\end{proposition}

\begin{proposition}\label{thm:monoid_completion_to_abelian_group}\mcite{nLab:grothendieck_group_of_a_commutative_monoid}
  Every \hyperref[def:magma/commutative]{commutative} \hyperref[def:unital_magma/monoid]{monoid} can be \hyperref[def:first_order_homomorphism_invertibility/embedding]{embedded} into an abelian group using the \term{Grothendieck completion} presented in the proof.
\end{proposition}
\begin{proof}
  Let \( M \) be a commutative monoid. Define the relation \( \cong \) on tuples of members of \( M \) as
  \begin{equation*}
    (x_1, x_2) \cong (y_1, y_2) \iff \exists a: x_1 + y_2 + a = y_1 + x_2 + a.
  \end{equation*}

  This is an equivalence relation because
  \SubProofOf{def:binary_relation/reflexive}
  \begin{equation*}
    (x_1, x_2) \cong (x_1, x_2) \iff x_1 + x_2 + 0 = x_1 + x_2 + 0
  \end{equation*}

  \SubProofOf{def:binary_relation/symmetric} By commutativity,
  \begin{balign*}
    (x_1, x_2) \cong (y_1, y_2)
     & \iff
    \exists a: x_1 + y_2 + a = y_1 + x_2 + a
    \\ &\iff
    \exists a: y_1 + x_2 + a = x_1 + y_2 + a
    \\ &\iff
    (y_1, y_2) \cong (x_1, x_2)
  \end{balign*}

  \SubProofOf{def:binary_relation/transitive} Let \( (x_1, x_2) \cong (y_1, y_2) \) and \( (y_1, y_2) \cong (z_1, z_2) \). Thus, there exist \( a, b \in \BbbN \) such that
  \begin{equation*}
    [x_1 + y_2 + a = y_1 + x_2 + a] \T{and} [y_1 + z_2 + b = z_1 + y_2 + b]
  \end{equation*}

  Summing both sides, we have
  \begin{equation*}
    x_1 + y_2 + a + y_1 + z_2 + b = y_1 + x_2 + a + z_1 + y_2 + b
  \end{equation*}

  We reorder both sides to obtain
  \begin{equation*}
    (x_1 + z_2) + (y_1 + y_2 + a + b) = (x_2 + z_1) + (y_1 + y_2 + a + b),
  \end{equation*}
  which implies \( (x_1, x_2) \cong (z_1, z_2) \).

  Define \( G \coloneqq M^2 / \cong \) to be the equivalence partition\fullref{thm:equivalence_partition} of \( M \times M \). Define addition in \( G \) on members of \( M \times M \) by
  \begin{equation*}
    (x_1, x_2) + (y_1, y_2)
    \coloneqq
    (x_1 + y_1, x_2 + y_2).
  \end{equation*}

  This addition does not depend on the representative of the equivalence class since \( (x_1, x_2) \cong (x_1', x_2') \) and \( (y_1, y_2) \cong (y_1', y_2') \) implies the existence of \( k, m \in \BbbN \), such that
  \begin{balign*}
    x_1 + x_2' + a&= x_2 + x_1' + a,
    y_1 + y_2' + b&= y_2 + y_1' + b,
  \end{balign*}
  which, when combined, give
  \begin{balign*}
    (x_1 + x_2' + a) + (y_1 + y_2' + b)
    &=
    (x_2 + x_1' + a) + (y_2 + y_1' + b)
    \\
    (x_1 + y_1) + (x_2' + y_2') + (a + b)
    &=
    (x_2 + y_2) + (y_1 + x_1) + (a + b).
  \end{balign*}

  This implies
  \begin{balign*}
    (x_1 + y_1, x_2 + y_2)
    \cong
    (x_1' + y_1', x_2' + y_2').
  \end{balign*}

  The equivalence class \( [(0, 0)] \) is obviously an identity in \( G \) and contains exactly the pairs \( (x, x) \) of identical elements.

  For each member \( (x_1, x_2) \in M \times M \) we define its inverse as \( (x_2, x_1) \). It is indeed an inverse since
  \begin{equation*}
    (x_1, x_2) + (x_2, x_1) = (x_1 + x_2, x_2 + x_1),
  \end{equation*}
  which, by commutativity, belongs to \( [(0, 0)] \).

  If \( (x_1, x_2) \cong (x_1', x_2') \), then
  \begin{equation*}
    (x_1, x_2) + (x_2', x_1')
    =
    (x_1 + x_2', x_2 + x_1'),
  \end{equation*}
  where the two representatives of a pair of inverses are equal because of the equivalence \( \cong \).

  Thus, \( + \) is a well-defined commutative operation on \( G \) with identity, making it an abelian group.

  Furthermore, the function
  \begin{balign*}
     & \varphi: M \to G              \\
     & \varphi(x) \coloneqq [(x, 0)]
  \end{balign*}
  is a monoid homomorphism, hence \( M \) is indeed embedded in the group. Furthermore, any group that embeds \( G \) must also embed \( M \) since \( G \setminus \varphi(M) \) consists only of the \enquote{inverse} elements of \( \varphi(M) \).
\end{proof}

\begin{definition}\label{def:group_commutator}
  Let \( G \) be any group. The commutator of \( x, y \in G \) is defined as
  \begin{equation*}
    [x, y] \coloneqq xyx^{-1}y^{-1}.
  \end{equation*}

  The commutator subgroup of \( G \) is the subgroup \hyperref[def:group_presentation]{generated} by all the commutators in \( G \).
\end{definition}

\begin{proposition}\label{thm:quotient_by_commutator_subgroup}\mcite[prop. 7.4]{Knapp2016BasicAlgebra}
  The commutator group \( G' \) of any group \( G \) is \hyperref[def:normal_subgroup]{normal} and the quotient \( G / G' \) is \hyperref[def:abelian_group]{abelian}.
\end{proposition}


% Ring theory
\section{Ring theory}\label{sec:ring_theory}
\subsection{Rings}\label{subsec:rings}

\begin{remark}\label{remark:rings}
  As for groups (see \fullref{remark:additive_group}), commutative and non-commutative rings are quite different despite having similar definitions.

  Rather than regarding rings as standalone algebraic structures, it is often convenient to regard rings as abelian groups with an additional second operation that extends multiplication by integers as defined in \fullref{def:magma_exponentiation} to arbitrary elements of the ring.

  For noncommutative ring, this second operation is usually given by function composition and for commutative rings, this operation is often truly an extension of \fullref{def:magma_exponentiation} to arbitrary ring elements.
\end{remark}

\begin{definition}\label{def:semiring}
  A \Def{semiring} \( (R, +, \cdot) \) is an algebraic \hyperref[def:algebraic_theory]{structure} with two binary operations:
  \begin{DefEnum}[series=def:semiring]
    \ILabel{def:semiring/addition} The \Def{addition} \( + \) is \hyperref[def:magma/associative]{associative}, \hyperref[def:unital_magma]{unital} and \hyperref[def:magma/commutative]{commutative}, i.e. \( (R, +) \) is a commutative \hyperref[def:unital_magma/associative]{monoid}. The additive identity is usually denoted by \( 0 \). We denote this monoid by \( R^+ \).

    \ILabel{def:semiring/multiplication} The \Def{multiplication} \( \cdot \) (usually written using juxtaposition) is \hyperref[def:magma/associative]{associative}, i.e. \( (R, \cdot) \) a \hyperref[def:magma/associative]{semigroup}. We denote this semigroup by \( R^\times \).

    \ILabel{def:semiring/distributivity} We require that \( \cdot \) \Def{distributes} over \( + \), i.e.
    \begin{align}\label{eq:def:semiring/distributivity}
      (x + y)z = xz + yz
      &&
      x(y + z) = xy + xz
    \end{align}
  \end{DefEnum}

  The \Def{trivial semiring} consists only of the additive identity \( \{ 0 \} \).

  We say that \( x \) is \Def{nilpotent} if \( x^n = 0 \) for some nonnegative integer \( n \).

  The following are special kinds of semirings:
  \begin{DefEnum}[resume=def:semiring]
    \ILabel{def:semiring/dioid} \Def{Dioids} are unital semirings, that is, both \( (R, +) \) and \( (R, \cdot) \) are monoids. The term \enquote{semiring} is sometimes reserved for dioids while \enquote{hemiring} is used for what we call semirings.

    We define integer exponentiation as a shorthand for iterated multiplication. That is, if multiplication is invertible, for \( r \in R \) and \( n \in \BZ \) we define
    \begin{equation*}
      r^n \coloneqq \begin{cases}
        1,                      & n = 0  \\
        r^{n - 1} \cdot r,      & n > 0  \\
        r^{n + 1} \cdot r^{-1}, & n < 0.
      \end{cases}
    \end{equation*}

    If multiplication is not invertible, we skip defining negative exponents.

    \ILabel{def:semiring/no_zero_divisor} If \( xy = 0 \) whenever \( x \) and \( y \) are nonzero, we say that the semiring has \Def{no zero divisors}.

    \ILabel{def:semiring/ring} \Def{Rings} are semirings with invertible addition, i.e. \( (R, +) \) forms an \hyperref[def:abelian_group]{abelian group}. The category of rings is denoted by \( \Cat{Ring} \).

    \ILabel{def:semiring/unital_ring} \Def{Unital rings} are \hyperref[def:semiring/ring]{rings} in which multiplication is unital, i.e. \( R, \cdot \) is a \hyperref[def:unital_magma/associative]{monoid} with identity \( 1 \). Some authors define all rings to be unital. Invertible \hyperref[def:algebraic_theory/invertibile_element]{elements} under multiplication are called \Def{units} (see \fullref{remark:units_in_rings_etymology}) and the operation itself is called \Def{division}.

    We may additionally require that the ring is nontrivial so that \( 0 \neq 1 \) (see \fullref{thm:semiring_properties/identities_are_equal_iff_trivial_ring}).

    \ILabel{def:semiring/commutative_ring} \Def{Commutative rings} are \hyperref[def:semiring/ring]{rings} in which multiplication is \hyperref[def:magma/commutative]{commutative}, i.e. \( R, \cdot \) is a commutative semigroup.

    \ILabel{def:semiring/commutative_unital_ring} \Def{Commutative unital rings} are (obviously) both commutative and unital and as usually assumed to be nontrivial. Despite being ubiquitous, they do not have an established one-word name.

    We are mostly interested in these rings because if \( R \) is a commutative unital ring, so is its polynomial \hyperref[def:algebra_of_polynomials]{ring} \( R[X] \) and, by induction\IND, multivariate polynomial \hyperref[def:multivariate_polynomial]{rings} (\( R[X, Y] = R[X][Y] \)). So we are interested in properties of \( R \) that are preserves by \( R[X] \). This is the reason so many types of commutative rings and ideals are studied. We are interested in integral domains when we speak of divisibility and factorization and since zero is absorbing, it does not interact nicely with factorization. Refer to \fullref{sec:commutative_algebra}.

    \ILabel{def:semiring/integral_domain} \Def{Integral domains} are nontrivial commutative unital \hyperref[def:semiring/commutative_unital_ring]{rings} with no zero \hyperref[def:commutative_ring_division]{divisors}. This implies that \( 1 \neq 0 \) because otherwise \( 1 \cdot 1 = 0 \) and \( 1 \) would be a zero divisor. This in turn implies that \fullref{thm:semiring_properties/identities_are_equal_iff_trivial_ring}.

    \Fullref{thm:semiring_properties/cancellable_iff_not_zero_divisor} shows that a commutative unital ring \( R \) is an integral domain if and only if its multiplication is cancellable.

    \ILabel{def:semiring/unique_factorization_domain} \Def{Unique factorization domains} are integral domains in which every element has unique \hyperref[def:factorization_in_ring]{factorization} exists.

    \ILabel{def:semiring/principal_ideal_domain} \Def{Principal ideal domains} are integral \hyperref[def:semiring/integral_domain]{domains} in which every \hyperref[def:semiring_ideal]{ideal} is \hyperref[def:principal_ideal]{principal}.

    By \fullref{thm:pid_is_ufd}, every principal ideal domain is a unique factorization domain.

    \ILabel{def:semiring/euclidean_domain} \Def{Euclidean domains} are integral \hyperref[def:semiring/integral_domain]{domains} which allow division with remainders (see \fullref{def:euclidean_domain}).

    By \fullref{thm:euclidean_domain_is_pid}, every Euclidean domain is a principal ideal domain.

    \ILabel{def:semiring/division_ring} \Def{Division rings} are nontrivial unital \hyperref[def:semiring/unital_ring]{rings} in which all nonzero elements are units, i.e. \( (F \setminus \{ 0 \}, \cdot) \) is a \hyperref[def:group]{group}. The nontriviality is a requirement because we want \( 1 \) not to be a zero divisor (see discussion in \fullref{def:semiring/integral_domain}).

    The multiplicative inverse of an element in a division ring is called its \Def{reciprocal}.

    In order to fit multiplicative invertibility as an axiom for \fullref{def:algebraic_theory}, we can use the following formula:
    \begin{equation*}
      \forall \xi ((\xi \doteq 0) \lor \exists \eta (\xi \cdot \eta \doteq 1))
    \end{equation*}
    or add an additional operation \( (\cdot)^{-1} \) that inverts all nonzero elements and fixes zero, that is,
    \begin{equation*}
      \forall \xi (\xi \cdot \xi^{-1} \doteq 1),
    \end{equation*}
    where we define \( 0^{-1} = 0 \). This is only a formalism since \( 0 \) is not actually \enquote{invertible}, but it is required if we wish to avoid existential quantifiers.

    \ILabel{def:semiring/field} \Def{Fields} are \hyperref[def:magma/commutative]{commutative} division \hyperref[def:semiring/division_ring]{rings}, i.e. \( (F \setminus \{ 0 \}, \cdot) \) is an \hyperref[def:abelian_group]{abelian group}. The category of fields is denoted by \( \Cat{Field} \).
  \end{DefEnum}
\end{definition}

\begin{example}\label{ex:semirings}
  We give examples and counterexamples of semirings. Note that the order of definitions in \fullref{def:semiring} is not preserved.

  \begin{RefList}
    \IRef{def:semiring/euclidean_domain} The base building block for the examples will be the ring \( (\BZ, +, \cdot) \) of \hyperref[def:integers]{integers}, which itself is an Euclidean domain (see \fullref{def:integers}).

    Another example of Euclidean domains are the polynomial rings over a field (see \fullref{thm:polynomials_over_field_are_euclidean_domain}).

    \IRef{def:semiring/dioid} By removing additive inverses from the integers, we obtain the dioid \( (\BN, +, \cdot) \) of \hyperref[def:natural_numbers]{natural numbers}.

    Take only the commutative monoid \( (\BN, +) \) of \hyperref[def:natural_numbers]{natural numbers} with addition. The endomorphism semiring
    \begin{equation*}
      \End(\BN)
    \end{equation*}
    is a noncommutative dioid.

    Tropical \hyperref[def:tropical_semiring]{semirings} are another example noncommutative dioids.

    \IRef{def:semiring} Simple examples of semirings (but not rings) without unity are proper semiring ideals. For example, the ideal \( 2\BN \) of positive even numbers is a non-unital semiring.

    Another example can be given by taking a subsemiring but not a unital subsemiring of the endomorphism dioid \( \End(\BN, \oplus) \), for example the functions
    \begin{BreakableAlign*}
       & f_n: \BN \to \BN        \\
       & f_n(x) \coloneqq x + n,
    \end{BreakableAlign*}
    where \( n > 0 \). They are closed under composition and thus form a semiring themselves, but they do not contain the identity function so the semiring is not unital.

    \IRef{def:semiring/unital_ring} The endomorphism rings \( \End(G) \) for any abelian group \( G \) are unital but non-commutative rings. This includes the matrix space \( R^{n \times n} \) (see \fullref{thm:finite_dimensional_operators_are_isomorphic_to_matrices}).

    \IRef{def:semiring/ring} Consider the Banach space \( C_0(\BC) \) of complex functions vanishing at \hyperref[def:function_spaces/c0]{infinity}. If we take addition to be pointwise addition and multiplication to be composition, then \( C_0(\Co) \) becomes a non-commutative ring with no multiplicative identity because \( C_0(\Co) \) does not contain the identity function.

    \IRef{def:semiring/commutative_ring} For an example of a commutative ring without unit, consider again the Banach space \( C_0(\Co) \), however define multiplication as pointwise function multiplication rather than by composition. The constant function \( f(x) = 1 \) does not vanish at infinity, hence \( (C_0(\Co), +, \cdot) \) is a commutative but not unital ring.

    \IRef{def:semiring/commutative_unital_ring} We are mostly interested in different types of commutative unital rings since the polynomials over well-behaved commutative rings preserve this behavior. This is important because multivariate \hyperref[def:multivariate_polynomial]{polynomials} are defined inductively as polynomials over polynomial rings. See \fullref{thm:geometric_nullstellensatz} for an application.

    An example of a nontrivial commutative unital ring that has zero divisors is the matrix algebra \( \BZ^{n \times n} \) over the integers. It is a ring under addition and matrix multiplication. We have
    \begin{equation*}
      \begin{pmatrix}
        1 & 0 \\
        1 & 0
      \end{pmatrix}
      \begin{pmatrix}
        0 & 0 \\
        0 & 1
      \end{pmatrix}
      =
      \begin{pmatrix}
        0 & 0 \\
        0 & 0
      \end{pmatrix},
    \end{equation*}
    thus there are zero divisors in \( \BZ^{n \times n} \).

    \IRef{def:semiring/integral_domain}\cite[388]{Knapp2016BAlg} The integral domain \( \BZ[\sqrt{-5}] \) is not a unique factorization domain because
    \begin{equation*}
      6 = (1 + \sqrt{-5}) (1 - \sqrt{-5}) = 2 \cdot 3.
    \end{equation*}

    Note that \( \BZ[\sqrt{-5}] \) is an integral domain by \fullref{thm:polynomials_over_integral_domain_are_integral_domain}.

    \IRef{def:semiring/unique_factorization_domain}\cite{ProofWiki:polynomials_in_integers_is_not_principal_ideal_domain} The unique factorization domain \( \BZ[X] \) is not a principal ideal domain.

    Note that \( \BZ[X] \) is a unique factorization domain by \fullref{thm:polynomials_over_integral_domain_are_integral_domain}.

    Consider the ideal \( I \) of polynomials with an even constant term.

    Assume\LEM that \( I \) is generated by the polynomial \( p(X) \in \BZ[X] \). Since \( 2 \in I \), then \( p(X) \) divides \( 2 \) so \( p(X) \in \{ -2, -1, 1, 2 \} \), that is \( p(X) \) is a unit of \( \BZ[X] \). But then \( I = \Gen{p(X)} = \BZ[X] \), which contradicts the definition of \( I \).

    The obtained contradiction proves that \( \BZ[X] \) is not a principal ideal domain.

    \IRef{def:semiring/principal_ideal_domain} Principal ideal domains are not Euclidean domains in general. Such domains are discussed in \cite{Anderson1986}.

    \IRef{def:semiring/division_ring} The \hyperref[def:ring_localization]{localization} of a noncommutative ring over its nonzero cancellative elements (characterized by \fullref{thm:ring_localization_universal_property}), if it exists, forms a division ring.

    \IRef{def:semiring/field} The canonical examples of fields include the rational \hyperref[def:rational_numbers]{numbers} \( \BQ \), the \hyperref[def:real_numbers]{real numbers} \( \BR \) and the \hyperref[def:complex_numbers]{complex numbers} \( \BC \).

    More generally, any nontrivial commutative unital can be embedded in a field by \fullref{def:field_of_fractions}.
  \end{RefList}
\end{example}

\begin{proposition}\label{thm:semiring_properties}
  Any semiring \( R \) has the following basic properties:
  \begin{PropEnum}
    \ILabel{thm:semiring_properties/zero_is_absorbing} Multiplication by \( 0 \) is \hyperref[def:algebraic_theory/absorbing_element]{absorbing}, that is, \( x0 = 0x = 0 \) for any \( x \in R \).
    \ILabel{thm:semiring_properties/one_exponents} For any integer \( n \), \( 1^n = 1 \).
    \ILabel{thm:semiring_properties/identities_are_equal_iff_trivial_ring} In a unital \hyperref[def:semiring/unital_ring]{ring}, the additive and multiplicative identities are equal if and only if the ring is trivial.
    \ILabel{thm:semiring_properties/cancellable_iff_not_zero_divisor} An element \( x \in R \) of a commutative ring is a zero divisor if and only if it is cancellable (with respect to multiplication).
  \end{PropEnum}
\end{proposition}
\begin{proof}
  \SubProofOf{thm:semiring_properties/zero_is_absorbing} Follows from \fullref{thm:left_module_properties/ring_zero_is_absorbing} and \fullref{thm:left_module_properties/module_zero_is_absorbing}.
  \SubProofOf{thm:semiring_properties/one_exponents} By definition, \( r \cdot 1 = r \) for any \( r \in R \). Hence for \( 1 \cdot 1 = 1 \). Proceeding by induction\IND, we can show that for \( 1^n = 1 \) for positive \( n \). For negative \( n \), since \( 1 \) is its own inverse, we also have \( 1^n = 1 \).

  \SubProofOf{thm:semiring_properties/identities_are_equal_iff_trivial_ring}
  Let \( 0 = 1 \) in a unital ring \( R \). Let \( r \in R \). Then, by \fullref{thm:semiring_properties/zero_is_absorbing},
  \begin{equation*}
    r = 1r = 0r = 0.
  \end{equation*}

  Thus \( r = 0 \). Since \( r \) was arbitrary, we conclude that \( R = \{ 0 \} \) is the trivial ring.

  \SubProofOf{thm:semiring_properties/zero_is_absorbing} Follows from \fullref{thm:left_module_properties/ring_zero_is_absorbing} and \fullref{thm:left_module_properties/module_zero_is_absorbing}.

  \SubProofOf{thm:semiring_properties/one_exponents} By definition, \( r \cdot 1 = r \) for any \( r \in R \). Hence for \( 1 \cdot 1 = 1 \). Proceeding by induction\IND, we can show that for \( 1^n = 1 \) for positive \( n \). For negative \( n \), since \( 1 \) is its own inverse, we also have \( 1^n = 1 \).

  \SubProofOf{thm:semiring_properties/identities_are_equal_iff_trivial_ring} The trivial ring only has one element, hence \( 0 = 1 \).

  \SufficiencyOf{thm:semiring_properties/cancellable_iff_not_zero_divisor}
  Suppose that \( x \in R \) is not cancellable. Then there exist \( y \neq z \) for which \( xy = xz \). We have \( y - z \neq 0 \) and
  \begin{equation*}
    x(y - z) = xy - xz = 0.
  \end{equation*}

  Thus \( x \) is a zero divisor.

  \NecessityOf{thm:semiring_properties/cancellable_iff_not_zero_divisor} If \( x \) is a zero divisor, fix \( y \in R \) such that \( xy = 0 \). For any \( z \in R \) we have
  \begin{equation*}
    xy = 0 = x(yz)
  \end{equation*}
  but \( y \neq yz \) in general.

  Thus \( x \) is not cancellable.
\end{proof}

\begin{definition}\label{def:semiring_characteristic}
  Fix a nontrivial unital semiring \( R \) (the nontrivial condition is essential because of \fullref{thm:embedding_preserves_characteristic}). We define its \Def{characteristic} \( \Char(R) \) via any of the equivalent definitions:
  \begin{DefEnum}
    \ILabel{def:semiring_characteristic/direct} \( \Char(R) \) is the smallest number of times \( 1_R \) must be added to itself in order to obtain \( 0_R \) and zero if \( 0_R \) cannot be obtained in this way.

    \ILabel{def:semiring_characteristic/homomorphism} The semiring \( \BZ_{\geq 0} \) of nonnegative integers can be embedded into \( R \) via the unique homomorphism
    \begin{BreakableAlign*}
       & \iota: \BZ_{\geq 0} \to R                     \\
       & \iota(k) \coloneqq \begin{cases}
        0_R,                & k = 0 \\
        \iota(k - 1) + 1_R, & k > 1
      \end{cases}
    \end{BreakableAlign*}
    that adds the identity \( 1_R \) to itself \( n \) times. This allows us to use the positive integers in any semiring. Note that if addition in \( R \) is invertible (that is, if \( R \) is a ring), we can embed all integers by defining \( \iota(n) \coloneqq -\iota(-n) \) for \( n < 0 \).

    We define \( \Char(R) \) as the positive integer \( n \) such that
    \begin{equation*}
      n\BZ_{\geq 0} \cong \ker\iota.
    \end{equation*}
  \end{DefEnum}
\end{definition}
\begin{proof}
  We will show that \( \iota \) is unique. Let \( \varphi: \BZ_{\geq 0} \to R \) be another embedding. Obviously
  \begin{equation*}
    \iota(1_{\BZ}) = 1_R = \varphi(1_{\BZ}).
  \end{equation*}

  Then by induction\IND on \( k \) we can show that
  \begin{equation*}
    \iota(k) = \underbrace{1_R + \cdots + 1_R}_{k \text{ times }} = \varphi(k).
  \end{equation*}

  Hence \( \iota = \varphi \).
\end{proof}

\begin{proposition}\label{thm:embedding_preserves_characteristic}
  If \( R \) is a nontrivial unital semiring and \( T \) is a unital supersemiring of \( R \), then
  \begin{equation*}
    \Char(T) = \Char(R).
  \end{equation*}
\end{proposition}
\begin{proof}
  Note that if \( R \) was the trivial unital semiring, its characteristic would be \( 1 \) and yet the characteristic of \( T \) could be nonzero. So the theorem would not hold if \( R \) was trivial.

  Now assume that \( R \) is nontrivial and let \( \iota: \BZ_{\geq 0} \to R \) be the embedding defining \( \Char(R) \). Then it is obviously also an embedding of \( \BZ_{\geq 0} \) into \( T \). Since \( 0_R = 0_T \), then \( \ker \iota = \ker \varphi = n\BZ_{\geq 0} \), where \( n = \Char(R) \).
\end{proof}

\begin{example}\label{ex:semiring_characteristic}
  We find examples of semiring characteristics via the embedding \( \iota \) defined in \fullref{def:semiring_characteristic}:

  \begin{ExEnum}
    \ILabel{ex:semiring_characteristic/nonnegative_integers} The zero-based \hyperref[def:natural_numbers]{natural numbers} \( \BN \) have characteristic \( \Char(\BN) = 0 \) because \( \iota \) is an isomorphism. Consequently, any supersemiring of \( \BN \) has characteristic zero, most notably the integers \( \BZ \) and the fields \( \BQ \), \( \BR \), \( \BC \).

    \ILabel{ex:semiring_characteristic/integers_modulo} The integers modulo \( n \) (see \fullref{def:ring_of_integers_modulo}) have characteristic \( \Char(\BZ_n) = n \) because of \fullref{thm:integers_modulo_isomorphic_to_quotient_group}.

    \ILabel{ex:semiring_characteristic/polynomial_ring} An \hyperref[def:algebra_over_ring]{algebra} \( A \) over a nontrivial commutative unital ring \( R \) has characteristic \( \Char(A) = R \) because of the canonical embedding of \( R \) in \( A \). In particular, polynomial \hyperref[def:algebra_of_polynomials]{rings} \( R[X] \) have the same characteristic as their ring.

    \ILabel{ex:semiring_characteristic/galois_fields} The \hyperref[thm:galois_field_existence]{Galois field} \( \BF_{p^n} \) has characteristic \( p \) because it is a field extension of \( \BF_p \).
  \end{ExEnum}
\end{example}

\begin{definition}\label{def:semiring_kernel}
  The \Def{kernel} \( \ker(f) \) of a semiring homomorphism \( f: R \to S \) is the zero \hyperref[def:zero_locus]{locus} of \( f \), that is, \hyperref[def:function/preimage]{preimage} \( f^{-1}(0_S) \).

  It is an instance of \fullref{def:categorical_kernel}.
\end{definition}

\begin{definition}\label{def:quotient_semiring}
  Let \( R \) be a ring and \( I \) be an ideal of \( M \). Define the \Def{quotient ring} to be the quotient \hyperref[def:quotient_left_module]{module} when considering \( R \) as a module over itself.
\end{definition}

\begin{theorem}\label{thm:homomorphism_theorem_for_rings}
  Let \( \varphi: R \to T \) be a homomorphism of rings. We have the isomorphism
  \begin{equation*}
    R / \ker \varphi \cong \Img \varphi.
  \end{equation*}
\end{theorem}
\begin{proof}
  Special case of \fullref{thm:homomorphism_theorem_for_left_modules}.
\end{proof}

\begin{proposition}\label{thm:ring_homomorphism_simpler_conditions}
  A function \( f: R \to S \) between the rings \( R \) and \( S \) is a homomorphism in the sense of \fullref{def:first_order_homomorphism} if and only if for any \( x, y \in R \) it satisfies
  \begin{equation}\label{thm:ring_homomorphism_simpler_conditions/condition}
    \begin{dcases}
      f(x + y) & = f(x) + f(y), \\
      f(xy)    & = f(x) f(y),   \\
      f(1_R)   & = 1_S.
    \end{dcases}
  \end{equation}

  Note that the last condition is only for unital rings.

  In other words, if a function satisfies \fullref{thm:ring_homomorphism_simpler_conditions/condition}, the following are automatically satisfied:
  \begin{itemize}
    \item \( f(0_R) = 0_S \)
    \item for all \( x \in R \), we have \( f(-x) = -f(x) \)
    \item for all units \( x \in R \), we have \( f(x^{-1}) = f(x)^{-1} \)
  \end{itemize}
\end{proposition}
\begin{proof}
  Since \( (R, +) \) and \( (S, +) \) are groups, the first two equalities from \fullref{thm:group_homomorphism_single_condition}.

  The proof of \( f(x^{-1}) = f(x)^{-1} \) is analogous to \fullref{thm:group_homomorphism_single_condition}.
\end{proof}

\begin{definition}\label{def:ring_of_integers_modulo}
  The \hyperref[def:integers]{integers} \( \BZ \) form a ring under addition and multiplication. Fix a positive integer \( n > 1 \). We extend the group \( \BZ_n \) of integers modulo \( n \) (see \fullref{def:group_of_integers_modulo}) with the operation
  \begin{equation*}
    x \odot y \coloneqq \Rem(xy, n).
  \end{equation*}

  The ring \( \BZ_n \) is called the \Def{ring of integers modulo} \( n \).
\end{definition}
\begin{proof}
  Note that
  \begin{BreakableAlign*}
     & \phantom{\equiv}\; \Rem(x, n) \Rem(y, n)
     & \pmod n \equiv                           \\ &\equiv
    (x - n \Quot(x, n)) (y - n \Quot(y, n))
     & \pmod n \equiv                           \\ &\equiv
    xy - n \Quot(x, n) - n \Quot(y, n) + n^2 \Quot(x, n) \Quot(y, n)
     & \pmod n \equiv                           \\ &\equiv
    xy
     & \pmod n. \phantom{\equiv}
  \end{BreakableAlign*}

  The proof that multiplication in \( \BZ_n \) is associative, unital and commutative becomes trivial.

  We will prove that multiplication distributes over addition. Fix \( x, y, z \in \BZ_n \). We have
  \begin{BreakableAlign*}
    (x \oplus y) \odot z
     & =
    \Rem((x \oplus y) z, n)
    =    \\ &=
    \Rem(\Rem(x + y, n) z, n)
    =    \\ &=
    \Rem((x + y - n \Quot(x + y, n)) z, n)
    =    \\ &=
    \Rem((x + y)z, n).
  \end{BreakableAlign*}
  and
  \begin{BreakableAlign*}
    (x \odot z) \oplus (y \odot z)
     & =
    \Rem([(x \odot z) + (y \odot z)], n)
    =    \\ &=
    \Rem([xz - n \Quot(xz, n) + yz - n \Quot(yz, n)], n)
    =    \\ &=
    \Rem(xz + yz, n)
    =    \\ &=
    \Rem((x + y)z, n).
  \end{BreakableAlign*}

  Hence
  \begin{equation*}
    (x \oplus y) \odot z = (x \odot z) \oplus (y \odot z).
  \end{equation*}
\end{proof}

\begin{definition}\label{def:tropical_semiring}\MarginCite{nLab:tropical_semiring}
  Fix a partially \hyperref[def:poset]{ordered} \hyperref[def:abelian_group]{abelian group} \( (M, +, \leq) \). Let \( \infty \) be a sentinel symbol not in \( M \). Define
  \begin{equation*}
    T \coloneqq M \cup \{ \infty \}
  \end{equation*}
  with operations
  \begin{BreakableAlign*}
     & \oplus: T \times T \to T                        \\
     & x \oplus y \coloneqq \begin{cases}
      \min \{ x, y \}, & x \neq \infty \text{ and } y \neq \infty \text{ and they are comparable}, \\
      \infty,          & x = \infty \text{ or } y = \infty
    \end{cases} \\
    \\
     & \odot: T \times T \to T                         \\
     & x \odot y \begin{cases}
      x + y,  & x \neq \infty \text{ and } y \neq \infty, \\
      \infty, & x = \infty \text{ or } y = \infty
    \end{cases}
  \end{BreakableAlign*}

  This makes \( (T, \oplus, \odot) \) into a \hyperref[def:semiring/dioid]{dioid} with additive identity \( \infty \) and multiplicative identity \( 0 \). We call \( (T, \oplus, \odot) \) the \( \min \)-\Def{tropical semiring} or simply the \Def{tropical semiring} over \( M \). We define the \( \max \)-\Def{tropical semiring} analogously by simply replacing \( \min \) with \( \max \).
\end{definition}

\begin{definition}\label{def:semiring_direct_product}
  Let \( \{ X_k \}_{k \in \CK} \) be a nonempty family of rings.

  Analogously to \fullref{def:group_direct_product}, we define their \Def{direct product} as the ring \( \prod_{k \in \CK} X_k \), the operations defined componentwise as
  \begin{BreakableAlign*}
     & \{ x_k \}_{k \in \CK} + \{ y_k \}_{k \in \CK}
    \coloneqq
    \{ x_k + y_k \}_{k \in \CK},                         \\
     & \{ x_k \}_{k \in \CK} \cdot \{ y_k \}_{k \in \CK}
    \coloneqq
    \{ x_k \cdot y_k \}_{k \in \CK}.
  \end{BreakableAlign*}

  We define their \Def{direct sum} as the subring of \( \prod_{k \in \CK} X_k \) (see \fullref{def:semiring_direct_product}) where only finitely many components of any ring element are different from zero.
\end{definition}

\begin{proposition}\label{thm:ring_categorical_limits}
  We are interested in \hyperref[def:categorical_limit]{categorical limits} and \hyperref[def:categorical_colimit]{colimits} in \( \Cat{Ring} \). Fix an indexed family  \( \{ X_k \}_{k \in \CK} \) of rings.
  \begin{DefEnum}
    \ILabel{thm:ring_categorical_limits/product} Their \hyperref[def:categorical_product]{categorical product} is their direct \hyperref[def:semiring_direct_product]{product} \( \prod_{k \in \CK} X_k \), the projection morphisms being inherited from \fullref{thm:set_categorical_limits/product}.
  \end{DefEnum}
\end{proposition}

\begin{definition}\label{def:opposite_ring}\MarginCite[555]{Knapp2016BAlg}
  The opposite ring \( R^{-1} \) of \( R \) is defined as the same abelian group with the order of multiplication reversed. They are obviously isomorphic for commutative rings.
\end{definition}

\begin{definition}\label{def:ring_commutator}
  Let \( R \) be a ring. The commutator of \( x, y \in R \) is defined as
  \begin{equation*}
    [x, y] \coloneqq xy - yx.
  \end{equation*}

  The commutator ideal of \( R \) is the ideal \hyperref[def:generated_ring_ideal]{generated} by all the commutators in \( G \).
\end{definition}

\begin{proposition}\label{thm:quotient_by_commutator_ideal}
  The quotient \( R / I \) of any unital \hyperref[def:semiring/unital_ring]{ring} \( R \) by its commutator ideal \( I \) is \hyperref[def:semiring/commutative_ring]{commutative}.
\end{proposition}

\begin{definition}\label{def:endomorphism_dioid}
  Let \( (X, +) \) be an monoid and let \( \End(X) \) be set of endomorphism over \( X \). We define two operations:
  \begin{itemize}
    \item Pointwise addition \( [f + g](x) \coloneqq f(x) + g(x) \).
    \item Multiplication by composition \( [fg](x) \coloneqq f(g(x)) \).
  \end{itemize}

  These operations make \( \End(X) \) into a dioid. If \( X \) is a group, then \( \End(X) \) is a ring.

  If \( X \) is a dioid, we define \( \End(X) \) to be a set of dioid endomorphisms (that is, we want the additive group homomorphisms to preserve multiplication and units). Then \( \End(X) \) is again a dioid and, if \( X \) is a unital ring, so is \( \End(X) \).
\end{definition}

\begin{definition}\label{def:function_support}
  The \Def{support} of a function \( f: S \to R \) from a set \( S \) to a semiring \( R \) is the set
  \begin{equation*}
    \Supp(f) \coloneqq \{ x \in S \colon f(x) \neq 0_R \}.
  \end{equation*}
\end{definition}

\begin{definition}\label{def:functions_vanish_nowhere}
  Let \( \Cal{F} \) be a family of functions from a set \( S \) to a ring \( R \). We say that \( \Cal{F} \) \Def{vanishes nowhere} if for every \( x \in S \) there exists a function \( f \in \Cal{F} \) such that \( f(x) \neq 0_R \).
\end{definition}

\begin{definition}\label{def:ordered_semiring}
  Extending \fullref{def:preordered_magma} to (semi)rings, we define a \Def{preordered semiring} to be a semiring \( \BR \) with a magma preorder \( \leq \) that additionally satisfies
  \begin{equation}\label{eq:def:ordered_semiring/nonnegativity}
    0 \leq y \Tand 0 \leq y \Timplies 0 \leq xy.
  \end{equation}
\end{definition}

\subsection{Ring ideals}\label{subsec:ring_ideals}

\begin{definition}\label{def:semiring_ideal}
  Let \( R \) be a semiring and \( I \) be a subset of \( R \). We say that \( I \) is an \term{ideal} (left, right or two-sided) if \( (I, +) \) is a subgroup of \( (R, +) \) and \( (I, \cdot) \) is a \hyperref[def:magma_ideal]{magma ideal} of \( (R, \cdot) \).
\end{definition}

\begin{proposition}\label{thm:semiring_ideal_is_nonunital_subsemiring}
  Two-sided semiring ideals are subsemirings. In the special case where the semiring is unital, an ideal is a unital subsemiring if and only if it is not a proper ideal.
\end{proposition}
\begin{proof}
  Follows from \fullref{thm:magma_ideal_is_submagma} and \fullref{thm:unital_magma_ideal_is_submagma_iff_contains_identity}.
\end{proof}

\begin{definition}\label{thm:semiring_ideal_iff_kernel}
  A subset of a ring is a two-sided \hyperref[def:semiring_ideal]{ideal} if and only if it is the \hyperref[def:semiring_kernel]{kernel} of some ring homomorphism.
\end{definition}
\begin{proof}
  \SufficiencySubProof Let \( I \) be a two-sided ideal. Since it is an abelian group, \( I \) is a normal subgroup and thus we can form the quotient \hyperref[def:normal_subgroup]{group} \( R / I \) with the canonical projection
  \begin{balign*}
     & \pi: R \to R / I       \\
     & \pi(x) \coloneqq x + I
  \end{balign*}

  Multiplication in \( R \) induces multiplication in \( R / I \) by
  \begin{equation*}
    (x + I) \cdot (y + I) \coloneqq (xy + I).
  \end{equation*}

  It is well-defined since if \( x + I = x' + I \) and \( y + I = y' + I \), then
  \begin{balign*}
    (x + I) (y + I)
     & =
    xy + (Iy + xI + II)
    =    \\ &=
    xy + I
    =    \\ &=
    x'y' + I
    =    \\ &=
    x'y' + (Iy' + x'I + II)
    =    \\ &=
    (x' + I) (y' + I).
  \end{balign*}

  Thus, the ring structure on \( R \) induces a ring structure on \( R / I \).

  The canonical projection \( \pi \) is an additive group homomorphism. Since we just showed that \( \pi(xy) = \pi(x) \pi(y) \), it follows that it is also a ring homomorphism.

  It only remains to show that \( \ker(\pi) = I \). Since \( I \) is closed under addition, naturally \( I \subseteq \ker(\pi) \). Conversely, if \( x \in \ker(\pi) \), then \( \pi(x) = \pi(0) = I \), i.e. \( x \in I \). Hence, \( \ker(\pi) = I \).

  \NecessitySubProof Let \( f: R \to T \) is a ring homomorphism. We must show that \( \ker(f) \) is an ideal. If \( x \in R \) and \( y \in \ker(f) \), then
  \begin{equation*}
    f(xy) = f(x) f(y) = f(x) 0 = 0.
  \end{equation*}

  Thus, \( xy \in \ker(f) \). Similarly, we can show that \( yx \in \ker(f) \). Thus, \( R \ker(f) = \ker(f) R = \ker(f) \) and \( \ker(f) \) is a two-sided ideal.
\end{proof}

\begin{definition}\label{def:generated_ring_ideal}
  Let \( R \) be a commutative ring, so that left and right ideals coincide. Let \( S \subseteq R \) be any nonempty subset of \( R \). We define the ideal generated by \( S \) equivalently as either
  \begin{thmenum}
    \thmitem{def:generated_ring_ideal/minimal} the smallest ideal of \( R \) that contains \( S \).
    \thmitem{def:generated_ring_ideal/direct} the ideal
    \begin{equation*}
      \braket S \coloneqq \left\{ \sum_{k=1}^n r_k s_k \mid r_1, \ldots, r_n \in R, s_1, \ldots, s_n \in S, n \in \BbbZ_{>0} \right\}
    \end{equation*}
    of finite linear combinations.

    \thmitem{def:generated_ring_ideal/polynomials} the ideal
    \begin{equation*}
      \braket S \coloneqq \left\{ p(s_1, s_2, \ldots, s_n) \mid s_1, \ldots, s_n \in S, p \in R[X_1, \ldots, X_n], n \in \BbbZ_{>0} \right\}
    \end{equation*}
  \end{thmenum}

  If \( S \) is finite, then \( \braket S \) is called \term{finitely generated}. If \( S = \{ s_1, \ldots, s_n \} \), then
  \begin{equation*}
    \braket S = s_1 R + s_2 R + \cdots s_n R.
  \end{equation*}
\end{definition}

\begin{definition}\label{def:principal_ideal}
  If an ideal \( I \) is \hyperref[def:generated_ring_ideal]{generated} by a single element, it is called a \term{principal ideal}.
\end{definition}

\begin{proposition}\label{thm:product_of_principal_ideals}
  In a commutative unital ring \( R \) the product of the principal ideals \( \braket{x} \) and \( \braket{y} \) is \( \braket{xy} \).
\end{proposition}

\subsection{Fields}\label{subsec:fields}

\begin{definition}\label{def:field}
  As mentioned in \fullref{def:semiring/field}, fields are commutative division rings.
\end{definition}

\begin{proposition}\label{thm:ideals_of_field}
  The only \hyperref[def:semiring_ideal]{ideals} of a field are \( \{ 0 \} \) and \( \BbbK \).
\end{proposition}

\begin{theorem}\label{thm:ring_of_integers_module_prime_is_field}
  The ring \( \BbbZ_n \) (see \fullref{def:ring_of_integers_modulo}) of integers modulo \( n \) is a field if \( n \) is a prime \hyperref[def:prime_number]{number}.
\end{theorem}
\begin{proof}
  We only need to show that \( \BbbZ_n \) has a multiplicative inverse for any nonzero element.

  Fix \( x \in \BbbZ_n \). If \( y \) is a multiplicative inverse of \( x \), we should have
  \begin{equation*}
    xy \cong 1 \pmod n,
  \end{equation*}
  which is the same as
  \begin{equation*}
    n \mid (xy - 1).
  \end{equation*}

  \Fullref{thm:bezout_identity} gives us integers \( a, b \in \BbbZ \) such that
  \begin{equation*}
    ax + bn = \gcd(x, n) = 1,
  \end{equation*}
  which is the same as
  \begin{equation*}
    -bn = xa - 1.
  \end{equation*}

  Define \( y \coloneqq \rem(a, n) \). This is the multiplicative inverse of \( x \).
\end{proof}

\begin{definition}\label{def:field_extension}
  If \( \Bbbk \) and \( \BbbK \) are fields and \( \Bbbk \) is a unital \hyperref[def:first_order_substructure]{subring} of \( \BbbK \), we say that \( \Bbbk \) is a \term{subfield} of \( \BbbK \) and that \( \BbbK \) is a \term{field extension} of \( \Bbbk \). If \( \BbbK = \Bbbk \), we say that \( \BbbK \) is a \term{trivial field extension} of \( \Bbbk \).

  Field extension are also denoted as \( \BbbK / \Bbbk \) to highlight the roles of \( \BbbK \) and \( \Bbbk \). This is not a quotient ring, but simply a notation. See \fullref{def:galois_group}.

  We define the following
  \begin{thmenum}
    \thmitem{def:field_extension/dimension} The extension \( \BbbK \) is a vector space over \( \Bbbk \). We denote the dimension of this vector space by
    \begin{equation*}
      [\BbbK : \Bbbk].
    \end{equation*}

    We call \( [\BbbK : \Bbbk] \) the dimension of \( \BbbK \) over \( \Bbbk \) and if \( [\BbbK : \Bbbk] \) is finite, we say that \( \BbbK \) is a \term{finite extension} of \( \Bbbk \).

    \thmitem{def:field_extension/generated_extension} If \( x_1, \ldots, x_n \) are members of \( \BbbK \), we will use the following
    \begin{itemize}
      \item The ring \( \Bbbk[x_1, \ldots, x_n] \) obtained by evaluating \hyperref[thm:polynomial_ring_universal_property]{polynomials}.
      \item The field \( \Bbbk(x_1, \ldots, x_n) \) obtained by evaluating rational algebraic \hyperref[def:rational_algebraic_function]{functions}.
    \end{itemize}
  \end{thmenum}
\end{definition}

\begin{remark}\label{rem:adjoint_extension_field}
  Any field of rational algebraic \hyperref[def:rational_algebraic_function]{functions} over \( \BbbK \) is always a field extension of \( \BbbK \). We say that the field \( \BbbK(X) \) is obtained from \( \BbbK \) by \term{adjoining} a new element \( X \). Although formally \( X \) is a polynomial, we regard it as a symbol in the sense of \fullref{def:language}.
\end{remark}

\begin{definition}\label{def:galois_group}\mcite[124]{Knapp2016BasicAlgebra}
  Let \( \BbbK \) be a field \hyperref[def:field_extension]{extension} of \( \Bbbk \). The group \( \op{Gal}(\BbbK / \Bbbk) \) of automorphisms of \( \BbbK \) that leave \( \Bbbk \) fixed is called the \term{Galois group} of the field extension \( \BbbK / \Bbbk \).
\end{definition}

\begin{example}\label{thm:galois_group_complex_over_real}
  The Galois \hyperref[def:galois_group]{group} \( \op{Gal}(\BbbC / \BbbR) \) is the group of all \( \BbbR \)-linear functions \( \varphi: \BbbC \to \BbbC \) such that
  \begin{equation*}
    \varphi(\BbbR) = \BbbR.
  \end{equation*}

  The only such functions are rotations and axial symmetries. No nontrivial rotations of the complex plane leave \( \BbbR \) intact and the only nontrivial axial symmetry that fixes \( \BbbR \) is \( a + bi \mapsto a - bi \). Hence,
  \begin{equation*}
    \op{Gal}(\BbbC / \BbbR) \cong \BbbZ_2.
  \end{equation*}
\end{example}

\begin{definition}\label{def:transcendetal_element}\mcite[454]{Knapp2016BasicAlgebra}
  We say that the element \( a \in \BbbK \) of the field extension \( \BbbK \) of \( \Bbbk \) is \term{transcendental} over \( \BbbK \) if any of the equivalent conditions hold:
  \begin{thmenum}
    \thmitem{def:transcendetal_element/evaluation} The evaluation \hyperref[thm:polynomial_ring_universal_property]{map} \( \Phi_a: \Bbbk[X] \to \Bbbk[a] \) is injective.

    \thmitem{def:transcendetal_element/polynomial} There exists no polynomial \( p(X) \in \Bbbk[X] \) such that \( p(a) = 0 \).
  \end{thmenum}

  If \( a \) is not transcendental, we say that is is \term{algebraic}.
\end{definition}

\begin{definition}\label{def:algebraic_extension}\mcite[456]{Knapp2016BasicAlgebra}
  We say that the field extension \( \BbbK \) of \( \Bbbk \) is an \term{algebraic extension} if every element of \( \BbbK \) is algebraic over \( \Bbbk \).
\end{definition}

\begin{proposition}\label{thm:field_elements_are_algebraic}
  Every field is an \hyperref[def:algebraic_extension]{algebraic} of itself.
\end{proposition}
\begin{proof}
  If \( a \in \BbbK \), then \( \BbbK[a] = \Bbbk \) because every polynomial evaluates to some real element, depending on \( a \), and the constant polynomials already take all possible values. Thus, \( \Phi_a \) is not injective.

  Since \( a \in \BbbK \) was arbitrary, we conclude that all elements from a field are algebraic over \( \BbbK \).
\end{proof}

\begin{theorem}\label{thm:algebraic_extension_always_exists}\mcite[485]{Knapp2016BasicAlgebra}
  If \( p(X) \) is a prime polynomial over the field \( \BbbK \), there exists an algebraic extension of \( \BbbK = \Bbbk[u] \), where \( u \in \BbbK \) is a root of \( p(X) \).
\end{theorem}
\begin{proof}
  Since \( p(X) \) is prime, \( \braket{p(X)} \) is a nontrivial prime ideal. By \fullref{thm:prime_ideals_are_maximal_in_pid}, \( \braket{p(X)} \) is maximal and by \fullref{def:maximal_ring_ideal}, the quotient \( \BbbK \coloneqq \Bbbk / \braket{p(X)} \) is a field. It is an extension field of \( \Bbbk \).

  Define
  \begin{equation*}
    u \coloneqq X + \braket{p(X)}.
  \end{equation*}

  Then
  \begin{equation*}
    p(u) = p(X + \braket{p(X)}) = p(X) + \braket{p(X)} = \braket{p(X)},
  \end{equation*}
  hence \( u \) is a root of \( p(X) \) in \( \BbbK \).

  Thus, \( u \) is algebraic over \( \Bbbk \) and \( \BbbK \) is an algebraic extension of \( \Bbbk \).

  It remains to show that \( \BbbK = \Bbbk[u] \). First, take a coset \( q(X) + \braket{p(X)} \) in \( \BbbK \). We have
  \begin{equation*}
    q(X) + \braket{p(X)} = (q(X) - X) + (X + \braket{p(X)}) = (q(X) - X) + u,
  \end{equation*}
  hence this belongs in \( \Bbbk[u] \). Conversely, evaluating a polynomial \( q(X) \in \Bbbk[X] \) at \( u \) gives us
  \begin{equation*}
    q(u) = q(X + \braket{p(X)}) = q(X) + \braket{p(X)},
  \end{equation*}
  which is a coset of \( \BbbK \).
\end{proof}

\begin{proposition}\label{thm:finite_field_extensions_are_algebraic}
  Finite field extensions are algebraic.
\end{proposition}
\begin{proof}
  Fix a finite field extension \( \BbbK / \Bbbk \) and denote by \( n \) the dimension \( [\BbbK : \Bbbk] \). Assume that \( a \in \BbbK \) is transcendental. Then the evaluation map
  \begin{equation*}
    \Phi_a: \Bbbk[X] \to \Bbbk[a]
  \end{equation*}
  is injective. But \( \Bbbk[X] \) has a countably infinite monomial basis, so \( \Bbbk[a] \) must also have a countable basis consisting of \( 1, a, a^2, \ldots \). But \( \Bbbk[a] \) is a subspace of \( \Bbbk \), which is finite dimensional over \( \Bbbk \).

  The obtained contradiction proves the theorem.
\end{proof}

\begin{theorem}\label{thm:e_is_transcendental}\label{thm:eulers_constant_is_transcendental}
  \hyperref[def:exponential_function]{Euler's constant} \( e \) is transcendental over \( \BbbQ \).
\end{theorem}

\begin{theorem}\label{thm:pi_is_transcendental}\mcite[454]{Knapp2016BasicAlgebra}
  The number \( \pi \) (see \fullref{def:pi}) is transcendental over \( \BbbQ \).
\end{theorem}

\begin{example}\label{ex:polynomials_over_pi}
  \Fullref{thm:pi_is_transcendental} implies that the polynomials \( \BbbQ[X] \) can be embedded into \( \BbbR \) via \( \Phi_\pi: \BbbQ[X] \to \BbbR \). We can identify a polynomial
  \begin{equation*}
    p(X) = \sum_{i=0}^n a_i X^i
  \end{equation*}
  with rational coefficients with the number
  \begin{equation*}
    p(\pi) = \sum_{i=0}^n a_i \pi^i.
  \end{equation*}
\end{example}

\begin{definition}\label{def:algebraically_closed_field}\mcite[prop. 9.20]{Knapp2016BasicAlgebra}
  We say that the field \( \BbbK \) is algebraically closed if any of the equivalent conditions are satisfied:
  \begin{thmenum}
    \thmitem{def:algebraically_closed_field/trivial_algebraic_extensions} \( \BbbK \) has no nontrivial algebraic \hyperref[def:algebraic_extension]{extensions}.
    \thmitem{def:algebraically_closed_field/linear_irreducible_polynomials} Every irreducible polynomial in \( \Bbbk[X] \) is linear.
    \thmitem{def:algebraically_closed_field/at_least_one_root} Every nonconstant polynomial in \( \Bbbk[X] \) has at least one root in \( \Bbbk \).
    \thmitem{def:algebraically_closed_field/factorization} Every polynomial in \( \Bbbk[X] \) \hyperref[def:factorization_in_ring]{factors} into a product of linear polynomials.
    \thmitem{def:algebraically_closed_field/exactly_n_roots} Every polynomial in \( \Bbbk[X] \) of degree \( n \) has exactly \( n \) roots in \( \Bbbk \).
  \end{thmenum}
\end{definition}
\begin{proof}
  \ImplicationSubProof{def:algebraically_closed_field/trivial_algebraic_extensions}{def:algebraically_closed_field/linear_irreducible_polynomials} Let \( p(X) \) be an irreducible polynomial in \( \Bbbk[X] \). By \fullref{thm:ufd_prime_iff_irreducible}, \( p(X) \) is prime. By \fullref{thm:algebraic_extension_always_exists}, there exists an algebraic extension \( \BbbK \) of \( \Bbbk \) such that the prime polynomial \( p(X) \) has a root in \( \BbbK \). But \( \Bbbk \) has no nontrivial algebraic extensions, hence \( F = G \) and \( p(X) \) has a root \( u \in F \).

  If \( p(X) \) is not linear, we can divide \( p(X) \) by \( (X - u) \) to obtain a lower-degree non-constant polynomial. Hence, \( p(X) \) is linear.

  \ImplicationSubProof{def:algebraically_closed_field/linear_irreducible_polynomials}{def:algebraically_closed_field/factorization} With induction on the polynomial degree, we split a polynomial \( p(X) \) into a product of linear polynomials.

  This is obvious for \( \deg p = 1 \). Assume that the statement holds for polynomial of degree strictly less than \( n \) and let \( p(X) \) be a polynomial of degree \( n \). By \fullref{def:factorization_in_ring}, it is \hyperref[def:irreducible_ring_element]{reducible}, that is, there exist non-invertible (that is, non-constant) polynomials \( r_1(X) \) and \( r_2(X) \), such that
  \begin{equation*}
    p(X) = r_1(X) r_2(X).
  \end{equation*}

  Since both \( r_1(X) \) and \( r_2(X) \) are non-constant, they have a positive degree less than \( n \). Hence, the induction conjecture holds for them and both can be factored into linear polynomials. Therefore, their product \( p(X) \) can also be factored into linear polynomials.

  This completes the proof.

  \ImplicationSubProof{def:algebraically_closed_field/at_least_one_root}{def:algebraically_closed_field/factorization} Suppose that \( u_1 \) is a root of \( p(X) \). \Fullref{thm:polynomial_root_iff_divisible} tells us that \( p(X) \) is divisible by \( (X - u_1) \). Using induction on the degree of \( p(X) \), we can factor \( p(X) \) into
  \begin{equation*}
    p(X) = a (X - u_1) (X - u_2) \cdots (X - u_n),
  \end{equation*}
  where \( a \in F \).

  \ImplicationSubProof{def:algebraically_closed_field/factorization}{def:algebraically_closed_field/exactly_n_roots} Follows from \fullref{thm:polynomial_root_iff_divisible} Follows from \fullref{thm:polynomial_root_iff_divisible} by induction on the polynomial degree. Note that the number of roots is bounded by \( n \) (see \fullref{thm:integral_domain_polynomial_root_limit}).

  \ImplicationSubProof{def:algebraically_closed_field/exactly_n_roots}{def:algebraically_closed_field/trivial_algebraic_extensions} By \fullref{thm:integral_domain_polynomial_root_limit}, if \( p(X) \) has degree \( n \) and exactly \( n \) roots, then it has no more roots. Hence, all roots of \( p(X) \) are already in the field \( \Bbbk \) and \( \Bbbk \) is the only algebraic extension of itself.
\end{proof}

\begin{proposition}\label{thm:no_finite_extensions_of_closed_fields}
  There exist no nontrivial finite extensions of an algebraically closed field.
\end{proposition}
\begin{proof}
  Let \( \BbbK \) is a finite extension of the algebraically closed field \( \Bbbk \). By \fullref{thm:finite_field_extensions_are_algebraic}, \( \BbbK \) is an algebraic extension. But every element of \( \Bbbk \) is already algebraic over \( \Bbbk \), therefore \( \BbbK \subseteq \Bbbk \).

  We conclude that \( \Bbbk = \BbbK \), hence the only finite extension of an algebraically closed field is trivial.
\end{proof}

\begin{definition}\label{def:splitting_field}\mcite[458]{Knapp2016BasicAlgebra}
  We say that a polynomial \( p(X) \in \BbbK[X] \) over the field \( \BbbK \) \term{splits} if \( p(X) \) can be \hyperref[def:factorization_in_ring]{factored} into a product of linear polynomials in \( \BbbK[X] \).

  A \term{splitting field} of \( p(X) \) over \( \Bbbk \) is a field extension \( \BbbK / \Bbbk \) such that
  \begin{itemize}
    \item \( p(X) \) splits over \( \BbbK \).
    \item \( \BbbK \) is \hyperref[def:generated_ring_ideal]{generated} by \( \Bbbk \) and the roots of \( p(X) \) over \( \BbbK \).
  \end{itemize}
\end{definition}

\begin{proposition}\label{thm:splitting_field_existence}\mcite[thm. 9.12]{Knapp2016BasicAlgebra}
  A splitting field exists for every polynomial \( p(X) \in \BbbK[X] \).
\end{proposition}

\medskip

\begin{theorem}\label{thm:galois_field_existence}\mcite[thm. 9.14]{Knapp2016BasicAlgebra}
  Fix a prime \hyperref[def:prime_number]{number} \( p \) and a positive integer \( n \). Then there exists up to an isomorphism a unique field with \( p^n \) elements. Furthermore, this is a splitting field for \( X^{p^n} - X \) for the field \( \BbbZ_p \).

  We call this field the \term{Galois field} of \( p^n \) elements over the prime field \( \BbbZ_p \) and denote it by
  \( \BbbF_{p^n} \). We identify \( \BbbZ_p \) with \( \BbbF_p \).
\end{theorem}

\begin{theorem}\label{thm:f2_is_boolean_algebra}
  The Galois field \( \BbbF_2 \) is a \hyperref[def:boolean_algebra]{Boolean algebra} with joins and meets induced by the ordering (see \fullref{thm:binary_lattice_operations}) and complements given by \( \neg x \coloneqq x \mapsto x \oplus 1 \).

  More concretely:
  \begin{itemize}
    \item The top element is \( \sup \{ 0, 1 \} = 1 \)
    \item The bottom element is \( \inf \{ 0, 1 \} = 0 \)
    \item Joins are given by \( \inf \{ x, y \} = xy \)
    \item Meets are given by multiplication \( \sup \{ x, y \} = x \oplus y \oplus (x \odot y) = x \oplus y \oplus \inf \{ x, y \} \)
  \end{itemize}
\end{theorem}
\begin{proof}
  Addition and multiplication in \( \BbbF_2 \) works as usual integer \hyperref[def:set_of_integers]{arithmetic}, except that \( 1 \oplus 1 = 0 \):
  \begin{equation*}
    \begin{tabular}{c c | c c}
      \( x \) & \( y \) & \( x \oplus y \) & \( x \odot y \) \\
      \hline
      \( 0 \) & \( 0 \) & \( 0 \)          & \( 0 \)         \\
      \( 0 \) & \( 1 \) & \( 1 \)          & \( 0 \)         \\
      \( 1 \) & \( 0 \) & \( 1 \)          & \( 0 \)         \\
      \( 1 \) & \( 1 \) & \( 0 \)          & \( 1 \)
    \end{tabular}
  \end{equation*}

  Evidently \( 1 \) is a top element and \( 0 \) is a bottom element, thus \( \BbbF_2 \) is a \hyperref[def:semilattice/lattice]{lattice}.

  Distributivity of multiplication over addition is inherited from \( \BbbZ \), however unlike in \( \BbbZ \), addition distributes over multiplication:
  \begin{equation*}
    \begin{tabular}{c c c | c c}
      \( x \) & \( y \) & \( z \) & \(x \odot (y \oplus z) \) & \( (x \odot y) \oplus (x \odot z) \) \\
      \hline
      \( 0 \) & \( 0 \) & \( 0 \) & \( 0 \)                   & \( 0 \)                              \\
      \( 0 \) & \( 0 \) & \( 1 \) & \( 0 \)                   & \( 0 \)                              \\
      \( 0 \) & \( 1 \) & \( 0 \) & \( 0 \)                   & \( 0 \)                              \\
      \( 0 \) & \( 1 \) & \( 1 \) & \( 0 \)                   & \( 0 \)                              \\
      \( 1 \) & \( 0 \) & \( 0 \) & \( 0 \)                   & \( 0 \)                              \\
      \( 1 \) & \( 0 \) & \( 1 \) & \( 1 \)                   & \( 1 \)                              \\
      \( 1 \) & \( 1 \) & \( 0 \) & \( 1 \)                   & \( 1 \)                              \\
      \( 1 \) & \( 1 \) & \( 1 \) & \( 1 \)                   & \( 1 \)
    \end{tabular}
  \end{equation*}

  Thus, \( \BbbF_2 \) is a distributive \hyperref[def:semilattice/distributive_lattice]{lattice}.

  It is also evident that the complementation \( \neg x = x \mapsto x \oplus 1 \) gives the desired result:
  \begin{itemize}
    \item \( \inf \{ x, \neg x \} = x \odot \neg x = x \odot (x \oplus 1) = (x \odot x) \oplus (x \odot 1) = 0 \) since \( x \odot x = x \odot 1 \).
    \item \( \sup \{ x, \neg x \} = x \oplus \neg x \oplus (x \odot \neg x) = x \oplus \neg x \oplus 0 = x \oplus (x \oplus 1) = 1 \).
  \end{itemize}

  Therefore, \( (\BbbF_2, 1, 0, \inf, \sup, \neg x ) \) is a Boolean algebra.
\end{proof}

\begin{proposition}
  For any function \( f: \BbbF_n \to \BbbF_n \) over any Galois field \( \BbbF_n \), there exists a unique polynomial \( p(X) \in \BbbF_n[X] \) of degree \( n - 1 \) such that the corresponding function \( p(x) \) agrees with \( f(x) \) on all of \( \BbbF_n \).
\end{proposition}
\begin{proof}
  We simply use \fullref{thm:lagrange_interpolation} on all points of the field.
\end{proof}

\begin{algorithm}\label{alg:finite_field_polynomial_reduction}
  Let \( \BbbF_p \) be the \hyperref[thm:galois_field_existence]{Galois field} for some prime number \( p \). Consider the nonzero polynomial
  \begin{equation*}
    f(X_1, \ldots, X_n) \coloneqq \sum_{k_1=0}^{m_1} \cdots \sum_{k_n=0}^{m_n} a_{k_1,\ldots,k_n} X_1^{k_1} \cdots X_n^{k_n}.
  \end{equation*}

  We will build a polynomial \( \hat f(X_1, \ldots, X_n) \) of degree at most \( n - 1 \) that corresponds to the same function. In terms of the \hyperref[thm:polynomial_ring_universal_property]{evaluation homomorphism}
  \begin{equation*}
    \Phi: \BbbF_p[X_1, \ldots, X_n] \to \fun(\BbbF_p^n, \BbbF_p),
  \end{equation*}
  this means that
  \begin{equation*}
    \Phi(f) = \Phi(\hat f).
  \end{equation*}

  This can be achieved by grouping some of the coefficients. The univariate case is more understandable, so we will instead initially consider the polynomial
  \begin{equation*}
    g(X) \coloneqq \sum_{k=0}^m a_k X^k.
  \end{equation*}

  We will now use (and later prove) that the following two univariate polynomials evaluate to the same function:
  \begin{equation}\label{eq:alg:finite_field_polynomial_reduction/reduction}
    \Phi(X^s) = \Phi(X^{\rho(s)}),
  \end{equation}
  where
  \begin{align*}
    &\rho: \{ 1, 2, 3, \ldots \} \to \{ 1, 2, \ldots, p - 1 \} \\
    &\rho(s) \coloneqq \begin{cases}
      p - 1,          &(p - 1) \mid s \\
      \rem(s, p - 1), &\T{times}.
    \end{cases}
  \end{align*}

  Obviously \( \rho(s) = s \) for \( s < p \). It follows from the linearity of \( \Phi \) that
  \begin{equation*}
    \hat g(X)
    \coloneqq
    \sum_{k=0}^m a_k X^{r(k)}
    =
    \sum_{j=0}^{p-1} \left( \sum_{k \in r^{-1}(j)} a_k \right) X^j
  \end{equation*}
  is the desired polynomial.

  In the multivariate case, the reduced polynomial is
  \begin{equation*}
    \hat f(X_1, \ldots, X_n) \coloneqq \sum_{j_1=0}^{p-1} \cdots \sum_{j_n=0}^{p-1} \left( \sum_{k_1 \in r^{-1}(j_1)} \cdots \sum_{k_n \in r^{-1}(j_n)} a_{k_1,\ldots,k_n} \right) X_1^{k_1} \cdots X_n^{k_n}.
  \end{equation*}

  Correctness follows, again, from the linearity of \( \Phi \).
\end{algorithm}
\begin{proof}[Proof of \eqref{eq:alg:finite_field_polynomial_reduction/reduction}]
  By \fullref{thm:fermats_little_theorem},
  \begin{equation}\label{eq:alg:finite_field_polynomial_reduction/fermat}
    \Phi(X^p) = \Phi(X).
  \end{equation}

  We now use induction on \( q \coloneqq \quot(s, p) \).
  \begin{itemize}
    \item If \( q = 0 \), then \( s < p \) and no reduction is necessary.
    \item If \( q > 0 \), let \( s = (p - 1) q + r \). We have two cases
    \begin{itemize}
      \item If \( (p - 1) \mid s \), i.e. if \( r = 0 \), then
      \begin{balign*}
        \Phi(X^s)
        &=
        \Phi(X^{(p-1)q})
        = \\ &=
        \Phi(X^{(p-1)(q-1) + (p-1)})
        = \\ &=
        \Phi(X^{(p-1)(q-1)}) \Phi(X^{p-1})
        \reloset {\T{ind.}} = \\ &=
        \Phi(X^{p-1}) \Phi(X^{p-1})
        = \\ &=
        \Phi(X^{2(p-1)})
        = \\ &=
        \Phi(X^{p+(p-2)})
        = \\ &=
        \Phi(X^p) \Phi(X^{p-2})
        \reloset {\eqref{eq:alg:finite_field_polynomial_reduction/fermat}} = \\ &=
        \Phi(X^{p-1}).
      \end{balign*}

      \item If \( r > 0 \), we have
      \begin{balign*}
        \Phi(X^s)
        &=
        \Phi(X^{(p-1)q+r})
        = \\ &=
        \Phi(X^{[(p-1)(q-1) + r] + (p-1)})
        = \\ &=
        \Phi(X^{(p-1)(q-1)+r}) \Phi(X^{p-1})
        \reloset {\T{ind.}} = \\ &=
        \Phi(X^r) \Phi(X^{p-1})
        = \\ &=
        \Phi(X^{r-1}) \Phi(X^p)
        \reloset {\eqref{eq:alg:finite_field_polynomial_reduction/fermat}} = \\ &=
        \Phi(X^{r-1}) \Phi(X)
        = \\ &=
        \Phi(X^r).
      \end{balign*}
    \end{itemize}
  \end{itemize}
\end{proof}

\subsection{Modules}\label{subsec:modules}

\begin{definition}\label{def:left_module}
  Let \( R \) be a \hyperref[def:semiring/dioid]{dioid} and \( M \) be an abelian group. Analogously to \fullref{def:monoid_action}, we say that \( M \) is a left \( R \)-\term{module} if it has some additional structure, which can be defined equivalently as
  \begin{thmenum}
    \thmitem{def:left_module/homomorphism} a dioid homomorphism from \( R \) to the endomorphism \hyperref[def:endomorphism_dioid]{ring} \( \End(M) \)

    \thmitem{def:left_module/multiplication}\cite[374]{Knapp2016BasicAlgebra} an \hyperref[def:magma/associative]{associative} and \hyperref[def:unital_magma]{unital} \hyperref[def:monoid_action/family]{operation} \( \cdot: R \times M \to M \), written using juxtaposition.

    We require that \( \cdot \) is associative, distributive over \( + \), and compatible with the identity in \( R \). Explicitly, the following are satisfied for \( x, y \in M \) and \( s, t \in R \):
    \begin{thmenum}
      \thmitem{def:left_module/associativity}(associativity) \( s \cdot (t \cdot x) = (s t) \cdot x \).
      \thmitem{def:left_module/scalar_distributivity}(scalar distributivity) \( (s + t) \cdot x = s \cdot x + t \cdot x \).
      \thmitem{def:left_module/vector_distributivity}(vector distributivity) \( t \cdot (x + y) = t \cdot x + t \cdot y \).
      \thmitem{def:left_module/identity}(identity) \( 1_R \cdot x = x \).
    \end{thmenum}
  \end{thmenum}

  In analogy with \hyperref[def:vector_space]{vector spaces}, we call elements of \( R \) scalars and elements of \( M \) vectors. See \fullref{def:vector_space}.

  We denote the category of modules over \( R \) by \( \cat{Mod}_R \).
\end{definition}
\begin{proof}
  \ImplicationSubProof{def:left_module/homomorphism}{def:left_module/multiplication} Let \( \tau: R \to \End(M) \) be a ring homomorphism. Define the operation
  \begin{balign*}
     & \cdot: R \times M \to M           \\
     & \cdot(r, m) \coloneqq \tau(r)(m).
  \end{balign*}

  Let \( x, y \in M \) and \( s, t \in R \). From the fact that \( \tau \) is a ring homomorphism, we have
  \SubProofOf{def:left_module/associativity}
  \begin{equation*}
    s \cdot (t \cdot x)
    =
    \tau(s)(\tau(t)(x))
    =
    (\tau(s) \circ \tau(t))(x)
    =
    \tau(st)(x).
  \end{equation*}

  \SubProofOf{def:left_module/scalar_distributivity}
  \begin{equation*}
    (s + t) \cdot x
    =
    \tau(s + t)(x)
    =
    \tau(s)(x) + \tau(t)(x)
    =
    s \cdot x + t \cdot x.
  \end{equation*}

  \SubProofOf{def:left_module/vector_distributivity}
  \begin{equation*}
    t \cdot (x + y)
    =
    \tau(t)(x + y).
  \end{equation*}

  Now since \( \tau(t) \) is a group endomorphism, we have \( \tau(t)(x + y) = \tau(t)(x) + \tau(t)(y) \). Thus,
  \begin{equation*}
    t \cdot (x + y)
    =
    t \cdot x + t \cdot y.
  \end{equation*}

  \SubProofOf{def:left_module/identity}
  \begin{equation*}
    1_R \cdot x
    =
    \tau(1_R)(x)
    =
    \id(x)
    =
    x.
  \end{equation*}

  \ImplicationSubProof{def:left_module/multiplication}{def:left_module/homomorphism} Let \( \cdot: R \times M \to M \) be a left scalar multiplication operation. We define
  \begin{balign*}
     & \tau: R \to \End(M)                      \\
     & \tau(t) \coloneqq (x \mapsto t \cdot x).
  \end{balign*}

  This function is well-defined since for each \( t \in R \), the function \( \tau(t) \) is an abelian group homomorphism (due to \fullref{def:left_module/associativity}).

  \( \tau \) is a ring homomorphism because
  \begin{itemize}
    \item it preserves addition:
          \begin{equation*}
            \tau(s + t)
            =
            (x \mapsto (s + t) \cdot x)
            =
            (x \mapsto s \cdot x + t \cdot x)
            =
            \tau(s) + \tau(t).
          \end{equation*}

    \item it preserves multiplication:
          \begin{equation*}
            \tau(st)
            =
            (x \mapsto (st) \cdot x)
            =
            (x \mapsto (s \cdot (t \cdot x)))
            =
            \tau(s) \circ \tau(t).
          \end{equation*}

    \item it preserves identities:
          \begin{equation*}
            \tau(1_R)
            =
            \id,
          \end{equation*}
          which is the multiplicative unit in \( \End(M) \).
  \end{itemize}
\end{proof}

\begin{definition}\label{def:right_module}
  We say that \( \tau: R \to \End(A) \) is a \term{right \( R \)-module} if the same function is a \hyperref[def:left_module]{left module} on the opposite \hyperref[def:opposite_ring]{ring} \( R^{-1} \).
\end{definition}

\begin{definition}\label{def:bimodule}
  An abelian group \( M \) that is both a left \( L \)-module and right \( R \)-module is called an \( L, R \)-\term{bimodule} if \( l \in L, r \in R \) and \( x \in M \) implies
  \begin{equation*}
    (lx)r = l(xr).
  \end{equation*}
\end{definition}

\begin{proposition}\label{thm:def:left_module/properties}
  Any left \( R \)-module \( M \) has the following basic properties:
  \begin{thmenum}
    \thmitem{thm:def:left_module/properties/ring_zero_is_absorbing} Multiplication by \( 0_R \) is \term{absorbing}, that is, \( 0_R x = 0_M \) for any \( x \in M \).

    \thmitem{thm:def:left_module/properties/module_zero_is_absorbing} Multiplication by \( 0_M \) is absorbing, that is, \( t 0_M = 0_M \) for any \( t \in R \).
  \end{thmenum}
\end{proposition}
\begin{proof}
  \SubProofOf{thm:def:left_module/properties/ring_zero_is_absorbing} For any \( x \in M \) we have that \( 0_R x = (0_R + 0_R)x = 0_R x + 0_R x \), thus \( 0_R x \) is an additive identity and \( 0_R x = 0_M \).

  \SubProofOf{thm:def:left_module/properties/module_zero_is_absorbing} For any \( t \in R \) we have that \( t 0_M = t (0_M + 0_M) = t 0_M + t 0_M \), thus \( t 0_M \) is an additive identity and \( t 0_M = 0_M \).
\end{proof}

\begin{example}\label{ex:module/ideal_of_ring}
  Every unital ring \( R \) is a module over itself. Every ideal \( I \subseteq R \) is an \( R \)-module since it is closed under multiplication with \enquote{scalars} from \( R \).
\end{example}

\begin{definition}\label{def:left_module_kernel}
  The \term{kernel} \( \ker(f) \) of a left \( R \)-module homomorphism \( f: M \to N \) is the \hyperref[def:zero_locus]{zero locus} of \( f \), that is, \hyperref[thm:def:function/properties/preimage]{preimage} \( f^{-1}(0_N) \).

  It is an instance of \fullref{def:zero_morphisms/kernel}.
\end{definition}

\begin{definition}\label{def:quotient_left_module}
  Let \( M \) be a left module and \( N \) be a submodule of \( M \). Define the \term{quotient module}
  \begin{equation*}
    M / N \coloneqq \{ x + N \colon x \in M \}
  \end{equation*}
  with the operations
  \begin{balign*}
    x + N \oplus y + N \coloneqq x + y + N.
    t \odot x + N \coloneqq tx + N.
  \end{balign*}

  Define the canonical projection homomorphism
  \begin{balign*}
     & \pi: G \to G / N        \\
     & \pi(x) \coloneqq x + N.
  \end{balign*}

  The kernel of \( \pi \) is precisely \( N \).
\end{definition}
\begin{proof}
  The proof of correctness is similar to \fullref{def:quotient_group}.
\end{proof}

\begin{theorem}\label{thm:homomorphism_theorem_for_left_modules}
  Fix a ring \( R \). Let \( \varphi: M \to K \) be a homomorphism of left \( R \)-modules. We have the isomorphism
  \begin{equation*}
    M / \ker \varphi \cong \img \varphi.
  \end{equation*}
\end{theorem}
\begin{proof}
  Analogous to \fullref{thm:homomorphism_theorem_for_groups}.
\end{proof}

\begin{definition}\label{def:linear_operator}
  Let \( M \) and \( N \) be two left \( R \)-modules. We say that the function \( f: M \to N \) is \term{linear} or a \term{linear operator} if it satisfies the conditions
  \begin{thmenum}
    \thmitem{def:linear_operator/additivity}(additivity) \( f(x + y) = f(x) + f(y) \) for any \( x, y \in M \).
    \thmitem{def:linear_operator/homogeneity}(homogeneity) \( f(tx) = t f(x) \) for any \( t \in R \) and \( x \in M \) (see \fullref{def:homogenous_function}).
  \end{thmenum}
\end{definition}

\begin{proposition}\label{thm:map_is_linear_iff_homomorphism}
  A function \( f: M \to N \) between left \( R \)-modules is \hyperref[def:linear_operator]{linear} if and only if it is a module homomorphism in the sense of \fullref{def:first_order_homomorphism}.
\end{proposition}
\begin{proof}
  Since \( (M, +) \) and \( (N, +) \) are groups, it follows from \fullref{thm:group_homomorphism_single_condition} that \fullref{def:linear_operator/additivity} is equivalent to the requirement that \( f \) is a homomorphism between \( (M, +) \) and \( (N, +) \).

  Now fix \( t \in R \). For \( f \) to be a homomorphism, it must satisfy
  \begin{equation*}
    f(\cdot_M^{(t)}(x)) = \cdot_t^{(N)}(f(x)),
  \end{equation*}
  which is just a more formal way to write \fullref{def:linear_operator/homogeneity}.
\end{proof}

\begin{definition}\label{def:affine_operator}
  Let \( M \) and \( N \) be two left \( R \)-modules. We say that the function \( f: M \to N \) is \term{affine} if it is a translation of a \hyperref[def:linear_operator]{linear function}, that is, if there exists a linear function \( l: M \to N \) and a constant \( a \in N \) such that \( f(x) = l(x) + a \).
\end{definition}

\begin{definition}\label{def:multilinear_function}
  Generalizing \fullref{def:linear_operator}, if \( M_1, \ldots, M_k \) and \( N \) are \( R \)-modules, we say that the function
  \begin{equation*}
    f: M_1 \times \ldots \times M_k \to N
  \end{equation*}
  is \term{multilinear} or \term{\( k \)-linear} (\term{bilinear} for \( k = 2 \), \term{trilinear} for \( k = 3 \)) if it is linear in each component, that is, for each component \( i = 1, \ldots, k \), and for each tuple not containing elements from \( M_i \),
  \begin{equation*}
    (u_1, \ldots, u_{i-1}, u_{i+1}, \ldots, u_k) \in M_1 \times \ldots \times M_{i-1} \times M_{i+1} \times \ldots \times M_k \to N
  \end{equation*}
  the following function is linear:
  \begin{balign*}
     & f_i: M_i \to N                                                         \\
     & f_i(u_i) \coloneqq f(u_1, \ldots, u_{i-1}, u_i, u_{i+1}, \ldots, u_k).
  \end{balign*}
\end{definition}

\begin{definition}\label{def:abelian_group_z_module}\mcite[375]{Knapp2016BasicAlgebra}
  Let \( G \) be an abelian group. Associate with \( G \) the \( \BbbZ \)-module \( M \) with scalar multiplication
  \begin{balign*}
    nu \coloneqq \begin{cases}
      0,              & n = 0  \\
      u + \ldots + u, & n > 0  \\
      -((-n)u),       & n < 0.
    \end{cases}
  \end{balign*}
\end{definition}

\begin{proposition}\label{thm:abelian_group_iff_z_module}\mcite[375]{Knapp2016BasicAlgebra}
  Every abelian group is isomorphic to exactly one \( \BbbZ \)-module.
\end{proposition}
\begin{proof}
  We already saw in \fullref{def:abelian_group_z_module} how every abelian group can be regarded as a \( \BbbZ \)-module. Every \( \BbbZ \)-module can then be identified with its additive group.

  Scalar multiplication ensures that there is exactly one way to define a \( \BbbZ \)-module structure on an abelian group since \( na = (n-1)a + a \) and \( 0a = 0 \).
\end{proof}

\begin{definition}\label{def:left_module_direct_product}
  Let \( \{ X_k \}_{k \in \mscrK} \) be a nonempty family of left \( R \)-modules.

  Analogously to \fullref{def:group_direct_product}, we define their \term{direct product} as the module \( \prod_{k \in \mscrK} X_k \), the operations defined componentwise as
  \begin{balign*}
     & \{ x_k \}_{k \in \mscrK} + \{ y_k \}_{k \in \mscrK}
    \coloneqq
    \{ x_k + y_k \}_{k \in \mscrK},                     \\
     & t \{ x_k \}_{k \in \mscrK}
    \coloneqq
    t \{ t x_k \}_{k \in \mscrK}.
  \end{balign*}

  We define their \term{direct sum} as the submodule of \( \prod_{k \in \mscrK} X_k \) (see \fullref{def:left_module_direct_product}) where only finitely many components of any module element are different from zero.
\end{definition}

\begin{proposition}\label{thm:module_categorical_limits}
  We are interested in \hyperref[def:category_of_cones/limit]{categorical limits} and \hyperref[def:category_of_cones/colimit]{colimits} in the category \( \cat{Mod}_R \) of left-modules over \( R \). Fix an indexed family  \( \{ X_k \}_{k \in \mscrK} \) of \( R \)-modules.
  \begin{thmenum}
    \thmitem{thm:module_categorical_limits/product} Their \hyperref[def:discrete_category_limits]{categorical product} is their direct \hyperref[def:left_module_direct_product]{product} \( \prod_{k \in \mscrK} X_k \), the projection morphisms being inherited from \fullref{thm:discrete_category_limits_in_set}.

    \thmitem{thm:module_categorical_limits/coproduct} Their \hyperref[def:discrete_category_limits]{categorical coproduct} is the \hyperref[def:group_direct_product]{direct sum} \( \oplus_{k \in \mscrK} X_k \), the embedding morphisms being inherited from \fullref{thm:abelian_group_categorical_limits/coproduct}.
  \end{thmenum}
\end{proposition}

\begin{definition}\label{def:linear_combination}
  Let \( M \) be a left \( R \)-module. Like \hyperref[def:polynomial]{polynomials}, we define linear combinations to be tuples \( (t_1, t_2, \ldots, t_n) \) of scalars from \( R \). Unlike with polynomials, we are not interested in defining operations on them, but rather defining the function
  \begin{equation}\label{def:linear_combination/function}
    (x_1, \ldots, x_n) \mapsto \sum_{k=1}^n t_k x_k.
  \end{equation}

  The scalars \( t_1, \ldots, t_n \) are called the \term{coefficients} of the linear combination. A linear combination is said to be \term{trivial} if all coefficients are equal to \( 0_R \).

  For convenience, given set of vectors \( x_1, \ldots, x_n \in M \), we also call the sum \( \sum_{k=1}^n t_k x_k \) a linear combination.

  In the special case where \( R \) is a superring of \( \BbbR \), we define the following special types of linear combinations:
  \begin{thmenum}
    \thmitem{def:linear_combination/affine} an \term{affine combination} if \( \sum_{k=1}^n t_k = 1 \).
    \thmitem{def:linear_combination/conic} a \term{conic combination} if all of the coefficients are nonnegative real numbers.
    \thmitem{def:linear_combination/convex} a \term{convex combination} if it is both affine and conic.
  \end{thmenum}
\end{definition}

\begin{definition}\label{def:left_module_linear_dependence}
  Let \( M \) be a left \( R \)-module and let \( A \subseteq M \). We say that the set \( A \) is \term{linearly independent} if for any linear \hyperref[def:linear_combination/function]{combination}, the equality
  \begin{equation*}
    \sum_{i=1}^n t_i x_k = 0_M
  \end{equation*}
  for \( x_1, \ldots, x_n \in A \) implies that the combination is trivial.

  We say that the vectors \( x_1, \ldots, x_n \) are linearly independent if the corresponding set \( \{ x_1, \ldots, x_n \} \) is linearly independent.

  We say that \( A \) is \term{linearly dependent} if it is not linearly independent.
\end{definition}

\begin{definition}\label{def:left_module_hamel_basis}
  A subset \( B \) of the left \( R \)-module \( M \) is called a \term{Hamel basis} or simply \term{basis} of \( M \) if \( B \) is a \hyperref[def:partially_ordered_set_extremal_points/maximal_and_minimal_element]{minimal} (with respect to set inclusion) linearly independent subset of \( M \).
\end{definition}

\begin{definition}\label{def:free_left_module}\mcite[377]{Knapp2016BasicAlgebra}
  We say that the left \( R \)-module \( M \) is a \term{free left module} if it has a \hyperref[def:left_module_hamel_basis]{basis}.

  Let \( S \) be any set. If we regard \( R \) as a left module over itself, then the \hyperref[thm:module_categorical_limits/coproduct]{direct sum}
  \begin{equation*}
    F(S) \coloneqq \oplus_{s \in S} R
  \end{equation*}
  with injections \( \{ \iota_s \}_{s \in S} \) is called the free left module \term{generated by \( S \)}. Define the function
  \begin{balign*}
     & \varphi: S \to F(S)                \\
     & \varphi(s) \coloneqq \iota_s(1_R).
  \end{balign*}

  The image \( \varphi(S) \) is then a basis of \( F(S) \).

  The cardinality of the basis of a free left module \( M \) is called the \term{rank} \( \rank M \) of \( M \). \Fullref{thm:left_module_basis_cardinality} tells us that this rank is unique for commutative unital rings. If the rank of a module is finite, we say that the module is \term{finitely generated}.

  We also denote \( F(S) = \braket S \), especially in finitely generated modules.
\end{definition}

\begin{proposition}\label{thm:free_module_is_free_functor}
  The functor \( F: \cat{Set} \to \cat{Mod}_R \), defined pointwise in \fullref{def:free_left_module}, is \hyperref[def:category_adjunction]{free}.
\end{proposition}

\begin{proposition}\label{thm:left_module_basis_decomposition}
  Let \( B \) be a basis of the free left \( R \)-module \( M \). Then each element \( u \) of \( M \) can be uniquely (up to a permutation) represented as a linear \hyperref[def:linear_combination]{combination} of elements of \( B \).
\end{proposition}
\begin{proof}
  Let
  \begin{equation*}
    u = \sum_{i=1}^n t_i v_i
  \end{equation*}
  and
  \begin{equation*}
    u = \sum_{j=1}^m s_i w_i
  \end{equation*}
  be two representations of \( u \) as a linear combination over \( B \).

  Define the function
  \begin{balign*}
     & t: M \to R                                \\
     & t(v) \coloneqq \begin{cases}
      t_i, & v = v_i,                          \\
      0,   & v \not\in \{ v_1, \ldots, v_n \}.
    \end{cases}
  \end{balign*}
  and analogously for \( s: M \to R \). Define the set
  \begin{equation*}
    B' \coloneqq \{ v_1, \ldots, v_n, w_1, \ldots, w_m \}.
  \end{equation*}

  Thus,
  \begin{equation*}
    u = \sum_{b \in B'} t(b) b = \sum_{b \in B'} s(b) b
  \end{equation*}
  and
  \begin{equation*}
    0 = u - u = \sum_{b \in B'} (t(b) - s(b)) b.
  \end{equation*}

  The set \( B' \) is linearly independent as a subset of the basis \( B \), hence only a trivial linear combination can be the zero vector. This gives us
  \begin{equation*}
    t(b) = s(b), b \in B'.
  \end{equation*}

  In particular, the two decompositions of \( u \) along \( B \) are identical up to a permutation.
\end{proof}

\begin{definition}\label{def:left_module_basis_projection}
  Let \( M \) be a left \( R \)-module and let \( B \) be a basis of \( M \). For each \( b \in B \), we define the \text{coordinate projection functional} \( \pi_b: M \to R \) that gives us the unique coefficient in the basis decomposition. Thus, for every \( x \in M \) we have
  \begin{equation*}
    x = \sum_{b \in B} \pi_b(x) b.
  \end{equation*}

  The sum is well-defined since only finitely many terms are nonzero.

  When the basis \( B \) is finite and ordered:
  \begin{equation*}
    B = \{ b_1, \ldots, b_n \},
  \end{equation*}
  we also write
  \begin{equation*}
    x = \sum_{i=1}^n x_k b_i.
  \end{equation*}
\end{definition}
\begin{proof}
  By \fullref{thm:left_module_basis_decomposition}, this decomposition is unique given a basis \( B \).
\end{proof}

\begin{proposition}\label{thm:left_module_basis_projections_are_linear}
  The basis projection \hyperref[def:left_module_basis_projection]{maps} are linear.
\end{proposition}
\begin{proof}
  \SubProofOf{def:linear_operator/homogeneity} Let \( t \in R \) and \( x \in M \). We have the unique decompositions
  \begin{balign*}
    x  & = \sum_{b \in B} \pi_b(x) b,  \\
    tx & = \sum_{b \in B} \pi_b(tx) b.
  \end{balign*}

  Since both decompositions have only finitely many terms, their difference also has only finitely many nonzero terms. Thus,
  \begin{equation*}
    0
    =
    tx - tx
    =
    t \left( \sum_{b \in B} \pi_b(x) b \right) - \sum_{b \in B} \pi_b(tx) b
    =
    \sum_{b \in B} (t \pi_b(x) - \pi_b(tx)) b.
  \end{equation*}

  Since the vectors in \( B \) are linearly independent, no nontrivial linear combination can equal the zero vector. Thus, for all \( b \in B \),
  \begin{equation*}
    t \pi_b(x) = \pi_b(tx).
  \end{equation*}

  \SubProofOf{def:linear_operator/additivity} Analogous.
\end{proof}

\begin{theorem}\label{thm:linear_map_iff_function_on_basis}
  Let \( M \) and \( N \) be left \( R \)-modules and let \( B \) be a basis of \( M \). Then there exists a bijection between the \hyperref[def:function]{functions} from \( B \) to \( N \) and the module \hyperref[def:linear_operator]{homomorphisms} from \( M \) to \( N \), such that each linear map is an \hyperref[def:multi_valued_function/restriction]{extension} of the corresponding function.
\end{theorem}
\begin{proof}
  Let \( \varphi: M \to N \) be a homomorphism. Define the function
  \begin{balign*}
     & f: B \to N                 \\
     & f(b) \coloneqq \varphi(b).
  \end{balign*}

  Now define the linear map
  \begin{balign*}
     & \hat \varphi: M \to N                                   \\
     & \hat \varphi(x) \coloneqq \sum_{b \in B} \pi_b(x) f(b).
  \end{balign*}

  Since the projections \( \pi_b(x) \) are linear functions by \fullref{thm:left_module_basis_projections_are_linear} and since we only use the value of \( f \) on fixed vectors, it follows that \( \hat \varphi \) is also linear.

  It remains to show that \( \varphi = \hat \varphi \). For each \( x \in M \), by linearity of \( \varphi \) we have
  \begin{equation*}
    \hat \varphi(x)
    =
    \sum_{b \in B} \pi_b(x) f(b)
    =
    \sum_{b \in B} \pi_b(x) \varphi(b)
    =
    \varphi(x).
  \end{equation*}
\end{proof}

\begin{remark}\label{rem:linear_map_iff_function_on_basis}
  \Fullref{thm:linear_map_iff_function_on_basis} is very powerful in that is allows to study linear maps given their value at only a small subset of vectors. The connection between multilinear \hyperref[def:multilinear_function]{maps} and \hyperref[def:left_module_tensor_product]{tensors} is based on this idea.
\end{remark}

\begin{corollary}\label{thm:linear_maps_agree_on_free_module_if_they_agree_on_basis}
  If two linear maps from the free left module \( M \) to \( N \) agree on a basis of \( M \), they agree on the whole module.
\end{corollary}

\begin{proposition}\label{thm:left_module_basis_cardinality}\mcite{ProofWiki:bases_of_free_module_have_same_cardinality}
  All bases in a free left module over a commutative unital ring have the same cardinality.
\end{proposition}

\begin{definition}\label{def:left_module_tensor_product}\mcite[574]{Knapp2016BasicAlgebra}
  Let \( R \) be a unital ring. Let \( M \) be a right \( R \)-module and \( N \) be a left \( R \)-module. Define the free \hyperref[def:free_abelian_group]{abelian group} \( G \) generated by the basis \( M \times N \), that is,
  \begin{equation*}
    G \coloneqq \oplus_{(m,n) \in M \times N} \BbbZ.
  \end{equation*}

  Denote by \( e_{m,n} \) the \( (m,n) \)-th basis vector and by \( t_{m,n} \) the \( (m,n) \)-th coordinate of \( t \in G \) (we can have only a finite amount of nonzero coordinates since \( G \) is a direct sum).

  We can regard \( G \) as a left \( R \)-module with scalar multiplication given by
  \begin{equation*}
    (r t)_{(m,n)} \coloneqq t_{(rm,n)}.
  \end{equation*}

  Let \( H \) be the submodule of \( G \) generated by
  \begin{itemize}
    \item \( e_{(m_1 - m_2, n)} - e_{(m_1,n)} - e_{(m_2,n)} \), \( m_1, m_2, n \in G \)
    \item \( e_{(m, n_1 - n_2)} - e_{(m,n_1)} - e_{(m,n_2)} \), \( m, n_1, n_2 \in G \)
    \item \( e_{(rm,n)} - e_{(m,rn)} \), \( m, n \in G \) and \( r \in R \)
  \end{itemize}

  Define the \term{tensor product} of \( M \) and \( N \) to be the \( R \)-module
  \begin{equation*}
    M \otimes N \coloneqq G / H.
  \end{equation*}
\end{definition}

\begin{theorem}\label{thm:tensor_product_universal_property}\mcite[thm. 10.18]{Knapp2016BasicAlgebra}
  Let \( R \) be a unital ring. Let \( M \) be a right \( R \)-module and \( N \) be a left \( R \)-module and let \( M \otimes N \) be their \hyperref[def:left_module_tensor_product]{tensor product} with \( q: M \times N \to M \otimes N \) being the corresponding quotient map.

  The tensor product \( M \otimes N \) satisfies the following universal mapping property: for every \( R \)-module \( K \) and any bilinear \hyperref[def:multilinear_function]{map} \( f: M \times N \to K \) there exists a unique map \( \hat f: M \otimes N \to K \) such that
  \begin{equation*}
    f = \hat f \circ q,
  \end{equation*}
  that is, the following diagram commutes:

  \begin{alignedeq}\label{thm:tensor_product_universal_property/diagram}
    \text{\todo{Add diagram}}\iffalse\begin{mplibcode}
      beginfig(1);
      input metapost/graphs;

      v1 := thelabel("$M \times N$", origin);
      v2 := thelabel("$M \otimes N$", (2, 0) scaled u);
      v3 := thelabel("$K$", (1, -1) scaled u);

      a1 := straight_arc(v1, v2);
      a2 := straight_arc(v2, v3);
      a3 := straight_arc(v1, v3);

      draw_vertices(v);
      draw_arcs(a);

      label.top("$q$", straight_arc_midpoint of a1);
      label.lrt("$\hat f$", straight_arc_midpoint of a2);
      label.llft("$f$", straight_arc_midpoint of a3);
      endfig;
    \end{mplibcode}\fi
  \end{alignedeq}
\end{theorem}

\begin{proposition}\label{thm:tensor_product_with_underlying_ring}\mcite[677]{Knapp2016BasicAlgebra}
  Let \( R \) be a unital ring (regarded as a right module) and \( B \) be a left \( R \)-module. Their \hyperref[def:left_module_tensor_product]{tensor product} satisfies
  \begin{equation*}
    R \otimes M \cong M.
  \end{equation*}
\end{proposition}

\begin{definition}\label{def:algebra_over_ring}\mcite[408]{Knapp2016BasicAlgebra}
  Let \( R \) be a commutative unital ring. We say that the left \( R \)-module \( A \) is an \( R \)-\term{algebra} if we define an additional bilinear \term{vector multiplication} operation
  \begin{equation*}
    \odot: A \times A \to A
  \end{equation*}
  such that for \( x, y \in M \) and \( t \in R \)
  \begin{equation*}
    t \cdot (x \odot y) = (t \cdot x) \odot y = x \odot (t \cdot y).
  \end{equation*}

  Both vector and scalar multiplication are usually denoted by juxtaposition.

  If \( \odot \) is associative, commutative, unital or invertible, we add this prefix to \( A \), e.g. A is a commutative algebra if \( \odot \) is commutative.
\end{definition}

\begin{proposition}\label{thm:functions_over_ring_form_algebra}
  Let \( X \) be an arbitrary nonempty set and \( R \) be a commutative unital ring. Define
  \begin{equation*}
    A \coloneqq \cat{Set}(X, R)
  \end{equation*}
  to be the set of all functions from \( X \) to \( R \) (see \fullref{def:category_of_small_sets}). Then \( A \) is an \( R \)-algebra with the operations being defined pointwise, that is,
  \begin{balign*}
    [f + g](x)     & \coloneqq f(x) + g(x)     \\
    [f \odot g](x) & \coloneqq f(x) \circ g(x) \\
    [rf](x)        & \coloneqq r f(x)
  \end{balign*}

  We call the algebra \( A \) the \term{algebra of functions} from \( X \) to \( R \).

  If \( X \) itself has a ring structure, we consider the set of ring \hyperref[thm:ring_homomorphism_simpler_conditions]{homomorphisms}
  \begin{equation*}
    \cat{Ring}(X, R),
  \end{equation*}
  which is usually a strict subset of \( \cat{Set}(X, R) \). This set is usually denoted by \( \hom(X, R) \).

  If \( R \) is a module, but not necessarily a ring, then \( \cat{Set}(X, R) \) is a only module since we do not necessarily have multiplication. See \fullref{def:linear_operator}.
\end{proposition}

\begin{definition}\label{def:homogenous_function}
  Let \( M \) and \( N \) be left \( R \)-modules. We say that the function \( f: M \to N \) is homogeneous with degree \( n \) if for all \( t \in R \) and \( x \in M \) we have
  \begin{equation*}
    f(t x) = t^n f(x).
  \end{equation*}
\end{definition}

\begin{definition}\label{def:left_module_annihilator}\mcite[30]{КоцевСидеров2016}
  Fix a subset \( S \subseteq M \) of a left \( R \)-module \( M \). We define the \term{annihilator} of \( S \) as the ideal
  \begin{equation*}
    \op{ann}(S) \coloneqq \{ r \in R \colon rS = \{ 0 \} \}.
  \end{equation*}
\end{definition}


% Commutative algebra
\section{Commutative algebra}\label{sec:commutative_algebra}

\begin{remark}\label{rem:polynomial_commutative_ring}
  In this whole section, \( R \) will refer to a nontrivial commutative unital \hyperref[def:semiring/commutative_unital_ring]{ring}.

  Since \( R \) is commutative, left and right modules over \( R \) are equivalent. We will only refer to either of them as simply \enquote{modules}.
\end{remark}

\subsection{Euclidean division}\label{subsec:euclidean_division}

\begin{definition}\label{def:commutative_ring_division}
  Let \( a, b \in R \). We say that \( b \) is a \term{divisor} of \( a \) or that \( a \) is a \term{factor} of \( b \) and write \( b \mid a \) if there exists \( c \in R \) such that \( a = bc \).

  We say that \( b \) is a \term{trivial} divisor of \( a \) if \( a = bc \) and either \( b \) or \( c \) is invertible.
\end{definition}

\begin{definition}\label{def:modulo}
  Fix an ideal \( I \) of \( R \).

  If for some \( a, b \in R \) we have \( a - b \in I \), we say that \( a \) and \( b \) are \term{congruent modulo} \( I \) and write
  \begin{equation*}
    a \cong b \pmod I.
  \end{equation*}

  If the ideal \( I \) is generated by the single element \( n \) and if \( a = b \mod I \), we say that \( a \) and \( b \) are congruent modulo \( n \) and write
  \begin{equation*}
    a \cong b \pmod n.
  \end{equation*}
\end{definition}

\begin{definition}\label{def:greatest_common_denominator}
  The \term{greatest common denominator}  of \( a \) and \( b \) is defined, if it exists and is unique, as
  \begin{equation*}
    \gcd(a, b) \coloneqq \max_{\cdot \mid \cdot} \{ c \in R : c \mid a \wedge c \mid b \},
  \end{equation*}
  where the maximum is taken with respect to the divisibility partial order.
\end{definition}

\begin{definition}\label{def:factorization_in_ring}
  A factorization of \( x \in R \) is a finite sequence \( p_1, \ldots, p_n \) of \hyperref[def:irreducible_ring_element]{irreducible} elements with multiplicities \( k_1, \ldots, k_n \) so that
  \begin{equation*}
    x = e p_1^{k_1} p_2^{k_2} \cdots p_n^{k_n},
  \end{equation*}
  where \( e \) is a unit in \( R \).
\end{definition}

\begin{definition}\label{def:euclidean_domain}\mcite\cite{nLab:euclidean_domain}
  Let \( R \) be an integral \hyperref[def:semiring/integral_domain]{domain}. Multiplication is not invertible in general, but we can instead define \term{Euclidean division} with remainders.

  We endow \( R \) with an additional function \( \delta: R \to \BbbZ_{\geq 0} \). Let \( a, b \in R \). If there exists a pair \( (q, r) \) such that
  \begin{equation*}
    a = bq + r
  \end{equation*}
  holds and either \( r = 0 \) or \( \delta(r) < \delta(b) \), we say that \( (R, \delta) \) is an \term{Euclidean domain}.

  We say that \( b \) \term{divides} \( a \) with \term{quotient} \( q \) and \term{remainder} \( r \).

  If the pair \( (q, r) \) is unique, we use the special notation
  \begin{balign*}
    q & = \quot(a, b),                    \\
    r & = \rem(a, b) = a - b \quot(a, b).
  \end{balign*}
\end{definition}

\begin{algorithm}\label{alg:euclidean_algorithm}
  Let \( R \) be an Euclidean domain. Fix \( a, b \in R \) with \( b \neq 0 \). The \term{Euclidean algorithm} for finding \( \gcd(a, b) \) proceeds as follows:
  \begin{algenum}
    \ilabel{alg:euclidean_algorithm/initialization} Define \( r_0 \coloneqq a \) and \( r_1 \coloneqq b \).
    \ilabel{alg:euclidean_algorithm/step} Starting with \( i = 2 \), obtain \( q_i \) and \( r_i \) from \hyperref[def:semiring/euclidean_domain]{division}
    \begin{equation*}
      r_{i-2} = r_{i-1} q_i + r_i.
    \end{equation*}

    If \( r_i = 0 \), halt the algorithm with result \( \gcd(a, b) = r_i \).

    Otherwise, proceed by incrementing \( i \) and repeating this step.
  \end{algenum}
\end{algorithm}
\begin{proof}
  Euclidean division ensures that \( \delta(r_i) < \delta(r_{i-1}) \) on every step. Thus the algorithm terminates at some point. Denote by \( n \) the (minimum) number of steps necessary to obtain \( r_n = 0 \).

  We show by induction\IND on \( k < n \) that \( r_n \) divides \( r_{n-k} \). The case \( k = 0 \) is obvious since \( r_n \) divides itself.

  Assume that \( r_n \) divides \( r_{n-i} \) for \( 0 < i < k \). Now since
  \begin{equation*}
    r_{n-k} = r_{n-(k-1)} q_{n-(k-2)} + r_{n-(k-2)}
  \end{equation*}
  and both of the terms on the right-hand side are factors of \( r_n \), the left-hand side \( r_{n-k} \) is also a factor.

  We conclude that \( r_n \) divides both \( r_{n-(n-1)} = r_1 = b \) and \( r_{n-n} = r_0 = a \).

  Furthermore, \( r_n \) is the greatest common divisor of \( a \) and \( b \). Indeed, assume that there exists \( d \in R \) such that \( r_n \mid d \) and both \( d \mid a \) and \( d \mid b \) hold. But he have
  \begin{equation*}
    a = b q_2 + r_2,
  \end{equation*}
  hence \( d \mid r_2 \). Proceeding by induction\IND, we obtain that \( d \mid r_n \). But we assumed that \( r_n \mid d \), therefore \( r_n = d \) and \( r_n \) is the greatest common divisor of \( a \) and \( b \).
\end{proof}

\begin{proposition}\label{thm:euclidean_domain_is_pid}
  Every Euclidean \hyperref[def:semiring/euclidean_domain]{domain} is a principal ideal \hyperref[def:semiring/principal_ideal_domain]{domain}.
\end{proposition}
\begin{proof}
  Fix an ideal \( I \) of the Euclidean domain \( R \). By \fullref{thm:natural_numbers_are_well_ordered}, the set \( \delta(I) \) has a minimum. Choose\AOC an element \( m \in I \) such that \( \delta(m) = \min \delta(I) \). We will prove that \( I = \braket m \).

  Let \( x \in I \). We divide it by \( m \) to obtain
  \begin{equation*}
    x = mq + r,
  \end{equation*}
  such that either \( r = 0 \) or \( \delta(r) < \delta(m) \). Since both \( x \) and \( m \) are in \( I \), we have \( r = mq - x \in I \). But \( m \) minimizes \( \delta \) over \( I \), thus \( \delta(m) \leq \delta(r) \), which contradicts \( \delta(r) < \delta(m) \). Therefore \( r = 0 \) and
  \begin{equation*}
    x = mq,
  \end{equation*}
  which implies that \( x \in \braket m \). This proves \( I \subseteq \braket m \).

  Let \( x \in \braket m \), that is, \( x = mq \) for some \( q \in R \). Since \( I \) is an ideal and \( m \in I \), all multiples of \( m \) are in \( I \) and thus \( x \in I \). This proves \( \braket m \subseteq I \).

  We have now obtained \( \braket m = I \). Since \( I \) was an arbitrary ideal, we conclude that every ideal in \( R \) is principal.
\end{proof}

\begin{proposition}\label{thm:pid_is_ufd}
  Every principal ideal \hyperref[def:semiring/principal_ideal_domain]{domain} is a unique factorization \hyperref[def:semiring/unique_factorization_domain]{domain}.
\end{proposition}

\begin{theorem}[Bezout's identity]\label{thm:bezout_identity}
  Let \( R \) be a principal ideal domain. Let \( a, b \in R \) with \( b \neq 0 \). Then \( \gcd(a, b) \) exists and, furthermore, there exist \( x, y \) such that
  \begin{equation*}
    ax + by = \gcd(a, b).
  \end{equation*}
\end{theorem}
\begin{proof}
  We will first prove the existence of \( \gcd(a, b) \). Define the ideal
  \begin{equation*}
    I \coloneqq \braket{a, b}.
  \end{equation*}

  Since every ideal is principal, there exists \( g \in I \) such that \( I = \braket g \). Let \( x, y \in R \) such that
  \begin{equation*}
    g = ax + by.
  \end{equation*}

  Note that \( g \) is a divisor of both \( a \) and \( b \) because \( a \in I \) and \( b \in I \). We will show that it is the greatest divisor. Let \( d \) be another divisor of both \( a \) and \( b \) such that \( g \mid d \) (or, equivalently, \( d \in \braket g \)).

  Let \( a = da' \) and \( b = db' \). We have
  \begin{equation*}
    g = ax + by = d(a'x + b'y),
  \end{equation*}
  which implies that \( d \mid g \). Thus \( g = d \) and \( g \) is a greatest common denominator.
\end{proof}

\begin{algorithm}\label{alg:extended_euclidean_algorithm}
  Let \( R \) be an Euclidean domain. Fix \( a, b \in R \) with \( b \neq 0 \). We will explicitly find \( x \) and \( y \) so that \fullref{thm:bezout_identity} is satisfied:
  \begin{equation*}
    \gcd(a, b) = ax + by.
  \end{equation*}

  Let \( r_0, r_1, \ldots, r_n \) be the sequence of remainders from \fullref{alg:euclidean_algorithm}. The \term{extended Euclidean algorithm} proceeds as follows:

  \begin{algenum}
    \ilabel{alg:extended_euclidean_algorithm/initialization} For \( i = 2 \), define
    \begin{balign*}
      x_2 & \coloneqq 1,    \\
      y_2 & \coloneqq -q_2.
    \end{balign*}

    \ilabel{alg:extended_euclidean_algorithm/step} For \( i = 2, \ldots, n \), define
    \begin{balign*}
      x_i & \coloneqq x_{i-2} - x_{i-1} q_i, \\
      y_i & \coloneqq y_{i-2} - y_{i-1} q_i.
    \end{balign*}

    \ilabel{alg:extended_euclidean_algorithm/completion} Halt the algorithm with result
    \begin{balign*}
      x & \coloneqq x_n, \\
      y & \coloneqq y_n.
    \end{balign*}
  \end{algenum}
\end{algorithm}
\begin{proof}
  We will prove with induction\IND on \( i = 2, \ldots, n \) that
  \begin{equation*}
    r_i = ax_i + by_i.
  \end{equation*}

  \begin{reflist}
    \iref{alg:extended_euclidean_algorithm/initialization} For \( i = 2 \), we have
    \begin{balign*}
      r_0       & = r_1 q_2 + r_2, \\
      a         & = b q_2 + r_2,   \\
      a - b q_2 & = r_2,
    \end{balign*}
    that is, \( r_2 = a + b (-q_2) \).

    \iref{alg:extended_euclidean_algorithm/step} For \( i = 3, \ldots, n \), we have
    \begin{balign*}
      r_{i-2}                                               & = r_{i-1} q_i + r_i,                 \\
      a x_{i-2} + b y_{i-2}                                 & = (a x_{i-1} + b y_{i-1}) q_i + r_i, \\
      a (x_{i-2} - x_{i-1} q_i) + b (y_{i-2} - y_{i-1} q_i) & = r_i.
    \end{balign*}

    \iref{alg:extended_euclidean_algorithm/completion} Since \( r_n = \gcd(a, b) \), we conclude that
    \begin{equation*}
      \gcd(a, b) = a x_n + b y_n.
    \end{equation*}
  \end{reflist}
\end{proof}

\subsection{Polynomials}\label{subsec:polynomials}

\begin{remark}\label{rem:polynomials_vs_polynomial_functions}
  Polynomial are a purely algebraic framework for describing certain well-behaved functions. The link between a polynomial and a function is given by \fullref{def:polynomial_function}, but in general these are different concepts. The link between a polynomial and its conventional symbolic expression is given by \fullref{def:algebra_of_polynomials}.

  \Fullref{thm:polynomial_embedding_behavior} allow us to ignore this distinction in certain special cases.
\end{remark}

\begin{definition}\label{def:polynomial}\mcite[149]{Knapp2016BasicAlgebra}
  A \term{polynomial} \( p \) over \( R \) is a sequence of members of \( R \) called \term{coefficients},
  \begin{equation*}
    p \coloneqq ( a_0, a_1, a_2, \ldots ) \subseteq R,
  \end{equation*}
  such that only finitely many coefficients are nonzero. We can regard \( p \) as a \hyperref[def:topological_net]{net} over \( \BbbZ_{\geq 0} \) with finite \hyperref[def:function_support]{support}.

  \begin{thmenum}
    \thmitem{def:polynomial/zero_polynomial} An exception to most rules is the \term{zero polynomial}, all of whose coefficients are zeroes.

    \thmitem{def:polynomial/leading_coefficient} The last nonzero coefficient of a nonzero polynomial is called the \term{leading coefficient} and is denoted by \( \op{LC}(p) \). If \( \op{LC}(p) = 1 \), we call the polynomial \term{2}.

    \thmitem{def:polynomial/degree} The zero-based index of the leading coefficient is called the \term{degree} of the polynomial as is denoted by \( \deg(p) \). That is, if only \( a_0 \) is nonzero, then \( \deg(p) = a_0 \). The degree of the zero polynomial is left undefined.

    \thmitem{def:polynomial/monomial} A \term{monomial} is a polynomial with only one nonzero element.

    \thmitem{def:polynomial/degree_names} Polynomials of degree \( n \) with special names include
    \begin{itemize}
      \item \term{constant} for the zero polynomial or if \( n = 0 \).
      \item \term{linear} if \( n = 1 \).
      \item \term{quadratic} if \( n = 2 \).
      \item \term{cubic} if \( n = 3 \).
      \item \term{quartic} if \( n = 4 \).
      \item \term{quintic} if \( n = 5 \).
    \end{itemize}
  \end{thmenum}
\end{definition}

\begin{definition}\label{def:algebra_of_polynomials}
  Denote by \( R[X] \) the set of polynomials over \( R \). Note that it is bijective with \( c_{00} \) and we can inherit pointwise addition and scalar multiplication from there. That is,

  \begin{thmenum}
    \thmitem{def:algebra_of_polynomials/addition} We define \term{polynomial addition} point as
    \begin{balign*}
       & +: R[X] \times R[X] \to R[X]                                                     \\
       & (a_0, a_1, \ldots) + (b_0, b_1, \ldots) \coloneqq (a_0 + b_0, b_0 + b_1, \ldots)
    \end{balign*}

    \thmitem{def:algebra_of_polynomials/scalar_multiplication} We define \term{scalar multiplication} as
    \begin{balign*}
       & \cdot: R[X] \times R[X] \to R[X]                            \\
       & t \cdot (a_0, a_1, \ldots) \coloneqq (t a_0, t b_1, \ldots)
    \end{balign*}

    \thmitem{def:algebra_of_polynomials/polynomial_multiplication} In order to make \( R[X] \) into an \hyperref[def:algebra_over_ring]{algebra}, we define \term{Cauchy multiplication} \( \odot: R[X] \times R[X] \to R[X] \) as follows: if \( (a_0, a_1, \ldots) \) and \( (b_0, b_1, \ldots) \) are polynomials, their product is defined to be the polynomial with coefficients
    \begin{equation}
      c_l \coloneqq \sum_{i+j=l} a_i b_j, l = 0, 1, \ldots.
    \end{equation}

    Polynomial multiplication is bilinear, associative and commutative.
  \end{thmenum}

  We will implicitly use the canonical embedding \( \iota: R \to R[X] \), which sends an element \( r \) of \( R \) into the constant polynomial \( p(X) \coloneqq r \).

  We usually refer to \( R[X] \) as the \term{polynomial ring}, especially since scalar multiplication is the same multiplication by constants.
\end{definition}

\begin{remark}\label{rem:polynomial_symbolic_expression}
  We choose a \hyperref[def:language]{symbol}, usually \( X \), called an \term{indeterminate}, and give it \hyperref[def:first_order_semantics/satisfiability]{semantics} using the \hyperref[def:polynomial/monomial]{monomial}
  \begin{equation*}
    X \coloneqq (0, 1, 0, 0, \ldots)
  \end{equation*}
  (the assignment should be understood as \enquote{defining the interpretation of \( X \) in a certain first-order language}).

  Using the definition of multiplication, we see that the coefficients \( c_0, c_1, \ldots \) of \( X^2 = X \cdot X \) can be expressed in terms of the coefficients \( a_0, a_1, \ldots \) of \( X \) as
  \begin{balign*}
    c_0 & = a_0 \cdot a_0 = 0                                                                 \\
    c_1 & = a_0 \cdot a_1 + a_1 \cdot a_0 = 0 + 0 = 0                                         \\
    c_2 & = a_0 \cdot a_2 + a_1 \cdot a_1 + a_2 \cdot a_0 = 0 + 1 + 0 = 1                     \\
    c_3 & = a_0 \cdot a_3 + a_1 \cdot a_2 + a_2 \cdot a_1 + a_0 \cdot a_3 = 0 + 0 + 0 + 0 = 0 \\
    c_4 & = \cdots = 0                                                                        \\
    c_5 & = \cdots = 0                                                                        \\
    \vdots
  \end{balign*}

  Proceeding by induction\IND, we see that \( X^k \) corresponds to the sequence
  \begin{equation*}
    (\underbrace{0, 0, \ldots, 0, 0}_{k \text{times}}, 1, 0, 0, \ldots).
  \end{equation*}

  We define \( X^0 \coloneqq 1 \).

  It is conventional to then write a polynomial of degree \( n \) as the \hyperref[def:language]{expression}
  \begin{equation}\label{eq:rem:polynomial_symbolic_expression}
    p(X) = a_n X^n + a_{n-1} X^{n-1} + \ldots + a_2 X^2 + a_1 X + a_0 = \sum_{i=0}^n a_i X^i
  \end{equation}
  and the zero polynomial as
  \begin{equation*}
    p(X) \coloneqq 0.
  \end{equation*}

  We use capital letters to highlight that this is not a function - see \fullref{def:polynomial_function}. We say that \( p(X) \) is a polynomial in one indeterminate.
\end{remark}

\begin{proposition}\label{thm:polynomial_degree_properties}
  The polynomial degree has the following basic properties:
  \begin{thmenum}
    \thmitem{thm:polynomial_degree_properties/sum} For nonzero polynomials \( p, q \in R[X] \) with \( p \neq -q \), we have
    \begin{equation*}
      \deg (p + q) \leq \max \{ \deg p, \deg q \}.
    \end{equation*}

    \thmitem{thm:polynomial_degree_properties/product} For nonzero polynomials \( p, q \in R[X] \) with \( pq \neq 0 \), we have
    \begin{equation*}
      \deg (pq) \leq \deg p + \deg q,
    \end{equation*}
    with equality holding if \( R \) is an integral domain.

    The requirement that \( pq \neq 0 \) may also be dropped if \( R \) is an integral domain as per \fullref{thm:polynomials_over_integral_domain_are_integral_domain}.
  \end{thmenum}
\end{proposition}
\begin{proof}
  Fix nonzero polynomials
  \begin{align}
    p(X) &\coloneqq \sum_{k=0}^n a_k X^k, \label{eq:thm:polynomial_degree_properties/p} \\
    q(X) &\coloneqq \sum_{k=0}^m b_k X^k. \label{eq:thm:polynomial_degree_properties/q}
  \end{align}

  \SubProofOf{thm:polynomial_degree_properties/sum} Additionally assume that \( p \neq -q \) since otherwise \( p + q = 0 \) and \( \deg(p + q) \) is undefined. Thus there exists at least one index \( k = 1, 2, \ldots \) so that \( a_k \neq b_k \). Denote by \( k_0 \) the largest such index (only finitely many are nonzero). Then
  \begin{equation*}
    a_k = b_k = 0 \T{for} k > k_0.
  \end{equation*}

  Therefore \( \deg(p + q) = k_0 \). Note that \( k_0 \) cannot exceed both \( \deg p \) and \( \deg q \) because it corresponds to a nonzero coefficient. Thus \( k \leq \max\{ \deg p, \deg q \} \).

  \SubProofOf{thm:polynomial_degree_properties/product} By assumption \( pq \neq 0 \) so there exists at least one nonzero coefficient, say \( c_k \). Obviously \( k \leq \deg p + \deg q \) since the highest possible degree of \( pq \) is \( \deg p + \deg q \). Thus
  \begin{equation*}
    k = \deg (pq) \leq \deg p + \deg q.
  \end{equation*}

  If \( R \) is an integral domain, the product of nonzero elements is nonzero. Thus the leading coefficient \( \op{LC}(pq) = \op{LC}(p)\op{LC}(q) \) is nonzero and
  \begin{equation*}
    \deg(pq) = \deg p + \deg q.
  \end{equation*}
\end{proof}

\begin{proposition}\label{thm:polynomial_ring_universal_property}\mcite[150]{Knapp2016BasicAlgebra}
  For any nontrivial commutative unital ring \( T \), any unital ring homomorphism \( \varphi: R \to T \) and any \( t \in T \), there exists a unique homomorphism \( \Phi_t: R[X] \to T \) such that \( \iota(1) = t \) and the following diagram commutes:

  \begin{alignedeq}\label{thm:polynomial_ring_universal_property/diagram}
    \includegraphics{figures/thm__polynomial_ring_universal_property__diagram.pdf}
  \end{alignedeq}

  We call the map \( \Phi_t \) a \term{substitution homomorphism}.

  If \( T \) is a superring of \( R \), we call \( \Phi_t \) an \term{evaluation} at \( t \in T \). We also write
  \begin{equation}
    R[t] \coloneqq \Phi_t(R[X]).
  \end{equation}

  This allows us to \term{adjoin} elements from a superring to a subring. See \fullref{ex:polynomial_evaluation_gaussian_integers}.
\end{proposition}
\begin{proof}
  We will first prove uniqueness. Let \( \Phi_t: R[X] \to T \) and \( \Psi_t: R[X] \to T \) are two such homomorphisms and take their difference \( \Theta_t \coloneqq \Phi_t - \Psi_t \).

  Then for \( r \in R \) we have
  \begin{equation*}
    \Theta_t(\iota(r)) = \Psi_t(\iota(r)) - \Psi_t(\iota(r)) = \varphi(r) - \varphi(r) = 0.
  \end{equation*}

  Now since for any polynomial we have \( p = \iota(1) p \) and since \( \Theta_t \) is a homomorphism of rings, we have
  \begin{equation*}
    \Theta_t(p) = \Theta_t(\iota(1) p) = 0 \Theta_t(p) = 0.
  \end{equation*}

  Thus \( \Theta_t = 0 \) and \( \Phi_t = \Psi_t \).

  Now we will show existence. Take a polynomial
  \begin{equation*}
    p = (a_0, a_1, \ldots, a_n, 0, 0, \ldots).
  \end{equation*}

  Define
  \begin{equation*}
    \Phi_t(p) \coloneqq \sum_{i=1}^n \phi(a_0) t^i.
  \end{equation*}

  By additivity, \( \Phi_t: R[X] \to T \) is obviously a homomorphism and \( \Phi_t((0, 1, 0, \ldots)) = t \). Therefore we have proven existence.
\end{proof}

\begin{example}\label{ex:polynomial_evaluation_gaussian_integers}
  The Gaussian \hyperref[def:gaussian_integers]{integers} \( \BbbZ[i] \) are complex numbers with integer real and complex parts. We will now motivate this notation.

  Consider the substitution map \( \Phi_i: \BbbZ[X] \to \BbbC \) for the imaginary unit given by \fullref{thm:polynomial_ring_universal_property}. Let \( p(X) \in \BbbZ[X] \). Then
  \begin{equation*}
    p(i)
    =
    \Phi_i(p)
    =
    \sum_{j=0}^n a_j i^n
    =
    \sum_{\rem(j, 4) = 0}^n a_j - \sum_{\rem(j, 4) = 2}^n a_j + i \left(\sum_{\rem(j, 4) = 1}^n a_j - \sum_{\rem(j, 4) = 3}^n a_j \right).
  \end{equation*}

  This is clearly a Gaussian integer.

  Now fix a Gaussian integer \( z = a + bi \). It can given by the polynomial
  \begin{equation*}
    p_z(X) \coloneqq a + bX.
  \end{equation*}

  It remains to show that multiplication in \( \BbbZ[X] \) is compatible with multiplication in \( \BbbC \). But complex \hyperref[def:complex_numbers]{multiplication} is defined to be compatible with the notation \( a + bi \), that is,
  \begin{equation*}
    (a + bi) (c + di)
    =
    ac + ibc + iad - bd
    =
    (ac - bd) + i(bc + ad).
  \end{equation*}

  Thus the Gaussian integers are precisely the homomorphic image of \( Z[X] \) under \( \Phi_i \).
\end{example}

\begin{proposition}\label{thm:polynomial_ring_units}
  The units of the polynomial ring \( R[X] \) are precisely the units of \( R \).
\end{proposition}
\begin{proof}
  Any unit of \( R \) is obviously a unit of \( R[X] \).

  For the converse, fix a nonzero constant polynomial \( p(X) = r \). In order for it to have an inverse \( q(X) \), we should have
  \begin{equation*}
    1 = p(X) q(X) = r q(X),
  \end{equation*}
  which can only happen if \( q(X) \) is a constant and a multiplicative inverse of \( r \).
\end{proof}

\begin{proposition}\label{thm:polynomial_algebra_basis}
  The polynomial \hyperref[def:algebra_of_polynomials]{algebra} \( R[X] \) has a Hamel \hyperref[def:left_module_hamel_basis]{basis} consisting of all \hyperref[def:polynomial/monomial]{monomials}
  \begin{equation*}
    B \coloneqq \{ 1, X, X^2, X^3, \ldots \}.
  \end{equation*}
\end{proposition}
\begin{proof}
  By \fullref{def:algebra_of_polynomials/addition}, every polynomial can easily be represented as a sum of finitely many monomials.
\end{proof}

\begin{definition}\label{def:polynomial_free_module}
  It is convenient, especially in \hyperref[sec:approximation_theory]{approximation theory}, to work with the free module of polynomials of degree at most \( n \). We define
  \begin{equation*}
    \pi_n(R[X]) \coloneqq \linspan \{ 1, X, \ldots, X^{n-1}, X^n \}.
  \end{equation*}

  We will use \( \pi_n \) and when the ring is clear from the context.
\end{definition}

\begin{proposition}\label{thm:polynomials_over_integral_domain_are_integral_domain}
  If \( R \) is an integral domain, the polynomial ring \( R[X] \) is also an integral domain.
\end{proposition}
\begin{proof}
  Polynomial multiplication inherits its commutativity from multiplication in \( R \). It remains only to show that \( R[X] \) has no zero divisors.

  Fix two polynomials \( p, q \in R[X] \). If either of them is zero, their product \( pq \) is zero.

  Assume that both are nonzero polynomials. The leading coefficient of their product is, by definition of multiplication, \( \op{LC}(pq) = \op{LC}(p) \op{LC}(q) \). Since \( R \) has no zero divisors, then \( \op{LC}(pq) \neq 0 \) and thus \( pq \) is a nonzero polynomial.

  Therefore \( R[X] \) is an integral domain.
\end{proof}

\begin{proposition}\label{thm:polynomials_over_unique_factorization_domain_are_unique_factorization_domain}
  If \( R \) is a unique factorization domain, the polynomial ring \( R[X] \) is also a unique factorization domain.
\end{proposition}

\begin{theorem}[Euclidean division of polynomials]\label{thm:euclidean_division_of_polynomials}\mcite[10]{Knapp2016BasicAlgebra}
  Let \( a, b \in R[X] \) and \( b \) be \hyperref[def:polynomial/leading_coefficient]{monic} (in particular, \( b \neq 0 \)). Then there exist unique polynomials \( q, r \in R[X] \), where \( r \) is either zero or \( \deg r < \deg b \), such that
  \begin{equation*}
    a = bq + r.
  \end{equation*}
\end{theorem}
\begin{proof}
  Let \( a, b \in R[X] \) and \( b \neq 0 \). If \( a = 0 \) or \( \deg a < \deg b \), define
  \begin{balign*}
    q & \coloneqq 0, \\
    r & \coloneqq a.
  \end{balign*}

  In this case, \( \deg r = \deg a < \deg b \).

  Suppose that \( \deg b \leq \deg a \). We will use proof by induction\IND on \( \deg a \). If \( \deg a = 0 \), obviously \( \deg b = 0 \) (thus \( b = 1 \)) and we define
  \begin{balign*}
    q & \coloneqq a, \\
    r & \coloneqq 0.
  \end{balign*}

  In this case, \( r \) is the zero polynomial.

  Assume the result holds for \( \deg a < n \) and let \( \deg a = n, \deg b = m \). Then there exists a polynomial \( \hat a(X) \) that is either zero or \( \deg \hat a = n - 1 \) such that
  \begin{equation*}
    a(X) = a_n a^n + \hat a(X).
  \end{equation*}

  Analogously, we find \( \hat b(X) \) that is either zero or \( \deg \hat b = m - 1 \) such that
  \begin{equation*}
    b(X) = X^m + \hat b(X).
  \end{equation*}

  Thus
  \begin{balign*}
    \hat r(X)
     & \coloneqq
    a(X) - b(X) a_n X^{n-m}
    =            \\ &=
    a_n X^n + \hat a(X) - (b_m X^m + \hat b(X)) a_n X^{n-m}
    =            \\ &=
    a_n X^n + \hat a(X) - a_n X^n - \hat b(X) a_n X^{n-m}
    =            \\ &=
    \hat a(X) - \hat b(X) a_n X^{n-m}.
  \end{balign*}

  Therefore \( \hat r(X) \) is either the zero polynomial (in which case we define \( r(X) \coloneqq \hat r(X) \)) or \( \deg \hat r \leq n - 1 \). In the latter case, we can divide \( \hat r \) by \( b \) to obtain \( \hat q(X) \) and \( r(X) \) such that
  \begin{equation*}
    \hat r(X) \coloneqq b(X) \hat q(X) + r(X),
  \end{equation*}
  where \( r = 0 \) or \( \deg r < \deg b \).

  Substitute into the definition of \( \hat r(X) \):
  \begin{balign*}
    \hat r(X)                                         & = a(X) - b(X) a_n X^{n-m} \\
    b(X) \hat q(X) + r(X)                             & = a(X) - b(X) a_n X^{n-m} \\
    b(X) \left(\hat q(X) - a_n X^{n-m} \right) + r(X) & = a(X).
  \end{balign*}

  Define
  \begin{equation*}
    q(X) \coloneqq \hat q(X) - a_n X^{n-m}.
  \end{equation*}

  We have obtained polynomials \( r(X) \) and \( q(X) \) where \( r(X) \) is either zero or \( \deg r < \deg b \).

  It remains only to show uniqueness. Suppose that
  \begin{equation*}
    a = bq + r = bq' + r'.
  \end{equation*}

  If \( r - r' \) is the zero polynomial, so is \( q - q' \) and uniqueness follows.

  If \( r - r' \) is not zero, then neither is \( q - q' \). Then \( b(q - q') = -(r - r') \) and
  \begin{equation*}
    \deg b + \deg(q - q') = \deg[b (q - q')] = \deg(r - r') \leq \max(\deg r, \deg r') < \deg b,
  \end{equation*}
  which is a contradiction.

  This proves uniqueness.
\end{proof}

\begin{corollary}\label{thm:polynomials_over_field_are_euclidean_domain}\mcite[10]{Knapp2016BasicAlgebra}
  The polynomial \hyperref[def:semiring/integral_domain]{ring} \( \BbbK[X] \) over a field \( \BbbK \) is an \hyperref[def:semiring/euclidean_domain]{Euclidean} domain with \( \delta(p) \coloneqq \deg p \). Furthermore, the remainder and quotient are unique.
\end{corollary}
\begin{proof}
  By \fullref{thm:polynomials_over_integral_domain_are_integral_domain}, \( \BbbK[X] \) is an integral domain.

  To show that it is Euclidean, fix two polynomials \( a, b \in \BbbK[X] \) with \( b \neq 0 \). We use \fullref{thm:euclidean_division_of_polynomials} to perform Euclidean division of \( a \) by the monic polynomial \( \frac {b} {\op{LC}(b)} \) and obtain polynomials \( q, r \), where either \( r = 0 \) or \( \deg r < \deg b \), such that
  \begin{equation*}
    a = \frac {b} {\op{LC}(b)} q + r.
  \end{equation*}

  Instead of dividing \( b \) by its leading coefficient \( \op{LC}(b) \), we can divide \( q \) and thus obtain the required factorization.
\end{proof}

\begin{definition}\label{def:polynomial_function}
  Denote by \( \cat{Set}(R) \) the ring of all functions over \( R \) (see \fullref{thm:functions_over_ring_form_algebra}). Define the unital ring homomorphism
  \begin{balign*}
     & \Phi: R[X] \to \cat{Set}(R)                                                                      \\
     & \Phi((a_0, a_1, \ldots, a_n, 0, 0, \ldots)) \coloneqq \left( x \to \sum_{i=0}^n a_i x^n \right),
  \end{balign*}
  which constructs a \term{polynomial function} from a polynomial. This map is not injective in general and multiple polynomials may be equivalent as \hyperref[def:function]{functions}.

  If we want to highlight that we are referring to a polynomial function rather than a polynomial \( p(X) \), we usually use a lowercase letter for the variable, i.e.
  \begin{equation*}
    p(x) = \sum_{i=0}^n a_i x^i.
  \end{equation*}

  If the ring is \hyperref[def:finite_set]{infinite}, \fullref{thm:polynomial_embedding_behavior} allows us to set aside the difference between polynomials and polynomial functions and we usually identify the two when working over \( \BbbR \) or \( \BbbC \).
\end{definition}
\begin{proof}
  It is obvious that \( \Phi \) is a homomorphism of the additive groups of \( R \) and \( \cat{Set}(R) \).

  We will prove that multiplication of polynomials corresponds to multiplication of polynomial functions. Take
  \begin{balign*}
    p(X) & \coloneqq \sum_{i=0}^n a_i X^i, \\
    q(X) & \coloneqq \sum_{j=0}^m b_j X^j.
  \end{balign*}

  For their product \( s(X) = \sum_{i=0}^{n + m} c_i X^i \) by definition we have
  \begin{equation*}
    c_l = \sum_{i+j=l} a_i b_j, l = 0, 1, \ldots, n + m.
  \end{equation*}

  We need to show that for any \( r \in R \),
  \begin{equation*}
    \Phi_r(p) \Phi_r(q) = \Phi_r(s).
  \end{equation*}

  Using associativity and commutativity of multiplication and distributivity of multiplication over addition, we obtain
  \begin{balign*}
    \Phi_r(p) \Phi_r(q)
     & =
    \left( \sum_{i=0}^n a_i r^i \right) \left( \sum_{j=0}^m b_j r^j \right)
    =    \\ &=
    \sum_{i=0}^n a_i r^i \left( \sum_{j=0}^m b_j r^j \right)
    =    \\ &=
    \sum_{i=0}^n a_i r^i \left( \sum_{j=0}^m b_j r^j \right)
    =    \\ &=
    \sum_{i=0}^n \sum_{j=0}^m a_i r^i b_j r^j
    =    \\ &=
    \sum_{i=0}^n \sum_{j=0}^m a_i b_j r^{i + j}
    =    \\ &=
    \sum_{l=0}^{n + m} \sum_{i+j=l} a_i b_j r^l
    =    \\ &=
    \sum_{l=0}^{n + m} c_l r^l
    =
    \Phi_r(s),
  \end{balign*}
  where we have used that \( a_i = 0, i > n \) and \( b_j = 0, j > m \).

  Thus \( \Phi: R[X] \to \fun(R) \) is indeed a homomorphism of rings.
\end{proof}

\begin{definition}\label{def:formal_power_series}
  If we extend \fullref{def:polynomial} to allow for polynomial with infinite terms (that is, allow for the sequence to have infinitely many nonzero items), we obtain a set \( R\Bracks{X} \) which we call the \term{formal power series} over \( R \).

  Note that the operations in \fullref{def:algebra_of_polynomials} are well defined and still make \( R\Bracks{X} \) into an algebra. Evaluation (see \fullref{thm:polynomial_ring_universal_property} and \fullref{def:polynomial_function}) is problematic, however, since algebraic operations are finitary by nature. In practice, we use a topology over \( R \) and speak of convergent and divergent power series.
\end{definition}

\begin{theorem}[Newton's binomial theorem]\label{thm:binomial_theorem}
  \begin{equation*}
    (X + Y)^n = \sum_{k=0}^n \binom n k X^k Y^{n-k}
  \end{equation*}
\end{theorem}
\begin{proof}
  We use induction\IND on \( n \). For \( n = 0 \), the theorem trivially holds. Assume that the theorem holds for \( 1, \ldots, n \). Then
  \begin{balign*}
    (X + Y)^{n+1}
     & =
    X (X + Y)^n + Y (X + Y)^n
    =    \\ &=
    \sum_{k=0}^n \binom n k X^{k+1} Y^{n-k} + Y \sum_{k=0}^n \binom n k X^k Y^{n-k}
    =    \\ &=
    X^{n+1} + Y \sum_{k=0}^{n-1} \binom n k X^{k+1} Y^{n-(k+1)} + Y \sum_{k=0}^n \binom n k X^k Y^{n-k}
    =    \\ &=
    X^{n+1} + Y \left[ \sum_{k=1}^n \binom n {k-1} X^k Y^{n-k} + Y^n \sum_{k=1}^n \binom n k X^k Y^{n-k} \right] + Y^{n+1}
    =    \\ &\overset {\ref{thm:pascals_identity}} =
    X^{n+1} + Y \sum_{k=1}^n \binom {n+1} k X^k Y^{n-k} + Y^{n+1}
    =    \\ &=
    \sum_{k=0}^n \binom {n+1} k X^k Y^{(n+1)-k}.
  \end{balign*}
\end{proof}

\begin{proposition}\label{thm:xn_minus_one_factorization}
  \begin{equation*}
    X^n - 1 = (X - 1)(X^{n-1} + X^{n-2} + \cdots + 1).
  \end{equation*}
\end{proposition}
\begin{proof}
  Trivial application of induction\IND.
\end{proof}

\begin{proposition}\label{thm:polynomial_root_iff_divisible}
  The value \( u \in R \) is a \hyperref[def:semiring_kernel]{root} of the polynomial function \( p(x) \) if any only if the polynomial \( (X - u) \) divides \( p(X) \).
\end{proposition}
\begin{proof}
  \SufficiencySubProof Suppose that \( u \in R \) is a root of \( p(x) \). By \fullref{thm:euclidean_division_of_polynomials}, we can divide \( p(X) \) by the monic polynomial \( (X - u) \):
  \begin{equation*}
    p(X) = (X - u) q(X) + r(X).
  \end{equation*}

  Assume\DNE that \( r(X) \) is nonzero. Evaluating \( p(X) \) at \( u \) gives us
  \begin{equation*}
    0 = p(u) = (u - u) q(r) + r(u),
  \end{equation*}
  hence \( u \) is a root of \( r(X) \). But \( \deg r < \deg (X - u) = 1 \), that is, \( r \) is a nonzero constant and cannot have roots. The obtained contradiction proves the statement.

  \NecessitySubProof Suppose that \( (X - u) \) divides \( p(X) \) with quotient \( q(X) \). Then
  \begin{equation*}
    p(X) = (X - u) q(X).
  \end{equation*}

  Evaluation at \( u \) gives us
  \begin{equation*}
    p(u) = (u - u) q(u) = 0.
  \end{equation*}

  Therefore \( u \) is a root of \( p(X) \).
\end{proof}

\begin{definition}\label{def:polynomial_root_multiplicity}
  We say that the polynomial \( p \in R[X] \) has the root \( r \in R \) with multiplicity \( m \in \BbbZ_{>0} \) if there exists a polynomial \( q \in R[X] \) of degree \( \deg q = \deg p - m \) such that \( (X - r) \) does not divide \( q(X) \) and
  \begin{equation*}
    p(X) = (X - r)^m q(X).
  \end{equation*}
\end{definition}

\begin{proposition}\label{thm:integral_domain_polynomial_root_limit}
  If \( R \) is an integral domain, a nonzero polynomial of degree \( n \) has at most \( n \) (not necessarily \hyperref[def:polynomial_root_multiplicity]{distinct}) roots.
\end{proposition}
\begin{proof}
  We will use induction\IND on \( n \).

  In the case \( n = 0 \), we have that \( p \) is a nonzero constant polynomial. Such a polynomial cannot\DNE have roots, hence the statement holds.

  Now assume that the statement holds for polynomials of degrees \( 1, \ldots, n - 1 \). Let \( p \in R[X] \) be a polynomial of degree \( n \) and let \( r \in R \) be a root of \( p \). \Fullref{thm:polynomial_root_iff_divisible} implies that there exists a polynomial \( q(X) \) of degree \( n - 1 \) such that
  \begin{equation*}
    p(X) = (X - r) q(X).
  \end{equation*}

  Fix \( t \neq r \) that is a not a root of \( q(X) \). Evaluation at \( t \) gives us
  \begin{equation*}
    p(t) = (t - r) q(t).
  \end{equation*}

  Both \( (t - r) \neq 0 \) and \( q(t) \neq 0 \). Since \( R \) has no zero divisors, the product \( p(t) \) of \( (t - r) \) and \( q(t) \) is also nonzero. Thus the only roots of \( p(X) \) are \( r \) and the roots of \( q(X) \).

  By the induction conjecture, \( q(X) \) has at most \( n - 1 \) roots (counting multiplicities). Thus \( p(X) \) has at most \( (n - 1) + 1 = n \) roots.
\end{proof}

\begin{proposition}\label{thm:polynomials_with_identical_values}
  In an integral domain, two polynomials \eqref{eq:thm:polynomial_degree_properties/p} and \eqref{eq:thm:polynomial_degree_properties/q} with \( m \leq n \) are equal (i.e. have the same coefficients) if and only if their functions agree at \( n + 1 \) points.
\end{proposition}
\begin{proof}
  \SufficiencySubProof Obvious

  \NecessitySubProof Define
  \begin{equation*}
    r(X) \coloneqq p(X) - q(X) = \sum_{k=0}^n (a_k - b_k) X^k.
  \end{equation*}

  This is a polynomial of degree at most \( n \) that has \( n + 1 \) roots. By \fullref{thm:integral_domain_polynomial_root_limit}, \( r \) is the zero polynomial. Hence \( p = q \).
\end{proof}

\begin{definition}\label{def:primitive_polynomial}\mcite[394]{Knapp2016BasicAlgebra}
  A nonzero polynomial is called \term{primitive} if its coefficients are \hyperref[def:coprime_ring_ideals]{coprime}.
\end{definition}

\begin{proposition}\label{thm:polynomial_quotient_rings_equinumerous_with_module_of_polynomials}
  Fix a monic polynomial \( p(X) \in R[X] \) of degree \( n \). The free module \( \pi_n(R[X]) \), as defined in \fullref{def:polynomial_free_module}, is \hyperref[def:equinumerous_sets]{equinumerous} with the quotient ring \( R[X] / \braket{p(X)} \). This allows us to choose a \enquote{canonical representative} of the cosets of the quotient ring in a similar manner to \fullref{thm:cyclic_group_isomorphic_to_integers_modulo_n}.

  Consequently, for different polynomials \( p(X) \), only the ring structure on \( R[X] / \braket{p(X)} \) differs.
\end{proposition}
\begin{proof}
  Euclidean \hyperref[thm:euclidean_division_of_polynomials]{division} allows us to define the homomorphism
  \begin{balign*}
     & \Theta: R[X] / \braket{p(X)} \to \pi_n(R[X])             \\
     & \Theta(q(X) + \braket{p(X)}) \coloneqq \rem(q(X), p(X)).
  \end{balign*}

  It is injective since if \( q_1(X) \) and \( q_2(X) \) are not congruent modulo \( \braket{p(X)} \), they have different remainders. Conversely, \( \Theta \) is surjective because any remainder \( r(X) \) belongs to the coset
  \begin{equation*}
    r(X) + \braket{p(X)}.
  \end{equation*}

  See \fullref{ex:polynomial_quotient_rings_gaussian_integers} and \fullref{ex:polynomial_quotient_rings_z2} for differing ring structures in \( \pi_n(R[X]) \).
\end{proof}

\begin{example}\label{ex:polynomial_quotient_rings_gaussian_integers}
  The value of \fullref{thm:polynomial_quotient_rings_equinumerous_with_module_of_polynomials} is in that, like \fullref{thm:integers_modulo_isomorphic_to_quotient_group}, it allows us to identify elements of the quotient rings of the form \( R[X] / \braket{p(X)} \), for monic \( p \), with concrete polynomials.

  \Fullref{thm:polynomial_quotient_rings_equinumerous_with_module_of_polynomials} tells us that we can choose a concrete polynomial from \( \pi_{n-1}(R[X]) \) for every equivalence class in \( R[X] / \braket{p(X)} \) and that these polynomials have degree strictly less than \( n \) (if they are not zero).

  For example, consider the polynomial \( p(X) \coloneqq X^2 + 1 \) over the \hyperref[def:integers]{integers} \( \BbbZ \). It has degree \( 2 \), so the quotient ring \( R[X] / \braket{p(X)} \) can be identified with polynomials of the form
  \begin{equation}\label{ex:polynomial_quotient_rings_gaussian_integers/linear_polynomial}
    bX + a
  \end{equation}
  with integer coefficients.

  In order to make sense of the imposed ring structure in \( \BbbZ[X] / \braket{X^2 + 1} \), we can see how multiplication modulo \( X^2 + 1 \) works. We have
  \begin{balign*}
    (bX + a) (dX + c)
     & \cong
    bdX^2 + (ad + bc)X + ac
     & \pmod {X^2 + 1} \cong            \\ &\cong
    bd[X^2 + 1] + [(ad + bc)X - bd + ac]
     & \pmod {X^2 + 1} \cong            \\ &\cong
    (ad + bc)X + (ac - bd)
     & \pmod {X^2 + 1}. \phantom{\cong}
  \end{balign*}

  This is precisely the definition of multiplication of complex numbers (see \fullref{def:complex_numbers}). Thus we can identify
  \begin{equation*}
    \BbbZ[X] / \braket{X^2 + 1} \cong \BbbZ[i].
  \end{equation*}

  Like in \fullref{ex:polynomial_evaluation_gaussian_integers}, we arrive at the Gaussian \hyperref[def:gaussian_integers]{integers}, but using a different approach.
\end{example}

\begin{example}\label{ex:polynomial_quotient_rings_z2}
  Similarly to how the Gaussian integers were identified using \fullref{ex:polynomial_quotient_rings_gaussian_integers}, we will provide a different ring structure on \( \BbbZ[X] / \braket{p(X)} \) for a polynomial \( p(X) \) of degree \( 2 \).

  Consider the polynomial \( p(X) \coloneqq X^2 - 2 \) over the \hyperref[def:integers]{integers} \( \BbbZ \). We know that \( \sqrt 2 \) is a root of \( X^2 - 2 \) in \( \BbbR \) so we can identify
  \begin{equation*}
    \BbbZ[X] / \braket{X^2 - 2} \cong \BbbZ[\sqrt 2].
  \end{equation*}

  We will verify that multiplication is indeed compatible. Multiplication modulo \( X^2 - 2 \) works as follows:
  \begin{balign*}
    (aX + b) (cX + bd)
     & \cong
    acX^2 + (bc + ad)X + bd
     & \pmod X^2 - 2 \cong            \\ &\cong
    ac[X^2 - 2] + [(bc + ad)X + 2ac + bd]
     & \pmod X^2 - 2 \cong            \\ &\cong
    (bc + ad)X + (2ac + bd)
     & \pmod X^2 - 2. \phantom{\cong}
  \end{balign*}

  Multiplication in \( \BbbZ[\sqrt 2] \) works as follows:
  \begin{balign*}
    (a \sqrt 2 + b) (c \sqrt b + d)
    =
    2ac + (bc + ad) \sqrt 2 + bd.
  \end{balign*}

  The two results are identical.
\end{example}

\begin{definition}\label{def:algebraic_derivative}
  Generalizing \fullref{def:differentiability} from analysis, we define the \term{algebraic derivative} of a polynomial \( p(X) \in R[X] \) as
  \begin{equation*}
    p'(X) \coloneqq n a_n X^{n-1} + (n-1) a_{n-1} X^{n-2} + \cdots + a_2 X + a_1.
  \end{equation*}
\end{definition}

\begin{proposition}\label{thm:algebraic_derivative_product_rule}
  Algebraic \hyperref[def:algebraic_derivative]{derivatives} satisfy the product rule
  \begin{equation*}
    (pq)' = p'q + pq'.
  \end{equation*}
\end{proposition}
\begin{proof}
  By linearity, it is enough to consider the case where both \( p(X) \) and \( q(X) \) are monomials.

  \begin{balign*}
    p'(X) q(X) + p(X) q'(X)
     & =
    n a_n X^{n-1} b_m X^m + a_n X^n m b_m X^{m-1}
    =    \\ &=
    (n + m) a_n b_m X^{n+m-1}
    =    \\ &=
    (a_n b_m X^{n+m})
    =    \\ &=
    (pq)'(X).
  \end{balign*}
\end{proof}

\begin{proposition}\label{thm:algebraic_derivative_of_linear_polynomial_power}
  The algebraic \hyperref[def:algebraic_derivative]{derivative} of
  \begin{equation*}
    p(X) \coloneqq a (X - u)^n
  \end{equation*}
  is
  \begin{equation*}
    p'(X) = an(X - u)^{n-1}.
  \end{equation*}
\end{proposition}
\begin{proof}
  We use induction\IND on \( n \). The case \( n = 1 \) is obvious. Assume that the statement holds for \( 1, \ldots, n - 1 \).

  By \fullref{thm:algebraic_derivative_product_rule},
  \begin{balign*}
    p'(X)
     & =
    [a(X - u)^{n-1}]' (X - u) + a(X - u)^{n-1} [(X - u)]'
    =    \\ &=
    a(n-1)(X - u)^{n-2} (X - u) + a(X - u)^{n-1}
    =    \\ &=
    an(X - u)^{n-1}.
  \end{balign*}
\end{proof}

\begin{corollary}\label{thm:repeated_root_iff_derivatives_divisible}
  The value \( u \in R \) is a \hyperref[def:semiring_kernel]{root} of multiplicity \( m \) of the polynomial function \( p(x) \) if any only if \( u \) is a root of multiplicity \( m - 1 \) of its algebraic \hyperref[def:algebraic_derivative]{derivative} \( p'(x) \).
\end{corollary}
\begin{proof}
  \SufficiencySubProof Let \( u \) be a root of \( p(X) \) of multiplicity \( m \), i.e. there exists a polynomial \( q(X) \) of degree \( \deg(q) = \deg(p) - m \) such that \( (X - u) \) does not divide \( q(X) \) and
  \begin{equation*}
    p(X) = (X - u)^m q(X).
  \end{equation*}

  By \fullref{thm:algebraic_derivative_product_rule} and \fullref{thm:algebraic_derivative_of_linear_polynomial_power},
  \begin{equation*}
    p'(X)
    =
    m (X - u)^{m-1} q(X) + (X - u)^m q'(X)
    =
    (X - u)^{m-1} [m q(X) + (X - u) q'(X)].
  \end{equation*}

  Then \( u \) is a root of \( p'(X) \) of multiplicity at least \( m - 1 \). Assume\DNE that the multiplicity is at least \( m \), that is,
  \begin{equation*}
    (X - u) \mid [mq(X) + (X - u) q'(X)].
  \end{equation*}

  Since \( X - u \) obviously divides \( (X - u) q'(X) \), then the above implies
  \begin{equation*}
    (X - u) \mid mq(X).
  \end{equation*}

  But this contradicts our choice of \( q(X) \) that does not divide \( (X - u) \).

  The obtained contradiction proves that \( u \) if a root of \( p'(X) \) of multiplicity exactly \( n \).
\end{proof}

\begin{definition}\label{def:multivariate_polynomial}
  We define the \term{multivariate polynomial ring} in \( n \) indeterminates as the \enquote{iterated} single-variable polynomial ring
  \begin{equation*}
    R[X_1, X_2, \ldots, X_n] \coloneqq R[X_1][X_2] \cdots [X_n].
  \end{equation*}

  Note that we are using the letter \( n \) to represent the number of indeterminates rather than the degree of the multivariate polynomial. When we need the degree, we will note it specifically.

  Just like a polynomial \( p(X) \) in one indeterminate is a sequence, that is, a \hyperref[def:topological_net]{net} over \( \BbbZ_{>0} \), each multivariate polynomial \( p(X_1, \ldots, X_n) \) is a map from \( \BbbZ_{>0}^n \) to \( R \) (there is no natural order on \( \BbbZ_{>0}^n \) for defining \( p \) as a net). We can regard it as an \( n \)-dimensional \hyperref[def:array]{array} over \( R \), although arrays are purposely defined to only have finitely many elements, which makes it more difficult to define operations for an arbitrary pair of polynomials. For example, if there are only two variables, multivariate polynomials are \enquote{infinite matrices}
  \begin{equation*}
    p(X_1, X_2) \coloneqq \begin{pmatrix}
      a_{0,0} & a_{0,1} & \cdots \\
      a_{1,0} & a_{1,1} & \cdots \\
      \vdots  & \vdots  & \ddots \\
    \end{pmatrix}
  \end{equation*}
  with only finitely many nonzero elements.

  A \term{monomial} is a polynomial with only one nonzero element. The monomial
  \begin{equation*}
    p(X_1, X_2) \coloneqq \begin{pmatrix}
      0      & 0      & 0      & \cdots \\
      0      & 0      & 0      & \cdots \\
      0      & r      & 0      & \cdots \\
      0      & 0      & 0      & \cdots \\
      \vdots & \vdots & \vdots & \ddots \\
    \end{pmatrix}
  \end{equation*}
  can also be written symbolically as \( p(X_1, X_2) = r X_1^2 X_2 \), with the power for each variable corresponding to the zero-based index of the element along the corresponding axis.

  The sum of the indices over all axes is called the \term{degree} of this monomial. The above monomial has degree \( 3 = 2 + 1 \). We leave the degree for the zero monomial undefined.

  Every polynomial can be regarded as the sum of finitely many monomials by taking each element of the array and putting it in its own monomial array.

  The \term{degree} \( \deg p \) of a multivariate polynomial \( p \) is defined as the maximal degree among all of its nonzero monomials. If all monomials are zero, the degree is left undefined.

  If all nonzero monomials have the same degree, the polynomial is said to be \term{homogeneous}. Homogenous polynomials of degree \( 1 \) are called \term{linear}.
\end{definition}

\begin{theorem}\label{thm:polynomial_embedding_behavior}
  Let \( \mscrR \) be an integral domain and \( \xi \coloneqq \min \{ \card(\mscrR), \aleph_0 \} \).

  \begin{thmenum}
    \thmitem{thm:polynomial_embedding_behavior/zero} The only polynomial corresponding to the zero function \( f(x) = 0 \) is the zero polynomial.

    \thmitem{thm:polynomial_embedding_behavior/univariate} Let \( p(X) \) be a nonzero polynomial and let \( \Phi(p) \) is its polynomial function. Then there exists exactly one polynomial \( q(X) \) of degree \( \deg(p) < \xi \) such that \( \Phi(q) = \Phi(p) \).

    \thmitem{thm:polynomial_embedding_behavior/multivariate} Let \( p(X_1, \ldots, X_n) \) be a nonzero multivariate polynomial and let \( \Phi(p) \) be its polynomial function. Then there exists exactly one polynomial \( q(X_1, \ldots, X_n) \) such that the power of every variable in every monomial is strictly less than \( \xi \) and \( \Phi(q) = \Phi(p) \).
  \end{thmenum}
\end{theorem}
\begin{proof}
  \SubProofOf{thm:polynomial_embedding_behavior/zero} The nonzero constant polynomials are obviously different from the zero function. The nonconstant polynomials should have at least one value different from zero, hence are again different from the zero function.

  Therefore only the zero polynomial gives rise to the zero function.

  \SubProofOf{thm:polynomial_embedding_behavior/univariate} In the univariate case, by \fullref{thm:polynomials_with_identical_values}, two nonzero polynomials of degree less than \( \xi \) are equal if and only if they agree at \( \xi \) points.

  If \( \deg(p) < \xi \), which is automatically true if \( \mscrR \) is an infinite ring, then \( q \coloneqq p \) is the desired polynomial. It is unique by the previous paragraph.

  If \( \deg(p) \geq \xi \), which is only possible if \( \mscrR \) is a finite ring, we can easily give a non-constructive proof in the general case. \Fullref{alg:finite_field_polynomial_reduction} gives a concrete procedure for finding \( q \) in the case of certain Galois fields.

  If \( \mscrR \) is finite, both the \hyperref[def:polynomial_free_module]{polynomial free module} \( \pi_{\xi - 1} \) and the \hyperref[thm:functions_over_ring_form_algebra]{function space} \( \fun(\mscrR) \) have exactly \( \xi^\xi \) elements. Furthermore, by the first paragraph of the proof, two distinct polynomials in \( \pi_{\xi - 1} \) give rise to distinct functions. Therefore the evaluation \( \Phi: \pi_{\xi - 1} \to \fun(\mscrR) \) is an injective function between finite sets of the same cardinality. It is therefore a bijection. That is, to each endofunction over \( \mscrR \), in particular to \( \Phi(p) \), there corresponds a unique univariate polynomial over \( \mscrR \) of degree less than \( \xi \).

  \SubProofOf{thm:polynomial_embedding_behavior/multivariate} As in the univariate case, if \( \mscrR \) is infinite, the statement trivially holds.

  Assume that \( \mscrR \) is a finite ring. A simple counting argument shows that the vector spaces \( \fun(\mscrR^n, \mscrR) \) and
  \begin{equation*}
    \mscrL \coloneqq \linspan \{ X_1^{k_1} \cdots X_n^{k_n} \colon k_j = 1, \ldots, \xi \T{for} j = 1, \ldots, n \}
  \end{equation*}
  have exactly \( \xi^{\xi^n} \) vectors.

  We must, therefore, only show that \( \Phi \) is injective on \( \mscrL \).

  We use induction on the number of variables. We already showed injectivity for one variable and we now assume that the statement is true for all polynomials with (strictly) less than \( n \) variables.

  Assume\DNE that \( \Phi \) is not injective for polynomials in \( n \) variables. Then there exist polynomials \( f, g \in \mscrL \) such that \( r \coloneqq f - g \) is nonzero and yet \( \Phi(r) = 0 \). We know that
  \begin{equation*}
    f \in \mscrR[X_1, \ldots, X_n] = \mscrR[X_1, \ldots, X_{n-1}][X_n],
  \end{equation*}
  hence
  \begin{equation*}
    f(X_1, \ldots, X_n) = \sum_{k=0}^{q-1} f_k(X_1, \ldots, X_{n-1}) X_n^k,
  \end{equation*}
  where \( f_k(X_1, \ldots, X_{n-1}) \in \mscrL \) satisfy the inductive hypothesis and analogously for \( g \).

  Then, since \( \Phi \) is a homomorphism,
  \begin{equation*}
    \Phi(r(X_1, \ldots, X_n)) = \sum_{k=0}^{q-1} \Phi(f_k(X_1, \ldots, X_{n-1}) - g_k(X_1, \ldots, X_{n-1})) \Phi(X_n^k) = 0.
  \end{equation*}

  By the inductive hypothesis, the vectors \( \Phi[f_k - g_k], k = 0, \ldots, q - 1 \) are linearly independent. Therefore \( \Phi[r(X_1, \ldots, X_n)] = 0 \) if and only if \( \Phi(X_n^k) = 0 \) for all \( k = 0, \ldots, q - 1 \). But evaluating \( \Phi_t(X_n) \) would give \( t \), which may be nonzero, hence \( \Phi(X_n) \) is necessarily a nonzero function.

  Hence \( \Phi(r) \) is a nonzero function as a nontrivial combination of linearly independent nonzero functions. But this contradicts our assumption that \( f \neq g \). Therefore \( \Phi \) is injective over \( \mscrL \).

  Therefore, as in the univariate case, to each function from \( \fun(\mscrR^n, \mscrR) \), in particular to \( \Phi(p) \), there corresponds a unique polynomial from \( \mscrL \).
\end{proof}

\begin{definition}\label{def:rational_algebraic_function}
  We denote the field of \hyperref[def:field_of_fractions]{fractions} of \( R[X_1, \ldots, X_n] \) by \( R(X_1, \ldots, X_n) \) and call it the field of \term{rational algebraic functions}.
\end{definition}

\begin{definition}\label{def:laurent_polynomial}
  We generalize \hyperref[def:polynomial]{polynomials} by allowing negative coefficients.

  \begin{thmenum}
    \thmitem{def:laurent_polynomial/polynomial} A \term{Laurent polynomial} over \( R \) is, formally, a \hyperref[def:topological_net]{net} with finite support over the integers \( \BbbZ \) rather than a net over \( \BbbZ_{\geq 0} \) (i.e. the nonnegative integers). The operations \fullref{def:algebra_of_polynomials} make the Laurent polynomials into a ring, which we denote by \( R[X^\pm] \). This notation is consistent with \fullref{thm:polynomial_ring_universal_property}. Note that
    \begin{equation*}
      R[X^\pm] \cong R[X, X^{-1}] \cong R[X, Y] / (XY - 1),
    \end{equation*}
    so all three notations are used.

    The terms \enquote{\hyperref[def:polynomial/degree]{degree}} and \enquote{\hyperref[def:polynomial/leading_coefficient]{leading coefficient}} do not generalize naturally so we leave them undefined.

    We use the notation
    \begin{equation*}
      p(X) = \sum_{k \in \BbbZ} a_k X^k = \sum_{k=-\infty}^\infty a_k X^k.
    \end{equation*}

    \thmitem{def:laurent_polynomial/multivariate} Analogously to \fullref{def:multivariate_polynomial}, we define the ring of multivariate Laurent polynomials \( R[X_1^\pm, \ldots, X_n^\pm] \) as maps from \( \BbbZ^n \) to \( R \) rather than from \( \BbbZ \) to \( R \) or, inductively, as
    \begin{equation*}
      R[X_1^\pm, \ldots, X_n^\pm] = R[X_1^\pm, \ldots, X_{n-1}^\pm][X_n^\pm]
    \end{equation*}

    \thmitem{def:laurent_polynomial/series} A \term{formal Laurent series} is simply a Laurent polynomial in which we remove the restriction of only finitely many nonzero coefficients. We denote the set of all formal Laurent series over \( R \) by \( R\Bracks{X^\pm} \).
  \end{thmenum}
\end{definition}

\subsection{Prime ideals}\label{subsec:prime_ideals}

\begin{definition}\label{def:prime_ring_ideal}\mcite[384]{Knapp2016BasicAlgebra}
  An ideal \( P \) is called \term{prime} if it is proper and satisfies any of the equivalent conditions:
  \begin{thmenum}
    \thmitem{def:prime_ring_ideal/direct} If \( x, y \in R \) are such that \( xy \in P \), then either \( x \in P \) or \( y \in P \).
    \thmitem{def:prime_ring_ideal/ideals} If \( I, J \subseteq R \) are ideals such that \( IJ \subseteq P \), then either \( I \subseteq P \) or \( J \subseteq P \).
    \thmitem{def:prime_ring_ideal/quotient} The quotient \( R / P \) is an integral domain.
  \end{thmenum}

  An element \( r \in R \) is called \term{prime} if the ideal \( \braket r \) is prime.
\end{definition}
\begin{proof}
  \ImplicationSubProof{def:prime_ring_ideal/direct}{def:prime_ring_ideal/ideals} Fix ideals \( I, J \) of \( R \) such that \( IJ \subseteq P \).

  Assume that neither \( I \not\subseteq P \) nor \( J \not\subseteq P \). Take \( x \in I \setminus P \) and \( y \in J \setminus P \). It follows that \( xy \in P \) and either \( x \in P \) or \( y \in P \). This contradicts our assumption.

  The obtained contradiction proves that either \( I \subseteq P \) or \( J \subseteq P \).

  \ImplicationSubProof{def:prime_ring_ideal/ideals}{def:prime_ring_ideal/quotient} Fix an ideal \( P \) such that if \( I, J \subseteq R \) are ideals and \( IJ \subseteq P \), then either \( I \subseteq P \) or \( J \subseteq P \).

  We will prove that \( R / P \) is an integral domain. If \( R \) is an integral domain, this is obvious. If not, we fix nonzero \( x, y \in R \), so that \( xy = 0 \). Thus, \( [x][y] = (x + P)(y + P) = xy + P = P = [0] \). We will show that either \( x = 0 \) or \( y = 0 \).

  Consider the ideals
  \begin{balign*}
    \braket{x} & = xR, \\
    \braket{y} & = yR.
  \end{balign*}

  By \fullref{thm:product_of_principal_ideals}, we have \( \braket{x} \braket{y} = \braket{xy} = \braket{0} = \{ 0 \} \).

  Since \( \braket{x} \braket{y} \subseteq P \), then either \( \braket{x} \subseteq P \) or \( \braket{y} \subseteq P \). That is, either \( [x] = 0 \) or \( [y] = 0 \).

  Thus, \( R / P \) is an integral domain.

  \ImplicationSubProof{def:prime_ring_ideal/quotient}{def:prime_ring_ideal/direct} Suppose that \( R / P \) is an integral domain. Fix \( x, y \in R \), so that \( xy \in P \). If \( x = 0 \), obviously \( x = 0 \in R \) and similarly for \( y \). Suppose that both \( x \) and \( y \) are nonzero. We will show that either \( x \in P \) or \( y \in P \).

  We have
  \begin{equation*}
    [x][y] = [xy] = xy + P = P = [0].
  \end{equation*}

  Since \( R / P \) is an integral domain, either \( [x] = [0] \) or \( [y] = [0] \). That is, either \( x \in P \) or \( y \in P \).
\end{proof}

\begin{proposition}\label{thm:prime_ideal_iff_prime_quotient_ideal}
  If \( J \subseteq I \) are ideals of \( R \), then \( I \) is a \hyperref[def:prime_ring_ideal]{prime ideal} in \( R \) if and only if \( I / J \) is a prime ideal in  \( R / J \).
\end{proposition}

\begin{definition}\label{def:irreducible_ring_element}
  A nonzero element \( r \in R \) of an integral domain is called \term{reducible} if there exist non-invertible elements \( r_1, r_2 \in R \) such that
  \begin{equation*}
    r = r_1 r_2.
  \end{equation*}

  If \( r \) is not reducible, we say that it is \term{irreducible}.
\end{definition}

\begin{definition}\label{def:coprime_ring_ideals}
  Two ring ideals \( I \subseteq R \) and \( J \subseteq R \) are said to be \term{coprime} if \( I + J = R \).
\end{definition}

\begin{proposition}\label{thm:prime_implies_irreducible}\mcite[389]{Knapp2016BasicAlgebra}
  All \hyperref[def:prime_ring_ideal]{prime} elements in an integral domain are \hyperref[def:irreducible_ring_element]{irreducible}.
\end{proposition}
\begin{proof}
  Let \( p \) be prime. Assume that \( p \) is reducible, that is, there exist non-invertible elements \( r_1, r_2 \in R \) such that
  \begin{equation*}
    p = r_1 r_2.
  \end{equation*}

  Since \( p \) is prime, it must divide either \( r_1 \) or \( r_2 \). Without loss of generality, assume that \( p | r_1 \) and \( r_1 = pc \) for some \( c \in R \).

  Then \( p = r_1 r_2 = pc r_2 \). By \fullref{thm:def:semiring/properties/cancellable_iff_not_zero_divisor}, \( 1 = c r_2 \), which implies that \( r_2 \) is invertible with inverse \( c \). This contradicts our assumption that both \( r_1 \) and \( r_2 \) are invertible.

  The obtained contradiction proves that \( p \) is irreducible.
\end{proof}

\begin{definition}\label{def:maximal_ring_ideal}
  A two-sided ideal \( M \) is called \term{maximal} if it is proper and satisfies any of the equivalent conditions:
  \begin{thmenum}
    \thmitem{def:maximal_ring_ideal/maximality} \( M \) is maximal with respect to set inclusion among proper two-sided ideals.
    \thmitem{def:maximal_ring_ideal/quotient} The quotient \( R / M \) is a field.
  \end{thmenum}
\end{definition}
\begin{proof}
  \ImplicationSubProof{def:maximal_ring_ideal/maximality}{def:maximal_ring_ideal/quotient} Suppose that \( M \) is maximal among proper ideals. We will prove that every nonzero element of \( R / M \) is invertible.

  Fix \( x \not\in M \), so that \( [x] = x + M \neq M = [0] \). Define the set
  \begin{equation*}
    I \coloneqq Rx + M.
  \end{equation*}

  It is a ideal since both \( Rx \) and \( M \) are ideals. Furthermore, it contains \( M \) strictly because \( M \subseteq I \) and \( x \in I \). Since \( M \) is maximal, we have that \( I = R \).

  Hence, there exists \( y \in R \) such that \( 1 = yx + M \). Hence, \( [y] = y + M \) is an inverse of \( [x] \) in \( R / M \).

  Since \( [x] \in R / M \) was an arbitrary nonzero element, we conclude that \( R / M \) is a field.

  \ImplicationSubProof{def:maximal_ring_ideal/quotient}{def:maximal_ring_ideal/maximality} Suppose that \( R / M \) is a field. Assume that \( M \) is not maximal. Then there exists a proper ideal \( I \supsetneq M \).

  Assume that \( I \neq M \) and take \( x \in I \setminus M \). Then \( x \not\in M \) and hence \( [x] \neq [0] \) and is invertible in \( R / M \). Denote by \( y \) any representative of this inverse. Thus, \( [xy] - [1] = [0] \), that is, \( xy - 1 \in M \).

  Note that \( xy \in I \) because \( x \in I \) and \( y \in R \). Since \( I \) is closed under addition, it follows that \( 1 \in I \) and hence \( I = R \). But this contradicts our assumption that \( I \) is proper.

  The obtained contradiction proves that \( M \) is maximal.
\end{proof}

\begin{theorem}[Krull's theorem]\label{thm:krulls_theorem}\mcite{Hodges1979}
  Every nontrivial \hyperref[def:semiring/commutative_unital_ring]{commutative unital ring} has a \hyperref[def:maximal_ring_ideal]{maximal ideal}.

  In \hyperref[def:zfc]{\logic{ZF}} this theorem is equivalent to the \hyperref[def:zfc/choice]{axiom of choice} --- see \fullref{thm:axiom_of_choice_equivalences/krull}.
\end{theorem}

\begin{proposition}\label{thm:maximal_ideals_are_prime}
  Maximal ring \hyperref[def:maximal_ring_ideal]{ideals} are \hyperref[def:prime_ring_ideal]{prime}.
\end{proposition}
\begin{proof}
  If \( M \) is a maximal ideal of \( R \), by \fullref{def:maximal_ring_ideal/quotient} \( R / M \) is a field. Thus, \( R / M \) is an integral domain, which by \fullref{def:prime_ring_ideal/quotient} means that \( M \) is a prime ideal.
\end{proof}

\begin{proposition}\label{thm:field_maximal_ideal_representation}\mcite[exer. 8.1]{КоцевСидеров2016}
  If \( r_1, \ldots, r_n \) are elements of the field \( \BbbK \), then \( \braket{X_1 - r_1, \ldots, X_n - r_n} \) is a maximal ideal of \( \BbbK[X_1, \ldots, X_n] \).
\end{proposition}

\begin{proposition}\label{thm:ufd_prime_iff_irreducible}
  An element in a unique factorization domain is \hyperref[def:prime_ring_ideal]{prime} if and only if it is \hyperref[def:irreducible_ring_element]{irreducible}.
\end{proposition}
\begin{proof}
  \SufficiencySubProof Follows from \fullref{thm:prime_implies_irreducible}.

  \NecessitySubProof Let \( r \) be an irreducible element and let \( p_1 p_2 \in \braket{r} \). We will show that either  \( p_1 \in \braket{r} \) or \( p_2 \in \braket{r} \).

  Since \( \braket{r} \) is an ideal, there exists an element \( q \in R \) such that \( qr = p_1 p_2 \). Because of unique factorization, there exists a unit \( u \in R \) such that \( uqr = p_1 p_2  \).

  Therefor either
  \begin{itemize}
    \item \( p_1 = 1 \), in which case \( p_2 = uqr \in \braket{r} \).
    \item \( p_1 = u \), in which case \( p_2 = qr \in \braket{r} \).
    \item \( p_1 = uq \), in which case \( p_2 = r \in \braket{r} \).
    \item \( p_1 = ur \in \braket{r} \).
    \item \( p_1 = qr \in \braket{r} \).
    \item \( p_1 = uqr \in \braket{r} \).
  \end{itemize}
\end{proof}

\begin{proposition}\label{thm:prime_ideals_are_maximal_in_pid}
  Prime ring \hyperref[def:prime_ring_ideal]{ideals} in a principal ideal domain are \hyperref[def:maximal_ring_ideal]{maximal}.
\end{proposition}
\begin{proof}
  Let \( P \) be a prime ideal of \( R \) and let \( I \supsetneq P \) be an ideal strictly containing \( P \). We will show that \( I = R \).

  Since \( R \) is a principal ideal domain, both \( P \) and \( I \) are principal. Let \( p \) and \( i \) be their respective generators. Since \( I \) contains \( P \), there exists \( r \in R \) such that
  \begin{equation*}
    p = ir.
  \end{equation*}

  But \( p \) is prime, and thus irreducible by \fullref{thm:ufd_prime_iff_irreducible}, and hence either \( i \) or \( r \) must be a unit. If \( r \) is a unit, then \( \braket i = \braket {ir} = \braket p \), which contradicts our choice of \( I \supsetneq P \). It remains for \( i \) to be a unit.

  Therefore, \( I = \braket i = \braket 1 = R \). This proves that \( P \) is maximal with respect to inclusion of ideals.
\end{proof}

\begin{definition}\label{def:krull_dimension}\mcite[67]{КоцевСидеров2016}
  Consider \hyperref[def:tower_diagram]{chains}
  \begin{equation*}
    P_0 \subsetneq P_1 \subsetneq \cdots
  \end{equation*}
  of prime \hyperref[def:prime_ring_ideal]{ideals} in \( R \) under strict inclusion. The length of this chain is defined as the zero-based index of its last element and is allowed to be infinite. Zero-based means that a chain with only one ideal has length zero.

  We call the supremum of the lengths of these chains the \term{Krull dimension} of the ring \( R \) and denote it by \( \dim R \).
\end{definition}

\begin{proposition}\label{thm:def:krull_dimension/properties}
  The Krull \hyperref[def:krull_dimension]{dimension} of a ring \( R \) has the following basic properties:
  \begin{thmenum}
    \thmitem{thm:def:krull_dimension/properties/monotone} If \( R = T / S \) is the quotient of some rings \( S \subseteq T \), \( \dim R \leq \dim T \).
    \thmitem{thm:def:krull_dimension/properties/pid} If \( R \) is a principal ideal domain, \( \dim R \in \{ 0, 1 \} \).
    \thmitem{thm:def:krull_dimension/properties/field} If \( R \) is a \hyperref[def:field]{field}, \( \dim R = 0 \).
    \thmitem{thm:def:krull_dimension/properties/polynomials_over_field}\cite[exercise 8.19]{КоцевСидеров2016} If \( R = \BbbK[X_1, \ldots, X_n] \) for some \hyperref[def:field]{field} \( \BbbK \), \( \dim R = n \).
  \end{thmenum}
\end{proposition}

\begin{corollary}\label{thm:multivariate_polynomial_rings_are_not_pid}
  Multivariate polynomial rings are not principal ideal domains.
\end{corollary}
\begin{proof}
  Follows from \fullref{thm:def:krull_dimension/properties/pid} and \fullref{thm:def:krull_dimension/properties/polynomials_over_field}.
\end{proof}

\begin{definition}\label{def:radical_ideal}\mcite[15]{КоцевСидеров2016}
  We define the \term{radical ideal} of the ideal \( I \) of \( R \) as
  \begin{equation*}
    \sqrt I \coloneqq \{ x \in R \colon \exists n: x^n \in I \}.
  \end{equation*}
\end{definition}
\begin{proof}
  We verify that the set \( \sqrt I \) is an ideal of \( R \).

  It is closed under addition because if \( x^n \in I \) and \( y^m \in I \), then by \fullref{thm:binomial_theorem},
  \begin{equation*}
    (x + y)^{n+m}
    =
    \sum_{k=0}^{n+m} \binom n k x^k y^{n+m-k}
  \end{equation*}

  In this sum, either \( k \geq n \) and thus the product \( x^k y^{n+m-k} \in I \), or \( k < n \), in which case \( n + m - k > n + m - n = m \) and the same holds. Thus,
  \begin{equation*}
    (x + y)^{n+m} \in I
  \end{equation*}
  and \( x + y \in \sqrt I \).

  The radical is also closed under multiplication with \( R \) since if \( x^n \in I \), then \( (rx)^n = r^n x^n \in I \).

  Therefore, it is an ideal of \( R \).
\end{proof}

\begin{definition}\label{def:primary_ring_ideal}\mcite[74]{КоцевСидеров2016}
  We call the proper ideal \( P \) of \( R \) \term{primary} if \( xy \in P \) implies that \( x \in P \) or \( y \in \sqrt P \).
\end{definition}

\begin{theorem}[Chinese remainder theorem]\label{thm:chinese_remained_theorem}\mcite[thm. 8.27]{Knapp2016BasicAlgebra}
  Let \( I_1, \ldots, I_n \) be pairwise \hyperref[def:coprime_ring_ideals]{coprime} ideals. Then
  \begin{equation*}
    R / \bigcap_{i=1}^n I_n \cong R / I_1 \times \cdots \times R / I_n.
  \end{equation*}
\end{theorem}

\begin{definition}\label{def:spectrum_of_ring}
  The set of all prime ideals on a ring \( R \) is called the \term{spectrum} of \( R \) denoted by \( \op{Spec}(R) \).
\end{definition}

\subsection{Localization}\label{subsec:localization}

\begin{definition}\label{def:ring_localization}\mcite\cite[428]{Knapp2016BasicAlgebra}
  Let \( S \subseteq R \) be closed under multiplication.

  Define the following equivalence relation on \( R \times S \):
  \begin{equation*}
    (r, s) \cong (r', s') \iff \exists t \in S: t(rs' - sr') = 0.
  \end{equation*}

  Define the ring
  \begin{equation*}
    S^{-1} R \coloneqq R \times S / \cong
  \end{equation*}
  with operations inherited from \( R \) using the injection
  \begin{balign*}
     & \iota: R \to S^{-1} R        \\
     & \iota(r) \coloneqq [(r, 1)].
  \end{balign*}
\end{definition}

\begin{proposition}\label{thm:ring_localization_universal_property}\mcite\cite[431]{Knapp2016BasicAlgebra}
  Let \( S \subseteq R \) be closed under multiplication. The ring \hyperref[def:ring_localization]{localization} \( S^{-1} R \) satisfies the following universal property: if \( T \) is a nontrivial commutative unital ring and \( \varphi: R \to T \) is a unital ring homomorphism such that \( \varphi(S) \) are units in \( T \), there exists a unique ring homomorphism \( \hat \varphi \) such that the following diagram commutes:

  \begin{alignedeq}\label{thm:ring_localization_universal_property/diagram}
    \todo{Add diagram}\iffalse\begin{mplibcode}
      beginfig(1);
      input metapost/graphs;

      v1 := thelabel("$S^{-1} R$", origin);
      v2 := thelabel("$T$", (2, 0) scaled u);
      v3 := thelabel("$R$", (1, 1) scaled u);

      a1 := straight_arc(v3, v2);
      a2 := straight_arc(v3, v1);

      d1 := straight_arc(v1, v2);

      draw_vertices(v);
      draw_arcs(a);

      drawarrow d1 dotted;

      label.urt("$\varphi$", straight_arc_midpoint of a1);
      label.ulft("$\iota$", straight_arc_midpoint of a2);
      label.top("$\hat\varphi$", straight_arc_midpoint of d1);
      endfig;
    \end{mplibcode}\fi
  \end{alignedeq}
\end{proposition}

\begin{proposition}\label{thm:ring_localization_preserves_ideals}\mcite\cite[432]{Knapp2016BasicAlgebra}
  If \( I \) is an ideal in \( R \), then
  \begin{equation*}
    S^{-1} I \coloneqq \{ s^{-1} x \mid s \in S, x \in I \}
  \end{equation*}
  is an ideal in the \hyperref[def:ring_localization]{localization} \( S^{-1} R \).
\end{proposition}

\begin{definition}\label{def:local_ring}
  If \( R \) has a unique \hyperref[def:maximal_ring_ideal]{maximal ideal}, we say that it is a \term{local ring}.
\end{definition}

\begin{proposition}\label{thm:localization_of_prime_is_local}\mcite\cite[cor. 8.50]{Knapp2016BasicAlgebra}
  Fix a prime ideal \( P \). Its complement
  \begin{equation*}
    S \coloneqq \BbbZ \setminus P
  \end{equation*}
  is closed under multiplication and we can perform \hyperref[def:ring_localization]{localization}.

  In this case, the ring \( S^{-1} R \) is a local \hyperref[def:local_ring]{ring} and \( M \coloneqq S^{-1} P \) is its unique maximal ideal.
\end{proposition}

\begin{example}\label{ex:ring_localization}\mcite\cite[430]{Knapp2016BasicAlgebra}
  Let \( p \) be a prime \hyperref[def:prime_number]{number} and \( P = \braket p \) be the corresponding \hyperref[def:prime_ring_ideal]{prime ideal}. Denote its complement by \( S \).

  The ring \( S^{-1} R \) then consists of all rational \hyperref[def:rational_numbers]{numbers} whose denominators are not divisible by \( p \).

  In particular, if \( p = 2 \), then \( S^{-1} R \) is the set of all rational numbers with odd denominators.
\end{example}

\begin{definition}\label{def:field_of_fractions}
  The \term{field of fractions} of \( R \) is defined as the \hyperref[def:ring_localization]{localization} of \( R \) by the set
  \begin{equation*}
    S \coloneqq R \setminus \{ 0 \}.
  \end{equation*}
\end{definition}
\begin{proof}
  This is indeed a field since all nonzero elements are invertible.
\end{proof}

\subsection{Noetherian rings}\label{subsec:noetherian_rings}

\begin{definition}\label{def:noetherian_module}\mcite\cite[prop. 8.30]{Knapp2016BasicAlgebra}
  A module \( M \) over \( R \) is called \term{Noetherian} if it satisfies any of the following conditions:
  \begin{thmenum}
    \ilabel{def:noetherian_module/ascending_chain} Every strict chain of submodules of \( M \)
    \begin{equation*}
      M_1 \subsetneq M_2 \subsetneq \ldots
    \end{equation*}
    is finite.

    \ilabel{def:noetherian_module/finite_basis} Every submodule of \( M \) is finitely \hyperref[def:free_left_module]{generated}.
  \end{thmenum}
\end{definition}
\begin{proof}
  \ImplicationSubProof{def:noetherian_module/ascending_chain}{def:noetherian_module/finite_basis} We can construct a basis as follows: choose\AOC any element \( x_1 \) of \( M \). Next, choose\AOC an element \( x_2 \in M \setminus \braket {x_1} \), then \( x_3 \in M \setminus \braket {x_1, x_2} \) and so on.

  This process must stop after finitely many steps because we obtain the strict chain
  \begin{equation*}
    \braket {x_1} \subsetneq \braket {x_1, x_2} \cdots
  \end{equation*}
  of submodules.

  \ImplicationSubProof{def:noetherian_module/finite_basis}{def:noetherian_module/ascending_chain} Suppose that all submodules of \( M \) are finitely generated. Let
  \begin{equation*}
    M_1 \subsetneq M_2 \subsetneq \ldots
  \end{equation*}
  be a strictly ascending chain of submodules.

  Then the submodule \( N \coloneqq \bigcup_{i=1}^\infty M_i \) is also finitely generated. There exists a member \( M_N \) of the chain containing all the generators of \( N \). Then no further strict inclusion of modules is possible. We conclude that the chain
  \begin{equation*}
    M_1 \subsetneq M_2 \subsetneq \ldots
  \end{equation*}
  is finite.
\end{proof}

\begin{definition}\label{def:noetherian_ring}
  A \term{Noetherian ring} is an Noetherian submodule over itself, i.e. it satisfies the conditions in \fullref{def:noetherian_module} on its ideals.
\end{definition}

\begin{theorem}[Hilbert basis theorem]\label{thm:hilberts_basis_theorem}\mcite\cite[418]{Knapp2016BasicAlgebra}
  If \( R \) is Noetherian, so is \( R[X] \).
\end{theorem}

\begin{theorem}\label{thm:noetherian_rings_closed_under}
  The \hyperref[def:set_zfc]{classes} of Noetherian \hyperref[def:noetherian_module]{modules} and Noetherian \hyperref[def:noetherian_ring]{rings} are closed with respect to many operations, including the following:

  \begin{thmenum}
    \ilabel{thm:noetherian_rings_closed_under/localization}\cite[corollary 8.48]{Knapp2016BasicAlgebra} The \hyperref[def:ring_localization]{localizations} \( S^{-1} R \) of a Noetherian ring \( R \) are also Noetherian.

    \ilabel{thm:noetherian_rings_closed_under/modules}\cite[proposition 8.34]{Knapp2016BasicAlgebra} Any \hyperref[def:free_left_module]{finitely generated module} over a Noetherina ring is also \hyperref[def:noetherian_module]{Noetherian}.

    \ilabel{thm:noetherian_rings_closed_under/submodules}\cite[proposition 6.3(а)]{Коцев2016} Every submodule of a Noetherian module is Noetherian.

    \ilabel{thm:noetherian_rings_closed_under/quotients}\cite[proposition 6.3(a)]{Коцев2016} Every quotient of a Noetherian module is Noetherian.

    \ilabel{thm:noetherian_rings_closed_under/restoration}\cite[proposition 6.3(b)]{Коцев2016} If the module \( M \) has a Noetherian submodule \( N \) such that \( M / N \) is also Noetherian, then \( M \) itself is Noetherian.

    \ilabel{thm:noetherian_rings_closed_under/polynomial_ring} The polynomial ring \( R[X] \) over a Noetherian ring \( R \) is also Noetherian. See \fullref{thm:hilberts_basis_theorem}.
  \end{thmenum}
\end{theorem}

\begin{theorem}[Primary decomposition]\label{thm:primary_decomposition}\mcite\cite[thm. 10.6]{Коцев2016}
  In a Noetherian ring \( R \), every ideal \( I \) can be represented as an intersection
  \begin{equation*}
    I = \bigcap_{i=1}^n P_i
  \end{equation*}
  of finitely many primary \hyperref[def:primary_ring_ideal]{ideals}.
\end{theorem}


% Linear algebra
\section{Linear algebra}\label{sec:linear_algebra}
\subsection{Vector spaces}\label{subsec:vector_spaces}

This subsection is about the algebraic properties of vector spaces. See \fullref{subsec:vector_space_geometry} for \enquote{geometric} concepts like hyperplanes and convexity.

\begin{definition}\label{def:vector_space}
  A \Def{vector space} \( (V, +, \cdot) \) is a \hyperref[def:left_module]{left module} over a field \( \BK \).

  We call elements of \( \BK \) \Def{scalars} and elements of \( V \) \Def{vectors}.

  The category of vector spaces over \( \BK \) is denoted by \( \Cat{Vect}_{\BK} \).
\end{definition}

\begin{definition}\label{def:vector_field}
  Let \( V \) be a vector space over \( \BK \). Functions of the type
  \begin{equation*}
    f: \BK \to V
  \end{equation*}
  are called \Def{vector fields}. To avoid confusion, \( \BK \) is sometimes referred to as a \Def{scalar field}. This convention comes from physics and is dominant in areas that are far from algebraic field theory, hence in practice it does not cause a lot of confusion.
\end{definition}

\begin{remark}\label{rem:real_vector_space}
  Outside of algebra, we are usually only interested in vector spaces over the fields \( \BR \) or \( \BC \). We call them \Def{real vector spaces} and \Def{complex vector spaces}, respectively.
\end{remark}

\begin{definition}\label{def:complex_conjucate_vector_space}
  Let \( V \) be a vector space over the complex numbers \( \BC \). Its \Def{complex conjugate vector space} \( \Ol V \) is the same space but with scalar multiplication defined as
  \begin{equation*}
    t \cdot_{\Ol V} x \coloneqq \Ol t \cdot_V x.
  \end{equation*}
\end{definition}

\begin{proposition}\label{thm:field_extension_is_vector_space}
  Let \( \BK \) be a field \hyperref[def:field_extension]{extension} of \( G \). Then \( \BK \) is a vector space over \( G \).
\end{proposition}
\begin{proof}
  Since \( \BK \) already has the structure of an abelian group, we must only define scalar multiplication
  \begin{BreakableAlign*}
     & \circ: G \times \BK \to \BK, \\
     & g \circ f \coloneqq gf,
  \end{BreakableAlign*}
  where the product in the definition is simply multiplication in \( \BK \). The well-definedness of \( \circ \) follows from the well-definedness of multiplication in \( \BK \).
\end{proof}

\begin{remark}\label{rem:linear_span_only_for_vector_spaces}
  The definition for linear \hyperref[def:linear_span]{span} applies to general commutative \hyperref[def:left_module]{modules}. However, since \fullref{thm:vector_space_linear_dependence,thm:vector_space_basis} do not apply to general commutative modules, it makes sense to only use linear spans withing the context of vector spaces.
\end{remark}

\begin{definition}\label{def:linear_span}
  For a set \( A \subseteq X \) of vectors, the set of all linear combinations of finite subsets of \( A \) is called its span and is denoted by \( \Span{A} \).
\end{definition}

\begin{proposition}\label{thm:vector_space_linear_dependence}
  The set \( A \subseteq V \) is linearly dependent in the sense of \fullref{def:left_module_linear_dependence} if and only if there exists a vector \( x \in V \) such that
  \begin{equation*}
    x \in \Span{A} \setminus \{ x \}.
  \end{equation*}
\end{proposition}
\begin{proof}
  \Sufficiency Let \( A \subseteq M \) and let
  \begin{equation*}
    0_M \coloneqq \sum_{k=1}^n t_k x_k,
  \end{equation*}
  where \( t_1, \ldots, t_n \) have at least one nonzero scalar and where \( x_1, \ldots, x_n \) are nonzero vectors. Without loss of generality, assume that \( t_{n_0} \) is the nonzero scalar. Then
  \begin{BreakableAlign*}
    0_M             & = \sum_{k=1}^n t_k x_k,                                                                                  \\
    t_{k_0} x_{k_0} & = -\sum_{k=1}^n t_k x_k,                                                                                 \\
    x_{k_0}         & = \sum_{k=1}^n \left(-\frac {t_k} {t_{k_0}} \right) x_k \in \Span{A} \setminus \left\{ x_{k_0} \right\}.
  \end{BreakableAlign*}

  \Necessity Let \( A \subseteq M \) and \( x \in \Span{A} \setminus \{ 0_M, x \} \). By \fullref{def:linear_combination}, there exist nonzero vectors \( x_1, \ldots, x_n \in A \) and scalars \( t_1, \ldots, t_n \in R \) such that
  \begin{equation*}
    x \coloneqq \sum_{k=1}^n t_k x_k,
  \end{equation*}
  where at least one of \( t_1, \ldots, t_n \) is nonzero.

  Then \( 0_M \) is a nontrivial linear combination of the nonzero vectors \( x_1, \ldots, x_n, x \):
  \begin{equation*}
    0_M = \sum_{k=1}^n t_k x_k - x.
  \end{equation*}
\end{proof}

\begin{definition}\label{affine_independence}
  Given a vector space \( V \) over \( F \), we say that a set \( A \subseteq V \) of vectors is \Def{affinely independent} in \( V \) if the set
  \begin{equation*}
    \{ (x, 1) \colon x \in A \}
  \end{equation*}
  is linearly independent in \( V \times F \).
\end{definition}

\begin{proposition}\label{thm:vector_space_basis}
  The set \( B \subseteq V \) is a basis in the sense of \fullref{def:left_module_hamel_basis} if and only if it is linearly independent and
  \begin{equation*}
    V = \Span{B}.
  \end{equation*}
\end{proposition}
\begin{proof}
  \Sufficiency We will prove the contraposition, that is, if \( \Span{B} \neq M \), then \( B \) is not a maximal linearly independent set.

  If \( \Span{B} \subsetneq M \), then there exists a vector \( x \in M \) such that \( x \) is not a linear combination of any subset of \( B \). Thus \( B \cup \{ x \} \) does not have a nontrivial linear combination that equals zero. Hence \( B \cup \{ x \} \) is linearly independent.

  \Necessity Let \( B \subseteq M \) and \( \Span{B} = M \). Assume\LEM that there exists a vector \( x \in M \setminus B \) such that the set \( B \cup \{ x \} \) is linearly independent.

  Then, for any \( b \in B \), the vector \( x + b \) is a linear combination of elements, one of which is independent of \( B \). Thus, by \fullref{thm:vector_space_linear_dependence},
  \begin{equation*}
    M = \Span{B} \subsetneq \Span{B \cup \{ x \}} \subseteq M,
  \end{equation*}
  which is a contradiction.
\end{proof}

\begin{theorem}\label{thm:all_vector_spaces_are_free_left_modules}
  All vector spaces have a \hyperref[def:left_module_hamel_basis]{basis}. Equivalently, all vector spaces are \hyperref[def:free_left_module]{free modules}. See \fullref{thm:aoc/vector_space_bases}.
\end{theorem}
\begin{proof}
  Let \( V \) be a vector space. Assume that it does not have a basis. Let \( \Cal{B} \) be the family of all linearly independent \hyperref[def:linear_combination]{subsets} of \( V \).

  The family \( \Cal{B} \) is obviously nonempty since any \hyperref[rem:singleton_sets]{singleton} from \( V \) belongs to \( \Cal{B} \). The union of any chain \( \Cal{B}' \subseteq \Cal{B} \) can then contain only linearly independent elements since otherwise\LEM we would have that some set in \( \Cal{B}' \) is not linearly independent. Thus we can apply \fullref{thm:zorns_lemma} to obtain a maximal element \( B \).

  Assume\LEM that \( B \) is not a basis, that is,
  \begin{equation*}
    \Span B \subsetneq V.
  \end{equation*}

  Take \( V \in V \setminus \Span B \). Then the set \( B \cup \{ v \} \) is linearly independent, which contradicts the assumption that \( B \) is not a basis. Thus \( B \) is a basis of \( V \) and \( V \) is a free module.
\end{proof}

\begin{definition}\label{def:vector_space_dimension}
  The \hyperref[def:free_left_module]{free module rank} of a vector space \( V \) is called the \Def{dimension} \( \dim V \) of \( V \). If \( U \) is a vector subspace of \( V \), we call \( \Co\dim_V U \coloneqq \dim(V/U) \) the \Def{codimension} of \( U \) relative to \( V \).
\end{definition}

\begin{proposition}\label{thm:linear_maps_form_algebra}
  The set \( \Hom(U, V) \) is a vector space.
\end{proposition}
\begin{proof}
  By \fullref{thm:functions_over_ring_form_algebra}, \( \Hom(U, V) \) forms an \( \BK \)-vector space.
\end{proof}

\begin{remark}\label{rem:functional}
  The term \enquote{functional} does not have a strict meaning. For example, logicians use terms like \enquote{primitive recursive functional} for certain generalized functions. Functions are also ill-defined, see \fullref{rem:function_definition}. Outside of logic, however, the term \enquote{functional} usually refers to a function from a vector space \( V \) to its base field \( \BK \). Examples include linear \hyperref[def:linear_operator]{functionals}, like projection \hyperref[def:left_module_basis_projection]{maps} and \hyperref[def:differentiability]{derivatives}, and nonlinear functionals, like the Minkowski \hyperref[def:minkowski_functional]{functionals}.
\end{remark}

\begin{definition}\label{def:eigenpair}
  Let \( f: U \to V \) be a function between vector spaces over \( \BK \).

  An \Def{eigenpair} of \( f \) consists of an \Def{eigenvalue} \( \lambda \in \BK \) and an \Def{eigenvector} \( x \in U \) such that
  \begin{equation*}
    f(x) = \lambda x.
  \end{equation*}
\end{definition}

\subsection{Algebraic dual spaces}\label{subsec:algebraic_dual_spaces}

In this subsection, we restrict ourselves to fields rather than arbitrary ring.

\begin{definition}\label{def:dual_vector_space}\mcite[50]{Knapp2016BasicAlgebra}
  Let \( V \) be a \hyperref[def:vector_space]{vector space} over the \hyperref[def:field]{field} \( \BbbK \). By \fullref{thm:functions_over_algebra}, the set \( \hom(V, \BbbK) \) of all \hyperref[def:semimodule/homomorphism]{linear maps} from \( V \) to the underlying field \( \BbbK \) also form a vector space over \( \BbbK \).

  We will call this space the \term{algebraic dual space} of \( V \) and denote it by \( V^* \). We will call the functions in \( V^* \) \term{linear functionals}. The prefix \enquote{algebraic} is important when confusion is possible with \hyperref[def:continuous_dual_space]{continuous dual spaces}.
\end{definition}

\begin{remark}\label{rem:dual_space_bilinear_form}\mcite[16]{ИоффеТихомиров1974}
  If \( l \) is a \hyperref[def:dual_vector_space]{linear functional} over \( V \), we often use the notation \( \inprod l x \) rather than the function notation \( l(x) \). This is an extension of the notation for \hyperref[def:inner_product_space]{inner product spaces}.

  Moreover, \( \inprod \anon \anon \) is a \hyperref[def:multilinear_function]{bilinear function} from the Cartesian product \( V^* \times V \) to \( \BbbK \). Hence, if \( V \) is isomorphic to \( V^* \), then this is precisely an inner product.
\end{remark}

\begin{remark}\label{rem:functional}
  The term \enquote{functional} as a noun has no definite meaning.

  \begin{itemize}
    \item In the context of linear algebra, and in particular \fullref{def:dual_vector_space}, the term \enquote{functional} refers to \enquote{linear functional}, i.e. a \hyperref[def:semimodule/homomorphism]{linear map} from a \hyperref[def:vector_space]{vector space} to its base field.

    This terminology can be found, for example, in \cite[50]{Knapp2016BasicAlgebra} and \cite[sec. 26.1]{Тыртышников2004Лекции}.

    \item In the context of functional analysis, \enquote{linear functional} may refer to either \hyperref[def:continuous_dual_space]{continuous linear functionals} from some \hyperref[def:topological_vector_space]{topological vector space} to its base field, or to arbitrary linear functionals.

    The former terminology can be found, for example, in \cite[def. 3.1]{Rudin1991Functional} and \cite[sec. 1.3]{Clarke2013}.

    An arbitrary map from a topological vector space to its field may also be called a functional. For example, \cite[102]{KufnerFucik1980} and \cite[223]{Deimling1985} refer to \enquote{nonlinear functionals}. \hyperref[def:minkowski_functional]{Minkowski functionals} are notoriously nonlinear.

    \item In the context of recursive functions, for example in \cite{StanfordPlato:recursive_functions}, functionals are defined as \enquote{operations which map one or more functions of type \( \BbbN^k \to \BbbN \) (possibly of different arities) to other functions}.
  \end{itemize}

  The commonality between linear algebra and functional analysis is that \enquote{functional} refers to a map from a vector space to its base field. The commonality between functional analysis and logic is that \enquote{functional} refers to a map acting on a set of functions.
\end{remark}

\begin{definition}\label{def:dual_linear_operator}\mimprovised
  We define the \term{dual linear operator} of \( T: U \to V \) as
  \begin{equation*}
    \begin{aligned}
      &T^*: V^* \to U^* \\
      &T^*(v^*) \coloneqq v^* \bincirc T.
    \end{aligned}
  \end{equation*}
\end{definition}

\begin{remark}\label{rem:vector_space_and_dual_space}
  A vector space \( V \) over \( \BbbK \) with \hyperref[def:hamel_basis]{basis} \( B \) is, by definition, isomorphic to the \hyperref[def:free_semimodule]{free module} \( \BbbK^{\oplus B} \). We can thus regard \( V \) as the set of all \hyperref[def:set_finiteness]{finitely}-\hyperref[def:function_support]{supported} functions from \( B \) to \( \BbbK \).

  By \fullref{thm:free_semimodule_universal_property}, the linear functions from \( V \) to \( \BbbK \) are precisely the linear extensions of the functions from \( B \) to \( \BbbK \).

  It is now clear that \( V \) can be embedded in \( V^* \). This is explicitly given by the map \( e \mapsto \pi_e \), where \( \pi_e \) is the \hyperref[def:basis_decomposition]{projection} onto the basis vector \( e \).

  The space \( V \) is thus finite-dimensional if and only if \( V \) and \( V^* \) are isomorphic. We often restrict ourselves to \hyperref[def:continuous_dual_space]{continuous linear functionals}, in which case even infinite-dimensional vector spaces can be isomorphic to their duals --- see \fullref{subsec:hilbert_spaces}.
\end{remark}

\begin{remark}\label{rem:finite_dimensional_dual_space_isomorphism}
  As discussed in \fullref{rem:vector_space_and_dual_space}, the vector space \( \BbbK^n \) is isomorphic to its dual.

  We discussed in \fullref{rem:matrices_as_functions} that vectors in \( \BbbK^n \) can be regarded as \hyperref[def:array/column_vector]{column vectors}. Depending on the situation, we regard linear functionals as either:
  \begin{itemize}
    \item Functions acting on vectors.
    \item Row vectors, which can be multiplied with column vectors from \( \BbbK^n \).
    \item Given the \hyperref[def:inner_product_space]{inner product} \( \inprod l x \coloneqq l^T x \), we can identify functionals with column vectors so that the functional \( l \) can be identified in \( x \mapsto \inprod l x \).
  \end{itemize}

  For example, given the real \hyperref[def:differentiability]{differentiable} function \( f(x, y) = xy \), we can regard its gradient at the point \( (x_0, y_0) \) as the row vector
  \begin{balign*}
    f'(x_0, y_0) =
    \begin{pmatrix}
      y_0 & x_0
    \end{pmatrix}.
  \end{balign*}

  This is a linear functional that acts on vectors from \( \BbbR^2 \) by multiplying them from the left.
\end{remark}

\begin{definition}\label{def:vector_space_annihilator}\mcite[52]{Knapp2016BasicAlgebra}
  Fix a subset \( S \subseteq V \) of the vector space \( V \) over \( \BbbK \). We define the \term{annihilator} of \( S \) as the vector space of functionals
  \begin{equation*}
    \op{ann}(S) \coloneqq \set{ l \in V^* \given \qforall {x \in S} l(x) = 0_\BbbK }.
  \end{equation*}
\end{definition}

\begin{example}\label{ex:def:vector_space_annihilator}
  We list several examples of \hyperref[def:vector_space_annihilator]{vector space annihilators}:
  \begin{thmenum}
    \thmitem{ex:def:vector_space_annihilator/whole} The annihilator of the entire space \( V \) is the zero subspace
    \begin{equation*}
      \op{ann}(V) = \set{ 0_{V^*} }.
    \end{equation*}

    \thmitem{ex:def:vector_space_annihilator/zero} The annihilator of the zero subspace \( \set{ 0_V } \) is the entire space
    \begin{equation*}
      \op{ann}(V) = V^*.
    \end{equation*}

    \thmitem{ex:def:vector_space_annihilator/complement} Consider the space \( \BbbR^2 \) with basis \( \set{ x, y } \). The annihilator of the subspace
    \begin{equation*}
      \set{ tx \given t \in \BbbR }
    \end{equation*}
    is
    \begin{equation*}
      \set{ ty \given t \in \BbbR }.
    \end{equation*}
  \end{thmenum}
\end{example}

\begin{remark}\label{rem:double_dual}
  We discussed in \fullref{rem:vector_space_and_dual_space} that any vector space \( V \) can be embedded into its dual \( V^* \). The dual can, in turn, be embedded into the double dual \( V^{**} \).

  What is more remarkable is that \( V \) can be directly embedded into \( V^{**} \) via by identifying the vector \( x \) with the map \( l \mapsto l(x) \).

  This is an isomorphism if and only if \( V \) is finite dimensional. When restricted to only \hyperref[def:continuous_dual_space]{continuous functionals}, it is possible that \( V \) is isomorphic to \( V^{**} \) --- see \fullref{subsec:reflexive_spaces}.
\end{remark}

\subsection{Matrices}\label{subsec:matrices}

\begin{definition}\label{def:array}
  Let \( X \) be any nonempty set. A \( k \)-dimensional \term{array} \( A \) of shape \( (n_1, \ldots, n_k) \) over \( X \) is a function of type
  \begin{equation*}
    A: \{ 1, 2, \ldots, n_1 \} \times \ldots \times \{ 1, 2, \ldots, n_k \} \to X.
  \end{equation*}

  In particular,
  \begin{defenum}
    \ilabel{def:array/matrix} two-dimensional arrays of shape \( n, m \) are usually called \term{matrices}. An \( n, m \)-matrix \( A \) is denoted as
    \begin{equation*}
      A = \{ a_{i,j} \}_{i,j=1}^{n,m}
    \end{equation*}
    or graphically as tables
    \begin{equation*}
      \begin{pmatrix}
        a_{1,1} & a_{1,2} & \cdots & a_{1,m} \\
        a_{2,1} & a_{2,2} & \cdots & a_{2,m} \\
        \vdots  & \vdots  & \ddots & \vdots  \\
        a_{n,1} & a_{n,2} & \cdots & a_{n,m}
      \end{pmatrix}.
    \end{equation*}

    The elements \( a_{1,1}, \ldots, a_{\min{n, m}, \min{n, m}} \) of a matrix are called its \term{main diagonal}.

    \ilabel{def:array/square_matrix} If \( n = m \), we call the matrix a \term{square matrix} of order \( n \).

    \ilabel{def:array/column_vector} matrices with only one column are called \term{column matrices}:
    \begin{equation*}
      \begin{pmatrix}
        a_{1,1} \\
        \vdots  \\
        a_{n,1}
      \end{pmatrix}.
    \end{equation*}

    \ilabel{def:array/row_vector} matrices with only one row are called \term{row matrices}:
    \begin{equation*}
      \begin{pmatrix}
        a_{1,1} & \cdots & a_{1,m}
      \end{pmatrix}.
    \end{equation*}

    \ilabel{def:array/vector} one-dimensional arrays are called simply \term{vectors} or \hyperref[def:cartesian_product]{tuples} and are usually written as either column vectors or row vectors.
  \end{defenum}
\end{definition}

\begin{remark}\label{rem:arrays_vs_tensors}
  Multidimensional arrays, as defined in \fullref{def:array}, are often called tensors, especially in machine learning where they are often used. This is a confusing practice since tensors (see \fullref{def:left_module_tensor_product}) are defined in a coordinate-independent fashion.

  A single tensor can be represented by different arrays and the same array can represent multiple tensors.
\end{remark}

\begin{definition}\label{def:block_matrix}
  A \term{block matrix} is a \enquote{matrix of matrices}, that is, a matrix of the form
  \begin{equation*}
    \begin{pmatrix}
      A_{1,1} & A_{1,2} & \cdots & A_{1,m} \\
      A_{2,1} & A_{2,2} & \cdots & A_{2,m} \\
      \vdots  & \vdots  & \ddots & \vdots  \\
      A_{n,1} & A_{n,2} & \cdots & A_{n,m}
    \end{pmatrix},
  \end{equation*}
  where all \( A_{i,j} \) are matrices of compatible dimensions.

  We usually write the block matrix
  \begin{equation*}
    \begin{pmatrix}
      A      & \cdots & B      \\
      \vdots & \ddots & \vdots \\
      C      & \cdots & D
    \end{pmatrix}
  \end{equation*}
  as
  \begin{equation*}
    \p*{\begin{array}{ccc|c|ccc}
      a_{1,1}   & \cdots & a_{1,m_A}   & \cdots & b_{1,1}   & \cdots & b_{1,m_B} \\
      \vdots    & \ddots & \vdots      & \cdots & \vdots    & \ddots & \vdots \\
      a_{n_A,1} & \cdots & a_{n_A,m_A} & \cdots & b_{n_B,1} & \cdots & b_{n_B,m_B} \\
      \hline
      \vdots    & \vdots & \vdots      & \ddots & \vdots    & \vdots & \vdots \\
      \hline
      c_{1,1}   & \cdots & c_{1,m_C}   & \cdots & d_{1,1}   & \cdots & d_{1,m_D} \\
      \vdots    & \ddots & \vdots      & \cdots & \vdots    & \ddots & \vdots \\
      c_{n_C,1} & \cdots & c_{n_C,m_C} & \cdots & d_{n_D,1} & \cdots & d_{n_D,m_D} \\
    \end{array}}.
  \end{equation*}

  Given any matrix \( A = \{ a_{i,j} \}_{i,j=1}^{n,m} \), we sometimes consider its block matrix of \term{rows}
  \begin{equation*}
    \p*{\begin{array}{c}
      a_{1,-} \\
      \hline
      a_{2,-} \\
      \hline
      \vdots \\
      \hline
      a_{n,-}
    \end{array}},
  \end{equation*}
  consisting of row vectors, and its block matrix of of \term{columns}
  \begin{equation*}
    \p*{\begin{array}{c|c|c|c}
      a_{-,1} & a_{-,2} & \cdots & a_{-,m},
    \end{array}}
  \end{equation*}
  consisting of column vectors.
\end{definition}

\begin{definition}\label{def:left_module_of_tuples}
  Let \( R \) be a \hyperref[def:semiring]{semiring}. Let \( R^n \) be the set of all \( n \)-\hyperref[def:array/vector]{tuples} over \( R \), that is,
  \begin{equation*}
    R^n = R \times R \times \cdots \times R.
  \end{equation*}

  It is customary to denote elements of \( R^n \) by
  \begin{equation*}
    x = \begin{pmatrix} x_1 \\ \vdots \\ x_n \end{pmatrix}.
  \end{equation*}
  rather than
  \begin{equation*}
    x = (x_1, \ldots, x_n).
  \end{equation*}

  This highlights that \( R^n \) is usually treated as a space of column vectors.

  Define the operations
  \begin{balign*}
     & +: R^n \times R^n \to R^n
    \\
     & \begin{pmatrix} x_1 \\ \vdots \\ x_n \end{pmatrix}
    +
    \begin{pmatrix} y_1 \\ \vdots \\ y_n \end{pmatrix}
    =
    \begin{pmatrix} x_1 + y_1 \\ \vdots \\ x_n + y_n \end{pmatrix}
    \\
    \\
     & \cdot: R \times R^n \to R^n
    \\
     & \lambda \cdot \begin{pmatrix} x_1 \\ \vdots \\ x_n \end{pmatrix}
    =
    \begin{pmatrix} \lambda x_1 \\ \vdots \\ \lambda x_n \end{pmatrix}.
  \end{balign*}

  With these operations defined, \( R^n \) becomes a semiring \hyperref[def:left_module]{module}.

  In particular, if \( R \) is a \hyperref[def:semiring/field]{field}, \( R^n \) is a \hyperref[def:vector_space]{vector space} and we refer to it as a \term{tuple space}. We are usually only concerned with the vector spaces \( \BbbR^n \) and \( \BbbC^n \).
\end{definition}

\begin{proposition}\label{thm:matrix_spaces_are_tuple_spaces}
  The vector spaces \( F^{n \times m} \) and \( F^{nm} \) are isomorphic with an isomorphism defined by \fullref{def:double_index_maps}.
\end{proposition}

\begin{remark}\label{rem:vector_spaces_of_tuples_and_matrices}
  \Fullref{thm:finite_dimensional_spaces_are_isomorphic} provides a justification for working with vector spaces of tuples instead of arbitrary vector spaces.

  \Fullref{thm:finite_dimensional_operators_are_isomorphic_to_matrices} provides a justification for working with vector spaces of matrices instead of arbitrary spaces of linear operators.
\end{remark}

\begin{theorem}\label{thm:finite_dimensional_spaces_are_isomorphic}
  Every \( n \)-\hyperref[def:vector_space_dimension]{dimensional} \hyperref[def:vector_space]{vector space} over the field \( \BbbK \) is isomorphic to \( \BbbK^n \) (see \fullref{def:left_module_of_tuples}).
\end{theorem}
\begin{proof}
  Let \( V \) be an arbitrary \( n \)-dimensional vector space over \( \BbbK \). Since a basis of \( V \) exists\AOC by \fullref{thm:all_vector_spaces_are_free_left_modules}, fix a basis and fix an ordering \( b_1, \ldots, b_n \) of the basis vectors. Denote the projection \hyperref[def:left_module_basis_projection]{maps} by \( \pi_{b_i} \).

  Define the function
  \begin{balign*}
     & L: V \to \BbbK^n                                      \\
     & L(x) \coloneqq (\pi_{b_1}(x), \ldots, \pi_{b_n}(x))
  \end{balign*}
  that maps a vector \( x \in V \) into an \( n \)-tuple of the projections of \( x \) along the ordered basis \( b_1, \ldots, b_n \). It is linear since, by \fullref{thm:left_module_basis_projections_are_linear}, the projections are linear.

  Now define the inverse function
  \begin{balign*}
     & P: \BbbK^n \to V                                      \\
     & P(y_1, \ldots, y_n) \coloneqq \sum_{i=1}^n y_i b_i,
  \end{balign*}
  which is obviously linear.

  The composition of \( L \) and \( P \) is the identity mapping on \( V \). Indeed, for any \( x \in V \),
  \begin{equation*}
    (P \circ L)(x)
    =
    P(\pi_{b_1}(x), \ldots, \pi_{b_n}(x))
    =
    \sum_{i=1}^n \pi_{b_i}(x) b_i
    =
    x.
  \end{equation*}
\end{proof}

\begin{definition}\label{def:algebra_of_matrices}
  Denote by \( R^{n \times m} \) the set of \( n, m \)-matrices over the semiring \( R \). We define three operations on matrices:

  \begin{defenum}
    \ilabel{def:algebra_of_matrices/addition} We define \term{matrix addition} as
    \begin{balign*}
       & +: R^{n,m} \times R^{n,m} \to R^{n,m} \\
       & \begin{pmatrix}
        a_{1,1} & \cdots & a_{1,m} \\
        \vdots  & \ddots & \vdots  \\
        a_{n,1} & \cdots & a_{n,m}
      \end{pmatrix}
      +
      \begin{pmatrix}
        b_{1,1} & \cdots & b_{1,m} \\
        \vdots  & \ddots & \vdots  \\
        b_{n,1} & \cdots & b_{n,m}
      \end{pmatrix}
      \coloneqq
      \begin{pmatrix}
        a_{1,1} + b_{1,1} & \cdots & a_{1,m} + b_{1,m} \\
        \vdots            & \ddots & \vdots            \\
        a_{n,1} + b_{n,1} & \cdots & a_{n,m} + b_{n,m}
      \end{pmatrix}
    \end{balign*}

    \ilabel{def:algebra_of_matrices/scalar_multiplication} We define \term{scalar multiplication} as
    \begin{balign*}
       & \cdot: R \times R^{n,m} \to R^{n,m}      \\
       & \lambda \cdot \begin{pmatrix}
        a_{1,1} & \cdots & a_{1,m} \\
        \vdots  & \ddots & \vdots  \\
        a_{n,1} & \cdots & a_{n,m}
      \end{pmatrix}
      \coloneqq
      \begin{pmatrix}
        \lambda a_{1,1} & \cdots & \lambda a_{1,m} \\
        \vdots          & \ddots & \vdots          \\
        \lambda a_{n,1} & \cdots & \lambda a_{n,m}
      \end{pmatrix}
    \end{balign*}

    \ilabel{def:algebra_of_matrices/matrix_multiplication} We define \term{matrix multiplication} in two steps. The complexity of the definition is justified by \fullref{thm:finite_dimensional_operators_are_isomorphic_to_matrices}. First, if \( a \in R^{1,n} \) is a row \hyperref[def:array/row_vector]{vector} and \( b \in R^{n,1} \) is a column \hyperref[def:array/column_vector]{vector}, we define their \term{inner product} to be
    \begin{equation}
      a \cdot b \coloneqq \sum_{i=1}^n a_i b_i.
    \end{equation}

    We can now define matrix multiplication as
    \begin{balign*}
       & \odot: R^{n,m} \times R^{m,k} \to R^{n,k} \\
       & \p*{\begin{array}{c}
        a_{1,-} \\
        \hline
        a_{2,-} \\
        \hline
        \vdots \\
        \hline
        a_{n,-}
      \end{array}}
      \odot
      \p*{\begin{array}{c|c|c|c}
        \scriptstyle{b_{-,1}} & \scriptstyle{b_{-,2}} & \cdots & \scriptstyle{b_{-,m}}
      \end{array}}
      \coloneqq
      \begin{pmatrix}
        a_{1,-} \cdot b_{-,1} & a_{1,-} \cdot b_{-,2} & \vdots & a_{1,-} \cdot b_{-,m} \\
        a_{2,-} \cdot b_{-,1} & a_{2,-} \cdot b_{-,2} & \vdots & a_{2,-} \cdot b_{-,m} \\
        \vdots                & \vdots                & \ddots & \vdots                \\
        a_{n,-} \cdot b_{-,1} & a_{n,-} \cdot b_{-,2} & \cdots & a_{n,-} \cdot b_{-,m}
      \end{pmatrix}.
    \end{balign*}
  \end{defenum}

  With \hyperref[def:algebra_of_matrices/addition]{addition} and scalar \hyperref[def:algebra_of_matrices/scalar_multiplication]{multiplication}, \( R^{n \times m} \) becomes a semiring \hyperref[def:left_module]{module}.

  In the special case where \( R \) is a commutative unital ring and \( n = m \), we can add matrix \hyperref[def:algebra_of_matrices/matrix_multiplication]{multiplication} to the module \( R^{n \times n} \) so that it becomes an \hyperref[def:algebra_over_ring]{algebra} over \( R \).
\end{definition}

\begin{example}\label{ex:matrix_multiplication_is_noncommutative}
  The matrix algebra \( R^{n \times n} \) is a noncommutative ring. Consider the following example:
  \begin{balign*}
    \begin{pmatrix}
      0 & 0 \\
      0 & 1
    \end{pmatrix}
    \begin{pmatrix}
      1 & 0 \\
      1 & 0
    \end{pmatrix}
     & =
    \begin{pmatrix}
      0 & 0 \\
      1 & 0
    \end{pmatrix}
    \\
    \begin{pmatrix}
      1 & 0 \\
      1 & 0
    \end{pmatrix}
    \begin{pmatrix}
      0 & 0 \\
      0 & 1
    \end{pmatrix}
     & =
    \begin{pmatrix}
      0 & 0 \\
      0 & 0
    \end{pmatrix}
  \end{balign*}
\end{example}

\begin{proposition}\label{thm:finite_dimensional_operators_are_isomorphic_to_matrices}
  Fix a dioid \( R \). The \hyperref[def:algebra_of_matrices]{matrix vector space} \( R^{n \times m} \) is isomorphic to the vector space of all linear maps \( \hom(R^m, R^n) \) (note that the maps are from \( R^m \) to \( R^n \)).

  In the special case where \( R \) is a commutative unital ring and \( n = m \), the algebra \( R^{n \times n} \) with matrix multiplication as vector multiplication is isomorphic to \( \End(R^n) \) with function composition as vector multiplication. In particular, this justifies using juxtaposition for application of linear functions, e.g. \( Lx \) rather than \( L(x) \).
\end{proposition}
\begin{proof}
  Let \( L: R^m \to R^n \) be a linear map and let \( e_1, \ldots, e_m \) be the basis vectors in \( R^m \). Denote by \( \pi_i, i = 1, \ldots, m \) be the basis \hyperref[def:left_module_basis_projection]{projections}. We construct a matrix as follows:
  \begin{equation*}
    A_L \coloneqq \p*{\begin{array}{c|c|c|c}
      L(e_1) & L(e_2) & \cdots & L(e_m)
    \end{array}}
  \end{equation*}

  Conversely, given a matrix \( A \in R^{n \times m} \), we define the linear map
  \begin{equation*}
    \hat L_A(x) \coloneqq Ax
  \end{equation*}
  by left multiplication of a vector with \( A \).

  It remains to show that these are mutually inverse. Let \( L: R^m \to R^n \) and \( x \in R^m \). We have
  \begin{equation*}
    L_{A_L}(x) = A_L x = \sum_{i=1}^m \pi_i(x) L(e_i) = L\left(\sum_{i=1}^m \pi_i(x) e_i \right) = L(x).
  \end{equation*}

  Conversely, let \( A \in R^{n \times m} \) and \( x \in R^m \). We have
  \begin{balign*}
    A_{L_A} x
     & =
    \p*{\begin{array}{c|c|c|c}
      L_A(e_1) & L_A(e_2) & \cdots & L_A(e_m)
    \end{array}}
    x
    =    \\ &=
    \p*{\begin{array}{c|c|c|c}
      A e_1 & A e_2 & \cdots & A e_m
    \end{array}}
    x
    =    \\ &=
    \sum_{i=1}^m \pi_i(x) A e_i
    =    \\ &=
    A \left( \sum_{i=1}^m \pi_i(x) e_i \right)
    =    \\ &=
    Ax.
  \end{balign*}

  It trivially follows that linear function composition corresponds to matrix multiplication.
\end{proof}

\begin{definition}\label{def:matrix_determinant}\mcite\cite[215]{Knapp2016BAlg}
  Fix the \hyperref[def:algebra_of_matrices]{matrix space} \( R^{n \times n} \) over a commutative unital ring \( R \). We define its determinant as
  \begin{balign*}
     & \det: R^{n \times n} \to R                                                                  \\
     & \det(\{ a_{i,j} \}_{i,j=1}^n) \coloneqq \sum_{p \in S_n} \sgn(p) \prod_{i=1}^n a_{i,p(i)},
  \end{balign*}
  where \( S_n \) is the \hyperref[def:symmetric_group]{symmetric group} of order \( n \).

  The determinant of a matrix is not invertible, we say that it is \term{singular}. \Fullref{thm:matrix_invertible_iff_nonsingular} gives a strong link between the invertibility of a matrix and the invertibility of its determinant.

  If \( R \) is a field, then only the zero is not invertible and hence only matrices with \( \det A = 0 \) are singular.
\end{definition}

\begin{proposition}\label{thm:matrix_determinant_properties}\mcite\cite[prop. 5.1]{Knapp2016BAlg}
  Matrix determinants over the commutative unital ring \( R \) have the following basic properties:
  \begin{propenum}
    \ilabel{thm:matrix_determinant_properties/identity} For the identity matrix \( E_n \in R^n \) we have
    \begin{equation*}
      \det(E_n) = 1.
    \end{equation*}

    \ilabel{thm:matrix_determinant_properties/transpose} For the transpose \hyperref[def:matrix_transpose]{matrix} \( A^T \) of \( A \in R^n \), we have
    \begin{equation*}
      \det(A^T) = \det(A).
    \end{equation*}

    \ilabel{thm:matrix_determinant_properties/product} For matrices \( A, B \in R^n \) we have
    \begin{equation*}
      \det(A) \det(B) = \det(AB).
    \end{equation*}
  \end{propenum}
\end{proposition}

\begin{proposition}\label{thm:matrix_invertible_iff_nonsingular}\mcite\cite[cor. 5.5]{Knapp2016BAlg}
  A matrix over \( R \) is invertible if and only if its determinant is invertible in \( R \).

  In particular, a matrix over a field is invertible if and only if its determinant is nonzero.
\end{proposition}

\begin{definition}\label{def:inverse_matrix}
  Let \( A \) be a square matrix of order \( n \) over a dioid \( R \). We say that \( B \) is an \term{inverse matrix} of \( A \) if
  \begin{equation*}
    AB = BA = E_n.
  \end{equation*}

  An inverse matrix, if it exists, is unique. We denote this inverse of \( A \) by \( A^{-1} \).

  The set of all invertible matrices of order \( n \) over \( R \) is called the \term{general linear group} and is denoted by \( \op{GL}_n(R) \). It forms a group with respect to matrix multiplication.

  If \( R \) is commutative, we also consider the \term{special linear group} \( \op{SL}_n(R) \) of matrices with \hyperref[def:matrix_determinant]{determinant} \( 1 \).
\end{definition}
\begin{proof}
  The inverse is unique by \fullref{def:unital_magma_inverse_element_unique}.
\end{proof}

\begin{proposition}\label{thm:general_linear_group_isomorphic_to_automorphism_group}
  Fix a dioid \( R \). The general linear group \( \op{GL}_n(R) \) is isomorphic to the group of all invertible linear transformations over \( R^n \) under composition.
\end{proposition}
\begin{proof}
  Follows from \fullref{thm:finite_dimensional_operators_are_isomorphic_to_matrices}.
\end{proof}

\begin{definition}\label{def:orthogonal_matrix}
  Let \( R \) be a dioid. We say that the square matrix \( A \) is \term{orthogonal} if \( A^T = A^{-1} \). If \( R \) is a commutative unital ring, the set of all orthogonal matrices of order \( n \) forms a subgroup of \( \op{GL}_n(R) \) called the \term{orthogonal group} \( \op{O}_n(R) \).
\end{definition}

\begin{definition}\label{def:unitary_matrix}
  We say that the complex square matrix \( A \) is \term{unitary} if \( A^\dagger = A^{-1} \). The set of all unitary matrices of order \( n \) is called the unitary group \( \op{U}_n \) and is a subgroup of \( \op{GL}_n(\co) \).
\end{definition}

\begin{definition}\label{def:matrix_column_and_row_space}
  Fix a matrix \( A \in R^{n \times m} \) over a semiring \( R \). We define its \term{row space} as
  \begin{equation*}
    \linspan \{ a_{i,-} \colon i = 1, \ldots, n \}
  \end{equation*}
  and its \term{column space} as
  \begin{equation*}
    \linspan \{ a_{-,j} \colon j = 1, \ldots, m \}.
  \end{equation*}
\end{definition}

\begin{definition}\label{def:matrix_transpose}
  Let \( A = \{ a_{i,j} \}_{i,j=1}^{n,m} \) be a matrix. We define its \term{transpose matrix} by \term{flipping it over its main diagonal}, that is,
  \begin{equation*}
    A^T \coloneqq \begin{pmatrix}
      a_{1,1} & a_{2,1} & \cdots & a_{n,1} \\
      a_{1,2} & a_{2,2} & \cdots & a_{n,2} \\
      \vdots  & \vdots  & \ddots & \vdots  \\
      a_{1,m} & a_{2,m} & \cdots & a_{n,m}
    \end{pmatrix}.
  \end{equation*}
\end{definition}

\begin{definition}\label{def:symmetric_matrix}
  A square matrix \( A \) is said to be \term{symmetric} if \( A = A^T \).
\end{definition}

\begin{definition}\label{def:matrix_conjugate_transpose}
  Let \( A \) be a complex matrix. We define its \term{conjugate transpose matrix} as
  \begin{equation*}
    A^\dagger \coloneqq \begin{pmatrix}
      \overline{a_{1,1}} & \overline{a_{2,1}} & \cdots & \overline{a_{n,1}} \\
      \overline{a_{1,2}} & \overline{a_{2,2}} & \cdots & \overline{a_{n,2}} \\
      \vdots       & \vdots       & \ddots & \vdots       \\
      \overline{a_{1,m}} & \overline{a_{2,m}} & \cdots & \overline{a_{n,m}}
    \end{pmatrix}.
  \end{equation*}
\end{definition}

\begin{definition}\label{def:hermitian_matrix}
  If \( A \) is a complex matrix, we say that it is Hermitian if \( A = A^\dagger \).
\end{definition}

\begin{proposition}\label{thm:dual_linear_operator_matrix_transpose}
  Let \( L: F^m \to F^n \) be a linear operator and let \( L^*: {F^n}^* \to {F^m}^* \) be its dual \hyperref[def:dual_linear_operator]{operator}.

  If \( A \in F^{n \times m} \) is the matrix of \( L \), then its \hyperref[def:matrix_transpose]{transpose} \( A^T \) is the matrix of \( L^* \) when regarding \( L^* \) as an operator acting on column vectors.
\end{proposition}
\begin{proof}
  Let \( l \in {F^n}^* \) be a linear functional regarded as a function and \( \overrightarrow l \) be the same functional regarded as a column \hyperref[rem:finite_dimensional_dual_space_isomorphism]{vector}. We have
  \begin{equation*}
    L^*(l)
    =
    l \circ L
    =
    (x \mapsto l(L(x)))
    =
    (x \mapsto \overrightarrow l^T Ax)
    =
    \overrightarrow l^T A.
  \end{equation*}

  Thus
  \begin{equation*}
    L^*(l) = A^T \overrightarrow l,
  \end{equation*}
  i.e. the matrix \( A^T \) corresponds to the dual operator \( L^* \).
\end{proof}

\begin{proposition}\label{thm:column_and_row_spaces_are_images}
  Fix a semiring \( R \). Let \( L: R^m \to R^n \) be a linear map and let \( A \in R^{n \times m} \) be the corresponding \hyperref[thm:finite_dimensional_operators_are_isomorphic_to_matrices]{matrix}. The \hyperref[def:matrix_column_and_row_space]{column space} of \( A \) is isomorphic to the image \( \img(L) \) and the row space is isomorphic to \( \img(L^*) \).
\end{proposition}
\begin{proof}
  The column space of \( A \) lies within \( R^{n \times 1} \), which is isomorphic to \( R^n \). We will assume that it is a subset of \( R^n \) and will prove that it is equal to \( \img(L) \).

  Denote by \( e_1, \ldots, e_m \) the basis of \( R^m \). The \( j \)-th column \( a_{-,j} \) of \( A \) can be represented as
  \begin{equation*}
    A e_j = a_{-,j}.
  \end{equation*}

  Thus \( a_{-,j} \in \img(L), j = 1, \ldots, m \). Since \( \img(L) \) is a linear subspace of \( R^n \), it contains the linear span of any finite collection of its vectors. Consequently, the column space of \( A \) is a subspace of \( \img(L) \).

  To see the converse, let \( x \in \BbbR^m \). We have
  \begin{equation*}
    L(x) = Ax = \sum_{j=1}^m x_i A e_j = \sum_{j=1}^m a_{-,j}.
  \end{equation*}

  Hence the image of any vector \( x \) under \( L \) is a linear combination of the columns of \( A \).

  Thus proves that the column space of \( A \) is equal to \( \img(L) \).

  The proof that the row space is isomorphic to \( \img(L^*) \) is identical, noting that \( A^T \) corresponds to \( L^* \) by \fullref{thm:dual_linear_operator_matrix_transpose}.
\end{proof}

\subsection{Diagonalization}\label{subsec:diagonalization}

\begin{definition}\label{def:diagonal_matrix}
  We say that the square \hyperref[def:array/matrix]{matrix} \( \{ a_{i,j} \}_{i,j=1}^{n,n} \) over a field \( \BbbK \) is \term{diagonal} if \( a_{i,j} = 0 \) whenever \( i \neq j \), that is, only the main diagonal has nonzero elements:
  \begin{equation*}
    \begin{pmatrix}
      a_{1,1} & 0       & \cdots & 0       \\
      0       & a_{2,2} & \cdots & 0       \\
      \vdots  & \vdots  & \ddots & \vdots  \\
      0       & 0       & \cdots & a_{n,n}
    \end{pmatrix}.
  \end{equation*}

  If all elements along the main diagonal are \( 1 \), we call the matrix the \term{identity matrix} of order \( n \) and denote it by
  \begin{equation*}
    E_n \coloneqq
    \begin{pmatrix}
      1      & 0      & \cdots & 0      \\
      0      & 1      & \cdots & 0      \\
      \vdots & \vdots & \ddots & \vdots \\
      0      & 0      & \cdots & 1
    \end{pmatrix}.
  \end{equation*}
\end{definition}

\begin{definition}\label{def:diagonalizable_matrix}
  A matrix over a field is \term{diagonalizable} if it is isomorphic to a diagonal matrix.
\end{definition}

\begin{theorem}[Spectral theorem for matrices]\label{def:spectral_theorem_for_matrices}\mcite\cite[thm. 6.5]{Knapp2016BAlg}
  If \( R \) is a ring and \( 2 \) is a unit in \( R \), every symmetric matrix over \( R \) is diagonalizable.
\end{theorem}

\subsection{Bilinear forms}\label{subsec:bilinear_forms}

\begin{definition}\label{def:bilinear_form}\mcite[249]{Knapp2016BasicAlgebra}
  Let \( M \) and \( N \) be left \( R \)-modules and \( L: M \times N \to R \) be a \hyperref[def:multilinear_function]{multilinear function}. We say that \( L \) is a \term{bilinear form}.

  If \( M = N \), we have the following additional types of bilinear forms:
  \begin{thmenum}
    \thmitem{def:bilinear_form/symmetric} If \( L \) is a \hyperref[def:multi_valued_function/symmetric]{symmetric function}, we say that is is a \term{symmetric bilinear form}

    \thmitem{def:bilinear_form/skew_symmetric} If for all \( x, y \in M \) instead of \( L(x, y) = L(y, x) \) we have \( L(x, y) = -L(y, x) \), we say that \( L \) is \term{skew-symmetric}.

    \thmitem{def:bilinear_form/alternating} If for all \( x \in M \) we have \( L(x, x) = 0 \), we say that \( L \) is \term{alternating}.
  \end{thmenum}
\end{definition}

\begin{proposition}\label{thm:skew_symmetric_iff_alternating}
  Let \( 1 + 1 = 2 \) be a unit in the ring \( R \). Let \( M \) be a left module over \( R \). Then the bilinear form \( L: M \times M \to R \) is \hyperref[def:bilinear_form/alternating]{alternating} if and only if it is \hyperref[def:bilinear_form/skew_symmetric]{skew-symmetric}.

  Alternating implies skew-symmetric even if \( 2 \) is not invertible.
\end{proposition}
\begin{proof}
  \SufficiencySubProof Let \( L \) be alternating. Then
  \begin{balign*}
    L(x + y, x + y) & = L(x, x) + L(x, y) + L(y, x) + L(y, y) \\
    0               & = L(x, y) + L(y, x)                     \\
    L(x, y) = -L(y, x).
  \end{balign*}

  \NecessitySubProof Let \( L \) be skew-symmetric. Then
  \begin{equation*}
    L(x, x) = -L(x, x),
  \end{equation*}
  which implies that \( 2L(x, y) = 0 \). Hence, \( L \) is alternating if we are able to divide by \( 2 \).
\end{proof}

\begin{definition}\label{def:bilinear_form_radicals}\mcite[250]{Knapp2016BasicAlgebra}
  Let \( L: M \times N \to R \) be a bilinear form. We define its \term{left radical}
  \begin{equation*}
    \{ x \in M \colon \forall y \in N, \inprod x y = 0 \}
  \end{equation*}
  and \term{right radical}
  \begin{equation*}
    \{ y \in N \colon \forall x \in M, \inprod x y = 0 \}.
  \end{equation*}

  Note that if \( L \) is symmetric or skew-symmetric (which also implies \( M = N \)), the two are identical and we speak simply of the \term{radical} \( \sqrt L \).
\end{definition}

\begin{definition}\label{def:nondegenerate_bilinear_form}\mcite[249]{Knapp2016BasicAlgebra}
  We say that a bilinear form \( L: M \times N \to R \) is \term{nondegenerate} if both its left and right \hyperref[def:bilinear_form_radicals]{radicals} are nontrivial.
\end{definition}

\begin{theorem}\label{thm:bilinear_form_matrix_presentation}
  Fix a commutative unital ring \( R \) and a bilinear form \( L: R^n \times R^m \to R \). Then there exists a matrix \( A \in R^{n \times m} \) such that
  \begin{equation*}
    L(x, y) \coloneqq x^T A y.
  \end{equation*}

  This matrix is called the generalized \term{Gram matrix}.

  In particular, if \( L \) is \hyperref[def:multi_valued_function/symmetric]{symmetric}, so it \( A \).
\end{theorem}
\begin{proof}
  Denote by \( e_1, \ldots, e_n \) the basis of \( R^n \) and by \( f_1, \ldots, f_m \) the basis of \( R^m \).

  Define the matrix \( A = \{ a_{i,j} \}_{i,j=1}^{n,m} \) by
  \begin{equation*}
    a_{i,j} \coloneqq L(e_i, f_j).
  \end{equation*}

  Note that if \( n = m \) and if \( L \) is symmetric, then the matrix \( A \) is obviously symmetric too.

  For any fixed basis vector \( e_i, i = 1, \ldots, n \) of \( R^n \), we have
  \begin{equation*}
    L(e_i, y)
    =
    \sum_{j=1}^m y_i L(e_i, f_j)
    =
    y_i a_{(i,-)},
  \end{equation*}
  where \( a_{(i,-)} \) is the \( i \)-th row of \( A \).

  Thus, for an arbitrary \( x \in R^n \)
  \begin{equation*}
    L(x, y)
    =
    \sum_{i=1}^n x_i L(e_i, y)
    =
    \sum_{i=1}^n x_i (a_{(i,-)} y)
    =
    \left( \sum_{i=1}^n x_i a_{(i,-)} \right) y
    =
    x^T A y.
  \end{equation*}
\end{proof}

\begin{corollary}\label{thm:bilinear_forms_isomorphic_to_matrices}
  Fix a commutative unital ring \( R \). The vector space of bilinear forms of type \( R^n \times R^m \to R \) is isomorphic to the matrix space \( A \in R^{n \times m} \).
\end{corollary}

\begin{definition}\label{def:sesquilinear_form}\mcite[258]{Knapp2016BasicAlgebra}
  Let \( V \) be a complex vector space and let \( \overline V \) be its conjugate \hyperref[def:complex_conjucate_vector_space]{transpose}. We call the bilinear form \( L: V \times \overline V \to \BbbC \) a \term{sesquilinear form} (we say that \( L \) is \enquote{semilinear} in its second argument and \enquote{sesqui} means \enquote{one and a half} is Latin).

  Similar to \fullref{def:bilinear_form}, we have
  \begin{thmenum}
    \thmitem{def:sesquilinear_form/hermitian} If for all \( x, y \in V \) we have \( L(x, y) = \overline{L(y, x)} \), we say that \( L \) is \term{Hermitian}.

    \thmitem{def:sesquilinear_form/skew_hermitian} If for all \( x, y \in V \) we have \( L(x, y) = -\overline{L(y, x)} \), we say that \( L \) is \term{skew-Hermitian}.
  \end{thmenum}
\end{definition}

\begin{definition}\label{def:duality_pairing}
  Let \( M \) and \( N \) be left \( R \)-modules. A \term{duality pairing} \( \inprod \cdot \cdot: M \times N \to R \) is a \hyperref[def:nondegenerate_bilinear_form]{nondegenerate} bilinear form.

  See \fullref{def:canonical_duality_pairing} and \fullref{def:locally_convex_duality_pairing}.
\end{definition}

\begin{definition}\label{def:quadratic_form}
  If \( L: M \times M \to R \) be a bilinear \hyperref[def:bilinear_form]{form}, we call the function
  \begin{equation*}
    Q(x) \coloneqq L(x, x)
  \end{equation*}
  a \term{quadratic form} over \( M \).
\end{definition}

\begin{definition}\label{def:quadratic_form_definiteness}
  \todo{Define definiteness of quadratic forms}
\end{definition}

\begin{definition}\label{def:homogenous_function}
  Let \( M \) and \( N \) be left \( R \)-modules. We say that the function \( f: M \to N \) is homogeneous with degree \( n \) if for all \( t \in R \) and \( x \in M \) we have
  \begin{equation*}
    f(t x) = t^n f(x).
  \end{equation*}
\end{definition}

\begin{proposition}\label{thm:bilinear_forms_vs_to_quadratic_forms}
  A \term{quadratic form} \( Q: M \to R \) is a \hyperref[def:homogenous_function]{homogeneous function} of degree \( 2 \). In particular, \( Q(x) = Q(-x) \).
\end{proposition}
\begin{proof}
  Let \( L: M \times M \to R \) be the corresponding bilinear form. Then, by \fullref{def:semimodule/homomorphism/homogeneity},
  \begin{equation*}
    Q(tx) = L(tx, tx) = t^2 L(x, x) = t^2 Q(x).
  \end{equation*}
\end{proof}

\begin{proposition}\label{thm:polarization_identity}\mcite{nLab:polarization_identity}
  Let \( L: M \times M \to R \) be a bilinear \hyperref[def:bilinear_form]{form} and \( Q: M \to R \) be its associated quadratic \hyperref[def:quadratic_form]{form}. Then the \term{polarization identity} holds:
  \begin{equation}\label{thm:polarization_identity/polarization_identity}
    2 L(x, y) + 2 L(y, x) = Q(x + y) - Q(x - y)
  \end{equation}

  The similar looking, but slightly less useful parallelogram law also holds:
  \begin{equation}\label{thm:polarization_identity/parallelogram_law}
    2 Q(x) + 2 Q(y) = Q(x + y) + Q(x - y)
  \end{equation}

  If \( 2 = 1 + 1 \) is a unit in \( R \), we can \enquote{recover} from \( Q \) the bilinear form:
  \begin{equation}\label{thm:polarization_identity/symmetrization_definition}
    \hat L(x, y) \coloneqq \frac 1 2 \left[ Q(x + y) - Q(x) - Q(y) \right]
  \end{equation}

  The function \( \hat L \) is \hyperref[def:multi_valued_function/symmetric]{symmetric} and is called the \term{symmetrization} of \( L \). If \( L \) itself is symmetric, \( L = \hat L \).
\end{proposition}
\begin{proof}
  Identities \fullref{thm:polarization_identity/polarization_identity,thm:polarization_identity/parallelogram_law,thm:polarization_identity/symmetrization_definition} all follow from the bilinearity of \( L \), that is,
  \begin{equation*}
    Q(x \pm y)
    =
    L(x, x) \pm L(x, y) \pm L(y, x) + L(y, y)
    =
    [Q(x) + Q(y)] \pm [L(x, y) + L(y, x)].
  \end{equation*}
\end{proof}

\begin{definition}\label{def:orthogonality}
  Let \( U \) and \( V \) be vector spaces over \( \BbbK \) and let \( L: U \times V \to \BbbK \) be a nondegenerate bilinear form. We say that the vectors \( x \in U \) and \( y \in V \) are \term{orthogonal} with respect to \( L \) if
  \begin{equation*}
    L(x, y) = 0.
  \end{equation*}

  For every subspace \( U \subseteq V \) we define its \term{orthogonal complement} with respect to \( L \) as
  \begin{equation*}
    U^\perp \coloneqq \set{ x \in U \colon L(x, y) = 0 \T{for all} y \in V }
  \end{equation*}
  and analogously for submodules of \( V \).

  Let \( \mscrK \) be an index set and \( \seq{ x_k }_{k \in \mscrK} \subseteq U \), \( \seq{ y_k }_{k \in \mscrK} \subseteq V \) be two families of vectors indexed by \( \mscrK \). We say that these families form a \term{biorthogonal system} with respect to \( L \) if
  \begin{equation*}
    L(x_k, y_m) = 0 \text{ follows from } k \neq m
  \end{equation*}

  If \( U = V \), we usually consider \term{orthogonal systems} \( \seq{ x_k }_{k \in \mscrK} \subseteq V \) where
  \begin{equation*}
    L(x_k, x_m) = 0 \iff k \neq m
  \end{equation*}
\end{definition}

\begin{definition}\label{def:inner_product_space}
  An \term{inner product space} is a vector space \( V \) over \( F \) equipped with a positive \hyperref[def:quadratic_form_definiteness]{definite} \hyperref[def:bilinear_form/symmetric]{symmetric} bilinear form \( \inprod \cdot \cdot: V \times V \to F \).

  In the special case where \( F = \BbbC \), by convention we require \( V \) to instead be equipped with a positive definite \hyperref[def:sesquilinear_form/hermitian]{Hermitian} sesquilinear form instead.
\end{definition}

\begin{definition}\label{def:symplectic_vector_space}
  A \term{symplectic vector space} is a vector space \( V \) over \( F \) equipped with a \hyperref[def:bilinear_form/symmetric]{nondegenerate}  \hyperref[def:bilinear_form/alternating]{alternating} bilinear form \( \inprod \cdot \cdot: V \times V \to F \).
\end{definition}

\begin{lemma}\label{thm:inner_product_quadratic_form_is_positive_definite}
  Let \( V \) be a real or complex \hyperref[def:inner_product_space]{inner product space} with product \( \inprod \cdot \cdot \). The function \( Q(x) \coloneqq \inprod x x \) (which is not a quadratic form in the complex case) is positive definite.
\end{lemma}
\begin{proof}
  The real case is trivial. Assume that \( V \) is a complex vector space and that \( \inprod \cdot \cdot \) is Hermitian. This implies that \( \inprod x x = \overline{\inprod x x} \), thus \( \inprod x x \in \BbbR \). Furthermore, since the inner product is positive definite, we have \( Q(x) = \inprod x x \geq 0 \). Thus, \( Q \) is nonnegative real valued.

  Since \( \inprod \cdot \cdot \) is positive definite, so is \( Q \).
\end{proof}

\begin{theorem}[Cauchy-Bunyakovsky-Schwarz inequality]\label{thm:cauchy_bunyakovsky_schwarz_inequality}
  Let \( V \) be a real or complex \hyperref[def:inner_product_space]{inner product space} with product \( \inprod \cdot \cdot \). For every \( x, y \in V \) it holds that
  \begin{equation}\label{thm:cauchy_bunyakovsky_schwarz_inequality/inequality}
    {\abs{\inprod x y}}^2 \leq \inprod x x \inprod y y.
  \end{equation}

  Furthermore, equality is achieved if and only if \( x \) and \( y \) are linearly dependent.
\end{theorem}
\begin{proof}
  Note that we use this theorem to prove that the induced norm is a norm, so we cannot use the norm here. Associate with \( \inprod \cdot \cdot \) the function \( Q(x) \coloneqq \inprod x x \). By \fullref{thm:inner_product_quadratic_form_is_positive_definite}, \( Q \) is positive definite.

  Fix \( x, y \in V \) and \( t \in \BbbC \). If either vector is zero the statement is trivially true, so let both be nonzero. We have
  \begin{balign*}
    Q(x + ty)
     & =
    \inprod {x + ty} {x + ty}
    =    \\ &=
    Q(x) + \overline t \inprod x y + t \inprod y x + \abs{t}^2 Q(y)
    =    \\ &=
    Q(x) + 2\real t \overline{\inprod x y} + \abs{t}^2 Q(y)
  \end{balign*}

  Take \( t \coloneqq - \frac {\inprod x y} {Q(y)} \), so that
  \begin{equation*}
    Q(x + ty)
    =
    Q(x) - 2 \frac {\abs{\inprod x y}^2} {Q(y)} + \frac {\abs{\inprod x y}^2} {Q(y)}
    =
    Q(x) - \frac {\abs{\inprod x y}^2} {Q(y)}
  \end{equation*}

  Since \( Q(x + ty) \geq 0 \), it follows that
  \begin{balign*}
    Q(x) - \frac {\abs{\inprod x y}^2} {Q(y)} & \geq 0                  \\
    Q(x) Q(y)                               & \geq \abs{\inprod x y}^2.
  \end{balign*}

  If \( x \) and \( y \) are linearly dependent, equality obviously holds. Conversely, suppose that equality holds. This implies that
  \begin{equation*}
    Q(x + ty) = 0,
  \end{equation*}
  which by the positive definiteness of \( Q \) means that \( x = -ty \). Thus, \( x \) and \( y \) are linearly dependent.
\end{proof}

\begin{definition}\label{def:bilinear_form_induced_norm}
  Let \( V \) be a real or complex \hyperref[def:inner_product_space]{inner product space} with product \( \inprod \cdot \cdot \). We define its induced \hyperref[def:norm]{norm} as
  \begin{balign*}
     & \norm \cdot : V \to \BbbR_{\geq 0}    \\
     & \norm x \coloneqq \sqrt{\inprod x x}.
  \end{balign*}

  If \( V \) is a real inner product space, the induced norm is a square root of the induced quadratic \hyperref[def:quadratic_form]{form} of \( \inprod \cdot \cdot \).
\end{definition}
\begin{proof}
  We will only prove the complex case because the real case is identical, but slightly simpler.

  Note that \( \norm \cdot \) is well-defined (that is, positive definite) by \fullref{thm:inner_product_quadratic_form_is_positive_definite}.

  Now we will show that it is a norm.
  \SubProofOf{def:norm/N1} Follows from the positive definiteness of \( \inprod \cdot \cdot \)

  \SubProofOf{def:norm/N2} For \( t \in \BbbC \) and \( x \in V \) we have
  \begin{equation*}
    \norm{tx} = \sqrt{\inprod{tx} {tx}} = \abs{t} \sqrt{\inprod x x} = \abs t \norm x.
  \end{equation*}

  \SubProofOf{def:norm/N3} For \( x, y \in V \) we have
  \begin{balign*}
    \norm{x + y}^2
     & =
    \inprod{x + y} {x + y}
    =                                                            \\ &=
    \inprod x x + \inprod x y + \inprod y x + \inprod y y
    =                                                            \\ &=
    \norm{x}^2 + 2 \real \inprod x y + \norm{y}^2
    \leq                                                         \\ &\leq
    \norm{x}^2 + 2 \abs{\real \inprod x y} + \norm{y}^2
    \reloset {\ref{thm:cauchy_bunyakovsky_schwarz_inequality}} = \\ &=
    \norm{x}^2 + 2 \norm x \norm y + \norm{y}^2
    =
    (\norm{x} + \norm{y})^2
  \end{balign*}

  Therefore,
  \begin{equation*}
    \norm{x + y} \leq \norm x + \norm y.
  \end{equation*}
\end{proof}


% Logic
\begin{definition}\label{def:logic_syntax}
  In logic, we are interested in \Def{terms} (see \fullref{def:first_order_language/term}) and \Def{formulas} (see \fullref{def:propositional_language/formula} or \fullref{def:first_order_language/formula}), which are specific strings of symbols, and their \Def{valuation} (see \fullref{def:propositional_valuation} or \fullref{def:first_order_valuation}), which maps terms and formulas into certain \hyperref[def:concrete_category]{structured set}.
\end{definition}

\begin{definition}\label{rem:syntax_and_semantics}


  A set \( \CF \) of \Def{logical formulas}. 


  \begin{RemEnum}
    \ILabel{def:propositional_model/satisfiability} If \( \varphi \) is true in \( \CA \) under every variable assignment, we say that \( \varphi \) is \Def{valid} in \( \CA \) and that \( \CA \) is a \Def{model} of \( \varphi \) and write \( \CA \models \varphi \). We extend this to sets of formulas \( \Gamma \) via conjunction.

    \ILabel{def:propositional_model/entailment} If every interpretation that satisfies all formulas in the set \( \Gamma \) also satisfies the formula \( \varphi \), we say that \( \Gamma \) \Def{entails} \( \varphi \) and write \( \Gamma \models \varphi \).

    \ILabel{def:propositional_model/tautology} If all interpretations satisfy \( \varphi \), we call \( \varphi \) a \Def{tautology}.

    \ILabel{def:propositional_model/contradiction} Dually, if no interpretations satisfy \( \varphi \), \( \varphi \) is a \Def{contradiction}.

    \ILabel{def:propositional_model/equivalence} If \( \varphi\Val{I} = \psi\Val{I} \) for every interpretation \( I \), we say that \( \varphi \) and \( \psi \) are \Def{semantically equivalent} and write \( \varphi \cong \psi \).
  \end{RemEnum}
\end{definition}

\subsection{Languages}\label{subsec:languages}

\begin{remark}\label{rem:language_definitions_using_sets}
  Languages are used to define formulas for expressing the \hyperref[def:set]{axioms of set theory}. Here, sets are used to formally define languages. A simple way out of this vicious cycle is the logic-metalogic relationship described at the beginning of \fullref{sec:mathematical_logic} --- we define languages within the metalogic using the implicitly available concept of set and we later define formulas, again in the metalogic, which allows us to subsequently formally define sets via axioms.
\end{remark}

\begin{definition}\label{def:language}
  Fix a nonempty set \( \mscrA \).

  \begin{thmenum}
    \thmitem{def:language/alphabet} We call \( \mscrA \) an \term{alphabet}.

    \thmitem{def:language/symbol} We call each element of \( \mscrA \) a \term{symbol}.

    \thmitem{def:language/word} A \term{word} over \( \mscrA \) is a \hyperref[def:cartesian_product]{tuple} of symbols. If \( (a, b, c) \) is a word, for convenience we write it as the string \( abc \). This is the reason words are also referred to as \term{strings}. This notation only makes sense if each symbol of the language is actually represented by one typographic symbol.

    \thmitem{def:language/empty_word} We denote the empty word by \( \varepsilon \).

    \thmitem{def:language/word_length} The \term{length} \( \len(w) \) of a word \( w \) is the number of elements of the tuple \( w \).

    \thmitem{def:language/concatenation} The \term{concatenation} of the words \( v = (v_1, \ldots, v_n) \) and \( w = (w_1, \ldots, w_m) \) is the word
    \begin{equation*}
      vw \coloneqq (v_1, \ldots, v_n, w_1, \ldots, w_m).
    \end{equation*}

    We abbreviate \( \underbrace{w w \ldots w}_{k \T{times}} \) as \( w^k \). This is only a notation. We do not distinguish, formally, between the words \( aaabbaa \) and \( a^3 b^2 a^2 \), nor between \( a \varepsilon b \) and \( ab \).

    \thmitem{def:language/reverse} The \term{reverse word} of \( w = (w_1, \ldots, w_n) \) is
    \begin{equation*}
      \op{rev}(w) \coloneqq (w_n, \ldots, w_1).
    \end{equation*}

    \thmitem{def:language/prefix} The word \( p = (p_1, \ldots, p_m) \) is a \term{prefix} of \( w = (w_1, \ldots, w_n) \) if
    \begin{equation*}
      w = (\underbrace{p_1, \ldots, p_m}_p, w_{m+1}, \ldots, w_n).
    \end{equation*}

    \thmitem{def:language/suffix} The word \( s \) is a \term{suffix} of \( w \) if \( \op{rev}(s) \) is a prefix of \( \op{rev}(w) \).

    \thmitem{def:language/subword} The word \( v \) is a \term{subword} of \( w \) if there exists a prefix \( p \) and a suffix \( s \) of \( v \) such that
    \begin{equation*}
      w = pvs.
    \end{equation*}

    \thmitem{def:language/kleene_star} The \term{Kleene star} \( \mscrA^{\ast} \) of \( \mscrA \) is the set of all (finite) words over \( \mscrA \).

    \thmitem{def:language/language} A \term{language} over \( \mscrA \) is any subset of \( \mscrA^{\ast} \). Note that, in some contexts like \hyperref[subsec:propositional_logic]{propositional logic} or \hyperref[subsec:first_order_logic]{first-order logic}, the term \enquote{language} may refer to the alphabet itself (see \fullref{rem:propositional_language_is_alphabet}).
  \end{thmenum}
\end{definition}

\begin{proposition}\label{thm:kleene_star_is_monoid}
  For any alphabet \( \mscrA \), the Kleene star \( \mscrA^{\ast} \) is a \hyperref[def:unital_magma/associative]{monoid} under concatenation.
\end{proposition}
\begin{proof}
  Concatenation is clearly associative and the empty word \( \varepsilon \) is a \hyperref[def:magma_identity]{two-sided identity} under concatenation.
\end{proof}

\subsection{Grammars}\label{subsec:grammars}

\begin{definition}\label{def:grammar}\mcite\cite[def. 2.2]{Sipser2013}
  Let \( \mscrA \) be some \hyperref[def:language/alphabet]{alphabet} and \( V, \Sigma \subseteq \mscrA \) be nonempty disjoint subsets of \( \mscrA \).

  \begin{defenum}
    \ilabel{def:grammar/variables} We call elements of \( V \) \term{variables} or \term{non-terminals}.

    \ilabel{def:grammar/terminals} We call elements of elements of \( \Sigma \) \term{terminals}.

    \ilabel{def:grammar/start} We assume that a special \term{start symbol} \( S \in V \) is fixed.

    \ilabel{def:grammar/production_rules} We define a binary \hyperref[def:relation]{relation} \( \to \) of \term{production rules} over \( (V \cup \Sigma)^* \), that is, rules are \enquote{transformations} that define how a language is \enquote{generated} starting from \( S \in V \) (see \fullref{def:grammar_derivation} and \fullref{ex:natural_arithmetic_grammar/derivation}). By convention, we treat uppercase symbols as variables and lowercase symbols as terminals. See, for example, \fullref{ex:natural_arithmetic_grammar/backus_naur_form}. When speaking about general grammars, however, we usually use the letters \( u \), \( v \) and \( w \) to denote words (that may contain variables) rather than terminals.

    \ilabel{def:grammar/terminal_rules} Rules of the form \( u \to \sigma \), where \( \sigma \in \Sigma \), are called \term{terminal rules}. Note that \( u \) here is a word and not a terminal.

    \ilabel{def:grammar/grammar} The tuple \( G \coloneqq (V, \Sigma, \to, S) \) is called a \term{formal grammar}.

    \ilabel{def:grammar/context_free} If every production rule has only a single variable for a source, i.e. if for every rule \( u \to v \) we have \( u = A \) for some \( A \in V \), we say that the grammar is \term{context-free}.
  \end{defenum}
\end{definition}

\begin{example}\label{ex:natural_arithmetic_grammar/backus_naur_form}
  We will define a grammar for addition and multiplication of \hyperref[def:natural_numbers]{natural numbers}. Note that we only consider the number in \( \BbbN \) only as symbols, not as the numbers themselves.

  Let \( V \coloneqq \{ E \} \) and \( \Sigma \coloneqq \BbbN \cup \{ +, (, ) \} \). Define the grammar
  \begin{alignedeq}\label{eq:ex:natural_arithmetic_grammar/backus_naur_form/simple}
    &E \to 0 \\
    &E \to 1 \\
    &\phantom{E \to} \vdots \\
    &E \to n \\
    &\phantom{E \to} \vdots \\
    &E \to (E + E) \\
    &E \to (E \cdot E)
  \end{alignedeq}

  We can use the following shorthand:
  \begin{equation}\label{eq:ex:natural_arithmetic_grammar/backus_naur_form/shorthand}
    E \to 0 \mid 1 \mid \ldots \mid (E + E) \mid (E \cdot E).
  \end{equation}

  The infinitude of possible rules may not bother us formally, but when dealing with software implementations (e.g. the Python grammar that can be found in \cite{Python:39_grammar}), we must have a finite numbers of rules.

  There are also other advantages of introducing a more convenient metasyntax, i.e. a syntax for describing language syntax.

  For \hyperref[def:grammar/context_free]{context-free grammars}, is often convenient to use the \term{Backus-Naur form (BNF)}. In our example, this becomes
  \begin{bnf*}
    \bnfprod{nonzero digit} {\bnfts{1} \bnfor \bnfts{2} \bnfor \bnfts{3} \bnfor \bnfts{4} \bnfor \bnfts{5} \bnfor \bnfts{6} \bnfor \bnfts{7} \bnfor \bnfts{8} \bnfor \bnfts{9}} \\
    \bnfprod{digit}         {\bnfts{0} \bnfor \bnfpn{nonzero digit}} \\
    \bnfprod{number}        {\bnfpn{nonzero digit} \bnfor \bnfpn{number} \bnfsp \bnfpn{digit}} \\
    \bnfprod{operation}     {\bnfts{+} \bnfor \bnfts{\( \cdot \)}} \\
    \bnfprod{expression}    {\bnfpn{number} \bnfor \bnfts{(} \bnfsp \bnfpn{number} \bnfsp \bnfpn{operation} \bnfsp \bnfpn{number} \bnfsp \bnfts{)}}.
  \end{bnf*}
  with \( \bnfpn{expression} \) as the starting variable.

  The obvious difference is that we explicitly define numbers via their decimal representation, which only requires a finite amount of rules. Compared to \eqref{eq:ex:natural_arithmetic_grammar/backus_naur_form/simple}, some other differences are:
  \begin{enumerate}
    \item Variables are denoted by \( \langle \)words enclosed in angle brackets\( \rangle \) so that we can name variables using more than one symbol.
    \item Terminals are denoted using \enquote{quotes}. In human-readable rich text documents like this one, it is sometimes possible to use different fonts and so instead of \enquote{quotes} we specify terminals using an \texttt{upright monospaced font}.
    \item Free-text rules can be specified using a normal font. This is also only used in human-readable rich text documents, however this usage is justified because such rules are only beneficial for human understanding and not for machine parsing.
    \item The symbol \( :\coloneqq \) is used instead of \( \to \) for specifying transition rules.
    \item Different rules with the same source are concatenated as in \eqref{eq:ex:natural_arithmetic_grammar/backus_naur_form/shorthand}.
    \item In order to fully describe a context-free grammar, we must only specify its Backus-Naur form and its starting variable.
  \end{enumerate}
\end{example}

\begin{definition}\label{def:backus_naur_form}
  We defined the \term{Backus-Naur form} of a \hyperref[def:grammar/context_free]{context-free grammar} in \fullref{ex:natural_arithmetic_grammar/backus_naur_form}.

  Although formally necessary for \fullref{def:grammar}, it is of slight inconvenience to explicitly specify the starting variable for a nontrivial grammar in Backus-Naur form because the same Backus-Naur form can be used with different starting variables.

  For this reason, we will say that the Backus-Naur form specifies \term{grammar schemas} and not grammars. Given a grammar schema, we can select any of its variables to obtain a grammar.
\end{definition}

\begin{definition}\label{def:grammar_derivation}\mcite\cite[page 104 \\ page 108]{Sipser2013}
  Fix a \hyperref[def:grammar]{formal grammar} \( G = (V, \Sigma, \to, S) \). Note that all lowercase symbols in this definitions are words rather than terminals.

  \begin{defenum}
    \ilabel{def:grammar_derivation/yields} Fix a \hyperref[def:language/word]{word} \( pvs \). If \( u \to v \) is a production rule, we say that \( pvs \) \term{yields} the word \( pws \) and write
    \begin{equation*}
      pvs \Rightarrow pws.
    \end{equation*}

    \ilabel{def:grammar_derivation/derivation} We say that \( u \) \term{derives} \( v \) and write \( u \Rightarrow v \) if there exists a finite sequence of words \( u_1, \ldots, u_n \) such that
    \begin{equation*}
      u \Rightarrow u_1 \Rightarrow \ldots \Rightarrow u_n \Rightarrow v.
    \end{equation*}

    The sequence \( u, u_1, \ldots, u_n, v \) is called a \term{derivation} of \( v \) from \( u \).

    \ilabel{def:grammar_derivation/leftmost_rightmost_derivation} If on every step of the derivation the leftmost (resp. rightmost) variable is replaced, we say that it is a \term{leftmost} (resp. \term{rightmost}) derivation.

    \ilabel{def:grammar_derivation/grammar_language} Define the \term{language} of the grammar to be
    \begin{equation*}
      \mscrL(G) \coloneqq \{ w \in \Sigma^* \colon S \Rightarrow w \},
    \end{equation*}
    that is, all words that can be derived from \( S \) and contains only terminals.

    We also say that strings in \( \mscrL(G) \) are \term{generated} by the grammar \( G \).

    If a language can be generated by a \hyperref[def:grammar/context_free]{context-free grammar}, we say that it is a \term{context-free language}.

    \ilabel{def:grammar_derivation/ambiguity}\mcite\cite[def. 2.7]{Sipser2013}We say that the word \( w \) can be derived \term{unambiguously} if there exists a unique leftmost derivation from \( S \). Otherwise we say that \( w \) is generated \term{ambiguously} and that the grammar itself is \term{ambiguous}.
  \end{defenum}
\end{definition}

\begin{example}\label{ex:natural_arithmetic_grammar/derivation}
  We continue \fullref{ex:natural_arithmetic_grammar/backus_naur_form}. Depending on our choice of starting symbol, we can derive different sets of words. We do not consider the language as a big single set but instead as a family of smaller sets whose union is \( \mscrL(G) \). In this example, this family consists of sets of words representing nonzero digits, arbitrary digits, numbers, operations and arithmetic expressions.

  For the sake of simplifying our exposition and proof, however, we will assume the simpler grammar described in \eqref{eq:ex:natural_arithmetic_grammar/backus_naur_form/simple}.

  Choose the starting symbol to be \( E \). Then the grammar can produce the arithmetic expression \( ((1 + 2) + 3) \) by applying the
  \begin{equation*}
    \begin{mplibcode}
      u := 2cm;

      beginfig(1);
      input metapost/graphs;

      v1 := thelabel("$E$", origin);
      v2 := thelabel("$(E + E)$", (0, -1) scaled u);
      v3 := thelabel("$3$", (1, -2) scaled u);
      v4 := thelabel("$(E + E)$", (-1, -2) scaled u);
      v5 := thelabel("$1$", (-2, -3) scaled u);
      v6 := thelabel("$2$", (0, -3) scaled u);

      a1 := straight_arc(v1, v2);
      a2 := straight_arc(v2, v3);
      a3 := straight_arc(v2, v4);
      a4 := straight_arc(v4, v5);
      a5 := straight_arc(v4, v6);

      draw_vertices(v);
      draw_arcs(a);

      label.lft("$E \to (E + E$)", straight_arc_midpoint of a1);
      label.urt("$E \to 3$", straight_arc_midpoint of a2);
      label.ulft("$E \to (E + E)$", straight_arc_midpoint of a3);
      label.ulft("$E \to 1$", straight_arc_midpoint of a4);
      label.urt("$E \to 2$", straight_arc_midpoint of a5);
      endfig;
    \end{mplibcode}
  \end{equation*}

  Note that the grammar is unambiguous because of the parentheses. If we omit the parentheses, it will no longer be unambiguous since \( 1 + 2 + 3 \) can be derived by both
  \begin{equation*}
    \begin{mplibcode}
      u := 2cm;

      beginfig(1);
      input metapost/graphs;

      v1 := thelabel("$E$", origin);
      v2 := thelabel("$E + E$", (0, -1) scaled u);
      v3 := thelabel("$1$", (-1, -2) scaled u);
      v4 := thelabel("$E + E$", (1, -2) scaled u);
      v5 := thelabel("$2$", (0, -3) scaled u);
      v6 := thelabel("$3$", (2, -3) scaled u);

      a1 := straight_arc(v1, v2);
      a2 := straight_arc(v2, v3);
      a3 := straight_arc(v2, v4);
      a4 := straight_arc(v4, v5);
      a5 := straight_arc(v4, v6);

      draw_vertices(v);
      draw_arcs(a);

      label.lft("$E \to E + E$", straight_arc_midpoint of a1);
      label.ulft("$E \to 1$", straight_arc_midpoint of a2);
      label.urt("$E \to E + E$", straight_arc_midpoint of a3);
      label.ulft("$E \to 2$", straight_arc_midpoint of a4);
      label.urt("$E \to 3$", straight_arc_midpoint of a5);
      endfig;
    \end{mplibcode}
    \hspace{1cm}
    \begin{mplibcode}
      u := 2cm;

      beginfig(1);
      input metapost/graphs;

      v1 := thelabel("$E$", origin);
      v2 := thelabel("$E + E$", (0, -1) scaled u);
      v3 := thelabel("$3$", (1, -2) scaled u);
      v4 := thelabel("$E + E$", (-1, -2) scaled u);
      v5 := thelabel("$1$", (-2, -3) scaled u);
      v6 := thelabel("$2$", (0, -3) scaled u);

      a1 := straight_arc(v1, v2);
      a2 := straight_arc(v2, v3);
      a3 := straight_arc(v2, v4);
      a4 := straight_arc(v4, v5);
      a5 := straight_arc(v4, v6);

      draw_vertices(v);
      draw_arcs(a);

      label.lft("$E \to E + E$", straight_arc_midpoint of a1);
      label.urt("$E \to 3$", straight_arc_midpoint of a2);
      label.ulft("$E \to E + E$", straight_arc_midpoint of a3);
      label.ulft("$E \to 1$", straight_arc_midpoint of a4);
      label.urt("$E \to 2$", straight_arc_midpoint of a5);
      endfig;
    \end{mplibcode}
  \end{equation*}
\end{example}
\begin{proof}
  We will show that \( G \) is unambiguous. Let \( w \) be a word in \( \mscrL(G) \). We explicitly build the derivation of \( w \) by induction\IND on \( \len(w) \):
  \begin{itemize}
    \item If \( \len(w) = 1 \), then \( w = n \in \BbbN \) and the word has been generated by the single rule \( A \to n \).

    \item Assume that \( w \) is unambiguously derived for \( \len(w) < m + 2 \) and let \( \len(w) = m + 2 \), then \( w \) is necessarily enclosed in parentheses. Let \( w = ( \sigma_1 \ldots \sigma_m ) \) be the symbols of \( w \). Because of the parentheses, the only possibility for \( \sigma_1 \ldots \sigma_m \) is that it consists of two words in \( \mscrL(G) \) with either an addition symbol \( + \) or a multiplication symbol \( \cdot \) between them. Let \( k \) be the index of the operator, that is, the index such that \( \sigma_1 \ldots \sigma_{k-1} \) and \( \sigma_{k+1} \ldots \sigma_m \) both belong to \( \mscrL(G) \). Furthermore, by inductive hypothesis, both \( \sigma_1 \ldots \sigma_{k-1} \) and \( \sigma_{k+1} \ldots \sigma_m \) are unambiguously derived. Therefore \( w \) is also unambiguously derived.
  \end{itemize}
\end{proof}

\subsection{Boolean functions}\label{subsec:boolean_functions}

\begin{definition}\label{def:boolean_value}
  Fix a two-element set \( \set{ T, F } \). We can think of \( T \) as a value denoting truth and \( F \) as denoting falsity. See \fullref{rem:mathematical_logic_conventions/propositional_constants} for notation conventions.

  There is a natural \hyperref[def:boolean_algebra/trivial]{trivial Boolean algebra} structure on \( \set{ T, F } \) where \( T \) is the \hyperref[def:semilattice/join]{top} and \( F \) is the \hyperref[def:semilattice/meet]{bottom} and the operations are defined in an obvious way. Without further context, we can assume that \( \set{ T, F } \) the \hyperref[thm:galois_field_existence]{Galois field} \( \BbbF_2 \) where we identify \( F \) with \( 0 \) and \( T \) with \( 1 \).
\end{definition}

\begin{definition}\label{def:boolean_function}
  We call functions from any set to \( \set{ T, F } \) \term{Boolean-valued} and functions from \( \set{ T, F }^n \) to \( \set{ T, F } \) \term{Boolean}.
\end{definition}

\begin{remark}\label{rem:boolean_valued_functions_and_predicates}
  \hyperref[def:boolean_function]{Boolean-valued functions} and \hyperref[def:relation]{relations} represent the same concept. In particular, the relation \( R \subseteq A_1 \times \cdots \times A_n \) corresponds to a unique Boolean-valued function
  \begin{equation*}
    \begin{aligned}
      &f: X_1 \times \cdots \times X_n \to \set{ T, F } \\
      &f(x_1, \ldots, x_n) = \begin{cases}
        T, &(x_1, \ldots, x_n) \in R, \\
        F, &\T{otherwise}
      \end{cases}
    \end{aligned}
  \end{equation*}
  and vice versa.
\end{remark}

\begin{definition}\label{def:boolean_closure}
  Fix a set \( B \) of Boolean functions of arbitrary arities.

  The \term{closure} \( \cl{B} \) of \( B \) is defined inductively as follows:
  \begin{itemize}
    \item If \( f \in B \), then \( f \in \cl{B} \)
    \item If \( f_k(x_1, \ldots, x_n) \in \cl{B} \) for \( k \in 1, \ldots, m \) and if \( g(x_1, \ldots, x_m) \in \cl{B} \), then their \hyperref[def:multi_valued_function/superposition]{superposition}
    \begin{equation*}
      h(x_1, \ldots, x_n) \coloneqq g(f_1(x_1, \ldots, x_n), \ldots, f_m(x_1, \ldots, x_n))
    \end{equation*}
    is also in \( \cl{B} \).
  \end{itemize}

  We say that \( B \) is \term{closed} if \( \cl{B} = B \) and \term{complete} if \( \cl{B} \) is the set of all Boolean functions of arbitrary arity.

  If \( B \) is complete, then from \fullref{thm:functions_over_model_form_model} it follows that \( B \) is a Boolean algebra. This is used in \fullref{thm:lindenmaum_tarski_algebra_of_full_propositional_logic/bijection}.
\end{definition}

\begin{definition}\label{def:zhegalkin_polynomial}
  A \term{Zhegalkin polynomial} is a \hyperref[def:polynomial]{polynomial} in the \hyperref[thm:galois_field_existence]{Galois field} \( \BbbF_2 \).

  Due to \fullref{thm:polynomial_embedding_behavior}, we may restrict ourselves to polynomials with no powers higher than \( 1 \). For example, for binary operations, we may restrict ourselves to the Zhegalkin polynomials
  \begin{equation}\label{eq:def:zhegalkin_polynomial/binary_polynomial}
    f(x, y) = axy \oplus bx \oplus cy \oplus d,
  \end{equation}
  where \( a, b, c, d \in \BbbF_2 \), since these are exactly the polynomials that correspond to unique binary Boolean functions.

  Using this conventions, we can avoid distinguishing between polynomials and polynomial functions (see \fullref{rem:polynomials_vs_polynomial_functions}).
\end{definition}

\begin{definition}\label{def:standard_boolean_operators}
  Unlike \hyperref[def:function]{arbitrary functions}, \hyperref[def:boolean_function]{Boolean functions} only have a small finite number of possible values that can easily be enumerated.

  Out of the following binary operations, \( \vee \), \( \wedge \) and \( \overline{\placeholder} \) form the \hyperref[thm:f2_is_boolean_algebra]{Boolean algebra structure} on \( \BbbF_2 \) and \( \oplus \) and \( \wedge \) form the \hyperref[def:field]{field structure} on \( \BbbF_2 \). The operations \( \rightarrow \) and \( \leftrightarrow \) are also defined in any \hyperref[def:boolean_algebra]{Boolean algebra}.

  \begin{center}
    \begin{tabular}{c | c || c c | c c c c c c}
      \( x \) & \( \overline{x} \) & \( x \) & \( y \) & \( x \vee y \) & \( x \oplus y \)    & \( x \wedge y \) & \( x \rightarrow y \)   & \( x \leftrightarrow y \) \\
      \hline
              & not \( x \)        &         &         & \( x \) or \( y \)  & \( x \) xor \( y \) & \( x \) and \( y \)     & \( x \) implies \( y \) & \( x \) iff \( y \)       \\
      \hline
      \( F \) & \( T \)            & \( F \) & \( F \) & \( F \)             & \( F \)             & \( F \)                 & \( T \)                 & \( T \)                   \\
      \( T \) & \( F \)            & \( F \) & \( T \) & \( F \)             & \( T \)             & \( F \)                 & \( T \)                 & \( F \)                   \\
              &                    & \( T \) & \( F \) & \( F \)             & \( T \)             & \( F \)                 & \( F \)                 & \( F \)                   \\
              &                    & \( T \) & \( T \) & \( T \)             & \( F \)             & \( T \)                 & \( T \)                 & \( T \)                   \\
      \hline
              & \( x \oplus 1 \) &         &         & \( xy \oplus x \oplus y \) & \( x \oplus y \)    & \( xy \)            & \( xy \oplus x \oplus 1 \) & \( x \oplus y \oplus 1 \)
    \end{tabular}
  \end{center}

  See \fullref{thm:boolean_equivalences} for direct consequences of these definitions.
\end{definition}

\begin{definition}\label{def:boolean_functions_in_f2}
  Fix a \hyperref[def:boolean_function]{Boolean function} \( f(x_1, \ldots, x_n) \) in the \hyperref[thm:galois_field_existence]{Galois field} \( \BbbF_2 \),

  \begin{thmenum}
    \thmitem{def:boolean_function_in_f2/dual} Its \term{dual function} is
    \begin{equation*}
      \overline{f}(x_1, \ldots, x_n) \coloneqq \overline{f(\overline{x_1}, \ldots, \overline{x_n})}.
    \end{equation*}

    \thmitem{def:boolean_functions_in_f2/self_dual} \( f \) is \term{self-dual} if it is its own \hyperref[def:boolean_function_in_f2/dual]{dual}.

    \thmitem{def:boolean_functions_in_f2/truth_preserving} \( f \) is \term{truth-preserving} if \( f(T, \ldots, T) = T \).

    \thmitem{def:boolean_functions_in_f2/falsity_preserving} \( f \) is \term{falsity-preserving} if \( f(F, \ldots, F) = F \).

    \thmitem{def:boolean_functions_in_f2/monotone} \( f \) is \term{monotone} if, for any two tuples of arguments \( x_1, \ldots, x_n \in \BbbF_2 \) and \( y_1, \ldots, y_n \in \BbbF_2 \), the inequalities \( x_k \leq y_k \) for all \( k \in \set{ 1, \ldots, n } \) imply that
    \begin{equation*}
      f(x_1, \ldots, x_n) \leq f(y_1, \ldots, y_n).
    \end{equation*}

    \thmitem{def:boolean_functions_in_f2/linear} \( f \) is \term{linear} if its \hyperref[def:zhegalkin_polynomial]{Zhegalkin polynomial} is linear, i.e. has only monomials of degree \( 0 \) or \( 1 \). In the case of binary Boolean functions, this means that the coefficient \( a \) in \eqref{eq:def:zhegalkin_polynomial/binary_polynomial} is zero.
  \end{thmenum}
\end{definition}

\begin{theorem}[Post's completeness theorem]\label{thm:posts_completeness_theorem}\mcite{Martin1990}
  The family \( B \) of Boolean functions is \hyperref[def:boolean_closure]{complete} if and only if all of the following conditions are satisfied:
  \begin{thmenum}
    \thmitem{thm:posts_completeness_theorem/truth_preserving} \( B \) contains a function that is not \hyperref[def:boolean_functions_in_f2/truth_preserving]{truth-preserving}.
    \thmitem{thm:posts_completeness_theorem/falsity_preserving} \( B \) contains a function that is not \hyperref[def:boolean_functions_in_f2/falsity_preserving]{falsity-preserving}.
    \thmitem{thm:posts_completeness_theorem/self_dual} \( B \) contains a function that is not \hyperref[def:boolean_functions_in_f2/self_dual]{self-dual}.
    \thmitem{thm:posts_completeness_theorem/monotone} \( B \) contains a function that is not \hyperref[def:boolean_functions_in_f2/monotone]{monotone}.
    \thmitem{thm:posts_completeness_theorem/linear} \( B \) contains a function that is not \hyperref[def:boolean_functions_in_f2/linear]{linear}.
  \end{thmenum}
\end{theorem}

\begin{example}\label{ex:posts_completeness_theorem}
  We give examples of complete sets of Boolean functions in \( \BbbF_2 \).

  \begin{thmenum}
    \thmitem{ex:posts_completeness_theorem/and_or} The archetypic example of a complete set of Boolean functions is the triple \( \vee, \wedge, \overline{\placeholder} \) that forms the Boolean algebra structure on \( \BbbF_2 \).

    We verify that the conditions of \fullref{thm:posts_completeness_theorem} are satisfied:
    \begin{refenum}
      \refitem{thm:posts_completeness_theorem/truth_preserving} \( \overline{\placeholder} \) is not truth-preserving.
      \refitem{thm:posts_completeness_theorem/falsity_preserving} \( \overline{\placeholder} \) is not falsity-preserving.
      \refitem{thm:posts_completeness_theorem/self_dual} Neither \( \vee \) nor \( \wedge \) are self-dual. In fact, due to \fullref{thm:de_morgans_laws}, \( \wedge \) is the dual of \( \vee \) and vice versa.
      \refitem{thm:posts_completeness_theorem/monotone} \( \overline{\placeholder} \) is not monotone.
      \refitem{thm:posts_completeness_theorem/linear} Neither \( \vee \) nor \( \wedge \) have linear Zhegalkin polynomials.
    \end{refenum}

    Thus \( \set{ \wedge, \vee, \overline{\placeholder} } \) is indeed a complete set of Boolean functions. Note that having both \( \vee \) and \( \wedge \) is redundant and we usually include both for symmetry. The families \( \set{ \wedge, \overline{\placeholder} } \) and \( \set{ \vee, \overline{\placeholder} } \) are both complete.

    This is used for \hyperref[alg:conjunctive_normal_form_reduction]{conjunctive normal forms}.

    \thmitem{ex:posts_completeness_theorem/nand} We can go even further and have a single binary Boolean function generate all others. We will use the function
    \begin{equation}\label{eq:ex:posts_completeness_theorem/nand}
      (x \uparrow y) \coloneqq \overline{x \wedge y} = xy \oplus 1.
    \end{equation}

    This operation is called \term{Sheffer's stroke} or \term{nand} (\enquote{not and}).

    We have
    \begin{equation*}
      \begin{array}{ccc}
        \overline{x} = (x \uparrow 1)
        &
        \T{and}
        &
        (x \wedge y) = \overline{x \uparrow y},
      \end{array}
    \end{equation*}
    which allows us to reduce the case to \fullref{ex:posts_completeness_theorem/and_or}. We conclude that the singleton set \( \set{ \uparrow } \) is a complete set of Boolean operations.

    \thmitem{ex:posts_completeness_theorem/conditional_negation} Another commonly used complete family is \( \set{ \rightarrow, \overline{\placeholder} } \).
    We verify that the conditions of \fullref{thm:posts_completeness_theorem} are satisfied:
    \begin{refenum}
      \refitem{thm:posts_completeness_theorem/truth_preserving} \( \overline{\placeholder} \) is not truth-preserving.
      \refitem{thm:posts_completeness_theorem/falsity_preserving} \( \rightarrow \) is not falsity-preserving because \( (F \rightarrow F) = T \).
      \refitem{thm:posts_completeness_theorem/self_dual} \( \rightarrow \) is not self-dual because \( \overline{\overline{x} \rightarrow \overline{y}} \reloset {\eqref{eq:thm:boolean_equivalences/contrapositive}} = (y \rightarrow x) \neq (x \rightarrow y) \).
      \refitem{thm:posts_completeness_theorem/monotone} \( \rightarrow \) is not monotone because \( F \rightarrow T = F \).
      \refitem{thm:posts_completeness_theorem/linear} \( \rightarrow \) doesn't have a linear Zhegalkin polynomial.
    \end{refenum}

    \thmitem{ex:posts_completeness_theorem/conditional_bottom} Given the family \( \set{ \rightarrow, F } \), we can define
    \begin{equation*}
      \overline{x} \coloneqq (x \rightarrow F),
    \end{equation*}
    which shows that \( \set{ \rightarrow, F } \) is also a complete family.
  \end{thmenum}
\end{example}

\subsection{Propositional logic}\label{subsec:propositional_logic}

\begin{remark}\label{rem:propositional_language_is_alphabet}
  The \hyperref[def:propositional_language]{language of propositional logic} is, strictly speaking, an \hyperref[def:formal_language/alphabet]{alphabet} rather than a \hyperref[def:formal_language/language]{language}. Nonetheless, this is the established terminology.
\end{remark}

\begin{definition}\label{def:propositional_language}\mcite[sec. 7.2]{OpenLogicFull}
  The \term{language of propositional logic} consists of:

  \begin{thmenum}
    \thmitem{def:propositional_language/prop} A nonempty, \hyperref[def:set_countability/at_most_countable]{at most countable} set \( \boldop{Prop} \) of \term{propositional variables}. Technically, we can have different languages with different variables, but it is safe to assume that there is only one single language of propositional language.

    \thmitem{def:propositional_language/constants} Two \term{propositional constants} (also known as \term{truth values}):
    \begin{thmenum}
      \thmitem{def:propositional_language/constants/verum} The \term{verum} \( \top \).
      \thmitem{def:propositional_language/constants/falsum} The \term{falsum} \( \bot \).
    \end{thmenum}

    \thmitem{def:propositional_language/negation} \term{Negation} \( \neg \).
    \thmitem{def:propositional_language/connectives} The set \( \Sigma \) of \term{propositional connectives}, namely
    \begin{thmenum}
      \thmitem{def:propositional_language/connectives/conjunction} \term{Conjunction} \( \wedge \) (also known as \hyperref[def:standard_boolean_operators]{\term{and}} and \hyperref[def:semilattice/meet]{\term{meet}}).
      \thmitem{def:propositional_language/connectives/disjunction} \term{Disjunction} \( \vee \) (also known as \hyperref[def:standard_boolean_operators]{\term{or}} and \hyperref[def:semilattice/join]{\term{join}}).
      \thmitem{def:propositional_language/connectives/conditional} \term{Conditional} \( \rightarrow \) (also known as \term{if\ldots then} and \hyperref[def:material_implication]{\term{material implication}}).
      \thmitem{def:propositional_language/connectives/biconditional} \term{Biconditional} \( \leftrightarrow \) (also known as \term{iff} and \term{material equivalence}).
    \end{thmenum}

     Note that \enquote{conditional} and \enquote{biconditional} are nouns in this context.

    \thmitem{def:propositional_language/parentheses} Parentheses \( ( \) and \( ) \) for defining the order of operations unambiguously (see \fullref{rem:propositional_formula_parentheses} for a further discussion).
  \end{thmenum}

  \Fullref{rem:smaller_propositional_language} shows we can actually utilize a smaller propositional language without losing any of its semantics.
\end{definition}

\begin{definition}\label{def:propositional_syntax}
  The following related definitions constitute what is called the \term{syntax of propositional logic}.

  \begin{thmenum}
    \thmitem{def:propositional_syntax/grammar_schema} Consider the following \hyperref[ex:natural_number_arithmetic_grammar/backus_naur_form]{grammar schema}:
    \begin{bnf*}
      \bnfprod{variable}   {P \in \boldop{Prop}} \\
      \bnfprod{connective} {\circ \in \Sigma} \\
      \bnfprod{formula}    {\bnfpn{variable} \bnfor} \\
      \bnfmore             {\bnfts{\( \top \)} \bnfor \bnfts{\( \bot \)} \bnfor} \\
      \bnfmore             {\bnfts{\( \neg \)} \bnfpn{formula} \bnfor} \\
      \bnfmore             {\bnfts{(} \bnfsp \bnfpn{formula} \bnfsp \bnfpn{connective} \bnfsp \bnfpn{formula} \bnfsp \bnfts{)}}
    \end{bnf*}

    Note that \( \boldop{Prop} \) may be infinite, in which case the grammars may have infinitely many rules. If needed, we can circumvent this by introducing an appropriate naming convention for variables, for example by allowing arbitrary strings of alphanumeric characters for variable names.

    For the sake of readability, we will be using the conventions in \fullref{rem:propositional_formula_parentheses} regarding parentheses.

    \thmitem{def:propositional_syntax/formula} The set \( \boldop{Form} \) of \term{propositional formulas} is the language \hyperref[def:grammar_derivation/grammar_language]{generated} by this grammar schema with \( \bnfpn{formula} \) as a starting rule. Propositional formulas are also called sentenced unlike in first-order logic where only specific formulas are called sentences --- see \fullref{def:first_order_syntax/ground_formula}.

    The grammar of propositional formulas is unambiguous as shown by \fullref{thm:propositional_formulas_are_unambiguous}, which makes it possible to perform proofs via \fullref{thm:structural_induction_on_unambiguous_grammars}.

    \thmitem{def:propositional_syntax/subformula} If \( \varphi \) and \( \psi \) are formulas and \( \psi \) is a \hyperref[def:formal_language/subword]{subword} of \( \varphi \), we say that \( \psi \) is a \term{subformula} of \( \varphi \).

    \thmitem{def:propositional_syntax/variables} For each formula \( \varphi \), we inductively define its \term{variables} to be elements of the set
    \begin{equation}\label{eq:def:propositional_syntax/varables}
      \boldop{Var}(\varphi) \coloneqq \begin{cases}
        \varnothing,                                  &\varphi \in \set{ \top, \bot } \\
        \set{ P },                                    &\varphi = P \in \boldop{Prop} \\
        \boldop{Var}(\psi),                           &\varphi = \neg \psi \\
        \boldop{Var}(\psi) \cup \boldop{Var}(\theta), &\varphi = \psi \bincirc \theta, \bincirc \in \Sigma.
      \end{cases}
    \end{equation}

    Note that \( \boldop{Var}(\varphi) \) can naturally be totally ordered by the position of the first occurrence of a variable.
  \end{thmenum}
\end{definition}

\begin{proposition}\label{thm:propositional_formulas_are_unambiguous}
  The grammar of \hyperref[def:propositional_syntax/formula]{propositional formulas} is \hyperref[def:grammar_derivation/unambiguous]{unambiguous}.
\end{proposition}
\begin{proof}
  The proof is analogous to \fullref{ex:natural_number_arithmetic_grammar/derivation}.
\end{proof}

\begin{remark}\label{rem:propositional_formula_parentheses}
  We use the following two \enquote{abuse-of-notation} conventions regarding parentheses:
  \begin{thmenum}
    \thmitem{rem:propositional_formula_parentheses/outermost} We may skip the outermost parentheses in formulas with top-level \hyperref[def:propositional_language/connectives]{connectives}, e.g. we may write \( P \wedge Q \) rather than \( (P \wedge Q) \).

    \thmitem{rem:propositional_formula_parentheses/associative} Because of the associativity of \( \wedge \) and \( \vee \), which is implied by \fullref{def:propositional_formula_induced_function} and \fullref{def:standard_boolean_operators}, we may skip the parentheses in chains like
    \begin{equation*}
      ( \ldots ((P_1 \wedge P_2) \wedge P_3) \wedge \ldots \wedge P_{n-1} ) \wedge P_n.
    \end{equation*}
    and instead write
    \begin{equation*}
      P_1 \wedge P_2 \wedge \ldots \wedge P_{n-1} \wedge P_n.
    \end{equation*}

    \thmitem{rem:first_order_formula_parentheses/additional} Although not formally necessary, for the sake of readability we may choose to add parentheses around certain formulas like
    \begin{equation*}
      \neg P \vee \neg Q.
    \end{equation*}
    and instead write
    \begin{equation*}
      (\neg P) \vee \neg Q.
    \end{equation*}

    This latter convention is more useful for quantifiers in \hyperref[def:first_order_syntax/formula]{first-order formulas}.
  \end{thmenum}

  These are only notations shortcuts in the \hyperref[rem:metalogic]{metalanguage} and the formulas themselves (as abstract mathematical objects) are still assumed to contain parentheses that help them avoid syntactic ambiguity (see \fullref{thm:propositional_formulas_are_unambiguous}).
\end{remark}

\begin{definition}\label{def:material_implication}
  Theorems in mathematics usually have the form \( P \rightarrow Q \). Formulas of this form are called \term{material implications} in order to distinguish them from logical implication, which relates to the metatheoretic concept of \hyperref[def:propositional_semantics/entailment]{entailment} (see \cite{MathSE:material_vs_logical_implication}). Note that the term \enquote{material implication} sometimes also refers to the \hyperref[def:propositional_language/connectives/conditional]{conditional connective \( \rightarrow \)} itself.

  We introduce terminology that is conventionally used when dealing with theorems.

  \begin{thmenum}
    \thmitem{def:material_implication/sufficient_condition} \( P \) is a \term{sufficient condition} for \( Q \).

    \thmitem{def:material_implication/necessary_condition} \( Q \) is a \term{necessary condition} for \( P \).

    \thmitem{def:material_implication/antecedent} \( P \) the \term{antecedent} of \( \varphi \).

    \thmitem{def:material_implication/consequent} \( Q \) the \term{consequent} of \( \varphi \).

    \thmitem{def:material_implication/inverse} The formula \( \neg P \rightarrow \neg Q \) is the \term[bg=противоположна,ru=противоположная]{inverse} of \( \varphi \).

    \thmitem{def:material_implication/converse} The formula \( Q \rightarrow P \) is the \term[bg=обратна,ru=обратная]{converse} of \( \varphi \).

    \thmitem{def:material_implication/contrapositive} The formula \( \neg Q \rightarrow \neg P \) is the \term{contrapositive} of \( \varphi \). In classical logic, it is \hyperref[def:propositional_semantics/equivalence]{equivalent} to the original formula due to \fullref{thm:boolean_equivalences/contrapositive}.
  \end{thmenum}
\end{definition}

\begin{definition}\label{def:propositional_valuation}
  We define \term[bg=оценка,ru=оценка]{valuations} for propositional formulas. It is possible to define different valuations, so in case of doubt, we will refer to the one defined here as the \term{classical valuation} giving \term{classical semantics}.

  This valuation implicitly depends on the \hyperref[def:boolean_algebra]{Boolean algebra} fixed in \fullref{def:boolean_function}. When dealing with \hyperref[def:propositional_heyting_algebra_semantics]{Heyting semantics}, we use more general Heyting algebras where not only the top and bottom, but also other values are utilized.

  \begin{thmenum}
    \thmitem{def:propositional_valuation/interpretation} A \term{propositional interpretation} is a function with signature \( I: \boldop{Prop} \to \set{ T, F } \). See \fullref{def:boolean_value} for remarks regarding the \hyperref[def:boolean_algebra]{Boolean algebra} \( \set{ T, F } \) and the \fullref{def:standard_boolean_operators} for a list of some standard Boolean operators.

    \thmitem{def:propositional_valuation/formula_valuation} Given an interpretation \( I \), we define the \term{valuation} of a formula \( \varphi \) inductively as
    \begin{equation}\label{eq:def:propositional_valuation/formula_interpretation}
      \varphi\Bracks{I} \coloneqq \begin{cases}
        T,                                         &\varphi = \top \\
        F,                                         &\varphi = \bot \\
        I(P),                                      &\varphi = P \in \boldop{Prop} \\
        \overline{\psi\Bracks{I}},                 &\varphi = \neg \psi \\
        \psi_1\Bracks{I} \bincirc \psi_2\Bracks{I} &\varphi = \psi_1 \bincirc \psi_2, \bincirc \in \Sigma,
      \end{cases}
    \end{equation}
  \end{thmenum}
  where \( \bincirc \) on the left denotes the \hyperref[def:standard_boolean_operators]{Boolean operator} corresponding to the connective \( \bincirc \) on the right.
\end{definition}

\begin{remark}\label{rem:propositional_formula_valuation_without_variable_assignment}
  If we know that \( \boldop{Var}(\varphi) \subseteq \{ P_1, \ldots, P_n \} \), it follows that the \hyperref[def:first_order_valuation/formula_valuation]{valuation} \( \varphi\Bracks{I} \) only depends on the particular values \( I(P_1), \ldots, I(P_n) \) of \( I \).

  Let \( x_1, \ldots, x_n \in \set{ F, T } \) and let \( I \) be such that \( I(P_k) = x_k \) for \( k = 1, \ldots, n \). We introduce the notation
  \begin{equation}\label{eq:rem:propositional_formula_valuation_without_variable_assignment/short_semantic}
    \varphi\Bracks{x_1, \ldots, x_n}
  \end{equation}
  for \( \varphi\Bracks{I} \) because the rest of the interpretation \( I \) plays no role here. We may also use
  \begin{equation}\label{eq:rem:propositional_formula_valuation_without_variable_assignment/short_syntactic}
    \varphi[\psi_1, \ldots, \psi_n]
  \end{equation}
  to denote \hyperref[def:propositional_substitution]{substitution}.

  When using this notation, we implicitly assume that \( \boldop{Var}(\varphi) \subseteq \set{ P_1, \ldots, P_n } \).
\end{remark}

\begin{definition}\label{def:propositional_formula_induced_function}
  Let \( \varphi \) be a propositional formula and let \( \boldop{Var}(\varphi) = \set{ P_1, \ldots, P_n } \) be an ordering of the free variables of \( \varphi \). We define the \hyperref[def:boolean_function]{Boolean function}
  \begin{equation}\label{eq:def:propositional_formula_induced_function}
    \begin{split}
      &\fun_\varphi: \set{ T, F }^n \to \set{ T, F } \\
      &\fun_\varphi(x_1, \ldots, x_n) \coloneqq \varphi\Bracks{x_1, \ldots, x_n}.
    \end{split}
  \end{equation}
\end{definition}

\begin{definition}\label{def:propositional_semantics}
  We now define \term{semantical} properties of propositional formulas. Because of the connection with \hyperref[def:boolean_function]{Boolean functions} given in \fullref{def:propositional_formula_induced_function}, we also formulate some of the properties using Boolean functions.

  \begin{thmenum}
    \thmitem{def:propositional_semantics/satisfiability}\mcite[def. 7.14]{OpenLogicFull} Given an interpretation \( I \) and a set \( \Gamma \) of formulas, we say that \( I \) \term{satisfies} \( \Gamma \) if, for every formula \( \varphi \in \Gamma \) we have \( \varphi\Bracks{I} = T \).

    We also say that \( I \) is a \term{model} of \( \Gamma \) and write \( I \vDash \Gamma \).

    If \( \Gamma = \set{ \gamma_1, \ldots, \gamma_n } \) is a finite ordered set, we use the shorthand \( I \vDash \gamma_1, \ldots, \gamma_n \) rather than \( I \vDash \set{ \gamma_1, \ldots, \gamma_n } \). In particular, if \( \Gamma = \set{ \varphi } \) we write \( I \vDash \varphi \).

    Note that every interpretation vacuously satisfies the empty set \( \Gamma = \varnothing \) of formulas.

    We say that \( \Gamma \) is \term{satisfiable} if there exists a model for \( \Gamma \).

    \thmitem{def:propositional_semantics/entailment} We say that the set of formulas \( \Gamma \) \term{entails} the set of formulas \( \Delta \) and write \( \Gamma \vDash \Delta \) if either of the following hold:
    \begin{itemize}
      \thmitem{def:propositional_semantics/entailment/direct} Every model of \( \Gamma \) is also a model of \( \Delta \).
      \thmitem{def:propositional_semantics/entailment/functional} The following \hyperref[thm:def:function/properties/preimage]{preimage} inclusion holds:
      \begin{equation*}
        \bigcap_{\varphi \in \Gamma} \fun_\varphi^{-1}(T) \subseteq \bigcap_{\psi \in \Delta} \fun_\psi^{-1}(T).
      \end{equation*}
    \end{itemize}

    \thmitem{def:propositional_semantics/tautology} The formula \( \varphi \) is a (semantic) \term{tautology} if either:
    \begin{itemize}
      \thmitem{def:propositional_semantics/tautology/interpretations} Every interpretation satisfies \( \varphi \).
      \thmitem{def:propositional_semantics/tautology/entailment} The empty set \( \Gamma = \varnothing \) of formulas entails \( \varphi \), i.e. \( \vDash \varphi \).
      \thmitem{def:propositional_semantics/tautology/functional} The function \( \fun_\varphi \) is canonically true.
    \end{itemize}

    We also say that \( \varphi \) is \term{valid}.

    \thmitem{def:propositional_semantics/contradiction} Dually, \( \varphi \) is a (semantic) \term{contradiction} if either:
    \begin{itemize}
      \thmitem{def:propositional_semantics/contradiction/interpretations} No interpretation satisfies \( \varphi \).
      \thmitem{def:propositional_semantics/contradiction/entailment} The formula \( \varphi \) entails \( \bot \), i.e. \( \varphi \vDash \bot \).
      \thmitem{def:propositional_semantics/contradiction/functional} The function \( \fun_\varphi \) is canonically false.
    \end{itemize}

    \thmitem{def:propositional_semantics/equivalence} We say that \( \varphi \) and \( \psi \) are \term{semantically equivalent} and write \( \varphi \gleichstark \psi \) if either:
    \begin{itemize}
      \thmitem{def:propositional_semantics/equivalence/interpretations} We have \( \varphi\Bracks{I} = \psi\Bracks{I} \) for every interpretation \( I \).
      \thmitem{def:propositional_semantics/equivalence/entailment} Both \( \varphi \vDash \psi \) and \( \psi \vDash \varphi \).
    \end{itemize}

    \thmitem{def:propositional_semantics/equisatisfiability} A weaker notion than that of semantic equivalence is that of \term{equisatisfiability}. We say that the families \( \Gamma \) and \( \Delta \) are equisatisfiable if the following holds: \enquote{\( \Gamma \) is satisfiable if and only if \( \Delta \) is satisfiable}. For single-formula families \( \Gamma = \set{ \varphi } \) and \( \Delta = \set{ \psi } \), the following are equivalent conditions for equisatisfiability:
    \begin{itemize}
      \thmitem{def:propositional_semantics/equisatisfiability/interpretations} There exist interpretations \( I \) and \( J \) such that \( \varphi\Bracks{I} = \psi\Bracks{J} \).
      \thmitem{def:propositional_semantics/equisatisfiability/functional} We have \( \fun_\varphi = \fun_\psi \) for the induced functions.
    \end{itemize}

    A trivial example of equisatisfiable, but not equivalent formulas are \( \varphi = P \) and \( \psi = Q \) for \( P \neq Q \).
  \end{thmenum}
\end{definition}

\begin{theorem}\label{thm:lindenmaum_tarski_algebra_of_full_propositional_logic}
  We give an explicit connection between \hyperref[def:propositional_syntax/formula]{propositional formulas} and \hyperref[def:boolean_function]{Boolean functions}.

  \begin{thmenum}
    \thmitem{thm:lindenmaum_tarski_algebra_of_full_propositional_logic/equivalence_classes} The \hyperref[def:propositional_semantics/equivalence]{semantic equivalence} \( \gleichstark \) is an equivalence relation on the set \( \boldop{Form} \) of all propositional formulas.

    \thmitem{thm:lindenmaum_tarski_algebra_of_full_propositional_logic/bijection} The \hyperref[def:lindenbaum_tarski_algebra]{Lindenbaum-Tarski algebra}  \( \boldop{Form} / {{}\gleichstark} \) of all propositional formulas with respect to semantic equivalence is bijective with the set of all \hyperref[def:boolean_function]{Boolean functions} of arbitrary arity.

    Both are provably Boolean algebras, but with very different proofs --- the Lindenbaum-Tarski algebra is Boolean due to the purely syntactic \fullref{thm:intuitionistic_lindenbaum_tarski_algebra} and the set of all Boolean functions is a Boolean algebra due to the semantic \fullref{thm:functions_over_model_form_model}. This is another demonstration of \fullref{thm:classical_propositional_logic_is_sound_and_complete}.

    See \fullref{rem:thm:intuitionistic_lindenbaum_tarski_algebra/syntactic_proof}.
  \end{thmenum}
\end{theorem}
\begin{proof}
  \SubProofOf{thm:lindenmaum_tarski_algebra_of_full_propositional_logic/equivalence_classes} Follows from the equivalences in \fullref{def:equivalence_relation}.

  \SubProofOf{thm:lindenmaum_tarski_algebra_of_full_propositional_logic/bijection} Follows from the equivalences in \fullref{def:propositional_semantics/equivalence}.
\end{proof}

\begin{proposition}\label{thm:boolean_equivalences}
  The following (and many more) are called \term{Boolean equivalences} because they are actually statements about our choice of \hyperref[def:standard_boolean_operators]{standard Boolean operators}. They are formulated here because the framework of propositional logic is more convenient for stating the equivalences. Note that most of these equivalences fail in \hyperref[def:intuitionistic_propositional_deductive_systems]{intuitionistic logic}.

  For arbitrary propositional formulas \( \varphi \) and \( \psi \), the following semantic equivalences hold:
  \begin{thmenum}
    \thmitem{thm:boolean_equivalences/negation_bottom} \hyperref[def:propositional_language/negation]{Negation} can be expressed via the \hyperref[def:propositional_language/constants/falsum]{falsum}:
    \begin{equation}\label{eq:thm:boolean_equivalences/negation_bottom}
      \begin{split}
        \mathllap{\neg \varphi} &\gleichstark \mathrlap{\varphi \rightarrow \bot}.
      \end{split}
    \end{equation}

    \thmitem{thm:boolean_equivalences/double_negation} \hyperref[def:propositional_language/negation]{Negation} is an \hyperref[def:set_with_involution]{involution}:
    \begin{equation}\label{eq:thm:boolean_equivalences/double_negation}
      \begin{split}
        \mathllap{\neg \neg \varphi} &\gleichstark \mathrlap{\varphi}.
      \end{split}
    \end{equation}

    \thmitem{thm:boolean_equivalences/contrapositive} A \hyperref[def:material_implication]{material implication} is equivalent to its \hyperref[def:material_implication/contrapositive]{contrapositive}:
    \begin{equation}\label{eq:thm:boolean_equivalences/contrapositive}
      \begin{split}
        \mathllap{\varphi \rightarrow \psi} &\gleichstark \mathrlap{\neg \psi \rightarrow \neg \varphi.}
      \end{split}
    \end{equation}

    \thmitem{thm:boolean_equivalences/conditional_as_disjunction} A \hyperref[def:propositional_language/connectives/conditional]{conditional} is a \hyperref[def:propositional_language/connectives/disjunction]{disjunction} with the \hyperref[def:material_implication/antecedent]{antecedent} negated:
    \begin{equation}\label{eq:thm:boolean_equivalences/conditional_as_disjunction}
      \begin{split}
        \mathllap{\varphi \rightarrow \psi} &\gleichstark \mathrlap{ \neg \varphi \vee \psi. }
      \end{split}
    \end{equation}

    \thmitem{thm:boolean_equivalences/biconditional_via_conditionals} A \hyperref[def:propositional_language/connectives/biconditional]{biconditional} is a \hyperref[def:propositional_language/connectives/conjunction]{conjunction} of \hyperref[def:propositional_language/connectives]{conditionals}:
    \begin{equation}\label{eq:thm:boolean_equivalences/biconditional_via_conditionals}
      \begin{split}
        \mathllap{\varphi \leftrightarrow \psi} &\gleichstark \mathrlap{(\varphi \rightarrow \psi) \wedge (\psi \rightarrow \varphi).}
      \end{split}
    \end{equation}

    \thmitem{thm:boolean_equivalences/biconditional_as_conjunction} The \hyperref[def:propositional_language/connectives/biconditional]{biconditional} is a \hyperref[def:propositional_language/connectives/disjunction]{conjunction} of \hyperref[def:propositional_language/connectives/conjunction]{disjunctions}:
    \begin{equation}\label{eq:thm:boolean_equivalences/biconditional_as_conjunction}
      \begin{split}
        \mathllap{\varphi \leftrightarrow \psi} &\gleichstark \mathrlap{(\neg \varphi \vee \psi) \wedge (\neg \varphi \vee \psi).}
      \end{split}
    \end{equation}

    \thmitem{thm:boolean_equivalences/biconditional_as_disjunction} The \hyperref[def:propositional_language/connectives/biconditional]{biconditional} is a \hyperref[def:propositional_language/connectives/disjunction]{disjunction} of \hyperref[def:propositional_language/connectives/conjunction]{conjunctions}:
    \begin{equation}\label{eq:thm:boolean_equivalences/biconditional_as_disjunction}
      \begin{split}
        \mathllap{\varphi \leftrightarrow \psi} &\gleichstark \mathrlap{(\varphi \wedge \psi) \vee (\neg \varphi \wedge \neg \psi).}
      \end{split}
    \end{equation}

    \thmitem{thm:boolean_equivalences/biconditional_member_negation} A \hyperref[def:propositional_language/connectives/biconditional]{biconditional} is equivalent its termwise negation:
    \begin{equation}\label{eq:thm:boolean_equivalences/biconditional_member_negation}
      \begin{split}
        \mathllap{\neg \varphi \leftrightarrow \neg \psi} &\gleichstark \mathrlap{\varphi \leftrightarrow \psi.}
      \end{split}
    \end{equation}

    \thmitem{thm:boolean_equivalences/biconditional_negation} A negation of a \hyperref[def:propositional_language/connectives/biconditional]{biconditional} is again a biconditional with one of the terms negated:
    \begin{equation}\label{eq:thm:boolean_equivalences/biconditional_negation}
      \begin{split}
        \mathllap{\neg \parens{\varphi \leftrightarrow \psi}}
        &\gleichstark
        \mathrlap{\neg \varphi \leftrightarrow \psi \gleichstark}
        \\ &\gleichstark
        \mathrlap{\varphi \leftrightarrow \neg \psi.}
      \end{split}
    \end{equation}
  \end{thmenum}
\end{proposition}
\begin{proof}
  The proofs follow directly from the table in \fullref{def:standard_boolean_operators}.
\end{proof}

\begin{definition}\label{def:propositional_substitution}
  We sometimes want to substitute a propositional variable with another variable or even with a formula. This is akin to applying a \hyperref[def:boolean_function]{Boolean function} like \( x \vee y \) to different variables (e.g. to obtain \( x \vee x \)) or even concrete values (e.g. \( F \vee T \)), except that it is done on a purely syntactic level without involving any semantics involved.

  It does not pose any technical difficulty to extend this definition beyond replacing a variable like it is usually done (e.g. \cite[def. 7.8]{OpenLogicFull}). Not only that, we can then use this mechanism to define complicated rewriting rules as in \fullref{alg:conjunctive_normal_form_reduction} and have semantic equivalence automatically follow from \fullref{thm:propositional_substitution_equivalence}.

  \begin{thmenum}
    \thmitem{def:propositional_substitution/single} We define the \term{substitution} of the propositional formula \( \theta \) with \( \chi \) in \( \varphi \) as
    \begin{equation}\label{eq:def:propositional_substitution/single}
      \varphi[\theta \mapsto \chi] \coloneqq \begin{cases}
        \chi,                                                             &\varphi = \theta \\
        \varphi,                                                          &\varphi \neq \theta \T{and} \varphi \in \set{ \top, \bot } \cup \boldop{Prop} \\
        \neg \psi[\theta \mapsto \chi],                                   &\varphi \neq \theta \T{and} \varphi = \neg \psi \\
        \psi_1[\theta \mapsto \chi] \bincirc \psi_2[\theta \mapsto \chi], &\varphi \neq \theta \T{and} \varphi = \psi_1 \bincirc \psi_2, \circ \in \Sigma.
      \end{cases}
    \end{equation}

    Note that it is not strictly necessary for \( \theta \) to be a subformula of \( \varphi \).

    In the case where \( \theta \) is a single variable, if \( P \in \boldop{Var}(\varphi) \), then \( \varphi[P \mapsto \chi] \) is said to be an \term{instance} of \( \varphi \).

    \thmitem{def:propositional_substitution/simultaneous} We will now define \term{simultaneous substitution} of \( \theta_1, \ldots, \theta_n \) with \( \chi_1, \ldots, \chi_n \). We wish to avoid the case where \( \theta_k \) is a subformula of \( \chi_{k-1} \) and it accidentally gets replaced during \( \varphi[\theta_{k-1} \mapsto \chi_{k-1}][\theta_k \mapsto \chi_k] \).

    Define
    \begin{equation*}
      \cat{Bound} \coloneqq \boldop{Var}(\chi_1) \cup \ldots \cup \boldop{Var}(\chi_n).
    \end{equation*}
    and, for each variable \( P_k \) in \( \cat{Bound} \), pick a variable \( Q_k \) from \( \boldop{Prop} \setminus \boldop{Bound} \) (we implicitly assume the existence of enough variables in \( \boldop{Prop} \)). Let \( m \) be the \hyperref[def:cardinal]{cardinality} of \( \boldop{Bound} \). The simultaneous substitution can now be defined as
    \begin{align*}
      \varphi[\theta_1 \mapsto \chi_1, \ldots, \theta_n \mapsto \chi_n] \coloneqq \varphi
      [\theta_1 \mapsto \chi_1[P_1 \mapsto Q_1, \ldots, P_m \mapsto Q_m]] \\
      \vdots \hspace{3cm} \\
      [\theta_n \mapsto \chi_n[P_1 \mapsto Q_1, \ldots, P_m \mapsto Q_m]] \\
      [Q_1 \mapsto P_1, \ldots, Q_m \mapsto P_m].
    \end{align*}
  \end{thmenum}
\end{definition}

\begin{proposition}\label{thm:propositional_substitution_equivalence}
  If \( \theta \) is a subformula of \( \varphi \) and if \( \theta \gleichstark \chi \), then
  \begin{equation}\label{eq:thm:propositional_substitution_equivalence}
    \varphi[\theta \mapsto \chi] \gleichstark \varphi.
  \end{equation}

  By induction, this also holds for \hyperref[def:propositional_substitution/simultaneous]{simultaneous substitution}.
\end{proposition}
\begin{proof}
  We use structural induction on \( \varphi \):

  \begin{itemize}
    \item If \( \varphi = \theta \), then \( \varphi[\theta \mapsto \chi] = \chi \) and, by definition,
    \begin{equation*}
      \varphi = \theta \gleichstark \chi = \varphi[\theta \mapsto \chi].
    \end{equation*}

    \item If \( \varphi \neq \theta \) and \( \varphi \in \set{ \top, \bot } \cup \boldop{Prop} \), then \( \varphi[\theta \mapsto \chi] = \varphi \) and \eqref{eq:thm:propositional_substitution_equivalence} again holds trivially.

    \item If \( \varphi \neq \theta \) and \( \varphi = \neg \chi \) and if the inductive hypothesis holds for \( \chi \), then \( \varphi[\theta \mapsto \chi] = \neg \psi[\theta \mapsto \chi] \). For any interpretation \( I \),
    \begin{equation*}
      \parens[\Big]{ \varphi[\theta \mapsto \chi] }\Bracks{I}
      =
      \overline{\parens[\Big]{ \psi[\theta \mapsto \chi] }\Bracks{I}}
      \reloset {\T{ind.}} =
      \overline{\psi\Bracks{I}}
      =
      \varphi\Bracks{I}.
    \end{equation*}

    Therefore, \eqref{eq:thm:propositional_substitution_equivalence} holds in this case.

    \item If \( \varphi \neq \theta \) and \( \varphi = \psi_1 \bincirc \psi_2, \bincirc \in \Sigma \) and if the inductive hypothesis holds for both \( \psi_1 \) and \( \psi_2 \), then for any interpretation \( I \),
    \begin{equation*}
      \parens[\Big]{ \varphi[\theta \mapsto \chi] }\Bracks{I}
      =
      \parens[\Big]{ \psi_1[\theta \mapsto \chi] }\Bracks{I} \bincirc \parens[\Big]{ \psi_2[\theta \mapsto \chi] }\Bracks{I}
      \reloset {\T{ind.}} =
      \psi_1\Bracks{I} \bincirc \psi_2\Bracks{I}
      =
      \varphi\Bracks{I}.
    \end{equation*}

    Therefore, \eqref{eq:thm:propositional_substitution_equivalence} holds in this case also.
  \end{itemize}

  We have verified that \eqref{eq:thm:propositional_substitution_equivalence} holds in all cases.
\end{proof}

\begin{remark}\label{rem:smaller_propositional_language}
  For \hyperref[def:propositional_semantics]{semantical} concepts, it is immaterial which element of an equivalence class we consider. \hyperref[def:boolean_closure]{Complete sets of Boolean operations} allow us to represent each formula using a strict subset of the \hyperref[def:propositional_language/constants]{propositional constants}, \hyperref[def:propositional_language/negation]{negation} and \hyperref[def:propositional_language/connectives]{connectives}. \Fullref{ex:posts_completeness_theorem} shows some concrete commonly used complete sets of Boolean operations. This is also the motivation for studying \hyperref[def:lindenbaum_tarski_algebra]{Lindenbaum-Tarski algebras}.

  This is useful in
  \begin{itemize}
    \item Reduction to normal forms such as the \hyperref[def:conjunctive_disjunctive_normal_form]{conjunctive normal form} in \fullref{alg:conjunctive_normal_form_reduction}.

    \item \hyperref[def:propositional_semantics/satisfiability]{Satisfiability} proofs that rely on \hyperref[rem:structural_recursion_and_induction]{structural induction} because it allows us to consider less cases in the induction.

    \item Having fewer rules in \hyperref[alg:conjunctive_normal_form_reduction]{deductive systems}. For example, we may choose to add \eqref{eq:thm:minimal_propositional_negation_laws/pierce} to the axioms of the \hyperref[def:positive_implicational_deductive_system]{positive implicational derivation system} and due to \fullref{thm:minimal_propositional_negation_laws} this derivation system would be able to emulate the \hyperref[def:classical_propositional_deductive_systems]{classical derivation system}.
  \end{itemize}
\end{remark}

\begin{definition}\label{def:conjunctive_disjunctive_normal_form}
  We will now introduce the conjunctive and disjunctive normal forms.

  \begin{thmenum}
    \thmitem{def:conjunctive_disjunctive_normal_form/literal} A \term{literal} is either a propositional variable \( L = P \) or a negation \( L = \neg P \) of a propositional variable. These are called \term{positive} and \term{negative} literals, correspondingly.

    \thmitem{def:conjunctive_disjunctive_normal_form/normal_form} A propositional formula \( \varphi \) is in \term{conjunctive normal form} (resp. \term{disjunctive normal form}) if \( \varphi \) is a finite conjunction of disjunctions (resp. finite disjunction of conjunctions) of literals. That is, if \( \varphi \) is in conjunctive normal form, it has the form
    \begin{equation*}
      (L_{1,1} \vee \ldots \vee L_{1,n_1}) \wedge \cdots \wedge (L_{k,1} \vee \ldots \vee L_{k,n_k}).
    \end{equation*}

    \thmitem{def:conjunctive_disjunctive_normal_form/conjunct_disjunct} A \term{disjunct} (resp. a \term{conjunct}) is a set of literals, the difference between the two being the context in which they are used. To each formula in conjunctive normal form there corresponds a set of disjuncts and to each formula in disjunctive normal form there corresponds a set of conjuncts.
  \end{thmenum}
\end{definition}

\begin{algorithm}\label{alg:conjunctive_normal_form_reduction}
  Let \( \varphi \) be any formula. We explicitly derive a formula in conjunctive normal form that is \hyperref[def:propositional_semantics/equivalence]{semantically equivalent} to \( \varphi \). In a software implementation, we can easily construct an efficient recursive procedure based on the following steps.

  \begin{thmenum}
    \thmitem{alg:conjunctive_normal_form_reduction/constants} Pick any variable \( P \) and substitute
    \begin{align*}
      \top \T{with} P \vee \neg P, && \bot \T{with} P \wedge \neg P
    \end{align*}
    to get rid of the \hyperref[def:propositional_language/constants]{propositional constants}.

    \thmitem{alg:conjunctive_normal_form_reduction/iff} For any subformulas \( \psi \) and \( \theta \) of \( \varphi \), perform the substitution
    \begin{equation*}
      \psi \leftrightarrow \theta \T{with} (\psi \rightarrow \theta) \wedge (\theta \rightarrow \psi)
    \end{equation*}
    to get rid of \hyperref[def:propositional_language/connectives/biconditional]{biconditional connectives}. Semantic equivalence with \( \varphi \) is then justified by \fullref{thm:boolean_equivalences/biconditional_via_conditionals}.

    \thmitem{alg:conjunctive_normal_form_reduction/implies} For any subformulas \( \psi \) and \( \theta \) of \( \varphi \), perform the substitution
    \begin{equation*}
      \psi \rightarrow \theta \T{with} \neg \psi \vee \theta.
    \end{equation*}
    to get rid of \hyperref[def:propositional_language/connectives/conditional]{conditional connectives}. Equivalence with \( \varphi \) is justified by \fullref{thm:boolean_equivalences/conditional_as_disjunction}.

    \thmitem{alg:conjunctive_normal_form_reduction/de_morgan} For any subformulas \( \psi \) and \( \theta \) of \( \varphi \), use \fullref{thm:de_morgans_laws} to justify the substitution
    \begin{align*}
      \neg(\psi \vee \theta) \T{with} \neg \psi \wedge \neg \theta
      &&
      \neg(\psi \wedge \theta) \T{with} \neg \psi \vee \neg \theta.
    \end{align*}

    In order to ensure that \hyperref[def:propositional_language/negation]{negation} is only present before propositional variables, repeat \ref{alg:conjunctive_normal_form_reduction/de_morgan} until we reach a fixed point, i.e. until nothing is substituted anymore.

    \thmitem{alg:conjunctive_normal_form_reduction/double_negation} For any variable \( P \) of \( \varphi \), use \fullref{eq:thm:boolean_equivalences/double_negation} to justify the substitution
    \begin{equation*}
      \neg \neg P \T{with} P.
    \end{equation*}

    \thmitem{alg:conjunctive_normal_form_reduction/distributivity} Finally, for any subformulas \( \psi \), \( \theta \) and \( \chi \) of \( \varphi \), use \hyperref[eq:def:semilattice/distributive_lattice/finite/join_over_meet]{distributivity} of \( \vee \) over \( \wedge \) to justify the substitution
    \begin{equation*}
      \psi \vee (\theta \wedge \chi) \T{with} (\psi \vee \theta) \wedge (\theta \vee \chi).
    \end{equation*}

    In order to ensure that conjunction is always one level higher than disjunction, repeat \ref{alg:conjunctive_normal_form_reduction/distributivity} until we reach a fixed point.
  \end{thmenum}

  The resulting formula is in conjunctive normal form. As a consequence of \fullref{thm:propositional_substitution_equivalence}, it is equivalent to \( \varphi \).
\end{algorithm}

\subsection{First-order logic}\label{subsec:first_order_logic}

\begin{definition}\label{def:first_order_language}\mcite[sec. 15.2]{OpenLogicFull}
  The idea of first-order predicate logic (we will omit \enquote{predicate} and only refer to \enquote{first-order logic}) is to create a formal language whose semantics (given by structures) support boolean operations and can quantify over all elements of an ambient universe. Unlike in \hyperref[subsec:propositional_logic]{propositional logic}, there are different first-order logic languages.

  The alphabet for a \term{first-order logic \hyperref[def:formal_language]{language}} \( \mscrL \) extends that of \hyperref[subsec:propositional_logic]{propositional logic} and consists of two types of symbols (note that \fullref{rem:propositional_language_is_alphabet} holds here also).

  \begin{description}
    \item[Logical symbols]
    \hfill
    \begin{thmenum}[series=def:first_order_language]
      \thmitem{def:first_order_language/propositional} The entirety of the \hyperref[subsec:propositional_logic]{propositional logic language} except for the \hyperref[def:propositional_language/prop]{propositional variables}.

      \thmitem{def:first_order_language/var} A nonempty at most \hyperref[thm:omega_is_a_cardinal]{countable} alphabet of \term{individual variables} \( \boldop{Var} \), usually denoted by small Greek letters \( \xi_1, \xi_2, \ldots \) or \( \xi, \eta, \zeta \) --- see \fullref{rem:mathematical_logic_conventions/variable_symbols}.

      \thmitem{def:first_order_language/quantifiers} The quantifiers \( Q = \set{ \forall, \exists } \):
      \begin{thmenum}
        \thmitem{def:first_order_language/quantifiers/universal} The \term{universal quantifier} \( \forall \).
        \thmitem{def:first_order_language/quantifiers/existential} The \term{existential quantifier} \( \exists \).
        \thmitem{def:first_order_language/quantifiers/dot} The dot \( . \) for separating a quantifier from its formula.
      \end{thmenum}

      The dot is not itself a quantifier and is not formally necessary --- we use it only for readability.

      \thmitem{def:first_order_language/equality} A symbol for \term{formal equality} \( \doteq \). Equality is sometimes omitted by logicians, but examples of first-order languages without formal equality are obscure.
    \end{thmenum}

    \item[Non-logical symbols]
    \hfill
    \begin{thmenum}[resume=def:first_order_language]
      \thmitem{def:first_order_language/func} A set of functional symbols, \( \boldop{Fun} \), whose elements are usually denoted by \( f_1, f_2, \ldots \) or \( f, g \) or by symbols like \( \otimes \). In the latter case we usually use the infix notation discussed in \fullref{rem:first_order_formula_conventions/infix}. Each functional symbol has an associated natural number called its \term{arity}, denoted by \( \# f \). Functional symbols with a zero arity are called \term{constants}.

      \thmitem{def:first_order_language/pred} A set of predicate symbols, \( \boldop{Pred} \), whose elements are usually denoted by \( p_1, p_2, \ldots \) or by symbols like \( \leq \). Again, in the latter case we use infix notation. Predicate symbols also have an associated arity. Predicate symbols with zero arity are called \term{propositional variables}.
    \end{thmenum}
  \end{description}

  The logical symbols are common for all first-order languages. Thus, first-order languages differ by their non-logical symbols. The collection of functional and predicate symbols of a language are sometimes called its \term{signature}.
\end{definition}

\begin{definition}\label{def:first_order_syntax}
  Similarly to the \hyperref[def:propositional_syntax]{syntax of propositional logic}, we define the \term{syntax} of a fixed first-order language \( \mscrL \).

  \begin{thmenum}
    \thmitem{def:first_order_syntax/grammar_schema} Consider the following \hyperref[rem:backus_naur_form]{grammar schema}:
    \begin{bnf*}
      \bnfprod{variable}        {v \in \boldop{Var}} \\
      \bnfprod{connective}      {\bincirc \in \Sigma} \\
      \bnfprod{quantifier}      {\bnfts{\( \forall \)} \bnfor \bnfts{\( \exists \)}} \\
      \bnfprod{unary function}  {f \in \boldop{Fun}, \#f = 1} \\
      \bnfmore                  {\vdots} \\
      \bnfprod{\( n \)-ary function}  {f \in \boldop{Fun}, \#f = n \T{(standalone rule for each \( n \))}} \\
      \bnfmore                  {\vdots} \\
      \bnfprod{unary predicate} {p \in \boldop{Pred}, \#p = 1} \\
      \bnfmore                  {\vdots} \\
      \bnfprod{\( n \)-ary predicate} {p \in \boldop{Pred}, \#p = n \T{(standalone rule for each \( n \))}} \\
      \bnfmore                  {\vdots} \\
      \bnfprod{term}            {\bnfpn{variable} \bnfor} \\
      \bnfmore                  {\bnfpn{unary function} \bnfsp \bnfts{(} \bnfsp \bnfpn{term} \bnfsp \bnfts{)} \bnfor} \\
      \bnfmore                  {\hspace{3cm} \vdots} \\
      \bnfmore                  {\bnfpn{n-ary function} \bnfsp \underbrace{\bnfts{(} \bnfsp \bnfpn{term} \bnfts{,} \bnfsk \bnfts{,} \bnfpn{term} \bnfsp \bnfts{)}}_{n \T{terms}} \bnfor} \\
      \bnfmore                  {\hspace{3cm} \vdots} \\
      \bnfprod{atomic formula}  {\bnfts{\( \top \)} \bnfor \bnfts{\( \bot \)} \bnfor} \\
      \bnfmore                  {\bnfts{(} \bnfsp \bnfpn{term} \bnfsp \bnfts{\( \doteq \)} \bnfsp \bnfpn{term} \bnfsp \bnfts{)} \bnfor} \\
      \bnfmore                  {\bnfpn{unary predicate} \bnfsp \bnfts{(} \bnfsp \bnfpn{term} \bnfsp \bnfts{)} \bnfor} \\
      \bnfmore                  {\hspace{3cm} \vdots} \\
      \bnfmore                  {\bnfpn{n-ary predicate} \bnfsp \underbrace{\bnfts{(} \bnfsp \bnfpn{term} \bnfts{,} \bnfsk \bnfts{,} \bnfpn{term} \bnfsp \bnfts{)}}_{n \T{terms}} \bnfor} \\
      \bnfmore                  {\hspace{3cm} \vdots} \\
      \bnfprod{formula}         {\bnfpn{atomic formula} \bnfor} \\
      \bnfmore                  {\bnfts{\( \neg \)} \bnfpn{formula} \bnfor} \\
      \bnfmore                  {\bnfts{(} \bnfsp \bnfpn{formula} \bnfsp \bnfpn{connective} \bnfsp \bnfpn{formula} \bnfsp \bnfts{)} \bnfor} \\
      \bnfmore                  {\bnfpn{quantifier} \bnfsp \bnfpn{variable} \bnfsp \bnfts{.} \bnfsp \bnfpn{formula}}
    \end{bnf*}

    In practice, we usually only have functions and predicates of specific arities. Note that we can have infinitely many functions, but only finitely many different arities. The \hyperref[def:semimodule/theory]{theory of semimodules} is an example of a first-order language with infinitely many unary functional symbols, one constant and one binary functional symbol.

    If we need the grammars to have a finite set of rules, except for having only finitely many different arities, we need to introduce appropriate naming conventions for variables, functions and predicates, analogously to the \hyperref[def:propositional_syntax/grammar_schema]{grammar schema of propositional logic}.

    We use the conventions in \fullref{rem:propositional_formula_parentheses} regarding parentheses by extending them wherever appropriate.

    In order to simplify notation, we also use the conventions in \fullref{rem:first_order_formula_conventions}.

    \thmitem{def:first_order_syntax/term} The set \( \boldop{Term} \) of \term{terms} in \( \mscrL \) is the language \hyperref[def:grammar_derivation/grammar_language]{generated} by this grammar schema with \( \bnfpn{term} \) as a starting rule.

    The grammar of first-order terms is unambiguous due to \fullref{thm:first_order_terms_and_formulas_are_unambiguous}, which makes it possible to perform proofs via \fullref{thm:structural_induction_on_unambiguous_grammars}.

    \thmitem{def:first_order_syntax/subterm} If \( \tau \) and \( \kappa \) are terms and \( \kappa \) is a \hyperref[def:formal_language/subword]{subword} of \( \tau \), we say that \( \kappa \) is a \term{subterm} of \( \tau \).

    \thmitem{def:first_order_syntax/term_variables} For each term \( \tau \), we define its variables as
    \begin{equation}\label{eq:def:first_order_syntax/term_variables}
      \boldop{Var}(\tau) \coloneqq \begin{cases}
        \xi,                                                        &\tau = \xi \in \boldop{Var}, \\
        \boldop{Var}(\tau_1) \cup \ldots \cup \boldop{Var}(\tau_n), &\tau = f(\tau_1, \ldots, \tau_n).
      \end{cases}
    \end{equation}

    As in \fullref{def:propositional_syntax/variables}, \( \boldop{Var} \) is ordered by the position of the first occurrence of a variable.

    \thmitem{def:first_order_syntax/ground_term} A term \( \tau \) is called a \term{ground term} if \( \boldop{Var}(\tau) = \varnothing \). Ground terms are also called \term{closed terms}.

    \thmitem{def:first_order_syntax/atomic_formula} The set \( \boldop{Atom} \) of \term{atomic formulas} in \( \mscrL \) is the language \hyperref[def:grammar_derivation/grammar_language]{generated} by this grammar schema with \( \bnfpn{atomic formula} \) as a starting rule.

    \thmitem{def:first_order_syntax/formula} The set \( \boldop{Form} \) of \term{formulas} in \( \mscrL \) is the language \hyperref[def:grammar_derivation/grammar_language]{generated} by this grammar schema with \( \bnfpn{formula} \) as a starting rule.

    The \term{atomic formulas} are the ones generated from \( \bnfpn{atomic formula} \).

    The grammar of first-order formulas is unambiguous as shown by \fullref{thm:first_order_terms_and_formulas_are_unambiguous}.

    See \fullref{ex:first_order_substitution} for examples of first-order formulas.

    \thmitem{def:first_order_syntax/subformula} If \( \varphi \) and \( \psi \) are formulas and \( \psi \) is a \hyperref[def:formal_language/subword]{subword} of \( \varphi \), we say that \( \psi \) is a \term{subformula} of \( \varphi \).

    \thmitem{def:first_order_syntax/formula_terms} If \( \varphi \) is a formula, if \( \tau \) is a term and if \( \tau \) is a \hyperref[def:formal_language/subword]{subword} of \( \varphi \), we say that \( \tau \) is a \term{term} of \( \varphi \).

    \thmitem{def:first_order_syntax/formula_free_variables} The \term{free variables} of a formula are defined as
    \begin{equation}\label{eq:def:first_order_syntax/formula_free_variables}
      \boldop{Free}(\varphi) \coloneqq \begin{cases}
        \varnothing,                                                &\varphi \in \set{ \top, \bot } \\
        \boldop{Var}(\tau_1) \cup \ldots \cup \boldop{Var}(\tau_n), &\varphi = p(\tau_1, \ldots, \tau_n) \\
        \boldop{Var}(\tau_1) \cup \boldop{Var}(\tau_2),             &\varphi = \tau_1 \doteq \tau_2, \\
        \boldop{Free}(\psi),                                        &\varphi = \neg \psi, \\
        \boldop{Free}(\psi_1) \cup \boldop{Free}(\psi_2),           &\varphi = \psi_1 \bincirc \psi_2, \bincirc \in \Sigma, \\
        \boldop{Free}(\psi) \setminus \set{ \xi },                  &\varphi = \quantifier{Q}{\xi} \psi, Q \in \set{ \forall, \exists }
      \end{cases}
    \end{equation}

    \thmitem{def:first_order_syntax/ground_formula} A formula \( \varphi \) is called a \term{ground formula} if \( \boldop{Free}(\varphi) = \varnothing \). Ground formulas are also called \term{closed formulas} or \term{sentences} (unlike in propositional logic where all formulas are called sentences --- see \fullref{def:propositional_syntax/formula}).

    We will not restrict our attention only to closed formulas, and we will even rely on implicit quantification as mentioned in \fullref{rem:mathematical_logic_conventions/quantification}. That being said, certain important theorems like \fullref{thm:semantic_deduction_theorem} and \fullref{thm:syntactic_deduction_theorem} require some of the formulas to be closed, so we will often follow this restriction.

    \thmitem{def:first_order_syntax/formula_bound_variables} Dually, the \term{bound variables} of a formula are defined as
    \begin{equation}\label{eq:def:first_order_syntax/formula_bound_variables}
      \boldop{Bound}(\varphi) \coloneqq \begin{cases}
        \varnothing,                                        &\varphi \T{is atomic,} \\
        \boldop{Bound}(\psi),                               &\varphi = \neg \psi, \\
        \boldop{Bound}(\psi_1) \cup \boldop{Bound}(\psi_2), &\varphi = \psi_1 \bincirc \psi_2, \bincirc \in \Sigma, \\
        \boldop{Bound}(\psi) \cup \set{ \xi },              &\varphi = \quantifier{Q}{\xi} \psi, Q \in \set{ \forall, \exists }.
      \end{cases}
    \end{equation}

    \thmitem{def:first_order_syntax/formula_variables} Finally, the set of all variables of a formula \( \varphi \) is
    \begin{equation}\label{eq:def:first_order_syntax/formula_variables}
      \boldop{Var}(\varphi) \coloneqq \boldop{Free}(\varphi) \cup \boldop{Bound}(\varphi).
    \end{equation}
  \end{thmenum}
\end{definition}

\begin{proposition}\label{thm:first_order_terms_and_formulas_are_unambiguous}
  The grammars of \hyperref[def:first_order_syntax/term]{first-order terms} and of \hyperref[def:first_order_syntax/formula]{first-order formulas} are \hyperref[def:grammar_derivation/unambiguous]{unambiguous}.
\end{proposition}
\begin{proof}
  The proof is more complicated, but similar in spirit to \fullref{thm:propositional_formulas_are_unambiguous}.
\end{proof}

\begin{remark}\label{rem:first_order_formula_conventions}
  In order to simplify exposition, we use the following conventions
  \begin{thmenum}
    \thmitem{rem:first_order_formula_conventions/infix} Binary functional symbols are often written using \term{infix notation}, i.e.
    \begin{equation*}
      \zeta \doteq \xi + \eta
    \end{equation*}
    rather than the \term{prefix notation}
    \begin{equation*}
      \zeta \doteq +(\xi, \eta).
    \end{equation*}

    This also applies to predicates --- we write \( \xi \sim \eta \) rather than \( \sim(\xi, \eta) \).

    \thmitem{rem:first_order_formula_conventions/negation} Negation of an infix binary predicate symbol \( \sim \) can be written more simply as
    \begin{equation*}
      \xi \not\sim \eta
    \end{equation*}
    rather than
    \begin{equation*}
      \neg(\xi \sim \eta).
    \end{equation*}

    \thmitem{rem:first_order_formula_conventions/relativization}\mcite{LeanFOL} If \( \sim \) is a binary predicate, to further shorten notation, we write
    \begin{equation*}
      \qforall {\xi \sim \eta} \varphi
    \end{equation*}
    as a shorthand for
    \begin{equation*}
      \qforall \xi (\xi \sim \eta \rightarrow \varphi)
    \end{equation*}
    and
    \begin{equation*}
      \qexists {\xi \sim \eta} \varphi
    \end{equation*}
    as a shorthand for
    \begin{equation*}
      \qexists \xi (\xi \sim \eta \wedge \varphi).
    \end{equation*}

    This is called \term{relativization} of the quantifier and is immensely useful when working with heterogeneous objects or even in \hyperref[sec:set_theory]{set theory}.

    \thmitem{rem:first_order_formula_conventions/exists_unique} We sometimes want to specify not only existence, but also uniqueness. This is the case in \eqref{eq:def:zfc/choice}, for example. It is conventional to write
    \begin{equation*}
      \qExists \xi \varphi
    \end{equation*}
    as a shorthand for
    \begin{equation*}
      \qexists \xi \parens[\Big]{ \varphi \wedge \parens[\Big]{ \qforall \eta \varphi[\xi \mapsto \eta] \rightarrow (\xi \doteq \eta) } }.
    \end{equation*}

    \thmitem{rem:first_order_formula_conventions/necessary_signature} We only add to the language itself the functional and predicate symbols that are necessary for our desired axioms --- see \fullref{def:first_order_theory}. We can define additional functions and predicates in terms of these, but we avoid using them as much as possible when writing formulas in the object language. For example, we avoid adding the functional symbols \hyperref[thm:well_ordered_order_type_existence]{\( \ord(A) \)} and \hyperref[def:cardinal]{\( \card(A) \)} or even \hyperref[def:basic_set_operations/union]{\( \cup \)} and \hyperref[def:basic_set_operations/intersection]{\( \cap \)} to \hyperref[def:zfc]{\logic{ZFC}}.

    If needed, we can consider these new functions and predicates to be abbreviations for more verbose terms and formulas as described in \fullref{rem:predicate_formula}.
  \end{thmenum}

  As for \fullref{rem:propositional_formula_parentheses}, both of these conventions exist only in the metalanguage and the formulas themselves are assumed to have the former form within the object language.
\end{remark}

\begin{definition}\label{def:first_order_structure}\mcite[def. 16.1]{OpenLogicFull}
  Fix a first-order logic language \( \mscrL \). A \term{structure} for \( \mscrL \) is a pair \( \mscrX = (X, I) \), where
  \begin{thmenum}
    \thmitem{def:first_order_structure/set} \( X \) is a nonempty set called the \term{domain} or \term{universe} of the structure \( \mscrX \). See \fullref{rem:empty_models}.

    \thmitem{def:first_order_structure/interpretation} The \term{interpretation} \( I \) of the structure \( \mscrX \) is a \hyperref[def:function]{function} that is defined on the signature of \( \mscrL \) and satisfies the following conditions:
    \begin{thmenum}
      \thmitem{def:first_order_structure/interpretation/function} For every \( n \)-ary function symbol \( f \), its interpretation is a \hyperref[def:function]{function} with signature \( I(f): X^n \to X \).

      \thmitem{def:first_order_structure/interpretation/predicate} For every \( n \)-ary predicate \( p \), its interpretation is a an n-ary \hyperref[def:boolean_function]{Boolean-valued function} with signature \( I(p): X^n \to \set{ T, F } \). A tuple \( (x_1, \ldots, x_n) \) satisfies \( p \) if \( p(x_1, \ldots, x_n) = T \).

      It is conventional to define the interpretation of a predicate to be a \hyperref[def:relation]{relation} \( I(p) \subseteq X^n \) (see e.g. \mcite[def. 16.1]{OpenLogicFull}), however it is more convenient for us to work with Boolean-valued functions. The two approaches are equivalent as explained in \fullref{rem:boolean_valued_functions_and_predicates}.
    \end{thmenum}
  \end{thmenum}

  Unlike in the rest of this document, when dealing with first-order structures, it is important to distinguish between the structure \( \mscrX \) as a pair and its domain \( X \) as a set. See \fullref{rem:first_order_model_notation}.
\end{definition}

\begin{remark}\label{rem:empty_models}
   If we allow for the domain of a structure to be empty, we would have to reformulate a lot of important theorems (e.g. see the proof of \fullref{thm:renaming_assignment_compatibility/formulas}), which would complicate compatibility between semantics and \hyperref[def:deductive_system]{deductive systems}.

   See \fullref{thm:substructures_form_complete_lattice/bottom} for a context where empty sets are justified as domains of first-order structures.
\end{remark}

\begin{definition}\label{def:first_order_valuation}
  Fix a structure \( \mscrX = (X, I) \) for a first-order logic language \( \mscrL \).

  \begin{thmenum}
    \thmitem{def:first_order_valuation/variable_assignment}\mcite[def. 16.7]{OpenLogicFull} A \term{variable assignment} for the variables of \( \mscrL \) is any function \( v: \boldop{Var} \to X \) (loosely similar to \hyperref[def:propositional_valuation/interpretation]{propositional interpretations}).

    \thmitem{def:first_order_valuation/modified_assignment} For every variable \( \xi \) and every domain element \( x \in X \) we also define the \term{modified assignment} at \( \xi \) with \( x \):
    \begin{equation*}
      v_{\xi \mapsto x}(\zeta) \coloneqq \begin{cases}
        x,        &\zeta = \xi, \\
        v(\zeta), &\zeta \neq \xi.
      \end{cases}
    \end{equation*}

    We can also modify the value at \( \xi \) with another variable, e.g.
    \begin{equation*}
      v_{\xi \mapsto \eta}(\zeta) \coloneqq \begin{cases}
        v(\eta),  &\zeta = \xi, \\
        v(\zeta), &\zeta \neq \xi.
      \end{cases}
    \end{equation*}

    Inductively,
    \begin{equation*}
      v_{\xi_1 \mapsto x_1, \ldots, \xi_n \mapsto x_n}(\eta) \coloneqq ((\ldots(v_{\xi_1 \mapsto x_1})\ldots)_{\xi_n \mapsto x_n})(\eta).
    \end{equation*}

    Except for semantics of quantification, these are also used in other places like \fullref{thm:renaming_assignment_compatibility} and \fullref{rem:first_order_formula_valuation_without_variable_assignment}.

    \thmitem{def:first_order_valuation/term_valuation}\mcite[def. 16.8]{OpenLogicFull} The \term{valuation} of a term \( \tau \) is a value in the domain \( X \) given by
    \begin{equation}\label{eq:def:first_order_valuation/term_valuation}
      \tau\Bracks{v} \coloneqq \begin{cases}
        v(\xi),                                           &\tau = \xi \in \boldop{Var}, \\
        I(f)(\tau_1\Bracks{v}, \ldots, \tau_n\Bracks{v}), &\tau = f(\tau_1, \ldots, \tau_n).
      \end{cases}
    \end{equation}

    \thmitem{def:first_order_valuation/formula_valuation}\mcite[def. 16.11]{OpenLogicFull} We extend the classical propositional valuations from \fullref{def:propositional_valuation}. The (classical) \term{valuation} of a formula \( \varphi \) is a \hyperref[def:boolean_value]{Boolean value} given by
    \begin{equation}\label{eq:def:first_order_valuation/formula_valuation}
      \varphi\Bracks{v} \coloneqq \begin{cases}
        T,                                                              &\varphi = \top, \\
        F,                                                              &\varphi = \bot, \\
        \tau_1\Bracks{v} = \tau_2\Bracks{v},                            &\varphi = \tau_1 \doteq \tau_2, \\
        I(p)(\tau_1\Bracks{v}, \ldots, \tau_n\Bracks{v}),               &\varphi = p(\tau_1, \ldots, \tau_n), \\
        \overline{\psi\Bracks{v}},                                      &\varphi = \neg \psi, \\
        \psi_1\Bracks{v} \bincirc \psi_2\Bracks{v},                     &\varphi = \psi_1 \bincirc \psi_2, \bincirc \in \Sigma, \\
        \bigvee\set{ \psi\Bracks{v_{\xi \mapsto x}} \given x \in X },   &\varphi = \qforall \xi \psi, \\
        \bigwedge\set{ \psi\Bracks{v_{\xi \mapsto x}} \given x \in X }, &\varphi = \qexists \xi \psi.
      \end{cases}
    \end{equation}

    The rules for evaluating constants, negations and connectives are a direct extension of the \hyperref[def:propositional_valuation/formula_valuation]{rules for propositional logic}.

    It is important that the domain is nonempty because \( \bigwedge\varnothing = T \), which directly contradicts our intent of defining \( \exists \) as a quantifier for existence.
  \end{thmenum}
\end{definition}

\begin{remark}\label{rem:first_order_formula_valuation_without_variable_assignment}
  Somewhat similar to \fullref{rem:propositional_formula_valuation_without_variable_assignment}, if we know that \( \boldop{Free}(\varphi) \subseteq \{ \xi_1, \ldots, \xi_n \} \), we know that the \hyperref[def:first_order_valuation/formula_valuation]{valuation} \( \varphi\Bracks{v} \) only depends on the values \( v(\xi_1), \ldots, v(\xi_n) \). This allows us to introduce the shorthand
  \begin{equation}\label{eq:rem:first_order_formula_valuation_without_variable_assignment/long_semantic}
    \varphi\Bracks{\xi_1 \mapsto x_1, \ldots, \xi_n \mapsto x_n}
  \end{equation}
  or even
  \begin{equation}\label{eq:rem:first_order_formula_valuation_without_variable_assignment/short_semantic}
    \varphi\Bracks{x_1, \ldots, x_n}
  \end{equation}
  for
  \begin{equation*}
    \varphi\Bracks{v_{\xi_1 \mapsto x_1, \ldots, \xi_n \mapsto x_n}}
  \end{equation*}
  because the variable assignment \( v \) plays no role here.

  When using either of these shorthand notations, we implicitly assume that \( \boldop{Free}(\varphi) \subseteq \set{ \xi_1, \ldots, \xi_n } \).

  When \( \varphi = p(\xi_1, \ldots, \xi_n) \) is a predicate formula, the shorter notation \eqref{eq:rem:first_order_formula_valuation_without_variable_assignment/short_semantic} translates to
  \begin{equation*}
    p\Bracks{x_1, \ldots, x_n} = I(p)(x_1, \ldots, x_n) \in \set{ T, F }
  \end{equation*}
  and analogously for function terms we have
  \begin{equation*}
    f\Bracks{x_1, \ldots, x_n} = I(f)(x_1, \ldots, x_n) \in X.
  \end{equation*}

  Of course, we avoid this notation for formulas like \( p(f(\xi)) \) because \( p\Bracks{x} \) would mean \( I(p)(I(f)(x)) \) rather than \( I(p)(x) \), which would be confusing.

  We apply this notation for terms and, in particular, functions.

  We also sometimes use the shortened notation
  \begin{equation}\label{eq:rem:first_order_formula_valuation_without_variable_assignment/short_syntactic}
    \varphi[\tau_1, \ldots, \tau_n]
  \end{equation}
  for \hyperref[def:first_order_substitution/term_in_formula]{substituting terms in formulas}.
\end{remark}

\begin{definition}\label{def:first_order_equation}
  A \term{first-order equation} is a formula of the form
  \begin{equation}\label{eq:def:first_order_equation}
    f(\xi_1, \ldots, \xi_n) \doteq g(\xi_1, \ldots, \xi_n),
  \end{equation}
  where both \( f(\xi_1, \ldots, \xi_n) \) and \( g(\xi_1, \ldots, \xi_n) \) are functional symbols with the same free variables.

  Given a structure \( \mscrX = (X, I) \), we call the elements of the set defined by this formula \term{solutions}. That is, we say that the tuple \( (x_1, \ldots, x_n) \) is a, solution to the equation \eqref{eq:def:first_order_equation} if
  \begin{equation*}
    f\Bracks{x_1, \ldots, x_n} = g\Bracks{x_1, \ldots, x_n}.
  \end{equation*}

  We can actually replace \( f \) and \( g \) with more general terms \( \tau \) and \( \sigma \), in which case the equation becomes \( (\tau \doteq \sigma) \) and is satisfied if
  \begin{equation*}
    \tau\Bracks{x_1, \ldots, x_n} = \sigma\Bracks{x_1, \ldots, x_n}.
  \end{equation*}
\end{definition}

\begin{example}\label{ex:equations}
  A remarkable portion of mathematics concerns the study of different types of equations (even though they are not generally restricted to \hyperref[def:first_order_equation]{equations in first-order logic}). The reason for this is that equations provide a simple way to specify rich semantic structure using simple syntactic objects.

  \begin{itemize}
    \item Matrix theory can be regarded as the study of linear equations. See \fullref{subsec:matrices_over_rings} and \fullref{subsec:matrices_over_fields}.
    \item Differential equations is aptly named since it studies equations in functional spaces concerning functions and their derivatives. See \fullref{sec:diffeq}.
    \item Roots of generalized derivatives are studied in optimization. See \fullref{subsec:nonsmooth_derivatives}.
    \item Diophantine equations are studied in number theory. See \fullref{subsec:integers}.
    \item Fixed points of functions are studied in different branches of mathematics. See \fullref{thm:banach_fixed_point_theorem} or \fullref{thm:knaster_tarski_theorem}.
    \item Affine varieties, which are sets of roots of polynomials, are studied in algebraic geometry. See \fullref{subsec:quadratic_curves}.
  \end{itemize}
\end{example}

\begin{definition}\label{def:first_order_semantics}
  Fix a first-order logic language \( \mscrL \). We introduce notions analogous to \hyperref[def:propositional_semantics]{propositional semantics}:
  \begin{thmenum}
    \thmitem{def:first_order_semantics/satisfiability}\mcite[def. 16.11]{OpenLogicFull} Given a \hyperref[def:first_order_structure]{structure} \( \mscrX = (X, I) \), a \hyperref[def:first_order_valuation/variable_assignment]{variable assignment} \( v \) and a set \( \Gamma \) of \hyperref[def:first_order_syntax/formula]{first-order formulas}, we say that the variable assignment \( v \) \term{satisfies} \( \Gamma \) and we write \( \mscrX \vDash_v \Gamma \) if, for every formula \( \gamma \in \Gamma \) we have \( \gamma\Bracks{v} = T \).

    If every variable assignment in \( \mscrX \) satisfies \( \Gamma \), we say that \( \mscrX \) itself satisfies \( \Gamma \) or that \( \mscrX \) is a \term{model} of \( \Gamma \) and write \( \mscrX \vDash \Gamma \) (or simply \( \mscrX \vDash \Gamma \) if the interpretation is clear from the context).

    Analogously to \fullref{def:propositional_semantics/satisfiability}, we say that \( \Gamma \) is satisfiable if there exists a model for \( \Gamma \).

    \thmitem{def:first_order_semantics/entailment} We say that the set of formulas \( \Gamma \) \term{entails} the set of formulas \( \Delta \) and write \( \Gamma \vDash \Delta \) if every model of \( \Gamma \) is also a model of \( \Delta \).

    \thmitem{def:first_order_semantics/tautology} The formula \( \varphi \) is a \term{tautology} if every structure is a model of \( \varphi \).

    \thmitem{def:first_order_semantics/contradiction} Dually, \( \varphi \) is a \term{contradiction} is no structure is a model of \( \varphi \).

    \thmitem{def:first_order_semantics/equivalence} As in the simplest case with \hyperref[def:propositional_semantics/equivalence]{propositional semantical equivalence}, we say that \( \Gamma \) and \( \Delta \) are \term{semantically equivalent} and write \( \Gamma \gleichstark \Delta \) if both \( \Gamma \vDash \Delta \) and \( \Delta \vDash \Gamma \).

    \thmitem{def:first_order_semantics/equisatisfiability} Again as in the simplest case with \hyperref[def:propositional_semantics/equisatisfiability]{propositional equisatisfiability}, we say that the sets of formulas \( \Gamma \) and \( \Delta \) are \term{equisatisfiable} when it holds that \( \Gamma \) is satisfiable if and only if \( \Delta \) is satisfiable.

    \Fullref{thm:quantifier_satisfiability/existential} provides an important example of equisatisfiable formulas that are not equivalent.
  \end{thmenum}
\end{definition}

\begin{remark}\label{rem:propositional_logic_as_first_order_logic}
  It is now clear that the \hyperref[subsec:propositional_logic]{propositional logic language} can be regarded as a degenerate first-order logic language with at most countably many predicate symbols, all of arity \( 0 \), and no functional symbols. Thus, first-order logic is indeed an extension of propositional logic.
\end{remark}

\begin{remark}\label{rem:first_order_model_notation}
  In first-order logic, \hyperref[def:first_order_semantics/satisfiability]{models} are defined as pairs \( \mscrX = (X, I) \). Each area of mathematics has its own conventions and models are usually specified as simply as possible without being unambiguous (and sometimes even beyond non-ambiguity).

  A popular convention is to use compatible letters like we did with \( X \) and \( \mscrX \) or \( G \) and \( \mscrG \), where the structure itself is named using calligraphic letters while the domain is named using the corresponding capital letter in normal font. This only works very simple cases where we can say \enquote{Let \( \mscrP = (P, \leq) \) be a \hyperref[def:partially_ordered_set]{partially ordered set}}.

  The language of the \hyperref[def:group/theory]{theory of groups} has a signature consisting of three functional symbols and no predicate symbols. Specifying a structure for this language is thus the same as specifying a quadruple \( \mscrG = (G, e, (\anon)^{-1}, \cdot) \). We usually specify only the domain \( G \) and the basic structure needed to avoid ambiguity, for example \enquote{Let \( (G, \cdot) \) be a group}. This is technically wrong, but it is both convenient and conventional. The rest of the definition of the group can easily be inferred. In case of ambiguity, the simplest disambiguation is to use lower indices with the name of the structure, e.g. \( +_G \) and \( +_H \) may be the addition operation in different abelian groups.

  Furthermore, stating that \( (G, \cdot, \leq, \mscrT) \) is a totally ordered topological group is cumbersome and can even raise questions; for example, is \( \mscrT \) the \hyperref[def:order_topology]{order topology} or just an arbitrary \hyperref[rem:topological_first_order_structures]{group topology}?
\end{remark}

\subsection{First-order models}\label{subsec:first_order_models}

Much of \fullref{subsec:first_order_logic} is dedicated to semantic equivalences between logical formulas, which are formulated and proved using \hyperref[def:first_order_structure]{structures}. This section is dedicated to the study of structures themselves and relations between them. While model theory is a wide topic, for the purposes of this document we are only interested in the following questions:

\begin{itemize}
  \item Which subsets of a structure form a \hyperref[def:first_order_substructure]{substructure}?

  This is answered by \fullref{def:first_order_substructure} and by \fullref{def:first_order_generated_substructure}. Vacuously, if the language contains no functional symbols, by \fullref{thm:first_order_substructure_properties/no_functions} every subset of a structure is a substructure. Such is the case with \hyperref[def:set]{sets} themselves, with \hyperref[def:poset]{partially ordered sets} or with \hyperref[def:metric_space]{metric} and \hyperref[def:topological_space]{topological spaces}.

  \Fullref{thm:substructures_form_complete_lattice} shows that the set of all substructures of a structure is worth studying in itself.

  \item Given a model of some set \( \Gamma \) of formulas, which substructures and \hyperref[def:first_order_homomorphism]{homomorphic} images of the model are again models of \( \Gamma \)?

  This is answered by \fullref{thm:positive_formulas_preserved_under_homomorphism}, \fullref{thm:arbitrary_formulas_preserved_under_isomorphisms} and \fullref{thm:functions_over_model_form_model}.
\end{itemize}

\begin{definition}\label{def:first_order_substructure}
  Let \( \mscrX = (X, I) \) be a structure for the language \( \mscrL \) and let \( Y \subseteq X \). We say that \( \mscrY = (Y, I) \) is a \term{substructure} of \( \mscrX \) if it satisfies any of the following equivalent conditions:

  \begin{thmenum}
    \thmitem{def:first_order_substructure/deductive} If \( Y \) is closed under function application, that is, for any functional symbol \( f \) in \( \mscrL \) with arity \( n \), we have \( I(f)(Y^n) \subseteq Y \).

    \thmitem{def:first_order_substructure/inductive} \( \mscrY \) is a \hyperref[def:fixed_point]{fixed point} of the operator
    \begin{equation}\label{eq:def:first_order_substructure/inductive/operator}
      \begin{aligned}
        &T: \pow(X) \to \pow(X) \\
        &T(A) \coloneqq A \cup \set*{ x \in X \given \qexists{f \in \boldop{Fun}} \qexists{x_1, \ldots, x_{\#f} \in A} f\Bracks{x_1, \ldots, x_{\#f}} = x },
      \end{aligned}
    \end{equation}
    which enlarges \( A \) with the union of all image of \( A \) under functions of the language \( \mscrL \).

    Note that the formula inside \eqref{eq:def:first_order_substructure/inductive/operator} is in the metalanguage despite using syntax similar to first-order logic formulas.
  \end{thmenum}
\end{definition}
\begin{proof}
  By definition of \( T \), \( Y \) if a fixed point if and only if
  \begin{equation*}
    \set*{ x \in X \given \qexists{f \in \boldop{Fun}} \qexists{x_1, \ldots, x_{\#f} \in A} f\Bracks{x_1, \ldots, x_{\#f}} = x } \subseteq Y.
  \end{equation*}

  This condition is clearly satisfied if \( B \) satisfies \fullref{def:first_order_substructure/deductive}.

  If, instead \( Y \) is a fixed point of \( T \), for the \( n \)-ary functional symbol \( f \in \boldop{Fun} \) and for any tuple \( x_1, \ldots, x_n \), the value \( I(f)(x_1, \ldots, x_n) \) belongs to \( Y \). Therefore \fullref{def:first_order_substructure/deductive} is satisfied.
\end{proof}

\begin{example}\label{ex:def:first_order_substructure/vector_space}
  The classic definition for a subset \( U \) of a \hyperref[def:vector_space]{vector space} \( \mscrV \) being a vector subspace is that \( U \) is closed under \hyperref[def:linear_combination]{linear combinations}. Linear combinations are simply finite \hyperref[def:function/superposition]{superpositions} of addition and scalar multiplication in \( \mscrV \). So this condition ensures that \( U \) is closed under application of the functional symbols corresponding to addition and scalar multiplication.
\end{example}

\begin{proposition}\label{thm:first_order_substructure_properties}
  Fix a language \( \mscrL \). The \hyperref[def:first_order_substructure]{first order substructures} of a structure \( \mscrX = (X, I) \) have the following basic properties:
  \begin{thmenum}
    \thmitem{thm:first_order_substructure_properties/no_functions} If \( \mscrL \) has no functional symbols, then \( (Y, I) \) is a substructure where \( Y \) is any subset of \( X \).

    \thmitem{thm:first_order_substructure_properties/intersection} Let \( \{ (Y_k, I) \}_{k \in \mscrK} \) be a family of substructures of \( \mscrX \). Then their \term{intersection structure} \( \parens*{\bigcap_{k \in \mscrK} Y_k, I} \) is again a substructure of \( \mscrX \).
  \end{thmenum}
\end{proposition}
\begin{proof}
  \SubProofOf{thm:first_order_substructure_properties/no_functions} Both conditions in \fullref{def:first_order_substructure} are vacuously satisfied if there are no functional symbols in \( \mscrL \).

  \SubProofOf{thm:first_order_substructure_properties/intersection} For any functional symbol \( f \) in \( \mscrL \) with arity \( n \), we have
  \begin{equation*}
    I(f)\parens*{\parens*{\bigcap_{\smash{k \in \mscrK}} Y_k}^n}
    \reloset {\ref{thm:function_image_properties/intersection}} \subseteq
    \bigcap_{k \in \mscrK} I(f)(Y_k^n) \subseteq \bigcap_{k \in \mscrK} \mscrY_k.
  \end{equation*}

  Therefore \( \parens*{\bigcap_{k \in \mscrK} Y_k, I} \) is indeed a substructure of \( \mscrX \).
\end{proof}

\begin{definition}\label{def:first_order_generated_substructure}
  Let \( \mscrX = (X, I) \) be a structure over \( \mscrL \) and let \( A \subseteq X \) be any set. The set \( A \) is said to \term[bg=поражда,ru=порождает]{generate} or \term{induce} the substructure \( \mscrY = (Y, I) \) if it satisfies any of the equivalent statements:
  \begin{thmenum}
    \thmitem{def:first_order_generated_substructure/smallest} Out of all substructures of \( \mscrX \) whose domain contains \( A \), the domain \( \mscrY \) is the smallest with respect to \hyperref[def:subset]{set inclusion}.

    \thmitem{def:first_order_generated_substructure/intersection} \( \mscrY \) is the intersection structure of all substructures of \( \mscrX \) that contain \( A \).
  \end{thmenum}
\end{definition}
\begin{proof}
  Let \( \{ (Y_k, I) \}_{k \in \mscrK} \) be the family of all substructures of \( \mscrX \) whose domains contain \( A \). Fix one of these substructures, say \( (Y_{k_0}, I) \).

  We have the obvious inclusion
  \begin{equation*}
    \bigcap_{k \in \mscrK} Y_k \subseteq Y_{k_0}.
  \end{equation*}

  The reverse inclusion holds if and only if \( Y_{k_0} \) is contained in each one of domains \( Y_k \) for \( k \in \mscrK \). In other words, \( Y_{k_0} \) is the smallest of the domains \( \set{ Y_k }_{k \in \mscrK} \) with respect to set inclusion if and only if \( Y_{k_0} \) equals their intersection.
\end{proof}

\begin{example}\label{ex:def:first_order_generated_substructure}
  Common examples of generated substructures are the \hyperref[def:linear_span]{linear span} and the \hyperref[def:generated_ring_ideal]{generated ring ideals}.
\end{example}

\begin{remark}\label{rem:induction}
  \Fullref{def:first_order_substructure} highlights an important aspect of logic. It actually provides two definitions (that are not equivalent in general):
  \begin{itemize}
    \item \Fullref{def:first_order_substructure/deductive} defines a substructure of \( \mscrX = (X, I) \) to be \( \mscrY = (Y, I) \), where \( Y \) is any subset of \( X \) that satisfies certain conditions. Given a subset \( Y \) of \( X \), we this definition allows us to deduce if \( (Y, I) \) is a substructure or not.

    We say that this definition is \term{deductive}. \Fullref{thm:semantic_deduction_theorem} and \fullref{thm:syntactic_deduction_theorem} allow us to connect deduction in the object language with deduction in the metalanguage so we can mechanize this substructure validation by introducing it as a unary predicate symbol in the object language.

    \item \Fullref{def:first_order_substructure/inductive} instead provides a procedure for building a substructure given a subset of the domain \( X \) of \( \mscrX \) by having explicit rules for which additional related elements from \( X \) we need to add in order to obtain the corresponding generated substructure --- see \fullref{def:first_order_generated_substructure} and \fullref{ex:def:first_order_generated_substructure}.

    We say that this definition is \term{inductive}. Rather than using it for verifying whether a subset forms a substructure or not, we can instead use it for generating substructures. In \hyperref[def:zfc]{ZFC}, we can even add a unary function to the object language for obtaining the generated substructure of a set (which set in this case will be an individual member of a model of ZFC rather than a subset of the model).
  \end{itemize}

  The ambient domain \( X \) used for the inductive definition only guarantees that the result is well-defined but the operator \eqref{eq:def:first_order_substitution/term_in_formula/quantifiers/trivial} is not the only one that is useful for definitions. In fact, any monotone operator on the \hyperref[thm:boolean_algebra_of_subsets]{Boolean algebra of subsets} of \( X \) has a least fixed point by \fullref{thm:knaster_tarski_theorem}.

  For example, provided that \( X \) is a large enough to be a superset of the \hyperref[def:smallest_inductive_set]{smallest inductive set \( \omega \)}, the set \( \omega \) itself can be defined as the smallest fixed point of the operator
  \begin{equation*}
    T(A) \coloneqq \set{ \varnothing } \cup \set{ S(x) \given x \in A },
  \end{equation*}
  where \( S \) is the \hyperref[def:ordinal_successor_operator]{ordinal successor operator}.

  This general procedure for constructing in stages an object that satisfies certain conditions is called \term{structural induction} or simply \term{induction}. The number of steps required by the procedure may not be finite (as in the example with \( \omega \)) or may not even be countable, which further encourages us to think of induction as a general procedure rather than an algorithm that works in discrete steps.

  Rather than constructing objects, we can also use inductive reasoning for proofs. In this context, the procedure for constructing objects is sometimes called recursion. We introduce specific axiom schemas representing induction --- \ref{def:peano_arithmetic/PA3} on \hyperref[def:set_of_natural_numbers]{natural numbers}, \fullref{thm:well_ordered_induction} on \hyperref[def:well_ordered_set]{well-ordered sets} and \fullref{thm:transfinite_induction} on \hyperref[def:ordinal]{ordinals}. These are formulated entirely within the object language and are, within the \hyperref[def:first_order_axiomatic_derivation_system]{first-order axiomatic derivation system}, a useful way to introduce universal quantification where deductive derivations are more difficult or even nonexistent.
\end{remark}

\begin{proposition}\label{thm:substructures_form_complete_lattice}
  Fix a structure \( \mscrX = (X, I) \) for the language \( \mscrL \).

  The set of all substructures of a \( \mscrX \) forms a complete lattice with respect to set inclusion of domains. It is isomorphic to a complete \hyperref[def:semilattice/submodel]{sublattice} of the Boolean algebra \( \pow(X) \) described in \fullref{thm:boolean_algebra_of_subsets}.

  Explicitly:
  \begin{thmenum}
    \thmitem{thm:substructures_form_complete_lattice/join} The \hyperref[def:semilattice/join]{join} of the family of substructures with domains \( \set{ (Y_k, I) }_{k \in \mscrK} \) is the \hyperref[def:first_order_generated_substructure]{generated substructure} of the set \( \bigcup_{k \in \mscrK} Y_k \).

    \thmitem{thm:substructures_form_complete_lattice/top} The \hyperref[def:poset_extremal_points/top_and_bottom]{top element} is the substructure \( \mscrX \) itself. Any substructures that are different from \( \mscrX \) are called \term{proper}.

    \thmitem{thm:substructures_form_complete_lattice/meet} The \hyperref[def:semilattice/meet]{meet} of the family of substructures \( \set{ (Y_k, I) }_{k \in \mscrK} \) is simply the \hyperref[thm:first_order_substructure_properties/intersection]{intersection structure} of the family.

    \thmitem{thm:substructures_form_complete_lattice/bottom} The \hyperref[def:poset_extremal_points/top_and_bottom]{bottom element} of this lattice is the intersection of all substructures. This is called the \term{trivial substructure}. As a matter of fact, because the trivial substructures for any two structures are isomorphic, we refer to them collectively as the \term{trivial structure} because trivial substructures tend to be isomorphic.

    Note that the empty set is not allowed to be the domain of a structure by definition (see \fullref{rem:empty_models}). Nevertheless, for the sake of having a bottom element we allow structures with empty domains in this lattice.

    The trivial substructure usually (but not necessarily) consists only of the constants of \( \mscrL \) --- for example, the \hyperref[def:group/trivial]{trivial group} \( \set{ e } \) or the \hyperref[def:semilattice/trivial]{trivial bounded lattice} \( \set{ \top, \bot } \).
  \end{thmenum}
\end{proposition}
\begin{proof}
  \SubProofOf{thm:substructures_form_complete_lattice/join} Let \( (Y, I) \) be the generated substructure of the set \( A \coloneqq \bigcup_{k \in \mscrK} \mscrY_k \). By \fullref{def:first_order_generated_substructure/smallest}, out of the domains of all substructures of \( \mscrX \), \( Y \) is the smallest that contains \( A \) and hence the smallest that contains \( Y_k \) for all \( k \in \mscrK \). Therefore it is indeed the supremum of the family \( \set{ (Y_k, I) }_{k \in \mscrK} \) with respect to set inclusion of domains.

  \SubProofOf{thm:substructures_form_complete_lattice/top} Since \( \mscrX \) is a substructure of itself, it is not only the supremum of the entire lattice but actually the maximum.

  \SubProofOf{thm:substructures_form_complete_lattice/meet} The domain of the intersection structure of the family \( \set{ (Y_k, I) }_{k \in \mscrK} \) of substructures of \( \mscrX \) is obviously the infimum of the family since its domain are the infimum in the Boolean algebra \( \pow(X) \) (see \fullref{thm:boolean_algebra_of_subsets}).

  \SubProofOf{thm:substructures_form_complete_lattice/bottom} It trivially follows from \fullref{thm:substructures_form_complete_lattice/meet} that the bottom element is the intersection of all substructures of \( \mscrX \).
\end{proof}

\begin{definition}\label{def:first_order_homomorphism}
  Let \( \mscrX = (X, I) \) and \( \mscrY = (Y, I) \) be structures over a common language. We say that the \hyperref[def:function]{function} \( h: X \to Y \) is a \term{homomorphism} between \( \mscrX \) and \( \mscrY \) if it preserves all functions and relations. Explicitly:
  \begin{thmenum}
    \thmitem{def:first_order_homomorphism/functions} For any functional symbol \( f \in \boldop{Fun} \) of arity \( n \) and any tuple \( x_1, \ldots, x_n \in X \) we have
    \begin{equation*}
      h\parens[\Big]{ I_\mscrX(f)(x_1, \ldots, x_n) } = I_\mscrY(f) \parens[\Big]{ h(x_1), \ldots, h(x_n) }
    \end{equation*}

    \thmitem{def:first_order_homomorphism/predicates} For any predicate symbol \( p \in \boldop{Pred} \) of arity \( n \) and any \( x_1, \ldots, x_n \in X \),
    \begin{equation*}
      I_\mscrX(p) (x_1, \ldots, x_n) = T \T{implies} I_\mscrY(p) \parens[\Big]{ h(x_1), \ldots, h(x_n) } = T.
    \end{equation*}
  \end{thmenum}
\end{definition}

\begin{remark}\label{rem:first_order_strong_homomorphism}
  Homomorphisms as they are defined in \fullref{def:first_order_homomorphism} are sometimes called \term{weak homomorphisms}. Under weak homomorphisms, it is possible that \( I_\mscrX(p) (x_1, \ldots, x_n) = F \) and yet \( I_\mscrY(p) \parens[\Big]{ h(x_1), \ldots, h(x_n) } = T \).

  Logicians sometimes define \term{strong homomorphisms} where they replace \fullref{def:first_order_homomorphism/predicates} with the stronger condition
  \begin{equation*}
    I_\mscrX(p) (x_1, \ldots, x_n) = I_\mscrY(p) \parens[\Big]{ h(x_1), \ldots, h(x_n) }.
  \end{equation*}

  This condition seems much more natural at first but it is less useful in practice. For example, \hyperref[def:preordered_set/homomorphism]{order homomorphisms} and \hyperref[def:graph_homomorphism]{graph homomorphisms} are both weak homomorphisms and these are the most used definitions of homomorphisms in languages with predicate symbols. For this reason, we avoid studying strong homomorphisms and only mention them in a few remarks.
\end{remark}

\begin{proposition}\label{thm:first_order_homomorphism_properties}
  \hyperref[def:first_order_homomorphism]{First-order structure homomorphisms} have the following basic properties:
  \begin{thmenum}
    \thmitem{thm:first_order_homomorphism_properties/substructure} If \( \mscrX = (X, I) \) is a structure and \( \mscrY = (Y, I) \) is a \hyperref[def:first_order_substructure]{substructure} of \( \mscrX \), then the \term{canonical embedding} function
    \begin{equation}\label{thm:first_order_homomorphism_properties/substructure/canonical_embedding}
      \begin{aligned}
        &\iota: Y \to X \\
        &\iota(y) \coloneqq y
      \end{aligned}
    \end{equation}
    is indeed a \hyperref[def:first_order_homomorphism_invertibility/projection]{homomorphism} (and thus an embedding in the sense of \fullref{def:first_order_homomorphism_invertibility}).

    \thmitem{thm:first_order_homomorphism_properties/preserves_substructures} If \( \mscrX = (X, I_\mscrX) \) and \( \mscrY = (Y, I_\mscrY) \) are structures and \( h: X \to Y \) is a (weak) homomorphism, then the \hyperref[def:function/image]{image} \( h(\mscrX) \coloneqq (h(X), I_\mscrY) \) is a substructure of \( \mscrY \).

    \thmitem{thm:first_order_homomorphism_properties/composition} The \hyperref[def:multi_valued_function/composition]{composition} of two homomorphisms is again a homomorphism.

    \thmitem{thm:first_order_homomorphism_properties/term_valuation} Fix a homomorphism \( h: X \to Y \) and a term \( \tau \). For any variable assignments \( v_\mscrX \) and \( v_\mscrY \) such that \( v_\mscrY(\xi) = h(v_\mscrX(\xi)) \) for all \( \xi \in \boldop{Var}(\tau) \), we have
    \begin{equation*}
      h(\tau\Bracks{v_\mscrX}) = \tau\Bracks{v_\mscrY}.
    \end{equation*}
  \end{thmenum}
\end{proposition}
\begin{proof}
  \SubProofOf{thm:first_order_homomorphism_properties/substructure} The interpretation in the substructure \( \mscrY \) is the \hyperref[def:multi_valued_function/restriction]{restriction} \( I\restr_\mscrY \) of \( I \) to \( \mscrY \) which simply restricts the domain of any predicate and function is indeed an interpretation in \( \mscrY \). Thus \( (Y, I\restr_\mscrY) \) is a structure.

  Conditions \fullref{def:first_order_homomorphism/functions} and \fullref{def:first_order_homomorphism/predicates} are both satisfied since the interpretation of any function and predicate is restricted to \( \mscrY \). Thus \( \iota \) is a homomorphism.

  \SubProofOf{thm:first_order_homomorphism_properties/preserves_substructures} To prove that \( (f(X), I_\mscrY) \) is a substructure of \( \mscrY \), we will show that \fullref{def:first_order_substructure/deductive} holds.

  Indeed, by \fullref{def:first_order_homomorphism/functions}, for any \( n \)-ary functional symbol and any tuple \( {x_1, \ldots, x_n \in X} \), we have that
  \begin{equation*}
    I_\mscrY(f) \parens[\Big]{ h(x_1), \ldots, f(x_n) }
    \reloset {\ref{def:first_order_homomorphism/functions}} =
    h\parens[\Big]{ I_\mscrX(f)(x_1, \ldots, x_n) }
    \reloset {\ref{def:first_order_substructure/deductive}} \in
    h(X).
  \end{equation*}

  \SubProofOf{thm:first_order_homomorphism_properties/composition} Let \( h: \mscrX \mapsto \mscrY \) and \( l: \mscrY \mapsto \mscrZ \) both be homomorphisms.

  \begin{itemize}
    \item \Fullref{def:first_order_homomorphism/functions} is satisfied because for any \( n \)-ary functional symbol \( f \) and any tuple \( x_1, \ldots, x_n \in X \),
    \begin{align*}
      &\phantom{{}={}}
      (l \bincirc h) \parens[\Big]{ I_\mscrX(f)(x_1, \ldots, x_n) }
      \reloset {\ref{def:first_order_homomorphism/functions}} = \\ &=
      l\parens[\Big]{ I_\mscrY(f) \parens[\Big]{ h(x_1), \ldots, h(x_n) } }
      \reloset {\ref{def:first_order_homomorphism/functions}} = \\ &=
      I_{\mscrZ}(f) \parens[\Big]{ (l \bincirc h)(x_1), \ldots, (l \bincirc h)(x_n) }.
    \end{align*}

    \item \Fullref{def:first_order_homomorphism/predicates} is satisfied because for any \( n \)-ary predicate symbol \( p \) and any tuple \( x_1, \ldots, x_n \in X \),
    \begin{align*}
      &\phantom{{}={}}
      I_\mscrX(p) (x_1, \ldots, x_n)
      \reloset {\ref{def:first_order_homomorphism/predicates}} = \\ &=
      I_\mscrY(p) \parens[\Big]{ h(x_1), \ldots, h(x_n) }
      \reloset {\ref{def:first_order_homomorphism/predicates}} = \\ &=
      I_{\mscrZ}(p) \parens[\Big]{ (l \bincirc h)(x_1), \ldots, (l \bincirc h)(x_n) }.
    \end{align*}
  \end{itemize}

  \SubProofOf{thm:first_order_homomorphism_properties/term_valuation} We use induction on the structure of \( \tau \). If \( \tau \) is a variable, the statement is obvious from the compatibility condition for \( v_\mscrX \) and \( v_\mscrY \). If \( \tau = f(\kappa_1, \ldots, \kappa_m) \), then
  \begin{align*}
    \tau\Bracks{v_\mscrY}
    &=
    I(f) \parens[\Big]{ \kappa_1\Bracks{v_\mscrY}, \ldots, \kappa_m\Bracks{v_\mscrY} }
    = \\ &=
    I(f) \parens[\Big]{ h(\kappa_1\Bracks{v_\mscrX}), \ldots, h(\kappa_m\Bracks{v_\mscrX}) }
    \reloset {\ref{def:first_order_homomorphism/functions}} = \\ &=
    h\parens[\Big]{ I(f) \parens[\Big]{ \kappa_1\Bracks{v_\mscrX}, \ldots, \kappa_m\Bracks{v_\mscrX} } }
    = \\ &=
    h(\tau\Bracks{v_\mscrX}).
  \end{align*}
\end{proof}

\begin{definition}\label{def:first_order_homomorphism_invertibility}
  In connection with \fullref{def:function_invertibility} and \fullref{def:morphism_invertibility}, we introduce the following terminology for homomorphisms:
  \begin{thmenum}
    \thmitem{def:first_order_homomorphism_invertibility/embedding} An \term{embedding}, also called a \term{monomorphism}, is an \hyperref[def:function_invertibility/injection]{injective} homomorphism. We sometimes use the categorical notation \( f: A \hookrightarrow B \).

    \thmitem{def:first_order_homomorphism_invertibility/projection} Dually, a \term{projection}, also called an \term{epimorphism}, is an \hyperref[def:function_invertibility/surjection]{surjective} homomorphism. We sometimes use the categorical notation \( f: A \twoheadrightarrow B \).

    \thmitem{def:first_order_homomorphism_invertibility/isomorphism} An \term{isomorphism} is a \hyperref[def:function_invertibility/bijection]{bijective} homomorphism.

    \thmitem{def:first_order_homomorphism_invertibility/endomorphism} An \term{endomorphism} is a homomorphism that is also an \hyperref[def:multi_valued_function/endofunction]{endofunction}.

    \thmitem{def:first_order_homomorphism_invertibility/automorphism} A homomorphism that is both an endomorphism and an isomorphism is called an \term{automorphism}.
  \end{thmenum}
\end{definition}

\begin{definition}\label{def:positive_formula}
  We say that a \hyperref[def:propositional_syntax/formula]{propositional formula} \( \varphi \) is \term{positive} if it contains only \hyperref[def:conjunctive_disjunctive_normal_form/literal]{positive literals} and the propositional constants \hyperref[def:propositional_language/constants/verum]{top \( \top \)} and \hyperref[def:propositional_language/constants/falsum]{bottom \( \bot \)} connected using \hyperref[def:propositional_language/connectives/conjunction]{conjunction \( \wedge \)} and \hyperref[def:propositional_language/connectives/disjunction]{disjunction \( \vee \)}.

  The point of positive formulas is to avoid \hyperref[def:propositional_language/negation]{negation \( \neg \)}. This definition is not equivalent to \hyperref[def:positive_implicational_propositional_derivation_system]{positive implicational formulas} where \( \rightarrow \) is the only connective. We avoid adding \( \rightarrow \) because that would allow us, assuming classical logic, to derive negation using \fullref{thm:boolean_equivalences/negation_bottom}.

  Positive formulas are used in \fullref{thm:positive_formulas_preserved_under_homomorphism}, which fails to hold for some non-positive formulas (see \fullref{ex:monoid_cancellation_not_preserved_by_homomorphism}).

  When dealing with first-order logic, we simply use \hyperref[thm:first_order_substitution_equivalence/propositional]{substitution} to replace propositional variables with atomic formulas. This way we obtain positive first-order formulas with \hyperref[thm:semantic_implicit_universal_quantification]{implicit universal quantification}. Of course, we can always add explicit universal quantifiers but we avoid existential quantifiers because of \fullref{thm:first_order_quantifiers_are_dual}.
\end{definition}

\begin{proposition}\label{thm:positive_formulas_preserved_under_homomorphism}
  Let \( \mscrX = (X, I) \) and \( \mscrY = (Y, I) \) be structures over a common language \( \mscrL \) and let \( h: X \to Y \) be a \hyperref[def:first_order_homomorphism]{homomorphism} between them. Take \( \Gamma \) to be a nonempty set of \hyperref[def:positive_formula]{positive formulas}.

  Then \( h \) preserves models of \( \Gamma \). That is, if \( \mscrX \vDash \Gamma \) then \( (h(X), I_\mscrY) \vDash \Gamma \).
\end{proposition}
\begin{proof}
  Let \( v \) a variable assignment in \( \mscrY \). Let \( v_\mscrX: \boldop{Var} \to X \) be an assignment such that for any variable \( \xi \) we have
  \begin{equation*}
    v_\mscrX(\xi) \in h^{-1}(v_\mscrY(\xi)).
  \end{equation*}

  At least one such assignment exists by \fullref{def:zfc/A8}. If \( h \) is injective, this assignment is unique.

  We will show that
  \begin{equation}\label{thm:positive_formulas_preserved_under_homomorphism/ind_hyp_x}
    \mscrX \vDash_{v_\mscrX} \varphi
  \end{equation}
  implies
  \begin{equation}\label{thm:positive_formulas_preserved_under_homomorphism/ind_hyp_y}
    (h(X), I_\mscrY) \vDash_{v_\mscrY} \varphi.
  \end{equation}

  We assume \eqref{thm:positive_formulas_preserved_under_homomorphism/ind_hyp_x} for \( \varphi \) and we use induction on the structure of \( \varphi \) to prove \eqref{thm:positive_formulas_preserved_under_homomorphism/ind_hyp_y}, starting with different \hyperref[def:first_order_syntax/atomic_formula]{atomic formulas}:
  \begin{itemize}
    \item The constant \( \top \) is vacuously preserved by homomorphisms because it does not depend on the interpretation or variable assignment.

    \item Suppose that \( \varphi = (\tau_1 \doteq \tau_2) \). We have \( \tau_1\Bracks{v_\mscrX} = \tau_2\Bracks{v_\mscrX} \) and hence \( h(\tau_1\Bracks{v_\mscrX}) = h(\tau_2\Bracks{v_\mscrX}) \) and
    \begin{equation*}
      \tau_1\Bracks{v_\mscrY}
      \reloset {\ref{thm:first_order_homomorphism_properties/term_valuation}} =
      h(\tau_1\Bracks{v_\mscrX})
      =
      h(\tau_2\Bracks{v_\mscrX})
      \reloset {\ref{thm:first_order_homomorphism_properties/term_valuation}} =
      \tau_2\Bracks{v_\mscrY}.
    \end{equation*}

    \item Suppose that \( \varphi \) is the predicate formula \( p(\tau_1, \ldots, \tau_n) \). By assumption for every variable assignment in \( \mscrX \) and, in particular, for any \( v_\mscrX \),
    \begin{equation*}
      \mscrX \vDash_{v_\mscrX} p(\tau_1, \ldots, \tau_n),
    \end{equation*}
    then
    \begin{equation}\label{eq:thm:positive_formulas_preserved_under_homomorphism/predicates/x}
      I_\mscrX(p) \parens[\Big]{ \tau_1\Bracks{v_\mscrX}, \ldots, \tau_n\Bracks{v_\mscrX} } = T.
    \end{equation}

    By definition of homomorphism, we have
    \begin{equation}\label{eq:thm:positive_formulas_preserved_under_homomorphism/predicates/y}
      I_\mscrY(p) \parens[\Big]{ h(\tau_1\Bracks{v_\mscrX}), \ldots, h(\tau_n\Bracks{v_\mscrX}) } = T.
    \end{equation}

    Now
    \begin{equation*}
      (h(X), I_\mscrY) \vDash_{v_\mscrY} p(\tau_1, \ldots, \tau_n),
    \end{equation*}
    follows from \fullref{thm:first_order_homomorphism_properties/term_valuation}.

    If \( h \) is a \hyperref[rem:first_order_strong_homomorphism]{strong homomorphism}, then the converse also holds, i.e. \eqref{eq:thm:positive_formulas_preserved_under_homomorphism/predicates/x} follows from \eqref{eq:thm:positive_formulas_preserved_under_homomorphism/predicates/y}. See \fullref{thm:arbitrary_formulas_preserved_under_isomorphisms} for an application of this converse.

    \item Suppose that \( \varphi = \psi_1 \wedge \psi_2 \) and that the inductive hypothesis holds for \( \psi_1 \) and \( \psi_2 \).

    Since \( \varphi\Bracks{v_\mscrX} = T \) by assumption, by definition of valuation of conjunction we have
    \begin{equation*}
      \psi_1\Bracks{v_\mscrX}
      =
      \psi_2\Bracks{v_\mscrX}
      =
      T.
    \end{equation*}

    This allows us to apply the inductive hypothesis to obtain
    \begin{equation*}
      \psi_1\Bracks{v_\mscrY}
      =
      \psi_2\Bracks{v_\mscrY}
      =
      T.
    \end{equation*}
    and conclude that
    \begin{equation*}
      \varphi\Bracks{v_\mscrY}
      =
      \psi_1\Bracks{v_\mscrY} \wedge \psi_2\Bracks{v_\mscrY}
      =
      T \wedge T
      =
      T.
    \end{equation*}

    \item Suppose that \( \varphi = \psi_1 \vee \psi_2 \) and that the inductive hypothesis holds for \( \psi_1 \) and \( \psi_2 \).

    Since the formula \( \varphi \) is valid in \( \mscrX \), at least one of \( \psi_1 \) or \( \psi_2 \) is valid under \( v_\mscrX \). For different \( v_\mscrX \) the valuation pair \( (\psi_1\Bracks{v_\mscrX}, \psi_2\Bracks{v_\mscrX}) \) may be different but will always have at least one \( T \) value.

    The inductive hypothesis holds for both \( \psi_1 \) and \( \psi_2 \) and therefore \( (\psi_1\Bracks{v_\mscrY}, \psi_2\Bracks{v_\mscrY}) \) also contains at least one \( T \) value.

    This allows us to conclude that
    \begin{equation*}
      \varphi\Bracks{v_\mscrY}
      =
      \psi_1\Bracks{v_\mscrY} \vee \psi_2\Bracks{v_\mscrY}
      =
      T.
    \end{equation*}

    \item To see how this proof fails for conditionals, consider \( \varphi = (\psi_1 \rightarrow \psi_2) \). Then \( \varphi\Bracks{v_\mscrX} = T \) implies either \( \psi_1\Bracks{v_\mscrX} = F \) or \( \psi_1\Bracks{v_\mscrX} = \psi_2\Bracks{v_\mscrX} = T \).

    If \( \psi_1\Bracks{v_\mscrX} = \psi_2\Bracks{v_\mscrX} = F \), we have \( \varphi\Bracks{v_\mscrX} = T \) but we cannot conclude that \( \varphi\Bracks{v_\mscrY} = T \) because that would require the \hyperref[def:material_implication/inverse]{inverse} of the inductive hypothesis to hold for \( \psi_1 \) and \( \psi_2 \).

    See \fullref{ex:monoid_cancellation_not_preserved_by_homomorphism} for an example where a conditional is not preserved by a homomorphism.
  \end{itemize}

  Since \( v \) was chosen arbitrarily, we conclude that
  \begin{equation*}
    (h(X), I_\mscrY) \vDash \varphi.
  \end{equation*}
\end{proof}

\begin{corollary}\label{thm:substructure_is_model}
  If \( \Gamma \) is a set of positive formulas, any \hyperref[def:first_order_substructure]{substructure} of a model of \( \Gamma \) is again a model of \( \Gamma \).
\end{corollary}
\begin{proof}
  Follows from \fullref{thm:first_order_homomorphism_properties/substructure} and \fullref{thm:positive_formulas_preserved_under_homomorphism}.
\end{proof}

\begin{proposition}\label{thm:arbitrary_formulas_preserved_under_isomorphisms}\mcite[thm. 25.9]{OpenLogicFull}
  If \( \mscrX = (X, I_\mscrX) \) is a model of \( \Gamma \) and if \( h: X \to Y \) is a embedding from \( \mscrX \) to \( \mscrY = (Y, I_\mscrY) \), then \( (h(X), I_\mscrY) \) is also a model of \( \Gamma \).

  We say that embeddings preserve arbitrary formulas.

  If \( (h(X), I_\mscrY) \) is a model of \( \Gamma \), then \( \mscrX \) is also a model if \( h \) is a \hyperref[rem:first_order_strong_homomorphism]{strong homomorphism}.

  We say that strong embeddings reflect arbitrary formulas.
\end{proposition}
\begin{proof}
  The proof simply extends the induction in the proof of \fullref{thm:arbitrary_formulas_preserved_under_isomorphisms} to
  \begin{equation*}
    \varphi\Bracks{v_\mscrY} = \varphi\Bracks{v_\mscrY}
  \end{equation*}
  which allows us to use the usual induction on the negation and all connectives and quantifiers.

  The result regarding strong homomorphisms is shown in the note about \eqref{eq:thm:positive_formulas_preserved_under_homomorphism/predicates/x} following from \eqref{eq:thm:positive_formulas_preserved_under_homomorphism/predicates/y} under strong homomorphisms.
\end{proof}

\begin{proposition}\label{thm:functions_over_model_form_model}
  Let \( \Gamma \) to be a nonempty set of \hyperref[def:positive_formula]{positive formulas}. Let \( \mscrX \) be a model of \( \Gamma \) and let \( S \) be any nonempty set. Consider the set \( Y \coloneqq \fun(S, \mscrX) \) of \hyperref[def:function]{all set functions} from \( S \) to \( X \).

  Define the function \( \iota: X \mapsto Y \) by sending each \( x \in X \) to the corresponding constant function in \( Y \).

  Define the interpretation \( I_\mscrY \) as follows:
  \begin{itemize}
    \item For each \( n \)-ary functional symbol \( f \) in \( \mscrL \), define the interpretation of the functions \( k_1, \ldots, k_n \) componentwise as
    \begin{equation*}
      \begin{aligned}
        &I_\mscrY(f): Y^n \to Y \\
        &I_\mscrY(f) \parens[\Big]{ k_1, \ldots, k_n } \coloneqq \parens[\Big]{ s \mapsto I(f) \parens[\Big]{ k_1(s), \ldots, k_n(s) } }.
      \end{aligned}
    \end{equation*}

    \item For each \( n \)-ary predicate symbol \( p \) in \( \mscrL \), define \( I_\mscrY(p) \subseteq Y^n \) via
    \begin{equation*}
      \begin{aligned}
        &I_\mscrY(p): Y^n \to \set{ T, F } \\
        &I_\mscrY(p) \parens[\Big]{ k_1, \ldots, k_n } \coloneqq \bigwedge \set[\Big]{ I(p) \parens[\Big]{ k_1(s), \ldots, k_n(s) } \given* s \in S }.
      \end{aligned}
    \end{equation*}

    That is, \( I_\mscrY(p) (k_1, \ldots, k_n) = T \) if and only if \( I(p) (k_1(s), \ldots, k_n(s)) = T \) simultaneously for all \( s \in S \).
  \end{itemize}

  Then the structure \( \mscrY = (Y, I_\mscrY) \) is also a model of \( \Gamma \) and \( \iota: \mscrX \to \mscrY \) is an embedding.
\end{proposition}
\begin{proof}
  It is obvious that \( \mscrY \) is a structure and that \( \iota \) is an embedding. We will prove using induction on the structure of a formula \( \varphi \) that \( \mscrX \vDash \varphi \) implies \( \mscrY \vDash \varphi \).

  Let \( v_\mscrY \) be a variable assignment in \( \mscrY \).

  Suppose that \( \mscrX \vDash \varphi \). We use induction on the structure of \( \varphi \) to show that \( \varphi\Bracks{v_\mscrY} = T \).
  \begin{itemize}
    \item If \( \varphi \) is a propositional constant, its value does not depend on \( v_\mscrY \) and thus \( \varphi\Bracks{v_\mscrY} = \varphi\Bracks{v_\mscrX} \).

    \item If \( \varphi = (\tau_1 \doteq \tau_2) \), then \( \tau_1\Bracks{v_\mscrX} = \tau_2\Bracks{v_\mscrX} \) for all assignments \( v_\mscrX \) in \( \mscrX \), hence for any \( s \in S \) we have \( \parens[\Big]{\tau_1\Bracks{v_\mscrY}}(s) = \parens[\Big]{\tau_2\Bracks{v_\mscrY}}(s) \) since both sides of the equality here are elements of \( \mscrX \).

    \item Analogously, if \( \varphi = p(\tau_1, \ldots, \tau_n) \), then
    \begin{equation*}
      I_\mscrY(p) \parens[\Big]{ k_1, \ldots, k_n }
      =
      \bigwedge \set[\Big]{ I(p) \parens[\Big]{ k_1(s), \ldots, k_n(s) } \given* s \in S }
      =
      \bigwedge \set{ T \given s \in S }
      =
      T.
    \end{equation*}

    \item Analogous to the proof of \fullref{thm:positive_formulas_preserved_under_homomorphism}, conjunction and disjunction formulas that are valid in \( \mscrX \) are valid in \( \mscrY \) while conditional formulas may fail to be valid. See \fullref{ex:thm:functions_over_model_of_positive_formulas_form_model} for examples where this proposition fails.
  \end{itemize}
\end{proof}

\begin{example}\label{ex:thm:functions_over_model_of_positive_formulas_form_model}
  While the statement of \fullref{thm:functions_over_model_form_model} may be a little cryptic, a few examples show that it is actually obvious.
  \begin{itemize}
    \item \hyperref[def:boolean_function]{Boolean functions} have their values in the Boolean algebra \( \set{ T, F } \). Let \( S \) be the set of all tuples of values in \( \set{ T, F }^n \) for arbitrary \( n \). That is,
    \begin{equation*}
      S \coloneqq \bigcup_{n \geq 1} \set{ T, F }^n.
    \end{equation*}

    Then by \fullref{thm:functions_over_model_form_model}, the set \( B = \fun(S, \set{ T, F }) \) of all Boolean functions of arbitrary arities is again a Boolean algebra. See \fullref{thm:lindenmaum_tarski_algebra_of_full_propositional_logic/bijection} for further discussion.

    \item If \( \mscrR \) is a ring and \( S \) is any set, then \( \fun(S, \mscrR) \) is again a ring with componentwise operations --- see \fullref{thm:functions_over_ring_form_algebra}.

    This is useful in functional analysis where we study real-valued and complex-valued functions over arbitrary sets.

    \item If \( \BbbK \) is a field, then in general \( \fun(S, \BbbK) \) is not a field. The simplest example are the real-valued real functions --- \( \sin(x) \) has no multiplicative inverse since \( \sfrac 1 {\sin(x)} \) is not defined for \( x = 2k\pi, k = 1, 2, \ldots \). We can form a \hyperref[def:field_of_fractions]{field of fractions} but in general fields of fractions over function rings no longer correspond to functions --- they are purely algebraic constructions, just like \hyperref[def:formal_power_series]{formal power series}.

    This happens because the definition of a field and, more generally, of a division ring (see \fullref{def:semiring/division_ring}), has a non-positive axiom --- it requires every nonzero element to have a multiplicative inverse, which can be described formally as
    \begin{equation*}
      \qforall \xi \parens[\Big]{ (\xi \doteq 0) \vee \qexists \eta (\xi \cdot \eta \doteq 1) }.
    \end{equation*}

    As discussed in \fullref{def:positive_formula}, a formula with an existential quantifier may fail to be positive.
  \end{itemize}
\end{example}


% Set theory
\section{Set theory}\label{sec:set_theory}
\subsection{Sets}\label{subsec:sets}

\begin{definition}\label{def:set_naive}\mcite[chapter 1]{Enderton1977Sets}
  Naive set theory is not based on a strict axiom set but rather on the intuitive notion of a set as an unordered collection without repetition. Set equality \( A = B \), set membership \( x \in A \) and \hyperref[def:subset]{set inclusion} \( A \subseteq B \) are assumed to be understood. Sets can be explicitly constructed by specifying their elements, e.g.
  \begin{equation*}
    \set{ 3, 7, 31, 127, 8191 }
  \end{equation*}
  or by specifying a unary logical \hyperref[def:first_order_syntax/formula]{formula} \( \varphi(x) \) in an implicitly assumed logical language:
  \begin{equation*}
    \set{ a \given \varphi(a) }
  \end{equation*}

  If \( \varphi(x) = x \in A \land \psi(x) \), we often write
  \begin{equation*}
    \set{ x \in A \given \psi(x) }.
  \end{equation*}

  In a suitable context, the definitions can be made precise. For example, in the ring of integers \( \BbbZ \) with equality, addition, multiplication and predicates partial ordering \( \leq \) and divisibility \( \vert \), each set can be thought of an equivalence class of formulas in the corresponding \hyperref[def:first_order_language]{first-order logic language}. Given formulas \( \varphi_A \) and \( \varphi_B \) with a free variable \( x \) and sets
  \begin{balign*}
    A \coloneqq \{ x \colon \varphi_A(x) \} &  & B \coloneqq \{ x \colon \varphi_B(x) \}
  \end{balign*}

  \begin{itemize}
    \item the membership relation \( x \in A \) holds precisely when \( \BbbZ \models \varphi_A(x) \).

    \item the inclusion relation \( A \subseteq B \) holds when for any \hyperref[def:first_order_valuation/variable_assignment]{evaluation} \( v \) in \( \BbbZ \) and any integer \( x \), we have \( \varphi_A(x) \implies \varphi_B(x) \).

    \item set equality \( A = B \) holds precisely when \( A \subseteq B \) and \( B \subseteq A \)
  \end{itemize}

  Naive set theory easily leads to \hyperref[ex:russels_paradox_sets]{paradoxes}) and so some axiomatization (e.g. \fullref{def:set_zfc}) is required.
\end{definition}

\begin{example}\label{ex:russels_paradox_sets}
  Define
  \begin{equation*}
    R \coloneqq \{ x \colon x \neq x \}.
  \end{equation*}

  We have both \( R \in R \) and \( R \not\in R \).
\end{example}

\begin{definition}\label{def:set_zfc}\mcite[271]{Enderton1977Sets}
  In contrast to \hyperref[def:set_naive]{na\"ive set theory}), \term{Z}ermelo – \term{F}raenkel set theory with the axiom of choice (ZF\term{C}) can be made precise. Consider the \hyperref[def:first_order_language]{first-order logic language} with equality \( = \), no functional symbols and a single binary predicate \( \in \). Note that we can take the language not to have formal equality and then use \fullref{def:set_zfc/A1} as an axiom schema to define equality in terms of \( \in \).

  Given a unary formula \( \varphi(x) \), we can construct a (syntactic) object
  \begin{equation*}
    A = \{ a \colon \varphi(a) \}
  \end{equation*}
  that we call a \term{class}. Not all classes can be defined to have meaningful semantics (e.g. the class of all classes easily leads to paradoxes like \fullref{ex:russels_paradox_sets}). We define sets in ZFC as classes with semantics given by a model for the following axioms (exclude \fullref{def:set_zfc/A8} to obtain ZF). Classes that do not satisfy these axioms are called \term{proper classes} and are often said to be \term{too big} to be sets, e.g. the class of all sets or the class of all vector spaces). In this document, our main limitation when working with classes rather than sets is not being able to talk about a class not being a member of another class, however this is also not necessary for us.

  \begin{thmenum}
    \thmitem[def:set_zfc/A1]{A1}(extensionality) Two sets are equal if they have the same elements (given by set membership)

    \thmitem[def:set_zfc/A2]{A2}(empty set) The following class is a set
    \begin{equation*}
      \varnothing \coloneqq \{ x \colon x \neq x \}.
    \end{equation*}

    \thmitem[def:set_zfc/A3]{A3}(pairing) If \( A \) and \( B \) are sets, then
    \begin{equation*}
      \{ A, B \}
    \end{equation*}
    is also a set. In particular, \( \{ A \} = \{ A, A \} \) is a set.

    \thmitem[def:set_zfc/A4]{A4}(union) If \( A \) is a set, then \( \bigcup A \) (see \fullref{def:set_union}) is also a set.

    \thmitem[def:set_zfc/A5]{A5}(power set) If \( A \) is a set, \( \pow(A) \) (see \fullref{def:power_set}) is also a set.

    \thmitem[def:set_zfc/A6]{A6}(specification) If \( A \) is a set and \( \varphi \) is a formula, then
    \begin{equation*}
      \{ x \in A \colon \varphi(x) \}
    \end{equation*}
    is a set.

    \thmitem[def:set_zfc/A7]{A7}(infinity) There exists an \hyperref[def:inductive_set]{inductive set}.

    \thmitem[def:set_zfc/A8]{A8}(choice; see \fullref{thm:aoc}) Let \( \mscrM \neq 0 \) and for all \( m \in \mscrM \), let \( X_m \) be a nonempty set and \( X_k \cap X_m = \varnothing \) whenever \( m \neq k \). Then there exists a set \( M \) such that for every \( m \in \mscrM \), the intersection \( M \cap X_m \) (see \fullref{def:set_intersection}) has exactly one member.

    \thmitem[def:set_zfc/A9]{A9}(replacement) Given a set \( X \) and a formula \( \varphi(x, y) \), if for every set \( x \in X \) there exists a unique set \( y \) such that \( \varphi(x, y) \) holds, then
    \begin{equation*}
      Y \coloneqq \{ y \colon \exists x \in X, \varphi(x, y) \}
    \end{equation*}
    is a set.

    \thmitem[def:set_zfc/A10]{A10}(regularity) For every nonempty set \( A \), there exists a member \( a \in A \) such that
    \begin{equation*}
      a \cap A \neq \varnothing.
    \end{equation*}
  \end{thmenum}
\end{definition}

\begin{remark}\label{rem:family_of_sets}
  In ZFC \fullref{def:set_zfc}, everything is a set. However, it is often the case that we are not interested in how a set's elements are represented and only in how they behave, e.g. when working with \hyperref[def:natural_numbers]{natural numbers} we are interested in the elements of \( \BbbN \) and not in the way every element of \( \BbbN \) is encoded as a set.

  In order to reduce repetitiveness, sets whose elements we consider to be other sets, are often called \term{families} of sets. In particular if all (different) sets are \hyperref[def:set_intersection]{disjoint}, we say that the family is a \term{disjoint family}. We usually assume that the sets are nonempty.
\end{remark}

\begin{remark}\label{rem:singleton_sets}
  Sets with a single elements are usually called \term{singletons}. It is sometimes convenient, especially with connection to geometry or multivalued \hyperref[def:function]{functions} (e.g. when dealing with \hyperref[def:net_convergence/limit]{limits} or \hyperref[def:subdifferentials]{subdifferentials}), to not distinguish between singleton sets and their corresponding element.
\end{remark}

\begin{definition}\label{def:subset}
  We say that \( A \) is a \term{subset} of \( B \) and write \( A \subseteq B \) if \( x \in A \implies x \in B \). If \( A \) is a subset of \( B \), we say that B is a \term{superset} of \( A \).

  If \( A \subseteq B \) and \( A \neq B \), we say that \( A \) is a \term{proper subset} of \( B \) and write \( A \subsetneq B \).
\end{definition}

\begin{remark}\label{rem:subset_notation}
  Some authors, such as \cite{Kelley1955}, use the notation \( A \subseteq B \) to mean \enquote{all elements of \( A \) belong to \( B \)}, even in the case when \( A = B \). To avoid confusion, we use the notations \( A \subseteq B \) and \( A \subsetneq B \) (see \fullref{def:subset}).
\end{remark}

\begin{remark}\label{rem:subset_and_membership_relations}
  Both \( \in \) and \( \subseteq \) are binary \hyperref[def:relation]{relations}, called the \term{membership} and \term{inclusion} relations, correspondingly.
\end{remark}

\begin{definition}\label{def:set_intersection}\mcite[24]{Enderton1977Sets}
  If \( A \) is a set, define their \term{intersection} as
  \begin{equation*}
    \bigcap A \coloneqq \{ x \colon \forall a \in A, x \in a \}.
  \end{equation*}

  We leave \( \bigcap \varnothing \) undefined.

  By \fullref{def:set_zfc/A6}, \( \bigcap A \) is a set.

  For two sets \( A \) and \( B \), we define the \term{binary intersection} as
  \begin{equation*}
    A \cup B \coloneqq \bigcap \{ A, B \} = \{ x \colon x \in A \land x \in B \}.
  \end{equation*}

  The class \( \{ A, B \} \) is a set by \fullref{def:set_zfc/A3} and \( A \cup B \) is a set by \fullref{def:set_zfc/A6}.

  If \( A \cap B = \varnothing \), we say that \( A \) and \( B \) are \term{disjoint}. If they are not disjoint, we say that they \term{intersect}.
\end{definition}

\begin{definition}\label{def:set_union}\mcite[24]{Enderton1977Sets}
  If \( A \) is a set, define its \term{union} as
  \begin{equation*}
    \bigcup A \coloneqq \{ x \colon \exists a \in A, x \in A \}.
  \end{equation*}

  In particular, \( \bigcup \varnothing = \varnothing \).

  By \fullref{def:set_zfc/A4}, \( \bigcup A \) is a set.

  For two sets \( A \) and \( B \), we define the \term{binary union} as
  \begin{equation*}
    A \cup B \coloneqq \bigcup \{ A, B \} = \{ x \colon x \in A \lor x \in B \}.
  \end{equation*}

  The class \( \{ A, B \} \) is a set by \fullref{def:set_zfc/A3} and \( A \cup B \) is a set by \fullref{def:set_zfc/A4}.
\end{definition}

\begin{definition}\label{def:set_difference}\mcite[27]{Enderton1977Sets}
  If \( A \) and \( B \) are sets, define their \term{difference} as
  \begin{equation*}
    A \setminus B \coloneqq \{ a \in A \colon a \not\in B \}.
  \end{equation*}

  By \fullref{def:set_zfc/A6}, \( A \setminus B \) is a set.
\end{definition}

\begin{proposition}\label{thm:set_difference_properties}
  Set \hyperref[def:set_difference]{difference} has the following basic properties:
  \begin{thmenum}
    \thmitem{thm:set_difference_properties/intersection} If \( A \) and \( B \) are subsets of \( C \), then \( A \setminus B = A \cap (C \setminus B) \).
    \thmitem{thm:set_difference_properties/double_difference} If \( A \subseteq B \), then \( B \setminus (B \setminus A) = A \)
  \end{thmenum}
\end{proposition}
\begin{proof}
  \SubProofOf{thm:set_difference_properties/intersection} Since \( a \in A \) implies \( a \in C \), we have
  \begin{balign*}
    A \setminus B
     & =
    \{ a \in A \colon a \not\in B \}
    =    \\ &=
    \{ a \in A \colon a \in C \text{ and } a \not\in B \}
    =    \\ &=
    A \cap (C \setminus B).
  \end{balign*}

  \SubProofOf{thm:set_difference_properties/double_difference} By the law of the excluded middle,
  \begin{balign*}
    B \setminus (B \setminus A)
     & =
    \{ b \in B \colon b \not\in \{ b \in B \colon b \not\in A \} \}
    =    \\ &=
    \{ b \in B \colon b \in A \}
    =    \\ &=
    A.
  \end{balign*}
\end{proof}

\begin{definition}\label{def:power_set}\mcite[19]{Enderton1977Sets}
  If \( A \) is a set, define its \term{power set} as
  \begin{equation*}
    \pow(A) \coloneqq \{ B \colon B \subseteq A \}.
  \end{equation*}

  By \fullref{def:set_zfc/A5}, \( \pow(A) \) is a set.
\end{definition}

\begin{proposition}\label{thm:subsets_form_boolean_algebra}
  Let \( X \) be an arbitrary set. Then \( \pow(X) \) is a \hyperref[def:boolean_algebra]{Boolean algebra} with the \hyperref[def:poset]{partial order} \( \subseteq \) (see \fullref{def:subset}) and complements given by \( \overline{A} \coloneqq X \setminus A \). More concretely,
  \begin{refenum}
    \refitem{def:binary_lattice_operations/join} Joins are given by \hyperref[def:set_union]{unions \( \bigcup \)}.
    \refitem{def:binary_lattice_operations/meet} Meets are given by \hyperref[def:set_intersection]{intersections \( \bigcap \)}.
    \refitem{def:lattice/top} The top element is \( X \).
    \refitem{def:lattice/bottom} The bottom is \( \varnothing \).
  \end{refenum}
\end{proposition}
\begin{proof}
  The structure is based on \fullref{thm:propositional_formula_cosets_are_boolean_functions/boolean_algebra} because every set corresponds to a certain equivalence class of first-order logic formulas.
\end{proof}

\begin{remark}\label{rem:binary_vs_arbitrary_tuples}
  We give two pairs of definitions for tuples and Cartesian products. The first pair, \fullref{def:kuratowski_pair,def:binary_cartesian_product}, is quite restricted and is mostly necessary for defining \hyperref[def:function]{functions} and ensuring that everything along the way is indeed a set. The second pair of definitions, given in \fullref{def:cartesian_product}, can then be used freely.
\end{remark}

\begin{definition}\label{def:kuratowski_pair}\mcite[36]{Enderton1977Sets}
  If \( A \) and \( B \) are sets, define the \term{(binary) tuple} or \term{Kuratowski pair} as
  \begin{equation*}
    (A, B) \coloneqq \{ \{ A \}, \{ A, B \} \}.
  \end{equation*}

  By \fullref{def:set_zfc/A3}, \( (A, B) \) is a set.
\end{definition}

\begin{definition}\label{def:binary_cartesian_product}\mcite[37]{Enderton1977Sets}
  If \( A \) and \( B \) are sets, define their \term{binary Cartesian product} as
  \begin{equation*}
    A \times B \coloneqq \{ (a, b) \colon a \in A \land b \in B \}.
  \end{equation*}
\end{definition}

\begin{proposition}\label{thm:binary_cartesian_product_is_set}
  If \( A \) and \( B \) are sets, their product \( A \times B \) is also a set.
\end{proposition}
\begin{proof}
  Fix \( a \in A \) and \( b \in B \).
  \begin{itemize}
    \item \( \{ a \} \) is a set by \fullref{def:set_zfc/A6} since \( \{ a \} \subseteq A \)
    \item \( A \cup B \) is a set by \fullref{def:set_union}
    \item \( \{ a, b \} \) is a set by \fullref{def:set_zfc/A6} since \( \{ a \} \subseteq A \cup B \)
    \item \( (a, b) = \{ \{ a \}, \{ a, b \} \} \) is a set by \fullref{def:set_zfc/A6} since \( (a, b) \subseteq \pow(A \cup B) \).
  \end{itemize}

  Thus \( A \times B \) is a set since \( A \times B \subseteq \pow(\pow(A \cup B)) \).
\end{proof}

\begin{definition}\label{def:cartesian_product}\mcite[54]{Enderton1977Sets}
  Let \( \{ X_k \}_{k \in \mscrK} \) be a nonempty \hyperref[def:indexed_family]{family} of (potentially nonempty) sets.

  We define their \term{Cartesian product} as
  \begin{equation*}
    \prod_{k \in \mscrK} X_k \coloneqq \set*{ f: \mscrK \to \bigcup_{k \in \mscrK} X_k \given \qforall{m \in \mscrK} f(m) \in X_m }.
  \end{equation*}

  The definition makes sense when any of the sets is empty because the product itself is then empty.

  Any element of the Cartesian product is called a \term{tuple}.
\end{definition}

\begin{definition}\label{def:disjoint_union}
  Let \( \{ X_k \}_{k \in \mscrK} \) be a nonempty \hyperref[def:indexed_family]{family} of nonempty sets.

  We define their \term{disjoint union} as
  \begin{equation*}
    \coprod_{k \in \mscrK} X_k \coloneqq \{ (k, x) \colon k \in \mscrK, x \in X_k \}.
  \end{equation*}
\end{definition}

\begin{definition}\label{def:ordinal_successor_operator}\mcite[68]{Enderton1977Sets}
  For any set \( X \), we define the \term{successor} operation
  \begin{equation*}
    S(X) \coloneqq X \cup \{ X \}.
  \end{equation*}
\end{definition}

\begin{definition}\label{def:inductive_set}\mcite[68]{Enderton1977Sets}
  A set \( A \) is called \term{inductive} if
  \begin{thmenum}
    \item \( \varnothing \in A \)
    \item \( a \in A \implies S(a) \in A \)
  \end{thmenum}
\end{definition}

\begin{definition}\label{def:smallest_inductive_set}
  The smallest \hyperref[def:inductive_set]{inductive set} is
  \begin{equation*}
    \omega \coloneqq \bigcap \{ A \colon A \text{ is an inductive set} \}.
  \end{equation*}

  The elements of \( \omega \) are
  \begin{equation*}
    \varnothing, S(\varnothing), S(S(\varnothing)),
  \end{equation*}
  where \( S \) is the set-theoretic successor operator (see \fullref{def:ordinal_successor_operator}).

  These numbers are called \term{von Neumann ordinals}.
\end{definition}

\subsection{The category of sets}\label{subsec:category_of_sets}

\begin{definition}\label{def:category_of_sets}
  We define the \hyperref[def:category]{category} \( \cat{Set} \) as follows:
  \begin{RefList}
    \iref{def:category/C1} The \hyperref[def:set_zfc]{class} of objects is the class of all sets.

    \iref{def:category/C2} The morphisms between two sets \( A, B \) are the \hyperref[def:function]{functions} \( f: A \to B \).

    \iref{def:category/C3} Composition of morphisms is the usual \hyperref[def:function/composition]{function composition}.
  \end{RefList}
\end{definition}
\begin{proof}
  To see that \( \cat{Set} \) is indeed a category, we verify the axioms
  \SubProofOf{def:category/identity} For any set \( A \in \cat{Set} \), we have the identity function
  \begin{balign*}
     & \id_A: A \to A        \\
     & \id_A(x) \coloneqq A.
  \end{balign*}

  If \( f: A \to B \) is any function, for all \( x \in A \) we have
  \begin{balign*}
    [\id_B \circ f](x) = \id_B(f(x)) = f(x),
     &  &
    [f \circ \id_A](x) = f(\id_A(x)) = f(x),
  \end{balign*}
  thus \( \id_A \) and \( \id_B \) are indeed identity morphisms.

  \SubProofOf{def:category/associativity} Let \( f: A \to B \), \( g: B \to C \) and \( h: C \to D \) be arbitrary functions. For any \( x \in A \), we have
  \begin{balign*}
    [[h \circ g] \circ f](x)
    =
    [h \circ g](f(x))
    =
    h(g(f(x)))
    =
    h([g \circ f](x))
    =
    [h \circ [g \circ f]](x).
  \end{balign*}
\end{proof}

\begin{corollary}\label{thm:functions_over_set_form_monoid}
  Let \( X \) be a set. Then the family \( \cat{Set}(X) \) of all \hyperref[rem:category_obj_hom]{functions} is a \hyperref[def:unital_magma/associative]{monoid} with function composition as the operation.
\end{corollary}

\begin{proposition}\label{thm:set_is_locally_small}
  The category \( \cat{Set} \) is \hyperref[def:category_cardinality]{locally small}.
\end{proposition}
\begin{proof}
  Since a function \( f: X \to Y \) is \hyperref[def:function]{formally} a subset of the \hyperref[def:cartesian_product]{Cartesian product} \( X^2 \), which is a set, it is itself a set.

  The class of all functions \( \cat{Set}(X, Y) \) from \( X \) to \( Y \) is then a subset of \( \pow(X \times Y) \), which is also a set. Thus \( \cat{Set} \) is a locally small category.
\end{proof}

\begin{proposition}\label{thm:set_categorical_limits}
  We are interested in \hyperref[def:categorical_limit]{categorical limits} and \hyperref[def:categorical_colimit]{colimits} in \( \cat{Set} \). Fix an indexed \hyperref[def:indexed_family]{family} \( \{ X_k \}_{k \in \mscrK} \) of sets.
  \begin{defenum}
    \ilabel{thm:set_categorical_limits/product} Their \hyperref[def:categorical_product]{categorical product} is their \hyperref[def:cartesian_product]{Cartesian product} \( \prod_{k \in \mscrK} X_k \), the projection morphisms being
    \begin{balign*}
       & \pi_m: \prod_{k \in \mscrK} X_k \to X_m        \\
       & \pi_m(\{ x_k \}_{k \in \mscrK}) \coloneqq x_m.
    \end{balign*}

    \ilabel{thm:set_categorical_limits/coproduct} Their \hyperref[def:categorical_coproduct]{categorical coproduct} is their disjoint \hyperref[def:disjoint_union]{union} \( \coprod_{k \in \mscrK} X_k \), the embedding morphisms being
    \begin{balign*}
       & \iota_m: X_m \to \coprod_{k \in \mscrK} X_k \\
       & \iota_m(x_k) \coloneqq (m, x_k).
    \end{balign*}

    \ilabel{thm:set_categorical_limits/equalizer} An equalizer of two functions \( f, g: X \to Y \) in \( \cat{Set} \) is the set
    \begin{equation*}
      \{ x \in X \colon f(x) = g(x) \}.
    \end{equation*}

    Compare this with pullbacks in \( \cat{Set} \) (see \fullref{thm:set_categorical_limits/pullback}).

    \ilabel{thm:set_categorical_limits/pullback} The pullback of two functions \( f: X \to Z \) and \( g: Y \to Z \) in \( \cat{Set} \) is the set
    \begin{equation*}
      \{ (x, y) \in X \times Y \colon f(x) = g(y) \}.
    \end{equation*}

    Compare this with equalizers in \( \cat{Set} \) (see \fullref{thm:set_categorical_limits/equalizer}).

    \ilabel{thm:set_categorical_limits/coequalizer} A coequalizer of two functions \( f, g: X \to Y \) in \( \cat{Set} \) is the quotient set formed by the reflexive, symmetric and transitive closure of the relation \( x \sim y \iff f(x) = g(x) \).

    In particular, if \( \sim \subseteq X^2 \) is an \hyperref[def:equivalence_relation]{equivalence relation}) on \( X \), then the coequalizer of the two projection maps of the product \( X^2 \) is the pair \( (X / ~, \pi) \), where \( \pi \) is the quotient map
    \begin{balign*}
       & \pi: X \to X / \sim \\
       & \pi(x) = [x].
    \end{balign*}

    \ilabel{thm:set_categorical_limits/pushout} Let \( X \) and \( Y \) be two sets, let \( Z \) be a subset of \( X \) and let \( i: Z \to X \) be the inclusion map. For any function \( f: Z \to Y \), we define a pushout of \( i \) and \( f \) in \( \cat{Set} \) to be the set obtained as the quotient of the coproduct \( X \coprod Y \) and the relation \( x \sim y \iff i^{-1}(x) = f^{-1}(y) \).
  \end{defenum}
\end{proposition}

\begin{proposition}\label{thm:set_is_monoidal}
  The category \( \cat{Set} \) is monoidal with
  \begin{itemize}
    \item the \hyperref[def:cartesian_product]{Cartesian product} acting as a monoidal product
    \item the singleton set \( \{ \varnothing \} \) acting as an identity object
    \item natural transformations
          \begin{balign*}
            \sigma                              & \coloneqq \id \\
            \lambda(\{ \varnothing \} \times A) & \coloneqq A   \\
            \rho(A \times \{ \varnothing \})    & \coloneqq A
          \end{balign*}
  \end{itemize}
\end{proposition}
\begin{proof}
  All conditions in \fullref{def:monoidal_category} are trivially satisfied.
\end{proof}

\begin{proposition}\label{thm:monoids_are_monoids_in_set}
  A monoid in the sense of \fullref{def:unital_magma/associative} is a monoid in \( \cat{Set} \) in the sense of \fullref{def:categorical_monoid}.
\end{proposition}
\begin{proof}
  By \fullref{thm:set_is_monoidal}, \( \cat{Set} \) is monoidal with the Cartesian product as a monoidal product. Let \( \mscrM \) be a monoid in the sense of \fullref{def:unital_magma/associative}. We define the morphism \( \mu: \mscrM \times \mscrM \to \mscrM \) to be the monoid operation and the morphism \( \eta: \{ \varnothing \} \to \mscrM \) to be the identity operation. Then the diagrams in \fullref{def:categorical_monoid} commute.

  The categorical definition of morphism between monoids in \( \cat{Set} \) is then a restatement of the definition of homomorphism of a unital magma. If \( (\mscrM, \mu, \eta) \) and \( (\mscrM', \mu', \eta') \) are monoids in \( \cat{Set} \) and \( f \) is a morphism between them, then
  \begin{equation*}
    (f \circ \mu)(x, y)
    =
    f(xy)
    =
    f(x) f(y)
    =
    (\mu' \circ (f \otimes f))(x, y)
  \end{equation*}
  and
  \begin{equation*}
    (f \circ \eta)(\{ \varnothing \})
    =
    f(e_{\mscrM})
    =
    e_{\mscrM'}
    =
    \eta'(\{ \varnothing \}).
  \end{equation*}
\end{proof}

\subsection{Axiom of choice}\label{subsec:axiom_of_choice}

\begin{remark}\label{rem:aoc}
  \Fullref{thm:aoc} famously leads to counterintuitive results, which is a reason why it is frowned upon. It is insanely useful in the form of its many equivalent statements (see \fullref{def:set_zfc/A8} and \fullref{sec:index}).
\end{remark}

\begin{definition}\label{def:choice_function}
  Let \( X \) be a nonempty set. A \term{choice function} is a function of type \( f: \pow(X) \to X \) such that, for all subsets \( A \subseteq X \),
  \begin{equation*}
    f(A) \in A.
  \end{equation*}

  That is, the function \( f \) \enquote{chooses} an element out of any subset of \( X \).
\end{definition}

\begin{theorem}[Axiom of choice]\label{thm:aoc}
  The following are equivalent:

  \begin{thmenum}
    \ilabel{thm:aoc/transversal}\mcite\cite[thm. 6M(4)]{Enderton1977Sets}Every hypergraph has a minimal transversal (compare to \fullref{thm:finite_hypergraphs_have_minimal_transversal} that only mentions finite sets).

    \ilabel{thm:aoc/product}\mcite\cite[thm. 6M(2)]{Enderton1977Sets}The \hyperref[def:cartesian_product]{Cartesian product} of a family of nonempty sets is nonempty.

    \medskip

    \ilabel{thm:aoc/choice}\mcite\cite[thm. 6M(3)]{Enderton1977Sets}For any nonempty set \( X \) there exists a \hyperref[def:choice_function]{choice function}.

    \medskip

    \ilabel{thm:aoc/function}\mcite\cite[thm. 6M(1)]{Enderton1977Sets}Every \hyperref[def:function/multivalued]{multivalued function} has a \hyperref[def:function/selection]{selection}.

    \ilabel{thm:aoc/cardinals} \Fullref{thm:cardinal_trichotomy}.

    \ilabel{thm:aoc/zorn} \Fullref{thm:zorns_lemma}.

    \ilabel{thm:aoc/well_ordering_principle} \Fullref{thm:well_ordering_principle}.

    \ilabel{thm:aoc/vector_space_bases} \Fullref{thm:all_vector_spaces_are_free_left_modules}. Compare this to finite-dimensional vector spaces of order \( n \) over \( \BbbK \), which are all isomorphic to \( \BbbK^n \).

    \ilabel{thm:aoc/tychonoff} \Fullref{thm:tychonoffs_product_theorem}.

    \ilabel{thm:aoc/krull} \Fullref{thm:krulls_theorem}.
  \end{thmenum}
\end{theorem}

\subsection{Relations}\label{subsec:relations}

\begin{definition}\label{def:relation}
  Let \( \{ X_k \}_{k \in \mscrK} \) be a family of sets and let
  \begin{equation*}
    R \subseteq \prod_{k \in \mscrK} X_k
  \end{equation*}
  be a subset of their Cartesian product.

  The \hyperref[def:cartesian_product]{tuple} \( (R,  \{ X_k \}_{k \in \mscrK}) \) is called a \term{relation}.

  \begin{thmenum}
    \thmitem{def:relation/graph} \( R \) is called the \term{graph} of the relation. In case the family \( \{ X_k \}_{k \in \mscrK} \) is clear from the context, we say that the graph \( R \) itself is a relation.

    \thmitem{def:relation/arity} When \( \mscrK \) is a finite set of \hyperref[rem:cardinals]{cardinality} \( n \), the relation is called \term{n-ary}. In particular,
    \begin{itemize}
      \item \( R \) is nullary if \( n = 0 \)
      \item \( R \) is unary if \( n = 1 \)
      \item \( R \) is binary if \( n = 2 \)
      \item \( R \) is ternary if \( n = 3 \)
    \end{itemize}

    This is not to be confused with \fullref{def:function/arity}.

    \thmitem{def:relation/single_set} If all \( X_k \) are equal to some set \( X \), \( X \) is clear from the context and we usually say that \( R \subseteq X^{\mscrK} \) is a \term{relation on} \( X \).
  \end{thmenum}
\end{definition}

\begin{example}\label{ex:relation}
  Relations are used in vastly different contexts:
  \begin{itemize}
    \item Functions (see \fullref{def:function}) are special binary relations.
    \item Orders (see \fullref{sec:order_theory}) are also special binary relations.
    \item Directed graphs (see \fullref{def:directed_graph}) are binary relations over finite sets.
    \item Entourages (see \fullref{def:entourage}) are binary relations in \hyperref[def:uniform_space]{uniform spaces}.
    \item Structures in first-order logic (see \fullref{def:first_order_structure}) use relations for giving semantics to predicates. Predicates are ubiquitous in mathematics, e.g.
          \begin{itemize}
            \item Questions of the form \enquote{does some property \( \varphi(x) \) hold for \( x_0 \)} are unary predicates. This is probably the most common type of questions in mathematics. Two examples are \enquote{is a real \hyperref[def:real_numbers]{number} positive} and \enquote{is a uniform space \hyperref[def:complete_uniform_space]{complete}}.
            \item Affine planes (see \fullref{def:affine_plane}) define several binary predicates.
            \item Production rules and derivations in formal grammars (see \fullref{def:grammar}) are binary predicates.
            \item Elementary questions in set theory like \enquote{is \( A \) a \hyperref[def:subset]{subset} of \( B \)} or \enquote{are \( A \) and \( B \) equinumerous} are modeled using predicates.
            \item Questions in general \hyperref[sec:general_topology]{topology} like \enquote{are two spaces homeomorphic} or questions in algebra like \enquote{are two groups homomorphic} are modeled using predicates. These questions, however, are more suitable for \hyperref[def:morphism_invertibility/isomorphism]{isomorphisms} in category theory.
          \end{itemize}
  \end{itemize}
\end{example}

\begin{definition}\label{def:binary_relation}
  Let \( R \subseteq X \times Y \) be a binary relation. We introduce the following terminology:
  \begin{thmenum}[series=def:binary_relation]
    \thmitem{def:binary_relation/domain} We define the \term{domain} of \( R \) as the set
    \begin{equation*}
      \dom(R) \coloneqq \{ x \in X \colon \exists y: (x, y) \in R \}
    \end{equation*}
    of all members of \( X \) that belong to at least one tuple.

    \thmitem{def:binary_relation/image} Similarly, we define the \term{image} of \( R \) as the set
    \begin{equation*}
      \img(R) \coloneqq \{ y \in Y \colon \exists x: (x, y) \in R \}
    \end{equation*}
    of all members of \( Y \) that belong to at least one tuple.

    \thmitem{def:binary_relation/range} The set \( Y \) is called the \term{range} of \( R \). There is no similar established terminology for \( X \).

    \thmitem{def:binary_relation/inverse} We define \term{inverse relation} of \( R \) as
    \begin{equation*}
      \neg R \coloneqq (X \times Y) \setminus R.
    \end{equation*}

    \thmitem{def:binary_relation/converse} We define \term{converse relation} of \( R \) as
    \begin{equation*}
      R^{-1} \coloneqq \{ (y, x) \colon (x, y) \in R \}.
    \end{equation*}

    \thmitem{def:binary_relation/diagonal} A very special relation is the \term{diagonal relation} on a set \( X \):
    \begin{equation*}
      \Delta_X \coloneqq \{ (x, x) \colon x \in X \}.
    \end{equation*}

    \thmitem{def:binary_relation/composition} Given two relations \( R \subseteq X \times Y \) and \( T \subseteq Y \times Z \), we define their composition as
    \begin{equation*}
      T \circ R \coloneqq \{ (x, z) \in X \times Z \colon \exists y \in Y: (x, y) \in R \T{and} (y, z) \in T \}.
    \end{equation*}
  \end{thmenum}

  Whenever \( X = Y \), the following are ubiquitous axioms a binary relation \( R \):
  \begin{thmenum}[resume=def:binary_relation]
    \thmitem{def:binary_relation/reflexive} \( R \) is \term{reflexive} if \( \Delta_X \subseteq R \).

    \thmitem{def:binary_relation/irreflexive} \( R \) is \term{irreflexive} if \( \Delta_X \cap R = \varnothing \).

    \thmitem{def:binary_relation/symmetric} \( R \) is \term{symmetric} if \( R = R^{-1} \).

    \thmitem{def:binary_relation/antisymmetric} \( R \) is \term{antisymmetric} if \( R \cap R^{-1} = \Delta_X \).

    \thmitem{def:binary_relation/transitive} \( R \) is \term{transitive} if
    \begin{equation*}
      (x, y) \in R \T{and} (y, z) \in R \T{implies} (x, z) \in R.
    \end{equation*}

    \thmitem{def:binary_relation/total} \( R \) is \term{total} if, for all \( x, y \in X \), either \( (x, y) \in R \) or \( (y, x) \in R \). This is not to be confused with \fullref{def:function/total}.

    \thmitem{def:binary_relation/trichotomic} \( R \) is \term{trichotomic} if, for all \( x, y \in X \), either \( x = y \), \( (x, y) \in R \) or \( (y, x) \in R \).
  \end{thmenum}
\end{definition}

\begin{definition}\label{def:derived_relations}
  Let \( R \) be a binary relation on \( X \).

  \begin{thmenum}
    \thmitem{def:derived_relations/reflexive} The \term{reflexive closure} of \( R \) is defined as
    \begin{equation*}
      \cl^R(R) \coloneqq R \cup \Delta_X.
    \end{equation*}

    \thmitem{def:derived_relations/symmetric} The \term{symmetric closure} of \( R \) is defined as
    \begin{equation*}
      \cl^S(R) \coloneqq R \cup R^{-1}.
    \end{equation*}

    \thmitem{def:derived_relations/transitive} The \term{transitive closure} of \( R \) is defined as
    \begin{equation*}
      \cl^T(R) \coloneqq \bigcup_{k=1}^\infty R^k,
    \end{equation*}
    where \( R^k \) is iterated \hyperref[def:binary_relation/composition]{composition}. In other words, it is the smallest relation such that \( (x, y) \in R \) and \( (y, z) \in R \) together imply then \( (x, z) \in \cl^T(R) \).

    The \term{transitive reduction} \( \red^T(R) \) of \( R \) is the smallest (with respect to set inclusion) relation such that \( \cl^T(\red^T(R)) = \cl^T(R) \).
  \end{thmenum}
\end{definition}

\begin{proposition}\label{thm:derived_relations_characterization}
  The is reflexive (resp. symmetric or transitive) closure of a relation \( R \) is the smallest reflexive (resp. symmetric or transitive) relation that contains \( R \).
\end{proposition}
\begin{proof}
  Every other reflexive (resp. symmetric or transitive) relation strictly contains the closure.
\end{proof}

\begin{definition}\label{def:equivalence_relation}
  A relation that is \hyperref[def:binary_relation/reflexive]{reflexive}, \hyperref[def:binary_relation/symmetric]{symmetric} and \hyperref[def:binary_relation/transitive]{transitive} is called an equivalence relation. In other words, an equivalence relation is a symmetric \hyperref[def:preordered_set]{preorder}.

  Using the infix notation convention (see \fullref{rem:first_order_formula_conventions/infix}), we usually denote equivalence relations by \( \cong \).

  \begin{thmenum}
    \thmitem{def:equivalence_relation/coset} We define \term{equivalence classes} or \term{cosets} to be sets of the form
    \begin{equation*}
      [x] \coloneqq \{ y \in X \colon x \cong y \}.
    \end{equation*}

    \thmitem{def:equivalence_relation/quotient} We define the \term{quotient set} of \( X \) by \( \cong \) as
    \begin{equation*}
      X / \cong \ \coloneqq \{ [x] \colon x \in X \}.
    \end{equation*}

    \thmitem{def:equivalence_relation/projection} We call the function
    \begin{balign*}
       & \pi: X \to X / \cong  \\
       & \pi(a) \coloneqq [a].
    \end{balign*}
    the \term{canonical projection}. See \fullref{thm:equivalence_partition}.

    The function \( \pi \) can be regarded as a \hyperref[def:function/multivalued]{multivalued function} from \( X \) to \( X \).
  \end{thmenum}
\end{definition}

\begin{proposition}\label{thm:equality_is_smallest_equivalence_relation}
  The equality \hyperref[def:relation]{relation} \( = \) is the intersection of all equivalence relations.
\end{proposition}
\begin{proof}
  It is equivalent to the \hyperref[def:binary_relation/diagonal]{diagonal relation} \( \Delta_X \). By \fullref{thm:derived_relations_characterization}, it is the smallest reflexive relation on \( X \), i.e. the intersection of all reflexive relations.
\end{proof}

\begin{definition}\label{def:set_partition}
  Let \( X \) be a set. A \term{cover of \( X \)} is a \hyperref[rem:family_of_sets]{family} \( \mscrP \subseteq \pow(X) \) of nonempty sets such that \( X = \bigcup \mscrP \).

  A \term{partition} of \( X \) is a pairwise disjoint cover. In other words, each element of \( X \) belong to exactly one set in a partition \( \mscrP \).
\end{definition}

\begin{lemma}\label{thm:equivalence_relation_inheriance}
  If \( f: X \to Y \) is a function, then the relation \( \cong \) defined by \( x \cong y \iff f(x) = f(y) \) is an equivalence relation on \( X \).
\end{lemma}
\begin{proof}
  Follows from the fact that \( = \) is an equivalence relation.
\end{proof}

\begin{proposition}\label{thm:equivalence_partition}
  Fix a set \( X \). Let \( \sim \) be a relation of \( X \). The following are equivalent:
  \begin{thmenum}
    \thmitem{thm:equivalence_partition/equivalence} \( \sim \) is an \hyperref[def:equivalence_relation]{equivalence relation}.

    \thmitem{thm:equivalence_partition/partition} There exists a \hyperref[def:set_partition]{partition} \( \mscrP \) of \( X \) such that
    \begin{equation}\label{thm:equivalence_partition/partition/property}
      x \sim y \iff \exists P \in \mscrP: \{ x, y \} \subseteq P.
    \end{equation}

    \thmitem{thm:equivalence_partition/function} There exists a set \( Y \) and a function \( f: X \to Y \) such that \( f(a) = f(b) \iff a \sim b \).
  \end{thmenum}
\end{proposition}
\begin{proof}
  \ImplicationSubProof{thm:equivalence_partition/equivalence}{thm:equivalence_partition/partition} Let \( \sim \) be an equivalence relation on \( X \). The quotient set \( X / \sim \) is a partition since
  \begin{itemize}
    \item Every element \( x \in X \) belongs exactly one equivalence class \( [x] \).
    \item The equivalence classes are disjoint. Indeed, let \( [x] \cap [y] \neq \varnothing \) and let \( z \in [x] \cap [y] \). Assume\DNE that \( x \not\sim x \). Then \( z \sim x \) and \( z \sim y \), thus \( x \sim z \sim y \) and \( x \sim y \), which is a contradiction. Thus either \( [x] = [x] \) or \( [y] \cap [z] = \varnothing \).
  \end{itemize}

  \ImplicationSubProof{thm:equivalence_partition/partition}{thm:equivalence_partition/function} Let \( \mscrP \) be a partition of \( X \) satisfying \fullref{thm:equivalence_partition/partition/property}. Denote by \( P_x \) the set in \( \mscrP \) which contains \( x \) and define the function
  \begin{balign*}
     & f: X \to \mscrP \\
     & f(x) = P_x.
  \end{balign*}

  This function is well defined since since \( \mscrP \) is a partition, which means that \( x \) belongs to exactly one set in \( \mscrP \).

  \ImplicationSubProof{thm:equivalence_partition/function}{thm:equivalence_partition/equivalence} Follows from \fullref{thm:equivalence_relation_inheriance}.
\end{proof}

\subsection{Functions}\label{subsec:functions}

\begin{remark}\label{rem:function_definition}
  It is not straightforward to formalize the notion of correspondence between two values. We will reserve the term \term{mapping} for this informal notion and use \term{function} in the sense of \fullref{def:function}. There are several drawbacks of using material (that is, membership-based) set theory for defining functions:

  \begin{itemize}
    \item Mappings are often more general than what can be formalized, i.e. there exist correspondences between logical \hyperref[def:first_order_syntax/formula]{formulas} and between proper \hyperref[def:set]{classes} that cannot be defined in set theory without reaching contradictions.

    \item The ambient space often has an additional structure, e.g. algebraic or topological, that is not carried by functions. This leads to definitions such as \hyperref[def:first_order_homomorphism]{homomorphism} and \hyperref[def:isometry]{isometry}. This is a motivating example for the benefits of \hyperref[sec:category_theory]{category theory}, where the notion of \hyperref[def:category/C2]{morphism} is able to capture this additional structure (see \fullref{def:category_of_sets}).

    \item Several generalizations of set-theoretic functions are often used, e.g. \hyperref[def:function]{multivalued} or \hyperref[def:function/partial]{partial functions}, however most formalisms of set theory often only concern functions.

    \item Set-theoretic functions are often used in contexts where they do not refer to the intuitive notion of a mapping, e.g. for Cartesian \hyperref[def:cartesian_product]{products} or for indexed \hyperref[def:indexed_family]{families}.
  \end{itemize}
\end{remark}

\begin{definition}\label{def:function}
  A \term{multivalued function} from \( X \) to \( Y \) is simply a \hyperref[def:binary_relation]{binary relation} \( (F, X, Y) \), the difference being in how we treat multivalued functions and relations. We use the notation \( F: X \rightrightarrows Y \).

  \begin{thmenum}[series=def:function]
    \thmitem{def:function/single_valued} The \hyperref[def:function/multivalued]{multivalued function} \( F: X \rightrightarrows Y \) is called a \term{single-valued function} or simply a \term{function} if \( F(x) \) is a \hyperref[rem:singleton_sets]{singleton set} for each \( x \in X \). In this case, we write \( F: X \to Y \) rather than \( F: X \rightrightarrows Y \).

    Formally, the value of a single-valued function is an element of \( Y \) rather than a subset of \( Y \). \Fullref{rem:singleton_sets} allows us to ignore this distinction unless is would cause confusion.

    Note that single-valued functions are total by definition, which explains why there is no established terminology for the set \( X \).

    We denote the set of all functions from \( X \) to \( Y \) by \( \fun(X, Y) \). We may also use either \( \cat{Set}(X, Y) \) (which is consistent with \fullref{def:category_of_sets}) or by \( Y^X \) (which is consistent with \hyperref[def:cardinal_arithmetic]{cardinal arithmetic}). We abbreviate \( \fun(X, X) \) as \( \fun(X) \)

    \thmitem{def:function/multivalued} We only considered multivalued functions until now. In practice, \enquote{function} usually refers to single-valued functions and we call a function multivalued if it is not single-valued.

    \thmitem{def:function/partial} A \term{partial function} is a generalization of a single-valued function which we allow to not be total, i.e. we allow \( F(x) \) to be an empty set for some \( x \in X \).

    \thmitem{def:function/selection} A \term{selection} of a multivalued function \( F: X \rightrightarrows Y \) is a single-valued function \( f: X \to Y \) such that \( \gph(f) \subseteq \gph(F) \) (see \fullref{def:function/graph}).

    \thmitem{def:function/value} We call the set \( B \) the \term{value} of \( F \) at \( x \) if \( y \in B \iff (x, y) \in f \) and use the notation \( F(x) = B \).

    \thmitem{def:function/set_value} If \( A \subseteq X \), we define the value of \( F \) at \( A \), also called the \term{action} of \( F \) on \( A \) or the \term{image} of \( A \) under \( F \), as
    \begin{equation*}
      F(A) \coloneqq \cup_{a \in A} \{ F(a) \}.
    \end{equation*}

    Even if \( Y \) is a proper class, \( f(X) \) is a set by \fullref{def:zfc/A9}.

    \thmitem{def:function/argument} The notion \term{function arguments} is somewhat informal. If \( X = X_1 \times \cdots \times X_n \) is a finite Cartesian product, we denote the function \( F: X_1 \times \cdots \times X_n \rightrightarrows Y \) by \( F(x_1, \ldots, x_n) \). The variables \( x_1, \ldots, x_n \) in this \hyperref[def:first_order_syntax/term]{term} are called the function's \term{parameters} or \term{arguments} or even \term{independent variables}. In the latter terminology, we say that \( F \) itself is a \term{dependent variable}, depending on the independent variables \( x_1, \ldots, x_n \).

    Note that \( X \) may to be uniquely representable as a Cartesian product, however this does not cause confusion in practice.

    Even for single-argument functions, it is conventional to write \( F(x) \) rather than simply \( F \). This does not cause confusion in practice because it is usually clear from the context whether \( x \) is a \enquote{free} or \enquote{bound} variable. See \fullref{def:first_order_syntax/formula} for a formal justification.
  \end{thmenum}

  The following terminology is consistent with \fullref{def:binary_relation}:
  \begin{thmenum}[resume=def:function]
    \thmitem{def:function/graph} The \term{graph} \( \gph(F) \) of \( F \) is the relation \( F \) itself independent from \( X \) and \( Y \) as defined in \fullref{def:relation/graph}.

    \thmitem{def:function/domain} The \term{domain} \( \dom(F) \) of \( F \) is the set of all values for which \( \dom(F) \neq \varnothing \). This is consistent with \fullref{def:binary_relation/domain}.

    \thmitem{def:function/image} The \term{image} \( \img(F) \) is the set of all \( y \in Y \) that belong to the set \( F(x) \) for at least one \( x \in X \). This is consistent with \fullref{def:binary_relation/image}.

    \thmitem{def:function/range} The \term{range} \( \range(F) \) is the set \( Y \) as defined in \fullref{def:binary_relation/range}.

    \thmitem{def:function/composition} The \term{composition} \( G \circ F \) of two functions \( F: X \rightrightarrows Y \) and \( Y \rightrightarrows Z \) is the function
    \begin{equation*}
      [G \circ F](x) \coloneqq G(F(x)).
    \end{equation*}

    This is consistent with \fullref{def:binary_relation/composition}.

    Note that the composition of single-valued functions is single-valued (see \fullref{def:category_of_sets}).
  \end{thmenum}

  The following terminology is not consistent with \fullref{def:binary_relation}:
  \begin{thmenum}[resume=def:function]
    \thmitem{def:function/total} The term \term{total multivalued function} means that \( \dom(F) = X \), that is, that \( F(x) \neq \varnothing \) for all \( x \in X \). This is not to be confused with \fullref{def:binary_relation/total}.

    \thmitem{def:function/arity} The \term{arity} of a function is its number of \hyperref[def:function/arity]{arguments}. This should not to be confused with \fullref{def:relation/arity}.

    \thmitem{def:function/inverse} The \term{inverse} \( F^{-1}: Y \rightrightarrows X \) of a multivalued function \( F: X \rightrightarrows Y \) is its \hyperref[def:binary_relation/converse]{converse} (rather than \hyperref[def:binary_relation/inverse]{inverse}) relation.

    \thmitem{def:function/diagonal} The function corresponding to the \hyperref[def:binary_relation/diagonal]{diagonal relation} is called the \term{identity}. Instead, the (single-valued) \term{diagonal function} \( f: \mscrX \to X^2 \) is defined as \( f(x) \coloneqq (x, x) \).
  \end{thmenum}

  We define some additional terminology:
  \begin{thmenum}[resume=def:function]
    \thmitem{def:function/involution} If \( F = F^{-1} \), we say that \( F \) is an \term{involution}. See \fullref{def:set_with_involution}.

    \thmitem{def:function/preimage} For a concrete set \( B \subseteq Y \), its \term{large preimage} or simply \term{preimage} under \( F: X \to Y \) is the value of \( B \) under the inverse function \( F^{-1}: Y \rightrightarrows X \). For a single value \( y \in Y \), we call \( F^{-1}(y) \) the \term{fiber} of \( y \) under \( F \).

    We define its \term{small preimage} as \( F_{-1}(B) \coloneqq \{ x \in X \colon F(x) \subseteq B \} \)

    \thmitem{def:function/extension} Let \( X \) and \( Y \) be sets and let \( A \subseteq X \). For the multivalued functions \( F: A \rightrightarrows Y \) and \( G: X \rightrightarrows Y \), we say that \( G \) is an \term{extension} of \( F \) to \( X \) and that \( F \) is a \term{restriction} of \( G \) to \( A \).

    \thmitem{def:function/superposition} Although the terms \enquote{composition} and \enquote{superposition} are used interchangeably (see \cite[44]{Enderton1977Sets} and \cite[\textnumero 25]{Фихтенгольц1968Том1}), \enquote{superposition} usually refers to the following special case of composition:

    If we are given the functions \( F_k: X \rightrightarrows Y_k, k = 1, \ldots, n \) and \( G: Y_1 \times \cdots \times Y_n \rightrightarrows Z \), their \term{superposition} \( H: X \rightrightarrows Z \) is
    \begin{equation*}
      H(x) \coloneqq G(F_1(x), \ldots, F_n(x)).
    \end{equation*}
  \end{thmenum}
\end{definition}

\begin{definition}\label{def:function_invertibility}(Compare with \fullref{def:morphism_invertibility})
  We list equivalent conditions for three types of invertibility:
  \begin{thmenum}
    \thmitem{def:function_invertibility/injection} \( f \) is called \term{injective}, \term{left-invertible}, \term{one-to-one} if any of the following equivalent conditions hold:
    \begin{thmenum}
      \thmitem{def:function_invertibility/injection/points} Different points in \( X \) have different images under \( f \).
      \thmitem{def:function_invertibility/injection/preimage} The preimage of any point in \( Y \) is either empty or a singleton.
      \thmitem{def:function_invertibility/injection/monomorphism} There exists a function \( g: Y \to X \) such that \( g \circ f = \id_X \).
      \thmitem{def:function_invertibility/injection/inverse} The inverse is a single-valued partial function.
    \end{thmenum}

    We sometimes use the \hyperref[def:morphism_invertibility/monomorphism]{monomorphism} notation \( f: X \hookrightarrow Y \).

    \thmitem{def:function_invertibility/surjection} \( f \) is called \term{surjective}, \term{right-invertible} or \term{onto} if any of the equivalent conditions hold:
    \begin{thmenum}
      \thmitem{def:function_invertibility/surjection/points} Each point in \( Y \) is the image of at least one point in \( X \).
      \thmitem{def:function_invertibility/surjection/image} The image of \( f \) equals the range of \( f \).
      \thmitem{def:function_invertibility/surjection/epimorphism} There exists a function \( g: Y \to X \) such that \( f \circ g = \id_Y \).
      \thmitem{def:function_invertibility/surjection/inverse} The inverse is a total multivalued function.
    \end{thmenum}

    We sometimes use the \hyperref[def:morphism_invertibility/epimorphism]{epimorphism} notation \( f: X \twoheadrightarrow Y \).

    \thmitem{def:function_invertibility/bijection} \( f \) is called \term{bijective} or simply \term{invertible} if any of the equivalent conditions hold:
    \begin{thmenum}
      \thmitem{def:function_invertibility/bijection/direct} it is both injective and surjective.
      \thmitem{def:function_invertibility/bijection/points} each point in \( Y \) is the image of exactly one point in \( X \).
      \thmitem{def:function_invertibility/bijection/preimage} the preimage of any point in \( Y \) is a singleton.
      \thmitem{def:function_invertibility/bijection/isomorphism} there exists a function \( g: Y \to X \) such that both \( g \circ f = \id_X \) and \( f \circ g = \id_Y \).
      \thmitem{def:function_invertibility/bijection/inverse} the inverse is a single-valued total function.
    \end{thmenum}

    We sometimes use the \hyperref[def:morphism_invertibility/isomorphism]{isomorphism} notation \( f: X \cong Y \). See also \fullref{def:equinumerous_sets}.
  \end{thmenum}
\end{definition}

\begin{definition}\label{def:endofunction}
  A function from a set to itself is called an \term{endofunction}.
\end{definition}

\begin{definition}\label{def:currying}
  This notation is abused in practice as long as it does not cause confusion.

  Given the two-argument function \( f: A \times B \to C \), we may define
  \begin{equation*}
    g(y)(x) \coloneqq f(x, y).
  \end{equation*}

  Here \( g \) is itself an operator from \( A \) to \( \cat{Set}(B, C) \). This is called \term{currying}, although the latter term is more specific and refers to lambda calculus.

  We often wish to \enquote{fix} some value \( x \), i.e. bind it using modified variable \hyperref[def:first_order_valuation/variable_assignment]{assignment}, so that \( g(y): B \to C \) is well-defined.
\end{definition}

\begin{proposition}\label{thm:function_image_properties}
  Functions images have the following basic properties (compare to \fullref{thm:function_preimage_properties}):
  \begin{thmenum}
    \thmitem{thm:function_image_properties/monotonicity} If \( A \subseteq B \), then \( f(A) \subseteq f(B) \).

    \thmitem{thm:function_image_properties/union} \( f(\bigcup_{k \in \mscrK} X_k) = \bigcup_{k \in \mscrK} f(X_k) \).

    \thmitem{thm:function_image_properties/intersection} \( f(\bigcap_{k \in \mscrK} X_k) \subseteq \bigcap_{k \in \mscrK} f(X_k) \) with equality holding if \( f \) is injective.

    \thmitem{thm:function_image_properties/difference} \( f(A \setminus B) \subseteq f(A) \setminus f(B) \) with equality holding if \( f \) is surjective.
  \end{thmenum}
\end{proposition}
\begin{proof}
  \SubProofOf{thm:function_image_properties/monotonicity} If \( x_0 \in A \), then \( x_0 \in B \) and hence \( f(x_0) \in f(B) \). Therefore \( f(A) \subseteq f(B) \).

  \SubProofOf{thm:function_image_properties/union} If \( x_0 \in X_{k_0} \) for some \( k_0 \in \mscrK \), clearly \( f(x_0) \in f(X_{k_0}) \subseteq \bigcup_{k \in \mscrK} f(X_k) \). Therefore \( f(\bigcup_{k \in \mscrK} X_k) \subseteq \bigcup_{k \in \mscrK} f(X_k) \).

  Conversely, if \( y_0 \in f(X_{k_0}) \) for some \( k_0 \in \mscrK \), by \fullref{thm:function_image_properties/monotonicity} obviously \( y_0 \in f\left( \bigcup_{k \in \mscrK} X_k \right) \). Therefore \( f(\bigcup_{k \in \mscrK} X_k) \supseteq \bigcup_{k \in \mscrK} f(X_k) \).

  \SubProofOf{thm:function_image_properties/intersection} If \( x_0 \in \bigcap_{k \in \mscrK} X_{k} \), then \( x_0 \in X_k \) for all \( k \in \mscrK \). We have \( f(x_0) \in f(X_k) \) for all \( k \in \mscrK \), therefore \( f(\bigcap_{k \in \mscrK} X_k) \subseteq \bigcap_{k \in \mscrK} f(X_k) \).

  Conversely, if \( f \) is injective and \( y_0 \in f(X_k) \) for all \( k \in \mscrK \), then there exists a unique \( x_0 \in \bigcap_{k \in X_k} \) such that \( f(x_0) = y_0 \). Therefore \( f(\bigcap_{k \in \mscrK} X_k) \supseteq \bigcap_{k \in \mscrK} f(X_k) \).

  \SubProofOf{thm:function_image_properties/difference} If \( x_0 \in A \) and \( x_0 \not\in B \), then \( f(x_0) \in f(A) \setminus f(B) \). Therefore \( f(A \setminus B) \subseteq f(A) \setminus f(B) \).

  Conversely, suppose that \( f \) is surjective. For \( y_0 \in f(A) \setminus f(B) \) there exists a \( x_0 \in A \) such that \( f(x_0) = y_0 \in f(A) \setminus B \). Since \( y_0 \not\in f(B) \), and, by surjectivity, \( y_0 \) has the preimage of \( y_0 \) has only one member \( x_0 \), we conclude that \( x_0 = \not\in f(B) \). Therefore \( f(A \setminus B) \supseteq f(A) \setminus f(B) \).
\end{proof}

\begin{proposition}\label{thm:function_preimage_properties}
  Functions \hyperref[def:function/preimage]{preimages} have the following basic properties (compare to \fullref{thm:function_image_properties}):
  \begin{thmenum}
    \thmitem{thm:function_preimage_properties/monotonicity} If \( A \subseteq B \), then \( f^{-1}(A) \subseteq f^{-1}(B) \).

    \thmitem{thm:function_preimage_properties/union} \( f^{-1}(\bigcup_{k \in \mscrK} Y_k) = \bigcup_{k \in \mscrK} f^{-1}(Y_k) \).

    \thmitem{thm:function_preimage_properties/intersection} \( f^{-1}(\bigcap_{k \in \mscrK} Y_k) = \bigcap_{k \in \mscrK} f^{-1}(Y_k) \).

    \thmitem{thm:function_preimage_properties/difference} \( f^{-1}(A \setminus B) = f^{-1}(A) \setminus f^{-1}(B) \).
  \end{thmenum}
\end{proposition}
\begin{proof}
  \SubProofOf{thm:function_image_properties/monotonicity} Analogous to \fullref{thm:function_image_properties/monotonicity}.

  \SubProofOf{thm:function_image_properties/union} Analogous to \fullref{thm:function_image_properties/union}.

  \SubProofOf{thm:function_image_properties/intersection} If \( y_0 \in \bigcap_{k \in \mscrK} Y_{k} \), then \( y_0 \in Y_k \) for all \( k \in \mscrK \). We have \( f^{-1}(y_0) \in f^{-1}(Y_k) \) for all \( k \in \mscrK \), therefore \( f^{-1}(\bigcap_{k \in \mscrK} Y_k) \subseteq \bigcap_{k \in \mscrK} f^{-1}(Y_k) \).

  Conversely, if \( x_0 \in f^{-1}(Y_k) \) for all \( k \in \mscrK \), then, since \( f \) is a function, there exists a unique \( y_0 \in \bigcap_{k \in Y_k}Y_k \) such that \( f^{-1}(y_0) = x_0 \). Therefore \( f^{-1}(\bigcap_{k \in \mscrK} X_k) \supseteq \bigcap_{k \in \mscrK} f^{-1}(X_k) \).

  \SubProofOf{thm:function_image_properties/difference} If \( y_0 \in A \) and \( y_0 \not\in B \), then \( f^{-1}(y_0) \in f^{-1}(A) \setminus f^{-1}(B) \). Therefore \( f^{-1}(A \setminus B) \subseteq f^{-1}(A) \setminus f^{-1}(B) \).

  Conversely, for \( x_0 \in f^{-1}(A) \setminus f^{-1}(B) \), since \( f \) is a function, there exists a \( y_0 \in A \) such that \( f^{-1}(y_0) = x_0 \in f^{-1}(A) \setminus f^{-1}(B) \). Since \( x_0 \not\in f^{-1}(B) \), we conclude that \( x_0 = \not\in B \). Therefore \( f^{-1}(A \setminus B) \supseteq f^{-1}(A) \setminus f^{-1}(B) \).
\end{proof}

\begin{proposition}\label{thm:function_image_preimage_composition}
  \hfill
  \begin{thmenum}
    \thmitem{thm:function_image_preimage_composition/image_first} \( A \subseteq f^{-1}(f(A)) \) with equality holding if \( f \) is injective.
    \thmitem{thm:function_image_preimage_composition/preimage_first} \( f(f^{-1}(A)) \subseteq A \) with equality holding if \( f \) is surjective
  \end{thmenum}
\end{proposition}
\begin{proof}
  \SubProofOf{thm:function_image_preimage_composition/image_first} Equality obviously holds unless the image \( f(A) \) of \( A \) contains other points except those in \( A \). In this case, \( f^{-1}(f(A)) \) may contain those points in addition to the points of \( A \). If \( f \) is injective, however, no such additional points are possible and equality indeed holds.

  \SubProofOf{thm:function_image_preimage_composition/preimage_first} Equality obviously holds unless \( A \) contains points that do not belong to the image \( \imag f \). If \( f \) is surjective, however, all point in \( A \) have preimages and equality indeed holds.
\end{proof}

\begin{definition}\label{def:indexed_family}
  When considering finite families of sets, it is enough to consider n-tuples. For example, given sets \( X_1, \ldots, X_n \), we can think of the family \( \{ X_k \}_k \) as the ordered tuple
  \begin{equation*}
    (X_1, \ldots, X_n)
  \end{equation*}
  where the \( k \)-th coordinate of the tuple gives us the \( k \)-th set of the family.

  This approach has two flaws:
  \begin{itemize}
    \item The family \term{must} be ordered since the natural numbers are ordered. Families of sets often have no obvious ordering.
    \item The family \term{must} be at most countable.
  \end{itemize}

  A more natural approach to indexed families is given by functions. We choose an arbitrary set \( \mscrK \), called the \term{index set}. Every function \( f: \mscrK \to \mathcal C \) from \( \mscrK \) into some class \( \mathcal C \) of sets is then called an \term{indexed family}. The function \( f \) maps every element \( k \) of \( \mscrK \) into a set \( X_k \coloneqq f(k) \). For convenience, this family is denoted as
  \begin{equation*}
    \{ X_k \}_{k \in \mscrK}.
  \end{equation*}

  We will write \( \{ X_k \}_{k \in \mscrK} \subseteq \mathcal{C} \), despite the net actually being an \( \mscrK \)-shaped generalized \hyperref[def:generalized_element]{element} of \( \mathcal{C} \) rather than a subset. See \fullref{rem:indexed_family_notation} for further discussion of the notation.

  A more general framework than indexed families that also considers relations between the family's elements is given by diagrams in category \hyperref[def:categorical_diagram]{theory}.
\end{definition}

\begin{example}\label{ex:indexed_families}
  \hfill
  \begin{thmenum}
    \item Every n-tuple \( (x_1, \ldots, x_n) \) is an indexed family with domain \( \mscrK = \{ 1, \ldots, n \} \).

    \item An important corner case is when \( \mscrK \) is the empty set. Since the only possible indexing function is then the empty function, we simply say that the resulting family is empty.

    \item In continuous stochastic processes, it is convenient to consider families of random variables \( \{ X_t \}_{t \geq 0} \) indexed by \( \mscrK = \BbbR^+ \). The indexing parameter is often denoted by \( t \geq 0 \) is often interpreted as time.

    \item An \( n \times m \) \hyperref[def:array/matrix]{matrix} \( A = \{ a_{i,j} \} \) is a family of scalars indexed by the unordered set \( \mscrK = \{ 1, \ldots, n \} \times \{ 1, \ldots, m \} \).

    \item \hyperref[def:topological_net]{Nets} is topology are indexed families where the domain is a directed \hyperref[def:directed_set]{set}.
  \end{thmenum}
\end{example}

\begin{definition}\label{def:sequence}
  A \term{sequence} \( \{ X_k \}_{k=1}^\infty \) is an indexed family with domain \( \mscrK = \BbbN \). Sometimes finite \( n \)-tuples are referred to as \term{finite sequences}, in which case the usual sequences are referred to as \term{infinite sequences}. See \fullref{def:topological_net}.

  We say that \( \{ X_{k_m} \}_{k=1}^\infty \) is a \term{subsequence} of \( \{ X_k \}_{k=1}^\infty \) if the sequence \( \{ k_m \}_{k=1}^\infty \) of positive integers is strictly monotone.

  Subsequences of \( \{ X_k \}_{k=1}^\infty \) are usually denoted by adding another index as a subscript, i.e. \( \{ x_{k_m} \}_{k=1}^\infty \).
\end{definition}

\begin{remark}\label{rem:indexed_family_notation}
  Since we denote \hyperref[def:cartesian_product]{tuples} as \( (x_1, \ldots, x_n) \), it is consistent to denote indexed \hyperref[def:indexed_family]{families} by
  \begin{equation*}
    ( X_k )_{k \in \mscrK}
  \end{equation*}
  instead of
  \begin{equation*}
    \{ X_k \}_{k \in \mscrK}.
  \end{equation*}

  This is actually done when we want to enumerate elements of a sequence, e.g. see \fullref{def:polynomial}.

  In general, however, we prefer the latter notation because
  \begin{equation*}
    \left\{ \log \left( f^{(n)}(x_k) \right) \right\}_{k=1}^\infty.
  \end{equation*}
  is both more conventional (in analysis) and more aesthetically pleasing than
  \begin{equation*}
    \left( \log \left( f^{(n)}(x_k) \right) \right)_{k=1}^\infty
  \end{equation*}

  The difference may be more visible in simpler cases like
  \begin{balign*}
    (\sin(k))_{k \in \mscrK}
     &  &
    \{\sin(k)\}_{k \in \mscrK}.
  \end{balign*}
\end{remark}

\begin{definition}\label{def:family_of_functions_separates_points}
  Let \( \mathcal{F} \) be a family of functions between the sets \( A \) and \( B \). We say that \( \mathcal{F} \) \term{separates points} if for every two points \( x, y \in A \) there exists a function \( f \in \mathcal{F} \) such that \( f(x) \neq f(y) \).
\end{definition}

\begin{definition}\label{def:symmetric_function}
  Fix arbitrary sets \( X \) and \( Y \). A function \( f: X \times X \to Y \) is called \term{symmetric} if, for all \( x, y \in X \), we have
  \begin{equation*}
    f(x, y) = f(y, x).
  \end{equation*}

  Symmetric functions should not be confused with \hyperref[def:derived_relations/symmetric]{symmetric relations}.
\end{definition}

\begin{definition}\label{def:fixed_point}
  Given a function \( f: A \to B \), we call \( x \in A \) a \term{fixed point} of \( f \) if \( x = f(x) \).
\end{definition}

\begin{definition}\label{def:cartesian_product}
  Let \( \set{ A_k }_{k \in \mscrK} \) be a nonempty \hyperref[def:indexed_family]{family} of arbitrary sets.

  Their \term{Cartesian product} is
  \begin{equation*}
    \prod_{k \in \mscrK} A_k \coloneqq \set*{ f: \mscrK \to \bigcup_{k \in \mscrK} X_k \given* \qforall{m \in \mscrK} f(m) \in A_m }.
  \end{equation*}

  The definition makes sense when any of the sets is empty because the product itself is then empty.

  Any element of the Cartesian product is called a \term{tuple}. In particular, tuples of length two are bijective with the \hyperref[def:binary_cartesian_product]{Kuratowski pairs}.
\end{definition}

\begin{definition}\label{def:disjoint_union}
  The \term{disjoint union} of the \hyperref[def:indexed_family]{family} \( \set{ A_k }_{k \in \mscrK} \) of nonempty sets is
  \begin{equation*}
    \coprod_{k \in \mscrK} A_k \coloneqq \set{ (k, x) \given k \in \mscrK \T{and} x \in X_k }.
  \end{equation*}
\end{definition}

\subsection{Cardinality}\label{subsec:cardinality}

\begin{theorem}[Well-Ordering Principle]\label{thm:well_ordering_principle}\mcite[196]{Enderton1977Sets}
  Any \hyperref[def:set_zfc]{set} can be \hyperref[def:well_ordered_set]{well-ordered}.

  This theorem is equivalent to \fullref{thm:aoc}.
\end{theorem}

\begin{definition}\label{def:set_domination}\mcite[145]{Enderton1977Sets}
  We say that the set \( X \) is \term{dominated by \( Y \)} and write \( \abs{X} \leq \abs{Y} \) if there exists an \hyperref[def:function_invertibility/injection]{injection} from \( X \) to \( Y \).
\end{definition}

\begin{definition}\label{def:equinumerous_sets}\mcite[129]{Enderton1977Sets}
  We say that the sets \( X \) and \( Y \) are \term{equinumerous} and write \( X \cong Y \) if there exists a \hyperref[def:function_invertibility/bijection]{bijection} between \( X \) and \( Y \).
\end{definition}

\begin{theorem}[Cantor-Schröder-Bernstein]\label{thm:cantor_schroder_bernstein}\mcite[147]{Enderton1977Sets}
  If \( \abs{X} \leq \abs{Y} \) and \( \abs{Y} \leq \abs{X} \), then \( X \cong Y \).
\end{theorem}

\medskip

\begin{proposition}\label{thm:equinumerousity_equivalence}\mcite[thm. 6A]{Enderton1977Sets}
  \hyperref[def:equinumerous_sets]{Equinumerosity} satisfies the equivalence relation \hyperref[def:equivalence_relation]{axioms} (however it is formally not an equivalence relation since we cannot define relations on the class of all sets; see \fullref{def:set_zfc}).
\end{proposition}

\begin{theorem}[Cantor]\label{thm:cantor_power_set_theorem}\mcite[thm. 6B]{Enderton1977Sets}
  No set \( X \) is equinumerous with its power set \( \pow(X) \).
\end{theorem}
\begin{proof}
  Fix some function \( f: X \to \pow(X) \). Define the set
  \begin{equation*}
    Y \coloneqq \{ x \in X \colon x \not\in f(x) \}.
  \end{equation*}

  Note that \( Y \subseteq X \) and thus \( Y \in \pow(X) \), however \( Y \) is not in the \hyperref[def:function]{image} \( \img f \) and thus \( f \) is not a \hyperref[def:function_invertibility/surjection]{surjection}.

  Since \( f \) was arbitrary, we conclude that no function \( f: X \to \pow(X) \) is a surjection and, hence, \( X \not\cong \pow(X) \).
\end{proof}

\begin{definition}\label{def:finite_set}\mcite[133]{Enderton1977Sets}
  A set \( A \) is \term{finite} if it is equinumerous to a natural number as defined in \fullref{def:natural_numbers} (using the convention that \( \varnothing \) corresponds to zero).

  If \( A \) is not finite, we say that it is \term{infinite}.
\end{definition}

\begin{proposition}\label{thm:infinite_set_iff_equinumerous_to_proper_subset}\mcite[cor. 6D]{Enderton1977Sets}
  A set is \hyperref[def:finite_set]{infinite} if and only if it is equinumerous to a proper subset of itself.
\end{proposition}

\medskip

\begin{theorem}\label{thm:equinumerous_ordinal_existence}\mcite[197]{Enderton1977Sets}
  For every set, there exists at least one \hyperref[def:ordinal]{ordinal} equinumerous to it.
\end{theorem}

\medskip

\begin{definition}\label{def:cardinal}\mcite[197]{Enderton1977Sets}
  For each set \( A \), define its \term{cardinal} or \term{cardinal number} \( \card A \) as the smallest ordinal that is equinumerous to \( A \) (a smallest ordinal always exists by \fullref{def:ordinal}).

  If \( \xi \) and \( \eta \) are cardinal numbers, we define \( \xi \leq \eta \) to mean that \( \eta \) \hyperref[def:set_domination]{dominates} \( \xi \), i.e.
  \begin{equation*}
    \xi \leq \eta \iff \abs{\xi} \leq \abs{\eta}.
  \end{equation*}
\end{definition}

\begin{theorem}\label{thm:cardinal_trichotomy}\mcite[thm. 6M(5)]{Enderton1977Sets}
  If \( \xi \) and \( \eta \) are cardinals, then either
  \begin{itemize}
    \item \( \xi \leq \eta \)
    \item \( \xi = \eta \)
    \item \( \xi \geq \eta \)
  \end{itemize}
\end{theorem}

\begin{corollary}[Pigeonhole principle]\label{def:pigeonhole_principle}
  If the cardinality of \( X \) is greater than the cardinality of \( Y \), then there exists no injective function from \( X \) to \( Y \).
\end{corollary}

\begin{remark}\label{rem:cardinals}
  We can think of cardinal numbers as \enquote{choosing}\AOC a special set out of the equivalence classes obtained from \fullref{thm:equinumerousity_equivalence}.

  Since the natural numbers as defined in \fullref{def:natural_numbers} are ordinals and no two different natural numbers are equinumerous, we identify the cardinal numbers for \hyperref[def:finite_set]{finite sets} with natural numbers.

  We give special names to
  \begin{thmenum}
    \thmitem{rem:cardinals/countably_infinite} \( \aleph_0 \coloneqq \card(\omega) \), the \term{cardinality of the natural numbers}.
    \thmitem{rem:cardinals/continuum} \( c \coloneqq \card(\BbbR) = \card(\pow(\omega)) \), the \term{cardinality of the continuum}.
  \end{thmenum}
\end{remark}

\begin{proposition}\label{thm:cardinals_well_ordered}
  The class of all \hyperref[def:cardinal]{cardinals} is \hyperref[def:well_ordered_set]{well-ordered}, that is, every set of cardinals has a least element.
\end{proposition}
\begin{proof}
  Direct consequence of \fullref{thm:ordinal_properties/set_of_ordinals_has_minimum} and \fullref{def:cardinal}.
\end{proof}

\begin{conjecture}[Continuum hypothesis]\label{hyp:continuum_hypothesis}\mcite[165]{Enderton1977Sets}
  There exists no cardinal \( \xi \) such that \( \aleph_0 < \xi < c \).
\end{conjecture}

\medskip

\begin{remark}\label{rem:continuum_hypothesis}\mcite[165]{Enderton1977Sets}
  \Fullref{hyp:continuum_hypothesis} has been shown by G\"odel to not be disprovable in \hyperref[def:set_zfc]{ZFC} and by Cohen to not be provable in ZFC.
\end{remark}

\begin{definition}\label{def:cardinal_arithmetic}
  Let \( \xi \) and \( \eta \) be cardinal numbers. We define
  \begin{thmenum}
    \thmitem{def:cardinal_arithmetic/addition}(addition) \( \xi + \eta \coloneqq \card(\xi \coprod \eta) \), where \( \coprod \) denotes disjoint \hyperref[def:disjoint_union]{unions}.
    \thmitem{def:cardinal_arithmetic/multiplication}(multiplication) \( \xi \cdot \eta \coloneqq \card(\xi \times \eta) \)
    \thmitem{def:cardinal_arithmetic/exponentiation}(exponentiation) \( \xi^\eta \coloneqq \card(\cat{Set}(\eta, \xi)) \) (see \fullref{def:category_of_sets})
  \end{thmenum}
\end{definition}

\begin{proposition}\label{thm:countable_union_of_countable_sets}\mcite[thm. 6Q]{Enderton1977Sets}
  A countable union of countable sets is countable.
\end{proposition}


% Category theory
\subsection{Categories}\label{subsec:categories}

\begin{definition}\label{def:category}\mcite[def. 1.1.1]{Leinster2016Basic}
  A \term{category} is a \hyperref[def:quiver]{quiver} \( \cat{C} \) equipped with a \hyperref[def:partial_function]{partial operation} \( \bincirc \) on the arrows of \( \cat{C} \) and another operation \( \id \) that selects a distinguished arrow for each vertex.

  In tradition regarding \hyperref[def:concrete_category]{forgetful functors}, we denote the underlying quiver of \( \cat{C} \) by \( U(\cat{C}) \).

  \begin{thmenum}[series=def:category]
    \thmitem{def:category/objects} We call the vertices of the quiver \term{objects} and denote the set of all objects by \( \obj(\cat{C}) \). We will often write \( A \in \cat{C} \) as a shorthand for \( A \in \obj(\cat{C}) \).

    \thmitem{def:category/morphisms} We call the arrows of the quiver \term{morphisms} or sometimes \term{maps}. If \( f \) is a morphism, we call its head its \term{domain} \( \dom(f) \) and its tail its \term{codomain} \( \co\dom(f) \). We denote a morphism from \( A \) to \( B \) by \( f: A \to B \) or \( A \reloset f \to B \).

    We call the set \( \cat{C}(A, B) \) of all morphisms from \( A \) to \( B \) a \term{morphism set} or \term{\( \hom \)-set}. We use the shorthand \( \cat{C}(A) \) for \( \cat{C}(A, A) \). Another established notation is \( \op{hom}(A, B) \) instead of \( \cat{C}(A, B) \).

    Both of these notations highlight that \( \cat{C}(A, B) \), when parameterized by \( A \) and \( B \), is a \hyperref[def:functor]{functor}, as discussed in \fullref{def:hom_functor}.

    \thmitem{def:category/composition} We require the \term{composition} \( \bincirc \) of the arrows \( f \) and \( g \) to be defined only if \( \co\dom(f) = \dom(g) \). In this case, we require \( g \bincirc f \) to be a morphism from \( \dom(f) \) to \( \co\dom(g) \).

    Note how the order of \( f \) and \( g \) may seem confusing: we write the composition of \( f: A \to B \) and \( g: B \to C \) as \( g \bincirc f: A \to C \). This is set up so that it matches \hyperref[def:multi_valued_function/composition]{function composition}. The order may seem different compared to multiplication in \hyperref[def:group]{groups}, for example, however \fullref{def:monoid_delooping} shows that this is actually a generalization of multiplication.

    This order of composition is used in \cite[7]{MacLane1994}, \cite[def. 1.1.1]{Leinster2016Basic} and \cite[def 3.1.]{Aluffi2009}.

    \thmitem{def:category/identity} We denote the \term{identity morphism} of an object \( A \) by \( \id_A \).
  \end{thmenum}

  The definition of a category additionally requires the following conditions to hold:
  \begin{thmenum}[resume=def:category]
    \thmitem[def:category/C1]{C1} For any morphism \( f: A \to B \), the identities \( \id_A \) and \( \id_B \) must satisfy
    \begin{equation}\label{eq:def:category/C1}\tag{\logic{C1}}
      f \bincirc \id_A = \id_B \bincirc f = f.
    \end{equation}

    \thmitem[def:category/C2]{C2} Composition must be associative. That is, for each triple of morphism \( f: A \to B \), \( g: B \to C \) and \( h: C \to D \), the following must hold:
    \begin{equation}\label{eq:def:category/C2}\tag{\logic{C2}}
      (h \bincirc g) \bincirc f = h \bincirc (g \bincirc f).
    \end{equation}
  \end{thmenum}
\end{definition}

\begin{example}\label{ex:def:category}
  Examples of categories include:

  \begin{itemize}
    \item The category \( \cat{Set} \) of \hyperref[def:large_and_small_sets]{small} \hyperref[def:set]{sets} and \hyperref[def:function]{functions} defined in \fullref{def:category_of_small_sets}.

    \item The category \( \cat{Cat} \) of small categories defined in \fullref{def:category_of_small_categories}.

    \item All the \hyperref[def:category_of_small_first_order_models]{categories of small first-order models} listed in \fullref{ex:def:category_of_small_first_order_models}

    \item The category \( \cat{Top} \) of small \hyperref[def:topological_space]{topological spaces} and \hyperref[def:global_continuity]{continuous functions} defined in \fullref{def:category_of_small_topological_spaces}.

    \item For every topological space, the fundamental groupoid defined in \fullref{def:fundamental_groupoid}.

    \item The category \( \cat{Quiv} \) of small \hyperref[def:quiver]{quivers} defined in \fullref{def:category_of_small_quivers}.

    \item For every quiver, the free category defined in \fullref{def:quiver_free_category}.

    \item For every \hyperref[def:preordered_set]{preordered set}, the induced category defined in \fullref{thm:order_category_isomorphism}.
  \end{itemize}
\end{example}

\begin{definition}\label{def:category_size}
  As can be seen from \fullref{ex:def:category}, some of the categories we are working with, like \( \cat{Set} \), contain as objects all \hyperref[def:large_and_small_sets]{small sets}. As mentioned in \fullref{def:large_and_small_sets}, the concept of a small set is defined relative to the smallest Grothendieck universe that suits our needs.

  \Fullref{thm:russels_paradox} demonstrates that the set of all sets easily leads to a paradox, which is the reason we restrict our attention only to sets within some Grothendieck universe. This universe is implicit by default, however we will occasionally need to make it explicit.

  We will say that the category \( \cat{C} \) is \term{locally \( \mscrU \)-small} if the morphism set \( \cat{C}(A, B) \) is \( \mscrU \)-small for every pair of objects \( A \) and \( B \). If, in addition, the set \( \obj(\cat{C}) \) of objects is also \( \mscrU \)-small, we will say that the category \( \cat{C} \) is \term{\( \mscrU \)-small}. If a category is not \( \mscrU \)-small, we say that it is \term{\( \mscrU \)-large}.

  In particular, \term{finite} and \term{locally finite} categories are ones who are \( V_\omega \)-small and \( V_\omega \)-locally small for the universe of hereditary finite sets \hyperref[def:universe_of_hereditary_finite_sets]{\( V_\omega \)}. This notion of local finiteness is unrelated to local finiteness of graphs defined in \fullref{def:hypergraph/degree}.

  Universes are crucial to be able to do a lot of categorical constructions within set theory, most importantly \( \mscrU \)-large \hyperref[def:functor_category]{functor categories} but also \hyperref[def:product_category]{product categories} and, as discussed in \fullref{rem:functor_size}, even the \hyperref[def:functor]{functors} themselves.

  Note that, even if a category is \( \mscrU \)-small, the category itself as a tuple \( (\mscrQ, \bincirc, \id) \) (see \fullref{def:category}) may not be a \( \mscrU \)-small set.

  Also note that, in a locally small category, it is possible for the set of all morphisms to be \( \mscrU \)-large. This is impossible for small categories due to \ref{def:grothendieck_universe/union}.

  We sometimes skip the prefix \enquote{\( \mscrU \)-} if it is unimportant, and simply speak of \enquote{large categories} or \enquote{locally small categories}.
\end{definition}

\begin{definition}\label{def:category_of_small_sets}
  Suppose that we are given a \hyperref[def:grothendieck_universe]{Grothendieck universe} \( \mscrU \), which is safe to assume to be the smallest suitable one as explained in \fullref{def:large_and_small_sets}.

  We denote the \hyperref[def:category]{category} of \( \mscrU \)-small \hyperref[def:set]{sets} by \( \ucat{Set} \) or, if the universe is clear from the context, simply by \( \cat{Set} \). See \fullref{def:category_size} for a further discussion of universes and categories.

  \begin{itemize}
    \item The \hyperref[def:category/objects]{set of objects} \( \obj(\cat{Set}) \) is the set of all \( \mscrU \)-small sets, i.e. all members of \( \mscrU \).

    \item The \hyperref[def:category/morphisms]{set of morphisms} \( \cat{Set}(A, B) \) from \( A \) to \( B \) is the set \hyperref[def:function/set_of_functions]{\( \fun(A, B) \)} of all total single-valued functions from \( A \) to \( B \).

    \item The \hyperref[def:category/composition]{composition of morphisms} is the usual \hyperref[def:multi_valued_function/composition]{function composition}.

    \item The \hyperref[def:category/identity]{identity morphism} on the set \( A \) is the \hyperref[def:multi_valued_function/identity]{identity function}
    \begin{equation*}
      \begin{aligned}
        &\id_A: A \to A \\
        &\id_A(x) \coloneqq A.
      \end{aligned}
    \end{equation*}
  \end{itemize}
\end{definition}
\begin{defproof}
  To see that \( \ucat{Set} \) is indeed a category, we verify the conditions \ref{def:category/C1} and \ref{def:category/C2}.

  \SubProofOf{def:category/C1} For every two sets \( A, B \in \mscrU \) and every function \( f: A \to B \), for all \( x \in A \) we have
  \begin{equation*}
    [\id_B \bincirc f](x)
    =
    \id_B(f(x))
    =
    f(x)
    =
    f(\id_A(x))
    =
    [f \bincirc \id_A](x).
  \end{equation*}

  Therefore, \( \id_A \) and \( \id_B \) satisfy \eqref{eq:def:category/C1}.

  \SubProofOf{def:category/C2} Associativity of function composition is proved in \fullref{thm:def:multivalued_function/properties/associative}.
\end{defproof}

\begin{proposition}\label{thm:category_of_small_sets_properites}
  We collect here important properties of the category \hyperref[def:category_of_small_sets]{\( \ucat{Set} \)} of \( \mscrU \)-small sets. Most of them require forward references.

  \begin{thmenum}
    \thmitem{thm:category_of_small_sets_properites/large} It is a \( \mscrU \)-large category in the sense of \fullref{def:category_size} because \( \mscrU \) itself is the set of objects and, defined as a \hyperref[def:quiver]{quiver} with additional operations, the category is a \( \mscrU \)-large set in the sense of \fullref{def:large_and_small_sets}.

    \thmitem{thm:category_of_small_sets_properites/locally_small} It is a \hyperref[def:category_size]{\( \mscrU \)-locally small category} because \( \mscrU \) is a model of \hyperref[def:zfc]{\( \logic{ZFC} \)} and \fullref{thm:zfc_existence_theorems/set_of_functions} holds.

    \thmitem{thm:category_of_small_sets_properites/morphism_invertibility} All \hyperref[def:morphism_invertibility/right_cancellative]{epimorphisms} and \hyperref[def:multi_valued_function/empty]{nonempty} \hyperref[def:morphism_invertibility/left_cancellative]{monomorphisms} \hyperref[def:morphism_invertibility/left_invertible]{split} and are precisely the \hyperref[def:function_invertibility/surjective]{surjective} and nonempty \hyperref[def:function_invertibility/injective]{injective functions}, respectively.

    This is stated in \fullref{thm:function_invertibility_categorical}. See also \fullref{thm:epimorphisms_split_in_set}.

    \thmitem{thm:category_of_small_sets_properites/universal_objects} The empty set \( \varnothing \) is an \hyperref[def:universal_objects/initial]{initial object} and the singleton set \( \set{ A } \) is a \hyperref[def:universal_objects/terminal]{terminal object} for every \( A \in \ucat{Set} \). No \hyperref[def:universal_objects/zero]{zero objects} exist in \( \ucat{Set} \) by \fullref{thm:def:universal_objects/properties/no_zero}.

    This is discussed in \fullref{ex:def:universal_objects}.

    \thmitem{thm:category_of_small_sets_properites/discrete_category} The \hyperref[def:discrete_category]{discrete category} functor \( D: \ucat{Set} \to \ucat{Cat} \) is left adjoint to the forgetful functor \( U: \ucat{Cat} \to \ucat{Set} \)

    This is discussed in \fullref{ex:def:category_adjunction/set_cat}.

    \thmitem{thm:category_of_small_sets_properites/limits} The \hyperref[def:discrete_category_limits]{products} and \hyperref[def:discrete_category_limits]{coproducts} are the \hyperref[def:cartesian_product/product]{Cartesian products} and the \hyperref[def:disjoint_union]{disjoint unions}, respectively.

    This is stated in \fullref{thm:discrete_category_limits_in_set}.
  \end{thmenum}
\end{proposition}

\begin{definition}\label{def:dual_category}\mcite[def. 1.1.9]{Leinster2016Basic}
  The \term{opposite} or \term{dual} category of \( \cat{C} \) is obtained by \enquote{reversing} all arrows. This reversing is merely a relabeling of the domain and codomain --- the underlying morphisms are the same. This concept is quite powerful because it allows performing constructions and proofs by duality --- see \fullref{thm:categorical_principle_of_duality}.

  Formally, the category \( \cat{C}^{\opcat} \) is defined as follows:
  \begin{itemize}
    \item The \hyperref[def:category/objects]{set of objects} \( \obj(\cat{C}^{\opcat}) \) is the set of objects \( \obj(\cat{C}) \) of \( \cat{C}^{\opcat} \).

    \item The \hyperref[def:category/morphisms]{set of morphisms} \( \cat{C}(A, B) \) is the set \( \cat{C}(B, A) \). Thus, any morphism \( f^{\opcat}: A \to B \) in the dual category \( \cat{C}^{\opcat} \) is a morphism \( f: B \to A \) in \( \cat{C}^{\opcat} \).

    The superscript here is used solely to distinguish between \( f \) being regarded as a morphism of \( \cat{C} \) and of \( \cat{C}^{\opcat} \) --- the morphisms in \( \cat{C} \) are exactly those of \( \cat{C}^{\opcat} \), simply relabeled.

    \item The \hyperref[def:category/composition]{composition of the morphisms}
    \begin{align*}
      f^{\opcat} &\in \cat{C}^{\opcat}(A, B) = \cat{C}(B, A) \\
      g^{\opcat} &\in \cat{C}^{\opcat}(B, C) = \cat{C}(C, B)
    \end{align*}
    is the morphism
    \begin{equation*}
      \underbrace{g^{\opcat} \bincirc f^{\opcat}}_{\cat{C}^{\opcat}(A, C)} \coloneqq \underbrace{f \bincirc g}_{\cat{C}(C, A)}.
    \end{equation*}

    \item The \hyperref[def:category/identity]{identity morphism} on the object \( A \in \cat{C} \) is again \( \id_A \).
  \end{itemize}
\end{definition}

\begin{remark}\label{rem:double_dual_category}
  The double-dual of a category or morphism is obviously the original. This is made precise with the oppositization functor defined in \fullref{def:dual_functor}.
\end{remark}

\begin{example}\label{ex:def:dual_category}
  A morphism \( f^{\opcat}: A \to B \) in the category \( \cat{Set}^{\opcat} \) is a function from the set \( B \) to the set \( A \). We cannot apply \( f \) to a point in \( B \) unless \( B \subseteq A \). Thus, we cannot regard, in general, the morphism \( f^{\opcat} \) as a function, although only the \hyperref[def:multi_valued_function]{signature} of \( f \) is different from that of \( f^{\opcat} \) --- their \hyperref[def:multi_valued_function/graph]{graphs} are the same.
\end{example}

\begin{proposition}\label{thm:categorical_principle_of_duality}
  We can extend the principle of duality for preordered sets discussed in \fullref{def:preordered_set/duality} to categories. Since we have defined categories in \hyperref[def:axiom_of_universes]{\logic{ZFC+U}} rather than as a first-order theory, we will state this principle informally:
  \begin{displayquote}
    If a statement holds for every category, its dual statement obtained by \enquote{reversing} all morphisms as in \fullref{def:dual_category}, also holds for every category.
  \end{displayquote}

  See \fullref{thm:def:morphism_invertibility/properties/split_epimorphism} for how this principle can be utilized easily.

  We list here results that heavily utilize this principle. Note that it is now always obvious what exactly needs to reversed in order for this principle to hold. For example, as discussed in \fullref{def:dual_functor}, for dual functors we have
  \begin{equation*}
    [F \bincirc G]^{\opcat} = F^{\opcat} \bincirc G^{\opcat},
  \end{equation*}
  which is somewhat unexpected.

  \begin{thmenum}
    \thmitem{thm:categorical_principle_of_duality/morphism_invertibility} \Fullref{thm:morphism_invertibility_duality}: A morphism \( f: A \to B \) in \( \cat{C} \) is a \hyperref[def:morphism_invertibility/left_invertible]{(split)} \hyperref[def:morphism_invertibility/left_cancellative]{monomorphism} if and only if \( f^{\opcat}: B \to A \) in the dual category \( \cat{C}^{\opcat} \) is a \hyperref[def:morphism_invertibility/right_invertible]{(split)} \hyperref[def:morphism_invertibility/right_cancellative]{epimorphism}.

    In particular, \( f \) is an \hyperref[def:morphism_invertibility/isomorphism]{isomorphism} in \( \cat{C} \) if and only if \( f^{\opcat} \) is an isomorphism in \( \cat{C}^{\opcat} \).

    \thmitem{thm:categorical_principle_of_duality/universal_objects} \Fullref{thm:universal_object_duality}: An object is \hyperref[def:universal_objects/initial]{initial} if and only if it is a \hyperref[def:universal_objects/terminal]{terminal object} the \hyperref[def:dual_category]{dual category}.

    \thmitem{thm:categorical_principle_of_duality/functor_categories} \Fullref{thm:dual_functor_category}: For the \hyperref[def:dual_category]{dual} of the \hyperref[def:functor_category]{functor category} \( [\cat{C}, \cat{D}] \) we have
    \begin{equation*}
      [\cat{C}, \cat{D}]^{\opcat} = [\cat{C}^{\opcat}, \cat{D}^{\opcat}].
    \end{equation*}

    \thmitem{thm:categorical_principle_of_duality/equivalences} \Fullref{thm:opposite_of_category_equivalence}: The \hyperref[def:dual_category]{duals} of \hyperref[def:category_equivalence]{equivalent categories} are equivalent.

    \thmitem{thm:categorical_principle_of_duality/adjunctions} \Fullref{thm:category_adjunction_duality}: The functor \( F \) is \hyperref[def:category_adjunction]{left adjoint} to \( G \) if and only if the \hyperref[def:dual_functor]{dual functor} \( F^{\opcat} \) is right adjoint to \( G^{\opcat} \).

    \thmitem{thm:categorical_principle_of_duality/limits} \Fullref{thm:categorical_limit_duality}: For every \hyperref[def:category_of_cones/cone]{cone} \( (A, \alpha) \) of the \hyperref[def:categorical_diagram]{diagram} \( D \) in \( \cat{C} \), \( (A, \alpha^{\opcat}) \) is a \hyperref[def:category_of_cones/cone]{cocone} of \( D^{\opcat} \) in \( \cat{C}^{\opcat} \).

    Even more, for every \hyperref[def:category_of_cones/limit]{limit} \( (L, \pi) \) of \( D \) in \( \cat{C} \), \( (L, \pi^{\opcat}) \) is a \hyperref[def:category_of_cones/colimit]{colimit} of \( D^{\opcat} \) in \( \cat{C}^{\opcat} \).
  \end{thmenum}
\end{proposition}

\begin{definition}\label{def:morphism_invertibility}
  In connection with \fullref{def:function_invertibility} and \fullref{def:first_order_homomorphism_invertibility}, we introduce the following terminology:
  \begin{thmenum}
    \thmitem{def:morphism_invertibility/left_cancellative} The morphism \( g: B \to C \) is \term{left-cancellative} if, for any pair of morphisms \( f_1, f_2: A \to B \), the equality \( g \bincirc f_1 = g \bincirc f_2 \) implies \( f_1 = f_2 \).

    Left-cancellative morphisms are also called \term{monic morphisms} or \term{monomorphisms}.

    A conventional notation for monomorphisms is \( g: B \hookrightarrow C \).

    \thmitem{def:morphism_invertibility/left_invertible} The morphism \( f: A \to B \) is \term{left-invertible} if there exists a morphism \( g: B \to A \) such that \( g \bincirc f = \id_A \). We call \( g \) a \term{left inverse} of \( f \).

    Using forward references to \fullref{def:categorical_diagram}, we can restate this condition by saying that the following diagram commutes:
    \begin{equation}\label{eq:def:morphism_invertibility/left_invertible}
      \begin{aligned}
        \includegraphics[page=1]{output/def__morphism_invertibility.pdf}
      \end{aligned}
    \end{equation}

    Left-invertible morphisms are sometimes called \term{split monomorphisms} because they \enquote{split} the identity \( \id_A \) into a composition of \( f \) and \( g \).

    \thmitem{def:morphism_invertibility/right_cancellative} \hyperref[thm:categorical_principle_of_duality]{Dually}, the morphism \( f: A \to B \) is \term{right-cancellative} if, for any pair of morphisms \( g_1, g_2: B \to C \), the equality \( g_1 \bincirc f = g_2 \bincirc f \) implies \( g_1 = g_2 \).

    Right-cancellative morphisms are also called \term{epic morphisms} or \term{epimorphisms}.

    A conventional notation for epimorphisms is \( f: A \twoheadrightarrow B \).

    \thmitem{def:morphism_invertibility/right_invertible} The morphism \( g: B \to A \) is \term{right-invertible} if there exists a morphism \( f: A \to B \) such that \( f \bincirc g = \id_B \). We call \( g \) a \term{right inverse} of \( f \).

    Using forward references to \fullref{def:categorical_diagram}, we can restate this condition by saying that the following diagram commutes:
    \begin{equation}\label{eq:def:morphism_invertibility/right_invertible}
      \begin{aligned}
        \includegraphics[page=2]{output/def__morphism_invertibility.pdf}
      \end{aligned}
    \end{equation}

    Right-invertible morphisms are sometimes called \term{split epimorphisms} because they \enquote{split} the identity \( \id_B \) into a composition of \( g \) and \( f \).

    \thmitem{def:morphism_invertibility/isomorphism} The morphism \( f: A \to B \) is \term{fully invertible} it is both left-invertible and right-invertible. By \fullref{thm:def:morphism_invertibility/properties/left_and_right}, in this case, there exists a unique morphism \( f^{-1}: B \to A \) that is a \term{two-sided inverse}, i.e. it is both a left inverse and a right inverse.

    A fully invertible morphism is usually called an \term{isomorphism}. If there exists an isomorphism between \( A \) and \( B \), we say that they are \term{isomorphic} and write \( A \cong B \).

    \thmitem{def:morphism_invertibility/endomorphism} A morphism \( f: A \to A \) from an object to itself is called an \term{endomorphism}.

    \thmitem{def:morphism_invertibility/automorphism} A morphism that is both an endomorphism and an isomorphism is called an \term{automorphism}.
  \end{thmenum}
\end{definition}

\begin{example}\label{ex:def:morphism_invertibility}
  \Fullref{thm:function_invertibility_categorical} characterizes the cancellative and invertible morphisms defined in \fullref{def:morphism_invertibility} for \hyperref[def:category_of_small_sets]{\( \cat{Set} \)} in terms of \hyperref[def:function_invertibility/injective]{injectivity} and \hyperref[def:function_invertibility/injective]{surjectivity}.

  A very simple example of a monomorphism which does not split is the empty function with nonempty domain. These are discussed in \fullref{thm:function_invertibility_categorical/empty}.

  \Fullref{thm:surjective_functions_are_right_invertible} is important enough to have a categorical interpretation via \fullref{thm:epimorphisms_split_in_set}, where its relation to the \hyperref[def:zfc/choice]{axiom of choice} is also discussed.
\end{example}

\begin{proposition}\label{thm:morphism_invertibility_duality}
  A morphism \( f: A \to B \) in \( \cat{C} \) is a \hyperref[def:morphism_invertibility/left_invertible]{(split)} \hyperref[def:morphism_invertibility/left_cancellative]{monomorphism} if and only if \( f^{\opcat}: B \to A \) in the dual category \( \cat{C}^{\opcat} \) is a \hyperref[def:morphism_invertibility/right_invertible]{(split)} \hyperref[def:morphism_invertibility/right_cancellative]{epimorphism}.

  In particular, \( f \) is an \hyperref[def:morphism_invertibility/isomorphism]{isomorphism} in \( \cat{C} \) if and only if \( f^{\opcat} \) is an isomorphism in \( \cat{C}^{\opcat} \).

  This is part of the duality principles listed in \fullref{thm:categorical_principle_of_duality}.
\end{proposition}
\begin{proof}
  Trivial.
\end{proof}

\begin{proposition}\label{thm:def:morphism_invertibility/properties}
  Morphisms have the following basic properties regarding their \hyperref[def:morphism_invertibility]{invertibility} (compare to \fullref{thm:function_composition_invertibility}):

  \begin{thmenum}
    \thmitem{thm:def:morphism_invertibility/properties/split_monomorphism} Any \hyperref[def:morphism_invertibility/left_invertible]{left-invertible morphism} is \hyperref[def:morphism_invertibility/left_cancellative]{left-cancellative}.

    In more categorical terms, every split monomorphism is a monomorphism.

    \thmitem{thm:def:morphism_invertibility/properties/split_epimorphism} Any \hyperref[def:morphism_invertibility/right_invertible]{right-invertible morphism} is \hyperref[def:morphism_invertibility/right_cancellative]{right-cancellative}.

    In more categorical terms, every split epimorphism is an epimorphism.

    \thmitem{thm:def:morphism_invertibility/properties/at_most_one_inverse}\mcite[exer. 1.1.13]{Leinster2016Basic} Any morphism has at most one two-sided inverse.

    \thmitem{thm:def:morphism_invertibility/properties/left_and_right} If a morphism is both left-invertible and right-invertible, the two inverses are equal, and the morphism is fully invertible.

    \thmitem{thm:def:morphism_invertibility/properties/inverse_interchanges} The morphism \( f: A \to B \) is a right inverse of \( g: B \to A \) if and only if \( g \) is a left inverse of \( f \).

    \thmitem{thm:def:morphism_invertibility/properties/monomorphism_and_split_epimorphism} If a morphism left-cancellative and right-invertible, it is an isomorphism.

    \thmitem{thm:def:morphism_invertibility/properties/split_monomorphism_and_epimorphism} If a morphism left-invertible and right-cancellative, it is an isomorphism.

    \thmitem{thm:def:morphism_invertibility/properties/cancellative_composition} The composition of two monomorphisms (resp. epimorphisms) is again a monomorphism (resp. epimorphism).

    \thmitem{thm:def:morphism_invertibility/properties/invertible_composition} The composition of two split monomorphisms (resp. epimorphisms) is again a split monomorphism (resp. epimorphism).
  \end{thmenum}
\end{proposition}
\begin{proof}
  \SubProofOf{thm:def:morphism_invertibility/properties/split_monomorphism} Suppose that \( g: B \to C \) is left-invertible with inverse \( h: C \to B \). Suppose that \( f_1, f_2: A \to B \) are morphisms such that
  \begin{equation*}
    g \bincirc f_1 = g \bincirc f_2.
  \end{equation*}

  Then
  \begin{equation*}
    f_1
    \reloset {\eqref{eq:def:category/C1}} =
    \id_B \bincirc f_1
    =
    (h \bincirc g) \bincirc f_1
    \reloset {\eqref{eq:def:category/C2}} =
    h \bincirc (g \bincirc f_1)
    =
    h \bincirc (g \bincirc f_2)
    =
    \cdots
    =
    f_2.
  \end{equation*}

  \SubProofOf{thm:def:morphism_invertibility/properties/split_epimorphism} This is an exemplar proof using duality. By \fullref{thm:morphism_invertibility_duality}, every split epimorphism \( f: A \to B \) in \( \cat{C} \) is a split monomorphism in \( \cat{C}^{\opcat} \). By \fullref{thm:def:morphism_invertibility/properties/split_monomorphism}, \( f^{\opcat} \) is a monomorphism. Then gain by \fullref{thm:morphism_invertibility_duality}, \( f \) is an epimorphism.

  \SubProofOf{thm:def:morphism_invertibility/properties/at_most_one_inverse} If \( f: A \to B \) has no inverse, it vacuously has at most one inverse.

  Now assume that \( f: A \to B \) has two inverses \( g_1: B \to A \) and \( g_2: B \to A \):
  \begin{align*}
    g_1 \bincirc f = \id_A &&& f \bincirc g_1 = \id_B, \\
    g_2 \bincirc f = \id_A &&& f \bincirc g_2 = \id_B.
  \end{align*}

  Then
  \begin{equation*}
    g_1
    \reloset {\eqref{eq:def:category/C1}} =
    g_1 \bincirc \id_B
    =
    g_1 \bincirc (f \bincirc g_2)
    \reloset {\eqref{eq:def:category/C2}} =
    (g_1 \bincirc f) \bincirc g_2
    =
    \id_A \bincirc g_2
    \reloset {\eqref{eq:def:category/C1}} =
    g_2.
  \end{equation*}

  \SubProofOf{thm:def:morphism_invertibility/properties/left_and_right} Suppose that \( f: A \to B \) has a left-inverse \( l: B \to A \) and a right-inverse \( r: B \to A \). Then

  \SubProofOf{thm:def:morphism_invertibility/properties/inverse_interchanges} Trivial.

  \SubProofOf{thm:def:morphism_invertibility/properties/monomorphism_and_split_epimorphism} Let \( g: B \to A \) be left-cancellative and right-invertible. Let \( f: A \to B \) be a right inverse of \( g \). Then
  \begin{equation*}
    f
    =
    \reloset {\eqref{eq:def:category/C1}} =
    f \bincirc \id_A
    =
    f \bincirc (g \bincirc f)
    =
    (f \bincirc g) \bincirc f.
  \end{equation*}

  Because \( g \) is a left inverse of \( f \), from \fullref{thm:def:morphism_invertibility/properties/split_monomorphism} it follows that \( f \) is left-cancellative. Since we have
  \begin{equation*}
    \id_B \bincirc f
    =
    (f \bincirc g) \bincirc f,
  \end{equation*}
  it follows that \( f \bincirc g = \id_B \).

  Therefore, \( f \) is a left inverse of \( g \) and hence an isomorphism.

  \SubProofOf{thm:def:morphism_invertibility/properties/split_monomorphism_and_epimorphism} The proof is analogous to \fullref{thm:def:morphism_invertibility/properties/monomorphism_and_split_epimorphism}.

  \SubProofOf{thm:def:morphism_invertibility/properties/cancellative_composition} Let \( g: B \to C \) and \( h: C \to D \) be monomorphisms (left-cancellative).

  Let \( f_1, f_2: A \to B \) be two arbitrary morphisms with codomain \( B \). Suppose that
  \begin{equation*}
    (h \bincirc g) \bincirc f_1 = (h \bincirc g) \bincirc f_2.
  \end{equation*}

  Then, by \ref{def:category/C2},
  \begin{equation*}
    h \bincirc (g \bincirc f_1) = h \bincirc (g \bincirc f_2).
  \end{equation*}

  Since \( h \) is left-cancellative, it follows that
  \begin{equation*}
    g \bincirc f_1 = g \bincirc f_2.
  \end{equation*}

  Since \( g \) is also left-cancellative, \( f_1 = f_2 \).

  Therefore, \( h \bincirc g \) is a monomorphism.

  The proof for composition of epimorphisms is identical.

  \SubProofOf{thm:def:morphism_invertibility/properties/invertible_composition} Let \( f: A \to B \) and \( g: B \to C \) be split monomorphisms (left-invertible).

  Then there exist left inverses \( l_f: B \to A \) and \( l_g: C \to B \) of \( f \) and \( g \), respectively. We have
  \begin{equation*}
    (l_f \bincirc l_g) \bincirc (g \bincirc f)
    \reloset {\eqref{eq:def:category/C2}} =
    l_f \bincirc (l_g \bincirc g) \bincirc f
    =
    l_f \bincirc \id_B \bincirc f
    \reloset {\eqref{eq:def:category/C1}} =
    l_f \bincirc f
    =
    \id_A.
  \end{equation*}

  Therefore, \( g \bincirc f \) is also left-invertible.

  The proof for composition of split epimorphisms is identical.
\end{proof}

\begin{theorem}[Epimorphisms split in Set]\label{thm:epimorphisms_split_in_set}
  Every \hyperref[def:morphism_invertibility/right_cancellative]{epimorphism} in \hyperref[def:category_of_small_sets]{\( \cat{Set} \)} splits. That is, all epimorphisms in \( \cat{Set} \) are \hyperref[def:morphism_invertibility/right_invertible]{split epimorphisms}.

  Assuming the existence of the \hyperref[def:grothendieck_universe]{Grothendieck universe} containing \( \cat{Set} \), in \hyperref[def:zfc]{\logic{ZF}} this theorem is equivalent to the \hyperref[def:zfc/choice]{axiom of choice} --- see \fullref{thm:axiom_of_choice_equivalences/epimorphisms}.

  Since not every epimorphism splits in a general category, this theorem is sometimes considered to be a categorical statement of the axiom of choice, which holds in some categories but not in others.
\end{theorem}
\begin{proof}
  By \fullref{thm:function_invertibility_categorical/right_cancellative}, a function is an epimorphism if and only if it is surjective. Thus, the theorem is equivalent to \fullref{thm:surjective_functions_are_right_invertible}.
\end{proof}

\begin{definition}\label{def:universal_objects}\mcite[def. 2.1.7]{Leinster2016Basic}
  Fix a category \( \cat{C} \).

  \begin{thmenum}
    \thmitem{def:universal_objects/initial} We call the object \( I \in \cat{C} \) an \term{initial object} if for any other object \( A \in \cat{C} \) there exists a unique morphism \( f: I \to A \),

    \thmitem{def:universal_objects/terminal} \hyperref[thm:categorical_principle_of_duality]{Dually}, we call the object \( T \in \cat{C} \) a \term{terminal object} or \term{final object} if for any other object \( A \in \cat{C} \) there exists a unique morphism \( f: A \to T \).

    The initial and terminal objects are collectively called \term{universal objects}.

    \thmitem{def:universal_objects/zero}\mcite{nLab:pointed_category} If \( Z \) is both an initial and a terminal object, we say that \( Z \) is a \term{zero object}. A category with a zero object is called a \term{pointed category}.
  \end{thmenum}
\end{definition}

\begin{example}\label{ex:def:universal_objects}
  \begin{thmenum}
    \thmitem{ex:def:universal_objects/set} In the category \hyperref[def:category_of_small_sets]{\( \cat{Set} \)} of small sets, for any set \( A \) there is a unique \hyperref[def:multi_valued_function/empty]{empty function} from \( \varnothing \) to \( A \). Therefore, \( \varnothing \) is an \hyperref[def:universal_objects/initial]{initial object} in \( \cat{Set} \).

    For any set \( A \), there is a unique function that contracts any set \( B \) to \( \set{ A } \). Therefore, every singleton set is a \hyperref[def:universal_objects/terminal]{final object} in \( \cat{Set} \).

    We often denote the initial and terminal objects in \( \cat{Set} \) by \( 0 \) and \( 1 \) respectively, which corresponds to their definition as \hyperref[def:ordinal]{ordinals}.

    By \fullref{thm:def:universal_objects/properties/no_zero}, \( \cat{Set} \) has no zero object.

    \thmitem{ex:def:universal_objects/grp} In the category \hyperref[def:group/category]{\( \cat{Grp} \)} of small groups, the \hyperref[def:group/trivial]{trivial group} is a \hyperref[def:universal_objects/zero]{zero object}. Indeed, it can be embedded into any other group and any group can be contracted into the corresponding trivial group. Furthermore, all trivial groups are isomorphic.
  \end{thmenum}
\end{example}

\begin{proposition}\label{thm:universal_object_duality}
  An object is \hyperref[def:universal_objects/initial]{initial} if and only if it is a \hyperref[def:universal_objects/terminal]{terminal object} the dual category.

  This is part of the duality principles listed in \fullref{thm:categorical_principle_of_duality}.
\end{proposition}
\begin{proof}
  Trivial.
\end{proof}

\begin{proposition}\label{thm:def:universal_objects/properties}
  \hfill
  \begin{thmenum}
    \thmitem{thm:def:universal_objects/properties/initial} An \hyperref[def:universal_objects/initial]{initial object} is unique up to an isomorphism.
    \thmitem{thm:def:universal_objects/properties/terminal} \hyperref[thm:categorical_principle_of_duality]{Dually}, a \hyperref[def:universal_objects/initial]{terminal object} is also unique up to an isomorphism.
    \thmitem{thm:def:universal_objects/properties/zero} If a category has an initial and a terminal object and if they are isomorphic, then both are zero objects.

    In particular, a zero object is unique up to an isomorphism.

    \thmitem{thm:def:universal_objects/properties/no_zero} If an initial and a terminal object exists and are not isomorphic, then there exist no zero objects.
  \end{thmenum}
\end{proposition}
\begin{proof}
  \SubProofOf{thm:def:universal_objects/properties/initial} Suppose that \( A \) and \( B \) are both initial objects in \( \cat{C} \). Then there exist morphisms \( f: A \to B \) and \( g: B \to A \). Their composition \( g \bincirc f \) is an \hyperref[def:morphism_invertibility/endomorphism]{endomorphism} on \( A \).

  But there exists a unique \hyperref[def:morphism_invertibility/endomorphism]{endomorphism} on \( A \), which must be the identity \( \id_A \). Thus, \( g \bincirc f = \id_A \) and \( g \) is a left inverse of \( f \).

  We can analogously show that \( g \) is a right inverse of \( f \). Therefore, \( f \) is fully invertible, and \( A \) and \( B \) are isomorphic.

  \SubProofOf{thm:def:universal_objects/properties/terminal} If \( T' \) and \( T^\dprime \) are terminal objects in \( \cat{C} \), by \fullref{thm:universal_object_duality}, they are initial objects in \( \cat{C}^{\opcat} \). By \fullref{thm:def:universal_objects/properties/initial}, they are isomorphic in \( \cat{C}^{\opcat} \) and by \fullref{thm:morphism_invertibility_duality}, they are isomorphic in \( \cat{C} \).

  \SubProofOf{thm:def:universal_objects/properties/zero} Suppose that \( A \) is an initial object and that \( B \) is a final object in \( \cat{C} \). Let \( f: A \to B \) be an isomorphism between them.

  Let \( C \in \cat{C} \) be any other object and let \( g: C \to B \) be the unique morphism to \( B \). Then \( f^{-1} \bincirc g: C \to A \) is a morphism from \( C \) to \( A \). The inverse \( f^{-1}: B \to A \) is unique by \fullref{thm:def:morphism_invertibility/properties/at_most_one_inverse}, therefore its composition with \( g: C \to B \) is also unique. Hence, any object has a unique morphism to \( A \). This makes \( A \) a terminal object and thus a zero object.

  We can analogously show that \( B \) is a zero object.

  \SubProofOf{thm:def:universal_objects/properties/no_zero} By \fullref{thm:def:universal_objects/properties/zero}, all zero objects are isomorphic. By \fullref{thm:def:universal_objects/properties/initial}, all initial objects are isomorphic and analogously for terminal objects. Hence, if a zero object exists, all initial objects are isomorphic to all terminal objects.

  If some initial object is not isomorphic to some terminal object, then by contraposition it follows that no zero object exists.
\end{proof}

\subsection{Functors}\label{subsec:functors}

\begin{definition}\label{def:functor}\mcite[def. 1.2.1 \\ def. 1.2.10]{Leinster2016Basic}
  Fix some \hyperref[def:category]{categories} \( \cat{C} \) and \( \cat{D} \). A \term{functor} \( F: \cat{C} \to \cat{D} \) is a \hyperref[eq:def:category_of_small_quivers/homomorphism]{quiver homomorphism} between the underlying quivers that is compatible with composition and identities.

  Explicitly, a functor is a family of functions
  \begin{equation}\label{eq:def:functor_as_family_of_function}
    \begin{aligned}
      F_{\obj}:       &\obj(\cat{C}) \to \obj(\cat{D}) \\
      F_{\hom(A, B)}: &\cat{C}(A, B) \to \cat{D}(F_{\obj}(A), F_{\obj}(B)),
    \end{aligned}
  \end{equation}
  where \( F_{\hom(A, B)} \) is a distinct function for every pair of objects \( A \) and \( B \).

  In practice, we usually define the functor as the set
  \begin{equation}\label{eq:def:functor_as_single_function}
    F \coloneqq F_{\obj} \cup \bigcup\set{ F_{\hom(A, B)} \given A, B \in \obj(\cat{C}) }
  \end{equation}

  Since the domains of all constituent functions are disjoint, \( F \) is again a total single-valued function. This allows us to justify the notation \( F(A) \) for objects and \( F(f) \) for morphisms.

  \begin{thmenum}[resume=def:functor]
    \thmitem{def:functor/domain_and_codomain} We say that the category \( \cat{C} \) is the \term{domain} and \( \cat{D} \) --- the \term{codomain} of the functor \( F \). These are technically not the domain and codomain of \( F \) when regarded as a function, however it is consistent with \fullref{def:category_of_small_categories}.

    \thmitem{def:functor/endofunctor} Similarly to \fullref{def:multi_valued_function/endofunction} for functions, if the domain \( \cat{C} \) and codomain \( \cat{D} \) of a functor coincide, we say that it is an \term{endofunctor}.
  \end{thmenum}

  The definition of a functor additionally requires the following compatibility conditions to hold:
  \begin{thmenum}[series=def:functor]
    \thmitem[def:functor/CF1]{CF1} Functors must preserve identities, meaning that for any object \( A \in \cat{C} \) the following equality must hold:
    \begin{equation}\label{eq:def:functor/CF1}\tag{\logic{CF1}}
      F(\id_A) = \id_{F(A)}.
    \end{equation}

    \thmitem[def:functor/CF2]{CF2} Functors must preserve composition, meaning that for any pair of morphism \( f: A \to B \) and \( g: B \to C \) in \( \cat{C} \),
    \begin{equation}\label{eq:def:functor/CF2}\tag{\logic{CF2}}
      F(g \bincirc f) = F(g) \bincirc F(f).
    \end{equation}
  \end{thmenum}
\end{definition}
\begin{defproof}
  The definition \eqref{eq:def:functor_as_family_of_function} ensures that the quiver homomorphism conditions \eqref{eq:def:category_of_small_quivers/homomorphism/head} and \eqref{eq:def:category_of_small_quivers/homomorphism/tail} hold.

  Indeed, for any morphism \( f: A \to B \) in \( \cat{C} \) we have
  \begin{equation*}
    F(\dom(f)) = F(A) = \dom(F(f)),
  \end{equation*}
  which implies \eqref{eq:def:category_of_small_quivers/homomorphism/head}. We also have
  \begin{equation*}
    F(\co\dom(f)) = F(B) = \co\dom(F(f)),
  \end{equation*}
  which implies \eqref{eq:def:category_of_small_quivers/homomorphism/tail}.
\end{defproof}

\begin{remark}\label{rem:functor_size}
  It is possible that \( \cat{C} \) is \( \mscrU \)-small in the sense of \fullref{def:category_size}, but the \hyperref[def:functor]{functor} \( F \), as the set \eqref{eq:def:functor_as_single_function}, is not \( \mscrU \)-small in the sense of \fullref{def:large_and_small_sets}. Without using universes, we cannot prove the existence of any functor from the category of smalls sets to itself, for example.
\end{remark}

\begin{example}\label{ex:unary_functors_in_set}
  In \fullref{def:basic_set_operations}, we defined some operations on the category \hyperref[def:category_of_small_sets]{\( \cat{Set} \)} of small sets.

  \begin{thmenum}
    \thmitem{ex:unary_functors_in_set/power} The \hyperref[def:basic_set_operations/power_set]{power set} \( \pow: \cat{Set} \to \cat{Set} \) is a canonical example of an \hyperref[def:functor/endofunctor]{endofunctor}. Explicitly:
    \begin{equation*}
      \begin{aligned}
        &\pow: \cat{Set} \to \cat{Set}, \\
        &\pow(A) \coloneqq \set{ S \given S \subseteq A }, \\
        &\pow(f: A \to B) \coloneqq (S \mapsto f[S]). \\
      \end{aligned}
    \end{equation*}

    We must verify that it is indeed a functor. \ref{def:functor/CF2} is satisfied because
    \begin{equation*}
      \pow(g) \bincirc \pow(f) = (S \mapsto g[f[S]]) = \pow(g \bincirc f).
    \end{equation*}

    The condition \ref{def:functor/CF2} is also obviously satisfied.

    The nuance here is that we send every function \( f: A \to B \) to its \hyperref[def:multi_valued_function/set_value]{set value} \( f[S] \) of some subset \( S \) of \( A \).

    \thmitem{ex:unary_functors_in_set/union} The \hyperref[def:basic_set_operations/union]{union} \( \bigcup \) and \hyperref[def:basic_set_operations/intersection]{intersection} \( \bigcap \) may seem to be good examples of endofunctors in \( \cat{Set} \). Unfortunately, there is no natural way to extend a morphism (function) \( f: A \to B \) to a morphism from \( \bigcup A \) to \( \bigcup B \) or \( \bigcap A \) to \( \bigcap B \).
  \end{thmenum}
\end{example}

\begin{definition}\label{def:subcategory}\mcite[def. 1.2.18]{Leinster2016Basic}
  We call the category \( \cat{D} \) a \term{subcategory} of \( \cat{C} \) if the following hold:
  \begin{itemize}
    \item The underlying quiver \( U(\cat{D}) \) is a \hyperref[def:quiver/submodel]{subquiver} of \( U(\cat{C}) \). That is, every object in \( \cat{D} \) is an object in \( \cat{C} \) and every morphism in \( \cat{D} \) is a morphism in \( \cat{C} \).
    \item Composition and identity in \( \cat{D} \) are \hyperref[def:multi_valued_function/restriction]{restrictions} of composition and identity in \( \cat{C} \).
  \end{itemize}

  \begin{thmenum}
    \thmitem{def:subcategory/inclusion} For every subcategory there exists an \term{inclusion functor} \( \Iota: \cat{D} \to \cat{C} \), which sends every object and morphism of \( \cat{D} \) to itself in \( \cat{C} \).

    \thmitem{def:subcategory/full} We say that \( \cat{D} \) is a \term{full subcategory} if the underlying quiver \( U(\cat{C}) \) is a \hyperref[def:quiver/submodel]{full subquiver}. That is, in case \( \cat{D}(A, B) = \cat{C}(A, B) \) for every pair of objects \( A \) and \( B \) of \( \cat{D} \).

    By \fullref{thm:def:functor_invertibility/properties/full_subcategory}, this is equivalent to the inclusion functor being \hyperref[def:functor_invertibility/full]{full}.

    \thmitem{def:subcategory/induced} Every \hyperref[rem:family_of_sets]{family} \( \mscrD \) of objects in \( \cat{C} \) induces a full subcategory \( \cat{D} \) of \( \cat{C} \), whose objects are those of \( \mscrD \) and whose morphisms are restricted to those whose domain and codomain are both in \( \mscrD \).
  \end{thmenum}
\end{definition}

\begin{remark}\label{rem:contravariant_functor}\mcite[def. 1.2.10]{Leinster2016Basic}
  We can invert the order of composition in \ref{def:functor/CF2} in the definition of a functor given in \fullref{def:functor}.
  \begin{thmenum}
    \thmitem[rem:contravariant_functor/CF2]{CF2\Textprime} We can replace \ref{def:functor/CF2} with
    \begin{equation}\label{eq:rem:contravariant_functor/CF2}\tag{\logic{CF2\Textprime}}
      F(g \bincirc f) = F(f) \bincirc F(g).
    \end{equation}
  \end{thmenum}

  This also requires some other straightforward modifications to the definition of a functor.

  A functor that satisfies \ref{rem:contravariant_functor/CF2} rather than \ref{def:functor/CF2} is called \term{contravariant}. In this context, a functor satisfying \ref{def:functor/CF2} is called \term{covariant}.

  Fortunately, a contravariant functor from \( \cat{C}^{\opcat} \) to \( \cat{D} \) is identical to a covariant functor from \( \cat{C} \) to \( \cat{D} \). Therefore, there is no formal difference between the two concepts.

  The usage of the terms are entirely dictated by context. Unless necessary, we will avoid speaking about contravariant functors to avoid confusion. Some examples where this terminology may be useful are \fullref{def:dual_functor}, \fullref{def:hom_functor/unary} and \fullref{ex:dual_space_contravariant_functor}.
\end{remark}

\begin{example}\label{ex:dual_space_contravariant_functor}
  We can try to na\"ively define a functor that assigns to a \hyperref[def:vector_space]{vector space} its \hyperref[def:dual_vector_space]{algebraic dual}:
  \begin{equation*}
    \begin{aligned}
      &F: \cat{Vect_\BbbK} \to \cat{Vect_\BbbK}, \\
      &F(V) \coloneqq V^*, \\
      &F(f: V \to W) \coloneqq (\varphi: W \to \BbbK \mapsto \varphi \bincirc f).
    \end{aligned}
  \end{equation*}

  Unfortunately, \( F(f) \) is supposed to be a morphism from \( V^* \) to \( W^* \), but is actually a morphism from \( W^* \) to \( V^* \). This makes \( F \) a \hyperref[rem:contravariant_functor]{contravariant functor} or, equivalently, a functor from \( \cat{Vect_\BbbK}^{\opcat} \) to \( \cat{Vect_\BbbK} \).
\end{example}

\begin{definition}\label{def:discrete_category}
  A \term{discrete category} is a category with no morphisms except for the identities. Clearly to any set there corresponds exactly one discrete category and vice versa.
\end{definition}

\begin{example}\label{ex:discrete_category_adjunction}
  Denote by
  \begin{equation*}
    U: \cat{Cat} \to \cat{Set}
  \end{equation*}
  the forgetful functor that for any small category \( \cat{C} \) gives us its set of objects \( \obj(\cat{C}) \). There is also a functor
  \begin{equation*}
    D: \cat{Set} \to \cat{Cat}
  \end{equation*}
  that for any small set \( A \) gives us the \hyperref[def:discrete_category]{discrete category} whose set of objects is \( A \).

  This is actually an \hyperref[def:category_adjunction]{adjunction} --- see \fullref{ex:def:category_adjunction/set_cat}.
\end{example}

\begin{proposition}\label{thm:def:functor/properties}
  \hyperref[def:functor]{Functors} have the following basic properties:
  \begin{thmenum}
    \thmitem{thm:def:functor/properties/half_inverses} Functors preserve inverses. For every functor \( F: \cat{C} \to \cat{D} \) and every morphism \( f: A \to B \) in \( \cat{C} \) with a right inverse \( g: B \to A \), \( F(f) \) is a right inverse of \( F(g) \). Similarly, if \( g \) is a left inverse of \( f \), then \( F(g) \) is a left inverse of \( F(f) \).

    \thmitem{thm:def:functor/properties/inverses} For every functor \( F: \cat{C} \to \cat{D} \) and every isomorphism \( f: A \to B \) in \( \cat{C} \),
    \begin{equation}\label{eq:thm:def:functor/properties/inverses}
      [F(f)]^{-1} = F(f^{-1}).
    \end{equation}

    \thmitem{thm:def:functor/properties/isomorphisms} Functors preserve \hyperref[def:morphism_invertibility/isomorphism]{isomorphisms}. That is, for every functor \( F: \cat{C} \to \cat{D} \), if \( f: A \to B \) is an isomorphism in \( \cat{C} \), \( F(f) \) is an isomorphism in \( \cat{D} \).

    Consequently, for every pair of objects \( A \) and \( B \) in \( \cat{C} \), from \( A \cong B \) it follows that \( F(A) \cong F(B) \).

    The converse sometimes also holds --- see \fullref{thm:def:functor_invertibility/properties/fully_faithful_reflects_isomorphisms}.
  \end{thmenum}
\end{proposition}
\begin{proof}
  \SubProofOf{thm:def:functor/properties/half_inverses} Let \( f: A \to B \) be a right inverse of \( g: B \to A \) in \( \cat{C} \). Then
  \begin{equation*}
    F(g) \bincirc F(f)
    \reloset {\eqref{eq:def:functor/CF2}} =
    F(g \bincirc f)
    =
    F(\id_A)
    \reloset {\eqref{eq:def:functor/CF1}} =
    \id_{F(A)}.
  \end{equation*}

  Thus, \( F(f) \) is a right inverse of \( F(g) \). Since \( g \) is a left inverse of \( f \), automatically \( F(g) \) is a left inverse of \( F(f) \).

  \SubProofOf{thm:def:functor/properties/inverses} If \( f^{-1} \) is a left inverse of \( f \), by \fullref{thm:def:functor/properties/half_inverses} we have that \( F(f^{-1}) \) is a left inverse of \( F(f) \). But \( F(f^{-1}) \) is also a right inverse, and again by \fullref{thm:def:functor/properties/half_inverses} \( F(g) \) is a right inverse of \( F(f^{-1}) \).

  Therefore, \( F(f^{-1}) \) is a two-sided inverse of \( F(f) \). By \fullref{thm:def:morphism_invertibility/properties/at_most_one_inverse}, it is the only two-sided inverse, hence
  \begin{equation*}
    [F(f)]^{-1} = F(f^{-1}).
  \end{equation*}

  \SubProofOf{thm:def:functor/properties/isomorphisms} Follows from \fullref{thm:def:functor/properties/inverses}.
\end{proof}

\begin{definition}\label{def:category_of_small_categories}
  Suppose that we are given a \hyperref[def:grothendieck_universe]{Grothendieck universe} \( \mscrU \), which is safe to assume to be the smallest suitable one as explained in \fullref{def:large_and_small_sets}.

  We denote the \hyperref[def:category]{category} of \( \mscrU \)-small \hyperref[def:category]{categories} by \( \ucat{Cat} \) or, if the universe is clear from the context, simply by \( \cat{Cat} \). See \fullref{def:category_size} for a further discussion of universes and categories.

  \begin{itemize}
    \item The \hyperref[def:category/objects]{set of objects} \( \obj(\cat{Cat}) \) is the set of all \( \mscrU \)-small categories.

    \item The \hyperref[def:category/morphisms]{set of morphisms} \( \cat{Cat}(A, B) \) from \( A \) to \( B \) is the set of all \hyperref[def:functor]{functors} from \( A \) to \( B \).

    \item The \hyperref[def:category/composition]{composition of morphisms} is the \hyperref[def:multi_valued_function/composition]{function composition} of the functors regarded as the functions \eqref{eq:def:functor_as_single_function}. That is, the composition of \( F: \cat{C} \to \cat{D} \) and \( G: \cat{D} \to \cat{E} \) is the functor
    \begin{equation}\label{eq:def:category_of_small_categories/composition}
      \begin{aligned}
        &[G \bincirc F]: \cat{C} \to \cat{E}, \\
        &[G \bincirc F](A) \coloneqq G(F(A)), \\
        &[G \bincirc F](f) \coloneqq G(F(f)).
      \end{aligned}
    \end{equation}

    \item The \hyperref[def:category/identity]{identity morphism} on the category \( \cat{C} \) is the \term{identity functor}
    \begin{equation}\label{eq:def:category_of_small_categories/identity}
      \begin{aligned}
        &\id_{\cat{C}}: \cat{C} \to \cat{C}, \\
        &\id_{\cat{C}}(A) \coloneqq A, \\
        &\id_{\cat{C}}(f) \coloneqq f.
      \end{aligned}
    \end{equation}
  \end{itemize}
\end{definition}
\begin{defproof}
  To see that \( \ucat{Cat} \) is indeed a category, we verify the conditions \ref{def:category/C1} and \ref{def:category/C2}.

  \SubProofOf{def:category/C1} For every two \( \mscrU \)-small categories \( \cat{C} \) and \( \cat{D} \) and every functor \( F: \cat{C} \to \cat{D} \), for every object \( A \in \cat{C} \) we have
  \begin{equation*}
    [\id_{\cat{D}} \bincirc F](A)
    =
    \id_{\cat{D}}(F(A))
    =
    F(A)
    =
    F(\id_{\cat{C}}(A))
    =
    [F \bincirc \id_{\cat{C}}](A)
  \end{equation*}
  and analogously for morphisms.

  Therefore, \( \id_{\cat{C}} \) and \( \id_{\cat{D}} \) satisfy \eqref{eq:def:category/C1}.

  \SubProofOf{def:category/C2} Associativity of functor composition is inherited from the associativity of function composition.
\end{defproof}

\begin{definition}\label{def:universal_categories}
  For any \hyperref[def:grothendieck_universe]{Grothendieck universe} \( \mscrU \), the \hyperref[def:category_of_small_categories]{category of \( \mscrU \)-small categories} \( \ucat{Cat} \) has an initial and a terminal object.

  Similarly to how we use the \hyperref[def:ordinal]{ordinals} \( 0 \) and \( 1 \) to denote the initial and terminal object in the category of sets, we denote the initial category by \( \cat{0} \) and the final category by \( \cat{1} \). Note that the final category is only unique up to an isomorphism. They are identical, however, for all universes \( \mscrU \).

  These categories are precisely the \hyperref[def:discrete_category]{discrete categories} induced by the ordinals \( 0 \) and \( 1 \) as described in \fullref{thm:order_category_isomorphism}.
\end{definition}

\begin{definition}\label{def:dual_functor}\mcite{nLab:opposite_category}
  The \term{opposite} or \term{dual} functor of \( F: \cat{C} \to \cat{D} \) is the functor
  \begin{equation*}
    \begin{aligned}
      &F^{\opcat}: \cat{C}^{\opcat} \to \cat{D}^{\opcat} \\
      &F^{\opcat}(A) \coloneqq A \\
      &F^{\opcat}(f^{\opcat}: B \to A) \coloneqq [F(f: A \to B)]^{\opcat}.
    \end{aligned}
  \end{equation*}

  For the composition of functors, we then have
  \begin{equation}\label{eq:def:dual_functor/composition}
    [G \bincirc F]^{\opcat} = G^{\opcat} \bincirc F^{\opcat}.
  \end{equation}

  This is somewhat in contrast to the general practice of inverting morphisms when taking duals. Thus, we define, for any \hyperref[def:grothendieck_universe]{Grothendieck universe} \( \mscrU \), the \term{oppositization functor}
  \begin{equation*}
    (\anon*)^{\opcat}: \ucat{Cat}^{\opcat} \to \ucat{Cat}.
  \end{equation*}

  As an \hyperref[def:multi_valued_function/endofunction]{endofunction} on \( \obj(\ucat{Cat}) \), the oppositization functor is clearly an \hyperref[def:set_with_involution]{involution}.

  Dual functors also arise naturally in \fullref{thm:dual_functor_category}.
\end{definition}

\begin{definition}\label{def:functor_image}
  The \term{image} of a functor \( F: \cat{C} \to \cat{D} \) is the \hyperref[def:quiver]{quiver} whose vertex set is
  \begin{equation*}
    V \coloneqq \set{ F(A) \given A \in \cat{C} }
  \end{equation*}
  and whose arc set is
  \begin{equation*}
    A \coloneqq \set{ F(f) \given A, B \in \cat{C} \T{and} f \in \cat{C}(A, B) }.
  \end{equation*}

  This quiver has no categorical structure --- it is merely a directed multigraph. As shown in \fullref{ex:functor_image_not_a_category}, imposing a categorical structure na\"ively may fail.
\end{definition}

\begin{example}\label{ex:functor_image_not_a_category}\mcite{MathSE:image_of_functor_is_not_a_category}
  \begin{figure}
    \hfill
    \includegraphics[page=1]{output/ex__functor_image_not_a_category.pdf}
    \hfill
    \hfill
    \caption{A functor whose image is not a category.}\label{fig:ex:functor_image_not_a_category}
  \end{figure}

  Consider the functor \( F: \cat{C} \to \cat{D} \) from \cref{fig:ex:functor_image_not_a_category}.

  \begin{itemize}
    \item The solid arrows are the morphisms in \( \cat{C} \) and their images in \( F(\cat{C}) \).
    \item The dashed arrows denote the action of the functor \( F \).
    \item The dotted arrow exists in \( \cat{D} \) as the composition of the other two arrows, however it is missing in the image \( F(\cat{C}) \). Thus, composition is not fully defined in \( F(\cat{C}) \), and \( F(\cat{C}) \) fails to be a category.
  \end{itemize}
\end{example}

\begin{definition}\label{def:categorical_diagram}
  Fix a category \( \cat{I} \), called an \term{index category}. A \term{diagram} in \( \cat{C} \) of shape \( \cat{I} \) is simply a functor \( D: \cat{I} \to \cat{C} \), whose domain is \( \cat{I} \). We sometimes identify a diagram functor with its image \( D(\cat{I}) \).

  It is often convenient to draw graphically the \hyperref[def:quiver_geometric_realization]{geometric realizations} of the \hyperref[def:quiver]{quiver} \( D(\cat{I}) \). An established convention is to allow multiple vertices representing the same object, which can be achieved formally by actually adjoining new vertices to the quiver and labeling them as in \fullref{def:weighted_set}. Other established conventions for drawing diagrams include not drawing identity morphisms and adding various visual aids. In this regard, categorical diagrams correspond to the everyday sense of the word \enquote{diagram}.

  We say that the diagram \( D \) over \( \cat{C} \) \term{commutes} if, whenever \( p = (f_1, \ldots, f_n) \) and \( q = (g_1, \ldots, g_m) \) are two \hyperref[def:quiver_path/directed]{directed paths} in \( D(\cat{I}) \) with identical endpoints and either \( n > 1 \) or \( m > 1 \), then
  \begin{equation*}
    f_n \bincirc f_{n-1} \bincirc \cdots \bincirc f_2 \bincirc f_1
    =
    g_m \bincirc g_{m-1} \bincirc \cdots \bincirc g_2 \bincirc g_1.
  \end{equation*}

  We do not really care about how the objects and morphisms in \( \cat{I} \) are labeled, hence we often use placeholder dots like in \eqref{eq:ex:quivers_as_functors/index/dots}.

  The requirement that the one of the paths is nontrivial, however, is crucial in \fullref{def:equalizers}.
\end{definition}

\begin{remark}\label{rem:inverting_isomorphisms_may_preserve_commutativity}
  Inverting isomorphisms in a \hyperref[def:categorical_diagram]{commutative diagram} may or may not preserve commutativity.

  If \( p = (f_1, \ldots, f_n) \) and \( q = (g_1, \ldots, g_m) \) are two paths in a commutative diagram, and if \( f_1 \) is invertible, then obviously
  \begin{equation*}
    f_n \bincirc \cdots \bincirc f_1 = g_m \bincirc \cdots \bincirc g_1
  \end{equation*}
  if and only if
  \begin{equation*}
    f_n \bincirc \cdots \bincirc f_2 = g_m \bincirc \cdots \bincirc g_1 \bincirc f_1
  \end{equation*}
  and similarly if \( f_n \) is invertible.

  On the other hand, consider \hyperref[def:ordinal]{ordinals} in \( \cat{Set} \). Denote by \( \iota \) the inclusion maps and by \( f: \omega^2 \to \omega \) the bijective map from \fullref{thm:omega_equinumerous_with_omega_squared}. Then the following diagram commutes:
  \begin{equation}\label{eq:rem:inverting_isomorphisms_may_preserve_commutativity/ordinals_commuting}
    \begin{aligned}
      \includegraphics[page=1]{output/rem__inverting_isomorphisms_may_preserve_commutativity.pdf}
    \end{aligned}
  \end{equation}
  but the following does not:
  \begin{equation}\label{eq:rem:inverting_isomorphisms_may_preserve_commutativity/ordinals_not_commuting}
    \begin{aligned}
      \includegraphics[page=2]{output/rem__inverting_isomorphisms_may_preserve_commutativity.pdf}
    \end{aligned}
  \end{equation}
\end{remark}

\begin{definition}\label{def:functor_invertibility}
  In connection with \fullref{def:morphism_invertibility} and \fullref{def:function_invertibility}, we introduce the following terminology:
  \begin{thmenum}
    \thmitem{def:functor_invertibility/injective_on_objects} The \hyperref[def:functor]{functor} \( F: \cat{C} \to \cat{D} \) is \term{injective on objects} if the \hyperref[def:multi_valued_function/restriction]{restriction}
    \begin{equation*}
      F\restr_{\obj(C)}: \obj(C) \to \obj(D)
    \end{equation*}
    is \hyperref[def:function_invertibility/injective]{injective}.

    That is, for every pair of objects \( A \) and \( B \) in \( \cat{C} \), from \( F(A) = F(B) \) it follows that \( A = B \).

    If, instead, from \( F(A) \cong F(B) \) it follows that \( A \cong B \), we say that \( F \) if \term{essentially injective on objects}.

    \thmitem{def:functor_invertibility/injective_on_morphisms} The \hyperref[def:functor]{functor} \( F: \cat{C} \to \cat{D} \) is \term{injective on morphisms} if its restriction to the set
    \begin{equation*}
      \bigcup\set{ \cat{C}(A, B) \given A, B \in \obj(\cat{C}) }
    \end{equation*}
    of all morphisms is injective.

    That is, for every pair of morphisms \( f \) and \( g \) in \( \cat{C} \), from \( F(f) = F(g) \) it follows that \( f = g \). Note that if the morphisms are not parallel, we assume that they are not equal.

    \thmitem{def:functor_invertibility/faithful}\mcite[def. 1.2.16]{Leinster2016Basic} The functor \( F: \cat{C} \to \cat{D} \) is \term{faithful} if it is \hyperref[def:function_invertibility/injective]{injective} on \( \hom \)-sets, i.e. for all pairs of objects \( A \) and \( B \) in \( \cat{C} \), the restriction of \( F \) to \( \cat{C}(A, B) \) is an injective function.

    That is, for every pair of objects \( A \) and \( B \) in \( \cat{C} \) and every pair of morphisms \( f \) and \( g \) in \( \cat{C}(A, B) \), from \( F(f) = F(g) \) it follows that \( f = g \).

    See \fullref{thm:def:functor_invertibility/properties/injective} for how faithful functors relate to functors injective on objects or on morphisms.

    \thmitem{def:functor_invertibility/surjective_on_objects}\mcite[def. 1.3.17]{Leinster2016Basic} The functor \( F: \cat{C} \to \cat{D} \) is \term{surjective on objects} if the restriction
    \begin{equation*}
      F\restr_{\obj(C)}: \obj(C) \to \obj(D)
    \end{equation*}
    is \hyperref[def:function_invertibility/surjective]{surjective}.

    That is, for every object \( B \) in \( \cat{D} \), there exists at least one object \( A \) in \( \cat{C} \) such that \( F(A) = B \).

    If, instead, there exists at least one object \( A \in \cat{C} \) such that \( F(A) \cong B \), we say that \( F \) is \term{essentially surjective on objects}.

    \thmitem{def:functor_invertibility/surjective_on_morphisms} Similarly, \( F: \cat{C} \to \cat{D} \) is \term{surjective on morphisms} if its restriction to the set of all morphisms is surjective.

    That is, for every morphism \( g \) in \( \cat{D} \), there exists at least one morphism \( f \) in \( \cat{C} \) such that \( F(f) = g \).

    \thmitem{def:functor_invertibility/full}\mcite[def. 1.2.16]{Leinster2016Basic} The functor \( F: \cat{C} \to \cat{D} \) is \term{full} if it is surjective on \( \hom \)-sets, i.e. for all pairs of objects \( A \) and \( B \) in \( \cat{C} \), the restriction of \( F \) to \( \cat{C}(A, B) \) is a surjective function.

    That is, for every pair of objects \( A \) and \( B \) in \( \cat{C} \) and every morphism \( g: F(A) \to F(B) \) in \( \cat{D} \), there exists at least one morphism in \( f: A \to B \) in \( \cat{C} \) such that \( F(f) = g \).

    \thmitem{def:functor_invertibility/fully_faithful} Finally, \( F: \cat{C} \to \cat{D} \) is \term{fully faithful} if it is both full and faithful.
  \end{thmenum}
\end{definition}

\begin{proposition}\label{thm:def:functor_invertibility/properties}
  \hyperref[def:functor]{Functors} have the following basic properties regarding their \hyperref[def:functor_invertibility]{invertibility}:

  \begin{thmenum}
    \thmitem{thm:def:functor_invertibility/properties/injective} A functor is \hyperref[def:functor_invertibility/injective_on_morphisms]{injective on morphisms} if and only if it is both \hyperref[def:functor_invertibility/injective_on_objects]{injective on objects} and \hyperref[def:functor_invertibility/faithful]{faithful}.

    \thmitem{thm:def:functor_invertibility/properties/surjective} A functor is \hyperref[def:functor_invertibility/surjective_on_morphisms]{surjective on morphisms} if and only if it is both \hyperref[def:functor_invertibility/surjective_on_objects]{surjective on objects} and \hyperref[def:functor_invertibility/full]{full}.

    \thmitem{thm:def:functor_invertibility/properties/full_subcategory} A \hyperref[def:subcategory]{subcategory} \( \cat{D} \) of \( \cat{C} \) is full in the sense of \fullref{def:subcategory} if and only if the \hyperref[def:subcategory]{inclusion functor} \( \Iota: \cat{D} \to \cat{C} \) is full in the sense of \fullref{def:functor_invertibility/full}.

    \thmitem{thm:def:functor_invertibility/properties/faithful_reflects_composition} \hyperref[def:functor_invertibility/faithful]{Faithful} functors reflect composition. That is, for every functor \( F: \cat{C} \to \cat{D} \), if the following diagram commutes:
    \begin{equation}\label{eq:thm:def:functor_invertibility/properties/faithful_reflects_composition/image}
      \begin{aligned}
        \includegraphics[page=1]{output/thm__def__functor_invertibility__properties.pdf}
      \end{aligned}
    \end{equation}
    then the following diagram are identities:
    \begin{equation}\label{eq:thm:def:functor_invertibility/properties/faithful_reflects_composition/source}
      \begin{aligned}
        \includegraphics[page=2]{output/thm__def__functor_invertibility__properties.pdf}
      \end{aligned}
    \end{equation}

    \thmitem{thm:def:functor_invertibility/properties/faithful_reflects_cancellative} A \hyperref[def:functor_invertibility/faithful]{faithful} functor reflects monomorphisms and epimorphisms. For every functor \( F: \cat{C} \to \cat{D} \) and morphism \( f: A \to B \) in \( \cat{C} \), if \( F(f) \) is a monomorphism (resp. epimorphism), so is \( f \).

    \thmitem{thm:def:functor_invertibility/properties/fully_faithful_reflects_identities} A \hyperref[def:functor_invertibility/fully_faithful]{fully faithful} functor reflects identities. For every functor \( F: \cat{C} \to \cat{D} \) and endomorphism \( f: A \to A \) in \( \cat{C} \), if \( F(f) = \id_{F(A)} \), then \( f = \id_A \).

    \thmitem{thm:def:functor_invertibility/properties/fully_faithful_reflects_isomorphisms} A \hyperref[def:functor_invertibility/fully_faithful]{fully faithful} functor reflects split monomorphisms and split epimorphisms, and hence also isomorphisms.

    That is, for every functor \( F: \cat{C} \to \cat{D} \) and morphism \( f: A \to B \) in \( \cat{C} \), if \( F(f) \) is a split monomorphism (resp. split epimorphism or isomorphism), so is \( f \).

    \thmitem{thm:def:functor_invertibility/properties/isomorphism} A functor between \( \mscrU \)-small categories that is both injective and surjective on morphisms is itself an isomorphism in \( \ucat{Cat} \).
  \end{thmenum}
\end{proposition}
\begin{proof}
  \SubProofOf{thm:def:functor_invertibility/properties/injective}
  \SufficiencySubProof* Let \( F: \cat{C} \to \cat{D} \) be injective on morphisms. It is trivially faithful since faithfulness is a more restrictive condition.

  To see that \( F \) is injective on objects, let \( A, B \in \cat{C} \) and suppose that \( F(A) = F(B) \). Then \( \id_{F(A)} = \id_{F(B)} \) and
  \begin{equation*}
    F(\id_A)
    \reloset {\eqref{eq:def:functor/CF2}} =
    \id_{F(A)}
    =
    \id_{F(B)}
    \reloset {\eqref{eq:def:functor/CF2}} =
    F(\id_B).
  \end{equation*}

  Since \( F \) is injective on morphisms, it follows that \( \id_A = \id_B \), hence \( A = B \). Thus, \( F \) is injective on objects.

  \NecessitySubProof* Let \( F: \cat{C} \to \cat{D} \) be faithful and injective on objects. Let \( f: A \to B \) and \( g: C \to D \) be morphisms in \( \cat{C} \) such that \( F(f) = F(g) \).

  Then both \( F(f) \) and \( F(g) \) have the same domain \( F(A) = F(C) \) and codomain \( F(B) = F(D) \). Hence, since \( F \) is injective on objects, we have \( A = C \) and \( B = D \).

  Thus, \( f \) and \( g \) are both morphisms from \( A \) to \( B \). Since \( F \) is also faithful, from \( F(f) = F(g) \) it follows that \( f = g \).

  Therefore, \( F \) is injective on morphisms.

  \SubProofOf{thm:def:functor_invertibility/properties/surjective}
  \SufficiencySubProof* Let \( F: \cat{C} \to \cat{D} \) be surjective on morphisms. It is trivially full since fullness is a more restrictive condition.

  To see that \( F \) is surjective on objects, let \( C \in \cat{D} \). Then there exists some morphism \( f: A \to B \) in \( \cat{C} \) such that \( F(f) = \id_Z \). We thus necessarily have \( F(A) = C \) and \( F(B) = C \).

  \NecessitySubProof* Let \( F: \cat{C} \to \cat{D} \) be full and injective on objects. Let \( g: C \to D \) be a morphism in \( \cat{D} \).

  Since \( F \) is surjective on objects, there exists preimages \( A \) of \( C \) and \( B \) of \( D \) under \( F \). Thus, \( g \in \cat{D}(F(A), F(B)) \).

  Since \( F \) is also full, there exists some morphism \( f: A \to B \) such that \( F(f) = g \).

  Therefore, \( F \) is surjective on morphisms.

  \SubProofOf{thm:def:functor_invertibility/properties/full_subcategory} Trivial.

  \SubProofOf{thm:def:functor_invertibility/properties/faithful_reflects_composition} Suppose that \eqref{eq:thm:def:functor_invertibility/properties/faithful_reflects_composition/image} commutes. Then, since \( F \) is faithful and thus injective on the morphism set \( \cat{C}(A, C) \), the equality \( F(g \bincirc f) = F(g) \bincirc F(f) = F(h) \) implies that \( g \bincirc f = h \). Hence, \eqref{eq:thm:def:functor_invertibility/properties/faithful_reflects_composition/source} also commutes.

  \SubProofOf{thm:def:functor_invertibility/properties/faithful_reflects_cancellative} Let \( F(g) \) be a monomorphism and let \( f_1, f_2: A \to B \) be parallel morphisms such that
  \begin{equation*}
    g \bincirc f_1 = g \bincirc f_2.
  \end{equation*}

  Then, since \( F(g) \) is a monomorphism, we have that \( F(f_1) = F(f_2) \). Since \( F \) is faithful, the restriction \( F\restr_{C(A, B)} \) is injective, and \( f_1 = f_2 \).

  The proof when \( F(g) \) is an epimorphism is analogous.

  \SubProofOf{thm:def:functor_invertibility/properties/fully_faithful_reflects_identities} If \( F: \cat{C} \to \cat{D} \) is fully faithful, for every object \( A \) in \( \cat{C} \), the identity morphism \( \id_{F(A)} \) has a unique preimage under \( F \). By \ref{def:functor/CF1}, this preimage can only be \( \id_A \).

  \SubProofOf{thm:def:functor_invertibility/properties/fully_faithful_reflects_isomorphisms} Let \( q \) be a left inverse of \( F(f) \). Since \( F \) is fully faithful, there exists a unique morphism \( g: B \to A \) such that \( F(g) = q \).

  Since
  \begin{equation*}
    F(g) \bincirc F(f) = \id_{F(A)},
  \end{equation*}
  by \fullref{thm:def:functor_invertibility/properties/fully_faithful_reflects_identities} we have
  \begin{equation*}
    g \bincirc f = \id_A.
  \end{equation*}

  Therefore, \( g \) is a left inverse of \( F(f) \).

  The proof for right inverses follows from \fullref{thm:def:morphism_invertibility/properties/inverse_interchanges}.

  From \fullref{thm:def:morphism_invertibility/properties/left_and_right} it follows that if \( F(f) \) is an isomorphism, so is \( f \).

  \SubProofOf{thm:def:functor_invertibility/properties/isomorphism} If \( F \) is both injective and surjective on morphisms, it is also injective and surjective on objects and hence, as a function, is bijective. Therefore, it is both left and right invertible as a consequence of \fullref{thm:function_invertibility_categorical/fully_invertible}.
\end{proof}

\begin{example}\label{ex:def:functor_invertibility}
  \hfill
  \begin{thmenum}
    \thmitem{ex:def:functor_invertibility/power} The power set functor described in \fullref{ex:unary_functors_in_set} is clearly \hyperref[def:functor_invertibility/injective_on_morphisms]{injective on morphisms}, hence by \fullref{thm:def:functor_invertibility/properties/injective}, it is also \hyperref[def:functor_invertibility/injective_on_objects]{injective on objects} and \hyperref[def:functor_invertibility/faithful]{faithful}.

    It is not full, nor surjective on objects.

    \thmitem{ex:def:functor_invertibility/cat_to_set} The forgetful functor \( D: \ucat{Cat} \to \ucat{Set} \) discussed in \fullref{def:discrete_category} is \hyperref[def:functor_invertibility/surjective_on_morphisms]{surjective on morphisms}, hence by \fullref{thm:def:functor_invertibility/properties/surjective}, it is also \hyperref[def:functor_invertibility/surjective_on_objects]{surjective on objects} and \hyperref[def:functor_invertibility/full]{full}.

    It is not faithful, nor injective on objects.
  \end{thmenum}
\end{example}

\begin{proposition}\label{thm:commutative_diagrams_preserved_and_reflected}
  Functors preserve commutative diagrams and faithful functors also reflect commutative diagrams.

  More precisely, let \( \cat{C} \) be an arbitrary category, let \( D \) be a diagram in \( \cat{C} \), and let \( p = (f_1, \ldots, f_n) \) and \( q = (g_1, \ldots, g_m) \) be two \hyperref[def:quiver_path/directed]{directed paths} with the same endpoints in \( D \).

  For any functor \( F: \cat{C} \to \cat{D} \), if
  \begin{equation}\label{eq:thm:commutative_diagrams_preserved_and_reflected/source}
    f_n \bincirc \cdots \bincirc \bincirc f_1 = g_m \bincirc \cdots \bincirc \bincirc g_1,
  \end{equation}
  then
  \begin{equation}\label{eq:thm:commutative_diagrams_preserved_and_reflected/image}
    F(f_n) \bincirc \cdots \bincirc \bincirc F(f_1) = F(g_m) \bincirc \cdots \bincirc \bincirc F(g_1),
  \end{equation}

  Conversely, if \( F \) is faithful, then \eqref{eq:thm:commutative_diagrams_preserved_and_reflected/image} implies \eqref{eq:thm:commutative_diagrams_preserved_and_reflected/source}.
\end{proposition}
\begin{proof}
  Functors preserve composition by \ref{eq:def:functor/CF2}, hence \eqref{eq:thm:commutative_diagrams_preserved_and_reflected/image} follows from \eqref{eq:thm:commutative_diagrams_preserved_and_reflected/source} directly.

  Now suppose that \eqref{eq:thm:commutative_diagrams_preserved_and_reflected/image} holds for a faithful functor \( F \). \ref{eq:def:functor/CF2} allows us to reduce \eqref{eq:thm:commutative_diagrams_preserved_and_reflected/image} to
  \begin{equation*}
    F(f_n \bincirc \cdots \bincirc \bincirc f_1) = F(g_m \bincirc \cdots \bincirc \bincirc g_1).
  \end{equation*}

  Then, by injectivity of \( F \) on the morphism set \( \cat{C}(\dom(f_1), \co\dom(f_1)) \), \eqref{eq:thm:commutative_diagrams_preserved_and_reflected/source} holds.
\end{proof}

\begin{definition}\label{def:natural_transformation}\mcite[def. 1.3.1]{Leinster2016Basic}
  Let \( F \) and \( G \) be parallel \hyperref[def:functor]{functors} from the category \( \cat{C} \) to \( \cat{D} \).

  A \term{natural transformation} \( \alpha \) from \( F \) to \( G \) is an \hyperref[def:cartesian_product/indexed_family]{indexed family} of
  \begin{equation}\label{eq:def:natural_transformation/family}
    \seq{ \alpha_A: F(A) \to G(A) }_{A \in \cat{C}}
  \end{equation}
  of morphisms in \( \cat{D} \) such that, for every morphism \( f: A \to B \) in \( \cat{C} \), the following \hyperref[def:categorical_diagram]{diagram commutes}:
  \begin{equation}\label{eq:def:natural_transformation/diagram}
    \begin{aligned}
      \includegraphics[page=1]{output/def__natural_transformation.pdf}
    \end{aligned}
  \end{equation}

  The morphisms \( \alpha_A \) are called the components of \( \alpha \). We denote natural transformations by \( \alpha: F \Rightarrow G \) and, when used in diagrams, by
  \begin{equation}\label{eq:def:natural_transformation/notation}
    \begin{aligned}
      \includegraphics[page=2]{output/def__natural_transformation.pdf}
    \end{aligned}
  \end{equation}
\end{definition}

\begin{example}\label{ex:quivers_as_functors}\mcite[exmpl. 1.3.46]{Perrone2019}
  In \fullref{def:quiver}, we have defined a quiver as a set \( V \) of vertices, a set \( A \) of arcs and two functions --- the head \( h: A \to V \)and tail \( t: A \to V \) of an arc.

  Now consider the following \hyperref[def:categorical_diagram]{index category} \( \cat{I}: \)
  \begin{equation}\label{eq:ex:quivers_as_functors/index/dots}
    \begin{aligned}
      \includegraphics[page=1]{output/ex__quivers_as_functors.pdf}
    \end{aligned}
  \end{equation}

  For the sake of readability, we will give the following explicit labels in this category:
  \begin{equation}\label{eq:ex:quivers_as_functors/index/annotated}
    \begin{aligned}
      \includegraphics[page=2]{output/ex__quivers_as_functors.pdf}
    \end{aligned}
  \end{equation}

  A quiver can then be defined as a functor \( Q: \cat{I} \to \ucat{Set} \) to the category \hyperref[def:category_of_small_sets]{\( \ucat{Set} \)} of \( \mscrU \)-small sets (for a \hyperref[def:category_size]{fixed Grothendieck universe} \( \mscrU \)).

  A \hyperref[def:natural_transformation]{natural transformation} from the quiver \( Q: \cat{I} \to \ucat{Set} \) to \( R: \cat{I} \to \ucat{Set} \) is then a pair of functions \( f_V: Q(V) \to R(V) \) and \( f_A: Q(A) \to R(A) \) such that the following \hyperref[def:categorical_diagram]{diagrams commute}:
  \begin{equation}\label{eq:ex:quivers_as_functors/index/diagram}
    \begin{aligned}
      \includegraphics[page=3]{output/ex__quivers_as_functors.pdf}
      \quad\quad\quad\quad
      \includegraphics[page=4]{output/ex__quivers_as_functors.pdf}
    \end{aligned}
  \end{equation}

  See \fullref{ex:isomorphism_of_quiver_categories} for how these functors related to quivers as defined in \fullref{def:quiver}.
\end{example}

\begin{remark}\label{rem:natural_transformations_into_set}
  Let \( \cat{C} \) be an arbitrary \( \mscrU \)-small category. A \hyperref[def:natural_transformation]{natural transformation} \( \alpha \) from \( F: \cat{C} \to \ucat{Set} \) to \( G: \cat{C} \to \ucat{Set} \) is then a family of functions
  \begin{equation*}
    \seq{ \alpha_A: F(A) \to G(A) }_{A \in \cat{C}}.
  \end{equation*}

  Suppose that for every two objects \( A \) and \( B \) in \( \cat{C} \), the functions \( \alpha_A \) and \( \alpha_B \) agree on \( F(A) \cap F(B) \). This is automatically satisfied in \( F(A) \) and \( F(B) \) are disjoint whenever \( A \neq B \).

  We can then take the set-theoretic union of \( \alpha \) to obtain the function
  \begin{equation*}
    \bigcup_{A \in \cat{C}} \alpha_A: \bigcup\set{ F(A) \given A \in \cat{C} } \to \bigcup\set{ G(A) \given A \in \cat{C} }.
  \end{equation*}

  Both the domain and codomain are sets as a consequence of \ref{def:grothendieck_universe/union}, therefore the function is well-defined in the universe \( \mscrU \). Denote it on \( \Alpha \) for brevity.

  An advantage of this is that we can define a natural transformation to be a function on a general enough set and then prove that its restrictions satisfy \eqref{eq:def:natural_transformation/diagram}.

  For example, consider the power set functor \( \pow: \ucat{Set} \to \ucat{Set} \) discussed in \fullref{ex:unary_functors_in_set}. The \hyperref[def:multi_valued_function/identity]{identity function} \( \id_\mscrU \) is then a natural transformation from the identity functor \( \id_{\ucat{Set}} \) to \( \pow \).

  Another natural transformation between the same functors is the singleton set operation \( \Sigma \) on sets defined as \( A \mapsto \set{ A } \). Note that, in this context, \( \Sigma \) operates not on the sets \( \id_{\ucat{Set}}(A) \) and \( \pow(A) \), but on their members. The diagram \eqref{eq:def:natural_transformation/diagram} becomes
  \begin{equation}\label{eq:rem:natural_transformations_into_set}
    \begin{aligned}
      \includegraphics[page=1]{output/rem__natural_transformations_into_set.pdf}
    \end{aligned}
  \end{equation}

  This diagram commutes because, for every function \( f: A \to B \) and every \( x \in A \), we have
  \begin{equation*}
    f[\set{ x }] = \set{ f(x) }.
  \end{equation*}
\end{remark}

\begin{definition}\label{def:functor_category}
  Let \( \cat{C} \) and \( \cat{D} \) be arbitrary \hyperref[def:category]{categories}. The \term{functor category} \( [\cat{C}, \cat{D}] \), also denoted as \( \cat{D}^{\cat{C}} \), is defined as follows:

  \begin{itemize}
    \item The \hyperref[def:category/objects]{set of objects} \( \obj([\cat{C}, \cat{D}]) \) is the set of all functors from \( \cat{C} \) to \( \cat{D} \).

    \item The \hyperref[def:category/morphisms]{set of morphisms} \( [\cat{C}, \cat{D}](F, G) \) from \( F \) to \( G \) is the set of all \hyperref[def:natural_transformation]{natural transformations} from \( F \) to \( G \).

    \item The \hyperref[def:category/composition]{composition of the morphisms} \( \alpha: F \Rightarrow G \) and \( \beta: G \Rightarrow H \) is the natural transformation \( \beta \bincirc \alpha: F \Rightarrow H \) defined in terms of componentwise morphism composition, i.e.
    \begin{equation}\label{eq:def:functor_category/composition}
      (\beta \bincirc \alpha)_A \coloneqq \beta_A \bincirc \alpha_A.
    \end{equation}

    \item The \hyperref[def:category/identity]{identity morphism} on the functor \( F: \cat{C} \to \cat{D} \) is the \term{identity natural transformation} \( \id_F: F \Rightarrow F \) with components
    \begin{equation}\label{eq:def:functor_category/identity}
      (\id_F)_A \coloneqq \underbrace{\id_{F(A)}}_{F(\id_A)}
    \end{equation}
  \end{itemize}
\end{definition}
\begin{defproof}
  Just to verify that the composition \( \beta \bincirc \alpha \) defined in \eqref{eq:def:functor_category/composition} is indeed a natural transformation from \( F \) to \( H \), note that the following diagram trivially commutes:
  \begin{equation}\label{def:functor_category/composition}
    \begin{aligned}
      \includegraphics[page=1]{output/def__functor_category.pdf}
    \end{aligned}
  \end{equation}

  Now, to see that \( [\cat{C}, \cat{D}] \) is indeed a category, we verify the conditions \ref{def:category/C1} and \ref{def:category/C2}, which are in turn inherited from the same conditions on the categories \( \cat{C} \) and \( \cat{D} \).

  \SubProofOf{def:category/C1} For every two functors \( F, G: \cat{C} \to \cat{D} \) and natural transformation \( \alpha: F \Rightarrow G \), for every object \( A \in \cat{C} \) we have
  \begin{equation*}
    \id_{G(A)} \bincirc \alpha_A
    \reloset{\eqref{def:category/C1}} =
    \alpha_A
    \reloset{\eqref{def:category/C1}} =
    \alpha_A \bincirc \id_{F(A)}
  \end{equation*}

  Therefore,
  \begin{equation*}
    \id_G \bincirc \alpha = \alpha = \alpha \bincirc \id_F
  \end{equation*}
  and, after generalizing, we obtain that \eqref{eq:def:category/C1} holds in \( [\cat{C}, \cat{D}] \).

  \SubProofOf{def:category/C2} For any quadruple \( F \), \( G \), \( H \) and \( T \) of functors from \( \cat{C} \) to \( \cat{D} \) and every combination of natural transformations \( \alpha: F \Rightarrow G \), \( \beta: G \Rightarrow H \) and \( \gamma: H \Rightarrow T \), for every object \( A \in \cat{C} \) we have
  \begin{equation*}
    (\gamma_A \bincirc \beta_A) \bincirc \alpha_A
    \reloset{\eqref{def:category/C2}} =
    \gamma_A \bincirc (\beta_A \bincirc \alpha_A).
  \end{equation*}

  Therefore, after generalizing, we obtain that \eqref{eq:def:category/C2} holds in \( [\cat{C}, \cat{D}] \).
\end{defproof}

\begin{remark}\label{rem:functor_category_size}
  If \( \cat{C} \) and \( \cat{D} \) are \( \mscrU \)-large categories in the sense of \fullref{def:category_size}, we cannot construct the \hyperref[def:functor_category]{functor category} \( [\cat{C}, \cat{D}] \). This is the main motivation for the \hyperref[def:axiom_of_universes]{axiom of universes}, which is discussed in \fullref{def:large_and_small_sets} and, in relation to category theory, in \fullref{def:category_size}.
\end{remark}

\begin{example}\label{ex:isomorphism_of_quiver_categories}
  In \fullref{ex:quivers_as_functors}, we defined \hyperref[def:quiver]{quivers} as functors from a certain index category \( \cat{I} \) to \( \ucat{Set} \) (for a \hyperref[def:category_size]{fixed Grothendieck universe} \( \mscrU \)).

  There is then an obvious correspondence between quivers as objects of \hyperref[def:category_of_small_quivers]{\( \ucat{Quiv} \)}, defined in \fullref{def:quiver}, and quivers as objects in the \hyperref[def:functor_category]{functor category} \( [\cat{I}, \ucat{Set}] \), defined in \fullref{ex:quivers_as_functors}. Indeed, given any functor \( Q: \cat{I} \to \ucat{Set} \), the quadruple
  \begin{equation*}
    \parens[\Big]{ Q(V), Q(A), Q(h), Q(t) }
  \end{equation*}
  is a quiver in the sense of \fullref{def:quiver}.

  No object in \( \ucat{Quiv} \) is formally equal to any object in \( [\cat{I}, \ucat{Set}] \) in the sense of \hyperref[def:zfc]{\logic{ZFC}}. They are, however, equivalent, as shown above, and this can be formalized by stating that the two categories are isomorphic, in the sense of \fullref{def:morphism_invertibility/isomorphism}, as objects of the category \( \ucat[\mscrV]{Cat} \), where \( \mscrV \) is a Grothendieck universe that strictly contains \( \mscrU \). We have already defined this isomorphism explicitly.

  This is an example of \term{isomorphism of categories}. In practice, if two categories are not so obviously identical, we are usually better served by \term{equivalences of categories} defined in \fullref{def:category_equivalence}.
\end{example}

\begin{definition}\label{def:dual_natural_transformation}\mcite{nLab:opposite_category}
  The \term{opposite} or \term{dual} natural transformation of \( \alpha: F \Rightarrow G \), where \( F \) and \( G \) are functors from \( \cat{C} \) to \( \cat{D} \), is the natural transformation \( \alpha^{\opcat}: G^{\opcat} \Rightarrow F^{\opcat} \), in which we take the opposite of each component in \( \alpha \).

  Dual natural transformation arise naturally in \fullref{thm:dual_functor_category}.
\end{definition}
\begin{defproof}
  The naturality diagram \eqref{eq:def:natural_transformation/diagram} commutes for \( \alpha^{\opcat} \) because all morphisms are simply reversed.
\end{defproof}

\begin{proposition}\label{thm:dual_functor_category}
  For the \hyperref[def:dual_category]{dual} of the \hyperref[def:functor_category]{functor category} \( [\cat{C}, \cat{D}] \) we have
  \begin{equation*}
    [\cat{C}, \cat{D}]^{\opcat} = [\cat{C}^{\opcat}, \cat{D}^{\opcat}].
  \end{equation*}

  This is part of the duality principles listed in \fullref{thm:categorical_principle_of_duality}.
\end{proposition}
\begin{proof}
  In \fullref{def:dual_functor}, we have defined the opposite functor \( F^{\opcat}: \cat{C}^{\opcat} \to \cat{D}^{\opcat} \) of \( F: \cat{C} \to \cat{D} \) in a way that allows us to regard it as an object of \( [\cat{C}^{\opcat}, \cat{D}^{\opcat}] \).

  In \fullref{def:dual_natural_transformation}, we have defined the opposite natural transformation \( \alpha^{\opcat}: G^{\opcat} \to F^{\opcat} \) of \( \alpha: F \Rightarrow G \) in a way that allows us to regard it as a morphism of \( [\cat{C}^{\opcat}, \cat{D}^{\opcat}] \).

  Furthermore, \( \alpha^{\opcat}: G^{\opcat} \to F^{\opcat} \) reverses the direction of its morphisms, and hence it is the dual to \( \alpha \) in the category \( [\cat{C}, \cat{D}]^{\opcat} \).
\end{proof}

\begin{definition}\label{def:diagonal_functor}\mcite[143]{Leinster2016Basic}
  Given an \term{index category} \( \cat{I} \) and an arbitrary category \( \cat{C} \), for any object \( A \) in \( \cat{C} \), we can define the \term{constant functor}
  \begin{equation*}
    \begin{aligned}
      &\Delta_A^{\cat{I}}: \cat{I} \to \cat{C}, \\
      &\Delta_A^{\cat{I}}(X) \coloneqq X, \\
      &\Delta_A^{\cat{I}}(g: X \to Y) \coloneqq \id_A.
    \end{aligned}
  \end{equation*}

  Given two objects \( A \) and \( B \) in \( \cat{C} \), a natural transformation \( \alpha: \Delta_A^{\cat{I}} \Rightarrow \Delta_B^{\cat{I}} \) is an \hyperref[def:cartesian_product/indexed_family]{indexed family} that gives the same morphism for every object of the index category \( \cat{I} \).

  Indeed, the diagram \eqref{eq:def:natural_transformation/diagram} in this case becomes
  \begin{equation}\label{eq:def:diagonal_functor/nat}
    \begin{aligned}
      \includegraphics[page=1]{output/def__diagonal_functor.pdf}
    \end{aligned}
  \end{equation}

  This diagram implies that \( \alpha_A = \alpha_B \) for any two objects \( A \) and \( B \) in \( \cat{I} \). Therefore, all components of \( \alpha \) are equal to some morphism in \( \cat{C}(A, B) \).

  We can now define the \( \cat{I} \)-shaped \term{diagonal functor} on \( \cat{C} \)
  \begin{equation*}
    \begin{aligned}
      &\Delta^{\cat{I}}: \cat{C} \to [\cat{I}, \cat{C}], \\
      &\Delta^{\cat{I}}(A) \coloneqq \Delta_A^{\cat{I}}, \\
      &\Delta^{\cat{I}}(f: A \to B) \coloneqq \seq{ f: A \to B }_{k \in \cat{I}}.
    \end{aligned}
  \end{equation*}

  It is called a diagonal functor because, if \( \cat{I} \) is a discrete category of two objects, then \( \Delta_{\cat{I}} \) gives the diagonal of the \hyperref[def:product_category]{product category} \( \cat{C}^2 \) by providing, for each object \( A \) of \( \cat{C} \), the ordered pair \( (A, A) \) (and similarly for morphisms).
\end{definition}

\begin{proposition}\label{thm:natural_isomorphism}\mcite{math3ma:natural_transformations}
  Let \( F \) and \( G \) be parallel \hyperref[def:functor]{functors} from the category \( \cat{C} \) to \( \cat{D} \). The family \eqref{eq:def:natural_transformation/family} is an isomorphism in the corresponding \hyperref[def:functor_category]{functor category} \( [\cat{C}, \cat{D}] \) if and only if all of its components are isomorphisms and, for any morphism \( f: A \to B \) in \( \cat{C} \), the following diagram commutes:
  \begin{equation}\label{eq:thm:natural_isomorphism/diagram}
    \begin{aligned}
      \includegraphics[page=1]{output/thm__natural_isomorphism.pdf}
    \end{aligned}
  \end{equation}

  We say that \( \alpha \) is a \term{natural isomorphism}.
\end{proposition}
\begin{proof}
  If all components of \( \alpha \) are isomorphisms, the condition
  \begin{equation*}
    \alpha_B \bincirc F(f) = G(f) \bincirc \alpha_A
  \end{equation*}
  is equivalent to
  \begin{equation*}
    F(f) = \alpha_B^{-1} \bincirc G(f) \bincirc \alpha_A.
  \end{equation*}

  We must now show that, if \( \alpha \) is an isomorphism in \( [\cat{C}, \cat{D}] \), all of its components are isomorphisms.

  If \( \alpha: F \Rightarrow G \) is an isomorphism in \( [\cat{C}, \cat{D}] \). Then there exists some natural transformation \( \beta: G \Rightarrow F \) such that
  \begin{equation*}
    \beta \bincirc \alpha = \id_F \quad\T{and}\quad \alpha \bincirc \beta = \id_G.
  \end{equation*}

  For every object \( A \) in \( \cat{C} \), the morphism \( \alpha_A: F(A) \to G(A) \) composed with \( \beta_A: G(A) \to F(A) \) is
  \begin{equation*}
    \beta_A \bincirc \alpha_A = \id_{F(A)}.
  \end{equation*}

  Therefore, \( \alpha_A \) is left-invertible. Analogously,
  \begin{equation*}
    \alpha_A \bincirc \beta_A = \id_{F(A)}
  \end{equation*}
  and hence \( \alpha_A \) is right-invertible.

  Therefore, for every object \( A \) in \( \cat{C} \), the morphism \( \alpha_A \) is fully invertible, i.e. an isomorphism.
\end{proof}

\begin{definition}\label{def:product_category}\mcite[const. 1.1.11]{Leinster2016Basic}
  We define the \term{product category} \( \cat{C} \times \cat{D} \) of \( \cat{C} \) and \( \cat{D} \) as follows:

  \begin{itemize}
    \item The \hyperref[def:category/objects]{set of objects} is the \hyperref[def:cartesian_product]{Cartesian product}
    \begin{equation}\label{eq:def:product_category/objects}
      \obj(\cat{C} \times \cat{D}) \coloneqq \obj(\cat{C}) \times \obj(\cat{D}).
    \end{equation}

    \item The \hyperref[def:category/morphisms]{set of morphisms} from the pair of objects \( (A, X) \) to \( (B, Y) \) is the product
    \begin{equation}\label{eq:def:product_category/morphisms}
      (\cat{C} \times \cat{D})\parens[\Big]{ (A, X), (B, Y) } \coloneqq \cat{C}(A, B) \times \cat{D}(X, Y).
    \end{equation}

    \item The \hyperref[def:category/composition]{composition of the morphisms}
    \begin{align*}
      (f, r)&: (A, X) \to (B, Y) \\
      (g, s)&: (B, Y) \to (C, Z)
    \end{align*}
    is the pairwise composition
    \begin{equation}\label{eq:def:product_category/composition}
      (g, s) \bincirc (f, r) \coloneqq \underbrace{(g \bincirc f, s \bincirc r)}_{(A, X) \to (C, Z)}.
    \end{equation}

    \item The \hyperref[def:category/identity]{identity morphism} of the pair \( (A, X) \) is simply the pair of identity morphisms \( (\id_A, \id_X) \).
  \end{itemize}
\end{definition}

\begin{definition}\label{def:hom_functor}
  Let \( \cat{C} \) be a \hyperref[def:category_size]{locally \( \mscrU \)-small} category. We can regard the morphism sets \( \cat{C}(A, B) \) as a functor parameterized by objects of \( \cat{C} \).

  \begin{thmenum}
    \thmitem{def:hom_functor/binary} For any pair of morphisms \( f: B \to A \) and \( g: X \to Y \) in \( \cat{C} \), define the operator
    \begin{equation}\label{eq:def:hom_functor/t}
      \begin{aligned}
        &T_{f,g}: \cat{C}(A, X) \to \cat{C}(B, Y) \\
        &T_{f,g}(s) \mapsto g \bincirc s \bincirc f.
      \end{aligned}
    \end{equation}

    The action of \( T_{f,g} \) can be expressed graphically as
    \begin{equation}\label{eq:def:hom_functor/t_diagram}
      \begin{aligned}
        \includegraphics[page=1]{output/def__hom_functor.pdf}
      \end{aligned}
    \end{equation}

    We can now define the following \term{binary hom-functor}:
    \begin{equation}\label{eq:def:hom_functor/binary}
      \begin{aligned}
        &\cat{C}(\anon*, \anon*): \cat{C}^{\opcat} \times \cat{C} \to \ucat{Set} \\
        &\cat{C}(A, X) \coloneqq \set{ s: A \to X } \\
        &\cat{C}(f, g) \coloneqq T_{f,g}
      \end{aligned}
    \end{equation}

    \thmitem{def:hom_functor/unary} Fixing the first argument \( A \) in \eqref{eq:def:hom_functor/binary}, we instead obtain a covariant unary hom-functor:
    \begin{equation}\label{eq:def:hom_functor/unary/covariant}
      \cat{C}(A, \anon*): \cat{C} \to \ucat{Set}
    \end{equation}

    Analogously, fixing the second argument \( X \), we obtain a \hyperref[def:hom_functor/unary]{contravariant} unary hom-functor:
    \begin{equation}\label{eq:def:hom_functor/unary/contravariant}
      \cat{C}(\anon*, X): \cat{C}^{\opcat} \to \ucat{Set}
    \end{equation}
  \end{thmenum}
\end{definition}
\begin{defproof}
  It is sufficient to verify that \eqref{eq:def:hom_functor/binary} defines a functor. \ref{def:functor/CF2} can be seen to hold by inspecting the diagram:
  \begin{equation}\label{eq:def:hom_functor/inv_composition}
    \begin{aligned}
      \includegraphics[page=2]{output/def__hom_functor.pdf}
    \end{aligned}
  \end{equation}

  The other functor condition \ref{def:functor/CF1} is straightforward to prove.
\end{defproof}

\begin{proposition}\label{thm:currying_is_natural_isomorphism}
  \hyperref[def:function/currying]{Function currying} is a natural isomorphism between the functors
  \begin{align*}
    &\cat{Set}(A \times B, C)
    &\cat{Set}(A, \cat{Set}(B, C))
  \end{align*}

  More concretely, consider the following functors, which are loosely based on \fullref{def:hom_functor}:
  \begin{equation*}
    \begin{aligned}
      &V: \cat{Set}^3 \to \cat{Set} \\
      &V(A, B, C) \coloneqq \cat{Set}(A \times B, C) \\
      \Big[& V(f: X \to A, g: Y \to B, h: C \to Z) \Big](s: A \times B \to C) \coloneqq \smash{ \overbrace{ (x, y) \mapsto h\parens[\Bigg]{ s\parens[\Big]{ \underbrace{ f(x), g(y) }_{A \times B} } } }^{X \times Y \to Z} } \\
    \end{aligned}
  \end{equation*}
  and
  \begin{equation*}
    \begin{aligned}
      &W: \cat{Set}^3 \to \cat{Set} \\
      &W(A, B, C) \coloneqq \cat{Set}(A, \cat{Set}(B, C)) \\
      \Big[& W(f: X \to A, g: Y \to B, h: C \to Z) \Big](t: A \to \cat{Set}(B, C)) \coloneqq \smash{ \underbrace{ x \mapsto \overbrace{y \mapsto h\parens[\Bigg]{ \overbrace{t(f(x))}^{B \to C}\parens[\Big]{ g(y) } } }^{Y \to Z} }_{X \to \cat{Set}(Y, Z)} } \\
    \end{aligned}
  \end{equation*}

  Then the family of functions
  \begin{equation*}
    \begin{aligned}
      &\alpha: V \Rightarrow W \\
      &\alpha_{A,B,C}(s: A \times B \to C) \coloneqq a \mapsto b \mapsto s(a, b)
    \end{aligned}
  \end{equation*}
  is a \hyperref[thm:natural_isomorphism]{natural isomorphism}.
\end{proposition}
\begin{proof}
  The function \( \varphi \) is clearly invertible. Fix a triple of functions \( f: X \to A \), \( g: Y \to B \) and \( h: C \to Z \). For every \( s: A \times B \to C \) we have
  \begin{align*}
    [W(f, g, h)](\alpha_{A,B,C}(s))
    &=
    x \mapsto y \mapsto \parens[\Big]{ h\parens[\Big]{ [a \mapsto b \mapsto s(a, b)](f(x))(g(y)) } }
    = \\ &=
    x \mapsto y \mapsto h\parens[\Big]{ s(f(x), g(y)) }
    = \\ &=
    \alpha_{X,Y,Z}(V(f, g, h)),
  \end{align*}
  which proves that the following diagram commutes:
  \begin{equation}\label{eq:thm:currying_is_natural_isomorphism/diagram}
    \begin{aligned}
      \includegraphics[page=1]{output/thm__currying_is_natural_isomorphism.pdf}
    \end{aligned}
  \end{equation}
\end{proof}

\subsection{Adjoint functors}\label{subsec:adjoint_functors}

\begin{remark}\label{def:adjoint_functors}
  Adjoint functors are a generalization of invertible functors. When the functor \( F: \cat{A} \to \boldop B \) is left adjoint to \( G: \cat{A} \to \boldop B \) and \( G \) is not invertible, then \( F \) finds a \enquote{generalized inverse} for every object in \( \boldop B \) that tries to preserve its morphisms.

  Jean-Pierre Marquis in \cite{StanfordPlato:category_theory} refers to adjoint functors as \enquote{conceptual inverses} that reverse the idea rather than the realization.
\end{remark}

\begin{definition}\label{def:adjoint_functor}\mcite\cite[exer. 4.1.32]{Leinster2016Basic}
  Let \( \cat{A} \) and \( \boldop B \) be locally small categories and \( F: \cat{A} \to \boldop B \) and \( G: \boldop B \to \cat{A} \) be functors.

  We say that \( F \) is \term{left-adjoint} to \( G \) and \( G \) is \term{right-adjoint} to \( F \) and write \( F \dashv G \) if there is a natural \hyperref[def:natural_isomorpism]{isomorphism} between the functors \( \cat{A}(-, G(-)) \) and \( \cat{B}(F(-), -) \).
\end{definition}

\begin{example}\label{ex:top_adjoint_functor}\mcite\cite[exmpl. 2.1.5]{Leinster2016Basic}
  Consider the functors
  \begin{itemize}
    \item \( U: \cat{Top} \to \cat{Set} \), which maps topological spaces to their underlying sets.
    \item \( D: \cat{Set} \to \cat{Top} \), which maps sets to topological spaces equipped with the discrete \hyperref[def:standard_topologies/discrete]{topology}.
    \item \( I: \cat{Set} \to \cat{Top} \), which maps sets to topological spaces equipped with the indiscrete \hyperref[def:standard_topologies/indiscrete]{topology}.
  \end{itemize}

  Let \( T \in \cat{Top} \) and \( S \in \cat{Set} \). Every function from a discrete topological space is continuous, therefore
  \begin{equation*}
    \cat{Top}(D(S), T) = \cat{Set}(S, U(T)),
  \end{equation*}
  where equality means that all the underlying functions are equal.

  Similarly, every function into an indiscrete topological space is continuous, therefore
  \begin{equation*}
    \cat{Top}(T, I(S)) = \cat{Set}(U(T), S).
  \end{equation*}
\end{example}

\subsection{Representable functors}\label{subsec:representable_functors}

\begin{definition}\label{def:representable_functor}\mcite\cite[def. 4.1.3 \\ def. 4.1.16]{Leinster2014}
  Let \( \cat{A} \) be a locally small category and \( A \in \cat{A} \). Define
  \begin{balign*}
     & H^A: \cat{A} \to \cat{Set},                                          \\
     & H^A(B) \coloneqq \cat{A}(A, B),                                      \\
     & H^A(f: B \to C) \coloneqq (p: A \to B) \mapsto (f \circ p: A \to C).
  \end{balign*}

  We say that the functor \( F: \cat{A} \to \cat{Set} \) is \term{representable} with \term{representation} \( H^A \) if \( F \cong H^A \).

  Analogously, the presheaf \( G: \cat{A}^{-1} \to \cat{Set} \) is representable if for some \( A \in \cat{A}^{-1} \) we have \( G \cong H_A \), where
  \begin{balign*}
     & H_A: \cat{A}^{-1} \to \cat{Set},                                     \\
     & H_A(B) \coloneqq \cat{A}(B, A),                                      \\
     & H_A(f: C \to B) \coloneqq (p: B \to A) \mapsto (p \circ f: C \to A).
  \end{balign*}
\end{definition}

\begin{example}\label{def:top_representable_functor}\mcite\cite[exmpl. 4.1.4]{Leinster2014}
  Let \( U: \cat{Top} \to \cat{Set} \) be the forgetful functor which maps a topological space to its underlying set.

  Let \( 1 \) be the one-element topological space. There is a correspondence between points \( x \) in \( T \) and continuous functions \( p_x: 1 \to T \). Thus the functor \( H^1 \) maps
  \begin{itemize}
    \item any topological space \( T \) into the set of morphisms
          \begin{equation*}
            H^1(T) = \cat{Top}(1, T) = \{ p_x: 1 \to T \} \cong U(T).
          \end{equation*}
    \item any continuous function \( f: T \to S \) to
          \begin{equation*}
            H^1(f) = p_x \mapsto f \circ p_x \cong x \mapsto f(x) = f.
          \end{equation*}
  \end{itemize}

  Thus \( U \) is representable with representation \( H^1 \).
\end{example}

\begin{definition}\label{def:yoneda_embedding}\mcite\cite[def. 4.1.15 \\ def. 4.1.21]{Leinster2014}
  Let \( \cat{A} \) be a locally small category. For each pair \( A, B \in \cat{A} \) and morphism \( f: A \to B \) we define the natural transformation \( H^f: H^A \to H^B \) with \( C \)-components (note the reversal)
  \begin{balign*}
     & H^f(C): H^A(C) \to H^B(C),                     \\
     & H^f(C) \coloneqq H_C(f) = p \mapsto p \circ f.
  \end{balign*}

  Thus allows us to define the functor \( H^\bullet: \cat{A}^{-1} \to [\cat{A}, \cat{Set}] \) by
  \begin{balign*}
    H^\bullet(A) \coloneqq H^A &  & H^\bullet(f) \coloneqq H^f.
  \end{balign*}

  Analogously, we define \( H_f \) by \( H_f(C) = H^C(f), C \in \cat{A} \) and the \term{Yoneda embedding} \( H_\bullet: \cat{A} \to [\cat{A}^{-1}, \cat{Set}] \) by
  \begin{balign*}
    H_\bullet(A) \coloneqq H_A &  & H_\bullet(f) \coloneqq H_f.
  \end{balign*}
\end{definition}

\begin{proposition}\label{thm:yoneda_embedding_is_injective}\mcite\cite[exer. 4.1.27]{Leinster2014}
  Let \( \cat{A} \) be a locally small category and let \( A, A' \in \cat{A} \) be such that \( H_A \cong H_{A'} \). Then \( A \cong A' \).
\end{proposition}
\begin{proof}
  First, let \( A \) and \( A' \) be arbitrary. Given a natural isomorphism \( \eta: H_A \to H_{A'} \), its components are \( \eta_B: H_A(B) \to H_{A'}(B) \).

  We are interested in the morphisms
  \begin{balign*}
     & f \coloneqq \eta_A(\id_A): A \to A',          \\
     & g \coloneqq \eta_{A'}^{-1}(\id_A'): A' \to A.
  \end{balign*}

  We need to show that \( g \) is inverse to \( f \). We will use the commutativity of the following diagram:
  \begin{equation*}
    \begin{mplibcode}
      beginfig(1);
      input metapost/graphs;

      v1 := thelabel("$H_A(A)$", origin);
      v2 := thelabel("$H_{A'}(A)$", (0, -1) scaled u);
      v3 := thelabel("$H_A(A')$", (2, 0) scaled u);
      v4 := thelabel("$H_{A'}(A')$", (2, -1) scaled u);

      a1 := straight_arc(v1, v2);
      a2 := straight_arc(v3, v1);
      a3 := straight_arc(v4, v2);
      a4 := straight_arc(v3, v4);

      draw_vertices(v);
      draw_arcs(a);

      label.lft("$\eta_A$", straight_arc_midpoint of a1);
      label.top("$H_A(f)$", straight_arc_midpoint of a2);
      label.bot("$H_{A'}(f)$", straight_arc_midpoint of a3);
      label.rt("$\eta_{A'}$", straight_arc_midpoint of a4);
      endfig;
    \end{mplibcode}
  \end{equation*}
  where
  \begin{balign*}
     & H_A(f: A \to A') = (p: A' \to A) \mapsto (p \circ f: A \to A),      \\
     & H_{A'}(f: A \to A') = (p: A' \to A') \mapsto (p \circ f: A \to A').
  \end{balign*}

  In particular,
  \begin{equation*}
    \begin{mplibcode}
      beginfig(1);
      input metapost/graphs;

      v1 := thelabel("$g \circ f$", origin);
      v2 := thelabel("$\eta_A(g \circ f) = H_{A'}(f)(\id_{A'}) = f$", (-1, -1) scaled u);
      v3 := thelabel("$g$", (3, 0) scaled u);
      v4 := thelabel("$\eta_{A'}(\eta_{A'}^{-1}(\id_{A'})) = \id_{A'}$", (4, -1) scaled u);

      a1 := straight_arc(v1, v2);
      a2 := straight_arc(v3, v1);
      a3 := straight_arc(v4, v2);
      a4 := straight_arc(v3, v4);

      draw_vertices(v);
      draw_arcs(a);

      label.ulft("$\eta_A$", straight_arc_midpoint of a1);
      label.top("$H_A(f)$", straight_arc_midpoint of a2);
      label.bot("$H_{A'}(f)$", straight_arc_midpoint of a3);
      label.urt("$\eta_{A'}$", straight_arc_midpoint of a4);
      endfig;
    \end{mplibcode},
  \end{equation*}
  that is,
  \begin{equation*}
    \eta_A(g \circ f) = f = \eta_A(\id_A).
  \end{equation*}

  Since \( \eta_A \) is a bijection, we conclude that \( g \circ f = \id_A \).

  Analogously, we obtain that \( f \circ g = \id_{A'} \). Thus \( f: A \to A' \) is an isomorphism, the inverse being \( g: A' \to A \).
\end{proof}

\begin{theorem}(Yoneda's lemma)\label{def:yoneda_lemma}\mcite\cite[thm. 4.2.1]{Leinster2014}
  Let \( \cat{A} \) be a locally small category. Then there is a natural isomorphism between the functors
  \begin{balign*}
     & \cat{A}^{-1} \times [\cat{A}^{-1}, \cat{Set}] \to \cat{Set} \\
     & (A, X) \mapsto X(A)
  \end{balign*}
  and
  \begin{balign*}
     & \cat{A}^{-1} \times [\cat{A}^{-1}, \cat{Set}] \to \cat{Set} \\
     & (A, X) \mapsto [\cat{A}^{-1}, \cat{Set}](H_A, X).
  \end{balign*}
\end{theorem}

\subsection{Free objects}\label{subsec:free_objects}

\begin{remark}\label{rem:free_construction_etymology}
  Free constructions in algebra such as \hyperref[def:free_monoid]{free monoids}, \hyperref[def:free_group]{free groups} and \hyperref[def:free_left_module]{free modules} all share a common feature - they allow us to construct an algebraic structure \enquote{freely}, without imposing any restrictions on the target structure. See \fullref{def:group_presentation} for a more precise justification. Jean-Pierre Marquis in \cite{StanfordPlato:category_theory} suggests that this is the reason free constructions are called \enquote{free}.

  The category-theoretic formulation of free functors and, more generally, free objects, is based on the idea that free \hyperref[def:free_functor]{functors} \enquote{remember} what \hyperref[def:forgetful_functor]{forgetful functors} \enquote{forget}.
\end{remark}

\begin{definition}\label{def:forgetful_functor}\mcite\cite[exmpl. 1.2.3]{Leinster2016Basic}
  A functor \( U: \cat{A} \to \cat{B} \) is called \term{forgetful} if it \enquote{forgets structure}. This \enquote{definition} is informal, but often a unique free functor is clear from the context and so the definition often makes perfect sense. See \fullref{ex:forgetful_functors}.
\end{definition}

\begin{example}\label{ex:forgetful_functors}
  Examples of forgetful functors include

  \begin{thmenum}
    \item The identity functor \( U = \id \) on \( \cat{Set} \), which forgets nothing.
    \item The functors \( U: \cat{Grp} \to \cat{Set} \) and \( U: \cat{Ring} \to \cat{Set} \), which forget the group and ring operations.
    \item The functor \( U: \cat{Top} \to \cat{Set} \) from \fullref{ex:top_adjoint_functor}, which forgets the topology.
    \item More generally, given a \hyperref[def:first_order_theory]{first-order theory} \( \Gamma \) and its corresponding category \( \cat{Model}_\Gamma \), the functor \( U: \cat{Model}_\Gamma \to \cat{Set} \) forgets any additional conditions imposed on the underlying sets of the models.
  \end{thmenum}
\end{example}

\begin{definition}\label{def:free_functor}\mcite\cite{nLab:free_object}
  A left \hyperref[subsec:adjoint_functors]{adjoint} to a \hyperref[def:forgetful_functor]{forgetful functor} is called a free functor.
\end{definition}

\begin{example}\label{ex:free_functors}
  Examples of free functors include

  \begin{thmenum}
    \item The identity functor \( U = \id \) on \( \cat{Set} \).
    \item The functor \( D: \cat{Set} \to \cat{Top} \) from \fullref{ex:top_adjoint_functor}, which sends a set to its corresponding topological space.
    \item The functor \( F: \cat{Set} \to \cat{Grp} \), which sends a set to its corresponding \hyperref[def:free_group]{free group}.
    \item The functor \( F: \cat{Set} \to \cat{Mod}_R \), which sends a set \( S \) to a \hyperref[def:free_left_module]{free module} whose basis is \( S \).
  \end{thmenum}
\end{example}

\begin{definition}\label{def:free_object}\mcite\cite{nLab:free_object}
  A free object satisfies the adjoint condition at a single point. Let \( U: \cat{A} \to \cat{B} \) be a forgetful functor and let \( b \in \cat{B} \). We say that \( a \in \cat{A} \) is a \term{free object} with respect to the forgetful functor \( U \) if there exists a morphism \( \eta_{a}: b \to Ua \) that satisfy the following universal property: for every object \( a' \in \cat{A} \) and morphism \( \eta_{a'}: b \to Ua' \), there exists a unique morphism \( f: a \to a' \) such that \( U(f) \circ \eta_a = \eta_{a'} \), that is, the following diagram commutes:
  \begin{equation*}
    \todo{Add diagram}\iffalse\begin{mplibcode}
      beginfig(1);
      input metapost/graphs;

      v1 := thelabel("$b$", origin);
      v2 := thelabel("$U(a)$", (-1, -1) scaled u);
      v3 := thelabel("$U(a')$", (1, -1) scaled u);

      a1 := straight_arc(v1, v2);
      a2 := straight_arc(v1, v3);

      d1 := straight_arc(v2, v3);

      draw_vertices(v);
      draw_arcs(a);

      drawarrow d1 dotted;

      label.ulft("$\eta_{a}$", straight_arc_midpoint of a1);
      label.urt("$\eta_{a'}$", straight_arc_midpoint of a2);
      label.top("$U(f)$", straight_arc_midpoint of d1);
      endfig;
    \end{mplibcode}\fi
  \end{equation*}
\end{definition}

\subsection{Limits}\label{subsec:categorical_limits}

\begin{remark}\label{def:categorical_limit_examples}
  Examples of limits and colimits can be found in \fullref{thm:set_categorical_limits}, \fullref{thm:group_categorical_limits} and \fullref{subsec:initial_final_topologies}.
\end{remark}

\begin{definition}\label{def:diagonal_functor}\mcite\cite[143]{Leinster2016Basic}
  Let \( \cat{I} \) be a small index category and let \( \cat{A} \) be any category. For each object \( A \in \cat{A} \), we define the functor \( \Delta A: \cat{I} \to \cat{A} \) as
  \begin{itemize}
    \item for every object \( k \in \cat{I} \), define \( \Delta A(k) = A \)
    \item for every morphism \( u: k \to \beta \), define \( \Delta A(u) = \id_A \)
  \end{itemize}

  We combine these functors for every object \( A \in \cat{A} \) to obtain the functor \( \Delta: \cat{A} \to [\cat{I}, \cat{A}] \).
\end{definition}

\begin{definition}\label{def:categorical_cone}\mcite\cite[def. 5.1.19(a)]{Leinster2016Basic}
  Let \( \cat{A} \) be a category and \( \cat{I} \) be a \hyperref[def:categorical_diagram]{category} which we shall call an \term{index} category. Let \( D: \cat{I} \to \cat{A} \) be a diagram. A \term{cone} on \( D \) can be defined equivalently as:

  \begin{thmenum}
    \labitem{def:categorical_cone/explicit} A family of \term{projection} morphisms \( \{ \pi_k: A \to D(k) \}_{k \in \cat{I}} \) from the \term{vertex} \( A \) such that for all morphisms \( u: k \to \beta \) in \( \cat{I} \), the following diagram commutes:
    \begin{alignedeq}\label{def:categorical_cone/universal_property}
      \todo{Add diagram}\iffalse\begin{mplibcode}
        beginfig(1);
        input metapost/graphs;

        v1 := thelabel("$A$", origin);
        v2 := thelabel("$D(k)$", (-1, -1) scaled u);
        v3 := thelabel("$D(\beta)$", (1, -1) scaled u);

        a1 := straight_arc(v1, v2);
        a2 := straight_arc(v1, v3);
        a3 := straight_arc(v2, v3);

        draw_vertices(v);
        draw_arcs(a);

        label.ulft("$\pi_k$", straight_arc_midpoint of a1);
        label.urt("$\pi_\beta$", straight_arc_midpoint of a2);
        label.bot("$D(u)$", straight_arc_midpoint of a3);
        endfig;
      \end{mplibcode}\fi
    \end{alignedeq}

    \labitem{def:categorical_cone/natural} A natural transformation in \( [\cat{I}, \cat{A}](\Delta A, D) \).

    \labitem{def:categorical_cone/comma} An object of the comma category \( (\Delta \downarrow D) \) (see the equivalence proof for details).
  \end{thmenum}
\end{definition}
\begin{proof}
  \EquivalenceSubProof{def:categorical_cone/explicit}{def:categorical_cone/natural} Let \( k, \beta \in \cat{I} \) and \( u: k \to \beta \). Then a natural transformation \( f \) in  satisfies the following commutative diagram:
  \begin{equation*}
    \todo{Add diagram}\iffalse\begin{mplibcode}
      beginfig(1);
      input metapost/graphs;

      v1 := thelabel("$\Delta A(k)$", (-1, 0) scaled u);
      v2 := thelabel("$\Delta A(\beta)$", (1, 0) scaled u);
      v3 := thelabel("$D(k)$", (-1, -1) scaled u);
      v4 := thelabel("$D(\beta)$", (1, -1) scaled u);

      a1 := straight_arc(v1, v2);
      a2 := straight_arc(v1, v3);
      a3 := straight_arc(v2, v4);
      a4 := straight_arc(v3, v4);

      draw_vertices(v);
      draw_arcs(a);

      label.top("$\Delta A(u)$", straight_arc_midpoint of a1);
      label.lft("$\pi_k$", straight_arc_midpoint of a2);
      label.rt("$\pi_\beta$", straight_arc_midpoint of a3);
      label.bot("$D(u)$", straight_arc_midpoint of a4);
      endfig;
    \end{mplibcode}\fi
  \end{equation*}

  Since \( \Delta A(k) = \Delta A(\beta) = A \), the above diagram is the same as \fullref{def:categorical_cone/universal_property}.

  \EquivalenceSubProof{def:categorical_cone/natural}{def:categorical_cone/comma} We can regard \( D: \cat{I} \to \cat{A} \) as an object in the functor category \( [\cat{I}, \cat{A}] \). Since \( \Delta: \cat{A} \to [\cat{I}, \cat{A}] \), an object \( (A, h) \) in \( (\Delta \downarrow D) \) consists of an object \( A \) of \( \cat{A} \) and a natural transformation from \( \Delta A \) to \( D \). The converse also applies.
\end{proof}

\begin{definition}\label{def:categorical_limit}\mcite\cite[def. 5.1.19(b), 6.3.6]{Leinster2016Basic}
  Let \( \cat{A} \) be a category and \( \cat{I} \) be an index category. The (unique up to an isomorphism, if it exists) \term{limit} or \term{limit cone} \( \varprojlim D \) of \( D \) is a cone
  \begin{equation*}
    \{ L \overset {\pi_k} \to D(k) \}_{k \in \cat{I}}
  \end{equation*}
  such that for every cone
  \begin{equation*}
    \{ L' \overset {\pi_k'} \to D(k) \}_{k \in \cat{I}}
  \end{equation*}
  there exists exactly one morphism \( f: L' \to L \) such that \( f \circ \pi_k' = \pi_k, k \in \cat{I} \), i.e. the following diagram commutes:
  \begin{equation*}
    \todo{Add diagram}\iffalse\begin{mplibcode}
      beginfig(1);
      input metapost/graphs;

      v1 := thelabel("$D(k)$", origin);
      v2 := thelabel("$L'$", (-1, 1) scaled u);
      v3 := thelabel("$L$", (1, 1) scaled u);

      a1 := straight_arc(v2, v1);
      a2 := straight_arc(v3, v1);

      d1 := straight_arc(v2, v3);

      draw_vertices(v);
      draw_arcs(a);

      drawarrow d1 dotted;

      label.llft("$\pi_k$", straight_arc_midpoint of a1);
      label.lrt("$\pi_k'$", straight_arc_midpoint of a2);
      label.top("$f$", straight_arc_midpoint of d1);
      endfig;
    \end{mplibcode}\fi
  \end{equation*}

  If the diagram \( \cat{I} \) is small, its limit is called a \term{small limit}. If a category \( \cat{A} \) has all small limits, it is called \term{complete}.
\end{definition}

\begin{definition}\label{def:categorical_product}\mcite\cite[def. 5.1.1, 5.1.7]{Leinster2016Basic}
  If the index category \( \cat{I} \) is discrete, then any diagram \( D: \cat{I} \to \cat{A} \) is simply an indexed family \( \{ X_k \}_{k \in \cat{I}} \) of objects of \( \cat{A} \). In this case, the limit \( L \) does not depend on the functor \( D \). We call it the \term{product in \( \cat{A} \) indexed by \( \cat{I} \)} and denote it by \( \prod_{k \in \cat{I}} X_k \).

  Explicitly, the \term{product} of \( \{ X_k \}_{k \in \cat{I}} \) is an object \( P \coloneqq \prod_{k \in \cat{I}} X_k \) with associated \term{projection morphisms} \( \{ \pi_k: P \to X_k \}_{k \in \cat{I}} \), that satisfy the following universal property: for any object \( P' \) and any family of morphisms \( \{ \pi_k': P' \to {X_k} \}_{k \in \cat{I}} \) there exists exactly one morphism \( f: P' \to P \) such that for every \( k \in \cat{I} \) we have \( f \circ \pi_k = \pi_k' \), i.e. the following diagram commutes:
  \begin{equation*}
    \todo{Add diagram}\iffalse\begin{mplibcode}
      beginfig(1);
      input metapost/graphs;

      v1 := thelabel("$X_k$", origin);
      v2 := thelabel("$P'$", (-1, 1) scaled u);
      v3 := thelabel("$P$", (1, 1) scaled u);

      a1 := straight_arc(v2, v1);
      a2 := straight_arc(v3, v1);

      d1 := straight_arc(v2, v3);

      draw_vertices(v);
      draw_arcs(a);

      drawarrow d1 dotted;

      label.llft("$\pi_k'$", straight_arc_midpoint of a1);
      label.lrt("$\pi_k$", straight_arc_midpoint of a2);
      label.top("$f$", straight_arc_midpoint of d1);
      endfig;
    \end{mplibcode}\fi
  \end{equation*}

  The function \( f \) is also denoted as \( \{ f_k \}_{k \in \cat{I}} \).

  In particular, for two objects \( X, Y \in \cat{A} \) (i.e. when \( \cat{I} \) is a two-object discrete category), the product is an object \( X \times Y \) with projections \( \pi_X: X \times Y \to X \) and \( \pi_Y: X \times Y \to Y \) such that for each object \( P' \) and morphisms \( \pi_X': P' \to X \) and \( \pi_Y': P' \to Y \) the following diagram commutes:
  \begin{equation*}
    \todo{Add diagram}\iffalse\begin{mplibcode}
      beginfig(1);
      input metapost/graphs;

      v1 := thelabel("$P'$", (-1, 1) scaled u);
      v2 := thelabel("$X \times Y$", origin);
      v3 := thelabel("$X$", (0, -1) scaled u);
      v4 := thelabel("$Y$", (1, 0) scaled u);

      a1 := straight_arc(v1, v3);
      a2 := straight_arc(v1, v4);
      a3 := straight_arc(v2, v3);
      a4 := straight_arc(v2, v4);

      d1 := straight_arc(v1, v2);

      draw_vertices(v);
      draw_arcs(a);

      drawarrow d1 dotted;

      label.llft("$\pi_X'$", straight_arc_midpoint of a1);
      label.urt("$\pi_Y'$", straight_arc_midpoint of a2);
      label.rt("$\pi_X$", straight_arc_midpoint of a3);
      label.bot("$\pi_Y$", straight_arc_midpoint of a4);

      fill fullcircle scaled 0.25u shifted (center d1) withcolor white;
      label("$f$", straight_arc_midpoint of d1);
      endfig;
    \end{mplibcode}\fi
  \end{equation*}
\end{definition}

\begin{remark}\label{rem:small_categorical_product}
  If the discrete category \( \cat{I} \) is small, denote the set of its objects by \( I \). This allows us to talk about products of families \( \{ X_k \}_{k \in I} \) indexed by the set \( I \) rather than the category \( \cat{I} \).
\end{remark}

\begin{remark}\label{rem:empty_categorical_product}
  The product \( \prod_{k \in \varnothing} X_k \) of an empty family of objects is the terminal object of the category.
\end{remark}

\begin{definition}\label{def:categorical_fork}\mcite\cite[112]{Leinster2016Basic}
  A \term{fork} in the category \( \cat{A} \) is a commutative diagram of the form
  \begin{equation*}
    \todo{Add diagram}\iffalse\begin{mplibcode}
      beginfig(1);
      input metapost/graphs;

      v1 := thelabel("$A$", origin);
      v2 := thelabel("$X$", (1, 0) scaled u);
      v3 := thelabel("$Y$", (2, 0) scaled u);

      a1 := straight_arc(v1, v2);
      a2 := straight_arc_shifted(v2, v3, (0, safe_arc_spacing));
      a3 := straight_arc_shifted(v2, v3, (0, -safe_arc_spacing));

      draw_vertices(v);
      draw_arcs(a);

      label.top("$f$", straight_arc_midpoint of a1);
      label.top("$s$", straight_arc_midpoint of a2);
      label.bot("$t$", straight_arc_midpoint of a3);
      endfig;
    \end{mplibcode}\fi
  \end{equation*}

  Commutativity simply means that \( s \circ f = t \circ f \).
\end{definition}

\begin{definition}\label{def:categorical_equalizer}\mcite\cite[def. 5.1.11]{Leinster2016Basic}
  Assume that the index category \( \cat{I} \) consists of two objects and two unidirectional morphisms:
  \begin{equation*}
    \todo{Add diagram}\iffalse\begin{mplibcode}
      beginfig(1);
      input metapost/graphs;

      v1 := thelabel("$\bullet$", (1, 0) scaled u);
      v2 := thelabel("$\bullet$", (2, 0) scaled u);

      a1 := straight_arc_shifted(v1, v2, (0, safe_arc_spacing));
      a2 := straight_arc_shifted(v1, v2, (0, -safe_arc_spacing));

      draw_vertices(v);
      draw_arcs(a);
      endfig;
    \end{mplibcode}\fi
  \end{equation*}

  Diagrams \( D \) of shape \( \cat{I} \) are simply subcategories of \( \cat{A} \) of the shape
  \begin{equation*}
    \todo{Add diagram}\iffalse\begin{mplibcode}
      beginfig(1);
      input metapost/graphs;

      v1 := thelabel("$X$", (1, 0) scaled u);
      v2 := thelabel("$Y$", (2, 0) scaled u);

      a1 := straight_arc_shifted(v1, v2, (0, safe_arc_spacing));
      a2 := straight_arc_shifted(v1, v2, (0, -safe_arc_spacing));

      draw_vertices(v);
      draw_arcs(a);

      label.top("$s$", straight_arc_midpoint of a1);
      label.bot("$t$", straight_arc_midpoint of a2);
      endfig;
    \end{mplibcode}\fi
  \end{equation*}

  Cones with vertex \( A \) are then given by commutative diagrams of shape
  \begin{equation*}
    \todo{Add diagram}\iffalse\begin{mplibcode}
      beginfig(1);
      input metapost/graphs;

      v1 := thelabel("$A$", origin);
      v2 := thelabel("$X$", (-1, -1) scaled u);
      v3 := thelabel("$Y$", (1, -1) scaled u);

      a1 := straight_arc(v1, v2);
      a2 := straight_arc(v1, v3);
      a3 := straight_arc_shifted(v2, v3, (0, safe_arc_spacing));
      a4 := straight_arc_shifted(v2, v3, (0, -safe_arc_spacing));

      draw_vertices(v);
      draw_arcs(a);

      label.ulft("$f$", straight_arc_midpoint of a1);
      label.urt("$g$", straight_arc_midpoint of a2);
      label.top("$s$", straight_arc_midpoint of a3);
      label.bot("$t$", straight_arc_midpoint of a4);
      endfig;
    \end{mplibcode}\fi
  \end{equation*}

  Since the morphism \( g: A \to Y \) is determined uniquely by \( f \) and \( s \), the cones are actually forks:
  \begin{equation*}
    \todo{Add diagram}\iffalse\begin{mplibcode}
      beginfig(1);
      input metapost/graphs;

      v1 := thelabel("$A$", origin);
      v2 := thelabel("$X$", (1, 0) scaled u);
      v3 := thelabel("$Y$", (2, 0) scaled u);

      a1 := straight_arc(v1, v2);
      a2 := straight_arc_shifted(v2, v3, (0, safe_arc_spacing));
      a3 := straight_arc_shifted(v2, v3, (0, -safe_arc_spacing));

      draw_vertices(v);
      draw_arcs(a);

      label.top("$f$", straight_arc_midpoint of a1);
      label.top("$s$", straight_arc_midpoint of a2);
      label.bot("$t$", straight_arc_midpoint of a3);
      endfig;
    \end{mplibcode}\fi
  \end{equation*}

  The limit \( (L, l) \) of \( D \) then satisfies the universal property: for any fork \( (L', l') \), there exists a unique morphism \( f: L' \to L \) such that the following diagram commutes:
  \begin{equation*}
    \todo{Add diagram}\iffalse\begin{mplibcode}
      beginfig(1);
      input metapost/graphs;

      v1 := thelabel("$X$", origin);
      v2 := thelabel("$Y$", (1, 0) scaled u);
      v3 := thelabel("$L'$", (-1, 1) scaled u);
      v4 := thelabel("$L$", (-1, -1) scaled u);

      a1 := straight_arc_shifted(v1, v2, (0, safe_arc_spacing));
      a2 := straight_arc_shifted(v1, v2, (0, -safe_arc_spacing));
      a3 := straight_arc(v3, v1);
      a4 := straight_arc(v4, v1);

      d1 := straight_arc(v3, v4);

      draw_vertices(v);
      draw_arcs(a);

      drawarrow d1 dotted;

      label.top("$s$", straight_arc_midpoint of a1);
      label.bot("$t$", straight_arc_midpoint of a2);
      label.urt("$l'$", straight_arc_midpoint of a3);
      label.lrt("$l$", straight_arc_midpoint of a4);
      label.rt("$f$", straight_arc_midpoint of d1);
      endfig;
    \end{mplibcode}\fi
  \end{equation*}

  This limit is called the \term{equalizer} of \( s \) and \( t \).
\end{definition}

\begin{definition}\label{def:categorical_pullback}\mcite\cite[def. 5.1.16]{Leinster2016Basic}
  Assume that the index category \( \cat{I} \) has the shape
  \begin{equation*}
    \bullet \longrightarrow \bullet \longleftarrow \bullet
  \end{equation*}

  Cones of shape \( \cat{I} \) with vertex \( A \) are then given by commutative diagrams of shape
  \begin{equation*}
    \todo{Add diagram}\iffalse\begin{mplibcode}
      beginfig(1);
      input metapost/graphs;

      v1 := thelabel("$A$", origin);
      v2 := thelabel("$X$", (0, -1) scaled u);
      v3 := thelabel("$Y$", (1, 0) scaled u);
      v4 := thelabel("$Z$", (1, -1) scaled u);

      a1 := straight_arc(v1, v2);
      a2 := straight_arc(v1, v3);
      a3 := straight_arc(v2, v4);
      a4 := straight_arc(v3, v4);

      draw_vertices(v);
      draw_arcs(a);

      label.lft("$\pi_X$", straight_arc_midpoint of a1);
      label.top("$\pi_Y$", straight_arc_midpoint of a2);
      label.bot("$s$", straight_arc_midpoint of a3);
      label.rt("$t$", straight_arc_midpoint of a4);
      endfig;
    \end{mplibcode}\fi
  \end{equation*}

  The limit \( (L, \pi_X, \pi_Y) \) then satisfies the universal property: for any \( \cat{I} \)-cone \( (L', \pi_X', \pi_Y') \), there exists a unique morphism \( f: L' \to L \) such that the following diagram commutes:
  \begin{equation*}
    \todo{Add diagram}\iffalse\begin{mplibcode}
      beginfig(1);
      input metapost/graphs;

      v1 := thelabel("$L$", origin);
      v2 := thelabel("$X$", (0, -1) scaled u);
      v3 := thelabel("$Y$", (1, 0) scaled u);
      v4 := thelabel("$Z$", (1, -1) scaled u);
      v5 := thelabel("$L'$", (-1, 1) scaled u);

      a1 := straight_arc(v1, v2);
      a2 := straight_arc(v1, v3);
      a3 := straight_arc(v2, v4);
      a4 := straight_arc(v3, v4);
      a5 := straight_arc(v5, v2);
      a6 := straight_arc(v5, v3);

      d1 := straight_arc(v5, v1);

      draw_vertices(v);
      draw_arcs(a);

      drawarrow d1 dotted;

      label.rt("$\pi_X$", straight_arc_midpoint of a1);
      label.bot("$\pi_Y$", straight_arc_midpoint of a2);
      label.bot("$s$", straight_arc_midpoint of a3);
      label.rt("$t$", straight_arc_midpoint of a4);
      label.llft("$\pi_X'$", straight_arc_midpoint of a5);
      label.urt("$\pi_Y'$", straight_arc_midpoint of a6);

      fill fullcircle scaled 0.25u shifted (center d1) withcolor white;
      label("$f$", straight_arc_midpoint of d1);
      endfig;
    \end{mplibcode}\fi
  \end{equation*}

  This limit is called the \term{pullback} or \term{fibered product} of \( s \) and \( t \).
\end{definition}

\begin{definition}\label{def:categorical_cocone}\mcite\cite[def. 5.2.1]{Leinster2016Basic}
  The dual notion of a \hyperref[def:categorical_cone]{cone} is that of a cocone. Given a category \( \cat{A} \), an index category \( \cat{I} \) and a diagram \( D: \cat{I} \to \cat{A} \), we say that the family of morphisms
  \begin{equation*}
    \{ D(k) \overset {\iota_k} \to A \}_{k \in \cat{I}}
  \end{equation*}
  is a \term{cocone} for D if it is a cone for \( D^{-1}: \cat{I}^{-1} \to \cat{A}^{-1} \).

  Explicitly, a \term{cocone} on \( D \) consists of
  \begin{itemize}
    \item an object \( A \in \cat{A} \), called the \term{vertex} of the cocone
    \item a family of \term{coprojection} morphisms \( \{ \iota_k: D(k) \to A \}_{k \in \cat{I}} \)
  \end{itemize}
  such that for all morphisms \( u: k \to \beta \) in \( \cat{I} \), the following diagram commutes:
  \begin{alignedeq}\label{def:categorical_cocone/universal_property}
    \todo{Add diagram}\iffalse\begin{mplibcode}
      beginfig(1);
      input metapost/graphs;

      v1 := thelabel("$A$", origin);
      v2 := thelabel("$D(k)$", (-1, 1) scaled u);
      v3 := thelabel("$D(\beta)$", (1, 1) scaled u);

      a1 := straight_arc(v2, v1);
      a2 := straight_arc(v3, v1);
      a3 := straight_arc(v2, v3);

      draw_vertices(v);
      draw_arcs(a);

      label.llft("$\iota_k$", straight_arc_midpoint of a1);
      label.lrt("$\iota_\beta$", straight_arc_midpoint of a2);
      label.top("$D(u)$", straight_arc_midpoint of a3);
      endfig;
    \end{mplibcode}\fi
  \end{alignedeq}
\end{definition}

\begin{definition}\label{def:categorical_colimit}\mcite\cite[def. 5.1.19(b)]{Leinster2016Basic}
  Analogously to \hyperref[def:categorical_limit]{limits}, we define the \term{colimit} \( \varinjlim D \) of \( D \) to be a cocone
  \begin{equation*}
    \{ D(k) \overset {\iota_k} \to L \}_{k \in \cat{I}}
  \end{equation*}
  such that for every cocone
  \begin{equation*}
    \{ D(k) \overset {\iota_k'} \to L' \}_{k \in \cat{I}}
  \end{equation*}
  there exists exactly one morphism \( f: L \to L' \) such that \( \iota_k' = f \circ \iota_k, k \in \cat{I} \), i.e. the following diagram commutes:
  \begin{equation*}
    \todo{Add diagram}\iffalse\begin{mplibcode}
      beginfig(1);
      input metapost/graphs;

      v1 := thelabel("$D(k)$", origin);
      v2 := thelabel("$L'$", (-1, -1) scaled u);
      v3 := thelabel("$L$", (1, -1) scaled u);

      a1 := straight_arc(v1, v2);
      a2 := straight_arc(v1, v3);

      d1 := straight_arc(v3, v2);

      draw_vertices(v);
      draw_arcs(a);

      drawarrow d1 dotted;

      label.ulft("$\iota_k'$", straight_arc_midpoint of a1);
      label.urt("$\iota_k$", straight_arc_midpoint of a2);
      label.top("$f$", straight_arc_midpoint of d1);
      endfig;
    \end{mplibcode}\fi
  \end{equation*}

  If all small colimits exist, we say that \( \cat{A} \) is a \term{cocomplete category}.
\end{definition}

\begin{definition}\label{def:cocomplete_category}
  If a category is both \hyperref[def:categorical_limit]{complete} and \hyperref[def:categorical_colimit]{cocomplete}, it is said to be a \term{cocomplete category}.
\end{definition}

\begin{definition}\label{def:categorical_coproduct}\mcite\cite[def. 5.2.2]{Leinster2016Basic}
  If the index category \( \cat{I} \) is discrete, specifying a functor \( D: \cat{I} \to \cat{A} \) is analogous to specifying a \( \cat{I} \)-indexed family \( \{ X \}_{k \in \cat{I}} \) of objects in \( \cat{A} \) (see \fullref{def:categorical_product}).

  The \term{coproduct} or \term{categorical sum}
  \begin{equation*}
    S \coloneqq \coprod_{k \in \cat{I}} X_k = \sum_{k \in \cat{I}} X_k.
  \end{equation*}
  satisfies the following universal property: for any object \( S' \) and any family of morphisms \( \{ \iota_k': {X_k} \to S' \}_{k \in \cat{I}} \) there exists exactly one morphism \( f: S \to S' \) such that for every \( k \in \cat{I} \) we have \( \iota_k' = f \circ \iota_k \), i.e. the following diagram commutes:
  \begin{equation*}
    \todo{Add diagram}\iffalse\begin{mplibcode}
      beginfig(1);
      input metapost/graphs;

      v1 := thelabel("$X_j$", origin);
      v2 := thelabel("$S'$", (-1, 1) scaled u);
      v3 := thelabel("$S$", (1, 1) scaled u);

      a1 := straight_arc(v1, v2);
      a2 := straight_arc(v1, v3);

      d1 := straight_arc(v3, v2);

      draw_vertices(v);
      draw_arcs(a);

      drawarrow d1 dotted;

      label.llft("$\iota_k'$", straight_arc_midpoint of a1);
      label.lrt("$\iota_k$", straight_arc_midpoint of a2);
      label.top("$f$", straight_arc_midpoint of d1);
      endfig;
    \end{mplibcode}\fi
  \end{equation*}

  The function \( f \) is also denoted as \( \{ f_k \}_{k \in \cat{I}} \).

  In particular, for two objects \( X, Y \in \cat{A} \), the coproduct is an object \( X + Y \) with coprojections \( \pi_X: X \to X \times Y \) and \( \pi_Y: Y \to X \times Y \) such that for each object \( S' \) and morphisms \( \iota_X': X \to S' \) and \( \iota_Y': X \to P' \) the following diagram commutes:
  \begin{equation*}
    \todo{Add diagram}\iffalse\begin{mplibcode}
      beginfig(1);
      input metapost/graphs;

      v1 := thelabel("$S'$", (-1, 1) scaled u);
      v2 := thelabel("$X + Y$", origin);
      v3 := thelabel("$X$", (0, -1) scaled u);
      v4 := thelabel("$Y$", (1, 0) scaled u);

      a1 := straight_arc(v3, v1);
      a2 := straight_arc(v4, v1);
      a3 := straight_arc(v3, v2);
      a4 := straight_arc(v4, v2);

      d1 := straight_arc(v2, v1);

      draw_vertices(v);
      draw_arcs(a);

      drawarrow d1 dotted;

      label.llft("$\iota_X'$", straight_arc_midpoint of a1);
      label.urt("$\iota_Y'$", straight_arc_midpoint of a2);
      label.rt("$\iota_X$", straight_arc_midpoint of a3);
      label.bot("$\iota_Y$", straight_arc_midpoint of a4);

      fill fullcircle scaled 0.25u shifted (center d1) withcolor white;
      label("$f$", straight_arc_midpoint of d1);
      endfig;
    \end{mplibcode}\fi
  \end{equation*}
\end{definition}

\begin{remark}\label{rem:empty_categorical_coproduct}
  The coproduct \( \prod_{k \in \varnothing} X_k \) of an empty family of objects is the initial object of the category.
\end{remark}

\begin{definition}\label{def:categorical_coequalizer}\mcite\cite[def. 5.2.7]{Leinster2016Basic}
  As for \hyperref[def:categorical_coequalizer]{equalizers}, assume that the index category \( \cat{I} \cong \cat{I}^{-1} \) consists of two objects and two unidirectional morphisms:
  \begin{equation*}
    \todo{Add diagram}\iffalse\begin{mplibcode}
      beginfig(1);
      input metapost/graphs;

      v1 := thelabel("$\bullet$", (1, 0) scaled u);
      v2 := thelabel("$\bullet$", (2, 0) scaled u);

      a1 := straight_arc_shifted(v1, v2, (0, safe_arc_spacing));
      a2 := straight_arc_shifted(v1, v2, (0, -safe_arc_spacing));

      draw_vertices(v);
      draw_arcs(a);
      endfig;
    \end{mplibcode}\fi
  \end{equation*}

  Cocones with vertex \( A \) are then given by commutative diagrams of shape
  \begin{equation*}
    \todo{Add diagram}\iffalse\begin{mplibcode}
      beginfig(1);
      input metapost/graphs;

      v1 := thelabel("$A$", (3, 0) scaled u);
      v2 := thelabel("$X$", (1, 0) scaled u);
      v3 := thelabel("$Y$", (2, 0) scaled u);

      a1 := straight_arc(v3, v1);
      a2 := straight_arc_shifted(v2, v3, (0, safe_arc_spacing));
      a3 := straight_arc_shifted(v2, v3, (0, -safe_arc_spacing));

      draw_vertices(v);
      draw_arcs(a);

      label.top("$f$", straight_arc_midpoint of a1);
      label.top("$s$", straight_arc_midpoint of a2);
      label.bot("$t$", straight_arc_midpoint of a3);
      endfig;
    \end{mplibcode}\fi
  \end{equation*}

  The \term{coequalizer} \( (L, l) \) then satisfies the universal property: for any \( \cat{I} \)-cocone \( (L', l') \), there exists a unique morphism \( f: L \to L' \) such that the following diagram commutes:
  \begin{equation*}
    \todo{Add diagram}\iffalse\begin{mplibcode}
      beginfig(1);
      input metapost/graphs;

      v1 := thelabel("$X$", origin);
      v2 := thelabel("$Y$", (1, 0) scaled u);
      v3 := thelabel("$L'$", (2, 1) scaled u);
      v4 := thelabel("$L$", (2, -1) scaled u);

      a1 := straight_arc_shifted(v1, v2, (0, safe_arc_spacing));
      a2 := straight_arc_shifted(v1, v2, (0, -safe_arc_spacing));
      a3 := straight_arc(v2, v3);
      a4 := straight_arc(v2, v4);

      d1 := straight_arc(v4, v3);

      draw_vertices(v);
      draw_arcs(a);

      drawarrow d1 dotted;

      label.top("$s$", straight_arc_midpoint of a1);
      label.bot("$t$", straight_arc_midpoint of a2);
      label.ulft("$l'$", straight_arc_midpoint of a3);
      label.llft("$l$", straight_arc_midpoint of a4);
      label.rt("$f$", straight_arc_midpoint of d1);
      endfig;
    \end{mplibcode}\fi
  \end{equation*}
\end{definition}

\begin{definition}\label{def:categorical_pushout}\mcite\cite[def. 5.2.11]{Leinster2016Basic}
  A \term{pushout} in \( \cat{A} \) is a \term{pullback} in \( \cat{A}^{-1} \).

  Explicitly, the index category \( \cat{I} \) has the shape
  \begin{equation*}
    \bullet \longleftarrow \bullet \longrightarrow \bullet
  \end{equation*}

  Cocones of shape \( \cat{I} \) with vertex \( A \) are then given by commutative diagrams of shape
  \begin{equation*}
    \todo{Add diagram}\iffalse\begin{mplibcode}
      beginfig(1);
      input metapost/graphs;

      v1 := thelabel("$A$", origin);
      v2 := thelabel("$X$", (-1, 0) scaled u);
      v3 := thelabel("$Y$", (0, 1) scaled u);
      v4 := thelabel("$Z$", (-1, 1) scaled u);

      a1 := straight_arc(v2, v1);
      a2 := straight_arc(v3, v1);
      a3 := straight_arc(v4, v2);
      a4 := straight_arc(v4, v3);

      draw_vertices(v);
      draw_arcs(a);

      label.bot("$\iota_X$", straight_arc_midpoint of a1);
      label.rt("$\iota_Y$", straight_arc_midpoint of a2);
      label.lft("$s$", straight_arc_midpoint of a3);
      label.top("$t$", straight_arc_midpoint of a4);
      endfig;
    \end{mplibcode}\fi
  \end{equation*}

  The pushout \( (L, \iota_X, \iota_Y) \) of \( D \) then satisfies the universal property: for any \( \cat{I} \)-cocone \( (L', \iota_X', \iota_Y') \), there exists a unique morphism \( f: L \to L' \) such that the following diagram commutes:
  \begin{equation*}
    \todo{Add diagram}\iffalse\begin{mplibcode}
      beginfig(1);
      input metapost/graphs;

      v1 := thelabel("$L$", origin);
      v2 := thelabel("$X$", (-1, 0) scaled u);
      v3 := thelabel("$Y$", (0, 1) scaled u);
      v4 := thelabel("$Z$", (-1, 1) scaled u);
      v5 := thelabel("$L'$", (1, -1) scaled u);

      a1 := straight_arc(v2, v1);
      a2 := straight_arc(v3, v1);
      a3 := straight_arc(v4, v2);
      a4 := straight_arc(v4, v3);
      a5 := straight_arc(v2, v5);
      a6 := straight_arc(v3, v5);

      draw_vertices(v);
      draw_arcs(a);

      d1 := straight_arc(v1, v5);

      draw_vertices(v);
      draw_arcs(a);

      drawarrow d1 dotted;

      label.top("$\iota_X$", straight_arc_midpoint of a1);
      label.lft("$\iota_Y$", straight_arc_midpoint of a2);
      label.lft("$s$", straight_arc_midpoint of a3);
      label.top("$t$", straight_arc_midpoint of a4);
      label.llft("$\iota_X'$", straight_arc_midpoint of a5);
      label.urt("$\iota_Y'$", straight_arc_midpoint of a6);

      fill fullcircle scaled 0.25u shifted (center d1) withcolor white;
      label("$f$", straight_arc_midpoint of d1);
      endfig;
    \end{mplibcode}\fi
  \end{equation*}
\end{definition}

\begin{definition}\label{def:categorical_limit_preservation}\mcite\cite[def. 5.3.1, 5.3.5]{Leinster2016Basic}
  Let \( F: \cat{A} \to \boldop B \) be a functor. We say that
  \begin{thmenum}
    \labitem{def:categorical_limit_preservation/preserve} \( F \) \term{preserves} limits of shape \( \cat{I} \) for some index category \( \cat{I} \) if, given a \( \cat{I} \)-shaped limit cone
    \begin{equation*}
      \{ L \overset {\pi_k} \to D(k) \}_{k \in \cat{I}},
    \end{equation*}
    its image
    \begin{equation*}
      \{ F(L) \overset {F(\pi_k)} \to F(D(k)) \}_{k \in \cat{I}}
    \end{equation*}
    is also a limit cone. We say that \( F \) simply preserves limits if it preserves limits for every index category \( \cat{I} \).

    \labitem{def:categorical_limit_preservation/reflect} \( F \) \term{reflects} limits of shape \( \cat{I} \) if, given any \( \cat{I} \)-shaped cone, if its image is a limit cone, then is it itself a limit cone.

    \labitem{def:categorical_limit_preservation/create} \( F \) \term{creates} limits of shape \( \cat{I} \) if it both preserves and reflects limits.

    \labitem{def:categorical_limit_preservation/lift} \( F \) \term{lifts} limits of shape \( \cat{I} \) if, given a diagram \( D: \cat{I} \to \boldop B \), any limit cone \( \varprojlim D \) is the image of some limit cone in \( A \).
  \end{thmenum}
\end{definition}

\begin{remark}\label{rem:categorical_colimit_preservation}
  Analogous definitions can be given for colimits.
\end{remark}

\subsection{Monoidal categories}\label{subsec:monoidal_categories}

\begin{definition}\label{def:monoidal_category}\mcite\cite[158]{MacLane1994}
  A \term{monoidal category} is a generalization of a \hyperref[def:magma]{monoid} from sets to categories. Formally, it is a category \( \cat M \) along with
  \begin{itemize}
    \item a \term{monoidal product} functor \( \otimes: C \times C \to C \)
    \item an identity object \( 1 \in \bop M \)
    \item natural transformations
          \begin{balign*}
            \sigma  & : ((-) \otimes (-)) \otimes (-) \cong (-) \otimes ((-) \otimes (-)) \\
            \lambda & : 1 \times (-) \cong (-)                                            \\
            \rho    & : (-) \times 1 \cong (-)
          \end{balign*}
  \end{itemize}
  such that
  \begin{defenum}
    \item for every object \( A \in \bop M \),
    \begin{balign*}
      1 \otimes A \overset {\lambda_a} \cong A
      \\
      A \otimes 1 \overset {\rho_a} \cong A
    \end{balign*}

    \item for all objects \( A, B, C \in \bop M \),
    \begin{equation*}
      A \otimes (B \otimes C) \overset {\sigma_{A,B,C}} \cong (A \otimes B) \otimes C
    \end{equation*}

    \item the following diagram commutes for all objects \( A, B, C, D \in \bop M \)
    \begin{equation*}
      \begin{mplibcode}
        u := 2cm;

        beginfig(1);
        input metapost/graphs;

        v1 := thelabel("$(A \otimes B) \otimes (C \otimes D)$", origin);
        v2 := thelabel("$A \otimes (B \otimes (C \otimes D))$", (-2, -1) scaled u);
        v3 := thelabel("$A \otimes ((B \otimes C) \otimes D)$", (-2, -2) scaled u);
        v4 := thelabel("$((A \otimes B) \otimes C) \otimes D$", (2, -1) scaled u);
        v5 := thelabel("$(A \otimes (B \otimes C)) \otimes D$", (2, -2) scaled u);

        a1 := straight_arc(v2, v1);
        a2 := straight_arc(v1, v4);
        a3 := straight_arc(v2, v3);
        a4 := straight_arc(v5, v4);
        a5 := straight_arc(v3, v5);

        draw_vertices(v);
        draw_arcs(a);

        label.ulft("$\sigma_{A, B, (C \otimes D)}$", straight_arc_midpoint of a1);
        label.urt("$\sigma_{(A \otimes B), C, D}$", straight_arc_midpoint of a2);
        label.lft("$\id \otimes \sigma_{B, C, D}$", straight_arc_midpoint of a3);
        label.rt("$\sigma_{A, B, C} \otimes \id$", straight_arc_midpoint of a4);
        label.bot("$\sigma_{A, (B \otimes C), D}$", straight_arc_midpoint of a5);
        endfig;
      \end{mplibcode}
    \end{equation*}

    \item the following diagram commutes for all objects \( A, B \in \bop M \)
    \begin{equation*}
      \begin{mplibcode}
        beginfig(1);
        input metapost/graphs;

        v1 := thelabel("$A \otimes B$", origin);
        v2 := thelabel("$A \otimes (1 \otimes B)$", (-2, 1) scaled u);
        v3 := thelabel("$(A \otimes 1) \otimes B$", (2, 1) scaled u);

        a1 := straight_arc(v2, v1);
        a2 := straight_arc(v3, v1);
        a3 := straight_arc(v2, v3);

        draw_vertices(v);
        draw_arcs(a);

        label.llft("$\id \otimes \lambda_b$", straight_arc_midpoint of a1);
        label.lrt("$\rho_a \otimes \id$", straight_arc_midpoint of a2);
        label.top("$\sigma_{A, 1, B}$", straight_arc_midpoint of a3);
        endfig;
      \end{mplibcode}
    \end{equation*}
  \end{defenum}

  If the natural isomorphisms \( \sigma \), \( \lambda \) and \( \rho \) are identities, we say that \( \bop M \) is a \term{strict monoidal category}.
\end{definition}

\begin{definition}\label{def:categorical_monoid}\mcite\cite[167]{MacLane1994}
  Let \( (\cat{M}, \otimes, 1) \) be a monoidal \hyperref[def:monoidal_category]{category}. A monoid in \( \cat{M} \) consists of
  \begin{itemize}
    \item the monoid itself, an object \( M \in \cat{M} \)
    \item the monoid operation, a morphism \( \mu: M \otimes M \to M \)
    \item the identity \hyperref[def:generalized_element]{element}, a morphism \( \eta: 1 \to M \)
  \end{itemize}
  such that the following diagrams commute:
  \begin{equation*}
    \begin{mplibcode}
      beginfig(1);
      input metapost/graphs;

      v1 := thelabel("$M \otimes (M \otimes M)$", (-2, 0) scaled u);
      v2 := thelabel("$(M \otimes M) \otimes M$", (0, 1) scaled u);
      v3 := thelabel("$M \otimes M$", (2, 0) scaled u);
      v4 := thelabel("$M \otimes M$", (-2, -1) scaled u);
      v5 := thelabel("$M$", (2, -1) scaled u);

      a1 := straight_arc(v1, v2);
      a2 := straight_arc(v1, v4);
      a3 := straight_arc(v2, v3);
      a4 := straight_arc(v3, v5);
      a5 := straight_arc(v4, v5);

      draw_vertices(v);
      draw_arcs(a);

      label.ulft("$\sigma_{M,M,M}$", straight_arc_midpoint of a1);
      label.lft("$\id \otimes \mu$", straight_arc_midpoint of a2);
      label.urt("$\mu \otimes \id$", straight_arc_midpoint of a3);
      label.rt("$\mu$", straight_arc_midpoint of a4);
      label.top("$\mu$", straight_arc_midpoint of a5);
      endfig;
    \end{mplibcode}
  \end{equation*}
  and
  \begin{equation*}
    \begin{mplibcode}
      beginfig(1);
      input metapost/graphs;

      v1 := thelabel("$M \otimes M$", (0, 1) scaled u);
      v2 := thelabel("$M$", (0, -1) scaled u);
      v3 := thelabel("$\id \otimes M$", (-2, 0) scaled u);
      v4 := thelabel("$M \otimes \id$", (2, 0) scaled u);

      a1 := straight_arc(v1, v2);
      a2 := straight_arc(v3, v1);
      a3 := straight_arc(v3, v2);
      a4 := straight_arc(v4, v1);
      a5 := straight_arc(v4, v2);

      draw_vertices(v);
      draw_arcs(a);

      label.rt("$\mu$", straight_arc_midpoint of a1);
      label.ulft("$\eta \otimes \id$", straight_arc_midpoint of a2);
      label.llft("$\lambda_M$", straight_arc_midpoint of a3);
      label.urt("$\id \otimes \eta$", straight_arc_midpoint of a4);
      label.lrt("$\rho_M$", straight_arc_midpoint of a5);
      endfig;
    \end{mplibcode}
  \end{equation*}

  A morphism \( f: (M, \mu, \eta) \to (M', \mu', \eta') \) between two monoids is an arrow \( f: M \to M' \) such that
  \begin{equation*}
    f \circ \mu = \mu' \circ (f \otimes f): M \times M \to M'
  \end{equation*}
  and
  \begin{equation*}
    f \circ \eta = \eta': 1 \to M'.
  \end{equation*}

  The category of all monoids over \( \cat{M} \) is denoted by \( \cat{Mon}(\cat{M}) \).
\end{definition}

\begin{definition}\label{def:enriched_category}\mcite\cite[180]{MacLane1994},\cite{nLab:enriched_category}
  Enriched categories provide additional structure to the morphism sets of locally small categories. The definition can be compared with \fullref{def:category}. We say that \( \cat{C} \) is an \term{enriched category} over the small monoidal category \( \bop M \) if
  \begin{itemize}
    \item there exists a class of objects, where the membership is denoted by \( A \in \cat{C} \).
    \item for each object \( A \in \cat{C} \), there exists an \term{identity morphism} \( j_A: 1 \to \cat{C}(A, A) \).
    \item for each pair of objects \( A, B \in \cat{C} \), there exists an object \( \cat{C}(A, B) \) in \( \bop M \).
    \item for each triple of objects \( A, B, C \in \cat{C} \), there exists a \term{composition morphism} in \( \bop M \):
          \begin{equation*}
            \circ_{A,B,C}: {\cat{C}}(B, C) \times {\cat{C}}(A, B) \to {\cat{C}}(A, C)
          \end{equation*}
  \end{itemize}
  such that
  \begin{defenum}
    \item the following diagram commutes for all objects \( A, B, C, D \in \cat{C} \)
    \begin{equation*}
      \begin{mplibcode}
        u := 2cm;

        beginfig(1);
        input metapost/graphs;

        v1 := thelabel("$C(A, D)$", origin);
        v2 := thelabel("$C(C, D) \otimes C(A, C)$", (-2, -1) scaled u);
        v3 := thelabel("$C(C, D) \otimes (C(B, C) \otimes C(A, B))$", (-2, -2) scaled u);
        v4 := thelabel("$C(B, D) \otimes C(A, B)$", (2, -1) scaled u);
        v5 := thelabel("$(C(C, D) \otimes C(B, C)) \otimes C(A, B)$", (2, -2) scaled u);

        a1 := straight_arc(v2, v1);
        a2 := straight_arc(v4, v1);
        a3 := straight_arc(v2, v3);
        a4 := straight_arc(v5, v4);
        a5 := straight_arc(v3, v5);

        draw_vertices(v);
        draw_arcs(a);

        label.ulft("$\circ_{A, C, D}$", straight_arc_midpoint of a1);
        label.urt("$\circ_{A, B, D}$", straight_arc_midpoint of a2);
        label.lft("$\id \otimes \circ_{A, B, C}$", straight_arc_midpoint of a3);
        label.rt("$\circ_{B, C, D} \otimes \id$", straight_arc_midpoint of a4);
        label.bot("$\sigma$", straight_arc_midpoint of a5);
        endfig;
      \end{mplibcode}
    \end{equation*}

    \item the following diagram commutes for all objects \( A, B \in \bop M \)
    \begin{equation*}
      \begin{mplibcode}
        u := 2cm;

        beginfig(1);
        input metapost/graphs;

        v1 := thelabel("$C(A, B)$", origin);
        v2 := thelabel("$C(B, B) \otimes C(A, B)$", (-2, 1) scaled u);
        v3 := thelabel("$1 \otimes C(A, B)$", (-2, -1) scaled u);
        v4 := thelabel("$C(A, B) \otimes C(A, A)$", (2, 1) scaled u);
        v5 := thelabel("$C(A, B) \otimes 1$", (2, -1) scaled u);

        a1 := straight_arc(v2, v1);
        a2 := straight_arc(v4, v1);
        a3 := straight_arc(v3, v1);
        a4 := straight_arc(v5, v1);
        a5 := straight_arc(v3, v2);
        a6 := straight_arc(v5, v4);

        draw_vertices(v);
        draw_arcs(a);

        label.urt("$\circ_{A, B, B}$", straight_arc_midpoint of a1);
        label.ulft("$\circ_{A, A, B}$", straight_arc_midpoint of a2);
        label.lrt("$\lambda$", straight_arc_midpoint of a3);
        label.llft("$\rho$", straight_arc_midpoint of a4);
        label.lft("$j_B \otimes \id_{C(A, B)}$", straight_arc_midpoint of a5);
        label.rt("$\id_{C(A, B)} \otimes j_A$", straight_arc_midpoint of a6);
        endfig;
      \end{mplibcode}
    \end{equation*}
  \end{defenum}

  In order for monoidal categories to actually be categories (more specifically, locally small categories), formally we need a functor \( U: \cat{M} \to \cat{Set} \) so that morphism objects \( \cat{C}(A, B) \) become sets \( U(\cat{C}(A, B)) \). This is usually defined implicitly, for example \( U(\cat{C}(A, B)) \coloneqq \bop{M}(1, C(A, B)) \).
\end{definition}

\subsection{Abelian categories}\label{subsec:abelian_categories}

\begin{definition}\label{def:preadditive_category}\mcite\cite[28]{MacLane1994}
  A \term{preadditive category \( \cat{C} \)} is any category enriched over the category \( \cat{Ab} \) of \hyperref[thm:ab_is_abelian]{abelian groups}, such that composition
  \begin{equation*}
    \circ_{A,B,C}: \cat{Ab}(B, C) \times \cat{Ab}(A, B) \to \cat{Ab}(A, C)
  \end{equation*}
  is bilinear, e.g. given group homomorphisms \( f, f': A \to B \) and \( g, g': B \to C \), we have
  \begin{equation*}
    (g + g') \circ (f + f') = g \circ f + g \circ f' + g' \circ f + g' \circ f'.
  \end{equation*}
\end{definition}

\begin{definition}\label{def:zero_morphism}
  Let \( \cat{C} \) be a category. We say that the morphism \( f: A \to B \) is
  \begin{thmenum}
    \labitem{def:zero_morphism/left} a \term{left-zero morphism} or a \term{constant morphism} if \( f \circ g = f \circ h \) for any two morphisms \( g, h: A' \to A \) for any object \( A' \).
    \labitem{def:zero_morphism/right} a \term{right-zero morphism} or a \term{coconstant morphism} if \( g \circ f = h \circ f \) for any two morphisms \( g, h: B \to B' \) for any object \( B' \).
    \labitem{def:zero_morphism/bidirectional} a \term{zero morphism} if it is both a left-zero and a right-zero morphism. We denote it by \( 0_{A,B} \) if it is unique (for example, in \hyperref[def:preadditive_category]{preadditive categories}).
  \end{thmenum}
\end{definition}

\begin{proposition}\label{thm:preadditive_zero_morphisms}
  If \( \cat{C} \) is a \hyperref[def:preadditive_category]{preadditive category} and \( A, B \in \cat{C} \), the identity of \( \cat{C}(A, B) \) is the unique zero \hyperref[def:zero_morphism]{morphism} from \( A \) to \( B \).
\end{proposition}
\begin{proof}
  Denote the identity of \( \cat{C}(A, B) \) by \( 0_{A,B} \). We will show that it is a zero morphism in the sense of \fullref{def:zero_morphism}.

  Let \( C \in \cat{C} \) and fix a morphism \( f: B \to C \). Then, by linearity,
  \begin{balign*}
    f \circ 0_{A,B} + f \circ 0_{A,B}
    =
    f \circ (0_{A,B} + 0_{A,B})
    =
    f \circ 0_{A,B}.
  \end{balign*}

  Thus \( f \circ 0_{A,B} = 0_{A,C} \). Since this holds for any function, we conclude that \( g \circ 0_{A,B} = h \circ 0_{A,B} = 0_{A,C} \) for any two morphisms in \( g, h \in \cat{C}(B,C) \) and hence \( 0_{A,B} \) is a left zero morphism. The proof that \( 0_{A,B} \) is a right zero morphism is identical. Hence \( 0_{A,B} \) is a zero morphism.

  Now we will show that these are the only zero morphisms in \( \cat{C} \). Assume that \( z: A \to B \) is a zero morphism. Then
  \begin{equation*}
    z = 0_{B,B} \circ z = (0_{B,B} + 0_{B,B}) \circ z = z + z,
  \end{equation*}
  hence \( z = 0_{A,B} \).
\end{proof}

\begin{proposition}\label{thm:preadditive_category_biproducts}
  If \( \cat{C} \) is a preadditive category, the vertices of nonempty finite products and coproducts coincide.
\end{proposition}
\begin{proof}
  Let \( X: \cat{I} \to \cat{C} \) be a finite discrete diagram. Denote the objects \( X(k) \) by \( X_k \) and their product by
  \begin{equation*}
    (X, \pi) \coloneqq \varprojlim D
  \end{equation*}
  where \( X \) is an object in \( C \) and
  \begin{equation*}
    \pi = \{ \pi_k: X \to X_k \}_{k \in \cat{I}}
  \end{equation*}
  is the family of projections.

  Consider the object \( X_k \in \cat{C} \) with the family of morphisms
  \begin{balign*}
    \begin{dcases}
      \begin{drcases}
        \id_{X_k},       & \beta = k    \\
        0_{X_k,X_\beta}, & \beta \neq k
      \end{drcases}
    \end{dcases}_{\beta \in \cat{I}}
  \end{balign*}

  By the definition of \hyperref[def:categorical_product]{product}, there exists a unique family of morphisms \( \{ \iota_k \}_{k \in \cat{I}} \) such that the following diagram commutes
  \begin{equation*}
    \todo{Add diagram}\iffalse\begin{mplibcode}
      beginfig(1);
      input metapost/graphs;

      v1 := thelabel("$X_k$", origin);
      v2 := thelabel("$X_k$", (-1, 1) scaled u);
      v3 := thelabel("$P$", (1, 1) scaled u);

      a1 := straight_arc(v2, v1);
      a2 := straight_arc(v3, v1);

      d1 := straight_arc(v2, v3);

      draw_vertices(v);
      draw_arcs(a);

      drawarrow d1 dotted;

      label.llft("$\id_{X_k}$", straight_arc_midpoint of a1);
      label.lrt("$\pi_k$", straight_arc_midpoint of a2);
      label.top("$\iota_k$", straight_arc_midpoint of d1);
      endfig;
    \end{mplibcode}\fi
    \hspace{1cm}
    \todo{Add diagram}\iffalse\begin{mplibcode}
      beginfig(1);
      input metapost/graphs;

      v1 := thelabel("$X_k$", origin);
      v2 := thelabel("$X_\beta$", (-1, 1) scaled u);
      v3 := thelabel("$P$", (1, 1) scaled u);

      a1 := straight_arc(v2, v1);
      a2 := straight_arc(v3, v1);

      d1 := straight_arc(v2, v3);

      draw_vertices(v);
      draw_arcs(a);

      drawarrow d1 dotted;

      label.llft("$0_{X_k, X_\beta}$", straight_arc_midpoint of a1);
      label.lrt("$\pi_k$", straight_arc_midpoint of a2);
      label.top("$\iota_\beta$", straight_arc_midpoint of d1);
      endfig;
    \end{mplibcode}\fi
  \end{equation*}

  Define \( \iota \coloneqq \{ \iota_k \}_{k \in \cat{I}} \). We will prove that \( (X, \iota) \) is a \hyperref[def:categorical_coproduct]{coproduct}.

  Let \( \Gamma \in \cat{C} \) be an arbitrary object such that there exists a family of morphisms
  \begin{equation*}
    \{ \gamma_k: X_k \to \Gamma \}_{k \in \cat{I}}.
  \end{equation*}

  Define
  \begin{equation*}
    f \coloneqq \sum_{k \in I} (\gamma_k \circ \pi_k): X \to \Gamma.
  \end{equation*}

  Fix \( k \in \cat{I} \). Now we show that the following diagrams commute:
  \begin{equation*}
    \todo{Add diagram}\iffalse\begin{mplibcode}
      beginfig(1);
      input metapost/graphs;

      v1 := thelabel("$X_k$", origin);
      v2 := thelabel("$\Gamma$", (-1, 1) scaled u);
      v3 := thelabel("$X$", (1, 1) scaled u);

      a1 := straight_arc(v1, v2);
      a2 := straight_arc(v1, v3);

      d1 := straight_arc(v3, v2);

      draw_vertices(v);
      draw_arcs(a);

      drawarrow d1 dotted;

      label.llft("$\gamma_k$", straight_arc_midpoint of a1);
      label.lrt("$\iota_k$", straight_arc_midpoint of a2);
      label.top("$f$", straight_arc_midpoint of d1);
      endfig;
    \end{mplibcode}\fi
  \end{equation*}

  Indeed,
  \begin{balign*}
    f \circ \iota_k
    =
    \left(\sum_{\beta \in \cat{I}} \gamma_\beta \circ \pi_\beta \right) \circ \iota_k
    =
    \sum_{k \in \cat{I}} (\gamma_\beta \circ (\pi_\beta \circ \iota_k))
    =
    \gamma_k \circ \id_{X_k} + \sum_{\substack{\beta \in \cat{I} \\ {\beta \neq k}}} \gamma_\beta \circ 0_{X_k,X_\beta}
    =
    \gamma_k.
  \end{balign*}

  Note that the sum is well-defined since the indexing category \( \cat{I} \) is finite.

  Now we will show that the morphism \( f \) is unique.

  First define
  \begin{equation*}
    g \coloneqq \sum_{\beta \in \cat{I}} \iota_\beta \circ \pi_\beta: X \to X.
  \end{equation*}

  Note that for each \( k \in \cat{I} \),
  \begin{balign*}
    \pi_k \circ g
    =
    \pi_k \circ \left( \sum_{\beta \in \cat{I}} \iota_\beta \circ \pi_\beta \right)
    =
    \sum_{\beta \in \cat{I}} ((\pi_k \circ \iota_\beta) \circ \pi_\beta)
    =
    \id_k \circ \pi_k + \sum_{\substack{\beta \in \cat{I} \\ {\beta \neq k}}} 0_{X_k,X_\beta}
    =
    \pi_k.
  \end{balign*}

  We claim that \( g = \id_X \). Since \( X \) is a product, there exists a unique morphism such that the following diagram commutes for each \( k \in \cat{I} \):
  \begin{alignedeq}\label{thm:preadditive_biproducts/product_identity}
    \todo{Add diagram}\iffalse\begin{mplibcode}
      beginfig(1);
      input metapost/graphs;

      v1 := thelabel("$X_k$", origin);
      v2 := thelabel("$X$", (-1, 1) scaled u);
      v3 := thelabel("$X$", (1, 1) scaled u);

      a1 := straight_arc(v2, v1);
      a2 := straight_arc(v3, v1);
      d1 := straight_arc(v2, v3);

      draw_vertices(v);
      draw_arcs(a);

      drawarrow d1 dotted;

      label.llft("$g \circ \pi_k$", straight_arc_midpoint of a1);
      label.lrt("$\pi_k$", straight_arc_midpoint of a2);
      endfig;
    \end{mplibcode}\fi
  \end{alignedeq}

  Both \( g \) and \( \id_X \) satisfy the universal property in \fullref{thm:preadditive_biproducts/product_identity}, hence they are equal.

  To show that the morphism \( f \) is unique, assume that there exists \( f': \Gamma \to X \) such that for each \( k \in \cat{I} \),
  \begin{equation*}
    f' \circ \iota_k = \gamma_k.
  \end{equation*}

  But
  \begin{balign*}
    f - f'
     & =
    (f - f') \circ \id_X
    =    \\ &=
    (f - f') \circ \left( \sum_{k \in \cat{I}} \iota_k \circ \pi_k \right)
    =    \\ &=
    \sum_{k \in \cat{I}} ((f \circ \iota_k) \circ \pi_k - (f' \circ \iota_k) \circ \pi_k)
    =    \\ &=
    \sum_{k \in \cat{I}} (\gamma_k \circ \pi_k - \gamma_k \circ \pi_k)
    =
    0_{\Gamma,X},
  \end{balign*}
  thus \( f = f' \).

  Hence the definition of coproduct is satisfied by \( (X, \iota) \).
\end{proof}

\begin{definition}\label{def:categorical_biproduct}
  Let \( \cat{C} \) be a preadditive category. A \term{biproduct} of the finite family \( \{ X_k \}_{k \in I} \) of objects in \( \cat{C} \) is a triple \( (X, \pi, \iota) \), such that \( (X, \pi) \) is a product, \( (X, \iota) \) is a coproduct.
\end{definition}

\begin{remark}\label{rem:preadditive_category_biproducts}
  By \fullref{thm:preadditive_category_biproducts}, if a nonempty finite product exists in a preadditive category, so does the corresponding coproduct, hence it is a biproduct. If the empty product exists, however, it may not be a coproduct.

  In order to ensure some regularity, \hyperref[def:additive_category]{additive categories} are introduced.
\end{remark}

\begin{definition}\label{def:additive_category}\mcite\cite[196]{MacLane1994}
  A \hyperref[def:preadditive_category]{preadditive category} is called additive if it has all finite \hyperref[def:categorical_biproduct]{biproducts}, including empty biproducts (see \fullref{thm:additive_category_biproducts}).
\end{definition}

\begin{theorem}\label{thm:additive_category_biproducts}
  If \( \cat{C} \) is an additive category, the vertices of finite products and coproducts coincide, that is, they are biproducts.
\end{theorem}
\begin{proof}
  The proof follows from \fullref{thm:preadditive_category_biproducts} and the fact that the \hyperref[rem:empty_categorical_coproduct]{initial} and \hyperref[rem:empty_categorical_product]{terminal} object coincide.
\end{proof}

\begin{definition}\label{def:categorical_kernel}
  Let \( \cat{C} \) be a preadditive category and \( f: A \to B \) be a morphism in \( \cat{C} \). We define the \term{kernel} \( \ker(f) \) of \( f \) as the \hyperref[def:categorical_equalizer]{equalizer} of \( f \) and \( 0_{A,B} \). Thus \( \ker(f) \) is a morphism from \( L \) (the limit vertex) to \( A \).

  Analogously, we define the \term{cokernel} \( \co\ker(f) \) of \( f \) as the \hyperref[def:categorical_coequalizer]{coequalizer} of \( f \) and \( 0_{A,B} \). Thus \( \co\ker(f): B \to C \), where \( C \) is the colimit vertex.
\end{definition}

\begin{definition}\label{def:abelian_category}\mcite\cite[196]{MacLane1994}
  An \hyperref[def:additive_category]{additive category} \( \cat{C} \) is called an \term{abelian category} if:
  \begin{thmenum}
    \item \( \cat{C} \) has a kernel and a cokernel for every \hyperref[def:categorical_kernel]{morphism}
    \item every monomorphism is a kernel and every epimorphism is a \hyperref[def:morphism_invertibility]{cokernel}
  \end{thmenum}
\end{definition}

\begin{proposition}\label{thm:abelian_category_morphism_factorization}\mcite\cite[prop. 8.3.1]{MacLane1994}
  In an abelian category \( \cat{C} \), every morphism \( f: A \to B \) has a factorization \( f = \img f \circ \co\img f \), where
  \begin{itemize}
    \item \( \img f \coloneqq \ker(\co\ker f: B \to C_1): L_1 \to B \) is a monomorphism
    \item \( \co\img f \coloneqq \co\ker(\ker f: L_2 \to A): A \to C_2 \) is an epimorphism
  \end{itemize}
  Here \( L_1 \) and \( L_2 \) are the limit vertices and \( C_1 \) and \( C_2 \) are the colimit vertices as in \fullref{def:categorical_kernel}. Necessarily \( L_1 \cong C_2 \).
\end{proposition}

\begin{definition}\label{def:exact_morphism_pair}\mcite\cite[196]{MacLane1994}
  In an abelian category \( \cat{C} \), a composable pair of morphisms \( f: A \to B \) and \( g: B \to C \) is said to be \term{exact} at \( B \) if \( \img f \cong \ker g \) as subobjects of \( B \) (or, equivalently, \( \co\ker f \cong \co\img g \); see \fullref{def:categorical_subobject}).
\end{definition}

\subsection{Exact sequences}\label{subsec:exact_sequences}

\begin{definition}\label{def:short_exact_sequence}\mcite[196]{MacLane1994}
  In an \hyperref[def:abelian_category]{abelian category} \( \cat{C} \), the tower \hyperref[def:tower_diagram]{diagram}
  \begin{equation}\label{def:short_exact_sequence/diagram}
    0
    \overset \iota \longrightarrow
    A
    \overset f \longrightarrow
    B
    \overset g \longrightarrow
    C
    \overset \pi \longrightarrow
    0
  \end{equation}
  is called a \term{short exact sequence} (SES) if it is exact at \( A \), \( B \) and \( C \) (in the sense of \fullref{def:exact_morphism_pair}).

  Equivalently, \fullref{def:short_exact_sequence/diagram} is short exact if and only if \( f \cong \ker g \) as subobjects of \( B \) and \( g \cong \co\ker f \) as subobjects of \( C \).
\end{definition}

\begin{remark}\label{rem:short_exact_sequence_factorization}
  Since \( 0 \) is an initial object, the morphism \( \iota: 0 \to A \) exists and is unique. Analogously, \( \pi: C \to 0 \) exists and is unique (up to an isomorphism). This is why \( \iota \) and \( \pi \) can be skipped entirely when defining short exact sequences.

  The morphism \( f \) is necessarily a monomorphism (\enquote{f} stands for \enquote{injection}) since it is equivalent to a kernel and \( g \) is necessarily an epimorphism (\enquote{g} stands for \enquote{projection}). When either \( f \) or \( g \) is obvious, they may also be skipped.

  This makes SES a good framework for describing factorization of algebraic structures, as can be seen in \fullref{ex:short_exact_sequences}.
\end{remark}

\begin{definition}\label{def:exact_sequence_morphisms}\mcite[198]{MacLane1994}
  Consider the two short exact sequences over the same category \( \cat{C} \):
  \begin{equation*}
    \begin{mplibcode}
      beginfig(1);
      input metapost/graphs;

      v1 := thelabel("$0$", origin);
      v2 := thelabel("$A$", (1, 0) scaled u);
      v3 := thelabel("$B$", (2, 0) scaled u);
      v4 := thelabel("$C$", (3, 0) scaled u);
      v5 := thelabel("$0$", (4, 0) scaled u);

      v6 := thelabel("$0$", (0, -1) scaled u);
      v7 := thelabel("$A'$", (1, -1) scaled u);
      v8 := thelabel("$B'$", (2, -1) scaled u);
      v9 := thelabel("$C'$", (3, -1) scaled u);
      v10 := thelabel("$0$", (4, -1) scaled u);

      a1 := straight_arc(v1, v2);
      a2 := straight_arc(v2, v3);
      a3 := straight_arc(v3, v4);
      a4 := straight_arc(v4, v5);

      a5 := straight_arc(v6, v7);
      a6 := straight_arc(v7, v8);
      a7 := straight_arc(v8, v9);
      a8 := straight_arc(v9, v10);

      draw_vertices(v);
      draw_arcs(a);

      label.top("$f$", straight_arc_midpoint of a2);
      label.top("$g$", straight_arc_midpoint of a3);

      label.top("$f'$", straight_arc_midpoint of a6);
      label.top("$g'$", straight_arc_midpoint of a7);
      endfig;
    \end{mplibcode}
  \end{equation*}
  We say that the triple
  \begin{equation*}
    \varphi = (\varphi_A: A \to A', \varphi_B: B \to B', \varphi_C: C \to C')
  \end{equation*}
  is a \term{homomorphism} of short exact sequences if the following diagram commutes:
  \begin{equation*}
    \begin{mplibcode}
      beginfig(1);
      input metapost/graphs;

      v1 := thelabel("$0$", origin);
      v2 := thelabel("$A$", (1, 0) scaled u);
      v3 := thelabel("$B$", (2, 0) scaled u);
      v4 := thelabel("$C$", (3, 0) scaled u);
      v5 := thelabel("$0$", (4, 0) scaled u);

      v6 := thelabel("$0$", (0, -1) scaled u);
      v7 := thelabel("$A'$", (1, -1) scaled u);
      v8 := thelabel("$B'$", (2, -1) scaled u);
      v9 := thelabel("$C'$", (3, -1) scaled u);
      v10 := thelabel("$0$", (4, -1) scaled u);

      a1 := straight_arc(v1, v2);
      a2 := straight_arc(v2, v3);
      a3 := straight_arc(v3, v4);
      a4 := straight_arc(v4, v5);

      a5 := straight_arc(v6, v7);
      a6 := straight_arc(v7, v8);
      a7 := straight_arc(v8, v9);
      a8 := straight_arc(v9, v10);

      a9 := straight_arc(v2, v7);
      a10 := straight_arc(v3, v8);
      a11 := straight_arc(v4, v9);

      draw_vertices(v);
      draw_arcs(a);

      label.top("$f$", straight_arc_midpoint of a2);
      label.top("$g$", straight_arc_midpoint of a3);

      label.bot("$f'$", straight_arc_midpoint of a6);
      label.bot("$g'$", straight_arc_midpoint of a7);

      label.rt("$\varphi_A$", straight_arc_midpoint of a9);
      label.rt("$\varphi_B$", straight_arc_midpoint of a10);
      label.rt("$\varphi_C$", straight_arc_midpoint of a11);
      endfig;
    \end{mplibcode}
  \end{equation*}

  If each component of \( \varphi \) is an isomorphism, we say that the short exact sequences are \term{isomorphic}.
\end{definition}

\begin{definition}\label{def:split_exact_sequence}\mcite{nLab:split_exact_sequence}
  A short exact sequence
  \begin{equation}\label{def:split_exact_sequence/short_diagram}
    0
    \longrightarrow
    A
    \overset f \longrightarrow
    B
    \overset g \longrightarrow
    C
    \longrightarrow
    0
  \end{equation}
  is said to be \term{splitting} or \term{split exact} if any of the following equivalent conditions hold:
  \begin{defenum}
    \item \( f \) has a left inverse
    \item \( g \) has a right inverse
    \item the sequence \fullref{def:split_exact_sequence/short_diagram} is isomorphic to the SES
    \begin{equation}\label{def:short_exact_sequence/split_diagram}
      0
      \longrightarrow
      A
      \longrightarrow
      A \otimes C
      \longrightarrow
      C
      \longrightarrow
      0
    \end{equation}
    with the canonical embedding and projection morphisms
  \end{defenum}

  The equivalence of the three conditions is called the \term{splitting lemma}.
\end{definition}

\begin{example}\label{ex:short_exact_sequences}
  \mbox{}
  \begin{defenum}
    \ilabel{ex:short_exact_sequences/cyclic_groups} Fix a natural number \( n > 0 \) and consider the category of \( \cat{Ab} \) of abelian groups and the following short exact sequence:
    \begin{equation*}
      0
      \longrightarrow
      \BbbZ
      \overset {n \cdot} \longrightarrow
      \BbbZ
      \overset {\lbrack \cdot \rbrack_n} \longrightarrow
      \BbbZ_n
      \longrightarrow
      0
    \end{equation*}
    where
    \begin{itemize}
      \item \( i(x) \coloneqq nx \) multiplies any integer by \( n \) to obtain the subgroup \( n \BbbZ \).
      \item \( p(x) \coloneqq [x]_n \) projects any integer into the corresponding remainder when divided by \( n \) (see \fullref{def:group_of_integers_modulo}).
    \end{itemize}

    The (group-theoretic) image \( n \BbbZ \) of \( f \) is precisely the (group-theoretic) kernel of \( [\cdot]_n \). The sequence does not split since \( f \) does not have a left inverse.

    \ilabel{ex:short_exact_sequences/real_number_splitting} Consider the additive groups \( \BbbZ \), \( \BbbR \) and the unit circle group \( S_{\BbbR^2} \) with the group operation given by addition of polar angles and with the vector \( (1, 0)^T \) as a unit.
    \begin{equation*}
      0
      \longrightarrow
      \BbbZ
      \overset f \longrightarrow
      \BbbR
      \overset g \longrightarrow
      S_{\BbbR^2}
      \longrightarrow
      0
    \end{equation*}
    where
    \begin{itemize}
      \item \( f \) is the canonical embedding of \( \BbbZ \) is \( \BbbR \)
      \item \( g(x) \coloneqq (\cos(\op{frac}(x)), \sin(\op{frac}(x)))^T \) (see \fullref{def:floor_ceiling_functions} and \fullref{def:quadratic_plane_curve/ellipse/parametric_equations}).
    \end{itemize}

    Since each integer has fractional part \( 0 \) and \( p(0) = (1, 0)^T \), the image \( \cat{Z} \) of \( \cat{Z} \) under \( f \) is the kernel of the group homomorphism \( g \).

    The sequence does not split since \( f \) is not left-invertible.

    \ilabel{ex:short_exact_sequences/vector_space_sum} The following SES of real vector spaces splits
    \begin{equation*}
      0
      \longrightarrow
      A
      \overset {\left(\begin{smallmatrix}1 \\ 0\end{smallmatrix}\right)} \longrightarrow
      B
      \overset {\left(\begin{smallmatrix}0 & 1\end{smallmatrix}\right)} \longrightarrow
      C
      \longrightarrow
      0
    \end{equation*}
    since all of the following equivalent conditions hold
    \begin{itemize}
      \item \( \left(\begin{smallmatrix}1 & 0\end{smallmatrix}\right) \) is a left inverse to \( \left(\begin{smallmatrix}1 \\ 0\end{smallmatrix}\right) \)
      \item \( \left(\begin{smallmatrix}0 \\ 1\end{smallmatrix}\right) \) is a right inverse to \( \left(\begin{smallmatrix}0 & 1\end{smallmatrix}\right) \)
      \item \( \BbbR^2 \) is a direct product and a biproduct of two copies of \( \BbbR \)
    \end{itemize}

    \ilabel{ex:short_exact_sequences/fundamental_theorem_of_calculus} The fundamental theorem of calculus is a splitting of the SES of vector spaces
    \begin{equation*}
      0
      \longrightarrow
      \BbbR
      \longrightarrow
      C^n(\BbbR, \BbbR)
      \overset {\frac d {dx}} \longrightarrow
      C^{n-1}(\BbbR, \BbbR)
      \longrightarrow
      0
    \end{equation*}
  \end{defenum}
\end{example}

\begin{definition}\label{def:chain_complex}\mcite{nLab:chain_complex}
  In an abelian category \( \cat{C} \), the tower \hyperref[def:tower_diagram]{diagram} with objects \( \{ C_n \}_{n \in \BbbZ} \) and morphisms \( \partial_n: C_n \to C_{n-1} \)
  \begin{equation}\label{def:chain_complex/chain_diagram}
    \cdots
    \overset {\partial_2} \longrightarrow
    C_1
    \overset {\partial_1} \longrightarrow
    C_0
    \overset {\partial_0} \longrightarrow
    C_{-1}
    \overset {\partial_{-1}} \longrightarrow
    \cdots
  \end{equation}
  is called a \term{chain complex} if for every \( n \),
  \begin{equation*}
    \partial_n \circ \partial_{n+1} = 0_{C_{n+1},C_{n-1}}.
  \end{equation*}

  Chain complexes may be finite or infinite in one or both directions. The morphisms \( \partial_n \) are called \term{boundary maps}.

  A \term{cochain complex} is a chain complex on \( \cat{C}^{-1} \), i.e.
  \begin{equation}\label{def:chain_complex/cochain_diagram}
    \cdots
    \overset {\partial_1} \longleftarrow
    C_1
    \overset {\partial_0} \longleftarrow
    C_0
    \overset {\partial_{-1}} \longleftarrow
    C_{-1}
    \overset {\partial_{-2}} \longleftarrow
    \cdots
  \end{equation}
  such that for any \( n \),
  \begin{equation*}
    \partial_{n+1} \circ \partial_n = 0_{C_{n-1},C_{n+1}}.
  \end{equation*}
\end{definition}


% Order theory
\subsection{Preorders}\label{subsec:preorders}

\begin{definition}\label{def:preordered_set}\mcite{nLab:preorder}
  A \hyperref[def:binary_relation]{binary relation} \( \leq \) on a set \( \mscrP \) is called a \term{preorder} if it is \hyperref[def:binary_relation/reflexive]{reflexive} and \hyperref[def:binary_relation/transitive]{transitive}.

  A \term{preordered set} is a set \( \mscrP \) equipped with a preorder \( \leq \). It is conventional to use the same symbol as for \hyperref[def:poset]{partial orders}, however the lack of \hyperref[def:binary_relation/antisymmetric]{antisymmetry} may be confusing --- see \fullref{ex:preorder_nonuniqueness}.

  \begin{thmenum}[series=def:preordered_set]
    \thmitem{def:preordered_set/theory} Consider a \hyperref[def:first_order_language]{first-order language} \( \mscrL \) with a single \hyperref[rem:order_infix_notation]{infix} binary predicate symbol \( \leq \). The \hyperref[def:first_order_semantics/theory]{theory} in \( \mscrL \) axiomatized by \hyperref[def:binary_relation/reflexive]{reflexivity} and \hyperref[def:binary_relation/transitive]{transitivity} for \( \leq \) is called the \term{theory of preordered sets}.

    \thmitem{def:preordered_set/homomorphism} A \hyperref[def:first_order_homomorphism]{homomorphism} between the preordered sets \( (\mscrP, \leq_\mscrP) \) and \( (\mscrQ, \leq_\mscrQ) \) is, explicitly, a function \( \varphi: \mscrP \to \mscrQ \) such that
    \begin{equation}\label{eq:def:preordered_set/homomorphism}
      x \leq_\mscrP y \T{implies} \varphi(x) \leq_\mscrQ \varphi(y)
    \end{equation}

    Such as function is called \term{monotone} or \term{order-preserving} or simply an \term{order homomorphism}.

    If
    \begin{equation*}
      x \neq y \T{implies} \varphi(x) \neq \varphi(y),
    \end{equation*}
    we call the function \( \varphi \) \term{strictly monotone}.

    In particular, if \( \mscrP \) is the set of \hyperref[rem:peano_arithmetic_zero/nonnegative]{nonnegative integers}, then we speak of \term{monotone sequences}
    \begin{equation*}
      \{ x_k \}_{k=1}^\infty,
    \end{equation*}
    where \( x_{k-1} \leq_Q x_k \) for all \( k = 1, 2, 3, \ldots \).

    \thmitem{def:preordered_set/substructure} Since the theory contains only positive formulas over a language with no functional symbols, any subset \( A \) of a preordered set \( (X, \leq) \) becomes a preordered set with the induced preorder \( \leq_A \) defined as the restriction of \( \leq \) to only elements of \( A \).

    \thmitem{def:preordered_set/category} We give no special name for the \hyperref[def:category_of_first_order_models]{category of models} for the theory of preordered sets.

    \thmitem{def:preordered_set/dual} The partially ordered set \( (\mscrP, \geq) \) where \( \geq \) is the \hyperref[def:binary_relation/converse]{converse relation}, is the called the \term{dual preordered set}. See \fullref{def:thin_category} for a discussion of the duality.
  \end{thmenum}
\end{definition}

\begin{definition}\label{def:directed_set}\mcite[8]{Engelking1989}
  A \hyperref[def:preordered_set]{preordered set} \( (\mscrP, \leq) \) is called a \term{directed set} if every finite subset of \( \mscrP \) has an upper \hyperref[def:preordered_set/upper_and_lower_bounds]{bound}, i.e. for all \( x, y \in \mscrP \) there exists \( z \in \mscrP \) such that \( x \leq z \) and \( y \leq z \).

  There is no established name for the relation itself.

  Note that \( \{ x, y \} \) may not have a supremum, i.e. the set of its upper bounds may not have a \hyperref[def:preordered_set/maximum_and_minimum]{smallest element}. Thus the upper bound condition is strictly weaker than every two-element set having a supremum.

  Directed sets are used to define nets in topological spaces, see \fullref{def:topological_net}.
\end{definition}

\begin{definition}\label{def:thin_category}\mcite{nLab:thin_category}
  A \hyperref[def:category]{category} \( \cat{P} \) is called a \term{thin category} if, for every two objects \( A, B \in \boldop{P} \), whenever \( f, g \in \boldop{P}(A, B) \), we have \( f = g \).

  If \( \cat{P} \) is locally small, this is equivalent to saying that any set of morphisms \( \boldop{P}(A, B) \) is at most a singleton.
\end{definition}

\begin{proposition}\label{thm:preorder_category_correspondence}
  To every \hyperref[def:preordered_set]{preordered set} there corresponds exactly one \hyperref[def:category_cardinality]{small} \hyperref[def:thin_category]{thin} category.

  Furthermore, \hyperref[def:preordered_set/supremum_and_infimum]{infima} correspond to categorical \hyperref[def:categorical_product]{products}, suprema to \hyperref[def:categorical_coproduct]{coproducts} and dual \hyperref[def:preordered_set/dual]{preordered sets} correspond to dual \hyperref[def:opposite_category]{categories}.

  Compare this result to \fullref{thm:partial_order_category_correspondence}.
\end{proposition}
\begin{proof}
  \SufficiencySubProof Let \( (P, \leq) \) be a preordered set. We define the category \( \cat{P} \) as follows:
  \begin{itemize}
    \item The \hyperref[def:category/C1]{class of objects} in \( \cat{P} \) is the set \( P \).
    \item The \hyperref[def:category/C2]{set of morphisms} between two elements \( x, y \in \cat{P} \) is the singleton \( \{ (x, y) \} \) when \( x \leq y \) and the empty set otherwise.
    \item The \hyperref[def:category/C3]{composition} of two morphisms \( (x, y) \) and \( (y, z) \) is simply \( (x, z) \) (such a morphism exists by transitivity of \( \leq \)).

    The axiom \ref{def:category/identity} follows from reflexivity of \( \leq \) and the axiom \ref{def:category/associativity} is trivial.
  \end{itemize}

  We showed that \( \cat{P} \) is indeed a category. We will only prove the equivalence of products and infima since the argument for suprema and coproducts is completely analogous.

  Let \( p \) be the categorical product of the set \( A \subseteq P \). Then \( p \leq x \) for all \( x \in A \), hence it is a lower bound. If \( q \) is another lower bound, then by definition of product, there exists a unique morphism \( q \leq p \). Therefore \( p = q \) is the infimum.

  \NecessitySubProof Now assume that \( \cat{P} \) is a thin small category. Define the relation \( \leq \) on the set \( \cat{P} \) as
  \begin{equation*}
    x \leq y \T{if and only if} \cat{P}(x, y) \neq \varnothing.
  \end{equation*}

  This is indeed a preorder because
  \begin{itemize}
    \item \( \leq \) is \hyperref[def:binary_relation/reflexive]{reflexivity} because of the existence of identity morphisms.
    \item \( \leq \) is \hyperref[def:binary_relation/transitive]{transitive} since if \( x \leq y \) and \( y \leq z \), composition of morphisms gives us \( x \leq z \).
  \end{itemize}

  Note that the infimum of a set \( A \subseteq \cat{P} \) (if it exists) has a unique morphism \( \inf A \) such that \( \inf A \leq x \) for any \( x \in A \). If \( y \leq x \) for all \( x \in A \) is another \hyperref[def:categorical_cone]{cone}, then necessarily \( y \leq \inf A \). Therefore the infimum is the categorical product.

  We proved that for each partially ordered set there corresponds at least one thin small category and vice versa. The fact that to each poset corresponds at most one category \( \cat{P} \) is obvious. Therefore we have a correspondence between the two.

  Duality is also obvious from our constructions.
\end{proof}

\subsection{Partial orders}\label{subsec:partial_orders}

\begin{definition}\label{def:poset}
  Fix a set \( \mscrP \). A \term{partially ordered set} structure, also called a \term{poset} structure, can be defined in the following equivalent ways:
  \begin{thmenum}[series=def:poset]
    \thmitem{def:poset/nonstrict} A \hyperref[def:preordered_set]{preorder} \( \leq \) on \( \mscrP \) such that \( \leq \) is \hyperref[def:binary_relation/antisymmetric]{antisymmetric} in addition to being \hyperref[def:binary_relation/reflexive]{reflexive} and \hyperref[def:binary_relation/transitive]{transitive}. This definition is the more common one. If we wish to distinguish it from the other definition, we call such a relation a \term{nonstrict partial order}.

    \thmitem{def:poset/strict} An \hyperref[def:binary_relation/irreflexive]{irreflexive} and \hyperref[def:binary_relation/transitive]{transitive} binary relation \( < \) on \( \mscrP \). This relation is called a \term{strict partial order}.
  \end{thmenum}

  If both relations are present, in order for the them to be equivalent, \( \leq \) must be the union of \( < \) and the \hyperref[def:binary_relation/diagonal]{diagonal} \( \Delta \). This condition corresponds to the following axiom:
  \begin{equation}\label{def:poset/compatibility_nonstrict}
    (x \leq y) \leftrightarrow \parens[\Big]{(x < y) \vee (x = y)}.
  \end{equation}

  By adding \( \placeholder \wedge \neg (x = y) \) to both sides of \eqref{def:poset/compatibility_nonstrict}, using \fullref{thm:de_morgans_laws} and taking irreflexivity of \( < \) into account, we obtain
  \begin{equation}\label{def:poset/compatibility_strict}
    (x < y) \leftrightarrow \parens[\Big]{(x \leq y) \wedge \neg (x = y)}.
  \end{equation}

  \begin{thmenum}[resume=def:poset]
    \thmitem{def:poset/theory} Since we can interdefine nonstrict and strict orders, it makes little sense to study separately theories for the two. Thus to obtain the \term{theory of posets}, we add to the \hyperref[def:preordered_set/theory]{theory of preordered sets}:
    \begin{itemize}
      \item \hyperref[def:binary_relation/antisymmetric]{Antisymmetry} for \( \leq \).
      \item A single binary predicate symbol \( < \).
      \item Either of the compatibility condition \eqref{def:poset/compatibility_nonstrict} or \eqref{def:poset/compatibility_strict} (it is unnecessary to add both).
    \end{itemize}

    We can also add \hyperref[def:binary_relation/irreflexive]{irreflexivity} and \hyperref[def:binary_relation/transitive]{transitivity} for \( < \) but that would also be redundant.

    \thmitem{def:poset/homomorphism} The \hyperref[def:first_order_homomorphism]{homomorphisms} in the theory are again \hyperref[def:poset/homomorphism]{monotone maps}. If we instead build a theory of strict partial orders without \( \leq \), homomorphisms are instead the more restrictive strict monotone maps.

    \thmitem{def:poset/substructure} As for preorders, any subset of a poset is itself a poset.

    \thmitem{def:poset/category} We denote the \hyperref[def:first_order_model_category]{model category} by \( \cat{Pos} \). It is a full subcategory of the \hyperref[def:preordered_set/category]{category of preordered sets}.
  \end{thmenum}
\end{definition}
\begin{proof}
  \ImplicationSubProof{def:poset/nonstrict}{def:poset/strict} Let \( \leq \) be a nonstrict partial order. We will show that \( < \) is a strict partial order.

  \begin{itemize}
    \item The relation \( < \) is \hyperref[def:binary_relation/transitive]{transitive}. To see this, let \( x < y \) and \( y < z \). In particular, \( x \leq y \) and \( y \leq z \). By transitivity, \( x \leq z \).

    Additionally, \( x \neq y \) and \( y \neq z \). Assume\DNE that \( x = z \). By reflexivity of \( \leq \), we have \( z \leq x \) and, since \( y \leq z \), by transitivity we obtain \( y \leq x \). But since \( x \leq y \), by the antisymmetry of \( \leq \), we have \( x = y \), which contradicts the assumption that \( x < y \).

    Therefore \( x < z \).

    \item \hyperref[def:binary_relation/irreflexive]{Irreflexivity} of \( < \) follows directly from reflexivity of \( \leq \) and the compatibility condition.
  \end{itemize}

  Since the right side is false, the left side \( x < x \) is also false.

  \ImplicationSubProof{def:poset/strict}{def:poset/nonstrict} Let \( < \) be a strict partial order. We will show that \( \leq \) is a nonstrict partial order.

  \begin{itemize}
    \item To see \hyperref[def:binary_relation/reflexive]{reflexivity}, fix \( x \in \mscrP \) and assume\DNE that \( x \not\leq x \). Then \( x \neq x \) which contradicts the reflexivity of equality. Hence \( x \leq x \).

    \item To see \hyperref[def:binary_relation/antisymmetric]{antisymmetry}, let \( x \leq y \) and \( y \leq x \), that is, either \( x = y \) or both \( x < y \) and \( y < x \) hold. Assume\DNE the latter. By the transitivity of \( \leq \), we have \( x < x \), which contradicts the irreflexivity of \( < \). Hence \( x = y \).

    \item To see \hyperref[def:binary_relation/transitive]{transitivity}, let \( x \leq y \) and \( y \leq z \). Then we have four cases depending on which of \( x \), \( y \) and \( z \) are equal. Since both relations \( < \) and \( = \) are transitive, it follows that in all four cases \( x \leq z \).
  \end{itemize}
\end{proof}

\begin{proposition}\label{thm:preorder_to_partial_order}
  Let \( (\mscrP, \leq) \) be a preordered set. Define the relation \( \cong \) by
  \begin{equation*}
    x \cong y \iff x \leq y \T{and} y \leq x.
  \end{equation*}

  That is, \( \cong \) is the intersection of the relation \( \leq \) with its \hyperref[def:binary_relation/converse]{converse}.

  Since \( \cong \) is an \hyperref[def:equivalence_relation]{equivalence relation} we can for the the quotient set \( \mscrP / \cong \). Define the relation \( \preceq \) on this quotient set by
  \begin{equation*}
    [x] \preceq [y] \iff x \leq y.
  \end{equation*}

  The pair \( (\mscrP / \cong, \preceq) \) is then a \hyperref[def:poset]{partially ordered set}.
\end{proposition}
\begin{proof}
  The relation \( \preceq \) is well-defined. Indeed, let \( x \cong x' \) and \( y \cong y' \), that is, both \( x \leq x' \) and \( x' \leq x \) and similarly for \( y \). If \( x \leq y \), by transitivity \( x \leq y \leq y' \). But \( x' \leq x \), hence \( x' \leq y' \).

  It is then clear that \( \preceq \) is a partial order because it inherits reflexivity and transitivity from \( \leq \) and antisymmetry is imposed by taking quotient sets --- equality in \( \mscrP / \cong \) holds precisely when \( \cong \) holds in \( \mscrP \).
\end{proof}

\begin{proposition}\label{thm:partial_order_category_correspondence}
  To every \hyperref[def:poset]{poset} there corresponds exactly one \hyperref[def:category_cardinality]{small} \hyperref[def:thin_category]{thin} \hyperref[def:skeletal_category]{skeletal} category.

  Compare this result to \fullref{thm:preorder_category_correspondence}.
\end{proposition}
\begin{proof}
  The statement follows from \fullref{thm:preorder_category_correspondence} by noting that the factorization in \fullref{thm:preorder_to_partial_order} makes the corresponding category skeletal.
\end{proof}

\subsection{Total orders}\label{subsec:total_orders}

\begin{definition}\label{def:totally_ordered_set}
  We say that the partially ordered set \( P \) is \term{totally ordered} if either the nonstrict order \( \leq \) is \hyperref[def:binary_relation/total]{total} or if the strict order \( < \) is \hyperref[def:binary_relation/trichotomic]{trichotomic}.
\end{definition}
\begin{proof}
  The two conditions are equivalent because \( x < y \) is trichotomic if and only if \( x \leq y \) is total.
\end{proof}

\begin{definition}\label{def:poset_chain}
  We call a subset \( A \subseteq X \) of a \hyperref[def:poset]{poset} \( (X, \leq) \) a \term{chain} if \( (A, \leq_A) \) is a totally \hyperref[def:totally_ordered_set]{ordered} set.

  Dually, we call the subset \( A \subseteq X \) an \term{antichain} if no two elements of \( A \) are \hyperref[def:preordered_set/comparability]{comparable}.
\end{definition}

\begin{definition}\label{def:total_order_interval}\mcite\cite{nLab:order_topology}
  In a \hyperref[def:poset]{totally ordered set} \( (P, \leq) \), for any \( a, b \in P \) with \( a \leq b \), we define
  \begin{thmenum}
    \labitem{def:total_order_interval/closed} the \term{closed interval}
    \begin{equation*}
      [a, b] \coloneqq \{ x \in P \colon a \leq x \leq b \}
    \end{equation*}

    \labitem{def:total_order_interval/open} the \term{open interval}
    \begin{equation*}
      (a, b) \coloneqq \{ x \in P \colon a < x < b \}
    \end{equation*}

    \labitem{def:total_order_interval/half_open} the \term{half-open intervals}
    \begin{balign*}
      (a, b] & \coloneqq \{ x \in P \colon a < x \leq b \}
      \\
      [a, b) & \coloneqq \{ x \in P \colon a \leq x < b \}
    \end{balign*}

    \labitem{def:total_order_interval/open_ray} the \term{open rays}
    \begin{balign*}
      (a, \infty)  & \coloneqq \{ x \in P \colon a < x \}
      \\
      (-\infty, b) & \coloneqq \{ x \in P \colon x < b \}
    \end{balign*}

    \labitem{def:total_order_interval/closed_ray} the \term{closed rays}
    \begin{balign*}
      [a, \infty)  & \coloneqq \{ x \in P \colon a \leq x \}
      \\
      (-\infty, b] & \coloneqq \{ x \in P \colon x \leq b \}
    \end{balign*}
  \end{thmenum}
\end{definition}

\begin{definition}\label{def:order_topology}\mcite\cite{nLab:order_topology}
  Let \( (P, <) \) be a \hyperref[def:poset]{totally ordered set}. The \term{order topology induced by \( < \)} is the topology generated by the \hyperref[def:topological_subbase]{subbase} of open \hyperref[def:total_order_interval/open_ray]{rays}
  \begin{equation*}
    \mathcal{P} \coloneqq \{ (a, \infty) \colon a \in P \} \cup \{ (-\infty, b) \colon b \in P \}.
  \end{equation*}
\end{definition}

\begin{lemma}[Zorn's lemma]\label{thm:zorns_lemma}\mcite\cite{nLab:zorns_lemma}
  If any \hyperref[def:poset_chain]{chain} in a \hyperref[def:poset]{partially ordered set} has an upper \hyperref[def:preordered_set/upper_lower_bound]{bound}, there exists a \hyperref[def:preordered_set/maximal_minimal_element]{maximal} set in \( X \).
\end{lemma}

\begin{definition}\label{def:well_ordered_set}
  A totally ordered \hyperref[def:totally_ordered_set]{set} \( (X, \leq) \) is said to be \term{well-ordered} if every nonempty subset \( A \subseteq X \) has a \hyperref[def:preordered_set/largest_smallest_element]{minimum}.
\end{definition}

\begin{theorem}[Well-Ordering Principle]\label{thm:well_ordering_principle}\mcite\cite[196]{Enderton1977Sets}
  Any \hyperref[def:set_zfc]{set} can be \hyperref[def:well_ordered_set]{well-ordered}.
\end{theorem}

\subsection{Lattices}\label{subsec:lattices}

\begin{definition}\label{def:semilattice}\mcite[3]{Gratzer1978}
  Lattices are \hyperref[def:partially_ordered_set]{partially ordered sets} in which \hyperref[def:partially_ordered_set_extremal_points/supremum_and_infimum]{suprema and infima} are taken as basic operations called \enquote{joins} and \enquote{meets}. See \fullref{rem:lattice_operation_etymology} for a discussion of the operation names. This shifts the focus from ordering to operations, i.e. from predicates to functions.

  Joins and meets may also be defined axiomatically as binary operations (see \fullref{thm:binary_lattice_operations}) rather than via, some partial order, however this restricts us to taking suprema of finite sets and prevents us from taking the supremum of an arbitrary set. In other words, it is possible for the order to carry more information than joins and meets. See \fullref{thm:binary_lattice_operations/new_lattice} for a discussion. Unless explicitly noted otherwise, we assume that lattices have their partial order defined.

  \begin{thmenum}[series=def:semilattice]
    \thmitem{def:semilattice/join} A \term{join-semilattice} is a \hyperref[def:partially_ordered_set_extremal_points/top_and_bottom]{bounded from above} partially ordered set in which every finite supremum exists. The operation itself is denoted by \( \vee \) and referred to as \term{join} and rather than supremum. In contrast to suprema, joins are usually written in \hyperref[rem:first_order_formula_conventions/infix]{infix} notation, e.g. \( x \vee y \vee z \) rather than \( \sup\set{ x, y, z } \).

    \thmitem{def:semilattice/meet} Analogously, a \term{meet-semilattice} is a partially ordered set in which every finite infimum exists. The infimum is denoted by \( \wedge \) and called \term{meet}.

    \thmitem{def:semilattice/bounded} A \term{bounded semilattice} is a semilattice that is \hyperref[def:partially_ordered_set_extremal_points/top_and_bottom]{bounded} as a \hyperref[def:partially_ordered_set]{partially ordered set}, either from above for join-semilattices or from below for meet-semilattices.

    \thmitem{def:semilattice/complete}\mcite[24]{Gratzer1978} A semilattice is said to be \term{complete} if the corresponding operation is defined for arbitrary sets rather than only finite ones. Finite lattices are trivially complete and complete lattices are trivially bounded.

    \thmitem{def:semilattice/lattice} A \term{lattice} is a partially ordered set which is both a join-semilattice and a meet-semilattice. It is called \term{bounded} if both semilattices are bounded, i.e. if the partially ordered set itself is \hyperref[def:partially_ordered_set_extremal_points/top_and_bottom]{bounded}. It is called \term{complete} if both semilattices are complete.

    \thmitem{def:semilattice/distributive_lattice}\mcite[30]{Gratzer1978} A lattice is said to be \term{distributive} if the following two distributive conditions hold:
    \begin{align}
      x \vee (y_1 \wedge y_2) &= (x \vee y_1) \wedge (x \vee y_2) \label{eq:def:semilattice/distributive_lattice/finite/join_over_meet} \\
      x \wedge (y_1 \vee y_2) &= (x \wedge y_1) \vee (x \wedge y_2) \label{eq:def:semilattice/distributive_lattice/finite/meet_over_join}.
    \end{align}

    If the lattice is \hyperref[def:semilattice/complete]{complete}, the above conditions are not enough. A complete lattice \( \mscrX \) it is said to be \term{distributive} if any of the following more general distributive axioms hold for any \( x \in \mscrX \) and \hyperref[def:indexed_family]{family} \( \seq{ y_k }_{k \in \mscrK} \subseteq \mscrX \):
    \begin{align}
      x \vee \parens*{ \bigwedge_{k \in \mscrK} y_k } &= \bigwedge_{k \in \mscrK} \parens{ x \vee y_k } \label{eq:def:semilattice/distributive_lattice/arbitrary/join_over_meet} \\
      x \wedge \parens*{ \bigvee_{k \in \mscrK} y_k } &= \bigvee_{k \in \mscrK} \parens{ x \wedge y_k } \label{eq:def:semilattice/distributive_lattice/arbitrary/meet_over_join}
    \end{align}
  \end{thmenum}

  Lattices have the following metamathematical structure:
  \begin{thmenum}[resume=def:semilattice]
    \thmitem{def:semilattice/theory} The language of the theory of lattices consists of the language of the \hyperref[def:partially_ordered_set/theory]{theory of partially ordered sets} with the addition of the binary infix functional symbols \( \vee \) and \( \wedge \). If we only want to restrict ourselves to semilattices, we can add only one of the two operations as functional symbols. If we wish to study \hyperref[def:semilattice/bounded]{bounded lattices}, as it is often done, we must also add the constants \( \top \) and \( \bot \).

    For meet-semilattices, we add the following axiom schema to the theory to ensure compatibility between infima and meets (we use \( \mathbin\& \) to denote \hyperref[def:propositional_language/connectives/conjunction]{logical conjunction} to avoid symbol collision with meets):
    \begin{equation}\label{eq:def:semilattice/theory/meet_compat}
      \parens[\Big]{ \xi \wedge \eta \doteq \alpha } \leftrightarrow \parens[\Big]{ \alpha \leq \xi \mathbin\& \alpha \leq \eta \mathbin\& \qforall \alpha ((\alpha \leq \xi \mathbin\& \alpha \leq \eta) \rightarrow \alpha \leq \alpha) }
    \end{equation}
    and, for bounded lattices, the following axiom to ensure that \( \bot \) is indeed the minimum:
    \begin{equation}\label{eq:def:semilattice/theory/bottom_compat}
      \qforall \xi (\bot \leq \xi).
    \end{equation}

    Analogous axioms need to be added for join-semilattices.

    We cannot proper express the theory of complete (semi)lattices as an extension of this theory since we must define join and meet as unary operations on subsets of the domain rather than binary operations on members of the domain. Complete semilattices can instead be defined within \hyperref[def:zfc]{\logic{ZFC}}.

    \thmitem{def:semilattice/submodel} Unlike for partially ordered sets (see \fullref{def:partially_ordered_set/submodel}), not any subset of a semilattice is a subsemilattice because \( \vee \) and \( \wedge \) are now regarded as functional symbols. The axiom \eqref{eq:def:semilattice/theory/meet_compat} is not a positive formula, but does not cause trouble itself as it merely specifies compatibility of \( \leq \) and \( \wedge \).

    For bounded semilattices, the relevant constants should be present in any bounded subsemilattice.

    \thmitem{def:semilattice/trivial} The \hyperref[thm:substructures_form_complete_lattice/bottom]{trivial join-semilattice} and the trivial meet-semilattice are the empty set. The trivial bounded join-semilattice is the singleton \( \set{ \top } \) and the trivial bounded meet-semilattice is \( \set{ \bot } \).

    The trivial bounded lattice consists of \( \set{ \top, \bot } \).

    \thmitem{def:semilattice/homomorphism} \hyperref[def:first_order_homomorphism]{Homomorphisms} between (semi)lattices are simply the monotone maps.

    Alternatively, without referring to the order, we can characterize homomorphisms as functions preserving joins, meets and constants. No axioms follow automatically as in \fullref{thm:group_homomorphism_single_condition}.

    \thmitem{def:semilattice/category} The \hyperref[def:category_of_first_order_models]{categories of models} for (semi)lattices are full subcategories of \hyperref[def:partially_ordered_set/category]{\( \cat{Pos} \)}. We only give a special name for the category \( \cat{Lat} \) of lattices.

    \thmitem{def:semilattice/lattice_duality} The \hyperref[def:partially_ordered_set/duality]{principle of duality for partially ordered sets} holds for lattices if we also swap the binary operations \( \vee \) and \( \wedge \).

    If the lattice is bounded, we must additionally swap the constants \( \top \) and \( \bot \).

    If the lattice is bounded from only one side, the principle of duality does not hold unless we restrict ourselves to formulas that do not contain the constants.
  \end{thmenum}
\end{definition}

\begin{remark}\label{rem:lattice_operation_etymology}
  The terms \hyperref[def:semilattice/join]{\enquote{join}} for \( \vee \) and \hyperref[def:semilattice/meet]{\enquote{meet}} for \( \wedge \) are notoriously difficult to remember. A helpful accident is the ability to write \enquote{meet} as \enquote{\( \wedge \wedge \)eet}.
\end{remark}

\begin{proposition}\label{thm:binary_lattice_operations}
  Let \( (\mscrP, \leq) \) be a partially ordered set.

  \begin{thmenum}
    \thmitem{thm:binary_lattice_operations/semilattices} If it is a \hyperref[def:semilattice/join]{join-semilattice} (resp. \hyperref[def:semilattice/meet]{meet-semilattice}), then \( \vee \) (resp. \( \wedge \)) is \hyperref[def:magma/associative]{associative}, \hyperref[def:magma/commutative]{commutative} and \hyperref[def:magma/idempotent]{idempotent} when considered as a binary operation.

    \thmitem{thm:binary_lattice_operations/identity} If \( \mscrP \) is a (semi)lattice, the constants act as \hyperref[def:magma_identity]{magma identities}. That is, for each \( x \in \mscrP \),
    \begin{align}
      x \vee \bot = x \label{eq:thm:binary_lattice_operations/identity/join} \\
      x \wedge \top = x \label{eq:thm:binary_lattice_operations/identity/meet}
    \end{align}

    \thmitem{thm:binary_lattice_operations/absorption} If \( \mscrP \) is a lattice, then the following absorption laws hold:
    \begin{align}
      x \vee (x \wedge y) &= x \label{eq:thm:binary_lattice_operations/absorption/join} \\
      x \wedge (x \vee y) &= x \label{eq:thm:binary_lattice_operations/absorption/meet}.
    \end{align}

    \thmitem{thm:binary_lattice_operations/compatibility} The following conditions for compatibility with \( \leq \) hold:
    \begin{align}
      x \leq y &\T{if and only if} x \vee y = y \label{eq:thm:binary_lattice_operations/compatibility/join} \\
      x \leq y &\T{if and only if} x \wedge y = x \label{eq:thm:binary_lattice_operations/compatibility/meet}.
    \end{align}

    \thmitem{thm:binary_lattice_operations/new_lattice} If \( S \) is an arbitrary \hyperref[def:set]{set} and if \( \vee \) is a binary operation that is associative, commutative and idempotent (the conclusion of \fullref{thm:binary_lattice_operations/semilattices}), then \( (S, \leq) \) is a join-semilattice with an ordering defined by \eqref{eq:thm:binary_lattice_operations/compatibility/join}. If there exists a distinguished element \( \top \) such that \eqref{eq:thm:binary_lattice_operations/identity/join} holds, then \( (S, \leq) \) is a meet-semilattice.

    A completely analogous statement holds for meet-semilattices.

    If \( (S, \leq) \) is both a join-semilattice and meet-semilattice and if \( \vee \) and \( \wedge \) satisfy the absorption conditions \eqref{eq:thm:binary_lattice_operations/absorption/join} and \eqref{eq:thm:binary_lattice_operations/absorption/meet}, then \( (S, \leq) \) is a lattice. Furthermore, proving idempotence for \( \vee \) or \( \wedge \) is unnecessary because both follow from the absorption conditions.

    It may turn out that \( (S, \leq) \) is a complete lattice under this definition. This can allow us, for example, to transparently extend the binary operations join and meet into infinitary operations.
  \end{thmenum}
\end{proposition}
\begin{proof}
  \SubProofOf{thm:binary_lattice_operations/semilattices} Suprema and infima are obviously associative and commutative as binary operations because ordering is immaterial for pure sets and \( x \vee y \) is defined as \( \sup\set{ x, y } \).

  Idempotence is also obvious because \( x \vee x = \sup\set{ x } = x \).

  \SubProofOf{thm:binary_lattice_operations/identity} Obvious since \( \bot \leq x \leq \top \) for all \( x \in \mscrP \).

  \SubProofOf{thm:binary_lattice_operations/absorption} If we rewrite \eqref{eq:thm:binary_lattice_operations/absorption/join} using suprema and infima, we obtain
  \begin{equation*}
    \sup\set{ x, \inf\set{ y, x } } = x.
  \end{equation*}

  If \( x \leq y \), then \( \inf\set{ y, x } = x \) and \( \sup\set{ x, \inf\set{ y, x } } = \sup\set{ x, x } = x \).

  If \( x \geq y \), then \( \inf\set{ y, x } = y \) and \( \sup\set{ x, \inf\set{ y, x } } = \sup\set{ x, y } = x \).

  This proves \eqref{eq:thm:binary_lattice_operations/absorption/join}. Since \( \wedge \) is \( \vee \) in the \hyperref[def:preordered_set/duality]{dual partially ordered set}, \eqref{eq:thm:binary_lattice_operations/absorption/meet} follows automatically.

  \SubProofOf{thm:binary_lattice_operations/compatibility} We have
  \begin{equation*}
    x \vee y
    =
    \sup\set{ x, y }
    =
    \begin{cases}
      y, &x \leq y \\
      x, &x > y
    \end{cases}
  \end{equation*}
  and dually for \( \wedge \).

  \SubProofOf{thm:binary_lattice_operations/new_lattice} Since the binary join and/or meet are defined for all members of the set \( S \), it is indeed a join-semilattice because all finite joins and meets exist by definition.

  Idempotence of \( \vee \) follows from \eqref{eq:thm:binary_lattice_operations/absorption/meet}:
  \begin{equation*}
    x \vee x = x \vee (x \wedge (x \vee x)) = x
  \end{equation*}
  and dually for \( \wedge \).
\end{proof}

\begin{definition}\label{def:fixed_point}
  Given a \hyperref[def:function]{function} \( f: A \to A \) between arbitrary sets, we call \( x \in A \) a \term{fixed point} of \( f \) if \( x = f(x) \).
\end{definition}

\begin{theorem}[Knaster-Tarski theorem]\label{thm:knaster_tarski_theorem}
  The \hyperref[def:fixed_point]{fixed points} of a \hyperref[def:partially_ordered_set/homomorphism]{monotone} \hyperref[def:multi_valued_function/endofunction]{endofunction} in a \hyperref[def:semilattice/lattice]{complete lattice} form a complete sublattice. In particular, the function has at least one fixed point.
\end{theorem}
\begin{proof}
  Let \( (\mscrX, \leq) \) be a complete lattice and let \( \varphi: \mscrX \to \mscrX \) be a monotone function. Define
  \begin{equation*}
    L \coloneqq \{ x \in \mscrX \colon f(x) \leq x \}.
  \end{equation*}

  We know that \( L \) is nonempty because \( \top \in L \).

  Since the lattice is complete, we can take \( l \coloneqq \inf L \). Note that \( f(l) \) is a lower bound of \( L \) because for any \( y \in L \) we have
  \begin{equation*}
    f(l) \leq f(y) \leq y.
  \end{equation*}

  But \( l \) is the largest lower bound of \( L \), hence
  \begin{equation}\label{eq:thm:knaster_tarski/f_lower}
    f(l) \leq l.
  \end{equation}

  Therefore, \( f(f(l)) \leq f(l) \) and \( f(l) \in L \). Hence, \( l \) is a lower bound for \( \{ f(l) \} \) and
  \begin{equation}\label{eq:thm:knaster_tarski/f_upper}
    l \leq f(l).
  \end{equation}

  From \eqref{eq:thm:knaster_tarski/f_lower} and \eqref{eq:thm:knaster_tarski/f_upper} it follows that \( l = f(l) \), that is, \( l \) is a fixed point of \( f \).

  Denote by \( F \) the set of all fixed points of \( \mscrX \). We just showed that \( F \) is nonempty. Let \( G \subseteq F \). We will show that the infimum and supremum of \( G \) is in \( F \).

  Denote
  \begin{equation*}
    l_G \coloneqq \inf G.
  \end{equation*}

  For any \( g \in G \) we have \( l_G \leq g \). From monotonicity of \( f \),
  \begin{equation*}
    f(l_G) \leq f(g) = g,
  \end{equation*}
  therefore \( f(l_G) \leq l_G \) because \( l_G \) is the greatest lower bound of \( G \). But, from monotonicity of \( f \), we have \( l_G \leq f(l_G) \). Therefore, \( f(l_G) = l_G \) and \( l_G \in F \).

  We can analogously show that \( \sup G \in F \) and conclude that \( (F, \leq) \) is itself a complete lattice.
\end{proof}

\begin{remark}\label{def:semilattice/lattice_categorical_product}
  The existence of finite joins and meets is equivalent to the existence of finite products and coproducts in the respective partially ordered set category defined in \fullref{thm:partial_order_category_correspondence}.
\end{remark}


% Combinatorics
\subsection{Enumerative combinatorics}\label{subsec:enumerative_combinatorics}

This subsection lists several results of various importance that don't really belong to any more consistent theory.

\begin{theorem}[Dirichlet's pigeonhole principle]\label{def:pigeonhole_principle}
  If we have more pigeons than pigeonholes, then at least one pigeonhole must contain multiple pigeons in it.

  More formally, if \( \card(A) > \card(B) \), then there exists no injective function from \( A \) to \( B \).
\end{theorem}
\begin{proof}
  This is a corollary of \fullref{thm:set_domination_relation_trichotomy}.
\end{proof}

\begin{definition}\label{def:binomial_coefficient}
  The \term{binomial coefficient} of the \hyperref[rem:peano_arithmetic_zero/positive]{positive integers} \( n \) and \( k \) is
  \begin{equation*}
    \binom n k \coloneqq \frac {n!} {k!(n-k)!}
  \end{equation*}

  They are motivated by \fullref{thm:binomial_theorem}.
\end{definition}

\begin{theorem}[Pascal's identity]\label{thm:pascals_identity}
  \begin{equation*}
    \binom n k = \binom {n - 1} k + \binom {n - 1} {k - 1}.
  \end{equation*}
\end{theorem}
\begin{proof}
  \begin{balign*}
    \binom {n - 1} k + \binom {n - 1} {k - 1}
    &=
    \frac {(n - 1)!} {k! (n - 1 - k)!} + \frac {(n - 1)!} {(k - 1)! (n - k)!}
    = \\ &=
    \frac {(n - 1)!} {(k - 1)! (n - 1 - k)!} \bracks*{ \frac 1 k + \frac 1 {n - k} }
    = \\ &=
    \frac {(n - 1)!} {(k - 1)! (n - 1 - k)!} \frac n {k(n - k)}
    = \\ &=
    \frac {n!} {k! (n - k)!}
    = \\ &=
    \binom n k.
  \end{balign*}
\end{proof}

\begin{definition}\label{def:factorial}
  The \term{factorial} of a \hyperref[rem:peano_arithmetic_zero/nonnegative]{nonnegative integer} \( n \) is defined recursively as
  \begin{equation*}
    n! \coloneqq \begin{cases}
      1,          &n = 0 \\
      (n - 1)! n, &n > 0.
    \end{cases}
  \end{equation*}
\end{definition}

\begin{proposition}\label{thm:gamma_function_interpolates_factorial}
  For every \hyperref[rem:peano_arithmetic_zero/nonnegative]{nonnegative integer} \( n \) we have
  \begin{equation*}
    \Gamma(n + 1) \coloneqq n!,
  \end{equation*}
  where \( \Gamma \) is the Gamma function defined in \fullref{def:gamma_function}.
\end{proposition}
\begin{proof}
  We use induction on \( n \).
  \begin{itemize}
    \item If \( n = 0 \), then
    \begin{equation*}
      \Gamma(1)
      =
      \int_0^\infty x^0 e^{-x} \dl x
      =
      -e^{-x}\restr_{x=0}^\infty
      =
      -\underbrace{\lim_{x \to \infty} e^{-x}}_{0} + 1
      =
      1
      =
      0!
    \end{equation*}

    \item If \( n > 0 \) and \( \Gamma(n) = (n - 1)! \), then
    \begin{balign*}
      \Gamma(n + 1)
      &=
      \int_0^\infty x^n \cdot e^{-x} \dl x
      = \\ &=
      \underbrace{(- x^n e^{-x})\restr_{x=0}^\infty}_{-(0 - 0)} + n \int_0^\infty e^{-x} x^{n-1} \dl x
      = \\ &=
      n \Gamma(n)
      = \\ &=
      n (n - 1)!
      = \\ &=
      n!
    \end{balign*}
  \end{itemize}
\end{proof}

\begin{theorem}[Stirling's factorial approximation]\label{thm:stirlings_factorial_approximation}
  For every \hyperref[rem:peano_arithmetic_zero/nonnegative]{nonnegative integer} \( n \) there exists some constant \( \theta \in (0, 1) \) such that
  \begin{equation*}
    n! = \sqrt{2 \pi n} \cdot \parens*{ \frac n e }^n \cdot e^{\frac 1 {12n + \theta}}.
  \end{equation*}
\end{theorem}
\begin{proof}
  Follows from \fullref{thm:gamma_function_interpolates_factorial} and \fullref{thm:stirlings_gamma_approximation}.
\end{proof}

\begin{remark}\label{rem:double_index_maps}
  We want to be able to map single indices to double indices and vice versa, for example for the purpose of \fullref{thm:matrix_spaces_are_tuple_spaces}. As an example, we want to be able to \enquote{linearize} an \( m \times n \) matrix such as the \( 2 \times 3 \) matrix
  \begin{equation}\label{eq:rem:double_index_maps/example/matrix}
    \begin{pmatrix}
      1 & 2 & 3 \\
      4 & 5 & 6
    \end{pmatrix}
  \end{equation}
  into the tuple
  \begin{equation}\label{eq:rem:double_index_maps/example/row_major}
    (1, 2, 3, 4, 5, 6)
  \end{equation}
  and vice versa. This is called \term{row-major order} of the elements of a matrix. The \term{column-major order} would instead be
  \begin{equation}\label{eq:rem:double_index_maps/example/column_major}
    (1, 4, 2, 5, 3, 6).
  \end{equation}

  Let \( m \) and \( n \) be \hyperref[rem:peano_arithmetic_zero/positive]{positive integers}. We will explicitly define functions for linearizing a matrix like \eqref{eq:rem:double_index_maps/example/matrix} into its row-major order \eqref{eq:rem:double_index_maps/example/row_major}. Define the sets
  \begin{align*}
    S &\coloneqq \overbrace{ \set{ 1, \ldots, mn - 1, mn } }^{\T{single indices}}
    \\
    D &\coloneqq \underbrace{ \set{ 1, \ldots, m } \times \set{ 1, \ldots, n } }_{\T{double indices}}
  \end{align*}
  and the mutually inverse operations
  \begin{align}
    &\begin{aligned}\label{eq:rem:double_index_maps/sharp}
      &\sharp: S \to D \\
      &\sharp(k) \coloneqq \parens[\Big]{ \quot(k - 1, m) + 1, \rem(k - 1, m) + 1 } \\
    \end{aligned}
    \\[0.5\baselineskip]
    &\begin{aligned}\label{eq:rem:double_index_maps/flat}
      &\flat: D \to S \\
      &\flat(i, j) \coloneqq (i - 1) \cdot m + (j - 1) + 1.
    \end{aligned}
  \end{align}

  The operation \( \sharp \) encodes the matrix \eqref{eq:rem:double_index_maps/example/matrix} into its row-major order \eqref{eq:rem:double_index_maps/example/row_major} and \( \flat \) does the opposite. Both operations are trivial except for the shifting needed in to allow us to use \hyperref[def:euclidean_domain]{remainders and quotients}.

  We can easily verify that \( \sharp \) is a \hyperref[def:morphism_invertibility/left_invertible]{left inverse} of \( \flat \) (note that \( j < m \)):
  \begin{align*}
    \sharp(\flat(i, j))
    &=
    \sharp\parens[\Big]{ (i - 1) \cdot m + (j - 1) + 1 }
    = \\ &=
    \parens[\Big]{ \quot(\cdots, m) + 1, \rem(\cdots, m) + 1 }
    = \\ &=
    \parens[\Big]{ (i - 1) + 1, (j - 1) + 1 }
    = \\ &=
    (i, j).
  \end{align*}

  We can just as easily verify that \( \flat \) is a \hyperref[def:morphism_invertibility/right_invertible]{right inverse} of \( \sharp \):
  \begin{align*}
    \flat(\sharp(k))
    &=
    \flat\parens[\Big]{ \quot(k, m) + 1, \rem(k, m) + 1 }
    = \\ &=
    \quot(k, m) \cdot m + \rem(k, m)
    = \\ &=
    k.
  \end{align*}

  Hence, \( \sharp \) is fully invertible with inverse \( \flat \). By \fullref{thm:function_invertibility_categorical/fully_invertible}, it is bijective.
\end{remark}

\subsection{Progressions}\label{subsec:progressions}

Progressions are an elementary concept that happens to be useful quite often. There is no definition of progression, but rather the term \enquote{progression} refers to specific recursively defined \hyperref[def:sequence]{sequences}.

\begin{definition}\label{def:arithmetic_progression}
  The \term{arithmetic progression} with \term{base} \( a_0 \) and \term{difference} \( d \) is the sequence
  \begin{equation}\label{eq:def:arithmetic_progression}
    a_k \coloneqq \begin{cases}
      a_0,         & k = 0, \\
      a_{k-1} + d, & k > 0.
    \end{cases}
  \end{equation}

  Clearly every index \( k \geq 0 \) we have the closed form representation \( a_k = a_0 + kd \).
\end{definition}

\begin{proposition}\label{thm:arithmetic_progression_partial_sums}
  The \hyperref[def:convergent_series]{series} constructed from the arithmetic progression \eqref{eq:def:arithmetic_progression} has partial sums
  \begin{equation}\label{eq:thm:arithmetic_progression_partial_sums}
    \sum_{k=0}^n a_k = \frac {(n + 1) (a_n - a_0)} 2.
  \end{equation}

  In the special case where \( a_0 = 0 \) and \( d = 1 \), this reduces to
  \begin{equation}\label{eq:thm:arithmetic_progression_partial_sums/integers}
    \sum_{k=0}^n k = \sum_{k=1}^n k = \frac {n (n + 1)} 2.
  \end{equation}
\end{proposition}
\begin{proof}
  \begin{balign*}
    2 \sum_{k=0}^n a_k
     & =
    2 \sum_{k=0}^n (a_0 + kd)
    =    \\ &=
    \sum_{k=0}^n (a_0 + kd) + \sum_{k=0}^n (a_0 + (n-k)d)
    =    \\ &=
    \sum_{k=0}^n (2 a_0 + nd)
    =    \\ &=
    (n + 1) (a_0 + a_n).
  \end{balign*}
\end{proof}

\begin{definition}\label{def:geometric_progression}
  The \term{geometric progression} with \term{base} \( a_0 \) and \term{denominator} \( q \) is the sequence
  \begin{equation}\label{eq:def:geometric_progression}
    a_k \coloneqq \begin{cases}
      a_0,       & k = 0, \\
      a_{k-1} q, & k > 0.
    \end{cases}
  \end{equation}

  Clearly every index \( k \geq 0 \) we have the closed form representation \( a_k = a_0 q^k \).

  The \hyperref[def:convergent_series]{series}
  \begin{equation}\label{eq:def:geometric_progression/series}
    \sum_{k=0}^\infty a_k = a_0 \sum_{k=0}^\infty q^k.
  \end{equation}
  is called the \term{geometric series} for \( q \). Without loss of generality, we will assume \( a_0 = 1 \) when speaking about geometric series.
\end{definition}

\begin{proposition}\label{thm:geometric_series_properties}
  The geometric series \eqref{eq:def:geometric_progression/series} has the following basic properties:
  \begin{thmenum}
    \thmitem{thm:geometric_series_properties/finite_sum} For all \( q \in \BbbC \setminus \{ 1 \} \), the geometric series \eqref{eq:def:geometric_progression/series} has partial sums
    \begin{equation}\label{thm:geometric_progression/partial_sum}
      \sum_{k=0}^n q^k = \frac {1 - q^{n+1}} {1 - q}.
    \end{equation}

    Compare this to \fullref{thm:xn_minus_one_factorization}.

    \thmitem{thm:geometric_series_properties/degenerate} In the degenerate case \( q = 1 \), the progression itself is constant, and its partial sums are instead
    \begin{equation}\label{thm:geometric_progression/degenerate}
      \sum_{k=0}^n q^k = n + 1.
    \end{equation}

    \thmitem{thm:geometric_series_properties/series_sum_exterior} For \( \abs{q} \geq 1 \), the geometric series diverges.

    \thmitem{thm:geometric_series_properties/series_sum_interior} For \( 0 < \abs{q} < 1 \), the geometric series converges absolutely with sum
    \begin{equation}\label{thm:geometric_progression/series_sum}
      \sum_{k=0}^\infty q^k = \frac 1 {1 - q}.
    \end{equation}
  \end{thmenum}
\end{proposition}
\begin{proof}
  \SubProofOf{thm:geometric_series_properties/finite_sum} Follows from \fullref{thm:xn_minus_one_factorization}.
  \SubProofOf{thm:geometric_series_properties/degenerate} Obvious.

  \SubProofOf{thm:geometric_series_properties/series_sum_exterior} For \( q = 1 \), \fullref{thm:geometric_series_properties/degenerate} implies that the series diverges because it grows indefinitely. If \( \abs{q} = 1 \) and \( q \neq 1 \), the integer powers \( q^k \) are rotations around the complex plane unit circle, which do not tend to a limit. Hence, the series diverges again.

  When \( \abs{q} > 1 \), \( \abs{q^n} \) grows indefinitely with \( n \), and it follows that
  \begin{equation*}\label{thm:geometric_progression/cauchy_partial_sum}
    \sum_{k=m}^n q^k
    =
    q^m \sum_{k=0}^{n-m} q^k
    =
    q^m \frac {1 - q^{n-m+1}} {1 - q}
    =
    \frac {q^m - q^{n+1}} {1 - q}.
  \end{equation*}
  can get arbitrarily large. Therefore, in this case the series also diverges.

  \SubProofOf{thm:geometric_series_properties/series_sum_interior} Fix \( q \in B(0, 1) \). Since only \( q^{n + 1} \) depends on \( n \) in \eqref{thm:geometric_progression/partial_sum}, we obtain \eqref{thm:geometric_progression/series_sum} by simply noting that \( q^n \to 0 \) when \( n \to \infty \).
\end{proof}

\begin{example}\label{ex:n_ary_decomposition}
  A simple but important practical example of a \hyperref[eq:def:geometric_progression/series]{geometric series} is
  \begin{equation}\label{eq:ex:n_ary_decomposition/binary}
    \sum_{k=0}^\infty \frac 1 {2^k} = \frac 1 {1 - \sfrac 1 2} = 2.
  \end{equation}

  Note that if the series starts at \( k = 1 \) instead of \( k = 0 \), it sums to \( 1 \). This is often applied in analysis indirectly via \fullref{thm:continuous_function_series_powers_of_two}.

  Another application of \eqref{eq:ex:n_ary_decomposition/binary} is showing that \( 0.\overline{1} = 2 \) in the binary number system. More generally, for the \( n \)-ary number system we have
  \begin{equation}\label{eq:ex:n_ary_decomposition/general}
    \sum_{k=0}^\infty \parens*{ \frac {n-1} n }^k = \frac 1 {1 - \ifrac {(n-1)} n} = n.
  \end{equation}
\end{example}

\begin{remark}\label{rem:progressions_and_interest}
  In this example we exploit the equivalence between the closed form representations in \fullref{def:arithmetic_progression} and \fullref{def:geometric_progression} and the corresponding inductive definitions. The equivalences are obvious from a mathematical standpoint, however outside of mathematics they have highly nontrivial consequences. Indeed, they highlight the difference between simple interest and compound interest.

  As an example, a savings account with \( 1000\$ \) with a simple monthly interest of \( 2\% \) will earn \( 240\$ \) over a year:
  \begin{equation*}
    1000 (1 + 12 \cdot \sfrac 2 {100}) = 1240.
  \end{equation*}

  The same account with a compound interest of \( 2\% \) will earn a bit more - about \( 268\$ \):
  \begin{equation*}
    1000 (1 + \sfrac 2 {100})^{12} \approx 1268.24.
  \end{equation*}

  Over the course of ten years, however, simple interest will earn a total of \( 2400\$ \), while compound interest will earn \( \approx 9765\$ \).

  The difference between linear and exponential growth appears staggering in a real world situation even though the difference may not be very noticeable short-term.
\end{remark}

\begin{definition}\label{def:harmonic_progression}
  The \term{harmonic progression} with \term{base} \( a_0 \) and \term{difference} \( d \) is the sequence
  \begin{equation}\label{eq:def:harmonic_progression}
    a_k \coloneqq \frac 1 {a_0 + kd}.
  \end{equation}

  That is, each term is the reciprocal of the corresponding term in an \hyperref[def:arithmetic_progression]{arithmetic progression} with the same base and difference. In order for \eqref{eq:def:harmonic_progression} to be well-defined, either
  \begin{itemize}
    \item \( d = 0 \) and \( a_0 \neq 0 \), which turns \eqref{eq:def:harmonic_progression} into the constant sequence \( \seq{ \sfrac 1 {a_0} }_{k=0}^\infty \).
    \item \( d \neq 0 \), in which case
    \begin{equation*}
      a_k = \frac d {\sfrac {a_0} d + k}.
    \end{equation*}

    Thus, if \( d \neq 0 \), \( \ifrac {a_0} d \) must not be a negative integer unless we are satisfied with only the first \( -\ifrac {a_0} d \) terms of the progression existing.
  \end{itemize}

  Furthermore, the series may only start at \( k = 0 \) if \( a_0 \neq 0 \).

  For series related to harmonic progressions, see \fullref{ex:harmonic_series}
\end{definition}

\begin{remark}\label{rem:harmonic_progression_recursive_form}
  Unlike \fullref{def:arithmetic_progression} and \fullref{def:geometric_progression}, we have defined the harmonic progressions via closed-form expressions. Indeed, the equivalent inductive definition is more awkward to work with:
  \begin{equation*}
    a_k \coloneqq \begin{cases}
      \ifrac 1 {a_0},                    & k = 0, \text{ only defined if } a_0 \neq 0, \\
      \ifrac 1 {(a_0 + d)},                & k = 1,                                      \\
      \ifrac 1 {(\sfrac 1 {a_{k-1}} + d)}, & k > 0.
    \end{cases}
  \end{equation*}
\end{remark}

\subsection{Abstract simplicial complexes}\label{subsec:abstract_simplicial_complexes}

\begin{definition}\label{def:abstract_simplicial_complex}\mcite\cite[def. 2.1]{Carlsson2009}
  An \term{abstract simplicial complex} is a pair \( (V, \Sigma) \), where
  \begin{itemize}
    \item \( V \) is a finite set
    \item \( \Sigma \subseteq \pow(V) \) such that \( \sigma \in \Sigma \) and \( \tau \subseteq \sigma \) implies \( \tau \in \Sigma \).
  \end{itemize}

  Due to the equivalence with families of simplices (see \fullref{thm:abstract_simplicial_complex_iff_simplicial_complex}), elements of \( V \) are called \term{vertices} and elements \( \Sigma \) are called \term{simplices}.

  Denote by \( \Sigma_k \) the family of all simplices \( S \) with \( \abs{S} = k + 1 \), that is, all \term{\( k \)-simplices}.
\end{definition}

\begin{definition}\label{def:simplicial_complex}
  A \term{simplicial complex} in \( \BbbR^n \) is a set \( K \) of \hyperref[def:simplex]{simplices}, such that
  \begin{itemize}
    \item For any simplex \( S \in K \), any face of \( S \) is also in \( K \).
    \item The intersection of any two simplices \( S_1 \) and \( S_2 \) of \( K \) is either empty or is a face of both \( S_1 \) and \( S_2 \).
  \end{itemize}

  Denote by \( K_k \) the family of all \( k \)-simplices in \( K \).
\end{definition}

\begin{proposition}\label{thm:abstract_simplicial_complex_iff_simplicial_complex}
  Let \( (V, \Sigma) \) be an abstract simplicial \hyperref[def:abstract_simplicial_complex]{complex} and let \( v_1 < \ldots < v_n \) be an ordering of elements of \( V \). Define the map \( E(v_k) \coloneqq e_k, k = 1, \ldots, n \), where \( e_k \) are the corresponding basis vectors in \( \BbbR^n \). Then the set
  \begin{equation*}
    K \coloneqq \{ \conv E(S) \colon S \in \Sigma \}
  \end{equation*}
  is a simplicial \hyperref[def:simplicial_complex]{complex}.

  Conversely, if \( K \) is a simplicial complex in \( \BbbR^n \), denote by \( V \) all \( 0 \)-simplices (vertices) in \( K \) and
  \begin{equation*}
    \Sigma \coloneqq \{ U \subseteq V \colon \conv U \in K \}.
  \end{equation*}

  Then \( (V, \Sigma) \) is an abstract simplicial complex.
\end{proposition}

\begin{definition}\label{def:group_of_chains}\mcite\cite[262]{Carlsson2009}
  Let \( X = (V, \Sigma) \) be an abstract simplicial complex. For each nonnegative integer \( k \), define the corresponding \term{group of \( k \)-chains} \( C_k(X) \) as the \hyperref[def:free_abelian_group]{free abelian group} generated by the \( k \)-simplices \( \Sigma_k \).

  Let \( v_1 < \ldots < v_n \) be a \hyperref[def:totally_ordered_set]{total order} on the vertex set \( V \). We define the functions
  \begin{balign*}
     & d_i: \Sigma \to \Sigma                  \\
     & d_i(S) \coloneqq S \setminus \{ v_i \}.
  \end{balign*}
  and the homomorphisms
  \begin{balign*}
     & \partial_k: C_k(X) \to C_{k-1}(X)                  \\
     & \partial_k(S) \coloneqq \sum_{i=1}^k (-1)^i d_i(S)
  \end{balign*}

  We can use the induced ordering to represent the operators \( \partial_k \) via matrices.
\end{definition}

\begin{proposition}\label{thm:abstract_simplicial_chain_complex}
  In an abstract simplicial complex \( X = (V, \Sigma) \), the homomorphisms \( \partial_k: C_k(X) \to C_{k-1}(X) \) form a chain \hyperref[def:chain_complex]{complex}.
\end{proposition}

\subsection{Graphs}\label{subsec:graphs}

\begin{remark}\label{rem:directed_and_undirected_graphs}
  Unfortunately, the word \enquote{graph} has at least three popular meanings withing mathematics:
  \begin{itemize}
    \item Graphs of functions (see \fullref{def:function/graph})
    \item Directed graphs (see \fullref{def:directed_graph})
    \item Undirected graphs (see \fullref{def:undirected_graph})
  \end{itemize}

  Graphs of functions are different enough from the other two notions to not cause any confusion, however it is often not clear from the context whether \enquote{graph} refers to directed or undirected graphs. Both are formalisms corresponding to dots in the plane connected with (directed or undirected) lines (see \fullref{ex:directed_graph}).

  We define undirected graphs as a special case of directed graphs. This approach makes some definitions more awkward. In programming, however, implementing undirected graphs as a special case of directed graphs is often more versatile (see \cite[sec. 5.4]{Erickson2019} and \cite[ch. 1, sec. 2.4]{GondranMinoux1984Graphs}).
\end{remark}

\begin{definition}\label{def:directed_graph}\mcite[ch. 1, sec. 1.1]{GondranMinoux1984Graphs}
  A \term{directed graph} \( G = (V, E) \) is a pair where
  \begin{itemize}
    \item \( V \) is a set, whose elements are called \term{vertices} or \term{nodes}.
    \item \( E \subseteq V^2 \) is a \hyperref[def:relation]{relation} over \( V \), whose elements are called \term{arcs}. If \( u, v \in V \) are vertices and \( e = (u, v) \) is an arc, we say that \( u \) is the \term{head} or \term{initial endpoint} and \( v \) is the \term{tail} or \term{terminal endpoint} of the arc. We denote \( u = h(e) \) and \( v = t(e) \).
  \end{itemize}

  For readability, we use the infix notation \( u \to v \) rather than \( (u, v) \) for arcs.

  We say that
  \begin{thmenum}
    \thmitem{def:directed_graph/order} the number
    \begin{equation*}
      \ord G \coloneqq \card V
    \end{equation*}
    is the \term{order} of the graph \( G = (V, E) \).
    \thmitem{def:directed_graph/empty} a graph \( G = (V, E) \) is \term{empty} if \( E = \varnothing \).
    \thmitem{def:directed_graph/subgraph} the graph \( G' = (V', E') \) is a \term{subgraph} of \( G = (V, E) \) if we have both \( V' \subseteq V \) and \( E' \subseteq E \).
    \thmitem{def:directed_graph/loop} the arc \( u \to v \) is a \term{loop} if \( u = v \).
    \thmitem{def:directed_graph/simple}\mcite[ch. 1, sec. 1.3]{GondranMinoux1984Graphs}the graph \( G \) is \term{simple} if it has no loops.
  \end{thmenum}
\end{definition}

\medskip

\begin{definition}\label{def:graph_matrices}
  Let \( G = (V, E) \) be a simple finite directed graph.
  \begin{thmenum}
    \thmitem{def:graph_matrices/incidence}\mcite[ch. 1, sec. 2.1]{GondranMinoux1984Graphs} The \term{incidence matrix} \( I = \{ a_{ve} \}_{v \in V, e \in E} \) of \( G \) is defined as
    \begin{equation*}
      a_{ve} \coloneqq \begin{cases}
        1,  & v \text{ is the head of } e \\
        -1, & v \text{ is the tail of } e \\
        0,  & \text{otherwise}
      \end{cases}
    \end{equation*}

    \thmitem{def:graph_matrices/adjacency}\mcite[ch. 1, sec. 2.3]{GondranMinoux1984Graphs} The \term{adjacency matrix} \( A = \{ a_{ve} \}_{u, v \in V} \) of \( G \) is defined as
    \begin{equation*}
      a_{uv} \coloneqq \begin{cases}
        1, & (u, v) \in E     \\
        0, & \text{otherwise}
      \end{cases}
    \end{equation*}
  \end{thmenum}
\end{definition}

\begin{example}\label{ex:directed_graph}
  Consider the directed graph \( G = (V, E) \), corresponding to the following drawing
  \begin{alignedeq}\label{ex:directed_graph/embedding}
    \text{\todo{Add diagram}}\iffalse\begin{mplibcode}
      u := 0.6cm;

      beginfig(1);
      input metapost/graphs;
      lax_bboxmargin := 2pt;

      v1 := thelabel("$a$", origin);
      v2 := thelabel("$b$", (2, 2) scaled u);
      v3 := thelabel("$c$", (5, 2) scaled u);
      v4 := thelabel("$d$", (2, -2) scaled u);
      v5 := thelabel("$e$", (5, -2) scaled u);
      v6 := thelabel("$f$", (7, 0) scaled u);

      a1 := straight_arc(v1, v2);
      a2 := straight_arc(v1, v4);
      a3 := straight_arc(v2, v3);
      a5 := straight_arc(v4, v3);
      a4 := straight_arc(v4, v5);
      a6 := straight_arc(v3, v6);
      a7 := straight_arc(v5, v6);

      draw_vertices(v);
      draw_arcs(a);

      label.ulft("$1$", straight_arc_midpoint of a1);
      label.llft("$2$", straight_arc_midpoint of a2);
      label.top("$3$", straight_arc_midpoint of a3);
      label.bot("$5$", straight_arc_midpoint of a4);
      label.ulft("$4$", straight_arc_midpoint of a5);
      label.urt("$6$", straight_arc_midpoint of a6);
      label.lrt("$7$", straight_arc_midpoint of a7);
      endfig;
    \end{mplibcode}\fi
  \end{alignedeq}

  The vertices are labeled \( a \) through \( f \) and the arcs are labeled 1 through 7.

  The corresponding incidence matrix is
  \begin{balign*}
    \bordermatrix{
      & 1  & 2  & 3  & 4  & 5  & 6  & 7  \cr
    a & 1  & 1  &    &    &    &    & \cr
    b & -1 &    & 1  &    &    &    & \cr
    c &    &    & -1 & -1 &    & 1  & \cr
    d &    & -1 &    & 1  & 1  &    & \cr
    e &    &    &    &    & -1 &    & 1  \cr
    f &    &    &    &    &    & -1 & -1
    }
  \end{balign*}
  and the adjacency matrix is
  \begin{balign*}
    \bordermatrix{
      & a & b & c & d & e & f  \cr
    a &   & 1 &   & 1 &   & \cr
    b &   &   & 1 &   &   & \cr
    c &   &   &   &   &   & 1  \cr
    d &   &   & 1 &   & 1 & \cr
    e &   &   &   &   &   & 1  \cr
    f &   &   &   &   &   &
    }
  \end{balign*}
\end{example}

\begin{definition}\label{def:graph_paths}
  Let \( G = (V, E) \) be a directed graph.

  \begin{thmenum}
    \thmitem{def:graph_paths/adjacent_vertices} Two vertices \( u \)  and \( v \)  are called \term{adjacent} if there exists an arc from \( u \)  to \( v \) .

    \thmitem{def:graph_paths/adjacent_arcs}\mcite[ch. 1, sec. 1.4]{GondranMinoux1984Graphs} Two arcs are called \term{adjacent} if they have a common endpoint. Thus the first three pairs of arcs are adjacent and the fourth is not (assuming all vertices are distinct):
    \begin{enumerate}
      \item \( u \to v \) and \( v \to w \)
      \item \( u \to v \) and \( u \to w \)
      \item \( u \to w \) and \( v \to w \)
      \item \( u \to u' \) and \( v \to v' \)
    \end{enumerate}

    \thmitem{def:graph_paths/undirected_path}\mcite[ch. 1, sec. 3.1]{GondranMinoux1984Graphs} An \term{undirected path} or \term{chain} is a sequence of distinct arcs
    \begin{equation*}
      p \coloneqq ( e_1, \ldots, e_n ),
    \end{equation*}
    such that any two consecutive arcs are adjacent, that is, the arcs \( e_i \) and \( e_{i+1} \) are adjacent for \( i = 1, \ldots, n - 1 \). We say that \( u \) is the \term{head} of \( p \) if it is an endpoint of \( e_1 \) but not \( e_2 \) and that \( v \) is the \term{tail} of \( p \) if it is an endpoint of \( e_n \) but not \( e_{n-1} \). The number \( n \) is called the \term{length} of the path and is denoted by \( \len p \).

    In the graph \fullref{ex:directed_graph/embedding}, \( (1, 2, 3) \) is a path with head \( a \), tail \( f \) and length 3, while \( (3, 4) \) is a path with head \( b \), tail \( d \) and length 2.

    If all vertices in a path are distinct, that is, if there are exactly \( n + 1 \) distinct vertices, we say that the path is \term{simple}.

    Some authors (e.g. \cite[sec. 5.2]{Erickson2019}) call undirected paths \term{walks} and reserve the term \enquote{path} for simple undirected paths.

    If the head and the tail of a path coincide, we say that the path is a \term{cycle}.

    A \term{simple cycle} is a cycle where all non-endpoint vertices are distinct.

    \thmitem{def:graph_paths/directed_path}\mcite[ch. 1, sec. 3.2]{GondranMinoux1984Graphs}If the tail of each non-endpoint arc in a path coincides with the head of the next arc, we say that the path is a \term{directed path}.

    In the graph \fullref{ex:directed_graph/embedding}, \( (1, 2, 3) \) is a directed path, while \( (3, 4) \) is not.

    A directed cycle is also called a \term{circuit}.

    \thmitem{def:graph_paths/dag}\mcite[231]{Erickson2019}A \term{directed acyclic graph} or \term{dag} is a directed graph without directed cycles.

    \medskip

    \thmitem{def:graph_paths/eulerian_path}\mcite[ch. 8, sec. 1.1]{GondranMinoux1984Graphs}A path (either directed or undirected) is called \term{Eulerian} if it contains every arc of the graph exactly once, that is, the path induces an ordering of the arcs. A graph with an Eulerian cycle is called an \term{Eulerian graph}.

    \thmitem{def:graph_paths/hamiltonian_path}\mcite[ch. 8, sec. 3.1]{GondranMinoux1984Graphs}A simple path (either directed or undirected) is called \term{Hamiltonian} if it contains every vertex of the graph, that is, it induces an ordering of the vertices. A graph with a Hamiltonian cycle is called a \term{Hamiltonian graph}.
  \end{thmenum}
\end{definition}

\begin{definition}\label{def:graph_incidence}
  Let \( G = (V, E) \) be a directed graph. We define the multivalued \hyperref[def:function/multivalued]{functions} with signature \( \pow(V) \rightrightarrows E \):
  \begin{balign*}
     & w^+(A) \coloneqq \{ (u, v) \in E \colon u \in A \} \\
     & w^-(A) \coloneqq \{ (u, v) \in E \colon v \in A \} \\
     & w(A) \coloneqq w^+(A) \cup w^-(A).
  \end{balign*}

  That is, for a set \( A \) of vertices, \( w^+(A) \) gives us the set of arcs whose head is in \( A \), \( w^-(A) \) gives us the set of arcs whose tail is in \( A \) and \( w(A) \) gives us all arcs with at least one endpoint in \( A \).

  \begin{thmenum}
    \thmitem{def:graph_incidence/incident_arcs} The arc \( e \) is said to be \term{incident} with the vertex \( v \) if \( e \in w(v) \), that is, if \( v \) is an endpoint of \( e \).

    \thmitem{def:graph_incidence/degree}\mcite[ch. 1, sec. 1.4]{GondranMinoux1984Graphs}Given a vertex \( v \), the \term{degree} \( d(v) \) (resp. \term{in-degree} \( d^+(v) \) and \term{out-degree} \( d^-(v) \)) of the vertex is defined as
    \begin{equation*}
      d(v) \coloneqq \card w(v).
    \end{equation*}

    The degree of the graph is then defined as
    \begin{equation*}
      d(G) \coloneqq \max_{v \in V} d(v).
    \end{equation*}
  \end{thmenum}
\end{definition}

\begin{definition}\label{def:graph_connectivity}
  Let \( G = (V, E) \) be a directed graph.

  \begin{thmenum}
    \thmitem{def:graph_connectivity/reachable_vertices}The vertex \( v \) is \term{reachable} from the vertex \( u \) if there exists a directed \hyperref[def:graph_paths/directed_path]{path} from \( u \) to \( v \).

    \thmitem{def:graph_connectivity/strongly_connected_graph}\mcite[ch. 1, sec. 3.5]{GondranMinoux1984Graphs}The graph \( G \) is \term{strongly connected} if every pair of distinct vertices are reachable, that is, if there exists a directed path between every pair of distinct vertices.

    \thmitem{def:graph_connectivity/weakly_connected_graph}\mcite[ch. 1, sec. 3.3]{GondranMinoux1984Graphs}The graph \( G \) is \term{weakly connected} if there exists an undirected path between every pair of distinct vertices.

    \thmitem{def:graph_connectivity/connected_component}\mcite[ch. 1, sec. 3.3 \\ ch. 1, sec. 3.5]{GondranMinoux1984Graphs}The subgraph \( G' \) of \( G \) is a \term{connected component} (resp. \term{strongly connected component}) if it is connected (resp. strongly connected) and there exists no connected (resp. strongly connected) subgraph of \( G \) that properly contains \( G' \).

    \thmitem{def:graph_connectivity/connectivity_number}\mcite[ch. 1, sec. 3.3 \\ ch. 1, sec. 3.5]{GondranMinoux1984Graphs}\( G \) has \term{connectivity number} (resp. \term{strong connectivity number}) \( n \) if it has \( n \) connected (resp. strongly connected) components.

    \thmitem{def:graph_connectivity/cut}\mcite[ch. 1, sec. 3.4]{GondranMinoux1984Graphs}The set \( U \subseteq V \) of vertices is a \term{cut} (resp. \term{directed cut}) if removing \( T \) from the graph would increase the connectivity number (resp. strong connectivity number) of the graph.

    \thmitem{def:graph_connectivity/cocycle}\mcite[ch. 1, sec. 4.4]{GondranMinoux1984Graphs}The set \( F \subseteq E \) of arcs is a \term{cocycle} (resp. \term{cocircuit}) if there exists a set \( U \subseteq V \) of vertices such that \( F = w(T) \) (resp. \( F \in \{ w^+(T), w^-(T) \} \)).
  \end{thmenum}
\end{definition}

\begin{definition}\label{def:graph_adjacency}
  Let \( G = (V, E) \) be an undirected graph.

  \begin{thmenum}
    \thmitem{def:graph_adjacency/clique}\mcite[ch. 1, sec. 1.4]{GondranMinoux1984Graphs}The set \( U \subseteq V \) is called a \term{clique} if all two vertices in \( U \) are adjacent.

    \medskip

    \thmitem{def:graph_adjacency/complete_graph}\mcite[ch. 1, sec. 1.4]{GondranMinoux1984Graphs}If \( V \) itself is a clique, we say that \( G \) is a \term{complete graph}.

    \medskip

    \thmitem{def:graph_adjacency/anticlique}\mcite[120]{Erickson2019}Dually, \( U \subseteq V \) is an \term{anticlique} or \term{independent set} of vertices if no two vertices in \( U \) are adjacent.

    \thmitem{def:graph_adjacency/matching}\mcite[ch. 5, exer. 11]{GondranMinoux1984Graphs}The set \( F \subseteq E \) of arcs is a \term{matching} or \term{independent set} of arcs if no two arcs in \( F \) are adjacent.

    \thmitem{def:graph_adjacency/bipartite_graph}\mcite[7]{GondranMinoux1984Graphs}The graph is called \term{bipartite} if there exists a partition \( \{ A, B \} \) of \( V \) such that both \( A \) and \( B \) are anticliques. We also write \( G = (A, B, E) \).

    If \( G \) is undirected and if for every pair of vertices \( a \in A, b \in B \) there is an arc \( a \to b \), we say that \( G \) is a complete bipartite graph.
  \end{thmenum}
\end{definition}

\begin{definition}\label{def:undirected_graph}
  An \term{undirected graph} is a directed graph \( G = (V, E) \) where \( E \) is a symmetric relation (see \fullref{rem:directed_and_undirected_graphs}). When dealing with undirected graphs, instead of speaking about the arcs \( u \to v \) and \( v \to u \), we speak about the \term{edges} \( \{ u, v \} \). Thus, we can also define an undirected graph to be the tuple \( G = (V, E) \), where
  \begin{itemize}
    \item \( V \) is a set of \term{vertices}.
    \item \( E \subseteq \pow(V) \) is a family of unordered pairs of vertices, that is, singletons and two-element sets.
  \end{itemize}

  Defining undirected graphs as a special case of directed graphs allows us to somewhat unify their study and usage, however we need to keep in mind some remarks:
  \begin{itemize}
    \item All paths are directed and hence we only speak of \term{paths} and \term{cycles}. It is necessary, however, to not allow consecutive arcs to represent the same edge, that is, we must treat paths as sequences of edges rather than sequences of arcs.

          Otherwise, since every edge corresponds to two \enquote{inverse} arcs, for all adjacent vertices \( u \) and \( v \) the path \( (u \to v, v \to u) \) is a cycle and hence all undirected graphs would be cyclic.

    \item The incidence matrix is usually defined as \( I = \{ a_{ve} \}_{v \in V, e \in E} \), where
          \begin{equation*}
            a_{ve} \coloneqq \begin{cases}
              1, & v \text{ is an endpoint of } e \\
              0, & \text{otherwise}
            \end{cases}
          \end{equation*}

    \item The adjacency matrix is symmetric if and only if the graph is undirected.

    \item If \( G \) contains to cycles, we say that it is \term{acyclic}.

    \item The notions of connectedness and strong connectedness coincide. Connected acyclic graphs are called \term{trees}. Undirected acyclic graphs are often called \term{forests} since the connected components are trees.

    \item The in-degrees and out-degrees of vertices coincide with the degree. Also, \( w(A) = w^+(A) = w^-(A) \).
  \end{itemize}
\end{definition}

\begin{example}\label{ex:petersen_graph}\mcite[347]{GondranMinoux1984Graphs}
  The Petersen graph
  \begin{alignedeq}\label{ex:petersen_graph/embedding}
    \text{\todo{Add diagram}}\iffalse\begin{mplibcode}
      u := 1cm;

      beginfig(1);
      input metapost/graphs;
      lax_bboxmargin := 2pt;

      for i = 1 upto 5:
      v[i] := thelabel("$\bullet$", dir(18 + i * 72) scaled u);
      v[5 + i] := thelabel("$\bullet$", dir(18 + i * 72) scaled 2u);
      endfor;

      for i = 1 upto 4:
      a[i] := straight_arc(v[5 + i], v[5 + i + 1]);
      a[5 + i] := straight_arc(v[i], v[5 + i]);
      endfor;

      a5 := straight_arc(v10, v6);
      a10 := straight_arc(v5, v10);

      a11 := straight_arc(v1, v3);
      a12 := straight_arc(v1, v4);
      a13 := straight_arc(v2, v4);
      a14 := straight_arc(v2, v5);
      a15 := straight_arc(v3, v5);

      draw_vertices(v);
      draw_edges(a);
      endfig;
    \end{mplibcode}\fi
  \end{alignedeq}
  is connected, but not acyclic nor \hyperref[def:graph_paths/hamiltonian_path]{Hamiltonian}.
\end{example}

\begin{definition}\label{def:graph_homomorphism}
  Let \( G = (V, E) \) and \( G' = (V', E') \) be directed graphs. We say that the function \( f: V \to V' \) is a \term{graph homomorphism} if for every two vertices \( u, v \in V \),
  \begin{equation*}
    (u, v) \in E \text{ implies } (f(u), f(v)) \in E'.
  \end{equation*}

  The terminology from \fullref{def:morphism_invertibility} applies to graph homomorphisms because of the category \( \cat{Graph} \) of \hyperref[def:category_of_graphs]{graphs}.
\end{definition}

\begin{definition}\label{def:category_of_graphs}
  We denote by \( \cat{DGraph} \) the \hyperref[def:category]{category} where
  \begin{itemize}
    \item the \hyperref[def:set_zfc]{class} of objects is the class of all directed \hyperref[def:directed_graph]{graphs}.
    \item the morphisms between two graphs the homomorphisms between them, with morphism composition being the usual \hyperref[def:function/composition]{function composition}.
  \end{itemize}

  We denote by \( \cat{UGraph} \) the category of undirected graphs. Given the convention established in \fullref{rem:directed_and_undirected_graphs} about undirected graphs being a special case of directed graphs, we can regard \( \boldop{UGraph} \) as a subcategory of \( \boldop{DGraph} \).

  Both categories is \hyperref[def:category_cardinality]{locally small}.
\end{definition}

\subsection{Hypergraphs}\label{subsec:hypergraphs}

\begin{definition}\label{def:hypergraph}\mcite\cite[30]{GondranMinoux1984Graphs}
  Let \( V \) be a set and \( E \subseteq \pow(V) \) be a family of nonempty subsets of \( V \). We call the pair \( H = (V, E) \) a \term{hypergraph} if \( V = \bigcup E \).

  In analogy with undirected \hyperref[def:undirected_graph]{graphs}, we call elements of \( V \) \term{vertices} and elements of \( E \) \term{edges}. If the edges have cardinality at most \( 2 \), the hypergraph is an undirected graph.
\end{definition}

\begin{definition}\label{def:hypergraph_transversal}\mcite\cite[32]{GondranMinoux1984Graphs}
  Let \( H = (V, E) \) be a \hyperref[def:hypergraph]{hypergraph}. We say that the set \( T \subseteq V \) is a \term{transversal} if it intersects every edge of \( H \).
\end{definition}

\begin{example}\label{ex:trivial_hypergraph_transversal}
  Every hypergraph has at least one transversal since the vertex set it itself a transversal.
\end{example}

\begin{definition}\label{def:minimal_hypergraph_transversal}
  A transversal \( T \) of the hypergraph \( H = (V, E) \) is said to be \term{minimal} if any of the following equivalent conditions hold:
  \begin{thmenum}
    \ilabel{def:minimal_hypergraph_transversal/order} \( T \) is a minimal \hyperref[def:preordered_set/maximal_minimal_element]{element} under \hyperref[rem:subset_and_membership_relations]{set inclusion} in the set of all transversals of \( H \).
    \ilabel{def:minimal_hypergraph_transversal/singleton} for every vertex \( x \) in \( T \) there exists an edge \( E_x \in E \) such that
    \begin{equation*}
      T \cap E_x = \{ x \}.
    \end{equation*}
  \end{thmenum}
\end{definition}
\begin{proof}
  \ImplicationSubProof{def:minimal_hypergraph_transversal/order}{def:minimal_hypergraph_transversal/singleton} Let \( T \) be minimal under inclusion among all transversals. Fix \( x \in T \). Since \( T \) is minimal, the set \( T \setminus \{ x \} \) is not transversal. So there exists an edge \( E_x \in E \) such that
  \begin{equation*}
    \varnothing = (T \setminus \{ x \}) \cap E_x = (T \cap E_x) \setminus \{ x \}.
  \end{equation*}

  Now since \( T \) is a transversal for \( H \), \( T \cap E_x \) is nonempty and thus
  \begin{equation*}
    T \cap E_x = \{ x \}.
  \end{equation*}

  \ImplicationSubProof{def:minimal_hypergraph_transversal/singleton}{def:minimal_hypergraph_transversal/order} Now suppose that for every \( x \in T \) there exists an edge \( E_x \in \mathcal{F} \) such that \( T \cap E_x = \{ x \} \).

  Assume\LEM that \( T \) is not minimal and let \( y \in T \) be such that \( T \setminus \{ y \} \) is a transversal. But our assertion gives us an edge \( E_y \in \mathcal{F} \) such that \( T \cap E_y = \{ y \} \). Clearly the set \( T \setminus \{ y \} \) cannot be a transversal of \( H \) since
  \begin{equation*}
    (T \setminus \{ y \}) \cap E_y = \varnothing.
  \end{equation*}

  This contradiction proves that \( T \) is minimal under set inclusion.
\end{proof}

\begin{example}\label{ex:no_minimal_set_transversal}
  Let
  \begin{equation*}
    X_n \coloneqq \{ n, n + 1, n + 2, \ldots \}, n \in \BbbZ_{>0}
  \end{equation*}
  and \( H \coloneqq (\BbbZ_{>0}, \{ X_n \colon n \in \BbbZ_{>0} \}) \).

  The hypergraph \( H \) obviously has a transversal, e.g. \( \BbbZ_{>0} \), but it does not have a minimal transversal.

  To see this, assume that \( T \) is a minimal transversal. Since the natural numbers are well-\hyperref[thm:natural_numbers_are_well_ordered]{ordered}, \( T \) has a minimum. Let \( n_0 \coloneqq \min T \).

  But \( T \setminus \{ n_0 \} \) is also a transversal because each \( X_n \) intersects \( T \) at infinitely many points besides \( n_0 \).
\end{example}

\begin{proposition}\label{thm:finite_hypergraphs_have_minimal_transversal}
  Every finite hypergraph has a minimal transversal.
\end{proposition}
\begin{proof}
  A minimal transversal must exist\LEM because the vertex set is a transversal and we can only remove finitely many elements from it.
\end{proof}


\printindex
\printbibliography

\end{document}
