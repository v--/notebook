\documentclass{notebook}

% Bibliography
\addbibresource{bib/books.bib}
\addbibresource{bib/papers.bib}
\addbibresource{bib/articles.bib}
\addbibresource{bib/nlab.bib}
\addbibresource{bib/proofwiki.bib}
\addbibresource{bib/mathcounterexamples.bib}
\addbibresource{bib/mathse.bib}
\addbibresource{bib/mathof.bib}
\addbibresource{bib/misc.bib}
\addbibresource{bib/code.bib}

% Document
\title{Notebook}
\subtitle{\boldtt{\url{https://github.com/v--/notebook}}}
\author{Ianis Vasilev}

% Failsafe means "use defauls if no file is found", see common/git_commit_info.sty
\GitCommitInfoReadFailsafe {git-commit-info}
{
  \date
  {
    \normalsize
    \textbf{Date:} \GitCommitInfoDate \\
    \textbf{Commit:} \texttt{\GitCommitInfoHash}
  }
}

\begin{document}

\maketitle

\begin{abstract}
  This ever-expanding document started as a set of study notes and exercises and gradually outgrew itself to become more encyclopedic. Having all these notes in one place is quite helpful for both expressing my own thoughts clearly and for later reference. It is also helpful for tracking connections between seemingly unrelated concepts --- the entire document is inter-hyperlinked. Even though I aim at understanding every concept in the way it is meant to be used, most concepts are presented insomuch as they are relevant to probability and optimization.

  Since these are study notes, they will naturally have a lot of errors, so read them with caution. The document claims no referential nor pedagogical value. Everything is written at the level of abstraction I am comfortable with. Furthermore, some sections in the document are much more polished than others. Feel free to contact me if something in this document happens to distress you.

  I tried putting citations on as many things as possible. The citations themselves are usually put in the left margin. If there is no citation on a definition or theorem, that means that I have either recalled it from memory or discovered it on my own. Many of the unoriginal definitions and theorems are restated. The simple proofs are mostly original, and the difficult ones are, often loosely, based on proofs from the places cited. Some proofs simply state \enquote{Trivial.} in order to distinguish themselves from the proofs that are omitted for other reasons.

  See the remark in the introduction of the sections on \hyperref[sec:mathematical_logic]{mathematical logic} and \hyperref[sec:set_theory]{set theory} for clarification regarding seemingly arbitrary conventions.
\end{abstract}

\newpage
\addcontentsline{toc}{section}{Contents}
\tableofcontents
\newpage

% Numbers
\import{src}{natural_numbers.tex}
\import{src}{integers.tex}
\import{src}{rational_numbers.tex}
\import{src}{real_numbers.tex}
\import{src}{complex_numbers.tex}

% Real analysis
\import{src}{topology_of_euclidean_spaces.tex}
\import{src}{real_valued_functions.tex}
\import{src}{real_convergence.tex}
\import{src}{real_differentiability.tex}
\import{src}{real_series.tex}
\import{src}{riemann_integration.tex}
\import{src}{line_integrals.tex}
\import{src}{total_variation.tex}

% Complex analysis
\import{src}{complex_functions.tex}
\import{src}{series.tex}
\import{src}{power_series.tex}
\import{src}{trigonometric_functions.tex}
\import{src}{exponential_function.tex}
\import{src}{trigonometric_polynomials.tex}
\import{src}{special_functions.tex}
\import{src}{norms.tex}

% Functional analysis
\import{src}{topological_groups.tex}
\import{src}{topological_vector_spaces.tex}
\import{src}{hahn_banach.tex}
\import{src}{frechet_spaces.tex}
\import{src}{banach_spaces.tex}
\import{src}{hilbert_spaces.tex}
\import{src}{asplund_spaces.tex}
\import{src}{minkowski_functionals.tex}
\import{src}{dentable_sets.tex}
\import{src}{differentiability.tex}
\import{src}{banach_space_interpolation.tex}

% Nonsmooth analysis
\import{src}{nonsmooth_derivatives.tex}
\import{src}{convex_functions.tex}
\import{src}{subdifferentials.tex}
\import{src}{clarke_gradients.tex}

% Approximation theory
\import{src}{lagrange_polynomials.tex}
\import{src}{bernstein_inequalities.tex}

% Differential equations
\import{src}{ordinary_differential_equations.tex}

% General topology
\import{src}{topological_spaces.tex}
\import{src}{topological_nets.tex}
\import{src}{function_convergence.tex}
\import{src}{topological_continuity.tex}
\import{src}{initial_final_topologies.tex}
\import{src}{separation_axioms.tex}
\import{src}{connected_spaces.tex}
\import{src}{compact_spaces.tex}
\import{src}{baire_spaces.tex}
\import{src}{uniform_spaces.tex}

% Metric spaces
\import{src}{metric_topology.tex}
\import{src}{complete_metric_spaces.tex}
\import{src}{hausdorff_distance.tex}
\import{src}{totally_bounded_sets.tex}
\import{src}{noncompactness_measures.tex}
\import{src}{lipschitz_continuity.tex}
\import{src}{function_oscillation.tex}

% Geometry
\import{src}{affine_coordinate_systems.tex}
\import{src}{vector_space_geometry.tex}
\import{src}{analytic_geometry_in_the_plane.tex}
\import{src}{manifolds.tex}
\import{src}{affine_varieties.tex}

% Group theory
\import{src}{pointed_sets.tex}
\import{src}{magmas.tex}
\import{src}{unital_magmas.tex}
\import{src}{involutions.tex}
\import{src}{magma_ideals.tex}
\import{src}{monoid_actions.tex}
\import{src}{groups.tex}
\import{src}{group_presentations.tex}
\import{src}{group_actions.tex}
\import{src}{category_of_groups.tex}
\import{src}{abelian_groups.tex}

% Ring theory
\import{src}{rings.tex}
\import{src}{ring_ideals.tex}
\import{src}{fields.tex}
\import{src}{modules.tex}

% Commutative algebra
\import{src}{euclidean_division.tex}
\import{src}{polynomials.tex}
\import{src}{prime_ideals.tex}
\import{src}{localization.tex}
\import{src}{noetherian_rings.tex}

% Linear algebra
\import{src}{vector_spaces.tex}
\import{src}{algebraic_dual_spaces.tex}
\import{src}{matrices.tex}
\import{src}{diagonalization.tex}
\import{src}{bilinear_forms.tex}

\import{src}{mathematical_logic.tex}
\import{src}{formal_languages.tex}
\import{src}{boolean_functions.tex}
\import{src}{propositional_logic.tex}
\import{src}{first_order_logic.tex}
\import{src}{first_order_satisfiability.tex}
\import{src}{first_order_models.tex}
\import{src}{proof_derivation_systems.tex}
\import{src}{logical_theories.tex}

\import{src}{set_theory.tex}
\import{src}{naive_set_theory.tex}
\import{src}{relations.tex}
\import{src}{functions.tex}
\import{src}{zermelo_fraenkel_set_theory.tex}
\import{src}{well_ordered_sets.tex}
\import{src}{ordinals.tex}
\import{src}{cardinals.tex}
\import{src}{transfinite_arithmetic.tex}
\import{src}{von_neumanns_cumulative_hierarchy.tex}
\import{src}{grothendieck_universes.tex}

\import{src}{category_theory.tex}
\import{src}{categories.tex}
\import{src}{functors.tex}
\import{src}{category_equivalences.tex}
\import{src}{category_adjunctions.tex}
\import{src}{categorical_limits.tex}

\import{src}{order_theory.tex}
\import{src}{preordered_sets.tex}
\import{src}{partially_ordered_sets.tex}
\import{src}{totally_ordered_sets.tex}
\import{src}{lattices.tex}
\import{src}{boolean_algebras.tex}

\import{src}{combinatorics.tex}
\import{src}{enumerative_combinatorics.tex}
\import{src}{progressions.tex}
\import{src}{hypergraphs.tex}
\import{src}{undirected_graphs.tex}
\import{src}{quivers.tex}
\import{src}{trees.tex}
\import{src}{graph_embedding.tex}

\begin{sloppypar}
  \printbibliography
\end{sloppypar}

\end{document}
