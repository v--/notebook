\subsection{Propositional logic}\label{subsec:propositional_logic}

Propositional logic allows us to express basic relations between atomic propositions --- variables that can be true or false or, in the case of \hyperref[def:propositional_semantics]{non-classical semantics}, have some intermediate value.

\paragraph{Syntax of propositional logic}\hfill

There are different approaches to formalizing the syntax of propositional logic. We use the theory of formal grammars that we have developed in \fullref{subsec:formal_languages} and \fullref{subsec:syntax_trees}. This requires specific adjustments related to other treatments of propositional logic --- e.g. \incite[ch. 1]{Hinman2005}, \incite[pt. I]{Smullyan1995} and \incite[ch. 1]{КолмогоровДрагалин2006} --- which rely on a more informal treatment of syntax more akin to what can be achieved via \fullref{thm:least_fixed_point_recursion}. Our approach allows us to use powerful tools while treating the object logic in complete formality. \incite[45]{Mimram2020} also describes syntax via formal grammars, but does not utilize the theory of formal languages.

\begin{remark}\label{rem:grammar_rules_for_variables}
  We need to introduce conventions for identifiers --- i.e. names of variables, functions and so forth --- in order to be able to be able to express infinitely many variables with finitely many grammar rules. The variable names that we will actually use will be a very specific subset of the ones we allow formally. Consider the following \hyperref[def:formal_grammar/schema]{grammar schema}:
  \begin{bnf*}
    \bnfprod{Latin letter}           {\bnftsq{A} \bnfor \bnftsq{B} \bnfor \cdots \bnfor \bnftsq{Z} \bnfor \bnftsq{a} \bnfor \bnftsq{b} \bnfor \cdots \bnfor \bnftsq{z}}, \\
    \bnfprod{Latin string}           {\bnfpn{Latin letter} \bnfor \bnfpn{Latin letter} \bnfsp \bnfpn{Latin string}}, \\
    \bnfprod{Latin identifier}       {\bnfpn{Latin string} \bnfor \bnfpn{Latin string} \bnfsp \bnfpn{natural number}},
  \end{bnf*}
  where grammar rules for natural numbers can be taken from \fullref{rem:decimal_notation_grammar}.

  An identifier is a nonempty sequence of Latin letters, optionally followed by a number. Within the metalogic, we treat this number as a suffix and write it as a subscript, that is, we write \enquote{\( p_1 \)} instead of \enquote{\( p1 \)}.

  We use a number of related rules defined in an obvious way:
  \begin{itemize}
    \item In \fullref{def:untyped_lambda_term}, we use the nonterminal \( \bnfpn{Short Latin identifier} \), which consists of a single Latin letter with a numeric subscript.
    \item In \fullref{def:first_order_syntax} and \fullref{def:inference_rule}, we will use the nonterminal \( \bnfpn{Greek identifier} \) with analogous rules for denoting first-order variables.
    \item In \fullref{def:logical_context}, we have used distinct rules for small and capital Greek letters.
  \end{itemize}

  A benefit of this naming schema is that we can introduce a \hyperref[thm:def:well_ordered_set/lexicographic]{lexicographic well-ordering} on the set of identifiers. This ordering can help reduce ambiguity in \fullref{def:propositional_valuation/valuation_function}, \fullref{def:lambda_substitution} and \fullref{def:first_order_substitution/term_in_formula} (although, in practice, we will prefer the ordering \( \xi < \eta < \zeta \) resembling \( x < y < z \)). We use the following conventions:
  \begin{itemize}
    \item Individual digits are ordered as they are listed in \fullref{rem:decimal_notation_grammar}, i.e.
    \begin{equation*}
      0 < 1 < \cdots < 9.
    \end{equation*}

    \item Individual letters are ordered as they are listed above, i.e.
    \begin{equation*}
      A < B < \cdots < Z < a < b < \cdots < z.
    \end{equation*}

    \item Digits are smaller than letters.

    \item In the case of words of different lengths, if one word is a prefix of another, we assume that the shorter word is smaller, i.e.
    \begin{equation*}
      a < a_{13} < aa.
    \end{equation*}

    Comparing words of different lengths is discussed in \fullref{ex:def:lexicographic_order/heterogenous_words}.
  \end{itemize}
\end{remark}

\begin{definition}\label{def:propositional_alphabet}\mcite[4; 5]{Smullyan1995}
  The \hyperref[def:formal_language/alphabet]{alphabet} of \term[ru=логика высказываний (\cite[43]{КолмогоровДрагалин2006})]{propositional logic} consists of:

  \begin{thmenum}
    \thmitem{def:propositional_alphabet/constants} Two \term{propositional constants}:
    \begin{thmenum}
      \thmitem{def:propositional_alphabet/constants/verum}\mcite[70]{CitkinMuravitsky2021} \term{verum} \enquote{\( \syntop \)}.
      \thmitem{def:propositional_alphabet/constants/falsum}\mcite[70]{CitkinMuravitsky2021} \term{falsum} \enquote{\( \synbot \)}.
    \end{thmenum}

    \thmitem{def:propositional_alphabet/negation} \term[ru=отрицание (\cite[17]{КолмогоровДрагалин2006})]{Negation} \enquote{\( \synneg \)}.
    \thmitem{def:propositional_alphabet/connectives} The set \( \op*{Conn} \) of \term{binary propositional connectives}, namely
    \begin{thmenum}
      \thmitem{def:propositional_alphabet/connectives/disjunction} \term[ru=дизъюнкция (\cite[17]{КолмогоровДрагалин2006})]{Disjunction} \enquote{\( \synvee \)}, also known as \hyperref[thm:standard_boolean_functions]{\term{or}}.
      \thmitem{def:propositional_alphabet/connectives/conjunction} \term[ru=конъюнкция (\cite[17]{КолмогоровДрагалин2006})]{Conjunction} \enquote{\( \synwedge \)}, also known as \hyperref[thm:standard_boolean_functions]{\term{and}}.
      \thmitem{def:propositional_alphabet/connectives/conditional} \term{Conditional}\fnote{Note that \enquote{conditional} and \enquote{biconditional} are used nouns in this context, although we prefer the phrase \enquote{conditional formula}. We use these terms to avoid confusion with the same connectives in the metalogic, for example \enquote{A biconditional formula is equivalent \dots} could otherwise become \enquote{An equivalence is equivalent \dots}.} \( \synimplies \), also known as \term{if\ldots then} and \hyperref[thm:standard_boolean_functions]{\term[ru=импликация (\cite[17]{КолмогоровДрагалин2006})]{implication}}.
      \thmitem{def:propositional_alphabet/connectives/biconditional} \term{Biconditional} \enquote{\( \syniff \)}, also known as \term{if and only if} (\hyperref[thm:standard_boolean_functions]{\term{iff}}) and \term[ru=эквиваленция (\cite[17]{КолмогоровДрагалин2006})]{equivalence}.
    \end{thmenum}

    \thmitem{def:propositional_alphabet/parentheses} Parentheses \enquote{\( ( \)} and \enquote{\( ) \)} for defining the order of operations unambiguously.
  \end{thmenum}
\end{definition}
\begin{comments}
  \item If desired, we can utilize a smaller propositional language without losing its semantical properties. Important example are the \hyperref[def:cnf_and_dnf]{conjunctive normal forms} in \fullref{alg:perfect_cnf_and_dnf}, although similar constructions hold for other \hyperref[def:boolean_closure/complete]{complete sets of Boolean functions} like those from \fullref{thm:complete_sets_of_boolean_functions}.

  \item We place dots over the various symbols in order to highlight that they are merely symbols without semantics --- see \fullref{rem:mathematical_logic_conventions/metavariable_syntax} for a general discussion.

  \item \enquote{Conjunctio} and \enquote{disjunctio} are Latin words for \enquote{union} and \enquote{separation}, respectively. \enquote{Implicatio} is Latin for \enquote{entangled}.
\end{comments}

\begin{definition}\label{def:propositional_syntax}\mimprovised
  We will introduce a \hyperref[def:formal_grammar/schema]{grammar schema} whose rules and generated languages we will collectively call the \term[en=syntax (\cite[8]{Hinman2005})]{syntax} of propositional logic:
  \begin{thmenum}
    \thmitem{def:propositional_syntax/schema} Consider the following \hyperref[def:formal_grammar/schema]{grammar schema}:
    \begin{bnf*}
      \bnfprod{variable}    {\bnfpn{Latin identifier}} \\
      \bnfprod{connective}  {\bnftsq{\( \synvee \)} \bnfor \bnftsq{\( \synwedge \)} \bnfor \bnftsq{\( \synimplies \)}\bnfor \bnftsq{\( \syniff \)}} \\
      \bnfprod{formula}     {\bnftsq{\( \syntop \)} \bnfor \bnftsq{\( \synbot \)} \bnfor} \\
      \bnfmore              {\bnfpn{variable} \bnfor} \\
      \bnfmore              {\bnftsq{\( \synneg \)} \bnfsp \bnfpn{formula} \bnfor} \\
      \bnfmore              {\bnftsq{(} \bnfsp \bnfpn{formula} \bnfsp \bnfpn{connective} \bnfsp \bnfpn{formula} \bnfsp \bnftsq{)}}
    \end{bnf*}
    where we use the identifiers discussed in \fullref{rem:grammar_rules_for_variables}.

    \thmitem{def:propositional_syntax/prop} We denote by \( \op*{Prop} \) the set of all \term[ru=пропозициональные переменные (\cite[43]{КолмогоровДрагалин2006})]{propositional variables}, that is, all strings generated by the above grammar schema with starting nonterminal \( \bnfpn{variable} \).

    \thmitem{def:propositional_syntax/formula} Similarly, we denote by \( \op*{Form} \) the set of \term[ru=формула (\cite[43]{КолмогоровДрагалин2006})]{formulas}, that is, strings generated with starting nonterminal \( \bnfpn{formula} \).

    Within the metalanguage, we will denote abstract formulas via \( \varphi \), \( \psi \), \( \theta \) and other letters as discussed in \fullref{rem:mathematical_logic_conventions/greek_alphabet}. This convention will later lead us to a formal definition of formula schemas in \fullref{def:propositional_formula_schema}.

    We will also call \term{sentences} in relation to abstract \hyperref[def:logical_framework]{logical frameworks}. This contrasts with first-order logic, where only specific formulas are called sentences --- see \fullref{def:first_order_syntax/closed_formula}.

    \thmitem{def:propositional_syntax/language} By \enquote{language of propositional logic} we will mean the set \( \op*{Form} \) of formulas (and hence also the variables).

    \thmitem{def:propositional_syntax/fragment} By a \term[en=fragment (\cite[49]{Mimram2020})]{fragment} of propositional logic we will mean a subset of \( \op*{Form} \), most easily obtained by restricting which rules from the full schema are used. See \fullref{def:implicational_propositional_fragment}.
  \end{thmenum}
\end{definition}
\begin{comments}
  \item Various authors refer to the \enquote{propositional language} --- for example \incite[13]{Hinman2005} --- in relation to various syntactic constructions. These authors do not formally define this language, however. Since we use the mechanism of \hyperref[def:formal_language]{formal languages}, we prefer being more concrete about which syntactic notions we refer to.

  \item We implicitly associate with each propositional formula an \hyperref[con:abstract_syntax_tree]{abstract syntax tree} --- see \fullref{def:propositional_formula_ast}. The grammar of propositional formulas is unambiguous as shown via \fullref{thm:propositional_formulas_are_unambiguous}, which makes it possible to perform proofs via \fullref{thm:induction_on_syntax_trees}.

  \item If the root of the tree is a conjunction, we refer to the formula itself as a conjunction, and similarly for other propositional connectives.
\end{comments}

\begin{remark}\label{rem:object_language_dots}
  In the metalanguage we distinguish between the numeral \enquote{\( 1 \)} and the abstract integer \enquote{\( n \)}. Similarly, we want to distinguish between a literal representing a propositional variable and a metalingual variable denoting an unspecified propositional variable. Following \incite{Hinman2005}, we resolve this via dots --- we will denote the literal strings \enquote{\( p \)} and \enquote{\( q \)} as \enquote{\( \synp \)} and \enquote{\( \syn q \)}, while reserving the corresponding un-dotted symbols \( p \) and \( q \) for metalingual variables.

  A similar convention is used by \incite[rem. 2.1.3]{Hinman2005}.

  We will follow this convention for first-order variables, although we will avoid it for predicate symbols and functional symbols since we will almost exclusively use unambiguous symbols, and adding dots can even introduce clashes (for example in for the formulas in \fullref{subsec:lattices}).

  We will also follow it for the propositional alphabet, as well as for \hyperref[def:first_order_language/quantifiers]{first-order quantifiers} and \hyperref[def:first_order_language/equality]{first-order equality}, which helps us to distinguish the \hyperref[def:lattice/join]{join} \( x \vee y \) from the \hyperref[def:propositional_alphabet/connectives/conjunction]{conjunction} \( \syn p \synvee \syn q \) and the literal equality \( x = y \) from the atomic formula \( \syn x \syneq \syn y \).
\end{remark}

\begin{proposition}\label{thm:propositional_formulas_are_unambiguous}
  The \hyperref[def:formal_grammar]{grammar} of \hyperref[def:propositional_syntax/formula]{propositional formulas} is \hyperref[def:grammar_ambiguity]{unambiguous}.
\end{proposition}
\begin{proof}
  It is straightforward to adapt the proof from \fullref{ex:natural_number_arithmetic_grammar/unambiguous}.
\end{proof}

\begin{definition}\label{def:conditional_formula}
  For a given \hyperref[def:propositional_syntax/formula]{conditional formula} \( \varphi \synimplies \psi \), we introduce the following terminology:
  \begin{thmenum}
    \thmitem{def:conditional_formula/sufficient_condition}\mcite[def. I.4.1]{Эдельман1975} \( \varphi \) is a \term[ru=достаточное условие]{sufficient condition} for \( \psi \).

    \thmitem{def:conditional_formula/necessary_condition}\mcite[def. I.4.1]{Эдельман1975} \( \psi \) is a \term[ru=необходимое условие]{necessary condition} for \( \varphi \).

    \thmitem{def:conditional_formula/antecedent}\mcite[16]{Эдельман1975} \( \varphi \) is the \term[ru=посылка, en=antecedent (\cite[35]{Rosen1999})]{antecedent} of the conditional \( \varphi \synimplies \psi \).

    \thmitem{def:conditional_formula/consequent}\mcite[16]{Эдельман1975} \( \psi \) is the \term[ru=следствие, en=consequent (\cite[37]{Rosen1999})]{consequent} of \( \varphi \synimplies \psi \).

    \thmitem{def:conditional_formula/inverse}\mcite[def. I.4.3]{Эдельман1975} We call the formula \( \synneg \varphi \synimplies \synneg \psi \) the \term[ru=противоположная (теорема), en=inverse (\cite[49]{Rosen1999})]{inverse} of \( \varphi \).

    \thmitem{def:conditional_formula/converse}\mcite[13]{Kleene2002Logic} We call the formula \( \psi \synimplies \varphi \) the \term[ru=обратная (теорема) (\cite[def. I.4.2]{Эдельман1975})]{converse} of \( \varphi \).

    \thmitem{def:conditional_formula/contrapositive}\mcite[13]{Kleene2002Logic} The formula \( \synneg \psi \synimplies \synneg \varphi \) is the \term[ru=контрапозиция (\cite[26]{Эдельман1975})]{contrapositive} of \( \varphi \).
  \end{thmenum}
\end{definition}
\begin{comments}
  \item In \hyperref[def:classical_logic]{classical logic}, the contrapositive is \hyperref[def:semantic_equivalence]{equivalent} to the original formula due to \fullref{thm:classical_equivalences/contrapositive}.
\end{comments}

\begin{definition}\label{def:propositional_formula_ast}\mimprovised
  We implicitly associate with each propositional formula \( \varphi \) an \hyperref[con:abstract_syntax_tree]{abstract syntax tree} \( T(\varphi) \) defined as follows:
  \begin{thmenum}
    \thmitem{def:propositional_fomula_ast/atomic} If \( \varphi \) is a propositional constant or variable, let \( T(\varphi) \) be the singleton tree with value \( \varphi \).

    \thmitem{def:propositional_fomula_ast/negation} If \( \varphi = \synneg \psi \), assuming we have already built \( T(\psi) \), we obtain \( T(\varphi) \) by adding a new root with value \( \synneg \):
    \begin{equation*}
      \includegraphics[page=1]{output/def__propositional_formula_ast}
    \end{equation*}

    \thmitem{def:propositional_fomula_ast/connective} If \( \varphi = \psi \syncirc \theta \), assuming we have built \( T(\psi) \) and \( T(\chi) \), we obtain \( T(\varphi) \) by joining them by a new root with value \( \syncirc \):
    \begin{equation*}
      \includegraphics[page=2]{output/def__propositional_formula_ast}
    \end{equation*}
  \end{thmenum}
\end{definition}
\begin{comments}
  \item This is similar to \enquote{formation trees} used by \incite[9]{Smullyan1995}, but drawn in reverse and having no excess information present.
  \item The \hyperref[def:rooted_tree/leaf]{leaves} of the tree are variables and constants, while every other node is either a binary connective or negation.
\end{comments}

\begin{example}\label{ex:def:propositional_fomula_ast}
  We list examples of \hyperref[def:propositional_fomula_ast]{abstract syntax trees for propositional formulas}:
  \begin{thmenum}
    \thmitem{ex:def:propositional_fomula_ast/lnc} The law of non-contradiction \eqref{eq:thm:intuitionistic_tautologies/lnc} has the following AST:
    \begin{equation*}
      \includegraphics[page=1]{output/ex__def__propositional_formula_ast}
    \end{equation*}

    \thmitem{ex:def:propositional_fomula_ast/associative_conjunction} We will define semantics for conjunction and disjunction so that they become associative Boolean operators. Following our general discussion of such binary operations in \fullref{rem:binary_operation_syntax_trees}, we will generally conflate the formula
    \begin{equation*}
      ((\synp \synwedge \syn q) \synwedge \syn r)
    \end{equation*}
    with AST
    \begin{equation*}
      \includegraphics[page=2]{output/ex__def__propositional_formula_ast}
    \end{equation*}
    and the formula
    \begin{equation*}
      ((\synp \synwedge \syn q) \synwedge \syn r)
    \end{equation*}
    with AST
    \begin{equation*}
      \includegraphics[page=3]{output/ex__def__propositional_formula_ast}
    \end{equation*}
  \end{thmenum}
\end{example}

\begin{proposition}\label{thm:propositional_formula_balanced_parentheses}
  Propositional formulas have \hyperref[ex:thm:regular_pumping_lemma/balanced_parentheses]{balanced parentheses}, that is, for any formula there are as many left parentheses as there are right parentheses.
\end{proposition}
\begin{proof}
  Denote by \( l_\varphi \) and \( r_\varphi \) the number of left and right parentheses in the formula \( \varphi \).

  We will use \fullref{thm:induction_on_syntax_trees} on \( \varphi \) to prove that \( l_\varphi = r_\varphi \).
  \begin{itemize}
    \item If \( \varphi \) is a constant or variable, there are no parentheses, and \( l_\varphi = r_\varphi = 0 \).
    \item If \( \varphi = \synneg \psi \) and if the inductive hypothesis holds for \( \psi \), then
    \begin{equation*}
      l_\varphi = l_\psi \reloset{\T{ind.}} = r_\psi = r_\varphi
    \end{equation*}

    \item If \( \varphi = (\psi \syncirc \theta) \) and if the hypothesis holds for \( \psi \) and \( \theta \), then
    \begin{equation*}
      l_\varphi = l_\psi + l_\theta + 1 \reloset{\T{ind.}} = r_\psi + r_\theta + 1 = r_\varphi.
    \end{equation*}
  \end{itemize}
\end{proof}

\begin{remark}\label{rem:propositional_formula_parentheses}
  We use several following \enquote{abuse-of-notation} conventions regarding parentheses. These are only notations shortcuts in the \hyperref[con:metalogic]{metalanguage} and the formulas themselves (as abstract mathematical objects) are still assumed to contain parentheses that help them avoid syntactic ambiguity.

  \begin{thmenum}
    \thmitem{rem:propositional_formula_parentheses/outermost} We may skip the outermost parentheses in formulas with top-level \hyperref[def:propositional_alphabet/connectives]{connectives}, e.g. we may write \( \varphi \synwedge \psi \) rather than \( (\varphi \synwedge \psi) \).

    \thmitem{rem:propositional_formula_parentheses/associative} Because of the associativity of \( \synwedge \) and \( \synvee \), which is implied by \fullref{def:propositional_valuation/valuation_function} and \fullref{thm:standard_boolean_functions}, we may skip the parentheses in chains like
    \begin{equation*}
      ( \ldots ((\varphi_1 \synwedge \varphi_2) \synwedge \varphi_3) \synwedge \ldots \synwedge \varphi_{n-1} ) \synwedge \varphi_n.
    \end{equation*}
    and instead write
    \begin{equation*}
      \varphi_1 \synwedge \varphi_2 \synwedge \ldots \synwedge \varphi_{n-1} \synwedge \varphi_n.
    \end{equation*}

    \thmitem{rem:propositional_formula_parentheses/additional} Although not formally necessary, for the sake of readability we may choose to add parentheses around certain formulas like
    \begin{equation*}
      \synneg \varphi \synvee \synneg \psi.
    \end{equation*}
    and instead write
    \begin{equation*}
      (\synneg \varphi) \synvee (\synneg \psi).
    \end{equation*}

    This latter convention is more useful for quantifiers in \hyperref[def:first_order_syntax/formula]{first-order formulas}.
  \end{thmenum}
\end{remark}

\begin{definition}\label{def:propositional_subformula}\incite[8]{Smullyan1995}
  For every \hyperref[def:propositional_syntax/formula]{propositional formula}, we define the set of \term{subformulas} as follows:
  \begin{equation*}
    \op*{Subform}(\varphi) \coloneqq \begin{cases}
      \set{ \varphi },                                                        &\varphi \in \set{ \syntop, \synbot } \T{or} \varphi \in \op*{Prop}, \\
      \set{ \varphi } \bigcup \op*{Subform}(\psi),                            &\varphi = \synneg \psi, \\
      \set{ \varphi } \bigcup \op*{Subform}(\psi) \cup \op*{Subform}(\theta), &\varphi = \psi \syncirc \theta, {\syncirc} \in \op*{Conn}.
    \end{cases}
  \end{equation*}
\end{definition}

\begin{lemma}\label{thm:propositional_subformula_lemma}
  If the formula \( \psi \) is a \hyperref[def:formal_language/substring]{substring} of the formula \( \varphi \), we have the following possibilities:
  \begin{thmenum}
    \thmitem{thm:propositional_subformula/variable} \( \varphi \) is a variable or constant and \( \psi = \varphi \).
    \thmitem{thm:propositional_subformula/negation_self} \( \varphi = \synneg \theta \) and \( \psi \) coincides with \( \varphi \).
    \thmitem{thm:propositional_subformula/negation} \( \varphi = \synneg \theta \) and \( \psi \) is a subformula of \( \theta \).
    \thmitem{thm:propositional_subformula/connective_self} \( \varphi = (\theta \syncirc \chi) \) and \( \psi \) coincides with \( \varphi \).
    \thmitem{thm:propositional_subformula/connective_left} \( \varphi = (\theta \syncirc \chi) \) and \( \psi \) is a subformula of \( \theta \).
    \thmitem{thm:propositional_subformula/connective_right} \( \varphi = (\theta \syncirc \chi) \) and \( \psi \) is a subformula of \( \chi \) but not \( \theta \).
  \end{thmenum}
\end{lemma}
\begin{proof}
  We use \fullref{thm:induction_on_syntax_trees} on \( \varphi \):
  \begin{itemize}
    \item If \( \varphi \) is a variable or constant, it is a single lexeme, and the only possible substring that is a formula is \( \varphi \) itself. This corresponds to \fullref{thm:propositional_subformula/variable}.
    \item If \( \varphi = \synneg \theta \) and if the inductive hypothesis holds for \( \theta \), we have the following possibilities:
    \begin{itemize}
      \item If \( \psi = \varphi \), then \fullref{thm:propositional_subformula/negation_self} holds.
      \item If \( \psi = \synneg \), it is a substring of \( \varphi \), but not itself a formula.
      \item If \( \psi \) is a substring of \( \theta \), we apply the inductive hypothesis --- then \fullref{thm:propositional_subformula/negation} holds.
    \end{itemize}

    \item If \( \varphi = (\theta \syncirc \chi) \), where the inductive hypothesis holds for \( \theta \) and \( \chi \), we have the following possibilities:
    \begin{itemize}
      \item If \( \psi = \varphi \), then \fullref{thm:propositional_subformula/connective_self} holds.
      \item If \( \psi \) is a substring of \( \theta \), then \fullref{thm:propositional_subformula/connective_left} holds.
      \item If \( \psi \) is a substring of \( \chi \) but not of \( \theta \), then \fullref{thm:propositional_subformula/connective_right} holds.
      \item If \( \psi = (\theta \syncirc \psi_\chi \), where \( \psi_\chi \) is a prefix of \( \chi \), then \( \psi \) has unbalanced parentheses, which contradicts \fullref{thm:propositional_formula_balanced_parentheses}, and thus \( \psi \) is not a formula.
      \item Similarly, if \( \psi = \psi_\theta \syncirc \chi) \), where \( \psi_\theta \) is a suffix of \( \theta \), then again \( \psi \) has unbalanced parentheses.
      \item If \( \psi = \psi_\theta \syncirc \psi_\chi \), where \( \psi_\theta \) is a suffix of \( \theta \) and \( \psi_\chi \) is a prefix of \( \chi \), then \( \psi \) is again not a formula because it is not wrapped in parentheses.
    \end{itemize}
  \end{itemize}
\end{proof}

\begin{proposition}\label{thm:propositional_formula_characterization}
  The \hyperref[def:formal_language/substring]{substring} \( \psi \) of the formula \( \varphi \) is a \hyperref[def:propositional_subformula]{subformula} of \( \varphi \) if and only if \( \psi \) is itself a formula.
\end{proposition}
\begin{proof}
  \SufficiencySubProof Straightforward.
  \NecessitySubProof Suppose that \( \psi \) is itself a formula. Then \fullref{thm:propositional_subformula_lemma} implies that it falls into one of the cases of \fullref{def:propositional_subformula}, and is thus a subformula of \( \varphi \).
\end{proof}

\paragraph{Intuitionistic propositional semantics}

\begin{definition}\label{def:truth_value_algebra}
  In order to determine whether a sentence holds under given circumstances, we must have a set of \term[ru=истинностное значение (\cite[17]{Герасимов2011}), en=truth value (\cite[9]{Smullyan1995})]{truth values}, for example the Boolean values \( T \) and \( F \) from \fullref{con:boolean_value}. This set may need some additional structure depending on the syntax of the sentences. The aforementioned set is naturally a \hyperref[def:boolean_algebra]{Boolean algebra}, and the most general setting we will consider are \hyperref[def:heyting_algebra]{Heyting algebras}.

  When working with \hyperref[def:institution/models]{models} and notions related to them, we will presume that an underlying Heyting algebra \( \BbbH \) is fixed. In accordance with \fullref{rem:mathematical_logic_conventions/propositional_constants}, we will denote the top and bottom element of \( \BbbH \) by \( T \) and \( F \).
\end{definition}
\begin{comments}
  \item In \fullref{def:propositional_semantics} we will introduce new terminology based on the particular choice of \( \BbbH \).
\end{comments}

\begin{definition}\label{def:propositional_formula_variables}\mimprovised
  For each formula \( \varphi \), we \hyperref[con:evaluation]{recursively define} the set of variables occurring in the \hyperref[def:propositional_syntax/formula]{propositional formulas}:
  \begin{equation*}
    \op*{Var}(\varphi) \coloneqq \begin{cases}
      \varnothing,                            &\varphi \in \set{ \syntop, \synbot }, \\
      \set{ \varphi },                        &\varphi \in \op*{Prop}, \\
      \op*{Var}(\psi),                        &\varphi = \synneg \psi, \\
      \op*{Var}(\psi) \cup \op*{Var}(\theta), &\varphi = \psi \syncirc \theta, {\syncirc} \in \op*{Conn}.
    \end{cases}
  \end{equation*}
\end{definition}

\begin{definition}\label{def:propositional_valuation}\mimprovised
  We will define valuations for propositional formulas in a fixed \hyperref[def:truth_value_algebra]{truth value Heyting algebra} \( \BbbH \).

  \begin{thmenum}
    \thmitem{def:propositional_valuation/interpretation} An \term[ru=интерпретация (\cite[17]{Герасимов2011}), en=interpretation (\cite[10]{Smullyan1995})]{propositional interpretation} is a function with signature \( I: \op*{Prop} \to \BbbH \).

    We may call an interpretation a \enquote{propositional model} in accordance with \fullref{rem:institutional_model_terminology}. This is further discussed in \fullref{rem:classical_propositional_interpretations}.

    \thmitem{def:propositional_valuation/formula_valuation} Given an interpretation \( I \), we define the corresponding \term[ru=значение истинности (формулы) (\cite[8]{Эдельман1975}), en=valuation (\cite[10]{Smullyan1995})]{valuation} of a formula \( \varphi \) inductively as follows:
    \begin{equation}\label{eq:def:propositional_valuation/formula_valuation}
      \Bracks{\varphi}_I \coloneqq \begin{cases}
        T,                                         &\varphi = \syntop \\
        F,                                         &\varphi = \synbot \\
        I(\varphi),                                &\varphi \in \op*{Prop} \\
        \widetilde{\Bracks{\psi}_I},               &\varphi = \synneg \psi \\
        \Bracks{\psi}_I \relcirc \Bracks{\theta}_I &\varphi = \psi \syncirc \theta, {\syncirc} \in \op*{Conn},
      \end{cases}
    \end{equation}
    where \( \relcirc \) denotes \hyperref[def:heyting_algebra]{Heyting algebra} operation corresponding to the connective \( \syncirc \).

    \thmitem{def:propositional_valuation/valuation_function} If \( \op*{Var}(\varphi) \subseteq \set{ p_1, \ldots, p_n } \), the valuation \( \Bracks{\varphi}_I \) only depends on the particular values \( I(p_1), \ldots, I(p_n) \) of \( I \). Hence, if the variables are clear from the context, we obtain a Boolean function
    \begin{equation*}
      \begin{aligned}
        &\Bracks{\varphi}: \BbbH^n \to \BbbH, \\
        &\Bracks{\varphi}(x_1, \ldots, x_n) \coloneqq \Bracks{\varphi}_I,
      \end{aligned}
    \end{equation*}
    where \( I \) is any interpretation such that \( I(p_k) = x_k \) for \( k = 1, \ldots, n \).

    Unless otherwise noted, we assume that \( p_1, \ldots, p_n \) are precisely the variables of \( \varphi \), ordered lexicographically as discussed in \fullref{rem:grammar_rules_for_variables}. We will call this the \term{valuation function} of \( \varphi \).
  \end{thmenum}
\end{definition}
\begin{comments}
  \item Different authors use different terminology for the concepts in this definition:
  \begin{itemize}
    \item In a very general setting where \hyperref[def:operation_on_set]{algebraic operations} are assigned to abstract logical connectives, \incite[def. 3.2.3]{CitkinMuravitsky2021} use \enquote{valuation} for what we call \enquote{interpretation}. The aforementioned authors explain how valuations can be extended for all formulas without introducing additional terminology.

    \item When restricted to the case where \( \BbbH = \set{ T, F } \), which in accordance to \hyperref{def:propositional_semantics} we will call \enquote{classical semantics}, \incite[10]{Smullyan1995} uses terminology analogous to ours, but instead allows valuations to be arbitrary functions satisfying \eqref{eq:def:propositional_valuation/formula_valuation}. \incite[80]{Mimram2020} proceeds similarly for interpretations, but introduces no special term for valuations. \incite[def. 1.1.6]{Hinman2005} follows a similar approach, but uses the term \enquote{atomic truth assignment} for what we call \enquote{propositional interpretation}, and \enquote{truth assignment} for what we call \enquote{valuation}.
  \end{itemize}
\end{comments}

\begin{definition}\label{def:propositional_institution}\mimprovised
  For a fixed \hyperref[def:truth_value_algebra]{truth value Heyting algebra} \( \BbbH \), \hyperref[def:propositional_valuation/interpretation]{propositional interpretations} naturally give rise to an \hyperref[def:institution]{institution} as follows:
  \begin{thmenum}
    \thmitem{thm:propositional_institution/signatures} For the category of \hyperref[def:institution/signatures]{signatures}, fix some symbol \( \anon \) and let \( \cat{Sign} \) be the \hyperref[def:discrete_category]{discrete category} on \( \set{ \anon } \).

    \thmitem{thm:propositional_institution/sentences} The \hyperref[def:institution/sentences]{sentence functor} is \( \anon \mapsto \op*{Form} \).

    \thmitem{thm:propositional_institution/models} Let \( \BbbI \) be the discrete category on the function set \( \fun(\op*{Prop}, \BbbH) \). The \hyperref[def:institution/models]{model functor} can then be described as \( \anon \mapsto \BbbI \).

    \thmitem{thm:propositional_institution/satisfaction} Finally, let the \hyperref[def:institution/satisfaction]{satisfaction} relation \( I \vDash_{\anon} \varphi \) hold if \( \Bracks{\varphi}_I = T \).
  \end{thmenum}
\end{definition}
\begin{defproof}
  We must show that this is indeed an institution, which requires verifying \eqref{eq:def:institution/satisfaction}. But this is vacuous because there are no nontrivial signature morphisms.
\end{defproof}

\begin{definition}\label{def:propositional_semantics}\mimprovised
  By the \term[ru=семантика (\cite[54]{КолмогоровДрагалин2006}), en=semantics (\cite[8]{Hinman2005})]{semantics} of propositional logic we mean the \hyperref[def:propositional_institution]{propositional institution} and all related notions like \hyperref[def:institution/models]{models}, \hyperref[def:institution/satisfaction]{satisfaction}, \hyperref[def:institutional_entailment]{semantic entailment} and \hyperref[def:semantic_equivalence]{semantic equivalence}. We will say that the semantics are \term{\hyperref[def:intuitionistic_logic]{intuitionistic}}, while in the special case where the \hyperref[def:truth_value_algebra]{truth value Heyting algebra} is the two-element Boolean algebra \( \set{ T, F } \), we will call them \term{\hyperref[def:classical_logic]{classical}} or \term{Boolean}.
\end{definition}

\begin{remark}\label{rem:classical_propositional_interpretations}
  Note that the same propositional interpretation may be used to define different valuations, and the notion of \enquote{classical} or \enquote{intuitionistic} semantics from \fullref{def:propositional_semantics} only applies when \eqref{eq:def:propositional_valuation/formula_valuation} is used. For example, a slight variation of the aforementioned valuation results in \fullref{def:minimal_propositional_semantics}. Thus, an interpretation cannot, by itself, be classical.

  On the other hand, (institutional) models can be classical, and propositional models are interpretations, so we are free to use the term \enquote{classical model} when referring to interpretations in an appropriate context.
\end{remark}

\begin{definition}\label{def:propositional_entailment}\mimprovised
  For a fixed \hyperref[def:truth_value_algebra]{truth value Heyting algebra}, we will now consider the \hyperref[def:institutional_entailment]{institutional entailment} relation obtained via \fullref{def:propositional_institution}, according to which that the set \( \Gamma \) of \hyperref[def:propositional_syntax/formula]{propositional formulas} entails \( \psi \) if, whenever some \hyperref[def:propositional_valuation/interpretation]{interpretation} \hyperref[thm:propositional_institution/satisfaction]{satisfies} every formula of \( \Gamma \), it also satisfies \( \psi \).

  We denote the corresponding relation via \( {\vDash} \). Similarly to other \hyperref[def:entailment_system/entailment]{entailment relations}, we use the sequent notation discussed in \fullref{rem:sequent_notation}.
\end{definition}

\begin{example}\label{ex:trivial_heyting_semantics}
  Within the one-element Heyting algebra, where \( T = F \), every formula is satisfied because there is simply no \enquote{non-true} truth value. \Fullref{ex:inconsistent_lindenbaum_tarski_algebra} implies that this precisely is the \hyperref[def:lindenbaum_tarski_algebra]{Lindenbaum-Tarski algebra} of any \hyperref[def:consistent_set_of_sentences]{inconsistent} set of formulas.
\end{example}

\begin{example}\label{ex:heyting_semantics_lem_counterexample}
  Consider the three-valued Heyting algebra from \fullref{ex:def:heyting_algebra/three_valued}, where \( F < N < T \).

  Let \( I \) be a \hyperref[def:propositional_valuation]{propositional interpretation} such that \( I(\synp) = N \). Then the valuation of \eqref{eq:thm:classical_tautologies/lem} is
  \begin{equation*}
    \Bracks{\synp \synvee \synneg \synp}_I
    =
    \Bracks{\synp}_I \vee \widetilde{\Bracks{\synp}_I}
    =
    N \vee F
    =
    N.
  \end{equation*}

  Therefore, \eqref{eq:thm:classical_tautologies/lem} does not hold in general.
\end{example}

\begin{example}\label{ex:topological_semantics_lem_counterexample}
  Consider the standard topology in \( \BbbR \) and the corresponding topological Heyting algebra from \fullref{ex:def:heyting_algebra/topology}.

  Let \( U \) be an open set. We will examine \eqref{eq:thm:classical_tautologies/lem}. Given any \hyperref[def:propositional_valuation]{propositional interpretation} \( I \) such that \( I(\synp) = U \), we have
  \begin{equation*}
    \Bracks{\synp \synvee \synneg \synp}_I
    =
    \Bracks{\synp}_I \cup \widetilde{\Bracks{\synp}_I}
    =
    U \cup \widetilde{U}
    =
    U \cup \Int(\BbbR \setminus U).
  \end{equation*}

  If \( U \) is empty, then \( \Bracks{\synp \synvee \synneg \synp}_I = \BbbR \) and \eqref{eq:thm:classical_tautologies/lem} holds in this case. If \( U \) is the open unit interval \( (0, 1) \), then \( \Bracks{\synp \synvee \synneg \synp}_I = \BbbR \setminus \set{ 0, 1 } \) and \eqref{eq:thm:classical_tautologies/lem} does not hold.
\end{example}

\begin{proposition}\label{thm:intuitionistic_equivalences}
  We will list some \hyperref[def:propositional_semantics]{intuitionistic} propositional \hyperref[def:semantic_equivalence]{semantic equivalences}:
  \begin{thmenum}
    \thmitem{thm:intuitionistic_equivalences/negation_bottom} Negation can be expressed as a conditional formula whose \hyperref[def:conditional_formula/consequent]{consequent} is the falsum:
    \begin{equation}\label{eq:thm:intuitionistic_equivalences/negation_bottom}
      \mathllap{\synneg \varphi} \gleichstark \mathrlap{\varphi \synimplies \synbot.}
    \end{equation}

    \thmitem{thm:intuitionistic_equivalences/top_elim} We can eliminate verum from the antecedent of conditional formulas:
    \begin{equation}\label{eq:thm:intuitionistic_equivalences/top_elim}
      \mathllap{\syntop \synimplies \varphi} \gleichstark \mathrlap{\varphi.}
    \end{equation}

    \thmitem{thm:intuitionistic_equivalences/contradiction} Falsum is equivalent to a conjunction of a formula and its negation:
    \begin{equation}\label{eq:thm:intuitionistic_equivalences/contradiction}
      \mathllap{\varphi \synwedge \synneg \varphi} \gleichstark \mathrlap{\synbot.}
    \end{equation}
  \end{thmenum}
\end{proposition}
\begin{comments}
  \item For every formula \( \varphi \), \eqref{eq:thm:intuitionistic_equivalences/negation_bottom} is a different axiom, and we refer to \eqref{eq:thm:intuitionistic_equivalences/negation_bottom} itself as an \enquote{axiom schema}. We will formalize schemas via \hyperref[def:propositional_formula_schema]{formula schemas} in \fullref{subsec:axiomatic_derivations}.
  \item Both \eqref{eq:thm:intuitionistic_equivalences/negation_bottom} and \eqref{eq:thm:intuitionistic_equivalences/top_elim} hold in \hyperref[def:minimal_propositional_semantics]{minimal semantics}, and their proofs do not use any semantic properties of the falsum.
\end{comments}
\begin{proof}
  \SubProofOf{thm:intuitionistic_equivalences/negation_bottom} Follows from \fullref{def:heyting_algebra/pseudocomplement}.

  \SubProofOf{thm:intuitionistic_equivalences/top_elim} Follows from \fullref{thm:def:heyting_algebra/top_left}.

  \SubProofOf{thm:intuitionistic_equivalences/contradiction} For any interpretation \( I \), we have
  \begin{equation*}
    \Bracks{ \varphi \synwedge \synneg \varphi }_I
    =
    \Bracks{ \varphi }_I \synwedge (\Bracks{ \varphi }_I \rightarrow F)
    \reloset {\eqref{eq:def:heyting_algebra/axioms/modus_ponens}} =
    \Bracks{ \varphi }_I \wedge F
    =
    F
    =
    \Bracks{\synbot}_I.
  \end{equation*}
\end{proof}

\begin{definition}\label{def:propositional_tautology}
  We say that \( \varphi \) is a \term[ru=пропозициональная тавтология (\cite[44]{КолмогоровДрагалин2006})]{propositional tautology} if any of the following equivalent conditions hold:
  \begin{thmenum}
    \thmitem{def:propositional_tautology/interpretations}\mcite[11]{Smullyan1995} We have \( \Bracks{\varphi}_I = T \) for every interpretation \( I \).
    \thmitem{def:propositional_tautology/entailment} We have the entailment \( \vDash \varphi \).
    \thmitem{def:propositional_tautology/equivalence} We have the equivalence \( \varphi \gleichstark \syntop \).
  \end{thmenum}
\end{definition}

\begin{definition}\label{def:propositional_contradiction}
  Dually, we say that \( \varphi \) is a \term[en=contradictory (formula) (\cite[28]{Kleene2002Logic})]{propositional contradiction} if any of the following equivalent conditions hold:
  \begin{thmenum}
    \thmitem{def:propositional_contradiction/interpretations}\mcite[def. 1.4.1(i)]{Hinman2005} We have \( \Bracks{\varphi}_I = F \) for every interpretation \( I \).
    \thmitem{def:propositional_contradiction/entailment} We have the entailment \( \varphi \vDash \synbot \).
    \thmitem{def:propositional_contradiction/equivalence} We have the equivalence \( \varphi \gleichstark \synbot \).
  \end{thmenum}
\end{definition}

\begin{proposition}\label{thm:intuitionistic_tautologies}
  We will list some \hyperref[def:propositional_semantics]{intuitionistic} propositional \hyperref[def:propositional_tautology]{tautologies}:
  \begin{thmenum}
    \thmitem{thm:intuitionistic_tautologies/self} Every formula implies itself:
    \begin{equation}\label{eq:thm:intuitionistic_tautologies/self}
      \varphi \synimplies \varphi.
    \end{equation}

    \thmitem{thm:intuitionistic_tautologies/dni} Any formula implies its double negation:
    \begin{equation}\label{eq:thm:intuitionistic_tautologies/dni}
      \varphi \synimplies \neg \neg \varphi \tag{\( \logic{DNI}_A \)}
    \end{equation}

    \enquote{DNI} stands for \enquote{double negation introduction}.

    \thmitem{thm:intuitionistic_tautologies/efq} Falsum implies anything:
    \begin{equation}\label{eq:thm:intuitionistic_tautologies/efq}
      \synbot \synimplies \varphi \tag{\( \logic{EFQ}_A \)}
    \end{equation}

    \enquote{EFQ} stands for \enquote{ex falso quodlibet}, which is Latin for \enquote{from falsehood, anything}. It is also known as the \enquote{the principle of explosion} because it allows us to show that intuitionistic semantics are \hyperref[def:paraconsistent_consequence_operator]{explosive} --- see \fullref{thm:intuitionistic_semantics_are_explosive}. Both terms are used by \incite[47]{Mimram2020}.

    \thmitem{thm:intuitionistic_tautologies/ecq} A \hyperref[def:propositional_contradiction]{contradiction} implies anything:
    \begin{equation}\label{eq:thm:intuitionistic_tautologies/ecq}
      (\varphi \synwedge \neg \varphi) \synimplies \psi \tag{\( \logic{ECQ}_A \)}
    \end{equation}

    \enquote{EFQ} stands for \enquote{ex contradictione quodlibet}, which is Latin for \enquote{from a contradiction, anything}. The term is used by \incite[2]{DienerMcKubreJordens2016}.

    \thmitem{thm:intuitionistic_tautologies/lnc} A formula and its negation cannot both hold:
    \begin{equation}\label{eq:thm:intuitionistic_tautologies/lnc}
      \synneg (\varphi \synwedge \synneg \varphi). \tag{\( \logic{LNC}_A \)}
    \end{equation}

    \enquote{LCN} stands for \enquote{law of non-contradiction}.
  \end{thmenum}
\end{proposition}
\begin{comments}
  \item Some tautologies that hold in \hyperref[def:propositional_semantics]{classical semantics} are listed in \fullref{thm:classical_tautologies}.
  \item Both \eqref{eq:thm:intuitionistic_tautologies/dni} and \eqref{eq:thm:intuitionistic_tautologies/lnc} hold under \hyperref[def:minimal_propositional_semantics]{minimal semantics}, but our semantic proofs for them are not valid in minimal logic. We will demonstrate simple syntactic proofs in \fullref{thm:syntactic_minimal_tautologies}.
\end{comments}
\begin{proof}
  Fix an interpretation \( I \).

  \SubProofOf{thm:intuitionistic_tautologies/self}
  \begin{equation*}
    \Bracks{\varphi \synimplies \varphi}_I
    =
    \Bracks{\varphi}_I \synimplies \Bracks{\varphi}_I
    \reloset {\ref{thm:def:heyting_algebra/leq}} =
    T.
  \end{equation*}

  \SubProofOf{thm:intuitionistic_tautologies/dni}
  \begin{equation*}
    \Bracks{\neg \neg \varphi}_I
    =
    \Bracks{\varphi}_I \rightarrow \widetilde{\widetilde{\Bracks{\varphi}_I}}
    \reloset {\ref{thm:def:heyting_algebra/dni}} =
    T.
  \end{equation*}

  \SubProofOf{thm:intuitionistic_tautologies/efq}
  \begin{equation*}
    \Bracks{\synbot \synimplies \varphi}_I
    =
    \Bracks{\synbot}_I \implies \Bracks{\varphi}_I
    \reloset {\ref{thm:def:heyting_algebra/leq}} =
    T.
  \end{equation*}

  \SubProofOf{thm:intuitionistic_tautologies/ecq}
  \begin{equation*}
    \Bracks{(\varphi \synwedge \neg \varphi) \synimplies \psi}_I
    \reloset {\eqref{eq:thm:intuitionistic_equivalences/contradiction}} =
    \Bracks{\synbot \synimplies \psi}_I
    \reloset {\eqref{eq:thm:intuitionistic_tautologies/efq}} =
    T.
  \end{equation*}

  \SubProofOf{thm:intuitionistic_tautologies/lnc}
  \begin{equation*}
    \Bracks{\synneg (\varphi \synwedge \synneg \varphi)}_I
    =
    \widetilde{\Bracks{\varphi}_I \wedge \widetilde{\Bracks{\varphi}_I}}
    \reloset {\eqref{eq:thm:intuitionistic_equivalences/contradiction}} =
    \widetilde{F}
    =
    F \rightarrow F
    \reloset {\eqref{eq:thm:intuitionistic_tautologies/self}} =
    T.
  \end{equation*}
\end{proof}

\paragraph{Brouwer-Heyting-Kolmogorov interpretation}

\begin{concept}\label{con:brouwer_heyting_kolmogorov_interpretation}\mcite[sec. 1.3.1]{TroelstraSchwichtenberg2000}
  \hyperref[def:propositional_semantics]{Intuitionistic semantics} correspond to the less formal \term{Brouwer-Heyting-Kolmogorov interpretation}. This interpretation is based on the notion of a \enquote{construction}, which is also why we will refer to intuitionistic logic as \term{constructive logic}.

  \begin{thmenum}
    \thmitem{con:brouwer_heyting_kolmogorov_interpretation/verum} We suppose that \( \syntop \) is evident without a construction.

    \thmitem{con:brouwer_heyting_kolmogorov_interpretation/atomic} We suppose that we know what constitutes a construction proving a propositional variable.

    \thmitem{con:brouwer_heyting_kolmogorov_interpretation/falsum} There is no construction that proves \( \synbot \).

    \thmitem{con:brouwer_heyting_kolmogorov_interpretation/disjunction} A construction proving \( \varphi \synvee \psi \) is a construction proving \( \varphi \), or a construction proving \( \psi \).

    \thmitem{con:brouwer_heyting_kolmogorov_interpretation/conjunction} A construction proving \( \varphi \synwedge \psi \) is a pair of constructions, one proving \( \varphi \) and one proving \( \psi \).

    \thmitem{con:brouwer_heyting_kolmogorov_interpretation/conditional} A construction proving \( \varphi \synimplies \psi \) is a transformation of constructions proving \( \varphi \) into constructions proving \( \psi \).

    \thmitem{con:brouwer_heyting_kolmogorov_interpretation/biconditional} Based on \fullref{con:brouwer_heyting_kolmogorov_interpretation/conditional}, the biconditional \( \varphi \syniff \varphi \) then corresponds to a pair of transformations (not necessarily inverses).

    \thmitem{con:brouwer_heyting_kolmogorov_interpretation/negation} A construction of \( \neg \varphi \) demonstrates the impossibility of a construction of \( \varphi \). Based on \fullref{con:brouwer_heyting_kolmogorov_interpretation/falsum} and \fullref{con:brouwer_heyting_kolmogorov_interpretation/conditional}, a construction of \( \neg \varphi \) should correspond to a transformation of constructions proving \( \varphi \) into elements of the empty set, and such a transformation is only possible if there exist no constructions proving \( \varphi \).
  \end{thmenum}
\end{concept}

\begin{example}\label{ex:con:brouwer_heyting_kolmogorov_interpretation/well_ordering_principle_zfc}
  \Fullref{thm:well_ordering_theorem} in \hyperref[def:zfc]{\( \logic{ZFC} \)} does not provide a way to well-order an arbitrary set. The theorem relies on the axiom of choice, whose consequence \fullref{thm:diaconescu_goodman_myhill_theorem} proves the law of the excluded middle \eqref{eq:thm:classical_tautologies/lem} from the axioms of \logic{ZF}.

  Since \logic{LEM} may not hold in intuitionistic logic, it follows that both \fullref{thm:well_ordering_theorem} and the axiom of choice itself should not in general hold under the Brouwer-Heyting-Kolmogorov interpretation, hence by the terminology in \fullref{con:brouwer_heyting_kolmogorov_interpretation}, \fullref{thm:well_ordering_theorem} is a non-constructive theorem.
\end{example}

\paragraph{Minimal propositional semantics}

\begin{definition}\label{def:minimal_propositional_semantics}\mimprovised
  In addition to classical and intuitionistic semantics discussed in \fullref{def:propositional_semantics}, we will also occasionally consider \term{minimal semantics}, obtained by interpreting \( \synbot \) as a propositional variable. Thus, a propositional interpretation must now provide a value for \( \synbot \).

  Formula valuations are thus defined as follows:
  \begin{equation}\label{eq:def:minimal_propositional_semantics/formula_valuation}
    \Bracks{\varphi}_I \coloneqq \begin{cases}
      I(\synbot),                             &\varphi = \synbot \\
      \Bracks{\psi}_I \rightarrow I(\synbot), &\varphi = \synneg \psi \\
      \vdots, &
    \end{cases}
  \end{equation}
  the other cases being the same as in \eqref{eq:def:propositional_valuation/formula_valuation}.
\end{definition}
\begin{comments}
  \item \Fullref{thm:lindenbaum_tarski_algebras} provides a concrete motivation for this precise definition.
  \item Unless we are specifically interested in \( \synbot \) and \( \synneg \), a more reasonable approach may be to only consider the \hyperref[def:propositional_syntax/fragment]{fragment} of propositional logic without them. This is approach to minimal logic mentioned in \mcite[49]{Mimram2020}, and this is the approach used in \fullref{def:minimal_implication_logic}. In this case we lose the distinction between minimal and intuitionistic logic. Perhaps this is the entire point of working in minimal logic?
  \item Minimal logic was introduced by Ingebrigt Johansson as an attempt to further refine intuitionistic logic. An analysis of his works and his interactions with Gerhard Gentzen is given by \incite{VanDerMolen2016}. No semantics are discussed there.
  \item Our definition for minimal semantics is based on the general principles outlined by \incite[3]{VanDerMolen2016}, \incite[35]{TroelstraSchwichtenberg2000} and \incite[1]{DienerMcKubreJordens2016}. None of the aforementioned discuss semantics.
\end{comments}

\begin{definition}\label{def:paraconsistent_consequence_operator}\mcite{StanfordPlato:paraconsistent_logic}
  We say that the \hyperref[def:consequence_operator]{consequence operator} \( \vdash \) for propositional logic is \term{explosive} if the following generalization of \eqref{eq:thm:intuitionistic_tautologies/ecq} holds:
  \begin{equation*}
    \varphi, \synneg \varphi \vdash \psi.
  \end{equation*}

  If \( \vdash \) is not explosive, we say that it is \term{paraconsistent}.
\end{definition}

\begin{proposition}\label{thm:minimal_semantics_are_paraconsistent}
  \hyperref[def:minimal_propositional_semantics]{Minimal semantics} are \hyperref[def:paraconsistent_consequence_operator]{paraconsistent}.
\end{proposition}
\begin{proof}
  Fix an interpretation \( I \) such that \( I(\synbot) = T \). Then
  \begin{equation*}
    \Bracks{\synneg \syntop}_I
    =
    \Bracks{\syntop}_I \rightarrow I(\synbot)
    =
    T \rightarrow T
    \reloset {\ref{thm:intuitionistic_tautologies/self}} =
    T
  \end{equation*}

  Then \( I \vDash \syntop \) and \( I \vDash \synneg \syntop \). Yet, if \( I(P) = F \) for some variable \( P \), we cannot conclude that \( I \vDash P \).
\end{proof}

\begin{proposition}\label{thm:intuitionistic_semantics_are_explosive}
  \hyperref[def:propositional_semantics]{Intuitionistic semantics} are \hyperref[def:paraconsistent_consequence_operator]{explosive}.
\end{proposition}
\begin{proof}
  Fix two formulas, \( \varphi \) and \( \psi \), and an interpretation \( I \) such that \( I \vDash \varphi \) and \( I \vDash \psi \). Then \( I \vDash \varphi \synwedge \synneg \varphi \) because
  \begin{equation*}
    \Bracks{\varphi \synwedge \synneg \varphi}_I
    =
    \Bracks{\varphi}_I \synwedge \Bracks{\synneg \varphi}_I
    =
    T \wedge T
    =
    T.
  \end{equation*}

  But also
  \begin{equation*}
    \Bracks{\varphi \synwedge \synneg \varphi}_I
    =
    \Bracks{\varphi}_I \synwedge (\Bracks{\varphi}_I \rightarrow F)
    \reloset {\eqref{eq:def:heyting_algebra/axioms/modus_ponens}} =
    \Bracks{\varphi}_I \synwedge F
    =
    F.
  \end{equation*}

  The obtained contradiction shows that such an interpretation \( I \) does not exist.

  Therefore, for all zero interpretations \( I \) for which both \( I \vDash \varphi \) and \( I \vDash \psi \), we can conclude that \( I \vDash \psi \).
\end{proof}

\begin{proposition}\label{thm:semantic_propositional_conjunction_of_premises}
  With respect to \hyperref[def:minimal_propositional_semantics]{minimal semantics}, we have \( \varphi, \psi \vDash \theta \) if and only if \( (\varphi \synwedge \psi) \vDash \theta \).
\end{proposition}
\begin{proof}
  Trivial.
\end{proof}

\begin{theorem}[Propositional semantic deduction theorem]\label{thm:propositional_semantic_deduction_theorem}
  With respect to \hyperref[def:minimal_propositional_semantics]{minimal semantics}, for an arbitrary \hyperref[def:truth_value_algebra]{truth value Heyting algebra} and arbitrary propositional formulas, we have
  \begin{equation*}
    \Gamma, \varphi \vDash \psi \T{if and only if} \Gamma \vDash \varphi \synimplies \psi.
  \end{equation*}
\end{theorem}
\begin{comments}
  \item See \fullref{rem:deduction_theorem_list} for a list of similar theorems.
\end{comments}
\begin{proof}
  Fix an interpretation \( I \) satisfying \( \Gamma \).

  \SufficiencySubProof Suppose that from \( I \vDash \varphi \) it follows that \( I \vDash \psi \). If \( I \vDash \varphi \), then
  \begin{equation*}
    \Bracks{\varphi \synimplies \psi}_I
    =
    \underbrace{\Bracks{\varphi}_I}_T \rightarrow \underbrace{\Bracks{\psi}_I}_T
    \reloset {\ref{thm:def:heyting_algebra/top_left}} =
    \underbrace{\Bracks{\psi}_I}_T,
  \end{equation*}
  thus \( I \vDash (\varphi \synimplies \psi) \).

  \NecessitySubProof Suppose that \( I \vDash (\varphi \synimplies \psi) \). If \( I \vDash \varphi \), then
  \begin{equation*}
    \underbrace{\Bracks{\varphi \synimplies \psi}_I}_{T}
    =
    \underbrace{\Bracks{\varphi}_I}_T \rightarrow \Bracks{\psi}_I
    \reloset {\ref{thm:def:heyting_algebra/top_left}} =
    \Bracks{\psi}_I,
  \end{equation*}
  thus \( I \vDash \psi \).
\end{proof}

\begin{corollary}\label{thm:intuitionistic_deduction_consequences}
  We list some consequences of \Fullref{thm:propositional_semantic_deduction_theorem}:
  \begin{thmenum}
    \thmitem{thm:intuitionistic_deduction_consequences/top} Under \hyperref[def:minimal_propositional_semantics]{minimal semantics}, \( \varphi \vDash \syntop \) for every formula \( \varphi \).

    \thmitem{thm:intuitionistic_deduction_consequences/bot} Under \hyperref[def:propositional_semantics]{intuitionistic semantics}, \( \synbot \vDash \varphi \) for every formula \( \varphi \).
  \end{thmenum}
\end{corollary}
\begin{proof}
  \SubProofOf{thm:intuitionistic_deduction_consequences/top} Follows from \fullref{thm:def:heyting_algebra/top_right}.

  \SubProofOf{thm:intuitionistic_deduction_consequences/bot} Follows from \fullref{thm:def:heyting_algebra/leq}.
\end{proof}

\begin{remark}\label{rem:deduction_theorem_list}
  The following is a list of different flavors of the \enquote{deductive theorem} that allows us to use \hyperref[def:propositional_alphabet/connectives/conditional]{implication} and \hyperref[def:entailment_system/entailment]{entailment} interchangeably:
  \begin{thmenum}
    \thmitem{rem:deduction_theorem_list/propositional_semantic} \Fullref{thm:propositional_semantic_deduction_theorem}
    \thmitem{rem:deduction_theorem_list/implicational_syntactic} \Fullref{thm:implicational_syntactic_deduction_theorem}
    \thmitem{rem:deduction_theorem_list/propositional_syntactic} \Fullref{thm:propositional_syntactic_deduction_theorem}
  \end{thmenum}
\end{remark}

\paragraph{Classical propositional semantics}

\begin{proposition}\label{thm:classical_equivalences}
  We will list some \hyperref[def:propositional_semantics]{classical} propositional \hyperref[def:semantic_equivalence]{equivalences}\fnote{These are actually statements about \hyperref[thm:standard_boolean_functions]{standard Boolean functions}, but nevertheless we find propositional logic more convenient for stating them.}:
  \begin{thmenum}
    \thmitem{thm:classical_equivalences/double_negation} Negation is an \hyperref[def:involution]{involution}:
    \begin{equation}\label{eq:thm:classical_equivalences/double_negation}
      \mathllap{\synneg \synneg \varphi} \gleichstark \mathrlap{\varphi.}
    \end{equation}

    This equivalence is a combination of \eqref{eq:thm:intuitionistic_tautologies/dni} and \eqref{eq:thm:classical_tautologies/dne}.

    \thmitem{thm:classical_equivalences/conditional_as_disjunction} A conditional formula is a disjunction with the \hyperref[def:conditional_formula/antecedent]{antecedent} negated:
    \begin{equation}\label{eq:thm:classical_equivalences/conditional_as_disjunction}
      \mathllap{\varphi \synimplies \psi} \gleichstark \mathrlap{ \synneg \varphi \synvee \psi. }
    \end{equation}

    This equivalence is known under the name \enquote{material implication} and is discussed in \fullref{con:material_implication}.

    \thmitem{thm:classical_equivalences/contrapositive} A conditional formula is equivalent to its \hyperref[def:conditional_formula/contrapositive]{contrapositive}:
    \begin{equation}\label{eq:thm:classical_equivalences/contrapositive}
      \mathllap{\varphi \synimplies \psi} \gleichstark \mathrlap{\synneg \psi \synimplies \synneg \varphi.}
    \end{equation}

    \thmitem{thm:classical_equivalences/distributivity} Disjunctions and conjunctions distribute over each other:
    \begin{subequations}
      \begin{align}
        \mathllap{\varphi \synvee (\psi \synwedge \theta)} &\gleichstark \mathrlap{(\varphi \synvee \psi) \synwedge (\varphi \synvee \theta),}   \label{eq:thm:classical_equivalences/distributivity/join_over_meet} \\
        \mathllap{\varphi \synwedge (\psi \synvee \theta)} &\gleichstark \mathrlap{(\varphi \synwedge \psi) \synvee (\varphi \synwedge \theta).} \label{eq:thm:classical_equivalences/distributivity/meet_over_join}
      \end{align}
    \end{subequations}

    This equivalence motivates axioms \eqref{eq:def:distributive_lattice/join_over_meet} and \eqref{eq:def:distributive_lattice/meet_over_join} for \hyperref[def:distributive_lattice]{distributive lattices}.

    \thmitem{thm:classical_equivalences/de_morgan} Conjunctions and disjunctions are obtained from each other via negation:
    \begin{subequations}
      \begin{align}
        \mathllap{\synneg (\varphi \synvee \psi)}   &\gleichstark \mathrlap{\synneg \varphi \synwedge \synneg \psi,} \label{eq:thm:classical_equivalences/de_morgan/complement_of_join} \\
        \mathllap{\synneg (\varphi \synwedge \psi)} &\gleichstark \mathrlap{\synneg \varphi \synvee \synneg \psi.}   \label{eq:thm:classical_equivalences/de_morgan/complement_of_meet}
      \end{align}
    \end{subequations}

    See \fullref{rem:de_morgans_laws} for a list of related theorems.

    \thmitem{thm:classical_equivalences/biconditional_member_negation} A biconditional formula is equivalent to its termwise negation:
    \begin{equation}\label{eq:thm:classical_equivalences/biconditional_member_negation}
      \mathllap{\varphi \syniff \psi} \gleichstark \mathrlap{\synneg \varphi \syniff \synneg \psi.}
    \end{equation}

    \thmitem{thm:classical_equivalences/biconditional_negation} A negation of a biconditional formula is again a biconditional with one of the terms negated:
    \begin{equation}\label{eq:thm:classical_equivalences/biconditional_negation}
      \begin{aligned}
        \mathllap{\synneg \parens{\varphi \syniff \psi}}
        &\gleichstark
        \mathrlap{\synneg \varphi \syniff \psi \gleichstark}
        \\ &\gleichstark
        \mathrlap{\varphi \syniff \synneg \psi.}
      \end{aligned}
    \end{equation}
  \end{thmenum}
\end{proposition}
\begin{comments}
  \item These equivalences fail more generally in \hyperref[def:intuitionistic_logic]{intuitionistic logic}.
\end{comments}
\begin{proof}
  The proofs follow directly from the table in \fullref{thm:standard_boolean_functions}.
\end{proof}

\begin{proposition}\label{thm:classical_tautologies}
  We will list some \hyperref[def:propositional_semantics]{classical} propositional \hyperref[def:propositional_tautology]{tautologies}:
  \begin{thmenum}
    \thmitem{thm:classical_tautologies/dne} The double negation of a formula implies it:
    \begin{equation}\label{eq:thm:classical_tautologies/dne}
      \synneg \synneg \varphi \synimplies \varphi \tag{\( \logic{DNE}_A \)}
    \end{equation}

    \enquote{DNE} stands for \enquote{double negation elimination}.

    \thmitem{thm:classical_tautologies/pierce} The following, named \enquote{Pierce's law}, holds:
    \begin{equation}\label{eq:thm:classical_tautologies/pierce}
      ((\varphi \synimplies \psi) \synimplies \varphi) \synimplies \varphi \tag{\( \logic{Pierce}_A \)}
    \end{equation}

    \thmitem{thm:classical_tautologies/lem} Either a formula holds or its negation does:
    \begin{equation}\label{eq:thm:classical_tautologies/lem}
      \varphi \synvee \synneg \varphi. \tag{\( \logic{LEM}_A \)}
    \end{equation}

    \enquote{LEM} stands for \enquote{law of the excluded middle}.
  \end{thmenum}
\end{proposition}
\begin{comments}
  \item Some tautologies that hold in \hyperref[def:propositional_semantics]{intuitionistic semantics} are listed in \fullref{thm:intuitionistic_tautologies}.
\end{comments}
\begin{proof}
  The proofs follow directly from the table in \fullref{thm:standard_boolean_functions}.
\end{proof}

\begin{concept}\label{con:material_implication}
  Consider the conditional formula \( \varphi \synimplies \psi \). Via \fullref{thm:propositional_semantic_deduction_theorem}, the equivalence \eqref{eq:thm:classical_equivalences/conditional_as_disjunction} can be restated as follows:
  \begin{displayquote}
    The formula \( \varphi \synimplies \psi \) holds unless the antecedent \( \varphi \) holds and the consequent \( \psi \) doesn't.
  \end{displayquote}

  This equivalence goes back to Philo of Megara, who lived in the third century B.C. We refer to it as \term[ru=материальная импликация (\cite[74]{КолмогоровДрагалин2006}), en=material implication (\cite[9]{Kleene2002Logic})]{material implication} to distinguish it from \enquote{strict} and \enquote{relevant} implications, as well as \enquote{assertions} and other possibilities that try to avoid the so-called \enquote{paradoxes of material implication}. A survey of the aforementioned concepts has been written by \incite{StanfordPlato:logic_of_conditionals}.
\end{concept}
