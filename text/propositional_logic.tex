\section{Propositional logic}\label{sec:propositional_logic}

Propositional logic allows us to express basic relations between atomic \hyperref[con:proposition]{propositions}, irrespective of how these propositions are structured.

We give here the following definition, and then dedicate several sections to explaining it:
\begin{definition}\label{def:propositional_logic}\mimprovised
  We call \term[ru=логика высказываний (\cite[10]{Герасимов2014Вычислимость}), en=propositional logic (\cite[4]{Kleene2002Logic})]{propositional logic} the \hyperref[def:general_logic]{general logic} specified by the \hyperref[def:propositional_institution]{propositional institution} (with \hyperref[def:minimal_propositional_semantics]{minimal}, \hyperref[def:truth_value_algebra/intuitionistic]{intuitionistic} or \hyperref[def:truth_value_algebra/classical]{classical} semantics) and the corresponding \hyperref[def:propositional_natural_deduction_systems]{propositional natural deduction system}.
\end{definition}
\begin{comments}
  \item This general logics are well-defined due to \fullref{thm:propositional_natural_deduction_soundness}.

  \item We may replace natural deduction with a suitable \hyperref[def:propositional_axiomatic_derivation_system]{axiomatic derivation system} or possibly another proof system, as long as it produces the same entailment system.

  \item As explained in \cref{ex:def:bounded_hol/zeroth}, propositional logic can be viewed as a degenerate case of \hyperref[con:predicate_logic]{predicate logic}, which suggests the name \enquote{zeroth-order logic}.
\end{comments}

\paragraph{Syntax of propositional logic}

There are different approaches to formalizing the syntax of propositional logic. We use the theory of formal grammars that we have developed in \fullref{sec:formal_languages} and \fullref{sec:syntax_trees}. This requires specific adjustments related to other treatments of propositional logic, for example \cite[ch. 1]{Hinman2005Logic}, \cite[def. 1.1.2]{VanDalen2004LogicAndStructure}, \cite[\S I.1]{Kleene2002Logic}, \cite[ch. 1]{КолмогоровДрагалин2006Логика} and \cite[\S 1.1]{ШеньВерещагин2017ЯзыкиИИсчисления}. These rely on a more informal treatment of syntax more akin to what can be achieved via \fullref{thm:knaster_tarski_iteration}. Our approach allows us to use powerful tools while treating the object logic in complete formality. \incite[45]{Mimram2020ProgramEqualsProof} also describes syntax via formal grammars, but does not rely on the theory of formal languages.

\begin{definition}\label{def:propositional_alphabet}
  The \hyperref[def:formal_language/alphabet]{alphabet} of \hyperref[con:improper_symbol]{improper} and \hyperref[con:variable]{constant} \hyperref[def:formal_language/symbol]{symbols} of propositional logic consists of the following:

  \begin{thmenum}
    \thmitem{def:propositional_alphabet/constants} Two \term{propositional constants}:
    \begin{thmenum}
      \thmitem{def:propositional_alphabet/constants/verum}\mcite[VIII]{Peano1889ArithmeticesPrincipia} The \term{verum}, \enquote{\( \syntop \)}.
      \thmitem{def:propositional_alphabet/constants/falsum}\mcite[VIII]{Peano1889ArithmeticesPrincipia} The \term{falsum}, \enquote{\( \synbot \)}.
    \end{thmenum}

    \thmitem{def:propositional_alphabet/negation}\mcite[def. 1.1.1(ii)]{VanDalen2004LogicAndStructure} The \term[ru=отрицание (\cite[17]{КолмогоровДрагалин2006Логика})]{negation} symbol \enquote{\( \synneg \)}.

    \thmitem{def:propositional_alphabet/connectives}\mcite[def. 1.1.1(ii)]{VanDalen2004LogicAndStructure} The set \( \op*{Conn} \) of \term{binary propositional connectives}, namely
    \begin{thmenum}
      \thmitem{def:propositional_alphabet/connectives/conjunction} The \term[ru=конъюнкция (\cite[14]{Эдельман1975Логика})]{conjunction} connective \enquote{\( \synwedge \)}, also known as \hyperref[def:standard_boolean_functions]{\term{and}}.
      \thmitem{def:propositional_alphabet/connectives/disjunction} The \term[ru=дизъюнкция (\cite[14]{Эдельман1975Логика})]{disjunction} connective \enquote{\( \synvee \)}, also known as \hyperref[def:standard_boolean_functions]{\term{or}}.
      \thmitem{def:propositional_alphabet/connectives/conditional} The \term[en=conditional (\cite[70]{Kleene2002Logic})]{conditional} connective\fnote{Although we prefer the phrase \enquote{conditional formula} to simply \enquote{conditional}, the words \enquote{conditional} and \enquote{biconditional} are used nouns in this context. We use these terms to avoid confusion with the same connectives in the metalogic, for example \enquote{A biconditional formula is equivalent \dots} could otherwise become \enquote{An equivalence is equivalent \dots}.} \( \synimplies \), also known as \term{if\ldots then} or \term[ru=импликация (\cite[15]{Эдельман1975Логика})]{implication}.
      \thmitem{def:propositional_alphabet/connectives/biconditional} The \term[en=conditional (\cite[70]{Kleene2002Logic})]{biconditional} connective \enquote{\( \syniff \)}, also known as \term{if and only if} (\term{iff}) or \term[ru=эквиваленция (\cite[16]{Эдельман1975Логика})]{equivalence}.
    \end{thmenum}

    \thmitem{def:propositional_alphabet/parentheses}\mcite[def. 1.1.1(iii)]{VanDalen2004LogicAndStructure} Parentheses \enquote{\( ( \)} and \enquote{\( ) \)} for defining the order of operations unambiguously.
  \end{thmenum}
\end{definition}
\begin{comments}
  \item To highlight that the symbols belong to the object language and do not have implicitly associated semantics, we place dots over them; see \cref{rem:mathematical_logic_conventions/terminal_dots} for a general discussion of this convention.

  \item If desired, we can utilize a smaller propositional language without losing its semantical properties. Important example are the \hyperref[def:cnf_and_dnf]{conjunctive normal forms} in \fullref{alg:full_cnf_and_dnf}, although similar constructions hold for other \hyperref[def:boolean_closure/complete]{complete sets of Boolean functions} like those from \cref{thm:complete_sets_of_boolean_functions}.

  \item \incite{Эдельман1975Логика} states that \enquote{conjunctio} and \enquote{disjunctio} are Latin words for \enquote{union} and \enquote{separation}, while \enquote{implicatio} is Latin for \enquote{entangled}.

  \item A popular alternative for the conditional symbol \enquote{\( \rightarrow \)} is \enquote{\( \rightimply \)}. It is used in \cite[3]{Peano1908FormulatioMathematico}, and hence also in \cite[xvi]{WhiteheadRussell1927PrincipiaMathematicaVol1} and subsequently in \cite[44]{Church1956LogicVol1} and \cite[5]{Kleene2002Logic}, as well as \cite[44]{КолмогоровДрагалин2006Логика}, \cite[13]{Герасимов2014Вычислимость} and \cite[15]{Эдельман1975Логика}.

  Modern authors seem to prefer the arrow \enquote{\( \rightarrow \)}, for example \cite[14]{Hinman2005Logic}, \cite[def. 1.1.1(ii)]{VanDalen2004LogicAndStructure}, \cite[1]{TroelstraSchwichtenberg2000BasicProofTheory}, \cite[\S 2.1.1]{Mimram2020ProgramEqualsProof} and \cite[9]{ШеньВерещагин2017ЯзыкиИИсчисления}.
\end{comments}

\begin{definition}\label{def:propositional_formula}\mimprovised
  We introduce a \hyperref[def:formal_grammar]{formal grammar} for \hyperref[con:proposition]{formulas} of \hyperref[def:propositional_logic]{propositional logic} based on the \hyperref[def:propositional_alphabet]{eponymous alphabet}:
  \begin{bnf*}
    \bnfprod{propositional variable} {\bnfpn{Small Latin identifier}} \\
    \bnfprod{constant formula}       {\bnftsq{\( \syntop \)} \bnfor \bnftsq{\( \synbot \)}} \\
    \bnfprod{atomic formula}         {\bnfpn{propositional variable} \bnfor \bnfpn{constant formula}} \\
    \bnfprod{negation formula}       {\bnftsq{\( \synneg \)} \bnfsp \bnfpn{formula}} \\
    \bnfprod{connective formula}     {\bnftsq{(} \bnfsp \bnfpn{formula} \bnfsp \bnfpn{connective} \bnfsp \bnfpn{formula} \bnfsp \bnftsq{)}} \\
    \bnfprod{formula}                {\bnfpn{constant formula} \bnfor} \\
    \bnfmore                         {\bnfpn{propositional variable} \bnfor} \\
    \bnfmore                         {\bnfpn{negation formula} \bnfor} \\
    \bnfmore                         {\bnfpn{connective formula}}
  \end{bnf*}

  We have used here the connectives from \cref{def:propositional_alphabet/connectives} and the variable identifier rules from \cref{def:variable_identifier}.

  Similarly to the case of propositional formulas in \cref{def:propositional_subformula}, we denote by \( \op*{Form}_\Sigma \) the set of all \term[en=first-order formulas (\cite[23]{CitkinMuravitsky2022ConsequenceRelations})]{first-order formulas} over \( \Sigma \), and by \( \op*{Atom}_\Sigma \) those that are \term[ru=атомарные формулы (\cite[56]{КолмогоровДрагалин2006Логика}), en=atomic formulas (\cite[def. 2.1.8]{Hinman2005Logic})]{atomic}.

  We will denote the set of all \term[ru=пропозициональные переменные (\cite[43]{КолмогоровДрагалин2006Логика}), en=propositional variables (\cite[27]{Church1956LogicVol1})]{propositional variables} by \( \op*{Prop} \). Similarly, we will denote the set of all \term[ru=формулы (\cite[43]{КолмогоровДрагалин2006Логика}), en=propositional formulas (\cite[41]{Mimram2020ProgramEqualsProof})]{propositional formulas} by \( \op*{Form} \), and by \( \op*{Atom}_\Sigma \) those that are \term{atomic}.

  We will also call the formulas \term{sentences} in relation to \hyperref[def:general_logic]{general logics}. This contrasts with predicate logic, where only specific formulas are called sentences --- see \cref{thm:fol_semantic_deduction_theorem}.
\end{definition}
\begin{comments}
  \item Within the metalanguage, we will denote abstract formulas via \( \varphi \), \( \psi \), \( \theta \) and other letters in accordance with \cref{rem:mathematical_logic_conventions/greek_alphabet}. This convention will later lead us to a formal definition of formula schemas in \cref{def:propositional_formula_schema}.

  \item We implicitly associate with each propositional formula an \hyperref[con:abstract_syntax_tree]{abstract syntax tree} --- see \cref{def:propositional_formula_ast}. The grammar of propositional formulas is unambiguous as shown via \cref{thm:propositional_formula_grammar/unambiguous}, which makes it possible to perform proofs via \fullref{thm:induction_on_rooted_trees}.

  \item If the root of the tree is a conjunction, we may refer to the formula itself as a conjunction, and similarly for other propositional connectives.

  \item Propositional formulas are implemented in \identifier{math.logic.propositional} as a special case of first-order formulas in accordance with \cref{rem:propositional_logic_as_first_order_logic}.
\end{comments}

\begin{proposition}\label{thm:propositional_formula_grammar}
  The \hyperref[def:formal_grammar]{grammar} of \hyperref[def:propositional_formula]{propositional formulas} has the following basic properties:
  \begin{thmenum}
    \thmitem{thm:propositional_formula_grammar/unambiguous} It is \hyperref[def:grammar_ambiguity]{unambiguous}.

    \thmitem{thm:propositional_formula_grammar/balanced} The \hyperref[def:formal_grammar/language]{generated language} has \hyperref[def:paired_delimiters]{balanced parentheses}.

    \thmitem{thm:propositional_formula_grammar/not_regular} The grammar is not \hyperref[def:chomsky_hierarchy/regular]{regular}.
  \end{thmenum}
\end{proposition}
\begin{proof}
  \SubProofOf{thm:propositional_formula_grammar/unambiguous} The proof in \cref{ex:natural_number_arithmetic_grammar/unambiguous} can be adapted straightforwardly.

  \SubProofOf{thm:propositional_formula_grammar/balanced} The proof in \cref{thm:dyck_language_balanced} can be adapted straightforwardly.

  \SubProofOf{thm:propositional_formula_grammar/not_regular} Follows from \cref{thm:propositional_formula_grammar/balanced} and \cref{thm:paired_delimiters_not_regular}.
\end{proof}

\begin{remark}\label{rem:propositional_formula_notation_conventions}
  We will use several following \enquote{abuse-of-notation} conventions regarding parentheses. These are only notations shortcuts in the \hyperref[con:metalogic]{metalanguage} and the formulas themselves (as abstract mathematical objects) are still assumed to contain parentheses that help them avoid syntactic ambiguity.

  \begin{thmenum}
    \thmitem{rem:propositional_formula_notation_conventions/outermost} We may skip the outermost parentheses in formulas with top-level \hyperref[def:propositional_alphabet/connectives]{connectives}, e.g. we may write \( \varphi \synwedge \psi \) rather than \( (\varphi \synwedge \psi) \).

    \thmitem{rem:propositional_formula_notation_conventions/associative} Because of the associativity of \( \synwedge \) and \( \synvee \), which is implied by \cref{def:propositional_denotation/function} and \cref{def:standard_boolean_functions}, we may skip the parentheses in chains like
    \begin{equation*}
      ( \ldots ((\varphi_1 \synwedge \varphi_2) \synwedge \varphi_3) \synwedge \ldots \synwedge \varphi_{n-1} ) \synwedge \varphi_n.
    \end{equation*}
    and instead write
    \begin{equation*}
      \varphi_1 \synwedge \varphi_2 \synwedge \ldots \synwedge \varphi_{n-1} \synwedge \varphi_n.
    \end{equation*}

    \thmitem{rem:propositional_formula_notation_conventions/additional} Although not formally necessary, for the sake of readability we may choose to add parentheses around certain formulas like
    \begin{equation*}
      \synneg \varphi \synvee \synneg \psi.
    \end{equation*}
    and instead write
    \begin{equation*}
      (\synneg \varphi) \synvee (\synneg \psi).
    \end{equation*}

    This latter convention is more useful for quantifiers in \hyperref[con:predicate_logic]{predicate logic}.
  \end{thmenum}
\end{remark}

\begin{definition}\label{def:conditional_formula}
  For a given \hyperref[def:propositional_formula]{conditional formula} \( \varphi \synimplies \psi \), we introduce the following terminology:
  \begin{thmenum}
    \thmitem{def:conditional_formula/sufficient_condition}\mcite[def. I.4.1]{Эдельман1975Логика} \( \varphi \) is a \term[ru=достаточное условие, en=sufficient condition (\cite[7]{Rosen2019DiscreteMathematics})]{sufficient condition} for \( \psi \).

    \thmitem{def:conditional_formula/necessary_condition}\mcite[def. I.4.1]{Эдельман1975Логика} \( \psi \) is a \term[ru=необходимое условие, en=necessary condition (\cite[7]{Rosen2019DiscreteMathematics})]{necessary condition} for \( \varphi \).

    \thmitem{def:conditional_formula/antecedent}\mcite[16]{Эдельман1975Логика} \( \varphi \) is the \term[ru=посылка, en=antecedent (\cite[71]{Church1956LogicVol1})]{antecedent} of the conditional \( \varphi \synimplies \psi \).

    \thmitem{def:conditional_formula/consequent}\mcite[16]{Эдельман1975Логика} \( \psi \) is the \term[ru=следствие, en=consequent (\cite[71]{Church1956LogicVol1})]{consequent} of \( \varphi \synimplies \psi \).

    \thmitem{def:conditional_formula/converse}\mcite[def. I.4.2]{Эдельман1975Логика} We call the formula \( \psi \synimplies \varphi \) the \term[ru=обратная (теорема), en=converse (\cite[9]{Rosen2019DiscreteMathematics})]{converse} of \( \varphi \synimplies \psi \).

    \thmitem{def:conditional_formula/contrapositive}\mcite[26]{Эдельман1975Логика} The formula \( \synneg \psi \synimplies \synneg \varphi \) is the \term[ru=контрапозиция, en=contrapositive (\cite[9]{Rosen2019DiscreteMathematics})]{contrapositive} of \( \varphi \synimplies \psi \).

    In \hyperref[con:classical_logic]{classical logic}, the contrapositive is \hyperref[def:semantic_equivalence]{equivalent} to the original formula due to \cref{thm:classical_equivalences/contrapositive}.

    \thmitem{def:conditional_formula/inverse}\mcite[def. I.4.3]{Эдельман1975Логика} We call the formula \( \synneg \varphi \synimplies \synneg \psi \) the \term[ru=противоположная (теорема), en=inverse (\cite[9]{Rosen2019DiscreteMathematics})]{inverse} of \( \varphi \synimplies \psi \).
  \end{thmenum}
\end{definition}

\begin{definition}\label{def:propositional_formula_ast}\mcite[11]{VanDalen2004LogicAndStructure}
  We implicitly associate with each propositional formula \( \varphi \) an \hyperref[con:abstract_syntax_tree]{abstract syntax tree} \( T(\varphi) \) as follows:
  \begin{thmenum}
    \thmitem{def:propositional_formula_ast/atomic} If \( \varphi \) is a propositional constant or variable, we define \( T(\varphi) \) as the \hyperref[def:canonical_singleton_tree]{canonical singleton tree} with label \( \varphi \).

    \thmitem{def:propositional_formula_ast/negation} If \( \varphi = \synneg \psi \), assuming we have already built \( T(\psi) \), we define \( T(\varphi) \) by \hyperref[def:ordered_tree_grafting_product]{grafting} \( T(\psi) \) to a new root labeled with \( \synneg \):
    \begin{equation*}
      \includegraphics[page=1]{output/def__propositional_formula_ast}
    \end{equation*}

    \thmitem{def:propositional_formula_ast/connective} If \( \varphi = \psi \syncirc \theta \), assuming we have built \( T(\psi) \) and \( T(\theta) \), we define \( T(\varphi) \) by grafting \( T(\psi) \) and \( T(\theta) \) to a new root labeled with \( \syncirc \):
    \begin{equation*}
      \includegraphics[page=2]{output/def__propositional_formula_ast}
    \end{equation*}
  \end{thmenum}
\end{definition}
\begin{comments}
  \item The \hyperref[def:rooted_tree/leaf]{leaves} of the tree are variables and constants, while every other node is either a binary connective or negation.
\end{comments}

\begin{example}\label{ex:def:propositional_formula_ast}
  We list examples of \hyperref[def:propositional_formula_ast]{abstract syntax trees for propositional formulas}:
  \begin{thmenum}
    \thmitem{ex:def:propositional_formula_ast/lnc} The formula \( \synneg (\varphi \synwedge \synneg \varphi) \) expressing the law of non-contradiction \eqref{eq:thm:intuitionistic_tautologies/lnc} has the following AST:
    \begin{equation*}
      \includegraphics[page=1]{output/ex__def__propositional_formula_ast}
    \end{equation*}

    \thmitem{ex:def:propositional_formula_ast/associative_conjunction} In \cref{def:propositional_denotation/function} we will define semantics for conjunction and disjunction so that they become associative as Boolean operators. Following our general discussion of such binary operations in \cref{rem:binary_operation_syntax_trees}, we will generally conflate the formula
    \begin{equation*}
      ((\varphi \synwedge \psi) \synwedge \theta)
    \end{equation*}
    with AST
    \begin{equation*}
      \includegraphics[page=2]{output/ex__def__propositional_formula_ast}
    \end{equation*}
    and the formula
    \begin{equation*}
      ((\varphi \synwedge \psi) \synwedge \theta)
    \end{equation*}
    with AST
    \begin{equation*}
      \includegraphics[page=3]{output/ex__def__propositional_formula_ast}
    \end{equation*}
  \end{thmenum}
\end{example}

\paragraph{Subformulas}

\begin{definition}\label{def:propositional_subformula}\mcite[def. 1.1.7]{VanDalen2004LogicAndStructure}
  For every \hyperref[def:propositional_formula]{propositional formula}, we define the set of \term{subformulas} as follows:
  \begin{equation*}
    \op*{Subform}(\varphi) \coloneqq \begin{cases}
      \set{ \varphi },                                                        &\varphi \in \op*{Atom}, \\
      \set{ \varphi } \bigcup \op*{Subform}(\psi),                            &\varphi = \synneg \psi, \\
      \set{ \varphi } \bigcup \op*{Subform}(\psi) \cup \op*{Subform}(\theta), &\varphi = \psi \syncirc \theta, {\syncirc} \in \op*{Conn}.
    \end{cases}
  \end{equation*}

  We denote by \( \op*{Var}(\varphi) \) those subterms of \( \varphi \) that are propositional variables.
\end{definition}

\begin{lemma}\label{thm:propositional_subformula_lemma}
  If the formula \( \psi \) is a \hyperref[def:formal_language/substring]{substring} of the formula \( \varphi \), we have the following possibilities:
  \begin{thmenum}
    \thmitem{thm:propositional_subformula/variable} \( \varphi \) is a variable or constant and \( \psi = \varphi \).
    \thmitem{thm:propositional_subformula/negation_self} \( \varphi = \synneg \theta \) and \( \psi \) coincides with \( \varphi \).
    \thmitem{thm:propositional_subformula/negation} \( \varphi = \synneg \theta \) and \( \psi \) is a subformula of \( \theta \).
    \thmitem{thm:propositional_subformula/connective_self} \( \varphi = (\theta \syncirc \chi) \) and \( \psi \) coincides with \( \varphi \).
    \thmitem{thm:propositional_subformula/connective_left} \( \varphi = (\theta \syncirc \chi) \) and \( \psi \) is a subformula of \( \theta \).
    \thmitem{thm:propositional_subformula/connective_right} \( \varphi = (\theta \syncirc \chi) \) and \( \psi \) is a subformula of \( \chi \) but not \( \theta \).
  \end{thmenum}
\end{lemma}
\begin{proof}
  We use \fullref{thm:induction_on_rooted_trees} on \( \varphi \):
  \begin{itemize}
    \item If \( \varphi \) is a variable or constant, it is a single lexeme, and the only possible substring that is a formula is \( \varphi \) itself. This corresponds to \cref{thm:propositional_subformula/variable}.
    \item If \( \varphi = \synneg \theta \) and if the inductive hypothesis holds for \( \theta \), we have the following possibilities:
    \begin{itemize}
      \item If \( \psi = \varphi \), then \cref{thm:propositional_subformula/negation_self} holds.
      \item If \( \psi = \synneg \), it is a substring of \( \varphi \), but not itself a formula.
      \item If \( \psi \) is a substring of \( \theta \), we apply the inductive hypothesis --- then \cref{thm:propositional_subformula/negation} holds.
    \end{itemize}

    \item If \( \varphi = (\theta \syncirc \chi) \), where the inductive hypothesis holds for \( \theta \) and \( \chi \), we have the following possibilities:
    \begin{itemize}
      \item If \( \psi = \varphi \), then \cref{thm:propositional_subformula/connective_self} holds.
      \item If \( \psi \) is a substring of \( \theta \), then \cref{thm:propositional_subformula/connective_left} holds.
      \item If \( \psi \) is a substring of \( \chi \) but not of \( \theta \), then \cref{thm:propositional_subformula/connective_right} holds.
      \item If \( \psi = (\theta \syncirc \psi_\chi \), where \( \psi_\chi \) is a prefix of \( \chi \), then \( \psi \) has unbalanced parentheses, which contradicts \cref{thm:propositional_formula_grammar/balanced}, and thus \( \psi \) is not a formula.
      \item Similarly, if \( \psi = \psi_\theta \syncirc \chi) \), where \( \psi_\theta \) is a suffix of \( \theta \), then again \( \psi \) has unbalanced parentheses.
      \item If \( \psi = \psi_\theta \syncirc \psi_\chi \), where \( \psi_\theta \) is a suffix of \( \theta \) and \( \psi_\chi \) is a prefix of \( \chi \), then \( \psi \) is again not a formula because it is not wrapped in parentheses.
    \end{itemize}
  \end{itemize}
\end{proof}

\begin{proposition}\label{thm:propositional_formula_characterization}
  The \hyperref[def:formal_language/substring]{substring} \( \psi \) of the formula \( \varphi \) is a \hyperref[def:propositional_subformula]{subformula} of \( \varphi \) if and only if \( \psi \) is itself a formula.
\end{proposition}
\begin{comments}
  \item An analogous statement holds in first-order logic --- see \cref{thm:fol_formula_characterization} --- but not in higher-order logic because the latter is based on \( \muplambda \)-terms, where \cref{thm:lambda_subterm_characterization} holds instead.
\end{comments}
\begin{proof}
  \SufficiencySubProof Straightforward.
  \NecessitySubProof Suppose that \( \psi \) is itself a formula. Then \cref{thm:propositional_subformula_lemma} implies that it falls into one of the cases of \cref{def:propositional_subformula}, and is thus a subformula of \( \varphi \).
\end{proof}

\begin{proposition}\label{thm:propositional_ast_subformula}
  The formula \( \psi \) is a \hyperref[def:propositional_subformula]{subformula} of \( \varphi \) if and only if the \hyperref[def:propositional_formula_ast]{abstract syntax tree} \( T(\varphi) \) has a \hyperref[def:tree/subtree]{subtree} \hyperref[def:labeled_tree/homomorphism]{isomorphic} to \( T(\psi) \).
\end{proposition}
\begin{proof}
  Trivial.
\end{proof}

\paragraph{Intuitionistic propositional semantics}

\begin{definition}\label{def:truth_value_algebra}\mimprovised
  In order to determine whether a sentence holds under given circumstances, we must have a set of \term[ru=истинностное значение (\cite[def. 1.1.12]{Герасимов2014Вычислимость}), en=truth value (\cite[\S 6.5.9]{Mimram2020ProgramEqualsProof})]{truth values}, for example the Boolean values \( T \) and \( F \) from \cref{con:boolean_value}. The aforementioned set is naturally a \hyperref[def:boolean_algebra]{Boolean algebra}. The most general setting we will consider are \hyperref[def:heyting_algebra]{Heyting algebras}.

  When working with \hyperref[def:institution/models]{models} and notions related to them, we will presume that an underlying Heyting algebra \( \BbbH \) is fixed. In accordance with \cref{rem:mathematical_logic_conventions/propositional_constants}, we will denote the top and bottom element of \( \BbbH \) by \( T \) and \( F \).

  \begin{thmenum}
    \thmitem{def:truth_value_algebra/intuitionistic} In the general case we consider here, we will say that the semantics is \term{intuitionistic} since it formalizes \hyperref[con:intuitionistic_logic]{intuitionistic logic}.

    \thmitem{def:truth_value_algebra/classical} In the special case where the \hyperref[def:truth_value_algebra]{truth value algebra} is the two-element Boolean algebra \( \set{ T, F } \), we will call the semantics \term{classical} since it formalizes \hyperref[con:classical_logic]{classical logic}, or \term{Boolean} since it is described via \hyperref[def:boolean_function]{Boolean functions}.

    For the reasons outlined in \cref{rem:intuitionistic_first_order_semantics}, we will assume classical semantics for first-order logic.

    \thmitem{def:truth_value_algebra/topological} If the Heyting algebra is a \hyperref[def:topological_space]{topology}, as in \cref{ex:def:heyting_algebra/topology}, we obtain \term{topological semantics}.

    \thmitem{def:truth_value_algebra/fuzzy} If the Heyting algebra is the unit interval \( [0, 1] \), as in \cref{ex:def:heyting_algebra/totally_ordered}, we obtain \term{fuzzy semantics}, more often called \term[en=fuzzy logic (\cite[17]{Rosen2019DiscreteMathematics})]{fuzzy logic}.
  \end{thmenum}
\end{definition}
\begin{comments}
  \item Fuzzy logic is based on \hyperref[def:fuzzy_set]{fuzzy sets}, which we discuss in \fullref{sec:fuzzy_sets}. A brief description of fuzzy logic can be found in \cite[17]{Rosen2019DiscreteMathematics}.
\end{comments}

\begin{definition}\label{def:propositional_interpretation}\mimprovised
  A \term[ru=интерпретация (\cite[def. 1.1.10]{Герасимов2014Вычислимость})]{propositional interpretation} in a \hyperref[def:truth_value_algebra]{truth value algebra} \( \BbbH \) is any function \( I: \op*{Prop} \to \BbbH \) assigning truth values to propositional variables.
\end{definition}
\begin{comments}
  \item Propositional interpretations act as models in the propositional institution. Unlike in general institutions, however, we do not refer to an arbitrary propositional interpretation \( I \) as a \enquote{model}; see \cref{rem:model_theory_structure_terminology} for a broader discussion.

  \item The terms \enquote{interpretation} and \enquote{denotation} are used inconsistently in the literature; see \cref{rem:model_theory_structure_terminology}.
\end{comments}

\begin{definition}\label{def:propositional_denotation}\mimprovised
  Fix a \hyperref[def:truth_value_algebra]{truth value algebra} \( \BbbH \). For any interpretation \( I \), we define the \term{denotation} of the formula \( \varphi \) recursively as follows:
  \begin{empheq}[left={\Bracks{\varphi}_I} \coloneqq \empheqlbrace]{align}
    &T,                                         &&\varphi = \syntop,                                         \label{eq:def:propositional_denotation/top} \\
    &F,                                         &&\varphi = \synbot,                                         \label{eq:def:propositional_denotation/bot} \\
    &I(\varphi),                                &&\varphi \in \op*{Prop},                                    \label{eq:def:propositional_denotation/prop} \\
    &\oline{\Bracks{\psi}_I},                   &&\varphi = \synneg \psi,                                    \label{eq:def:propositional_denotation/neg} \\
    &\Bracks{\psi}_I \relcirc \Bracks{\theta}_I &&\varphi = \psi \syncirc \theta, {\syncirc} \in \op*{Conn}, \label{eq:def:propositional_denotation/conn}
  \end{empheq}
  where \( \relcirc \) denotes \hyperref[def:heyting_algebra]{Heyting algebra} operation corresponding to the connective \( \syncirc \).

  \begin{thmenum}
    \thmitem{def:propositional_denotation/function} If \( \op*{Var}(\varphi) \subseteq \set{ p_1, \ldots, p_n } \), the valuation \( \Bracks{\varphi}_I \) only depends on the particular values \( I(p_1), \ldots, I(p_n) \) of \( I \). Hence, if the variables are clear from the context, we obtain a Boolean function
    \begin{equation*}
      \begin{aligned}
        &\Bracks{\varphi}: \BbbH^n \to \BbbH, \\
        &\Bracks{\varphi}(x_1, \ldots, x_n) \coloneqq \Bracks{\varphi}_I,
      \end{aligned}
    \end{equation*}
    where \( I \) is any interpretation such that \( I(p_k) = x_k \) for \( k = 1, \ldots, n \).

    Unless otherwise noted, we assume that \( p_1, \ldots, p_n \) are precisely the variables of \( \varphi \) ordered lexicographically.
  \end{thmenum}
\end{definition}
\begin{comments}
  \item The term \enquote{denotation} is inspired by Russell's theory of denotations discussed in \cref{con:denotation}. Along with \enquote{interpretation}, it is used inconsistently in the literature. See \cref{rem:model_theory_structure_terminology}.
\end{comments}

\begin{concept}\label{con:improper_symbol}
  In \cref{def:propositional_denotation}, we have defined valuations for \hyperref[def:propositional_formula]{propositional formulas}. Based on this definition, the symbols used to define the formal grammar of formulas can be divided into two collections:
  \begin{thmenum}
    \thmitem{con:improper_symbol/proper} The propositional variables and the constants \( \syntop \) and \( \synbot \) can be regarded, in the words of \incite[32]{Church1956LogicVol1}, as \enquote{having meaning in isolation}.

    That is to mean that, given a fixed \hyperref[def:propositional_interpretation]{propositional interpretation}, any variable and any constant has a definite valuation.

    Following Church, we call such symbols \term[en=proper (symbol) (\cite[32]{Church1956LogicVol1})]{proper}.

    \thmitem{con:improper_symbol/improper} On the other hand, the rest of the symbols from the \hyperref[def:propositional_alphabet]{propositional alphabet} \enquote{have no meaning in isolation} but \enquote{combine with proper symbols (one or more) to form longer expressions that do have meaning in isolation.}

    Indeed, the conjunction symbol \( {\synwedge} \) has no valuation by itself, but the conjunction formula \( \varphi \synwedge \psi \) does.

    Again, following Church, we call such symbols \term[en=improper (symbol) (\cite[32]{Church1956LogicVol1})]{improper}.
  \end{thmenum}
\end{concept}
\begin{comments}
  \item An alternative to the adjective \enquote{improper} is \enquote{reserved}. In programming languages, certain words are marked as reserved and cannot be used for naming. For example, \cite[\S 6.4.1]{ISO:9899:2018} lists several dozens of words that are called \enquote{reserved} and cannot be used as identifiers in the programming language C.

  \item It is common to call improper symbols \enquote{logical}, but this is unfortunately ambiguous --- see \cref{rem:logical_symbol_terminology}. Still, with some disambiguation, we find useful the adjectives \enquote{logical} and \enquote{nonlogical} when discussing higher-order logic constants in \fullref{sec:higher_order_logic}.
\end{comments}

\begin{remark}\label{rem:logical_symbol_terminology}
  It is common to call \hyperref[con:improper_symbol]{improper symbols} \enquote{logical}, distinguishing them from \enquote{nonlogical} symbols that appear in a signature. This terminology is unfortunately ambiguous, as discussed in \cite{MathSE:are_variables_logical_symbols}. For example, for the \hyperref[def:first_order_syntax]{syntax of first-order logic}, the terms are used as follows:
  \begin{itemize}
    \item \incite[178]{Gentzen1935LogischeSchließen}, when presenting first-order logic, calls \enquote{Logische Zeichen} (\enquote{logical symbols}) the \hyperref[def:propositional_alphabet/negation]{negation}, \hyperref[def:propositional_alphabet/connectives]{propositional connectives} and \hyperref[def:predicate_logic_alphabet/quantifiers]{quantifiers}.

    He separately lists other syntactic classes:
    \begin{itemize}
      \item Two \hyperref[def:propositional_alphabet/constants]{logical constants}.
      \item \enquote{Hilfszeichen} (\enquote{auxiliary symbols}) consisting of two \hyperref[def:propositional_alphabet/parentheses]{parentheses} and a \hyperref[def:sequent]{sequent relation symbol}.
      \item \hyperref[con:variable]{Variables}, separated into \hyperref[con:variable_binding]{free} object variables, bound object variables and propositional variables.
    \end{itemize}

    \item \incite[69]{Kleene1971Metamathematics}, \incite[57]{КолмогоровДрагалин2006Логика}, \incite[140]{Мальцев1970АлгебраическиеСистемы} and \incite[86]{Герасимов2014Вычислимость} extend Gentzen's list of logical symbols with quantifiers. Neither of them uses logical constants, but it is reasonable to expect them to regard constants as logical symbols.

    \item \incite*[2]{Hinman2005Logic} writes
    \begin{displayquote}
      \textellipsis the symbols will fall into two classes: \textit{\textbf{logical symbols}}, which have a fixed meaning, and \textit{\textbf{non-logical symbols}}, which are subject to multiple interpretations.
    \end{displayquote}

    As stated, this principle suggests that constants, being a part of a signature and having a predefined interpretation, are logical symbols, while variables, whose interpretation depends on a \hyperref[def:fol_variable_assignment]{variable assignment}, are nonlogical symbols.

    Later, in \cite[def. 2.1.2]{Hinman2005Logic}, Hinman designates as logical symbols a limited set of quantifiers and connectives, as well as the variables and the equality symbol. Constants, functional and relational symbols are instead designated as nonlogical.
  \end{itemize}
\end{remark}

\begin{remark}\label{rem:improper_symbols_and_parsing}
  We generally disallow \hyperref[con:improper_symbol]{improper symbols} as variables because the usefulness of a formula such as \( \synneg \synneg \) is highly questionable.

  There are however also technical considerations for this. For example, it is important for us to work with \hyperref[def:grammar_ambiguity]{syntactically unambiguous} \hyperref[def:formal_grammar]{grammars}. In more complicated languages, allowing improper symbols as variable identifiers can actually lead to syntactic ambiguity. For example, the specification \cite[\S 6.4.1]{ISO:9899:2018} of the programming language C lists several dozens of words that are called \enquote{reserved} and cannot be used as identifiers.

  We have defined the propositional formulas in \cref{def:propositional_formula} so that they are syntactically unambiguous. In fact, even if we allow improper symbols such as \( \synneg \) or \( {\synwedge} \) to denote variables, the grammar would again be unambiguous because we would be able to determine that \( \synneg \synneg \) is a negation of the variable \( \synneg \) and that \( (\synwedge \synwedge \synwedge) \) is a conjunction of the variable \( {\synwedge} \) with itself.

  Even if the grammar remains unambiguous, however, parsing becomes more complicated because the context in which a symbol occurs becomes significant\fnote{Even if parsing requires additional context, generating expressions does not, so the language remains \hyperref[def:chomsky_hierarchy/context_free]{context-free}.} --- the parser must at all times know whether to expect at a certain position a variable symbol or the corresponding improper symbol.

  We even disallow \hyperref[con:logical_system_signature]{signature symbols} to clash with variables or placeholders (except in the special case of \hyperref[def:quantifiable_type]{quantifiable types} in \hyperref[def:higher_order_logic]{higher-order logic}, where we instead disallow type variables).
\end{remark}

\begin{definition}\label{def:propositional_institution}\mimprovised
  For a fixed \hyperref[def:truth_value_algebra]{truth value algebra} \( \BbbH \), \hyperref[def:propositional_interpretation]{propositional interpretations} naturally give rise to an \hyperref[def:institution]{institution} as follows:
  \begin{thmenum}
    \thmitem{def:propositional_institution/signatures} For the category of \hyperref[def:institution/signatures]{signatures}, fix some symbol \( \anon \) and let \( \cat{Sign} \) be the \hyperref[def:discrete_category]{discrete category} on \( \set{ \anon } \).

    \thmitem{def:propositional_institution/sentences} Define the \hyperref[def:institution/sentences]{sentence functor} as \( \anon \mapsto \op*{Form} \).

    \thmitem{def:propositional_institution/models} Let \( \BbbI \) be the discrete category on the \hyperref[def:set_of_all_functions]{function set} \( \fun(\op*{Prop}, \BbbH) \). The \hyperref[def:institution/models]{model functor} can then be described as \( \anon \mapsto \BbbI \).

    \thmitem{def:propositional_institution/satisfaction} Finally, let the \hyperref[def:institution/satisfaction]{satisfaction} relation \( I \vDash_{\anon} \varphi \) hold if and only if \( \Bracks{\varphi}_I = T \).
  \end{thmenum}
\end{definition}
\begin{comments}
  \item The cardinality restrictions discussed in \cref{rem:language_alphabet_cardinality/category} ensure that the formulas are hereditarily finite sets and thus belong to every \hyperref[def:grothendieck_universe]{Grothendieck universe}. Thus, we do not have to concerns ourselves about the \hyperref[def:category_size]{size} of the above categories.
\end{comments}
\begin{defproof}
  We must show that this is indeed an institution, which requires verifying \eqref{eq:def:institution/satisfaction}. But this is vacuous because there are no nontrivial signature morphisms.
\end{defproof}

\begin{definition}\label{def:propositional_semantics}\mimprovised
  By the \enquote{\hyperref[con:syntax_semantics_duality]{semantics} of propositional logic} we mean the \hyperref[def:propositional_institution]{propositional institution} and all related notions like \hyperref[def:institution/models]{models}, \hyperref[def:institution/satisfaction]{satisfaction}, \hyperref[def:institutional_entailment]{semantic entailment} and \hyperref[def:semantic_equivalence]{semantic equivalence}.

  \begin{thmenum}
    \thmitem{def:propositional_semantics/satisfaction} We use the terminology from \cref{def:institutional_satisfaction} regarding satisfaction and validity.

    However, as per \cref{rem:model_theory_structure_terminology}, unlike in general institutions, we refer to interpretations as \enquote{models} only with respect to a theory.

    Instead, we say that \( I \) is a \term[en=model (\cite[def. 1.4.1(i)]{Hinman2005Logic})]{propositional model} \hi{of} the set of \hyperref[def:propositional_formula]{propositional formulas} \( \Gamma \) and write \( I \vDash \Gamma \) if and only if \( \Bracks{\varphi}_I = T \) for every formula \( \varphi \in \Gamma \).

    \thmitem{def:propositional_semantics/entailment} We use the terminology and notation from \cref{def:institutional_entailment} regarding entailment: \( \Gamma \vDash \varphi \) if and only if, whenever \( I \) satisfies \( \Gamma \), it also satisfies \( \varphi \).
  \end{thmenum}
\end{definition}

\begin{remark}\label{rem:classical_propositional_interpretations}
  Note that a propositional interpretation may be used to define valuations different from the one in \cref{def:propositional_denotation}. For example, \hyperref[con:minimal_logic]{minimal logic} as defined in \cref{def:minimal_propositional_semantics} requires a slight variation.

  Thus, an interpretation cannot, by itself, be \hyperref[con:classical_logic]{classical} or \hyperref[con:intuitionistic_logic]{intuitionistic}.

  On the other hand, (institutional) models can be classical, and propositional models are interpretations, so we are free to use the term \enquote{classical model} when referring to interpretations in an appropriate context.
\end{remark}

\begin{example}\label{ex:def:propositional_semantics}
  We list some examples related to \hyperref[def:propositional_semantics]{propositional semantics}:
  \begin{thmenum}
    \thmitem{ex:def:propositional_semantics/trivial} Within the one-element Heyting algebra, where \( T = F \), every formula is satisfied because there is simply no \enquote{non-true} truth value. \Cref{thm:inconsistent_lindenbaum_tarski_algebra} implies that this precisely is the \hyperref[def:lindenbaum_tarski_algebra]{Lindenbaum-Tarski algebra} of any \hyperref[def:consistent_set_of_sentences]{inconsistent} set of formulas.

    \thmitem{ex:def:propositional_semantics/three_valued_lem} We show in \cref{thm:classical_tautologies/lem} that the law of the excluded middle \eqref{eq:thm:classical_tautologies/lem} holds in classical logic.

    We will give a simple counterexample when \( \BbbH \) is the more general three-valued Heyting algebra from \cref{ex:def:heyting_algebra/three_valued}, where \( F < N < T \).

    Let \( I \) be a \hyperref[def:propositional_denotation]{propositional interpretation} such that \( I(\synp) = N \). Then the valuation of \eqref{eq:thm:classical_tautologies/lem} is
    \begin{equation*}
      \Bracks{\synp \synvee \synneg \synp}_I
      =
      \Bracks{\synp}_I \vee \oline{\Bracks{\synp}_I}
      =
      N \vee F
      =
      N.
    \end{equation*}

    Thus, the law does not hold with respect to \( \BbbH \).

    \thmitem{ex:def:propositional_semantics/topological_semantics_lem} Similarly, the law of the excluded middle \eqref{eq:thm:classical_tautologies/lem} does not hold with respect to \hyperref[def:truth_value_algebra/topological]{topological semantics} based on the standard topology of \( \BbbR \).

    Indeed, let \( U \) be an open set. For any \hyperref[def:propositional_denotation]{propositional interpretation} \( I \) such that \( I(\synp) = U \), we have
    \begin{equation*}
      \Bracks{\synp \synvee \synneg \synp}_I
      =
      \Bracks{\synp}_I \cup \oline{\Bracks{\synp}_I}
      =
      U \cup \oline{U}
      =
      U \cup \Int(\BbbR \setminus U).
    \end{equation*}

    If \( U \) is empty, then \( \Bracks{\synp \synvee \synneg \synp}_I = \BbbR \) and the law holds in this case. If \( U \) is the open unit interval \( (0, 1) \), then \( \Bracks{\synp \synvee \synneg \synp}_I = \BbbR \setminus \set{ 0, 1 } \) and the law does not hold.
  \end{thmenum}
\end{example}

\begin{proposition}\label{thm:intuitionistic_equivalences}
  We list some \hyperref[def:truth_value_algebra/intuitionistic]{intuitionistic} propositional \hyperref[def:semantic_equivalence]{semantic equivalences}:
  \begin{thmenum}
    \thmitem{thm:intuitionistic_equivalences/negation_bottom} Negation can be expressed as a conditional formula whose \hyperref[def:conditional_formula/consequent]{consequent} is the falsum:
    \begin{equation}\label{eq:thm:intuitionistic_equivalences/negation_bottom}
      \mathllap{\synneg \varphi} \gleichstark \mathrlap{\varphi \synimplies \synbot.}
    \end{equation}

    \thmitem{thm:intuitionistic_equivalences/top_elim} We can eliminate verum from the antecedent of conditional formulas:
    \begin{equation}\label{eq:thm:intuitionistic_equivalences/top_elim}
      \mathllap{\syntop \synimplies \varphi} \gleichstark \mathrlap{\varphi.}
    \end{equation}

    \thmitem{thm:intuitionistic_equivalences/contradiction} Falsum is equivalent to a conjunction of a formula and its negation:
    \begin{equation}\label{eq:thm:intuitionistic_equivalences/contradiction}
      \mathllap{\varphi \synwedge \synneg \varphi} \gleichstark \mathrlap{\synbot.}
    \end{equation}
  \end{thmenum}
\end{proposition}
\begin{comments}
  \item For every formula \( \varphi \), \eqref{eq:thm:intuitionistic_equivalences/negation_bottom} is a different axiom, and we refer to \eqref{eq:thm:intuitionistic_equivalences/negation_bottom} itself as an \enquote{axiom schema}. We will formalize schemas in \fullref{sec:propositional_axiomatic_derivations}.

  \item Both \eqref{eq:thm:intuitionistic_equivalences/negation_bottom} and \eqref{eq:thm:intuitionistic_equivalences/top_elim} hold in \hyperref[def:minimal_propositional_semantics]{minimal semantics} since their proofs do not use any semantic properties of the falsum.
\end{comments}
\begin{proof}
  \SubProofOf{thm:intuitionistic_equivalences/negation_bottom} Follows from \cref{def:heyting_algebra/pseudocomplement}.

  \SubProofOf{thm:intuitionistic_equivalences/top_elim} Follows from \cref{thm:def:heyting_algebra/top_left}.

  \SubProofOf{thm:intuitionistic_equivalences/contradiction} For any interpretation \( I \), we have
  \begin{equation*}
    \Bracks{ \varphi \synwedge \synneg \varphi }_I
    =
    \Bracks{ \varphi }_I \synwedge (\Bracks{ \varphi }_I \rightarrow F)
    \reloset {\eqref{eq:def:heyting_algebra/axioms/modus_ponens}} =
    \Bracks{ \varphi }_I \wedge F
    =
    F
    =
    \Bracks{\synbot}_I.
  \end{equation*}
\end{proof}

\begin{definition}\label{def:propositional_tautology}
  We say that \( \varphi \) is a \term[ru=пропозициональная тавтология (\cite[44]{КолмогоровДрагалин2006Логика}), en=tautology (\cite[def. 1.2.4]{VanDalen2004LogicAndStructure})]{propositional tautology} if any of the following equivalent conditions hold:
  \begin{thmenum}
    \thmitem{def:propositional_tautology/interpretations}\mcite[def. 1.2.4(i)]{VanDalen2004LogicAndStructure} We have \( \Bracks{\varphi}_I = T \) for every interpretation \( I \).
    \thmitem{def:propositional_tautology/entailment}\mcite[def. 1.2.4(ii)]{VanDalen2004LogicAndStructure} We have the entailment \( \vDash \varphi \).
    \thmitem{def:propositional_tautology/equivalence}\mimprovised We have the equivalence \( \varphi \gleichstark \syntop \).
  \end{thmenum}
\end{definition}
\begin{comments}
  \item In \cite[419]{Tarski1983LogicalConsequence}, an English translation of Tarski, \enquote{tautology} is described as \enquote{a statement which \enquote*{says nothing about reality}}.
\end{comments}

\begin{definition}\label{def:propositional_contradiction}
  Dually to \cref{def:propositional_tautology}, we say that \( \varphi \) is a \term[en=contradictory (formula) (\cite[28]{Kleene2002Logic})]{propositional contradiction} if any of the following equivalent conditions hold:
  \begin{thmenum}
    \thmitem{def:propositional_contradiction/interpretations}\mcite[def. 1.4.1(i)]{Hinman2005Logic} We have \( \Bracks{\varphi}_I = F \) for every interpretation \( I \).
    \thmitem{def:propositional_contradiction/entailment}\mimprovised We have the entailment \( \varphi \vDash \synbot \).
    \thmitem{def:propositional_contradiction/equivalence}\mimprovised We have the equivalence \( \varphi \gleichstark \synbot \).
  \end{thmenum}
\end{definition}

\begin{proposition}\label{thm:intuitionistic_tautologies}
  We list some \hyperref[def:truth_value_algebra/intuitionistic]{intuitionistic} propositional \hyperref[def:propositional_tautology]{tautologies}:
  \begin{thmenum}
    \thmitem{thm:intuitionistic_tautologies/self} Every formula implies itself:
    \begin{equation}\label{eq:thm:intuitionistic_tautologies/self}
      \varphi \synimplies \varphi.
    \end{equation}

    \thmitem{thm:intuitionistic_tautologies/dni} Any formula implies its double negation:
    \begin{equation}\label{eq:thm:intuitionistic_tautologies/dni}
      \varphi \synimplies \neg \neg \varphi \tag{\ensuremath{ \logic{DNI}_A }}
    \end{equation}

    \enquote{DNI} stands for \enquote{double negation introduction}.

    \thmitem{thm:intuitionistic_tautologies/efq} Falsum implies anything:
    \begin{equation}\label{eq:thm:intuitionistic_tautologies/efq}
      \synbot \synimplies \varphi \tag{\ensuremath{ \logic{EFQ}_A }}
    \end{equation}

    \enquote{EFQ} stands for \enquote{ex falso quodlibet}. It is a shortening of \enquote{ex falso sequitur quodlibet}, which \incite[35]{TroelstraSchwichtenberg2000BasicProofTheory} translate from Latin as \enquote{from a falsehood follows whatever you like}. It is also known as the \enquote{the principle of explosion} because it allows us to show that intuitionistic semantics is \hyperref[def:paraconsistent_consequence_operator]{explosive} --- see \cref{thm:intuitionistic_semantics_are_explosive}.

    Both \enquote{ex falso quodlibet} and \enquote{principle of explosion} are used by \incite[47]{Mimram2020ProgramEqualsProof}.

    \thmitem{thm:intuitionistic_tautologies/ecq} A \hyperref[def:propositional_contradiction]{contradiction} implies anything:
    \begin{equation}\label{eq:thm:intuitionistic_tautologies/ecq}
      (\varphi \synwedge \neg \varphi) \synimplies \psi \tag{\ensuremath{ \logic{ECQ}_A }}
    \end{equation}

    \enquote{EFQ} stands for \enquote{ex contradictione quodlibet} --- a variation of the last principle without an explicit falsum. The term is used by \incite[2]{DienerMcKubreJordens2020MaterialImplication}.

    \thmitem{thm:intuitionistic_tautologies/lnc} A formula and its negation cannot both hold:
    \begin{equation}\label{eq:thm:intuitionistic_tautologies/lnc}
      \synneg (\varphi \synwedge \synneg \varphi). \tag{\ensuremath{ \logic{LNC}_A }}
    \end{equation}

    \enquote{LCN} stands for \enquote{law of non-contradiction}.
  \end{thmenum}
\end{proposition}
\begin{comments}
  \item Some other useful tautologies hold in classical logic --- see \cref{thm:classical_tautologies}.

  \item Both \eqref{eq:thm:intuitionistic_tautologies/dni} and \eqref{eq:thm:intuitionistic_tautologies/lnc} hold under \hyperref[def:minimal_propositional_semantics]{minimal semantics}, but our semantic proofs for them are not valid in minimal logic. We will demonstrate simple syntactic proofs in \cref{thm:syntactic_minimal_tautologies}.
\end{comments}
\begin{proof}
  Fix an interpretation \( I \).

  \SubProofOf{thm:intuitionistic_tautologies/self}
  \begin{equation*}
    \Bracks{\varphi \synimplies \varphi}_I
    \reloset {\eqref{eq:def:propositional_denotation/conn}} =
    \Bracks{\varphi}_I \synimplies \Bracks{\varphi}_I
    \reloset {\ref{thm:def:heyting_algebra/leq}} =
    T.
  \end{equation*}

  \SubProofOf{thm:intuitionistic_tautologies/dni}
  \begin{equation*}
    \Bracks{\varphi \synimplies \neg \neg \varphi}_I
    \reloset {\eqref{eq:def:propositional_denotation/conn}} =
    \Bracks{\varphi} \rightarrow \Bracks{\neg \neg \varphi}_I
    \reloset {\eqref{eq:def:propositional_denotation/neg}} =
    \Bracks{\varphi}_I \rightarrow \underbrace{\oline{\oline{\Bracks{\varphi}_I}}}_{\mathclap{\leq \Bracks{\varphi}_I \T*{by} \ref{thm:def:heyting_algebra/dni}}}
    \reloset {\ref{thm:def:heyting_algebra/leq}}
    T
  \end{equation*}

  \Cref{thm:def:heyting_algebra/dni} via \ref{thm:def:heyting_algebra/leq} implies that the latter equals the top element \( T \).

  \SubProofOf{thm:intuitionistic_tautologies/efq}
  \begin{equation*}
    \Bracks{\synbot \synimplies \varphi}_I
    \reloset {\eqref{eq:def:propositional_denotation/conn}} =
    \Bracks{\synbot}_I \rightarrow \Bracks{\varphi}_I
    \reloset {\ref{thm:def:heyting_algebra/leq}} =
    T.
  \end{equation*}

  \SubProofOf{thm:intuitionistic_tautologies/ecq}
  \begin{equation*}
    \Bracks{(\varphi \synwedge \neg \varphi) \synimplies \psi}_I
    \reloset {\eqref{eq:def:propositional_denotation/conn}} =
    \Bracks{\varphi \synwedge \neg \varphi}_I \rightarrow \Bracks{\synimplies \psi}_I
    \reloset {\eqref{eq:thm:intuitionistic_equivalences/contradiction}} =
    \Bracks{\synbot}_I \rightarrow \Bracks{\synimplies \psi}_I
    \reloset {\eqref{eq:thm:intuitionistic_tautologies/efq}} =
    T.
  \end{equation*}

  \SubProofOf{thm:intuitionistic_tautologies/lnc}
  \begin{equation*}
    \Bracks{\synneg (\varphi \synwedge \synneg \varphi)}_I
    \reloset {\eqref{eq:def:propositional_denotation/neg}} =
    \oline{\Bracks{\varphi \synwedge \synneg \varphi}_I}
    \reloset {\eqref{eq:thm:intuitionistic_equivalences/contradiction}} =
    \oline{\Bracks{\synbot}_I}
    \reloset {\eqref{eq:def:propositional_denotation/bot}} =
    \oline{F}
    \reloset {\ref{thm:def:heyting_algebra/extrema_complement}} =
    T.
  \end{equation*}
\end{proof}

\paragraph{Brouwer-Heyting-Kolmogorov interpretation}

\begin{concept}\label{con:brouwer_heyting_kolmogorov_interpretation}\mcite[sec. 1.3.1]{TroelstraSchwichtenberg2000BasicProofTheory}
  \hyperref[def:propositional_semantics]{Intuitionistic semantics} correspond to the less formal \term{Brouwer-Heyting-Kolmogorov interpretation}. This interpretation is based on the notion of a \enquote{construction}, which is also why we will refer to intuitionistic logic as \term{constructive logic}.

  \begin{thmenum}
    \thmitem{con:brouwer_heyting_kolmogorov_interpretation/verum} We suppose that \( \syntop \) is evident without a construction.

    \thmitem{con:brouwer_heyting_kolmogorov_interpretation/falsum} There is no construction that proves \( \synbot \).

    \thmitem{con:brouwer_heyting_kolmogorov_interpretation/atomic} We suppose that we know what constitutes a construction proving a propositional variable.

    \thmitem{con:brouwer_heyting_kolmogorov_interpretation/disjunction} A construction proving \( \varphi \synvee \psi \) is a construction proving \( \varphi \) or a construction proving \( \psi \).

    \thmitem{con:brouwer_heyting_kolmogorov_interpretation/conjunction} A construction proving \( \varphi \synwedge \psi \) is a pair of constructions, one proving \( \varphi \) and one proving \( \psi \).

    \thmitem{con:brouwer_heyting_kolmogorov_interpretation/conditional} A construction proving \( \varphi \synimplies \psi \) is a transformation of constructions proving \( \varphi \) into constructions proving \( \psi \).

    \thmitem{con:brouwer_heyting_kolmogorov_interpretation/biconditional} Based on \cref{con:brouwer_heyting_kolmogorov_interpretation/conditional}, the biconditional \( \varphi \syniff \varphi \) then corresponds to a pair of transformations (not necessarily inverses).

    \thmitem{con:brouwer_heyting_kolmogorov_interpretation/negation} A construction of \( \neg \varphi \) demonstrates the impossibility of a construction of \( \varphi \). Based on \cref{con:brouwer_heyting_kolmogorov_interpretation/falsum} and \cref{con:brouwer_heyting_kolmogorov_interpretation/conditional}, a construction of \( \neg \varphi \) should correspond to a transformation of constructions transforming proofs of \( \varphi \) into elements of the empty set, and such a transformation is only possible if there exist no constructions proving \( \varphi \).
  \end{thmenum}
\end{concept}

\begin{example}\label{ex:con:brouwer_heyting_kolmogorov_interpretation/well_ordering_principle_zfc}
  \Fullref{thm:well_ordering_theorem} in \hyperref[def:zfc]{\( \logic{ZFC} \)} does not provide a way to construct a well-ordering of an arbitrary set. The theorem relies on the axiom of choice, whose consequence \fullref{thm:diaconescu_goodman_myhill_theorem} proves the law of the excluded middle \eqref{eq:thm:classical_tautologies/lem} from the axioms of \logic{ZF}.

  Since \eqref{eq:thm:classical_tautologies/lem} does not generally hold in intuitionistic logic, it follows that both \fullref{thm:well_ordering_theorem} and the axiom of choice itself should not in general hold under the Brouwer-Heyting-Kolmogorov interpretation. By the terminology in \cref{con:brouwer_heyting_kolmogorov_interpretation}, then, \fullref{thm:well_ordering_theorem} is a non-constructive theorem.
\end{example}

\paragraph{Minimal propositional semantics}

\begin{definition}\label{def:minimal_propositional_semantics}\mimprovised
  In addition to classical and intuitionistic semantics discussed in \cref{def:truth_value_algebra}, we will also occasionally consider \term{minimal semantics}, obtained by interpreting \( \synbot \) as a propositional variable. Thus, a propositional interpretation must now provide a value for \( \synbot \).

  Formula valuations are thus defined as follows:
  \begin{empheq}[left={\Bracks{\varphi}_I} \coloneqq \empheqlbrace]{align}
    &I(\synbot),                             &&\varphi = \synbot,      \label{eq:def:minimal_propositional_semantics/bot} \\
    &\Bracks{\psi}_I \rightarrow I(\synbot), &&\varphi = \synneg \psi, \label{eq:def:minimal_propositional_semantics/neg} \\
    &\vdots,                                 && \notag
  \end{empheq}
  the other cases being the same as in \cref{def:propositional_denotation}.
\end{definition}
\begin{comments}
  \item Minimal logic was introduced by Ingebrigt Johansson as an attempt to further refine intuitionistic logic. An analysis of his works and his interactions with Gerhard Gentzen is given by \incite{VanDerMolen2016MinimalLogic}. No semantics are discussed there.
\end{comments}

\begin{remark}\label{rem:minimal_semantics_and_bottom}
  For achieving minimal semantics, in \cref{def:minimal_propositional_semantics} we treat the falsum \( \synbot \) as a propositional variable.

  Unless we are specifically interested in \( \synbot \) and \( \synneg \), a more reasonable approach may be to only consider the \hyperref[con:syntax_fragment]{fragment} of propositional logic without them. This is approach to minimal logic mentioned in \cite[49]{Mimram2020ProgramEqualsProof} and used in \cref{def:minimal_implication_logic}.

  In \fullref{sec:propositional_natural_deduction} and in \fullref{sec:propositional_completeness}, we will find it simpler to generalize the semantics as we did here. Our definition for minimal semantics is based on the principles outlined by \incite[3]{VanDerMolen2016MinimalLogic}, \incite[35]{TroelstraSchwichtenberg2000BasicProofTheory} and \incite[1]{DienerMcKubreJordens2020MaterialImplication}, although none of the aforementioned authors discuss semantics.
\end{remark}

\begin{definition}\label{def:paraconsistent_consequence_operator}\mcite{StanfordPlato:paraconsistent_logic}
  We say that the \hyperref[def:consequence_operator]{consequence operator} \( \vdash \) for propositional logic is \term{explosive} if the following translation of \eqref{eq:thm:intuitionistic_tautologies/ecq} holds:
  \begin{equation*}
    \varphi, \synneg \varphi \vdash \psi.
  \end{equation*}

  If \( \vdash \) is not explosive, we say that it is \term{paraconsistent}.
\end{definition}

\begin{proposition}\label{thm:minimal_semantics_are_paraconsistent}
  \hyperref[def:minimal_propositional_semantics]{Minimal semantics} is \hyperref[def:paraconsistent_consequence_operator]{paraconsistent}.
\end{proposition}
\begin{proof}
  Fix an interpretation \( I \) such that \( I(\synbot) = T \) and \( I(p) = F \) for some propositional variable \( p \). Then
  \begin{equation*}
    \Bracks{\synneg \syntop}_I
    \reloset {\eqref{eq:def:minimal_propositional_semantics/neg}} =
    \Bracks{\syntop}_I \rightarrow I(\synbot)
    =
    T \rightarrow T
    \reloset {\ref{thm:intuitionistic_tautologies/self}} =
    T
  \end{equation*}

  Then \( I \vDash \syntop \) and \( I \vDash \synneg \syntop \), yet \( I \nvDash p \).
\end{proof}

\begin{proposition}\label{thm:intuitionistic_semantics_are_explosive}
  \hyperref[def:propositional_semantics]{Intuitionistic semantics} is \hyperref[def:paraconsistent_consequence_operator]{explosive}.
\end{proposition}
\begin{proof}
  Fix two formulas, \( \varphi \) and \( \psi \), and an interpretation \( I \) such that \( I \vDash \varphi \) and \( I \vDash \synneg \varphi \). Then \( I \vDash \varphi \synwedge \synneg \varphi \) because
  \begin{equation*}
    \Bracks{\varphi \synwedge \synneg \varphi}_I
    \reloset {\eqref{eq:def:propositional_denotation/conn}} =
    \Bracks{\varphi}_I \synwedge \Bracks{\synneg \varphi}_I
    =
    T \wedge T
    =
    T.
  \end{equation*}

  But also
  \begin{equation*}
    \Bracks{\varphi \synwedge \synneg \varphi}_I
    \reloset {\eqref{eq:def:propositional_denotation/conn}} =
    \Bracks{\varphi}_I \wedge {\synneg \varphi}_I
    \reloset {\eqref{eq:thm:intuitionistic_equivalences/negation_bottom}} =
    \Bracks{\varphi}_I \wedge (\Bracks{\varphi}_I \rightarrow F)
    \reloset {\eqref{eq:def:heyting_algebra/axioms/modus_ponens}} =
    \Bracks{\varphi}_I \wedge F
    =
    F.
  \end{equation*}

  The obtained contradiction shows that such an interpretation \( I \) does not exist.

  Therefore, for all zero interpretations \( I \) for which both \( I \vDash \varphi \) and \( I \vDash \psi \), we can conclude that \( I \vDash \psi \) for any formula \( \psi \).
\end{proof}

\begin{proposition}\label{thm:semantic_propositional_conjunction_of_premises}
  With respect to \hyperref[def:minimal_propositional_semantics]{minimal semantics}, we have \( \varphi_1, \ldots, \varphi_n \vDash \psi \) if and only if \( (\varphi_1 \synwedge \cdots \synwedge \varphi_n) \vDash \psi \).
\end{proposition}
\begin{proof}
  By induction on \( n \), we can easily show that
  \begin{equation*}
    \Bracks{\varphi_1 \synwedge \cdots \synwedge \varphi_n}_I
    \reloset {\eqref{eq:def:propositional_denotation/conn}} =
    \Bracks{\varphi_1}_I \wedge \cdots \wedge \Bracks{\varphi_n}_I,
  \end{equation*}
  so \( \Bracks{\varphi_1 \synwedge \cdots \synwedge \varphi_n}_I = T \) if and only if \( \Bracks{\varphi_k}_I = T \) for every \( k = 1, \ldots, n \).
\end{proof}

\begin{theorem}[Propositional semantic deduction theorem]\label{thm:propositional_semantic_deduction_theorem}
  With respect to \hyperref[def:minimal_propositional_semantics]{minimal semantics}, for an arbitrary \hyperref[def:truth_value_algebra]{truth value algebra} and for arbitrary propositional formulas, we have
  \begin{equation*}
    \Gamma, \varphi \vDash \psi \T{if and only if} \Gamma \vDash \varphi \synimplies \psi.
  \end{equation*}
\end{theorem}
\begin{comments}
  \item This is one of several deduction theorems presented here --- see \cref{rem:deduction_theorem_list/propositional_semantical}.
\end{comments}
\begin{proof}
  Fix an interpretation \( I \) satisfying \( \Gamma \).

  \SufficiencySubProof Suppose that from \( I \vDash \varphi \) it follows that \( I \vDash \psi \). If \( I \vDash \varphi \), then
  \begin{equation*}
    \Bracks{\varphi \synimplies \psi}_I
    =
    \underbrace{\Bracks{\varphi}_I}_T \rightarrow \underbrace{\Bracks{\psi}_I}_T
    \reloset {\ref{thm:def:heyting_algebra/top_left}} =
    \underbrace{\Bracks{\psi}_I}_T,
  \end{equation*}
  thus \( I \vDash (\varphi \synimplies \psi) \).

  \NecessitySubProof Suppose that \( I \vDash (\varphi \synimplies \psi) \). If additionally \( I \vDash \varphi \), then
  \begin{equation*}
    \underbrace{\Bracks{\varphi \synimplies \psi}_I}_{T}
    =
    \underbrace{\Bracks{\varphi}_I}_T \rightarrow \Bracks{\psi}_I
    \reloset {\ref{thm:def:heyting_algebra/top_left}} =
    \Bracks{\psi}_I,
  \end{equation*}
  thus \( I \vDash \psi \).
\end{proof}

\begin{remark}\label{rem:deduction_theorem_list}
  The following is a list of different flavors of the \enquote{deductive theorem} that allows us to use \hyperref[def:propositional_alphabet/connectives/conditional]{implication} and \hyperref[def:entailment_system/entailment]{entailment} interchangeably:
  \begin{thmenum}
    \thmitem{rem:deduction_theorem_list/propositional_semantic} \Fullref{thm:propositional_semantic_deduction_theorem}.
    \thmitem{rem:deduction_theorem_list/implicational_syntactic} \Fullref{thm:implicational_syntactic_deduction_theorem}.
    \thmitem{rem:deduction_theorem_list/propositional_syntactic} \Fullref{thm:propositional_syntactic_deduction_theorem}.
    \thmitem{rem:deduction_theorem_list/fol_semantic} \Fullref{thm:fol_semantic_deduction_theorem}.
  \end{thmenum}
\end{remark}

\begin{corollary}\label{thm:intuitionistic_deduction_consequences}
  We list some consequences of \fullref{thm:propositional_semantic_deduction_theorem}:
  \begin{thmenum}
    \thmitem{thm:intuitionistic_deduction_consequences/top} Under \hyperref[def:minimal_propositional_semantics]{minimal semantics}, \( \varphi \vDash \syntop \) for every formula \( \varphi \).

    \thmitem{thm:intuitionistic_deduction_consequences/bot} Under \hyperref[def:truth_value_algebra/intuitionistic]{intuitionistic semantics}, \( \synbot \vDash \varphi \) for every formula \( \varphi \).
  \end{thmenum}
\end{corollary}
\begin{proof}
  \SubProofOf{thm:intuitionistic_deduction_consequences/top} Follows from \cref{thm:def:heyting_algebra/top_right}.

  \SubProofOf{thm:intuitionistic_deduction_consequences/bot} Follows from \cref{thm:def:heyting_algebra/leq}.
\end{proof}

\paragraph{Classical propositional semantics}

\begin{proposition}\label{thm:classical_equivalences}
  We list some \hyperref[def:truth_value_algebra/classical]{classical} propositional \hyperref[def:semantic_equivalence]{equivalences}\fnote{These are actually statements about \hyperref[def:standard_boolean_functions]{standard Boolean functions}, but nevertheless we find propositional logic more convenient for stating them.}:
  \begin{thmenum}
    \thmitem{thm:classical_equivalences/double_negation} Negation is an \hyperref[def:morphism_invertibility/involution]{involution}:
    \begin{equation}\label{eq:thm:classical_equivalences/double_negation}
      \mathllap{\synneg \synneg \varphi} \gleichstark \mathrlap{\varphi.}
    \end{equation}

    This equivalence combines \eqref{eq:thm:intuitionistic_tautologies/dni} and \eqref{eq:thm:classical_tautologies/dne}.

    \thmitem{thm:classical_equivalences/conditional_as_disjunction} A conditional formula is a disjunction with the \hyperref[def:conditional_formula/antecedent]{antecedent} negated:
    \begin{equation}\label{eq:thm:classical_equivalences/conditional_as_disjunction}
      \mathllap{\varphi \synimplies \psi} \gleichstark \mathrlap{ \synneg \varphi \synvee \psi. }
    \end{equation}

    This equivalence is known as \enquote{material implication}; we discuss it in \cref{con:material_implication}.

    \thmitem{thm:classical_equivalences/contrapositive} A conditional formula is equivalent to its \hyperref[def:conditional_formula/contrapositive]{contrapositive}:
    \begin{equation}\label{eq:thm:classical_equivalences/contrapositive}
      \mathllap{\varphi \synimplies \psi} \gleichstark \mathrlap{\synneg \psi \synimplies \synneg \varphi.}
    \end{equation}

    \thmitem{thm:classical_equivalences/distributivity} Disjunctions and conjunctions distribute over each other:
    \begin{align}
      \mathllap{\varphi \synvee (\psi \synwedge \theta)} &\gleichstark \mathrlap{(\varphi \synvee \psi) \synwedge (\varphi \synvee \theta),}   \label{eq:thm:classical_equivalences/distributivity/join_over_meet} \\
      \mathllap{\varphi \synwedge (\psi \synvee \theta)} &\gleichstark \mathrlap{(\varphi \synwedge \psi) \synvee (\varphi \synwedge \theta).} \label{eq:thm:classical_equivalences/distributivity/meet_over_join}
    \end{align}

    This equivalence motivates axioms \eqref{eq:def:distributive_lattice/join_over_meet} and \eqref{eq:def:distributive_lattice/meet_over_join} for \hyperref[def:distributive_lattice]{distributive lattices}.

    \thmitem{thm:classical_equivalences/de_morgan} Conjunctions and disjunctions are obtained from each other via negation:
    \begin{align}
      \mathllap{\synneg (\varphi \synvee \psi)}   &\gleichstark \mathrlap{\synneg \varphi \synwedge \synneg \psi,} \label{eq:thm:classical_equivalences/de_morgan/complement_of_join} \\
      \mathllap{\synneg (\varphi \synwedge \psi)} &\gleichstark \mathrlap{\synneg \varphi \synvee \synneg \psi.}   \label{eq:thm:classical_equivalences/de_morgan/complement_of_meet}
    \end{align}

    This is one of several variants of De Morgan's laws presented here --- see \cref{ex:de_morgans_laws/propositional}.

    \thmitem{thm:classical_equivalences/biconditional_member_negation} A biconditional formula is equivalent to its termwise negation:
    \begin{equation}\label{eq:thm:classical_equivalences/biconditional_member_negation}
      \mathllap{\varphi \syniff \psi} \gleichstark \mathrlap{\synneg \varphi \syniff \synneg \psi.}
    \end{equation}

    \thmitem{thm:classical_equivalences/biconditional_negation} A negation of a biconditional formula is again a biconditional with one of the terms negated:
    \begin{equation}\label{eq:thm:classical_equivalences/biconditional_negation}
      \begin{aligned}
        \mathllap{\synneg \parens{\varphi \syniff \psi}}
        &\gleichstark
        \mathrlap{\synneg \varphi \syniff \psi \gleichstark}
        \\ &\gleichstark
        \mathrlap{\varphi \syniff \synneg \psi.}
      \end{aligned}
    \end{equation}
  \end{thmenum}
\end{proposition}
\begin{comments}
  \item These equivalences fail more generally in \hyperref[con:intuitionistic_logic]{intuitionistic logic}.
\end{comments}
\begin{proof}
  The proofs follow directly from the Boolean operator tables in \cref{def:standard_boolean_functions}.
\end{proof}

\begin{proposition}\label{thm:classical_tautologies}
  We list some \hyperref[def:truth_value_algebra/classical]{classical} \hyperref[def:propositional_tautology]{propositional tautologies}:
  \begin{thmenum}
    \thmitem{thm:classical_tautologies/dne} The double negation of a formula implies it:
    \begin{equation}\label{eq:thm:classical_tautologies/dne}
      \synneg \synneg \varphi \synimplies \varphi \tag{\ensuremath{ \logic{DNE}_A }}
    \end{equation}

    \enquote{DNE} stands for \enquote{double negation elimination}.

    \thmitem{thm:classical_tautologies/pierce} The following, named \enquote{Pierce's law}, holds:
    \begin{equation}\label{eq:thm:classical_tautologies/pierce}
      ((\varphi \synimplies \psi) \synimplies \varphi) \synimplies \varphi \tag{\ensuremath{ \logic{Pierce}_A }}
    \end{equation}

    \thmitem{thm:classical_tautologies/lem} Either a formula holds or its negation does:
    \begin{equation}\label{eq:thm:classical_tautologies/lem}
      \varphi \synvee \synneg \varphi. \tag{\ensuremath{ \logic{LEM}_A }}
    \end{equation}

    \enquote{LEM} stands for \enquote{law of the excluded middle}.
  \end{thmenum}
\end{proposition}
\begin{comments}
  \item Some tautologies that hold in \hyperref[def:truth_value_algebra/intuitionistic]{intuitionistic semantics} are listed in \cref{thm:intuitionistic_tautologies}.
\end{comments}
\begin{proof}
  The proofs follow directly from the Boolean operator tables in \cref{def:standard_boolean_functions}.
\end{proof}

\begin{concept}\label{con:material_implication}
  Consider the conditional formula \( \varphi \synimplies \psi \). Via \fullref{thm:propositional_semantic_deduction_theorem}, the equivalence \eqref{eq:thm:classical_equivalences/conditional_as_disjunction} can be restated as follows:
  \begin{displayquote}
    The formula \( \varphi \synimplies \psi \) holds unless the antecedent \( \varphi \) holds and the consequent \( \psi \) doesn't.
  \end{displayquote}

  This equivalence goes back to Philo of Megara, who lived in the third century B.C. We refer to it as \term[ru=материальная импликация (\cite[74]{КолмогоровДрагалин2006Логика}), en=material implication (\cite[9]{Kleene2002Logic})]{material implication} to distinguish it from \enquote{strict} and \enquote{relevant} implications, as well as \enquote{assertions} and other possibilities that try to avoid the so-called \enquote{paradoxes of material implication}. A survey of the aforementioned concepts has been written by \incite{StanfordPlato:logic_of_conditionals}.
\end{concept}
