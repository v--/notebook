\section{Boolean functions}\label{sec:boolean_functions}

\paragraph{Boolean values and functions}

\begin{proposition}\label{thm:two_element_lattice}
  Any two-element \hyperref[def:lattice]{lattice} is \hyperref[def:lattice/homomorphism]{isomorphic} to the \hyperref[def:finite_field]{finite field} \( \BbbF_2 = \set{ 0, 1 } \).
\end{proposition}
\begin{comments}
  \item \Fullref{ex:def:boolean_algebra/f2} shows how \( \BbbF_2 \) can be regarded as a \hyperref[def:boolean_algebra]{Boolean algebra}.
\end{comments}
\begin{proof}
  There is only one possible lattice homomorphism from \( \set{ \top, \bot } \) to \( \BbbF_2 = \set{ 0, 1 } \). Furthermore, it is invertible, and its inverse is also a lattice homomorphism. Hence, it is a lattice isomorphism.
\end{proof}

\begin{concept}\label{con:boolean_value}\mimprovised
  There is a natural \hyperref[def:boolean_algebra]{Boolean algebra} structure on any two-element set \( \set{ T, F } \) in which one of the elements is larger than the other. We can regard \( T \) as a value denoting truth and \( F \) as denoting falsity, in which case \( F < T \). We call \( T \) and \( F \) in this context \term{Boolean values}.
\end{concept}
\begin{comments}
  \item See \fullref{rem:mathematical_logic_conventions/propositional_constants} for more discussions regarding related notation conventions.

  \item For certain purposes, for example \hyperref[def:zhegalkin_polynomial]{Zhegalkin polynomials}, it makes sense to identify \( F \) with \( 0 \) and \( T \) with \( 1 \) in the \hyperref[def:finite_field]{finite field} \( \BbbF_2 \) (this is technically the isomorphism of Boolean algebras from \fullref{thm:two_element_lattice}).
\end{comments}

\begin{definition}\label{def:boolean_function}\mcite[9, 120]{Яблонский2003ДискретнаяМатематика}
  We call functions from any set to \( \set{ T, F } \) (Boolean-valued) \term[ru=предикаты, en=predicates (\cite[15]{Savage2008Computability})]{predicates} and functions from \( \set{ T, F }^n \) to \( \set{ T, F } \) \( n \)-ary \term[ru=булевые функции, en=Boolean functions (\cite[847]{Rosen2019DiscreteMathematics})]{Boolean functions}.
\end{definition}

\begin{remark}\label{rem:boolean_valued_functions_and_predicates}
  \hyperref[def:boolean_function]{Boolean-valued functions} and \hyperref[def:relation]{relations} represent the same concept. In particular, if we fix some sets \( X_1, \ldots, X_n \), any relation \( R \subseteq X_1 \times \cdots \times X_n \) corresponds to a unique Boolean-valued function
  \begin{equation*}
    \begin{aligned}
      &f: X_1 \times \cdots \times X_n \to \set{ T, F } \\
      &f(x_1, \ldots, x_n) = \begin{cases}
        T, &(x_1, \ldots, x_n) \in R, \\
        F, &\T{otherwise}
      \end{cases}
    \end{aligned}
  \end{equation*}
  and vice versa.
\end{remark}

\begin{definition}\label{def:boolean_closure}\mcite[30; 33]{Яблонский2003ДискретнаяМатематика}
  Consider the set of all Boolean functions
  \begin{equation*}
    \mscrB \coloneqq \set[\Big]{ f \given* f \colon \set{ T, F }^n \to \set{ T, F } \T{for some nonnegative integer} n }.
  \end{equation*}

  \begin{thmenum}
    \thmitem{def:boolean_closure/closed} We say that a subset \( B \) of \( \mscrB \) is \term{closed} if, whenever \( g(x_1, \ldots, x_m) \) is in \( B \) and \( f_k(x_1, \ldots, x_n) \) is in \( B \) for \( k = 1, \ldots, m \), then their \hyperref[con:function_superposition]{superposition}
    \begin{equation*}
      h(x_1, \ldots, x_n) \coloneqq g(f_1(x_1, \ldots, x_n), \ldots, f_m(x_1, \ldots, x_n))
    \end{equation*}
    is also in \( B \).

    \thmitem{def:boolean_closure/closure} We define the \hyperref[def:moore_closure_operator]{Moore closure operator} \( \cl \) on \( \pow(\mscrB) \) by assigning to every set \( B \) the smallest closed set containing \( B \).

    \thmitem{def:boolean_closure/complete} If the closure \( \cl{B} \) is the entire set \( \mscrB \), we say that \( B \) is \term[ru=полная система, en=functionally complete (\cite[857]{Rosen2019DiscreteMathematics})]{complete}.
  \end{thmenum}
\end{definition}
\begin{comments}
  \item If \( B \) is complete, then from \fullref{thm:functions_over_model_form_model} it follows that \( B \) is a Boolean algebra. This is used in \fullref{thm:propositional_formulas_and_boolean_functions/bijection}.
\end{comments}
\begin{defproof}
  Let \( B \) be an arbitrary set of Boolean functions (i.e. an arbitrary subset of \( \mscrB \)). If a function belongs to \( \cl{B} \), it must belong to any closed set containing \( B \), and vice versa - if a function belongs to every closed superset of \( B \), it must belong to \( \cl{B} \). Hence, \( \cl{B} \) is the intersection of all closed superset of \( B \). At least one such superset exists - \( \mscrB \) itself - hence \( \cl{B} \) is well-defined.

  The conditions for closure operator from \fullref{def:moore_closure_operator} are trivial to verify.
\end{defproof}

\paragraph{Zhegalkin polynomials}

\begin{definition}\label{def:square_free_element}\mcite[79]{JedrzejewiczEtAl2017SquareFree}
  We say that an element \( x \) if a \hyperref[def:semiring]{semiring} is \term[ru=свободное от квадратов (число) (\cite[30]{Бухштаб1966ТеорияЧисел})]{square-free} if there exists no \hi{non-invertible} element \( y \) such that \( y^2 \) divides \( x \).
\end{definition}

\begin{definition}\label{def:zhegalkin_polynomial}\mcite[32]{Яблонский2003ДискретнаяМатематика}
  A \term[ru=полином Жегалкина]{Zhegalkin polynomial} is a \hyperref[def:polynomial_algebra/polynomials]{polynomial} in the \hyperref[def:finite_field]{finite field} \( \BbbF_2 \).
\end{definition}
\begin{comments}
  \item \hyperref[def:square_free_element]{Square-free} Zhegalkin polynomials are unique --- see \fullref{thm:zhegalkin_polynomial_uniqueness}.
\end{comments}

\begin{algorithm}\label{alg:infer_zhegalkin_polynomial}
  Given a \hyperref[def:boolean_function]{Boolean function} \( f(x_1, \ldots, x_n) \), we can recursively build a square-free \hyperref[def:zhegalkin_polynomial]{Zhegalkin polynomial} \( p(X_1, \ldots, X_n) \) that \hyperref[con:evaluation_homomorphism]{evaluates} to \( f(x_1, \ldots, x_n) \).

  \begin{thmenum}
    \thmitem{alg:infer_zhegalkin_polynomial/base} If \( n = 0 \), then \( f \) is a constant; Let \( p \coloneqq f \).
    \thmitem{alg:infer_zhegalkin_polynomial/step} If \( n > 0 \), we can recursively apply the algorithm to obtain polynomials \( p_T(X_2, \ldots, X_n) \) and \( p_F(X_2, \ldots, X_n) \) corresponding to \( f(T, X_2, \ldots, X_n) \) and \( f(F, X_2, \ldots, X_n) \).

    We then define
    \begin{equation}\label{eq:alg:infer_zhegalkin_polynomial/step}
      p(X_1, \ldots, X_n) = X_1 \cdot \parens[\Big]{ p_T(X_2, \ldots, X_n) \oplus p_F(X_2, \ldots, X_n) } \oplus p_F(X_2, \ldots, X_n).
    \end{equation}
  \end{thmenum}
\end{algorithm}
\begin{comments}
  \item This algorithm can be found as \identifier{polynomials.zhegalkin.infer_zhegalkin} in \cite{notebook:code}.
\end{comments}
\begin{defproof}
  We will prove correctness by induction. The base case \( n = 0 \) is vacuous, so suppose that \( n > 0 \) and that the algorithm is correct for less than \( n \) variables.

  Fix a Boolean function \( f(X_1, \ldots, X_n) \) and let \( p_T \) and \( p_F \) be as in \fullref{alg:infer_zhegalkin_polynomial/step}.

  Both are elements of the ring \( \BbbF_2[X_2, \ldots, X_n] \). The polynomial \( p \) as defined by \eqref{eq:alg:infer_zhegalkin_polynomial/step} is a linear polynomial in \( X_1 \) over this ring.
  \begin{itemize}
    \item Evaluating \( p \) with \( X_1 \mapsto F \) yields the constant coefficient, which by definition is \( p_F \).
    \item Evaluating \( p \) with \( X_1 \mapsto T \) yields a value that is the sum of the linear and constant coefficient. Thus, the linear coefficient itself is \( p_T \ominus p_F = p_T \oplus p_F \).
  \end{itemize}

  By the inductive hypothesis, evaluating \( p \) over \( \BbbF_2 \) gives \( f \).
\end{defproof}

\begin{proposition}\label{thm:zhegalkin_polynomial_uniqueness}
  To every Boolean function in \( n \) variables there corresponds a unique square-free Zhegalkin polynomial in \( n \) indeterminates.
\end{proposition}
\begin{proof}
  Existence is provided by \fullref{alg:infer_zhegalkin_polynomial}, while uniqueness is provided by \fullref{thm:functions_over_prime_fields}.
\end{proof}

\begin{lemma}\label{thm:f2_sum_parity}
  In the \hyperref[def:finite_field]{finite field} \( \BbbF_2 \), the sum \( a_1 \oplus \cdots \oplus a_n \) is \( T \) if and only if there is an odd amount of summands with value \( T \).
\end{lemma}
\begin{proof}
  Trivial.
\end{proof}

\begin{proposition}\label{thm:unary_boolean_function_zhegalkin_polynomial}
  For a unary \hyperref[def:boolean_function]{Boolean function} \( f(x) \), the unique square-free Zhegalkin polynomial
  \begin{equation}\label{eq:thm:unary_boolean_function_zhegalkin_polynomial}
    p(x) = ax \oplus b
  \end{equation}
  has coefficients
  \begin{align*}
    b &\coloneqq f(F) \\
    a &\coloneqq f(T) \oplus f(F).
  \end{align*}
\end{proposition}
\begin{proof}
  This is simply a restatement of \eqref{eq:alg:infer_zhegalkin_polynomial/step}.
\end{proof}

\begin{proposition}\label{thm:binary_boolean_function_zhegalkin_polynomial}
  For a binary \hyperref[def:boolean_function]{Boolean function} \( f(x, y) \), the unique square-free Zhegalkin polynomial
  \begin{equation}\label{eq:thm:binary_boolean_function_zhegalkin_polynomial}
    p(x, y) = axy \oplus bx \oplus cy \oplus d
  \end{equation}
  has coefficients
  \begin{align*}
    d &\coloneqq f(F, F) \\
    c &\coloneqq f(F, T) \oplus f(F, F) \\
    b &\coloneqq f(T, F) \oplus f(F, F) \\
    a &\coloneqq \underbrace{f(T, T) \oplus f(T, F) \oplus f(T, F) \oplus f(F, F)}_{T \T*{if} 1 \T*{or} 3 \T*{of the values are} T}.
  \end{align*}
\end{proposition}
\begin{proof}
  \Fullref{alg:infer_zhegalkin_polynomial}
  \begin{equation*}
    f(x, y) = x(ay \oplus b) \oplus (cy \oplus d).
  \end{equation*}

  The expressions for the concrete values for the coefficients follow from \fullref{thm:unary_boolean_function_zhegalkin_polynomial} and \eqref{eq:alg:infer_zhegalkin_polynomial/step}.
\end{proof}

\paragraph{Concrete Boolean functions}

\begin{definition}\label{def:standard_boolean_functions}\mcite[sec. 1.1]{Rosen2019DiscreteMathematics}
  The following \hyperref[def:boolean_function]{Boolean functions} are well-established:
  \begin{equation*}
    \begin{array}{*{2}{c}}
      \toprule
      x & \oline{x}  \\
      \midrule
        & \T{not} x  \\
      \midrule
      F & T          \\
      T & F          \\
      \midrule
        & x \oplus T \\
      \bottomrule
    \end{array}
    \thinspace
    \begin{array}{*{8}{c}}
      \toprule
      x & y & x \vee y             & x \oplus y  & x \wedge y  & x \uparrow y & x \rightarrow y      & x \leftrightarrow y \\
      \midrule
        &   & x \T{or} y           & x \T{xor} y & x \T{and} y & x \T{nand} y & x \T{implies} y      & x \T{iff} y         \\
      \midrule
      F & F & F                    & F           & F           & T            & T                    & T                   \\
      T & F & T                    & T           & F           & T            & F                    & F                   \\
      F & T & T                    & T           & F           & T            & T                    & F                   \\
      T & T & T                    & F           & T           & F            & T                    & T                   \\
      \midrule
        &   & xy \oplus x \oplus y & x \oplus y  & xy          & xy \oplus T  & xy \oplus x \oplus T & x \oplus y \oplus T \\
      \bottomrule
    \end{array}
  \end{equation*}

  The following abbreviations have been used:
  \begin{itemize}
    \item \term{xor} for \enquote{exclusive or}, which is a special case of the \hyperref[def:symmetric_difference]{symmetric difference}.
    \item \term[en=nand (\cite[40]{Hinman2005Logic})]{nand} for \enquote{not and}, also known as \term[ru=штрих Шеффера (\cite[29]{Эдельман1975Логика}), en=Sheffer's stroke (\cite[40]{Hinman2005Logic})]{Sheffer's stroke}.
    \item \term{iff} for \enquote{if and only if}.
  \end{itemize}
\end{definition}
\begin{comments}
  \item The last rows of the two tables contain \hyperref[def:zhegalkin_polynomial]{Zhegalkin polynomials} of the corresponding functions. \Fullref{thm:unary_boolean_function_zhegalkin_polynomial} and \fullref{thm:binary_boolean_function_zhegalkin_polynomial} were used to determine the coefficients.

  \item See \fullref{ex:def:heyting_algebra/three_valued} for how \( {\rightarrow} \) becomes more complicated in three-valued logic.

  \item These tables can be used to prove \fullref{thm:classical_equivalences} and \fullref{thm:classical_tautologies}.
\end{comments}

\begin{proposition}\label{thm:complete_sets_of_boolean_functions}
  The following sets of Boolean functions are \hyperref[def:boolean_closure/complete]{complete}:
  \begin{thmenum}
    \thmitem{thm:complete_sets_of_boolean_functions/zhegalkin} \( \set{ \oplus, \wedge, F, T } \).
    \thmitem{thm:complete_sets_of_boolean_functions/or_not} \( \set{ \vee, \oline{\anon} } \).
    \thmitem{thm:complete_sets_of_boolean_functions/and_not} \( \set{ \wedge, \oline{\anon} } \).
    \thmitem{thm:complete_sets_of_boolean_functions/nand} \( \set{ \uparrow } \).
    \thmitem{thm:complete_sets_of_boolean_functions/conditional_falsum} \( \set{ \rightarrow, F } \).
  \end{thmenum}
\end{proposition}
\begin{proof}
  \SubProofOf{thm:complete_sets_of_boolean_functions/zhegalkin} \Fullref{alg:infer_zhegalkin_polynomial} gives an explicit Zhegalkin polynomial for every Boolean function. The polynomial's constant coefficient is either \( F \) or \( T \); \( \wedge \) is used when evaluating monomials, and \( \oplus \) is used when summing the values of the monomials.

  \SubProofOf{thm:complete_sets_of_boolean_functions/or_not} By inspecting \fullref{def:standard_boolean_functions}, we can conclude that we can express the operators from \fullref{thm:complete_sets_of_boolean_functions/zhegalkin} via \( \vee \) and \( \oline{\anon} \):
  \begin{itemize}
    \item \( T = x \vee \oline{x} \),
    \item \( F = \oline{T} \),
    \item \( x \wedge y = \oline*{\oline{x} \vee \oline{y}} \),
    \item \( x \oplus y = (x \vee y) \wedge (\oline{x} \vee \oline{y}) \).
  \end{itemize}

  Since the former operators are a complete set, the latter are also a complete set.

  \SubProofOf{thm:complete_sets_of_boolean_functions/and_not} Follows from \fullref{thm:complete_sets_of_boolean_functions/or_not} by noting that
  \begin{equation*}
    x \vee y = \oline*{\oline x \wedge \oline y}.
  \end{equation*}

  \SubProofOf{thm:complete_sets_of_boolean_functions/nand} Note that
  \begin{itemize}
    \item \( \oline{x} = x \uparrow T \),
    \item \( x \wedge y = \oline{x \uparrow y} \).
  \end{itemize}

  \Fullref{thm:complete_sets_of_boolean_functions/and_not} implies that \( \set{ \wedge, \oline{\anon} } \) is complete. Then so is \( \set{ \uparrow } \).

  \SubProofOf{thm:complete_sets_of_boolean_functions/conditional_falsum} Follows from \fullref{thm:complete_sets_of_boolean_functions/nand} by noting that
  \begin{equation*}
    x \uparrow y = x \rightarrow (y \rightarrow F).
  \end{equation*}
\end{proof}
