\subsection{Boolean functions}\label{subsec:boolean_functions}

\paragraph{Boolean values and operators}

\begin{proposition}\label{thm:two_element_lattice}
  Any two-element \hyperref[def:lattice]{lattice} is \hyperref[def:lattice/homomorphism]{isomorphic} to the \hyperref[def:finite_field]{finite field} \( \BbbF_2 = \set{ 0, 1 } \).
\end{proposition}
\begin{comments}
  \item \Fullref{ex:def:boolean_algebra/f2} shows how \( \BbbF_2 \) can be regarded as a \hyperref[def:boolean_algebra]{Boolean algebra}.
\end{comments}
\begin{proof}
  There is only one possible lattice homomorphism from \( \set{ \top, \bot } \) to \( \BbbF_2 = \set{ 0, 1 } \), and it is invertible.
\end{proof}

\begin{definition}\label{def:boolean_value}\mimprovised
  There is a natural \hyperref[def:boolean_algebra]{Boolean algebra} structure on any two-element set \( \set{ T, F } \) in which one of the elements is somehow \enquote{larger} than the other. We can regard \( T \) as a value denoting truth and \( F \) as denoting falsity, in which case \( F < T \). We call \( T \) and \( F \) in this context \term{Boolean values}.
\end{definition}
\begin{comments}
  \item See \fullref{rem:mathematical_logic_conventions/propositional_constants} for more discussions regarding related notation conventions.

  \item For certain purposes, for example \hyperref[def:zhegalkin_polynomial]{Zhegalkin polynomials}, it makes sense to identify \( F \) with \( 0 \) and \( T \) with \( 1 \) in the \hyperref[def:finite_field]{finite field} \( \BbbF_2 \) (this is technically the isomorphism of Boolean algebras from \fullref{thm:two_element_lattice}).
\end{comments}

\begin{definition}\label{def:boolean_function}\mcite[9, 120]{Яблонский2003}
  We call functions from any set to \( \set{ T, F } \) (Boolean-valued) \term[ru=предикаты, en=predicates (\cite[15]{Savage1998})]{predicates} and functions from \( \set{ T, F }^n \) to \( \set{ T, F } \) \( n \)-ary \term[ru=булевые функции, en=Boolean functions (\cite[12]{Savage1998})]{Boolean functions}.
\end{definition}

\begin{remark}\label{rem:boolean_valued_functions_and_predicates}
  \hyperref[def:boolean_function]{Boolean-valued functions} and \hyperref[def:relation]{relations} represent the same concept. In particular, if we fix some sets \( X_1, \ldots, X_n \), any relation \( R \subseteq X_1 \times \cdots \times X_n \) corresponds to a unique Boolean-valued function
  \begin{equation*}
    \begin{aligned}
      &f: X_1 \times \cdots \times X_n \to \set{ T, F } \\
      &f(x_1, \ldots, x_n) = \begin{cases}
        T, &(x_1, \ldots, x_n) \in R, \\
        F, &\T{otherwise}
      \end{cases}
    \end{aligned}
  \end{equation*}
  and vice versa.
\end{remark}

\begin{definition}\label{def:boolean_closure}
  Consider the set of all Boolean functions
  \begin{equation*}
    \mscrB \coloneqq \set[\Big]{ f \given* f \colon \set{ T, F }^n \to \set{ T, F } \T{for some nonnegative integer} n }.
  \end{equation*}

  \begin{thmenum}
    \thmitem{def:boolean_closure/closed}\mcite[33]{Яблонский2003} We say that a subset \( B \) of \( \mscrB \) is \term{closed} if, whenever \( g(x_1, \ldots, x_m) \) is in \( B \) and \( f_k(x_1, \ldots, x_n) \) is in \( B \) for \( k = 1, \ldots, m \), then their \hyperref[con:function_superposition]{superposition}
    \begin{equation*}
      h(x_1, \ldots, x_n) \coloneqq g(f_1(x_1, \ldots, x_n), \ldots, f_m(x_1, \ldots, x_n))
    \end{equation*}
    is also in \( B \).

    \thmitem{def:boolean_closure/closure}\mcite[33]{Яблонский2003} We define the \hyperref[def:moore_closure_operator]{Moore closure operator} \( \cl \) on \( \pow(\mscrB) \) by assigning to every set \( B \) the smallest closed set containing \( B \).

    \thmitem{def:boolean_closure/complete}\mcite[30]{Яблонский2003} If the closure \( \cl{B} \) is the entire set \( \mscrB \), we say that \( B \) is \term[ru=полная система, en=complete basis (\cite[84]{Savage1998})]{complete}.
  \end{thmenum}
\end{definition}
\begin{comments}
  \item If \( B \) is complete, then from \fullref{thm:functions_over_model_form_model} it follows that \( B \) is a Boolean algebra. This is used in \fullref{thm:propositional_formulas_and_boolean_functions/bijection}.
\end{comments}
\begin{defproof}
  Let \( B \) be an arbitrary set of Boolean functions (i.e. an arbitrary subset of \( \mscrB \)). If a function belongs to \( \cl{B} \), it must belong to any closed set containing \( B \), and vice versa - if a function belongs to every closed superset of \( B \), it must belong to \( \cl{B} \). Hence, \( \cl{B} \) is the intersection of all closed superset of \( B \). At least one such superset exists - \( \mscrB \) itself - hence \( \cl{B} \) is well-defined.

  The conditions for closure operator from \fullref{def:moore_closure_operator} are trivial to verify.
\end{defproof}

\begin{definition}\label{def:square_free_element}\mcite[79]{JedrzejewiczMatysiakZielinski2017}
  We say that an element \( x \) if a \hyperref[def:semiring]{semiring} is \term{square-free} if there exists no element \( y \) such that \( y^2 \) divides \( x \).
\end{definition}

\begin{definition}\label{def:zhegalkin_polynomial}\mcite[32]{Яблонский2003}
  A \term[ru=полином Жегалкина]{Zhegalkin polynomial} is a \hyperref[def:polynomial_algebra/polynomials]{polynomial} in the \hyperref[def:finite_field]{finite field} \( \BbbF_2 \).
\end{definition}
\begin{comments}
  \item \Fullref{thm:functions_over_prime_fields} implies that to every Boolean function there corresponds exactly one \hyperref[def:square_free_element]{square-free} Zhegalkin polynomial.

  For example, for every binary Boolean function there exist coefficients \( a, b, c, d \in \BbbF_2 \) such that
  \begin{equation}\label{eq:def:zhegalkin_polynomial/binary_polynomial}
    f(x, y) = axy \oplus bx \oplus cy \oplus d.
  \end{equation}
\end{comments}

\begin{definition}\label{def:standard_boolean_functions}\mcite[14]{Яблонский2003}
  Unlike \hyperref[def:function]{arbitrary functions}, \hyperref[def:boolean_function]{Boolean functions} only have several possible values that can easily be enumerated. Some important operators are:
  \begin{center}
    \begin{tabular}{c | c || c c | c c c c c c}
      \( x \) & \( \oline{x} \)  & \( x \) & \( y \) & \( x \vee y \)             & \( x \oplus y \)    & \( x \wedge y \)        & \( x \rightarrow y \)   & \( x \leftrightarrow y \) \\
      \hline
              & not \( x \)      &         &         & \( x \) or \( y \)         & \( x \) xor \( y \) & \( x \) and \( y \)     & \( x \) implies \( y \) & \( x \) iff \( y \)       \\
      \hline
      \( T \) & \( F \)          & \( T \) & \( T \) & \( T \)                    & \( F \)             & \( T \)                 & \( T \)                 & \( T \)                   \\
      \( F \) & \( T \)          & \( T \) & \( F \) & \( F \)                    & \( T \)             & \( F \)                 & \( F \)                 & \( F \)                   \\
              &                  & \( F \) & \( T \) & \( F \)                    & \( T \)             & \( F \)                 & \( T \)                 & \( F \)                   \\
              &                  & \( F \) & \( F \) & \( F \)                    & \( F \)             & \( F \)                 & \( T \)                 & \( T \)                   \\
      \hline
              & \( x \oplus 1 \) &         &         & \( xy \oplus x \oplus y \) & \( x \oplus y \)    & \( xy \)            & \( xy \oplus x \oplus 1 \) & \( x \oplus y \oplus 1 \)
    \end{tabular}
  \end{center}
\end{definition}
\begin{comments}
  \item \enquote{xor} is an abbreviation for \enquote{exclusive or}, while \enquote{iff} is an abbreviation for \enquote{if and only if}.

  \item Out of the listed binary operations, \( \vee \), \( \wedge \) and \( \oline{\anon} \) form the \hyperref[def:boolean_algebra]{Boolean algebra} structure on \( \BbbF_2 \) and \( \oplus \) and \( \wedge \) form the \hyperref[def:field]{field} structure on \( \BbbF_2 \).

  \item The operation \( \rightarrow \) is also defined in any \hyperref[def:boolean_algebra]{Boolean algebra}.

  \item See \fullref{thm:boolean_equivalences} for direct consequences of these definitions.
\end{comments}

\begin{proposition}\label{thm:complete_sets_of_boolean_functions}
  The following sets of Boolean functions are \hyperref[def:boolean_closure/complete]{complete}:
  \begin{thmenum}
    \thmitem{thm:complete_sets_of_boolean_functions/zhegalkin} \( \set{ \oplus, \wedge, F, T } \).
    \thmitem{thm:complete_sets_of_boolean_functions/or_not} \( \set{ \vee, \oline{\anon} } \).
    \thmitem{thm:complete_sets_of_boolean_functions/and_not} \( \set{ \wedge, \oline{\anon} } \).
    \thmitem{thm:complete_sets_of_boolean_functions/nand} \( \set{ \uparrow } \), where \( (x \uparrow y) \coloneqq \oline{x \wedge y} \)\fnote{This operation is called \term{Sheffer's stroke} or \term{nand} (\enquote{not and}).}
    \thmitem{thm:complete_sets_of_boolean_functions/conditional_falsum} \( \set{ \rightarrow, F } \).
  \end{thmenum}
\end{proposition}
\begin{proof}
  \SubProofOf{thm:complete_sets_of_boolean_functions/zhegalkin} \Fullref{thm:finite_field_lagrange_interpolation} implies that every Boolean function has a corresponding Zhegalkin polynomial. The polynomial's constant term is either \( 0 \) or \( 1 \); \( \wedge \) is used when evaluating monomials, and \( \oplus \) is used when summing the values of the monomials.

  \SubProofOf{thm:complete_sets_of_boolean_functions/or_not} By inspecting \fullref{def:standard_boolean_functions}, we can conclude that we can express the operators from \fullref{thm:complete_sets_of_boolean_functions/zhegalkin} via \( \vee \) and \( \oline{\anon} \):
  \begin{itemize}
    \item \( T = x \vee \oline{x} \),
    \item \( F = \oline{T} \),
    \item \( x \wedge y = \oline*{\oline{x} \vee \oline{y}} \),
    \item \( x \oplus y = (x \vee y) \wedge \oline{(x \vee y)} \).
  \end{itemize}

  Since the former operators are a complete set, the latter are also a complete set.

  \SubProofOf{thm:complete_sets_of_boolean_functions/and_not} Follows from \fullref{thm:complete_sets_of_boolean_functions/or_not} by noting that
  \begin{equation*}
    x \vee y = \oline*{\oline x \wedge \oline y}.
  \end{equation*}

  \SubProofOf{thm:complete_sets_of_boolean_functions/nand} Note that
  \begin{itemize}
    \item \( \oline{x} = x \uparrow T \),
    \item \( x \wedge y = \oline{x \uparrow y} \).
  \end{itemize}

  \Fullref{thm:complete_sets_of_boolean_functions/and_not} implies that \( \set{ \wedge, \oline{\anon} } \) is complete. Then so is \( \set{ \uparrow } \).

  \SubProofOf{thm:complete_sets_of_boolean_functions/conditional_falsum} Follows from \fullref{thm:complete_sets_of_boolean_functions/nand} by noting that
  \begin{equation*}
    x \uparrow y = x \rightarrow (y \rightarrow F).
  \end{equation*}
\end{proof}
