\subsection{Lattice fixed points}\label{subsec:lattice_fixed_points}

\paragraph{Knaster-Tarski theorem}

\begin{definition}\label{def:fixed_point}\mcite[286]{Golan2010}
  Given an \hyperref[def:function/endofunction]{endofunction} \( f: A \to A \) on an arbitrary \hyperref[def:set]{set} \( A \), we call \( x \in A \) a \term[bg=неподвижна точка (\cite[138]{Боянов2008}), ru=неподвижная точка (\cite[23]{Зорич2019Том1})]{fixed point} of \( f \) if \( f(x) = x \).
\end{definition}

\begin{theorem}[Knaster-Tarski theorem]\label{thm:knaster_tarski_theorem}\mcite[thm. 1]{Tarski1955}
  The \hyperref[def:fixed_point]{fixed points} of an \hyperref[def:order_function]{order-preserving} \hyperref[def:function/endofunction]{endofunction} in a \hyperref[def:complete_lattice]{complete lattice} form a complete lattice.
\end{theorem}
\begin{comments}
  \item The lattice of fixed points may have different operations, so it is not in general a sublattice.
  \item In particular, the function has at least one fixed point.
  \item This particular name for the theorem is used by \incite[50]{DaveyPriestley2002} and \incite[321]{PicadoPultr2012}, although only special cases are stated.
\end{comments}
\begin{proof}
  Let \( (X, \leq) \) be a complete lattice and let \( f: X \to X \) be an order-preserving function. Denote by \( F \) the set of all fixed points of \( X \).

  \SubProof{Proof that \( X \) has a least fixed point} Define
  \begin{equation*}
    D \coloneqq \set{ x \in X \given f(x) \leq x }.
  \end{equation*}

  At this point, we do not know whether \( D \) is empty.

  Since the lattice is complete, we can take \( l \coloneqq \inf D \). Then, for every \( y \) with \( f(y) \leq y \), we have \( l \leq y \), and thus \( f(l) \leq f(y) \). Hence,
  \begin{equation*}
    f(l) \leq f(y) \leq y.
  \end{equation*}

  Thus, \( f(l) \) is a lower bound of \( D \). But \( l \) is the greatest lower bound, hence
  \begin{equation*}
    f(l) \leq l.
  \end{equation*}

  We obtain that \( l \) belongs to \( D \). Furthermore, \( f(l) \leq l \) implies \( f(f(l)) \leq f(l) \), thus \( f(l) \) also belongs to \( D \). Since \( l \) is a lower bound of \( D \), we have
  \begin{equation*}
    l \leq f(l).
  \end{equation*}

  Therefore, \( l \) is a fixed point of \( f \). Furthermore, \( l \) is the least element in \( D \), which is a superset of \( F \), so \( l \) is the least fixed point of \( X \).

  \SubProof{Proof that every subset of \( F \) has a supremum in \( F \)} We have shown that \( F \) is nonempty. Let \( G \) be some subset of \( F \). We will show that \( G \) has a supremum in \( F \).

  Define
  \begin{equation*}
    s \coloneqq \sup G.
  \end{equation*}

  We do not know whether this element \( s \) belongs to \( F \). We must find an upper bound of \( G \) belonging to \( F \).

  First note that \( f(s) \) is also an upper bound of \( G \) because, for any \( g \in G \), we have \( g \leq s \), hence
  \begin{equation*}
    g = f(g) \leq f(s).
  \end{equation*}

  Then the \hyperref[def:order_interval/unbounded]{final segment} \( X_{\geq s} \), which is a complete lattice, is \hyperref[def:invariant_subset]{invariant} under \( f \). We can thus restrict \( f \) to \( X_{\geq s} \).

  By what we have already shown, \( X_{\geq s} \) contains a least fixed point of \( f \). This fixed point is an upper bound of \( G \), thus it is the supremum of \( G \) in \( F \).

  \SubProof{Proof that every subset of \( F \) has an infimum in \( F \)} We can regard \( (F, \geq) \) as an ordered subset of \( (X, \geq) \). Then, by what we have already shown, if \( G \) is a subset of \( X \), then \( G \) has a supremum in \( (F, \geq) \). This is precisely the infimum of \( G \) in \( (F, \leq) \).
\end{proof}

\begin{example}\label{ex:thm:knaster_tarski}
  We list examples of lattice fixed points:
  \begin{thmenum}
    \thmitem{ex:thm:knaster_tarski/closure} Every \hyperref[def:moore_closure_operator]{Moore closure operator} introduces a concept of \enquote{closed} elements, which are precisely its fixed points. The join of two closed elements may differ from the ambient join:
    \begin{itemize}
      \item Unions of subgroups are not subgroups in general, and similarly for subrings, ideals and for arbitrary first-order structures; as shown in \fullref{thm:substructures_form_complete_lattice}, their join in instead the substructure generated by their union.
      \item Arbitrary unions of topological closed sets are not closed in general; their join is instead the closure of their union.
    \end{itemize}

    \thmitem{ex:thm:knaster_tarski/derived} The \hyperref[def:derived_set]{derived set operation} in a topological space is not idempotent; \fullref{thm:transfinite_knaster_tarski_iteration} may therefore require transfinite iteration to reach a fixed point.

    \thmitem{ex:thm:knaster_tarski/inductive_relation} \hyperref[con:recursively_defined_relation]{Recursively defined relations} in logic are fixed points of order-preserving operators.
  \end{thmenum}
\end{example}

\begin{definition}\label{def:scott_continuity}\mcite[56]{Gratzer2011}
  We say that a map between \hyperref[def:complete_lattice]{complete lattices} is \term{Scott-continuous} if it preserves joins of \hyperref[def:directed_set]{upward-directed families}.
\end{definition}

\begin{proposition}\label{thm:def:scott_continuous}
  \hyperref[def:scott_continuity]{Scott-continuous} maps have the following basic properties:
  \begin{thmenum}
    \thmitem{thm:def:scott_continuous/order_preserving} Every Scott-continuous map is order-preserving.
    \thmitem{thm:def:scott_continuous/latice_of_subsets} On a \hyperref[thm:boolean_algebra_of_subsets]{lattice of subsets} \( X \), the endofunction \( f: X \to X \) is Scott-continuous if and only if the following holds:
    \begin{displayquote}
      If \( \mscrA \) is an upward-directed family in \( X \) and if \( x \in f(\bigcup \mscrA) \), then there exists a set \( A \in \mscrA \) such that \( x \in f(A) \).
    \end{displayquote}
  \end{thmenum}
\end{proposition}
\begin{proof}
  \SubProofOf{thm:def:scott_continuous/order_preserving} Let \( f \) be a Scott-continuous map and fix some elements \( x \) and \( y \) in the domain of \( f \). We need to show that \( f(x) \leq f(y) \).

  The set \( \set{ x, y } \) is directed, so \( f(x \vee y) = f(x) \vee f(y) \). Furthermore, \eqref{eq:thm:lattice_operation_characterization/compatibility/join} implies that \( x \vee y = y \), so \( f(x) \vee f(y) = f(y) \), and using \eqref{eq:thm:lattice_operation_characterization/compatibility/join} again we obtain \( f(x) \leq f(y) \).

  \SubProofOf{thm:def:scott_continuous/latice_of_subsets} Straightforward.
\end{proof}

\begin{theorem}[Knaster-Tarski iteration theorem]\label{thm:knaster_tarski_iteration}\mcite[lemma 1]{Echenique2005}
  Fix a \hyperref[def:complete_lattice]{complete lattice} \( X \) and an endofunction \( f: X \to X \). Let \( \mu \) be (the initial ordinal of) a cardinal strictly larger than the cardinality of \( X \).

  Fix some element \( x \) such that \( x \leq f(x) \) and consider the \( \mu \)-indexed \hyperref[def:transfinite_sequence]{transfinite sequence}
  \begin{equation*}
    x_\alpha \coloneqq \begin{dcases}
      x,                                &\alpha = 0, \\
      f(x_\beta),                       &\alpha = \op{sc}(\beta), \\
      \bigvee_{\beta < \alpha} x_\beta, &\alpha \T{is a limit ordinal} \\
    \end{dcases}
  \end{equation*}

  \begin{thmenum}
    \thmitem{thm:knaster_tarski_iteration/transfinite} If \( f \) is \hyperref[def:order_function]{order-preserving}, the sequence stabilizes, that is, there exists an ordinal \( \sigma < \mu \) such that \( x_\alpha = x_\sigma \) whenever \( \alpha > \sigma \).

    \thmitem{thm:knaster_tarski_iteration/minimum} Furthermore, \( f \) has a least fixed point greater than or equal to \( x \), and it is precisely \( x_\sigma \).

    \thmitem{thm:knaster_tarski_iteration/continuous} If \( f \) is \hyperref[def:scott_continuous]{Scott-continuous}, then \( \sigma \leq \omega \), so the transfinite sequence reduces to an ordinary sequence, perhaps even a finite one.
  \end{thmenum}
\end{theorem}
\begin{comments}
  \item Although we call this \enquote{the} Knaster-Tarski theorem, it has a weaker conclusion than \fullref{thm:knaster_tarski_theorem}, concerned with the existence of only one particular fixed point. Only the existence of certain fixed points is actually how the general (i.e. non-sequential) theorem is stated by some authors such as \incite[50]{DaveyPriestley2002} and \incite[321]{PicadoPultr2012}.

  \item Since we have not required \( \mu \) to be infinite, if \( X \) itself is finite, so is the \enquote{sequence} \( \seq{ x_\alpha }_{\alpha < \mu} \).

  \item The fact that \( \mu \) is an ordinal is important as we will use successor ordinals (rather than successor cardinals).

  \item This theorem is called \enquote{constructive} by \incite{Echenique2005}, but it utilizes the law of double negation elimination \eqref{eq:thm:classical_tautologies/dne} so it can be argued that it is not, in fact, constructive, at least not in the sense of the \hyperref[con:brouwer_heyting_kolmogorov_interpretation]{Brouwer-Heyting-Komogorov interpretation}.

  \item There is an uncanny resemblance between \fullref{thm:knaster_tarski_iteration/continuous} and \fullref{thm:banach_fixed_point_theorem}.
\end{comments}
\begin{proof}
  \SubProofOf{thm:knaster_tarski_iteration/existence} First note that the sequence \( \seq{ x_\alpha }_{\alpha < \mu} \) coincides with the following, which is easily seen to be nonstrictly increasing:
  \begin{equation*}
    x'_\alpha \coloneqq \begin{dcases}
      x,                                    &\alpha = 0, \\
      \bigvee_{\beta < \alpha} f(x'_\beta), &\alpha > 0 \\
    \end{dcases}
  \end{equation*}

  \SubProof*{Proof of equivalence with the simplified sequence} We can show \( x_\alpha = x'_\alpha \) via induction on \( \alpha \):
  \begin{itemize}
    \item The case \( \alpha = 0 \) is obvious.
    \item If \( \alpha = \op{sc}(\beta) \) and \( \beta = 0 \), we simply use our assumption \( x \leq f(x) \).
    \item If \( \alpha = \op{sc}(\beta) \) and \( \beta > 0 \) and if the inductive hypothesis holds for \( \beta \), then
    \begin{equation*}
      x'_\alpha
      =
      \bigvee_{\gamma < \alpha} f(x'_\gamma)
      =
      \parens[\Big]{ \bigvee_{\gamma < \beta} f(x'_\gamma) } \vee f(x'_\beta)
    \end{equation*}

    Since the sequence \( \seq{ x'_\alpha }_{\alpha < \mu} \) is nonstrictly increasing, we have
    \begin{equation*}
      \bigvee_{\gamma < \beta} f(x_\gamma) \leq f(x_\beta)
    \end{equation*}
    and thus
    \begin{equation*}
      x'_\alpha = f(x'_\beta),
    \end{equation*}
    which by the inductive hypothesis equals \( f(x_\beta) = x_\alpha \).

    \item If \( \alpha \) is a limit ordinal and if the inductive hypothesis holds for all \( \beta < \alpha \), then
    \begin{equation*}
      x'_\alpha
      =
      \bigvee_{\beta < \alpha} f(x'_\beta)
      \reloset {\T{ind.}} =
      \bigvee_{\beta < \alpha} f(x_\beta)
      =
      \bigvee_{\beta < \alpha} x_{\op{sc}(\beta)}.
    \end{equation*}

    Since \( \op{sc}(\beta) < \alpha \) whenever \( \beta < \alpha \), the above is equivalent to
    \begin{equation*}
      x_\alpha = \bigvee_{\beta < \alpha} x_\beta.
    \end{equation*}
  \end{itemize}

  \SubProof*{Proof that the simplified sequence stabilizes} Since the cardinality of \( X \) is smaller than \( \mu \), assuming that \( \seq{ x'_\alpha }_{\alpha < \mu} \) is strictly increasing leads to a contradiction, so there must exist indices \( \delta < \varepsilon \) such that \( x'_\delta = x'_\varepsilon \). Let \( \sigma \) be the smallest ordinal such that \( x'_\sigma = x'_\delta \).

  Then the sequence \( \seq{ x'_\alpha }_{\alpha < \mu} \) stabilizes at \( \sigma \), that is, \( x'_\alpha = x'_\sigma \) whenever \( \alpha > \sigma \). The common value of the tail of the sequence is a fixed point of \( f \).

  \SubProofOf{thm:knaster_tarski_iteration/minimum} We have to show that, if \( y \geq x \) is a fixed point of \( f \), we have \( y \geq x_\sigma \). Via induction on \( \gamma < \mu \), we will prove the more general statement \( y \geq x_\gamma \).
  \begin{itemize}
    \item If \( \gamma = 0 \), then \( x_\gamma = x \) and \( x \leq y \) by assumption.
    \item If \( \gamma = \op{sc}(\delta) \) and  \( x_\delta \leq y \), then, since \( f \) preserves order,
    \begin{equation*}
      x_\gamma = f(x_\delta) \leq f(y) = y.
    \end{equation*}

    \item If \( \gamma \) is a limit ordinal and if \( x_\delta \leq y \) whenever \( \delta < \gamma \), then
    \begin{equation*}
      x_\gamma = \bigvee_{\delta < \gamma} x_\delta \leq y.
    \end{equation*}
  \end{itemize}

  Therefore, \( x_\sigma \) is the least fixed point of \( f \) out of those greater than or equal to \( x \).

  \SubProofOf{thm:knaster_tarski_iteration/continuous} If \( f \) is Scott-continuous, since the sequence \( \seq{ x_\alpha }_{\alpha < \mu} \) is increasing, either \( \mu \) is finite or we have
  \begin{equation*}
    f(x_\omega)
    =
    f\parens[\Big]{ \bigvee_{k=0}^\infty x_k }
    =
    \bigvee_{k=0}^\infty f(x_k)
    =
    \bigvee_{k=1}^\infty x_k
    =
    x \vee \bigvee_{k=1}^\infty x_k
    =
    x_\omega.
  \end{equation*}

  Thus, if \( \mu \) is infinite, \( x_\omega \) is necessarily a fixed point of \( f \).
\end{proof}

\paragraph{Least fixed point recursion and induction}

\begin{theorem}[Least fixed point recursion]\label{thm:least_fixed_point_recursion}\mimprovised
  Assume fixed the following:
  \begin{thmenum}
    \thmitem{thm:least_fixed_point_recursion/domain} A nonempty set \( D \), which we call our \term{domain of discourse}.
    \thmitem{thm:least_fixed_point_recursion/base} A subset \( B \) of \( D \), whose members we call \term{base elements}.
    \thmitem{thm:least_fixed_point_recursion/constructors} A set \( \mscrC \) of \hyperref[def:set_valued_map/partial]{partial} \hyperref[def:operation_on_set]{operations} on \( D \), of possibly different arity. We call elements of \( \mscrC \) \term{constructors}.
  \end{thmenum}

  Then there exists a superset \( P \) of \( B \), which is closed under all constructors, and such that no subset of \( P \) containing \( B \) is closed.

  Furthermore, every member of \( P \) either belongs to \( B \) or can be obtained from elements of \( B \) via finitely many applications of the constructors.

  We call \( P \) the \term{least fixed point} of the recursion.
\end{theorem}
\begin{comments}
  \item This theorem is based on \cite[ch. 8]{DaveyPriestley2002}, although it is not explicitly stated there.
  \item We call this theorem \enquote{fixed point recursion} because it utilizes \fullref{thm:knaster_tarski_iteration} and it resembles \fullref{thm:omega_recursion} and \fullref{thm:bounded_transfinite_recursion}.
\end{comments}
\begin{proof}
  \Fullref{thm:boolean_algebra_of_subsets} implies that \( \pow(D) \) is a complete lattice, so we will use \fullref{thm:knaster_tarski_iteration}.

  Consider the operator
  \begin{equation*}
    \begin{aligned}
      &T: \pow(D) \to \pow(D) \\
      &T(A) \coloneqq A \cup \set{ c(x_1, \ldots, x_n) \given c \T{is a constructor of arity} n \T{defined for} x_1, \ldots, x_n \in A^n }.
    \end{aligned}
  \end{equation*}

  This operator is obviously \hyperref[def:extensive_function]{extensive} and preserves order. Furthermore, it is Scott-continuous because it satisfies \fullref{thm:def:scott_continuous/latice_of_subsets}. Indeed, if the family \( \mscrA \) of subsets of \( D \) is upward-directed and if \( x \in T(\bigcup \mscrA) \), we have the following possibilities:
  \begin{itemize}
    \item If \( x \in \bigcup \mscrA \), then necessarily there exists a set \( A_0 \) in the family \( \mscrA \) such that \( x \in A \) and thus \( x \in T(A_0) \).
    \item Otherwise, there exists a constructor \( c \) of arity \( n \) and elements \( x_1, \ldots, x_n \) of \( \bigcup \mscrA \) such that \( x = c(x_1, \ldots, x_n) \). Suppose that \( x_i \) belongs to \( A_i \) for \( i = 1, \ldots, n \). Since the family is directed, \( \mscrA \) contains an upper limit \( A_0 \) for \( A_1, \ldots, A_n \) a set \( A_0 \). Then \( x_1, \ldots, x_n \) all belong to \( A_0 \), and \( x = c(x_1, \ldots, x_n) \in T(A_0) \).
  \end{itemize}

  \Fullref{thm:knaster_tarski_iteration/continuous} then implies that
  \begin{equation*}
    P \coloneqq \bigcup_{k=0}^\infty T^k(B)
  \end{equation*}
  is the least fixed point of \( T \).
\end{proof}

\begin{example}\label{ex:thm:least_fixed_point_recursion}
  We list examples of how \fullref{thm:least_fixed_point_recursion} can be used:
  \begin{thmenum}
    \thmitem{ex:thm:least_fixed_point_recursion/natural_numbers} Let \( \kappa \) be any uncountable cardinal and consider \fullref{thm:least_fixed_point_recursion} with base element \( \varnothing \) (i.e. \( B = \set{ \varnothing } \)) and sole constructor the \hyperref[def:ordinal_successor]{successor operation} \( \op{sc}: \kappa \to \kappa \).

    Consider the operator \( T \) from our proof of \fullref{thm:least_fixed_point_recursion}. The fixed points of \( T \) containing \( B \) are precisely the \hyperref[def:inductive_set]{inductive sets}, and the least fixed point is the \hyperref[thm:smallest_inductive_set_existence]{smallest inductive set} \( \omega \), which we conflate with the set \( \BbbN \) of natural numbers.

    Furthermore, since \( T \) is Scott-continuous, every member \( n \) of \( \omega \) is of the form
    \begin{equation*}
      \underbrace{ \op{sc}( \cdots \op{sc}(\varnothing) \cdots ) }_{n \T*{times}}.
    \end{equation*}

    The other fixed points contain \hyperref[rem:standard_models_of_arithmetic]{nonstandard natural numbers}, which cannot be obtained by iterative application of \( \op{sc} \) to \( \varnothing \).

    \thmitem{ex:thm:least_fixed_point_recursion/omega_recursion} We can view \fullref{thm:omega_recursion} as a special case of least fixed point recursion. In that setting, we have a nonempty set \( A \), a base element \( a_0 \) of \( A \) and some function \( f: A \to A \), and we want to build a sequence \( \seq{ a_k }_{k=0}^\infty \) such that \( a_{k+1} = f(a_k) \).

    Let our domain of discourse be the Cartesian product \( D \coloneqq \omega \times A \), let our sole base element be \( (0, a_0) \) and let our sole constructor be
    \begin{equation*}
      F(k, a) \coloneqq \parens[\Big]{ k + 1, f(a) }.
    \end{equation*}

    An arbitrary fixed point under \fullref{thm:least_fixed_point_recursion} needs not even be a function (merely a subset of \( \omega \times A \)), but the least fixed point will have exactly one tuple for every natural number, and will coincide with the sequence \( \seq{ a_k }_{k=0}^\infty \).

    \thmitem{ex:thm:least_fixed_point_recursion/relations} Generalizing the above example, in \fullref{sec:mathematical_logic} we use \hyperref[def:inference_rule]{inference rules} to define relations. This is discussed in \fullref{ex:fixed_point_recursion_for_relations}.
  \end{thmenum}
\end{example}

\begin{theorem}[Least fixed point induction]\label{thm:fixed_point_recursion}\mimprovised
  In the setting of \fullref{thm:least_fixed_point_recursion}, in order to prove a statement for every element of \( P \), we can do the following:
  \begin{itemize}
    \item Prove the statement for every element of \( B \).
    \item For every constructor \( c \), if \( c \) has arity \( n \), assume that the statement holds for some members \( x_1, \ldots, x_n \) of \( D \) and show that it also holds for \( c(x_1, \ldots, x_n) \).
  \end{itemize}
\end{theorem}
\begin{comments}
  \item Unlike for most induction principles in \fullref{con:induction}, we did not formulate this one via logical formulas. This would make the statement unnecessarily convoluted with no gains.
\end{comments}
\begin{proof}
  \Fullref{thm:knaster_tarski_iteration/continuous} demonstrates that every element of the least fixed point is either a base element or is obtained from base elements via successive applications of the constructors.
\end{proof}
