\section{Lattice fixed points}\label{sec:lattice_fixed_points}

\paragraph{Fixed points}

\begin{definition}\label{def:function_fixed_point}\mcite[286]{Golan1999Semirings}
  Given an \hyperref[def:function/endofunction]{endofunction} \( f: A \to A \) on an arbitrary \hyperref[def:set]{set} \( A \), we call \( x \in A \) a \term[bg=неподвижна точка (\cite[138]{Боянов2008ЧислениМетоди}), ru=неподвижная точка (\cite[\S 1.8.6]{Новиков2013ДискретнаяМатематика})]{fixed point} of \( f \) if \( f(x) = x \).
\end{definition}

\begin{theorem}[Fixed point existence]\label{thm:fixed_point_existence}
  We list some sufficient conditions for an \hyperref[def:function/endofunction]{endofunction} to have a \hyperref[def:function_fixed_point]{fixed point}:
  \begin{thmenum}
    \thmitem{thm:fixed_point_existence/knaster_tarski} \fullref{thm:knaster_tarski_theorem}: Every \hyperref[def:order_function]{order-preserving} endofunction on a \hyperref[def:complete_lattice]{complete lattice} has a fixed point.

    \thmitem{thm:fixed_point_existence/banach} \fullref{thm:banach_fixed_point_theorem}: Every \hyperref[def:lipschitz_continuity/contraction]{contraction mapping} in a \hyperref[def:complete_metric_space]{complete metric space} has a unique fixed \hyperref[def:function_fixed_point]{point}.
  \end{thmenum}
\end{theorem}

\paragraph{Knaster-Tarski theorem}

\begin{theorem}[Knaster-Tarski theorem]\label{thm:knaster_tarski_theorem}\mcite[thm. 1]{Tarski1955FixedPointTheorem}
  The \hyperref[def:function_fixed_point]{fixed points} of an \hyperref[def:order_function]{order-preserving} \hyperref[def:function/endofunction]{endofunction} in a \hyperref[def:complete_lattice]{complete lattice} form a complete lattice.
\end{theorem}
\begin{comments}
  \item The lattice of fixed points may have different operations, so it is not in general a sublattice.

  \item In particular, the function has at least one fixed point. This is part of our collection of fixed point theorems --- see \fullref{thm:fixed_point_existence}.

  \item This particular name for the theorem is used by \incite[50]{DaveyPriestley2002LatticeTheory} and \incite[321]{PicadoPultr2012FramesAndLocales}, although only special cases are stated.
\end{comments}
\begin{proof}
  Let \( (X, \leq) \) be a complete lattice and let \( f: X \to X \) be an order-preserving function. Denote by \( F \) the set of all fixed points of \( X \).

  \SubProof{Proof that \( X \) has a least fixed point} Define
  \begin{equation*}
    D \coloneqq \set{ x \in X \given f(x) \leq x }.
  \end{equation*}

  At this point, we do not know whether \( D \) is empty.

  Since the lattice is complete, we can take \( l \coloneqq \inf D \). Then, for every \( y \) with \( f(y) \leq y \), we have \( l \leq y \), and thus \( f(l) \leq f(y) \). Hence,
  \begin{equation*}
    f(l) \leq f(y) \leq y.
  \end{equation*}

  Thus, \( f(l) \) is a lower bound of \( D \). But \( l \) is the greatest lower bound, hence
  \begin{equation*}
    f(l) \leq l.
  \end{equation*}

  We obtain that \( l \) belongs to \( D \). Furthermore, \( f(l) \leq l \) implies \( f(f(l)) \leq f(l) \), thus \( f(l) \) also belongs to \( D \). Since \( l \) is a lower bound of \( D \), we have
  \begin{equation*}
    l \leq f(l).
  \end{equation*}

  Therefore, \( l \) is a fixed point of \( f \). Furthermore, \( l \) is the least element in \( D \), which is a superset of \( F \), so \( l \) is the least fixed point of \( X \).

  \SubProof{Proof that every subset of \( F \) has a supremum in \( F \)} We have shown that \( F \) is nonempty. Let \( G \) be some subset of \( F \). We will show that \( G \) has a supremum in \( F \).

  Define
  \begin{equation*}
    s \coloneqq \sup G.
  \end{equation*}

  We do not know whether this element \( s \) belongs to \( F \). We must find an upper bound of \( G \) belonging to \( F \).

  First note that \( f(s) \) is also an upper bound of \( G \) because, for any \( g \in G \), we have \( g \leq s \), hence
  \begin{equation*}
    g = f(g) \leq f(s).
  \end{equation*}

  Then the \hyperref[def:order_interval/unbounded]{final segment} \( X_{\geq s} \), which is a complete lattice, is \hyperref[def:invariant_subset]{invariant} under \( f \). We can thus restrict \( f \) to \( X_{\geq s} \).

  By what we have already shown, \( X_{\geq s} \) contains a least fixed point of \( f \). This fixed point is an upper bound of \( G \), thus it is the supremum of \( G \) in \( F \).

  \SubProof{Proof that every subset of \( F \) has an infimum in \( F \)} We can regard \( (F, \geq) \) as an ordered subset of \( (X, \geq) \). Then, by what we have already shown, if \( G \) is a subset of \( X \), then \( G \) has a supremum in \( (F, \geq) \). This is precisely the infimum of \( G \) in \( (F, \leq) \).
\end{proof}

\begin{example}\label{ex:thm:knaster_tarski}
  We list examples of lattice fixed points:
  \begin{thmenum}
    \thmitem{ex:thm:knaster_tarski/closure} Every \hyperref[def:moore_closure_operator]{Moore closure operator} introduces a concept of \enquote{closed} elements, which are precisely its fixed points. The join of two closed elements may differ from the ambient join:
    \begin{itemize}
      \item Unions of subgroups are not subgroups in general, and similarly for subrings, ideals and for arbitrary first-order structures; as shown in \cref{thm:substructures_form_complete_lattice}, their join in instead the substructure generated by their union.
      \item Arbitrary unions of topological closed sets are not closed in general; their join is instead the closure of their union.
    \end{itemize}

    \thmitem{ex:thm:knaster_tarski/derived} The \hyperref[def:derived_set]{derived set operation} in a topological space is not idempotent; \fullref{thm:knaster_tarski_iteration} may therefore require transfinite iteration to reach a fixed point.

    \thmitem{ex:thm:knaster_tarski/inductive_relation} \hyperref[thm:recursively_defined_relations]{Recursively defined relations} in logic are fixed points of order-preserving operators.
  \end{thmenum}
\end{example}

\begin{definition}\label{def:scott_continuity}\mcite[56]{Grätzer2011LatticeTheory}
  We say that a map between \hyperref[def:complete_lattice]{complete lattices} is \term{Scott-continuous} if it preserves joins of \hyperref[def:directed_set]{upward-directed families}.
\end{definition}
\begin{comments}
  \item Instead of complete lattices, we can work over \hyperref[def:partially_ordered_set]{partially ordered sets} closed under upward-directed suprema. \incite[def. 8.1]{DaveyPriestley2002LatticeTheory} call such a partially ordered sets \enquote{complete}, which unfortunately conflicts with our notion of complete lattice. \incite[192]{Birkhoff1967LatticeTheory} instead requires all \hyperref[def:partial_order_chain/chain]{chains} to have a supremum and calls such partially ordered sets \enquote{inductive}. \incite[83]{БелоусовТкачёв2004ДискретнаяМатематика} calls a partially ordered set \enquote{индуктивным} (\enquote{inductive}) if it satisfies Davey and Priestley's condition.
\end{comments}

\begin{proposition}\label{thm:def:scott_continuity}
  \hyperref[def:scott_continuity]{Scott-continuous} maps have the following basic properties:
  \begin{thmenum}
    \thmitem{thm:def:scott_continuity/order_preserving} Every Scott-continuous map is order-preserving.
    \thmitem{thm:def:scott_continuity/lattice_of_subsets} On a \hyperref[thm:boolean_algebra_of_subsets]{lattice of subsets} \( X \), the endofunction \( f: X \to X \) is Scott-continuous if and only if the following holds:
    \begin{displayquote}
      If \( \mscrA \) is an upward-directed family in \( X \) and if \( x \in f(\bigcup \mscrA) \), then there exists a set \( A \in \mscrA \) such that \( x \in f(A) \).
    \end{displayquote}
  \end{thmenum}
\end{proposition}
\begin{proof}
  \SubProofOf{thm:def:scott_continuity/order_preserving} Let \( f \) be a Scott-continuous map and fix some elements \( x \) and \( y \) in the domain of \( f \). We need to show that \( f(x) \leq f(y) \).

  The set \( \set{ x, y } \) is directed, so \( f(x \vee y) = f(x) \vee f(y) \). Furthermore, \eqref{eq:thm:lattice_operation_characterization/compatibility/join} implies that \( x \vee y = y \), so \( f(x) \vee f(y) = f(y) \), and using \eqref{eq:thm:lattice_operation_characterization/compatibility/join} again we obtain \( f(x) \leq f(y) \).

  \SubProofOf{thm:def:scott_continuity/lattice_of_subsets} Straightforward.
\end{proof}

\begin{theorem}[Knaster-Tarski iteration theorem]\label{thm:knaster_tarski_iteration}\mcite[lemma 1]{Echenique2005TarskiFixedPointTheorem}
  Fix a \hyperref[def:complete_lattice]{complete lattice} \( X \) and an endofunction \( f: X \to X \). Let \( \mu \) be (the initial ordinal of) a cardinal strictly larger than the cardinality of \( X \).

  Fix some element \( x_0 \) such that \( x_0 \leq f(x_0) \) and extend it into the \( \mu \)-indexed \hyperref[def:transfinite_sequence]{transfinite sequence}
  \begin{equation*}
    x_\alpha \coloneqq \begin{dcases}
      f(x_\beta),                       &\alpha = \op{sc}(\beta), \\
      \bigvee_{\beta < \alpha} x_\beta, &\alpha \T{is a limit ordinal} \\
    \end{dcases}
  \end{equation*}

  \begin{thmenum}
    \thmitem{thm:knaster_tarski_iteration/transfinite} If \( f \) is \hyperref[def:order_function]{order-preserving}, the sequence stabilizes, that is, there exists an ordinal \( \sigma < \mu \) such that \( x_\alpha = x_\sigma \) whenever \( \alpha > \sigma \).

    \thmitem{thm:knaster_tarski_iteration/minimum} Furthermore, \( f \) has a least fixed point greater than or equal to \( x \), and it is precisely \( x_\sigma \).

    \thmitem{thm:knaster_tarski_iteration/continuous} If \( f \) is \hyperref[def:scott_continuity]{Scott-continuous}, then \( \sigma \leq \omega \), so the transfinite sequence reduces to an ordinary sequence, perhaps even a finite one.
  \end{thmenum}
\end{theorem}
\begin{comments}
  \item Although we call this \enquote{the} Knaster-Tarski theorem, it has a weaker conclusion than \fullref{thm:knaster_tarski_theorem}, concerned with the existence of only one particular fixed point. Only the existence of certain fixed points is actually how the general (i.e. non-sequential) theorem is stated by some authors such as \incite[50]{DaveyPriestley2002LatticeTheory} and \incite[321]{PicadoPultr2012FramesAndLocales}.

  \item Since we have not required \( \mu \) to be infinite, if \( X \) itself is finite, so is the \enquote{sequence} \( \seq{ x_\alpha }_{\alpha < \mu} \).

  \item The fact that \( \mu \) is an ordinal is important as we will use successor ordinals (rather than successor cardinals).

  \item This theorem is called \enquote{constructive} by \incite{Echenique2005TarskiFixedPointTheorem}, but it utilizes the law of double negation elimination \eqref{eq:thm:classical_tautologies/dne} so it can be argued that it is not, in fact, constructive, at least not in the sense of the \hyperref[con:brouwer_heyting_kolmogorov_interpretation]{Brouwer-Heyting-Komogorov interpretation}.

  \item There is an uncanny resemblance between \cref{thm:knaster_tarski_iteration/continuous} and \fullref{thm:banach_fixed_point_theorem}.
\end{comments}
\begin{proof}
  \SubProofOf{thm:knaster_tarski_iteration/transfinite} First note that the sequence \( \seq{ x_\alpha }_{\alpha < \mu} \) coincides with the following, which is easily seen to be nonstrictly increasing:
  \begin{equation*}
    x'_\alpha \coloneqq \begin{dcases}
      x_0,                                  &\alpha = 0, \\
      \bigvee_{\beta < \alpha} f(x'_\beta), &\alpha > 0 \\
    \end{dcases}
  \end{equation*}

  \SubProof*{Proof of equivalence with the simplified sequence} We can show \( x_\alpha = x'_\alpha \) via induction on \( \alpha \):
  \begin{itemize}
    \item The case \( \alpha = 0 \) is obvious.
    \item If \( \alpha = \op{sc}(\beta) \) and \( \beta = 0 \), we simply use our assumption \( x \leq f(x) \).
    \item If \( \alpha = \op{sc}(\beta) \) and \( \beta > 0 \) and if the inductive hypothesis holds for \( \beta \), then
    \begin{equation*}
      x'_\alpha
      =
      \bigvee_{\gamma < \alpha} f(x'_\gamma)
      =
      \parens[\Big]{ \bigvee_{\gamma < \beta} f(x'_\gamma) } \vee f(x'_\beta)
    \end{equation*}

    Since the sequence \( \seq{ x'_\alpha }_{\alpha < \mu} \) is nonstrictly increasing, we have
    \begin{equation*}
      \bigvee_{\gamma < \beta} f(x_\gamma) \leq f(x_\beta)
    \end{equation*}
    and thus
    \begin{equation*}
      x'_\alpha = f(x'_\beta),
    \end{equation*}
    which by the inductive hypothesis equals \( f(x_\beta) = x_\alpha \).

    \item If \( \alpha \) is a limit ordinal and if the inductive hypothesis holds for all \( \beta < \alpha \), then
    \begin{equation*}
      x'_\alpha
      =
      \bigvee_{\beta < \alpha} f(x'_\beta)
      \reloset {\T{ind.}} =
      \bigvee_{\beta < \alpha} f(x_\beta)
      =
      \bigvee_{\beta < \alpha} x_{\op{sc}(\beta)}.
    \end{equation*}

    Since \( \op{sc}(\beta) < \alpha \) whenever \( \beta < \alpha \), the above is equivalent to
    \begin{equation*}
      x_\alpha = \bigvee_{\beta < \alpha} x_\beta.
    \end{equation*}
  \end{itemize}

  \SubProof*{Proof that the simplified sequence stabilizes} Since the cardinality of \( X \) is smaller than \( \mu \), assuming that \( \seq{ x'_\alpha }_{\alpha < \mu} \) is strictly increasing leads to a contradiction, so there must exist indices \( \delta < \varepsilon \) such that \( x'_\delta = x'_\varepsilon \). Let \( \sigma \) be the smallest ordinal such that \( x'_\sigma = x'_\delta \).

  Then the sequence \( \seq{ x'_\alpha }_{\alpha < \mu} \) stabilizes at \( \sigma \), that is, \( x'_\alpha = x'_\sigma \) whenever \( \alpha > \sigma \). The common value of the tail of the sequence is a fixed point of \( f \).

  \SubProofOf{thm:knaster_tarski_iteration/minimum} We have to show that, if \( y \geq x_0 \) is a fixed point of \( f \), we have \( y \geq x_\sigma \). Via induction on \( \gamma < \mu \), we will prove the more general statement \( y \geq x_\gamma \).
  \begin{itemize}
    \item If \( \gamma = 0 \), then \( x_\gamma = x_0 \) and \( x_0 \leq y \) by assumption.
    \item If \( \gamma = \op{sc}(\delta) \) and  \( x_\delta \leq y \), then, since \( f \) preserves order,
    \begin{equation*}
      x_\gamma = f(x_\delta) \leq f(y) = y.
    \end{equation*}

    \item If \( \gamma \) is a limit ordinal and if \( x_\delta \leq y \) whenever \( \delta < \gamma \), then
    \begin{equation*}
      x_\gamma = \bigvee_{\delta < \gamma} x_\delta \leq y.
    \end{equation*}
  \end{itemize}

  Therefore, \( x_\sigma \) is the least fixed point of \( f \) out of those greater than or equal to \( x_0 \).

  \SubProofOf{thm:knaster_tarski_iteration/continuous} If \( f \) is Scott-continuous, since the sequence \( \seq{ x_\alpha }_{\alpha < \mu} \) is increasing, either \( \mu \) is finite or we have
  \begin{equation*}
    f(x_\omega)
    =
    f\parens[\Big]{ \bigvee_{k=0}^\infty x_k }
    =
    \bigvee_{k=0}^\infty f(x_k)
    =
    \bigvee_{k=1}^\infty x_k
    =
    x \vee \bigvee_{k=1}^\infty x_k
    =
    x_\omega.
  \end{equation*}

  Thus, if \( \mu \) is infinite, \( x_\omega \) is necessarily a fixed point of \( f \).
\end{proof}

\begin{example}\label{ex:thm:knaster_tarski_iteration}
  We list examples of how \fullref{thm:knaster_tarski_iteration} can be used:
  \begin{thmenum}
    \thmitem{ex:thm:knaster_tarski_iteration/natural_numbers} Let \( \kappa \) be any uncountable cardinal and consider \fullref{thm:knaster_tarski_iteration} of the \hyperref[def:ordinal_successor]{successor operation} starting from \( \varnothing \) (i.e. \( x_0 = \set{ \varnothing } \) and  \( f(x) = \op{sc} \) in the notation of \fullref{thm:knaster_tarski_iteration}).

    The fixed points of \( f \) containing \( x_0 \) are precisely the \hyperref[def:inductive_set]{inductive sets}, and the least fixed point is the \hyperref[thm:smallest_inductive_set_existence]{smallest inductive set} \( \omega \), which we generally conflate with the set \( \BbbN \) of natural numbers.

    Furthermore, since \( f \) is Scott-continuous, every member \( n \) of \( \omega \) is of the form
    \begin{equation*}
      \underbrace{ \op{sc}( \cdots \op{sc}(\varnothing) \cdots ) }_{n \T*{times}}.
    \end{equation*}

    The other fixed points contain \hyperref[rem:standard_models_of_arithmetic]{nonstandard natural numbers}, which cannot be obtained by iterative application of \( \op{sc} \) to \( \varnothing \).

    \thmitem{ex:thm:knaster_tarski_iteration/relations} Generalizing the above example, in \fullref{ch:mathematical_logic} we use \hyperref[def:inference_rule]{inference rules} to define relations. This is discussed in \fullref{thm:recursively_defined_relations}.
  \end{thmenum}
\end{example}

\paragraph{Recursively defined relations}

\begin{theorem}[Recursively defined relations]\label{thm:recursively_defined_relations}
  Consider some \hyperref[def:first_order_signature]{first-order signature} \( \Sigma \) and let \( \Sigma' \) be an extension with a binary relation symbol \( \rho \).

  Suppose that we have \( n \) we have an \hyperref[def:inference_rule]{inference rules} \( R_1, \ldots, R_n \), where, for \( k = 1, \ldots, n \),
  \begin{equation*}
    \begin{prooftree}
      \hypo{ \varphi_1 }
      \hypo{ \cdots }
      \hypo{ \varphi_{m_k} }
      \infer3[\( R_k \)]{ \tau_k \mathrel{\rho} \sigma_k }.
    \end{prooftree}
  \end{equation*}

  Here \( \tau_k \) and \( \sigma_k \) are terms over \( \Sigma \), while \( \varphi_1, \ldots, \varphi_{m_k} \) are general formulas over \( \Sigma' \) (usually also atomic formulas for \( \rho \)).

  Then, for any structure \( \mscrD = (D, I) \) over \( \Sigma \), there exists a \hyperref[def:binary_relation]{binary relation} \( \Rho \) on \( D \) such that:
  \begin{thmenum}
    \thmitem{thm:recursively_defined_relations/closed} \( \Rho \) is closed with respect to the rules --- for every rule \( R_k \), we have \( (\Bracks{\tau_k}_v, \Bracks{\sigma_k}_v) \in \Rho \) whenever \( v \) is a \hyperref[def:first_order_valuation/variable_assignment]{variable assignment} such that, for every \( s = 1, \ldots, m_k \), we have \( \Bracks{\varphi_s}_v = T \) and \( (\Bracks{\tau'}_v, \Bracks{\sigma'}_v) \in \Rho \) whenever \( \tau' \mathrel{\rho} \sigma' \) is a subformula of \( \varphi_s \).

    \thmitem{thm:recursively_defined_relations/minimal} \( \Rho \) is minimal --- every pair of elements of \( D \) related by \( \Rho \) is the conclusion of some rule, the premises of which are either conclusions of rules or are not atomic formulas for \( \rho \).
  \end{thmenum}
\end{theorem}
\begin{comments}
  \item The gist of the theorem is that \( \Rho \) is completely determined by recursive applications of the inference rules and no \enquote{extraneous} elements of \( D \) are related. Thus, for every pair of elements related by \( \Rho \) there exists a \enquote{deduction tree} similar to those from \fullref{sec:propositional_natural_deduction}.

  \item For the premises we allow general first-order formulas because, in practice, rules can involve arbitrarily complicated judgments, like in the rules \ref{inf:def:lambda_term_alpha_equivalence/abs} and \ref{inf:def:beta_eta_reduction/eta}.

  \item The syntax-semantics duality is easily visible here --- \( x_k \) and \( y_k \) are syntactic variables, while \( a_k \) and \( b_k \) are concrete elements of \( D \). Similarly, \( \rho \) is simply a symbol, while \( \Rho \) is an actual relation.
\end{comments}
\begin{proof}
  \Cref{thm:boolean_algebra_of_subsets} implies that \( \pow(D^2) \) is a complete lattice, so we will use \cref{thm:knaster_tarski_iteration/continuous}.

  Consider the operator \( E: D^2 \to D^2 \) that extends a relation \( r \) with \( (\Bracks{\tau_k}_v, \Bracks{\sigma_k}_v) \in \Rho \) if there exists a rule
  \begin{equation*}
    \begin{prooftree}
      \hypo{ \varphi_1 }
      \hypo{ \cdots }
      \hypo{ \varphi_{m_k} }
      \infer3[\( R_k \)]{ x_k \mathrel{\rho} y_k },
    \end{prooftree}
  \end{equation*}
  such that, for every \( s = 1, \ldots, m_k \), we have \( \Bracks{\varphi_s}_v = T \) and \( (\Bracks{\tau'}_v, \Bracks{\sigma'}_v) \in \Rho \) whenever \( \tau' \mathrel{\rho} \sigma' \) is a subformula of \( \varphi_s \).

  This operator is obviously \hyperref[def:extensive_function]{extensive} and preserves order. We will now show that it is Scott-continuous because it satisfies \cref{thm:def:scott_continuity/lattice_of_subsets}. Indeed, if the family \( \mscrA \) of subsets of \( D \) is upward-directed and if \( (c, d) \in E(\bigcup \mscrA) \), we have the following possibilities:
  \begin{itemize}
    \item If \( (c, d) \in \bigcup \mscrA \), then there exists a set \( A_0 \in \mscrA \) such that \( (c, d) \in A_0 \).

    Then, as desired, \( (c, d) \in E(A_0) \).

    \item Otherwise, \( c = \Bracks{\tau_k}_v \) and \( d = \Bracks{\sigma_k}_v \) for some rule \( R_k \), in which case, for every \( s = 1, \ldots, m_k \), we have \( \Bracks{\varphi_s}_v = T \) and, whenever \( \tau' \mathrel{\rho} \sigma' \) is a subformula of \( \varphi_s \), there exists a set \( A_s \) from \( \mscrA \) such that \( (\Bracks{\tau'}_v, \Bracks{\sigma'}_v) \in A_s \).

    There are finitely many subformulas in \( \varphi_s \), so, since the family \( \mscrA \) is directed, there exists an upper limit \( A_0 \) that contains \( (\Bracks{\tau'}_v, \Bracks{\sigma'}_v) \) whenever \( \tau' \mathrel{\rho} \sigma' \) is a subformula of \( \varphi_s \) for some \( s = 1, \ldots, m_k \).

    Then, by construction, \( (c, d) \) must belong to \( E(A_0) \).
  \end{itemize}

  \Cref{thm:knaster_tarski_iteration/continuous} then implies that
  \begin{equation*}
    \Rho \coloneqq \bigcup_{k=0}^\infty E^k(\varnothing)
  \end{equation*}
  is the least fixed point of \( E \). It satisfies \cref{thm:recursively_defined_relations/closed} as a fixed point of \( T \) and \cref{thm:recursively_defined_relations/minimal} because of continuity.
\end{proof}

\begin{example}\label{ex:recursively_defined_relation}
  We can use \fullref{thm:recursively_defined_relations} to succinctly define the \hyperref[thm:equivalence_closure]{equivalence closure} \( {\congdot} \) of a binary relation \( {\dotsim} \):
  \begin{align*}
    \begin{prooftree}
      \hypo{ \synx \dotsim \syny }
      \infer1[\logic{I}]{ \synx \congdot \syny }
    \end{prooftree}
    &&
    \begin{prooftree}
      \infer0[\logic{R}]{ \synx \congdot \synx }
    \end{prooftree}
    \\\\
    \begin{prooftree}
      \hypo{ \synx \congdot \syny }
      \infer1[\logic{S}]{ \syny \congdot \synx }
    \end{prooftree}
    &&
    \begin{prooftree}
      \hypo{ \synx \congdot \syny }
      \hypo{ \syny \congdot \synz }
      \infer2[\logic{T}]{ \synx \congdot \synz }
    \end{prooftree}
  \end{align*}

  Note that we have used \( {\congdot} \) and \( {\dotsim} \) as our relation symbols to highlight that inference rules are syntactic constructions, and the relation symbols are merely schemas for actual relations.

  For \fullref{thm:recursively_defined_relations}, we consider a first-order signature with two binary relation symbols --- \( {\congdot} \) and \( {\dotsim} \). For any set \( D \) over which \( {\sim} \) is defined, the corresponding operator from the proof of \fullref{thm:recursively_defined_relations} becomes
  \begin{equation*}
    \begin{aligned}
      E(r) \coloneqq \thickspace & r \thickspace {\cup} \\
                                 & \set{ (x, y) \given x \sim y } \thickspace {\cup} \\
                                 & \set{ (x, x) \given x \in D } \thickspace {\cup} \\
                                 & \set{ (y, x) \given (x, y) \in r } \thickspace {\cup} \\
                                 & \set{ (x, z) \given (x, y) \in r \T{and} (y, z) \in r }.
    \end{aligned}
  \end{equation*}

  To obtain \( {\cong} \), we must recursively apply \( E \), starting at the empty set.
  \begin{itemize}
    \item At the zeroth step, no elements of \( D \) are related.
    \item At the first step \( E(\varnothing) \), we only have \( x \cong y \) if \( x \sim y \) or \( x = y \).
    \item At the second step \( E(E(\varnothing)) \), we have \( x \cong y \) if \( x \sim y \), if \( y \sim x \), if \( x = y \) or if, for some \( z \), both \( x \sim z \) and \( z \sim y \).
    \item After the second step, we have a lot more cases to consider.
  \end{itemize}
\end{example}

\begin{theorem}[Induction on recursively defined relations]\label{thm:induction_on_recursively_defined_relations}
  In the setting of \fullref{thm:recursively_defined_relations}, in order to prove a statement for a specific pair \( (a, b) \in \Rho \), it is sufficient to do the following:
  \begin{displayquote}
    If \( (a, b) \in \Rho \) due to \( R_k \), suppose that the statement holds for all premises of \( R_k \) that are atomic formulas for \( \rho \), and prove the statement for the conclusion \( (a, b) \).
  \end{displayquote}

  Generalizing on \( (a, b) \), we can prove the statement for the entire relation \( \Rho \).
\end{theorem}
\begin{comments}
  \item Unlike for most induction principles discussed in \cref{con:induction}, we did not formulate this one via logical formulas since that would make the statement unnecessarily convoluted with little gain.
\end{comments}
\begin{proof}
  By \cref{thm:recursively_defined_relations/minimal}, every pair of elements of \( D \) related by \( \Rho \) is the conclusion of some rule.
\end{proof}
