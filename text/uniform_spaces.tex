\section{Uniform spaces}\label{sec:uniform_spaces}

\begin{remark}\label{rem:entourage_notation}
  \hyperref[def:uniform_space]{Uniform spaces} are an extension of both \hyperref[def:metric_space]{metric spaces} and \hyperref[def:topological_group]{topological groups} (including \hyperref[def:topological_vector_space]{topological vector spaces}). They are topological spaces that are \enquote{uniform} in the sense that different parts of the space behave the same.

  In metric spaces, we use the notation \( \mu(x, y) < \varepsilon \) to mean that \( x \) and \( y \) are close (at a distance less than \( \varepsilon \)).

  In (\hyperref[con:additive_semigroup]{additive}) topological groups, we instead have linear operations and use \( x - y \in U \) to mean that \( x \) and \( y \) are close (their difference belongs to some neighborhood of \( 0 \)).

  A proper generalization needs to make both metric spaces and topological groups feel natural as special cases. Generalizing metric space \hyperref[def:metric_space/ball]{balls} or neighborhoods of \hyperref[thm:origin_neighborhoods_in_topological_groups]{zero} are nice options which unfortunately introduces some asymmetry since, for example for metric spaces, \( \mu(x, y) < \varepsilon \) can be written as either \( y \in B(x, \varepsilon) \) or \( x \in B(y, \varepsilon) \). This approach does not go far beyond what general topological spaces offer as a notation.

  \cite[section 8]{Engelking1989GeneralTopology} uses the notation \( \abs{x - y} < V \) to mean that \( x \) and \( y \) belong to the \hyperref[def:entourage]{entourage} \( V \). This is a bit confusing because no absolute value nor subtraction are defined in uniform spaces. We find it simpler to not introduce any special notation beyond that of \hyperref[def:relation]{relations}, and so we denote the same by \( (x, y) \in V \).
\end{remark}

\begin{definition}\label{def:entourage}\mcite[sec. 8.1]{Engelking1989GeneralTopology}
  Let \( X \) be a set. For two binary \hyperref[def:relation]{relations} \( V \) and \( U \) on \( X \) we define their sum as
  \begin{equation*}
    V + U \coloneqq \{ (x, z) \colon \exists y \in X: (x, y) \in U, (y, z) \in V \}
  \end{equation*}
  and \( nV \) by \( n \)-fold iterative addition.

  For any relation \( V \), we denote by \( -V \) the \hyperref[def:binary_relation/converse]{inverse relation}.

  A relation \( V \) on \( X \) is called an \term{entourage} if \( V \) is \hyperref[def:binary_relation/reflexive]{reflexive} and \hyperref[def:binary_relation/symmetric]{symmetric}.

  In analogy to \hyperref[def:metric_space]{metric spaces}, we define
  \begin{thmenum}
    \thmitem{def:entourage/ball} We define the \term{open ball} or simply \term{ball} with \term{center} \( x \) and \term{radius} \( V \) to be the set
    \begin{equation*}
      B(x, V) \coloneqq \{ y \in X \colon (y, x) \in V \}.
    \end{equation*}

    \thmitem{def:entourage/bounded_set} We say that the set \( S \subseteq X \) is \term{bounded} if it is contained in some ball.
  \end{thmenum}
\end{definition}

\begin{proposition}\label{thm:entourage_simulates_metric}\mcite[sec. 8.1]{Engelking1989GeneralTopology}
  Using the notation of \cref{def:entourage}, we obtain properties similar to those of metrics:
  \begin{thmenum}
    \refitem{def:metric_space/M1} \( (x, x) \in V \)
    \refitem{def:metric_space/M2} \( (x, y) \in V \) if and only if \( (y, x) \in V \)
    \refitem{def:metric_space/M3} If \( (x, y) \in U \) and \( (y, z) \in V \), then \( (x, y) \in U + V \).
  \end{thmenum}
\end{proposition}

\begin{definition}\label{def:uniform_space}\mcite[sec. 8.1]{Engelking1989GeneralTopology}
  A \term{uniform space} is a set \( X \) with a nonempty family \( \mscrV \) of \hyperref[def:entourage]{entourages} on \( X \) such that
  \begin{thmenum}
    \thmitem[def:uniform_space/U1]{U1} If \( V \in V \) and \( V \subseteq W \) for some entourage \( W \) on \( X \), then \( W \in V \).
    \thmitem[def:uniform_space/U2]{U2} If \( V_1, V_2 \in V \), then \( V_1 \cap V_2 \in V \).
    \thmitem[def:uniform_space/U3]{U3} For every \( V \in V \) there exists \( W \in V \) such that \( 2W \subseteq V \).
    \thmitem[def:uniform_space/U4]{U4} \( \bigcap \mscrV = \Delta_X \), where \( \Delta_X \) is the \hyperref[def:binary_relation/diagonal]{diagonal relation}.
  \end{thmenum}

  The family \( \mscrV \) is called a \term{uniform structure} or \term{uniformity} on \( X \).
\end{definition}

\begin{definition}\label{def:uniform_topology}
  Let \( (X, \mscrV) \) be a \hyperref[def:uniform_space]{uniform space}. We define its \term{uniform topology} or \term{induced topology} as the \hyperref[def:topological_space]{topology} generated by the \hyperref[def:topological_local_base]{neighborhood filter}
  \begin{equation*}
    \mathcal{B}(x) \coloneqq \{ B(x, V) \colon V \in V \}.
  \end{equation*}

  If for some topological space \( (X, \mscrT) \) there exists a uniformity such that \( \mscrT \) is its induced topology, we say that the topology \( \mscrT \) is \term{uniformizable}.
\end{definition}
\begin{proof}
  This proof of correctness does not actually rely on the uniform structure (except for \( \mscrV \) being nonempty), but rather on the properties of entourages.

  It is indeed a neighborhood filter because
  \begin{refenum}
    \refitem{thm:topology_from_local_base/BP1} Every \hyperref[def:entourage]{entourage} is reflexive, hence \( x \) is contained in every ball in \( \mathcal{B}(x) \).

    \refitem{thm:topology_from_local_base/BP3} For \( B(x, U) \) and \( B(x, V) \) we have
    \begin{balign*}
      B(x, U \cap V)
       & =
      \{ y \in X \colon (x, y) \in U \cap V \}
      =    \\ &=
      \{ y \in X \colon (x, y) \in U \text{ and } (x, y) \in V \}
      =    \\ &=
      B(x, U) \cap B(x, V).
    \end{balign*}

    \refitem{thm:topology_from_local_base/BP2} Fix \( x, y \in X \) and a ball \( B(y, V) \in \mathcal{B}(y) \) that contains \( x \). We will show that \( B(y, V) \subseteq B(x, 2V) \).

    Fix \( z \in B(y, V) \). We have \( (z, y) \in V \). Then \( (z, x) \in V + V = 2V \). Since \( z \in B(y, V) \) was arbitrary, we conclude that \( B(y, V) \subseteq B(x, 2V) \).
  \end{refenum}
\end{proof}

\begin{theorem}\label{thm:tychonoff_spaces_are_uniformizable}\mcite[thm. 8.1.20]{Engelking1989GeneralTopology}
  A topological space is \hyperref[def:uniform_topology]{uniformizable} if and only if it is a \hyperref[def:separation_axioms/T3.5]{Tychonoff space}.
\end{theorem}

\begin{definition}\label{def:uniform_space_base}
  Fix a uniform space \( (X, \mscrV) \). The subfamily \( \mscrB \subseteq V \) if entourages is called a \term{base} for \( \mscrV \) if every entourage \( V \in V \) contains a member of \( \mscrV \).
\end{definition}

\begin{definition}\label{thm:uniform_space_base_axioms}\mcite[prop. 8.1.14]{Engelking1989GeneralTopology}
  Let \( X \) be an arbitrary set and let \( \mscrB \) be a family of entourages satisfying the following axioms:
  \begin{thmenum}
    \thmitem[thm:uniform_space_base_axioms/BU1]{BU1} If \( V_1, V_2 \in B \), there exists an entourage \( V \in B \) such that \( V \subseteq V_1 \cap V_2 \).
    \thmitem[thm:uniform_space_base_axioms/BU2]{BU2} For every \( V \in B \) there exists \( W \in B \) such that \( 2W \subseteq V \).
    \thmitem[thm:uniform_space_base_axioms/BU3]{BU3} \( \bigcap \mscrB = \Delta_X \)
  \end{thmenum}

  Then the family of entourages
  \begin{balign}\label{thm:uniform_space_base_axioms/uniformity}
    \mscrV \coloneqq \left\{ V \subseteq X \times X \colon \exists B \in B: B \in V \T{and} V \text{ is reflexive and symmetric} \right\}
  \end{balign}
  is a uniform structure on \( X \). Furthermore, \( \mscrB \) is a \hyperref[def:uniform_space_base]{base} of \( \mscrV \).

  In particular, the base on any topology satisfies \cref{thm:uniform_space_base_axioms/BU1} -- \cref{thm:uniform_space_base_axioms/BU2}.
\end{definition}

\begin{lemma}\label{thm:uniform_space_neighborhood_contains_ball}
  In a uniform space \( (X, \mscrV) \), for every neighborhood \( A \) (in the topology) of a point \( x_0 \in X \) there exists an entourage \( V \in V \) such that \( B(x_0, V) \subseteq A \).
\end{lemma}
\begin{proof}
  By \cref{def:uniform_topology} and \cref{def:topological_base/union}, \( A \) is a union of balls centered at \( x_0 \). For any ball \( B(x_0, V) \) of this union, we have \( B(x_0, V) \subseteq A \).
\end{proof}

\begin{proposition}\label{thm:def:uniform_topology}
  The \hyperref[def:uniform_topology]{uniform topology} \( \mscrT \) on \( (X, \mscrV) \) the following basic properties:
  \begin{thmenum}
    \thmitem{thm:def:uniform_topology/ball_is_open} All \hyperref[def:entourage/ball]{balls} are open sets.
    \thmitem{thm:def:uniform_topology/neighborhood_contains_ball} Every neighborhood of every point a ball centered at that point.
  \end{thmenum}
\end{proposition}
\begin{proof}
  \SubProofOf{thm:def:uniform_topology/ball_is_open} We defined the balls to be the base of the uniform topology, therefore they are open.
  \refitem{thm:def:uniform_topology/neighborhood_contains_ball} Fix a point \( x_0 \). It is a trivial consequence of \cref{def:topological_base/subset} that every neighborhood of \( x_0 \) contains some ball centered at a point that is not necessarily \( x_0 \). By \cref{thm:topology_from_local_base/BP3}, this ball contains another ball centered at \( x_0 \).
\end{proof}

\begin{proposition}\label{thm:uniform_space_local_convergence}
  Fix a topological space \( (X, \mscrT) \) and a uniform space \( (Y, \mscrU) \). Let \( A \subseteq X \) be a nonempty set and let \( f: A \to Y \) be a function. Then \( y_0 \) is a limit point of \( f \) at \( x_0 \in X \) in the sense of \cref{def:local_continuity} if and only if
  \begin{equation}\label{thm:uniform_space_local_convergence/topological_source}
    \forall V \in V \ \exists A \in T(x_0) : x \in A \implies (f(x), y_0) \in V.
  \end{equation}

  If instead, \( (X, \mscrU) \) is a uniform space, then \( y_0 \) is a limit point of \( f \) at \( x_0 \in X \) if and only if
  \begin{equation}\label{thm:uniform_space_local_convergence/uniform_source}
    \forall V \in V \ \exists U \in U : (x, x_0) \in U \implies (f(x), y_0) \in V.
  \end{equation}

  Note that the limit point may not be unique because uniform spaces are not \hyperref[def:separation_axioms/T2]{Hausdorff} in general.
\end{proposition}
\begin{proof}
  We will only prove \cref{thm:uniform_space_local_convergence/uniform_source} because our proof of \cref{thm:uniform_space_local_convergence/topological_source} is a special case.

  \SufficiencySubProof Suppose that \( y_0 \) is a limit point of \( f \) at \( x_0 \) and fix a neighborhood \( B \) of \( y_0 \). Then there exists a neighborhood \( A \) of \( x_0 \) such that \( f(A) \subseteq B \).

  Fix an entourage \( V \in V \). Then \( B(y_0, V) \) is also a neighborhood of \( y_0 \). By \cref{thm:uniform_space_neighborhood_contains_ball} and \cref{def:uniform_space/U2}, there exists an entourage \( V' \subseteq V \) such that \( B(f(x), V') \subseteq B \cap B(y_0, V) \).

  Fix an entourage \( U \in U \) such that \( B(x_0, U) \subseteq A \). Then for any \( x \in X \), if \( (x, x_0) \in U \), we have \( (f(x), y_0) \in V' \). But \( V' \subseteq V \), therefore
  \begin{equation*}
    (x, x_0) \in U \implies (f(x), y_0) \in V.
  \end{equation*}

  This concludes the proof.

  \NecessitySubProof Fix a neighborhood \( B \) of \( y_0 \) and an entourage \( V \in V \) such that \( B(x_0, V) \subseteq B \) (see \cref{thm:uniform_space_neighborhood_contains_ball} for a justification). Then there exists \( U \in U \) such that
  \begin{equation*}
    (x, x_0) \in U \implies (f(x), y_0) \in V.
  \end{equation*}

  Therefore, \( A \coloneqq B(x_0, U) \) is a neighborhood of \( x_0 \) such that \( f(A) \subseteq B \).
\end{proof}

\begin{corollary}\label{thm:uniform_space_local_continuity}
  A function \( f: (X, \mscrV) \to (Y, \mscrU) \) between uniform spaces is continuous at \( x_0 \in X \) if and only if
  \begin{equation*}
    \forall V \in V \ \exists U \in U : (x, x_0) \in U \implies (f(x), f(x_0)) \in V.
  \end{equation*}
\end{corollary}

\begin{definition}\label{def:bounded_function}
  Fix a set \( X \) and a \hyperref[def:uniform_space]{uniform space} \( (Y, \mscrV) \). Fix a function \( f: X \to Y \).

  \begin{thmenum}
    \thmitem{def:bounded_function/bounded} We say that the function \( f: X \to Y \) is \term{bounded} if \( f(X) \) is a bounded set, that is, if there exists a \hyperref[def:entourage/ball]{ball} \( B(y, V) \) such that \( f(X) \subseteq B(y, V) \).

    \thmitem{def:bounded_function/bounded_family} We say that the family of functions \( \mscrF \) from \( X \) to \( Y \) is \term{bounded} at \( x_0 \) if there exists a ball \( B(y, V) \) such that the set \( \mscrF(x_0) \coloneqq \{ f(x_0) \colon f \in F \} \) is contained in \( B(y, V) \).

    \thmitem{def:bounded_function/pointwise} We say that \( \mscrF \) is \term{pointwise bounded} on the set \( S \subseteq X \) if
    \begin{equation*}
      \forall x \in S \ \exists B(y, V) : \mscrF(x) \subseteq B(y, V).
    \end{equation*}

    \thmitem{def:bounded_function/uniform} We say that \( \mscrF \) is \term{uniformly bounded} on \( S \subseteq X \) if
    \begin{equation*}
      \exists B(y, V) \ \forall x \in S : \mscrF(x) \subseteq B(y, V).
    \end{equation*}

    \thmitem{def:bounded_function/locally_bounded} If there is a topology \( \mscrT \) on \( X \), we say that the function \( f: X \to Y \) is \term{locally bounded} if there exists an entourage \( V \in V \) such that for each neighborhood \( A \in T(x) \) we have \( \diam{f(A)} < V \).
  \end{thmenum}
\end{definition}

\begin{proposition}\label{thm:continuous_implies_locally_bounded}
  Let \( (X, \mscrT) \) be a topological space and \( (Y, \mscrV) \) be a uniform space. Any \hyperref[thm:uniform_space_local_convergence/topological_source]{continuous function} from \( X \) to \( Y \) is locally \hyperref[def:bounded_function/locally_bounded]{bounded}.
\end{proposition}
\begin{proof}
  Trivial.
\end{proof}

\begin{definition}\label{def:function_net_convergence}
  Fix a set \( X \) and a \hyperref[def:uniform_space]{uniform space} \( (Y, \mscrV) \). Let \( \{ f_k \}_{k \in \mscrK} \) be a \hyperref[def:topological_net]{net} of functions from \( X \) to \( Y \). We say that \( \{ f_k \}_{k \in \mscrK} \) \term{converges pointwise} to the function \( f \) and write \( f_k \to f \) if
  \begin{equation}\label{def:function_net_convergence/pointwise}
    \forall V \in V \ \underbrace{\forall x \in X \ \exists k_0 \in \mscrK} : k \geq k_0 \implies (f_k(x), f(x)) \in V
  \end{equation}
  and that \( \{ f_k \}_{k \in \mscrK} \) \term{converges uniformly} to \( f \) and write \( f_k \multto f \) if
  \begin{equation}\label{def:function_net_convergence/uniform}
    \forall V \in V \ \underbrace{\exists k_0 \in \mscrK \ \forall x \in X} : k \geq k_0 \implies (f_k(x), f(x)) \in V
  \end{equation}

  In the special case where \( X \) is a topological space with topology \( \mscrT \), we call the sequence \( \{ f_k \}_{k \in \mscrK} \) \term{locally uniformly convergent} (see \cite{ProofWiki:locally_uniform_convergence}) if every point in \( S \) has a neighborhood in which the sequence converges uniformly. Symbolically,
  \begin{equation}\label{def:function_net_convergence/locally_uniform}
    \forall V \in V \ \forall x_0 \in S \ \exists A \in T(x_0) \ \exists k_0 \in \mscrK \ \forall x \in A : k \geq k_0 \implies (f_k(x), f(x)) \in V.
  \end{equation}

  If the index \( k_0 \) does not depend on the neighborhood \( A \) and the point \( x_0 \), then this is equivalent to uniform convergence. It is still more powerful than pointwise convergence. For example, \hyperref[def:convergent_power_series]{power series} are locally uniformly convergent on the interior of their domain of convergence - see \cref{thm:power_series_are_locally_uniform_convergent}.

  A slightly weaker notion is that of \term{compact convergence} (see \cite{ProofWiki:compact_convergence}), which is defined as uniform convergence on any compact subset. Symbolically,
  \begin{equation}\label{def:function_net_convergence/compact}
    \forall V \in V \ \forall \text{ compact } C \subseteq S \ \exists k_0 \in \mscrK \ \forall x \in C : k \geq k_0 \implies (f_k(x), f(x)) \in V.
  \end{equation}
\end{definition}

\begin{definition}\label{def:uniform_continuity}\mcite[435]{Engelking1989GeneralTopology}
  Fix two \hyperref[def:uniform_space]{uniform spaces} \( (X, \mscrU) \) and \( (Y, \mscrV) \). We say that the function \( f: X \to Y \) if is \term{uniformly continuous} on the set \( S \subseteq X \) if
  \begin{equation}\label{def:uniform_continuity/uniform}
    \forall V \in V \ \underbrace{\exists U \in U \ \forall x_1, x_2 \in S} : (x_1, x_2) \in U \implies (f(x_1), f(x_2)) \in V.
  \end{equation}

  Compare this to \term{pointwise continuity} on \( S \), which is defined by \cref{thm:uniform_space_local_convergence/uniform_source} as convergence for any \( x_1 \in X \):
  \begin{equation}\label{def:uniform_continuity/pointwise}
    \forall V \in V \ \underbrace{\forall x_1, x_2 \in S \ \exists U \in U} : (x_1, x_2) \in U \implies (f(x_1), f(x_2)) \in V.
  \end{equation}
\end{definition}

\begin{definition}\label{def:function_set_continuity}\mcite[285]{BouziadTroallic2004Equicontinuity}
  Fix a topological space \( (X, \mscrT) \) and a \hyperref[def:uniform_space]{uniform space} \( (Y, \mscrV) \). We say that the family \( \mscrF \) of functions from \( X \) to \( Y \) is \term{functionwise continuous} at \( x_0 \in X \) if
  \begin{equation}\label{def:function_set_continuity/functionwise}
    \forall V \in V \ \underbrace{\forall f \in F \ \exists A \in T(x_0)} : f(A) \subseteq B(f(x_0), V),
  \end{equation}
  and \term{equicontinuous} at \( x_0 \in X \) if
  \begin{equation}\label{def:function_set_continuity/equicontinuous}
    \forall V \in V \ \underbrace{\exists A \in T(x_0) \ \forall f \in F} : f(A) \subseteq B(f(x_0), V).
  \end{equation}

  In the special case where \( (X, \mscrU) \) is a uniform space, then we can define \term{uniform equicontinuity} of the family \( \mscrF \) on the set \( S \subseteq X \) as
  \begin{equation}\label{def:function_set_continuity/uniform_equicontinuous}
    \forall V \in V \ \underbrace{\exists U \in U \ \forall f \in F \ \forall x_1, x_2 \in S} : (x_1, x_2) \in U \implies (f(x_1), f(x_2)) \in V
  \end{equation}

  Compare this to \term{pointwise equicontinuity} of \( \mscrF \) on \( S \), as defined by \cref{def:function_set_continuity/equicontinuous} for all \( x_1, x_2 \in S \),
  \begin{equation}\label{def:function_set_continuity/pointwise_equicontinuous}
    \forall V \in V \ \underbrace{\forall x_1, x_2 \in S \ \exists U \in U \ \forall f \in F} : (x_1, x_2) \in U \implies (f(x_1), f(x_2)) \in V
  \end{equation}
  to \term{functionwise uniform continuity} of \( \mscrF \) on \( S \), which is defined by \cref{def:uniform_continuity/uniform} for all \( f \in F \),
  \begin{equation}\label{def:function_set_continuity/uniform_functionwise}
    \forall V \in V \ \underbrace{\forall f \in F \ \exists U \in U \ \forall x_1, x_2 \in S} : (x_1, x_2) \in U \implies (f(x_1), f(x_2)) \in V
  \end{equation}
  and to \term{functionwise pointwise continuity} of \( \mscrF \) on \( S \), i.e. regular continuity as defined by \cref{thm:uniform_space_local_continuity} for all \( x_1, x_2 \in S \) and all \( f \in F \),
  \begin{equation}\label{def:function_set_continuity/functionwise_pointwise}
    \forall V \in V \ \underbrace{\forall f \in F \ \forall x_1, x_2 \in S \ \exists U \in U} : (x_1, x_2) \in U \implies (f(x_1), f(x_2)) \in V
  \end{equation}
\end{definition}

\begin{proposition}\label{thm:uniform_limit_of_continuous_functions}
  \hfill
  \begin{thmenum}
    \thmitem{thm:uniform_limit_of_continuous_functions/continuous} A locally \hyperref[def:function_net_convergence]{uniform limit} of functions continuous at a \hyperref[thm:uniform_space_local_continuity]{point} is continuous at that point.
    \thmitem{thm:uniform_limit_of_continuous_functions/uniform} A uniform limit of functions uniformly \hyperref[def:uniform_continuity]{continuous} on a set is uniformly continuous on the set.
  \end{thmenum}
\end{proposition}
\begin{proof}
  The two proofs are similar, but have a lot of subtle differences.

  Fix uniform spaces \( (X, \mscrU) \) and \( (Y, \mscrV) \). Let \( \{ f_k \}_{k \in \mscrK} \) be a \hyperref[def:topological_net]{net} of functions from \( S \subseteq X \) to \( (Y, \mscrV) \).

  \SubProofOf{thm:uniform_limit_of_continuous_functions/continuous} Assume that the functions \( f_k, k \in \mscrK \) are continuous and that they converge to the function \( f \) locally \hyperref[def:function_net_convergence/locally_uniform]{uniformly}.

  Fix an entourage \( W \in V \) and use \cref{def:uniform_space/U3} to obtain \( V \subseteq W \) such that \( 3V \subseteq W \).

  . and a point \( x_0 \in S \). Let \( A \) be a neighborhood of \( x_0 \). From locally \hyperref[def:function_net_convergence]{uniform convergence}, there exists an index \( k_0 \in \mscrK \) such that
  \begin{equation*}
    \forall k > k_0 \ \forall x \in A : (f_k(x), f(x)) \in V.
  \end{equation*}

  Fix \( k > k_0 \). From \hyperref[def:function_net_convergence/locally_uniform]{uniform continuity}, there exists an entourage \( U \in U \) such that
  \begin{equation*}
    \forall x \in S : (x, x_0) \in U \implies (f_k(x_0), f_k(x)) \in V.
  \end{equation*}

  Combining the last two inequalities, we note that for any \( x \in A \),
  \begin{itemize}
    \item \( (f(x_0), f(x)) \in V \),
    \item \( (f_k(x_0), f(x_0)) \in V \),
    \item \( (f_k(x), f(x)) \in V \),
  \end{itemize}
  thus by applying the triangle inequality in \cref{thm:entourage_simulates_metric} twice, we obtain
  \begin{equation*}
    (f(x_0), f(x)) \in 3V \subseteq W \quad\forall x \in A \cap B(x_0, U).
  \end{equation*}

  Given an entourage \( W \in V \), we found a neighborhood \( A \cap B(x_0, U) \) of \( x_0 \) such that \cref{thm:uniform_space_local_convergence/topological_source} is satisfied. Thus, \( f \) is continuous at \( x_0 \).

  \SubProofOf{thm:uniform_limit_of_continuous_functions/continuous} Assume that the functions \( f_k, k \in \mscrK \) are uniformly continuous and that they converge to \( f \) \hyperref[def:function_net_convergence/locally_uniform]{uniformly}.

  As in \cref{thm:uniform_limit_of_continuous_functions/continuous}, fix entourages \( V, W \in V \) such that \( 3V \subseteq W \). From \hyperref[def:uniform_continuity]{uniform continuity},
  \begin{equation*}
    \forall k \in \mscrK \ \exists U \in U \ \forall x_1, x_2 \in S : (x_1, x_2) \in U \implies (f_k(x_1), f_k(x_2)) \in V.
  \end{equation*}

  From \hyperref[def:function_net_convergence]{uniform convergence}, there exists an index \( k_0 \in \mscrK \) such that
  \begin{equation*}
    \forall k > k_0 \ \forall x \in S : (f_k(x), f(x)) \in V.
  \end{equation*}

  Fix an index \( k > k_0 \) and let \( U \in U \) be such that
  \begin{equation}\label{thm:uniform_limit_of_continuous_functions/uniform/continuity}
    \forall x_1, x_2 \in S : (x_1, x_2) \in U \implies (f_k(x_1), f_k(x_2)) \in V.
  \end{equation}

  For any two points \( x_1, x_2 \in S \), we also have that
  \begin{equation}\label{thm:uniform_limit_of_continuous_functions/uniform/convergence}
    (f(x_i), f_k(x_i)) \in V, i = 1, 2.
  \end{equation}

  Analogously to \cref{thm:uniform_limit_of_continuous_functions/continuous}, from \eqref{thm:uniform_limit_of_continuous_functions/uniform/continuity} and \eqref{thm:uniform_limit_of_continuous_functions/uniform/convergence}, we obtain
  \begin{equation*}
    \forall x_1, x_2 \in S : (x_1, x_2) \in U \implies (f(x_1), f(x_2)) \in 3V \subseteq W.
  \end{equation*}

  Thus, the entourage \( U \) depends on \( W \) and not on \( x_1 \) and \( x_2 \). Technically, it also depends on \( k_0 \), however we are only concerned with existence and not uniqueness. Hence, \( f \) is uniformly continuous.
\end{proof}

\begin{definition}\label{def:category_of_uniform_spaces}
  Uniform spaces and \hyperref[def:uniform_continuity]{uniformly continuous functions} form a subcategory of \( \cat{Top} \) (see \cref{def:category_of_small_topological_spaces}). We denote this category by \( \cat{Met} \).
\end{definition}

\begin{definition}\label{def:fundamental_net}
  A \hyperref[def:topological_net]{net} \( \seq{ x_k }_{k \in \mscrK} \) in a uniform space \( (X, \mscrV) \) is called a \term{fundamental net} or \term{Cauchy net} if
  \begin{equation*}
    \forall V \in V \ \exists k_0 \in \mscrK \ \forall k, m \geq k_0 : (x_k, x_m) \in V.
  \end{equation*}
\end{definition}

\begin{lemma}\label{thm:convergent_net_is_fundamental}
  A net in a uniform space that has a limit \hyperref[def:net_limit_point]{point} is \hyperref[def:fundamental_net]{fundamental}.
\end{lemma}
\begin{proof}
  If \( x_0 \) is a limit point of the net \( \seq{ x_k }_{k \in \mscrK} \), the net is eventually in every ball \( B(x_0, V) \), which implies \cref{def:fundamental_net}.
\end{proof}

\begin{definition}\label{def:complete_uniform_space}\mcite[446]{Engelking1989GeneralTopology}
  A uniform space is called \term{complete} if it is \hyperref[def:separation_axioms/T2]{Hausdorff} and if every \hyperref[def:fundamental_net]{fundamental net} \hyperref[def:net_limit_point]{converges}.

  The \term{completion} of uniform space \( (X, \mscrV) \) is a (\hyperref[def:uniform_continuity]{uniformly continuous}) \hyperref[def:morphism_invertibility/left_cancellative]{embedding} \( f: X \to Y \) into a complete uniform space \( (Y, \mscrU) \) such that \( \img(X) \) is \hyperref[def:topologically_dense_set]{dense} in \( Y \).
\end{definition}

\begin{theorem}[Uniform space completion]\label{thm:uniform_space_completion}\mcite[thm. 8.3.12]{Engelking1989GeneralTopology}
  Every uniform space has a unique (up to an isomorphism) \hyperref[def:complete_uniform_space]{completion}.

  See also \fullref{thm:metric_space_completion}.
\end{theorem}

\begin{theorem}[Cauchy's net convergence criterion]\label{thm:cauchys_net_convergence_criterion}
  A net in a complete \hyperref[def:complete_uniform_space]{uniform space} is convergent if and only if it \hyperref[def:fundamental_net]{fundamental}.

  Explicitly, a net \( \seq{ x_k }_{k \in \mscrK} \) in a complete uniform space \( (X, \mscrV) \) is \hyperref[def:net_limit_point]{convergent} if and only if for every entourage \( V \in V \) there exists an index \( k_0 \) such that
  \begin{equation*}
    (x_k, x_m) \in V \quad\forall k, m \geq k_0.
  \end{equation*}
\end{theorem}
\begin{proof}
  \SufficiencySubProof Given by \cref{thm:convergent_net_is_fundamental}
  \NecessitySubProof Given by \cref{def:complete_uniform_space}
\end{proof}
