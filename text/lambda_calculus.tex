\chapter{Lambda calculus}\label{ch:lambda_calculus}

Alonzo Church developed, in his own words, \enquote{a set of postulates for the foundation of formal logic}. This was first published in \cite{Church1932Untyped}, where he used the letter \( \synlambda \) as a logical symbol as follows\fnote{We quote Church's text verbatim, however in our monograph we use a modernized syntax. See \fullref{def:lambda_term}.}:
\begin{displayquote}
  If \( M \) is any formula containing the variable \( x \), then \( \synlambda x [M] \) is a symbol for the function whose values are those given by the formula.
\end{displayquote}

It turned out that this theory is interesting in itself, or in contexts other than formal logic such as computability. This holds in particular for \hyperref[def:rewriting_system]{rewriting systems} of \( \synlambda \)-terms, commonly referred to as \enquote{\( \synlambda \)-conversion}\fnote{\( \synlambda \)-conversion as a concept is discussed in \fullref{con:lambda_conversion}.}. Distilling these ideas lead to what is now known as \enquote{\( \synlambda \)-calculus}.

We present a modern formulation of some fundamental ideas of \( \synlambda \)-calculus in \fullref{sec:untyped_lambda_terms}, \fullref{sec:lambda_term_alpha_equivalence}, \fullref{sec:lambda_term_reductions} and \fullref{sec:church_rosser_theorem}.

The adjective \enquote{untyped} is a contradistinction from further developments by Church himself. He starts his paper \cite{Church1940STT} as follows:
\begin{displayquote}
  The purpose of the present paper is to give a formulation of the simple theory of types which incorporates certain features of the calculus of \( \synlambda \)-conversion.
\end{displayquote}

This \enquote{simple theory of types} is attributed as a development by different logicians of the \enquote{ramified theory of types} by Bertrand Russell. Russell himself was concerned with self-reference as explained in \fullref{rem:self_reference}. We briefly discuss different type theories in \fullref{rem:type_theory}, and present a modern formulation of the so-called simple type theory in \fullref{sec:simply_typed_lambda_terms} and \fullref{sec:lambda_term_normalization}. Two connections to logic will be investigated, in \fullref{sec:curry_howard_correspondence} and, within \fullref{ch:mathematical_logic}, in \fullref{sec:higher_order_logic}.
