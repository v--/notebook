\chapter{Lambda calculus}\label{ch:lambda_calculus}

Alonzo Church developed, in his own words, \enquote{a set of postulates for the foundation of formal logic}. This was first published in \cite{Church1932LambdaCalculus}, where he used the letter \( \muplambda \) as a logical symbol as follows\fnote{We quote Church's text verbatim, however in our monograph we use a modernized syntax. See \fullref{def:lambda_term}.}:
\begin{displayquote}
  If \( M \) is any formula containing the variable \( x \), then \( \synlambda x [M] \) is a symbol for the function whose values are those given by the formula.
\end{displayquote}

It turned out that this theory is interesting in itself, or in contexts other than formal logic such as computability. This holds in particular for \hyperref[def:rewriting_system]{rewriting} of \( \muplambda \)-terms, commonly referred to as \enquote{\( \muplambda \)-conversion}\fnote{\( \muplambda \)-conversion as a concept is discussed in \fullref{con:lambda_conversion}.}. Distilling these ideas lead to what is now known as \enquote{\( \muplambda \)-calculus}.

\begin{itemize}
  \item \Fullref{sec:untyped_lambda_terms} presents a modern formulation of the syntax of \( \muplambda \)-calculus.
  \item \Fullref{sec:lambda_term_alpha_equivalence} studies equivalence of \( \muplambda \)-terms under changes of bound variables.
  \item \Fullref{sec:lambda_term_reductions} studies rewriting of \( \muplambda \)-terms.
  \item \Fullref{sec:church_rosser_theorem} studies some coherence properties of \( \muplambda \)-term rewriting.
\end{itemize}

The adjective \enquote{untyped} is a contradistinction from further developments by Church himself. He starts his paper \cite{Church1940STT} as follows:
\begin{displayquote}
  The purpose of the present paper is to give a formulation of the simple theory of types which incorporates certain features of the calculus of \( \muplambda \)-conversion.
\end{displayquote}

This \enquote{simple theory of types} is attributed as a development by different logicians of the \enquote{ramified theory of types} by Bertrand Russell. Russell himself was concerned with self-reference as explained in \fullref{rem:self_reference}. We briefly discuss different type theories in \fullref{rem:type_theory}, and present a modern formulation of the so-called \enquote{simple type theory}.

\begin{itemize}
  \item \Fullref{sec:simply_typed_lambda_terms} presents type systems in general, and simple type systems with arrow types in particular.
  \item \Fullref{sec:curry_howard_correspondence} extends arrow types with what we call \enquote{simple algebraic types} and show an important connection with propositional logic.
  \item \Fullref{sec:simply_typed_subject_reduction} demonstrates compatibility with untyped \( \muplambda \)-calculus.
  \item \Fullref{sec:algebra_of_simple_types} studied some algebraic properties of simple algebraic types.
  \item \Fullref{sec:lambda_term_normalization} studies reduction chains of well-typed terms.
  \item \Fullref{sec:dependent_types} briefly presents more complicated type systems.
\end{itemize}

Two connections to logic will be investigated --- one in \fullref{sec:curry_howard_correspondence} and, within \fullref{ch:mathematical_logic}, another one in \fullref{sec:higher_order_logic}.
