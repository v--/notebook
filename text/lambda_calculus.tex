\chapter{Lambda calculus}\label{ch:lambda_calculus}

Alonzo Church developed, in his own words, \enquote{a set of postulates for the foundation of formal logic}. This was first published in \cite{Church1932Untyped}, where he used the letter \( \synlambda \) as a logical symbol as follows:
\begin{displayquote}
  If \( M \) is any formula containing the variable \( x \), then \( \synlambda x [M] \) is a symbol for the function whose values are those given by the formula.
\end{displayquote}

It turned out that this theory is interesting in itself, or in contexts other than formal logic such as \fullref{ch:computability_theory}. This holds in particular for \hyperref[def:rewriting_system]{rewriting systems} of \( \synlambda \)-terms, commonly referred to as \enquote{\( \synlambda \)-conversion}\fnote{\( \synlambda \)-conversion as a concept is discussed in \fullref{con:lambda_conversion}.}. This leads to what is now known as \enquote{\( \synlambda \)-calculus}.

We present a modern formulation of some fundamental ideas of \( \synlambda \)-calculus in \fullref{sec:untyped_lambda_calculus} and \fullref{sec:lambda_calculus_reductions}.

The adjective \enquote{untyped} is a contradistinction from further developments by Church himself. He starts his paper \cite{Church1940STT} as follows:
\begin{displayquote}
  The purpose of the present paper is to give a formulation of the simple theory of types which incorporates certain features of the calculus of \( \synlambda \)-conversion.
\end{displayquote}

This \enquote{simple theory of types} is attributed as a development by different logicians of the \enquote{ramified theory of types} by Bertrand Russell. Russell himself was concerned with self-reference as explained in \fullref{rem:self_reference}.

Under Church's flavor of type theory, given the types \( \alpha \) and \( \beta \) ranging over some objects, Church regarded the type \( (\alpha\beta) \) as
\begin{displayquote}
  \textellipsis the type of functions of one variable for which the range of the independent variable comprises the type \( \beta \) and the range of the dependent variable is contained in the type \( \alpha \).
\end{displayquote}

Then, given a \( \synlambda \)-term \( M \) of type \( \alpha \) and a variable \( x \) of type \( \beta \), the term \( \synlambda x M \) should have type \( (\alpha\beta) \). We prefer to denote the latter type by \enquote{\( \beta \synimplies \alpha \)} because of the Curry-Howard correspondence discussed in \fullref{con:curry_howard_correspondence}. This correspondence provides a connection between types and propositional formulas, which provides an alternative connection with logic compared to Church's idea of the \( \synlambda \)-terms themselves encoding formulas.

We present a modern formulation of these ideas in \fullref{sec:simply_typed_lambda_calculus} and, in connection to logic, in \fullref{sec:higher_order_logic}.
