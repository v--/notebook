\subsection{Category equivalences}\label{subsec:category_equivalences}

\begin{remark}\label{rem:category_similarity}
  We have the following notions for expressing that two categories \( \cat{C} \) and \( \cat{D} \) are similar:

  \begin{thmenum}
    \thmitem{rem:category_similarity/equality} Obviously, if \( \cat{C} \) and \( \cat{D} \) are equal, they are similar.

    \thmitem{rem:category_similarity/isomorphism} A slightly less obvious notion is \term{isomorphism of categories}. This is an isomorphism, in the sense of \fullref{def:morphism_invertibility/isomorphism}, in the category \hyperref[def:category_of_small_categories]{\( \ucat{Cat} \)} of small categories for a suitable \hyperref[def:grothendieck_universe]{Grothendieck universe} \( \mscrU \). That is, \( \cat{C} \) and \( \cat{D} \) are isomorphic if there exists an invertible functor between them.

    We rarely distinguish between objects and arrows of isomorphic categories, even if we do not have strict equality in the sense of the \hyperref[def:zfc/extensionality]{axiom of extensionality} in \hyperref[def:zfc]{\logic{ZFC}}.

    Examples of isomorphic categories include \fullref{thm:order_category_isomorphism} and \fullref{ex:isomorphism_of_directed_multigraph_categories}.

    \thmitem{rem:category_similarity/equivalence} A weaker but very useful notion is \term{category equivalence} defined in \fullref{def:category_equivalence}.
  \end{thmenum}
\end{remark}

\begin{example}\label{ex:set_discr_cat_isomorphism}
  There is an isomorphism between the category \( \ucat{Set} \) of small sets and \( \ucat{DiscrCat} \) of all small \hyperref[def:discrete_category]{discrete categories}.

  Consider the pair of functors
  \begin{equation*}
    U: \ucat{DiscrCat} \to \ucat{Set},
  \end{equation*}
  which for any small category \( \cat{C} \) gives us its set of objects \( \obj(\cat{C}) \) and
  \begin{equation*}
    D: \ucat{Set} \to \ucat{DiscrCat},
  \end{equation*}
  which for any small set \( A \) gives us the \hyperref[def:discrete_category]{discrete category} whose set of objects is \( A \).

  These were discussed in \fullref{ex:discrete_category_adjunction}, although with \( \cat{Cat} \) rather than \( \ucat{DiscrCat} \).

  It is obvious that for any small category \( \cat{C} \), the functor \( D \bincirc U \) is bijective on objects. We need to verify that it is bijective on morphisms, however, in order to prove that \( D \bincirc U \) is the identity functor on \( \ucat{DiscrCat} \).

  But any functor \( F: \cat{C} \to \cat{D} \) is completely determined by how \( F \) acts on the objects of \( \cat{C} \). Indeed, the only morphisms in \( \cat{C} \) are the identity morphisms and, by \ref{def:functor/CF1}, every object \( X \) in \( \cat{C} \) determines how the identity \( \id_X \in \cat{C}(X) \) is mapped by \( F \).

  Therefore, \( D \) is a left inverse of \( U \). It is also a right inverse --- for any function \( f: A \to B \) between small sets,
  \begin{equation*}
    [U \bincirc D](f) = D(f)\restr_{A} = f.
  \end{equation*}

  Therefore, the forgetful functor \( U \) is invertible, and its inverse is \( D \).
\end{example}

\begin{definition}\label{def:category_equivalence}\mcite[def. 1.3.15]{Leinster2016Basic}
  An \term{equivalence} between the \hyperref[def:category]{categories} \( \cat{C} \) and \( \cat{D} \) is a quadruple
  \begin{equation}\label{eq:def:category_equivalence/signature}
    \begin{aligned}
                F &: \cat{C} \to \cat{D}, \\
                G &: \cat{D} \to \cat{C}, \\
             \eta &: \id_{\cat{C}} \Rightarrow G \bincirc F, \\
      \varepsilon &: F \bincirc G \Rightarrow \id_{\cat{D}},
    \end{aligned}
  \end{equation}
  where \( \eta \) and \( \varepsilon \) are \hyperref[thm:natural_isomorphism]{natural isomorphisms}.

  We call \( \eta \) the \term{unit} of the equivalence and \( \varepsilon \) the \term{counit}.

  If \( (F, G, \eta, \varepsilon) \) is an equivalence, we say that \( \cat{C} \) and \( \cat{D} \) are equivalent categories. This is justified because equivalence of categories is an equivalence relation --- see \fullref{thm:category_equivalence_is_equivalence_relation}.

  Note that an equivalence is not necessarily an \hyperref[def:category_adjunction]{adjunction}, they simply have a common setup. This is discussed in \fullref{thm:adjoint_equivalence}.
\end{definition}

\begin{proposition}\label{thm:category_equivalence_is_equivalence_relation}
  For every \hyperref[def:grothendieck_universe]{Grothendieck universe} \( \mscrU \), \hyperref[def:category_equivalence]{category equivalence} is an \hyperref[def:equivalence_relation]{equivalence relation} on the set \( \obj(\ucat{Cat}) \)
\end{proposition}
\begin{proof}
  \SubProofOf[def:binary_relation/reflexive]{reflexivity} Clearly \( (\id_{\cat{C}}, \id_{\cat{C}}, \id_{\id_{\cat{C}}}, \id_{\id_{\cat{C}}}]) \) is a self-equivalence for the category \( \cat{C} \).

  \SubProofOf[def:binary_relation/symmetric]{symmetry} If \( (F, G, \eta, \varepsilon) \) is an \hyperref[def:category_equivalence]{equivalence} between the categories \( \cat{C} \) and \( \cat{D} \), then \( (G, F, \varepsilon^{-1}, \eta^{-1}) \) is an equivalence between \( \cat{D} \) and \( \cat{C} \).

  \SubProofOf[def:binary_relation/transitive]{transitivity} By \fullref{thm:def:morphism_invertibility/invertible_composition}, the composition of invertible natural isomorphisms is again a natural isomorphism, hence equivalence in \( \obj(\ucat{Cat}) \) is a transitive relation.
\end{proof}

\begin{proposition}\label{thm:discrete_category_equivalence}
  Let \( \cat{C} \) and \( \cat{D} \) be \hyperref[def:discrete_category]{discrete categories}. Then \( \cat{C} \) and \( \cat{D} \) are \hyperref[def:category_equivalence]{equivalent} if and only if the underlying sets \( \obj(\cat{C}) \) and \( \obj(\cat{D}) \) are \hyperref[def:equinumerosity]{equinumerous}.
\end{proposition}
\begin{proof}
  \SufficiencySubProof Suppose that \( (F, G, \eta, \varepsilon) \) be a category equivalence.

  The unit natural transformation \( \eta: \id_{\cat{C}} \Rightarrow G \bincirc F \) consists of a morphism
  \begin{equation*}
    \eta_A: A \to [G \bincirc F](A)
  \end{equation*}
  for every object \( A \) of \( \cat{C} \). Since the only morphisms in \( \cat{C} \) are the identities, it follows that \( \eta_A = \id_A \) and hence \( [G \bincirc F](A) = A \). In particular, this implies that \( \eta \) is the \hyperref[eq:def:functor_category/identity]{identity natural transformation} on \( \id_{\cat{C}} \) and that the restriction \( G\restr_{\obj(D)} \) is a left inverse of \( F\restr_{\obj(\cat{C})} \).

  Similarly, for the counit \( \varepsilon: F \bincirc G \Rightarrow \id_{\cat{D}} \), for every object \( X \) in \( \cat{D} \) we have \( \varepsilon_X = \id_X \) and hence \( [F \bincirc G](X) = X \). Thus, \( \eta \) is the identity natural transformation on \( \id_{\cat{D}} \) and \( G\restr_{\obj(D)} \) is a right inverse of \( F\restr_{\obj(\cat{C})} \).

  Therefore, the sets \( \obj(\cat{C}) \) and \( \obj(D) \) are equinumerous.

  \NecessitySubProof Suppose that \( F: \obj(\cat{C}) \to \obj(D) \) is a bijective function. Then it is an isomorphism in the category \( \ucat{Cat} \) for an appropriate universe \( \mscrU \), hence it induces an equivalence between \( \cat{C} \) and \( \cat{D} \).
\end{proof}

\begin{proposition}\label{thm:opposite_of_category_equivalence}
  The \hyperref[def:opposite_category]{opposite} of \hyperref[def:category_equivalence]{equivalent categories} are equivalent.

  More precisely, if \( (F, G, \eta, \varepsilon) \) is an \hyperref[def:category_equivalence]{equivalence} between the categories \( \cat{C} \) and \( \cat{D} \), then
  \begin{equation*}
    \begin{aligned}
                G^\oppos &: \cat{D}^\oppos \to \cat{C}^\oppos, \\
                F^\oppos &: \cat{C}^\oppos \to \cat{D}^\oppos, \\
      \varepsilon^\oppos &: \id_{\cat{D}} \Rightarrow [F \bincirc G]^\oppos, \\
             \eta^\oppos &: \underbrace{[G \bincirc F]^\oppos}_{G^\oppos \bincirc F^\oppos} \Rightarrow \id_{\cat{C}^\oppos},
    \end{aligned}
  \end{equation*}
  is an equivalence between \( \cat{C}^\oppos \) and \( \cat{D}^\oppos \).

  This is part of the duality principles listed in \fullref{thm:categorical_principle_of_duality}.
\end{proposition}
\begin{proof}
  Trivial.
\end{proof}

\begin{proposition}\label{thm:equivalence_induces_fully_faithful_and_essentially_surjective_functor}
  In any \hyperref[def:category_equivalence]{category equivalence} \( (F, G, \eta, \varepsilon) \), the functor \( F \) is \hyperref[def:functor_invertibility/fully_faithful]{fully faithful} and \hyperref[def:functor_invertibility/surjective_on_objects]{essentially surjective on objects}.

  The converse of this statement is \fullref{thm:fully_faithful_and_essentially_surjective_functor_induces_equivalence}.
\end{proposition}
\begin{proof}
  \SubProofOf[def:functor_invertibility/surjective_on_objects]{essential surjectivity} For any object \( X \) in \( \cat{D} \), \( A \coloneqq G(X) \) is an object in \( \cat{C} \).

  By definition of category equivalence, the morphism
  \begin{equation*}
    \varepsilon_X: \underbrace{[F \bincirc G](X)}_{F(A)} \to X
  \end{equation*}
  is an isomorphism.

  Therefore, for every object \( X \) in \( \cat{D} \), there exists some object \( A \) in \( \cat{C} \) such that \( F(A) \cong X \). Thus, \( F \) is essentially surjective.

  \SubProofOf[def:functor_invertibility/faithful]{faithfulness} Fix some objects \( A \) and \( B \) in \( \cat{C} \). Let \( f_1: A \to B \) and \( f_2: A \to B \) be morphisms such that \( F(f_1) = F(f_2) \).

  From the naturality of \( \eta \) it follows that the following diagram commutes:
  \begin{equation}\label{eq:thm:equivalence_induces_fully_faithful_and_essentially_surjective_functor/faithfullness}
    \begin{aligned}
      \includegraphics[page=1]{output/thm__equivalence_induces_fully_faithful_and_essentially_surjective_functor}
    \end{aligned}
  \end{equation}

  Therefore, \( \eta_B \bincirc f_1 = \eta_B \bincirc f_2 \) and, since \( \eta_B \) is left-cancellative, \( f_1 = f_2 \).

  \SubProofOf[def:functor_invertibility/full]{fullness} Fix some objects \( A \) and \( B \) in \( \cat{C} \). Let \( g: F(A) \to F(B) \) be an arbitrary morphism.

  We can define a morphism \( f: A \to B \) via the composition
  \begin{equation}\label{eq:thm:equivalence_induces_fully_faithful_and_essentially_surjective_functor/fullness/def}
    \begin{aligned}
      \includegraphics[page=2]{output/thm__equivalence_induces_fully_faithful_and_essentially_surjective_functor}
    \end{aligned}
  \end{equation}

  By naturality of \( \eta \), the following diagram commutes:
  \begin{equation}\label{eq:thm:equivalence_induces_fully_faithful_and_essentially_surjective_functor/fullness/eta_nat}
    \begin{aligned}
      \includegraphics[page=3]{output/thm__equivalence_induces_fully_faithful_and_essentially_surjective_functor}
    \end{aligned}
  \end{equation}

  Therefore,
  \begin{equation*}
     G(g) = [G \bincirc F](f).
  \end{equation*}

  We have already shown that \( F \) is faithful and, by \fullref{thm:category_equivalence_is_equivalence_relation}, \( G \) is also faithful. Since \( g \) and \( F(f) \) are parallel, it follows that they are equal.

  Therefore, \( F \) is full.
\end{proof}

\begin{remark}\label{rem:adjoint_equivalence_induces_fully_faithful_and_essentially_surjective_functor}
  In the fullness proof of \fullref{thm:equivalence_induces_fully_faithful_and_essentially_surjective_functor}, we can use another argument if \( (F, G, \eta, \varepsilon) \) is an \hyperref[def:adjoint_equivalence]{adjoint equivalence}.

  If the triangle diagram \eqref{eq:def:category_adjunction/d_triangle} commutes, the dashed lines in the following diagram also commute:
  \begin{equation}\label{eq:thm:equivalence_induces_fully_faithful_and_essentially_surjective_functor/fullness/triangles}
    \begin{aligned}
      \includegraphics[page=1]{output/rem__adjoint_equivalence_induces_fully_faithful_and_essentially_surjective_functor}
    \end{aligned}
  \end{equation}
\end{remark}

\begin{theorem}[Fully faithful and essentially surjective functor induces equivalence]\label{thm:fully_faithful_and_essentially_surjective_functor_induces_equivalence}
  Every \hyperref[def:functor_invertibility/fully_faithful]{fully faithful} and \hyperref[def:functor_invertibility/surjective_on_objects]{essentially surjective on objects} functor induces a \hyperref[def:category_equivalence]{category equivalence}.

  More precisely, given a functor \( F: \cat{C} \to \cat{D} \) that is fully faithful and essentially surjective on objects, there exists a functor \( G: \cat{D} \to \cat{C} \) and \hyperref[def:natural_transformation]{natural transformations}
  \begin{align*}
    \eta        &: \id_{\cat{C}} \Rightarrow G \bincirc F, \\
    \varepsilon &: F \bincirc G \Rightarrow \id_{\cat{D}},
  \end{align*}
  such that the quadruple \( (F, G, \eta, \varepsilon) \) is a \hyperref[def:category_equivalence]{category equivalence}.

  In \hyperref[def:zfc]{\logic{ZF}}, this theorem is equivalent to the \hyperref[def:zfc/choice]{axiom of choice} --- see \fullref{thm:axiom_of_choice_equivalences/fully_faithful_essentially_surjective}.

  We prove the converse of this statement separately in \fullref{thm:equivalence_induces_fully_faithful_and_essentially_surjective_functor}.
\end{theorem}
\begin{proof}
  \ImplicationSubProof[def:zfc/choice]{the axiom of choice}[thm:fully_faithful_and_essentially_surjective_functor_induces_equivalence]{functors induce equivalences} Suppose that the axiom of choice holds and let \( F \) be a fully faithful functor that is surjective on objects.

  From essential surjectivity of \( F \), it follows that for every object \( X \) in \( \cat{D} \), the preimage of \( X \) under \( F \) is nonempty. The preimage of \( X \) is the set \( \mscrA_X \) of objects in \( \cat{C} \) such that \( A \in \mscrA_X \) if and only if \( F(A) \cong X \). We use the axiom of choice on the family \( \set{ \mscrA_X }_{X \in \cat{D}} \) to select a single preimage for every \( X \), which we denote by \( G(X) \).

  Again using the axiom of choice, we pick an isomorphism \( \varepsilon_X: F(G(X)) \to X \) for every \( X \).

  We have defined a function \( G \) from \( \obj(\cat{D}) \) to \( \obj(\cat{C}) \). In order to \( G \) to become a functor, we must extend it to morphisms. Let \( X \) and \( Y \) be objects in \( \cat{D} \) and \( g: X \to Y \) be any morphism.

  Consider the morphism
  \begin{equation*}
    \varepsilon_Z^{-1} \bincirc f \bincirc \varepsilon_X: [F \bincirc G](X) \to [F \bincirc G](Y).
  \end{equation*}

  Since \( F \) is fully faithful, there exists a unique morphism \( g \) in \( \cat{D}(G(X), G(Y)) \) such that the following diagram commutes:
  \begin{equation}\label{eq:thm:fully_faithful_and_essentially_surjective_functor_induces_equivalence/inverse_morphism_definition}
    \begin{aligned}
      \includegraphics[page=1]{output/thm__fully_faithful_and_essentially_surjective_functor_induces_equivalence}
    \end{aligned}
  \end{equation}

  We define \( G(g) \coloneqq g \).

  In order to prove that \( G \) is a functor, we need to show that \ref{def:functor/CF1} and \ref{def:functor/CF2} hold.

  For \ref{def:functor/CF1}, note that the following diagram commutes for any object \( X \) in \( \cat{D} \):
  \begin{equation}\label{eq:thm:fully_faithful_and_essentially_surjective_functor_induces_equivalence/identity}
    \begin{aligned}
      \includegraphics[page=2]{output/thm__fully_faithful_and_essentially_surjective_functor_induces_equivalence}
    \end{aligned}
  \end{equation}

  Note that \eqref{eq:thm:fully_faithful_and_essentially_surjective_functor_induces_equivalence/inverse_morphism_definition} also commutes if we replace \( [F \bincirc G](\id_X) \) with \( F(\id_{G(X)}) \). Since \( F \) is fully faithful, this morphism is unique and it follows that
  \begin{equation*}
    [F \bincirc G](\id_X) = F(\id_{G(X)}).
  \end{equation*}

  For \ref{def:functor/CF2}, analogously, given morphisms \( g: X \to Y \) and \( f: Y \to Z \), the following diagram commutes:
  \begin{equation}\label{eq:thm:fully_faithful_and_essentially_surjective_functor_induces_equivalence/composition}
    \begin{aligned}
      \includegraphics[page=3]{output/thm__fully_faithful_and_essentially_surjective_functor_induces_equivalence}
    \end{aligned}
  \end{equation}

  We have implicitly used \fullref{rem:inverting_isomorphisms_may_preserve_commutativity} above.

  By the same uniqueness argument used for \ref{def:functor/CF1}, we conclude that
  \begin{equation*}
    G(f \bincirc q) = G(f) \bincirc G(g).
  \end{equation*}

  We have shown that \( G \) is a functor. Furthermore, \( \varepsilon \) is a natural transformation since, for any morphism \( g: X \to Y \) in \( \cat{D} \), the following diagram commutes:
  \begin{equation}\label{eq:thm:fully_faithful_and_essentially_surjective_functor_induces_equivalence/varepsilon}
    \begin{aligned}
      \includegraphics[page=4]{output/thm__fully_faithful_and_essentially_surjective_functor_induces_equivalence}
    \end{aligned}
  \end{equation}

  To show that \( F \) induces an equivalence, it now only remains to define a unit natural transformation \( \eta: \id_{\cat{C}} \to G \bincirc F \). For every object \( A \) in \( \cat{C} \) we have an isomorphism
  \begin{equation*}
    \varepsilon_{F(A)}^{-1}: F(A) \to [F \bincirc G \bincirc F](A).
  \end{equation*}

  Using \( G(\varepsilon_{F(A)}^{-1}) \) will get us nowhere. Fortunately, \( F \) is fully faithful, so there is a bijective function
  \begin{equation*}
    \varphi: \cat{D}(F(A), [F \bincirc G \bincirc F](A)) \to \cat{C}(A, [F \bincirc G](A)).
  \end{equation*}

  Hence, we can define
  \begin{equation*}
    \begin{aligned}
      &\eta: \id_{\cat{C}} \Rightarrow F \bincirc G \\
      &\eta_A \coloneqq \varphi(\varepsilon_{F(A)}^{-1})
    \end{aligned}
  \end{equation*}
  so that \( F(\eta_A) = \varepsilon_{F(A)}^{-1} \). By \fullref{thm:def:functor_invertibility/fully_faithful_reflects_invertible}, since \( \varepsilon_{F(A)} \) is an isomorphism, \( \eta_A \) is also an isomorphism.

  By naturality of \( F(\eta_A) \), the following diagram commutes:
  \begin{equation}\label{eq:thm:fully_faithful_and_essentially_surjective_functor_induces_equivalence/varepsilon_image_nat}
    \begin{aligned}
      \includegraphics[page=5]{output/thm__fully_faithful_and_essentially_surjective_functor_induces_equivalence}
    \end{aligned}
  \end{equation}

  Hence, by \fullref{thm:commutative_diagrams_preserved_and_reflected}, the following diagram also commutes:
  \begin{equation}\label{eq:thm:fully_faithful_and_essentially_surjective_functor_induces_equivalence/varepsilon_source_nat}
    \begin{aligned}
      \includegraphics[page=6]{output/thm__fully_faithful_and_essentially_surjective_functor_induces_equivalence}
    \end{aligned}
  \end{equation}

  Therefore, the quadruple \( (F, G, \eta, \varepsilon) \) is an equivalence of categories.

  \ImplicationSubProof[thm:fully_faithful_and_essentially_surjective_functor_induces_equivalence]{functors induce equivalences}[def:zfc/choice]{the axiom of choice} Let \( \mscrA \) be a family of nonempty sets. Let \( \cat{D} \) be the \hyperref[def:discrete_category]{discrete category} induced by \( \mscrA \).

  Define the category \( \cat{C} \) as follows:
  \begin{itemize}
    \item The \hyperref[def:category/objects]{set of objects} \( \obj(\cat{C}) \) is the \hyperref[def:disjoint_union]{disjoint union} \( \coprod_{A \in \mscrA} A \).

    \item The \hyperref[def:category/morphisms]{set of morphisms} \( \cat{C}((A, x), (B, y)) \) has a single morphism if \( A = B \) and no morphisms otherwise. This single morphism can be encoded as the triple \( (A, x, y) \).

    \item The \hyperref[def:category/composition]{composition of the morphisms} \( (A, x, y) \) and \( (A, y, z) \) is the morphism \( (A, x, z) \).

    \item The \hyperref[def:category/identity]{identity morphism} on the object \( (A, x) \in \cat{C} \) is \( (A, x, x) \).
  \end{itemize}

  Define the functor
  \begin{equation*}
    \begin{aligned}
      &F: \cat{C} \to \cat{D} \\
      &F(A, x) \coloneqq A \\
      &F(A, x, y) \coloneqq \id_A
    \end{aligned}
  \end{equation*}
  that maps each point \( x \in A \in \mscrA \) into the set \( A \) it belongs to. We have taken the disjoint union of \( \mscrA \) since otherwise there may not be a canonical choice of set \( A \) for \( F \) to send \( x \) to. Thus, the functor is surjective on objects (not essentially surjective but actually surjective).

  Note that \( \cat{D}(F(A, x), F(B, y)) \) has a single morphism if \( A = B \) and is empty otherwise. From this it follows that \( F \) is fully faithful.

  Therefore, \( F \) induces a \hyperref[def:category_equivalence]{category equivalence} \( (F, G, \eta, \varepsilon) \). The functor \( G \) chooses an object \( (A, x) \) of \( \cat{C} \) for each object \( A \) of \( \cat{D} \). This induces a \hyperref[def:choice_function]{choice function} on \( \mscrA \).

  We have shown that the axiom of choice holds.
\end{proof}

\begin{definition}\label{def:skeletal_category}\mcite[93]{MacLane1998Categories}
  We say that a category is \term{skeletal} if wherever two objects are isomorphic, they are equal.

  If \( \cat{S} \) is a subcategory of \( \cat{C} \) and if they are \hyperref[def:category_equivalence]{equivalent}, we say that \( \cat{S} \) is a \term{skeleton} of \( \cat{C} \).
\end{definition}

\begin{example}\label{ex:skeleton_of_set}
  For a Grothendieck universe \( \mscrU \), the \hyperref[def:cardinal]{cardinal numbers} in \( \mscrU \) are a \hyperref[def:skeletal_category]{skeletal subcategory} of \( \ucat{Set} \).
\end{example}

\begin{theorem}[Category skeleton existence]\label{thm:category_skeleton_existence}
  Every \hyperref[def:category]{category} has a \hyperref[def:skeletal_category]{skeleton}.
\end{theorem}
\begin{comments}
  \item In \hyperref[def:zfc]{\logic{ZF}}, this theorem is equivalent to the \hyperref[def:zfc/choice]{axiom of choice} --- see \fullref{thm:axiom_of_choice_equivalences/skeletons}.
\end{comments}
\begin{proof}
  \ImplicationSubProof[def:zfc/choice]{the axiom of choice}[thm:category_skeleton_existence]{skeleton existence} Suppose that the axiom of choice holds and let \( F \) be a fully faithful functor that is surjective on objects.

  Fix a category \( \cat{C} \). We will build a subcategory \( \cat{S} \) of \( \cat{C} \) whose inclusion functor \( \Iota: \cat{S} \to \cat{C} \) is essentially surjective and fully faithful. By \fullref{thm:fully_faithful_and_essentially_surjective_functor_induces_equivalence}, this is sufficient for \( \cat{S} \) and \( \cat{C} \) to be equivalent.

  In order for \( \Iota \) to be a full functor, \( \cat{S} \) must be a full subcategory. Therefore, when building \( \cat{S} \), we can only remove objects and must preserve the morphism sets for the remaining objects.

  Denote by \( \obj(\cat{C}) / \cong \) the quotient of \( \cat{C} \) by the isomorphism relation. Using the axiom of choice, we can obtain a \hyperref[def:choice_function]{choice function} \( c: (\obj(\cat{C}) / \cong) \to \obj(\cat{C}) \).

  Define \( \cat{S} \) as the subcategory induced by the image \( c[\obj(\cat{C}) / \cong] \).

  Now consider the \hyperref[def:subcategory]{inclusion functor} \( \Iota: \cat{S} \to \cat{C} \). For every pair \( X \) and \( Y \) of objects in \( \cat{S} \), clearly
  \begin{equation*}
    \cat{S}(X, Y) = \cat{C}(\Iota(X), \Iota(Y)).
  \end{equation*}

  Hence, \( \Iota \) is fully faithful.

  Now let \( X \) and \( Y \) be objects of \( \cat{C} \). Since the objects of \( \cat{S} \) were chosen from isomorphism classes of \( \cat{C} \), there exist objects \( X' \) and \( Y' \) in \( \cat{S} \) that are isomorphic to \( X \) and \( Y \), correspondingly. Hence, \( \Iota \) is essentially surjective.

  Therefore, \( F \) satisfies \fullref{thm:fully_faithful_and_essentially_surjective_functor_induces_equivalence}, from which is follows that \( \cat{C} \) and \( \cat{S} \) are equivalent.

  \ImplicationSubProof[thm:category_skeleton_existence]{skeleton existence}[def:zfc/choice]{the axiom of choice} Suppose that every category has a skeleton.

  Let \( \mscrA \) be a family of nonempty sets. Construct a category \( \cat{C} \)from the \hyperref[def:disjoint_union]{disjoint union} \( \coprod_{A \in \mscrA} A \), where a morphism exists only between members of the same set. This construction is performed in detail in our proof of \fullref{thm:fully_faithful_and_essentially_surjective_functor_induces_equivalence}.

  Then \( \cat{C} \) has a skeleton \( \cat{S} \). All morphisms in \( \cat{C} \) are isomorphisms, hence the set \( \obj(\cat{S}) \) contains exactly one representative for each set in the family \( \mscrA \).

  More precisely, define the set
  \begin{equation*}
    S \coloneqq \set{ x \given (A, x) \in \obj(\cat{S}) }.
  \end{equation*}

  Then \( S \) satisfies \fullref{thm:axiom_of_choice_equivalences/choice_sets}.

  Since the family \( \mscrA \) is arbitrary, we conclude that the axiom of choice holds.
\end{proof}

\begin{remark}\label{rem:skeletons_and_preorder_categories}
  Rather than defining representatives of equivalence classes, as in \fullref{thm:category_skeleton_existence}, we can define morphisms between the equivalence classes themselves, as in \fullref{def:antisymmetric_quotient}.

  This does not require the \hyperref[def:zfc/choice]{axiom of choice}, but is rarely applicable, unfortunately. One case where it is applicable is in \hyperref[def:preorder_category]{preorder categories} --- see \fullref{thm:order_category_isomorphism}.
\end{remark}

\begin{definition}\label{def:groupoid}
  A \term{groupoid} is a category whose only morphisms are \hyperref[def:morphism_invertibility/isomorphism]{isomorphisms}.
\end{definition}

\begin{definition}\label{def:monoid_delooping}\mcite[def. 1.1.7]{Perrone2021Categories}
  Let \( (M, \cdot, e) \) be a \hyperref[def:monoid]{monoid}. The \term{delooping} \( \cat{B}_M \) of \( M \) is the following category:
  \begin{itemize}
    \item The \hyperref[def:category/objects]{set of objects} \( \obj(\cat{B}_M) \) is the singleton set \( \set{ \Anon } \), where \( \Anon \) is any set not in the set of all morphisms \( M \).

    \item The only \hyperref[def:category/morphisms]{set of morphisms} \( \cat{B}_M(\Anon) \) is the underlying set \( M \) of the monoid.

    \item The \hyperref[def:category/composition]{composition of the morphisms} \( x \) and \( y \) is the multiplication:
    \begin{equation*}
      y \bincirc x \coloneqq y \cdot x.
    \end{equation*}

    Note how we write composition in the same order as multiplication. This may seem to contradict the general convention, however it is consistent with groups being regarded as sets of invertible transformations.

    \item The \hyperref[def:category/identity]{identity morphism} is \( e \).
  \end{itemize}
\end{definition}

\begin{proposition}\label{thm:delooping_of_group}
  The \hyperref[def:monoid_delooping]{delooping} of a \hyperref[def:group]{group} \( G \) is a \hyperref[def:groupoid]{groupoid}.
\end{proposition}
\begin{comments}
  \item There is a restricted form of a converse --- see \fullref{thm:connected_delooping}.
\end{comments}
\begin{proof}
  Trivial.
\end{proof}

\begin{definition}\label{def:connected_category}\mcite[88]{MacLane1998Categories}
  We say that a \hyperref[def:category]{category} is \term{connected} its \hyperref[def:category]{underlying multigraph} is \hyperref[def:graph_connectedness/weak]{weakly connected}.
\end{definition}

\begin{proposition}\label{thm:connected_delooping}
  For every \hyperref[def:connected_category]{connected} \hyperref[def:groupoid]{groupoid} \( \cat{G} \) there exists a \hyperref[def:group]{group} \( G \) such that \( \cat{G} \) is \hyperref[def:category_equivalence]{equivalent} to the \hyperref[def:monoid_delooping]{delooping} \( \cat{B}_G \). Furthermore, if \( \cat{G} \) has only one object, then this equivalence is an \hyperref[rem:category_similarity/isomorphism]{isomorphism}.
\end{proposition}
\begin{comments}
  \item See \fullref{thm:delooping_of_group} for a much simpler converse.
\end{comments}
\begin{proof}
  \SubProof{Proof that all objects are isomorphic} Fix two objects \( A \) and \( B \) in \( \cat{G} \). We will show that they are isomorphic.

  Since \( \cat{G} \) is connected, there exists an undirected path connecting \( A \) with \( B \). Since \( \cat{G} \) is a groupoid, all morphisms in the path are invertible, and hence there also exists a directed path from \( A \) to \( B \). This directed path can be composed into an isomorphism from \( A \) to \( B \).

  Therefore, all objects in \( \cat{G} \) are isomorphic.

  \SubProof{Construction of group} \Fullref{thm:category_skeleton_existence} implies that there exists a \hyperref[def:skeletal_category]{skeleton} \( \cat{S} \) of \( \cat{G} \). Let \( (F, G, \eta, \varepsilon) \) be the equivalence between \( \cat{G} \) and \( \cat{S} \).

  Since all objects in \( \cat{G} \) are isomorphic, \( \cat{S} \) has a single object \( \anon \). Hence, it is possible to compose any two morphisms in \( \cat{S} \).

  Define the group \( G \) as follows:
  \begin{itemize}
    \item Let the underlying set of \( G \) be the set of all morphisms of \( \cat{S} \).
    \item Define the group operation as
    \begin{equation*}
      x \cdot y \coloneqq x \bincirc y.
    \end{equation*}
  \end{itemize}

  It follows that the neutral element on \( G \) is then the identity morphism on \( \anon \) and that the inverse of \( x \) is the inverse morphism \( x^{-1} \).

  \SubProof{Proof of equivalence} Note that the skeleton \( \cat{S} \) and the delooping \( \cat{B}_G \) are equal. Since \( \cat{G} \) and \( \cat{S} \) are equivalent, so are \( \cat{G} \) and \( \cat{B}_G \)
\end{proof}

\begin{definition}\label{def:preorder_category}\mcite[11]{MacLane1998Categories}
  We will say that a \hyperref[def:category]{category} is a \term{preorder category} if any two parallel morphisms are equal.

  This is equivalent to saying that the function for every two objects \( A \) and \( B \) in \( \cat{P} \), whenever the set \( \cat{P}(A, B) \) is at most a singleton.
\end{definition}
\begin{comments}
  \item Preorder categories are often conflated with preordered sets due to \fullref{thm:order_category_isomorphism}.

  \item As shown in \fullref{ex:preorder_nonuniqueness} and discussed in \fullref{thm:order_category_isomorphism}, a preorder category may not be \hyperref[def:skeletal_category]{skeletal}.
\end{comments}

\begin{theorem}[Ordered sets as categories]\label{thm:order_category_isomorphism}
  The category \( \cat{PreOrd} \) of small \hyperref[def:preordered_set]{preordered sets} and (nonstrict) \hyperref[def:preordered_set/homomorphism]{monotone maps} is \hyperref[rem:category_similarity/isomorphism]{isomorphic} to the category of small \hyperref[def:preorder_category]{preorder categories} and their functors.

  Furthermore, \hyperref[def:partially_ordered_set]{partially ordered sets} correspond to \hyperref[def:skeletal_category]{skeletal} preorder categories.
\end{theorem}
\begin{proof}
  \SubProof{Proof that preordered sets induce preorder categories} Let \( (P, \leq) \) be a small preordered set. Let \( \cat{P} \) be the \hyperref[def:directed_multigraph_free_category]{free category} obtained by regarding \( (P, \leq) \) as a \hyperref[def:directed_multigraph]{directed multigraph}. Explicitly, the category \( \cat{P} \) is built as follows:
  \begin{itemize}
    \item The \hyperref[def:category/objects]{set of objects} \( \obj(\cat{P}) \) is simply \( P \).

    \item The \hyperref[def:category/morphisms]{set of morphisms} \( \cat{P}(x, y) \) consists of the tuple \( (x, y) \) if \( x \leq y \) and is empty otherwise.

    \item The \hyperref[def:category/composition]{composition of the morphisms} \( (x, y) \) and \( (y, z) \) is simply \( (x, y) \). This is well-defined because of the \hyperref[def:binary_relation/transitive]{transitivity} of \( \leq \).

    \item The \hyperref[def:category/identity]{identity morphism} on the object \( x \in \cat{C} \) is \( (x, x) \). This is well-defined because of the \hyperref[def:binary_relation/reflexive]{reflexivity} of \( \leq \).
  \end{itemize}

  This is a preorder category because \( \leq \) is a binary relation and ordered tuples with the same elements are equal.

  Now let \( f: P \to Q \) be a nonstrict monotone map. It induces the functor
  \begin{equation*}
    \begin{aligned}
      &F: \cat{P} \to \cat{Q} \\
      &F(x) \coloneqq f(x) \\
      &F(x, y) \coloneqq (f(x), f(y)).
    \end{aligned}
  \end{equation*}

  \ref{def:functor/CF1} is immediate and \ref{def:functor/CF2} follows from \eqref{eq:def:order_function/preserving}, hence \( F \) is indeed a functor.

  \SubProof{Proof that preorder categories induce preordered sets} Let \( \cat{P} \) be a small category. Define the binary relation
  \begin{equation*}
    X \leq Y \T{if and only if} \cat{P}(X, Y) \neq \varnothing.
  \end{equation*}

  This is a binary relation over the set \( P \coloneqq \obj(\cat{P}) \). It is reflexive because of the existence of identity morphisms in \( \cat{P} \) and transitive because of the requirement that the composition of compatible morphisms exists.

  Therefore, \( (P, \leq) \) is a preordered set.

  Given a functor \( F: \cat{P} \to \cat{Q} \), the restriction \( F\restr_{\obj(\cat{P})} \) is a monotone map from \( (P, \leq_P) \) to \( (Q, \leq_Q) \).

  Indeed, if \( X \leq Y \) for \( X, Y \in P \), then \( \cat{P}(X, Y) \neq \varnothing \). Hence, \( \cat{P}(F(X), F(Y)) \neq \varnothing \) and \( F(X) \leq F(Y) \).

  \SubProof{Proof of category isomorphism} We have implicitly defined a functor between the categories of preordered sets and of preorder categories. Isomorphism requires that these functors are mutually inverse. The verification of this is trivial and we only mention it to highlight that we have not ignored this issue.

  \SubProof{Proof for partial orders} For a preorder category, there is an isomorphism between \( x \) and \( y \) if and only if there is both a morphism from \( x \) to \( y \) and one from \( y \) to \( x \). This isomorphism may not be unique as a consequence of \fullref{ex:preorder_nonuniqueness}. Uniqueness requires \( x = y \) to hold in this case, which is in turn equivalent to partial order \hyperref[def:binary_relation/antisymmetric]{antisymmetry}.
\end{proof}

\begin{definition}\label{def:category_disjoint_union}\mimprovised
  We define the \term{disjoint union} \( \coprod_{k \in \mscrK} \cat{C}_k \) of the family \( \seq{ \cat{C}_k }_{k \in \mscrK} \) of \hyperref[def:category]{categories} as follows:

  \begin{itemize}
    \item The \hyperref[def:category/objects]{set of objects} is the \hyperref[def:disjoint_union]{disjoint union} of the sets of objects of the family.

    \item The \hyperref[def:category/morphisms]{set of morphisms} from \( \iota_k(A) \) to \( \iota_m(B) \) is the set of morphisms from \( A \) to \( B \), which is empty if \( k \neq m \).

    \item The \hyperref[def:category/composition]{composition of morphisms} from \( \iota_k(A) \) to \( \iota_k(B) \) is inherited from \( \cat{C}_k \).

    \item The \hyperref[def:category/identity]{identity morphism} on \( \iota_k(A) \) is the identity in \( \cat{C}_k \).
  \end{itemize}
\end{definition}

\begin{proposition}\label{thm:category_coproduct}
  The \hyperref[def:discrete_category_limits]{categorical coproduct} of the family \( \seq{ \cat{C}_k }_{k \in \mscrK} \) in the category \hyperref[def:category_of_small_categories]{\( \ucat{Cat} \)} of \( \mscrU \)-small categories is their \hyperref[def:category_disjoint_union]{disjoint union} \( \coprod_{k \in \mscrK} \cat{C}_k \).
\end{proposition}
\begin{proof}
  For any \hyperref[def:category_of_cones/cocone]{cocone} \( (\cat{A}, \alpha) \) for the discrete diagram \( \seq{ \cat{C}_k }_{k \in \mscrK} \), we can define the following functor:
  \begin{equation*}
    \begin{aligned}
      &L: \coprod_{k \in \mscrK} \cat{C}_k \to \cat{A}, \\
      &L_{\cat{A}}\parens[\Big]{ \iota_k(X) } \coloneqq \alpha_k(X), \\
      &L_{\cat{A}}\parens[\Big]{ f: X \to Y } \coloneqq \alpha_k(f).
    \end{aligned}
  \end{equation*}
\end{proof}

\begin{proposition}\label{thm:order_category_isomorphism_properties}
  Let \( (P, \leq) \) be a \hyperref[def:partially_ordered_set]{partially ordered set} and let \( \cat{P} \) be its corresponding category induced by \fullref{thm:order_category_isomorphism}.

  \begin{thmenum}
    \thmitem{thm:order_category_isomorphism_properties/opposite} The \hyperref[def:opposite_category]{opposite category} \( \cat{P}^\oppos \) corresponds to the \hyperref[thm:preorder_duality]{opposite partially ordered set} \( (P, \geq) \).

    \thmitem{thm:order_category_isomorphism_properties/universal} If \( \cat{P} \) has an \hyperref[def:universal_objects/initial]{initial object}, since it is a skeletal category, this initial object is unique.

    An object \( I \) is an initial object if and only if it is the \hyperref[def:extremal_points/top_and_bottom]{bottom} of \( P \).

    \hyperref[thm:categorical_principle_of_duality]{Dually}, \( T \) is a \hyperref[def:universal_objects/terminal]{terminal object} if and only if it is the \hyperref[def:extremal_points/top_and_bottom]{top} of \( P \).

    \thmitem{thm:order_category_isomorphism_properties/coproduct} The coproduct of partially ordered sets is their disjoint union in the sense of \fullref{def:category_disjoint_union}.
  \end{thmenum}
\end{proposition}
\begin{proof}
  \SubProofOf{thm:order_category_isomorphism_properties/opposite} Trivial.

  \SubProofOf{thm:order_category_isomorphism_properties/universal} Trivial.

  \SubProofOf{thm:order_category_isomorphism_properties/coproduct} Follows from \fullref{thm:category_coproduct}.
\end{proof}
