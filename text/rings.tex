\subsection{Rings}\label{subsec:rings}

\paragraph{Rings}

\begin{definition}\label{def:ring}\cite[141]{Knapp2016BasicAlgebra}
  A \term[bg=пръстен (\cite[3]{КоцевСидеров2016}), ru=кольцо (\cite[sec. 15.1]{Тыртышников2007})]{ring} is a \hyperref[def:semiring]{semiring} with additive inverses. More precisely, this means that the additive monoid is a group.

  The absorption condition \eqref{eq:def:semiring/absorption} becomes redundant due to \fullref{thm:def:ring/zero_absorbing}.

  As for semirings, rings can also be nonunital, with \hyperref[def:semiring_ideal]{ring ideals} being the main example.

  Rings have the following metamathematical properties:
  \begin{thmenum}
    \thmitem{def:ring/theory} We can construct a \hyperref[def:first_order_theory]{first-order theory} for rings by adding a unary functional symbol \( - \) and the involution axiom \eqref{eq:def:group/theory/inverse_axiom} to the \hyperref[def:semiring/theory]{theory of semirings}.

    \thmitem{def:ring/homomorphism} A \hyperref[def:first_order_homomorphism]{first-order homomorphism} between the rings \( R \) and \( T \) is a \hyperref[def:semiring/homomorphism]{semiring homomorphism} \( \varphi: R \to T \) that additionally preserves additive inverses.

    As shown in \fullref{thm:group_homomorphism_single_condition}, this condition is not only redundant, but the additive invertibility automatically implies that \( \varphi(0_R) = 0_S \).

    \thmitem{def:ring/submodel} The set \( A \subseteq R \) is a \hyperref[def:first_order_submodel]{submodel} of \( R \) if it is both a \hyperref[def:semiring]{sub-semiring} of \( R \) and an additive submonoid of \( R \).

    As a consequence of \fullref{thm:positive_formulas_preserved_under_homomorphism}, the image of a ring homomorphism is a subring of its codomain.

    \thmitem{def:ring/category} We denote the corresponding \hyperref[def:category_of_small_first_order_models]{category of \( \mscrU \)-small models} by \( \ucat{Ring} \) and the category of nonunital rings by \( \ucat{Rng} \).

    \begin{itemize}
      \item Obviously \( \ucat{Ring} \) is \hyperref[def:concrete_category]{concrete} over \( \ucat{Rng} \).
      \item The latter is \hyperref[def:concrete_category]{concrete} over the category of semirings \hyperref[def:semiring/category]{\( \ucat{SRing} \)}, and hence also \hyperref[def:monoid/category]{\( \ucat{Mon} \)} and \hyperref[def:pointed_set/category]{\( \ucat{Set}^* \)}.

      It is also concrete over \hyperref[def:group/category]{\( \ucat{Grp} \)} via the additive group.

      \item The category \( \ucat{Rng} \) has a \hyperref[def:universal_objects/zero]{zero object} --- the one-element ring \( \set{ 0 } \) --- which allows us to consider kernels and quotients of nonunital rings.

      Although \( \ucat{Ring} \) has no zero object, we consider kernels and quotients with respect to \( \ucat{Rng} \).
    \end{itemize}

    \thmitem{def:ring/commutative} If multiplication is commutative, we call the ring itself \term{commutative}. Unless multiplication corresponds to function composition, most rings we will encounter will be commutative.

    We denote the category of commutative rings by \( \cat{CRing} \).

    \thmitem{def:ring/trivial} Similarly to semirings, any single-element ring is as trivial object in \( \ucat{Rng} \) in the sense of \fullref{def:trivial_object}. It is not a zero object in \( \ucat{Ring} \), although we may refer to \( \set{ 0 } \) as \enquote{the trivial ring}.

    \thmitem{def:ring/kernel} \Fullref{thm:ring_zero_morphisms/kernel} implies that the \hyperref[def:zero_morphisms/kernel]{categorical kernel} of a morphism \( \varphi: R \to T \) in \( \ucat{Rng} \) is the additive group kernel
    \begin{equation*}
      \ker \varphi \coloneqq \varphi^{-1}(0_T) = \set{ x \in R \given \varphi(x) = 0_T }.
    \end{equation*}

    It is a \hyperref[def:semiring_ideal]{two-sided ideal} (and hence a nonunital subring) as a consequence of \fullref{thm:kernel_is_ideal}.

    We use this definition for morphisms in \( \ucat{Ring} \) also, although the latter are no longer categorical kernels.

    \thmitem{def:ring/quotient} Similarly to groups, \hyperref[def:first_order_congruence]{first-order congruences} for rings are well-behaved. \Fullref{thm:ideals_and_congruences} shows that congruences correspond exactly to \hyperref[def:semiring_ideal]{two-sided ideals}.
  \end{thmenum}
\end{definition}

\begin{proposition}\label{thm:def:ring}
  \hyperref[def:ring]{Rings} have the following basic properties:
  \begin{thmenum}
    \thmitem{thm:def:ring/minus_one} For any ring element \( x \) we have \( (-1) \cdot x = -x \).

    \thmitem{thm:def:ring/multiplication_even_sign} For any ring elements \( x \) and \( y \) we have
    \begin{equation}\label{eq:thm:def:ring/multiplication_even_sign}
      (-x) \cdot (-y) = x \cdot y.
    \end{equation}

    \thmitem{thm:def:ring/multiplication_odd_sign} For any ring elements \( x \) and \( y \) we have
    \begin{equation}\label{eq:thm:def:ring/multiplication_odd_sign}
      (-x) \cdot y = x \cdot (-y) = -(x \cdot y).
    \end{equation}

    \thmitem{thm:def:ring/zero_absorbing} The zero element of a ring is absorbing, that is, it satisfies \eqref{eq:def:semiring/absorption}.

    \thmitem{thm:def:ring/quotient_equality_via_difference} Given an \hyperref[def:semiring_ideal]{ideal} \( I \) in a \hyperref[def:ring]{ring} \( R \), we have \( x + I = y + I \) if and only if \( x - y \in I \).

    \thmitem{thm:def:ring/semiring_cancellative_iff_no_zero_divisors} An element of a \hyperref[def:semiring]{ring} is left (resp. right) \hyperref[def:binary_operation/cancellative]{cancellable} if and only if it is not a left (resp. right) \hyperref[def:divisibility/zero]{nontrivial zero divisor}. In other words, \( xy = xz \) implies \( y = z \) if and only if \( xa = 0 \) implies \( a = 0 \).

    \thmitem{thm:def:ring/ring_entire_iff_cancellative} In an \hyperref[def:entire_semiring]{entire} \hyperref[def:ring]{ring} \( R \), the set \( R \setminus \set{ 0_R } \) is a \hyperref[def:binary_operation/cancellative]{cancellative} \hyperref[def:monoid]{monoid} with respect to multiplication.

    \thmitem{thm:def:ring/simple_ring_homomorphism_is_injective} If \( R \) is a \hyperref[def:simple_object]{simple ring}, every \hyperref[def:ring/homomorphism]{unital ring homomorphism} \( \varphi: R \to T \) is an injective function.
  \end{thmenum}
\end{proposition}
\begin{proof}
  \SubProofOf{thm:def:ring/minus_one} Distributivity implies
  \begin{equation*}
    0
    \reloset {\ref{eq:def:semiring/absorption}} =
    0 \cdot x
    =
    (1 - 1) \cdot x
    \reloset {\ref{eq:def:semiring/left_distributivity}} =
    x + (-1) \cdot x.
  \end{equation*}

  But \( -x \) is also an additive inverse of \( x \). Furthermore, \fullref{thm:monoid_inverse_unique} implies that the additive inverse is unique. Hence, \( -x = (-1) \cdot x \).

  \SubProofOf{thm:def:ring/multiplication_even_sign} First note that \( (-1) \cdot (-1) = 1 \) because
  \begin{equation*}
    (-1) \cdot (-1) - 1
    \reloset {\ref{eq:def:semiring/left_distributivity}} =
    (-1) \cdot (-1 + 1)
    =
    (-1) \cdot 0
    \reloset {\ref{eq:def:semiring/absorption}} =
    0.
  \end{equation*}

  We have
  \begin{equation*}
    (-x) \cdot (-y)
    (-x) \cdot (-y)
    \reloset {\ref{thm:def:ring/minus_one}} =
    (-1) \cdot x \cdot (-1) \cdot y
    =
    [(-1) \cdot (-1)] \cdot [x \cdot y].
    =
    x \cdot y.
  \end{equation*}

  \SubProofOf{thm:def:ring/multiplication_odd_sign} We have
  \begin{equation*}
    -(x \cdot y)
    \reloset {\ref{thm:def:ring/minus_one}} =
    (-1) \cdot (x \cdot y)
    =
    (-1 \cdot x) \cdot y
    \reloset {\ref{thm:def:ring/minus_one}} =
    (-x) \cdot y
    \reloset {\ref{thm:def:group/involution}} =
    (-x) \cdot [-(-y)]
    \reloset {\ref{thm:def:ring/multiplication_even_sign}} =
    x \cdot (-y).
  \end{equation*}

  \SubProofOf{thm:def:ring/zero_absorbing} We have
  \begin{equation*}
    x \cdot 0
    =
    x \cdot (1 - 1)
    \reloset {\ref{def:semiring/left_distributivity}} =
    x + (-1) \cdot x
    \reloset {\ref{thm:def:ring/minus_one}} =
    x - x
    =
    0.
  \end{equation*}

  \SubProofOf{thm:def:ring/quotient_equality_via_difference} Trivial.
  \SubProofOf{thm:def:ring/semiring_cancellative_iff_no_zero_divisors}
  \SufficiencySubProof* Suppose that \( xy = xz \) implies \( y = z \). Suppose also that \( xa = 0 \). Then \( xa = 0 = x0 \), implying that \( a = 0 \).

  \NecessitySubProof* Suppose that \( xa = 0 \) implies \( a = 0 \). Suppose also that \( xy = xz \). Then \( x(y - z) = 0 \), implying that \( y = z \).

  \SubProofOf{thm:def:ring/ring_entire_iff_cancellative} Follows from \fullref{thm:def:ring/semiring_cancellative_iff_no_zero_divisors}.

  \SubProofOf{thm:def:ring/simple_ring_homomorphism_is_injective} Let \( \varphi: R \to T \) be a unital ring homomorphism. Its kernel is an ideal of \( R \), and the only ideals are the trivial ring or \( R \) itself. Since \( \varphi(1_R) = 1_T \), the kernel cannot be \( R \), so it can only be the trivial ring. \Fullref{thm:group_homomorphism_zero_kernel} then implies that \( \varphi \) is injective.
\end{proof}

\paragraph{Integers modulo \( n \)}

\begin{definition}\label{def:ring_of_integers_modulo}\mimprovised
  For a positive integer \( n > 1 \), we extend the group \hyperref[def:group_of_integers_modulo]{\( \BbbZ_n \)} of integers modulo \( n \) with the operation
  \begin{equation*}
    x \odot y \coloneqq \rem(xy, n).
  \end{equation*}

  Thus, \( \BbbZ_n \) becomes a \hyperref[def:ring/commutative]{commutative ring}, which we call the \term{ring of integers modulo} \( n \).
\end{definition}
\begin{proof}
  Note that
  \begin{balign*}
    &\phantom{{}\cong{}} \rem(x, n) \rem(y, n)
    &\cong \pmod n \\ &\cong
    (x - n \quot(x, n)) (y - n \quot(y, n))
    &\cong \pmod n \\ &\cong
    xy - n \quot(x, n) - n \quot(y, n) + n^2 \quot(x, n) \quot(y, n)
    &\cong \pmod n \\ &\cong
    xy.
  \end{balign*}

  The proof that multiplication in \( \BbbZ_n \) is associative, unital and commutative becomes trivial.

  We will prove that multiplication distributes over addition. Fix \( x, y, z \in \BbbZ_n \). We have
  \begin{balign*}
    (x \oplus y) \odot z
     & =
    \rem((x \oplus y) z, n)
    =    \\ &=
    \rem(\rem(x + y, n) z, n)
    =    \\ &=
    \rem((x + y - n \quot(x + y, n)) z, n)
    =    \\ &=
    \rem((x + y)z, n).
  \end{balign*}
  and
  \begin{balign*}
    (x \odot z) \oplus (y \odot z)
     & =
    \rem([(x \odot z) + (y \odot z)], n)
    =    \\ &=
    \rem([xz - n \quot(xz, n) + yz - n \quot(yz, n)], n)
    =    \\ &=
    \rem(xz + yz, n)
    =    \\ &=
    \rem((x + y)z, n).
  \end{balign*}

  Hence,
  \begin{equation*}
    (x \oplus y) \odot z = (x \odot z) \oplus (y \odot z).
  \end{equation*}
\end{proof}

\paragraph{Ring quotients}

\begin{proposition}\label{thm:kernel_is_ideal}
  The \hyperref[def:ring/kernel]{kernel} of a \hyperref[def:ring/homomorphism]{ring homomorphism} is a \hyperref[def:semiring_ideal]{two-sided ideal} of its domain.
\end{proposition}
\begin{proof}
  Follows from \fullref{thm:kernel_is_normal_subgroup} and absorption.
\end{proof}

\begin{proposition}\label{thm:ideals_and_congruences}
  Fix a (unital) \hyperref[def:ring]{ring} \( R \). \hyperref[def:first_order_congruence]{First order congruences} on \( R \) and \hyperref[def:semiring_ideal]{ideals} of \( R \) are related as follows:
  \begin{thmenum}
    \thmitem{thm:ideals_and_congruences/cong_to_ideal} For every congruence in \( R \), the coset \( [0] \) of the zero element is a two-sided ideal of \( R \).

    \thmitem{thm:ideals_and_congruences/ideal_to_cong} Conversely, for every two-sided ideal \( I \), the relation \( x \cong y \) defined to hold if \( x - y \) is in \( I \), is a congruence.

    \thmitem{thm:ideals_and_congruences/inverse} The procedures for obtaining a congruence from a two-sided ideal and vice versa are inverses.
  \end{thmenum}
\end{proposition}
\begin{comments}
  \item This allows us to consider the quotient \( R / I \) with respect to an ideal rather than \( R / {\cong} \) with respect to a congruence.

  \item This is based on a similar statement for groups --- see \fullref{thm:normal_subgroups_and_congruences}. Another similar statement holds for submodules --- see \fullref{thm:submodules_and_congruences}
\end{comments}
\begin{proof}
  \SubProofOf{thm:ideals_and_congruences/cong_to_ideal} Since ring congruences are also congruences of the additive group, \fullref{thm:normal_subgroups_and_congruences/cong_to_subgroup} implies that \( [0] \) is a \hyperref[def:normal_subgroup]{normal subgroup} of the additive group of \( R \).

  \Fullref{thm:def:ring/zero_absorbing} implies that \( 0 \) is absorbing, thus \( [0] \) is indeed an ideal.

  \SubProofOf{thm:ideals_and_congruences/ideal_to_cong} Similarly, since every ideal is a normal subgroup of the additive group, it follows from \fullref{thm:normal_subgroups_and_congruences/subgroup_to_cong} that \( {\cong} \) defined as \( x \cong y \) if \( x - y \in I \) is a congruence on the additive group\footnote{\Fullref{thm:normal_subgroups_and_congruences/subgroup_to_cong} actually defines the congruence to hold \( (-x) + y \in I \), but the two conditions are equivalent}.

  To show that it is a ring congruence, we must show that \( a \cong b \) and \( x \cong y \) implies \( ax \cong by \). This holds because
  \begin{equation*}
    ax - by
    =
    (ax - ay) + (ay - by)
    =
    a(\underbrace{x - y}_{I}) + (\underbrace{a - b}_{I})y.
  \end{equation*}

  Since \( I \) is closed under multiplication with anything and addition with elements of \( I \), we conclude that \( ax - by \in I \) and hence \( ax \cong by \).

  \SubProofOf{thm:ideals_and_congruences/inverse} This is a special case of \fullref{thm:normal_subgroups_and_congruences/inverse}.
\end{proof}

\begin{theorem}[Lattice theorem for ideals]\label{thm:lattice_theorem_for_ideals}
  Given a \hyperref[def:semiring_ideal]{two-sided ideal} \( I \) of the \hyperref[def:ring]{ring} \( R \), the function \( J \mapsto J / I \) is an \hyperref[def:semilattice/homomorphism]{isomorphism of complete lattices} between the \hyperref[def:lattice_ideal/principal]{principal filter} of ideals of \( R \) containing \( I \) and the \hyperref[thm:semiring_of_ideals/lattice]{lattice of ideals} of the \hyperref[def:ring/quotient]{quotient} \( R / I \).

  Furthermore, \hyperref[def:semiring_ideal/prime]{prime}, \hyperref[def:semiring_ideal/maximal]{maximal} and \hyperref[def:radical_ideal]{radical} ideals of \( R \) containing \( I \) correspond to the same kind of ideals in \( R / I \).
\end{theorem}
\begin{comments}
  \item See \fullref{thm:lattice_theorem_for_substructures} for a very verbose formulation of this theorem in a general setting.
  \item This is based on a similar statement for \hyperref[def:group/submodel]{subgroups} --- see \fullref{thm:lattice_theorem_for_subgroups}. Another similar statement holds for \hyperref[def:module/submodel]{submodules} --- see \fullref{thm:lattice_theorem_for_submodules}.
\end{comments}
\begin{proof}
  Fix an ideal \( I \) in \( R \) and let \( x \cong y \) if \( x - y \) is in \( I \).

  \SubProof{Proof of lattice isomorphism} Note that every ideal of \( R \) is a subgroup of (the additive group of) \( R \) and that \( {\cong} \) is also a congruence on the additive group of \( R \). \Fullref{thm:lattice_theorem_for_subgroups} allows us to conclude that the ideal \( J \) of \( R \) contains \( I \) if and only if it is compatible with \( {\cong} \).

  The lattice isomorphism then follows from \fullref{thm:lattice_theorem_for_substructures}.

  \SubProof{Proof that prime ideals correspond to prime ideals}

  \SufficiencySubProof* Let \( P \) be a prime ideal of \( R \) compatible with \( {\cong} \), that is, containing \( I \).

  Fix members \( x \) and \( y \) of \( P \) such that \( [xy] = [x][y] \) is in \( P / {\cong} \). Then \( P \) contains \( xy \), hence \( x \) or \( y \) (or both) is in \( P \). Then \( [x] \) or \( [y] \) is in \( P / {\cong} \).

  It follows that the quotient \( P / {\cong} \) is prime.

  \NecessitySubProof* Let \( Q \) be a prime ideal of \( R / {\cong} \).

  If \( xy \in \bigcup Q \), then \( [x][y] = [xy] \in Q \). But \( Q \) is prime, hence \( [x] \) or \( [y] \) is in \( Q \). Then \( x \) or \( y \) is in \( \bigcup Q \).

  It follows that the union \( \bigcup Q \) is prime.

  \SubProof{Proof that maximal ideals correspond to maximal ideals} Straightforward corollary of the isomorphism of lattices.

  \SubProof{Proof that radical ideals correspond to radical ideals}

  \SufficiencySubProof* Let \( J \) be a radical ideal of \( R \) compatible with \( {\cong} \).

  Suppose that, for some positive integer \( n \), the element \( [x]^n \) is in \( J / I \). Then, since \( [x]^n = [x^n] \), it follows that \( x^n \) is in \( J \), and since \( J \) is radical, it follows that \( x \) is in \( J \). Thus, \( [x] \) is in \( J / I \).

  It follows that the quotient \( J / {\cong} \) is radical.

  \NecessitySubProof* Let \( H \) be a radical ideal of \( R / {\cong} \).

  Then, if \( x^n \) is in \( \bigcup H \), it follows that \( [x^n] = [x]^n \) is in \( H \) and, since \( H \) is radical, that \( [x] \) is in \( H \). Thus, \( x \) is in \( \bigcup H \).

  It follows that the union \( \bigcup H \) is radical.
\end{proof}

\begin{corollary}\label{thm:quotient_by_maximal_ideal}
  The two-sided ideal \( I \) of the \hyperref[def:ring]{ring} \( R \) is \hyperref[def:semiring_ideal/maximal]{maximal} if and only if the \hyperref[def:ring/quotient]{quotient} \( R / M \) is a \hyperref[def:simple_object]{simple ring}.

  In particular, if \( R \) is commutative, \( R / M \) is a \hyperref[def:field]{field}.
\end{corollary}
\begin{comments}
  \item See \fullref{thm:quotient_by_prime_ideal} for the corresponding statement for \hyperref[def:semiring_ideal/prime]{prime ideals} in commutative rings.
\end{comments}
\begin{proof}
  Since \( M \) is maximal, only \( M \) and \( R \) are ideals of \( R \) containing \( M \). Therefore, by \fullref{thm:lattice_theorem_for_ideals}, \( R / M \) has only two ideals. The converse also follows from the lattice theorem.
\end{proof}

\begin{proposition}\label{thm:ring_zero_morphisms}
  Consider the \hyperref[def:ring/category]{category of nonunital rings} and let \( \varphi: R \to T \) be a \hyperref[def:ring/homomorphism]{ring homomorphism}.

  \begin{thmenum}
    \thmitem{thm:ring_zero_morphisms/kernel} The \hyperref[def:zero_morphisms/kernel]{categorical kernel} \( \ker \varphi \) of \( \varphi \) is the preimage of \( 0_T \).

    \thmitem{thm:ring_zero_morphisms/cokernel} The \hyperref[def:zero_morphisms/kernel]{categorical cokernel} \( \co\ker \varphi \) of \( \varphi \) is the \hyperref[def:group/quotient]{quotient} of \( T \) by the \hyperref[def:semiring_ideal/generated]{semiring generated} by the \hyperref[def:set_valued_map/image]{set-theoretic image} \( \varphi[R] \).

    \thmitem{thm:ring_zero_morphisms/image} The \hyperref[def:zero_morphisms/image]{categorical image} \( \img \varphi \) of \( \varphi \) is the \hyperref[def:set_valued_map/image]{set-theoretic image} \( \varphi[R] \).

    \thmitem{thm:ring_zero_morphisms/coimage} The \hyperref[def:zero_morphisms/coimage]{categorical coimage} of \( \varphi \) is the \hyperref[def:ring/quotient]{quotient} of \( R \) by the \hyperref[def:ring/kernel]{kernel} \( \ker \varphi \).

    \thmitem{thm:ring_zero_morphisms/isomorphism} The image and coimage of \( \varphi \) are isomorphic via the following map:
    \begin{equation}\label{eq:thm:ring_zero_morphisms/isomorphism}
      \begin{aligned}
        &\psi: R / \ker \varphi \to \img \varphi, \\
        &\psi(\pi(x)) \coloneqq \varphi(x).
      \end{aligned}
    \end{equation}
  \end{thmenum}
\end{proposition}
\begin{comments}
  \item This is based on a similar statement for \hyperref[def:group]{groups} --- see \fullref{thm:group_zero_morphisms}. Another similar statement holds for \hyperref[def:module]{modules} --- see \fullref{thm:module_zero_morphisms}.
\end{comments}
\begin{proof}
  \SubProofOf{thm:ring_zero_morphisms/kernel} Follows from \fullref{thm:zero_morphisms_pointed/kernel}.
  \SubProofOf{thm:ring_zero_morphisms/cokernel} Analogously to \fullref{thm:group_zero_morphisms/cokernel}, the statement follows from \fullref{thm:zero_morphisms_pointed/cokernel} by noting that \( \braket{ \varphi[A] } \) is the smallest ideal whose congruence \( \cong \) satisfies \( y \cong 0_T \) if and only if \( y \in \varphi[R] \).

  \SubProofOf{thm:ring_zero_morphisms/image} Follows from \fullref{thm:zero_morphisms_pointed/image}.
  \SubProofOf{thm:ring_zero_morphisms/coimage} Analogously to \fullref{thm:group_zero_morphisms/coimage}, we conclude that the coimage, defined as the cokernel of the kernel of \( \varphi \), is the quotient of \( R \) by \( \ker \varphi \), the latter being a two-sided ideal as a consequence of \fullref{thm:kernel_is_ideal}.

  \SubProofOf{thm:ring_zero_morphisms/isomorphism} \Fullref{thm:group_zero_morphisms/isomorphism} implies that \( \psi \) as defined in \eqref{eq:thm:ring_zero_morphisms/isomorphism} is an isomorphism between the the additive groups of \( R / \ker \varphi \) and \( \img \varphi \). Furthermore, \( \psi \) is a ring isomorphism because \( \varphi \) is.
\end{proof}

\paragraph{Ring characteristic}

\begin{proposition}\label{thm:ring_characteristic_homomorphism}
  Similarly to how \( \BbbN \) is an \hyperref[def:universal_objects/initial]{initial object} in the category \hyperref[def:semiring/category]{\( \cat{SRing} \)} of semirings, \( \BbbZ \) is an initial object in the category \hyperref[def:ring/category]{\( \cat{Ring} \)} of unital rings.
\end{proposition}
\begin{proof}
  Follows from \fullref{thm:semiring_characteristic_homomorphism} by noting that \( (-n)x = -nx \).
\end{proof}

\begin{definition}\label{def:ring_characteristic}\mimprovised
  We define the \term{characteristic} \( \op{char}(R) \) of a ring \( R \) via the following equivalent definitions:
  \begin{thmenum}
    \thmitem{def:ring_characteristic/embedding} \( \op{char}(R) \) is the unique nonnegative integer \( n \) for which \( \BbbZ_n \) can be embedded into \( R \).

    Via the homomorphism \( \iota \) from the integers defined via \eqref{eq:thm:semiring_characteristic_homomorphism}, we can write this condition as
    \begin{equation}\label{eq:def:ring_characteristic/embedding}
      \BbbZ_n \cong \BbbZ / \ker \iota,
    \end{equation}

    We use here that \( \BbbZ_0 \) is the \hyperref[def:ring/trivial]{ring with one element}.

    \thmitem{def:ring_characteristic/direct} \( \op{char}(R) \) is the \hyperref[def:group_order]{order} of \( 1 \) in the additive group of \( R \) and \( 0 \) if the order is infinite.

    Explicitly, \( \op{char}(R) \) is the smallest positive integer \( n \) such that \( n \cdot 1 = 0 \) and \( \op{char}(R) = 0 \) if such an integer does not exist.
  \end{thmenum}
\end{definition}
\begin{defproof}
  \EquivalenceSubProof{def:ring_characteristic/embedding}{def:ring_characteristic/direct} Let \( n \) be such that
  \begin{equation*}
    n \BbbZ = \ker\iota.
  \end{equation*}

  In particular, \( \iota(0) = \iota(n) \).

  If \( n = 0 \), \( \ker\iota \) is a trivial group and \( \iota \) is an embedding. Then there cannot exist a positive integer \( n \) such that
  \begin{equation*}
    n \cdot 1_R = 0_R.
  \end{equation*}

  Otherwise, \( n \) is the smallest positive integer such that
  \begin{equation*}
    n \cdot 1_R = 0 \cdot 1_R = 0_R.
  \end{equation*}
\end{defproof}

\begin{proposition}\label{thm:ring_embedding_preserves_characteristic}
  If \( \varphi: R \to S \) is a \hyperref[def:ring/homomorphism]{ring embedding}, then \( S \) inherits its \hyperref[def:ring_characteristic]{characteristics} from \( R \).
\end{proposition}
\begin{proof}
  First suppose that \( R \) has positive characteristic \( n \). Then \( n \cdot 1 = 0 \), which implies \( n \cdot \varphi(1) = \varphi(0) \), hence \( \op{char}(T) \leq n \). But \( \varphi \) is an embedding, hence if \( k \cdot 1 \neq 0 \), then
  \begin{equation*}
    k \cdot \varphi(1) \neq \varphi(0).
  \end{equation*}

  This implies that \( \op{char}(S) \geq \op{char}(R) \), which in turn shows that \( \op{char}(S) = \op{char}(R) \).

  If \( R \) has characteristic zero, then \( \iota: \BbbN \to R \) is an embedding and thus \( \varphi \bincirc \iota: \BbbN \to S \) is also an embedding. It is unique as shown in \fullref{thm:ring_characteristic_homomorphism}. Therefore, \( S \) also has characteristic zero.
\end{proof}

\begin{example}\label{ex:def:ring_characteristic}
  The following are examples of \hyperref[def:ring_characteristic]{ring characteristics}:
  \begin{thmenum}
    \thmitem{ex:def:ring_characteristic/natural_numbers} The \hyperref[def:integers]{integers} \( \BbbZ \) have characteristic \( \op{char}(\BbbZ) = 0 \) because \( \iota \) is an isomorphism. Consequently, by \fullref{thm:ring_embedding_preserves_characteristic}, any superring of \( \BbbZ \) has characteristic zero, most notably the fields \( \BbbQ \), \( \BbbR \) and \( \BbbC \).

    \thmitem{ex:def:ring_characteristic/integers_modulo} The ring \hyperref[def:ring_of_integers_modulo]{\( \BbbZ_n \)} of integers modulo \( n \) has characteristic \( \op{char}(\BbbZ_n) = n \) because of \fullref{thm:integers_modulo_isomorphic_to_quotient_group}.

    \thmitem{ex:def:ring_characteristic/polynomial_ring} An \hyperref[def:algebra_over_semiring]{algebra} \( M \) over a commutative unital ring \( R \) has the same characteristic as \( R \) because of the canonical embedding of \( R \) in \( M \). In particular, the \hyperref[def:polynomial_algebra]{polynomial ring} \( R[X] \) has the same characteristic as its ring.
  \end{thmenum}
\end{example}

\paragraph{Totally ordered rings}

\begin{proposition}\label{thm:ordered_ring_inversion}
  In any \hyperref[def:ordered_semiring]{ordered} \hyperref[def:ring]{ring}, the element \( x \) is positive if and only if \( -x \) is negative.
\end{proposition}
\begin{comments}
  \item This follows from \fullref{def:totally_ordered_ring_signum} in the special case of entire totally ordered rings because it states that \( \sgn(-x) = -\sgn(x) \).
\end{comments}
\begin{proof}
  If \( 0 < x \), then \( -x \leq x + (-x) = 0 \). If \( -x = 0 \), this would contradict \fullref{thm:def:group/inverse_identity}, hence \( -x < 0 \).

  The converse direction is analogous.
\end{proof}

\begin{proposition}\label{thm:ordered_ring_order_inversion}
  In any \hyperref[def:ordered_semiring]{ordered} \hyperref[def:ring]{ring}, we have \( x < y \) if and only if \( -y < -x \).
\end{proposition}
\begin{proof}
  If \( x < y \), \fullref{thm:def:ordered_semiring/strict_sum} implies that \( x - y < y - y = 0 \), and similarly \( x - y - x < 0 - x \), that is, \( -y < -x \).

  The converse direction follows by double negation.
\end{proof}

\begin{definition}\label{def:totally_ordered_ring_signum}\mimprovised
  In a \hyperref[def:totally_ordered_set]{totally} \hyperref[def:ordered_semiring]{ordered} \hyperref[def:ring]{ring} \( R \), we can define the \term{signum function}
  \begin{equation*}
    \begin{aligned}
      &\sgn: R \to \BbbZ \\
      &\sgn(x) \coloneqq \begin{cases}
        0,  &x = 0_R, \\
        1,  &x > 0_R, \\
        -1, &x < 0_R.
      \end{cases}
    \end{aligned}
  \end{equation*}

  We say that \( x \) has a \term{positive sign} if \( \sgn(x) = 1 \) or a \term{negative sign} if \( \sgn(x) = -1 \).
\end{definition}

\begin{proposition}\label{thm:def:totally_ordered_ring_signum}
  \hyperref[def:totally_ordered_ring_signum]{The signum function} on the totally ordered \hi{\hyperref[def:entire_semiring]{entire}} ring \( R \) is a \hyperref[def:group/homomorphism]{group homomorphism} from the multiplicative group of \( R \) to the group \( \set{ -1, 0, 1 } \) under integer multiplication.

  Explicitly, for any \( x \) and \( y \) we have \( \sgn(x \cdot y) = \sgn(x) \cdot \sgn(y) \).

  If the ring has zero divisors, we can only state that
  \begin{equation*}
    \abs{\sgn(x \cdot y)} \leq \abs{\sgn(x) \cdot \sgn(y)}.
  \end{equation*}
\end{proposition}
\begin{proof}
  \hfill
  \begin{itemize}
    \item If both \( x \) and \( y \) are positive, the proposition follows from \fullref{thm:def:ordered_semiring/positive_prod}.
    \item If \( x \) and \( y \) have different signs, the proposition follows from \fullref{thm:def:ordered_semiring/alternating_prod}.
    \item If \( x \) and \( y \) are both negative, then \( -x \) is positive, and the previous case implies that \( (-x)y \) is negative. \Fullref{thm:ordered_ring_inversion} then implies that \( xy \) is positive.
  \end{itemize}
\end{proof}

\begin{definition}\label{def:totally_ordered_ring_absolute_value}\mimprovised
  For any \hyperref[def:totally_ordered_set]{totally} \hyperref[def:ordered_semiring]{ordered} \hyperref[def:ring]{ring} \( R \), we can define an \term{absolute value} function as
  \begin{equation*}
    \begin{aligned}
      &\abs{\anon}: R \to R \\
      &\abs{x} \coloneqq \sgn(x) \cdot x.
    \end{aligned}
  \end{equation*}
\end{definition}

\begin{proposition}\label{thm:def:totally_ordered_ring_absolute_value}
  The \hyperref[def:totally_ordered_ring_absolute_value]{absolute value} function has the following basic properties:
  \begin{thmenum}
    \thmitem{thm:def:totally_ordered_ring_absolute_value/nonnegative_value} If \( x \) is nonnegative, we have \( \abs{x} = x \).
    \thmitem{thm:def:totally_ordered_ring_absolute_value/negative_value} If \( x \) is negative, we have \( \abs{x} = -x \).
    \thmitem{thm:def:totally_ordered_ring_absolute_value/nonnegative} For any value \( x \), the absolute value \( \abs{x} \) is nonnegative.
  \end{thmenum}
\end{proposition}
\begin{proof}
  \SubProofOf{thm:def:totally_ordered_ring_absolute_value/nonnegative_value} If \( x = 0_R \), then
  \begin{equation*}
    \abs{x} = \sgn(x) \cdot x = 0 \cdot 0_R = 0 = x.
  \end{equation*}

  If \( x > 0_R \), then
  \begin{equation*}
    \abs{x} = \sgn(x) \cdot x = 1 \cdot x = x.
  \end{equation*}

  \SubProofOf{thm:def:totally_ordered_ring_absolute_value/negative_value} If \( x < 0_R \), then
  \begin{equation*}
    \abs{x} = \sgn(x) \cdot x = -1 \cdot x = -x.
  \end{equation*}

  \SubProofOf{thm:def:totally_ordered_ring_absolute_value/nonnegative} It is clear that \( \abs{x} \) is nonnegative if \( x \) is. If \( x \) is negative, then \( \abs{x} = -x \) is positive by \fullref{thm:ordered_ring_inversion}.
\end{proof}

\paragraph{Grothendieck completion}

\begin{proposition}\label{thm:grothendieck_semiring_completion}\mcite[95]{Enderton1977Sets}
  The \hyperref[def:monoid_grothendieck_completion]{Grothendieck completion} \( \overline{R} \) of the additive monoid of a \hyperref[def:semiring]{semiring} \( R \) becomes a \hyperref[def:ring]{ring} with the operation
  \begin{equation*}
    [(a, b)] \odot [(c, d)] \coloneqq [(ac + bd, ad + bc)].
  \end{equation*}
\end{proposition}
\begin{comments}
  \item This definition is motivated in our proof of \fullref{thm:grothendieck_semiring_completion_universal_property}.
\end{comments}
\begin{proof}
  Multiplication on \( R \) does not depend on the representative of the equivalence class. Indeed, let \( (a, b) \sim (a', b') \) and \( (c, d) \sim (c', d') \). Then there exist \( u \) and \( v \) such that
  \begin{align*}
    a + b' + u &= a' + b + u, \\
    c + d' + v &= c' + d + v.
  \end{align*}

  Then
  \begin{align*}
    &\phantom{{}={}}
    \hi{ac} + b'c + uc + a'd + \hi{bd} + ud + a'c + \hi{a'd'} + a'v + \hi{b'c'} + b'd + b'v
    = \\ &=
    (a + b' + u)c + (a' + b + u)d + a'(c + d' + v) + b'(c' + d + v)
    = \\ &=
    (a' + b + u)c + (a + b' + u)d + a'(c' + d + v) + b'(c + d' + v)
    = \\ &=
    a'c + \hi{bc} + uc + \hi{ad} + b'd + ud + \hi{a'c'} + a'd + a'v + b'c + \hi{b'd'} + b'v.
  \end{align*}

  Therefore,
  \begin{equation*}
    (a \cdot c + b \cdot d, a \cdot d + b \cdot c) \sim (a' \cdot c' + b' \cdot d', a' \cdot d' + b' \cdot c').
  \end{equation*}

  Associativity and distributivity in \( \overline{R} \) are inherited from \( R \).
\end{proof}

\begin{theorem}[Grothendieck semiring completion universal property]\label{thm:grothendieck_semiring_completion_universal_property}
  The \hyperref[thm:grothendieck_semiring_completion]{Grothendieck completion} \( \overline{R} \) of a semiring \( R \) satisfies the following \hyperref[rem:universal_mapping_property]{universal mapping property}:
  \begin{displayquote}
    For every ring \( T \) and every semiring homomorphism \( \varphi: R \to T \), there exists a unique ring homomorphism \( \widetilde{\varphi}: \overline{R} \to T \) such that the following diagram commutes:
    \begin{equation}\label{eq:thm:grothendieck_semiring_completion_universal_property/diagram}
      \begin{aligned}
        \includegraphics[page=1]{output/thm__grothendieck_semiring_completion_universal_property}
      \end{aligned}
    \end{equation}
  \end{displayquote}
\end{theorem}
\begin{comments}
  \item Via \fullref{rem:universal_mapping_property}, \( \overline{\anon} \) becomes \hyperref[def:category_adjunction]{left adjoint} to the \hyperref[def:concrete_category]{forgetful functor}
  \begin{equation*}
    U: \cat{CRing} \to \cat{CSRing}.
  \end{equation*}
\end{comments}
\begin{proof}
  \Fullref{thm:grothendieck_monoid_completion_universal_property} suggests the definition
  \begin{equation*}
    \overline{\varphi}([(a, b)]) \coloneqq \varphi(a) - \varphi(b).
  \end{equation*}

  We must only show that \( \overline{\varphi} \) is a ring homomorphism. Clearly
  \begin{equation*}
    \overline{\varphi}([(1, 0)]) = \varphi(1) - \varphi(0),
  \end{equation*}
  which implies that \( \varphi \) preserves multiplicative identities. Also,
  \begin{balign*}
    \overline{\varphi}\parens[\Big]{ [(a, b)] \odot [(c, d)] }
    &=
    \overline{\varphi}\parens[\Big]{ [(a \cdot b + c \cdot d, a \cdot d + b \cdot c)] }
    = \\ &=
    \varphi(a \cdot b + c \cdot d) - \varphi(a \cdot d + b \cdot c)
    = \\ &=
    \varphi(c) \parens[\Big]{ \varphi(d) - \varphi(b) } - \varphi(a) \parens[\Big]{ \varphi(d) - \varphi(b) }
    = \\ &=
    \parens[\Big]{ \varphi(c) - \varphi(a) } \parens[\Big]{ \varphi(d) - \varphi(b) }
    = \\ &=
    \overline{\varphi}\parens[\Big]{ [(a, c)] } \overline{\varphi}\parens[\Big]{ [(b, d)] }.
  \end{balign*}
\end{proof}

\begin{proposition}\label{thm:def:grothendieck_semiring_completion}
  The \hyperref[thm:grothendieck_semiring_completion]{Grothendieck completion} \( \overline{R} \) of a semiring \( R \) satisfies the following basic properties:
  \begin{thmenum}
    \thmitem{thm:def:grothendieck_semiring_completion/commutative} If \( R \) is commutative, so is \( \overline{R} \).
    \thmitem{thm:def:grothendieck_semiring_completion/entire} If \( R \) is \hyperref[def:entire_semiring]{entire}, so is \( \overline{R} \).
  \end{thmenum}
\end{proposition}
\begin{proof}
  \SubProofOf{thm:def:grothendieck_semiring_completion/commutative} This is clear from the definition of multiplication.
  \SubProofOf{thm:def:grothendieck_semiring_completion/entire} Suppose that
  \begin{equation*}
    \underbrace{[(a, b)] \cdot [(c, d)]}_{[(ac + bd, ad + bc)]} = [(0, 0)]
  \end{equation*}

  Then there exists an element \( u \) in \( R \) such that
  \begin{equation*}
    (ac + bd) + 0 + u = 0 + (ad + bc) + u.
  \end{equation*}

  Suppose that \( d = c + e \). Then
  \begin{equation*}
    ac + b(c + e) = a(c + e) + bc
  \end{equation*}
  and
  \begin{equation*}
    (ac + bc) + be = (ac + bc) + ae.
  \end{equation*}

  Cancelling \( e \), we obtain that \( a = b \). But \( [(a, b)] = [(0, 0)] \).
\end{proof}

\paragraph{Ring abelianization}

\begin{definition}\label{def:ring_commutator}\mimprovised
  Let \( R \) be an arbitrary ring. We define the \term{commutator} of the elements \( x \) and \( y \) as
  \begin{equation*}
    [x, y] \coloneqq xy - yx.
  \end{equation*}

  We call the ideal \hyperref[def:semiring_ideal/generated]{generated} by all the commutators in \( R \) the \term{commutator ideal} and denote it by \( [R, R] \).
\end{definition}
\begin{comments}
  \item Compare this to group commutators from \fullref{def:group_commutator}.
\end{comments}

\begin{proposition}\label{thm:quotient_is_abelian_iff_ideal_contains_commutator}
  The \hyperref[def:ring/quotient]{quotient} \( R / I \) of a \hyperref[def:ring]{ring} is \hyperref[def:abelian_group]{abelian} if and only if \( I \) contains the \hyperref[def:ring_commutator]{commutator ideal} \( [R, R] \) of \( R \).
\end{proposition}
\begin{comments}
  \item Compare this to \fullref{thm:quotient_is_abelian_iff_subgroup_contains_commutator} for group commutators.
\end{comments}
\begin{proof}
  Note that the cosets \( [x] \) and \( [y] \) commute if and only if
  \begin{equation*}
    [x] [y] - [y] [x] = [xy - yx] = I,
  \end{equation*}
  which holds if and only if the commutator \( xy - yx \) belongs to \( I \).
\end{proof}

\begin{definition}\label{def:ring_abelianization}\mcite[example 2.1.3(b)]{Leinster2016Basic}
   We define the \term{abelianization} of a group \( G \) as its \hyperref[def:group/quotient]{quotient} \( G / [G, G] \) by its \hyperref[def:group_commutator]{commutator group} \( [G, G] \).
\end{definition}
\begin{comments}
  \item \Fullref{thm:quotient_is_abelian_iff_ideal_contains_commutator} implies that \( R / [R, R] \) is abelian, which justifies the name.
  \item Compare this to ring abelianization from \fullref{def:ring_abelianization}.
\end{comments}
\begin{defproof}
  \Fullref{thm:commutator_subgroup_is_normal} implies that \( [G, G] \) is normal, hence we are allowed to take quotients.
\end{defproof}

\begin{theorem}[Ring abelianization universal property]\label{thm:ring_abelianization_universal_property}
  The \hyperref[def:ring_abelianization]{abelianization} \( R / [R, R] \) of a ring \( R \) satisfies the following \hyperref[rem:universal_mapping_property]{universal mapping property}:
  \begin{displayquote}
    For every commutative ring \( T \) and every ring homomorphism \( \varphi: R \to T \), \( \varphi \) \hyperref[def:factors_through]{uniquely factors through} \( R / [R, R] \). More precisely, there exists a unique ring homomorphism \( \widetilde{\varphi}: R / [R, R] \to T \) such that the following diagram commutes:
    \begin{equation}\label{eq:thm:ring_abelianization_universal_property/diagram}
      \begin{aligned}
        \includegraphics[page=1]{output/thm__ring_abelianization_universal_property}
      \end{aligned}
    \end{equation}
  \end{displayquote}
\end{theorem}
\begin{comments}
  \item Via \fullref{rem:universal_mapping_property}, the abelianization functor becomes \hyperref[def:category_adjunction]{left adjoint} to the \hyperref[def:concrete_category]{forgetful functor}
  \begin{equation*}
    U: \cat{CRing} \to \cat{Ring}.
  \end{equation*}

  \item Compare this result to \fullref{thm:group_abelianization_universal_property} for group abelianization.
\end{comments}
\begin{proof}
  Let \( T \) be a commutative ring and let \( \varphi: R \to T \) be a ring homomorphism.

  We want \( \overline{\varphi}: R / [R, R] \to T \) to satisfy
  \begin{equation}\label{eq:thm:ring_abelianization_universal_property/homomorphism}
    \overline{\varphi}(\pi_R(x)) = \varphi(x).
  \end{equation}

  We will use \eqref{eq:thm:ring_abelianization_universal_property/homomorphism} as a definition, but we must verify that \( \overline{\varphi} \) is well-defined.

  Suppose that \( x \cong y \pmod {[R, R]} \). \Fullref{thm:def:ring/quotient_equality_via_difference} implies that \( x - y \in [R, R] \), hence there exist some \( a \) and \( b \) in \( R \) such that
  \begin{equation*}
    x - y = ab - ba.
  \end{equation*}

  Then, because of commutativity in \( T \), we have
  \begin{equation*}
    \varphi(x - y)
    =
    \varphi(ab - ba)
    =
    \varphi(a) \cdot \varphi(b) - \underbrace{\varphi(b) \cdot \varphi(a)}_{\varphi(a) \cdot \varphi(b)}
    =
    0_T.
  \end{equation*}

  Therefore, \( \varphi(x) = \varphi(y) \), and thus the value of \( \overline{\varphi} \) in \eqref{eq:thm:ring_abelianization_universal_property/homomorphism} does not depend on the representative from \( \pi_R(x) \).
\end{proof}

\paragraph{Ring localization}

\begin{definition}\label{def:multiplicative_set_in_ring}\mcite[428]{Knapp2016BasicAlgebra}
  We call the subset of the \hyperref[def:ring]{ring} \( R \) a \term[bg=мултипликативно затворено множество (\cite[23]{КоцевСидеров2016})]{multiplicative set} if it is closed under multiplication and, furthermore, if it contains \( 1_R \).
\end{definition}

\begin{proposition}\label{thm:complement_of_prime_ideal}
  The \hyperref[def:semiring_ideal]{ideal} \( P \) in the \hyperref[def:ring/commutative]{commutative ring} \( R \) is \hyperref[def:semiring_ideal/prime]{prime} if and only if \( R \setminus P \) is a \hyperref[def:multiplicative_set_in_ring]{multiplicative set}.
\end{proposition}
\begin{comments}
  \item Not all multiplicative sets are obtained as complements of prime ideals --- see \fullref{ex:def:ring_localization/powers_of_two}.
\end{comments}
\begin{proof}
  By \fullref{thm:def:semiring_ideal/ideal_containing_unit}, \( P \) is a proper ideal if and only if \( 1_R \in R \setminus P \).

  By \fullref{thm:def:semiring_ideal/prime_pointwise}, \( P \) is prime if and only if \( x, y \in R \setminus P \) implies \( xy \in R \setminus P \), that is, if \( R \setminus P \) is closed under multiplication.
\end{proof}

\begin{definition}\label{def:ring_localization}\mcite[428]{Knapp2016BasicAlgebra}
  Let \( R \) be a \hyperref[def:ring/commutative]{commutative ring} and let \( S \subseteq R \) be a \hyperref[def:multiplicative_set_in_ring]{multiplicative set}.

  Define the equivalence relation \( (r, s) \sim (r', s') \) on \( R \times S \) to hold if and only if there exists some \( u \in S \) such that \( u r s' = u r' s \).

  Consider the set
  \begin{equation*}
    S^{-1} R \coloneqq R \times S / \sim,
  \end{equation*}
  whose cosets we will denote by \( \ifrac r s \) rather than \( [(r, s)] \).

  Define on \( S^{-1} R \) the operations
  \begin{align*}
    \frac a b + \frac c d     &\coloneqq \frac {a d + b c} {b d}, \\
    \frac a b \cdot \frac c d &\coloneqq \frac {a c} {b d},
  \end{align*}
  and the canonical inclusion
  \begin{equation*}
    \begin{aligned}
      &\iota: R \to S^{-1} R \\
      &\iota(r) \coloneqq \frac r {1_R}.
    \end{aligned}
  \end{equation*}

  This ring is called the \term[bg=локализация (\cite[23]{КоцевСидеров2016})]{localization} of \( R \) with respect to \( A \); we denote it by \( S^{-1} R \). In case \( S \) is the \hyperref[thm:boolean_algebra_of_subsets/complement]{complement} of a \hyperref[def:semiring_ideal/prime]{prime ideal}, we may denote the localization by \( R_P \) (or \( R_p \) if \( P = \braket{ p } \)).

  The image under \( \iota \) of every element \( s \) of \( S \) is invertible in \( S^{-1} R \), and we call the inverse \( \ifrac {1_R} s \) the \term{reciprocal} of \( s \).
\end{definition}
\begin{comments}
  \item This construction is very similar to the \hyperref[def:monoid_grothendieck_completion]{Grothendieck completion} of a monoid or semiring, although with notable differences --- the set \( S \) may be a strict subset of \( R \), and addition in the Grothendieck completion is defined similarly to multiplication in the localization, while addition in the completion has no analogy.
\end{comments}
\begin{defproof}
  The proof that \( {\sim} \) is an equivalence relation is the same as in \fullref{def:monoid_grothendieck_completion}. The result is then a ring if the operations are well-defined.

  We will show that both operations are well-defined. Let \( u ab' = u a'b \), meaning that \( (a, b) \sim (a', b') \) and hence \( \ifrac a b = \ifrac {a'} {b'} \), and let \( v cd' = v c'd \).

  For addition, we have
  \begin{align*}
    u v (ad + bc) b' d'
    &=
    v dd' (u ab') + u bb' (v cd')
    = \\ &=
    v dd' (u a'b) + u bb' (v c'd)
    = \\ &=
    u v (a'd' + b'c') b d,
  \end{align*}
  hence \( (ad + bc, bd) \sim (a'd' + b'c', b'd') \).

  The proof for correctness of multiplication is the same as our proof of correctness of addition in \fullref{def:monoid_grothendieck_completion}.
\end{defproof}

\begin{theorem}[Ring localization universal property]\label{thm:ring_localization_universal_property}\mcite[431]{Knapp2016BasicAlgebra}
  The \hyperref[def:ring_localization]{localization} of \( R \) by \( S \) satisfies the following \hyperref[rem:universal_mapping_property]{universal mapping property}:
  \begin{displayquote}
    For every commutative ring \( T \) and every ring homomorphism \( \varphi: R \to T \) such that \( \varphi(s) \) is invertible in \( T \) for every \( s \in S \), \( \varphi \) \hyperref[def:factors_through]{uniquely factors through} \( S^{-1} R \). More precisely, there exists a unique ring homomorphism \( \widetilde{\varphi}: S^{-1} R \to T \) such that the following diagram commutes:
    \begin{equation}\label{eq:thm:ring_localization_universal_property/diagram}
      \begin{aligned}
        \includegraphics[page=1]{output/thm__ring_localization_universal_property}
      \end{aligned}
    \end{equation}
  \end{displayquote}
\end{theorem}
\begin{proof}
  The condition suggests the definition
  \begin{equation*}
    \widetilde{\varphi}\parens*{ \frac r s } \coloneqq \varphi(r) \varphi(s)^{-1}.
  \end{equation*}
\end{proof}

\begin{example}\label{ex:def:ring_localization}
  We list several examples of \hyperref[def:ring/commutative]{commutative ring} \hyperref[def:ring_localization]{localization}.

  \begin{thmenum}
    \thmitem{ex:def:ring_localization/zero} If \( S \) contains \( 0_R \), then \( S^{-1} R \) is the trivial ring.

    \thmitem{ex:def:ring_localization/powers_of_two} The localization \( S^{-1} \BbbZ \) by the set \( S \coloneqq \set{ 2^n \given n \geq 0 } \) is (a ring isomorphic to) the rational numbers with denominators that are powers of two. This is an example of a multiplicative set that is not the complement of a prime ideal.

    This ring is isomorphic to the ring \( \BbbZ[\ifrac 1 2] \) obtained by \hyperref[thm:adjoining_elements_to_semiring]{adjoining} the rational number \( \ifrac 1 2 \) to \( \BbbZ \).

    \thmitem{ex:def:ring_localization/prime_number} Let \( p \) be a \hyperref[def:prime_number]{prime number}. The localization \( S^{-1} \BbbZ \) by \( S \coloneqq \BbbZ \setminus \braket{ p } \) is (a ring isomorphic to) the rational numbers with denominators coprime to \( p \).

    For \( p = 2 \), this localization consists of rational numbers whose denominator is an odd number.
  \end{thmenum}
\end{example}

\begin{proposition}\label{thm:def:ring_localization}
  \hyperref[def:ring_localization]{Ring localizations} have the following basic properties:

  \begin{thmenum}
    \thmitem{thm:def:ring_localization/image_of_ideal}\mcite[432]{Knapp2016BasicAlgebra} Localization preserves \hyperref[def:semiring_ideal]{ideals}. More precisely, given a commutative ring \( R \), a multiplicative set \( S \) and an ideal \( I \), the set
    \begin{equation*}
      S^{-1} I \coloneqq \set*{ \frac r s \given* r \in I \T{and} s \in S }
    \end{equation*}
    is an ideal of the localization \( S^{-1} R \).

    \thmitem{thm:def:ring_localization/prime_ideals}\mcite[exer. 4.3]{КоцевСидеров2016} The map \( I \mapsto S^{-1} I \) is a \hyperref[def:order_homomorphism/isomorphism]{strict order isomorphism} between the set of \hyperref[def:ring/submodel]{prime ideals} of \( R \) not intersecting \( S \) and the set of all prime ideals of \( S^{-1} R \).

    \thmitem{thm:def:ring_localization/by_prime_ideal}\mcite[exer. 4.2a)]{КоцевСидеров2016} The localization \( R_P \) by a \hyperref[def:semiring_ideal/prime]{prime ideal} \( P \) has a unique maximal ideal \( S^{-1} P \) (here \( S \coloneqq R \setminus P \)).

    \thmitem{thm:def:ring_localization/injective_inclusion} The canonical inclusion \( \iota: R \to S^{-1} R \) is injective if and only if \( S \) contains no zero divisors.
  \end{thmenum}
\end{proposition}
\begin{proof}
  \SubProofOf{thm:def:ring_localization/image_of_ideal} Trivial since \( S \) is closed under multiplication.

  \SubProofOf{thm:def:ring_localization/prime_ideals} Let \( P \) be a prime ideal in \( R \) disjoint from \( S \). By \fullref{thm:def:ring_localization/image_of_ideal}, \( S^{-1} P \) is an ideal of \( S^{-1} R \). If the product \( \ifrac {ac} {bd} \) belong to \( S^{-1} P \), then \( ac \in P \) and \( bd \in S \). Since \( P \) is prime, \( a \in P \) or \( c \in P \). If \( a \in P \), then \( ba \in P \) and \( \ifrac a d = \ifrac {ba} {bd} \in S^{-1} P \); if \( c \in P \), we proceed analogously. Thus, \( S^{-1} P \) is a prime ideal, i.e. the image under \( I \mapsto S^{-1} I \) of a prime ideal is a prime ideal.

  \SubProofOf[def:function_invertibility/injective/equality]{injectivity} Let \( S^{-1} P = S^{-1} Q \) for prime ideals \( P \) and \( Q \) disjoint from \( S \). Suppose that \( P \setminus Q \) contains at least one element, say \( p \). Then \( \iota(p) = \ifrac p 1 \) belongs to both \( S^{-1} P \) and \( S^{-1} Q \); hence, \( Q \) contains an element \( q \) such that, for some \( s \in S \) and \( u \in S \),
  \begin{equation*}
    p \cdot s \cdot u = 1 \cdot q \cdot u.
  \end{equation*}

  Since \( Q \) is an ideal, \( qu \in Q \), and hence \( psu \in Q \). But neither \( p \), \( s \) nor \( u \) belong to \( Q \), which contradicts the assumption that \( Q \) is prime. Therefore, \( P \setminus Q \) is empty. Generalizing, we obtain that \( I \mapsto S^{-1} I \) is injective on prime ideals.

  \SubProofOf[def:function_invertibility/surjective/existence]{surjectivity} Fix a prime ideal \( T \) in \( S^{-1} R \) and let \( P \) be the set of numerators in \( T \), i.e. if \( \ifrac p s \in T \), then \( p \in P \). We will show that \( P \) is a prime ideal; clearly \( T = S^{-1} P \).

  Clearly \( 0_R \in P \). Let \( a, c \in P \). Then there exist \( b, d \in S \) such that \( \ifrac a b \) and \( \ifrac c d \) belong to \( T \). But \( T \) is closed under multiplication with members of \( R \), hence \( \ifrac a {1_R} = b (\ifrac a b) \) and \( \ifrac c {1_R} = d (\ifrac c d) \) also belong to \( T \). Then their sum \( \ifrac {a + c} 1 \) belongs to \( T \), and hence also to \( P \). Thus, \( P \) is closed under addition. We analogously obtain that it is closed under multiplication.

  We have shown that \( P \) is an ideal in \( R \). We must show that it is a prime ideal. Let \( ac \in P \). Then
  \begin{equation*}
    \frac a b \cdot \frac c d \in T
  \end{equation*}
  for some \( b, d \in S \). Hence, \( \ifrac a b \) or \( \ifrac c d \) belongs to \( T \), implying that \( a \in P \) or \( c \in P \).

  \SubProofOf[def:order_homomorphism]{monotonicity} Follows from \fullref{thm:order_embedding_is_strict}.

  \SubProofOf{thm:def:ring_localization/by_prime_ideal} In the localization \( R_P \) be a prime ideal, all members of \( P \) become invertible. Hence, a maximal ideal cannot contain members of \( P \). By \fullref{thm:def:ring_localization/image_of_ideal}, \( S^{-1} P \) is an ideal, therefore it must be the largest proper ideal.

  \SubProofOf{thm:def:ring_localization/injective_inclusion} Let \( sr = 0 \) for \( s \in S \). Then \( \iota(s) = \ifrac s {1_R} \) is invertible in \( S^{-1} R \) and hence
  \begin{equation*}
    \frac {0_R} {1_R}
    =
    \frac {sr} {1_R}
    =
    \frac {1_R} s \cdot \frac {sr} {1_R}
    =
    \frac r {1_R}.
  \end{equation*}

  Hence, \( \iota(r) = \iota(0_R) \).

  It follows that \( \iota \) is injective if and only if \( S \) contains no zero divisors.
\end{proof}

\paragraph{Fields}

\begin{definition}\label{def:division_ring}\mcite[144]{Knapp2016BasicAlgebra}
  If every nonzero element of a ring is \hyperref[def:divisibility/unit]{invertible}, we call it a \term{division ring}.
\end{definition}

\begin{proposition}\label{thm:division_ring_is_entire}
  Every \hyperref[def:ring/trivial]{nontrivial} \hyperref[def:division_ring]{division ring} is \hyperref[def:entire_semiring]{entire}.
\end{proposition}
\begin{proof}
  Let \( xy = 0 \). If \( x \) is nonzero, multiplying both sides by \( x^{-1} \), we obtain \( y = 0 \). Analogously, \( y \neq 0 \) implies that \( x = 0 \). In all cases, either \( x \) or \( y \) is necessarily zero.

  Therefore, the ring has no nontrivial zero divisors.
\end{proof}

\begin{definition}\label{def:field}\mimprovised
  We will call the \hyperref[def:ring/trivial]{nontrivial} \hyperref[def:ring]{ring} \( \BbbK \) a \term[bg=поле (\cite[4]{КоцевСидеров2016}), ru=поле (\cite[sec. 15.1]{Тыртышников2007})]{field} if any of the following equivalent conditions hold:
  \begin{thmenum}
    \thmitem{def:field/simple} \( \BbbK \) is \hyperref[def:ring/commutative]{commutative} and \hyperref[def:simple_object]{simple}.
    \thmitem{def:field/division_ring} \( \BbbK \) is a \hyperref[def:ring/commutative]{commutative} \hyperref[def:division_ring]{division ring}.
  \end{thmenum}

  Fields have the following metamathematical properties:
  \begin{thmenum}
    \thmitem{def:field/theory} We can construct a \hyperref[def:first_order_theory]{first-order theory} for fields by adding to the \hyperref[def:semiring/theory]{theory of rings} the axioms \( \neg (0 \doteq 1) \) and
    \begin{equation}\label{eq:def:field/theory/invertibility}
      (\xi \doteq 0) \vee \qexists \eta (\xi \cdot \eta \doteq 1).
    \end{equation}

    These axioms are not \hyperref[def:positive_formula]{positive formulas}, hence many structural theorems from \fullref{subsec:first_order_models} fail to hold for them.

    \thmitem{def:field/homomorphism}\mcite[453]{Knapp2016BasicAlgebra} A \hyperref[def:first_order_homomorphism]{first-order homomorphism} between fields is simply a \hyperref[def:ring/homomorphism]{unital ring homomorphism}.

    \thmitem{def:field/submodel} If \( \Bbbk \) and \( \BbbK \) are fields such that \( \Bbbk \subseteq \BbbK \), we say that \( \BbbK \) is a \term{field extension} of \( \Bbbk \) and that \( \Bbbk \) is a \term{subfield} of \( \BbbK \). In particular, if \( \BbbK = \Bbbk \), we say that the extension is trivial.

    \thmitem{def:field/category} We denote the category of \hyperref[def:large_and_small_sets]{\( \mscrU \)-small} fields by \( \ucat{Field} \). It is a full subcategory of \hyperref[def:ring/category]{\( \ucat{CRing} \)} with objects restricted to fields.
  \end{thmenum}
\end{definition}
\begin{defproof}
  The equivalence of definitions follows from \fullref{thm:def:semiring_ideal/units}.
\end{defproof}
