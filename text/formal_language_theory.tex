\section{Formal language theory}\label{sec:formal_language_theory}

Our purpose here is to study artificial languages, including ways to generate and recognize them. Chomsky's hierarchy, defined in \fullref{def:chomsky_hierarchy}, is a fundamental classification of languages by which \hyperref[def:formal_grammar]{formal grammars} can be used to generate them. We dedicate \fullref{subsec:syntax_trees} to studying \hyperref[def:chomsky_hierarchy/context_free]{context-free languages} via \hyperref[def:parse_tree]{parse trees} and \fullref{subsec:regular_languages} to studying \hyperref[def:chomsky_hierarchy/regular]{regular languages} via \hyperref[def:finite_automaton]{finite automata}.

\begin{concept}\label{con:metalanguage}
  We dedicate this entire section to studying rigidly structured languages, which we will call our \term[ru=предметный язык (\cite[35]{Герасимов2011Вычислимость}), en=object language (\cite[3]{Kleene2002Logic})]{object languages}. The monograph itself is written in a language with looser rules, which we call our \term[ru=метаязык (\cite[35]{Герасимов2011Вычислимость}), en=metalanguage (\cite[3]{Kleene2002Logic})]{metalanguage}. This distinction is important and leads to conventions like \fullref{rem:object_language_dots} that allow us to more easily disambiguate between the object language and metalanguage.

  In relation to logic, \fullref{con:metalogic} introduces more related notions like object logic and object theories, as well as their metalingual counterparts. \Fullref{fig:con:metalogic} allows us to hierarchically visualize these concepts.
\end{concept}

\begin{concept}\label{con:syntax_and_semantics}
  As long as the object language allows specifying numbers in \hyperref[def:positional_number_system/decimal]{decimal notation}, in the \hyperref[con:metalanguage]{metalanguage} we distinguish between the following:
  \begin{itemize}
    \item The numeral \enquote{\( 1 \)} as a single-symbol string in the object language with no inherent meaning. In many programming languages this corresponds to the three-symbol \enquote{string literal} expression \texttt{"1"}.

    \item The value \( 1 \) is a metalingual object, which we can formally define as the set \( \set{ \varnothing } \) (see \fullref{thm:omega_is_model_of_pa} for a broader discussion). In many programming languages this corresponds to the single-symbol \enquote{number literal} expression \texttt{1}.
  \end{itemize}

  Every numeral can be interpreted \enquote{within the metalanguage} as the corresponding numeric value in decimal notation, and every numeric value can be expressed as a decimal string within the object language. Distinguishing between the two in the metalanguage leads to the dot conventions from \fullref{con:syntax_and_semantics}.

  Of course, more complicated expressions in the object language often have intermediate form likes \hyperref[con:abstract_syntax_tree]{abstract syntax trees}. We call these strings and intermediate forms the \term[en=syntax (\cite[8]{Hinman2005Logic})]{syntax} of the object language. We call the systematic assignment of values to these objects the \term[ru=семантика (\cite[54]{КолмогоровДрагалин2006Логика}), en=semantics (\cite[8]{Hinman2005Logic})]{semantics} of the language, and we call the assignment itself \term[ru=интерпретация (\cite[17]{Герасимов2011Вычислимость}), en=interpretation (\cite[10]{Smullyan1995FOL})]{interpretation} or \term[ru=оценка (\cite[77]{ШеньВерещагин2017Логика})]{evaluation} (see \fullref{con:evaluation} regarding the latter). These concepts are thoroughly studied in \fullref{sec:mathematical_logic}.
\end{concept}

\begin{remark}\label{rem:object_language_dots}
  The object language is part of the metalanguage, hence we may expect a clash of notation. Numeric strings and numbers were briefly discussed in \fullref{con:syntax_and_semantics}, but the distinctions are often more subtle. For example, in \fullref{def:propositional_valuation}, the symbol \( \wedge \) refers both to a logical connective and to a metalogical operation. Furthermore, their role is reversed in \fullref{def:lattice/theory}, where we study the first-order theory of lattices.

  For this reason, we introduce the following conventions:
  \begin{thmenum}
    \thmitem{rem:object_language_dots/terminals} Whenever the object language features some kind of alphabetic symbols, such as the variable identifiers as defined in \fullref{def:variable_identifier}, these symbols may coincide with our metalingual variable identifiers. In the case of such an ambiguity, we put a dot on top of the symbols in the object language.

    For example, \enquote{\( \dot u \dot v \)} is a definite two-symbol string in the object language, while \enquote{\( uv \)} refers to an expression where both are metalingual variables whose values are unspecified. This makes it theoretically possible to feature both dotted and un-dotted variables within a single string, but we will not find this useful.

    This becomes helpful because, though the entire monograph, we use the same letters for metalingual variables, and this allows highlighting variables in the object language --- see, for example, \fullref{def:semiring/theory} or \fullref{def:binary_relation}.

    \thmitem{rem:object_language_dots/connectives} We also put dots over all the propositional connectives from \fullref{def:propositional_alphabet}, as well as the formal equality from first-order quantifiers from \fullref{def:first_order_language}. This is motivated by the examples from the beginning of this remark.

    \thmitem{rem:object_language_dots/ambiguity} In other cases like \fullref{ex:natural_number_arithmetic_grammar/evaluation}, where ambiguity is possible, we also place dots over symbols in the object language.
  \end{thmenum}

  \incite[rem. 2.1.3]{Hinman2005Logic} follows similar conventions, but uses the dots over function and predicate symbols mostly and avoids placing them over variables and connectives. We chose different conventions because most of the monograph is not concerned with syntax.
\end{remark}
