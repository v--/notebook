\chapter{Formal language theory}\label{ch:formal_language_theory}

Our purpose here is to study artificial languages, including ways to generate and recognize them. Chomsky's hierarchy, defined in \cref{def:chomsky_hierarchy}, is a fundamental classification of languages by which \hyperref[def:formal_grammar]{formal grammars} can be used to generate them. We dedicate \fullref{sec:syntax_trees} to studying \hyperref[def:chomsky_hierarchy/context_free]{context-free languages} via \hyperref[def:parse_tree]{parse trees} and \fullref{sec:regular_languages} to studying \hyperref[def:chomsky_hierarchy/regular]{regular languages} via \hyperref[def:finite_automaton]{finite automata}.

\begin{concept}\label{con:metalanguage}
  We dedicate this entire chapter to studying rigidly structured languages, which we will call our \term[ru=предметный язык (\cite[35]{Герасимов2011Вычислимость}), en=object language (\cite[3]{Kleene2002Logic})]{object languages}. The monograph itself is written in a language with looser rules, which we call our \term[ru=метаязык (\cite[35]{Герасимов2011Вычислимость}), en=metalanguage (\cite[3]{Kleene2002Logic})]{metalanguage}. This distinction is important and leads to conventions like those in \cref{rem:object_language_dots} that allow us to more easily disambiguate between the object language and metalanguage.

  In relation to logic, \cref{con:metalogic} introduces more related notions like object logic and object theories, as well as their metalingual counterparts. \Cref{fig:con:metalogic} allows us to hierarchically visualize these concepts.
\end{concept}

\begin{concept}\label{con:syntax_semantics_duality}
  As long as the object language allows specifying numbers in \hyperref[def:positional_number_system/decimal]{decimal notation}, in the \hyperref[con:metalanguage]{metalanguage} we distinguish between the following:
  \begin{itemize}
    \item The numeral \enquote{\( 1 \)} as a single-symbol string in the object language with no inherent meaning. In many programming languages this corresponds to the three-symbol string literal expression \texttt{"1"}.

    \item The value \( 1 \) as a metalingual object, which we can formally define as the set \( \set{ \varnothing } \) (see \cref{thm:omega_is_model_of_pa} for a broader discussion). In many programming languages this corresponds to the single-symbol number literal expression \texttt{1}.
  \end{itemize}

  Every numeral can be interpreted \enquote{within the metalanguage} as the corresponding numeric value in decimal notation, and every numeric value can be expressed as a decimal string within the object language. Distinguishing between the two in the metalanguage leads to the dot conventions from \cref{rem:object_language_dots}.

  Of course, more complicated expressions in the object language often have intermediate forms like \hyperref[con:abstract_syntax_tree]{abstract syntax trees}. We call these strings and intermediate forms the \term[en=syntax (\cite[8]{Hinman2005Logic})]{syntax} of the object language. We call the systematic assignment of values to these objects the \term[ru=семантика (\cite[54]{КолмогоровДрагалин2006Логика}), en=semantics (\cite[8]{Hinman2005Logic})]{semantics} of the language, and we call the assignment itself \term[ru=интерпретация (\cite[17]{Герасимов2011Вычислимость}), en=interpretation (\cite[10]{Smullyan1995FOL})]{interpretation} or \term[ru=оценка (\cite[77]{ШеньВерещагин2017ЯзыкиИИсчисления})]{evaluation} (see \cref{con:evaluation} regarding the latter). These concepts are thoroughly studied in \fullref{ch:mathematical_logic}, as well as, more abstractly, in other places like \fullref{sec:free_groups}.

  We will refer to the interaction between syntactic objects and their semantic counterparts as the \term{syntax-semantics duality}.
\end{concept}

\begin{remark}\label{rem:object_language_dots}
  The object language is part of the metalanguage, hence we may expect a clash of notation.

  For instance, as discussed in \cref{con:syntax_semantics_duality}, when discussing \hyperref[def:positional_number_system/decimal]{decimal strings}, it is not clear whether \enquote{10} refers to a numeric string or to a number. Similarly, in \cref{def:propositional_valuation}, the symbol \( {\wedge} \) may refer both to a logical connective and to a metalogical operation.

  It is also not clear whether the metalingual variables in the \hyperref[def:first_order_syntax/atomic_formula]{atomic formula} \( (x \doteq y) \) are allowed to refer to the same symbol or not. Regarding this issue, \incite[78]{Kleene2002Logic} remarks
  \begin{displayquote}
    Our convention here is that distinct small Roman letters will be names for \textit{distinct} variables (or their equivalent) in the object language, \textit{except when} we say they need not be distinct.
  \end{displayquote}

  To surely eliminate ambiguity, we will rely on simple convention --- to place dots over (or occasionally inside) the symbols of the object language. Thus, we can be sure that \( \enquote{\syn1\syn0} \) refers to a digit string, while \( \enquote{10} \) refers to a number. Similarly, we can distinguish the connective symbol \( {\synwedge} \) from the operation \( {\wedge} \), while the metalingual variables \( x \) and \( y \) may both refer to the same object language variable \( \synx \).
\end{remark}
\begin{comments}
  \item This convention is nonstandard --- see \cref{rem:object_language_disambiguation_conventions}
\end{comments}

\begin{remark}\label{rem:object_language_disambiguation_conventions}
  The convention from \cref{rem:object_language_dots} we use of placing dots over symbols in the object language is not standard. Semantics are rarely discussed outside formal logic, and propositional connectives are mostly avoided in the metalanguage, so most of the ambiguity comes from the usage of \hyperref[con:variable]{variables} and \hyperref[con:logical_system_signature]{signature} symbols.

  Among the authors we cite, disambiguation conventions can be found in the following:
  \begin{itemize}
    \item \incite[rem. 2.1.3]{Hinman2005Logic} follows a similar convention to ours, but he places dots much more sparsely --- mostly over the formal equality symbol \( {\syneq} \) and the symbols in \hyperref[def:first_order_signature]{first-order signatures}.

    \item \incite*[\S 51]{Andrews2002Logic} relies on another convention --- to use bold letters for metalingual variables and normal-weight letters for variables in the object language.

    We find Andrews' convention more inconvenient since it heavily depends on the font.

    \item \incite[5]{Smullyan1995FOL} relies on yet another convention --- propositional variables in the object language are denoted by \( p_i \), where \( i = 1, 2, \ldots \), while metalingual variables that refer to them are denoted by single-symbol small Latin identifiers --- \( x \), \( y \), \( z \), even possibly \( p \).

    We find this convention very fragile, and with a very limited scope.
  \end{itemize}
\end{remark}
