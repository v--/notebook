\section{Complex polynomials}\label{sec:complex_polynomials}

\begin{proposition}\label{thm:topological_ring_polynomials_are_continuous}
\end{proposition}

\begin{proposition}\label{thm:complex_polynomials_dominated_by_power}
  Consider the complex polynomial
  \begin{equation*}
    f(z) = \sum_{k=0}^n a_k z^k.
  \end{equation*}

  Define
  \begin{equation*}
    r \coloneqq \max\set[\Big]{ 1 + \abs{a_k} \given k = 0, \ldots, n }.
  \end{equation*}

  Then \( \abs{z^{n+1}} > \abs{f(z)} \) whenever \( \abs{z} > r \).
\end{proposition}
\begin{proof}
  We will use induction on \( n \).
  \begin{itemize}
    \item The base case where \( f(z) \) is a constant polynomial is straightforward --- \( f(z) = a_0 \) and \( r = 1 + \abs{a_0} \), so, whenever \( \abs{z} > r \), we have
    \begin{equation*}
      \abs{z} > 1 + \underbrace{\abs{a_0}}_{\abs{f(z)}} > \abs{f(z)}.
    \end{equation*}

    \item Suppose that the statement holds for \( n - 1 \).

    We have
    \begin{equation*}
      f(z) = \sum_{k=0}^{n-1} a_k z^k + a_n z^n.
    \end{equation*}

    The inductive hypothesis implies that, whenever \( \abs{z} > r \), we have
    \begin{equation*}
      \abs{z^n} > \abs{f(z) - a_n z^n}.
    \end{equation*}

    By subadditivity of the absolute value, we have
    \begin{equation*}
      \abs{f(z)} \leq \abs{f(z) - a_n z^n} + \abs{a_n z^n},
    \end{equation*}
    thus
    \begin{equation*}
      (1 + \abs{a_n}) \abs{z^n} > \abs{f(z) - a_n z^n} + \abs{a_n z^n} \geq \abs{f(z)}.
    \end{equation*}

    By definition of \( r \), we have
    \begin{equation*}
      \abs{z} > r \geq 1 + \abs{a_n},
    \end{equation*}
    thus
    \begin{equation*}
      \abs{z^{n+1}} > (1 + \abs{a_n}) \abs{z^n} > \abs{f(z)}.
    \end{equation*}
  \end{itemize}
\end{proof}

\begin{lemma}\label{thm:rational_function_field_ordering_lemma}
  Consider the \hyperref[def:integers]{integer} \hyperref[def:rational_function_field]{rational function}
  \begin{equation}\label{ex:thm:rational_function_field_ordering_lemma/d}
    d(X) = \frac {\sum_{k=0}^n a_k X^k} {\sum_{k=0}^m b_k X^k}.
  \end{equation}

  Then \( {a_n} / {b_m} \geq 0 \) if and only if there exists a positive integer \( s \) such that \( d(x) \geq 0 \) whenever \( x \geq s \).
\end{lemma}
\begin{proof}
  If \( a_n \) is zero, then \( d(X) \) is the zero rational function and thus \( d(x) = 0 \) for every \( x \).

  Otherwise, \( a_n \) is nonzero. The denominator is a nonzero polynomial and hence \( b_m \) is a nonzero integer; we can thus rewrite the expression \eqref{ex:thm:rational_function_field_ordering_lemma/d} as
  \begin{equation*}
    d(X) = \frac {a_n} {b_m} \cdot \frac {\sum_{k=0}^n (a_k / a_n) X^k} {\sum_{k=0}^m (b_k / b_m) X^k}.
  \end{equation*}

  Let \( p \) be the smallest integer that exceeds \( 1 + \abs{a_k} \) for \( k = 0, \ldots, n \). \Fullref{thm:complex_polynomials_dominated_by_power} implies that, for every \( x \geq p \), we have
  \begin{equation*}
    x^n + \abs{a_0} > \abs*{\sum_{k=1}^{n-1} \frac {a_k} {a_n} x^k},
  \end{equation*}
  hence
  \begin{equation*}
    x^n + \sum_{k=1}^{n-1} \frac {a_k} {a_n} x^k > 0.
  \end{equation*}

  Similarly, there exists a positive integer \( q \) such that, for every \( x \geq q \),
  \begin{equation*}
    x^m + \sum_{k=1}^{m-1} \frac {b_k} {b_m} x^k > 0.
  \end{equation*}

  Let \( s \coloneqq \max\set{ p, q } \). When \( x \geq s \), the sign of
  \begin{equation*}
    d(x) = \frac {a_n} {b_m} \cdot \frac {\sum_{k=0}^n (a_k / a_n) x^k} {\sum_{k=0}^m (b_k / b_m) x^k}
  \end{equation*}
  matches the sign of \( a_n / b_m \).
\end{proof}

\begin{example}\label{ex:rational_function_field_non_archimedean}\mcite[193]{Mendelson2008NumberSystems}
  The \hyperref[def:rational_function_field]{field of rational functions} \( \BbbZ(X) \) can be given a non-\hyperref[def:archimedean_semiring]{Archimedean} \hyperref[def:totally_ordered_set]{total order} --- define \( f(X) \sqsupseteq g(X) \) for two rational functions if their difference \( f(X) - g(X) \) satisfies the equivalent conditions from \fullref{thm:rational_function_field_ordering_lemma}.

  We must first prove that this is a total order in \( \BbbZ(X) \).
  \begin{itemize}
    \item Reflexivity of \( {\sqsubseteq} \) holds vacuously.

    \item For antisymmetry, suppose that \( f(X) \sqsupseteq g(X) \) and \( f(X) \sqsubseteq g(X) \). Denote their difference by \( d(X) \) with coefficients as in \eqref{ex:thm:rational_function_field_ordering_lemma/d}.

    Both \( d(X) \) and \( -d(X) \) must be nonnegative, thus \( a_n / b_m \geq 0 \) and \( -(a_n / b_m) \geq 0 \). \Fullref{thm:def:ordered_semiring_positivity/alternating_prod} implies that \( a_n / b_m \leq 0 \). Antisymmetry on the rational numbers then implies that \( a_n / b_m = 0 \). Multiplying by \( b_m \), we obtain \( a_n = 0 \). Since \( a_n \) is the leading coefficient of the numerator, it follows that it is the zero polynomial, and thus \( d(X) \) is the zero rational function.

    Therefore, antisymmetry holds.

    \item For transitivity, suppose that \( f(X) \sqsupseteq g(X) \) and \( g(X) \sqsupseteq h(X) \). Then there exist positive integers \( p > 0 \) and \( q > 0 \) such that \( f(x) - g(x) \geq 0 \) when \( x \geq p \) and \( g(x) - h(x) \geq 0 \) when \( x \geq q \).

    Let \( s \coloneqq \max\set{ p, q } \). Then, if \( x \geq s \), we have
    \begin{equation*}
      f(x) - h(x) = \parens[\Big]{ f(x) - g(x) } + \parens[\Big]{ g(x) - h(x) } \geq 0 + 0 = 0.
    \end{equation*}

    Therefore, \( f(X) \sqsupseteq h(X) \).

    \item To show that \( {\sqsubseteq} \) is \hyperref[def:binary_relation/connected]{connected}, note that either \( a_n / b_n \geq 0 \) or \( -(a_n / b_n) \geq 0 \). Thus, we rely on the connectedness of the rational number ordering to show that the ordering on \( \BbbZ(X) \) is connected.
  \end{itemize}

  Compatibility with the field operations is directly inherited from the compatibility in the rational numbers. We have thus shown that \( \BbbZ(X) \) can be totally ordered.

  It remains to demonstrate the gist of the example --- that the order is not Archimedean. Consider the polynomials \( 1 \) and \( X \). The Archimedean property requires there to exist a positive integer \( s \) such that \( X \sqsubseteq s \cdot 1 = s \). But \( X \sqsubseteq s \) because their difference \( X - s \) has a positive leading coefficient. Therefore, the order is not Archimedean.
\end{example}
