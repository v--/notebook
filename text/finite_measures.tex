\section{Finite measures}\label{sec:finite_measures}

\paragraph{Finite measures}

\begin{definition}\label{def:finite_measure}\mcite[31]{Halmos1976MeasureTheory}
  Fix a \hyperref[def:measure_space/positive]{positive measure space} \( (\Omega, \mscrA, \mu) \). We say that \( \mu \) is \term{finite measure} if \( \mu(\Omega) \) is finite.

  Due to \fullref{thm:def:boolean_algebra_additive_function/nonnegative_order}, we can regard \( \mu \) as a function from \( X \) to \( [0, \mu(\Omega)] \).
\end{definition}

\paragraph{\( \sigma \)-finite measures}

\begin{definition}\label{def:exhausting_sequence}\mimprovised
  We say that an increasing sequence
  \begin{equation*}
    A_1 \subseteq A_2 \subseteq \cdots
  \end{equation*}
  \term[ru=монотонно исчерпывает (\cite[def. 3.8.1]{ИльинСадовничийСендов1987АнализТом2}), en=exhaustion sequence (\cite[13]{Malliavin1995Probability})]{exhausts} \( B \) if
  \begin{equation*}
    B = \bigcup_{k=1}^\infty A_k.
  \end{equation*}
\end{definition}
\begin{comments}
  \item We generalize our definition from the following:
  \begin{itemize}
    \item \incite[def. 3.8.1]{ИльинСадовничийСендов1987АнализТом2} provide a definition for an increasing sequence of open sets \enquote{monotonically exhausting} (\enquote{монотонно исчерпывающий}) some set \( D \) when their union equals \( D \).

    \item \incite[13]{Malliavin1995Probability} defines \enquote{exhaustion sequences} as increasing sequences of measurable sets whose union coincides with the whole space.
  \end{itemize}
\end{comments}

\begin{definition}\label{def:sigma_finite_measure}\mcite[31]{Halmos1976MeasureTheory}
  Fix a \hyperref[def:measure_space/positive]{positive measure space} \( (\Omega, \mscrA, \mu) \). We say that \( \mu \) is \term[ru=\( \sigma \)-крайний (\cite[def. 1.6.1]{Богачёв2003ТеорияМерыТом1})]{\( \sigma \)-finite} if \( \Omega \) can be \hyperref[def:exhausting_sequence]{exhausted} by a sequence of sets of finite measure.
\end{definition}
