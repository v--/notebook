\subsection{Categorical limits}\label{subsec:categorical_limits}

\begin{definition}\label{def:comma_category}\mcite[def. 2.3.1]{Leinster2016Basic}
  Comma categories allow us to define morphisms between morphisms, which becomes useful in, for example, the concise definition of a limit in \fullref{def:category_of_cones/limit}.

  \begin{thmenum}
    \thmitem{def:comma_category/variable} We first prove the most general construction. Let \( F: \cat{A} \to \cat{C} \) and \( G: \cat{B} \to \cat{C} \) be any two functors with a common codomain. We define their \term{comma category} \( (F \downarrow G) \) as follows:

    \begin{itemize}
      \item The \hyperref[def:category/objects]{objects} are the triples \( (A, s, B) \), where \( A \in \cat{A} \), \( B \in \cat{B} \) and \( s: F(A) \to G(B) \).

      \item The \hyperref[def:category/morphisms]{morphisms} from \( (A, s, B) \) to \( (A', s', B')) \) are the pairs
      \begin{equation*}
        (f: A \to A', g: B \to B'),
      \end{equation*}
      such that the following diagram commutes:
      \begin{equation}\label{eq:def:comma_category/variable}
        \begin{aligned}
          \includegraphics[page=1]{output/def__comma_category}
        \end{aligned}
      \end{equation}

      \item The \hyperref[def:category/composition]{composition of morphisms} is their pairwise composition, analogically to \hyperref[def:product_category]{product categories}.

      \item The \hyperref[def:category/identity]{identity morphism} on the object \( (A, s, B) \) is the identity pair \( (\id_A, \id_B) \).
    \end{itemize}

    \thmitem{def:comma_category/fixed} It is often the case where either \( F \) or \( G \) are constant. If, instead of \( G \), we are given an object \( X \) in \( \cat{C} \), we use the constant functor \( \Delta_X^{\cat{1}}: \cat{1} \to \cat{C} \) with domain the \hyperref[def:universal_categories]{terminal category} \( \cat{1} \), in order to utilize the comma category \( (F \downarrow \Delta_X^{\cat{1}}) \).

    We can thus simplify \fullref{def:comma_category/variable} as follows:

    \begin{minipage}[t]{0.43\textwidth}
      \begin{equation*}
        (F \downarrow X) \coloneqq (F \downarrow \Delta_X^{\cat{1}}).
      \end{equation*}

      \begin{itemize}
        \item The \hyperref[def:category/objects]{set of objects} \( \obj(F \downarrow X) \) is
        \begin{equation*}
          \set{ (A, s) \given s: F(A) \to X }
        \end{equation*}

        \item The \hyperref[def:category/morphisms]{hom-set}
        \begin{equation*}
          [F \downarrow X]\parens[\Big]{ (A, s), (A', s') }
        \end{equation*}
        is the set of \( f: A \to A' \), such that
        \begin{equation}\label{eq:def:comma_category/fixed/right}
          \begin{aligned}
            \includegraphics[page=2]{output/def__comma_category}
          \end{aligned}
        \end{equation}
      \end{itemize}
    \end{minipage}
    \begin{minipage}[t]{0.43\textwidth}
      \begin{equation*}
        (X \downarrow G) \coloneqq (\Delta_X^{\cat{1}} \downarrow G)
      \end{equation*}

      \begin{itemize}
        \item The \hyperref[def:category/objects]{set of objects} \( \obj(X \downarrow G) \) is
        \begin{equation*}
          \set{ (s, B) \given s: X \to F(B) }
        \end{equation*}

        \item The \hyperref[def:category/morphisms]{hom-set}
        \begin{equation*}
          [X \downarrow G]\parens[\Big]{ (s, B), (s', B') }
        \end{equation*}
        is the set of \( g: B \to B' \), such that
        \begin{equation}\label{eq:def:comma_category/fixed/left}
          \begin{aligned}
            \includegraphics[page=3]{output/def__comma_category}
          \end{aligned}
        \end{equation}
      \end{itemize}
    \end{minipage}
  \end{thmenum}
\end{definition}

\begin{definition}\label{def:factors_through}\mcite[133]{Knapp2016BasicAlgebra}
  Consider a morphism \( f: A \to B \). We say that \( f \) \term{factors through} the object \( X \) if there exist morphisms \( g: A \to X \) and \( h: X \to B \) such that the following diagram commutes:
  \begin{equation}\label{eq:def:factors_through}
    \begin{aligned}
      \includegraphics[page=1]{output/def__factors_through}
    \end{aligned}
  \end{equation}

  If, given \( g \), \( h \) is unique or vice versa, we say that \( f \) \term{uniquely factors through} \( X \). In some cases, for example in \fullref{def:zero_morphisms/morphism}, \( f \) there exists unique \( g \) and \( h \) and neither of them must be given beforehand.
\end{definition}

\begin{definition}\label{def:category_of_cones}
  Let \( D: \cat{I} \to \cat{C} \) be a \hyperref[def:categorical_diagram]{diagram}.

  \begin{thmenum}
    \thmitem{def:category_of_cones/category} We define \term{category of cones} to \( D \) as the \hyperref[def:comma_category/fixed]{constant-functor comma category}
    \begin{equation*}
      \cat{Cone}(D) \coloneqq \underbrace{ (\Delta^{\cat{I}} \downarrow D) }_{(\Delta^{\cat{I}} \downarrow \Delta^{\cat{1}}_{D})},
    \end{equation*}
    where \( \Delta^{\cat{I}}: \cat{C} \to [\cat{I}, \cat{C}] \) is the \( \cat{I} \)-shaped \hyperref[def:diagonal_functor]{diagonal functor} on \( \cat{C} \).

    \thmitem{def:category_of_cones/cone} A \( D \)-\term{cone} is simply a member of \( \cat{Cone}(D) \).

    Explicitly, a cone with \term{vertex} \( A \in \cat{C} \) is a pair \( (\Delta^{\cat{I}}_A, \alpha) \), where \( \Delta^{\cat{I}}_A \) is the constant functor at \( A \) and \( \alpha \) is a natural transformation from \( \Delta^{\cat{I}}_A \) to \( D \).

    Even more explicitly, a cone is a family of morphisms
    \begin{equation}\label{eq:def:category_of_cones/cone}
      \seq{ \alpha_k: A \to D(k) }_{k \in \cat{I}}.
    \end{equation}
    satisfying a simplified naturality condition (compared to \eqref{eq:def:natural_transformation/diagram}). For every morphism \( u: k \to m \) in \( \cat{I} \), the following diagram must commute:
    \begin{equation}\label{eq:def:category_of_cones/cone_nat}
      \begin{aligned}
        \includegraphics[page=1]{output/def__category_of_cones}
      \end{aligned}
    \end{equation}

    Note that this diagram is very different from \eqref{eq:def:comma_category/fixed/right}, they merely look similar.

    \thmitem{def:category_of_cones/limit} A \term{limit cone} of the diagram \( D \) is a \hyperref[def:universal_objects/terminal]{terminal object} of the cone category \( \cat{Cone}(D) \).

    Explicitly, \( (L, \lambda) \) is limit cone if, for every cone \( (A, \alpha) \), there exists a unique \hyperref[eq:def:comma_category/fixed/right]{cone morphism} \( l_A: (A, \alpha) \to (L, \lambda) \).

    Even more explicitly, \( (L, \lambda) \) is a limit cone if it satisfies the following \hyperref[rem:limit_universal_mapping_property]{universal mapping property}:
    \begin{displayquote}
      For every cone \( (A, \alpha) \), there exists a unique morphism \( l_A: A \to L \) such that following diagram commutes for every index morphism \( u: k \to m \):
      \begin{equation}\label{eq:def:category_of_cones/limit}
        \begin{aligned}
          \includegraphics[page=2]{output/def__category_of_cones}
        \end{aligned}
      \end{equation}
    \end{displayquote}

    Thus, \( \alpha_k \) \hyperref[def:factors_through]{uniquely factors through} \( L \) for every \( k \in \cat{I} \). Furthermore, \( l_A \) is compatible with morphisms in \( \cat{I} \) and does not depend on \( k \).

    From \fullref{thm:def:universal_objects/terminal} and \fullref{thm:universal_objects_as_adjunctions/terminal} it follows that a limit, if it exists, is unique up to a unique isomorphism.

    Without further context, we usually refer to \( L \) as the limit vertex and \( (L, \lambda) \) as the limit cone. By \enquote{the limit}, we usually mean the vertex \( L \).

    \thmitem{def:category_of_cones/cocone} \hyperref[thm:categorical_principle_of_duality]{Dually}, a \( D \)-\term{cocone} with vertex \( A \) is a family of morphisms
    \begin{equation}\label{eq:def:category_of_cones/cocone}
      \seq{ \alpha_k: D(k) \to A }_{k \in \cat{I}}.
    \end{equation}
    satisfying the naturality condition that for every morphism \( u: k \to m \) in \( \cat{I} \), the following diagram must commute:
    \begin{equation}\label{eq:def:category_of_cones/cocone_nat}
      \begin{aligned}
        \includegraphics[page=3]{output/def__category_of_cones}
      \end{aligned}
    \end{equation}

    \thmitem{def:category_of_cones/colimit} A \term{colimit cocone} of the diagram \( D \) is an \hyperref[def:universal_objects/initial]{initial object} of the cocone category \( (D \downarrow \Delta) \).

    There are two major differences compared to limits: a colimit cocone is an initial object, not a terminal object, and its underlying comma category is \( (D \downarrow \Delta) \), not \( (\Delta^{\cat{I}} \downarrow D) \).

    The analogous diagram to \eqref{eq:def:category_of_cones/limit} is exactly its \hyperref[thm:categorical_principle_of_duality]{opposite}:
    \begin{equation}\label{eq:def:category_of_cones/colimit}
      \begin{aligned}
        \includegraphics[page=4]{output/def__category_of_cones}
      \end{aligned}
    \end{equation}

    In particular, \( \alpha_k \) \hyperref[def:factors_through]{uniquely factors through} \( L \) for every \( k \in \cat{I} \).

    Without further context, we usually refer to \( L \) as the colimit vertex and \( (L, \lambda) \) as the colimit cocone. By \enquote{the colimit}, we usually mean the vertex \( L \).
  \end{thmenum}
\end{definition}

\begin{proposition}\label{thm:categorical_limit_duality}
  For every \hyperref[def:category_of_cones/cone]{cone} \( (A, \alpha) \) of the \hyperref[def:categorical_diagram]{diagram} \( D \) in \( \cat{C} \), \( (A, \alpha^\oppos) \) is a \hyperref[def:category_of_cones/cone]{cocone} of \( D^\oppos \) in the \hyperref[def:opposite_category]{opposite category} \( \cat{C}^\oppos \).

  Even more, for every \hyperref[def:category_of_cones/limit]{limit} \( (L, \lambda) \) of \( D \) in \( \cat{C} \), \( (L, \lambda^\oppos) \) is a \hyperref[def:category_of_cones/colimit]{colimit} of \( D^\oppos \) in \( \cat{C}^\oppos \).

  This is part of the duality principles listed in \fullref{thm:categorical_principle_of_duality}.
\end{proposition}
\begin{proof}
  Note that the defining diagrams \eqref{eq:def:category_of_cones/cone}, \eqref{eq:def:category_of_cones/cone_nat} and \eqref{eq:def:category_of_cones/limit} are dual to \eqref{eq:def:category_of_cones/cocone}, \eqref{eq:def:category_of_cones/cocone_nat} and \eqref{eq:def:category_of_cones/colimit}.
\end{proof}

\begin{lemma}\label{thm:categorical_limit_uniqueness_lemma}
  Any two limits (resp. colimits) of a diagram are isomorphic.

  We prove a stronger result in \fullref{thm:categorical_limit_uniqueness}.
\end{lemma}
\begin{proof}
  Let \( (L', \lambda') \) and \( (L^\dprime, \lambda^\dprime) \) be two limit cones over the diagram \( D: \cat{I} \to \cat{C} \). The definition \eqref{eq:def:category_of_cones/limit} of a limit implies that
  \begin{equation}\label{eq:thm:categorical_limit_uniqueness}
    \begin{aligned}
      \includegraphics[page=1]{output/thm__categorical_limit_uniqueness}
    \end{aligned}
  \end{equation}
  commutes.

  Therefore, the limits \( L' \) and \( L^\dprime \) are isomorphic.

  Now let \( (L', \lambda') \) and \( (L^\dprime, \lambda^\dprime) \) be colimit cocones. By \fullref{thm:categorical_limit_duality}, \( (L', \lambda'^\oppos) \) and \( (L^\dprime, {\lambda^\dprime}^\oppos) \) are limits in the opposite category and are thus isomorphic. By \fullref{thm:morphism_invertibility_duality}, the colimits are isomorphic.
\end{proof}

\begin{proposition}\label{thm:categorical_limit_is_adjoint}
  Suppose that, for a given category \( \cat{C} \), the limits over all \( \cat{I} \)-shaped diagrams exist. Denote by \( \lim(D) \) the vertex of the limiting cone of the diagram \( D \).

  Given a natural transformation \( \alpha: D \to E \) between diagrams, the diagram \eqref{eq:thm:categorical_limit_is_adjoint/c_triangle} defining a limit uniquely determines a morphism \( \lim(D) \) to \( \lim(E) \). Denote this morphism by \( \lim(\alpha) \).

  We have defined a functor
  \begin{equation*}
    \lim: [\cat{I}, \cat{C}] \to \cat{C}.
  \end{equation*}

  This functor is \hyperref[def:category_adjunction]{right adjoint} to the diagonal functor
  \begin{equation*}
    \Delta: \cat{C} \to [\cat{I}, \cat{C}]
  \end{equation*}

  \hyperref[thm:categorical_principle_of_duality]{Dually}, the colimit functor
  \begin{equation*}
    \co\lim: [\cat{I}, \cat{C}] \to \cat{C}
  \end{equation*}
  is left adjoint to \( \Delta \).
\end{proposition}
\begin{proof}
  It is sufficient to prove this for limits since the statement for colimits follows from the duality principles \fullref{thm:category_adjunction_duality} and \fullref{thm:categorical_limit_duality}. The unit \( \eta: \id_{\cat{C}} \to [{\lim} \bincirc \Delta] \) of the adjunction
  \begin{equation*}
    \Delta \dashv \lim
  \end{equation*}
  is the unique morphism from an object \( A \) of \( \cat{C} \) to the limit of its constant diagram \( \Delta_A: \cat{I} \to \cat{C} \) such that \eqref{eq:def:category_of_cones/limit} commutes. The counit is more complicated because its components are themselves natural transformations:
  \begin{equation*}
    \begin{aligned}
      \varepsilon:       &[\Delta \bincirc \lim] \Rightarrow \id_{[\cat{I}, \cat{C}]} \\
      \varepsilon_D:     &\Delta(\lim D) \Rightarrow D \\
      \varepsilon_{D,k}: &\lim(D) \to D(k),
    \end{aligned}
  \end{equation*}
  where \( \varepsilon_{D,k} \) are the projections of the limit.

  For any diagram \( D: \cat{I} \to \cat{C} \), the following triangle commutes:
  \begin{equation}\label{eq:thm:categorical_limit_is_adjoint/ic_triangle}
    \begin{aligned}
      \includegraphics[page=1]{output/thm__categorical_limit_is_adjoint}
    \end{aligned}
  \end{equation}

  Indeed,
  \begin{equation*}
    \eta_{\lim(D)}: \lim(D) \to \smash{ \overbrace{\lim(\Delta_{\lim(D)})}^{[{\lim} \bincirc \Delta \bincirc {\lim}](D)} }
  \end{equation*}
  is the unique morphism such that \eqref{eq:def:category_of_cones/limit} commutes, and (somewhat) similarly for
  \begin{equation*}
    \lim(\varepsilon_D): [{\lim} \bincirc \Delta \bincirc {\lim}](D) \to \lim(D).
  \end{equation*}

  It follows that both \( \lim(D) \) and \( [{\lim} \bincirc \Delta \bincirc {\lim}](D) \) are limits over the same diagram. By \fullref{thm:categorical_limit_is_adjoint}, they are isomorphic, and hence \eqref{eq:thm:categorical_limit_is_adjoint/ic_triangle} commutes.

  Also, for any object \( A \) in \( \cat{C} \), the following triangle also commutes:
  \begin{equation}\label{eq:thm:categorical_limit_is_adjoint/c_triangle}
    \begin{aligned}
      \includegraphics[page=2]{output/thm__categorical_limit_is_adjoint}
    \end{aligned}
  \end{equation}

  Indeed,
  \begin{equation*}
    \Delta(\eta_A) = \seq{ \eta_A: A \to \lim(\Delta_A) }_{k \in \cat{I}}
  \end{equation*}
  is the constant family consisting of \( \eta_A \) and
  \begin{equation*}
    \varepsilon_{\Delta(A)} = \seq{ \lambda_k^{\Delta_A}: \lim(\Delta_A) \to \underbrace{\Delta_A(k)}_{A} }_{k \in \cat{I}}.
  \end{equation*}
  is a constant family of the single projection of the limit to \( A \). From the commutativity of \eqref{eq:def:category_of_cones/limit} it follows that \eqref{eq:thm:categorical_limit_is_adjoint/c_triangle} also commutes.
\end{proof}

\begin{corollary}\label{thm:categorical_limit_uniqueness}
  A limit (resp. colimit) of a diagram, if it exists, is unique up to a unique isomorphism.

  This statement strengthens \fullref{thm:categorical_limit_uniqueness_lemma}.
\end{corollary}
\begin{proof}
  Follows from \fullref{thm:functor_adjoint_uniqueness} and \fullref{thm:categorical_limit_is_adjoint}.
\end{proof}

\begin{remark}\label{rem:limit_universal_mapping_property}
  The limit diagram \eqref{eq:def:category_of_cones/limit} may seem unrelated to the universal mapping properties discussed in \fullref{rem:universal_mapping_property}, however it is actually a special case.

  Suppose that, for a given category \( \cat{C} \), the limits over all \( \cat{I} \)-shaped diagrams exist and fix a diagram \( D: \cat{I} \to \cat{C} \). Consider the functors
  \begin{align*}
    \lim:   &[\cat{I}, \cat{C}] \to \cat{C} \\
    \Delta: &\cat{C} \to [\cat{I}, \cat{C}]
  \end{align*}
  discussed in \fullref{thm:categorical_limit_is_adjoint}. We have established that \( \Delta \dashv \lim \).

  For every diagram \( D: \cat{I} \to \cat{C} \), there exist unique up to a unique isomorphism object \( \lim(D) \) in \( \cat{C} \) and canonical projection map \( \lambda: [\Delta \bincirc {\lim}(D)] \to D \) satisfying the following universal mapping property:
  \begin{displayquote}
    For every object \( A \) in \( \cat{C} \) and every natural transformation \( \alpha: \Delta(A) \Rightarrow D \), there exists a unique morphism \( l: A \to \lim(D) \) such that the following diagram commutes:
    \begin{equation}\label{eq:rem:limit_universal_mapping_property/ic_triangle}
      \begin{aligned}
        \includegraphics[page=1]{output/rem__limit_universal_mapping_property}
      \end{aligned}
    \end{equation}
  \end{displayquote}

  For a fixed index object \( k \in \cat{I} \), this becomes:
  \begin{equation}\label{eq:rem:limit_universal_mapping_property/c_triangle_basic}
    \begin{aligned}
      \includegraphics[page=2]{output/rem__limit_universal_mapping_property}
    \end{aligned}
  \end{equation}

  The defining diagram \eqref{eq:def:category_of_cones/limit} of a limit simply encodes the cone naturality condition \eqref{eq:def:category_of_cones/cone_nat} into \eqref{eq:rem:limit_universal_mapping_property/c_triangle_basic}.

  Except for being simpler to check, limits defined via universal mapping properties have the advantage (compared to adjoint functors) that limits can exist for some diagrams and not for others.

  The construction for colimits is dual.
\end{remark}

\begin{proposition}\label{thm:limits_of_empty_diagram}
  The \hyperref[def:category_of_cones/limit]{limits} of an empty diagram over \( \cat{C} \) are the \hyperref[def:universal_objects/terminal]{terminal objects} and the \hyperref[def:category_of_cones/colimit]{colimits} are the \hyperref[def:universal_objects/initial]{initial object}.

  Compare this result with \fullref{thm:limits_of_identity_functor}.
\end{proposition}
\begin{proof}
  A cone of the empty diagram is simply an object of \( \cat{C} \). A limit cone is then an object \( L \) such that every other object \( A \) has a unique morphism \( l_A: A \to L \). This is precisely the definition of a terminal object.

  The statement for colimits follows by \hyperref[thm:categorical_principle_of_duality]{duality}.
\end{proof}

\begin{proposition}\label{thm:limits_of_identity_functor}
  Fix an arbitrary category \( \cat{C} \).

  \begin{thmenum}
    \thmitem{thm:limits_of_identity_functor/initial_object_ks_limit} If \( I \) is an \hyperref[def:universal_objects/initial]{initial object} of \( \cat{C} \), \( (I, \xi) \) is a limit cone of the \hyperref[eq:def:category_of_small_categories/identity]{identity functor} \( \id_{\cat{C}} \), where
    \begin{equation*}
      \xi \coloneqq \seq{ \xi_A: I \to A }_{A \in \cat{C}}
    \end{equation*}
    is the family of unique morphisms with domain \( I \).

    \thmitem{thm:limits_of_identity_functor/limit_ks_knitial_object} Conversely, if \( (L, \lambda) \) is a limit cone of the identity, then \( L \) is an initial object.
  \end{thmenum}

  \hyperref[thm:categorical_principle_of_duality]{Dually}, by \fullref{thm:universal_object_duality} and \fullref{thm:categorical_limit_duality}, the cocones (and colimits) of the identity functor are the terminal objects.

  Compare this result with \fullref{thm:limits_of_empty_diagram}.
\end{proposition}
\begin{proof}
  \SubProofOf{thm:limits_of_identity_functor/initial_object_ks_limit} For an initial object \( I \) with morphisms \( \xi \), and for any morphism \( f: B \to C \), from the uniqueness of the arrows in \( \xi \) it follows that \( \xi_C = f \bincirc \xi_B \). Thus, the following naturality diagram commutes:
  \begin{equation}\label{eq:def:thm:limits_of_identity_functor/initial_object_ks_limit/nat}
    \begin{aligned}
      \includegraphics[page=1]{output/thm__limits_of_identity_functor}
    \end{aligned}
  \end{equation}

  Therefore, \( (I, \xi) \) is a cone.

  Now let \( (A, \alpha) \) be another cone. The naturality of \( \alpha \) implies that the following diagram commutes for every object \( B \):
  \begin{equation}\label{eq:def:thm:limits_of_identity_functor/initial_object_ks_limit/half_limit}
    \begin{aligned}
      \includegraphics[page=2]{output/thm__limits_of_identity_functor}
    \end{aligned}
  \end{equation}

  In particular, since \( \xi_I = \id_I \), for every morphism \( g: A \to I \) we have
  \begin{equation}\label{eq:def:thm:limits_of_identity_functor/initial_object_ks_limit/cone_morphism_uniqueness}
    \begin{aligned}
      \includegraphics[page=3]{output/thm__limits_of_identity_functor}
    \end{aligned}
  \end{equation}
  which shows that \( \alpha_I = g \) and thus \( \alpha_I \) is the unique morphism from \( A \) to \( I \).

  Therefore, for any morphism \( f: B \to C \), the following diagram commutes:
  \begin{equation}\label{eq:def:thm:limits_of_identity_functor/initial_object_ks_limit/limit}
    \begin{aligned}
      \includegraphics[page=4]{output/thm__limits_of_identity_functor}
    \end{aligned}
  \end{equation}

  This is precisely the defining diagram for a limit cone. Therefore, \( (I, \xi) \) is a limit cone.

  \SubProofOf{thm:limits_of_identity_functor/limit_ks_knitial_object} Let \( (L, \lambda) \) be a limit cone of the identity. The component \( \lambda_B \) of the natural transformation \( \lambda \) is a morphism from \( L \) to \( B \). In order for \( L \) to be an initial object, these morphisms must be unique.

  Since \( (L, \lambda) \) is a limit cone, for any cone \( (A, \alpha) \) and any morphism \( f: B \to C \), there exists a unique morphism \( l_A: A \to L \) such that the following diagram commutes:
  \begin{equation}\label{eq:def:thm:limits_of_identity_functor/limit_ks_knitial_object/limit}
    \begin{aligned}
      \includegraphics[page=5]{output/thm__limits_of_identity_functor}
    \end{aligned}
  \end{equation}

  In particular, there exists a unique map \( l_L: L \to L \) for the cone \( (L, \lambda) \) such that the following diagram commutes:
  \begin{equation}\label{eq:def:thm:limits_of_identity_functor/limit_ks_knitial_object/endomorphism_uniqueness}
    \begin{aligned}
      \includegraphics[page=6]{output/thm__limits_of_identity_functor}
    \end{aligned}
  \end{equation}

  Since \eqref{eq:def:thm:limits_of_identity_functor/limit_ks_knitial_object/endomorphism_uniqueness} commutes with \( \id_L \) instead of \( l_L \), by uniqueness it follows that \( l_L = \id_L \). Since, by the naturality of \( \lambda \), \( \lambda_L \) also satisfies this condition, \( \lambda_L = \id_L \).

  From the naturality of \( \lambda \), for any map \( f: L \to C \) it follows that the following diagram commutes:
  \begin{equation}\label{eq:def:thm:limits_of_identity_functor/limit_ks_knitial_object/morphism_uniqueness}
    \begin{aligned}
      \includegraphics[page=7]{output/thm__limits_of_identity_functor}
    \end{aligned}
  \end{equation}

  Therefore, \( f = \lambda_B \). Since \( f \) was an arbitrary morphism with domain \( L \), we conclude that \( L \) has a unique morphism to every object in \( \cat{C} \). Hence, \( L \) is an initial object.
\end{proof}

\begin{example}\label{ex:limits_of_partially_ordered_set}
  We will show that limits and colimits correspond to suprema and infima of partially ordered sets.

  Let \( (P, \leq) \) be a \hyperref[def:partially_ordered_set]{partially ordered set} and \( \cat{P} \) be its corresponding \hyperref[def:skeletal_category]{skeletal} \hyperref[def:preorder_category]{preorder category}. The correspondence is discussed in \fullref{thm:order_category_isomorphism}.

  The image of a diagram \( D: \cat{I} \to \cat{P} \) is a pair \( (A, R) \), where \( A \) is the set of objects \( D(\obj(\cat{I})) \) in the image \( D(\cat{I}) \) and \( R \) is a subrelation of \( \leq \). This relation is reflexive, however it may not even be a preorder as shown in \fullref{ex:functor_image_not_a_category}.

  Conversely, every subset \( A \subseteq P \) with a corresponding category \( \cat{A} \) is given by the diagram \( \Iota_A \), where the \hyperref[def:subcategory]{inclusion functor} \( \Iota_A: \cat{A} \to \cat{P} \).

  A cone of a diagram \( D: \cat{I} \to \cat{P} \) is a \hyperref[def:extremal_points/bounds]{lower bound} of the set \( D(\obj(\cat{I})) \). The relation induced by \( D \) does not matter here.

  Indeed, the cone \( (x, R) \) in \( \cat{P} \) consists of morphisms with domain \( x \). That is, \( R \) is a subrelation of \( \leq \) whose first component is \( x \). Clearly then \( x \) is a lower bound of the set \( D(\obj(\cat{I})) \).

  For the limiting cone \( (y, S) \), it holds that \( x \leq y \) for every other cone \( (x, R) \). Both \( R \) and \( S \) are subrelations of \( \leq \) with the same second components. Thus, \( y \) is the greatest lower bound of \( D(\obj(\cat{I})) \).

  \hyperref[thm:categorical_principle_of_duality]{Dually}, cocones are upper bounds and colimits are suprema.

  As mentioned, the relation induced by the diagram \( D \) does not actually matter. Therefore, we may choose, without loss of generality, \( D \) to be a \hyperref[def:discrete_category]{discrete category}. It then follows that infima and suprema correspond to \hyperref[def:discrete_category_limits]{products} and \hyperref[def:discrete_category_limits]{coproduct}, however what we have shown here is more general.
\end{example}

\begin{definition}\label{def:direct_and_inverse_limits}
  We will define limits and colimits of diagrams over infinite \hyperref[def:partial_order_chain]{chains} of integers. These notions predate limits and colimits, which explains why their names may seem inconsistent with other limits and colimits.

  \begin{thmenum}
    \thmitem{def:direct_and_inverse_limits/direct}\mcite[example 5.2.15]{Leinster2016Basic} Consider the category \hyperref[thm:order_category_isomorphism]{induced} by the positive integers
    \begin{equation}\label{eq:def:direct_and_inverse_limits/direct}
      \begin{aligned}
        \includegraphics[page=1]{output/def__direct_and_inverse_limits}
      \end{aligned}
    \end{equation}

    A \hyperref[def:category_of_cones/colimit]{colimit} over a diagram of this shape is called a \term{direct limit}. In this context, the diagram itself is sometimes called a \term{direct system}.

    \thmitem{def:direct_and_inverse_limits/inverse}\mcite[example 5.1.21(d)]{Leinster2016Basic} Now consider the opposite category
    \begin{equation}\label{eq:def:direct_and_inverse_limits/inverse}
      \begin{aligned}
        \includegraphics[page=2]{output/def__direct_and_inverse_limits}
      \end{aligned}
    \end{equation}

    Somewhat confusingly, a \hyperref[def:category_of_cones/limit]{limit} of a diagram of this shape is called an \term{inverse limit}. In this context, the diagram itself is sometimes called an \term{inverse system}.
  \end{thmenum}
\end{definition}

\begin{example}\label{ex:def:direct_and_inverse_limits}
  We will list several examples of \hyperref[def:direct_and_inverse_limits/direct]{direct} and \hyperref[def:direct_and_inverse_limits/inverse]{inverse} limits.

  \begin{thmenum}
    \thmitem{ex:def:direct_and_inverse_limits/vector_space_direct} Consider the chain of \hyperref[def:vector_space]{vector spaces}
    \begin{equation}\label{eq:ex:def:direct_and_inverse_limits/vector_space_direct/diagram}
      \begin{aligned}
        \includegraphics[page=1]{output/ex__def__direct_and_inverse_limits}
      \end{aligned}
    \end{equation}
    where \( \iota_n^m: \BbbK^n \to \BbbK^m \) is simply the canonical inclusion map.

    Let \( \BbbK_0^\infty \) be the vector space of all sequences in \( \BbbK \) with only finitely many nonzero elements. For each positive integer \( n \), denote by \( \iota_n^\infty: \BbbK^n \to \BbbK_0^\infty \) the canonical inclusion.

    Then \( (\BbbK_0^\infty, \iota) \) is a \hyperref[def:category_of_cones/cocone]{cocone} of \eqref{eq:ex:def:direct_and_inverse_limits/vector_space_direct/diagram}. We will show that it is a colimit cocone, i.e. that \( \BbbK_0^\infty \) is a \hyperref[def:direct_and_inverse_limits]{direct limit} of \eqref{eq:ex:def:direct_and_inverse_limits/vector_space_direct/diagram}.

    Let \( (A, \alpha) \) be another cocone. We want to define a linear map \( l_A: \BbbK_0^\infty \to A \) so that the following diagram commutes:
    \begin{equation}\label{eq:ex:def:direct_and_inverse_limits/vector_space_direct/limit}
      \begin{aligned}
        \includegraphics[page=2]{output/ex__def__direct_and_inverse_limits}
      \end{aligned}
    \end{equation}

    For every vector \( x \in \BbbK^m \), we must have
    \begin{equation*}
      \alpha_n(x) = l_A(\iota_n^\infty(x)).
    \end{equation*}

    This implies the obvious definition where, given a vector \( x \in \BbbK_0^\infty \) whose greatest nonzero element has index \( m \), we define
    \begin{equation*}
      l_A(x) \coloneqq \alpha_m(x).
    \end{equation*}

    This map is well-defined because the compatibility with inclusion maps guarantees that \( \alpha_{m+1}(x) = \iota_{m+1}(\alpha^{m+1}_m(x)) \), i.e. the result obtained by using \( \alpha_m \) and \( \alpha_{m+1} \) is the same.

    Therefore, \( (\BbbK_0^\infty, \iota) \) is a direct limit of the diagram \eqref{eq:ex:def:direct_and_inverse_limits/vector_space_direct/diagram}.

    \thmitem{ex:def:direct_and_inverse_limits/vector_space_inverse} \hyperref[thm:categorical_principle_of_duality]{Dually}, consider the chain
    \begin{equation}\label{eq:ex:def:direct_and_inverse_limits/vector_space_inverse/diagram}
      \begin{aligned}
        \includegraphics[page=3]{output/ex__def__direct_and_inverse_limits}
      \end{aligned}
    \end{equation}
    where
    \begin{equation*}
      \begin{aligned}
        &\pi_n^m: \BbbK^m \to \BbbK^n \\
        &\pi_n^m(x_1, \ldots, x_n, x_{k+1}, \ldots, x_m) \coloneqq (x_1, \ldots, x_n).
      \end{aligned}
    \end{equation*}

    Let \( \BbbK^\infty \) be the vector space of all sequences in \( \BbbK \) and, for each positive integer \( n \), define \( \pi_n^\infty \) as a truncation in the obvious way.

    Then \( (\BbbK^\infty, \pi^\infty) \) is a \hyperref[def:category_of_cones/cone]{cone} of \eqref{eq:ex:def:direct_and_inverse_limits/vector_space_direct/diagram}. We will show that it is a limit cone, i.e. that \( \BbbK^\infty \) is a \hyperref[def:direct_and_inverse_limits]{direct limit} of \eqref{eq:ex:def:direct_and_inverse_limits/vector_space_direct/diagram}.

    Let \( (A, \alpha) \) be another cone. We want to define \( l_A: A \to \BbbK^\infty \) so that the following diagram commutes:
    \begin{equation}\label{eq:ex:def:direct_and_inverse_limits/vector_space_inverse/limit}
      \begin{aligned}
        \includegraphics[page=4]{output/ex__def__direct_and_inverse_limits}
      \end{aligned}
    \end{equation}

    For every \( a \in A \), we must have
    \begin{equation*}
      \alpha_n(a) = \pi_n^\infty(l_A(a)),
    \end{equation*}
    which implies the obvious definition where the \( n \)-th coordinate of the vector \( l_A(a) \) is
    \begin{equation*}
      [l_A(a)]_n \coloneqq [\alpha_n(a)]_n.
    \end{equation*}

    That is, \( l_A(a) \) is a sequence whose \( n \)-th coordinate is the \( n \)-th coordinate \( \alpha_n(a) \). This is well-defined because, for \( m > n \), we have
    \begin{equation*}
      \pi_m^n \bincirc \alpha_m = \alpha_n.
    \end{equation*}

    \thmitem{ex:def:direct_and_inverse_limits/generalized_intersection} In \hyperref[def:concrete_category]{concrete categories}, \hyperref[def:direct_and_inverse_limits/inverse]{inverse limits} are generalizations of \hyperref[thm:zfc_existence_theorems/arbitrary_intersection]{set intersections}.

    As an example, consider the chain of sets
    \begin{equation}\label{eq:ex:inverse_limit_as_intersection/diagram}
      \begin{aligned}
        \includegraphics[page=5]{output/ex__def__direct_and_inverse_limits}
      \end{aligned}
    \end{equation}
    where \( A_{k+1} \subseteq A_k \) for every positive integer \( k \) and \( \iota_k^m: A_m \to A_k \) is simply the canonical inclusion map.

    Then the intersection \( \bigcap_{k=1}^\infty A_k \) along with its inclusion maps is a limit of \eqref{eq:ex:inverse_limit_as_intersection/diagram}. This is a consequence of the discussion in \fullref{ex:limits_of_partially_ordered_set}.

    We can replace the inclusion maps \( \iota_k^m \) with other injective functions, or even non-injective functions. In this case, we would obtain a \enquote{generalized intersection}.
  \end{thmenum}
\end{example}

\begin{definition}\label{def:discrete_category_limits}
  Fix an arbitrary category \( \cat{C} \). A \term{product} in \( \cat{C} \) is a \hyperref[def:category_of_cones/limit]{limit} over a diagram \( D: \cat{I} \to \cat{C} \), whose domain \( \cat{I} \) is a \hyperref[def:discrete_category]{discrete category}. \hyperref[thm:categorical_principle_of_duality]{Dually}, a \term{coproduct} or \term{sum} in \( \cat{C} \) is a colimit of \( D \).

  It will be convenient for us to speak about the product of an \hyperref[def:cartesian_product/indexed_family]{indexed family} \( \seq{ X_k }_{k \in \mscrK} \) of objects in \( \cat{C} \).

  \begin{minipage}[t]{0.47\textwidth}
    A cone with vertex \( A \) consists of morphisms with signatures
    \begin{equation*}
      \alpha = \seq{ \alpha_k: A \to X_k }.
    \end{equation*}
  \end{minipage}
  \hfill
  \begin{minipage}[t]{0.47\textwidth}
    A cocone with vertex \( A \) consists of morphisms with signatures
    \begin{equation*}
      \alpha = \seq{ \alpha_k: X_k \to A }.
    \end{equation*}
  \end{minipage}

  \begin{minipage}[t]{0.47\textwidth}
    A product \hyperref[def:category_of_cones/cone]{cone} \( (\prod_{k \in \mscrK} X_k, \pi) \) satisfies the following \hyperref[rem:limit_universal_mapping_property]{universal mapping property}:
    \begin{displayquote}
      For every cone \( (A, \alpha) \), there exists a unique morphism
      \begin{equation*}
        l_A: A \to \prod_{k \in \mscrK} X_k,
      \end{equation*}
      such that the following diagram commutes:
    \end{displayquote}
    \begin{equation}\label{eq:def:discrete_category_limits/product}
      \begin{aligned}
        \includegraphics[page=1]{output/def__discrete_category_limits}
      \end{aligned}
    \end{equation}
  \end{minipage}
  \hfill
  \begin{minipage}[t]{0.47\textwidth}
    A coproduct \hyperref[def:category_of_cones/cocone]{cocone} \( (\coprod_{k \in \mscrK} X_k, \iota) \) satisfies the following \hyperref[rem:limit_universal_mapping_property]{universal mapping property}:
    \begin{displayquote}
      For every cone \( (A, \alpha) \), there exists a unique morphism
      \begin{equation*}
        l_A: A \to \coprod_{k \in \mscrK} X_k,
      \end{equation*}
      such that the following diagram commutes:
    \end{displayquote}
    \begin{equation}\label{eq:def:discrete_category_limits/coproduct}
      \begin{aligned}
        \includegraphics[page=3]{output/def__discrete_category_limits}
      \end{aligned}
    \end{equation}
  \end{minipage}

  \begin{minipage}[t]{0.47\textwidth}
    We call the morphism \( \pi_k \) the \term{canonical projection} of the product onto \( X_k \), even though it may not be a surjective function.
  \end{minipage}
  \hfill
  \begin{minipage}[t]{0.47\textwidth}
    We call the morphism \( \iota_k \) the \term{canonical inclusion} of \( X_k \) into the coproduct, even though it may not be an injective function.
  \end{minipage}
  \medskip

  From \fullref{thm:limits_of_empty_diagram} it follows that the product (resp. coproduct) of an empty family is a terminal (resp. initial) object of \( \cat{C} \).

  As in the case of general limits, we call \( \prod_{k \in \mscrK} X_k \) \hi{the} product of the family \( X \).

  In the case of only two objects, their product is given by the following diagram:
  \begin{equation}\label{eq:def:discrete_category_limits/product/binary}
    \begin{aligned}
      \includegraphics[page=2]{output/def__discrete_category_limits}
    \end{aligned}
  \end{equation}
  and their coproduct by
  \begin{equation}\label{eq:def:discrete_category_limits/coproduct/binary}
    \begin{aligned}
      \includegraphics[page=4]{output/def__discrete_category_limits}
    \end{aligned}
  \end{equation}
\end{definition}

\begin{proposition}\label{thm:discrete_category_limits_in_set}
  The \hyperref[def:discrete_category_limits]{product} in the category \hyperref[def:category_of_small_sets]{\( \ucat{Set} \)} of \( \mscrU \)-small sets of a family \( \mscrA = \seq{ A_k }_{k \in \mscrK} \) is their \hyperref[def:cartesian_product/product]{Cartesian product} \( \prod_{k \in \mscrK} A_k \) and the \hyperref[def:discrete_category_limits]{coproduct} is their \hyperref[def:disjoint_union]{disjoint union} \( \coprod_{k \in \mscrK} A_k \).
\end{proposition}
\begin{proof}
  \SubProof{Proof for products} Consider the Cartesian product
  \begin{equation*}
    L \coloneqq \prod_{k \in \mscrK} A_k = \set*{ f: A \to \bigcup_{k \in \mscrK} A_k \given* \qforall {k \in \mscrK} k \in A_k }.
  \end{equation*}

  Since \( \mscrU \) is a model of \logic{ZFC} (with or without the axiom of infinity), hence the product is an object of \( \ucat{Set} \). Define the projection morphisms
  \begin{equation*}
    \pi_k(f) \coloneqq f(k).
  \end{equation*}

  Then \( (L, \pi) \) is a cone for (some diagram forming) \( \mscrA \). Let \( (B, \beta) \) also be a cone and consider the following diagram:
  \begin{equation}\label{eq:thm:discrete_category_limits_in_set/limit}
    \begin{aligned}
      \includegraphics[page=1]{output/thm__discrete_category_limits_in_set}
    \end{aligned}
  \end{equation}

  We want to define a function \( l_B: B \to L \) such that
  \begin{equation*}
    \underbrace{ \pi_k(l_B(b)) }_{[l_B(b)](k)} = \beta_k(b)
  \end{equation*}
  for every \( b \in B \) and \( k \in \mscrK \). This is uniquely determined by \( b \) and \( k \), hence
  \begin{equation*}
    l_B(b) \coloneqq \seq{ \beta_k(b) }_{k \in \mscrK}.
  \end{equation*}

  The cone \( (B, \beta) \) was chosen randomly, and we showed that there exists a unique morphism \( l_B: B \to L \) such that \eqref{eq:thm:discrete_category_limits_in_set/limit} commutes. Therefore, \( (L, \pi) \) is a categorical product of the family \( \mscrA \).

  \SubProof{Proof for coproducts} Now consider the disjoint union
  \begin{equation*}
    L \coloneqq \coprod_{k \in \mscrK} A_k = \set{ (k, a) \given k \in \mscrK \T{and} a \in A_k }
  \end{equation*}
  and the inclusions
  \begin{equation*}
    \iota_k\parens[\Big]{ (k, a) } = a.
  \end{equation*}

  We must show that, for any other cocone \( (B, \beta) \), the following diagram commutes:
  \begin{equation}\label{eq:thm:discrete_category_limits_in_set/colimit}
    \begin{aligned}
      \includegraphics[page=2]{output/thm__discrete_category_limits_in_set}
    \end{aligned}
  \end{equation}

  Analogously to products, the condition
  \begin{equation*}
    \underbrace{ l_B(\iota_k(a)) }_{l_B(k, a)} = \beta_k(a)
  \end{equation*}
  for every \( b \in B \) and \( k \in \mscrK \) uniquely identifies the function
  \begin{equation*}
    l_B(k, a) \coloneqq \beta_k(a).
  \end{equation*}
\end{proof}

\begin{definition}\label{def:equalizers}\mcite[def. 5.1.11]{Leinster2016Basic}
  Consider the index category
  \begin{equation}\label{eq:def:equalizers/index}
    \begin{aligned}
      \includegraphics[page=1]{output/def__equalizers}
    \end{aligned}
  \end{equation}

  A diagram indexed by \eqref{eq:def:equalizers/index} is sometimes called a \term{fork}. A \hyperref[def:category_of_cones/limit]{limit} of a fork is called an \term[ru=уравнитель \cite[36]{ЦаленкоШульгейфер1974}]{equalizer} and a \hyperref[def:category_of_cones/colimit]{colimit} --- a \term[ru=коуравнитель \cite[37]{ЦаленкоШульгейфер1974}]{coequalizer}.

  Note that in \fullref{def:categorical_diagram} we defined commutativity only when at least one path is nontrivial. That is, \eqref{eq:def:equalizers/index} cannot commute by definition since all of its paths have length either \( 0 \) or \( 1 \).

  We will describe equalizers in more detail. Fix a fork in \( \cat{C} \) which we will denote by
  \begin{equation}\label{eq:def:equalizers/raw_diagram}
    \begin{aligned}
      \includegraphics[page=2]{output/def__equalizers}
    \end{aligned}
  \end{equation}

  \begin{minipage}[t]{0.47\textwidth}
    A \hyperref[def:category_of_cones/cone]{cone} with vertex \( A \) over this diagram has a single morphism \( f: A \to X \) such that the following diagram commutes:
    \begin{equation}\label{eq:def:equalizers/cone}
      \begin{aligned}
        \includegraphics[page=3]{output/def__equalizers}
      \end{aligned}
    \end{equation}

    This is equivalent to requiring
    \begin{equation*}
      s \bincirc f = t \bincirc f.
    \end{equation*}

    An equalizer limit cone \( (L, \iota) \) satisfies the following \hyperref[rem:limit_universal_mapping_property]{universal mapping property}:
    \begin{displayquote}
      For every cone \( (A, f) \), \( f \) \hyperref[def:factors_through]{uniquely factors through} \( L \). More precisely, there exists a unique morphism
      \begin{equation*}
         l_A: A \to L
      \end{equation*}
      such that the following diagram commutes:
    \end{displayquote}
    \begin{equation}\label{eq:def:equalizers/equalizer}
      \begin{aligned}
        \includegraphics[page=4]{output/def__equalizers}
      \end{aligned}
    \end{equation}
  \end{minipage}
  \hfill
  \begin{minipage}[t]{0.47\textwidth}
    A \hyperref[def:category_of_cones/cocone]{cocone} with vertex \( A \) over this diagram has a single morphism \( f: Y \to A \) such that the following diagram commutes:
    \begin{equation}\label{eq:def:equalizers/cocone}
      \begin{aligned}
        \includegraphics[page=5]{output/def__equalizers}
      \end{aligned}
    \end{equation}

    This is equivalent to requiring
    \begin{equation*}
      f \bincirc s = f \bincirc t.
    \end{equation*}

    A coequalizer cocone \( (L, \pi) \) satisfies the following \hyperref[rem:limit_universal_mapping_property]{universal mapping property}:
    \begin{displayquote}
      For every cocone \( (A, f) \), \( f \) \hyperref[def:factors_through]{uniquely factors through} \( L \). More precisely, there exists a unique morphism
      \begin{equation*}
        l_A: L \to A
      \end{equation*}
      such that the following diagram commutes:
    \end{displayquote}
    \begin{equation}\label{eq:def:equalizers/coequalizer}
      \begin{aligned}
        \includegraphics[page=6]{output/def__equalizers}
      \end{aligned}
    \end{equation}
  \end{minipage}

  The requirement that \eqref{eq:def:equalizers/equalizer} commutes does not mean that \( p = q \). As discussed in \fullref{def:categorical_diagram}, for commutative diagrams, we only consider a pair of paths if at least one of them is nontrivial. We made this requirement in order to allow parallel morphisms.

  Note how we interchanged the notation for the projections and inclusions compared to \fullref{def:discrete_category_limits} --- \( (L, \iota) \) is a \hi{limit} of a fork, while \( (L, \pi) \) is a \hi{colimit}. The reason for this is that equalizers are usually canonical inclusions, while coequalizers are projections.
\end{definition}

\begin{proposition}\label{thm:equalizer_invertibility}
  If \( (L, \iota) \) is an \hyperref[def:equalizers]{equalizer} in the category \( \cat{C} \), then \( \iota \) is a \hyperref[def:morphism_invertibility/left_cancellative]{monomorphism}.

  \hyperref[thm:categorical_principle_of_duality]{Dually}, if \( (L, \pi) \) is a \hyperref[def:equalizers]{coequalizer}, then \( \pi \) is an \hyperref[def:morphism_invertibility/right_cancellative]{epimorphism}.
\end{proposition}
\begin{proof}
  Fix an equalizer cone \( (L, \iota) \) of the fork \eqref{eq:def:equalizers/raw_diagram}. Fix any object \( A \) in \( \cat{C} \) and any two parallel morphisms \( a_1, a_2: A \to X \) such that
  \begin{equation*}
    \iota \bincirc a_1 = \iota \bincirc a_2.
  \end{equation*}

  Then
  \begin{equation*}
    g \bincirc \iota \bincirc a_1 = h \bincirc \iota \bincirc a_2,
  \end{equation*}
  and thus \( \iota \bincirc a_1 = \iota \bincirc a_2 \) is the morphism of a cone. Hence, both \( (A, \iota \bincirc a_1) \) and \( (A, \iota \bincirc a_2) \) are cones and thus \( a_1 \) and \( a_2 \) are the unique maps such that the following diagram commutes:
  \begin{equation}\label{eq:thm:equalizer_invertibility/monomorphism}
    \begin{aligned}
      \includegraphics[page=1]{output/thm__equalizer_invertibility}
    \end{aligned}
  \end{equation}

  Therefore, \( a_1 = a_2 \) and, since \( a_1 \) and \( a_2 \) were arbitrary, it follows that \( \iota \) is a monomorphism.

  Dually, if \( (L, \pi) \) is a coequalizer colimit cocone in \( \cat{C} \), then by \fullref{thm:categorical_limit_duality}, \( (L, \pi^\oppos) \) is an equalizer in \( \cat{C}^\oppos \). Hence, \( \pi^\oppos \) is a monomorphism, and by \fullref{thm:morphism_invertibility_duality}, \( \pi \) is an epimorphism.
\end{proof}

\begin{example}\label{ex:equalizers_in_set}
  \hfill
  \begin{thmenum}
    \thmitem{ex:equalizers_in_set/equalizer} For a pair of functions \( s, t: X \to Y \) in \( \cat{Set} \), define the set
    \begin{equation*}
      E = \set{ x \in X \given s(x) = t(x) }.
    \end{equation*}

    With the inclusion map \( \iota: E \to X \), this is an \hyperref[def:equalizers]{equalizer} cone for \( s \) and \( t \).

    We will now prove that it is a limit cone. The pair \( (E, \iota) \) is obviously a cone. For any other cone \( (A, f) \), we must find a map \( l_A: A \to E \) such that the following equalizer diagram commutes:
    \begin{equation}\label{eq:ex:equalizers_in_set/equalizer}
      \begin{aligned}
        \includegraphics[page=1]{output/ex__equalizers_in_set}
      \end{aligned}
    \end{equation}

    In order for \( (A, f) \) to be a cone, the image \( f[A] \) must be a subset of \( E \). In order for \eqref{eq:ex:equalizers_in_set/equalizer} to commute, it only makes sense to take \( l_A \) to be \( f \) with its codomain restricted to \( E \).

    It follows that \( (E, \iota) \) is a limit cone.

    \thmitem{ex:equalizers_in_set/coequalizer} The coequalizer of \( s, t: X \to Y \) is more nuanced. Suppose that \( (L, \pi) \) is a colimit cocone and \( (A, f) \) is any cocone for the coequalizer diagram
    \begin{equation}\label{eq:ex:equalizers_in_set/coequalizer}
      \begin{aligned}
        \includegraphics[page=2]{output/ex__equalizers_in_set}
      \end{aligned}
    \end{equation}

    In order for \( (A, f) \) to be a cocone, for every \( x \in X \), we must have \( f(s(x)) = f(t(x)) \). Thus, \( A \) must be a partition of \( Y \) in a way such that \( s(x) \) and \( t(x) \) belong to the same coset if and only if \( f(s(x)) = f(t(x)) \). Outside the images of \( s \) and \( t \), \( f \) is free to take any value.

    Let \( {\sim} \) be the smallest equivalence relation on \( Y \) such that \( s(x) \sim t(x) \) for every \( x \in X \). Explicitly, this is the \hyperref[thm:equivalence_closure]{equivalence closure} of the relation
    \begin{equation*}
      \set{ (s(x), t(x)) \given x \in X }.
    \end{equation*}

    Consider the partition \( Y / {\sim} \) with the projection map
    \begin{equation*}
      \begin{aligned}
        &\pi: Y \to Y / {\sim} \\
        &\pi(y) \coloneqq [y]
      \end{aligned}
    \end{equation*}

    This is a cocone. Furthermore, it is a colimit cocone because, for any other cocone \( (A, f) \), we can define \( l_A([y]) \coloneqq f(y) \) so that \eqref{eq:ex:equalizers_in_set/coequalizer} commutes.

    Coequalizers mostly make sense in the context of \hyperref[def:group/quotient]{quotient groups}, where partitions are especially well-behaved and admit a much simpler description. See \fullref{def:group/quotient}.
  \end{thmenum}
\end{example}
