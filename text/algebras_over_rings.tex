\subsection{Algebras over rings}\label{subsec:algebras_over_rings}

\paragraph{Algebras over rings}

\begin{definition}\label{def:algebra_over_ring}\mimprovised
  An \term[bg=алгебра (\cite[4]{КоцевСидеров2016}), ru=алгебра (\cite[def. 1.7.1]{Винберг2014})]{algebra} over a commutative ring \( R \) rather than over a \hyperref[def:algebra_over_semiring]{semiring} exhibits some more interesting metamathematical properties.

  \begin{thmenum}
    \thmitem{def:algebra_over_ring/theory} The first-order theory is identical to the \hyperref[def:algebra_over_semiring/theory]{theory of algebras over semimodules}.

    \thmitem{def:algebra_over_ring/homomorphism} A \hyperref[def:first_order_homomorphism]{first-order homomorphism} between two \( R \)-algebras is a \hyperref[def:semimodule/homomorphism]{linear map} that preserves multiplication. This is the same as for semirings.

    \thmitem{def:algebra_over_ring/submodel} The set \( A \subseteq M \) is a \hyperref[def:first_order_submodel]{submodel} of \( M \) if it is a \hyperref[def:monoid/submodel]{submodule} of \( M \) that contains \( 1 \) and is closed under algebra multiplication. We say that \( A \) is an \( R \)-\term{subalgebra} of \( M \).

    If \( A \) does not contain \( 1 \), we may instead refer to nonunital \( R \)-subalgebras. We will use them for quotients.

    As a consequence of \fullref{thm:positive_formulas_preserved_under_homomorphism}, the image of an \( R \)-algebra homomorphism is an \( R \)-subalgebra of its codomain.

    \thmitem{def:algebra_over_ring/category} For a fixed ring \( R \), we denote the \hyperref[def:category_of_small_first_order_models]{category of \( \mscrU \)-small models} by \( \ucat{Alg}_R \). It is concrete with respect to both \( \ucat{CRing} \) and \( \ucat{Mod}_R \).

    Unfortunately, these categories are not as well-behaved as categories of modules. Similarly to rings, unital and nonunital algebras behave differently.

    \thmitem{def:algebra_over_ring/trivial} Similarly to rings, \enquote{the} \hyperref[def:trivial_object]{trivial object} is the one-element algebra \( \set{ 0 } \).

    \thmitem{def:algebra_over_ring/kernel} \Fullref{thm:ring_zero_morphisms/kernel} implies that the \hyperref[def:zero_morphisms/kernel]{categorical kernel} of a homomorphism \( \varphi: M \to N \) between \hi{nonunital} \( R \)-algebras is the additive group kernel
    \begin{equation*}
      \ker \varphi \coloneqq \varphi^{-1}(0_N) = \set{ x \in M \given \varphi(x) = 0_N }.
    \end{equation*}

    The kernel is a both a two-sided ideal of \( M \) as a consequence of \fullref{thm:kernel_is_ideal} and a submodule of \( M \) as a consequence of \fullref{thm:kernel_is_submodule}.

    \thmitem{def:algebra_over_ring/quotient} Similarly to rings, we define quotient \( R \)-algebras of \( M \) by nonunital \( R \)-subalgebras. In particular, \fullref{thm:algebra_ideal_is_subalgebra} implies that we can take the quotient by any ideal of \( M \).

    \thmitem{def:algebra_over_ring/commutative} As in the case of algebras over semirings, by \enquote{\( M \) is commutative}, we will mean that vector multiplication is commutative.

    We denote the subcategory of commutative algebras by \( \cat{CAlg}_R \).
  \end{thmenum}
\end{definition}

\begin{proposition}\label{thm:algebra_ideal_is_subalgebra}
  Every \hyperref[def:semiring_ideal]{left ideal} of an \( R \)-\hyperref[def:algebra_over_ring]{algebra} is a nonunital \( R \)-\hyperref[def:algebra_over_ring/submodel]{subalgebra}.
\end{proposition}
\begin{proof}
  Let \( I \) be a left ideal of the \( R \)-algebra \( M \). By definition of left ideal, \( I \) is a left \( M \)-submodule of \( M \), and, because \( R \) is a subring of \( M \), \( I \) is a left \( R \)-submodule of \( M \).

  Furthermore, \eqref{eq:def:semiring_ideal/direct/multiplicative} implies that \( I \) is closed under vector multiplication with arbitrary elements of \( M \), hence \( I \) is also an \( R \)-subalgebra.
\end{proof}

\begin{proposition}\label{thm:ring_is_integer_algebra}
  The categories \( \hyperref[def:ring/category]{\cat{Ring}} \) of rings and \( \hyperref[def:algebra_over_ring/category]{\cat{Alg}_\BbbZ} \) of integer algebras are \hyperref[rem:category_similarity/isomorphism]{isomorphic}.
\end{proposition}
\begin{comments}
  \item Compare this result to \fullref{thm:abelian_group_is_module} for modules and \fullref{thm:semiring_is_natural_number_algebra} for algebras over semirings.
\end{comments}
\begin{proof}
  Follows from \fullref{thm:semiring_is_algebra}.
\end{proof}

\paragraph{Algebra presentations}

\begin{definition}\label{def:algebra_presentation}
  A \term{presentation} of the \( R \)-\hyperref[def:algebra_over_ring]{algebra} \( M \) is a surjective \hyperref[def:module/homomorphism]{homomorphism}\footnote{Epimorphisms may be too general.} \( \varphi: R[\mscrS] \to M \), where \( R[\mscrS] \) is a \hyperref[def:polynomial_algebra]{polynomial ring} with a set of indeterminates \( \mscrS \).

  Analogously to \hyperref[def:group_presentation]{group presentations}, we say that \( M \) is finitely generated/related/presented if there exists an appropriate presentation.
\end{definition}
\begin{comments}
  \item Compare this to \hyperref[def:group_presentation]{group presentations} and \hyperref[def:module_presentation]{module presentations}.
\end{comments}

\begin{proposition}\label{thm:algebra_presentation_existence}
  Every algebra has at least one \hyperref[def:algebra_presentation]{presentation}.
\end{proposition}
\begin{proof}
  This can be proven analogously to \fullref{thm:polynomial_algebra_universal_property} by using rather than \fullref{thm:free_semimodule_universal_property}.
\end{proof}

\paragraph{Noetherian algebras}

\begin{definition}\label{def:noetherian_semimodule}\mcite[prop. 6.16]{Golan2010}
  We say that an \( R \)-\hyperref[def:semimodule]{semimodule} is \term[bg=ньотеров (\cite[41]{КоцевСидеров2016})]{noetherian} if any of the following equivalent conditions hold:
  \begin{thmenum}
    \thmitem{def:noetherian_semimodule/acc} It satisfies the \hyperref[def:chain_condition]{ascending chain condition} on \( R \)-sub-semimodules.

    \thmitem{def:noetherian_semimodule/generated} Every \( R \)-sub-semimodule is \hyperref[def:module_presentation]{finitely generated}, i.e. is the \hyperref[def:semimodule/submodel]{linear span} of finitely many elements.
  \end{thmenum}
\end{definition}
\begin{defproof}
  Fix an \( R \)-semimodule \( M \).

  \ImplicationSubProof{def:noetherian_semimodule/acc}{def:noetherian_semimodule/generated} Suppose that \( M \) satisfies \fullref{def:chain_condition/maximal}, i.e. every nonempty family of \( R \)-sub-semimodules of \( M \) has a maximal element.

  Let \( K \coloneqq \linspan{ x_1, \ldots, x_n } \) be maximal in the family of all finitely-generated \( R \)-sub-semimodules. Let \( N \) be any sub-semimodule. Adding a particular element from \( N \) does not change \( K \), because otherwise it would not be maximal. Thus, \( N \subseteq K \).

  \ImplicationSubProof{def:noetherian_semimodule/generated}{def:noetherian_semimodule/acc} Suppose that every \( R \)-sub-semimodule is finitely generated.

  Consider the ascending chain of \( R \)-sub-semimodules
  \begin{equation}\label{eq:def:noetherian_semimodule/chain}
    N_1 \subseteq N_2 \subseteq N_3 \subseteq \cdots
  \end{equation}

  By \fullref{thm:def:semimodule/union}, their union \( \bigcup_{k \in \mscrK} N_k \) is also an \( R \)-sub-semimodule.

  Let \( x_1, \ldots, x_n \) be the set of generators for the union. Let \( k_m \) be the index of the first sub-semimodule that contains \( x_m \). Every next sub-semimodule in the chain contains the previous, hence \( k \leq k_m \) implies that \( x_k \in N_{k_m} \).

  Let \( k_{m_0} \) be a maximal index. Then \( N_{k_{m_0}} \) contains all the generators, and hence it coincides with the union \( \bigcup_{k \in \mscrK} N_k \). Every sub-semimodule with a greater index is simply equal to the previous.

  Therefore, the chain \eqref{eq:def:noetherian_semimodule/chain} stabilizes.
\end{defproof}

\begin{proposition}\label{thm:def:noetherian_semimodule}
  \hyperref[def:noetherian_semimodule]{Noetherian modules} over an arbitrary ring \( R \) have the following basic properties:
  \begin{thmenum}
    \thmitem{thm:def:noetherian_semimodule/submodule} If \( M \) is noetherian, then every \( R \)-submodule of \( M \) also is.
    \thmitem{thm:def:noetherian_semimodule/quotient}\mcite[prop. 6.3b)]{КоцевСидеров2016} Let \( N \) be an \( R \)-\hyperref[def:module/submodel]{submodule} of \( M \). Then \( M \) is noetherian if and only if both \( N \) and their \hyperref[def:module/quotient]{quotient} \( M / N \) are.
  \end{thmenum}
\end{proposition}
\begin{proof}
  \SubProofOf{thm:def:noetherian_semimodule/submodule} Trivial.
  \SubProofOf{thm:def:noetherian_semimodule/quotient} By \fullref{thm:lattice_theorem_for_submodules}, every chain of \( R \)-submodules of \( M / N \) corresponds to a chain of \( R \)-submodules in \( M \). Thus, if \( M \) is noetherian, clearly \( M / N \) also is.

  Conversely, suppose that both \( N \) and \( M / N \) are noetherian. Let
  \begin{equation}\label{eq:thm:def:noetherian_semimodule/quotient/chain}
    K_1 \subseteq K_2 \subseteq K_3 \subseteq \cdots
  \end{equation}
  be an ascending chain of \( R \)-submodule of \( M \). Then
  \begin{equation}\label{eq:thm:def:noetherian_semimodule/quotient/chain/intersection}
    K_1 \cap N \subseteq K_2 \cap N \subseteq K_3 \cap N \subseteq \cdots
  \end{equation}
  is an ascending chain of \( R \)-submodules of \( N \) and
  \begin{equation}\label{eq:thm:def:noetherian_semimodule/quotient/chain/quotient}
    (K_1 + N) / N \subseteq (K_2 + N) / N \subseteq (K_3 + N) / N \subseteq \cdots
  \end{equation}
  is an ascending chain of \( R \)-submodule of \( M / N \).

  Both \eqref{eq:thm:def:noetherian_semimodule/quotient/chain/intersection} and \eqref{eq:thm:def:noetherian_semimodule/quotient/chain/quotient} stabilize. Let \( n \) be an index such that, for every positive integer \( k \), \( K_n \cap N = K_{n + k} \cap N \) and \( (K_n + N) / N = (K_{n + k} + N) / N \). For a fixed \( k \), we will show that \( K_n = K_{n + k} \).

  Let \( x \in K_{n + k} \). If \( x \in N \), then \( x \in K_n \) since \( K_n \cap N = K_{n + k} \cap N \). Suppose that \( x \in K_{n + k} \setminus N \). For any \( n \in N \), we have \( x + n \in K_{n + k} + N \), and hence
  \begin{equation*}
    x + n + N = x + N \in (K_{n + k} + N) / N = (K_n + N) / N.
  \end{equation*}

  Then there exists some \( y \in K_n \) such that \( x - y \in N \). Actually
  \begin{equation*}
    x - y \in K_{n + k} \cap N = K_n \cap N.
  \end{equation*}

  Since both \( y \) and \( x - y \) are in \( K_n \), so is their sum \( x \). Generalizing on \( x \), we conclude that \( K_n = K_{n + k} \).

  Therefore, the chain \eqref{eq:thm:def:noetherian_semimodule/quotient/chain} stabilizes, implying that \( M \) is noetherian.
\end{proof}

\begin{definition}\label{def:noetherian_semiring}\mcite[prop. 6.16]{Golan2010}
  We say that a (not necessarily commutative) \hyperref[def:semiring]{semiring} is \term{left noetherian} (resp. right noetherian) if it is a left (resp. right) \hyperref[def:noetherian_semimodule]{noetherian semimodule} over itself.

  Explicitly, any of the following equivalent conditions characterize a left noetherian semiring:
  \begin{thmenum}
    \thmitem{def:noetherian_semiring/acc} It satisfies the \hyperref[def:chain_condition]{ascending chain condition} on left (resp. right) ideals.
    \thmitem{def:noetherian_semiring/generated} Every left (resp. right) ideal is \hyperref[def:semiring_ideal/generated]{finitely generated}.
  \end{thmenum}
\end{definition}

\begin{proposition}\label{thm:noetherian_free_module}
  For a \hyperref[def:noetherian_semiring]{noetherian ring} \( R \), the \hyperref[def:sequence_space]{coordinate space} \( R^n \) is a \hyperref[def:noetherian_semimodule]{noetherian module}.
\end{proposition}
\begin{proof}
  We will use induction on \( n \). The cases \( n = 0 \) and \( n = 1 \) are trivial.

  Suppose that \( R^{n-1} \) is noetherian. We can identify \( R \) with the submodule of \( R^n \) generated by the vector \( (0, \ldots, 0, 1) \). Two vectors \( \seq{ x_k }_{i=1}^n \) and  \( \seq{ y_k }_{i=1}^n \) in \( R^n \) belong to this submodule if and only if \( x_k = y_k \) for \( k = 1, \ldots, n - 1 \).

  By \fullref{thm:def:ring/quotient_equality_via_difference}, these vectors get mapped to the same vector in the quotient \( R^n / R \). Then \( R^n / R \cong R^{n-1} \), which is noetherian by the inductive hypothesis. By \fullref{thm:def:noetherian_semimodule/quotient}, \( R^{n-1} \) is noetherian if and only if \( R^n \) is noetherian.

  Therefore, \( R^n \) is noetherian.
\end{proof}

\begin{lemma}\label{thm:surjective_endomorphism_over_noetherian_module}
  Every surjective endomorphism \( f: M \to M \) of a noetherian \( R \)-module \( M \) is an isomorphism.
\end{lemma}
\begin{proof}
  Consider the equation
  \begin{equation*}
    f(f(x)) = 0_M.
  \end{equation*}

  It is obviously satisfied for \( x \in \ker f \), but it is also possible that \( f(x) \neq 0_M \) while \( f(f(x)) = 0_M \). Therefore,
  \begin{equation*}
    \ker f \subseteq \ker f^2 \subseteq \ker f^3 \subseteq \cdots,
  \end{equation*}
  where \( f^k \) is \( k \)-fold iterated composition.

  Since \( M \) is noetherian, this chain stabilizes. Let \( n \) be an index such that \( \ker f^n = \ker f^{n + k} \) for every positive integer \( k \).

  Let \( y \in \ker f^n \). Since \( f \) is surjective, so is \( f^n \), and hence there exists some \( x \) be such that \( f^n(x) = y \). Then \( f^n(y) = f^n(f^n(x)) = 0_M \). But \( \ker f^n = \ker f^{2n} \), hence \( x \in \ker f^n \). Therefore, \( y = f^n(x) = 0 \).

  It follows that \( f^n \) has a trivial kernel. Then so does \( f \). By \fullref{thm:group_homomorphism_zero_kernel}, this implies that \( f \) is injective, and hence an isomorphism.
\end{proof}

\begin{proposition}\label{thm:surjective_endomorphism_in_free_module}
  Consider the \hyperref[def:sequence_space]{coordinate space} \( R^n \) for a \hyperref[def:noetherian_semiring]{noetherian ring} \( R \). If the endomorphism \( \varphi: R^n \to R^n \) is surjective, then it is also injective and hence an automorphism.
\end{proposition}
\begin{proof}
  Follows from \fullref{thm:noetherian_free_module} and \fullref{thm:surjective_endomorphism_over_noetherian_module}.
\end{proof}

\begin{theorem}[Hilbert's basis theorem]\label{thm:hilberts_basis_theorem}\mcite[thm. 7.4]{КоцевСидеров2016}
  If \( R \) is a \hyperref[def:noetherian_semiring]{noetherian commutative ring}, then so is \( R[X] \).
\end{theorem}
\begin{proof}
  Let \( I \subseteq R[X] \) be an arbitrary ideal. We will prove that \( I \) is finitely generated.

  Denote by \( L \) the set of all leading coefficients of polynomials in \( I \). The leading coefficient of the product \( p(X) q(X) \) of univariate polynomials is the product of their leading coefficients, hence \( L \) is an ideal as a consequence of \( I \) being an ideal.

  As a consequence of \( R \) being noetherian, \( L \) is finitely generated. Suppose that \( L = \set{ l_1, \ldots, l_n } \).

  For every generator \( l_k \), there exists a polynomial \( p_k(X) \) in \( I \) whose leading coefficient is \( l_k \). Denote by \( d_k \) the degree of \( p_k \) and let \( d \) be the maximum of the degrees. We will show that \( I \) itself is equal to the sum of the finitely generated ideals
  \begin{equation*}
    J \coloneqq \underbrace{ \braket{ p_1, \ldots, p_n } + \braket{ X, X^2, \ldots, X^d } }_{ \braket{ p_1, \ldots, p_n, X, X^2, \ldots, X^d } }.
  \end{equation*}

  Let \( f(X) \) be some polynomial from \( I \) and let \( l X^m \) be its leading coefficient.

  We proceed by induction on \( m \) to show that \( f(X) \) belongs to \( J \).
  \begin{itemize}
    \item If \( m \leq d \), then \( f(X) \) belongs to the second ideal \( \braket{ X, X^2, \ldots, X^d } \).

    \item Suppose that \( m > d \) and that every polynomial in \( I \) of degree less than \( m \) belongs to \( J \).

    Since \( l \in L \), it is a linear combination \( l = \sum_{k=1}^n t_k l_k \) with coefficients in \( R \). Consider the polynomial
    \begin{equation*}
      p(X) \coloneqq \sum_{k=1}^n t_k p_k(X) X^{m - d_k}.
    \end{equation*}

    Define \( r(X) \coloneqq f(X) - p(X) \). Since \( p(X) \) belongs to \( I \), \( r(X) \) does too. It is a polynomial in \( I \) of degree less than \( m \), hence it belongs to \( J \). Then
    \begin{equation*}
      f(X) = \underbrace{ p(X) }_{\mathclap{\braket{ p_1(X) \cdots, p_n(X) }}} + \overbrace{ r(X) }^J.
    \end{equation*}

    Hence, \( f(X) \in J \).
  \end{itemize}

  Our choice of polynomial \( f(X) \in I \) was arbitrary. Therefore,
  \begin{equation*}
    I \subseteq \braket{ p_1, \ldots, p_n } + \braket{ X, X^2, \ldots, X^d } \subseteq I,
  \end{equation*}
  demonstrating that \( I \) is finitely generated.
\end{proof}

\paragraph{Algebraic dependence}

\begin{definition}\label{def:algebraic_dependence}\mimprovised
  Let \( M \) be an \hyperref[def:algebra_over_ring]{algebra} over a \hyperref[def:ring/commutative]{commutative ring} \( R \). Fix some indexed set \( \seq{ u_e }_{e \in E} \) from \( M \) and consider the \hyperref[def:polynomial_algebra]{polynomial algebra} \( R[X_e \given e \in E] \) over some fixed indeterminates.

  We say that the elements of \( E \) are \term{algebraically independent} if any of the following conditions hold:

  \begin{thmenum}
    \thmitem{def:algebraic_dependence/direct} If \( \seq{ u_e }_{e \in E} \) is a root of some polynomial \( p(X_e \given e \in E) \) with coefficients in \( R \), then \( p \) is the zero polynomial.

    \thmitem{def:algebraic_dependence/evaluation} The \hyperref[thm:polynomial_algebra_universal_property]{evaluation map} \( \Phi_u: R[X_e \given e \in E] \to M \) sending \( X_e \) to \( u_e \) is injective.
  \end{thmenum}

  Unsurprisingly, if the elements of \( E \) are not \term{algebraically independent}, we say that they are \term{algebraically dependent}.
\end{definition}
\begin{comments}
  \item Compare this concept to linear dependence defined in \fullref{def:linear_dependence}.
\end{comments}
\begin{defproof}
  \ImplicationSubProof{def:algebraic_dependence/direct}{def:algebraic_dependence/evaluation} Suppose that \( \Phi_e \) is injective and that there exists a polynomial \( p(X_e \given e \in E) \) such that \( \Phi_e(p) = 0_M \).

  For any other polynomial \( q(X_e \given e \in E) \), we have \( \Phi_e(p q) = 0_M \), and hence either \( p \) is the zero polynomial or the evaluation map is not injective. We have assumed that it is injective, hence \( p \) is the zero polynomial.

  \ImplicationSubProof{def:algebraic_dependence/evaluation}{def:algebraic_dependence/direct} Conversely, suppose that \( E \) is a root only of the zero polynomial. Let \( \Phi_e(p) = \Phi_e(q) \). Then \( \seq{ u_e }_{e \in E} \) is a root of \( p - q \) and hence the latter is the zero polynomial. But this implies that \( p = q \). Hence, the evaluation map is injective.
\end{defproof}

\begin{proposition}\label{thm:def:algebraic_dependence}
  \hyperref[def:algebraic_dependence]{Algebraic (in)dependence} for the \hyperref[def:ring/commutative]{commutative ring} \( R \) has the following basic properties:
  \begin{thmenum}
    \thmitem{thm:def:algebraic_dependence/element} Every nonzero element of \( R \) is algebraically dependent over \( R \).

    \thmitem{thm:def:algebraic_dependence/n_independent} Different indeterminates are algebraically independent over \( R \).

    \thmitem{thm:def:algebraic_dependence/two_univariate_dependent}\mcite{MathOF:univariate_polynomials_algebraically_dependent} Every two univariate polynomials in \( R \) are algebraically dependent over \( R \).

    \thmitem{thm:def:algebraic_dependence/n_plus_one_dependent} Every \( n + 1 \) polynomials in \( R[X_1, \ldots, X_n] \) are algebraically dependent over \( R \).
  \end{thmenum}
\end{proposition}
\begin{proof}
  \SubProofOf{thm:def:algebraic_dependence/element} Every nonzero element \( x \) is a root of the univariate polynomial \( X - x \).

  \SubProofOf{thm:def:algebraic_dependence/n_independent} Fix some indeterminates \( X_1, \ldots, X_n \). For a nonzero polynomial \( f(Y_1, \ldots, Y_n) \), the evaluation \( \Phi_{X_1, \dots, X_n}(f) \) is zero if and only if \( f \) is zero because the evaluation simply renames the variables.

  \SubProofOf{thm:def:algebraic_dependence/two_univariate_dependent} Fix polynomials \( p(X) \) and \( q(X) \) over \( R \). We will construct a polynomial \( f(Y, Z) \) over \( R \) such that \( \Phi_{p,q}(f) = 0 \).

  If \( p(X) \) is zero, simply define \( f(Y, Z) \coloneqq Z \). If \( q(X) \) is zero, put \( f(Y, Z) \coloneqq Y \).

  Suppose that both are nonzero; denote by \( n \) be the degree of \( p(X) \) and by \( m \) the degree of \( q(X) \). We will consider polynomials of the form \( p^l q^k \).

  Fix a positive integer \( d \). We want the degree of \( p^l q^k \) to be at most \( d \). If
  \begin{align*}
    l < \frac d {2n} && k < \frac d {2m},
  \end{align*}
  then, by \fullref{thm:def:polynomial_degree/product}, either \( p^l q^k \) is the zero polynomial or
  \begin{equation*}
    \deg(p^l q^k) = nl + km < \frac d 2 + \frac d 2 = d.
  \end{equation*}

  These polynomials are all in
  \begin{equation*}
    L_d \coloneqq \linspan\set{ 1, X, X^2, X^3, \ldots, X^{d-1} }.
  \end{equation*}

  This is a module of \hyperref[thm:commutative_module_rank]{rank} \( d \).

  Furthermore, there are \( \ifrac* {d^2} {4nm} \) such polynomials. If \( d > 4nm \), there are more polynomials of the form \( p^l q^k \) than the \hyperref[thm:commutative_module_rank]{rank} of \( L_d \). Hence, every \( d + 1 \) such polynomials are linearly dependent, and hence there exists some linear combination
  \begin{equation*}
    a_1 p^{l_1} q^{k_1} + \cdots + a_{d+1} p^{l_{d+1}} q^{k_{d+1}} = 0.
  \end{equation*}

  We can thus define the following polynomial in \( R[Y, Z] \):
  \begin{equation*}
    f(Y, Z) \coloneqq a_1 Y^{l_1} Z^{k_1} + \cdots + a_{d+1} Y^{l_{d+1}} Z^{k_{d+1}}.
  \end{equation*}

  Then clearly \( \Phi_{p,q}(f) = 0 \), so \( p \) and \( q \) are algebraically dependent over \( R \).

  \SubProofOf{thm:def:algebraic_dependence/n_plus_one_dependent} Let \( p_1, \ldots, p_{n+1} \) be polynomials in \( R[X_1, \ldots, X_{n-1}][X_n] \). By \fullref{thm:def:algebraic_dependence/two_univariate_dependent}, the polynomials \( p_n \) and \( p_{n+1} \) are algebraically dependent over \( R[X_1, \ldots, X_{n-1}] \).

  Let \( f(Y_n, Y_{n+1}) \) be a polynomial in \( R[X_1, \ldots, X_{n-1}][Y_n, Y_{n+1}] \) such that \( \Phi_{p_n,p_{n+1}}(f) = 0 \). The coefficients of \( f \) are themselves polynomials. Let
  \begin{equation*}
    \widehat{f}(Y_1, \ldots, Y_{n-1}, Y_n, Y_{n+1})
  \end{equation*}
  be the polynomial obtained from
  \begin{align*}
    f(X_1, \ldots, X_{n-1}, Y_n, Y_{n+1})
  \end{align*}
  by renaming the corresponding variables.

  Then \( \Phi_{p_1,\ldots,p_{n+1}}(\widehat{f}) = 0 \). Therefore, \( p_1, \ldots, p_{n+1} \) are algebraically dependent over \( R \).
\end{proof}

\begin{proposition}\label{thm:change_of_polynomial_basis}
  Let \( R \) be a \hyperref[def:ring/commutative]{commutative ring}, let \( X_1, \ldots, X_n \) be arbitrary symbols and consider some polynomials \( Y_k(X_1, \ldots, X_n) \), \( k = 1, \ldots, n \) from \( R[X_1, \ldots, X_n] \) that are \hyperref[def:algebraic_dependence]{algebraically independent}.

  Then the \hyperref[thm:polynomial_algebra_universal_property]{evaluation map} \( \Phi: R[X_1, \ldots, X_n] \to R[X_1, \ldots, X_n][Y_1, \ldots, Y_n] \), given by
  \begin{equation*}
    X_k \mapsto Y_k(X_1, \ldots, X_n),
  \end{equation*}
  is an isomorphic embedding of \( R \)-algebras.
\end{proposition}
\begin{comments}
  \item A polynomial in the image of \( \Phi \) does not explicitly contain any of \( X_1, \ldots, X_n \), hence we can regard it as a polynomial in the indeterminates \( Y_1, \ldots, Y_n \) and then regard \( \Phi \) as an isomorphism between \( R[X_1, \ldots, X_n] \) and \( R[Y_1, \ldots, Y_n] \).

  \item This is a generalization of the \hyperref[rem:change_of_basis]{change of basis} of linear operators to polynomials.
\end{comments}
\begin{proof}
  The evaluation map is a homomorphism of \( R \)-algebras, hence it remains only to show that it is injective. Suppose that \( \Phi(p) = \Phi(q) \) for some polynomials \( p \) and \( q \) from \( R[X_1, \ldots, X_n] \).

  Then \( \Phi(p - q) \) is the zero polynomial. Since \( Y_1, \ldots, Y_n \) are algebraically independent, it follows that \( p - q \) is the zero polynomial, that is, \( p = q \).
\end{proof}
