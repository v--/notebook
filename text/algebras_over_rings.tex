\section{Algebras over rings}\label{sec:algebras_over_rings}

\paragraph{Algebras over rings}

\begin{definition}\label{def:algebra_over_ring}\mcite[15]{Kaplansky1974CommutativeRings}
  An \term[bg=алгебра (\cite[4]{КоцевСидеров2016КомутативнаАлгебра}), ru=алгебра (\cite[def. 1.7.1]{Винберг2014КурсАлгебры}), en=commutative algebra (\cite[28]{Eisenbud1995CommutativeAlgebra})]{algebra} over a commutative ring \( R \) rather than over a \hyperref[def:algebra_over_semiring]{semiring} exhibits some more interesting metamathematical properties.

  \begin{thmenum}
    \thmitem{def:algebra_over_ring/theory}\mimprovised The first-order theory is identical to the \hyperref[def:algebra_over_semiring/theory]{theory of algebras over semimodules}.

    \thmitem{def:algebra_over_ring/homomorphism}\mcite[28]{Eisenbud1995CommutativeAlgebra} A \hyperref[def:first_order_homomorphism]{first-order homomorphism} between two \( R \)-algebras is a \hyperref[def:linear_function]{linear map} that preserves multiplication. This is the same as for semirings.

    \thmitem{def:algebra_over_ring/submodel}\mcite[28]{Eisenbud1995CommutativeAlgebra}) The set \( A \subseteq M \) is a \hyperref[def:first_order_submodel]{submodel} of \( M \) if it is a \hyperref[def:monoid/submodel]{submodule} of \( M \) that contains \( 1 \) and is closed under algebra multiplication. We say that \( A \) is an \( R \)-\term{subalgebra} of \( M \).

    If \( A \) does not contain \( 1 \), we may instead refer to nonunital \( R \)-subalgebras. We will use them for quotients.

    As a consequence of \fullref{thm:positive_formulas_preserved_under_homomorphism}, the image of an \( R \)-algebra homomorphism is an \( R \)-subalgebra of its codomain.

    \thmitem{def:algebra_over_ring/category}\mimprovised For a fixed ring \( R \), we denote the \hyperref[def:category_of_small_first_order_models]{category of \( \mscrU \)-small models} by \( \ucat{Alg}_R \). It is concrete with respect to both \( \ucat{CRing} \) and \( \ucat{Mod}_R \).

    Unfortunately, these categories are not as well-behaved as categories of modules. Similarly to rings, unital and nonunital algebras behave differently.

    \thmitem{def:algebra_over_ring/trivial}\mimprovised Similarly to rings, \enquote{the} \hyperref[def:trivial_object]{trivial object} is the one-element algebra \( \set{ 0 } \).

    \thmitem{def:algebra_over_ring/kernel}\mimprovised \Fullref{thm:ring_zero_morphisms/kernel} implies that the \hyperref[def:zero_morphisms/kernel]{categorical kernel} of a homomorphism \( \varphi: M \to N \) between \hi{nonunital} \( R \)-algebras is the additive group kernel
    \begin{equation*}
      \ker \varphi \coloneqq \varphi^{-1}(0_N) = \set{ x \in M \given \varphi(x) = 0_N }.
    \end{equation*}

    The kernel is a both a two-sided ideal of \( M \) as a consequence of \fullref{thm:kernel_is_ideal} and a submodule of \( M \) as a consequence of \fullref{thm:kernel_is_submodule}.

    \thmitem{def:algebra_over_ring/quotient}\mimprovised Similarly to rings, we define quotient \( R \)-algebras of \( M \) by nonunital \( R \)-subalgebras. In particular, \fullref{thm:algebra_ideal_is_subalgebra} implies that we can take the quotient by any ideal of \( M \).

    \thmitem{def:algebra_over_ring/commutative}\mimprovised As in the case of algebras over semirings, by \enquote{\( M \) is commutative}, we will mean that vector multiplication is commutative.

    We denote the subcategory of commutative algebras by \( \cat{CAlg}_R \).
  \end{thmenum}
\end{definition}
\begin{comments}
  \item We adopt Eisenbud's definitions for submodules and homomorphisms from \cite[28]{Eisenbud1995CommutativeAlgebra}, however for the main definition we stick to Kaplansky since he allows vector products that are not commutative.
\end{comments}

\begin{proposition}\label{thm:algebra_ideal_is_subalgebra}
  Every \hyperref[def:semiring_ideal]{left ideal} of an \( R \)-\hyperref[def:algebra_over_ring]{algebra} is a nonunital \( R \)-\hyperref[def:algebra_over_ring/submodel]{subalgebra}.
\end{proposition}
\begin{proof}
  Let \( I \) be a left ideal of the \( R \)-algebra \( M \). By definition of left ideal, \( I \) is a left \( M \)-submodule of \( M \), and, because \( R \) is a subring of \( M \), \( I \) is a left \( R \)-submodule of \( M \).

  Furthermore, \eqref{eq:def:semiring_ideal/direct/multiplicative} implies that \( I \) is closed under vector multiplication with arbitrary elements of \( M \), hence \( I \) is also an \( R \)-subalgebra.
\end{proof}

\begin{proposition}\label{thm:ring_is_integer_algebra}
  The categories \( \hyperref[def:ring/category]{\cat{Ring}} \) of rings and \( \hyperref[def:algebra_over_ring/category]{\cat{Alg}_\BbbZ} \) of integer algebras are \hyperref[rem:category_similarity/isomorphism]{isomorphic}.
\end{proposition}
\begin{comments}
  \item Compare this result to \fullref{thm:abelian_group_is_module} for modules and \fullref{thm:semiring_is_natural_number_algebra} for algebras over semirings.
\end{comments}
\begin{proof}
  Follows from \fullref{thm:semiring_is_natural_number_algebra}.
\end{proof}

\paragraph{Quotients of polynomial algebras by principal ideals}

\begin{proposition}\label{thm:representatives_in_univariate_polynomial_quotient_set}
  Fix a \hyperref[def:monic_polynomial]{monic polynomial} \( f(X) \) in a nontrivial commutative ring \( R \).

  Every coset in \( R[X] / \braket{ f(X) } \) has a unique representative that is either zero or has degree less than that of \( f(X) \).
\end{proposition}
\begin{proof}
  Let \( g(X) \) be an arbitrary polynomial. \Fullref{alg:euclidean_division_of_polynomials} gives us polynomials \( q(X) \) and \( r(X) \) such that
  \begin{equation*}
    g(X) = f(X) q(X) + r(X),
  \end{equation*}
  where \( r(X) \) is either zero or has degree less than that of \( f(X) \).

  Multiples of \( q(X) \) are congruent to \( 0_R \) modulo the ideal \( \braket{ q(X) } \), hence \( g(X) \) is congruent to \( r(X) \).

  By the uniqueness of \( r(X) \), the statement of the corollary follows.
\end{proof}

\begin{corollary}\label{thm:polynomial_quotient_modules_vs_algebras}
  For two nonzero monic polynomials \( f(X) \) and \( g(X) \), the degrees of \( f(X) \) and \( g(X) \) coincide if and only if the \hyperref[def:algebra_over_ring/quotient]{quotient algebras} \( R[X] / \braket{ f(X) } \) and \( R[X] / \braket{ g(X) } \) are isomorphic as \( R \)-modules.
\end{corollary}
\begin{comments}
  \item As shown in \fullref{ex:gaussian_integers} and \fullref{ex:integers_with_sqrt2}, the vector multiplication operation on the quotients may differ --- the quotients may be isomorphic as \( R \)-modules, but not as \( R \)-algebras.
\end{comments}
\begin{proof}
  By \fullref{thm:representatives_in_univariate_polynomial_quotient_set}, for every coset in the quotient, \fullref{alg:euclidean_division_of_polynomials} gives us a unique representative of the corresponding degree. Addition and scalar multiplication must be the same in both.
\end{proof}

\begin{corollary}\label{thm:polynomial_quotient_module_dimension}
  The quotient of the polynomial algebra \( R[X] \) by the principal ideal \( \braket{ f(X) } \), where \( f(X) \) is monic, has as \hyperref[def:module_rank]{module rank} the degree of \( f \).
\end{corollary}
\begin{proof}
  Follows from \fullref{thm:polynomial_quotient_module_dimension} by noting that \( 1, X, X^2, \ldots, X^{\deg f-1} \) is a basis.
\end{proof}

\begin{proposition}\label{thm:adjoint_roots_and_quotients}
  Fix arbitrary commutative rings \( R \subseteq S \) and some element \( u \) of \( S \). Consider the \hyperref[con:evaluation_homomorphism]{evaluation homomorphism} \( \Phi_u: R[X] \to S \) sending \( X \) to \( u \). Additionally suppose that the kernel of \( \Phi_u \) is principal with generator \( f(X) \).

  Then the quotient ring \( R[X] / \braket{ f(X) } \) is isomorphic to the ring \( R[u] \) obtained by \hyperref[def:semiring_adjunction]{adjoining} \( u \) to \( R \).
\end{proposition}
\begin{comments}
  \item This is useful when \( R \) is a \hyperref[def:field]{field} and thus \( R[X] \) is a \hyperref[def:principal_ideal_domain]{principal ideal domain}. See the equivalences in \fullref{def:algebraic_element}.
\end{comments}
\begin{proof}
  \Fullref{thm:ring_zero_morphisms/isomorphism} implies that
  \begin{equation*}
    R[X] / \underbrace{\ker \Phi_u}_{\braket{ f(X) }} \cong \underbrace{\img \Phi_u}_{R[u]}.
  \end{equation*}
\end{proof}

\begin{definition}\label{def:gaussian_integers}\mcite[\S V.6.2]{Aluffi2009Algebra}
  A \term[ru=целые гауссовые числа (\cite[example 3.5.1]{Винберг2014КурсАлгебры})]{Gaussian integer} is a \hyperref[def:complex_numbers]{complex number} whose real and imaginary part are \hyperref[def:integers]{integers}.
\end{definition}

\begin{example}\label{ex:gaussian_integers}
  We can define several isomorphic rings for the \hyperref[def:gaussian_integers]{Gaussian integers}, demonstrating \fullref{thm:adjoint_roots_and_quotients}.

  We assume that the field of complex numbers is available to us.

  \begin{thmenum}
    \thmitem{ex:gaussian_integers/quotient} Analogous to our discussion in \fullref{def:complex_numbers}, we can take the \hyperref[def:algebra_over_ring/quotient]{quotient algebra} \( \BbbZ[X] / \braket{X^2 + 1} \).

    \thmitem{ex:gaussian_integers/evaluation} We can also \hyperref[def:semiring_adjunction]{adjoin} \( i \) to \( \BbbZ \) to obtain the ring \( \BbbZ[i] \).

    Given a Gaussian integer \( z = a + bi \), it corresponds to the polynomial
    \begin{equation*}
      f_z(X) \coloneqq a + bX.
    \end{equation*}

    Conversely, consider the \hyperref[con:evaluation_homomorphism]{evaluation homomorphism} \( \Phi_i: \BbbZ[X] \to \BbbC \) for the imaginary unit. Let \( f(X) \in \BbbZ[X] \). Then
    \begin{equation*}
      f(i)
      =
      \Phi_i(f)
      =
      \sum_{k=0}^n a_k i^n
      =
      \thickspace \sum_{\scriptscriptstyle{\mathclap{\rem(k, 4) = 0}}}^n a_k - \sum_{\scriptscriptstyle{\mathclap{\rem(k, 4) = 2}}}^n a_k + i \parens[\Bigg]{ \quad \sum_{\scriptscriptstyle{\mathclap{\rem(k, 4) = 1}}}^n a_k - \sum_{\scriptscriptstyle{\mathclap{\rem(k, 4) = 3}}}^n a_k }.
    \end{equation*}

    This is clearly again a Gaussian integer.

    Thus, although we skipped proving that \( \braket{X^2 + 1} \) is the kernel of \( \Phi_i: \BbbZ[X] \to \BbbC \), we have shown that the algebras \( \BbbZ[X] / \braket{ X^2 + 1 } \) and \( \BbbZ[i] \) behave identically.
  \end{thmenum}
\end{example}

\begin{example}\label{ex:integers_mod_xx_minus_1}
  In \fullref{ex:gaussian_integers}, we discussed how the ring of \hyperref[def:gaussian_integers]{Gaussian integers} equals the quotient
  \begin{equation*}
    \BbbZ[X] / \braket{ X^2 + 1 }.
  \end{equation*}

  If we instead take
  \begin{equation*}
    \BbbZ[X] / \braket{ X^2 - 1 },
  \end{equation*}
  multiplication of \( a + bX \) and \( c + dX \) would behave as
  \begin{equation*}
    (a + bX) (c + dX) = (ac + bd) + (bc + ad)X.
  \end{equation*}

  This is less useful since \( -1 \) and \( 1 \) are already roots of \( X^2 - 1 \), and we adjoin a new root. Nevertheless, the example shows that polynomial quotient algebras are constructed similarly when the polynomial is not irreducible.
\end{example}

\begin{example}\label{ex:integers_with_sqrt2}
  Similarly to \fullref{ex:gaussian_integers}, we have
  \begin{equation*}
    \BbbZ[X] / \braket{X^2 - 2} \cong \BbbZ[\sqrt 2].
  \end{equation*}

  The gist of this example is that, even though \( \BbbZ[\sqrt 2] \) and \( \BbbZ[i] \) are isomorphic as modules, their vector multiplication operation is different. Indeed, since \( X^2 = 2 \), we have
  \begin{align*}
    (a + bX) (c + dX) = (ac + 2bd) + (bc + ad)X
  \end{align*}
  which is different compared to complex multiplication.
\end{example}

\paragraph{Noetherian algebras}

\begin{definition}\label{def:noetherian_semimodule}\mcite[prop. 6.16]{Golan1999Semirings}
  We say that an \( R \)-\hyperref[def:semimodule]{semimodule} is \term[bg=ньотеров (\cite[41]{КоцевСидеров2016КомутативнаАлгебра}), ru=нётеровый (\cite[def. 9.4.1]{Винберг2014КурсАлгебры})]{noetherian} if any of the following equivalent conditions hold:
  \begin{thmenum}
    \thmitem{def:noetherian_semimodule/acc} It satisfies the \hyperref[def:chain_condition]{ascending chain condition} on \( R \)-sub-semimodules.

    \thmitem{def:noetherian_semimodule/generated} Every \( R \)-sub-semimodule is \hyperref[def:semimodule/generated]{finitely generated}, i.e. is the \hyperref[def:semimodule/submodel]{linear span} of finitely many elements.
  \end{thmenum}
\end{definition}
\begin{defproof}
  Fix an \( R \)-semimodule \( M \).

  \ImplicationSubProof{def:noetherian_semimodule/acc}{def:noetherian_semimodule/generated} Suppose that \( M \) satisfies \fullref{def:chain_condition/maximal}, i.e. every nonempty family of \( R \)-sub-semimodules of \( M \) has a maximal element.

  Let \( K \coloneqq \linspan{ x_1, \ldots, x_n } \) be maximal in the family of all finitely generated \( R \)-sub-semimodules. Let \( N \) be any sub-semimodule. Adding a particular element from \( N \) does not change \( K \), because otherwise it would not be maximal. Thus, \( N \subseteq K \).

  \ImplicationSubProof{def:noetherian_semimodule/generated}{def:noetherian_semimodule/acc} Suppose that every \( R \)-sub-semimodule is finitely generated.

  Consider the ascending sequence of \( R \)-sub-semimodules
  \begin{equation}\label{eq:def:noetherian_semimodule/chain}
    N_1 \subseteq N_2 \subseteq N_3 \subseteq \cdots
  \end{equation}

  By \fullref{thm:def:semimodule/union}, their union \( \bigcup_{k \in \mscrK} N_k \) is also an \( R \)-sub-semimodule.

  Let \( x_1, \ldots, x_n \) be the set of generators for the union. Let \( k_m \) be the index of the first sub-semimodule that contains \( x_m \). Every next sub-semimodule in the sequence contains the previous, hence \( k \leq k_m \) implies that \( x_k \in N_{k_m} \).

  Let \( k_{m_0} \) be a maximal index. Then \( N_{k_{m_0}} \) contains all the generators, and hence it coincides with the union \( \bigcup_{k \in \mscrK} N_k \). Every sub-semimodule with a greater index is simply equal to the previous.

  Therefore, the sequence \eqref{eq:def:noetherian_semimodule/chain} stabilizes.
\end{defproof}

\begin{proposition}\label{thm:def:noetherian_semimodule}
  \hyperref[def:noetherian_semimodule]{Noetherian modules} over an arbitrary ring \( R \) have the following basic properties:
  \begin{thmenum}
    \thmitem{thm:def:noetherian_semimodule/submodule} If \( M \) is noetherian, then every \( R \)-submodule of \( M \) also is.
    \thmitem{thm:def:noetherian_semimodule/quotient}\mcite[prop. 6.3b)]{КоцевСидеров2016КомутативнаАлгебра} Let \( N \) be an \( R \)-\hyperref[def:module/submodel]{submodule} of \( M \). Then \( M \) is noetherian if and only if both \( N \) and their \hyperref[def:module/quotient]{quotient} \( M / N \) are.
  \end{thmenum}
\end{proposition}
\begin{proof}
  \SubProofOf{thm:def:noetherian_semimodule/submodule} Trivial.
  \SubProofOf{thm:def:noetherian_semimodule/quotient} By \fullref{thm:lattice_theorem_for_submodules}, every sequence of \( R \)-submodules of \( M / N \) corresponds to a sequence of \( R \)-submodules in \( M \). Thus, if \( M \) is noetherian, clearly \( M / N \) also is.

  Conversely, suppose that both \( N \) and \( M / N \) are noetherian. Let
  \begin{equation}\label{eq:thm:def:noetherian_semimodule/quotient/chain}
    K_1 \subseteq K_2 \subseteq K_3 \subseteq \cdots
  \end{equation}
  be an ascending sequence of \( R \)-submodule of \( M \). Then
  \begin{equation}\label{eq:thm:def:noetherian_semimodule/quotient/chain/intersection}
    K_1 \cap N \subseteq K_2 \cap N \subseteq K_3 \cap N \subseteq \cdots
  \end{equation}
  is an ascending sequence of \( R \)-submodules of \( N \) and
  \begin{equation}\label{eq:thm:def:noetherian_semimodule/quotient/chain/quotient}
    (K_1 + N) / N \subseteq (K_2 + N) / N \subseteq (K_3 + N) / N \subseteq \cdots
  \end{equation}
  is an ascending sequence of \( R \)-submodule of \( M / N \).

  Both \eqref{eq:thm:def:noetherian_semimodule/quotient/chain/intersection} and \eqref{eq:thm:def:noetherian_semimodule/quotient/chain/quotient} stabilize. Let \( n \) be an index such that, for every positive integer \( k \), \( K_n \cap N = K_{n + k} \cap N \) and \( (K_n + N) / N = (K_{n + k} + N) / N \). For a fixed \( k \), we will show that \( K_n = K_{n + k} \).

  Let \( x \in K_{n + k} \). If \( x \in N \), then \( x \in K_n \) since \( K_n \cap N = K_{n + k} \cap N \). Suppose that \( x \in K_{n + k} \setminus N \). For any \( n \in N \), we have \( x + n \in K_{n + k} + N \), and hence
  \begin{equation*}
    x + n + N = x + N \in (K_{n + k} + N) / N = (K_n + N) / N.
  \end{equation*}

  Then there exists some \( y \in K_n \) such that \( x - y \in N \). Actually
  \begin{equation*}
    x - y \in K_{n + k} \cap N = K_n \cap N.
  \end{equation*}

  Since both \( y \) and \( x - y \) are in \( K_n \), so is their sum \( x \). Generalizing on \( x \), we conclude that \( K_n = K_{n + k} \).

  Therefore, the sequence \eqref{eq:thm:def:noetherian_semimodule/quotient/chain} stabilizes, implying that \( M \) is noetherian.
\end{proof}

\begin{definition}\label{def:noetherian_semiring}\mcite[prop. 6.16]{Golan1999Semirings}
  We say that a (not necessarily commutative) \hyperref[def:semiring]{semiring} is \term{left noetherian} (resp. right noetherian) if it is a left (resp. right) \hyperref[def:noetherian_semimodule]{noetherian semimodule} over itself.

  Explicitly, any of the following equivalent conditions characterize a left noetherian semiring:
  \begin{thmenum}
    \thmitem{def:noetherian_semiring/acc} It satisfies the \hyperref[def:chain_condition]{ascending chain condition} on left (resp. right) ideals.
    \thmitem{def:noetherian_semiring/generated} Every left (resp. right) ideal is \hyperref[def:semiring_ideal/generated]{finitely generated}.
  \end{thmenum}
\end{definition}

\begin{proposition}\label{thm:noetherian_free_module}
  For a \hyperref[def:noetherian_semiring]{noetherian ring} \( R \), the \hyperref[def:coordinate_space]{coordinate space} \( R^n \) is a \hyperref[def:noetherian_semimodule]{noetherian module}.
\end{proposition}
\begin{proof}
  We will use induction on \( n \). The cases \( n = 0 \) and \( n = 1 \) are trivial.

  Suppose that \( R^{n-1} \) is noetherian. We can identify \( R \) with the submodule of \( R^n \) generated by the vector \( (0, \ldots, 0, 1) \). Two vectors \( \seq{ x_k }_{i=1}^n \) and  \( \seq{ y_k }_{i=1}^n \) in \( R^n \) belong to this submodule if and only if \( x_k = y_k \) for \( k = 1, \ldots, n - 1 \).

  By \fullref{thm:def:ring/quotient_equality_via_difference}, these vectors get mapped to the same vector in the quotient \( R^n / R \). Then \( R^n / R \cong R^{n-1} \), which is noetherian by the inductive hypothesis. By \fullref{thm:def:noetherian_semimodule/quotient}, \( R^{n-1} \) is noetherian if and only if \( R^n \) is noetherian.

  Therefore, \( R^n \) is noetherian.
\end{proof}

\begin{lemma}\label{thm:surjective_endomorphism_over_noetherian_module}
  Every surjective endomorphism \( f: M \to M \) of a noetherian \( R \)-module \( M \) is an isomorphism.
\end{lemma}
\begin{proof}
  Consider the equation
  \begin{equation*}
    f(f(x)) = 0_M.
  \end{equation*}

  It is obviously satisfied for \( x \in \ker f \), but it is also possible that \( f(x) \neq 0_M \) while \( f(f(x)) = 0_M \). Therefore,
  \begin{equation*}
    \ker f \subseteq \ker f^2 \subseteq \ker f^3 \subseteq \cdots,
  \end{equation*}
  where \( f^k \) is \( k \)-fold iterated composition.

  Since \( M \) is noetherian, this sequence stabilizes. Let \( n \) be an index such that \( \ker f^n = \ker f^{n + k} \) for every positive integer \( k \).

  Let \( y \in \ker f^n \). Since \( f \) is surjective, so is \( f^n \), and hence there exists some \( x \) be such that \( f^n(x) = y \). Then \( f^n(y) = f^n(f^n(x)) = 0_M \). But \( \ker f^n = \ker f^{2n} \), hence \( x \in \ker f^n \). Therefore, \( y = f^n(x) = 0 \).

  It follows that \( f^n \) has a trivial kernel. Then so does \( f \). By \fullref{thm:group_homomorphism_trivial_kernel}, this implies that \( f \) is injective, and hence an isomorphism.
\end{proof}

\begin{proposition}\label{thm:surjective_endomorphism_in_free_module}
  Consider the \hyperref[def:coordinate_space]{coordinate space} \( R^n \) for a \hyperref[def:noetherian_semiring]{noetherian ring} \( R \). If the endomorphism \( \varphi: R^n \to R^n \) is surjective, then it is also injective and hence an automorphism.
\end{proposition}
\begin{proof}
  Follows from \fullref{thm:noetherian_free_module} and \fullref{thm:surjective_endomorphism_over_noetherian_module}.
\end{proof}

\begin{theorem}[Hilbert's basis theorem]\label{thm:hilberts_basis_theorem}\mcite[thm. 7.4]{КоцевСидеров2016КомутативнаАлгебра}
  If \( R \) is a \hyperref[def:noetherian_semiring]{noetherian} commutative ring, then so is \( R[X] \).
\end{theorem}
\begin{proof}
  Let \( I \subseteq R[X] \) be an arbitrary ideal. We will prove that \( I \) is finitely generated.

  Denote by \( L \) the set of all leading coefficients of polynomials in \( I \). \Fullref{thm:leading_coefficient_of_product} implies that the leading coefficient of the product of univariate polynomials is the product of their leading coefficients, hence \( L \) is an ideal as a consequence of \( I \) being an ideal.

  As a consequence of \( R \) being noetherian, \( L \) is finitely generated. Suppose that \( L = \set{ l_1, \ldots, l_n } \).

  For every generator \( l_k \), there exists a polynomial \( f_k(X) \) in \( I \) whose leading coefficient is \( l_k \). Denote by \( d_k \) the degree of \( f_k \) and let \( d \) be the maximum of the degrees. We will show that \( I \) itself is equal to the sum of the finitely generated ideals
  \begin{equation*}
    J \coloneqq \underbrace{ \braket{ f_1, \ldots, f_n } + \braket{ X, X^2, \ldots, X^d } }_{ \braket{ f_1, \ldots, f_n, X, X^2, \ldots, X^d } }.
  \end{equation*}

  Let \( g(X) \) be some polynomial from \( I \). Denote by \( m \) its degree and by \( l \) its leading coefficient. We proceed by induction on \( m \) to show that \( g(X) \) belongs to \( J \).
  \begin{itemize}
    \item If \( m \leq d \), then \( g(X) \) belongs to the second ideal \( \braket{ X, X^2, \ldots, X^d } \).

    \item Suppose that \( m > d \) and that every polynomial in \( I \) of degree less than \( m \) belongs to \( J \).

    Since \( l \in L \), it is a linear combination \( l = \sum_{k=1}^n t_k l_k \) with coefficients in \( R \). Consider the polynomial
    \begin{equation*}
      f(X) \coloneqq \sum_{k=1}^n t_k X^{m - d_k} \cdot f_k(X).
    \end{equation*}

    Define \( r(X) \coloneqq g(X) - f(X) \). Since \( f(X) \) belongs to \( I \), \( r(X) \) does too. The difference \( r(X) \) is a polynomial in \( I \) of degree less than \( m \), hence it also belongs to \( J \). Then
    \begin{equation*}
      g(X) = \underbrace{ f(X) }_{\mathclap{\braket{ f_1(X) \cdots, f_n(X) }}} + \overbrace{ r(X) }^J.
    \end{equation*}

    Hence, \( g(X) \) belongs to \( J \).
  \end{itemize}

  Generalizing on \( g(X) \), we conclude that
  \begin{equation*}
    I \subseteq J = \braket{ f_1, \ldots, f_n } + \braket{ X, X^2, \ldots, X^d }.
  \end{equation*}

  Therefore, \( I \) is finitely generated.
\end{proof}

\begin{example}\label{ex:countable_indeterminates_non_noetherian}
  \Fullref{thm:hilberts_basis_theorem} implies that if \( R \) is a noetherian commutative ring, so is \( R[X_1, \ldots, X_n] \). If we instead consider the polynomial algebra \( R[X_1, X_2, X_3, \ldots] \) in countably many indeterminates, it will not be noetherian because we have the following non-stabilizing sequence of ideals:
  \begin{equation*}
    \braket{ X_1 } \subseteq \braket{ X_1, X_2 } \subseteq \braket{ X_1, X_2, X_3 } \subseteq \cdots.
  \end{equation*}
\end{example}

\paragraph{Algebraic dependence}

\begin{definition}\label{def:algebraic_dependence}\mimprovised
  Let \( M \) be an \hyperref[def:algebra_over_ring]{algebra} over a \hyperref[def:ring/commutative]{commutative ring} \( R \). Fix some indexed set \( \seq{ u_e }_{e \in E} \) from \( M \) and consider the \hyperref[def:polynomial_algebra]{polynomial algebra} \( R[X_e \given e \in E] \) over some fixed indeterminates.

  We say that the elements of \( E \) are \term[ru=алгебрически независимые (елементы) (\cite[408]{Винберг2014КурсАлгебры})]{algebraically independent} if any of the following conditions hold:

  \begin{thmenum}
    \thmitem{def:algebraic_dependence/direct} If \( \seq{ u_e }_{e \in E} \) is a root of some polynomial \( f(X_e \given e \in E) \) with coefficients in \( R \), then \( f \) is the zero polynomial.

    \thmitem{def:algebraic_dependence/evaluation} The \hyperref[thm:polynomial_algebra_universal_property]{evaluation map} \( \Phi_u: R[X_e \given e \in E] \to M \) sending \( X_e \) to \( u_e \) is injective.
  \end{thmenum}
\end{definition}
\begin{comments}
  \item Unsurprisingly, if the elements of \( E \) are not \term{algebraically independent}, we say that they are \term{algebraically dependent}.
  \item Compare this concept to linear dependence defined in \fullref{def:linear_dependence}.
\end{comments}
\begin{defproof}
  \ImplicationSubProof{def:algebraic_dependence/direct}{def:algebraic_dependence/evaluation} Suppose that \( \Phi_e \) is injective and that there exists a polynomial \( f(X_e \given e \in E) \) such that \( \Phi_e(f) = 0_M \).

  For any other polynomial \( g(X_e \given e \in E) \), we have \( \Phi_e(f g) = 0_M \), and hence either \( f \) is the zero polynomial or the evaluation map is not injective. We have assumed that it is injective, hence \( f \) is the zero polynomial.

  \ImplicationSubProof{def:algebraic_dependence/evaluation}{def:algebraic_dependence/direct} Conversely, suppose that \( E \) is a root only of the zero polynomial. Let \( \Phi_e(f) = \Phi_e(g) \). Then \( \seq{ u_e }_{e \in E} \) is a root of \( f - g \) and hence the latter is the zero polynomial. But this implies that \( f = g \). Hence, the evaluation map is injective.
\end{defproof}

\begin{proposition}\label{thm:def:algebraic_dependence}
  \hyperref[def:algebraic_dependence]{Algebraic (in)dependence} for the \hyperref[def:ring/commutative]{commutative ring} \( R \) has the following basic properties:
  \begin{thmenum}
    \thmitem{thm:def:algebraic_dependence/element} Every nonzero element of \( R \) is algebraically dependent over \( R \).

    \thmitem{thm:def:algebraic_dependence/n_independent} Different indeterminates are algebraically independent over \( R \).

    \thmitem{thm:def:algebraic_dependence/two_univariate_dependent}\mcite{MathOF:univariate_polynomials_algebraically_dependent} Every two univariate polynomials in \( R \) are algebraically dependent over \( R \).

    \thmitem{thm:def:algebraic_dependence/n_plus_one_dependent} Every \( n + 1 \) polynomials in \( R[X_1, \ldots, X_n] \) are algebraically dependent over \( R \).
  \end{thmenum}
\end{proposition}
\begin{proof}
  \SubProofOf{thm:def:algebraic_dependence/element} Every nonzero element \( x \) is a root of the univariate polynomial \( X - x \).

  \SubProofOf{thm:def:algebraic_dependence/n_independent} Fix some indeterminates \( X_1, \ldots, X_n \). For a nonzero polynomial \( f(Y_1, \ldots, Y_n) \), the evaluation \( \Phi_{X_1, \dots, X_n}(f) \) is zero if and only if \( f \) is zero because the evaluation simply renames the variables.

  \SubProofOf{thm:def:algebraic_dependence/two_univariate_dependent} Fix polynomials \( p(X) \) and \( q(X) \) over \( R \). We will construct a polynomial \( f(Y, Z) \) over \( R \) such that \( \Phi_{p,q}(f) = 0 \).

  If \( p(X) \) is zero, simply define \( f(Y, Z) \coloneqq Z \). If \( q(X) \) is zero, put \( f(Y, Z) \coloneqq Y \).

  Suppose that both are nonzero; denote by \( n \) be the degree of \( p(X) \) and by \( m \) the degree of \( q(X) \). We will consider polynomials of the form \( p^l q^k \).

  Fix a positive integer \( d \). We want the degree of \( p^l q^k \) to be at most \( d \). If
  \begin{align*}
    l < \frac d {2n} && k < \frac d {2m},
  \end{align*}
  then, by \fullref{thm:polynomial_degree_arithmetic/product}, either \( p^l q^k \) is the zero polynomial or
  \begin{equation*}
    \deg(p^l q^k) = nl + km < \frac d 2 + \frac d 2 = d.
  \end{equation*}

  These polynomials are all in
  \begin{equation*}
    L_d \coloneqq \linspan\set{ 1, X, X^2, X^3, \ldots, X^{d-1} }.
  \end{equation*}

  This is a module of \hyperref[def:module_rank]{rank} \( d \).

  Furthermore, there are \( d^2 / 4nm \) such polynomials. If \( d > 4nm \), there are more polynomials of the form \( p^l q^k \) than the \hyperref[def:module_rank]{rank} of \( L_d \). Hence, every \( d + 1 \) such polynomials are linearly dependent, and hence there exists some linear combination
  \begin{equation*}
    a_1 p^{l_1} q^{k_1} + \cdots + a_{d+1} p^{l_{d+1}} q^{k_{d+1}} = 0.
  \end{equation*}

  We can thus define the following polynomial in \( R[Y, Z] \):
  \begin{equation*}
    f(Y, Z) \coloneqq a_1 Y^{l_1} Z^{k_1} + \cdots + a_{d+1} Y^{l_{d+1}} Z^{k_{d+1}}.
  \end{equation*}

  Then clearly \( \Phi_{p,q}(f) = 0 \), so \( p \) and \( q \) are algebraically dependent over \( R \).

  \SubProofOf{thm:def:algebraic_dependence/n_plus_one_dependent} Let \( p_1, \ldots, p_{n+1} \) be polynomials in \( R[X_1, \ldots, X_{n-1}][X_n] \). By \fullref{thm:def:algebraic_dependence/two_univariate_dependent}, the polynomials \( p_n \) and \( p_{n+1} \) are algebraically dependent over \( R[X_1, \ldots, X_{n-1}] \).

  Let \( f(Y_n, Y_{n+1}) \) be a polynomial in \( R[X_1, \ldots, X_{n-1}][Y_n, Y_{n+1}] \) such that \( \Phi_{p_n,p_{n+1}}(f) = 0 \). The coefficients of \( f \) are themselves polynomials. Let
  \begin{equation*}
    \widehat{f}(Y_1, \ldots, Y_{n-1}, Y_n, Y_{n+1})
  \end{equation*}
  be the polynomial obtained from
  \begin{align*}
    f(X_1, \ldots, X_{n-1}, Y_n, Y_{n+1})
  \end{align*}
  by renaming the corresponding variables.

  Then \( \Phi_{p_1,\ldots,p_{n+1}}(\widehat{f}) = 0 \). Therefore, \( p_1, \ldots, p_{n+1} \) are algebraically dependent over \( R \).
\end{proof}

\begin{proposition}\label{thm:change_of_polynomial_basis}
  Let \( R \) be a \hyperref[def:ring/commutative]{commutative ring}, let \( X_1, \ldots, X_n \) be arbitrary symbols and consider some polynomials \( Y_k(X_1, \ldots, X_n) \), \( k = 1, \ldots, n \) from \( R[X_1, \ldots, X_n] \) that are \hyperref[def:algebraic_dependence]{algebraically independent}.

  Then the \hyperref[thm:polynomial_algebra_universal_property]{evaluation map} \( \Phi: R[X_1, \ldots, X_n] \to R[X_1, \ldots, X_n][Y_1, \ldots, Y_n] \), given by
  \begin{equation*}
    X_k \mapsto Y_k(X_1, \ldots, X_n),
  \end{equation*}
  is an isomorphic embedding of \( R \)-algebras.
\end{proposition}
\begin{comments}
  \item A polynomial in the image of \( \Phi \) does not explicitly contain any of \( X_1, \ldots, X_n \), hence we can regard it as a polynomial in the indeterminates \( Y_1, \ldots, Y_n \) and then regard \( \Phi \) as an isomorphism between \( R[X_1, \ldots, X_n] \) and \( R[Y_1, \ldots, Y_n] \).

  \item This is a generalization of the \hyperref[con:change_of_basis]{change of basis} of linear operators to polynomials.
\end{comments}
\begin{proof}
  The evaluation map is a homomorphism of \( R \)-algebras, hence it remains only to show that it is injective. Suppose that \( \Phi(f) = \Phi(g) \) for some polynomials \( f \) and \( g \) from \( R[X_1, \ldots, X_n] \).

  Then \( \Phi(f - g) \) is the zero polynomial. Since \( Y_1, \ldots, Y_n \) are algebraically independent, it follows that \( f - g \) is the zero polynomial, that is, \( f = g \).
\end{proof}

\paragraph{Quaternions}

We have defined the field \( \BbbC \) of \hyperref[def:complex_numbers]{complex numbers} as the \hyperref[def:algebra_over_ring/quotient]{quotient algebra} \( \BbbR[X] / \braket{ X^2 - 1 } \). Furthermore, as discussed in \fullref{rem:real_field_extensions}, there are no nontrivial finite-dimensional field extensions of \( \BbbC \).

Nevertheless, we can generalize the construction by instead considering \hyperref[def:noncommutative_polynomial_algebra]{noncommutative polynomials}.

\begin{definition}\label{def:quaternion_algebra}\mimprovised
  We define the \term[bg=кватерниони (\cite[454]{Обрешков1962ВисшаАлгебра}), ru=алгебра кватернионов (\cite[41]{Винберг2014КурсАлгебры}), en=quaternion algebra (\cite[exerc. III.1.12]{Aluffi2009Algebra})]{quaternion algebra} \( \BbbH \) as the \hyperref[def:algebra_over_ring/quotient]{quotient} of the \hyperref[def:noncommutative_polynomial_algebra]{noncommutative polynomial algebra} \( \BbbR\braket{ i, j, k } \) by the congruence \hyperref[def:first_order_generated_congruence]{generated} by the following relation:
  \begin{align*}
    i^2 \sim -1, && jk \sim i, && (kj \sim -i), \\
    j^2 \sim -1, && ki \sim j, && (ik \sim -j), \\
    k^2 \sim -1, && ij \sim k, && (ji \sim -k).
  \end{align*}

  \begin{figure}[!ht]
    \centering
    \includegraphics[page=1]{output/def__quaternions}
    \caption{An illustration of the congruence in \fullref{def:quaternion_algebra} --- by multiplying the head of an arc by its tail, we obtain the third vertex; by multiplying the tail by the head, we instead obtain the negation.}\label{fig:def:quaternion_algebra}
  \end{figure}

  Since every possible pair of symbols can now be reduced to a single one, we are left with only four monomials, and we can denote a quaternion as a (real) linear combination:
  \begin{equation*}
    a + bi + cj + dk.
  \end{equation*}
\end{definition}
\begin{comments}
  \item The last column defining the congruence is redundant because
  \begin{equation*}
    kj \sim (-1)^2 kj \sim (-1) j^2 (kj) \sim -j (jk) j \sim -j (ij) \sim -jk \sim -i
  \end{equation*}
  and analogously for the others.

  \item The letter \( \BbbH \) is chosen in honor of William Hamilton, who introduced them in a multi-part paper over the course of several years, from 1846 to 1850. The compiled paper can be found in \cite{Hamilton2000Quaternions}.
\end{comments}

\begin{proposition}\label{thm:def:quaternion_algebra}
  \hyperref[def:quaternion_algebra]{Quaternions} have the following basic properties:
  \begin{thmenum}
    \thmitem{thm:def:quaternion_algebra/commutative} Multiplication of quaternions is not commutative.
    \thmitem{thm:def:quaternion_algebra/inverse} \( \BbbH \) is a \hyperref[def:division_ring]{division ring} --- every nonzero quaternion \( a + bi + cj + dk \) has a (two-sided) multiplicative inverse:
    \begin{equation}\label{eq:thm:def:quaternion_algebra/inverse}
      \frac {a - bi - cj - dk} {a^2 + b^2 + c^2 + d^2}
    \end{equation}
  \end{thmenum}
\end{proposition}
\begin{proof}
  \SubProofOf{thm:def:quaternion_algebra/commutative} We have \( ij = k = -ji \).
  \SubProofOf{thm:def:quaternion_algebra/inverse} We have
  \begin{align*}
    &\phantom{{}={}}
    (a + bi + cj + dk) (a - bi - cj - dk)
    = \\ &=
    [\hi{a^2} - (ab)i - (ac)j - (ad)k]
    +
    [(ab)i + \hi{b^2} - (bc)k + (bd)j]
    + \\ &+
    [(ac)j + (bc)k + \hi{c^2} - (cd)i]
    +
    [(ad)k - (bd)j + (cd)i + \hi{d^2}]
    = \\ &=
    a^2 + b^2 + c^2 + d^2.
  \end{align*}

  Furthermore,
  \begin{align*}
    &\phantom{{}={}}
    (a - bi - cj - dk) (a + bi + cj + dk)
    = \\ &=
    [a + (-b) i + (-c) j + (-d) k] [a - (-b) i - (-c) j - (-d) k]
    = \\ &=
    a^2 + (-b)^2 + (-c)^2 + (-d)^2.
  \end{align*}
\end{proof}

\begin{definition}\label{def:quaternionic_group}\mcite[128]{Aluffi2009Algebra}
  We also define the \term{quaternionic group} \( Q_8 \) as the multiplicative subgroup \( \braket{ 1, i, j, k } \) of the \hyperref[def:quaternion_algebra]{quaternion algebra} \( \BbbH \) consisting of the four basis elements and their inverses.
\end{definition}
\begin{comments}
  \item It is tempting to define the quaternion algebra as the group algebra of \( Q_8 \), but that would make the algebra eight-dimensional over \( \BbbR \), with distinct coordinates for \( i \) and \( -i \), \( j \) and \( -j \) and so forth.
\end{comments}

\begin{example}\label{ex:quaternion_polynomial_with_infinitely_many_roots}
  Consider the \hyperref[def:noncommutative_polynomial_algebra]{noncommutative polynomial} \( f(X) = X^2 + 1 \) over the \hyperref[def:quaternion_algebra]{quaternion algebra} \( \BbbH \).

  The square of the quaternion \( x \coloneqq bi + cj + dk \) is
  \begin{align*}
    &\phantom{{}={}}
    (bi + cj + dk) (bi + cj + dk)
    = \\ &=
    [-\hi{b^2} + (bc)k - (bd)j]
    + \\ &+
    [-(bc)k - \hi{c^2} + (cd)i]
    +
    [(bd)j - (cd)i - \hi{d^2}]
    = \\ &=
    -b^2 - c^2 - d^2.
  \end{align*}

  Therefore, if \( b^2 + c^2 + d^2 = 1 \), then \( x^2 = -1 \), and \( f(x) = 0 \). Therefore, if we extend \fullref{def:root_of_polynomial} of polynomial roots to noncommutative polynomials, we obtain that \( f(X) \) has infinitely many roots.

  This contrasts with the commutative case, where \fullref{thm:def:integral_domain/root_limit} implies that \hyperref[def:entire_semiring]{entire} rings have at most \( n \) roots for a polynomial of degree \( n \).
\end{example}
