\section{Symmetry groups}\label{sec:symmetry_groups}

\paragraph{Symmetry groups}

\begin{definition}\label{def:symmetry_group}\mcite[example 4.1.11]{Винберг2014КурсАлгебры}
  For each subset \( A \) of the \hyperref[def:euclidean_space]{Euclidean space} \( \BbbR^n \), we define its \term[ru=группа симметрий]{symmetry group} \( \op{Sym}(A) \) as the subgroup of the \hyperref[def:symmetric_group]{symmetric group} of \( \BbbR^n \) consisting of all \hyperref[def:isometry]{isometries} that \hyperref[def:fixed_point]{fix} \( A \).
\end{definition}

\begin{proposition}\label{thm:symmetry_group_of_sphere}
  The \hyperref[def:symmetry_group]{symmetry group} of the \hyperref[def:metric_space/ball]{unit ball} consists of rotations and reflections about the origin.
\end{proposition}

\paragraph{Dihedral group}

\begin{definition}\label{def:dihedral_group}\mcite[70]{Jacobson1985BasicAlgebraI}
  Assume fixed two \hyperref[def:formal_language/symbol]{symbols} \( a \) and \( b \). We define the \term{dihedral group} via the \hyperref[def:group_presentation]{presentation}
  \begin{equation}\label{eq:def:dihedral_group}
    D_n \coloneqq \braket{ a, b \given a^n = e, b^2 = e, b^{-1} = aba }.
  \end{equation}
\end{definition}
