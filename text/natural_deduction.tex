\subsection{Natural deduction}\label{subsec:natural_deduction}

\paragraph{Inference rules}

\begin{definition}\label{def:inference_rule}\mimprovised
  An \term[ru=правило вывода (\cite[31]{Герасимов2011}), en=inference rule (\cite[2]{MartinLof1996Meanings})]{inference rule} is a string generated by the corresponding grammar rule from the \hyperref[def:formal_grammar/schema]{schema}
  \begin{bnf*}
    \bnfprod{entry list}     {\bnfpn{entry} \bnfor \bnfpn{entry list} \bnfsp \bnftsq{,} \bnfsp \bnfpn{entry list}}, \\
    \bnfprod{premises}       {\bnfes \bnfor \bnfpn{entry} \bnfor \bnfpn{entry list}}, \\
    \bnfprod{conclusion}     {\bnfpn{entry}}, \\
    \bnfprod{inference rule} {\bnftsq{(} \bnfsp \bnfpn{rule name} \bnfsp \bnftsq{)} \bnfsp \bnfpn{premises} \bnfsp \bnftsq{\( \vDdash \)} \bnfsp \bnfpn{conclusion}}
  \end{bnf*}
  where, similarly to \fullref{def:logical_context}, we have purposely not specified rules for the nonterminal \( \bnfpn{entry} \) in order to encompass entries with different syntax. We have also not specified a precise syntax for rule names because we use a variety of symbols as there is little benefit in listing them explicitly here.

  \incite[291]{Szabo1964Gentzen} introduces the notation
  \begin{equation*}
    \begin{prooftree}
      \hypo{ \varphi_1 }
      \hypo{ \cdots }
      \hypo{ \varphi_n }
      \infer3[\logic{R}]{ \psi }
    \end{prooftree}
  \end{equation*}
  for the rule
  \begin{equation*}
    (\logic{R}) \enspace \varphi_1, \ldots, \varphi_n \vDdash \psi.
  \end{equation*}

  We call \( n \) the \term{arity} of the rule \( \logic{R} \).
\end{definition}
\begin{comments}
  \item Identifier grammar rules are motivated and discussed in \fullref{rem:grammar_rules_for_variables}.
  \item This definition is based on the one by \incite[291]{Szabo1964Gentzen}, but formalized via grammars and not tied to either formulas nor sequents.
\end{comments}

\begin{example}\label{ex:fixed_point_recursion_for_relations}
  Among their other uses, \hyperref[def:inference_rule]{inference rules} are a convenient mechanism for defining relations. We will illustrate this with an example.

  The following rules express how to construct the \hyperref[thm:equivalence_closure]{equivalence closure} \( {\congdot} \) of a binary relation \( {\dotsim} \). In the following, \( \alpha \), \( \beta \) and \( \gamma \) are variables whose purpose is act as placeholders for concrete values:
  \begin{align*}
    \begin{prooftree}
      \hypo{ \alpha \dotsim \beta }
      \infer1[\logic{I}]{ \alpha \congdot \beta }
    \end{prooftree}
    &&
    \begin{prooftree}
      \infer0[\logic{R}]{ \alpha \congdot \alpha }
    \end{prooftree}
    \\\\
    \begin{prooftree}
      \hypo{ \alpha \congdot \beta }
      \infer1[\logic{S}]{ \beta \congdot \alpha }
    \end{prooftree}
    &&
    \begin{prooftree}
      \hypo{ \alpha \congdot \beta }
      \hypo{ \beta \congdot \gamma }
      \infer2[\logic{T}]{ \alpha \congdot \gamma }
    \end{prooftree}
  \end{align*}

  Note that we have used \( {\congdot} \) and \( {\dotsim} \) as our relation symbols to highlight that inference rules are syntactic constructions, and the relation symbols are merely placeholders for actual relations.

  Rather than giving an all-encompassing formal definition for how such inference rules are interpreted, we will translate the aforementioned rules into an endofunction on \( M \to M \) suitable for \fullref{thm:knaster_tarski_theorem}:
  \begin{equation*}
    \begin{aligned}
      T(r) \coloneqq \thickspace & r \thickspace {\cup} \\
                                 & \set{ (x, y) \given x \sim y } \thickspace {\cup} \\
                                 & \set{ (x, x) \given x \in M } \thickspace {\cup} \\
                                 & \set{ (y, x) \given (x, y) \in r } \thickspace {\cup} \\
                                 & \set{ (x, z) \given (x, y) \in r \T{and} (y, z) \in r }.
    \end{aligned}
  \end{equation*}

  The equivalence closure is simply the least fixed point of \( T \).

  A downside of this approach is that it may be difficult to determine whether \( T \) is \hyperref[def:scott_continuity]{Scott-continuous}, and thus whether the least fixed point can be reached by taking the union
  \begin{equation*}
    \bigcup_{k=0}^\infty T^k(\varnothing).
  \end{equation*}

  An alternative is to use \fullref{thm:least_fixed_point_recursion}, which guarantees that the underlying operator is Scott-continuous, but has the downside of being more verbose in this particular case. Suppose that our domain of discourse is \( D \coloneqq M \times M \) and consider the following constructors:
  \begin{itemize}
    \item Corresponding to rule \( \logic{I} \), for each pair \( (x, y) \) in \( {\sim} \), the nullary constructor\fnote{This family of constructors could of course be replaced by choosing the set of base elements to be the underlying set of the relation \( {\sim} \), but our purpose is to illustrate how the particular inference rules can be translated into constructors.}
    \begin{equation*}
      i_{x,y} \coloneqq (x, y).
    \end{equation*}

    \item Corresponding to rule \( \logic{R} \), for each element \( x \) of \( M \), the nullary constructor
    \begin{equation*}
      r_x \coloneqq (x, x).
    \end{equation*}

    \item Corresponding to rule \( \logic{S} \), the unary constructor
    \begin{equation*}
      s((x, y)) \coloneqq (y, x).
    \end{equation*}

    \item Corresponding to rule \( \logic{T} \), the partial binary constructor
    \begin{equation*}
      t((x, y), (y', z)) \coloneqq \begin{cases}
        (x, z),        &y = y', \\
        \T{undefined}, &\T{otherwise.} \\
      \end{cases}
    \end{equation*}
  \end{itemize}
\end{example}

\begin{remark}\label{rem:inference_rules_semantics}
  We have introduced inference rules in \fullref{def:inference_rule} as strings with the intention to use them in vastly differing situations:
  \begin{itemize}
    \item Expressing typing rules in \fullref{subsec:simple_type_theory}. This is done, among others, by \incite{Hindley1997} and \incite{Mimram2020}.
    \item Succinctly defining relations as shown in \fullref{ex:fixed_point_recursion_for_relations}. This is done by \incite{Mimram2020}.
    \item Expressing the possibilities of constructing proof trees in natural deduction systems. This is done by \incite{TroelstraSchwichtenberg2000}, \incite{КолмогоровДрагалин2006}, \incite{Герасимов2011} and \incite{Hindley1997}.
  \end{itemize}

  Inference rules can easily be formalized as \hyperref[def:relation]{relations}, as is done by \incite[31]{Герасимов2011}, \incite[115]{CitkinMuravitsky2021} and \incite[63]{Galatos2007}, but this makes it difficult to change the underlying semantics. Other authors only use the notion of inference rules informally.

  For describing his \enquote{natural deduction} systems, \incite{Szabo1964Gentzen} uses the notation
  \begin{equation*}
    \begin{prooftree}
      \hypo{ \varphi_1 }
      \hypo{ \cdots }
      \hypo{ \varphi_n }
      \infer3[\logic{R}]{ \psi }
    \end{prooftree},
  \end{equation*}
  where \( \varphi_1, \ldots, \varphi_n \) and \( \psi \) are placeholders for formulas. A problem with this approach is that the rules may depend on additional metalogical context.

  \incite[2]{MartinLof1996Meanings} argues that, in situations like the one above, we should instead use metalogical rules to affirm the corresponding formulas:
  \begin{equation*}
    \begin{prooftree}
      \hypo{ \vdash \varphi_1 }
      \hypo{ \cdots }
      \hypo{ \vdash \varphi_n }
      \infer3[\logic{R}]{ \vdash \psi }
    \end{prooftree}
  \end{equation*}

  Later in \cite{Szabo1964Gentzen}, the original of which was written prior to Martin-L\"of's birth, Gentzen introduces \enquote{sequent calculus} that uses \hyperref[def:sequent]{sequents} instead of formulas, which results in an extension of the latter notation that also allows formulas on the left. Some authors, like \incite[46]{Mimram2020}, prefer to add the necessary context for natural deduction on the left side of sequents, which would otherwise be empty. \incite[ch. 2]{TroelstraSchwichtenberg2000} present both of these variants of natural deduction.

  To clarify whether rules act inside the object theory or the metatheory, \incite[117]{CitkinMuravitsky2021}, who define rules as relations, refer as \enquote{rule} to inference rules acting on sequents and as \enquote{hyperrules} to inference rules acting on judgments, and give distinct definitions for the two.
\end{remark}

\paragraph{Natural deduction systems}

\begin{definition}\label{def:propositional_natural_deduction_system}\mimprovised
  Consider the grammar schema of propositional logic from \fullref{def:propositional_syntax} extended with \hyperref[def:propositional_formula_placeholder]{formula placeholders}.

  A \term{propositional natural deduction system} consists of a nonempty collection of \hyperref[def:inference_rule]{inference rules}, implicitly assumed to have different names, whose entries can be either formula placeholders or pairs of placeholders:
  \begin{bnf*}
    \bnfprod{entry} {\bnfpn{formula placeholder} \bnfor} \\
    \bnfmore        {\bnftsq{[} \bnfsp \bnfpn{formula placeholder} \bnfsp \bnftsq{]} \bnfsp \bnfpn{formula placeholder}}.
  \end{bnf*}

  These bracketed placeholders allow \enquote{closing} certain assumptions --- see \fullref{def:natural_deduction_proof_tree}.

  We forbid the conclusion to have such a bracketed placeholder by replacing the existing rules from \fullref{def:inference_rule} with
  \begin{bnf*}
    \bnfprod{conclusion} {\bnfpn{formula placeholder}}
  \end{bnf*}

  We prefer to denote the rule
  \begin{equation*}
    (\logic{R}) \enspace \Phi_1, \ldots, [\Phi_k] \thickspace \Theta_k, \ldots, \Theta_n \vDdash \Psi
  \end{equation*}
  as
  \begin{equation*}
    \begin{prooftree}
      \hypo{ \Phi_1 }
      \hypo{ \cdots }
      \hypo{ [\Theta_k] }
      \ellipsis {} { \Phi_k }
      \hypo{ \cdots }
      \hypo{ \Phi_n }
      \infer5[\logic{R}]{ \Psi }
    \end{prooftree}
  \end{equation*}
\end{definition}
\begin{comments}
  \item This definition resembles \enquote{deductive systems} defined by \incite[31]{Герасимов2011} and \enquote{axiom systems} defined by \incite[80]{Smullyan1995}. We have restricted ourselves to propositional logic and to natural deduction because we want our definition to be as precise as possible. The rules are formalized as strings so that they can be treated rigorously, and then they are specifically adjusted for natural deduction. The adjustments are based on \cite[sec. 2.1]{TroelstraSchwichtenberg2000}. Furthermore, we assume no axioms here.
\end{comments}

\begin{definition}\label{def:natural_deduction_proof_tree}\mimprovised
  Fix a \hyperref[def:propositional_natural_deduction_system]{propositional natural deduction system}. We will define a family of \hyperref[def:multiway_tree]{multiway trees}, which we will call \term{proof trees}.

  We will use \fullref{thm:least_fixed_point_recursion} on multiway trees whose values are triples consisting of the following objects:
  \begin{thmenum}[series=def:natural_deduction_proof_tree]
    \thmitem{def:natural_deduction_proof_tree/conclusion} A formula, called the \term[ru=conclusion (\cite[35]{TroelstraSchwichtenberg2000}), en=conclusion (\cite[36]{TroelstraSchwichtenberg2000})]{conclusion} of the proof.

    \thmitem{def:natural_deduction_proof_tree/rule_name} An empty string or the name of a rule from the natural deduction system.

    \thmitem{def:natural_deduction_proof_tree/context} A \hyperref[def:logical_context]{context} whose entries are marked formulas generated by the following grammar:
    \begin{bnf*}
      \bnfprod{marker}          {\bnfpn{Latin identifier}}, \\
      \bnfprod{marked formula}  {\bnfpn{marker} \bnfsp \bnftsq{:} \bnfsp \bnfpn{formula}}.
    \end{bnf*}
  \end{thmenum}

  We will define constructors on general multiway trees of this form, however by \enquote{proof tree} we will only mean those recursively obtained via \fullref{thm:least_fixed_point_recursion}.

  We will define two kinds of trees:
  \begin{thmenum}[resume=def:natural_deduction_proof_tree]
    \thmitem{def:natural_deduction_proof_tree/assumption} We will first define trees corresponding to assumptions, which will be the \hyperref[thm:least_fixed_point_recursion/base]{base elements} of the recursion.

    An \term{assumption} for a propositional formula \( \varphi \) and marker \( u \) is the singleton tree with the following value:
    \begin{itemize}
      \item The conclusion is, unsurprisingly, \( \varphi \).
      \item The rule name is empty (because the conclusion has not been obtained by an application of a rule).
      \item The context is the singleton list with entry \( u: \varphi \).
    \end{itemize}

    We will denote such a tree as follows:
    \begin{equation*}
      [\varphi]^u.
    \end{equation*}

    \thmitem{def:natural_deduction_proof_tree/application} We will now define trees called \term{rule applications}. In terms of \fullref{thm:least_fixed_point_recursion}, we define an \( n \)-ary \hyperref[thm:least_fixed_point_recursion/base]{partial constructor} for every \( n \)-ary rule and every (uniform) \hyperref[def:uniform_placeholder_substitution]{placeholder substitution}.

    Consider the substitution \( \sigma \) and the rule
    \begin{equation*}
      \begin{prooftree}
        \hypo{ \Phi_1 }
        \hypo{ \cdots }
        \hypo{ [\Theta_i] }
        \ellipsis {} { \Phi_i }
        \hypo{ \cdots }
        \hypo{ \Phi_n }
        \infer5[\logic{R}]{ \Psi }
      \end{prooftree}
    \end{equation*}

    Fix a tuple \( (T_1, \ldots, T_n) \) of trees such that, for \( i = 1, \ldots, n \), the conclusion of \( T_i \) is \( \Phi_i[\sigma] \). We leave the constructor undefined for tuples that do not satisfy this condition.

    The delicate part of defining rule application trees is to choose a context \( \Gamma \) of assumptions \enquote{closed} at this step. We start with an empty context and use recursion on \( i = 1, \ldots, n \) to define \( \Gamma \). At step \( i \), if \( \Phi_i \) has a bracketed formula \( [\Theta_i] \) attached, if \( T_i \) contains an assumption node labeled by the formula \( \Theta_i[\sigma] \) and if the marker \( u \) of the assumption is not in the context of any rule application subtree of \( T_i \), we will close this assumption now by adding \( u: \Theta_i[\sigma] \) to \( \Gamma \).

    We can now define a node value that will serve as the root of an application tree:
    \begin{itemize}
      \item The conclusion is \( \Psi[\sigma] \).
      \item The rule name is \( \logic{R} \).
      \item The context is \( \Gamma \).
    \end{itemize}

    Finally, we define a rule application tree, the value of our constructor, as the tree whose root has the aforementioned value and whose immediate subtrees are \( T_1, \ldots, T_n \). We draw this tree as follows (resembling stacked inference rules):
    \begin{equation*}
      \begin{prooftree}
        \hypo{ T_1 }
        \hypo{ \cdots }
        \hypo{ T_n }
        \infer[left label=\( \Gamma \)]3[\logic{R}]{ \Psi[\sigma] }
      \end{prooftree}
    \end{equation*}
  \end{thmenum}
\end{definition}
\begin{comments}
  \item This definition somewhat resembles \enquote{prooftrees} from \incite{TroelstraSchwichtenberg2000}, but is completely rigorous and explicitly encodes which inference rules have been used. The treatment of the rules is based on \incite[sec. 2.1]{TroelstraSchwichtenberg2000}.
  \item It is possible to avoid markers entirely and close assumptions independent of their label. \incite[subsec. 2.1.9]{TroelstraSchwichtenberg2000} call this the \enquote{complete discharge convention}. We prefer not to use this convention for two reasons:
  \begin{itemize}
    \item It is easier to track assumptions in proofs.
    \item The relationship to \( \synlambda \)-terms becomes apparent.
  \end{itemize}
\end{comments}

\begin{definition}\label{def:natural_deduction_open_assumptions}\mimprovised
  We say that the formula \( \varphi \) is an \term{open assumption} of a \hyperref[def:natural_deduction_proof_tree]{natural deduction proof tree} \( T \) if there exists an assumption subtree \( [\varphi]^u \) in \( T \) and if \( u: \varphi \) is not closed by any rule application in \( T \).
\end{definition}
\begin{comments}
  \item By this definition, \( \varphi \) is an open assumption if, for distinct labels, \( [\varphi]^v \) is closed and \( [\varphi]^u \) is not. To avoid such situations, it is best to only have one marker per formula.
  \item Another curious situation is if \( [\varphi]^u \) is closed in one subtree but not in another. We do not consider it an open assumption in this case.
\end{comments}

\begin{definition}\label{def:natural_deduction_entailment}\mcite[205]{Hinman2005}
  For a \hyperref[def:natural_deduction_system]{natural deduction system}, we define a \hyperref[def:consequence_relation]{consequence relation} \( \Gamma \vdash \varphi \) that holds when there is a \hyperref[def:natural_deduction_proof_tree]{proof tree} with conclusion \( \varphi \) whose \hyperref[def:natural_deduction_open_assumptions]{open assumptions} are all in \( \Gamma \).
\end{definition}
\begin{defproof}
  We must show that \( {\vdash} \) satisfies the conditions for being a consequence relation.

  \SubProofOf[def:consequence_relation/reflexivity]{reflexivity} If \( \varphi \in \Gamma \), then, for any marker \( u \), the tree \( [\varphi]^u \) proves \( \varphi \) from \( \Gamma \).

  \SubProofOf[def:consequence_relation/monotonicity]{monotonicity} If \( \Gamma \vdash \varphi \), then there exists a proof tree of \( \varphi \) from \( \Gamma \). Such a tree also proves \( \varphi \) from any superset of \( \Gamma \).

  \SubProofOf[def:consequence_relation/transitivity]{transitivity} Suppose that \( \Delta, \Epsilon \vdash \varphi \) and that, for every \( \psi \in \Delta \), we have \( \Gamma, \Epsilon \vdash \psi \).

  Let \( T_\varphi \) be a proof tree of \( \varphi \) from \( \Delta \cup \Epsilon \) and let \( T_\psi \) be a proof tree of \( \psi \) from \( \Gamma \cup \Epsilon \).

  Then, for any formula \( \psi \in \Delta \) and any marker \( u \), we can replace the assumption \( [\psi]^u \) in \( T_\varphi \) with \( T_\psi \) and remove \( u: \psi \) from the contexts of the rule applications in \( T_\varphi \).

  The resulting tree will have the same conclusion as \( T_\varphi \) --- namely, \( \varphi \) --- but different open assumptions --- namely, members of \( \Gamma \cup \Epsilon \).
\end{defproof}

\begin{proposition}\label{thm:axiomatic_derivation_as_natural_deduction}
  Consider a (propositional) \hyperref[def:axiomatic_derivation_system]{axiomatic derivation system}. We regard it as a \hyperref[def:natural_deduction_system]{natural deduction system} as follows:
  \begin{thmenum}
    \thmitem{thm:axiomatic_derivation_as_natural_deduction/axiom} For every axiom schema \( \Phi \), we introduce the rule
    \begin{equation*}
      \begin{prooftree}
        \infer0[\logic{Ax}]{ \Phi }
      \end{prooftree}
    \end{equation*}

    Note that \( \Phi \) here is a metalogical variable that refers to a \hyperref[def:propositional_formula_placeholder]{formula placeholder}.

    For convenience, here we must break our convention of rules having unique names.

    \thmitem{thm:axiomatic_derivation_as_natural_deduction/mp} For deducing proofs, we introduce the following rule, called \term{modus ponens}:
    \begin{equation*}
      \begin{prooftree}
        \hypo{ \varphi \rightarrow \psi }
        \hypo{ \varphi }
        \infer2[\logic{MP}]{ \psi }
      \end{prooftree}
    \end{equation*}

    Note that \( \varphi \rightarrow \psi \) here refers to a concrete formula placeholder.
  \end{thmenum}

  Then there exists an \hyperref[def:axiomatic_derivation]{axiomatic derivation} of \( \varphi \) from \( \Gamma \) if and only if there exists a \hyperref[def:natural_deduction_proof_tree]{proof tree} concluding \( \varphi \) from \( \Gamma \).
\end{proposition}

\begin{example}\label{ex:minimal_implication_logic_identity/trees}
  We will demonstrate \fullref{thm:axiomatic_derivation_as_natural_deduction} by reformulating \fullref{ex:minimal_implication_logic_identity/derivations} via natural deduction rather than axiomatic derivations.

  Consider \hyperref[def:minimal_implication_logic]{minimal implicational logic}, which consists of two axiom schemas --- \eqref{eq:def:minimal_implication_logic/intro} and \eqref{eq:def:minimal_implication_logic/dist}. As per \fullref{thm:axiomatic_derivation_as_natural_deduction}, these become inference rules.

  Then the final derivation from \fullref{ex:minimal_implication_logic_identity/derivations} can be presented as follows:
  \begin{equation}\label{eq:ex:minimal_implication_logic_identity/trees/derivation/proof}
    \begin{prooftree}[separation=3em]
      \hypo
        {
          \eqref{eq:def:minimal_implication_logic/intro}
        }

      \ellipsis
        {
          \( \begin{array}{l}
            \psi \mapsto (\varphi \synimplies \varphi)
            \\
            \mbox{}
          \end{array} \)
        }
        {
          \eqref{eq:ex:propositional_implicational_logic/dagger}
        }

      \hypo
        {
          \eqref{eq:def:minimal_implication_logic/trans}
        }

      \ellipsis
        {
          \( \begin{array}{l}
            \psi \mapsto (\varphi \synimplies \varphi)
            \\
            \theta \mapsto \varphi
          \end{array} \)
        }
        {
          \eqref{eq:ex:propositional_implicational_logic/dagger}
          \synimplies ((\varphi \synimplies (\varphi \synimplies \varphi)) \synimplies (\varphi \synimplies \varphi))
        }

      \infer2[\ref{eq:def:def:axiomatic_derivation_system/mp}]{(\varphi \synimplies (\varphi \synimplies \varphi)) \synimplies (\varphi \synimplies \varphi)}

      \hypo
        {
          \eqref{eq:def:minimal_implication_logic/intro}
        }

      \ellipsis
        {
          \( \psi \mapsto \varphi \)
        }
        {
          \varphi \synimplies (\varphi \synimplies \varphi)
        }

      \infer2[\ref{eq:def:def:axiomatic_derivation_system/mp}]{\varphi \synimplies \varphi}
    \end{prooftree}
  \end{equation}
  where
  \begin{equation}\label{eq:ex:propositional_implicational_logic/dagger}
    \varphi \synimplies ((\varphi \synimplies \varphi) \synimplies \varphi).
  \end{equation}
\end{example}

\paragraph{Minimal logic}

\begin{definition}\label{def:minimal_implicational_logic}\mcite[def. 2.4.1]{TroelstraSchwichtenberg2000}
  While the \hyperref[def:minimal_implication_logic]{minimal implicational logic} is simple, it is of more practical use to have all propositional connectives available. As it turns out, we cannot utilize \hyperref[def:boolean_closure/complete]{complete sets of Boolean functions} unless we are dealing with \hyperref[def:propositional_entailment]{classical semantics} --- see \fullref{ex:heyting_semantics_lem_counterexample} and \fullref{ex:topological_semantics_lem_counterexample} for cases where \fullref{thm:classical_equivalences/conditional_as_disjunction} fails to hold.

  Our goal is to define the \term{minimal implicational logic}, which would correspond to \hyperref[def:minimal_logic]{minimal logic}. It is axiomatic in the sense that we do not use new rules to express the rest of the propositional syntax, but instead we need axiom schemas for each connective. The only exception is \hyperref[def:propositional_alphabet/constants/verum]{\( \synbot \)}, the axioms for which tend to change semantics by a lot --- see \fullref{thm:minimal_propositional_negation_laws}.

  Axioms with \( + \) in the superscript are called \term{introduction axioms} and axioms with \( - \) are called \term{elimination axioms}.

  The following axioms are essential in the sense that they cannot be defined in terms of others:
  \begin{thmenum}[series=def:minimal_implicational_logic]
    \thmitem{def:minimal_implicational_logic/top} The simplest axiom states that the constant \hyperref[def:propositional_alphabet/constants/verum]{\( \syntop \)} is itself an axiom:
    \begin{equation}\label{eq:def:minimal_implicational_logic/top/intro}
      \syntop \tag{\( \syntop_A^+ \)}
    \end{equation}

    \thmitem{def:minimal_implicational_logic/and} Axioms for \hyperref[def:propositional_alphabet/connectives/conjunction]{conjunction}:
    \begin{align}
      \mathllap{ (\psi \synwedge \psi) } &\synimplies \mathrlap{ \psi } \tag{\( \wedge_A^{+_L} \)} \label{eq:def:minimal_implicational_logic/and/elim_left} \\
      \mathllap{ (\psi \synwedge \psi) } &\synimplies \mathrlap{ \psi } \tag{\( \wedge_A^{+_R} \)} \label{eq:def:minimal_implicational_logic/and/elim_right} \\
      \mathllap{ \psi }               &\synimplies \mathrlap{ \parens[\Big]{ \psi \synimplies (\psi \synwedge \psi) } } \tag{\( \wedge_A^{-} \)} \label{eq:def:minimal_implicational_logic/and/intro}
    \end{align}

    \thmitem{def:minimal_implicational_logic/or} Axioms for \hyperref[def:propositional_alphabet/connectives/disjunction]{disjunction}:
    \begin{align}
      \mathllap{ \psi }                      &\synimplies \mathrlap{ (\psi \synvee \psi) } \tag{\( \vee_A^{+_L} \)} \label{eq:def:minimal_implicational_logic/or/intro_left} \\
      \mathllap{ \psi }                      &\synimplies \mathrlap{ (\psi \synvee \psi) } \tag{\( \vee_A^{+_R} \)} \label{eq:def:minimal_implicational_logic/or/intro_right} \\
      \mathllap{ (\psi \synimplies \theta) } &\synimplies \mathrlap{ \parens[\Big]{ (\psi \synimplies \theta) \synimplies ((\psi \synvee \psi) \synimplies \theta) } } \tag{\( \vee_A^{-} \)} \label{eq:def:minimal_implicational_logic/or/elim}
    \end{align}
  \end{thmenum}

  The following axioms and are said to be \enquote{abbreviations} and do not affect semantics:
  \begin{thmenum}[resume=def:minimal_implicational_logic]
    \thmitem{def:minimal_implicational_logic/iff} The axioms for the biconditional are motivated by \fullref{def:heyting_algebra/biconditional}:
    \begin{align}
      \mathllap{ (\psi \synimplies \psi)     } &\synimplies \mathrlap{ \parens[\Big]{ (\psi \synimplies \psi) \synimplies (\psi \syniff \psi) } } \tag{\( \leftrightarrow_A^{+} \)} \label{def:minimal_implicational_logic/iff/intro} \\
      \mathllap{ (\psi \syniff \psi)  }&\synimplies \mathrlap{ (\psi \synimplies \psi) } \tag{\( \leftrightarrow_A^{-_L} \)} \label{eq:def:minimal_implicational_logic/iff/elim_left} \\
      \mathllap{ (\psi \syniff \psi) } &\synimplies \mathrlap{ (\psi \synimplies \psi) } \tag{\( \leftrightarrow_A^{-_R} \)} \label{eq:def:minimal_implicational_logic/iff/elim_right}
    \end{align}

    \thmitem{def:minimal_implicational_logic/negation} The axioms for negation are motivated by \fullref{thm:intuitionistic_equivalences/negation_bottom}:
    \begin{align}
      \mathllap{ \synneg \psi }               &\synimplies \mathrlap{ (\psi \synimplies \synbot) } \tag{\( \neg_A^- \)} \label{eq:def:minimal_implicational_logic/neg/elim} \\
      \mathllap{ (\psi \synimplies \synbot) } &\synimplies \mathrlap{ \synneg \psi } \tag{\( \neg_A^+ \)} \label{eq:def:minimal_implicational_logic/neg/intro}
    \end{align}
  \end{thmenum}
\end{definition}

\begin{definition}\label{def:propositional_natural_deduction_system}\mcite[sec. 1.3.2]{TroelstraSchwichtenberg2000}
  \term{Natural deduction systems} are \hyperref[def:deduction_system]{deduction systems} whose set of rules allows \enquote{discharging} certain assumptions of the proof tree. These rules correspond to \enquote{bringing in} the sequent \( \psi \vdash \psi \) as a formula \( \psi \to \psi \), thus eliminating \( \psi \) as an assumption as justified by \fullref{thm:syntactic_deduction_theorem}.

  In a natural deduction system, every assumption in a \hyperref[def:natural_deduction_proof_tree]{proof trees} has a unique \hyperref[def:labeled_set]{label}. By default, we say that assumptions are \term{undischarged}. When applying a rule that supports discharging, we add a label to the subproof that matches the label of the assumption which we discharge. We additionally add a \hyperref[def:natural_deduction_proof_tree/premises]{non-premise label} to every discharged assumption so that it does not affect \hyperref[def:proof_derivability]{derivability}.

  We allow applying rules without discharging any assumptions. For example, sometimes we do not have any assumptions to discharge e.g. in our proof of correctness of \fullref{def:minimal_propositional_natural_deduction_system/imp}.
\end{definition}

\begin{definition}\label{def:minimal_propositional_natural_deduction_system}\mcite[def. 2.1.1]{TroelstraSchwichtenberg2000}
  We define the \term{minimal propositional natural deduction system}, which is the \hyperref[def:propositional_natural_deduction_system]{natural deduction} equivalent of the \hyperref[def:minimal_implicational_logic]{minimal implicational logic}.

  \begin{thmenum}
    \thmitem{def:minimal_propositional_natural_deduction_system/imp} The following rules corresponds to the conditional axiom schemas in \fullref{def:minimal_implication_logic}:

    \begin{minipage}[t]{0.45\textwidth}
      This rule is inspired by \eqref{eq:def:minimal_implication_logic/intro}:
      \begin{equation*}\taglabel[\( \rightarrow^+ \)]{eq:def:minimal_propositional_natural_deduction_system/imp/intro}
        \begin{prooftree}
          \hypo{ [\psi]^n }
          \ellipsis {} { \varphi }
          \infer[left label=\( n \)]1[\ref{eq:def:minimal_propositional_natural_deduction_system/imp/intro}]{ \psi \synimplies \varphi }
        \end{prooftree}
      \end{equation*}
    \end{minipage}
    \hfill
    \begin{minipage}[t]{0.45\textwidth}
      This rule is merely a renaming of \eqref{eq:def:def:axiomatic_derivation_system/mp}:
      \begin{equation*}\taglabel[\( \rightarrow^- \)]{eq:def:minimal_propositional_natural_deduction_system/imp/elim}
        \begin{prooftree}
          \hypo{ \varphi \synimplies \psi }
          \hypo{ \varphi }
          \infer2[\ref{eq:def:minimal_propositional_natural_deduction_system/imp/elim}]{ \psi }
        \end{prooftree}
      \end{equation*}
    \end{minipage}

    The additional notation in \eqref{eq:def:minimal_propositional_natural_deduction_system/imp/intro} means that the premise marked with \( n \), if any, can be discharged.

    Note that there is no rule corresponding to \eqref{eq:def:minimal_implication_logic/trans} because this axiom schema follows from \eqref{eq:def:minimal_propositional_natural_deduction_system/imp/intro} and \eqref{eq:def:minimal_propositional_natural_deduction_system/imp/elim}. Unlike in the Hilbert system where \eqref{eq:def:minimal_implication_logic/trans} is used to prove \fullref{thm:syntactic_deduction_theorem}, here we have a stronger connection between \( \synimplies \) in the object language and \( \vdash \) in the metalanguage given by \eqref{eq:def:minimal_propositional_natural_deduction_system/imp/intro}.

    \thmitem{def:minimal_propositional_natural_deduction_system/top} The following rule corresponds to the axiom \eqref{eq:def:minimal_implicational_logic/top/intro}:
    \begin{equation*}\taglabel[\( \top^+ \)]{eq:def:minimal_propositional_natural_deduction_system/top/intro}
      \begin{prooftree}
        \infer0[\ref{eq:def:minimal_propositional_natural_deduction_system/top/intro}]{ \syntop }
      \end{prooftree}
    \end{equation*}

    As discussed in \fullref{def:deduction_system/rule}, applications of this rule have a \hyperref[def:natural_deduction_proof_tree/premises]{non-premise label} in order to prevent \( \syntop \) as an undischarged assumption.

    \thmitem{def:minimal_propositional_natural_deduction_system/and} The following rules corresponds to the conjunction axiom schemas in \fullref{def:minimal_implicational_logic/and}:

    \begin{minipage}{0.3\textwidth}
      \begin{equation*}\taglabel[\( \wedge^+ \)]{eq:def:minimal_propositional_natural_deduction_system/and/intro}
        \begin{prooftree}
          \hypo{ \varphi }
          \hypo{ \psi }
          \infer2[\ref{eq:def:minimal_propositional_natural_deduction_system/and/intro}]{ \varphi \synwedge \psi }
        \end{prooftree}
      \end{equation*}
    \end{minipage}
    \hfill
    \begin{minipage}{0.3\textwidth}
      \begin{equation*}\taglabel[\( \wedge^{-_L} \)]{eq:def:minimal_propositional_natural_deduction_system/and/elim_left}
        \begin{prooftree}
          \hypo{ \varphi \synwedge \psi }
          \infer1[\ref{eq:def:minimal_propositional_natural_deduction_system/and/elim_left}]{ \psi }
        \end{prooftree}
      \end{equation*}
    \end{minipage}
    \hfill
    \begin{minipage}{0.3\textwidth}
      \begin{equation*}\taglabel[\( \wedge^{-_R} \)]{eq:def:minimal_propositional_natural_deduction_system/and/elim_right}
        \begin{prooftree}
          \hypo{ \varphi \synwedge \psi }
          \infer1[\ref{eq:def:minimal_propositional_natural_deduction_system/and/elim_right}]{ \varphi }
        \end{prooftree}
      \end{equation*}
    \end{minipage}

    \thmitem{def:minimal_propositional_natural_deduction_system/or} The following rules corresponds to the disjunction axiom schemas in \fullref{def:minimal_implicational_logic/or}:

    \begin{minipage}{0.3\textwidth}
      \begin{equation*}\taglabel[\( \vee^{+_L} \)]{eq:def:minimal_propositional_natural_deduction_system/or/intro_left}
        \begin{prooftree}
          \hypo{ \varphi }
          \infer1[\ref{eq:def:minimal_propositional_natural_deduction_system/or/intro_left}]{ \varphi \synvee \psi }
        \end{prooftree}
      \end{equation*}
    \end{minipage}
    \hfill
    \begin{minipage}{0.3\textwidth}
      \begin{equation*}\taglabel[\( \vee^{+_R} \)]{eq:def:minimal_propositional_natural_deduction_system/or/intro_right}
        \begin{prooftree}
          \hypo{ \psi }
          \infer1[\ref{eq:def:minimal_propositional_natural_deduction_system/or/intro_right}]{ \varphi \synvee \psi }
        \end{prooftree}
      \end{equation*}
    \end{minipage}
    \hfill
    \begin{minipage}{0.3\textwidth}
      \begin{equation*}\taglabel[\( \vee^- \)]{eq:def:minimal_propositional_natural_deduction_system/or/elim}
        \begin{prooftree}
          \hypo{ \varphi \synvee \psi }
          \hypo{ [\varphi]^n }
          \ellipsis {} { \theta }
          \hypo{ [\psi]^n }
          \ellipsis {} { \theta }
          \infer[left label=\( n \)]3[\ref{eq:def:minimal_propositional_natural_deduction_system/or/elim}]{ \theta }
        \end{prooftree}
      \end{equation*}
    \end{minipage}

    \thmitem{def:minimal_propositional_natural_deduction_system/iff} The following rules corresponds to the biconditional axiom schemas in \fullref{def:minimal_implicational_logic/iff}:

    \begin{minipage}{0.3\textwidth}
      \begin{equation*}\taglabel[\( \leftrightarrow^+ \)]{eq:def:minimal_propositional_natural_deduction_system/iff/intro}
        \begin{prooftree}
          \hypo{ \varphi \synimplies \psi }
          \hypo{ \psi \synimplies \varphi }
          \infer[left label=\( n \)]2[\ref{eq:def:minimal_propositional_natural_deduction_system/iff/intro}]{ \varphi \syniff \psi }
        \end{prooftree}
      \end{equation*}
    \end{minipage}
    \hfill
    \begin{minipage}{0.3\textwidth}
      \begin{equation*}\taglabel[\( \leftrightarrow^{-_L} \)]{eq:def:minimal_propositional_natural_deduction_system/iff/elim_left}
        \begin{prooftree}
          \hypo{ \varphi \syniff \psi }
          \hypo{ \psi }
          \infer2[\ref{eq:def:minimal_propositional_natural_deduction_system/iff/elim_left}]{ \varphi }
        \end{prooftree}
      \end{equation*}
    \end{minipage}
    \hfill
    \begin{minipage}{0.3\textwidth}
      \begin{equation*}\taglabel[\( \leftrightarrow^{-_R} \)]{eq:def:minimal_propositional_natural_deduction_system/iff/elim_right}
        \begin{prooftree}
          \hypo{ \varphi \syniff \psi }
          \hypo{ \varphi }
          \infer2[\ref{eq:def:minimal_propositional_natural_deduction_system/iff/elim_right}]{ \psi }
        \end{prooftree}
      \end{equation*}
    \end{minipage}

    \thmitem{def:minimal_propositional_natural_deduction_system/negation} The following rules corresponds to the negation axiom schemas in \fullref{def:minimal_implicational_logic/negation}:

    \begin{minipage}{0.45\textwidth}
      \begin{equation*}\taglabel[\( \neg^+ \)]{eq:def:minimal_propositional_natural_deduction_system/neg/intro}
        \begin{prooftree}
          \hypo{ [\varphi]^n }
          \ellipsis {} { \synbot }
          \infer[left label=\( n \)]1[\ref{eq:def:minimal_propositional_natural_deduction_system/neg/intro}]{ \synneg \varphi }
        \end{prooftree}
      \end{equation*}
    \end{minipage}
    \hfill
    \begin{minipage}{0.45\textwidth}
      \begin{equation*}\taglabel[\( \neg^- \)]{eq:def:minimal_propositional_natural_deduction_system/neg/elim}
        \begin{prooftree}
          \hypo{ \varphi }
          \hypo{ \synneg \varphi }
          \infer2[\ref{eq:def:minimal_propositional_natural_deduction_system/neg/elim}]{ \synbot }
        \end{prooftree}
      \end{equation*}
    \end{minipage}
  \end{thmenum}
\end{definition}
\begin{defproof}
  We will prove that the axiomatic \hyperref[def:minimal_implicational_logic]{minimal implicational logic} is equivalent to the rules of natural deduction described in this proposition.

  \SubProofOf{def:minimal_propositional_natural_deduction_system/imp} Consider first the axiom \eqref{eq:def:minimal_implication_logic/intro}. It can be derived via \eqref{eq:def:minimal_propositional_natural_deduction_system/imp/intro} as follows:
  \begin{equation*}
    \begin{prooftree}
      \hypo{ [\varphi]^1 }
      \infer1[\ref{eq:def:minimal_propositional_natural_deduction_system/imp/intro}]{ \psi \synimplies \varphi }
      \infer[left label=\( 1 \)]1[\ref{eq:def:minimal_propositional_natural_deduction_system/imp/intro}]{ \varphi \synimplies (\psi \synimplies \varphi) }
    \end{prooftree}
  \end{equation*}

  Note that we have used \ref{eq:def:minimal_propositional_natural_deduction_system/imp/intro} twice, but have only discharged an assumption the second time.

  Conversely, fix two formulas \( \varphi \) and \( \psi \). Then \( \psi \synimplies (\varphi \synimplies \psi) \) is an instance of \eqref{eq:def:minimal_implication_logic/intro}. Thus, we obtain \( \psi \vdash \varphi \synimplies \psi \) by applying \eqref{eq:def:def:axiomatic_derivation_system/mp}, which in turn shows the validity of the rule \eqref{eq:def:minimal_propositional_natural_deduction_system/imp/intro}.

  Now we will show that \eqref{eq:def:minimal_implication_logic/trans} can be derived using only the rules \eqref{eq:def:minimal_propositional_natural_deduction_system/imp/intro} and \eqref{eq:def:minimal_propositional_natural_deduction_system/imp/elim}:
  \begin{equation}\label{eq:def:minimal_propositional_natural_deduction_system/imp/trans_derivation}
    \begin{prooftree}
      \hypo{ [\varphi \synimplies (\psi \synimplies \theta)]^1 }
      \hypo{ [\varphi]^2 }
      \infer2[\ref{eq:def:minimal_propositional_natural_deduction_system/imp/elim}]{ \psi \synimplies \theta }

      \hypo{ [\varphi \synimplies \psi]^3 }
      \hypo{ [\varphi]^2 }
      \infer2[\ref{eq:def:minimal_propositional_natural_deduction_system/imp/elim}]{ \psi }

      \infer2[\ref{eq:def:minimal_propositional_natural_deduction_system/imp/elim}]{ \theta }

      \infer[left label=\( 2 \)]1[\ref{eq:def:minimal_propositional_natural_deduction_system/imp/intro}]{ \varphi \synimplies \theta }
      \infer[left label=\( 3 \)]1[\ref{eq:def:minimal_propositional_natural_deduction_system/imp/intro}]{ (\varphi \synimplies \psi) \synimplies (\varphi \synimplies \theta) }
      \infer[left label=\( 1 \)]1[\ref{eq:def:minimal_propositional_natural_deduction_system/imp/intro}]{ \eqref{eq:def:minimal_implication_logic/trans} }
    \end{prooftree}
  \end{equation}

  \SubProofOf{def:minimal_propositional_natural_deduction_system/top} Obvious.

  \SubProofOf{def:minimal_propositional_natural_deduction_system/and} The rule \eqref{eq:def:minimal_implicational_logic/and/intro} is equivalent but more readable than proving \( \set{ \varphi, \psi } \vdash \varphi \synwedge \psi \) directly. Indeed, compare it to
  \begin{equation*}
    \begin{prooftree}
      \hypo{ \varphi }
      \hypo{ \eqref{eq:def:minimal_implicational_logic/and/intro} }
      \infer2[\ref{eq:def:def:axiomatic_derivation_system/mp}]{ \psi \synimplies (\varphi \synwedge \psi) }

      \hypo{ \psi }
      \infer2[\ref{eq:def:def:axiomatic_derivation_system/mp}]{ \varphi \synwedge \psi },
    \end{prooftree}
  \end{equation*}
  which is a derivation of \( \varphi \synwedge \psi \) from \( \set{ \varphi, \psi } \) using the axiomatic system. The other direction is also simple:
  \begin{equation}\label{eq:def:minimal_propositional_natural_deduction_system/and_intro_axiom_derivation}
    \begin{prooftree}
      \hypo{ [\varphi]^1 }
      \hypo{ [\psi]^2 }
      \infer2[\ref{eq:def:minimal_propositional_natural_deduction_system/and/intro}]{ \varphi \synwedge \psi }
      \infer[left label=\( 2 \)]1[\ref{eq:def:minimal_propositional_natural_deduction_system/imp/intro}]{ \psi \synimplies (\varphi \synwedge \psi) },
      \infer[left label=\( 1 \)]1[\ref{eq:def:minimal_propositional_natural_deduction_system/imp/intro}]{ \eqref{eq:def:minimal_implicational_logic/and/intro} },
    \end{prooftree}
  \end{equation}

  The other two rules are trivially connected to the corresponding axioms using a single application of \eqref{eq:def:def:axiomatic_derivation_system/mp}.

  \SubProofOf{def:minimal_propositional_natural_deduction_system/or} For a more complicated example, consider \eqref{eq:def:minimal_implicational_logic/or/elim}. We have
  \begin{equation*}
    \begin{prooftree}
      \hypo{ \eqref{eq:def:minimal_implicational_logic/or/elim} }
      \hypo{ \varphi \synimplies \theta }
      \infer2[\ref{eq:def:def:axiomatic_derivation_system/mp}]{ (\psi \synimplies \theta) \synimplies ((\varphi \synvee \psi) \synimplies \theta) },

      \hypo{ \psi \synimplies \theta }
      \infer2[\ref{eq:def:def:axiomatic_derivation_system/mp}]{ (\varphi \synvee \psi) \synimplies \theta }.

      \hypo{ \varphi \synvee \psi }
      \infer2[\ref{eq:def:def:axiomatic_derivation_system/mp}]{ \theta }.
    \end{prooftree}
  \end{equation*}

  The assumptions of this derivations are \( \varphi \synimplies \theta \), \( \psi \synimplies \theta \) and \( \varphi \synvee \psi \). Instead of adding them directly as premises of the inference rule \eqref{eq:def:minimal_propositional_natural_deduction_system/or/elim}, we replace the conditional \( \synimplies \) with marked assumptions that correspond to \( \varphi \vdash \theta \) and \( \psi \vdash \theta \).

  We can prove that \eqref{eq:def:minimal_propositional_natural_deduction_system/or/elim} implies \eqref{eq:def:minimal_implicational_logic/or/elim} analogously to \eqref{eq:def:minimal_propositional_natural_deduction_system/and_intro_axiom_derivation}.

  The other two rules are again trivial to obtain from the corresponding axioms and vice versa.

  \SubProofOf{def:minimal_propositional_natural_deduction_system/iff} Analogous to what we have already shown.

  \SubProofOf{def:minimal_propositional_natural_deduction_system/negation} \eqref{eq:def:minimal_propositional_natural_deduction_system/neg/intro} is obtained from \eqref{eq:def:minimal_implicational_logic/neg/intro} by applying \eqref{eq:def:def:axiomatic_derivation_system/mp} once and \eqref{eq:def:minimal_propositional_natural_deduction_system/neg/elim} is obtained from \eqref{eq:def:minimal_implicational_logic/neg/elim} by applying \eqref{eq:def:def:axiomatic_derivation_system/mp} twice. Using the rules to derive the axioms is similar to \eqref{eq:def:minimal_propositional_natural_deduction_system/and_intro_axiom_derivation}.
\end{defproof}

\begin{proposition}\label{thm:conjunction_of_premises}
  In deduction systems that extend the \hyperref[def:minimal_propositional_natural_deduction_system]{minimal propositional natural deduction system}, we have \( \psi_1, \psi_1 \vdash \varphi \) if and only if \( (\psi_1 \synwedge \psi_2) \vdash \varphi \).
\end{proposition}
\begin{proof}
  \SufficiencySubProof If \( \psi_1, \psi_2 \vdash \varphi \), then
  \begin{equation*}
    \begin{prooftree}
      \hypo{ \psi_1 \synwedge \psi_2 }
      \infer1[\eqref{eq:def:minimal_propositional_natural_deduction_system/and/elim_right}]{ \psi_1 }

      \hypo{ \psi_1 \synwedge \psi_2 }
      \infer1[\eqref{eq:def:minimal_propositional_natural_deduction_system/and/elim_right}]{ \psi_2 }

      \infer2{}

      \ellipsis{}{ \varphi }
    \end{prooftree}
  \end{equation*}

  \NecessitySubProof If \( (\psi_1 \synwedge \psi_2) \vdash \varphi \), then
  \begin{equation*}
    \begin{prooftree}
      \hypo{ \psi_1 }
      \hypo{ \psi_2 }
      \infer2[\eqref{eq:def:minimal_propositional_natural_deduction_system/and/intro}]{ \psi_1 \synwedge \psi_2 }
      \ellipsis{}{ \varphi }
    \end{prooftree}
  \end{equation*}
\end{proof}

\begin{theorem}\label{thm:minimal_propositional_negation_laws}
  Assuming the \hyperref[def:minimal_implicational_logic]{minimal implicational logic}, we have the following derivations:
  \begin{center}
    \begin{forest}
      [
        {\eqref{eq:thm:classical_tautologies/dne}}
          [
            {\eqref{eq:thm:classical_tautologies/pierce}}
              [{\eqref{eq:thm:classical_tautologies/lem}}]
          ]
          [
            {\eqref{eq:thm:intuitionistic_tautologies/efq}}
              [{\eqref{eq:thm:intuitionistic_tautologies/lnc}}]
          ]
      ]
    \end{forest}
  \end{center}

  As it turns out, \eqref{eq:thm:intuitionistic_tautologies/lnc}, which is often associated with intuitionistic logic, is a theorem of \hyperref[def:minimal_logic]{minimal logic}.

  Conversely, \eqref{eq:thm:intuitionistic_tautologies/efq} and \eqref{eq:thm:classical_tautologies/lem} together can be used to derive \eqref{eq:thm:classical_tautologies/dne}.
\end{theorem}
\begin{proof}
  Most proofs are given in \cite[prop. 3]{DienerMcKubreJordens2016} and \cite[prop. 13]{DienerMcKubreJordens2016}. We will only show that \eqref{eq:thm:intuitionistic_tautologies/lnc} is strictly weaker than \eqref{eq:thm:intuitionistic_tautologies/efq}.

  For any formula \( \varphi \), we have the \hyperref[def:minimal_propositional_natural_deduction_system]{natural deduction} proof that \( \eqref{eq:thm:intuitionistic_tautologies/lnc} \) is a tautology:
  \begin{equation*}
    \begin{prooftree}[separation=3em]
      \hypo{ [\varphi \synwedge \synneg \varphi]^1 }
      \infer1[\ref{eq:def:minimal_propositional_natural_deduction_system/and/elim_left}]{ \varphi }

      \hypo{ [\varphi \synwedge \synneg \varphi]^1 }
      \infer1[\ref{eq:def:minimal_propositional_natural_deduction_system/and/elim_right}]{ \synneg \varphi }

      \infer2[\ref{eq:def:minimal_propositional_natural_deduction_system/neg/elim}]{ \synbot }

      \infer[left label=\( 1 \)]1[\ref{eq:def:minimal_propositional_natural_deduction_system/neg/intro}]{ \synneg (\varphi \synwedge \synneg \varphi) }
    \end{prooftree}
  \end{equation*}

  Hence, \eqref{eq:thm:intuitionistic_tautologies/lnc} is a theorem of \hyperref[def:minimal_logic]{minimal logic}. If it were to imply \eqref{eq:thm:intuitionistic_tautologies/efq}, then minimal and intuitionistic logic would be equivalent, which would contradict \cite[prop. 3]{DienerMcKubreJordens2016}. Therefore, \eqref{eq:thm:intuitionistic_tautologies/lnc} is indeed strictly weaker than \eqref{eq:thm:intuitionistic_tautologies/efq}.
\end{proof}

\begin{proposition}\label{thm:syntactic_contraposition}
  In the \hyperref[def:minimal_propositional_natural_deduction_system]{minimal propositional natural deduction system}, we have
  \begin{align}
    (\varphi \synimplies \psi) &\vdash (\synneg \psi \synimplies \synneg \varphi) \label{eq:thm:syntactic_contraposition/straight} \\
    \eqref{eq:thm:classical_tautologies/dne}, (\synneg \varphi \synimplies \synneg \psi) &\vdash (\psi \synimplies \varphi) \label{eq:thm:syntactic_contraposition/reverse}
  \end{align}
\end{proposition}
\begin{proof}
  We can easily derive \eqref{eq:thm:syntactic_contraposition/straight}:
  \begin{equation*}
    \begin{prooftree}
      \hypo{ \varphi \synimplies \psi }
      \hypo{ [\varphi]^1 }
      \infer2[\eqref{eq:def:minimal_propositional_natural_deduction_system/imp/elim}]{ \psi }

      \hypo{ [\synneg \psi]^2 }
      \infer2[\eqref{eq:def:minimal_propositional_natural_deduction_system/neg/elim}]{ \synbot }

      \infer[left label=\( 1 \)]1[\eqref{eq:def:minimal_propositional_natural_deduction_system/neg/intro}]{ \synneg \varphi }
      \infer[left label=\( 2 \)]1[\eqref{eq:def:minimal_propositional_natural_deduction_system/imp/intro}]{ \synneg \psi \synimplies \synneg \varphi }
    \end{prooftree}
  \end{equation*}

  We can similarly derive \eqref{eq:thm:syntactic_contraposition/reverse}, the only difference being in applying \eqref{eq:thm:classical_tautologies/dne}:
  \begin{equation*}
    \begin{prooftree}
      \hypo{ \eqref{eq:thm:classical_tautologies/dne} }

      \hypo{ \synneg \varphi \synimplies \synneg \psi }
      \hypo{ [\synneg \varphi]^1 }
      \infer2[\eqref{eq:def:minimal_propositional_natural_deduction_system/imp/elim}]{ \synneg \psi }

      \hypo{ [\psi]^2 }
      \infer2[\eqref{eq:def:minimal_propositional_natural_deduction_system/neg/elim}]{ \synbot }

      \infer[left label=\( 1 \)]1[\eqref{eq:def:minimal_propositional_natural_deduction_system/neg/intro}]{ \synneg \synneg \varphi }
      \infer2[\eqref{eq:def:def:axiomatic_derivation_system/mp}]{ \varphi }
      \infer[left label=\( 2 \)]1[\eqref{eq:def:minimal_propositional_natural_deduction_system/imp/intro}]{ \psi \synimplies \varphi }
    \end{prooftree}
  \end{equation*}
\end{proof}

\begin{definition}\label{def:intuitionistic_propositional_deduction_systems}\mcite[def. 2.1.1]{TroelstraSchwichtenberg2000}
  The \term{intuitionistic propositional natural deduction system} extends the \hyperref[def:minimal_propositional_natural_deduction_system]{minimal propositional natural deduction system} with the rule
  \begin{equation*}\taglabel[\logic{EFQ}]{eq:def:intuitionistic_propositional_deduction_systems/rules/efq}
    \begin{prooftree}
      \hypo{ \synbot }
      \infer1[\ref{eq:def:intuitionistic_propositional_deduction_systems/rules/efq}]{ \varphi }
    \end{prooftree}
  \end{equation*}
\end{definition}
\begin{comments}
  \item This corresponds to the axiom \eqref{eq:thm:intuitionistic_tautologies/efq}, which we can add to the \hyperref[def:minimal_implicational_logic]{minimal implicational logic}.
  \item The corresponding semantics are defined in \fullref{def:propositional_heyting_algebra_semantics} and their link with the deductive system is given in \fullref{thm:intuitionistic_propositional_logic_is_sound_and_complete}.
\end{comments}

\begin{definition}\label{def:propositional_heyting_algebra_semantics}\mcite[14]{BezhanishviliHolliday2019}
  We define \term{Heyting semantics} for propositional formulas similarly to how it is done with classical Boolean semantics in \fullref{def:propositional_entailment}, except that instead of using a \hyperref[def:boolean_algebra]{Boolean algebra} we use a more general \hyperref[def:heyting_algebra]{Heyting algebra}.

  Logical negations depend on complements in Boolean algebras. Since Heyting algebras do not have complements, we instead use \hyperref[def:heyting_algebra/pseudocomplement]{pseudocomplements}.

  Fix a Heyting algebra \( \mscrH = (H, \sup, \inf, T, F, \synimplies) \). \hyperref[def:propositional_valuation/interpretation]{Propositional interpretations} in Heyting semantics may take any value in \( X \), as can \hyperref[def:propositional_valuation/formula_valuation]{formula valuations}.

  Given an interpretation \( I \) and a formula \( \varphi \), we define \( \Bracks{\varphi}_I \) via \eqref{eq:def:propositional_valuation/formula_valuation}, the sole difference being that negation valuation is defined via the pseudocomplement:
  \begin{equation*}
    \Bracks{\synneg \psi}_I \coloneqq \widetilde{\Bracks{\varphi}_I}.
  \end{equation*}

  We say that \( I \) satisfies \( \varphi \) if \( \Bracks{\varphi}_I = T \). Thus, if the valuation of \( \varphi \) takes any value in \( H \setminus \set{ T } \), then \( I \) does not satisfy \( \varphi \), but that does not necessarily mean that \( I \) satisfies \( \synneg \varphi \).

  Then \( \Gamma \) entails \( \varphi \) if for every \( \psi \in \Gamma \) and every interpretation \( I \) in every Heyting algebra, we have \( \Bracks{\varphi}_I = \Bracks{\psi}_I \).

  Note that different Heyting algebras may provide different semantics --- see \fullref{ex:heyting_semantics_lem_counterexample} for an example of what is impossible in a Boolean algebra.
\end{definition}

\begin{example}\label{ex:heyting_semantics_lem_counterexample}
  Let \( \mscrX \) be an extension of the trivial Boolean algebra \( \set{ T, F } \) with the \enquote{indeterminate} symbol \( N \). That is, the domain of \( \mscrX \) is \( \set{ F, N, T } \) and the order is \( F \leq N \leq T \).

  The pseudocomplement of \( N \) is
  \begin{equation*}
    \widetilde{N}
    \reloset {\eqref{eq:def:heyting_algebra/pseudocomplement}} =
    \sup\set{ a \in X \given a \synwedge N = \synbot }
    =
    F.
  \end{equation*}

  Consider any \hyperref[def:propositional_valuation]{propositional interpretation} \( I \) such that \( I(P) = N \).

  Then the valuation of \eqref{eq:thm:classical_tautologies/lem} is
  \begin{equation*}
    \Bracks{P \synvee \synneg P}_I
    =
    \sup\set{ \Bracks{P}_I, \widetilde{\Bracks{P}_I} }
    =
    \sup\set{ N, \widetilde{N} }
    =
    \sup\set{ N, F }
    =
    N.
  \end{equation*}

  Therefore, \eqref{eq:thm:classical_tautologies/lem} does not hold.
\end{example}

\begin{theorem}[Intuitionistic propositional logic is sound and complete]\label{thm:intuitionistic_propositional_logic_is_sound_and_complete}\mcite[11]{BezhanishviliHolliday2019}
  The \hyperref[def:intuitionistic_propositional_deduction_systems]{intuitionistic propositional deductive system} is \hyperref[def:logical_framework/soundness]{sound} and \hyperref[def:logical_framework/completeness]{complete} with respect to both \hyperref[def:propositional_heyting_algebra_semantics]{Heyting semantics}. To elaborate,
  \begin{thmenum}
    \thmitem{thm:intuitionistic_propositional_logic_is_sound_and_complete/sound} If \( \vdash \varphi \), then \( \vDash \varphi \) for every Heyting algebra.
    \thmitem{thm:intuitionistic_propositional_logic_is_sound_and_complete/complete} If \( \vDash \varphi \) in every finite Heyting algebra, then \( \vdash \varphi \).
  \end{thmenum}
\end{theorem}

\begin{definition}\label{def:propositional_topological_semantics}\mcite[15]{BezhanishviliHolliday2019}
  Since arbitrary \hyperref[def:heyting_algebra]{Heyting algebras} can be cumbersome to come up with when used for \hyperref[def:propositional_heyting_algebra_semantics]{propositional Heyting semantics}, we can instead utilize \fullref{ex:def:heyting_algebra/topology} and define \term{topological semantics} for some nonempty \hyperref[def:topological_space]{topological space}.

  The truth values of interpretations and valuations are then open sets in some topological space and a formula is said to be valid if its valuation is the whole space.
\end{definition}

\begin{example}\label{ex:topological_semantics_lem_counterexample}
  Let \( U \) be an open set in the standard topology in \( \BbbR \). We will examine \eqref{eq:thm:classical_tautologies/lem} with respect to \hyperref[def:propositional_topological_semantics]{topological semantics} for \( \BbbR \). Due to \fullref{ex:def:heyting_algebra/topology}, given any \hyperref[def:propositional_valuation]{propositional interpretation} \( I \) such that \( I(P) = U \), we have
  \begin{equation*}
    \Bracks{P \synvee \synneg P}_I
    =
    \Bracks{P}_I \cup \widetilde{\Bracks{P}_I}
    =
    U \cup \widetilde{U}
    =
    U \cup \Int(\BbbR \setminus U).
  \end{equation*}

  If \( U = \varnothing \), then \( \Bracks{P \synvee \synneg P}_I = \BbbR \) and \eqref{eq:thm:classical_tautologies/lem} holds. If \( U = (0, 1) \), then \( \Bracks{P \synvee \synneg P}_I = \BbbR \setminus \set{ 0, 1 } \) and \eqref{eq:thm:classical_tautologies/lem} does not hold.

  Compare this result with \fullref{ex:heyting_semantics_lem_counterexample}.
\end{example}

\begin{concept}\label{con:brouwer_heyting_kolmogorov_interpretation}\mcite[sec. 1.3.1]{TroelstraSchwichtenberg2000}
  Another semantical framework for the \hyperref[def:intuitionistic_propositional_deduction_systems]{intuitionistic propositional deductive system} is the \term{Brouwer-Heyting-Kolmogorov interpretation}.

  It uses a less formal approach than \hyperref[def:propositional_heyting_algebra_semantics]{Heyting algebra semantics} that is based on the notion of a \enquote{construction}, which is also why it is sometimes called \term{constructive logic}.

  \begin{thmenum}
    \thmitem{con:brouwer_heyting_kolmogorov_interpretation/atomic} We assume that we know what constitutes a construction of propositional variables.
    \thmitem{con:brouwer_heyting_kolmogorov_interpretation/constant} There is no construction of \( \synbot \) and no construction of \( \syntop \) is needed.
    \thmitem{con:brouwer_heyting_kolmogorov_interpretation/disjunction} A construction of \( \psi_1 \synvee \psi_2 \) is a pair \( (k, M) \), where \( k = 1, 2 \) and \( M \) is a construction of \( \psi_m \) if and only if \( k = m \). The notion of a pair here is informal.
    \thmitem{con:brouwer_heyting_kolmogorov_interpretation/conjunction} A construction of \( \psi_1 \synwedge \psi_2 \) is a pair \( (M_1, M_2) \), where \( M_k \) is a construction of \( \psi_k \) for \( k = 1, 2 \).
    \thmitem{con:brouwer_heyting_kolmogorov_interpretation/conditional} A construction of \( \psi_1 \synimplies \psi_2 \) is a function that converts a construction of \( \psi_1 \) into a construction of \( \psi_2 \). The notion of a function here is informal.
  \end{thmenum}

  The negation \( \synneg\psi \) that corresponds to pseudocomplements in Heyting algebra semantics corresponds to the metastatement \enquote{a construction of \( \psi \) is impossible} under the Brouwer-Heyting-Kolmogorov interpretation.

  If the set \( \Gamma \) of formulas does not derive \( \varphi \), we say that \( \varphi \) is non-constructive under the axioms \( \Gamma \).
\end{concept}

\begin{remark}\label{rem:brouwer_heyting_kolmogorov_interpretation_compatibility}
  Since the \hyperref[con:brouwer_heyting_kolmogorov_interpretation]{Brouwer-Heyting-Kolmogorov interpretation} is not very formal, we cannot properly prove its soundness or completeness with respect to the \hyperref[def:intuitionistic_propositional_deduction_systems]{intuitionistic propositional deductive system}.

  Nevertheless, we generally accept the interpretation and conflate \enquote{constructive} and \enquote{intuitionistic} statements.
\end{remark}

\begin{example}\label{ex:con:brouwer_heyting_kolmogorov_interpretation/well_ordering_principle_zfc}
  \Fullref{thm:well_ordering_theorem} in \hyperref[def:zfc]{\( \logic{ZFC} \)} does not provide a way to well-order an arbitrary set. The theorem relies on the axiom of choice, whose consequence \fullref{thm:diaconescu_goodman_myhill_theorem} implies the law of the excluded middle (\logic{LEM}) assuming the axioms of \logic{ZFC}.

  Since \logic{LEM} may not hold in intuitionistic logic, it follows that both \fullref{thm:well_ordering_theorem} and the axiom of choice itself should not in general hold under the Brouwer-Heyting-Kolmogorov interpretation, hence by the terminology in \fullref{con:brouwer_heyting_kolmogorov_interpretation}, \fullref{thm:well_ordering_theorem} is a non-constructive theorem.
\end{example}

\begin{definition}\label{def:classical_propositional_deduction_systems}
  In order to obtain a deductive system that matches \hyperref[def:propositional_entailment]{classical propositional semantics}, we may extend the \hyperref[def:minimal_propositional_natural_deduction_system]{minimal propositional natural deduction system} with the rule
  \begin{equation*}\taglabel[\logic{DNE}]{eq:def:classical_propositional_deduction_systems/rules/dne}
    \begin{prooftree}
      \hypo{ [\synneg \varphi]^n }
      \ellipsis {} { \synbot }
      \infer[left label=\( n \)]1[\ref{eq:def:classical_propositional_deduction_systems/rules/dne}]{ \varphi }
    \end{prooftree}
  \end{equation*}

  This corresponds to the axiom \eqref{eq:thm:classical_tautologies/dne}, which we can add to the \hyperref[def:minimal_implicational_logic]{minimal implicational logic}. As per \fullref{thm:minimal_propositional_negation_laws}, we can instead add \eqref{eq:thm:classical_tautologies/lem} to the \hyperref[def:intuitionistic_propositional_deduction_systems]{intuitionistic propositional Hilbert system}, since
  \begin{equation*}
    \eqref{eq:thm:classical_tautologies/lem}, \eqref{eq:thm:intuitionistic_tautologies/efq} \vdash \eqref{eq:thm:classical_tautologies/dne}
  \end{equation*}

  We call this, very simply, the (classical) \term{propositional deductive system}.
\end{definition}

\begin{theorem}[Glivenko's double negation theorem]\label{thm:glivenkos_double_negation_theorem}\mcite[4]{BezhanishviliHolliday2019}
  A formula \( \varphi \) is derivable in the \hyperref[def:classical_propositional_deduction_systems]{classical propositional natural deduction system} if and only if it's double negation \( \synneg \synneg \varphi \) is derivable in the \hyperref[def:intuitionistic_propositional_deduction_systems]{intuitionistic propositional natural deduction system}.
\end{theorem}
