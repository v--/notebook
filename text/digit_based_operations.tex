\section{Digit-based operations}\label{sec:digit_based_operations}

\paragraph{Integer radix expansions}

\begin{definition}\label{def:integer_radix_expansion}
  Similarly to how we have defined real number radix expansions in \fullref{def:real_number_radix_expansion}, we define a radix \( b \) expansion for the integer \( n \) as a \hyperref[def:univariate_polynomial]{univariate polynomial}
  \begin{equation*}
    p(X) = \sgn(n) \cdot \sum_{k=0}^m a_k X^k,
  \end{equation*}
  where \( 0 \leq a_k < b \) for all \( a_k \) and \( n = p(b) \).
\end{definition}
\begin{comments}
  \item Introducing this notion is not strictly necessary, however we find it more convenient than real number radix expansions as defined in \fullref{def:real_number_radix_expansion}.

  \item Due to compatibility with real number radix expansions shown in \fullref{thm:integer_and_real_radix_expansions}, \fullref{thm:radix_expansion_non_uniqueness} implies that integer expansions are unique.
\end{comments}

\begin{proposition}\label{thm:integer_and_real_radix_expansions}
  For an integer \( n \), the \hyperref[def:formal_laurent_series]{Laurent series}
  \begin{equation}\label{eq:thm:integer_and_real_radix_expansions/series}
    \sgn(n) \cdot \sum_{k=-m}^0 \frac {a_k} {b^k}
  \end{equation}
  is a radix \( b \) \hyperref[def:real_number_radix_expansion]{real number expansion} of \( n \) if and only if
  \begin{equation}\label{eq:thm:integer_and_real_radix_expansions/polynomial}
    \sgn(n) \cdot \sum_{k=0}^m a_k b^k
  \end{equation}
  is an \hyperref[def:integer_radix_expansion]{integer radix expansion}.
\end{proposition}
\begin{comments}
  \item The gist here, discussed in \fullref{def:endianness}, is that the real number expansion gives the \hyperref[def:endianness]{little-endian order} of the coefficients while integer expansion \hyperref[def:endianness]{big-endian order}.
\end{comments}
\begin{proof}
  Trivial.
\end{proof}

\begin{algorithm}[Integer radix expansion]\label{alg:integer_radix_expansion}
  We will give a simplified form of \fullref{alg:real_number_radix_expansion} for \hyperref[def:integer_radix_expansion]{integer radix expansions}. Fix an integer \( n \) and a radix \( b \geq 2 \).

  \begin{thmenum}
    \thmitem{alg:integer_radix_expansion/negative} As in \fullref{alg:real_number_radix_expansion/negative}, if \( n < 0 \) we build an expansion for \( -n \) and then multiply it by \( -1 \).

    \thmitem{alg:integer_radix_expansion/base} If \( 0 \leq n < b \), halt the algorithm with \( n \) regarded as a constant polynomial.

    \thmitem{alg:integer_radix_expansion/recursive} Finally, if \( n > b \), let \( n = bq + r \) be the result of \fullref{alg:integer_division} and let \( p(X) \) be the expansion of \( q \). Then
    \begin{equation*}
      X \cdot p(X) + r
    \end{equation*}
    is an expansion of \( n \).
  \end{thmenum}
\end{algorithm}
\begin{comments}
  \item This algorithm can be found as \identifier{arithmetic.bases.get_integer_expansion} in \cite{notebook:code}.
\end{comments}

\paragraph{Digit-based integer arithmetic}

We will present several algorithms which allow performing integer arithmetic using radix \( b \) expansions knowing only how the operations behave for positive integers smaller than \( b \). This is especially convenient for binary arithmetic, and is the basis upon which computer arithmetic is implemented.

\begin{algorithm}[Addition with carrying]\label{alg:addition_with_carrying}
  Fix a radix \( b \) and consider the \hyperref[def:integer_radix_expansion]{integer expansions}
  \begin{equation*}
    n = \sgn(n) \cdot \sum_{k=0}^p a_k b^k
  \end{equation*}
  and
  \begin{equation*}
    m = \sgn(m) \cdot \sum_{k=0}^l c_k b^k.
  \end{equation*}

  We will find the digits of the expansion
  \begin{equation*}
    n + m = \sgn(n + m) \cdot \sum_{k=0}^{\max\set{ p, l } + 1} d_k b^k.
  \end{equation*}

  \begin{thmenum}
    \thmitem{alg:addition_with_carrying/n_guard} If \( n < 0 \), we can apply the algorithm to \( -n \) and \( -m \) since
    \begin{equation*}
      n + m = -(-n - m).
    \end{equation*}

    Thus, we can suppose that \( n \geq 0 \).

    \thmitem{alg:addition_with_carrying/m_guard} If \( m < 0 \) and \( \abs{m} > \abs{n} \), we instead apply the algorithm to \( -m \) and \( -n \).

    Thus, we can suppose that \( \abs{m} \leq \abs{n} \) and \( r \leq p \).

    Along with \( n \geq 0 \), this ensures that \( n + m \) is nonnegative.

    \thmitem{alg:addition_with_carrying/init} We start with \( r_{-1} \coloneqq 0 \).

    \thmitem{alg:addition_with_carrying/step} At step \( k = 0, 1, \ldots \), let \( q_k \) and \( d_k \) be the quotient and remainder of dividing via \fullref{alg:integer_division} the integer
    \begin{equation}\label{eq:alg:addition_with_carrying/step/quot}
      a_k + \sgn(m) \cdot c_k + q_{k - 1}
    \end{equation}
    by \( b \).

    We will show that \( q_k \) is either \( -1 \), \( 0 \) or \( 1 \), so, instead of relying on \fullref{alg:integer_division}, we can compute \( q_k \) by simply checking whether \( a_k + \sgn(m) \cdot c_k + q_{k - 1} \) is larger than \( q - 1 \) or smaller than \( 0 \).
  \end{thmenum}
\end{algorithm}
\begin{comments}
  \item The term \( q_k \) \enquote{carries} either \( -1 \), \( 0 \) or \( 1 \) to be used in the next step.
  \item This algorithm can be found as \identifier{arithmetic.bases.add_with_carrying} in \cite{notebook:code}.
\end{comments}
\begin{defproof}
  \SubProof{Proof that \( q_k \) is either \( -1 \), \( 0 \) or \( 1 \)} We will show by induction on \( k \) that \( q_k \) is either \( -1 \), \( 0 \) or \( 1 \). Note that, since \( b \geq 2 \), we have
  \begin{equation*}
    a_k + \sgn(m) \cdot c_k \leq (b - 1) + (b - 1) = 2b - 2 \leq b^2 - 2
  \end{equation*}
  and
  \begin{equation*}
    a_k + \sgn(m) \cdot c_k \geq 0 - (b - 1) = 1 - b.
  \end{equation*}

  \begin{itemize}
    \item In the base case \( k = 0 \), we divide \( a_0 + \sgn(m) \cdot c_0 \) by \( b \).
    \begin{itemize}
      \item If \( m \geq 0 \), we use that \( 0 \leq a_0 + c_0 < b^2 \) to conclude that \( q_0 \) is either \( 0 \) or \( 1 \).
      \item If \( m < 0 \), we use that \( a_k - c_k > 1 - b \) to conclude that \( q_0 \) is \( 0 \) if \( a_k \geq \abs{c_k} \) or \( -1 \) if \( a_k < \abs{c_k} \).
    \end{itemize}

    \item For \( k > 0 \), we must divide \( a_k + \sgn(m) \cdot c_k + q_{k-1} \).
    \begin{itemize}
      \item If \( m \geq 0 \), we use the inductive hypothesis on \( q_{k-1} \) to obtain
      \begin{equation*}
        -b < -1 \leq a_k + c_k + q_{k-1} \leq b^2 - 2 + q_{k-1} \leq b^2 - 1 < b^2
      \end{equation*}
      and conclude that \( q_k \) is either \( -1 \), \( 0 \) or \( 1 \).

      \item If \( m < 0 \), we instead have
      \begin{equation*}
        -b \leq -b + 1 + q_{k-1} \leq a_k - c_k + q_{k-1} \leq (b - 1) + (-1) + q_{k-1} \leq b - 1,
      \end{equation*}
      hence \( q_k \) is either \( -1 \) or \( 0 \).
    \end{itemize}
  \end{itemize}

  \SubProof{Proof that the expansion polynomial degree is at most \( \max\set{ p, l } + 1 \)} \Fullref{alg:addition_with_carrying/m_guard} ensures that \( l \leq p \), thus \( \max\set{ p, l } = p \). We must show that \( d_k = 0 \) when \( k > p + 1 \).

  Since \( n \geq \abs{m} \), we have \( a_p \geq c_p \) with equality holding if \( q_{p - 1} = 0 \) --- we easily obtain a contradiction otherwise.

  \begin{itemize}
    \item If \( a_p > c_p \), then \( a_p + \sgn(p) \cdot c_p \geq 1 \), and, since \( q_{p - 1} \geq -1 \),
    \begin{equation*}
      a_p + \sgn(p) \cdot c_p + q_{p - 1} \geq 0.
    \end{equation*}

    \item If \( a_p = c_p \), then \( q_{p - 1} = 0 \) as discussed, and again
    \begin{equation*}
      a_p + \sgn(p) \cdot c_p + q_{p - 1} \geq 0.
    \end{equation*}
  \end{itemize}

  In both case, we have excluded the possibility that \( q_p = -1 \), hence \( q_p = 0 \) or \( q_p = 1 \).

  We have
  \begin{equation*}
    b \cdot q_{p + 1} = (\underbrace{a_{p + 1} + \sgn(m) \cdot c_{p + 1}}_0 + q_p) - d_{p + 1},
  \end{equation*}

  Since \( 0 \leq d_{p + 1} \leq b - 1 \), it follows that
  \begin{equation*}
    q_p - (b - 1) \leq b \cdot q_{p + 1} \leq q_p,
  \end{equation*}
  and, since \( q_p \) is either \( 0 \) or \( 1 \),
  \begin{equation*}
    -(b - 1) \leq b \cdot q_{p + 1} \leq 1.
  \end{equation*}

  It remains for \( q_{p + 1} \) to be \( 0 \). Then
  \begin{equation*}
    d_{p + 2} = \underbrace{q_{p + 1}}_{0} - b \cdot q_{p + 2},
  \end{equation*}
  which is only possible if \( d_{p + 2} = q_{p + 2} = 0 \).

  For \( k > p + 1 \), we can analogously show that \( d_k = 0 \).

  \SubProof{Proof that \( d_k \) are the coefficients of the expansion of \( n + m \)} Let \( n \geq 0 \) and \( \abs{m} \leq n \).

  We will use induction on \( s \) to show that
  \begin{equation*}
    \sum_{k=0}^s a_k b^k + \sgn(m) \cdot \sum_{k=0}^s c_k b^k = \sum_{k=0}^s d_k b^k + q_s b^{s+1}.
  \end{equation*}

  \begin{itemize}
    \item The base case \( s = 0 \) requires us to show that
    \begin{equation*}
      a_0 + \sgn(m) \cdot c_0 = d_0 + q_0 \cdot b,
    \end{equation*}
    which holds by definition.

    \item Suppose that the inductive hypothesis holds for \( s - 1 \). Then
    \begin{align*}
      \sum_{k=0}^s d_k b^k + q_s b^{s+1}
      &=
      \sum_{k=0}^{s-1} d_k b^k + \underbrace{d_k b^s + q_s b^{s+1}}_{b^s(a_s + \sgn(m) \cdot c_s)}
      \reloset{\T{ind.}} = \\ &=
      \parens[\Big]{ \sum_{k=0}^{s-1} a_k b^k + \sgn(m) \cdot \sum_{k=0}^{s-1} c_k b^k } + \parens[\Big]{ a_s b^s + \sgn(m) \cdot c_s b^s }.
    \end{align*}
  \end{itemize}

  We have already shown that \( d_k = q_k = 0 \) for \( k > p + 1 \), and clearly \( a_k = 0 \) for \( k > p \) and \( c_k = 0 \) for \( k > s \), thus
  \begin{equation*}
    \sum_{k=0}^p a_k b^k + \sgn(m) \cdot \sum_{k=0}^r c_k b^k = \sum_{k=0}^{p + 1} d_k b^k.
  \end{equation*}
\end{defproof}

\begin{algorithm}[Single-digit multiplication with carrying]\label{alg:single_digit_multiplication_with_carrying}
  Fix a radix \( b \) and consider the \hyperref[def:integer_radix_expansion]{integer expansion}
  \begin{equation*}
    n = \sgn(n) \cdot \sum_{k=0}^p a_k b^k.
  \end{equation*}

  Let \( m \) be any integer whose absolute value is smaller than \( b \).

  We will find the digits of the expansion
  \begin{equation*}
    nm = \sgn(nm) \cdot \sum_{k=0}^{p + 1} d_k b^k.
  \end{equation*}

  \begin{thmenum}
    \thmitem{alg:single_digit_multiplication_with_carrying/double_neg_guard} If \( n < 0 \) and \( m < 0 \), we can apply the algorithm to \( -n \) and \( -m \) since
    \begin{equation*}
      nm = (-n) \cdot (-m).
    \end{equation*}

    We have thus reduced this case to nonnegative \( n \) and \( m \).

    \thmitem{alg:single_digit_multiplication_with_carrying/single_neg_guard} If \( n \) and \( m \) have different \hi{nonzero} signs, we can apply the algorithm to \( \abs{n} \) and \( \abs{m} \) since
    \begin{equation*}
      nm = -\abs{n} \cdot \abs{m}.
    \end{equation*}

    We have thus reduced this case to nonnegative \( n \) and \( m \).

    \thmitem{alg:single_digit_multiplication_with_carrying/init} If both \( n \) and \( m \) are nonnegative, we start with \( r_{-1} \coloneqq 0 \), similarly to \fullref{alg:addition_with_carrying/init}.

    \thmitem{alg:single_digit_multiplication_with_carrying/step} At step \( k = 0, 1, \ldots \), let \( q_k \) and \( d_k \) be the quotient and remainder of dividing via \fullref{alg:integer_division} the integer
    \begin{equation}\label{eq:alg:single_digit_multiplication_with_carrying/step/quot}
      m \cdot a_k + q_{k - 1}
    \end{equation}
    by \( b \).

    We will show that \( 0 \leq q_k < m \). Then \( d_k \) and \( q_k \) are the first two coefficients in the radix \( b \) expansion of \( m \cdot a_k + q_{k-1} \). Thus, if we have the expansion available, we can take \( d_k \) and \( q_k \) from there instead of relying on \fullref{alg:integer_division}.
  \end{thmenum}
\end{algorithm}
\begin{comments}
  \item This algorithm can be found as \identifier{arithmetic.bases.single_digit_mult_with_carrying} in \cite{notebook:code}.
\end{comments}
\begin{defproof}
  The proof of correctness is similar, but much simpler than that of \fullref{alg:integer_radix_expansion}.

  The guard steps \fullref{alg:single_digit_multiplication_with_carrying/double_neg_guard} and \fullref{alg:single_digit_multiplication_with_carrying/single_neg_guard} allow us to assume that \( n \geq 0 \) and \( m \geq 0 \).

  \SubProof{Proof that \( 0 \leq q_k < m \)} Since \( 0 \leq a_k \leq b - 1 \) and \( 0 \leq m \leq b - 1 \), we have
  \begin{equation*}
    m \cdot a_k \leq (b - 1) \cdot m = bm - m.
  \end{equation*}

  \begin{itemize}
    \item In the base case \( k = 0 \), the quotient of dividing \( m \cdot a_0 \) by \( b \) is thus strictly less than \( m \) because \( m < b \).

    \item For \( k > 0 \), we must divide \( m \cdot a_k + q_{k-1} \) by \( b \). By the inductive hypothesis we have \( 0 \leq q_{k-1} < m \), thus
    \begin{equation*}
      m \cdot a_k + q_{k-1} \leq (b - 1) \cdot m + (m - 1) = bm - 1.
    \end{equation*}

    It follows that the quotient \( q_k \) is at most \( m - 1 \).
  \end{itemize}

  \SubProof{Proof that the expansion polynomial degree is at most \( p + 1 \)} We will show that \( d_k = 0 \) when \( k > p + 1 \).

  For \( k = p + 1 \), since \( a_{p + 1} = 0 \), to obtain \( q_{p + 1} \) we divide \( q_p \) by \( b \). But \( q_p < m < b \), thus \( q_{p + 1} = 0 \).

  Then \( d_{p + 2} \) is the remainder of dividing \( q_{p + 1} = 0 \) by \( b \), that is, \( d_{p + 2} = 0 \). This also holds for all \( k > p + 1 \).

  \SubProof{Proof that \( d_k \) are the coefficients of the expansion of \( nm \)}

  We will use induction on \( s \) to show that
  \begin{equation*}
    m \cdot \sum_{k=0}^s a_k b^k = \sum_{k=0}^s d_k b^k + q_s b^{s+1}.
  \end{equation*}

  \begin{itemize}
    \item The base case \( s = 0 \) requires us to show that
    \begin{equation*}
      m \cdot a_0 = d_0 + q_0 \cdot b,
    \end{equation*}
    which holds by definition.

    \item Suppose that the inductive hypothesis holds for \( s - 1 \). Then
    \begin{equation*}
      \sum_{k=0}^s d_k b^k + q_s b^{s+1}
      =
      \sum_{k=0}^{s-1} d_k b^k + \underbrace{d_k b^s + q_s b^{s+1}}_{b^s(m \cdot a_s)}
      \reloset{\T{ind.}} =
      (m \cdot \sum_{k=0}^{s-1} a_k b^k) + m \cdot a_s b^s.
    \end{equation*}
  \end{itemize}

  We have already shown that \( d_k = q_k = 0 \) for \( k > p + 1 \), and clearly \( a_k = 0 \) for \( k > p \), thus
  \begin{equation*}
    m \cdot \sum_{k=0}^p a_k b^k = \sum_{k=0}^{p + 1} d_k b^k.
  \end{equation*}
\end{defproof}

\begin{algorithm}[Multi-digit multiplication with carrying]\label{alg:multi_digit_mult_with_carrying}
  Fix a radix \( b \) and consider the \hyperref[def:integer_radix_expansion]{integer expansions}
  \begin{equation*}
    n = \sgn(n) \cdot \sum_{k=0}^p a_k b^k
  \end{equation*}
  and
  \begin{equation*}
    m = \sgn(m) \cdot \sum_{k=0}^l c_k b^k.
  \end{equation*}

  We will only use the power \( p \) in the expansion of \( n \), not its coefficients.

  Rather than finding the individual digits of \( nm \), we will use a combination of \fullref{alg:addition_with_carrying} and \fullref{alg:single_digit_multiplication_with_carrying}.

  We start with the observation that
  \begin{equation*}
    nm = \sgn(m) \cdot \sum_{k=0}^l (n \cdot c_k) b^k.
  \end{equation*}

  We will construct a sequence \( f_{-1}, f_0, \ldots \) of integer expansion polynomials such that
  \begin{equation*}
    f_s(b) = \sgn(m) \cdot \sum_{k=0}^s (n \cdot c_k) b^k,
  \end{equation*}
  where \( f_s \) is either zero or its polynomial degree is at most \( p + s + 2 \).

  Then \( nm = f_l(b) \).

  \begin{thmenum}
    \thmitem{alg:multi_digit_multiplication_with_carrying/init} Start with the expansion \( f_{-1} = 0 \).

    \thmitem{alg:multi_digit_multiplication_with_carrying/step} At step \( s = 0, 1, \ldots \), use \fullref{alg:single_digit_multiplication_with_carrying} to obtain the expansion
    \begin{equation*}
      n \cdot c_s = \sgn(n) \cdot \sum_{k=0}^{p + 1} d_k b^k.
    \end{equation*}

    Define \( f_s \) to be the sum of \( f_{s-1} \) with
    \begin{equation*}
      n \cdot c_s \cdot b^s = \sgn(n) \cdot \sum_{k=s}^{s + p + 1} d_{k - s} b^k
    \end{equation*}
    obtained via \fullref{alg:addition_with_carrying}.

    Then \( f_s \) is an expansion of
    \begin{equation*}
      \sgn(m) \cdot \sum_{k=0}^s (n \cdot c_k) b^k.
    \end{equation*}

    If \( f_{s-1} \) is zero of if the polynomial degree of \( f_{s-1} \) is at most \( p + s \), \fullref{alg:addition_with_carrying} implies that the degree of \( f_s \) is at most
    \begin{equation*}
      \max\set{ p + s, s + p + 1 } + 1 = p + s + 2.
    \end{equation*}
  \end{thmenum}
\end{algorithm}
\begin{comments}
  \item This algorithm can be found as \identifier{arithmetic.bases.multi_digit_mult_with_carrying} in \cite{notebook:code}.
\end{comments}

\begin{algorithm}[Long division]\label{alg:long_division}
  Fix a radix \( b \) and consider the \hyperref[def:integer_radix_expansion]{integer expansions}
  \begin{equation*}
    n = \sgn(n) \cdot \sum_{k=0}^p a_k b^k
  \end{equation*}
  and
  \begin{equation*}
    m = \sgn(m) \cdot \sum_{k=0}^l c_k b^k.
  \end{equation*}

  We will only use the powers \( p \) and \( l \) of the expansions of \( n \) and \( m \), not their coefficients.

  \Fullref{alg:integer_division} gives us integers \( q \) and \( r \) such that \( 0 \leq r < m \) and
  \begin{equation*}
    n = qm + r.
  \end{equation*}

  We will find the digits of the expansions
  \begin{equation*}
    q = \sgn(q) \cdot \sum_{k=0}^{p - l} d_k b^k
  \end{equation*}
  and
  \begin{equation*}
    r = \sum_{k=0}^l e_k b^k.
  \end{equation*}

  \begin{thmenum}
    \thmitem{alg:long_division/n_guard} If \( n < 0 \), let \( q \) and \( r \) be the quotient and remainder of the division of \( \abs{n} \) by \( \abs{m} \). We have
    \begin{equation*}
      n = -\abs{n} = -(q \cdot \abs{m} + r) = q \cdot (-\abs{m}) - r = (q + 1) \cdot (-\abs{m}) + (\abs{m} - r).
    \end{equation*}

    Then \( -\sgn(m) * (q + 1) \) is the quotient and \( \abs{m} - r \) is the remainder of dividing \( n \) by \( m \).

    We have thus reduced this case to nonnegative \( n \) and \( m \).

    \thmitem{alg:long_division/m_guard} If \( n \geq 0 \) and \( m < 0 \), let \( q \) and \( r \) be the quotient and remainder of the division of \( n \) by \( -m \). We have
    \begin{equation*}
      n = q \cdot (-m) + r = (-q) \cdot m + r.
    \end{equation*}

    Then \( -q \) is the quotient and \( r \) is the remainder of dividing \( n \) by \( m \).

    We have thus reduced this case to nonnegative \( n \) and \( m \).

    \thmitem{alg:long_division/init} Finally, if \( n \geq 0 \) and \( m \geq 0 \), let \( r_{-1} \coloneqq n \) and \( q_{-1} \coloneqq 0 \).

    \thmitem{alg:long_division/step} At step \( s = 0, 1, \ldots \), let \( k \coloneqq p - r - s \) and define
    \begin{equation*}
      d_k \coloneqq \max\set{ d \leq b - 1 \given d m b^k \leq r_{s - 1} }.
    \end{equation*}

    Let \( r_s \) be the sum of \( r_{s - 1} \) and \( -d_k m b^k \). The sum can be obtained via \fullref{alg:addition_with_carrying}, while multiplying \( m b^k \) by \( -d_k \) can be achieved via \fullref{alg:single_digit_multiplication_with_carrying}.

    \thmitem{alg:long_division/finish} Finally, at step \( p - l \), the remainder is \( r = r_{p - l} \), as we will show, and we have used \fullref{alg:addition_with_carrying} and \fullref{alg:single_digit_multiplication_with_carrying} to explicitly compute its expansion at every step.

    We have also constructed all coefficients of the expansion of \( q \).
  \end{thmenum}
\end{algorithm}
\begin{comments}
  \item This algorithm can be found as \identifier{arithmetic.bases.long_integer_division} in \cite{notebook:code}.
\end{comments}
\begin{defproof}
  The guard steps \fullref{alg:long_division/n_guard} and \fullref{alg:long_division/m_guard} allow us to assume that \( n \geq 0 \) and \( m \geq 0 \).

  \SubProof{Proof that \( r_{s - p} \) is the remainder} By construction, we have
  \begin{equation*}
    d_k m b^k \leq r_{s - 1},
  \end{equation*}
  and we can add \( -d_k m b^k \) to both sides to conclude that
  \begin{equation*}
    0 \leq r_{s - 1} - d_k m b^k = r_s.
  \end{equation*}

  Then \( r_s \) is a nonnegative integer at every step \( s \). We must also show that \( r_s < m b^{p - l - s} \). If \( r_s \geq m b^{p - l - s} \), then
  \begin{equation*}
    r_{s-1} - d_{p - l - s} \cdot m \cdot b^{p - l - s} \geq m \cdot b^{p - l - s}
  \end{equation*}
  and thus
  \begin{equation*}
    r_{s-1} - (d_{p - l - s} - 1) \cdot m \cdot b^{p - l - s} \geq 0,
  \end{equation*}
  which contradicts the maximality of \( d_{p - l - s} \)

  Therefore, \( r_s < m b^{p - l - s} \), which implies that \( r_{p - l} < m \). Then \( r = r_{p - l} \) is the remainder of dividing \( n \) by \( m \).

  \SubProof{Proof that \( d_k \) are the coefficients of \( q \)}
  We will use induction on \( s \) to show that
  \begin{equation}\label{eq:alg:long_division/proof/hypothesis}
    r_s = n - m \cdot \sum_{k=p-l-s}^{p-l} d_k b^k.
  \end{equation}

  \begin{itemize}
    \item In the base case \( s = 0 \), we have
    \begin{equation*}
      r_0 = \underbrace{r_{-1}}_n - d_{p-l} \cdot m \cdot b^{p-l},
    \end{equation*}
    hence \eqref{eq:alg:long_division/proof/hypothesis} holds.

    \item For \( s > 0 \), we have
    \begin{equation*}
      r_s
      =
      r_{s-1} - d_{p-l-s} \cdot m \cdot b^{p-l}
      \reloset{\T{ind.}} =
      n - m \cdot \sum_{k=p-l-s+1}^{p-l} d_k b^k - d_{p-l-s} \cdot m \cdot b^{p-l}
      =
      n - m \cdot \sum_{k=p-l-s}^{p-l} d_k b^k.
    \end{equation*}
  \end{itemize}

  Then
  \begin{equation*}
    r_{p-l} = n - m \cdot \sum_{k=0}^{p-l} d_k b^k.
  \end{equation*}

  Since \( r_{p-l} < m \), we have obtained the unique quotient and remainder of dividing \( n \) by \( m \). Since \( 0 \leq d_k < b \), we conclude that these are the coefficients of the radix \( b \) expansion of \( q \).
\end{defproof}
