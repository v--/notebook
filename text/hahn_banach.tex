\section{The Hahn-Banach theorem}\label{sec:hahn_banach}

The Hahn-Banach theorem is an important result that can be stated differently and in different levels of generality.

\begin{theorem}[Geometric Hahn-Banach theorem/Mazur's theorem]\label{thm:geometric_hahn_banach}\mcite[24]{ИоффеТихомиров1974ЭкстремЗадачи}
  Fix a \hyperref[def:topological_vector_space]{topological vector space} \( X \). Let \( A \subseteq X \) be an open \hyperref[def:convex_hull]{convex} set and \( L \subseteq X \) be a subspace that is disjoint from \( A \). Then there exists a continuous linear functional \( x^* \in X^* \) such that
  \begin{equation*}
    \begin{aligned}
      \real \inprod{x^*} x > 0, &x \in A \\
      \real \inprod{x^*} x = 0, &x \in L
    \end{aligned}
  \end{equation*}

  See \fullref{rem:linear_functionals_over_c} for a justification of only considering the real part of \( x^* \).
\end{theorem}

\begin{corollary}\label{thm:hahn_banach_implies_functionals_vanish_nowhere}\mcite[24]{ИоффеТихомиров1974ЭкстремЗадачи}
  The \hyperref[def:dual_vector_space]{dual} of a Hausdorff \hyperref[def:locally_convex_space]{locally convex space} \( X \) does not \hyperref[def:functions_vanish_nowhere]{vanish} at the nonzero vectors of \( X \).
\end{corollary}
\begin{proof}
  Fix a nonzero point \( x \in X \). The result follows from \fullref{thm:geometric_hahn_banach} with \( L \coloneqq \{ 0 \} \) and \( A \) --- any convex set containing \( x \) and not containing zero. Such a set \( A \) exists because the topology is Hausdorff and \( x \) has a neighborhood disjoint from any point in \( L \).
\end{proof}

\begin{corollary}\label{thm:hahn_banach_implies_annihilator_nontrivial}\mcite[25]{ИоффеТихомиров1974ЭкстремЗадачи}
  The \hyperref[def:vector_space_annihilator]{annihilator} of any proper subspace of a Hausdorff \hyperref[def:locally_convex_space]{locally convex space} contains nonzero elements.
\end{corollary}
\begin{proof}
  Denote the proper subspace by \( L \subsetneq X \). Fix \( x \in X \setminus L \) and let \( A \) be a convex neighborhood of \( x \) that is disjoint from \( L \). The result follows from \fullref{thm:geometric_hahn_banach}.
\end{proof}

\begin{corollary}\label{thm:hahn_banach_implies_duality_mapping_nonempty}\mcite[25]{ИоффеТихомиров1974ЭкстремЗадачи}
  In a \hyperref[def:norm]{normed} space \( X \), for any nonzero vector \( x \in X \) there exists a continuous functional \( x^* \in S_{X^*} \) such that \( \inprod {x^*} x = \norm x \). In other words, the duality \hyperref[def:duality_mapping]{mapping} is nonempty for any point.
\end{corollary}
\begin{proof}
  This follows from \fullref{thm:hahn_banach_implies_annihilator_nontrivial} by taking \( A \coloneqq B(x, \abs{x}) \) and \( L \coloneqq \{ 0 \} \) and then scaling the obtained functional.
\end{proof}

\begin{definition}\label{def:hyperplane_separation}
  We again restrict our attention to real affine spaces in order to define \term{hyperplane separation}.

  \begin{thmenum}
    \thmitem{def:hyperplane_separation/nonstrict}\mcite[def. 3.3]{Gallier2011Geometry} We say that two sets of points are \term{separated} by a \hyperref[def:affine_hyperplane]{hyperplane} if the sets belongs to different closed half-spaces with respect to the hyperplane.

    More concretely, the sets \( A \) and \( B \) are strongly separated by a hyperplane \( H \) if there exists a parametrization \( f(x) = \inprod l x - c \) such that, whenever \( x \in A \) and \( y \in B \), we have
    \begin{equation}\label{eq:def:hyperplane_separation/nonstrict}
      \inprod l x \leq c \leq \inprod l y.
    \end{equation}

    \begin{figure}[!ht]
      \centering
      \includegraphics[page=1]{output/def__hyperplane_separation__nonstrict}
      \caption{A \hyperref[def:hyperplane_separation/nonstrict]{separating line} between \hyperref[def:circle]{circles}.}\label{fig:def:hyperplane_separation/nonstrict}
    \end{figure}

    \thmitem{def:hyperplane_separation/strong}\mcite[def. 3.3]{Gallier2011Geometry} If the sets belong to different \hi{open} half-spaces, we say that they are \term{strongly separated}. The inequalities then become
    \begin{equation}\label{eq:def:hyperplane_separation/strong}
      \inprod l x < c < \inprod l y.
    \end{equation}

    \begin{figure}[!ht]
      \centering
      \includegraphics[page=1]{output/def__hyperplane_separation__strong}
      \caption{Parallel \hyperref[def:hyperplane_separation/strong]{strongly separating lines} between \hyperref[def:circle]{circles}.}\label{fig:def:hyperplane_separation/strong}
    \end{figure}

    \thmitem{def:hyperplane_separation/supporting}\mcite[def. 3.4]{Gallier2011Geometry} If \( B \) consists of a single point \( x_0 \) from \( A \), we say that a separating hyperplane is a \term{supporting hyperplane} for \( x_0 \) in \( A \). For each \( x \in A \) we have
    \begin{equation}\label{eq:def:hyperplane_separation/supporting}
      \inprod l x \leq \inprod l {x_0}.
    \end{equation}

    \begin{figure}[!ht]
      \centering
      \includegraphics[page=1]{output/def__hyperplane_separation__supporting}
      \caption{Multiple \hyperref[def:hyperplane_separation/supporting]{supporting lines} for a \hyperref[def:circle]{circle}.}\label{fig:def:hyperplane_separation/supporting}
    \end{figure}
  \end{thmenum}
\end{definition}

\begin{theorem}[Hahn-Banach hyperplane separation theorem]\label{thm:hahn_banach_hyperplane_separation}\mcite[25]{ИоффеТихомиров1974ЭкстремЗадачи}
  Fix a \hyperref[def:topological_vector_space]{topological vector space} \( X \). Let \( A, B \subseteq X \) be disjoint \hyperref[def:convex_hull]{convex} sets. If \( \int{A} \neq \varnothing \), there exists a continuous linear functional \hyperref[def:hyperplane_separation]{separating} \( A \) and \( B \).
\end{theorem}
