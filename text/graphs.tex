\subsection{Graphs}\label{subsec:graphs}

\paragraph{Four kinds of graphs}\hfill

The term \enquote{graph} is unfortunately very ambiguous\fnote{The graph of a relation or function is unrelated to the combinatorial graphs discussed here.} - it is a set of vertices connected by either directed arcs or undirected edges. To quote \incite[362]{Knuth1997Vol1}:
\begin{displayquote}
  Unfortunately, there will probably never be a standard terminology in this field, and so the author has followed the usual practice of contemporary books on graph theory, namely to use words that are similar but not identical to the terms used in any other books on graph theory.
\end{displayquote}

 We introduce distinct definitions for the four types of graphs from \cref{fig:def:graph_functors}, along with comments on how the definitions are used by different authors, and then in \fullref{def:graph_functors} we will define functors that help transparently transform some types of graphs into others. It is an established convention to not go through these hoops and implicitly transfer concepts between different kinds of graphs without even mentioning morphisms and categories. We will later follow this convention, but will nonetheless first describe the details.

\begin{figure}[!ht]
  \caption{Functors between the categories of different kinds of graphs.}\label{fig:def:graph_functors}
  \smallskip
  \hfill
  \begin{tikzpicture}
    \node[align=center] (dm) at (0, 0) {\hyperref[def:directed_graph/category]{Simple} \\ \hyperref[def:directed_graph/category]{undirected} \\ \hyperref[def:directed_graph/category]{graphs}};
    \node[align=center] (um) at (5cm, 0) {\hyperref[def:undirected_graph/category]{Simple} \\ \hyperref[def:undirected_graph/category]{undirected} \\ \hyperref[def:undirected_graph/category]{graphs}};
    \node[align=center] (ds) at (0, -3cm) {\hyperref[def:directed_multigraph/category]{Directed} \\ \hyperref[def:directed_multigraph/category]{multigraphs}};
    \node[align=center] (us) at (5cm, -3cm) {\hyperref[def:undirected_multigraph/category]{Undirected} \\ \hyperref[def:undirected_multigraph/category]{multigraphs}};
    \draw[->] (dm) to node[midway, above] {\hyperref[def:graph_functors/multi_forgetful]{\( U_S \)}} (um);
    \draw[->, bend left] (um) to node[midway, below] {\hyperref[def:graph_functors/simple_doubling]{\( D_S \)}} (dm);
    \draw[->] (ds) to node[midway, below] {\hyperref[def:graph_functors/multi_forgetful]{\( U_M \)}} (us);
    \draw[->, bend left] (dm) to node[midway, right] {\hyperref[def:graph_functors/directed_forgetful]{\( U_D \)}} (ds);
    \draw[->, bend left] (ds) to node[midway, left] {\hyperref[def:graph_functors/directed_inclusion]{\( I_D \)}} (dm);
    \draw[->, bend left] (um) to node[midway, right] {\hyperref[def:graph_functors/undirected_forgetful]{\( U_U \)}} (us);
    \draw[->, bend left] (us) to node[midway, left] {\hyperref[def:graph_functors/undirected_inclusion]{\( I_U \)}} (um);
  \end{tikzpicture}
  \hfill\hfill
\end{figure}

\begin{definition}\label{def:directed_multigraph}\mcite[def. 1.1.1; def. 1.1.2]{Knauer2011}
  A \term[bg=ориентиран (\cite[6]{Мирчев2001}) мултиграф (\cite[7]{Мирчев2001}), ru=ориентированый мультиграф (\cite[220]{Зыков2004})]{directed multigraph} \( G \) consists of the following:
  \begin{thmenum}[series=def:directed_multigraph]
    \thmitem{def:directed_multigraph/vertices} A set \( V_G \), whose elements we call \term[bg=върхове (\cite[6]{Мирчев2001}), ru=вершины (\cite[5]{Зыков2004})]{vertices}.

    \thmitem{def:directed_multigraph/arcs} A disjoint from \( V_G \) set \( A_G \), whose elements we call \term[bg=дъги (\cite[6]{Мирчев2001}), ru=дуги (\cite[6]{Зыков2004})]{arcs}.
    \thmitem{def:directed_multigraph/head} A function \( h_G: A \to V \), giving the \term[en=head (\cite[544]{Rosen1999})]{head} or \term[bg=начален връх (\cite[7]{Мирчев2001}), ru=начало (\cite[277]{БелоусовТкачёв2004}), en=initial vertex (\cite[28]{Diestel2005})]{initial endpoint} of an arc.

    \thmitem{def:directed_multigraph/tail} A function \( t_G: A \to V \), giving the \term[en=tail (\cite[544]{Rosen1999})]{tail} or \term[bg=краен връх (\cite[7]{Мирчев2001}), ru=конец (\cite[277]{БелоусовТкачёв2004}), en=terminal vertex (\cite[28]{Diestel2005})]{terminal endpoint} of an arc.
  \end{thmenum}

  \Cref{fig:def:directed_multigraph} illustrates this definition. The figure is not merely illustrative --- it is a \hyperref[def:graph_geometric_realization/embedding]{graph embedding}.

  \begin{figure}[!ht]
    \begin{equation}\label{eq:fig:def:directed_multigraph}
      \begin{aligned}
        \includegraphics[page=1]{output/def__directed_multigraph}
      \end{aligned}
    \end{equation}
    \caption{A \hyperref[def:directed_multigraph]{directed multigraph} with a pair of parallel arcs, a pair of oppositely directed arcs and a loop. Removing the dashed arcs makes it a \hyperref[def:directed_graph]{simple directed graph}.}\label{fig:def:directed_multigraph}
  \end{figure}

  We will need the following basic notions:
  \begin{thmenum}[resume=def:directed_multigraph]
    \thmitem{def:directed_multigraph/loop} We call the arc \( e \) a \term[bg=примка (\cite[7]{Мирчев2001}), ru=петля (\cite[6]{Зыков2004})]{loop} if \( h(e) = t(e) \).

    \medskip

    \thmitem{def:directed_multigraph/parallel}\mcite[28]{Diestel2005} We call the arcs \( e \) and \( f \) \term[bg=паралелни (ребра) (\cite[7]{Мирчев2001}), ru=параллельные (рёбра) (\cite[7]{Зыков2004})]{parallel} if \( h(e) = h(f) \) and \( t(e) = t(f) \) and \term{opposite} if it is not a loop and \( h(e) = t(f) \) and \( t(e) = h(f) \).

    \thmitem{def:directed_multigraph/homomorphism}\mcite[def. 1.4.1]{Knauer2011} A \term{homomorphism} between directed multigraphs \( G \) and \( H \) is a pair of functions
    \begin{align*}
      &f_V: V_G \to V_H, \\
      &f_A: A_G \to A_H,
    \end{align*}
    such that
    \begin{subequations}
      \begin{align}
        h_H \bincirc f_A &= f_V \bincirc h_G, \label{eq:def:directed_multigraph/homomorphism/head} \\
        t_H \bincirc f_A &= f_V \bincirc t_G. \label{eq:def:directed_multigraph/homomorphism/tail}
      \end{align}
    \end{subequations}

    \thmitem{def:directed_multigraph/category}\mcite[48]{MacLane1998} Given a Grothendieck universe \( \mscrU \), we can define the \hyperref[def:category]{category} of \( \mscrU \)-small directed multigraphs (where both the vertex and arc sets are \( \mscrU \)-small) and their homomorphisms.

    \thmitem{def:directed_multigraph/subgraph}\mcite[3]{Diestel2005} We say that \( H \) is a \term{subgraph} of \( G \) if \( V_G \subseteq V_H \), \( A_G \subseteq A_H \) and \( h_H \) and \( t_H \) are restrictions of \( h_G \) and \( t_G \) to \( A_H \).
  \end{thmenum}
\end{definition}
\begin{comments}
  \item Formally, we define a directed multigraph as a quadruple \( G = (V_G, A_G, h_G, t_G) \). We may skip indices when unnecessary, i.e. we are free to write \( G = (V, A, h, t) \) or even \( H = (W, B, j, u) \).

  \item \incite[220]{Зыков2004} defines directed multigraphs similarly to how we do it. \incite[def. 1.1.1]{Knauer2011}, \incite[10]{MacLane1998} and \incite[28]{Diestel2005} also provide similar definitions, but call them \enquote{directed graphs}. Knauer later distinguishes between simple and multigraphs, while Diestel uses \enquote{oriented graph} for what we call \enquote{simple directed graph}. \incite[8]{Bollobas1998} briefly mentions directed multigraphs and calls them as such.

  Diestel and Bollobas reserve the term \enquote{multigraph} for \hyperref[def:undirected_multigraph]{undirected multigraphs}. The latter also implicitly restricts graphs to finite order. These two books use \hyperref[def:undirected_graph]{simple undirected graphs} almost exclusively.
\end{comments}

\begin{remark}\label{rem:digraph}
  The term \enquote{digraph} is used as a shorthand for \enquote{directed graph}, for example by \incite[def. 1.1.1]{Knauer2011}, \incite[372]{Knuth1997Vol1} and \incite[559]{Rosen1999}. \incite[28]{Diestel2005} also uses this terminology, and in his usage \enquote{graph} exclusively refers to \enquote{undirected graph}.

  Similarly, in Russian, \enquote{орграф} is used as a shorthand for \enquote{ориентированный граф}, for example by \incite[\textnumero 7.1.5]{Новиков2013}.

  We avoid the term.
\end{remark}

\begin{remark}\label{rem:subgraphs_and_subobjects}
  Subgraphs are not \hyperref[def:subobject_and_quotient]{categorical subobjects} because a graph can have distinct isomorphic subgraphs. For example, all one-vertex subgraphs are isomorphic, but nonetheless we consider them to be different subgraphs.
\end{remark}

\begin{definition}\label{def:directed_graph}\mcite[def. 1.1.2]{Knauer2011}
  A \term{simple directed graph} \( G \) consists of the following:
  \begin{thmenum}[series=def:directed_graph]
    \thmitem{def:directed_graph/vertices} A set \( V_G \), whose elements we call \term{vertices}.
    \thmitem{def:directed_graph/arcs} A disjoint from \( V_G \) set \( A_G \) of \hyperref[def:cartesian_product/kuratowski_pair]{ordered pairs} of \hi{distinct} vertices. As in the case of directed multigraphs, we call the elements of \( A_G \) \term{arcs}. If we allow vertices whose endpoints coincide, we will say that \( G \) is a \enquote{simple directed graph with loops}.
  \end{thmenum}

  We will use the following basic terminology:
  \begin{thmenum}[resume=def:directed_graph]
    \thmitem{def:directed_graph/homomorphism}\mimprovised A \term{homomorphism} between the directed graphs \( G \) and \( H \) is a function \( f: V_G \to V_H \) such that \( (u, v) \in A_G \) implies \( (f(u), f(v)) \in A_H \).

    \thmitem{def:directed_graph/category}\mimprovised Given a Grothendieck universe \( \mscrU \), we can define the \hyperref[def:category]{category} of \( \mscrU \)-small simple directed graphs and their homomorphisms.

    \thmitem{def:directed_graph/subgraph}\mimprovised We say that \( H \) is a \term{subgraph} of \( G \) if \( V_G \subseteq V_H \) and \( A_G \subseteq A_H \).
  \end{thmenum}
\end{definition}
\begin{comments}
  \item We can regard directed graphs as directed multigraphs without loops and parallel arcs. This is made precise via the inclusion functor \hyperref[def:graph_functors/directed_inclusion]{\( I_D \)}.
\end{comments}

\begin{definition}\label{def:undirected_multigraph}\mcite[def. 1.1.2]{Knauer2011}
  An \term{undirected multigraph} \( G \) consists of the following:
  \begin{thmenum}[series=def:undirected_multigraph]
    \thmitem{def:undirected_multigraph/vertices} A set \( V_G \), whose elements we call \term{vertices}.
    \thmitem{def:undirected_multigraph/edges} A disjoint from \( V_G \) set \( E_G \), whose elements we call \term[bg=ребра (\cite[6]{Мирчев2001}), ru=рёбра (\cite[6]{Зыков2004})]{edges}.
    \thmitem{def:undirected_multigraph/endpoints} A map
    \begin{equation*}
      \mscrE: E \to \set[\Big]{ \set{ u, v } \given* u, v \in V },
    \end{equation*}
    giving an unordered pair of \term{endpoints} of an edge.
  \end{thmenum}

  \begin{figure}[!ht]
    \begin{equation}\label{eq:fig:def:undirected_multigraph}
      \begin{aligned}
        \includegraphics[page=1]{output/def__undirected_multigraph}
      \end{aligned}
    \end{equation}
    \caption{An undirected multigraph, which becomes simple after removing the dashed edges.}\label{fig:def:undirected_multigraph}
  \end{figure}

  We will use the following basic terminology:
  \begin{thmenum}[resume=def:undirected_multigraph]
    \thmitem{def:undirected_multigraph/loop} We call the edge \( e \) a \term{loop} if \( \mscrE(e) \) is a one-element set.

    \medskip

    \thmitem{def:undirected_multigraph/parallel} We call the edges \( e \) and \( f \) \term{parallel} if \( \mscrE(e) = \mscrE(f) \).

    \thmitem{def:undirected_multigraph/homomorphism}\mimprovised A \term{homomorphism} between the undirected multigraphs \( G \) and \( H \) is a pair of functions
    \begin{align*}
      &f_V: V_G \to V_H, \\
      &f_E: E_G \to E_H,
    \end{align*}
    such that, for each edge \( e \in E_G \),
    \begin{equation}\label{eq:def:undirected_multigraph/homomorphism}
      \mscrE_H(f_E(e)) = \set[\Big]{ f_V(v) \given* v \in \mscrE_G(e) }.
    \end{equation}

    \thmitem{def:undirected_multigraph/category}\mimprovised Given a Grothendieck universe \( \mscrU \), we can define the \hyperref[def:category]{category} of \( \mscrU \)-small undirected multigraphs and their homomorphisms.

    \thmitem{def:undirected_multigraph/subgraph}\mimprovised We say that \( H \) is a \term{subgraph} of \( G \) if \( V_G \subseteq V_H \), \( E_G \subseteq E_H \) and \( \mscrE_H = \mscrE_G\restr_{E_H} \).
  \end{thmenum}
\end{definition}
\begin{comments}
  \item \incite[def. 1.1.2]{Knauer2011} does not explicitly mention undirected multigraphs, but their existence is implied.

  \incite[28]{Diestel2005} and \incite[6]{Bollobas1998} define undirected multigraphs, although both omit the prefix \enquote{undirected}. Bollobas later explicitly defines directed multigraphs, while \incite[28]{Diestel2005} uses the term \enquote{directed graph} for what we call a \hyperref[def:directed_multigraph]{directed multigraph} and \enquote{oriented graph} for what we call \enquote{simple directed graph}.

  \incite[3]{GondranMinoux1984Graphs} define multigraphs without specifying whether they are directed or not, but later implicitly assume that they are undirected.
\end{comments}

\begin{definition}\label{def:undirected_graph}\mcite[3]{Diestel2005}
  A \term{simple undirected graph} \( G \) consists of the following:
  \begin{thmenum}[series=def:undirected_graph]
    \thmitem{def:undirected_graph/vertices} A set \( V_G \), whose elements we call \term{vertices}.
    \thmitem{def:undirected_graph/edges} A disjoint from \( V_G \) set \( E_G \) of unordered pairs of \hi{distinct} vertices, i.e. sets of the form \( \set{ u, v } \), whose elements we call \term{edges}. If we allow vertices whose endpoints coincide, we will say that \( G \) is a \enquote{simple undirected graph with loops}.
  \end{thmenum}

  We will use the following basic terminology:
  \begin{thmenum}[resume=def:undirected_graph]
    \thmitem{def:undirected_graph/homomorphism}\mcite[def. 1.4.3]{Knauer2011} A \term{homomorphism} between the simple undirected graphs \( G \) and \( H \) is a function \( f: V_G \to V_H \) such that \( \set{ u, v } \in E_G \) implies \( \set{ f(u), f(v) } \in E_H \).

    \thmitem{def:undirected_graph/category}\mcite[example 3.1.12]{Knauer2011} Given a Grothendieck universe \( \mscrU \), we can define the \hyperref[def:category]{category} of \( \mscrU \)-small simple undirected graphs and their homomorphisms.

    By \fullref{thm:edgeless_graph_universal_property}, the monomorphisms are precisely the injective homomorphisms and, by \fullref{thm:complete_graph_universal_property}, the epimorphisms are precisely the surjective homomorphisms.

    \thmitem{def:undirected_graph/subgraph}\mcite[3]{Diestel2005} We say that \( H \) is a \term{subgraph} of \( G \) if \( V_G \subseteq V_H \) and \( E_G \subseteq E_H \).
  \end{thmenum}
\end{definition}
\begin{comments}
  \item We can regard undirected graphs as undirected multigraphs without loops and parallel arcs. This is made precise via the inclusion functor \hyperref[def:graph_functors/undirected_inclusion]{\( I_U \)}.

  \item \incite[2]{Diestel2005} and \incite[1]{Bollobas1998} define simple undirected graphs similarly to how we have done it. \incite[def. 1.1.2]{Knauer2011} also does, however he allows loops.

  \item \incite[def. 1.4.8]{Knauer2011} define subgraphs more generally as graphs which can be embedded via injective graph homomorphisms. We prefer the vertices and edges of subgraphs to be subsets.
\end{comments}

\begin{remark}\label{rem:simple_graphs}
  Whether or simple graphs are allowed have loops again depends on the authors.

  Defining \enquote{directed graphs} as pairs \( (V, A) \), where \( A \subseteq V \times V \), without further restrictions, is common. It is done by \incite[def. 1.1.2]{Knauer2011}, \incite[21]{GondranMinoux1984Graphs}, \incite[10]{Savage1998}, \incite[190]{Erickson2019}, \incite[277]{БелоусовТкачёв2004} and \incite[6]{Мирчев2001}. \incite[\textnumero 7.1.5]{Новиков2013} also provides the same definition, however he forbids loops unless explicitly mentioned. \incite[39]{Diestel2005} uses the term \enquote{oriented graph} for what we call \enquote{simple directed graph}.

  For undirected graphs we instead have several conventions:
  \begin{itemize}
    \item Some of the aforementioned authors, namely \incite[21]{GondranMinoux1984Graphs}, \incite[10]{Savage1998} and \incite[277]{БелоусовТкачёв2004}, define directed graphs as \enquote{symmetric} undirected graphs, in which \( (u, v) \) is an arc whenever \( (v, u) \) is. This definition allows loops since the directed counterparts do.

    \item Other aforementioned authors, namely \incite[def. 1.1.2]{Knauer2011}, \incite[190]{Erickson2019}, and \incite[7]{Мирчев2001} define undirected graphs as pairs \( (V, E) \), where \( E \) consists of one-element or two-element subsets of \( V \). This definition again allows loops.

    \item \incite[\textnumero 7.1.2]{Новиков2013}, \incite[2]{Diestel2005} and \incite[1]{Bollobas1998} define directed graphs as pairs \( (V, E) \), where \( E \) consists of two-element subsets of \( V \). This explicitly forbid loops.

    \item \incite[377]{Knuth1997Vol1} explicitly forbids loops in graphs, however he defines \enquote{graphs} as \enquote{a set of points together with a set of lines}.
  \end{itemize}

  \incite[def. 1.1.2]{Knauer2011} uses the adjective \enquote{simple} to refer to (multi)graphs without multiple edges, without referring to loops. So do \incite[2]{Diestel2005} and \incite[22]{Bollobas1998}, however their definitions of graphs forbid loops by default. \incite[22]{GondranMinoux1984Graphs}, \incite[191]{Erickson2019}, \incite[540]{Rosen1999} and \incite[12]{Мирчев2001} use \enquote{simple} to additionally forbid loops.

  We prefer our terminology to be as unambiguous as possible, for which reason we use \enquote{simple} in the latter sense and, if loops are allowed, we mention it explicitly.
\end{remark}

\begin{definition}\label{def:multigraph_orientation}\mcite[def. 6.2.1]{Knauer2011}
  We call the \hyperref[def:directed_multigraph]{directed multigraph} \( D = (V, A, h, t) \) an \term[ru=ориентация (\cite[412]{Зыков2004})]{orientation} of the \hyperref[def:undirected_multigraph]{undirected multigraph} \( G = (U, E, \mscrE) \) if \( U = V \), \( A = E \), and, for every arc \( e \in A \), we have \( \mscrE(e) = \set{ h(e), t(e) } \).
\end{definition}
\begin{comments}
  \item This definition also applies to simple graphs via the inclusion functors \hyperref[def:graph_functors/directed_inclusion]{\( I_D \)} and \hyperref[def:graph_functors/undirected_inclusion]{\( I_U \)}.
\end{comments}

\begin{definition}\label{def:graph_functors}\mimprovised
  Different kinds of graphs are related via the following functors (shown graphically in \fullref{fig:def:graph_functors}):
  \begin{thmenum}
    \thmitem{def:graph_functors/directed_forgetful} The simple \hyperref[def:concrete_category]{forgetful functor}:
    \begin{flalign*}
      &U_D: \hyperref[def:directed_multigraph/category]{\T*{Directed multigraphs}} \to \hyperref[def:directed_graph/category]{\T*{Simple directed graphs with loops}}, &&\\
      &U_D(V, A, h, t) \coloneqq \parens[\Big]{ V, \set[\Big]{ \parens[\Big]{ h(a), t(a) } \given* a \in A } }, &&\\
      &U_D(f_V, f_A) \coloneqq f_V.
    \end{flalign*}

    \thmitem{def:graph_functors/directed_inclusion} A \hyperref[def:category_adjunction]{left adjoint}:
    \begin{flalign*}
      &I_D: \hyperref[def:directed_graph/category]{\T*{Simple directed graphs}} \to \hyperref[def:directed_multigraph/category]{\T*{Directed multigraphs}}, &&\\
      &I_D(V, A) \coloneqq \parens[\Big]{ V, A, (u, v) \mapsto u, (u, v) \mapsto v }, &&\\
      &I_D(f) \coloneqq \parens[\Big]{ f, (u, v) \mapsto \parens[\Big]{ f(u), f(v) } }.
    \end{flalign*}

    \thmitem{def:graph_functors/undirected_forgetful} A similar forgetful functor for undirected graphs:
    \begin{flalign*}
      &U_U: \hyperref[def:undirected_multigraph/category]{\T*{Undirected multigraphs}} \to \hyperref[def:undirected_graph/category]{\T*{Simple undirected graphs with loops}}, &&\\
      &U_U(V, E, \mscrE) \coloneqq \parens[\Big]{ V, \set[\Big]{ \mscrE(e) \given* e \in E } }, &&\\
      &U_U(f_V, f_E) \coloneqq f_V.
    \end{flalign*}

    \thmitem{def:graph_functors/undirected_inclusion} A left adjoint:
    \begin{flalign*}
      &I_U: \hyperref[def:undirected_graph/category]{\T*{Simple undirected graphs}} \to \hyperref[def:undirected_multigraph/category]{\T*{Undirected multigraphs}}, &&\\
      &I_U(V, E) \coloneqq \parens[\Big]{ V, E, e \mapsto e }, &&\\
      &I_U(f) \coloneqq \parens[\Big]{ f, \set{ u, v } \mapsto \set{ f(u), f(v) } }.
    \end{flalign*}

    \thmitem{def:graph_functors/multi_forgetful} A functor that forgets the \hyperref[def:multigraph_orientation]{orientation} of a multigraph:
    \begin{flalign*}
      &U_M: \hyperref[def:directed_multigraph/category]{\T*{Directed multigraphs}} \to \hyperref[def:undirected_multigraph/category]{\T*{Undirected multigraphs}}, &&\\
      &U_M(V, A, h, t) \coloneqq \parens[\Big]{ V, A, e \mapsto \set{ h(e), t(e) } }, &&\\
      &U_M(f_V, f_A) \coloneqq (f_V, f_A).
    \end{flalign*}

    \thmitem{def:graph_functors/simple_forgetful} A functor that forgets the orientation of a simple graph:
    \begin{flalign*}
      &U_S: \hyperref[def:directed_graph/category]{\T*{Simple directed graphs}} \to \hyperref[def:undirected_graph/category]{\T*{Simple undirected multigraphs}}, &&\\
      &U_S(V, A) \coloneqq \parens[\Big]{ V, A, (u, v) \mapsto \set{ u, v } }, &&\\
      &U_S(f) \coloneqq f.
    \end{flalign*}

    \thmitem{def:graph_functors/simple_doubling} A right adjoint to \( U_S \), which doubles each edge to produce arcs in both directions:
    \begin{flalign*}
      &D_S: \hyperref[def:undirected_graph/category]{\T*{Simple undirected graphs}} \to \hyperref[def:directed_multigraph/category]{\T*{Simple directed graphs}}, &&\\
      &D_S(V, E) \coloneqq \parens[\Big]{ V, \set[\Big]{ (u, v) \in V^2 \given \set{ u, v } \in E } }, &&\\
      &D_S(f) \coloneqq f.
    \end{flalign*}
  \end{thmenum}
\end{definition}
\begin{comments}
  \item \Fullref{ex:def:category_adjunction/us_ds} discusses the doubling functor from \fullref{def:graph_functors/simple_doubling}.
  \item We can introduce a doubling functor for multigraphs similar to \hyperref[def:graph_functors/simple_doubling]{\( D_S \)} that introduces two opposite copies of each edge in an undirected multigraph.

  Unfortunately, that would require choosing one vertex as the head and the other as a tail, and thus the functor depends on a \hyperref[def:choice_function]{choice function}. This choice function is unnecessary for simple graphs because ordered pairs having a first and second element, while multigraphs have abstract objects as arcs.

  Furthermore, even given a canonical choice function, this functor would not be a right inverse to \hyperref[def:graph_functors/multi_forgetful]{\( U_M \)} --- instead, we would obtain an undirected multigraph with twice as many edges as the original.

  We avoid introducing such a functor altogether.
\end{comments}

\begin{remark}\label{rem:arbitrary_graph}
  We will henceforth use the term \enquote{arbitrary graph} to refer to any of the four kinds of graphs defined in this section.

  The inclusion functors \hyperref[def:graph_functors/directed_inclusion]{\( I_D \)} and \hyperref[def:graph_functors/undirected_inclusion]{\( I_U \)} allow all definitions for multigraphs to apply to simple graphs. The transition between directed and undirected graphs is more complicated, but we will nonetheless utilize \hyperref[def:graph_functors/multi_forgetful]{\( U_M \)}, \hyperref[def:graph_functors/simple_forgetful]{\( U_S \)} and \hyperref[def:graph_functors/simple_doubling]{\( D_S \)} when necessary.
\end{remark}

\begin{remark}\label{rem:trivial_graph}
  Unlike the \hyperref[def:group/trivial]{trivial group} \( \set{ e } \) or \hyperref[def:module/trivial]{trivial \( R \)-module} \( \set{ 0 } \), which are unique up to an isomorphism, there is no trivial graph in the sense of \fullref{def:trivial_object}.

  Every graph has a subgraph with \hyperref[def:graph_cardinality/order]{order} zero, and hence up to an isomorphism we have an order-zero graph (for every kind of graph discussed here).

  Another unambiguous concept is that of an edgeless graph. Every \hyperref[def:complete_graph]{complete graph} has \( 2^n \) edgeless subgraphs (one for each set of edges).
\end{remark}

\paragraph{Cardinalities in graphs}

\begin{definition}\label{def:graph_incidence}\mcite[3]{Diestel2005}
  We say that the vertex \( v \) and the arc/edge \( e \) are \term[bg=инцидентни (ребра) (\cite[7]{Мирчев2001}), ru=инцидентные (рёбра) (\cite[6]{Зыков2004})]{incident} of \( v \) is an endpoint of \( e \).
\end{definition}

\begin{definition}\label{def:graph_cardinality}
  Graphs have the following notions of \hyperref[thm:cardinality_existence]{cardinality}:
  \begin{thmenum}
    \thmitem{def:graph_cardinality/order}\mcite[2]{Diestel2005} We define the \term{order} \( \ord(G) \) of an \hyperref[rem:arbitrary_graph]{arbitrary graph} \( G \) as the (cardinal) number of vertices.

    For a simple graph, finitely many vertices imply finitely many arcs/edges, which justifies terminology like \enquote{finite simple graph}. For multigraphs, however, we will prefer being more concrete and use \enquote{finite-order (multi)graph}.

    \thmitem{def:graph_cardinality/directed_degree}\mcite[def. 1.1.7]{Knauer2011} We define the \term[bg=полустепен на изхода (\incite[8]{Мирчев2001}), ru=полустепень исхода (\cite[277]{БелоусовТкачёв2004})]{out-degree} \( \deg_{\op{out}}(v) \) (resp. \term[bg=полустепен на входа (\incite[8]{Мирчев2001}), ru=полустепень захода (\cite[277]{БелоусовТкачёв2004})]{in-degree} \( \deg_{\op{in}}(v) \)) of a vertex \( v \) in a \hyperref[def:directed_multigraph]{directed (multi)graph} as the (cardinal) number of arcs starting (resp. ending) at \( v \).

    We define the \term[bg=степен (\cite[8]{Мирчев2001}), ru=степень (\cite[21]{Зыков2004})]{degree} \( \deg(v) \) of \( v \) as the sum of the two.

    \thmitem{def:graph_cardinality/undirected_degree}\mimprovised If the graph is undirected, consider the \hyperref[def:multiset]{multiset} of all edges incident to \( v \), with loops having multiplicity \( 2 \) and all other edges having multiplicity \( 1 \).

    We define the \term{degree} \( \deg(v) \) as the \hyperref[def:multiset/cardinality]{cardinality} of this multiset.

    \thmitem{def:graph_cardinality/local}\mcite[196]{Diestel2005} We say that a graph is \term{locally finite} (resp. \term{locally countable}) if the degree of any vertex is finite (resp. countable).
  \end{thmenum}
\end{definition}

\begin{proposition}\label{thm:degree_of_undirected_counterpart}
  The degree, in the sense of \fullref{def:graph_cardinality/directed_degree}, of a vertex \( v \) in a \hyperref[def:directed_multigraph]{directed multigraph} \( G \), is equal to the degree, in the sense of \fullref{def:graph_cardinality/undirected_degree}, of \( v \) in the undirected counterpart \( \hyperref[def:graph_functors/multi_forgetful]{U_M}(G) \) of \( G \).
\end{proposition}
\begin{comments}
  \item When proving statements about vertex degrees, it thus makes sense to prove it for undirected multigraphs, since it will then also automatically apply to directed graphs.
\end{comments}
\begin{proof}
  We have adjusted our definitions so that this holds.
\end{proof}

\begin{remark}\label{rem:counting_loops_twice}
  Loops may contribute either \( 1 \) or \( 2 \) towards the degree of a graph. This choice simplifies \fullref{thm:sum_of_endpoint_degrees} and hence \fullref{thm:sum_of_graph_degrees}, as well as \fullref{def:graph_cycle}, among others.

  We describe here several conventions.
  \begin{itemize}
    \item \incite[def. 1.1.7]{Knauer2011} avoids defining degrees for directed graphs, leaving only out-degrees and in-degrees. For undirected graphs, however, in \cite[def. 1.1.8]{Knauer2011} he defines the degree of a vertex \( v \) as the number of edges incident to \( v \), which without our multiset trick only counts loops once.

    \item \incite[372]{Knuth1997Vol1} also defines out-degrees and in-degrees, but avoids defining degrees for both directed and undirected graphs.

    \item \incite[191]{Erickson2019} \incite[8]{Мирчев2001} defines degrees without counting loops twice for both directed and undirected graphs.

    \item \incite[277]{БелоусовТкачёв2004}, who diligently distinguish between directed and undirected graphs, defines degrees in (simple) directed graphs as sums of out-degrees and in-degrees, like we do, however for undirected graphs they also avoid counting loops twice.

    \item \incite[544]{Rosen1999} defines the degree of a vertex similarly to us, however without referring to multisets. \incite[8]{Bollobas1998} hints that loops should be counted twice.
  \end{itemize}
\end{remark}

\begin{definition}\label{def:isolated_vertex}\mcite[5]{Diestel2005}
  We say that a vertex in an \hyperref[rem:arbitrary_graph]{arbitrary graph} is \term[bg=изолирани (върхове) (\cite[8]{Мирчев2001}), ru=изолированные (вершины) (\cite[6]{Зыков2004})]{isolated} if its \hyperref[def:graph_cardinality/directed_degree]{degree} is zero.
\end{definition}

\begin{lemma}\label{thm:sum_of_endpoint_degrees}
  Fix an undirected multigraph \( G = (V, E, \mscrE) \). Fix an edge \( e \) and consider the graph \( G = (V, E \setminus \set{ e }, \mscrE) \). Denote by \( \deg'(v) \) the vertex degree in \( G' \).

  Then
  \begin{equation}\label{eq:thm:sum_of_endpoint_degrees}
    \sum_{v \in \mscrE(e)} \deg(v) = \sum_{v \in \mscrE(e)} \deg'(v) + 2.
  \end{equation}
\end{lemma}
\begin{proof}
  If \( e \) is a loop at \( v \), by definition we have
  \begin{equation*}
    \sum_{w \in \mscrE(e)} \deg(w) = \deg(v) = \deg'(v) + 2 = \sum_{w \in \mscrE(e)} \deg'(w) + 2.
  \end{equation*}

  If \( e \) is an edge between \( u \) and \( v \), then
  \begin{equation*}
    \sum_{w \in \mscrE(e)} \deg(w)
    =
    \deg(u) + \deg(v)
    =
    \deg'(u) + 1 + \deg'(v) + 1
    =
    \sum_{w \in \mscrE(e)} \deg'(w) + 2.
  \end{equation*}
\end{proof}

\begin{proposition}\label{thm:sum_of_graph_degrees}
  In an \hyperref[rem:arbitrary_graph]{arbitrary graph} with finitely many arcs/edges, the sum of the degrees of all vertices is twice the number of arcs/edges.
\end{proposition}
\begin{comments}
  \item As long as a graph only has finitely many arcs/edges, only finitely many vertices have positive degree, so summing all degrees is justified.
  \item As a consequence, the sum of degrees if even. \incite[4]{Bollobas1998} calls this consequence the \enquote{handshaking lemma} because of the observation that the total number of hands shaken at a party is even.
\end{comments}
\begin{proof}
  As per \fullref{thm:degree_of_undirected_counterpart}, it is sufficient to only prove the statement for undirected multigraphs.

  We will use induction on the number \( n \) of arcs.

  \begin{itemize}
    \item If there are zero edges, then the degree of every vertex is zero, hence the sum of all degrees is also zero.

    \item Suppose that the statement holds for multigraphs with \( n \) arcs and consider an undirected multigraph \( G = (V, E, \mscrE) \) with \( n + 1 \) arcs.

    Fix any arc \( e \) and consider \( G' \coloneqq (V, E \setminus \set{ e }, \mscrE) \). Denote by \( \deg'(v) \) the degree of \( v \) in \( G' \). Then \fullref{thm:sum_of_endpoint_degrees} implies that
    \begin{equation*}
      \overbrace{\sum_{v \in V} \deg(v) = \sum_{v \in V \setminus \mscrE(e)} \underbrace{\deg(v)}_{\deg'(v)} + \underbrace{\sum_{v \in \mscrE(e)} \deg(v)}_{\sum_{v \in \mscrE(e)} \deg'(v) + 2} - 2}^{2n \T*{by the inductive hypothesis}} + 2
      =
      2(n + 1)
    \end{equation*}
  \end{itemize}
\end{proof}

\begin{corollary}\label{thm:odd_degree_vertices}\mcite[prop. 1.2.1]{Diestel2005}
  In an \hyperref[rem:arbitrary_graph]{arbitrary graph} with finitely many arcs/edges, the number of vertices of odd \hyperref[def:graph_cardinality/directed_degree]{degree} is even.
\end{corollary}
\begin{comments}
  \item This corollary, while by itself inappreciable, is essential for proving \fullref{thm:eulers_theorem_for_graphs}.
\end{comments}
\begin{proof}
  As per \fullref{thm:degree_of_undirected_counterpart}, it is sufficient to consider some undirected multigraph \( G = (V, E, \mscrE) \).

  We have
  \begin{equation}\label{eq:thm:odd_degree_vertices/proof}
    \underbrace{\sum_{v \in V} \deg(v)}_{\T*{even by} \fullref{thm:sum_of_graph_degrees}} = \underbrace{\sum_{\deg(v) \T*{is even}} \deg(v)}_{\T{sum of even numbers}} + \underbrace{\sum_{\deg(v) \T*{is odd}} \deg(v)}_{\T{sum of odd numbers}}.
  \end{equation}

  In order for \eqref{eq:thm:odd_degree_vertices/proof} to hold, there must be an even number of vertices of odd degree.
\end{proof}

\begin{example}\label{ex:infinite_integer_graphs}
  An example of an infinite \hyperref[def:directed_graph]{simple directed graph} is the \hyperref[def:transitive_reduction]{transitive reduction} of the positive integers:
  \begin{equation}\label{eq:ex:infinite_integer_graphs/positive}
    \begin{aligned}
      \includegraphics[page=1]{output/ex__infinite_integer_graphs}
    \end{aligned}
  \end{equation}

  Since the graph is simple, we have
  \begin{equation*}
    \deg(n) = \begin{cases}
      \deg_{\op{in}}(n) + \deg_{\op{out}}(n) = 0 + 1 = 1, &n = 0, \\
      \deg_{\op{in}}(n) + \deg_{\op{out}}(n) = 2 + 1 = 2, &n > 0 \\
    \end{cases}
  \end{equation*}

  The \hyperref[def:categorical_diagram]{categorical diagram} corresponding to this graph is used to define direct limits in \fullref{def:direct_and_inverse_limits/direct}.

  Another related graph is based on the negative integers:
  \begin{equation}\label{eq:ex:infinite_integer_graphs/negative}
    \begin{aligned}
      \includegraphics[page=2]{output/ex__infinite_integer_graphs}
    \end{aligned}
  \end{equation}

  The categorical diagram corresponding to this graph is used to define inverse limits in \fullref{def:direct_and_inverse_limits/inverse}.

  Finally, the union of the two with zero added gives us the following directed graph:
  \begin{equation}\label{eq:ex:infinite_integer_graphs/two_sided}
    \begin{aligned}
      \includegraphics[page=3]{output/ex__infinite_integer_graphs}
    \end{aligned}
  \end{equation}

  All three graphs \hyperref[def:graph_cardinality/local]{locally finite} but have infinite \hyperref[def:graph_cardinality/order]{order}.
\end{example}
