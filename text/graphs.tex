\section{Graphs}\label{sec:graphs}

\paragraph{Four kinds of graphs}

\begin{figure}[!ht]
  \caption{Functors between the categories of different kinds of graphs and hypergraphs.}\label{fig:def:graph_functors}
  \smallskip
  \hfill
  \begin{tikzpicture}
    \node[align=center] (hy) at (5cm, 2cm) {\hyperref[def:hypergraph/category]{Hypergraphs}};
    \node[align=center] (ds) at (0, -3cm) {\hyperref[def:directed_graph/category]{Simple} \\ \hyperref[def:directed_graph/category]{directed} \\ \hyperref[def:directed_graph/category]{graphs} \\ {\scriptsize \hyperref[def:directed_graph/category]{(with loops)}}};
    \node[align=center] (us) at (5cm, -3cm) {\hyperref[def:undirected_graph/category]{Simple} \\ \hyperref[def:undirected_graph/category]{undirected} \\ \hyperref[def:undirected_graph/category]{graphs} \\ {\scriptsize \hyperref[def:undirected_graph/category]{(with loops)}}};
    \node[align=center] (dm) at (0, 0) {\hyperref[def:directed_multigraph/category]{Directed} \\ \hyperref[def:directed_multigraph/category]{multigraphs}};
    \node[align=center] (um) at (5cm, 0) {\hyperref[def:hypergraph/category]{Undirected} \\ \hyperref[def:hypergraph/category]{multigraphs}};

    \draw[->] (ds) to node[midway, above] {\hyperref[def:graph_functors/simple_forgetful]{\( U_S \)}} (us);
    \draw[->, bend left] (us) to node[midway, below] {\hyperref[def:graph_functors/simple_doubling]{\( D_S \)}} (ds);
    \draw[->] (dm) to node[midway, below] {\hyperref[def:graph_functors/multi_forgetful]{\( U_M \)}} (um);
    \draw[->, bend left] (dm) to node[midway, right] {\hyperref[def:graph_functors/directed_forgetful]{\( U_D \)}} (ds);
    \draw[->, bend left] (ds) to node[midway, left] {\hyperref[def:graph_functors/directed_inclusion]{\( I_D \)}} (dm);
    \draw[->, bend left] (um) to node[midway, right] {\hyperref[def:graph_functors/undirected_forgetful]{\( U_U \)}} (us);
    \draw[->, bend left] (us) to node[midway, left] {\hyperref[def:graph_functors/undirected_inclusion]{\( I_U \)}} (um);
    \draw[->] (um) to node[midway, left, align=center] {{\scriptsize Subcategory} \\ {\scriptsize inclusion}} (hy);
  \end{tikzpicture}
  \hfill\hfill
\end{figure}

\begin{definition}\label{def:directed_multigraph}\mcite[def. 1.1.1; def. 1.1.2]{Knauer2019AlgebraicGraphTheory}
  A \term[bg=ориентиран (\cite[6]{Мирчев2001Графи}) мултиграф (\cite[7]{Мирчев2001Графи}), ru=ориентированый мультиграф (\cite[16]{ЕмеличевИПр1990ТеорияГрафов})]{directed multigraph} is a quadruple \( G = (V, A, h, t) \) consisting of:
  \begin{thmenum}[series=def:directed_multigraph]
    \thmitem{def:directed_multigraph/vertices} A set \( V \), whose elements we call \term[bg=върхове (\cite[6]{Мирчев2001Графи}), ru=вершины (\cite[279]{ЕмеличевИПр1990ТеорияГрафов})]{vertices}.

    \thmitem{def:directed_multigraph/arcs} A disjoint from \( V \) set \( A \), whose elements we call \term[bg=дъги (\cite[6]{Мирчев2001Графи}), ru=дуги (\cite[279]{ЕмеличевИПр1990ТеорияГрафов})]{arcs}.

    \thmitem{def:directed_multigraph/head} A function \( h: A \to V \), giving the \term[en=head (\cite[191]{Erickson2019Algorithms})]{head} or \term[bg=начален връх (\cite[7]{Мирчев2001Графи}), ru=начало (\cite[279]{ЕмеличевИПр1990ТеорияГрафов}), en=initial vertex (\cite[28]{Diestel2017GraphTheory})]{initial endpoint} of an arc.

    \thmitem{def:directed_multigraph/tail} A function \( t: A \to V \), giving the \term[en=tail (\cite[191]{Erickson2019Algorithms})]{tail} or \term[bg=краен връх (\cite[7]{Мирчев2001Графи}), ru=конец (\cite[279]{ЕмеличевИПр1990ТеорияГрафов}), en=terminal vertex (\cite[28]{Diestel2017GraphTheory})]{terminal endpoint} of an arc.
  \end{thmenum}

  \Cref{fig:def:directed_multigraph} illustrates this definition. The figure is not merely schematic --- it is a \hyperref[def:graph_geometric_realization/embedding]{graph embedding}.

  \begin{figure}[!ht]
    \begin{equation}\label{eq:fig:def:directed_multigraph}
      \begin{aligned}
        \includegraphics[page=1]{output/def__directed_multigraph}
      \end{aligned}
    \end{equation}
    \caption{A \hyperref[def:directed_multigraph]{directed multigraph} with a pair of parallel arcs, a pair of oppositely directed arcs and a loop. Removing the dashed arcs \( \widehat{e}_2 \), \( \widehat{e}_6 \) and \( e_8 \) makes it a \hyperref[def:directed_graph]{simple directed graph}.}\label{fig:def:directed_multigraph}
  \end{figure}

  We will need the following basic notions:
  \begin{thmenum}[resume=def:directed_multigraph]
    \thmitem{def:directed_multigraph/loop} We call the arc \( e \) a \term[bg=примка (\cite[7]{Мирчев2001Графи}), ru=петля (\cite[279]{ЕмеличевИПр1990ТеорияГрафов})]{loop} if \( h(e) = t(e) \).

    \medskip

    \thmitem{def:directed_multigraph/parallel}\mcite[28]{Diestel2017GraphTheory} We call the arcs \( e \) and \( f \) \term[bg=паралелни (ребра) (\cite[7]{Мирчев2001Графи}), ru=параллельные (рёбра) (\cite[279]{ЕмеличевИПр1990ТеорияГрафов})]{parallel} if \( h(e) = h(f) \) and \( t(e) = t(f) \) and \term{opposite} if it is not a loop and \( h(e) = t(f) \) and \( t(e) = h(f) \).

    \thmitem{def:directed_multigraph/homomorphism}\mcite[def. 1.4.1]{Knauer2019AlgebraicGraphTheory} A \term{homomorphism} between directed multigraphs \( G \) and \( H \) is a pair of functions
    \begin{align*}
      &f_V: V_G \to V_H, \\
      &f_A: A_G \to A_H,
    \end{align*}
    such that
    \begin{align}
      h_H \bincirc f_A &= f_V \bincirc h_G, \label{eq:def:directed_multigraph/homomorphism/head} \\
      t_H \bincirc f_A &= f_V \bincirc t_G. \label{eq:def:directed_multigraph/homomorphism/tail}
    \end{align}

    Isomorphisms are discussed in \cref{thm:graph_isomorphisms/multi_directed}.

    \thmitem{def:directed_multigraph/category}\mcite[48]{MacLane1998CategoryTheory} Given a Grothendieck universe \( \mscrU \), we can define the \hyperref[def:category]{category} of \( \mscrU \)-small directed multigraphs as follows:
    \begin{itemize}
      \item The \hyperref[def:category/objects]{set of objects} is the family of directed multigraphs, where both the set of vertices and the set of arcs are \( \mscrU \)-small.

      \item The \hyperref[def:category/morphisms]{morphisms} between two directed multigraphs are their homomorphisms as defined in \cref{def:directed_multigraph/homomorphism}.

      \item The \hyperref[def:category/composition]{composition of the morphisms} \( (f_V, f_A): G \to H \) and \( (g_V, g_A): H \to K \) is their componentwise function composition
      \begin{equation*}
        (g_V \bincirc f_V, g_A \bincirc f_A): G \to K.
      \end{equation*}

      \item The \hyperref[def:category/identity]{identity morphism} on the directed multigraph \( G \) is \( (\id_{V_G}, \id_{A_G}) \).
    \end{itemize}

    \thmitem{def:directed_multigraph/subgraph}\mimprovised We say that \( H \) is a \term{subgraph} of \( G \) if \( V_G \subseteq V_H \), \( A_G \subseteq A_H \) and \( h_H \) and \( t_H \) are restrictions of \( h_G \) and \( t_G \) to \( A_H \).
  \end{thmenum}
\end{definition}
\begin{comments}
  \item \incite[675]{Rosen2019DiscreteMathematics} defines \enquote{directed multigraphs}, but prefers to treat the edge set as a multiset rather than having a head and tail function.

  \incite[def. 1.1.1]{Knauer2019AlgebraicGraphTheory}, \incite[10]{MacLane1998CategoryTheory} and \incite[28]{Diestel2017GraphTheory} provide definitions similar to ours, but call them \enquote{directed graphs}. Knauer later distinguishes between simple and multigraphs, while Diestel uses \enquote{oriented graph} for what we call \enquote{simple directed graph}. \incite[8]{Bollobás1998ModernGraphTheory} briefly mentions directed multigraphs and calls them as such.

  Diestel and Bollobás reserve the term \enquote{multigraph} for \hyperref[def:hypergraph/multigraph]{undirected multigraphs}. The latter also implicitly restricts graphs to finite order. These two books use \hyperref[def:undirected_graph]{simple undirected graphs} almost exclusively.

  \incite[1]{König1986Graphentheorie}, which, as explained in \cref{rem:graph_definition}, is considered the first book on graph theory, gives a vague definition of a graph that roughly coincides with our notion of directed multigraph, with infinite graphs explicitly considered.
\end{comments}

\begin{remark}\label{rem:digraph}
  The term \enquote{digraph} is used as a shorthand for \enquote{directed graph}, for example by \incite[10]{Harary1969GraphTheory}, \incite[def. 1.1.1]{Knauer2019AlgebraicGraphTheory}, \incite[def. 10.1.2]{Rosen2019DiscreteMathematics}, \incite[571]{Stanley2012EnumerativeCombinatoricsVol1} and \incite[372]{Knuth1997ArtVol1}. \incite[28]{Diestel2017GraphTheory} also uses this terminology, and in his usage \enquote{graph} exclusively refers to \enquote{undirected graph}.

  Similarly, in Russian, \enquote{орграф} is used as a shorthand for \enquote{ориентированный граф}, for example by \incite[\S 7.1.5]{Новиков2013ДискретнаяМатематика} or \incite[279]{ЕмеличевИПр1990ТеорияГрафов}.

  We avoid the term.
\end{remark}

\begin{remark}\label{rem:subgraphs_and_subobjects}
  Subgraphs are not \hyperref[def:subobject_and_quotient]{categorical subobjects} because a graph can have distinct \hyperref[thm:graph_isomorphisms]{isomorphic} subgraphs. For example, all one-vertex subgraphs are isomorphic, but nonetheless we consider them to be different subgraphs.
\end{remark}

\begin{definition}\label{def:directed_graph}\mcite[def. 1.1.2]{Knauer2019AlgebraicGraphTheory}
  A \term{simple directed graph} is a pair \( G = (V, A) \) consisting of the following:
  \begin{thmenum}[series=def:directed_graph]
    \thmitem{def:directed_graph/vertices} A set \( V \), whose elements we call \term{vertices}.

    \thmitem{def:directed_graph/arcs} A disjoint from \( V \) set \( A \) of \hyperref[def:ordered_tuple]{ordered pairs} of \hi{distinct} vertices. As in the case of directed multigraphs, we call the elements of \( A \) \term{arcs}. If we allow vertices whose endpoints coincide, we will say that \( G \) is a \enquote{simple directed graph, possibly with loops}.
  \end{thmenum}

  Simple directed graphs have the following metamathematical properties:
  \begin{thmenum}[resume=def:directed_graph]
    \thmitem{def:directed_graph/theory}\mimprovised We can regard directed graphs as models of a \hyperref[def:first_order_theory]{first-order theory} over a \hyperref[def:first_order_signature]{first-order signature} \( \Sigma \) with a single \hyperref[rem:first_order_formula_conventions/infix]{infix} predicate symbol, possibly with the irreflexivity axiom \eqref{eq:def:binary_relation/irreflexive}.

    \thmitem{def:directed_graph/homomorphism}\mimprovised A \hyperref[def:first_order_homomorphism]{first-order homomorphism} between the \( G \) and \( H \) is a function \( f: V_G \to V_H \) such that \( (u, v) \in A_G \) implies \( (f(u), f(v)) \in A_H \).

    Isomorphisms are discussed in \cref{thm:graph_isomorphisms/simple_directed}.

    \thmitem{def:directed_graph/category}\mimprovised Given a Grothendieck universe \( \mscrU \), we can define the \hyperref[def:category]{category} of \( \mscrU \)-small simple directed graphs, possibly with loops, and their homomorphisms. Disallowing loops leads to a full subcategory.

    \thmitem{def:directed_graph/subgraph}\mimprovised We say that \( H \) is a \term{subgraph} of \( G \) if \( V_G \subseteq V_H \) and \( A_G \subseteq A_H \).
  \end{thmenum}
\end{definition}
\begin{comments}
  \item We can regard directed graphs as directed multigraphs without loops and parallel arcs. This is made precise via the inclusion functor \hyperref[def:graph_functors/directed_inclusion]{\( I_D \)}.
\end{comments}

\begin{proposition}\label{def:directed_graph_induced_by_relation}
  Every \hyperref[def:binary_relation]{binary relation} \( R \) on the set \( A \) induces the \hyperref[def:directed_graph]{directed graph} \( G = (A, R) \), in which there is an arc from \( x \) to \( y \) and only if \( x R y \).
\end{proposition}
\begin{proof}
  Trivial.
\end{proof}

\begin{definition}\label{def:hypergraph}\mcite[185]{DörflerWaller1980Hypergraphs}
  A \term[ru=гиперграф (\cite[298]{ЕмеличевИПр1990ТеорияГрафов})]{hypergraph} is a triple \( G = (V, E, \mscrE) \) consisting of the following:
  \begin{thmenum}[series=def:hypergraph]
    \thmitem{def:hypergraph/vertices} A set \( V \), whose elements we call \term[ru=вершины (\cite[298]{ЕмеличевИПр1990ТеорияГрафов})]{vertices}.

    \thmitem{def:hypergraph/edges} A disjoint from \( V \) set \( E \), whose elements we call \term[ru=рёбра (\cite[298]{ЕмеличевИПр1990ТеорияГрафов})]{edges}.

    \thmitem{def:hypergraph/endpoints} A map \( \mscrE: E \to \pow(V) \setminus \set{ \varnothing } \), giving a nonempty set of \term{endpoints} for every edge.
  \end{thmenum}

  \begin{figure}[!ht]
    \begin{equation}\label{eq:fig:def:hypergraph}
      \begin{aligned}
        \includegraphics[page=1]{output/def__hypergraph}
      \end{aligned}
    \end{equation}
    \caption{A hypergraph, which becomes an undirected multigraph after removing \( e_9 \) and a simple graph after further removing \( \widehat{e}_2 \), \( \widehat{e}_6 \) and \( e_8 \).}\label{fig:def:hypergraph}
  \end{figure}

  We will use the following basic terminology:
  \begin{thmenum}[resume=def:hypergraph]
    \thmitem{def:hypergraph/loop} We call the edge \( e \) a \term{loop} if \( \mscrE(e) \) is a one-element set.

    \thmitem{def:hypergraph/hyperedge} We say that \( e \) is a \term[ru=гипердуги (\cite[\S 7.1.5]{Новиков2013ДискретнаяМатематика}), en=hyperedge (\cite[7]{Bollobás1998ModernGraphTheory})]{hyperedge} if it has more than two endpoints.

    \thmitem{def:hypergraph/multigraph} We say that \( G \) is an \term{undirected multigraph} if it has no hyperedges.

    \thmitem{def:hypergraph/parallel}\mimprovised We call the (hyper)edges \( e \) and \( f \) \term{parallel} if \( \mscrE(e) = \mscrE(f) \).

    \thmitem{def:hypergraph/uniform}\mcite[191]{DörflerWaller1980Hypergraphs} If all edges have \( \kappa \) vertices, we say that the hypergraph is \term[ru=\( \kappa \)-однородный]{\( \kappa \)-uniform}.

    \thmitem{def:hypergraph/homomorphism}\mcite[186]{DörflerWaller1980Hypergraphs} A \term{homomorphism} between the hypergraphs \( G \) and \( H \) is a pair of functions
    \begin{align*}
      &f_V: V_G \to V_H, \\
      &f_E: E_G \to E_H,
    \end{align*}
    such that, for each edge \( e \in E_G \),
    \begin{equation}\label{eq:def:hypergraph/homomorphism}
      \mscrE_H(f_E(e)) = \set[\Big]{ f_V(v) \given* v \in \mscrE_G(e) }.
    \end{equation}

    Isomorphisms are discussed in \cref{thm:graph_isomorphisms/hypergraphs}.

    \thmitem{def:hypergraph/category}\mcite[186]{DörflerWaller1980Hypergraphs} Similarly to \hyperref[def:directed_multigraph]{directed multigraphs}, given a Grothendieck universe \( \mscrU \), we can define the \hyperref[def:category]{category} of \( \mscrU \)-small hypergraphs and hypergraph homomorphisms with componentwise composition.

    An important subcategory is the one induced by undirected multigraphs.

    \thmitem{def:hypergraph/subgraph}\mimprovised We say that \( H \) is a \term{sub(hyper)graph} of \( G \) if \( V_G \subseteq V_H \), \( E_G \subseteq E_H \) and \( \mscrE_H = \mscrE_G\restr_{E_H} \).
  \end{thmenum}
\end{definition}
\begin{comments}
  \item Unlike \enquote{hypergraph}, the term \enquote{undirected multigraphs} is not nearly as ubiquitous.
  \begin{itemize}
    \item \incite[def. 1.1.2]{Knauer2019AlgebraicGraphTheory} does not explicitly mention undirected multigraphs, but their existence is implied.

    \item \incite[10]{Harary1969GraphTheory}, \incite[28]{Diestel2017GraphTheory}, \incite[6]{Bollobás1998ModernGraphTheory} and \incite[674]{Rosen2019DiscreteMathematics} define undirected multigraphs, although all omit the prefix \enquote{undirected}. Harary disallows loops, instead referring to multigraphs with loops as \enquote{pseudographs}. Bollobás explicitly defines directed multigraphs, while Diestel uses the term \enquote{directed graph} for what we call a \hyperref[def:directed_multigraph]{directed multigraph} and \enquote{oriented graph} for what we call \enquote{simple directed graph}.

    \item \incite[3]{GondranMinoux1984GraphsAndAlgorithms} define multigraphs without specifying whether they are directed or not, but later implicitly assume that they are undirected.
  \end{itemize}
\end{comments}

\begin{definition}\label{def:undirected_graph}\mcite[2; 3]{Diestel2017GraphTheory}
  A \term{simple undirected graph} is a pair \( G = (V, E) \) consisting of the following:
  \begin{thmenum}[series=def:undirected_graph]
    \thmitem{def:undirected_graph/vertices} A set \( V \), whose elements we call \term{vertices}.
    \thmitem{def:undirected_graph/edges} A disjoint from \( V \) set \( E \) of unordered pairs of \hi{distinct} vertices, i.e. sets of the form \( \set{ u, v } \), whose elements we call \term{edges}. If we allow vertices whose endpoints coincide, we will say that \( G \) is a \enquote{simple undirected graph, possibly with loops}.
  \end{thmenum}

  We will use the following basic terminology:
  \begin{thmenum}[resume=def:undirected_graph]
    \thmitem{def:undirected_graph/homomorphism} A \term{homomorphism} between the simple undirected graphs \( G \) and \( H \) is a function \( f: V_G \to V_H \) such that \( \set{ u, v } \in E_G \) implies \( \set{ f(u), f(v) } \in E_H \).

    Isomorphisms are discussed in \cref{thm:graph_isomorphisms/simple_undirected}.

    \thmitem{def:undirected_graph/category}\mcite[example 3.1.12]{Knauer2019AlgebraicGraphTheory} Given a Grothendieck universe \( \mscrU \), we can define the \hyperref[def:category]{category} of \( \mscrU \)-small simple undirected graphs, possibly with loops, and their homomorphisms. Disallowing loops leads to a full subcategory.

    By \fullref{thm:edgeless_graph_universal_property}, the monomorphisms are precisely the injective homomorphisms and, by \fullref{thm:complete_graph_universal_property}, the epimorphisms are precisely the surjective homomorphisms.

    \thmitem{def:undirected_graph/subgraph}\mcite[4]{Diestel2017GraphTheory} We say that \( H \) is a \term{subgraph} of \( G \) if \( V_G \subseteq V_H \) and \( E_G \subseteq E_H \).
  \end{thmenum}
\end{definition}
\begin{comments}
  \item We can regard undirected graphs as undirected multigraphs without loops and parallel arcs. This is made precise via the \hyperref[con:inclusion_and_projection]{inclusion} functor \hyperref[def:graph_functors/undirected_inclusion]{\( I_U \)}.

  \item \incite[9]{Harary1969GraphTheory}, \incite[2]{Diestel2017GraphTheory} and \incite[1]{Bollobás1998ModernGraphTheory} define simple undirected graphs similarly to how we have done it. \incite[def. 10.1.1]{Rosen2019DiscreteMathematics} and \incite[def. 1.1.2]{Knauer2019AlgebraicGraphTheory} also do, however they allow loops.

  \item \incite[def. 1.4.8]{Knauer2019AlgebraicGraphTheory} define subgraphs more generally as graphs which can be embedded via injective graph homomorphisms. We prefer the vertices and edges of subgraphs to be subsets.

  \item Although simple undirected graphs are much less general than \hyperref[def:hypergraph]{hypergraphs}, \cref{thm:hypergraph_representation} allows us to regard hypergraphs as \hyperref[def:graph_coloring/proper]{properly \( 2 \)-colored} simple graphs.
\end{comments}

\begin{remark}\label{rem:theory_of_simple_undirected_graphs}
  For the purpose of fitting \hyperref[def:undirected_graph]{simple undirected graphs} in the framework of first-order logic, we can extend the \hyperref[def:directed_graph/theory]{theory of directed graphs} with the symmetry axiom \eqref{eq:def:binary_relation/symmetric}.

  This leads to an incompatible notion of homomorphisms, however --- the \hyperref[def:first_order_homomorphism]{first-order homomorphisms} of this theory are the directed graph homomorphisms defined in \cref{def:directed_graph/homomorphism} and not the more general undirected graph homomorphisms defined in \cref{def:undirected_graph/homomorphism}.

  We will still find this theory useful, for example when defining quotient graphs in \cref{def:quotient_graph}.
\end{remark}

\begin{remark}\label{rem:simple_graphs}
  Whether or simple graphs are allowed have loops again depends on the authors.

  Defining \enquote{directed graphs} as pairs \( (V, A) \), where \( A \subseteq V \times V \), without further restrictions, is common. It is done by \incite[10]{Harary1969GraphTheory}, \incite[def. 1.1.2]{Knauer2019AlgebraicGraphTheory}, \incite[21]{GondranMinoux1984GraphsAndAlgorithms}, \incite[571]{Stanley2012EnumerativeCombinatoricsVol1}, \incite[def. 10.1.2]{Rosen2019DiscreteMathematics}, \incite[190]{Erickson2019Algorithms}, \incite[279]{ЕмеличевИПр1990ТеорияГрафов}, \incite[277]{БелоусовТкачёв2004ДискретнаяМатематика} and \incite[6]{Мирчев2001Графи}. \incite[\S 7.1.5]{Новиков2013ДискретнаяМатематика} also provides the same definition, however he forbids loops unless explicitly mentioned. \incite[28]{Diestel2017GraphTheory} uses the term \enquote{oriented graph} for what we call \enquote{simple directed graph}.

  For undirected graphs we instead have several conventions:
  \begin{itemize}
    \item Some of the aforementioned authors, namely \incite[21]{GondranMinoux1984GraphsAndAlgorithms} and \incite[277]{БелоусовТкачёв2004ДискретнаяМатематика}, define directed graphs as \enquote{symmetric} undirected graphs, in which \( (u, v) \) is an arc whenever \( (v, u) \) is. This definition allows loops since the directed counterparts do.

    \item Other aforementioned authors, namely \incite[def. 1.1.2]{Knauer2019AlgebraicGraphTheory}, \incite[571]{Stanley2012EnumerativeCombinatoricsVol1}, \incite[def. 10.1.1]{Rosen2019DiscreteMathematics} \incite[190]{Erickson2019Algorithms}, and \incite[7]{Мирчев2001Графи} define undirected graphs as pairs \( (V, E) \), where \( E \) consists of one-element or two-element subsets of \( V \). This definition again allows loops.

    \item \incite[9]{Harary1969GraphTheory}, \incite[2]{Diestel2017GraphTheory}, \incite[1]{Bollobás1998ModernGraphTheory}, \incite[9]{ЕмеличевИПр1990ТеорияГрафов} and \incite[\S 7.1.2]{Новиков2013ДискретнаяМатематика} define directed graphs as pairs \( (V, E) \), where \( E \) consists of two-element subsets of \( V \). This explicitly forbid loops.

    \item \incite[377]{Knuth1997ArtVol1} explicitly forbids loops in graphs, however he defines \enquote{graphs} as \enquote{a set of points together with a set of lines}.
  \end{itemize}

  \incite[def. 1.1.2]{Knauer2019AlgebraicGraphTheory}, \incite[571]{Stanley2012EnumerativeCombinatoricsVol1} and \incite[def. 10.1.1]{Rosen2019DiscreteMathematics} use the adjective \enquote{simple} to refer to (multi)graphs without multiple edges, without referring to loops. So do \incite[2]{Diestel2017GraphTheory} and \incite[22]{Bollobás1998ModernGraphTheory}, however their definitions of graphs forbid loops by default. \incite[22]{GondranMinoux1984GraphsAndAlgorithms}, \incite[191]{Erickson2019Algorithms}, \incite[17]{ЕмеличевИПр1990ТеорияГрафов} and \incite[12]{Мирчев2001Графи} use \enquote{simple} to additionally forbid loops.

  We prefer our terminology to be as unambiguous as possible, for which reason we use \enquote{simple} in the latter sense and, if loops are allowed, we mention it explicitly.
\end{remark}

\begin{proposition}\label{thm:graph_isomorphisms}
  In order to discuss graph isomorphisms systematically, we study \hyperref[def:morphism_invertibility/isomorphism]{categorical isomorphisms} on their respective categories.

  \begin{thmenum}
    \thmitem{thm:graph_isomorphisms/multi_directed} Given \hyperref[def:directed_multigraph]{directed multigraphs} \( G \) and \( H \), the \hyperref[def:directed_multigraph/homomorphism]{directed multigraph homomorphism}
    \begin{align*}
      &f_V: V_G \to V_H, \\
      &f_A: A_G \to A_H,
    \end{align*}
    is a categorical isomorphism if and only if both \( f_V \) and \( f_A \) are bijective.

    Furthermore, the componentwise inverse \( (f_V^{-1}, f_A^{-1}) \) of \( (f_V, f_A) \) is also a homomorphism, and it is the two-sided inverse of \( (f_V, f_A) \) in the category of directed multigraphs.

    \thmitem{thm:graph_isomorphisms/simple_directed} Given \hyperref[def:directed_graph]{simple directed graphs} \( G \) and \( H \), possibly with loops, the \hyperref[def:directed_graph/homomorphism]{directed graph homomorphism} \( f: V_G \to V_H \) is a categorical isomorphism if and only if it is bijective and satisfies
    \begin{equation}\label{eq:thm:graph_isomorphisms/simple_directed}
      (u, v) \in A_G \T{if and only if} (f(u), f(v)) \in A_H.
    \end{equation}

    Furthermore, the inverse function \( f^{-1} \) of \( f \) is also a homomorphism, it satisfies \eqref{eq:thm:graph_isomorphisms/simple_directed}, and it is the two-sided inverse of \( f \) in the category of simple directed graphs.

    \thmitem{thm:graph_isomorphisms/hypergraphs} Given \hyperref[def:hypergraph]{hypergraphs} \( G \) and \( H \), the \hyperref[def:hypergraph/homomorphism]{hypergraph homomorphism}
    \begin{align*}
      &f_V: V_G \to V_H, \\
      &f_E: E_G \to E_H,
    \end{align*}
    is a categorical isomorphism if and only if both \( f_V \) and \( f_E \) are bijective.

    Furthermore, the componentwise inverse \( (f_V^{-1}, f_E^{-1}) \) of \( (f_V, f_E) \) is also a homomorphism, and it is the two-sided inverse of \( (f_V, f_E) \) in the category of directed multigraphs.

    \thmitem{thm:graph_isomorphisms/simple_undirected} Given \hyperref[def:undirected_graph]{simple undirected graphs} \( G \) and \( H \), possibly with loops, the \hyperref[def:undirected_graph/homomorphism]{undirected graph homomorphism} \( f: V_G \to V_H \) is a categorical isomorphism if and only if it is bijective and satisfies
    \begin{equation}\label{eq:thm:graph_isomorphisms/simple_undirected}
      \set{ u, v } \in E_G \T{if and only if} \set{ f(u), f(v) } \in E_H.
    \end{equation}
  \end{thmenum}
\end{proposition}
\begin{comments}
  \item The category of hypergraphs is discussed in detail by \incite{DörflerWaller1980Hypergraphs}.

  \incite[def. 1.4.1]{Knauer2019AlgebraicGraphTheory} give \eqref{eq:thm:graph_isomorphisms/simple_directed} as a definition of graph isomorphism.

  \incite[3]{Diestel2017GraphTheory}, \incite[3]{Bollobás1998ModernGraphTheory}, \incite[def. 1.4.3]{Knauer2019AlgebraicGraphTheory}, \incite[def. 10.3.1]{Rosen2019DiscreteMathematics}, \incite[13]{ЕмеличевИПр1990ТеорияГрафов}, \incite[def. 5.14]{БелоусовТкачёв2004ДискретнаяМатематика} and \incite[\S 7.1.6]{Новиков2013ДискретнаяМатематика} define simple undirected graph isomorphisms via \eqref{eq:thm:graph_isomorphisms/simple_undirected}, however avoid discussing isomorphisms for other kinds of graphs.
\end{comments}
\begin{proof}
  \SubProofOf{thm:graph_isomorphisms/multi_directed}

  \SufficiencySubProof* Suppose that \( (f_V, f_A): G \to H \) is a categorical isomorphism. Clearly it is a homomorphism. We must show that both \( f_V \) and \( f_A \) are bijective.

  We have two forgetful functors into \( \cat{Set} \) --- one for vertices and functions between them, the other one for arcs and functions between them. Hence, by \cref{thm:def:functor_invertibility/preserves_inverses}, both \( f_V \) and \( f_A \) are set isomorphisms, and, by \cref{thm:function_invertibility_categorical/fully_invertible}, both are bijective.

  \NecessitySubProof* Suppose that \( (f_V, f_A): G \to H \) is a homomorphism and that both \( f_V \) and \( f_A \) are bijective. We will show that \( (f_V^{-1}, f_A^{-1}) \) is a homomorphism from \( H \) to \( G \).

  For any arc \( e \) in \( H \), we have
  \begin{equation*}
    f_V(h_G(f^{-1}_A(e)))
    \reloset {\eqref{eq:def:directed_multigraph/homomorphism/head}} =
    h_H(f_A(f^{-1}_A(e)))
    =
    h_H(e),
  \end{equation*}
  hence
  \begin{equation*}
    h_G(f^{-1}_A(e))
    =
    f_V^{-1}(f_V(h_G(f^{-1}_A(e))))
    =
    f_V^{-1}(h_H(e)).
  \end{equation*}

  We can analogously show that
  \begin{equation*}
    t_G(f^{-1}_A(e)) = f_V^{-1}(t_H(e)).
  \end{equation*}

  Thus, the pair \( (f_V^{-1},f_A^{-1}) \) satisfies both \eqref{eq:def:directed_multigraph/homomorphism/head} and \eqref{eq:def:directed_multigraph/homomorphism/tail}, which makes it a directed multigraph homomorphism.

  Composing \( (f_V, f_A) \) and \( (f_V^{-1}, f_A^{-1}) \) (componentwise), we obtain identity homomorphisms on either \( G \) or \( H \) depending on the order of composition. Therefore, \( (f_V, f_A) \) is fully invertible, that is, a categorical isomorphism.

  \SubProofOf{thm:graph_isomorphisms/simple_directed}

  \SufficiencySubProof* Suppose that \( f: V_G \to V_H \) is a categorical isomorphism between \( G \) and \( H \). Again, by \cref{thm:def:functor_invertibility/preserves_inverses}, \( f \) is an isomorphism between \( V_G \) and \( V_H \) in the category of sets, that is, a bijective function.

  Let \( g: V_G \to V_H \) be the categorical inverse of \( f \). Then it is the inverse of \( f \) in \( \cat{Set} \), which due to the uniqueness of two-sided inverses implies that \( g \) is the inverse function \( f^{-1} \) of \( f \).

  Then, if \( (f(u), f(v)) \) is an arc of \( H \) for some vertices \( u \) and \( v \),
  \begin{equation*}
    \parens[\Big]{ f^{-1}(f(u)), f^{-1}(f(v)) } = (u, v)
  \end{equation*}
  is an edge of \( G \). Therefore, \eqref{eq:thm:graph_isomorphisms/simple_directed} holds.

  \NecessitySubProof* Suppose that \( f: V_G \to V_H \) is a bijective homomorphism from \( G \) to \( H \) and that \eqref{eq:thm:graph_isomorphisms/simple_directed} holds.

  For any arc \( (u, v) \) in \( H \), we have
  \begin{equation*}
    (u, v) = \parens[\Big]{ f(f^{-1}(u)), f(f^{-1}(v)) }.
  \end{equation*}
  and thus, by \eqref{eq:thm:graph_isomorphisms/simple_directed}, \( (f^{-1}(u), f^{-1}(v)) \) is an arc in \( G \).

  Therefore, \( f^{-1} \) is itself a homomorphism from \( H \) to \( G \), and it clearly satisfies \eqref{eq:thm:graph_isomorphisms/simple_directed}. When composed with \( f \), we obtain the identity on either \( V_G \) or \( V_H \) depending on the order of composition, hence \( f^{-1} \) is the two-sided inverse of \( f \).

  \SubProofOf{thm:graph_isomorphisms/hypergraphs}

  \SufficiencySubProof* If \( (f_V, f_E): G \to H \) is a categorical isomorphism of hypergraphs, we can prove that both \( f_V \) and \( f_E \) are bijective analogously to the case for directed multigraphs.

  \NecessitySubProof* Suppose that \( (f_V, f_E): G \to H \) is a homomorphism and that both \( f_V \) and \( f_E \) are bijective.

  For any edge \( e \) of \( H \), we have
  \begin{equation*}
    \mscrE_H(e)
    =
    \mscrE_H\parens[\Big]{ f_E(f_E^{-1}(e)) }
    \reloset {\eqref{eq:def:hypergraph/homomorphism}} =
    \set{ f_V(u) \given u in \mscrE_G(f_E^{-1}(e)) }.
  \end{equation*}

  Then
  \begin{equation*}
    \set{ f_V^{-1}(v) \given v \in \mscrE_H(e) }
    =
    \set{ f_V^{-1}(f_V(u)) \given u \in \mscrE_G(f_E^{-1}(e)) }
    =
    \mscrE_G(f_E^{-1}(e)).
  \end{equation*}

  Therefore, the pair \( (f_V^{-1}, f_E^{-1}) \) satisfies \eqref{eq:def:hypergraph/homomorphism}, which makes it a hypergraph homomorphism.

  Analogously to the case of directed multigraphs, we conclude that \( (f_V, f_E) \) is a categorical isomorphism with inverse \( (f_V^{-1}, f_E^{-1}) \).

  \SubProofOf{thm:graph_isomorphisms/simple_undirected} Both sufficiency and necessity can be proven similarly to the case of simple directed graphs.
\end{proof}

\begin{definition}\label{def:multigraph_orientation}\mcite[def. 6.2.1; def. 1.1.2]{Knauer2019AlgebraicGraphTheory}
  We call the \hyperref[def:directed_multigraph]{directed multigraph} \( D = (V, A, h, t) \) an \term[ru=ориентация (\cite[32]{ЕмеличевИПр1990ТеорияГрафов})]{orientation} of the \hyperref[def:hypergraph/multigraph]{undirected multigraph} \( G = (U, E, \mscrE) \) if \( U = V \), \( A = E \), and, for every arc \( e \in A \), we have \( \mscrE(e) = \set{ h(e), t(e) } \).

  We call \( G \) the \term{underlying undirected (multi)graph} of \( D \).
\end{definition}
\begin{comments}
  \item This definition also applies to simple graphs via the inclusion functors \hyperref[def:graph_functors/directed_inclusion]{\( I_D \)} and \hyperref[def:graph_functors/undirected_inclusion]{\( I_U \)}.
\end{comments}

\begin{definition}\label{def:graph_functors}\mimprovised
  Different kinds of graphs are related via the following functors (shown graphically in \cref{fig:def:graph_functors}):
  \begin{thmenum}
    \thmitem{def:graph_functors/directed_forgetful} The simple \hyperref[def:concrete_category]{forgetful functor}:
    \begin{flalign*}
      &U_D: \hyperref[def:directed_multigraph/category]{\T*{Directed multigraphs}} \to \hyperref[def:directed_graph/category]{\T*{Simple directed graphs with loops}}, &&\\
      &U_D(V, A, h, t) \coloneqq \parens[\Big]{ V, \set[\Big]{ \parens[\Big]{ h(a), t(a) } \given* a \in A } }, &&\\
      &U_D(f_V, f_A) \coloneqq f_V.
    \end{flalign*}

    \thmitem{def:graph_functors/directed_inclusion} A \hyperref[def:category_adjunction]{left adjoint}:
    \begin{flalign*}
      &I_D: \hyperref[def:directed_graph/category]{\T*{Simple directed graphs}} \to \hyperref[def:directed_multigraph/category]{\T*{Directed multigraphs}}, &&\\
      &I_D(V, A) \coloneqq \parens[\Big]{ V, A, (u, v) \mapsto u, (u, v) \mapsto v }, &&\\
      &I_D(f) \coloneqq \parens[\Big]{ f, (u, v) \mapsto \parens[\Big]{ f(u), f(v) } }.
    \end{flalign*}

    \thmitem{def:graph_functors/undirected_forgetful} A similar forgetful functor for undirected graphs:
    \begin{flalign*}
      &U_U: \hyperref[def:hypergraph/category]{\T*{Undirected multigraphs}} \to \hyperref[def:undirected_graph/category]{\T*{Simple undirected graphs with loops}}, &&\\
      &U_U(V, E, \mscrE) \coloneqq \parens[\Big]{ V, \set[\Big]{ \mscrE(e) \given* e \in E } }, &&\\
      &U_U(f_V, f_E) \coloneqq f_V.
    \end{flalign*}

    \thmitem{def:graph_functors/undirected_inclusion} A left adjoint:
    \begin{flalign*}
      &I_U: \hyperref[def:undirected_graph/category]{\T*{Simple undirected graphs}} \to \hyperref[def:hypergraph/category]{\T*{Undirected multigraphs}}, &&\\
      &I_U(V, E) \coloneqq \parens[\Big]{ V, E, e \mapsto e }, &&\\
      &I_U(f) \coloneqq \parens[\Big]{ f, \set{ u, v } \mapsto \set{ f(u), f(v) } }.
    \end{flalign*}

    \thmitem{def:graph_functors/multi_forgetful} A functor that forgets the \hyperref[def:multigraph_orientation]{orientation} of a multigraph:
    \begin{flalign*}
      &U_M: \hyperref[def:directed_multigraph/category]{\T*{Directed multigraphs}} \to \hyperref[def:hypergraph/category]{\T*{Undirected multigraphs}}, &&\\
      &U_M(V, A, h, t) \coloneqq \parens[\Big]{ V, A, e \mapsto \set{ h(e), t(e) } }, &&\\
      &U_M(f_V, f_A) \coloneqq (f_V, f_A).
    \end{flalign*}

    \thmitem{def:graph_functors/simple_forgetful} A functor that forgets the orientation of a simple graph:
    \begin{flalign*}
      &U_S: \hyperref[def:directed_graph/category]{\T*{Simple directed graphs}} \to \hyperref[def:undirected_graph/category]{\T*{Simple undirected multigraphs}}, &&\\
      &U_S(V, A) \coloneqq \parens[\Big]{ V, A, (u, v) \mapsto \set{ u, v } }, &&\\
      &U_S(f) \coloneqq f.
    \end{flalign*}

    Note that opposite arcs map to the same edge.

    \thmitem{def:graph_functors/simple_doubling} A right adjoint to \( U_S \), which doubles each edge to produce arcs in both directions:
    \begin{flalign*}
      &D_S: \hyperref[def:undirected_graph/category]{\T*{Simple undirected graphs}} \to \hyperref[def:directed_multigraph/category]{\T*{Simple directed graphs}}, &&\\
      &D_S(V, E) \coloneqq \parens[\Big]{ V, \set[\Big]{ (u, v) \in V^2 \given \set{ u, v } \in E } }, &&\\
      &D_S(f) \coloneqq f.
    \end{flalign*}
  \end{thmenum}
\end{definition}
\begin{comments}
  \item \Cref{ex:def:category_adjunction/us_ds} discusses the doubling functor from \cref{def:graph_functors/simple_doubling}.
  \item We can introduce a doubling functor for multigraphs similar to \hyperref[def:graph_functors/simple_doubling]{\( D_S \)} that introduces two opposite copies of each edge in an undirected multigraph.

  Unfortunately, that would require choosing one vertex as the head and the other as a tail, and thus the functor depends on a \hyperref[def:choice_function]{choice function}. This choice function is unnecessary for simple graphs because ordered pairs having a first and second element, while multigraphs have abstract objects as arcs.

  Furthermore, even given a canonical choice function, this functor would not be a right inverse to \hyperref[def:graph_functors/multi_forgetful]{\( U_M \)} --- instead, we would obtain an undirected multigraph with twice as many edges as the original.

  We avoid introducing such a functor altogether.
\end{comments}

\begin{remark}\label{rem:arbitrary_kind_graph}
  We will henceforth use the phrase \enquote{arbitrary-kind graph} to refer to any of the four kinds of (un)directed (multi)graphs. We will mostly avoid working with hypergraphs, hence we will be explicit about them and say, for example, \enquote{arbitrary-kind graph of hypergraph}.

  The inclusion functors \hyperref[def:graph_functors/directed_inclusion]{\( I_D \)} and \hyperref[def:graph_functors/undirected_inclusion]{\( I_U \)} allow all definitions for multigraphs to apply to simple graphs. The transition between directed and undirected graphs is more complicated, but we will nonetheless utilize \hyperref[def:graph_functors/multi_forgetful]{\( U_M \)}, \hyperref[def:graph_functors/simple_forgetful]{\( U_S \)} and \hyperref[def:graph_functors/simple_doubling]{\( D_S \)} when necessary.

  These functors will be used implicitly, but when increased attention is required, we use these functors explicitly. See \cref{thm:graph_coloring_as_homomorphism} for an example.
\end{remark}

\begin{definition}\label{def:induced_subgraph}\mcite[def. 1.4.8]{Knauer2019AlgebraicGraphTheory}
  In an \hyperref[rem:arbitrary_kind_graph]{arbitrary-kind graph or hypergraph} \( G \), for every set \( U \) of vertices, we define the \term[ru=порождённый (подграф) (\cite[17]{ЕмеличевИПр1990ТеорияГрафов}), bg=породен (подграф) (\cite[18]{Мирчев2001Графи})]{induced subgraph} \( G[U] \) as the unique subgraph with vertex set \( U \) that contains all arcs/edges of \( G \) with endpoints in \( U \).
\end{definition}

\begin{definition}\label{def:graph_incidence}\mcite[def. 1.1.1]{Knauer2019AlgebraicGraphTheory}
  In an \hyperref[rem:arbitrary_kind_graph]{arbitrary-kind graph or hypergraph}, we say that the vertex \( v \) and the arc/edge \( e \) are \term[bg=инцидентни (ребра) (\cite[7]{Мирчев2001Графи}), ru=инцидентные (рёбра) (\cite[9]{ЕмеличевИПр1990ТеорияГрафов})]{incident} of \( v \) is an endpoint of \( e \).
\end{definition}

\begin{definition}\label{def:graph_adjacency}\mcite[def. 10.2.1; def. 10.2.2]{Rosen2019DiscreteMathematics}
  In an \hyperref[rem:arbitrary_kind_graph]{arbitrary-kind graph or hypergraph}, if two vertices (resp. arcs/edges) are \hyperref[def:graph_incidence]{incident} to a common arc/edge (resp. vertex), we say that they are \term[bg=съседни (върхове/дъги/ребра) (\cite[7]{Мирчев2001Графи}), ru=смежные (вершины/рёбра) (\cite[9]{ЕмеличевИПр1990ТеорияГрафов})]{adjacent}.
\end{definition}

\begin{definition}\label{def:graph_independent_set}\mcite[777]{Rosen2019DiscreteMathematics}
  We say that a subset of the vertices (resp. arcs/edges) of an \hyperref[rem:arbitrary_kind_graph]{arbitrary-kind graph or hypergraph} is an \term[bg=независимо множество (\cite[103]{Мирчев2001Графи})]{independent set} if no two vertices (resp. arcs/edges) in it are \hyperref[def:graph_adjacency]{adjacent}.
\end{definition}

\begin{remark}\label{rem:trivial_graph}
  Unlike the \hyperref[def:group/trivial]{trivial group} \( \set{ e } \) or \hyperref[def:module/trivial]{trivial \( R \)-module} \( \set{ 0 } \), which are unique up to an isomorphism, there is no trivial graph in the sense of \cref{def:trivial_object}.

  Every graph has a subgraph with \hyperref[def:graph_cardinality/order]{order} zero, and hence up to an isomorphism we have an order-zero graph (for every kind of graph discussed here).

  Another unambiguous concept is that of an edgeless graph. Every \hyperref[def:complete_graph]{complete graph} has \( 2^n \) edgeless subgraphs (one for each set of edges).
\end{remark}

\paragraph{Cardinalities in graphs}

\begin{definition}\label{def:graph_cardinality}
  Graphs have the following notions of \hyperref[thm:cardinality_existence]{cardinality}:
  \begin{thmenum}
    \thmitem{def:graph_cardinality/order}\mcite[2]{Diestel2017GraphTheory} We define the \term[ru=порядок (\cite[9]{ЕмеличевИПр1990ТеорияГрафов})]{order} \( \ord(G) \) of an \hyperref[rem:arbitrary_kind_graph]{arbitrary-kind graph or hypergraph} \( G \) as the (cardinal) number of vertices.

    For a simple graph, finitely many vertices imply finitely many arcs/edges, which justifies terminology like \enquote{finite simple graph}. For multigraphs, however, we will prefer being more concrete and use \enquote{finite-order (multi)graph}.

    \thmitem{def:graph_cardinality/directed_degree}\mimprovised We define the \term[bg=полустепен на изхода (\cite[8]{Мирчев2001Графи}), ru=полустепень исхода (\cite[283]{ЕмеличевИПр1990ТеорияГрафов}), en=out-degree (\cite[def. 1.1.7]{Knauer2019AlgebraicGraphTheory})]{out-degree} \( \deg_{\op{out}}(v) \) (resp. \term[bg=полустепен на входа (\cite[8]{Мирчев2001Графи}), ru=полустепень захода (\cite[283]{ЕмеличевИПр1990ТеорияГрафов}), en=in-degree (\cite[def. 1.1.8]{Knauer2019AlgebraicGraphTheory})]{in-degree} \( \deg_{\op{in}}(v) \)) of a vertex \( v \) in a \hyperref[def:directed_multigraph]{directed (multi)graph} as the (cardinal) number of arcs starting (resp. ending) at \( v \).

    We define the \term[bg=степен (\cite[8]{Мирчев2001Графи}), ru=степень (\cite[283]{ЕмеличевИПр1990ТеорияГрафов})]{total degree} \( \deg(v) \) of \( v \) as the sum of the two.

    \thmitem{def:graph_cardinality/undirected_degree}\mimprovised In an undirected graph or hypergraph, consider the \hyperref[def:multiset]{multiset} of all edges incident to \( v \), with loops having multiplicity \( 2 \) and all other edges having multiplicity \( 1 \).

    We define the \term{degree} \( \deg(v) \) as the \hyperref[def:multiset/cardinality]{cardinality} of this multiset.

    \thmitem{def:graph_cardinality/local}\mcite[210]{Diestel2017GraphTheory} We say that a graph is \term{locally finite} (resp. \term{locally countable}) if the degree of any vertex is finite (resp. countable).
  \end{thmenum}
\end{definition}

\begin{proposition}\label{thm:degree_of_undirected_counterpart}
  The degree, in the sense of \cref{def:graph_cardinality/directed_degree}, of a vertex \( v \) in a \hyperref[def:directed_multigraph]{directed multigraph} \( G \), is equal to the degree, in the sense of \cref{def:graph_cardinality/undirected_degree}, of \( v \) in the undirected counterpart \( \hyperref[def:graph_functors/multi_forgetful]{U_M}(G) \) of \( G \).
\end{proposition}
\begin{comments}
  \item When proving statements about vertex degrees, it thus makes sense to prove it for undirected multigraphs, since it will then also automatically apply to directed graphs.
\end{comments}
\begin{proof}
  We have adjusted our definitions so that this holds.
\end{proof}

\begin{remark}\label{rem:counting_loops_twice}
  Loops may contribute either \( 1 \) or \( 2 \) towards the degree of a graph. This choice simplifies \cref{thm:sum_of_endpoint_degrees} and hence \cref{thm:sum_of_graph_degrees}, as well as \cref{def:graph_cycle}, among others.

  We describe here several conventions.
  \begin{itemize}
    \item \incite[27]{ЕмеличевИПр1990ТеорияГрафов}, who diligently distinguish between directed and undirected graphs, defines degrees in (simple) directed graphs as sums of out-degrees and in-degrees, like we do, and in undirected graphs they, like us, count loops twice.

    In chapter XI, dedicated to hypergraphs, they define the degree of a vertex as the number of incident hyperedges without mentioning how loops are handled. There is no reason to believe that they would generalize from undirected graphs without counting loops twice.

    \item \incite[277]{БелоусовТкачёв2004ДискретнаяМатематика}, who also give separate definitions for directed and undirected graphs, also define degrees in (simple) directed graphs as sums of out-degrees and in-degrees, but for undirected graphs they avoid counting loops twice.

    \item \incite[def. 10.2.3]{Rosen2019DiscreteMathematics}, who also give separate definitions between directed and undirected graphs, defines degrees for (simple) undirected graphs by explicitly counting loops twice, but for directed graphs they only define the in-degrees and out-degrees without the \enquote{total} degrees.

    \item \incite[191]{Erickson2019Algorithms} and \incite[8]{Мирчев2001Графи} define degrees without counting loops twice for both directed and undirected graphs.

    \item \incite[def. 1.1.7]{Knauer2019AlgebraicGraphTheory} avoids defining \enquote{total} degrees for directed graphs, leaving only out-degrees and in-degrees. For undirected graphs, however, in \cite[def. 1.1.8]{Knauer2019AlgebraicGraphTheory} he defines the degree of a vertex \( v \) as the number of edges incident to \( v \), which without our multiset trick only counts loops once.

    \item \incite[372]{Knuth1997ArtVol1} also defines out-degrees and in-degrees, but avoids defining degrees for both directed and undirected graphs.

    \item \incite[8]{Bollobás1998ModernGraphTheory} hints that loops should be counted twice.
  \end{itemize}
\end{remark}

\begin{definition}\label{def:isolated_vertex}\mcite[5]{Diestel2017GraphTheory}
  We say that a vertex in an \hyperref[rem:arbitrary_kind_graph]{arbitrary-kind graph or hypergraph} is \term[bg=изолирани (върхове) (\cite[8]{Мирчев2001Графи}), ru=изолированные (вершины) (\cite[26]{ЕмеличевИПр1990ТеорияГрафов})]{isolated} if its \hyperref[def:graph_cardinality/directed_degree]{total degree} is zero.
\end{definition}

\begin{lemma}\label{thm:sum_of_endpoint_degrees}
  Fix an undirected multigraph \( G = (V, E, \mscrE) \). Fix an edge \( e \) and consider the graph \( G = (V, E \setminus \set{ e }, \mscrE) \). Denote by \( \deg'(v) \) the vertex degree in \( G' \).

  Then
  \begin{equation}\label{eq:thm:sum_of_endpoint_degrees}
    \sum_{v \in \mscrE(e)} \deg(v) = \sum_{v \in \mscrE(e)} \deg'(v) + 2.
  \end{equation}
\end{lemma}
\begin{proof}
  If \( e \) is a loop at \( v \), by definition we have
  \begin{equation*}
    \sum_{w \in \mscrE(e)} \deg(w) = \deg(v) = \deg'(v) + 2 = \sum_{w \in \mscrE(e)} \deg'(w) + 2.
  \end{equation*}

  If \( e \) is an edge between \( u \) and \( v \), then
  \begin{equation*}
    \sum_{w \in \mscrE(e)} \deg(w)
    =
    \deg(u) + \deg(v)
    =
    \deg'(u) + 1 + \deg'(v) + 1
    =
    \sum_{w \in \mscrE(e)} \deg'(w) + 2.
  \end{equation*}
\end{proof}

\begin{proposition}\label{thm:sum_of_graph_degrees}
  In an \hyperref[rem:arbitrary_kind_graph]{arbitrary-kind graph} with finitely many arcs/edges, the sum of the degrees of all vertices is twice the number of arcs/edges.
\end{proposition}
\begin{comments}
  \item As long as a graph only has finitely many arcs/edges, only finitely many vertices have positive degree, so summing all degrees is justified.
  \item As a consequence, the sum of degrees if even. \incite[4]{Bollobás1998ModernGraphTheory} and \incite[prop. 5.1]{ЕмеличевИПр1990ТеорияГрафов} call this consequence the \enquote{handshaking lemma} because of the observation that the total number of hands shaken at a party is even. \incite[thm 10.2.1]{Rosen2019DiscreteMathematics} uses \enquote{handshaking lemma} for the proposition itself.
\end{comments}
\begin{proof}
  As per \cref{thm:degree_of_undirected_counterpart}, it is sufficient to only prove the statement for undirected multigraphs.

  We will use induction on the number \( n \) of arcs.

  \begin{itemize}
    \item If there are zero edges, then the degree of every vertex is zero, hence the sum of all degrees is also zero.

    \item Suppose that the statement holds for multigraphs with \( n \) arcs and consider an undirected multigraph \( G = (V, E, \mscrE) \) with \( n + 1 \) arcs.

    Fix any arc \( e \) and consider \( G' \coloneqq (V, E \setminus \set{ e }, \mscrE) \). Denote by \( \deg'(v) \) the degree of \( v \) in \( G' \). Then \cref{thm:sum_of_endpoint_degrees} implies that
    \begin{equation*}
      \overbrace{\sum_{v \in V} \deg(v) = \sum_{v \in V \setminus \mscrE(e)} \underbrace{\deg(v)}_{\deg'(v)} + \underbrace{\sum_{v \in \mscrE(e)} \deg(v)}_{\sum_{v \in \mscrE(e)} \deg'(v) + 2} - 2}^{2n \T*{by the inductive hypothesis}} + 2
      =
      2(n + 1)
    \end{equation*}
  \end{itemize}
\end{proof}

\begin{corollary}\label{thm:odd_degree_vertices}\mcite[prop. 1.2.1]{Diestel2017GraphTheory}
  In an \hyperref[rem:arbitrary_kind_graph]{arbitrary-kind graph} with finitely many arcs/edges, the number of vertices of odd \hyperref[def:graph_cardinality/directed_degree]{total degree} is even.
\end{corollary}
\begin{comments}
  \item This corollary, while by itself inappreciable, is essential for proving \fullref{thm:eulers_theorem_for_graphs}. It first appeared in \cite{Euler1741Bridges} by Leonhard Euler, considered the first paper on graph theory and discussed in \cref{ex:konigsberg_bridges}.
\end{comments}
\begin{proof}
  As per \cref{thm:degree_of_undirected_counterpart}, it is sufficient to consider some undirected multigraph \( G = (V, E, \mscrE) \).

  We have
  \begin{equation}\label{eq:thm:odd_degree_vertices/proof}
    \underbrace{\sum_{v \in V} \deg(v)}_{\T*{even by} \cref{thm:sum_of_graph_degrees}} = \underbrace{\sum_{\deg(v) \T*{is even}} \deg(v)}_{\T{sum of even numbers}} + \underbrace{\sum_{\deg(v) \T*{is odd}} \deg(v)}_{\T{sum of odd numbers}}.
  \end{equation}

  In order for \eqref{eq:thm:odd_degree_vertices/proof} to hold, there must be an even number of vertices of odd degree.
\end{proof}

\begin{example}\label{ex:infinite_integer_graphs}
  An example of an infinite \hyperref[def:directed_graph]{simple directed graph} is the \hyperref[def:transitive_reduction]{transitive reduction} of the positive integers:
  \begin{equation}\label{eq:ex:infinite_integer_graphs/positive}
    \begin{aligned}
      \includegraphics[page=1]{output/ex__infinite_integer_graphs__positive}
    \end{aligned}
  \end{equation}

  Since the graph is simple, we have
  \begin{equation*}
    \deg(n) = \begin{cases}
      \deg_{\op{in}}(n) + \deg_{\op{out}}(n) = 0 + 1 = 1, &n = 0, \\
      \deg_{\op{in}}(n) + \deg_{\op{out}}(n) = 2 + 1 = 2, &n > 0 \\
    \end{cases}
  \end{equation*}

  The \hyperref[def:categorical_diagram]{categorical diagram} corresponding to this graph is used to define direct limits in \cref{def:direct_and_inverse_limits/direct}.

  Another related graph is based on the negative integers:
  \begin{equation}\label{eq:ex:infinite_integer_graphs/negative}
    \begin{aligned}
      \includegraphics[page=1]{output/ex__infinite_integer_graphs__negative}
    \end{aligned}
  \end{equation}

  The categorical diagram corresponding to this graph is used to define inverse limits in \cref{def:direct_and_inverse_limits/inverse}.

  Finally, the union of the two with zero added gives us the following directed graph:
  \begin{equation}\label{eq:ex:infinite_integer_graphs/two_sided}
    \begin{aligned}
      \includegraphics[page=1]{output/ex__infinite_integer_graphs__two_sided}
    \end{aligned}
  \end{equation}

  All three graphs \hyperref[def:graph_cardinality/local]{locally finite} but have infinite \hyperref[def:graph_cardinality/order]{order}.
\end{example}
