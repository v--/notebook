\section{Group theory}\label{sec:group_theory}

Modern algebra takes its roots in generalizing the operations on \hyperref[def:integers]{integers} and \hyperref[def:real_numbers]{real numbers}. Both addition and multiplication are \hyperref[def:binary_operation/commutative]{commutative} and, if we want to generalize their properties, it is sensible to study commutative operations.

Another type of objects that usually fits in the same algebraic framework are \hyperref[def:function]{functions} and their \hyperref[def:set_valued_map/composition]{composition}. Functions from a set to itself can be composed to form another function of the same type, similarly to how two integers can be added to obtain another integer. The main difference is in the non-commutativity of function composition.

This suggests that we use the same algebraic structures to study both generalizations of numbers and generalizations of functions over a set. The first case is commutative, the second is not. This is why commutative and non-commutative structures, even though they are similarly defined, can have very different properties and applications.

We shall not attempt to give a precise definition for an \term{algebraic structure}. There are very general frameworks for doing so --- for example \term{algebras over monads} discussed in \cite[ch. VI]{MacLane1998Categories} or \term{universal algebras} discussed in \cite[sec. 1.2.]{ЦаленкоШульгейфер1974Категории}. Their complexity, however, is unjustified for us. We will instead build standard algebraic structures from \enquote{base building blocks}, although we will utilize very general definitions like categorical kernels and images defined in \fullref{def:zero_morphisms}.

The simplest algebraic structures that will be of interest to us are \hyperref[def:group]{groups} and their less well-behaved generalizations, \hyperref[def:semigroup]{semigroups} and \hyperref[def:monoid]{monoids}.

Except as a building block for more complicated algebraic structures, groups arise whenever some mathematical structure exhibits symmetries, and this concept is formalized via \hyperref[def:automorphism_group]{automorphism groups} and \hyperref[def:group_action]{group actions}. If, instead of symmetries we have non-invertible but nonetheless well-behaved transformations, we can instead study \hyperref[def:endomorphism_monoid]{endomorphism monoids} and \hyperref[def:monoid_action]{monoid actions}.
