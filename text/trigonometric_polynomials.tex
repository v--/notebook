\subsection{Trigonometric polynomials}\label{subsec:trigonometric_polynomials}

\begin{definition}\label{def:trigonometric_polynomial}
  We define the \term{trigonometric polynomials} over \( \BbbC \) as the \hyperref[def:laurent_polynomial_algebra]{Laurent polynomials} \( \BbbC[e^{iz}] \). A trigonometric polynomial \( p \in \BbbC[e^{iz}] \) can be written as
  \begin{equation}\label{def:trigonometric_polynomial/exponential}
    p(z) = \sum_{k \in \BbbZ} c_k e^{ikz}
  \end{equation}
  or, using \hyperref[thm:exponential_trigonometric_identities/eulers_formula]{Euler's formula}, rewritten in the more conventional notation (see \cite[1]{Боянов2008ЧислениМетоди} or \cite[88]{Rudin1987RealAndComplexAnalysis}):
  \begin{equation}\label{def:trigonometric_polynomial/trigonometric}
    p(z) = a_0 + \sum_{k=1}^\infty [ a_k \cos(kz) + b_k \sin(kz) ],
  \end{equation}
  where we denote \( a_k \coloneqq c_k \) and \( b_k \coloneqq ic_k \).

  In particular, when using \fullref{def:trigonometric_polynomial/trigonometric}, we may regard the coefficients \( \{ a_k \}_{k=0}^\infty \) and \( \{ b_k \}_{k=1}^\infty \) as either real or complex, which is a downside of \fullref{def:trigonometric_polynomial/exponential}.

  Denote by \( \tau_n(\BbbK) \) the vector space of all trigonometric polynomials of degree at most \( n \) with coefficients in \( \BbbK \). We also introduce the subspaces \( \tau_n^\alpha{\BbbK} \) of those polynomials which \( a_0 = 0 \).
\end{definition}
