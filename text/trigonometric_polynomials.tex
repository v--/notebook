\subsection{Trigonometric polynomials}\label{subsec:trigonometric_polynomials}

\begin{definition}\label{def:ring_of_laurent_polynomials}
  The ring of \term{Laurent polynomials} in the indeterminates \( \mscrX \) over the integral domain \( D \) is obtained from the \hyperref[def:polynomial_algebra]{polynomial ring} \( D[\mscrX] \) by \hyperref[thm:adjoining_elements_to_field]{adjoining} to \( D[\mscrX] \) the set
  \begin{equation*}
    \set*{ \frac 1 X \given* X \in \mscrX }
  \end{equation*}
  of reciprocals of the indeterminates from the field of rational functions \( D(\mscrX) \).

  If \( \mscrX = \set{ X_1, \ldots, X_n } \), this ring is denoted by \( D[X_1^\pm, \ldots, X_n^\pm] \) or \( D[X_1, X_1^{-1}, \ldots, X_n, X_n^{-1}] \). Individual polynomials are written as
  \begin{equation*}
    p(X_1, \ldots, X_n) = \sum_{k_1=-\infty}^\infty \cdots \sum_{k_n=-\infty}^\infty a_{k_1, \ldots, k_n} X^{k_1} \cdots X_n^{k_n}.
  \end{equation*}
\end{definition}
\begin{comments}
  \item This is precisely the \hyperref[def:semigroup_algebra]{group algebra} \( D[\mscrX^{\oplus \BbbZ}] \).
\end{comments}

\begin{definition}\label{def:trigonometric_polynomial}
  We define the \term{trigonometric polynomials} over \( \BbbC \) as the \hyperref[def:ring_of_laurent_polynomials]{Laurent polynomials} \( \BbbC[e^{iz}] \). A trigonometric polynomial \( p \in \BbbC[e^{iz}] \) can be written as
  \begin{equation}\label{def:trigonometric_polynomial/exponential}
    p(z) = \sum_{k \in \BbbZ} c_k e^{ikz}
  \end{equation}
  or, using \hyperref[thm:exponential_trigonometric_identities/eulers_formula]{Euler's formula}, rewritten in the more conventional notation (see \cite[1]{Боянов2008} or \cite[88]{Rudin1986RealAndComplex}):
  \begin{equation}\label{def:trigonometric_polynomial/trigonometric}
    p(z) = a_0 + \sum_{k=1}^\infty [ a_k \cos(kz) + b_k \sin(kz) ],
  \end{equation}
  where we denote \( a_k \coloneqq c_k \) and \( b_k \coloneqq ic_k \).

  In particular, when using \fullref{def:trigonometric_polynomial/trigonometric}, we may regard the coefficients \( \{ a_k \}_{k=0}^\infty \) and \( \{ b_k \}_{k=1}^\infty \) as either real or complex, which is a downside of \fullref{def:trigonometric_polynomial/exponential}.

  Denote by \( \tau_n(\BbbK) \) the vector space of all trigonometric polynomials of degree at most \( n \) with coefficients in \( \BbbK \). We also introduce the subspaces \( \tau_n^\alpha{\BbbK} \) of those polynomials which \( a_0 = 0 \).
\end{definition}
