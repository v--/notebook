\subsection{Natural numbers}\label{subsec:natural_numbers}

\begin{definition}\label{def:peano_arithmetic}\mcite[def. 2.7.7]{VanDalen2004}
  Peano arithmetic (commonly abbreviated as \logic{PA}) is a \hyperref[def:first_order_theory]{theory} in \hyperref[subsec:first_order_logic]{first-order predicate logic} for describing \hyperref[def:natural_numbers]{natural numbers} and their operations. It can also be formulated in \hyperref[rem:higher_order_logic]{second-order logic} or entirely within \hyperref[sec:set_theory]{set theory} (especially considering that we are working inside an ambient \hyperref[rem:standard_model_of_set_theory]{standard} \hyperref[rem:transitive_model_of_set_theory]{transitive} model of \hyperref[def:axiom_of_universes]{\logic{ZFC+U}}), however in this document we usually give preference to the first-order logic formulation of a theory.

  Peano's original axioms from \cite[1]{Peano1889} used sets as fundamental notions. We prefer using first-order logic, as it is done by Dirk van Dalen in \cite[def. 2.7.7]{VanDalen2004}. This allows us to use isolate concepts related to natural numbers that do not depend on sets. We do work with a model of Peano arithmetic --- see \fullref{def:natural_numbers} --- and we distinguish statements about that model and statements about the logical theory presented here.

  The \hyperref[def:first_order_language]{language} of the theory consists of
  \begin{thmenum}[series=def:peano_arithmetic]
    \thmitem{def:peano_arithmetic/zero} A constant \( 0 \) for representing \term{zero}. We can alternatively require a constant for \( 1 \), but this would lead to worse metamathematical properties as discussed in \fullref{rem:peano_arithmetic_zero}.

    \thmitem{def:peano_arithmetic/succ} A unary \hyperref[def:first_order_language/func]{functional symbol} \( s \), called the \term{successor operation}.

    The successor function is only a technicality used for establishing basic properties and for defining addition and multiplication, both in this subsection and in \fullref{def:omega_operations}.

    We will only use the abstract successor operation prior to proving the familiar properties of addition and multiplication, although we will later use the \hyperref[def:ordinal_successor]{ordinal successor operation} for building a model of \logic{PA} --- see \fullref{thm:omega_is_model_of_pa}.

    \thmitem{def:peano_arithmetic/plus} An \hyperref[rem:first_order_formula_conventions/infix]{infix} binary functional symbol \( + \) for denoting \term{addition}.

    See \fullref{thm:natural_number_addition_properties} for the algebraic properties of natural number addition.

    \thmitem{def:peano_arithmetic/mult} Another infix binary functional symbol \( \cdot \) for denoting \term{multiplication}. Outside the object language we usually use juxtaposition instead.

    See \fullref{thm:natural_number_multiplication_properties} for the algebraic properties of natural number multiplication.

    As with all \hyperref[def:semiring]{semirings}, multiplication has higher priority than addition. In the unambiguous language defined in \fullref{ex:natural_number_arithmetic_grammar/rules}, this means that we can use the shorthand \( \xi + \eta \cdot \zeta \) for \( ((\xi \cdot \eta) + \zeta) \). We often use juxtaposition for denoting multiplication.
  \end{thmenum}

  As usual, in order to avoid parentheses, we assume that multiplication has a higher precedence and thus the right-hand side of axiom \eqref{eq:def:peano_arithmetic/PA7} should be parenthesized as \( ((\eta \cdot \xi) + \xi) \). We avoid excessive parentheses in formulas as per our convention \fullref{rem:propositional_formula_parentheses}.

  We impose the following base \hyperref[def:first_order_theory]{axioms}:
  \begin{thmenum}[resume=def:peano_arithmetic]
    \thmitem[def:peano_arithmetic/PA1]{PA1} The successor function is \hyperref[thm:function_invertibility_categorical/nonempty_left_invertible]{injective}. This can be stated as follows:
    \begin{equation}\label{eq:def:peano_arithmetic/PA1}\tag{\logic{PA1}}
      s(\xi) \doteq s(\eta) \rightarrow \xi \doteq \eta.
    \end{equation}

    We use here the convention for implicit universal quantification described in \fullref{rem:mathematical_logic_conventions/quantification}.

    \thmitem[def:peano_arithmetic/PA2]{PA2} Zero is not the successor of any natural number. Symbolically,
    \begin{equation}\label{eq:def:peano_arithmetic/PA2}\tag{\logic{PA2}}
      \neg \qexists \xi (s(\xi) \doteq 0).
    \end{equation}

    \thmitem[def:peano_arithmetic/PA3]{PA3} The \term{axiom schema of induction} roughly states that for a property to hold for all natural numbers it is sufficient for the following two conditions to be met:
    \begin{itemize}
      \item The property holds for \( 0 \).
      \item We can prove that is holds for any number by assuming that it holds for its predecessor.
    \end{itemize}

    See the proof of \fullref{thm:nonzero_natural_numbers_have_predecessors} for a detailed discussion.

    To describe this formally, we state that for any \hyperref[def:first_order_language/var]{variables} \( \xi \) and \( \eta \) and any formula \( \varphi \) not containing \underline{\( \eta \)} as a \hyperref[def:first_order_syntax/formula_free_variables]{free variable}, the following is an axiom:
    \begin{equation}\label{eq:def:peano_arithmetic/PA3}\tag{\logic{PA3}}
      \parens[\Big]
        {
          \underbrace{\varphi[\xi \mapsto 0]}_{\T{base case}}
          \wedge
          \qforall \eta \parens[\Big]
            {
              \overbrace
                {
                  \underbrace{ \varphi[\xi \mapsto \eta] }_{\mathclap{\substack{\T{inductive} \\ \T{hypothesis}}}}
                  \rightarrow
                  \underbrace{ \varphi[\xi \mapsto s(\eta)] }_{\mathclap{\substack{\T{inductive step} \\ \T{conclusion}}}}
                }^{\T{inductive step}}
            }
        }
      \rightarrow
      \underbrace{ \qforall \eta \varphi[\xi \mapsto \eta] }_{\T{conclusion}}.
    \end{equation}

    It is important to highlight that \( \varphi \) may have any set of free variables, as long as \( \eta \) is not among them. As explained in \fullref{rem:mathematical_logic_conventions/quantification}, we avoid excessive universal quantification. Of course, the axiom is only interesting if \( \xi \in \boldop{Free}(\varphi) \). If \( \zeta_1, \ldots, \zeta_n \) are all the other free variables of \( \varphi \), then the \hyperref[thm:implicit_universal_quantification]{universal closure} of the corresponding axiom is
    \begin{equation}\label{eq:def:peano_arithmetic/PA3_quantified}\tag{PA3'}
      \qforall {\zeta_1} \cdots \qforall {\zeta_n}
      \parens[\Bigg]
      {
        \parens[\Big]
          {
            \underbrace{\varphi[\xi \mapsto 0]}_{\T{base case}}
            \wedge
            \qforall \eta \parens[\Big]
              {
                \overbrace
                  {
                    \underbrace{ \varphi[\xi \mapsto \eta] }_{\mathclap{\substack{\T{inductive} \\ \T{hypothesis}}}}
                    \rightarrow
                    \underbrace{ \varphi[\xi \mapsto s(\eta)] }_{\mathclap{\substack{\T{inductive step} \\ \T{conclusion}}}}
                  }^{\T{inductive step}}
              }
          }
        \rightarrow
        \underbrace{ \qforall \eta \varphi[\xi \mapsto \eta] }_{\T{conclusion}}
      }.
    \end{equation}

    Thus, the axiom holds for any assignment for the variables \( \zeta_1, \ldots, \zeta_n \). For this reason, we call these variables \term{parameters}. Parameters in axiom schemas are further discussed in \fullref{def:set_builder_notation} in relation to comprehension in set theory.

    See \fullref{rem:induction} for a more detailed discussion of induction in general and \fullref{thm:omega_recursion} for the corresponding recursion principle.
  \end{thmenum}

  The theory we obtain without the binary operations and with only the axioms \eqref{eq:def:peano_arithmetic/PA1}-\eqref{eq:def:peano_arithmetic/PA3} is itself sometimes called Peano arithmetic. The operations are defined inductively, however, and there is no way for us to formalize them within the object theory without adding them to the language and theory itself.

  \begin{thmenum}[resume=def:peano_arithmetic]
    \thmitem[def:peano_arithmetic/PA4+5]{PA4+5} The next two axioms inductively define how addition is supposed to work:
    \begin{align}
      \xi + 0       &\doteq \xi           \label{eq:def:peano_arithmetic/PA4}\tag{\logic{PA4}} \\
      \xi + s(\eta) &\doteq s(\xi + \eta) \label{eq:def:peano_arithmetic/PA5}\tag{\logic{PA5}}
    \end{align}

    \thmitem[def:peano_arithmetic/PA6+7]{PA6+7} The final two axioms are for multiplication:
    \begin{align}
      \xi \cdot 0       &\doteq 0                    \label{eq:def:peano_arithmetic/PA6}\tag{\logic{PA6}} \\
      \xi \cdot s(\eta) &\doteq \xi \cdot \eta + \xi \label{eq:def:peano_arithmetic/PA7}\tag{\logic{PA7}}
    \end{align}
  \end{thmenum}
\end{definition}

\begin{remark}\label{rem:peano_arithmetic_zero}
  It is common to consider the first natural numbers to be \( 0 \). This is done By Dirk van Dalen in \cite[def. 2.7.7]{VanDalen2004} and by Herbert Enderton in \cite[71]{Enderton1977Sets}. Peano himself, however, considered \( 1 \) to be the first natural number - see \cite[1]{Peano1889}.

  Whether \( 0 \) is considered to be a natural number is a matter of convention. The operations defined via \eqref{eq:def:peano_arithmetic/PA4}-\eqref{eq:def:peano_arithmetic/PA7} can be modified to work if \( 1 \) was instead the first natural number.

  We make choose for \( \BbbN \) to start with \( 0 \), however we often avoid referring to the set \( \BbbN \) of natural numbers and instead rely on the concepts \enquote{nonnegative} and \enquote{positive} integers formally defined in \fullref{def:integer_ordering}.
\end{remark}

\begin{definition}\label{def:natural_numbers}
  We define the set of \term[bg=естествени числа,ru=натуральные числа]{natural numbers} \( \BbbN \) as the \hyperref[thm:smallest_inductive_set_existence]{smallest inductive set} \( \omega \) with the \hyperref[def:first_order_structure/interpretation]{interpretation} described in \fullref{thm:omega_is_model_of_pa}.

  We do not depend on any particular properties of \( \omega \), but we use it because our construction of it is careful and purposely does not use natural numbers to avoid circularity. We are working in an ambient \hyperref[rem:standard_model_of_set_theory]{standard} \hyperref[rem:transitive_model_of_set_theory]{transitive} model of \hyperref[def:axiom_of_universes]{\logic{ZFC+U}} and hence we will conflate \( \BbbN \) with \( \omega \) as sets, however the first is also a \hyperref[def:first_order_structure]{structure of first-order logic}.

  We use the usual notation
  \begin{align*}
    0 &\coloneqq \varnothing \\
    1 &\coloneqq \op{succ}(\varnothing) = \set{ \varnothing } \\
    2 &\coloneqq \op{succ}(\op{succ}(\varnothing)) = \set{ \varnothing, \set{ \varnothing } } \\
      &\vdots
  \end{align*}
  and continue to use the notation functional symbols from \fullref{def:peano_arithmetic}, however we now denote the corresponding interpretations in the structure \( \BbbN \).

  See \fullref{ex:natural_number_arithmetic_grammar/rules} for a simple \hyperref[def:formal_grammar]{grammar} that produces numeric symbols in their decimal notation.
\end{definition}

\begin{remark}\label{rem:standard_models_of_arithmetic}
  At this point, we have two kinds of natural numbers:
  \begin{itemize}
    \item We have natural numbers within the metatheory. This is our mental model of the natural numbers, and it is used for distinguishing between \enquote{unary} functional symbols like \( s \) and \enquote{binary} functional symbols like \( + \). This is mostly used within logic itself.

    \item We have the set of natural numbers \( \BbbN \) defined in \fullref{def:natural_numbers}. These are the numbers which we have defined formally, whose properties we study and the numbers which we use in the entire document. The properties of \( \BbbN \) help us develop a better mental model, which in turn changes our perception of the natural numbers within the metatheory.
  \end{itemize}

  We want the two sets of natural numbers to coincide. This is important when talking about, for example, \hyperref[def:sequence]{sequences}. If a number in \( \BbbN \) is not a natural number within the metatheory, we say that it is \term{nonstandard}. The existence of nonstandard models is guaranteed by \fullref{thm:upward_lowenheim_skolem_theorem}. There cannot be numbers in the metatheory that are not in \( \BbbN \) because a model of \logic{PA} cannot have a finite domain and the natural numbers are the smallest metalogical infinite set.

  A model of \logic{PA} which contains precisely the numbers in the metatheory is called a \term{standard model}. For the purpose of this document, it is sufficient to accept the convention that \( \BbbN \) is a standard model of \logic{PA}.
\end{remark}

\begin{proposition}\label{thm:nonzero_natural_numbers_have_predecessors}
  Every nonzero natural number has a unique predecessor. More precisely, zero has no predecessor and for any nonzero number \( n \) there exists a unique number \( m \) such that \( n = s(m) \). We will denote this predecessor by \( p(n) \).
\end{proposition}
\begin{proof}
  This proof is exemplar because it clearly demonstrates both the distinction between inductive and deductive reasoning and the role of the main three axioms.

  \begin{itemize}
    \item The axiom \eqref{eq:def:peano_arithmetic/PA1} states that the function \( s \) is injective. By the equivalences in \fullref{def:function_invertibility/injective} its \hyperref[def:multi_valued_function/inverse]{inverse multi-valued function} is actually a \hyperref[def:partial_function]{single-valued partial function}. Denote this inverse by \( p \).

    \item The axiom \eqref{eq:def:peano_arithmetic/PA2} states that the function \( s \) is not surjective. By the equivalences in \fullref{def:function_invertibility/surjective}, the inverse \( p \) is not a \hyperref[def:multi_valued_function/total]{total function}.
  \end{itemize}

  What we have shown up until this point in the proof is deductive --- we have restated the first two axioms of \logic{PA} and used some equivalent conditions that allowed us to deduce properties of the inverse function \( p \) of \( s \). We did all of this by following the precise rules of \hyperref[rem:classical_logic]{classical logic} described formally in \fullref{subsec:deductive_systems}. This reasoning emulates \hyperref[eq:def:def:axiomatic_deductive_system/mp]{modus ponens}.

  Now we will show that every nonzero natural number has a predecessor. That is, that the function \( p \) is not defined only at \( 0 \). To highlight the logical structure of this proof, we will use \hyperref[def:first_order_natural_deduction_system]{first-order natural deduction} rather than work with the model \( \BbbN \) of \logic{PA}.

  Denote by \( \theta \) the formula
  \begin{equation*}
    \xi \doteq 0 \vee \qexists \zeta (\xi \doteq s(\zeta)).
  \end{equation*}

  Clearly \( \xi \) is the only free variable in \( \theta \). We want to derive the formula \( \qforall \eta \theta[\xi \mapsto \eta] \) from the axioms of \logic{PA}.

  In this part of the proof we will use inductive reasoning. This will highlight that \eqref{eq:def:peano_arithmetic/PA3} is not an axiom schema about specifying properties, but rather about introducing a proof technique that does not hold for general \hyperref[def:first_order_theory]{logical theories}. We will not attempt to prove \( \qforall \eta \theta[\xi \mapsto \eta] \) directly. Instead, we will prove a more complicated formula that is easier to prove and then by one of the many induction principles, it will follow that our desired result holds.

  We can deduce the following \hyperref[def:proof_derivability]{logical theorem}:
  \begin{equation*}
    \begin{prooftree}
      \infer0[\eqref{eq:def:first_order_natural_deduction_system/equality/intro}]{ (\xi \doteq s(\zeta))[\zeta \mapsto \eta, \xi \mapsto s(\eta)] }
      \infer1[\eqref{eq:def:first_order_natural_deduction_system/exists/intro}]{ \parens[\Big]{ \qexists \zeta (\xi \doteq s(\zeta) }[\xi \mapsto s(\eta)] }
      \infer1[\eqref{eq:def:minimal_propositional_natural_deduction_system/or/intro_right}]{ \theta[\xi \mapsto s(\eta)] }
      \infer1[\eqref{eq:def:minimal_propositional_natural_deduction_system/imp/intro}]{ \theta[\xi \mapsto \eta] \rightarrow \theta[\xi \mapsto s(\eta)] }
      \infer1[\eqref{eq:def:first_order_natural_deduction_system/forall/intro}]{ \qforall \eta (\theta[\xi \mapsto \eta] \rightarrow \theta[\xi \mapsto s(\eta)]) }

      \infer0[\eqref{eq:def:first_order_natural_deduction_system/equality/intro}]{ (\xi \doteq 0)[\xi \mapsto 0] }
      \infer1[\eqref{eq:def:minimal_propositional_natural_deduction_system/or/intro_left}]{ \theta[\xi \mapsto s(\eta)] }

      \infer2[\eqref{eq:def:minimal_propositional_natural_deduction_system/and/intro}]{ \theta[\xi \mapsto 0] \wedge \qforall \eta \parens[\Big] { \theta[\xi \mapsto \eta] \rightarrow \theta[\xi \mapsto s(\eta)] } }
    \end{prooftree}
  \end{equation*}

  This is precisely the antecedent of the instance of \eqref{eq:def:peano_arithmetic/PA3} with \( \varphi = \theta \). By \fullref{thm:syntactic_deduction_theorem} we have
  \begin{equation*}
    \eqref{eq:def:peano_arithmetic/PA3} \vdash \qforall \eta \theta[\xi \mapsto \eta].
  \end{equation*}

  When interpreted in \( \BbbN \), this formula \( \qforall \eta \theta[\xi \mapsto \eta] \) simply states that every natural number is either zero or has a predecessor. The statement does not concern itself with uniqueness nor with whether \( 0 \) has a predecessor.

  But we have already shown uniqueness --- the predecessor function \( p \) is a partial single-valued function. And we have shown that \( p \) is not defined at zero. The last part of the proof shows \( p \) is defined for all nonzero values.

  We may choose to define \( p \) at zero by giving it a sentinel value. This is precisely the technique we use in \fullref{thm:function_invertibility_categorical/nonempty_left_invertible} to show that \( s \) has a left inverse if it is injective. We can use only \eqref{eq:def:peano_arithmetic/PA1} to show that \( s \) is injective and then pick \( p \) to be any of its left inverses. We also want \( p \) to be as close as possible to a right inverse, however. The latter, as we have seen, is more tricky.
\end{proof}

\begin{proposition}\label{thm:natural_number_addition_properties}
  The \hyperref[def:natural_numbers]{natural numbers} \( \BbbN \) with \hyperref[def:peano_arithmetic/plus]{addition} form a \hyperref[def:magma/cancellative]{cancellative} \hyperref[def:magma/commutative]{commutative} \hyperref[def:zerosumfree]{zerosumfree} \hyperref[def:monoid]{monoid} with \( 0 \) as the identity.

  Furthermore, the sum of two natural numbers is nonzero if and only if both numbers are nonzero, that is,
  \begin{equation}\label{eq:thm:natural_number_addition_properties/nonzero_sum}
    n + m = 0 \T{if and only if} n = 0 \T{and} m = 0.
  \end{equation}

  This generalizes to \fullref{thm:cardinal_addition_algebraic_properties} and \fullref{thm:cardinal_addition_algebraic_properties}.
\end{proposition}
\begin{proof}
  \SubProofOf[def:magma/commutative]{commutativity} Consider the sum \( n + m \). We use induction on \( m \) to prove its commutativity.
  \begin{itemize}
    \item If \( m = 0 \), nested induction by \( n \) yields:
    \begin{itemize}
      \item If \( n = m = 0 \), clearly \( n + m = 0 + 0 = m + n \).
      \item If the inductive hypothesis holds for its predecessor \( p(n) \),
      \begin{balign*}
        n + m
        &=
        n + 0
        = \\ &=
        s(p(n)) + 0
        \reloset {\eqref{eq:def:peano_arithmetic/PA4}} = \\ &=
        s(p(n))
        \reloset {\eqref{eq:def:peano_arithmetic/PA4}} = \\ &=
        s(p(n) + 0)
        = \\ &=
        s(p(n) + m)
        \reloset {\T{ind.}} = \\ &=
        s(m + p(n))
        \reloset {\eqref{eq:def:peano_arithmetic/PA5}} = \\ &=
        m + s(p(n))
        =
        m + n.
      \end{balign*}
    \end{itemize}

    \item If \( m \neq 0 \) and if the inductive hypothesis holds for \( p(m) \), \eqref{eq:def:peano_arithmetic/PA1} yields that \( n + m = m + n \) if and only if \( n + p(m) = p(m) + n \). But the last equality is satisfied because of the inductive hypothesis, hence commutativity of \( n \) and \( m \) follows.
  \end{itemize}

  \SubProofOf[def:magma/associative]{associativity} Fix natural numbers \( n \), \( m \), \( k \). We will prove associativity by induction on \( k \). If \( k = 0 \), we have
  \begin{equation*}
    (n + m) + 0
    \reloset {\eqref{eq:def:peano_arithmetic/PA4}} =
    n + m
    \reloset {\eqref{eq:def:peano_arithmetic/PA4}} =
    n + (m + 0).
  \end{equation*}

  If \( k \neq 0 \), the proof follows directly from \eqref{eq:def:peano_arithmetic/PA1} as in the proof of commutativity.

  \SubProofOf[def:monoid]{identity} We have \( n + 0 = n \) by \eqref{eq:def:peano_arithmetic/PA4} and \( 0 + n = n \) by commutativity.

  \SubProofOf[def:magma/cancellative]{cancellation} Let \( n + k = m + k \). We will prove that \( n = m \) by induction. This is obvious for \( k = 0 \). For \( k \neq 0 \) we have
  \begin{equation*}
    n + s(p(k))
    =
    n + k
    =
    m + k
    =
    m + s(p(k)),
  \end{equation*}
  which by \eqref{eq:def:peano_arithmetic/PA5} is equivalent to \( s(n + p(k))) = s(m + p(k))) \).

  By \eqref{eq:def:peano_arithmetic/PA1}, we have \( n + p(k) = m + p(k) \). The inductive hypothesis implies that \( n = m \).

  \SubProofOf[def:zerosumfree]{zerosumfree} We will use induction on \( m \) in \( n + m = 0 \).
  \begin{itemize}
    \item If \( m = 0 \), then \( n + m = n \) by \eqref{eq:def:peano_arithmetic/PA4}, and hence \( n = 0 \).

    \item If \( m > 0 \), then \( n + m = s(n + p(m)) \) by \eqref{eq:def:peano_arithmetic/PA5}, and by \eqref{eq:def:peano_arithmetic/PA2}, \( n + m \neq 0 \).
  \end{itemize}

  Therefore, \( n + m = 0 \) if and only if \( n = m = 0 \).
\end{proof}

\begin{remark}\label{rem:natural_number_multiplication}
  \hyperref[rem:additive_magma/multiplication]{Multiplication in commutative monoids} (i.e. monoid exponentiation) is defined in \fullref{def:monoid/exponentiation} for a natural number and a monoid member. It just to happens that, by \fullref{thm:natural_number_addition_properties}, the natural numbers are themselves a monoid. We cannot rely on \fullref{thm:magma_exponentiation_properties}, however, if we want to avoid circular definitions and proofs.

  Having multiplication as part of the signature of \hyperref[def:peano_arithmetic]{Peano arithmetic} allows us to avoid this circularity.
\end{remark}

\begin{proposition}\label{thm:natural_number_multiplication_properties}
  The \hyperref[def:natural_numbers]{natural numbers} \( \BbbN \) with \hyperref[def:peano_arithmetic/mult]{multiplication} form a \hyperref[def:magma/commutative]{commutative} \hyperref[def:monoid]{monoid} with \( 1 \) as the identity.

  When combined with addition, the natural numbers become an \hyperref[def:entire_semiring]{entire} \hyperref[def:semiring/commutative]{commutative semiring}.

  This generalizes to \fullref{thm:ordinal_multiplication_algebraic_properties} and \fullref{thm:cardinal_multiplication_algebraic_properties}.
\end{proposition}
\begin{proof}
  \SubProofOf[def:monoid]{identity} Multiplication by \( 1 \) on the right preserves any natural number:
  \begin{equation*}
     n \cdot 1
     \reloset{\eqref{eq:def:peano_arithmetic/PA7}} =
     n \cdot 0 + n
     \reloset{\eqref{eq:def:peano_arithmetic/PA6}} =
     0 + n
     =
     n.
  \end{equation*}

  Multiplication from the left is handled by induction. Indeed, the case \( n = 0 \) is trivial and for nonzero \( n \) we have
  \begin{equation*}
     1 \cdot n
     \reloset{\eqref{eq:def:peano_arithmetic/PA7}} =
     1 \cdot p(n) + 1
     \reloset{\eqref{eq:def:peano_arithmetic/PA6}} =
     p(n) + 1
     =
     n.
  \end{equation*}

  \SubProofOf[def:semiring/left_distributivity]{distributivity} We will prove that \( (n + m)k = n \cdot k + n \cdot k \) with induction on \( k \).

  If \( k = 0 \),
  \begin{equation*}
    (n + m) \cdot 0
    \reloset{\eqref{eq:def:peano_arithmetic/PA6}} =
    0
    \reloset{\eqref{eq:def:peano_arithmetic/PA4}} =
    0 + 0
    \reloset{\eqref{eq:def:peano_arithmetic/PA6}} =
    n \cdot 0 + m \cdot 0.
  \end{equation*}

  For all nonzero \( k \), if the inductive hypothesis holds for \( p(k) \), then
  \begin{balign*}
    (n + m) \cdot k
    &\reloset*{\eqref{eq:def:peano_arithmetic/PA7}} =
    (n + m) + (n + m) \cdot p(k)
    \reloset {\T{ind.}} = \\ &=
    (n + m) + n \cdot p(k) + n \cdot p(k)
    = \\ &=
    (n + n \cdot p(k)) + (m + m \cdot p(k))
    \reloset{\eqref{eq:def:peano_arithmetic/PA7}} = \\ &=
    n \cdot k + m \cdot k.
  \end{balign*}

  \SubProofOf[def:magma/associative]{associativity} With distributivity proven, associativity of multiplication follows by induction. Indeed,
  \begin{equation*}
    (n \cdot m) \cdot k = n \cdot (m \cdot k)
  \end{equation*}
  is trivially satisfied for \( k = 0 \) and for all nonzero \( k \), whenever the inductive hypothesis holds for all \( n, m \in \BbbN \), it follows that
  \begin{balign*}
    (n \cdot m) \cdot k
    &\reloset*{\eqref{eq:def:peano_arithmetic/PA7}} =
    n \cdot m + (n \cdot m) \cdot p(k)
    \reloset {\T{ind.}} = \\ &=
    n \cdot m + n \cdot (m \cdot p(k))
    \reloset{\eqref{eq:def:semiring/left_distributivity}} = \\ &=
    n \cdot (m + m \cdot p(k))
    \reloset{\eqref{eq:def:peano_arithmetic/PA7}} = \\ &=
    n \cdot (m \cdot k).
  \end{balign*}

  \SubProofOf[def:magma/commutative]{commutativity} By induction on \( m \) we prove
  \begin{equation*}
    n \cdot m = m \cdot n.
  \end{equation*}

  The base case is trivial for nonzero \( m \), if \( n \cdot p(m) = p(m) \cdot n \) for all \( n \in \BbbN \), then
  \begin{equation*}
    n \cdot m
    \reloset{\eqref{eq:def:peano_arithmetic/PA7}} =
    n + n \cdot p(m)
    \reloset {\T{ind.}} =
    n + p(m) \cdot n
    \reloset{\eqref{def:semiring/left_distributivity}} =
    (1 + p(m)) \cdot n
    \reloset{\eqref{eq:def:peano_arithmetic/PA5}} =
    m \cdot n.
  \end{equation*}

  \SubProof{Proof of no zero divisors} We will now show that \( \BbbN \) has no zero divisors.

  If \( m = 0 \), then \( n \cdot m = 0 \) by \eqref{eq:def:peano_arithmetic/PA6}. If \( n = 0 \), then by induction on \( m \) we can easily show that \( n \cdot m = n \cdot p(m) + n = 0 + 0 = 0 \).

  Conversely, let \( n \cdot m = 0 \). By induction on \( m \), either \( m = 0 \) or we have \( n \cdot m = n \cdot p(m) + n \), in which case \eqref{eq:thm:natural_number_addition_properties/nonzero_sum} both \( n \cdot p(m) \) and \( n \) are zero. Thus, if we assume that \( m \neq 0 \), then we can conclude that \( n = 0 \).
\end{proof}

\begin{remark}\label{rem:natural_number_successor_via_addition}
  In \cite[1]{Peano1889}, Peano defined an \enquote{\( n \mapsto n + 1 \)} operation rather than a successor operation. It has since become common practice to instead define a \enquote{successor} operation, define addition and then show that the two are compatible:
  \begin{equation*}
    n + 1
    \reloset {\eqref{eq:def:peano_arithmetic/PA5}} =
    s(n + 0)
    \reloset {\eqref{eq:def:peano_arithmetic/PA4}} =
    s(n).
  \end{equation*}

  The predecessor operation then corresponds to integer subtraction by \( 1 \). We avoid subtraction in this subsection --- we are only interested in the fact that every nonzero natural number \( n \) has a predecessor \( m \) such that \( n = m + 1 \).

  It is dangerous to conflate \( n + 1 \) and \( s(n) \) until we have proved the familiar properties of addition. We have already done, so in \fullref{thm:natural_number_addition_properties}, however, and will further avoid mentioning directly the operations \( s(n) \) and \( p(n) \).

  See \fullref{rem:ordinal_successor_via_addition} for the more general case of ordinal addition.
\end{remark}

\begin{definition}\label{def:natural_numbers_ordering}
  We can define the familiar order \( \leq \) on the natural numbers via addition as the \hyperref[rem:predicate_formula]{predicate formula}
  \begin{equation}\label{eq:def:natural_numbers_ordering/predicate}
    \alpha \leq \beta \coloneqq \qexists \xi \parens[\Big]{ \alpha + \xi \doteq \beta }.
  \end{equation}

  We use the infix notation for convenience, however we do not assume that \( \leq \) is part of the language of \logic{PA} (as explained in \fullref{rem:first_order_formula_conventions/necessary_signature}).

  The following relation
  \begin{equation}\label{eq:def:natural_numbers_ordering/strict_predicate}
    \alpha < \beta \coloneqq \qexists {\xi \neq 0} \parens[\Big]{ \alpha + \xi \doteq \beta }.
  \end{equation}
  is then connected to \( \leq \) via \eqref{eq:def:preordered_set/compatibility_strict}.

  For the specific model of \logic{PA} based on the smallest inductive set \( \omega \), the latter relation \( < \) is equivalent to \( \in \) as discussed in \fullref{rem:ordinal_definition}.

  We will show in \fullref{thm:natural_numbers_are_well_ordered} that \( \BbbN \) is \hyperref[def:totally_ordered_set]{total ordered} (even \hyperref[def:well_ordered_set]{well-ordered}) with \( \leq \) as the nonstrict order and \( < \) as the strict order.
\end{definition}
\begin{proof}
  We will show that \( n < m \) if and only if \( n \leq m \) and \( n \neq m \).

  \SufficiencySubProof Assume that \( n < m \). Then there exists some nonzero \( a \) such that \( n + a = m \). In particular, we have \( n \leq m \). If we suppose that \( n = m \), then \( n + a = m \) and since addition is cancellative, it would follow that \( a = 0 \), contradicting the assumption that \( a \) is nonzero.

  Therefore, \( n \leq m \) and \( n \neq m \).

  \NecessitySubProof Assume that \( n \leq m \) and \( n \neq m \). Then there exists some \( a \) such that \( n + a = m \) If we suppose that \( a = 0 \), then we would obtain that \( n = m \), which would contradict our choice of \( n \) and \( m \).

  Therefore, \( n < m \).
\end{proof}

\begin{proposition}\label{thm:natural_numbers_are_well_ordered}
  The natural numbers are \hyperref[def:well_ordered_set]{well-ordered} by the relation \( < \) defined by \eqref{eq:def:natural_numbers_ordering/strict_predicate}.

  Furthermore, \( \BbbN \) is an \hyperref[def:ordered_semiring]{ordered semiring}. That is, the nonstrict order \( \leq \) is compatible with addition and multiplication.
\end{proposition}
\begin{proof}
  As discussed in \fullref{def:well_ordered_set}, in order to show that \( < \) well-orders \( \BbbN \), we only need to show that \( < \) is \hyperref[def:binary_relation/transitive]{transitive}, satisfies \hyperref[def:binary_relation/trichotomic]{trichotomy} and does not allow an infinitely descending chain.

  \SubProofOf[def:binary_relation/transitive]{transitivity} Let \( n < m \) and \( m < k \). Then there exist nonzero numbers \( a \) such that \( n + a = m \) and \( b \) such that \( m + b = k \). Thus, \( n + a + b = k \), which demonstrates that \( n \leq k \). Furthermore, because \( \BbbN \) is zerosumfree, it also follows that \( a + b \neq 0 \).

  Therefore, \( n < k \).

  \SubProofOf[def:binary_relation/trichotomic]{trichotomy} Let \( n \) and \( m \) be natural numbers.

  We have already shown in \fullref{def:natural_numbers_ordering} that due to \eqref{eq:def:preordered_set/compatibility_strict} the equality \( n = m \) holds if and only if neither \( n < m \) nor \( n > m \).

  Aiming at a contradiction, suppose that both \( n < m \) and \( n > m \) hold. There must exist nonzero numbers \( a \) and \( b \) such that \( n + a = m \) and \( n = m + b \). Then
  \begin{equation*}
    n = m + b = (n + a) + b.
  \end{equation*}

  Since addition is cancellative, we have \( a + b = 0 \). Therefore, \( n = m \), which as we have shown is incompatible with neither \( n < m \) nor \( n > m \).

  Therefore, at most one of the three conditions \( n = m \), \( n < m \) or \( n > m \) holds.

  We will use induction on \( m \) to show that at least one of the conditions hold. If \( m = 0 \), then either \( n = 0 \) and \( m = n \) or \( n \neq 0 \) and \( m < n \). Now suppose that the inductive hypothesis holds for \( m \). We will show that it also holds for \( m + 1 \).
  \begin{itemize}
    \item If \( n = m \), then clearly \( n < m + 1 \).
    \item If \( n < m \), then since \( m < m + 1 \) by transitivity of \( < \), we have \( n < m + 1 \).
    \item If \( n > m \), then there exists some nonzero \( a \) such that \( n = m + a \). If \( a = 1 \), then \( n = m + 1 \). If \( a \) is neither \( 0 \) nor \( 1 \), then \( n > m + 1 \).
  \end{itemize}

  \SubProofOf[def:well_founded_relation]{well-foundedness} We will show by induction on \( n \) that an infinitely descending chain ending at \( n \) cannot exist.

  If \( n = 0 \), by \eqref{eq:def:peano_arithmetic/PA2} \( n \) has no predecessor and thus there cannot exist a natural number \( m \) such that \( m < n \).

  Now assume that the inductive hypothesis holds for \( n \) and suppose that there exists an infinitely descending chain ending in \( n + 1 \):
  \begin{equation}\label{eq:thm:natural_numbers_are_well_ordered/descending_chain}
    \cdots < k < m < n + 1.
  \end{equation}

  If \( m = n \), it follows that
  \begin{equation*}
    \cdots < k < n
  \end{equation*}
  is an infinitely descending chain ending in \( n \).

  If \( m < n \), then
  \begin{equation*}
    \cdots < k < m < n
  \end{equation*}
  is again an infinitely descending chain ending in \( n \).

  By the inductive hypothesis, a chain ending at \( n \) cannot exist, therefore neither does \eqref{eq:thm:natural_numbers_are_well_ordered/descending_chain}.

  \SubProofOf[def:ordered_magma]{compatibility with addition} We will show that the nonstrict order is compatible with addition in \( \BbbN \). Let \( n \leq m \) and let \( k \) be an arbitrary natural number. Since \( n \leq m \), there exists a number \( a \) such that \( n + a = m \). Then
  \begin{equation*}
    m + k = (n + a) + k = (n + k) + a.
  \end{equation*}

  Therefore,
  \begin{equation*}
    n + k \leq m + k.
  \end{equation*}

  \SubProofOf[def:ordered_semiring]{compatibility with multiplication} If \( n \geq 0 \) and \( m \geq 0 \), then \( n \cdot m \geq 0 \) for the simple reason that all natural numbers are greater than or equal to zero.
\end{proof}

\begin{proposition}\label{thm:natural_number_divisibility_lattice}
  The semiring \hyperref[def:natural_numbers]{\( \BbbN \)} of natural numbers is a \hyperref[def:semilattice/bounded]{bounded lattice} with respect to \hyperref[def:divisibility]{semiring divisibility}. Explicitly:
  \begin{thmenum}
    \thmitem{thm:natural_number_divisibility_lattice/join} The \hyperref[def:semilattice/join]{join} of \( n \) and \( m \) is their \hyperref[def:gcd_and_lcm]{least common multiple} \( \lcm\set{ n, m } \) (defined via \fullref{alg:euclidean_algorithm} and \fullref{thm:gcd_and_lcm}).

    \thmitem{thm:natural_number_divisibility_lattice/bottom} The \hyperref[def:extremal_points/top_and_bottom]{bottom element} is \( 1 \) since \( 1 \) divides every natural number.

    \thmitem{thm:natural_number_divisibility_lattice/meet} Dually, the \hyperref[def:semilattice/meet]{meet} of \( n \) and \( m \) is their \hyperref[def:gcd_and_lcm]{greatest common divisor} \( \gcd\set{ n, m } \) (defined via \fullref{alg:euclidean_algorithm}).

    \thmitem{thm:natural_number_divisibility_lattice/top} The \hyperref[def:extremal_points/top_and_bottom]{top element} is \( 0 \) since every natural number divides \( 0 \).
  \end{thmenum}

  Furthermore, divisibility is compatible with the standard ordering in the sense that \( n \mid m \) implies \( n \leq m \).

  \begin{figure}[!ht]
    \centering
    \includegraphics[page=1]{output/thm__natural_number_divisibility_order.pdf}
    \caption{A spatial \hyperref[def:hasse_diagram]{Hasse diagram} for a fragment of the \hyperref[thm:natural_number_divisibility_lattice]{natural number divisibility lattice}}
    \label{fig:thm:natural_number_divisibility_lattice/divisibility}
  \end{figure}

  \begin{figure}[!ht]
    \centering
    \includegraphics[page=2]{output/thm__natural_number_divisibility_order.pdf}
    \caption{A comparison of the \hyperref[thm:natural_number_divisibility_lattice]{divisibility lattice} of \( \BbbN \) and the \hyperref[thm:semiring_of_ideals/lattice]{lattice of ideals} of \( \BbbZ \).}
    \label{fig:thm:natural_number_divisibility_lattice/ideals}
  \end{figure}
\end{proposition}
\begin{proof}
  By \fullref{thm:semiring_divisibility_order}, divisibility is a preorder.

  \SubProofOf[def:binary_relation/antisymmetric]{antisymmetry} If \( n \mid m \) and \( m \mid n \), there exist numbers \( a \) and \( b \) such that \( n = ay \) and \( m = bx \). Then \( n = abx \), and we can cancel \( n \) to obtain that \( ab = 1 \). But \( 1 \) is the only unit in \( \BbbN \), hence \( a = b = 1 \), and thus \( n = m \).

  \SubProofOf[def:semilattice/lattice]{lattice structure} By \fullref{alg:euclidean_algorithm}, every pair of integers has a positive greatest common divisor, and also a least common multiple.

  By \fullref{thm:def:semiring_ideal/division}, the lattice of principal ideals in \( \BbbN \) must be dual to it. Indeed, by \fullref{thm:bezouts_identity}, we have that
  \begin{equation*}
    \braket{n} + \braket{m} = \braket{ \gcd(n, m) },
  \end{equation*}
  and by \fullref{thm:def:semiring_ideal/division}, \( \braket{n} \cap \braket{m} \) contains the common multiples of \( n \) and \( m \), hence
  \begin{equation*}
    \braket{n} \cap \braket{m} = \braket{ \lcm(n, m) }.
  \end{equation*}

  \SubProof{Proof that the order are compatible} If \( n \mid m \), then there exists a positive natural number \( a \) such that \( an = m \). We have
  \begin{equation*}
    an
    \reloset{\eqref{eq:def:peano_arithmetic/PA7}} =
    (a - 1)n + n
    =
    m.
  \end{equation*}

  Thus, by \eqref{eq:def:natural_numbers_ordering/predicate}, \( n \leq m \).
\end{proof}
