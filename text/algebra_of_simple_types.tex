\section{Algebra of simple types}\label{sec:algebra_of_simple_types}

\paragraph{Algebra of simple types}

\begin{proposition}\label{thm:simple_algebraic_type_arithmetic}
  Denote by \( T \) the set of all \hyperref[def:simple_algebraic_types]{simple algebraic types}. 

  Define the relation \( \tau \vdash \sigma \) on \( T \) to hold if, for some variable \( x \) and term \( M \), there is a \hyperref[def:type_derivation_tree]{derivation} of \( M: \sigma \) from \( x: \tau \). Let \( \tau \sim \sigma \) if \( \tau \vdash \sigma \) and \( \sigma \vdash \tau \).

  We will show that \( {\sim} \) is an \hyperref[def:equivalence_relation]{equivalence relation} and that \( {\synplus} \) and \( {\syntimes} \) are well-behaved with respect to \( {\sim} \). On the quotient \( T / {\sim} \), these operations act as \hyperref[def:lattice]{lattice} join and meet, with \( {\syn\Bbbzero} \) as the bottom element and \( {\syn\Bbbone} \) as the top element.

  In the corresponding \hyperref[ex:def:semiring/lattice]{join-meet semiring} they act as addition and multiplication, with zero element \( {\syn\Bbbzero} \) and unity element \( {\syn\Bbbone} \).

  First, we will show well-behavedness of \( {\sim} \):
  \begin{thmenum}[series=thm:simple_algebraic_type_arithmetic]
    \thmitem{thm:simple_algebraic_type_arithmetic/equivalence} \( {\sim} \) is an equivalence relation.

    \thmitem{thm:simple_algebraic_type_arithmetic/addition_well_behaved} Addition is well-behaved:
    \begin{equation}\label{eq:thm:simple_algebraic_type_arithmetic/addition_well_behaved}
      \tau \sim \tau' \T{and} \sigma \sim \sigma' \T{imply} \tau \synplus \sigma \sim \tau' \synplus \sigma'.
    \end{equation}

    \thmitem{thm:simple_algebraic_type_arithmetic/multiplication_well_behaved} Multiplication is well-behaved:
    \begin{equation}\label{eq:thm:simple_algebraic_type_arithmetic/multiplication_well_behaved}
      \tau \sim \tau' \T{and} \sigma \sim \sigma' \T{imply} \tau \syntimes \sigma \sim \tau' \syntimes \sigma'.
    \end{equation}
  \end{thmenum}

  We will prove the following for addition:
  \begin{thmenum}[resume=thm:simple_algebraic_type_arithmetic]
    \thmitem{thm:simple_algebraic_type_arithmetic/addition_associative} Addition is \hyperref[def:binary_operation/associative]{associative}:
    \begin{equation}\label{eq:thm:simple_algebraic_type_arithmetic/addition_associative}
      (\tau \synplus \sigma) \synplus \rho \sim \tau \synplus (\sigma \synplus \rho).
    \end{equation}

    \thmitem{thm:simple_algebraic_type_arithmetic/addition_commutative} Addition is \hyperref[def:binary_operation/commutative]{commutative}:
    \begin{equation}\label{eq:thm:simple_algebraic_type_arithmetic/addition_commutative}
      \tau \synplus \sigma \sim \sigma \synplus \tau.
    \end{equation}

    \thmitem{thm:simple_algebraic_type_arithmetic/addition_neutral} \( \syn\Bbbzero \) acts as a \hyperref[def:monoid]{neutral element} of addition (and thus satisfies the bounded lattice axiom \eqref{eq:def:bounded_lattice/theory/top}):
    \begin{equation}\label{eq:thm:simple_algebraic_type_arithmetic/addition_neutral}
      \tau \synplus \syn\Bbbzero \sim \tau.
    \end{equation}
  \end{thmenum}

  We will prove the following for multiplication:
  \begin{thmenum}[resume=thm:simple_algebraic_type_arithmetic]
    \thmitem{thm:simple_algebraic_type_arithmetic/multiplication_associative} Multiplication is associative:
    \begin{equation}\label{eq:thm:simple_algebraic_type_arithmetic/multiplication_associative}
      (\tau \syntimes \sigma) \syntimes \rho \sim \tau \syntimes (\sigma \syntimes \rho).
    \end{equation}

    \thmitem{thm:simple_algebraic_type_arithmetic/multiplication_commutative} Multiplication is commutative:
    \begin{equation}\label{eq:thm:simple_algebraic_type_arithmetic/multiplication_commutative}
      \tau \syntimes \sigma \sim \sigma \syntimes \tau.
    \end{equation}

    \thmitem{thm:simple_algebraic_type_arithmetic/multiplication_neutral} \( \syn\Bbbone \) acts as a neutral element of multiplication (and thus satisfies the bounded lattice axiom \eqref{eq:def:bounded_lattice/theory/bot}):
    \begin{equation}\label{eq:thm:simple_algebraic_type_arithmetic/multiplication_neutral}
      \tau \syntimes \syn\Bbbone \sim \tau.
    \end{equation}
  \end{thmenum}

  Additionally, we will also prove the following:
  \begin{thmenum}[resume=thm:simple_algebraic_type_arithmetic]
    \thmitem{thm:simple_algebraic_type_arithmetic/multiplication_distributes} Multiplication \hyperref[def:semiring]{distributes} over addition:
    \begin{equation}\label{eq:thm:simple_algebraic_type_arithmetic/multiplication_distributes}
      \tau \syntimes (\sigma \synplus \rho) \sim (\tau \syntimes \sigma) \synplus (\tau \syntimes \sigma).
    \end{equation}

    \thmitem{thm:simple_algebraic_type_arithmetic/addition_distributes} Addition distributes over multiplication:
    \begin{equation}\label{eq:thm:simple_algebraic_type_arithmetic/addition_distributes}
      \tau \synplus (\sigma \syntimes \rho) \sim (\tau \synplus \sigma) \syntimes (\tau \synplus \sigma).
    \end{equation}

    \thmitem{thm:simple_algebraic_type_arithmetic/multiplication_absorbs} The operations satisfy the absorption law \eqref{eq:thm:lattice_operation_characterization/absorption/meet}:
    \begin{equation}\label{eq:thm:simple_algebraic_type_arithmetic/multiplication_absorbs}
      \tau \syntimes (\tau \synplus \sigma) \sim \tau
    \end{equation}

    \thmitem{thm:simple_algebraic_type_arithmetic/addition_absorbs} The operations satisfy the absorption law \eqref{eq:thm:lattice_operation_characterization/absorption/join}:
    \begin{equation}\label{eq:thm:simple_algebraic_type_arithmetic/addition_absorbs}
      \tau \synplus (\tau \syntimes \sigma) \sim \tau
    \end{equation}
  \end{thmenum}
\end{proposition}
\begin{comments}
  \item See \cref{ex:con:curry_howard_correspondence/algebraic_types} for how this statement relates to \hyperref[def:lindenbaum_tarski_algebra]{Lindenbaum-Tarski algebras} via the \hyperref[con:curry_howard_correspondence]{Curry-Howard correspondence}.
\end{comments}
\begin{proof}
  \SubProofOf{thm:simple_algebraic_type_arithmetic/equivalence} Obviously \( {\sim} \) is reflexive, and it is by definition symmetric.

  We will show transitivity. Suppose that \( \synx: \tau \vdash M: \sigma \) and \( \syny: \sigma \vdash N: \rho \). We can use \fullref{alg:simply_typed_substitution} to construct a derivation of \( N[\syny \mapsto M]: \rho \) from \( \synx: \tau \), i.e. \( \synx: \tau \vdash N[\syny \mapsto M]: \rho \).

  Therefore, \( {\sim} \) is indeed an equivalence relation.

  \SubProofOf{thm:simple_algebraic_type_arithmetic/addition_well_behaved} Suppose that \( \syna: \tau \vdash M: \tau' \) and \( \synb: \sigma \vdash N: \sigma' \). From the corresponding derivations we can construct the tree
  \begin{equation*}
    \begin{prooftree}
      \hypo{ \synx: \tau \synplus \sigma }

      \hypo{ \syna: \tau }
      \ellipsis {} { M: \tau' }
      \infer1[\ref{inf:def:simple_sum_type/intro_left}]{ \synS_{+L} M: \tau' \synplus \sigma' }

      \hypo{ \synb: \sigma }
      \ellipsis {} { N: \sigma' }
      \infer1[\ref{inf:def:simple_sum_type/intro_right}]{ \synS_{+R} N: \tau' \synplus \sigma' }

      \infer[left label={\( \syna, \synb \)}]3[\ref{inf:def:simple_sum_type/elim}]{ \synS_- (\qabs {\syna^\tau} \synS_{+L} M) (\qabs {\synb^{\sigma}} \synS_{+R} N) \synx: \tau' \synplus \sigma' }
    \end{prooftree}
  \end{equation*}

  We have shown that \( \tau \synplus \sigma \vdash \tau' \synplus \sigma' \). We can obtain the converse by exchanging \( \tau \) with \( \tau' \) and \( \sigma \) with \( \sigma' \).

  Therefore, we have shown that \( {\synplus} \) is well-defined on the cosets of \( T / {\sim} \).

  \SubProofOf{thm:simple_algebraic_type_arithmetic/multiplication_well_behaved} Again, suppose that \( \syna: \tau \vdash M: \tau' \) and \( \synb: \sigma \vdash N: \sigma' \).

  Given the assertion \( \synx: \tau \syntimes \sigma \), \ref{inf:def:simple_product_type/elim_left} gives us \( \synP_{-L} \synx: \tau \), and \fullref{alg:simply_typed_substitution} constructs a derivation tree of \( M[\syna \mapsto \synP_{-L} \synx]: \tau' \). Similarly, the same assertion allows deriving \( N[\syna \mapsto \synP_{-R} \synx]: \sigma' \).

  Then
  \begin{equation*}
    \begin{prooftree}
      \hypo{ \synx: \tau \syntimes \sigma }
      \infer1[\ref{inf:def:simple_product_type/elim_left}]{ \synP_{-L} \synx: \tau }

      \infer[dashed]1{ M[\syna \mapsto \synP_{-L} \synx]: \tau' }

      \hypo{ \synx: \tau \syntimes \sigma }
      \infer1[\ref{inf:def:simple_product_type/elim_right}]{ \synP_{-R} \synx: \sigma }

      \infer[dashed]1{ N[\synb \mapsto \synP_{-R} \synx]: \sigma' }

      \infer2[\ref{inf:def:simple_product_type/intro}]{ \synP_+ M[\syna \mapsto \synP_{-L} \synx] \thinspace N[\synb \mapsto \synP_{-R} \synx]: \tau' \syntimes \sigma' }
    \end{prooftree}
  \end{equation*}

  The converse is clear, hence \( {\syntimes} \) is well-defined on the cosets of \( T / {\sim} \).

  \SubProofOf{thm:simple_algebraic_type_arithmetic/addition_commutative} We will show that \( \tau \synplus \sigma \sim \sigma \synplus \tau \). Again, it is sufficient to only prove one direction:
  \begin{equation*}
    \begin{prooftree}
      \hypo{ \synx: \tau \synplus \sigma }

      \hypo{ \syna: \tau }
      \infer1[\ref{inf:def:simple_sum_type/intro_right}]{ \synS_{+R} \syna: \sigma \synplus \tau }

      \hypo{ \synb: \sigma }
      \infer1[\ref{inf:def:simple_sum_type/intro_left}]{ \synS_{+L} \synb: \sigma \synplus \tau }

      \infer[left label={\( a, b \)}]3[\ref{inf:def:simple_sum_type/elim}]{ \synS_- (\qabs {\syna^\tau} \syna) (\qabs {\synb^{\sigma}} \synb) \synx: \sigma \synplus \tau }
    \end{prooftree}
  \end{equation*}

  \SubProofOf{thm:simple_algebraic_type_arithmetic/addition_associative} Associativity is a little more complicated but analogous to \cref{thm:simple_algebraic_type_arithmetic/addition_commutative}.

  \SubProofOf{thm:simple_algebraic_type_arithmetic/addition_neutral} We must show that \( \tau \synplus \syn\Bbbzero \sim \tau \). The two directions require distinct derivations:
  \begin{paracol}{2}
    \begin{leftcolumn}
      \ParacolAlignmentHack
      \begin{equation*}
        \begin{prooftree}
          \hypo{ \synx: \tau }
          \infer1[\ref{inf:def:simple_sum_type/intro_left}]{ \synS_{+L} \synx: \tau \synplus \syn\Bbbzero }
        \end{prooftree}
      \end{equation*}
    \end{leftcolumn}

    \begin{rightcolumn}
      \ParacolAlignmentHack
      \begin{equation*}
        \begin{prooftree}
          \hypo{ \synx: \tau \synplus \syn\Bbbzero }

          \hypo{ \syna: \tau }

          \hypo{ \synb: \syn\Bbbzero }
          \infer1[\ref{inf:def:simple_empty_type/elim}]{ \synE_- \synb: \tau }

          \infer[left label={\( \syna, \synb \)}]3[\ref{inf:def:simple_sum_type/elim}]{ \synS_- (\qabs {\syna^\tau} \syna) (\qabs {\synb^{\syn\Bbbzero}} \synE_- \synb) \synx: \tau }
        \end{prooftree}
      \end{equation*}
    \end{rightcolumn}
  \end{paracol}

  \SubProofOf{thm:simple_algebraic_type_arithmetic/multiplication_associative} We must show that multiplication is associative. In one direction, we have
  \begin{equation*}
    \begin{prooftree}
      \hypo{ \synx: (\tau \syntimes \sigma) \syntimes \rho }
      \infer1[\ref{inf:def:simple_product_type/elim_left}]{ \synP_{-L} \synx: \tau \syntimes \sigma }
      \infer1[\ref{inf:def:simple_product_type/elim_left}]{ \synP_{-L} \synP_{-L} \synx: \tau }

      \hypo{ \synx: (\tau \syntimes \sigma) \syntimes \rho }
      \infer1[\ref{inf:def:simple_product_type/elim_left}]{ \synP_{-L} \synx: \tau \syntimes \sigma }
      \infer1[\ref{inf:def:simple_product_type/elim_right}]{ \synP_{-R} \synP_{-L} \synx: \sigma }

      \hypo{ \synx: (\tau \syntimes \sigma) \syntimes \rho }
      \infer1[\ref{inf:def:simple_product_type/elim_right}]{ \synP_{-R} \synx: \rho }

      \infer2[\ref{inf:def:simple_product_type/intro}]{ \synP_+ (\synP_{-R} \synP_{-L} \synx) (\synP_{-R} \synx): \sigma \syntimes \rho }

      \infer2[\ref{inf:def:simple_product_type/intro}]{ \synP_+ (\synP_{-L} \synP_{-L} \synx) (\synP_+ (\synP_{-R} \synP_{-L} \synx) (\synP_{-R} \synx)): \tau \syntimes (\sigma \syntimes \rho) }
    \end{prooftree}
  \end{equation*}

  In the other direction, we proceed similarly.

  \SubProofOf{thm:simple_algebraic_type_arithmetic/multiplication_commutative} Commutativity is simpler to prove than for addition:
  \begin{equation*}
    \begin{prooftree}
      \hypo{ \synx: \tau \syntimes \sigma }
      \infer1[\ref{inf:def:simple_product_type/elim_right}]{ \synP_{-R} \synx: \sigma }

      \hypo{ \synx: \tau \syntimes \sigma }
      \infer1[\ref{inf:def:simple_product_type/elim_left}]{ \synP_{-L} \synx: \tau }

      \infer2[\ref{inf:def:simple_product_type/intro}]{ \synP_+ (\synP_{-R} \synx) (\synP_{-L} \synx): \sigma \syntimes \tau }
    \end{prooftree}
  \end{equation*}

  \SubProofOf{thm:simple_algebraic_type_arithmetic/multiplication_neutral} When showing that \( \tau \syntimes \syn\Bbbone \sim \syn\Bbbone \), the two directions require distinct derivations:
  \begin{paracol}{2}
    \begin{leftcolumn}
      \ParacolAlignmentHack
      \begin{equation*}
        \begin{prooftree}
          \hypo{ \synx: \tau \syntimes \syn\Bbbone }
          \infer1[\ref{inf:def:simple_product_type/elim_left}]{ \synP_{-L} \synx: \tau }
        \end{prooftree}
      \end{equation*}
    \end{leftcolumn}

    \begin{rightcolumn}
      \ParacolAlignmentHack
      \begin{equation*}
        \begin{prooftree}
          \hypo{ \synx: \tau }

          \infer0[\ref{inf:def:simple_unit_type/intro}]{ \synU_+: \syn\Bbbone }

          \infer2[\ref{inf:def:simple_product_type/intro}]{ \synP_+ \synx \synU_+: \tau \syntimes \syn\Bbbone }
        \end{prooftree}
      \end{equation*}
    \end{rightcolumn}
  \end{paracol}

  \SubProofOf{thm:simple_algebraic_type_arithmetic/multiplication_distributes} For distributivity of multiplication over addition, in one direction, we have
  \begin{equation*}
    \begin{prooftree}
      \hypo{ \synx: \tau \syntimes (\sigma \synplus \rho) }
      \infer1[\ref{inf:def:simple_product_type/elim_right}]{ \synP_{-R} \synx: \sigma \synplus \rho }

      \hypo{ \synx: \tau \syntimes (\sigma \synplus \rho) }
      \infer1[\ref{inf:def:simple_product_type/elim_left}]{ \synP_{-L} \synx: \tau }

      \hypo{ \syna: \sigma }
      \infer2[\ref{inf:def:simple_product_type/intro}]{ \synP_+ (\synP_{-L} \synx) \syna: \tau \syntimes \sigma }
      \infer1[\ref{inf:def:simple_sum_type/intro_left}]{ \synS_{+L} (\synP_+ (\synP_{-L} \synx) \syna): (\tau \syntimes \sigma) \synplus (\tau \syntimes \rho) }

      \hypo{ \cdots }

      \infer[left label={\( \syna, \synb \)}]3[\ref{inf:def:simple_sum_type/elim}]{ \cdots: (\tau \syntimes \sigma) \synplus (\tau \syntimes \rho) }
    \end{prooftree}
  \end{equation*}

  In the other,
  \begin{equation*}
    \begin{prooftree}
      \hypo{ \synx: (\tau \syntimes \sigma) \synplus (\tau \syntimes \rho) }

      \hypo{ \syna: \tau \syntimes \sigma }
      \infer1[\ref{inf:def:simple_product_type/elim_left}]{ \synP_{-L} \syna: \tau }

      \hypo{ \syna: \tau \syntimes \sigma }
      \infer1[\ref{inf:def:simple_product_type/elim_left}]{ \synP_{-R} \syna: \sigma }
      \infer1[\ref{inf:def:simple_sum_type/intro_left}]{ \synS_{+L} (\synP_{-R} \syna): \sigma \synplus \rho }

      \infer2[\ref{inf:def:simple_product_type/intro}]{ \synP_+ (\synP_{-L} \syna) (\synS_{+L} (\synP_{-R} \syna)): \tau \syntimes (\sigma \synplus \rho) }

      \hypo{ \cdots }

      \infer[left label={\( \syna, \synb \)}]3[\ref{inf:def:simple_sum_type/elim}]{ \cdots: \tau \syntimes (\sigma \synplus \rho) }
    \end{prooftree}
  \end{equation*}

  \SubProofOf{thm:simple_algebraic_type_arithmetic/addition_distributes} For distributivity of addition over multiplication, in one direction, we have
  \begin{equation*}
    \begin{prooftree}
      \hypo{ \synx: \tau \synplus (\sigma \syntimes \rho) }

      \hypo{ \syna: \tau }
      \infer1[\ref{inf:def:simple_sum_type/intro_left}]{ \synS_{+L} \syna: \tau \synplus \sigma }

      \hypo{ \syna: \tau }
      \infer1[\ref{inf:def:simple_sum_type/intro_left}]{ \synS_{+L} \syna: \tau \synplus \rho }

      \infer2[\ref{inf:def:simple_product_type/intro}]{ \synP_+ (\synS_{+L} \syna) (\synS_{+L} \syna): (\tau \synplus \sigma) \syntimes (\tau \syntimes \rho) }

      \hypo{}
      \ellipsis { \( A \) } { \cdots: \tau \synplus (\rho \syntimes \sigma) }

      \infer[left label={\( \syna, \synb \)}]3[\ref{inf:def:simple_sum_type/elim}]{ \cdots: (\tau \synplus \sigma) \syntimes (\tau \syntimes \rho) }
    \end{prooftree}
  \end{equation*}
  where \( A \) is the following derivation tree with open assumption \( \synb: \sigma \syntimes \rho \):
  \begin{equation*}
    \begin{prooftree}
      \hypo{ \synb: \sigma \syntimes \rho }
      \infer1[\ref{inf:def:simple_product_type/elim_left}]{ \synP_{-L} \synb: \sigma }
      \infer1[\ref{inf:def:simple_sum_type/intro_left}]{ \synS_{+R} \synP_{-L} \synb: \tau \synplus \sigma }

      \hypo{ \synb: \sigma \syntimes \rho }
      \infer1[\ref{inf:def:simple_product_type/elim_left}]{ \synP_{-R} \synb: \rho }
      \infer1[\ref{inf:def:simple_sum_type/intro_left}]{ \synS_{+R} \synP_{-R} \synb: \tau \synplus \rho }

      \infer2[\ref{inf:def:simple_product_type/intro}]{ \cdots: (\tau \synplus \sigma) \syntimes (\tau \syntimes \rho) }
    \end{prooftree}
  \end{equation*}

  In the other,
  \begin{equation*}
    \begin{prooftree}
      \hypo{ \synx: (\tau \synplus \sigma) \syntimes (\tau \synplus \rho) }
      \infer1[\ref{inf:def:simple_product_type/elim_left}]{ \synP_{-R} \synx: \tau \synplus \sigma }

      \hypo{ \syna: \tau }
      \infer1[\ref{inf:def:simple_sum_type/intro_left}]{ \synP_{+L} \syna: \tau \synplus (\sigma \syntimes \rho) }

      \hypo{}
      \ellipsis { \( B \) } { \cdots: \tau \synplus (\rho \syntimes \sigma) }

      \infer[left label={\( \syna, \synb \)}]3[\ref{inf:def:simple_sum_type/elim}]{ \cdots: \tau \synplus (\sigma \syntimes \rho) }
    \end{prooftree}
  \end{equation*}
  where \( B \) is the following derivation tree with open assumption \( \synb: \sigma \)
  \begin{equation*}
    \begin{prooftree}
      \hypo{ \synx: (\tau \synplus \sigma) \syntimes (\tau \synplus \rho) }
      \infer1[\ref{inf:def:simple_product_type/elim_right}]{ \synP_{-R} \synx: \tau \synplus \rho }

      \hypo{ \sync: \tau }
      \infer1[\ref{inf:def:simple_sum_type/intro_left}]{ \synS_{+L} \sync: \tau \synplus (\sigma \syntimes \rho) }

      \hypo{ \synb: \sigma }
      \hypo{ \synd: \rho }
      \infer2[\ref{inf:def:simple_product_type/intro}]{ \synP_+ \synb \synd: \sigma \syntimes \rho }
      \infer1[\ref{inf:def:simple_sum_type/intro_right}]{ \synS_{+R} (\synP_+ \synb \synd): \tau \synplus (\sigma \syntimes \rho) }

      \infer[left label={\( \sync, \synd \)}]3[\ref{inf:def:simple_sum_type/elim}]{ \synS_- (\qabs {\sync^\tau} \synS_{+L} \sync) (\qabs {\synd^\rho} \synS_{+R} \synP_+ \synb\synd): \tau \synplus (\rho \syntimes \sigma) }
    \end{prooftree}
  \end{equation*}

  \SubProofOf{thm:simple_algebraic_type_arithmetic/multiplication_absorbs} For absorption of multiplication, both directions are rather simple:
  \begin{paracol}{2}
    \begin{leftcolumn}
      \ParacolAlignmentHack
      \begin{equation*}
        \begin{prooftree}
          \hypo{ \synx: \tau \syntimes (\tau \synplus \sigma) }
          \infer1[\ref{inf:def:simple_product_type/elim_left}]{ \synP_{-L} \synx: \tau }
        \end{prooftree}
      \end{equation*}
    \end{leftcolumn}

    \begin{rightcolumn}
      \ParacolAlignmentHack
      \begin{equation*}
        \begin{prooftree}
          \hypo{ \synx: \tau }

          \hypo{ \synx: \tau }
          \infer1[\ref{inf:def:simple_sum_type/intro_left}]{ \synS_{+L} \synx: \tau \synplus \sigma }

          \infer2[\ref{inf:def:simple_product_type/intro}]{ \synP_+ \synx (\synS_{+L} \synx): \tau \syntimes (\tau \synplus \sigma) }
        \end{prooftree}
      \end{equation*}
    \end{rightcolumn}
  \end{paracol}

  \SubProofOf{thm:simple_algebraic_type_arithmetic/addition_absorbs} For absorption of addition, both directions are also simple:
  \columnratio{0.3,0.7}
  \begin{paracol}{2}
    \begin{leftcolumn}
      \ParacolAlignmentHack
      \begin{equation*}
        \begin{prooftree}
          \hypo{ \synx: \tau }
          \infer1[\ref{inf:def:simple_sum_type/intro_left}]{ \synS_{+L} \synx: \tau \synplus (\tau \syntimes \sigma) }
        \end{prooftree}
      \end{equation*}
    \end{leftcolumn}

    \begin{rightcolumn}
      \ParacolAlignmentHack
      \begin{equation*}
        \begin{prooftree}
          \hypo{ \synx: \tau \synplus (\tau \syntimes \sigma) }

          \hypo{ \syna: \tau }

          \hypo{ \synb: \tau \syntimes \sigma }
          \infer1[\ref{inf:def:simple_product_type/elim_left}]{ \synP_{-L} \synb: \tau }

          \infer[left label={\( \syna, \synb \)}]3[\ref{inf:def:simple_sum_type/elim}]{ \synS_- (\qabs {\syna^\tau} \syna) (\qabs {\synb^{\tau \syntimes \sigma}} \synP_{-L} \synb): \tau }
        \end{prooftree}
      \end{equation*}
    \end{rightcolumn}
  \end{paracol}
  \columnratio{}
\end{proof}
