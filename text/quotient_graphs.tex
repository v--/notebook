\section{Quotient graphs}\label{sec:quotient_graphs}

\paragraph{Quotient graphs}

\begin{definition}\label{def:quotient_graph}\mimprovised
  For any \hyperref[def:undirected_graph]{simple undirected graph} \( G = (V, E) \) and any \hyperref[def:equivalence_relation]{equivalence relation} \( {\cong} \) on \( V \), we define the \term{quotient graph}
  \begin{equation*}
    G / {\cong} \coloneqq \parens[\Big]{ V / {\cong}, \set{ \pi^E(e) \given e \in E \T{and} \pi^E(e) \T{is not a loop} } },
  \end{equation*}
  where \( \pi^V: V \to V / {\cong} \) is the canonical projection from \fullref{def:equivalence_relation} and
  \begin{equation*}
    \pi^E(\set{ u, v }) \coloneqq \set{ \pi^V(u), \pi^V(v) }.
  \end{equation*}

  \begin{thmenum}
    \thmitem{def:quotient_graph/vertex} Given a set \( U \) of vertices, we define the \term[ru=отождествление вершин (\cite[22]{ЕмеличевИПр1990ТеорияГрафов})]{vertex contraction} \( G / U \) as the quotient of \( G \) by the \hyperref[thm:equivalence_closure]{equivalence closure} of the relation
    \begin{equation*}
      u \sim v \T{if} u, v \in U.
    \end{equation*}

    \thmitem{def:quotient_graph/edge} Given a set \( F \) of edges, we define the \term[ru=стягивание ребра (\cite[21]{ЕмеличевИПр1990ТеорияГрафов}), en=edge contraction (\cite[697]{Rosen2019DiscreteMathematics})]{edge contraction} \( G / F \) as the vertex contraction of \( G \) by the set of all endpoints (of edges) in \( F \).
  \end{thmenum}
\end{definition}
\begin{comments}
  \item If we remove the restriction for \( \pi^E(e) \) not to be a loop, quotient graphs become the \hyperref[def:first_order_quotient]{first-order quotients} in the \hyperref[rem:theory_of_simple_undirected_graphs]{theory of simple undirected graphs}.

  \item \incite[19]{Diestel2017GraphTheory} defines quotients via vertex set partitions, and even considers infinite graphs, but requires the vertices in each equivalence class to be reachable from each other. Furthermore, he doesn't introduce a term for the quotient.

  \item \incite[24]{Bollobás1998ModernGraphTheory} and \incite[21]{ЕмеличевИПр1990ТеорияГрафов} define edge contractions one-edge-at-a-time. \incite[113]{Harary1969GraphTheory} uses the term \enquote{elementary contraction} for this one-edge operation. Our definition generalizes this to arbitrary sets of edges.

  \item Another approach is used by \incite[def. 1.4.6]{Knauer2019AlgebraicGraphTheory}, who defines \enquote{contractions} as compositions of homomorphisms, each of which contracts the vertices of an edge into one. To avoid unnecessary loops, he uses \enquote{weak homomorphisms} --- functions \( f: (V_G, E_G) \to (V_H, E_H) \), for which \( \set{ u, v } \in E_G \) implies \( \set{ f(u), f(v) } \in E_H \), but only if \( f(u) \neq f(v) \).
\end{comments}

\begin{definition}\label{def:graph_minor}\mcite[19]{Diestel2017GraphTheory}
  A \term{minor} of a simple undirected graph \( G \) is a \hyperref[def:undirected_graph/subgraph]{subgraph} of a finite \hyperref[def:quotient_graph/edge]{edge contraction} of \( G \).
\end{definition}
