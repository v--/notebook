\section{Polynomial factorization}\label{sec:polynomial_factorization}

\paragraph{Homogeneous polynomials}

\begin{definition}\label{def:homogeneous_polynomial}\mcite[139]{Jacobson1985AlgebraPart1}
  We say that a \hyperref[def:polynomial_algebra]{polynomial} is \term[bg=хомогенен (полином) (\cite[58]{ГеновМиховскиМоллов1991Алгебра}), ru=однородный (многочлен) (\cite[314]{Курош1968ВысшаяАлгебра})]{homogeneous} of degree \( d \) all of its monomials have \hyperref[def:polynomial_degree]{total degree} \( d \).
\end{definition}
\begin{comments}
  \item Some authors use \enquote{form} as a secondary term for homogeneous polynomials, for example \incite[13]{Eisenbud1995CommAlgebra}, \incite[245]{Roman2005LinearAlgebra}, \incite[314]{Курош1968ВысшаяАлгебра} and \incite[2]{Обрешков1962ВисшаАлгебра}. We only use the term for \hyperref[def:bilinear_form]{bilinear} and \hyperref[def:quadratic_form]{quadratic forms}.
\end{comments}

\begin{proposition}\label{thm:homogeneous_polynomial_iff_homogeneous_function}
  Fix a \hyperref[def:semiring]{(semi)ring} \( R \), an arbitrary \hyperref[def:algebra_over_semiring]{\( R \)-algebra} \( M \) and a \hyperref[def:polynomial_algebra]{polynomial algebra} \( R[\mscrX] \).

  If a \( p \in R[\mscrX] \) is \hyperref[def:homogeneous_polynomial]{homogeneous} of degree \( d \), then the \hyperref[con:evaluation_homomorphism]{evaluation} \( \Phi_M(p) \) is a \hyperref[def:real_homogeneous_function]{homogeneous function} of degree \( d \). The converse holds if \( R \) is an \hyperref[def:integral_domain]{integral domain}.
\end{proposition}
\begin{proof}
  \SufficiencySubProof Let \( p \in R[\mscrX] \) be a homogeneous polynomial of degree \( d \) over \( \mscrX \). Fix a variable assignment \( e \) and a scalar \( r \).

  Let
  \begin{equation*}
    p(\mscrX) = \sum_{X_1 \ldots X_m} a_{X_1 \ldots X_m} X_1 \ldots X_m.
  \end{equation*}

  Then, by definition of \( \Phi_M \),
  \begin{equation*}
    \Phi_M(p)(X \mapsto r \cdot e(X))
    =
    \sum_{X_1 \ldots X_d} a_{X_1 \ldots X_d} \parens[\Big]{ r \cdot e(X_1) } \ldots \parens[\Big]{ r \cdot e(X_d) }
    =
    r^d \cdot \Phi_M(p)(e).
  \end{equation*}

  Therefore, \( \Phi_M \) is homogeneous of degree \( d \).

  \NecessitySubProof Suppose that \( R \) is an integral domain. Let \( p \in R[\mscrX] \) and suppose that the function \( \Phi_M(p) \) is homogeneous of degree \( d \).

  Then, for a nonzero scalar \( r \),
  \begin{equation*}
    \Phi_M(p)(X \mapsto r \cdot e(X))
    =
    r^d \cdot \sum_{X_1 \ldots X_m} r^{m - d} \cdot a_{X_1 \ldots X_m} \cdot X_1 \cdots X_m.
  \end{equation*}

  Since \( R \) is entire, \( r^{m - d} \cdot a_{X_1 \ldots X_m} \) is nonzero if the coefficient is nonzero. Then \( \Phi_M(p)(X \mapsto r \cdot e(X)) \) does not equal \( r^d \cdot \Phi_M(p)(e) \) unless \( m = d \) for every monomial \( X_1 \cdots X_m \) with nonzero coefficients. Since we have assumed that equality holds, \( p \) must be a homogeneous polynomial of degree \( d \).
\end{proof}

\begin{proposition}\label{thm:degree_of_multivariate_polynomial_product}
  The \hyperref[def:polynomial_degree]{total degree} of the product of polynomials over an \hyperref[def:integral_domain]{integral domain} is the sum of their total degrees.
\end{proposition}
\begin{proof}
  Follows from \fullref{thm:polynomial_degree_arithmetic/product} by induction on the number of variables.
\end{proof}

\begin{proposition}\label{thm:divisors_of_homogeneous_polynomial}
  The divisors of a \hyperref[def:homogeneous_polynomial]{homogeneous polynomial} of are homogeneous.
\end{proposition}
\begin{proof}
  Let \( p(X_1, \ldots, X_n) \) be a homogeneous polynomial and let
  \begin{equation*}
    P(X_1, \ldots, X_n) = f(X_1, \ldots, X_n) \cdot g(X_1, \ldots, X_n).
  \end{equation*}

  Let \( d^- \) and \( d^+ \) be the minimal and maximal total degree of monomials in \( f \), and let \( e^- \) and \( e^+ \) be the corresponding degrees in \( g \).

  \Fullref{thm:degree_of_multivariate_polynomial_product} applied to individual terms implies that \( p \) has a monomial of degree \( d^- + e^- \) and a monomial of degree \( d^+ + e^+ \). By homogeneity,
  \begin{equation*}
    d = d^- + e^- = d^+ + e^+,
  \end{equation*}
  thus
  \begin{equation*}
    d^+ - d^- = - (e^+ - e^-).
  \end{equation*}

  Since both sides are nonnegative, we conclude that they are both \( 0 \). Then \( f \) is homogeneous of degree \( d^+ \) and \( g \) is homogeneous of degree \( e^+ \).

  This concludes the proof.
\end{proof}

\begin{proposition}\label{thm:homogeneous_polynomial_constant}
  In an \hyperref[def:integral_domain]{integral domain} \( D \), the \hyperref[def:homogeneous_polynomial]{homogeneous} polynomial \( p(X_1, \ldots, X_n, Y) \) is \hyperref[def:domain_divisibility/irreducible]{irreducible} in \( D[X_1, \ldots, X_n, Y] \) if and only if \( p(X_1, \ldots, X_n, 1) \) is irreducible in \( D[X_1, \ldots, X_n] \).
\end{proposition}
\begin{proof}
  Suppose that \( p(X_1, \ldots, X_n, Y) \) is a homogeneous polynomial of degree \( d \) over the domain \( D \). Define
  \begin{equation*}
    q(X_1, \ldots, X_n) \coloneqq p(X_1, \ldots, X_n, 1).
  \end{equation*}

  The latter is a polynomial in \( D[X_1, \ldots, X_n] \) and a scalar in \( D[X_1, \ldots, X_n][Y] \). \Fullref{thm:def:domain_divisibility/irreducible_in_polynomial_ring} thus implies that \( q \) is irreducible in \( D[X_1, \ldots, X_n, Y] \) if and only if it is irreducible in \( D[X_1, \ldots, X_n] \), hence it is sufficient to consider irreducibility in \( D[X_1, \ldots, X_n, Y] \).

  \SufficiencySubProof Suppose that \( p \) is irreducible. We will show that \( q \) also is.

  Suppose that
  \begin{equation*}
    q(X_1, \ldots, X_n) = f(X_1, \ldots, X_n) \cdot g(X_1, \ldots, X_n).
  \end{equation*}

  Consider the field of algebraic functions
  \begin{equation*}
    D(X_1, \ldots, X_n, Y).
  \end{equation*}

  \Fullref{thm:homogeneous_polynomial_iff_homogeneous_function} implies that
  \begin{equation*}
    p(X_1, \ldots, X_n, Y) = Y^d \cdot q\parens[\Big]{ \frac {X_1} Y, \cdots, \frac {X_n} Y }.
  \end{equation*}

  Then
  \begin{equation*}
    p(X_1, \ldots, X_n, Y) = Y^d \cdot f\parens[\Big]{ \frac {X_1} Y, \cdots, \frac {X_n} Y } \cdot g\parens[\Big]{ \frac {X_1} Y, \cdots, \frac {X_n} Y }.
  \end{equation*}

  Then
  \begin{equation*}
    \widehat{f}(X_1, \ldots, X_n, Y) \coloneqq Y^{\deg f} \cdot f\parens[\Big]{ \frac {X_1} Y, \cdots, \frac {X_n} Y }
  \end{equation*}
  is a polynomial in \( D[X_1, \ldots, X_n, Y] \) (and not a more general rational function). Thus,
  \begin{equation*}
    \widehat{f}(X_1, \ldots, X_n, 1) = f(X_1, \ldots, X_n).
  \end{equation*}

  Define \( \widehat{g} \) similarly.

  Therefore,
  \begin{equation*}
    p(X_1, \ldots, X_n, Y) = \widehat f(X_1, \ldots, X_n, Y) \cdot \widehat g(X_1, \ldots, X_n, Y).
  \end{equation*}

  Since \( p \) is irreducible, we conclude that \( \widehat{f} \) or \( \widehat{g} \) (or possibly both) are invertible. Without loss of generality, suppose that \( \widehat{f} \) is invertible. Then \fullref{thm:def:polynomial_algebra/invertible} implies that it is an invertible constant polynomial, and thus \( f \) is also an invertible constant.

  Generalizing on \( f \) and \( g \), we conclude that \( q(X_1, \ldots, X_n) = p(X_1, \ldots, X_n, 1) \) is irreducible.

  \NecessitySubProof Suppose that \( q \) is irreducible. We will show that \( p \) also is.

  If
  \begin{equation*}
    p(X_1, \ldots, X_n, Y) = f(X_1, \ldots, X_n, Y) \cdot g(X_1, \ldots, X_n, Y),
  \end{equation*}
  then
  \begin{equation*}
    p(X_1, \ldots, X_n, 1) = f(X_1, \ldots, X_n, 1) \cdot g(X_1, \ldots, X_n, 1).
  \end{equation*}

  Since \( q \) is irreducible, \( f(X_1, \ldots, X_n, 1) \) or \( g(X_1, \ldots, X_n, 1) \) (or both) is invertible.

  Without loss of generality, suppose that \( f(X_1, \ldots, X_n, 1) \) is invertible, thus, in particular, a constant polynomial. \Fullref{thm:divisors_of_homogeneous_polynomial} implies that \( f \) is homogeneous, thus
  \begin{equation*}
    f(X_1, \ldots, X_n, Y) = Y^{\deg f} \underbrace{f\parens[\Big]{ \frac {X_1} Y, \cdots, \frac {X_n} Y, 1 }}_{\T{invertible constant}}
  \end{equation*}
  is also an invertible constant polynomial.

  Generalizing on \( f \) and \( g \), we conclude that \( p \) is irreducible.
\end{proof}

\paragraph{Symmetric polynomials}

\begin{definition}\label{def:symmetric_polynomial}\mcite[139]{Тыртышников2017Алгебра}
  We say that a polynomial \( p(X_1, \ldots, X_n) \) over the \hyperref[def:semiring]{(semi)ring} \( R \) is \term[bg=симетричен (полином) (\cite[58]{ГеновМиховскиМоллов1991Алгебра}), ru=симметрический (многочлен), en=symmetric (polynomial) (\cite[190]{Lang2002Algebra})]{symmetric} if, for any \hyperref[def:algebra_over_semiring]{\( R \)-algebra} \( M \), the \hyperref[con:evaluation_homomorphism]{evaluation} \( \Phi_M(p) \) is a \hyperref[def:symmetric_function]{symmetric function}, i.e. \( \Phi_M(p) \) is invariant under permutations of \( X_1, \ldots, X_n \).
\end{definition}

\begin{definition}\label{def:elementary_symmetric_polynomial}\mcite[139]{Тыртышников2017Алгебра}
  We define the \term[bg=прости симетрични функции (\cite[183]{Обрешков1962ВисшаАлгебра}), ru=элементарный симметрический многочлен, en=elementary symmetric polynomial (\cite[190]{Lang2002Algebra})]{elementary symmetric polynomial} \( \sigma_k \) in \( X_1, \ldots, X_n \) as
  \begin{equation}\label{eq:def:elementary_symmetric_polynomial}
    \sigma_k(X_1, \ldots, X_n) \coloneqq \sum_{i_1 < \cdots < i_k} X_{i_1} \cdots X_{i_k}.
  \end{equation}
\end{definition}

\begin{lemma}\label{thm:symmetric_polynomial_recurrence}
  For the \hyperref[def:elementary_symmetric_polynomial]{elementary symmetric polynomials}, we have the following recurrence:
  \begin{equation}\label{eq:thm:symmetric_polynomial_recurrence}
    \sigma_k(X_1, \ldots, X_n) = \sigma_{k+1}(X_1, \ldots, X_{n-1}) + X_n \sigma_k(X_1, \ldots, X_{n-1})
  \end{equation}
\end{lemma}
\begin{proof}
  We have
  \begin{balign*}
    &\phantom{{}={}}
    \sigma_{k+1}(X_1, \ldots, X_{n-1}) + X_n \sigma_k(X_1, \ldots, X_{n-1})
    = \\ &=
    \sum_{\substack{i_1 < \cdots < i_{k+1} \\ i_{k+1} \leq n - 1}} X_{i_1} \cdots X_{i_{k+1}} + X_n \cdot \sum_{\substack{i_1 < \cdots < i_{k+1} \\ i_{k+1} \leq n - 1}} X_{i_1} \cdots X_{i_k}
    = \\ &=
    \sum_{i_1 < \cdots < i_{k+1}} X_{i_1} \cdots X_{i_{k+1}}
    = \\ &=
    \sigma_k(X_1, \ldots, X_n)
  \end{balign*}
\end{proof}

\begin{theorem}[Vieta's formulas]\label{thm:vietas_formulas}\mcite[thm. 3.11.1]{Тыртышников2017Алгебра}
  Suppose that, over some \hyperref[def:integral_domain]{integral domain}, the polynomial
  \begin{equation*}
    p(X) = \sum_{k=0}^n a_k X_k
  \end{equation*}
  \hyperref[def:polynomial_splits_into_linear_factors]{splits into linear factors}, and let \( \alpha_1, \ldots, \alpha_n \) be an enumeration of its roots.

  Then the non-leading coefficients \( a_0, \ldots, a_{n-1} \) can be expressed via the \hyperref[def:elementary_symmetric_polynomial]{elementary symmetric polynomials} applied to the roots as follows:
  \begin{equation}\label{eq:thm:vietas_formulas}
    a_k = a_n \cdot (-1)^{n-k} \cdot \sigma_{n-k}(\alpha_1, \ldots, \alpha_n).
  \end{equation}
\end{theorem}
\begin{proof}
  \Fullref{thm:polynomial_into_linear_factors} implies that
  \begin{equation*}
    p(X) = a_n \cdot \prod_{k=1}^n (X - \alpha_k).
  \end{equation*}

  We will use induction on \( n \). The case \( n = 0 \) is vacuous. Suppose that the theorem holds for polynomials of degree \( n - 1 \). Then we can apply the inductive hypothesis to
  \begin{equation*}
    q(X) \coloneqq a_n \cdot \prod_{k=1}^{n-1} (X - \alpha_k).
  \end{equation*}

  Let \( b_0, \ldots, b_{n-1} \) be the coefficients of \( q(X) \). The leading coefficient \( b_{n-1} \) of \( q(X) \) is \( a_n \). By the inductive hypothesis, for \( k = 0, \ldots, n - 2 \),
  \begin{equation*}
    b_k = a_n \cdot (-1)^{n-1-k} \cdot \sigma_{n-1-k}(\alpha_1, \ldots, \alpha_{n-1}).
  \end{equation*}

  Since \( p(X) = q(X) \cdot (X - \alpha_n) \), by the definition of convolution product, for \( k = 0, \ldots, n - 1 \),
  \begin{equation*}
    a_k = b_{k+1} - \alpha_n \cdot b_k.
  \end{equation*}

  Thus,
  \begin{balign*}
    a_k
    &=
    a_n \cdot \parens[\Big]{ (-1)^{n-k} \sigma_{n-k}(\alpha_1, \ldots, \alpha_{n-1}) - \alpha_n \cdot (-1)^{n-1-k} \cdot \sigma_{n-1-k}(\alpha_1, \ldots, \alpha_{n-1}) }
    = \\ &=
    a_n \cdot (-1)^{n-k} \cdot \parens[\Big]{ \sigma_{n-k}(\alpha_1, \ldots, \alpha_{n-1}) + \alpha_n \sigma_{n-1-k}(\alpha_1, \ldots, \alpha_{n-1}) }
    \reloset {\eqref{eq:thm:symmetric_polynomial_recurrence}} = \\ &=
    a_n \cdot (-1)^{n-k} \cdot \sigma_{n-k}(\alpha_1, \ldots, \alpha_n).
  \end{balign*}
\end{proof}

\paragraph{Resultants and discriminants}

\begin{definition}\label{def:sylvester_matrix}\mcite[136]{Тыртышников2017Алгебра}
  We define the \term[ru=матрица Сильвестра, en=Sylvester matrix (\cite[def. 3.6.2]{CoxLittleOShea2015AlgGeometry})]{Sylvester matrix} of the non-constant polynomials
  \begin{align*}
    p(X) = \sum_{k=0}^n a_k X^k
    &&\T{and}&&
    q(X) = \sum_{k=0}^m b_k X^k
  \end{align*}
  as the  of the \( (n + m) \times (n + m) \) matrix
  \begin{equation*}
    S(f, g) \coloneqq
    \begin{pmatrix}
      a_n    & a_{n-1} & \cdots  & \cdots  & a_0     &        &        &        \\
             & a_n     & a_{n-1} & \cdots  & \cdots  & a_0    &        &        \\
             &         & \cdots  & \cdots  &         &        & \cdots &        \\
             &         &         & a_n     & a_{n-1} & \cdots & \cdots & a_0    \\
      b_m    & b_{m-1} & \cdots  & \cdots  & b_0     &        &        &        \\
             & b_m     & b_{m-1} & \cdots  & \cdots  & b_0    &        &        \\
             &         & \cdots  & \cdots  &         &        & \cdots &        \\
             &         &         & b_n     & b_{m-1} & \cdots & \cdots & b_0
    \end{pmatrix}
  \end{equation*}
\end{definition}

\begin{definition}\label{def:resultant}\mcite[136]{Тыртышников2017Алгебра}
  We define the \term[bg=резултанта (\cite[198]{Обрешков1962ВисшаАлгебра}), ru=результанта, en=resultant (\cite[def. 3.6.2]{CoxLittleOShea2015AlgGeometry})]{resultant} of the non-constant univariate polynomials \( p(X) \) and \( q(X) \) as the \hyperref[def:matrix_determinant]{determinant}
  \begin{equation*}
    R(p, q) \coloneqq \det S(p, q)
  \end{equation*}
  of the corresponding \hyperref[def:sylvester_matrix]{Sylvester matrix}.
\end{definition}
\begin{comments}
  \item A useful characterization of resultants is given in \fullref{thm:resultant_as_product}.
\end{comments}

\begin{lemma}\label{thm:2x2_block_antidiagonal_determinant}
  We have the following \hyperref[def:block_matrix]{block matrix} \hyperref[def:matrix_determinant]{determinant} identity:
  \begin{equation}\label{eq:thm:2x2_block_determinant}
    \det
    \begin{pmatrix}
      0_{m \times n} & B              \\
      C              & 0_{n \times m}
    \end{pmatrix}
    =
    (-1)^{mn} \cdot \det(B) \cdot \det(C),
  \end{equation}
  where \( n > 0 \) and \( m > 0 \).
\end{lemma}
\begin{proof}
  We will use \fullref{thm:laplace_expansion}. Denote by \( c_{i,j} \) the \( (i, j) \)-th entry of \( C \). Then we can expand
  \begin{equation*}
    \det
    \begin{pmatrix}
      0_{m \times n} & B              \\
      C              & 0_{n \times m}
    \end{pmatrix}
    =
    \det
    \parens*
      {
        \begin{array}{c | c}
          0_{m \times n} & B              \\
          \hline
          \begin{matrix}
             c_{1,1} & \cdots & c_{n,1} \\
             \vdots  & \ddots & \vdots  \\
             c_{1,n} & \cdots & c_{n,n}
          \end{matrix} & 0_{n \times m}
        \end{array}
      }
  \end{equation*}
  along the first column to obtain
  \begin{equation*}
    \sum_{k=1}^n (-1)^{(m + k) + 1} \cdot c_{k,1} \cdot \det
    \begin{pmatrix}
      0_{m \times (n-1)} & B                  \\
      C_{k,1}            & 0_{(n-1) \times m}
    \end{pmatrix}
  \end{equation*}

  We will use induction on \( n \) to show \eqref{eq:thm:2x2_block_determinant}. In the base case \( n = 1 \) we have
  \begin{equation*}
    \det
    \begin{pmatrix}
      0_{m \times 1} & B              \\
      C              & 0_{1 \times m}
    \end{pmatrix}
    =
    (-1)^{(m + 1) + 1} \cdot \underbrace{c_{1,1}}_{\det C} \cdot \det B.
  \end{equation*}

  For the inductive step, suppose that the proposition holds for \( n - 1 \). Then
  \begin{balign*}
    \det \begin{pmatrix}
      0_{m \times n} & B              \\
      C              & 0_{n \times m}
    \end{pmatrix}
    &=
    \sum_{k=1}^n (-1)^{(m + k) + 1} \cdot c_{k,1} \cdot \det
    \begin{pmatrix}
      0_{m \times (n-1)} & B                  \\
      C_{k,1}            & 0_{(n-1) \times m}
    \end{pmatrix}
    = \\ &=
    \sum_{k=1}^n (-1)^{(m + k) + 1} \cdot c_{k,1} \cdot \det B \cdot (-1)^{m(n-1)} \cdot \det C_{k,1}
    = \\ &=
    (-1)^{mn} \cdot \det B \cdot \sum_{k=1}^n (-1)^{k+1} \cdot c_{k,1} \cdot \det C_{k,1}.
    = \\ &=
    (-1)^{mn} \cdot \det B \cdot \det C,
  \end{balign*}
  as desired.
\end{proof}

\begin{proposition}\label{thm:resultant_as_product}\mcite[prop. 3.10.2]{Тыртышников2017Алгебра}
  Fix two non-constant polynomials
  \begin{align*}
    p(X) = \sum_{k=0}^n a_k X^k
    &&\T{and}&&
    q(X) = \sum_{k=0}^m b_k X^k
  \end{align*}
  over an \hyperref[def:integral_domain]{integral domain} and suppose both \hyperref[def:polynomial_splits_into_linear_factors]{splits into linear factors}.

  Let \( \alpha_1, \ldots, \alpha_n \) and \( \beta_1, \ldots, \beta_m \) be enumerations of the roots of \( p(X) \) and \( q(X) \). Then for their \hyperref[def:resultant]{resultant} we have
  \begin{equation}\label{eq:thm:resultant_as_product}
    R(p, q) = a_n^m \cdot b_m^n \cdot \prod_{i=1}^n \prod_{j=1}^m (\alpha_i - \beta_j).
  \end{equation}
\end{proposition}
\begin{proof}
  Let
  \begin{equation*}
    W(x_1, \ldots, x_s) =
    \begin{pmatrix}
      x_1^{s-1} & x_2^{s-1} & \cdots & x_s^{s-1} \\
      x_1^{s-2} & x_2^{s-2} & \cdots & x_s^{s-2} \\
      \vdots    & \vdots    & \ddots & \vdots    \\
      x_1^1     & x_2^1     & \cdots & x_s^1     \\
      x_1^0     & x_2^0     & \cdots & x_s^0
    \end{pmatrix}
  \end{equation*}

  Due to its similarity with the \hyperref[ex:vandermonde_matrix]{Vandermonde matrix}, we can analogously deduce its determinant\fnote{The Vandermonde matrix has \( x_j - x_i \) in its determinant instead of \( x_i - x_j \).}:
  \begin{equation}\label{eq:thm:resultant_as_product/proof/w_determinant}
    \det W(x_1, \ldots, x_s) = \prod_{i < j} (x_i - x_j).
  \end{equation}

  Consider the matrices
  \begin{align*}
    Z        &\coloneqq W(\alpha_1, \ldots, \alpha_n, \beta_1, \ldots, \beta_m), \\
    Z_\alpha &\coloneqq W(\alpha_1, \ldots, \alpha_n), \\
    Z_\beta  &\coloneqq W(\beta_1, \ldots, \beta_m), \\
    D_p      &\coloneqq \op{diag}(p(\beta_1), \cdots, p(\beta_m)), \\
    D_q      &\coloneqq \op{diag}(q(\alpha_1), \cdots, q(\alpha_n)).
  \end{align*}

  Then
  \begin{equation}\label{eq:thm:resultant_as_product/proof/matrix_product}
    S(p, q) \cdot Z =
    \parens*
      {
        \begin{array}{c | c}
          0_{m \times n} & Z_\beta D_p   \\
          \hline
          Z_\alpha D_q   & 0_{n \times m}
        \end{array}
      }
  \end{equation}

  Indeed, denoting by \( c_{i,j} \) the \( (i, j) \)-th entry of the product \( S(p, q) \cdot Z \), we have
  \begin{equation*}
    c_{i,j} = \begin{cases}
      \alpha_j^{m-i} \cdot p(\alpha_j),         &1 \leq j \leq n \T{and} 1 \leq i \leq m, \\
      \beta_{j-n}^{m-i} \cdot p(\beta_{j-n}),   &n + 1 \leq j \leq n + m \T{and} 1 \leq i \leq m, \\
      \alpha_j^{n+m-i} \cdot q(\alpha_j),       &1 \leq j \leq m \T{and} m + 1 \leq i \leq n + m, \\
      \beta_{j-n}^{n+m-i} \cdot q(\beta_{j-n}), &n + 1 \leq j \leq n + m \T{and} m + 1 \leq i \leq n + m.
    \end{cases}
  \end{equation*}

  On the other hand, \( Z_\beta D_p \) is simpler:
  \begin{equation*}
    \begin{pmatrix}
      \beta_1^{m-1} \cdot p(\beta_1) & p(\beta_2) \cdot \beta_2^{m-1} & \cdots & \beta_m^{m-1} \cdot p(\beta_m) \\
      \beta_1^{m-2} \cdot p(\beta_1) & p(\beta_2) \cdot \beta_2^{m-2} & \cdots & \beta_m^{m-2} \cdot p(\beta_m) \\
      \vdots                         & \vdots                         & \ddots & \vdots                         \\
      \beta_1 \cdot p(\beta_1)       & p(\beta_2) \cdot \beta_2^{m-1} & \cdots & \beta_m \cdot p(\beta_m)       \\
      p(\beta_1)                     & p(\beta_2)                     & \cdots & p(\beta_m)
    \end{pmatrix}.
  \end{equation*}

  This demonstrates equality in \eqref{eq:thm:resultant_as_product/proof/matrix_product}.

  We can now use \eqref{eq:thm:resultant_as_product/proof/w_determinant} to determine the determinant of \( S(p, q) \cdot Z \):
  \begin{equation*}
    R(p, q) \cdot \det Z
    =
    R(p, q) \cdot \prod_{i=1}^n \prod_{j=1}^n (\alpha_i - \beta_j) \cdot \prod_{i_1 < i_2} (\alpha_{i_1} - \alpha_{i_2}) \cdot \prod_{j_1 < j_2} (\beta_{j_1} - \beta_{j_2}).
  \end{equation*}

  The determinant of \( Z_\beta D_p \) is
  \begin{equation*}
    \det(Z_\beta) \cdot \det(D_p)
    =
    \prod_{j_1 < j_2 \leq m} (\beta_{j_1} - \beta_{j_2}) \cdot \prod_{j=1}^m p(\beta_j)
    =
    \prod_{j_1 < j_2 \leq m} (\beta_{j_1} - \beta_{j_2}) \cdot a_n^m \prod_{j=1}^m \prod_{i=1}^n \underbrace{(\beta_j - \alpha_i)}_{(-1)(\alpha_i - \beta_j)},
  \end{equation*}
  where we have used that
  \begin{equation*}
    p(X) = a_n \prod_{i=1}^n (X - \alpha_i).
  \end{equation*}

  We can analogously obtain the determinant of \( Z_\alpha D_q \). From \fullref{thm:2x2_block_antidiagonal_determinant} it follows that the determinant of the right side of \eqref{eq:thm:resultant_as_product/proof/matrix_product} is
  \begin{equation*}
    (-1)^{nm} \cdot \det(Z_\beta D_p) \cdot \det(Z_\alpha D_q)
  \end{equation*}
  which we can expand to
  \begin{equation*}
    a_n^m \cdot b_m^n \cdot \prod_{i=1}^n \prod_{j=1}^m (\alpha_i - \beta_j) \cdot \det Z.
  \end{equation*}

  The determinant of the left side of \eqref{eq:thm:resultant_as_product/proof/matrix_product} is
  \begin{equation*}
    R(p, q) \cdot \det Z.
  \end{equation*}

  Cancelling \( \det Z \), we obtain \eqref{eq:thm:resultant_as_product}.
\end{proof}

\begin{corollary}\label{thm:resultant_invertibility}
  For polynomials \( p(X) \) and \( q(X) \) over an \hyperref[def:algebraically_closed_field]{algebraically closed field}, the \hyperref[def:resultant]{resultant} \( R(p, q) \) is zero if and only if \( p(X) \) and \( q(X) \) have a common root.
\end{corollary}
\begin{proof}
  In an algebraically closed field, both \( p(X) \) and \( q(X) \) split into linear factors. The result is the immediate from \fullref{thm:resultant_as_product}.
\end{proof}

\begin{definition}\label{def:discriminant}\mcite[prop. 8.5]{Lang2002Algebra}
  We define the \term[bg=дискриминанта (\cite[215]{Обрешков1962ВисшаАлгебра}), ru=дискриминант (\cite[141]{Винберг2014Алгебра}), en=discriminant (\cite[223]{Rotman2015AlgebraVol1})]{discriminant} of a non-constant polynomial
  \begin{equation*}
    p(X) = \sum_{k=0}^n a_k X_k
  \end{equation*}
  as
  \begin{equation*}
    D(p) \coloneqq \frac {(-1)^{n(n-1)/2}} {a_n} \cdot R(p, p'),
  \end{equation*}
  where \( R(p, p') \) is the \hyperref[def:resultant]{resultant} of \( p(X) \) and its \hyperref[def:algebraic_derivative]{algebraic derivative} \( p'(X) \).
\end{definition}
\begin{comments}
  \item Discriminants are often defined via the roots of \( p(X) \) and later this property is proven to be equivalent, for example by
  \incite[227]{Кострикин2000АлгебраТом1},
  \incite[258]{Jacobson1985AlgebraPart1},
  \incite[192]{Lang2002Algebra} and
  \incite[223]{Rotman2015AlgebraVol1}.

  We prefer the given definition because it does not explicitly use roots, the existence of which is a strong assumption and requires relying on \hyperref[def:splitting_field]{splitting fields}.

  We state the other definition as a characterization in \fullref{thm:discriminant_as_product}.
\end{comments}

\begin{lemma}\label{thm:derivative_at_polynomial_root}
  For the \hyperref[def:algebraic_derivative]{algebraic derivative} of
  \begin{equation*}
    p(X) = \prod_{k=1}^n (X - \alpha_k)
  \end{equation*}
  we have, for every \( m = 1, \ldots, n \),
  \begin{equation}\label{eq:thm:derivative_at_polynomial_root}
    p'(\alpha_m) = \prod_{k \neq m} (\alpha_m - \alpha_k).
  \end{equation}
\end{lemma}
\begin{proof}
  We will use induction on \( n \). The case \( n = 0 \) is vacuous, so suppose that the lemma holds for \( n - 1 \). \Fullref{thm:def:algebraic_derivative/product} implies that
  \begin{equation}\label{eq:thm:derivative_at_polynomial_root/product_derivative}
    p'(X) = \parens[\Big]{ \prod_{k=1}^{n-1} (X - \alpha_k) }' \cdot (X - \alpha_n) + \prod_{k=1}^{n-1} (X - \alpha_k) \cdot \underbrace{(X - \alpha_n)'}_{1}.
  \end{equation}

  \begin{itemize}
    \item If \( m = n \), the first term of \eqref{eq:thm:derivative_at_polynomial_root/product_derivative} vanishes and we are left with
    \begin{equation*}
      p'(\alpha_n) = \prod_{k=1}^{n-1} (\alpha_m - \alpha_k),
    \end{equation*}
    as desired.

    \item If \( m < n \), by the inductive hypothesis
    \begin{equation*}
      p'(\alpha_m) = \parens[\Big]{ \prod_{\substack{k \neq m \\ k \neq n}} (\alpha_m - \alpha_k) } \cdot (\alpha_m - \alpha_n) + \underbrace{\prod_{k=1}^{n-1} (\alpha_m - \alpha_k)}_{0}.
    \end{equation*}
  \end{itemize}
\end{proof}

\begin{proposition}\label{thm:discriminant_as_product}
  Fix an \hyperref[def:algebraically_closed_field]{\hi{algebraically closed}} \hyperref[def:field]{field} \( \BbbK \) and a polynomial
  \begin{equation*}
    p(X) = \sum_{k=0}^n a_k X^k.
  \end{equation*}

  Let \( \alpha_1, \ldots, \alpha_n \) be an enumeration of its roots. Then for its \hyperref[def:matrix_determinant]{determinant} we have
  \begin{equation}\label{eq:thm:discriminant_as_product}
    D(p) = a_n^{2n - 2} \prod_{i < j} (\alpha_i - \alpha_j)^2.
  \end{equation}
\end{proposition}
\begin{proof}
  Let \( \beta_1, \ldots, \beta_{n-1} \) be the roots of \( p' \).

  \Fullref{thm:resultant_as_product} implies that
  \begin{equation}\label{eq:thm:discriminant_as_product/proof/resultant}
    D(p) = \frac {(-1)^{n(n-1)/2}} {a_n} \cdot a_n^{n-1} \cdot (n a_n)^n \cdot \prod_{i=1}^n \prod_{j=1}^{n-1} (\alpha_i - \beta_j).
  \end{equation}

  Note that
  \begin{equation*}
    p'(\alpha_i) = n a_n \prod_{j=1}^{n-1} (\alpha_i - \beta_j),
  \end{equation*}
  however from \fullref{thm:derivative_at_polynomial_root} it follows that
  \begin{equation*}
    p'(\alpha_i) = a_n \prod_{j \neq i} (\alpha_i - \alpha_j).
  \end{equation*}

  Therefore, we can simplify \eqref{eq:thm:discriminant_as_product/proof/resultant} to
  \begin{equation*}
    D(p) = (-1)^{n(n-1)/2} \cdot a_n^{2n-2} \prod_{i=1}^n \prod_{j \neq i} (\alpha_i - \alpha_j).
  \end{equation*}

  It remains to use
  \begin{equation*}
    (\alpha_i - \alpha_j) = (-1)(\alpha_j - \alpha_i)
  \end{equation*}
  whenever \( j > i \) to conclude \eqref{eq:thm:discriminant_as_product}.
\end{proof}

\begin{corollary}\label{thm:discriminant_invertibility}
  For a polynomial \( p(X) \) over an \hyperref[def:algebraically_closed_field]{algebraically closed field}, the \hyperref[def:discriminant]{discriminant} \( D(p) \) is zero if and only if \( p(X) \) has repeated roots.
\end{corollary}
\begin{proof}
  Immediate from \fullref{thm:discriminant_as_product}.
\end{proof}

\paragraph{Quadratic polynomials}

\begin{lemma}\label{thm:quadratic_polynomial_discriminant}
  The \hyperref[def:discriminant]{discriminant} of the quadratic polynomial \( p(X) = a X^2 + b X + c \) is
  \begin{equation}\label{eq:thm:quadratic_polynomial_discriminant}
    D(p) = b^2 - 4ac
  \end{equation}
\end{lemma}
\begin{proof}
  We have
  \begin{align*}
    D(p)
    &=
    \frac {(-1)} a \cdot \begin{pmatrix}
      a  & b  & c  \\
      2a & b  &    \\
         & 2a & b
    \end{pmatrix}
    \reloset {\eqref{eq:thm:3x3_determinant}} = \\ &=
    \frac {(-1)} a \cdot \parens[\Big]{ ab^2 + 0 + 4a^2c - 0 - 0 - 2ab^2 }
    = \\ &=
    b^2 - 4ac.
  \end{align*}
\end{proof}

\begin{proposition}\label{thm:quadratic_polynomial_roots}
  Fix a quadratic univariate polynomial \( p(X) = aX^2 + b X + c \) over an \hyperref[def:integral_domain]{integral domain}.

  If \( \alpha \) and \( \beta \) are the roots of \( p(X) \), then the \hyperref[def:discriminant]{discriminant} \( b^2 - 4ac \) is the square of \( s = a(\alpha + \beta) \) and
  \begin{align*}
    2a\alpha = -b + s
    &&\T{and}&&
    2a\beta = -b - s.
  \end{align*}
\end{proposition}
\begin{proof}
  Suppose that \( \alpha \) and \( \beta \) are the roots of \( p(X) \). \Fullref{thm:vietas_formulas} implies that
  \begin{equation*}
    b =  -a \cdot \sigma_1(\alpha, \beta) = -a(\alpha + \beta)
  \end{equation*}
  and
  \begin{equation*}
    c = a \cdot \sigma_2(\alpha, \beta) = a\alpha\beta.
  \end{equation*}

  The discriminant is
  \begin{equation*}
    D(p) = b^2 - 4ac = a^2(\alpha + \beta)^2 - 4a^2 \alpha\beta = a^2 (\alpha^2 + 2\alpha\beta + v^2) - 4a^2 \alpha\beta = a^2 (\alpha - \beta)^2.
  \end{equation*}

  Denote \( a(\alpha - \beta) \) by \( s \). Then
  \begin{equation*}
    2a\alpha = a(\alpha + \beta) + a(\alpha - \beta) = -b + s
  \end{equation*}
  and
  \begin{equation*}
    2a\beta = a(\alpha + \beta) - a(\alpha - \beta) = -b - s
  \end{equation*}
\end{proof}

\paragraph{Irreducible polynomials}

\begin{proposition}\label{thm:axx_byy_irreducible}
  Fix a \hyperref[def:totally_ordered_set]{totally} \hyperref[def:ordered_semiring]{ordered} \hyperref[def:field]{field} \( \BbbK \).

  The polynomial \( p(X, Y) = a X^2 + b Y^2 \), where \( a \) and \( b \) are either both \hyperref[def:ordered_semiring_positivity]{positive} or \hyperref[def:ordered_semiring_positivity]{negative}, is irreducible in \( \BbbK[X, Y] \).
\end{proposition}
\begin{proof}
  Fix some decomposition \( p(X, Y) = f(X, Y) \cdot g(X, Y) \).

  Aiming at a contradiction, suppose that both are not invertible. Then both have positive degree. \Fullref{thm:divisors_of_homogeneous_polynomial} implies that both \( f(X, Y) \) and \( g(X, Y) \) are homogeneous, and \fullref{thm:polynomial_degree_arithmetic/product} leaves only the possibility that both are linear homogeneous.

  Let
  \begin{align*}
    f(X, Y) &= c X + d Y + e \\
    g(X, Y) &= f X + g Y + h.
  \end{align*}

  We have
  \begin{equation*}
    f(X, Y) \cdot g(X, Y) = c f X^2 + c g X Y + c h X + d f X Y + d g Y^2 + d h Y + e f X + e g Y + e h.
  \end{equation*}

  In order for there to be no mixed monomials, we must have \( c g = - d f \). Furthermore, \( c \), \( d \), \( f \) and \( g \) are nonzero because otherwise either \( a = c f \) or \( b = d g \) would be zero.

  Since \( a = c f \), it follows that \( f = a / c \), and similarly \( g = b / d \). Then
  \begin{equation*}
    c g = c \cdot \frac b d = - d \cdot \frac a c = - d f.
  \end{equation*}

  Multiplying by \( cd \), we obtain
  \begin{equation}\label{eq:thm:axx_byy_irreducible/proof/contradiction}
    c^2 \cdot b = - d^2 \cdot a.
  \end{equation}

  We have assumed that \( a \) and \( b \) have matching signs. \Fullref{thm:ordered_ring_power} implies that both \( c^2 \) and \( d^2 \) are positive. Then \fullref{thm:def:signum} implies that \( c^2 \cdot b \) and \( d^2 \cdot a \) also have matching signs.

  But then they cannot satisfy \eqref{eq:thm:axx_byy_irreducible/proof/contradiction}. The obtained contradiction shows that \( p(X, Y) \) is irreducible.
\end{proof}

\begin{corollary}\label{thm:ordered_field_not_algebraically_closed}
  A \hyperref[def:totally_ordered_set]{totally} \hyperref[def:ordered_semiring]{ordered} \hyperref[def:field]{field} cannot be \hyperref[def:algebraically_closed_field]{algebraically closed}.
\end{corollary}
\begin{proof}
  \Fullref{thm:axx_byy_irreducible} implies that \( X^2 + Y^2 \) is an irreducible polynomial, and \fullref{thm:homogeneous_polynomial_constant} imply that \( X^2 + 1 \) is also irreducible.
\end{proof}

\begin{proposition}\label{thm:axx_byy_czz_irreducible}
  The polynomial \( p(X, Y, Z) = a X^2 + b Y^2 + c Z^2 \), where \( a \), \( b \) and \( c \) are nonzero scalars from an arbitrary \hyperref[def:field]{field} \( \BbbK \), is irreducible in \( \BbbK[X, Y, Z] \).
\end{proposition}
\begin{proof}
  As in \fullref{thm:axx_byy_irreducible}, suppose that \( p(X, Y, Z) \) is a product of the linear polynomials
  \begin{align*}
    q(X, Y, Z) &= d X + e Y + f Z + g, \\
    r(X, Y, Z) &= h X + i Y + j Z + k.
  \end{align*}

  Then \( q(X, Y, Z) \cdot r(X, Y, Z) \) is
  \begin{balign*}
    &\phantom{{}+{}}
    d h X^2 + d i X Y + d j X Z + d k X
    + \\ &+
    e h X Y + e i Y^2 + e j Y Z + e k Y
    + \\ &+
    f h X Z + f i Y Z + f j Z^2 + f k Z
    + \\ &+
    g h X + g i Y + g j Z + g k.
  \end{balign*}

  Since \( a = dh \), \( b = ei \) and \( c = fj \) are nonzero, it follows that the corresponding scalars are nonzero. Furthermore, we must have
  \begin{align*}
    (d i + e h) X Y &= 0, \\
    (d j + f h) X Z &= 0, \\
    (e j + f i) Y Z &= 0,
  \end{align*}
  that is,
  \begin{align*}
    d i &= - e h, \\
    d j &= - f h, \\
    e j &= - f i.
  \end{align*}

  We can divide the first two equalities to obtain \( i / j = e / f \), i.e. \( ej = fi \). But the third equality states that \( ej = -fi \). Hence, \( ej \) and \( fi \) can both only be zero. But we know that \( e \), \( f \), \( i \) and \( j \) are all nonzero, hence \( ej \) and \( fi \) must also be nonzero.

  The obtained contradiction shows that the polynomial \( p(X, Y, Z) = a X^2 + b Y^2 + c Z^2 \) is irreducible over any field.
\end{proof}

\begin{proposition}\label{thm:axz_byy_irreducible}
  The polynomial \( p(X, Y, Z) = a XZ + b Y^2 \), where \( a \) and \( b \) are nonzero scalars from an arbitrary \hyperref[def:field]{field} \( \BbbK \), is irreducible in \( \BbbK[X, Y, Z] \).
\end{proposition}
\begin{proof}
  As in \fullref{thm:axx_byy_czz_irreducible}, suppose that the homogeneous polynomial \( p(X, Y, Z) \) is a product of the linear polynomials
  \begin{align*}
    q(X, Y, Z) &= d X + e Y + f Z + g, \\
    r(X, Y, Z) &= h X + i Y + j Z + k.
  \end{align*}

  Unlike in \fullref{thm:axx_byy_czz_irreducible}, we need \( dh \) and \( fj \), the coefficients of \( X^2 \) and \( Z^2 \), to be zero. The coefficient \( dj + fh \) of \( XZ \) must be nonzero, however. Then either \( d \) and \( j \) are nonzero and \( h = f = 0 \), or vice versa.

  The coefficients \( ei \) of \( Y^2 \) is also nonzero, hence \( e \neq 0 \) and \( i \neq 0 \).

  \begin{itemize}
    \item If \( h = f = 0 \), then, since the coefficient \( di + eh \) of \( XY \) must be zero, it follows that \( i = 0 \).

    \item If \( d = j = 0 \), then, since the coefficient \( ej + fi \) of \( YZ \) must be zero, it again follows that \( i = 0 \).
  \end{itemize}

  In both cases we obtain a contradiction. Therefore, \( p(X, Y, Z) \) is irreducible.
\end{proof}
