\section{Numbers}\label{sec:numbers}

Numbers are perhaps the most ubiquitous concept in mathematics. Even among non-mathematicians, division by zero or \( 0.\oline{9} = 1 \) seem to be a common topic of discourse, either as a joke or a sincere misunderstanding.

The aforementioned topics were studied extensively by mathematicians and have simple justifications from the point of view of abstract mathematics:
\begin{itemize}
  \item We may want to somehow define division by zero in the \hyperref[def:real_numbers]{field \( \BbbR \) of real numbers}, however that would introduce \hyperref[def:divisibility/zero]{zero divisors} and hence deprive \( \BbbR \) of being an \hyperref[def:integral_domain]{integral domain}. The cancellative property of multiplication would not hold, and hence \( xy = zy \) would not imply that \( x = z \) when \( y \neq 0 \).

  Our familiar arithmetic of real numbers heavily relies on the cancellative property, therefore we simply disallow division by zero.

  \item The set \( \BbbR \) of real numbers is a uniform space and thus every \hyperref[def:fundamental_net]{fundamental sequence} is convergent by \fullref{thm:cauchys_net_convergence_criterion}. Furthermore, since \( \BbbR \) is also a \hyperref[def:separation_axioms/T2]{Hausdorff} space, by \fullref{thm:t2_iff_singleton_limits}, every fundamental sequence has a unique limit.

  Now consider the following two fundamental sequences:
  \begin{align*}
    &1, 1, 1, 1, \ldots \\
    &0, 0.9, 0.99, \ldots
  \end{align*}

  Their absolute difference
  \begin{equation*}
    1, 0.1, 0.01, \ldots
  \end{equation*}
  converges to \( 0 \).

  Therefore, the two original sequences converge to the same real number, namely \( 1 \).

  This example is generalized in \fullref{ex:def:real_number_radix_expansion/geometric}.
\end{itemize}

Unfortunately, a formal study of numbers also leads to artifacts such as the nonstandard natural numbers discussed in \fullref{rem:standard_models_of_arithmetic}.

We will build and describe from the perspective of \fullref{sec:order_theory} and \fullref{sec:ring_theory} the following:
\begin{itemize}
  \item \Fullref{subsec:natural_numbers}, built upon \fullref{sec:mathematical_logic}.
  \item \Fullref{subsec:integers}.
  \item \Fullref{subsec:rational_numbers}.
  \item \Fullref{subsec:real_numbers}.
  \item \Fullref{subsec:complex_numbers}.
\end{itemize}

Additionally, we will describe syntactic presentations of numbers in \fullref{subsec:positional_number_systems}.
