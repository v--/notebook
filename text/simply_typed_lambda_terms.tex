\section{Simply typed lambda terms}\label{sec:simply_typed_lambda_terms}

Unlike untyped lambda calculus, whose presentation is mostly uniform across authors, type theory has many incompatible flavors. A concise history is described in \fullref{rem:type_theory}. We will only consider so-called \enquote{simple types}. We base the exposition here on \fullref{sec:natural_deduction}.

\paragraph{Type systems}

\begin{remark}\label{rem:type_theory}
  \incite*[237]{Russell1908TypeTheory} gives the following definition:
  \begin{displayquote}
    A \textit{type} is defined as the range of significance of a propositional function, i.e., as the collection of arguments for which the said function has values.
  \end{displayquote}

  Russell himself was trying to resolve \hyperref[rem:self_reference]{self-referential} paradoxes by circumventing \hyperref[con:impredicativity]{impredicativity}. This is briefly discussed in \fullref{rem:self_reference}.

  He used \enquote{individuals} loosely for \enquote{terms of elementary propositions}, where he defined terms as \enquote{whatever can be regarded as the \textit{subject} of the proposition}.

  A clearer formalization is due to Alonzo Church. In \cite{Church1940STT}, he extends untyped \( \synlambda \)-calculus with what he, based on intermediate works by other authors, calls \enquote{simple types}. He uses the letter \( \omicron \) for \enquote{the type of propositions} and \( \iota \) for \enquote{the type of individuals}. Given the types \( \alpha \) and \( \beta \), Church regards the type \( (\alpha\beta) \) as
  \begin{displayquote}
    \textellipsis the type of functions of one variable for which the range of the independent variable comprises the type \( \beta \) and the range of the dependent variable is contained in the type \( \alpha \).
  \end{displayquote}

  Thus, he used two \enquote{base types} and one \enquote{type constructor}, resulting in infinitely many types. He assigned to each \( \synlambda \)-term a subscript indicating its type, with certain coherence conditions that we will discuss in \fullref{def:simple_type_theory}. The term \( x_\omicron \) would thus correspond to our notion of propositional formula, while the term \( x_\iota \) --- to a first-order variable.

  Subsequent works by Curry and Howard lead to the Curry-Howard correspondence discussed in \fullref{con:curry_howard_correspondence}. This in turn lead Martin-Löf to introduce \hyperref[rem:dependent_type]{dependent types} and systemize the presentation of type theory via \hyperref[def:typing_rule]{typing rules}.

  At this point, notation started unifying across authors, ultimately leading to \enquote{\( M: \alpha \)} rather than \enquote{\( M_\alpha \)} or \enquote{\( M^\alpha \)} for \hyperref[def:type_assertion]{type assertions} postulating that the term \( M \) has type \( \alpha \), and to \( \alpha \synimplies \beta \) for \enquote{function types}, replacing Church's equivalent \( (\beta\alpha) \) (and swapping \( \alpha \) and \( \beta \)). As we will see in \fullref{con:type_annotation} and \fullref{def:pseudoterm}, superscript notation is still convenient for type annotation.

  These conventions were already present in \bycite{CoquandHuet1986CoC}. His paper introduces the \enquote{calculus of constructions} in which functions can be defined from terms to types, from types to terms and from types to types. This generalizes Martin-L\"of's dependent products, which can be regarded as functions from terms to types. \incite[192]{Barendregt1992Types} places simple type theory and the calculus of constructions as the least and most powerful type systems in his \enquote{\( \synlambda \)-cube}, where the axes correspond to the aforementioned three kinds of functions. These are further generalized by \enquote{pure type systems}, again discussed in the same article. As types become more complicated, they require flexible syntax, and the line between types and terms starts blurring.

  Another significant development is \enquote{homotopy type theory}, introducing semantics for types based on topological spaces. In the eponymous community-written book, \cite{UnivalentProject2024OctoberHoTT}, homotopy type theory is attributed to \cite{Voevodski2006HoTT} and \cite{AwodeyWarren2009HoTT}.

  Simple, dependent and homotopy type theory are described in a unified manner in \cite{Mimram2020Types} in the context of proof theory.
\end{remark}

\begin{definition}\label{def:type_assertion}\mimprovised
  Suppose we are given two \hyperref[def:formal_grammar/schema]{formal grammars} producing strings called \term{terms} and \term{types}, correspondingly. Terms and types may intersect or even coincide, as mentioned in \fullref{rem:type_theory}.

  We will be interested in expressions generated by the following grammar:
  \begin{bnf*}
    \bnfprod{type assertion} {\bnfpn{term} \bnfsp \bnftsq{\( : \)} \bnfsp \bnfpn{type}}.
  \end{bnf*}
\end{definition}
\begin{comments}
  \item When terms are given an interpretation as some objects, we may say \enquote{\( x \) is an object of type \( \alpha \)} rather than \enquote{\( x \) is a term of type \( \alpha \)}.

  \item On a metatheoretic level, this is one of several kinds of \hyperref[con:judgment]{judgments} that Martin-L\"of uses for his type systems; see \fullref{rem:typing_judgments} for a broader discussion.

  \item This terminology is unfortunately not well-established. For example, \incite[def. 2A4]{Hindley1997STT} and \incite[380]{BarendregtDekkersStatman2013Types} prefer the phrase \enquote{type assignment}, \incite[def. 5.1.10]{Barendregt1992Types} uses \enquote{statement}, while \incite[ch. 4]{Mimram2020Types} uses the type assertion syntax without introducing terminology for such expressions.

  \item Unlike the terminology, the colon notation itself is ubiquitous, as discussed in \cite{MathSE:origin_of_colon_notation_for_type_assertions}. Predecessors include the following:
  \begin{itemize}
    \item The subscript notation \( M_\alpha \) used by \cite{Church1940STT} and the modern by old-fashioned \cite{Andrews2002STT}.
    \item The superscript notation \( M^\alpha \) used by \cite{Howard1980FormulasAsTypes}.
    \item The set membership notation \( M \in \alpha \) used by Martin-L\"of in his earlier works \cite{MartinLöf1975IntTypeTheory} and \cite{MartinLöf1984IntTypeTheory}. In \cite{MartinLöf1994TypeJudgments} he already uses colons.
  \end{itemize}
\end{comments}

\begin{remark}\label{rem:type_universes}
  If in some flavors of \hyperref[rem:type_theory]{type theory} terms and types share a common syntax, we can express the statement \enquote{\( \alpha \) is a type} via the \hyperref[def:type_assertion]{type assertion}
  \begin{equation*}
    \alpha: \op*{Type},
  \end{equation*}
  where \( \op*{Type} \) is a predefined constant term.

  Unfortunately, as described in \incite[\S 8.2.3]{Mimram2020Types}, in a sufficiently general setting this leads to a variant of \fullref{thm:russels_paradox} due to Jean-Yves Girard. A workaround suggested by Mimram resembling the \hyperref[def:axiom_of_universes]{axiom of Grothendieck universes} is the introduction of a sequence of type universes
  \begin{equation*}
    \op*{Type}_0: \op*{Type}_1: \op*{Type}_2: \cdots
  \end{equation*}
  such that \( \alpha: \op*{Type}_k \) implies \( \alpha: \op*{Type}_{k+1} \).
\end{remark}

\begin{definition}\label{con:definitional_equality}
  An important concept in type theory is \term{definitional equality}. \incite[40]{MartinLöf1984IntTypeTheory} describes it as follows:
  \begin{displayquote}
    Definitional equality is intensional equality, or equality of meaning (synonymy). We use the symbol \( {\equiv} \) or \( {=_{\T{def.}}} \) (which was first introduced by Burali-Forti). Definitional equality \( {\aequiv} \) is a relation between linguistic expressions; it should not be confused with equality between objects (sets, elements of a set, etc.) which we denote by \( {=} \). Definitional equality is the equivalence relation generated by abbreviatory definitions, changes of bound variables and the principle of substituting equals for equals.
  \end{displayquote}

  Thus, roughly, definitional equality is a binary relation in an \hyperref[con:metalogic]{object language} with the intended interpretation of an \hyperref[def:equivalence_relation]{equivalence relation}.

  Martin-L\"of himself uses \hyperref[def:inference_rule]{inference rules} to specify definitional equality, which on a metatheoretic level amounts to using \fullref{thm:recursively_defined_relations}.
\end{definition}

\begin{remark}\label{rem:typing_judgments}
  Martin-L\"of builds his type theory on several kinds of \hyperref[con:judgment]{judgments}. \cite[19]{UnivalentProject2024OctoberHoTT} classifies these as either \hyperref[def:type_assertion]{type assertion} judgment or \hyperref[con:definitional_equality]{definitional equality} judgments.

  \incite*[14]{MartinLöf1984IntTypeTheory} distinguishes between the judgments \enquote{\( \alpha \) is a type} and \enquote{\( x \) is an object of type \( \alpha \)}\fnote{To be precise, these aren't exact quotes because Martin-L\"of refers to types as \enquote{sets} in this concrete work, and uses set-membership notation, which is a deviation from his other, even earlier, works.}, and similarly between \enquote{\( \alpha \) and \( \beta \) are equal sets} and \enquote{\( x \) and \( y \) are equal objects of type \( \alpha \)}. As discussed in \fullref{rem:type_universes}, however, these amount to the same judgment since we can regard types as objects of a \enquote{type universe}.
\end{remark}

\begin{definition}\label{def:type_context}\mcite[def. 2A5]{Hindley1997STT}
  A \term[en=type-context (\cite[def. 2A5]{Hindley1997STT})]{type context} a (set-based) \hyperref[def:logical_context]{logical context} whose entries are \hyperref[def:type_assertion]{type assertion} for \hi{distinct variables}. We have purposely specified no syntax for neither the terms nor types, and we simply assume that the terms contain a subset consisting of variables.

  We say that the assertion \( x: \alpha \) is \term{consistent} with the context \( \Gamma \) if \( \Gamma \) contains no type assertion for \( x \).
\end{definition}
\begin{comments}
  \item We generalize the definition from the case of \hyperref[def:simple_type]{simple types} considered by Hindley.

  \item The decision to restrict ourselves to variables comes is discussed in \fullref{rem:type_context_only_variables}.

  \item Another possibility, instead of requiring the variables to be distinct, is, as in \cite[159]{Mimram2020Types}, to allow contexts to contain more than one type assertion per variable and only consider the last assertion.
\end{comments}

\begin{definition}\label{def:abstract_type_system}\mimprovised
  A \term{abstract type system} is, roughly, a systematic way of evaluating \hyperref[def:type_assertion]{type assertions}, usually via a collection of \hyperref[con:typing_rule]{typing rules}.

  Such a system must feature a \hyperref[def:consequence_relation]{consequence relation} on \hyperref[def:type_assertion]{type assertions} whose contexts must be \hyperref[def:type_context]{type contexts}, i.e. contain type assertions for distinct variables.
\end{definition}
\begin{comments}
  \item A very general definition is given by \incite[def. 5.2.1]{Barendregt1992Types}, who considers \enquote{pure type systems} with a concrete common syntax for both terms and types, but with only an informal notion of typing rules. Here we abstract away both the syntax and the way of obtaining a consequence relation.
\end{comments}

\paragraph{Simple types}

\begin{definition}\label{def:simple_type}\mimprovised
  Fix a nonempty finite \hyperref[def:formal_language/alphabet]{alphabet} \( \op*{Base} \), whose elements we will call \term{base types}. We introduce \term{simple types} based on the following \hyperref[def:formal_grammar/schema]{grammar schema}:
  \begin{bnf*}
    \bnfprod{arrow type}   {\bnftsq{(} \bnfsp \bnfpn{type} \bnfsp \bnftsq{\( \synimplies \)} \bnfsp \bnfpn{type} \bnfsp \bnftsq{)}} \\
    \bnfprod{type}         {\bnfpn{base type} \bnfor \bnfpn{arrow type}}
  \end{bnf*}

  We will also find useful the following extensions:
  \begin{bnf*}
    \bnfprod{product type} {\bnftsq{(} \bnfsp \bnfpn{type} \bnfsp \bnftsq{\( \syntimes \)} \bnfsp \bnfpn{type} \bnfsp \bnftsq{)}} \\
    \bnfprod{sum type}     {\bnftsq{(} \bnfsp \bnfpn{type} \bnfsp \bnftsq{\( \synplus \)} \bnfsp \bnfpn{type} \bnfsp \bnftsq{)}}
  \end{bnf*}

  We handle product types in \fullref{def:product_type} and sum types in \fullref{def:sum_type}, and use both in \fullref{alg:proof_tree_to_type_derivation} and \fullref{alg:type_derivation_to_proof_tree}, but avoid them otherwise. In order to be explicit about specifying these and possibly other extensions, we introduce simple type systems in \fullref{def:simple_type_system}.
\end{definition}
\begin{comments}
  \item As it often happens, usage in the literature is not consistent. We discuss this in \fullref{rem:simple_type_terminology_and_notation}. To avoid ambiguity, we will be explicit about which types we allow in our applications.

  \item We will mostly restrict ourselves to arrow types because handling each extension requires additional work. The literature listed in \fullref{rem:simple_type_terminology_and_notation} avoids extensions unless necessary.
\end{comments}

\begin{remark}\label{rem:simple_type_terminology_and_notation}
  \hyperref[def:simple_type]{Simple types} are unfortunately not standardized, and naturally, allow variations in their definition.

  For example, the following authors consider arrow types (but not e.g. product types), however their definitions differ by their handling of base types:
  \begin{itemize}
    \item \incite[56]{Church1940STT}, \incite[\S 51]{Andrews2002STT} and \incite[269]{Farmer2008STTVirtues} allow exactly two base types --- of propositions and individuals, as discussed in \fullref{rem:type_theory}.

    \item \incite[def. 3.2.2]{Barendregt1992Types}, \incite{Hindley1997STT} and \incite[def. 1.2.1]{TroelstraSchwichtenberg2000Proofs} disallow base types, thus leaving only arrow types and \enquote{type variables} (resembling the variable placeholders from \fullref{def:lambda_term_schema}).

    \item \incite[def. 1.1.11]{BarendregtDekkersStatman2013Types} allow an arbitrary nonempty set of base types, which they call \enquote{type atoms}.
  \end{itemize}

  \incite[ch. 4]{Mimram2020Types} initially only considers arrow types without base types, but later allows two base types --- the empty and unit types -- as well as product and sum types.

  Additionally, what we call \enquote{arrow type} is also treated differently:
  \begin{itemize}
    \item \incite[56]{Church1940STT} and \incite[\S 51]{Andrews2002STT} use the notation \enquote{\( (\alpha\beta) \)} for what we would write as \( \beta \synimplies \alpha \). This is discussed in \fullref{rem:type_theory}.

    Neither introduce a dedicated term for arrow types.

    \item \incite[def. 1.2.1]{TroelstraSchwichtenberg2000Proofs} and \incite[269]{Farmer2008STTVirtues} use arrow notation and call them \enquote{function types}.

    \item \incite[\S 4.1.1]{Mimram2020Types} uses the arrow notation and calls them \enquote{arrow types}.

    \item \incite[def. 1.1.11]{BarendregtDekkersStatman2013Types} uses the arrow notation and calls them \enquote{function space types}.

    \item \incite[def. 2A1]{Hindley1997STT} uses the arrow notation and calls them \enquote{composite types}.

    \item \incite[def. 3.2.2]{Barendregt1992Types} use the arrow notation but introduces no special term.
  \end{itemize}
\end{remark}

\paragraph{Schemas for simply typed \( \synlambda \)-terms}\hfill

To formalize typing rules in \fullref{def:simple_typing_rule}, we first need to introduce schemas for \( \synlambda \)-terms and simple types.

\begin{definition}\label{def:lambda_term_schema}
  Similarly to \hyperref[def:propositional_formula_schema]{propositional formula schemas}, we define schemas for \hyperref[def:lambda_term]{\( \synlambda \)-terms}:
  \begin{bnf*}
    \bnfprod{variable placeholder} {\bnfpn{Small Latin identifier}}, \\
    \bnfprod{term placeholder}     {\bnfpn{Capital Latin identifier}}, \\
    \bnfprod{atom schema}          {\bnfpn{variable placeholder} \bnfor \bnfpn{constant}}, \\
    \bnfprod{application schema}   {\bnftsq{\( ( \)} \bnfsp \bnfpn{term schema} \bnfsp \bnfpn{term schema} \bnfsp \bnftsq{\( ) \)}}, \\
    \bnfprod{abstraction schema}   {\bnftsq{\( ( \)} \bnfsp \bnftsq{\( \synlambda \)} \bnfsp \bnfpn{variable schema} \bnfsp \bnftsq{.} \bnfsp \bnfpn{term schema} \bnfsp \bnftsq{\( ) \)}}, \\
    \bnfprod{term schema}          {\bnfpn{atom schema} \bnfor \bnfpn{term placeholder} \bnfor} \\
    \bnfmore                       {\bnfpn{application schema} \bnfor \bnfpn{abstraction schema}}.
  \end{bnf*}
\end{definition}
\begin{comments}
  \item See \fullref{con:schemas_and_instances} for a discussion of schemas in general.

  \item This syntax does not allow us to distinguish between variables and variable placeholders, however the distinction should be quite clear from the context.

  \item Similarly to propositional formula schemas, we will denote arbitrary schemas via capital Greek identifiers like \( \Phi \).
\end{comments}

\begin{definition}\label{def:simple_type_schema}
  We define schemas for \hyperref[def:simple_type]{simple types} in an obvious way:
  \begin{bnf*}
    \bnfprod{type placeholder}    {\bnfpn{Small Greek identifier}} \\
    \bnfprod{arrow type schema}   {\bnftsq{(} \bnfsp \bnfpn{type schema} \bnfsp \bnftsq{\( \synimplies \)} \bnfsp \bnfpn{type schema} \bnfsp \bnftsq{)}} \\
    \bnfprod{type schema}         {\bnfpn{base type} \bnfor \bnfpn{type placeholder} \bnfor \bnfpn{arrow type schema}}
  \end{bnf*}

  Product and sum types are handled similarly.
\end{definition}
\begin{comments}
  \item See \fullref{con:schemas_and_instances} for a discussion of schemas in general.
  \item To distinguish type schemas from term schemas, which we also denote via capital Greek identifiers, for types we will prefer letters from the start of the alphabet. Thus, \( \Psi: \Alpha \) is a type assertion for term and type schemas.
\end{comments}

\begin{definition}\label{def:lambda_schema_instantiation}
  In analogy with \hyperref[def:propositional_schema_instantiation]{schema instantiations} for \hyperref[def:propositional_formula_schema]{propositional formula schemas}, we define instantiations for \hyperref[def:type_assertion]{type assertions} for \hyperref[def:lambda_term_schema]{\( \synlambda \)-term schemas} and \hyperref[def:simple_type_schema]{simple type schemas} as \hyperref[def:set_valued_map/partial]{partial maps} sending \hyperref[def:lambda_term_schema]{variable placeholders} to \hyperref[def:lambda_term]{variables} and \hyperref[def:lambda_term_schema]{\( \synlambda \)-term placeholders} to \hyperref[def:lambda_term]{\( \synlambda \)-terms} and \hyperref[def:simple_type_schema]{type placeholders} to \hyperref[def:simple_type]{simple types}.

  \begin{thmenum}
    \thmitem{def:lambda_schema_instantiation/terms} We will not concern ourselves with \hyperref[def:lambda_variable_occurrence]{free and bound occurrence} of variables as for \hyperref[def:lambda_term_schema_substitution]{substitution}, but extend an instantiation \( \sigma \) to arbitrary \( \synlambda \)-term schemas straightforwardly:
    \begin{equation}\label{eq:def:lambda_schema_instantiation/terms}
      \Phi[\sigma] \coloneqq \begin{cases}
        M,                               &M \T{is a constant}, \\
        \sigma(M),                       &M \T{is a variable}, \\
        \sigma(M),                       &M \T{is a placeholder}, \\
        N[\sigma] \thinspace K[\sigma],  &M = NK,              \\
        \qabs {\sigma(x)} N[\sigma],     &M = \qabs x N.       \\
      \end{cases}
    \end{equation}

    \thmitem{def:lambda_schema_instantiation/types} For simple types we also extend the instantiation straightforwardly:
    \begin{equation}\label{eq:def:lambda_schema_instantiation/types}
      \Phi[\sigma] \coloneqq \begin{cases}
        \alpha,                                   &\alpha \T{is a base type},   \\
        \sigma(\alpha),                           &\alpha \T{is a placeholder}, \\
        \beta[\sigma] \synimplies \gamma[\sigma], &\alpha = \beta \synimplies \gamma,  \\
      \end{cases}
    \end{equation}

    \thmitem{def:lambda_schema_instantiation/assertion} Finally, given an assertion \( \Psi: \Alpha \), we let
    \begin{equation}\label{eq:def:lambda_schema_instantiation/assertion}
      (\Phi: \Alpha)[\sigma] \coloneqq \Phi[\sigma]: \Alpha[\sigma]
    \end{equation}
  \end{thmenum}
\end{definition}
\begin{comments}
  \item Of course, an application of an instantiation is only valid if the variables and placeholders of are in its domain.

  \item Another possibility is to keep instantiations for term and type schemas distinct, however our main use case is to instantiate type assertions and that would complicate us unnecessarily.
\end{comments}

\paragraph{Typing rules}

\begin{concept}\label{con:typing_rule}
  A \term{typing rule} is an \hyperref[def:inference_rule]{inference rule} whose entries are \hyperref[rem:typing_judgments]{typing judgments}, possibly for \hyperref[con:schemas_and_instances]{schemas} rather than terms and types.
\end{concept}
\begin{comments}
  \item We base this informal definition on Martin-L\"of discussion in \cite{MartinLöf1984IntTypeTheory} of rules for his type theory consisting of \hyperref[def:type_assertion]{type assertion} and \hyperref[con:definitional_equality]{definitional equality} judgments.
\end{comments}

\begin{definition}\label{def:simple_typing_rule}\mimprovised
  A \term{simple typing rule} is a \hyperref[con:typing_rule]{typing rule} whose entries are \hyperref[def:type_assertion]{type assertions} for \hyperref[def:lambda_term_schema]{\( \synlambda \)-term schemas} and \hyperref[def:simple_type_schema]{simple type schemas}.

  As in the case of natural deduction rules defined in \fullref{def:natural_deduction_rule}, and using the same formalism, we allow dischargeable type assertions to be attached to the premises.
\end{definition}
\begin{comments}
  \item This precise definition will allow us to mechanize the \hyperref[con:curry_howard_correspondence]{Curry-Howard correspondence} for simple types.

  \item Typing rules are often presented in another form that resembles sequent calculus rather than natural deduction --- see \fullref{rem:typing_rule_style}.
\end{comments}

\begin{definition}\label{def:base_typing_rules}\mcite[26]{TroelstraSchwichtenberg2000Proofs}
  The fundamental \hyperref[def:type_derivation_tree]{typing rules} for \hyperref[def:simple_type]{simple types} are:
  \begin{paracol}{2}
    \begin{leftcolumn}
      \begin{equation*}\taglabel[\ensuremath{ \logic{App}_\tau }]{inf:def:base_typing_rules/app}
        \begin{prooftree}
          \hypo{ \synM: \syn\alpha \synimplies \syn\beta }
          \hypo{ \synN: \syn\alpha }
          \infer2[\ref{inf:def:base_typing_rules/app}]{ \synM \synN: \syn\beta }
        \end{prooftree}
      \end{equation*}
    \end{leftcolumn}

    \begin{rightcolumn}
      \begin{equation*}\taglabel[\ensuremath{ \logic{Abs}_\tau }]{inf:def:base_typing_rules/abs}
        \begin{prooftree}
          \hypo{ [\synx: \syn\alpha] }
          \ellipsis {} { \synM: \syn\beta }
          \infer1[\ref{inf:def:base_typing_rules/abs}]{ \qabs \synx \synM: \syn\alpha \synimplies \syn\beta }
        \end{prooftree}
      \end{equation*}
    \end{rightcolumn}
  \end{paracol}
\end{definition}
\begin{comments}
  \item These are the base type rules present in this or a more general form in all our cited works of type theory --- see \fullref{rem:typing_rule_style}. We will require them to be a part of every \hyperref[def:simple_type_system]{simple type system}.
\end{comments}

\begin{remark}\label{rem:typing_rule_style}
  Consider the \hyperref[def:simple_typing_rule]{typing rule}
  \begin{equation*}
    \begin{prooftree}
      \hypo{ [\synx: \syn\alpha] }
      \ellipsis {} { \synM: \syn\beta }
      \infer1[\ref{inf:def:base_typing_rules/abs}]{ \qabs \synx \synM: \syn\alpha \synimplies \syn\beta }
    \end{prooftree}
  \end{equation*}

  \incite[\S 4.1.4]{Mimram2020Types} and \incite[def. 2A8]{Hindley1997STT} express this rule in a form that resembles \hyperref[rem:sequent_calculus]{sequent calculus}:
  \begin{equation*}
    \begin{prooftree}
      \hypo{ \syn\Gamma, x: \syn\alpha \synvdash \synM: \syn\beta }
      \infer1{ \syn\Gamma \synvdash \qabs \synx \synM: \syn\alpha \synimplies \syn\beta }
    \end{prooftree}
  \end{equation*}

  For expressing a generalization of this rule in their more complicated type systems, the sequent calculus style is used by \incite{CoquandHuet1986CoC}, \incite{AwodeyWarren2009HoTT}, \incite{Voevodski2006HoTT}.

  Barendregt introduces both styles of rules in \cite[def. 3.1.3]{Barendregt1992Types}, however prefers natural deduction style discharging for the actual proof trees.

  We prefer the natural deduction style rules since we find them easier to define formally, analyze and mechanize. This convention is used by Martin-L\"of, for example in \cite{MartinLöf1984IntTypeTheory}. \incite[26]{TroelstraSchwichtenberg2000Proofs} also use this style to highlight the connection with natural deduction for formulas.
\end{remark}

\paragraph{Type derivations}

\begin{definition}\label{def:simple_type_signature}\mimprovised
  A \term{simple type signature} \( \Sigma \) is a specification of the following:
  \begin{itemize}
    \item A possibly empty set \( \op*{Const}_\Sigma \) of \hyperref[def:lambda_term]{\( \synlambda \)-term} \hyperref[def:lambda_term/const]{constants}, along with a subset \( \op*{Term}_\Sigma \) of \( \synlambda \)-terms we wish to restrict ourselves to.

    \item A nonempty set \( \op*{Base}_\Sigma \) of \hyperref[def:simple_type]{base types}, along with a subset \( \op*{Types}_\Sigma \) of simple types we wish to restrict ourselves to.
  \end{itemize}
\end{definition}

\begin{definition}\label{def:simple_type_system}\mimprovised
  A \term{simple type system} over a \hyperref[def:simple_type_signature]{simple type signature} is, unsurprisingly, a collection of \hyperref[def:simple_typing_rule]{simple typing rules}. We will suppose that every such system includes the two rules from \fullref{def:base_typing_rules}.

  We will associate with each simple type system its derivability relation, which we will define via type derivation trees in \fullref{def:simple_type_derivability}.
\end{definition}
\begin{comments}
  \item Even though we will not require this, for the concrete systems we will make sure that the properties in \fullref{thm:def:simple_type_derivability} hold.
\end{comments}

\begin{definition}\label{def:type_derivation_tree}\mimprovised
  In analogy with natural deduction proof trees, which we define in \fullref{def:natural_deduction_proof_tree}, we will define here a family of \hyperref[def:labeled_tree]{labeled trees}, which we will call \term{type derivation trees}. Although we will be working in \hyperref[def:simple_type_system]{simple type systems}, the definition easily extends to more complicated systems.

  A fundamental difference compared to proof trees is that type derivations allow deriving the type of an arbitrarily complex \( \synlambda \)-term a \hyperref[def:type_context]{type context} specifying only the types of variables. This is justified in \fullref{rem:type_context_only_variables}.

  As in the case of natural deduction proof trees, the labels of type derivations are triples:
  \begin{thmenum}[series=def:type_derivation_tree]
    \thmitem{def:type_derivation_tree/conclusion} A \hyperref[def:type_assertion]{type assertion}, which we call the \term{conclusion}.
    \thmitem{def:type_derivation_tree/rule_name} An empty string or the name of a \hyperref[def:simple_typing_rule]{typing rule}.
    \thmitem{def:type_derivation_tree/context} A \hyperref[def:type_context]{type context}, whose entries we will call \term{open assumptions}.
  \end{thmenum}

  We define two kinds of derivation trees:
  \begin{thmenum}[resume=def:type_derivation_tree]
    \thmitem{def:type_derivation_tree/assumption} We define \term{assumption} trees similarly to natural deduction assumptions, but only for variable type assertions. For the assertion \( x: \alpha \), we define assumption tree as the \hyperref[def:canonical_singleton_tree]{canonical singleton tree} with the following label:
    \begin{itemize}
      \item The conclusion is \( x: \alpha \).
      \item The rule name is empty.
      \item The context is the singleton set with entry \( x: \alpha \).
    \end{itemize}

    We will not need to annotate terms with markers --- the variable plays the role of a marker for the type.

    \thmitem{def:type_derivation_tree/application} We define \term{rule application} trees based on a straightforward adaptation of \fullref{def:natural_deduction_proof_tree/application} for natural deduction rule applications, with schema instantiation based on \fullref{def:lambda_schema_instantiation}.

    Again, some assumptions of the subtrees can be \term{discharged} if the rule allows it. The important difference is that we have no markers, so there is a unique choice of type assertion that can be discharged.

    Of course, we must only allow rule applications whose context of open assumptions satisfies \fullref{def:type_context} --- each variable must have at most one assertion.
  \end{thmenum}
\end{definition}
\begin{comments}
  \item Rule application is implemented programmatically as \identifier{lambda_.type_derivation.tree.apply} in \cite{notebook:code}.
\end{comments}

\begin{definition}\label{def:simple_type_derivability}\mimprovised
  In a fixed \hyperref[def:simple_type_system]{simple type system} (specifying terms, types and rules), we define a \hyperref[def:consequence_relation]{consequence relation} on type assertions as follows: for a \hyperref[def:logical_context]{logical context} \( \Gamma \) and a \hyperref[def:type_assertion]{type assertion} \( M: \alpha \), we let \( \Gamma \vdash M: \alpha \) if there exists a \hyperref[def:type_derivation_tree]{type derivation tree} with conclusion \( M: \alpha \) whose \hyperref[def:natural_deduction_proof_tree/context]{open assumptions} are all in \( \Gamma \).

  As per \fullref{def:logical_framework}, we say that the type assertion \( M: \alpha \) is \term[en=derivable (\cite[def. 3.1.4]{Barendregt1992Types})]{derivable} from \( \Gamma \) if this relation holds.
\end{definition}
\begin{defproof}
  We can prove that \( {\vdash} \) is indeed a consequence relation analogously to our proof of consistency in \fullref{def:natural_deduction_entailment}.
\end{defproof}
\begin{comments}
  \item We introduce additional terminology in \fullref{def:typability} and \fullref{def:type_habitation}.

  \item Note that \( \Gamma \) by definition consists of type assertions for variables and not for more general \( \synlambda \)-terms. This is discussed in \fullref{rem:type_context_only_variables}.
\end{comments}

\begin{remark}\label{rem:type_context_only_variables}
  In \fullref{def:simple_type_derivability} we have only defined a \hyperref[def:consequence_relation]{consequence relation} for \hyperref[def:type_context]{type context} consisting of type assertions for variables, i.e. \( \Gamma \vdash M: \alpha \) can hold for an arbitrarily complex term \( M \), however the context \( \Gamma \) must have only variable type assertions. The types in both the context and the consequence are allowed to be arbitrarily complex.

  This is seemingly different from, for example, natural deduction proofs, where the context may consist of arbitrarily complex formulas. We would like to point out the following:
  \begin{itemize}
    \item Consistency conditions are simple to express for formulas, but in type theory they would depend circularly on typing derivations.

    \item While logic studies \hyperref[def:consequence_relation]{logical consequence} and \hyperref[con:hypothetical_judgment]{hypothetical judgments}, for type theory terms should ideally be typed unconditionally. Thus, while in logic \hyperref[def:propositional_tautology]{tautologies} with empty contexts are a curiosity, in type theory contexts are an instrument for dealing with \hyperref[def:lambda_variable_freeness]{free variables}.

    \item By the \hyperref[con:curry_howard_correspondence]{Curry-Howard correspondence}, formulas correspond to types, not to terms.
  \end{itemize}

  Our sources which define type contexts only allow variable type assertions in them ---
  \incite[19]{MartinLöf1984IntTypeTheory},
  \incite[def. 2A5]{Hindley1997STT},
  \incite[def. 5.1.5]{Barendregt1992Types} and
  \incite[\S 4.1]{Mimram2020Types}.
\end{remark}

\begin{proposition}\label{thm:def:simple_type_derivability}
  \hyperref[def:simple_type_derivability]{Type derivability} for \hyperref[def:simple_type_system]{simple type systems} only featuring the rules \ref{inf:def:base_typing_rules/abs} and \ref{inf:def:base_typing_rules/app} has the following basic properties:
  \begin{thmenum}
    \thmitem{thm:def:simple_type_derivability/free} If \( \Gamma, x: \alpha \vdash M: \beta \), then \( x \) is not free in \( M \) if and only if \( \Gamma \vdash M: \beta \).

    \thmitem{thm:def:simple_type_derivability/combinator} If \( {}\vdash M: \beta \), then \( M \) is a \hyperref[def:lambda_combinator]{combinator}.

    \thmitem{thm:def:simple_type_derivability/assumption_removal} We have \( \Gamma, x: \alpha \vdash M: \beta \) if and only if \( \Gamma \vdash \qabs x M: \alpha \to \beta \).
  \end{thmenum}
\end{proposition}
\begin{comments}
  \item Of course, systems may feature other rules, but we must explicitly prove the proposition for each of them.

  \item \Fullref{thm:def:simple_type_derivability/assumption_removal} demonstrates the \hyperref[con:inference_rule_admissibility]{admissible} of the \hyperref[rem:sequent_calculus]{sequent calculus} style variation of the rule \ref{inf:def:base_typing_rules/abs} described in \fullref{rem:typing_rule_style}.

  \item \Fullref{thm:def:simple_type_derivability/assumption_removal} shows that we can remove assumptions by prefixing a term with abstractions.
\end{comments}
\begin{proof}
  \SubProofOf{thm:def:simple_type_derivability/free} Let \( T \) be a tree deriving \( M: \beta \) from the type context \( \Gamma, x: \alpha \) (as usually, regarded as a set).

  We will use \fullref{thm:induction_on_rooted_trees} on \( T \):
  \begin{itemize}
    \item If \( T \) is an assumption tree, then \( M \) must be a variable, and thus \( x: \alpha \) is an open assumption in \( T \) if and only if \( M = x \).

    \item If \( T \) is an application of \ref{inf:def:base_typing_rules/app}, then \( M = NK \) and \( T \) has a subtree \( T_L \) deriving \( N: \gamma \to \beta \) for some type \( \gamma \), and a subtree \( T_R \) deriving \( K: \gamma \). We suppose that the inductive hypothesis holds for \( T_L \) and \( T_R \).

    \begin{itemize}
      \item If \( x \) is not free in \( M \), it is not free in \( N \) nor \( K \). Since nothing is discharged by this rule application, we have \( \Gamma, x: \alpha \vdash N \) and \( \Gamma, x: \alpha \vdash K \). The inductive hypothesis implies that \( x: \alpha \) is not an open assumption in \( T_L \) nor \( T_R \), and hence also not in \( T \). Hence, \( \Gamma \vdash M \).

      \item Conversely, if \( \Gamma \vdash M \), then \( \Gamma \vdash N \) and \( \Gamma \vdash K \), and the inductive hypothesis implies that \( x \) is not free in either \( N \) nor \( K \). Hence, \( x \) is not free in \( M \).
    \end{itemize}

    \item If \( T \) is an application of \ref{inf:def:base_typing_rules/abs}, \( M = \qabs y N \), \( \beta = \gamma \synimplies \delta \), and \( T \) has a subtree \( T' \) deriving \( N: \delta \) from \( \Gamma \setminus \set{ y: \gamma } \). Again, we suppose that the inductive hypothesis holds for \( T' \).

    \begin{itemize}
      \item If \( x = y \), uniqueness of the variables in the context implies that \( \alpha = \gamma \), and \( x: \alpha \) gets discharged, and hence \( \Gamma \vdash M \). Furthermore, in this case \( x \) is not free in \( M \).

      \item Otherwise, we have \( x \neq y \), and thus also \( \alpha \neq \gamma \).

      \begin{itemize}
        \item If \( x \) is not free in \( M \), it is also not free in \( N \), and the inductive hypothesis implies that \( \Gamma \vdash N: \delta \), from where it follows that
        \begin{equation*}
          \Gamma \vdash \underbrace{\qabs y N}_M: \underbrace{\gamma \synimplies \delta}_\beta.
        \end{equation*}

        \item Conversely, if \( \Gamma \vdash M: \beta \), then \( \Gamma, y: \gamma \vdash N: \delta \), and the inductive hypothesis implies that \( x \) is not free in \( N \). Then \( x \) is also not free in \( M \).
      \end{itemize}
    \end{itemize}
  \end{itemize}

  \SubProofOf{thm:def:simple_type_derivability/combinator} It follows from \fullref{thm:def:simple_type_derivability/free} that \( x: \alpha \) is an open assumption for a derivation of \( M: \beta \) if and only if \( x \) is not free in \( M \). If \( M \) is a combinator, it has no free variables, hence derivations of \( M \) have no open assumptions.

  \SubProofOf{thm:def:simple_type_derivability/assumption_removal}

  \SufficiencySubProof* Let \( P \) be a derivation tree for \( M: \beta \) from \( \Gamma, x: \alpha \). We will build a tree deriving \( \qabs x M: \alpha \to \beta \) from \( \Gamma \).

  \begin{itemize}
    \item If \( x: \alpha \) is an open assumption, it is discharged during a simple application of \ref{inf:def:base_typing_rules/abs}:
    \begin{equation*}
      \begin{prooftree}
        \hypo{ x: \alpha }
        \ellipsis { \( P \) } { M: \beta }
        \infer[left label=\( x \)]1[\ref{inf:def:base_typing_rules/abs}]{ \qabs x M: \alpha \synimplies \beta }
      \end{prooftree}
    \end{equation*}

    Discharging \( x: \alpha \) leaves us with \( \Gamma \vdash \qabs x M: \alpha \synimplies \beta \).

    \item Otherwise, if \( x: \alpha \) is not an open assumption in \( P \), then \( \Gamma \vdash P \) by \fullref{def:simple_type_derivability}. During application of \ref{inf:def:base_typing_rules/abs} nothing is discharged, and we again obtain \( \Gamma \vdash \qabs x M: \alpha \synimplies \beta \).
  \end{itemize}

  \NecessitySubProof* Conversely, if \( P \) is a derivation tree of \( \qabs x M: \alpha \to \beta \) from \( \Gamma \), it must be an application tree for \ref{inf:def:base_typing_rules/abs}, and hence \( x: \alpha \) is either discharged during the application, or it is not and the derivation of \( M: \beta \) does not depend on it. In both cases, we conclude that \( \Gamma, x: \alpha \vdash M \).
\end{proof}

\begin{definition}\label{def:typability}\mcite[def. 2D1]{Hindley1997STT}
  For a fixed \hyperref[def:abstract_type_system]{abstract type system}, we say that the term \( M \) is \term{typable} in the \hyperref[def:type_context]{type context} \( \Gamma \) if there exists a type \( \alpha \) such that \( \Gamma \vdash M: \alpha \).
\end{definition}
\begin{comments}
  \item We will often be interested in \hyperref[def:lambda_combinator]{combinators}, in which case, due to \fullref{thm:def:simple_type_derivability/combinator}, we may restrict ourselves to empty contexts: a combinator is typable if and only if there exists a type \( \alpha \) such that \( {}\vdash M: \alpha \).

  \item We generalize Hindley's definition for simple types to our notion of abstract type systems.
\end{comments}

\begin{definition}\label{def:type_habitation}\mcite[8A1]{Hindley1997STT}
  For a fixed \hyperref[def:abstract_type_system]{abstract type system}, if \( {}\vdash M: \alpha \), we say that \( M \) is an \term[en=untyped inhabitant (\cite[8A1]{Hindley1997STT})]{inhabitant} of \( \alpha \). If the type \( \alpha \) has at least one habitant, we say that it is \term{inhabited}.
\end{definition}
\begin{comments}
  \item We base our definition of habitant on the notion of \enquote{untyped inhabitants} for \hyperref[def:simple_type]{simple type} from \cite[def. 8A1]{Hindley1997STT}. Hindley also defines \enquote{typed habitants} for valid \hyperref[def:pseudoterm]{pseudoterms} and \enquote{\( \beta \)-normal habitants} for terms in \hyperref[def:lambda_reduction_normal_form]{\( \beta \)-normal form}.

  \item Unlike for \hyperref[def:typability]{typability}, this definition only applies to derivations from the empty context. This restricts us to inhabitants without free variables.
\end{comments}

\begin{example}\label{ex:def:type_derivation_tree}
  We list examples of \hyperref[def:type_derivation_tree]{type derivation trees} for \hyperref[def:simple_type_system]{simple type systems}:
  \begin{thmenum}
    \thmitem{ex:def:type_derivation_tree/var} Given a variable \( x \), it is typable in any \hyperref[def:type_context]{type context} containing \( x \). Due to \fullref{thm:def:simple_type_derivability/free}, it is not typable without a context unless we introduce a special rule for this.

    \thmitem{ex:def:type_derivation_tree/i} The combinator \( \ref{eq:ex:def:lambda_term/combinator/i} = \qabs \synx \synx \) inhabits \( \alpha \synimplies \alpha \) for any type \( \alpha \):
    \begin{equation*}
      \begin{prooftree}
        \hypo{ \synx: \alpha }
        \infer[left label=\( \synx \)]1[\ref{inf:def:base_typing_rules/abs}]{ \qabs \synx \synx: \alpha \synimplies \alpha }
      \end{prooftree}
    \end{equation*}

    \hyperref[rem:explicit_and_implicit_typing]{Explicit typing} can restrict each term to at most one type --- see \fullref{thm:pseudoterm_uniqueness}.

    \thmitem{ex:def:type_derivation_tree/k} The combinator \( \ref{eq:ex:def:lambda_term/combinator/k} = \qabs \synx \qabs \syny \synx \) inhabits \( \alpha \synimplies (\beta \synimplies \alpha) \) for any pair of types \( \alpha \) and \( \beta \):
    \begin{equation*}
      \begin{prooftree}
        \hypo{ \synx: \alpha }
        \infer1[\ref{inf:def:base_typing_rules/abs}]{ \qabs \syny \synx: \beta \synimplies \alpha }
        \infer[left label=\( \synx \)]1[\ref{inf:def:base_typing_rules/abs}]{ \qabs \synx \qabs \syny \synx: \alpha \synimplies (\beta \synimplies \alpha) }
      \end{prooftree}
    \end{equation*}

    During the first rule application, we do not discharge any assumptions since there are no compatible assumptions to discharge.

    Via the \hyperref[con:curry_howard_correspondence]{Curry-Howard correspondence}, this type corresponds to the \hyperref[def:minimal_implication_logic]{minimal implicational logic} axiom \eqref{eq:def:minimal_implication_logic/intro} --- see \fullref{ex:con:curry_howard_correspondence/minimal_implicational}.

    \thmitem{ex:def:type_derivation_tree/s} The combinator \( \ref{eq:ex:def:lambda_term/combinator/s} = \qabs \synx \qabs \syny \qabs \synz \synx \synz (\syny \synz) \) is a little more complicated:
    \begin{equation*}
      \begin{prooftree}
        \hypo{ \synx: \alpha \synimplies (\beta \synimplies \gamma) }
        \hypo{ \synz: \alpha }
        \infer2[\ref{inf:def:base_typing_rules/app}]{ \synx \synz: \beta \synimplies \gamma }

        \hypo{ \syny: \alpha \synimplies \beta }
        \hypo{ \synz: \alpha }
        \infer2[\ref{inf:def:base_typing_rules/app}]{ \syny \synz: \beta }

        \infer2[\ref{inf:def:base_typing_rules/app}]{ (\synx \synz) (\syny \synz): \gamma }
        \infer[left label=\( \synz \)]1[\ref{inf:def:base_typing_rules/abs}]{ \qabs \synz \synx \synz (\syny \synz): \alpha \synimplies \gamma }
        \infer[left label=\( \syny \)]1[\ref{inf:def:base_typing_rules/abs}]{ \qabs \syny \qabs \synz \synx \synz (\syny \synz): (\alpha \synimplies \beta) \synimplies (\alpha \synimplies \gamma) }
        \infer[left label=\( \synx \)]1[\ref{inf:def:base_typing_rules/abs}]{ S: (\alpha \synimplies (\beta \synimplies \gamma)) \synimplies ((\alpha \synimplies \beta) \synimplies (\alpha \synimplies \gamma)) }
      \end{prooftree}
    \end{equation*}

    Via the Curry-Howard correspondence, this type corresponds to the axiom \eqref{eq:def:minimal_implication_logic/dist} --- see \fullref{ex:con:curry_howard_correspondence/minimal_implicational}.

    \thmitem{ex:def:type_derivation_tree/pairs} Consider the \hyperref[def:ordered_tuple]{ordered pair} combinators \ref{ex:def:beta_eta_reduction/pairs/intro}, \ref{ex:def:beta_eta_reduction/pairs/elim_left} and \ref{ex:def:beta_eta_reduction/pairs/elim_right} from \fullref{ex:def:beta_eta_reduction/pairs}.

    We have shown that, for arbitrary terms \( A \) and \( B \), we have
    \begin{align*}
      P_{-L} (P_+ A B) \bred* A
      &&
      P_{-R} (P_+ A B) \bred* B.
    \end{align*}

    Similarly to the previous cases, we can find a type for
    \begin{equation*}
      \ref{ex:def:beta_eta_reduction/pairs/intro} = \qabs \synx \qabs \syny \qabs \synf \synf \synx \syny
    \end{equation*}
    by fixing types \( \alpha \), \( \beta \), \( \gamma \) and matching them to obtain the following derivation:
    \begin{equation*}
      \begin{prooftree}
        \hypo{ \synf: \alpha \synimplies (\beta \synimplies \gamma) }
        \hypo{ \synx: \alpha }
        \infer2[\ref{inf:def:base_typing_rules/app}]{ \synf \synx: \beta \synimplies \gamma }

        \hypo{ \syny: \beta }
        \infer2[\ref{inf:def:base_typing_rules/app}]{ \synf \synx \syny: \gamma }
        \infer[left label=\( \synf \)]1[\ref{inf:def:base_typing_rules/abs}]{ \qabs \synf \synf \synx \syny: (\alpha \synimplies (\beta \synimplies \gamma)) \synimplies \gamma }
        \infer[left label=\( \syny \)]1[\ref{inf:def:base_typing_rules/abs}]{ \qabs \syny \qabs \synf \synf \synx \syny: \beta \synimplies ((\alpha \synimplies (\beta \synimplies \gamma)) \synimplies \gamma) }
        \infer[left label=\( \synx \)]1[\ref{inf:def:base_typing_rules/abs}]{ P_+: \alpha \synimplies (\beta \synimplies ((\alpha \synimplies (\beta \synimplies \gamma)) \synimplies \gamma)) }
      \end{prooftree}
    \end{equation*}

    If \( A: \alpha \) and \( B: \beta \), after two applications of \ref{inf:def:base_typing_rules/app} we obtain
    \begin{equation*}
      P_+ A B: (\alpha \synimplies (\beta \synimplies \gamma)) \synimplies \gamma.
    \end{equation*}

    Similarly, for
    \begin{equation*}
      \ref{ex:def:beta_eta_reduction/pairs/elim_left} = \qabs \synp \synp (\qabs \synx \qabs \syny \synx)
    \end{equation*}
    we have
    \begin{equation*}
      \begin{prooftree}
        \hypo{ \synp: (\alpha \synimplies (\beta \synimplies \alpha)) \synimplies \gamma }

        \hypo{ \synx: \alpha }
        \infer1[\ref{inf:def:base_typing_rules/abs}]{ \qabs \syny \synx: \beta \synimplies \alpha }
        \infer[left label=\( \synx \)]1[\ref{inf:def:base_typing_rules/abs}]{ \qabs \synx \qabs \syny \synx: \alpha \synimplies (\beta \synimplies \alpha) }

        \infer2[\ref{inf:def:base_typing_rules/app}]{ \synp (\qabs \synx \qabs \syny \synx): \gamma }
        \infer[left label=\( \synp \)]1[\ref{inf:def:base_typing_rules/abs}]{ P_{-L}: ((\alpha \synimplies (\beta \synimplies \alpha)) \synimplies \gamma) \synimplies \gamma }
      \end{prooftree}
    \end{equation*}

    We have derived types for both \( P_+ A B \) and \( P_{-L} \), and can conclude that the rule \ref{inf:def:base_typing_rules/app} is only applicable to \( P_{-L} (P_+ A B) \) if \( \gamma = \alpha \), in which case \( P_{-L} (P_+ A B): \alpha \), as expected.

    On the other hand, for
    \begin{equation*}
      \ref{ex:def:beta_eta_reduction/pairs/elim_right} = \qabs \synp \synp (\qabs \synx \qabs \syny \syny)
    \end{equation*}
    we can similarly derive
    \begin{equation*}
      \begin{prooftree}
        \hypo{ \synp: (\alpha \synimplies (\beta \synimplies \beta)) \synimplies \gamma }

        \hypo{ \syny: \beta }
        \infer[left label=\( \syny \)]1[\ref{inf:def:base_typing_rules/abs}]{ \qabs \syny \syny: \beta \synimplies \beta }
        \infer1[\ref{inf:def:base_typing_rules/abs}]{ \qabs \synx \qabs \syny \syny: \alpha \synimplies (\beta \synimplies \beta) }

        \infer2[\ref{inf:def:base_typing_rules/app}]{ \synp (\qabs \synx \qabs \syny \syny): \gamma }
        \infer[left label=\( \synp \)]1[\ref{inf:def:base_typing_rules/abs}]{ P_{-R}: ((\alpha \synimplies (\beta \synimplies \beta)) \synimplies \gamma) \synimplies \gamma }
      \end{prooftree}
    \end{equation*}

    The rule \ref{inf:def:base_typing_rules/app} is only applicable to \( P_{-R} (P_+ A B) \) if \( \gamma = \beta \), in which case \( P_{-R} (P_+ A B): \beta \).

    We see here how the ability to derive multiple types for \( P_+ \) is actually beneficial --- we can use the same combinator in different situations.

    On the other hand, if \( P_+ \) has a unique type, perhaps hinted by a \hyperref[con:type_annotation]{type annotation}, then either \( P_{-L} (P_+ A B) \) or \( P_{-R} (P_+ A B) \) is \hyperref[def:typability]{untypable} in general (unless \( \alpha = \beta \)). This is discussed in \cite{MathOF:product_type_in_simply_typed_lambda_terms}, where it is also proven that no other such triple of combinators exist.

    \thmitem{ex:def:type_derivation_tree/disjoint} Consider the combinators \ref{eq:ex:def:beta_eta_reduction/disjoint/intro_left}, \ref{eq:ex:def:beta_eta_reduction/disjoint/intro_right} and \ref{eq:ex:def:beta_eta_reduction/disjoint/elim} from \fullref{ex:def:beta_eta_reduction/disjoint}.

    Since \( S_- \) and \( P_+ \) are \hyperref[def:lambda_term_alpha_equivalence]{\( \alpha \)-equivalent}, we conclude that they can be assigned the same types\fnote{This will be proven formally in \fullref{thm:alpha_equivalent_term_typing}.}.

    On the other hand, for
    \begin{equation*}
      S_{+L} = \qabs \synx \qabs \synf \qabs \syng \synf \synx
    \end{equation*}
    we have
    \begin{equation*}
      \begin{prooftree}
        \hypo{ \synf: \alpha \synimplies \gamma }
        \hypo{ \synx: \alpha }
        \infer2[\ref{inf:def:base_typing_rules/app}]{ \synf \synx: \gamma }
        \infer1[\ref{inf:def:base_typing_rules/abs}]{ \qabs \syng \synf \synx: (\beta \synimplies \gamma) \synimplies \gamma }
        \infer[left label=\( \synf \)]1[\ref{inf:def:base_typing_rules/abs}]{ \qabs \synf \qabs \syng \synf \synx: (\alpha \synimplies \beta) \synimplies ((\beta \synimplies \gamma) \synimplies \gamma) }
        \infer[left label=\( \synx \)]1[\ref{inf:def:base_typing_rules/abs}]{ S_{+L}: \alpha \synimplies ((\alpha \synimplies \beta) \synimplies ((\beta \synimplies \gamma) \synimplies \gamma)) }
      \end{prooftree}
    \end{equation*}

    Similarly, we can derive
    \begin{equation*}
      S_{+R}: \beta \synimplies ((\alpha \synimplies \beta) \synimplies ((\beta \synimplies \gamma) \synimplies \gamma)).
    \end{equation*}

    Suppose that \( A: \alpha \), \( F: \alpha \synimplies \gamma \) and \( G: \beta \synimplies \gamma \). Then, in order for \ref{inf:def:base_typing_rules/app} to be applicable to
    \begin{equation*}
      S_- F G (S_{+L} A),
    \end{equation*}
    the combinator \( S_- \) must inhabit
    \begin{equation*}
      (\alpha \synimplies \gamma) \synimplies ((\beta \synimplies \gamma) \synimplies (((\alpha \synimplies \beta) \synimplies ((\beta \synimplies \gamma) \synimplies \gamma)) \synimplies \delta))
    \end{equation*}
    for some type \( \delta \).

    According to \fullref{ex:def:type_derivation_tree/pairs}, such a derivation is possible if \( \delta = \gamma \). This type corresponds to Haskell's
    \begin{verbatim}
      either :: (a -> c) -> (b -> c) -> Either a b -> c,
    \end{verbatim}\vspace{-\baselineskip}
    as long as we identify \verb|Either a b| with
    \begin{equation*}
      (\alpha \synimplies \beta) \synimplies ((\beta \synimplies \gamma) \synimplies \gamma).
    \end{equation*}

    Hence, as expected
    \begin{equation*}
      S_- F G (S_{+L} A): \gamma
    \end{equation*}
    and
    \begin{equation*}
      S_- F G (S_{+R} B): \gamma
    \end{equation*}

    As long as \( F \) and \( G \) are fixed, we do not need different type derivations for handling \( S_{+L} A \) and \( S_{+R} B \).
  \end{thmenum}
\end{example}
\begin{comments}
  \item Many of these examples are verified programmatically in the module \identifier{lambda_.type_derivation.test_tree} in \cite{notebook:code}.
\end{comments}

\paragraph{Explicit typing}

\begin{concept}\label{con:type_annotation}
  The \hyperref[def:type_assertion]{type assertion} \( M: \alpha \) is a syntactic construction, and it is possible to assign different meanings to it. The obvious one is to perceive it as a \hyperref[rem:typing_judgments]{typing judgment} expressing that the term \( M \) has type \( \alpha \).

  Another possibility, used across programming languages, is merely to highlight intention. In this case we call \( \alpha \) a \term{type annotation} for \( M \) --- it is part of the language's syntax, and it merely expresses the intended type of \( M \). Whether or not \( M \) will have type \( \alpha \) depends on how the language is interpreted. We will make this precise in \fullref{def:pseudoterm_abstraction_rule} after introducing \hyperref[def:pseudoterm]{pseudoterms}.

  For convenience, we follow Howard's convention from \cite{Howard1980FormulasAsTypes} of denoting type annotations by superscripts, i.e. \( M^\alpha \) rather than \( M: \alpha \). Not only is it the notation more concise, it also allows us to distinguish annotations from assertions. Superscripts are also used in \cite{Mimram2020Types} and \cite[ch. 5]{Hindley1997STT}. On the other hand, \incite{Barendregt1992Types} uses colons for both annotations and assertions.

  We call \term[en=type-erasure (\cite[def. 5A2]{Hindley1997STT})]{type erasure} the procedure \( M^\alpha \mapsto M \) of removing the annotations of terms. If erasure is performed, verifying that \( M \) has type \( \alpha \) has to be done before that.
\end{concept}
\begin{comments}
  \item The phrase \enquote{type annotation} is used, for example, in TypeScript's handbook, \cite{TypeScript:5.7:handbook}, as well as in Python's documentation, \cite{PythonDocs:3.13:typing} (the latter also call annotations \enquote{type hints}).

  \incite[186]{Mimram2020Types} also uses \enquote{type annotation} when discussing Church style typing.

  \item This definition for type erasure is used by \bycite[def. 5A2]{Hindley1997STT} in the context of annotated \( \synlambda \)-terms.
\end{comments}

\begin{definition}\label{def:pseudoterm}\mimprovised
  We consider the following extension of the grammar schema of untyped \( \synlambda \)-term from \fullref{def:lambda_term} in which \hyperref[def:lambda_term/const]{constants} and \hyperref[def:lambda_abstractor]{abstractor variables} have \hyperref[con:type_annotation]{type annotations} attached. We call them \term[en=pseudoterm (\cite[def. 3.2.1]{Barendregt1992Types})]{pseudoterms}.

  Note that annotated variables are not by themselves considered pseudoterms --- variable annotations are only used in the abstractor.

  More precisely, we have the following grammar based on that of untyped \( \synlambda \)-terms from \fullref{def:lambda_term}:
  \begin{bnf*}
    \bnfprod{annotated variable}     {\bnfpn{variable} \bnfsp \bnftsq{:} \bnfsp \bnfpn{type}}, \\
    \bnfprod{annotated constant}     {\bnfpn{constant} \bnfsp \bnftsq{:} \bnfsp \bnfpn{type}}, \\
    \bnfprod{application pseudoterm} {\bnftsq{\( ( \)} \bnfsp \bnfpn{pseudoterm} \bnfsp \bnfpn{pseudoterm} \bnfsp \bnftsq{\( ) \)}}, \\
    \bnfprod{abstraction pseudoterm} {\bnftsq{\( ( \)} \bnfsp \bnftsq{\( \synlambda \)} \bnfsp \bnfpn{annotated variable} \bnfsp \bnftsq{.} \bnfsp \bnfpn{pseudoterm} \bnfsp \bnftsq{\( ) \)}}, \\
    \bnfprod{pseudoterm}             {\bnfpn{annotated constant} \bnfor \bnfpn{variable} \bnfor} \\
    \bnfmore                         {\bnfpn{application pseudoterm} \bnfor \bnfpn{abstraction pseudoterm}.}
  \end{bnf*}

  Following \fullref{con:type_annotation}, in the \hyperref[con:metalanguage]{metalanguage}, we will use the notation \( \qabs {x^\alpha} M \) rather than \( \qabs {x: \alpha} M \).
\end{definition}
\begin{comments}
  \item This definition is dependent on our choice for the syntax of types; unless noted otherwise, we suppose we are dealing with \hyperref[def:simple_type]{simple types}.

  \item We base our definition on \bycite[def. 3.2.1]{Barendregt1992Types}, but also adapt it to constants and make the syntax explicit.

  We can see in \fullref{def:pseudoterm_typing_rules} how type annotations help guide us.
\end{comments}

\begin{definition}\label{def:pseudoterm_abstraction_rule}\mcite[def. 3.2.3]{Barendregt1992Types}
  The typing rule \ref{inf:def:base_typing_rules/app} adapts straightforwardly for \hyperref[def:pseudoterm]{pseudoterms}, however \ref{inf:def:base_typing_rules/abs} must be adapted to only apply when \( \synx \) is annotated correctly:
  \begin{equation*}\taglabel[\ensuremath{ \logic{PsAbs}_\tau }]{inf:def:pseudoterm_abstraction_rule}
    \begin{prooftree}
      \hypo{ [\synx: \syn\alpha] }
      \ellipsis {} { \synM: \syn\beta }
      \infer1[\ref{inf:def:pseudoterm_abstraction_rule}]{ \qabs {\synx^{\syn\alpha}} \synM: \syn\alpha \synimplies \syn\beta }
    \end{prooftree}
  \end{equation*}
\end{definition}
\begin{comments}
  \item Of course, actually using this rule also requires introducing pseudoterm schemas. Fortunately, the formalisms are straightforward.
\end{comments}

\begin{proposition}\label{thm:pseudoterm_uniqueness}
  We can \hyperref[def:simple_type_derivability]{derive} at most one \hyperref[def:simple_type]{simple type} for any \hyperref[def:pseudoterm]{pseudoterm} from a fixed \hyperref[def:type_context]{type context}.
\end{proposition}
\begin{comments}
  \item In particular, a pseudoterm combinator can \hyperref[def:type_habitation]{inhabit} at most one \hyperref[def:simple_type]{simple type}.

  \item We can also interpret this result as a statement for untyped \( \synlambda \)-terms --- each of them can be typed uniquely \enquote{up to a choice of types for bound variables}.

  \item This result depends on the typing rules; we will prove it assuming \ref{inf:def:base_typing_rules/app} and \ref{inf:def:pseudoterm_abstraction_rule}. This also excludes constants since special rules must be introduced for them.
\end{comments}
\begin{proof}
  Let \( T \) be a tree deriving \( M: \alpha \) from the type context \( \Gamma \).

  We will use \fullref{thm:induction_on_rooted_trees} on \( T \) to show that if \( T' \) derives \( M: \alpha' \) from \( \Gamma \), the trees are equal and \( \alpha = \alpha' \):
  \begin{itemize}
    \item If \( T \) is an assumption tree, then \( M \) must be a variable, and \( \Gamma \vdash M \) only if \( \Gamma \) contains a type assertion for \( M \). Since type contexts have at most one type assertion per variable, it follows that the type \( \alpha \) of \( M \) is uniquely determined.

    \item If \( T \) is an application of \ref{inf:def:base_typing_rules/app}, then \( M = NK \) and \( T \) has a subtree \( T_L \) deriving \( N: \beta \to \alpha \) for some type \( \beta \), and a subtree \( T_R \) deriving \( K: \beta \). We suppose that the inductive hypothesis holds for \( T_L \) and \( T_R \).

    Let \( T' \) be a tree deriving \( M: \alpha' \) from \( \Gamma \). Since \( M = NK \), \( T' \) must be an application tree of \ref{inf:def:base_typing_rules/app}, with subtrees \( T_L' \) and \( T_R' \) deriving \( N: \beta' \to \alpha' \) and \( K: \beta' \) from the assumptions in \( \Gamma \).

    The inductive hypothesis on \( T_L \) implies that the types \( \beta' \to \alpha' \) and \( \beta \to \alpha \) are equal, and thus so are \( \alpha' \) and \( \alpha \).

    \item If \( T \) is an application of \ref{inf:def:pseudoterm_abstraction_rule}, then \( \alpha = \beta \synimplies \gamma \), \( M = \qabs {x^\beta} N \) and \( T \) has a subtree \( S \) deriving \( N: \gamma \) from \( \Gamma, x: \beta \).

    Let \( T' \) be a tree deriving \( M: \alpha' \) from \( \Gamma \). Then \( \alpha' = \beta \synimplies \gamma' \) \( T' \) is an application of \ref{inf:def:pseudoterm_abstraction_rule}, and we have a subtree \( S' \) deriving \( N: \gamma' \) from \( \Gamma, x: \beta \).

    The inductive hypothesis implies that \( S \) and \( S' \) are equal, and that \( \gamma = \gamma' \). Then also \( \alpha = \alpha' \).
  \end{itemize}
\end{proof}

\begin{remark}\label{rem:explicit_and_implicit_typing}
  The rule \ref{inf:def:pseudoterm_abstraction_rule} demonstrates how \hyperref[con:type_annotation]{type annotations} help guide us during type derivation.

  \Fullref{thm:pseudoterm_uniqueness} shows how this tackles the non-uniqueness of type derivations for \hyperref[def:lambda_combinator]{combinators} shown in \fullref{ex:def:type_derivation_tree/i}. For example, the combinator \( \ref{eq:ex:def:lambda_term/combinator/i} = \qabs \synx \synx \) inhabits \( \alpha \synimplies \alpha \) for any type \( \alpha \), however, when \( \alpha \) is fixed, \ref{inf:def:pseudoterm_abstraction_rule} uniquely determines \hi{the} type of \( I_\alpha \coloneqq \qabs {\synx^\alpha} \synx \) as \( \alpha \synimplies \alpha \) at the cost of introducing a distinct abstraction term for every possible annotation of the abstractor variable.

  Another possibility is to annotate every term like Church originally did in \cite{Church1940STT}. In this case type contexts become obsolete. The notation becomes much more verbose, however, and, in light of \fullref{thm:pseudoterm_uniqueness}, there is little gain. Compare the following typed variations of the combinator \ref{eq:ex:def:lambda_term/combinator/k}:
  \begin{equation*}
    \qabs {\synx^\alpha} \qabs {\syny^\beta} \synx
  \end{equation*}
  and
  \begin{equation*}
    (\qabs {\synx^\alpha} (\qabs {\syny^\beta} \synx^\alpha)^{\beta \synimplies \alpha})^{\alpha \synimplies (\beta \synimplies \alpha)}.
  \end{equation*}

  The second convention is referred to as \enquote{Church style typing}, for example by \cite[ch. 5]{Hindley1997STT}. Even though it is not as historically accurate, the convention of using pseudoterms is also referred to as \enquote{Church style}, for example by \incite[ch. 4]{Mimram2020Types}.

  As a contradistinction, type systems for untyped \( \synlambda \)-terms are referred to by both the aforementioned authors as \enquote{Curry style} in honor of Haskell Curry for his work \cite{Curry1934CombinatoryLogic}. Curry did not consider \( \synlambda \)-terms nor types, but worked with abstract combinators applied to each other in what he called \enquote{combinatory logic}. We can deduce a relation to typing systems from the following quote:
  \begin{displayquote}
    Let \( F \) denote an entity whose interpretation is such that \( FXY \) represents, in the notation of combinatory logic, the category of functions on \( X \) to \( Y \), while the formula \( \vdash FXYf \) represents the statement that \( f \) belongs to that category.
  \end{displayquote}

  Here \( f \) is merely a symbol, while \( \vdash FXYf \) translates into the type assertion \( f: X \synimplies Y \). Since the type \( X \synimplies Y \) is not attached to \( f \), the latter is also allowed to inhabit another type, which contrasts with Church's usage mentioned earlier.

  Ironically, the programming language Haskell, also named after Haskell Curry, requires type annotations. Unlike a true Church style type system, however, Haskell's allows not annotating terms for which the types can be inferred.

  We will refer to the two aforementioned extremes as explicit and implicit typing, correspondingly --- see \fullref{con:explicit_and_implicit_typing}.
\end{remark}

\begin{concept}\label{con:explicit_and_implicit_typing}\mcite[119]{Barendregt1992Types}
  We say that an \hyperref[def:abstract_type_system]{(abstract) type system} uses \term{explicit typing} if its terms have \hyperref[con:type_annotation]{type annotations} as part of their syntax --- perhaps only for some of the constructions, like in the case of \hyperref[def:pseudoterm]{pseudoterms}.

  If the terms have no type annotations, we say that the system uses \term{implicit typing}. If the type annotations are optional, we speak of \term{gradual typing}.
\end{concept}
\begin{comments}
  \item The phrases \enquote{explicit typing} and \enquote{implicit typing} are used by \incite[119]{Barendregt1992Types} as alternatives for what we described in \fullref{rem:explicit_and_implicit_typing} as Church and Curry style typing, correspondingly.

  \item Some programming languages allow but do not enforce type annotations. One such language is Python. Another one is TypeScript, which adds optional type annotations to JavaScript. TypeScript's documentation, \cite{TypeScript:5.7:handbook}, describes this approach as \enquote{gradual typing}.
\end{comments}
