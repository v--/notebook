\section{Abstract dynamical systems}\label{sec:abstract_dynamical_systems}

\paragraph{Classification of dynamical systems}

\begin{concept}\label{con:discrete_and_continuous_variables}
  A fundamental contradistinction in mathematics is between \term[ru=дискретная (величина) (\cite[16]{АлександровМаркушевичХинчинИПр1963ЭнциклопедияТом4}), en=discrete (quantity) ({\cite[36]{Aristotle1984Categories})}]{discrete} and \term[ru=непрерывная (величина) (\cite[16]{АлександровМаркушевичХинчинИПр1963ЭнциклопедияТом4}), en=continuous ({\cite[36]{Aristotle1984Categories}})]{continuous} \hyperref[con:variable]{variables}. We will try to make these notions precise analyzing the ideas of Aristotle, to whom they are attributed. For disambiguation, then, we will use the terms \term{Aristotelian discrete} and \term{Aristotelian continuous}.

  Several areas rely on this dichotomy, like time in \hyperref[def:dynamical_system]{dynamical systems} and \hyperref[def:stochastic_process]{stochastic processes}, or the classification of \hyperref[def:random_variable]{random variables}. The commonality in these cases is that the variables called \enquote{discrete} range over \hyperref[def:discrete_set]{discrete subsets} of the \hyperref[def:euclidean_plane]{Euclidean space} \( \BbbR^n \) such as \hyperref[def:point_lattice]{point lattices}, while variables called \enquote{continuous} range over connected subsets of \( \BbbR^n \). For \( n = 1 \), these reduce to variables on integers and variables on intervals.

  Boris Rosenfeld in \cite[17]{АлександровМаркушевичХинчинИПр1963ЭнциклопедияТом4} attributes the concept of continuity to Aristotle. In \cite[36]{Aristotle1984Categories}, an English translation of Aristotle's works, we can find the following:
  \begin{displayquote}
    Of quantities some are discrete, others continuous; \textellipsis

    Discrete are number and language; continuous are lines, surfaces, bodies, and also, besides these, time and place. For the parts of a number have no common boundary at which they join together. For example, if five is a part of ten the two fives do not join together at any common boundary but are separate; nor do the three and the seven join together at any common boundary. Nor could you ever in the case of a number find a common boundary of its parts, but they are always separate. Hence number is one of the discrete quantities. Similarly, language also is one of the discrete quantities (that language is a quantity is evident, since it is measured by long and short syllables; I mean here language that is spoken). For its parts do not join together at any common boundary. For there is no common boundary at which the syllables join together, but each is separate in itself. A line, on the other hand, is a continuous quantity. For it is possible to find a common boundary at which its parts join together, a point. And for a surface, a line; for the parts of a place join together at some common boundary. Similarly in the case of a body one could find a common boundary --- a line or a surface --- at which the parts of the body join together. Time also and place are of this kind. For present time joins on to both past time and future time. Place, again, is one of the continuous quantities. For the parts of a body occupy some place, and they join together at a common boundary. So the parts of this place occupied by the various parts of the body, themselves join together at the same boundary at which the parts of the body do. Thus place also is a continuous quantity, since its parts join together at one common boundary.
  \end{displayquote}

  Aristotle's definition of \enquote{discrete quantity} can be recast into the language of topology by regarding topological spaces as sets of possible values for \enquote{quantities}. Consider a \hyperref[def:topological_space]{topological space} \( X \). Under Aristotle's definition, \( X \) should be discrete if, given any \hyperref[def:set_partition]{partition} \( X = A \cup B \), the set \( A \) and \( B \) have no common \hyperref[def:topological_boundary_operator]{topological boundary}.

  Note that \( B \) is uniquely determined as the complement of \( A \), and that, due to \cref{thm:def:topological_boundary_operator/complement}, the boundaries of \( A \) and \( B \) coincide. Then Aristotle's requirement can be simplified to every subset \( A \) having an empty boundary. The boundary of \( A \) is empty if its \hyperref[def:topological_closure_operator]{closure} and \hyperref[def:topological_interior_operator]{interior} coincide. This happens if \( A \) is both the smallest closed subset containing itself and the smallest open set containing itself, i.e. if it is both open and closed.

  Thus, Aristotle's notion of \enquote{discrete quantity} directly corresponds to a discrete topological space as defined in \cref{def:discrete_topology}, in which every subset is open or, equivalently, every subset is closed.

  Another characterization of discrete spaces, perhaps more enlightening, is that all its points are \hyperref[def:set_cluster_point]{isolated}, as shown in \cref{thm:def:set_cluster_point/discrete}.

  On the other end of the spectrum must then be \enquote{continuous quantities}, in which the common boundary is always nonempty. These quantities must then correspond to \hyperref[def:connected_space]{connected spaces}, which satisfy \cref{def:connected_space/boundary} --- the boundary of every nonempty proper subset must be nonempty.
\end{concept}
\begin{comments}
  \item Both Aristotelian notions of quantity appear unrelated to the continuity of a function in a topological space, which we discuss in \fullref{sec:topological_continuity}. Thus, while we were able to formalize discrete and continuous quantities, a lot of the dichotomy that has developed since Aristotle relates to his ideas only tangentially.

  \item Some notions like \hyperref[def:discrete_category]{discrete categories} that seem related are derived from \hyperref[def:discrete_topology]{discrete topologies} rather than discrete variables.
\end{comments}

\begin{remark}\label{rem:discrete_mathematics}
  Due to the historical development of the notions of discrete and continuous variables, which we discuss in \cref{con:discrete_and_continuous_variables}, the dichotomy has created a vague division of mathematical areas based on the kind of variables studied.

  \incite[\S III.6]{Kuyk1977ComplementarityInMathematics}, in a section called \enquote{Bridging the Abyss Between the Discrete and the Continuous}, splits a small list of mathematical disciplines into \enquote{Discrete (Algebra)} and \enquote{Continuous (Topology)}, with set theory and category theory kept separate. To this, \incite[xx]{Johnstone1982StoneSpaces} remarks
  \begin{displayquote}
    In particular, it should be clear that any attempt, as in [Kuyk 1977], to draw a distinction between \enquote*{discrete} and \enquote*{continuous} mathematics belittles the subject: abstract algebra cannot develop to its fullest extent without the infusion of topological ideas, and conversely if we do not recognize the algebraic aspects of the fundamental structures of analysis our view of them will be one-sided.
  \end{displayquote}

  Nevertheless, unlike \enquote{continuous mathematics}, \enquote{discrete mathematics} has established itself as a conglomerate of \fullref{ch:mathematical_logic}, \fullref{ch:graph_theory}, \fullref{ch:combinatorics} and aspects of other areas such as \fullref{ch:group_theory} and \fullref{ch:order_theory}.

  Some major journals are explicitly dedicated to discrete mathematics, like \enquote{Discrete Mathematics} with eISSN \identifier{1872-681X} and \enquote{Дискретная математика} with eISSN \identifier{2305-3143}.

  Furthermore, we use as references several books explicitly dedicated to the subject --- \bycite{Rosen2019DiscreteMathematics}, \bycite{RosenEtAl2018DiscreteMathematicsHandbook}, \bycite{БелоусовТкачёв2004ДискретнаяМатематика}, \bycite{Новиков2013ДискретнаяМатематика} and \bycite{Яблонский2003ДискретнаяМатематика}.
\end{remark}

\begin{remark}\label{rem:dynamical_system_time_etymology}
  Dynamical systems are rarely presented in a uniform manner; instead, different types of dynamical systems are defined and studied separately.

  One major distinction is time. Fix an arbitrary \hyperref[def:monoid]{monoid} \( T \). We define a general dynamical system in \cref{def:dynamical_system} as an \hyperref[def:monoid_action]{action} \( \Phi: T \times X \to X \). Based on \( T \), we define discrete-time systems in \cref{def:discrete_dynamical_system} and continuous-time systems in \cref{def:continuous_dynamical_system}. Examples and counterexamples of the notions we will introduce can be found in \cref{ex:def:dynamical_system}.

  Here we present how some other authors handle this:
  \begin{itemize}
    \item \incite[362]{Rozikov2012MultiDimensionalTimeDynamicalSystem}, upon which definition we base ours, uses \enquote{real dynamical system} or \enquote{flow} if \( T \) is an \hyperref[def:order_interval/open]{open interval} of real numbers, and \enquote{discrete dynamical system} or \enquote{cascade} if \( T \) is the additive group of integers.

    If negative time is disallowed, he uses \enquote{semi-flow} and \enquote{semi-cascade} instead.

    In the general definition Rozikov allows \( X \) to be any set, but for flows and cascades he requires it to be a manifold. We find this requirement unnecessarily restrictive.

    \item \incite[def. 1]{GiuntiMazzola2012DynamicalSystemsOnMonoids} provide a similar general definition as a monoid action, and avoid discussing discrete and continuous systems.

    \item \incite[5]{Юмагулов2015ДинамическиеСистемы} defines a dynamical system as a phase space with a partial evolution function on nonnegative real numbers.

    If the evolution function is defined for all real numbers, he calls the system \enquote{непрерывная} (\enquote{continuous}). If it is defined in at most countably many real numbers, he calls the system \enquote{дискретная} (\enquote{discrete}). He highlights that the moments of a discrete dynamical system are immaterial and can, without loss of generality, be taken to be an initial segment of the natural numbers.

    \item \incite[1]{HasselblattKatok1995DynamicalSystems} give an informal definition where a dynamical system should consist of a \enquote{phase space}, \enquote{time} and a \enquote{time-evolution law}. They assume that time is a real number further classify it into \enquote{discrete} and \enquote{continuous} based on whether non-integers are considered, as well as \enquote{reversible} and \enquote{nonreversible} based on whether negative time is allowed.

    \item \incite[2]{BrinStuck2002DynamicalSystems} do not give a common definition. Instead, they define a \enquote{discrete-time dynamical system} via its \hyperref[def:one_step_evolution_function]{one-step evolution function}, and give a distinct definition for a \enquote{continuous-time dynamical system} as a family of functions indexed by real numbers satisfying \ref{eq:def:monoid_action/family/identity} and \ref{eq:def:monoid_action/family/compatibility}. They also use \enquote{flow} or \enquote{semiflow} for continuous-time systems.

    \item \incite[18]{Müller2022HandbookOfDynamicsAndProbability} generalizes dynamical systems so that they are allowed to depend on a starting time; see \cref{def:non_autonomous_dynamical_system}.
  \end{itemize}
\end{remark}

\begin{definition}\label{def:dynamical_system}\mimprovised
  Let \( T \) be an \hyperref[con:additive_semigroup]{additive} \hyperref[def:monoid]{monoid} and let \( X \) be a set or, more generally, an object in some \hyperref[def:concrete_category]{concrete category} \( \cat{C} \).

  We define a \term[ru=динамическая система (\cite[5]{Юмагулов2015ДинамическиеСистемы}), en=dynamical system (\cite[362]{Rozikov2012MultiDimensionalTimeDynamicalSystem})]{dynamical system} with \term[ru=фазовое пространство (\cite[5]{Юмагулов2015ДинамическиеСистемы}), en=phase space (\cite[362]{Rozikov2012MultiDimensionalTimeDynamicalSystem})]{phase space} \( X \) and \term[en=time (\cite[1]{HasselblattKatok1995DynamicalSystems})]{time} \( T \) as a \hyperref[def:monoid_action]{monoid action} \( \Phi: T \times X \to X \). We call the function \( \Phi \) an \term[ru=оператор эволюции (\cite[5]{Юмагулов2015ДинамическиеСистемы}), en=evolution function (\cite[362]{Rozikov2012MultiDimensionalTimeDynamicalSystem})]{evolution function}.

  We will refer to the elements of \( T \) as \term[ru=моменты (\cite[5]{Юмагулов2015ДинамическиеСистемы}), en=moments (\cite[2]{HasselblattKatok1995DynamicalSystems})]{moments} and to the elements of \( X \) as \term[ru=состояния (\cite[5]{Юмагулов2015ДинамическиеСистемы}), en=states (\cite[2]{HasselblattKatok1995DynamicalSystems})]{states} of the system. For this reason \( X \) can also be called the \term[ru=пространство состояний (\cite[5]{Юмагулов2015ДинамическиеСистемы}), en=state space (\cite[362]{Rozikov2012MultiDimensionalTimeDynamicalSystem})]{state space}.
\end{definition}
\begin{comments}
  \item We base our definition mostly on \bycite[362]{Rozikov2012MultiDimensionalTimeDynamicalSystem}; this and other general definitions reviewed in \cref{rem:dynamical_system_time_etymology}. We prefer to delegate technicalities to the definition \cref{def:monoid_action} of monoid actions. The flexibility of a using concrete categories is discussed in \cref{rem:group_action_category}. In particular, this makes it easy to define subtypes, like topological dynamical systems in \cref{def:topological_dynamical_system}, or linear dynamical systems in \cref{def:linear_dynamical_system}.

  \item Unlike the other concepts, time monoids seem to have no established terminology. \incite[def. 1]{GiuntiMazzola2012DynamicalSystemsOnMonoids} suggest \enquote{time model}, however we prefer \enquote{time monoid} or \enquote{time group}.
\end{comments}

\begin{definition}\label{def:non_autonomous_dynamical_system}\mimprovised
  We can generalize \cref{def:dynamical_system} so that the evolution function depends on the starting time. \incite[18]{Müller2022HandbookOfDynamicsAndProbability} suggests allowing time to consist of pairs \( (t_0, t) \), where \( t_0 \) is regarded as starting time and \( t \) as time passed since \( t_0 \). The evolution function \( \Phi: T^2 \times X \to X \) then needs to satisfy
  \begin{align}
    &\Phi_{(t, t)} = \id_X,                                            \label{eq:def:non_autonomous_dynamical_system/identity} \\
    &\Phi_{(t_3, t_1)} = \Phi_{(t_3, t_2)} \bincirc \Phi_{(t_2, t_1)}. \label{eq:def:non_autonomous_dynamical_system/compatibility}
  \end{align}

  We call this a \term{non-autonomous dynamical system} because it encompasses non-autonomous systems of differential equations, difference equations and recurrence relations.
\end{definition}
\begin{comments}
  \item \incite[18]{Müller2022HandbookOfDynamicsAndProbability} calls this a \enquote{dynamical system} and then calls \cref{def:dynamical_system} a \enquote{non-autonomous dynamical system}.
\end{comments}

\begin{definition}\label{def:reversible_dynamical_system}\mimprovised
  We say that a \hyperref[def:dynamical_system]{dynamical system} is \term[en=reversible (\cite[1]{HasselblattKatok1995DynamicalSystems})]{reversible} if its time monoid is a \hyperref[def:group]{group}. In this case the evolution function is invertible at every moment, and the system itself becomes a \hyperref[def:group_action]{group action}.

  Otherwise, we say that it is \term[en=irreversible (\cite[1]{HasselblattKatok1995DynamicalSystems})]{irreversible}.
\end{definition}
\begin{comments}
  \item We generalize the usage of \incite[1]{HasselblattKatok1995DynamicalSystems}, who call a dynamical system \enquote{reversible} if it includes negative integers in the discrete case or negative real numbers on the continuous case. See \cref{rem:dynamical_system_time_etymology} for a broader discussion of terminology.
\end{comments}

\begin{definition}\label{def:discrete_dynamical_system}\mimprovised
  We say that a \hyperref[def:dynamical_system]{dynamical system} has \term[ru=(динамическая система с) дискретным временем (\cite[10]{Юмагулов2015ДинамическиеСистемы}), en=discrete-time (dynamical system) (\cite[1]{Rozikov2012MultiDimensionalTimeDynamicalSystem})]{discrete time} if its time monoid is either the additive monoid of \hyperref[def:natural_numbers]{natural numbers} for \hyperref[def:reversible_dynamical_system]{irreversible systems} or the additive group of \hyperref[def:integers]{integers} for reversible ones. These systems can be described by the iteration of a single function --- see \cref{def:one_step_evolution_function}.

  The term \enquote{discrete-time} helps disambiguate cases where it is not clear whether the phase space is discrete or continuous in the sense of \cref{con:discrete_and_continuous_variables}. If this ambiguity is avoided, we can refer to the system itself as a \term[ru=дискретная динамическая система (\cite[10]{Юмагулов2015ДинамическиеСистемы}), en=discrete dynamical system (\cite[363]{Rozikov2012MultiDimensionalTimeDynamicalSystem})]{discrete dynamical system}.

  A reversible discrete-time system is also called a \term[ru=каскада (\cite[10]{Юмагулов2015ДинамическиеСистемы}), en=cascade (\cite[363]{Rozikov2012MultiDimensionalTimeDynamicalSystem})]{cascade}, while an irreversible system is called a \term[en=semi-cascade (\cite[363]{Rozikov2012MultiDimensionalTimeDynamicalSystem})]{semi-cascade}.
\end{definition}
\begin{comments}
  \item It follows from \cref{thm:order_topology_on_integers_is_discrete} that the integers with the \hyperref[def:order_topology]{order topology} form a \hyperref[def:discrete_topology]{discrete topological space}. Although it seems plausible to allow any discrete time monoid for a \enquote{discrete-time system}, only the natural numbers and integers are usually considered. It is also important that time is a \hyperref[def:cyclic_monoid]{cyclic monoid}; see \cref{def:one_step_evolution_function}.

  \item Our definition is based on \incite{Rozikov2012MultiDimensionalTimeDynamicalSystem}, but, unlike him, we do not require \( X \) to be a manifold. See \cref{rem:dynamical_system_time_etymology} for a broader discussion of terminology.
\end{comments}

\begin{definition}\label{def:continuous_dynamical_system}\mimprovised
  We say that a \hyperref[def:dynamical_system]{dynamical system} has \term[ru=(динамическая система с) непрерывным временем (\cite[8]{Юмагулов2015ДинамическиеСистемы}), en=continuous-time (dynamical system) (\cite[2]{BrinStuck2002DynamicalSystems})]{continuous time} if its time monoid is the additive group of \hyperref[def:real_numbers]{real numbers} or, if the system is \hyperref[def:reversible_dynamical_system]{irreversible}, the submonoid of nonnegative real numbers.

  In with discrete-time systems, in case no ambiguity with the other notions of continuity are possible, we can refer to the system itself as a \term[ru=непрарывная динамическая система (\cite[8]{Юмагулов2015ДинамическиеСистемы}), en=continuous dynamical system (\cite[21]{Müller2022HandbookOfDynamicsAndProbability})]{continuous dynamical system}.

  A reversible continuous-time system is also called a \term[ru=поток (\cite[10]{Юмагулов2015ДинамическиеСистемы}), en=flow (\cite[2]{BrinStuck2002DynamicalSystems})]{flow}, while an irreversible system is called a \term[en=semiflow (\cite[2]{BrinStuck2002DynamicalSystems}) / semi-flow (\cite[363]{Rozikov2012MultiDimensionalTimeDynamicalSystem})]{semi-flow}.
\end{definition}
\begin{comments}
  \item \incite[363]{Rozikov2012MultiDimensionalTimeDynamicalSystem} suggests using \enquote{real-time} instead of \enquote{continuous-time}. See \cref{rem:dynamical_system_time_etymology} for a broader discussion of terminology.
\end{comments}

\begin{proposition}\label{thm:def:dynamical_system}
  A \hyperref[def:dynamical_system]{dynamical system} with evolution function \( \Phi: T \times X \to X \) has the following basic properties:
  \begin{thmenum}
    \thmitem{thm:def:dynamical_system/negation} If the system is reversible, for any moment \( t \), we have
    \begin{equation}\label{eq:thm:def:dynamical_system/negation}
      \Phi_{-t} = (\Phi_t)^{-1}.
    \end{equation}

    \thmitem{thm:def:dynamical_system/iterated} For any moment \( t \) and any integer \( n \), the map \( \Phi_{nt} \) equals the iterated composition \( \Phi_t^n \). In case the system is irreversible, this only holds for nonnegative \( n \).
  \end{thmenum}
\end{proposition}
\begin{proof}
  \SubProofOf{thm:def:dynamical_system/negation} Follows from \cref{thm:def:group_action/negation}.

  \SubProofOf{thm:def:dynamical_system/iterated} Follows from \cref{thm:def:group_action/iterated}.
\end{proof}

\begin{definition}\label{def:one_step_evolution_function}\mimprovised
  Let \( \Phi: T \times X \to X \) be the evolution function of a \hyperref[def:discrete_dynamical_system]{discrete-time dynamical system}. Denote the map \( \Phi_1: X \to X \) by \( \varphi \).

  \Cref{thm:def:dynamical_system/iterated} implies that, for every moment \( n \), \( \Phi_n \) is simply the \( n \)-th composition of \( \varphi \) with itself. Thus, \( \varphi \) uniquely determines the entirety of \( \Phi \).

  We call \( \varphi \) the \term{one-step evolution function}.
\end{definition}
\begin{comments}
  \item Given this definition, we can conclude that the essence of discrete dynamical systems is that their time is a \hyperref[def:cyclic_monoid]{cyclic monoid}.
\end{comments}

\begin{example}\label{ex:def:dynamical_system}
  We list examples of \hyperref[def:dynamical_system]{dynamical systems}:
  \begin{thmenum}
    \thmitem{ex:def:dynamical_system/cellular_automata} Cellular automata, which we discuss in \fullref{sec:cellular_automata}, are discrete-time dynamical systems whose phase space is also Aristotelian discrete in the sense of \cref{con:discrete_and_continuous_variables}.

    \thmitem{ex:def:dynamical_system/fixed_point} As discussed in \cref{def:one_step_evolution_function}, discrete dynamical systems are characterized by iterated composition of a single one-step evolution endofunction.

    Such iterations occur naturally in \fullref{thm:knaster_tarski_iteration} and \fullref{thm:banach_fixed_point_theorem}.

    The latter requires the phase space to be a \hyperref[def:complete_metric_space]{complete metric space}, and it often happens to be a connected subset of an \hyperref[def:euclidean_plane]{Euclidean space}, making the phase space Aristotelian continuous in the sense of \cref{con:discrete_and_continuous_variables}.

    \thmitem{ex:def:dynamical_system/collatz} \Fullref{hyp:collatz_conjecture} naturally induces a discrete-time dynamical system with a discrete space. Both the time and phase space are the set of nonnegative integers.

    \thmitem{ex:def:dynamical_system/difference_equations} Difference equations, which we discuss in \fullref{sec:difference_equations}, are discrete-time dynamical systems whose phase space may.

    \thmitem{ex:def:dynamical_system/recurrence_relations} As shown in \cref{thm:autonomous_recurrence_as_dynamical_system}, \hyperref[def:recurrence_relation/autonomous]{autonomous recurrence relations}, which we discuss in \fullref{sec:recurrence_relations}, can be regarded as discrete-time dynamical systems acting on state vectors. The phase space is then the space of state vectors defined in \cref{def:recurrence_relation_space}. This phase space may again be discrete or continuous or neither.

    \thmitem{ex:def:dynamical_system/rotation} A dynamical system whose phase space is an \hyperref[def:euclidean_plane]{Euclidean space} --- the quintessential continuous phase space in the sense of \cref{con:discrete_and_continuous_variables} --- corresponds to movement of points.

    For example, in \cref{ex:def:group_action_orbit/rotation} we discussed that rotation is a group action on the plane, and hence we can view rotation as a dynamical system whose time is given by the \hyperref[def:circle_group]{circle group}. We can extend this system to any real moment by periodicity, thus obtaining a continuous-time dynamical system with continuous state space.

    \thmitem{ex:def:dynamical_system/multi_dimensional_time} Consider the \hyperref[def:free_monoid]{free monoid} \( A^* \) on a nonempty set \( A \).

    A dynamical system with time \( A^* \) generalizes discrete-time systems, whose time is a \hyperref[def:cyclic_monoid]{cyclic monoid}. Instead of iteratively composing a single function, these dynamical systems iteratively compose multiple functions.

    \Cref{thm:def:dynamical_system/iterated} implies that, for any symbol \( a \) in \( A^* \), the map \( \Phi_a \) determines all the maps \( \Phi_{a^n} \), where \( n \) is any integer. In this example we use multiplicative notation for time.

    For a word like \( a^n b^m \) then, the map \( \Phi_{a^n b^m} \) is the composition \( \Phi_b^m \bincirc \Phi_a^n \). The word \( abcba \) corresponds to
    \begin{equation*}
      \Phi_a \bincirc \Phi_b \bincirc \Phi_c \bincirc \Phi_b \bincirc \Phi_a.
    \end{equation*}

    If we consider the \hyperref[def:free_group]{free group} \( F(A) \) instead of the free monoid \( A^* \), all these functions are invertible, and the words \( ab^{-1}b \) and \( a \) lead to the same function
    \begin{equation*}
      (\Phi_b \bincirc \Phi_{b^{-1}}) \bincirc \Phi_a = \Phi_a.
    \end{equation*}

    \incite[363]{Rozikov2012MultiDimensionalTimeDynamicalSystem} dedicates his paper to dynamical systems whose time is a free group. He calls them \enquote{multi-dimensional time dynamical systems}.
  \end{thmenum}
\end{example}

\paragraph{Trajectories}

\begin{definition}\label{def:dynamical_system_trajectory}\mimprovised
  Consider a \hyperref[def:dynamical_system]{dynamical system} with evolution function \( \Phi: T \times X \to X \). A \term[ru=траектория (\cite[35]{Юмагулов2015ДинамическиеСистемы}), en=trajectory (\cite[361]{Rozikov2012MultiDimensionalTimeDynamicalSystem})]{trajectory} starting at the \term[ru=начальное состояние (\cite[35]{Юмагулов2015ДинамическиеСистемы})]{initial state} \( x_0 \in X \) is an \hyperref[def:indexed_family]{indexed family} \( \seq{ x_t }_{t \in T} \) obtained as\fnote{Note that \( \Phi_0 \) is the identity, so \( \Phi_0(x_0) = x_0 \).}
  \begin{equation*}
    x_t \coloneqq \Phi_t(x_0).
  \end{equation*}

  We call the underlying set \( \set{ \Phi_t(x_0) \given t \in T } \) the \term[en=orbit (\cite[361]{Rozikov2012MultiDimensionalTimeDynamicalSystem})]{orbit} of \( x_0 \) and denote it by \( O(x_0) \).

  If \( T \) is \hyperref[def:totally_ordered_set]{totally ordered}, we define \term{positive trajectories} and \term{positive orbits} \( O^+(x_0) \) by only considering positive moments, and similarly \term{negative trajectories} and \term{negative orbits} \( O^-(x_0) \).
\end{definition}
\begin{comments}
  \item The orbit \( O(x_0) \) coincides with the group action orbit of \( x_0 \) in the sense of \cref{def:group_action_orbit}. Usage in the literature may be ambiguous, however; see our usage comment below, as well as the discussion in \cite{MathSE:orbit_and_trajectory_in_dynamical_system}.

  \item We base our definition on the following:
  \begin{itemize}
    \item \incite[361]{Rozikov2012MultiDimensionalTimeDynamicalSystem} calls the function \( t \mapsto \Phi_t(x) \) the \enquote{flow through \( x \)} (even though he uses \enquote{flow} as a general term for continuous-time systems). He calls the \hyperref[def:set_valued_map/graph]{graph} of a flow a \enquote{trajectory through \( x \)}, which coincides with our usage, and he calls the underlying set the \enquote{orbit through \( x \)}, which also coincides with our usage.

    \item \incite[35]{Юмагулов2015ДинамическиеСистемы} defines a \enquote{траектория} (\enquote{trajectory}) a for discrete-time systems after informally discussing them for continuous-time systems. He uses \enquote{орбита} (\enquote{orbit}) synonymously, however we prefer our notion of orbit to coincide with the one given in \cref{def:group_action_orbit} for group actions.

    \item \incite[def. 3.4.4]{HadelerMüller2017CellularAutomata} defines a \enquote{trajectory} and \enquote{semi-orbit} synonymously as the zero-based trajectory in a discrete-time topological dynamical system.

    \item \incite[2]{BrinStuck2002DynamicalSystems} avoid defining trajectories, but define an orbit similarly to us.
  \end{itemize}
\end{comments}

\begin{remark}\label{rem:dynamical_system_trajectory_terminology}
  For example, the \hyperref[def:dynamical_system_trajectory]{trajectories} of a dynamical system are so named because, for an autonomous system of differential equations, they encode movement through the phase space. In this sense, a trajectory is local since the corresponding parametric curve is not considered is isolation, but as part of the phase space.

  At the level of generality of an abstract dynamical system, however, the term is not justified so well. For example, for a \hyperref[def:cellular_automaton]{cellular automaton}, the phase space is merely a formalism and a trajectory encodes the evolution of the entire global configuration.
\end{remark}

\begin{proposition}\label{thm:def:dynamical_system_trajectory}
  \hyperref[def:dynamical_system_trajectory]{Dynamical system trajectories} have the following basic properties:
  \begin{thmenum}
    \thmitem{thm:def:dynamical_system_trajectory/composition} For a dynamical system with evolution function \( \Phi: T \times X \to X \), the trajectory of \( x_0 \) satisfies, for every \( t \in X \),
    \begin{equation*}
      x_{t+s} = \Phi_s(x_t).
    \end{equation*}

    The order of \( s \) and \( t \) is important unless \( T \) is commutative.

    \thmitem{thm:def:dynamical_system_trajectory/discrete} For a \hyperref[def:discrete_dynamical_system]{discrete-time} dynamical system with \hyperref[def:one_step_evolution_function]{one-step evolution function} \( \varphi: X \to X \), the trajectory of \( x_0 \) is a sequence, perhaps two-sided, such that, for any integer \( n \) in \( T \), we have
    \begin{equation*}
      x_n = \varphi^n(x_0).
    \end{equation*}
  \end{thmenum}
\end{proposition}
\begin{proof}
  \SubProofOf{thm:def:dynamical_system_trajectory/composition} Follows from \ref{eq:def:monoid_action/family/compatibility}.

  \SubProofOf{thm:def:dynamical_system_trajectory/discrete} Follows from \cref{thm:def:dynamical_system_trajectory/composition}.
\end{proof}

\paragraph{Periodicity}

\begin{definition}\label{def:dynamical_system_periodicity}\mimprovised
  Consider a \hyperref[def:dynamical_system]{dynamical system} with evolution function \( \Phi: T \times X \to X \). As in \cref{def:periodic_function}, we suppose that \( T \) is infinite (even though this it not strictly necessary). For a fixed state \( x \), define the set
  \begin{equation*}
    P(\Phi, x) \coloneqq \set{ p \in T \given \Phi_p(x) = x }.
  \end{equation*}

  For every \hi{nonzero} moment \( p \) in \( P(\Phi, x) \), we say that \( x \) is a \term[en=periodic (point) (\incite[def. 11(iii)]{GiuntiMazzola2012DynamicalSystemsOnMonoids}]{periodic} with \term[en=period (\incite[def. 11(ii)]{GiuntiMazzola2012DynamicalSystemsOnMonoids}]{period} \( p \), and similarly that the \hyperref[def:dynamical_system_trajectory]{orbit} \( O(x) \) is \term[en=periodic (orbit) (\incite[def. 12(i)]{GiuntiMazzola2012DynamicalSystemsOnMonoids}]{periodic} with \term{period} \( p \).
\end{definition}
\begin{comments}
  \item We formulate this definition to match the analogous (but less tidy) definition of periodic functions in \cref{def:periodic_function}. Modulo the set of periods \( P(\Phi, x) \) being made explicit, the definition mostly coincides with those presented by \incite[def. 11(iii)]{GiuntiMazzola2012DynamicalSystemsOnMonoids} and \incite[2]{BrinStuck2002DynamicalSystems}.
\end{comments}

\begin{proposition}\label{thm:dynamical_system_trajectory_period}
  In the \hyperref[def:dynamical_system]{dynamical system} with evolution function \( \Phi: T \times X \to X \), the moment \( p \) is a period of the point \( x_0 \) in the sense of \cref{def:dynamical_system_periodicity} if and only if it is a period of the \hyperref[def:dynamical_system_trajectory]{trajectory} \( \seq{ x_t }_{t \in T} \) in the sense of \cref{def:periodic_function}.
\end{proposition}
\begin{proof}
  We must show that the sets
  \begin{equation*}
    P(t \mapsto \Phi_t(x_0)) = \set{ p \in T \given \qforall* {t \in T} \Phi_{t + p}(x_0) = \Phi_t(x_0) }
  \end{equation*}
  and
  \begin{equation*}
    P(\Phi, x_0) = \set{ p \in T \given \Phi_p(x_0) = x_0 }
  \end{equation*}
  coincide.

  First, fix \( p \) from \( P(t \mapsto \Phi_t(x_0)) \). Then
  \begin{equation*}
    \Phi_p(x_0)
    =
    \Phi_0(x_0)
    \reloset {\ref{eq:def:monoid_action/family/identity}} =
    x_0,
  \end{equation*}
  so \( p \) belongs to \( P(\Phi, x_0) \).

  Conversely, fix \( p \) from \( P(\Phi, x_0) \) and \( t \) from \( T \). Then
  \begin{equation*}
    \Phi_{t + p}(x_0)
    \reloset {\ref{eq:def:monoid_action/family/compatibility}} =
    \Phi_t(\Phi_p(x_0))
    =
    \Phi_t(x_0),
  \end{equation*}
  so \( p \) belongs to \( P(t \mapsto \Phi_t(x_0)) \).
\end{proof}

\begin{definition}\label{def:dynamical_system_fixed_point}
  Consider a \hyperref[def:dynamical_system]{dynamical system} with evolution function \( \Phi: T \times X \to X \). We say that the state \( x \) is a \term[ru=неподвижная точка (\cite[39]{Юмагулов2015ДинамическиеСистемы}), en=fixed point (\cite[2]{BrinStuck2002DynamicalSystems})]{fixed point} if any of the following equivalent conditions hold:

  \begin{thmenum}
    \thmitem{def:dynamical_system_fixed_point/direct}\mcite[2]{BrinStuck2002DynamicalSystems} \( \Phi_t(x) = x \) for every moment \( t \).
    \thmitem{def:dynamical_system_fixed_point/period_set}\mimprovised The set \( P(\Phi, x) \) of periods coincides with \( T \).
    \thmitem{def:dynamical_system_fixed_point/orbit}\mcite[def. 11(i)]{GiuntiMazzola2012DynamicalSystemsOnMonoids} The orbit \( O(x) \) contains only \( x \).
  \end{thmenum}
\end{definition}

\begin{definition}\label{def:dynamical_system_fundamental_period}\mimprovised
  In a \hyperref[def:dynamical_system]{dynamical system}, if the \hyperref[def:dynamical_system_trajectory]{trajectory} of a \hyperref[def:dynamical_system_periodicity]{periodic point} has a fundamental period in the sense of \cref{def:fundamental_function_period}, we call it the \term{fundamental period} of the point.
\end{definition}
\begin{comments}
  \item \Cref{thm:dynamical_system_trajectory_period} and \cref{thm:monoid_of_function_periods} implies that the set \( P(\Phi, x) \) of periods is an infinite monoid. If \( P(\Phi, x) \) is either a \hyperref[def:cyclic_monoid]{cyclic monoid} or infinite \hyperref[def:cyclic_group]{cyclic group}, we can pick a generator as the fundamental period. The details are delegated to \cref{def:fundamental_function_period}.

  \item Even though this definition is presented differently, it leads to the same notion as the definition by \incite[def. 11(iv)]{GiuntiMazzola2012DynamicalSystemsOnMonoids}, where the authors call it \enquote{the period} to distinguish it from \enquote{a period}. We briefly describe their approach to periodicity in \cref{rem:periodic_functions_and_periods}.
\end{comments}

\begin{definition}\label{def:dynamical_system_eventual_periodicity}\mcite[2]{BrinStuck2002DynamicalSystems}
  Consider a \hyperref[def:dynamical_system]{dynamical system} with evolution function \( \Phi: T \times X \to X \).

  We say that the point \( x \) is \term[en=eventually periodic (point) (\cite[2]{BrinStuck2002DynamicalSystems})]{eventually periodic} if there exists a moment \( p_0 \) such that \( \Phi_{p_0}(x) \) is a \hyperref[def:dynamical_system_periodicity]{periodic point}.

  In this case, we also call the orbit \( O(x) \) \term[en=eventually periodic (orbit) (\cite[def. 12(ii)]{GiuntiMazzola2012DynamicalSystemsOnMonoids})]{eventually periodic}.
\end{definition}
\begin{comments}
  \item We base our definition of eventually periodic point on \bycite[2]{BrinStuck2002DynamicalSystems}, but relax the requirement for \( p_0 \) to be positive (it is assumed to be a real number or integer). They describe \cref{thm:def:dynamical_system_eventual_periodicity/reversible}, and we ensure that it holds.

  Similarly, we base our definition of eventually periodic orbit on \bycite[def. 12(ii)]{GiuntiMazzola2012DynamicalSystemsOnMonoids}, but relax the requirement for \( p_0 \) to be nonzero (it is assumed to be a member of a monoid).

  Allowing \( p_0 \) to be zero leads to the properties in \cref{thm:def:dynamical_system_eventual_periodicity}

  \item This notion is related but distinct from \hyperref[def:ultimately_periodic_sequence]{ultimately periodic sequence}. The two are linked via \cref{thm:def:dynamical_system_eventual_periodicity/ultimately_periodic}. A noticeable difference is that we make no attempt to define a preperiod in general.
\end{comments}

\begin{proposition}\label{thm:def:dynamical_system_eventual_periodicity}
  \hyperref[def:dynamical_system_eventual_periodicity]{Eventually periodic} points in a \hyperref[def:dynamical_system]{dynamical system} have the following basic properties:
  \begin{thmenum}
    \thmitem{thm:def:dynamical_system_eventual_periodicity/periodic} Every periodic point is eventually periodic.

    \thmitem{thm:def:dynamical_system_eventual_periodicity/reversible} If the system is \hyperref[def:reversible_dynamical_system]{reversible}, every eventually periodic point is periodic.

    \thmitem{thm:def:dynamical_system_eventual_periodicity/discrete} If the system is \hyperref[def:discrete_dynamical_system]{discrete}, every eventually periodic sequence has a fundamental period.

    \thmitem{thm:def:dynamical_system_eventual_periodicity/ultimately_periodic} If the system is irreversible and discrete, i.e. if the time consists of nonnegative integers, then a point is periodic if and only if its \hyperref[def:dynamical_system_trajectory]{trajectory} is an \hyperref[def:ultimately_periodic_sequence]{ultimately periodic sequence}.

    In this case the concept of \enquote{preperiod} from \cref{def:ultimately_periodic_sequence} makes sense.
  \end{thmenum}
\end{proposition}
\begin{proof}
  \SubProofOf{thm:def:dynamical_system_eventual_periodicity/periodic} Trivial.

  \SubProofOf{thm:def:dynamical_system_eventual_periodicity/reversible} If \( \Phi_{p_0}(x) \) is periodic with period \( p \), then so is \( x \) because
  \begin{equation*}
    x
    \reloset {\ref{eq:def:monoid_action/family/identity}} =
    \Phi_{-p_0 + p_0}(x)
    \reloset {\ref{eq:def:monoid_action/family/compatibility}} =
    \Phi_{-p_0}(\Phi_{p_0}(x))
    =
    \Phi_{-p_0}(\Phi_{p_0}(\Phi_p(x)))
    \reloset {\ref{eq:def:monoid_action/family/compatibility}} =
    \Phi_{-p_0 + p_0 + p}(x)
    =
    \Phi_p(x).
  \end{equation*}

  \SubProofOf{thm:def:dynamical_system_eventual_periodicity/discrete} Follows from \cref{thm:def:fundamental_function_period/sequence}.

  \SubProofOf{thm:def:dynamical_system_eventual_periodicity/ultimately_periodic} Trivial.
\end{proof}

\paragraph{Linear dynamical systems}

\begin{definition}\label{def:linear_dynamical_system}\mcite[11]{Юмагулов2015ДинамическиеСистемы}
  Consider a \hyperref[def:dynamical_system]{dynamical system} with evolution function \( \Phi: T \times X \to X \). Suppose that \( X \) is a \hyperref[def:vector_space]{vector space} over \( \BbbK \).

  If, for every moment \( t \), the function \( \Phi_t: X \to X \) is \hyperref[def:linear_function]{linear}, we say that the system itself is \term[ru=линейная (динамическая система) (\cite[11]{Юмагулов2015ДинамическиеСистемы})]{linear}. Otherwise, we call the system \term[ru=нелинейная]{nonlinear}.
\end{definition}
\begin{comments}
  \item Only systems whose phase space is a vector space can be nonlinear; otherwise the notion does not make sense.
\end{comments}

\paragraph{Limiting behavior}

\begin{definition}\label{def:topological_dynamical_system}\mcite[28]{BrinStuck2002DynamicalSystems}
  Consider a \hyperref[def:dynamical_system]{dynamical system} with evolution function \( \Phi: T \times X \to X \). If \( T \) is a \hyperref[def:topological_semigroup]{topological monoid} and \( X \) is a \hyperref[def:topological_space]{topological space}, we may require \( \Phi \) to be continuous. In this case we say that the system is a \term{topological dynamical system}.
\end{definition}
\begin{comments}
  \item Regarded as a group action, a topological dynamical system is precisely a \hyperref[def:continuous_monoid_action]{continuous action}. Since \enquote{continuous dynamical system} refers to \cref{def:continuous_dynamical_system}, a distinct term is needed.
\end{comments}

\begin{definition}\label{def:dynamical_system_invariant_set}
  In a \hyperref[def:dynamical_system]{dynamical system}, we say that the subset of the phase space is \term[ru=инвариантное множество (\cite[94]{Юмагулов2015ДинамическиеСистемы})]{invariant} if it contains the \hyperref[def:dynamical_system_trajectory]{orbit} of each of its members.
\end{definition}
\begin{comments}
  \item We generalize the definition given by \incite[94]{Юмагулов2015ДинамическиеСистемы} from discrete-time to arbitrary dynamical systems.
\end{comments}

\begin{definition}\label{def:dynamical_system_limit_point}\mcite[def. 1.6.2]{HasselblattKatok1995DynamicalSystems}
  Consider a \hyperref[def:topological_dynamical_system]{topological dynamical system} with a \hyperref[def:totally_ordered_set]{totally ordered} time set \( T \) and evolution function \( \Phi: T \times X \to X \).

  We say that the point \( y \) from \( X \) is a \term{\( \omega \)-limit point} (resp. \( \alpha \)-limit point) of \( x \) if there exists a sequence \( t_1, t_2, \ldots \) of moments converging to \( \infty \) (resp. \( -\infty \)) such that \( \Phi_{t_1}(x), \Phi_{t_2}(x), \ldots \) converges to \( y \).

  We denote the set of \( \omega \)-limit points of \( x \) by \( \omega(x) \), and similarly for \( \alpha(x) \).
\end{definition}
\begin{comments}
  \item We generalize the definition from \cite[def. 1.6.2]{HasselblattKatok1995DynamicalSystems}, where \( T \) is the set of integers or real numbers.
\end{comments}

\begin{proposition}\label{thm:limit_point_set_is_invariant}
  In a \hyperref[def:topological_dynamical_system]{topological dynamical system} with \hyperref[def:totally_ordered_set]{totally ordered} time, the \hyperref[def:dynamical_system_limit_point]{\( \omega \)-limit set} and \hyperref[def:dynamical_system_limit_point]{\( \alpha \)-limit set} of any point are \hyperref[def:dynamical_system_invariant_set]{invariant}.
\end{proposition}
\begin{proof}
  Denote the evolution function by \( \Phi: T \times X \to X \).

  If \( y \) is an \( \omega \)-limit of \( x \), there exists a sequence \( t_1, t_2, \ldots \) of moments such that
  \begin{equation*}
    y = \lim_{k \to \infty} \Phi_{t_k}(x).
  \end{equation*}

  Then, for any moment \( t \), by continuity of \( \Phi \),
  \begin{equation*}
    \Phi_t(y)
    =
    \Phi_t(\lim_{k \to \infty} \Phi_{t_k}(x))
    =
    \lim_{k \to \infty} \Phi_t(\Phi_{t_k}(x))
    =
    \lim_{k \to \infty} \Phi_{t + t_k}(x).
  \end{equation*}

  This is again an \( \omega \)-limit point of \( x \). Generalizing, we conclude hence the set \( \omega(x) \) is invariant.

  We can analogously show that \( \alpha(x) \) is invariant.
\end{proof}

\begin{definition}\label{def:dynamical_system_recurrent_point}\mcite[def. 3.3.2]{HasselblattKatok1995DynamicalSystems}
  We say that a point \( x \) in a \hyperref[def:topological_dynamical_system]{topological dynamical system} with \hyperref[def:totally_ordered_set]{totally ordered} time is \term{positive recurrent} (resp. \term{negative recurrent}) if it belongs to its own \( \omega \)-limit set (resp. \( \alpha \)-limit set).
\end{definition}

\begin{definition}\label{def:dynamical_system_stability}\mimprovised
  Let \( x^* \) be a \hyperref[def:dynamical_system_fixed_point]{fixed point} of a \hyperref[def:topological_dynamical_system]{topological dynamical system} with \hyperref[def:totally_ordered_set]{totally ordered} time and evolution function \( \Phi: T \times X \to X \).

  \begin{thmenum}
    \thmitem{def:dynamical_system_stability/stable} We say that \( x^* \) is \term[en=stable (fixed point) (\cite[def. 1.2(1)]{Elaydi2007DiscreteChaos})]{stable} if, for any neighborhood \( V \) of \( x^* \), there exists a neighborhood \( U \) of \( x^* \) such that, for every \( x \) in \( U \), the entire \hyperref[def:dynamical_system_trajectory]{positive orbit} \( O^+(x) \) belongs to \( V \).

    \thmitem{def:dynamical_system_stability/attracting} We say that \( x^* \) is \term[en=attracting (fixed point) (\cite[def. 1.2(2)]{Elaydi2007DiscreteChaos})]{attracting} if there exists a neighborhood \( V \) of \( x^* \) such that, for every \( x \) in \( V \), we have
    \begin{equation*}
      \lim_{t \to \infty} \Phi_t(x) = x^*.
    \end{equation*}

    \thmitem{def:dynamical_system_stability/local_asymptotic} We say that \( x^* \) is \term[en=asymptotically stable (fixed point) (\cite[def. 1.2(3)]{Elaydi2007DiscreteChaos})]{locally asymptotically stable} if it is stable and attracting.

    \thmitem{def:dynamical_system_stability/global_asymptotic} If every positive trajectory converges to \( x^* \), we say that \( x^* \) is \term[en=globally asymptotically stable (fixed point) (\cite[def. 1.2(3)]{Elaydi2007DiscreteChaos})]{globally asymptotically stable}.
  \end{thmenum}
\end{definition}
\begin{comments}
  \item The definition is based on \bycite[def. 1.2]{Elaydi2007DiscreteChaos}, but generalized from discrete-time to arbitrary dynamical systems.
\end{comments}

\begin{definition}\label{def:dynamical_system_attractor}\mimprovised
  Consider a \hyperref[def:topological_dynamical_system]{topological dynamical system} with a \hyperref[def:totally_ordered_set]{totally ordered} time set \( T \) and evolution function \( \Phi: T \times X \to X \).

  We call a \hyperref[def:compact_space]{compact set} \( A \) in \( X \) an \term[ru=аттрактор (\cite[94]{Юмагулов2015ДинамическиеСистемы}), en=attractor (\cite[def. 3.1.1]{HasselblattKatok1995DynamicalSystems})]{attractor} if there exists a neighborhood \( V \) of \( A \) and a moment \( t_0 \) such that \( \Phi_{t_0}(V) \subseteq V \) and
  \begin{equation}\label{eq:def:dynamical_system_attractor}
    A = \bigcap_{t \geq t_0} \Phi_t(V).
  \end{equation}
\end{definition}
\begin{comments}
  \item \incite[94]{Юмагулов2015ДинамическиеСистемы} defines attractors for discrete-time systems and later in \cite[169]{Юмагулов2015ДинамическиеСистемы} for continuous-time systems. We generalize this usage to arbitrary dynamical systems.
\end{comments}

\paragraph{The Collatz conjecture}

\begin{definition}\label{def:collatz_map}\mcite[1]{Lagarias2021CollatzOverview}
  We define the \term[en=Collatz function (\cite[1]{Lagarias2021CollatzOverview})]{Collatz map} on positive integers as follows:
  \begin{equation}\label{eq:def:collatz_map}
    C(n) \coloneqq \begin{cases}
      3n + 1, &n \T{is odd}, \\
      n / 2,  &n \T{is even}.
    \end{cases}
  \end{equation}
\end{definition}
\begin{comments}
  \item \Cref{thm:integer_product_parity} implies that, if \( n \) is odd, \( 3n \) is also odd, thus \( 3n + 1 \) is even, and repeated application of \( C(n) \) would result in division by \( 2 \). This leads to the reduced Collatz map
  \begin{equation}\label{eq:def:collatz_conjecture/reduced}
    T(n) \coloneqq \begin{cases}
      (3n + 1) / 2, &n \T{is odd}, \\
      n / 2,        &n \T{is even}.
    \end{cases}
  \end{equation}

  \item \incite[11]{Wirsching1998CollatzDynamicalSystem} claims that Collatz himself called \eqref{eq:def:collatz_map} the \enquote{\( 3n + 1 \)}-Function. Meanwhile, \incite[11]{Wirsching1998CollatzDynamicalSystem} himself refers to \eqref{eq:def:collatz_conjecture/reduced} as the \enquote{\( 3n + 1 \) function}, while \incite[4]{Lagarias2021CollatzOverview}, calls it the \enquote{\( 3x + 1 \) function}.
\end{comments}

\begin{conjecture}[Collatz conjecture]\label{hyp:collatz_conjecture}\mcite[1]{Lagarias2021CollatzOverview}
  For every positive integer \( n \), iterated application of the \hyperref[def:collatz_map]{Collatz map} yields \( 1 \), that is, there exists some positive integer \( m \) such that \( C^m(n) = 1 \).
\end{conjecture}
\begin{comments}
  \item \incite[1]{Lagarias1985CollatzGeneralizations} provide a brief history of this conjecture and the origin of other names used for it --- \enquote{The \( 3x + 1 \) problem}, \enquote{The Syracuse problem}, \enquote{Hesse's algorithm}, \enquote{Kakutani's problem} and \enquote{Ulam's problem}. He also lists several equivalent conjectures.

  Since it is unclear how to approach proving the conjecture, Paul Erd\"os is credited for the remark
  \begin{displayquote}
    Mathematics is not yet ready for such problems.
  \end{displayquote}
\end{comments}

\begin{definition}\label{def:collatz_graph}\mcite[2]{Lagarias1985CollatzGeneralizations}
  We define the \term{Collatz graph} as the \hyperref[def:directed_graph]{directed graph} whose vertices are positive integers with an arc from \( n \) to \( m \) if \( m = C(n) \).

  \begin{figure}[!ht]
    \centering
    \includegraphics[page=1]{output/def__collatz_graph}
    \caption{A fragment of the \hyperref[def:collatz_graph]{Collatz graph}.}\label{fig:def:collatz_graph}
  \end{figure}
\end{definition}
\begin{comments}
  \item Both \incite[1]{Lagarias1985CollatzGeneralizations} and \incite[31]{Wirsching1998CollatzDynamicalSystem} actually define the Collatz graph for the reduced Collatz map \( T(n) \); we find it more illustrative to use \( C(n) \).
\end{comments}

\begin{theorem}[Collatz conjecture equivalences]\label{thm:collatz_conjectures_equivalences}
  The following statements are \hyperref[def:logical_theory/equivalent]{equivalent} to \fullref{hyp:collatz_conjecture}:
  \begin{thmenum}[series=thm:axiom_of_choice_equivalences]
    \thmitem{thm:collatz_conjecture_equivalences/recurrence} Consider the \hyperref[def:recurrence_relation]{recurrence relation} \( X_{N+1} = C(X_N) \), where \( C \) is the \hyperref[def:collatz_map]{Collatz map}.

    We claim that this recurrence relation \hyperref[def:dynamical_system_stability/stable]{stabilizes} at \( 1 \) for every initial value.

    \thmitem{thm:collatz_conjecture_equivalences/orbit} Consider the \hyperref[def:discrete_dynamical_system]{discrete-time} \hyperref[def:dynamical_system]{dynamical system} whose \hyperref[def:one_step_evolution_function]{one-step evolution function} is the Collatz map.

    We claim that every \hyperref[def:dynamical_system_trajectory]{orbit} of this system contains \( 1 \).

    \thmitem{thm:collatz_conjecture_equivalences/graph_connected}\mcite[1]{Lagarias1985CollatzGeneralizations} The \hyperref[def:collatz_graph]{Collatz graph} is \hyperref[def:graph_connectedness/weak]{weakly connected}.

    \thmitem{thm:collatz_conjecture_equivalences/operation}\mcite{MathSE:collatz_as_a_binary_operation} Consider the \hyperref[def:operation_on_set]{binary operation} \( a \ast b \) from \cref{ex:def:binary_operation/collatz}.

    We claim that, for every positive integer \( n \), the \hyperref[def:recurrence_relation]{recurrence relation} \( X_{N+1} = X_N \ast 1 \) stabilizes at \( 1 \) for every initial value.
  \end{thmenum}
\end{theorem}
\begin{proof}
  \SubProofOf{thm:collatz_conjecture_equivalences/recurrence} This is clearly a restatement of \fullref{hyp:collatz_conjecture}.

  \SubProofOf{thm:collatz_conjecture_equivalences/orbit} This is also clearly a restatement of \fullref{hyp:collatz_conjecture}.

  \SubProofOf{thm:collatz_conjecture_equivalences/graph_connected} The trajectories of the dynamical system are precisely the graph paths ending at \( 1 \). We can concatenate two such paths into a generalized walk, thus a trajectory contains \( 1 \) if and only if some generalized walk passes through \( 1 \).

  \SubProofOf{thm:collatz_conjecture_equivalences/operation} We have
  \begin{equation*}
    a \ast 1 = \frac {3a + 1} {2^{v_2(3a + 1)}} = C^{v_2(3a + 1)} (a).
  \end{equation*}

  Thus, if the recurrence \( X_{N+1} = C(X_N) \) stabilizes, so does \( X_{N+1} = X_N \ast 1 \).
\end{proof}
