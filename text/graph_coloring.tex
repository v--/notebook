\subsection{Graph coloring}\label{subsec:graph_coloring}

\paragraph{Graph coloring}

\begin{definition}\label{def:graph_coloring}\mcite[395]{Erickson2019Algorithms}
  A \term{vertex coloring} (resp. \term{edge coloring} or \term{arc coloring}) of a \hyperref[rem:arbitrary_kind_graph]{arbitrary-kind graph or hypergraph} is, unsurprisingly, a \hyperref[def:set_coloring]{coloring} of its vertices (resp. edges or arcs).

  \begin{figure}[!ht]
    \begin{subcaptionblock}{\textwidth/2}
      \centering
      \includegraphics[page=1]{output/def__graph_coloring}
      \caption{A proper vertex coloring of \hyperref[def:petersen_graph]{\( P_{5,2} \)}}\label{fig:def:graph_coloring/petersen}
    \end{subcaptionblock}
    \hfill
    \begin{subcaptionblock}{\textwidth/2}
      \centering
      \includegraphics[page=2]{output/def__graph_coloring}
      \caption{An improper edge coloring of \hyperref[def:complete_graph]{\( K_6 \)}}\label{fig:def:graph_coloring/triangle}
    \end{subcaptionblock}
    \caption{\hyperref[def:graph_coloring]{Graph colorings}}\label{fig:def:graph_coloring}
  \end{figure}

  \begin{thmenum}
    \thmitem{def:graph_coloring/proper} We say that the vertex coloring (resp. arc/edge coloring) is \term[bg=правилно (оцветяване) (\cite[141]{Мирчев2001Графи}), ru=правильная (раскраска) (\cite[306]{Емеличев1990Графы})]{proper} if no two adjacent vertices (resp. arcs/edges) have the same color.

    \thmitem{def:graph_coloring/colorable} If a graph has a \( \rho \)-coloring of its \hi{vertices}, we say that it is \( \rho \)-\term[bg=\( \rho \)-оцветим (граф) (\cite[141]{Мирчев2001Графи}), ru=\( \rho \)-раскрашиваемый (граф) (\cite[\S 53]{Емеличев1990Графы}), en=\( \rho \)-colorable (\cite[111]{Diestel2005Graphs})]{colorable}.
  \end{thmenum}
\end{definition}
\begin{comments}
  \item We can adopt the view that \( r \)-colorable graphs are generalizations of \( r \)-partite graphs that introduce more natural conditions for directed graphs. Furthermore, we allow colorings with infinitely many colors.

  \item A useful characterization of colorable graphs is \fullref{thm:graph_coloring_as_homomorphism}.

  \item Some authors like \incite[112]{Diestel2005Graphs}, \incite[126]{Harary1969Graphs}, \incite[145]{Bollobas1998Graphs} and \incite[374]{Новиков2013ДискретнаяМатематика} do not distinguish between proper and improper colorings by requiring all colorings to be proper.

  We prefer the more explicit terminology of \mcite[395]{Erickson2019Algorithms}, Arthur White in \cite[590]{Rosen1999DiscreteHandbook}, \incite[\S 53]{Емеличев1990Графы} and \incite[141]{Мирчев2001Графи}.
\end{comments}

\begin{proposition}\label{thm:k_colorable_iff_multipartite}
  A \hyperref[def:undirected_graph]{simple undirected graph} is \( r \)-\hyperref[def:multipartite_graph]{partite} if and only if it is \( r \)-\hyperref[def:graph_coloring/colorable]{colorable}.
\end{proposition}
\begin{proof}
  Trivial.
\end{proof}

\begin{proposition}\label{thm:graph_coloring_as_homomorphism}
  \hyperref[def:graph_coloring/proper]{Proper colorings} can be characterized as follows:

  \begin{thmenum}
    \thmitem{thm:graph_coloring_as_homomorphism/simple_undirected} In a \hyperref[def:undirected_graph]{simple undirected graph} \( G = (V, E) \), a function \( c: V \to \rho \) is a proper coloring if and only if it is a \hyperref[def:undirected_graph/homomorphism]{homomorphism} into the \hyperref[def:complete_graph]{complete graph} \( K_\rho \).

    \thmitem{thm:graph_coloring_as_homomorphism/simple_directed} In a \hyperref[def:directed_graph]{simple directed graph} \( G = (V, A) \), a function \( c: V \to \rho \) is a proper coloring if and only if it is a \hyperref[def:directed_graph/homomorphism]{homomorphism} into the \hyperref[def:graph_functors/simple_doubling]{doubling} \( D_S(K_\rho) \).

    \thmitem{thm:graph_coloring_as_homomorphism/hypergraph} In a \hyperref[def:hypergraph]{hypergraph} \( G = (V, E, \mscrE) \), a function \( c: V \to \rho \) is a proper coloring if and only if the pair \( (c, \mscrE) \) is a \hyperref[def:hypergraph/homomorphism]{homomorphism} into the \hyperref[def:graph_functors/undirected_inclusion]{inclusion} \( I_U(K_\rho) \).

    \thmitem{thm:graph_coloring_as_homomorphism/multi_directed} Finally, in a \hyperref[def:directed_multigraph]{directed multigraph} \( G = (V, E, t, h) \), a function \( c: V \to \rho \) is a proper coloring if and only if pairing \( c \) with \( e \mapsto (t(h), e(h)) \) gives a \hyperref[def:directed_multigraph/homomorphism]{homomorphism} into the \hyperref[def:graph_functors/directed_inclusion]{inclusion} of the \hyperref[def:graph_functors/simple_doubling]{doubling} \( I_D(D_S(K_\rho)) \).
  \end{thmenum}
\end{proposition}
\begin{proof}
  We will only prove \fullref{thm:graph_coloring_as_homomorphism/simple_undirected} since the rest are obvious corollaries.

  \SufficiencySubProof Suppose that \( c: V \to \rho \) is a proper coloring. For any edge \( \set{ u, v } \), since \( c \) is proper, \( c(u) \neq c(v) \), thus \( \set{ c(u), c(v) } \) is not a loop. Therefore, it is an edge of \( K_\rho \).

  Generalizing, we conclude that \( c \) is a homomorphism.

  \NecessitySubProof Suppose that \( c: V \to \rho \) is a homomorphism from \( G \) to \( K_\rho \). For any edge \( \set{ u, v } \), since \( G \) is simple, we have \( u \neq v \). Then \( c(u) \neq c(v) \) because \( \set{ c(u), c(v) } \) is an edge and \( K_\rho \) has no loops.

  Generalizing, we conclude that no two adjacent vertices have the same color.
\end{proof}

\begin{proposition}\label{thm:def:graph_coloring}
  \hyperref[def:multipartite_graph]{Vertex coloring} has the following basic properties:
  \begin{thmenum}
    \thmitem{thm:def:graph_coloring/edgeless} A graph without arcs/edges is \( \rho \)-colorable for any nonzero cardinal \( \rho \).

    \thmitem{thm:def:graph_coloring/succ} If a graph is \( \rho \)-colorable and \( \rho < \kappa \), it is also \( \kappa \)-colorable.

    \thmitem{thm:def:graph_coloring/prec} Given a \hyperref[def:graph_coloring/proper]{proper} \( (\rho + 1) \)-coloring of a graph, if there exists a pair of colors \( p \) and \( q \) such that no arc/edge has its endpoints colored by them, then the \( p \)-colored vertices can be recolored via \( q \) to produce a proper \( \rho \)-coloring.

    \thmitem{thm:def:graph_coloring/order} Every graph of order at most \( \rho \) is \( \rho \)-colorable.
  \end{thmenum}
\end{proposition}
\begin{proof}
  \SubProofOf{thm:def:graph_coloring/edgeless} If there are no arcs/edges in a graph, no two vertices are adjacent, and any vertex coloring is proper.

  \SubProofOf{thm:def:graph_coloring/succ} Given any vertex coloring \( f: V \to \rho \), we can simply extend the range of \( f \) with additional colors.

  \SubProofOf{thm:def:graph_coloring/prec} Trivial.

  \SubProofOf{thm:def:graph_coloring/order} Given a graph \( G \) of order \( \kappa \leq \rho \) with a vertex set \( V \), by definition of cardinal number there exists a bijective function \( c: V \to \kappa \). This is a proper \( \kappa \)-coloring because all color classes have only one vertex. By \fullref{thm:def:graph_coloring/succ}, \( G \) is \( \rho \)-colorable.
\end{proof}

\begin{proposition}\label{thm:hypergraph_representation}
  \hyperref[def:hypergraph]{Hypergraph} have the following representations as \hyperref[def:graph_coloring/proper]{properly \( 2 \)-colored} \hyperref[def:undirected_graph]{simple undirected graph}:
  \begin{thmenum}
    \thmitem{thm:hypergraph_representation/hypergraph_to_graph} To every hypergraph \( H = (V, E, \mscrE) \) we associate a simple graph with vertex set \( V \cup E \), edge set
    \begin{equation*}
      \set[\Big]{ \set{ v, e } \given v \in \mscrE(e) }
    \end{equation*}
    and \( 2 \)-coloring \( l \) such that \( l(v) = 0 \) for every vertex \( v \) and \( l(e) \) for every hyperedge \( e \).

    \begin{figure}[!ht]
      \hfill
      \includegraphics[page=1]{output/thm__hypergraph_representation}
      \hfill
      \includegraphics[page=2]{output/thm__hypergraph_representation}
      \hfill
      \hfill
      \caption{A \hyperref[def:hypergraph]{hypergraph} and its associated \hyperref[def:graph_coloring/proper]{properly \( 2 \)-colored} \hyperref[def:undirected_graph]{simple undirected graph} from \fullref{thm:hypergraph_representation}.}\label{fig:thm:hypergraph_representation}
    \end{figure}

    \thmitem{thm:hypergraph_representation/graph_to_hypergraph} To every properly \( 2 \)-colored simple undirected graph \( G = (V, E, l) \) we associate a hypergraph with vertex set \( l^{-1}(0) \), edge set \( l^{-1}(1) \) and endpoint map
    \begin{equation*}
      \mscrE(e) \coloneqq \set{ v \in V \given \set{ v, e } \in E }.
    \end{equation*}

    \thmitem{thm:hypergraph_representation/equivalence} The category of hypergraphs and hypergraph homomorphisms is \hyperref[rem:category_similarity/isomorphism]{isomorphic} to the category of \( 2 \)-colored simple undirected graphs and color-preserving graph homomorphisms.
  \end{thmenum}
\end{proposition}
\begin{comments}
  \item The construction \fullref{thm:hypergraph_representation/hypergraph_to_graph} is often stated via \hyperref[def:multipartite_graph]{bipartite graphs} --- for example by \incite[187]{DorflerWaller1980Hypergraphs} and \incite[300]{Емеличев1990Графы}. Bipartite graphs may have different partitions into independent sets, thus we do not generally know what is supposed to be the vertex set in the inverse construction \fullref{thm:hypergraph_representation/graph_to_hypergraph}. For this reason, we decided to explicitly rely on \( 2 \)-colorings.
\end{comments}
\begin{proof}
  Straightforward.
\end{proof}

\paragraph{Chromatic number}

\begin{definition}\label{def:graph_chromatic_number}\mcite[127]{Harary1969Graphs}
  The (vertex) \term[bg=хроматично число (\cite[142]{Мирчев2001Графи}), ru=хроматическое число (\cite[235]{Емеличев1990Графы})]{chromatic number} \( \chi(G) \) of the \hyperref[rem:arbitrary_kind_graph]{arbitrary-kind graph or hypergraph} \( G \) is the minimum nonzero cardinal \( \rho \) for which \( G \) is \( \rho \)-\hyperref[def:graph_coloring/colorable]{colorable}.

  If \( \chi(G) = \rho \), we say that \( G \) is \term[ru=\( \rho \)-хроматический (граф) (\cite[236]{Емеличев1990Графы})]{\( \rho \)-chromatic}.
\end{definition}
\begin{comments}
  \item Different authors use a different notation for chromatic numbers. We use \enquote{\( \chi \)} similarly to \incite[122]{Diestel2005Graphs}, \incite[145]{Bollobas1998Graphs}, \incite[152]{Knauer2011AlgGraphTheory}, Arthur White in \cite[590]{Rosen1999DiscreteHandbook}, \incite[235]{Емеличев1990Графы} and \incite[374]{Новиков2013ДискретнаяМатематика}. \incite[28]{GondranMinoux1984Graphs} use \enquote{\( \gamma \)}, while \incite[142]{Мирчев2001Графи} uses \enquote{\( \varkappa \)}.
\end{comments}

\begin{proposition}\label{thm:chromatic_number_edgeless}
  The \hyperref[def:graph_chromatic_number]{chromatic number} of a graph is \( 1 \) if and only if it has no arcs/edges.
\end{proposition}
\begin{proof}
  \SufficiencySubProof Given a \( 1 \)-coloring of a graph, the endpoints of every arc/edge are monochromatic, hence either there are no arcs/edges or the \( 1 \)-coloring is not proper.

  \NecessitySubProof If there exists an arc/edge, its endpoints must be colored with different colors, hence the chromatic number should be at least \( 2 \).
\end{proof}

\begin{proposition}\label{thm:chromatic_number_color_pairs}
  For every positive integer \( r \), for every \hyperref[def:graph_chromatic_number]{\( r \)-chromatic} graph and for every pair of distinct colors \( p \) and \( q \), there exists an arc/edge whose endpoints are colored by \( p \) and \( q \).
\end{proposition}
\begin{comments}
  \item If \( r < 2 \), there are no pairs of distinct colors, and thus the conclusion holds vacuously.
\end{comments}
\begin{proof}
  We will use induction on \( \rho \).
  \begin{itemize}
    \item If \( r < 2 \), as already discussed, the proposition holds vacuously.

    \item Suppose that the proposition holds for \( r \) and consider a \( (r + 1) \)-chromatic graph \( G \) and fix distinct colors \( p \) and \( q \).

    If no arc/edge exists whose endpoints are colored \( p \) and \( q \), \fullref{thm:def:graph_coloring/prec} implies that the graph has a proper \( \rho \)-coloring, which contradicts the assumption that \( G \) is \( (r + 1) \)-chromatic.
  \end{itemize}
\end{proof}

\begin{corollary}\label{thm:chromatic_number_coloring_splits}
  For every positive integer \( r \), for every \hyperref[def:graph_chromatic_number]{\( r \)-chromatic} graph \( G = (V, E) \) and for every proper coloring \( c: V \to \set{ 0, \ldots, r - 1 } \), the function \( c \) is a \hyperref[def:morphism_invertibility/right_cancellative]{categorical split epimorphism} from \( G \) to the \hyperref[def:complete_graph]{complete graph} \( K_r \).
\end{corollary}
\begin{proof}
  By \fullref{thm:graph_coloring_as_homomorphism/simple_undirected}, \( c \) is a homomorphism from \( G \) to \( K_r \).

  By \fullref{thm:chromatic_number_color_pairs}, the function \( { u, v } \mapsto \set{ c(u), c(v) } \) is surjective on edges. Then \( c \) is surjective on vertices, hence, by \fullref{thm:function_invertibility_categorical/right_invertible}, it has a right inverse
  \begin{equation*}
    d: \set{ 0, \ldots, r - 1 } \to V.
  \end{equation*}

  Furthermore, the edge \( \set{ i, j } \) in \( K_r \) maps to \( \set{ d(i), d(j) } \), making \( d \) a homomorphism from \( K_r \) to \( G \).

  Obviously \( d \) is a right inverse of \( c \). Therefore, \( c \) is right invertible in the category of simple undirected graphs.
\end{proof}

\begin{proposition}\label{thm:complete_graph_chromatic_number}
  For a nonempty \hyperref[def:complete_graph]{complete graph} \( K_\rho \) of order \( \rho \), the \hyperref[def:graph_chromatic_number]{chromatic number} is \( \rho \).

  Conversely, for a \hyperref[def:directed_graph]{simple directed graph} \( G \) of \hi{finite} order \( r \), if \( G \) is \( r \)-chromatic, then it is \hyperref[def:complete_subgraph]{complete}\fnote{By \enquote{\( G \) is complete} we mean that \( G \) is a complete subgraph of itself, i.e. it is isomorphic to \( K_r \)}.
\end{proposition}
\begin{proof}
  \SufficiencySubProof By \fullref{thm:def:graph_coloring/order}, \( \chi(K_\rho) \leq \rho \). Furthermore, for every \( \rho < \rho \) and every \( \rho \)-coloring, by \fullref{thm:pigeonhole_principle}, there will be two adjacent vertices with the same color, making the coloring improper.

  Therefore, \( \chi(K_\rho) = \rho \).

  \NecessitySubProof Suppose that \( G = (V, E) \) is a \( r \)-chromatic graph of order \( r \). Since \( G \) has order \( r \), there exists a bijective function \( c: V \to \set{ 0, \ldots, r - 1 } \), which also happens to be a \( r \)-coloring. It is proper because no two vertices have the same color.

  By \fullref{thm:graph_coloring_as_homomorphism/simple_undirected}, \( c \) is a homomorphism from \( G \) to \( K_\rho \). We will show that it satisfies \eqref{eq:thm:graph_isomorphisms/simple_undirected}, which will make it an isomorphism.

  Given the edge \( \set{ i, j } \) in \( K_\rho \), \fullref{thm:chromatic_number_color_pairs} implies that \( \set{ c^{-1}(i), c^{-1}(j) } \) is an edge in \( G \).

  Therefore, \( c \) is a bijective homomorphism satisfying \eqref{eq:thm:graph_isomorphisms/simple_undirected}, and \fullref{thm:graph_isomorphisms/simple_undirected} implies that it is an isomorphism between \( G \) and \( K_\rho \).
\end{proof}

\begin{proposition}\label{thm:complete_multipartite_graph_chromatic_number}
  If the sets \( V_1, \ldots, V_r \) are nonempty, the \hyperref[def:complete_multipartite_graph]{complete multipartite graph} \( K_{V_1,\ldots,V_k} \) has \hyperref[def:graph_chromatic_number]{chromatic number} \( r \).
\end{proposition}
\begin{proof}
  First note that \( K_{V_1, \ldots, V_k} \) is \( r \)-colorable due to \fullref{thm:k_colorable_iff_multipartite}.

  Furthermore, if \( V_1, \ldots, V_r \) are nonempty, again via \fullref{thm:pigeonhole_principle} we can conclude that every coloring with less colors cannot be proper.

  Therefore, the chromatic number is \( r \).
\end{proof}

\begin{proposition}\label{thm:chromatic_number_bound}\mcite[prop. 5.2.1]{Diestel2005Graphs}
  If a simple undirected graph \( G \) with \( m \) edges, where \( m \) is finite, the \hyperref[def:graph_chromatic_number]{chromatic number} satisfies the following inequality:
  \begin{equation}\label{eq:thm:chromatic_number_bound/edges}
    \chi(G) \leq \frac 1 2 + \sqrt{ \frac 1 4 + 2m }.
  \end{equation}

  In case \( G \) has finite order \( n \), we have the weaker upper bound
  \begin{equation}\label{eq:thm:chromatic_number_bound/vertices}
    \chi(G) \leq \frac 1 2 + \sqrt{ \frac 1 4 + n(n - 1) } \leq n.
  \end{equation}
\end{proposition}
\begin{proof}
  Let \( c: V \to \set{ 0, \ldots, r - 1 } \) be a minimal proper coloring of \( G = (V, E) \), i.e. \( \chi(G) = r \).

  \Fullref{thm:graph_coloring_as_homomorphism/simple_undirected} implies that \( c \) is a homomorphism from \( G \) to \( K_r \). By \fullref{thm:complete_graph_edge_count}, \( K_r \) has \( \binom r 2 \) edges. They represent color classes. By \fullref{thm:chromatic_number_color_pairs}, each of the color classes has a preimage in \( G \), i.e. an edge whose endpoints have the colors of the class.

  Therefore,
  \begin{equation*}
    \binom r 2 \leq m,
  \end{equation*}
  which leads to the quadratic inequality
  \begin{equation*}
    r^2 - r - 2m \leq 0.
  \end{equation*}

  By \fullref{thm:quadratic_polynomial_roots}, the roots of the polynomial are
  \begin{equation*}
    \frac { 1 \pm \sqrt{ 1 + 8m } } 2 = \frac 1 2 \pm \sqrt{ \frac 1 4 + 2m }.
  \end{equation*}

  Then
  \begin{equation}\label{eq:thm:chromatic_number_bound/proof/product}
    \parens[\Big]{ r - \frac 1 2 - \sqrt{ \frac 1 4 + 2m } } \cdot \underbrace{\parens[\Big]{ r - \frac 1 2 + \sqrt{ \frac 1 4 + 2m } }}_{\geq 0 \T*{for positive integers} r} \leq 0.
  \end{equation}

  In order for \eqref{eq:thm:chromatic_number_bound/proof/product} to hold, we must have
  \begin{equation*}
    r \leq \frac 1 2 + \sqrt{ \frac 1 4 + 2m }.
  \end{equation*}

  Since \( r = \chi(G) \), this completes the proof of \eqref{eq:thm:chromatic_number_bound/edges}.

  The other bound \eqref{eq:thm:chromatic_number_bound/vertices} follows from \fullref{thm:complete_graph_edge_count}.
\end{proof}

\begin{example}\label{ex:def:graph_chromatic_number}
  We list some examples of \hyperref[def:graph_chromatic_number]{chromatic numbers}:
  \begin{thmenum}
    \thmitem{ex:def:graph_chromatic_number/edgeless} As shown in \fullref{thm:chromatic_number_edgeless}, a graph is arcless/edgeless if and only if its chromatic number is \( 1 \).

    \thmitem{ex:def:graph_chromatic_number/complete} As shown in \fullref{thm:complete_graph_chromatic_number}, the chromatic number of a \hyperref[def:complete_graph]{complete graph} is its order.

    \thmitem{ex:def:graph_chromatic_number/compete_multipartite} As shown in \fullref{thm:complete_multipartite_graph_chromatic_number}, the chromatic number of a \hyperref[def:complete_multipartite_graph]{complete \( r \)-partite} graph with nonempty partitions is \( r \).

    \thmitem{ex:def:graph_chromatic_number/path} As per the discussion in \fullref{ex:def:multipartite_graph/path}, the \hyperref[def:path_graph]{path graph} \( P_n \) has chromatic number \( 2 \).

    \thmitem{ex:def:graph_chromatic_number/cycle} As per the discussion in \fullref{ex:def:multipartite_graph/cycle}, the \hyperref[def:cycle_graph]{cycle graph} \( C_n \) has chromatic number \( 2 \) if \( n \) is odd and \( 3 \) if \( n \) is even.

    \thmitem{ex:def:graph_chromatic_number/petersen} The \hyperref[def:petersen_graph]{Petersen graph} \( P_{5,2} \), as can be seen in \cref{fig:def:graph_coloring/petersen}, is \( 3 \)-colorable.

    Per its definition, it contains the cycle graph \( C_5 \), and hence it has an odd-length cycle. \Fullref{thm:bipartite_iff_no_odd_cycles} then implies that \( P_{5,2} \) is not bipartite, and thus not \( 2 \)-colorable.

    Therefore, \( \chi(P_{5,2}) = 3 \).
  \end{thmenum}
\end{example}

\paragraph{Ramsey numbers}

\begin{example}\label{ex:ramsey_party_problem}
  Among every group of \( 6 \) people, there will either be \( 3 \) who know each other or \( 3 \) who don't, or even possibly both.

  We can restate this in terms of \hyperref[def:graph_coloring]{graph coloring}. In the \hyperref[def:complete_graph]{complete graph} \( K_6 \), whose vertices we regard as people, color the edges in two colors depending on whether their endpoints know each other. Note that this coloring is not generally \hyperref[def:graph_coloring/proper]{proper}.

  Then the statement is that \( K_6 \) has a \hyperref[def:set_coloring/monochromatic]{monochromatic} \hyperref[def:triangle_graph]{triangle subgraph}. This can be seen in \cref{fig:def:graph_coloring/triangle}.

  This example is generalized by \fullref{thm:ramseys_theorem_single}, which is in turn a special case of \fullref{thm:ramseys_theorem}.
\end{example}

\begin{lemma}\label{thm:ramsey_number_lemma}
  Fix positive integers \( s_1, \ldots, s_r \) and let \( A \) be a set with \( s_1 + \cdots + s_r \) elements. Then, for every \hyperref[def:set_partition]{partition} \( A_1, \ldots, A_r \) of \( A \) into \( r \) disjoint subsets, there exists an index \( k \) such that \( S_k \) has at least \( s_k \) elements.
\end{lemma}
\begin{proof}
  Suppose that, for every \( i = 1, \ldots, k \), \( A_k \) has less than \( s_k \) elements. Then the union \( A \) of \( S_1, \ldots, S_m \) has at most
  \begin{equation*}
    (s_1 - 1) + (s_2 - 1) + \cdots (s_r - 1)
  \end{equation*}
  elements, which contradicts our choice of \( A \).

  Therefore, there exists at least one index \( k \) such that \( A_k \) has at least \( s_k \) elements.
\end{proof}

\begin{theorem}[Ramsey's theorem]\label{thm:ramseys_theorem}
  For any \( r \)-tuple of positive integers \( s_1, \ldots, s_r \), there exists a positive integer \( n \) such that, for every (possibly improper) \hyperref[def:graph_coloring]{edge coloring} of the \hyperref[def:complete_graph]{complete graph} \( K_n \) into \( r \) colors, there exists a color \( k \) and a \hyperref[def:complete_subgraph]{complete subgraph} of order \( s_k \) edge-colored with \( k \).
\end{theorem}
\begin{comments}
  \item Frank Ramsey proved a theorem, \cite[thm. B]{Ramsey1930LogicProblem}, stated without using graphs or coloring, amounting to a generalization to \hyperref[def:hypergraph/uniform]{uniform hypergraphs}.

  The theorem has later been restated and re-proved multiple times. We generalized the statement and proof from \incite[thm. VI.1]{Bollobas1998Graphs} to \( r \) colors rather than only \( 2 \).
\end{comments}
\begin{proof}
  Let us denote by \( R(s_1, \ldots, s_r) \) the corresponding Ramsey number, defined in \fullref{def:ramsey_number} as the minimal positive integer \( n \) satisfying the theorem. Then our proof amounts to proving that \( R(s_1, \ldots, s_r) \) is well-defined. We will do this by induction on \( s_1 + \ldots + s_r \).

  \begin{itemize}
    \item If there exists some \( k \) for which \( s_k = 1 \), then vacuously there exists a complete subgraph of order \( s_k \), whose empty set of edges is vacuously colored by \( k \).

    \item Suppose that the inductive hypothesis holds for \( s_1 + \ldots + s_r - 1 \). As per our base case, we can assume without loss of generality that \( s_k > 1 \) for every \( k = 1, \ldots, r \).

    By the inductive hypothesis, for every \( k \), the Ramsey number
    \begin{equation*}
      r_k \coloneqq R(s_1, \ldots, s_k - 1, \ldots, s_r)
    \end{equation*}
    is well-defined. Denote their sum by \( n \) and consider any edge coloring \( c \) of \( K_n \).

    Fix any vertex \( u \) of \( K_n \) and consider the partition of its other vertices into sets \( V_1, \ldots, V_r \), where \( v \) is in \( V_k \) if the edge from \( u \) to \( v \) has color \( k \). By \fullref{thm:ramsey_number_lemma}, there exists an index \( k_0 \) such that \( V_{k_0} \) has at least \( r_{k_0} \) elements.

    Let \( V' \) be a subset of \( V_{k_0} \) of exactly \( r_{k_0} \) elements and consider the\hyperref[def:induced_subgraph]{induced subgraph} \( G[V] \). Note that \( G[V] \) is itself complete, and we can apply the theorem to \( G[V] \). By the property of Ramsey numbers, we have the following two possibilities:
    \begin{itemize}
      \item \( G[V] \) has a complete subgraph of order \( s_k \) edge-colored by \( k \). This is then also a subgraph of \( K_n \), which proves the inductive step.

      \item Otherwise, \( G[V] \) has a complete subgraph \( H \) of order \( s_{k_0} - 1 \) edge-colored by \( k_0 \). Denote its vertices by \( U \). Then, since all vertices of \( G[V] \) (and hence of \( H \)) are connected to \( u \) by an edge with color \( k_0 \), the subgraph \( K_n[U \cup \set{ u }] \) is a complete subgraph of order \( s_{k_0} \) edge-colored by \( k_0 \).
    \end{itemize}

    Therefore, \( n \) satisfies the theorem, and \( R(s_1, \ldots, s_r) \) is well-defined since it is bounded from above by \( n \).
  \end{itemize}
\end{proof}

\begin{corollary}\label{thm:ramseys_theorem_single}
  For any positive integer \( s \), there exists a positive integer \( n \) such that, for every (possibly improper) binary \hyperref[def:graph_coloring]{edge coloring} of the \hyperref[def:complete_graph]{complete graph} \( K_n \), there exists a \hyperref[def:set_coloring/monochromatic]{monochromatic} \hyperref[def:complete_subgraph]{complete subgraph} of order \( s \).
\end{corollary}
\begin{proof}
  Special case of \fullref{thm:ramseys_theorem} when \( r = 2 \) and \( s_1 = s_2 \).
\end{proof}

\begin{definition}\label{def:ramsey_number}\mcite[183]{Bollobas1998Graphs}
  Given an \( r \)-tuple \( s_1, \ldots, s_r \), where \hi{\( r \geq 2 \)}, we define the \term{Ramsey number} \( R(s_1, \ldots, s_r) \) as the least positive integer for which \fullref{thm:ramseys_theorem} holds.
\end{definition}
\begin{comments}
  \item We avoid defining Ramsey numbers for \( r = 1 \) due to possible ambiguity. \incite[183]{Bollobas1998Graphs} defines \( R(s) \) to be \( R(s, s) \), while our general definition would suggest that \( R(s) \) should instead be the degenerate case of monochromatic coloring.
\end{comments}

\begin{proposition}\label{thm:def:ramsey_number}
  \hyperref[def:ramsey_number]{Ramsey numbers} have the following basic properties:
  \begin{thmenum}
    \thmitem{thm:def:ramsey_number/one} If at least one of \( s_1, \ldots, s_r \) is \( 1 \), then
    \begin{equation}\label{eq:thm:def:ramsey_number/one}
      R(s_1, \ldots, s_r) = 1
    \end{equation}

    \thmitem{thm:def:ramsey_number/symmetric} As a function, \( R(s_1, \ldots, s_r) \) is \hyperref[def:symmetric_function]{symmetric} --- for every permutation \( \sigma \) of \( \set{ 1, \ldots, r } \), we have
    \begin{equation}\label{eq:thm:def:ramsey_number/symmetric}
      R(s_1, \ldots, s_r) = R(s_{\sigma(1)}, \ldots, s_{\sigma(r)}).
    \end{equation}

    \thmitem{thm:def:ramsey_number/two} For \( s \geq 2 \), we have
    \begin{equation}\label{eq:thm:def:ramsey_number/two}
      R(s, 2) = R(2, s) = s.
    \end{equation}

    \thmitem{thm:def:ramsey_number/triangle} If all \( s_1, \ldots, s_r \) are greater than \( 1 \), then
    \begin{equation}\label{eq:thm:def:ramsey_number/triangle}
      R(s_1, \ldots, s_r) \leq \sum_{k=1}^r R(s_1, \ldots, s_k - 1, \ldots, s_r).
    \end{equation}

    \thmitem{thm:def:ramsey_number/superposition} For \( r > 2 \), we have
    \begin{equation}\label{eq:thm:def:ramsey_number/superposition}
      R(s_1, \ldots, s_k, s_{k+1}, \ldots, s_r) \leq R(s_1, \ldots, R(s_k, s_{k+1}), \ldots, s_r).
    \end{equation}

    \thmitem{thm:def:ramsey_number/binary_bound} We have
    \begin{equation}\label{eq:thm:def:ramsey_number/binary_bound}
      R(s, t) \leq \binom { (s - 1) + (t - 1) } { s - 1 }
    \end{equation}
  \end{thmenum}
\end{proposition}
\begin{proof}
  \SubProofOf{thm:def:ramsey_number/one} There are no edges in the complete graph \( K_1 \), thus its edges are monochromatic with respect to any possible edge coloring.

  \SubProofOf{thm:def:ramsey_number/symmetric} The colors have no inherent ordering, so we can reorder them as needed.

  \SubProofOf{thm:def:ramsey_number/two} For any binary edge coloring of \( K_s \), it is possible that one color is used for all edges, hence \( R(s, 2) \geq s \). On the other hand, if there is an edge with the other color, then \( K_s \) has the necessary complete subgraph of order \( 2 \leq s \). Therefore, \( R(s, 2) = s \).

  The other case \( R(2, s) \) follows from \fullref{thm:def:ramsey_number/symmetric}.

  \SubProofOf{thm:def:ramsey_number/triangle} This has been shown in our proof of \fullref{thm:ramseys_theorem}.

  \SubProofOf{thm:def:ramsey_number/superposition} Fix \( r > 2 \) positive integers \( s_1, \ldots, s_r \). Let
  \begin{equation*}
    n \coloneqq R(s_1, \ldots, R(s_k, s_{k+1}), \ldots, s_r)
  \end{equation*}
  and consider the complete graph \( K_n \) colored by \( r \) colors.

  \begin{itemize}
    \item If \( K_n \) has a complete subgraph of order \( s_i \) and color \( i \) for \( i < k \) or \( i > k + 1 \), then naturally \eqref{eq:thm:def:ramsey_number/superposition} follows.

    \item Otherwise, \( K_n \) has a complete subgraph of order \( R(s_k, s_{k+1}) \) whose edges may be colored by both \( k \) or \( k + 1 \). But this complete subgraph itself either has a complete subgraph of order \( s_k \) colored by \( k \), or of order \( s_{k+1} \) colored by \( k + 1 \).

    Thus, again, \eqref{eq:thm:def:ramsey_number/superposition} follows.
  \end{itemize}

  \SubProofOf{thm:def:ramsey_number/binary_bound} We will use induction on \( s + t \) to show \eqref{eq:thm:def:ramsey_number/binary_bound}.
  \begin{itemize}
    \item If \( s = 1 \), \fullref{thm:def:ramsey_number/one} implies that \( R(s, t) = 1 \), while
    \begin{equation*}
      \binom { (s - 1) + (t - 1) } {t - 1} = \binom { t - 1 } { t - 1 } = 1,
    \end{equation*}
    hence the bound holds.

    \item The case \( t = 1 \) follows by symmetry.

    \item Suppose that the inequality holds for \( s + t - 1 \). Without loss of generality, suppose also that \( s > 1 \) and \( t > 1 \), since we have already covered the other cases. Then
    \begin{balign*}
      R(s, t)
      &\reloset {\eqref{eq:thm:def:ramsey_number/triangle}} \leq
      R(s - 1, t) + R(s, t - 1)
      \reloset {\T{ind.}} \leq \\ &\leq
      \binom { (s - 2) + (t - 1) } { t - 1 } + \binom { (s - 1) + (t - 2) } { t - 2 }
      \reloset {\eqref{eq:thm:pascals_binomial_recurrence}} = \\ &=
      \binom { (s - 1) + (t - 1) } { t - 1 }.
    \end{balign*}
  \end{itemize}
\end{proof}

\begin{proposition}\label{thm:ramsey_number_33}
  The \hyperref[def:ramsey_number]{Ramsey number} \( R(3, 3) \) is \( 6 \).

  \begin{figure}[!ht]
    \centering
    \includegraphics[page=1]{output/thm__ramsey_number_33}
    \caption{An \hyperref[def:graph_coloring]{edge coloring} of \( K_5 \) with no monochromatic triangles.}\label{fig:thm:ramsey_number_33}
  \end{figure}
\end{proposition}
\begin{proof}
  \Cref{fig:thm:ramsey_number_33} illustrates an edge coloring of \( K_{5,5} \) without monochromatic triangle subgraphs, thus \( R(3, 3) > 5 \). By \fullref{thm:def:ramsey_number/triangle}, we have \( R(3, 3) \leq 6 \).
\end{proof}

\begin{remark}\label{rem:estimating_ramsey_numbers}
  Even though \fullref{thm:def:ramsey_number/binary_bound} and \fullref{thm:def:ramsey_number/superposition} provide upper bounds for \hyperref[def:ramsey_number]{Ramsey numbers}, obtaining the numbers themselves can be cumbersome. The survey article \cite{Radziszowski2021RamseyNumbers} provides a lot of upper and lower bounds since only a few Ramsey numbers are known.

  For example, by \eqref{eq:thm:def:ramsey_number/binary_bound}, we have \( R(3, 3) \leq 6 \). We already know about an explicit edge coloring of \( K_5 \) without monochromatic triangle subgraphs --- \cref{fig:thm:ramsey_number_33} --- and this allows us to conclude that \( R(3, 3) = 6 \). If we do not have such a counterexample, an exhaustive search would require enumerating all possible binary colorings of \( K_5 \) and searching for ones without monochromatic triangles. For \( R(3, 3) \), we would have to traverse all edge colorings of \( K_5 \), and, if all of them have monochromatic triangles, proceed with \( K_4 \).

  By \fullref{thm:complete_graph_edge_count}, there are \( \binom n 2 \) edges in the complete graph \( K_n \), hence the number of \( r \)-colorings is
  \begin{equation*}
    r^{\binom n 2}.
  \end{equation*}

  The number of binary colorings of \( K_5 \) is thus \( 2^{10} = 1024 \). There are \( \binom 5 3 = 10 \) triangles in \( K_5 \), so an exhaustive search for a coloring of \( K_5 \) without monochromatic triangles will require checking at most \( \numprint{10 240} \) subgraphs.

  If we are lucky, exhaustive search can yield a result early. In \cite{notebook:code}, the function \identifier{combinatorics.ramsey.compute_ramsey_number_exhaustively} needs to only check \( \numprint{221} \) colorings and a total of \( \numprint{729} \) triangles to conclude that \( R(3, 3) = 6 \).

  For \( R(3, 3, 2) \), which has the same upper bound, there are \( 3^{10} \) edge colorings, \( 10 \) triangles and \( 10 \) edges (complete subgraphs of order \( 2 \)). The aforementioned function needs to traverse only \( \numprint{3034} \) colorings and \( \numprint{11 168} \) subgraphs to find an \enquote{inappropriate} coloring of \( K_5 \) and conclude that \( R(3, 3, 2) = 6 \).

  For \( R(3, 4) \), which has an upper bound of \( 10 \), we need to traverse at most \( 2^{36} \) edge colorings and for each of them check at most \( 84 \) triangles and \( 126 \) complete subgraphs of order \( 4 \). The aforementioned function did not yield a result after traversing \( \numprint{250 000 000} \) colorings of \( K_9 \) and finding an appropriately colored subgraph in each of them.

  In general, if we are to use exhaustive search, we need to perform an inconceivable amount of checks. For slightly larger numbers like \( R(5, 5) \), \eqref{eq:thm:def:ramsey_number/binary_bound} gives an upper bound of \( 70 \). In order to check whether some coloring of \( K_{69} \) has no monochromatic complete subgraphs of order \( 5 \), we need to inspect at most
  \begin{equation*}
    2^{\binom {69} 2} = 2^{2346}
  \end{equation*}
  binary edge colorings, and for each of them check at most \( \binom {69} 5 = \numprint{11 238 513} \) complete subgraphs of order \( 5 \).

  Attempting to compute this exhaustively turns our not to be fruitful. Other techniques must be used to give more precise bounds. The survey article \cite[4]{Radziszowski2021RamseyNumbers} mentioned earlier states that, while an exact value for \( R(5, 5) \) is now known, we have the inequalities
  \begin{equation*}
    43 \leq R(5, 5) \leq 48.
  \end{equation*}

  For larger numbers the lower and upper bounds can differ wildly, for example
  \begin{equation*}
    798 \leq R(10, 10) \leq \numprint{17 730}.
  \end{equation*}
\end{remark}
