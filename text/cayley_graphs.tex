\section{Cayley graphs}\label{sec:cayley_graphs}

\paragraph{Cayley graphs}

\begin{remark}\label{rem:cayley_graph_overview}
  We investigate here how the Cayley graphs of a \hyperref[def:group_presentation]{finitely generated} \hyperref[def:group]{group}, which we define in \fullref{def:cayley_graph}, are defined in the literature:
  \begin{itemize}
    \item \incite[226]{Knauer2019AlgebraicGraphTheory} ascribe the concept to the paper \bycite[1]{Cayley1878Graphs}. It was published before \cite{König1986Graphentheorie}, which, as explained in \fullref{rem:graph_definition}, is considered the first book on graph theory. Thus, the concept of graph is not yet established at that time, and Cayley speaks of \enquote{diagrams} instead. He only sketches how such diagrams can be constructed for finite \hyperref[def:symmetric_group]{symmetric groups}, with directed arcs \hyperref[def:labeled_set]{labeled} differently whenever \enquote{no one of them can be obtained from the others by compounding them together in any manner}.

    He gives an example of a diagram of \( S_{12} \) with red and black edges.

    Based on Cayley's work, \incite[156]{König1990GraphTheory} constructs a \hyperref[def:directed_graph]{directed graph} from a finite group \( G \) by taking the members of \( G \) as vertices and connecting each pair of vertices with an arc \( (x, y) \) colored by \( x^{-1} y \).

    In an attempt to stay consistent with Cayley, in \fullref{def:cayley_graph} we consider directed Cayley graphs induced by generating sets.

    \item \incite[def. 7.3.1]{Knauer2019AlgebraicGraphTheory} define a \enquote{Cayley graph} of a finite \hyperref[def:semigroup]{semigroup} (or even \hyperref[rem:magma_terminology]{magma}) \( G \) and a \enquote{connection set} \( C \) as the directed graph whose vertices are the members of \( G \) and whose edges are pairs \( (x, xc) \), where \( x \) is in \( G \) and \( c \) is in \( C \).

    They define both colors and uncolored, as well as directed and undirected Cayley graphs. For directed graphs of groups the color can be determined from the arc, but not in general, so here it makes sense to explicitly label an arc with a color.

    We use a mixture of their two definitions for the case of finitely-generated groups.

    Later, in \cite[def. 7.3.2]{Knauer2019AlgebraicGraphTheory}, they define a \enquote{K\"onig graph} of an arbitrary group \( G \) and a generating subset \( C \) that is closed under inverses.

    \incite[def. 5.4.3]{Steinberg2012RepresentationTheory} calls a \enquote{Cayley graph} what Knauer call a \enquote{K\"onig graph}, but restricted to finite groups, and with the condition that the neutral element is not in \( C \).

    \item \incite[exerc. II.8.6]{Aluffi2009Algebra} defines a \enquote{Cayley graph} of a finitely generated group \( G \) and a set \( C \) of generators as the directed graph whose vertices are members of \( G \) and whose edges are pairs \( (x, xc) \), where \( x \) is in \( G \) and \( c \) is in \( C \), colored with \( c \).

    We generally follow Aluffi's definition but do not require \( C \) to be finite.

    \incite[18]{Rotman2017AdvancedModernAlgebraPart2} provides a similar definition, but without explicitly mentioning coloring.

    \incite[def. 2.1.3]{HadelerMüller2017CellularAutomata} also provide a similar definition, but take the \hyperref[def:multigraph_orientation]{underlying undirected graph} instead, also without mentioning coloring. We highlight some subtleties of undirected Cayley graphs in \fullref{rem:undirected_cayley_graph}.
  \end{itemize}
\end{remark}

\begin{definition}\label{def:cayley_graph}\mimprovised
  Fix a \hyperref[def:group]{group} \( G \) and let \( C \) be any set of generators. We will call \( C \) the \term[en=connection set (\cite[def. 7.3.1]{Knauer2019AlgebraicGraphTheory})]{connection set} and regard its elements as \hyperref[def:set_coloring]{colors}.

  We define the \term[en=Cayley graph (\cite[def. 2.1.3]{HadelerMüller2017CellularAutomata})]{Cayley graph} \( \Gamma(G, C) \) as the \hyperref[def:directed_graph]{directed graph} whose vertices are the elements of \( G \) and whose arcs are \( (x, xc) \), where \( x \) is a member of \( G \) and \( c \) is a color from \( C \). We call the neutral element the \term[en=origin (\cite[def. 2.1.3]{HadelerMüller2017CellularAutomata})]{origin} of \( \Gamma(G, C) \).
\end{definition}
\begin{comments}
  \item In a directed graph, we can determine \( c \) from the arc \( (x, xc) \) by left multiplication with \( x^{-1} \), hence it is immaterial whether we explicitly label the arc with \( c \) or not.

  \item We avoid undirected Cayley graphs due to the subtleties described in \fullref{rem:undirected_cayley_graph}.

  \item There are also other definitions based on Cayley's work; see \fullref{rem:cayley_graph_overview} for a broader discussion.
\end{comments}

\begin{example}\label{ex:def:cayley_graph}
  We list examples of \hyperref[def:cayley_graph]{Cayley graphs}:
  \begin{thmenum}
    \thmitem{ex:def:cayley_graph/finite_cyclic}\mcite[example 7.3.3]{Knauer2019AlgebraicGraphTheory} The Cayley graph of the same group varies based on the connection set.

    \Cref{fig:ex:def:cayley_graph/finite_cyclic} shows several Cayley graphs for the \hyperref[def:cyclic_group]{cyclic group} \( C_5 \).
    \begin{itemize}
      \item \Cref{fig:ex:def:cayley_graph/finite_cyclic/1} is simply a counter-clockwise \hyperref[def:multigraph_orientation]{orientation} of corresponding \hyperref[def:cycle_graph]{cycle graph}.

      \item \Cref{fig:ex:def:cayley_graph/finite_cyclic/14} features \( a^{-1} = a^4 \) as an additional generator, which adds for each arc its opposite --- the Cayley graph then corresponds to the \hyperref[def:graph_functors/simple_doubling]{doubling} of the cycle graph.

      \item Finally, \cref{fig:ex:def:cayley_graph/finite_cyclic/12} features \( a^2 \) as a generator in addition to \( a \), which results in an orientation of the corresponding \hyperref[def:complete_graph]{complete graph}.
    \end{itemize}

    \begin{figure}[!ht]
      \begin{subcaptionblock}{0.3\textwidth}
        \centering
        \includegraphics[page=1]{output/ex__def__cayley_graph__finite_cyclic}
        \caption{\( \Gamma(C_5, \set{ a }) \)}\label{fig:ex:def:cayley_graph/finite_cyclic/1}
      \end{subcaptionblock}
      \hfill
      \begin{subcaptionblock}{0.3\textwidth}
        \centering
        \includegraphics[page=2]{output/ex__def__cayley_graph__finite_cyclic}
        \caption{\( \Gamma(C_5, \set{ a, a^4 }) \)}\label{fig:ex:def:cayley_graph/finite_cyclic/14}
      \end{subcaptionblock}
      \hfill
      \begin{subcaptionblock}{0.3\textwidth}
        \centering
        \includegraphics[page=3]{output/ex__def__cayley_graph__finite_cyclic}
        \caption{\( \Gamma(C_5, \set{ a, a^2 }) \)}\label{fig:ex:def:cayley_graph/finite_cyclic/12}
      \end{subcaptionblock}
      \caption{The \hyperref[def:cayley_graph]{Cayley graphs} for the \hyperref[def:cyclic_group]{cyclic group} \( C_5 \) with different connection sets.}\label{fig:ex:def:cayley_graph/finite_cyclic}
    \end{figure}

    The example is generalized in \fullref{thm:cayley_graph_of_finite_cyclic_group}.

    \thmitem{ex:def:cayley_graph/free} The Cayley graph of the \hyperref[def:free_group]{free group} \( F(\set{ a, b }) \) with \( a \) and \( b \) as generators is an orientation of an infinite \hyperref[def:tree]{tree}. A fragment of this tree is shown in \cref{fig:ex:def:cayley_graph/free}.

    \begin{figure}[!ht]
      \centering
      \includegraphics[page=1]{output/ex__def__cayley_graph__free}
      \caption{A fragment of the \hyperref[def:cayley_graph]{Cayley graph} of the \hyperref[def:free_group]{free group} \( F(\set{ a, b }) \) with the canonical connection set. Solid and dashed lines correspond to arcs colored by \( a \) and \( b \), correspondingly.}\label{fig:ex:def:cayley_graph/free}
    \end{figure}

    The example is generalized in \fullref{thm:cayley_graph_of_free_group}.

    \thmitem{ex:def:cayley_graph/dihedral} The triangular \hyperref[def:dihedral_group]{dihedral group} \( D_3 \) has a Cayley graph that is an orientation of the \hyperref[def:petersen_graph]{generalized Petersen graph} \( P_{3,1} \). This is illustrated in \fullref{fig:ex:def:cayley_graph/dihedral}.

    \begin{figure}[!ht]
      \centering
      \includegraphics[page=1]{output/ex__def__cayley_graph__dihedral}
      \caption{A fragment of the \hyperref[def:cayley_graph]{Cayley graph} of the triangular \hyperref[def:dihedral_group]{dihedral group} \( D_3 \) with the canonical connection set. Solid lines denote rotation, while dashed lines denote reflection.}\label{fig:ex:def:cayley_graph/dihedral}
    \end{figure}

    \thmitem{ex:def:cayley_graph/point_lattice} In the \hyperref[def:euclidean_space]{Euclidean space} \( \BbbR^n \), consider the \hyperref[def:point_lattice]{point lattice} \( L \) with \hyperref[def:point_lattice_basis]{basis vectors} \( v_1, \ldots, v_n \).

    Then, by definition,
    \begin{equation*}
      L = \set[\Bigg]{ \sum_{k=1}^n a_k v_k \given* (a_1, \ldots, a_n) \in \BbbZ^n }.
    \end{equation*}

    Consider the \hyperref[def:free_abelian_group]{free abelian group} \( \BbbZ^n \) with its \hyperref[def:sequence_space]{standard basis} \( e_1, \ldots, e_n \). The vertices of the \hyperref[def:cayley_graph]{Cayley graph} \( \Gamma(\BbbZ^n, \set{ e_1, \ldots, e_n }) \) can be placed on the points of \( L \), and can then be connected to form a \hyperref[def:graph_geometric_realization/embedding]{graph embedding}.
  \end{thmenum}
\end{example}

\begin{proposition}\label{thm:def:cayley_graph}
  \hyperref[def:cayley_graph]{Cayley graphs} have the following basic properties:
  \begin{thmenum}
    \thmitem{thm:def:cayley_graph/connected} Every Cayley graph is \hyperref[def:graph_connectedness/weak]{weakly connected}.

    \thmitem{thm:def:cayley_graph/loops} The Cayley graph \( \Gamma(G, C) \) has \hyperref[def:directed_multigraph/loop]{loops} if and only if \( C \) contains the neutral element.

    \thmitem{thm:def:cayley_graph/degree} All vertices of the Cayley graph \( \Gamma(G, C) \) have an \hyperref[def:graph_cardinality/directed_degree]{in-degree} and \hyperref[def:graph_cardinality/directed_degree]{out-degree} equal to \( \card C \).

    \thmitem{thm:def:cayley_graph/finitely_generated} A group is \hyperref[def:group_presentation]{finitely generated} if and only if it has a \hyperref[def:graph_cardinality/local]{locally finite} Cayley graph.
  \end{thmenum}
\end{proposition}
\begin{proof}
  \SubProofOf{thm:def:cayley_graph/connected} Fix a Cayley graph \( \Gamma(G, C) \).

  Let \( x \) and \( y \) be members of \( G \). Since \( C \) is a generating set, \fullref{thm:group_presentation_generating_set} implies that \( \Gamma(G, C) \) is isomorphic to \( F(C) / {\cong} \) for some congruence \( {\cong} \) on the free group.

  Then there exists a string \( z_1 \ldots z_n \) of (inclusions of) members of \( C \) and their inverses such that \( x^{-1} y \cong z_1 \ldots z_n \). We can thus multiply by \( x \) to obtain \( x z_1 \ldots, z_n \cong y \). The string \( x z_1 \ldots z_n \) induces a \hyperref[def:graph_walk/generalized]{generalized walk} from \( x \) to \( y \).

  Since \( x \) and \( y \) were arbitrary, we conclude that \( \Gamma(G, C) \) is weakly connected.

  \SubProofOf{thm:def:cayley_graph/loops} Trivial.

  \SubProofOf{thm:def:cayley_graph/degree} For every vertex \( v \) in \( \Gamma(G, C) \), the edges starting at \( v \) are \( (v, vc) \) for every \( c \) in \( C \), while the edges ending at \( v \) are \( (vc^{-1}, v) \) for every \( c \) in \( C \).

  \SubProofOf{thm:def:cayley_graph/finitely_generated} Follows from \fullref{thm:def:cayley_graph/degree}.
\end{proof}

\begin{proposition}\label{thm:undirected_cayley_graph}
  Let \( C \) be a generating set of the group \( G \) and suppose that \( C \) is closed under inverses. Consider the \hyperref[def:cayley_graph]{Cayley graph} \( \Gamma(G, C) \).

  Its \hyperref[def:multigraph_orientation]{underlying undirected graph} \hyperref[def:graph_functors/simple_forgetful]{\( U_S \)}\( (\Gamma(G, C)) \) has an edge \( \set{ x, y } \) if and only if \( x^{-1} y \) belongs to \( C \).
\end{proposition}
\begin{defproof}
  \SufficiencySubProof Fix an edge \( \set{ x, y } \) in \( U_S(\Gamma(G, C)) \).

  Then \( (x, y) \) or \( (y, x) \) is an arc in \( \Gamma(G, C) \).

  \begin{itemize}
    \item If \( (x, y) \) is an arc, then \( y = xc \) for some color \( c \) in \( C \), and \( c = x^{-1} y \).
    \item Otherwise, \( (y, x) \) is an arc, and \( x = yc \) for some color \( c \). Then \( c = y^{-1} x \) is in \( C \), and hence so is \( c^{-1} = x^{-1} y \).
  \end{itemize}

  In both cases, \( x^{-1} y \) is in \( C \).

  \NecessitySubProof Suppose that, for every edge \( \set{ x, y } \) of \( \Gamma(G, C) \), \( x^{-1} y \) is in \( C \).

  Suppose that \( (x, y) \) is not an arc in \( \Gamma(G, C) \). Then \( y^{-1} x = (x^{-1} y)^{-1} \) is in \( C \), hence
  \begin{equation*}
    (y, y y^{-1} x) = (y, x)
  \end{equation*}
  is an edge colored by \( y^{-1} x \).
\end{defproof}

\begin{remark}\label{rem:undirected_cayley_graph}
  \Fullref{thm:undirected_cayley_graph} highlights some subtleties of defining an undirected \hyperref[def:cayley_graph]{Cayley graph} \( \oline{\Gamma}(G, C) \). We assume here the natural condition
  \begin{equation*}
    \oline{\Gamma}(G, C) = \hyperref[def:graph_functors/simple_forgetful]{U_S}(\Gamma(G, C)).
  \end{equation*}

  \begin{itemize}
    \item We must require \( C \) to be closed under inverses, which makes constructing even the simple examples from \fullref{ex:def:cayley_graph} less convenient.

    \item The edge \( \set{ x, y } \) can be colored via either \( x^{-1} y \) or \( y^{-1} x \), which are in general distinct. There is no canonical choice, so authors like \incite[def. 2.1.3]{HadelerMüller2017CellularAutomata} who discuss undirected Cayley graphs avoid defining colorings altogether.
  \end{itemize}
\end{remark}

\begin{proposition}\label{thm:cayley_graph_of_finite_cyclic_group}
  The finite group \( G \) with \( n > 2 \) elements is a \hyperref[def:cyclic_group]{cyclic group} if and only if it has a generating set \( C \) such that the \hyperref[def:cayley_graph]{Cayley graph} \( \Gamma(G, C) \) is isomorphic to a \hyperref[def:cycle_graph]{cycle graph} \hyperref[def:multigraph_orientation]{oriented} so that \( (k, m) \) is an arc if \( m = k + 1 \pmod n \).
\end{proposition}
\begin{comments}
  \item This is visualized in \fullref{ex:def:cayley_graph/finite_cyclic}.
  \item Since the vertices of \( \Gamma(G, C) \) are simply the group members, the group and the graph necessarily have the same cardinality.
\end{comments}
\begin{proof}
  \SufficiencySubProof Consider the cyclic group \( C_n = \braket{ a \given a^n = e } \) and the Cayley graph \( \Gamma(C_n, \set{ a }) \).

  Then \( (a^k, a^m) \) is an edge of \( C_n \) if and only if \( a^{m - k} \) equals the only color \( a \), which happens if \( m - k = 1 \pmod n \).

  \NecessitySubProof Conversely, let \( G \) be group of finite order \( n \) and suppose that, for some generating set \( C \), there exists a \hyperref[def:directed_graph/homomorphism]{graph isomorphism} \( \varphi: \Gamma(G, C) \to H \), where \( H \) is a cycle graph with \( n \) vertices oriented so that \( (k, m) \) is an arc if \( m = k + 1 \pmod n \).

  For every arc \( (x, xa) \), we have
  \begin{equation*}
    \varphi(xa) = \varphi(x) + 1 \pmod n.
  \end{equation*}

  Since \( 0 \leq \varphi(x) < n \) and similarly for \( \varphi(xa) \), we conclude that \( \varphi(x) = n - 1 \) if and only if \( \varphi(xa) = 0 \), and otherwise \( \varphi(xa) = \varphi(x) + 1 \). Thus, if \( (x, xb) \) is also an arc, we have \( \varphi(xa) = \varphi(xb) \).

  Since \( \varphi \) is an isomorphism, it follows that \( xa = xb \), and we can cancel \( x \) to conclude that \( a = b \).

  Therefore, \( G \) is generated by a singleton set \( C \), and is hence a cyclic group.
\end{proof}

\begin{proposition}\label{thm:cayley_graph_of_free_group}\mcite[exerc. II.8.6]{Aluffi2009Algebra}
  Fix a set \( C \) and consider the \hyperref[def:free_group]{free group} \( F(C) \) with canonical inclusion \( \iota: C \to F(C) \). The corresponding \hyperref[def:cayley_graph]{Cayley graph} \( \Gamma(F(C), \iota[C]) \) is \hyperref[def:acyclic_graph]{acyclic}.
\end{proposition}
\begin{comments}
  \item Distinguishing between \( c \) and \( \iota(c) \) is essential due to our construction of the free group.
\end{comments}
\begin{proof}
  Consider the following \hyperref[def:graph_walk]{walk} in \( \Gamma(F(A), A) \):
  \begin{equation}\label{eq:thm:cayley_graph_of_free_group/proof/walk}
    a_0 \reloset {d_1} \to a_1 \reloset {d_2} \to \cdots \reloset {d_{n-1}} \to a_{n-1} \reloset {d_n} \to a_n.
  \end{equation}

  By construction, \( d_k \) is a member of \( \iota[C] \) such that \( a_k = a_{k-1} d_k \), hence
  \begin{equation*}
    d_k = a_{k-1}^{-1} a_k.
  \end{equation*}

  Then
  \begin{equation*}
    a_0^{-1} a_n = d_1 \cdots d_n.
  \end{equation*}

  Aiming at a contradiction, suppose that \eqref{eq:thm:cayley_graph_of_free_group/proof/walk} is a \hyperref[def:graph_cycle]{cycle}, i.e. it satisfies \fullref{def:graph_cycle/direct}.

  Then \( n > 1 \) and \( a_1, \ldots, a_n \) are all distinct. Hence, \( d_k \) is not the identity for any \( k = 1, \ldots, n \).

  Furthermore, \( a_0 = a_n \), and thus \( d_1 \cdots d_n = e \).

  Following the definition \eqref{eq:def:free_group/reduction} of reduction in \( F(C) \), we conclude that there exists some index \( k_0 \) and some color \( c \) in \( C \) such that \( d_{k_0} = c^+ = \iota(c) \) and \( d_{k_0 + 1} = c^- \) or vice versa. But by construction of \( \Gamma(F(C), C) \), both \( d_{k_0} \) and \( d_{k_0 + 1} \) must be inclusions of members of \( C \).

  The obtained contradiction shows that \eqref{eq:thm:cayley_graph_of_free_group/proof/walk} cannot be a cycle.
\end{proof}

\paragraph{Distance in Cayley graphs}

\begin{definition}\label{def:graph_geodesic}\mcite[1]{Bollobás1984GeodesicsInOrientedGraphs}
  Among all \hyperref[def:graph_walk/directed]{directed walks} between two vertices in an \hyperref[rem:arbitrary_kind_graph]{arbitrary-kind graph or hypergraph}, the ones having shortest length are called \term[ru=геодезические (цепи) (\cite[\S 7.2.5]{Новиков2013ДискретнаяМатематика})]{geodesic}.
\end{definition}
\begin{comments}
  \item \incite[1]{Bollobás1984GeodesicsInOrientedGraphs} defines geodesics for directed graphs, but the notion is generally ambiguous since it is possible to consider both directed and generalized walks.

  Geodesics are defined for undirected graphs by \incite[14]{Harary1969GraphTheory} and \incite[\S 7.2.5]{Новиков2013ДискретнаяМатематика}.
\end{comments}

\begin{definition}\label{def:graph_geodesic_distance}\mcite[def. 1.1.5, rem. 1.1.6]{Knauer2019AlgebraicGraphTheory}
  We define the \term[ru=расстояние (\cite[34]{ЕмеличевИПр1990ТеорияГрафов}), en=distance (\cite[def. 1.1.5]{Knauer2019AlgebraicGraphTheory})]{geodesic distance} between two vertices in an \hyperref[def:undirected_graph]{undirected graph} or \hyperref[def:hypergraph]{hypergraph} as the length of the corresponding \hyperref[def:graph_geodesic]{geodesic walks}, and \( \infty \) if there is no walk between them.
\end{definition}
\begin{comments}
  \item We avoid defining a geodesic distance for directed graphs because it is not clear whether we should consider directed or generalized walks. Specifically for \hyperref[def:cayley_graph]{Cayley graphs}, we introduce the Cayley metric in \fullref{def:cayley_metric} based on generalized walks.

  \item We regard \( \infty \) as the corresponding \hyperref[def:extended_real_number]{extended real number}, but it is only important for it to be larger than any integer. For the general usage of the symbol \( \infty \), see \fullref{rem:lemniscate_symbol}.

  \item There are different conventions on handling the case where two vertices are not connected.
  \begin{itemize}
    \item Our convention of using a placeholder infinite value is also used by \incite[24]{Harary1969GraphTheory} and \incite[6]{Bollobás1998ModernGraphTheory}, whose books are dedicated to finite undirected graphs. \incite[def. 1.1.5, rem. 1.1.6]{Knauer2019AlgebraicGraphTheory} use this convention for more general graphs.

    \item \incite[202]{Зыков2004ТеорияГрафов} also uses the above convention, but considers a generalization in finite undirected graphs with weighted edges. In this case finiteness becomes a necessary assumption.

    \item \incite[34]{ЕмеличевИПр1990ТеорияГрафов} and \incite[\S 7.2.5]{Новиков2013ДискретнаяМатематика} assume that the graph is finite, undirected and connected. \incite[69]{König1990GraphTheory} allows the graph to be infinite.
  \end{itemize}
\end{comments}

\begin{proposition}\label{thm:graph_geodesic_distance_is_metric}
  The \hyperref[def:graph_geodesic_distance]{geodesic distance} in a \hyperref[def:graph_connectedness/undirected]{connected} \hyperref[def:undirected_graph]{undirected graph} or \hyperref[def:hypergraph]{hypergraph} is a \hyperref[def:metric_space]{metric} on the vertices.
\end{proposition}
\begin{proof}
  Trivial.
\end{proof}

\begin{definition}\label{def:cayley_distance}\mimprovised
  We define the \term[en=Cayley distance (\cite[def. 2.1.10]{HadelerMüller2017CellularAutomata})]{Cayley distance} between the vertices \( x \) and \( y \) in the \hyperref[def:cayley_graph]{Cayley graph} \( \Gamma(G, C) \) as the minimum of the lengths of all \hyperref[def:graph_walk/generalized]{generalized walks} from \( x \) to \( y \).
\end{definition}
\begin{comments}
  \item We adapt the definition in \bycite[def. 2.1.10]{HadelerMüller2017CellularAutomata} to directed Cayley graphs.
\end{comments}
\begin{defproof}
  \Fullref{thm:def:cayley_graph/connected} implies that \( \Gamma(G, C) \) is weakly connected, so at least one walk exists, and the distance is well-defined.
\end{defproof}

\begin{proposition}\label{thm:cayley_and_geodesic_distance}
  \hyperref[def:cayley_distance]{Cayley distance} in a \hyperref[def:cayley_graph]{Cayley graph} coincides with the \hyperref[def:graph_geodesic_distance]{geodesic distance} in the \hyperref[def:multigraph_orientation]{underlying undirected graph}.
\end{proposition}
\begin{proof}
  Trivial.
\end{proof}

\begin{corollary}\label{thm:cayley_distance_is_metric}
  The \hyperref[def:cayley_distance]{Cayley distance} in a \hyperref[def:cayley_graph]{Cayley graph} is a \hyperref[def:metric_space]{metric} on its vertices.
\end{corollary}
\begin{proof}
  Follows from \fullref{thm:cayley_and_geodesic_distance} and \fullref{thm:graph_geodesic_distance_is_metric}.
\end{proof}

\begin{proposition}\label{thm:cayley_distance_integer_lattice}
  The \hyperref[def:cayley_distance]{Cayley distance} in the \hyperref[def:cayley_graph]{Cayley graph} of the \hyperref[def:free_abelian_group]{free abelian group} \( \BbbZ^n \) with the \hyperref[def:sequence_space]{standard basis} coincides with the (metric corresponding to the) \hyperref[def:p_norm]{\( 1 \)-norm}.
\end{proposition}

\paragraph{Neighborhoods}

\begin{definition}\label{def:cayley_graph_neighborhood}\mcite[def. 2.2.1]{HadelerMüller2017CellularAutomata}
  Fix a \hyperref[def:cayley_graph]{Cayley graph} \( \Gamma(G, C) \), and assume given a finite set \( D_e \) of vertices, which we call a \term{neighborhood} of the origin.

  We define a neighborhood of any vertex \( x \) as the translation
  \begin{equation*}
    D_x \coloneqq x \cdot D_e = \set{ xv \given v \in D_e }.
  \end{equation*}

  \begin{thmenum}
    \thmitem{def:cayley_graph_neighborhood/von_neumann}\mcite[def. 2.2.3]{HadelerMüller2017CellularAutomata} We define the \term{von Neumann neighborhood} as the \hyperref[def:metric_space/ball]{unit ball} with respect to the \hyperref[def:cayley_distance]{Cayley distance}, i.e. the set of vertices adjacent to the origin.

    \thmitem{def:cayley_graph_neighborhood/moore}\mcite[def. 2.2.2]{HadelerMüller2017CellularAutomata} Consider \( \Gamma(\BbbZ^n, \set{ e_1, \ldots, e_n }) \), where \( e_1, \ldots, e_n \) is the \hyperref[def:sequence_space]{standard basis} of the \hyperref[def:free_abelian_group]{\( \BbbZ^n \)}.

    We define the \term{Moore neighborhood} as the unit ball with respect to the \hyperref[def:p_norm]{uniform norm}.
  \end{thmenum}
\end{definition}

\begin{example}\label{ex:def:cayley_graph_neighborhood}
  We list examples of \hyperref[def:cayley_graph_neighborhood]{Cayley graph neighborhoods}:
  \begin{thmenum}
    \thmitem{ex:def:cayley_graph_neighborhood/finite_cyclic} In \( \Gamma(C_n, \set{ a }) \), only \( a \) and \( a^{-1} = a^{n-1} \) are adjacent to the origin, so the von Neumann neighborhood is \( \set{ e, a, a^{-1} } \).

    \begin{figure}[!ht]
      \centering
      \includegraphics[page=1]{output/ex__def__cayley_graph_neighborhood__finite_cyclic}
      \caption{The \hyperref[def:cayley_graph_neighborhood/von_neumann]{von Neumann neighborhood} (of the origin) in the \hyperref[def:cayley_graph]{Cayley graph} \( \Gamma(C_5, \set{ a }) \).}\label{fig:ex:def:cayley_graph_neighborhood/finite_cyclic}
    \end{figure}

    \thmitem{ex:def:cayley_graph_neighborhood/free} In \( \Gamma(F(\set{ a, b }, \set{ a, b }) \), the neighbors of the origin are \( a \), \( b \), \( a^{-1} \) and \( b^{-1} \).

    \begin{figure}[!ht]
      \centering
      \includegraphics[page=1]{output/ex__def__cayley_graph_neighborhood__free}
      \caption{The \hyperref[def:cayley_graph_neighborhood/von_neumann]{von Neumann neighborhood} in the \hyperref[def:cayley_graph]{Cayley graph} \( \Gamma(F(\set{ a, b }, \set{ a, b }) \).}\label{fig:ex:def:cayley_graph_neighborhood/free}
    \end{figure}

    \thmitem{ex:def:cayley_graph_neighborhood/integer_lattice} In \( \Gamma(\BbbZ^n, \set{ e_1, \ldots, e_n }) \), the von Neumann neighborhood consists of \( e \), the basis vectors and their inverses.

    The Moore neighborhood additionally contains, for all indices \( i \) and \( j \), \( e_i + e_j \), \( e_i - e_j \) as well as their inverses.

    \begin{figure}[!ht]
      \begin{subcaptionblock}{0.45\textwidth}
        \centering
        \includegraphics[page=1]{output/ex__def__cayley_graph_neighborhood__integer_lattice}
        \caption{The \hyperref[def:cayley_graph_neighborhood/von_neumann]{von Neumann neighborhood}.}\label{fig:ex:def:cayley_graph_neighborhood/integer_lattice/von_neumann}
      \end{subcaptionblock}
      \hfill
      \begin{subcaptionblock}{0.45\textwidth}
        \centering
        \includegraphics[page=2]{output/ex__def__cayley_graph_neighborhood__integer_lattice}
        \caption{The \hyperref[def:cayley_graph_neighborhood/moore]{Moore neighborhood}.}\label{fig:ex:def:cayley_graph_neighborhood/integer_lattice/moore}
      \end{subcaptionblock}
      \caption{\hyperref[def:cayley_graph_neighborhood]{Neighborhoods} of the \hyperref[def:cayley_graph]{Cayley graph} \( \Gamma(\BbbZ^n, \set{ e_1, \ldots, e_n }) \).}\label{fig:ex:def:cayley_graph_neighborhood/integer_lattice}
    \end{figure}
  \end{thmenum}
\end{example}
