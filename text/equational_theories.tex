\section{Equational theories}\label{sec:equational_theories}

\paragraph{Equational theories}

\begin{definition}\label{def:fol_equational_theory}\mimprovised
  Consider some \hyperref[def:fol_theory]{first-order theory} \( T \) over a signature with no predicate symbols.

  We say that \( T \) is an \term[en=equational theory (\cite[def. 3.5.3]{BaaderNipkow2012TermRewriting})]{equational theory} if it has an axiomatization consisting of \hyperref[def:fol_quantifier_closure]{universal closures} of \hyperref[def:fol_equation]{equations}.
\end{definition}
\begin{comments}
  \item The definition is based in \cite[def. 3.5.3]{BaaderNipkow2012TermRewriting}, who, without dealing with logic, present an equational theory as a set of pairs of terms (their terms resemble our first-order terms). The general notion is attributed to Garrett Birkhoff's discussion of universal algebra in \cite[ch. VI]{Birkhoff1967LatticeTheory}.
\end{comments}

\begin{proposition}\label{thm:fol_equational_theory_models}
  We will show that the \hyperref[def:fol_semantics/model]{models} of an \hyperref[def:fol_equational_theory]{equational theory} \( T \) are particularly well-behaved:

  \begin{thmenum}
    \thmitem{thm:fol_equational_theory_models/image} If \( \mscrX \) is a model of \( T \), for every \hyperref[def:fol_homomorphism]{homomorphism} \( h: \mscrX \to \mscrT \), the image \( h[X] \) is (the domain of) a model of \( T \).

    \thmitem{thm:fol_equational_theory_models/substructure} Every \hyperref[def:fol_substructure]{substructure} of a model of \( T \) is also a model of \( T \).

    \thmitem{thm:fol_equational_theory_models/direct_product} The \hyperref[def:fol_direct_product]{direct product} of models of \( T \) is also a model of \( T \).

    \thmitem{thm:fol_equational_theory_models/quotient} Every \hyperref[def:fol_quotient_structure]{quotient structure} of a model of \( T \) is also a model of \( T \).

    \thmitem{thm:fol_equational_theory_models/functions} If \( \mscrX \) is a model of \( T \), for every set \( S \), the \hyperref[def:fol_structure_of_functions]{structure of all functions} \( \fun(S, \mscrX) \) is also a model of \( T \).

    \thmitem{thm:fol_equational_theory_models/lattice} If \( \mscrX \) is a model of \( T \), the \hyperref[def:fol_lattice_of_substructures]{lattice of substructures} of \( \mscrX \) consists of models of \( T \).
  \end{thmenum}
\end{proposition}
\begin{proof}
  Let \( E \) be an equational axiomatization of \( T \), with the universal quantifiers stripped (i.e. \( E \) is a set of \hyperref[def:fol_equation]{equations}). These equations are \hyperref[def:fol_universal_validity]{universally valid} in every model of \( T \).

  \SubProofOf{thm:fol_equational_theory_models/image} Since there are no predicate symbols, every homomorphism is strong. \Cref{thm:def:fol_function_respect/atomic} implies that \( h: \mscrX \to \mscrY \) \hyperref[def:fol_function_respect/formula]{respects} atomic formulas, hence it respects the axioms from \( E \). Then, by definition, it also respects all formulas from \( T \).

  \SubProofOf{thm:fol_equational_theory_models/substructure} Let \( \mscrY = (Y, J) \) be a substructure of \( \mscrX = (X, I) \).

  For every equation \( (\tau \syneq \sigma) \) from \( E \) and every variable assignment \( v \) into \( Y \), \cref{thm:fol_equality_characterization} implies that \( \Bracks{\tau}_\mscrX^v = \Bracks{\sigma}_\mscrX^v \).

  Since \( Y \) is closed under signature function application, \( \Bracks{\tau}_\mscrX^v \) is in \( Y \), and, since \( J \) restricts the interpretation of these symbols, \( \Bracks{\tau}_\mscrX^v = \Bracks{\tau}_\mscrY^v \). Similarly, \( \Bracks{\sigma}_\mscrX^v = \Bracks{\sigma}_\mscrY^v \). \Cref{thm:fol_equality_characterization} implies that \( \Bracks{\tau \syneq \sigma}_\mscrY^v = \semtop \).

  Generalizing on \( v \) and on \( (\tau \syneq \sigma) \), we conclude that all equations from \( E \) are universally valid in \( \mscrY \), and hence \( \mscrY \) is a model of \( T \).

  \SubProofOf{thm:fol_equational_theory_models/direct_product} Let \( \seq{ \mscrX_k }_{k \in \mscrK} \) be models of \( T \), and consider their product \( \prod_{k \in \mscrK} \mscrX_k \).

  By induction on the term \( \tau \), we can show that, assuming \( \tau \) has \( n \) free variables,
  \begin{equation*}
    \Bracks{\tau}_{\prod_{k \in \mscrX} \mscrX_k}\parens[\big]{ \seq{ a_{1,k} }_{k \in \mscrK}, \ldots, \seq{ a_{n,k} }_{k \in \mscrK} } = \seq[\big]{\Bracks{\tau}_{\mscrX_k}(a_{1,k}, \ldots, a_{n,k}) }_{k \in \mscrK}.
  \end{equation*}

  Hence, an equation is universally valid in \( \prod_{k \in \mscrX} \mscrX_k \) if and only if it is universally valid in \( \mscrX_k \) for every \( k \) simultaneously.

  This holds in particular for the equations from \( E \), which we have assumed are universally valid in \( \mscrX_k \). They are thus also universally valid in the product \( \prod_{k \in \mscrX} \mscrX_k \), i.e. the product is thus a model of \( T \).

  \SubProofOf{thm:fol_equational_theory_models/quotient} Let \( \mscrX = (X, I) \) be a model of \( T \) and let \( {\cong} \) be a congruence on \( \mscrX \).

  \Cref{thm:fol_equational_theory_models/image} implies that the quotient homomorphism \( \pi: \mscrX \to \mscrX / {\cong} \) is surjective, so \cref{thm:fol_equational_theory_models/image} implies that the image \( \mscrX / {\cong} \) of \( \pi \) is a model of \( T \).

  \SubProofOf{thm:fol_equational_theory_models/functions} Let \( \mscrX = (X, I) \) be a model of \( T \). Fix some set \( S \) and consider the structure of functions \( \fun(S, \mscrX) \).

  By induction on the term \( \tau \), we can show that, assuming \( \tau \) has \( n \) free variables,
  \begin{equation*}
    \Bracks{\tau}_{\fun(S, \mscrX)}(a_1, \ldots, a_n) = \parens[\big]{ s \mapsto \Bracks{\tau}_\mscrX(a_1(s), \ldots, a_n(s)) }.
  \end{equation*}

  Fix an equation \( (\tau \syneq \sigma) \) from \( E \). It is by assumption universally valid in \( \mscrX \), so \cref{thm:fol_equality_characterization} implies that \( \Bracks{\tau}_\mscrX^v = \Bracks{\sigma}_\mscrX^v \) for every variable assignment \( v \). This holds in particular for the values assigned by functions \( s: S \to X \).

  Therefore, \( (\tau \syneq \sigma) \) is universally valid in \( \fun(S, \mscrX) \). Generalizing on \( (\tau \syneq \sigma) \), we conclude that the equations from \( E \) are universally valid in \( \fun(S, \mscrX) \), and hence \( \fun(S, \mscrX) \) is a model of \( T \).

  \SubProofOf{thm:fol_equational_theory_models/lattice} Follows from \cref{thm:fol_equational_theory_models/substructure}.
\end{proof}

\paragraph{Theory of pure equality}

\begin{definition}\label{def:pure_equality}\mimprovised
  The simplest \hyperref[def:fol_equational_theory]{equational theory} is simply the empty theory over the empty signature.

  Its only terms are variables and its atomic formulas are either logical constants or equality formulas. For this reason, we call it the \term{theory of pure equality}.
\end{definition}

\begin{proposition}\label{thm:pure_equality_isomorphism}
  The \hyperref[def:pure_equality]{theory of pure equality} \( T_= \) (over the empty signature \( \Sigma_= \)) has the following basic properties:
  \begin{thmenum}
    \thmitem{thm:pure_equality_isomorphism/initial_theory} The theory \( T_= \) is an \hyperref[def:universal_objects/initial]{initial object} in any \hyperref[def:fol_theory/category]{category of theories} \( \cat{Th} \): every theory over every signature is an \hyperref[def:fol_theory/extension]{extension} of \( T_= \).

    \thmitem{thm:pure_equality_isomorphism/structure} Every \hi{nonempty} set induces a structure over \( \Sigma_= \) (with an empty interpretation).

    \thmitem{thm:pure_equality_isomorphism/substructure} For a given model \( \mscrX = (X, I) \), every subset \( A \) of \( X \) is (the domain of) a \hyperref[def:fol_substructure]{substructure}.

    \thmitem{thm:pure_equality_isomorphism/congruence} For a given model \( \mscrX = (X, I) \), every \hyperref[def:equivalence_relation]{equivalence relation} on \( X \) is a \hyperref[def:fol_congruence]{congruence}.

    \thmitem{thm:pure_equality_isomorphism/quotient_structure} For a given model \( \mscrX = (X, I) \), every \hyperref[def:set_partition]{partition} of \( X \) is (the domain of) a \hyperref[def:fol_quotient_structure]{quotient structure}.

    \thmitem{thm:pure_equality_isomorphism/model} Every structure over \( \Sigma_= \) is a model of \( T_= \).

    \thmitem{thm:pure_equality_isomorphism/function} Every function \( f: X \to Y \) between the structures \( \mscrX = (X, I) \) and \( \mscrY = (Y, J) \) over \( \Sigma_= \) is a \hyperref[def:fol_homomorphism]{homomorphism}.

    \thmitem{thm:pure_equality_isomorphism/isomorphism} Two structures over \( \Sigma_= \) are \hyperref[def:fol_isomorphism]{isomorphic} if and only if they have the same \hyperref[thm:cardinality_existence]{cardinality}.

    \thmitem{thm:pure_equality_isomorphism/finite_elementary_equivalence} Two \hi{finite} structures over \( \Sigma_= \) are \hyperref[def:elementary_equivalence]{elementary equivalent} if and only if they have the same cardinality.
  \end{thmenum}
\end{proposition}
\begin{proof}
  \SubProofOf{thm:pure_equality_isomorphism/initial_theory} Trivial.

  \SubProofOf{thm:pure_equality_isomorphism/structure} Vacuous.

  \SubProofOf{thm:pure_equality_isomorphism/substructure} Also vacuous.

  \SubProofOf{thm:pure_equality_isomorphism/congruence} Also vacuous.

  \SubProofOf{thm:pure_equality_isomorphism/quotient_structure} Follows from \cref{thm:pure_equality_isomorphism/congruence}.

  \SubProofOf{thm:pure_equality_isomorphism/model} Since \( T_= \) is the consequence closure of the empty set, \cref{thm:institutional_satisfaction_closure} implies that every structure \( \mscrX \) over \( \Sigma_= \) satisfying \( \varnothing \) also satisfies \( T_= \).

  \SubProofOf{thm:pure_equality_isomorphism/function} The function \( f: X \to Y \) is vacuously a homomorphism because there are no function or predicate symbols.

  \SubProofOf{thm:pure_equality_isomorphism/isomorphism} Let \( \mscrX = (X, I) \) and \( \mscrY = (Y, J) \) be two structures over \( \Sigma_= \).

  \SufficiencySubProof* Let \( h: X \to Y \) be an isomorphism. Then it is bijective, hence \( X \) and \( Y \) have the same cardinality.

  \NecessitySubProof* Suppose that \( X \) and \( Y \) have the same cardinality. Then there exists a bijective map \( h: X \to Y \), which is vacuously an isomorphism because there are no function or predicate symbols.

  \SubProofOf{thm:pure_equality_isomorphism/finite_elementary_equivalence}

  \SufficiencySubProof* \Cref{thm:def:elementary_equivalence/finite} implies that, if two structures are finite and elementarily equivalent, they are isomorphic.

  \NecessitySubProof* \Cref{thm:pure_equality_isomorphism/isomorphism} implies that, if two structures have the same cardinality, they are isomorphic. Then \cref{thm:def:elementary_embedding/isomorphism} implies that they are elementarily equivalent.
\end{proof}

\paragraph{Term models}

\begin{definition}\label{def:fol_free_term_model}\mimprovised
  Fix an \hyperref[def:fol_equational_theory]{equational theory} \( T \) over the \hyperref[def:fol_signature]{signature} \( \Sigma \) with no relation symbols. Also fix an \hyperref[def:formal_language/alphabet]{alphabet} \( V \), whose elements we will treat as \hyperref[con:free_construction/indeterminate]{indeterminates} (i.e. generalized syntactic variables).

  We will build a \hyperref[def:fol_semantics/model]{model} \( \mscrT_T(V) \) of \( T \) consisting of generalized \hyperref[def:fol_term_ast]{abstract syntax trees} of terms over \( \Sigma \). We will call \( \mscrT_T(V) \) a \term[en=term algebra (\cite[def. 3.4.1]{BaaderNipkow2012TermRewriting})]{free term model} of \( T \).

  \begin{thmenum}
    \thmitem{def:fol_free_term_model/intermediate} First, let \( P \) be the set of syntax trees obtained via \fullref{thm:recursively_defined_abstract_syntax} from the function symbols in \( \Sigma \) and from the indeterminates \( V \) considered as nullary symbols.

    On \( P \), we define the interpretation \( I(f) \) of the \( n \)-ary function \( f \) as
    \begin{equation*}
      I(f)(\tau_1, \ldots, \tau_n) \coloneqq B_+^f(\tau_1, \ldots, \tau_n),
    \end{equation*}
    where \( B_+^f(\tau_1, \ldots, \tau_n) \) is the tree obtained by \hyperref[def:ordered_tree_grafting]{grafting} \( \tau_1, \ldots, \tau_n \) to a new root labeled by \( f \). By construction, this tree belongs to \( \mscrP \).

    Then \( \mscrP \coloneqq (P, I) \) is a \hyperref[def:fol_structure]{structure} over \( \Sigma \). At this point is is unrelated to the theory \( T \).

    \thmitem{def:fol_free_term_model/congruence} Fix a particular equational axiomatization \( A \) of \( T \). Let \( {\cong} \) be the \hyperref[def:fol_congruence]{congruence} constructed via \fullref{thm:recursively_defined_relations} by extending the equivalence relation rules from \cref{thm:generated_equivalence_relation} with the following:

    \begin{thmenum}
      \thmitem{def:fol_free_term_model/congruence/functions} For every function symbol \( f \) of arity \( n \), we impose
      \begin{equation*}\taglabel[\ensuremath{ \logic{TM}_f }]{inf:def:fol_free_term_model/congruence/functions}
        \begin{prooftree}
          \hypo{ \tau_1 \cong \tau_1' }
          \hypo{ \cdots }
          \hypo{ \tau_n \cong \tau_n' }
          \infer3[\ref{inf:def:fol_free_term_model/congruence/functions}]{ I(f)(\tau_1, \ldots, \tau_1) \cong I(f)(\tau_1', \ldots, \tau_n') }
        \end{prooftree}
      \end{equation*}

      \thmitem{def:fol_free_term_model/congruence/axioms} For every axiom \( \varphi = \qforall {x_1} \cdots \qforall {x_n} (\tau \syneq \sigma) \) in \( A \), we impose
      \begin{equation*}\taglabel[\ensuremath{ \logic{TM}_{\tau \syneq \sigma} }]{inf:def:fol_free_term_model/congruence/axioms}
        \begin{prooftree}
          \hypo{ \rho_1 \cong \rho_1' }
          \hypo{ \cdots }
          \hypo{ \rho_n \cong \rho_n' }
          \infer3[\ref{inf:def:fol_free_term_model/congruence/axioms}]{ \Bracks{\tau}_\mscrP(\rho_1, \ldots, \rho_n) \cong \Bracks{\sigma}_\mscrP(\rho_1', \ldots, \rho_n') }
        \end{prooftree}
      \end{equation*}
    \end{thmenum}

    Any two axiomatizations of \( T \) are by definition \hyperref[def:fol_semantics/equivalence]{equivalent}, so the congruence does not depend on the particular choice of equational axiomatization \( A \). Note that the congruence would be trivial without the last rule.

    \thmitem{def:fol_free_term_model/quotient} The desired free term model \( \mscrT_T(V) \) is then the \hyperref[def:fol_quotient_structure]{quotient structure} \( \mscrP / {\cong} \).
  \end{thmenum}
\end{definition}
\begin{comments}
  \item This construction resembles term models used for proving \fullref{thm:fol_model_existence_theorem}, which we discuss in \cref{def:fol_term_model}.

  \item The terminology of free constructions, discussed in \cref{con:free_construction}, applies to all free term models.
\end{comments}

\begin{theorem}[Free term model universal property]\label{thm:fol_free_term_model_universal_property}\mimprovised
  The \hyperref[def:fol_free_term_model]{free term model} \( \mscrT_T(V) = (M, I) \) is the unique up to a unique isomorphism \hyperref[def:fol_semantics/model]{model} of \( T \) that satisfies the following \hyperref[rem:universal_mapping_property]{universal mapping property}:
  \begin{displayquote}
    For every model \( \mscrY = (Y, J) \) of \( T \) and every function \( e: V \to Y \) into the universe \( Y \) of \( \mscrY \), there exists a unique \hyperref[def:fol_homomorphism]{first-order homomorphism} \( \Phi_e: \mscrT_T(V) \to \mscrY \) such that the following diagram commutes:
    \begin{equation}\label{eq:thm:fol_term_model_universal_property}
      \begin{aligned}
        \includegraphics[page=1]{output/thm__fol_free_term_model_universal_property}
      \end{aligned}
    \end{equation}
  \end{displayquote}
\end{theorem}
\begin{comments}
  \item Via \cref{rem:universal_mapping_property}, \( V \mapsto \mscrT_T(V) \) becomes \hyperref[def:category_adjunction]{left adjoint} to the \hyperref[def:fol_theory_forgetful_functor/set]{forgetful functor}
  \begin{equation*}
    U: \ucat{Mod}(T) \to \ucat{Set},
  \end{equation*}
  which sends a model of \( T \) to its universe.

  \item Many theories have distinct definitions for their free objects; these include \hyperref[def:free_group]{free groups}, \hyperref[def:free_semimodule]{free (semi)modules} and \hyperref[def:polynomial_algebra]{polynomial algebras}. They satisfy the same universal property, so, by \cref{thm:functor_adjoint_uniqueness}, they are isomorphic to the corresponding free term models. See \cref{ex:def:fol_free_term_model/free_monoid} for a broader discussion.

  The terminology of free constructions, discussed in \cref{con:free_construction}, applies to all of them.
\end{comments}
\begin{proof}
  Let \( \pi: \mscrP \to \mscrT_T(V) \) be the quotient map from the construction of term models.

  We want the function \( \Phi_e: M \to Y \) to extend \( e \), thus the image of the coset \( \pi(B_+^v) \), where \( B_+^v \) is an indeterminate tree, must be \( e(v) \). This is well-defined because no other tree is congruent to \( B_+^v \).

  We also want \( \Phi_e \) to be a homomorphism. Thus, for the function application tree coset \( \pi(B_+^f(\tau_1, \ldots, \tau_n)) \), \( \Phi_e \) must recursively apply itself to the subtrees. The rule \ref{inf:def:fol_free_term_model/congruence/axioms} makes sure that \( \Phi_e \) does not depend on the choice of function application tree in a coset. This leads to the only possible definition
  \begin{equation*}
    \Phi_e(\pi(M)) \coloneqq \begin{cases}
      e(v),                                                                  &M = \pi(B_+^v), \\
      J(f)\parens[\big]{ \Phi_e(\pi(\tau_1)), \ldots, \Phi_e(\pi(\tau_n)) }, &M = \pi(B_+^f(\tau_1, \ldots, \tau_n)),
    \end{cases}
  \end{equation*}
\end{proof}

\begin{example}\label{ex:def:fol_free_term_model}
  We list examples of \hyperref[def:fol_free_term_model]{free term models}:
  \begin{thmenum}
    \thmitem{ex:def:fol_free_term_model/free_magma} A simple example of a free term model is based on the empty theory over a single binary operation.

    As explained in \cref{rem:magma_terminology}, models of this theory are sometimes referred to as \enquote{magmas}, so for this example let us call \( \mscrT_\varnothing(V) \) the \term{free magma} over \( V \).

    The congruence in the construction of \( \mscrT_\varnothing(V) \) is trivial, so the cosets are singleton sets. We thus do not need to distinguish between the syntax trees and their cosets.

    The following trees correspond to elements more conveniently written as \( (xy)z \) and \( x(yz) \):
    \begin{equation*}
      \includegraphics[page=1]{output/ex__def__fol_free_term_model}
      \qquad
      \includegraphics[page=2]{output/ex__def__fol_free_term_model}
    \end{equation*}

    Here \( x \), \( y \) and \( z \) are metalinguistic variables denoting arbitrary syntax trees.

    \thmitem{ex:def:fol_free_term_model/free_semigroup} By adding associativity to the empty theory, we obtain a very different free term model.

    The trees above corresponding to \( (xy)z \) and \( x(yz) \) are now congruent. It is thus less convenient to regard them as trees and more convenient to regard them as \hyperref[def:ordered_tuple]{ordered triples}. This extends to quadruples, quintuples and so forth. The grafting product of trees corresponds to concatenation of tuples.

    Thus, we may regard the free term model \( \mscrT_{T_1}(V) \) over the \hyperref[def:semigroup/theory]{first-order theory of semigroups} \( T_1 \) as the \hyperref[def:formal_language/kleene_plus]{Kleene plus} \( V^+ \), the set of nonempty strings of indeterminates. More formally, we have just demonstrated an isomorphism between \( \mscrT_{T_1}(V) \) and \( V^+ \).

    \thmitem{ex:def:fol_free_term_model/free_monoid} This naturally leads to the definition of free monoid in \cref{def:free_monoid} as the \hyperref[def:formal_language/kleene_star]{Kleene star} \( V^* \).

    Indeed, consider the free term model \( \mscrT_{T_2}(V) \) over the \hyperref[def:monoid/theory]{first-order theory of monoids} \( T_2 \). In \( \mscrT_{T_2}(V) \), we expect the equalities \( x = ex = xe \), where \( x \) is arbitrary and \( e = B_+^{\synneutral} \) is the neutral element in \( \mscrT_{T_2}(V) \). Thus, the following syntax trees must be regarded as congruent to \( x \):
    \begin{equation*}
      \includegraphics[page=3]{output/ex__def__fol_free_term_model}
      \qquad
      \includegraphics[page=4]{output/ex__def__fol_free_term_model}
    \end{equation*}

    This ensures that the grafting product with the neutral element tree \( e = B_+^{\synneutral} \) acts as concatenation with the empty string.

    Again, we have just demonstrated an isomorphism between \( \mscrT_{T_2}(V) \) and \( V^* \). However, even if an explicit isomorphism is not obvious, the universal properties provide one of us. In this case \( \mscrT_{T_2}(V) \) satisfies \fullref{thm:fol_free_term_model_universal_property}, while \( V^* \) satisfies \fullref{thm:free_monoid_universal_property}. Both are left adjoint to the same forgetful functor \( U: \cat{Mon} \to \cat{Set} \), so, by \cref{thm:functor_adjoint_uniqueness}, they are isomorphic.

    \thmitem{ex:def:fol_free_term_model/free_commutative_monoid} Further in one direction is the free commutative monoid, defined in \cref{def:free_commutative_monoid} as the \hyperref[def:free_semimodule]{free semimodule} \( \BbbN^{\oplus V} \).

    Consider the theory \( T_3 \) obtained by adding the commutativity axiom \eqref{eq:def:binary_operation/commutative} to \( T_2 \). Due to \fullref{thm:free_commutative_monoid_universal_property}, we again have an implicit isomorphism between \( \mscrT_{T_3}(V) \) and \( \BbbN^{\oplus V} \).

    This also follows naturally because, in the free term model \( \mscrT_{T_3}(V) \), the condition \( xy = yx \) requires us to additionally regard the following trees as congruent:
    \begin{equation*}
      \includegraphics[page=5]{output/ex__def__fol_free_term_model}
      \qquad
      \includegraphics[page=6]{output/ex__def__fol_free_term_model}
    \end{equation*}

    It thus becomes convenient to regard cosets in \( \mscrT_{T_3}(V) \) as \hyperref[def:multiset]{multisets} of \hyperref[def:multiset_cardinality/infinite]{finite order} over \( V \), i.e. as finitely-supported functions from \( V \) to \( \BbbN \).

    \thmitem{ex:def:fol_free_term_model/free_group} Instead of extending monoids to commutative monoids, we may extend them to \hyperref[def:group]{groups}. Let \( T_4 \) be the \hyperref[def:group/theory]{first-order theory of groups}.

    Compared to the free term model \( \mscrT_{T_2}(V) \) of monoids, in \( \mscrT_{T_4}(V) \) each element \( x \) has a corresponding inverse \( x^{-1} \), encoded by the tree
    \begin{equation*}
      \includegraphics[page=7]{output/ex__def__fol_free_term_model}
    \end{equation*}

    The theory additionally requires us to regard the following trees as congruent, for any \( x \), to the neutral element \( e = B_+^{\synneutral} \):
    \begin{equation*}
      \includegraphics[page=8]{output/ex__def__fol_free_term_model}
      \qquad
      \includegraphics[page=9]{output/ex__def__fol_free_term_model}
    \end{equation*}

    Symbolically, this amounts to reducing \( x^{-1}x \) and \( xx^{-1} \) to \( e \).

    The free group \( F(V) \), as defined in \cref{def:free_group}, provides an alternative definition that leads to a similar result. The corresponding universal property is \fullref{thm:free_group_universal_property}.

    \thmitem{ex:def:fol_free_term_model/polynomial_algebra} A more complicated example stems from the \hyperref[def:algebra_over_semiring/commutative]{first-order theory of commutative algebras} \( T_5 \) over a fixed \hyperref[def:semiring]{(semi)ring} \( R \).

    The free term model \( \mscrT_{T_5}(V) \) requires many adjustments. However, due to \fullref{thm:polynomial_algebra_universal_property}, we know that the free term model is isomorphic to the \hyperref[def:polynomial_algebra]{polynomial algebra} \( R[V] \).

    This is shown explicitly in \cref{ex:def:fol_free_term_model}.

    \thmitem{ex:def:fol_free_term_model/lattices} Unlike the examples discussed above, for the \hyperref[def:lattice/theory]{theory of lattices} \( T_6 \), we use free term model \( \mscrT_{T_6}(V) \) when defining free lattices in \cref{def:free_lattice}.
  \end{thmenum}
\end{example}
