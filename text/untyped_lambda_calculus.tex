\subsection{Untyped lambda calculus}\label{subsec:untyped_lambda_calculus}

\paragraph{Untyped \( \synlambda \)-terms}

\begin{definition}\label{def:untyped_lambda_term}\mimprovised
  We introduce \enquote{\( \synlambda \)-terms}, \enquote{\( \synlambda \)-applications} and \enquote{\( \synlambda \)-abstractions} via the following \hyperref[def:formal_grammar/schema]{grammar schema}:
  \begin{bnf*}
    \bnfprod{variable}    {\bnfpn{Short Latin identifier}} \\
    \bnfprod{application} {\bnftsq{\( ( \)} \bnfsp \bnfpn{term} \bnfsp \bnfpn{term} \bnfsp \bnftsq{\( ) \)}} \\
    \bnfprod{abstraction} {\bnftsq{\( ( \)} \bnfsp \bnftsq{\( \synlambda \)} \bnfsp \bnfpn{variable} \bnfsp \bnftsq{.} \bnfsp \bnfpn{term} \bnfsp \bnftsq{\( ) \)}} \\
    \bnfprod{term}        {\bnfpn{variable} \bnfor \bnfpn{application} \bnfor \bnfpn{abstraction}}
  \end{bnf*}
\end{definition}
\begin{comments}
  \item These notions are based on \cite[352]{Church1932}. It is not clear why has Church used \( \synlambda \) as part of the grammar.

  \item We use the nonterminal \( \bnfpn{Short Latin identifier} \) rather than \( \bnfpn{Latin identifier} \) because otherwise it would be unclear how to parse applications --- in \( (xyz) \), are the variables \( xy \) and \( z \) or \( x \) and \( yz \)?. An alternative would be to introduce a separator for applications, but that would only make our notation more cumbersome.

  \item We need all the parentheses so that we can prove unambiguity in \fullref{thm:lambda_terms_are_unambiguous}. Informally, we will use the conventions from \fullref{rem:propositional_formula_parentheses} regarding parentheses.
\end{comments}

\begin{remark}\label{rem:lambda_object_language}
  Similarly to \fullref{rem:object_language_dots}, we want to introduce a convention for distinguishing concrete variable literals from metalingual variables denoting unspecified variables. Unlike there, however, we will not use dots, but rather use different letters:
  \begin{itemize}
    \item The letters \( x \), \( y \) and \( z \), commonly used for metalingual variables elsewhere, will be used for literals here.
    \item The letters \( u \), \( v \) and \( w \), also often used for metalingual variables, will be used for metalingual variables here.
    \item All other letters may have different meaning depending on the context.
  \end{itemize}
\end{remark}

\begin{proposition}\label{thm:lambda_terms_are_unambiguous}
  The grammar of \hyperref[def:untyped_lambda_term]{\( \synlambda \)-terms} is \hyperref[def:grammar_ambiguity]{unambiguous}.
\end{proposition}
\begin{proof}
  Similarly to \fullref{thm:propositional_formulas_are_unambiguous}, the proof is again based on \fullref{ex:natural_number_arithmetic_grammar/unambiguous}.
\end{proof}

\begin{definition}\label{def:untyped_lambda_term_ast}
  We implicitly associate with each \( \synlambda \)-term \( M \) an \hyperref[con:abstract_syntax_tree]{abstract syntax tree} \( T(M) \) defined as follows:
  \begin{thmenum}
    \thmitem{def:untyped_lambda_term_ast/variable} If \( M \) is a variable, let \( T(M) \) be the singleton tree with value \( M \).

    \thmitem{def:untyped_lambda_term_ast/application} If \( M = NK \), assuming we have already built \( T(N) \) and \( T(K) \), we obtain \( T(M) \) by joining them by a new root with value \( \cdot \):
    \begin{equation*}
      \includegraphics[page=1]{output/def__lambda_term_ast}
    \end{equation*}

    \thmitem{def:untyped_lambda_term_ast/abstraction} If \( \varphi = \abs x N \), assuming we have already built \( T(N) \), we obtain \( T(M) \) by adding a new root with value \( \abs* x \):
    \begin{equation*}
      \includegraphics[page=2]{output/def__lambda_term_ast}
    \end{equation*}
  \end{thmenum}
\end{definition}
\begin{comments}
  \item Formally our ASTs for abstractions are not ideal because we need to \enquote{parse} the string at the root. They are, however, convenient for defining variable occurrences in \fullref{def:lambda_variable_occurrence}, and are also easier to parse visually.
\end{comments}

\begin{remark}\label{rem:lambda_term_parentheses}
  We use some \enquote{abuse-of-notation} syntactic conventions somewhat resembling \fullref{rem:propositional_formula_parentheses}:
  \begin{thmenum}
    \thmitem{rem:lambda_term_parentheses/outermost} As in \fullref{rem:propositional_formula_parentheses/outermost}, we avoid writing the outermost parentheses in terms like \( (xy) \) or \( ((xy)z) \).

    \thmitem{rem:lambda_term_parentheses/abstraction} We generally avoid writing parentheses around \( \synlambda \)-abstractions.

    Actually, it is not necessary to put parentheses around \( \synlambda \)-abstractions in order for the grammar to be unambiguous. Not requiring them, however, leads to the following unintuitive artifact.

    Consider the term \( \qabs x y z \). If parentheses are required only for \( \synlambda \)-abstractions but not for \( \synlambda \)-applications, it would unambiguously correspond to the following \hyperref[rem:lambda_term_ast]{abstract syntax tree}:
    \begin{equation*}
      \includegraphics[page=1]{output/rem__lambda_term_parentheses}
    \end{equation*}

    But it is reasonable to expect instead the following tree:
    \begin{equation*}
      \includegraphics[page=2]{output/rem__lambda_term_parentheses}
    \end{equation*}

    Since we will generally expect the latter, it is simpler to require parentheses around \( \synlambda \)-abstractions in the formal syntax and then, when convenient, avoid writing them within the metalogic.

    \thmitem{rem:lambda_term_parentheses/left_associative} We suppose that \( \synlambda \)-application is \hyperref[rem:binary_operation_syntax_trees/associativity]{left-associative}, which allows us to simplify notation in some cases like \eqref{eq:ex:def:untyped_lambda_term/combinator/s}, where we use \( xz(yz) \) rather than \( (xz)(yz) \).
  \end{thmenum}
\end{remark}

\begin{definition}\label{def:lambda_subterm}\mcite[def. 1A3]{Hindley1997}
  We define the set of all \term[ru=подтерм (\cite[189]{Герасимов2011})]{subterms} of a term \( M \) as follows:
  \begin{equation*}
    \op*{Subterm}(M) \coloneqq \begin{cases}
      \set{ x },                                                 &M = x, \\
      \set{M} \cup \op*{Subterm}(N) \cup \op*{Subterm}(K), &M = N K, \\
      \set{M} \cup \op*{Subterm}(N),                          &M = \qabs x N.
    \end{cases}
  \end{equation*}
\end{definition}
\begin{comments}
  \item Note that the analog of \fullref{thm:propositional_formula_characterization} no longer holds --- a variable that is a substring of a term is not necessarily a subterm --- see \fullref{ex:def:untyped_lambda_term/naive_subterm}.
\end{comments}

\begin{proposition}\label{thm:lambda_subterm_characterization}
  Suppose that the \hyperref[def:formal_language/substring]{substring} \( N \) of the \( \synlambda \)-term \( M \) is \hi{not a variable}. Then \( N \) is a \hyperref[def:lambda_subterm]{subterm} of \( M \) if and only if \( N \) is itself a \( \synlambda \)-term.
\end{proposition}
\begin{proof}
  We can give a proof similar to \fullref{thm:propositional_formula_characterization}.
\end{proof}

\begin{example}\label{ex:def:untyped_lambda_term}
  We list examples of \hyperref[def:untyped_lambda_term]{\( \synlambda \)-terms}:
  \begin{thmenum}
    \thmitem{ex:def:untyped_lambda_term/var} The simplest \( \synlambda \)-terms are the variables themselves, for example \( x \) or \( y \).

    \thmitem{ex:def:untyped_lambda_term/naive_subterm} Suppose that, as in \fullref{def:propositional_subformula} for propositional subformulas, we say that \( N \) is a subterm of \( M \) if it is itself a \hyperref[def:formal_language/substring]{substring} of \( M \).

    Then the term \( \qabs x y \) would have both \( x \) and \( y \) as subterms, while according to \fullref{def:untyped_lambda_term/subterm}, only \( y \) is a subterm.

    \thmitem{ex:def:untyped_lambda_term/combinator}\mcite[def. 1A10.1]{Hindley1997} The following terms have established names:
    \begin{align}
      &\qabs x x, \label{eq:ex:def:untyped_lambda_term/combinator/i}\tag{I} \\
      &\qabs x \qabs y yx, \nonumber\label{eq:ex:def:untyped_lambda_term/combinator/k}\tag{K} \\
      &\qabs x \qabs y \qabs z xz(yz), \label{eq:ex:def:untyped_lambda_term/combinator/s}\tag{S} \\
      &\qabs x (\qabs y xyy) (\qabs y xyy). \label{eq:ex:def:untyped_lambda_term/combinator/y}\tag{Y}
    \end{align}

    They are combinators in the sense that will be defined in \fullref{def:lambda_combinator}, and are referred to as such.
  \end{thmenum}
\end{example}

\paragraph{Variable scope}

\begin{definition}\label{def:lambda_abstractor}\mcite[def. 1A5]{Hindley1997}
  Given an abstraction \( M = \qabs x N \), we call \( \quantifier* \synlambda x \) the \term{abstractor} and \( N \) --- the \term{body} of \( M \).

  We say that the body \( N \) is the \term[ru=область действия (\cite[64]{Герасимов2011})]{scope} of the abstractor \( \quantifier* \synlambda x \) and that the abstractor \term{binds} the variable \( x \) in \( N \).
\end{definition}

\begin{definition}\label{def:lambda_variable_occurrence}\mimprovised
  An \term[ru=вхождение (\cite[64]{Герасимов2011})]{occurrence} of a variable \( x \) in a \( \synlambda \)-term \( M \) is a variable node labeled by \( x \) in the \hyperref[def:untyped_lambda_term_ast]{AST} of \( M \).

  We say that the occurrence is \term[ru=свободное (вхождение) (\cite[64]{Герасимов2011})]{free} in \( M \) if the root of the AST can be reached without passing through a corresponding abstractor. If a variable is not free, there exists an abstractor that binds it, and we say that the occurrence is \term[ru=связанное (вхождение) (\cite[64]{Герасимов2011})]{bound} in \( M \).
\end{definition}

\begin{definition}\label{def:lambda_variable_freeness}\mimprovised
  We say that a variable \( x \) is \term{free} in a \( \synlambda \)-term \( M \) if \( x \) has at least one \hyperref[def:lambda_variable_occurrence]{free occurrence} in \( M \) and \term{bound} if it has a \hyperref[def:lambda_variable_occurrence]{bound occurrence} in \( M \).
\end{definition}
\begin{comments}
  \item We may use the recursive definitions from \fullref{thm:lambda_variable_freeness_characterization} instead.
\end{comments}

\begin{example}\label{ex:def:lambda_variable_freeness}
  We list examples of free and bound variables and variables occurrences:
  \begin{thmenum}
    \thmitem{ex:def:lambda_variable_freeness/abstractor} The term \( \ref{eq:ex:def:untyped_lambda_term/combinator/i} = \qabs x x \) has exactly one occurrence of \( x \):
    \begin{equation*}
      \includegraphics[page=1]{output/def__lambda_variable_freeness}
    \end{equation*}

     This occurrence is bound in the term \( I \) and free in the subterm \( x \). Then \( x \) is a bound variable in \( I \) and a free variable in \( I \).

    \thmitem{ex:def:lambda_variable_freeness/both} The term \( M = I x = (\qabs x x) x \) has two occurrences of the variable \( x \):
    \begin{equation*}
      \includegraphics[page=2]{output/def__lambda_variable_freeness}
    \end{equation*}

    One of the occurrences is free, which makes \( x \) a free variable of \( M \), and one of the occurrences if bound, which makes \( x \) a bound variable of \( M \).
  \end{thmenum}
\end{example}

\begin{proposition}\label{thm:lambda_variable_freeness_characterization}
  The set of all \hyperref[def:lambda_variable_freeness]{free variables} of a \( \synlambda \)-term can be characterized as follows:
  \begin{equation*}
    \op*{Free}(M) \coloneqq \begin{cases}
      \set{ x },                              &M = x, \\
      \op*{Free}(N) \cup \op*{Free}(K), &M = N K, \\
      \op*{Free}(N) \setminus \set{ x },   &M = \qabs x N. \\
    \end{cases}
  \end{equation*}

  Similarly, the bound variables can be characterized via
  \begin{equation*}
    \op*{Bound}(M) \coloneqq \begin{cases}
      \varnothing,                              &M = y, \\
      \op*{Bound}(N) \cup \op*{Bound}(K), &M = N K, \\
      \op*{Bound}(N) \cup \set{ x },         &M = \qabs x N. \\
    \end{cases}
  \end{equation*}
\end{proposition}
\begin{proof}
  Straightforward.
\end{proof}

\begin{definition}\label{def:lambda_combinator}\mcite[def. 1A10]{Hindley1997}
  If a \( \synlambda \)-term has no \hyperref[def:lambda_variable_freeness]{free variables}, we say that it is \term{closed}. Closed terms are also called \term[ru=комбинаторы (\cite[188]{Герасимов2011})]{combinators}.
\end{definition}

\paragraph{Variable substitution}

\begin{definition}\label{def:lambda_substitution}\mcite[def. 1A7]{Hindley1997}
  We define the \term{substitution} of the variable \( x \) in the term \( M \) with \( N \) as
  \begin{subequations}
    \begin{empheq}[left={M[x \mapsto L]} \coloneqq \empheqlbrace]{align}
      &N,                                        &&M = x,                                                            \label{eq:def:lambda_substitution/var/direct} \\
      &x,                                        &&M = y \neq x,                                                     \label{eq:def:lambda_substitution/var/noop} \\
      &N[x \mapsto L] \thinspace K[x \mapsto L], &&M = NK,                                                           \label{eq:def:lambda_substitution/application} \\
      &M,                                        &&M = \qabs x N,                                                    \label{eq:def:lambda_substitution/abstraction/self} \\
      &M,                                        &&M = \qabs y N \T{and} x \not\in \op*{Free}(N),                 \label{eq:def:lambda_substitution/abstraction/noop} \\
      &\qabs y N[x \mapsto L],                   &&M = \qabs y N \T{and} x \in \op*{Free}(N) \T{and} \nonumber \\
      &                                          &&\quad y \not\in \op*{Free}(L),                                 \label{eq:def:lambda_substitution/abstraction/direct} \\
      &\qabs z N[y \mapsto z][x \mapsto L],      &&M = \qabs y N \T{and} x \in \op*{Free}(N) \T{and} \nonumber \\
      &                                          &&\quad y \in \op*{Free}(L) \T{and} z \not\in \op*{Free}(NP), \label{eq:def:lambda_substitution/abstraction/renaming}
    \end{empheq}
  \end{subequations}
  where \( z \) in \eqref{eq:def:lambda_substitution/abstraction/renaming} can be any variable free in neither \( N \) nor \( P \).

  We call the general process \eqref{eq:def:lambda_substitution/abstraction/renaming} of substituting a free variable with a suitable one \term[ru=переименование переменной (\cite[71]{Герасимов2011})]{variable renaming}.
\end{definition}
\begin{comments}
  \item For the sake of determinism, considering the identifier ordering discussed in \fullref{rem:grammar_rules_for_variables}, it makes sense for \( z \) in \eqref{eq:def:lambda_substitution/abstraction/renaming} to be the \enquote{smallest} suitable variable.

  \item The substitution rules \eqref{eq:def:lambda_substitution/abstraction/self} through \eqref{eq:def:lambda_substitution/abstraction/renaming} are adjusted so that \fullref{thm:lambda_substitution_free_variables} holds and the pitfalls of \fullref{ex:def:lambda_substitution} are avoided.

  \item The rule \eqref{eq:def:lambda_substitution/abstraction/self} is not strictly necessary, see \fullref{ex:def:lambda_substitution/self} for a discussion.
\end{comments}

\begin{proposition}\label{thm:lambda_substitution_free_variables}
  For any \( \synlambda \)-terms \( M \) and \( L \) and any variable \( x \), we have
  \begin{subequations}
    \begin{equation}\label{eq:thm:lambda_substitution_free_variables/free}
      \op*{Free}\parens[\Big]{ M[x \mapsto L] } = \parens[\Big]{ \op*{Free}(M) \setminus \set{ x } } \cup \op*{Free}(L)
    \end{equation}
    if \( x \) is free in \( M \) and
    \begin{equation}\label{eq:thm:lambda_substitution_free_variables/not_free}
      M[x \mapsto L] = M
    \end{equation}
  \end{subequations}
  otherwise.
\end{proposition}
\begin{proof}
  We will use \fullref{thm:induction_on_syntax_trees} on \( M \):
  \begin{itemize}
    \item If \( M \) is a variable, we have the following possibilities:
    \begin{itemize}
      \item If \( M = x \), then \( M[x \mapsto L] \reloset {\eqref{eq:def:lambda_substitution/var/direct}} = L \) and hence
      \begin{equation*}
        \op*{Free}\parens[\Big]{ M[x \mapsto L] }
        =
        \op*{Free}(L).
      \end{equation*}

      Furthermore,
      \begin{equation*}
        \parens[\Big]{ \underbrace{\op*{Free}(M)}_{\set{ x }} \setminus \set{ x } } \cup \op*{Free}(L)
        =
        \op*{Free}(L).
      \end{equation*}

      The two are equal, thus \eqref{eq:thm:lambda_substitution_free_variables/free} holds.

      \item If \( M = y \neq x \), then \( M[x \mapsto L] \reloset {\eqref{eq:def:lambda_substitution/var/noop}} = M \) and thus  \eqref{eq:thm:lambda_substitution_free_variables/non_free} holds.
    \end{itemize}

    \item If \( M = NK \), where the inductive hypothesis holds for \( N \) and \( K \), then
    \begin{align*}
      \op*{Free}\parens[\Big]{ M[x \mapsto L] }
      &\reloset {\eqref{eq:def:lambda_substitution/application}} =
      \op*{Free}\parens[\Big]{ N[x \mapsto L] \thinspace K[x \mapsto L] }
      = \\ &=
      \op*{Free}\parens[\Big]{ N[x \mapsto L] } \cup \op*{Free}\parens[\Big]{ K[x \mapsto L] }.
    \end{align*}

    We have several possibilities:
    \begin{itemize}
      \item If \( x \) is free in both \( N \) and \( K \), then
      \begin{balign*}
        &\phantom{{}={}}
        \op*{Free}\parens[\Big]{ N[x \mapsto L] } \cup \op*{Free}\parens[\Big]{ K[x \mapsto L] }
        \reloset {\T{ind.}} = \\ &=
        \parens[\bigg]{ \parens[\Big]{ \op*{Free}(N) \setminus \set{ x } } \cup \op*{Free}(L) }
        \cup
        \parens[\bigg]{ \parens[\Big]{ \op*{Free}(K) \setminus \set{ x } } \cup \op*{Free}(L) }
        = \\ &=
        \parens[\bigg]{ \parens[\Big]{ \op*{Free}(N) \cup \op*{Free}(K) } \setminus \set{ x } } \cup \op*{Free}(L)
        = \\ &=
        \parens[\Big]{ \op*{Free}(NK) \setminus \set{ x } } \cup \op*{Free}(L).
      \end{balign*}

      Then \eqref{eq:thm:lambda_substitution_free_variables/free} holds.

      \item If \( x \) is free in \( N \) but not \( K \), then
      \begin{balign*}
        &\phantom{{}={}}
        \op*{Free}\parens[\Big]{ N[x \mapsto L] } \cup \op*{Free}\parens[\Big]{ K[x \mapsto L] }
        \reloset {\T{ind.}} = \\ &=
        \parens[\bigg]{ \parens[\Big]{ \op*{Free}(N) \setminus \set{ x } } \cup \op*{Free}(L) } \cup \op*{Free}(K)
        = \\ &=
        \parens[\Big]{ \op*{Free}(N) \setminus \set{ x } } \cup \op*{Free}(L) \cup \op*{Free}(K)
        = \\ &=
        \parens[\Big]{ \op*{Free}(NK) \setminus \set{ x } } \cup \op*{Free}(L),
      \end{balign*}
      and again \eqref{eq:thm:lambda_substitution_free_variables/free} holds.

      \item If \( x \) is free in \( K \) but not in \( N \), we obtain \eqref{eq:thm:lambda_substitution_free_variables/free} as in the preceding case.

      \item If \( x \) is free in neither \( N \) nor \( K \), then
      \begin{equation*}
        \op*{Free}\parens[\Big]{ N[x \mapsto L] } \cup \op*{Free}\parens[\Big]{ K[x \mapsto L] }
        \reloset {\T{ind.}} =
        \op*{Free}(N) \cup \op*{Free}(K)
        =
        \op*{Free}(NK).
      \end{equation*}

      In this case, \eqref{eq:thm:lambda_substitution_free_variables/non_free} holds.
    \end{itemize}

    \item If \( M = \qabs y N \), where the inductive hypothesis holds for \( N \), we must again consider distinct cases.
    \begin{itemize}
      \item If \( y = x \), then \( M[x \mapsto L] \reloset {\eqref{eq:def:lambda_substitution/abstraction/self}} = M \), and thus \eqref{eq:thm:lambda_substitution_free_variables/non_free} holds.

      \item If \( y \neq x \) and \( x \) is not free in \( N \), then \( M[x \mapsto L] \reloset {\eqref{eq:def:lambda_substitution/abstraction/noop}} = M \), and again \eqref{eq:thm:lambda_substitution_free_variables/non_free} holds.

      \item If \( y \neq x \), \( x \) is free in \( N \) and \( y \) is not free in \( L \), then
      \begin{equation*}
        M[x \mapsto L]
        =
        (\qabs y N)[x \mapsto L]
        \reloset {\eqref{eq:def:lambda_substitution/abstraction/direct}} =
        \qabs y N[x \mapsto L].
      \end{equation*}

      For the corresponding free variables, we have
      \begin{balign*}
        \op*{Free}\parens[\Big]{ M[x \mapsto L] }
        &=
        \op*{Free}\parens[\Big]{ \qabs y N[x \mapsto L] }
        = \\ &=
        \op*{Free}\parens[\Big]{ N[x \mapsto L] } \setminus \set{ y }
        \reloset {\T{ind.}} = \\ &=
        \parens[\bigg]{ \parens[\Big]{ \op*{Free}(N) \setminus \set{ x } } \cup \op*{Free}(L) } \setminus \set{ y }
        = \\ &=
        \parens[\Big]{ \op*{Free}(N) \setminus \set{ x, y } } \cup \op*{Free}(L)
        = \\ &=
        \parens[\Big]{ \op*{Free}(M) \setminus \set{ x } } \cup \op*{Free}(L),
      \end{balign*}
      hence \eqref{eq:thm:lambda_substitution_free_variables/free} holds.

      \item Finally, if \( y \neq x \), \( x \) is free in \( N \) and \( y \) is free in \( L \), then, given a variable \( z \) not free in either, we have
      \begin{equation*}
        M[x \mapsto L]
        =
        (\qabs y N)[x \mapsto L]
        \reloset {\eqref{eq:def:lambda_substitution/abstraction/renaming}} =
        \qabs z N[y \mapsto z][x \mapsto L].
      \end{equation*}

      For the corresponding free variables, we have
      \begin{balign*}
        \op*{Free}\parens[\Big]{ M[x \mapsto L] }
        &=
        \op*{Free}\parens[\Big]{ \qabs z N[y \mapsto z][x \mapsto L] }
        = \\ &=
        \op*{Free}\parens[\Big]{ N[y \mapsto z][x \mapsto L] } \setminus \set{ z }
        \reloset {\T{ind.}} = \\ &=
        \parens[\bigg]{ \parens[\Big]{ \op*{Free}(N[y \mapsto z]) \setminus \set{ x } } \cup \op*{Free}(L) } \setminus \set{ z }
        = \\ &=
        \parens[\Big]{ \op*{Free}(N[y \mapsto z]) \setminus \set{ x, z } } \cup \op*{Free}(L)
        \reloset {\T{ind.}} = \\ &=
        \parens[\bigg]{ \parens[\bigg]{ \parens[\Big]{ \op*{Free}(N) \setminus \set{ y } } \cup \op*{Free}(z) } \setminus \set{ x, z } } \cup \op*{Free}(L)
        = \\ &=
        \parens[\Big]{ \op*{Free}(N) \setminus \set{ x, y } } \cup \op*{Free}(L)
        = \\ &=
        \parens[\Big]{ \op*{Free}(M) \setminus \set{ x } } \cup \op*{Free}(L)
      \end{balign*}
      hence \eqref{eq:thm:lambda_substitution_free_variables/free} holds.
    \end{itemize}
  \end{itemize}

  We have obtained either \eqref{eq:thm:lambda_substitution_free_variables/free} or \eqref{eq:thm:lambda_substitution_free_variables/non_free} in any case of the induction, and have thus proven the proposition.
\end{proof}

\begin{example}\label{ex:def:lambda_substitution}
  We list examples of \hyperref[def:lambda_substitution]{substitution} of \( \synlambda \)-terms:
  \begin{thmenum}
    \thmitem{ex:def:lambda_substitution/capture} The gist of \fullref{thm:lambda_substitution_free_variables} is that substitution avoids \enquote{capturing} free variables under the scope of some abstraction. Instead of \( (\qabs x y)[y \mapsto x] \) giving
    \begin{equation*}
      \qabs x y[y \mapsto x] = \qabs x x,
    \end{equation*}
    which \enquote{captures} \( x \) under the scope of the closest abstraction, our definition of substitution gives
    \begin{equation*}
      (\qabs x y)[y \mapsto x] = \qabs z x
    \end{equation*}
    for some new variable \( z \).

    \thmitem{ex:def:lambda_substitution/self} The rule \eqref{eq:def:lambda_substitution/abstraction/self} is not strictly necessary. Without it, in the substitution \( (\qabs x x)[x \mapsto y] \), we would instead use \eqref{eq:def:lambda_substitution/abstraction/renaming} and rename \( x \) to \( z \) first:
    \begin{equation*}
      (\qabs x x)[x \mapsto y]
      \reloset {\eqref{eq:def:lambda_substitution/abstraction/renaming}} =
      \qabs z x[x \mapsto z][x \mapsto y]
      =
      \qabs z z[x \mapsto y]
      =
      \qabs z z.
    \end{equation*}

    One could argue that \eqref{eq:def:lambda_substitution/abstraction/self} prevents unnecessary renaming, and this is true, but it doesn't help in \enquote{nested} cases like
    \begin{align*}
      \ref{eq:ex:def:untyped_lambda_term/combinator/k}[y \mapsto x]
      &=
      (\qabs x \qabs y yx)[y \mapsto x]
      = \\ &=
      (\qabs x \qabs y yx)[y \mapsto x]
      \reloset {\eqref{eq:def:lambda_substitution/abstraction/renaming}} = \\ &=
      \qabs z (\qabs y yx)[x \mapsto z][y \mapsto x]
      \reloset {\eqref{eq:def:lambda_substitution/abstraction/direct}} = \\ &=
      \qabs z (\qabs y yz)[y \mapsto x]
      \reloset {\eqref{eq:def:lambda_substitution/abstraction/renaming}} = \\ &=
      \qabs z \qabs t (yz)[y \mapsto t][y \mapsto x]
       = \\ &=
      \qabs z \qabs t tz.
    \end{align*}

    We include the aforementioned substitution rule \eqref{eq:def:lambda_substitution/abstraction/self} simply because it is simpler to work with during inductive proofs.
  \end{thmenum}
\end{example}

\paragraph{\( \alpha \)-equivalence}

\begin{definition}\label{def:untyped_lambda_term_alpha_equivalence}\mimprovised
  We say that the \( \synlambda \)-terms \( M \) and \( N \) are \( \alpha \)-\term{equivalent} and write \( M \aequiv N \) if any of the following conditions hold:
  \begin{thmenum}
    \thmitem{def:untyped_lambda_term_alpha_equivalence/variable} If both \( M = x = N \) for some variable \( x \).
    \thmitem{def:untyped_lambda_term_alpha_equivalence/application} If \( M = A B \) and \( N = C D \) and both \( A \aequiv C \) and \( B \aequiv D \).
    \thmitem{def:untyped_lambda_term_alpha_equivalence/abstraction} If \( M = \qabs a A \) and \( N = \qabs b B \) and \( A[a \mapsto c] \aequiv B[b \mapsto c] \) for every variable \( c \) free in neither \( A \) nor \( B \).
  \end{thmenum}
\end{definition}
\begin{comments}
  \item Two terms are \( \alpha \)-equivalent if they are \enquote{essentially the same} in the sense that they differ only by the names of the variables bound by abstractors.

  \item \Fullref{thm:def:untyped_lambda_term_alpha_equivalence/matching_abstraction} gives us an additional rule that can be used for simplifying induction proofs. We choose not to include it, however.
\end{comments}

\begin{example}\label{ex:def:untyped_lambda_term_alpha_equivalence}
  We list examples of \hyperref[def:untyped_lambda_term_alpha_equivalence]{\( \alpha \)-equivalence}:
  \begin{thmenum}
    \thmitem{ex:def:untyped_lambda_term_alpha_equivalence/combinator} The combinator \( \ref{eq:ex:def:untyped_lambda_term/combinator/i} \) has the same essential structure regardless of how we name its variables:
    \begin{equation*}
      I = \qabs x x \aequiv \qabs y y \aequiv \qabs a a \aequiv \cdots
    \end{equation*}

    \thmitem{ex:def:untyped_lambda_term_alpha_equivalence/freeing} In \fullref{ex:def:lambda_variable_freeness/both} we discussed how \( x \) is both bound and free in \( M = Ix = (\qabs x x) x \).

    We can use the term \( M' = (\qabs y y) x \) instead, where all occurrences of \( x \) are free and all occurrences of \( y \) are bound.

    This is generalized by \fullref{alg:alpha_equivalent_term_with_distinct_variables}.
  \end{thmenum}
\end{example}

\begin{algorithm}\label{alg:alpha_equivalent_term_with_distinct_variables}
  Fix a \( \synlambda \)-term \( M \). We will build an \hyperref[def:untyped_lambda_term_alpha_equivalence]{\( \alpha \)-equivalent} term \( M' \) where the \hyperref[def:lambda_variable_freeness]{free and bound variables} are distinct.

  We will use the auxiliary function
  \begin{equation*}
    D(M, G) \coloneqq \begin{cases}
      x,                                          &M = x, \\
      D(N, G) \thinspace D(K, G),                 &M = NK, \\
      \qabs y D(N, G \cup \set{ y })[x \mapsto y] &M = \qabs x N,
    \end{cases}
  \end{equation*}
  where \( G \) is a set of variables that should be avoided and \( y \not\in G \).

  We then simply define
  \begin{equation*}
    M' \coloneqq D(M, \op*{Free}(M)).
  \end{equation*}
\end{algorithm}

\paragraph{\( \beta \)-reduction}

\begin{definition}\label{def:beta_redex}\mcite[def. 1B1]{Hindley1997}
  A \( \beta \)-regex is a \( \synlambda \)-application whose first term is a \( \synlambda \)-abstraction, that is, a \( \synlambda \)-term of the form \( (\qabs x M) N \)
\end{definition}
