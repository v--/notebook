\section{First-order models}\label{sec:first_order_models}

\paragraph{Theories}

\begin{definition}\label{def:fol_theory}\mimprovised
  We restate here some definitions related to \hyperref[def:logical_theory]{logical theories}, adapted to \hyperref[def:first_order_logic]{first-order logic}.

  First and foremost, a (syntactic or semantic) \term[ru=теория (\cite[def. 3.1.1]{Герасимов2014Вычислимость}), en=theory (\cite[def. 2.4.1]{Hinman2005Logic})]{theory} is, as in \cref{def:general_logic_theory}, a set of sentences closed under logical consequence. For any set of sentences \( \Gamma \), we denote its consequence closure by \( \op*{Th}(\Gamma) \).

  \begin{thmenum}
    \thmitem{def:fol_theory/morphism} As in \cref{def:entailment_system_theory/morphism}, we call the \hyperref[def:fol_signature_category/morphisms]{first-order signature morphism} \( t: \Sigma \to \Theta \) a \term{theory morphisms} from \( (\Sigma, \Gamma) \) to \( (\Theta, \Delta) \) if the translation via \fullref{alg:fol_formula_signature_translation} of the formulas in \( \Gamma \) belong to \( \Delta \).

    \thmitem{def:fol_theory/extension} Of special interest is the case where \( t: \Sigma \to \Sigma^+ \) is the inclusion map of a \hyperref[def:fol_signature_extension]{signature extension}. In this case translation is does nothing, and \( t \) is a theory morphism from \( (\Sigma, \Gamma) \) to \( (\Sigma^+, \Gamma^+) \) if and only if \( \Gamma \) is a subset of \( \Gamma^+ \).

    We call \( \Gamma^+ \) an \term{extension} of \( \Gamma \).

    \thmitem{def:fol_theory/category} Based on theories and their morphisms, as per \cref{def:category_of_theories}, we have a (syntactic or semantic) category of theories \( \ucat{Th} \) for every \hyperref[def:grothendieck_universe]{Grothendieck universe} \( \mscrU \).

    \thmitem{def:fol_theory/conservative} As in \cref{def:entailment_system_theory/conservative}, we say that a theory morphism or extension is \term[en=conservative extension (\cite[180]{Hinman2005Logic})]{conservative} when \( \Gamma \nvDash_\Sigma \varphi \) implies \( \Delta \nvDash_\Theta \op*{Sen}(t)(\varphi) \).

    \thmitem{def:fol_theory/model}\mcite[def. 2.4.4]{Hinman2005Logic} As in \cref{def:theory_of_institutional_model}, we define the \term{theory} \( \op*{Th}(\mscrX) \) \hi{of} a structure \( \mscrX \) as the set of all sentences valid in \( \mscrX \).

    \thmitem{def:fol_theory/consistent}\mcite[def. 2.6.19]{Герасимов2014Вычислимость} We call a first-order theory \term[ru=непротиворечивое (множество формул)]{consistent} if no contradiction can be derived, i.e. if \( \synbot \) is not derivable.

    We discuss consistent theories in a general setting in \cref{def:consistent_set_of_sentences}, and give equivalent conditions for syntactic theories in \cref{thm:propositional_semantic_inconsistency} and for semantic theories in \cref{thm:propositional_semantic_inconsistency}. The latter also shows that semantic consistency is equivalent to \hyperref[def:propositional_semantics/satisfaction]{satisfiability}.

    \thmitem{def:fol_theory/complete}\mcite[def. 2.4.i]{Hinman2005Logic} We call a first-order theory \( \Gamma \) \term[ru=полное (множество формул) (\cite[def. 2.6.20]{Герасимов2014Вычислимость})]{complete} if, for every sentence \( \varphi \), \( \Gamma \) contains either \( \varphi \) or \( \synneg \varphi \) (or possibly both, if the theory is inconsistent).

    We will discuss complete theories in a general setting in \cref{def:complete_set_of_sentences}, and prove in \cref{thm:propositional_syntactically_complete_set} that it coincides with our notion when restricted to classical logic.
  \end{thmenum}
\end{definition}

\begin{proposition}\label{thm:def:fol_theory}
  \hyperref[def:fol_theory]{First-order theories} have the following basic properties:
  \begin{thmenum}
    \thmitem{thm:def:fol_theory/model_theory_consistent_complete} For every structure \( \mscrX \), the theory \( \op*{Th}(\mscrX) \) is consistent and complete.
  \end{thmenum}
\end{proposition}
\begin{proof}
  \SubProofOf{thm:def:fol_theory/model_theory_consistent_complete} Analogous to \cref{thm:def:propositional_theory/model_theory_consistent_complete}.
\end{proof}

\paragraph{Categories of models}

\begin{definition}\label{def:category_of_first_order_models}\mimprovised
  Consider the \hyperref[def:fol_institution]{first-order institution} over the \hyperref[def:grothendieck_universe]{Grothendieck universe} \( \mscrU \).

  We are interested in the category \( \ucat{Mod}(\Sigma) \) of \hyperref[def:fol_structure]{first-order structures} over the \hyperref[def:fol_signature]{signature} \( \Sigma \).

  More specifically, given a \hyperref[def:fol_theory]{first-order theory} \( \Gamma \), we are interested in the \hyperref[def:subcategory/full]{full subcategory} of structures in which \( \Gamma \) is \hyperref[def:fol_semantics/model]{valid}. We will denote this subcategory by \( \ucat{Mod}(\Gamma) \), and refer to it as the \term{category of \( \mscrU \)-small models of \( \Gamma \)}.
\end{definition}
\begin{comments}
  \item For established theories, we will introduce specific names. For example, we use \( \ucat{Grp} \) for the \hyperref[def:group/category]{category of groups} or \( \ucat{Lat} \) for \hyperref[def:lattice/theory]{category of lattices}.

  \item We have included the \( \mscrU \) prefix here for clarity, but will usually drop it.

  \item This definition is based on some remarks from \bycite[105]{GoguenBurstall1992Institutions}.
\end{comments}

\begin{remark}\label{rem:def:category_of_first_order_models}
  It is important that in \cref{def:category_of_first_order_models} we have required the subcategory \( \ucat{Mod}(\Gamma) \) to be full. This means that a structure from \( \ucat{Mod}(\Gamma) \) may or may not be a model of \( \Gamma \), but a homomorphism between models of \( \Gamma \) cannot have restrictions that do not depend solely on \( \Sigma \)
\end{remark}
