\section{First-order models}\label{sec:first_order_models}

\paragraph{Theories}

\begin{definition}\label{def:fol_theory}\mimprovised
  We restate here some definitions related to \hyperref[def:logical_theory]{logical theories}, adapted to \hyperref[def:first_order_logic]{first-order logic}.

  First and foremost, a (syntactic or semantic) \term[ru=теория (\cite[def. 3.1.1]{Герасимов2014Вычислимость}), en=theory (\cite[def. 2.4.1]{Hinman2005Logic})]{theory} is, as in \cref{def:general_logic_theory}, a set of sentences closed under logical consequence. For any set of sentences \( \Gamma \), we denote its consequence closure by \( \cat{Th}(\Gamma) \), and say that \( \Gamma \) \term{axiomatizes} \( \cat{Th}(\Gamma) \).

  \begin{thmenum}
    \thmitem{def:fol_theory/morphism} As in \cref{def:entailment_system_theory/morphism}, we call the \hyperref[def:fol_signature_category/morphisms]{first-order signature morphism} \( t: \Sigma \to \Theta \) a \term{theory morphisms} from \( (\Sigma, \Gamma) \) to \( (\Theta, \Delta) \) if the translation via \fullref{alg:fol_formula_signature_translation} of the formulas in \( \Gamma \) belong to \( \Delta \).

    \thmitem{def:fol_theory/extension} Of special interest is the case where \( t: \Sigma \to \Sigma^+ \) is the inclusion map of a \hyperref[def:fol_signature_extension]{signature extension}. In this case translation is does nothing, and \( t \) is a theory morphism from \( (\Sigma, \Gamma) \) to \( (\Sigma^+, \Gamma^+) \) if and only if \( \Gamma \) is a subset of \( \Gamma^+ \).

    We call \( \Gamma^+ \) an \term{extension} of \( \Gamma \).

    \thmitem{def:fol_theory/category} Based on theories and their morphisms, as per \cref{def:category_of_theories}, we have a category of (syntactic or semantic) theories \( \ucat{Th} \) for every \hyperref[def:grothendieck_universe]{Grothendieck universe} \( \mscrU \).

    \thmitem{def:fol_theory/conservative} As in \cref{def:entailment_system_theory/conservative}, we say that a theory morphism \( t: (\Sigma, \Gamma) \to (\Theta, \Delta) \) is \term[en=conservative extension (\cite[180]{Hinman2005Logic})]{conservative} when \( \Gamma \vdash_\Sigma \varphi \) implies \( \Delta \vdash_\Theta \op*{Sen}(t)(\varphi) \).

    Sufficiency is obvious for both semantic entailment and natural deduction, so it usually suffices to check that \( \Delta \vdash_\Theta \op*{Sen}(t)(\varphi) \) implies \( \Gamma \vdash_\Sigma \varphi \).

    \thmitem{def:fol_theory/model}\mcite[def. 2.4.4]{Hinman2005Logic} As in \cref{def:theory_of_institutional_model}, we define the \term{theory} \( \cat{Th}(\mscrX) \) \hi{of} a structure \( \mscrX \) as the set of all sentences valid in \( \mscrX \).

    \thmitem{def:fol_theory/consistent}\mcite[def. 2.6.19]{Герасимов2014Вычислимость} We call a first-order theory \term[ru=непротиворечивое (множество формул)]{consistent} if \( \Gamma \) contains no \hyperref[def:fol_semantics/tautology]{contradictions}, i.e. if \( \Gamma \) does not contain \( \synbot \).

    More generally, a set of sentences \( \Gamma \) is consistent if \( \synbot \) cannot be derived from it.

    We discuss consistent theories in a general setting in \cref{def:consistent_set_of_sentences}, and give equivalent conditions for syntactic theories in \cref{thm:propositional_semantic_inconsistency} and for semantic theories in \cref{thm:propositional_semantic_inconsistency}. The latter also shows that semantic consistency is equivalent to \hyperref[def:propositional_semantics/satisfaction]{satisfiability}.

    \thmitem{def:fol_theory/complete}\mcite[def. 2.4.i]{Hinman2005Logic} We call a first-order theory \( \Gamma \) \term[ru=полное (множество формул) (\cite[def. 2.6.20]{Герасимов2014Вычислимость})]{complete} if, for every sentence \( \varphi \), \( \Gamma \) contains either \( \varphi \) or \( \synneg \varphi \) (or possibly both, if the theory is inconsistent).

    More generally, a set of sentences \( \Gamma \) is complete if, for every sentence \( \varphi \), we can derive either \( \varphi \) or \( \synneg \varphi \) (or possibly both) from \( \Gamma \).

    We will discuss complete theories in a general setting in \cref{def:complete_set_of_sentences}, and prove in \cref{thm:propositional_syntactically_complete_set} that it coincides with our notion when restricted to classical logic.
  \end{thmenum}
\end{definition}

\begin{proposition}\label{thm:def:fol_theory}
  \hyperref[def:fol_theory]{First-order theories} have the following basic properties:
  \begin{thmenum}
    \thmitem{thm:def:fol_theory/model_theory_consistent_complete} For every structure \( \mscrX \), the theory \( \cat{Th}(\mscrX) \) is consistent and complete.

    \thmitem{thm:def:fol_theory/theory_morphism_reduct} For every \hyperref[def:fol_theory/morphism]{theory morphism} \( t: (\Sigma, \Gamma) \to (\Theta, \Delta) \), if \( \mscrY \) is a model of \( \Delta \), its \hyperref[def:fol_reduct_along_morphism]{reduct} \( \red_t(\mscrY) \) is a model of \( \Gamma \).

    The reduct is taken with respect to the underlying signature morphism of \( t \). We show that the theory morphism is also compatible with this reduct.
  \end{thmenum}
\end{proposition}
\begin{proof}
  \SubProofOf{thm:def:fol_theory/model_theory_consistent_complete} Analogous to \cref{thm:def:propositional_theory/model_theory_consistent_complete}.

  \SubProofOf{thm:def:fol_theory/theory_morphism_reduct} Suppose that \( \mscrY \) is a model of \( \Delta \).

  Consider the theory morphism \( t: (\Sigma, \Gamma) \to (\Theta, \Delta) \). By definition, it is a signature morphism that translates formulas from \( \Gamma \) to formulas from \( \Delta \).

  The reduct \( \red_t(\mscrY) \) along the \hi{signature morphism} \( t \). Suppose that \( \mscrY \) is a model of \( \Delta \), i.e. \( \Bracks{\psi}_\mscrY = \semtop \) for every formula \( \psi \) in \( \Delta \). In particular, \( \Bracks{\varphi[t]}_\mscrY = \semtop \) for every \( \varphi \) in \( \Gamma \).

  \Cref{thm:fol_structure_reduct_denotation/formulas} implies that \( \Bracks{\varphi}_{\red_t(\mscrY)}^v = \semtop \) for every \( \varphi \) in \( \Gamma \); hence, \( \red_t(\mscrY) \) is a model of \( \Gamma \).
\end{proof}

\begin{definition}\label{def:fol_theory_model_functor}\mimprovised
  Consider the \hyperref[def:fol_institution]{first-order institution} over the \hyperref[def:grothendieck_universe]{Grothendieck universe} \( \mscrU \).

  This institution has a \hyperref[def:fol_institution/models]{model functor} \( \cat{Mod}: \cat{Sign} \to \cat{Cat}^{\oppos} \) which sends each \hyperref[def:fol_signature]{signature} \( \Sigma \) to the category \( \cat{Mod}(\Sigma) \) of all \hyperref[def:fol_structure]{structures} and \hyperref[def:fol_homomorphism]{homomorphisms} over \( \Sigma \), and each \hyperref[def:fol_signature_category/morphisms]{signature morphism} \( t: \Sigma \to \Theta \) to the corresponding \hyperref[def:fol_institution/models/hom]{reduct functor} \( \cat{Mod}(t): \cat{Mod}(\Theta) \to \cat{Mod}(\Sigma) \).

  We can extend \( \cat{Mod} \) to a functor \( \cat{Mod}: \cat{Th} \to \cat{Cat}^{\oppos} \) on the corresponding \hyperref[def:fol_theory/category]{category of theories}:
  \begin{thmenum}
    \thmitem{def:fol_theory_model_functor/obj} We send a theory \( \Gamma \) to the \hyperref[def:subcategory/full]{\hi{full} subcategory} of \( \cat{Mod}(\Sigma) \) of all structures in which \( \Gamma \) is \hyperref[def:fol_semantics/model]{valid}.

    We denote it by \( \cat{Mod}(\Gamma) \) and refer to it as the \term{category of \( \mscrU \)-small models of \( \Gamma \)}.

    It is important that we have required the subcategory \( \cat{Mod}(\Gamma) \) to be full. Thus, what counts as a model in \( \cat{Mod}(\Gamma) \) depend on the axioms of \( \Gamma \), but what counts as a homomorphism depend solely on \( \Sigma \). In particular, if \( \Gamma \) is the empty theory, then \( \cat{Mod}(\Gamma) \) coincides with \( \cat{Mod}(\Sigma) \).

    \thmitem{def:fol_theory_model_functor/hom} We send the \hyperref[def:fol_theory/morphism]{theory morphism} \( t: (\Sigma, \Gamma) \to (\Theta, \Delta) \) to the aforementioned reduct functor \( \cat{Mod}(t): \cat{Mod}(\Theta) \to \cat{Mod}(\Sigma) \). To highlight that the functor acts on theories, we denote it by \( \cat{Mod}(t): \cat{Mod}(\Delta) \to \cat{Mod}(\Gamma) \).

    The reduct functor only depends on the underlying signature morphism, but, as shown in \cref{thm:def:fol_theory/theory_morphism_reduct}, whether the reduct is a model of \( \Gamma \) generally depends on \( \Delta \).
  \end{thmenum}
\end{definition}
\begin{comments}
  \item For established theories, instead of \( \cat{Mod}(\Gamma) \), we introduce specific names. For example, we use \( \cat{Grp} \) for the \hyperref[def:group/category]{category of groups} or \( \cat{Lat} \) for \hyperref[def:lattice/theory]{category of lattices}.

  \item We have skipped the \( \mscrU \) prefix here for brevity, but all the aforementioned categories depend on \( \mscrU \).

  \item This definition is based on the discussion in \bycite[105]{GoguenBurstall1992Institutions}.
\end{comments}

\begin{proposition}\label{thm:def:fol_theory_model_functor}
  The \hyperref[def:fol_theory_model_functor]{theory model functor} \( \cat{Mod}: \cat{Th} \to \cat{Cat}^{\oppos} \) has the following basic properties:
  \begin{thmenum}
    \thmitem{thm:def:fol_theory_model_functor/reduct_faithful} For every theory morphism \( t: (\Sigma, \Gamma) \to (\Theta, \Delta) \), the corresponding reduct functor \( \cat{Mod}(t): \cat{Mod}(\Delta) \to \cat{Mod}(\Gamma) \) is \hyperref[def:functor_invertibility/faithful]{faithful}.
  \end{thmenum}
\end{proposition}
\begin{proof}
  \SubProofOf{thm:def:fol_theory_model_functor/reduct_faithful} Fix two models \( \mscrX = (X, I) \) and \( \mscrY = (Y, J) \) of \( \Delta \), and consider their reducts \( \red_t(\mscrX) = (X, I_t) \) and \( \red_t(\mscrY) = (Y, J_t) \). \Cref{thm:def:fol_theory/theory_morphism_reduct} implies that they are models of \( \Gamma \).

  If \( \red_t(h) \) and \( \red_t(g) \) coincide for homomorphisms \( h \) and \( g \) in \( \cat{Mod}(\Sigma) \), they agree as functions on \( X \), thus \( h \) and \( g \) also coincide as homomorphisms in \( \cat{Mod}(\Sigma) \). Hence, \( \cat{Mod}(t) \) is faithful.
\end{proof}

\begin{definition}\label{def:fol_theory_forgetful_functor}\mimprovised
  The \hyperref[def:fol_theory_model_functor/obj]{theory model functor} \( \cat{Mod}: \cat{Th} \to \cat{Cat}^{\oppos} \) naturally induces many \hyperref[def:concrete_category]{forgetful functors}.

  \begin{thmenum}
    \thmitem{def:fol_theory_forgetful_functor/reduct} For every theory morphism \( t: (\Sigma, \Gamma) \to (\Theta, \Delta) \), the corresponding reduct functor \( \cat{Mod}(t): \cat{Mod}(\Delta) \to \cat{Mod}(\Gamma) \) can be regarded as forgetful.

    \thmitem{def:fol_theory_forgetful_functor/set} For any category of models \( \cat{Mod}(\Gamma) \), there is a natural functor \( U: \cat{Mod}(\Gamma) \to \cat{Set} \) sending each structure to its universe and each homomorphism to its underlying function.
  \end{thmenum}
\end{definition}
\begin{defproof}
  \Cref{thm:def:fol_theory_model_functor/reduct_faithful} implies that the reduct functors are faithful, and the proof for \( U \) is even simpler, so the implicit requirement of faithfulness from \cref{def:concrete_category} is satisfied.
\end{defproof}

\begin{example}\label{ex:def:fol_theory_model_functor}
  We list examples related to \hyperref[def:fol_theory_model_functor]{theory model functor} \( \cat{Mod}: \cat{Th} \to \cat{Cat}^{\oppos} \):
  \begin{thmenum}
    \thmitem{ex:def:fol_theory_model_functor/order} The \hyperref[def:preordered_set/theory]{partially ordered sets} extend the \hyperref[def:preordered_set/theory]{theory of preordered sets} with the antisymmetry condition \eqref{eq:def:binary_relation/antisymmetric}.

    The homomorphisms of partially ordered sets are \hyperref[def:order_function/preserving]{order-preserving maps}, as in the case of the more general preordered sets. This is to be expected since we have not modified the signature.

    On the other hand, the \hyperref[def:lattice/theory]{theory of lattices} introduces new function symbols and only introduces the familiar inequalities via \hyperref[def:fol_definitional_extension]{definitional extensions}.

    \thmitem{ex:def:fol_theory_model_functor/groups} The \hyperref[def:group/theory]{theory of groups} extends \hyperref[def:monoid/theory]{that of monoids} with the unary symbol \( {\syninv} \), which forces it to also adapt the notion of homomorphisms.

    On the other hand, the \hyperref[def:abelian_group]{abelian groups} extend only the theory, so the homomorphisms of abelian groups are precisely those of groups.
  \end{thmenum}
\end{example}

\paragraph{Definitional extensions}

\begin{concept}\label{con:primitive_notion}
  In a sufficiently complex \hyperref[con:metalogic]{object theory}, different notions can be defined via each other. If we want to use \hyperref[def:well_founded_relation]{well-founded} notions that are not defined circularly via each other, we must choose a subset of them that should to be characterized via axioms, and then use them to define the rest. Following \incite[28]{Kleene1971Metamathematics}, we will refer call these notions as \term{primitive}.

  For instance, when formalizing \hyperref[def:lattice]{lattices}, we may use an object theory based on one \hyperref[def:partially_ordered_set]{partial order relation} \( {\synleq} \), as we have done in \cref{def:lattice/theory}, or we may use a theory based on two operations \( {\latwedge} \) and \( {\latvee} \).

  The object language will feature all three symbols anyway. If we use the formalization from \cref{def:lattice/theory}, however, we will specify the behavior of \( {\synleq} \) via axioms, while the behavior of \( {\latwedge} \) and \( {\latvee} \) will be defined via that of \( {\synleq} \). Conversely, \cref{thm:lattice_from_binary_operations} shows how we can take \( {\latwedge} \) and \( {\latvee} \) as primitive and use them to specify how \( {\synleq} \) behaves.

  Usually the essential object of study is taken as primitive --- for example, sets and set membership are primitive in \fullref{ch:set_theory}, groups and group operations are primitive in \fullref{ch:group_theory}, vectors and vector space operations are primitive in \fullref{ch:linear_algebra} and so forth. This is mildly ironic because, for example, linear algebra cannot answer what a vector is beyond characterizing it as \enquote{an element of an abstract vector space}.

  Since it is impractical to make a definitive list of all required symbols beforehand, we will rely on \hyperref[def:fol_definitional_extension]{definitional extensions} and \hyperref[con:syntactic_abbreviation]{metalingual abbreviations}.
\end{concept}
\begin{comments}
  \item In addition to the adjective \enquote{primitive}, \incite[28]{Kleene1971Metamathematics} suggests \enquote{technical} and \enquote{undefined}. We will avoid the latter term because it would conflict with undefinedness as described in \cref{con:undefinedness}. See \cref{rem:undefined_and_primitive_terms} for disambiguation of the different notions of undefinedness.

  \item We avoid introducing a distinct term for symbols whose behavior is defined via others. \incite[28]{Kleene1971Metamathematics} suggests \enquote{ordinary}, \enquote{logical} and \enquote{defined}.
\end{comments}

\begin{concept}\label{con:syntactic_abbreviation}\mimprovised
  It is sometimes convenient to introduce a simplified notation for a more intricate \hyperref[con:expression]{expression}. We call this process \term{abbreviation}.

  Inside the \hyperref[con:metalogic]{object languages}, we can use \hyperref[def:fol_signature_category/morphisms]{signature translation} and \hyperref[def:fol_definition]{definitional extensions}. In case they are either insufficient or inconvenient, we may rely on notational shorthands entirely within the \hyperref[con:metalogic]{metalanguage}. For example, we use the \hyperref[con:description_operator/unique_existence]{unique existence quantifier} \( \qExists* x \varphi \) as an abbreviation of the more complicated expression \eqref{eq:rem:fol_formula_conventions/unique_existence}.
\end{concept}

\begin{definition}\label{def:fol_definition}\mcite[def. 2.6.11]{Hinman2005Logic}
  Fix a \hyperref[def:fol_theory]{first-order theory} \( \Gamma \) over the signature \( \Sigma \).

  \begin{thmenum}
    \thmitem{def:fol_definition/predicate} Consider some symbol \( p \) not in \( \Sigma \) and some formula \( \varphi_p \) whose free variables are among \( x_1, \ldots, x_n \).

    We can \hyperref[def:fol_signature_extension]{extend} \( \Sigma \) with \( p \) as an \( n \)-ary predicate and then form an \hyperref[def:fol_theory/extension]{extended theory} as the closure of \( \Gamma \) and the formula
    \begin{equation}\label{eq:def:fol_definition/predicate}
      \qforall {x_1} \cdots \qforall {x_n} (p(x_1, \ldots, x_n) \syniff \varphi_p).
    \end{equation}

    We call \eqref{eq:def:fol_definition/predicate} a \term{definition} of \( p \) in \( \Gamma \).

    \thmitem{def:fol_definition/function} Consider some symbol \( f \) not in \( \Sigma \) and some formula \( \varphi_f \) whose free variables are among \( x_1, \ldots, x_n \), and such that \( \Gamma \) contains
    \begin{equation}\label{eq:def:fol_definition/function/condition}
      \qforall {x_1} \cdots \qforall {x_n} \qexists y \varphi_f.
    \end{equation}

    We can extend \( \Sigma \) with \( f \) as an \( n \)-ary predicate and then form an extended theory as the closure of \( \Gamma \) and the formula
    \begin{equation}\label{eq:def:fol_definition/function}
      \qforall {x_1} \cdots \qforall {x_n} \qforall y (f(x_1, \ldots, x_n) \syneq y \syniff \varphi_f).
    \end{equation}

    We call \eqref{eq:def:fol_definition/function} a \term{definition} of \( f \) in \( \Gamma \).
  \end{thmenum}
\end{definition}
\begin{comments}
  \item Of course, the new symbols are allowed to have any \hyperref[def:fol_signature/notation]{notation}.
\end{comments}

\begin{definition}\label{def:fol_definitional_extension}\mcite[def. 2.6.12]{Hinman2005Logic}
  We say that the \hyperref[def:fol_theory/extension]{theory extension} \( (\Sigma^+, \Gamma^+) \) of \( (\Sigma, \Gamma) \) is a \term{definitional extension} if there exist finitely many function and predicate symbols in \( \Sigma^+ \) but not \( \Sigma \) such that \( (\Sigma^+, \Gamma^+) \) can be obtained from \( (\Sigma, \Gamma) \) by adding definitions for these symbols in accordance with \cref{def:fol_definition}.
\end{definition}
\begin{comments}
  \item We do not disallow trivial definitional extensions, but will not find them useful.
  \item See \cref{thm:def:fol_definability} for how \hyperref[def:fol_definability]{definability} relates to definitional extensions.
\end{comments}

\begin{definition}\label{def:fol_definitional_structure_expansion}\mimprovised
  In the setting of \cref{def:fol_definitional_extension}, for every structure \( \mscrX = (X, I) \), we can define an \hyperref[def:fol_structure_expansion]{expansion} \( \mscrX^+ = (X, I^+) \), where \( I^+ \) extends \( I \) as follows:
  \begin{thmenum}
    \thmitem{def:fol_definitional_structure_expansion/predicate} For each new \( n \)-ary predicate symbol \( p \), we use the \hyperref[def:fol_parameterized_formula_denotation]{parametrized formula denotation}:
    \begin{equation*}
      I^+(p)(a_1, \ldots, a_n) \coloneqq \Bracks{\varphi_p}_\mscrX(a_1, \ldots, a_n).
    \end{equation*}

    \thmitem{def:fol_definitional_structure_expansion/function} For each new \( n \)-ary function symbol \( f \), we define \( I^+(p)(a_1, \ldots, a_n) \) as the unique value \( y \) for which \( \Bracks{\varphi_p}_\mscrX(a_1, \ldots, a_n, y) = \semtop \).

    Uniqueness of \( y \) is guaranteed by \eqref{eq:def:fol_definition/function/condition}.
  \end{thmenum}

  We call it the \term{definitional expansion} of \( \mscrX \).
\end{definition}

\begin{lemma}\label{thm:fol_definitional_extension_model_uniqueness}
  In the setting of \cref{def:fol_definitional_structure_expansion}, if \( \mscrX \) is a model of \( \Gamma \), then \( \mscrX^+ \) is the unique expansion of \( \mscrX \) that is a model of \( \Gamma^+ \).
\end{lemma}
\begin{proof}
  It suffices to consider the case where \( (\Sigma^+, \Gamma^+) \) extends \( (\Sigma, \Gamma) \) with one definition. Suppose that \( \mscrX = (X, I) \) is a model of \( \Gamma \).

  \begin{itemize}
    \item Suppose that \( \Sigma^+ \) extends \( \Sigma \) with the \( n \)-ary predicate symbol \( p \).

    Every formula \( \varphi \) in \( \mscrX^+ \) is by definition a semantic consequence of \( \Gamma \) and \eqref{eq:def:fol_definition/predicate}, both of which \( \mscrX^+ \) satisfies. Then \( \mscrX^+ \) also satisfies \( \varphi \).

    Furthermore, any other interpretation of \( p \) would not satisfy \eqref{eq:def:fol_definition/predicate}, which demonstrates the uniqueness of \( \mscrX^+ \).

    \item The case for function symbols is similar.
  \end{itemize}
\end{proof}

\begin{proposition}\label{thm:fol_definitional_extension_conservative}
  \hyperref[def:fol_definitional_extension]{First-order definitional extensions} are \hyperref[def:fol_theory/conservative]{conservative}.
\end{proposition}
\begin{proof}
  It suffices to consider the case where \( (\Sigma^+, \Gamma^+) \) extends \( (\Sigma, \Gamma) \) with the definition of one symbol. Let \( \mscrX = (X, I) \) be a model of \( \Gamma \) and let \( \mscrX^+ = (X, I^+) \) be the corresponding expansion from \cref{def:fol_definitional_structure_expansion}.
  \begin{itemize}
    \item Suppose that \( \Sigma^+ \) extends \( \Sigma \) with the \( n \)-ary predicate symbol \( p \).

    Let \( \varphi \) be a formula from \( \Gamma^+ \) that does not contain \( f \).

    By \cref{thm:fol_definitional_extension_model_uniqueness}, the expansion \( \mscrX^+ \) is a model of \( \Gamma^+ \), and of \( \varphi \) in particular. Since \( \varphi \) does not contain \( f \), the denotation of \( \varphi \) in \( \mscrX^+ \) only uses the interpretation \( I \). Thus, \( \mscrX \) is also a model of \( \varphi \).

    Therefore, \( \Gamma^+ \) is conservative.

    \item The case for function symbols is similar.
  \end{itemize}
\end{proof}

\begin{example}\label{ex:def:fol_definitional_extension}
  We list examples of \hyperref[def:fol_definitional_extension]{definitional extensions}:
  \begin{thmenum}
    \thmitem{ex:def:fol_definitional_extension/sets} In \fullref{sec:naive_set_theory}, we define many operators on sets, and in \cref{thm:zfc_existence_theorems} we prove that they satisfy the condition \eqref{eq:def:fol_definition/function} for function definitions.

    \thmitem{ex:def:fol_definitional_extension/lattices} The \hyperref[def:lattice/theory]{theory of lattices} is only formulated using function symbols, and is adapted to the \hyperref[def:preordered_set/theory]{theory of preordered sets} via definitional extensions.
  \end{thmenum}
\end{example}

\paragraph{Isomorphism}

\begin{definition}\label{def:fol_isomorphism}
  We say that a \hyperref[def:fol_homomorphism]{first-order homomorphism} \( h: \mscrX \to \mscrY \) is an \term[ru=изоморфизм (\cite[170]{Герасимов2014Вычислимость}), en=isomorphism (\cite[def. 2.3.1]{Hinman2005Logic})]{isomorphism} if any of the following equivalent conditions hold:
  \begin{thmenum}
    \thmitem{def:fol_isomorphism/categorical}\mimprovised \( h \) is a \hyperref[def:fol_isomorphism]{categorical isomorphism} in \( \cat{Mod}(\Sigma) \): there exists an inverse homomorphism \( g \) such that \( g \bincirc h \) is the identity in \( \mscrX \) and \( h \bincirc g \) is the identity in \( \mscrY \).

    \thmitem{def:fol_isomorphism/inverse}\mimprovised \( h \) is a \hyperref[def:function_invertibility/bijective]{bijective} homomorphism and its set-theoretic inverse \( h^{-1} \) is a homomorphism.

    \thmitem{def:fol_isomorphism/direct}\mcite[170]{Герасимов2014Вычислимость} \( h \) is a \hyperref[def:function_invertibility/bijective]{bijective} \hyperref[def:fol_homomorphism/predicates]{strong homomorphism}.

    If the signature has no predicate symbols, all homomorphisms are strong, so it is sufficient for the homomorphism to be bijective.
  \end{thmenum}
\end{definition}
\begin{defproof}
  \ImplicationSubProof{def:fol_isomorphism/categorical}{def:fol_isomorphism/inverse} Suppose that \( h \) is a categorical isomorphism with inverse \( g \).

  \Cref{thm:function_invertibility_categorical/fully_invertible} implies that the image \( U(h) \) under the \hyperref[def:fol_theory_forgetful_functor/set]{forgetful functor} \( U: \cat{Mod}(\Sigma) \to \cat{Set} \) is a bijective function, and \cref{thm:def:functor/isomorphisms} implies that \( U(g) \) is the inverse of \( U(h) \) in \( \cat{Set} \).

  Therefore, the homomorphism \( g \) is the set-theoretic inverse of \( h \).

  \ImplicationSubProof{def:fol_isomorphism/inverse}{def:fol_isomorphism/categorical} Suppose that \( h \) is a bijective homomorphism whose set-theoretic inverse \( h^{-1} \) is a homomorphism.

  To be precise, our assumptions is that \( U(h) \) is bijective and that there exists a homomorphism \( g \) in \( \cat{Mod}(\Sigma) \) such that \( U(g) = U(h)^{-1} \), i.e. \( U(g) \) is the set-theoretic inverse of \( h \). \Cref{thm:def:functor_invertibility/faithful_reflects_isomorphisms} implies that \( g \) is the inverse of \( h \) in \( \cat{Mod}(\Sigma) \).

  \EquivalenceSubProof{def:fol_isomorphism/inverse}{def:fol_isomorphism/direct} If \( h \) is bijective, the inequality in \eqref{eq:def:fol_homomorphism/predicates} for \( h^{-1} \) coincides with the converse inequality for \( h \).
\end{defproof}

\paragraph{Definability}

\begin{definition}\label{def:fol_definability}\mcite[def. 2.3.37]{Hinman2005Logic}
  In a \hyperref[def:fol_structure]{first-order structure} \( \mscrX = (X, I) \), we say that the subset \( A \) of \( X^n \) is \term{definable} with \term{parameters} \( b_1, \ldots, b_m \) if there exists a formula \( \varphi_A \) with at most \( n + m \) free variables such that
  \begin{equation*}
    A = \set[\big]{ (a_1, \ldots, a_n) \in X^n \given* \Bracks{\varphi_A}_\mscrX^v(a_1, \ldots, a_n, b_1, \ldots, b_m) = \semtop }.
  \end{equation*}

  If \( A \) is definable without parameters, we simply say that it is \term{definable}.

  \begin{thmenum}
    \thmitem{def:fol_definability/predicate} As per \cref{rem:boolean_valued_functions_and_relations}, we may regard the \hyperref[def:subset_characteristic_function]{characteristic function} of \( A \) as the interpretation of a predicate symbol.

    \thmitem{def:fol_definability/function} If \( n > 1 \), we may also regard \( A \) as the interpretation of a \hyperref[def:set_valued_map]{set-valued map} from \( X^{n-1} \) to \( X \).
  \end{thmenum}
\end{definition}
\begin{comments}
  \item See \cref{thm:def:fol_definability} for how definability relates to \hyperref[def:fol_definitional_extension]{definitional extensions}.
  \item See \cref{def:set_builder_notation} and \fullref{thm:cumulative_hierarchy_model_of_zfc} for how this concept deeply relates to set theory.
\end{comments}

\begin{proposition}\label{thm:fol_definability_and_definitions}
  \hyperref[def:fol_definitional_extension]{Definitional extensions} and \hyperref[def:fol_definability]{definability} are related as follows:
  \begin{thmenum}
    \thmitem{thm:fol_definability_and_definitions/function} If the formula \( \varphi_f \) is used for a \hyperref[def:fol_definition/function]{function definition} of the symbol \( f \) in a \hyperref[def:fol_definitional_extension]{definitional extension} \( \Gamma^+ \) of \( \Gamma \), and if \( \mscrX \) is a model of \( \Gamma \), then the interpretation \( I^+(f) \) in the \hyperref[def:fol_definitional_structure_expansion]{definitional expansion} \( \mscrX^+ \) is \hyperref[def:fol_definability/function]{definable} by \( \varphi_f \) in \( \mscrX \).

    \thmitem{thm:def:fol_definability/predicates_extension} Similarly, if \( \varphi_p \) is used for a \hyperref[def:fol_definition/predicate]{predicate definition} of the symbol \( p \) in a \hyperref[def:fol_definitional_extension]{definitional extension} \( \Gamma^+ \) of \( \Gamma \), and if \( \mscrX \) is a model of \( \Gamma \), then the interpretation \( I^+(p) \) in the definitional expansion of \( \mscrX^+ \) is the \hyperref[def:subset_characteristic_function]{\hi{characteristic function}} of the set \hyperref[def:fol_definability/function]{definable} by \( \varphi_p \) in \( \mscrX \).
  \end{thmenum}
\end{proposition}
\begin{proof}
  Straightforward.
\end{proof}

\begin{proposition}\label{thm:fol_definable_set_automorphism}\mcite[corr. 2.3.40]{Hinman2005Logic}
  \hyperref[def:fol_definability]{Definable sets} are \hyperref[def:invariant_subset]{invariant} under \hyperref[def:morphism_invertibility/automorphism]{automorphisms}.
\end{proposition}
