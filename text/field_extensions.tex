\subsection{Field extensions}\label{subsec:field_extensions}

\paragraph{Algebraic and transcendental elements}

\begin{definition}\label{def:algebraic_element}
  Let \( \BbbK \) be an \hyperref[def:field/submodel]{extension} of the \hyperref[def:field]{field} \( \Bbbk \). We say that an element \( u \) of \( \BbbK \) is \term[ru=алгебрический (елемент) (\cite[407]{Винберг2014})]{algebraic} over \( \Bbbk \) if any of the following equivalent conditions hold:
  \begin{thmenum}
    \thmitem{def:algebraic_element/direct}\mcite[124]{Jacobson1985Vol1} \( u \) is a \hyperref[def:polynomial_root]{root} of some nonzero polynomial from \( \Bbbk[X] \).

    \thmitem{def:algebraic_element/quotient}\mcite[391]{Aluffi2009} There exists a unique \hyperref[def:monic_polynomial]{monic} \hyperref[def:domain_divisibility/irreducible]{irreducible polynomial} \( m(X) \) such that the \hyperref[def:algebra_over_ring/quotient]{quotient} of the \hyperref[def:polynomial_algebra]{polynomial algebra} \( \Bbbk[X] \) by the principal ideal \( \braket{ m(X) } \) is isomorphic to the ring \( \Bbbk[u] \), obtained by \hyperref[def:semiring_adjunction]{adjoining} \( u \) to \( \Bbbk \).

    We call \( m(X) \) the \term[bg=минимален полином (\cite[def. VI.2]{ГеновМиховскиМоллов1991}), ru=минимальный многочлен (\cite[410]{Винберг2014})]{minimal polynomial} of \( u \).

    \thmitem{def:algebraic_element/field} The ring \( \Bbbk[u] \) is a \hyperref[def:field]{field}.

    \thmitem{def:algebraic_element/embedding} The \hyperref[rem:substitution_homomorphism]{evaluation map} \( \Phi_u: \Bbbk[X] \to \BbbK \), sending the monomial \( X \) to \( u \), is not \hyperref[def:function_invertibility/injective]{injective}.

    \thmitem{def:algebraic_element/dimensions} The ring \( \Bbbk[u] \) is a finite-dimensional vector space over \( \Bbbk \).
  \end{thmenum}
\end{definition}
\begin{defproof}
  \ImplicationSubProof{def:algebraic_element/direct}{def:algebraic_element/quotient} Let \( p(X) \) be a polynomial over \( \Bbbk \) such that \( \Phi_u(p) = 0 \). Let
  \begin{equation*}
    p(X) = a(X) p_1(X) \cdots p_n(X)
  \end{equation*}
  be an \hyperref[def:irreducible_factorization]{irreducible factorization} of \( p(X) \).

  Since fields have no zero divisors, \( u \) is a root of at least one of the factors, say \( p_{k_0}(X) \). Furthermore, \fullref{thm:def:coprime_elements/irreducible} implies that any two irreducible polynomials are either associates or are coprime, thus, \( p_{k_0}(X) \) is the \enquote{unique} up to an invertible scalar. Let \( m(X) \) be the monic associate of \( p_{k_0}(X) \).

  Then \( m(X) \) is a common divisor of all polynomials in \( \Bbbk[X] \) which are zero at \( u \). Taking the quotient \( \Bbbk[X] / \braket{ m(X) } \) \enquote{collapses} all these polynomials to zero. We will show that \( \Bbbk[X] / \braket{ m(X) } \) is isomorphic to \( \Bbbk[u] \) by demonstrating that \( \braket{ m(X) } \) is the \hyperref[def:algebra_over_ring/kernel]{kernel} of the evaluation map \( \Phi_u \).

  Consider any polynomial \( f(X) \) from \( \Bbbk[X] \). We can use \fullref{alg:euclidean_division_of_polynomials} to obtain polynomials \( q(X) \) and \( r(X) \), where \( r(X) \) is zero or \( \deg r < \deg m \), such that
  \begin{equation*}
    f(X) = q(X) m(X) + r(X).
  \end{equation*}

  Clearly
  \begin{equation*}
    \Phi_u(f) = \underbrace{\Phi_u(q) \Phi_u(m)}_{0} + \Phi_u(r).
  \end{equation*}

  Furthermore, if \( r(X) \) is not the zero polynomial, then \( \Phi_u(r) \neq 0 \) because \( m(X) \) doesn't divide \( r(X) \), and we saw that \( m(X) \) divides all polynomials that have \( u \) as a root.

  Therefore, \( f(X) \) is in the kernel of \( \Phi_u \) if and only if \( m(X) \) divides \( f(X) \), that is, if it is in \( \braket{ m(X) } \). By \fullref{thm:ring_zero_morphisms/isomorphism},
  \begin{equation*}
    \Bbbk[X] / \braket{ m(X) } \cong \Bbbk[u].
  \end{equation*}

  \ImplicationSubProof{def:algebraic_element/quotient}{def:algebraic_element/field} Suppose that there exists a monic irreducible polynomial \( m(X) \) such that \( \Bbbk[X] / \braket{ m(X) } \) is isomorphic to \( \Bbbk[u] \).

  Since \( m(X) \) is irreducible, \fullref{thm:def:gcd_domain/irreducible_is_prime} implies that it is prime, and since \( \Bbbk[X] \) is a principal ideal domain, \fullref{thm:def:gcd_domain/irreducible_is_prime} implies that the ideal \( \braket{ m(X) } \) is maximal.

  By \fullref{thm:quotient_by_maximal_ideal}, \( \Bbbk[X] / \braket{ m(X) } \) is a field, and hence so is \( \Bbbk[u] \).

  \ImplicationSubProof{def:algebraic_element/field}{def:algebraic_element/embedding} Suppose that \( \Bbbk[u] \) is a field. \Fullref{thm:ring_zero_morphisms/isomorphism} implies that it is isomorphic to \( \Bbbk[X] / \ker \Phi_u \). Since \( \Bbbk[X] \) is not a field, \( \ker \Phi_u \) cannot be the zero ideal. Then, by \fullref{thm:group_homomorphism_zero_kernel}, \( \Phi_u \) is not injective.

  \ImplicationSubProof{def:algebraic_element/embedding}{def:algebraic_element/dimensions} Suppose that \( \Phi_u \) is not injective. \Fullref{thm:group_homomorphism_zero_kernel} implies that its kernel is nonzero, thus there exists a nonzero polynomial \( m(X) \) that generates it. \Fullref{thm:ring_zero_morphisms/isomorphism} then implies that
  \begin{equation*}
    \Bbbk[u] \cong \Bbbk[X] / \braket{ m(X) }.
  \end{equation*}

  \Fullref{thm:polynomial_quotient_module_dimension} implies that, as a vector space, \( \Bbbk[X] / \braket{ m(X) } \) has dimension \( \deg m \), and hence so does \( \Bbbk[u] \). Therefore, \( \Bbbk[u] \) is finite-dimensional.

  \ImplicationSubProof{def:algebraic_element/dimensions}{def:algebraic_element/direct} Suppose that \( \Bbbk[u] \) has finite dimension \( n \).

  Isomorphic vector spaces have the same dimension. Since \( \Bbbk[X] \) is infinite-dimensional, the evaluation map \( \Phi_u: \Bbbk[X] \to \Bbbk[u] \) must not be injective. Furthermore, \( \Bbbk[X] \) is a principal ideal domain by \fullref{thm:def:principal_ideal_domain/field_polynomials}, and thus the kernel of \( \Phi_u \) is generated by a single nonzero element, say \( p(X) \). Then \( \Phi_u(p) = 0 \), so \( u \) is a root of \( p(X) \).
\end{defproof}

\begin{example}\label{ex:def:algebraic_element}
  We list examples of \hyperref[def:algebraic_element]{algebraic elements}:
  \begin{thmenum}
    \thmitem{ex:def:algebraic_element/member} Any member of the field \( \Bbbk \) itself is algebraic over \( \Bbbk \).
    \thmitem{ex:def:algebraic_element/sqrt} Consider the \hyperref[def:nth_root]{\( n \)-th root} \( \sqrt[n]{ p } \) of (the real embedding of) an arbitrary \hyperref[def:prime_number]{prime number}.

    \Fullref{thm:power_of_nth_root} implies that \( \sqrt[n]{ p } \) is a root of the integer polynomial \( X^n - p \). Hence, it is algebraic over \( \BbbQ \). But \fullref{thm:nth_root_is_not_rational} implies that \( \sqrt[n]{ p } \) is not rational.

    Its \hyperref[def:algebraic_element/quotient]{minimal polynomial} is \( X^n - p \) and hence the dimension of \( \BbbQ[\sqrt[n]{ p }] \) over \( \BbbQ \) is \( n \).

    \thmitem{ex:def:algebraic_element/complex} Similarly to the above example, the \hyperref[def:complex_numbers]{imaginary unit} \( i \) is, by definition, algebraic over \( \BbbR \) with minimal polynomial \( X^2 + 1 \). It is also algebraic over \( \BbbQ \) with the same polynomial, but we obtain the field \( \BbbQ[i] \) instead.

    \thmitem{ex:def:algebraic_element/multiple_minimal}\mcite{MathSE:non_unique_minimal_polynomial} If \( \Bbbk \) is not a field, there may be multiple minimal polynomials.

    Consider the ring \( \BbbZ[X^2, X^3] \) discussed in \fullref{ex:common_polynomial_divisors/incomparable} of all polynomials without linear terms. The monomial \( X \) from \( \BbbZ[X] \) satisfies both \( Y^2 - X^2 \) and \( Y^2 + X^2Y - X^2 - X^3 \) from \( \BbbZ[X^2, X^3][Y] \), and the two are not associates.
  \end{thmenum}
\end{example}

\begin{definition}\label{def:transcendental_element}\mcite[124]{Jacobson1985Vol1}
  We say that an element of a field extension is \term[bg=трансцендентен (елемент) (\cite[135]{ГеновМиховскиМоллов1991}), ru=трансцендентный (елемент) (\cite[407]{Винберг2014})]{transcendental} if it is not \hyperref[def:algebraic_element]{algebraic}.
\end{definition}

\paragraph{Simple extensions}

\begin{definition}\label{def:field_adjunction}\mcite[213]{Jacobson1985Vol1}
  Let \( \BbbK \) be an \hyperref[def:field/submodel]{extension} of the \hyperref[def:field]{field} \( \Bbbk \) and let \( A \) be a subset of \( \Bbbk \). Denote by \( \Bbbk(A) \) the (unique) smallest extension of \( \Bbbk \) that contains all elements of \( A \). We say that \( \Bbbk(A) \) is generated by \term{adjoining} \( A \).

  As with \hyperref[def:semiring_adjunction]{(semi)ring adjunctions}, in case \( A \) is finite, we can list the individual elements as \( \Bbbk(a_1, \ldots, a_n) \).
\end{definition}
\begin{defproof}
  We must show the existence of \( \Bbbk(A) \).

  The intersection of any family of subfields of \( \BbbK \) is clearly also a subfield. \Fullref{thm:closure_operator_from_set_semilattice} implies that the intersection of all subfields containing an arbitrary subset of \( \BbbK \) is itself a subfield containing that subset.

  \Fullref{thm:closure_operator_minimality} then implies that the (unique) smallest subfield containing \( \Bbbk \cup A \) is the intersection of all subfields containing \( \Bbbk \cup A \). Since \( \BbbK \) is a subfield of itself, this intersection is nonempty, and \( \Bbbk(A) \) is well-defined.
\end{defproof}

\begin{definition}\label{def:simple_field_extension}\mcite[214]{Jacobson1985Vol1}
  We say that the field \( \BbbK \) is a \term[bg=просто разширение (\cite[def. VI.3]{ГеновМиховскиМоллов1991}), ru=простое расширение (\cite[409]{Винберг2014})]{simple extension} of \( \Bbbk \) if \( \BbbK \) is obtained by \hyperref[def:field_adjunction]{adjoining} a single element to \( \Bbbk \). We call the element itself a \term{primitive element} of the extension.
\end{definition}

\begin{proposition}\label{thm:simple_field_extension_characterization}
  Let \( \Bbbk(u) \) be a \hyperref[def:simple_field_extension]{simple field extension}.
  \begin{thmenum}
    \thmitem{thm:simple_field_extension_characterization/algebraic} \( u \) is \hyperref[def:algebraic_element]{algebraic} over \( \Bbbk \) if and only if \( \Bbbk(u) \) equals the ring \( \Bbbk[u] \), obtained by \hyperref[def:semiring_adjunction]{adjoining} \( u \) to \( \Bbbk \).

    \thmitem{thm:simple_field_extension_characterization/transcendental} \( u \) is \hyperref[def:transcendental_element]{transcendental} over \( \Bbbk \) if and only if \( \Bbbk(u) \) is isomorphic to the \hyperref[def:rational_function_field]{rational function field} \( \Bbbk(X) \).
  \end{thmenum}
\end{proposition}
\begin{proof}
  \SubProofOf{thm:simple_field_extension_characterization/algebraic} One of the equivalent definitions in \fullref{def:algebraic_element} for \( u \) being algebraic is \fullref{def:algebraic_element/field}, which requires \( \Bbbk[u] \) to be a field. Since \( \Bbbk(u) \) is by definition the smallest extension of \( \Bbbk \) containing \( u \), we conclude that \( \Bbbk(u) \) equals \( \Bbbk[u] \), the smallest ring containing \( u \).

  \SubProofOf{thm:simple_field_extension_characterization/transcendental}

  \SufficiencySubProof* Suppose that \( u \) is transcendental.

  Consider the map
  \begin{equation*}
    \begin{aligned}
      &\Psi: \Bbbk(X) \to \Bbbk(u), \\
      &\Psi\parens*{ \frac p q } \coloneqq \frac {\Phi_u(p)} {\Phi_u(q)}.
    \end{aligned}
  \end{equation*}

  It is well-defined because \( u \) is, by definition, not he root of any nonzero polynomial, and hence \( \Phi_u(q) \) is always nonzero. We will show that \( \Psi \) is an isomorphism.

  \SubProofOf[def:function_invertibility/injective/equality]{injectivity}* Suppose that \( \Psi(p / q) = \Psi(f / g) \). Then
  \begin{equation*}
    \Phi_u(pg) = \Phi_u(p) \Phi_u(g) = \Phi_u(f) \Phi_u(q) = \Phi_u(fq).
  \end{equation*}

  Since \( u \) is transcendental, \( \Phi_u: \Bbbk[X] \to \Bbbk[u] \) is injective, hence \( p(X) g(X) = f(X) q(X) \), and thus
  \begin{equation*}
    \frac {p(X)} {q(X)} = \frac {f(X)} {g(X)}.
  \end{equation*}

  \SubProofOf[def:function_invertibility/surjective/existence]{surjectivity} The image of \( \Psi \) is a field extension of \( \Bbbk \) containing \( u \). Since \( \Bbbk(u) \) is the minimal such extension, it follows that the image of \( \Psi \) is \( \Bbbk(u) \).

  \NecessitySubProof* Suppose that \( \Bbbk[u] \) is isomorphic to the rational function field \( \Bbbk(X) \). Then \( \Bbbk[X] \) is infinite-dimensional as a vector space over \( \Bbbk \), hence \( u \) is transcendental because it satisfies the negation of \fullref{def:algebraic_element/dimensions}.
\end{proof}

\begin{definition}\label{def:field_extension_degree}\mcite[def. VII.1.1]{Aluffi2009}
  We define the \term[ru=степень (def. \cite[9.5.1]{Винберг2014})]{degree} of the \hyperref[def:field/submodel]{field extension} \( \BbbK \) of \( \Bbbk \) as the \hyperref[thm:vector_space_dimension]{vector space dimension} of \( \BbbK \) over \( \Bbbk \).

  We use \enquote{finite extension} rather than \enquote{extension of finite degree}.
\end{definition}

\begin{proposition}\label{thm:simple_extension_dimension}
  The \hyperref[def:simple_field_extension]{simple field extension} \( \Bbbk(u) \) is \hyperref[def:field_extension_degree]{finite} if and only if \( u \) is \hyperref[def:algebraic_element]{algebraic}, in which case the degree of \( \Bbbk(u) \) is the degree of the minimal polynomial.
\end{proposition}
\begin{proof}
  \SufficiencySubProof Suppose that \( \Bbbk(u) \) is finite-dimensional. Then \fullref{thm:simple_field_extension_characterization/transcendental} implies that \( u \) cannot be transcendental because \( \Bbbk(X) \) is infinite-dimensional. It remains for \( u \) to be algebraic.

  \NecessitySubProof Suppose that \( u \) is algebraic over \( \Bbbk \).

  \Fullref{thm:simple_field_extension_characterization/algebraic} implies that \( \Bbbk(u) \) equals \( \Bbbk[u] \), and the latter is by definition isomorphic to \( \Bbbk[X] / \braket{ m(X) } \), where \( m(X) \) is the zero polynomial.

  \Fullref{thm:polynomial_quotient_module_dimension} implies that the dimension of \( \Bbbk[X] \) is the degree of \( m(X) \).
\end{proof}

\paragraph{Splitting fields}

\begin{definition}\label{def:splitting_field}\mcite[458]{Knapp2016BasicAlgebra}
  A \term[ru=поле разложения (def. \cite[9.5.2]{Винберг2014})]{splitting field} for a nonconstant polynomial \( f(X) \in \Bbbk[X] \) of degree \( n \) is a \hyperref[def:field/submodel]{field extension} \( \BbbK \) of \( \Bbbk \) in which \( f(X) \) has \( n \) \hyperref[def:polynomial_root]{roots} \( u_1, \ldots, u_n \) (counting multiplicities) and
  \begin{equation*}
    \BbbK \cong \Bbbk(u_1, \ldots, u_n).
  \end{equation*}
\end{definition}
\begin{comments}
  \item \Fullref{thm:splitting_field_existence} implies that splitting fields always exist and are unique up to an isomorphism.
\end{comments}

\begin{proposition}\label{thm:splitting_field_existence}
  For every nonconstant univariate polynomial over a field, there exists a \hyperref[def:splitting_field]{splitting field} that is unique up to a possibly nonunique isomorphism.
\end{proposition}
\begin{proof}
  \ExistenceSubProof We use induction on the polynomial degree to show the existence of a splitting field. Since the base case \( n = 1 \) is trivial, assume that the inductive hypothesis holds for all polynomials of degree less than \( n \) and fix a polynomial \( f(X) \) over \( \Bbbk \) of degree \( n \).

  \Fullref{thm:maximal_ideal_theorem} implies that the ideal \( \braket{ f(X) } \) is contained in some maximal ideal \( M \), and \fullref{thm:quotient_by_maximal_ideal} implies that \( \Bbbk[X] / M \) is a field.

  Denote by \( u_1 \) the coset \( X + M \). In order to evaluate \( f \) at \( u_1 \), we use \fullref{thm:quotient_structure_universal_property} to extend the map \( p(X) \mapsto p(X) + M \) from \( \Bbbk[X] \) to \(\Bbbk[X] / M \), obtaining
  \begin{equation*}
    f(u_1) = f(X + M) = f(X) + M = M.
  \end{equation*}

  Thus, \( u_1 \) is a root of \( f \) in \( \Bbbk[X] / M \), and hence also a root in \( \Bbbk(u_1) \), which may be smaller than \( \Bbbk[X] / M \).

  We can divide \( f(X) \) by \( (X - u_1) \) over \( \Bbbk(u_1) \) to obtain the quotient \( g(X) \) of degree \( n - 1 \).

  The inductive hypothesis gives us an extension \( \Bbbk(u_1)(u_2, \ldots, u_n) \) over which \( g(X) \) splits into linear terms. \Fullref{thm:def:polynomial_algebra/iterated} implies that
  \begin{equation*}
    \Bbbk(u_1)(u_2, \ldots, u_n) = \Bbbk(u_1, u_2, \ldots, u_n).
  \end{equation*}

  Then this is the desired extension field.

  \UniquenessSubProof Suppose that \( \BbbK \) and \( \BbbL \) are both splitting fields for \( f(X) \).
\end{proof}

\begin{theorem}[Classification of finite fields]\label{thm:finite_fields}
  \hfill
  \begin{thmenum}
    \thmitem{thm:finite_fields/characteristic} The \hyperref[def:ring_characteristic]{characteristic} of a \hyperref[def:field]{field} with \( q \) elements is a \hyperref[def:prime_number]{prime number} \( p \), and \( q \) is a power of \( p \).

    The fields of prime cardinality are sometimes called \term{prime fields}.

    \thmitem{thm:finite_fields/prime_field} For a prime number \( p \), the ring \hyperref[def:ring_of_integers_modulo]{\( \BbbZ_p \)} of integers modulo \( p \) is a field.

    \thmitem{thm:finite_fields/splitting} All \hyperref[def:field]{fields} with \( q \) elements are \hyperref[def:field/homomorphism]{isomorphic} as \hyperref[def:splitting_field]{splitting fields} for the polynomial
    \begin{equation*}
      X^q - X \in \BbbZ_p[X].
    \end{equation*}

    Utilizing the general conventions of identifying isomorphic objects in algebra, we denote by \( \BbbF_q \) \enquote{the} finite field with \( q \) elements. Finite fields are also called \term{Galois fields}.

    Every member of \( \BbbF_q \) is a root of \( X^q - X \).
  \end{thmenum}
\end{theorem}
\begin{proof}
  \SubProofOf{thm:finite_fields/characteristic} Let \( \BbbK \) be a field with \( q \) elements and let \( p \) be the \hyperref[def:ring_characteristic]{characteristic} of \( \BbbK \). Then \( \BbbZ_p \) is a subring of \( \BbbK \). By \fullref{thm:multiplicative_group_of_integers_modulo}, \( \BbbZ_p \) is a field.

  By \fullref{thm:quotient_by_maximal_ideal}, \( \braket{ p } \) is a maximal ideal in \( \BbbZ \), and, by \fullref{thm:def:semiring_ideal/maximal_is_prime}, \( p \) is a prime number.

  By \fullref{thm:lagranges_subgroup_theorem}, \( p \) divides \( q \). But \( \BbbK / \BbbZ_p \) is again a field by \fullref{thm:lattice_theorem_for_ideals}, and again has prime characteristic. Continuing by induction, we eventually obtain a sequence \( p_1, \ldots, p_n \) of prime numbers such that
  \begin{equation*}
    q = p_1 \cdots p_n.
  \end{equation*}

  By \fullref{thm:lagranges_subgroup_theorem}, \( q \) cannot contain subgroups of prime cardinalities \( p_1 \) and \( p_2 \) unless \( p_1 = p_2 \). Hence, again by induction, we conclude that
  \begin{equation*}
    p_1 = \cdots = p_n.
  \end{equation*}

  Therefore, \( q = p^n \).

  \SubProofOf{thm:finite_fields/prime_field} Follows from \fullref{thm:multiplicative_group_of_integers_modulo}.

  \SubProofOf{thm:finite_fields/splitting} Let \( \BbbK \) be a field with \( q \) elements with characteristic \( p \). We will show that every element of \( \BbbK \) is a root of \( X^q - X \in \BbbZ_p[X] \).

  The multiplicative group of \( \BbbK \) has order \( q - 1 \). The order of a non-zero element \( a \in \BbbK \) divides \( q - 1 \) by \fullref{thm:prime_groups_are_simple}, hence \( a^{q - 1} = 1 \pmod q \). We also have \( 0^q = 0 \). Therefore, for every element of \( \BbbF_q \), we have \( a^q = a \).

  Then
  \begin{equation*}
    X^q - X = \prod_{u \in \BbbK} (X - u).
  \end{equation*}
\end{proof}

\begin{proposition}\label{thm:functions_over_prime_fields}
  For every \hyperref[thm:finite_fields]{finite field} \( \BbbF_q \) and every \hyperref[def:polynomial_algebra]{polynomial ring} \( \BbbF_q[X_1, \ldots, X_n] \) in finitely many indeterminates, there exists an \( \BbbF_q \)-\hyperref[def:algebra_over_ring]{algebra} isomorphism
  \begin{equation*}
    \frac {\BbbF_q[X_1, \ldots, X_n]} {\braket{ X_i^q - X_i \given i = 1, \ldots, n }} \cong \fun(\BbbF_q^n, \BbbF_q),
  \end{equation*}
  where \( \fun(\BbbF_q^n, \BbbF_q) \) is the \hyperref[thm:functions_over_algebra]{\( \BbbF_q \)-algebra of all functions} from \( \BbbF
  _q^n \) to \( \BbbF_q \).

  Furthermore, every coset of polynomials has a unique representative given by \fullref{thm:finite_field_lagrange_interpolation}.
\end{proposition}
\begin{proof}
  Consider the \hyperref[rem:substitution_homomorphism]{functional evaluation homomorphism}
  \begin{equation*}
    \Phi: \BbbF_q[X_1, \ldots, X_m] \to \fun(\BbbF_q^m, \BbbF_q).
  \end{equation*}

  By \fullref{thm:finite_field_lagrange_interpolation}, \( \Phi \) is surjective. Then, by \fullref{thm:quotient_structure_universal_property},
  \begin{equation*}
    \BbbF_q[X_1, \ldots, X_n] / \ker \Phi \cong \fun(\BbbF_q^m, \BbbF_q).
  \end{equation*}

  We will now prove that \( \ker \Phi \) equals
  \begin{equation*}
    I \coloneqq \braket{ X_i^q - X_i \given i = 1, \ldots, n }.
  \end{equation*}

  First, let \( e: \mscrX \to \BbbF_q \) be the variable assignment that assigns \( u_1, \ldots, u_n \) to the corresponding indeterminates. By \fullref{thm:finite_fields/splitting}, every member of \( \BbbF_q \) is a root of \( X_i^q - X_i \). Then, for any indeterminate \( X_i \),
  \begin{equation*}
    \Phi_e(X_i^q - X_i) = u_i^q - u_i = 0 \pmod q.
  \end{equation*}

  Hence, the polynomial function \( \Phi(X_i^q - X_i) \) is the zero constant function. It follows that any linear combination of the polynomials \( X_i^q - X_i \) for \( i = 1, \ldots, n \) is also the zero function. Therefore, \( I \subseteq \ker \Phi \).

  We will prove the converse inclusion via induction on \( n \).

  In the case of a single indeterminate \( X \), for every polynomial \( f(X) \in \ker \Phi \), we know that the entirety of \( \BbbF_q \) are roots of \( f(X) \). By \fullref{thm:def:integral_domain/root_limit}, \( f(X) \) has at most \( q \) roots. Hence, \( X - u \) divides \( f(X) \) for every \( u \in \BbbF_q \). We have
  \begin{equation*}
    \underbrace{\prod_{u \in \BbbF_q} (X - u)}_{\mathclap{ X^q - X \T*{by \fullref{thm:finite_fields/splitting}}}} \mid f(X),
  \end{equation*}
  and hence \( f(X) \in \braket{ X^q - X } \).

  We have, up until now, shown that the entire proposition holds for the case of one indeterminate. Suppose that the proposition holds for \( n - 1 \) indeterminates and let \( f \in \BbbF_q[X_1, \ldots, X_n] \) be a nonconstant polynomial such that \( \Phi(f) \) is the zero function. Due to \fullref{thm:def:polynomial_algebra/iterated}, we can regard \( f \) as a univariate polynomial in \( X_n \)over \( \BbbF_q[X_1, \ldots, X_{n-1}] \). Thus,
  \begin{equation*}
    f(X_1, \ldots, X_n) = \sum_{k =0}^\infty \underbrace{\parens*{ \sum_\gamma a_{(k,\gamma)} X_1^{\gamma_1} X_2^{\gamma_1} \cdots X_{n-1}^{\gamma_{n-1}} }}_{s_k(X_1, \ldots, X_{n-1})} {X_n}^k,
  \end{equation*}
  where \( \gamma \) is a multi-index over the first \( n - 1 \) indeterminates.

  As a polynomial in \( X_n \), \( f \) has \( m \coloneqq (n-1)p \) roots \( s_1, \ldots, s_m \), which are themselves polynomials from \( \BbbF_q[X_1, \ldots, X_{n-1}] \). For some \( c \), we have
  \begin{equation*}
    f(X_1, \ldots, X_n) = c(X_1, \ldots, X_{n-1}) \prod_{j=1}^m (X_n - s_j(X_1, \ldots, X_{n-1}))
  \end{equation*}
  and
  \begin{equation*}
    0 = \Phi(f) = \Phi(c) \cdot \prod_{j=1}^m \parens[\Big]{ \Phi(X_n) - \Phi(s_j) }.
  \end{equation*}

  Since \( \BbbF_q[X_1, \ldots, X_{n-1}] \) is \hyperref[def:entire_semiring]{entire}, we conclude that either \( \Phi(c) \) is the zero function or \( \Phi(X_n) = \Phi(s_j) \) for at least one index \( 1 \leq j \leq m \). The latter is impossible, because \( \Phi(X_n) \) is linearly independent from polynomials in the first \( n - 1 \) variables.

  The inductive hypothesis holds for the polynomial \( c \), and \( \Phi(c) \) being the zero function implies
  \begin{equation*}
    c \in \braket{ X_i^q - X_i \given i = 1, \ldots, n - 1 } \subsetneq I.
  \end{equation*}

  Therefore, \( f \in I \) since \( f \) divides \( c \). We have chosen \( f \) to be an arbitrary member of \( \ker \Phi \), which implies \( \ker \Phi \subseteq I \).

  We have already shown that \( I \subseteq \ker \Phi \). We thus conclude that \( I = \ker \Phi \) and
  \begin{equation*}
    \BbbF_q[X_1, \ldots, X_m] / I \cong \fun(\BbbF_q^m, \BbbF_q).
  \end{equation*}
\end{proof}

\begin{definition}\label{def:transcendental_element}
  We say that the element \( a \in \BbbK \) of the field extension \( \BbbK \) of \( \Bbbk \) is \term{transcendental} over \( \BbbK \) if it is \hyperref[def:algebraic_dependence]{algebraically independent}.

  If \( a \) is not transcendental, we say that it is \term{algebraic}. If every element of \( \BbbK \) is algebraic over \( \Bbbk \), we say that \( \BbbK \) is an \term{algebraic extension} of \( \Bbbk \).
\end{definition}

\begin{proposition}\label{thm:field_is_algebraic_over_itself}
  Every field is an \hyperref[def:transcendental_element]{algebraic extension} of itself.
\end{proposition}
\begin{proof}
  Every element \( a \in \BbbK \) is a root of the polynomial \( X - a \).
\end{proof}

\begin{theorem}[Euler's constant is transcendental]\label{thm:eulers_constant_is_transcendental}
  \hyperref[def:exponential_function]{Euler's constant} \( e \) is \hyperref[def:algebraic_element/transcendental]{transcendental} over \( \BbbQ \).
\end{theorem}

\begin{theorem}[Pi is transcendental]\label{thm:pi_is_transcendental}\mcite[454]{Knapp2016BasicAlgebra}
  The number \hyperref[def:pi]{\( \pi \)} is \hyperref[def:algebraic_element/transcendental]{transcendental} over \( \BbbQ \).
\end{theorem}

\begin{example}\label{ex:polynomials_over_pi}
  \Fullref{thm:pi_is_transcendental} implies that the polynomials \( \BbbQ[X] \) can be embedded into \( \BbbR \) via \( \Phi_\pi: \BbbQ[X] \to \BbbR \). We can identify a polynomial
  \begin{equation*}
    p(X) = \sum_{i=0}^n a_k X^k
  \end{equation*}
  with rational coefficients with the number
  \begin{equation*}
    p(\pi) = \sum_{i=0}^n a_k \pi^k.
  \end{equation*}
\end{example}

\begin{proposition}\label{thm:splitting_field_is_finite_extension}
  Every \hyperref[def:splitting_field]{splitting field} is a \hyperref[def:field_extension_degree]{finite extension}.
\end{proposition}
\begin{proof}
  \Fullref{thm:splitting_field_existence} gives us a splitting field \( \BbbK \) of \( p(X) \) over \( \Bbbk \). Then there exist some elements \( a_1, \ldots, a_n \in \BbbK \) such that \( \BbbK = \Bbbk(a_1, \ldots, a_n) \). Then for every element \( x \) of \( \BbbK \) there exist some polynomial \( p(X_1, \ldots, X_n) \in \Bbbk[X] \) such that
  \begin{equation*}
    x = p(a_1, \ldots, a_n).
  \end{equation*}

  Thus, the set
  \begin{equation*}
    \set{ a_1^{k_1} \cdots a_n^{k_n} \given k_1 + \cdots + k_n \leq n }
  \end{equation*}
  is a basis of \( \BbbK \) over \( \Bbbk \). It is a finite set, hence \( \BbbK \) is a finite extension of \( \Bbbk \).
\end{proof}

\begin{proposition}\label{thm:finite_field_extensions_are_algebraic}
  Every \hyperref[def:field_extension_degree]{finite field extension} is \hyperref[def:transcendental_element]{algebraic}.
\end{proposition}
\begin{proof}
  Let \( \BbbK \) be a field extension of \( \Bbbk \). Consider the evaluation map \( \Phi_a: \Bbbk[X] \to \Bbbk[u] \) for some \( u \in \BbbK \).

  Since the polynomials \( X^k \) for \( k = 0, 1, 2, \ldots \) form a basis for \( \Bbbk[X] \). If \( \Phi_a \) is injective, then \( \Phi_a(X_k) \) are linearly independent over \( \BbbK \). But \( \BbbK \) has finite dimension over \( \Bbbk \).

  The obtained contradiction shows that \( \Phi_a \) is not injective.
\end{proof}

\begin{definition}\label{def:algebraically_closed_field}\mcite[prop. 9.20]{Knapp2016BasicAlgebra}
  We say that the field \( \BbbK \) is algebraically closed if any of the equivalent conditions are satisfied:
  \begin{thmenum}
    \thmitem{def:algebraically_closed_field/trivial_algebraic_extensions} \( \BbbK \) has no nontrivial \hyperref[def:transcendental_element]{algebraic extensions}.
    \thmitem{def:algebraically_closed_field/linear_irreducible_polynomials} Every irreducible polynomial in \( \BbbK[X] \) is linear.
    \thmitem{def:algebraically_closed_field/at_least_one_root} Every nonconstant polynomial in \( \BbbK[X] \) has at least one root in \( \BbbK \).
    \thmitem{def:algebraically_closed_field/factorization} Every polynomial in \( \BbbK[X] \) \hyperref[def:irreducible_factorization]{factors} into a product of linear polynomials.
    \thmitem{def:algebraically_closed_field/exactly_n_roots} Every polynomial in \( \BbbK[X] \) of degree \( n \) has exactly \( n \) roots in \( \BbbK \), counting the root multiplicities.
  \end{thmenum}
\end{definition}
\begin{proof}
  \ImplicationSubProof{def:algebraically_closed_field/trivial_algebraic_extensions}{def:algebraically_closed_field/linear_irreducible_polynomials} Let \( p(X) \) be an irreducible polynomial in \( \BbbK[X] \).

  Since \( \BbbK[X] \) is a factorial domain, it is a GCD domain, hence it satisfies \fullref{thm:def:gcd_domain/irreducible_is_prime}, and hence \( p(X) \) is a prime element. Thus, \( \braket {p(X)} \) is a \hyperref[def:semiring_ideal/prime]{prime ideal} in \( \BbbK[X] \).

  Since \( \BbbK[X] \) is a principal ideal domain, by \fullref{thm:def:principal_ideal_domain/prime_ideal_is_maximal}, \( \braket{ p(X) } \) is also a maximal ideal. By \fullref{thm:quotient_by_maximal_ideal}, the quotient \( Q \coloneqq \BbbK[X] / \braket{ p(X) } \) is a field. The vectors \( 1, X, X^2, \cdots, X^n \) for a basis of \( Q \) over \( \BbbK \), where \( n \) is the degree of \( p(X) \).

  By \fullref{thm:finite_field_extensions_are_algebraic}, \( Q \) is an algebraic extension of \( \BbbK \). Since \( \BbbK \) has no nontrivial algebraic extensions, it follows that \( \BbbK = Q \). Thus, \( Q \) is unidimensional, and we have already discussed that \( \dim Q = \deg p \). Therefore, \( p \) is a linear polynomial.

  \ImplicationSubProof{def:algebraically_closed_field/linear_irreducible_polynomials}{def:algebraically_closed_field/at_least_one_root} Suppose that every irreducible polynomial is linear.

  By \fullref{thm:def:factorial_domain/polynomial_ring}, \( \BbbK[X] \) is a factorial domain, and thus there exist irreducible polynomials \( q_1(X), \ldots, q_n(X) \) and an invertible element \( a \) such that
  \begin{equation*}
    p(X) = a q_1(X) \cdots q_n(X).
  \end{equation*}

  By assumption, the irreducible polynomials are linear, and hence have roots. Therefore, \( p(X) \) has at least one root.

  \ImplicationSubProof{def:algebraically_closed_field/at_least_one_root}{def:algebraically_closed_field/factorization} Suppose that \( u_1 \) is a root of \( p(X) \). Then \( p(X) \) is divisible by \( (X - u_1) \). Using induction on the degree of \( p(X) \), we can factor \( p(X) \) into
  \begin{equation*}
    p(X) = a (X - u_1) (X - u_2) \cdots (X - u_n),
  \end{equation*}
  where \( a \) is invertible in \( \BbbK \). This is the desired factorization.

  \ImplicationSubProof{def:algebraically_closed_field/factorization}{def:algebraically_closed_field/exactly_n_roots} Follows from the equivalence in \fullref{def:polynomial_root} by induction on the polynomial degree. By \fullref{thm:def:integral_domain/root_limit}, the number of roots is bounded by \( n \).

  \ImplicationSubProof{def:algebraically_closed_field/exactly_n_roots}{def:algebraically_closed_field/trivial_algebraic_extensions} Suppose that every nonconstant polynomial of degree \( n \) has exactly \( n \) roots in \( \Bbbk \) and let \( \BbbK \) be an algebraic extension of \( \Bbbk \).

  By \fullref{thm:def:integral_domain/root_limit}, every polynomial in \( \BbbK[X] \) has at most \( n \) roots. By assumption, every root of every polynomial is contained in \( \Bbbk \). Since \( \BbbK \) is algebraic over \( \Bbbk \), it follows that every element of \( \BbbK \) is a root of some polynomial. Therefore, \( \BbbK = \Bbbk \).
\end{proof}

\begin{proposition}\label{thm:algebraic_closure_existence}\mcite{Jelonek1991}
  Every \hyperref[def:field]{field} has a unique up to a (possibly non-unique) isomorphism \hyperref[def:transcendental_element]{algebraic extension} that is \hyperref[def:algebraically_closed_field]{algebraically closed}, called its \term{algebraic closure}.
\end{proposition}
\begin{proof}
  \UniquenessSubProof Let \( \BbbK \) and \( \BbbL \) be two algebraically closed algebraic extensions of \( \Bbbk \).

  Let \( \mscrF \) be the family of all field homomorphisms \( \varphi: \Bbbl \to \BbbK \), where \( \Bbbl \) is a subfield of \( \BbbL \), ordered such that \( \varphi_1 \leq \varphi_2 \) if \( \varphi_2 \) is an extension of \( \varphi_1 \). The set is nonempty because of the canonical inclusion \( \iota: \Bbbk \to \BbbK \).

  The supremum of an ascending chain in \( \mscrF \) if the (set-theoretic) union of the homomorphisms. \Fullref{thm:zorns_lemma} then implies that \( \mscrF \) has a maximal element \( \Phi \). The domain of \( \Phi \) is \( \BbbL \), because otherwise \( \Phi \) would not be maximal.

  Therefore, we have a field homomorphism \( \Phi: \BbbL \to \BbbK \). It is injective as a consequence of \fullref{thm:def:ring/simple_ring_homomorphism_is_injective}. It is also surjective because, if some element \( y \) of \( \BbbK \) has an empty preimage under \( \Phi \), and if \( y \) is a root of \( p(X) \in \Bbbk \), then \( \BbbL \) has a nontrivial splitting field that contains the roots of \( p(X) \). But the latter contradicts the assumption that \( \BbbL \) is algebraically closed, because splitting fields are algebraic, as shown in \fullref{thm:splitting_field_is_finite_extension} and \fullref{thm:finite_field_extensions_are_algebraic}.

  The obtained contradiction shows that \( \Phi \) is an isomorphism.

  \ExistenceSubProof Let \( \Bbbk \) be a field. Let \( U \) be the \hyperref[def:grothendieck_universe]{Grothendieck universe} containing \( \Bbbk \). Let \( \mscrL \) be the family of all algebraic extensions of \( \Bbbk \) contained in \( U \), ordered by set inclusion. \( \mscrL \) is nonempty since any nonconstant polynomial induces a \hyperref[def:splitting_field]{splitting field}, as shown in \fullref{thm:splitting_field_existence}, and this field is algebraic.

  The supremum of an ascending chain in \( \mscrL \) is their union, which is again an algebraic extension --- every element comes from an algebraic extension, where it is the root of a polynomial over \( \Bbbk \). \Fullref{thm:zorns_lemma} then implies that \( \mscrL \) has a maximal element \( \BbbK \).

  Then \( \BbbK \) is algebraically closed because it has no nontrivial algebraic field extensions. It is thus the desired algebraic closure.
\end{proof}

\begin{proposition}\label{thm:no_finite_extensions_of_closed_fields}
  An \hyperref[def:algebraically_closed_field]{algebraically closed field} has no nontrivial finite field extensions.
\end{proposition}
\begin{proof}
  Follows from \fullref{thm:finite_field_extensions_are_algebraic} applied to \fullref{def:algebraically_closed_field/trivial_algebraic_extensions}.
\end{proof}
