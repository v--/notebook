\section{Field extensions}\label{sec:field_extensions}

\paragraph{Algebraic and transcendental elements}

\begin{definition}\label{def:field_extension_degree}\mcite[page A.V.10]{Bourbaki2003Algebra4to7}
  We define the \term[ru=степень (def. \cite[9.5.1]{Винберг2014КурсАлгебры})]{degree} of the \hyperref[def:field/submodel]{field extension} \( \BbbK \) of \( \Bbbk \) as the \hyperref[thm:vector_space_dimension]{vector space dimension} of \( \BbbK \) over \( \Bbbk \).

  We will prefer the term \term{finite extension} rather than \enquote{extension of finite degree}. We use the prefixes from the polynomial degree terminology described in \fullref{def:polynomial_degree_terminology}, e.g. we call extensions of degree \( 2 \) \enquote{quadratic extensions}.
\end{definition}
\begin{comments}
  \item As shown in \fullref{thm:quadratic_extension_minimal_polynomial}, quadratic extensions are described by quadratic minimal polynomials. Although this does not hold for extensions of higher degrees, the terminology for polynomials seems to be established.
\end{comments}

\begin{proposition}\label{thm:intermediate_field_extension_degree}
  If \( \BbbL \) is a field extension of \( \BbbK \) and \( \BbbK \) --- of \( \Bbbk \), then
  \begin{equation}\label{eq:thm:intermediate_field_extension_degree}
    \dim_\Bbbk \BbbL = \dim_\BbbK \BbbL \cdot \dim_\Bbbk \BbbK.
  \end{equation}
\end{proposition}
\begin{proof}
  We use \fullref{thm:cardinality_product_rule} on the cardinality of the corresponding bases.
\end{proof}

\begin{lemma}\label{thm:quotient_by_irreducible_polynomial}
  Let \( f(X) \) be an \hyperref[def:domain_divisibility/irreducible]{irreducible polynomial} over some \hyperref[def:field]{field} \( \Bbbk \). The \hyperref[def:algebra_over_ring/quotient]{quotient} of the \hyperref[def:polynomial_algebra]{polynomial algebra} \( \Bbbk[X] \) by the principal ideal \( \braket{ f(X) } \) is then a \hyperref[def:field/submodel]{field extension} of \( \Bbbk \) of \hyperref[def:field_extension_degree]{degree} \( \deg f \).
\end{lemma}
\begin{proof}
  Since \( f(X) \) is irreducible, \fullref{thm:def:gcd_domain/irreducible_is_prime} implies that it is prime, and since \( \Bbbk[X] \) is a principal ideal domain, \fullref{thm:def:principal_ideal_domain/prime_ideal_is_maximal} implies that the ideal \( \braket{ f(X) } \) is maximal.

  By \fullref{thm:quotient_by_maximal_ideal}, \( \Bbbk[X] / \braket{ f(X) } \) is a field.

  Furthermore, \fullref{thm:polynomial_quotient_module_dimension} implies that \( \Bbbk[X] / \braket{ f(X) } \) has dimension \( \deg f \) as a vector space over \( \Bbbk \).
\end{proof}

\begin{definition}\label{def:algebraic_element}
  Let \( \BbbK \) be an \hyperref[def:field/submodel]{extension} of the \hyperref[def:field]{field} \( \Bbbk \). We say that an element \( \alpha \) of \( \BbbK \) is \term[bg=алгебричен (елемент) (\cite[422]{Обрешков1962ВисшаАлгебра}), ru=алгебрический (елемент) (\cite[407]{Винберг2014КурсАлгебры}), en=algebraic (element) (\cite[124]{Jacobson1985BasicAlgebraI})]{algebraic} over \( \Bbbk \) if any of the following equivalent conditions hold:
  \begin{thmenum}
    \thmitem{def:algebraic_element/direct}\mcite[124]{Jacobson1985BasicAlgebraI} \( \alpha \) is a \hyperref[def:root_of_polynomial]{root} of some nonzero polynomial from \( \Bbbk[X] \).

    \thmitem{def:algebraic_element/embedding}\mimprovised The \hyperref[con:evaluation_homomorphism]{evaluation map} \( \Phi_\alpha: \Bbbk[X] \to \BbbK \), sending the monomial \( X \) to \( \alpha \), is not \hyperref[def:function_invertibility/injective]{injective}.

    \thmitem{def:algebraic_element/quotient}\mcite[391]{Aluffi2009Algebra} There exists a \hyperref[def:monic_polynomial]{monic} \hyperref[def:domain_divisibility/irreducible]{irreducible polynomial} \( m(X) \) such that the quotient \( \Bbbk[X] / \braket{ m(X) } \) is isomorphic to the \hyperref[def:semiring_adjunction]{ring adjunction} \( \Bbbk[\alpha] \).

    \thmitem{def:algebraic_element/field}\mimprovised The ring \( \Bbbk[\alpha] \) is a \hyperref[def:field]{field}.

    \thmitem{def:algebraic_element/dimensions}\mimprovised The ring \( \Bbbk[\alpha] \) has finite rank over \( \Bbbk \).
  \end{thmenum}
\end{definition}
\begin{defproof}
  \ImplicationSubProof{def:algebraic_element/direct}{def:algebraic_element/embedding} Let \( f(X) \) be a polynomial over \( \Bbbk \) such that \( \Phi_\alpha(f) = 0 \). Then \( \Phi_\alpha(fg) = 0 \) for any polynomial \( g \), hence \( \Phi_\alpha \) is not injective.

  \ImplicationSubProof{def:algebraic_element/embedding}{def:algebraic_element/quotient} Suppose that \( \Phi_\alpha \) is not injective. Since \( \Bbbk \) is a field, \fullref{thm:def:principal_ideal_domain/field_polynomials} implies that \( \Bbbk[X] \) is a principal ideal domain. So the kernel of \( \Phi_\alpha \) is generated by a single polynomial. Let \( m(X) \) be a monic generator of \( \ker \Phi_\alpha \). By \fullref{thm:ring_zero_morphisms/isomorphism},
  \begin{equation*}
    \Bbbk[X] / \braket{ m(X) } \cong \Bbbk[\alpha].
  \end{equation*}

  It remains to show that \( m(X) \) is irreducible. Let \( m(X) = a(X) b(X) \). Since \( \Phi_\alpha(m) = 0 \), either \( \Phi_\alpha(a) = 0 \) or \( \Phi_\alpha(b) = 0 \) or both. Then at least one of the factors must belong to \( \ker \Phi_\alpha \). But \( m(X) \) is a generator, hence either \( a(X) \) or \( b(X) \) has the same degree (and thus the other is a constant polynomial). It follows that \( m(X) \) is irreducible.

  \ImplicationSubProof{def:algebraic_element/quotient}{def:algebraic_element/field} Suppose that, for some monic irreducible polynomial \( m(X) \), the quotient \( \Bbbk[X] / \braket{ m(X) } \) is isomorphic to \( \Bbbk[\alpha] \). \Fullref{thm:quotient_by_irreducible_polynomial} then implies that \( \Bbbk[X] / \braket{ m(X) } \) is a field, and hence so is \( \Bbbk[\alpha] \).

  \ImplicationSubProof{def:algebraic_element/field}{def:algebraic_element/dimensions} Suppose that \( \Bbbk[\alpha] \) is a field.

  Since \( \Bbbk[X] \) is not a field, the evaluation map \( \Phi_\alpha: \Bbbk[X] \to \Bbbk[\alpha] \) cannot be injective. \Fullref{thm:group_homomorphism_trivial_kernel} implies that the kernel of \( \Phi_\alpha \) is not trivial, and hence there is a monic polynomial \( m(X) \) of positive degree that generates it.

  \Fullref{thm:polynomial_quotient_module_dimension} then implies that \( \Bbbk[X] / \braket{ m(X) } \) has finite rank, and thus so does \( \Bbbk[\alpha] \).

  \ImplicationSubProof{def:algebraic_element/dimensions}{def:algebraic_element/direct} Suppose that \( \Bbbk[\alpha] \) has finite rank over \( \Bbbk \).

  Since \( \Bbbk[X] \) has infinite rank, the evaluation map \( \Phi_\alpha: \Bbbk[X] \to \Bbbk[\alpha] \) must not be injective and thus the kernel of \( \Phi_\alpha \) has a nonzero element \( p(X) \). Then \( \Phi_\alpha(p) = 0 \), so \( \alpha \) is a root of \( p(X) \).
\end{defproof}

\begin{definition}\label{def:algebraic_element_minimal_polynomial}\mcite[388]{Aluffi2009Algebra}
  Let \( \alpha \) be an \hyperref[def:algebraic_element]{algebraic element} over \( \Bbbk \). It thus satisfies \fullref{def:algebraic_element/quotient}, and there exists an irreducible monic element \( m(X) \) such that \( \Bbbk[\alpha] \cong \Bbbk[X] / \braket{ m(X) } \). Furthermore, \( m(X) \) is unique, up to a scalar multiple, as a generator of the kernel of \( \Phi_\alpha: \Bbbk[X] \to \Bbbk[\alpha] \).

  We call \( m(X) \) the \term[bg=минимален полином (\cite[def. VI.2]{ГеновМиховскиМоллов1991Алгебра}), ru=минимальный многочлен (\cite[410]{Винберг2014КурсАлгебры})]{minimal polynomial} of \( \alpha \).
\end{definition}

\begin{proposition}\label{thm:irreducible_polynomial_is_minimal}
  Fix a \hyperref[def:monic_polynomial]{monic} \hyperref[def:domain_divisibility/irreducible]{irreducible polynomial} \( f(X) \) over \( \Bbbk \). If \( \alpha \) is a root of \( f(X) \) is the extension \( \BbbK \), then \( f(X) \) is the \hyperref[def:algebraic_element_minimal_polynomial]{minimal polynomial} of \( \alpha \).
\end{proposition}
\begin{proof}
  The set \( I \) of all polynomials over \( \Bbbk \) for which \( \alpha \) is a root is an ideal in \( \Bbbk[X] \).

  Since \( \Bbbk \) is a field, \fullref{thm:def:euclidean_domain/polynomials} implies that \( \Bbbk[X] \) is a \hyperref[def:principal_ideal_domain]{principal ideal domain}. Then \( I \) has a unique monic generator \( m(X) \).

  Since \( f(X) \) belongs to \( I \), we must have \( f(X) = m(X) \cdot g(X) \) for some polynomial \( g(X) \). But \( f(X) \) is irreducible and \( m(X) \) is not a invertible, so it remains for \( g(X) \) to be invertible. Then \( f(X) \) generates \( I \).

  Since \( f(X) \) is also monic, it follows that it is the minimal polynomial for \( \alpha \).
\end{proof}

\begin{example}\label{ex:def:algebraic_element}
  We list examples of \hyperref[def:algebraic_element]{algebraic elements}:
  \begin{thmenum}
    \thmitem{ex:def:algebraic_element/member} Any member of the field \( \Bbbk \) itself is algebraic over \( \Bbbk \).

    \thmitem{ex:def:algebraic_element/sqrt} Consider the \hyperref[def:principal_nonnegative_nth_root]{\( n \)-th root} \( \sqrt[n]{ p } \) of (the real embedding of) an arbitrary \hyperref[def:prime_number]{prime number}.

    As an \hyperref[def:nth_root]{\( n \)-th root}, \( \sqrt[n]{ p } \) is a root of the integer polynomial \( X^n - p \). Hence, it is algebraic over \( \BbbQ \). But \fullref{thm:nth_root_is_not_rational} implies that \( \sqrt[n]{ p } \) is not rational.

    Its \hyperref[def:algebraic_element_minimal_polynomial]{minimal polynomial} is \( X^n - p \) and hence the dimension of \( \BbbQ[\sqrt[n]{ p }] \) over \( \BbbQ \) is \( n \).

    \thmitem{ex:def:algebraic_element/complex} Similarly to the above example, the \hyperref[def:complex_numbers]{imaginary unit} \( i \) is, by definition, algebraic over \( \BbbR \) with minimal polynomial \( X^2 + 1 \). It is also algebraic over \( \BbbQ \) with the same polynomial, but we obtain the field \( \BbbQ[i] \) instead.

    \thmitem{ex:def:algebraic_element/multiple_minimal}\mcite{MathSE:non_unique_minimal_polynomial} If \( \Bbbk \) is not a field, there may be multiple minimal polynomials.

    Consider the ring \( \BbbZ[X^2, X^3] \) discussed in \fullref{ex:common_polynomial_divisors/incomparable} of all polynomials without linear terms. The monomial \( X \) from \( \BbbZ[X] \) satisfies both \( Y^2 - X^2 \) and \( Y^2 + X^2Y - X^2 - X^3 \) from \( \BbbZ[X^2, X^3][Y] \), and the two are not associates.
  \end{thmenum}
\end{example}

\begin{definition}\label{def:transcendental_element}\mcite[124]{Jacobson1985BasicAlgebraI}
  We say that an element of a field extension is \term[bg=трансцендентен (елемент) (\cite[423]{Обрешков1962ВисшаАлгебра}), ru=трансцендентный (елемент) (\cite[407]{Винберг2014КурсАлгебры})]{transcendental} if it is not \hyperref[def:algebraic_element]{algebraic}.
\end{definition}

\paragraph{Algebraic extensions}

\begin{definition}\label{def:algebraic_extension}\mcite[216]{Jacobson1985BasicAlgebraI}
  We say that a field extension is \term[bg=алгебрично разширение (\cite[201]{ГеновМиховскиМоллов1991Алгебра}), ru=алгебраическое расширение (\cite[408]{Винберг2014КурсАлгебры})]{algebraic} if every element is \hyperref[def:algebraic_element]{algebraic} over the base field.
\end{definition}

\begin{proposition}\label{thm:field_is_algebraic_over_itself}
  Every field is an \hyperref[def:algebraic_extension]{algebraic extension} of itself.
\end{proposition}
\begin{proof}
  Every field element \( a \) is a root of the polynomial \( X - a \).
\end{proof}

\begin{proposition}\label{thm:finite_field_extensions_are_algebraic}
  Every \hyperref[def:field_extension_degree]{finite field extension} is \hyperref[def:algebraic_extension]{algebraic}.
\end{proposition}
\begin{proof}
  If \( \BbbK \) be a field extension of \( \Bbbk \), for every element \( \alpha \), \( \Bbbk(\alpha) \) is also a finite extension, and thus every element is algebraic.
\end{proof}

\paragraph{Simple extensions}

\begin{definition}\label{def:field_adjunction}\mcite[213]{Jacobson1985BasicAlgebraI}
  Let \( \BbbK \) be an \hyperref[def:field/submodel]{extension} of the \hyperref[def:field]{field} \( \Bbbk \) and let \( A \) be a subset of \( \Bbbk \). Denote by \( \Bbbk(A) \) the (unique) smallest extension of \( \Bbbk \) that contains all elements of \( A \). We say that \( \Bbbk(A) \) is generated by \term{adjoining} \( A \).

  As with \hyperref[def:semiring_adjunction]{(semi)ring adjunctions}, in case \( A \) is finite, we can list the individual elements as \( \Bbbk(a_1, \ldots, a_n) \).
\end{definition}
\begin{defproof}
  We must show the existence of \( \Bbbk(A) \).

  The intersection of any family of subfields of \( \BbbK \) is clearly also a subfield. \Fullref{thm:closure_operator_from_set_semilattice} implies that the intersection of all subfields containing an arbitrary subset of \( \BbbK \) is itself a subfield containing that subset.

  \Fullref{thm:closure_operator_minimality} then implies that the (unique) smallest subfield containing \( \Bbbk \cup A \) is the intersection of all subfields containing \( \Bbbk \cup A \). Since \( \BbbK \) is a subfield of itself, this intersection is nonempty, and \( \Bbbk(A) \) is well-defined.
\end{defproof}

\begin{proposition}\label{thm:field_adjunction_tower}
  Let \( \BbbK \) be an extension of \( \Bbbk \) and let \( A \) and \( B \) be subsets of \( \BbbK \). For the corresponding \hyperref[def:field_adjunction]{field adjunctions} we have
  \begin{equation*}
    \Bbbk(A \cup B) = \Bbbk(A)(B).
  \end{equation*}
\end{proposition}
\begin{comments}
  \item This also holds for \hyperref[def:semiring_adjunction]{(semi)ring adjunctions} --- see \fullref{thm:semiring_adjunction_tower}.
\end{comments}
\begin{proof}
  Clearly \( \Bbbk(A) \subseteq \Bbbk(A \cup B) \) and \( B \subseteq \Bbbk(A \cup B) \), hence \( \Bbbk(A)(B) \subseteq \Bbbk(A \cup B) \).

  Conversely, since \( \Bbbk(A)(B) \) contains \( A \cup B \), it follows that \( \Bbbk(A \cup B) \subseteq \Bbbk(A)(B) \).
\end{proof}

\begin{definition}\label{def:simple_field_extension}\mcite[214]{Jacobson1985BasicAlgebraI}
  We say that the field \( \BbbK \) is a \term[bg=просто разширение (\cite[def. VI.3]{ГеновМиховскиМоллов1991Алгебра}), ru=простое расширение (\cite[409]{Винберг2014КурсАлгебры})]{simple extension} of \( \Bbbk \) if \( \BbbK \) is obtained by \hyperref[def:field_adjunction]{adjoining} a single element to \( \Bbbk \).
\end{definition}
\begin{comments}
  \item We refer to \( \alpha \) as a \enquote{generating element} of the extension. The term \enquote{primitive element} is also used, although is usage is inconsistent among authors --- see \fullref{rem:primitive_element_terminology} --- for which reason we avoid it.
\end{comments}

\begin{proposition}\label{thm:simple_field_extension_characterization}
  Let \( \Bbbk(\alpha) \) be a \hyperref[def:simple_field_extension]{simple field extension}.
  \begin{thmenum}
    \thmitem{thm:simple_field_extension_characterization/algebraic} \( \alpha \) is \hyperref[def:algebraic_element]{algebraic} over \( \Bbbk \) if and only if \( \Bbbk(\alpha) \) equals the ring \( \Bbbk[\alpha] \), obtained by \hyperref[def:semiring_adjunction]{adjoining} \( \alpha \) to \( \Bbbk \).

    \thmitem{thm:simple_field_extension_characterization/transcendental} \( \alpha \) is \hyperref[def:transcendental_element]{transcendental} over \( \Bbbk \) if and only if \( \Bbbk(\alpha) \) is isomorphic to the \hyperref[def:rational_function_field]{field of rational functions} \( \Bbbk(X) \).
  \end{thmenum}
\end{proposition}
\begin{proof}
  \SubProofOf{thm:simple_field_extension_characterization/algebraic} One of the equivalent definitions in \fullref{def:algebraic_element} for \( \alpha \) being algebraic is \fullref{def:algebraic_element/field}, which requires \( \Bbbk[\alpha] \) to be a field. Since \( \Bbbk(\alpha) \) is by definition the smallest extension of \( \Bbbk \) containing \( \alpha \), we conclude that \( \Bbbk(\alpha) \) equals \( \Bbbk[\alpha] \), the smallest ring containing \( \alpha \).

  \SubProofOf{thm:simple_field_extension_characterization/transcendental}

  \SufficiencySubProof* Suppose that \( \alpha \) is transcendental.

  Consider the map
  \begin{equation*}
    \begin{aligned}
      &\Psi: \Bbbk(X) \to \Bbbk(\alpha), \\
      &\Psi\parens*{ \frac f g } \coloneqq \frac {\Phi_\alpha(f)} {\Phi_\alpha(g)}.
    \end{aligned}
  \end{equation*}

  It is well-defined because \( \alpha \) is, by definition, not the root of any nonzero polynomial, and hence \( \Phi_\alpha(g) \) is always nonzero. We will show that \( \Psi \) is an isomorphism.

  It is injective --- if \( \Psi(f / g) = \Psi(p / q) \), then
  \begin{equation*}
    \Phi_\alpha(fq) = \Phi_\alpha(f) \Phi_\alpha(q) = \Phi_\alpha(p) \Phi_\alpha(g) = \Phi_\alpha(pg).
  \end{equation*}

  Since \( \alpha \) is transcendental, \( \Phi_\alpha: \Bbbk[X] \to \Bbbk[\alpha] \) is injective, hence \( f(X) q(X) = p(X) g(X) \), and thus
  \begin{equation*}
    \frac {f(X)} {g(X)} = \frac {p(X)} {q(X)}.
  \end{equation*}

  The map \( \Psi \) is also surjective. The image of \( \Psi \) is a field extension of \( \Bbbk \) containing \( \alpha \). Since \( \Bbbk(\alpha) \) is the minimal such extension, it follows that the image of \( \Psi \) is \( \Bbbk(\alpha) \).

  \NecessitySubProof* Suppose that \( \Bbbk[\alpha] \) is isomorphic to the rational function field \( \Bbbk(X) \). Then \( \Bbbk[X] \) is infinite-dimensional as a vector space over \( \Bbbk \), hence \( \alpha \) is transcendental because it satisfies the negation of \fullref{def:algebraic_element/dimensions}.
\end{proof}

\begin{proposition}\label{thm:simple_extension_degree}
  The \hyperref[def:simple_field_extension]{simple field extension} \( \Bbbk(\alpha) \) is \hyperref[def:field_extension_degree]{finite} if and only if \( \alpha \) is \hyperref[def:algebraic_element]{algebraic}, in which case the \hyperref[def:vector_space_dimension]{vector space dimension} of \( \Bbbk(\alpha) \) over \( \Bbbk \) equals the \hyperref[def:polynomial_degree]{degree} of the \hyperref[def:algebraic_element_minimal_polynomial]{minimal polynomial}.
\end{proposition}
\begin{proof}
  \SufficiencySubProof Suppose that \( \Bbbk(\alpha) \) is finite-dimensional. Then \fullref{thm:simple_field_extension_characterization/transcendental} implies that \( \alpha \) cannot be transcendental because \( \Bbbk(X) \) is infinite-dimensional. It remains for \( \alpha \) to be algebraic.

  \NecessitySubProof Suppose that \( \alpha \) is algebraic over \( \Bbbk \).

  \Fullref{thm:simple_field_extension_characterization/algebraic} implies that \( \Bbbk(\alpha) \) equals \( \Bbbk[\alpha] \), and the latter is by definition isomorphic to \( \Bbbk[X] / \braket{ m(X) } \), where \( m(X) \) is the zero polynomial.

  \Fullref{thm:polynomial_quotient_module_dimension} implies that the dimension of \( \Bbbk[X] \) is the degree of \( m(X) \).
\end{proof}

\begin{corollary}\label{thm:finite_extension_degree_bound}
  Let \( \BbbK \) be a \hyperref[thm:finite_field_extensions_are_algebraic]{finite extension} of \( \Bbbk \). Then the \hyperref[def:polynomial_degree]{degree} of the \hyperref[def:algebraic_element_minimal_polynomial]{minimal polynomial} of any element of \( \BbbK \) is bounded by the \hyperref[def:vector_space_dimension]{vector space dimension} of \( \BbbK \) over \( \Bbbk \)
\end{corollary}
\begin{proof}
  \Fullref{thm:finite_field_extensions_are_algebraic} implies that \( \BbbK \) is an \hyperref[def:algebraic_extension]{algebraic extension} of \( \Bbbk \); thus, all elements of \( \BbbK \) are \hyperref[def:algebraic_element]{algebraic} and have well-defined minimal polynomials.

  Let \( \alpha \) be any element of \( \BbbK \), and let \( m(X) \) be its minimal polynomial. \Fullref{thm:simple_extension_degree} implies that the dimension of \( \Bbbk(\alpha) \) over \( \Bbbk \) is the degree of \( m(X) \). Since the dimension of \( \Bbbk(\alpha) \) cannot exceed that of \( \BbbK \), neither can the degree of \( m(X) \).
\end{proof}

\begin{proposition}\label{thm:finite_adjunction_finite_extension}
  The \hyperref[def:field_adjunction]{adjunction} \( \Bbbk(\alpha_1, \ldots, \alpha_n) \) of finitely many \hyperref[def:algebraic_element]{algebraic} elements is a \hyperref[def:field_extension_degree]{finite extension} of \( \Bbbk \).
\end{proposition}
\begin{proof}
  We will use induction on \( n \). The base case \( n = 0 \) is vacuous. Furthermore, if \( \Bbbk(\alpha_1, \ldots, \alpha_k) \) is a finite extension of \( \Bbbk \), then \fullref{thm:simple_extension_degree} implies that \( \Bbbk(\alpha_1, \ldots, \alpha_k)(u_{k+1}) \) is a finite extension of \( \Bbbk(\alpha_1, \ldots, \alpha_k) \) and, by \fullref{thm:lagranges_subgroup_theorem}, also a finite extension of \( \Bbbk \).
\end{proof}

\paragraph{Radical extensions}

\begin{definition}\label{def:radical_extension}\mcite[def. VII.7.9]{Aluffi2009Algebra}
  We say that the \hyperref[def:field/submodel]{field extension} \( \BbbK \) of \( \Bbbk \) is \term[ru=радикальное (рассширение) (\cite[\S D18.3]{Тыртышников2007ЛинейнаяАлгебра})]{radical} if there exist elements \( \alpha_1, \ldots, \alpha_n \) of \( \BbbK \) such that \( \BbbK = \Bbbk(\alpha_1, \cdots, \alpha_m) \) and, for \( k = 1, \ldots, m \), there exists a nonnegative integer \( s_k \) such that \( \alpha_k^{s_k} \) is in \( \Bbbk(\alpha_1, \ldots, \alpha_{k-1}) \).
\end{definition}
\begin{comments}
  \item There is also another distinct notion of radical extensions. Let \( p \) be \( 1 \) or a prime number, \incite[def. V.5.2]{Bourbaki2003Algebra4to7} calls a \enquote{\( p \)-radical extension} what \incite[exerc. VII.4.14]{Aluffi2009Algebra} calls a \enquote{purely inseparable extension}: in order for \( \BbbK \) to be \enquote{\( p \)-radical} over \( \Bbbk \), every element \( x \) of \( \BbbK \) must be \enquote{\( p \)-radical}, meaning that, for some nonnegative integer \( m \), \( x^{p^m} \) must belong to \( \Bbbk \).
\end{comments}

\begin{definition}\label{def:polynomial_solvable_by_radicals}\mcite[188]{Rotman2015AdvancedModernAlgebraPart1}
  We say that the univariate polynomial \( f(X) \) over \( \Bbbk \) is \term{solvable by radicals} if its \hyperref[def:splitting_field]{splitting field} belongs to some \hyperref[def:radical_extension]{radical extension} of \( \Bbbk \).
\end{definition}

\begin{example}\label{ex:def:radical_extension}
  We list some examples of\hyperref[def:radical_extension]{radical extensions}:
  \begin{thmenum}
    \thmitem{ex:def:radical_extension/triple}\mcite{MathSE:nested_radical_extensions} Consider the polynomial
    \begin{equation*}
      f(X) = X^4 - 6X^2 + 7
    \end{equation*}
    over the \hyperref[def:rational_numbers]{rational numbers}.

    We can use \Fullref{thm:real_quadratic_polynomial_roots} to solve
    \begin{equation*}
      g(Y) = Y^2 - 6Y + 7
    \end{equation*}
    as
    \begin{equation*}
      y_\pm = \frac {6 \pm \sqrt{ 36 - 28 }} 2 = 3 \pm \sqrt 2.
    \end{equation*}

    Then \( \pm \sqrt y_+ \) and \( \pm \sqrt y_- \) are the four solutions to \( f(X) \).

    We have the following tower of radical extensions:
    \begin{equation*}
      \BbbQ \subseteq \BbbQ(\sqrt 2) \subseteq \BbbQ(\sqrt 2, \sqrt{3 + \sqrt 2}) \subseteq \BbbQ(\sqrt 2, \sqrt{3 + \sqrt 2}, \sqrt{3 - \sqrt 2}).
    \end{equation*}

    Indeed, \( 2 = (\sqrt 2)^2 \) is in \( \BbbQ \), while \( 3 + \sqrt 2 \) and \( 3 - \sqrt 2 \) are both in \( \BbbQ(\sqrt 2) \).

    Therefore, \( f(X) \) is \hyperref[def:polynomial_solvable_by_radicals]{solvable by radicals}.
  \end{thmenum}
\end{example}

\begin{proposition}\label{thm:linear_monic_polynomial_irreducible}
  Over an integral domain, any monic linear polynomial is irreducible.
\end{proposition}
\begin{proof}
  Consider \( f(X) = X + a \). \Fullref{thm:polynomial_degree_arithmetic/product} implies that if
  \begin{equation*}
    f(X) = g(X) \cdot h(X),
  \end{equation*}
  then \( g(X) \) or \( h(X) \) is linear and the other is constant.

  If \( g(X) = bX + c \) and \( h(X) = d \), then
  \begin{equation*}
    g(X) \cdot h(X) = bdX + cd.
  \end{equation*}

  Then \( bd = 1 \), implying that \( h(X) = d \) is invertible. Generalizing on \( g(X) \) and \( h(X) \), we conclude that \( f(X) \) is irreducible.
\end{proof}

\begin{lemma}\label{thm:linear_polynomial_over_field_has_root}
  Every linear polynomial whose leading coefficient is invertible has a root.
\end{lemma}
\begin{proof}
  For the linear polynomial \( f(X) = aX + b \), if \( a \) is invertible, the unique root is \( -b/a \).
\end{proof}

\begin{proposition}\label{thm:small_degree_polynomial_without_roots_irreducible}
  A \hyperref[def:monic_polynomial]{\hi{monic}} quadratic or cubic univariate polynomial over an integral domain \( D \) without \hyperref[def:root_of_polynomial]{roots} in \( D \) is \hyperref[def:domain_divisibility/irreducible]{irreducible} over \( D \).
\end{proposition}
\begin{proof}
  Let \( f(X) \) be a polynomial over \( D \) of degree at most \( 3 \) that has no roots in \( D \).

  Let \( f(X) = g(X) \cdot h(X) \) and suppose that \( g(X) \) is not invertible. \Fullref{thm:leading_coefficient_of_product} implies that the leading coefficients or \( g(X) \) and \( h(X) \) multiply to \( 1 \) since \( f(X) \) is monic, hence the leading coefficient of \( g(X) \) is invertible. But \( g(X) \) itself is not invertible, and \fullref{thm:def:domain_divisibility/irreducible_polynomial_coefficients} implies it is not constant. Then \( g(X) \) has positive degree.

  Then \fullref{thm:polynomial_degree_arithmetic/product} gives us three possibilities:
  \begin{itemize}
    \item If \( g(X) \) is linear, we already saw that the leading coefficient of \( g(X) \) is invertible, and \fullref{thm:linear_polynomial_over_field_has_root} implies that \( g(X) \) has a root in \( D \). This is also a root of \( f(X) \), contradicting our initial assumption that \( f(X) \) has no root in \( D \).

    \item If \( g(X) \) is quadratic, then \( h(X) \) is linear and hence has a root, again leading to a contradiction.

    \item If \( g(X) \) is cubic, then \( h(X) \) is constant, and hence invertible because we saw that its leading coefficient is invertible.
  \end{itemize}

  Therefore, \( f(X) \) is irreducible.
\end{proof}

\begin{lemma}\label{thm:quadratic_extension_minimal_polynomial}
  The \hyperref[def:field/submodel]{field extension} \( \BbbK \) of \( \Bbbk \) is \hyperref[def:field_extension_degree]{quadratic} (i.e. has \hyperref[def:vector_space_dimension]{vector space dimension} \( 2 \)) if and only if there exists an element \( \alpha \) of \( \BbbK \) such that \( \BbbK = \Bbbk(\alpha) \) and the \hyperref[def:algebraic_element_minimal_polynomial]{minimal polynomial} of \( \alpha \) has degree \( 2 \).
\end{lemma}
\begin{comments}
  \item In particular, \( \alpha \) cannot be a member of \( \Bbbk \).
\end{comments}
\begin{proof}
  \SufficiencySubProof Suppose that \( \BbbK \) is quadratic. Let \( \alpha \) be any element of \( \BbbK \) not in \( \Bbbk \). Then \( \BbbK = \Bbbk(\alpha) \).

  \Fullref{thm:finite_field_extensions_are_algebraic} implies that \( \BbbK \) is an \hyperref[def:algebraic_extension]{algebraic extension}. Then \( \alpha \) is an \hyperref[def:algebraic_element]{algebraic element}. Let \( m(X) \) be its \hyperref[def:algebraic_element_minimal_polynomial]{minimal polynomial}.

  \Fullref{thm:simple_extension_degree} implies that the degree of \( m(X) \) is the dimension of \( \Bbbk(\alpha) \) over \( \Bbbk \). This dimension can only be \( 1 \) or \( 2 \), and, since \( \alpha \) is not in \( \Bbbk \), it must be \( 2 \).

  Therefore, \( m(X) \) has degree \( 2 \).

  \NecessitySubProof Suppose that \( \BbbK = \Bbbk(\alpha) \) and that the minimal polynomial \( m(X) \) of \( \alpha \) has degree \( 2 \).

  \Fullref{thm:simple_extension_degree} implies that the dimension of \( \Bbbk(\alpha) \) over \( \Bbbk \) equals the degree of \( m(X) \), i.e. \( \BbbK \) has dimension \( 2 \) over \( \Bbbk \).
\end{proof}

\begin{proposition}\label{thm:quadratic_extension_square_root}
  Let \( \Bbbk \) be a field whose \hyperref[def:ring_characteristic]{characteristic} is not \( 2 \) and let \( \BbbK \) be an extension of \( \Bbbk \).

  Then \( \BbbK \) is \hyperref[def:field_extension_degree]{quadratic} if and only if there exists an element \( \alpha \) of \( \BbbK \) such that \( \BbbK = \Bbbk(\alpha) \) and \( \alpha^2 \) is in \( \Bbbk \).
\end{proposition}
\begin{proof}
  \SufficiencySubProof Let \( \BbbK \) be a quadratic extension of \( \Bbbk \).

  \Fullref{thm:quadratic_extension_minimal_polynomial} implies the existence of some \( \beta \) such that \( \BbbK = \Bbbk(\beta) \) and the minimal polynomial \( m(X) \) of \( \beta \) has degree \( 2 \).

  Then there exist constants \( b \) and \( c \) such that
  \begin{equation*}
    m(X) = X^2 + bX + c.
  \end{equation*}

  Since the characteristic is different from \( 2 \), we can divide \( b \) by \( 2 \). Let \( d \) be the quotient. Then
  \begin{equation*}
    m(X) = X^2 + 2dX + c = (X + d)^2 + (c - d^2).
  \end{equation*}

  Then
  \begin{equation*}
    0 = m(\beta) = (\beta + d)^2 + (c - d^2),
  \end{equation*}
  hence
  \begin{equation*}
    (\beta + d)^2 = d^2 - c.
  \end{equation*}

  Both \( d^2 \) and \( c \) are in \( \Bbbk \), consequently so is \( (\beta + d)^2 \). Let \( \alpha \coloneqq \beta + d \).

  Then \( \alpha^2 \) is in \( \Bbbk \), but \( \alpha = \beta \) is not. Furthermore, \( \BbbK = \Bbbk(\beta) = \Bbbk(\alpha) \).

  \NecessitySubProof Suppose that \( \BbbK = \Bbbk(\alpha) \) and that \( \alpha^2 \) is in \( \Bbbk \).

  Then the polynomial \( m(X) \coloneqq X^2 - \alpha^2 \) has roots \( \alpha \) and \( -\alpha \), neither of which belong to \( \Bbbk \). \Fullref{thm:small_degree_polynomial_without_roots_irreducible} implies that \( m(X) \) is irreducible over \( \Bbbk \), and \fullref{thm:irreducible_polynomial_is_minimal} implies that \( m(X) \) is the minimal polynomial of \( \alpha \).
\end{proof}

\begin{corollary}\label{thm:quadratic_extension_is_radical}
  \hyperref[def:field_extension_degree]{Quadratic extensions} over characteristic different from \( 2 \) are \hyperref[def:radical_extension]{radical}.
\end{corollary}
\begin{proof}
  Special case of \fullref{thm:quadratic_extension_square_root}.
\end{proof}

\begin{definition}\label{def:conjugate_algebraic_element}
  Let \( \BbbK \) be an extension of \( \Bbbk \). We say that two \hyperref[def:algebraic_element]{algebraic elements} \( x \) and \( y \) of \( \BbbK \) are \term[en=conjugate (elements) (\cite[def. V.9.1]{Bourbaki2003Algebra4to7})]{conjugate} over \( \Bbbk \) if they have the same \hyperref[def:algebraic_element_minimal_polynomial]{minimal polynomial} over \( \Bbbk \).
\end{definition}
\begin{comments}
  \item If \( y \) is the unique conjugate of \( x \), like in the setting of \fullref{thm:quadratic_extension_conjugate}, we may refer to \enquote{the conjugate \( x \) over \( \Bbbk \)}.

  \item This definition is stated as a characterization by \incite[prop. V.9.2(c)]{Bourbaki2003Algebra4to7}, while the definition requires the existence of an automorphism \( \varphi \) of \( \BbbK \) fixing \( \Bbbk \) such that \( \varphi(x) = y \).

  We only state the simpler condition to avoid proving their equivalence.
\end{comments}

\begin{proposition}\label{thm:quadratic_extension_conjugate}
  Let \( \Bbbk \) be a field of characteristic different from \( 2 \) and let \( \Bbbk(\alpha) \) be a quadratic extension such that \( \alpha^2 \) is in \( \Bbbk \).

  Then every member \( a + b\alpha \) of \( \Bbbk(\alpha) \) has a unique \hyperref[def:conjugate_algebraic_element]{conjugate algebraic element} over \( \Bbbk \), namely \( a - b\alpha \).
\end{proposition}
\begin{comments}
  \item \Fullref{thm:quadratic_extension_square_root} implies that for every quadratic extension of \( \Bbbk \) such an element \( \alpha \) exists.
\end{comments}
\begin{proof}
  If \( b \) is zero, then \( a \) is a double root of \( (X - a)^2 \), and \enquote{both} roots have the same minimal polynomial --- \( X - a \).

  Otherwise, \( b \) is nonzero, and
  \begin{equation*}
    (X - (a + b \alpha)) (X - (a - b \alpha))
    =
    X^2 - 2aX + (a + b \alpha) (a - b \alpha)
    \reloset {\eqref{eq:thm:xn_minus_yn_factorization}}
    X^2 - 2aX + (a^2 - b^2 \alpha).
  \end{equation*}

  Neither \( a + b \alpha \) nor \( a - b \alpha \) belongs to \( \Bbbk \), hence polynomial has no roots over \( \Bbbk \), and \fullref{thm:small_degree_polynomial_without_roots_irreducible} implies that it is irreducible. Hence, by \fullref{thm:irreducible_polynomial_is_minimal}, it is minimal for \( a + b \beta \) and \( a - b \beta \).
\end{proof}

\paragraph{Splitting fields}

\begin{definition}\label{def:polynomial_splits_into_linear_factors}\mimprovised
  Fix \hyperref[def:integral_domain]{integral domains} \( D \subseteq E \). We say that a polynomial
  \begin{equation*}
    f(X) = \sum_{k=1}^n a_K X^k
  \end{equation*}
  over \( D \) \term[ru=разлагается на линейные множители (\cite[\S 17.6]{Тыртышников2007ЛинейнаяАлгебра}), en=splits into linear factors (\cite[235]{Lang2002Algebra})]{splits into linear factors} over \( E \) if there exist elements \( \alpha_1, \ldots, \alpha_n \) in \( E \) such that
  \begin{equation}\label{eq:def:polynomial_splits_into_linear_factors}
    f(X) = a_n \cdot \prod_{k=1}^n (X - \alpha_k).
  \end{equation}
\end{definition}
\begin{comments}
  \item \incite[235]{Lang2002Algebra} and \incite[\S 17.6]{Тыртышников2007ЛинейнаяАлгебра}) discusses splitting into linear factors for fields, however the concept seems to generalize well enough to integral domains.
\end{comments}

\begin{proposition}\label{thm:polynomial_into_linear_factors}
  A nonzero polynomial
  \begin{equation*}
    f(X) = \sum_{k=1}^n a_K X^k
  \end{equation*}
  \hyperref[def:polynomial_splits_into_linear_factors]{splits into linear factors} as
  \begin{equation*}
    f(X) = a_n \cdot \prod_{k=1}^n (X - \alpha_k)
  \end{equation*}
  if and only if \( \alpha_1, \ldots, \alpha_n \) is an enumeration of all roots of \( f(X) \) (counting multiplicities).
\end{proposition}
\begin{proof}
  \SufficiencySubProof Suppose that
  \begin{equation*}
    f(X) = a_n \cdot \prod_{k=1}^n (X - \alpha_k).
  \end{equation*}

  For every \( k = 1, \ldots, n \), since \( X - \alpha_k \) divides \( f(X) \), by definition, \( \alpha_k \) is a root of \( f(X) \). \Fullref{thm:def:integral_domain/root_limit} implies that \( f(X) \) has at most \( n \) roots counting multiplicities, therefore \( \alpha_1, \ldots, \alpha_n \) are precisely the roots of \( f(X) \).

  \NecessitySubProof Suppose that \( \alpha_1, \ldots, \alpha_n \) are the roots of \( p(X) \).

  We will show that \( f(X) \) splits into linear factors by induction on \( n \). The case \( n = 0 \) is vacuous. Suppose that every polynomial of degree \( n - 1 \) with \( n - 1 \) roots splits.

  Then \( X - \alpha_n \) divides \( f(X) \). Let \( g(X) \) be the quotient of \( f(X) \) by \( a_n (X - \alpha_n) \). \Fullref{thm:polynomial_degree_arithmetic/product} implies that \( g(X) \) has degree \( n - 1 \) and \fullref{thm:leading_coefficient_of_product} implies that \( g(X) \) is monic.

  Furthermore, for each \( k = 1, \ldots, n - 1 \), \fullref{thm:linear_monic_polynomial_irreducible} implies that \( X - \alpha_k \) is irreducible. Then, by \fullref{thm:def:gcd_domain/irreducible_is_prime}, it is prime, and thus, since \( X - \alpha_k \) divides \( f(X) = (X - \alpha_n) \cdot g(X) \), it divides either \( X - \alpha_n \) or \( g(X) \).

  If \( X - \alpha_k \) divides \( X - \alpha_n \), then \( \alpha_k = \alpha_n \). Thus, \( \alpha_k \) has multiplicity at least \( 2 \), meaning that \( (X - \alpha_k)^2 \) divides \( f(X) \). Thus, \( X - \alpha_k \) divides \( g(X) \).

  Since in both cases \( X - \alpha_k \) divides \( g(X) \), we conclude that \( \alpha_1, \ldots, \alpha_{n-1} \) are the roots of \( g(X) \). Since \( g(X) \) is monic, we have, by the inductive hypothesis,
  \begin{equation*}
    g(X) = \prod_{k=1}^{n-1} (X - \alpha_k).
  \end{equation*}

  Multiplying by \( a_n (X - \alpha_n) \), we obtain the desired factorization of \( f(X) \).
\end{proof}

\begin{definition}\label{def:splitting_field}\mcite[def. VII.4.1]{Aluffi2009Algebra}
  Suppose that the field \( \BbbK \) is an extension of \( \Bbbk \). Suppose that the polynomial
  \begin{equation*}
     f(X) = \sum_{k=0}^n a_k X^k
  \end{equation*}
  over \( \Bbbk \) \hyperref[def:polynomial_splits_into_linear_factors]{splits into linear factors} over \( \BbbK \), that is, if there exist elements \( \alpha_1, \ldots, \alpha_n \) of \( \BbbK \) such that
  \begin{equation*}
    f(X) = a_n \prod_{k=1}^n (X - \alpha_k).
  \end{equation*}

  We say that \( \BbbK \) is a \term[bg=поле на разлагане (\cite[429]{Обрешков1962ВисшаАлгебра}), ru=поле разложения (\cite[def. 9.5.2]{Винберг2014КурсАлгебры})]{splitting field} for \( f(X) \) if
  \begin{equation*}
    \BbbK \cong \Bbbk(u_1, \ldots, \alpha_n).
  \end{equation*}
\end{definition}
\begin{comments}
  \item \Fullref{thm:splitting_field_existence} implies that splitting fields always exist and are unique up to an isomorphism.
  \item Every splitting field is obviously an \hyperref[def:algebraic_extension]{algebraic extension}.
\end{comments}

\begin{lemma}\label{thm:splitting_field_uniqueness_step}
  Let \( \varphi: \Bbbk \to \Bbbl \) be an isomorphism of fields. Let \( \alpha \) be an \hyperref[def:algebraic_element]{algebraic element} over \( \Bbbk \) with \hyperref[def:algebraic_element_minimal_polynomial]{minimal polynomial}
  \begin{equation*}
     m(X) = \sum_{k=0}^n a_k X^k.
  \end{equation*}

  Consider the following polynomial over \( \Bbbl \):
  \begin{equation*}
    \widehat{m}(X) \coloneqq \sum_{k=1}^n \varphi(a_k) X^k.
  \end{equation*}

  Let \( \BbbL \) be an extension of \( \Bbbl \) in which \( \widehat{m}(X) \) has a root, say \( \beta \). Then there exists a field isomorphism \( \varphi: \Bbbk(\alpha) \to \Bbbk(\beta) \) extending \( \varphi \) and sending \( \alpha \) to \( \beta \).
\end{lemma}
\begin{proof}
  Note that the map \( f(X) \mapsto \widehat{f}(X) \) is an isomorphism from the polynomial algebra \( \Bbbk[X] \) to \( \Bbbl[X] \). It induces the following isomorphism of fields:
  \begin{equation*}
    \Bbbk(\alpha) = \Bbbk[\alpha] \cong \Bbbk[X] / \braket{ m(X) } \cong \Bbbl[X] / \braket{ \widehat{m}(X) } \cong \Bbbl[\beta] = \Bbbl(\beta).
  \end{equation*}
\end{proof}

\begin{proposition}\label{thm:splitting_field_existence}
  Every polynomial of positive degree over a field has a unique up to an isomorphism \hyperref[def:splitting_field]{splitting field}.
\end{proposition}
\begin{comments}
  \item Note that, unlike for \hyperref[rem:universal_mapping_property]{universal mapping properties}, we do not claim that this isomorphism is unique.
\end{comments}
\begin{proof}
  \ExistenceSubProof\mcite[thm. 4.3]{Jacobson1985BasicAlgebraI} Fix a field \( \Bbbk \) and consider the polynomial \( f(X) \) of positive degree \( n \). Suppose that \( f(X) \) has \( m \) irreducible factors over \( \Bbbk \).

  Clearly \( m \leq n \). We will use induction on \( s = n - m \) to show that \( f(X) \) has a splitting field.

  The base case \( n = m \) is trivial because every factor is then linear in \( \Bbbk \), hence it is its own splitting field.

  Now suppose that splitting fields exist when the difference is less than \( s \) and suppose that \( n - m = s \). Then there exists a nonlinear irreducible factor, say \( g(X) \). Let
  \begin{equation*}
    g(X) = \sum_{i=1}^l a_i X^i.
  \end{equation*}

  \Fullref{thm:quotient_by_irreducible_polynomial} implies that \( \BbbL \coloneqq \Bbbk[X] / \braket{ g(X) } \) is a field. Denote by \( \pi: \Bbbk[X] \to \BbbL \) the canonical projection.

  Consider the evaluation homomorphism \( \Phi_{\pi(X)}: \Bbbk[X] \to \BbbL \) sending \( X \) to \( \pi(X) \). We have
  \begin{equation*}
    \Phi_{\pi(X)}(g) = \sum_{k=1}^l a_i \pi(X)^i = \pi\parens*{ \sum_{k=1}^l a_i X^i } = \pi(g(X)) = 0_\BbbL.
  \end{equation*}

  Therefore, \( \alpha \coloneqq \pi(X) \) is a root of \( g(X) \) in \( \BbbL \), and the latter splits as \( g(X) = (X - \alpha) h(X) \).

  Thus, in \( \BbbL \), \( f(X) \) has at least \( m + 1 \) irreducible factors. We can now apply the inductive hypothesis to obtain a splitting field for \( p(X) \) over \( \BbbL \), where
  \begin{equation*}
    f(X) = b \cdot \prod_{j=1}^n (X - \alpha_j).
  \end{equation*}

  Since \( X - \alpha \) divides \( f(X) \), we conclude that there exists some index \( k_0 \) such that \( \alpha = \alpha_{k_0} \). Then \( \Bbbk(u_1, \ldots, \alpha_n) \) is a splitting field for \( f(X) \).

  \UniquenessSubProof Suppose that \( \BbbK \) and \( \BbbL \) are both splitting fields for the polynomial \( f(X) \) of degree \( n \) over \( \Bbbk \). Suppose that
  \begin{equation*}
    \BbbK = \Bbbk(u_1, \ldots, \alpha_n).
  \end{equation*}

  Via recursion on \( k = 0, \ldots, n \), we can construct a sequence of monomorphisms
  \begin{equation*}
    \varphi_k: \Bbbk(u_1, \ldots, \alpha_k) \to \BbbL
  \end{equation*}
  such that \( \varphi_{k+1} \) extends \( \varphi_k \).

  The base case is trivial: define \( \varphi_0 \) to be the identity on \( \Bbbk \).

  For any monomorphism \( \varphi_k \), \fullref{thm:splitting_field_uniqueness_step} gives us a monomorphism \( \varphi_{k+1} \) via the algebraic element \( \alpha_{k+1} \) over \( \Bbbk(u_1, \ldots, \alpha_k) \). Then \fullref{thm:field_adjunction_tower} implies that
  \begin{equation*}
    \Bbbk(u_1, \ldots, \alpha_k)(u_{k+1})
    =
    \Bbbk(u_1, \ldots, \alpha_k, \alpha_{k+1})
  \end{equation*}

  The last element of the sequence is
  \begin{equation*}
    \varphi_n: \BbbK \to \BbbL.
  \end{equation*}

  Since for every \( k = 1, \ldots, n \), the element \( \varphi_n(u_k) \) is a root of \( f(X) \) in \( \BbbL \), we conclude that the image of \( \varphi_n \) contains \( n \) roots of \( f(X) \), and must thus coincide with \( \BbbL \) itself.

  Thus, \( \varphi_n \) is the desired isomorphism between \( \BbbK \) and \( \BbbL \).
\end{proof}

\begin{proposition}\label{thm:splitting_field_is_finite_extension}
  Every \hyperref[def:splitting_field]{splitting field} is a \hyperref[def:field_extension_degree]{finite extension}.
\end{proposition}
\begin{proof}
  Follows from \fullref{thm:finite_adjunction_finite_extension}.
\end{proof}

\paragraph{Finite fields}

\begin{definition}\label{def:finite_field}\mimprovised
  Unsurprisingly, if a \hyperref[def:field]{field} has finite \hyperref[thm:cardinality_existence]{cardinality}, we call it a \term{finite field}.

  \Fullref{thm:finite_fields/uniqueness} implies that finite fields of the same cardinality are isomorphic, and we will not distinguish between them. Thus, if a finite field has \( q \) elements, we denote it by \( \BbbF_q \).
\end{definition}
\begin{comments}
  \item \Fullref{thm:zp_is_field} implies that, if \( p \) is a prime number, the ring of integers modulo \( p \) is a field. \Fullref{thm:finite_fields} further classifies finite fields.

  \item Finite fields are also referred to as \enquote{Galois fields}, in which case they are denoted by \( \op{GF}(q) \). Such usage can be found in \incite[104]{Berlekamp2015AlgebraicCodingTheory}, \incite[153]{Hall1986CombinatorialTheory}, \incite[163]{Кострикин2000АлгебраЧасть1} and \incite[60]{Фаддеев1984ЛекцииПоАлгебре}. \incite[def. 1.41]{LidlNiederreiter1997FiniteFields} refer only to fields of prime cardinality as \enquote{Galois fields}.
\end{comments}

\begin{theorem}[Classification of finite fields]\label{thm:finite_fields}
  \hyperref[def:finite_field]{Finite fields} can be classified as follows:
  \begin{thmenum}
    \thmitem{thm:finite_fields/characteristic} The \hyperref[def:ring_characteristic]{characteristic} of a finite field is a \hyperref[def:prime_number]{prime number}.

    \thmitem{thm:finite_fields/cardinality} The \hyperref[thm:cardinality_existence]{cardinality} of a finite field of characteristic \( p \) is a positive power of \( p \).

    \thmitem{thm:finite_fields/polynomial} Every element of a field with \( q \) elements is a root of the polynomial
    \begin{equation}\label{eq:thm:finite_fields/polynomial}
      X^q - X.
    \end{equation}

    \thmitem{thm:finite_fields/splitting} A field with \( q = p^n \) elements is a \hyperref[def:splitting_field]{splitting field} over \( \BbbZ_p \) for the polynomial \eqref{eq:thm:finite_fields/polynomial}.

    \thmitem{thm:finite_fields/uniqueness} All finite fields having the same cardinality are isomorphic.
  \end{thmenum}
\end{theorem}
\begin{proof}
  \SubProofOf{thm:finite_fields/characteristic} \Fullref{thm:def:ring_characteristic/entire} implies that the characteristic is either a prime number or zero. \Fullref{thm:def:ring_characteristic/zero} implies that it cannot be zero because then \( \BbbK \) would be infinite.

  \SubProofOf{thm:finite_fields/cardinality} Let \( \BbbK \) be a field with \( q \) elements and let \( p \) be the characteristic of \( \BbbK \). \Fullref{thm:finite_fields/characteristic} implies that \( p \) is prime.

  By definition of characteristic, \( \BbbZ_p \) is a subring of \( \BbbK \). \Fullref{thm:zp_is_field} implies that \( \BbbZ_p \) is a field.

  Then \( \BbbK \) is a vector space over \( \BbbZ_p \). \Fullref{thm:vector_space_basis_existence} shows that \( \BbbK \) has a basis, and since \( \BbbK \) has finitely many elements, this basis must be finite.

  Therefore, if \( n \) is the dimension of \( \BbbK \) over \( \BbbZ_p \), then \( q = p^n \).

  \SubProofOf{thm:finite_fields/polynomial} Let \( \BbbK \) be a field with \( q = p^n \) elements with characteristic \( p \).

  The multiplicative group of \( \BbbK \) has order \( q - 1 \). \Fullref{thm:lagranges_subgroup_theorem} implies that the order of a non-zero element \( a \) of \( \BbbK \) divides \( q - 1 \), hence \( a^{q - 1} = 1 \). We also have \( 0^q = 0 \). Therefore, for every element \( a \) of \( \BbbF_q \), we have \( a^q = a \).

  Therefore, every element of \( \BbbK \) is a root of \( X^q - X \).

  \SubProofOf{thm:finite_fields/splitting} Clearly
  \begin{equation*}
    X^q - X = \prod_{\alpha \in \BbbK} (X - \alpha).
  \end{equation*}

  Clearly adjoining all elements of \( \BbbK \) to \( \BbbZ_p \) will give \( \BbbK \), thus the latter is indeed a splitting field.

  \SubProofOf{thm:finite_fields/uniqueness} Follows from \fullref{thm:finite_fields/splitting} and \fullref{thm:splitting_field_existence}.
\end{proof}

\begin{example}\label{ex:def:finite_field}
  We list some examples of \hyperref[def:finite_field]{finite fields}.
  \begin{thmenum}
    \thmitem{ex:def:finite_field/prime} \Fullref{thm:zp_is_field} implies that, for every prime number \( p \), the ring of integers modulo \( p \) is a field.

    \thmitem{ex:def:finite_field/f4} The smallest finite field is the field \( \BbbF_2 \) of two elements. Let us consider \( \BbbF_4 \), the smallest proper extension of \( \BbbF_2 \).

    Let \( \zeta \) be an element of \( \BbbF_4 \) not in \( \BbbF_2 \). \Fullref{thm:finite_fields/splitting} implies that \( \BbbF_4 \) is a splitting field for the polynomial \( X^4 - X \), hence \( \zeta \) is algebraic over \( \BbbF_2 \).

    Since \( \BbbF_4 \) is the smallest proper extension of \( \BbbF_2 \), it coincides with \hyperref[def:simple_field_extension]{simple extension} \( \BbbF_2(\zeta) \). \Fullref{thm:simple_field_extension_characterization/algebraic} implies that, since \( \zeta \) is algebraic, then \( \BbbF_2(\zeta) = \BbbF_2[\zeta] \).

    Then every element of \( \BbbF_4 \) has the form
    \begin{equation*}
      a \zeta + b,
    \end{equation*}
    where \( a \) and \( b \) are in \( \BbbF_2 \).

    The elements of \( \BbbF_4 \) are thus \( 0 \), \( 1 \), \( \zeta \) and \( \zeta + 1 \).

    Let us examine arithmetic in \( \BbbF_4 \). \Fullref{thm:binomial_theorem} implies that
    \begin{equation*}
      (\zeta + 1)^3 = \zeta^3 + 3\zeta^2 + 3\zeta + 1^3 = \sum_{k=0}^3 \zeta^k.
    \end{equation*}

    But as a root of \( X^4 - X \), \( \zeta + 1 \) satisfies
    \begin{equation*}
      (\zeta + 1)^3 = 1.
    \end{equation*}

    Then
    \begin{equation*}
      \zeta^3 + \zeta^2 + \zeta = 0.
    \end{equation*}

    But \( \zeta^3 = 1 \) also, hence
    \begin{equation}\label{eq:ex:def:finite_field/f4/minimal/characteristic}
      \zeta^2 = -(\zeta + 1) = \zeta + 1.
    \end{equation}

    We have thus obtained the minimal polynomial of \( \zeta \):
    \begin{equation}\label{eq:ex:def:finite_field/f4/minimal}
      X^2 + X + 1.
    \end{equation}

    Indeed, it is clearly monic, and \fullref{thm:small_degree_polynomial_without_roots_irreducible} implies that is irreducible because neither \( 0 \) nor \( 1 \) are roots.

    This is the third \hyperref[def:cyclotomic_polynomial]{cyclotomic polynomial}.

    It is also the minimal polynomial of \( \zeta + 1 \) because
    \begin{equation*}
      (\zeta + 1)^2 + \zeta + \underbrace{1 + 1}_0
      \reloset {\eqref{eq:thm:binomial_theorem_positive_characteristic}} =
      (\zeta^2 + 1) + \zeta
      \reloset {\eqref{eq:ex:def:finite_field/f4/minimal/characteristic}} =
      0.
    \end{equation*}
  \end{thmenum}
\end{example}

\begin{proposition}\label{thm:functions_over_finite_fields}
  For every \hyperref[def:finite_field]{finite field} \( \BbbF_q \) and every \hyperref[def:polynomial_algebra]{polynomial ring} \( \BbbF_q[X_1, \ldots, X_n] \) in finitely many indeterminates, there exists an \( \BbbF_q \)-\hyperref[def:algebra_over_ring]{algebra} isomorphism
  \begin{equation*}
    \frac {\BbbF_q[X_1, \ldots, X_n]} {\braket{ X_i^q - X_i \given i = 1, \ldots, n }} \cong \fun(\BbbF_q^n, \BbbF_q),
  \end{equation*}
  where \( \fun(\BbbF_q^n, \BbbF_q) \) is the \hyperref[thm:functions_over_algebra]{\( \BbbF_q \)-algebra of all functions} from \( \BbbF
  _q^n \) to \( \BbbF_q \).
\end{proposition}
\begin{comments}
  \item Every coset of polynomials has a unique representative of minimal degree as discussed in \fullref{thm:representatives_in_univariate_polynomial_quotient_set}.
\end{comments}
\begin{proof}
  Consider the \hyperref[con:evaluation_homomorphism]{functional evaluation homomorphism}
  \begin{equation*}
    \Phi: \BbbF_q[X_1, \ldots, X_m] \to \fun(\BbbF_q^m, \BbbF_q).
  \end{equation*}

  By \fullref{thm:finite_field_lagrange_interpolation}, \( \Phi \) is surjective. Then, by \fullref{thm:quotient_structure_universal_property},
  \begin{equation*}
    \BbbF_q[X_1, \ldots, X_n] / \ker \Phi \cong \fun(\BbbF_q^m, \BbbF_q).
  \end{equation*}

  We will now prove that \( \ker \Phi \) equals
  \begin{equation*}
    I \coloneqq \braket{ X_i^q - X_i \given i = 1, \ldots, n }.
  \end{equation*}

  First, let \( e: \mscrX \to \BbbF_q \) be the variable assignment that assigns \( \alpha_1, \ldots, \alpha_n \) to the corresponding indeterminates. By \fullref{thm:finite_fields/polynomial}, every member of \( \BbbF_q \) is a root of \( X_i^q - X_i \). Then, for any indeterminate \( X_i \),
  \begin{equation*}
    \Phi_e(X_i^q - X_i) = \alpha_i^q - \alpha_i = 0 \pmod q.
  \end{equation*}

  Hence, the polynomial function \( \Phi(X_i^q - X_i) \) is the zero constant function. It follows that any linear combination of the polynomials \( X_i^q - X_i \) for \( i = 1, \ldots, n \) is also the zero function. Therefore, \( I \subseteq \ker \Phi \).

  We will prove the converse inclusion via induction on \( n \).

  In the case of a single indeterminate \( X \), for every polynomial \( f(X) \in \ker \Phi \), we know that the entirety of \( \BbbF_q \) are roots of \( f(X) \). By \fullref{thm:def:integral_domain/root_limit}, \( f(X) \) has at most \( q \) roots, counting multiplicities. Hence, \( X - \alpha \) divides \( f(X) \) for every \( \alpha \in \BbbF_q \). We have
  \begin{equation*}
    \underbrace{\prod_{\alpha \in \BbbF_q} (X - \alpha)}_{\mathclap{ X^q - X \T*{by \fullref{thm:finite_fields/splitting}}}} \mid f(X),
  \end{equation*}
  and hence \( f(X) \in \braket{ X^q - X } \).

  We have, up until now, shown that the entire proposition holds for the case of one indeterminate. Suppose that the proposition holds for \( n - 1 \) indeterminates and let \( f \in \BbbF_q[X_1, \ldots, X_n] \) be a nonconstant polynomial such that \( \Phi(f) \) is the zero function. Due to \fullref{thm:def:polynomial_algebra/union}, we can regard \( f \) as a univariate polynomial in \( X_n \)over \( \BbbF_q[X_1, \ldots, X_{n-1}] \). Thus,
  \begin{equation*}
    f(X_1, \ldots, X_n) = \sum_{k =0}^\infty \underbrace{\parens*{ \sum_\gamma a_{(k,\gamma)} X_1^{\gamma_1} X_2^{\gamma_1} \cdots X_{n-1}^{\gamma_{n-1}} }}_{s_k(X_1, \ldots, X_{n-1})} {X_n}^k,
  \end{equation*}
  where \( \gamma \) is a multi-index over the first \( n - 1 \) indeterminates.

  As a polynomial in \( X_n \), \( f \) has \( m \coloneqq (n-1)p \) roots \( s_1, \ldots, s_m \), which are themselves polynomials from \( \BbbF_q[X_1, \ldots, X_{n-1}] \). For some \( c \), we have
  \begin{equation*}
    f(X_1, \ldots, X_n) = c(X_1, \ldots, X_{n-1}) \prod_{j=1}^m (X_n - s_j(X_1, \ldots, X_{n-1}))
  \end{equation*}
  and
  \begin{equation*}
    0 = \Phi(f) = \Phi(c) \cdot \prod_{j=1}^m \parens[\Big]{ \Phi(X_n) - \Phi(s_j) }.
  \end{equation*}

  Since \( \BbbF_q[X_1, \ldots, X_{n-1}] \) is \hyperref[def:entire_semiring]{entire}, we conclude that either \( \Phi(c) \) is the zero function or \( \Phi(X_n) = \Phi(s_j) \) for at least one index \( 1 \leq j \leq m \). The latter is impossible, because \( \Phi(X_n) \) is linearly independent from polynomials in the first \( n - 1 \) variables.

  The inductive hypothesis holds for the polynomial \( c \), and \( \Phi(c) \) being the zero function implies
  \begin{equation*}
    c \in \braket{ X_i^q - X_i \given i = 1, \ldots, n - 1 } \subsetneq I.
  \end{equation*}

  Therefore, \( f \in I \) since \( f \) divides \( c \). We have chosen \( f \) to be an arbitrary member of \( \ker \Phi \), which implies \( \ker \Phi \subseteq I \).

  We have already shown that \( I \subseteq \ker \Phi \). We thus conclude that \( I = \ker \Phi \) and
  \begin{equation*}
    \BbbF_q[X_1, \ldots, X_m] / I \cong \fun(\BbbF_q^m, \BbbF_q).
  \end{equation*}
\end{proof}

\paragraph{Algebraically closed fields}

\begin{definition}\label{def:algebraically_closed_field}
  We say that the field \( \BbbK \) is \term[bg=алгебрически затворено (поле) (\cite[217]{ГеновМиховскиМоллов1991Алгебра}), ru=алгебраически замкнутое (поле) (\cite[106]{Винберг2014КурсАлгебры})]{algebraically closed} if any of the equivalent conditions are satisfied:
  \begin{thmenum}
    \thmitem{def:algebraically_closed_field/trivial_algebraic_extensions}\mcite[prop. 9.20(a)]{Knapp2016BasicAlgebra} \( \BbbK \) has no nontrivial \hyperref[def:algebraic_extension]{algebraic extensions}.

    \medskip

    \thmitem{def:algebraically_closed_field/linear_irreducible_polynomials}\mcite[prop. 9.20(b)]{Knapp2016BasicAlgebra} Every irreducible polynomial in \( \BbbK[X] \) is linear.

    \medskip

    \thmitem{def:algebraically_closed_field/at_least_one_root}\mcite[224]{Jacobson1985BasicAlgebraI} Every nonconstant polynomial in \( \BbbK[X] \) has at least one root in \( \BbbK \).

    \thmitem{def:algebraically_closed_field/factorization}\mimprovised Every polynomial in \( \BbbK[X] \) \hyperref[def:splitting_field]{splits into linear factors}.

    \thmitem{def:algebraically_closed_field/exactly_n_roots}\mimprovised Every polynomial in \( \BbbK[X] \) of degree \( n \) has exactly \( n \) roots in \( \BbbK \), counting the root multiplicities.
  \end{thmenum}
\end{definition}
\begin{defproof}
  \ImplicationSubProof{def:algebraically_closed_field/trivial_algebraic_extensions}{def:algebraically_closed_field/linear_irreducible_polynomials} Suppose that \( \BbbK \) has no nontrivial algebraic extensions. Let \( f(X) \) be an irreducible polynomial over \( \BbbK \) of degree \( n \). We will show that \( f(X) \) is linear.

  \Fullref{thm:quotient_by_irreducible_polynomial} implies that \( \BbbK[X] / \braket{ f(X) } \) is a field extension of \( \BbbK \). By our assumption, it must be isomorphic to \( \BbbK \) itself, that is, has dimension \( 1 \) over \( \BbbK \).

  \Fullref{thm:polynomial_quotient_module_dimension} then implies that \( f(X) \) has degree \( 1 \).

  \ImplicationSubProof{def:algebraically_closed_field/linear_irreducible_polynomials}{def:algebraically_closed_field/at_least_one_root} Suppose that every irreducible polynomial is linear.

  Any nonconstant polynomial \( f(X) \) has an \hyperref[def:irreducible_factorization]{irreducible factorization} and hence at least one irreducible factor. Each irreducible factor has exactly one root, therefore \( f(X) \) also has at least one root.

  \ImplicationSubProof{def:algebraically_closed_field/at_least_one_root}{def:algebraically_closed_field/factorization} Suppose that every nonconstant polynomial has at least one root.

  Let \( \alpha_1 \) be a root of \( f(X) \). Then \( f(X) \) is divisible by \( (X - \alpha_1) \). Using induction on the degree \( n \) of \( f(X) \), we can factor \( f(X) \) into
  \begin{equation*}
    f(X) = a (X - \alpha_1) (X - \alpha_2) \cdots (X - \alpha_n).
  \end{equation*}

  This is the desired factorization.

  \ImplicationSubProof{def:algebraically_closed_field/factorization}{def:algebraically_closed_field/exactly_n_roots} Suppose that every nonconstant polynomial splits into linear factors.

  For a polynomial of degree \( n \), there can be at most \( n \) linear factors, and each one represents a root. Then the polynomial itself has at least \( n \) roots, counting multiplicities. \Fullref{thm:def:integral_domain/root_limit} on the other hand implies that the number of roots is bounded from above \( n \), and is thus exactly \( n \).

  \ImplicationSubProof{def:algebraically_closed_field/exactly_n_roots}{def:algebraically_closed_field/trivial_algebraic_extensions} Suppose that every nonconstant polynomial of degree \( n \) has exactly \( n \) roots in \( \Bbbk \).

  Let \( \BbbK \) be an algebraic extension of \( \Bbbk \). Let \( \alpha \) be an element of \( \BbbK \). It is necessarily algebraic over \( \Bbbk \), and thus has a minimal polynomial \( m(X) \) of degree \( n \).

  If \( n > 1 \), then \( m(X) \) can be further split into linear factors, which in turn implies that \( m(X) \) is not irreducible. The obtained contradiction show that \( m(X) \) itself is linear.

  Since \( \alpha \) is a root of the linear polynomial \( m(X) \), it follows that \( \alpha \) itself belongs to \( \Bbbk \). Generalizing on \( \alpha \), we conclude that \( \BbbK \) belongs to \( \Bbbk \). The converse inclusion holds by assumption, thus \( \BbbK = \Bbbk \).
\end{defproof}

\begin{definition}\label{def:algebraic_closure}\mcite[465]{Jacobson1989BasicAlgebraII}
  We say that the \hyperref[def:algebraic_extension]{algebraic extension} \( \BbbK \) of \( \Bbbk \) is an \term[ru=алгебраическое замыкание (\cite[412]{Винберг2014КурсАлгебры})]{algebraic closure} if it is \hyperref[def:algebraically_closed_field]{algebraically closed}.
\end{definition}
\begin{comments}
  \item \Fullref{thm:algebraic_closure_existence} implies that algebraic closures always exist and are unique up to an isomorphism.
\end{comments}

\begin{proposition}\label{thm:algebraic_closure_existence}
  Every \hyperref[def:field]{field} has a unique up to a isomorphism \hyperref[def:algebraic_extension]{algebraic extension} that is \hyperref[def:algebraically_closed_field]{algebraically closed}.
\end{proposition}
\begin{comments}
  \item Like in \fullref{thm:splitting_field_existence}, we do not claim that this isomorphism is unique.
\end{comments}
\begin{proof}
  \ExistenceSubProof\mcite{Jelonek1993AlgebraicClosureProof} Let \( \Bbbk \) be a field. Let \( U \) be the \hyperref[def:grothendieck_universe]{Grothendieck universe} containing \( \Bbbk \). Let \( \mscrL \) be the family of all algebraic extensions of \( \Bbbk \) contained in \( U \), ordered by set inclusion. \( \mscrL \) is nonempty since any nonconstant polynomial induces a \hyperref[def:splitting_field]{splitting field}, as shown in \fullref{thm:splitting_field_existence}, and this field is algebraic.

  The supremum of an ascending sequence in \( \mscrL \) is their union, which is again an algebraic extension --- every element comes from an algebraic extension, where it is the root of a polynomial over \( \Bbbk \). \Fullref{thm:zorns_lemma} then implies that \( \mscrL \) has a maximal element \( \BbbK \).

  Then \( \BbbK \) is algebraically closed because it has no nontrivial algebraic field extensions. It is thus the desired algebraic closure.

  \UniquenessSubProof Let \( \BbbK \) and \( \BbbL \) be two algebraically closed algebraic extensions of \( \Bbbk \).

  Let \( \mscrF \) be the family of all field homomorphisms \( \varphi: \Bbbl \to \BbbK \), where \( \Bbbl \) is a subfield of \( \BbbL \), ordered such that \( \varphi_1 \leq \varphi_2 \) if \( \varphi_2 \) is an extension of \( \varphi_1 \). The set is nonempty because of the canonical inclusion \( \iota: \Bbbk \to \BbbK \).

  The supremum of an ascending sequence in \( \mscrF \) if the (set-theoretic) union of the homomorphisms. \Fullref{thm:zorns_lemma} then implies that \( \mscrF \) has a maximal element \( \Phi \). The domain of \( \Phi \) is \( \BbbL \), because otherwise \( \Phi \) would not be maximal.

  Therefore, we have a field homomorphism \( \Phi: \BbbL \to \BbbK \). It is also surjective because, if some element \( y \) of \( \BbbK \) has an empty preimage under \( \Phi \), and if \( y \) is a root of \( f(X) \in \Bbbk[X] \), then \( \BbbL \) has a nontrivial splitting field that contains the roots of \( f(X) \). But the latter contradicts the assumption that \( \BbbL \) is algebraically closed, because splitting fields are algebraic, as shown in \fullref{thm:splitting_field_is_finite_extension} and \fullref{thm:finite_field_extensions_are_algebraic}.

  The obtained contradiction shows that \( \Phi \) must be surjective. It is also injective as a consequence of \fullref{thm:def:ring/simple_ring_homomorphism_is_injective}.

  Thus, \( \Phi \) is the desired isomorphism.
\end{proof}

\begin{proposition}\label{thm:no_finite_extensions_of_closed_fields}
  An \hyperref[def:algebraically_closed_field]{algebraically closed field} has no nontrivial finite field extensions.
\end{proposition}
\begin{proof}
  Follows from \fullref{thm:finite_field_extensions_are_algebraic} applied to \fullref{def:algebraically_closed_field/trivial_algebraic_extensions}.
\end{proof}

\begin{proposition}\label{thm:no_finite_field_is_algebraically_closed}
  No \hyperref[def:finite_field]{finite field} is \hyperref[def:algebraically_closed_field]{algebraically closed}.
\end{proposition}
\begin{proof}
  Let \( a_1, \ldots, a_q \) be all elements of \( \BbbF_q \). Then the polynomial
  \begin{equation*}
    1 + \prod_{k=1}^q (X - a_k)
  \end{equation*}
  has no root in \( \BbbF_q \).
\end{proof}
