\subsection{Semirings}\label{subsec:semirings}

\paragraph{Distributivity in monoids}

We will start by defining semirings, and to do that we will first motivate distributivity.

\begin{proposition}\label{thm:monoid_distributivity}
  Fix an \hyperref[rem:additive_semigroup/multiplication]{additive} \hyperref[def:monoid]{monoid} \( (R, +, \cdot) \), where \( +: R \times R \to R \) is the monoid operation and \( \cdot: \BbbN \times R \to R \) is defined via \eqref{eq:rem:additive_semigroup/multiplication}.

  We have the following property, which we call \term{distributivity} of \( \cdot \) over \( + \):
  \begin{equation}\label{eq:thm:monoid_distributivity}
    n \cdot (x + y) = n \cdot x + n \cdot y.
  \end{equation}
\end{proposition}
\begin{proof}
  We use induction on \( n \). The case \( n = 0 \) is trivial. Suppose that \eqref{eq:thm:monoid_distributivity} holds. Then
  \begin{equation*}
    (n + 1) \cdot (x + y)
    \reloset {\eqref{eq:def:semigroup/exponentiation}} =
    n \cdot (x + y) + (x + y)
    \reloset {\T{ind.}} =
    n \cdot x + n \cdot y + (x + y)
    \reloset {\eqref{eq:def:semigroup/exponentiation}} =
    (n + 1) \cdot x + (n + 1) \cdot y.
  \end{equation*}
\end{proof}

\paragraph{Semirings}

\begin{definition}\label{def:semiring}\mcite[1]{Golan2010}
  A \term[ru=полупръстен (\cite[372]{ГеновМиховскиМоллов1991}), ru=полукольцо (\cite[4]{ВечтомовПетров2022})]{semiring} is a \hyperref[def:binary_operation/commutative]{commutative} \hyperref[def:monoid]{monoid} \( (R, +) \) with a second \hyperref[def:binary_operation/associative]{associative} operation \( \cdot: R \times R \to R \) called \term{multiplication}, which extends multiplication with natural numbers. The precise compatibility axioms are listed in \fullref{def:semiring/theory} because they fit nicely into first-order logic (unlike the \hyperref[def:semimodule/theory]{theory of semimodules}, for example, for which we prefer expressing these conditions in the metalogic).

  Although not strictly necessary, it will be convenient for us to assume that multiplication has an identity. If a multiplicative identity does not exist, we say that \( (R, +, \cdot) \) is \term{nonunital}. A canonical example of a nonunital semiring is a \hyperref[def:semiring_ideal]{semiring ideal}. We will not use nonunital semirings, but it is important to acknowledge their existence. In this context, if an identity exists, we will say that \( (R, + \cdot) \) is \term{unital}.

  We call \( (R, +) \) the \term{additive monoid} and \( (R, \cdot) \) the \term{multiplicative monoid} of the semiring. We also consider the \term{additive group} and the \term{multiplicative group} as the subsets of \hyperref[def:monoid_inverse]{invertible} elements. Both are instances of \fullref{thm:invertible_submonoid_is_group}. The multiplicative group is denoted by \( R^\times \); it is discussed further in \fullref{def:divisibility/invertible}.

  Semirings have the following metamathematical properties:
  \begin{thmenum}
    \thmitem{def:semiring/theory} The \hyperref[def:first_order_theory]{first-order theory} for semirings extends the \hyperref[def:monoid/theory]{theory of monoids}.

    First, we add another \hyperref[rem:first_order_formula_conventions/infix]{infix} binary functional symbol \( \cdot \) and a constant \( 1 \). The notation for the constant is justified by \fullref{thm:semiring_characteristic_homomorphism}.

    We then extend the theory of monoids with \hyperref[def:binary_operation/commutative]{commutativity} for \( + \), \hyperref[def:binary_operation/associative]{associativity} for \( \cdot \), and the following axioms:
    \begin{thmenum}
      \thmitem{def:semiring/left_distributivity} Multiplication on the left distributes over addition:
      \begin{equation}\label{eq:def:semiring/left_distributivity}
        \xi \cdot (\eta + \zeta) \doteq \xi \cdot \eta + \xi \cdot \zeta.
      \end{equation}

      \thmitem{def:semiring/right_distributivity} Multiplication on the right also distributes over addition:
      \begin{equation}\label{eq:def:semiring/right_distributivity}
        (\xi + \eta) \cdot \zeta \doteq \xi \cdot \zeta + \eta \cdot \zeta.
      \end{equation}

      If multiplication is commutative, right distributivity follows from left distributivity.

      \thmitem{def:semiring/absorption} Zero is an absorbing element:
      \begin{equation}\label{eq:def:semiring/absorption}
        \xi \cdot 0 \doteq 0 \wedge 0 \cdot \xi \doteq 0.
      \end{equation}
    \end{thmenum}

    \thmitem{def:semiring/homomorphism} A \hyperref[def:first_order_homomorphism]{first-order homomorphism} from the semiring \( R \) to \( T \) is a function \( \varphi: R \to T \) that is a \hyperref[def:monoid/homomorphism]{monoid homomorphism} both for their additive monoids also for their multiplicative monoids.

    \thmitem{def:semiring/submodel} The set \( A \subseteq R \) is a \hyperref[def:first_order_submodel]{first-order submodel} of \( R \) if it is a both \hyperref[def:monoid/submodel]{submonoid} of the additive monoid and also of the multiplicative monoid. We call \( A \) a \term{sub-semiring}.

    As a consequence of \fullref{thm:positive_formulas_preserved_under_homomorphism}, the \hyperref[def:set_valued_map/image]{image} of a homomorphism \( \varphi: R \to T \) is a sub-semiring of \( A \).

    \thmitem{def:semiring/generated} For an arbitrary set \( A \), we denote the \hyperref[def:first_order_generated_substructure]{generated submodel} by \( \braket{ A } \).

    \thmitem{def:semiring/exponentiation} As we shall see in \fullref{thm:semiring_characteristic_homomorphism}, multiplication in \( \cdot \) extends left multiplication with natural numbers in the monoid \( (R, +) \). We do have a third operation, however --- \hyperref[def:monoid/exponentiation]{monoid exponentiation} in \( (R, \cdot) \).

    For any integer \( n \), we have the fundamental property \( 1^n = 1 \).

    \thmitem{def:semiring/category} We denote the corresponding \hyperref[def:category_of_small_first_order_models]{category of \( \mscrU \)-small models} for unital semirings by \( \ucat{SRing} \) and for non-unital semirings by \( \ucat{SRng} \).

    Both are \hyperref[def:concrete_category]{concrete} over \hyperref[def:monoid/category]{\( \ucat{Mon} \)} with the forgetful functor taking the additive monoids.

    \thmitem{def:semiring/opposite}\mcite[555]{Knapp2016BasicAlgebra} The \term{opposite semiring} of \( (R, +, \cdot) \) is the semiring \( (R, +, \star) \), with multiplication defined as \( x \star y = y \cdot x \).

    \thmitem{def:semiring/commutative} If multiplication is commutative, we call the semiring itself \enquote{commutative}. Unless multiplication corresponds to function composition, most semirings we will encounter will be commutative\footnote{
    Notable exceptions to this rule are \hyperref[def:ordinal]{ordinals}. A \hyperref[def:successor_and_limit_ordinal]{limit ordinal} \( \alpha \), regarded as the set of all smaller ordinals, is a semiring. It is not commutative, however, as shown in \fullref{ex:ordinal_addition}.}.

    We denote the category of commutative semirings by \( \ucat{CSRng} \).

    \thmitem{def:semiring/trivial} Any single-element semiring is as trivial object in \( \ucat{SRng} \) in the sense of \fullref{def:trivial_object}. It is not a zero object in \( \ucat{SRing} \), although we may refer to \( \set{ 0 } \) as \enquote{the trivial semiring}.
  \end{thmenum}
\end{definition}
\begin{comments}
  \item We can also restate the identity axiom \eqref{eq:def:monoid/theory/neutral} for the multiplicative unit \( 1 \) to highlight its connection with \eqref{eq:def:semiring/absorption}:
  \begin{equation}\label{eq:def:semiring/identity}
    \xi \cdot 1 \doteq \xi \wedge 1 \cdot \xi \doteq \xi.
  \end{equation}
\end{comments}

\begin{remark}\label{rem:semiring_etymology}
  In \fullref{def:semiring}, we require semirings to have both an additive identity and a multiplicative identity. This is not consistent with semigroups defined in \fullref{def:binary_operation/associative}, which in general do not have identities.

  \incite[ch. 3]{GondranMinoux1984Graphs} suggest using \enquote{dioid} (short for \enquote{double monoid}) instead of \enquote{semiring}. \incite[xi]{Golan2010} describes how the term \enquote{dioid} may refer to semirings with idempotent addition, i.e. a general form of the tropical semirings defined in \fullref{def:tropical_semiring}.

  We thus prefer using the term \enquote{semiring} as we have defined it in \fullref{def:semiring}.
\end{remark}

\begin{example}\label{ex:def:semiring}
  We list several examples of \hyperref[def:semiring]{semirings} that are not \hyperref[def:ring]{rings}.

  \begin{thmenum}
    \thmitem{ex:def:semiring/trivial} A semiring is \hyperref[def:semiring/trivial]{trivial} if and only if \( 0_R = 1_R \). This follows from \eqref{eq:def:semiring/absorption} and \eqref{eq:def:semiring/identity}.

    As a consequence, if \( \varphi: \set{ 0 } \to R \) is a \hyperref[def:semiring/homomorphism]{homomorphism of unital semirings}, \( R \) is trivial. This is further strengthened by \fullref{thm:ring_embedding_preserves_characteristic}.

    \thmitem{ex:def:semiring/natural_numbers} The \hyperref[def:natural_numbers]{natural numbers} are the quintessential example of a semiring. We prove in \fullref{thm:natural_number_multiplication_properties} that \( \BbbN \) is a semiring.

    \thmitem{ex:def:semiring/weak_limit_cardinal} More generally, for every \hyperref[def:successor_and_limit_cardinal/weak_limit]{weak limit cardinal}, the set of smaller cardinals is a commutative semiring under \hyperref[def:cardinal_arithmetic/addition]{cardinal addition} and \hyperref[def:cardinal_arithmetic/multiplication]{cardinal multiplication}.

    \thmitem{ex:def:semiring/lattice} We discussed in \fullref{ex:def:monoid/semilattice} that in a \hyperref[def:extremal_points/bounds]{bounded lattice} \( (X, \vee, \wedge, \top, \bot) \), both \( (X, \vee, \bot) \) and \( (X, \wedge, \top) \) are monoids.

    As a consequence of \fullref{thm:def:lattice/bounded_absorption}, \( \bot \) is absorbing with respect to \( \wedge \) and \( \top \) with respect to \( \vee \). Therefore, if the lattice is \hyperref[def:distributive_lattice]{distributive}, as a consequence of \fullref{thm:def:lattice/bounded_absorption}, both \( (X, \vee, \wedge) \) and \( (X, \wedge, \vee) \) are semirings.

    We refer to these semirings are the \term{join-meet} semiring and the \term{meet-join} semiring of the lattice.

    \thmitem{def:def:semiring/power} We may presume that the power set of a semiring is also a semiring under pointwise operations, analogously to the power semigroups defined in \fullref{def:power_semigroup}.

    If \( R \) is a \hyperref[def:semiring]{semiring}, then \( \pow(R) \) is a semigroup with respect to both operations, however \( \pow(R) \) is not a semiring because the operations do not \hyperref[def:semiring/left_distributivity]{distribute}.

    Consider the ring of integers and let
    \begin{align*}
      A = \set{ -1, 1 },
      &&
      B = \set{ -1 },
      &&
      C = \set{ 1 }.
    \end{align*}

    Then
    \begin{equation*}
      A(\underbrace{B + C}_{\set{ 0 }})
      =
      \set{ 0 },
    \end{equation*}
    however
    \begin{equation*}
      \underbrace{AB}_{A} + \underbrace{AC}_{A}
      =
      \set{ -2, 0, 0, 2 }.
    \end{equation*}
  \end{thmenum}
\end{example}

\begin{definition}\label{def:tropical_semiring}\mcite[example 1.12]{Golan2010}
  Consider the additive monoid \( (\BbbN, +) \) of natural numbers or, more generally, an \hyperref[def:ordered_semigroup]{ordered} \hyperref[def:binary_operation/commutative]{commutative} \hyperref[def:monoid]{monoid} \( (M, +, \leq) \).

  We adjoin a \hyperref[def:extremal_points/top_and_bottom]{top element} \( \infty \) to \( M \) that is absorbing with respect to addition. That is, \( x + \infty = \infty \) for every \( x \in M \).

  The \( \min \)-plus semiring over \( M \) is the triple \( (M \cup \set{ \infty }, \min, +) \). The \hyperref[def:extremal_points/maximum_and_minimum]{minimum} as a binary operation plays the role of semiring addition, with \( \infty \) as the absorbing element. The usual addition in \( M \) extended with \( \infty \) plays the role of semiring multiplication, with \( 0 \) as the multiplicative identity.

  We analogously define the \( \max \)-plus semiring, adjoining a \hyperref[def:extremal_points/top_and_bottom]{bottom element} \( -\infty \) rather than a top element \( \infty \).

  We will collectively call these the \term{tropical semirings}.
\end{definition}
\begin{comments}
  \item According to \incite{Pin1994}, the name \enquote{tropical semiring} is a dedication to the Brazilian Imre Simon. The paper also introduces the terms \enquote{tropical integers}, \enquote{tropical reals}, etc. \incite[3]{Golan2010} refers to the more general notion of additively-idempotent semirings. Both reserve the term \enquote{tropical semiring} for the case where \( M = \BbbN \). \incite[ch. 3]{GondranMinoux1984Graphs} does not explicitly use the word \enquote{tropical}, but instead refers to semirings as \enquote{dioids}, and the latter term sometimes refers to additively-idempotent semirings.
\end{comments}
\begin{defproof}
  We will only show \hyperref[def:semiring/left_distributivity]{distributivity}. If \( x \leq y \), since \( \leq \) is compatible with \( + \), we have
  \begin{equation*}
    \underbrace{\min\set{ x, y }}_{x} + z = x + z \leq y + z.
  \end{equation*}

  Therefore,
  \begin{equation*}
    \min\set{ x, y } + z = \min\set{ x + z , y + z }.
  \end{equation*}
\end{defproof}

\begin{proposition}\label{thm:semiring_characteristic_homomorphism}
  For every \hyperref[def:semiring]{semiring}, multiplication extends the abelian group multiplication.

  More precisely, denote the additive identity by \( 0_R \) and the multiplicative identity by \( 1_R \). Define the following semiring homomorphism:
  \begin{equation}\label{eq:thm:semiring_characteristic_homomorphism}
    \begin{aligned}
      &\iota: \BbbN \to R \\
      &\iota(n) \coloneqq \begin{cases}
        0_R                &n = 0, \\
        \iota(n - 1) + 1_R &n > 0.
      \end{cases}
    \end{aligned}
  \end{equation}

  This is the unique homomorphism from \( \BbbN \) to \( R \). Furthermore, we have the following analogue to \eqref{eq:def:semigroup/exponentiation}:
  \begin{equation}\label{eq:thm:semiring_characteristic_homomorphism/multiplication}
    \iota(n) \cdot x \coloneqq \begin{cases}
      0_R,                      &n = 0, \\
      \iota(n - 1) \cdot x + x, &n > 1.
    \end{cases}
  \end{equation}
\end{proposition}
\begin{proof}
  First note that \eqref{eq:thm:semiring_characteristic_homomorphism/multiplication} follows from \eqref{eq:thm:semiring_characteristic_homomorphism} via \hyperref[def:semiring/right_distributivity]{right distributivity}.

  It remains to show that \( \iota \) is a monoid homomorphism, and that it is unique. Clearly \( \iota(0) = 0_R \) and \( \iota(1) = 1_R \). Proving \( \iota(n + m) = \iota(n) + \iota(m) \) and \( \iota(nm) = \iota(n) \cdot \iota(m) \) can be done via nested induction.

  Now suppose \( \varphi: \BbbN \to R \) is a homomorphism. It is clear that \( \varphi(0) = 0_R \) and \( \varphi(1) = 1_R \), and also
  \begin{equation*}
    \varphi(n + 1) = \varphi(n) + \varphi(1) = \varphi(n) + 1_R.
  \end{equation*}

  This implies \( \iota = \varphi \).
\end{proof}

\begin{proposition}\label{thm:category_of_semirings_properties}
  The \hyperref[def:semiring/category]{category of unital semirings} has the following basic properties:
  \begin{thmenum}
    \thmitem{thm:category_of_semirings_properties/initial} The \hyperref[def:integers]{ring of integers} \( \BbbZ \) is an \hyperref[def:universal_objects/initial]{initial object}.

    \thmitem{thm:category_of_semirings_properties/terminal} The \hyperref[def:semiring/trivial]{trivial semiring} \( \set{ 0 } \) is a \hyperref[def:universal_objects/terminal]{terminal object}.
  \end{thmenum}
\end{proposition}
\begin{proof}
  \SubProofOf{thm:category_of_semirings_properties/initial} Follows from \fullref{thm:semiring_characteristic_homomorphism}.
  \SubProofOf{thm:category_of_semirings_properties/terminal} Follows from \fullref{ex:def:semiring/trivial}.
\end{proof}

\paragraph{Divisibility}

\begin{definition}\label{def:divisibility}\mcite[11]{Golan2010}
  Fix an arbitrary element \( x \) in a \hyperref[def:semiring]{semiring}. If there exist elements \( l \) and \( r \) such that \( x = lr \), we say that \( l \) is a \term{left divisor} of \( x \), and that \( r \) is a \term{right divisor}.

  In a \hyperref[def:semiring/commutative]{commutative semiring}, the two notions coincide, and we simply use the term \enquote{divisor}. If \( x \) is a divisor of \( y \), we say that \( x \) \term{divides} \( y \) and write \( x \mid y \). We also say that \( y \) is a \term{multiple} of \( x \). Most rings we will encounter will be commutative, but it is useful to have the weaker notions of left and right divisors.

  \begin{thmenum}
    \thmitem{def:divisibility/zero}\mcite[4]{Golan2010} Divisors of \( 0 \) are called \term[bg=делител на нулата (\cite[def. V.2]{ГеновМиховскиМоллов1991}), ru=делитель нуля (\cite[20]{Винберг2014})]{zero divisors}. Due to \hyperref[def:semiring/absorption]{absorption}, every semiring element is a zero divisor. If \( lr = 0 \) for nonzero \( l \) and \( r \), we say that \( l \) (resp. \( r \)) is a \term{nontrivial} left (resp. right) zero divisor.

    \thmitem{def:divisibility/invertible}\mcite[84]{Lang2002} Divisors of \( 1 \) are called \term[bg=обратим (\cite[def. V.3]{ГеновМиховскиМоллов1991}), ru=обратимый (\cite[20]{Винберг2014})]{invertible}, since they are precisely the \hyperref[def:monoid_inverse]{monoid inverses} under multiplication.
  \end{thmenum}
\end{definition}
\begin{comments}
  \item The set of all invertible elements of \( R \) is precisely the \hyperref[def:semiring]{multiplicative group} \( R^\times \).
  \item Divisibility is extensively studied in \fullref{subsec:integral_domains} and, more generally, in \fullref{subsec:semiring_ideals}.
\end{comments}

\begin{example}\label{ex:def:divisibility}
  We list examples of \hyperref[def:divisibility]{semiring divisibility}:
  \begin{thmenum}
    \thmitem{ex:def:divisibility/integers} The positive integers are commutative and their left and right divisors coincide. They have no \hyperref[def:divisibility/zero]{nontrivial zero divisors} as a consequence of \fullref{thm:natural_number_multiplication_properties}.

    \thmitem{ex:def:divisibility/matrix_zero_divisors} A simple example of nontrivial zero divisors is given by the \hyperref[thm:matrix_algebra]{matrix algebra} \( \BbbZ^{2 \times 2} \). We have
    \begin{equation*}
      \underbrace
      {
        \begin{pmatrix}
          0 & 1 \\
          0 & 0
        \end{pmatrix}
      }_{L}
      \underbrace
      {
        \begin{pmatrix}
          0 & 0 \\
          0 & 1
        \end{pmatrix}
      }_{R}
      =
      \begin{pmatrix}
        0 & 0 \\
        0 & 0
      \end{pmatrix}.
    \end{equation*}

    Therefore, \( L \) is a left zero divisor and \( R \) is a right zero divisor. The two do not commute because
    \begin{equation*}
      \underbrace
      {
        \begin{pmatrix}
          0 & 0 \\
          0 & 1
        \end{pmatrix}
      }_{R}
      \underbrace
      {
        \begin{pmatrix}
          0 & 1 \\
          0 & 0
        \end{pmatrix}
      }_{L}
      =
      \begin{pmatrix}
        0 & 0 \\
        1 & 0
      \end{pmatrix}.
    \end{equation*}

    Nevertheless, \( RLRL \) is the zero matrix, so \( R \) is a left zero divisor and \( L \) is a right zero divisor.

    \thmitem{ex:def:divisibility/i_sqrt5}\mcite[388]{Knapp2016BasicAlgebra} Consider the semiring \( \BbbN[\sqrt{-5}] \) obtained by \hyperref[def:semiring_adjunction]{adjoining} the complex number \( \sqrt{-5} \) to \( \BbbN \).

    Consider the (complex) absolute value
    \begin{equation*}
      \abs[\Big]{a + b \sqrt{-5}}
      =
      \sqrt{\parens[\Big]{ a + b \sqrt{-5} }\parens[\Big]{ a - b \sqrt{-5}}}
      =
      \sqrt{a^2 + 5b^2}.
    \end{equation*}

    It preserves products, i.e.
    \begin{equation*}
      \abs[\Big]{\parens[\Big]{a + b \sqrt{-5}} \cdot \parens[\Big]{c + d \sqrt{-5}}}
      =
      \abs[\Big]{a + b \sqrt{-5}} \cdot \abs[\Big]{c + d \sqrt{-5}}.
    \end{equation*}

    In order for \( a + b\sqrt{-5} \) to be invertible, we must have \( \abs{a + b\sqrt{-5}} = 1 \), which is equivalent to
    \begin{equation*}
      \abs[\Big]{a + b\sqrt{-5}}^2 = a^2 + 5b^2 = 1.
    \end{equation*}

    Since both \( a \) and \( b \) are integers, we have \( a \geq 1 \) and \( b \geq 1 \). Thus, \( b \) must be \( 0 \) and \( a \) must be \( 1 \).

    Therefore, the unit of \( \BbbN[\sqrt{-5}] \) is \( 1 \), just like in \( \BbbN \).
  \end{thmenum}
\end{example}

\begin{proposition}\label{thm:divisibility_and_isomorphisms}
  Suppose that \( R \) and \( S \) are \hyperref[def:semiring/commutative]{commutative semirings}.

  \begin{thmenum}
    \thmitem{thm:divisibility_and_isomorphisms/divisibility} If \( \varphi: R \to S \) is any homomorphism, then \( x \mid y \) implies \( \varphi(x) \mid \varphi(y) \). The converse holds if \( \varphi \) is an isomorphism.

    \thmitem{thm:divisibility_and_isomorphisms/zero} If \( R \) and \( S \) are isomorphic, the \hyperref[def:divisibility/zero]{zero divisors} of \( R \) are precisely the zero divisors of \( S \).

    \thmitem{thm:divisibility_and_isomorphisms/invertible} If \( R \) and \( S \) are isomorphic, the \hyperref[def:divisibility/invertible]{invertible elements} of \( R \) are precisely the invertible elements of \( S \).
  \end{thmenum}
\end{proposition}
\begin{proof}
  \SubProofOf{thm:divisibility_and_isomorphisms/divisibility} If \( x \mid y \), then \( xr = y \) for some \( r \in R \). Then \( \varphi(x) \varphi(r) = \varphi(y) \), hence \( \varphi(x) \mid \varphi(y) \). If \( \varphi \) is an isomorphism, the converse follows by using \( \varphi^{-1}: S \to R \).

  \SubProofOf{thm:divisibility_and_isomorphisms/zero} Follows from \fullref{thm:divisibility_and_isomorphisms/divisibility} by noting that homomorphisms preserve zeros.

  \SubProofOf{thm:divisibility_and_isomorphisms/invertible} Follows from \fullref{thm:divisibility_and_isomorphisms/divisibility} by noting that homomorphisms preserve ones.
\end{proof}

\begin{proposition}\label{thm:semiring_divisibility_order}
  In a \hyperref[def:semiring/commutative]{commutative semiring}, the \hyperref[def:divisibility]{divisibility} relation is a \hyperref[def:preordered_set]{preorder}.
\end{proposition}
\begin{comments}
  \item Divisibility is not a partial order in general. To avoid the nonuniqueness problems described in \fullref{ex:preorder_nonuniqueness}, we sometimes instead prefer working with ideals.
\end{comments}
\begin{proof}
  Fix a semiring \( R \).

  \SubProofOf[def:binary_relation/reflexive]{reflexivity} Clearly every element of \( R \) divides itself.

  \SubProofOf[def:binary_relation/transitive]{transitivity} Let \( x \mid y \mid z \). Then there exist elements \( a \) and \( b \) such that \( y = a x \) and \( z = b y \). Hence, \( z = (ba) x \), and hence \( x \mid z \).
\end{proof}

\paragraph{Entire semirings}

\begin{definition}\label{def:entire_semiring}\mcite[4]{Golan2010}
  We say that a \hyperref[def:semiring]{semiring} is \term[ru=целостное (\cite[def. 3.5.1]{Винберг2014})]{entire} if it has no \hyperref[def:divisibility]{nontrivial zero divisors}, neither left nor right.
\end{definition}

\begin{example}\label{ex:def:entire_semiring}
  We list examples of (non-)\hyperref[def:entire_semiring]{entire} semirings:
  \begin{thmenum}
    \thmitem{ex:def:entire_semiring/domain} \hyperref[def:integral_domain]{Integral domains}, which is the dominating type of rings studied, are, by definition, entire.

    \thmitem{ex:def:entire_semiring/trivial} The \hyperref[def:semiring/trivial]{trivial semiring} is not entire because \( 1 \cdot 1 = 1 = 0 \).

    \thmitem{ex:def:entire_semiring/matrix} For \( n > 1 \), the \hyperref[thm:matrix_algebra]{matrix algebra} \( R^{n \times n} \) over any semiring is not entire.

    Indeed, consider the product
    \begin{equation*}
      \begin{pmatrix}
        1      & 0      & \cdots & 0 & 0 \\
        0      & 1      & \cdots & 0 & 0 \\
        \vdots & \vdots & \ddots & 0 & 0 \\
        0      & 0      & 0      & 1 & 0 \\
        0      & 0      & 0      & 0 & 0
      \end{pmatrix}
      \begin{pmatrix}
        0      & 0      & \cdots & 0 & 0 \\
        0      & 0      & \cdots & 0 & 0 \\
        \vdots & \vdots & \ddots & 0 & 0 \\
        0      & 0      & 0      & 0 & 0 \\
        0      & 0      & 0      & 0 & 1
      \end{pmatrix}
      \begin{pmatrix}
        0      & 0      & \cdots & 0 & 0 \\
        0      & 0      & \cdots & 0 & 0 \\
        \vdots & \vdots & \ddots & 0 & 0 \\
        0      & 0      & 0      & 0 & 0 \\
        0      & 0      & 0      & 0 & 0
      \end{pmatrix}
    \end{equation*}
  \end{thmenum}
\end{example}

\paragraph{Ordered semirings}

\begin{definition}\label{def:ordered_semiring}\mcite[223]{Golan2010}
  A (partially) \term{ordered semiring} is a semiring \( R \) with a \hyperref[def:partially_ordered_set]{partial order} \( \leq \) such that \( (R, +) \) is an \hyperref[def:ordered_semigroup]{ordered semigroup} and, additionally, \( x \leq y \) and \( 0 \leq z \) together imply \( xz \leq yz \) and \( zx \leq zy \).

  We will also use the following terminology for \( x \) itself:
  \begin{center}
    \begin{tabular}{l | l || l | l}
      Zero        & \( x = 0 \) & Nonzero     & \( x \neq 0 \) \\
      Positive    & \( x > 0 \) & Nonpositive & \( x \not> 0 \) \\
      Negative    & \( x < 0 \) & Nonnegative & \( x \not< 0 \)
    \end{tabular}
  \end{center}
\end{definition}

\begin{proposition}\label{thm:def:ordered_semiring}
  \hyperref[def:ordered_semiring]{Ordered semirings} have the following basic properties:
  \begin{thmenum}
    \thmitem{thm:def:ordered_semiring/sum} If two elements are simultaneously both positive or negative, then so is their sum.
    \thmitem{thm:def:ordered_semiring/strict_sum} If addition is \hyperref[def:binary_operation/cancellative]{cancellative}, for any element \( z \), the strict inequality \( x < y \) implies \( x + z < y + z \).
    \thmitem{thm:def:ordered_semiring/positive_prod} The product of positive elements is positive in \hyperref[def:entire_semiring]{entire} semirings and nonnegative in general.
    \thmitem{thm:def:ordered_semiring/alternating_prod} The product of a positive and negative element is negative in \hyperref[def:entire_semiring]{entire} semirings and nonpositive in general.
  \end{thmenum}
\end{proposition}
\begin{proof}
  \SubProofOf{thm:def:ordered_semiring/sum} If \( 0 < x \), then \( y \leq x + y \). If additionally \( 0 < y \), then \( 0 < y \leq x + y \).

  Similarly, if \( x < 0 \), then \( y + x \leq y \). If additionally \( y < 0 \), then \( y + x \leq y < 0 \).

  \SubProofOf{thm:def:ordered_semiring/strict_sum} Clearly \( x < y \) implies \( x + z \leq y + z \). If \( x + z = y + z \), we can cancel \( z \) to obtain \( x = y \), which is a contradiction. Hence, \( x + z < y + z \).

  \SubProofOf{thm:def:ordered_semiring/positive_prod} If \( 0 < x \) and \( 0 < y \), then \( 0 = 0 \cdot y \leq xy \).

  In an entire semiring we have \( 0 < xy \) because otherwise \( xy = 0 \) implies that \( x = 0 \) or \( y = 0 \).

  \SubProofOf{thm:def:ordered_semiring/alternating_prod} If \( x < 0 \) and \( 0 < y \), then \( xy \leq 0 \cdot y = 0 \). Again, in an entire ring the inequality would be strict.
\end{proof}

\begin{example}\label{ex:def:ordered_semiring}
  We list several examples of \hyperref[def:ordered_semiring]{ordered semirings}.

  \begin{thmenum}
    \thmitem{ex:def:ordered_semiring/natural_numbers} The \hyperref[def:natural_numbers]{natural numbers} form an ordered semiring as shown in \fullref{thm:natural_numbers_are_well_ordered}.

    \thmitem{ex:def:ordered_semiring/lattice} We discussed in \fullref{ex:def:semiring/lattice} that a \hyperref[def:extremal_points/bounds]{bounded} \hyperref[def:distributive_lattice]{distributive} \hyperref[def:lattice]{lattice} \( (X, \vee, \wedge) \) can be regarded as a semiring, and so can its opposite lattice.

    We discussed in \fullref{ex:def:ordered_semigroup/semilattice} that both \( (X, \vee) \) and \( (X, \wedge) \) are \hyperref[def:ordered_semigroup]{ordered semigroups}. Both \( (X, \vee, \wedge) \) and \( (X, \wedge, \vee) \) vacuously satisfy the condition from \fullref{def:ordered_semiring}, which makes them ordered semirings.

    All elements of the ordered semiring \( (X, \vee, \wedge) \) are nonnegative and all elements of \( (X, \wedge, \vee) \) are nonpositive. With a slight abuse of terminology, we refer to them as the \term{positive} and \term{negative} semirings of the lattice.
  \end{thmenum}
\end{example}

\begin{definition}\label{def:zerosumfree}\mcite[4]{Golan2010}
  We say that an \hyperref[rem:additive_semigroup]{additive} \hyperref[def:monoid]{monoid} is \term{zerosumfree} if the \hyperref[thm:invertible_submonoid_is_group]{additive group} is trivial. That is, if \( x + y = 0 \) implies \( x = y = 0 \).
\end{definition}

\begin{example}\label{ex:def:zerosumfree}
  We list several examples of \hyperref[def:zerosumfree]{zerosumfree} semirings:
  \begin{thmenum}
    \thmitem{ex:def:zerosumfree/natural_numbers} By \fullref{thm:natural_number_addition_properties}, the natural numbers are zerosumfree.

    \thmitem{ex:def:zerosumfree/lattice} We discussed in \fullref{ex:def:semiring/lattice} that every bounded distributive lattice \( (X, \vee, \wedge) \) has two associated semirings.

    We will show that the join-meet semiring \( (X, \vee, \wedge) \) is zerosumfree. The proof only relies on \( \vee \) being idempotent. Suppose that \( x \vee y = \bot \). Then
    \begin{equation*}
      \bot
      =
      x \vee y
      \reloset {\eqref{eq:def:binary_operation/idempotent}} =
      (x \vee x) \vee y
      \reloset {\eqref{eq:def:binary_operation/associative}} =
      x \vee (x \vee y)
      =
      x \vee \bot
      \reloset {\eqref{eq:thm:def:lattice/bounded_absorption/join}} =
      x.
    \end{equation*}

    Therefore, \( x = \bot \). But \( \bot \vee y = y \), hence \( x \vee y = \bot \) implies \( y = \bot \).

    This demonstrates that the join-meet semiring is zerosumfree.

    \thmitem{ex:def:zerosumfree/tropical} The \hyperref[def:tropical_semiring]{\( \min \)-plus semiring} \( (\BbbN \cup \set{ \infty }, \min, +) \) discussed in \fullref{def:tropical_semiring} is also zerosumfree. Indeed, \( \min \) is idempotent, and the proof is analogous to the one for lattices in \fullref{ex:def:zerosumfree/lattice}.
  \end{thmenum}
\end{example}
