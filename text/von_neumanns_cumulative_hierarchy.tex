\subsection{Von Neumann's cumulative hierarchy}\label{subsec:von_neumanns_cumulative_hierarchy}

We will now investigate how we can use set theory itself to build models of set theory. \Fullref{thm:cumulative_hierarchy_model_of_zfc} contains the important results.

\begin{definition}\label{def:cumulative_hierarchy}\mcite[63]{Jech2003}
  For every ordinal \( \alpha \) we use \fullref{rem:unbounded_transfinite_recursion} to define
  \begin{equation}\label{eq:def:cumulative_hierarchy}
    V_\alpha \coloneqq \begin{cases}
      \varnothing,                                  &\alpha = 0 \\
      \pow(V_\beta),                                &\alpha = \beta + 1 \\
      \bigcup\set{ V_\beta \given \beta < \alpha }, &\alpha \T{is a limit ordinal.}
    \end{cases}
  \end{equation}

  Each \( V_\alpha \) is called a \term{stage} and the index \( \alpha \) of a stage is called its \term{rank}. The entire proper class of stages is called the \term{cumulative hierarchy}.

  If some set \( A \) is a subset of \( V_\alpha \), but not of \( V_\beta \) for any \( \beta < \alpha \), we say that \( \alpha \) is the \term{rank} of the set \( A \) and denote it by \( \rank(A) \). We will see in \fullref{thm:axiom_of_regularity} that every set has a rank.
\end{definition}

\begin{proposition}\label{thm:def:cumulative_hierarchy}
  Without relying on the \hyperref[def:zfc/foundation]{axiom of foundation} and by assuming that ordinals are well-founded by definition, we can prove the following basic properties for \hyperref[def:cumulative_hierarchy]{Von Neumann's cumulative hierarchy}:
  \begin{thmenum}
    \thmitem{thm:def:cumulative_hierarchy/transitive} Each stage \( V_\alpha \) is a transitive set.
    \thmitem{thm:def:cumulative_hierarchy/membership} For any two ordinals \( \alpha < \beta \) we have \( V_\alpha \in V_\beta \).
    \thmitem{thm:def:cumulative_hierarchy/rank_inequality} If \( A \in B \) and both sets have ranks, then \( \rank(A) \in \rank(B) \).
    \thmitem{thm:def:cumulative_hierarchy/well_founded} Each stage \( V_\alpha \) is well-founded by set membership.
    \thmitem{thm:def:cumulative_hierarchy/subsets} We have \( \alpha < \beta \) if and only if \( V_\alpha \subsetneq V_\beta \).
    \thmitem{thm:def:cumulative_hierarchy/ordinals} For every ordinal \( \alpha \) we have \( \rank(\alpha) = \alpha \).
    \thmitem{thm:def:cumulative_hierarchy/stage_rank} For every stage \( V_\alpha \) we have \( \rank(V_\alpha) = \alpha \).
  \end{thmenum}
\end{proposition}
\begin{proof}
  \SubProofOf{thm:def:cumulative_hierarchy/transitive} The statement is vacuous for \( \alpha = 0 \). Suppose that \( \alpha > 0 \), let \( A \in V_\alpha \) and \( B \in A \). We will show that \( B \in V_\alpha \).
  \begin{itemize}
    \item Suppose that \( \alpha = \beta + 1 \) and that \( V_\beta \) is a transitive set. Then \( A \in V_\alpha \) implies that \( A \subseteq V_\beta \). Thus, \( B \in V_\beta \) and, since \( V_\beta \) is a transitive set, \( B \subseteq V_\beta \).

    Therefore, \( B \in V_\alpha = \pow(V_\beta) \).

    \item Suppose that \( \alpha \) is a limit ordinal and that \( V_\beta \) are transitive sets for every \( \beta < \alpha \). Then \( A \in V_\alpha \) implies that \( A \) belongs to \( V_{\beta_0} \) for some \( \beta_0 < \alpha \). The inductive hypothesis implies that \( A \subseteq V_{\beta_0} \). Therefore, \( A \subseteq V_\alpha \).
  \end{itemize}

  \SubProofOf{thm:def:cumulative_hierarchy/membership} Let \( \alpha < \beta \) be some ordinals. We will show that \( V_\alpha \in V_\beta \) using induction on \( \beta \).
  \begin{itemize}
    \item Suppose that \( \beta \) is a successor ordinal, i.e. \( \beta = \mu + 1 \) for some \( \mu \), and suppose that for every \( \alpha < \mu \) we have \( V_\alpha \in V_\mu \). Clearly \( V_\mu \in V_\beta \) because \( V_\mu \) is a subset of itself.

    If \( \alpha < \mu \), then \( V_\alpha \in V_\mu \) by the inductive hypothesis and, since \( V_\beta \) is a transitive set by \fullref{thm:def:cumulative_hierarchy/transitive}, \( V_\alpha \in V_\beta \).

    \item Suppose that \( \beta \) is a limit ordinal. For some fixed \( \alpha_0 \in \beta \) we have \( V_{\alpha_0} \in V_{\alpha_0 + 1} \) by what we have already proved. We also have
    \begin{equation*}
      V_\beta
      =
      \bigcup_{\alpha < \beta} V_\alpha,
    \end{equation*}
    hence \( V_{\alpha_0} \in V_\beta \).
  \end{itemize}

  \SubProofOf{thm:def:cumulative_hierarchy/rank_inequality} Let \( A \in B \) be arbitrary sets for which ranks are defined. Trichotomy holds for ordinals, so we have to show that \( \rank(A) \geq \rank(B) \) leads to a contradiction. Denote the ranks by \( \alpha \) and \( \beta \) for brevity.

  We have \( A \subseteq V_\alpha \) by definition and \( A \subseteq V_\beta \) since \( V_\alpha \in V_\beta \) and \( V_\beta \) is transitive. If we suppose that \( \rank(B) < \rank(A) \), this would mean that \( A \) belongs to a stage below \( V_\alpha \), which contradicts the minimality of \( \alpha = \rank(A) \).

  Now suppose that \( \beta = \rank(B) = \rank(A) = \alpha \).
  \begin{itemize}
    \item If \( \beta = 0 \), then \( A \in B \) is impossible.

    \item If \( \beta \) is a successor ordinal of \( \alpha \), then \( V_\beta = \pow(V_\alpha) \). Since \( B \subseteq V_\beta \) is a set of subsets of \( V_\alpha \), \( A \in B \) is a subset of \( V_\alpha \). Thus, we have \( \rank(A) \leq \alpha < \rank(A) \), which contradicts the well-foundedness of the ordinal ordering.

    \item If \( \beta \) is a limit ordinal, then \( V_\beta = \bigcup\set{ V_\alpha \given \alpha < \beta } \). As a member of \( B \), the set \( A \) belongs to some lower stage \( V_{\alpha_0} \), which again leads to \( \rank(A) < \rank(A) \).
  \end{itemize}

  Thus, it remains for \( \rank(A) < \rank(B) \).

  \SubProofOf{thm:def:cumulative_hierarchy/well_founded} We will use induction on \( \alpha \). Suppose that \( V_\beta \) is well-founded for every \( \beta < \alpha \). Aiming at a contradiction, suppose that there exists an infinitely descending sequence \( \seq{ x_k }_{k=1}^\infty \subseteq V_\alpha \). Then the sequence \( \seq{ \rank(x_k) }_{k=0}^\infty \) of ranks is an infinitely descending set of ordinals. The \hyperref[def:transitive_closure_of_a_set]{transitive closure} of the underlying set is then an ordinal by \fullref{thm:transitive_set_of_transitive_sets}. But this contradicts the well-foundedness of ordinals.

  Therefore, \( V_\alpha \) must be well-founded.

  \SubProofOf{thm:def:cumulative_hierarchy/subsets} If \( \alpha < \beta \), then from \fullref{thm:def:cumulative_hierarchy/membership} and \fullref{thm:def:cumulative_hierarchy/transitive} it follows that \( V_\alpha \subseteq V_\beta \). We will show that \( V_\alpha \neq V_\beta \). Fix some \( \mu < \beta \) and suppose that \( V_\alpha \subsetneq V_\mu \) holds for all \( \alpha < \mu \). If \( V_\alpha = V_\beta \), this would mean that \( V_\alpha \subsetneq V_\mu \subseteq V_\beta = V_\alpha \), which is a contradiction. Therefore, \( V_\alpha \subsetneq V_\beta \).

  Conversely, suppose that \( V_\alpha \subsetneq V_\beta \). Since trichotomy holds for ordinals and since \( V_\alpha \neq V_\beta \), it is sufficient to show that \( \alpha > \beta \) leads to a contradiction. If \( \alpha > \beta \), from \fullref{thm:def:cumulative_hierarchy/membership} it follows that \( V_\beta \in V_\alpha \), which implies that \( V_\beta \subsetneq V_\beta \). The obtained contradiction shows that \( \alpha < \beta \).

  \SubProofOf{thm:def:cumulative_hierarchy/ordinals} We will use transfinite induction to show that \( \rank(\alpha) = \alpha \) for every ordinal.
  \begin{itemize}
    \item The case \( \alpha = 0 \) is trivial because \( \alpha = \varnothing \subseteq \varnothing = V_0 \).

    \item Suppose that \( \alpha = \beta + 1 \) is a successor ordinal and \( \rank(\beta) = \beta \). Clearly \( \beta \in \pow(V_\beta) = V_\alpha \) and \( \set{ \beta } \in V_\alpha \). Since \( V_\alpha \) is a transitive set, we also have \( \beta \subseteq V_\alpha \). Thus,
    \begin{equation*}
      \alpha = \beta + 1 = \beta \cup \set{ \beta } \subseteq V_\alpha.
    \end{equation*}

    \item Suppose that \( \alpha \) is a limit ordinal and that for \( \rank(\beta) = \beta \) for all \( \beta < \alpha \). Since \( \beta \in V_{\beta + 1} \) for any \( \beta < \alpha \), we have
    \begin{equation*}
      \alpha
      =
      \set{ \beta \T{is an ordinal} \given \beta < \alpha }
      \subseteq
      \bigcup\set{ V_{\beta + 1} \given \beta < \alpha }
      =
      V_\alpha.
    \end{equation*}
  \end{itemize}

  \SubProofOf{thm:def:cumulative_hierarchy/stage_rank} Clearly \( V_\alpha \) is a subset of itself, hence \( \rank(V_\alpha) \leq \alpha \). In particular, the rank of \( V_\alpha \) exists.

  We will use induction on \( \alpha \) to show that \( \rank(V_\alpha) \geq \alpha \).
  \begin{itemize}
    \item For \( \alpha = 0 \) this is obvious.
    \item If \( \rank(V_\alpha) = \alpha \), then
    \begin{equation*}
      \rank(V_{\alpha + 1})
      =
      \rank(\pow(V_\alpha))
      \reloset {\ref{thm:def:cumulative_hierarchy/rank_inequality}} >
      \rank(V_\alpha)
      =
      \alpha.
    \end{equation*}

    \item Let \( \lambda \) be a limit ordinal and suppose that \( \rank(V_\alpha) = \alpha \) for any \( \alpha < \lambda \). Then
    \begin{equation*}
      \rank(V_\lambda)
      \reloset {\ref{thm:def:cumulative_hierarchy/rank_inequality}} \geq
      \sup\set{ \rank(V_\alpha) \given \alpha < \lambda }
      =
      \sup\set{ \alpha \given \alpha < \lambda }
      =
      \lambda.
    \end{equation*}
  \end{itemize}

  Therefore, \( \rank(V_\alpha) = \alpha \).
\end{proof}

\begin{theorem}[Axiom of regularity]\label{thm:axiom_of_regularity}\mcite[lemma 6.3]{Jech2003}
  Every set belongs to a stage in \hyperref[def:cumulative_hierarchy]{von Neumann's cumulative hierarchy}.

  This statement is called the \term{axiom of regularity} and in the presence of the other axioms of \logic{ZF}, it is equivalent to the \hyperref[def:zfc/foundation]{axiom of foundation}. It is much more difficult to state in the language of set theory, however.
\end{theorem}
\begin{proof}
  \ImplicationSubProof[def:zfc/foundation]{axiom of foundation}[thm:axiom_of_regularity]{axiom of regularity} Let \( A \) be a set. By \fullref{thm:transitive_closure_of_a_set}, its \hyperref[def:transitive_closure_of_a_set]{transitive closure} \( \cl^T(A) \) is a transitive set. Define
  \begin{equation*}
    D \coloneqq \set{ B \in \cl^T(A) \given B \T{does not belong to the cumulative hierarchy} }.
  \end{equation*}

  We will show that \( D \) is empty. Assume the contrary. Then by the \hyperref[def:zfc/foundation]{axiom of foundation} there exists \( B_0 \in D \) such that \( B_0 \cap D = \varnothing \). Since \( \cl^T(A) \) is a transitive set, \( B_0 \subseteq \cl^T(A) \). Thus, \( B_0 \) consists of members \( x \) of \( \cl^T(A) \) that themselves have ranks, i.e. a minimal ordinal \( \beta \) such that \( x \subseteq V_\beta \). It follows from the \hyperref[def:zfc/replacement]{axiom schema of replacement} that these ordinals form a set. Denote this set by \( C \).

  If \( x \in B_0 \), then \( x \subseteq V_\beta \) for some \( \beta \in C \) and \( x \in V_{\beta + 1} \). Thus,
  \begin{equation*}
    B_0 \subseteq \bigcup\set{ V_{\beta + 1} \given \beta \in C }.
  \end{equation*}

  Denote the union on the right by \( \alpha \). From \fullref{thm:union_of_set_of_ordinals} it follows that \( \alpha \) is an ordinal strictly larger than the ordinals in \( C \). By \fullref{thm:def:cumulative_hierarchy/membership} we have that \( V_{\beta + 1} \in V_\alpha \) for every \( \beta \in C \). Thus,
  \begin{equation*}
    B_0 \subseteq \bigcup\set{ V_{\beta + 1} \given \beta \in C } \subseteq V_\alpha.
  \end{equation*}

  This contradicts our assumption that \( B_0 \) does not belong to the cumulative hierarchy. Therefore, \( D = \varnothing \) and every member of \( \cl^T(A) \) also belongs to the cumulative hierarchy. In particular, every member of \( A \) belongs to the cumulative hierarchy.

  Define
  \begin{equation*}
    \mu \coloneqq \bigcup\set{ \rank(B) + 1 \given B \in A }.
  \end{equation*}

  Since \( B \) is a member of the stage with rank \( \rank(B) + 1 \) for every \( B \in A \), with the same reasoning as above it follows that \( A \subseteq V_\mu \).

  \ImplicationSubProof[thm:axiom_of_regularity]{axiom of regularity}[def:zfc/foundation]{axiom of foundation} Let \( A \) be any nonempty set. We will show that there exists a subset of \( A \) that is disjoint from \( A \).

  The \hyperref[thm:axiom_of_regularity]{axiom of regularity} ensures that \( A \) belongs to the von Neumann cumulative hierarchy. Let \( B \in A \) be a set with minimal rank.

  Suppose that \( B \cap A \) is not empty. Then there exists some set \( C \in A \setminus B \). From \fullref{thm:def:cumulative_hierarchy/rank_inequality} it follows that \( \rank(C) < \rank(B) < \rank(A) \). But \( C \) belongs to \( A \) and has a rank strictly smaller than \( \rank(B) \), which contradicts the minimality of \( \rank(B) \).

  The obtained contradiction shows that \( B \cap A = \varnothing \).
\end{proof}

\begin{theorem}[Model of Z]\label{thm:cumulative_hierarchy_model_of_z}\mcite[exer. 12.7 \\ exer. 12.8]{Jech2003}
  The stage \( V_{\omega + \omega} \) of the \hyperref[def:cumulative_hierarchy]{von Neumann's cumulative hierarchy} is a standard model of \logic{Z}, i.e. \logic{ZFC} without the \hyperref[def:zfc/replacement]{axiom schema of replacement}.

  More generally, a necessary and sufficient condition for \( V_\alpha \) to be a model of \logic{Z} is for \( \alpha \) to be a limit ordinal larger than \( \omega \).
\end{theorem}
\begin{proof}
  The following axioms are automatically satisfied for any stage \( V_\alpha \):
  \begin{itemize}
    \item The validity of the \hyperref[def:zfc/extensionality]{axiom of extensionality} is inherited from the metatheory.

    \item The \hyperref[def:zfc/specification]{axiom schema of specification} is satisfied because each axiom in the schema \hyperref[def:first_order_definability]{defines} a subset of \( V_\alpha \) and because \( V_\alpha \) is a transitive set. This does not necessarily require the axiom schema of specification in the metatheory --- we only need the subsets of \( V_\alpha \) defined in \fullref{def:first_order_definability}.

    \item The \hyperref[def:zfc/union]{axiom of unions} is satisfied because if \( A \in V_\alpha \) and \( C \in \bigcup A \), then there exists some \( B \in A \) such that \( C \in B \in A \). Since \( V_\alpha \) is a transitive set, it follows that \( C \in V_\alpha \).

    \item The \hyperref[def:zfc/foundation]{axiom of foundation} is satisfied for any \( V_\alpha \) due to \fullref{thm:def:cumulative_hierarchy/well_founded}. Its validity is also inherited from the metatheory, but we do not actually need the \hyperref[def:zfc/foundation]{axiom of foundation} in the metatheory.

    \item The validity of the \hyperref[def:zfc/choice]{axiom of choice} is inherited from the metatheory.

    Indeed, for any family of sets \( \mscrA \in V_\alpha \) there exists a \hyperref[def:choice_function]{choice function} \( c: \mscrA \to \bigcup \mscrA \). The set \( \set{ c(A) \given A \in \mscrA } \) then has a lower rank by \fullref{thm:def:cumulative_hierarchy/rank_inequality} and thus by \fullref{thm:def:cumulative_hierarchy/subsets} it belongs to \( V_\alpha \).
  \end{itemize}

  The rest of the axioms are satisfied whenever some easy restrictions are imposed on \( \alpha \):
  \begin{itemize}
    \item The \hyperref[def:zfc/power_set]{axiom of power sets} is satisfied by \( V_\lambda \) for a limit ordinal \( \lambda \) if the axiom of power sets holds in the metatheory.

    Indeed, if \( A \in V_\lambda \) and \( A \) has rank \( \beta \), then necessarily \( \beta < \lambda \). Since \( A \subseteq V_\beta \), it follows that \( \pow(A) \subseteq \pow(V_\beta) = V_{\beta + 1} \). But \( \beta + 1 < \lambda \) since \( \lambda \) is a limit ordinal. Therefore, \( \rank(\pow(A)) = \beta + 1 < \lambda \). From \fullref{thm:def:cumulative_hierarchy/subsets} it follows that \( V_{\beta + 1} \subsetneq V_\lambda \) and thus \( \pow(A) \subsetneq V_\lambda \).

    \item The \hyperref[def:zfc/pairing]{axiom of pairing} is also satisfied by \( V_\lambda \) for a limit ordinal \( \lambda \) if the axiom of pairing holds in the theory.

    Let \( A \) and \( B \) be members of \( V_\lambda \). Let \( \beta \) be the larger of their ranks. Then \( A \) and \( B \) are subsets of \( V_\beta \), hence members of \( \pow(V_\beta) = V_{\beta + 1} \) and thus the set \( \set{ A, B } \) has rank \( \beta + 2 \).

    Since \( \lambda \) is a limit ordinal, we have \( \beta + 2 < \lambda \) and hence \( \set{ A, B } \in V_\lambda \) by \fullref{thm:def:cumulative_hierarchy/subsets}.

    \item The \hyperref[def:zfc/infinity]{axiom of infinity} is satisfied by any ordinal \( \alpha > \omega \).

    Indeed, by \fullref{thm:def:cumulative_hierarchy/ordinals} we have \( \omega \subseteq V_\omega \) and by \fullref{thm:def:cumulative_hierarchy/subsets} we have \( \omega \in V_\alpha \) for \( \alpha \geq \omega + 1 \).
  \end{itemize}

  As discussed in \fullref{ex:countable_limit_ordinals}, \( \omega + \omega \) is the smallest ordinal that is both strictly larger than \( \omega \) and is a limit ordinal. Therefore, \( \omega + \omega \) or any larger limit ordinal is a model of \( \logic{ZFC} \) without the axiom of replacement.
\end{proof}

\begin{definition}\label{def:cofinality}\mcite[31]{Jech2003}
  The \term{cofinality} of a \hyperref[def:preordered_set]{preordered set} \( (P, \leq) \) is defined as
  \begin{equation*}
    \op{cf}(P, \leq) \coloneqq \min\set{ \card(A) \given A \T{is a cofinal subset of} P }.
  \end{equation*}
\end{definition}

\begin{proposition}\label{thm:cardinal_cofinality}
  The \hyperref[def:cofinality]{cofinality} of an infinite cardinal \( \kappa \) is
  \begin{equation*}
    \op{cf}(\kappa) = \min\set{ \card(A) \given A \T{is an unbounded subset of} \kappa }.
  \end{equation*}
\end{proposition}
\begin{proof}
  Well-foundedness of \( \kappa \) ensures that it is bounded from below, hence a subset \( A \) of \( \kappa \) is bounded from above if and only if it is bounded.

  Since \( \kappa \) itself as a limit ordinal by \fullref{thm:cardinal_is_infinite_iff_limit_ordinal}, it has no maximum. Hence, it is unbounded.

  The above reflections along with \fullref{thm:totally_ordered_cofinal_equivalences} imply that a subset \( A \) of \( \kappa \) is cofinal if and only if it is unbounded.
\end{proof}

\begin{definition}\label{def:regular_cardinal}
  We say that the infinite cardinal \( \kappa \) is \term{regular} if any of the following equivalent conditions hold:
  \begin{thmenum}
    \thmitem{def:regular_cardinal/cofinality} \( \kappa \) it is equal to its own \hyperref[def:cofinality]{cofinality}.
    \thmitem{def:regular_cardinal/unbounded_subsets} Every unbounded subset of \( \kappa \) has cardinality \( \kappa \).
  \end{thmenum}

  Note that the term \enquote{regular cardinal} is unrelated to \fullref{thm:axiom_of_regularity}.

  If \( \kappa \) is not regular, we say that it is \term{singular}. Finite ordinals are neither regular nor singular.
\end{definition}
\begin{proof}
  \ImplicationSubProof{def:regular_cardinal/cofinality}{def:regular_cardinal/unbounded_subsets} Note that \( \card(A) \leq \kappa \) for \( A \subseteq \kappa \). \Fullref{thm:cardinal_cofinality} implies that if \( A \) is unbounded, we have \( \card(A) \geq \op{cf}(\kappa) \). But \( \op{cf}(\kappa) = \kappa \), hence the result follows.

  \ImplicationSubProof{def:regular_cardinal/unbounded_subsets}{def:regular_cardinal/cofinality} If every unbounded subset of \( \kappa \) has cardinality \( \kappa \), the minimum of all such cardinalities is \( \kappa \) and hence \( \op{cf}(\kappa) = \kappa \).
\end{proof}

\begin{remark}\label{rem:strongly_inaccessible_cardinal}
  If \( \kappa \) is an \hi{uncountable} \hyperref[def:regular_cardinal]{regular} (weak or strong) \hyperref[def:successor_and_limit_cardinal]{limit cardinal}, it is commonly called a (weakly or strongly) \term{inaccessible cardinal}.

  The assumption of uncountability is sometimes dropped, however. We avoid this ambiguity by being explicit and using \enquote{uncountable regular strong limit cardinal} rather than \enquote{strongly inaccessible cardinal}.

  Furthermore, we often do not need to restrict ourselves to uncountable regular strong limit cardinals. An added benefit to this is that we can utilize the universe of hereditary finite sets \hyperref[def:universe_of_hereditary_finite_sets]{\( V_\omega \)} as a \hyperref[def:grothendieck_universe]{Grothendieck universe} in category theory.
\end{remark}

\begin{proposition}\label{thm:aleph_zero_is_regular}
  The first infinite cardinal \( \aleph_0 \) is a \hyperref[def:regular_cardinal]{regular}.
\end{proposition}
\begin{proof}
  The only strict subsets of \( \aleph_0 \) are either countable or finite and the finite sets are bounded. The only unbounded subsets of \( \aleph_0 \) have cardinality \( \aleph_0 \). Hence, \( \aleph_0 \) equals its own cofinality and is thus regular.
\end{proof}

\begin{lemma}\label{thm:regular_cardinal_stage_supremum_lemma}
  Let \( \kappa \) be a \hyperref[def:regular_cardinal]{regular cardinal} and let \( R \subseteq \kappa \). If \( \card(R) < \kappa \), then \( \sup R < \kappa \).
\end{lemma}
\begin{proof}
  We shall prove that the set \( R \) is bounded from above with respect to the membership ordering of \( \kappa \). Indeed, suppose that it is unbounded. Then \( \card(R) = \kappa \) since \( \kappa \) is regular. But this contradicts our assumption that \( \card(R) < \kappa \).

  The obtained contradiction shows that \( R \) is bounded from above (with respect to membership in \( \kappa \)). Hence, there exists some ordinal \( \rho < \kappa \) that is an upper bound of \( R \). The supremum \( \sup R \) then satisfies \( \sup R \leq \rho < \kappa \).
\end{proof}

\begin{proposition}\label{thm:regular_cardinal_stage_inverse_transitivity}
  For any regular cardinal \( \kappa \), we have
  \begin{equation*}
    A \subseteq V_\kappa \T{if and only if} (A \in V_\kappa \T{and} \card(A) < \kappa).
  \end{equation*}
\end{proposition}
\begin{proof}
  \SufficiencySubProof Let \( A \subseteq V_\kappa \) and \( \card(A) < \kappa \). Define
  \begin{equation*}
    R \coloneqq \set{ r(x) \given x \in A }.
  \end{equation*}

  Then from \fullref{thm:regular_cardinal_stage_supremum_lemma} it follows that \( \sup R < \kappa \). Denote \( \sup R \) by \( \rho \). Then
  \begin{equation*}
    x \subseteq V_\rho \T{for every} x \in A.
  \end{equation*}

  Therefore, \( A \in V_{\rho + 2} \). Since \( \kappa \) is an infinite cardinal, from \fullref{thm:cardinal_is_infinite_iff_limit_ordinal} it follows that it is a limit ordinal and \( \rho + 1 < \kappa \); then from \fullref{thm:def:cumulative_hierarchy/subsets} it follows that \( A \in V_\kappa \).

  \NecessitySubProof Follows from \fullref{thm:def:cumulative_hierarchy/transitive}.
\end{proof}

\begin{proposition}\label{thm:strong_regular_cardinal_stages}
  If \( \kappa \) is a \hyperref[rem:strongly_inaccessible_cardinal]{regular strong limit cardinal}, then \( \card(V_\alpha) < \kappa \) for every \( \alpha < \kappa \).
\end{proposition}
\begin{proof}
  We use \fullref{thm:bounded_transfinite_induction} on \( \alpha \).
  \begin{itemize}
    \item If \( \alpha = 0 \), then \( V_\alpha = \varnothing \) and hence \( \card(V_\alpha) = 0 < \kappa \).
    \item If \( \alpha < \kappa \) and \( \card(V_\alpha) < \kappa \), then
    \begin{equation*}
      \card(V_{\alpha + 1})
      \reloset {\eqref{eq:def:cumulative_hierarchy}} =
      \card(\pow(V_\alpha))
      \reloset {\ref{thm:cardinal_exponentiation_power_set}} =
      2^{\card(V_\alpha)}.
    \end{equation*}

    Since \( \kappa \) is a strong limit, we have \( \card(V_{\alpha + 1}) < \kappa \).

    \item Suppose that \( \lambda < \kappa \) is a limit ordinal and that \( \card(V_\alpha) < \kappa \) for every \( \alpha < \lambda \).

    From \fullref{thm:def:cumulative_hierarchy/subsets} we have \( V_\alpha \subsetneq V_{\alpha + 1} \) for any \( \alpha < \lambda \), hence
    \begin{equation}\label{eq:thm:def:cumulative_hierarchy/subsets/limit_inclusion}
      \card(V_\alpha) \leq \card(V_{\alpha + 1}).
    \end{equation}

    Define the set
    \begin{equation*}
      C \coloneqq \set{ \card(V_\alpha) \given \alpha < \lambda }.
    \end{equation*}

    Then
    \begin{equation*}
      \card(V_\lambda)
      \reloset {\eqref{eq:def:cumulative_hierarchy}} =
      \card\parens*{ \bigcup \set{ V_\alpha \given \alpha < \lambda } }
      \reloset {\eqref{eq:thm:def:cumulative_hierarchy/subsets/limit_inclusion}} =
      \sup\set{ \card(V_\alpha) \given \alpha < \lambda }
      =
      \sup C.
    \end{equation*}

    From \fullref{thm:regular_cardinal_stage_supremum_lemma} it follows that \( \sup C < \kappa \). Therefore, \( \card(V_\lambda) = \sup C < \kappa \).
  \end{itemize}
\end{proof}

\begin{corollary}\label{thm:strong_regular_cardinal_stage_cardinality}
  If \( \kappa \) is a \hyperref[rem:strongly_inaccessible_cardinal]{regular strong limit cardinal}, then \( \card(A) < \kappa \) for every \( A \in V_\kappa \).
\end{corollary}
\begin{proof}
  Denote by \( \alpha \) the rank of \( A \). Then
  \begin{equation*}
    \card(A) \leq \card(V_\alpha) < \kappa,
  \end{equation*}
  where the last inequality follows from \fullref{thm:strong_regular_cardinal_stages}.
\end{proof}

\begin{theorem}[Model of ZFC]\label{thm:cumulative_hierarchy_model_of_zfc}\mcite[lemma 12.13]{Jech2003}
  The stage \( V_\kappa \) of the \hyperref[def:cumulative_hierarchy]{von Neumann's cumulative hierarchy} is a standard model of \logic{ZFC} for every \hyperref[rem:strongly_inaccessible_cardinal]{uncountable regular strong limit cardinal} \( \kappa \).
\end{theorem}
\begin{comments}
  \item This theorem is an extension of \fullref{thm:cumulative_hierarchy_model_of_z}.
  \item Assuming that at least one strongly inaccessible cardinal exists, \logic{ZFC} is consistent since it has a model.
\end{comments}
\begin{proof}
  First note that \( \kappa \) is necessarily a limit ordinal by \fullref{thm:cardinal_is_infinite_iff_limit_ordinal}. It is also larger than \( \omega \) since it is uncountable. Hence, \fullref{thm:cumulative_hierarchy_model_of_z} is satisfied. We must only show that the \hyperref[def:zfc/replacement]{axiom schema of replacement} holds in \( V_\kappa \).

  Let \( A \in V_\kappa \). The \hyperref[def:zfc/replacement]{axiom schema of replacement} requires the image of every definable function from \( A \) to \( V_\kappa \) to be a member of \( V_\kappa \).

  Let \( \varphi \) be a \hyperref[def:first_order_syntax/formula]{formula} of \hyperref[def:zfc]{\logic{ZFC}} not containing \( \tau \) nor \( \sigma \) as free variables. Suppose additionally that for every \( x \in V_\kappa \) there exists a unique \( y \in V_\kappa \) such that \( \Bracks{\varphi}_{v_{\xi \to x, \eta \to y}} = T \) for every variable assignment \( v \) in (the structure of) \( V_\kappa \). These conditions ensure that \( \varphi \) can be plugged into the \hyperref[def:zfc/replacement]{axiom schema of replacement}. Define the relation
  \begin{equation*}
    f \coloneqq \set{ (x, y) \in V_\kappa^2 \given \varphi\Bracks{\xi \to x, \eta \to y} = T }.
  \end{equation*}

  Then \( f \) is a function because of our earlier restrictions on \( \varphi \). Since \( V_\kappa \) is transitive and \( A \in V_\kappa \), it is sufficient to consider the restriction of \( f\restr_A \) of \( f \). It will not be necessary to do even that since we will explicitly restrict ourselves to values of \( f \) on \( A \).

  Clearly \( f[A] \) is a subset of \( V_\kappa \) and hence \( \rank(f[A]) \leq \kappa \). In order to show that the instance of the axiom schema of replacement with \( \varphi \) holds, is sufficient to show that \( \rank(f[A]) < \kappa \).

  It is clear that
  \begin{equation*}
    \card(f[A]) \leq \card(\dom(f\restr_A)) = \card(A).
  \end{equation*}

  From \fullref{thm:strong_regular_cardinal_stage_cardinality} it follows that \( \card(A) < \kappa \) and since \( \card(f[A]) < \kappa \), from \fullref{thm:regular_cardinal_stage_inverse_transitivity} it follows that \( f[A] \in V_\kappa \).
\end{proof}

\begin{definition}\label{def:universe_of_hereditary_finite_sets}
  For this reason, \( V_\omega \) is known as the \term{universe of hereditary finite sets}.

  From \fullref{thm:aleph_zero_is_strong_limit} and \fullref{thm:aleph_zero_is_regular} it follows that \( \omega = \aleph_0 \) is a regular strong limit cardinal. Then by \fullref{thm:strong_regular_cardinal_stage_cardinality}, every member of \( V_\omega \) is finite. Because \( V_\omega \) is a transitive set, every member of every member of \( V_\omega \) is also finite. So is every member of every member of every member.
\end{definition}

\begin{proposition}\label{thm:cumulative_hierarchy_model_of_zfc_without_infinity}
  The universe of hereditary finite sets \hyperref[def:universe_of_hereditary_finite_sets]{\( V_\omega \)} is a standard model of \logic{ZFC} without the \hyperref[def:zfc/infinity]{axiom of infinity}.
\end{proposition}
\begin{proof}
  From the proof of \fullref{thm:cumulative_hierarchy_model_of_z} is follows that \( V_\omega \) is a model of \logic{ZFC} without the axiom of infinity and the axiom schema of replacement.

  \Fullref{thm:cumulative_hierarchy_model_of_zfc} shows that being a regular strong limit cardinal is sufficient for \( V_\omega \) to satisfy the axiom of replacement.
\end{proof}
