\subsection{Propositional normal forms}\label{subsec:propositional_normal_forms}

\paragraph{Conjunctive and disjunctive normal forms}

\begin{definition}\label{def:cnf_and_dnf}\mimprovised
  We will now introduce \term[ru=конъюнктивная нормальная форма (\cite[def. 6.2]{Эдельман1975}), en=conjunctive normal form (\cite[def. 1.3.10]{Hinman2005})]{conjunctive normal forms} (CNF) and \term[ru=дизъюнктивная нормальная форма (\cite[def. 6.2]{Эдельман1975}), en=disjunctive normal form (\cite[def. 1.3.10]{Hinman2005})]{disjunctive normal forms} (DNF) for propositional formulas.

  The structure of these formulas is best described by the following extension of the \hyperref[def:propositional_syntax]{propositional syntax grammar schema}:
  \begin{bnf*}
    \bnfprod{positive literal}       {\bnfpn{variable}} \\
    \bnfprod{negative literal}       {\synneg \bnfpn{variable}} \\
    \bnfprod{literal}                {\bnfpn{positive literal}       \bnfor \bnfpn{negative literal}} \\
    \bnfprod{elementary disjunction} {\bnfpn{literal} \bnfor} \\
    \bnfmore                         {\bnftsq{(} \bnfsp \bnfpn{elem. disjunction} \bnfsp \bnftsq{\( \synvee \)} \bnfsp \bnfpn{elem. disjunction} \bnfsp \bnftsq{)}} \\
    \bnfprod{CNF}                    {\bnfpn{elementary disjunction} \bnfor \bnftsq{(} \bnfsp \bnfpn{CNF}              \bnfsp \bnftsq{\( \synwedge \)} \bnfsp \bnfpn{CNF}             \bnfsp \bnftsq{)}} \\
    \bnfprod{elementary conjunction} {\bnfpn{literal} \bnfor} \\
    \bnfmore                         {\bnftsq{(} \bnfsp \bnfpn{elem. conjunction} \bnfsp \bnftsq{\( \synvee \)} \bnfsp \bnfpn{elem. conjunction} \bnfsp \bnftsq{)}} \\
    \bnfprod{DNF}                    {\bnfpn{elementary conjunction} \bnfor \bnftsq{(} \bnfsp \bnfpn{DNF}              \bnfsp \bnftsq{\( \synvee \)}   \bnfsp \bnfpn{DNF}             \bnfsp \bnftsq{)}}
  \end{bnf*}
\end{definition}
\begin{comments}
  \item The terms \enquote{literal}, \enquote{positive literal} and \enquote{negative literal} are used by \incite[40; 108]{Rosen1999}. The adjective \enquote{elementary} for conjunctions and disjunctions is based on \cite[36]{Эдельман1975}. \incite[13]{Smullyan1995} instead uses \enquote{basic conjunction}.
  \item The concepts are related but distinct from that of \hyperref[rem:lattice_polynomials]{lattice polynomials}.
  \item As usual, we utilize the convention in \fullref{rem:propositional_formula_parentheses} and avoid excessive parentheses.
\end{comments}

\begin{theorem}[Principle of duality for CNF and DNF]\label{thm:cnf_and_dnf_duality}
  Suppose we are working with \hyperref[def:propositional_semantics]{classical semantics}.

  Fix a propositional formula \( \varphi \) in \hyperref[def:cnf_and_dnf]{CNF} (resp. \hyperref[def:cnf_and_dnf]{DNF}). Consider its opposite \( \varphi^\oppos \), in which we swap all conjunctions and disjunctions, then make all positive literals negative and vice versa. Obviously \( \varphi^\oppos \) is in DNF (resp. CNF).

  Furthermore, \( \neg \varphi \) is \hyperref[def:semantic_equivalence]{semantically equivalent} to \( \varphi^\oppos \).
\end{theorem}
\begin{proof}
  Consider an elementary disjunction
  \begin{equation*}
    \psi = L_1 \synvee L_2 \synvee \cdots \synvee L_n,
  \end{equation*}
  where, for every \( k = 1, \ldots, n \), \( L_k \) is either \( P_k \) or \( \neg P_k \).

  We have
  \begin{equation*}
    \psi^\oppos = L_1^\oppos \synwedge L_2^\oppos \synwedge \cdots \synwedge L_n^\oppos.
  \end{equation*}

  Then, for any interpretation \( I \),
  \begin{equation*}
    \Bracks{\psi^\oppos}_I
    =
    \Bracks{L_1^\oppos}_I \wedge \cdots \wedge \Bracks{L_n^\oppos}_I
    =
    \oline{\Bracks{L_1}_I} \wedge \cdots \wedge \oline{\Bracks{L_n}_I}
    \reloset {\ref{thm:boolean_equivalences/de_morgan}} =
    \oline{ \Bracks{L_1}_I \vee \cdots \vee \Bracks{L_n}_I }
    =
    \oline{ \Bracks{\psi}_I }
    =
    \Bracks{\neg \psi}_I.
  \end{equation*}

  Now consider a formula in CNF
  \begin{equation*}
    \varphi = \psi_1 \synwedge \psi_2 \synwedge \cdots \synwedge \psi_m,
  \end{equation*}
  where, for every \( k = 1, \ldots, m \), \( \psi_k \) is an elementary disjunction.

  Then
  \begin{equation*}
    \Bracks{\varphi^\oppos}_I
    =
    \Bracks{\psi_1^\oppos}_I \vee \cdots \vee \Bracks{\psi_m^\oppos}_I
    =
    \oline{\Bracks{\psi_1}_I} \vee \cdots \vee \oline{\Bracks{\psi_m}_I}
    \reloset {\ref{thm:boolean_equivalences/de_morgan}} =
    \oline{ \Bracks{\psi_1}_I \wedge \cdots \wedge \Bracks{\psi_m}_I }
    =
    \oline{ \Bracks{\varphi}_I }
    =
    \Bracks{\neg \varphi}_I.
  \end{equation*}
\end{proof}

\begin{remark}\label{rem:straightforward_traversal}
  Consider the identity operator for formulas:
  \begin{equation*}
    \begin{aligned}
      \id(\varphi) \coloneqq \begin{cases}
        \varphi,                        &\varphi \in \set{ \syntop, \synbot } \\
        \varphi,                        &\varphi \in \op*{Prop} \\
        \synneg \id(\psi),              &\varphi = \synneg \psi, \\
        \id(\psi) \syncirc \id(\theta), &\varphi = \psi \syncirc \theta, {\syncirc} \in \op*{Conn}, \\
      \end{cases}
    \end{aligned}
  \end{equation*}

  When defining an operator acting on formulas, we usually handle only a few cases, and the others become straightforward. For example, replacing \( \synneg \varphi \) with \( \varphi \synimplies \synbot \), as justified by \fullref{thm:intuitionistic_equivalences/negation_bottom}, can be achieved via
  \begin{equation*}
    \begin{aligned}
      R(\varphi) \coloneqq \begin{cases}
        \varphi,                     &\varphi \in \set{ \syntop, \synbot } \\
        \varphi,                     &\varphi \in \op*{Prop} \\
        R(\psi) \synimplies \synbot, &\varphi = \synneg \psi, \\
        R(\psi) \syncirc R(\theta),  &\varphi = \psi \syncirc \theta, {\syncirc} \in \op*{Conn}, \\
      \end{cases}
    \end{aligned}
  \end{equation*}

  We can instead use the notational shorthand
  \begin{equation*}
    \begin{aligned}
      R(\varphi) \coloneqq \begin{cases}
        R(\psi) \synimplies \synbot,                                             &\varphi = \synneg \psi, \\
        \hyperref[rem:straightforward_traversal]{\T{straightforward traversal}}, &\T{otherwise.}
      \end{cases}
    \end{aligned}
  \end{equation*}

  In computer programming, this exact problem is solved via the visitor design pattern --- visitors for formulas of first-order logic can be found in \cite{code}.
\end{remark}

\begin{algorithm}[Formula to CNF or DNF]\label{alg:cnf_and_dnf}
  Again, suppose we are working with \hyperref[def:propositional_semantics]{classical semantics}.

  We will define four operators, \( C_1 \) through \( C_4 \), such that, for a \hyperref[def:propositional_syntax/formula]{propositional formula} \( \varphi \)
  \begin{equation*}
    [C_4 \bincirc C_3 \bincirc C_2 \bincirc C_1](\varphi)
  \end{equation*}
  is a formula in \hyperref[def:cnf_and_dnf]{conjunctive normal form} that is \hyperref[def:semantic_equivalence]{semantically equivalent} to \( \varphi \).

  In order to obtain a \hyperref[def:cnf_and_dnf]{disjunctive normal form} instead, we will need to use \( C_4^\oppos \) rather than \( C_4 \).

  \begin{thmenum}
    \thmitem{alg:cnf_and_dnf/constants} First, we remove the propositional constants. Choose any propositional variable, say \( P \), and let
    \begin{equation*}
      C_1(\varphi) \coloneqq \begin{cases}
        P \vee \synneg P,                                                        &\varphi = \syntop, \\
        P \wedge \synneg P,                                                      &\varphi = \synbot, \\
        \hyperref[rem:straightforward_traversal]{\T{straightforward traversal}}, &\T{otherwise.}
      \end{cases}
    \end{equation*}

    \Fullref{thm:propositional_substitution_equivalence} implies that, since \( P \vee \synneg P \gleichstark \syntop \) and \( P \wedge \synneg P \gleichstark \synbot \), then \( C_1(\varphi) \gleichstark \varphi \).

    \thmitem{alg:cnf_and_dnf/conditional} Next, we remove the conditional \( \synimplies \) and biconditional \( \syniff \). Let
    \begin{equation*}
      C_2(\varphi) \coloneqq \begin{cases}
        \synneg C_2(\psi) \synvee C_2(\theta),                                   &\varphi = \psi \synimplies \theta, \\
        C_2(\psi \synimplies \theta) \synwedge C_2(\theta \synimplies \psi),     &\varphi = \psi \syniff \theta, \\
        \hyperref[rem:straightforward_traversal]{\T{straightforward traversal}}, &\T{otherwise.}
      \end{cases}
    \end{equation*}

    \Fullref{thm:boolean_equivalences/conditional_as_disjunction} and \fullref{def:heyting_algebra/biconditional} justify the substitutions and ensure that \( C_2(\varphi) \gleichstark \varphi \).

    \thmitem{alg:cnf_and_dnf/negation} We \enquote{push} negations inwards. Let
    \begin{equation*}
      C_3(\varphi) \coloneqq \begin{cases}
        C_3(\synneg \psi) \synwedge C_3(\synneg \theta),                         &\varphi = \synneg (\psi \synvee \theta), \\
        C_3(\synneg \psi) \synvee C_3(\synneg \theta),                           &\varphi = \synneg (\psi \synwedge \theta), \\
        C_3(\psi),                                                               &\varphi = \synneg \synneg \psi, \\
        \synneg C_3(\psi),                                                       &\T{otherwise if} \varphi = \synneg \psi, \\
        \hyperref[rem:straightforward_traversal]{\T{straightforward traversal}}, &\T{otherwise.}
      \end{cases}
    \end{equation*}

    \Fullref{thm:boolean_equivalences/de_morgan} and \fullref{thm:boolean_equivalences/double_negation} ensure that \( C_3(\varphi) \gleichstark \varphi \) for any formula \( \varphi \).

    If case \( \varphi \) has no (bi)conditionals (e.g. due to \( C_2 \)), the result \( C_3(\varphi) \) has negations only before propositional variables.

    \thmitem{alg:cnf_and_dnf/distributive} Finally, we \enquote{pull} conjunctions outwards. Let
    \begin{equation*}
      C_4(\varphi) \coloneqq \begin{cases}
        C_4(\psi_1 \synvee \theta) \synwedge C_4(\psi_1 \synvee \theta),         &\varphi = \psi \synvee \theta \T{and} C_4(\psi) = \psi_1 \synwedge \psi_2, \\
        C_4(\psi \synvee \theta_1) \synwedge C_4(\psi \synvee \theta_2),         &\T{otherwise if} \varphi = \psi \synvee \theta \T{and} C_4(\theta) = \theta_1 \synwedge \theta_2, \\
        C_4(\psi) \synvee C_4(\theta),                                           &\T{otherwise if} \varphi = \psi \synvee \theta, \\
        C_4(\psi) \synwedge C_4(\theta),                                         &\varphi = \psi \synwedge \theta, \\
        \hyperref[rem:straightforward_traversal]{\T{straightforward traversal}}, &\T{otherwise.}
      \end{cases}
    \end{equation*}

    The last case is due to the usual traversal rules, and we could avoid writing it, but we wanted to be explicit about how conjunction behaves with \( C_4 \).

    \Fullref{thm:boolean_equivalences/distributivity} ensures that \( C_4(\varphi) \gleichstark \varphi \) for any formula \( \varphi \).

    To \enquote{pull} out disjunctions instead, we will need the operator \( C_4^\oppos \), which is defined like \( C_4 \) but with \( \synvee \) and \( \synwedge \) swapped.
  \end{thmenum}
\end{algorithm}
\begin{comments}
  \item This algorithm can be found as \identifier{fol.cnf.to_cnf} in \cite{code}.
\end{comments}
\begin{defproof}
  It is obvious from our comments that \( \varphi \) is equivalent to
  \begin{equation*}
    \varphi' \coloneqq [C_4 \bincirc C_3 \bincirc C_2 \bincirc C_1](\varphi).
  \end{equation*}

  We will use \hyperref[con:induction/peano_arithmetic]{natural number induction} on the length of \( \varphi \) to show that it is in conjunctive normal form.

  \begin{itemize}
    \item If \( \varphi = \syntop \), then \( \varphi' = C_1(\varphi) = P \synvee \synneg P \), which is a disjunction of literals.
    \item If \( \varphi = \synbot \), then \( \varphi' = C_1(\varphi) = P \synwedge \synneg P \), which is a conjunction of \enquote{unary} disjuncts of literals.
    \item If \( \varphi \) is a propositional variable, then \( \varphi' = \varphi \) is a literal.

    \item If \( \varphi = \psi_1 \synvee \psi_2 \), then
    \begin{equation*}
      \varphi'
      =
      [C_4 \bincirc C_3 \bincirc C_2 \bincirc C_1](\psi_1 \synvee \psi_2)
      =
      C_4\parens[\Big]{ [C_3 \bincirc C_2 \bincirc C_1](\psi_1) \synvee [C_3 \bincirc C_2 \bincirc C_1](\psi_2) }
    \end{equation*}

    The result of applying \( C_4 \) depends on \( \psi_1' \) and \( \psi_2' \). Note that both are shorter than \( \varphi \), so both must be in CNF.
    \begin{itemize}
      \item If \( (\synneg \theta_1)' = \chi_1 \synwedge \chi_2 \), then
      \begin{balign*}
        \varphi'
        &=
        C_4\parens[\bigg]{ \parens[\Big]{ \chi_1 \synvee [C_3 \bincirc C_2 \bincirc C_1](\psi_2) } \synwedge \parens[\Big]{ \chi_2 \synvee [C_3 \bincirc C_2 \bincirc C_1](\psi_2) } }
        = \\ &=
        C_4\parens[\Big]{ \chi_1 \synvee [C_3 \bincirc C_2 \bincirc C_1](\psi_2) } \synwedge C_4\parens[\Big]{ \chi_2 \synvee [C_3 \bincirc C_2 \bincirc C_1](\psi_2) }
        = \\ &=
        (\chi_1 \synvee \psi_2)' \synwedge (\chi_2 \synvee \psi_2)'.
      \end{balign*}

      Since both \( (\chi_1 \synvee \psi_2)' \) and \( (\chi_2 \synvee \psi_2)' \) are shorter than \( \varphi \), the inductive hypothesis implies that they are in CNF. Then \( \varphi' \) itself is also in CNF.

      \item If \( \psi_1' \) is not a conjunction and if \( \psi_2' = \chi_1 \synwedge \chi_2 \), then, by analogy with the case above, we conclude that \( \varphi' \) is in CNF.

      \item If neither \( \psi_1' \) nor \( \psi_2' \) are conjunctions, then they must be disjunctions of literals, and hence \( \varphi' \) is also a disjunction of literals.
    \end{itemize}

    \item If \( \varphi = \psi_1 \synwedge \psi_2 \), the inductive hypothesis holds for both \( \psi_1 \) and \( \psi_2 \), hence \( \varphi' = \psi_1' \synwedge \psi_2' \) is in CNF.

    \item If \( \varphi = \psi_1 \synimplies \psi_2 \), then
    \begin{balign*}
      \varphi'
      &=
      [C_4 \bincirc C_3 \bincirc C_2 \bincirc C_1](\psi_1 \synimplies \psi_2)
      = \\ &=
      [C_4 \bincirc C_3 \bincirc C_2]\parens[\Big]{ C_1(\psi_1) \synimplies C_1(\psi_2) }
      = \\ &=
      [C_4 \bincirc C_3]\parens[\Big]{ [C_2 \bincirc C_1](\synneg \psi_1) \synvee [C_2 \bincirc C_1](\psi_2) }
      = \\ &=
      [C_4 \bincirc C_3 \bincirc C_2 \bincirc C_1](\synneg \psi_1 \synvee \psi_2).
    \end{balign*}

    This reduces to the case where \( \varphi \) is a disjunction, hence we can conclude that \( \varphi' \) is in CNF.

    \item If \( \varphi = \psi_1 \syniff \psi_1 \), then
    \small
    \begin{balign*}
      \varphi'
      &=
      [C_4 \bincirc C_3 \bincirc C_2 \bincirc C_1](\psi_1 \syniff \psi_2)
      = \\ &=
      [C_4 \bincirc C_3 \bincirc C_2]\parens[\Big]{ C_1(\psi_1) \syniff C_1(\psi_2) }
      = \\ &=
      [C_4 \bincirc C_3]\parens[\bigg]{ \parens[\Big]{\synneg [C_2 \bincirc C_1](\psi_1) \synvee [C_2 \bincirc C_1](\psi_2)} \synwedge \parens[\Big]{ [C_2 \bincirc C_1](\psi_1) \synvee \synneg [C_2 \bincirc C_1](\psi_2) } }
      = \\ &=
      [C_4 \bincirc C_3 \bincirc C_2 \bincirc C_1]\parens[\Big]{ (\synneg \psi_1 \synvee \psi_2) \synwedge (\psi_1 \synvee \synneg \psi_2) },
    \end{balign*}
    \normalsize
    which reduces to the case where \( \varphi \) is a conjunction.

    \item If \( \varphi = \synneg \psi \), where \( \psi' \) is in CNF, then we must use nested induction on \( \psi' \):
    \begin{itemize}
      \item If \( \psi = \syntop \), then
      \begin{multline*}
        \varphi'
        =
        C_3(C_1(\synneg \syntop))
        =
        C_3(\synneg C_1(\syntop))
        =
        C_3(\synneg (P \synvee \synneg P))
        = \\ =
        C_3(\synneg P) \synvee C_3(\synneg \synneg P)
        =
        \synneg C_3(P) \synwedge C_3(P)
        =
        \synneg P \synwedge P,
      \end{multline*}
      which is again a conjunction of unary disjuncts of literals.

      \item Dually, if \( \psi = \synbot \), then
      \begin{equation*}
        \varphi'
        =
        C_3(C_1(\synneg \synbot))
        =
        C_3(\synneg (P \synwedge \synneg P))
        =
        \synneg P \synvee P,
      \end{equation*}
      which is a disjunction of literals.

      \item If \( \psi = \synneg \theta \), we have \( \varphi' = \theta' \). Furthermore, since \( \theta \) is shorter than \( \varphi \), \( \theta' \) is in CNF, and hence so is.

      \item If \( \psi = \theta_1 \synvee \theta_2 \), then
      \begin{balign*}
        \varphi'
        &=
        [C_4 \bincirc C_3 \bincirc C_2 \bincirc C_1]\parens[\Big]{\synneg (\theta_1 \synvee \theta_2) }
        = \\ &=
        [C_4 \bincirc C_3]\parens[\Big]{ \synneg \parens[\Big]{ [C_2 \bincirc C_1](\theta_1) \synvee [C_2 \bincirc C_1](\theta_2) } }
        = \\ &=
        C_4\parens[\Big]{ C_3\parens[\Big]{ \synneg [C_2 \bincirc C_1](\theta_1) } \synwedge C_3\parens[\Big]{ \synneg [C_2 \bincirc C_1](\theta_2) } }
        = \\ &=
        C_4\parens[\Big]{ [C_3 \bincirc C_2 \bincirc C_1](\synneg \theta_1) \synwedge [C_3 \bincirc C_2 \bincirc C_1](\synneg \theta_2) }
        = \\ &=
        (\synneg \theta_1 \synwedge \synneg \theta_2)'
      \end{balign*}

      This reduces to the case where \( \varphi' \) is a conjunction.

      \item If \( \psi = \theta_1 \synwedge \theta_2 \), then
      \begin{align*}
        \varphi'
        =
        (\synneg \theta_1 \synvee \synneg \theta_2)',
      \end{align*}
      which reduces to the case where \( \varphi \) is a disjunction.

      \item If \( \psi = \theta_1 \synimplies \theta_2 \), this reduces to the case where \( \psi \) is a disjunction and \( \varphi \) is a conjunction.
      \item If \( \psi = \theta_1 \syniff \theta_2 \), this reduces to the case where \( \psi \) is a conjunction and \( \varphi \) is a disjunction.
    \end{itemize}

    In all cases for \( \psi \) in \( \varphi = \synneg \psi \), we have shown that \( \varphi' \) is in CNF.
  \end{itemize}

  We have finished the inductive proof that, for any formula \( \varphi \), the result \( \varphi' \) of applying the transformations \( C_1 \) through \( C_4 \) is a formula in CNF.
\end{defproof}

\begin{definition}\label{def:perfect_cnf_and_dnf}\mimprovised
  We say that the \hyperref[def:cnf_and_dnf]{CNF} (resp. \hyperref[def:cnf_and_dnf]{DNF}) is \term[ru=совершенная (\cite[def. 6.3; def. 6.4]{Эдельман1975}), en=full (\cite[38]{Rosen1999}) / strict (\cite[def. 1.3.10]{Hinman2005})]{perfect} if the following hold:
  \begin{thmenum}
    \thmitem{def:perfect_cnf_and_dnf/literals} Every elementary disjunction (resp. conjunction) contains exactly one literal for each variable encountered in the formula, and these variables are ordered as per \fullref{rem:grammar_rules_for_variables}.

    More explicitly, if the variables are \( P_1, \ldots, P_n \), every elementary disjunction has the form
    \begin{equation*}
      L_1 \synvee L_2 \synvee \cdots \synvee L_n,
    \end{equation*}
    where, for every \( k = 1, \ldots, n \), \( L_k \) is either \( P_k \) or \( \neg P_k \).

    \thmitem{def:perfect_cnf_and_dnf/elementary} The elementary disjunctions (resp. conjunctions) are ordered \hyperref[def:lexicographic_order]{lexicographically} so that \( P_k < \neg P_k \) for every variable \( P_k \).
  \end{thmenum}
\end{definition}
\begin{comments}
  \item This definition is based on \cite[def. 6.3; def. 6.4]{Эдельман1975}, but with adjustments made in order to improve rigor.
\end{comments}

\begin{proposition}\label{thm:perfect_cnf_and_dnf_uniqueness}
  A \hyperref[def:perfect_cnf_and_dnf]{perfect} \hyperref[def:cnf_and_dnf]{CNF} (resp. \hyperref[def:cnf_and_dnf]{DNF}) is unique.
\end{proposition}
\begin{proof}
  Trivial.
\end{proof}

\begin{example}\label{ex:def:cnf_and_dnf}
  We list examples of formulas in \hyperref[def:cnf_and_dnf]{conjunctive and disjunctive normal forms}:
  \begin{thmenum}
    \thmitem{def:cnf_and_dnf/perfect_cnf} The \hyperref[thm:boolean_equivalences]{Boolean equivalence} \eqref{eq:thm:boolean_equivalences/conditional_as_disjunction} allows us to represent convert the \hyperref[def:propositional_alphabet/connectives/conditional]{conditional} \( \varphi \synimplies \psi \) to \( \synneg \varphi \synvee \psi \), which is both in CNF and in DNF.

    It is an elementary disjunction, hence it is vacuously in perfect CNF.

    The DNF is not perfect, however, because the condition \fullref{def:perfect_cnf_and_dnf/literals} is not satisfied.

    \thmitem{def:cnf_and_dnf/perfect_dnf} Consider instead the formula
    \begin{equation*}
      (\varphi \synwedge \psi) \synvee (\synneg \varphi \synwedge \psi) \synvee (\synneg \varphi \synwedge \synneg \psi)
    \end{equation*}

    It is in perfect DNF, and it is equivalent to \( \varphi \synimplies \psi \).
  \end{thmenum}
\end{example}

\begin{algorithm}[Boolean function to perfect CNF or DNF]\label{alg:perfect_cnf_and_dnf}\mcite[thm. I.1.3]{Яблонский2003}
  Let \( f(x_1, \ldots, x_n) \) be an arbitrary \hyperref[def:boolean_function]{Boolean function}.

  We will build a formula \( \varphi \) in \hyperref[def:perfect_cnf_and_dnf]{perfect} \hyperref[def:cnf_and_dnf]{conjunctive normal form} whose \hyperref[def:propositional_valuation/valuation_function]{induced function} under \hyperref[def:propositional_semantics]{classical semantics} will be \( f \).

  \begin{thmenum}
    \thmitem{alg:perfect_cnf_and_dnf/true} If \( f \) is canonically true, let \( \varphi \coloneqq P \synvee \neg P \).

    \thmitem{alg:perfect_cnf_and_dnf/false} If \( n = 0 \) and \( f = F \), let \( \varphi \coloneqq P \synwedge \neg P \).

    \thmitem{alg:perfect_cnf_and_dnf/elementary} Otherwise, fix some propositional variables \( P_1, \ldots, P_n \). Given a tuple \( x_1, \ldots, x_n \) of Boolean values, we can construct the following elementary disjunction:
    \begin{equation}\label{alg:perfect_cnf_and_dnf/cnf}
      P_1^{x_1} \synvee \cdots \synvee P_n^{x_n},
    \end{equation}
    where
    \begin{equation*}
      P_k^{x_k} \coloneqq \begin{cases}
        P_k,      &x_k = F, \\
        \neg P_k, &x_k = T.
      \end{cases}
    \end{equation*}

    \thmitem{alg:perfect_cnf_and_dnf/total} Finally, order the disjunctions with respect to the \hyperref[def:lexicographic_order]{lexicographic order} on the set \( \set{ T, F }^n \) (\( F < T \)) to which the tuples of Boolean values \( (x_1, \ldots, x_n) \) belong. Then let \( \varphi \) be the conjunction of the corresponding elementary disjunctions for which
    \begin{equation*}
      f(x_1, \ldots, x_n) = F.
    \end{equation*}

    \thmitem{alg:perfect_cnf_and_dnf/dual} In order to obtain a perfect DNF instead, we can utilize \fullref{thm:cnf_and_dnf_duality} by instead considering the function
    \begin{equation*}
      \oline{f(x_1, \ldots, x_n)},
    \end{equation*}
    obtaining a formula \( \varphi \) in CNF and then using the dual formula \( \varphi^\oppos \).
  \end{thmenum}
\end{algorithm}
\begin{comments}
  \item This algorithm can be found as \identifier{fol.cnf.function_to_cnf} in \cite{code}.
\end{comments}
\begin{defproof}
  We will use induction on \( n \) to show that
  \begin{equation*}
    \Bracks{\varphi}(x_1, \ldots, x_n) = f(x_1, \ldots, x_n).
  \end{equation*}

  \begin{itemize}
    \item The case \( n = 0 \) is special.
    \begin{itemize}
      \item If \( f = T \), then, by \fullref{alg:perfect_cnf_and_dnf/true}, \( \varphi = P \synvee \neg P \), which is a tautology.
      \item If \( f = F \), then, by \fullref{alg:perfect_cnf_and_dnf/false}, \( \varphi = P \synwedge \neg P \), which is a contradictory formula.
    \end{itemize}

    \item Let \( n = 1 \).
    \begin{itemize}
      \item If \( f(x) = T \), then again by \fullref{alg:perfect_cnf_and_dnf/true}, \( \varphi = P \synvee \neg P \) is a tautology.
      \item If \( f(x) = F \), then \fullref{alg:perfect_cnf_and_dnf/elementary} provides us with the elementary disjunctions \( P^F = P \) and \( P^T = \neg P \) and \fullref{alg:perfect_cnf_and_dnf/total} gives us \( \varphi = P \wedge \neg P \).
      \item If \( f(x) = x \), then we only consider the elementary disjunction \( P^F = P \), which leads us to \( \varphi = P \).
      \item If \( f(x) = \oline{x} \), then \( \varphi = \neg P \).
    \end{itemize}

    Therefore, if \( f \) is unary, the formula \( \varphi \) constructed via this algorithm satisfies \( \Bracks{\varphi}(x) = f(x) \).

    \item Consider the function \( f(x_1, \ldots, x_n, x_{n+1}) \) and suppose that, for any \( n \)-ary Boolean function \( g \), the interpretation of its corresponding formula coincides with \( g \).

    Let \( \varphi_F \) be the formula obtained for \( f(x_1, \ldots, x_n, F) \) and let \( \varphi_F' \) be the formula obtained from \( \varphi_F \) by adding the literal \( P_{n+1} \) to each elementary disjunction.

    Dually, let \( \varphi_T \) be the formula for \( f(x_1, \ldots, x_n, T) \) and let \( \varphi_T' \) be the formula obtained by adding the literal \( \neg P_{n+1} \) to each elementary disjunction.

    Then
    \begin{equation*}
      \Bracks{\varphi_F' \wedge \varphi_T'}(x_1, \ldots, x_n, F)
      =
      \underbrace{\Bracks{\varphi_F'}(x_1, \ldots, x_n, F)}_{\mathclap{\Bracks{\varphi_F}(x_1, \ldots, x_n) \T*{because} \Bracks{P_{n+1}}(F) = F}} \wedge \overbrace{\Bracks{\varphi_T'}(x_1, \ldots, x_n, F)}^{T \T*{because} \Bracks{\neg P_{n+1}}(F) = T}
      =
      \underbrace{\Bracks{\varphi_F}(x_1, \ldots, x_n)}_{f(x_1, \ldots, x_n, F)}
    \end{equation*}
    and similarly
    \begin{equation*}
      \Bracks{\varphi_F' \wedge \varphi_T'}(x_1, \ldots, x_n, T)
      =
      \Bracks{\varphi_T}(x_1, \ldots, x_n)
      =
      f(x_1, \ldots, x_n, T).
    \end{equation*}
  \end{itemize}

  We conclude that the interpretation of a formula obtained through this algorithm is the initial function.
\end{defproof}

\paragraph{Equivalence of propositional formulas and Boolean functions}

\begin{proposition}\label{thm:propositional_formulas_and_boolean_functions}
  \hyperref[def:propositional_syntax/formula]{Propositional formulas} under \hyperref[def:propositional_semantics]{classical semantics} have the following structural properties when treated as \hyperref[def:boolean_function]{Boolean functions}:

  \begin{thmenum}
    \thmitem{thm:propositional_formulas_and_boolean_functions/equivalence_classes} \hyperref[def:semantic_equivalence]{Semantic equivalence} \( \gleichstark \) is an \hyperref[def:equivalence_relation]{equivalence relation} on the set \( \op*{Form} \) of all propositional formulas.

    \thmitem{thm:propositional_formulas_and_boolean_functions/bijection} Given the set \( \mscrB \) of all \hyperref[def:boolean_function]{Boolean functions} of arbitrary arity, the following map is bijective:
    \begin{equation*}
      \begin{aligned}
        &\Phi: \op*{Form} / {\gleichstark} \to \mscrB \\
        &\Phi([\varphi]) \coloneqq \Bracks{\varphi}.
      \end{aligned}
    \end{equation*}
  \end{thmenum}
\end{proposition}
\begin{comments}
  \item This is one of the motivations for studying \hyperref[def:lindenbaum_tarski_algebra]{Lindenbaum-Tarski algebras}.
  \item Both \( \op*{Form} / {\gleichstark} \) and \( \mscrB \) are provably Boolean algebras, but we give very different proofs --- the former is a Boolean algebra due to the syntactic \fullref{thm:intuitionistic_lindenbaum_tarski_algebra}, and the latter is a Boolean algebra due to the semantic \fullref{thm:functions_over_model_form_model}. This is another demonstration of the soundness and completeness stated in \fullref{thm:classical_first_order_logic_is_sound_and_complete}. See also \fullref{rem:thm:intuitionistic_lindenbaum_tarski_algebra/syntactic_proof}.
\end{comments}
\begin{proof}
  \SubProofOf{thm:propositional_formulas_and_boolean_functions/equivalence_classes} Straightforward.

  \SubProofOf{thm:propositional_formulas_and_boolean_functions/bijection} The map \( \Phi \) is well-defined since, by definition of semantic equivalence, \( \varphi \gleichstark \psi \) whenever \( \Bracks{\varphi}_I = \Bracks{\psi}_I \) for every interpretation \( I \).

  Injectivity of \( \Phi \) is also obvious from the definition of semantic equivalence, while surjectivity is given by \fullref{alg:perfect_cnf_and_dnf}.
\end{proof}
