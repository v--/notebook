\subsection{Propositional normal forms}\label{subsec:propositional_normal_forms}

We sometimes want to substitute a propositional variable with another variable or even with a formula. This is akin to applying a \hyperref[def:boolean_function]{Boolean function} like \( x \synvee y \) to different variables (e.g. to obtain \( x \synvee x \)) or even concrete values (e.g. \( F \synvee T \)), except that it is done on a purely syntactic level without involving any semantics involved.

\paragraph{Simultaneous substitution}\hfill

\begin{definition}\label{def:propositional_substitution}\mimprovised
  Consider a map \( \sigma: \op*{Prop} \to \op*{Form} \), which allows us to replace propositional variables with arbitrary propositional formulas. In the case where only finitely many variables are not \hyperref[def:fixed_point]{fixed points}, we call such a map a \term[ru=одновременная замена (\cite[23]{Герасимов2011Вычислимость})]{simultaneous substitution} or simply \term{substitution}.

  Given a substitution \( \sigma \), we define via \hyperref[rem:straightforward_traversal]{straightforward traversal} the following operation on arbitrary formulas:
  \begin{equation*}
    \varphi[\sigma] \coloneqq \begin{cases}
      \varphi,                              &\varphi \in \set{ \syntop, \synbot }, \\
      \sigma(\varphi),                      &\varphi \in \op*{Prop}, \\
      \neg \psi[\sigma],                    &\varphi = \neg \psi, \\
      \psi[\sigma] \syncirc \theta[\sigma], &\varphi = \psi \syncirc \theta, {\syncirc} \in \op*{Conn}. \\
    \end{cases}
  \end{equation*}
\end{definition}
\begin{comments}
  \item Our definition is based on \incite[def. 3.1.3]{CitkinMuravitsky2022ConsequenceRelations}, with three notable differences:
  \begin{itemize}
    \item The authors define substitutions in the general setting of logical matrices, while we prefer to avoid them and instead specialize the definition to propositional logic.

    \item The authors define substitutions to be maps from formulas to formulas satisfying additional recursive conditions, while we define substitutions to be maps from variables to formulas.

    This allows us to stay consistent with \fullref{def:propositional_valuation}, where we make a distinction between interpretations and valuations

    \item The requirement of only finitely many variables is inspired by \incite[222]{Mimram2020Types}.
  \end{itemize}

  \item We focus on simultaneous substitutions directly instead of defining substituting a single variable. The latter is often done similarly to \fullref{def:literal_propositional_substitution}. See \fullref{rem:simulating_simultaneous_substitution} for an alternative approach.
\end{comments}

\begin{remark}\label{rem:substitution_notation}
  Suppose that the substitution \( \sigma \) replaces the variables \( p_1, \cdots, p_n \) with the formulas \( \omega_1, \ldots, \omega_n \), and fixes all other variables. This allows us to use the notation
  \begin{equation*}
    \varphi[p_1 \mapsto \omega_1, \ldots, p_n \mapsto \omega_n]
  \end{equation*}
  without referring to \( \sigma \) itself.

  A popular alternative that we find less straightforward is
  \begin{equation*}
    \varphi[\omega_1 / p_1, \cdots, \omega_n / p_n].
  \end{equation*}

  It is used, among others, by \incite[1A7]{Hindley1997STT}, \incite[55]{Mimram2020Types}, \incite[23]{Герасимов2011Вычислимость} and \incite[161]{ШеньВерещагин2017Логика}.

  Other authors use more esoteric notations --- \incite[47]{КолмогоровДрагалин2006Логика} use
  \begin{equation*}
    \varphi(p_1, \ldots, p_n \Vert \omega_1, \cdots, \omega_n),
  \end{equation*}
  while \incite[131]{Эдельман1975Логика} uses
  \begin{equation*}
    S_{b_{p_1, \ldots, p_n}}(\varphi; \omega_1, \ldots, \omega_n).
  \end{equation*}
\end{remark}

\begin{remark}\label{rem:simulating_simultaneous_substitution}
  Substitution as defined in \fullref{def:propositional_substitution} allows us to replace more than one variable at a time:
  \begin{equation*}
    (pqr)[p \mapsto r, r \mapsto p]
    =
    rqp.
  \end{equation*}

  We can also simulate this via single-variable substitution with the help of temporary variables:
  \begin{equation*}
    (pqr)[p \mapsto t][r \mapsto p][t \mapsto r]
    =
    (tqr)[r \mapsto p][t \mapsto r]
    =
    (tqp)[t \mapsto r]
    =
    (rqp).
  \end{equation*}
\end{remark}

\paragraph{Subformula substitution}\hfill

For propositional logic, We will also find useful the ability to replace an entire subformula with another formula. We will use this, for example, to justify the substitutions in \fullref{alg:cnf_and_dnf}.

\begin{definition}\label{def:literal_propositional_substitution}\mcite[def. 1.3.13]{Hinman2005Logic}
  We define the \term[en=substitution]{subformula substitution} of the propositional formula \( \chi \) with \( \omega \) in \( \varphi \) as follows:
  \begin{equation}\label{eq:def:propositional_substitution/single}
    \varphi[\chi \mapsto \omega] \coloneqq \begin{cases}
      \omega,                                                         &\varphi = \chi \\
      \varphi,                                                        &\varphi \neq \chi \T{and} \varphi \in \set{ \syntop, \synbot } \cup \op*{Prop} \\
      \synneg \psi[\chi \mapsto \omega],                              &\varphi \neq \chi \T{and} \varphi = \synneg \psi \\
      \psi[\chi \mapsto \omega] \syncirc \theta[\chi \mapsto \omega], &\varphi \neq \chi \T{and} \varphi = \psi \syncirc \theta, \circ \in \op*{Conn}.
    \end{cases}
  \end{equation}
\end{definition}
\begin{comments}
  \item Note that it is not necessary for \( \chi \) to be a subformula of \( \varphi \) --- the substitution simply does nothing otherwise.
  \item It is safe to reuse the notation for simultaneous substitutions because in the case where \( \chi \) is a variable, the two notions coincide.
\end{comments}

\begin{proposition}\label{thm:propositional_substitution_equivalence}
  If \( \chi \gleichstark \omega \), then
  \begin{equation*}
    \varphi[\chi \mapsto \omega] \gleichstark \varphi.
  \end{equation*}
\end{proposition}
\begin{proof}
  Suppose that \( \chi \gleichstark \omega \). We will use \fullref{thm:induction_on_rooted_trees} on \( \varphi \) to show that
  \begin{equation}\label{eq:thm:propositional_substitution_equivalence/proof}
    \varphi[\chi \mapsto \omega] \gleichstark \varphi.
  \end{equation}

  \begin{itemize}
    \item If \( \varphi = \chi \), then \( \varphi[\chi \mapsto \omega] = \omega \) and, by definition,
    \begin{equation*}
      \varphi = \chi \gleichstark \omega = \varphi[\chi \mapsto \omega].
    \end{equation*}

    \item If \( \varphi \neq \chi \) and \( \varphi \in \set{ \syntop, \synbot } \cup \op*{Prop} \), then \( \varphi[\chi \mapsto \omega] = \varphi \) and \eqref{eq:thm:propositional_substitution_equivalence/proof} again holds trivially.

    \item If \( \varphi \neq \chi \) and \( \varphi = \synneg \omega \) and if the inductive hypothesis holds for \( \omega \), then
    \begin{equation*}
      \varphi[\chi \mapsto \omega] = \synneg \psi[\chi \mapsto \omega].
    \end{equation*}

    For any interpretation \( I \),
    \begin{equation*}
      \Bracks[\Big]{ \varphi[\chi \mapsto \omega] }_I
      =
      \oline{\Bracks[\Big]{\psi[\chi \mapsto \omega] }_I}
      \reloset {\T{ind.}} =
      \oline{\Bracks{\psi}_I}
      =
      \Bracks{\varphi}_I.
    \end{equation*}

    Generalizing on \( I \), we conclude that \eqref{eq:thm:propositional_substitution_equivalence/proof} holds in this case.

    \item If \( \varphi \neq \chi \) and \( \varphi = \psi \syncirc \theta, {\syncirc} \in \op*{Conn} \) and if the inductive hypothesis holds for both \( \psi \) and \( \theta \), then for any interpretation \( I \),
    \begin{equation*}
      \Bracks[\Big]{ \varphi[\chi \mapsto \omega] }_I
      =
      \Bracks[\Big]{ \psi[\chi \mapsto \omega] }_I \relcirc \Bracks[\Big]{ \theta[\chi \mapsto \omega] }_I
      \reloset {\T{ind.}} =
      \Bracks[\Big]{\psi}_I \relcirc \Bracks[\Big]{\theta}_I
      =
      \Bracks{\varphi}_I.
    \end{equation*}

    Again, generalizing on \( I \), we conclude that \eqref{eq:thm:propositional_substitution_equivalence/proof} holds.
  \end{itemize}

  We have verified that \eqref{eq:thm:propositional_substitution_equivalence/proof} holds in all cases.
\end{proof}

\paragraph{Conjunctive and disjunctive normal forms}

\begin{definition}\label{def:cnf_and_dnf}\mimprovised
  We will now introduce \term[ru=конъюнктивная нормальная форма (\cite[def. 6.2]{Эдельман1975Логика}), en=conjunctive normal form (\cite[def. 1.3.10]{Hinman2005Logic})]{conjunctive normal forms} (CNF) and \term[ru=дизъюнктивная нормальная форма (\cite[def. 6.2]{Эдельман1975Логика}), en=disjunctive normal form (\cite[def. 1.3.10]{Hinman2005Logic})]{disjunctive normal forms} (DNF) for propositional formulas.

  The structure of these formulas is best described by the following extension of the \hyperref[def:propositional_syntax]{propositional syntax grammar schema}:
  \begin{bnf*}
    \bnfprod{positive literal}       {\bnfpn{variable}} \\
    \bnfprod{negative literal}       {\synneg \bnfpn{variable}} \\
    \bnfprod{literal}                {\bnfpn{positive literal}       \bnfor \bnfpn{negative literal}} \\
    \bnfprod{elementary disjunction} {\bnfpn{literal} \bnfor} \\
    \bnfmore                         {\bnftsq{(} \bnfsp \bnfpn{elem. disjunction} \bnfsp \bnftsq{\( \synvee \)} \bnfsp \bnfpn{elem. disjunction} \bnfsp \bnftsq{)}} \\
    \bnfprod{CNF}                    {\bnfpn{elementary disjunction} \bnfor \bnftsq{(} \bnfsp \bnfpn{CNF}              \bnfsp \bnftsq{\( \synwedge \)} \bnfsp \bnfpn{CNF}             \bnfsp \bnftsq{)}} \\
    \bnfprod{elementary conjunction} {\bnfpn{literal} \bnfor} \\
    \bnfmore                         {\bnftsq{(} \bnfsp \bnfpn{elem. conjunction} \bnfsp \bnftsq{\( \synvee \)} \bnfsp \bnfpn{elem. conjunction} \bnfsp \bnftsq{)}} \\
    \bnfprod{DNF}                    {\bnfpn{elementary conjunction} \bnfor \bnftsq{(} \bnfsp \bnfpn{DNF}              \bnfsp \bnftsq{\( \synvee \)}   \bnfsp \bnfpn{DNF}             \bnfsp \bnftsq{)}}
  \end{bnf*}
\end{definition}
\begin{comments}
  \item The terms \enquote{literal}, \enquote{positive literal} and \enquote{negative literal} are used by \incite[def. 12.2.1]{Rosen2019DiscreteMathematics}, \incite[103]{DaveyPriestley2002Lattices}, \incite[77]{Mimram2020Types} and \incite[132]{Savage1998Computability}. The adjective \enquote{elementary} for conjunctions and disjunctions is based on \cite[36]{Эдельман1975Логика}. \incite[13]{Smullyan1995FOL} instead uses \enquote{basic conjunction}.
  \item The concepts are related but distinct from that of \hyperref[rem:lattice_polynomials]{lattice polynomials}.
  \item As usual, we utilize the convention in \fullref{rem:propositional_formula_parentheses} and avoid excessive parentheses.
\end{comments}

\begin{theorem}[Principle of duality for CNF and DNF]\label{thm:cnf_and_dnf_duality}
  Suppose we are working with \hyperref[def:propositional_semantics/classical]{classical semantics}.

  Fix a propositional formula \( \varphi \) in \hyperref[def:cnf_and_dnf]{CNF} (resp. \hyperref[def:cnf_and_dnf]{DNF}). Consider its opposite \( \varphi^\oppos \), in which we swap all conjunctions and disjunctions, then make all positive literals negative and vice versa. Obviously \( \varphi^\oppos \) is in DNF (resp. CNF).

  Furthermore, \( \neg \varphi \) is \hyperref[def:semantic_equivalence]{semantically equivalent} to \( \varphi^\oppos \).
\end{theorem}
\begin{proof}
  Consider an elementary disjunction
  \begin{equation*}
    \psi = L_1 \synvee L_2 \synvee \cdots \synvee L_n,
  \end{equation*}
  where, for every \( k = 1, \ldots, n \), \( L_k \) is either \( p_k \) or \( \neg p_k \).

  We have
  \begin{equation*}
    \psi^\oppos = L_1^\oppos \synwedge L_2^\oppos \synwedge \cdots \synwedge L_n^\oppos.
  \end{equation*}

  Then, for any interpretation \( I \),
  \begin{equation*}
    \Bracks{\psi^\oppos}_I
    =
    \Bracks{L_1^\oppos}_I \wedge \cdots \wedge \Bracks{L_n^\oppos}_I
    =
    \oline{\Bracks{L_1}_I} \wedge \cdots \wedge \oline{\Bracks{L_n}_I}
    \reloset {\ref{thm:classical_equivalences/de_morgan}} =
    \oline{ \Bracks{L_1}_I \vee \cdots \vee \Bracks{L_n}_I }
    =
    \oline{ \Bracks{\psi}_I }
    =
    \Bracks{\neg \psi}_I.
  \end{equation*}

  Now consider a formula in CNF
  \begin{equation*}
    \varphi = \psi_1 \synwedge \psi_2 \synwedge \cdots \synwedge \psi_m,
  \end{equation*}
  where, for every \( k = 1, \ldots, m \), \( \psi_k \) is an elementary disjunction.

  Then
  \begin{equation*}
    \Bracks{\varphi^\oppos}_I
    =
    \Bracks{\psi_1^\oppos}_I \vee \cdots \vee \Bracks{\psi_m^\oppos}_I
    =
    \oline{\Bracks{\psi_1}_I} \vee \cdots \vee \oline{\Bracks{\psi_m}_I}
    \reloset {\ref{thm:classical_equivalences/de_morgan}} =
    \oline{ \Bracks{\psi_1}_I \wedge \cdots \wedge \Bracks{\psi_m}_I }
    =
    \oline{ \Bracks{\varphi}_I }
    =
    \Bracks{\neg \varphi}_I.
  \end{equation*}
\end{proof}

\begin{remark}\label{rem:straightforward_traversal}
  Consider the identity operator for formulas:
  \begin{equation*}
    \begin{aligned}
      \id(\varphi) \coloneqq \begin{cases}
        \varphi,                        &\varphi \in \set{ \syntop, \synbot } \\
        \varphi,                        &\varphi \in \op*{Prop} \\
        \synneg \id(\psi),              &\varphi = \synneg \psi, \\
        \id(\psi) \syncirc \id(\theta), &\varphi = \psi \syncirc \theta, {\syncirc} \in \op*{Conn}, \\
      \end{cases}
    \end{aligned}
  \end{equation*}

  When defining an operator acting on formulas, we usually handle only a few cases, and the others become straightforward. For example, replacing \( \synneg \varphi \) with \( \varphi \synimplies \synbot \), as justified by \fullref{thm:intuitionistic_equivalences/negation_bottom}, can be achieved via
  \begin{equation*}
    \begin{aligned}
      R(\varphi) \coloneqq \begin{cases}
        \varphi,                     &\varphi \in \set{ \syntop, \synbot } \\
        \varphi,                     &\varphi \in \op*{Prop} \\
        R(\psi) \synimplies \synbot, &\varphi = \synneg \psi, \\
        R(\psi) \syncirc R(\theta),  &\varphi = \psi \syncirc \theta, {\syncirc} \in \op*{Conn}, \\
      \end{cases}
    \end{aligned}
  \end{equation*}

  We can instead use the notational shorthand
  \begin{equation*}
    \begin{aligned}
      R(\varphi) \coloneqq \begin{cases}
        R(\psi) \synimplies \synbot,                                             &\varphi = \synneg \psi, \\
        \hyperref[rem:straightforward_traversal]{\T{straightforward traversal}}, &\T{otherwise.}
      \end{cases}
    \end{aligned}
  \end{equation*}

  In computer programming, this exact problem is solved via the visitor design pattern --- visitors for formulas of first-order logic can be found in \cite{notebook:code}.
\end{remark}

\begin{algorithm}[Formula to CNF or DNF]\label{alg:cnf_and_dnf}
  Again, suppose we are working with \hyperref[def:propositional_semantics/classical]{classical semantics}.

  We will define four operators, \( C_1 \) through \( C_4 \), such that, for a \hyperref[def:propositional_syntax/formula]{propositional formula} \( \varphi \)
  \begin{equation*}
    [C_4 \bincirc C_3 \bincirc C_2 \bincirc C_1](\varphi)
  \end{equation*}
  is a formula in \hyperref[def:cnf_and_dnf]{conjunctive normal form} that is \hyperref[def:semantic_equivalence]{semantically equivalent} to \( \varphi \).

  In order to obtain a \hyperref[def:cnf_and_dnf]{disjunctive normal form} instead, we will need to use \( C_4^\oppos \) rather than \( C_4 \).

  \begin{thmenum}
    \thmitem{alg:cnf_and_dnf/constants} First, we remove the propositional constants. Choose any propositional variable, say \( \synp \), and let
    \begin{equation*}
      C_1(\varphi) \coloneqq \begin{cases}
        \synp \synvee \synneg \synp,                                           &\varphi = \syntop, \\
        \synp \synwedge \synneg \synp,                                         &\varphi = \synbot, \\
        \hyperref[rem:straightforward_traversal]{\T{straightforward traversal}}, &\T{otherwise.}
      \end{cases}
    \end{equation*}

    \Fullref{thm:propositional_substitution_equivalence} implies that, since \( \synp \synvee \synneg \synp \gleichstark \syntop \) and \( \synp \synwedge \synneg \synp \gleichstark \synbot \), then \( C_1(\varphi) \gleichstark \varphi \).

    \thmitem{alg:cnf_and_dnf/conditional} Next, we remove the conditional \( \synimplies \) and biconditional \( \syniff \). Let
    \begin{equation*}
      C_2(\varphi) \coloneqq \begin{cases}
        \synneg C_2(\psi) \synvee C_2(\theta),                                   &\varphi = \psi \synimplies \theta, \\
        C_2(\psi \synimplies \theta) \synwedge C_2(\theta \synimplies \psi),     &\varphi = \psi \syniff \theta, \\
        \hyperref[rem:straightforward_traversal]{\T{straightforward traversal}}, &\T{otherwise.}
      \end{cases}
    \end{equation*}

    \Fullref{thm:classical_equivalences/conditional_as_disjunction} and \fullref{def:heyting_algebra/biconditional} justify the substitutions and ensure that \( C_2(\varphi) \gleichstark \varphi \).

    \thmitem{alg:cnf_and_dnf/negation} We \enquote{push} negations inwards. Let
    \begin{equation*}
      C_3(\varphi) \coloneqq \begin{cases}
        C_3(\synneg \psi) \synwedge C_3(\synneg \theta),                         &\varphi = \synneg (\psi \synvee \theta), \\
        C_3(\synneg \psi) \synvee C_3(\synneg \theta),                           &\varphi = \synneg (\psi \synwedge \theta), \\
        C_3(\psi),                                                               &\varphi = \synneg \synneg \psi, \\
        \synneg C_3(\psi),                                                       &\T{otherwise if} \varphi = \synneg \psi, \\
        \hyperref[rem:straightforward_traversal]{\T{straightforward traversal}}, &\T{otherwise.}
      \end{cases}
    \end{equation*}

    \Fullref{thm:classical_equivalences/de_morgan} and \fullref{thm:classical_equivalences/double_negation} ensure that \( C_3(\varphi) \gleichstark \varphi \) for any formula \( \varphi \).

    If case \( \varphi \) has no (bi)conditionals (e.g. due to \( C_2 \)), the result \( C_3(\varphi) \) has negations only before propositional variables.

    \thmitem{alg:cnf_and_dnf/distributive} Finally, we \enquote{pull} conjunctions outwards. Let
    \begin{equation*}
      C_4(\varphi) \coloneqq \begin{cases}
        C_4(\psi_1 \synvee \theta) \synwedge C_4(\psi_1 \synvee \theta),         &\varphi = \psi \synvee \theta \T{and} C_4(\psi) = \psi_1 \synwedge \psi_2, \\
        C_4(\psi \synvee \theta_1) \synwedge C_4(\psi \synvee \theta_2),         &\T{otherwise if} \varphi = \psi \synvee \theta \T{and} C_4(\theta) = \theta_1 \synwedge \theta_2, \\
        C_4(\psi) \synvee C_4(\theta),                                           &\T{otherwise if} \varphi = \psi \synvee \theta, \\
        C_4(\psi) \synwedge C_4(\theta),                                         &\varphi = \psi \synwedge \theta, \\
        \hyperref[rem:straightforward_traversal]{\T{straightforward traversal}}, &\T{otherwise.}
      \end{cases}
    \end{equation*}

    The last case is due to the usual traversal rules, and we could avoid writing it, but we wanted to be explicit about how conjunction behaves with \( C_4 \).

    \Fullref{thm:classical_equivalences/distributivity} ensures that \( C_4(\varphi) \gleichstark \varphi \) for any formula \( \varphi \).

    To \enquote{pull} out disjunctions instead, we will need the operator \( C_4^\oppos \), which is defined like \( C_4 \) but with \( \synvee \) and \( \synwedge \) swapped.
  \end{thmenum}
\end{algorithm}
\begin{comments}
  \item This algorithm can be found as \identifier{fol.cnf.to_cnf} in \cite{notebook:code}.
\end{comments}
\begin{defproof}
  It is obvious from our comments that \( \varphi \) is equivalent to
  \begin{equation*}
    \varphi' \coloneqq [C_4 \bincirc C_3 \bincirc C_2 \bincirc C_1](\varphi).
  \end{equation*}

  We will use \hyperref[con:induction/peano_arithmetic]{natural number induction} on the length of \( \varphi \) to show that it is in conjunctive normal form.

  \begin{itemize}
    \item If \( \varphi = \syntop \), then \( \varphi' = C_1(\varphi) = \synp \synvee \synneg \synp \), which is a disjunction of literals.
    \item If \( \varphi = \synbot \), then \( \varphi' = C_1(\varphi) = \synp \synwedge \synneg \synp \), which is a conjunction of \enquote{unary} disjuncts of literals.
    \item If \( \varphi \) is a propositional variable, then \( \varphi' = \varphi \) is a literal.

    \item If \( \varphi = \psi_1 \synvee \psi_2 \), then
    \begin{equation*}
      \varphi'
      =
      [C_4 \bincirc C_3 \bincirc C_2 \bincirc C_1](\psi_1 \synvee \psi_2)
      =
      C_4\parens[\Big]{ [C_3 \bincirc C_2 \bincirc C_1](\psi_1) \synvee [C_3 \bincirc C_2 \bincirc C_1](\psi_2) }
    \end{equation*}

    The result of applying \( C_4 \) depends on \( \psi_1' \) and \( \psi_2' \). Note that both are shorter than \( \varphi \), so both must be in CNF.
    \begin{itemize}
      \item If \( (\synneg \theta_1)' = \chi_1 \synwedge \chi_2 \), then
      \begin{balign*}
        \varphi'
        &=
        C_4\parens[\bigg]{ \parens[\Big]{ \chi_1 \synvee [C_3 \bincirc C_2 \bincirc C_1](\psi_2) } \synwedge \parens[\Big]{ \chi_2 \synvee [C_3 \bincirc C_2 \bincirc C_1](\psi_2) } }
        = \\ &=
        C_4\parens[\Big]{ \chi_1 \synvee [C_3 \bincirc C_2 \bincirc C_1](\psi_2) } \synwedge C_4\parens[\Big]{ \chi_2 \synvee [C_3 \bincirc C_2 \bincirc C_1](\psi_2) }
        = \\ &=
        (\chi_1 \synvee \psi_2)' \synwedge (\chi_2 \synvee \psi_2)'.
      \end{balign*}

      Since both \( (\chi_1 \synvee \psi_2)' \) and \( (\chi_2 \synvee \psi_2)' \) are shorter than \( \varphi \), the inductive hypothesis implies that they are in CNF. Then \( \varphi' \) itself is also in CNF.

      \item If \( \psi_1' \) is not a conjunction and if \( \psi_2' = \chi_1 \synwedge \chi_2 \), then, by analogy with the case above, we conclude that \( \varphi' \) is in CNF.

      \item If neither \( \psi_1' \) nor \( \psi_2' \) are conjunctions, then they must be disjunctions of literals, and hence \( \varphi' \) is also a disjunction of literals.
    \end{itemize}

    \item If \( \varphi = \psi_1 \synwedge \psi_2 \), the inductive hypothesis holds for both \( \psi_1 \) and \( \psi_2 \), hence \( \varphi' = \psi_1' \synwedge \psi_2' \) is in CNF.

    \item If \( \varphi = \psi_1 \synimplies \psi_2 \), then
    \begin{balign*}
      \varphi'
      &=
      [C_4 \bincirc C_3 \bincirc C_2 \bincirc C_1](\psi_1 \synimplies \psi_2)
      = \\ &=
      [C_4 \bincirc C_3 \bincirc C_2]\parens[\Big]{ C_1(\psi_1) \synimplies C_1(\psi_2) }
      = \\ &=
      [C_4 \bincirc C_3]\parens[\Big]{ [C_2 \bincirc C_1](\synneg \psi_1) \synvee [C_2 \bincirc C_1](\psi_2) }
      = \\ &=
      [C_4 \bincirc C_3 \bincirc C_2 \bincirc C_1](\synneg \psi_1 \synvee \psi_2).
    \end{balign*}

    This reduces to the case where \( \varphi \) is a disjunction, hence we can conclude that \( \varphi' \) is in CNF.

    \item If \( \varphi = \psi_1 \syniff \psi_1 \), then
    \small
    \begin{balign*}
      \varphi'
      &=
      [C_4 \bincirc C_3 \bincirc C_2 \bincirc C_1](\psi_1 \syniff \psi_2)
      = \\ &=
      [C_4 \bincirc C_3 \bincirc C_2]\parens[\Big]{ C_1(\psi_1) \syniff C_1(\psi_2) }
      = \\ &=
      [C_4 \bincirc C_3]\parens[\bigg]{ \parens[\Big]{\synneg [C_2 \bincirc C_1](\psi_1) \synvee [C_2 \bincirc C_1](\psi_2)} \synwedge \parens[\Big]{ [C_2 \bincirc C_1](\psi_1) \synvee \synneg [C_2 \bincirc C_1](\psi_2) } }
      = \\ &=
      [C_4 \bincirc C_3 \bincirc C_2 \bincirc C_1]\parens[\Big]{ (\synneg \psi_1 \synvee \psi_2) \synwedge (\psi_1 \synvee \synneg \psi_2) },
    \end{balign*}
    \normalsize
    which reduces to the case where \( \varphi \) is a conjunction.

    \item If \( \varphi = \synneg \psi \), where \( \psi' \) is in CNF, then we must use nested induction on \( \psi' \):
    \begin{itemize}
      \item If \( \psi = \syntop \), then
      \begin{multline*}
        \varphi'
        =
        C_3(C_1(\synneg \syntop))
        =
        C_3(\synneg C_1(\syntop))
        =
        C_3(\synneg (\synp \synvee \synneg \synp))
        = \\ =
        C_3(\synneg \synp) \synvee C_3(\synneg \synneg \synp)
        =
        \synneg C_3(\synp) \synwedge C_3(\synp)
        =
        \synneg \synp \synwedge \synp,
      \end{multline*}
      which is again a conjunction of unary disjuncts of literals.

      \item Dually, if \( \psi = \synbot \), then
      \begin{equation*}
        \varphi'
        =
        C_3(C_1(\synneg \synbot))
        =
        C_3(\synneg (\synp \synwedge \synneg \synp))
        =
        \synneg \synp \synvee \synp,
      \end{equation*}
      which is a disjunction of literals.

      \item If \( \psi = \synneg \theta \), we have \( \varphi' = \theta' \). Furthermore, since \( \theta \) is shorter than \( \varphi \), \( \theta' \) is in CNF, and hence so is.

      \item If \( \psi = \theta_1 \synvee \theta_2 \), then
      \begin{balign*}
        \varphi'
        &=
        [C_4 \bincirc C_3 \bincirc C_2 \bincirc C_1]\parens[\Big]{\synneg (\theta_1 \synvee \theta_2) }
        = \\ &=
        [C_4 \bincirc C_3]\parens[\Big]{ \synneg \parens[\Big]{ [C_2 \bincirc C_1](\theta_1) \synvee [C_2 \bincirc C_1](\theta_2) } }
        = \\ &=
        C_4\parens[\Big]{ C_3\parens[\Big]{ \synneg [C_2 \bincirc C_1](\theta_1) } \synwedge C_3\parens[\Big]{ \synneg [C_2 \bincirc C_1](\theta_2) } }
        = \\ &=
        C_4\parens[\Big]{ [C_3 \bincirc C_2 \bincirc C_1](\synneg \theta_1) \synwedge [C_3 \bincirc C_2 \bincirc C_1](\synneg \theta_2) }
        = \\ &=
        (\synneg \theta_1 \synwedge \synneg \theta_2)'
      \end{balign*}

      This reduces to the case where \( \varphi' \) is a conjunction.

      \item If \( \psi = \theta_1 \synwedge \theta_2 \), then
      \begin{align*}
        \varphi'
        =
        (\synneg \theta_1 \synvee \synneg \theta_2)',
      \end{align*}
      which reduces to the case where \( \varphi \) is a disjunction.

      \item If \( \psi = \theta_1 \synimplies \theta_2 \), this reduces to the case where \( \psi \) is a disjunction and \( \varphi \) is a conjunction.
      \item If \( \psi = \theta_1 \syniff \theta_2 \), this reduces to the case where \( \psi \) is a conjunction and \( \varphi \) is a disjunction.
    \end{itemize}

    In all cases for \( \psi \) in \( \varphi = \synneg \psi \), we have shown that \( \varphi' \) is in CNF.
  \end{itemize}

  We have finished the inductive proof that, for any formula \( \varphi \), the result \( \varphi' \) of applying the transformations \( C_1 \) through \( C_4 \) is a formula in CNF.
\end{defproof}

\begin{definition}\label{def:full_cnf_and_dnf}\mimprovised
  We say that the \hyperref[def:cnf_and_dnf]{CNF} (resp. \hyperref[def:cnf_and_dnf]{DNF}) is \term[ru=совершенная (\cite[def. 6.3; def. 6.4]{Эдельман1975Логика}), en=full (\incite[103]{DaveyPriestley2002Lattices}); strict (\cite[def. 1.3.10]{Hinman2005Logic})]{full} if the following hold:
  \begin{thmenum}
    \thmitem{def:full_cnf_and_dnf/literals} Every elementary disjunction (resp. conjunction) contains exactly one literal for each variable encountered in the formula, and these variables are ordered as per \fullref{def:variable_identifier}.

    More explicitly, if the variables are \( p_1, \ldots, p_n \), every elementary disjunction has the form
    \begin{equation*}
      L_1 \synvee L_2 \synvee \cdots \synvee L_n,
    \end{equation*}
    where, for every \( k = 1, \ldots, n \), \( L_k \) is either \( p_k \) or \( \neg p_k \).

    \thmitem{def:full_cnf_and_dnf/elementary} The elementary disjunctions (resp. conjunctions) are ordered \hyperref[def:lexicographic_order]{lexicographically} so that \( p_k < \neg p_k \) for every variable \( p_k \).
  \end{thmenum}
\end{definition}
\begin{comments}
  \item This definition is based on \cite[def. 6.3; def. 6.4]{Эдельман1975Логика}, but with adjustments made in order to improve rigor.
\end{comments}

\begin{proposition}\label{thm:full_cnf_and_dnf_uniqueness}
  A \hyperref[def:full_cnf_and_dnf]{perfect} \hyperref[def:cnf_and_dnf]{CNF} (resp. \hyperref[def:cnf_and_dnf]{DNF}) is unique.
\end{proposition}
\begin{proof}
  Trivial.
\end{proof}

\begin{example}\label{ex:def:cnf_and_dnf}
  We list examples of formulas in \hyperref[def:cnf_and_dnf]{conjunctive and disjunctive normal forms}:
  \begin{thmenum}
    \thmitem{def:cnf_and_dnf/perfect_cnf} The \hyperref[thm:classical_equivalences]{Boolean equivalence} \eqref{eq:thm:classical_equivalences/conditional_as_disjunction} allows us to represent convert the \hyperref[def:propositional_alphabet/connectives/conditional]{conditional} \( \varphi \synimplies \psi \) to \( \synneg \varphi \synvee \psi \), which is both in CNF and in DNF.

    It is an elementary disjunction, hence it is vacuously in perfect CNF.

    The DNF is not perfect, however, because the condition \fullref{def:full_cnf_and_dnf/literals} is not satisfied.

    \thmitem{def:cnf_and_dnf/perfect_dnf} Consider instead the formula
    \begin{equation*}
      (\varphi \synwedge \psi) \synvee (\synneg \varphi \synwedge \psi) \synvee (\synneg \varphi \synwedge \synneg \psi)
    \end{equation*}

    It is in perfect DNF, and it is equivalent to \( \varphi \synimplies \psi \).
  \end{thmenum}
\end{example}

\begin{algorithm}[Boolean function to perfect CNF or DNF]\label{alg:full_cnf_and_dnf}\mcite[thm. I.1.3]{Яблонский2003ДискретнаяМатематика}
  Let \( f(x_1, \ldots, x_n) \) be an arbitrary \hyperref[def:boolean_function]{Boolean function}.

  We will build a formula \( \varphi \) in \hyperref[def:full_cnf_and_dnf]{perfect} \hyperref[def:cnf_and_dnf]{conjunctive normal form} whose \hyperref[def:propositional_valuation/valuation_function]{induced function} under \hyperref[def:propositional_semantics/classical]{classical semantics} will be \( f \).

  \begin{thmenum}
    \thmitem{alg:full_cnf_and_dnf/true} If \( f \) is canonically true, let \( \varphi \coloneqq \synp \synvee \neg \synp \).

    \thmitem{alg:full_cnf_and_dnf/false} If \( n = 0 \) and \( f = F \), let \( \varphi \coloneqq \synp \synwedge \neg \synp \).

    \thmitem{alg:full_cnf_and_dnf/elementary} Otherwise, fix some propositional variables \( p_1, \ldots, p_n \). Given a tuple \( x_1, \ldots, x_n \) of Boolean values, we can construct the following elementary disjunction:
    \begin{equation}\label{alg:full_cnf_and_dnf/cnf}
      p_1^{x_1} \synvee \cdots \synvee p_n^{x_n},
    \end{equation}
    where
    \begin{equation*}
      p_k^{x_k} \coloneqq \begin{cases}
        p_k,      &x_k = F, \\
        \neg p_k, &x_k = T.
      \end{cases}
    \end{equation*}

    \thmitem{alg:full_cnf_and_dnf/total} Finally, order the disjunctions with respect to the \hyperref[def:lexicographic_order]{lexicographic order} on the set \( \set{ T, F }^n \) (\( F < T \)) to which the tuples of Boolean values \( (x_1, \ldots, x_n) \) belong. Then let \( \varphi \) be the conjunction of the corresponding elementary disjunctions for which
    \begin{equation*}
      f(x_1, \ldots, x_n) = F.
    \end{equation*}

    \thmitem{alg:full_cnf_and_dnf/dual} In order to obtain a perfect DNF instead, we can utilize \fullref{thm:cnf_and_dnf_duality} by instead considering the function
    \begin{equation*}
      \oline{f(x_1, \ldots, x_n)},
    \end{equation*}
    obtaining a formula \( \varphi \) in CNF and then using the dual formula \( \varphi^\oppos \).
  \end{thmenum}
\end{algorithm}
\begin{comments}
  \item This algorithm can be found as \identifier{fol.cnf.function_to_cnf} in \cite{notebook:code}.
\end{comments}
\begin{defproof}
  We will use induction on \( n \) to show that
  \begin{equation*}
    \Bracks{\varphi}(x_1, \ldots, x_n) = f(x_1, \ldots, x_n).
  \end{equation*}

  \begin{itemize}
    \item The case \( n = 0 \) is special.
    \begin{itemize}
      \item If \( f = T \), then, by \fullref{alg:full_cnf_and_dnf/true}, \( \varphi = p \synvee \neg p \), which is a tautology.
      \item If \( f = F \), then, by \fullref{alg:full_cnf_and_dnf/false}, \( \varphi = p \synwedge \neg p \), which is a contradictory formula.
    \end{itemize}

    \item Let \( n = 1 \).
    \begin{itemize}
      \item If \( f(x) = T \), then again by \fullref{alg:full_cnf_and_dnf/true}, \( \varphi = p \synvee \neg p \) is a tautology.
      \item If \( f(x) = F \), then \fullref{alg:full_cnf_and_dnf/elementary} provides us with the elementary disjunctions \( p^F = p \) and \( p^T = \neg p \) and \fullref{alg:full_cnf_and_dnf/total} gives us \( \varphi = p \synwedge \neg p \).
      \item If \( f(x) = x \), then we only consider the elementary disjunction \( p^F = p \), which leads us to \( \varphi = p \).
      \item If \( f(x) = \oline{x} \), then \( \varphi = \neg p \).
    \end{itemize}

    Therefore, if \( f \) is unary, the formula \( \varphi \) constructed via this algorithm satisfies \( \Bracks{\varphi}(x) = f(x) \).

    \item Consider the function \( f(x_1, \ldots, x_n, x_{n+1}) \) and suppose that, for any \( n \)-ary Boolean function \( g \), the interpretation of its corresponding formula coincides with \( g \).

    Let \( \varphi_F \) be the formula obtained for \( f(x_1, \ldots, x_n, F) \) and let \( \varphi_F' \) be the formula obtained from \( \varphi_F \) by adding the literal \( p_{n+1} \) to each elementary disjunction.

    Dually, let \( \varphi_T \) be the formula for \( f(x_1, \ldots, x_n, T) \) and let \( \varphi_T' \) be the formula obtained by adding the literal \( \neg p_{n+1} \) to each elementary disjunction.

    Then
    \begin{equation*}
      \Bracks{\varphi_F' \synwedge \varphi_T'}(x_1, \ldots, x_n, F)
      =
      \underbrace{\Bracks{\varphi_F'}(x_1, \ldots, x_n, F)}_{\mathclap{\Bracks{\varphi_F}(x_1, \ldots, x_n) \T*{because} \Bracks{p_{n+1}}(F) = F}} \wedge \overbrace{\Bracks{\varphi_T'}(x_1, \ldots, x_n, F)}^{T \T*{because} \Bracks{\neg p_{n+1}}(F) = T}
      =
      \underbrace{\Bracks{\varphi_F}(x_1, \ldots, x_n)}_{f(x_1, \ldots, x_n, F)}
    \end{equation*}
    and similarly
    \begin{equation*}
      \Bracks{\varphi_F' \synwedge \varphi_T'}(x_1, \ldots, x_n, T)
      =
      \Bracks{\varphi_T}(x_1, \ldots, x_n)
      =
      f(x_1, \ldots, x_n, T).
    \end{equation*}
  \end{itemize}

  We conclude that the interpretation of a formula obtained through this algorithm is the initial function.
\end{defproof}

\paragraph{Equivalence of propositional formulas and Boolean functions}

\begin{proposition}\label{thm:propositional_formulas_and_boolean_functions}
  \hyperref[def:propositional_syntax/formula]{Propositional formulas} under \hyperref[def:propositional_semantics/classical]{classical semantics} have the following structural properties when treated as \hyperref[def:boolean_function]{Boolean functions}:

  \begin{thmenum}
    \thmitem{thm:propositional_formulas_and_boolean_functions/equivalence_classes} \hyperref[def:semantic_equivalence]{Semantic equivalence} \( \gleichstark \) is an \hyperref[def:equivalence_relation]{equivalence relation} on the set \( \op*{Form} \) of all propositional formulas.

    \thmitem{thm:propositional_formulas_and_boolean_functions/bijection} Given the set \( \mscrB \) of all \hyperref[def:boolean_function]{Boolean functions} of arbitrary arity, the following map is bijective:
    \begin{equation*}
      \begin{aligned}
        &\Phi: \op*{Form} / {\gleichstark} \to \mscrB \\
        &\Phi([\varphi]) \coloneqq \Bracks{\varphi}.
      \end{aligned}
    \end{equation*}
  \end{thmenum}
\end{proposition}
\begin{comments}
  \item This is one of the motivations for studying \hyperref[def:lindenbaum_tarski_algebra]{Lindenbaum-Tarski algebras}.
  \item Both \( \op*{Form} / {\gleichstark} \) and \( \mscrB \) are provably Boolean algebras, but we give very different proofs --- the former is a Boolean algebra due to the syntactic \fullref{thm:lindenbaum_tarski_algebras}, and the latter is a Boolean algebra due to the semantic \fullref{thm:functions_over_model_form_model}. This is another demonstration of the soundness and completeness stated in \fullref{thm:classical_first_order_logic_soundness_and_complete}.
\end{comments}
\begin{proof}
  \SubProofOf{thm:propositional_formulas_and_boolean_functions/equivalence_classes} Straightforward.

  \SubProofOf{thm:propositional_formulas_and_boolean_functions/bijection} The map \( \Phi \) is well-defined since, by definition of semantic equivalence, \( \varphi \gleichstark \psi \) whenever \( \Bracks{\varphi}_I = \Bracks{\psi}_I \) for every interpretation \( I \).

  Injectivity of \( \Phi \) is also obvious from the definition of semantic equivalence, while surjectivity is given by \fullref{alg:full_cnf_and_dnf}.
\end{proof}
