\subsection{Monoids}\label{subsec:monoids}

We list here several basic algebraic structures that we will use mostly as building blocks for more complicated structures.

As discussed in \fullref{rem:first_order_model_notation}, listing all operations explicitly is cumbersome, and we will usually avoid it.

\begin{figure}[!ht]
  \caption{Generalizations of groups}\label{fig:monoid_hierarchy}
  \smallskip
  \hfill
  \begin{forest}
    for tree={grow=north}
    [
      {\hyperref[def:group]{Group}}
        [
          {\hyperref[def:set_with_involution]{Set with involution}}, edge label={node[midway, right=1em]{\( x^{-1} \) (unary)}}
        ]
        [
          {\hyperref[def:monoid]{Monoid}},
            [{\hyperref[def:semigroup]{Semigroup}}, edge label={node[midway, right=1em]{\( x \cdot y \) (binary)}}]
            [{\hyperref[def:pointed_set]{Pointed set}}, edge label={node[midway, left=1em]{\( e \) (nullary)}}]
        ]
    ]
  \end{forest}
  \hfill\hfill
\end{figure}

\begin{definition}\label{def:operation_on_set}\mcite[65]{Rosen1999}
  An \term[bg=операция (\cite[def. IV.1]{ГеновМиховскиМоллов1991}), ru=операция (\cite[12]{ЦаленкоШульгейфер1974})]{operation} of arity \( n \) or \( n \)-ary operation on a \hyperref[def:set]{plain set} \( A \) is a \hyperref[def:function]{function} from \( A^n \) to \( A \).
\end{definition}
\begin{comments}
  \item The arity of an operation is not to be confused with the corresponding relation arity as defined in \fullref{def:relation/arity} --- functions are always binary relations.
  \item We introduce in \fullref{def:operation_arity} some common terminology based on the arity.
\end{comments}

\begin{definition}\label{def:operation_arity}\mcite[132]{Birkhoff1967}
  We introduce the following terminology based on the arity \( n \) of an \hyperref[def:operation_on_set]{operation}:
  \begin{itemize}
    \item \term{Nullary} for \( n = 0 \).
    \item \term[bg=унарна (операция) (\cite[75]{ГеновМиховскиМоллов1991}), ru=унарная (операция) (\cite[13]{Мальцев1970})]{Unary} for \( n = 1 \).
    \item \term[bg=бинарна (операция) (\cite[def. IV.1]{ГеновМиховскиМоллов1991}), ru=бинарная (операция) (\cite[13]{Мальцев1970})]{Binary} for \( n = 2 \).
    \item \term[bg=тернарна (операция) (\cite[75]{ГеновМиховскиМоллов1991}), ru=тернарное (отношение) (\cite[53]{Ляпин1960})]{Ternary} for \( n = 3 \).
  \end{itemize}
\end{definition}

\paragraph{Pointed sets}

\begin{definition}\label{def:pointed_set}\mcite[26]{MacLane1998}
  A \term[ru=множество с отмеченной точкой (\cite[16]{ЦаленкоШульгейфер1974})]{pointed set} is a simple algebraic structure --- a nonempty set \( X \) equipped with a distinguished element \( e \). It is an algebraic structure because \( e \) can be regarded as the sole value of a nullary \hyperref[def:operation_on_set]{operation} \( \ast: X^0 \to X \).

  We will call \( e \) the \term{origin} of \( X \) based on the terminology for \hyperref[def:affine_coordinate_system]{affine coordinate systems}.

  Pointed sets have the following metamathematical properties:
  \begin{thmenum}
    \thmitem{def:pointed_set/theory} Pointed sets can also be viewed as \hyperref[def:first_order_model]{models} of an empty \hyperref[def:first_order_theory]{theory} for a \hyperref[def:first_order_language]{first-order logic language} with a constant symbol, i.e. a nullary \hyperref[def:first_order_language/fun]{functional symbol}.

    \thmitem{def:pointed_set/homomorphism} A \hyperref[def:first_order_homomorphism]{first-order homomorphism} between the pointed sets \( (X, e_{X}) \) and \( (Y, e_{Y}) \) is, explicitly, a function \( \varphi: X \to Y \) that satisfies
    \begin{equation}\label{eq:def:pointed_set/homomorphism}
      \varphi(e_{X}) = e_{Y}.
    \end{equation}

    \thmitem{def:pointed_set/submodel} The set \( S \subseteq X \) is a \hyperref[def:first_order_submodel]{first-order submodel} of \( X \) if \( e \in S \).

    In particular, as a consequence of \fullref{thm:positive_formulas_preserved_under_homomorphism}, the image of a pointed set homomorphism is a submodel of its codomain.

    \thmitem{def:pointed_set/category} We denote the \hyperref[def:category_of_small_first_order_models]{category of \( \mscrU \)-small models} for this theory by \( \ucat{Set}_* \).
  \end{thmenum}
\end{definition}

\paragraph{Trivial objects}

\begin{remark}\label{rem:trivial_object}
  Some algebraic theories have an established \term{trivial structure} --- for example the \term{trivial group} \( \set{ e } \) defined by \incite[146]{Knapp2016BasicAlgebra} and \incite[42]{Aluffi2009}, also called a \term{zero group} by \incite[25]{MacLane1998} or the \term{trivial module} \( \set{ 0 } \), used by \incite[175]{Aluffi2009}, also called a \term{zero module} by \incite[146]{Knapp2016BasicAlgebra}, \incite[25]{MacLane1998} and even Paolo Aluffi himself in \cite[341]{Aluffi2009}. Dimension theory for modules allows characterizing the trivial module as a zero-dimensional module or vector space.

  All these are single-element models of the corresponding theories. We may also define trivial \hyperref[def:partially_ordered_set]{partially ordered sets}, which would be empty sets, trivial \hyperref[def:semilattice/lattice]{lattices}, which could either be one-element or two-element sets depending on whether \( \top = \bot \), and trivial \hyperref[def:directed_graph]{directed graphs}, which could be any of the things discussed in \fullref{rem:trivial_graph}. The trouble is that these are not well-established concepts, and hence it could not be clear from the context what we mean.

  For the few purposes that we will need them for, it is possible to give a general definition for trivial objects --- see \fullref{def:trivial_object}.
\end{remark}

\begin{definition}\label{def:trivial_object}
  Let \( \cat{C} \) be a \hyperref[def:category]{category} that is \hyperref[def:concrete_category]{concrete} over the category \hyperref[def:pointed_set/category]{\( \cat{Set}_* \)} of pointed sets, and let \( U: \cat{C} \to \cat{Set}_* \) be the corresponding (\hyperref[def:functor_invertibility/faithful]{faithful}) forgetful functor.

  Then any preimage under \( U \) of any single-element pointed set is a \hyperref[def:universal_objects/zero]{zero object} in \( \cat{C} \). We will call such objects \term{trivial}.
\end{definition}
\begin{comments}
  \item Even if the category is concrete over \( \cat{Set}_* \), it may not have a zero object. As an example, we can remove the trivial pointed set and all related morphisms from \( \cat{Set}_* \) itself.
  \item We decided to use the term \enquote{trivial} rather than \enquote{zero} since it is better established in the context of algebra.
  \item In particular, any single-element pointed set is a zero object in the category \( \cat{Set}_* \).
  \item Other examples for which we will use this definition include \hyperref[def:group]{groups}, \hyperref[def:ring]{nonunital ring}, \hyperref[def:semimodule]{semimodules} and \hyperref[def:module]{modules} --- in the aforementioned cases, all single-element structures are isomorphic.
\end{comments}
\begin{defproof}
  Suppose that \( U(Z) = \set{ e } \).

  \SubProof{Proof that \( Z \) is an \hyperref[def:universal_objects/initial]{initial object}} For any morphism \( f: Z \to B \), the function \( U(f) \) \enquote{picks} an element out of \( U(B) \). The only choice of element that makes \( U(f) \) a pointed set homomorphism is the origin of \( U(B) \), and hence the image of the morphism set \( \cat{C}(Z, B) \) under \( U \) consists of only one function.

  Since \( U \) is a faithful functor, this means that \( \cat{C}(Z, B) \) also has only one element.

  The object \( B \) was chosen arbitrarily, so we conclude that \( Z \) is an initial object.

  \SubProof{Proof that \( Z \) is a \hyperref[def:universal_objects/initial]{terminal object}} If \( U(A) = \set{ e } \), then, for any morphism \( f: A \to Z \), the function \( U(f) \) must take the only possible value \( e \) at any element of \( U(Z) \). Again, since \( U \) is faithful, we conclude that \( Z \) is also a terminal object.
\end{defproof}

\paragraph{Kernels and cokernels}

\begin{definition}\label{def:zero_morphisms}
  Suppose that the category \( \cat{C} \) has a \hyperref[def:universal_objects/zero]{zero object} \( 0 \).

  \begin{thmenum}
    \thmitem{def:zero_morphisms/morphism}\mcite[20]{MacLane1998} For every pair of objects \( A \) and \( B \) in \( \cat{C} \), there exists a unique morphism, called the \term[ru=нулевой морфизм (\cite[75]{ЦаленкоШульгейфер1974})]{zero morphism}, that \hyperref[def:factors_through]{uniquely factors through} \( 0 \):
    \begin{equation}\label{eq:def:zero_morphisms/morphism}
      \begin{aligned}
        \includegraphics[page=1]{output/def__zero_morphisms}
      \end{aligned}
    \end{equation}

    We denote this zero morphism by \( 0_{A,B} \) or even \( 0 \).

    \thmitem{def:zero_morphisms/kernel}\mcite[191]{MacLane1998} We define the \term[ru=ядро (\cite[36]{ЦаленкоШульгейфер1974})]{kernel} cone of a morphism \( f: A \to B \) as the \hyperref[eq:def:equalizers/equalizer]{equalizer cone} of \( f \) and \( 0_{A,B} \).

    \Fullref{thm:equalizer_invertibility} implies that a kernel morphism is necessarily a monomorphism. The converse may not hold.

    \thmitem{def:zero_morphisms/cokernel}\mcite[64]{MacLane1998} \hyperref[thm:categorical_principle_of_duality]{Dually}, we define the \term[ru=коядро (\cite[37]{ЦаленкоШульгейфер1974})]{cokernel} cocone of a morphism \( f: A \to B \) as the \hyperref[eq:def:equalizers/coequalizer]{coequalizer cocone} of \( f \) and \( 0_{A,B} \).

    \Fullref{thm:equalizer_invertibility} implies that a cokernel morphism is necessarily an epimorphism. The converse may not hold.

    \thmitem{def:zero_morphisms/image}\mcite[def. IX.1.15]{Aluffi2009} We define the \term{image} of a morphism \( f: A \to B \) as the kernel of the cokernel morphism.

    \medskip

    \thmitem{def:zero_morphisms/coimage}\mcite[def. IX.1.15]{Aluffi2009} Dually, we define the \term{coimage} of a morphism \( f: A \to B \) as the cokernel of the kernel morphism.
  \end{thmenum}
\end{definition}
\begin{comments}
  \item Neither the kernel, cokernel, image nor coimage of a morphism is guaranteed to exist.
\end{comments}

\begin{lemma}\label{thm:zero_morphism_composition}
  The left or right composition of a \hyperref[def:zero_morphisms/morphism]{zero morphism} with any other morphism is also a zero morphism.
\end{lemma}
\begin{proof}
  Consider the composition \( 0_{B,C} \bincirc f \), where \( f: A \to B \). The following diagram commutes because \( 0 \) is a terminal object:
  \begin{equation*}
    \begin{aligned}
      \includegraphics[page=1]{output/thm__zero_morphism_composition}
    \end{aligned}
  \end{equation*}

  The morphism \( 0_{A,C} \) uniquely factors through \( 0 \) by definition, and such a factorization is provided by \( 0_{0,C} \bincirc 0_{A,0} \). Then
  \begin{equation*}
    0_{A,C} = 0_{0,C} \bincirc 0_{A,0} = \underbrace{0_{0,C} \bincirc 0_{B,0}}_{0_{B,C}} \bincirc f.
  \end{equation*}

  Therefore, \( 0_{B,C} \bincirc f \) is also a zero morphism.

  An analogous proof can be given when composing \( f \) with a zero morphism on the right.
\end{proof}

\begin{theorem}[Image-coimage factorization]\label{thm:image_coimage_factorization}
  In a category with a zero object, for every morphism \( f: A \to B \), if both its \hyperref[def:zero_morphisms/image]{image} and \hyperref[def:zero_morphisms/coimage]{coimage} exist, there exists a unique morphism \( \widehat{f}: \co\img f \to \img f \) such that the following diagram commutes:
  \begin{equation*}
    \begin{aligned}
      \includegraphics[page=1]{output/thm__image_coimage_factorization}
    \end{aligned}
  \end{equation*}
\end{theorem}
\begin{proof}
  Consider the extended diagram
  \begin{equation*}
    \begin{aligned}
      \includegraphics[page=2]{output/thm__image_coimage_factorization}
    \end{aligned}
  \end{equation*}

  \SubProof{Construction of \( f': \co\img f \to B \)}

  The projection \( \pi: A \to \co\img f \) is a cokernel morphism of \( \iota' \) above, which is in turn a kernel morphism of \( \ker f \). Another coequalizer cocone for \( \iota' \) is \( (B, f) \). Indeed, \( \iota' \) itself satisfies
  \begin{equation}\label{eq:thm:image_coimage_factorization/f_cocone/1}
    f \bincirc \iota' = 0_{A, B} \bincirc \iota'.
  \end{equation}

  \Fullref{thm:zero_morphism_composition} implies that
  \begin{equation}\label{eq:thm:image_coimage_factorization/f_cocone/2}
    0_{A, B} \bincirc \iota' = 0_{\ker f, B}
  \end{equation}
  and
  \begin{equation}\label{eq:thm:image_coimage_factorization/f_cocone/3}
    0_{\ker f, B} = f \bincirc 0_{\ker f, A}.
  \end{equation}

  We can combine \eqref{eq:thm:image_coimage_factorization/f_cocone/1}, \eqref{eq:thm:image_coimage_factorization/f_cocone/2} and \eqref{eq:thm:image_coimage_factorization/f_cocone/3} to obtain
  \begin{equation*}
    f \bincirc \iota' = f \bincirc 0_{\ker f, A}.
  \end{equation*}

  Thus, \( (B, f) \) is also a coequalizer cocone for \( \iota' \), and hence there exists a unique morphism \( f': \co\img f \to B \) such that \( f = f' \bincirc \pi \).

  \SubProof{Construction of \( \widetilde{f}: \co\img f \to \img f \)} The embedding \( \iota: \img f \to B \) is a kernel morphism of \( \pi' \), which is a cokernel morphism of \( f \). As a cokernel morphism of \( f \), \( \pi' \) satisfies the following:
  \begin{equation*}
    \pi' \bincirc f = \underbrace{\pi' \bincirc 0_{A,B}}_{0_{A,\co\ker f}}.
  \end{equation*}

  But \( f = f' \bincirc \pi \), thus
  \begin{equation*}
    \pi' \bincirc f' \bincirc \pi = 0_{A,\co\ker f} = 0_{A,B} \bincirc \pi.
  \end{equation*}

  Since \( \pi \) is a cokernel morphism, \fullref{thm:equalizer_invertibility} implies that it is an epimorphism, and thus we can cancel it:
  \begin{equation*}
    \pi' \bincirc f' = \underbrace{0_{A,B}}_{\mathclap{0_{B,\co\ker f} \bincirc f'}}.
  \end{equation*}

  Hence, \( (\co\img f, f') \) is another equalizer cone for \( \pi' \), and there exists a unique homomorphism \( \widetilde{f}: \co\img f \to \img f \) such that \( f' = \pi \bincirc \widetilde{f} \).

  This is the desired homomorphism.
\end{proof}

\begin{proposition}\label{thm:zero_morphisms_pointed}
  Let \( \mscrL \) be a \hyperref[def:first_order_language]{first-order language} with no predicate symbols and a fixed nullary functional symbol \( \ast \). Let \( \Gamma \) be a theory over \( \mscrL \) that consists of \hyperref[def:positive_formula]{positive formulas} without existential quantifiers. Finally, let \( \cat{C} \) be the \hyperref[def:category_of_small_first_order_models]{category of \( \mscrU \)-small models} of \( \Gamma \), and suppose that \( \cat{C} \) has a \hyperref[def:trivial_object]{trivial object}.\footnote{All these requirements seem restrictive, yet the categories we will encounter will satisfy them.}

  We will denote the interpretation of \( \ast \) in the structure \( A \) via \( e_A \) rather than \( I_A(\ast) \).

  \begin{thmenum}
    \thmitem{thm:zero_morphisms_pointed/morphism} The unique \hyperref[def:zero_morphisms/morphism]{zero morphism} between the structures \( A \) and \( B \) is the function that sends each element of \( A \) to the origin of \( B \):
    \begin{equation*}
      \begin{aligned}
        &0_{A,B}: A \to B, \\
        &0_{A,B}(x) \coloneqq e_B.
      \end{aligned}
    \end{equation*}

    \thmitem{thm:zero_morphisms_pointed/kernel} The \hyperref[def:zero_morphisms/kernel]{categorical kernel} of a morphism \( f: A \to B \) is the \hyperref[def:set_valued_map/inverse]{preimage}
    \begin{equation*}
      f^{-1}(e_B) = \set{ x \in A \given f(x) = e_B }.
    \end{equation*}

    We denote this set via \( \ker f \). The kernel pair is \( (\ker f, \iota) \), where \( \iota \) is the canonical embedding from \( \ker f \) to \( A \).

    \thmitem{thm:zero_morphisms_pointed/cokernel} Dually, the \hyperref[def:zero_morphisms/cokernel]{categorical cokernel} of a morphism \( f: A \to B \) is the \hyperref[def:first_order_quotient]{quotient} of \( B \) by the \hyperref[def:first_order_congruence]{congruence} \hyperref[def:first_order_generated_congruence]{generated} by the relation \( {\sim} \) defined as
    \begin{equation*}
      y \sim e_B \T{if and only if} y \in f[A].
    \end{equation*}

    We denote this quotient via \( \co\ker f \). The cokernel pair is \( (\co\ker f, \pi) \), where \( \pi \) is the canonical projection from \( B \) to \( \co\ker f \).

    \thmitem{thm:zero_morphisms_pointed/image} The \hyperref[def:zero_morphisms/image]{categorical image} of a morphism \( f: A \to B \) is the \hyperref[def:set_valued_map/image]{set-theoretic image} \( f[A] \).

    \thmitem{thm:zero_morphisms_pointed/coimage} Dually, the \hyperref[def:zero_morphisms/coimage]{categorical coimage} of a morphism \( f: A \to B \) is the \hyperref[def:first_order_quotient]{quotient} of \( A \) by the \hyperref[def:first_order_congruence]{congruence} \hyperref[def:first_order_generated_congruence]{generated} by the relation \( {\sim} \) defined as
    \begin{equation*}
      x \sim e_A \T{if and only if} f(x) = e_B.
    \end{equation*}
  \end{thmenum}
\end{proposition}
\begin{proof}
  \SubProofOf{thm:zero_morphisms_pointed/morphism} Trivial.

  \SubProofOf{thm:zero_morphisms_pointed/kernel}

  \SubProof*{Proof that \( \ker f \) is a submodel of \( A \)} As the preimage of the submodel \( \set{ e_B } \) of \( B \), \( \ker f \) is a substructure of \( A \) as a consequence of \fullref{thm:def:first_order_homomorphism/preimage_is_substructure}. As a substructure, \( \ker f \) is a model of \( \Gamma \) as a consequence of \fullref{thm:substructure_is_model}.

  \SubProof*{Proof that \( (\ker f, \iota) \) is an equalizer cone} The following diagram must commute:
  \begin{equation*}
    \begin{aligned}
      \includegraphics[page=1]{output/thm__zero_morphisms_pointed}
    \end{aligned}
  \end{equation*}

  Indeed, if \( a \in \ker f \), then by definition \( f(\iota(x)) = f(a) = e_B \).

  \SubProof*{Proof that \( (\ker f, \iota) \) is universal} If \( (X, g) \) is another cone, there must exist a homomorphism \( \widetilde g: X \to \ker f \) such that the following diagram commutes:
  \begin{equation*}
    \begin{aligned}
      \includegraphics[page=2]{output/thm__zero_morphisms_pointed}
    \end{aligned}
  \end{equation*}

  In order for \( (X, g) \) to be a cone, for every element \( a \) of \( X \) we need \( f(g(a)) \) to be \( e_B \), which in turn requires \( g(a) \) to be in the preimage \( f^{-1}(e_B) \).

  Therefore, we can restrict the codomain of \( g \) to \( \ker f \) to obtain the desired homomorphism \( \widetilde g: X \to \ker f \).

  \SubProofOf{thm:zero_morphisms_pointed/cokernel}

  \SubProof*{Proof that \( \co\ker f \) is a model of \( \Gamma \)} Follows from \fullref{thm:quotient_preserves_positive_formulas}.

  \SubProof*{Proof that \( (\co\ker f, \pi) \) is a coequalizer cocone} The following diagram must commute:
  \begin{equation*}
    \begin{aligned}
      \includegraphics[page=3]{output/thm__zero_morphisms_pointed}
    \end{aligned}
  \end{equation*}

  Indeed, for every \( b \) in \( f[A] \), we must have \( \pi(b) = \pi(e_B) \), which is guaranteed by \( f(x) \cong e_B \).

  \SubProof*{Proof that \( (\co\ker f, \pi) \) is universal} If \( (X, g) \) is another cocone, there must exist a homomorphism \( \widetilde g: \co\img f \to Y \) such that the following diagram commutes:
  \begin{equation*}
    \begin{aligned}
      \includegraphics[page=4]{output/thm__zero_morphisms_pointed}
    \end{aligned}
  \end{equation*}

  In order for \( (\co\ker f, \pi) \) to be a cokernel, for every \( b \) in \( B \), even outside \( f[A] \), \( \widetilde g \) must satisfy
  \begin{equation}\label{eq:thm:zero_morphisms_pointed/cokernel/homomorphism}
    \widetilde{g}(\pi(b)) = g(b).
  \end{equation}

  We will now show that \( b \cong b' \) implies \( g(b) = g(b') \), which would allow us to use \eqref{eq:thm:zero_morphisms_pointed/cokernel/homomorphism} as a definition\footnote{We could also use \fullref{thm:quotient_structure_universal_property} to prove the statement, but it would not make the proof simpler.}.

  \Fullref{thm:homomorphism_induces_congruence} implies that the relation \( b \congdot b' \) defined to hold when \( g(b) = g(b') \) is a \hyperref[def:first_order_congruence]{congruence}. Furthermore, since \( (X, g) \) is a cocone, it must satisfy \( g(b) = e_Y \) whenever \( b \) is in \( f[A] \), which in turn implies that \( b \congdot b' \) for all pairs \( b \) and \( b' \) in \( f[A] \).

  But \( {\cong} \) is defined as the intersection of all congruences satisfying the latter condition. Therefore, \( {\cong} \) must be a subset of \( {\congdot} \).

  Then \( b \cong b' \) implies \( b \congdot b' \). Restated in terms of the respective homomorphisms, this means that \( \pi(b) = \pi(b') \) implies \( g(b) = g(b') \). Thus, the function \( \widetilde{g} \) is well-defined by \eqref{eq:thm:zero_morphisms_pointed/cokernel/homomorphism}.

  This shows that \( (\co\ker f, \pi) \) is a cokernel pair of \( f \).

  \SubProofOf{thm:zero_morphisms_pointed/image} Based on \fullref{thm:zero_morphisms_pointed/cokernel}, the cokernel of \( f \) is \( B / {\cong} \) based on the relation \( {\cong} \) generated by \( f(x) \sim e_B \) for all \( x \in A \), and the corresponding projection \( \pi \) is the cokernel morphism.

  Based on \fullref{thm:zero_morphisms_pointed/kernel}, the category-theoretic image of \( f \), the kernel of \( \pi \), is the preimage of the origin \( \pi(e_B) \) of \( B / {\cong} \) under \( \pi \), i.e. the equivalence class of \( e_B \) under \( {\cong} \).

  Therefore, the set-theoretic image \( f[A] \) necessarily belongs to the category-theoretic image \( \img f \).

  \Fullref{thm:def:first_order_homomorphism/image_is_substructure} implies that \( f[A] \) is a substructure of \( B \) and \fullref{thm:substructure_is_model} implies that it is a model of \( \Gamma \). Hence, \( f[A] \) is an object in the category \( \cat{C} \).

  As a substructure of \( \mscrY \), it is closed under function applications in \( \mscrY \), hence it contains the equivalence class of \( e_B \) under \( {\cong} \).

  Therefore, it \( f[A] \) is both a subset and a superset of \( \img f \), which means they are equal.

  \SubProofOf{thm:zero_morphisms_pointed/coimage} Based on \fullref{thm:zero_morphisms_pointed/kernel}, the categorical kernel of \( f \) is the preimage of \( e_B \) under \( f \), and the corresponding embedding \( \iota \) is the kernel morphism.

  Based on \fullref{thm:zero_morphisms_pointed/cokernel}, the category-theoretic coimage of \( f \), the cokernel of \( \iota \), is \( A / {\cong} \) based on the relation \( {\cong} \) generated by \( x \sim e_A \) for every \( x \) in \( \ker f = f^{-1}(e_B) \).
\end{proof}

\paragraph{Sets with involutions}

\begin{definition}\label{def:involution}\mcite[65]{Rosen1999}
  We say that an \hyperref[def:function/endofunction]{endofunction} \( f: X \to X \) on a \hyperref[def:set]{plain set} is an \term[ru=инволюция (\cite[333]{Арнольд2012})]{involution} if it is its own \hyperref[def:morphism_invertibility/isomorphism]{two-sided inverse}, that is,
  \begin{equation*}
    f \circ f^{-1} = f^{-1} \circ f = \id_X
  \end{equation*}
\end{definition}

\begin{definition}\label{def:set_with_involution}\mimprovised
  A \term{set with an involution} is a \hyperref[def:set]{set} \( X \) with a unary \hyperref[def:involution]{involution} \( (\anon)^{-1} \).

  Sets with involutions have the following metamathematical properties:
  \begin{thmenum}
    \thmitem{def:set_with_involution/theory} We define the theory of sets with involution as a theory over the language consisting of a single unary functional symbol \( \anon^{-1} \) with the sole axiom
    \begin{equation}\label{eq:def:set_with_involution/theory/axiom}
      (\xi^{-1})^{-1} \doteq \xi.
    \end{equation}

    \thmitem{def:set_with_involution/homomorphism} A \hyperref[def:first_order_homomorphism]{first-order homomorphism} between sets with involutions \( X \) and \( Y \) is a function \( \varphi: X \to Y \) satisfying
    \begin{equation}\label{eq:def:set_with_involution/homomorphism}
      \varphi(x^{-1})
      =
      \varphi(x)^{-1}.
    \end{equation}

    \thmitem{def:set_with_involution/submodel} Any subset of a set with involution is again a set with involution.

    In particular, as a consequence of \fullref{thm:positive_formulas_preserved_under_homomorphism}, the \hyperref[def:set_valued_map/image]{image} of a homomorphism \( \varphi: X \to Y \) is a submodel of \( Y \).

    The \hyperref[thm:substructures_form_complete_lattice/bottom]{bottom substructure} of any set with involution is the empty set.

    See \fullref{rem:empty_first_order_structures} regarding allowing empty sets as first-order structures.

    \thmitem{def:set_with_involution/category} We introduce no special notation for the corresponding \hyperref[def:category_of_small_first_order_models]{category of \( \mscrU \)-small models}.
  \end{thmenum}
\end{definition}

\paragraph{Semigroups}

\begin{definition}\label{def:binary_operation}
  An important special case of \hyperref[def:operation_on_set]{operations} are binary operations. Given a set \( A \), a \term{binary operation} is simply a function \( \cdot: A^2 \to A \), usually denoted via \hyperref[rem:first_order_formula_conventions/infix]{infix notation}.

  The following \hyperref[def:first_order_syntax/formula]{formulas}, or rather their \hyperref[def:universal_closure]{universal closures}, are common \hyperref[def:first_order_theory/axiomatized]{axioms}:
  \begin{thmenum}
    \thmitem{def:binary_operation/associative}\mcite[\textnumero 1.18(\( \alpha \))]{Ляпин1960} \term[bg=асоциативност (\cite[11]{ГеновМиховскиМоллов1991}), ru=ассоциативность, en=associativity (\cite[41]{Savage1998})]{Associativity}:
    \begin{equation}\label{eq:def:binary_operation/associative}
      (\xi \cdot \eta) \cdot \zeta = \xi \cdot (\eta \cdot \zeta).
    \end{equation}

    \thmitem{def:binary_operation/commutative}\mcite[\textnumero 1.18(\( \beta \))]{Ляпин1960} \term[bg=комутативност (\cite[11]{ГеновМиховскиМоллов1991}), ru=коммутативность, en=commutativity (\cite[3]{Golan2010})]{Commutativity}:
    \begin{equation}\label{eq:def:binary_operation/commutative}
      \xi \cdot \eta \doteq \eta \cdot \xi.
    \end{equation}

    \thmitem{def:binary_operation/idempotent}\mcite[\textnumero 3.1]{Мальцев1970} \term[ru=идемпотентность, en=idempotence (\cite[65]{Rosen1999})]{Idempotence}:
    \begin{equation}\label{eq:def:binary_operation/idempotent}
      \xi \cdot \xi \doteq \xi.
    \end{equation}

    \thmitem{def:binary_operation/cancellative}\mcite[19]{MacLane1998} Left and right \term[bg=съкращаване (\cite[77]{ГеновМиховскиМоллов1991}), ru=сокращение (\cite[\( 1.18(\varepsilon) \)]{Ляпин1960})]{cancellation}:
    \begin{align}
      \qforall \zeta (\zeta \cdot \xi \doteq \zeta \cdot \eta) \rightarrow \xi = \eta
      \label{eq:def:binary_operation/cancellative/left}
      \\
      \qforall \zeta (\xi \cdot \zeta \doteq \eta \cdot \zeta) \rightarrow \xi = \eta
      \label{eq:def:binary_operation/cancellative/right}
    \end{align}
  \end{thmenum}
\end{definition}

\begin{example}\label{ex:def:binary_operation}
  We list several examples of \hyperref[def:semigroup]{semigroups} satisfying different properties.

  \begin{thmenum}
    \thmitem{ex:def:binary_operation/algebraic} \hyperref[eq:def:binary_operation/associative]{Associative} \hyperref[def:binary_operation]{binary operations} on a set are abundant and are part of the definition of essential algebraic structures like \hyperref[def:group]{groups}, \hyperref[def:ring]{rings}, \hyperref[def:vector_space]{vector spaces} and \hyperref[def:semilattice]{(semi)lattices}.

    These operations are \term{homogeneous} in the sense that their signature only contains a single set, unlike \hyperref[def:group_action]{group actions} and scalar products in \hyperref[def:vector_space]{vector spaces}.

    \thmitem{ex:def:binary_operation/composition} The quintessential example of a non-\hyperref[def:binary_operation/commutative]{commutative} operation is \hyperref[def:set_valued_map/composition]{composition} in any set of functions or, more generally, \hyperref[def:category/composition]{morphism composition} in any \hyperref[def:category]{category}.

    Composition is \hyperref[def:binary_operation/associative]{associative}. \hyperref[def:binary_operation/cancellative]{Cancellation} with respect to composition is discussed in \fullref{def:morphism_invertibility} and, for function composition, in \fullref{thm:function_invertibility_categorical}.

    \thmitem{ex:def:binary_operation/midpoint} The midpoint operation
    \begin{equation*}
      (x, y) \mapsto \dfrac {x + y} 2
    \end{equation*}
    on \( \BbbR \) is commutative and cancellative semigroup, but not associative.
  \end{thmenum}
\end{example}

\begin{remark}\label{rem:binary_operation_syntax_trees}
  \hyperref[def:binary_operation]{Binary operations} are easily extended to higher arities. Given a binary operation \( +: A \times A \to A \), we can extend it via \hyperref[rem:natural_number_recursion]{natural number recursion} to arbitrary \( n \)-tuples \( x_1, \ldots, x_n \) via parentheses:
  \begin{equation}\label{eq:rem:binary_operation_syntax_trees/right}
    (x_1 + (x_2 + \cdots + (x_{n-1} + x_n) \cdots )).
  \end{equation}

  This expression corresponds to the \hyperref[rem:abstract_syntax_tree]{abstract syntax tree}
  \begin{equation}\label{eq:rem:binary_operation_syntax_trees/tree/basic}
    \begin{aligned}
      \includegraphics[page=1]{output/rem__binary_operation_syntax_trees}
    \end{aligned}
  \end{equation}

  Several things can be noted here.
  \begin{thmenum}
    \thmitem{rem:binary_operation_syntax_trees/associativity} When exchanging the order of the parentheses in the expression \eqref{eq:rem:binary_operation_syntax_trees/right}, only the root is changed in the syntax tree \eqref{eq:rem:binary_operation_syntax_trees/tree/basic}.

    Thus, for an \hyperref[def:binary_operation/associative]{associative} binary operation, abstract syntax trees can instead be represented as finite ordered rooted trees like
    \begin{equation}\label{eq:rem:binary_operation_syntax_trees/tree/associative}
      \begin{aligned}
        \includegraphics[page=2]{output/rem__binary_operation_syntax_trees}
      \end{aligned}
    \end{equation}

    Of course, \( n \)-ary trees can still be used for non-associative binary operations, as long as we have selected a strategy for \hyperref[rem:evaluation]{evaluation}. If we evaluate \eqref{eq:rem:binary_operation_syntax_trees/tree/associative} as we would evaluate \eqref{eq:rem:binary_operation_syntax_trees/right}, we say that the operation is \term{right associative}. Dually, if we evaluate \eqref{eq:rem:binary_operation_syntax_trees/tree/associative} as we would evaluate
    \begin{equation}\label{eq:rem:binary_operation_syntax_trees/left}
      (( \cdots (x_1 + x_2) + \cdots + x_{n-1}) + x_n),
    \end{equation}
    we say that the operation is \term{left associative}.

    Note that, unlike associativity, left and right associativity are not properties of the operation, but rather conventions on how to evaluate expressions without explicit parentheses. For example, in a \hyperref[def:heyting_algebra]{Heyting algebra}, the \hyperref[eq:def:heyting_algebra/conditional]{conditional} \( \rightarrow \) is not associative, but it is often taken to be right associative so that
    \begin{equation*}
      x \rightarrow y \rightarrow z
    \end{equation*}
    is evaluated as
    \begin{equation*}
      (x \rightarrow (y \rightarrow z)).
    \end{equation*}

    \thmitem{rem:binary_operation_syntax_trees/commutativity} If, additionally, the operation is \hyperref[def:binary_operation/commutative]{commutative}, we can regard the syntax tree as if the tree was a plain unordered (but still node-labeled) tree.

    Commutativity allows us to swap summands inside parentheses, and associativity is needed to \enquote{remove} the parentheses. Thus, it makes sense to always place parentheses in operations that are commutative but not associative.

    An an example, consider the real number midpoint operation
    \begin{equation*}
      x \oplus y \coloneqq \frac {x + y} 2
    \end{equation*}
    from \fullref{ex:def:binary_operation/midpoint}. It is commutative and not associative, and in the expression \( x \oplus y \oplus z \), we can swap \( x \) with \( y \) but not with \( z \). This can be very unintuitive. We aim to always write parentheses for non-associative operations.

    \thmitem{rem:binary_operation_syntax_trees/infinite} Suppose that \( + \) is associative and commutative. Suppose also that we are given an \hyperref[def:cartesian_product/indexed_family]{indexed family} \( \seq{ x_k }_{k \in \mscrK} \) of elements of \( M \). We can construct an infinite tree with root \( + \) and children \( \seq{ x_k }_{k \in \mscrK} \).

    This is not strictly a syntax tree. Syntax trees are necessarily finite, and our family \( \mscrK \) may even be uncountable. Nevertheless, we can sometimes evaluate this tree to obtain a member of the monoid.
    \begin{thmenum}
      \thmitem{rem:binary_operation_syntax_trees/infinite/lattice} The operations of a \hyperref[def:semilattice/lattice]{lattice} arise by specializing \hyperref[def:extremal_points/supremum_and_infimum]{suprema and infima} to binary sets. Conversely, as discussed in \fullref{thm:binary_lattice_operations/new_lattice}, the binary lattice operations induce a \hyperref[def:partially_ordered_set]{partial order}, and we can define arbitrary suprema and infima. If the lattice happens to be \hyperref[def:semilattice/complete]{complete}, instead of the binary operations, we can use
      \begin{equation*}
        \bigvee_{k \in \mscrK} x_k = \sup\set{ x_k \given k \in \mscrK }.
      \end{equation*}

      \thmitem{rem:binary_operation_syntax_trees/infinite/convergence} If \( M \) is a \hyperref[rem:topological_first_order_structures]{topological monoid} and if \( \mscrK = \set{ 1, 2, 3, \ldots } \), the family is a \hyperref[def:sequence]{sequence}, and we can define the sequence of partial sums
      \begin{equation*}
        \seq*{ \sum_{k=1}^n x_k }_{n=1}^\infty
      \end{equation*}

      This gives rise to \hyperref[def:convergent_series]{series} discussed in \fullref{subsec:series} and \fullref{subsec:real_series}. A limit may not exist for the net, unfortunately, and if it does, it may not be unique (if the topology is not \hyperref[def:separation_axioms/T2]{Hausdorff}).

      \thmitem{rem:binary_operation_syntax_trees/infinite/direct_sum} Suppose that \( M \) has an \hyperref[def:monoid]{neutral element} \( 0_M \). If only finitely many elements of the family are different from \( 0_M \), we regard the ordinary summation operation as well-defined on the whole family and write
      \begin{equation*}
        s \coloneqq \sum_{k \in \mscrK} x_k.
      \end{equation*}

      Technically, this involves selecting a \hyperref[def:well_ordered_set]{well-ordering} \( x_{1_n}, \ldots, x_{k_n} \) on the set
      \begin{equation*}
        \set{ x_k \given k \in \mscrK \T{and} x_k \neq 0 }
      \end{equation*}
      and assigning to \( s \) the result of the iterated binary operation
      \begin{equation*}
        (x_{1_n} + (x_{2_n} + \cdots + (x_{k_{n-1}} + x_{k_n}) \cdots)).
      \end{equation*}

      Commutativity ensures that the sum \( s \) does not depend on the order of summands (and hence on the well-order we have chosen). Adding any member of the family would not change the sum, which justifies this shorthand definition. Furthermore, since we only sum finitely many summands, we can construct a well-ordering using \hyperref[rem:natural_number_recursion]{natural number recursion} without relying on the \hyperref[def:zfc/choice]{axiom of choice}.

      This is fundamental for the definition of \hyperref[def:semimodule_direct_sum]{direct sums}, which in turn are used to define \hyperref[rem:linear_combinations]{linear combinations} and \hyperref[def:polynomial_algebra]{polynomials}.
    \end{thmenum}
  \end{thmenum}
\end{remark}

\begin{remark}\label{rem:magma_terminology}
  The term \enquote{magma} for a set with a binary operation was used by \incite[18]{Serre1992}. Decades later, however, it is still not widely established. \incite[example 21.3]{Golan2010} uses the term \enquote{groupoid} instead, but the latter term seems to be overtaken by groupoids in category theory --- see \fullref{def:groupoid}. \incite{Ляпин1960} calls them \enquote{multiplicative sets}, noting that the terminology is not established and that some authors use the term \enquote{groupoid}.

  Semigroups, being a special case when the operation is \hyperref[def:binary_operation/associative]{associative}, are well-established concepts, unlike magmas. The term is defined by \incite[144]{MacLane1998}, \incite[1]{Golan2010} and \incite[28]{Ляпин1960}, and used without definition by others, including \incite[21]{Birkhoff1967} and \incite[402]{Rockafellar1997}.

  Since non-associative operation rarely arise, we prefer defining semigroups rather than magmas.
\end{remark}

\begin{definition}\label{def:semigroup}\mcite[144]{MacLane1998}
  A \term[ru=полугруппа (\cite[28]{Ляпин1960})]{semigroup} is a nonempty set \( G \) equipped with a \hyperref[def:operation_arity]{binary} \hyperref[def:operation_on_set]{operation} \( \cdot: G \times G \to G \), called the \term{semigroup operation}. We often denote this operation by juxtaposition as \( xy \) instead of \( x \cdot y \).

  We will call the operation \term{multiplication} or, in the case of \hyperref[def:endomorphism_monoid]{endomorphism monoids} --- \term{composition}. See also the notes in \fullref{rem:additive_semigroup} regarding additive semigroups and in \fullref{def:monoid_delooping} regarding the order of operands.

  \begin{thmenum}[series=def:semigroup]
    \thmitem{def:semigroup/theory} In analogy to the \hyperref[def:pointed_set/theory]{theory of pointed sets}, we can define the theory of semigroups as an empty theory over a language with a single \hyperref[rem:first_order_formula_conventions/infix]{infix} binary functional symbol.

    \thmitem{def:semigroup/homomorphism} A \hyperref[def:first_order_homomorphism]{first-order homomorphism} between the semigroups \( (G, \cdot_{G}) \) and \( (H, \cdot_{H}) \) is, explicitly, a function \( \varphi: G \to H \) such that
    \begin{equation}\label{eq:def:semigroup/homomorphism}
      \varphi(x \cdot_{G} y) = \varphi(x) \cdot_{H} \varphi(y)
    \end{equation}
    for all \( x, y \in G \).

    \thmitem{def:semigroup/submodel} The set \( A \subseteq G \) is a \hyperref[def:first_order_substructure]{first-order submodel} of \( G \) if it is closed under the semigroup operation, that is, if \( x, y \in A \) implies \( xy \in A \).

    We call \( A \) a \term{sub-semigroup} of \( G \).

    As a consequence of \fullref{thm:positive_formulas_preserved_under_homomorphism}, the image of a semigroup homomorphism is a sub-semigroup of its codomain.

    \thmitem{def:semigroup/category} We introduce no special notation for the \hyperref[def:category_of_small_first_order_models]{category of \( \mscrU \)-small models} for the theory of semigroups.

    \thmitem{def:semigroup/exponentiation} We define an additional \term{exponentiation} operation for positive integers \( n \) \hyperref[rem:natural_number_recursion]{recursively} as
    \begin{equation}\label{eq:def:semigroup/exponentiation}
      x^n \coloneqq \begin{cases}
        x,               &n = 1 \\
        x^{n-1} \cdot x, &n > 1
      \end{cases}
    \end{equation}

    \thmitem{def:semigroup/opposite} The \term{opposite semigroup} of \( (G, \cdot) \) is the semigroup \( (G, \star) \) with multiplication reversed:
    \begin{equation*}
      x \star y \coloneqq y \cdot x.
    \end{equation*}

    We denote the opposite semigroup by \( G^{\opcat} \). This is justified in \fullref{def:monoid/opposite}.
  \end{thmenum}
\end{definition}

\begin{definition}\label{def:power_semigroup}\mcite[1]{McCarthyHayes1973}
  Given a \hyperref[def:semigroup]{semigroup} \( (G, \cdot) \), its \hyperref[def:basic_set_operations/power_set]{power set} \( \pow(G) \) is also a semigroup with the operation
  \begin{equation*}
    A \star B \coloneqq \set{ a \cdot b \given a \in A \T{and} b \in B }.
  \end{equation*}

  We will call this the \term{power semigroup} of \( G \). There is an obvious injection
  \begin{equation*}
    \begin{aligned}
      &\iota: G \to \pow(G), \\
      &\iota(g) \coloneqq \set{ g }.
    \end{aligned}
  \end{equation*}
\end{definition}
\begin{comments}
  \item Operations on sets are used for \hyperref[def:topological_vector_space]{topological vector spaces} by \incite[6]{Rudin1991Functional}, \incite[4]{Rockafellar1997}, \incite[4]{Clarke2013}, \incite[25]{МагарилИльяевТихомиров2002ВыпуклыйАнализ}.

  \item Note that this concept is distinct from \hyperref[def:semiring_ideal/product]{product ideals}, which use the same notation.

  \item Power semigroups are ordered --- see \fullref{ex:def:ordered_semigroup/power}
\end{comments}

\begin{example}\label{ex:def:power_semigroup}
  We list some examples of \hyperref[def:power_semigroup]{power semigroups}:
  \begin{thmenum}
    \thmitem{ex:def:power_semigroup/cancellative} The cancellative property does not extend to the power semigroup.

    Consider the additive group \hyperref[def:group_of_integers_modulo]{\( \BbbZ_2 \)}. It is a cancellative semigroup as a consequence of \fullref{thm:def:group/cancellative}. Define the sets \( A \coloneqq \set{ 0, 1 } \) and \( B \coloneqq \set{ 0 } \). Then
    \begin{equation*}
      A + A = A = A + B,
    \end{equation*}
    however we cannot cancel \( A \) from the left because \( A \neq B \).
  \end{thmenum}
\end{example}

\begin{proposition}\label{thm:semigroup_exponentiation_properties}
  Fix a semigroup \( G \). \hyperref[def:semigroup/exponentiation]{Exponentiation} in \( G \) has the following basic properties:

  \begin{thmenum}
    \thmitem{thm:semigroup_exponentiation_properties/commutativity} We have the following \hyperref[def:binary_operation/commutative]{commutativity}-like property: for any \( x \) in \( G \) and any positive integer \( n \), we have
    \begin{equation}\label{eq:thm:semigroup_exponentiation_properties/commutativity}
      x^n = x x^{n-1} = x^{n-1} x.
    \end{equation}

    \thmitem{thm:semigroup_exponentiation_properties/distributivity} Exponentiation distributes over multiplication: for any member \( x \) of \( G \) and any two positive integers \( n \) and \( m \),
    \begin{equation}\label{eq:thm:semigroup_exponentiation_properties/multiplication}
      x^{n + m} = x^n x^m.
    \end{equation}

    \thmitem{thm:semigroup_exponentiation_properties/repeated} For any member \( x \) of \( M \) and any two positive integers \( n \) and \( m \),
    \begin{equation}\label{eq:thm:semigroup_exponentiation_properties/repeated}
      (x^n)^m = x^{nm}.
    \end{equation}
  \end{thmenum}
\end{proposition}
\begin{proof}
  \SubProofOf{thm:semigroup_exponentiation_properties/commutativity} We use induction on \( n \). The cases \( n = 1 \) and \( n = 2 \) are obvious. For \( n > 2 \), we have
  \begin{equation*}
    x^n
    \reloset {\eqref{eq:def:semigroup/exponentiation}} =
    x x^{n-1}
    \reloset {\T{ind.}} =
    x x^{n-2} x
    \reloset {\eqref{eq:def:semigroup/exponentiation}} =
    x^{n-1} x.
  \end{equation*}

  \SubProofOf{thm:semigroup_exponentiation_properties/distributivity} We use induction on \( n \). The case \( n = 1 \) follows directly from \eqref{eq:def:semigroup/exponentiation}. The case \( n > 1 \) follows from
  \begin{equation*}
    x^{n + m}
    \reloset {\eqref{eq:def:semigroup/exponentiation}} =
    x x^{n + (m - 1)}
    \reloset {\T{ind.}} =
    x x^{n-1} x^m
    \reloset {\eqref{eq:def:semigroup/exponentiation}} =
    x^n x^m.
  \end{equation*}

  \SubProofOf{thm:semigroup_exponentiation_properties/repeated} We use induction on \( n \). The case \( n = 1 \) is obvious and the rest follows from
  \begin{equation*}
    (x^n)^m
    \reloset {\eqref{eq:def:semigroup/exponentiation}} =
    x^n (x^n)^{m-1}
    \reloset {\T{ind.}} =
    x^n x^{n (m - 1)}
    \reloset {\eqref{eq:thm:semigroup_exponentiation_properties/multiplication}} =
    =
    x^{nm}.
  \end{equation*}
\end{proof}

\begin{definition}\label{def:ordered_semigroup}\mcite[535]{Ляпин1960}
  A (partially) \term[ru=частично упорядоченная полугруппа (\cite[535]{Ляпин1960})]{ordered semigroup} is a \hyperref[def:semigroup]{semigroup} \( M \) equipped with a \hyperref[def:partially_ordered_set]{partial order} \( \leq \) such that \( x \leq y \) implies \( xz \leq yz \) and \( zx \leq zy \) for every \( z \) in \( M \).
\end{definition}
\begin{comments}
  \item The category of small ordered semigroups is a \hyperref[def:concrete_category]{concrete category} over both the \hyperref[def:semigroup/category]{category of semigroups} and the \hyperref[def:partially_ordered_set]{category of partially ordered sets}.
\end{comments}

\begin{example}\label{ex:def:ordered_semigroup}
  We list several examples of \hyperref[def:ordered_semigroup]{ordered semigroups}:

  \begin{thmenum}
    \thmitem{ex:def:ordered_semigroup/natural_numbers} The \hyperref[def:natural_numbers]{natural numbers} with addition form an ordered semigroup as a consequence of \fullref{thm:natural_numbers_are_well_ordered}; and so do \( \BbbZ \), \( \BbbQ \), \( \BbbR \) and \( \BbbC \).

    \thmitem{ex:def:ordered_semigroup/ordinal} More generally, every \hyperref[def:successor_and_limit_ordinal]{limit ordinal} as the set of all smaller ordinals is an ordered semigroup under \hyperref[def:ordinal_arithmetic/addition]{ordinal addition}.

    Unlike addition in natural numbers, however, ordinal addition is not commutative as shown in \fullref{ex:ordinal_addition}.

    \thmitem{ex:def:ordered_semigroup/semilattice} Every \hyperref[def:semilattice/join]{join-semilattice} \( (X, \vee) \) is an ordered semigroup with the lattice order. Indeed, if \( x \leq y \), then
    \begin{itemize}
      \item If \( z \leq x \), then
      \begin{equation*}
        x = x \vee z \leq y \vee z = y.
      \end{equation*}

      \item If \( x \leq z \leq y \), then
      \begin{equation*}
        z = x \vee z \leq y \vee z = y.
      \end{equation*}

      \item If \( z \geq y \), then
      \begin{equation*}
        z = x \vee z \leq y \vee z = z.
      \end{equation*}
    \end{itemize}

    Every \hyperref[def:semilattice/meet]{meet-semilattice} is also an ordered semigroup.

    \thmitem{ex:def:ordered_semigroup/power} Any \hyperref[def:power_semigroup]{power semigroup} is a Boolean algebra as discussed in \fullref{thm:boolean_algebra_of_subsets}, and hence also an ordered semigroup as a consequence of \fullref{ex:def:ordered_semigroup/semilattice}.
  \end{thmenum}
\end{example}

\paragraph{Monoids}

\begin{definition}\label{def:monoid}\mcite[vii]{MacLane1998}
  A \term[ru=моноид (\cite[94]{Мальцев1970})]{monoid} is an \hyperref[eq:def:binary_operation/associative]{associative} \hyperref[def:semigroup]{semigroup} with a distinguished element \( e \) such that \( ex = x = xe \) for every \( x \). Such an element is obviously unique, and we call it the \term{neutral element} of the monoid. This makes monoids \hyperref[def:pointed_set]{pointed sets}.

  \begin{thmenum}
    \thmitem{def:monoid/theory} The theory of monoids consists of \hyperref[eq:def:binary_operation/associative]{associativity} and the axiom
    \begin{equation}\label{eq:def:monoid/theory/neutral}
      \qforall \xi (e \cdot \xi \doteq \xi \wedge \xi \cdot e \doteq \xi)
    \end{equation}
    over the combined language of \hyperref[def:pointed_set/theory]{pointed sets} and \hyperref[def:semigroup/theory]{semigroups}.

    \thmitem{def:monoid/homomorphism} A \hyperref[def:first_order_homomorphism]{first-order homomorphism} between monoids is a function that is both a \hyperref[def:pointed_set/homomorphism]{pointed set homomorphism} and a \hyperref[def:semigroup/homomorphism]{semigroup homomorphism}, that is, it satisfies both \eqref{eq:def:pointed_set/homomorphism} and \eqref{eq:def:semigroup/homomorphism}.

    \thmitem{def:monoid/submodel} The set \( A \subseteq M \) is a \hyperref[def:first_order_submodel]{first-order submodel} of the monoid \( M \) if it contains \( e \). This is equivalent to \( A \) being a pointed subset.

    We say that \( A \) is a \term{submonoid}.

    As a consequence of \fullref{thm:positive_formulas_preserved_under_homomorphism}, the image of a monoid homomorphism is a submonoid of its codomain.

    \thmitem{def:monoid/category} The \hyperref[def:category_of_small_first_order_models]{category of \( \mscrU \)-small models} \( \ucat{Mon} \) of monoids is \hyperref[def:concrete_category]{concrete} with respect to both the \hyperref[def:pointed_set/category]{category of \( \mscrU \)-small pointed sets} and \hyperref[def:semigroup/category]{category of \( \mscrU \)-small semigroups}.

    As such, it has a \hyperref[def:universal_objects/zero]{zero object} --- any one-element monoid --- and it satisfies \fullref{thm:zero_morphisms_pointed}.

    We will discuss the free monoid functor in \fullref{thm:free_monoid_universal_property}. Then from \fullref{thm:first_order_categorical_invertibility} it will follow that monoid monomorphisms are injective functions.

    For epimorphisms a similar statement does not hold, unfortunately. The embedding \( \iota: \BbbN \to \BbbZ \) is an epimorphism by \fullref{thm:grothendieck_monoid_completion_universal_property}, but it is clearly not surjective.

    \thmitem{def:monoid/trivial} Any single-element monoid, i.e. every zero object in \( \ucat{Mon} \), is as trivial object in the sense of \fullref{def:trivial_object}. We will call it \enquote{the} trivial monoid.

    \thmitem{def:monoid/exponentiation} We extend \hyperref[def:semigroup/exponentiation]{semigroup exponentiation} to all nonnegative integers by additionally defining
    \begin{equation*}
      x^0 \coloneqq e.
    \end{equation*}

    \thmitem{def:monoid/commutative} We usually write \hyperref[def:binary_operation/commutative]{commutative} monoids additively as explained in \fullref{rem:additive_semigroup}. We denote the subcategory of commutative monoids by \( \ucat{CMon} \).

    \thmitem{def:monoid/opposite} The \hyperref[def:monoid_delooping]{delooping} of the \hyperref[def:semigroup/opposite]{opposite semigroup} for a monoid \( M \) is the \hyperref[def:opposite_category]{opposite category} of the delooping of \( M \). This justifies the notation \( M^{\opcat} \)
  \end{thmenum}
\end{definition}
\begin{comments}
  \item The requirement of associativity is conventional but not strictly necessary. Non-associative monoids will not be useful to us, however.
\end{comments}

\begin{example}\label{ex:def:monoid}
  We list several important examples \hyperref[def:monoid]{monoids}.

  \begin{thmenum}
    \thmitem{ex:def:monoid/natural_numbers} The \hyperref[def:natural_numbers]{nonnegative natural numbers} with addition form a quintessential example of a monoid. We prove in \fullref{thm:natural_number_addition_properties} that they are a monoid.

    \thmitem{ex:def:monoid/weak_limit_cardinal} More generally, for every \hyperref[def:successor_and_limit_cardinal/weak_limit]{weak limit cardinal}, the set of smaller cardinals is a commutative monoid under \hyperref[def:cardinal_arithmetic/addition]{cardinal addition} as a consequence of \fullref{thm:cardinal_addition_algebraic_properties}.

    \thmitem{ex:def:monoid/limit_ordinal} For a \hyperref[def:successor_and_limit_ordinal]{limit ordinal}, however, the set of smaller ordinals is \hi{non-commutative} under \hyperref[def:ordinal_arithmetic/addition]{ordinal addition} as a consequence of \fullref{thm:ordinal_addition_algebraic_properties}.

    \thmitem{ex:def:monoid/kleene_star} Another important example of a monoid is the \hyperref[def:formal_language/kleene_star]{Kleene star} \( \mscrA \) over some \hyperref[def:formal_language]{alphabet} \( \mscrA \).

    The importance for monoid theory comes from the free monoid universal property described in \fullref{thm:free_monoid_universal_property}.

    \thmitem{ex:def:monoid/semilattice} Every \hyperref[def:semilattice/bounded]{bounded} \hyperref[def:semilattice/join]{join-semilattice} is a monoid as a consequence of \eqref{eq:thm:binary_lattice_operations/neutral/meet}, and similarly for \hyperref[def:semilattice/meet]{meet-semilattice}.

    \thmitem{ex:def:monoid/power} The \hyperref[def:power_semigroup]{power semigroup} \( \pow(M) \) of a monoid \( M \) is also a monoid --- \( \iota(1_M) = \set{ 1_M } \) is a neutral element.
  \end{thmenum}
\end{example}

\begin{example}\label{ex:monoid_cancellation_not_preserved_by_homomorphism}\mcite{MathSE:semigroup_cancellation_not_preserved}
  \hyperref[def:monoid/homomorphism]{Monoid homomorphisms} may not preserve the \hyperref[def:binary_operation/cancellative]{cancellation property}. For example, the \hyperref[def:natural_numbers]{natural numbers} \( \BbbN \) are a cancellative monoid under addition, as shown in \fullref{thm:natural_number_addition_properties}, but the homomorphism
  \begin{equation*}
    \begin{aligned}
      &h: (\BbbN, +) \to (\BbbF_2, \max) \\
      &h(n) \coloneqq \begin{cases}
        0, &n = 0 \\
        1, &n > 0
      \end{cases}
    \end{aligned}
  \end{equation*}
  does not preserve the cancellation property.

  Indeed, \( \max\set{ 0, 1 } = \max\set{ 1, 1 } \), but \( 0 \neq 1 \).
\end{example}

\begin{definition}\label{def:monoid_idempotent}\cite[1]{Golan2010}
  We say that an element \( x \) of a \hyperref[def:monoid]{monoid} (or, more generally, a \hyperref[def:semigroup]{semigroup}) is \term[ru=идемпотент (\cite[72]{Ляпин1960}), en=idempotent]{idempotent} if \( x^2 = x \). Since the neutral element is always idempotent, we call idempotent elements distinct from it \term{nontrivial}.
\end{definition}
\begin{comments}
  \item Thus, the monoid operation is itself idempotent in the sense of \fullref{def:binary_operation/idempotent} if and only if every element is idempotent.
\end{comments}
