\subsection{Abelian groups}\label{subsec:abelian_groups}

\begin{definition}\label{def:abelian_group}
  A \hyperref[def:magma/commutative]{commutative} \hyperref[def:group]{group} is usually called an \term{abelian group}. We denote by \( \cat{Ab} \) the category of abelian groups.

  By \fullref{thm:ring_is_integer_algebra}, the abelian groups are precisely the \hyperref[def:algebra_over_ring]{rings} over \( \BbbZ \), and we have an \hyperref[rem:category_similarity/isomorphism]{isomorphism of categories} \( \cat{Ab} \cong \cat{Mod}_\BbbZ \).
\end{definition}

\begin{remark}\label{rem:additive_magma}
  General groups often arise as \hyperref[def:automorphism_group]{automorphism groups}, which are, for the most part, non-commutative, while abelian groups are usually used as the main building block for \hyperref[def:ring]{rings} and \hyperref[def:module]{modules}.

  To make a further distinction, if the operation is denoted by \( \cdot \) or juxtaposition, we say that the group is a \term{multiplicative group}, and if the operation is denoted by \( + \), we say that the group is an \term{additive group}. This terminology usually, but not necessarily, coincides with the group (or, more generally, the \hyperref[def:magma]{magma}) being \hyperref[def:magma/commutative]{commutative}.

  To make things explicit, a \term{multiplicative magma} is any magma as defined in \fullref{def:magma}. Compare this to \term{additive magmas}, where
  \begin{thmenum}
    \thmitem{rem:additive_magma/addition} The magma operation is denoted by \( + \) and called \term{addition}.

    \thmitem{rem:additive_magma/multiplication} The magma \hyperref[def:magma/exponentiation]{exponentiation operation} \( x^n \) is denoted by \( n \cdot x \) or juxtaposition and called \term{multiplication}. Thus, multiplication is not defined for two elements of the magma, but defined for a positive integer and an element of the magma. That is,
    \begin{equation}\label{eq:rem:additive_magma/multiplication}
      \begin{aligned}
        &\cdot: \cdot: \BbbN \times R \to R \\
        &n \cdot x \coloneqq \begin{cases}
          0_M,           &n = 0, \T{initial condition if} M \T{is a monoid} \\
          x,             &n = 1, \T{initial condition if} M \T{is not a monoid} \\
          n \cdot x + x, &n > 1 \\
          -(n \cdot x),  &n < 0, \\
        \end{cases}
      \end{aligned}
    \end{equation}

    In the case of a \hyperref[def:magma/commutative]{commutative} \hyperref[def:monoid]{monoid}, if multiplication is extended to two elements of the monoid, we instead talk about \hyperref[def:semiring]{semirings}.

    \thmitem{rem:additive_magma/identity} The \hyperref[def:monoid]{identity} is usually denoted by \( 0 \).

    \thmitem{rem:additive_magma/inverse} If an \hyperref[def:monoid_inverse]{inverse} of \( x \) exists, it is denoted by \( -x \) rather than \( x^{-1} \).
  \end{thmenum}
\end{remark}

\begin{proposition}\label{thm:abelian_outer_automorphism_group}
  In an \hyperref[def:abelian_group]{abelian group}, the full \hyperref[def:automorphism_group]{automorphism group} \( \aut(G) \) is isomorphic to the \hyperref[def:inner_and_outer_automorphisms]{outer automorphism group} \( \op{out}(G) \).
\end{proposition}
\begin{proof}
  If the group operation is \hyperref[def:magma/commutative]{commutative}, then \( xyx^{-1} = yxx^{-1} = y \), which makes the \hyperref[def:inner_and_outer_automorphisms]{conjugation action} trivial. Thus, the \hyperref[def:inner_and_outer_automorphisms]{inner automorphism group} \( \op{int}(G) \) is trivial, and hence \( \aut(G) \cong \op{out}(G) \).
\end{proof}

\begin{proposition}\label{thm:abelian_normal_subgroups}
  All subgroups of an abelian group are \hyperref[thm:normal_subgroup_equivalences]{normal}.
\end{proposition}
\begin{proof}
  Let \( G \) be abelian and \( H \) be a subgroup of \( G \). Then \( x H x^{-1} = xx^{-1} H = H \) for any \( x \in G \) and thus \( H \) is normal.
\end{proof}

\begin{definition}\label{def:congruence_modulo_normal_subgroup}
  Given a \hyperref[thm:normal_subgroup_equivalences]{normal subgroup} \( N \) of an \hyperref[def:abelian_group]{abelian group} \( G \), we say that two elements \( x \) and \( y \) of \( G \) are \term{congruent modulo} \( N \) and write \( x \cong y \pmod N \) if \( x - y \in N \).

  If \( N = \braket{ z } \), this implies that \( x \cong y \pmod z \) if and only if \( x - y \in \braket{ z } \).

  This concept also extends to \hyperref[def:semiring_ideal]{ring ideals} rather than normal subgroups, in which case \( \braket{ z } \) is the \hyperref[def:semiring_ideal/generated]{ideal generated} by \( z \) rather than the \hyperref[def:cyclic_group]{cyclic subgroup} of \( z \).
\end{definition}

\begin{proposition}\label{thm:group_of_integers_modulo}
  The \hyperref[def:integers]{integers} \( \BbbZ \) form an abelian group under addition. For every positive integer \( n \), we define the group
  \begin{equation*}
    \BbbZ_n \coloneqq \set{ 0, 1, \ldots, n - 1 }
  \end{equation*}
  with the operation
  \begin{equation*}
    x \oplus y \coloneqq \rem(x + y, n)
  \end{equation*}
  so that
  \begin{equation*}
    x \oplus y \cong x + y \pmod n.
  \end{equation*}

  The group \( \BbbZ_n \) is called the \term{group of integers modulo} \( n \). Compare this result with \fullref{thm:ring_of_integers_modulo}.
\end{proposition}
\begin{proof}
  We will prove that \( \BbbZ_n \) is an abelian group.

  \SubProofOf[def:magma/associative]{associativity} Addition in \( \BbbZ_n \) is associative since
  \begin{balign*}
    (x \oplus y) \oplus z
    &=
    \rem((x \oplus y) + z, n)
    = \\ &=
    \rem(\rem(x + y, n) + z, n)
    = \\ &=
    \rem(x + y - n \quot(x + y, n) + z, n)
    = \\ &=
    \rem(x + y + z, n)
    = \\ &=
    \ldots
    = \\ &=
    x \oplus (y \oplus z).
  \end{balign*}

  \SubProofOf[def:monoid]{identity} The zero is the identity.

  \SubProofOf[def:monoid_inverse]{inverse} Fix \( x \in \BbbZ_n \). If \( x = 0 \), its inverse is \( 0 \). If \( x > 0 \), its inverse is \( n - x \) since \( n - x \in \BbbZ_n \) and
  \begin{equation*}
    x \oplus (n - x) = x + (n - x) - n = 0.
  \end{equation*}

  \SubProofOf[def:magma/commutative]{commutativity} Follows from
  \begin{equation*}
    x \oplus y
    =
    \rem(x + y, n)
    =
    \rem(y + x, n)
    =
    y \oplus x.
  \end{equation*}
\end{proof}

\begin{proposition}\label{thm:integers_modulo_isomorphic_to_quotient_group}
  The group \( \BbbZ_n \) of \hyperref[thm:group_of_integers_modulo]{integers modulo \( n \)} is isomorphic to the quotient of \( \BbbZ \) by \( n\BbbZ = \set{ nz \given z \in \BbbZ } \). That is,
  \begin{equation*}
    \BbbZ_n \cong \BbbZ / n\BbbZ.
  \end{equation*}
\end{proposition}
\begin{proof}
  Define the function
  \begin{align*}
    &\varphi: \BbbZ_n \to \BbbZ / n\BbbZ  \\
    &\varphi(x) \coloneqq x + n\BbbZ.
  \end{align*}

  It is a homomorphism because
  \begin{balign*}
    \varphi(x \oplus y)
    &=
    \varphi(\rem(x + y, n))
    = \\ &=
    \varphi(x + y - n \quot(x + y, n))
    = \\ &=
    x + y - n \quot(x + y, n) + n\BbbZ
    = \\ &=
    x + y + n\BbbZ
    = \\ &=
    (x + n\BbbZ) + (y + n\BbbZ)
    = \\ &=
    \varphi(x) + \varphi(y).
  \end{balign*}

  Furthermore, this shows that \( \varphi \) is also an isomorphism.
\end{proof}

\begin{example}\label{ex:lagranges_theorem_for_groups/direct_product_zn}
  \Fullref{thm:lagranges_theorem_for_groups} and \fullref{thm:integers_modulo_isomorphic_to_quotient_group} imply that, for any positive integer \( n \), \( (nm, k) \mapsto nm + k \) is a bijection between \( n \BbbZ \times \BbbZ_n \) and \( \BbbZ \). This bijection, however, is not necessarily a group isomorphism because \eqref{eq:def:magma/homomorphism} may not hold.

  Consider the tuples \( (nm_1, k_1) \) and \( (nm_2, k_2) \)  in \( n \BbbZ \times \BbbZ_n \). We have
  \begin{equation*}
    (nm_1, k_1) + (nm_2, k_2) = (nm_1 + nm_2, \rem(k_1 + k_2, n)).
  \end{equation*}

  Therefore, if \( k_1 + k_2 \geq n \),
  \begin{equation*}
    nm_1 + nm_2 + \rem(k_1 + k_2, n) < (nm_1 + k_1) + (nm_2 + k_2).
  \end{equation*}
\end{example}

\begin{proposition}\label{thm:cyclic_group_isomorphic_to_integers_modulo_n}
  The \hyperref[def:cyclic_group]{cyclic group} \( C_n \) is isomorphic to the group \hyperref[thm:group_of_integers_modulo]{\( \BbbZ_n \)} of integers modulo \( n \).
\end{proposition}
\begin{proof}
  The homomorphism
  \begin{equation*}
    \begin{aligned}
      &\varphi: \BbbZ_n \to C_n \\
      &\varphi(k) \coloneqq a^k,
    \end{aligned}
  \end{equation*}
  and the analogous homomorphism for the infinite group, are isomorphisms.
\end{proof}

\begin{definition}\label{def:monoid_grothendieck_completion}\mcite[sec. 2.1]{LimaFilho1993}
  Let \( M \) be a \hyperref[def:magma/commutative]{commutative} \hyperref[def:monoid]{monoid}. Define the \hyperref[def:equivalence_relation]{equivalence relation} \( \sim \) on tuples of members of \( M \) to hold for \( (a, b) \sim (a', b') \) if there exists an element \( u \) of \( M \) such that
  \begin{equation*}
    a + b' + u = a' + b + u.
  \end{equation*}

  Define addition on the \hyperref[thm:equivalence_partition]{equivalence partition} \( G \coloneqq (M \times M) / {\sim }\) componentwise as
  \begin{equation*}
    [(a, b)] \oplus [(c, d)] \coloneqq [(a + c, b + d)]
  \end{equation*}
  and fix a canonical embedding
  \begin{equation*}
    \begin{aligned}
      &\iota_M: M \to G \\
      &\iota_M(m) \coloneqq [(m, 0)].
    \end{aligned}
  \end{equation*}

  We call the obtained \hyperref[def:abelian_group]{abelian group} \( (G, \oplus) \) the \term{Grothendieck completion} of \( M \).
\end{definition}
\begin{defproof}
  \SubProof{Proof that \( \sim \) is an equivalence relation}
  \SubProofOf*[def:binary_relation/reflexive]{reflexivity}
  \begin{equation*}
    (a, b) \sim (a, b) \T{if and only if} a + b + 0 = a + b + 0
  \end{equation*}

  \SubProofOf*[def:binary_relation/symmetric]{symmetry} By commutativity, if \( (a, b) \sim (a', b') \), then there exists \( u \) such that
  \begin{equation*}
    a + b' + u = a' + b + u
    =
    a' + b + u = a + b' + u,
  \end{equation*}
  hence \( (a', b') \sim (a, b) \).

  \SubProofOf*[def:binary_relation/transitive]{transitivity} Suppose that \( (a, b) \sim (a', b') \) and \( (a', b') \sim (a^\dprime, b^\dprime) \). Thus, there exist elements \( u \) and \( b \) of \( M \) such that
  \begin{align*}
    a + b' + u         &= a' + b + u, \\
    a' + b^\dprime + v &= a^\dprime + b' + v.
  \end{align*}

  Summing both sides, we obtain
  \begin{equation*}
    (a + b' + u) + (a' + b^\dprime + v) = (a' + b + u) + (a^\dprime + b' + v)
  \end{equation*}

  We reorder both sides to obtain
  \begin{equation*}
    (a + b^\dprime) + (a' + b' + u + v) = (a^\dprime + b) + (a' + b' + u + v),
  \end{equation*}
  which implies \( (a, a^\dprime) \sim (b, b^\dprime) \).

  \SubProof{Proof that \( (G, \oplus) \) is an abelian group}

  \SubProof*{Proof that \( \oplus \) is well-defined} The addition operation on \( G \) does not depend on the representative of the equivalence class. Indeed, let \( (a, b) \sim (a', b') \) and \( (c, d) \sim (c', d') \). Then there exist \( u \) and \( b \) such that
  \begin{align*}
    a + b' + u &= a' + b + u, \\
    c + d' + v &= c' + d + v.
  \end{align*}

  When added combined, these give
  \begin{equation*}
    (a + c) + (b' + d') + (u + v)
    =
    (a' + c') + (b + d) + (u + v),
  \end{equation*}
  which implies that
  \begin{equation*}
    (a + c, b + d) \sim (a' + c', b' + d').
  \end{equation*}

  \SubProofOf*[def:magma/associative]{associativity} Associativity of multiplication in \( G \) is inherited from multiplication in \( M \).

  \SubProofOf*[def:monoid]{identity} The equivalence class \( [(0, 0)] \) is an identity in \( G \) and contains the pairs \( (x, x) \) of identical elements.

  \SubProofOf*[def:monoid_inverse]{inverse} For each member \( (a, b) \in M \times M \), its inverse is \( (b, a) \) because
  \begin{equation*}
    [(a, b)] \oplus [(b, a)] = [(a + b, b + c)],
  \end{equation*}
  which, by commutativity, belongs to \( [(0, 0)] \).

  \SubProofOf*[def:magma/commutative]{commutativity} Commutativity of the group operation \( \oplus \) is also inherited from the monoid operation \( + \).
\end{defproof}

\begin{theorem}[Grothendieck monoid completion universal property]\label{thm:grothendieck_monoid_completion_universal_property}\mcite[sec. 2.1]{LimaFilho1993}
  The \hyperref[def:monoid_grothendieck_completion]{Grothendieck completion} \( \overline{M} \) of a commutative monoid \( M \) satisfies the following \hyperref[rem:universal_mapping_property]{universal mapping property}:
  \begin{displayquote}
    For every abelian group \( G \) and every monoid homomorphism \( \varphi: M \to G \), there exists a unique group homomorphism \( \widetilde{\varphi}: \overline{M} \to G \) such that the following diagram commutes:
    \begin{equation}\label{eq:thm:grothendieck_monoid_completion_universal_property/diagram}
      \begin{aligned}
        \includegraphics[page=1]{output/thm__grothendieck_monoid_completion_universal_property.pdf}
      \end{aligned}
    \end{equation}
  \end{displayquote}

  Via \fullref{rem:universal_mapping_property}, \( \overline{\anon} \) becomes \hyperref[def:category_adjunction]{left adjoint} to the \hyperref[def:concrete_category]{forgetful functor}
  \begin{equation*}
    U: \cat{Ab} \to \cat{CMon}.
  \end{equation*}

  Compare this result to \fullref{thm:grothendieck_semiring_completion_universal_property}.
\end{theorem}
\begin{proof}
  Let \( \varphi: M \to G \) be a monoid homomorphism into an abelian group \( G \). We want to define a homomorphism \( \overline{\varphi} \) such that
  \begin{equation*}
    \overline{\varphi}(\iota_M(a)) = \overline{\varphi}([(a, 0)]) = \varphi(a).
  \end{equation*}

  Each equivalence class \( C \) in \( G \) has a unique member \( a \) such that \( (a, 0) \in C \), hence the above condition is well-posed.

  Fix pairs \( (a, b) \) and \( (a', b') \) from \( M \times M \). Suppose that \( (a, b) \sim (a', b') \). Then there exists \( u \in M \) such that
  \begin{equation*}
    a + b' + u = a' + b + u.
  \end{equation*}

  An additional restriction on \( \overline{\varphi} \) is then
  \begin{equation*}
    \overline{\varphi}\parens[\Big]{ [(a, b)] }
    =
    \overline{\varphi}\parens[\Big]{ [(a', b')] }.
  \end{equation*}

  We need to cancel out \( u \). This uniquely determines \( \overline{\varphi} \) as
  \begin{equation*}
    \overline{\varphi}([(a, b)]) \coloneqq \varphi(a) - \varphi(b).
  \end{equation*}
\end{proof}

\begin{definition}\label{def:group_commutator}\mcite[313]{Knapp2016BasicAlgebra}
  Let \( G \) be an arbitrary group. We define the \term{commutator} of the elements \( x \) and \( y \) as
  \begin{equation*}
    [x, y] \coloneqq xyx^{-1}y^{-1}.
  \end{equation*}

  The \term{commutator subgroup} \( [G, G] \) of \( G \) is the subgroup \hyperref[def:group/submodel]{generated} by all the commutators in \( G \).
\end{definition}

\begin{theorem}[Group abelianization universal property]\label{thm:group_abelianization_universal_property}\mcite[prop. 7.4]{Knapp2016BasicAlgebra}
  The commutator group \( [G, G] \) of any group \( G \) is \hyperref[thm:normal_subgroup_equivalences]{normal} and the quotient \( G / [G, G] \) is an abelian group, which we call the \term{abelianization} of \( G \), satisfies the following \hyperref[rem:universal_mapping_property]{universal mapping property}:
  \begin{displayquote}
    For every abelian group \( H \), every group homomorphism \( \varphi: G \to H \) \hyperref[def:factors_through]{uniquely factors through} \( G / [G, G] \). That is, there exists a unique group homomorphism \( \widetilde{\varphi}: G / [G, G] \to H \) such that the following diagram commutes:
    \begin{equation}\label{eq:thm:group_abelianization_universal_property/diagram}
      \begin{aligned}
        \includegraphics[page=1]{output/thm__group_abelianization_universal_property.pdf}
      \end{aligned}
    \end{equation}
  \end{displayquote}

  Via \fullref{rem:universal_mapping_property}, the abelianization functor becomes \hyperref[def:category_adjunction]{left adjoint} to the \hyperref[def:concrete_category]{forgetful functor}
  \begin{equation*}
    U: \cat{Ab} \to \cat{Grp}.
  \end{equation*}

  This result extends to \fullref{thm:ring_abelianization_universal_property}.
\end{theorem}
\begin{proof}
  Let \( C \coloneqq [G, G] \).

  \SubProof{Proof that \( G / C \) is abelian} Normality of \( G / C \) easily follows from
  \begin{equation*}
    a xyx^{-1}y^{-1} a^{-1}
    =
    (a x a^{-1}) (a y a^{-1}) (a x a^{-1})^{-1} (a y a^{-1})^{-1}.
  \end{equation*}

  Then for the cosets \( a C \) and \( b C \), we have
  \begin{equation*}
    a C \cdot b C
    =
    a b C
    =
    a b (b^{-1} a^{-1} b a) C
    =
    b a C.
  \end{equation*}

  Therefore, the quotient group \( G / C \) is abelian.

  \SubProof{Proof of universal mapping property} Let \( H \) be an abelian group and let \( \varphi: G \to H \) be a group homomorphism.

  Observe that \( \varphi(C) = e_H \). Indeed, since \( H \) is abelian, for \( [x, y] = xyx^{-1}y^{-1} \in C \) we have
  \begin{equation*}
    \varphi([x, y]) = \varphi(x) \varphi(y) \varphi(x^{-1}) \varphi(y^{-1}) = \varphi(x) \varphi(x^{-1}) \varphi(y) \varphi(y^{-1}).
  \end{equation*}

  We want \( \overline{\varphi}: G / C \to H \) to satisfy
  \begin{equation*}
    \overline{\varphi}(\underbrace{\pi_G(x)}_{xC}) = \varphi(x).
  \end{equation*}

  This suggests the definition
  \begin{equation*}
    \overline{\varphi}(xC) \coloneqq \varphi(x).
  \end{equation*}

  It is well-defined because if \( xC = yC \), we have
  \begin{equation*}
    \varphi(x)
    =
    \varphi(x) e_H
    =
    \varphi(x) \varphi(C)
    =
    \varphi(x C)
    =
    \varphi(y C)
    =
    \ldots
    =
    \varphi(y).
  \end{equation*}
\end{proof}
