\subsection{Algebras over semirings}\label{subsec:algebras_over_semirings}

\paragraph{Algebras over semirings}

\begin{definition}\label{def:multilinear_function}\mimprovised
  If \( M_1, \ldots, M_n \) and \( N \) are \( R \)-\hyperref[def:semimodule]{(semi)modules}, we say that the function
  \begin{equation*}
    f: M_1 \times \ldots \times M_n \to N
  \end{equation*}
  is \term{multilinear} (\term{bilinear} for \( n = 2 \)) if it is linear in each component --- for every tuple
  \begin{equation*}
    (x_1, \ldots, x_n) \in M_1 \times \cdots \times M_n,
  \end{equation*}
  and for every index \( k = 1, \ldots, n \), the following function is linear:
  \begin{equation*}
    y \mapsto f(x_1, \ldots, x_{k-1}, y, x_{k+1}, \ldots, x_n)
  \end{equation*}
\end{definition}
\begin{comments}
  \item We generalize terminology traditionally used for \hyperref[def:semimodule/homomorphism]{linear maps} --- for example by \incite[66]{Knapp2016BasicAlgebra}, \incite[522]{Aluffi2009} or \incite[239]{Treil2017}.
\end{comments}

\begin{definition}\label{def:algebra_over_semiring}\mimprovised
  An \term{algebra} over a \hyperref[def:semiring/commutative]{commutative (semi)ring} \( R \) is an \( R \)-\hyperref[def:semimodule]{semimodule} \( M \) with an \hyperref[def:binary_operation/associative]{associative} \hyperref[def:multilinear_function]{bilinear} vector multiplication operation. This makes \( M \) a nonunital ring. By default, we will also assume that \( M \) has a multiplicative unit, although nonunital algebras as just as valid as nonunital rings.

  We identify every element \( t \) of \( R \) with its canonical embedding \( t \cdot 1_M \) in \( M \), and thus we can also regard \( R \) as a sub-semiring of \( M \).

  Algebras have the following metamathematical properties:
  \begin{thmenum}
    \thmitem{def:algebra_over_semiring/theory} The \hyperref[def:first_order_theory]{first-order theory} for algebras extends the \hyperref[def:semimodule/theory]{theory of semimodules}. We add a new \hyperref[rem:first_order_formula_conventions/infix]{infix} binary function symbol \( \odot \) to the language, and add to the theory all semiring axioms from \fullref{def:semiring/theory} for \( + \) and \( \odot \). We must also add axioms ensuring that \( \odot \) is bilinear. Additivity follows from distributivity, hence it remains to account for homogeneity. Using the notation of \fullref{def:semimodule/theory}, this amounts to the following axiom schemas:
    \begin{align*}
      m_r(x) \odot y &= m_r(x \odot y), \\
      x \odot m_r(y) &= m_r(x \odot y).
    \end{align*}

    \thmitem{def:algebra_over_semiring/homomorphism} A \hyperref[def:first_order_homomorphism]{first-order homomorphism} between two \( R \)-algebras \( M \) and \( N \) is a linear map that also preserves vector multiplication.

    \thmitem{def:algebra_over_semiring/submodel} The set \( A \subseteq M \) is a \hyperref[def:first_order_submodel]{submodel} of \( M \) if it is a \hyperref[def:monoid/submodel]{submodule} of \( M \) that is closed under algebra multiplication. We say that \( A \) is a \term{subalgebra}.

    As for general submodules, \fullref{rem:span_over_different_semirings} shows how it is important to be unambiguous about over which semiring we consider the subalgebra.

    As a consequence of \fullref{thm:positive_formulas_preserved_under_homomorphism}, the image of an \( R \)-algebra homomorphism is a subalgebra of its codomain.

    \thmitem{def:algebra_over_semiring/generated} For an arbitrary set \( A \), we denote the \hyperref[def:first_order_generated_substructure]{generated submodel} by \( \linspan{ A } \).

    \thmitem{def:algebra_over_semiring/category} We denote the corresponding \hyperref[def:category_of_small_first_order_models]{category of \( \mscrU \)-small models} by \( \ucat{Alg}_R \).

    \thmitem{def:algebra_over_semiring/commutative} As in the case of general (semi)rings, by \enquote{\( M \) is commutative}, we will mean that vector multiplication is commutative.

    We denote the subcategory of commutative algebras by \( \cat{CAlg}_R \).
  \end{thmenum}
\end{definition}

\begin{proposition}\label{thm:semiring_is_algebra}
  Every \hyperref[def:semiring]{semiring} \( R \) is an \( R \)-\hyperref[def:algebra_over_semiring]{algebra} with both scalar and vector multiplication given by the multiplication in \( R \).
\end{proposition}
\begin{comments}
  \item This result extends \fullref{thm:commutative_monoid_is_semimodule} for \( R \)-semimodules.
\end{comments}
\begin{proof}
  Follows from \fullref{thm:commutative_monoid_is_bisemimodule} by noting that bilinearity follows from left distributivity in \( R \).
\end{proof}

\begin{proposition}\label{thm:semiring_is_natural_number_algebra}
  The categories \( \hyperref[def:semiring/category]{\cat{SRing}} \) of semirings and \( \hyperref[def:algebra_over_semiring/category]{\cat{Alg}_\BbbN} \) of natural number algebras are \hyperref[rem:category_similarity/isomorphism]{isomorphic}.
\end{proposition}
\begin{comments}
  \item Compare this result to \fullref{thm:commutative_monoid_is_semimodule} for semimodules and \fullref{thm:ring_is_integer_algebra} for algebras over rings.
\end{comments}
\begin{proof}
  Follows from \fullref{thm:commutative_monoid_is_semimodule} by noting that, as in our proof of \fullref{thm:semiring_is_algebra}, distributivity implies bilinearity.
\end{proof}

\begin{proposition}\label{thm:functions_over_algebra}
  For a \hyperref[def:set]{plain set} \( A \) and an \( R \)-\hyperref[def:algebra_over_semiring]{algebra} \( N \), the set of all functions from \( A \) to \( N \) is itself an \( R \)-algebra with the following operations:
  \begin{thmenum}
    \thmitem{thm:functions_over_algebra/addition} Pointwise addition
    \begin{equation*}
      [f + g](x) \coloneqq f(x) + g(x)
    \end{equation*}

    \thmitem{thm:functions_over_algebra/scalar_multiplication} Pointwise scalar multiplication
    \begin{equation*}
      [t \cdot f](x) \coloneqq t \cdot f(x)
    \end{equation*}

    \thmitem{thm:functions_over_algebra/vector_multiplication} Pointwise vector multiplication
    \begin{equation*}
      [f \odot g](x) \coloneqq f(x) \cdot g(x)
    \end{equation*}

    In practice, we use juxtaposition \( fg \) or \( f \cdot g \) instead of \( f \odot g \).
  \end{thmenum}

  If \( A \) is also an \( R \)-algebra, we denote the set of all \( R \)-\hyperref[def:algebra_over_semiring/homomorphism]{algebra homomorphisms} by \( \hom(A, N) \).
\end{proposition}
\begin{proof}
  By \fullref{thm:functions_over_model_form_model}, \( N \) is both an \( R \)-semiring and an \( R \)-semimodule. Compatibility comes from left distributivity in \( N \).
\end{proof}

\paragraph{Semigroup algebras}
\begin{definition}\label{def:semigroup_algebra}\mcite[29]{Golan2010}
  Fix a \hyperref[def:semigroup]{semigroup} \( G \) and an unital \hyperref[def:semiring]{semiring} \( R \).

  We define the corresponding \term[ru=групповая алгебра (\cite[504]{Винберг2014})]{semiring algebra} as the \hyperref[def:free_semimodule]{free semimodule} endowed with a vector multiplication operation \( \odot \) called \term[ru=свёртка (\cite[62]{Боровков1999Вероятности}) / конволюция (\cite[179]{ИоффеТихомиров1974})]{convolution product} and defined as follows:
  \begin{equation}\label{eq:def:semigroup_algebra/multiplication}
    \parens[\Big]{ \Bbbsum_{x \in G} a_x \cdot x } \odot \parens[\Big]{ \Bbbsum_{y \in G} b_y \cdot y }
    \coloneqq
    \Bbbsum_{g \in G} \parens[\Big]{ \sum_{xy = g} a_x b_y } \cdot g.
  \end{equation}

  We also define the embedding
  \begin{equation}\label{eq:def:semigroup_algebra/embedding}
    \begin{aligned}
      &\iota: G \to R[G] \\
      &\iota(g) \coloneqq 1_R \cdot g.
    \end{aligned}
  \end{equation}
\end{definition}
\begin{comments}
  \item More simply put, the coefficient before \( g \) in the product is the sum of products \( a_x b_y \) where \( xy = g \). This definition is justified in our proof of \fullref{thm:semigroup_algebra_universal_property}.

  \item Although we will often use the same symbols for all the operations, it is important to highlight that this definition uses six distinct operations --- four for multiplication and two for addition. In \eqref{eq:def:semigroup_algebra/multiplication} we have used juxtaposition for both the semigroup and the semiring multiplication, \( \cdot \) for the scalar multiplication and \( \odot \) for vector multiplication. The (semi)ring summation denoted via \( \sum \) and the (semi)module summation denoted via \( \Bbbsum \).
\end{comments}

\begin{theorem}[Semigroup algebra universal property]\label{thm:semigroup_algebra_universal_property}\mcite[381]{Knapp2016BasicAlgebra}
  The \hyperref[def:semigroup_algebra]{semigroup algebra} \( R[G] \) is the unique up to a unique isomorphism \hyperref[def:algebra_over_semiring]{algebra} over the \hyperref[def:semiring]{semiring} \( R \) that satisfies the following \hyperref[rem:universal_mapping_property]{universal mapping property}:
  \begin{displayquote}
    For every \( R \)-algebra \( M \) and every \hyperref[def:semigroup/homomorphism]{semigroup homomorphism} \( e: G \to M \) into the multiplicative group of \( M \), there exists a unique homomorphism of \( R \)-algebras \( e: R[G] \to M \) such that the following diagram commutes:
    \begin{equation}\label{eq:thm:semigroup_algebra_universal_property}
      \begin{aligned}
        \includegraphics[page=1]{output/thm__semigroup_algebra_universal_property}
      \end{aligned}
    \end{equation}
  \end{displayquote}
\end{theorem}
\begin{comments}
  \item Via \fullref{rem:universal_mapping_property}, \( G \mapsto R[G] \) becomes \hyperref[def:category_adjunction]{left adjoint} to the \hyperref[def:concrete_category]{forgetful functor} which sends \( R \)-algebras to their multiplicative group.
\end{comments}
\begin{proof}
  \Fullref{thm:free_semimodule_universal_property} implies that the following is an \( R \)-module homomorphism:
  \begin{equation*}
    \begin{aligned}
      &\Phi: R[G] \to M, \\
      &\Phi\parens[\Big]{ \Bbbsum_{x \in G} a_x \cdot x } \coloneqq \sum_{x \in G} a_x \cdot_M \varphi(x).
    \end{aligned}
  \end{equation*}

  We will show that \( \Phi \) preserves products, that is, that \( \Phi \) also a homomorphism of \( R \)-algebras.

  Fix some vector \( \Bbbsum_{y \in G} b_y \cdot y \). We will use induction on the number of nonzero coefficients in this sum to show that
  \begin{equation}\label{eq:thm:semigroup_algebra_universal_property/hypothesis}
    \Phi\parens[\Big]{ \Bbbsum_{x \in G} a_x \cdot x } \odot_M \Phi\parens[\Big]{ \Bbbsum_{y \in G} b_y \cdot y }
    =
    \Bbbsum_{g \in G} \parens[\Big]{ \sum_{x y = g} a_x b_y } \cdot_M \varphi(g).
  \end{equation}

  \begin{itemize}
    \item If \( b_y = 0 \) for every \( y \in G \), then \( a_x b_y = 0 \) and thus the product \eqref{eq:def:semigroup_algebra/multiplication} with any other vector is necessarily zero. Thus,
    \begin{equation*}
      \Phi\parens[\Big]{ \Bbbsum_{x \in G} a_x \cdot x } \odot_M \Phi\parens[\Big]{ \Bbbsum_{y \in G} \underbrace{b_y}_{\mathclap{\T*{zero}}} \cdot y }
      =
      \Bbbsum_{g \in G} \underbrace{\parens[\Big]{ \sum_{x y = g} a_x b_y }}_{\T{zero}} \cdot_M \varphi(g).
    \end{equation*}

    \item Otherwise, suppose that \eqref{eq:thm:semigroup_algebra_universal_property/hypothesis} holds for \( m \) nonzero coefficients and consider a vector
    \begin{equation*}
      \sum_{i=1}^{m+1} b_{y_i} \cdot y_i.
    \end{equation*}

    Since \( \odot_M \) is additive in its second argument, we have
    \begin{align}
      &\phantom{{}={}}
      \Phi\parens[\Big]{ \Bbbsum_{x \in G} a_x \cdot x } \odot_M \Phi\parens[\Big]{ \Bbbsum_{i=1}^{m+1} b_{y_i} \cdot y_i }
      = \nonumber \\ &=
      \Phi\parens[\Big]{ \Bbbsum_{x \in G} a_x \cdot x } \odot_M \Phi\parens[\Big]{ \Bbbsum_{i=1}^m b_{y_i} \cdot y_i } + \Phi\parens[\Big]{ \Bbbsum_{x \in G} a_x \cdot x } \odot_M \Phi\parens[\Big]{ b_{y_{m+1}} \cdot y_{m+1} }. \label{eq:thm:semigroup_algebra_universal_property/split}
    \end{align}

    We will study the second term. Because \( \odot_M \) is homogeneous in both arguments, and since \( \Phi \) is also homogeneous, we have
    \begin{align*}
      &\phantom{{}={}}
      \Phi\parens[\Big]{ \Bbbsum_{x \in G} a_x \cdot x } \odot_M \Phi\parens[\Big]{ b_{y_{m+1}} \cdot y_{m+1} }
      = \\ &=
      b_{y_{m+1}} \cdot \bracks[\Bigg]{ \Phi\parens[\Big]{ \Bbbsum_{x \in G} a_x \cdot x } \odot_M \Phi\parens[\Big]{ y_{m+1} } }
      = \\ &=
      \Phi\parens[\Big]{ \Bbbsum_{x \in G} a_x b_{y_{m+1}} \cdot x } \odot_M \Phi\parens[\Big]{ y_{m+1} }
      = \\ &=
      \Bbbsum_{x \in G} a_x b_{y_{m+1}} \cdot_M \underbrace{\varphi(x) \odot_M \varphi(y_{m+1})}_{\varphi(xy_{m+1})}.
    \end{align*}

    The last term can be rewritten by grouping coefficients before the same vectors:
    \begin{equation}\label{eq:thm:semigroup_algebra_universal_property/second}
      \Bbbsum_{g \in G} \parens[\Big]{ \sum_{xy_{m+1} = g} a_x b_{y_{m+1}} } \cdot_M \varphi(g).
    \end{equation}

    Substituting the second term in \eqref{eq:thm:semigroup_algebra_universal_property/split} for \eqref{eq:thm:semigroup_algebra_universal_property/second} and using the inductive hypothesis on the first term, we obtain
    \begin{align*}
      &\phantom{{}={}}
      \Phi\parens[\Big]{ \Bbbsum_{x \in G} a_x \cdot x } \odot_M \Phi\parens[\Big]{ \Bbbsum_{i=1}^m b_{y_i} \cdot y_i } + \Phi\parens[\Big]{ \Bbbsum_{x \in G} a_x \cdot x } \odot_M \Phi\parens[\Big]{ b_{y_{m+1}} \cdot y_{m+1} }
      = \\ &=
      \Bbbsum_{g \in G} \parens[\Big]{ \sum_{\substack{a y_i = g \\ 1 \leq y \leq m}} a_x b_{y_i} } \cdot_M \varphi(g) + \Bbbsum_{g \in G} \parens[\Big]{ \sum_{xy_{m+1} = g} a_x b_{y_{m+1}} } \cdot_M \varphi(g)
      = \\ &=
      \Bbbsum_{g \in G} \parens[\Big]{ \sum_{\substack{a y_i = g \\ 1 \leq y \leq m}} a_x b_{y_i} + \sum_{xy_{m+1} = g} a_x b_{y_{m+1}} } \cdot_M \varphi(g)
      = \\ &=
      \Bbbsum_{g \in G} \parens[\Big]{ \sum_{\substack{a y_i = g \\ 1 \leq y \leq m + 1}} a_x b_{y_i} } \cdot_M \varphi(g)
      = \\ &=
      \Bbbsum_{g \in G} \parens[\Big]{ \sum_{x y = g} a_x b_y } \cdot_M \varphi(g).
    \end{align*}
  \end{itemize}

  This completes the induction. We conclude the \eqref{eq:thm:semigroup_algebra_universal_property/hypothesis} holds no matter how many nonzero coefficients are there in the sum
  \begin{equation*}
    \Bbbsum_{y \in G} b_y \cdot y.
  \end{equation*}

  This in turn implies that \( \Phi \) preserves multiplication, making it a homomorphism of \( R \)-algebras.
\end{proof}

\begin{example}\label{ex:def:semigroup_algebra}
  We list examples of \hyperref[def:semigroup_algebra]{semigroup algebras}:
  \begin{thmenum}
    \thmitem{ex:def:semigroup_algebra/zero} If the semigroup is empty, the semimodule has only one element.

    \thmitem{ex:def:semigroup_algebra/one} If the semigroup has one element \( e \), then \( R[\set{ e }] \) is isomorphic as an algebra to \( R \).

    Indeed, the multiplication operation \eqref{eq:def:semigroup_algebra/multiplication} reduces to
    \begin{equation*}
      (a \cdot e) \odot (b \cdot e) = ab \cdot e.
    \end{equation*}

    The semimodule axiom \eqref{eq:def:semimodule/operation/scalar_addition_distributivity} implies that vector addition also satisfies a similar property:
    \begin{equation*}
      (a \cdot e) \oplus (b \cdot e) = (a + b) \cdot e.
    \end{equation*}

    Lastly, \eqref{eq:def:semimodule/operation/scalar_multiplication_action/compatibility} implies that scalar multiplication satisfies
    \begin{equation*}
      b \cdot (a \cdot e) = ba \cdot e.
    \end{equation*}

    \thmitem{ex:def:semigroup_algebra/polynomial} The group algebra of the \hyperref[def:semimodule_direct_sum]{direct sum} \( \BbbN^n \) over \( R \) is the \hyperref[def:polynomial_algebra]{polynomial algebra} in \( n \) indeterminates over \( R \).

    \thmitem{ex:def:semigroup_algebra/laurent} The group algebra of \( \BbbZ \) over \( R \) is the \hyperref[def:ring_of_laurent_polynomials]{algebra of Laurent polynomials} over \( R \).

    \thmitem{ex:def:semigroup_algebra/quaternions} The \hyperref[def:quaternion_algebra]{quaternion algebra} is a quotient of a semiring group of a free monoid.
  \end{thmenum}
\end{example}

\paragraph{Polynomial algebras}

\begin{definition}\label{def:polynomial_algebra}\mimprovised
  Fix a \hyperref[def:semiring/commutative]{commutative semiring} \( R \) and a set \( \mscrX \) of \hyperref[def:formal_language/symbol]{symbols}, which we will call \term{indeterminates}.

  \begin{thmenum}
    \thmitem{def:polynomial_algebra/monomials} Consider the \hyperref[def:free_commutative_monoid]{multiplicatively-written} \hyperref[def:free_commutative_monoid]{free commutative monoid} \( \BbbN^{\oplus \mscrX} \).

    As discussed in \fullref{rem:free_commutative_monoid_as_quotient}, we can regard elements of this monoid as congruence classes of strings of indeterminates. We call these congruence classes \term[ru=одночлен/моном (\cite[sec. 11.3]{Тыртышников2007})]{monomials}.

    We denote monomials as strings of indeterminates, e.g. \( XY \) or \( YX \), however regard them as their corresponding congruence classes and consider congruent strings to be equal.

    \thmitem{def:polynomial_algebra/polynomials} We define the \term[ru=алгебра многочленов (\cite[92]{Винберг2014})]{polynomial algebra} \( R[\mscrX] \) as the \hyperref[def:semigroup_algebra]{monoid semiring} \( R[\BbbN^{\oplus \mscrX}] \).

    If \( \mscrX = \set{ X_1, \ldots, X_n } \), we also write \( R[X_1, \ldots, X_n] \).

    We call elements of \( R[\mscrX] \) \term[bg=полиноми (\cite[22]{ГеновМиховскиМоллов1991}), ru=многочлены (\cite[92]{Винберг2014}; многочлены/полиномы (\cite[sec. 11.3]{Тыртышников2007})]{polynomials}. A polynomial is thus a linear combination of monomials with the \hyperref[def:semigroup_algebra]{convolution product}.

    \thmitem{def:polynomial_algebra/embedding} Thus, we have the following chain of canonical maps:
    \begin{equation*}
      \mscrX \to \BbbN^{\oplus \mscrX} \to R[\mscrX].
    \end{equation*}

    The first map is a plain function and the second one is a group homomorphism from the (multiplicatively written) \hyperref[def:free_commutative_monoid]{free commutative monoid} \( \BbbN^{\oplus \mscrX} \) of monomials into the multiplicative group of the \( R \)-algebra \( R[\mscrX] \) of polynomials.

    Whenever we need to be explicit, like in \fullref{thm:polynomial_algebra_universal_property}, we denote their composition by \( \iota: \mscrX \to R[\mscrX] \).
  \end{thmenum}
\end{definition}

\begin{theorem}[Polynomial algebra universal property]\label{thm:polynomial_algebra_universal_property}
  Fix a \hyperref[def:semiring/commutative]{commutative semiring} \( R \) and a set \( \mscrX \) of indeterminates. The \hyperref[def:polynomial_algebra]{polynomial algebra} \( R[\mscrX] \) is the unique up to a unique isomorphism \hyperref[def:algebra_over_semiring/commutative]{commutative \( R \)-algebra} that satisfies the following \hyperref[rem:universal_mapping_property]{universal mapping property}:
  \begin{displayquote}
    For every commutative \( R \)-algebra \( M \) and every function \( e: \mscrX \to M \), there exists a unique \( R \)-algebra homomorphism \( \Phi_e: R[\mscrX] \to M \) such that the following diagram commutes:
    \begin{equation}\label{eq:thm:polynomial_algebra_universal_property/diagram}
      \begin{aligned}
        \includegraphics[page=1]{output/thm__polynomial_algebra_universal_property}
      \end{aligned}
    \end{equation}
  \end{displayquote}
\end{theorem}
\begin{comments}
  \item Via \fullref{rem:universal_mapping_property}, \( R[\anon*] \) becomes \hyperref[def:category_adjunction]{left adjoint} to the \hyperref[def:concrete_category]{forgetful functor}
  \begin{equation*}
    U: \cat{CAlg}_R \to \cat{Set}.
  \end{equation*}

  The action of \( R[\anon*] \) on morphisms is given by \( \Phi \).
\end{comments}
\begin{proof}
  \Fullref{thm:free_semimodule_universal_property} gives us a unique monoid homomorphism \( \widehat{e}: \BbbN^{\oplus \mscrX} \to M \) such that \( \widehat{e}(X) = e(X) \) for every indeterminate \( X \).

  \Fullref{thm:semigroup_algebra_universal_property} gives us a unique homomorphism of \( R \)-algebras \( \Phi_e: R[\mscrX] \to M \) such that, for every indeterminate \( X \),
  \begin{equation*}
    \Phi_e(X) = \widehat{e}(X) = e(X).
  \end{equation*}
\end{proof}

\begin{remark}\label{rem:substitution_homomorphism}
  In \fullref{thm:polynomial_algebra_universal_property}, the function \( e \) evaluates each indeterminate from \( \mscrX \) within \( M \), while \( \Phi_e \) substitutes this value in the polynomials from \( R[\mscrX] \). We call \( \Phi_e \) the \term{substitution homomorphism} or \term{evaluation homomorphism} corresponding to the \term{variable assignment} \( e \). We can parameterize this by the evaluation functions to obtain the \enquote{functional substitution homomorphism}
  \begin{equation*}
    \begin{aligned}
      &\Phi: R[\mscrX] \to \fun(\fun(\mscrX, M), M), \\
      &\Phi(p) \coloneqq (e \mapsto \Phi_e(p)).
    \end{aligned}
  \end{equation*}

  To each polynomial \( p \) in \( R[\mscrX] \), the homomorphism \( \Phi \) sends \( p \) to a function that, given a value in \( M \) for each indeterminate from \( \mscrX \), evaluates \( p \) in \( M \). We call \( \Phi(p) \) the \term{polynomial function} corresponding to \( p \).

  In the case of \( n \) indeterminates, \( \Phi(p) \) is a function from \( M^n \) to \( M \).

  The precise relationship between polynomials and polynomial functions in \hyperref[thm:finite_fields]{finite fields} are discussed in \fullref{thm:functions_over_prime_fields}.
\end{remark}

\begin{remark}\label{rem:polynomials_over_infinitely_many_indeterminates}
  As we saw in \fullref{def:polynomial_algebra} and \fullref{thm:polynomial_algebra_universal_property}, it is possible to define polynomial algebras over infinitely many indeterminates.

  There is a problem in practice, however. Polynomials in one indeterminate, which we will call univariate based on \fullref{def:operation_arity}, have a \hyperref[def:well_ordered_set]{well-ordering} on their monomials, induced by the degree of their monomials. This is defined and discussed in \fullref{subsec:univariate_polynomials}.

  Polynomials in more than one variable do not have a well-ordering by default. If the indeterminates themselves are well-ordered, as is the case for finitely many indeterminates, we may introduce, for example, a \hyperref[def:lexicographic_order]{reverse lexicographic order} on the monomials. Furthermore, for finitely many variables, \fullref{thm:def:polynomial_algebra/iterated} allows us to use \hyperref[rem:induction/peano_arithmetic]{natural number induction} on the number of variables in order to prove statements about multivariate polynomial rings.

  For infinitely many, especially uncountably many variables, however, the theory is seriously crippled by the lack of the tools described above. For this reason, only polynomials in finitely many variables are often considered.
\end{remark}

\begin{proposition}\label{thm:def:polynomial_algebra}
  The following are basic properties of \hyperref[def:polynomial_algebra]{polynomial algebras}:
  \begin{thmenum}
    \thmitem{thm:def:polynomial_algebra/empty} If \( \mscrX \) is empty, \( R[\mscrX] \cong R \).

    \thmitem{thm:def:polynomial_algebra/iterated} If \( \mscrX \) is nonempty, the polynomial algebras \( R[\mscrX] \) and \( R[\mscrX \setminus \set{ X_0 }][X_0] \) are isomorphic for any \( X_0 \in \mscrX \).

    In particular,
    \begin{equation*}
      R[X_1, \ldots, X_{n-1}][X_n] \cong R[X_1, \ldots, X_n].
    \end{equation*}

    \thmitem{thm:def:polynomial_algebra/entire} The univariate \hyperref[def:polynomial_algebra]{polynomial semiring} \( R[X] \) is \hyperref[def:entire_semiring]{entire} if and only if \( R \) is entire.

    \thmitem{thm:def:polynomial_algebra/units} If \( R \) is entire, the \hyperref[def:divisibility/unit]{units} in \( R[X_1, \ldots, X_n] \) are precisely the (embeddings of) the units of \( R \).
  \end{thmenum}
\end{proposition}
\begin{proof}
  \SubProofOf{thm:def:polynomial_algebra/empty} Trivial.

  \SubProofOf{thm:def:polynomial_algebra/iterated} Polynomials in \( R[\mscrX] \) have the form
  \begin{equation*}
    p(\mscrX) = \sum_{X_1 \ldots X_m} a_{X_1 \ldots X_m} X_1 \ldots X_m,
  \end{equation*}
  where \( X_1 \ldots X_m \) is a monomial over \( \mscrX \).

  Due to distributivity, this can be rewritten as
  \begin{equation*}
    \widehat{p}(\mscrX) = \sum_{k=0}^n \parens[\Big]{ \sum_{X_0^k X_1 \ldots X_m} a_{X_0^k X_1 \ldots X_m} X_1 \ldots X_m } X_0^k.
  \end{equation*}

  This shows how every polynomial \( p \) in \( R[\mscrX] \) can be regarded as a polynomial \( \widehat{p} \) in \( X_0 \) over the ring \( R[\mscrX \setminus \set{ X_0 }] \).

  This map \( p \mapsto \widehat{p} \) is clearly an isomorphism of \( R \)-algebras.

  \SubProofOf{thm:def:polynomial_algebra/entire}

  \SufficiencySubProof Since \( R \) is an \( R \)-subalgebra of \( R[X] \), if the latter is entire, so is the former.

  \NecessitySubProof Suppose that \( R \) is entire and that \( R[X] \) isn't. Then there exist nonzero polynomials \( p(X) \) and \( q(X) \) such that \( p(X) q(X) = 0 \). If \( a_n \) is the leading coefficient (in the sense of \fullref{def:extremal_points/maximal_and_minimal_element}) of \( p(X) \) and \( b_m \) --- of \( q(X) \), the leading coefficient of \( p(X) q(X) \) is \( a_n b_m \). Since \( p(X) q(X) \) is the zero polynomial, \( a_n b_m = 0 \), which contradicts the assumption that \( R \) is entire.

  Therefore, \( R[X] \) is entire.

  \SubProofOf{thm:def:polynomial_algebra/units} As in \fullref{thm:def:polynomial_algebra/entire}, it is sufficient to prove the statement for one indeterminate.

  Clearly every constant is invertible as a constant polynomial.

  Now suppose that \( p(X) q(X) = 1 \). By definition of multiplication, the product has only one nonzero coefficient. Since \( R \) is entire, it follows that both \( p(X) \) and \( q(X) \) have only one nonzero coefficient, and are hence constants.
\end{proof}

\begin{example}\label{ex:def:polynomial_algebra}
  We list several examples of \hyperref[def:polynomial_algebra]{polynomials} over semirings.
  \begin{thmenum}
    \thmitem{ex:def:polynomial_algebra/natural_numbers} Consider the polynomial \( p(X) \coloneqq aX^2 + bX + c \) in \( \BbbN[X] \). A function from the set \( \set{ X } \) to \( \BbbN \) corresponds to an element of \( \BbbN \), and hence evaluating the polynomial is done by simply replacing \( X \) symbolically in \( p \) and then evaluating the obtained \hyperref[rem:binary_operation_syntax_trees]{syntax tree}.

    We seek the roots of \( p(X) \). We will only formally define roots in \fullref{def:polynomial_root}; for the purposes of the example, a root is a natural number \( n \) such that \( \Phi_n(p) = 0_R \).

    By \fullref{thm:fundamental_theorem_of_algebra} and \fullref{def:algebraically_closed_field/exactly_n_roots}, \( p \) has two roots in the \hyperref[def:complex_numbers]{complex plane} --- we regard \( \BbbC \) as an algebra over \( \BbbN \) and use \fullref{thm:polynomial_algebra_universal_property} to obtain a polynomial function on \( \BbbC \). Furthermore, over the complex numbers the roots can be explicitly found using
    \begin{equation*}
      \frac {-b \pm \sqrt{b^2 - 4ac}} {2a}.
    \end{equation*}

    Finding a root of \( p \) over the natural numbers cannot be done in general, however. If \( p(n) = 0 \), by the ordering of the natural numbers we have
    \begin{equation*}
      p(n) = an^2 + bn + c \geq c,
    \end{equation*}
    and hence \( c \) must necessarily be \( 0 \). If \( c = 0 \), then zero is a root of the polynomial \( p(X) = aX^2 + bX \).

    Now let \( n \) be any root of \( p \). We have
    \begin{equation*}
      an^2 + bn \geq bn,
    \end{equation*}
    and hence \( bn \) must also be \( 0 \). Thus, either \( b = 0 \) or \( n = 0 \). If we want a root other than \( n \), both \( a \) and \( b \) must be \( 0 \).

    Therefore, the only natural number solution to the general quadratic equation is \( 0 \), and it is only a solution if \( c = 0 \).

    \thmitem{ex:def:polynomial_algebra/tropical} Consider again the polynomial \( p(X) \coloneqq aX^2 + bX + c \) over \( \BbbN \), but this time evaluated over the \hyperref[def:tropical_semiring]{\( \min \)-plus semiring} \( (\BbbN \cup \set{ \infty }, \min, +) \).

    Expressed via the standard natural number operations, this polynomial becomes
    \begin{equation*}
      \min\set{ 2X + a, X + b, c }.
    \end{equation*}

    This allows us to express certain optimization problems via polynomials.

    This polynomial has a root if and only if \( a = b = c \). Roots in the tropical semiring are not very interesting, however.
  \end{thmenum}
\end{example}

\begin{definition}\label{def:formal_power_series}\mcite[196]{Salomaa1987}
  If we extend the concept of \hyperref[def:polynomial_algebra]{polynomials} to allow countably many monomials with nonzero coefficients, we obtain \( R\Bracks{\mscrX} \), which we call the set of \term[ru=формальные степянные ряды (\cite[28]{Шафаревич1999})]{formal power series} over \( R \) with indeterminates \( \mscrX \).
\end{definition}
\begin{comments}
  \item The evaluation homomorphism defined in \fullref{thm:polynomial_algebra_universal_property} is problematic since algebraic operations are finitary by nature. This is discussed in \fullref{rem:binary_operation_syntax_trees/infinite}, along with how sometimes we can make sense of infinitary algebraic operations.
\end{comments}

\paragraph{Adjoining elements via polynomial evaluation}

\begin{proposition}\label{thm:generators_via_polynomials}
  For a set \( A \) in an \( R \)-\hyperref[def:algebra_over_semiring]{algebra} \( M \), the \( R \)-subalgebra \hyperref[def:algebra_over_semiring/generated]{generated} by \( A \) coincides with the set
  \begin{equation*}
    \bigcup \set[\Big]{ R[a_1, \ldots, a_n] \given* a_1, \ldots, a_n \in A }
  \end{equation*}
  obtained by evaluating all multivariate polynomials over \( R \) with elements of \( A \).
\end{proposition}
\begin{comments}
  \item The \( \BbbN \)-subalgebras of \( M \) correspond to \hyperref[def:semiring/submodel]{sub-semirings} and the \( M \)-subalgebras correspond to \hyperref[def:semiring_ideal]{ideals}.

  \item Compare this result to \fullref{thm:span_via_linear_combinations} for linear spans in (semi)modules.
\end{comments}
\begin{proof}
  Similar to \fullref{thm:span_via_linear_combinations}.
\end{proof}

\begin{proposition}\label{thm:adjoining_elements_to_semiring}
  Let \( R \subseteq S \) be \hyperref[def:semiring/commutative]{commutative semirings} and let \( A \subseteq S \) be an arbitrary subset.

  Fix a set \( \mscrX \) of indeterminates and a bijective function \( e: \mscrX \to A \) and consider the \hyperref[rem:substitution_homomorphism]{evaluation homomorphism}
  \begin{equation*}
    \Phi_e: R[\mscrX] \to S.
  \end{equation*}

  Then the image \( R[A] \) of \( \Phi_e \) is the smallest super-semiring of \( R \) that contains \( A \).

  We say that \( R[A] \) is obtained by \term{adjoining} the elements of \( A \) to \( R \).
\end{proposition}
\begin{proof}
  Follows from \fullref{thm:generators_via_polynomials}.
\end{proof}

\begin{example}\label{ex:adjoining_root}
  Continuing \fullref{ex:def:polynomial_algebra/natural_numbers}, consider the polynomial equation
  \begin{equation*}
    X + 1 = 0.
  \end{equation*}

  It has no natural number root as a consequence of \eqref{eq:def:peano_arithmetic/PA2}.

  It does have an integer root, however, \( -1 \). We can \hyperref[thm:adjoining_elements_to_semiring]{adjoin} \( -1 \) to the semiring \( \BbbN \) to obtain the semiring \( \BbbN[-1] \). But this latter semiring is (isomorphic to) \( \BbbZ \).

  Therefore, \( \BbbZ \) is the smallest extension of \( \BbbN \) that contains a root to the polynomial \( X + 1 \).

  This example extends to the theory of \hyperref[def:transcendental_element]{transcendental and algebraic} elements of fields.
\end{example}

\begin{example}\label{ex:adjoining_polynomial}
  Consider again the polynomial algebra \( \BbbN[X] \). Rather than adjoining elements from the complex numbers, we can \enquote{adjoin} some polynomial from \( \BbbN[X] \) itself.

  For example, the algebra \( \BbbN[X^2] \) consists of polynomials of the form
  \begin{equation*}
    p(X) = \sum_{k=0}^\infty a_k X^{2k}.
  \end{equation*}

  We can further enhance this example. Consider the polynomial algebra \( \BbbN[X^2, X^3] \). We know that \( \BbbN[X, Y] \) consists of polynomials in two indeterminates, i.e.
  \begin{equation*}
    p(X, Y) = \sum_{k=0}^\infty \sum_{m=0}^\infty a_{k+m} X^k Y^m.
  \end{equation*}

  Thus, polynomials in \( \BbbN[X^2, X^3] \) have the form
  \begin{equation*}
    p(X) = \sum_{k=0}^\infty \sum_{m=0}^\infty a_{k+m} X^{2k + 3m}.
  \end{equation*}

  Note that, given some natural number \( n \) distinct from \( 1 \), we can obtain \( n = 2k + 3m \) as follows:
  \begin{itemize}
    \item If \( n \) is even, let \( a = n / 2 \) and \( b = 0 \).
    \item If \( n \) is odd, let \( a = (n - 1) / 2 - 1 \) and \( b = 1 \).
  \end{itemize}

  Such a representation is not possible for \( n = 1 \). Thus, \( \BbbN[X^2, X^3] \) contains all polynomials from \( \BbbN[X] \) whose coefficient for \( X \) is \( 0 \).
\end{example}

\paragraph{Noncommutative polynomials}

\begin{definition}\label{def:noncommutative_polynomial_algebra}\mcite[169]{Aluffi2009}
  We define the \term{noncommutative polynomial algebra} \( R\braket{ \mscrX } \) over the semiring \( R \) with \term{indeterminates} \( \mscrX \) as the \hyperref[def:semigroup_algebra]{monoid algebra} \( R[\mscrX^*] \), where \( \mscrX^* \) is the
  \hyperref[def:semigroup_algebra]{monoid algebra} over the \hyperref[def:free_monoid]{free monoid} (rather than the \hyperref[def:free_commutative_monoid]{free commutative monoid} \( \BbbN^\mscrX \)) of \( \mscrX \).
\end{definition}

\begin{theorem}[Noncommutative polynomial algebra universal property]\label{thm:noncommutative_polynomial_algebra_universal_property}
  Fix a \hyperref[def:semiring/commutative]{commutative semiring} \( R \) and a set \( \mscrX \) of indeterminates. The \hyperref[def:noncommutative_polynomial_algebra]{noncommutative polynomial algebra} \( R\braket{\mscrX} \) is the unique up to a unique isomorphism \hyperref[def:algebra_over_semiring]{\( R \)-algebra} that satisfies the following \hyperref[rem:universal_mapping_property]{universal mapping property}:
  \begin{displayquote}
    For every \( R \)-algebra \( M \) and every function \( e: \mscrX \to M \), there exists a unique \( R \)-algebra homomorphism \( \Phi_e: R\braket{\mscrX} \to M \) such that the following diagram commutes:
    \begin{equation}\label{eq:thm:noncommutative_polynomial_algebra_universal_property/diagram}
      \begin{aligned}
        \includegraphics[page=1]{output/thm__noncommutative_polynomial_algebra_universal_property}
      \end{aligned}
    \end{equation}
  \end{displayquote}
\end{theorem}
\begin{proof}
  Analogous to \fullref{thm:polynomial_algebra_universal_property}.
\end{proof}
