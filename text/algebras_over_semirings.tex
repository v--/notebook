\section{Algebras over semirings}\label{sec:algebras_over_semirings}

\paragraph{Algebras over semirings}

\begin{definition}\label{def:multilinear_function}\mcite[522]{Aluffi2009Algebra}
  If \( M_1, \ldots, M_n \) and \( N \) are \( R \)-\hyperref[def:semimodule]{(semi)modules}, we say that the function
  \begin{equation*}
    f: M_1 \times \ldots \times M_n \to N
  \end{equation*}
  is \term[bg=полилинейна (форма) (\cite[243]{ИлинСадовничиСендов1984АнализТом1}), ru=полилинейная / мультилиняйная (функция) (\cite[23]{Тыртышников2017ОсновыАлгебры})]{multilinear} (\term[en=билинейная (\cite[230]{Тыртышников2017ОсновыАлгебры})]{bilinear} when \( n = 2 \)) if it is linear in each component; that is, for every index \( k = 1, \ldots, n \), given a fixed \( (n - 1) \)-tuple
  \begin{equation*}
    (x_1, \ldots, x_{k-1}, x_{k+1}, \ldots, x_n) \in M_1 \times M_{k-1} \times M_{k+1} \cdots \times M_n,
  \end{equation*}
  the following function must be linear:
  \begin{equation*}
    y \mapsto f(x_1, \ldots, x_{k-1}, y, x_{k+1}, \ldots, x_n)
  \end{equation*}
\end{definition}

\begin{definition}\label{def:algebra_over_semiring}\mimprovised
  An \term{algebra} over a \hyperref[def:semiring/commutative]{commutative (semi)ring} \( R \) is an \( R \)-\hyperref[def:semimodule]{semimodule} \( M \) with an \hyperref[def:binary_operation/associative]{associative} \hyperref[def:multilinear_function]{bilinear} vector multiplication operation. This makes \( M \) a nonunital ring. By default, we will also assume that \( M \) has a multiplicative unit, although nonunital algebras as just as valid as nonunital rings.

  We identify every element \( t \) of \( R \) with its canonical embedding \( t \cdot 1_M \) in \( M \), and thus we can also regard \( R \) as a sub-semiring of \( M \).

  Algebras have the following metamathematical properties:
  \begin{thmenum}
    \thmitem{def:algebra_over_semiring/theory} The \hyperref[def:first_order_theory]{first-order theory} for algebras extends the \hyperref[def:semimodule/theory]{theory of semimodules}. We add a new \hyperref[def:fol_signature/notation]{infix} binary function symbol \( \ast \) to the language, and add to the theory all semiring axioms from \cref{def:semiring/theory} for \( + \) and \( \ast \). We must also add axioms ensuring that \( \ast \) is bilinear. Additivity follows from distributivity, hence it remains to account for homogeneity. Using the notation of \cref{def:semimodule/theory}, this amounts to the following axiom schemas:
    \begin{align*}
      m_r(x) \ast y &= m_r(x \ast y), \\
      x \ast m_r(y) &= m_r(x \ast y).
    \end{align*}

    \thmitem{def:algebra_over_semiring/homomorphism} A \hyperref[def:fol_homomorphism]{first-order homomorphism} between two \( R \)-algebras \( M \) and \( N \) is a linear map that also preserves vector multiplication.

    \thmitem{def:algebra_over_semiring/submodel} The set \( A \subseteq M \) is a \hyperref[def:first_order_submodel]{submodel} of \( M \) if it is a \hyperref[def:monoid/submodel]{submodule} of \( M \) that is closed under algebra multiplication. We say that \( A \) is a \term{subalgebra}.

    As for general submodules, \cref{rem:span_over_different_semirings} shows how it is important to be unambiguous about over which semiring we consider the subalgebra.

    As a consequence of \cref{thm:positive_formulas_preserved_under_homomorphism}, the image of an \( R \)-algebra homomorphism is a subalgebra of its codomain.

    \thmitem{def:algebra_over_semiring/generated} For an arbitrary set \( A \), we denote the \hyperref[def:first_order_generated_substructure]{generated submodel} by \( \lin{ A } \).

    \thmitem{def:algebra_over_semiring/category} We denote the corresponding \hyperref[def:fol_theory_model_functor/obj]{category of \( \mscrU \)-small models} by \( \ucat{Alg}_R \).

    \thmitem{def:algebra_over_semiring/commutative} As in the case of general (semi)rings, by \enquote{\( M \) is commutative}, we will mean that vector multiplication is commutative.

    We denote the subcategory of commutative algebras by \( \cat{CAlg}_R \).
  \end{thmenum}
\end{definition}
\begin{comments}
  \item We have adjusted a definition based on algebras over commutative rings discussed in \cref{def:algebra_over_ring}.
\end{comments}

\begin{proposition}\label{thm:algebra_over_subring}
  If \( S \) is (isomorphic to) a subring of \( R \), then \( R \) is an \( S \)-bimodule with scalar multiplication given by multiplication in \( R \).
\end{proposition}
\begin{comments}
  \item This result extends \cref{thm:bisemimodule_over_submonoid} for \( R \)-semimodules.
\end{comments}
\begin{proof}
  Follows from \cref{thm:bisemimodule_over_submonoid}.
\end{proof}

\begin{corollary}\label{thm:semiring_is_natural_number_algebra}
  The categories \( \hyperref[def:semiring/category]{\cat{SRing}} \) of semirings and \( \hyperref[def:algebra_over_semiring/category]{\cat{Alg}_\BbbN} \) of natural number algebras are \hyperref[rem:category_similarity/isomorphism]{isomorphic}.
\end{corollary}
\begin{comments}
  \item Compare this result to \cref{thm:commutative_monoid_is_semimodule} for semimodules and \cref{thm:ring_is_integer_algebra} for algebras over rings.
\end{comments}
\begin{proof}
  Follows from \cref{thm:commutative_monoid_is_semimodule} and \cref{thm:algebra_over_subring}.
\end{proof}

\begin{proposition}\label{thm:functions_over_algebra}
  For a \hyperref[def:set]{plain set} \( A \) and an \( R \)-\hyperref[def:algebra_over_semiring]{algebra} \( N \), the set of all functions from \( A \) to \( N \) is itself an \( R \)-algebra with the following operations:
  \begin{thmenum}
    \thmitem{thm:functions_over_algebra/addition} Pointwise addition
    \begin{equation*}
      [f + g](x) \coloneqq f(x) + g(x)
    \end{equation*}

    \thmitem{thm:functions_over_algebra/scalar_multiplication} Pointwise scalar multiplication
    \begin{equation*}
      [t \cdot f](x) \coloneqq t \cdot f(x)
    \end{equation*}

    \thmitem{thm:functions_over_algebra/vector_multiplication} Pointwise vector multiplication
    \begin{equation*}
      [f \ast g](x) \coloneqq f(x) \cdot g(x)
    \end{equation*}

    In practice, we use juxtaposition \( fg \) or \( f \cdot g \) instead of \( f \ast g \).
  \end{thmenum}
\end{proposition}
\begin{comments}
  \item If \( A \) is also an \( R \)-algebra, we denote the set of all \( R \)-\hyperref[def:algebra_over_semiring/homomorphism]{algebra homomorphisms} by \( \hom(A, N) \) in accordance with the general notation for \hyperref[def:category/morphisms]{categorical morphism sets}.
\end{comments}
\begin{proof}
  By \cref{thm:functions_over_model_form_model}, \( N \) is both an \( R \)-semiring and an \( R \)-semimodule. Compatibility comes from left distributivity in \( N \).
\end{proof}

\paragraph{Semigroup algebras}

\begin{definition}\label{def:semigroup_algebra}\mcite[29]{Golan1999Semirings}
  Fix a \hyperref[def:semigroup]{semigroup} \( G \) and an unital \hyperref[def:semiring]{semiring} \( R \).

  We define the corresponding \term[ru=групповая алгебра (\cite[504]{Винберг2014КурсАлгебры})]{semiring algebra} \( R[G] \) as the \hyperref[def:free_semimodule]{free semimodule} on \( G \) endowed with a vector multiplication operation \( \ast \) called \term[ru=свёртка (\cite[62]{Боровков1999ТеорияВероятностей}) / конволюция (\cite[179]{ИоффеТихомиров1974ЭкстремальныеЗадачи})]{convolution product} and defined as follows:
  \begin{equation}\label{eq:def:semigroup_algebra/multiplication}
    \parens[\Big]{ \Bbbsum_{x \in G} a_x \cdot x } \ast \parens[\Big]{ \Bbbsum_{y \in G} b_y \cdot y }
    \coloneqq
    \Bbbsum_{g \in G} \parens[\Big]{ \sum_{xy = g} a_x b_y } \cdot g.
  \end{equation}

  We also define the embedding
  \begin{equation}\label{eq:def:semigroup_algebra/embedding}
    \begin{aligned}
      &\iota: G \to R[G] \\
      &\iota(g) \coloneqq 1_R \cdot g.
    \end{aligned}
  \end{equation}
\end{definition}
\begin{comments}
  \item More simply put, the coefficient before \( g \) in the product is the sum of products \( a_x b_y \) where \( xy = g \). This definition is justified in our proof of \fullref{thm:semigroup_algebra_universal_property}.

  \item \Cref{ex:def:semigroup_algebra/cyclic} shows how \eqref{eq:def:semigroup_algebra/multiplication} is merely a formalization of bracket expansion with grouping.

  \item Although we will often use the same symbols for all the operations, it is important to highlight that this definition uses six distinct operations --- four for multiplication and two for addition. In \eqref{eq:def:semigroup_algebra/multiplication} we have used juxtaposition for both the semigroup and the semiring multiplication, \( \cdot \) for the scalar multiplication and \( \ast \) for vector multiplication. The (semi)ring summation denoted via \( \sum \) and the (semi)module summation denoted via \( \Bbbsum \).
\end{comments}

\begin{theorem}[Semigroup algebra universal property]\label{thm:semigroup_algebra_universal_property}\mcite[381]{Knapp2016BasicAlgebra}
  The \hyperref[def:semigroup_algebra]{semigroup algebra} \( R[G] \) is the unique up to a unique isomorphism \hyperref[def:algebra_over_semiring]{algebra} over the \hyperref[def:semiring]{semiring} \( R \) that satisfies the following \hyperref[rem:universal_mapping_property]{universal mapping property}:
  \begin{displayquote}
    For every \( R \)-algebra \( M \) and every \hyperref[def:semigroup/homomorphism]{semigroup homomorphism} \( e: G \to M \) into the multiplicative group of \( M \), there exists a unique homomorphism of \( R \)-algebras \( \Phi_e: R[G] \to M \) such that the following diagram commutes:
    \begin{equation}\label{eq:thm:semigroup_algebra_universal_property}
      \begin{aligned}
        \includegraphics[page=1]{output/thm__semigroup_algebra_universal_property}
      \end{aligned}
    \end{equation}
  \end{displayquote}
\end{theorem}
\begin{comments}
  \item Via \cref{rem:universal_mapping_property}, \( G \mapsto R[G] \) becomes \hyperref[def:category_adjunction]{left adjoint} to the \hyperref[def:concrete_category]{forgetful functor} which sends \( R \)-algebras to their multiplicative group.
\end{comments}
\begin{proof}
  \Fullref{thm:free_semimodule_universal_property} implies that the following is an \( R \)-module homomorphism:
  \begin{equation*}
    \begin{aligned}
      &\Phi_e: R[G] \to M, \\
      &\Phi_e\parens*{ \Bbbsum_{x \in G} a_x \cdot x } \coloneqq \sum_{x \in G} a_x \cdot_M \varphi(x).
    \end{aligned}
  \end{equation*}

  We will show that \( \Phi_e \) preserves products, that is, that \( \Phi_e \) also a homomorphism of \( R \)-algebras.

  Fix some vector \( \Bbbsum_{y \in G} b_y \cdot y \). We will use induction on the number of nonzero coefficients in this sum to show that
  \begin{equation}\label{eq:thm:semigroup_algebra_universal_property/hypothesis}
    \Phi_e\parens*{ \Bbbsum_{x \in G} a_x \cdot x } \ast_M \Phi_e\parens*{ \Bbbsum_{y \in G} b_y \cdot y }
    =
    \Bbbsum_{g \in G} \parens*{ \sum_{x y = g} a_x b_y } \cdot_M \varphi(g).
  \end{equation}

  \begin{itemize}
    \item If \( b_y = 0 \) for every \( y \in G \), then \( a_x b_y = 0 \) and thus the product \eqref{eq:def:semigroup_algebra/multiplication} with any other vector is necessarily zero. Thus,
    \begin{equation*}
      \Phi_e\parens[\Big]{ \Bbbsum_{x \in G} a_x \cdot x } \ast_M \Phi_e\parens[\Big]{ \Bbbsum_{y \in G} \underbrace{b_y}_{\mathclap{\T*{zero}}} \cdot y }
      =
      \Bbbsum_{g \in G} \underbrace{\parens[\Big]{ \sum_{x y = g} a_x b_y }}_{\T{zero}} \cdot_M \varphi(g).
    \end{equation*}

    \item Otherwise, suppose that \eqref{eq:thm:semigroup_algebra_universal_property/hypothesis} holds for \( m \) nonzero coefficients and consider a vector
    \begin{equation*}
      \sum_{i=1}^{m+1} b_{y_i} \cdot y_i.
    \end{equation*}

    Since \( \ast_M \) is additive in its second argument, we have
    \begin{align}
      &\phantom{{}={}}
      \Phi_e\parens[\Big]{ \Bbbsum_{x \in G} a_x \cdot x } \ast_M \Phi_e\parens[\Big]{ \Bbbsum_{i=1}^{m+1} b_{y_i} \cdot y_i }
      = \nonumber \\ &=
      \Phi_e\parens[\Big]{ \Bbbsum_{x \in G} a_x \cdot x } \ast_M \Phi_e\parens[\Big]{ \Bbbsum_{i=1}^m b_{y_i} \cdot y_i } + \Phi_e\parens[\Big]{ \Bbbsum_{x \in G} a_x \cdot x } \ast_M \Phi_e\parens[\Big]{ b_{y_{m+1}} \cdot y_{m+1} }. \label{eq:thm:semigroup_algebra_universal_property/split}
    \end{align}

    We will study the second term. Because \( \ast_M \) is homogeneous in both arguments, and since \( \Phi_e \) is also homogeneous, we have
    \begin{align*}
      &\phantom{{}={}}
      \Phi_e\parens[\Big]{ \Bbbsum_{x \in G} a_x \cdot x } \ast_M \Phi_e\parens[\Big]{ b_{y_{m+1}} \cdot y_{m+1} }
      = \\ &=
      b_{y_{m+1}} \cdot \bracks[\Bigg]{ \Phi_e\parens[\Big]{ \Bbbsum_{x \in G} a_x \cdot x } \ast_M \Phi_e\parens[\Big]{ y_{m+1} } }
      = \\ &=
      \Phi_e\parens[\Big]{ \Bbbsum_{x \in G} a_x b_{y_{m+1}} \cdot x } \ast_M \Phi_e\parens[\Big]{ y_{m+1} }
      = \\ &=
      \Bbbsum_{x \in G} a_x b_{y_{m+1}} \cdot_M \underbrace{\varphi(x) \ast_M \varphi(y_{m+1})}_{\varphi(xy_{m+1})}.
    \end{align*}

    The last term can be rewritten by grouping coefficients before the same vectors:
    \begin{equation}\label{eq:thm:semigroup_algebra_universal_property/second}
      \Bbbsum_{g \in G} \parens[\Big]{ \sum_{xy_{m+1} = g} a_x b_{y_{m+1}} } \cdot_M \varphi(g).
    \end{equation}

    Substituting the second term in \eqref{eq:thm:semigroup_algebra_universal_property/split} for \eqref{eq:thm:semigroup_algebra_universal_property/second} and using the inductive hypothesis on the first term, we obtain
    \begin{align*}
      &\phantom{{}={}}
      \Phi_e\parens[\Big]{ \Bbbsum_{x \in G} a_x \cdot x } \ast_M \Phi_e\parens[\Big]{ \Bbbsum_{i=1}^m b_{y_i} \cdot y_i } + \Phi_e\parens[\Big]{ \Bbbsum_{x \in G} a_x \cdot x } \ast_M \Phi_e\parens[\Big]{ b_{y_{m+1}} \cdot y_{m+1} }
      = \\ &=
      \Bbbsum_{g \in G} \parens[\Big]{ \sum_{\substack{a y_i = g \\ 1 \leq y \leq m}} a_x b_{y_i} } \cdot_M \varphi(g) + \Bbbsum_{g \in G} \parens[\Big]{ \sum_{xy_{m+1} = g} a_x b_{y_{m+1}} } \cdot_M \varphi(g)
      = \\ &=
      \Bbbsum_{g \in G} \parens[\Big]{ \sum_{\substack{a y_i = g \\ 1 \leq y \leq m}} a_x b_{y_i} + \sum_{xy_{m+1} = g} a_x b_{y_{m+1}} } \cdot_M \varphi(g)
      = \\ &=
      \Bbbsum_{g \in G} \parens[\Big]{ \sum_{\substack{a y_i = g \\ 1 \leq y \leq m + 1}} a_x b_{y_i} } \cdot_M \varphi(g)
      = \\ &=
      \Bbbsum_{g \in G} \parens[\Big]{ \sum_{x y = g} a_x b_y } \cdot_M \varphi(g).
    \end{align*}
  \end{itemize}

  This completes the induction. We conclude the \eqref{eq:thm:semigroup_algebra_universal_property/hypothesis} holds no matter how many nonzero coefficients are there in the sum
  \begin{equation*}
    \Bbbsum_{y \in G} b_y \cdot y.
  \end{equation*}

  This in turn implies that \( \Phi_e \) preserves multiplication, making it a homomorphism of \( R \)-algebras.
\end{proof}

\begin{example}\label{ex:def:semigroup_algebra}
  We list examples of \hyperref[def:semigroup_algebra]{semigroup algebras}:
  \begin{thmenum}
    \thmitem{ex:def:semigroup_algebra/zero} If the semigroup is empty, its semigroup algebra has only one element.

    \thmitem{ex:def:semigroup_algebra/natural} Consider the \hyperref[def:free_monoid]{free monoid} \( F(\set{ X }) \). An expression \( f \) from the semigroup algebra \( R[F(\set{ X })] \) can be written as follows:
    \begin{equation*}
      f = \sum_{k=0}^\infty a_k X^k,
    \end{equation*}
    where \( a_0, a_1, \ldots \) are coefficients from \( R \).

    We denote the neutral element of \( F(\set{ X }) \) by \( 1 \). Since only finitely many coefficients are nonzero, there is some index, \( n \), such that \( \alpha_n \) is the last nonzero coefficient; i.e. \( \alpha_n \neq 0 \) and \( \alpha_{n+k} = 0 \) for \( k = 1, 2, \ldots \).

    Furthermore, to highlight that the underlying monoid uses the symbol \( X \), we denote it by \( f(X) \). Thus, \( f \) can be written as
    \begin{equation*}
      f(X) = \sum_{k=0}^n a_k X^k.
    \end{equation*}

    This is the conventional syntax for \hyperref[def:univariate_polynomial]{univariate polynomials} used through this monograph (see e.g. \fullref{sec:polynomial_division}). In fact, we define general polynomials in \cref{def:polynomial_algebra} using direct products of free monoids.

    Given another expression
    \begin{equation*}
      g(X) = \sum_{k=0}^m b_k X^k,
    \end{equation*}
    we can form their convolution product
    \begin{equation}\label{ex:def:semigroup_algebra/natural/convolution}
      f(X) \ast g(X) = \sum_{k=0}^m \parens*{ \sum_{i+j=k} a_i b_j } X^k.
    \end{equation}

    In the special case \( n = m = 3 \), we will derive \eqref{ex:def:semigroup_algebra/natural/convolution} from first principles. Since multiplication distributes over addition, we have
    \small
    \begin{align*}
      &\phantom{{}={}}
      f(X) \ast g(X)
      = \\ &=
      (a_0 + a_1 X + a_2 X^2) g(X)
      = \\ &=
      a_0 g(X) + a_1 X g(X) + a_2 X^2 g(X)
      = \\ &=
      a_0 b_0 + a_0 b_1 X + a_0 b_2 X^2 + a_1 b_0 X + a_1 b_1 X^2 + a_1 b_2 X^3 + a_2 b_0 X^2 + a_2 b_1 X^3 + a_2 b_2 X^4.
    \end{align*}
    \normalsize

    Grouping the terms in from of the same powers of \( X \), we obtain
    \small
    \begin{equation*}
      f(X) \ast g(X) = a_0 b_0 + (a_0 b_1 + a_1 b_0) X + (a_0 b_2 + a_1 b_1 + a_2 b_0) X^2 + (a_1 b_2 + a_2 b_1) X^3 + a_2 b_2 X^4.
    \end{equation*}
    \normalsize

    We have grouped the coefficients so that their indices sum up to the corresponding power of \( X \). This leads precisely to \eqref{ex:def:semigroup_algebra/natural/convolution}.

    \thmitem{ex:def:semigroup_algebra/cyclic} Now consider the group algebra \( R[C_n] \) of the \hyperref[def:cyclic_group]{cyclic group} \( C_n \).

    Denote the generator of \( C_n \) by \( X \). Similarly to polynomials, an expression from \( R[C_n] \) can thus be written as
    \begin{equation*}
      f(X) = \sum_{k=0}^{n-1} a_k X^k.
    \end{equation*}

    The convolution product is more subtle than in polynomials. For example, in the case \( n = 2 \), we have
    \begin{equation*}
      f(X) \ast g(X) = (a_0 + a_1 X) (b_0 + b_1 X) = a_0 b_0 + (a_0 b_1 + a_1 b_0) X + a_1 b_1 X^2,
    \end{equation*}
    however, since \( X^2 = X^0 \), this becomes
    \begin{equation*}
      f(X) \ast g(X) = (a_0 b_0 + a_1 b_1) + (a_0 b_1 + a_1 b_0) X.
    \end{equation*}

    This hints at the following connection to ordinary polynomials. Suppose that \( R \) is a ring. Since \( C_n \) has the \hyperref[def:group_presentation]{presentation}
    \begin{equation*}
      C_n = \set{ X \given X^n = 1 },
    \end{equation*}
    we can infer that, if \( R \) is a ring, \( R[C_n] \) must be isomorphic to the \hyperref[def:algebra_over_ring/quotient]{quotient algebra}
    \begin{equation*}
      R[X] / \braket{ X^n - 1 },
    \end{equation*}
    in which the polynomials \( X^n \) and \( 1 \) are identified. This is easily verified to be the case.

    \thmitem{ex:def:semigroup_algebra/trivial} In the special case of the trivial group \( C_1 = \set{ e } \), by \cref{ex:def:semigroup_algebra/cyclic}, all powers of \( X \) in \( R[C_1] \) are identified, and thus the convolution product reduces to the usual product in \( R \).

    \thmitem{ex:def:semigroup_algebra/klein} The group algebra \( R[V_4] \) of the \hyperref[def:klein_four_group]{Klein four group} \( V_4 \) is more interesting.

    Denote the generators of \( V_4 \) by \( X \) and \( Y \) and its neutral element by \( 1 \). Then we have \( V_4 = \set{ 1, X, Y, XY } \). An expression in \( R[V_4] \) thus has the form
    \begin{equation*}
      f(X, Y) = aXY + bX + cY + d.
    \end{equation*}

    We can use the presentation
    \begin{equation*}
      V_4 = \braket{ X, Y \given X^2 = Y^2 = 1 \T{and} XY = YX }
    \end{equation*}
    to infer that, if \( R \) is a ring, \( R[V_4] \) should be isomorphic to the quotient
    \begin{equation*}
      R[X, Y] / \braket{ X^2 - 1, Y^2 - 1, XY - YX }
    \end{equation*}
    of the bivariate polynomial algebra \( R[X, Y] \).

    Since monomials commute by definition, we have the equality \( XY = YX \), and we can leave out \( XY - YX \) from the quotient. Thus,
    \begin{equation*}
      R[V_4] \cong R[X, Y] / \braket{ X^2 - 1, Y^2 - 1 }.
    \end{equation*}

    \thmitem{ex:def:semigroup_algebra/laurent} The group algebra of \( \BbbZ \) over \( R \) is the \hyperref[def:laurent_polynomial_algebra]{algebra of Laurent polynomials} over \( R \).

    \thmitem{ex:def:semigroup_algebra/noncommutative} The (semi)group is not necessarily commutative. For example, we define noncommutative polynomials in \cref{def:noncommutative_polynomial_algebra}. In the univariate case, it coincides with ordinary polynomials, but for bivariate noncommutative polynomials \( XY \) and \( YX \) are distinct.

    We define the quaternion algebra in \cref{def:quaternion_algebra} using noncommutative polynomials.
  \end{thmenum}
\end{example}

\paragraph{Polynomial algebras}

\begin{definition}\label{def:multi_index}\mimprovised
  A \term[en=multi-index (\cite[236]{Folland1999RealAnalysis}), ru=мультииндекс (\cite[42]{Тыртышников2017ОсновыАлгебры})]{multi-index} over the set \( \mscrK \) is simply a member \( \nu = \seq{ \nu_k }_{k \in \mscrK} \) of the \hyperref[def:free_commutative_monoid]{multiplicatively-written} \hyperref[def:free_commutative_monoid]{free commutative monoid} \( \BbbN^{\oplus \mscrK} \).

  \begin{thmenum}
    \thmitem{def:multi_index/dimension} We define the \term{dimension} of \( \nu \) as the \hyperref[thm:cardinality_existence]{cardinality} of \( \mscrK \)

    In the cases where \( \mscrK \) is \hyperref[def:totally_ordered_set]{totally ordered}, we regard the multi-indices of finite dimension as \hyperref[def:ordered_tuple]{ordered tuples} like \( \nu = (\nu_1, \ldots, \nu_n) \).

    \thmitem{def:multi_index/order} We define the \term{order} \( \abs\nu \) of \( \nu \) as the sum of its components.

    \thmitem{def:multi_index/power} Multi-indices allow us to extend \hyperref[def:monoid/exponentiation]{monoid exponentiation} by assigning, to each indexed family \( x = \seq{ x_k }_{k \in \mscrK} \), the value
    \begin{equation*}
      x^\nu \coloneqq \prod_{k \in \mscrK} x_k^{\nu_k}.
    \end{equation*}
  \end{thmenum}
\end{definition}
\begin{comments}
  \item Finite dimensional multi-indices are presented as a \hyperref[con:syntactic_abbreviation]{notational shorthand} for partial derivatives in \cite[236]{Folland1999RealAnalysis}, \cite[30]{AtkinsonHan2001TheoreticalNumericalAnalysis} and \cite[438]{Зорич2019АнализЧасть1}. We generalize this to arbitrary dimensions, mostly to simplify our exposition for general polynomial algebras.
\end{comments}

\begin{definition}\label{def:polynomial_algebra}\mimprovised
  Fix a \hyperref[def:semiring/commutative]{commutative semiring} \( R \) and an \hyperref[def:indexed_family]{indexed} \hyperref[def:formal_language/alphabet]{alphabet}  \( \mscrX = \seq{ X_k }_{k \in \mscrK} \). These will act as indeterminates in the sense of \cref{con:free_construction/indeterminate}.

  \begin{thmenum}
    \thmitem{def:polynomial_algebra/monomials} We refer to the \hyperref[def:multi_index]{multi-indices} over \( \mscrX \) as \term[ru=одночлены/мономы (\cite[\S 11.3]{Тыртышников2007ЛинейнаяАлгебра}), en=monomial (\cite[2]{Bourbaki2003Algebra4to7})]{monomials}. Formally, the monomials are elements of the \hyperref[def:free_commutative_monoid]{free commutative monoid} \( \BbbN^{\oplus \mscrX} \), \hyperref[con:additive_semigroup]{written multiplicatively}.

    Based on our discussion in \cref{rem:free_commutative_monoid_as_quotient}, we regard elements of \( \BbbN^{\oplus \mscrX} \) as congruence classes of strings of indeterminates. For denoting monomials, we will use the strings themselves, but regard congruent strings like \( XY \) and \( YX \) as equal.

    \thmitem{def:polynomial_algebra/polynomials} We call the corresponding \hyperref[def:semigroup_algebra]{monoid algebra} \( R[\BbbN^{\oplus \mscrX}] \) the \term[ru=алгебра многочленов (\cite[92]{Винберг2014КурсАлгебры}), en=polynomial algebra (\cite[473]{Bourbaki1998Algebra1to3})]{polynomial algebra} over \( \mscrX \) and denote it by \( R[\mscrX] \) or \( R[X_k \given k \in \mscrK] \). We call elements of \( R[\mscrX] \) \term[bg=полиноми (\cite[1]{Обрешков1962ВисшаАлгебра}), ru=многочлены/полиномы (\cite[\S 11.3]{Тыртышников2007ЛинейнаяАлгебра})]{polynomials}.

    A particular polynomial \( f(\mscrX) = f(X_k \given k \in \mscrK) \) can be written concisely with the aid of the \hyperref[def:multi_index/power]{multi-index exponentiation} as
    \begin{equation}\label{eq:def:polynomial_algebra/polynomials}
      f(\mscrX) = \sum_\nu a_\nu \mscrX^\nu,
    \end{equation}
    where \( a_\nu \) is the coefficient corresponding to the monomial \( \nu \).

    A polynomial is thus a \hyperref[def:linear_combination]{linear combination} of monomials. We retain the terminology for coefficients and terms from \cref{def:linear_combination}. In particular, the term of the monomial \( 1 = \mscrX^0 \) involves no indeterminates, so we call it the \term[ru=свободный член (\cite[118]{Тыртышников2007ЛинейнаяАлгебра}), en=constant term (\cite[1]{Bourbaki2003Algebra4to7})]{constant term}.

    The \hyperref[def:semigroup_algebra]{convolution product} acts as
    \begin{equation}\label{eq:def:polynomial_algebra/polynomials/product}
      f(\mscrX) \ast g(\mscrX)
      =
      \parens*{ \sum_\nu a_\nu \mscrX^\nu } \ast \parens*{ \sum_\nu b_\nu \mscrX^\nu }
      =
      \sum_{m=0}^\infty \parens*{ \sum_{\abs{\nu} = m } a_\nu \cdot b_\nu } \mscrX^\nu,
    \end{equation}

    Instead of the convolution symbol \( {\ast} \), we will henceforth use a dot \( f(X) \cdot g(X) \) or juxtaposition \( fg \).

    \thmitem{def:polynomial_algebra/inclusion} We have the following chain of canonical inclusions:
    \begin{equation*}
      \mscrX \to \BbbN^{\oplus \mscrX} \to R[\mscrX].
    \end{equation*}

    The first map is a plain function and the second one is a homomorphism from the monoid \( \BbbN^{\oplus \mscrX} \) of monomials into the multiplicative monoid of the \( R \)-algebra \( R[\mscrX] \) of polynomials.

    Whenever we need to be explicit, like in \fullref{thm:polynomial_algebra_universal_property}, we denote their composition by \( \iota: \mscrX \to R[\mscrX] \).
  \end{thmenum}
\end{definition}
\begin{comments}
  \item Polynomial algebras in a similar level of generality, but for rings rather than semirings, and without semigroup algebras, are discussed by \incite[\S III.9]{Bourbaki1998Algebra1to3}.

  \item Polynomials with a finite list of indeterminates correspond exactly to polynomial equations as defined in \cref{def:polynomial_equation}.

  \item See \cref{ex:def:fol_free_term_model/polynomial_algebra} for a comparison of \( R[\mscrX] \) to the isomorphic \hyperref[def:fol_free_term_model]{free term model} of the \hyperref[def:algebra_over_semiring/commutative]{first-order theory of commutative algebras} over \( R \).
\end{comments}

\begin{proposition}\label{thm:def:polynomial_algebra}
  The following are basic properties of \hyperref[def:polynomial_algebra]{polynomial algebras}:
  \begin{thmenum}
    \thmitem{thm:def:polynomial_algebra/empty} If \( \mscrX \) is empty, \( R[\mscrX] \cong R \).

    \thmitem{thm:def:polynomial_algebra/union}\mcite[119]{Jacobson1985BasicAlgebraI} The algebras \( R[\mscrX \cup \mscrY] \) and \( R[\mscrX][\mscrY] \) are isomorphic.

    In particular,
    \begin{equation*}
      R[X_1, \ldots, X_{n-1}][X_n] \cong R[X_1, \ldots, X_n].
    \end{equation*}
  \end{thmenum}
\end{proposition}
\begin{proof}
  \SubProofOf{thm:def:polynomial_algebra/empty} Trivial.

  \SubProofOf{thm:def:polynomial_algebra/union} Polynomials in \( R[\mscrX \cup \mscrY] \) have the form
  \begin{equation*}
    f(\mscrX \cup \mscrY) = \sum_\nu a_\nu \mscrX^\nu \mscrY^\nu
  \end{equation*}

  This can be rewritten as
  \begin{equation*}
    f(\mscrX \cup \mscrY) = \sum_\nu \parens*{ a_\nu \mscrX^\nu } \mscrY^\nu
  \end{equation*}
  which is a polynomial from \( R[\mscrX][\mscrY] \).

  The converse is obvious.
\end{proof}

\begin{remark}\label{rem:polynomials_over_infinitely_many_indeterminates}
  As shown in \cref{thm:def:polynomial_algebra/union}, the algebras \( R[X, Y] \) and \( R[X][Y] \) are isomorphic. Thus, if we prove a statement about the algebra \( R[X] \) in one indeterminate, oftentimes it automatically extends via induction to the algebra \( R[X_1, \ldots, X_n] \) over finitely many indeterminates.

  This reasoning fails for more general algebras \( R[X_k \given k \in \mscrK] \), where \( \mscrK \) is infinite. Even some results like \cref{thm:parametrized_polynomial_evaluation_isomorphism} require \( \mscrK \) to be finite due to cardinality considerations.

  Furthermore, the syntactic role of polynomials discussed in \cref{con:free_construction} is obscured when the variables are infinitely many.

  So, while polynomial algebras can be very general, we are almost exclusively restricted to finitely many indeterminates.
\end{remark}

\begin{theorem}[Polynomial algebra universal property]\label{thm:polynomial_algebra_universal_property}
  Fix a \hyperref[def:semiring/commutative]{commutative semiring} \( R \) and a set \( \mscrX \) of indeterminates. The \hyperref[def:polynomial_algebra]{polynomial algebra} \( R[\mscrX] \) is the unique up to a unique isomorphism of \hyperref[def:algebra_over_semiring/commutative]{commutative \( R \)-algebras} that satisfies the following \hyperref[rem:universal_mapping_property]{universal mapping property}:
  \begin{displayquote}
    For every commutative \( R \)-algebra \( M \) and every function \( e: \mscrX \to M \), there exists a unique \( R \)-algebra homomorphism \( \Phi_e: R[\mscrX] \to M \) such that the following diagram commutes:
    \begin{equation}\label{eq:thm:polynomial_algebra_universal_property/diagram}
      \begin{aligned}
        \includegraphics[page=1]{output/thm__polynomial_algebra_universal_property}
      \end{aligned}
    \end{equation}
  \end{displayquote}
\end{theorem}
\begin{comments}
  \item Via \cref{rem:universal_mapping_property}, \( R[\anon*] \) becomes \hyperref[def:category_adjunction]{left adjoint} to the \hyperref[def:concrete_category]{forgetful functor}
  \begin{equation*}
    U: \cat{CAlg}_R \to \cat{Set}.
  \end{equation*}

  The action of \( R[\anon*] \) on morphisms is given by \( \Phi \).

  \item The map \( \Phi_e \) acts as a substitution homomorphism in the sense of \cref{con:free_construction/substitution}.
\end{comments}
\begin{proof}
  \Fullref{thm:free_semimodule_universal_property} gives us a unique monoid homomorphism \( \widehat{e}: \BbbN^{\oplus \mscrX} \to M \) such that \( \widehat{e}(X) = e(X) \) for every indeterminate \( X \).

  \Fullref{thm:semigroup_algebra_universal_property} gives us a unique homomorphism of \( R \)-algebras \( \Phi_e: R[\mscrX] \to M \) such that, for every indeterminate \( X \),
  \begin{equation*}
    \Phi_e(X) = \widehat{e}(X) = e(X).
  \end{equation*}
\end{proof}

\begin{concept}\label{con:free_construction}
  We will discuss here some generalities regarding the so-called \term[en=free construction (\cite[192]{Perrone2024StartingCategoryTheory})]{free constructions}, of which \hyperref[def:polynomial_algebra]{polynomial algebras} are a special case; perhaps the most important one.

  It may be hard to discern from \cref{def:polynomial_algebra/polynomials}, but polynomials gradually arose as a systematic way to treat \hyperref[def:equation]{equations} involving only arithmetic operations. Some historical details are discussed in \cite{HSMSE:origin_of_polynomials}.

  This relationship is made precise via \cref{thm:integer_polynomials_and_logical_terms}, where we show how polynomials with integer coefficients are equivalence classes of \hyperref[def:fol_term]{logical terms} in the \hyperref[def:ring/theory]{first-order theory of rings} (or even the \hyperref[def:semiring/theory]{theory of semirings} if the coefficients are nonnegative).

  In more general (semi)rings, polynomials are the main form of abstract syntax.

  \begin{thmenum}
    \thmitem{con:free_construction/indeterminate} We start with a family \( \mscrX = \seq{ X_k }_{k \in \mscrK} \) of \hyperref[def:formal_language/symbol]{symbols}, which we call \term[bg=неизвестно (\cite[405]{Обрешков1962ВисшаАлгебра}), ru=неизвестная (\cite[131]{Курош1968КурсВысшейАлгебры}), en=indeterminate (\cite[def. III.1.19]{Aluffi2009Algebra})]{indeterminates}. These act as \hyperref[con:variable]{syntactic variables}.

    A polynomial \( f(\mscrX) \) in \( R[\mscrX] \) has its coefficients in \( R \), but the indeterminates are placeholders whose variables need to be \hi{explicitly} assignment values in some algebra \( M \) over \( R \) (not necessarily \( R \) itself).

    \thmitem{con:free_construction/assignment} In analogy with how we define denotations for first-order logical terms in \cref{alg:fol_term_denotation}, we rely on an \term{assignment} \( e: \mscrX \to M \) to supply a value for each indeterminate.

    This contrasts with a metalingual variable like \( x \) that can refer to some element of \( M \). The indeterminate \( X \) is a standalone entity that requires an explicit assignment \( e \).

    The convention of using corresponding uppercase and lowercase letters is due to \hyperref[rem:bourbaki]{Bourbaki}, but is not universal --- see \cref{rem:conventions_for_indeterminates}.

    \thmitem{con:free_construction/construction} Indeterminates are of little use by themselves. Each particular theory has its own abstract syntactic constructs that generalize the role of logical terms.

    We already saw that polynomials generalize logical terms in (semi)ring theory. The \hyperref[def:semimodule/theory]{theory of (semi)modules} has \hyperref[def:linear_combination]{linear combinations}.

    These entities are special because they form particular well-behaved collections. The polynomials in the indeterminates \( \mscrX \) over \( R \) form the \( R \)-algebra \( R[\mscrX] \), and similarly linear combinations in \( \mscrX \) form the free \( R \)-(semi)module \( R^{\oplus \mscrX} \).

    The commonality between \( R[\mscrX] \) and \( R^{\oplus \mscrX} \) is that they are constructed without any assumptions beyond the axioms of the theory. For this reason they are called \enquote{free} (polynomial algebras are free commutative \( R \)-algebras).

    In practice, free constructions are given by \hyperref[def:category_adjunction]{adjoint functors}. An example of a free construction related to \( R[\mscrX] \) is the \hyperref[def:noncommutative_polynomial_algebra]{noncommutative polynomial algebra} \( R\braket{ \mscrX } \). Important examples from \fullref{ch:group_theory} include \hyperref[def:free_monoid]{free monoids} and \hyperref[def:free_commutative_monoid]{free commutative monoids}, as well as \hyperref[def:free_group]{free groups} and \hyperref[def:free_abelian_group]{free Abelian groups}.

    \thmitem{con:free_construction/evaluation} Free constructions would not be very useful if they did not allow \hyperref[con:syntax_semantics_duality]{semantic interpretation}. In fact, they allow interpretation in a particularly uniform manner.

    In the particular case of the polynomial algebra \( R[\mscrX] \), \fullref{thm:polynomial_algebra_universal_property} allows extending the assignment \( e: \mscrX \to M \) to a map \( \Phi_e: R[\mscrX] \to M \). For a polynomial \( f(\mscrX) \), this map substitutes each indeterminate according to \( e \) and then performs the operations described by \( f \). The result is a value \( \Phi_e(f) \) in \( M \).

    It will be convenient for us to use the indexed family \( x = \seq{ x_k }_{k \in \mscrK} \), where \( x_k \coloneqq e(X_k) \), to introduce the notation
    \begin{equation*}
      f(x_k \given k \in \mscrK) \coloneqq \Phi_e(f).
    \end{equation*}

    We call \( \Phi_e \) the \term[en=evaluation homomorphism (\cite[98]{Lang2002Algebra})]{evaluation homomorphism} or \term[en=substitution homomorphism (\cite[151]{Knapp2016BasicAlgebra})]{substitution homomorphism}\fnote{For general free constructions, we will refer to these homomorphisms as simply \enquote{morphisms} in accordance to the established terminology of category theory.} corresponding to the assignment \( e \).

    A useful generalization is the \term{parameterized evaluation homomorphism}
    \begin{equation*}
      \begin{aligned}
        &\Phi_M: R[\mscrX] \to \fun(M^\mscrX, M), \\
        &\Phi_M(f) \coloneqq (e \mapsto \Phi_e(f)),
      \end{aligned}
    \end{equation*}
    which sends a polynomial \( f(\mscrX) \) to a function from \( M^\mscrX \) to \( M \). We call \( \Phi_M(f) \) the \term[en=polynomial function (\cite[4]{Bourbaki2003Algebra4to7}), ru=полиномиальная функция (\cite[202]{Кострикин2000АлгебраЧасть1})]{polynomial function} corresponding to \( f \).
  \end{thmenum}
\end{concept}

\begin{remark}\label{rem:conventions_for_indeterminates}
  We mentioned in \cref{con:free_construction/indeterminate} that we distinguish an indeterminate \( X \) from a metalinguistic variable \( x \) by the letter's case. This tradition is used, and perhaps introduced, by \hyperref[rem:bourbaki]{Bourbaki} in \cite[A III.25]{Bourbaki1970Algèbre1à3}. It is used in \cite[ch. 4]{Lang2002Algebra}, \cite[9]{Knapp2016BasicAlgebra} and \cite[182]{Кострикин2000АлгебраЧасть1}, but it is not very popular in general.

  Compared to that, lowercase letters are used for both indeterminates and variables in
  \cite[122]{Jacobson1985BasicAlgebraI},
  \cite[43]{Rotman2015AdvancedModernAlgebraPart1},
  \cite[23]{Eisenbud1995CommutativeAlgebra},
  \cite[def. III.1.19]{Aluffi2009Algebra},
  \cite[36]{Golan1999Semirings},
  \cite[97]{Lang2002Algebra},
  \cite[131]{Курош1968КурсВысшейАлгебры},
  \cite[405]{Обрешков1962ВисшаАлгебра},
  \cite[92]{Винберг2014КурсАлгебры},
  \cite[135]{ГеновМиховскиМоллов1991Алгебра},
  \cite[4]{КоцевСидеров2016КомутативнаАлгебра} and
  \cite[12]{Тыртышников2017ОсновыАлгебры}.

  As shown in \cref{thm:parametrized_polynomial_evaluation_isomorphism}, polynomials over infinite integral domains can be conflated with their polynomial functions. In these cases the role of indeterminates is not as important.
\end{remark}

\begin{definition}\label{def:polynomial_degree}\mcite[2]{Bourbaki2003Algebra4to7}
  We define the \term[en=total degree (\cite[103]{Lang2002Algebra})]{total degree} or simply \term[bg=степен (\cite[1]{Обрешков1962ВисшаАлгебра}), ru=степень (\cite[\S 8.1]{Тыртышников2007ЛинейнаяАлгебра})]{degree} of the \hyperref[def:polynomial_algebra/polynomial]{polynomial} as the largest \hyperref[def:multi_index/order]{multi-index order} among its monomials. For the zero polynomial, which has no monomials, we define the degree to be the \hyperref[def:extended_real_numbers]{extended real number} \( -\infty \).

  \begin{table}
    \begin{center}
      \begin{tabular}{L l}
        \toprule
          \text{Degree} & Term \\
        \midrule
          \leq 0        & \term[ru=константа (\cite[118]{Тыртышников2007ЛинейнаяАлгебра})]{constant} \\
          1             & \term[bg=линейна (функция) (\cite[1]{Обрешков1962ВисшаАлгебра}), ru=линейная (форма) (\cite[315]{Курош1968КурсВысшейАлгебры})]{linear} \\
          2             & \term[bg=квадратна (функция) (\cite[1]{Обрешков1962ВисшаАлгебра}), ru=квадратичная (форма) (\cite[315]{Курош1968КурсВысшейАлгебры}), en=quadratic (polynomial) (\cite[44]{Rotman2015AdvancedModernAlgebraPart1})]{quadratic} \\
        \bottomrule
      \end{tabular}
      \quad
      \begin{tabular}{L l}
        \toprule
          \text{Degree} & Term \\
        \midrule
          3             & \term[bg=кубична (форма) (\cite[1]{Обрешков1962ВисшаАлгебра}), ru=кубичная (форма) (\cite[315]{Курош1968КурсВысшейАлгебры}), en=cubic (polynomial) (\cite[44]{Rotman2015AdvancedModernAlgebraPart1})]{cubic} \\
          4             & \term[en=quartic (polynomial) (\cite[44]{Rotman2015AdvancedModernAlgebraPart1})]{quartic} \\
          5             & \term[en=quintic (polynomial) (\cite[44]{Rotman2015AdvancedModernAlgebraPart1})]{quintic} \\
        \bottomrule
      \end{tabular}
    \end{center}
    \caption{Common terms for polynomials of small \hyperref[def:polynomial_degree]{degrees}.}\label{tab:def:polynomial_degree}
  \end{table}
\end{definition}
\begin{comments}
  \item A linear polynomial corresponds to an \hyperref[def:affine_operator]{affine function}, not a \hyperref[def:linear_function]{linear function}. Thus, we must be precise when using the word \enquote{linear}. This also leads to a distinction between homogeneous and inhomogeneous linear equations characterized in \cref{thm:homogeneous_linear_equation}.

  \item Different conventions for handling the degree of the zero polynomial are discussed in \cref{rem:zero_polynomial_degree}.
\end{comments}

\begin{remark}\label{rem:zero_polynomial_degree}
  In \cref{def:polynomial_degree} we have defined the degree of the zero polynomial to be \( -\infty \). This will be very convenient in \cref{thm:univariate_polynomial_sum} and \cref{thm:univariate_polynomial_product}, as well as through \fullref{sec:polynomial_division}.

  On the other hand, both the definition of Euclidean domains in \cref{def:euclidean_domain} and the definition of homogeneous elements in \cref{def:homogeneous_element} hint that the degree of \( f(X) = 0 \) should be left \hyperref[con:undefinedness]{undefined}.

  We list here how different authors handle the situation.

  \begin{thmenum}
    \thmitem{rem:zero_polynomial_degree/negative_infinity} The degree of \( f(X) = 0 \) is defined as \( -\infty \) in
    \cite[97]{Lang2002Algebra},
    \cite[128]{Jacobson1985BasicAlgebraI},
    \cite[125]{Aluffi2009Algebra},
    \cite[28]{Ahlfors1979ComplexAnalysis},
    \cite[93]{Винберг2014КурсАлгебры} and
    \cite[183]{Кострикин2000АлгебраЧасть1}.

    This convention is useful when comparing degrees. In fact, \incite[155]{Knapp2016BasicAlgebra} defines the degree as \( -\infty \) for the section where he finds it convenient, but otherwise leaves the degree undefined.

    \thmitem{rem:zero_polynomial_degree/undefined} The degree of \( f(X) = 0 \) is left undefined in
    \cite[42]{Rotman2015AdvancedModernAlgebraPart1},
    \cite[118]{Тыртышников2017ОсновыАлгебры},
    \cite[132]{Курош1968КурсВысшейАлгебры} and
    \cite[22]{ГеновМиховскиМоллов1991Алгебра}.

    Although this approach has its appeal, it makes it difficult to compare the degrees of polynomials and instead forces us to do case analysis.

    \thmitem{rem:zero_polynomial_degree/minus_one} The degree of \( f(X) = 0 \) is defined as \( -1 \) in \cite[10]{FriedbergInselSpence2018LinearAlgebra}.

    This convention is also useful when comparing degrees, however it is incompatible with certain results like \cref{thm:univariate_polynomial_product/degree}.

    One justification of this precise definition is the following characterization of the degree of a univariate polynomial \( f \) over the real numbers:
    \begin{equation*}
      \deg(f) = \begin{cases}
        \min\set{ k = 0, 1, \ldots \given D^{k+1} f = 0 }, &f \neq 0, \\
        \T{undefined},                                     &\T{otherwise.}
      \end{cases}
    \end{equation*}

    Here the degree of \( f \) is expressed as the smallest nonnegative integer \( k \) such that the \( (k + 1) \)-th derivative of \( f \) is identically zero. If \( f \) is the zero polynomial, this definition is technically satisfied for \( k = -1 \).
  \end{thmenum}
\end{remark}

\begin{definition}\label{def:polynomial_root}\mimprovised
  A \term[bg=корен (на уравнение) (\cite[2]{Обрешков1962ВисшаАлгебра}), ru=корень (многочлена) (\cite[99]{Винберг2014КурсАлгебры}), en=root (of polynomial) (\cite[281]{Aluffi2009Algebra})]{root} in the \( R \)-module \( M \) of the polynomial \( p \) from \( R[\mscrX] \) is an assignment \( e: \mscrX \to M \) such that \( \Phi_e(f) = 0_M \).
\end{definition}
\begin{comments}
  \item Roots are zeros in the sense of \cref{def:zero_of_function} of a fixed polynomial function. See \cref{rem:root_terminology} for a broader discussion of the various uses of the term \enquote{root}.

  \item This concept extends to \enquote{roots with multiplicities} in the case of \hyperref[def:univariate_polynomial]{univariate polynomial} --- see \cref{def:polynomial_root_multiplicity}.
\end{comments}

\begin{remark}\label{rem:root_terminology}
  The term \enquote{root} has several distinct (but related) meanings:
  \begin{thmenum}
    \thmitem{rem:root_terminology/direct} The \( n \)-th root in the sense of \cref{def:semigroup_power}. The discussion in \cite{HSMSE:radical_symbol_history} suggests that this is the origin of the terms \enquote{root} and \enquote{radical}, based on the Latin \enquote{radix}.

    There is inherent ambiguity in picking \enquote{the} root, since there may be many. For positive real numbers we have an established canonical choice of \( n \)-th roots, which we call \enquote{principal roots}, and for negative real numbers --- of square roots. See \cref{def:principal_nonnegative_nth_root} for the former and \cref{def:principal_real_square_root} for the latter.

    \thmitem{rem:root_terminology/equation} The \hyperref[def:fol_equation/semantic_solution]{solutions} of an \hyperref[def:polynomial_equation]{algebraic equation} are also called \enquote{roots}. An explicit definition with this usage can be found in \cite[2]{Обрешков1962ВисшаАлгебра}.

    As noted in \cref{def:polynomial_equation}, polynomials naturally arise from equations on rings, thus it makes sense for this usage to predate polynomials.

    In particular, in a \hyperref[def:ring/commutative]{commutative ring}, \( b \) is an \( n \)-th root of \( a \) if and only if \( b \) is a solution to the algebraic equation \( X^n - a \).

    \thmitem{rem:root_terminology/polynomial} From a modern perspective, both uses described above are encompassed by roots of polynomials as defined in \cref{def:polynomial_root}. Roots are zeros in the sense of \cref{def:zero_of_function} of a fixed polynomial function. For \hyperref[def:univariate_polynomial]{univariate polynomials}, \fullref{thm:polynomial_factor_theorem} provides a characterization in terms of divisibility, and \cref{def:polynomial_root_multiplicity} provides the concept of \enquote{multiplicity} of a root.

    A \enquote{root} is defined as a zero of a univariate polynomial in
    \cite[282]{Aluffi2009Algebra},
    \cite[11]{Knapp2016BasicAlgebra} and
    \cite[119]{Тыртышников2017ОсновыАлгебры}.
  \end{thmenum}
\end{remark}

\begin{example}\label{ex:polynomial_root}
  We list several examples of \hyperref[def:polynomial_root]{polynomial roots} over semirings.
  \begin{thmenum}
    \thmitem{ex:polynomial_root/natural_numbers} Consider the polynomial \( f(X) \coloneqq aX^2 + bX + c \) in \( \BbbN[X] \). A function from the set \( \set{ X } \) to \( \BbbN \) corresponds to an element of \( \BbbN \), and hence evaluating the polynomial is done by simply replacing \( X \) symbolically in \( p \) and then evaluating the obtained \hyperref[def:fol_free_term_model]{syntax tree}.

    We seek the roots of \( f(X) \), i.e. the natural numbers \( n \) such that \( f(n) = \Phi_n(f) = 0_R \).

    By \fullref{thm:fundamental_theorem_of_algebra} and \cref{def:algebraically_closed_field/exactly_n_roots}, \( p \) has two roots in the \hyperref[def:complex_numbers]{complex plane} --- we regard \( \BbbC \) as an algebra over \( \BbbN \) and use \fullref{thm:polynomial_algebra_universal_property} to obtain a polynomial function in \( \BbbC \). Furthermore, over the complex numbers the roots can be explicitly found using \cref{thm:real_quadratic_polynomial_roots}.

    Finding a root of \( p \) over the natural numbers cannot be done in general, however. If \( f(n) = 0 \), by the ordering of the natural numbers we have
    \begin{equation*}
      f(n) = an^2 + bn + c \geq c,
    \end{equation*}
    and hence \( c \) must necessarily be \( 0 \). If \( c = 0 \), then zero is a root of the polynomial \( f(X) = aX^2 + bX \).

    Now let \( n \) be any root of \( p \). We have
    \begin{equation*}
      an^2 + bn \geq bn,
    \end{equation*}
    and hence \( bn \) must also be \( 0 \). Thus, either \( b = 0 \) or \( n = 0 \). If we want a root other than \( n \), both \( a \) and \( b \) must be \( 0 \). But if \( a \) and \( b \) are zero, the polynomial reduces to a constant.

    Therefore, the only natural number solution to the general quadratic equation is \( 0 \), and it is only a solution if \( c = 0 \).

    \thmitem{ex:polynomial_root/tropical} Consider again the polynomial \( f(X) \coloneqq aX^2 + bX + c \) over \( \BbbN \), but this time evaluated over the \hyperref[def:tropical_semiring]{\( \min \)-plus semiring} \( T \coloneqq (\BbbN \cup \set{ \infty }, \min, +) \).

    Now \( f(X) \) encodes an optimization problem. When expressed via the standard natural number operations, this polynomial becomes
    \begin{equation*}
      f(X) = \min\set{ 2X + a, X + b, c }.
    \end{equation*}

    It should be noted that the evaluation homomorphism maps \( 0 \) in \( \BbbN \) to \( \infty \) in \( T \) and \( 1 \) in \( \BbbN \) to \( 0 \) in \( T \).

    Based on our previous discussion, the extended natural number \( n \) from \( T \) is a root of \( f(X) \) if and only if \( n = \infty \), and only in the case where \( c = 0 \) (which translates to \( c = \infty \) in \( T \)).

    So roots over tropical semirings are not particularly exciting.
  \end{thmenum}
\end{example}

\paragraph{Univariate polynomials}

\begin{definition}\label{def:univariate_polynomial}\mimprovised
  We call polynomials in one indeterminate \term{univariate}. For a univariate polynomial \( f(X) \) of degree \( n \), the notation \eqref{eq:def:polynomial_algebra/polynomials} simplifies to
  \begin{equation}\label{eq:def:univariate_polynomial}
    f(X) = \sum_{k=0}^n a_k X^k = a_0 + a_1 X + a_2 X^2 + \cdots + a_{n-1} X^{n-1} + a_n X^n,
  \end{equation}
  where \( n \) is \( 0 \) for the zero polynomial and \( \deg f \) otherwise.

  We call \( a_n \) the \term[bg=старши коефициент (\cite[23]{ГеновМиховскиМоллов1991Алгебра}), ru=старший коефициент (\cite[118]{Тыртышников2007ЛинейнаяАлгебра})]{leading coefficient} of \( f(X) \).
\end{definition}

\begin{remark}\label{rem:univariate_polynomial_coefficient_order}
  There are two conventions for denoting the coefficients of \hyperref[def:univariate_polynomial]{univariate polynomials}:
  \begin{equation}\label{eq:rem:univariate_polynomial_coefficient_order/covariant}
    f(X) = \sum_{k=0}^n a_k X^k.
  \end{equation}
  and
  \begin{equation}\label{eq:rem:univariate_polynomial_coefficient_order/contravariant}
    f(X) = \sum_{k=0}^n a_{n-k} X^k.
  \end{equation}

  We prefer \eqref{eq:rem:univariate_polynomial_coefficient_order/covariant} because, when regarded as a sequence of coefficients, the \( k \)-th coefficients corresponds to the \( k \)-th power of \( X \), even for \( k \) larger than the polynomial degree.

  \begin{thmenum}
    \thmitem{rem:univariate_polynomial_coefficient_order/covariant} The first convention \eqref{eq:rem:univariate_polynomial_coefficient_order/covariant} is preferred in
    \cite[97]{Lang2002Algebra},
    \cite[44]{Rotman2015AdvancedModernAlgebraPart1},
    \cite[9]{Knapp2016BasicAlgebra},
    \cite[def. III.1.19]{Aluffi2009Algebra},
    \cite[120]{Jacobson1985BasicAlgebraI},
    \cite[10]{FriedbergInselSpence2018LinearAlgebra},
    \cite[28]{Ahlfors1979ComplexAnalysis},
    \cite[\S 8.24]{Schechter1997AnalysisHandbook},
    \cite[96]{Маркушевич1967АналитическиеФункцииТом1},
    \cite[93]{Винберг2014КурсАлгебры},
    \cite[180]{Кострикин2000АлгебраЧасть1},
    \cite[118]{Тыртышников2017ОсновыАлгебры},
    \cite[132]{Окунев1951КольцоМногочленов},
    \cite[22]{ГеновМиховскиМоллов1991Алгебра} and
    \cite[120]{ИлинСадовничиСендов1984АнализТом1}.

    \thmitem{rem:univariate_polynomial_coefficient_order/contravariant} The second convention \eqref{eq:rem:univariate_polynomial_coefficient_order/contravariant} is preferred in
    \cite[131]{Курош1968КурсВысшейАлгебры},
    \cite[1]{Обрешков1962ВисшаАлгебра} and
    \cite[1]{Боянов2008ЧислениМетоди}
  \end{thmenum}
\end{remark}

\begin{proposition}\label{thm:univariate_polynomial_sum}
  Over any \hyperref[def:semiring/commutative]{commutative semiring} \( R \), consider \hyperref[def:univariate_polynomial]{univariate polynomials}
  \begin{align*}
    f(X) = \sum_{k=0}^n a_k X^k, &&
    g(X) = \sum_{k=0}^m b_k X^k.
  \end{align*}

  Their sum \( f + g \) has the following properties:
  \begin{thmenum}
    \thmitem{thm:univariate_polynomial_sum/degree} We have
    \begin{equation}\label{eq:thm:univariate_polynomial_sum/degree}
      \deg(f + g) \leq \max\set{ \deg f, \deg g }.
    \end{equation}

    Furthermore, equality holds unless \( n = m \) and \( a_n + b_m = 0 \).
  \end{thmenum}
\end{proposition}
\begin{proof}
  Denote by \( c_k \) the \( k \)-th coefficient of \( f + g \). Clearly \( c_k = a_k + b_k \) for every index \( k \), so in particular \( c_k = 0 \) when \( k > n \) and \( k > m \). Hence, \( \deg(f + g) \leq \max\set{ n, m } \).

  Furthermore, if \( n > m \), then \( c_n = a_n \) and thus \( \deg(f + g) = n = \max\set{ n, m } \). The case \( n < m \) similarly implies equality.

  Equality also holds if \( n = m \), but only if \( a_n + b_m \neq 0 \).
\end{proof}

\begin{proposition}\label{thm:univariate_polynomial_product}
  Over any \hyperref[def:semiring/commutative]{commutative semiring} \( R \), consider \hyperref[def:univariate_polynomial]{univariate polynomials}
  \begin{align*}
    f(X) = \sum_{k=0}^n a_k X^k, &&
    g(X) = \sum_{k=0}^m b_k X^k.
  \end{align*}

  Their product \( fg \) has the following properties:
  \begin{thmenum}
    \thmitem{thm:univariate_polynomial_product/nonzero_leading} If the product \( a_n b_m \) of leading coefficients is nonzero in \( R \), the product \( fg \) is nonzero in \( R[X] \), and its leading coefficient is \( a_n b_m \).

    \thmitem{thm:univariate_polynomial_product/invertible_leading} If both \( a_n \) and \( b_m \) are nonzero and if at least one of them is \hyperref[def:divisibility/invertible]{invertible}, then \( fg \) is nonzero.

    \thmitem{thm:univariate_polynomial_product/degree} We have
    \begin{equation}\label{eq:thm:univariate_polynomial_product/degree}
      \deg (fg) \leq \deg f + \deg g.
    \end{equation}

    Furthermore, equality holds unless both \( a_n \) and \( b_m \) are nonzero and \( a_n b_m \) is not. Such an example is \( f(X) = g(X) = 2X \), whose product in \hyperref[def:ring_of_integers_modulo]{\( \BbbZ_4 \)} is \( 0 \) --- see \cref{ex:def:divisibility/z4_x2}.

    One sufficient condition for equality is for \( R \) to be \hyperref[def:entire_semiring]{entire}. Another sufficient condition is for either \( a_n \) or \( b_m \) to be invertible.
  \end{thmenum}
\end{proposition}
\begin{proof}
  \SubProofOf{thm:univariate_polynomial_product/nonzero_leading} The coefficient \( c_{n+m} \) of \( f g \) is \( a_n b_m \) by definition. For \( k > n + m \), if \( i + j = k \), either \( i > n \) or \( j > m \), so \( c_k \) is zero.

  Thus, \( c_{n+m} = a_n b_m \) is the leading coefficient of \( fg \), and \( fg \) is not the zero polynomial.

  \SubProofOf{thm:univariate_polynomial_product/invertible_leading} \Cref{thm:semiring_nonzero_product} implies that \( a_n b_m \) is nonzero in \( R \). Then \cref{thm:univariate_polynomial_product/nonzero_leading} implies that \( fg \) is nonzero in \( R[X] \).

  \SubProofOf{thm:univariate_polynomial_product/degree} We have shown in \cref{thm:univariate_polynomial_product/nonzero_leading} that, for \( k > n + m \), the coefficient \( c_k \) of \( fg \) is zero.

  Obviously \( \deg(fg) \leq n + m \). We have equality in the following cases:
  \begin{itemize}
    \item If \( a_n = b_m = 0 \), then \( c_{n+m} = 0 \) and
    \begin{equation*}
      \underbrace{\deg(fg)}_{-\infty} = \underbrace{\deg f}_{-\infty} + \underbrace{\deg g}_{-\infty} = -\infty.
    \end{equation*}

    \item If \( c_{n+m} = a_n b_m \) is nonzero, equality in \eqref{eq:thm:univariate_polynomial_product/degree} also holds.

    \item Suppose that \( a_n \) is invertible.
    \begin{itemize}
      \item If \( b_m \neq 0 \), then \( c_{n+m} \neq 0 \) due to \cref{thm:semiring_nonzero_product}, again giving equality in \eqref{eq:thm:univariate_polynomial_product/degree}.

      \item If \( b_m = 0 \), then \( c_{n+m} = 0 \), in which case
      \begin{equation*}
        \underbrace{\deg(fg)}_{-\infty} = \deg f + \underbrace{\deg g}_{-\infty} = -\infty.
      \end{equation*}
    \end{itemize}

    \item If \( b_m \) is invertible, the proof proceeds analogously.
  \end{itemize}
\end{proof}

\begin{corollary}\label{thm:entire_polynomial_algebra}
  The univariate \hyperref[def:polynomial_algebra]{polynomial semiring} \( R[X] \) is \hyperref[def:entire_semiring]{entire} if and only if \( R \) is entire.
\end{corollary}
\begin{proof}
  \SubProof{Proof that \( R[X] \) and \( R \) are entire simultaneously}

  \SufficiencySubProof* Since \( R \) is an \( R \)-subalgebra of \( R[X] \), if the latter is entire, so is the former.

  \NecessitySubProof* Suppose that \( R \) is entire.

  Then the product of leading coefficients of the nonzero polynomials \( f(X) \) and \( g(X) \) is nonzero, and \cref{thm:univariate_polynomial_product/nonzero_leading} implies that \( f(X) \cdot g(X) \) is nonzero.

  Therefore, \( R[X] \) to be entire.
\end{proof}

\paragraph{Formal power series}

\begin{definition}\label{def:formal_power_series}\mcite[41]{Rotman2015AdvancedModernAlgebraPart1}
  If we extend the concept of \hyperref[def:polynomial_algebra]{polynomials} to allow countably many monomials with nonzero coefficients, we obtain \( R\Bracks{\mscrX} \), which we call the set of \term[ru=формальные степянные ряды (\cite[28]{Шафаревич1999ОсновныеПонятияАлгебры})]{formal power series} over \( R \) with indeterminates \( \mscrX \).
\end{definition}
\begin{comments}
  \item The evaluation homomorphism defined in \fullref{thm:polynomial_algebra_universal_property} is problematic since algebraic operations are finitary by nature. This is discussed in \cref{rem:binary_operations_to_infinitary}, along with how sometimes we can make sense of infinitary algebraic operations.
\end{comments}

\paragraph{Laurent polynomials}

\begin{definition}\label{def:laurent_polynomial_algebra}\mimprovised
  Analogously to \hyperref[def:polynomial_algebra]{polynomial algebras}, we can consider the \hyperref[def:semigroup_algebra]{semigroup algebra} over \( \BbbZ^{\oplus \mscrX} \). We call members of this algebra \term{Laurent polynomials}.
\end{definition}
\begin{comments}
  \item \incite[exerc. B-1.7]{Rotman2015AdvancedModernAlgebraPart1} and \incite[127]{Aluffi2009Algebra} discuss the ring of Laurent polynomials, but restrict themselves to one variable.
\end{comments}

\begin{definition}\label{def:formal_laurent_series}\mcite[425]{Jacobson1989BasicAlgebraII}
  Similarly to how we extended \hyperref[def:polynomial_algebra/polynomials]{polynomials} to \hyperref[def:formal_power_series]{formal power series} by allowing infinitely many nonzero coefficients, we extend \hyperref[def:laurent_polynomial_algebra]{Laurent polynomials} to \term[ru=формальные ряды Лорана (\cite[23]{Шафаревич1999ОсновныеПонятияАлгебры})]{formal Laurent series} by allowing infinitely many nonzero coefficients \hi{with positive indices}. We denote the algebra of formal Laurent series in \( \mscrX \) over \( R \) by \( R\Parens{\mscrX} \).
\end{definition}

\paragraph{Adjoining elements via polynomial evaluation}

\begin{proposition}\label{thm:generators_via_polynomials}
  For a set \( A \) in an \( R \)-\hyperref[def:algebra_over_semiring]{algebra} \( M \), the \( R \)-subalgebra \hyperref[def:algebra_over_semiring/generated]{generated} by \( A \) coincides with the set
  \begin{equation*}
    \bigcup \set[\Big]{ R[a_1, \ldots, a_n] \given* a_1, \ldots, a_n \in A }
  \end{equation*}
  obtained by evaluating all multivariate polynomials over \( R \) with elements of \( A \).
\end{proposition}
\begin{comments}
  \item The \( \BbbN \)-subalgebras of \( M \) correspond to \hyperref[def:semiring/submodel]{sub-semirings} and the \( M \)-subalgebras correspond to \hyperref[def:semiring_ideal]{ideals}.

  \item Compare this result to \cref{thm:span_via_linear_combinations} for linear spans in (semi)modules.
\end{comments}
\begin{proof}
  Similar to \cref{thm:span_via_linear_combinations}.
\end{proof}

\begin{definition}\label{def:semiring_adjunction}
  Let \( R \) be a \hyperref[def:ring/submodel]{sub-semiring} of the \hyperref[def:semiring/commutative]{commutative semiring} \( S \) and let \( A \) be a subset of \( S \). Denote by \( R[A] \) the ring generated by \term[bg=присъединяване/адюнгиране (на елемент) (\cite[425]{Обрешков1962ВисшаАлгебра}), ru=присоединение (элемента) (\cite[288]{Курош1968КурсВысшейАлгебры}), en=adjoining (a subset) (\cite[119]{Jacobson1985BasicAlgebraI})]{adjoining} \( A \), which we define via any of the following equivalent definitions:
  \begin{thmenum}
    \thmitem{def:semiring_adjunction/smallest}\mcite[119]{Jacobson1985BasicAlgebraI} \( R[A] \) is the (unique) smallest super-semiring of \( R \) containing \( A \).

    \thmitem{def:semiring_adjunction/evaluation} \( R[A] \) is the image of the \hyperref[con:free_construction/evaluation]{evaluation homomorphism} \( \Phi_e: R[\mscrX] \to S \), where \( e: \mscrX \to S \) is a bijective function assigning the elements of \( A \) to indeterminates.
  \end{thmenum}

  In case \( A \) is finite, we can list the individual elements as \( \Bbbk(a_1, \ldots, a_n) \).
\end{definition}
\begin{comments}
  \item \Cref{def:semiring_adjunction/smallest} requires a proof of existence, but \cref{def:semiring_adjunction/evaluation} does not. Since the two are equal, we avoid proving existence.
  \item Compare this to the related but distinct notion for fields defined in \cref{def:field_adjunction}.
\end{comments}
\begin{defproof}
  The equivalence follows from \cref{thm:generators_via_polynomials}.
\end{defproof}

\begin{proposition}\label{thm:semiring_adjunction_tower}\mcite[119]{Jacobson1985BasicAlgebraI}
  Let \( R \) be a \hyperref[def:ring/submodel]{sub-semiring} of the \hyperref[def:semiring/commutative]{commutative semiring} \( S \) and let \( A \) and \( B \) be subsets of \( S \). For the corresponding \hyperref[def:semiring_adjunction]{semiring adjunctions} we have
  \begin{equation*}
    R[A \cup B] = R[A][B].
  \end{equation*}
\end{proposition}
\begin{comments}
  \item This also holds for \hyperref[def:field_adjunction]{field adjunctions} --- see \cref{thm:field_adjunction_tower}.
\end{comments}
\begin{proof}
  Follows from \cref{thm:def:polynomial_algebra/union}.
\end{proof}

\begin{example}\label{ex:adjoining_root}
  Continuing \cref{ex:polynomial_root/natural_numbers}, consider the polynomial equation
  \begin{equation*}
    X + 1 = 0.
  \end{equation*}

  It has no natural number root as a consequence of \eqref{eq:def:peano_arithmetic/PA2}.

  It does have an integer root, however, \( -1 \). We can \hyperref[def:semiring_adjunction]{adjoin} \( -1 \) to the semiring \( \BbbN \) to obtain the semiring \( \BbbN[-1] \). But this latter semiring is (isomorphic to) \( \BbbZ \).

  Therefore, \( \BbbZ \) is the smallest extension of \( \BbbN \) that contains a root to the polynomial \( X + 1 \).

  This example extends to the theory of \hyperref[def:algebraic_element]{algebraic} and \hyperref[def:transcendental_element]{transcendental} elements of fields.
\end{example}

\begin{example}\label{ex:adjoining_polynomial}
  Consider again the polynomial algebra \( \BbbN[X] \). Rather than adjoining elements from the complex numbers, we can \enquote{adjoin} some polynomial from \( \BbbN[X] \) itself.

  For example, the algebra \( \BbbN[X^2] \) consists of polynomials of the form
  \begin{equation*}
    f(X) = \sum_{k=0}^\infty a_k X^{2k}.
  \end{equation*}

  We can further enhance this example. Consider the polynomial algebra \( \BbbN[X^2, X^3] \). We know that \( \BbbN[X, Y] \) consists of polynomials in two indeterminates, i.e.
  \begin{equation*}
    f(X, Y) = \sum_{k=0}^\infty \sum_{m=0}^\infty a_{k+m} X^k Y^m.
  \end{equation*}

  Thus, polynomials in \( \BbbN[X^2, X^3] \) have the form
  \begin{equation*}
    f(X) = \sum_{k=0}^\infty \sum_{m=0}^\infty a_{k+m} X^{2k + 3m}.
  \end{equation*}

  Note that, given some natural number \( n \) distinct from \( 1 \), we can obtain \( n = 2k + 3m \) as follows:
  \begin{itemize}
    \item If \( n \) is even, let \( a = n / 2 \) and \( b = 0 \).
    \item If \( n \) is odd, let \( a = (n - 1) / 2 - 1 \) and \( b = 1 \).
  \end{itemize}

  Such a representation is not possible for \( n = 1 \). Thus, \( \BbbN[X^2, X^3] \) contains all polynomials from \( \BbbN[X] \) whose coefficient for \( X \) is \( 0 \).
\end{example}

\paragraph{Noncommutative polynomials}

\begin{definition}\label{def:noncommutative_polynomial_algebra}\mcite[169]{Aluffi2009Algebra}
  We define the \term[ru=алгебра некоммутативных многочленов (\cite[92]{Шафаревич1999ОсновныеПонятияАлгебры})]{noncommutative polynomial algebra} \( R\braket{ \mscrX } \) over the semiring \( R \) with \hyperref[con:free_construction/indeterminate]{indeterminate} \( \mscrX \) as the \hyperref[def:semigroup_algebra]{monoid algebra} \( R[\mscrX^*] \), where \( \mscrX^* \) is the
  \hyperref[def:semigroup_algebra]{monoid algebra} over the \hyperref[def:free_monoid]{free monoid} (rather than the \hyperref[def:free_commutative_monoid]{free commutative monoid} \( \BbbN^\mscrX \)) of \( \mscrX \).
\end{definition}

\begin{theorem}[Noncommutative polynomial algebra universal property]\label{thm:noncommutative_polynomial_algebra_universal_property}
  Fix a \hyperref[def:semiring/commutative]{commutative semiring} \( R \) and a set \( \mscrX \) of indeterminates. The \hyperref[def:noncommutative_polynomial_algebra]{noncommutative polynomial algebra} \( R\braket{\mscrX} \) is the unique up to a unique isomorphism \hyperref[def:algebra_over_semiring]{\( R \)-algebra} that satisfies the following \hyperref[rem:universal_mapping_property]{universal mapping property}:
  \begin{displayquote}
    For every \( R \)-algebra \( M \) and every function \( e: \mscrX \to M \), there exists a unique \( R \)-algebra homomorphism \( \Phi_e: R\braket{\mscrX} \to M \) such that the following diagram commutes:
    \begin{equation}\label{eq:thm:noncommutative_polynomial_algebra_universal_property/diagram}
      \begin{aligned}
        \includegraphics[page=1]{output/thm__noncommutative_polynomial_algebra_universal_property}
      \end{aligned}
    \end{equation}
  \end{displayquote}
\end{theorem}
\begin{proof}
  Analogous to \fullref{thm:polynomial_algebra_universal_property}.
\end{proof}
