\subsection{Hausdorff distance}\label{subsec:hausdorff_distance}

Let \( (X, \mu) \) be a \hyperref[def:complete_metric_space]{complete metric space}.

\begin{definition}\label{def:hausdorff_distance}\mcite[144]{DontchevRockafellar2014}
  Fix two sets \( E \subseteq X \) and \( F \subseteq X \).

  The \term{excess} of \( E \) beyond \( F \) is defined as
  \begin{balign*}
     & e: \pow X \times \pow X \to \BbbR \cup \{ \infty \} \\
     & e(E, F) \coloneqq \begin{cases}
      +\infty,                                                                                    & E = \varnothing, D = \varnothing                      \\
      0,                                                                                          & E = \varnothing, D \neq \varnothing                   \\
      \sup_{x \in E} \op{dist}(x, F) \reloset{*}{=} \inf \{\delta > 0 \colon E \subseteq F_\delta \}, & E \neq \varnothing \nonumber\refstepcounter{equation}
    \end{cases}
  \end{balign*}
  where \( F_\delta \coloneqq \{ y \in X \colon \op{dist}(y, F) \leq \delta \} \).

  The \term{Pompeiu-Hausdorff distance} or simply \term{Hausdorff} distance between them is then defined as
  \begin{equation*}
    h(E, F) \coloneqq \max\{ e(E, F), e(F, E) \} = \inf \{\delta > 0 \colon E \subseteq F_\delta, F \subseteq E_\delta \}.
  \end{equation*}
\end{definition}
\begin{proof}(of the equality \( * \))
  Note that the set
  \begin{equation*}
    F_{e(E, F)} = \{ x \in X \colon \op{dist}(x, F) \leq \sup_{x \in E} \op{dist}(x, F) \}
  \end{equation*}
  obviously includes \( E \).

  Now let \( \delta > 0 \) be any real number that satisfies \( E \subseteq F_\delta \), i.e.
  \begin{equation*}
    E \subseteq F_\delta = \{ x \in X \colon \op{dist}(x, F) \leq \delta \},
  \end{equation*}
  which implies that
  \begin{equation*}
    e(E, F) = \sup_{x \in E} \op{dist}(x, F) \leq \delta.
  \end{equation*}
\end{proof}

\begin{proposition}\label{thm:hausdorff_distance_is_metric}
  The Hausdorff distance is a metric on the nonempty compact subsets of \( X \).
\end{proposition}
\begin{proof}
  Let \( E \), \( F \) and \( G \) be nonempty compact subsets of \( X \).

  The function \( h \) is nonnegative. Since we exclude empty and unbounded sets, We do not care about infinite values.

  \SubProofOf{def:metric_space/M1} Obviously \( h(E, E) = 0 \). If \( h(E, F) = 0 \), then there exists no point of \( E \) outside \( F \) and vice versa, hence \( E = F \).

  \SubProofOf{def:metric_space/M2} This follows from the symmetry of the \( \max \) function.

  \SubProofOf{def:metric_space/M3} For any point \( y \in X \), we have
  \begin{balign*}
    \op{dist}(x, G)
    =
    \inf_{z \in G} \mu(x, z)
    \leq
    \mu(x, y) + \inf_{y \in G} \mu(y, z)
    =
    \mu(x, y) + \op{dist}(y, G).
  \end{balign*}

  Select \( y \in F \) that minimizes the distance \( \mu(x, y) \) over \( F \) (compactness allows us), so that \todo{Prove \hyperref[thm:weierstrass_extreme_value_theorem]{Weierstrass' theorem}}
  \begin{balign*}
    \op{dist}(x, G)
    \leq
    \mu(x, y) + \op{dist}(y, G)
    =
    \op{dist}(x, F) + \op{dist}(y, G)
    \leq
    \op{dist}(x, F) + e(F, G).
  \end{balign*}

  It now follows that
  \begin{balign*}
    e(E, G)
     & =
    \inf \{\delta > 0 \colon E \subseteq G_\delta \}
    =    \\ &=
    \inf \{\delta > 0 \colon E \subseteq \{ x \in X \colon \op{dist}(x, G) \leq \delta \}
    \leq \\ &\leq
    \inf \{\delta > 0 \colon E \subseteq \{ x \in X \colon \op{dist}(x, F) + e(F, G) \leq \delta, y \in X \}
    =    \\ &=
    e(F, G) + \inf \{\delta > 0 \colon E \subseteq F_\delta \}
    =    \\ &=
    e(F, G) + e(E, F).
  \end{balign*}
\end{proof}
