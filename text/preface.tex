\addchap{Preface}

This ever-expanding monograph started as a set of study notes and exercises and gradually outgrew itself to become more encyclopedic. Having all these notes in one place is quite helpful for both expressing my own thoughts clearly and for later reference. It is also helpful for tracking connections between seemingly unrelated concepts --- the entire monograph is hyperlinked. Even though I aim at understanding every concept in the way it is meant to be used, most concepts are presented insomuch as they are relevant to my previous experience.

Most of the document contains concepts that are new to me at the time of writing, and thus it is naturally rough around the edges. There are a lot of errors, both insignificant and substantial, so read with caution. The document claims no referential nor pedagogical value. Everything is written at the level of abstraction I am comfortable with. Furthermore, some sections in the monograph are much more polished than others. Feel free to contact me if something in this monograph happens to distress you.

I tried putting citations on as many things as possible. The citations themselves are usually put in the left margin. If there is no citation on a definition or theorem, that means that I have either recalled it from memory or discovered it on my own. I try to mark my own definitions to distinguish them from one taken from other sources. Many of the unoriginal definitions and theorems are restated. The simple proofs are mostly original, and the difficult ones are, often loosely, based on proofs from the places cited.

Last but not least, some auxiliary programs are provided in the \identifier{src/} subdirectory within the monograph's source. Staying close to the monograph exposition is a priority over performance or code quality considerations.

Even though the document starts by building the familiar number systems, from a formalist perspective, it should start at \fullref{ch:mathematical_logic}.

\begin{concept}\label{con:assumed_knowledge}
  It may seem that some of our formalization endeavors are silly, but they are merely a result of the following two driving forces:
  \begin{itemize}
    \item We strive to ensure that the concepts we use reflect the usage in the literature, or at least do not deviate far. This also leads to literature reviews like \cref{rem:sigma_algebra_terminology} and \cref{rem:graph_walk_terminology}.

    \item We wish to clarify the \hyperref[con:metalogic]{metatheory} as much as possible while remaining informal in our general treatment. This requires us to minimize the amount of implicit assumptions.
  \end{itemize}

  To raise our confidence, we try to provide references for all definitions, with explicit commentary in case we divert from the literature. A difficulty quickly arises --- a lot of concepts are so ubiquitous that authors do not bother defining them. For example, see our comments in \cref{def:equation} regarding equations or in \cref{def:linear_combination} regarding linear combinations.

  We will refer to such notions as \term{assumed knowledge}.
\end{concept}
