\section{Propositional natural deduction}\label{sec:propositional_natural_deduction}

\paragraph{Inference rules}

\begin{definition}\label{def:inference_rule_alphabet}\mimprovised
  The \hyperref[def:formal_language/alphabet]{alphabet} of \hyperref[con:improper_symbol]{improper symbols} for inference rules consists of the following:

  \begin{thmenum}
    \thmitem{def:inference_rule_alphabet/connective} The \hyperref[def:sequent]{sequent relation} symbol \enquote{\( \synVdash \)}.
    \thmitem{def:inference_rule_alphabet/aux} The following auxiliary symbols:
    \begin{thmenum}
      \thmitem{def:inference_rule_alphabet/aux/parentheses} Parentheses \enquote{\( ( \)} and \enquote{\( ) \)} for specifying rule names.

      \thmitem{def:inference_rule_alphabet/aux/brackets} Brackets \enquote{\( [ \)} and \enquote{\( ] \)} for specifying attachments in natural deduction rules.

      \thmitem{def:inference_rule_alphabet/aux/comma} The comma \enquote{\( , \)} for delimiting premises.
    \end{thmenum}
  \end{thmenum}
\end{definition}

\begin{definition}\label{def:inference_rule}\mimprovised
  An \term[ru=правило вывода (\cite[def. 1.2.2]{Герасимов2014Вычислимость}), en=inference rule (\cite[2]{MartinLöf1996LogicalLaws})]{inference rule} is a string generated by the corresponding grammar rule from the \hyperref[def:formal_grammar/schema]{grammar schema}
  \begin{bnf*}
    \bnfprod{entry list}     {\bnfpn{entry} \bnfor \bnfpn{entry} \bnfsp \bnftsq{,} \bnfsp \bnfpn{entry list}}, \\
    \bnfprod{premises}       {\bnfves \bnfor \bnfpn{entry list}}, \\
    \bnfprod{conclusion}     {\bnfpn{entry}}, \\
    \bnfprod{inference rule} {\bnftsq{(} \bnfsp \bnfpn{rule name} \bnfsp \bnftsq{)} \bnfsp \bnfpn{premises} \bnfsp \bnftsq{\( \synVdash \)} \bnfsp \bnfpn{conclusion}}
  \end{bnf*}

  Similarly to \cref{def:logical_context}, we have purposely not specified rules for the nonterminal \( \bnfpn{entry} \) in order to encompass entries with different syntax. We have also not specified a precise syntax for rule names because we use a variety of symbols and there is little benefit in listing them explicitly here.

  Following \incite[181]{Gentzen1935LogischeSchließen}, we will use the notation
  \begin{equation*}
    \begin{prooftree}
      \hypo{ \Phi_1 }
      \hypo{ \cdots }
      \hypo{ \Phi_n }
      \infer3[\logic{R}]{ \Psi }
    \end{prooftree}
  \end{equation*}
  for the rule
  \begin{equation*}
    (\logic{R}) \enspace \Phi_1, \ldots, \Phi_n \synVdash \Psi.
  \end{equation*}

  We call \( n \) the \term{arity} of the rule \( \logic{R} \).
\end{definition}
\begin{comments}
  \item This definition is based on Gentzen's syntax for rules in \cite[181]{Gentzen1935LogischeSchließen}, but formalized via grammars and not tied to either formulas nor sequents.

  \item Here \( \Phi_1, \ldots, \Phi_n \) and \( \Psi \) are metalogical variables representing placeholders for concrete entries.

  \item A somewhat similar syntax, albeit also informal, can be found in Barendregt's works \cite{Barendregt1984LambdaCalculus} and \cite{Barendregt1992LambdaCalculiWithTypes}, who expresses some of his rules linearly, with \enquote{\( {\implies} \)} as the connective. For example, the typing rules in \cite[def. 3.1.3]{Barendregt1992LambdaCalculiWithTypes} are presented in both linear and tree-like style.

  \item If we do not want to be explicit about syntax, rather than defining \hyperref[def:relation]{relations} based on inference rules, as in \fullref{thm:recursively_defined_relations}, we can define rules as relations. This latter is done by \incite[def. 1.2.2]{Герасимов2014Вычислимость} and \incite[115]{CitkinMuravitsky2022ConsequenceRelations}. Other authors only use the notion of inference rules informally.
\end{comments}

\begin{remark}\label{rem:inference_rules_usage}
  We have introduced inference rules in \cref{def:inference_rule} as strings with the intention to use them in vastly differing situations:
  \begin{itemize}
    \item We use inference rules for expressing the possibilities of constructing proof trees.

    We discuss proof trees in general in \cref{def:proof_tree}. Natural deduction systems are presented in \cref{def:propositional_natural_deduction} for propositional logic, in \cref{def:hol_consequence} for higher-order logic and in \cref{def:fol_natural_deduction} for first-order logic.

    We also describe sequent calculus rules in \cref{def:abstract_sequent_calculus_system} and discuss how natural deduction rules can be presented in \enquote{sequent calculus style} in \cref{rem:natural_deduction_in_sequent_style}.

    Using inference rules for logical rules is done in
    \cite{Gentzen1935LogischeSchließen},
    \cite{TroelstraSchwichtenberg2000BasicProofTheory},
    \cite{VanDalen2004LogicAndStructure},
    \cite{Mimram2020ProgramEqualsProof},
    \cite{КолмогоровДрагалин2006Логика} and
    \cite{Герасимов2014Вычислимость}.

    \item We use inference rules to succinctly define relations via \fullref{thm:recursively_defined_relations}, whose usage is demonstrated in \cref{ex:recursively_defined_relation}. This will be very useful in \fullref{sec:lambda_term_alpha_equivalence} and \fullref{sec:lambda_term_reductions}.

    Among others, such usage can also be found in \cite{Mimram2020ProgramEqualsProof} for \( \beta \)- and \( \eta \)-reduction, as well as \cite{MartinLöf1984IntuitionisticTypeTheory} and \cite{UnivalentFoundationsProgram2013HoTT} for \hyperref[rem:type_theory_rule_classification/equality]{judgmental equality rules}.

    \item We use inference rules to express \hyperref[con:typing_rule]{typing rules} in \fullref{sec:simply_typed_lambda_terms}, \fullref{sec:curry_howard_correspondence} and \fullref{sec:dependent_types}, as well as \fullref{sec:higher_order_logic}.

    See \cref{rem:dependent_type_rule_formalization} and \cref{rem:dependent_type_rule_sequents} for some subtleties of typing rules.
  \end{itemize}
\end{remark}

\begin{definition}\label{def:proof_tree}
  Given a collection of \hyperref[def:inference_rule]{inference rules}, we can define a family of \hyperref[def:labeled_tree]{labeled trees}, which we will call \term{proof trees}.

  We will assume that the entries of the rules are \hyperref[con:schemas_and_instances]{schemas} whose instances are entries of the proof trees. Thus, we suppose we are given an atomic instantiation \( \BbbI \), extended to handle arbitrary entries.

  In their most basic form, the labels of proof trees consist of the following objects:
  \begin{thmenum}[series=def:proof_tree]
    \thmitem{def:proof_tree/conclusion} An entry, called the \term[en=conclusion (\cite[22]{TroelstraSchwichtenberg2000BasicProofTheory})]{conclusion} of the proof.

    \thmitem{def:proof_tree/rule_name} An empty string or the name of a rule.
  \end{thmenum}

  We define two kinds of proof trees:
  \begin{thmenum}[resume=def:proof_tree]
    \thmitem{def:proof_tree/assumption} An \term{assumption} tree for \( \varphi \) is simply a \hyperref[def:canonical_singleton_tree]{canonical singleton tree} with conclusion \( \varphi \) and an empty rule name.

    \thmitem{def:proof_tree/application} We now define \term{rule application} trees.

    Consider the rule
    \begin{equation*}
      \begin{prooftree}
        \hypo{ \Phi_1 }
        \hypo{ \cdots }
        \hypo{ \Phi_n }
        \infer3[\logic{R}]{ \Psi }
      \end{prooftree}
    \end{equation*}

    Suppose we already have proof trees \( P_1, \ldots, P_n \) such that, for \( i = 1, \ldots, n \), the conclusion of \( P_i \) is \( \Phi_i[\BbbI] \).

    We may add additional content to the labels or even disallow rule application depending on the concrete entries and on the purpose of the trees.

    In the end, we can define the rule application tree by \hyperref[def:ordered_tree_grafting_product]{grafting} \( P_1, \ldots, P_n \) to a new root with conclusion \( \Psi[\BbbI] \) and rule name \( \logic{R} \).

    We draw this tree as follows (resembling stacked inference rules):
    \begin{equation*}
      \begin{prooftree}
        \hypo{}
        \ellipsis { \( P_1 \) } { \Phi_1[\BbbI] }

        \hypo{ \cdots }

        \hypo{}
        \ellipsis { \( P_n \) } { \Phi_n[\BbbI] }
        \infer3[\logic{R}]{ \Psi[\BbbI] }
      \end{prooftree}
    \end{equation*}
  \end{thmenum}
\end{definition}
\begin{comments}
  \item This formalism somewhat resembles \enquote{prooftrees} from \bycite{TroelstraSchwichtenberg2000BasicProofTheory}, but, given a suitable definition of schemas and instantiations, it becomes completely rigorous and explicitly encodes which inference rules have been used.
\end{comments}

\begin{remark}\label{rem:inference_rule_formalization}
  Using the fact that, in \cref{def:inference_rule}, we have formalized inference rules as strings, we are able to encode some concrete rules. Our end goal is to completely formalize the \hyperref[con:curry_howard_correspondence]{Curry-Howard correspondence} for \hyperref[def:propositional_formula]{propositional formulas} and \hyperref[def:simple_type]{simple types} in \fullref{alg:type_derivation_to_proof_tree} and \fullref{alg:proof_tree_to_type_derivation}. We will also formalize natural deduction rules for \hyperref[def:fol_formula]{first-order formulas} in \fullref{sec:first_order_natural_deduction}.

  As we shall see in this remark, other cases often require more significant effort. We will still present many other rules, but only as part of the metatheory. Handling rules like those occurring in \fullref{sec:dependent_types} requires formalization efforts that resemble the creation of a programming language.

  \begin{itemize}
    \thmitem{rem:inference_rule_formalization/proofs} The propositional \hyperref[def:natural_deduction_rule]{natural deduction rules} stated in \cref{def:propositional_natural_deduction} are encoded in the module \identifier{math.logic.deduction.classical_logic}, based on the schemas defined in \cref{def:propositional_formula_schema}.

    These rules can be applied to form proof trees as described in \cref{def:propositional_natural_deduction_proof_tree}. Constructs for building such trees can be found in the module \identifier{math.logic.deduction.proof_tree}.

    We will also extend proof trees for first-order logic rules in \cref{def:fol_natural_deduction}.

    \thmitem{rem:inference_rule_formalization/types} The \hyperref[def:simple_typing_rule]{simple typing rules} stated in \fullref{sec:simply_typed_lambda_terms} and \fullref{sec:curry_howard_correspondence} are encoded in the module \identifier{math.lambda_.type_system}.

    Similarly to the above, the rules can be used to form derivation trees as described in \cref{def:type_derivation_tree}, with an implementation in \identifier{math.lambda_.type_derivation.tree}.

    Formalizing well-formed contexts and types, judgmental equality and side conditions for the typing rules from \fullref{sec:dependent_types} however requires great effort, and we will not bother with those. Several impediments are highlighted in \cref{rem:dependent_type_rule_formalization}.
  \end{itemize}
\end{remark}

\begin{concept}\label{con:inference_rule_admissibility}\mimprovised
  We say that the \hyperref[def:inference_rule]{inference rule} \( R \) is \term[ru=допустимое (правило) (\cite[53]{Герасимов2014Вычислимость}), en=admissible (rule) (\cite[76]{TroelstraSchwichtenberg2000BasicProofTheory})]{admissible} with respect to a family \( \mscrR \) of rules if the conclusion of \( R \) can be obtained from its premises by using the rules from \( \mscrR \).
\end{concept}
\begin{comments}
  \item The precise meaning of \enquote{obtaining} a conclusion via a rule depends on the intricacies of the corresponding \hyperref[def:proof_tree]{proof trees}. A proof of admissibility may restructure the original proof tree, so it is not simply a shorthand for a pattern of rules.

  The most important example is \cref{thm:cut_elimination}, but other admissible rules can be found in \cref{thm:minimal_implicational_logic_axioms_nd_proof}, \cref{thm:propositional_admissible_rules}, \cref{thm:alpha_equivalence_simplified}, \cref{thm:typed_substitution_assertions} and \cref{thm:reduction_typing_rules}.
\end{comments}

\paragraph{Abstract natural deduction}

\begin{definition}\label{def:natural_deduction_rule}\mimprovised
  A \term{natural deduction rule} is an \hyperref[def:inference_rule]{inference rule} whose entries can be \hyperref[con:schemas_and_instances]{formula schemas} or pairs of schemas:
  \begin{bnf*}
    \bnfprod{attached schema} {\bnftsq{[} \bnfsp \bnfpn{formula schema} \bnfsp \bnftsq{]}} \\
    \bnfprod{attachments}     {\bnfves \bnfor \bnfpn{attached schema} \bnfsp \bnfpn{attachments}} \\
    \bnfprod{entry}           {\bnfpn{attachments} \bnfsp \bnfpn{formula schema}}
  \end{bnf*}

  We will use these attached schemas are used for different purposes --- namely, discharging explicit assumptions and indicating implicit premises. This is formalized in \cref{def:propositional_natural_deduction_proof_tree/application}. Some rules like \ref{inf:def:identity_type/j/elim} require more than one attachment. We refer to the numeric index of an attachment as its \term{position}.

  For the rule
  \begin{equation*}
    (\logic{R}) \enspace \Phi_1, \ldots, [\Theta_{k,1}] \ldots [\Theta_{k,m}] \thickspace \Phi_k, \ldots, \Phi_n \synVdash [\Omega_1] \ldots [\Omega_m] \thickspace \Psi
  \end{equation*}
  we use the more convenient notation
  \begin{equation*}
    \begin{prooftree}
      \hypo{ \Phi_1 }
      \hypo{ \cdots }

      \hypo{ [\Theta_{k,1}] }
      \hypo{ \cdots }
      \hypo{ [\Theta_{k,m_k}] }
      \infer[dashed]3{ \Phi_k }

      \hypo{ \cdots }
      \hypo{ \Phi_n }
      \infer[left label={\( [\Omega_1] \cdots [\Omega_m] \)}]5[\logic{R}]{ \Psi }
    \end{prooftree}
  \end{equation*}
\end{definition}
\begin{comments}
  \item Our purpose with this definition is to mechanize arbitrary natural deduction rules, breaking the tradition described in \cite[13]{MartinLöf1984IntuitionisticTypeTheory}:
  \begin{displayquote}
    In the usual natural deduction style, the rules given are not quite formal.
  \end{displayquote}

  The rules are formalized as strings so that they can be treated rigorously, and then they are specifically adjusted for natural deduction. The adjustments are based on \cite[\S 2.1]{TroelstraSchwichtenberg2000BasicProofTheory}.

  The notation is based on Gentzen's original \cite[186]{Gentzen1935LogischeSchließen}, with a dashed line added for clarity.

  \item The placeholder variables like \( \Psi \) represent schemas. For example, \( \Psi \) can be the propositional formula schema \( \syn\varphi \synimplies \syn\psi \), which are purely syntactic objects, and to reason about formulas instead we must use \fullref{alg:propositional_schema_instantiation}.

  \item The only schemas we have completely formalized are propositional formula schemas in \cref{def:propositional_formula_schema} and typed \( \muplambda \)-term schemas in \cref{def:simple_type_schema} and \cref{def:lambda_term_schema}.
\end{comments}

\begin{definition}\label{def:propositional_natural_deduction_system}\mimprovised
  An \term{abstract propositional natural deduction system} consists of a nonempty collection of \hyperref[def:natural_deduction_rule]{natural deduction rules} for \hyperref[def:propositional_formula_schema]{propositional formula schemas}. We require the rules to have distinct names, but otherwise impose no restrictions on them.
\end{definition}

\begin{definition}\label{def:propositional_natural_deduction_proof_tree}\mimprovised
  We will define \hyperref[def:proof_tree]{proof trees} for a fixed \hyperref[def:propositional_natural_deduction_system]{propositional natural deduction system}.

  Generic proof trees are assumed to have a conclusion and rule name attached. For natural deduction, we will need three additional components:
  \begin{thmenum}[series=def:propositional_natural_deduction_proof_tree]
    \thmitem{def:propositional_natural_deduction_proof_tree/open_assumptions} We will associate with every tree a set of marked formulas generated by the following grammar:
    \begin{bnf*}
      \bnfprod{marker}          {\bnfpn{Small Latin identifier}}, \\
      \bnfprod{marked formula}  {\bnfpn{marker} \bnfsp \bnftsq{:} \bnfsp \bnfpn{formula}}.
    \end{bnf*}

    We will call these formulas \term[en=open assumption (\cite[24]{TroelstraSchwichtenberg2000BasicProofTheory})]{open assumptions}. They are cumulative for the entire proof tree.

    \thmitem{def:propositional_natural_deduction_proof_tree/discharged_assumptions} Dually, we have marked formulas that are \term[en=discharged (\cite[9]{GirardEtAl1989ProofsAndTypes})]{discharged} or \term[en=closed (\cite[23]{TroelstraSchwichtenberg2000BasicProofTheory})]{closed}. It will be sufficient for us to only consider their markers.

    Unlike open assumptions, discharged ones are local to the root.

    \thmitem{def:propositional_natural_deduction_proof_tree/implicit_premises} We will also associate with every tree a set of (unmarked) formulas, which we will call \term{implicit open premises}.

    Just like ordinary premises, these will be local to the root. Even though we can collect all implicit premises across the proof tree, the notion starts to lose its meaning because an implicit premise in one subtree may be an explicit premise in another.

    We will primarily use implicit premises for handling \hyperref[def:fol_natural_deduction_proof_tree/eigenvariables]{eigenvariables} when extending proof trees to \hyperref[def:first_order_logic]{first-order logic} in \cref{def:fol_natural_deduction_proof_tree}.
  \end{thmenum}

  These additional components are maintained in proof trees as follows:
  \begin{thmenum}[resume=def:propositional_natural_deduction_proof_tree]
    \thmitem{def:propositional_natural_deduction_proof_tree/assumption} An assumption tree for a propositional formula \( \varphi \) with marker \( u \) is the \hyperref[def:canonical_singleton_tree]{canonical singleton tree} with the following label:
    \begin{thmenum}
      \thmitem{def:propositional_natural_deduction_proof_tree/assumption/conclusion} The conclusion is \( \varphi \).
      \thmitem{def:propositional_natural_deduction_proof_tree/assumption/rule_name} The rule name is the empty string.
      \thmitem{def:propositional_natural_deduction_proof_tree/assumption/open_assumptions} There is only one open assumption --- \( \varphi \) marked with \( u \).
      \thmitem{def:propositional_natural_deduction_proof_tree/assumption/discharged_assumptions} There are no discharged assumptions.
      \thmitem{def:propositional_natural_deduction_proof_tree/assumption/implicit_premises} There are no implicit open premises.
    \end{thmenum}

    We will denote this a tree as follows:
    \begin{equation*}
      [\varphi]^u.
    \end{equation*}

    \thmitem{def:propositional_natural_deduction_proof_tree/application} A rule application tree collects all open assumptions from its premises and optionally discharged some of them. Consider the rule
    \begin{equation*}
      \begin{prooftree}
        \hypo{ \Phi_1 }
        \hypo{ \cdots }

        \hypo{ [\Theta_{k,1}] }
        \hypo{ \cdots }
        \hypo{ [\Theta_{k,m_k}] }
        \infer[dashed]3{ \Phi_k }

        \hypo{ \cdots }
        \hypo{ \Phi_n }
        \infer[left label={\( [\Omega_1] \cdots [\Omega_m] \)}]5[\logic{R}]{ \Psi }
      \end{prooftree}
    \end{equation*}

    As per \cref{def:proof_tree/application}, fix an \hyperref[def:atomic_propositional_instantiation]{atomic schema instantiation} \( \BbbI \) compatible with all schemas of the rule. Fix also a list \( (P_1, \ldots, P_n) \) of proof trees such that, for \( k = 1, \ldots, n \), the conclusion of \( P_k \) is \( \Phi_k[\BbbI] \). We must \hyperref[def:ordered_tree_grafting_product]{graft} \( P_1, \ldots, P_n \) to a new root to obtain a rule application tree for \( \logic{R} \) with conclusion \( \Psi[\BbbI] \).

    \begin{thmenum}[series=def:propositional_natural_deduction_proof_tree/application]
      \thmitem{def:propositional_natural_deduction_proof_tree/application/dischargeable} With every attached schema \( [\Theta_{k,i}] \) of every premise \( \Phi_k \), we associate a set \( D_{k,i} \) of \term{dischargeable} assumptions --- those open assumptions \( [\varphi]^u \) from the premise subtree \( P_k \) for which \( \varphi = \Theta_{k,i}[\BbbI] \).

      To handle the nondeterminism with choosing which assumptions to discharge, we associate with our rule application tree a partial \hyperref[def:choice_function]{choice function} \( d \) on \( \seq{ D_{k,i} }_{k=1,i=1}^{n,m_k} \) (undefined only when \( D_{k,i} \) is empty).

      We require \( d \) to simultaneously discharge all instances of a given open assumption across all subtrees. If this is not possible, we consider \( d \) to be invalid. This restriction helps prevent name collisions in some dependent typing rules like \ref{inf:def:identity_type/j/elim}.
    \end{thmenum}

    For every choice \( d \) of assumptions to discharge, we associate a unique proof tree \( P_d \):
    \begin{thmenum}[resume=def:propositional_natural_deduction_proof_tree/application]
      \thmitem{def:propositional_natural_deduction_proof_tree/application/conclusion} The conclusion is \( \Psi[\BbbI] \).
      \thmitem{def:propositional_natural_deduction_proof_tree/application/rule_name} The rule name is \( \logic{R} \).
      \thmitem{def:propositional_natural_deduction_proof_tree/application/open_assumptions} We collect all open assumptions from the immediate subtrees and remove those discharged by \( d \). Namely, let
      \begin{equation*}
        O_d \coloneqq \bigcup_{k=1}^n \set[\big]{ [\varphi]^u \in O_k \given d(D_{k,i}) \neq [\varphi]^u \T{for} i=1, \ldots, m_k },
      \end{equation*}
      where \( O_k \) is the set of open assumptions of the premise subtree \( P_k \).

      \thmitem{def:propositional_natural_deduction_proof_tree/application/discharged_assumptions} The set of discharged markers is simply
      \begin{equation*}
        D_d \coloneqq \set{ u \given d(D_{k_i}) = [\varphi]^u }.
      \end{equation*}

      \thmitem{def:propositional_natural_deduction_proof_tree/application/implicit_premises} For the implicit open premises, we collect the attached schemas \( \Omega_1, \ldots, \Omega_n \) of the conclusion, instantiate them, and remove those that are discharged by \( d \):
      \begin{equation*}
        I_d \coloneqq \set{ \Omega_j[\BbbI] \given j = 1, \ldots, m \T{and} \Omega_j[\BbbI] \neq \varphi \T{for every discharged} [\varphi]^u }.
      \end{equation*}
    \end{thmenum}

    Discharging requires some elucidation, in part motivated from similarly-structured \hyperref[def:type_derivation_tree]{type derivation trees}. These are implemented programmatically in the function \identifier{math.logic.deduction.proof_tree.apply} in \cite{notebook:code}:
    \begin{itemize}
      \item In order to discharge an assumption (i.e. in order for \( D_{k,i} \) to be nonempty), two conditions must be satisfied  --- \( \Phi_i \) must have a bracketed formula \( [\Theta_{k,i}] \) attached and \( \Theta_{k,i}[\BbbI] \) must match at least one open assumption of \( P_i \).

      If the latter fails, the rule assumption \( \Theta_{k,i}[\BbbI] \) is satisfied vacuously. For type derivation trees, \incite[note 2A8.1]{Hindley1997BasicSTT} calls this \enquote{discharging vacuously}. \Cref{ex:def:propositional_natural_deduction/efq_vs_dne} demonstrates this, and a more practical example can be found in the proof of \cref{thm:simple_algebraic_type_arithmetic/equivalence}.

      \item If a premise has dischargeable assumptions, at least one of them \hi{must} be discharged\fnote{This is not enforced in the code because dischargeable assumptions are expected to be provided by the user.}. It will not no longer be open in the proof tree.

      When defining the entailment relation in \cref{def:propositional_natural_deduction_consequence}, for deducing \( \Gamma \vdash \varphi \), we only require \( \varphi \) to have a proof tree whose open assumptions are members of \( \Gamma \); discharged assumptions may just as well be in \( \Gamma \). But they will no longer be open in the proof tree.

      This restriction is motivated by type derivation trees, where we want to avoid confusing situations like \cref{ex:def:mltt_well_formed_context/discharging}. As discussed in \cref{rem:dependent_products_and_forall_quantifier_rules}, this particular example provides a justification for the eigenvariable conditions in predicate logic.

      \item We also disallow distinct formulas with the same marker because that would break \fullref{alg:proof_tree_to_type_derivation}. If two assumptions have the same marker, we require the corresponding formulas to coincide. This requirement is also explicit in \cite[\S 2.1.8]{TroelstraSchwichtenberg2000BasicProofTheory}.

      \item It is possible to discharge all dischargeable assumptions irrespective of their markers. \incite[\S 2.1.9]{TroelstraSchwichtenberg2000BasicProofTheory} call this the \term{complete discharge convention}. We avoid this because it complicates \fullref{alg:proof_tree_to_type_derivation}.
    \end{itemize}

    Let \( l_1, \ldots, l_k \) be the list of markers of assumptions discharged by this rule application. We draw this application tree as follows:
    \begin{equation*}
      \begin{prooftree}
        \hypo{}
        \ellipsis { \( P_1 \) } { \Phi_1[\BbbI] }

        \hypo{ \cdots }

        \hypo{}
        \ellipsis { \( P_n \) } { \Phi_n[\BbbI] }
        \infer[left label={\( l_1, \ldots, l_k \)}]3[\logic{R}]{ \Psi[\BbbI] }
      \end{prooftree}
    \end{equation*}
  \end{thmenum}
\end{definition}
\begin{comments}
  \item The general definition and the treatment of discharging are based on \cite[\S 2.1]{TroelstraSchwichtenberg2000BasicProofTheory}.

  \item Although discharging is a fundamental part of natural deduction, implicit open premises are not. In fact, they are only discussed
   in three special cases (those from \cref{def:propositional_natural_deduction}) in a passing remark in \cite[\S 2.1]{TroelstraSchwichtenberg2000BasicProofTheory} when the authors explain in detail how to handle \hyperref[def:fol_natural_deduction_proof_tree/eigenvariables]{eigenvariables}. Other authors like \cite{VanDalen2004LogicAndStructure} are less precise.

  \item Some important examples are listed in \cref{ex:def:propositional_natural_deduction}.

  \item Our formalization allows us to avoid some issues present in less formal systems; see, for example, \cref{ex:def:classical_propositional_sequent_calculus/proof_tree_non_uniqueness}.

  \item Propositional proof trees are special cases of the analogous \hyperref[def:first_order_logic]{first-order logic} trees defined in \cref{def:fol_natural_deduction_proof_tree}, implemented in the module \identifier{math.logic.deduction.proof_tree} in \cite{notebook:code}.
\end{comments}

\begin{definition}\label{def:propositional_natural_deduction_consequence}\mimprovised
  For every \hyperref[def:propositional_natural_deduction_system]{propositional natural deduction system}, we define a \hyperref[def:consequence_relation]{consequence relation} as follows: we let \( \Gamma \vdash \varphi \) if there exists a \hyperref[def:propositional_natural_deduction_proof_tree]{proof tree} with conclusion \( \varphi \) whose \hyperref[def:propositional_natural_deduction_proof_tree/open_assumptions]{open assumptions} are all in \( \Gamma \).

  In accordance with \cref{def:general_logic}, we say that \( \varphi \) is \term{derivable} from \( \Gamma \) if this relation holds.
\end{definition}
\begin{comments}
  \item We lose almost all information about the concrete proof by using such an entailment relation. Here \( \Gamma \) is a set of formulas somehow deriving \( \varphi \), and we only know that there exists a proof tree whose open assumptions include those from \( \Gamma \) with some label.
\end{comments}
\begin{defproof}
  We must show that \( {\vdash} \) satisfies the conditions for being a consequence relation.

  \SubProofOf[def:consequence_relation/reflexivity]{reflexivity} If \( \varphi \in \Gamma \), then, for any marker \( u \), the tree \( [\varphi]^u \) proves \( \varphi \) from \( \Gamma \).

  \SubProofOf[def:consequence_relation/monotonicity]{monotonicity} If \( \Gamma \vdash \varphi \), then there exists a proof tree deriving \( \varphi \) from \( \Gamma \). Such a tree also proves \( \varphi \) from any superset of \( \Gamma \).

  \SubProofOf[def:consequence_relation/transitivity]{transitivity} Suppose that \( \Delta, \Epsilon \vdash \varphi \) and that, for every \( \psi \in \Delta \), we have \( \Gamma, \Epsilon \vdash \psi \).

  Let \( P_\varphi \) be a proof tree deriving \( \varphi \) from \( \Delta \cup \Epsilon \) and let \( P_\psi \) be a proof tree deriving \( \psi \) from \( \Gamma \cup \Epsilon \).

  Then, for any formula \( \psi \in \Delta \) and any marker \( u \), we can replace the assumption \( [\psi]^u \) in \( P_\varphi \) with \( P_\psi \).

  The resulting tree will have the same conclusion as \( P_\varphi \) --- namely, \( \varphi \) --- but different open assumptions --- namely, members of \( \Gamma \cup \Epsilon \).
\end{defproof}

\begin{proposition}\label{thm:propositional_natural_deduction_derivation_compact}
  Every \hyperref[def:propositional_natural_deduction_consequence]{natural deduction entailment relation} is \hyperref[def:consequence_relation/compactness]{compact}.
\end{proposition}
\begin{comments}
  \item This is one of several compactness theorems presented here --- see \cref{rem:logical_compactness_theorems}.
\end{comments}
\begin{proof}
  A proof tree is can have at most finitely many assumptions.
\end{proof}

\paragraph{Axiomatic derivations via natural deduction}

\begin{definition}\label{def:axiomatic_derivation_as_natural_deduction}\mimprovised
  We can regard any (propositional) \hyperref[def:propositional_axiomatic_derivation_system]{axiomatic derivation system} as a \hyperref[def:propositional_natural_deduction_system]{natural deduction system}. For this, suppose that each axiom schema \( \Phi \) has a name \( \logic{R}_\Phi \).

  \begin{thmenum}
    \thmitem{def:axiomatic_derivation_as_natural_deduction/axiom} For the axiom schema \( \Phi \), we introduce the rule
    \begin{equation*}
      \begin{prooftree}
        \infer0[\ensuremath{ \logic{R}_\Phi }]{ \Phi }
      \end{prooftree}
    \end{equation*}

    Note that \( \Phi \) here is a metalogical variable that refers to a \hyperref[def:propositional_formula_schema]{formula schema}.

    \thmitem{def:axiomatic_derivation_as_natural_deduction/mp} For deducing proofs, we introduce the following rule, called \term{modus ponens}:
    \begin{equation*}\taglabel[\ensuremath{ \logic{MP} }]{inf:def:axiomatic_derivation_as_natural_deduction/mp}
      \begin{prooftree}
        \hypo{ \syn\varphi \synimplies \syn\psi }
        \hypo{ \syn\varphi }
        \infer2[\ref{inf:def:axiomatic_derivation_as_natural_deduction/mp}]{ \syn\psi }
      \end{prooftree}
    \end{equation*}

    Note that \( \syn\varphi \synimplies \syn\psi \) here refers to a concrete formula schema.
  \end{thmenum}
\end{definition}
\begin{comments}
  \item According to \cite[75]{Rosen2019DiscreteMathematics}, \enquote{modus ponens} is Latin for \enquote{mode that affirms}.
  \item If we are willing to break our convention of rules having unique names, we may use a generic name such as \( \logic{Ax} \) for each schema.
\end{comments}

\begin{example}\label{ex:minimal_implication_logic_identity/trees}
  Regard \hyperref[def:minimal_implication_logic]{minimal implicational logic} as a \hyperref[def:propositional_natural_deduction_system]{natural deduction system} in accordance with \cref{def:axiomatic_derivation_as_natural_deduction}.

  The two axiom schemas --- \eqref{eq:def:minimal_implication_logic/intro} and \eqref{eq:def:minimal_implication_logic/dist} --- become inference rules:
  \columnratio{0.3,0.7}
  \begin{paracol}{2}
    \begin{leftcolumn}
      \ParacolAlignmentHack
      \begin{equation*}\taglabel[\ensuremath{ \rightarrow^+ }]{inf:ex:minimal_implication_logic_identity/trees/intro}
        \begin{prooftree}
          \infer0[\ref{inf:ex:minimal_implication_logic_identity/trees/intro}]{ \syn\varphi \synimplies (\syn\psi \synimplies \syn\varphi) }
        \end{prooftree}
      \end{equation*}
    \end{leftcolumn}

    \begin{rightcolumn}
      \ParacolAlignmentHack
      \begin{equation*}\taglabel[\ensuremath{ \twoheadrightarrow }]{inf:ex:minimal_implication_logic_identity/trees/dist}
        \begin{prooftree}
          \infer0[\ref{inf:ex:minimal_implication_logic_identity/trees/dist}]{ \parens[\big]{ \syn\varphi \synimplies (\syn\psi \synimplies \syn\theta) } \synimplies \parens[\big]{ (\syn\varphi \synimplies \syn\psi) \synimplies (\syn\varphi \synimplies \syn\theta) } }
        \end{prooftree}
      \end{equation*}
    \end{rightcolumn}
  \end{paracol}
  \columnratio{}

  Then \cref{ex:minimal_implication_logic_identity/derivations} can be presented as follows:
  \begin{equation*}
    \begin{prooftree}[separation=3em]
      \infer0[\ref{inf:ex:minimal_implication_logic_identity/trees/intro}]{ \ref{ex:minimal_implication_logic_identity/trees/dagger} }
      \infer0[\ref{inf:ex:minimal_implication_logic_identity/trees/dist}]
        {
          \ref{ex:minimal_implication_logic_identity/trees/dagger}
          \synimplies ((\varphi \synimplies (\varphi \synimplies \varphi)) \synimplies (\varphi \synimplies \varphi))
        }

      \infer2[\ref{inf:def:axiomatic_derivation_as_natural_deduction/mp}]{(\varphi \synimplies (\varphi \synimplies \varphi)) \synimplies (\varphi \synimplies \varphi)}

      \infer0[\ref{inf:ex:minimal_implication_logic_identity/trees/intro}]{ \varphi \synimplies (\varphi \synimplies \varphi) }

      \infer2[\ref{inf:def:axiomatic_derivation_as_natural_deduction/mp}]{\varphi \synimplies \varphi}
    \end{prooftree}
  \end{equation*}
  where
  \begin{equation}\label{ex:minimal_implication_logic_identity/trees/dagger}
    \varphi \synimplies ((\varphi \synimplies \varphi) \synimplies \varphi). \tag{\ensuremath{ \dagger }}
  \end{equation}
\end{example}

\begin{algorithm}[Axiomatic derivation to proof tree]\label{alg:propositional_axiomatic_derivation_to_proof_tree}
  Fix an \hyperref[def:propositional_axiomatic_derivation_system]{axiomatic derivation system} and regard it as a \hyperref[def:propositional_natural_deduction_system]{natural deduction system} in accordance with \cref{def:axiomatic_derivation_as_natural_deduction}.

  Fix a derivation \( \psi_1, \ldots, \psi_n \) of \( \varphi \). We will build a proof tree whose conclusion is \( \varphi \) and whose \hyperref[def:propositional_natural_deduction_proof_tree/open_assumptions]{open assumptions} are the premises of the derivation.

  \begin{thmenum}
    \thmitem{alg:propositional_axiomatic_derivation_to_proof_tree/premise} If \( \varphi \) itself is a premise, then, for a fixed marker \( u \), we declare the result of the algorithm to be the assumption tree \( [\varphi]^u \).

    \thmitem{alg:propositional_axiomatic_derivation_to_proof_tree/axiom} If \( \varphi \) is an instance of the schema \( \Phi \), we declare the result of the algorithm to be
    \begin{equation*}
      \begin{prooftree}
        \infer0[\ensuremath{ \logic{R}_\Phi }]{ \varphi }
      \end{prooftree}
    \end{equation*}

    \thmitem{alg:propositional_axiomatic_derivation_to_proof_tree/recursive} Otherwise, there must exist indices \( i \) and \( j \) such that
    \begin{equation*}
      \varphi_i = \varphi_j \synimplies \varphi_n.
    \end{equation*}

    We assume that the algorithm already correctly produces results for derivations shorter than \( n \). Denote by \( P_i \) and \( P_j \) the proof trees corresponding to \( \varphi_i \) and \( \varphi_j \).

    We declare our result of the algorithm to be the tree
    \begin{equation*}
      \begin{prooftree}
        \hypo{}
        \ellipsis { \( P_i \) } { \varphi_j \synimplies \varphi }

        \hypo{}
        \ellipsis { \( P_j \) } { \varphi_j }

        \infer2[\ref{inf:def:axiomatic_derivation_as_natural_deduction/mp}]{ \varphi }
      \end{prooftree}
    \end{equation*}
  \end{thmenum}
\end{algorithm}
\begin{comments}
  \item This algorithm can be found as \identifier{math.logic.derivation.axiomatic_derivation.derivation_to_proof_tree} in \cite{notebook:code}.
\end{comments}

\begin{algorithm}[Proof tree to axiomatic derivation]\label{alg:proof_tree_to_propositional_axiomatic_derivation}
  Again, fix an \hyperref[def:propositional_axiomatic_derivation_system]{axiomatic derivation system} and regard it as a \hyperref[def:propositional_natural_deduction_system]{natural deduction system} in accordance with \cref{def:axiomatic_derivation_as_natural_deduction}.

  Then, given a \hyperref[def:propositional_natural_deduction_proof_tree]{proof tree}, enumerating its nodes' conclusions via \hyperref[def:ordered_tree_enumeration]{post-order traversal}, we obtain an axiomatic derivation whose premises are the open assumptions of the tree, and whose conclusion coincides with that of the tree.
\end{algorithm}
\begin{comments}
  \item This algorithm can be found as \identifier{math.logic.derivation.axiomatic_derivation.proof_tree_to_derivation} in \cite{notebook:code}.
\end{comments}

\paragraph{Concrete natural deduction}

\begin{definition}\label{def:propositional_natural_deduction}\mcite[def. 2.1.1]{TroelstraSchwichtenberg2000BasicProofTheory}
  We will now describe several \hyperref[def:propositional_natural_deduction_system]{propositional natural deduction systems}. The base set of rules describes \hyperref[con:minimal_logic]{minimal logic}, while \hyperref[con:intuitionistic_logic]{intuitionistic} and \hyperref[con:classical_logic]{classical logic} additionally require \ref{inf:def:propositional_natural_deduction/bot/efq} and \ref{inf:def:propositional_natural_deduction/bot/raa}, respectively.

  \begin{thmenum}
    \thmitem{def:propositional_natural_deduction/top}\mcite[fig. 2.5]{Mimram2020ProgramEqualsProof} A single rule for introducing \hyperref[def:propositional_alphabet/constants/verum]{verum}:
    \begin{equation*}\taglabel[\ensuremath{ \top_+ }]{inf:def:propositional_natural_deduction/top/intro}
      \begin{prooftree}
        \infer0[\ref{inf:def:propositional_natural_deduction/top/intro}]{ \syntop }
      \end{prooftree}
    \end{equation*}

    \thmitem{def:propositional_natural_deduction/bot}\mcite[31]{VanDalen2004LogicAndStructure} Rules corresponding to \fullref{thm:propositional_semantic_efq} and \fullref{thm:propositional_semantic_raa}:
    \begin{paracol}{2}
      \ParacolAlignmentHack
      \begin{equation*}\taglabel[\logic{EFQ}]{inf:def:propositional_natural_deduction/bot/efq}
        \begin{prooftree}
          \hypo{ \synbot }
          \infer1[\ref{inf:def:propositional_natural_deduction/bot/efq}]{ \syn\varphi }
        \end{prooftree}
      \end{equation*}

      \switchcolumn

      \ParacolAlignmentHack
      \begin{equation*}\taglabel[\logic{RAA}]{inf:def:propositional_natural_deduction/bot/raa}
        \begin{prooftree}
          \hypo{ [\synneg \syn\varphi] }
          \infer[dashed]1{ \synbot }
          \infer1[\ref{inf:def:propositional_natural_deduction/bot/raa}]{ \syn\varphi }
        \end{prooftree}
      \end{equation*}

      \switchcolumn*

      We add this rule to the system only when working with \hyperref[con:intuitionistic_logic]{intuitionistic object logic}.

      \switchcolumn

      We add this rule to the system only when working with \hyperref[con:classical_logic]{classical object logic}.
    \end{paracol}

    \thmitem{def:propositional_natural_deduction/negation}\mcite[fig. 2.5]{Mimram2020ProgramEqualsProof} Rules for expressing \hyperref[def:propositional_alphabet/negation]{negation} via the \hyperref[def:propositional_alphabet/constants/falsum]{falsum}:
    \begin{paracol}{2}
      \begin{leftcolumn}
        \ParacolAlignmentHack
        \begin{equation*}\taglabel[\ensuremath{ \neg_+ }]{inf:def:propositional_natural_deduction/neg/intro}
          \begin{prooftree}
            \hypo{ [\syn\varphi] }
            \infer[dashed]1{ \synbot }
            \infer1[\ref{inf:def:propositional_natural_deduction/neg/intro}]{ \synneg \syn\varphi }
          \end{prooftree}
        \end{equation*}
      \end{leftcolumn}

      \begin{rightcolumn}
        \ParacolAlignmentHack
        \begin{equation*}\taglabel[\ensuremath{ \neg_- }]{inf:def:propositional_natural_deduction/neg/elim}
          \begin{prooftree}
            \hypo{ \synneg \syn\varphi }
            \hypo{ \syn\varphi }
            \infer2[\ref{inf:def:propositional_natural_deduction/neg/elim}]{ \synbot }
          \end{prooftree}
        \end{equation*}
      \end{rightcolumn}
    \end{paracol}

    \thmitem{def:propositional_natural_deduction/and}\mcite[fig. 2.5]{Mimram2020ProgramEqualsProof} Rules for introducing and eliminating \hyperref[def:propositional_alphabet/connectives/conjunction]{conjunctions}:
    \begin{paracol}{3}
      \begin{nthcolumn}{0}
        \ParacolAlignmentHack
        \begin{equation*}\taglabel[\ensuremath{ \wedge_+ }]{inf:def:propositional_natural_deduction/and/intro}
          \begin{prooftree}
            \hypo{ \syn\varphi }
            \hypo{ \syn\psi }
            \infer2[\ref{inf:def:propositional_natural_deduction/and/intro}]{ \syn\varphi \synwedge \syn\psi }
          \end{prooftree}
        \end{equation*}
      \end{nthcolumn}

      \begin{nthcolumn}{1}
        \ParacolAlignmentHack
        \begin{equation*}\taglabel[\ensuremath{ \wedge_{-L} }]{inf:def:propositional_natural_deduction/and/elim_left}
          \begin{prooftree}
            \hypo{ \syn\varphi \synwedge \syn\psi }
            \infer1[\ref{inf:def:propositional_natural_deduction/and/elim_left}]{ \syn\varphi }
          \end{prooftree}
        \end{equation*}
      \end{nthcolumn}

      \begin{nthcolumn}{2}
        \ParacolAlignmentHack
        \begin{equation*}\taglabel[\ensuremath{ \wedge_{-R} }]{inf:def:propositional_natural_deduction/and/elim_right}
          \begin{prooftree}
            \hypo{ \syn\varphi \synwedge \syn\psi }
            \infer1[\ref{inf:def:propositional_natural_deduction/and/elim_right}]{ \syn\psi }
          \end{prooftree}
        \end{equation*}
      \end{nthcolumn}
    \end{paracol}

    \thmitem{def:propositional_natural_deduction/or} Rules for introducing and eliminating \hyperref[def:propositional_alphabet/connectives/disjunction]{disjunctions}:
    \begin{paracol}{3}
      \begin{nthcolumn}{0}
        \ParacolAlignmentHack
        \begin{equation*}\taglabel[\ensuremath{ \vee_{+L} }]{inf:def:propositional_natural_deduction/or/intro_left}
          \begin{prooftree}
            \hypo{ \syn\varphi }
            \infer[left label=\( [\syn\psi] \)]1[\ref{inf:def:propositional_natural_deduction/or/intro_left}]{ \syn\varphi \synvee \syn\psi }
          \end{prooftree}
        \end{equation*}
      \end{nthcolumn}

      \begin{nthcolumn}{1}
        \ParacolAlignmentHack
        \begin{equation*}\taglabel[\ensuremath{ \vee_{+R} }]{inf:def:propositional_natural_deduction/or/intro_right}
          \begin{prooftree}
            \hypo{ \syn\psi }
            \infer[left label=\( [\syn\varphi] \)]1[\ref{inf:def:propositional_natural_deduction/or/intro_right}]{ \syn\varphi \synvee \syn\psi }
          \end{prooftree}
        \end{equation*}
      \end{nthcolumn}

      \begin{nthcolumn}{2}
        \ParacolAlignmentHack
        \begin{equation*}\taglabel[\ensuremath{ \vee_- }]{inf:def:propositional_natural_deduction/or/elim}
          \begin{prooftree}
            \hypo{ \syn\varphi \synvee \syn\psi }

            \hypo{ [\syn\varphi] }
            \infer[dashed]1{ \syn\theta }

            \hypo{ [\syn\psi] }
            \infer[dashed]1{ \syn\theta }

            \infer3[\ref{inf:def:propositional_natural_deduction/or/elim}]{ \syn\theta }
          \end{prooftree}
        \end{equation*}
      \end{nthcolumn}
    \end{paracol}

    \thmitem{def:propositional_natural_deduction/imp} Rules for introducing and eliminating \hyperref[def:propositional_alphabet/connectives/conditional]{conditionals}:
    \begin{paracol}{2}
      \begin{leftcolumn}
        \ParacolAlignmentHack
        \begin{equation*}\taglabel[\ensuremath{ \rightarrow_+ }]{inf:def:propositional_natural_deduction/imp/intro}
          \begin{prooftree}[center=false]
            \hypo{ [\syn\varphi] }
            \infer[dashed]1{ \syn\psi }
            \infer[left label=\( [\syn\varphi] \)]1[\ref{inf:def:propositional_natural_deduction/imp/intro}]{ \syn\varphi \synimplies \syn\psi }
          \end{prooftree}
        \end{equation*}
      \end{leftcolumn}

      \begin{rightcolumn}
        \ParacolAlignmentHack
        \begin{equation*}\taglabel[\ensuremath{ \rightarrow_- }]{inf:def:propositional_natural_deduction/imp/elim}
          \begin{prooftree}[center=false]
            \hypo{ \syn\varphi \synimplies \syn\psi }
            \hypo{ \syn\varphi }
            \infer2[\ref{inf:def:propositional_natural_deduction/imp/elim}]{ \syn\psi }
          \end{prooftree}
        \end{equation*}
      \end{rightcolumn}
    \end{paracol}

    \thmitem{def:propositional_natural_deduction/iff}\mcite[99]{КолмогоровДрагалин2006Логика} Rules for introducing and eliminating \hyperref[def:propositional_alphabet/connectives/biconditional]{biconditionals}:
    \begin{paracol}{3}
      \begin{leftcolumn}
        \ParacolAlignmentHack
        \begin{equation*}\taglabel[\ensuremath{ \leftrightarrow_+ }]{inf:def:propositional_natural_deduction/iff/intro}
          \begin{prooftree}
            \hypo{ [\syn\varphi] }
            \infer[dashed]1{ \syn\psi }
            \hypo{ [\syn\psi] }
            \infer[dashed]1{ \syn\varphi }
            \infer2[\ref{inf:def:propositional_natural_deduction/iff/intro}]{ \syn\varphi \syniff \syn\psi }
          \end{prooftree}
        \end{equation*}
      \end{leftcolumn}

      \begin{nthcolumn}{1}
        \ParacolAlignmentHack
        \begin{equation*}\taglabel[\ensuremath{ \leftrightarrow_{-L} }]{inf:def:propositional_natural_deduction/iff/elim_left}
          \begin{prooftree}
            \hypo{ \syn\varphi \syniff \syn\psi }
            \hypo{ \syn\psi }
            \infer2[\ref{inf:def:propositional_natural_deduction/iff/elim_left}]{ \syn\varphi }
          \end{prooftree}
        \end{equation*}
      \end{nthcolumn}

      \begin{nthcolumn}{2}
        \ParacolAlignmentHack
        \begin{equation*}\taglabel[\ensuremath{ \leftrightarrow_{-R} }]{inf:def:propositional_natural_deduction/iff/elim_right}
          \begin{prooftree}
            \hypo{ \syn\varphi \syniff \syn\psi }
            \hypo{ \syn\varphi }
            \infer2[\ref{inf:def:propositional_natural_deduction/iff/elim_right}]{ \syn\psi }
          \end{prooftree}
        \end{equation*}
      \end{nthcolumn}
    \end{paracol}
  \end{thmenum}
\end{definition}
\begin{comments}
  \item The systems are mostly based on \bycite[def. 2.1.1]{TroelstraSchwichtenberg2000BasicProofTheory}, with the following additions:
  \begin{itemize}
    \item The rules for \( \synbot \) are based on \bycite[31]{VanDalen2004LogicAndStructure}.

    \item The additional rules for \( \syntop \) and \( \synneg \) obtained from \ref{inf:def:simple_unit_type/intro} and \ref{inf:def:simple_empty_type/elim} via the \hyperref[con:curry_howard_correspondence]{Curry-Howard correspondence}. They can also be found in \cite[fig. 2.5]{Mimram2020ProgramEqualsProof}.

    \item We swap the order of the conjunction elimination rules --- see \cref{rem:conjunction_elimination_order}.

    \item The rules for \( \syniff \) are based on \bycite[99]{КолмогоровДрагалин2006Логика}.
  \end{itemize}

  \item Although we have defined abstract natural deduction systems in great generality, we will refer to the above as \enquote{the} natural deduction systems.

  \item These precise rules are used in the module \identifier{logic.classical_logic} in \cite{notebook:code}.
\end{comments}

\begin{remark}\label{rem:conjunction_elimination_order}
  There are two conventions for conjunction elimination rules:
  \begin{paracol}{2}
    \begin{leftcolumn}
      \ParacolAlignmentHack
      \begin{equation*}
        \begin{prooftree}
          \hypo{ \syn\varphi \synwedge \syn\psi }
          \infer1[\ensuremath{ \wedge_{-L} }]{ \syn\varphi }
        \end{prooftree}
      \end{equation*}
    \end{leftcolumn}

    \begin{rightcolumn}
      \ParacolAlignmentHack
      \begin{equation*}
        \begin{prooftree}
          \hypo{ \syn\varphi \synwedge \syn\psi }
          \infer1[\ensuremath{ \wedge_{-L} }]{ \syn\psi }
        \end{prooftree}
      \end{equation*}
    \end{rightcolumn}
  \end{paracol}

  The first selects \( \syn\varphi \), while the second \enquote{eliminates} \( \syn\varphi \) and instead selects \( \syn\psi \).

  When defining the corresponding rule in \ref{inf:def:propositional_natural_deduction/and/elim_left}, we have used the first convention because it ensures compatibility with product type \enquote{projection} rules, which we present in \cref{def:simple_product_type}. This convention is also used by \incite[fig. 2.5]{Mimram2020ProgramEqualsProof} (who also discusses product types).

  The other convention, which eliminates \( \syn\varphi \), can be found, for example, in \cite[def. 2.1.1]{TroelstraSchwichtenberg2000BasicProofTheory}.
\end{remark}
\begin{comments}
  \item This reasoning has also affected our definition of biconditional elimination rules in \cref{def:propositional_natural_deduction/iff}.
\end{comments}

\begin{example}\label{ex:def:propositional_natural_deduction}
  We list examples of \hyperref[def:propositional_natural_deduction_proof_tree]{proof trees} for the \hyperref[def:propositional_natural_deduction_system]{natural deduction systems} from \cref{def:propositional_natural_deduction}:
  \begin{thmenum}
    \thmitem{ex:def:propositional_natural_deduction/instantiation_dependence} Most of the rules are straightforward to apply. For example, given assumption trees \( [\varphi]^u \) and \( [\psi]^v \), in that order, the rule \ref{inf:def:propositional_natural_deduction/and/intro} can only possibly produce the rule application tree
    \begin{equation*}
      \begin{prooftree}
        \hypo{ [\varphi]^u }
        \hypo{ [\psi]^v }
        \infer2[\ref{inf:def:propositional_natural_deduction/and/intro}]{ \varphi \synwedge \psi }
      \end{prooftree}
    \end{equation*}

    Some cases are more ambiguous, however. Consider the rule \ref{inf:def:propositional_natural_deduction/bot/efq}, which we defined as
    \begin{equation*}
      \begin{prooftree}
        \hypo{ \synbot }
        \infer1[\ref{inf:def:propositional_natural_deduction/bot/efq}]{ \syn\varphi }
      \end{prooftree}
    \end{equation*}

    It has two schemas --- \( \synbot \), which corresponds to only one formula, and \( \syn\varphi \), whose desired instance is impossible to determine based on the premises alone.

    Applying \ref{inf:def:propositional_natural_deduction/bot/efq} thus requires specifying an instance for \( \syn\varphi \). In informal usage this is resolved easily, but in a formal setting, this requires specifying an instantiation explicitly. This is crucial in some cases like \fullref{alg:type_derivation_to_proof_tree/instantiation}.

    \thmitem{ex:def:propositional_natural_deduction/efq_vs_dne} The only difference between \ref{inf:def:propositional_natural_deduction/bot/efq} and \ref{inf:def:propositional_natural_deduction/bot/raa} is the dischargeable negation in the latter.

    For example, by using the rule \ref{inf:def:propositional_natural_deduction/bot/raa}, we can derive the formula \( \eqref{eq:thm:classical_tautologies/dne} = \synneg \synneg \varphi \synimplies \varphi \):
    \begin{equation*}
      \begin{prooftree}
        \hypo{ [\synneg\synneg\varphi]^u }
        \hypo{ [\synneg\varphi]^v }
        \infer2[\ref{inf:def:propositional_natural_deduction/neg/elim}]{ \synbot }
        \infer[left label=\( v \)]1[\ref{inf:def:propositional_natural_deduction/bot/raa}]{ \varphi }
        \infer[left label=\( u \)]1[\ref{inf:def:propositional_natural_deduction/imp/intro}]{ \synneg\synneg\varphi \synimplies \varphi }
      \end{prooftree}
    \end{equation*}

    If we instead use the weaker \ref{inf:def:propositional_natural_deduction/bot/efq}, we cannot discharge \( [\synneg\varphi]^v \), and instead of
    \begin{equation*}
      \vdash \synneg\synneg\varphi \synimplies \varphi,
    \end{equation*}
    we are left with
    \begin{equation*}
      \synneg\varphi \vdash \synneg\synneg\varphi \synimplies \varphi.
    \end{equation*}

    \thmitem{ex:def:propositional_natural_deduction/imp_intro} Consider the proof tree
    \begin{equation*}
      \begin{prooftree}
        \hypo{ [\psi]^u }
        \infer1[\ref{inf:def:propositional_natural_deduction/imp/intro}]{ \varphi \synimplies \psi }
      \end{prooftree}
    \end{equation*}

    Here \( \varphi \) is an implicit premise, while \( [\psi]^u \) is an open assumption.

    On the other hand, if \( P \) is a tree deriving \( \psi \) from \( \varphi \), we can construct
    \begin{equation*}
      \begin{prooftree}
        \hypo{ [\varphi]^u }
        \ellipsis {\( P \)} { \psi }
        \infer[left label=\( u \)]1[\ref{inf:def:propositional_natural_deduction/imp/intro}]{ \varphi \synimplies \psi }
      \end{prooftree}
    \end{equation*}

    In this case there are neither implicit premises nor open assumptions.
  \end{thmenum}
\end{example}

\begin{proposition}\label{thm:minimal_implicational_logic_axioms_nd_proof}
  Every instance of the two axiom schemas --- \eqref{eq:def:minimal_implication_logic/intro} and \eqref{eq:def:minimal_implication_logic/dist} --- of \hyperref[def:minimal_implication_logic]{minimal implicational logic} is derivable in the \hyperref[def:propositional_natural_deduction]{minimal natural deduction system}.
\end{proposition}
\begin{comments}
  \item When regarding \hyperref[def:minimal_implication_logic]{minimal implicational logic} as a \hyperref[def:propositional_natural_deduction_system]{natural deduction system} in accordance with \cref{def:axiomatic_derivation_as_natural_deduction}, the rules of the resulting system are \hyperref[con:inference_rule_admissibility]{admissible} with respect to the minimal natural deduction system. Thus, we do not need to introduce additional rules in order for implicational proofs to hold in the natural deduction systems we will consider.
\end{comments}
\begin{proof}
  \SubProofOf{eq:def:minimal_implication_logic/intro}
  \begin{equation*}
    \begin{prooftree}
      \hypo{ [\varphi]^u }
      \infer1[\ref{inf:def:propositional_natural_deduction/imp/intro}]{ \psi \synimplies \varphi }
      \infer[left label=\( u \)]1[\ref{inf:def:propositional_natural_deduction/imp/intro}]{ \varphi \synimplies (\psi \synimplies \varphi) }
    \end{prooftree}
  \end{equation*}

  \SubProofOf{eq:def:minimal_implication_logic/dist}
  \begin{equation*}
    \begin{prooftree}
      \hypo{ [\varphi \synimplies (\psi \synimplies \theta)]^u }
      \hypo{ [\varphi]^v }
      \infer2[\ref{inf:def:propositional_natural_deduction/imp/elim}]{ \psi \synimplies \theta }

      \hypo{ [\varphi \synimplies \psi]^w }
      \hypo{ [\varphi]^v }
      \infer2[\ref{inf:def:propositional_natural_deduction/imp/elim}]{ \psi }

      \infer2[\ref{inf:def:propositional_natural_deduction/imp/elim}]{ \theta }

      \infer[left label=\( v \)]1[\ref{inf:def:propositional_natural_deduction/imp/intro}]{ \varphi \synimplies \theta }
      \infer[left label=\( w \)]1[\ref{inf:def:propositional_natural_deduction/imp/intro}]{ (\varphi \synimplies \psi) \synimplies (\varphi \synimplies \theta) }
      \infer[left label=\( u \)]1[\ref{inf:def:propositional_natural_deduction/imp/intro}]{ (\varphi \synimplies (\psi \synimplies \theta)) \synimplies ((\varphi \synimplies \psi) \synimplies (\varphi \synimplies \theta)) }
    \end{prooftree}
  \end{equation*}
\end{proof}

\begin{proposition}\label{thm:syntactic_minimal_tautologies}
  The intuitionistic tautologies \eqref{eq:thm:intuitionistic_tautologies/dni} and \eqref{eq:thm:intuitionistic_tautologies/lnc} are derivable in the \hyperref[def:propositional_natural_deduction]{minimal natural deduction system}.
\end{proposition}
\begin{proof}
  \SubProofOf{eq:thm:intuitionistic_tautologies/dni}
  \begin{equation*}
    \begin{prooftree}
      \hypo{ [\varphi]^u }
      \hypo{ [\synneg \varphi]^v }
      \infer2[\ref{inf:def:propositional_natural_deduction/neg/elim}]{ \synbot }
      \infer[left label=\( v \)]1[\ref{inf:def:propositional_natural_deduction/neg/intro}]{ \synneg \synneg \varphi }
    \end{prooftree}
  \end{equation*}

  \SubProofOf{eq:thm:intuitionistic_tautologies/lnc}
  \begin{equation*}
    \begin{prooftree}
      \hypo{ [\varphi \synwedge \synneg \varphi]^u }
      \infer1[\ref{inf:def:propositional_natural_deduction/and/elim_left}]{ \varphi }

      \hypo{ [\varphi \synwedge \synneg \varphi]^u }
      \infer1[\ref{inf:def:propositional_natural_deduction/and/elim_right}]{ \synneg \varphi }

      \infer2[\ref{inf:def:propositional_natural_deduction/neg/elim}]{ \synbot }

      \infer[left label=\( u \)]1[\ref{inf:def:propositional_natural_deduction/neg/intro}]{ \synneg (\varphi \synwedge \synneg \varphi) }
    \end{prooftree}
  \end{equation*}
\end{proof}

\begin{theorem}[Propositional syntactic deduction theorem]\label{thm:propositional_syntactic_deduction_theorem}
  With respect to \hyperref[def:propositional_natural_deduction]{minimal propositional natural deduction}, for arbitrary formulas we have
  \begin{equation*}
    \Gamma, \varphi \vdash \psi \T{if and only if} \Gamma \vdash \varphi \synimplies \psi.
  \end{equation*}
\end{theorem}
\begin{comments}
  \item This is one of several deduction theorems presented here --- see \cref{rem:deduction_theorem_list}.
\end{comments}
\begin{proof}
  \SufficiencySubProof Let \( P \) be a \hyperref[def:propositional_natural_deduction_proof_tree]{proof tree} of \( \psi \) from \( \Gamma \cup \set{ \varphi } \). If \( \varphi \) is marked with \( u \) in an open assumption in \( P \), the following tree derives \( \varphi \synimplies \psi \) from \( \Gamma \):
  \begin{equation*}
    \begin{prooftree}
      \hypo{ [\varphi]^u }
      \ellipsis { \( P \) } { \psi }
      \infer[left label=\( u \)]1[\ref{inf:def:propositional_natural_deduction/imp/intro}]{ \varphi \synimplies \psi }
    \end{prooftree}
  \end{equation*}

  Otherwise, we use the same rule but without discharging any assumptions:
  \begin{equation*}
    \begin{prooftree}
      \hypo{}
      \ellipsis { \( P \) } { \psi }
      \infer1[\ref{inf:def:propositional_natural_deduction/imp/intro}]{ \varphi \synimplies \psi }
    \end{prooftree}
  \end{equation*}

  \NecessitySubProof Let \( P \) be a proof tree deriving \( \varphi \synimplies \psi \) from \( \Gamma \). Then the following proves \( \psi \) from \( \Gamma \cup \set{ \varphi } \):
  \begin{equation*}
    \begin{prooftree}
      \hypo {\Gamma}
      \ellipsis { \( P \) } { \varphi \synimplies \psi }

      \hypo { [\varphi]^u }
      \infer2[\ref{inf:def:propositional_natural_deduction/imp/elim}]{ \psi }
    \end{prooftree}
  \end{equation*}
\end{proof}

\begin{proposition}\label{thm:minimal_propositional_negation_laws}\mcite[prop. 3; prop. 13]{DienerMcKubreJordens2020MaterialImplication}
  With respect to the \hyperref[def:propositional_natural_deduction]{minimal natural deduction system}, we have the following strict derivability graph (\eqref{eq:thm:classical_tautologies/dne} being the most powerful):
  \begin{center}
    \begin{forest}
      [
        {\eqref{eq:thm:classical_tautologies/dne}}
          [
            {\eqref{eq:thm:classical_tautologies/pierce}}
              [{\eqref{eq:thm:classical_tautologies/lem}}]
          ]
          [
            {\eqref{eq:thm:intuitionistic_tautologies/efq}}
              [{\eqref{eq:thm:intuitionistic_tautologies/lnc}}]
          ]
      ]
    \end{forest}
  \end{center}
\end{proposition}

\begin{proposition}\label{thm:syntactic_propositional_conjunction_of_premises}
  With respect to the \hyperref[def:propositional_natural_deduction]{minimal natural deduction system}, we have \( \varphi, \psi \vdash \theta \) if and only if \( (\varphi \synwedge \psi) \vdash \theta \).
\end{proposition}
\begin{proof}
  \SufficiencySubProof If \( \varphi, \psi \vdash \theta \), the following is a proof tree deriving \( \theta \) from \( \varphi \synwedge \psi \):
  \begin{equation*}
    \begin{prooftree}
      \hypo{ \varphi \synwedge \psi }
      \infer1[\ref{inf:def:propositional_natural_deduction/and/elim_left}]{ \varphi }

      \hypo{ \varphi \synwedge \psi }
      \infer1[\ref{inf:def:propositional_natural_deduction/and/elim_right}]{ \psi }

      \infer2{}

      \ellipsis {} { \theta }
    \end{prooftree}
  \end{equation*}

  \NecessitySubProof If \( (\varphi \synwedge \psi) \vdash \theta \), the following is a proof tree deriving \( \theta \) from \( \set{ \varphi, \psi } \):
  \begin{equation*}
    \begin{prooftree}
      \hypo{ \varphi }
      \hypo{ \psi }
      \infer2[\ref{inf:def:propositional_natural_deduction/and/intro}]{ \varphi \synwedge \psi }
      \ellipsis{}{ \theta }
    \end{prooftree}
  \end{equation*}
\end{proof}

\begin{proposition}\label{thm:propositional_admissible_rules}
  The following \hyperref[def:natural_deduction_rule]{natural deduction rules} are \hyperref[con:inference_rule_admissibility]{admissible} with respect to \hyperref[def:propositional_natural_deduction]{minimal propositional natural deduction}:
  \begin{thmenum}
    \thmitem{thm:propositional_admissible_rules/self_conditional} The conditional of a formula with itself is tautologous:
    \begin{equation*}\taglabel[\ensuremath{ \rightarrow_R }]{inf:thm:propositional_admissible_rules/self_conditional}
      \begin{prooftree}
        \infer0[\ref{inf:thm:propositional_admissible_rules/self_conditional}]{ \varphi \synimplies \varphi }
      \end{prooftree}
    \end{equation*}

    \thmitem{thm:propositional_admissible_rules/self_biconditional} The biconditional of a formula with itself is tautologous:
    \begin{equation*}\taglabel[\ensuremath{ \leftrightarrow_R }]{inf:thm:propositional_admissible_rules/self_biconditional}
      \begin{prooftree}
        \infer0[\ref{inf:thm:propositional_admissible_rules/self_biconditional}]{ \varphi \syniff \varphi }
      \end{prooftree}
    \end{equation*}

    \thmitem{thm:propositional_admissible_rules/biconditional_as_conjunction} The biconditional is a conjunction of conditionals:
    \begin{equation*}\taglabel[\ensuremath{ \leftrightarrow_\wedge }]{inf:thm:propositional_admissible_rules/biconditional_as_conjunction}
      \begin{prooftree}
        \hypo{ \varphi \syniff \psi }
        \infer1[\ref{inf:thm:propositional_admissible_rules/biconditional_as_conjunction}]{ (\varphi \synimplies \psi) \synwedge (\psi \synimplies \varphi) }
      \end{prooftree}
    \end{equation*}

    \thmitem{thm:propositional_admissible_rules/conjunction_of_conditionals} Conversely, a conjunction of compatible conditionals is a biconditional:
    \begin{equation*}\taglabel[\ensuremath{ \wedge_\leftrightarrow }]{inf:thm:propositional_admissible_rules/conjunction_of_conditionals}
      \begin{prooftree}
        \hypo{ (\varphi \synimplies \psi) \synwedge (\psi \synimplies \varphi) }
        \infer1[\ref{inf:thm:propositional_admissible_rules/conjunction_of_conditionals}]{ \varphi \syniff \psi }
      \end{prooftree}
    \end{equation*}

    \thmitem{thm:propositional_admissible_rules/contraposition} From a conditional we can derive its contraposition:
    \begin{equation*}\taglabel[\ensuremath{ \rightarrow_{\neg+} }]{inf:thm:propositional_admissible_rules/contraposition}
      \begin{prooftree}
        \hypo{ \varphi \synimplies \psi }
        \infer1[\ref{inf:thm:propositional_admissible_rules/contraposition}]{ \synneg \psi \synimplies \synneg \varphi }
      \end{prooftree}
    \end{equation*}

    \thmitem{thm:propositional_admissible_rules/contraposition_double_negation} From a contraposition we can derive a variation of the original:
    \begin{equation*}\taglabel[\ensuremath{ \rightarrow_{\neg\neg} }]{inf:thm:propositional_admissible_rules/contraposition_double_negation}
      \begin{prooftree}
        \hypo{ \synneg \psi \synimplies \synneg \varphi }
        \infer1[\ref{inf:thm:propositional_admissible_rules/contraposition_double_negation}]{ \varphi \synimplies \synneg \synneg \psi }
      \end{prooftree}
    \end{equation*}

    \thmitem{thm:propositional_admissible_rules/contraposition_elim} When additionally assuming \ref{inf:def:propositional_natural_deduction/bot/raa}, we can simplify \ref{inf:thm:propositional_admissible_rules/contraposition_double_negation} to
    \begin{equation*}\taglabel[\ensuremath{ \rightarrow_{\neg-} }]{inf:thm:propositional_admissible_rules/contraposition_elim}
      \begin{prooftree}
        \hypo{ \synneg \psi \synimplies \synneg \varphi }
        \infer1[\ref{inf:thm:propositional_admissible_rules/contraposition_elim}]{ \varphi \synimplies \psi }
      \end{prooftree}
    \end{equation*}

    \thmitem{thm:propositional_admissible_rules/conditional_as_conjunction} A conditional formula is the negation of a conjunction:
    \begin{equation*}\taglabel[\ensuremath{ \rightarrow_\wedge }]{inf:thm:propositional_admissible_rules/conditional_as_conjunction}
      \begin{prooftree}
        \hypo{ \varphi \synimplies \psi }
        \infer1[\ref{inf:thm:propositional_admissible_rules/conditional_as_conjunction}]{ \neg(\varphi \synwedge \synneg \psi) }
      \end{prooftree}
    \end{equation*}

    \thmitem{thm:propositional_admissible_rules/disjunction_as_conditional} When additionally assuming \ref{inf:def:propositional_natural_deduction/bot/efq}, a disjunction with the first subformula negated is a conditional:
    \begin{equation*}\taglabel[\ensuremath{ \vee_\rightarrow }]{inf:thm:propositional_admissible_rules/disjunction_as_conditional}
      \begin{prooftree}
        \hypo{ \synneg \varphi \synvee \psi }
        \infer1[\ref{inf:thm:propositional_admissible_rules/disjunction_as_conditional}]{ \varphi \synimplies \psi }
      \end{prooftree}
    \end{equation*}
  \end{thmenum}
\end{proposition}
\begin{comments}
  \item These rules will be valid for \hyperref[def:first_order_logic]{first-order logic} and \hyperref[def:higher_order_logic]{higher-order logic} because the systems we will consider these will extend the propositional natural deduction rules.
\end{comments}
\begin{proof}
  \SubProofOf{thm:propositional_admissible_rules/self_conditional}
  \begin{equation*}
    \begin{prooftree}
      \hypo{ [\varphi]^u }
      \hypo{ [\varphi]^u }
      \infer[left label=\( u \)]2[\ref{inf:def:propositional_natural_deduction/imp/intro}]{ \varphi \synimplies \varphi }
    \end{prooftree}
  \end{equation*}

  \SubProofOf{thm:propositional_admissible_rules/self_biconditional}
  \begin{equation*}
    \begin{prooftree}
      \hypo{ [\varphi]^u }
      \hypo{ [\varphi]^u }
      \infer[left label=\( u \)]2[\ref{inf:def:propositional_natural_deduction/iff/intro}]{ \varphi \syniff \varphi }
    \end{prooftree}
  \end{equation*}

  \SubProofOf{thm:propositional_admissible_rules/biconditional_as_conjunction}
  \begin{equation*}
    \begin{prooftree}
      \hypo{ [\varphi \syniff \psi]^u }
      \hypo{ [\varphi]^v }
      \infer2[\ref{inf:def:propositional_natural_deduction/iff/elim_right}]{ \psi }
      \infer[left label=\( v \)]1[\ref{inf:def:propositional_natural_deduction/imp/intro}]{ \varphi \synimplies \psi }

      \hypo{ [\varphi \syniff \psi]^u }
      \hypo{ [\psi]^w }
      \infer2[\ref{inf:def:propositional_natural_deduction/iff/elim_left}]{ \varphi }
      \infer[left label=\( w \)]1[\ref{inf:def:propositional_natural_deduction/imp/intro}]{ \psi \synimplies \varphi }

      \infer2[\ref{inf:def:propositional_natural_deduction/and/intro}]{ (\varphi \synimplies \psi) \synwedge (\psi \synimplies \varphi) }
    \end{prooftree}
  \end{equation*}

  \SubProofOf{thm:propositional_admissible_rules/conjunction_of_conditionals}
  \begin{equation*}
    \begin{prooftree}
      \hypo{ [(\varphi \synimplies \psi) \synwedge (\psi \synimplies \varphi)]^u }
      \infer1[\ref{inf:def:propositional_natural_deduction/and/elim_left}]{ \varphi \synimplies \psi }
      \hypo{ [\varphi]^v }
      \infer2[\ref{inf:def:propositional_natural_deduction/imp/elim}]{ \psi }

      \hypo{ [(\varphi \synimplies \psi) \synwedge (\psi \synimplies \varphi)]^u }
      \infer1[\ref{inf:def:propositional_natural_deduction/and/elim_right}]{ \psi \synimplies \varphi }
      \hypo{ [\psi]^w }
      \infer2[\ref{inf:def:propositional_natural_deduction/imp/elim}]{ \varphi }

      \infer[left label={\( v, w \)}]2[\ref{inf:def:propositional_natural_deduction/iff/intro}]{ \varphi \syniff \psi }
    \end{prooftree}
  \end{equation*}

  \SubProofOf{thm:propositional_admissible_rules/contraposition}
  \begin{equation*}
    \begin{prooftree}
      \hypo{ [\synneg \psi]^u }

      \hypo{ \varphi \synimplies \psi }
      \hypo{ [\varphi]^v }
      \infer2[\ref{inf:def:propositional_natural_deduction/imp/elim}]{ \psi }
      \infer2[\ref{inf:def:propositional_natural_deduction/neg/elim}]{ \synbot }

      \infer[left label=\( v \)]1[\ref{inf:def:propositional_natural_deduction/neg/intro}]{ \synneg \varphi }
      \infer[left label=\( u \)]1[\ref{inf:def:propositional_natural_deduction/imp/intro}]{ \synneg \psi \synimplies \synneg \varphi }
    \end{prooftree}
  \end{equation*}

  \SubProofOf{thm:propositional_admissible_rules/contraposition_double_negation}
  \begin{equation*}
    \begin{prooftree}
      \hypo{ \synneg \psi \synimplies \synneg \varphi }
      \hypo{ [\synneg \psi]^u }
      \infer2[\ref{inf:def:propositional_natural_deduction/imp/elim}]{ \synneg \varphi }

      \hypo{ [\varphi]^v }
      \infer2[\ref{inf:def:propositional_natural_deduction/neg/elim}]{ \synbot }

      \infer[left label=\( u \)]1[\ref{inf:def:propositional_natural_deduction/neg/intro}]{ \synneg \synneg \psi }
      \infer[left label=\( v \)]1[\ref{inf:def:propositional_natural_deduction/imp/intro}]{ \varphi \synimplies \synneg \synneg \psi }
    \end{prooftree}
  \end{equation*}

  \SubProofOf{thm:propositional_admissible_rules/contraposition_elim}
  \begin{equation*}
    \begin{prooftree}
      \hypo{ \synneg \psi \synimplies \synneg \varphi }
      \hypo{ [\synneg \psi]^u }
      \infer2[\ref{inf:def:propositional_natural_deduction/imp/elim}]{ \synneg \varphi }

      \hypo{ [\varphi]^v }
      \infer2[\ref{inf:def:propositional_natural_deduction/neg/elim}]{ \synbot }

      \infer[left label=\( u \)]1[\ref{inf:def:propositional_natural_deduction/bot/raa}]{ \psi }
      \infer[left label=\( v \)]1[\ref{inf:def:propositional_natural_deduction/imp/intro}]{ \varphi \synimplies \psi }
    \end{prooftree}
  \end{equation*}

  \SubProofOf{thm:propositional_admissible_rules/conditional_as_conjunction}
  \begin{equation*}
    \begin{prooftree}
      \hypo{ [\varphi \synwedge \synneg \psi]^u}
      \infer1[\ref{inf:def:propositional_natural_deduction/and/elim_right}]{ \synneg \psi }

      \hypo { [\varphi \rightarrow \psi]^v }

      \hypo{ [\varphi \synwedge \synneg \psi]^u}
      \infer1[\ref{inf:def:propositional_natural_deduction/and/elim_left}]{ \varphi }

      \infer2[\ref{inf:def:propositional_natural_deduction/imp/elim}]{ \psi }
      \infer2[\ref{inf:def:propositional_natural_deduction/imp/elim}]{ \synbot }

      \infer[left label=\( u \)]1[\ref{inf:def:propositional_natural_deduction/neg/intro}]{ \neg(\varphi \synwedge \synneg \psi) }
    \end{prooftree}
  \end{equation*}

  \SubProofOf{thm:propositional_admissible_rules/disjunction_as_conditional}
  \begin{equation*}
    \begin{prooftree}
      \hypo{ [\synneg \varphi \synvee \psi]^u }

      \hypo{ [\synneg \varphi]^v }
      \hypo{ [\varphi]^w }
      \infer2[\ref{inf:def:propositional_natural_deduction/neg/elim}]{ \bot }
      \infer1[\ref{inf:def:propositional_natural_deduction/bot/efq}]{ \psi }

      \hypo{ [\psi]^t }
      \infer[left label={\( v, t \)}]3[\ref{inf:def:propositional_natural_deduction/or/elim}]{ \psi }

      \infer[left label=\( w \)]1[\ref{inf:def:propositional_natural_deduction/imp/intro}]{ \varphi \synimplies \psi }
    \end{prooftree}
  \end{equation*}
\end{proof}

\paragraph{Soundness of natural deduction}

\begin{theorem}[Propositional natural deduction soundness]\label{thm:propositional_natural_deduction_soundness}
  Intuitionistic and classical \hyperref[def:propositional_natural_deduction]{propositional natural deduction} are \hyperref[def:general_logic]{sound} with respect to \hyperref[def:truth_value_algebra/intuitionistic]{intuitionistic} and \hyperref[def:truth_value_algebra/classical]{classical} \hyperref[def:propositional_semantics]{semantics}, respectively.
\end{theorem}
\begin{comments}
  \item This proof will also demonstrate that the minimal natural deduction system is sound with respect to the minimal semantics that we will sketch in \cref{rem:minimal_propositional_semantics}. Indeed, except for \ref{inf:def:propositional_natural_deduction/bot/efq} and \ref{inf:def:propositional_natural_deduction/bot/raa}, the other rules do not depend on a particular interpretation of \( \synbot \).

  \item This provides a proof that the propositional general logics in \cref{def:propositional_logic} are well-defined.

  \item See \cref{rem:soundness_and_completeness_theorem_list} for a list of soundness and completeness theorems.
\end{comments}
\begin{proof}
  Fix a \hyperref[def:propositional_natural_deduction_proof_tree]{proof tree} \( P \). We will use \fullref{thm:induction_on_rooted_trees} on \( P \) to show that if a \hyperref[def:propositional_interpretation]{propositional interpretation} satisfies the open assumptions of \( P \), it also satisfies its conclusion.

  For definiteness, let \( \Gamma \) be the list of open assumptions of \( P \). Fix an interpretation \( I \) that satisfies \( \Gamma \).

  The case where \( P \) is an assumption tree is vacuous, so suppose that \( P \) is a rule application tree. Let \( P_1, \ldots, P_n \) be the premise subtrees of \( P \) and suppose that the statement holds for them. We will perform case analysis by the rule used.

  \SubProofOf{inf:def:propositional_natural_deduction/top/intro} The conclusion of \( P \) is \( \syntop \). There is nothing to prove since the denotation of \( \syntop \) is defined to be \( \semtop \).

  \SubProofOf{inf:def:propositional_natural_deduction/bot/efq} Let \( \varphi \) be the conclusion of \( P \). The premise \( P_1 \) derives \( \synbot \).

  Since \( I \) satisfies \( \Gamma \), the premises of \( P \), it also satisfies those of \( P_1 \). The inductive hypothesis implies that \( I \) satisfies \( \synbot \). Such an interpretation cannot exist. Thus, all zero interpretations satisfying the assumptions of \( P \) vacuously satisfy any formula, including \( \varphi \).

  This case requires \fullref{thm:propositional_semantic_efq} on a metatheoretic level.

  \SubProofOf{inf:def:propositional_natural_deduction/neg/intro} Let \( \synneg \varphi \) be the conclusion of \( P \). Then the premise \( P_1 \) derives \( \synbot \) from \( \varphi \).

  By the inductive hypothesis, we have \( \Gamma, \varphi \vDash \synbot \). \Cref{thm:propositional_semantic_deduction_theorem} implies that \( \Gamma \vDash \varphi \synimplies \synbot \). Then \cref{thm:intuitionistic_equivalences/negation_bottom} implies that \( \Gamma \vDash \synneg \varphi \).

  Since \( I \) satisfies \( \Gamma \) (the premises of \( P \)), it also satisfies \( \synneg \varphi \) (the conclusion of \( P \)).

  Note that it is irrelevant whether \( \varphi \) is actually an open assumption --- a consequence relation allows adding to the left if it is not.

  \SubProofOf{inf:def:propositional_natural_deduction/bot/raa} The previous case implies that \( I \) satisfies \( \synneg \synneg \varphi \). \Cref{thm:classical_tautologies/dne} implies that, in classical logic, \( I \) satisfies \( \varphi \).

  \SubProofOf{inf:def:propositional_natural_deduction/neg/elim} The conclusion of \( P \) is \( \synbot \). For some formula \( \varphi \), the premises \( P_1 \) and \( P_2 \) have conclusions \( \synneg \varphi \) and \( \varphi \).

  By the inductive hypothesis, we have \( \Bracks{\synneg \varphi}_I = \Bracks{\varphi}_I = \semtop \). Thus,
  \begin{equation*}
    \underbrace{\Bracks{\synneg \varphi}_I}_{\semtop}
    \reloset {\eqref{eq:thm:intuitionistic_equivalences/negation_bottom}} =
    \underbrace{\Bracks{\varphi}_I}_{\semtop} \rightarrow \Bracks{\synbot}_I
    \reloset {\ref{thm:def:heyting_algebra/top_left}} =
    \Bracks{\synbot}_I.
  \end{equation*}

  \SubProofOf{inf:def:propositional_natural_deduction/and/intro} Let \( \varphi \synwedge \psi \) be the conclusion of \( P \). Then \( \varphi \) and \( \psi \) are the conclusions of \( P_1 \) and \( P_2 \).

  By the inductive hypothesis, we have \( \Bracks{\varphi}_I = \Bracks{\psi}_I = \semtop \), so
  \begin{equation*}
    \Bracks{\varphi \synwedge \psi}_I = \semtop \wedge \semtop = \semtop.
  \end{equation*}

  \SubProofOf{inf:def:propositional_natural_deduction/and/elim_left} Let \( \varphi \) be the conclusion of \( P \). Then, for some formula \( \psi \), the conclusion of \( P_1 \) is \( \varphi \synwedge \psi \).

  By the inductive hypothesis, we have \( \Bracks{\varphi \synwedge \psi}_I = \semtop \), hence \( \Bracks{\varphi}_I = \Bracks{\psi}_I = \semtop \).

  \SubProofOf{inf:def:propositional_natural_deduction/and/elim_right} Analogous.

  \SubProofOf{inf:def:propositional_natural_deduction/or/intro_left} Let \( \varphi \synvee \psi \) be the conclusion of \( P \). Then \( \varphi \) is the conclusion of \( P_1 \).

  We have \( \Bracks{\varphi}_I = \semtop \) by the inductive hypothesis, so
  \begin{equation*}
    \Bracks{\varphi \synvee \psi}_I = \semtop \vee \Bracks{\psi}_I = \semtop.
  \end{equation*}

  \SubProofOf{inf:def:propositional_natural_deduction/or/intro_right} Analogous.

  \SubProofOf{inf:def:propositional_natural_deduction/or/elim} Suppose that \( P \) has the form
  \begin{equation*}
    \begin{prooftree}
      \hypo{}
      \ellipsis { \( P_1 \) } { \varphi \synvee \psi }

      \hypo{ \varphi }
      \ellipsis { \( P_2 \) } { \theta }

      \hypo{ \psi }
      \ellipsis { \( P_3 \) } { \theta }

      \infer3[\ref{inf:def:propositional_natural_deduction/or/elim}]{ \theta }
    \end{prooftree}
  \end{equation*}

  By the inductive hypothesis on \( P_1 \), we have \( \Gamma \vDash \varphi \synvee \psi \). Since \( I \) satisfies \( \varphi \), it also satisfies \( \varphi \synvee \psi \).

  By the inductive hypothesis on \( P_2 \) and \( P_3 \), we have \( \Gamma, \varphi \vDash \theta \) and \( \Gamma, \psi \vDash \theta \). \Cref{thm:intuitionistic_deduction_consequences/leq} implies that \( \Bracks{\varphi}_I \leq \Bracks{\theta}_I \) and \( \Bracks{\psi}_I \leq \Bracks{\theta}_I \).

  Therefore, \( \Bracks{\theta}_I \) is an upper bound of \( \Bracks{\varphi}_I \) and \( \Bracks{\psi}_I \). Since \( \Bracks{\varphi \synvee \psi}_I \) is the least upper bound, we have \( \Bracks{\varphi \synvee \psi}_I \leq \Bracks{\theta}_I \). Via \cref{thm:intuitionistic_deduction_consequences/leq}, this implies that \( \Gamma, \varphi \synvee \psi \vDash \theta \).

  \SubProofOf{inf:def:propositional_natural_deduction/imp/intro} Let \( \varphi \synimplies \psi \) be the conclusion of \( P \). Then \( P_1 \) derives \( \psi \) from \( \varphi \). By the inductive hypothesis, \( \Gamma, \varphi \vDash \psi \). \Cref{thm:intuitionistic_deduction_consequences/leq} then implies that \( \Bracks{\varphi}_I \leq \Bracks{\psi}_I \) for every interpretation \( I \).

  It follows from \cref{thm:def:heyting_algebra/leq} that \( \Bracks{\varphi \synimplies \psi}_I = \semtop \).

  \SubProofOf{inf:def:propositional_natural_deduction/imp/elim} Let \( \psi \) be the conclusion of \( P \). Then, for some formula \( \varphi \), the conclusions of \( P_1 \) and \( P_2 \) are \( \varphi \synimplies \psi \) and \( \varphi \), correspondingly.

  We have \( \Bracks{\varphi \synimplies \psi}_I = \semtop \) and \( \Bracks{\varphi}_I = \semtop \) by the inductive hypothesis, so
  \begin{equation*}
    \underbrace{\Bracks{\varphi \synimplies \psi}_I}_{\semtop}
    \reloset {\eqref{eq:alg:propositional_denotation/conn}} =
    \underbrace{\Bracks{\varphi}_I}_{\semtop} \rightarrow \Bracks{\psi}_I
    \reloset {\ref{thm:def:heyting_algebra/top_left}} =
    \Bracks{\psi}_I.
  \end{equation*}

  \SubProofOf{inf:def:propositional_natural_deduction/iff/intro} Suppose that \( P \) has the form
  \begin{equation*}
    \begin{prooftree}
      \hypo{ \varphi }
      \ellipsis { \( P_1 \) } { \psi }

      \hypo{ \psi }
      \ellipsis { \( P_2 \) } { \varphi }

      \infer2[\ref{inf:def:propositional_natural_deduction/or/elim}]{ \varphi \syniff \psi }
    \end{prooftree}
  \end{equation*}

  By the inductive hypothesis on \( P_1 \), we have \( \Gamma, \varphi \vDash \psi \). \Cref{thm:intuitionistic_deduction_consequences/leq} implies that \( \Bracks{\varphi}_I \leq \Bracks{\psi}_I \) for every interpretation \( I \).

  Similarly, using the inductive hypothesis on \( P_2 \), we conclude that \( \Bracks{\varphi}_I \geq \Bracks{\psi}_I \). Therefore, \( \Bracks{\varphi}_I = \Bracks{\psi}_I \) for every interpretation \( I \).

  \Cref{thm:def:heyting_algebra/eq} implies that \( \Bracks{\varphi \syniff \psi}_I = \semtop \).

  \SubProofOf{inf:def:propositional_natural_deduction/iff/elim_left} Let \( \varphi \) be the conclusion of \( P \). Then, for some formula \( \psi \), the conclusions of \( P_1 \) and \( P_2 \) are \( \varphi \syniff \psi \) and \( \psi \), correspondingly.

  We have \( \Bracks{\varphi \syniff \psi}_I = \semtop \) and \( \Bracks{\psi}_I = \semtop \) by the inductive hypothesis, so
  \begin{multline*}
    \underbrace{\Bracks{\varphi \syniff \psi}_I}_{\semtop}
    \reloset {\eqref{eq:alg:propositional_denotation/conn}} =
    \Bracks{\varphi}_I \leftrightarrow \Bracks{\psi}_I
    \reloset {\eqref{eq:def:heyting_algebra/biconditional}} =
    (\Bracks{\varphi}_I \rightarrow \underbrace{\Bracks{\psi}_I}_{\semtop}) \wedge (\underbrace{\Bracks{\psi}_I}_{\semtop} \rightarrow \Bracks{\varphi}_I)
    \reloset {\ref{thm:def:heyting_algebra/top_left}} = \\ =
    (\Bracks{\varphi}_I \rightarrow \underbrace{\Bracks{\psi}_I}_{\semtop}) \wedge \Bracks{\varphi}_I
    \reloset {\eqref{eq:def:heyting_algebra/axioms/modus_ponens}} =
    \underbrace{\Bracks{\psi}_I}_{\semtop} \wedge \Bracks{\varphi}_I
    \reloset {\ref{thm:def:bounded_lattice/neutral}} =
    \Bracks{\varphi}_I.
  \end{multline*}

  \SubProofOf{inf:def:propositional_natural_deduction/iff/elim_right} Analogous.
\end{proof}

\begin{theorem}[Glivenko's double negation theorem]\label{thm:glivenkos_double_negation_theorem}\mcite[202]{CitkinMuravitsky2022ConsequenceRelations}
  A formula \( \varphi \) is derivable from \( \Gamma \) in \hyperref[def:propositional_natural_deduction]{classical propositional natural deduction} if and only if it's double negation \( \synneg \synneg \varphi \) is derivable from \( \Gamma \) in the minimal natural deduction system.
\end{theorem}
\begin{comments}
  \item We will not prove this theorem.
\end{comments}
