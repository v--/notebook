\section{Semimodules}\label{sec:semimodules}

\paragraph{Semimodules over semirings}

Semimodules are generalizations of monoid actions. Notation and terminology-wise, semimodules are somewhat special in that they are very much influenced by linear algebra and analysis, where vector spaces are crucial.

\begin{definition}\label{def:endomorphism_semiring}\mimprovised
  Let \( X \) be a monoid or, more generally, an object in a \hyperref[def:category]{category} that is \hyperref[def:concrete_category]{concrete} over \hyperref[def:monoid/category]{\( \cat{Mon} \)}.

  Let \( \End(X) \) be the \hyperref[def:endomorphism_monoid]{endomorphism monoid} over \( X \). These are necessarily monoid endomorphisms, however they may carry additional structure like being \hyperref[def:group/homomorphism]{group homomorphisms}, \hyperref[def:semimodule/homomorphism]{semimodule homomorphisms}, \hyperref[def:lattice/homomorphism]{(semi)lattice homomorphisms} or their \hyperref[rem:topological_first_order_structures]{continuous counterparts}.

  Define addition in \( \End(X) \) pointwise as \( [f + g](x) \coloneqq f(x) + g(x) \). Then \( \End(X) \) with pointwise addition and composition is a \hyperref[def:semiring]{semiring}, which we call the \term{endomorphism semiring} over \( X \).
\end{definition}

\begin{definition}\label{def:semimodule}\mcite[149]{Golan1999Semirings}
  Fix a \hyperref[def:semiring]{semiring} \( R \), whose elements we will call \term{scalars}, and an \hyperref[con:additive_semigroup]{additive} \hyperref[def:binary_operation/commutative]{commutative} \hyperref[def:monoid]{monoid} \( M \), whose elements we will call \term{vectors}. See \fullref{rem:vector_etymology} for a discussion of the term \enquote{vector}.

  We say that \( M \) is a \term[ru=полумодуль (\cite[99]{ВечтомовПетров2022Полукольца})]{semimodule} over \( R \) if they are compatible in any of the equivalent ways listed below. Analogously to \hyperref[def:monoid_action]{monoid actions}, if \( R \) is not commutative, we distinguish between left and right semimodules. Rather than \enquote{\( M \) is a semimodule over \( R \)}, it is often more convenient to say \enquote{\( M \) is an \( R \)-semimodule}.

  \begin{thmenum}[series=def:semimodule]
    \thmitem{def:semimodule/action}\mimprovised A left semimodule is a \hyperref[def:semiring/homomorphism]{homomorphism} from \( R \) to the \hyperref[def:endomorphism_semiring]{endomorphism semiring} \( \End(M) \). A right semimodule is a homomorphism from the \hyperref[def:semiring/opposite]{dual semiring} \( R^{-1} \) to \( \End(M) \).

    This definition is concise and natural, but unfortunately not very useful.

    \thmitem{def:semimodule/operation}\mcite[149]{Golan1999Semirings} The explicit way to define a left semimodule is via a binary operation \( \cdot: R \times M \to M \) called \term{scalar multiplication} that satisfies the following conditions:
    \begin{thmenum}
      \thmitem{def:semimodule/operation/scalar_multiplication_action} Scalar multiplication is a \hyperref[def:monoid_action]{monoid action} of the multiplicative monoid \( (R, \cdot_R) \) on \( M \). The following conditions correspond to \eqref{eq:def:monoid_action/family/identity} and \eqref{eq:def:monoid_action/family/compatibility}:
      \begin{align}
        &1_R \cdot x = x, \label{eq:def:semimodule/operation/scalar_multiplication_action/identity} \\
        &(r \cdot_R s) \cdot x = r \cdot (s \cdot x). \label{eq:def:semimodule/operation/scalar_multiplication_action/compatibility}
      \end{align}

      The second condition can be regarded as a form of associativity.

      \thmitem{def:semimodule/operation/scalar_addition_distributivity} Scalar addition distributes over scalar multiplication:
      \begin{equation}\label{eq:def:semimodule/operation/scalar_addition_distributivity}
        (r +_R s) \cdot x = r \cdot x + s \cdot x.
      \end{equation}

      \thmitem{def:semimodule/operation/vector_addition_distributivity} Vector addition distributes over scalar multiplication:
      \begin{equation}\label{eq:def:semimodule/operation/vector_addition_distributivity}
        r \cdot (x + y) = r \cdot x + r \cdot y.
      \end{equation}

      \thmitem{def:semimodule/operation/absorption} The scalar and vector zeros are compatible:
      \begin{equation}\label{eq:def:semimodule/operation/absorption}
        0_R \cdot x = 0_M = r \cdot 0_M.
      \end{equation}
    \end{thmenum}

    In practice, we use the same symbol for both scalar and vector addition, and we denote both scalar and vector multiplication via juxtaposition.
  \end{thmenum}

  Semimodules have the following metamathematical properties:
  \begin{thmenum}[resume=def:semimodule]
    \thmitem{def:semimodule/theory}\mimprovised In order to fit the heterogeneous operation \( \cdot \) into the framework of \hyperref[def:first_order_model]{first-order logic models}, we can extend the \hyperref[def:monoid/theory]{theory of monoids} by adding, for every semiring element \( r \), a unary \hyperref[def:first_order_language/fun]{functional symbol} \( m_r \).

    We have placed a restriction that the number of functional symbols must be finite, as discussed in \fullref{rem:uncountable_first_order_language}, hence this method is only available for finite semirings and is simply a conceptual sketch for infinite semirings.

    All conditions can then be reformulated via this operation. For example, \eqref{eq:def:semimodule/operation/scalar_multiplication_action/compatibility} corresponds to the axiom schema
    \begin{equation*}
      m_{rs}(\synx) = m_r(m_s(\synx)).
    \end{equation*}

    \thmitem{def:semimodule/homomorphism}\mcite[156]{Golan1999Semirings} A \hyperref[def:first_order_homomorphism]{first-order homomorphism} between two \( R \)-semimodules \( M \) and \( N \) is a function \( \varphi: M \to N \) that is a \hyperref[def:monoid/homomorphism]{monoid homomorphism} and satisfies
    \begin{equation}\label{eq:def:semimodule/homomorphism/compatibility}
      \varphi \bincirc m_r^M = m_r^N \bincirc \varphi.
    \end{equation}

    We will see in \fullref{thm:semimodule_homomorphism_iff_linear} that the homomorphisms are precisely the \hyperref[def:linear_function]{linear functions}.

    \thmitem{def:semimodule/submodel}\mcite[150]{Golan1999Semirings} The set \( A \subseteq M \) is a \hyperref[def:first_order_submodel]{submodel} of \( M \) if it is a \hyperref[def:monoid/submodel]{submonoid} of \( M \) that is closed under scalar multiplication, i.e. \( rM = m_r[M] \subseteq M \) for every \( r \in R \). We say that \( A \) is an \( R \)-\term{sub-semimodule} of \( M \).

    If \( M \) is a semimodule over some semiring extension \( T \) of \( R \), \( A \) may not be a \( T \)-sub-semimodule. For this reason, we should only use the term \enquote{sub-semimodule} (without specifying the semiring) if the underlying ring is clear from the context.

    As a consequence of \fullref{thm:positive_formulas_preserved_under_homomorphism}, the \hyperref[def:set_valued_map/image]{image} of an \( R \)-semimodule homomorphism \( \varphi: M \to N \) is an \( R \)-sub-semimodule of \( M \).

    \thmitem{def:semimodule/generated}\mimprovised For an arbitrary set \( A \), we denote the \hyperref[def:first_order_generated_substructure]{generated submodel} by \( \linspan{ A } \) and call it the \term[ru=линейная оболочка (\cite[sec. 3.2]{Тыртышников2007ЛинейнаяАлгебра})]{linear span} of \( A \).

    \Fullref{rem:span_over_different_semirings} shows how it is important to be unambiguous about over which semiring we take the span of \( A \). In case of possible ambiguity, we will use subscripts like \( \linspan_R A \).

    The linear span can be characterized via \hyperref[def:linear_combination]{linear combinations} --- see \fullref{ex:def:first_order_substructure/vector_space}.

    \thmitem{def:semimodule/category}\mimprovised For a fixed semiring \( R \), the \hyperref[def:category_of_small_first_order_models]{category of \( \mscrU \)-small models} \( \ucat{SMod}_R \) of left semimodules is \hyperref[def:concrete_category]{concrete} over \hyperref[def:monoid]{\( \ucat{Mon} \)}.

    Other notations are in use, for example \( R-\cat{Mod} \) for \( R \)-modules by \incite[158]{Aluffi2009Algebra}, that better highlight whether we are considering left or right (semi)modules. We will prefer \( \cat{SMod}_R^\oppos \) for the category of right semimodules.

    \thmitem{def:semimodule/bisemimodule}\mcite[149]{Golan1999Semirings} An \( (R, T) \)-\term{bisemimodule} is a triple \( (M, R, T) \), where \( M \) is an abelian group that is both a left \( R \)-semimodule and a right \( T \)-semimodule, and the following associativity condition holds for \( m \in M \), \( r \in R \) and \( t \in T \):
    \begin{equation}\label{eq:def:semimodule/bisemimodule/associativity}
      (r \cdot_A m) \cdot_B t = r \cdot_A (m \cdot_B t).
    \end{equation}
  \end{thmenum}
\end{definition}
\begin{defproof}
  \ImplicationSubProof{def:semimodule/action}{def:semimodule/operation} Fix a semiring homomorphism \( \varphi: R \to \End(M) \) and define the operation \( r \cdot x \coloneqq \varphi(r)(x) \).

  We will verify that all conditions from \fullref{def:semimodule/operation} hold for this operation.

  \begin{itemize}
    \item By definition, \( \varphi \) is a monoid action of \( (R, \cdot) \) on \( (M, \bincirc) \).

    \item Distributivity of scalar addition holds because \( \varphi \) is a \hyperref[def:semigroup/homomorphism]{semigroup homomorphism} from \( (R, +) \) to \( (M, +) \).

    \item Distributivity of vector addition holds because, for each \( r \), \( \varphi(r) \) is a semigroup endomorphism of \( (M, +) \).

    \item Since \( \varphi \) is a monoid homomorphism from \( (R,  +) \) to \( (R, \cdot) \), it preserves identities and hence
    \begin{equation*}
      0_R \cdot x = \varphi(0_R)(x) = [y \mapsto 0_M](x) = 0_M.
    \end{equation*}

    This proves half of \eqref{eq:def:semimodule/operation/absorption}.

    \item Since, for each \( r \), \( \varphi(r) \) is a monoid endomorphism of \( (M, +) \), we have
    \begin{equation*}
      r \cdot 0_M = \varphi(r)(0_M) = 0_M.
    \end{equation*}

    This proves the other half of \eqref{eq:def:semimodule/operation/absorption}.
  \end{itemize}

  \ImplicationSubProof{def:semimodule/operation}{def:semimodule/action} Let \( \cdot: R \times M \to M \) be an operation satisfying all conditions from \fullref{def:semimodule/operation}. Define the function \( \varphi(r) \coloneqq (x \mapsto r \cdot x) \). We will show that this is a semiring homomorphism.

  The operation preserves both identities because
  \begin{equation*}
    \varphi(0_R) = (x \mapsto 0) = 0_{\End(M)}
  \end{equation*}
  and
  \begin{equation*}
    \varphi(1_R) = (x \mapsto x) = \id_M.
  \end{equation*}

  We must also show that it preserves both binary operations. Clearly
  \begin{equation*}
    \varphi(r + s)
    =
    (x \mapsto (r + s) x)
    \reloset {\eqref{eq:def:semiring/right_distributivity}} =
    (x \mapsto r x + s x)
    =
    (x \mapsto r x) + (x \mapsto s x)
    =
    \varphi(r) + \varphi(s).
  \end{equation*}

  For multiplication, we have
  \begin{equation*}
    \varphi(rs)
    =
    (x \mapsto (rs)x)
    \reloset {\eqref{eq:def:binary_operation/associative}} =
    (x \mapsto r(sx))
    =
    \parens[\Big]{ x \mapsto \varphi(r)\parens[\Big]{ \varphi(s)(x) } }
    =
    \varphi(r) \bincirc \varphi(s).
  \end{equation*}
\end{defproof}

\begin{proposition}\label{thm:def:semimodule}
  \hyperref[def:semimodule]{Semimodules} have the following basic properties:
  \begin{thmenum}
    \thmitem{thm:def:semimodule/union} The union of a \hyperref[def:order_function/ascending]{ascending sequence} \( N_1 \subseteq N_2 \subseteq \cdots \) of \( R \)-\hyperref[def:semimodule/submodel]{sub-semimodules} of \( M \) is also an \( R \)-sub-semimodule of \( M \).
  \end{thmenum}
\end{proposition}
\begin{proof}
  \SubProofOf{thm:def:semimodule/union} Trivial.
\end{proof}

\begin{proposition}\label{thm:bisemimodule_over_submonoid}
  If \( S \) is (isomorphic to) a sub-semiring of \( R \), then \( R \) is an \( S \)-bisemimodule with scalar multiplication given by multiplication in \( R \).
\end{proposition}
\begin{comments}
  \item This result specializes to algebras over semirings in \fullref{thm:algebra_over_subring}.
\end{comments}
\begin{proof}
  We must show that \( \cdot \) satisfied the conditions in \fullref{def:semimodule/operation}.
  \begin{itemize}
    \item The identity law \eqref{eq:def:semimodule/operation/scalar_multiplication_action/identity} holds because \( 1 \) is a multiplicative identity of \( M \).
    \item The associativity-like law \eqref{eq:def:semimodule/operation/scalar_multiplication_action/compatibility} follows from associativity of multiplication.
    \item The two distributivity laws \eqref{eq:def:semimodule/operation/scalar_addition_distributivity} and \eqref{eq:def:semimodule/operation/vector_addition_distributivity} follow from left and right distributivity on \( R \).
    \item The absorption law \eqref{eq:def:semimodule/operation/absorption} follows from absorption on semirings.
  \end{itemize}

  All the above also hold for right semimodules rather than left.
\end{proof}

\begin{proposition}\label{thm:commutative_monoid_is_semimodule}
  The categories \( \hyperref[def:monoid/category]{\cat{CMon}} \) of commutative monoids and \( \hyperref[def:semimodule/category]{\cat{SMod}_\BbbN} \) of natural number semimodules are \hyperref[rem:category_similarity/isomorphism]{isomorphic}.

  More concretely, every commutative monoid \( M \) is a left semimodule over \( \BbbN \) with scalar multiplication given by \hyperref[con:additive_semigroup/multiplication]{recursively defined multiplication}
  \begin{equation}\label{eq:thm:commutative_monoid_is_semimodule/operation}
    \begin{aligned}
      &\cdot: \BbbN \times M \to M \\
      &n \cdot x \coloneqq \begin{cases}
        0_M,           &n = 0, \\
        n \cdot x + x, &n > 1.
      \end{cases}
    \end{aligned}
  \end{equation}

  Conversely, in every semimodule over \( \BbbN \), scalar multiplication matches the recursively defined multiplication.
\end{proposition}
\begin{comments}
  \item This result specializes to algebras over semirings in \fullref{thm:semiring_is_natural_number_algebra} and modules over rings in \fullref{thm:abelian_group_is_module}.
\end{comments}
\begin{proof}
  \SufficiencySubProof Let \( M \) be a commutative monoid. The operation \( \cdot: \BbbN \times M \to M \) defined in \fullref{thm:semiring_characteristic_homomorphism} satisfies the conditions in \fullref{def:semimodule/operation} as either a direct consequence of the definition or as a consequence of \fullref{thm:monoid_distributivity}.

  The homomorphisms are thus also compatible.

  \NecessitySubProof Let \( M \) be a semimodule over \( \BbbN \). We will use induction to show that \eqref{eq:thm:commutative_monoid_is_semimodule/operation} holds.
  \begin{itemize}
    \item For \( n = 0 \), this follows from the absorption law \eqref{eq:def:semimodule/operation/absorption}.
    \item If \( n \cdot x = n \cdot x + x \), then by scalar distributivity, \( (n + 1) \cdot x = n \cdot x + 1 \cdot x \). The multiplicative identity law \eqref{eq:def:semimodule/operation/scalar_multiplication_action/identity} then shows that \( 1 \cdot x = x \), which concludes our proof.
  \end{itemize}

  The homomorphisms are thus also compatible.
\end{proof}

\paragraph{Linear functions}

\begin{definition}\label{def:homogeneous_function}\mimprovised
  We say that a function \( f: M \to N \) between \( R \)-\hyperref[def:semimodule]{semimodules} is \term[ru=однородная (\cite[416]{Зорич2019АнализЧасть1}), en=homogeneous (\cite[def. 2.3.1]{HillePhillips1996FunctionalAnalysis})]{homogeneous} of degree \( d \) if, for every scalar \( r \) and ever vector \( x \), we have
  \begin{equation}\label{eq:def:homogeneous_function}
    f(rx) = r^d \cdot f(x).
  \end{equation}

  We will shorten \enquote{homogeneous of degree \( 1 \)} to simply \enquote{homogeneous}.
\end{definition}
\begin{comments}
  \item We generalize this definition from \incite[def. 2.3.1]{HillePhillips1996FunctionalAnalysis}, who define homogeneous functions between \hyperref[def:topological_vector_space]{topological vector spaces} without degrees, and \incite[416]{Зорич2019АнализЧасть1}, who defines homogeneous real-valued functions on \hyperref[def:euclidean_space]{Euclidean spaces} with positive real-valued degrees.
\end{comments}

\begin{definition}\label{def:linear_function}\mimprovised
  We say that a function between \( R \)-\hyperref[def:semimodule]{semimodules} is \term[bg=линейно преобразувание (\cite[101]{Обрешков1962ВисшаАлгебра}), ru=линейный оператор (\cite[236]{Тыртышников2007ЛинейнаяАлгебра}), en=linear transformation (\cite[def. 2.3.2]{HillePhillips1996FunctionalAnalysis})]{linear} if it is \hyperref[def:additive_function]{additive} and \hyperref[def:homogeneous_function]{homogeneous}.
\end{definition}
\begin{comments}
  \item As a consequence of \fullref{thm:semimodule_homomorphism_iff_linear}, the linear functions between semimodules are precisely the \hyperref[def:semimodule/homomorphism]{homomorphisms} between them.

  \item \incite*[def. 2.3.2]{HillePhillips1996FunctionalAnalysis} state
  \begin{displayquote}
    An additive and homogeneous transformation is said to be linear.
  \end{displayquote}

  Our underlying definitions of \enquote{additive} and \enquote{homogeneous} differ in their generality, however --- see the comments to \fullref{def:additive_function} and \fullref{def:homogeneous_function}.
\end{comments}

\begin{proposition}\label{thm:semimodule_homomorphism_iff_linear}
  A function between \( R \)-\hyperref[def:semimodule]{semimodules} is a \hyperref[def:semimodule/homomorphism]{semimodule homomorphism} if and only if it is \hyperref[def:linear_function]{linear}.
\end{proposition}
\begin{proof}
  Fix a function \( \varphi: M \to N \).

  Homogeneity is simply a restatement of \eqref{eq:def:semimodule/homomorphism/compatibility}: indeed, for any scalar \( r \) and any vector \( x \), we have
  \begin{equation}\label{eq:thm:semimodule_homomorphism_iff_linear/proof/homogeneity}
    \varphi(rx)
    =
    \varphi(m_r^M(x))
    \reloset{\eqref{eq:def:semimodule/homomorphism/compatibility}} =
     m_r^N(\varphi(x))
     =
     r \cdot \varphi(x).
  \end{equation}

  Thus, it remains to show that \( \varphi \) is a homomorphism of the additive monoids if and only if it is additive. This is false in general since monoid homomorphisms do not automatically preserve neutral elements. Due to homogeneity, however, for any element \( x \) of \( M \) we have
  \begin{equation*}
    \varphi(0_M)
    \reloset {\eqref{eq:def:semimodule/operation/absorption}} =
    \varphi(0 \cdot x)
    =
    0 \cdot \varphi(x)
    \reloset {\eqref{eq:def:semimodule/operation/absorption}} =
    0_N.
  \end{equation*}
\end{proof}

\paragraph{Semimodule direct sums}

\begin{definition}\label{def:function_support}\mcite[19]{Golan1999Semirings}
  We define the \term[bg=носител (\cite[58]{Боянов2008ЧислениМетоди}), ru=носитель (\cite[135]{КанторовичАкилов1984ФункциональныйАнализ})]{support} of a function \( f: S \to R \) from any set \( S \) to a semiring \( R \) as the set
  \begin{equation*}
    \supp(f) \coloneqq \set{ x \in S \given f(x) \neq 0_R }.
  \end{equation*}
\end{definition}

\begin{definition}\label{def:semimodule_direct_sum}\mimprovised
  We define the \term{external direct sum} or simply \term{direct sum} \( \bigoplus_{k \in \mscrK} M_k \) of the family of \( R \)-\hyperref[def:semimodule]{semimodules} \( \seq{ M_k }_{k \in \mscrK} \) of \( \Gamma \) as the subset of their \hyperref[def:first_order_direct_product]{direct product} \( \prod_{k \in \mscrK} M_k \) consisting of tuples with finite \hyperref[def:function_support]{support}.

  The sums and scalar products of tuples with finite support also have finite support, hence the direct sum is an \( R \)-\hyperref[def:semimodule/submodel]{sub-semimodule} of the direct product.

  \begin{thmenum}
    \thmitem{def:semimodule_direct_sum/inclusion} For every \( m \in \mscrK \), we define the following \term{canonical inclusion} \hyperref[def:semimodule/homomorphism]{homomorphism}:
    \begin{equation*}
      \begin{aligned}
        &\iota_m: M_m \to \bigoplus_{k \in \mscrK} M_k, \\
        &\iota_m(x) \coloneqq \begin{rcases}
          \begin{cases}
            x, &k = m \\
            0, &k \neq m
          \end{cases}
        \end{rcases}_{k \in \mscrK}
      \end{aligned}
    \end{equation*}

    \thmitem{def:semimodule_direct_sum/power} In case all summands are equal to \( M \), we denote the direct sum by \( M^{\oplus \mscrK} \).
    \thmitem{def:semimodule_direct_sum/internal} If all summands are submodels of \( M \) and if the sum \( \bigoplus_{k \in \mscrK} M_k \) is isomorphic to \( M \), we call it an \term{internal direct sum} and treat the tuple \( \seq{ x_k }_{k \in \mscrK} \) as the product \( x_{k_1} x_{k_2} \cdots x_{k_n} \) of the elements of \( M \) distinct from zero.
  \end{thmenum}
\end{definition}

\begin{proposition}\label{thm:semimodule_coproduct}
  The \hyperref[def:discrete_category_limits]{categorical coproduct} of the family \( \seq{ M_k }_{k \in \mscrK} \) in the \hyperref[def:semimodule/category]{category of \hi{commutative} semimodules} is their \hyperref[def:semimodule_direct_sum]{direct sum} \( \bigoplus_{k \in \mscrK} M_k \).
\end{proposition}
\begin{proof}
  First note that the sum \( A \coloneqq \bigoplus_{k \in \mscrK} A_k \) with the \hyperref[def:semimodule_direct_sum/inclusion]{inclusions} \( \iota \coloneqq \seq{ \iota_k }_{k \in \mscrK} \) are a \hyperref[def:category_of_cones/cocone]{cocone} for the \hyperref[def:discrete_category]{discrete} \hyperref[def:categorical_diagram]{diagram} \( \seq{ A_k }_{k \in \mscrK} \).

  Let \( (C, \alpha) \) also be a cocone. We want to define a semimodule homomorphism
  \begin{equation*}
    l_C: \bigoplus_{k \in \mscrK} A_k \to A
  \end{equation*}
  such that, for every \( m \in \mscrK \) and \( x \in M_m \),
  \begin{equation*}
    \alpha_m(x) = l_A(\iota_m(x)).
  \end{equation*}

  Thus, the value of \( l_A(c) \) on members of the inclusion \( \iota_m[M_m] \) is entirely determined by \( \alpha_m \). This suggests the definition
  \begin{equation*}
    l_C(\seq{ x_k }_{k \in \mscrK}) \coloneqq \prod_{k \in \mscrK}^n \alpha_k(x_k).
  \end{equation*}

  We discuss well-definedness of infinitary operations in direct sums in \fullref{rem:binary_operation_syntax_trees/infinite/direct_sum}.
\end{proof}

\paragraph{Free semimodules}

\begin{definition}\label{def:free_semimodule}\mimprovised
  Fix a \hyperref[def:semiring]{semiring} \( R \). We associate with every \hyperref[def:set]{plain set} \( A \) its \term{free \( R \)-semimodule} \( R^{\oplus A} \) over \( R \) defined as the set
  \begin{equation*}
    R^{\oplus A} \coloneqq \bigoplus_{x \in A} R = \set{ t: A \to R \given t \T{has finite \hyperref[def:function_support]{support}} }.
  \end{equation*}

  \begin{thmenum}
    \thmitem{def:free_semimodule/operations} \Fullref{thm:functions_over_model_form_model} implies that \( R^{\oplus A} \) is a semiring with the elementwise addition and multiplication inherited from \( R \). Scalar multiplication by \( r \in R \) can be defined as multiplication by the \hyperref[def:constant_function]{constant function} at \( r \).

    \thmitem{def:free_semimodule/inclusion} We start with the canonical inclusion
    \begin{equation*}
      \begin{aligned}
        &\iota_A: A \to R^{\oplus A}, \\
        &\iota_A(x) \coloneqq \parens[\Bigg]
          {
            y \mapsto \begin{rcases}
              \begin{cases}
                1_R, &y = x \\
                0_R, &y \neq x
              \end{cases}
            \end{rcases}
          }
      \end{aligned}
    \end{equation*}
  \end{thmenum}
\end{definition}
\begin{comments}
  \item In accordance with \fullref{rem:free_construction_univalence}, we refer to any (semi)module isomorphic to \( R^{\oplus A} \) as \enquote{free}, and to \( R^{\oplus A} \) itself as \enquote{the} free (semi)module. For \hyperref[def:module]{modules} this reduces to finding a \hyperref[def:hamel_basis]{Hamel basis} --- see \fullref{rem:free_module}.

  \item We will use the notation \fullref{rem:free_semimodule_notation} for semimodules.

  \item In the case when \( R \) is the semiring \( \BbbN \) of natural numbers, \( \BbbN^{\oplus A} \) is the set of finite \hyperref[def:multiset]{multisets} over \( S \).
\end{comments}

\begin{remark}\label{rem:free_semimodule_notation}
  We can use the inclusion \fullref{def:free_semimodule/inclusion} of \( A \) into its \hyperref[def:free_semimodule]{free semimodule} \( R^{\oplus A} \) to introduce a simpler notation.

  For any \( t: A \to R^{\oplus A} \) and any \( a \in A \), denote by \( t_a \) the constant function taking the value \( t(a) \)\fnote{In \fullref{def:free_semimodule/operations} we defined scalar multiplication by \( t(a) \) as multiplication with \( t_a \). Having a clear distinction between functions and scalars makes the derivation of \eqref{eq:def:free_semimodule/sum} clearer}.

  Then
  \begin{equation*}
    t_a(x) \cdot \iota_A(a)(x)
    =
    \begin{cases}
      t(a) \cdot_R 0_R, &a \neq x, \\
      t(a) \cdot_R 1_R, &a = x
    \end{cases}
  \end{equation*}

  On other words, the product \( t_a(x) \cdot \iota_A(a)(x) \) is zero for every \( x \neq a \) and coincides with \( t(x) \) when \( x = a \).  As discussed in \fullref{rem:binary_operation_syntax_trees/infinite/direct_sum}, an infinite sum is well-defined if only finitely many summands are nonzero, thus
  \begin{equation*}
    t(x) = \sum_{a \in A} t_a(x) \cdot_R \iota_A(a)(x),
  \end{equation*}
  which can also be written more succinctly as
  \begin{equation*}
    t = \sum_{a \in A} t_a \cdot \iota_A(a).
  \end{equation*}

  To make the notation simpler, we can write \( a \) instead of \( \iota_A(a) \):
  \begin{equation}\label{eq:def:free_semimodule/sum}
    t = \sum_{a \in A} t_a \cdot a.
  \end{equation}

  Outside this remark, we will regard \( t_a \) as a scalar and \( a \) as a vector.
\end{remark}

\begin{theorem}[Free semimodule universal property]\label{thm:free_semimodule_universal_property}
  Fix a semiring \( R \) and a set \( A \). The \hyperref[def:free_semimodule]{free \( R \)-semimodule} \( R^{\oplus A} \) over \( R \) is the unique up to a unique isomorphism semimodule that satisfies the following \hyperref[rem:universal_mapping_property]{universal mapping property}:
  \begin{displayquote}
    For every semimodule \( M \) over \( R \) and every function \( e: A \to M \), there exists a unique \( R \)-semimodule homomorphism \( \Phi_e: R^{\oplus A} \to M \) such that the following diagram commutes:
    \begin{equation}\label{eq:thm:free_semimodule_universal_property/diagram}
      \begin{aligned}
        \includegraphics[page=1]{output/thm__free_semimodule_universal_property}
      \end{aligned}
    \end{equation}
  \end{displayquote}
\end{theorem}
\begin{comments}
  \item Via \fullref{rem:universal_mapping_property}, \( A \mapsto R^{\oplus A} \) becomes \hyperref[def:category_adjunction]{left adjoint} to the \hyperref[def:concrete_category]{forgetful functor}
  \begin{equation*}
    U: \cat{SMod}_R \to \cat{Set}.
  \end{equation*}

  \item \hyperref[con:evaluation_homomorphism]{Evaluation homomorphisms} for polynomials are actually defined via the map \( \Phi_e \) discussed here. We thus refer to \( \Phi_e \) as an evaluation homomorphism.
\end{comments}
\begin{proof}
  \ExistenceSubProof For every function \( e: A \to M \), we want
  \begin{equation*}
    \Phi_e(\iota(x)) = e(x).
  \end{equation*}

  This suggests the definition
  \begin{equation*}
    \Phi_e(\sum_{a \in A} t_a \cdot a) \coloneqq \sum_{x \in A} \sum_{a \in A} t_a \cdot e(a).
  \end{equation*}

  \UniquenessSubProof Fix a linear map \( \Psi_e: R^{\oplus A} \to M \) such that
  \begin{equation*}
    \Psi_e(\iota(x)) = e(x).
  \end{equation*}

  We will use induction on the number of nonzero entries in \( \seq{ t_x }_{x \in A} \) to show that \( \Phi_e \) and \( \Psi_e \) coincide. The base case where all entries are zero is obvious since
  \begin{equation*}
    \Psi_e(\iota(0_R)) = e(0_r) = \Phi_e(\iota(0_R)).
  \end{equation*}

  Now suppose that
  \begin{equation*}
    \Psi_e(\iota(x)) = e(x),
  \end{equation*}
  and that \( \Psi_e \) coincides with \( \Phi_e \) for every element of \( R^{\oplus A} \) with \( n - 1 \) entries. Given \( \seq{ t_x }_{x \in A} \) with \( n \) nonzero entries, fix a nonzero \( t_{x_0} \) and define
  \begin{equation}
    t'_x \coloneqq \begin{cases}
      0,   &x = x_0, \\
      t_x, &\T{otherwise.}
    \end{cases}
  \end{equation}

  Then
  \begin{balign*}
    \Psi_e(\seq{ t_x }_{x \in A})
    &=
    \sum_{x \neq x_0} t_x \cdot e(x) + t_{x_0} \cdot e(x_0)
    = \\ &=
    \Psi_e(\seq{ t'_x }_{x \in A}) + t_{x_0} \cdot e(x_0)
    \reloset {\T{ind.}} = \\ &=
    \Phi_e(\seq{ t'_x }_{x \in A})
    = \\ &=
    \Phi_e(\seq{ t_x }_{x \in A}).
  \end{balign*}

  This concludes the proof.
\end{proof}

\begin{concept}\label{con:indeterminate}
  When discussing free and bound variables in \fullref{con:variable_binding}, we rely on the underlying \hyperref[con:logical_system]{logical systems} providing a distinction.

  Outside logic, such a distinction is, for the most part, clear from the context --- for example, in the \hyperref[def:semimodule/theory]{logical theory of (semi)modules}, \enquote{\( x \)} may denote a fixed vector if it is bound, or it may denote an \hyperref[con:variable_dependence]{independent variable} otherwise; and it is usually clear whether \( x \) is free or bound.

  We must sometimes take one step back, however. The expression \( x + y \) is simply a string of symbols, however, on a semantic level, it is either a vector if both \( x \) and \( y \) are \hyperref[con:variable_binding]{bound variables}, or otherwise it is an \hyperref[con:variable_dependence]{implicit function}. This removes from us the essence of the first-order term \( x + y \), encoding the precise relation of the variables.

  For this reason we introduce \hyperref[def:linear_combination]{linear combinations} and \hyperref[def:polynomial_algebra]{polynomials}, and even more general \hyperref[con:free_construction]{free constructions}. We start with a list of \hyperref[def:formal_language/symbol]{symbols} --- in our case \( X \) and \( Y \) --- which we call \term[bg=неизвестно (\cite[405]{Обрешков1962ВисшаАлгебра}), ru=неизвестная (\cite[131]{Курош1968КурсВысшейАлгебры}), en=indeterminate (\cite[def. III.1.19]{Aluffi2009Algebra})]{indeterminates}, and we use them to build a specially constructed free module. Then we proceed to perform operations on the indeterminates, and the particular structure of the free module ensures that the operations are simply algebraic encodings of the analogous logical terms. For example, the linear combination \( X + Y \) can be evaluated via \fullref{thm:free_semimodule_universal_property} in any module to produce a concrete vector. Linear combinations and their sums are thus instances of the \hyperref[con:syntax_semantics_duality]{syntax-semantics duality}.

  Even though this is not necessary, we will use capital letters to distinguish indeterminates from values they may evaluate to. This convention, as well as alternative terminology, is discussed in \fullref{rem:conventions_for_indeterminates}.
\end{concept}

\begin{definition}\label{def:linear_combination}\mimprovised
  Fix some list of \hyperref[con:indeterminate]{indeterminates} \( X_1, \ldots, X_n \).

  A \term[ru=линейная комбинация (\cite[\S 3.2]{Тыртышников2007ЛинейнаяАлгебра}), en=linear combination (\cite[39]{FriedbergInselSpence2018LinearAlgebra})]{linear combination} in \( X_1, \ldots, X_n \) is simply an element of the \hyperref[def:free_semimodule]{free semimodule} over the set of indeterminates.

  For every linear combination \( t \), we can adapt the notation from \fullref{rem:free_semimodule_notation} to write
  \begin{equation}\label{eq:def:linear_combination}
    \sum_{k=1}^n t_k \cdot X_k.
  \end{equation}

  We refer to the scalars \( t_k \) in \eqref{eq:def:linear_combination} as \term[ru=коэффициенты (линейной комбинации) (\cite[\S 3.2]{Тыртышников2007ЛинейнаяАлгебра}), en=coefficients (\cite[39]{FriedbergInselSpence2018LinearAlgebra})]{coefficients} and to the products \( t_k \cdot X_k \) as \term{terms}.

  \begin{thmenum}
    \thmitem{def:linear_combination/trivial} If all coefficients are zero, we say that the linear combination is \term[ru=тривиальная (линейная комбинация) (\cite[\S 3.3]{Тыртышников2007ЛинейнаяАлгебра}), en=trivial (linear combination) (\cite[8]{Treil2017LinearAlgebraDoneWrong})]{trivial}.

    \thmitem{def:linear_combination/sum} For every \( R \)-semimodule \( M \) and every list \( m_1, \ldots, m_n \) of vectors in \( M \), \fullref{thm:free_semimodule_universal_property} allows us to evaluate \eqref{eq:def:linear_combination} to obtain a vector in \( M \):
    \begin{equation}\label{eq:def:linear_combination/sum}
      \sum_{k=1}^n t_k \cdot m_k.
    \end{equation}

    We will refer to \eqref{eq:def:linear_combination/sum} as a \enquote{linear combination \hi{of the vectors}} \( m_1, \ldots, m_n \).
  \end{thmenum}
\end{definition}
\begin{comments}
  \item Linear combinations and their sums are thus an instance of the \hyperref[con:syntax_semantics_duality]{syntax-semantics duality}. We base our presentation on \hyperref[def:polynomial_algebra]{polynomials}, where this distinction is made clear.

  Unfortunately, it is not established practice to distinguish between a linear combination and its sum, which formally renders the discussions of coefficients and terms nonsensical. Linear combinations are \hyperref[con:assumed_knowledge]{assumed knowledge} and even informal definitions can rarely be found. Even those authors like \incite[\S 3.2]{Тыртышников2007ЛинейнаяАлгебра} and \incite[6]{Treil2017LinearAlgebraDoneWrong} that give definitions do not delve into the syntax-semantics duality.
\end{comments}

\begin{remark}\label{rem:conventions_for_indeterminates}
  In the cases where we wish to distinguish between \hyperref[con:indeterminate]{indeterminates} and more general metasyntactic variables, we employ the convention of denoting the variables with lower-case letters (e.g. \( x \) or \( y \)) and the indeterminates with their upper-case analogous (e.g. \( X \) or \( Y \)). This tradition is used, and perhaps introduced, by \hyperref[rem:bourbaki]{Bourbaki} in \cite[A III.25]{Bourbaki1970Algèbre1à3}, however it is not popular --- out of all other authors we mention here, only \incite[9]{Knapp2016BasicAlgebra} uses it.

  The term \enquote{indeterminate} is also not standardized. We summarize how different authors approach indeterminates in \hyperref[def:polynomial_algebra]{polynomial algebras}:
  \begin{itemize}
    \item The term \enquote{indeterminate} is used mostly preferred anglophone literature:  it is used by
    \incite[122]{Jacobson1985BasicAlgebraI},
    \incite[43]{Rotman2015AdvancedModernAlgebraPart1},
    \incite[23]{Eisenbud1995CommutativeAlgebra},
    \incite[def. III.1.19]{Aluffi2009Algebra},
    \incite[36]{Golan1999Semirings},
    \incite[9]{Knapp2016BasicAlgebra},
    \incite[182]{Кострикин2000АлгебраЧасть1} (as both \enquote{переменная} (\enquote{variable}) and \enquote{неизвестная} (\enquote{unknown}))
    \incite[131]{Курош1968КурсВысшейАлгебры} (as \enquote{неизвестная} (\enquote{unknown}))
    \incite[405]{Обрешков1962ВисшаАлгебра} (as \enquote{неизвестно} (\enquote{unknown}))

    \item The term \enquote{variable} is used by
    \incite[97]{Lang2002Algebra} (as \enquote{variable})
    \incite[92]{Винберг2014КурсАлгебры} (as \enquote{переменная} (\enquote{variable}))
    \incite[135]{ГеновМиховскиМоллов1991Алгебра} (as \enquote{променлива} (\enquote{variable}))
    \incite[4]{КоцевСидеров2016КомутативнаАлгебра} (as \enquote{променлива} (\enquote{variable}))

    \item The term \enquote{letter} is used by
    \incite[12]{Тыртышников2017ОсновыАлгебры} (as \enquote{буква} (\enquote{letter}))
  \end{itemize}
\end{remark}

\begin{proposition}\label{thm:span_via_linear_combinations}
  For a set \( A \) in an \( R \)-\hyperref[def:semimodule]{semimodule} \( M \), the \hyperref[def:semimodule/generated]{linear span} of \( A \) equals the set of all \hyperref[def:linear_combination]{linear combinations} over \( A \).
\end{proposition}
\begin{comments}
  \item Compare this result to \fullref{thm:generators_via_polynomials} for algebras and polynomials.
\end{comments}
\begin{proof}
  \Cref{fig:thm:span_via_linear_combinations} shows an \hyperref[con:abstract_syntax_tree]{abstract syntax tree} for a given linear combination, which can be traversed and evaluated to obtain a vector in \( M \). This vector must be a member of the span of \( S \) since the latter is closed under vector addition and scalar multiplication with members of \( S \). Hence, the set \( L \) of all linear combinations over \( S \) is a subset of the span.

  Generalizing the syntax tree construction from \cref{fig:thm:span_via_linear_combinations}, we see that \( L \) satisfies \fullref{def:first_order_substructure/universe/inductive}, and is thus a submodule of \( M \). Since the span is the smallest module containing \( S \), we have \( L = \linspan S \).

  \begin{figure}[!ht]
    \hfill
    \includegraphics[page=1]{output/thm__span_via_linear_combinations}
    \hfill\hfill
    \caption{A linear combination is simply a \hyperref[con:function_superposition]{superposition} of scalar multiplication and binary addition.}
    \label{fig:thm:span_via_linear_combinations}
  \end{figure}
\end{proof}

\begin{remark}\label{rem:span_over_different_semirings}
  If \( M \) is both an \( R \)-semimodule and a \( T \)-semimodule, \fullref{thm:span_via_linear_combinations} highlights a fundamental difference between the generated \( R \)-sub-semimodule and the generated \( T \)-sub-semimodule.

  For example, the \( \BbbN \)-sub-semimodule generated by \( 2 \) is the semiring \( 2\BbbN \) of even natural numbers, while the \( \BbbR_{\geq 0} \)-sub-semimodule generated by \( 2 \) is \( \BbbR_{\geq 0} \) itself.
\end{remark}

\paragraph{Free commutative monoids}

\begin{remark}\label{rem:free_commutative_monoid_as_quotient}
  Consider the \hyperref[def:free_monoid]{free monoid} \( A^* \) over some \hyperref[def:set]{plain set} \( A \) and also the \hyperref[def:free_semimodule]{free \( \BbbN \)-semimodule} \( \BbbN^{\oplus A} \).

  Define the homomorphism
  \begin{equation*}
    \begin{aligned}
      &\varphi: A^* \to \BbbN^{\oplus A}, \\
      &\varphi(x_1 \cdots x_n) \coloneqq \seq[\Big]{ \underbrace{\sum_{k=1}^n 1_{a = x_k}}_{\mathclap{\T*{repetitions of} a \T*{among} x_1 \cdots x_n} }}_{a \in A}
    \end{aligned}
  \end{equation*}

  By \fullref{thm:homomorphism_induces_congruence}, \( \varphi \) induces a \hyperref[def:first_order_congruence]{monoid congruence} \( \cong \) on \( A^* \) where two strings are congruent if they have the same amount of each symbol from \( A \).

  This ensures that the strings \enquote{\( abc \)}, \enquote{\( bac \)}, \enquote{\( bca \)}, \enquote{\( cba \)}, \enquote{\( cab \)} and \enquote{\( acb \)} are congruent --- these are precisely all variations of \enquote{\( abc \)} that can be obtained via \hyperref[def:transposition]{transpositions} due to commutativity.

  The \hyperref[def:first_order_quotient]{quotient} \( A^* / \cong \) is isomorphic to \( \BbbN^{\oplus A} \) as a monoid. This motivates the definition of free commutative monoids in \fullref{def:free_commutative_monoid}.
\end{remark}

\begin{definition}\label{def:free_commutative_monoid}\mimprovised
  We associate with every \hyperref[def:set]{plain set} \( A \) its \term{free commutative monoid} defined as the \hyperref[def:free_semimodule]{free \( \BbbN \)-module} \( \BbbN^{\oplus A} \).
\end{definition}
\begin{comments}
  \item In accordance with \fullref{rem:free_construction_univalence}, we refer to any monoid isomorphic to \( \BbbN^{\oplus A} \) as \enquote{free}, and to \( \BbbN^{\oplus A} \) itself as \enquote{the} free monoid.

  \item We regard \( \BbbN^{\oplus A} \) only as a monoid, without generally considering scalars.

  \item The relation to \hyperref[def:free_monoid]{free monoids} is given in \fullref{rem:free_commutative_monoid_as_quotient}.
\end{comments}

\begin{theorem}[Free commutative monoid universal property]\label{thm:free_commutative_monoid_universal_property}
  Given a set \( A \), the \hyperref[def:free_commutative_monoid]{free commutative monoid} \( \BbbN^{\oplus A} \) is the unique up to a unique isomorphism commutative monoid that satisfies the following \hyperref[rem:universal_mapping_property]{universal mapping property}:
  \begin{displayquote}
    For every commutative monoid \( M \) and every function \( e: A \to M \), there exists a unique monoid homomorphism \( \Phi_e: \BbbN^{\oplus A} \to M \) such that the following diagram commutes:
    \begin{equation}\label{eq:thm:free_commutative_monoid_universal_property/diagram}
      \begin{aligned}
        \includegraphics[page=1]{output/thm__free_commutative_monoid_universal_property}
      \end{aligned}
    \end{equation}
  \end{displayquote}
\end{theorem}
\begin{comments}
  \item Via \fullref{rem:universal_mapping_property}, \( A \mapsto \BbbN^{\oplus A} \) becomes \hyperref[def:category_adjunction]{left adjoint} to the \hyperref[def:concrete_category]{forgetful functor}
  \begin{equation*}
    U: \cat{CMon} \to \cat{Set}.
  \end{equation*}
\end{comments}
\begin{proof}
  Follows from \fullref{thm:free_semimodule_universal_property} by noting that, as shown in \fullref{thm:commutative_monoid_is_semimodule}, commutative monoids are semimodules over \( \BbbN \).
\end{proof}
