\subsection{Semimodules}\label{subsec:semimodules}

\paragraph{Semimodules over semirings}

Semimodules are generalizations of monoid actions. Notation and terminology-wise, semimodules are somewhat special in that they are very much influenced by linear algebra and analysis, where vector spaces are crucial.

\begin{definition}\label{def:endomorphism_semiring}\mimprovised
  Let \( X \) be a monoid or, more generally, an object in a \hyperref[def:category]{category} that is \hyperref[def:concrete_category]{concrete} over \hyperref[def:monoid/category]{\( \cat{Mon} \)}.

  Let \( \End(X) \) be the \hyperref[def:endomorphism_monoid]{endomorphism monoid} over \( X \). These are necessarily monoid endomorphisms, however they may carry additional structure like being \hyperref[def:group/homomorphism]{group homomorphisms}, \hyperref[def:semimodule/homomorphism]{semimodule homomorphisms}, \hyperref[def:lattice/homomorphism]{(semi)lattice homomorphisms} or their \hyperref[rem:topological_first_order_structures]{continuous counterparts}.

  Define addition in \( \End(X) \) pointwise as \( [f + g](x) \coloneqq f(x) + g(x) \). Then \( \End(X) \) with pointwise addition and composition is a \hyperref[def:semiring]{semiring}, which we call the \term{endomorphism semiring} over \( X \).
\end{definition}

\begin{definition}\label{def:semimodule}
  Fix a \hyperref[def:semiring]{semiring} \( R \), whose elements we will call \term{scalars}, and an \hyperref[rem:additive_semigroup]{additive} \hyperref[def:binary_operation/commutative]{commutative} \hyperref[def:monoid]{monoid} \( M \), whose elements we will call \term{vectors}. See \fullref{rem:vector_etymology} for a discussion of the term \enquote{vector}.

  We say that \( M \) is a \term[ru=полумодуль (\cite[99]{ВечтомовПетров2022})]{semimodule} over \( R \) if they are compatible in any of the equivalent ways listed below. Analogously to \hyperref[def:monoid_action]{monoid actions}, if \( R \) is not commutative, we distinguish between left and right semimodules. Rather than \enquote{\( M \) is a semimodule over \( R \)}, it is often more convenient to say \enquote{\( M \) is an \( R \)-semimodule}.

  \begin{thmenum}[series=def:semimodule]
    \thmitem{def:semimodule/action} A left semimodule is a \hyperref[def:semiring/homomorphism]{homomorphism} from \( R \) to the \hyperref[def:endomorphism_semiring]{endomorphism semiring} \( \End(M) \). A right semimodule is a homomorphism from the \hyperref[def:semiring/opposite]{dual semiring} \( R^{-1} \) to \( \End(M) \).

    This definition is concise and natural, but unfortunately not very useful.

    \thmitem{def:semimodule/operation} The explicit way to define a left semimodule is via a binary operation \( \cdot: R \times M \to M \) called \term{scalar multiplication} that satisfies the following conditions:
    \begin{thmenum}
      \thmitem{def:semimodule/operation/scalar_multiplication_action} Scalar multiplication is a \hyperref[def:monoid_action]{monoid action} of the multiplicative monoid \( (R, \cdot_R) \) on \( M \). The following conditions correspond to \eqref{eq:def:monoid_action/family/identity} and \eqref{eq:def:monoid_action/family/compatibility}:
      \begin{align}
        &1_R \cdot x = x, \label{eq:def:semimodule/operation/scalar_multiplication_action/identity} \\
        &(r \cdot_R s) \cdot x = r \cdot (s \cdot x). \label{eq:def:semimodule/operation/scalar_multiplication_action/compatibility}
      \end{align}

      The second condition can be regarded as a form of associativity.

      \thmitem{def:semimodule/operation/scalar_addition_distributivity} Scalar addition distributes over scalar multiplication:
      \begin{equation}\label{eq:def:semimodule/operation/scalar_addition_distributivity}
        (r +_R s) \cdot x = r \cdot x + s \cdot x.
      \end{equation}

      \thmitem{def:semimodule/operation/vector_addition_distributivity} Vector addition distributes over scalar multiplication:
      \begin{equation}\label{eq:def:semimodule/operation/vector_addition_distributivity}
        r \cdot (x + y) = r \cdot x + r \cdot y.
      \end{equation}

      \thmitem{def:semimodule/operation/absorption} The scalar and vector zeros are compatible:
      \begin{equation}\label{eq:def:semimodule/operation/absorption}
        0_R \cdot x = 0_M = r \cdot 0_M.
      \end{equation}
    \end{thmenum}

    In practice, we use the same symbol for both scalar and vector addition, and we denote both scalar and vector multiplication via juxtaposition.
  \end{thmenum}

  Semimodules have the following metamathematical properties:
  \begin{thmenum}[resume=def:semimodule]
    \thmitem{def:semimodule/theory} In order to fit the heterogeneous operation \( \cdot \) into the framework of \hyperref[def:first_order_model]{first-order logic models}, we can extend the \hyperref[def:monoid/theory]{theory of monoids} by adding, for every semiring element \( r \), a unary \hyperref[def:first_order_language/fun]{functional symbol} \( m_r \).

    We have placed a restriction that the number of functional symbols must be finite, as discussed in \fullref{rem:uncountable_first_order_language}, hence this method is only available for finite semirings and is simply a conceptual sketch for infinite semirings.

    All conditions can then be reformulated via this operation. For example, \eqref{eq:def:semimodule/operation/scalar_multiplication_action/compatibility} corresponds to the axiom schema
    \begin{equation*}
      m_{rs}(\xi) = m_r(m_s(\xi)).
    \end{equation*}

    \thmitem{def:semimodule/homomorphism} A \hyperref[def:first_order_homomorphism]{first-order homomorphism} between two \( R \)-semimodules \( M \) and \( N \) is a function \( \varphi: M \to N \) that is a \hyperref[def:monoid/homomorphism]{monoid homomorphism} and satisfies \( \varphi \bincirc m_r^M = m_r^N \bincirc \varphi \).

    This can be expressed more clearly via the following two conditions, which we call \term{additivity} and \term{homogeneity}:
    \begin{align}
      \varphi(x + y) &= \varphi(x) + \varphi(y) \label{eq:def:semimodule/homomorphism/additive} \\
         \varphi(rx) &= r \cdot \varphi(x) \label{eq:def:semimodule/homomorphism/homogeneity}
    \end{align}

    Functions satisfying additivity and homogeneity are commonly called \term{linear}. These are a central object of study in \hyperref[sec:linear_algebra]{linear algebra} and, to a lesser extent, (linear) \hyperref[sec:functional_analysis]{functional analysis}.

    \thmitem{def:semimodule/submodel} The set \( A \subseteq M \) is a \hyperref[def:first_order_submodel]{submodel} of \( M \) if it is a \hyperref[def:monoid/submodel]{submonoid} of \( M \) that is closed under scalar multiplication, i.e. \( rM = m_r[M] \subseteq M \) for every \( r \in R \). We say that \( A \) is an \( R \)-\term{sub-semimodule} of \( M \).

    If \( M \) is a semimodule over some semiring extension \( T \) of \( R \), \( A \) may not be a \( T \)-sub-semimodule. For this reason, we should only use the term \enquote{sub-semimodule} (without specifying the semiring) if the underlying ring is clear from the context.

    As a consequence of \fullref{thm:positive_formulas_preserved_under_homomorphism}, the \hyperref[def:set_valued_map/image]{image} of an \( R \)-semimodule homomorphism \( \varphi: M \to N \) is an \( R \)-sub-semimodule of \( M \).

    \thmitem{def:semimodule/generated} For an arbitrary set \( A \), we denote the \hyperref[def:first_order_generated_substructure]{generated submodel} by \( \linspan{ A } \) and call it the \term[ru=линейная оболочка (\cite[sec. 3.2]{Тыртышников2007})]{linear span} of \( A \).

    \Fullref{rem:span_over_different_semirings} shows how it is important to be unambiguous about over which semiring we take the span of \( A \). In case of possible ambiguity, we will use subscripts like \( \linspan_R A \).

    The linear span can be characterized via \hyperref[rem:linear_combinations]{linear combinations} --- see \fullref{ex:def:first_order_substructure/vector_space}.

    \thmitem{def:semimodule/category} For a fixed semiring \( R \), the \hyperref[def:category_of_small_first_order_models]{category of \( \mscrU \)-small models} \( \ucat{SMod}_R \) of left semimodules is \hyperref[def:concrete_category]{concrete} over \hyperref[def:monoid]{\( \ucat{Mon} \)}.

    Other notations are in use, for example \( R-\cat{Mod} \) for \( R \)-modules by \incite[158]{Aluffi2009}, that better highlight whether we are considering left or right (semi)modules. We will prefer \( \cat{SMod}_R^\oppos \) for the category of right semimodules.

    \thmitem{def:semimodule/bisemimodule}\mcite[149]{Golan2010} An \( (R, T) \)-\term{bisemimodule} is a triple \( (M, R, T) \), where \( M \) is an abelian group that is both a left \( R \)-semimodule and a right \( T \)-semimodule, and the following associativity condition holds for \( m \in M \), \( r \in R \) and \( t \in T \):
    \begin{equation}\label{eq:def:semimodule/bisemimodule/associativity}
      (r \cdot_A m) \cdot_B t = r \cdot_A (m \cdot_B t).
    \end{equation}
  \end{thmenum}
\end{definition}
\begin{defproof}
  \ImplicationSubProof{def:semimodule/action}{def:semimodule/operation} Fix a semiring homomorphism \( \varphi: R \to \End(M) \) and define the operation \( r \cdot x \coloneqq \varphi(r)(x) \).

  We will verify that all conditions from \fullref{def:semimodule/operation} hold for this operation.

  \begin{itemize}
    \item By definition, \( \varphi \) is a monoid action of \( (R, \cdot) \) on \( (M, \bincirc) \).

    \item Distributivity of scalar addition holds because \( \varphi \) is a \hyperref[def:semigroup/homomorphism]{semigroup homomorphism} from \( (R, +) \) to \( (M, +) \).

    \item Distributivity of vector addition holds because, for each \( r \), \( \varphi(r) \) is a semigroup endomorphism of \( (M, +) \).

    \item Since \( \varphi \) is a monoid homomorphism from \( (R,  +) \) to \( (R, \cdot) \), it preserves identities and hence
    \begin{equation*}
      0_R \cdot x = \varphi(0_R)(x) = [y \mapsto 0_M](x) = 0_M.
    \end{equation*}

    This proves half of \eqref{eq:def:semimodule/operation/absorption}.

    \item Since, for each \( r \), \( \varphi(r) \) is a monoid endomorphism of \( (M, +) \), we have
    \begin{equation*}
      r \cdot 0_M = \varphi(r)(0_M) = 0_M.
    \end{equation*}

    This proves the other half of \eqref{eq:def:semimodule/operation/absorption}.
  \end{itemize}

  \ImplicationSubProof{def:semimodule/operation}{def:semimodule/action} Let \( \cdot: R \times M \to M \) be an operation satisfying all conditions from \fullref{def:semimodule/operation}. Define the function \( \varphi(r) \coloneqq (x \mapsto r \cdot x) \). We will show that this is a semiring homomorphism.

  The operation preserves both identities because
  \begin{equation*}
    \varphi(0_R) = (x \mapsto 0) = 0_{\End(M)}
  \end{equation*}
  and
  \begin{equation*}
    \varphi(1_R) = (x \mapsto x) = \id_M.
  \end{equation*}

  We must also show that it preserves both binary operations. Clearly
  \begin{equation*}
    \varphi(r + s)
    =
    (x \mapsto (r + s) x)
    \reloset {\eqref{eq:def:semiring/right_distributivity}} =
    (x \mapsto r x + s x)
    =
    (x \mapsto r x) + (x \mapsto s x)
    =
    \varphi(r) + \varphi(s).
  \end{equation*}

  For multiplication, we have
  \begin{equation*}
    \varphi(rs)
    =
    (x \mapsto (rs)x)
    \reloset {\eqref{eq:def:binary_operation/associative}} =
    (x \mapsto r(sx))
    =
    \parens[\Big]{ x \mapsto \varphi(r)\parens[\Big]{ \varphi(s)(x) } }
    =
    \varphi(r) \bincirc \varphi(s).
  \end{equation*}
\end{defproof}

\begin{proposition}\label{thm:def:semimodule}
  \hyperref[def:semimodule]{Semimodules} have the following basic properties:
  \begin{thmenum}
    \thmitem{thm:def:semimodule/union} The union of a \hyperref[def:order_function/ascending]{ascending sequence} \( N_1 \subseteq N_2 \subseteq \cdots \) of \( R \)-\hyperref[def:semimodule/submodel]{sub-semimodules} of \( M \) is also an \( R \)-sub-semimodule of \( M \).
  \end{thmenum}
\end{proposition}
\begin{proof}
  \SubProofOf{thm:def:semimodule/union} Trivial.
\end{proof}

\begin{proposition}\label{thm:commutative_monoid_is_bisemimodule}
  Every semiring is a \hyperref[def:semimodule/bisemimodule]{bisemimodule} over itself with scalar multiplication given by the semiring multiplication.
\end{proposition}
\begin{comments}
  \item This result specializes to algebras over semirings in \fullref{thm:semiring_is_algebra}.
\end{comments}
\begin{proof}
  Fix a semiring \( R \). We will show that \( \cdot \) satisfied the conditions in \fullref{def:semimodule/operation}.
  \begin{itemize}
    \item The identity law \eqref{eq:def:semimodule/operation/scalar_multiplication_action/identity} holds because \( 1 \) is a multiplicative identity of \( M \).
    \item The associativity-like law \eqref{eq:def:semimodule/operation/scalar_multiplication_action/compatibility} follows from associativity of multiplication.
    \item The two distributivity laws \eqref{eq:def:semimodule/operation/scalar_addition_distributivity} and \eqref{eq:def:semimodule/operation/vector_addition_distributivity} follow from left and right distributivity on \( R \).
    \item The absorption law \eqref{eq:def:semimodule/operation/absorption} follows from absorption on semirings.
  \end{itemize}

  All the above also hold for right semimodules rather than left.
\end{proof}

\begin{proposition}\label{thm:commutative_monoid_is_semimodule}
  The categories \( \hyperref[def:monoid/category]{\cat{CMon}} \) of commutative monoids and \( \hyperref[def:semimodule/category]{\cat{SMod}_\BbbN} \) of natural number semimodules are \hyperref[rem:category_similarity/isomorphism]{isomorphic}.

  More concretely, every commutative monoid \( M \) is a left semimodule over \( \BbbN \) with scalar multiplication given by \hyperref[rem:additive_semigroup/multiplication]{recursively defined multiplication}
  \begin{equation}\label{eq:thm:commutative_monoid_is_semimodule/operation}
    \begin{aligned}
      &\cdot: \BbbN \times M \to M \\
      &n \cdot x \coloneqq \begin{cases}
        0_M,           &n = 0, \\
        n \cdot x + x, &n > 1.
      \end{cases}
    \end{aligned}
  \end{equation}

  Conversely, in every semimodule over \( \BbbN \), scalar multiplication matches the recursively defined multiplication.
\end{proposition}
\begin{comments}
  \item This result specializes to algebras over semirings in \fullref{thm:semiring_is_natural_number_algebra} and modules over rings in \fullref{thm:abelian_group_is_module}.
\end{comments}
\begin{proof}
  \SufficiencySubProof Let \( M \) be a commutative monoid. The operation \( \cdot: \BbbN \times M \to M \) defined in \fullref{thm:semiring_characteristic_homomorphism} satisfies the conditions in \fullref{def:semimodule/operation} as either a direct consequence of the definition or as a consequence of \fullref{thm:monoid_distributivity}.

  The homomorphisms are thus also compatible.

  \NecessitySubProof Let \( M \) be a semimodule over \( \BbbN \). We will use induction to show that \eqref{eq:thm:commutative_monoid_is_semimodule/operation} holds.
  \begin{itemize}
    \item For \( n = 0 \), this follows from the absorption law \eqref{eq:def:semimodule/operation/absorption}.
    \item If \( n \cdot x = n \cdot x + x \), then by scalar distributivity, \( (n + 1) \cdot x = n \cdot x + 1 \cdot x \). The multiplicative identity law \eqref{eq:def:semimodule/operation/scalar_multiplication_action/identity} then shows that \( 1 \cdot x = x \), which concludes our proof.
  \end{itemize}

  The homomorphisms are thus also compatible.
\end{proof}

\paragraph{Semimodule direct sums}

\begin{definition}\label{def:function_support}\mcite[def. 10.3]{Rudin1976Principles}
  We define the \term[bg=носител (\cite[58]{Боянов2008}), ru=носитель (\cite[135]{КанторовичАкилов1984})]{support} of a function \( f: S \to R \) from any set \( S \) to a semiring \( R \) as the set
  \begin{equation*}
    \supp(f) \coloneqq \set{ x \in S \given f(x) \neq 0_R }.
  \end{equation*}
\end{definition}

\begin{definition}\label{def:semimodule_direct_sum}\mimprovised
  We define the \term{external direct sum} or simply \term{direct sum} \( \bigoplus_{k \in \mscrK} M_k \) of the family of \( R \)-\hyperref[def:semimodule]{semimodules} \( \seq{ M_k }_{k \in \mscrK} \) of \( \Gamma \) as the subset of their \hyperref[def:first_order_direct_product]{direct product} \( \prod_{k \in \mscrK} M_k \) consisting of tuples with finite \hyperref[def:function_support]{support}.

  The sums and scalar products of tuples with finite support also have finite support, hence the direct sum is an \( R \)-\hyperref[def:semimodule/submodel]{sub-semimodule} of the direct product.

  \begin{thmenum}
    \thmitem{def:semimodule_direct_sum/inclusion} For every \( m \in \mscrK \), we define the following \term{canonical inclusion} \hyperref[def:semimodule/homomorphism]{homomorphism}:
    \begin{equation*}
      \begin{aligned}
        &\iota_m: M_m \to \bigoplus_{k \in \mscrK} M_k, \\
        &\iota_m(x) \coloneqq \begin{rcases}
          \begin{cases}
            x, &k = m \\
            0, &k \neq m
          \end{cases}
        \end{rcases}_{k \in \mscrK}
      \end{aligned}
    \end{equation*}

    \thmitem{def:semimodule_direct_sum/power} In case all summands are equal to \( M \), we denote the direct sum by \( M^{\oplus \mscrK} \).
    \thmitem{def:semimodule_direct_sum/internal} If all summands are submodels of \( M \) and if the sum \( \bigoplus_{k \in \mscrK} M_k \) is isomorphic to \( M \), we call it an \term{internal direct sum} and treat the tuple \( \seq{ x_k }_{k \in \mscrK} \) as the product \( x_{k_1} x_{k_2} \cdots x_{k_n} \) of the elements of \( M \) distinct from zero.
  \end{thmenum}
\end{definition}

\begin{proposition}\label{thm:semimodule_coproduct}
  The \hyperref[def:discrete_category_limits]{categorical coproduct} of the family \( \seq{ M_k }_{k \in \mscrK} \) in the \hyperref[def:semimodule/category]{category of \hi{commutative} semimodules} is their \hyperref[def:semimodule_direct_sum]{direct sum} \( \bigoplus_{k \in \mscrK} M_k \).
\end{proposition}
\begin{proof}
  First note that the sum \( A \coloneqq \bigoplus_{k \in \mscrK} A_k \) with the \hyperref[def:semimodule_direct_sum/inclusion]{inclusions} \( \iota \coloneqq \seq{ \iota_k }_{k \in \mscrK} \) are a \hyperref[def:category_of_cones/cocone]{cocone} for the \hyperref[def:discrete_category]{discrete} \hyperref[def:categorical_diagram]{diagram} \( \seq{ A_k }_{k \in \mscrK} \).

  Let \( (C, \alpha) \) also be a cocone. We want to define a semimodule homomorphism
  \begin{equation*}
    l_C: \bigoplus_{k \in \mscrK} A_k \to A
  \end{equation*}
  such that, for every \( m \in \mscrK \) and \( x \in M_m \),
  \begin{equation*}
    \alpha_m(x) = l_A(\iota_m(x)).
  \end{equation*}

  Thus, the value of \( l_A(c) \) on members of the inclusion \( \iota_m[M_m] \) is entirely determined by \( \alpha_m \). This suggests the definition
  \begin{equation*}
    l_C(\seq{ x_k }_{k \in \mscrK}) \coloneqq \prod_{k \in \mscrK}^n \alpha_k(x_k).
  \end{equation*}

  We discuss well-definedness of infinitary operations in direct sums in \fullref{rem:binary_operation_syntax_trees/infinite/direct_sum}.
\end{proof}

\paragraph{Free semimodules}

\begin{definition}\label{def:free_semimodule}\mimprovised
  Fix a \hyperref[def:semiring]{semiring} \( R \). We associate with every \hyperref[def:set]{plain set} \( A \) its \term{free \( R \)-semimodule} \( R^{\oplus A} \) over \( R \) defined as the set
  \begin{equation*}
    R^{\oplus A} \coloneqq \bigoplus_{x \in A} R = \set{ t: A \to R \given t \T{has finite \hyperref[def:function_support]{support}} }.
  \end{equation*}

  We regard the function \( t \) as an indexed family \( \seq{ t_x }_{x \in A} \), and we call the indexed family a \term{linear combination} over \( R \). We call the linear combination over the zero tuple \term{trivial}.

  \begin{thmenum}
    \thmitem{def:free_semimodule/operations} \Fullref{thm:functions_over_model_form_model} implies that \( R^{\oplus A} \) inherits addition and multiplication from \( R \) and is actually a semiring. Scalar multiplication can be defined as
    \begin{equation*}
      \begin{aligned}
        &\cdot: R \times R^{\oplus A} \to R^{\oplus A}, \\
        &r \cdot \seq{ t_x }_{x \in A} \coloneqq \seq{ r \cdot t_x }_{x \in A}.
      \end{aligned}
    \end{equation*}

    Free right semimodules require trivial adjustments here.

    \thmitem{def:free_semimodule/inclusion} We also define the \term{canonical inclusion} function
    \begin{equation*}
      \begin{aligned}
        &\iota_A: A \to R^{\oplus A}, \\
        &\iota_A(x) \coloneqq \parens[\Bigg]
          {
            y \mapsto \begin{rcases}
              \begin{cases}
                1_R, &y = x \\
                0_R, &y \neq x
              \end{cases}
            \end{rcases}
          }
      \end{aligned}
    \end{equation*}
  \end{thmenum}
\end{definition}
\begin{comments}
  \item In the case when \( R \) is the semiring \( \BbbN \) of natural numbers, \( \BbbN^{\oplus A} \) is the set of finite \hyperref[def:labeled_set/multiset]{multisets} over \( S \).
\end{comments}

\begin{theorem}[Free semimodule universal property]\label{thm:free_semimodule_universal_property}
  Fix a semiring \( R \) and a set \( A \). The \hyperref[def:free_semimodule]{free \( R \)-semimodule} \( R^{\oplus A} \) over \( R \) is the unique up to a unique isomorphism semimodule that satisfies the following \hyperref[rem:universal_mapping_property]{universal mapping property}:
  \begin{displayquote}
    For every semimodule \( M \) over \( R \) and every function \( e: A \to M \), there exists a unique \( R \)-semimodule homomorphism \( \Phi_e: R^{\oplus A} \to M \) such that the following diagram commutes:
    \begin{equation}\label{eq:thm:free_semimodule_universal_property/diagram}
      \begin{aligned}
        \includegraphics[page=1]{output/thm__free_semimodule_universal_property}
      \end{aligned}
    \end{equation}
  \end{displayquote}
\end{theorem}
\begin{comments}
  \item Via \fullref{rem:universal_mapping_property}, \( A \mapsto R^{\oplus A} \) becomes \hyperref[def:category_adjunction]{left adjoint} to the \hyperref[def:concrete_category]{forgetful functor}
  \begin{equation*}
    U: \cat{SMod}_R \to \cat{Set}.
  \end{equation*}
\end{comments}
\begin{proof}
  For every function \( e: A \to M \), we want
  \begin{equation*}
    \Phi_e(\iota(x)) = e(x).
  \end{equation*}

  This suggests the definition
  \begin{equation*}
    \Phi_e(\seq{ t_x }_{x \in A}) \coloneqq \sum_{x \in A}^n t_x \cdot e(x).
  \end{equation*}

  We discuss well-definedness of infinitary operations in direct sums in \fullref{rem:binary_operation_syntax_trees/infinite/direct_sum}.
\end{proof}

\begin{remark}\label{rem:linear_combinations}
  The \hyperref[def:free_semimodule]{linear combination} \( \sum_{x \in A} t_x x \) can instead be written as \( \sum_{k=1}^n t_k x_k \), where \( x_1, \ldots, x_n \) are the values in \( A \) for which the scalars \( t_1, \ldots, t_n \) are nonzero (we denote \( t_{x_k} \) by \( t_k \) for brevity). This is actually the dominating convention, although we sometimes use the former notation or the \hyperref[thm:direct_product_projections]{direct product projections}.

  This issue is discussed in more generality in \fullref{rem:binary_operation_syntax_trees/infinite/direct_sum}.
\end{remark}

\begin{proposition}\label{thm:span_via_linear_combinations}
  For a set \( A \) in an \( R \)-\hyperref[def:semimodule]{semimodule} \( M \), the \hyperref[def:semimodule/generated]{linear span} of \( A \) equals the set of all \hyperref[rem:linear_combinations]{linear combinations} over \( A \).
\end{proposition}
\begin{comments}
  \item Compare this result to \fullref{thm:generators_via_polynomials} for algebras and polynomials.
\end{comments}
\begin{proof}
  \Cref{fig:thm:span_via_linear_combinations} shows an \hyperref[rem:abstract_syntax_tree]{abstract syntax tree} for a given linear combination, which can be traversed and evaluated to obtain a vector in \( M \). This vector must be a member of the span of \( S \) since the latter is closed under vector addition and scalar multiplication with members of \( S \). Hence, the set \( L \) of all linear combinations over \( S \) is a subset of the span.

  Generalizing the syntax tree construction from \cref{fig:thm:span_via_linear_combinations}, we see that \( L \) satisfies \fullref{def:first_order_substructure/universe/inductive}, and is thus a submodule of \( M \). Since the span is the smallest module containing \( S \), we have \( L = \linspan S \).

  \begin{figure}[!ht]
    \hfill
    \includegraphics[page=1]{output/thm__span_via_linear_combinations}
    \hfill\hfill
    \caption{A linear combination is simply a \hyperref[rem:function_superposition]{superposition} of scalar multiplication and binary addition.}
    \label{fig:thm:span_via_linear_combinations}
  \end{figure}
\end{proof}

\begin{remark}\label{rem:span_over_different_semirings}
  If \( M \) is both an \( R \)-semimodule and a \( T \)-semimodule, \fullref{thm:span_via_linear_combinations} highlights a fundamental difference between the generated \( R \)-sub-semimodule and the generated \( T \)-sub-semimodule.

  For example, the \( \BbbN \)-sub-semimodule generated by \( 2 \) is the semiring \( 2\BbbN \) of even natural numbers, while the \( \BbbR_{\geq 0} \)-sub-semimodule generated by \( 2 \) is \( \BbbR_{\geq 0} \) itself.
\end{remark}

\paragraph{Free commutative monoids}

\begin{remark}\label{rem:free_commutative_monoid_as_quotient}
  Consider the \hyperref[def:free_monoid]{free monoid} \( A^* \) over some \hyperref[def:set]{plain set} \( A \) and also the \hyperref[def:free_semimodule]{free \( \BbbN \)-semimodule} \( \BbbN^{\oplus A} \).

  Define the homomorphism
  \begin{equation*}
    \begin{aligned}
      &\varphi: A^* \to \BbbN^{\oplus A}, \\
      &\varphi(x_1 \cdots x_n) \coloneqq \seq[\Big]{ \underbrace{\sum_{k=1}^n 1_{a = x_k}}_{\mathclap{\T*{repetitions of} a \T*{among} x_1 \cdots x_n} }}_{a \in A}
    \end{aligned}
  \end{equation*}

  By \fullref{thm:homomorphism_induces_congruence}, \( \varphi \) induces a \hyperref[def:first_order_congruence]{monoid congruence} \( \cong \) on \( A^* \) where two strings are congruence if they have the same amount of each symbol from \( A \).

  This ensures that the strings \enquote{\( abc \)}, \enquote{\( bac \)}, \enquote{\( bca \)}, \enquote{\( cba \)}, \enquote{\( cab \)} and \enquote{\( acb \)} are congruent --- these are precisely all variations of \enquote{\( abc \)} that can be obtained via \hyperref[def:transposition]{transpositions} due to commutativity.

  The \hyperref[def:first_order_quotient]{quotient} \( A^* / \cong \) is isomorphic to \( \BbbN^{\oplus A} \) as a monoid. This motivates the definition of free commutative monoids in \fullref{def:free_commutative_monoid}.
\end{remark}

\begin{definition}\label{def:free_commutative_monoid}\mimprovised
  We associate with every \hyperref[def:set]{plain set} \( A \) its \term{free commutative monoid} defined as the \hyperref[def:free_semimodule]{free \( \BbbN \)-module} \( \BbbN^{\oplus A} \).
\end{definition}
\begin{comments}
  \item We regard \( \BbbN^{\oplus A} \) only as a monoid, without generally considering scalars.
  \item The relation to \hyperref[def:free_monoid]{free monoids} is given in \fullref{rem:free_commutative_monoid_as_quotient}.
\end{comments}

\begin{theorem}[Free commutative monoid universal property]\label{thm:free_commutative_monoid_universal_property}
  Given a set \( A \), the \hyperref[def:free_commutative_monoid]{free commutative monoid} \( \BbbN^{\oplus A} \) is the unique up to a unique isomorphism commutative monoid that satisfies the following \hyperref[rem:universal_mapping_property]{universal mapping property}:
  \begin{displayquote}
    For every commutative monoid \( M \) and every function \( e: A \to M \), there exists a unique monoid homomorphism \( \Phi_e: \BbbN^{\oplus A} \to M \) such that the following diagram commutes:
    \begin{equation}\label{eq:thm:free_commutative_monoid_universal_property/diagram}
      \begin{aligned}
        \includegraphics[page=1]{output/thm__free_commutative_monoid_universal_property}
      \end{aligned}
    \end{equation}
  \end{displayquote}
\end{theorem}
\begin{comments}
  \item Via \fullref{rem:universal_mapping_property}, \( A \mapsto \BbbN^{\oplus A} \) becomes \hyperref[def:category_adjunction]{left adjoint} to the \hyperref[def:concrete_category]{forgetful functor}
  \begin{equation*}
    U: \cat{CMon} \to \cat{Set}.
  \end{equation*}
\end{comments}
\begin{proof}
  Follows from \fullref{thm:free_semimodule_universal_property} by noting that, as shown in \fullref{thm:commutative_monoid_is_semimodule}, commutative monoids are semimodules over \( \BbbN \).
\end{proof}
