\section{Univariate polynomials}\label{sec:univariate_polynomials}

We will discuss here the \hyperref[def:polynomial_algebra]{polynomial algebra} \( R[X] \) in one indeterminate over the \hyperref[def:ring/trivial]{nontrivial} \hyperref[def:ring/commutative]{commutative unital ring} \( R \). We call them \term{univariate polynomials} based on \fullref{def:operation_arity}, although we acknowledge the polynomials are not functions. \Fullref{rem:polynomials_over_infinitely_many_indeterminates} discusses why we often focus only on finitely many indeterminates, and why the theory of univariate polynomials is often sufficient.

Polynomials are not functions in general, and the exact relationship between polynomials and polynomial functions is discussed in \fullref{thm:polynomial_algebra_universal_property} and \fullref{thm:functions_over_prime_fields}.

\paragraph{Univariate polynomials}

\begin{definition}\label{def:monic_polynomial}
  We say that the nonzero univariate polynomial \( p(X) \) is \term[bg=нормиран (\cite[409]{Обрешков1962ВисшаАлгебра}), ru=нормированный/приведенный (\cite[102]{Винберг2014Алгебра})]{monic} if its leading coefficient is \( 1 \).
\end{definition}

\begin{proposition}\label{thm:polynomial_degree_arithmetic}
  The \hyperref[def:polynomial_degree]{polynomial degree} has the following basic properties:
  \begin{thmenum}
    \thmitem{thm:polynomial_degree_arithmetic/sum} For any two nonzero polynomials satisfying \( p(X) \neq -q(X) \), we have
    \begin{equation}\label{eq:thm:polynomial_degree_arithmetic/sum}
      \deg (p + q) \leq \max \set{ \deg p, \deg q }.
    \end{equation}

    \thmitem{thm:polynomial_degree_arithmetic/product} For any two nonzero polynomials \( p(X) \) and \( q(X) \) whose leading coefficients do not multiply to zero, we have
    \begin{equation}\label{eq:thm:polynomial_degree_arithmetic/product}
      \deg (pq) = \deg p + \deg q.
    \end{equation}
  \end{thmenum}
\end{proposition}
\begin{comments}
  \item An easy sufficient condition for \eqref{eq:thm:polynomial_degree_arithmetic/product} is for the ring to be \hyperref[def:entire_semiring]{entire}, although it is also sufficient for the ring to be nontrivial (so that \( 0_R \neq 1_R \)) and either \( p(X) \) or \( q(X) \) to be \hyperref[def:monic_polynomial]{monic}.

  \item We generalize a weaker statement for multivariate polynomials in \fullref{thm:degree_of_multivariate_polynomial_product}.
\end{comments}
\begin{proof}
  Fix nonzero polynomials
  \begin{align*}
    p(X) = \sum_{k=0}^n a_k X^k, &&
    q(X) = \sum_{k=0}^m b_k X^k.
  \end{align*}

  \SubProofOf{thm:polynomial_degree_arithmetic/sum} Additionally assume that \( p(X) \neq -q(X) \) since otherwise \( p(X) + q(X) = 0 \) and \( \deg(p + q) \) is undefined.

  Since the polynomials are not equal, there exists at least one index \( k = 1, 2, \ldots \) such that \( a_k \neq b_k \). Denote by \( k_0 \) the largest such index (only finitely many are nonzero). Then
  \begin{equation*}
    a_k - b_k = 0 \T{for} k > k_0.
  \end{equation*}

  Therefore, \( \deg(p + q) = k_0 \). Note that \( k_0 \) cannot exceed both \( \deg p \) and \( \deg q \) because it corresponds to a nonzero coefficient in both. Thus, \( k_0 \leq \max\set{ \deg p, \deg q } \).

  \SubProofOf{thm:polynomial_degree_arithmetic/product} The coefficient \( c_{n + m} \) of the product \( p(X) q(X) \) is \( a_n b_m \) by definition. By assumption, it is nonzero. Then, since \( c_{n+m+1} = 0 \), we have
  \begin{equation*}
    \deg (pq) = \deg p + \deg q.
  \end{equation*}
\end{proof}

\begin{corollary}\label{thm:leading_coefficient_of_product}
  If the leading coefficients of two univariate polynomials do not multiply to zero, the product of their leading coefficients is the leading coefficient of their product.
\end{corollary}
\begin{proof}
  Consider the polynomials
  \begin{align*}
    p(X) = \sum_{k=0}^n a_k X^k, &&
    q(X) = \sum_{k=0}^m b_k X^k.
  \end{align*}

  By definition of convolution product, the \( (n + m) \)-th coefficient of \( p(X) q(X) \) is \( a_n b_m \). \Fullref{thm:polynomial_degree_arithmetic/product} implies that this is their leading coefficient.
\end{proof}

\begin{algorithm}[Euclidean division of polynomials]\label{alg:euclidean_division_of_polynomials}\mcite[prop. 1.12]{Knapp2016BasicAlgebra}
  Fix two univariate polynomials \( f(X) \) and \( g(X) \), and assume that \( g(X) \) is \hyperref[def:monic_polynomial]{monic}.

  We will build polynomials \( q(X) \) and \( r(X) \), where \( r(X) \) is either zero or \( \deg r < \deg g \), such that
  \begin{equation*}
    f(X) = g(X) \cdot q(X) + r(X).
  \end{equation*}

  The algorithm only demonstrates existence; we will prove uniqueness right after it.

  \begin{thmenum}
    \thmitem{alg:euclidean_division_of_polynomials/zero_degree} If \( \deg f = \deg g = 0 \), necessarily \( g(X) = 1_R \), and in this case we define
    \begin{align*}
      q(X) &\coloneqq f(X), \\
      r(X) &\coloneqq 0_R.
    \end{align*}

    \thmitem{alg:euclidean_division_of_polynomials/no_division} If \( f(X) \) is the zero polynomial or \( \deg f < \deg g \), define
    \begin{align*}
      q(X) &\coloneqq 0_R, \\
      r(X) &\coloneqq f(X).
    \end{align*}

    In this case, \( r(X) \) is either zero or \( \deg r = \deg f < \deg b \).

    \thmitem{alg:euclidean_division_of_polynomials/positive_degree} Suppose that
    \begin{align*}
      f(X) = a_n X^n + \hat f(X), \\
      g(X) = X^m + \hat g(X),
    \end{align*}
    where \( n \) and \( m \) are positive, \( \hat f(X) \) is either zero or \( \deg \hat f < \deg f \), and similarly for \( \deg \hat g \).

    Then
    \begin{align*}
    f(X) - g(X) a_n X^{n-m}
    &=
    a_n X^n + \hat f(X) - (b_m X^m + \hat g(X)) a_n X^{n-m}
    = \\ &=
    a_n X^n + \hat f(X) - a_n X^n - \hat g(X) a_n X^{n-m}
    = \\ &=
    \underbrace{\hat f(X) - \hat g(X) a_n X^{n-m}}_{\hat r(X)}.
    \end{align*}

    The polynomial \( \hat r(X) \) is either zero, in which case we define \( r(X) \coloneqq \hat r(X) \), or \( \deg \hat r \leq n - 1 \).

    We use the algorithm recursively to divide \( \hat r(X) \) by \( g(X) \), and obtain \( \hat q(X) \) and \( r(X) \) such that
    \begin{equation*}
      \hat r(X) \coloneqq g(X) \hat q(X) + r(X),
    \end{equation*}
    where \( r(X) \) is either zero or \( \deg r < \deg g \).

    Then
    \begin{align*}
      \hat r(X)                                         &= f(X) - g(X) a_n X^{n-m} \\
      g(X) \hat q(X) + r(X)                             &= f(X) - g(X) a_n X^{n-m} \\
      g(X) \left(\hat q(X) + a_n X^{n-m} \right) + r(X) &= f(X).
    \end{align*}

    Define
    \begin{equation*}
      q(X) \coloneqq \hat q(X) + a_n X^{n-m}.
    \end{equation*}

    We have obtained polynomials \( r(X) \) and \( q(X) \) where \( r(X) \) is either zero or \( \deg r < \deg g \).
  \end{thmenum}
\end{algorithm}
\begin{comments}
  \item This algorithm can be found as \identifier{polynomials.univariate.euclidean_divmod} in \cite{notebook:code}.
\end{comments}
\begin{defproof}
  \UniquenessSubProof Suppose that
  \begin{equation*}
    a(X) = g(X)q(X) + r(X) = g(X) \widetilde{q}(X) + \widetilde{r}(X),
  \end{equation*}
  where \( r(X) \) and \( \widetilde{r}(X) \) are either zero or have degree less than \( g(X) \).

  Assume that \( r(X) \neq \widetilde{r}(X) \).

  \begin{itemize}
    \item If both \( r(X) \) and \( \widetilde{r}(X) \) are nonzero, we have
    \begin{equation*}
      g(X) \parens[\Big]{ q(X) - \widetilde{q}(X) } = -\parens[\Big]{ r(X) - \widetilde{r}(X) }.
    \end{equation*}

    Since \( g(X) \) is monic and its leading coefficient \( 1_R \) is not a zero divisor, \fullref{thm:polynomial_degree_arithmetic/product} holds, and thus
    \begin{equation*}
      \deg g + \deg(q - \widetilde{q})
      \reloset {\eqref{eq:thm:polynomial_degree_arithmetic/product}} =
      \deg(g (q - \widetilde{q}))
      =
      \deg(r - \widetilde{r})
      \reloset {\eqref{eq:thm:polynomial_degree_arithmetic/sum}} =
      \leq \max\set{ \deg r, \deg \widetilde{r} }
      <
      \deg g,
    \end{equation*}
    which is a contradiction.

    \item If \( r(X) \) is zero but \( \widetilde{r}(X) \) is not, then
    \begin{equation*}
      g(X) q(X) = g(X) \widetilde{q}(X) + \widetilde{r}(X),
    \end{equation*}
    implying that
    \begin{equation*}
      \widetilde{r}(X) = g(X) \parens[\Big]{ q(X) - \widetilde{q}(X) }.
    \end{equation*}

    By \eqref{thm:polynomial_degree_arithmetic/product}, \( \deg g \leq \widetilde{r} \), which contradicts our choice of \( \widetilde{r}(X) \).
  \end{itemize}
\end{defproof}

\begin{algorithm}[Horner's rule]\label{alg:horners_rule}
  Consider the polynomials
  \begin{equation*}
    f(X) = \sum_{k=0}^n a_k X^k
  \end{equation*}
  and
  \begin{equation*}
    g(X) = X + b.
  \end{equation*}

  The coefficients of the quotient of \( f(X) \) and \( g(X) \) with respect to \fullref{alg:euclidean_division_of_polynomials} can be computed recursively as follows:
  \begin{equation}\label{eq:alg:horners_rule/quot}
    c_{n-k} \coloneqq \begin{cases}
      a_n,                              &k = 1 \\
      a_{n-(k-1)} - b \cdot c_{n-(k-1)} &k > 1.
    \end{cases}
  \end{equation}

  Furthermore, \( r(X) \) is a constant that can be obtained from the above as
  \begin{equation}\label{eq:alg:horners_rule/rem}
    c_{-1} = a_0 - b \cdot c_0.
  \end{equation}
\end{algorithm}
\begin{comments}
  \item This algorithm can be found as \identifier{polynomials.division.horner_divmod} in \cite{notebook:code}.
\end{comments}
\begin{proof}
  Denote the quotient by
  \begin{equation*}
    q(X) = \sum_{k=0}^{n-1} c_k X^k.
  \end{equation*}

  We have
  \begin{equation*}
    f(X) = g(X) \cdot q(X) + r(X),
  \end{equation*}
  from where, for every \( k > 0 \),
  \begin{equation*}
    a_k = c_{k-1} + b \cdot c_k,
  \end{equation*}
  from which \eqref{eq:alg:horners_rule/quot} follows.

  Furthermore,
  \begin{equation*}
    r(X) = a_0 - b \cdot c_0,
  \end{equation*}
  which demonstrates \eqref{eq:alg:horners_rule/rem}.
\end{proof}

\paragraph{Polynomial roots}

\begin{definition}\label{def:algebraic_derivative}\mcite[461]{Knapp2016BasicAlgebra}
  Generalizing \fullref{def:differentiability} from analysis, we define the \term[ru=(алгебрическая) производная (\cite[sec. D17.1]{Тыртышников2007ЛинейнаяАлгебра})]{algebraic derivative} of a univariate polynomial
  \begin{equation*}
    p(X) = \sum_{k=0}^n a_k X^k = a_n X^n + a_{n-1} X^{n-1} + \cdots + a_2 X^2 + a_1 X + a_0
  \end{equation*}
  as
  \begin{equation*}
    p'(X) \coloneqq \sum_{k=1}^n k a_k X^{k-1} = n a_n X^{n-1} + (n-1) a_{n-1} X^{n-2} + \cdots + a_2 X + a_1.
  \end{equation*}

  Via \hyperref[rem:natural_number_recursion]{natural number recursion}, we can define algebraic derivatives of order \( m \) as
  \begin{equation*}
    p^{(m)}(X) \coloneqq \begin{cases}
      p(X)              &m = 0 \\
      \parens[\Big]{ p^{(m - 1)} }'(X) &m > 0
    \end{cases}
  \end{equation*}
\end{definition}

\begin{proposition}\label{thm:def:algebraic_derivative}
  \hyperref[def:algebraic_derivative]{Algebraic derivatives} have the following basic properties:
  \begin{thmenum}
    \thmitem{thm:def:algebraic_derivative/linear} The derivative operator \( p(X) \mapsto p'(X) \) is linear.
    \thmitem{thm:def:algebraic_derivative/degree} \( p^{(n)}(X) \) is either zero or has degree \( \deg p - n \).

    \thmitem{thm:def:algebraic_derivative/product} The product rule holds:
    \begin{equation}\label{eq:thm:def:algebraic_derivative/product}
      (pq)' = p'q + pq'.
    \end{equation}

    \thmitem{thm:def:algebraic_derivative/leibniz} \hyperref[thm:leibniz_rule]{Leibniz' rule} holds:
    \begin{equation}\label{eq:thm:def:algebraic_derivative/leibniz}
      (pq)^{(n)} = \sum_{k=0}^n \binom n k p^{(k)} q^{(n-k)}
    \end{equation}

    \thmitem{thm:def:algebraic_derivative/affine_power} If \( m \leq n \), the \( m \)-th derivative of \( (X - \alpha)^n \) is
    \begin{equation*}
      \frac {n!} {(n-m)!} (X - \alpha)^{n-m}.
    \end{equation*}
  \end{thmenum}
\end{proposition}
\begin{proof}
  \SubProofOf{thm:def:algebraic_derivative/linear} Trivial.

  \SubProofOf{thm:def:algebraic_derivative/degree} Trivial.

  \SubProofOf{thm:def:algebraic_derivative/product} By \fullref{thm:def:algebraic_derivative/linear}, it is enough to consider the case where both \( p(X) \) and \( q(X) \) are monomials.

  \begin{align*}
    p'(X) q(X) + p(X) q'(X)
    &=
    n a_n X^{n-1} \cdot b_m X^m + a_n X^n \cdot m b_m X^{m-1}
    = \\ &=
    (n + m) a_n b_m X^{n+m-1}
    = \\ &=
    (a_n b_m X^{n+m})'
    = \\ &=
    (pq)'(X).
  \end{align*}

  \SubProofOf{thm:def:algebraic_derivative/leibniz} The proof in \fullref{thm:leibniz_rule} relies only on the product rule, hence it holds here as well.

  \SubProofOf{thm:def:algebraic_derivative/affine_power} We use outer induction on \( m \) and inner induction on \( n \).

  The case \( m = n = 1 \) is obvious. Assume that the statement holds for \( m = 1 \) and \( n - 1 \). Then
  \begin{equation*}
    \parens[\Big]{ (X - \alpha)^n }'
    =
    \parens[\Big]{ (X - \alpha)^{n-1} \cdot (X - \alpha) }'
    \reloset {\eqref{eq:thm:def:algebraic_derivative/product}} =
    \parens[\Big]{ (X - \alpha)^{n-1} }' (X - \alpha) + (X - \alpha)^{n-1}
    \reloset {\T{ind.}} =
    n (X - \alpha)^{n-1}.
  \end{equation*}

  Now suppose that the statement holds for derivatives of order less than \( m \) and for every \( n \geq m \). Then,
  \begin{equation*}
    \parens[\Big]{ (X - \alpha)^n }^{(m)}
    =
    \parens*{\parens[\Big]{ (X - \alpha)^n }^{(m-1)}}'
    \reloset {\T{ind.}} =
    \parens*{ \frac {n!} {(n - m + 1)!} (X - \alpha)^{n - m + 1} }'
    \reloset {\T{ind.}} =
    \frac {n!} {(n - m)!} (X - \alpha)^{n - m}.
  \end{equation*}
\end{proof}

\begin{definition}\label{def:multiple_root}\mimprovised
  Let \( R \) be a nontrivial commutative ring.

  We say that the value \( \alpha \in R \) is a \term[bg=корен (\cite[33]{ГеновМиховскиМоллов1991Алгебра}), ru=корень (\cite[99]{Винберг2014Алгебра})]{root} of multiplicity \( m \) for the univariate polynomial \( p(X) \in R[X] \) of degree \( n \geq m \) if any of the following equivalent conditions hold:
  \begin{thmenum}
    \thmitem{def:multiple_root/division} The polynomial \( (X - \alpha)^m \) divides \( p(X) \).

    \thmitem{def:multiple_root/derivative_roots} The value \( \alpha \) is a \hyperref[def:zero_locus]{zero} of the \hyperref[def:algebraic_derivative]{algebraic derivatives} \( p^{(0)}(X), p^{(1)}(X), \ldots, p^{(m-1)}(X) \) of \( p(X) \).
  \end{thmenum}

  Every polynomial \( p(X) \) has a \hyperref[def:multiset]{multiset} of roots. If at least one of them has multiplicity greater than one, we say that it is a \term{repeated root}.
\end{definition}
\begin{comments}
  \item This is an extension of the general concept of polynomial roots discussed in \fullref{def:polynomial_root}.

  \item The equivalence between \( (X - \alpha) \) dividing \( p(X) \) and \( p(\alpha) \) being zero is called the \enquote{factor theorem} by \incite[corr. 1.13]{Knapp2016BasicAlgebra}.
\end{comments}
\begin{defproof}
  \ImplicationSubProof{def:multiple_root/division}{def:multiple_root/derivative_roots} Suppose that \( (X - \alpha)^m \) divides \( p(X) \). Then there exists a polynomial \( q(X) \) such that
  \begin{equation*}
    p(X) = (X - \alpha)^m q(X).
  \end{equation*}

  For the \( n < m \)-th derivative of \( p(X) \), by \fullref{thm:def:algebraic_derivative/leibniz}, we have
  \begin{equation*}
    p^{(n)}(X) = \sum_{k=0}^n \binom n k \underbrace{\parens[\Big]{ (X - \alpha)^m }^{(k)}}_{\mathclap{\frac {m!} {(m-k)!} (X - \alpha)^{m-k} \T*{by} \ref{thm:def:algebraic_derivative/affine_power}}} q^{(n-k)}(X).
  \end{equation*}

  Let \( \Phi_\alpha: R[X] \to R \) be the \hyperref[con:evaluation_homomorphism]{evaluation homomorphism} at \( \alpha \). Then
  \begin{equation*}
    \Phi_\alpha(p^{(n)}) = \sum_{k=0}^n \binom n k \frac {m!} {(m-k)!} 0_R^{m-k} \Phi_\alpha(q^{(n-k)})
  \end{equation*}

  For \( n < m \), clearly \( \Phi_\alpha(p^{(n)}) = 0_R \).

  \ImplicationSubProof{def:multiple_root/derivative_roots}{def:multiple_root/division} Suppose that \( \alpha \) is a root of \( p^{(0)}(X), \ldots, p^{(m-1)}(X) \). We will use induction on \( m \) to show that \( (X - \alpha)^m \mid p(X) \).

  The case \( m = 0 \) is trivial. Suppose that \( \alpha \) is root of \( p^{(0)}(X), \ldots, p^{(m-1)}(X) \) and that \( (X - \alpha)^{m-1} \) divides \( p(X) \). Additionally suppose that \( \alpha \) is a root of \( p^{(m)}(X) \).

  By the inductive hypothesis, there exists a polynomial \( q(X) \) such that
  \begin{equation*}
    p(X) = (X - \alpha)^{m-1} q(X).
  \end{equation*}

  By \fullref{thm:def:algebraic_derivative/leibniz},
  \begin{equation*}
    p^{(m-1)}(X) = \sum_{k=0}^{m-1} \binom {m-1} k \underbrace{\parens[\Big]{ (X - \alpha)^{m - 1} }^{(k)}}_{\mathclap{\frac {(m - 1)!} {(m - 1 - k)!} (X - \alpha)^{m - 1 - k} \T*{by} \ref{thm:def:algebraic_derivative/affine_power}}} q^{(m - 1 - k)}(X).
  \end{equation*}

  Then
  \begin{equation*}
    \Phi_\alpha(p^{(m-1)}) = \sum_{k=0}^{m-1} \binom {m-1} k \frac {(m - 1)!} {(m - 1 - k)!} 0_R^{m - 1 - k} \Phi_\alpha(q^{(m - 1 - k)}).
  \end{equation*}

  All terms on the right are zero except for the case where \( k = m - 1 \), where \( q^{(0)} = q \) and the entire expression reduces to \( \Phi_\alpha(q) \). The scalar \( \alpha \) is a root of \( p^{(m-1)} \), implying that it is also a root of \( q \).

  We use \fullref{alg:euclidean_division_of_polynomials} to obtain a polynomial \( s(X) \) and a constant polynomial \( r(X) = r_0 \) so that
  \begin{equation*}
    q(X) = (X - \alpha) s(X) + r_0.
  \end{equation*}

  Since \( \Phi_\alpha(q) = 0_R \), then necessarily \( r_0 = 0_R \). Therefore,
  \begin{equation*}
    p(X) = (X - \alpha)^{m-1} q(X) = (X - \alpha)^m s(X).
  \end{equation*}
\end{defproof}
