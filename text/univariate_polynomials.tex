\section{Univariate polynomials}\label{sec:univariate_polynomials}

\paragraph{Euclidean division}

\begin{algorithm}[Euclidean division of polynomials]\label{alg:euclidean_division_of_polynomials}
  In a nontrivial commutative ring \( R \), fix two univariate polynomials
  \begin{align*}
    f(X) = \sum_{k=0}^n a_k X^k, &&
    g(X) = \sum_{k=0}^m b_k X^k.
  \end{align*}

  Suppose that the leading coefficient \( b_m \) of \( g(X) \) is invertible. We will build polynomials \( q(X) \) and \( r(X) \), where \( r(X) \) is either zero or \( \deg r < \deg g = m \), such that
  \begin{equation*}
    f(X) = g(X) \cdot q(X) + r(X).
  \end{equation*}

  \Cref{thm:alg:euclidean_division_of_polynomials/unique} will demonstrate uniqueness of \( q(X) \) and \( r(X) \).

  \begin{thmenum}
    \thmitem{alg:euclidean_division_of_polynomials/zero} If \( f(X) \) is the zero polynomial, let
    \begin{align*}
      q(X) \coloneqq 0_R, &&
      r(X) \coloneqq 0_R.
    \end{align*}

    \thmitem{alg:euclidean_division_of_polynomials/no_division} Otherwise, if \( \deg f = n < m = \deg g \), let
    \begin{align*}
      q(X) \coloneqq 0_R, &&
      r(X) \coloneqq f(X).
    \end{align*}

    \thmitem{alg:euclidean_division_of_polynomials/division} Otherwise, \( \deg f = n \geq m = \deg g \). Let
    \begin{equation*}
      \widehat r(X) \coloneqq f(X) - (a_n / b_m) X^{n-m} \cdot g(X).
    \end{equation*}

    We use the algorithm recursively to divide \( \widehat r(X) \) by \( g(X) \), and obtain \( \widehat q(X) \) and \( r(X) \), where \( r(X) \) is either zero or of lower degree than \( g \), such that
    \begin{equation*}
      \widehat r(X) = g(X) \cdot \widehat q(X) + r(X).
    \end{equation*}

    Then
    \begin{align*}
      \widehat r(X)                                                          &= f(X) - (a_n / b_m) X^{n-m} \cdot g(X), \\
      g(X) \cdot \widehat q(X) + r(X)                                        &= f(X) - (a_n / b_m) X^{n-m} \cdot g(X), \\
      g(X) \cdot \parens[\big]{ \widehat q(X) + (a_n / b_m) X^{n-m} } + r(X) &= f(X).
    \end{align*}

    It remains to define
    \begin{equation*}
      q(X) \coloneqq \widehat q(X) + (a_n / b_m) X^{n-m}.
    \end{equation*}
  \end{thmenum}
\end{algorithm}
\begin{comments}
  \item In accordance with \cref{def:euclidean_domain}, we denote \( q(X) \) and \( r(X) \) by \( \quot(f, g) \) and \( \rem(f, g) \).
  \item This algorithm can be found as \identifier{math.polynomials.euclidean_divmod} in \cite{notebook:code}.
\end{comments}
\begin{defproof}
  We only need to show that the algorithm terminates. This is clear if \( f(X) \) is zero. Otherwise, we will do this by induction on the degree \( n \) of \( f(X) \).

  \begin{itemize}
    \item If \( n = 0 \), either \( m > 0 \) and the algorithm terminates at \cref{alg:euclidean_division_of_polynomials/no_division}, or \( m = 0 \) and we use the recursive step \cref{alg:euclidean_division_of_polynomials/division}.

    In the latter case, \( \widehat r(X) \) evaluates to \( 0_R \), so the recursive application uses \cref{alg:euclidean_division_of_polynomials/zero} and then the algorithm terminates.

    \item Suppose that \( n > 0 \) and that the inductive hypothesis holds for \( n - 1 \).

    Again, we use either \cref{alg:euclidean_division_of_polynomials/no_division} or the recursive step \cref{alg:euclidean_division_of_polynomials/division}.

    In the latter case, to use the inductive hypothesis, we only need to show that \( \widehat r(X) \) is either zero or has degree lower than \( n \). Let
    \begin{align*}
      \widehat f(X) &\coloneqq f(X) - a_n X^n, \\
      \widehat g(X) &\coloneqq g(X) - b_m X^m.
    \end{align*}

    Clearly \( \deg \widehat f < n \) and \( \deg \widehat g < m \). Moreover,
    \begin{equation*}
      \widehat r(X) = \widehat f(X) - (a_n / b_m) X^{n-m} \cdot \widehat g(X).
    \end{equation*}

    If \( \widehat r(X) \) is not zero, then
    \begin{equation*}
      \deg \widehat r
      \reloset {\eqref{eq:thm:univariate_polynomial_sum/degree}} \leq
      \max\set{ \deg \widehat f, n - 1 }
      =
      n - 1
    \end{equation*}
    because
    \begin{equation*}
      \deg\parens[\big]{ (a_n / b_m) X^{n-m} \cdot \widehat g(X) }
      \reloset {\eqref{eq:thm:univariate_polynomial_product/degree}} =
      (n - m) + \deg \widehat g
      \leq
      (n - m) + (m - 1)
      =
      n - 1.
    \end{equation*}
  \end{itemize}
\end{defproof}

\begin{proposition}\label{thm:alg:euclidean_division_of_polynomials}
  In a nontrivial commutative ring \( R \), fix a polynomial \( g(X) \) of degree \( n \) whose leading coefficient is invertible. \Fullref{alg:euclidean_division_of_polynomials} gives us maps \( \quot(\anon*, g) \) and \( \rem(\anon*, g) \) on \( R[X] \).

  These quotients and remainders have the following basic properties:
  \begin{thmenum}
    \thmitem{thm:alg:euclidean_division_of_polynomials/unique} They are unique in the following sense: if \( f = q \cdot g + r \), where \( r \) is either zero or \( \deg r < \deg g \), then \( r = \rem(f, g) \) and \( q = \quot(f, g) \).

    \thmitem{thm:alg:euclidean_division_of_polynomials/linear} Both functions \( \rem(\anon*, g) \) and \( \quot(\anon*, g) \) are \hyperref[def:linear_function]{linear}.

    \thmitem{thm:alg:euclidean_division_of_polynomials/rem_zeros} The kernel of \( \rem(\anon*, g) \) is the ideal \( \braket{ g(X) } \).

    \thmitem{thm:alg:euclidean_division_of_polynomials/quot_zeros} The kernel of \( \quot(\anon*, g) \) and the set of \hyperref[def:function_fixed_point]{fixed points} of \( \rem(\anon*, g) \) coincide.

    We will show in \cref{thm:truncated_polynomial_algebra/quot_zeros} that both sets coincide with the \hyperref[def:truncated_polynomial_module]{truncated polynomial module} \( \Pi_{n-1}(R[X]) \).
  \end{thmenum}
\end{proposition}
\begin{proof}
  \SubProofOf{thm:alg:euclidean_division_of_polynomials/unique} We have
  \begin{equation}\label{eq:thm:alg:euclidean_division_of_polynomials/unique/proof/diff}
    [\rem(f, g) - r] = -[\quot(f, g) - q] \cdot g.
  \end{equation}

  If either \( \quot(f, g) - q \) or \( \rem(f, g) - r \) is nonzero, \cref{thm:univariate_polynomial_product/zero_product} implies that both are nonzero.

  In such case, then \cref{thm:univariate_polynomial_product/degree} implies that
  \begin{equation*}
    \deg(\rem(f, g) - r) = \deg(\quot(f, g) - q) + \deg g.
  \end{equation*}

  In particular, \( \deg(\rem(f, g) - r) \geq \deg g \).

  \begin{itemize}
    \item If \( r \) is zero, then \( \rem(f, g) - r = \rem(f, g) \) and thus \( \deg g \leq \deg \rem(f, g) \). But we have assumed that \( \deg \rem(f, g) < \deg g \), which is a contradiction.

    \item Otherwise, if \( \rem(f, g) \) is zero, we similarly derive a contradiction.

    \item If both \( \rem(f, g) \) and \( r \) are nonzero, \cref{thm:univariate_polynomial_sum/degree} implies that
    \begin{equation*}
      \deg(\rem(f, g) - r) \leq \underbrace{\max\set{ \deg \rem(f, g), \deg r }}_{< \deg g},
    \end{equation*}
    which is again a contradiction.
  \end{itemize}

  The obtained contradictions show that both \( \rem(f, g) - r \) and \( \quot(f, g) - q \) are zero.

  \SubProofOf{thm:alg:euclidean_division_of_polynomials/linear}

  \SubProofOf[eq:def:additive_function]{additivity} We have
  \begin{equation*}
    f + p = \parens[\big]{ \quot(f, g) + \quot(p, g) } \cdot g + \parens[\big]{ \rem(f, g) + \rem(p, g) }.
  \end{equation*}

  The sum \( \rem(f, g) + \rem(p, g) \) is either zero or has degree less than \( n \), so the uniqueness shown in \cref{thm:alg:euclidean_division_of_polynomials/unique} implies that \( \rem(f + p, g) = \rem(f, g) + \rem(p, g) \).

  We similarly conclude that \( \quot(f + p, g) = \quot(f, g) + \quot(p, g) \).

  \SubProofOf[def:homogeneous_function]{homogeneity} Analogous.

  \SubProofOf{thm:alg:euclidean_division_of_polynomials/rem_zeros} If \( f \) is in the ideal \( \braket{ g(X) } \), it is a multiple of \( g \), and hence, by the uniqueness of division shown in \cref{thm:alg:euclidean_division_of_polynomials/unique}, its remainder \( \rem(f, g) \) is zero.

  Conversely, if \( \rem(f, g) \) is zero, then \( f = \quot(f, g) \cdot g \), hence \( f \) belongs to \( \braket{ g(X) } \).

  \SubProofOf{thm:alg:euclidean_division_of_polynomials/quot_zeros}
  \SufficiencySubProof* If \( f(X) \) is a fixed point of \( \rem(\anon*, g) \), then
  \begin{equation*}
    f = \quot(f, g) \cdot g + f,
  \end{equation*}
  hence \( \quot(f, g) \cdot g \) is zero. Because \( g \) is nonzero, \cref{thm:univariate_polynomial_product/zero_product} implies that \( \quot(f, g) \) must be zero.

  \NecessitySubProof* If \( f \) is in the kernel of \( \quot(\anon*, g) \), then \( \quot(f, g) \) is zero, and obviously \( f = \rem(f, g) \).
\end{proof}

\paragraph{Horner's rule}

\begin{algorithm}[Horner's rule]\label{alg:horners_rule}\mcite[\S III.V.2]{Обрешков1962ВисшаАлгебра}
  Consider the polynomials
  \begin{align*}
    f(X) = \sum_{k=0}^n a_k X^k
    &&
    g(X) = X + b.
  \end{align*}

  The coefficients of the quotient of \( f(X) \) and \( g(X) \) with respect to \fullref{alg:euclidean_division_of_polynomials} can be computed recursively as follows:
  \begin{equation}\label{eq:alg:horners_rule/quot}
    c_{n-k} \coloneqq \begin{cases}
      a_n,                              &k = 1 \\
      a_{n-(k-1)} - b \cdot c_{n-(k-1)} &k > 1.
    \end{cases}
  \end{equation}

  Furthermore, \( r(X) \) is a constant that equals
  \begin{equation}\label{eq:alg:horners_rule/rem}
    c_{-1} = a_0 - b \cdot c_0.
  \end{equation}
\end{algorithm}
\begin{comments}
  \item This algorithm can be found as \identifier{math.polynomials.division.horner_divmod} in \cite{notebook:code}.
\end{comments}
\begin{proof}
  Denote the quotient by
  \begin{equation*}
    q(X) = \sum_{k=0}^{n-1} c_k X^k.
  \end{equation*}

  We have
  \begin{equation*}
    f(X) = g(X) \cdot q(X) + r(X),
  \end{equation*}
  from where, for every \( k > 0 \),
  \begin{equation*}
    a_k = c_{k-1} + b \cdot c_k,
  \end{equation*}
  from which \eqref{eq:alg:horners_rule/quot} follows.

  Furthermore,
  \begin{equation*}
    r(X) = a_0 - b \cdot c_0,
  \end{equation*}
  which demonstrates \eqref{eq:alg:horners_rule/rem}.
\end{proof}

\begin{remark}\label{rem:alg:horners_rule}
  The phrase \enquote{Horner's rule} is ambiguous.

  \incite{Horner1819SolvingNumericalEquations} describes a method for seeking polynomial roots. Variations of method are described by \incite[144]{Курош1968КурсВысшейАлгебры}, \incite[97]{Винберг2014КурсАлгебры} and \incite[209]{Кострикин2000АлгебраЧасть1} as \enquote{схема Горнера} (\enquote{Horner's scheme}) and \incite[\S III.V.2]{Обрешков1962ВисшаАлгебра} as \enquote{правилото на Хорнер} (\enquote{Horner's rule}). We roughly follow their discussion in \fullref{alg:horners_rule}.

  On the other hand, \incite[486]{Knuth1997ArtVol2} calls \enquote{Horner's rule} the representation
  \begin{equation}\label{eq:rem:alg:horners_rule/knuth}
    f(X) = \sum_{k=0}^n a_k X^k = a_0 + X (a_1 + \cdots + X(a_{n-1} + X a_n) + \cdots),
  \end{equation}
  considered from the point of view of computational complexity --- at allows computing \( f(\alpha) \) via \( n \) multiplications and \( n \) additions, while direct evaluation requires  \( {n(n+1)} / 2 \) multiplications and \( n \) additions.

  \incite[exerc. 3.3.14]{Rosen2019DiscreteMathematics} calls this procedure \enquote{Horner's method}.
\end{remark}

\paragraph{\( n \)-th roots}

\begin{definition}\label{def:nth_root}\mimprovised
  In a \hyperref[def:ring/commutative]{commutative ring}, for \( n \geq 2 \), we say that an element is an \term{\( n \)-th root} of \( a \) if it is a \hyperref[def:polynomial_root]{polynomial root} of \( X^n - a \).

  We use the prefixes from the polynomial degree terminology described in \cref{def:polynomial_degree_terminology}, with one modification --- instead of \enquote{quadratic root}, we say \enquote{square root}.

  If \( b \) is a square root (resp. cubic root) of \( a \), we say that \( a \) is a \term{square} (resp. \term{cube}) of \( b \).
\end{definition}
\begin{comments}
  \item In the case of positive \hyperref[def:real_numbers]{real numbers}, we have a canonical choice of \( n \)-th roots given by \cref{def:principal_nonnegative_nth_root}, where we use the notation \( \sqrt[n]{ x } \). We also have such a canonical choice for square roots of negative real numbers --- see \cref{def:principal_real_square_root}.
\end{comments}

\begin{definition}\label{def:algebraic_equation}\mimprovised
  \hyperref[def:equation]{Equations} in the \hyperref[def:ring/theory]{theory of rings} are called \term{algebraic}. Such equations are equalities of polynomials in finitely many indeterminates. It is sufficient to only consider equations whose right side is zero (since we can cancel it otherwise), thus a general algebraic equation has the form
  \begin{equation*}\label{eq:def:algebraic_equation}
    f(X_1, \ldots, X_n) = 0.
  \end{equation*}

  We use the prefixes from the polynomial degree terminology described in \cref{def:polynomial_degree_terminology}, e.g. we call \eqref{eq:def:algebraic_equation} a \enquote{quadratic equation} if \( f(X_1, \ldots, X_n) \) is a quadratic polynomial.
\end{definition}

\begin{definition}\label{def:trivial_solution_of_algebraic_equation}\mimprovised
  If the polynomial corresponding to an \hyperref[def:algebraic_equation]{algebraic equation} has no \hyperref[def:univariate_polynomial]{constant term}, evaluating all indeterminates to zero provides a \hyperref[def:equation/solution]{solution}. We call it the \term{trivial solution}.
\end{definition}

\begin{remark}\label{rem:root_terminology}
  The term \enquote{root} has several distinct (but related) meanings:
  \begin{itemize}
    \item The \( n \)-th root in the sense of \cref{def:nth_root}. The discussion in \cite{HSMSE:radical_symbol_history} suggests that this is the origin of the terms \enquote{root} and \enquote{radical}, based on the Latin \enquote{radix}.

    There is inherent ambiguity in picking \enquote{the} root, since there may be many. For positive real numbers we have an established canonical choice of \( n \)-th roots, which we call \enquote{principal roots}, and for negative real numbers --- of square roots.

    \item The \hyperref[def:equation/solution]{solutions} of an \hyperref[def:algebraic_equation]{algebraic equation} are also called \enquote{roots}. An explicit definition with this usage can be found in \incite[2]{Обрешков1962ВисшаАлгебра}.

    As noted in \cref{def:algebraic_equation}, polynomials naturally arise from equations on rings, thus it makes sense for this usage to predate polynomials.

    \item From a modern perspective, both are encompassed by roots of polynomials as defined in \cref{def:polynomial_root}. Roots are zeros in the sense of \cref{def:zero_of_function} of a fixed polynomial function. For \hyperref[def:univariate_polynomial]{univariate polynomials}, \cref{def:polynomial_root_multiplicity} provides a characterization in terms of divisibility, as well as a concept of \enquote{multiplicity} of a root.

    A \enquote{root} is defined as a zero of a univariate polynomial by
    \incite[282]{Aluffi2009Algebra},
    \incite[11]{Knapp2016BasicAlgebra} and
    \incite[119]{Тыртышников2017ОсновыАлгебры}.
  \end{itemize}
\end{remark}

\paragraph{Multiple roots}

\begin{definition}\label{def:algebraic_derivative}\mcite[461]{Knapp2016BasicAlgebra}
  Generalizing \cref{def:differentiability} from analysis, we define the \term[ru=производная (\cite[163]{Тыртышников2017ОсновыАлгебры})]{algebraic derivative} of a univariate polynomial
  \begin{equation*}
    f(X) = \sum_{k=0}^n a_k X^k = a_n X^n + a_{n-1} X^{n-1} + \cdots + a_2 X^2 + a_1 X + a_0
  \end{equation*}
  as
  \begin{equation*}
    f'(X) \coloneqq \sum_{k=1}^n k a_k X^{k-1} = n a_n X^{n-1} + (n-1) a_{n-1} X^{n-2} + \cdots + a_2 X + a_1.
  \end{equation*}

  Via \fullref{thm:omega_recursion}, we can define algebraic derivatives of order \( m \) as
  \begin{equation*}
    f^{(m)}(X) \coloneqq \begin{cases}
      f(X)                             &m = 0 \\
      \parens[\big]{ f^{(m - 1)} }'(X) &m > 0
    \end{cases}
  \end{equation*}
\end{definition}

\begin{proposition}\label{thm:def:algebraic_derivative}
  \hyperref[def:algebraic_derivative]{Algebraic derivatives} have the following basic properties:
  \begin{thmenum}
    \thmitem{thm:def:algebraic_derivative/linear} The derivative operator \( f(X) \mapsto f'(X) \) is linear.
    \thmitem{thm:def:algebraic_derivative/degree} \( f^{(n)}(X) \) is either zero or has degree \( \deg f - n \).

    \thmitem{thm:def:algebraic_derivative/product} The product rule holds:
    \begin{equation}\label{eq:thm:def:algebraic_derivative/product}
      (fg)' = f'g + fg'.
    \end{equation}

    \thmitem{thm:def:algebraic_derivative/leibniz} A variant of \fullref{thm:leibniz_rule} holds:
    \begin{equation}\label{eq:thm:def:algebraic_derivative/leibniz}
      (fg)^{(n)} = \sum_{k=0}^n \binom n k f^{(k)} g^{(n-k)}
    \end{equation}

    \thmitem{thm:def:algebraic_derivative/affine_power} If \( m \leq n \), the \( m \)-th derivative of \( (X - \alpha)^n \) is
    \begin{equation*}
      \frac {n!} {(n-m)!} (X - \alpha)^{n-m}.
    \end{equation*}
  \end{thmenum}
\end{proposition}
\begin{proof}
  \SubProofOf{thm:def:algebraic_derivative/linear} Trivial.

  \SubProofOf{thm:def:algebraic_derivative/degree} Trivial.

  \SubProofOf{thm:def:algebraic_derivative/product} By \cref{thm:def:algebraic_derivative/linear}, it is enough to consider the case where both \( f(X) \) and \( g(X) \) are monomials.

  \begin{align*}
    f'(X) g(X) + f(X) g'(X)
    &=
    n a_n X^{n-1} \cdot b_m X^m + a_n X^n \cdot m b_m X^{m-1}
    = \\ &=
    (n + m) a_n b_m X^{n+m-1}
    = \\ &=
    (a_n b_m X^{n+m})'
    = \\ &=
    (fg)'(X).
  \end{align*}

  \SubProofOf{thm:def:algebraic_derivative/leibniz} The proof in \fullref{thm:leibniz_rule} relies only on the product rule, hence it holds here as well.

  \SubProofOf{thm:def:algebraic_derivative/affine_power} We use outer induction on \( m \) and inner induction on \( n \).

  The case \( m = n = 1 \) is obvious. Assume that the statement holds for \( m = 1 \) and \( n - 1 \). Then
  \begin{equation*}
    \parens[\big]{ (X - \alpha)^n }'
    =
    \parens[\big]{ (X - \alpha)^{n-1} \cdot (X - \alpha) }'
    \reloset {\eqref{eq:thm:def:algebraic_derivative/product}} =
    \parens[\big]{ (X - \alpha)^{n-1} }' (X - \alpha) + (X - \alpha)^{n-1}
    \reloset {\T{ind.}} =
    n (X - \alpha)^{n-1}.
  \end{equation*}

  Now suppose that the statement holds for derivatives of order less than \( m \) and for every \( n \geq m \). Then,
  \begin{equation*}
    \parens[\big]{ (X - \alpha)^n }^{(m)}
    =
    \parens*{\parens[\big]{ (X - \alpha)^n }^{(m-1)}}'
    \reloset {\T{ind.}} =
    \parens*{ \frac {n!} {(n - m + 1)!} (X - \alpha)^{n - m + 1} }'
    \reloset {\T{ind.}} =
    \frac {n!} {(n - m)!} (X - \alpha)^{n - m}.
  \end{equation*}
\end{proof}

\begin{definition}\label{def:polynomial_root_multiplicity}\mimprovised
  Let \( R \) be a nontrivial commutative ring.

  We say that the \hyperref[def:polynomial_root]{root} \( \alpha \) of the univariate polynomial \( f(X) \) of degree \( n \) has \term[bg=кратност (\cite[171]{Обрешков1962ВисшаАлгебра}), ru=кратность (\cite[163]{Тыртышников2017ОсновыАлгебры}), en=multiplicity (\cite[229]{Jacobson1985BasicAlgebraI})]{multiplicity} \( m \) if any of the following equivalent conditions hold:
  \begin{thmenum}
    \thmitem{def:polynomial_root_multiplicity/division}\mcite[163]{Тыртышников2017ОсновыАлгебры} The polynomial \( (X - \alpha)^m \) divides \( f(X) \).

    \thmitem{def:polynomial_root_multiplicity/derivative_roots} The value \( \alpha \) is a \hyperref[def:polynomial_root]{root} of the \hyperref[def:algebraic_derivative]{algebraic derivatives} \( f^{(0)}(X), f^{(1)}(X), \ldots, f^{(m-1)}(X) \).
  \end{thmenum}

  Every polynomial \( f(X) \) has a \hyperref[def:multiset]{multiset} of roots. If at least one of them has multiplicity greater than one, we say that it is a \term[bg=многократен (корен) (\cite[171]{Обрешков1962ВисшаАлгебра}), ru=кратный (корень) (\cite[163]{Тыртышников2017ОсновыАлгебры}), en=multiple root (\cite[229]{Jacobson1985BasicAlgebraI})]{multiple root}, otherwise we say that it is a \term[bg=прост корен (\cite[171]{Обрешков1962ВисшаАлгебра}), ru=простой корень (\cite[163]{Тыртышников2017ОсновыАлгебры}), en=simple root (\cite[229]{Jacobson1985BasicAlgebraI})]{simple root}.
\end{definition}
\begin{comments}
  \item This is an extension of the general concept of polynomial roots discussed in \cref{def:polynomial_root}.
\end{comments}
\begin{defproof}
  \ImplicationSubProof{def:polynomial_root_multiplicity/division}{def:polynomial_root_multiplicity/derivative_roots} Suppose that \( (X - \alpha)^m \) divides \( f(X) \). Then there exists a polynomial \( g(X) \) such that
  \begin{equation*}
    f(X) = (X - \alpha)^m g(X).
  \end{equation*}

  For the \( n < m \)-th derivative of \( f(X) \), by \cref{thm:def:algebraic_derivative/leibniz}, we have
  \begin{equation*}
    f^{(n)}(X) = \sum_{k=0}^n \binom n k \underbrace{\parens[\big]{ (X - \alpha)^m }^{(k)}}_{\mathclap{\frac {m!} {(m-k)!} (X - \alpha)^{m-k} \T*{by} \ref{thm:def:algebraic_derivative/affine_power}}} q^{(n-k)}(X).
  \end{equation*}

  Let \( \Phi_\alpha: R[X] \to R \) be the \hyperref[con:evaluation_homomorphism]{evaluation homomorphism} at \( \alpha \). Then
  \begin{equation*}
    \Phi_\alpha(f^{(n)}) = \sum_{k=0}^n \binom n k \frac {m!} {(m-k)!} 0_R^{m-k} \Phi_\alpha(g^{(n-k)})
  \end{equation*}

  For \( n < m \), clearly \( \Phi_\alpha(f^{(n)}) = 0_R \).

  \ImplicationSubProof{def:polynomial_root_multiplicity/derivative_roots}{def:polynomial_root_multiplicity/division} Suppose that \( \alpha \) is a root of \( f^{(0)}(X), \ldots, f^{(m-1)}(X) \). We will use induction on \( m \) to show that \( (X - \alpha)^m \mid f(X) \).

  The case \( m = 0 \) is trivial. Suppose that \( \alpha \) is root of \( f^{(0)}(X), \ldots, f^{(m-1)}(X) \) and that \( (X - \alpha)^{m-1} \) divides \( f(X) \). Additionally suppose that \( \alpha \) is a root of \( f^{(m)}(X) \).

  By the inductive hypothesis, there exists a polynomial \( g(X) \) such that
  \begin{equation*}
    f(X) = (X - \alpha)^{m-1} g(X).
  \end{equation*}

  By \cref{thm:def:algebraic_derivative/leibniz},
  \begin{equation*}
    f^{(m-1)}(X) = \sum_{k=0}^{m-1} \binom {m-1} k \underbrace{\parens[\big]{ (X - \alpha)^{m - 1} }^{(k)}}_{\mathclap{\frac {(m - 1)!} {(m - 1 - k)!} (X - \alpha)^{m - 1 - k} \T*{by} \ref{thm:def:algebraic_derivative/affine_power}}} q^{(m - 1 - k)}(X).
  \end{equation*}

  Then
  \begin{equation*}
    \Phi_\alpha(f^{(m-1)}) = \sum_{k=0}^{m-1} \binom {m-1} k \frac {(m - 1)!} {(m - 1 - k)!} 0_R^{m - 1 - k} \Phi_\alpha(q^{(m - 1 - k)}).
  \end{equation*}

  All terms on the right are zero except for the case where \( k = m - 1 \), where \( g^{(0)} = g \) and the entire expression reduces to \( \Phi_\alpha(g) \). The scalar \( \alpha \) is a root of \( f^{(m-1)} \), implying that it is also a root of \( g \).

  We use \fullref{alg:euclidean_division_of_polynomials} to obtain a polynomial \( s(X) \) and a constant polynomial \( r(X) = r_0 \) so that
  \begin{equation*}
    g(X) = (X - \alpha) s(X) + r_0.
  \end{equation*}

  Since \( \Phi_\alpha(g) = 0_R \), then necessarily \( r_0 = 0_R \). Therefore,
  \begin{equation*}
    f(X) = (X - \alpha)^{m-1} g(X) = (X - \alpha)^m s(X).
  \end{equation*}
\end{defproof}

\begin{theorem}[Polynomial factor theorem]\label{thm:polynomial_factor_theorem}\mcite[corr. 2.2]{Jacobson1985BasicAlgebraI}
  Over a nontrivial commutative ring, the value \( \alpha \) is a \hyperref[def:polynomial_root]{root} of the univariate polynomial \( f(X) \) if and only if \( (X - \alpha) \) divides \( f(X) \).
\end{theorem}
\begin{comments}
  \item A much stronger theorem holds in \hyperref[def:gcd_domain]{GCD domains} --- see \fullref{thm:polynomial_factorization_via_roots}.
\end{comments}
\begin{proof}
  Special case of the equivalence of definitions in \cref{def:polynomial_root_multiplicity}.
\end{proof}

\paragraph{Truncated polynomial modules}

\begin{definition}\label{def:truncated_polynomial_algebra}\mimprovised
  Corresponding to the \hyperref[def:polynomial_algebra]{polynomial algebra} \( R[X] \) (over an arbitrary commutative semiring), we define the \term{truncated polynomial module} of degree \( d \) as the \( R \)-submodule
  \begin{equation*}
    \Pi_d(R[X]) \coloneqq \lin\set{ 1, X, X^2, \ldots, X^d }
  \end{equation*}

  In a nontrivial commutative ring, by the argumentation in \cref{ex:def:hamel_basis/polynomial_algebra}, the monomials \( 1, X, \ldots, X^d \) form a \hyperref[def:hamel_basis]{Hamel basis} of \( \Pi_d(R[X]) \). Furthermore, due to the isomorphism in \cref{thm:truncated_polynomial_algebra/isomorphism}, we identify \( \Pi_d(R[X]) \) with the \hyperref[def:algebra_over_ring/quotient]{quotient algebra} \( R[X] / \braket{ X^{d+1} } \).
\end{definition}

\begin{proposition}\label{thm:truncated_polynomial_algebra}
  In a nontrivial commutative ring \( R \), fix a polynomial \( g(X) \) of degree \( n \) whose leading coefficient is invertible.

  We will now develop a correspondence between the \hyperref[def:algebra_over_ring/quotient]{quotient algebra} \( R[X] / \braket{ g(X) } \) and the \hyperref[def:truncated_polynomial_algebra]{truncated polynomial module} \( \Pi_{n-1}(R[X]) \).

  \begin{thmenum}
    \thmitem{thm:truncated_polynomial_algebra/quot_zeros} The kernel of \( \quot(\anon*, g) \) and the set of fixed points of \( \rem(\anon*, g) \) both coincide with \( \Pi_{n-1}(R[X]) \).

    \thmitem{thm:truncated_polynomial_algebra/principal} For every coset \( [f] \) in \( R[X] / \braket{ g(X) } \), we have
    \begin{equation*}
      [f] \cap \Pi_{n-1}(R[X]) = \set{ \rem(f, g) }.
    \end{equation*}

    \thmitem{thm:truncated_polynomial_algebra/algebra} With the operation
    \begin{equation*}
      (f \ast_g p) \coloneqq \rem(fp, g),
    \end{equation*}
    the truncated \( R \)-module \( \Pi_{n-1}(R[X]) \) becomes a commutative \( R \)-algebra.

    \thmitem{thm:truncated_polynomial_algebra/isomorphism} The map
    \begin{equation*}
      \varphi([f]) \coloneqq \rem(f, g)
    \end{equation*}
    is a well-defined \( R \)-algebra isomorphism between \( R[X] / \braket{ g(X) } \) and \( \Pi_{n-1}(R[X]) \).

    \thmitem{thm:truncated_polynomial_algebra/basis} The cosets \( [1], [X], \ldots, [X^{n-1}] \) form a \hyperref[def:hamel_basis]{Hamel basis} of \( R[X] / \braket{ g(X) } \).
  \end{thmenum}
\end{proposition}
\begin{proof}
  \SubProofOf{thm:truncated_polynomial_algebra/quot_zeros} \Cref{thm:alg:euclidean_division_of_polynomials/quot_zeros} implies that the zeros of \( \quot(\anon*, g) \) coincide with the fixed points of \( \rem(\anon*, g) \). We will now show that \( \rem(p, g) = r \) if and only if \( r \) is in \( \Pi_{n-1}(R[X]) \).

  \SufficiencySubProof* Suppose that \( \rem(r, g) = r \). Then \( r \) is either zero or \( \deg r < n \). In both cases, \( r \) is in \( \Pi_{n-1}(R[X]) \).

  \NecessitySubProof* If \( r \in \Pi_{n-1}(R[X]) \), then division must use either \cref{alg:euclidean_division_of_polynomials/zero} or \cref{alg:euclidean_division_of_polynomials/no_division}. In both cases, \( \rem(r, g) = r \).

  \SubProofOf{thm:truncated_polynomial_algebra/principal} Fix a polynomial \( r \) that is both in \( [f] \) and \( \Pi_{n-1}(R[X]) \).

  Since \( r \) is in \( [f] \), there exists a polynomial \( q \) such that \( f = qg + r \). Then, since \( r \) is in \( \Pi_{n-1}(R[X]) \), \cref{thm:alg:euclidean_division_of_polynomials/unique} implies that \( q = \quot(f, g) \) and \( r = \rem(f, g) \).

  \SubProofOf{thm:truncated_polynomial_algebra/algebra} We must show that \( (f \ast_g p) = \rem(fp, g) \) is associative, commutative and bilinear.

  \SubProofOf*[def:binary_operation/associative]{associativity} We have
  \begin{align*}
    (fp)s
    &=
    \parens[\big]{ \quot(fp, g) \cdot g + \rem(fp, g) } \cdot s
    = \\ &=
    \parens[\big]{ \quot(fp, g) \cdot s + \quot\parens[\big]{ \rem(fp, g) \cdot s, g } } \cdot g + \rem\parens[\big]{ \rem(fp, g) \cdot s, g }
  \end{align*}
  and similarly
  \begin{equation*}
    f(ps)
    =
    \parens[\big]{ f \cdot \quot(ps, g) + \quot\parens[\big]{ f \cdot \rem(ps, g), g } } \cdot g + \rem\parens[\big]{ f \cdot \rem(ps, g), g }
  \end{equation*}

  The uniqueness of remainders shown in \cref{thm:alg:euclidean_division_of_polynomials/unique} implies that
  \begin{equation*}
    \underbrace{\rem\parens[\big]{ \overbrace{\rem(fp, g)}^{f \ast_g p} \cdot s, g }}_{(f \ast_g p) \ast_g s} = \underbrace{\rem\parens[\big]{ f \cdot \overbrace{\rem(ps, g)}^{p \ast_g s}, g }}_{f \ast_g (p \ast_g s)},
  \end{equation*}
  as desired.

  \SubProofOf*[def:binary_operation/commutative]{commutativity} Follows directly from commutativity of the convolution product.

  \SubProofOf*[eq:def:additive_function]{additivity} Due to commutativity, it will be sufficient to show additivity in the first argument; that is, we want to show that
  \begin{equation*}
    (f + p) \ast_g s = \rem((f + p)s, g) = \rem(fs, g) + \rem(ps, g) = (f \ast_g s) + (p \ast_g s).
  \end{equation*}

  This follows from \cref{thm:alg:euclidean_division_of_polynomials/linear} and bilinearity of the polynomial convolution product.

  \SubProofOf*[eq:def:additive_function]{homogeneity} Analogous.

  \SubProofOf{thm:truncated_polynomial_algebra/isomorphism}

  \SubProof*{Proof of well-definedness} For any polynomial \( p \) in \( [f] \), there exists some \( s \) such that \( p = sg + f \). Then
  \begin{equation*}
    p = sg + \quot(f, g) \cdot g + \rem(f, g).
  \end{equation*}

  The uniqueness of remainders shown in \cref{thm:alg:euclidean_division_of_polynomials/unique} implies that
  \begin{equation*}
    \varphi([p]) = \rem(p, g) = \rem(f, g) = \varphi([f]).
  \end{equation*}

  \SubProofOf[def:function_invertibility/injective/equality]{injectivity} Suppose that \( \rem(f, g)  = \rem(p, g) \). Then
  \begin{equation*}
    f - p = \parens[\big]{ \quot(f, g) - \quot(p, g) } \cdot g + \parens[\big]{ \underbrace{\rem(f, g) - \rem(p, g)}_{0} },
  \end{equation*}
  thus \( [f] = [p] \).

  \SubProofOf[def:function_invertibility/surjective/existence]{surjectivity} Fix a polynomial \( r \) in \( \Pi_{n-1}(R[X]) \).

  \Cref{thm:truncated_polynomial_algebra/quot_zeros} implies that \( \rem(r, g) = r \), so \( [r] \) is the desired preimage.

  \SubProofOf[def:linear_function]{linearity} Follows from \cref{thm:alg:euclidean_division_of_polynomials/linear}.

  \SubProof{Proof that the isomorphism \hyperref[def:algebra_over_ring/homomorphism]{preserves multiplication}} We have
  \begin{equation*}
    fp = \quot(fp, g) \cdot g + \rem(fp, g)
  \end{equation*}
  and
  \begin{align*}
    fp
    &=
    \parens[\big]{ \quot(f, g) \cdot g + \rem(f, g) } \cdot \parens[\big]{ \quot(p, g) \cdot g + \rem(p, g) }
    = \\ &=
    \quot(f, g) \cdot \quot(f, g) \cdot g^2 + \parens[\big]{ \quot(f, g) + \rem(p, g) } \cdot g + \rem(f, g) \cdot \rem(p, g).
  \end{align*}

  The uniqueness of remainders shown in \cref{thm:alg:euclidean_division_of_polynomials/unique} implies that
  \begin{equation*}
    \underbrace{\rem(fp, g)}_{\varphi([fp])}
    =
    \underbrace{\rem\parens[\big]{ \overbrace{\rem(f, g)}^{\varphi([f])} \cdot \overbrace{\rem(p, g)}^{\varphi([p])}, g }}_{\varphi([f]) \ast_g \varphi([p])}
  \end{equation*}

  \SubProofOf{thm:truncated_polynomial_algebra/basis} Obvious considering the linear isomorphism with \( \Pi_{n-1}(R[X]) \).
\end{proof}

\begin{definition}\label{def:quotient_polynomial_algebra_as_truncated}\mimprovised
  In a nontrivial commutative ring \( R \), fix a polynomial \( g(X) \) of degree \( n \) whose leading coefficient is invertible.

  We showed in \cref{thm:truncated_polynomial_algebra/isomorphism} how the \hyperref[def:algebra_over_ring/quotient]{quotient algebra} \( R[X] / \braket{ g(X) } \) is isomorphic to the \hyperref[def:truncated_polynomial_algebra]{truncated polynomial module} \( \Pi_{n-1}(R[X]) \) with the product operation
  \begin{equation*}
     (f \ast_g p) \coloneqq \rem(fp, g).
  \end{equation*}

  Furthermore, we showed in \cref{thm:truncated_polynomial_algebra/principal} that \( \rem(f, g) \) is the unique representative of the coset \( [f] \) in \( \Pi_{n-1}(R[X]) \). We call it the \term{principal representative} of \( [f] \).

  We thus find it natural to regard \( R[X] / \braket{ g(X) } \) not as a family of cosets, but as the truncated module \( \Pi_{n-1}(R[X]) \) of its principal representatives endowed with the product operation \( {\ast_g} \).
\end{definition}

\paragraph{Quotient algebras and adjunction}

\begin{proposition}\label{thm:adjoint_roots_and_quotients}
  Fix arbitrary nontrivial commutative rings \( R \subseteq S \) and some element \( u \) of \( S \). Consider the \hyperref[con:evaluation_homomorphism]{evaluation homomorphism} \( \Phi_u: R[X] \to S \) sending \( X \) to \( u \). Additionally, suppose that the kernel of \( \Phi_u \) is a principal ideal with generator \( f(X) \).

  Then the quotient ring \( R[X] / \braket{ f(X) } \) is isomorphic to the ring \( R[u] \) obtained by \hyperref[def:semiring_adjunction]{adjoining} \( u \) to \( R \).
\end{proposition}
\begin{comments}
  \item Even though
  \begin{equation*}
    \ker \Phi_u
    =
    \set{ p(X) \in R[X] \given p(u) = 0 }
    \reloset {\ref{thm:polynomial_factor_theorem}} =
    \set{ p(X) \in R[X] \given (X - u) \T{divides} p(X) },
  \end{equation*}
  unless \( u \) itself belongs to \( R \), the above is not the principal ideal of \( (X - u) \).

  When \( R \) is a \hyperref[def:field]{field}, by \cref{thm:def:principal_ideal_domain/field_polynomials}, \( R[X] \) is a \hyperref[def:principal_ideal_domain]{principal ideal domain}, and thus the kernel of \( \Phi_u \) is always principal. If \( u \) is an \hyperref[def:algebraic_element]{algebraic element}, i.e. if \( \ker \Phi_u \) is nontrivial and thus has a monic generator, this generator is called the \hyperref[def:algebraic_element_minimal_polynomial]{minimal polynomial} of \( u \).

  \item When \( R \) is an \hyperref[def:integral_domain]{integral domain}, the elements of the ring \( R[u] \) correspond to the \hyperref[def:quotient_polynomial_algebra_as_truncated]{principal representatives} of the cosets in \( R[X] / \braket{ f(X) } \).
\end{comments}
\begin{proof}
  \Cref{thm:ring_zero_morphisms/isomorphism} implies that
  \begin{equation*}
    R[X] / \underbrace{\ker \Phi_u}_{\braket{ f(X) }} \cong \underbrace{\img \Phi_u}_{R[u]}.
  \end{equation*}
\end{proof}

\begin{definition}\label{def:gaussian_integers}\mcite[\S V.6.2]{Aluffi2009Algebra}
  A \term[ru=целые гауссовые числа (\cite[example 3.5.1]{Винберг2014КурсАлгебры})]{Gaussian integer} is a \hyperref[def:complex_numbers]{complex number} whose real and imaginary part are \hyperref[def:integers]{integers}.
\end{definition}

\begin{example}\label{ex:thm:adjoint_roots_and_quotients}
  We list several examples related to \cref{thm:adjoint_roots_and_quotients}:
  \begin{thmenum}
    \thmitem{ex:thm:adjoint_roots_and_quotients/gaussian} We can define several rings isomorphic to the \hyperref[def:gaussian_integers]{Gaussian integers}, demonstrating \cref{thm:adjoint_roots_and_quotients}.

    We assume that the field of complex numbers is available to us.

    \begin{thmenum}
      \thmitem{ex:thm:adjoint_roots_and_quotients/gaussian/quotient} Based on how we defined complex numbers in \cref{def:complex_numbers}, we can simply take the \hyperref[def:algebra_over_ring/quotient]{quotient algebra} \( \BbbZ[X] / \braket{X^2 + 1} \).

      \thmitem{ex:thm:adjoint_roots_and_quotients/gaussian/evaluation} Alternatively, we can \hyperref[def:semiring_adjunction]{adjoin} \( i \) to \( \BbbZ \) to obtain the ring \( \BbbZ[i] \).

      Given a Gaussian integer \( z = a + bi \), it corresponds to the polynomial
      \begin{equation*}
        f_z(X) \coloneqq a + bX.
      \end{equation*}

      Conversely, consider the \hyperref[con:evaluation_homomorphism]{evaluation homomorphism} \( \Phi_i: \BbbZ[X] \to \BbbC \) for the imaginary unit. Let \( f(X) \in \BbbZ[X] \). Then
      \begin{equation*}
        f(i)
        =
        \Phi_i(f)
        =
        \sum_{k=0}^n a_k i^n
        =
        \thickspace \sum_{\scriptscriptstyle{\mathclap{\rem(k, 4) = 0}}}^n a_k - \sum_{\scriptscriptstyle{\mathclap{\rem(k, 4) = 2}}}^n a_k + i \parens*{ \quad \sum_{\scriptscriptstyle{\mathclap{\rem(k, 4) = 1}}}^n a_k - \sum_{\scriptscriptstyle{\mathclap{\rem(k, 4) = 3}}}^n a_k }.
      \end{equation*}

      This is clearly again a Gaussian integer.

      Thus, although we skipped proving that \( \braket{X^2 + 1} \) is the kernel of \( \Phi_i: \BbbZ[X] \to \BbbC \), we have shown that the rings \( \BbbZ[X] / \braket{ X^2 + 1 } \) and \( \BbbZ[i] \) behave identically.
    \end{thmenum}

    \thmitem{ex:thm:adjoint_roots_and_quotients/integers_with_sqrt2} Similarly to \cref{ex:thm:adjoint_roots_and_quotients/gaussian}, we have
    \begin{equation*}
      \BbbZ[X] / \braket{X^2 - 2} \cong \BbbZ[\sqrt 2].
    \end{equation*}

    The gist of this example is that, even though \( \BbbZ[\sqrt 2] \) and \( \BbbZ[i] \) are isomorphic as modules, their vector multiplication operation is different. Indeed, since \( X^2 = 2 \), we have
    \begin{align*}
      (a + bX) (c + dX) = (ac + 2bd) + (bc + ad)X
    \end{align*}
    which is different compared to complex multiplication.

    \thmitem{ex:thm:adjoint_roots_and_quotients/integers_mod_xx_minus_1} In \cref{ex:thm:adjoint_roots_and_quotients/gaussian}, we discussed how the ring of \hyperref[def:gaussian_integers]{Gaussian integers} can be defined via the quotient
    \begin{equation*}
      \BbbZ[X] / \braket{ X^2 + 1 }.
    \end{equation*}

    If we instead take
    \begin{equation*}
      \BbbZ[X] / \braket{ X^2 - 1 },
    \end{equation*}
    multiplication of \( a + bX \) and \( c + dX \) would behave as
    \begin{equation*}
      (a + bX) (c + dX) = (ac + bd) + (bc + ad)X.
    \end{equation*}

    This is less useful since \( -1 \) and \( 1 \) are already roots of \( X^2 - 1 \), and we adjoin a new root. Nevertheless, the example shows that polynomial quotient algebras are constructed similarly when the polynomial is not irreducible.
  \end{thmenum}
\end{example}
