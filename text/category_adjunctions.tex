\subsection{Category adjunctions}\label{subsec:category_adjunctions}

\begin{remark}\label{rem:adjoint_functors}
  Suppose we have the functors \( F: \cat{C} \to \cat{D} \) and \( G: \cat{D} \to \cat{C} \).

  \begin{itemize}
    \item If \( F \) is a left inverse to \( G \), given only \( G(X) \), \( F \) can restore \( X \).

    \item If \( F \) is instead a left adjoint to \( G \), given \( G(X) \) and some object \( A \) in \( \cat{C} \), \( F \) can give us an object \( F(A) \) such that the morphisms from \( F(A) \) to \( X \) are uniquely defined by those from \( A \) to \( G(X) \). Thus, \( F \) cannot restore \( X \), but it can give us objects in \( \cat{D} \) that act on \( X \) as their images under \( G \) would act on \( G(X) \) in \( \cat{C} \). This is described in \cite{StanfordPlato:category_theory} as \( F \) being a \enquote{conceptual inverse} of \( G \).

    \item If \( F \) is a right adjoint to \( G \), the morphisms from \( X \) to \( F(A) \) are determined by those from \( G(X) \) to \( A \).
  \end{itemize}

  \Fullref{def:category_adjunction} contains two equivalent definition of an adjunction, and \fullref{rem:universal_mapping_property} describes how they can be characterized via universal mapping properties.
\end{remark}

\begin{definition}\label{def:category_adjunction}\mcite[sec. 2.2]{Leinster2016Basic}
  An \term{adjunction} between the \hyperref[def:category]{categories} \( \cat{C} \) and \( \cat{D} \) can be defined in several equivalent ways. Let \( F: \cat{C} \to \cat{D} \) and \( G: \cat{D} \to \cat{C} \) be arbitrary functors.

  In both cases below, if there exists an adjunction between \( F \) and \( G \), we say that \( F \) is \term{left adjoint} to \( G \) and, correspondingly, that \( G \) is \term{right adjoint} to \( F \). A conventional notation for adjoint functors is \( F \dashv G \).

  \begin{thmenum}
    \thmitem{def:category_adjunction/hom} A \term{hom-adjunction} is a triple \( (F, G, \varphi) \), where \( \varphi \) is \hyperref[thm:natural_isomorphism]{natural isomorphism}
    \begin{equation}\label{eq:def:category_adjunction/hom}
      \varphi: \cat{D}(F(\anon*), \anon*) \Rightarrow \cat{C}(\anon*, G(\anon*)).
    \end{equation}

    The functors
    \begin{align*}
      &\cat{D}(F(\anon*), \anon*): \cat{C}^\oppos \times \cat{D} \to \cat{Set}, \\
      &\cat{C}(\anon*, G(\anon*)): \cat{C}^\oppos \times \cat{D} \to \cat{Set}
    \end{align*}
    are straightforward modifications of the \hyperref[eq:def:hom_functor/binary]{binary hom-functor} on \( \cat{C} \).

    Naturality of \( \varphi \) means that, for every two morphisms \( f: B \to A \) in \( \cat{C} \) and \( g: X \to Y \) in \( \cat{D} \), the following diagram commutes:
    \begin{equation}\label{eq:def:category_adjunction/varphi_nat_diagram}
      \begin{aligned}
        \includegraphics[page=1]{output/def__category_adjunction}
      \end{aligned}
    \end{equation}

    The above diagram reduces to the simpler to verify condition that, for every triple of morphisms \( f: B \to A \), \( s: F(A) \to X \) and \( g: X \to Y \), we must have the equality
    \begin{equation}\label{eq:def:category_adjunction/varphi_nat}
      G(g) \bincirc \varphi_{A,X}(s) \bincirc f = \varphi_{B,Y}(g \bincirc s \bincirc F(f)).
    \end{equation}

    \thmitem{def:category_adjunction/unit_counit} A \term{unit-counit adjunction} is a quadruple \( (F, G, \eta, \varepsilon) \), where
    \begin{equation}\label{eq:def:category_adjunction/unit_counit/signature}
      \begin{aligned}
               \eta &: \id_{\cat{C}} \Rightarrow G \bincirc F, \\
        \varepsilon &: F \bincirc G \Rightarrow \id_{\cat{D}}
      \end{aligned}
    \end{equation}
    are natural transformations satisfying the condition that, for any pair of objects \( A \) in \( \cat{C} \) and \( Y \) in \( \cat{D} \), the following triangle diagrams commute:

    \begin{minipage}{0.43\textwidth}
      \begin{equation}\label{eq:def:category_adjunction/d_triangle}
        \begin{aligned}
          \includegraphics[page=2]{output/def__category_adjunction}
        \end{aligned}
      \end{equation}
    \end{minipage}
    \hfill
    \begin{minipage}{0.44\textwidth}
      \raggedright
      \begin{equation}\label{eq:def:category_adjunction/c_triangle}
        \begin{aligned}
          \includegraphics[page=3]{output/def__category_adjunction}
        \end{aligned}
      \end{equation}
    \end{minipage}
    \smallskip

    Note that an adjunction is not an \hyperref[def:category_equivalence]{equivalence}, they simply have a common setup. Similarly to \hyperref[def:category_equivalence]{equivalence}, we call the \hyperref[def:natural_transformation]{natural transformation} \( \eta \) the \term{unit} of the adjunction and \( \varepsilon \) the \term{counit}.
  \end{thmenum}
\end{definition}
\begin{defproof}
  \ImplicationSubProof{def:category_adjunction/hom}{def:category_adjunction/unit_counit} Let \( (F, G, \varphi) \) be a hom-adjunction.

  For every morphism \( f: B \to A \) in \( \cat{C} \), from the naturality of \( \varphi \) we have
  \begin{equation}\label{eq:def:category_adjunction/varphi_eta}
    \begin{aligned}
      \includegraphics[page=4]{output/def__category_adjunction}
    \end{aligned}
  \end{equation}

  Since \( \varphi_{A,F(B)} \) is a morphism in \( \cat{Set} \), it is a function, and we can apply it in order to define the family
  \begin{equation*}
    \begin{aligned}
      &\eta: \id_{\cat{C}} \Rightarrow G \bincirc F, \\
      &\eta_A \coloneqq \varphi_{A,F(A)}(\id_{F(A)}).
    \end{aligned}
  \end{equation*}

  We must show that \( \eta \) is a natural transformation. On the diagram \eqref{eq:def:category_adjunction/varphi_eta}, we can start in the top left corner with \( F(\id_A) \) and top right corner with \( F(\id_B) \) and reach the middle.

  We obtain that,
  \begin{equation*}
    \cat{C}(f, [G \bincirc F](\id_A))\parens[\Big]{ \underbrace{\varphi_{A, F(A)}(\id_A)}_{\eta_A} }
    =
    \eta_A \bincirc f
  \end{equation*}
  and
  \begin{equation*}
    \cat{C}(\id_B, [G \bincirc F](f))\parens[\Big]{ \underbrace{\varphi_{B, F(B)}(\id_B)}_{\eta_B} }
    =
    [G \bincirc F](f) \bincirc \eta_B
  \end{equation*}
  are equal. That is, the following diagram commutes:
  \begin{equation}\label{eq:def:category_adjunction/eta_nat}
    \begin{aligned}
      \includegraphics[page=5]{output/def__category_adjunction}
    \end{aligned}
  \end{equation}

  In order to define the natural transformation \( \varepsilon: F \bincirc G \Rightarrow \id_{\cat{D}} \), we use the inverse transformation \( \varphi^{-1} \). For every morphism \( g: X \to Y \) in \( \cat{D} \), we have
  \begin{equation}\label{eq:def:category_adjunction/varphi_varepsilon}
    \begin{aligned}
      \includegraphics[page=6]{output/def__category_adjunction}
    \end{aligned}
  \end{equation}

  Define the family
  \begin{equation*}
    \begin{aligned}
      &\varepsilon: F \bincirc G \Rightarrow \id_{\cat{D}}, \\
      &\varepsilon_X \coloneqq \varphi_{G(X),X}^{-1}(\id_{G(X)}).
    \end{aligned}
  \end{equation*}

  We can prove that \( \varepsilon \) is a natural transformation analogously to how we proved it for \( \eta \), and we will skip the details.

  We will now show that the triangle diagram \eqref{eq:def:category_adjunction/d_triangle} commutes. Consider the morphism \( (\eta_A, F(\id_A)) \) in \( \cat{C}^\oppos \times \cat{D} \). Applying the functors \( \cat{D}(F(\anon*), \anon*) \) and \( \cat{D}(\anon*, G(\anon*)) \) to this morphism and using the naturality of \( \varphi \), we obtain
  \begin{equation}\label{eq:def:category_adjunction/d_triangle_proof}
    \begin{aligned}
      \includegraphics[page=7]{output/def__category_adjunction}
    \end{aligned}
  \end{equation}

  Note that \( \varepsilon_{F(A)} \) is a member of \( \cat{D}([F \bincirc G \bincirc F](A), F(A)) \).

  Composing the functions in \eqref{eq:def:category_adjunction/d_triangle_proof} in one direction, we obtain
  \begin{balign*}
    &\phantom{{}={}}
    \varphi_{A,F(A)}^{-1} \parens[\Bigg]{ \cat{C}\parens[\Big]{ \eta_A, [G \bincirc F](\id_A) } \parens[\Big]{ \varphi_{[G \bincirc F](A),F(A)} (\varepsilon_{F(A)}) } }
    = \\ &=
    \varphi_{A,F(A)}^{-1} \parens[\Bigg]{ \parens[\Big]{ \varphi_{[G \bincirc F](A),F(A)} (\varepsilon_{F(A)}) } \bincirc \eta_A }
    = \\ &=
    \varphi_{A,F(A)}^{-1} \parens[\Big]{ \id_{[G \bincirc F](A)} \bincirc \eta_A }
    = \\ &=
    \id_{F(A)}.
  \end{balign*}

  Composing the functions in \eqref{eq:def:category_adjunction/d_triangle_proof} in the other direction, we obtain
  \begin{equation*}
    \cat{D}\parens[\Big]{ F(\eta_A), F(\id_A) } (\varepsilon_{F(A)})
    =
    \varepsilon_{F(A)} \bincirc F(\eta_A).
  \end{equation*}

  Therefore,
  \begin{equation*}
    \id_{F(A)} = \varepsilon_{F(A)} \bincirc F(\eta_A).
  \end{equation*}
  and thus \eqref{eq:def:category_adjunction/d_triangle} commutes.

  We can similarly prove that \eqref{eq:def:category_adjunction/c_triangle} commutes.

  Therefore, \( (F, G, \eta, \varepsilon) \) is a unit-counit adjunction.

  \ImplicationSubProof{def:category_adjunction/unit_counit}{def:category_adjunction/hom} Let \( (F, G, \eta, \varepsilon) \) be a unit-counit adjunction.

  For every pair of objects \( A \in \cat{C} \) and \( X \in \cat{D} \), define the functions
  \begin{equation*}
    \begin{aligned}
      &\varphi_{A,X}: \cat{D}(F(A), X) \to \cat{C}(A, G(X)) \\
      &\varphi_{A,X}(g) \coloneqq G(g) \bincirc \eta_A.
    \end{aligned}
  \end{equation*}
  and
  \begin{equation*}
    \begin{aligned}
      &\psi_{A,X}^{-1}: \cat{C}(A, G(X)) \to \cat{D}(F(A), X) \\
      &\psi_{A,X}^{-1}(f) \coloneqq \varepsilon_X \bincirc F(f),
    \end{aligned}
  \end{equation*}

  From the naturality of \( \varepsilon \) and from \eqref{eq:def:category_adjunction/d_triangle}, it follows that the following diagram commutes:
  \begin{equation}\label{eq:def:category_adjunction/varphi_inverse_def}
    \begin{aligned}
      \includegraphics[page=8]{output/def__category_adjunction}
    \end{aligned}
  \end{equation}

  Therefore,
  \begin{equation*}
    g = \varepsilon_X \bincirc \underbrace{[F \bincirc G](f) \bincirc F(\eta_A)}_{F(\varphi_{A,X}(g))} = [\varphi_{A,X} \bincirc \varphi_{A,X}](g)
  \end{equation*}
  and thus \( \psi_{A,X} \) is a left inverse of \( \varphi_{A,X} \).

  We can analogously show that \( \psi_{A,X} \) is a right inverse, and hence that \( \varphi_{A,X} \) is invertible.

  Since we have already shown that \( \varphi \) is a bijective function, it remains to verify the naturality of \( \varphi \) in order to show that it is a natural isomorphism. Let \( f: B \to A \) be a morphism in \( \cat{C} \) and \( g: X \to Y \) be a morphism in \( \cat{D} \). Fix some morphism \( s: F(A) \to X \).

  Composing the functions of \eqref{eq:def:category_adjunction/varphi_nat_diagram} in one direction, we obtain
  \begin{equation}\label{eq:def:category_adjunction/varphi_nat_diagram_chase_right}
    \varphi_{B, Y}\parens[\Big]{ \cat{D}(F(f), g)(s) }
    =
    \varphi_{B, Y}\parens[\Big]{ g \bincirc s \bincirc F(f) }
    =
    G(g) \bincirc G(s) \bincirc [G \bincirc F](f) \bincirc \eta_B.
  \end{equation}

  In the other direction, we have
  \begin{equation}\label{eq:def:category_adjunction/varphi_nat_diagram_chase_down}
    \cat{C}(f, G(g))\parens[\Big]{ \varphi_{A, X}(s) }
    =
    \cat{C}(f, G(g))\parens[\Big]{ G(s) \bincirc \eta_A }
    =
    G(g) \bincirc G(s) \bincirc \eta_A \bincirc f.
  \end{equation}

  From the naturality of \( \eta \), we have that \eqref{eq:def:category_adjunction/eta_nat} commutes and hence
  \begin{equation*}
    \eta_A \bincirc f
    =
    [G \bincirc F](s) \bincirc \eta_B.
  \end{equation*}

  Therefore, \eqref{eq:def:category_adjunction/varphi_nat_diagram_chase_right} and \eqref{eq:def:category_adjunction/varphi_nat_diagram_chase_down} are equal and, thus, \eqref{eq:def:category_adjunction/varphi_nat_diagram} also commutes.

  This proves the naturality of \( \varphi \).
\end{defproof}

\begin{proposition}\label{thm:category_adjunction_duality}
  The functor \( F: \cat{C} \to \cat{D} \) is \hyperref[def:category_adjunction]{left adjoint} to \( G: \cat{D} \to \cat{C} \) if and only if the \hyperref[def:opposite_functor]{dual functor} \( F^\oppos \) is right adjoint to \( G^\oppos \).

  This is part of the duality principles listed in \fullref{thm:categorical_principle_of_duality}.
\end{proposition}
\begin{proof}
  \begin{equation*}
    \cat{C^\oppos}(G^\oppos(X), A) = \cat{C}(A, G(X)) \cong \cat{D}(F(A), X) = \cat{D^\oppos}(X, F^\oppos(A)).
  \end{equation*}
\end{proof}

\begin{definition}\label{def:concrete_category}\mcite[26]{MacLane1998}
  A \term{concrete category} is a pair \( (\cat{C}, U) \), where \( \cat{C} \) is a category and \( U: \cat{C} \to \cat{Set} \) is a \hyperref[def:functor_invertibility/faithful]{faithful functor} that gives us a set for any object of \( \cat{C} \). More generally, a pair \( (\cat{C}, U) \), where \( U: \cat{C} \to \cat{D} \), is a concrete category over \( \cat{D} \).

  In the case of a concrete category over \( \cat{Set} \), we can regard any morphism \( f \) in \( \cat{C} \) as a function. Thus, for example, we can say that \( f \) is an \hyperref[def:function_invertibility/injective]{injective} morphism if \( U(f) \) is an injective function. This is discussed further in \fullref{thm:concrete_category_function_invertibility}.

  In the context of a concrete category, we call \( U \) a \term{forgetful functor} and any \hyperref[def:category_adjunction]{left adjoint} to \( U \) functor a \term{free functor}. According to Jean-Pierre Marquis in \cite{StanfordPlato:category_theory}, the motivation for this terminology is that free functors build objects that are free from additional restrictions.

  We list several examples in \fullref{ex:def:category_adjunction}. The forgetful functor is usually clear from the context, and we identify a concrete category \( (\cat{C}, U) \) with its underlying set \( \cat{C} \). The corresponding free functor, however, often requires a nontrivial but straightforward construction.
\end{definition}

\begin{remark}\label{rem:concrete_categories}
  Unless it is important, in \hyperref[def:concrete_category]{concrete categories} over \( \cat{Set} \), we will ignore writing the forgetful functor and always assume that morphism composition is function composition. This automatically implies that the identity morphism is the identity function.
\end{remark}

\begin{example}\label{ex:def:category_adjunction}
  We list some examples of \hyperref[def:category_adjunction]{category adjunctions}. Note that only some of them are commonly referred to as \enquote{free}.

  \begin{thmenum}
    \thmitem{ex:def:category_adjunction/set_top} Perhaps the simplest meaningful example of an adjunction is the \hyperref[def:discrete_topology]{discrete topology} functor \( D: \cat{Set} \to \cat{Top} \), which is left adjoint to the forgetful functor \( U: \cat{Top} \to \cat{Set} \), which maps a small \hyperref[def:topological_space]{topological space} \( (X, \mscrT) \) into its underlying set \( X \).

    Given a set \( A \) and a topological space \( (X, \mscrT) \), every function \( s: A \to X \) is \hyperref[def:global_continuity]{continuous} when \( A \) is endowed with the discrete topology. Conversely, every continuous function is obviously a \hyperref[def:function]{function}. It follows that there is an equality
    \begin{equation*}
      \cat{Top}\parens[\Big]{ \underbrace{(A, \pow(A))}_{D(A)}, (X, \mscrT) } = \cat{Set}\parens[\Big]{ A, X }.
    \end{equation*}

    Therefore, \( (D, U, \id) \) is a hom-adjunction. Furthermore, \( (D, U, \id, \id) \) is a unit-counit adjunction.

    \thmitem{ex:def:category_adjunction/top_set} The \hyperref[def:indiscrete_topology]{indiscrete topology} functor \( I: \cat{Set} \to \cat{Top} \) is right-adjoint to the same forgetful functor \( U: \cat{Top} \to \cat{Set} \), again with identities for all natural transformations of the adjunction.

    Therefore, we have
    \begin{equation*}
      D \dashv U \dashv I.
    \end{equation*}

    \thmitem{ex:def:category_adjunction/set_cat} We discussed in \fullref{ex:discrete_category_adjunction} the \hyperref[def:discrete_category]{discrete category} functor \( D: \cat{Set} \to \cat{Cat} \). We showed in \fullref{ex:set_discr_cat_isomorphism} that, when restricted to the subcategory \( \cat{DiscrCat} \) rather than \( \cat{Cat} \), \( D \) it is an inverse to the forgetful functor \( U \). In the general case, however, this is an adjunction rather than an isomorphism. More precisely, \( D \) is left adjoint to \( U \).

    Note that for any functor \( F: \cat{C} \to \cat{D} \), we have \( U(F) \coloneqq F\restr_{\obj(C)} \). Thus, \( U \) is not only a functor in \( [\cat{Cat}, \cat{Set}] \); it also induces a natural isomorphism between the functors \( \cat{Cat}(D(\anon*), \cat{\anon*}) \) and \( \cat{Set}(\anon*, U(\anon*)) \).

    Indeed, fix a small category \( \cat{C} \) and a set \( A \). From our discussion in \fullref{ex:set_discr_cat_isomorphism} it is obvious that the restriction
    \begin{equation*}
      U: \cat{Cat}(D(A), \cat{C}) \to \cat{Set}(A, U(\cat{C}))
    \end{equation*}
    is a bijective function.

    In order to verify the naturality of the transformation induced by \( U \), we must show that for any function \( f: B \to A \) and functor \( F: \cat{C} \to \cat{D} \), the following diagram commutes
    \begin{equation}\label{eq:ex:def:category_adjunction/set_cat/u_nat}
      \begin{aligned}
        \includegraphics[page=1]{output/ex__def__category_adjunction}
      \end{aligned}
    \end{equation}

    The commutativity of \eqref{eq:ex:def:category_adjunction/set_cat/u_nat} follows from the following: for every functor \( S: D(A) \to \cat{C} \) we have
    \begin{equation*}
      U(F \bincirc S \bincirc D(f))
      =
      U(F) \bincirc U(S) \bincirc U(D(f))
      =
      U(F) \bincirc U(S) \bincirc f.
    \end{equation*}

    Therefore, \( (D, U, U) \) is a \hyperref[def:category_adjunction/hom]{hom-adjunction}.

    We can also explicitly define a unit-counit adjunction. The unit \( \eta: \id_{\cat{Set}} \Rightarrow U \bincirc D \) is simply the identity.

    The counit is slightly more interesting. Given a small category \( \cat{C} \), applying \( D \bincirc U \) gives us the subcategory consisting only of the objects and identity morphisms of \( \cat{C} \). Then the counit \( \varepsilon: D \bincirc U \Rightarrow \id_{\cat{Cat}} \) is simply the inclusion functor \( \Iota \) from this subcategory to \( \cat{C} \).

    The triangle
    \begin{equation}\label{eq:ex:def:category_adjunction/set_cat/triangles}
      \begin{aligned}
        \includegraphics[page=2]{output/ex__def__category_adjunction}
        \quad\quad
        \includegraphics[page=3]{output/ex__def__category_adjunction}
      \end{aligned}
    \end{equation}
    corresponding to \eqref{eq:def:category_adjunction/d_triangle} and \eqref{eq:def:category_adjunction/c_triangle}, obviously commute.

    The quadruple \( (D, U, \eta, \varepsilon) \) is a \hyperref[def:category_adjunction/unit_counit]{unit-counit adjunction}.

    \thmitem{ex:def:category_adjunction/dm_cat} The left adjoint of the forgetful functor \( U \) from \( \cat{Cat} \) to the \hyperref[def:directed_multigraph/category]{category of directed multigraphs}, which we will denote by \( \cat{DMGph} \) is the free category functor defined in \fullref{def:directed_multigraph_free_category}. We denote this functor by \( F \).

    We can define the family of functions
    \begin{equation}\label{eq:ex:def:category_adjunction/dm_cat/varphi_family}
      \begin{aligned}
        &\varphi: \cat{Cat}(F(\anon*), \cat{\anon*}) \Rightarrow \cat{DMGph}(\anon*, U(\cat{\anon*})), \\
        &\varphi_{Q, \cat{C}}(S) \coloneqq \parens[\Big]{ v \mapsto S(v), a \mapsto S(\iota(a)) }.
      \end{aligned}
    \end{equation}

    For every functor \( S: F(Q) \to \cat{C} \), define \( \varphi_{Q, \cat{C}} \) so that it sends \( S \) to a \hyperref[def:graph_walk/directed]{walk} containing only one arc.

    Define the canonical embedding \( \iota \):
    \begin{equation*}
      \begin{aligned}
        &\iota: Q \to [U \bincirc F](Q) \\
        &\iota_V(v) \coloneqq v, \\
        &\iota_A(e) \coloneqq h(e) \overset e \to t(e).
      \end{aligned}
    \end{equation*}

    We will later see that \( \iota \) is the unit of a unit-counit adjunction.

    Now, from \eqref{eq:def:directed_multigraph_free_category/functor_from_homomorphism}, it is clear that the free category functor \( F \), when restricted to the set of directed multigraph homomorphisms \( \cat{DMGph}(Q, U(\cat{C})) \), is the two-sided inverse of \( \varphi_{Q, \cat{C}} \).

    We will show that \( \varphi \) is a natural transformation. Fix a functor \( G: \cat{C} \to \cat{D} \) and a homomorphism \( (g_V, g_A): Q \to R \). We must show that the following diagram commutes:
    \begin{equation}\label{eq:ex:def:category_adjunction/dm_cat/varphi_nat}
      \begin{aligned}
        \includegraphics[page=4]{output/ex__def__category_adjunction}
      \end{aligned}
    \end{equation}

    That is, for every functor \( S: F(A) \to \cat{C} \), we must show
    \begin{equation*}
      \varphi_{R, \cat{D}}(G \bincirc S \bincirc F(g_V, g_A))
      =
      U(G) \bincirc \varphi_{R, \cat{D}}(S) \bincirc (g_V, g_A).
    \end{equation*}

    This is also clear from \eqref{eq:def:directed_multigraph_free_category/functor_from_homomorphism}.

    Therefore, \( (F, U, \varphi) \) is a \hyperref[def:category_adjunction/hom]{hom-adjunction}.

    Furthermore, the canonical embedding \( \iota \) defined above, when parameterized by \( Q \), is a unit of adjunction.

    We can define a counit \( \varepsilon: F \bincirc U \Rightarrow \id_{\cat{Cat}} \) as follows. As discussed in \fullref{def:directed_multigraph_free_category}, for every path
    \begin{equation*}
      p = \anon \overset {e_1} \to \anon \overset \cdots \to \anon \overset {e_n} \to \anon
    \end{equation*}
    in the directed multigraph \( U(\cat{C}) \), the functor \( F \bincirc U \) \enquote{evaluates} \( p \) as
    \begin{equation*}
      e_n \bincirc e_{n-1} \bincirc \cdots \bincirc e_1.
    \end{equation*}

    Since the embedding only produces paths with a single arc, the adjunction triangles commute:
    \begin{equation}\label{eq:ex:def:category_adjunction/dm_cat/triangles}
      \begin{aligned}
        \includegraphics[page=5]{output/ex__def__category_adjunction}
        \quad\quad
        \includegraphics[page=6]{output/ex__def__category_adjunction}
      \end{aligned}
    \end{equation}

    \thmitem{ex:def:category_adjunction/us_ds}\mcite{MathOF:free_digraph} Denote the \hyperref[def:directed_graph/category]{category of simple directed graphs} by \( \cat{DGph} \) and the \hyperref[def:directed_graph/category]{category of simple undirected graphs} by \( \cat{UGph} \).

    The doubling functor \( D_S: \cat{UGph} \to \cat{DGph} \) from \fullref{def:graph_functors/simple_doubling} is a natural candidate for left adjoint to the \hyperref[def:concrete_category]{forgetful functor} \hyperref[def:graph_functors/simple_forgetful]{\( U_S: \cat{DGph} \to \cat{UGph} \)}. This is not the case, however, as there are \hyperref[def:undirected_multigraph/homomorphism]{undirected graph homomorphisms} that fail to be \hyperref[def:directed_multigraph/homomorphism]{directed graph homomorphisms}. An example is shown in \cref{fig:ex:def:category_adjunction/us_ds}.

    \begin{figure}[!ht]
      \hfill
      \includegraphics[page=7]{output/ex__def__category_adjunction}
      \hfill
      \includegraphics[page=8]{output/ex__def__category_adjunction}
      \hfill\hfill
      \caption{Two undirected homomorphisms from \( G \) to \( U_S(Q) \), denoted using dashed lines, only one of which is a directed multigraph homomorphism from \( D_S(G) \) to \( Q \).}
      \label{fig:ex:def:category_adjunction/us_ds}
    \end{figure}

    If we consider instead a directed graph homomorphisms \( f: Q \to D_S(G) \), it is clear that it is also an undirected graph homomorphism from \( U_S(Q) \) to \( G \). Thus, \( D_S \) is actually \hi{right adjoint} to \( U_S \).
  \end{thmenum}
\end{example}

\begin{definition}\label{def:adjoint_equivalence}
  We call the quadruple \( (F, G, \eta, \varepsilon) \) with signature \eqref{eq:def:category_equivalence/signature} an \term{adjoint equivalence} if it is both an \hyperref[def:category_adjunction]{adjunction} and \hyperref[def:category_equivalence]{equivalence}.
\end{definition}

\begin{proposition}\label{thm:adjoint_equivalence}
  Let \( (F, G, \eta, \varepsilon) \) be a \hyperref[def:category_equivalence]{category equivalence} between \( \cat{C} \) and \( \cat{D} \).

  There exists a natural isomorphism \( \zeta: \id_{\cat{C}} \Rightarrow G \bincirc F \) such that \( (F, G, \zeta, \varepsilon) \) is an \hyperref[def:adjoint_equivalence]{adjoint equivalence}.
\end{proposition}
\begin{proof}
  From \fullref{thm:equivalence_induces_fully_faithful_and_essentially_surjective_functor} it follows that \( F \) is fully faithful and essentially surjective.

  We will now use the same trick as in the end of our proof of \fullref{thm:fully_faithful_and_essentially_surjective_functor_induces_equivalence} to define \( \zeta \).

  Since \( F \) is fully faithful, there is a bijective function
  \begin{equation*}
    \varphi: \cat{D}\parens[\Big]{ F(A), [F \bincirc G \bincirc F](A) } \to \cat{C}\parens[\Big]{ A, [F \bincirc G](A) }.
  \end{equation*}

  Hence, we can define
  \begin{equation*}
    \begin{aligned}
      &\zeta: \id_{\cat{C}} \to G \bincirc F, \\
      &\zeta_A \coloneqq \varphi(\varepsilon_{F(A)}^{-1})
    \end{aligned}
  \end{equation*}
  so that \( F(\zeta_A) = \varepsilon_{F(A)}^{-1} \). By \fullref{thm:def:functor_invertibility/fully_faithful_reflects_invertible}, \( \zeta_A \) is also an isomorphism.

  As in \fullref{thm:fully_faithful_and_essentially_surjective_functor_induces_equivalence}, we use \fullref{thm:commutative_diagrams_preserved_and_reflected} and the naturality of \( \varepsilon \) to prove that \eqref{eq:thm:fully_faithful_and_essentially_surjective_functor_induces_equivalence/varepsilon_source_nat} implies \eqref{eq:thm:fully_faithful_and_essentially_surjective_functor_induces_equivalence/varepsilon_image_nat} (with \( \eta \) replaced by \( \zeta \)).

  Therefore, \( \zeta \) is a natural isomorphism and the quadruple \( (F, G, \zeta, \varepsilon) \) is an equivalence of categories.
\end{proof}

\begin{proposition}\label{thm:functor_adjoint_uniqueness}
  If a functor has two \hyperref[def:category_adjunction]{left adjoints} (resp. right adjoints), then there exists a unique natural isomorphism between them.

  We say that left adjoints (resp. right adjoints) are unique up to a unique natural isomorphism.
\end{proposition}
\begin{proof}
  We will first prove the statement for left adjoints. Suppose that \( (F', G, \eta', \varepsilon') \) and \( (F', G, \eta^\dprime, \varepsilon^\dprime) \) are two unit-counit adjunctions.

  \SubProof{Proof of existence of isomorphism}\mcite{MathSE:left_adjoint_uniqueness} We can utilize the naturality of the units \( \eta' \) and \( \eta^\dprime \) and counits \( \varepsilon' \) and \( \varepsilon^\dprime \) to show that the following diagram commutes:
  \begin{equation}\label{eq:thm:functor_adjoint_uniqueness/existence}
    \begin{aligned}
      \includegraphics[page=1]{output/thm__functor_adjoint_uniqueness}
    \end{aligned}
  \end{equation}

  By the commuting triangle \eqref{eq:def:category_adjunction/d_triangle}, all paths from \( F'(A) \) to \( F'(A) \) above are identities.

  The bottom-most path in \eqref{eq:thm:functor_adjoint_uniqueness/existence} justifies defining the natural transformation
  \begin{equation*}
    \begin{aligned}
      &\alpha: F' \Rightarrow F^\dprime \\
      &\alpha_A \coloneqq \varepsilon'_{F^\dprime(A)} \bincirc F'(\eta_A^\dprime).
    \end{aligned}
  \end{equation*}

  Then \( \alpha_A \) is an isomorphism for every object \( A \) in \( \cat{C} \) with inverse \( \varepsilon_{F'(A)}^\dprime \bincirc F^\dprime(\eta_A') \). Therefore, it is a natural isomorphism from \( F' \) to \( F^\dprime \).

  \SubProof{Proof of uniqueness of isomorphism} Suppose that \( \beta: F' \Rightarrow F^\dprime \) is another natural isomorphism. Then, by the commuting triangle \eqref{eq:def:category_adjunction/d_triangle}, the following diagram also commutes:
  \begin{equation}\label{eq:thm:functor_adjoint_uniqueness/uniqueness}
    \begin{aligned}
      \includegraphics[page=2]{output/thm__functor_adjoint_uniqueness}
    \end{aligned}
  \end{equation}

  Therefore,
  \begin{equation*}
    \beta_A
    =
    \varepsilon^\dprime_{F^\dprime(A)} \bincirc F^\dprime(\eta^\dprime_A) \bincirc \alpha_A
    \reloset {\eqref{eq:def:category_adjunction/d_triangle}} =
    \alpha_A.
  \end{equation*}

  This finishes the proof for left adjoints. The other direction is \hyperref[thm:categorical_principle_of_duality]{dual}. If \( G' \) and \( G^\dprime \) are two right adjoints to \( F \), then by \fullref{thm:category_adjunction_duality}, \( G'^\oppos \) and \( {G^\dprime}^\oppos \) are left adjoints and are thus isomorphic. Then by \fullref{thm:morphism_invertibility_duality}, \( G' \) and \( G^\dprime \) are also isomorphic.
\end{proof}

\begin{proposition}\label{thm:universal_objects_as_adjunctions}
  Fix a category \( \cat{C} \). We can characterize the universal objects in \( \cat{C} \) from \fullref{def:universal_objects} via adjunctions with the \hyperref[def:universal_categories]{terminal category} \( \cat{1} \).

  Let \( \Delta^{\cat{1}}: \cat{C} \to \cat{1} \) be the \hyperref[def:diagonal_functor]{constant functor} into \( \cat{1} \).

  \begin{thmenum}
    \thmitem{thm:universal_objects_as_adjunctions/initial} The object \( I \) of \( \cat{C} \) is \hyperref[def:universal_objects]{initial} if and only if it is (the unique value of) a left adjoint to \( \Delta^{\cat{1}} \) functor.

    In particular, the uniqueness proved in \fullref{thm:def:universal_objects/initial} follows from \fullref{thm:functor_adjoint_uniqueness}.

    \thmitem{thm:universal_objects_as_adjunctions/terminal} \hyperref[thm:categorical_principle_of_duality]{Dually}, the \hyperref[def:universal_objects]{terminal objects} are exactly the right adjoint to \( \Delta_I^{\cat{1}} \) functors.
  \end{thmenum}
\end{proposition}
\begin{proof}
  We will only prove \fullref{thm:universal_objects_as_adjunctions/initial} since the other direction is \hyperref[thm:categorical_principle_of_duality]{dual}.

  \SufficiencySubProof Let \( I \) be an initial object in \( \cat{C} \). We can then regard it as a functor \( F: \cat{1} \to \cat{C} \).

  We define the natural transformations
  \begin{equation*}
    \begin{aligned}
      &\eta: \id_{\cat{1}} \Rightarrow \Delta_I^{\cat{1}} \bincirc F \\
      &\eta_{\cat{0}} \coloneqq \id_{\cat{0}}
    \end{aligned}
  \end{equation*}
  and
  \begin{equation*}
    \begin{aligned}
      &\varepsilon: F \bincirc \Delta_I^{\cat{1}} \Rightarrow \id_{\cat{C}} \\
      &\varepsilon_A \T{is the unique morphism} I \to A
    \end{aligned}
  \end{equation*}

  Since \( I \) has a unique morphism into any other object of \( \cat{C} \), for every morphism \( f: A \to B \), the following diagram commutes:
  \begin{equation}\label{eq:thm:universal_objects_as_adjunctions/sufficiency_nat}
    \begin{aligned}
      \includegraphics[page=1]{output/thm__universal_objects_as_adjunctions}
    \end{aligned}
  \end{equation}

  It follows that both \( \eta \) and \( \varepsilon \) are natural transformations. Furthermore, they trivially satisfy the triangle diagrams triangle \eqref{eq:def:category_adjunction/d_triangle} and \eqref{eq:def:category_adjunction/c_triangle}.

  Hence, \( (F, \Delta_I^{\cat{1}}, \eta, \varepsilon) \) is a \hyperref[def:category_adjunction/unit_counit]{unit-counit adjunction}.

  \NecessitySubProof Conversely, suppose that \( (F, \Delta_I^{\cat{1}}, \eta, \varepsilon) \) is a unit-counit adjunction.

  Let \( I \coloneqq F(\cat{0}) \). Then \( \varepsilon_A \) is a morphism from \( I \) to \( A \). By \eqref{eq:def:category_adjunction/d_triangle}, \( \varepsilon_I = \id_I \) since the following diagram commutes:
  \begin{equation}\label{eq:thm:universal_objects_as_adjunctions/d_triangle}
    \begin{aligned}
      \includegraphics[page=2]{output/thm__universal_objects_as_adjunctions}
    \end{aligned}
  \end{equation}

  Suppose that \( \zeta \) is another morphism from \( I \) to \( A \). The naturality of \( \varepsilon \) implies that, for the morphism \( \id_A: A \to A \), the following diagram commutes:
  \begin{equation}\label{eq:thm:universal_objects_as_adjunctions/necessity_nat}
    \begin{aligned}
      \includegraphics[page=3]{output/thm__universal_objects_as_adjunctions}
    \end{aligned}
  \end{equation}

  The upper left triangle in \eqref{eq:thm:universal_objects_as_adjunctions/necessity_nat} is \eqref{eq:thm:universal_objects_as_adjunctions/d_triangle}.

  We conclude that \( \zeta = \varepsilon_A \) and, generalizing on \( A \), that every morphism from \( I \) is unique.
\end{proof}

\begin{remark}\label{rem:left_and_right_adjoint_not_equivalence}
  We discussed in \fullref{ex:def:universal_objects/grp} that \enquote{the} trivial group \( \set{ e } \) is a \hyperref[def:universal_objects/zero]{zero object} of \( \cat{Grp} \). By \fullref{thm:universal_objects_as_adjunctions}, this object induces a functor that is both left adjoint and right adjoint of \( \Delta_I^{\cat{1}} \). Nevertheless, the categories \( \cat{Grp} \) and \( \cat{1} \) are not \hyperref[def:category_equivalence]{equivalent}.
\end{remark}

\begin{remark}\label{rem:universal_mapping_property}
  We will now regard adjoint functors as a way to \enquote{construct} new objects.

  Let \( (F, G, \iota, \pi) \) be a \hyperref[def:category_adjunction/unit_counit]{unit-counit adjunction} between the categories \( \cat{C} \) and \( \cat{D} \). In the current context, especially in connection with \hyperref[def:category_of_cones/limit]{limits} and \hyperref[def:category_of_cones/colimit]{colimits}, we will call the components of the counit \( \pi: F \bincirc G \Rightarrow \id_{\cat{D}} \) --- \term{projections}, and the components of the unit \( \iota: \id_{\cat{C}} \Rightarrow G \bincirc F \) --- \term{coprojections}.

  Take objects \( A \) in \( \cat{C} \) and \( X \) in \( \cat{D} \) and a morphism \( f: A \to G(X) \). We want to obtain a morphism \( \widetilde{f}: F(A) \to X \), for which the following diagram commutes:
  \begin{equation}\label{eq:rem:universal_mapping_property/c_triangle}
    \begin{aligned}
      \includegraphics[page=1]{output/rem__universal_mapping_property}
    \end{aligned}
  \end{equation}

  From the naturality of \( \iota \) and from the triangle diagram \eqref{eq:def:category_adjunction/c_triangle} it follows that the following diagram commutes:
  \begin{equation}\label{eq:rem:universal_mapping_property/f_tilde_existence}
    \begin{aligned}
      \includegraphics[page=2]{output/rem__universal_mapping_property}
    \end{aligned}
  \end{equation}

  It is clear from \eqref{eq:rem:universal_mapping_property/f_tilde_existence} that
  \begin{equation*}
    G(\widetilde{f}) = G(\pi_X) \bincirc [G \bincirc F](f) = G(\pi_X \bincirc F(f)).
  \end{equation*}

  Furthermore, this value is unique. From the naturality of \( \pi \) and the triangle diagram \eqref{eq:def:category_adjunction/d_triangle} it follows that the following diagram commutes:
  \begin{equation}\label{eq:rem:universal_mapping_property/f_tilde_uniquness}
    \begin{aligned}
      \includegraphics[page=3]{output/rem__universal_mapping_property}
    \end{aligned}
  \end{equation}

  Therefore,
  \begin{equation*}
    \widetilde{f} = \pi_X \bincirc F(f)
  \end{equation*}

  Taking into account that the functor \( F \) itself is unique up to a unique isomorphism, as per \fullref{thm:functor_adjoint_uniqueness}, we have proved the following statement:
  \begin{displayquote}
    For every object \( A \) in \( \cat{C} \), there exist unique up to a unique isomorphism object \( F(A) \) in \( \cat{D} \) and canonical coprojection map \( \iota_A: A \to [G \bincirc F](A) \) satisfying the following property, called a \term{universal mapping property}:
    \begin{displayquote}
      For every object \( X \) in \( \cat{D} \) and every map \( f: A \to G(X) \) in \( \cat{C} \), there exists a unique map \( \widetilde{f}: F(A) \to X \) in \( \cat{D} \) such that the diagram \eqref{eq:rem:universal_mapping_property/c_triangle} commutes.
    \end{displayquote}
  \end{displayquote}

  Intuitively, this universal mapping property states that any map (morphism) with domain \( A \) in \( \cat{C} \) can be transformed into a map with domain \( F(A) \) in \( \cat{D} \).

  The statement becomes more meaningful when we regard \( G: \cat{D} \to \cat{C} \) as a \hyperref[def:concrete_category]{forgetful functor}. In this case, every object of \( \cat{D} \) is regarded as an object of \( \cat{C} \), and we write \( X \) rather than \( G(X) \). The universal mapping property then becomes:
  \begin{displayquote}
    For every object \( A \) in \( \cat{C} \), there exist unique up to a unique isomorphism object \( F(A) \) in \( \cat{D} \) and canonical coprojection map \( \iota_A: A \to F(A) \) satisfying the following universal mapping property:
    \begin{displayquote}
      For every object \( X \) in \( \cat{D} \) and every map \( f: A \to X \) in \( \cat{C} \), there exists a unique map \( \widetilde{f}: F(A) \to X \) in \( \cat{D} \) such that the following diagram commutes:
      \begin{equation}\label{eq:rem:universal_mapping_property/c_triangle_forgetful}
        \begin{aligned}
          \includegraphics[page=4]{output/rem__universal_mapping_property}
        \end{aligned}
      \end{equation}
    \end{displayquote}
  \end{displayquote}

  In \fullref{def:concrete_category} we mentioned that we will call the left adjoint of a forgetful functor a free functor. Universal mapping properties allow characterizing certain \enquote{free constructions}, such as the free groups defined in \fullref{def:free_group}, without explicitly building a free functor and proving that it is left adjoint. Indeed, for every suitable object and map, we explicitly build the natural isomorphism \( \varphi \) of a hom-adjunction, and the commutative triangle \eqref{eq:rem:universal_mapping_property/c_triangle} ensures that this \( \varphi \) is a natural transformation.

  Universal mapping properties of this form are used for \hyperref[def:category_of_cones/colimit]{colimits} --- see \fullref{rem:limit_universal_mapping_property}.

  Of course, there is a \hyperref[thm:categorical_principle_of_duality]{dual} universal mapping property:
  \begin{displayquote}
    For every object \( X \) in \( \cat{D} \) there exist unique up to a unique isomorphism object \( G(X) \) in \( \cat{C} \) and canonical projection map \( \pi_X: [F \bincirc G](X) \to X \) satisfying the following property, called a \term{universal mapping property}:
    \begin{displayquote}
      For every object \( A \) in \( \cat{C} \) and every map \( g: F(A) \to X \) in \( \cat{D} \), there exists a unique map \( \widetilde{g}: A \to G(X) \) in \( \cat{C} \) such that the following diagram commutes:
      \begin{equation}\label{eq:rem:universal_mapping_property/d_triangle}
        \begin{aligned}
          \includegraphics[page=5]{output/rem__universal_mapping_property}
        \end{aligned}
      \end{equation}
    \end{displayquote}
  \end{displayquote}

  In this case, we can regard \( F \) as a forgetful functor and \( G \) as a free functor to obtain the following:
  \begin{displayquote}
    For every object \( X \) in \( \cat{D} \) there exist unique up to a unique isomorphism object \( G(X) \) in \( \cat{C} \) and canonical projection map \( \pi_X: G(X) \to X \) satisfying the following property, called a \term{universal mapping property}:
    \begin{displayquote}
      For every object \( A \) in \( \cat{C} \) and every map \( g: A \to X \) in \( \cat{D} \), there exists a unique map \( \widetilde{g}: A \to G(X) \) in \( \cat{C} \) such that the following diagram commutes:
      \begin{equation}\label{eq:rem:universal_mapping_property/d_triangle_forgetful}
        \begin{aligned}
          \includegraphics[page=6]{output/rem__universal_mapping_property}
        \end{aligned}
      \end{equation}
    \end{displayquote}
  \end{displayquote}

  Universal mapping properties of this form are used for \hyperref[def:category_of_cones/limit]{limits} --- see \fullref{rem:limit_universal_mapping_property}.
\end{remark}

\begin{proposition}\label{thm:concrete_category_function_invertibility}
  If \( \cat{C} \) is a \hyperref[def:concrete_category]{concrete category} with forgetful functor \( U: \cat{C} \to \cat{Set} \).

  \begin{thmenum}
    \thmitem{thm:concrete_category_function_invertibility/injection_is_mono} Every \hyperref[def:set_valued_map/empty]{\hi{nonempty}} \hyperref[def:function_invertibility/injective]{injective} morphism in \( \cat{C} \) is a \hyperref[def:morphism_invertibility/left_cancellative]{monomorphism}.

    That is, for every morphism \( f \) in \( \cat{C} \), if \( U(f) \) is a \hyperref[def:set_valued_map/empty]{\hi{nonempty}} \hyperref[def:function_invertibility/injective]{injective function}, \( f \) is a \hyperref[def:morphism_invertibility/left_cancellative]{monomorphism}.

    \thmitem{thm:concrete_category_function_invertibility/surjection_is_epi} \hyperref[thm:categorical_principle_of_duality]{Dually}, every \hyperref[def:function_invertibility/surjective]{surjective} morphism in \( \cat{C} \) is an \hyperref[def:morphism_invertibility/right_cancellative]{epimorphism}.

    \thmitem{thm:concrete_category_function_invertibility/invertible} Every \hyperref[def:morphism_invertibility/left_invertible]{split monomorphism} is injective, every \hyperref[def:morphism_invertibility/right_cancellative]{split epimorphism} is surjective and every categorical isomorphism is a bijective function.

    \thmitem{thm:concrete_category_function_invertibility/mono_is_injection} If \( U \) has a \hyperref[def:category_adjunction]{left adjoint}, every monomorphism is injective.

    \thmitem{thm:concrete_category_function_invertibility/epi_is_surjection} If \( U \) has a \hyperref[def:category_adjunction]{right adjoint}, every epimorphism is surjective.

    \thmitem{thm:concrete_category_function_invertibility/isomorphisms} If \( U \) has both a left and right adjoint, isomorphisms are bijective morphisms whose inverses are also morphisms.
  \end{thmenum}
\end{proposition}
\begin{proof}
  \SubProofOf{thm:concrete_category_function_invertibility/injection_is_mono} By \fullref{thm:function_invertibility_categorical}, a nonempty injective function is a monomorphism in \( \cat{Set} \). This translates to morphisms in \( \cat{C} \) since they are special cases of functions. Formally, this follows from \fullref{thm:def:functor_invertibility/faithful_reflects_cancellative}.

  \SubProofOf{thm:concrete_category_function_invertibility/surjection_is_epi} Again by \fullref{thm:function_invertibility_categorical}, a surjective function is an epimorphism in \( \cat{Set} \). Again from \fullref{thm:def:functor_invertibility/faithful_reflects_cancellative}, this implies that every surjective morphism in \( \cat{C} \) is an epimorphism.

  \SubProofOf{thm:concrete_category_function_invertibility/invertible} Follows from \fullref{thm:def:functor_invertibility/preserves_inverses}.

  \SubProofOf{thm:concrete_category_function_invertibility/mono_is_injection} Let \( F: \cat{Set} \to \cat{C} \) be a left adjoint to \( U \) and fix a set \( A \). As discussed in \fullref{rem:universal_mapping_property}, this leads to the following \hyperref[rem:universal_mapping_property]{universal mapping property}:
  \begin{displayquote}
    For every object \( X \) in \( \cat{D} \) and every morphism \( g: A \to U(X) \) in \( \cat{C} \), there exists a unique morphism \( \widetilde{g}: F(A) \to X \) in \( \cat{D} \) such that the following diagram commutes:
    \begin{equation}\label{eq:thm:concrete_category_function_invertibility/left_universal_diagram}
      \begin{aligned}
        \includegraphics[page=1]{output/thm__concrete_category_function_invertibility}
      \end{aligned}
    \end{equation}
  \end{displayquote}

  First, suppose that \( f: X \to Y \) is a monomorphism in \( \cat{C} \) and let \( [U(f)](x_1) = [U(f)](x_2) \) (by \fullref{thm:function_invertibility_categorical/empty}, \( U(f) \) is nonempty). Aiming at a contradiction, suppose that \( x_1 \neq x_2 \). Define the function \( g: U(X) \to U(X) \) as the identity \( U(\id_X) \) for which \( x_1 \) and \( x_2 \) are swapped as in our proof of \fullref{thm:function_invertibility_categorical/left_cancellative}. By the property \eqref{eq:thm:concrete_category_function_invertibility/left_universal_diagram}, the following diagram commutes:
  \begin{equation}\label{eq:thm:concrete_category_function_invertibility/left_diagram}
    \begin{aligned}
      \includegraphics[page=2]{output/thm__concrete_category_function_invertibility}
    \end{aligned}
  \end{equation}

  In particular,
  \begin{equation*}
    U(f) \bincirc U(\widetilde{\id_X}) = U(f) \bincirc U(\widetilde{g}),
  \end{equation*}
  and thus \( U(\widetilde{\id_X}) = U(\widetilde{g}) \) and \( U(\id_X) = g \). But this contradicts our assumption that \( g \) differs from the identity by construction. Therefore, \( U(f) \) must be an injective function.

  \SubProofOf{thm:concrete_category_function_invertibility/epi_is_surjection} Dually, let \( F \) be a right adjoint to \( U \). It follows from \fullref{thm:category_adjunction_duality}, \fullref{thm:categorical_principle_of_duality/morphism_invertibility} and \fullref{thm:concrete_category_function_invertibility/mono_is_injection} that every epimorphism is a surjection.

  We will also give a direct proof. Since \( F \) is right adjoint, we have the following \hyperref[rem:universal_mapping_property]{universal mapping property}:
  \begin{displayquote}
    For every object \( X \) in \( \cat{C} \) and every function \( h: A \to U(X) \) in \( \cat{Set} \), there exists a unique morphism \( \widetilde{h}: F(A) \to X \) in \( \cat{C} \) such that the following diagram commutes:
    \begin{equation}\label{eq:thm:concrete_category_function_invertibility/right_universal_diagram}
      \begin{aligned}
        \includegraphics[page=3]{output/thm__concrete_category_function_invertibility}
      \end{aligned}
    \end{equation}
  \end{displayquote}

  Suppose that \( f \) is an epimorphism and that there exists some value \( y_0 \in U(Y) \) not in the image of \( f \). As in our proof of \fullref{thm:function_invertibility_categorical/right_cancellative}, fix some set \( z \) not in \( U(Y) \), define \( S \coloneqq U(Y) \cup \set{ z } \) and define \( g: U(Y) \to S \) by swapping \( y_0 \) and \( z \) in the identity \( \id_Z \). Let \( s: U(Y) \to S \) be the restriction of \( \id_Z \) to \( U(Y) \).

  By the property \eqref{eq:thm:concrete_category_function_invertibility/right_universal_diagram}, the following diagram commutes:
  \begin{equation}\label{eq:thm:concrete_category_function_invertibility/right_diagram}
    \begin{aligned}
      \includegraphics[page=4]{output/thm__concrete_category_function_invertibility}
    \end{aligned}
  \end{equation}

  In particular,
  \begin{equation*}
  U(\widetilde{\id_Z}) \bincirc U(f) = U(\widetilde{h}) \bincirc U(f),
  \end{equation*}
  and thus \( U(\widetilde{\id_Z}) = U(\widetilde{h}) \) and \( \id_Z = h \). But this contradicts our construction of \( h \). Therefore, \( U(f) \) must be a surjective function.
\end{proof}
