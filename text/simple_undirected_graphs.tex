\section{Simple undirected graphs}\label{sec:simple_undirected_graphs}

\paragraph{Basic graphs}

\begin{definition}\label{def:edgeless_graph}\mimprovised
  For a set \( S \), we define the \term{edgeless graph} \( \oline{K_S} \) as, unsurprisingly, the simple undirected graph with no edges whose vertices are \( S \).
\end{definition}
\begin{comments}
  \item See \fullref{rem:universal_graph_cardinals} regarding the case where \( S \) is a \hyperref[def:cardinal]{cardinal number}.

  \item Following \incite[16]{Harary1969GraphTheory}, we use the notation \( \oline{K_S} \). This is justified by edgeless graphs being \hyperref[def:graph_complement]{complements} of \hyperref[def:complete_graph]{complete graphs}, which are in turn denoted by \( K_n \). We find it simpler to define them the other way around.

  \item For \( \oline{K_n} \), \incite[18]{Bollobás1998ModernGraphTheory} prefers the term \enquote{empty graph}, while \incite[def. 1.3.1]{Knauer2019AlgebraicGraphTheory} uses \enquote{totally disconnected} and \enquote{discrete graph}. \incite[16]{Harary1969GraphTheory} also uses \enquote{totally disconnected}.
\end{comments}

\begin{remark}\label{rem:universal_graph_cardinals}
  In the special case where \( S \) is a \hyperref[def:cardinal]{cardinal number} \( \rho \), the vertices of \( \oline{K_\rho} \) are simply the smaller ordinals. This is discussed in \fullref{rem:cardinal_colorings} regarding \( \rho \)-colorings.

  An inconsistency arises --- we prefer indexing our vertices via positive integers, i.e. we would prefer a \( r \)-coloring with \( 1, \ldots, r \) rather than \( 0, \ldots, r - 1 \). Fortunately, graphs are generally considered up to an isomorphism, so don't have to bother too much with concrete choices for the vertices.
\end{remark}

\begin{theorem}[Edgeless graph universal property]\label{thm:edgeless_graph_universal_property}
  Given a set \( S \), the \hyperref[def:edgeless_graph]{edgeless graph} \( \oline{K_S} \) is the unique up to a unique isomorphism simple undirected graph that satisfies the following \hyperref[rem:universal_mapping_property]{universal mapping property}:
  \begin{displayquote}
    For every simple undirected graph \( G = (V, E) \) and every function \( f: S \to V \), there exists a unique \hyperref[def:undirected_graph/homomorphism]{homomorphism} \( \widetilde{f}: \oline{K_S} \to G \) such that the following diagram commutes:
    \begin{equation}\label{eq:thm:edgeless_graph_universal_property/diagram}
      \begin{aligned}
        \includegraphics[page=1]{output/thm__edgeless_graph_universal_property}
      \end{aligned}
    \end{equation}
  \end{displayquote}
\end{theorem}
\begin{comments}
  \item Via \fullref{rem:universal_mapping_property}, \( V \mapsto \oline{K_V} \) becomes \hyperref[def:category_adjunction]{left adjoint} to the \hyperref[def:concrete_category]{forgetful functor} from simple undirected graphs to sets.

  \Fullref{thm:first_order_categorical_invertibility/injective} then implies that the \hyperref[def:morphism_invertibility/left_cancellative]{categorical monomorphisms} are precisely the injective graph homomorphisms.
\end{comments}
\begin{proof}
  Since \( \oline{K_S} \) has no edges, \( f \) vacuously preserves all edges of \( \oline{K_V} \), and is thus the desired homomorphism.
\end{proof}

\begin{definition}\label{def:graph_complement}\mcite[def. 5.1.1]{Knauer2019AlgebraicGraphTheory}
  We define the \term[ru=дополнение (\cite[15]{ЕмеличевИПр1990ТеорияГрафов})]{complement} of the \hyperref[def:undirected_graph]{simple undirected graphs} \( G = (V, E) \) as
  \begin{equation*}
     G^{\complement} \coloneqq (V, V^2 \setminus E).
  \end{equation*}
\end{definition}

\begin{definition}\label{def:complete_graph}\mimprovised
  For every set \( V \), we define the \term[bg=пълен граф (\cite[12]{Мирчев2001Графи}), ru=полный граф (\cite[10]{ЕмеличевИПр1990ТеорияГрафов}), en=complete graph (\cite[def. 1.3.1]{Knauer2019AlgebraicGraphTheory})]{complete graph} \( K_V \) as the \hyperref[def:graph_complement]{complement} of the \hyperref[def:edgeless_graph]{edgeless graph} \( \oline{K_V} \) on \( V \).

  \begin{figure}[!ht]
    \begin{equation}\label{eq:fig:def:complete_graph/k4}
      \begin{aligned}
        \includegraphics[page=1]{output/def__complete_graph}
      \end{aligned}
    \end{equation}
    \caption{The \hyperref[def:complete_graph]{complete graph} \( K_4 \).}\label{fig:def:complete_graph/k4}
  \end{figure}
\end{definition}
\begin{comments}
  \item See \fullref{rem:universal_graph_cardinals} regarding the case where \( S \) is a \hyperref[def:cardinal]{cardinal number}.

  \item An \hyperref[def:multigraph_orientation]{orientation} of a complete graph is sometimes called a \enquote{tournament} --- for example by \incite[319]{Diestel2017GraphTheory}, \incite[30]{Bollobás1998ModernGraphTheory} and \incite[\S 7.3.3]{Новиков2013ДискретнаяМатематика} (as \enquote{турнир}).
\end{comments}

\begin{proposition}\label{thm:complete_graph_edge_count}
  The number of edges in the \hyperref[def:complete_graph]{complete graph} \( K_n \), and hence the maximum number of edges in a simple undirected graph of order \( n \), is \( \binom n 2 \).
\end{proposition}
\begin{proof}
  The cases where \( n < 2 \) are trivial. We use induction on \( n \geq 2 \) to show that the number of edges is \( \sum_{k=0}^{n-1} k \), which by \fullref{thm:numeric_arithmetic_progression_partial_sums} equals \( \binom n 2 \).

  \begin{itemize}
    \item If \( n = 2 \), \( K_n \) has only one edge.
    \item If \( K_n \) has \( \sum_{k=0}^{n-1} k \) edges, then \( K_{n+1} \) has \( \sum_{k=0}^n k \) edges since it has additional vertex compared to \( K_n \) and \( n \) additional edges connecting that vertex to all others.
  \end{itemize}
\end{proof}

\begin{theorem}[Complete graph universal property]\label{thm:complete_graph_universal_property}
  Given a set \( S \), the \hyperref[def:complete_graph]{complete graph} \( K_S \) is the unique up to a unique isomorphism simple undirected graph that satisfies the following \hyperref[rem:universal_mapping_property]{universal mapping property}:
  \begin{displayquote}
    For every simple undirected graph \( G = (V, E) \) and every function \( f: V \to S \), there exists a unique \hyperref[def:undirected_graph/homomorphism]{homomorphism} \( \widetilde{f}: G \to K_S \) such that the following diagram commutes:
    \begin{equation}\label{eq:thm:complete_graph_universal_property/diagram}
      \begin{aligned}
        \includegraphics[page=1]{output/thm__complete_graph_universal_property}
      \end{aligned}
    \end{equation}
  \end{displayquote}
\end{theorem}
\begin{comments}
  \item Via \fullref{rem:universal_mapping_property}, \( V \mapsto K_V \) becomes \hyperref[def:category_adjunction]{right adjoint} to the \hyperref[def:concrete_category]{forgetful functor} from simple undirected graphs to sets.

  \Fullref{thm:first_order_categorical_invertibility/surjective} then implies that the \hyperref[def:morphism_invertibility/right_cancellative]{categorical epimorphism} are precisely the surjective graph homomorphisms.
\end{comments}
\begin{proof}
  For every edge \( \set{ u, v } \) of \( G \), \( \set{ f(u), f(v) } \) is an edge of \( K_S \). Therefore, \( f \) is itself the desired homomorphism.
\end{proof}

\begin{definition}\label{def:complete_subgraph}\mimprovised
  We say that the \hyperref[def:undirected_graph/subgraph]{subgraph} \( H \) of the \hyperref[def:undirected_graph]{simple undirected graph} \( G \) is \term{complete} if any of the following equivalent conditions hold:
  \begin{thmenum}
    \thmitem{def:complete_subgraph/direct} There is an edge in \( H \) between any pair of vertices in \( H \).

    \thmitem{def:complete_subgraph/isomorphic} \( H \) is \hyperref[def:undirected_graph/homomorphism]{isomorphic} to the \hyperref[def:complete_graph]{complete graph} \( K_\rho \), where \( \rho \) is the (cardinal) \hyperref[def:graph_cardinality/order]{order} of \( H \).

    \thmitem{def:complete_subgraph/degree} \( H \) has \( \rho + 1 \) vertices and each of them has \hyperref[def:graph_cardinality/undirected_degree]{degree} \( \rho \) (in \( H \)).
  \end{thmenum}
\end{definition}
\begin{comments}
  \item The term \enquote{clique} is used as a synonym for complete subgraph by \incite[def. 1.4.11]{Knauer2019AlgebraicGraphTheory}, \incite[5]{GondranMinoux1984GraphsAndAlgorithms}, \incite[394]{Erickson2019Algorithms}, \incite[111]{ЕмеличевИПр1990ТеорияГрафов} (as \enquote{клика}) and \incite[\S 7.3.1]{Новиков2013ДискретнаяМатематика} (as \enquote{клика}). Other authors use it to mean a maximal complete subgraph, for example \incite[20]{Harary1969GraphTheory}, \incite[112]{Bollobás1998ModernGraphTheory} and \incite[103]{Мирчев2001Графи} (as \enquote{клика}).

  To prevent ambiguity, we avoid the term altogether.
\end{comments}
\begin{proof}
  \ImplicationSubProof{def:complete_subgraph/direct}{def:complete_subgraph/isomorphic} Let \( \seq{ H_\alpha }_{\alpha < \rho} \) be an enumeration of the vertices of \( H \). \Fullref{thm:complete_graph_universal_property} shows that this (transfinite) sequence is itself a homomorphism from \( H \) to \( K_\rho \). Furthermore, its inverse is also a homomorphism because \( H \) has an edge between every pair of vertices. Therefore, \( H \) and \( K_\rho \) are isomorphic.

  \ImplicationSubProof{def:complete_subgraph/isomorphic}{def:complete_subgraph/degree} Fix an isomorphism \( f: V_H \to V_{K_{\rho + 1}} \). Fix any vertex \( v \) in \( H \). Except for \( f(v) \), all \( \rho \) other vertices in \( K_{\rho + 1} \) are adjacent to \( f(v) \), thus the degree of \( f(v) \) is \( \rho \).

  Given a vertex \( k \) of \( K_{\rho + 1} \) adjacent to \( f(v) \), \( f^{-1}(k) \) is adjacent to \( v \) as per \eqref{eq:thm:graph_isomorphisms/simple_undirected}. Therefore, \( v \) has degree \( \rho \) in \( H \).

  Generalizing on \( v \), we conclude that every vertex of \( H \) has degree \( \rho \) in \( H \).

  \ImplicationSubProof{def:complete_subgraph/degree}{def:complete_subgraph/direct} Suppose that \( H \) has \( \rho + 1 \) vertices, each of degree \( \rho \).

  Fix a vertex \( v \) in \( H \). Since \( G \) does not allow loops or parallel edges, neither does \( H \), and thus \( v \) must have \( \rho \) adjacent vertices from \( H \). But there are \( \rho \) vertices in \( H \) distinct from \( v \), thus \( v \) is adjacent in \( H \) to any other vertex of \( H \).

  Since \( v \) was arbitrary, we conclude that \( H \) has an edge between every pair of vertices.
\end{proof}

\begin{definition}\label{def:path_graph}\mcite[def. 1.3.1]{Knauer2019AlgebraicGraphTheory}
  Given every positive integer \( n \), we regard it as the set of smaller cardinals and define the \term[en=path (\cite[4]{Bollobás1998ModernGraphTheory})]{path graph} \( P_n \) as the \hyperref[def:undirected_graph]{simple undirected graph}
  \begin{equation*}
    P_n \coloneqq \parens[\Big]{ n, \set[\Big]{ \set{ k, k + 1 } \given* 1 \leq k < n - 1 } }.
  \end{equation*}

  \begin{figure}[!ht]
    \begin{equation}\label{eq:fig:def:path_graph/p4}
      \begin{aligned}
        \includegraphics[page=1]{output/def__path_graph}
      \end{aligned}
    \end{equation}
    \caption{The \hyperref[def:path_graph]{path graph} \( P_4 \).}\label{fig:def:path_graph/p4}
  \end{figure}
\end{definition}

\begin{definition}\label{def:cycle_graph}\mcite[example 10.2.6]{Rosen2019DiscreteMathematics}
  For every integer \( n > 2 \), we define \term[en=cycle]{cycle graph} \( C_n \) as \hyperref[def:path_graph]{\( P_n \)} with an additional edge connecting \( n - 1 \) to \( 0 \).

  \begin{figure}[!ht]
    \begin{equation}\label{eq:fig:def:cycle_graph/c12}
      \begin{aligned}
        \includegraphics[page=1]{output/def__cycle_graph}
      \end{aligned}
    \end{equation}
    \caption{The \hyperref[def:cycle_graph]{cycle graph} \( C_{12} \).}\label{fig:def:cycle_graph/c12}
  \end{figure}
\end{definition}
\begin{comments}
  \item As shown in \fullref{thm:cayley_graph_of_finite_cyclic_group}, the \hyperref[def:cayley_graph]{Cayley graph} of a finite \hyperref[def:cyclic_group]{cyclic group} with a singleton connection set is a cycle graph.
\end{comments}

\begin{definition}\label{def:triangle_graph}\mcite[3]{Diestel2017GraphTheory}
  We call the graph \( K_3 = C_3 \) the \term{triangle graph} and any \hyperref[def:complete_subgraph]{complete subgraph} of order \( 3 \) --- a \term{triangle subgraph}.
\end{definition}

\paragraph{Combining graphs}

\begin{definition}\label{def:graph_disjoint_union}\mimprovised
  We define the \term[ru=дизъюнктное объедиение (графов) (\cite[19]{ЕмеличевИПр1990ТеорияГрафов})]{disjoint union} of the family of \hyperref[def:undirected_graph]{simple undirected graphs} \( \seq{ G_k }_{k \in \mscrK} \), where \( G_k = (V_k, E_k) \), as the graph
  \begin{equation*}
    \coprod_{k \in \mscrK} G_k \coloneqq \parens[\Big]{ \coprod_{k \in \mscrK} V_k, \bigcup_{k \in \mscrK} \set[\Big]{ \iota_k^E(e) \given* e \in E_k } },
  \end{equation*}
  where \( \iota_k^V \) is the canonical inclusion from \fullref{def:disjoint_union} and
  \begin{equation*}
    \iota_k^E(\set{ u, v }) \coloneqq \set{ \iota_k^V(u), \iota_k^V(v) }.
  \end{equation*}
\end{definition}
\begin{comments}
  \item This definition can be adjusted straightforwardly to other kinds of graphs, however we will not make use of that.
  \item \incite[21]{Harary1969GraphTheory}, \incite[def. 4.1.1]{Knauer2019AlgebraicGraphTheory}, \incite[def. 10.2.9]{Rosen2019DiscreteMathematics} and \incite[23]{Зыков2004ТеорияГрафов} use ordinary unions with the convention that the vertex sets and edge sets are disjoint. \incite[19]{ЕмеличевИПр1990ТеорияГрафов} also use regular unions, however they explicitly refer to unions as \enquote{disjoint} (\enquote{дизъюнктное объединение}) if the vertex sets and edge sets are disjoint.
\end{comments}

\begin{proposition}\label{thm:undirected_graph_coproduct}
  The \hyperref[def:discrete_category_limits]{categorical coproduct} of a family of graphs in the category of \hyperref[def:undirected_graph]{simple undirected graphs} is their \hyperref[def:graph_disjoint_union]{disjoint union}.
\end{proposition}
\begin{proof}
  Let \( (H, \alpha) \) be a \hyperref[def:category_of_cones/cocone]{cocone} for the discrete diagram \( \seq{ G_k }_{k \in \mscrK} \). We want to define a graph homomorphism \( l: \coprod_{k \in \mscrK} G_k \to H \) such that, for every \( m \in \mscrK \),
  \begin{equation*}
    \alpha_m(v) = l_A(\iota^V_m(v)).
  \end{equation*}

  This suggests the definition
  \begin{equation*}
    l_A\parens[\Big]{ \iota^V_m(v) } \coloneqq \alpha_k(v).
  \end{equation*}
\end{proof}

\begin{definition}\label{def:graph_join}\mcite[def. 4.1.4]{Knauer2019AlgebraicGraphTheory}
  We define the \term[ru=соединение (графов) (\cite[265]{Новиков2013ДискретнаяМатематика})]{graph join} \( G \Ast H \) of the \hyperref[def:undirected_graph]{simple undirected graphs} \( G \) and \( H \) as the \hyperref[def:graph_disjoint_union]{disjoint union} \( G \coprod H \) with an additional edge between every vertex from \( G \) and every vertex from \( H \).
\end{definition}
\begin{comments}
  \item \incite[21]{Harary1969GraphTheory} attributes the introduction of graph joins to \cite[164]{Зыков1949СоединенияГрафов}. Zykov calls them \enquote{произведения} (\enquote{products}) and denotes them via juxtaposition, however his paper generally uses arcane terminology like \enquote{complex} instead of \enquote{graph}. In a later book, \cite[23]{Зыков2004ТеорияГрафов}, Zykov uses the same term, but translates it as \enquote{join}.

  \incite[def. 4.1.4]{Knauer2019AlgebraicGraphTheory} uses both our terminology and our notation. \incite[4]{Diestel2017GraphTheory} uses \enquote{\( \Ast \)}, however avoids introducing a special term. The term \enquote{term} is used by \incite[21]{Harary1969GraphTheory} and \incite[\S 7.3.1]{Новиков2013ДискретнаяМатематика} (as \enquote{соединение}), however the latter two authors denote this operation by \enquote{\( + \)}.
\end{comments}

\begin{definition}\label{def:wheel_graph}\mcite[65]{Diestel2017GraphTheory}
  For a positive integer \( n \), we define the \term{wheel graph} \( W_n \) as the \hyperref[def:graph_join]{join} of the \hyperref[def:cycle_graph]{cycle graph} \( C_n \) and the single-point graph \( K_1 \).

  \begin{figure}[!ht]
    \begin{equation}\label{eq:fig:def:wheel_graph/w12}
      \begin{aligned}
        \includegraphics[page=1]{output/def__wheel_graph}
      \end{aligned}
    \end{equation}
    \caption{The \hyperref[def:wheel_graph]{wheel graph} \( W_{12} \).}\label{fig:def:wheel_graph/c12}
  \end{figure}
\end{definition}

\begin{definition}\label{def:graph_box_product}\mcite[def. 4.3.1]{Knauer2019AlgebraicGraphTheory}
  We define the \term{box product} \( G \boxprod H \) of two \hyperref[def:undirected_graph]{simple undirected graphs} \( G \) and \( H \) as the graph whose vertex set is the \hyperref[def:cartesian_product]{Cartesian product} \( V_G \times V_H \) and whose edge set is
  \begin{equation}\label{eq:def:graph_box_product/edges}
    \set[\Big]{ \set{ (u, v), (u, v') } \given \set{ v, v' } \in E_H } \cup \set[\Big]{ \set{ (u, v), (u', v) } \given \set{ u, u' } \in E_G }.
  \end{equation}

  We can recursively extend this to form \( n \)-fold box products:
  \begin{equation}\label{eq:def:graph_box_product/recursive}
    G^{\boxprod n} \coloneqq \begin{cases}
      K_1                           &n = 0 \\
      G^{\boxprod n - 1} \boxprod G &n > 1.
    \end{cases}
  \end{equation}
\end{definition}
\begin{comments}
  \item For complete graphs \eqref{eq:def:graph_box_product/edges} simplifies considerably --- see \fullref{thm:box_product_of_complete_graphs}.
\end{comments}

\begin{proposition}\label{thm:box_product_of_complete_graphs}
  In the case of \hyperref[def:complete_graph]{complete graphs} \( K_A \) and \( K_B \), two vertices in their \hyperref[def:graph_box_product]{box product} \( K_A \boxprod K_B \) are adjacent if and only if either their first or second components match (but not both).
\end{proposition}
\begin{proof}
  We simply have to note that the condition for \( v \) and \( w \) to be adjacent in \( K_A \) is equivalent to \( v \neq w \), which simplifies \eqref{eq:def:graph_box_product/edges} to
  \begin{equation*}
    \set[\Big]{ \set{ (u, v), (u, v) } \given v \neq v' } \cup \set[\Big]{ \set{ (u, v), (u', v) } \given u \neq u' }.
  \end{equation*}
\end{proof}

\begin{definition}\label{def:hypercube_graph}\mcite[def. 4.3.4]{Knauer2019AlgebraicGraphTheory}
  We define the \term[en=hypercube (\cite[example 10.2.8]{Rosen2019DiscreteMathematics}) / cube (\cite[def. 4.3.4]{Knauer2019AlgebraicGraphTheory})]{hypercube graph} \( Q_n \) as the \( n \)-fold \hyperref[def:graph_box_product]{box product} of \( K_2 \).

  We refer to \( Q_2 \) as a \term{square} and to \( Q_3 \) as a \term{cube}.

  \begin{figure}[!ht]
    \begin{equation}\label{eq:fig:def:hypercube_graph/q3}
      \begin{aligned}
        \includegraphics[page=1]{output/def__hypercube_graph}
      \end{aligned}
    \end{equation}
    \caption{The \hyperref[def:hypercube_graph]{cube graph} \( Q_3 \) labeled as per \fullref{thm:hypercube_graphs_as_strings}.}\label{fig:def:hypercube_graph/q3}
  \end{figure}
\end{definition}
\begin{comments}
  \item Other graphs isomorphic to \( Q_2 \) are also referred to as \enquote{square graphs} --- see \fullref{thm:square_graphs}.
  \item We can characterize \( Q_n \) via \hyperref[def:bit_string]{bit strings} --- see \fullref{thm:hypercube_graphs_as_strings}.
\end{comments}

\begin{proposition}\label{thm:hypercube_graphs_as_strings}
  The \hyperref[def:hypercube_graph]{hypercube graph} \( Q_n \) is \hyperref[def:undirected_graph/homomorphism]{isomorphic} to the graph whose vertices are \hyperref[def:bit_string]{bit strings} of length \( n \), with two strings adjacent if their \hyperref[def:hamming_distance]{Hamming distance} is \( 1 \).
\end{proposition}
\begin{proof}
  Denote by \( B_n \) the bit string graph described above.

  We will use induction on \( n \) to show that \( Q_n \) and \( B_n \) are isomorphic. The base case \( n = 0 \) is obvious --- there is only one string of length zero, and it is not connected to itself.

  Suppose that the proposition holds for \( n - 1 \) and consider \( Q_n \), which is, by definition, \( Q_{n-1} \boxprod K_2 \).

  By the inductive hypothesis, there exists an isomorphism \( f: V_{Q_{n-1}} \to V_{B_{n-1}} \). Since the vertices of \( K_2 \) are \( 0 \) and \( 1 \), we can extend \( f \) by concatenation:
  \begin{equation*}
    \begin{aligned}
      &\widehat f: V_{Q_n} \to V_{B_n}, \\
      &\widehat f(v, k) \coloneqq f(v) \cdot k.
    \end{aligned}
  \end{equation*}

  We must now show that \( \widehat f \) is an isomorphism.

  \begin{itemize}
    \item First suppose that there is an edge in \( Q_n \) between \( (v, k) \) and \( (w, m) \).
    \begin{itemize}
      \item If \( v = w \), then \( f(v) = f(w) \) and thus \( k \neq m \), implying that the Hamming distance of \( \widehat f(v, k) \) and \( \widehat f(w, m) \) is exactly \( 1 \).

      \item Otherwise, \( k = m \), and the inductive hypothesis implies that the Hamming distance of \( f(v) \) and \( f(w) \) is \( 1 \).
    \end{itemize}

    In both cases we obtain that \( \widehat f(v, n) \) and \( \widehat f(w, m) \) are adjacent in \( B_n \).

    \item Conversely, suppose that there is an edge in \( B_n \) between \( \widehat f(v, n) \) and \( \widehat f(w, m) \).
    \begin{itemize}
      \item If \( k = m \), then \( f(v) \) and \( f(w) \) have Hamming distance \( 1 \), thus they are adjacent in \( B_{n-1} \). By the inductive hypothesis, \( v \) and \( w \) are adjacent in \( Q_{n-1} \), and hence \( (v, k) \) and \( (w, m) \) are adjacent in \( Q_n \).

      \item Otherwise, \( f(v) = f(w) \), and \( v = w \) since \( f \) is injective. Then \( (v, k) \) and \( (w, m) \) are adjacent in \( Q_{n-1} \).
    \end{itemize}
  \end{itemize}

  This completes the proof.
\end{proof}

\paragraph{Generalized Petersen graphs}

\begin{definition}\label{def:petersen_graph}\mcite[2]{Watkins1969GeneralizedPetersenGraphs}
  For positive integers \( m < n \), the \term{generalized Petersen graph} \( P_{n,m} \) is the \hyperref[def:undirected_graph]{simple undirected graph} constructed as follows:
  \begin{thmenum}
    \thmitem{def:petersen_graph/union} We take the \hyperref[def:graph_disjoint_union]{disjoint union} \( \oline{K_n} \coprod C_n \).
    \thmitem{def:petersen_graph/inter} For each \( k = 0, \ldots, n - 1 \), we connect \( \iota_{\oline{K_n}}(k) \) to \( \iota_{C_n}(k) \).
    \thmitem{def:petersen_graph/inner} For each \( k = 0, \ldots, n - 1 \), we connect \( \iota_{\oline{K_n}}(k) \) to \( \iota_{\oline{K_n}}(\rem(k + m, n)) \).
  \end{thmenum}

  The usual \term{Petersen graph} is \( P_{5,2} \).

  \begin{figure}[!ht]
    \begin{subcaptionblock}{0.45\textwidth}
      \centering
      \includegraphics[page=1]{output/def__petersen_graph__p52}
      \caption{\( P_{5,2} \)}\label{fig:def:petersen_graph/p52}
    \end{subcaptionblock}
    \hfill
    \begin{subcaptionblock}{0.45\textwidth}
      \centering
      \includegraphics[page=1]{output/def__petersen_graph__p73}
      \caption{\( P_{7,3} \)}\label{fig:def:petersen_graph/p73}
    \end{subcaptionblock}
    \par
    \begin{subcaptionblock}{0.45\textwidth}
      \centering
      \includegraphics[page=1]{output/def__petersen_graph__p31}
      \caption{\( P_{3,1} \)}\label{fig:def:petersen_graph/p31}
    \end{subcaptionblock}
    \hfill
    \begin{subcaptionblock}{0.45\textwidth}
      \centering
      \includegraphics[page=1]{output/def__petersen_graph__p21}
      \caption{\( P_{2,1} \)}\label{fig:def:petersen_graph/p21}
    \end{subcaptionblock}
    \caption{\hyperref[def:petersen_graph]{Generalized Petersen graphs}.}\label{fig:def:petersen_graph}
  \end{figure}
\end{definition}

\begin{proposition}\label{thm:petersen_graph_cardinality}
  The \hyperref[def:petersen_graph]{generalized Petersen graph} \( P_{n,m} \) has \( 2n \) vertices and \( 3n \) edges.
\end{proposition}
\begin{proof}
  We have \( 2n \) vertices and \( n \) edges by \fullref{def:petersen_graph/union}. \Fullref{def:petersen_graph/inter} and \fullref{def:petersen_graph/inter} each add \( n \) edges.
\end{proof}

\begin{definition}\label{def:bridgeless_graph}\mimprovised
  We say that a graph is \term{bridgeless} if it has no \hyperref[def:graph_bridge]{bridge}.
\end{definition}

\begin{proposition}\label{thm:petersen_graph_bridgeless}
  Every \hyperref[def:petersen_graph]{generalized Petersen graph} is \hyperref[def:bridgeless_graph]{bridgeless}.
\end{proposition}
\begin{proof}
  Fix an edge \( e \) in \( P_{n,m} \). We will show that \( e \) is necessarily contained in a cycle, and thus cannot be a bridge.

  \begin{figure}[!ht]
    \begin{subcaptionblock}{0.3\textwidth}
      \centering
      \includegraphics[page=1]{output/thm__petersen_graph_bridgeless}
    \end{subcaptionblock}
    \hfill
    \begin{subcaptionblock}{0.3\textwidth}
      \centering
      \includegraphics[page=2]{output/thm__petersen_graph_bridgeless}
    \end{subcaptionblock}
    \hfill
    \begin{subcaptionblock}{0.3\textwidth}
      \centering
      \includegraphics[page=3]{output/thm__petersen_graph_bridgeless}
    \end{subcaptionblock}
    \caption{The three possibilities in the proof of \fullref{thm:petersen_graph_bridgeless}.}\label{fig:thm:petersen_graph_bridgeless/proof}
  \end{figure}

  \begin{itemize}
    \item If \( e \) is added during \fullref{def:petersen_graph/union}, it is an edge coming from \( C_n \), which induces a cycle on both endpoints of \( e \).

    \item If \( e \) is added during \fullref{def:petersen_graph/inter}, one of its endpoints comes from \( \oline{K_n} \), which necessarily connects to another endpoint from \( \oline{K_n} \), which in turn connects to \( C_n \). We can then traverse \( C_n \) to reach the other endpoint of \( e \).

    \item If \( e \) is added during \fullref{def:petersen_graph/inner}, both of its endpoints are connected to \( C_n \), and we can traverse \( C_n \) to obtain a cycle.
  \end{itemize}
\end{proof}

\paragraph{Multipartite graphs}

\begin{definition}\label{def:multipartite_graph}\mcite[6]{Bollobás1998ModernGraphTheory}
  For a \hyperref[def:undirected_graph]{simple undirected graph} \( G = (V, E) \), we say that the \hyperref[def:set_partition]{partition} \( V_1, \ldots, V_r \) is an \term{\( r \)-partition} of \( V \) if the subsets \( V_1, \ldots, V_r \) are \hyperref[def:graph_independent_set]{independent}.

  If \( G \) has at least one \( r \)-partition, we say that it is \term[ru=\( r \)-дольный (граф) (\cite[11]{ЕмеличевИПр1990ТеорияГрафов})]{\( r \)-partite}.

  We use the term \term{bipartition} (resp. \term[ru=двудольный (граф) (\cite[11]{ЕмеличевИПр1990ТеорияГрафов})]{bipartite graph}) when \( r = 2 \) and tripartition (resp. \term[ru=трёхдольный (граф) (\cite[11]{ЕмеличевИПр1990ТеорияГрафов})]{tripartite graph}) when \( r = 3 \). When no concrete value of \( r \) is meant, we say that the graph is \term[en=multipartite (graph) (\cite[ex. 8.5.5]{Knauer2019AlgebraicGraphTheory})]{multipartite}.
\end{definition}
\begin{itemize}
  \item It is tempting to require the subsets of the partition to be nonempty, however many authors choose not to do so. This includes \incite[17]{Diestel2017GraphTheory}, \incite[27]{Harary1969GraphTheory}, \incite[6]{Bollobás1998ModernGraphTheory}, \incite[def. 1.3.1]{Knauer2019AlgebraicGraphTheory}, \incite[26]{GondranMinoux1984GraphsAndAlgorithms} and \incite[11]{ЕмеличевИПр1990ТеорияГрафов}. We follow their lead.
\end{itemize}

\begin{example}\label{ex:def:multipartite_graph}
  We list some examples of \hyperref[def:multipartite_graph]{multipartite graphs}:
  \begin{thmenum}
    \thmitem{ex:def:multipartite_graph/edgeless} \hyperref[def:edgeless_graph]{Edgeless graphs} are vacuously \( r \)-partite for any positive integer \( r \).

    \thmitem{ex:def:multipartite_graph/path} Every \hyperref[def:path_graph]{path graph} is bipartite. Indeed, no two even vertices and no two odd vertices are adjacent.
    \begin{figure}[!ht]
      \centering
      \includegraphics[page=1]{output/ex__def__multipartite_graph__path}
      \caption{A drawing of the \hyperref[def:path_graph]{path graph} \( P_6 \) highlighting that it is \hyperref[def:multipartite_graph]{bipartite}.}\label{fig:ex:def:multipartite_graph/path}
    \end{figure}

    \thmitem{ex:def:multipartite_graph/cycle} Every \hyperref[def:cycle_graph]{cycle graph} is tripartite, however some are also bipartite.

    \begin{figure}[!ht]
    \hfill
      \begin{subcaptionblock}{0.3\textwidth}
        \centering
        \includegraphics[page=1]{output/ex__def__multipartite_graph__cycle__c6_tripartite}
        \caption{\( C_6 \) is tripartite.}\label{fig:ex:def:multipartite_graph/cycle/c6_tripartite}
      \end{subcaptionblock}
      \hfill
      \begin{subcaptionblock}{0.3\textwidth}
        \centering
        \includegraphics[page=1]{output/ex__def__multipartite_graph__cycle__c6_bipartite}
        \caption{\( C_6 \) is also bipartite.}\label{fig:ex:def:multipartite_graph/cycle/c6_bipartite}
      \end{subcaptionblock}
      \hfill
      \begin{subcaptionblock}{0.3\textwidth}
        \centering
        \includegraphics[page=1]{output/ex__def__multipartite_graph__cycle__c5}
        \caption{\( C_5 \) is tripartite.}\label{fig:ex:def:multipartite_graph/cycle/c5}
      \end{subcaptionblock}
      \caption{Drawings of \hyperref[def:cycle_graph]{cycle graphs} highlighting their \hyperref[def:multipartite_graph]{multipartite} structure.}\label{fig:ex:def:multipartite_graph/cycle}
    \end{figure}

    Again, we split the odd and even vertices, however we must also acknowledge that the first and last vertex may have the same parity, forcing the last vertex into its own color class.
  \end{thmenum}
\end{example}

\begin{proposition}\label{thm:bipartite_iff_no_odd_cycles}\mcite[thm. 1.3.7]{Knauer2019AlgebraicGraphTheory}
  A \hyperref[def:undirected_graph]{simple undirected graph} is \hyperref[def:multipartite_graph]{bipartite} if and only if it has no \hyperref[def:graph_cycle]{cycles} of odd length.
\end{proposition}
\begin{proof}
  \SufficiencySubProof Let \( A \cup B \) be a bipartition of \( G = (V, E) \).

  Consider a cycle
  \begin{equation*}
    v_0 \to v_1 \to \cdots \to v_n.
  \end{equation*}

  Without loss of generality, suppose that \( v_0 \) is in \( A \). Then \( v_1 \) must be in \( B \). By induction, we conclude that \( v_k \) is in \( B \) if \( k \) is odd and in \( A \) if \( k \) is even. Since \( v_n = v_1 \) is in \( A \), we conclude that \( n \) is even.

  \NecessitySubProof Suppose that \( G = (V, E) \) has no cycles of odd length.

  If \( V \) is empty, \( G \) is vacuously bipartite. Suppose that \( V \) is nonempty. Let \( G_1, \ldots, G_s \) be the connected components of \( G \).

  Fix a vertex \( u \) from \( G_k \). Let \( A_k \) be the set of all vertices from \( G_k \) reachable from \( u \) with even-length paths and let \( B_k \) be the complement of \( A_k \) in \( V \). We will show that both \( A_k \) and \( B_k \) are independent.

  Aiming at a contradiction, suppose that there exist adjacent vertices \( v \) and \( w \) in \( A_k \). We can concatenate even-length paths from \( w \) to \( u \) and from \( u \) to \( v \) to obtain an even-length path from \( w \) to \( v \). Appending the edge \( \set{ v, w } \), we obtain an odd-length cycle at \( w \), whose existence contradicts our initial assumption. Hence, \( A_k \) is an independent set.

  Similarly, suppose that there exist adjacent vertices \( v \) and \( w \) in \( B_k \). Then we can concatenate odd-length paths from \( w \) to \( u \) and from \( u \) to \( v \) to obtain an even-length path from \( w \) to \( v \). Appending \( \set{ u, v } \), we again obtain an odd-length cycle at \( w \), whose existence again leads to a contradiction. Hence, \( B_k \) is also an independent set.

  We can now take the unions \( A \coloneqq A_1 \cup \cdots \cup A_s \) and \( B \coloneqq B_1 \cup \cdots \cup B_s \). These are again independent sets because \( A_i \) and \( A_j \) belong to different connected components if \( i \neq j \).

  Then \( V = A \cup B \) is the desired partition of \( V \), which demonstrates that \( G \) is bipartite.
\end{proof}

\begin{proposition}\label{thm:multipartite_graph_complete_subgraph}
  If \hyperref[def:undirected_graph]{simple undirected graph} \( G \) has a \hyperref[def:complete_subgraph]{complete subgraph} of order \( r + 1 \), then \( G \) is not \hyperref[def:multipartite_graph]{\( r \)-partite}.
\end{proposition}
\begin{comments}
  \item The converse does not hold --- see \fullref{ex:thm:multipartite_graph_complete_subgraph/converse}.

  \item A partial \hyperref[def:conditional_formula/inverse]{inverse proposition} that only regards the number of edges is \fullref{thm:turans_theorem}.
\end{comments}
\begin{proof}
  Let \( H \) be a complete subgraph of \( G \) of order \( r + 1 \).

  Let \( V_1, \ldots, V_r \) be \hi{any} partition of the vertices of \( G \). By \fullref{thm:pigeonhole_principle/simple}, there exists an index \( k \) such that \( V_k \) contains at least two vertices of \( H \). But then \( V_k \) is not an independent set.

  Therefore, \( G \) cannot possibly be \( r \)-partite.
\end{proof}

\begin{example}\label{ex:thm:multipartite_graph_complete_subgraph}
  We list several examples related to \fullref{thm:multipartite_graph_complete_subgraph}:
  \begin{thmenum}
    \thmitem{ex:thm:multipartite_graph_complete_subgraph/triangle} The graph \( C_4 \) is bipartite, as can be seen in \cref{fig:thm:square_graphs}.

    Consider the following supergraph of \( C_4 \):
    \begin{equation}\label{eq:ex:thm:multipartite_graph_complete_subgraph/triangle}
      \begin{aligned}
        \includegraphics[page=1]{output/ex__thm__multipartite_graph_complete_subgraph}
      \end{aligned}
    \end{equation}

    It has two \hyperref[def:triangle_graph]{triangle subgraphs}, hence it is not bipartite.

    \thmitem{ex:thm:multipartite_graph_complete_subgraph/converse} The converse of \fullref{thm:multipartite_graph_complete_subgraph} does not hold.

    As discussed in \fullref{ex:def:multipartite_graph/cycle}, the cycle graph \( K_5 \) is not bipartite. It does not contain \hyperref[def:triangle_graph]{triangles}, however, and the converse of \fullref{thm:multipartite_graph_complete_subgraph} would imply that it is bipartite.
  \end{thmenum}
\end{example}

\begin{definition}\label{def:complete_multipartite_graph}\mcite[33]{Harary1969GraphTheory}
  For a positive integer \( r \), we define the \term[ru=полный \( r \)-дольный (граф) (\cite[\S 1]{ЕмеличевИПр1990ТеорияГрафов})]{complete \( r \)-partite} graph \( K_{V_1,\ldots,V_r} \) as the \hyperref[def:graph_join]{graph join}
  \begin{equation*}
    \oline{K_{V_1}} \Ast \oline{K_{V_2}} \Ast \cdots \Ast \oline{K_{V_r}}
  \end{equation*}
  of \( r \) \hyperref[def:edgeless_graph]{edgeless graphs}.

  \begin{figure}[!ht]
    \begin{subcaptionblock}{0.45\textwidth}
      \centering
      \includegraphics[page=1]{output/def__complete_multipartite_graph__k32}
      \caption{The bipartite \( K_{3,2} \)}\label{fig:def:complete_multipartite_graph/k32}
    \end{subcaptionblock}
    \hfill
    \begin{subcaptionblock}{0.45\textwidth}
      \centering
      \includegraphics[page=1]{output/def__complete_multipartite_graph__k15}
      \caption{The \hyperref[def:star_graph]{star} \( K_{1,5} \)}\label{fig:def:complete_multipartite_graph/k15}
    \end{subcaptionblock}
    \caption{\hyperref[def:complete_multipartite_graph]{Complete multipartite graphs}.}\label{fig:def:complete_multipartite_graph}
  \end{figure}
\end{definition}
\begin{comments}
  \item If we plain unions rather than disjoint unions in the definition of graph joins, we would have to ensure that \( V_1, \ldots, V_r \) are disjoint. When \( n \) and \( m \) are positive integers, this would make the definition of \( K_{n,m} \) more tricky that it has to be.
\end{comments}

\begin{definition}\label{def:star_graph}\mcite[18]{Diestel2017GraphTheory}
  For a positive integer \( n \), we call the \hyperref[def:complete_multipartite_graph]{complete multipartite graph} \( K_{1,n} \) a \term{star graph}.
\end{definition}

\begin{proposition}\label{thm:complete_multipartite_subgraphs}
  A \hyperref[def:undirected_graph]{simple undirected graph} \( G = (V, E) \) is \hyperref[def:multipartite_graph]{\( r \)-partite} if and only if there exists a partition \( V_1, \ldots, V_r \) of \( V \) such that \( G \) is (isomorphic to) a subgraph of \( K_{V_1, \ldots, V_r} \).
\end{proposition}
\begin{proof}
  \SufficiencySubProof Let \( V_1, \ldots, V_r \) be a partition of \( V \) into independent subsets. Every edge \( e \) in \( E \) uniquely corresponds to an edge \( \iota^E(e) \) in \( K_{V_1, \ldots, V_r} \). Thus, the graph \( (\iota^V[V], \iota^E[E]) \) is a subgraph of \( K_{V_1, \ldots, V_r} \).

  \NecessitySubProof Conversely, if \( G \) is (isomorphic to) a subgraph of \( K_{V_1, \ldots, V_r} \), it must itself be \( r \)-partite because otherwise \( K_{V_1, \ldots, V_r} \) would not be.
\end{proof}

\begin{proposition}\label{thm:small_complete_multipartite_graph}
  The \hyperref[def:complete_graph]{complete graph} \( K_n \) is isomorphic to the \hyperref[def:complete_multipartite_graph]{complete multipartite graph}
  \begin{equation*}
    K_{\underbrace{1,\ldots,1}_{n \T*{times}}}.
  \end{equation*}
\end{proposition}
\begin{proof}
  An explicit \hyperref[def:undirected_graph/homomorphism]{isomorphism} is
  \begin{equation*}
    \begin{aligned}
      &f: K_n \to K_{1,\ldots,1} \\
      &f(k) \coloneqq \iota^V_k(0).
    \end{aligned}
  \end{equation*}
\end{proof}

\begin{proposition}\label{thm:square_graphs}
  The \hyperref[def:cycle_graph]{cycle graph} \( C_4 \) and the \hyperref[def:complete_multipartite_graph]{complete bipartite graph} \( K_{2,2} \) are \hyperref[def:undirected_graph/homomorphism]{isomorphic} to the \hyperref[def:hypercube_graph]{square graph} \( Q_2 \)


  \begin{figure}[!ht]
    \begin{subcaptionblock}{0.3\textwidth}
      \centering
      \includegraphics[page=1]{output/thm__square_graph__q2}
      \caption{\( Q_2 \)}\label{fig:thm:square_graphs/q2}
    \end{subcaptionblock}
    \hfill
    \begin{subcaptionblock}{0.3\textwidth}
      \centering
      \includegraphics[page=1]{output/thm__square_graph__c4}
      \caption{\( C_4 \)}\label{fig:thm:square_graphs/c4}
    \end{subcaptionblock}
    \hfill
    \begin{subcaptionblock}{0.3\textwidth}
      \centering
      \includegraphics[page=1]{output/thm__square_graph__k22}
      \caption{\( K_{2,2} \)}\label{fig:thm:square_graphs/k22}
    \end{subcaptionblock}
    \caption{An explicit isomorphism for the graphs in \fullref{thm:square_graphs}.}\label{fig:thm:square_graphs}
  \end{figure}
\end{proposition}
\begin{proof}
  An explicit isomorphism is illustrated in \cref{fig:thm:square_graphs}.
\end{proof}

\paragraph{Turan's theorem}

\begin{definition}\label{def:turan_graph}\mimprovised
  For any two integers \( r > 0 \) and \( n \geq 0 \), we define the \term{Turan graph} \( T_{n,r} \) as the \hyperref[def:complete_multipartite_graph]{complete \( r \)-partite graph}
  \begin{equation*}
    K_{\underbrace{q+1,\ldots,q+1}_{s \T*{times}},\underbrace{q,\cdots,q}_{n - s \T*{times}}},
  \end{equation*}
  where \( n = qr + s \) is the result of \fullref{alg:integer_division}.

  \begin{figure}[!ht]
    \begin{subcaptionblock}{0.45\textwidth}
      \centering
      \includegraphics[page=1]{output/def__turan_graph}
      \caption{\( K_5 \)}\label{fig:def:turan_graph/k5}
    \end{subcaptionblock}
    \hfill
    \begin{subcaptionblock}{0.45\textwidth}
      \centering
      \includegraphics[page=2]{output/def__turan_graph}
      \caption{\( T_{5,3} \)}\label{fig:def:turan_graph/t52}
    \end{subcaptionblock}
    \caption{An illustration of how \hyperref[def:turan_graph]{Turan graphs} are the closest multipartite analogues of \hyperref[def:complete_graph]{complete graphs}.}\label{fig:def:turan_graph}
  \end{figure}
\end{definition}
\begin{comments}
  \item We base our notation and terminology on \cite[175]{Diestel2017GraphTheory}, however Diestel defines the Turan graph \( T_{n,r} \) as \enquote{the unique} \( r \)-partite graph whose partition classes differ by at most \( 1 \) vertex. This only characterizes Turan graphs up to an \hyperref[thm:graph_isomorphisms/simple_undirected]{isomorphism}.
\end{comments}

\begin{proposition}\label{thm:small_turan_graph}
  If \( 0 \leq n \leq r \), the \hyperref[def:turan_graph]{Turan graph} \( T_{n,r} \) is \hyperref[def:undirected_graph/homomorphism]{isomorphic} to the \hyperref[def:complete_graph]{complete graph} \( K_n \).
\end{proposition}
\begin{proof}
  If \( n < r \), then \( n = 0 \cdot r + s \), thus
  \begin{equation*}
    T_{n,r} = K_{\underbrace{1,\ldots,1}_{s \T*{times}},\underbrace{0,\cdots,0}_{n - s \T*{times}}}.
  \end{equation*}

  Otherwise, if \( n = r \),then
  \begin{equation*}
    T_{n,r} = K_{\underbrace{1,\ldots,1}_{n \T*{times}}}.
  \end{equation*}

  In both cases, by \fullref{thm:small_complete_multipartite_graph}, \( T_{n,r} \) is isomorphic to \( K_n \).
\end{proof}

\begin{lemma}\label{thm:multipartite_graph_degree}
  In an \hyperref[def:multipartite_graph]{\( r \)-partite} graph with vertex partition \( V_1, \ldots, V_r \), for every vertex \( v \in V_k \) we have
  \begin{equation*}
    \deg(v) \leq \sum_{i \neq k} \card(V_i),
  \end{equation*}
  with equality holding for \hyperref[def:complete_multipartite_graph]{complete \( r \)-partite graphs}.
\end{lemma}
\begin{proof}
  Equality for complete \( r \)-partite graphs easily follows by induction. The general inequality then follows from \fullref{thm:complete_multipartite_subgraphs}.
\end{proof}

\begin{proposition}\label{thm:turan_graph_edge_count}
  The maximum number of edges in a \hyperref[def:multipartite_graph]{\( r \)-partite} graph of finite order \( n \) is
  \begin{equation}\label{eq:thm:turan_graph_edge_count}
    \frac {r - 1} r \cdot \frac {n^2 - s^2} 2 + \binom s 2,
  \end{equation}
  where \( n = qr + s \) is the result of \fullref{alg:integer_division}.

  Furthermore, equality holds in \eqref{eq:thm:turan_graph_edge_count} for the \hyperref[def:turan_graph]{Turan graph} \( T_{n,r} \).
\end{proposition}
\begin{comments}
  \item If \( n < r \), then \( n = r \) and \eqref{eq:thm:turan_graph_edge_count} reduces to \( \binom n 2 \), which by \fullref{thm:complete_graph_edge_count} is the edge count of \( K_n \).
\end{comments}
\begin{proof}
  Let \( V_1, \ldots, V_r \) be a partition of the graph \( G \) into independent sets. Denote by \( n_k \) the cardinality of \( V_k \) and by \( m \) --- the number of edges in \( G \).

  \Fullref{thm:sum_of_graph_degrees} implies that
  \begin{equation*}
    m = 2 \cdot \sum_{k=1}^r \sum_{v \in V_k} \deg(v).
  \end{equation*}

  For \( v \in V_k \), \fullref{thm:multipartite_graph_degree} implies that
  \begin{equation*}
    \deg(v) \leq \underbrace{\sum_{i \neq k} n_i}_{n - n_k}.
  \end{equation*}

  Thus,
  \begin{equation}\label{eq:thm:turan_graph_edge_count/proof/degree_ineq}
    m \leq \frac 1 2 \cdot \sum_{k=1}^r \underbrace{\sum_{v \in V_k} (n - n_k)}_{n_k (n - n_k)}.
  \end{equation}

  We want to maximize \( m \). \Fullref{thm:multipartite_graph_degree} implies that equality is attained in \eqref{eq:thm:turan_graph_edge_count/proof/degree_ineq} for complete multipartite graphs, thus \( G \) must be complete \( s \)-partite.

  It remains to determine values for \( n_1, \ldots, n_r \) for which \eqref{eq:thm:turan_graph_edge_count/proof/degree_ineq} is maximized.

  The total change in the number of edges after moving a vertex from \( V_i \) to \( V_j \) is
  \begin{align*}
    &\phantom{{}={}}
    \frac 1 2 \cdot \parens[\Big]{ (n_i - 1)(n - n_i + 1) - n_i(n - n_i) } + \frac 1 2 \cdot \parens[\Big]{ (n_j + 1)(n - n_j - 1) - n_j(n - n_j) }
    = \\ &=
    \frac 1 2 \cdot \parens[\Big]{ n_i(n - n_i) + n_i - n + n_i - 1 - n_i(n - n_i) } + \cdots
    = \\ &=
    \frac 1 2 \cdot \parens[\Big]{ 2n_i - n - 1 } + \frac 1 2 \cdot \parens[\Big]{ -2n_j + n - 1 }
    = \\ &=
    n_i - n_j - 1.
  \end{align*}

  Thus,
  \begin{itemize}
    \item \( m \) increases if \( n_i > n_j + 1 \).
    \item \( m \) decreases if \( n_i \leq n_j \).
    \item \( m \) does not change if \( n_i = n_j + 1 \).
  \end{itemize}

  We can thus conclude that the total number of edges can only increase if there exist \( i \) and \( j \) such that \( n_i > n_j + 1 \). Therefore, the Turan graph \( T_{n,r} \) has the maximal amount of edges possible for an \( r \)-partite graph.

  For \( T_{n,r} \), \eqref{eq:thm:turan_graph_edge_count/proof/degree_ineq} becomes
  \begin{equation}\label{eq:thm:turan_graph_edge_count/proof/degree_turan}
    m = \frac 1 2 \cdot \underbrace{\sum_{k=1}^s (q+1) (n - (q+1))}_{s(q+1) (n - q-1)} + \frac 1 2 \cdot \underbrace{\sum_{k=s+1}^r q (n - q)}_{(r-s) q (n - q)}.
  \end{equation}

  It can be simplified as follows:
  \begin{balign*}
    m
    &=
    \frac {s(q+1) (n - q-1) + (r-s) q (n - q)} 2
    = \\ &=
    \frac {sq(n-q-1) + s(n-q-1) + (r-s)q(n-q)} 2
    = \\ &=
    \frac {\hi{sq(n-q)} - sq + s(n-q-1) + \hi{(r-s)q(n-q)}} 2
    = \\ &=
    \frac {rq(n-q) + s(n-q) - s(q+1)} 2
    = \\ &=
    \frac {(rq+s)(n-q) - s(q+1)} 2
    = \\ &=
    \frac {n(n-q) - s(q+1)} 2.
  \end{balign*}

  Since \( q = (n - s) / r \), we can further simplify as follows:
  \begin{balign*}
    r
    &=
    \frac {n((r-1)n+s) - s(n-s+r)} {2r}
    = \\ &=
    \frac {(r-1)n^2 + \hi{sn} - \hi{sn} + s^2 - sr} {2r}
    = \\ &=
    \frac {(r-1)n^2 - (r-1-r) s^2 - sr} {2r}
    = \\ &=
    \frac {r-1} r \cdot \frac {n^2 - s^2} 2 - \frac {s^2 - s} 2,
  \end{balign*}
  which is precisely \eqref{eq:thm:turan_graph_edge_count}.
\end{proof}

\begin{corollary}\label{thm:turan_graph_edge_count_bound}
  The number of edges in a \hyperref[def:multipartite_graph]{\( r \)-partite} graph of finite order \( n \) cannot exceed
  \begin{equation}\label{eq:thm:turan_graph_edge_count_bound}
    \frac {r - 1} r \cdot \frac {n^2} 2,
  \end{equation}
  where \( n = qr + s \) is the result of \fullref{alg:integer_division}.
\end{corollary}
\begin{proof}
  \Fullref{thm:turan_graph_edge_count} implies that the maximum number of edges is
  \begin{equation*}
    \frac {r - 1} r \cdot \frac {n^2 - s^2} 2 + \binom s 2.
  \end{equation*}

  If we subtract this bound from \eqref{eq:thm:turan_graph_edge_count_bound}, we obtain
  \begin{equation*}
    d = \frac {r - 1} r \cdot \frac {s^2} 2 - \binom s 2.
  \end{equation*}

  It only remains to show that this difference is nonnegative:
  \begin{equation*}
    d = \frac s {2r} \parens[\Big]{ s(r - 1) - r(s - 1) } = \frac s {2r} (\underbrace{r - s}_{> 0}) \geq 0.
  \end{equation*}
\end{proof}

\begin{theorem}[Turan's theorem]\label{thm:turans_theorem}\mcite[thm. 7.1.1]{Diestel2017GraphTheory}
  For any two integers \( r > 0 \) and \( n \geq 0 \), if a \hyperref[def:undirected_graph]{simple undirected graph} with \( n \) vertices has no \hyperref[def:complete_subgraph]{complete subgraph} of order \( r + 1 \), then it cannot have more edges than the \hyperref[def:turan_graph]{Turan graph} \( T_{n,r} \).
\end{theorem}
\begin{comments}
  \item The exact bound on the number of edges is \eqref{eq:thm:turan_graph_edge_count}, although the upper bound \eqref{eq:thm:turan_graph_edge_count_bound} may be simpler to work with.

  \item The important thing to note is that, although the graph may not be \( r \)-partite, the number of edges has the same bound as a \( r \)-partite graph.
\end{comments}
\begin{proof}
  We will use induction on \( n \).

  \SubProof{Proof of base case} If \( n \leq r \), \fullref{thm:small_turan_graph} implies that \( T_{n,r} \) is isomorphic to \( K_n \).

  \Fullref{thm:complete_graph_edge_count} implies that any graph \( G \) of order \( n \) has its edge count is bounded by \( \binom n 2 \), which is precisely the edge count of \( T_{n,r} \).

  \SubProof{Proof of inductive step} Let \( n > r \) and suppose that the theorem holds for graphs of order less than \( n \). Let \( G = (V, E) \) be a graph of order \( n \) with no complete subgraph of order \( r + 1 \).

  Since no subgraph of \( G \) is isomorphic to \( K_{r+1} \), then no subgraph satisfies \fullref{def:complete_subgraph/degree}, and thus there exists a vertex of degree at most \( r - 1 \). Let \( v \) be such a vertex. Define
  \begin{equation*}
    G' \coloneqq \parens[\Big]{ V \setminus \set{ v }, E \setminus \set{ e \given v \in e } }.
  \end{equation*}

  Then \( G' \) is a graph of order \( n - 1 \) with no complete subgraph of order \( r + 1 \). Hence, the inductive hypothesis is applicable to \( G' \).
  \begin{itemize}
    \item If \( r \) divides \( n \), by inductive hypothesis, the bound on the edge count of \( G' \) is
    \begin{equation*}
      \frac {r - 1} r \cdot \frac {(n - 1)^2 - (s - 1)^2} 2 + \binom {s - 1} 2.
    \end{equation*}

    This can be expanded as follows:
    \begin{equation*}
      \underbrace{\frac {r - 1} r \cdot \frac {n^2} 2}_b + \underbrace{\frac {r - 1} {2r} \cdot (-2n + \cancel{1} - s^2 + 2s - \cancel{1}) + \binom {s - 1} 2}_e.
    \end{equation*}

    The highlighted part \( b \) is the bound we want to obtain on \( G \). We must thus show that \( e \) is at most \( -r + 1 \) because at most \( r - 1 \) edges are removed from \( G \). We have
    \begin{align*}
      e
      &=
      \frac {r - 1} {2r} \cdot (-2n - s^2 + 2s) + \binom {s - 1} 2
      = \\ &=
      -\frac {r - 1} {2r} \cdot \parens[\Big]{ s(s - 2) + 2n ) + \frac {(s - 1)(s - 2)} 2 }
      = \\ &=
      \cancel{-\frac {(s - 1)(s - 2)} 2} - \frac {s - 1} r n + \cancel{\frac {(s - 1)(s - 2)} 2}.
    \end{align*}

    Since \( r \) divides \( n \) and \( r < n \), \( n / r \) is an integer greater than \( 1 \), thus
    \begin{equation*}
      e = - \frac {s - 1} r n = - \frac n r (s - 1) < -(s - 1) = -s + 1.
    \end{equation*}

    \item If \( r \) does not divide \( n \), then the remainder \( s \) is positive and the bound on the edge count of \( G' \) is
    \begin{equation*}
      \frac {r - 1} r \cdot \frac {(n - 1)^2 - (s - 1)^2} 2 + \binom {s - 1} 2.
    \end{equation*}

    This can be expanded as follows:
    \begin{equation*}
      \underbrace{\frac {r - 1} r \cdot \frac {n^2 - s^2} 2 + \binom s 2}_b + \underbrace{\frac {r - 1} {2m} \cdot (-2n + \cancel{1} + 2s - \cancel{1}) - (s - 1)}_e.
    \end{equation*}

    Again, we must show that \( e \leq -m + 1 \). We have
    \begin{equation*}
      e = \frac {r - 1} {2m} \cdot (-2n + 2s) - (s - 1) = -\frac {r - 1} r (n - s) - (s - 1).
    \end{equation*}

    Since \( r \) divides \( n - s \), their quotient is a positive integer. Thus,
    \begin{equation*}
      e \leq -(r - 1) - \underbrace{(s - 1)}_{\geq 0} \leq -r + 1.
    \end{equation*}
  \end{itemize}

  On both cases, we conclude that \( e \leq -r + 1 \) and thus the bound for the edge count of \( G \) coincides the edge count for \( T_{n,r} \) from \eqref{eq:thm:turan_graph_edge_count}.
\end{proof}

\begin{theorem}[Mantel's theorem]\label{thm:mantel_theorem}\mcite[thm. I.2]{Bollobás1998ModernGraphTheory}
  If a \hyperref[def:undirected_graph]{simple undirected graph} with \( n \) vertices has more than \( n^2 / 4 \) edges, it contains a \hyperref[def:triangle_graph]{triangle subgraph}.
\end{theorem}
\begin{proof}
  Let \( G \) be a graph of order \( n \) with more than \( n^2 / 4 \) edges.

  Suppose that it contains no triangles. By \fullref{thm:turans_theorem}, its edge count is at most
  \begin{equation*}
    \frac {n^2 - s^2} 4 + \binom s 2,
  \end{equation*}
  where \( s = \rem(n, 2) \) is \( 0 \) if \( n \) is even and \( 1 \) if \( n \) is odd. We have the following weaker bound:
  \begin{equation*}
    \frac {n^2} 4 + \frac {-s^2 + 2s^2 - 2s} 4
    =
    \frac {n^2} 4 + \underbrace{\frac {s(s - 2)} 4}_{\leq 0}
    \leq
    \frac {n^2} 4.
  \end{equation*}

  But this contradicts our initial assumption that \( G \) has more than \( \frac {n^2} 4 \) edges.

  Therefore, \( G \) does not contain triangles.
\end{proof}
