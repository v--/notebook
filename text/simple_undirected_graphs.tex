\subsection{Simple undirected graphs}\label{subsec:simple_undirected_graphs}

\paragraph{Basic graphs}

\begin{definition}\label{def:edgeless_graph}\mimprovised
  For a set \( S \), we define the \term[ru=безрёберный (\cite[16]{Зыков2004})]{edgeless graph} \( \oline{K_S} \) as, unsurprisingly, the simple undirected graph with no edges whose vertices are \( S \).

  For every positive integer \( n \), we define \( \oline{K_n} \) as the edgeless graph with vertices \( S = \set{ 1, \ldots, n } \).
\end{definition}
\begin{comments}
  \item The notation \( \oline{K_S} \) is used because edgeless graphs are \hyperref[def:graph_complement]{complements} of \hyperref[def:complete_graph]{complete graphs}, which are in turn denoted by \( K_n \). We find it simpler to define them the other way around.

  \item For \( \oline{K_n} \), \incite[18]{Bollobas1998} prefers the term \enquote{empty graph}, while \incite[def. 1.3.1]{Knauer2011} uses \enquote{totally disconnected} and \enquote{discrete graph}.
\end{comments}

\begin{theorem}[Edgeless graph universal property]\label{thm:edgeless_graph_universal_property}
  Given a set \( S \), the \hyperref[def:edgeless_graph]{edgeless graph} \( \oline{K_S} \) is the unique up to a unique isomorphism simple undirected graph that satisfies the following \hyperref[rem:universal_mapping_property]{universal mapping property}:
  \begin{displayquote}
    For every simple undirected graph \( G = (V, E) \) and every function \( f: S \to V \), there exists a unique \hyperref[def:undirected_graph/homomorphism]{homomorphism} \( \widetilde{f}: \oline{K_S} \to G \) such that the following diagram commutes:
    \begin{equation}\label{eq:thm:edgeless_graph_universal_property/diagram}
      \begin{aligned}
        \includegraphics[page=1]{output/thm__edgeless_graph_universal_property}
      \end{aligned}
    \end{equation}
  \end{displayquote}
\end{theorem}
\begin{comments}
  \item Via \fullref{rem:universal_mapping_property}, \( V \mapsto \oline{K_V} \) becomes \hyperref[def:category_adjunction]{left adjoint} to the \hyperref[def:concrete_category]{forgetful functor} from simple undirected graphs to sets.

  \Fullref{thm:first_order_categorical_invertibility/injective} then implies that the \hyperref[def:morphism_invertibility/left_cancellative]{categorical monomorphisms} are precisely the injective graph homomorphisms.
\end{comments}
\begin{proof}
  Since \( \oline{K_S} \) has no edges, \( f \) vacuously preserves all edges of \( \oline{K_V} \), and is thus the desired homomorphism.
\end{proof}

\begin{definition}\label{def:graph_complement}\mcite[def. 5.1.1]{Knauer2011}
  We define the \term{complement} of the \hyperref[def:undirected_graph]{simple undirected graphs} \( G = (V, E) \) as
  \begin{equation*}
     G^{\complement} \coloneqq (V, V^2 \setminus E).
  \end{equation*}
\end{definition}

\begin{definition}\label{def:complete_graph}\mimprovised
  For every set \( V \), we define the \term[bg=пълен граф (\cite[12]{Мирчев2001}), ru=полный граф (\cite[16]{Зыков2004}), en=complete graph (\cite[def. 1.3.1]{Knauer2011})]{complete graph} \( K_V \) as the \hyperref[def:graph_complement]{complement} of the \hyperref[def:edgeless_graph]{edgeless graph} \( \oline{K_V} \) on \( V \).

  In particular, for a nonnegative integer \( n \), \( K_n \coloneqq \oline{K_n}^{\complement} \).

  \begin{figure}[!ht]
    \begin{equation}\label{eq:fig:def:complete_graph/k4}
      \begin{aligned}
        \includegraphics[page=1]{output/def__complete_graph}
      \end{aligned}
    \end{equation}
    \caption{The \hyperref[def:complete_graph]{complete graph} \( K_4 \).}\label{fig:def:complete_graph/k4}
  \end{figure}
\end{definition}
\begin{comments}
  \item An \hyperref[def:graph_orientation]{orientation} of a complete graph is sometimes called a \enquote{tournament} --- for example by \incite[289]{Diestel2005}, \incite[30]{Bollobas1998}, \incite[570]{Rosen1999} and \incite[\textnumero 7.3.3]{Новиков2013} (as \enquote{турнир}).
\end{comments}

\begin{theorem}[Complete graph universal property]\label{thm:complete_graph_universal_property}
  Given a set \( S \), the \hyperref[def:complete_graph]{complete graph} \( K_S \) is the unique up to a unique isomorphism simple undirected graph that satisfies the following \hyperref[rem:universal_mapping_property]{universal mapping property}:
  \begin{displayquote}
    For every simple undirected graph \( G = (V, E) \) and every function \( f: V \to S \), there exists a unique \hyperref[def:undirected_graph/homomorphism]{homomorphism} \( \widetilde{f}: G \to K_S \) such that the following diagram commutes:
    \begin{equation}\label{eq:thm:edgeless_graph_universal_property/diagram}
      \begin{aligned}
        \includegraphics[page=1]{output/thm__complete_graph_universal_property}
      \end{aligned}
    \end{equation}
  \end{displayquote}
\end{theorem}
\begin{comments}
  \item Via \fullref{rem:universal_mapping_property}, \( V \mapsto K_V \) becomes \hyperref[def:category_adjunction]{right adjoint} to the \hyperref[def:concrete_category]{forgetful functor} from simple undirected graphs to sets.

  \Fullref{thm:first_order_categorical_invertibility/surjective} then implies that the \hyperref[def:morphism_invertibility/right_cancellative]{categorical epimorphism} are precisely the surjective graph homomorphisms.
\end{comments}
\begin{proof}
  For every edge \( \set{ u, v } \) of \( G \), \( \set{ f(u), f(v) } \) is an edge of \( K_S \). Therefore, \( f \) is itself the desired homomorphism.
\end{proof}

\begin{definition}\label{def:path_graph}\mcite[def. 1.3.1]{Knauer2011}
  For every positive integer \( n \), we define the \hyperref[def:undirected_graph]{simple undirected graph}
  \begin{equation*}
    P_n \coloneqq \parens[\Big]{ \set{ 1, \ldots, n }, \set[\Big]{ \set{ k, k + 1 } \given* 1 \leq k < n } }.
  \end{equation*}

  \begin{figure}[!ht]
    \begin{equation}\label{eq:fig:def:path_graph/p4}
      \begin{aligned}
        \includegraphics[page=1]{output/def__path_graph}
      \end{aligned}
    \end{equation}
    \caption{The graph \hyperref[def:path_graph]{\( P_4 \)}.}\label{fig:def:path_graph/p4}
  \end{figure}
\end{definition}

\begin{definition}\label{def:cycle_graph}\mcite[def. 1.3.1]{Knauer2011}
  For every positive integer \( n \), we define \term[en=cycle graph (\cite[532]{Rosen1999})]{cycle graph} \( C_n \) as \hyperref[def:path_graph]{\( P_n \)} with an additional edge connecting \( n \) to \( 1 \).

  \begin{figure}[!ht]
    \begin{equation}\label{eq:fig:def:cycle_graph/c12}
      \begin{aligned}
        \includegraphics[page=1]{output/def__cycle_graph}
      \end{aligned}
    \end{equation}
    \caption{The graph \hyperref[def:cycle_graph]{cycle graph} \( C_{12} \).}\label{fig:def:cycle_graph/c12}
  \end{figure}
\end{definition}

\paragraph{Combining graphs}

\begin{definition}\label{def:graph_disjoint_union}\mimprovised
  We define the \term{disjoint union} of the family of \hyperref[def:undirected_graph]{simple undirected graphs} \( \seq{ G_k }_{k \in \mscrK} \), where \( G_k = (V_k, E_k) \), as the graph
  \begin{equation*}
    \coprod_{k \in \mscrK} G_k \coloneqq \parens[\Big]{ \coprod_{k \in \mscrK} V_k, \bigcup_{k \in \mscrK} \set[\Big]{ \set{ \iota_k(u), \iota_k(v) } \given* \set{ u, v } \in E_k } },
  \end{equation*}
  whose vertex set is a \hyperref[def:disjoint_union]{disjoint union} of the vertex sets of the individual graphs and the edge set consists of the inclusions of the corresponding edges.
\end{definition}
\begin{comments}
  \item This definition can be adjusted straightforwardly to other kinds of graphs, however we will not make use of that.
  \item \incite[def. 4.1.1]{Knauer2011} and \incite[23]{Зыков2004} use regular unions with the convention that the vertex sets and edge sets are disjoint.
\end{comments}

\begin{proposition}\label{thm:undirected_graph_coproduct}
  The \hyperref[def:discrete_category_limits]{categorical coproduct} of a family of graphs in the category of \hyperref[def:undirected_graph]{simple undirected graphs} is their \hyperref[def:graph_disjoint_union]{disjoint union}.
\end{proposition}
\begin{proof}
  Let \( (H, \alpha) \) be a \hyperref[def:category_of_cones/cocone]{cocone} for the discrete diagram \( \seq{ G_k }_{k \in \mscrK} \). We want to define a graph homomorphism \( l: \coprod_{k \in \mscrK} G_k \to H \) such that, for every \( m \in \mscrK \),
  \begin{equation*}
    \alpha_m(v) = l_A(\iota_m(v)).
  \end{equation*}

  This suggests the definition
  \begin{equation*}
    l_A\parens[\Big]{ \iota_k(v) } \coloneqq \alpha_k(v).
  \end{equation*}
\end{proof}

\begin{definition}\label{def:graph_join}\mcite[def. 4.2.1]{Knauer2011}
  We define the \term[ru=соединение (графов) (\cite[265]{Новиков2013})]{graph join} \( G \Ast H \) of the \hyperref[def:undirected_graph]{simple undirected graphs} \( G \) and \( H \) as the \hyperref[def:graph_disjoint_union]{disjoint union} \( G \coprod H \) with an additional edge between every vertex from \( G \) and every vertex from \( H \).
\end{definition}
\begin{comments}
  \item \incite[def. 4.2.1]{Knauer2011} and \incite[265]{Новиков2013} denote this operation by \enquote{\( + \)}, while \incite[23]{Зыков2004} uses \enquote{\( + \)} for graph unions instead. Zykov calls the join operation a \enquote{произведение}, which literally translates to \enquote{product}, however he translates it as \enquote{join}. \incite[547]{Rosen1999} also calls this operation \enquote{join} and denotes it by \enquote{\( \Ast \)}. \incite[15]{Diestel2005} also uses \enquote{\( \Ast \)}, however avoids introducing a special term.
\end{comments}

\paragraph{Graph partitions}

\begin{definition}\label{def:graph_independent_set}\mcite[3]{Diestel2005}
  We say that a subset of the vertices of an \hyperref[rem:arbitrary_graph]{arbitrary graph} is an \term[bg=независимо множество (\cite[103]{Мирчев2001})]{independent set} if no two vertices in it are \hyperref[def:graph_adjacency]{adjacent}.
\end{definition}

\begin{definition}\label{def:n_partite_graph}\mcite[def. 1.3.1]{Knauer2011}
  For a positive integer \( n \), we say that a \hyperref[def:undirected_graph]{simple undirected graph} \( G = (V, E) \) is \( n \)-\term{partite} if \( V \) can be \hyperref[def:set_partition]{partitioned} into \( n \) (possibly empty) \hyperref[def:graph_independent_set]{independent} subsets \( V_1, \ldots, V_n \).

  We use the term \term[ru=двудольный (граф) (\cite[177]{Зыков2004})]{bipartite} when \( n = 2 \) and \term{tripartite} when \( n = 3 \).

  \begin{figure}[!ht]
    \centering
    \includegraphics[page=1]{output/def__n_partite_graph}
    \caption{A drawing of \hyperref[def:cycle_graph]{\( C_6 \)} highlighting that it is \hyperref[def:n_partite_graph]{tripartite}.}\label{fig:def:n_partite_graph}
  \end{figure}
\end{definition}
\begin{itemize}
  \item It is tempting to require the subsets of the partition to be nonempty, however many authors choose not to do so. This includes \incite[17]{Diestel2005}, \incite[21]{Bollobas1998}, \incite[def. 1.3.1]{Knauer2011}, \incite[26]{GondranMinoux1984Graphs}, \incite[530]{Rosen1999} and \incite[177]{Зыков2004}. We follow their lead.
\end{itemize}

\begin{definition}\label{def:complete_n_partite_graph}\mimprovised
  For an integer \( r \geq 2 \), we define the \term{complete \( n \)-partite} graph \( K_{V_1,\ldots,V_n} \) as the \hyperref[def:graph_join]{graph join}
  \begin{equation*}
    \oline{K_{V_1}} \Ast \oline{K_{V_2}} \Ast \cdots \Ast \oline{K_{V_n}}
  \end{equation*}
  of \( n \) \hyperref[def:edgeless_graph]{edgeless graphs}.

  \begin{figure}[!ht]
    \hfill
    \includegraphics[page=1]{output/def__complete_n_partite_graph}
    \hfill
    \includegraphics[page=2]{output/def__complete_n_partite_graph}
    \hfill
    \hfill
    \caption{The \hyperref[def:complete_n_partite_graph]{complete bipartite graphs} \( K_{1,5} \) and \( K_{3,3} \).}\label{fig:def:complete_n_partite_graph}
  \end{figure}
\end{definition}
\begin{comments}
  \item We generalize the definition by \incite[24]{Зыков2004} from bipartite to general \( n \)-partite graphs.
  \item If we allow \( r = 1 \), we obtain edgeless graphs.
  \item If we plain unions rather than disjoint unions in the definition of graph joins, we would have to ensure that \( V_1, \ldots, V_r \) are disjoint. This would make the definition of \( K_{n,m} \) more tricky that it has to be.
\end{comments}

\begin{proposition}\label{thm:bipartite_iff_no_odd_cycles}\mcite[thm. 1.3.5]{Knauer2011}
  A \hyperref[def:undirected_graph]{simple undirected graph} is \hyperref[def:n_partite_graph]{bipartite} if and only if it has no \hyperref[def:graph_cycle]{cycles} of odd length.
\end{proposition}
\begin{proof}
  \SufficiencySubProof Suppose that \( G = (V, E) \) is bipartite and let \( A \cup B \) be a partition of \( V \) into nonempty disjoint independent sets.

  Consider a cycle
  \begin{equation*}
    v_0 \to v_1 \to \cdots \to v_n.
  \end{equation*}

  Without loss of generality, suppose that \( v_0 \) is in \( A \). Then \( v_1 \) must be in \( B \). By induction, we conclude that \( v_k \) is in \( B \) if \( k \) is odd and in \( A \) if \( k \) is even. Since \( v_n = v_1 \) is in \( A \), we conclude that \( n \) is even.

  \NecessitySubProof Suppose that \( G = (V, E) \) has no cycles of odd length.

  If \( V \) is empty, \( G \) is vacuously bipartite. Suppose that \( V \) is nonempty. Let \( G_1, \ldots, G_s \) be the connected components of \( G \).

  Fix a vertex \( u \) from \( G_k \). Let \( A_k \) be the set of all vertices from \( G_k \) reachable from \( u \) with even-length paths and let \( B_k \) be the complement of \( A_k \) in \( V \). We will show that both \( A_k \) and \( B_k \) are independent.

  Aiming at a contradiction, suppose that there exist adjacent vertices \( v \) and \( w \) in \( A_k \). We can concatenate even-length paths from \( w \) to \( u \) and from \( u \) to \( v \) to obtain an even-length path from \( w \) to \( v \). Appending the edge \( \set{ v, w } \), we obtain an odd-length cycle at \( w \), whose existence contradicts our initial assumption. Hence, \( A_k \) is an independent set.

  Similarly, suppose that there exist adjacent vertices \( v \) and \( w \) in \( B_k \). Then we can concatenate odd-length paths from \( w \) to \( u \) and from \( u \) to \( v \) to obtain an even-length path from \( w \) to \( v \). Appending \( \set{ u, v } \), we again obtain an odd-length cycle at \( w \), whose existence again leads to a contradiction. Hence, \( B_k \) is also an independent set.

  We can now take the unions \( A \coloneqq A_1 \cup \cdots \cup A_s \) and \( B \coloneqq B_1 \cup \cdots \cup B_s \). These are again independent sets because \( A_i \) and \( A_j \) belong to different connected components if \( i \neq j \).

  Then \( V = A \cup B \) is the desired partition of \( V \), which demonstrates that \( G \) is bipartite.
\end{proof}

\paragraph{Graph coloring}

\begin{definition}\label{def:graph_coloring}\mcite[395]{Erickson2019}
  A \term{vertex coloring} (resp. \term{edge coloring} or \term{arc coloring}) of a \hyperref[rem:arbitrary_graph]{graph} is, unsurprisingly, a \hyperref[def:set_coloring]{coloring} of its vertices (resp. edges or arcs).

  \begin{thmenum}
    \thmitem{def:graph_coloring/proper} We say that the vertex coloring is \term[bg=правилно (оцветяване) (\cite[141]{Мирчев2001}), ru=правильная (раскраска) (\cite[28]{Зыков2004})]{proper} if no two adjacent vertices have the same color.

    \thmitem{def:graph_coloring/colorable} If \( A \) has an \( n \)-coloring, we say that it is \( n \)-\term[bg=\( n \)-оцветим (граф) (\cite[141]{Мирчев2001}), ru=\( n \)-раскрашиваемый (граф) (\cite[\textnumero 10.7.1]{Новиков2013}), en=\( n \)-colorable (\cite[111]{Diestel2005})]{colorable}.
  \end{thmenum}
\end{definition}
\begin{comments}
  \item We can adopt the view that \( n \)-colorable graphs are generalizations of \( n \)-partite graphs that introduce more natural conditions for directed graphs.

  \item Some authors like \incite[112]{Diestel2005}, \incite[145]{Bollobas1998} and \incite[374]{Новиков2013} do not distinguish between proper and improper colorings by requiring all colorings to be proper.

  We prefer the more explicit terminology of \mcite[395]{Erickson2019}, \incite[590]{Rosen1999}, \incite[28]{Зыков2004} and \incite[141]{Мирчев2001}.
\end{comments}

\begin{proposition}\label{thm:def:graph_coloring}
  \hyperref[def:n_partite_graph]{Graph coloring} has the following basic properties:
  \begin{thmenum}
    \thmitem{thm:def:graph_coloring/edgeless} A graph without arcs/edges is \( n \)-colorable for any positive integer \( n \).
    \thmitem{thm:def:graph_coloring/succ} If a graph is \( n \)-colorable, it is also \( (n + 1) \)-colorable.
    \thmitem{thm:def:graph_coloring/order} Every graph of order \( n \) is \( n \)-colorable (however it may not be \( (n - 1) \)-colorable).
    \thmitem{thm:def:graph_coloring/complete} A \hyperref[def:complete_graph]{complete graph} of order \( n \) is not \( (n - 1) \)-colorable.
  \end{thmenum}
\end{proposition}
\begin{proof}
  \SubProofOf{thm:def:graph_coloring/edgeless} If there are no arcs/edges in a graph, no two vertices are adjacent, and any vertex coloring is proper.

  \SubProofOf{thm:def:graph_coloring/succ} Given any vertex \( n \)-coloring \( f: V \to \set{ 1, \ldots, n } \), we can simply extend the range of \( f \) with additional colors.

  \SubProofOf{thm:def:graph_coloring/order} Given a graph with a vertex set \( V = \set{ v_1, \ldots, v_n } \), we can color \( V \) by assigning the color \( k \) to \( v_k \).

  \SubProofOf{thm:def:graph_coloring/complete} In a complete graph of order \( n \), by \fullref{thm:pigeonhole_principle} every vertex \( (n - 1) \)-coloring necessarily has two vertices with the same color.
\end{proof}

\begin{definition}\label{def:chromatic_number}\mcite[28]{Зыков2004}
  The \term[bg=хроматично число (\cite[142]{Мирчев2001}), ru=хроматическое число (\cite[28]{Зыков2004}), en=chromatic number (\cite[122]{Diestel2005})]{chromatic number} \( \chi(G) \) of the \hyperref[rem:arbitrary_graph]{arbitrary graph} \( G \) is the minimum positive integer \( k \) for which \( G \) is \( k \)-\hyperref[def:graph_coloring/colorable]{colorable}.
\end{definition}
\begin{comments}
  \item Different authors use a different notation for chromatic numbers. We use \enquote{\( \chi \)} similarly to \incite[122]{Diestel2005}, \incite[145]{Bollobas1998}, \incite[152]{Knauer2011}, \incite[531]{Rosen1999} and \incite[374]{Новиков2013}. \incite[27]{Зыков2004} and \incite[28]{GondranMinoux1984Graphs} use \enquote{\( \gamma \)}, while \incite[142]{Мирчев2001} uses \enquote{\( \varkappa \)}.
\end{comments}

\begin{proposition}\label{thm:n_colorable_iff_n_partite}
  A \hyperref[def:undirected_graph]{simple undirected graph} is \( n \)-\hyperref[def:n_partite_graph]{partite} if and only if it is \( n \)-\hyperref[def:graph_coloring/colorable]{colorable}.
\end{proposition}
\begin{proof}
  Trivial.
\end{proof}
