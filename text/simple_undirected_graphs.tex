\subsection{Simple undirected graphs}\label{subsec:simple_undirected_graphs}

\paragraph{Basic graphs}

\begin{definition}\label{def:edgeless_graph}\mimprovised
  For a set \( S \), we define the \term{edgeless graph} \( \oline{K_S} \) as, unsurprisingly, the simple undirected graph with no edges whose vertices are \( S \).
\end{definition}
\begin{comments}
  \item See \fullref{rem:universal_graph_cardinals} regarding the case where \( S \) is a \hyperref[def:cardinal_number]{cardinal number}.

  \item Following \incite[16]{Harary1969}, we use the notation \( \oline{K_S} \). This is justified by edgeless graphs being \hyperref[def:graph_complement]{complements} of \hyperref[def:complete_graph]{complete graphs}, which are in turn denoted by \( K_n \). We find it simpler to define them the other way around.

  \item For \( \oline{K_n} \), \incite[18]{Bollobas1998} prefers the term \enquote{empty graph}, while \incite[def. 1.3.1]{Knauer2011} uses \enquote{totally disconnected} and \enquote{discrete graph}. \incite[16]{Harary1969} also uses \enquote{totally disconnected}.
\end{comments}

\begin{remark}\label{rem:universal_graph_cardinals}
  In the special case where \( S \) is a \hyperref[def:cardinal_number]{cardinal number} \( \kappa \), the vertices of \( \oline{K_\kappa} \) are simply the smaller ordinals. This is discussed in \fullref{rem:cardinal_colorings} regarding \( \kappa \)-colorings.

  An inconsistency arises --- we prefer indexing our vertices via positive integers, i.e. \( 1, \ldots, n \) rather than \( 0, \ldots, n - 1 \). Fortunately, graphs are generally considered up to an isomorphism, so don't have to bother too much with concrete choices for the vertices.
\end{remark}

\begin{theorem}[Edgeless graph universal property]\label{thm:edgeless_graph_universal_property}
  Given a set \( S \), the \hyperref[def:edgeless_graph]{edgeless graph} \( \oline{K_S} \) is the unique up to a unique isomorphism simple undirected graph that satisfies the following \hyperref[rem:universal_mapping_property]{universal mapping property}:
  \begin{displayquote}
    For every simple undirected graph \( G = (V, E) \) and every function \( f: S \to V \), there exists a unique \hyperref[def:undirected_graph/homomorphism]{homomorphism} \( \widetilde{f}: \oline{K_S} \to G \) such that the following diagram commutes:
    \begin{equation}\label{eq:thm:edgeless_graph_universal_property/diagram}
      \begin{aligned}
        \includegraphics[page=1]{output/thm__edgeless_graph_universal_property}
      \end{aligned}
    \end{equation}
  \end{displayquote}
\end{theorem}
\begin{comments}
  \item Via \fullref{rem:universal_mapping_property}, \( V \mapsto \oline{K_V} \) becomes \hyperref[def:category_adjunction]{left adjoint} to the \hyperref[def:concrete_category]{forgetful functor} from simple undirected graphs to sets.

  \Fullref{thm:first_order_categorical_invertibility/injective} then implies that the \hyperref[def:morphism_invertibility/left_cancellative]{categorical monomorphisms} are precisely the injective graph homomorphisms.
\end{comments}
\begin{proof}
  Since \( \oline{K_S} \) has no edges, \( f \) vacuously preserves all edges of \( \oline{K_V} \), and is thus the desired homomorphism.
\end{proof}

\begin{definition}\label{def:graph_complement}\mcite[def. 5.1.1]{Knauer2011}
  We define the \term[ru=дополнение (\cite[15]{Емеличев1990})]{complement} of the \hyperref[def:undirected_graph]{simple undirected graphs} \( G = (V, E) \) as
  \begin{equation*}
     G^{\complement} \coloneqq (V, V^2 \setminus E).
  \end{equation*}
\end{definition}

\begin{definition}\label{def:complete_graph}\mimprovised
  For every set \( V \), we define the \term[bg=пълен граф (\cite[12]{Мирчев2001}), ru=полный граф (\cite[10]{Емеличев1990}), en=complete graph (\cite[def. 1.3.1]{Knauer2011})]{complete graph} \( K_V \) as the \hyperref[def:graph_complement]{complement} of the \hyperref[def:edgeless_graph]{edgeless graph} \( \oline{K_V} \) on \( V \).

  \begin{figure}[!ht]
    \begin{equation}\label{eq:fig:def:complete_graph/k4}
      \begin{aligned}
        \includegraphics[page=1]{output/def__complete_graph}
      \end{aligned}
    \end{equation}
    \caption{The \hyperref[def:complete_graph]{complete graph} \( K_4 \).}\label{fig:def:complete_graph/k4}
  \end{figure}
\end{definition}
\begin{comments}
  \item See \fullref{rem:universal_graph_cardinals} regarding the case where \( S \) is a \hyperref[def:cardinal_number]{cardinal number}.

  \item An \hyperref[def:graph_orientation]{orientation} of a complete graph is sometimes called a \enquote{tournament} --- for example by \incite[289]{Diestel2005}, \incite[30]{Bollobas1998}, \incite[570]{Rosen1999} and \incite[\S 7.3.3]{Новиков2013} (as \enquote{турнир}).
\end{comments}

\begin{proposition}\label{thm:complete_graph_edge_count}
  The number of edges in the \hyperref[def:complete_graph]{complete graph} \( K_n \), and hence the maximum number of edges in a simple undirected graph of order \( n \), is \( \binom n 2 \).
\end{proposition}
\begin{proof}
  The cases where \( n < 2 \) are trivial. We use induction on \( n \geq 2 \) to show that the number of edges is \( \sum_{k=0}^{n-1} k \), which by \fullref{thm:numeric_arithmetic_progression_partial_sums} equals \( \binom n 2 \).

  \begin{itemize}
    \item If \( n = 2 \), \( K_n \) has only one edge.
    \item If \( K_n \) has \( \sum_{k=0}^{n-1} k \) edges, then \( K_{n+1} \) has \( \sum_{k=0}^n k \) edges since it has additional vertex compared to \( K_n \) and \( n \) additional edges connecting that vertex to all others.
  \end{itemize}
\end{proof}

\begin{theorem}[Complete graph universal property]\label{thm:complete_graph_universal_property}
  Given a set \( S \), the \hyperref[def:complete_graph]{complete graph} \( K_S \) is the unique up to a unique isomorphism simple undirected graph that satisfies the following \hyperref[rem:universal_mapping_property]{universal mapping property}:
  \begin{displayquote}
    For every simple undirected graph \( G = (V, E) \) and every function \( f: V \to S \), there exists a unique \hyperref[def:undirected_graph/homomorphism]{homomorphism} \( \widetilde{f}: G \to K_S \) such that the following diagram commutes:
    \begin{equation}\label{eq:thm:edgeless_graph_universal_property/diagram}
      \begin{aligned}
        \includegraphics[page=1]{output/thm__complete_graph_universal_property}
      \end{aligned}
    \end{equation}
  \end{displayquote}
\end{theorem}
\begin{comments}
  \item Via \fullref{rem:universal_mapping_property}, \( V \mapsto K_V \) becomes \hyperref[def:category_adjunction]{right adjoint} to the \hyperref[def:concrete_category]{forgetful functor} from simple undirected graphs to sets.

  \Fullref{thm:first_order_categorical_invertibility/surjective} then implies that the \hyperref[def:morphism_invertibility/right_cancellative]{categorical epimorphism} are precisely the surjective graph homomorphisms.
\end{comments}
\begin{proof}
  For every edge \( \set{ u, v } \) of \( G \), \( \set{ f(u), f(v) } \) is an edge of \( K_S \). Therefore, \( f \) is itself the desired homomorphism.
\end{proof}

\begin{definition}\label{def:path_graph}\mcite[def. 1.3.1]{Knauer2011}
  For every positive integer \( n \), we regard it as the set of smaller cardinals and define the \hyperref[def:undirected_graph]{simple undirected graph}
  \begin{equation*}
    P_n \coloneqq \parens[\Big]{ n, \set[\Big]{ \set{ k, k + 1 } \given* 1 \leq k < n - 1 } }.
  \end{equation*}

  \begin{figure}[!ht]
    \begin{equation}\label{eq:fig:def:path_graph/p4}
      \begin{aligned}
        \includegraphics[page=1]{output/def__path_graph}
      \end{aligned}
    \end{equation}
    \caption{The \hyperref[def:path_graph]{path graph} \( P_4 \).}\label{fig:def:path_graph/p4}
  \end{figure}
\end{definition}

\begin{definition}\label{def:cycle_graph}\mcite[def. 1.3.1]{Knauer2011}
  For every integer \( n > 2 \), we define \term[en=cycle graph (\cite[532]{Rosen1999})]{cycle graph} \( C_n \) as \hyperref[def:path_graph]{\( P_n \)} with an additional edge connecting \( n - 1 \) to \( 0 \).

  \begin{figure}[!ht]
    \begin{equation}\label{eq:fig:def:cycle_graph/c12}
      \begin{aligned}
        \includegraphics[page=1]{output/def__cycle_graph}
      \end{aligned}
    \end{equation}
    \caption{The \hyperref[def:cycle_graph]{cycle graph} \( C_{12} \).}\label{fig:def:cycle_graph/c12}
  \end{figure}
\end{definition}

\paragraph{Combining graphs}

\begin{definition}\label{def:graph_disjoint_union}\mimprovised
  We define the \term[ru=дизъюнктное объедиение (графов) (\cite[19]{Емеличев1990})]{disjoint union} of the family of \hyperref[def:undirected_graph]{simple undirected graphs} \( \seq{ G_k }_{k \in \mscrK} \), where \( G_k = (V_k, E_k) \), as the graph
  \begin{equation*}
    \coprod_{k \in \mscrK} G_k \coloneqq \parens[\Big]{ \coprod_{k \in \mscrK} V_k, \bigcup_{k \in \mscrK} \set[\Big]{ \iota_k^E(e) \given* e \in E_k } },
  \end{equation*}
  where \( \iota_k^V \) is the canonical inclusion from \fullref{def:disjoint_union} and
  \begin{equation*}
    \iota_k^E(\set{ u, v }) \coloneqq \set{ \iota_k^V(u), \iota_k^V(v) }.
  \end{equation*}
\end{definition}
\begin{comments}
  \item This definition can be adjusted straightforwardly to other kinds of graphs, however we will not make use of that.
  \item \incite[21]{Harary1969} and \incite[def. 4.1.1]{Knauer2011} use regular unions with the convention that the vertex sets and edge sets are disjoint. \incite[19]{Емеличев1990} also use regular unions, however they explicitly refer to unions as \enquote{disjoint} (\enquote{дизъюнктное объединение}) if the vertex sets and edge sets are disjoint.
\end{comments}

\begin{proposition}\label{thm:undirected_graph_coproduct}
  The \hyperref[def:discrete_category_limits]{categorical coproduct} of a family of graphs in the category of \hyperref[def:undirected_graph]{simple undirected graphs} is their \hyperref[def:graph_disjoint_union]{disjoint union}.
\end{proposition}
\begin{proof}
  Let \( (H, \alpha) \) be a \hyperref[def:category_of_cones/cocone]{cocone} for the discrete diagram \( \seq{ G_k }_{k \in \mscrK} \). We want to define a graph homomorphism \( l: \coprod_{k \in \mscrK} G_k \to H \) such that, for every \( m \in \mscrK \),
  \begin{equation*}
    \alpha_m(v) = l_A(\iota^V_m(v)).
  \end{equation*}

  This suggests the definition
  \begin{equation*}
    l_A\parens[\Big]{ \iota^V_k(v) } \coloneqq \alpha_k(v).
  \end{equation*}
\end{proof}

\begin{definition}\label{def:graph_join}\mcite[def. 4.2.1]{Knauer2011}
  We define the \term[ru=соединение (графов) (\cite[265]{Новиков2013})]{graph join} \( G \Ast H \) of the \hyperref[def:undirected_graph]{simple undirected graphs} \( G \) and \( H \) as the \hyperref[def:graph_disjoint_union]{disjoint union} \( G \coprod H \) with an additional edge between every vertex from \( G \) and every vertex from \( H \).
\end{definition}
\begin{comments}
  \item \incite[21]{Harary1969} attributes the introduction of graph joins to \cite[164]{Зыков1949}. Zykov calls them \enquote{products} and denotes them via juxtaposition, however his paper generally uses arcane terminology like \enquote{complex} instead of \enquote{graph}.

  The term \enquote{join} is used by \incite[21]{Harary1969}, \incite[def. 4.2.1]{Knauer2011} and \incite[\S 265]{Новиков2013} (as \enquote{соединение}). The aforementioned authors denote this operation by \enquote{\( + \)}. \incite[547]{Rosen1999} also calls this operation \enquote{join}, however he denotes it by \enquote{\( \Ast \)}. \incite[15]{Diestel2005} also uses \enquote{\( \Ast \)}, however avoids introducing a special term.
\end{comments}

\paragraph{Generalized Petersen graphs}

\begin{definition}\label{def:petersen_graph}\mcite[2]{Watkins1969}
  For positive integers \( k < n \), the \term{generalized Petersen graph} \( P_{n,k} \) is the \hyperref[def:undirected_graph]{simple undirected graph} constructed as follows:
  \begin{thmenum}
    \thmitem{def:petersen_graph/union} We take the \hyperref[def:graph_disjoint_union]{disjoint union} \( \oline{K_n} \coprod C_n \).
    \thmitem{def:petersen_graph/inter} For each \( i = 0, \ldots, n - 1 \), we connect \( \iota_{\oline{K_n}}(i) \) to \( \iota_{C_n}(i) \).
    \thmitem{def:petersen_graph/inner} For each \( i = 0, \ldots, n - 1 \), we connect \( \iota_{\oline{K_n}}(i) \) to \( \iota_{\oline{K_n}}(\rem(i + k, n)) \).
  \end{thmenum}

  The usual \term{Petersen graph} is \( P_{5,2} \).

  \begin{figure}[!ht]
    \begin{subcaptionblock}{\textwidth/2}
      \centering
      \includegraphics[page=1]{output/def__petersen_graph}
      \caption{\( P_{5,2} \)}\label{fig:def:petersen_graph/p52}
    \end{subcaptionblock}
    \hfill
    \begin{subcaptionblock}{\textwidth/2}
      \centering
      \includegraphics[page=2]{output/def__petersen_graph}
      \caption{\( P_{7,3} \)}\label{fig:def:petersen_graph/p73}
    \end{subcaptionblock}
    \par
    \begin{subcaptionblock}{\textwidth/2}
      \centering
      \includegraphics[page=3]{output/def__petersen_graph}
      \caption{\( P_{3,1} \)}\label{fig:def:petersen_graph/p31}
    \end{subcaptionblock}
    \hfill
    \begin{subcaptionblock}{\textwidth/2}
      \centering
      \includegraphics[page=4]{output/def__petersen_graph}
      \caption{\( P_{2,1} \)}\label{fig:def:petersen_graph/p21}
    \end{subcaptionblock}

    \caption{\hyperref[def:petersen_graph]{Generalized Petersen graphs}.}\label{fig:def:petersen_graph}
  \end{figure}
\end{definition}

\begin{proposition}\label{thm:petersen_graph_cardinality}
  The \hyperref[def:petersen_graph]{generalized Petersen graph} \( P_{n,k} \) has \( 2n \) vertices and \( 3n \) edges.
\end{proposition}
\begin{proof}
  We have \( 2n \) vertices and \( n \) edges by \fullref{def:petersen_graph/union}. \Fullref{def:petersen_graph/inter} and \fullref{def:petersen_graph/inter} each add \( n \) edges.
\end{proof}

\begin{definition}\label{def:bridgeless_graph}\mimprovised
  We say that a graph is \term{bridgeless} if it has no \hyperref[def:graph_bridge]{bridge}.
\end{definition}

\begin{proposition}\label{thm:petersen_graph_bridgeless}
  Every \hyperref[def:petersen_graph]{generalized Petersen graph} is \hyperref[def:bridgeless_graph]{bridgeless}.
\end{proposition}
\begin{proof}
  Fix an edge \( e \) in \( P_{n,k} \). We will show that \( e \) is necessarily contained in a cycle, and thus cannot be a bridge.

  \begin{figure}[!ht]
    \begin{subcaptionblock}{\textwidth/4}
      \centering
      \includegraphics[page=1]{output/thm__petersen_graph_bridgeless}
    \end{subcaptionblock}
    \hfill
    \begin{subcaptionblock}{\textwidth/4}
      \centering
      \includegraphics[page=2]{output/thm__petersen_graph_bridgeless}
    \end{subcaptionblock}
    \hfill
    \begin{subcaptionblock}{\textwidth/4}
      \centering
      \includegraphics[page=3]{output/thm__petersen_graph_bridgeless}
    \end{subcaptionblock}

    \caption{The three possibilities in the proof of \fullref{thm:petersen_graph_bridgeless}.}\label{fig:thm:petersen_graph_bridgeless/proof}
  \end{figure}

  \begin{itemize}
    \item If \( e \) is added during \fullref{def:petersen_graph/union}, it is an edge coming from \( C_n \), which induces a cycle on both endpoints of \( e \).

    \item If \( e \) is added during \fullref{def:petersen_graph/inter}, one of its endpoints comes from \( \oline{K_n} \), which necessarily connects to another endpoint from \( \oline{K_n} \), which in turn connects to \( C_n \). We can then traverse \( C_n \) to reach the other endpoint of \( e \).

    \item If \( e \) is added during \fullref{def:petersen_graph/inner}, both of its endpoints are connected to \( C_n \), and we can traverse \( C_n \) to obtain a cycle.
  \end{itemize}
\end{proof}

\paragraph{Quotient graphs}

\begin{definition}\label{def:quotient_graph}\mimprovised
  For any \hyperref[def:undirected_graph]{simple undirected graph} \( G = (V, E) \) and any \hyperref[def:equivalence_relation]{equivalence relation} \( {\cong} \) on \( V \), we define the \term{quotient graph}
  \begin{equation*}
    G / {\cong} \coloneqq \parens[\Big]{ V / {\cong}, \set{ \pi^E(e) \given e \in E \T{and} \pi^E(e) \T{is not a loop} } },
  \end{equation*}
  where \( \pi^V: V \to V / {\cong} \) is the canonical projection from \fullref{def:equivalence_relation} and
  \begin{equation*}
    \pi^E(\set{ u, v }) \coloneqq \set{ \pi^V(u), \pi^V(v) }.
  \end{equation*}

  \begin{thmenum}
    \thmitem{def:quotient_graph/vertex} Given a set \( U \) of vertices, we define the \term[ru=отождествление вершин (\cite[22]{Емеличев1990}), en=contraction (\cite[532]{Rosen1999})]{vertex contraction} \( G / U \) as the quotient of \( G \) by the \hyperref[thm:equivalence_closure]{equivalence closure} of the relation
    \begin{equation*}
      u \sim v \T{if} u, v \in U.
    \end{equation*}

    \thmitem{def:quotient_graph/edge} Given a set \( F \) of edges, we define the \term[ru=стягивание ребра (\cite[21]{Емеличев1990}), en=contraction (\cite[532]{Rosen1999})]{edge contraction} \( G / F \) as the vertex contraction of \( G \) by the set of all endpoints (of edges) in \( F \).
  \end{thmenum}
\end{definition}
\begin{comments}
  \item If we remove the restriction for \( \pi^E(e) \) not to be a loop, quotient graphs become the \hyperref[def:first_order_quotient]{first-order quotients} in the \hyperref[rem:theory_of_simple_undirected_graphs]{theory of simple undirected graphs}.

  \item \incite[19]{Diestel2005} defines quotients via vertex set partitions, and even considers infinite graphs, but requires the vertices in each equivalence class to be reachable from each other. Furthermore, he doesn't introduce a term for the quotient.

  \item \incite[19]{Diestel2005}, \incite[24]{Bollobas1998} and \incite[21]{Емеличев1990} define edge contractions one-edge-at-a-time. \incite[113]{Harary1969} and \incite[532]{Rosen1999} use the term \enquote{elementary contraction} for this one-edge operation. Our definition generalizes this to arbitrary sets of edges.

  \item Another approach is used by \incite[rem. 1.4.5]{Knauer2011}, who defines \enquote{contractions} as compositions of homomorphisms, each of which contracts the vertices of an edge into one. To avoid unnecessary loops, he uses \enquote{weak homomorphisms} --- functions \( f: (V_G, E_G) \to (V_H, E_H) \), for which \( \set{ u, v } \in E_G \) implies \( \set{ f(u), f(v) } \in E_H \), but only if \( f(u) \neq f(v) \).
\end{comments}

\begin{definition}\label{def:graph_minor}\mcite[19]{Diestel2005}
  A \term{minor} of a simple undirected graph \( G \) is a \hyperref[def:undirected_graph/subgraph]{subgraph} of a finite \hyperref[def:quotient_graph/edge]{edge contraction} of \( G \).
\end{definition}

\paragraph{Graph partitions}

\begin{definition}\label{def:graph_independent_set}\mcite[3]{Diestel2005}
  We say that a subset of the vertices of an \hyperref[rem:arbitrary_kind_graph]{arbitrary-kind graph} is an \term[bg=независимо множество (\cite[103]{Мирчев2001})]{independent set} if no two vertices in it are \hyperref[def:graph_adjacency]{adjacent}.
\end{definition}

\begin{definition}\label{def:multipartite_graph}\mcite[def. 1.3.1]{Knauer2011}
  For a positive integer \( k \), we say that a \hyperref[def:undirected_graph]{simple undirected graph} \( G = (V, E) \) is \term[ru=\( k \)-дольный (граф) (\cite[11]{Емеличев1990})]{\( k \)-partite} if \( V \) can be \hyperref[def:set_partition]{partitioned} into \( k \) (possibly empty) \hyperref[def:graph_independent_set]{independent} subsets \( V_1, \ldots, V_k \).

  We use the term \term[ru=двудольный (граф) (\cite[11]{Емеличев1990})]{bipartite} when \( k = 2 \) and \term[ru=трёхдольный (граф) (\cite[11]{Емеличев1990})]{tripartite} when \( n = 3 \). When no concrete value of \( n \) is meant, we say that the graph is \term[en=multipartite (graph) (\cite[ex. 8.5.5]{Knauer2011})]{multipartite}.
\end{definition}
\begin{itemize}
  \item It is tempting to require the subsets of the partition to be nonempty, however many authors choose not to do so. This includes \incite[17]{Diestel2005}, \incite[27]{Harary1969}, \incite[21]{Bollobas1998}, \incite[def. 1.3.1]{Knauer2011}, \incite[26]{GondranMinoux1984Graphs}, \incite[530]{Rosen1999} and \incite[11]{Емеличев1990}. We follow their lead.
\end{itemize}

\begin{example}\label{ex:def:multipartite_graph}
  We list some examples of \hyperref[def:multipartite_graph]{multipartite graphs}:
  \begin{thmenum}
    \thmitem{ex:def:multipartite_graph/edgeless} \hyperref[def:edgeless_graph]{Edgeless graphs} are vacuously \( k \)-partite for any positive integer \( k \).

    \thmitem{ex:def:multipartite_graph/path} Every \hyperref[def:path_graph]{path graph} is bipartite. Indeed, no two even vertices and no two odd vertices are adjacent.
    \begin{figure}[!ht]
      \centering
      \includegraphics[page=1]{output/ex__def__multipartite_graph}
      \caption{A drawing of the \hyperref[def:path_graph]{path graph} \( P_6 \) highlighting that it is \hyperref[def:multipartite_graph]{bipartite}.}\label{fig:ex:def:multipartite_graph/path}
    \end{figure}

    \thmitem{ex:def:multipartite_graph/cycle} Every \hyperref[def:cycle_graph]{cycle graph} is tripartite, however some are also bipartite.

    \begin{figure}[!ht]
      \begin{subcaptionblock}{\textwidth/4}
        \centering
        \includegraphics[page=2]{output/ex__def__multipartite_graph}
        \caption{\( C_6 \) is tripartite.}\label{fig:ex:def:multipartite_graph/cycle/c6_tripartite}
      \end{subcaptionblock}
      \hfill
      \begin{subcaptionblock}{\textwidth/4}
        \centering
        \includegraphics[page=3]{output/ex__def__multipartite_graph}
        \caption{\( C_6 \) is also bipartite.}\label{fig:ex:def:multipartite_graph/cycle/c6_bipartite}
      \end{subcaptionblock}
      \hfill
      \begin{subcaptionblock}{\textwidth/4}
        \centering
        \includegraphics[page=4]{output/ex__def__multipartite_graph}
        \caption{\( C_5 \) is tripartite.}\label{fig:ex:def:multipartite_graph/cycle/c5}
      \end{subcaptionblock}
      \caption{Drawings of \hyperref[def:cycle_graph]{cycle graphs} highlighting their \hyperref[def:multipartite_graph]{multipartite} structure.}\label{fig:ex:def:multipartite_graph/cycle}
    \end{figure}

    Again, we split the odd and even vertices, however we must also acknowledge that the first and last vertex may have the same parity, forcing the last vertex into its own color class.
  \end{thmenum}
\end{example}

\begin{proposition}\label{thm:bipartite_iff_no_odd_cycles}\mcite[thm. 1.3.5]{Knauer2011}
  A \hyperref[def:undirected_graph]{simple undirected graph} is \hyperref[def:multipartite_graph]{bipartite} if and only if it has no \hyperref[def:graph_cycle]{cycles} of odd length.
\end{proposition}
\begin{proof}
  \SufficiencySubProof Suppose that \( G = (V, E) \) is bipartite and let \( A \cup B \) be a partition of \( V \) into nonempty disjoint independent sets.

  Consider a cycle
  \begin{equation*}
    v_0 \to v_1 \to \cdots \to v_n.
  \end{equation*}

  Without loss of generality, suppose that \( v_0 \) is in \( A \). Then \( v_1 \) must be in \( B \). By induction, we conclude that \( v_k \) is in \( B \) if \( k \) is odd and in \( A \) if \( k \) is even. Since \( v_n = v_1 \) is in \( A \), we conclude that \( n \) is even.

  \NecessitySubProof Suppose that \( G = (V, E) \) has no cycles of odd length.

  If \( V \) is empty, \( G \) is vacuously bipartite. Suppose that \( V \) is nonempty. Let \( G_1, \ldots, G_s \) be the connected components of \( G \).

  Fix a vertex \( u \) from \( G_k \). Let \( A_k \) be the set of all vertices from \( G_k \) reachable from \( u \) with even-length paths and let \( B_k \) be the complement of \( A_k \) in \( V \). We will show that both \( A_k \) and \( B_k \) are independent.

  Aiming at a contradiction, suppose that there exist adjacent vertices \( v \) and \( w \) in \( A_k \). We can concatenate even-length paths from \( w \) to \( u \) and from \( u \) to \( v \) to obtain an even-length path from \( w \) to \( v \). Appending the edge \( \set{ v, w } \), we obtain an odd-length cycle at \( w \), whose existence contradicts our initial assumption. Hence, \( A_k \) is an independent set.

  Similarly, suppose that there exist adjacent vertices \( v \) and \( w \) in \( B_k \). Then we can concatenate odd-length paths from \( w \) to \( u \) and from \( u \) to \( v \) to obtain an even-length path from \( w \) to \( v \). Appending \( \set{ u, v } \), we again obtain an odd-length cycle at \( w \), whose existence again leads to a contradiction. Hence, \( B_k \) is also an independent set.

  We can now take the unions \( A \coloneqq A_1 \cup \cdots \cup A_s \) and \( B \coloneqq B_1 \cup \cdots \cup B_s \). These are again independent sets because \( A_i \) and \( A_j \) belong to different connected components if \( i \neq j \).

  Then \( V = A \cup B \) is the desired partition of \( V \), which demonstrates that \( G \) is bipartite.
\end{proof}

\begin{definition}\label{def:complete_subgraph}\mimprovised
  We say that the \hyperref[def:undirected_graph/subgraph]{subgraph} \( H \) of the \hyperref[def:undirected_graph]{simple undirected graph} \( G \) is \term{complete} if any of the following equivalent conditions hold:
  \begin{thmenum}
    \thmitem{def:complete_subgraph/direct} There is an edge in \( H \) between any pair of vertices in \( H \).

    \thmitem{def:complete_subgraph/isomorphic} \( H \) is \hyperref[def:undirected_graph/homomorphism]{isomorphic} to the \hyperref[def:complete_graph]{complete graph} \( K_\kappa \), where \( \kappa \) is the (cardinal) \hyperref[def:graph_cardinality/order]{order} of \( H \).

    \thmitem{def:complete_subgraph/degree} \( H \) has \( \kappa + 1 \) vertices and each of them has \hyperref[def:graph_cardinality/undirected_degree]{degree} \( \kappa \) (in \( H \)).
  \end{thmenum}
\end{definition}
\begin{proof}
  \ImplicationSubProof{def:complete_subgraph/direct}{def:complete_subgraph/isomorphic} Let \( \seq{ H_\alpha }_{\alpha < \kappa} \) be an enumeration of the vertices of \( H \). \Fullref{thm:complete_graph_universal_property} shows that this (transfinite) sequence is itself a homomorphism from \( H \) to \( K_\kappa \). Furthermore, its inverse is also a homomorphism because \( H \) has an edge between every pair of vertices. Therefore, \( H \) and \( K_\kappa \) are isomorphic.

  \ImplicationSubProof{def:complete_subgraph/isomorphic}{def:complete_subgraph/degree} Fix an isomorphism \( f: V_H \to V_{K_{\kappa + 1}} \). Fix any vertex \( v \) in \( H \). Except for \( f(v) \), all \( \kappa \) other vertices in \( K_{\kappa + 1} \) are adjacent to \( f(v) \), thus the degree of \( f(v) \) is \( \kappa \).

  Given a vertex \( i \) of \( K_{\kappa + 1} \) adjacent to \( f(v) \), \( f^{-1}(i) \) is adjacent to \( v \) as per \eqref{eq:thm:graph_isomorphisms/simple_undirected}. Therefore, \( v \) has degree \( \kappa \) in \( H \).

  Generalizing on \( v \), we conclude that every vertex of \( H \) has degree \( \kappa \) in \( H \).

  \ImplicationSubProof{def:complete_subgraph/degree}{def:complete_subgraph/direct} Suppose that \( H \) has \( \kappa + 1 \) vertices, each of degree \( \kappa \).

  Fix a vertex \( v \) in \( H \). Since \( G \) does not allow loops or parallel edges, neither does \( H \), and thus \( v \) must have \( \kappa \) adjacent vertices from \( H \). But there are \( \kappa \) vertices in \( H \) distinct from \( v \), thus \( v \) is adjacent in \( H \) to any other vertex of \( H \).

  Since \( v \) was arbitrary, we conclude that \( H \) has an edge between every pair of vertices.
\end{proof}

\begin{proposition}\label{thm:multipartite_graph_complete_subgraph}
  If \hyperref[def:undirected_graph]{simple undirected graph} \( G \) has a \hyperref[def:complete_subgraph]{complete subgraph} of order \( k+1 \), then \( G \) is not \hyperref[def:multipartite_graph]{\( k \)-partite}.
\end{proposition}
\begin{comments}
  \item The converse does not hold --- see \fullref{ex:thm:multipartite_graph_complete_subgraph/converse}.

  \item A partial \hyperref[def:conditional_formula/inverse]{inverse proposition} that only regards the number of edges is \fullref{thm:turans_theorem}.
\end{comments}
\begin{proof}
  Let \( H \) be a complete subgraph of \( G \) of order \( k + 1 \).

  Let \( V_1, \ldots, V_k \) be \hi{any} partition of the vertices of \( G \). By \fullref{thm:pigeonhole_principle}, there exists an index \( i \) such that \( V_i \) contains at least two vertices of \( H \). But then \( V_i \) is not an independent set.

  Therefore, \( G \) cannot possibly be \( k \)-partite.
\end{proof}

\begin{definition}\label{def:triangle_graph}\mcite[3]{Diestel2005}
  We call the graph \( K_3 = C_3 \) the \term{triangle graph} and any \hyperref[def:complete_subgraph]{complete subgraph} of order \( 3 \) --- a \term{triangle subgraph}.
\end{definition}

\begin{example}\label{ex:thm:multipartite_graph_complete_subgraph}
  We list several examples related to \fullref{thm:multipartite_graph_complete_subgraph}:
  \begin{thmenum}
    \thmitem{ex:thm:multipartite_graph_complete_subgraph/triangle} The graph \( C_4 \) is bipartite, as can be seen in \cref{fig:thm:square_graph}.

    Consider the following supergraph of \( C_4 \):
    \begin{equation}\label{eq:ex:thm:multipartite_graph_complete_subgraph/triangle}
      \begin{aligned}
        \includegraphics[page=1]{output/ex__thm__multipartite_graph_complete_subgraph}
      \end{aligned}
    \end{equation}

    It has two \hyperref[def:triangle_graph]{triangle subgraphs}, hence it is not bipartite.

    \thmitem{ex:thm:multipartite_graph_complete_subgraph/converse} The converse of \fullref{thm:multipartite_graph_complete_subgraph} does not hold.

    As discussed in \fullref{ex:def:multipartite_graph/cycle}, the cycle graph \( K_5 \) is not bipartite. It does not contain \hyperref[def:triangle_graph]{triangles}, however, and the converse of \fullref{thm:multipartite_graph_complete_subgraph} would imply that it is bipartite.
  \end{thmenum}
\end{example}

\begin{definition}\label{def:complete_multipartite_graph}\mcite[33]{Harary1969}
  For a positive integer \( k \), we define the \term[ru=полный \( k \)-дольный (граф) (\cite[\S 1]{Емеличев1990})]{complete \( k \)-partite} graph \( K_{V_1,\ldots,V_k} \) as the \hyperref[def:graph_join]{graph join}
  \begin{equation*}
    \oline{K_{V_1}} \Ast \oline{K_{V_2}} \Ast \cdots \Ast \oline{K_{V_k}}
  \end{equation*}
  of \( k \) \hyperref[def:edgeless_graph]{edgeless graphs}.

  \begin{figure}[!ht]
    \hfill
    \includegraphics[page=1]{output/def__complete_multipartite_graph}
    \hfill
    \includegraphics[page=2]{output/def__complete_multipartite_graph}
    \hfill
    \hfill
    \caption{The \hyperref[def:complete_multipartite_graph]{complete bipartite graphs} \( K_{1,5} \) and \( K_{3,3} \).}\label{fig:def:complete_multipartite_graph}
  \end{figure}
\end{definition}
\begin{comments}
  \item If we plain unions rather than disjoint unions in the definition of graph joins, we would have to ensure that \( V_1, \ldots, V_k \) are disjoint. This would make the definition of \( K_{n,m} \) more tricky that it has to be.
\end{comments}

\begin{proposition}\label{thm:complete_multipartite_subgraphs}
  A \hyperref[def:undirected_graph]{simple undirected graph} \( G = (V, E) \) is \hyperref[def:multipartite_graph]{\( k \)-partite} if and only if there exists a partition \( V_1, \ldots, V_k \) of \( V \) such that \( G \) is (isomorphic to) a subgraph of \( K_{V_1, \ldots, V_k} \).
\end{proposition}
\begin{proof}
  \SufficiencySubProof Let \( V_1, \ldots, V_n \) be a partition of \( V \) into independent subsets. Every edge \( e \) in \( E \) uniquely corresponds to an edge \( \iota^E(e) \) in \( K_{V_1, \ldots, V_n} \). Thus, the graph \( (\iota^V[V], \iota^E[E]) \) is a subgraph of \( K_{V_1, \ldots, V_n} \).

  \NecessitySubProof Conversely, if \( G \) is (isomorphic to) a subgraph of \( K_{V_1, \ldots, V_n} \), it must itself be \( k \)-partite because otherwise \( K_{V_1, \ldots, V_n} \) would not be.
\end{proof}

\begin{proposition}\label{thm:small_complete_multipartite_graph}
  The \hyperref[def:complete_graph]{complete graph} \( K_n \) is isomorphic to the \hyperref[def:complete_multipartite_graph]{complete multipartite graph}
  \begin{equation*}
    K_{\underbrace{1,\ldots,1}_{n \T*{times}}}.
  \end{equation*}
\end{proposition}
\begin{proof}
  An explicit \hyperref[def:undirected_graph/homomorphism]{isomorphism} is
  \begin{equation*}
    \begin{aligned}
      &f: K_n \to K_{1,\ldots,1} \\
      &f(i) \coloneqq \iota^V_i(1).
    \end{aligned}
  \end{equation*}
\end{proof}

\begin{proposition}\label{thm:square_graph}
  The \hyperref[def:cycle_graph]{cycle graph} \( C_4 \) is \hyperref[def:undirected_graph/homomorphism]{isomorphic} to the \hyperref[def:complete_multipartite_graph]{complete bipartite graph} \( K_{2,2} \).

  \begin{figure}[!ht]
    \begin{subcaptionblock}{\textwidth/2}
      \centering
      \includegraphics[page=1]{output/thm__square_graph}
      \caption{\( C_4 \)}\label{fig:thm:square_graph/c4}
    \end{subcaptionblock}
    \hfill
    \begin{subcaptionblock}{\textwidth/2}
      \centering
      \includegraphics[page=2]{output/thm__square_graph}
      \caption{\( K_{2,2} \)}\label{fig:thm:square_graph/k22}
    \end{subcaptionblock}
    \caption{An explicit isomorphism for the graphs in \fullref{thm:square_graph}.}\label{fig:thm:square_graph}
  \end{figure}
\end{proposition}
\begin{proof}
  An explicit isomorphism is illustrated in \cref{fig:thm:square_graph}.
\end{proof}

\begin{definition}\label{def:turan_graph}\mimprovised
  For any two integers \( k > 0 \) and \( n \geq 0 \), we define the \term{Turan graph} \( T_{n,k} \) as the \hyperref[def:complete_multipartite_graph]{complete \( k \)-partite graph}
  \begin{equation*}
    K_{\underbrace{q+1,\ldots,q+1}_{r \T*{times}},\underbrace{q,\cdots,q}_{n - r \T*{times}}},
  \end{equation*}
  where \( n = qk + r \) is the result of \fullref{alg:integer_division}.

  \begin{figure}[!ht]
    \begin{subcaptionblock}{\textwidth/2}
      \centering
      \includegraphics[page=1]{output/def__turan_graph}
      \caption{\( K_5 \)}\label{fig:def:turan_graph/k5}
    \end{subcaptionblock}
    \hfill
    \begin{subcaptionblock}{\textwidth/2}
      \centering
      \includegraphics[page=2]{output/def__turan_graph}
      \caption{\( T_{5,3} \)}\label{fig:def:turan_graph/t52}
    \end{subcaptionblock}
    \caption{An illustration of how \hyperref[def:turan_graph]{Turan graphs} are the closest multipartite analogues of \hyperref[def:complete_graph]{complete graphs}.}\label{fig:def:turan_graph}
  \end{figure}
\end{definition}
\begin{comments}
  \item We base our notation and terminology on \cite[165]{Diestel2005}, however Diestel defines the Turan graph \( T_{n,k} \) as \enquote{the unique} \( k \)-partite graph whose partition classes differ by at most \( 1 \) vertex. This only characterizes Turan graphs up to an \hyperref[thm:graph_isomorphisms/simple_undirected]{isomorphism}.
\end{comments}

\begin{proposition}\label{thm:small_turan_graph}
  If \( 0 \leq n \leq k \), the \hyperref[def:turan_graph]{Turan graph} \( T_{n,k} \) is \hyperref[def:undirected_graph/homomorphism]{isomorphic} to the \hyperref[def:complete_graph]{complete graph} \( K_n \).
\end{proposition}
\begin{proof}
  If \( n < k \), then \( n = 0 \cdot k + r \), thus
  \begin{equation*}
    T_{n,k} = K_{\underbrace{1,\ldots,1}_{r \T*{times}},\underbrace{0,\cdots,0}_{n - r \T*{times}}}.
  \end{equation*}

  Otherwise, if \( n = k \),then
  \begin{equation*}
    T_{n,k} = K_{\underbrace{1,\ldots,1}_{n \T*{times}}}.
  \end{equation*}

  In both cases, by \fullref{thm:small_complete_multipartite_graph}, \( T_{n,k} \) is isomorphic to \( K_n \).
\end{proof}

\begin{lemma}\label{thm:multipartite_graph_degree}
  In an \hyperref[def:multipartite_graph]{\( k \)-partite} graph with vertex partition \( V_1, \ldots, V_k \), for every vertex \( v \in V_i \) we have
  \begin{equation*}
    \deg(v) \leq \sum_{j \neq i} \card(V_j),
  \end{equation*}
  with equality holding for \hyperref[def:complete_multipartite_graph]{complete \( k \)-partite graphs}.
\end{lemma}
\begin{proof}
  Equality for complete \( k \)-partite graphs easily follows by induction. The general inequality then follows from \fullref{thm:complete_multipartite_subgraphs}.
\end{proof}

\begin{proposition}\label{thm:turan_graph_edge_count}
  The maximum number of edges in a \hyperref[def:multipartite_graph]{\( k \)-partite} graph of finite order \( n \) is
  \begin{equation}\label{eq:thm:turan_graph_edge_count}
    \frac {k - 1} k \cdot \frac {n^2 - r^2} 2 + \binom r 2,
  \end{equation}
  where \( n = qk + r \) is the result of \fullref{alg:integer_division}.

  Furthermore, equality holds in \eqref{eq:thm:turan_graph_edge_count} for the \hyperref[def:turan_graph]{Turan graph} \( T_{n,k} \).
\end{proposition}
\begin{comments}
  \item If \( n < k \), then \( n = r \) and \eqref{eq:thm:turan_graph_edge_count} reduces to \( \binom n 2 \), which by \fullref{thm:complete_graph_edge_count} is the edge count of \( K_n \).
\end{comments}
\begin{proof}
  Let \( V_1, \ldots, V_k \) be a partition of the graph \( G \) into independent sets. Denote by \( n_i \) the cardinality of \( V_i \) and by \( m \) --- the number of edges in \( G \).

  \Fullref{thm:sum_of_graph_degrees} implies that
  \begin{equation*}
    m = 2 \cdot \sum_{i=1}^k \sum_{v \in V_i} \deg(v).
  \end{equation*}

  For \( v \in V_i \), \fullref{thm:multipartite_graph_degree} implies that
  \begin{equation*}
    \deg(v) \leq \underbrace{\sum_{j \neq i} n_j}_{n - n_i}.
  \end{equation*}

  Thus,
  \begin{equation}\label{eq:thm:turan_graph_edge_count/proof/degree_ineq}
    m \leq \frac 1 2 \cdot \sum_{i=1}^k \underbrace{\sum_{v \in V_i} (n - n_i)}_{n_i (n - n_i)}.
  \end{equation}

  We want to maximize \( m \). \Fullref{thm:multipartite_graph_degree} implies that equality is attained in \eqref{eq:thm:turan_graph_edge_count/proof/degree_ineq} for complete multipartite graphs, thus \( G \) must be complete \( k \)-partite.

  It remains to determine values for \( n_1, \ldots, n_k \) for which \eqref{eq:thm:turan_graph_edge_count/proof/degree_ineq} is maximized.

  The total change in the number of edges after moving a vertex from \( V_i \) to \( V_j \) is
  \begin{align*}
    &\phantom{{}={}}
    \frac 1 2 \cdot \parens[\Big]{ (n_i - 1)(n - n_i + 1) - n_i(n - n_i) } + \frac 1 2 \cdot \parens[\Big]{ (n_j + 1)(n - n_j - 1) - n_j(n - n_j) }
    = \\ &=
    \frac 1 2 \cdot \parens[\Big]{ n_i(n - n_i) + n_i - n + n_i - 1 - n_i(n - n_i) } + \cdots
    = \\ &=
    \frac 1 2 \cdot \parens[\Big]{ 2n_i - n - 1 } + \frac 1 2 \cdot \parens[\Big]{ -2n_j + n - 1 }
    = \\ &=
    n_i - n_j - 1.
  \end{align*}

  Thus,
  \begin{itemize}
    \item \( m \) increases if \( n_i > n_j + 1 \).
    \item \( m \) decreases if \( n_i \leq n_j \).
    \item \( m \) does not change if \( n_i = n_j + 1 \).
  \end{itemize}

  We can thus conclude that the total number of edges can only increase if there exist \( i \) and \( j \) such that \( n_i > n_j + 1 \). Therefore, the Turan graph \( T_{n,k} \) has the maximal amount of edges possible for a \( k \)-partite graph.

  For \( T_{n,k} \), \eqref{eq:thm:turan_graph_edge_count/proof/degree_ineq} becomes
  \begin{equation}\label{eq:thm:turan_graph_edge_count/proof/degree_turan}
    m = \frac 1 2 \cdot \underbrace{\sum_{i=1}^r (q+1) (n - (q+1))}_{r(q+1) (n - q-1)} + \frac 1 2 \cdot \underbrace{\sum_{i=r+1}^k q (n - q)}_{(k-r) q (n - q)}.
  \end{equation}

  It can be simplified as follows:
  \begin{balign*}
    m
    &=
    \frac {r(q+1) (n - q-1) + (k-r) q (n - q)} 2
    = \\ &=
    \frac {rq(n-q-1) + r(n-q-1) + (k-r)q(n-q)} 2
    = \\ &=
    \frac {\hi{rq(n-q)} - rq + r(n-q-1) + \hi{(k-r)q(n-q)}} 2
    = \\ &=
    \frac {kq(n-q) + r(n-q) - r(q+1)} 2
    = \\ &=
    \frac {(kq+r)(n-q) - r(q+1)} 2
    = \\ &=
    \frac {n(n-q) - r(q+1)} 2.
  \end{balign*}

  Since \( q = (n - r) / k \), we can further simplify as follows:
  \begin{balign*}
    m
    &=
    \frac {n((k-1)n+r) - r(n-r+k)} {2k}
    = \\ &=
    \frac {(k-1)n^2 + \hi{rn} - \hi{rn} + r^2 - rk} {2k}
    = \\ &=
    \frac {(k-1)n^2 - (k-1-k) r^2 - rk} {2k}
    = \\ &=
    \frac {k-1} k \cdot \frac {n^2 - r^2} 2 - \frac {r^2 - r} 2,
  \end{balign*}
  which is precisely \eqref{eq:thm:turan_graph_edge_count}.
\end{proof}

\begin{theorem}[Turan's theorem]\label{thm:turans_theorem}\mcite[7.1.1]{Diestel2005}
  For any two integers \( k > 0 \) and \( n \geq 0 \), if a \hyperref[def:undirected_graph]{simple undirected graph} with \( n \) vertices has no \hyperref[def:complete_subgraph]{complete subgraph} of order \( k + 1 \), then it cannot have more edges than the \hyperref[def:turan_graph]{Turan graph} \( T_{n,k} \).
\end{theorem}
\begin{comments}
  \item The exact bound on the number of edges is \eqref{eq:thm:turan_graph_edge_count}.

  \item The important thing to note is that, although the graph may not be \( k \)-partite, the number of edges has the same bound as a \( k \)-partite graph.
\end{comments}
\begin{proof}
  We will use induction on \( n \).

  \SubProof{Proof of base case} If \( n \leq k \), \fullref{thm:small_turan_graph} implies that \( T_{n,k} \) is isomorphic to \( K_n \).

  \Fullref{thm:complete_graph_edge_count} implies that any graph \( G \) of order \( n \) has its edge count is bounded by \( \binom n 2 \), which is precisely the edge count of \( T_{n,k} \).

  \SubProof{Proof of inductive step} Let \( n > k \) and suppose that the theorem holds for graphs of order less than \( n \). Let \( G = (V, E) \) be a graph of order \( n \) with no complete subgraph of order \( k + 1 \).

  Since no subgraph of \( G \) is isomorphic to \( K_{k+1} \), then no subgraph satisfies \fullref{def:complete_subgraph/degree}, and thus there exists a vertex of degree at most \( k - 1 \). Let \( v \) be such a vertex. Define
  \begin{equation*}
    G' \coloneqq \parens[\Big]{ V \setminus \set{ v }, E \setminus \set{ e \given v \in e } }.
  \end{equation*}

  Then \( G' \) is a graph of order \( n - 1 \) with no complete subgraph of order \( k + 1 \). Hence, the inductive hypothesis is applicable to \( G' \).
  \begin{itemize}
    \item If \( k \) divides \( n \), by inductive hypothesis, the bound on the edge count of \( G' \) is
    \begin{equation*}
      \frac {k - 1} k \cdot \frac {(n - 1)^2 - (k - 1)^2} 2 + \binom {k - 1} 2.
    \end{equation*}

    This can be expanded as follows:
    \begin{equation*}
      \underbrace{\frac {k - 1} k \cdot \frac {n^2} 2}_b + \underbrace{\frac {k - 1} {2k} \cdot (-2n + \cancel{1} - k^2 + 2k - \cancel{1}) + \binom {k - 1} 2}_e.
    \end{equation*}

    The highlighted part \( b \) is the bound we want to obtain on \( G \). We must thus show that \( e \) is at most \( -k + 1 \) because at most \( k - 1 \) edges are removed from \( G \). We have
    \begin{align*}
      e
      &=
      \frac {k - 1} {2k} \cdot (-2n - k^2 + 2k) + \binom {k - 1} 2
      = \\ &=
      -\frac {k - 1} {2k} \cdot \parens[\Big]{ k(k - 2) + 2n ) + \frac {(k - 1)(k - 2)} 2 }
      = \\ &=
      \cancel{-\frac {(k - 1)(k - 2)} 2} - \frac {k - 1} k n + \cancel{\frac {(k - 1)(k - 2)} 2}.
    \end{align*}

    Since \( k \) divides \( n \) and \( k < n \), \( n / k \) is an integer greater than \( 1 \), thus
    \begin{equation*}
      e = - \frac {k - 1} k n = - \frac n k (k - 1) < -(k - 1) = -k + 1.
    \end{equation*}

    \item If \( k \) does not divide \( n \), then the remainder \( r \) is positive and the bound on the edge count of \( G' \) is
    \begin{equation*}
      \frac {k - 1} k \cdot \frac {(n - 1)^2 - (r - 1)^2} 2 + \binom {r - 1} 2.
    \end{equation*}

    This can be expanded as follows:
    \begin{equation*}
      \underbrace{\frac {k - 1} k \cdot \frac {n^2 - r^2} 2 + \binom r 2}_b + \underbrace{\frac {k - 1} {2k} \cdot (-2n + \cancel{1} + 2r - \cancel{1}) - (r - 1)}_e.
    \end{equation*}

    Again, we must show that \( e \leq -k \). We have
    \begin{equation*}
      e = \frac {k - 1} {2k} \cdot (-2n + 2r) - (r - 1) = -\frac {k - 1} k (n - r) - (r - 1).
    \end{equation*}

    Since \( k \) divides \( n - r \), their quotient is a positive integer. Thus,
    \begin{equation*}
      e \leq -(k - 1) - \underbrace{(r - 1)}_{\geq 0} \leq -k + 1.
    \end{equation*}
  \end{itemize}

  On both cases, we conclude that \( e \leq -k + 1 \) and thus the bound for the edge count of \( G \) coincides the edge count for \( T_{n,k} \) from \eqref{eq:thm:turan_graph_edge_count}.
\end{proof}
