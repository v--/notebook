\section{Real numbers}\label{sec:real_numbers}

\paragraph{Field of real numbers}

\begin{definition}\label{def:real_numbers}\mcite[sec. I.V.]{Dedekind1901Essays}
  We define the set \( \BbbR \) of \term[bg=реални числа (\cite[ch. I]{Тагамлицки1971ДиференциалноСмятане}), ru=вещественные числа (\cite[\S 3]{Фихтенгольц1968ОсновыАнализаТом1}) / действительные числа (\cite[def. 25.1]{АлександровМаркушевичХинчинИПр1951ЭнциклопедияТом1})]{real numbers} as the \hyperref[def:lower_cut_completion]{unbounded lower cut completion} of the set \( \BbbQ \) of \hyperref[def:rational_numbers]{rational numbers}.
\end{definition}
\begin{comments}
  \item We consider the following an indispensable part of \( \BbbR \):
  \begin{itemize}
    \item The field structure defined in \cref{def:real_number_arithmetic}.
    \item The order topology defined via \cref{def:order_topology}.
  \end{itemize}
\end{comments}

\begin{definition}\label{def:real_number_arithmetic}\mimprovised
  For each pair of \hyperref[def:real_numbers]{real numbers} \( A \) and \( B \), considered as \hyperref[def:lower_cut]{lower cuts} of \hyperref[def:rational_numbers]{rational numbers}, we define the base operations:
  \begin{thmenum}
    \thmitem{def:real_number_arithmetic/zero}\mcite[325]{Mendelson2008NumberSystems} Zero:
    \begin{equation}\label{eq:def:real_number_arithmetic/zero}
      O \coloneqq \BbbQ_{< 0} = \set{ s \given s < 0 }.
    \end{equation}

    \thmitem{def:real_number_arithmetic/additive_inverse}\mcite[117]{Enderton1977SetTheory} Additive inverses:
    \begin{equation}\label{eq:def:real_number_arithmetic/additive_inverse}
      \boxminus A \coloneqq \set{ -s \given \qexists*{ r \in \BbbQ \setminus A } r < s }
    \end{equation}

    It is tempting to simply take the inverses of all upper bounds of \( A \), i.e.
    \begin{equation*}
      \set{ -s \given s \in \BbbQ \setminus A },
    \end{equation*}
    however we want \( \boxminus A \) to not have a maximum, and our additional condition excludes the infimum of \( \BbbQ \setminus A \).

    \thmitem{def:real_number_arithmetic/addition}\mcite[326]{Mendelson2008NumberSystems} Addition:
    \begin{equation}\label{eq:def:real_number_arithmetic/addition}
      A \boxplus B \coloneqq \set{ a + b \given a \in A \T{and} b \in B }.
    \end{equation}

    \thmitem{def:real_number_arithmetic/one}\mcite[325]{Mendelson2008NumberSystems} One:
    \begin{equation}\label{eq:def:real_number_arithmetic/one}
      I \coloneqq \BbbQ_{< 1} = \set{ s \given s < 1 }.
    \end{equation}

    \thmitem{def:real_number_arithmetic/multiplicative_inverse}\mcite[329]{Mendelson2008NumberSystems} One: Multiplicative inverses (for \( A \neq O \)):
    \begin{empheq}[left=A^\boxast \coloneqq \empheqlbrace]{align}
        O \cup \set{ 0 } \cup \set{ s^{-1} \given \qexists*{ r \in \BbbQ \setminus A } r < s }, &A \supsetneq O, \label{eq:def:real_number_arithmetic/multiplicative_inverses/pos} \\
        \boxminus ((\boxminus A)^\boxast)                                                       &A \subsetneq O. \label{eq:def:real_number_arithmetic/multiplicative_inverses/neg}
    \end{empheq}

    Note that, in \eqref{eq:def:real_number_arithmetic/multiplicative_inverses/pos} \( \BbbQ \setminus A \) contains only positive rational numbers, so all reciprocals will also be positive. For this reason we must add \( 0 \) and all negative numbers.

    \thmitem{def:real_number_arithmetic/multiplication}\mcite[329]{Mendelson2008NumberSystems} Multiplication:
    \begin{empheq}[left=A \boxtimes B \coloneqq \empheqlbrace]{align}
        O,                                                                               &O = A \T{or} O = B,                    \label{eq:def:real_number_arithmetic/multiplication/zero} \\
        O \cup \set{ a \cdot b \given a \in A \setminus O \T{and} b \in B \setminus O }, &O \subsetneq A \T{and} O \subsetneq B, \label{eq:def:real_number_arithmetic/multiplication/pos} \\
        \boxminus ((\boxminus A) \boxtimes B),                                           &O \supsetneq A \T{and} O \subsetneq B, \label{eq:def:real_number_arithmetic/multiplication/pos_a} \\
        \boxminus (A \boxtimes (\boxminus B)),                                           &O \subsetneq A \T{and} O \supsetneq B, \label{eq:def:real_number_arithmetic/multiplication/pos_b} \\
        (\boxminus A) \boxtimes (\boxminus B),                                           &O \supsetneq A \T{and} O \supsetneq B. \label{eq:def:real_number_arithmetic/multiplication/neg} \\
    \end{empheq}

    The different cases are adjusted so that \cref{thm:def:signum} holds. If multiplication is always treated like in \eqref{eq:def:real_number_arithmetic/multiplication/pos}, we would stumble upon a problem --- since \( I \) contains all negative rational numbers, when multiplying \( I \) by \( -I \), the result would coincide with \( \BbbQ \).
  \end{thmenum}
\end{definition}
\begin{defproof}
  We defer to \cref{thm:field_of_real_numbers} the proof that \( \BbbR \) becomes a field with these operations. We will only show here that the operations are well-defined, i.e. that they produce lower cuts.

  \SubProofOf{def:real_number_arithmetic/zero} \( O \) is the embedding of \( 0 \) into \( \BbbR \), so naturally it is a lower cut.

  \SubProofOf{def:real_number_arithmetic/additive_inverse} Given a lower cut \( A \), we need to show that \( \boxminus A \) is also a lower cut.

  \SubProofOf*[def:closed_ordered_subset]{downward closure} Fix \( -s \in \boxminus A \) and let \( t < -s \). \Cref{thm:def:signum} implies that \( s < -t \).

  By definition of \( \boxminus A \), there exists some upper bound \( r < s \) of \( A \). Then \( -t > r \) is also an upper bound, hence \( t \in \boxminus A \).

  \SubProof*{Proof of no maximum} Suppose that \( -s \) is a maximum of \( \boxminus A \). By definition, there exists some upper bound \( r < s \) of \( A \). By density, there also exists some \( t \in (r, s) \). Then \( -t > -s \) is also in \( \boxminus A \), contradicting the maximality of \( s \).

  \SubProofOf{def:real_number_arithmetic/addition} Given lower cuts \( A \) and \( B \), we need to show that \( A \boxplus B \) is also a lower cut.

  \SubProofOf*[def:closed_ordered_subset]{downward closure} Fix some element \( a + b \) of \( A \boxplus B \) and let \( s < a + b \).

  Define \( t \coloneqq a + b - s \) so that \( s = a + (b - t) \). Since \( t > 0 \) and since \( B \) is downward closed, we conclude that \( b - t \) is in \( B \). Then \( s \) is a sum of an element of \( A \) and an element of \( B \), implying that it belongs to \( A \boxplus B \). Therefore, \( A \boxplus B \) is downward closed.

  \SubProof*{Proof of no maximum} Suppose that \( a + b \) is a maximum of \( A \boxplus B \). Since \( A \) has no maximum, there exists some \( a' > a \) in \( A \). Then \( a' + b > a + b \) is also in \( A \boxplus B \), contradicting the maximality of \( a + b \).

  \SubProofOf{def:real_number_arithmetic/one} \( I \) is the embedding of \( 1 \) into \( \BbbR \).

  \SubProofOf{def:real_number_arithmetic/multiplicative_inverse} Given a lower cut \( A \), we need to show that \( A^\boxast \) is also a lower cut. We will restrict ourselves to the case where \( A \supsetneq O \) since the other case reduces to it.

  \SubProofOf*[def:closed_ordered_subset]{downward closure} If \( p \in A^\boxast \), either \( p \leq 0 \) or there exists some \( r \in \BbbQ \setminus A \) such that \( p^{-1} > r \).

  \begin{itemize}
    \item If \( p \leq 0 \) and \( t < p \), clearly \( t < 0 \) also belongs to \( A^\boxast \).

    \item Otherwise, \( p^{-1} > r \) for some \( r \in \BbbQ \setminus A \).

    Let \( t < p \). If \( t < 0 \), it is again clear that \( t \in A^\boxast \). Otherwise, \( 0 < t < p \) and \cref{thm:ordered_ring/reciprocal_inversion} implies that \( 0 < r < p^{-1} < t^{-1} \), hence \( t \) must belong to \( A^\boxast \).
  \end{itemize}

  \SubProof*{Proof of no maximum} Suppose that \( s^{-1} \) is a maximum of \( A^\boxast \). Then there exists some upper bound \( r \) of \( A \) such that \( s > r \). By density, there exists some \( t \in (r, s) \), thus \( t^{-1} \in A^\boxast \).

  But \cref{thm:ordered_ring/reciprocal_inversion} implies that \( s^{-1} < t^{-1} \), contradicting the maximality of \( s^{-1} \).

  \SubProofOf{def:real_number_arithmetic/multiplication} Given lower cuts \( A \) and \( B \), we need to show that \( A \boxtimes B \) is also a lower cut. This is clear when \( A = O \) or \( B = O \) since \( A \boxtimes B = O \). All other cases reduce to the case where \( A \supsetneq O \) and \( B \supsetneq O \), so we only need to consider the latter.

  \SubProofOf*[def:closed_ordered_subset]{downward closure} Fix some element \( a \cdot b \) of \( A \boxtimes B \) and let \( s < a \cdot b \).

  \begin{itemize}
    \item Suppose that \( s < 0 \). By definition, \( A \boxtimes B \) contains \( O \), thus \( s \in A \boxtimes B \).

    \item Suppose that \( s = 0 \). By assumption, both \( A \) and \( B \) contain positive rational numbers. Since both are downward closed, they both contain zero, thus \( s \in A \boxtimes B \).

    \item Suppose that \( s > 0 \).

    Then \( a \cdot b > 0 \) and \cref{thm:def:signum} implies that \( a \) and \( b \) have matching signs. If both are negative, we can consider \( -a \) and \( -b \), so, without loss of generality, suppose that both \( a \) and \( b \) are positive.

    Define \( t \coloneqq s / ab \). Then \( s = (t \cdot a) \cdot b \). Since \( t < 1 \), we conclude that \( t \cdot a < a \). Then \( A \) contains \( t \cdot a \). Therefore, \( s \) belongs to \( A \boxtimes B \) as a product of elements of \( A \) and \( B \).
  \end{itemize}

  \SubProof*{Proof of no maximum} Suppose that \( a \cdot b \) is a maximum of \( A \boxtimes B \). Then \( a \cdot b > 0 \). Without loss of generality, suppose that \( a > 0 \) and \( b > 0 \).

  Since \( A \) has no maximum, there exists some \( a' > a \) in \( A \). \Cref{thm:def:ordered_semigroup/cancellative} implies that \( a' \cdot b > a \cdot b \). But \( a' \cdot b \) is also in \( A \boxplus B \), contradicting the maximality of \( a \cdot b \).
\end{defproof}

\begin{proposition}\label{thm:field_of_real_numbers}
  With the operations from \cref{def:real_number_arithmetic}, the set \( \BbbR \) of \hyperref[def:real_numbers]{real numbers} becomes an \hyperref[def:ordered_semiring]{ordered} \hyperref[def:field]{field} extending the \hyperref[def:rational_numbers]{rational numbers}.

  We will prove the following for addition:
  \begin{thmenum}[series=thm:field_of_real_numbers]
    \thmitem{thm:field_of_real_numbers/addition_associative} Addition is \hyperref[def:binary_operation/associative]{associative}:
    \begin{equation}\label{eq:thm:field_of_real_numbers/addition_associative}
      (A \boxplus B) \boxplus C = A \boxplus (B \boxplus C)
    \end{equation}

    \thmitem{thm:field_of_real_numbers/addition_commutative} Addition is \hyperref[def:binary_operation/commutative]{commutative}:
    \begin{equation}\label{eq:thm:field_of_real_numbers/addition_commutative}
      A \boxplus B = B \boxplus A.
    \end{equation}

    \thmitem{thm:field_of_real_numbers/addition_order} \( \BbbR \) is an \hyperref[def:ordered_semigroup]{ordered semigroup} with respect to addition: when \( A \subseteq B \), for any \( C \) we have
    \begin{equation}\label{eq:thm:field_of_real_numbers/addition_order}
      A \boxplus C \subseteq B \boxplus C.
    \end{equation}

    \thmitem{thm:field_of_real_numbers/addition_neutral}  \( O \) acts as a \hyperref[def:monoid]{neutral element} of addition:
    \begin{equation}\label{eq:thm:field_of_real_numbers/addition_neutral}
      A \boxplus O = A.
    \end{equation}

    \thmitem{thm:field_of_real_numbers/zero_additive_inverse} The additive inverse of zero is zero:
    \begin{equation}\label{eq:thm:field_of_real_numbers/zero_additive_inverse}
      \boxminus O = O.
    \end{equation}

    \thmitem{thm:field_of_real_numbers/addition_invertible} Addition is \hyperref[def:monoid_inverse]{invertible}:
    \begin{equation}\label{eq:thm:field_of_real_numbers/addition_inverse}
      A \boxplus (\boxminus A) = O.
    \end{equation}

    \thmitem{thm:field_of_real_numbers/addition_sign} Additive inverses swap their sign:
    \begin{equation}\label{eq:thm:field_of_real_numbers/addition_sign}
      A \supseteq O \T{implies} \boxminus A^\boxast \subseteq O \T{and} A \subseteq O \T{implies} \boxminus A^\boxast \supseteq O.
    \end{equation}
  \end{thmenum}

  We will prove the following for multiplication:
  \begin{thmenum}[resume=thm:field_of_real_numbers]
    \thmitem{thm:field_of_real_numbers/multiplication_and_additive_inverses} Multiplication respects additive inverses:
    \begin{equation}\label{eq:thm:field_of_real_numbers/multiplication_and_additive_inverses}
      A \boxtimes (\boxminus B) = (\boxminus A) \boxtimes B = \boxminus (A \boxtimes B).
    \end{equation}

    \thmitem{thm:field_of_real_numbers/multiplication_associative} Multiplication is associative:
    \begin{equation}\label{eq:thm:field_of_real_numbers/multiplication_associative}
      (A \boxtimes B) \boxtimes C = A \boxtimes (B \boxtimes C)
    \end{equation}

    \thmitem{thm:field_of_real_numbers/multiplication_commutative} Multiplication is commutative:
    \begin{equation}\label{eq:thm:field_of_real_numbers/multiplication_commutative}
      A \boxtimes B = B \boxtimes A.
    \end{equation}

    \thmitem{thm:field_of_real_numbers/multiplication_order} \( \BbbR \) is an \hyperref[def:ordered_semiring]{ordered semiring}: when \( A \subseteq B \), for any \( C \supseteq O \) we have
    \begin{equation}\label{eq:thm:field_of_real_numbers/multiplication_order}
      A \boxtimes C \subseteq B \boxtimes C.
    \end{equation}

    \thmitem{thm:field_of_real_numbers/multiplication_sign} Multiplicative inverses have the same sign:
    \begin{equation}\label{eq:thm:field_of_real_numbers/multiplication_sign}
      A \supsetneq O \T{implies} A^\boxast \supsetneq O \T{and} A \subsetneq O \T{implies} A^\boxast \subsetneq O.
    \end{equation}

    \thmitem{thm:field_of_real_numbers/multiplication_neutral}  \( I \) acts as a neutral element of multiplication:
    \begin{equation}\label{eq:thm:field_of_real_numbers/multiplication_neutral}
      A \boxtimes I = A.
    \end{equation}

    \thmitem{thm:field_of_real_numbers/multiplication_invertible} Multiplication is invertible:
    \begin{equation}\label{eq:thm:field_of_real_numbers/multiplication_invertible}
      A \boxtimes A^\boxast = I.
    \end{equation}
  \end{thmenum}

  Additionally, we will also prove the following:
  \begin{thmenum}[resume=thm:field_of_real_numbers]
    \thmitem{thm:field_of_real_numbers/distributivity} Multiplication \hyperref[def:semiring]{distributes} over addition:
    \begin{equation}\label{eq:thm:field_of_real_numbers/distributivity}
      A \boxtimes (B \boxplus C) = (A \boxtimes B) \boxplus (A \boxtimes C).
    \end{equation}

    \thmitem{thm:field_of_real_numbers/field_embedding} The \hyperref[def:preordered_set/homomorphism]{order embedding} \( \iota(a) = \BbbQ_{<a} \) from \cref{def:lower_cut_completion} is a \hyperref[def:field/homomorphism]{field embedding} of \( \BbbQ \) into \( \BbbR \).
  \end{thmenum}
\end{proposition}
\begin{proof}
  \SubProofOf{thm:field_of_real_numbers/addition_associative} Follows from associativity of addition in \( \BbbQ \).

  \SubProofOf{thm:field_of_real_numbers/addition_commutative} Follows from commutativity of addition in \( \BbbQ \).

  \SubProofOf{thm:field_of_real_numbers/addition_order} We will demonstrate that addition is compatible with the subset order.

  Fix lower cuts \( A \subseteq B \) and \( C \). Then
  \begin{equation*}
    B \boxplus C = \set{ b \cdot c \given b \in B \T{and} c \in C }.
  \end{equation*}

  Since every element of \( A \) is in \( B \), it follows that
  \begin{equation*}
    A \boxplus C \subseteq B \boxplus C.
  \end{equation*}

  This completes the proof.

  \SubProofOf{thm:field_of_real_numbers/addition_neutral} We will show that \( A \boxplus O = A \).

  Fix elements \( a \) of \( A \) and \( q \) of \( O \). By definition, \( q \) is negative, thus \( a + q < a \). It follows that
  \begin{equation*}
    A \boxplus O \subseteq A.
  \end{equation*}

  For the converse inclusion, fix an element \( a \). Since \( A \) is downward closed, it contains some \( a' < a \). The number \( a' - a \) is negative and thus belongs to \( O \). Therefore, \( a = a' + (a' - a) \). It follows that
  \begin{equation*}
    A \subseteq A \boxplus O.
  \end{equation*}

  \SubProofOf{thm:field_of_real_numbers/zero_additive_inverse} Note that the members of \( O \) are simply the negative rational numbers.

  Then, if \( q \in O \), \( -q \) is positive, i.e. an upper bound of \( O \) that is not a supremum of \( O \), thus \( q = -(-q) \) is in \( \boxminus O \). This demonstrates
  \begin{equation*}
    O \subseteq \boxminus O.
  \end{equation*}

  Conversely, if \( -s \in \boxminus O \), there exists some upper bound \( r \) of \( O \) such that \( r < s \). But all upper bounds of \( O \) are nonnegative, implying that \( s \) is positive. Then \( -s \) is negative. This demonstrates
  \begin{equation*}
    \boxminus O \subseteq O.
  \end{equation*}

  \SubProofOf{thm:field_of_real_numbers/addition_invertible} We will show that \( A \boxplus (\boxminus A) = O \).

  \SubProof*{Proof when \( A = O \)} \Cref{thm:field_of_real_numbers/addition_neutral} implies that
  \begin{equation*}
    O \boxplus (\boxminus O) = O \boxplus O = O.
  \end{equation*}

  \SubProof*{Proof when \( A \neq O \)} Fix some \( a + s \) in \( A \boxplus (\boxminus A) \). By definition, \( -s \) is an upper bound of \( A \), thus \( -s > a \). \Cref{thm:ordered_ring/additive_inversion} implies that \( s < -a \), thus
  \begin{equation*}
    a + s < a + (-a) = 0.
  \end{equation*}

  Generalizing on \( a \) and \( s \), we conclude that
  \begin{equation*}
    A \boxplus (\boxminus A) \subseteq O.
  \end{equation*}

  Conversely, fix \( q \in O \).

  \begin{itemize}
    \item If \( A \supsetneq O \), since, by \cref{thm:def:rational_numbers_ordering/archimedean}, \( \BbbQ \) is \hyperref[def:archimedean_semiring]{Archimedean}, hence there exists a largest nonnegative integer \( n \) such that \( a \coloneqq (-q/2)n \) is in \( A \).

    \item If \( A \subsetneq O \), we instead take the smallest nonnegative integer \( n \) such that \( a \coloneqq (q/2)n \) is in \( A \).
  \end{itemize}

  \begin{figure}[!ht]
    \begin{subcaptionblock}{\textwidth}
      \centering
      \includegraphics[page=1]{output/thm__field_of_real_numbers__addition_invertible}
    \end{subcaptionblock}

    \begin{subcaptionblock}{\textwidth}
      \centering
      \includegraphics[page=2]{output/thm__field_of_real_numbers__addition_invertible}
    \end{subcaptionblock}

    \caption{Choice of points in proof of \( O \subseteq A \boxplus (\boxminus A) \) from \cref{thm:field_of_real_numbers}}
    \label{fig:thm:field_of_real_numbers/addition_invertible}
  \end{figure}

  In both cases, \( a - q/2 \) is an upper bound of \( A \), and \( a - q \) is an upper bound that is guaranteed not to be the supremum of \( A \). Then its inverse \( q - a \) is in \( \boxminus A \). Therefore,
  \begin{equation*}
    q = a + (q - a) \in A \boxplus (\boxminus A).
  \end{equation*}

  Generalizing on \( q \), we conclude that
  \begin{equation*}
    O \subseteq A \boxplus (\boxminus A).
  \end{equation*}

  \SubProofOf{thm:field_of_real_numbers/addition_sign} \Cref{thm:field_of_real_numbers/addition_invertible} implies that addition is a group operation and \cref{thm:field_of_real_numbers/addition_order} implies that the additive group of \( \BbbR \) is ordered. Then \eqref{eq:thm:field_of_real_numbers/addition_sign} follows from \cref{thm:ordered_group/sign_inversion}.

  \SubProofOf{thm:field_of_real_numbers/multiplication_and_additive_inverses}

  \SubProof*{Proof when \( A = O \) or \( B = O \)} Trivial.

  \SubProof*{Proof when \( O \subsetneq A \) and \( O \subsetneq B \)} Since \( \boxminus B \) is the additive inverse of \( B \), we have \( \boxminus B \subsetneq O \), and thus
  \begin{equation*}
    A \boxtimes (\boxminus B)
    \reloset {\eqref{eq:def:real_number_arithmetic/multiplication/pos_a}} =
    \boxminus(A \boxtimes (\boxminus\boxminus B))
    =
    \boxminus(A \boxtimes B),
  \end{equation*}
  where we have used that \( \boxminus\boxminus B = B \) due to \cref{thm:def:group/involution}.

  Similarly,
  \begin{equation*}
    (\boxminus A) \boxtimes B
    \reloset {\eqref{eq:def:real_number_arithmetic/multiplication/pos_b}} =
    \boxminus((\boxminus\boxminus A) \boxtimes B)
    =
    \boxminus(A \boxtimes B).
  \end{equation*}

  \SubProof*{Proof when \( O \supsetneq A \) and \( O \subsetneq B \)}
  \begin{equation*}
    A \boxtimes (\boxminus B)
    \reloset {\eqref{eq:def:real_number_arithmetic/multiplication/neg}} =
    (\boxminus A) \boxtimes (\boxminus\boxminus B)
    =
    (\boxminus A) \boxtimes B
  \end{equation*}
  and
  \begin{equation*}
    (\boxminus A) \boxtimes B
    =
    \boxminus ((\boxminus\boxminus A) \boxtimes B)
    \reloset {\eqref{eq:def:real_number_arithmetic/multiplication/pos_b}} =
    \boxminus(A \boxtimes B).
  \end{equation*}

  \SubProof*{Proof when \( O \subsetneq A \) and \( O \supsetneq B \)} Follows from the above via commutativity.

  \SubProof*{Proof when \( O \supsetneq A \) and \( O \supsetneq B \)} We have
  \begin{equation*}
    A \boxtimes (\boxminus B)
    \reloset {\eqref{eq:def:real_number_arithmetic/multiplication/pos_b}} =
    \boxminus ((\boxminus A) \boxtimes (\boxminus B))
    \reloset {\eqref{eq:def:real_number_arithmetic/multiplication/pos_a}} =
    (\boxminus A) \boxtimes B
  \end{equation*}
  and
  \begin{equation*}
    \boxminus(A \boxtimes B)
    =
    \boxminus \boxminus ((\boxminus A) \cdot (\boxminus B))
    =
    (\boxminus A) \cdot (\boxminus B).
  \end{equation*}

  \SubProofOf{thm:field_of_real_numbers/multiplication_associative} The case where \( A = B = C = O \) is trivial.

  If \( A \supsetneq O \) and \( B \supsetneq O \) and \( C \supsetneq O \), it follows from associativity of multiplication in \( \BbbQ \).

  We reduce all other cases to this one. For example suppose that only \( A \subsetneq O \). Via successive applications of \cref{thm:field_of_real_numbers/multiplication_and_additive_inverses}, we obtain
  \begin{equation*}
    (A \boxtimes B) \boxtimes C
    =
    (\boxminus ((\boxminus A) \boxtimes B)) \boxtimes C
    =
    \boxminus ((\boxminus A) \boxtimes B)) \boxtimes C).
  \end{equation*}
  and
  \begin{equation*}
    A \boxtimes (B \boxtimes C)
    =
    \boxminus ((\boxminus A) \boxtimes (B \boxtimes C))
  \end{equation*}.

  Thus, we have reduced this case to the previous one. We will not consider the other cases individually.

  \SubProofOf{thm:field_of_real_numbers/multiplication_commutative} Again, the case where \( A = B = C = O \) is trivial, and all others reduce to the case where \( A \supsetneq O \) and \( B \supsetneq O \), which follows from commutativity of multiplication in \( \BbbQ \).

  \SubProofOf{thm:field_of_real_numbers/multiplication_order} We will demonstrate that multiplication is compatible with the subset order.

  Fix lower cuts \( A \subseteq B \) and \( C \), and suppose that \( C \) is \hi{positive}.

  \SubProof*{Proof when \( A \supseteq O \)} We have
  \begin{equation*}
    B \boxtimes C
    \reloset {\eqref{eq:def:real_number_arithmetic/multiplication/pos}} =
    O \cup \set{ b \cdot c \given b \in B \setminus O \T{and} c \in C \setminus O }.
  \end{equation*}

  If \( A = O \), then \( A \boxtimes C = O \), which is clearly in \( B \boxtimes C \).

  Otherwise, since every element of \( A \) is in \( B \), it again follows that \( A \boxtimes C \subseteq B \boxtimes C \).

  \SubProof*{Proof when \( B \subseteq O \)} We have already shown that addition is compatible with the subset order. Then
  \begin{equation*}
    A \subseteq B \subseteq O
  \end{equation*}
  via \cref{thm:ordered_group/inversion} implies that
  \begin{equation*}
    \underbrace{\boxminus O}_O \subseteq \boxminus B \subseteq \boxminus A.
  \end{equation*}

  By what we have already shown,
  \begin{equation*}
    (\boxminus B) \boxtimes C \subseteq (\boxminus A) \boxtimes C.
  \end{equation*}

  Then, by \eqref{eq:thm:field_of_real_numbers/multiplication_and_additive_inverses},
  \begin{equation*}
    \boxminus (B \boxtimes C) \subseteq \boxminus (A \boxtimes C).
  \end{equation*}

  Applying \cref{thm:ordered_group/inversion} again, we obtain
  \begin{equation*}
    A \boxtimes C \subseteq B \boxtimes C.
  \end{equation*}

  \SubProof*{Proof when \( A \subsetneq O \subsetneq B \)} By what we have already shown,
  \begin{equation*}
    A \boxtimes C \subseteq O \boxtimes C = O
  \end{equation*}
  and
  \begin{equation*}
    O = O \boxtimes C \subseteq B \boxtimes C.
  \end{equation*}

  Then
  \begin{equation*}
    A \boxtimes C \subseteq B \boxtimes C.
  \end{equation*}

  \SubProofOf{thm:field_of_real_numbers/multiplication_neutral} We will show that \( A \boxtimes I = A \).

  \SubProof*{Proof when \( A = O \)} Trivial.

  \SubProof*{Proof when \( A \supsetneq O \)} Fix \( s \) in \( A \boxtimes I \).
  \begin{itemize}
    \item If \( s \leq 0 \), it belongs to \( A \) by definition.
    \item If \( s > 0 \), then there exist \( a \in A \) and \( i \in I \) such that \( s = ai \). If \( a \) and \( i \) are both negative, we can instead consider \( (-a) \) and \( (-i) \), so suppose that both are positive.

    Then \( 0 < i < 1 \), and \cref{thm:ordered_ring/strict_prod} implies that, by multiplying by \( a \), we obtain
    \begin{equation*}
      0 < ai < a.
    \end{equation*}

    Since \( A \) is downward closed, \( s = ai \) belongs to \( A \).
  \end{itemize}

  Generalizing on \( s \), we conclude that
  \begin{equation*}
    A \boxtimes I \subseteq A.
  \end{equation*}

  Conversely, fix \( a \in A \).
  \begin{itemize}
    \item Suppose first that \( a \leq 0 \).

    Note that \( A \boxtimes I \) contains at least one positive element. Indeed, since \( A \supsetneq O \), then \( A \) contains at least one positive element \( a' \). Then, since \( 1 / 2 \) is in \( I \), \( A \boxtimes I \) contains \( a' / 2 \), which is positive.

    Since \( A \boxtimes I \) is a lower cut, it is downward closed, and so it contains \( a \).

    \item Otherwise, \( a > 0 \). Since \( A \) has no maximum, it contains some \( b > a \). Let \( s \coloneqq a b^{-1} \). \Cref{thm:ordered_ring/strict_prod} implies that \( a = b s > a s \), and thus \( 1 > s \). Then \( s \in I \), and we conclude that \( a = b s \) is in \( A \boxtimes I \).
  \end{itemize}

  In both cases,
  \begin{equation*}
    A \subseteq A \boxtimes I.
  \end{equation*}

  \SubProof*{Proof when \( A \subsetneq O \)} We have
  \begin{equation*}
    A \boxtimes I
    \reloset {\ref{thm:def:group/involution}} =
    \boxminus (\boxminus (A \boxtimes I))
    \reloset {\eqref{eq:thm:field_of_real_numbers/multiplication_and_additive_inverses}} =
    \boxminus ((\boxminus A) \boxtimes I)
    =
    \boxminus (\boxminus A)
    \reloset {\ref{thm:def:group/involution}} =
    A.
  \end{equation*}

  \SubProofOf{thm:field_of_real_numbers/multiplication_sign}

  \SubProof*{Proof when \( A \supsetneq O \)} The case \eqref{eq:def:real_number_arithmetic/multiplicative_inverses/pos} defines \( A^\boxast \) so that it explicitly contains \( 0 \), meaning that \( A^\boxast \supsetneq O \).

  \SubProof*{Proof when \( A \subsetneq O \)} The case \eqref{eq:def:real_number_arithmetic/multiplicative_inverses/neg} defines \( A^\boxast \) as \( \boxminus ((\boxminus A)^\boxast) \).

  \Cref{thm:field_of_real_numbers/addition_sign} implies that \( \boxminus A \supsetneq O \) and, by what we have already shown, \( (\boxminus A)^\boxast \supsetneq O \). Then we simply take the additive inverse and conclude that \( A^\boxast \subsetneq O \).

  \SubProofOf{thm:field_of_real_numbers/multiplication_invertible} We will show that \( A \boxtimes (A^\boxast) = I \) when \( A \neq O \).

  \SubProof*{Proof when \( A \supsetneq O \)} First fix positive elements \( a \) in \( A \) and \( s \) in \( A^\boxast \). We must show that \( as < 1 \).

  By definition of \( A^\boxast \), there exists some upper bound \( r < s^{-1} \) of \( A \). Then \( a < r < s^{-1} \). \Cref{thm:ordered_ring/reciprocal_inversion} implies that \( s < r^{-1} < a^{-1} \). Multiplying by \( a \), we obtain \( as < a^{-1} a = 1 \).

  Generalizing on \( a \) and \( s \), we conclude that
  \begin{equation*}
    A \boxtimes (A^\boxast) \subseteq I.
  \end{equation*}

  Conversely, fix some \( i \) in \( I \).
  \begin{itemize}
    \item If \( i \leq 0 \), then it belongs to \( A \boxtimes (A^\boxast) \) by definition since both \( A \supsetneq O \) and, as \cref{thm:field_of_real_numbers/multiplication_sign} implies, \( A^\boxast \supsetneq O \).

    \item Otherwise, \( 0 < i < 1 \). As in \cref{thm:field_of_real_numbers/addition_invertible}, we will use that \( \BbbQ \) is \hyperref[def:archimedean_semiring]{Archimedean}.

    If \( i^{-1} > 1 \) is in \( A \), then \cref{thm:def:rational_numbers_ordering/archimedean_exponentiation} implies that there exists a largest nonnegative integer such that \( i^{-n} \) is in \( A \). Since \( A \) has no maximum, there also exists some element \( a > i^{-n} \).

    Then \( i^{-1} \cdot i^{-n} \) is an upper bound that is possibly the supremum, and \( i^{-1} \cdot a > i^{-1} \cdot i^{n-1} \) is not the supremum.

    It follows that \( (ai^{-1})^{-1} \) is in \( A^\boxast \). Therefore, \( i = a \cdot (ai^{-1})^{-1} \) is in \( A \boxtimes A^\boxast \).
  \end{itemize}

  \SubProof*{Proof when \( A \subsetneq O \)} \Cref{thm:field_of_real_numbers/multiplication_sign} implies that \( A^\boxast \subsetneq O \), and thus
  \begin{equation*}
    A \boxtimes A^\boxast
    \reloset {\eqref{eq:def:real_number_arithmetic/multiplication/neg}} =
    (\boxminus A) \boxtimes (\boxminus A^\boxast),
  \end{equation*}
  which, by what we have already shown, equals \( I \).

  \SubProofOf{thm:field_of_real_numbers/distributivity} Our approach here is similar to the one in \cref{thm:field_of_real_numbers/multiplication_associative} and \cref{thm:field_of_real_numbers/multiplication_commutative}.

  If \( A = O \), then \( A \boxtimes (B \boxplus C) = O = (A \boxtimes C) \boxplus (B \boxtimes C) \). Via \cref{thm:field_of_real_numbers/multiplication_and_additive_inverses} all other cases reduce to the one where \( A \supseteq O \), \( B \supseteq O \) and \( C \supseteq O \), which follows from distributivity in \( \BbbQ \).

  \SubProofOf{thm:field_of_real_numbers/field_embedding} Clearly the map \( \iota(a) = \BbbQ_{\leq a} \) is injective as an order embedding. We must show that it is a \hyperref[def:field/homomorphism]{field homomorphism}. Clearly \( O = \iota(0) \) and \( I = \iota(1) \).

  Furthermore, \( \iota(a + b) \) has \( a + b \) as its supremum, while \( \iota(a) \) and \( \iota(b) \) have \( a \) and \( b \) correspondingly. Then \( \iota(a) \boxplus \iota(b) \) also has \( a + b \) as its supremum, and hence
  \begin{equation*}
    \iota(a + b) = \iota(a) = \iota(b).
  \end{equation*}

  The base case of multiplication can be handled similarly and then extended via \cref{thm:field_of_real_numbers/multiplication_and_additive_inverses}.
\end{proof}

\begin{proposition}\label{thm:real_number_positive_power}
  For a positive real number cut \( A \supsetneq 0 \), \hyperref[def:semiring/exponentiation]{(semi)ring exponentiation} with a positive (integral) power \( n \) satisfies
  \begin{equation}\label{eq:thm:real_number_positive_power}
    A^{\boxtimes n} = O \cup \set{ a^n \given a \in A \setminus O }.
  \end{equation}
\end{proposition}
\begin{proof}
  Define
  \begin{equation*}
    P(A, n) \coloneqq O \cup \set{ a^n \given a \in A \setminus O }.
  \end{equation*}

  For a fixed lower cut \( A \), we will use induction on \( n \) to show that \( P(A, n) = A^{\boxtimes n} \).

  \begin{itemize}
    \item If \( n = 1 \), then
    \begin{equation*}
      P(A, n)
      =
      O \cup \set{ a^1 \given a \in A \setminus O }
      =
      A
      \reloset {\eqref{eq:def:semigroup/exponentiation}} =
      A^{\boxtimes 1}.
    \end{equation*}

    \item If \eqref{eq:thm:real_number_positive_power} holds, then
    \begin{align*}
      A^{\boxtimes n+1}
      &\reloset {\eqref{eq:def:semigroup/exponentiation}} =
      A^{\boxtimes n} \boxtimes A
      \reloset {\T{ind.}} = \\ &=
      (O \cup \set{ a^n \given a \in A \setminus O }) \boxtimes A
      \reloset {\eqref{eq:def:real_number_arithmetic/multiplication/pos}} = \\ &=
      O \cup \set{ a^n \cdot b \given a \in A \setminus O \T{and} b \in A \setminus O }.
    \end{align*}

    It is thus clear that
    \begin{equation*}
      A^{\boxtimes n+1} \subseteq P(A, n+1).
    \end{equation*}

    Conversely, given \( a^n \cdot b \) for positive rational numbers \( a \) and \( b \), we have two cases:
    \begin{itemize}
      \item If \( a \leq b \), then \( a^n \cdot b \leq b^{n+1} \), which belongs to \( A^{n+1} \).
      \item Otherwise, \( a^n \cdot b < a^{n+1} \), which also belongs to \( A^{n+1} \).
    \end{itemize}

    Generalizing on \( a \) and \( b \), we conclude that
    \begin{equation*}
      P(A, n+1) \subseteq A^{\boxtimes n+1}.
    \end{equation*}
  \end{itemize}
\end{proof}

\begin{proposition}\label{thm:real_numbers_archimedean}
  The field of real numbers is \hyperref[def:archimedean_semiring]{Archimedean}.
\end{proposition}
\begin{proof}
  Fix two lower cuts \( A \) and \( B \).

  \begin{itemize}
    \item The case \( A = B \) is clear.

    \item If \( A \subsetneq B \), let \( a \) be a rational number in \( A \) and let \( b \) be an upper bound of \( B \).

    By \cref{thm:def:rational_numbers_ordering/archimedean}, the field of rational numbers is Archimedean, thus there exists a positive integer \( n \) such that \( b < np \). Then, since \( \iota \) is a field embedding,
    \begin{equation*}
      B \subseteq \iota(b) \subseteq \iota(n) \cdot \iota(a) \subseteq \iota(n) \cdot A.
    \end{equation*}

    \item If \( B \subsetneq A \), we reuse the previous case.
  \end{itemize}
\end{proof}

\paragraph{Ceiling and floor}

\begin{definition}\label{def:real_floor_ceiling}\mimprovised
  For any \hyperref[def:real_numbers]{real number} \( A \), we define its \term{floor} as
  \begin{equation*}
    \floor(A) \coloneqq \max\set[\Big]{ \iota(n) \given* n \in \BbbZ \T{and} \iota(n) \subseteq A }
  \end{equation*}
  and its \term{ceiling} as
  \begin{equation*}
    \ceil(A) \coloneqq \min\set[\Big]{ \iota(n) \given* n \in \BbbZ \T{and} \iota(n) \supseteq A }.
  \end{equation*}
\end{definition}

\begin{proposition}\label{thm:real_floor_ceiling_interval}
  For any real number \( A \), we have
  \begin{equation*}
    \floor(A) \subseteq A \subseteq \ceil(A).
  \end{equation*}

  \begin{thmenum}
    \thmitem{thm:real_floor_ceiling_interval/integer} If \( A \) is an integer, then
    \begin{equation*}
      \floor(A) = A = \ceil(A).
    \end{equation*}

    \thmitem{thm:real_floor_ceiling_interval/non_integer} If \( A \) is not an integer, then
    \begin{equation*}
      \floor(A) \subsetneq A \subsetneq \ceil(A).
    \end{equation*}
  \end{thmenum}
\end{proposition}
\begin{proof}
  Let \( n \) be the smallest integer such that \( A \subseteq \iota(n) \).
  \begin{itemize}
    \item If \( A = \iota(n) \), then also \( \iota(n) \subseteq A \) and hence \( A \) is its own floor and ceiling.
    \item If \( A \subsetneq \iota(n) = \ceil(A) \), then \( n - 1 \) is the largest integer such that \( \iota(n - 1) \subseteq A \). We conclude that
    \begin{equation*}
      \floor(A) = \iota(n - 1) \subsetneq A \subsetneq \iota(n) = \ceil(A).
    \end{equation*}
  \end{itemize}
\end{proof}

\begin{proposition}\label{thm:rational_number_floor}
  For rational numbers \( a \) and \( b \), we have
  \begin{equation}\label{eq:thm:rational_number_floor}
    \floor(a / b) = \quot(a, b).
  \end{equation}
\end{proposition}
\begin{proof}
  \Fullref{alg:integer_division} gives us
  \begin{equation*}
    a = b \cdot \quot(a, b) + \rem(a, b).
  \end{equation*}

  Since \( 0 \leq \rem(a) < b \), we have
  \begin{equation*}
    \frac {\rem(a, b)} b < 1,
  \end{equation*}
  hence \eqref{eq:thm:rational_number_floor} follows.
\end{proof}

\paragraph{Positive real roots}

\begin{definition}\label{def:principal_nonnegative_nth_root}\mimprovised
  Given an integer \( n \geq 2 \), for every \hi{nonnegative} \hyperref[def:real_numbers]{real number} \( A \) we define its \term{principal \( n \)-th root} as
  \begin{equation}\label{eq:def:principal_nonnegative_nth_root}
    \sqrt[n]{ A } = O \cup \set{ s \geq 0 \given s^n \in A }.
  \end{equation}

  It is customary to simply write \( \sqrt{ A } \) when \( n = 2 \).
\end{definition}
\begin{comments}
  \item As will follow from \cref{thm:def:principal_nonnegative_nth_root/power_of_root}, this is indeed an \( n \)-th root in the sense of \cref{def:nth_root}.

  \item Based on this definition we introduce rational exponents in \cref{def:real_number_rational_exponent}.

  \item \Cref{def:principal_real_square_root} introduces canonical square roots for negative real values.

  \item Although not for lower sets, \incite[def. 5.6.4]{Tao2022AnalysisI} defines \( n \)-th roots for nonnegative real numbers similarly.
\end{comments}
\begin{defproof}
  We will show here that \( \sqrt[n]{ A } \) is a \hyperref[def:lower_cut]{lower cut}.

  \SubProofOf*[def:closed_ordered_subset]{downward closure} Fix some \( s \) in \( \sqrt[n]{ A } \) and let \( r < s \).

  \begin{itemize}
    \item If \( r < 0 \), it belongs to \( \sqrt[n]{ A } \) by definition.

    \item If \( r \geq 0 \), then \( s > 0 \). \Cref{thm:ordered_ring/power_preserves_inequality} implies that \( r^n \leq s^n \). Since \( A \) is downward closed, it thus contains \( r^n \), hence \( r \) belongs to \( \sqrt[n]{ A } \).
  \end{itemize}

  \SubProof*{Proof of no maximum} Suppose that \( s = \max \sqrt[n]{ A } \).

  Then \( s^n \in A \). Since \( A \) has no maximum, it contains some \( r > s^n \).

  \Fullref{alg:rational_number_power_bisection} implies that there exists some positive rational number \( t \) such that \( s^n < t^n < r \). Then \( s < t \) since otherwise \cref{thm:ordered_ring/power_preserves_inequality} would imply that \( s^n \geq t^n \).

  Therefore, \( t \) must be in \( \sqrt[n]{ A } \), contradicting the maximality of \( s \).
\end{defproof}

\begin{proposition}\label{thm:def:principal_nonnegative_nth_root}
  The \hyperref[def:principal_nonnegative_nth_root]{real \( n \)-th root} satisfies the following basic properties:
  \begin{thmenum}
    \thmitem{thm:def:principal_nonnegative_nth_root/zero} For any \( A \supsetneq O \), we have \( A = O \) if and only if \( \sqrt A = O \).

    \thmitem{thm:def:principal_nonnegative_nth_root/power_of_root} For any \( A \supsetneq O \), we have
    \begin{equation}\label{eq:thm:def:principal_nonnegative_nth_root/power_of_root}
      \parens[\Big]{ \sqrt[n]{ A } }^{\boxtimes n} = A.
    \end{equation}

    \thmitem{thm:def:principal_nonnegative_nth_root/root_of_power} For any \( A \supsetneq O \), we have
    \begin{equation}\label{eq:thm:def:principal_nonnegative_nth_root/root_of_power}
      \sqrt[n]{ A^{\boxtimes n} } = A.
    \end{equation}

    \thmitem{thm:def:principal_nonnegative_nth_root/multiplicative} Taking the \( n \)-th root is \hyperref[def:multiplicative_function]{multiplicative}:
    \begin{equation}\label{eq:thm:def:principal_nonnegative_nth_root/multiplicative}
      \sqrt[n]{ A \boxtimes B } = \parens[\Big]{\sqrt[n]{ A }} \boxtimes \parens[\Big]{\sqrt[n]{ B }}.
    \end{equation}
  \end{thmenum}
\end{proposition}
\begin{proof}
  \SubProofOf{thm:def:principal_nonnegative_nth_root/zero} Trivial.

  \SubProofOf{thm:def:principal_nonnegative_nth_root/power_of_root} We have
  \begin{equation*}
    \parens[\Big]{ \sqrt[n]{ A } }^{\boxtimes n}
    \reloset{\eqref{eq:thm:real_number_positive_power}} =
    O \cup \set{ s^n \given s \in \sqrt[n]{ A } }
    \reloset{\eqref{eq:def:principal_nonnegative_nth_root}} =
    O \cup \set{ s^n \given s \geq 0 \T{and} s^n \in A }
    =
    A.
  \end{equation*}

  \SubProofOf{thm:def:principal_nonnegative_nth_root/root_of_power} We have
  \begin{equation*}
    \sqrt[n]{ A^{\boxtimes n} }
    \reloset{\eqref{eq:def:principal_nonnegative_nth_root}} =
    O \cup \set{ s \geq 0 \given s^n \in A^{\boxtimes n} }
    \reloset{\eqref{eq:thm:real_number_positive_power}} =
    O \cup \set{ a \geq 0 \given a \in A }
    =
    A.
  \end{equation*}

  \SubProofOf{thm:def:principal_nonnegative_nth_root/multiplicative} If \( A = O \) or \( B = O \), then both sides of \eqref{eq:thm:def:principal_nonnegative_nth_root/multiplicative} are equal to \( O \).

  Suppose that \( A \supsetneq O \) and \( B \supsetneq O \). Then
  \begin{align*}
    \parens[\Big]{\sqrt[n]{ A }} \boxtimes \parens[\Big]{\sqrt[n]{ B }}
    &\reloset {\eqref{eq:def:real_number_arithmetic/multiplication/pos}} =
    O \cup \set{ s \cdot r \given s \in \sqrt[n]{ A } \T{and} r \in \sqrt[n]{ B }}
    \reloset {\eqref{eq:def:principal_nonnegative_nth_root}} = \\ &=
    O \cup \set{ s \cdot r \given s \geq 0 \T{and} s^n \in A \T{and} r \geq 0 \T{and} r^n \in B }
    \reloset {\eqref{eq:def:real_number_arithmetic/multiplication/pos}} = \\ &=
    O \cup \set{ t \given t \geq 0 \T{and} t^n \in A \boxtimes B }
    \reloset {\eqref{eq:def:principal_nonnegative_nth_root}} = \\ &=
    \sqrt[n]{ A \boxtimes B }.
  \end{align*}
\end{proof}

\begin{definition}\label{def:real_number_rational_exponent}
  For a nonnegative real number \( A \), \hyperref[def:principal_nonnegative_nth_root]{principal \( n \)-th roots} allow us to extend \hyperref[def:semiring/exponentiation]{(semi)ring exponentiation} from an integer power to an arbitrary rational number \( p / q \) as follows (assuming \( q > 1 \)):
  \begin{equation}\label{eq:def:real_number_rational_exponent}
    A^{\boxtimes p / q} \coloneqq \sqrt[q]{ P }^{\boxtimes p}
  \end{equation}
\end{definition}
\begin{comments}
  \item Via \cref{def:principal_real_square_root}, we can also extend this to square roots of negative real numbers.
\end{comments}

\paragraph{Irrational numbers}

\begin{definition}\label{def:irrational_number}\mcite[7]{Dedekind1901Essays}
  An \term[ru=иррациональное число (\cite[def. II.1]{Александров1977ОбщаяТопология})]{Irrational number} is a \hyperref[def:real_numbers]{real number} that is not (an embedding of) a \hyperref[def:rational_numbers]{rational number}.
\end{definition}
\begin{comments}
  \item These correspond exactly to lower cuts of rational numbers that have no supremum.
\end{comments}

\begin{proposition}\label{thm:real_nth_root_is_irrational}
  For any \hyperref[def:square_free_element]{square-free} integer \( m \), consider its embedding into the \hyperref[def:real_numbers]{real numbers} \( M = \iota(m) = \BbbQ_{<m} \).

  We claim that, for integers \( n \geq 2 \), the \( n \)-th \hyperref[def:principal_nonnegative_nth_root]{root} \( \sqrt[n]{ M } \) is \hyperref[def:irrational_number]{irrational}.
\end{proposition}
\begin{proof}
  Suppose that \( \sqrt[n]{ M } \) is rational, and let \( r \) be its supremum.

  Then \( r^n = r^2 \cdot r^{n-2} = m \), which contradicts the assumption that \( m \) is square-free.

  It remains for \( \sqrt[n]{ A } \) to be irrational.
\end{proof}
