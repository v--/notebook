\subsection{Real numbers}\label{subsec:real_numbers}

\paragraph{Field of real numbers}

\begin{definition}\label{def:real_numbers}\mcite[sec. I.V.]{Beman1901DedekindEssays}
  We define the set \( \BbbR \) of \term[bg=реални числа (\cite[ch. I]{Тагамлицки1971Диф}), ru=вещественные числа (\cite[\S 3]{Фихтенгольц1968ОсновыТом1}) / действительные числа (\cite[def. 25.1]{АлександровМаркушевичХинчинЭнциклопедия1951Том1})]{real numbers} as the \hyperref[def:dedekind_completion]{Dedekind completion} of the set \( \BbbQ \) of \hyperref[def:rational_numbers]{rational numbers}.
\end{definition}
\begin{comments}
  \item We consider the following an indispensable part of \( \BbbR \):
  \begin{itemize}
    \item The field structure defined in \fullref{def:real_number_arithmetic}.
    \item The order topology defined via \fullref{def:order_topology}.
  \end{itemize}

  \item Our definition of Dedekind completion uses \hyperref[def:dedekind_macnielle_closure]{Dedekind-MacNeille closed sets}, which has certain consequences compared to other authors --- see \fullref{rem:dedekind_completion_through_dedekind_macneille_closures}.
\end{comments}

\begin{definition}\label{def:real_number_arithmetic}\mimprovised
  For each pair of \hyperref[def:real_numbers]{real numbers} \( R \) and \( Q \), considered as \hyperref[def:dedekind_macnielle_closure]{Dedekind-MacNeille-closed} sets of rational numbers, we define the base operations:
  \begin{thmenum}
    \thmitem{def:real_number_arithmetic/zero} Zero:
    \begin{equation}\label{eq:def:real_number_arithmetic/zero}
      \begin{split}
        \mathllap{O} &\coloneqq \mathrlap{\set{ s \given s \leq 0 }}.
      \end{split}
    \end{equation}

    \thmitem{def:real_number_arithmetic/additive_inverse} Additive inverses:
    \begin{equation}\label{eq:def:real_number_arithmetic/additive_inverse}
      \begin{split}
        \mathllap{-P} \coloneqq \mathrlap{\set{ -p \given p \in P }^L}.
      \end{split}
    \end{equation}

    \thmitem{def:real_number_arithmetic/addition} Addition:
    \begin{equation}\label{eq:def:real_number_arithmetic/addition}
      \begin{split}
        \mathllap{P + Q} \coloneqq \mathrlap{\set{ p + q \given p \in P \T{and} q \in Q }^{UL}}.
      \end{split}
    \end{equation}

    \thmitem{def:real_number_arithmetic/one} One:
    \begin{equation}\label{eq:def:real_number_arithmetic/one}
      \begin{split}
        \mathllap{I} \coloneqq \mathrlap{\set{ s \given s \leq 1 }}.
      \end{split}
    \end{equation}

    \thmitem{def:real_number_arithmetic/multiplicative_inverse} Multiplicative inverses (for \( P \neq O \)):
    \begin{equation}\label{eq:def:real_number_arithmetic/multiplicative_inverses}
      \mathllap{P^{-1}} \coloneqq \mathrlap{\set{ p^{-1} \given p \in P }^L}.
    \end{equation}

    \thmitem{def:real_number_arithmetic/multiplication} Multiplication:
    \begin{equation}\label{eq:def:real_number_arithmetic/multiplication}
      P \cdot Q \coloneqq \begin{cases}
        O,                                                                             &O = P \T{or} O = Q \\
        \set{ p \cdot q \given p \in P \setminus O \T{and} q \in Q \setminus O }^{UL}, &O \subsetneq P \T{and} O \subsetneq Q \\
        (-P) \cdot Q,                                                                  &O \supsetneq P \T{and} O \subsetneq Q \\
        P \cdot (-Q),                                                                  &O \subsetneq P \T{and} O \supsetneq Q \\
        (-P) \cdot (-Q),                                                               &O \supsetneq P \T{and} O \supsetneq Q \\
      \end{cases}
    \end{equation}
  \end{thmenum}
\end{definition}
\begin{comments}
  \item \incite[113]{Enderton1977Sets} instead defines the arithmetic operations for cuts as discussed in \fullref{rem:dedekind_completion_through_dedekind_macneille_closures}, which makes addition simpler, but nearly all other operations more difficult.
  \item The complications in \eqref{eq:def:real_number_arithmetic/multiplication} arise from the fact that \( I \) contains all negative rational numbers and, when multiplied by \( -1 \), these become positive. Thus, if multiplication is always treated like in the first case, multiplying \( I \) by \( -1 \) would give the entire set \( \BbbR \).
\end{comments}

\begin{proposition}\label{thm:real_numbers_are_a_field}
  The operations from \fullref{def:real_number_arithmetic} make the set \( \BbbR \) of \hyperref[def:real_numbers]{real numbers} a \hyperref[def:field]{field}. Furthermore, the \hyperref[def:preordered_set/homomorphism]{order embedding} \( \iota: \BbbQ \to \BbbR \) from \fullref{def:dedekind_macnielle_completion} is a \hyperref[def:field/homomorphism]{field embedding}.
\end{proposition}
\begin{proof}
  \SubProof{Proof that addition is associative} First note that the set of sums \( r + q \), where \( r \in R \) and \( q \in Q \), is downward closed. Indeed, if \( s \leq r + q \), then \( s = (r + q) - s_0 \), where \( s_0 \coloneqq (r + q) - s \). The latter is nonnegative, hence \( q - s_0 \leq q \) and thus \( q - s_0 \in Q \). Then \( s = r + (q - s_0) \), where \( r \) belongs to \( R \) and \( q - s_0 \) belongs to \( Q \). Thus, the set \( \set{ r + q \given r \in R \T{and} q \in Q } \) is downward closed.

  \Fullref{thm:def:dedekind_macnielle_closure/point_in_closure} then implies that every member of \( R + Q \) is either a sum of the form \( r + q \) for \( r \in R \) and \( q \in Q \), or the supremum of such sums.

  Now let \( x \in (P + Q) + R \). We have two possibilities:
  \begin{itemize}
    \item If \( x = s + r \) for some \( s \in P + Q \) and \( r \in R \), we again have two possibilities:
    \begin{itemize}
      \item If \( s = p + q \) for some \( p \in P \) and \( q \in Q \), then \( x = (p + q) + r \). Then obviously \( x = p + (q + r) \) is a member of \( P + (Q + R) \).

      \item If \( s = \sup(P + Q) \), consider some \( s' < s \) and let \( r' \coloneqq r - (s - s') \). Since \( R \) is downward closed, it contains \( r' \). Since \( s' \) is not the supremum of \( P + Q \) (but belongs to the latter set), by \fullref{thm:def:dedekind_macnielle_closure/point_in_closure}, there exist some \( p' \in P \) and \( q' \in Q \) such that \( s' = p' + q' \). Then \( x = (p' + q') + r' \), which again implies \( x \in P + (Q + R) \).
    \end{itemize}

    \item Suppose that \( x = \sup((P + Q) + R) \).

    Let \( x' < x \) so that \( x' \) belongs to \( (P + Q) + R \) and \( x = x' + (x - x') \). Then there exist some \( p' \in P \), \( q' \in Q \) and \( r' \in R \) such that \( x' = (p' + q') + r' \). Clearly then \( x' \) belongs to \( P + (Q + R) \).

    Thus, every rational number strictly smaller than \( x \) belongs to \( P + (Q + R) \). But \( x \) is the supremum of such numbers, and \fullref{thm:def:dedekind_macnielle_closure/totally_ordered} implies that this supremum must also belong to \( P + (Q + R) \).
  \end{itemize}

  We have shown that \( (P + Q) + R \subseteq P + (Q + R) \). The converse inclusion can be proven symmetrically. Hence, we conclude that addition of real numbers is associative.

  \SubProof{Proof that addition is commutative} Follows directly from commutativity of addition of rational numbers.

  \SubProof{Proof that addition has a neutral element} We will show that \( P + O = P \).

  Since \( p + 0 = p \), clearly \( P \subseteq P + O \).

  For the converse, note that \Fullref{thm:def:dedekind_macnielle_closure/totally_ordered} implies that \( P \) is downward closed, hence \( p \in P \) and \( o < 0 \) imply that \( p + o \in P \). Since \( P \) is also closed under taking suprema of subsets for which suprema exist, it follows that \( P + O \subseteq P \).

  \SubProof{Proof that addition is invertible} We will show that \( P + (-P) = O \).

  A rational number \( p \) is in \( P \) if and only if \( -p \) is an upper bound of \( -P \), hence \( 0 \) is an upper bound of \( P + (-P) \), implying that \( P + (-P) \subseteq O \).

  For the converse, we will consider several cases.

  \SubProof*{Proof for \( P = O \)} \( P \) is simply the principal ideal of \( 0 \), and so is \( -P \). Any negative number \( s < 0 \) is then in \( P \), and, since \( 0 \) is in \( -P \), we have that \( s \) is in \( P + (-P) \).

  \SubProof*{Proof for \( O \subsetneq P \)} Let \( s < 0 \) and let \( n \) be the smallest positive integer such that \( (-s)n \) is an upper bound of \( P \). At least one such integer exists because, given some upper bound \( u \) of \( P \), \fullref{thm:def:rational_numbers_ordering/archimedean}, gives us a positive integer such that \( (-s)m > u \).

  Then \( (-s)n \) is in \( -P \), \( (-s)n + s \) is in \( P \), and their sum is \( s \). Hence, \( s \in P + (-P) \).

  \SubProof*{Proof for \( P \subsetneq O \)} Similarly to the above case, we can take the smallest \( n \) such that \( sn \) is an upper bound of \( P \). Then \( -sn \) is in \( -P \) and \( sn + s \) is in \( P \), and their sum is \( s \).

  \SubProof{Proof that multiplication is commutative} Follows directly from commutativity of multiplication of rational numbers.

  \SubProof{Proof that multiplication respects additive inverses} We will show that
  \begin{equation}\label{eq:thm:real_numbers_are_a_field/mult_additive_inverses}
    P \cdot (-Q) = (-P) \cdot Q = -(P \cdot Q).
  \end{equation}

  The case where either \( P = O \) or \( Q = O \) is trivial, and we will not consider it.

  \SubProof*{Proof for \( O \subsetneq P \) and \( O \subsetneq Q \)} Since \( -Q \) is the additive inverse of \( Q \), we have \( O \supsetneq -Q \), and thus
  \begin{equation*}
    P \cdot (-Q)
    \reloset {\eqref{eq:def:real_number_arithmetic/multiplication}} =
    -(P \cdot (--Q))
    =
    -(P \cdot Q).
  \end{equation*}

  Similarly,
  \begin{equation*}
    (-P) \cdot Q
    \reloset {\eqref{eq:def:real_number_arithmetic/multiplication}} =
    -((--P) \cdot Q)
    =
    -(P \cdot Q).
  \end{equation*}

  \SubProof*{Proof for \( O \supsetneq P \) and \( O \subsetneq Q \)}
  \begin{equation*}
    P \cdot (-Q)
    \reloset {\eqref{eq:def:real_number_arithmetic/multiplication}} =
    (-P) \cdot (--Q)
    =
    (-P) \cdot Q
    \reloset {\eqref{eq:def:real_number_arithmetic/multiplication}} =
    -((--P) \cdot Q)
    =
    -(P \cdot Q).
  \end{equation*}

  \SubProof*{Proof for \( O \subsetneq P \) and \( O \supsetneq Q \)} Follows from the above via commutativity.

  \SubProof*{Proof for \( O \supsetneq P \) and \( O \supsetneq Q \)}
  \begin{equation*}
    P \cdot (-Q)
    \reloset {\eqref{eq:def:real_number_arithmetic/multiplication}} =
    ((-P) \cdot (-Q))
    \reloset {\eqref{eq:def:real_number_arithmetic/multiplication}} =
    (-P) \cdot Q
  \end{equation*}
  and
  \begin{equation*}
    -(P \cdot Q)
    =
    --((-P) \cdot (-Q))
    =
    (-P) \cdot (-Q).
  \end{equation*}

  \SubProof{Proof that multiplication is associative} We have many cases, but fortunately most of them are similar. We will only consider two, excluding the trivial case where \( P = O \) or \( Q = O \) or \( R = O \).

  \SubProof*{Proof for \( O \subsetneq P \), \( O \subsetneq Q \) and \( O \subsetneq R \)} First note that \( O \subsetneq (P \cdot Q) \cdot R \) and \( O \subsetneq P \cdot (Q \cdot R) \) Indeed, there exist positive numbers \( p \in P \) and \( q \in Q \) such that \( pq \in P \cdot Q \), and thus \( O \subsetneq P \cdot Q \). Similarly, there exists a positive number \( r \in R \) such that \( pqr \in (P \cdot Q) \cdot R \). But \( pqr \) also belongs to \( P \cdot (Q \cdot R) \). Hence, both sets strictly contain \( O \).

  Now that we have show that the two products have the same sign, we will show that they are equal.

  Let \( x \in (P \cdot Q) \cdot R \) and \( x > 0 \). As in the case of addition, \fullref{thm:def:dedekind_macnielle_closure/point_in_closure} implies that we have the following cases:
  \begin{itemize}
    \item If there exist some \( s \in P \cdot Q \) and \( r \in R \) such that \( x = s \cdot r \), we again have two possibilities:
    \begin{itemize}
      \item If there exist some \( p \in P \) and \( q \in Q \) such that \( s = p \cdot q \), then \( x = (p \cdot q) \cdot r = p \cdot (q \cdot r) \) belongs to \( P \cdot (Q \cdot R) \).

      \item If \( s = \sup(P \cdot Q) \), consider some positive \( s' < s \) and let \( r' \coloneqq s' \cdot s^{-1} \cdot r \). Then \( x = s \cdot r = s' \cdot r' \). Furthermore, since \( s' < s \), the quotient \( s' / s \) is less than \( 1 \), hence \( r' < r \) and \( r' \) belongs to \( R \).

      Then \( s' \) is not the supremum of \( P \cdot Q \), and hence there exist positive numbers \( p' \in P \) and \( q' \in Q \) such that \( s' = p' \cdot q' \) and thus \( x' = (p' \cdot q') \cdot r' \).

      It is now clear that \( x \in P \cdot (Q \cdot R) \).
    \end{itemize}

    \item Suppose that \( x = \sup((P \cdot Q) \cdot R) \).

    Fix some \( x' \) such that \( 0 < x' < x \). Then there exist some positive \( p' \in P \), \( q' \in Q \) and \( r' \in R \) such that \( x' = (p' \cdot q') \cdot r' \). Clearly then \( x' \in P \cdot (Q \cdot R) \).

    Thus, every positive member of \( (P \cdot Q) \cdot R \) belongs to \( P \cdot (Q \cdot R) \). Since \( x \) is the supremum of such members, it follows that \( x \) must also belong to \( P \cdot (Q \cdot R) \), which is closed under suprema of subsets that have suprema.
  \end{itemize}

  We have shown that \( (P \cdot Q) \cdot R \subseteq P \cdot (Q \cdot R) \).  The converse inclusion can be proven symmetrically.

  \SubProof*{Proof for \( O \supsetneq P \), \( O \subsetneq Q \) and \( O \subsetneq R \)} We use \eqref{eq:thm:real_numbers_are_a_field/mult_additive_inverses} to reduce this case to the first:
  \begin{equation*}
    P \cdot (Q \cdot R)
    \reloset {\eqref{eq:thm:real_numbers_are_a_field/mult_additive_inverses}} =
    -[(-P) \cdot (Q \cdot R)]
    =
    -[((-P) \cdot Q) \cdot R]
    \reloset {\eqref{eq:thm:real_numbers_are_a_field/mult_additive_inverses}} =
    -[(-(P \cdot Q)) \cdot R]
    \reloset {\eqref{eq:thm:real_numbers_are_a_field/mult_additive_inverses}} =
    (P \cdot Q) \cdot R.
  \end{equation*}

  \SubProof*{Proof for all other cases} The other cases follow similarly from the first via \eqref{eq:thm:real_numbers_are_a_field/mult_additive_inverses}.

  \SubProof{Proof that multiplication has a neutral element} This is similar to the proof for addition. We will show that \( P \cdot I = P \).

  Since \( p \cdot 1 = p \), clearly \( P \subseteq P \cdot I \).

  For the converse, let \( x \in P \cdot I \). Note that \( O \cdot I = O \), hence we can consider the nontrivial cases where \( P \neq O \). First suppose that \( O \subsetneq P \).
  \begin{itemize}
    \item If \( x = p \cdot i \) for some positive \( p \in P \) and \( i \in I \), then \( 0 < i < 1 \) and thus \( x = pi < p \) and \( x \in P \).
    \item If \( x = \sup(P \cdot I) \), it is the supremum of members of \( P \) and is thus itself in \( P \).
  \end{itemize}

  Therefore, if \( O = P \) or \( O \subsetneq P \), then \( P = O \cdot P \). If \( O \supsetneq P \), then
  \begin{equation*}
    P \cdot I = -[(-P) \cdot I] = -(-P) = P.
  \end{equation*}

  \SubProof{Proof that multiplication is invertible} We will show that \( P \cdot P^{-1} = I \), somewhat similarly to how we did it for addition.

  A rational number \( p \) is in \( P \) if and only if \( p^{-1} \) is an upper bound of \( P^{-1} \), hence \( 1 = p \cdot p^{-1} \) is an upper bound of \( P \cdot P^{-1} = I \), implying that \( P \cdot P^{-1} \subsetneq I \).

  We will now show that every positive rational number less than \( 1 \) belongs to \( P \cdot P^{-1} \), and it will follow that so does their Dedekind-MacNeille closure \( I \).

  \SubProof*{Proof for \( O \subsetneq P \)} Let \( 0 < s < 1 \).
  \begin{itemize}
    \item If \( s^{-1} \) is in \( P^{-1} \), let \( n \) be the smallest positive integer such that \( s^{-n} \) is an upper bound of \( P^{-1} \). This is well-defined because \fullref{thm:def:rational_numbers_ordering/reciprocal} implies that \( s^{-1} > 1 \) and thus \( s^{-n} \) increases with \( n \). Furthermore, \fullref{thm:def:rational_numbers_ordering/archimedean_exponentiation} implies that, for some positive integer \( n \), \( s^{-n} \) must be an upper bound, that is, at least one upper bound of this form exists.

    Then \( s^{-n+1} \) is in \( P^{-1} \). The number \( n \) is necessarily greater than \( 1 \) because \( s^{-1} \) itself belongs to \( P^{-1} \).

    Furthermore, since \( s^{-n} \) is an upper bound of \( P^{-1} \), its reciprocal belongs to \( P \). Then
    \begin{equation*}
      s = s^n \cdot s^{-n+1} \in P \cdot P^{-1}.
    \end{equation*}

    \item If \( s^{-1} \) is an upper bound of \( P^{-1} \), let \( n \) be the smallest positive integer such that \( s^n \) is in \( P^{-1} \).

    \begin{itemize}
      \item If \( n = 1 \), then \( s \) belongs to both \( P \) and \( P^{-1} \). Either \( P \) or \( P^{-1} \) must contain \( 1 \), thus \( s = s \cdot 1 = 1 \cdot s \) belongs to their product \( P \cdot P^{-1} \).

      \item If \( n > 1 \), then \( s^{n-1} \) is an upper bound of \( P^{-1} \), and thus \( s^{-(n-1)} \) is in \( P \). Thus,
      \begin{equation*}
        s^{-(n-1)} \cdot s^n = s.
      \end{equation*}
    \end{itemize}
  \end{itemize}

  \SubProof*{Proof for \( P \subsetneq O \)} We can show similarly by considering \( -s^n \) rather than \( s^n \).

  \SubProof{Proof of distributivity} We will show that \( P(Q + R) = PQ + PR \). Skipping the trivial cases, we only need to consider the case where \( O \subsetneq P \) and \( O \subsetneq Q + R \) because the rest will follow via \eqref{eq:def:real_number_arithmetic/multiplication}.

  Again, \fullref{thm:def:dedekind_macnielle_closure/point_in_closure} gives us several possibilities for \( x \in P(Q + R) \):
  \begin{itemize}
    \item If \( x = p \cdot s \) for some \( p \in P \) and \( s \in Q + R \), then we have two possibilities again:
    \begin{itemize}
      \item If \( s = q + r \) for some \( q \in Q \) and \( r + R \), then \( x = p(q + r) \) and distributivity on the rational numbers imply \( x \in PQ + PR \).

      \item If \( s = \sup(Q + R) \), since \( p \) is positive, as in our proofs of associativity we have \( x = p' \cdot (q' + r') \) for some \( q' \in Q \) and \( r' \in Q \), and distributivity on the rational again implies \( x \in PQ + PR \).
    \end{itemize}

    \item If \( x = \sup(P(Q + R)) \), then, since \( O \subsetneq P(Q + R) \), we can again conclude that \( x' < x \) implies \( x' \in PQ + PR \), which in turn implies that \( x \in PR + PR \).
  \end{itemize}

  Therefore, \( P(Q + R) \subseteq PQ + PR \).

  Conversely, if \( x \in PQ + PR \), then:
  \begin{itemize}
    \item If \( x = q_0 + r_0 \) for \( q_0 \in PQ \) and \( r_0 \in PR \), then both are positive, and we have more possibilities:
    \begin{itemize}
      \item If \( q_0 = p_q q \) for positive \( p_q \in P \) and \( q \in Q \) and if similarly \( r_0 = p_r r \), let \( p \) be the larger of the two. Then
      \begin{equation*}
        x = p\parens[\Big]{ \underbrace{\frac {p_q} p}_{\leq 1} q + \underbrace{\frac {p_r} p}_{\leq 1} r },
      \end{equation*}
      hence \( x \in P(Q + R) \).

      \item If \( q_0 = p_q q \) as above but \( r_0 = \sup(PR) \), we must consider \( r_0' < r_0 \) and argue that \( x \) is the supremum by \( r_0' \).

      \item The case \( q = \sup(PQ) \) and \( r_0 = p_r r \) is symmetric.

      \item If \( q_0 = \sup(PQ) \) and \( r_0 = \sup(PR) \), we must simultaneously consider \( q_0' < q_0 \) and \( r_0' < r_0 \).
    \end{itemize}

    \item If \( x = \sup(PR + PR) \), then we must consider \( x' < x \) and, as in the proofs of associativity, argue that \( x \) is the supremum by \( x' \).
  \end{itemize}

  Therefore, \( PQ + PR \subseteq P(Q + R) \).

  \SubProof{Proof of field embedding} Clearly the map \( \iota(p) = \BbbQ_{\leq p} \) is injective as an order embedding. We must show that it is a \hyperref[def:field/homomorphism]{field homomorphism}. Clearly \( O = \iota(0) \) and \( I = \iota(1) \).

  Furthermore, \( \iota(p + q) \) has \( p + q \) as its supremum, while \( \iota(p) \) and \( \iota(q) \) have \( p \) and \( q \) correspondingly. Then \( \iota(p) + \iota(q) \) also has \( p + q \) as its supremum, and hence
  \begin{equation*}
    \iota(p + q) = \iota(p) = \iota(q).
  \end{equation*}

  The base case of multiplication can be handled similarly and then extended via \eqref{eq:thm:real_numbers_are_a_field/mult_additive_inverses}.
\end{proof}

\begin{proposition}\label{thm:real_numbers_archimedean}
  The field of real numbers is \hyperref[def:archimedean_field]{Archimedean}.
\end{proposition}
\begin{proof}
  Fix two real numbers \( P \) and \( Q \).

  \begin{itemize}
    \item The case \( P = Q \) is clear.

    \item If \( P \subsetneq Q \), let \( p \) be a rational number in \( P \) and let \( q \) be a rational number not in \( Q \).

    By \fullref{thm:def:rational_numbers_ordering/archimedean}, the field of rational numbers is Archimedean, thus there exists a positive integer \( n \) such that \( q < np \). Then
    \begin{equation*}
      Q \subseteq \iota(q) \subseteq \iota(n) \cdot \iota(p) \subseteq \iota(n) \cdot P.
    \end{equation*}

    \item If \( Q \subsetneq P \), we use the previous case.
  \end{itemize}
\end{proof}

\paragraph{Extended real numbers}

\begin{definition}\label{def:extended_real_numbers}\mimprovised
  We are sometimes interested in \term{extended real numbers}. This is the \hyperref[def:dedekind_macnielle_completion]{Dedekind-MacNeille completion} of the \hyperref[def:rational_numbers]{rational numbers}.
\end{definition}
\begin{comments}
  \item Note that in \fullref{def:dedekind_completion} we defined the Dedekind completion of an unbounded totally ordered set as its Dedekind-MacNeille completion without the top and bottom elements. Thus, the extended real numbers are the real numbers with the additional elements \( \infty = +\infty \), corresponding to the entire set \( \BbbQ \), and \( -\infty \), corresponding to the empty set \( \varnothing \).
\end{comments}

\begin{proposition}\label{thm:extended_real_numbers_are_not_field}
  Consider the \hyperref[def:extended_real_numbers]{extended real numbers} \( \BbbR \cup \set{ -\infty, \infty } \). If we extend \hyperref[def:real_number_arithmetic/addition]{addition} or \hyperref[def:real_number_arithmetic/multiplication]{multiplication} of real numbers to \( \infty \) and/or \( -\infty \) in any way, the obtained set is no longer an \hyperref[def:ordered_semiring]{ordered field}.
\end{proposition}
\begin{proof}
  Since \( 0 < 1 \), in an ordered field, \fullref{thm:def:ordered_semiring/strict_sum} implies that \( \infty = 0 + \infty < 1 + \infty \). But \( \infty \) is the top element.

  Similarly, for multiplication, since \( 1 < 2 \), we have \( \infty < 2 \cdot \infty \), which is again a contradiction.

  The analogous result for \( -\infty \) follows via \fullref{thm:lattice_duality}.
\end{proof}

\paragraph{Irrational numbers}

\begin{definition}\label{def:irrational_numbers}\mimprovised
  \term[ru=иррациональные числа (\cite[36]{Александров1977Топология})]{Irrational numbers} are \hyperref[def:real_numbers]{real numbers} that are not (the embedding of) \hyperref[def:rational_numbers]{rational numbers}.
\end{definition}
\begin{comments}
  \item These correspond exactly to Dedekind-MacNeille-closed sets of rational numbers that have no supremum.
\end{comments}

\begin{definition}\label{def:nth_root}\mimprovised
  For any \hyperref[def:integers]{integer} \( n \geq 2 \) and any \hi{nonnegative} \hyperref[def:real_numbers]{real number} \( P \), we define the \( n \)-th \term{root} of the \( P \) as the number
  \begin{equation*}
    \sqrt[n]{ P } = O \cup \set{ q > 0 \given q^n \in P }.
  \end{equation*}

  It is customary to simply write \( \sqrt{ P } \) when \( n = 2 \).
\end{definition}
\begin{defproof}
  We must show that \( \sqrt[n]{ P } \) is always \hyperref[def:dedekind_macnielle_closure]{Dedekind-MacNeille closed}.

  If \( P = O \), then \( \sqrt[n]{ P } = O \) and there is nothing to prove. Suppose that \( O \subsetneq P \).

  First note that if \( q > 0 \) is in \( \sqrt[n]{ P } \) and \( 0 < r < q \), \fullref{thm:def:rational_numbers_ordering/power_monotone} implies that \( r^n < q^n \) and, since \( P \) is \hyperref[def:closed_ordered_subset]{downward closed}, we have \( r^n \in P \) and hence \( r \in \sqrt[n]{ P } \). Thus, since it also contains all nonpositive numbers, \( \sqrt[n]{ P } \) is downward closed.

  Furthermore, suppose that \( q_0 \) is the supremum of \( \sqrt[n]{ P } \) and that \( q_0 \) does not belong to \( \sqrt[n]{ P } \). Then \( q_0^n \) is not in \( P \), but it is a supremum of \( P \). But \( P \) must contain its supremum. The obtained contradiction shows that, if \( \sqrt[n]{ P } \) has a supremum, the latter must belong to \( \sqrt[n]{ P } \).

  \Fullref{thm:def:dedekind_macnielle_closure/totally_ordered} then implies that \( \sqrt[n]{ P } \) is Dedekind-MacNeille closed.
\end{defproof}

\begin{proposition}\label{thm:nth_root_is_irrational}
  For any \hyperref[def:prime_number]{prime number} \( p \), consider its embedding into the \hyperref[def:real_numbers]{real numbers}
  \begin{equation*}
    P = \set{ s \in \BbbQ \given s \leq p }.
  \end{equation*}

  We claim that, for integers \( n > 2 \), the \( n \)-th \hyperref[def:nth_root]{root} \( \sqrt[n]{ P } \) is \hyperref[def:irrational_numbers]{irrational}.
\end{proposition}
\begin{proof}
   If \( \sqrt[n]{ P } \) is rational, then by definition it has a supremum. Denote this supremum by \( s \). Then \( s^n \) belongs to \( P \). Furthermore, \( s^n \) must be the supremum of \( P \) due to \fullref{thm:def:rational_numbers_ordering/power_monotone}. But \( p \) is the supremum of \( P \), and \fullref{thm:nth_root_is_not_rational} establishes that there exists no rational number \( q \) such that \( q^n = p \). But \( p = s^n \).

   The obtained contradiction shows that \( \sqrt[n]{ P } \) is irrational.
\end{proof}

\begin{proposition}\label{thm:power_of_nth_root}
  The \( n \)-th \hyperref[def:nth_root]{root} of \( P \) satisfies
  \begin{equation}\label{eq:thm:def:nth_root}
    \parens[\Big]{ \sqrt[n]{ P } }^n = P.
  \end{equation}
\end{proposition}
\begin{proof}
  We will use induction on \( m \geq 1 \) to show that
  \begin{equation}\label{eq:thm:def:nth_root/aux}
    \sqrt[n]{ P }^m = \set{ p^m \given p > 0 \T{and} p^n \in P }^{UL}.
  \end{equation}

  The base case \( m = 1 \) is the definition of \( n \)-th root. Suppose that \eqref{eq:thm:def:nth_root/aux} holds for some fixed \( m \). If \( \sqrt[n]{ P }^m \) has a supremum, then this supremum multiplied by any positive \( q \) in \( \sqrt[n]{ P } \) does not exceed the supremum of \( \sqrt[n]{ P }^{m+1} \). Hence,
  \begin{equation*}
    \sqrt[n]{ P }^{m+1} = \sqrt[n]{ P }^m \cdot \sqrt[n]{ P } = \set[\Big]{ p^m \cdot q \given p > 0 \T{and} q > 0 \T{and} p^n \in P \T{and} q^n \in P }^{UL}
  \end{equation*}

  Note that if \( p > q \), we have \( p^{m+1} > p^m \cdot q \). Since the same restrictions are placed on both \( p \) and \( q \), any upper bound of numbers of the form \( p^m \cdot q \) is also an upper bound of numbers of the form \( p^{m+1} \) and vice versa. Thus,
  \begin{equation*}
    \sqrt[n]{ P }^{m+1}
    =
    \set[\Big]{ p^{m+1} \given* p > 0 \T{and} p^n \in P }^{UL}.
  \end{equation*}

  Then \eqref{eq:thm:def:nth_root} follows for \( m = n \).
\end{proof}

\paragraph{Ceiling and floor}

\begin{definition}\label{def:real_floor_ceiling}\mimprovised
  For any \hyperref[def:real_numbers]{real number} \( P \), we define its \term{floor} as
  \begin{equation*}
    \floor(P) \coloneqq \max\set[\Big]{ \iota(n) \given* n \in \BbbZ \T{and} \iota(n) \subseteq P }
  \end{equation*}
  and its \term{ceiling} as
  \begin{equation*}
    \ceil(P) \coloneqq \min\set[\Big]{ \iota(n) \given* n \in \BbbZ \T{and} \iota(n) \supseteq P }.
  \end{equation*}
\end{definition}

\begin{proposition}\label{thm:real_floor_ceiling_interval}
  For any real number \( P \), we have
  \begin{equation*}
    \floor(P) \subseteq P \subseteq \ceil(P).
  \end{equation*}

  \begin{thmenum}
    \thmitem{thm:real_floor_ceiling_interval/integer} If \( P \) is an integer, then
    \begin{equation*}
      \floor(P) = P = \ceil(P).
    \end{equation*}

    \thmitem{thm:real_floor_ceiling_interval/non_integer} If \( P \) is not an integer, then
    \begin{equation*}
      \floor(P) \subsetneq P \subsetneq \ceil(P).
    \end{equation*}
  \end{thmenum}
\end{proposition}
\begin{proof}
  Let \( n \) be the smallest integer such that \( P \subseteq \iota(n) \).
  \begin{itemize}
    \item If \( P = \iota(n) \), then also \( \iota(n) \subseteq P \) and hence \( P \) is its own floor and ceiling.
    \item If \( P \subsetneq \iota(n) = \ceil(P) \), then \( n - 1 \) is the largest integer such that \( \iota(n - 1) \subseteq P \). We conclude that
    \begin{equation*}
      \floor(P) = \iota(n - 1) \subsetneq P \subsetneq \iota(n) = \ceil(P).
    \end{equation*}
  \end{itemize}
\end{proof}

\begin{proposition}\label{thm:rational_number_floor}
  For rational numbers, we have
  \begin{equation}\label{eq:thm:rational_number_floor}
    \floor(p / q) = \quot(p, q).
  \end{equation}
\end{proposition}
\begin{proof}
  \Fullref{alg:integer_division} gives us
  \begin{equation*}
    p = q \cdot \quot(p, q) + \rem(p, q).
  \end{equation*}

  Since \( 0 \leq \rem(p) < q \), we have
  \begin{equation*}
    \frac {\rem(p, q)} q < 1,
  \end{equation*}
  hence \eqref{eq:thm:rational_number_floor} follows.
\end{proof}

\paragraph{Polynomials}

\begin{proposition}\label{thm:nth_root_polynomial}
  The \hyperref[def:nth_root]{\( n \)-th root} \( \sqrt[n]{ P } \) of a nonnegative \hyperref[def:real_numbers]{real number} \( P \) is a \hyperref[def:polynomial_root]{polynomial root} of the \hyperref[def:polynomial_algebra/polynomials]{polynomial} \( X^n - P \).

  If \( n \) is even, the additive inverse \( -\sqrt[n]{ P } \) is also a root of the same polynomial.
\end{proposition}
\begin{comments}
  \item This proposition highlights how the phrase \enquote{the root of \( P \)} is possibly non-unique unless we clarify what the word \enquote{root} means.
\end{comments}
\begin{proof}
  Follows from \fullref{thm:power_of_nth_root}.
\end{proof}

\begin{proposition}\label{thm:x2_plus_one_no_root}
  The \hyperref[def:polynomial_algebra/polynomials]{polynomial} \( X^2 + 1 \) has no \hyperref[def:polynomial_root]{root} over the \hyperref[def:real_numbers]{real numbers}.
\end{proposition}
\begin{comments}
  \item We denote here by \( 1 \) the integer \( 1 \) and by \( I \) the corresponding real number. We take a polynomial over the integers and show that it does not have a root over the reals.
\end{comments}
\begin{proof}
  For every real number \( P \), by \hyperref[def:binary_relation/trichotomy]{trichotomy}, we have the following possibilities for \( P^2 \):
  \begin{itemize}
    \item If \( P = O \), then \( P^2 = O^2 = 0 \).
    \item If \( P \supsetneq O \), then \( P^2 \supsetneq O \).
    \item If \( P \subsetneq O \), then again \( P^2 = (-P)^2 \supsetneq O \).
  \end{itemize}

  In all cases, \( P^2 \supsetneq O \), and thus \( P^2 + I \supsetneq 1 \). Therefore, \( P \) cannot be a root of the (integer) polynomial \( X^2 + 1 \).

  Since \( P \) was arbitrary, we conclude that the polynomial \( X^2 + 1 \) has no real root.
\end{proof}

\begin{proposition}\label{thm:reals_not_algebraically_closed}
  The field \( \BbbR \) of \hyperref[def:real_numbers]{real numbers} is not \hyperref[def:algebraically_closed_field]{algebraically closed}.
\end{proposition}
\begin{proof}
  Follows from \fullref{thm:x2_plus_one_no_root}.
\end{proof}
