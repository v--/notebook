\subsection{Real numbers}\label{subsec:real_numbers}

\begin{definition}\label{def:real_numbers}\mcite[sec. I.V.]{Beman1901Dedekind}
  We define the set \( \BbbR \) of \term[bg=реални числа (\cite[sec. 1.1.2]{Соскова2015}), ru=вещественные числа (\cite[\textnumero 3]{ФихтенгольцОсновыТом1}) / действительные числа (\cite[36]{Александров1977Введение})]{real numbers} as the \hyperref[def:dedekind_completion]{Dedekind completion} of the set \( \BbbQ \) of \hyperref[def:rational_numbers]{rational numbers}.
\end{definition}
\begin{comments}
  \item We consider the following an indispensable part of \( \BbbR \):
  \begin{itemize}
    \item The field structure defined in \fullref{def:real_number_arithmetic}.
    \item The order topology defined via \fullref{def:order_topology}.
  \end{itemize}

  \item Our definition of Dedekind completion uses \hyperref[def:dedekind_macnielle_closure]{Dedekind-MacNeille-closed sets}, which has certain consequences compared to other authors --- see \fullref{rem:dedekind_completion_through_dedekind_macneille_closures}.
\end{comments}

\begin{definition}\label{def:real_number_arithmetic}\mimprovised
  For each pair of \hyperref[def:real_numbers]{real numbers} \( R \) and \( Q \), considered as \hyperref[def:dedekind_macnielle_closure]{Dedekind-MacNeille-closed sets} of rational numbers, we define the base operations:
  \begin{thmenum}
    \thmitem{def:real_number_arithmetic/zero} Zero:
    \begin{equation}\label{eq:def:real_number_arithmetic/addition}
      \begin{split}
        \mathllap{O} &\coloneqq \mathrlap{\set{ s \given s \leq 0 }}.
      \end{split}
    \end{equation}

    \thmitem{def:real_number_arithmetic/additive_inverse} Additive inverses:
    \begin{equation}\label{eq:def:real_number_arithmetic/additive_inverse}
      \begin{split}
        \mathllap{-P} \coloneqq \mathrlap{\set{ -p \given p \in P }^L}.
      \end{split}
    \end{equation}

    \thmitem{def:real_number_arithmetic/addition} Addition:
    \begin{equation}\label{eq:def:real_number_arithmetic/addition}
      \begin{split}
        \mathllap{P + Q} \coloneqq \mathrlap{\set{ p + q \given p \in P \T{and} q \in Q }^{UL}}.
      \end{split}
    \end{equation}

    \thmitem{def:real_number_arithmetic/one} One:
    \begin{equation}\label{eq:def:real_number_arithmetic/addition}
      \begin{split}
        \mathllap{I} \coloneqq \mathrlap{\set{ s \given s \leq 1 }}.
      \end{split}
    \end{equation}

    \thmitem{def:real_number_arithmetic/multiplicative_inverse} Multiplicative inverses (for \( P \neq O \)):
    \begin{equation}\label{eq:def:real_number_arithmetic/multiplicative_inverses}
      \mathllap{P^{-1}} \coloneqq \mathrlap{\set{ p^{-1} \given p \in P }^L}.
    \end{equation}

    \thmitem{def:real_number_arithmetic/multiplication} Multiplication:
    \begin{equation}\label{eq:def:real_number_arithmetic/multiplication}
      P \cdot Q \coloneqq \begin{cases}
        O,                                                                             &O = P \T{or} O = Q \\
        \set{ p \cdot q \given r \in P \setminus O \T{and} q \in Q \setminus O }^{UL}, &O \subsetneq P \T{and} O \subsetneq Q \\
        (-P) \cdot Q,                                                                  &O \supsetneq P \T{and} O \subsetneq Q \\
        P \cdot (-Q),                                                                  &O \subsetneq P \T{and} O \supsetneq Q \\
        (-P) \cdot (-Q),                                                               &O \supsetneq P \T{and} O \supsetneq Q \\
      \end{cases}
    \end{equation}
  \end{thmenum}
\end{definition}
\begin{comments}
  \item \incite[113]{Enderton1977Sets} instead defines the arithmetic operations for cuts as discussed in \fullref{rem:dedekind_completion_through_dedekind_macneille_closures}, which makes addition simpler, but nearly all other operations more difficult.
  \item The complications in \eqref{eq:def:real_number_arithmetic/multiplication} arise from the fact that \( I \) contains all negative rational numbers and, when multiplied by \( -1 \), these become positive. Thus, if multiplication is always treated like in the first case, multiplying \( I \) by \( -1 \) would give the entire set \( \BbbR \).
\end{comments}

\begin{proposition}\label{thm:real_numbers_are_a_field}
  The operations from \fullref{def:real_number_arithmetic} make the set \( \BbbR \) of \hyperref[def:real_numbers]{real numbers} a \hyperref[def:field]{field}. Furthermore, the \hyperref[def:order_homomorphism/embedding]{order embedding} \( \iota: \BbbQ \to \BbbR \) from \fullref{def:dedekind_macnielle_completion} is a \hyperref[def:field/homomorphism]{field embedding}.
\end{proposition}
\begin{proof}
  \SubProof{Proof that addition is associative} First note that the set of sums \( r + q \), where \( r \in R \) and \( q \in Q \), is downward closed. Indeed, if \( s \leq r + q \), then \( s = (r + q) - s_0 \), where \( s_0 \coloneqq (r + q) - s \). The latter is nonnegative, hence \( q - s_0 \leq q \) and thus \( q - s_0 \in Q \). Then \( s = r + (q - s_0) \), where \( r \) belongs to \( R \) and \( q - s_0 \) belongs to \( Q \). Thus, the set \( \set{ r + q \given r \in R \T{and} q \in Q } \) is downward closed.

  \Fullref{thm:def:dedekind_macnielle_closure/point_in_closure} then implies that every member of \( R + Q \) is either a sum of the form \( r + q \) for \( r \in R \) and \( q \in Q \), or the supremum of such sums.

  Now let \( x \in (P + Q) + R \). We have two possibilities:
  \begin{itemize}
    \item If \( x = s + r \) for some \( s \in P + Q \) and \( r \in R \), we again have two possibilities:
    \begin{itemize}
      \item If \( s = p + q \) for some \( p \in P \) and \( q \in Q \), then \( x = (p + q) + r \). Then obviously \( x = p + (q + r) \) is a member of \( P + (Q + R) \).

      \item If \( s = \sup(P + Q) \), consider some \( s' < s \) and let \( r' \coloneqq r - (s - s') \). Since \( R \) is downward closed, it contains \( r' \). Since \( s' \) is not the supremum of \( P + Q \) (but belongs to the latter set), by \fullref{thm:def:dedekind_macnielle_closure/point_in_closure}, there exist some \( p' \in P \) and \( q' \in Q \) such that \( s' = p' + q' \). Then \( x = (p' + q') + r' \), which again implies \( x \in P + (Q + R) \).
    \end{itemize}

    \item Suppose that \( x = \sup((P + Q) + R) \).

    Let \( x' < x \) so that \( x' \) belongs to \( (P + Q) + R \) and \( x = x' + (x - x') \). Then there exist some \( p' \in P \), \( q' \in Q \) and \( r' \in R \) such that \( x' = (p' + q') + r' \). Clearly then \( x' \) belongs to \( P + (Q + R) \).

    Thus, every rational number strictly smaller than \( x \) belongs to \( P + (Q + R) \). But \( x \) is the supremum of such numbers, and \fullref{thm:def:dedekind_macnielle_closure/closed_elements} implies that this supremum must also belong to \( P + (Q + R) \).
  \end{itemize}

  We have shown that \( (P + Q) + R \subseteq P + (Q + R) \). The converse inclusion can be proven symmetrically. Hence, we conclude that addition of real numbers is associative.

  \SubProof{Proof that addition is commutative} Follows directly from commutativity of addition of rational numbers.

  \SubProof{Proof that addition has a neutral element} We will show that \( P + O = P \).

  Since \( p + 0 = p \), clearly \( P \subseteq P + O \).

  For the converse, note that \Fullref{thm:def:dedekind_macnielle_closure/closed_elements} implies that \( P \) is downward closed, hence \( p \in P \) and \( o < 0 \) imply that \( p + o \in P \). Since \( P \) is also closed under taking suprema of subsets for which suprema exist, it follows that \( P + O \subseteq P \).

  \SubProof{Proof that addition is invertible} We will show that \( P + (-P) = O \). Fix \( p \in P \).

  The number \( -p \) is an upper bound for \( -P \), hence \( 0 = p + (-p) \) is an upper bound for \( P + (-P) \). Thus, \( P + (-P) \subseteq O \).

  Conversely, if \( s < 0 \), then \( p - s > p \) is an upper bound of \( P \), and hence \( s - p \) is in \( -P \). Then \( s = p + (s - p) \) is in \( P + (-P) \). Hence, \( O \subseteq P + (-P) \).

  \SubProof{Proof that multiplication is commutative} Follows directly from commutativity of multiplication of rational numbers.

  \SubProof{Proof that multiplication respects additive inverses} We will show that
  \begin{equation}\label{eq:thm:real_numbers_are_a_field/mult_additive_inverses}
    P \cdot (-Q) = (-P) \cdot Q = -(P \cdot Q).
  \end{equation}

  The case where either \( P = O \) or \( Q = O \) is trivial, and we will not consider it.

  \SubProof*{Proof for \( O \subsetneq P \) and \( O \subsetneq Q \)} Since \( -Q \) is the additive inverse of \( Q \), we have \( O \supsetneq -Q \), and thus
  \begin{equation*}
    P \cdot (-Q)
    \reloset {\eqref{eq:def:real_number_arithmetic/multiplication}} =
    -(P \cdot (--Q))
    =
    -(P \cdot Q).
  \end{equation*}

  Similarly,
  \begin{equation*}
    (-P) \cdot Q
    \reloset {\eqref{eq:def:real_number_arithmetic/multiplication}} =
    -((--P) \cdot Q)
    =
    -(P \cdot Q).
  \end{equation*}

  \SubProof*{Proof for \( O \supsetneq P \) and \( O \subsetneq Q \)}
  \begin{equation*}
    P \cdot (-Q)
    \reloset {\eqref{eq:def:real_number_arithmetic/multiplication}} =
    (-P) \cdot (--Q)
    =
    (-P) \cdot Q
    \reloset {\eqref{eq:def:real_number_arithmetic/multiplication}} =
    -((--P) \cdot Q)
    =
    -(P \cdot Q).
  \end{equation*}

  \SubProof*{Proof for \( O \subsetneq P \) and \( O \supsetneq Q \)} Follows from the above via commutativity.

  \SubProof*{Proof for \( O \supsetneq P \) and \( O \supsetneq Q \)}
  \begin{equation*}
    P \cdot (-Q)
    \reloset {\eqref{eq:def:real_number_arithmetic/multiplication}} =
    ((-P) \cdot (-Q))
    \reloset {\eqref{eq:def:real_number_arithmetic/multiplication}} =
    (-P) \cdot Q
  \end{equation*}
  and
  \begin{equation*}
    -(P \cdot Q)
    =
    --((-P) \cdot (-Q))
    =
    (-P) \cdot (-Q).
  \end{equation*}

  \SubProof{Proof that multiplication is associative} We have many cases, but fortunately most of them are similar. We will only consider two, excluding the trivial case where \( P = O \) or \( Q = O \) or \( R = O \).

  \SubProof*{Proof for \( O \subsetneq P \), \( O \subsetneq Q \) and \( O \subsetneq R \)} First note that \( O \subsetneq (P \cdot Q) \cdot R \) and \( O \subsetneq P \cdot (Q \cdot R) \) Indeed, there exist positive numbers \( p \in P \) and \( q \in Q \) such that \( pq \in P \cdot Q \), and thus \( O \subsetneq P \cdot Q \). Similarly, there exists a positive number \( r \in R \) such that \( pqr \in (P \cdot Q) \cdot R \). But \( pqr \) also belongs to \( P \cdot (Q \cdot R) \). Hence, both sets strictly contain \( O \).

  Now that we have show that the two products have the same sign, we will show that they are equal.

  Let \( x \in (P \cdot Q) \cdot R \) and \( x > 0 \). As in the case of addition, \fullref{thm:def:dedekind_macnielle_closure/point_in_closure} implies that we have the following cases:
  \begin{itemize}
    \item If there exist some \( s \in P \cdot Q \) and \( r \in R \) such that \( x = s \cdot r \), we again have two possibilities:
    \begin{itemize}
      \item If there exist some \( p \in P \) and \( q \in Q \) such that \( s = p \cdot q \), then \( x = (p \cdot q) \cdot r = p \cdot (q \cdot r) \) belongs to \( P \cdot (Q \cdot R) \).

      \item If \( s = \sup(P \cdot Q) \), consider some positive \( s' < s \) and let \( r' \coloneqq s' \cdot s^{-1} \cdot r \). Then \( x = s \cdot r = s' \cdot r' \). Furthermore, since \( s' < s \), the quotient \( \ifrac {s'} s \) is less than \( 1 \), hence \( r' < r \) and \( r' \) belongs to \( R \).

      Then \( s' \) is not the supremum of \( P \cdot Q \), and hence there exist positive numbers \( p' \in P \) and \( q' \in Q \) such that \( s' = p' \cdot q' \) and thus \( x' = (p' \cdot q') \cdot r' \).

      It is now clear that \( x \in P \cdot (Q \cdot R) \).
    \end{itemize}

    \item Suppose that \( x = \sup((P \cdot Q) \cdot R) \).

    Fix some \( x' \) such that \( 0 < x' < x \). Then there exist some positive \( p' \in P \), \( q' \in Q \) and \( r' \in R \) such that \( x' = (p' \cdot q') \cdot r' \). Clearly then \( x' \in P \cdot (Q \cdot R) \).

    Thus, every positive member of \( (P \cdot Q) \cdot R \) belongs to \( P \cdot (Q \cdot R) \). Since \( x \) is the supremum of such members, it follows that \( x \) must also belong to \( P \cdot (Q \cdot R) \), which is closed under suprema of subsets that have suprema.
  \end{itemize}

  We have shown that \( (P \cdot Q) \cdot R \subseteq P \cdot (Q \cdot R) \).  The converse inclusion can be proven symmetrically.

  \SubProof*{Proof for \( O \supsetneq P \), \( O \subsetneq Q \) and \( O \subsetneq R \)} We use \eqref{eq:thm:real_numbers_are_a_field/mult_additive_inverses} to reduce this case to the first:
  \begin{equation*}
    P \cdot (Q \cdot R)
    \reloset {\eqref{eq:thm:real_numbers_are_a_field/mult_additive_inverses}} =
    -[(-P) \cdot (Q \cdot R)]
    =
    -[((-P) \cdot Q) \cdot R]
    \reloset {\eqref{eq:thm:real_numbers_are_a_field/mult_additive_inverses}} =
    -[(-(P \cdot Q)) \cdot R]
    \reloset {\eqref{eq:thm:real_numbers_are_a_field/mult_additive_inverses}} =
    (P \cdot Q) \cdot R.
  \end{equation*}

  \SubProof*{Proof for all other cases} The other cases follow similarly from the first via \eqref{eq:thm:real_numbers_are_a_field/mult_additive_inverses}.

  \SubProof{Proof that multiplication has a neutral element} This is similar to the proof for addition. We will show that \( P \cdot I = P \).

  Since \( p \cdot 1 = p \), clearly \( P \subseteq P \cdot I \).

  For the converse, note that \Fullref{thm:def:dedekind_macnielle_closure/closed_elements} implies that \( P \) is downward closed, hence \( p \in P \) and \( 0 < i < 1 \) imply that \( pi \in P \). Since \( P \) is also closed under taking suprema of subsets for which suprema exist, it follows that \( P \cdot I \subseteq P \).

  \SubProof{Proof that multiplication is invertible} We will show that \( P \cdot P^{-1} = I \). This is similar to the proof for addition. Fix some \( p \in P \).

  The number \( p^{-1} \) is an upper bound for \( P^{-1} \), hence \( 1 = p \cdot p^{-1} \) is an upper bound for \( P \cdot P^{-1} \). Thus, \( P \cdot P^{-1} \subseteq I \).

  Conversely, if \( 0 < s < 1 \), then \( p \cdot s^{-1} > p \) is an upper bound of \( P \), and hence \( s \cdot p^{-1} \) is in \( P^{-1} \). Then \( s = p \cdot (s \cdot p^{-1}) \) is in \( P \cdot P^{-1} \). Hence, \( I \subseteq P \cdot P^{-1} \).

  \SubProof{Proof of distributivity} We will show that \( P(Q + R) = PQ + PR \). Skipping the trivial cases, we only need to consider the case where \( O \subsetneq P \) and \( O \subsetneq Q + R \) because the rest will follow via \eqref{eq:def:real_number_arithmetic/multiplication}.

  Again, \fullref{thm:def:dedekind_macnielle_closure/point_in_closure} gives us several possibilities for \( x \in P(Q + R) \):
  \begin{itemize}
    \item If \( x = p \cdot s \) for some \( p \in P \) and \( s \in Q + R \), then we have two possibilities again:
    \begin{itemize}
      \item If \( s = q + r \) for some \( q \in Q \) and \( r + R \), then \( x = p(q + r) \) and distributivity on the rational numbers imply \( x \in PQ + PR \).

      \item If \( s = \sup(Q + R) \), since \( p \) is positive, as in our proofs of associativity we have \( x = p' \cdot (q' + r') \) for some \( q' \in Q \) and \( r' \in Q \), and distributivity on the rational again implies \( x \in PQ + PR \).
    \end{itemize}

    \item If \( x = \sup(P(Q + R)) \), then, since \( O \subsetneq P(Q + R) \), we can again conclude that \( x' < x \) implies \( x' \in PQ + PR \), which in turn implies that \( x \in PR + PR \).
  \end{itemize}

  Therefore, \( P(Q + R) \subseteq PQ + PR \).

  Conversely, if \( x \in PQ + PR \), then:
  \begin{itemize}
    \item If \( x = q_0 + r_0 \) for \( q_0 \in PQ \) and \( r_0 \in PR \), then both are positive, and we have more possibilities:
    \begin{itemize}
      \item If \( q_0 = p_q q \) for positive \( p_q \in P \) and \( q \in Q \) and if similarly \( r_0 = p_r r \), let \( p \) be the larger of the two. Then
      \begin{equation*}
        x = p\parens[\Big]{ \underbrace{\frac {p_q} p}_{\leq 1} q + \underbrace{\frac {p_r} p}_{\leq 1} r },
      \end{equation*}
      hence \( x \in P(Q + R) \).

      \item If \( q_0 = p_q q \) as above but \( r_0 = \sup(PR) \), we must consider \( r_0' < r_0 \) and argue that \( x \) is the supremum by \( r_0' \).

      \item The case \( q = \sup(PQ) \) and \( r_0 = p_r r \) is symmetric.

      \item If \( q_0 = \sup(PQ) \) and \( r_0 = \sup(PR) \), we must simultaneously consider \( q_0' < q_0 \) and \( r_0' < r_0 \).
    \end{itemize}

    \item If \( x = \sup(PR + PR) \), then we must consider \( x' < x \) and, as in the proofs of associativity, argue that \( x \) is the supremum by \( x' \).
  \end{itemize}

  Therefore, \( PQ + PR \subseteq P(Q + R) \).

  \SubProof{Proof of embedding} Clearly the map \( \iota(p) = \BbbQ_{\leq p} \) is injective as an order embedding. We must show that it is a \hyperref[def:field/homomorphism]{field homomorphism}. Clearly \( O = \iota(0) \) and \( I = \iota(1) \).

  Furthermore, \( \iota(p + q) \) has \( p + q \) as its supremum, while \( \iota(p) \) and \( \iota(q) \) have \( p \) and \( q \) correspondingly. Then \( \iota(p) + \iota(q) \) also has \( p + q \) as its supremum, and hence
  \begin{equation*}
    \iota(p + q) = \iota(p) = \iota(q).
  \end{equation*}

  The base case of multiplication can be handled similarly and then extended via \eqref{eq:thm:real_numbers_are_a_field/mult_additive_inverses}.
\end{proof}
