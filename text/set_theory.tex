\chapter{Set theory}\label{ch:set_theory}

Sets are ubiquitous in mathematics, yet set theory itself is quite complicated. Attention needs to be put to define a \hyperref[def:first_order_theory]{logical theory} of sets that is both useful and \hyperref[def:first_order_theory/consistent]{consistent}.

We first use the simplicity of \hyperref[def:naive_set_theory]{na\"ive set theory} to introduce some fundamental definitions. This theory turns out to be inconsistent due to \fullref{thm:russels_paradox}.

We later introduce the more sophisticated \hyperref[def:zfc]{Zermelo-Fraenkel set theory} (\logic{ZFC}) and enhance it with Grothendieck's \hyperref[def:axiom_of_universes]{axiom of universes} to obtain \logic{ZFC+U}. The consistency of the latter theory is discussed in \cref{rem:set_definition_recursion}.

Depending on the context, by \enquote{set theory} we mean either na\"ive set theory, \logic{ZF}, \logic{ZFC}, \logic{ZFC+U} or further variations.

\begin{remark}\label{rem:set_definition_recursion}
  The relation between \hyperref[sec:first_order_logic]{first-order logic} and set theory is remarkably circular.

  \begin{itemize}
    \item We define set theory as a \hyperref[def:first_order_theory]{theory} of first-order logic.

    \item We define first-order logic itself via sets --- see even the basic definitions from \fullref{sec:first_order_logic}.
  \end{itemize}

  In order to resolve this circularity, we utilize the concept of \hyperref[con:metalogic]{metalogic}:
  \begin{itemize}
    \item We start with some intuitive understanding of sets and build first-order logic as in \fullref{ch:mathematical_logic}. This is our initial metatheory, where we assume the availability of first-order logic. We construct formulas specifying certain special sets that can be used as models of \logic{ZFC} --- see \fullref{sec:grothendieck_universes}

    \item We now transition to work within \logic{ZFC+U} as a metatheory and \logic{ZFC} as an object theory. We use Grothendieck's \hyperref[def:axiom_of_universes]{axiom of universes} in the metatheory because it provides us with \hyperref[def:set_countability/uncountable]{uncountable} \hyperref[def:grothendieck_universe]{Grothendieck universes}\fnote{We can assume the existence of only one such universe, but that would complicate our discussion of \hyperref[ch:category_theory]{category theory}}, which are models of \logic{ZFC} as a consequence of \cref{thm:grothendieck_universe_is_model_of_zfc}. Hence, within this metatheory, the object theory \logic{ZFC} is consistent.
  \end{itemize}

  There are, however, caveats, among which:
  \begin{itemize}
    \item We restrict ourselves to \hyperref[rem:standard_model_of_set_theory]{standard} \hyperref[rem:transitive_model_of_set_theory]{transitive} models in order to avoid very counterintuitive results.

    \item It is possible that the set theory which we use within the metatheory in order to provide models of \logic{ZFC} is itself inconsistent. In that case, due to \eqref{eq:thm:intuitionistic_tautologies/efq}, every theorem can be derived and a proof of consistency of \logic{ZFC} is insubstantial.
  \end{itemize}
\end{remark}

\begin{remark}\label{rem:first_order_theories_in_zfc}
  Instead of discussing first-order theories like the \hyperref[def:group/theory]{theory of groups}, we can instead reformulate the definition within set theory and add a \hyperref[con:formula_defined_predicate]{formula-defined predicate} \( \op{IsGroup}[\synx] \), which is valid only for groups.

  This is a natural approach, and we will use it implicitly. Furthermore, it makes no sense to speak about concepts like the \hyperref[thm:substructures_form_complete_lattice]{lattice of subgroups} or the \hyperref[def:cardinal]{cardinality} of a group otherwise.

  This is also a natural framework for defining \hyperref[def:topological_space]{topological spaces} and \hyperref[def:category]{categories} via \hyperref[def:directed_multigraph]{directed multigraphs}. Some theories like the \hyperref[def:partially_ordered_set]{partially ordered sets} are first-order theories, but \hyperref[def:well_ordered_set]{well-ordered sets} is an extension that requires an ambient set theory.

  Thus, roughly, set theory allows us to use \hyperref[def:simply_typed_hol]{higher-order relations and types} within first-order logic.

  We will end this remark with a quote from \cite[17]{UnivalentFoundationsProgram2013HoTT}:
  \begin{displayquote}
    We note that a set-theoretic foundation has two \enquote{layers}: the deductive system of first-order logic, and, formulated inside this system, the axioms of a particular theory, such as ZFC. Thus, set theory is not only about sets, but rather about the interplay between sets (the objects of the second layer) and propositions (the objects of the first layer).
  \end{displayquote}
\end{remark}
