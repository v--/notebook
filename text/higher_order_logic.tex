\section{Higher-order logic}\label{sec:higher_order_logic}

This section depends on most of \fullref{ch:lambda_calculus}, and provides general definitions that will be used in \fullref{sec:first_order_logic} and \fullref{sec:first_order_models}. At the same time, it is intended to generalize first-order logic, from which the presentation will inevitably suffer.

\begin{concept}\label{con:higher_order_logic}
  \term{Higher-order logic} is a loose term for \hyperref[def:logical_framework]{logical frameworks} whose sentences can encode \hyperref[con:judgment]{judgments} not only about objects of discourse --- called \term{individuals} --- but also encode judgments about simpler judgments.

  We will describe an \hyperref[con:explicit_and_implicit_typing]{implicitly-typed} variant of Peter Andrews' system \( \logic{Q}_0 \) presented in \cite[\S 51]{Andrews2002Logic} and, with some variations, in \incite{Farmer2008STTVirtues}. Andrews' system is itself a variation of Church's simply typed \( \lambda \)-calculus from \cite{Church1940STT}.

  There are also frameworks not based on \( \synlambda \)-calculus; some are described by \incite[\S 3.6]{Hinman2005Logic}.
\end{concept}

\begin{definition}\label{def:simply_typed_hol}\mimprovised
  We will define a family of \hyperref[def:lambda_term_typing_style]{explicitly typed} \hyperref[def:simple_type_system]{simple type systems} which we will call \term{simply typed higher-order logic}.

  First, we will need the following \hyperref[def:simple_type]{base types}\fnote{Note that, in accordance with \fullref{rem:object_language_dots/terminals}, we place dots over symbols in the object logic.}:
  \begin{thmenum}[series=def:simple_type_system/base]
    \thmitem{def:simply_typed_hol/base/propositions}\mcite[57]{Church1940STT} The \term{type of propositions} \( \syn\omicron \).
    \thmitem{def:simply_typed_hol/base/iota}\mcite[57]{Church1940STT} The \term{type of individuals} \( \syn\iota \) (Small Greek iota \( \iota \) with a dot).
  \end{thmenum}

  We will also need the following \hyperref[def:lambda_term]{constant terms}:
  \begin{thmenum}
    \thmitem{def:simply_typed_hol/const/equality} The \term{equality term} \( \synQ \).
    \thmitem{def:simply_typed_hol/const/description} The \term{definite description term} \( \synI \).
  \end{thmenum}
\end{definition}

\begin{definition}\label{def:nth_order_logic}
\end{definition}
