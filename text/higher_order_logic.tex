\section{Higher-order logic}\label{sec:higher_order_logic}

\begin{definition}\label{def:weak_q0}
  We will define a \hyperref[con:implicit_typing]{implicitly-typed} variant of Peter Andrews' system \( \logic{Q}_0 \), which we will call \term{weak \( \logic{Q}_0 \)}.

  \begin{thmenum}[series=def:simple_type_system/base]
    \thmitem{def:simple_type_system/base/propositions}\mcite[57]{Church1940STT} The \term{type of propositions} \( \synomicron \).
    \thmitem{def:simple_type_system/base/iota}\mcite[57]{Church1940STT} The \term{type of individuals} \( \syniota \) (Small Greek iota \( \iota \) with a dot).
  \end{thmenum}

  Note that, in accordance with \fullref{rem:object_language_dots/terminals}, we place dots over symbols in the object logic.

  Other useful base types include:
  \begin{thmenum}[resume=def:simple_type_system/base]
    \thmitem{def:simple_type_system/base/empty}\mcite[\S 4.3.4]{Mimram2020Types} The \term{empty type} \( \syn\Bbbzero \).
    \thmitem{def:simple_type_system/base/unit}\mcite[\S 4.3.2]{Mimram2020Types} The \term{unit type} \( \syn\Bbbone \).
  \end{thmenum}

  \thmitem{def:simple_type_system/const} Fix another, possibly empty, alphabet \( \op*{Const}_\Sigma \), whose entries we will use as \hyperref[def:lambda_term/const]{constant \( \synlambda \)-terms}.

  We will need the following constants based on Andrews' system \( \logic{Q}_0 \):
  \begin{thmenum}
    \thmitem{def:simple_type_system/const/equality}\mcite[208]{Andrews2002STT} The \term{equality term} \( \synQ \).
    \thmitem{def:simple_type_system/const/description}\mcite[208]{Andrews2002STT} The \term{definite description term} \( \synI \).
  \end{thmenum}
\end{definition}
