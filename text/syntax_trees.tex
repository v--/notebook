\subsection{Syntax trees}\label{subsec:syntax_trees}

\begin{definition}\label{def:parse_tree}\mimprovised
  A \term{parse tree} or \term{concrete syntax tree} for the \hyperref[def:chomsky_hierarchy/context_free]{context-free grammar} \( G = (V, \Sigma, \to, S) \) is an \hyperref[def:ordered_tree]{ordered tree} \( T = (X, E, r, \leq) \) \hyperref[def:weighted_set]{node-labeled} by \( l: X \to V \cup \Sigma \cup \set{ \varepsilon } \) such that:
  \begin{thmenum}
    \thmitem{def:parse_tree/root} The root \( r \) has label \( l(r) = S \).
    \thmitem{def:parse_tree/leaves} The \hyperref[def:arborescence/ancestry]{leaves} of \( T \) are labeled by terminals or by \( \varepsilon \).
    \thmitem{def:parse_tree/successors} The \hyperref[def:arborescence/ancestry]{successors} of \( x \) are \( y_1 < \ldots < y_n \) if and only if there is a rule
    \begin{equation*}
      l(x) \to l(y_1) \cdots l(y_n).
    \end{equation*}

    We handle the special case \( l(x) \to \varepsilon \) by introducing a single child for \( x \) with a label \( \varepsilon \). This is done in order to distinguish \( \varepsilon \)-rules from \enquote{dangling} non-terminals.
  \end{thmenum}
\end{definition}
\begin{comments}
  \item We associate with each parse tree a word as shown in \fullref{def:parse_tree_word}.
\end{comments}

\begin{remark}\label{rem:parse_tree_roots}
  In recursive definitions and inductive proofs, for example in relation to \hyperref[rem:evaluation]{evaluation}, we usually consider \hyperref[ex:natural_number_arithmetic_grammar/rules]{grammar schemas} and each subtree has a root with a different label. In this case, we work with parse trees for different grammars over the same grammar schema (differing by the start symbol). We also allow singleton trees consisting of a single terminal or \( \varepsilon \), acknowledging that these are not technically parse trees.
\end{remark}

\begin{definition}\label{def:parse_tree_word}\mimprovised
  Let \( T = (X, A, r, \leq, l) \) be a \hyperref[def:parse_tree]{parse tree}. Let \( x_1, \ldots, x_n \) be the nodes in the \hyperref[def:traversal_ordering]{pre-order} of \( T \) whose labels are terminals. We say that \( T \) \term{yields} the word
  \begin{equation*}
    l(x_1) \cdots l(x_n).
  \end{equation*}

  We implicitly associate this word with the parse tree.
\end{definition}

\begin{example}\label{ex:def:parse_tree}
  We give several examples of \hyperref[def:parse_tree]{parse trees}.

  \begin{thmenum}
    \thmitem{ex:def:parse_tree/an} Consider the grammar
    \begin{equation*}
      \begin{aligned}
        S &\to A \mid \varepsilon, \\
        A &\to aA \mid a.
      \end{aligned}
    \end{equation*}
    from \fullref{ex:def:chomsky_hierarchy/an}.

    It describes the language \( \mscrL = \set{ a^n \given n \geq 0 } \) discussed in \fullref{ex:def:formal_language/an}. The only possible parse tree for the word \( aaa \) is
    \begin{equation*}\label{eq:ex:def:parse_tree/an}
      \begin{aligned}
        \includegraphics[page=1]{output/ex__def__parse_tree.pdf}
      \end{aligned}
    \end{equation*}

    If we instead use the \enquote{simpler} grammar
    \begin{equation*}
      S \to aS \mid \varepsilon,
    \end{equation*}
    then we would have an \( \varepsilon \)-labeled node in the parse tree:
    \begin{equation*}
      \begin{aligned}
        \includegraphics[page=2]{output/ex__def__parse_tree.pdf}
      \end{aligned}
    \end{equation*}

    \thmitem{ex:def:parse_tree/anbn} Consider the grammar
    \begin{equation*}
      \begin{aligned}
        S &\to A \mid \varepsilon, \\
        A &\to aAb \mid ab
      \end{aligned}
    \end{equation*}
    from \fullref{ex:def:chomsky_hierarchy/anbn} describing \( \mscrL = \set{ a^n b^n \given n \geq 0 } \) from \fullref{ex:def:formal_language/anbn}.

    The only possible parse tree for the word \( aaabbb \) is
    \begin{equation*}
      \begin{aligned}
        \includegraphics[page=3]{output/ex__def__parse_tree.pdf}
      \end{aligned}
    \end{equation*}

    \thmitem{ex:def:parse_tree/arithmetic} We continue the binary natural number grammar example from \fullref{ex:natural_number_arithmetic_grammar/rules} with the grammar \eqref{eq:ex:natural_number_arithmetic_grammar/rules/shorthand}.

    The only possible parse tree for the expression \( (10 \times (1 + 10)) \) is
    \begin{equation*}
      \begin{aligned}
        \includegraphics[page=4]{output/ex__def__parse_tree.pdf}
      \end{aligned}
    \end{equation*}

    We will discuss abstract syntax trees in \fullref{rem:abstract_syntax_tree}, which are notably smaller.

    We can \hyperref[rem:evaluation]{evaluate} this parse tree by substituting every binary number string (such as \( 10 \)) with its numeric value in \( \BbbN \) and then recursively applying the corresponding expressions. Evaluating the above parse tree results in \( 10 \times 11 = 110 \). More formal details on this evaluation are discussed in \fullref{ex:natural_number_arithmetic_grammar/evaluation}.

    Now consider instead the same grammar without parentheses, that is, replace the expression expansion rule
    \begin{equation*}
      E \to (E O E)
    \end{equation*}
    with
    \begin{equation*}
      E \to E O E.
    \end{equation*}

    The corresponding expression \( 10 + 1 \times 10 \) has more than one parse tree:
    \begin{equation*}
      \begin{aligned}
        \includegraphics[page=5]{output/ex__def__parse_tree.pdf}
        \qquad\qquad
        \includegraphics[page=6]{output/ex__def__parse_tree.pdf}
      \end{aligned}
    \end{equation*}

    If we evaluate the first parse tree, we obtain \( 10 \times (1 + 10) = 10 \times 11 = 110 \). If we evaluate the second one, we obtain \( (10 \times 1) + 10 = 10 + 10 = 100 \) (\( 10 \) in binary notation is \( 2 \) is decimal notation, and \( 100 \) in binary is \( 4 \) in decimal).

    Thus, removing parentheses makes the grammar ambiguous in the sense of \fullref{def:grammar_ambiguity}. We will show in \fullref{ex:natural_number_arithmetic_grammar/unambiguous} that the parenthesized grammar is unambiguous.
  \end{thmenum}
\end{example}

\begin{definition}\label{def:leftmost_derivation}\mcite[53]{Salomaa1987}
  In a context-free grammar, we say that the derivation
  \begin{equation*}
    S \Rightarrow w_1 \Rightarrow \cdots \Rightarrow w_m = w
  \end{equation*}
  is \term{leftmost} if, given any step \( w_{k-1} = p_k A_k s_k \Rightarrow p_k v_k s_k = w_k \), the prefix \( p_k \) contains only terminals.

  We define \term{rightmost} by instead requiring that the suffix \( s_k \) contains only terminals.
\end{definition}

\begin{example}\label{ex:natural_number_arithmetic_grammar/derivation}
  We will consider derivations in the binary natural number grammar \eqref{eq:ex:natural_number_arithmetic_grammar/rules/shorthand} from \fullref{ex:natural_number_arithmetic_grammar/rules}.

  We have shown in \fullref{ex:def:parse_tree/arithmetic} that removing parentheses results in multiple parse trees yielding the same word.

  Similarly, using parentheses results in a single leftmost derivation:
  \begin{equation*}
    \begin{aligned}
      \includegraphics[page=1]{output/ex__natural_number_arithmetic_grammar__derivation.pdf}
    \end{aligned}
  \end{equation*}

  Removing parentheses allows for multiple leftmost derivations:
  \begin{equation*}
    \begin{aligned}
      \includegraphics[page=2]{output/ex__natural_number_arithmetic_grammar__derivation.pdf}
    \end{aligned}
  \end{equation*}
\end{example}

\begin{proposition}\label{thm:leftmost_derivation_existence}
  Every word in a grammar's generated language has at least one \hyperref[def:leftmost_derivation]{leftmost derivation}.
\end{proposition}
\begin{proof}
  A leftmost derivation can be achieved via reordering of the derivation steps.
\end{proof}

\begin{algorithm}[Parse tree from derivation]\label{alg:parse_tree_from_derivation}
  Fix a context-free grammar \( G = (V, \Sigma, \to, S) \) and a derivation
  \begin{equation*}
    S = w_0 \Rightarrow w_1 \Rightarrow \cdots \Rightarrow w_m = w
  \end{equation*}
  of the word
  \begin{equation*}
    w = (a_1, \ldots, a_n).
  \end{equation*}

  We will construct a parse tree recursively. At step \( i \), where \( i = 0, \ldots, m \), the leaves of the tree \( T_i \) (traversed via \hyperref[def:traversal_ordering]{pre-ordering}) should be the symbols of \( w_i \). Proceed as follows:
  \begin{thmenum}
    \thmitem{alg:parse_tree_from_derivation/initial} Let \( T_0 \) be the tree with only node \( S \).

    \thmitem{alg:parse_tree_from_derivation/step} For every \( i > 0 \), suppose that we have already built \( T_{i-1} \) for \( w_{i-1} \). Let \( A \to t_1 \cdots t_{n_i} \) be the rule producing \( w_i \) from \( w_{i-1} \).

    Since the leaves of \( T_{i-1} \) are the symbols of \( w_{i-1} \), there exists a leaf label in \( T_{i-1} \) is \( A \). For definiteness, fix the first such leaf with respect to the \hyperref[def:traversal_ordering]{pre-ordering relation}. Then \( T_{i-1} \) has the form
    \begin{equation*}
      \begin{aligned}
        \includegraphics[page=1]{output/alg__parse_tree_from_derivation.pdf}
      \end{aligned}
    \end{equation*}

    \begin{itemize}
      \item If \( n_i = 0 \), i.e. if \( w_i = \varepsilon \), let \( T_i \) be the tree
      \begin{equation*}
        \begin{aligned}
          \includegraphics[page=2]{output/alg__parse_tree_from_derivation.pdf}
        \end{aligned}
      \end{equation*}

      \item If \( n_i > 0 \), let \( T_i \) instead be the tree
      \begin{equation*}
        \begin{aligned}
          \includegraphics[page=3]{output/alg__parse_tree_from_derivation.pdf}
        \end{aligned}
      \end{equation*}
    \end{itemize}
  \end{thmenum}

  Then \( T_m \) is the desired parse tree for \( w \).
\end{algorithm}
\begin{comments}
  \item This algorithm can be found as \texttt{grammars.parse\_tree.parse\_tree\_to\_derivation} in \cite{code}.
\end{comments}

\begin{algorithm}[Leftmost derivation from parse tree]\label{alg:leftmost_derivation_from_parse_tree}
  Let \( T = (X, E, r, \leq, l) \) be a parse tree for \( w = a_1 \cdots a_n \).

  We will construct a derivation recursively. We will actually construct a sequence
  \begin{equation*}
    u_0, u_1, u_2, \ldots
  \end{equation*}
  of words over \( X \), i.e. sequences of nodes of \( T \), such that the ordering of the symbols of \( u_i \) agrees with the \hyperref[def:traversal_ordering]{pre-ordering traversal} in \( T \). Since there are only finitely many nodes in \( T \), this sequence stabilizes, and there exists some index \( m \) such that the labels of \( u_m \) are terminals and \( u_{m+i} = u_m \) for any \( i > 0 \).

  Proceed as follows:
  \begin{thmenum}
    \thmitem{alg:leftmost_derivation_from_parse_tree/initial} Define
    \begin{equation*}
      u_0 \coloneqq r
    \end{equation*}
    and
    \begin{equation*}
      w_0 \coloneqq l(r) = S.
    \end{equation*}

    \thmitem{alg:leftmost_derivation_from_parse_tree/step} For every \( i > 0 \), suppose that
    \begin{equation*}
      u_{i-1} = k_1 \cdots k_{n_{i-1}}.
    \end{equation*}

    \begin{itemize}
      \item If \( l(k_j) \) is a terminal for each \( j = 1, \ldots, n_{i-1} \), let \( u_i \coloneqq u_{i-1} \). This will mean that the \enquote{stabilizing index} \( m \) is either \( i - 1 \) or some smaller index.

      \item Otherwise, let \( j \) be the smallest index such that \( l(k_j) \) is a non-terminal, and let \( k'_1 < \cdots < k'_{n'_j} \) be the successors of \( k_j \). Then define
      \begin{equation*}
        u_i \coloneqq k_1 \cdots k_{j-1} \overline{ k'_1 \cdots k'_{n'_j} } k_{j+1} \cdots k_{n_j}.
      \end{equation*}

      Define also
      \begin{equation*}
        w_i \coloneqq l(k_1) \cdots l(k_{j-1}) \overline{ l(k'_1) \cdots l(k'_{n'_j}) } l(k_{j+1}) \cdots l(k_{n_j}).
      \end{equation*}

      \Fullref{def:parse_tree/successors} ensures there exists a rule
      \begin{equation*}
        l(k_j) \to l(k'_1) \cdots l(k'_{n'_j})
      \end{equation*}
      that allows us to transition from \( w_{i-1} \) to \( w_i \).
    \end{itemize}

    \thmitem{alg:leftmost_derivation_from_parse_tree/down} The desired derivation is
    \begin{equation*}
      S = w_0 \Rightarrow w_1 \cdots w_{m-1} \Rightarrow w_m.
    \end{equation*}
  \end{thmenum}
\end{algorithm}
\begin{comments}
  \item This algorithm can be found as \texttt{grammars.parse\_tree.derivation\_to\_parse\_tree} in \cite{code}.
\end{comments}

\begin{proposition}\label{thm:derivations_and_parse_trees}
  For every \hyperref[def:leftmost_derivation]{leftmost derivation} in a \hyperref[def:chomsky_hierarchy/context_free]{context-free grammar} there exists a unique \hyperref[def:parse_tree]{parse tree} for the same word and vice versa.
\end{proposition}
\begin{comments}
  \item When combined with \fullref{thm:leftmost_derivation_existence}, this implies that a word is derivable if and only if there exists a parse tree for it.
\end{comments}
\begin{proof}
  \Fullref{alg:parse_tree_from_derivation} allows us to construct a parse tree from any derivation. Multiple derivations can lead to the same parse tree, but our choice of non-terminal in \fullref{alg:parse_tree_from_derivation/step} ensures that a leftmost derivation leads to a unique parse tree.

  Conversely, \fullref{alg:leftmost_derivation_from_parse_tree} allows us to construct a leftmost derivation from any parse tree. Uniqueness is ensured by choosing at \fullref{alg:leftmost_derivation_from_parse_tree/step} the smallest possible index.
\end{proof}

\begin{algorithm}[Brute force parsing]\label{alg:brute_force_parsing}
  Fix a \hyperref[def:chomsky_hierarchy/context_free]{context-free grammar} \( G = (V, \Sigma, \to, S) \). For each non-terminal \( A \), we will introduce a \hyperref[def:multi_valued_function]{multi-valued function} \( P_A(w) \), whose values are parse trees of \( w \).

  To avoid unbounded recursion, we will also define the auxiliary multi-valued function \( P'_A(w, U) \), where \( U \) is a set of pairs \( (A, w) \) \enquote{already traversed} during recursion.

  Suppose that we are given the rule \( A \to v_1 \cdots v_m \), where each \( v_k \) is a single symbol (either a terminal or non-terminal). The algorithm relies on partitioning \( w \) into \( m \) parts - one for each symbol in \( v \) - by which we mean
  \begin{equation*}
    w = \underbrace{ a_1 \cdots a_{\gamma_1} }_{w_1} \underbrace{ a_{\gamma_1 + 1} \cdots a_{\gamma_1 + \gamma_2} }_{w_2} \cdots \underbrace{ a_{\gamma_1 + \cdots + \gamma_{m-1} + 1} \cdots a_{\gamma_1 + \cdots + \gamma_m} }_{w_m},
  \end{equation*}
  where \( \gamma \) is a \hyperref[def:multi_index]{multi-index} over \( 1, \ldots, m \) with \hyperref[def:multi_index]{norm} \( n \).

  Hence, we will introduce the auxiliary multi-valued function
  \begin{equation*}
    P^\dprime_A(A \to v_1 \cdots v_m, \gamma, U).
  \end{equation*}

  Finally, define
  \begin{equation*}
    P_A(w)
    \coloneqq
    P'_A(w, \varnothing)
    =
    \bigcup_{A \to v_1 \cdots v_m} \bigcup_{\norm \gamma = n} P^\dprime_A(A \to v_1 \cdots v_m, \gamma, \varnothing).
  \end{equation*}

  \begin{thmenum}
    \thmitem{alg:brute_force_parsing/init} In order to define
    \begin{equation*}
      P^\dprime_A(A \to v_1 \cdots v_m, \gamma, U),
    \end{equation*}
    we fix a rule \( A \to v_1 \cdots v_m \), a multi-index \( \gamma \) inducing the partition \( w = w_1 \cdots w_m \), and a set \( U \).

    \thmitem{alg:brute_force_parsing/tree_set} For each \( v_k \), construct a set \( \mscrT_k \) of parse trees as follows:
    \begin{itemize}
      \item If \( v_k \) is a terminal and \( w_k = v_k \), let \( \mscrT_k \) be a singleton set whose only element is the tree with only one node \( v_k \).

      \item If \( v_k \) is a non-terminal, let
      \begin{equation*}
        \mscrT_k \coloneqq \begin{cases}
          \varnothing,                          &(v_k, w_k) \in U, \\
          P'_{v_k}(w_k, U \cup \set{ (A, w) }), &\T{otherwise.}
        \end{cases}
      \end{equation*}
    \end{itemize}

    \thmitem{alg:brute_force_parsing/combine} Finally, combine the above sets \( \mscrT_1, \cdots, \mscrT_m \) as follows:
    \begin{itemize}
      \item If \( m = n = 0 \), i.e. if \( A \to \varepsilon \) and \( w = \varepsilon \), let
      \begin{equation*}
        P^\dprime_A(A \to v_1 \cdots v_m, \gamma, U)
      \end{equation*}
      consist of a single tree
      \begin{equation*}
        \begin{aligned}
          \includegraphics[page=1]{output/alg__brute_force_parsing.pdf}
        \end{aligned}
      \end{equation*}

      \item Otherwise, let
      \begin{equation*}
        P^\dprime_A(A \to v_1 \cdots v_m, \gamma, U)
      \end{equation*}
      consist of all trees of the form
      \begin{equation*}
        \begin{aligned}
          \includegraphics[page=2]{output/alg__brute_force_parsing.pdf}
        \end{aligned}
      \end{equation*}
      where \( (T_1, \ldots, T_m) \) is a tuple in the Cartesian product \( \mscrT_1 \times \cdots \times \mscrT_m \).
    \end{itemize}
  \end{thmenum}
\end{algorithm}
\begin{comments}
  \item This is a variation of Stephen Unger's algorithm described in \cite{Unger1968}.
  \item This algorithm can be found as \texttt{grammars.brute\_force\_parse.parse} in \cite{code}.
  \item We can avoid the auxiliary parameter \( U \) if the grammar has no \hyperref[def:epsilon_free_grammar]{\( \varepsilon \)-rules} and no \hyperref[alg:renaming_rule_collapse]{renaming rules}, but otherwise we can easily get stuck in a loop like \( A \Rightarrow B \Rightarrow A \) (if \( A \to B \) and \( B \to A \)) or \( A \Rightarrow AB \Rightarrow A \) (if \( B \to \varepsilon \)).
\end{comments}

\begin{definition}\label{def:grammar_ambiguity}\mcite[54]{Salomaa1987}
  We say that a word generated by a \hyperref[def:chomsky_hierarchy/context_free]{context-free grammar} is \term{ambiguous} if any of the following equivalent conditions hold:
  \begin{thmenum}
    \thmitem{def:grammar_ambiguity/tree} It has multiple \hyperref[def:parse_tree]{parse trees}.
    \thmitem{def:grammar_ambiguity/leftmost} It has multiple \hyperref[def:leftmost_derivation]{leftmost derivations}.
    \thmitem{def:grammar_ambiguity/rightmost} It has multiple \hyperref[def:leftmost_derivation]{right derivations}.
  \end{thmenum}

  We say that the grammar itself is \term{ambiguous} if it generates at least one ambiguous word, and \term{unambiguous} otherwise.
\end{definition}
\begin{comments}
  \item \Fullref{thm:derivations_and_parse_trees} ensures that every generated word has at least one parse tree.
  \item \Fullref{thm:leftmost_derivation_existence} ensures that every generated word has at least one leftmost derivation.
\end{comments}
\begin{proof}
  \EquivalenceSubProof{def:grammar_ambiguity/tree}{def:grammar_ambiguity/leftmost} Follows from \fullref{thm:derivations_and_parse_trees}.
  \EquivalenceSubProof{def:grammar_ambiguity/leftmost}{def:grammar_ambiguity/rightmost} Trivial.
\end{proof}

\begin{example}\label{ex:natural_number_arithmetic_grammar/unambiguous}
  We will show that the binary natural number grammar \eqref{eq:ex:natural_number_arithmetic_grammar/rules/shorthand} from \fullref{ex:natural_number_arithmetic_grammar/rules} is \hyperref[ex:natural_number_arithmetic_grammar/unambiguous]{unambiguous}.

  Let \( w \) be a word in \( \mscrL(G) \). We will build a derivation tree for \( w \) via \hyperref[rem:natural_number_recursion]{natural number recursion} on \( \len(w) \):

  \begin{itemize}
    \item If \( \len(w) = 1 \), then \( w = 0 \) or \( w = 1 \), and the possible parse trees are
    \begin{equation*}
      \begin{aligned}
        \includegraphics[page=1]{output/ex__natural_number_arithmetic_grammar__unambiguous.pdf}
        \qquad\qquad
        \includegraphics[page=2]{output/ex__natural_number_arithmetic_grammar__unambiguous.pdf}
      \end{aligned}
    \end{equation*}

    \item Let \( w = a_1 \cdots a_n \) be a word of length \( n \) and suppose that every word shorter than \( w \) is unambiguous.

    \begin{itemize}
      \item If \( a_1 \) is \( ( \), then \( w \) is an expression. Hence, \( a_n \) is \( ) \) and there exists some index \( k \) such that \( a_k \in \set{ +, -, \times, + } \). The inductive hypothesis applies to \( a_2 \cdots a_{k-1} \) and  \( a_{k+1} \cdots a_{n-1} \), and gives us parse trees \( T_1 \) and \( T_2 \) whose roots are labeled with \( E \). Then the following is the unique parse tree of \( w \):
      \begin{equation*}
        \begin{aligned}
          \includegraphics[page=3]{output/ex__natural_number_arithmetic_grammar__unambiguous.pdf}
        \end{aligned}
      \end{equation*}

      \item Otherwise, \( w \) is a binary numerical string. Given a parse tree of \( a_1 \cdots a_{n-2} \), since the grammar with start symbol \( N \) is right-linear, the parse tree for \( w \) requires simply adding the dotted edges
      \begin{equation*}
        \begin{aligned}
          \includegraphics[page=4]{output/ex__natural_number_arithmetic_grammar__unambiguous.pdf}
        \end{aligned}
      \end{equation*}
    \end{itemize}
  \end{itemize}
\end{example}

\begin{proposition}\label{thm:regular_grammars_are_unambiguous}
  All \hyperref[def:chomsky_hierarchy/regular]{regular grammars} are \hyperref[def:grammar_ambiguity]{unambiguous}.
\end{proposition}
\begin{proof}
  The right side of any rule contains at most one non-terminal, so there is only one possible derivation for every word.
\end{proof}

\begin{remark}\label{rem:evaluation}
  \term{Evaluation} is the process of transforming a word in some \hyperref[def:formal_grammar/language]{grammar's language} into some other object, perhaps even another word. This is easily achieved via partial definitions depending on the structure of the word.

  Fix a grammar \( G = (V, \Sigma, \to, S) \) with non-terminals \( A_1, \cdots, A_n \). Then an example evaluation function has the form
  \begin{equation*}
    F(w) \coloneqq \begin{cases}
      \cdots, &w \T{is generated from} S, \\
      \cdots, &w \T{is generated from} A_1, \\
              &\vdots,
    \end{cases}
  \end{equation*}

  \term{Pattern matching} is a term used for partial definition of \( F \) based on the structure of \( w \). Pattern matching depends on knowing the possible parse trees for \( w \). We use parse trees implicitly without explicitly mentioning it. \Fullref{ex:natural_number_arithmetic_grammar/evaluation} and \fullref{ex:evaluation_via_trees} provide justification for not using parse trees directly.

  We use the convention from \fullref{rem:parse_tree_roots} regarding parse tree for different starting nodes over the same grammar schema. Of course, more granular rules are possible based on the structure of words --- see \fullref{ex:natural_number_arithmetic_grammar/evaluation} for an example.
\end{remark}
\begin{comments}
  \item In order for \( F \) to be \hyperref[def:function]{single-valued}, the grammar must be \hyperref[def:grammar_ambiguity]{unambiguous}.
  \item We allow different starting non-terminals because, if \( F \) calls itself recursively, it must be able to process words generated in different ways.
\end{comments}

\begin{example}\label{ex:natural_number_arithmetic_grammar/evaluation}
  We will define \hyperref[rem:evaluation]{evaluation} for the binary natural number grammar \eqref{eq:ex:natural_number_arithmetic_grammar/rules/shorthand} from \fullref{ex:natural_number_arithmetic_grammar/rules}. Let
  \begin{equation*}
    E(w) \coloneqq \begin{cases}
      \Bracks{0},                    &w = 0, \\
      \Bracks{1},                    &w = 1, \\
      \Bracks{2} E(w'),              &w = w' 0, \\
      \Bracks{2} E(w') \Bracks{+} \Bracks{1}, &w = w' 1, \\
      E(u) \Bracks{+} E(v),          &w = \mathtt{(} \thinspace u + v \thinspace \mathtt{)}, \\
      E(u) \Bracks{\times} E(v),     &w = \mathtt{(} \thinspace u \times v \thinspace \mathtt{)},
    \end{cases}
  \end{equation*}
  where \( + \) and \( 1 \) denote symbols in the abstract alphabet, while \( \Bracks{+} \) and \( \Bracks{1} \) denote their obvious interpretation.

  In this example, we do pattern matching on the word \( w \). This pattern matching depends on knowing the possible parse trees and knowing that the grammar is unambiguous. We use parse trees implicitly without recognizing it.
\end{example}

\begin{example}\label{ex:evaluation_via_trees}
  If we use \hyperref[def:parse_tree]{parse trees} for \hyperref[rem:evaluation]{evaluation} instead of doing pattern matching on words, it will be more awkward for us. Consider the following example:

  Let \( v \) be a symbol, either a terminal or non-terminal, and let \( T \) be the tree
  \begin{equation*}
    \begin{aligned}
      \includegraphics[page=1]{output/ex__evaluation_via_trees.pdf}
    \end{aligned}
  \end{equation*}

  Then
  \begin{equation*}
    F(T) \coloneqq \begin{cases}
      \varepsilon,          &v = \varepsilon, \\
      v,                    &v \T{is a terminal}, \\
      F(T_1) \cdots F(T_n), &v \T{is a non-terminal}
    \end{cases}
  \end{equation*}
  produces the word \hyperref[def:parse_tree_word]{yielded} by \( T \).
\end{example}

\begin{remark}\label{rem:abstract_syntax_tree}
  \hyperref[def:parse_tree]{Parse trees} focus on how words are built from symbols. Take the following parse tree from \fullref{ex:def:parse_tree/arithmetic}:
  \begin{equation}\label{eq:rem:abstract_syntax_tree/base}
    \begin{aligned}
      \includegraphics[page=1]{output/rem__abstract_syntax_tree.pdf}
    \end{aligned}
  \end{equation}

  Most of the information in this tree is not necessary for \hyperref[rem:evaluation]{evaluating} it. An \term{abstract syntax tree} (AST) is instead a tree that contains information that is necessary for evaluation, but not for yielding the source word. The possible abstract syntax trees for a language differ depending on their use, as we will show, and are often produced from parse trees (which we called \enquote{concrete syntax trees} in \fullref{def:parse_tree}). The only restriction we put on abstract syntax trees is that there should be exactly one AST for every parse tree and vice versa.

  First, we can vastly simplify the tree by splitting the grammar \eqref{eq:ex:natural_number_arithmetic_grammar/rules/shorthand} into two parts:
  \begin{thmenum}
    \thmitem{rem:abstract_syntax_tree/lexical} The \term{lexical part}
    \begin{equation*}
      \begin{aligned}
        N &\to 0 \mid 1 \mid 1 B \\
        B &\to 0 \mid B 0 \mid 1 \mid B 1 \\
        O &\to + \mid - \mid \times \mid +
      \end{aligned}
    \end{equation*}

    Any words produced via the lexical rules are called \term{lexemes}. The goal is to isolate atoms of a language from arbitrarily complicated recursive structures. This distinction is important for evaluation and becomes apparent in more complicated use cases like first-order formula substitution in \fullref{def:first_order_substitution}.

    We rarely need parse trees for lexical rules. The lexical grammar is often \hyperref[def:chomsky_hierarchy/regular]{regular}.

    \thmitem{rem:abstract_syntax_tree/syntactic} The \term{syntactic part}
    \begin{equation*}
      E \to N \mid (E O E)
    \end{equation*}

    Syntactic rules regard lexemes as base symbols, although this is often not possible formally. In this example, \( N \) produces infinitely many words, and by definition a grammar's terminals are only finitely many.

    Compared to lexical rules, syntactic rules benefit from parse trees because they highlight the recursive structure of a word.
  \end{thmenum}

  One simplification we can do to \eqref{eq:rem:abstract_syntax_tree/base} is to collapse lexical rules and regard lexemes as words:
  \begin{equation}\label{eq:rem:abstract_syntax_tree/syntactic}
    \begin{aligned}
      \includegraphics[page=2]{output/rem__abstract_syntax_tree.pdf}
    \end{aligned}
  \end{equation}

  Another thing we can simplify is to remove auxiliary symbols like parentheses\footnote{Whitespace and comments in programming languages are also often useless in parse trees during evaluation, but they are useful for code refactoring tools, error reporting and metaprogramming.}. They are only necessary in order to the grammar to be unambiguous, but once we already have a parse tree, they become meaningless. This leads to
  \begin{equation}\label{eq:rem:abstract_syntax_tree/no_parens}
    \begin{aligned}
      \includegraphics[page=3]{output/rem__abstract_syntax_tree.pdf}
    \end{aligned}
  \end{equation}

  We can also remove unnecessarily long chains of non-terminals from \eqref{eq:rem:abstract_syntax_tree/no_parens} where that would not be ambiguous (i.e. collapse \( E \to E \to 10 \) to \( E \to 10 \)):
  \begin{equation}\label{eq:rem:abstract_syntax_tree/collapsed}
    \begin{aligned}
      \includegraphics[page=4]{output/rem__abstract_syntax_tree.pdf}
    \end{aligned}
  \end{equation}

  We can now do a simplification based on our intended semantics. Every node in \eqref{eq:rem:abstract_syntax_tree/collapsed} is labeled via either a number lexeme, operation symbol or \( E \). We know the operations are binary, and we can use the symbol of the operation instead of \( E \) as a label, removing the need to keep a separate node for the operation symbol. This leads to
  \begin{equation}\label{eq:rem:abstract_syntax_tree/final}
    \begin{aligned}
      \includegraphics[page=5]{output/rem__abstract_syntax_tree.pdf}
    \end{aligned}
  \end{equation}

  This last tree has only 5 nodes, compared to the original tree \eqref{eq:rem:abstract_syntax_tree/base} with 21 nodes. Yet, we can recover \eqref{eq:rem:abstract_syntax_tree/base} from \eqref{eq:rem:abstract_syntax_tree/final}.
\end{remark}

\begin{theorem}[Induction on syntax trees]\label{thm:induction_on_syntax_trees}
  Fix a \hyperref[def:chomsky_hierarchy/context_free]{context-free grammar} \( G = (V, \Sigma, \to, S) \). Suppose that we want to prove a statement for every word in \( \mscrL(G) \). Then it is sufficient to do the following for an arbitrary word \( w \):
  \begin{displayquote}
    For every \hyperref[def:parse_tree]{parse tree} \( T \) of \( w \), let \( T_1, \ldots, T_n \) be the immediate subtrees whose roots are labeled by \( S \).

    Suppose that the statement holds for all \( T_k \), \( k = 1, \ldots, n \), and prove the statement for \( T \).
  \end{displayquote}
\end{theorem}
\begin{comments}
  \item It is beneficial, but not strictly necessary for the grammar to be \hyperref[def:grammar_ambiguity]{unambiguous}. If it is unambiguous, we only have to consider one parse tree for every word.
  \item In practice, we skip the hairy part with building a parse tree and instead rely on \hyperref[rem:evaluation]{pattern matching}.
  \item Unlike for the other induction principles in \fullref{rem:induction}, we did not formulate this one via logical formulas. This would make the statement unnecessarily convoluted with no gains.
  \item We may just as well use \hyperref[rem:abstract_syntax_tree]{abstract syntax trees}, as long as we have an established convention on how they are represented.
\end{comments}
\begin{proof}
  This is an instance of \fullref{thm:well_founded_induction} if we order the tree nodes according to their \hyperref[def:arborescence/depth]{depth}.
\end{proof}

\begin{example}\label{ex:natural_number_arithmetic_grammar/induction}
  We will show how to use \fullref{thm:induction_on_syntax_trees} in the context of the binary natural number grammar \eqref{eq:ex:natural_number_arithmetic_grammar/rules/shorthand} from \fullref{ex:natural_number_arithmetic_grammar/rules}.

  We want to prove that an expression without ones \hyperref[rem:evaluation]{evaluates} to zero.

  \begin{itemize}
    \item The simplest expressions we have are number \hyperref[rem:abstract_syntax_tree/lexical]{lexemes}. The only one that does not contain ones is \( \mathtt{0} \), which evaluates to zero.

    \item Consider the expression \( (u + v) \).

    We implicitly assume that existence of a unique parse tree
    \begin{equation*}
      \begin{aligned}
        \includegraphics[page=1]{output/ex__natural_number_arithmetic_grammar__induction.pdf}
      \end{aligned}
    \end{equation*}
    where \( T_u \) and \( T_v \) are the (unique) parse trees for \( u \) and \( v \). It is easier for us to rely on \hyperref[rem:evaluation]{pattern matching} than to explicitly construct the tree, and have only done it for demonstrational purposes.

    Having assumed that both \( u \) and \( v \) don't contain ones and thus evaluate to zero, we conclude that their sum \( (u + v) \) should also evaluate to zero due to \ref{eq:def:peano_arithmetic/PA4}.

    \item Analogously, the expression \( (u \times v) \) evaluates to zero due to \ref{eq:def:peano_arithmetic/PA6}.
  \end{itemize}
\end{example}

\begin{remark}\label{rem:backus_normal_form}\mcite[14]{Backus1958}
  Practice requires introducing a more convenient metasyntax (a syntax for describing language syntax).

  For \hyperref[def:chomsky_hierarchy/context_free]{context-free grammars}, we can use the \term{Backus normal form} (BNF), sometimes also called \term{Backus-Naur form} (also BNF). Backus himself in \cite[14]{Backus1958} described it via examples.

  For \fullref{ex:natural_number_arithmetic_grammar/rules}, one possible BNF is
  \begin{bnf*}
    \bnfprod{expression}     {\bnfpn{number} \bnfor \bnftsq{(} \bnfsp \bnfpn{number} \bnfsp \bnfpn{operation} \bnfsp \bnfpn{number} \bnfsp \bnftsq{)}} \\
    \bnfprod{number}         {\bnftsq{\( 0 \)} \bnfor \bnftsq{\( 1 \)} \bnfor \bnftsq{\( 1 \)} \bnfsp \bnfpn{numeric suffix}} \\
    \bnfprod{numeric suffix} {\bnftsq{\( 0 \)} \bnfor \bnftsq{\( 0 \)} \bnfsp \bnfpn{numeric suffix} \bnfor \bnftsq{\( 1 \)} \bnfor \bnftsq{\( 1 \)} \bnfsp \bnfpn{numeric suffix}} \\
    \bnfprod{operation}      {\bnftsq{\( + \)} \bnfor \bnftsq{\( \times \)}}
  \end{bnf*}

  Note that we have placed the (only) \hyperref[rem:abstract_syntax_tree/syntactic]{syntactic rule} on top.

  Compared to \eqref{eq:ex:natural_number_arithmetic_grammar/rules/simple}, some differences are:
  \begin{itemize}
    \item Variables are denoted by \( \langle \)words enclosed in angle brackets\( \rangle \), so that we can name variables more descriptively using more than one symbol.
    \item Terminals are, by convention, put in \enquote{quotes}.
    \item Also by convention, the symbol \( \Coloneqq \) is used instead of \( \to \) for specifying transition rules.
    \item Different rules with the same source are combined as in \eqref{eq:ex:natural_number_arithmetic_grammar/rules/shorthand}.
    \item In order to fully describe a context-free grammar, we must only specify its Backus-Naur form and its starting variable.
  \end{itemize}
\end{remark}
\begin{comments}
  \item We will use Backus normal forms in more complicated grammars like \fullref{def:propositional_syntax/grammar_schema} and \fullref{def:first_order_syntax/grammar_schema}, while preferring more the primitive notation \eqref{eq:ex:natural_number_arithmetic_grammar/rules/simple} for generic examples relating to grammars.
\end{comments}
