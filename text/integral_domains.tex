\section{Integral domains}\label{sec:integral_domains}

\paragraph{Integral domains}

\begin{definition}\label{def:integral_domain}\mcite[def. III.1.10]{Aluffi2009Algebra}
  An \term[bg=област на цялостност (\cite[393]{Обрешков1962ВисшаАлгебра}), ru=область целостности (\cite[def. 3.5.1]{Винберг2014КурсАлгебры})]{integral domain} is an \hyperref[def:ring/trivial]{nontrivial} \hyperref[def:entire_semiring]{entire} \hyperref[def:ring/commutative]{commutative (unital) ring}.
\end{definition}

\begin{proposition}\label{thm:def:integral_domain}
  \hyperref[def:integral_domain]{Integral domains} have the following basic properties:
  \begin{thmenum}
    \thmitem{thm:def:integral_domain/subring} Any nontrivial \hyperref[def:ring/submodel]{subring} of an integral domain is also an integral domain.

    \thmitem{thm:def:integral_domain/polynomial_ring} A commutative ring \( R \) is an integral domain if and only if its \hyperref[def:polynomial_algebra]{polynomial ring} \( R[X] \) is.

    \thmitem{thm:def:integral_domain/polynomial_divides} The polynomial \( f(X) \) divides \( X^n \) if and only if \( f(X) = aX^m \), where \( a \) is invertible and \( m \leq n \).
  \end{thmenum}
\end{proposition}
\begin{proof}
  \SubProofOf{thm:def:integral_domain/subring} Trivial.

  \SubProofOf{thm:def:integral_domain/polynomial_ring}

  \NecessitySubProof* If \( R[X] \) is an integral domain, by \cref{thm:def:integral_domain/subring}, so is \( R \).

  \SufficiencySubProof* It is sufficient to prove the statement for one indeterminate. If \( f(X) \) and \( g(X) \) are nonzero polynomials, then so is \( f(X) g(X) \) by \cref{thm:polynomial_degree_arithmetic/product}.

  \SubProofOf{thm:def:integral_domain/polynomial_divides} Suppose that \( X^n = f(X) \cdot g(X) \).

  We have
  \begin{equation*}
    X^n
    =
    f(X) \cdot g(X)
    =
    \sum_{k=0}^\infty (\sum_{m+l=k} a_m b_l) X^k.
  \end{equation*}

  Then
  \begin{equation*}
    a_m b_l = \begin{cases}
      1, &m + l = n, \\
      0, &\T{otherwise}
    \end{cases}
  \end{equation*}

  Since we are working over a domain and there are no zero divisors, there exists only one pair of nonnegative integers \( m \) and \( l \) such that \( a_m b_l = 1 \). It follows that \( f(X) = a_m X^m \) and \( g(X) = a_l X^l \).
\end{proof}

\begin{proposition}\label{thm:quotient_by_prime_ideal}
  The ideal \( P \) of the \hyperref[def:ring/commutative]{commutative ring} \( R \) is \hyperref[def:semiring_ideal/prime]{prime} if and only if the \hyperref[def:ring/quotient]{quotient ring} \( R / P \) is an \hyperref[def:integral_domain]{integral domain}.
\end{proposition}
\begin{comments}
  \item See \cref{thm:quotient_by_maximal_ideal} for the corresponding statement for \hyperref[def:semiring_ideal/maximal]{maximal ideals} in possibly noncommutative rings.
\end{comments}
\begin{proof}
  \SufficiencySubProof Suppose that \( P \) is a prime ideal. Clearly \( R / P \) is a commutative ring. Since \( P \) is a proper ideal, \( R / P \) must be nontrivial. We will show that it is an \hyperref[def:entire_semiring]{entire ring}.

  Let \( [x] [y] = [0] = P \) (where \( [x] = x + P \) is the coset of \( x \) in \( R / P \)). By definition,
  \begin{equation*}
    [x] [y] = (x + P) (y + P) = (xy + P),
  \end{equation*}
  which implies \( xy + P = P \) and hence \( xy \in P \). Since \( P \) is prime, by \cref{thm:def:semiring_ideal/prime_pointwise}, we have \( x \in P \) or \( y \in P \).

  Therefore, \( [x] = [0] \) or \( [y] = [0] \). Generalizing on \( x \) and \( y \), we can conclude that \( R / P \) is entire, and thus an integral domain.

  \NecessitySubProof Suppose that \( R / P \) is an integral domain. Since \( R / P \) is nontrivial, \( P \) must be a proper ideal. We will show that it satisfies \cref{thm:def:semiring_ideal/prime_pointwise}.

  Let \( xy \in P \). We have
  \begin{equation*}
    P = [0] = [xy] = [x] [y],
  \end{equation*}
  hence \( [x] \) and \( [y] \) are zero divisors in \( R / P \). But \( R / P \) is entire, hence either \( [x] \) or \( [y] \) must be zero. That is, either \( x \in P \) or \( y \in P \).

  Generalizing on \( x \) and \( y \), we can conclude that \( P \) is a prime ideal.
\end{proof}

\paragraph{Divisibility in domains}

\begin{proposition}\label{thm:ring_entire_iff_unique_quotient}
  A \hyperref[def:ring/commutative]{commutative ring} is \hyperref[def:entire_semiring]{entire} if and only if, whenever \( y \) \hyperref[def:divisibility]{divides} \( x \), there exists a \hi{unique} element \( z \) such that \( x = yz \).
\end{proposition}
\begin{proof}
  \SufficiencySubProof Suppose that \( R \) is entire. Let \( x = yz = yz' \). \Cref{thm:def:ring/cancellable_iff_not_zero_divisor} implies that we can cancel \( y \) to obtain \( z = z' \). This demonstrates uniqueness.

  \NecessitySubProof Suppose instead that uniqueness holds and let \( 0 = yz \). By absorption, we have \( 0 = y0 \), and thus \( z = 0 \), showing that \( y \) has no nontrivial zero divisors.

  Since \( y \) was arbitrary, we conclude that the ring is entire.
\end{proof}

\begin{example}\label{ex:nonunique_divisor}
  The requirement in \cref{thm:ring_entire_iff_unique_quotient} for the ring to be entire is essential --- \cref{thm:idempotent_division} implies that (nontrivial) multiplicatively idempotent elements are nontrivial zero divisor.

  For example, if \( R \) is nontrivial, consider the \hyperref[thm:matrix_algebra]{matrix algebra} \( R^{2 \times 2} \) and the idempotent matrix
  \begin{equation*}
    \begin{pmatrix}
      1 & 0 \\
      0 & 0
    \end{pmatrix}
  \end{equation*}

  Because it is idempotent, it divides itself:
  \begin{equation*}
    \begin{pmatrix}
      1 & 0 \\
      0 & 0
    \end{pmatrix}^2
    =
    \begin{pmatrix}
      1 & 0 \\
      0 & 0
    \end{pmatrix}.
  \end{equation*}

  But another possible quotient is the identity:
  \begin{equation*}
    \begin{pmatrix}
      1 & 0 \\
      0 & 0
    \end{pmatrix}
    \begin{pmatrix}
      1 & 0 \\
      0 & 1
    \end{pmatrix}
    =
    \begin{pmatrix}
      1 & 0 \\
      0 & 1
    \end{pmatrix}
    \begin{pmatrix}
      1 & 0 \\
      0 & 0
    \end{pmatrix}
    =
    \begin{pmatrix}
      1 & 0 \\
      0 & 0
    \end{pmatrix}.
  \end{equation*}
\end{example}

\begin{definition}\label{def:domain_quotient}\mimprovised
  By \cref{thm:ring_entire_iff_unique_quotient}, in an \hyperref[def:integral_domain]{integral domain}, if \( y \) divides \( x \), then there exists a unique element \( z \) such that \( x = yz \). \Cref{thm:def:ring_localization/divisibility} motivates the notation \( x / y \).

  We call \( x / y \) the \term[ru=частное/отношение (\cite[16]{Винберг2014КурсАлгебры})]{quotient} of \( x \) by \( y \).
\end{definition}

\begin{proposition}\label{thm:domain_quotient_inverse}
  If the \hyperref[def:domain_quotient]{quotient} \( x / y \) is invertible, then \( y \) divides \( x \).
\end{proposition}
\begin{comments}
  \item Using the terminology from \cref{def:domain_divisibility/associates/direct}, \( x \) and \( y \) are thus associates.
\end{comments}
\begin{proof}
  If \( x = yz \) and \( z \) is invertible, then \( x z^{-1} = y \) and thus \( y \) is invertible.
\end{proof}

\begin{definition}\label{def:domain_divisibility}
  We will introduce several notions related to \hyperref[def:divisibility]{divisibility} in \hyperref[def:integral_domain]{integral domains}.

  \begin{thmenum}
    \thmitem{def:domain_divisibility/associates} We say that \( x \) and \( y \) are \term[bg=асоциирани (\cite[142]{ГеновМиховскиМоллов1991Алгебра}), ru=ассоциированные (\cite[118]{Винберг2014КурсАлгебры})]{associated} if any of the following conditions hold:
    \begin{thmenum}
      \thmitem{def:domain_divisibility/associates/direct}\mcite[246]{Aluffi2009Algebra} Both \( x \mid y \) and \( y \mid x \).

      \thmitem{def:domain_divisibility/associates/invertible}\mcite[393]{Knapp2016BasicAlgebra} There exists an \hyperref[def:divisibility/invertible]{invertible element} \( u \) such that \( x = uy \).

      \thmitem{def:domain_divisibility/associates/ideals}\mcite[246]{Aluffi2009Algebra} The \hyperref[def:semiring_ideal/principal]{principal ideals} \( \braket{ x } \) and \( \braket{ y } \) are equal.
    \end{thmenum}

    \thmitem{def:domain_divisibility/irreducible} We say that the nonzero non-invertible element \( x \) is \term{irreducible} if any of the following conditions hold:
    \begin{thmenum}
      \thmitem{def:domain_divisibility/irreducible/direct}\mcite[388]{Knapp2016BasicAlgebra} Whenever \( x = yz \), at least one of \( y \) or \( z \) is invertible.
      \thmitem{def:domain_divisibility/irreducible/ideals} \( \braket{ x } \) is maximal among all proper principal ideals. Maximality means that, if \( \braket{ x } \subseteq \braket{ y } \) for some nonzero non-invertible \( y \), then \( \braket{ x } = \braket{ y } \).
    \end{thmenum}

    \thmitem{def:domain_divisibility/prime} We say that the nonzero non-invertible element \( x \) is \term{prime} if any of the following equivalent conditions hold:
    \begin{thmenum}
      \thmitem{def:domain_divisibility/prime/direct}\mcite[388]{Knapp2016BasicAlgebra} If \( x \mid yz \), then \( x \mid y \) or \( x \mid z \) (or both).
      \thmitem{def:domain_divisibility/prime/ideals}\mcite[113]{Lang2002Algebra} The ideal \( \braket{ x } \) is \hyperref[def:semiring_ideal/prime]{prime}.
    \end{thmenum}

    This definition is motivated by \cref{thm:euclids_lemma}. It also applies more generally for \hyperref[def:entire_semiring]{entire semirings}.
  \end{thmenum}
\end{definition}
\begin{defproof}
  \SubProofOf{def:domain_divisibility/associates}
  \ImplicationSubProof*{def:domain_divisibility/associates/direct}{def:domain_divisibility/associates/invertible} If \( x \mid y \) and \( y \mid x \), then there exist \( a \) and \( b \) such that \( x = ay \) and \( y = bx \). Hence, \( x = abx \). Since we are working in an integral domain, we can cancel \( x \) to obtain \( ab = 1 \). Therefore, both \( a \) and \( b \) are \hyperref[def:divisibility/invertible]{units}.

  \ImplicationSubProof*{def:domain_divisibility/associates/invertible}{def:domain_divisibility/associates/ideals} Suppose that \( x = uy \) for some invertible element \( u \). If \( z \) is in \( \braket{ x } \), then \( x = uy \) divides \( z \) and hence \( y \) also divides \( z \), implying that \( \braket{ x } \subseteq \braket{ y } \). We obtain the converse inclusion by noting that \( y = u^{-1} x \).

  \ImplicationSubProof*{def:domain_divisibility/associates/ideals}{def:domain_divisibility/associates/direct} If \( \braket{ x } = \braket{ y } \), then, by \cref{thm:def:semiring_ideal/division}, \( x \mid y \) and \( y \mid x \).

  \SubProofOf{def:domain_divisibility/irreducible}
  \ImplicationSubProof*{def:domain_divisibility/irreducible/direct}{def:domain_divisibility/irreducible/ideals} Suppose that \( x \) is not invertible and that \( x = yz \) implies that \( y \) or \( z \) is invertible. Since we are working in an integral domain, \( x \) is necessarily nonzero.

  Let \( \braket{ x } \subseteq \braket{ w } \) for some non-invertible \( w \). By \cref{thm:def:semiring_ideal/division}, \( w \mid x \). Then there exists some element \( a \) such that \( x = aw \). Since \( w \) is not invertible by assumption, \( a \) must be invertible. By the equivalent definitions of associates in a domain, \( \braket{ x } = \braket{ w } \).

  \ImplicationSubProof*{def:domain_divisibility/irreducible/ideals}{def:domain_divisibility/irreducible/direct} Suppose that \( \braket{ x } \) is maximal among nonzero proper principal ideals.

  Let \( x = yz \). If, without loss of generality, \( \braket{ x } \subseteq \braket{ y } \), then \( \braket{ x } = \braket{ y } \) and, again by the equivalent conditions for associates, there exists some invertible element \( u \) such that \( x = uy \). Cancelling \( y \) in \( yu = yz \), we obtain \( u = z \). Hence, \( z \) is invertible.

  \SubProofOf{def:domain_divisibility/prime} Trivial.
\end{defproof}

\begin{remark}\label{rem:prime_and_irreducible_terminology}
  There is a certain discrepancy in the literature regarding prime and irreducible elements. It does not matter much because, in \hyperref[def:factorial_domain]{factorial domains}, an element is prime if and only if it is irreducible.

  \begin{itemize}
    \item \incite[111]{Lang2002Algebra} uses the term \enquote{irreducible} for what we call \enquote{prime}, and later notes that, in factorial domains, the principal ideal of an irreducible element is prime, and thus it makes sense to call irreducible elements prime in this context. In the Russian translation of the book, in \cite[89]{Ленг1968Алгебра}, irreducible elements are called \enquote{неприводимые элементы}, and prime elements are called \enquote{простые елементы}.

    Later Russian authors, for example \incite[def. 3.5]{Винберг2014КурсАлгебры} and \incite[30]{Шафаревич1999ОсновныеПонятияАлгебры}, call irreducible elements \enquote{простые} (which is used as a translation for \enquote{prime} elsewhere in the book, e.g. for prime ideals). This also transfers to Bulgarian books --- for example, \enquote{прост елемент} is used by \incite[def. VI.5]{ГеновМиховскиМоллов1991Алгебра} --- but \incite[413]{Обрешков1962ВисшаАлгебра} uses both \enquote{прост елемент} and \enquote{неразложим елемент} as synonyms for irreducible elements in general integral domains. \incite[125]{Тыртышников2017ОсновыАлгебры} instead defines irreducible elements as \enquote{неприводимые елементы} and avoid introducing prime elements (but uses prime ideals).

    In the context of polynomials over fields, however,
    \incite[161]{Курош1968КурсВысшейАлгебры},
    \incite[121]{Винберг2014КурсАлгебры},
    \incite[190]{Кострикин2000АлгебраЧасть1} and
    \incite[19]{Шафаревич1999ОсновныеПонятияАлгебры}
    use \enquote{неприводимый многочлен} and \incite[409]{Обрешков1962ВисшаАлгебра} and \incite[def. VI.3]{ГеновМиховскиМоллов1991Алгебра} use \enquote{неразложим полином} for what we call an irreducible polynomial, and avoid mentioning prime polynomials.

    \item On the other hand, a distinction between \enquote{prime} and \enquote{irreducible} elements in general integral domains is made in modern anglophone literature --- for example by
    \incite[343]{Jacobson1985BasicAlgebraI},
    \incite[388; 389]{Knapp2016BasicAlgebra},
    \incite[27; 29]{Eisenbud1995CommutativeAlgebra} and
    \incite[def. V.1.6]{Aluffi2009Algebra}.

    \incite[67]{Rotman2015AdvancedModernAlgebraPart1} only defines irreducible elements, mentioning \enquote{prime elements} in \cite[105]{Rotman2015AdvancedModernAlgebraPart1} as a remark.
  \end{itemize}
\end{remark}

\begin{proposition}\label{thm:def:domain_divisibility}
  The divisibility notions from \cref{def:domain_divisibility} have the following basic properties:
  \begin{thmenum}
    \thmitem{thm:def:domain_divisibility/prime_is_irreducible} Every \hyperref[def:domain_divisibility/prime]{prime element} is \hyperref[def:domain_divisibility/irreducible]{irreducible}.

    The converse is true in \hyperref[def:factorial_domain]{factorial domains}.

    \thmitem{thm:def:domain_divisibility/irreducible_in_polynomial_ring} An element of the domain \( D \) is irreducible in \( D \) if and only if it is irreducible in \( D[X] \).


    \thmitem{thm:def:domain_divisibility/irreducible_polynomial_coefficients} If a polynomial is irreducible in \( D[X] \), its nonzero non-invertible coefficients are irreducible in \( D \).

    \thmitem{thm:def:domain_divisibility/irreducible_polynomial_no_constant_term} If a polynomial over \( D \) is irreducible in \( D[X] \), its constant term is nonzero.

    \thmitem{thm:def:domain_divisibility/associates_and_isomorphisms} If \( \varphi: D \to E \) is an isomorphism, then \( x \) and \( y \) are \hyperref[def:domain_divisibility/associates]{associates} in \( D \) if and only if \( \varphi(x) \) and \( \varphi(y) \) are associates in \( E \).

    \thmitem{thm:def:domain_divisibility/primes_and_isomorphisms} If \( \varphi: D \to E \) is an isomorphism, then \( x \) is \hyperref[def:domain_divisibility/prime]{prime} (resp. \hyperref[def:domain_divisibility/irreducible]{irreducible}) in \( D \) if and only if \( \varphi(x) \) is prime (resp. irreducible) in \( E \).
  \end{thmenum}
\end{proposition}
\begin{proof}
  \SubProofOf{thm:def:domain_divisibility/prime_is_irreducible} Let \( x \) be a prime element and suppose that \( x = yz \). Then \( x \) divides \( y \) or \( z \). If, without loss of generality, \( x \) divides \( y \), then \( x \) and \( y \) are \hyperref[def:domain_divisibility/associates]{associates}, and, by the equivalence of conditions in \cref{def:domain_divisibility/associates}, \( z \) must be invertible.

  \SubProofOf{thm:def:domain_divisibility/irreducible_in_polynomial_ring}

  \SufficiencySubProof* Suppose that \( x \) is irreducible in \( D \) and let \( x = y(X) z(X) \). By \cref{thm:polynomial_degree_arithmetic/product}, both \( y(X) \) and \( z(X) \) must be constant polynomials. Therefore, they are scalars, and since \( x \) is irreducible, \( y \) or \( z \) is invertible. By \cref{thm:def:polynomial_algebra/invertible}, if \( y \) is invertible in \( D \), it is invertible in \( D[X] \).

  Generalizing on \( x \), it follows that every irreducible element in \( D \) is also irreducible in \( D[X] \).

  \NecessitySubProof* Suppose that \( x \) is irreducible in \( D[X] \) and let \( x = yz \). Then \( y \) or \( z \) is invertible in \( D[X] \), and thus again by \cref{thm:def:polynomial_algebra/invertible}, it is invertible in \( D \).

  Generalizing on \( x \), it follows that every element of \( D \) that is irreducible in \( D[X] \) is also irreducible in \( D \).

  \SubProofOf{thm:def:domain_divisibility/irreducible_polynomial_coefficients} Suppose that the polynomial
  \begin{equation*}
    f(X) = \sum_{k=0}^n a_k x^k
  \end{equation*}
  whose coefficients lie in \( D \), is irreducible in \( D[X] \).

  Suppose that the non-invertible element \( b \) divides \( a_0, a_1, \ldots, a_n \). Then \( b \) also divides \( f(X) \), and \cref{thm:def:domain_divisibility/irreducible_in_polynomial_ring} implies that \( b \) is not invertible in \( D[X] \). But this contradicts that \( f(X) \) is irreducible.

  \SubProofOf{thm:def:domain_divisibility/irreducible_polynomial_no_constant_term} Suppose that the polynomial
  \begin{equation*}
    f(X) = a_0 + \sum_{k=1}^n a_k x^k,
  \end{equation*}
  whose coefficients lie in \( D \), is irreducible in \( D[X] \).

  If the constant term \( a_0 \) is zero, then \( f(0) = 0 \), hence \( 0 \) is a root and \( X \) must divide \( f(X) \). This contradicts irreducibility; hence, \( a_0 \) must be nonzero.

  \SubProofOf{thm:def:domain_divisibility/associates_and_isomorphisms} Follows from \cref{thm:divisibility_and_isomorphisms}.

  \SubProofOf{thm:def:domain_divisibility/primes_and_isomorphisms} Follows from \cref{thm:divisibility_and_isomorphisms}.
\end{proof}

\begin{example}\label{ex:def:domain_divisibility}
  We list some examples of the divisibility notions from \cref{def:domain_divisibility}:
  \begin{thmenum}
    \thmitem{ex:def:domain_divisibility/integers} \hyperref[def:prime_number]{Prime numbers} are irreducible integers by definition. By \cref{thm:euclids_lemma}, they are also prime elements.

    The inverse \( -p \) of the prime number \( p \) is also irreducible and thus a prime element in \( \BbbZ \), but convention requires \enquote{prime numbers} to be positive.

    So, we make a distinction between \enquote{prime numbers} and \enquote{prime elements of \( \BbbZ \)}

    \thmitem{ex:def:domain_divisibility/irreducible_not_prime}\mcite[388]{Knapp2016BasicAlgebra} Consider the ring \( \BbbZ[\sqrt{-5}] \) obtained by \hyperref[def:semiring_adjunction]{adjoining} the complex number \( \sqrt{-5} \) to \( \BbbZ \). We will examine irreducible elements in this ring and show that irreducible elements are not necessarily prime.

    Based on our discussion in \cref{ex:def:divisibility/i_sqrt5}, we can conclude that \( 1 \) and \( -1 \) are the units of \( \BbbZ \).

    Let \( n \) be a positive integer strictly less than \( 5 \). Suppose that
    \begin{equation}\label{eq:ex:def:domain_divisibility/irreducible_not_prime/decomposition}
      n = \parens[\Big]{ a + b \sqrt{-5} }\parens[\Big]{ c + d \sqrt{-5}}.
    \end{equation}

    Then
    \begin{equation}\label{eq:ex:def:domain_divisibility/irreducible_not_prime/abs}
      \abs{n}^2
      =
      \abs[\Big]{a + b \sqrt{-5}}^2 \cdot \abs[\Big]{c + d \sqrt{-5}}^2
      =
      (a^2 + 5b^2) (c^2 + 5d^2)
      =
      a^2 c^2 + 5a^2 d^2 + 5b^2 c^2 + 25 b^2 d^2.
    \end{equation}

    Since \( n < 5 \), the product \( bd \) must be zero, and since \( \BbbZ \) is entire, it follows that either \( b \) or \( d \) or both must be zero. Furthermore, \( ad \) and \( bc \) must also be zero, hence \( b = d = 0 \). Then \( n = \abs{ac} \). Without loss of generality, suppose that both \( a \) and \( c \) are positive so that \( n = ac \).

    If \( n \) is \( 2 \) or \( 3 \), it is a prime number and is thus irreducible in \( \BbbZ \), hence also in \( \BbbZ[\sqrt{-5}] \).

    Now consider \( 1 \pm \sqrt{-5} \), whose (complex) absolute value is \( \sqrt 6 \). Suppose that it factors as \eqref{eq:ex:def:domain_divisibility/irreducible_not_prime/decomposition}. Then \eqref{eq:ex:def:domain_divisibility/irreducible_not_prime/abs} implies that either \( b = 0 \) or \( d = 0 \), but both cannot be zero because the condition \( a^2 c^2 = 6 \) cannot be satisfied.
    \begin{itemize}
      \item If \( b = 0 \), then
      \begin{equation*}
        6 = a^2 (c^2 + 5 d^2).
      \end{equation*}

      We have \( c^2 + 5 d^2 \geq 6 \), so \( a^2 \) must be \( 1 \).

      \item If \( d = 0 \), then similarly
      \begin{equation*}
        6 = c^2 (a^2 + 5b^2),
      \end{equation*}
      from which we conclude that \( c^2 \) must be \( 1 \).
    \end{itemize}

    It follows that \( 1 + \sqrt{5} \) and \( 1 - \sqrt{5} \) are also irreducible.

    Therefore, we have the following ways of representing \( 6 \) as a product of irreducible factors:
    \begin{equation*}
      6 = 2 \cdot 3 = \parens[\Big]{ 1 + \sqrt{-5} } \cdot \parens[\Big]{ 1 - \sqrt{-5}}.
    \end{equation*}

    Furthermore, \( 2 \) and \( 3 \) are irreducible, \( 1 + \sqrt{5} \) doesn't divide neither \( 2 \) nor \( 3 \) but divides their product. Hence, both \( 2 \) and \( 3 \) are irreducible elements that are not prime. The same holds for \( 1 + \sqrt{5} \) and \( 1 - \sqrt{5} \).
  \end{thmenum}
\end{example}

\begin{remark}\label{rem:choice_of_associates}
  If \( x \) and \( y \) are \hyperref[def:domain_divisibility/associates]{associates}, we generally have no reason to prefer \( x \) to \( y \). This leads to a non-uniqueness in certain contexts, e.g. choosing a \hyperref[def:gcd]{greatest common divisor} or, more generally, a generator for a principal ideal. In such cases, we often prefer working with ideals.

  Fortunately, in the majority of cases, we have good candidates for uniqueness:
  \begin{itemize}
    \item In the domain \( \BbbZ \) of integers, there are two units, \( 1 \) and \( -1 \). It is convenient to choose the positive greatest common divisor.

    \item In a polynomial ring over the integers \( \BbbZ \), by \cref{thm:def:polynomial_algebra/invertible}, the units are again \( 1 \) and \( -1 \), and we can choose the leading coefficient to be positive.

    \item If \( \BbbK \) is any \hyperref[def:field]{field}, any polynomial is associated with a unique \hyperref[def:monic_polynomial]{monic polynomial}.
  \end{itemize}
\end{remark}

\paragraph{Roots in domains}

\begin{lemma}\label{thm:roots_of_polynomial_product}
  In an \hyperref[def:integral_domain]{integral domain} \( D \), the roots of the product \( f(X) \cdot g(X) \) are roots of either \( f(X) \) or \( g(X) \).
\end{lemma}
\begin{comments}
  \item This lemma is very straightforward, but it highlights a crucial property of integral domains.
\end{comments}
\begin{proof}
  If \( f(\alpha) \cdot g(\alpha) = 0_D \), either \( f(\alpha) = 0_D \) or \( g(\alpha) = 0_D \).
\end{proof}

\begin{proposition}\label{thm:polynomial_degree_bounds_root_count}
  In an \hyperref[def:integral_domain]{integral domain}, the \hyperref[def:polynomial_root_multiplicity]{multiset of roots} of a (nonzero) univariate polynomial of \hyperref[def:polynomial_degree]{degree} \( n \) has total cardinality at most \( n \).
\end{proposition}
\begin{comments}
  \item In other words, a polynomial of degree \( n \) cannot have more that \( n \) roots, counting multiplicity.
\end{comments}
\begin{proof}
  We will use induction on the degree.
  \begin{itemize}
    \item A zero-degree polynomial is a nonzero constant. Such a polynomial has no roots.

    \item Suppose that all polynomials of degree \( n \) satisfy the inductive hypothesis. Let \( f(X) \) be a polynomial of degree \( n + 1 \).

    If \( f(X) \) has no roots, we have nothing to prove. Otherwise, let \( \alpha \) be a root of \( f(X) \).

    By \fullref{thm:polynomial_factor_theorem}, \( (X - \alpha) \) must divide \( f(X) \). Let \( q(X) \) be a quotient.

    \Cref{thm:polynomial_degree_arithmetic/product} implies that \( q(X) \) has degree \( n \). Let \( M \) be its multiset of roots. By the inductive hypothesis, \( M \) has total cardinality at most \( n - 1 \). After adding \( \alpha \) to \( M \), we increment its total cardinality by \( 1 \).

    Furthermore, \cref{thm:roots_of_polynomial_product} implies that \( f(X) \) cannot have other roots. So, counting multiplicity, it has at most \( n \) roots.
  \end{itemize}
\end{proof}

\begin{theorem}[Polynomial coefficient comparison principle]\label{thm:polynomial_coefficient_comparison_principle}
  In an \hyperref[def:integral_domain]{integral domain}, two univariate polynomials of degree \( n \) are equal if and only if they agree on \( n + 1 \) values for their arguments.
\end{theorem}
\begin{comments}
  \item The name of the theorem is borrowed from the Bulgarian phrase \enquote{принцип за сравняване на коефициентите при полиномите}, which is used to describe it in some books like \cite[192]{Тагамлицки1971ДиференциалноСмятане}.
\end{comments}
\begin{proof}
  \SufficiencySubProof Trivial.
  \NecessitySubProof Let \( f(X) \) and \( g(X) \) be polynomials of degree \( n \) over the domain \( D \). Consider their difference \( f(X) - g(X) \). By \cref{thm:polynomial_degree_arithmetic/sum}, this difference is either the zero polynomial or its degree is at most \( n \). In the latter case, \cref{thm:polynomial_degree_bounds_root_count} implies that \( f(X) - g(X) \) has at most \( n \) roots, counting multiplicities.

  Therefore, if we assume that \( f(X) \) and \( g(X) \) agree on \( n + 1 \) values from \( D \), these values become roots of \( f(X) - g(X) \), which is possible only if \( f(X) - g(X) \) is the zero polynomial.
\end{proof}

\begin{corollary}\label{thm:polynomial_euclidean_division_uniqueness}
  In an \hyperref[def:integral_domain]{integral domain}, the quotient and remainder from \fullref{alg:euclidean_division_of_polynomials} are unique.
\end{corollary}
\begin{proof}
  For polynomials \( f(X) \) and \( g(X) \) over the domain \( D \), where \( g(X) \) is monic, suppose that
  \begin{equation*}
    f(X) = q_1(X) \cdot g_1(X) + r_1(X) = q_2(X) \cdot g(X) + r_2(X),
  \end{equation*}
  where, for \( k = 1, 2 \), the remainder \( r_k(X) \) is either zero or has degree less than that of \( g(X) \).

  Then
  \begin{equation*}
    [r_1(X) - r_2(X)] = -[q_1(X) - q_2(X)] \cdot g(X).
  \end{equation*}

  Aiming at a contradiction, suppose that \( r_1(X) - r_2(X) \) is nonzero. \Cref{thm:def:integral_domain/polynomial_ring} implies that \( D[X] \) is also an integral domain, so the factors \( q_1(X) - q_2(X) \) and \( g(X) \) of a nonzero element \( -[r_1(X) - r_2(X)] \) of \( D[X] \) are also nonzero.

  \Cref{thm:polynomial_degree_arithmetic/product} then implies that
  \begin{equation*}
    \deg(r_1 - r_2) = \deg(q_1 - q_2) + \deg g \geq \deg g.
  \end{equation*}

  \begin{itemize}
    \item If \( r_2 \) is zero, then \( r_1 - r_2 = r_1 \) and thus \( \deg g \leq \deg r_1 \). But we have assumed that \( \deg r_1 < \deg g \), which is a contradiction.

    \item Otherwise, if \( r_1 \) is zero, we similarly derive a contradiction.

    \item If both \( r_1 \) and \( r_2 \) are nonzero, \cref{thm:polynomial_degree_arithmetic/sum} implies that
    \begin{equation*}
      \deg(r_1 - r_2) \leq \max\set{ \deg r_1, \deg r_2 } < \deg g,
    \end{equation*}
    which is again a contradiction.
  \end{itemize}

  It remains for \( r_1(X) - r_2(X) \) to be the zero polynomial. Then so is \( -[q_1(X) - q_2(X)] \cdot g(X) \). Since \( D[X] \) is an integral domain, either \( q_1(X) - q_2(X) \) or \( g(X) \) is zero. Since we have assumed that \( g(X) \) is monic, it remains for \( q_1(X) - q_2(X) \) to be zero.
\end{proof}

\begin{example}\label{ex:polynomial_root_uniqueness}
  We list some examples related to \fullref{thm:polynomial_coefficient_comparison_principle}:
  \begin{thmenum}
    \thmitem{ex:polynomial_root_uniqueness/boolean_ring} \hyperref[def:boolean_ring]{Boolean rings} provide several counterexamples.

    A Boolean ring \( R \) has idempotent multiplication, so \( x^2 = x \) for any element \( x \). Therefore, every element of \( R \) is a root of the polynomial \( f(X) \coloneqq X^2 - X \). If \( R \) has more than two elements, the comparison principle does not hold, and hence \( R \) cannot be an integral domain. In particular, if \( R \) is infinite, then \( f(X) \) has infinitely many roots.

    So, as long as \( R \) is big enough, we can easily find counterexamples --- namely, we can add \( f(X) \) to any polynomial of degree greater than \( 2 \). A simple example is the pair \( X^3 \) and \( X^3 + X^2 - X \).

    On the other end of the spectrum, if \( R \) is the \hyperref[def:finite_field]{finite field} \( \BbbF_2 = \set{ 0, 1 } \), all elements of the field at roots of \( f(X) \). Since there are only two elements, this in fact validates the comparison principle.

    The proof of \cref{thm:polynomial_degree_bounds_root_count} breaks because, although it provides us with a multiset of roots, we cannot show that there are no others.
  \end{thmenum}
\end{example}

\paragraph{Common divisors and multiples}

\begin{example}\label{ex:common_polynomial_divisors}\mcite{MathSE:polynomials_without_gcd}
  Consider two polynomials \( f(X) \) and \( g(X) \) over some domain that we wish to find the common roots of.

  The element \( \alpha \) is a common root if and only if the polynomial \( (X - \alpha) \) divides both \( f(X) \) and \( g(X) \). Thus, if \( r(X) \) is a common divisor of \( f(X) \) and \( g(X) \), every root of \( r(X) \) is a common root for \( f(X) \) and \( g(X) \).

  Let \( C \) be the set of all common divisors of \( f(X) \) and \( g(X) \). Every invertible element of the domain is itself a common divisor, so \( C \) is necessarily nonempty. Consider the \hyperref[thm:semiring_divisibility_order]{divisibility (pre)order} in \( C \). A \hyperref[def:extremal_points/greatest_and_least]{greatest element} with respect to divisibility must contain all common roots of \( f(X) \) and \( g(X) \), and thus it makes sense to search for greatest common divisors.

  \begin{thmenum}
    \thmitem{ex:common_polynomial_divisors/field} If \( f(X) \) and \( g(X) \) are polynomials over a \hyperref[def:field]{field} like \( \BbbR \), \fullref{alg:euclidean_algorithm} explicitly constructs a greatest common divisor.

    \thmitem{ex:common_polynomial_divisors/factorial} More generally, polynomials over \hyperref[def:factorial_domain]{factorial domains} always have a greatest common divisor, but the aforementioned algorithm may fail.

    \thmitem{ex:common_polynomial_divisors/distinct} As a simple concrete example, consider the polynomials \( X^5 \) and \( X^6 \) over any ring. Clearly \( X^5 \) is a common divisor and, furthermore, any polynomial dividing both \( X^5 \) and \( X^6 \) vacuously divides \( X^5 \).

    Thus, \( X^5 \) is a greatest common divisor of \( X^5 \) and \( X^6 \), but it is not unique - for any invertible ring element \( a \), the polynomial \( a X^5 \) is also a greatest common divisor.

    This general problem comes from the lack of antisymmetry in preorders --- see \cref{ex:preorder_nonuniqueness}.

    \thmitem{ex:common_polynomial_divisors/incomparable} Consider the polynomial algebra \( \BbbZ[X^2, X^3] \) ordered by divisibility. Generally there may be distinct maximal elements on a preordered set. Based on our discussion in \cref{ex:adjoining_polynomial}, we conclude that these polynomials have the form
    \begin{equation*}
      \sum_{k \neq 1} a_k X^k.
    \end{equation*}

    Because the ring features no monomial \( X \), the monomial \( X^2 \) does not divide \( X^3 \). This has some interesting consequences.

    Consider the polynomials \( X^5 \) and \( X^6 \) in this domain.

    \Cref{thm:def:integral_domain/polynomial_divides} implies that \( f(X) \) divides \( X^6 \) if and only if \( f(X) = aX^s \) for some invertible \( a \) and \( s \leq 6 \). The quotient of \( X^6 \) by \( f(X) \) is \( a^{-1} X^{6-s} \). Furthermore, neither \( s \) nor \( 6 - s \) must be \( 1 \), thus \( s \) must be among \( 0 \), \( 2 \), \( 3 \), \( 4 \) and \( 6 \).

    Similarly, we can conclude that \( f(X) = aX^s \) divides \( X^5 \) if and only if \( s \) is among \( 0 \), \( 2 \), \( 3 \) and \( 5 \).

    Hence, \( f(X) \) is a common divisor of \( X^5 \) and \( X^6 \) if and only if \( s \) is among \( 0 \), \( 2 \) and \( 3 \).

    \begin{figure}[!ht]
      \centering
      \includegraphics[page=1]{output/ex__common_polynomial_divisors__incomparable}
      \caption{A fragment of the \hyperref[def:hasse_diagram]{Hasse diagram} of the divisibility relation in \( \BbbZ[X^2, X^3] \).}
      \label{fig:ex:common_polynomial_divisors/incomparable}
    \end{figure}

    Thus, up to a choice of invertible element \( a \), the common divisors of \( X^5 \) and \( X^6 \) are \( a \) itself, \( aX^2 \) and \( aX^3 \). But \( aX^2 \) and \( aX^3 \) are not comparable because neither divides the other in \( \BbbZ[X^2, X^3] \).

    Therefore, the monomials \( X^5 \) and \( X^6 \) have two monic maximal divisors but no greatest common divisor in \( \BbbZ[X^2, X^3] \).
  \end{thmenum}
\end{example}

\begin{definition}\label{def:gcd}
  We say that, in a \hyperref[def:integral_domain]{integral domain}, \( g \) is a \term[bg=най-голям общ делител (на два полинома) (\cite[173]{Обрешков1962ВисшаАлгебра}), ru=наибольший общий делитель (\cite[def. 3.5.3]{Винберг2014КурсАлгебры}), en=greatest common divisor (\cite[def. V.2.2]{Aluffi2009Algebra})]{greatest common divisor} of \( x \) and \( y \) if it satisfies the following equivalent conditions:

  \begin{thmenum}
    \thmitem{def:gcd/direct}\mcite[2]{Knapp2016BasicAlgebra} \( g \) divides both \( x \) and \( y \) and, whenever \( d \) is also a common divisor, \( d \) divides \( g \).

    \thmitem{def:gcd/infimum} \( g \) is an \hyperref[def:extremal_points/supremum_and_infimum]{infimum} of \( x \) and \( y \) with respect to the \hyperref[thm:semiring_divisibility_order]{divisibility order}.

    \thmitem{def:gcd/ideals}\mcite[def. V.2.2]{Aluffi2009Algebra} The ideal \( \braket{ g } \) is the smallest principal ideal that contains the join \( \braket{ x } + \braket{ y } = \braket{ x, y } \) of \( \braket{ x } \) and \( \braket{ y } \).
  \end{thmenum}
\end{definition}
\begin{comments}
  \item There may be distinct greatest common divisors that divide each other --- see \cref{ex:common_polynomial_divisors/distinct} for a simple example or \cref{ex:preorder_nonuniqueness} for a discussion of this problem for general preordered sets. For the \hyperref[def:natural_numbers]{natural numbers}, where the GCD of \( x \) and \( y \) is unique, we denote it via \( \gcd(x, y) \).

  The \hyperref[def:binary_operation/associative]{associativity} property shown \Cref{thm:def:gcd/associative} even allows us to extend this notation to any (nonzero) finite amount of numbers as \( \gcd(x_1, \ldots, x_n) \).

  \item \Cref{def:gcd/ideals} can be simplified in \hyperref[def:bezout_domain]{Bezout domains}, where \( \braket{ x, y } \) is principal and must thus coincide with \( \braket{ g } \). Note that the case of least common multiples in \cref{def:lcm/ideals} always satisfies the corresponding analog to this stronger condition.

  \item Some authors like \incite[111]{Lang2002Algebra}, \incite[144]{Jacobson1985BasicAlgebraI} and \incite[2]{Knapp2016BasicAlgebra} leave greatest common divisors of \( 0 \) and \( 0 \) undefined, while others like \incite[304]{Rotman2015AdvancedModernAlgebraPart1}, \incite[def. V.2.2]{Aluffi2009Algebra}, \incite[119]{Винберг2014КурсАлгебры} and \incite[143]{ГеновМиховскиМоллов1991Алгебра} do not handle them as a special case.
\end{comments}
\begin{defproof}
  \ImplicationSubProof{def:gcd/direct}{def:gcd/infimum} Suppose that \( g \) is a common divisor of \( x \) and \( y \) and any other common divisor divides \( g \).

  Then \( g \) is a lower bound of \( x \) and \( y \) with respect to divisibility. Furthermore, if \( d \) is a common divisor, then \( d \) divides \( g \), and thus is smaller than \( g \) with respect to divisibility.

  Therefore, \( g \) is a greatest lower bound of \( x \) and \( y \).

  \ImplicationSubProof{def:gcd/infimum}{def:gcd/ideals} Suppose that \( g \) is an infimum of \( x \) and \( y \) with respect to divisibility.

  Since \( g \) divides both \( x \) and \( y \), any linear combination \( \alpha x + \beta y \) from \( \braket{ x, y } \) is a multiple of \( g \):
  \begin{equation*}
    \alpha x + \beta y
    =
    g \parens*{ \alpha \frac x g + \beta \frac y g }.
  \end{equation*}

  Hence, \( \braket{ x, y } \subseteq \braket{ g } \).

  Now let \( d \) be such that
  \begin{equation*}
    \braket{ x, y } \subseteq \braket{ d } \subseteq \braket{ g }.
  \end{equation*}

  Since \( g \) a greatest lower bound of \( x \) and \( y \) and since \( d \) divides \( g \), it follows that \( g \) divides \( d \) and thus \( \braket{ d } = \braket{ g } \).

  Therefore, \( \braket{ g } \) is the smallest principal ideal containing \( \braket{ x, y } \).

  \ImplicationSubProof{def:gcd/ideals}{def:gcd/direct} Suppose that \( \braket{ g } \) is the smallest principal ideal containing \( \braket{ x, y } \).

  Then \( \braket{ x } \subseteq \braket{ g } \) and thus \( g \mid x \), and similarly \( g \mid y \). It is thus a common divisor for \( x \) and \( y \).

  Furthermore, if \( d \) is also a common divisor, then, by the proof of the previous implication, \( \braket{ x, y } \subseteq \braket{ d } \). The minimality of \( \braket{ g } \) then ensures that \( \braket{ g } \subseteq \braket{ d } \) and, by \cref{thm:def:semiring_ideal/division}, \( d \mid g \).
\end{defproof}

\begin{proposition}\label{thm:def:gcd}
  \hyperref[def:gcd]{Greatest common divisors} have the following basic properties:
  \begin{thmenum}
    \thmitem{thm:def:gcd/divides} \( x \) divides \( y \) if and only if \( x \) is a GCD of \( x \) and \( y \).

    \thmitem{thm:def:gcd/associates} Let \( g \) be a \hyperref[def:gcd]{GCD} of \( x \) and \( y \). Then an element \( g' \) is also a greatest common divisor if and only if \( g \) and \( g' \) are \hyperref[def:domain_divisibility/associates]{associates}.

    \thmitem{thm:def:gcd/bezouts_identity_converse} If \( d \) is a common divisor for \( x \) and \( y \), and if there exist elements \( a \) and \( b \) such that \( ax + by = d \), then \( d \) is a GCD of \( x \) and \( y \).

    The premise here always holds in \hyperref[def:bezout_domain]{Bezout domains}.

    \thmitem{thm:def:gcd/power} If \( g \) is a GCD of \( x \) and \( y \), then \( g^n \) is a GCD of \( x^n \) and \( y^n \) for any positive integer \( n \).

    \thmitem{thm:def:gcd/associative} We have the following \hyperref[def:binary_operation/associative]{associativity} property:
    \begin{displayquote}
      If \( a \) is a GCD of \( x \) and \( y \) and \( b \) is a GCD of \( y \) and \( z \), then \( c \) is a GCD of \( a \) and \( z \) if and only if it is a GCD of \( x \) and \( b \).
    \end{displayquote}

    Ignoring problems of existence and uniqueness, this property can be expressed more succinctly as
    \begin{equation}\label{eq:thm:def:gcd/associative}
      \gcd(\gcd(x, y), z) = \gcd(x, \gcd(y, z)).
    \end{equation}
  \end{thmenum}
\end{proposition}
\begin{proof}
  \SubProofOf{thm:def:gcd/divides} Trivial.

  \SubProofOf{thm:def:gcd/associates} The element \( g' \) satisfies \cref{def:gcd/ideals} if and only if it has the same principal ideal as \( g \).

  \SubProofOf{thm:def:gcd/bezouts_identity_converse} Let \( e \) be a common divisor of \( x \) and \( y \). Then \( e \) divides both \( ax \) and \( by \), hence also \( ax + by = d \). Therefore, \( e \) divides \( d \), implying that \( d \) is a greatest common divisor of \( x \) and \( y \).

  \SubProofOf{thm:def:gcd/power} Let \( x = ag \) and \( y = bg \). Then \( x^n = a^n g^n \) and \( y^n = b^n g^n \). Clearly \( g^n \) is a common divisor of \( x^n \) and \( y^n \).

  Let \( d \) be a divisor of \( x^n \) and \( y^n \). Then there exist \( \alpha \) and \( \beta \) such that \( x^n = \alpha d \) and \( y^n = \beta d \).

  \SubProofOf{thm:def:gcd/associative}

  \SufficiencySubProof* Let \( c \) be a GCD of \( a \) and \( z \). Since \( a \) is a GCD of \( x \) and \( y \), due to the associativity of divisibility show in \cref{thm:semiring_divisibility_order}, it follows that \( c \) divides \( x \) and \( y \).

  Let \( d \) be a divisor of \( x \) and \( b \). Then \( d \) divides \( y \) and \( z \), and, since \( a \) is a GCD of \( x \) and \( y \), it follows that \( d \) divides \( a \). But \( c \) is a GCD of \( a \) and \( z \), thus \( d \) divides \( c \).

  Therefore, \( c \) is a GCD of \( x \) and \( b \).

  \NecessitySubProof* Analogous.
\end{proof}

\begin{definition}\label{def:lcm}
  \hyperref[thm:lattice_duality]{Dually} to \cref{def:gcd}, we say that, in a \hyperref[def:integral_domain]{integral domain}, \( l \) is a \term[bg=най-малко общо кратно (на числа) (\cite[381]{ГеновМиховскиМоллов1991Алгебра}), ru=наименьшее общее кратное (\cite[exerc. 3.6.3]{Винберг2014КурсАлгебры}), en=least common multiple (\cite[32]{Knapp2016BasicAlgebra})]{least common multiple} (LCM) of \( x \) and \( y \) if it satisfies the following equivalent conditions:

  \begin{thmenum}
    \thmitem{def:lcm/direct}\mcite[32]{Knapp2016BasicAlgebra} Both \( x \) and \( y \) divide \( l \) and, whenever \( m \) is also a common multiple, \( l \) divides \( m \).

    \thmitem{def:lcm/supremum} \( l \) is an \hyperref[def:extremal_points/supremum_and_infimum]{supremum} of \( x \) and \( y \) with respect to the \hyperref[thm:semiring_divisibility_order]{divisibility order}.

    \thmitem{def:lcm/ideals} The ideal \( \braket{ l } \) coincides with the meet \( \braket{ x } \cap \braket{ y } \) of \( \braket{ x } \) and \( \braket{ y } \).
  \end{thmenum}
\end{definition}
\begin{comments}
  \item Note how \cref{def:lcm/ideals} differs from \cref{def:gcd/ideals} --- the existence of least common multiples in general domains is indeed stronger, as we shall see in \cref{rem:gcd_but_no_lcm} and \cref{thm:gcd_and_lcm_existence}.

  \item A result similar to \cref{thm:def:gcd/associates} holds -- every pair of least common multiples are associates.

  \item As in the case of GCDs, for the \hyperref[def:natural_numbers]{natural numbers}, where the LCM of \( x \) and \( y \) is unique, we denote it via \( \lcm(x, y) \).
\end{comments}
\begin{proof}
  \ImplicationSubProof{def:lcm/direct}{def:lcm/supremum} Same as in the case of GCDs in \cref{def:gcd}.

  \ImplicationSubProof{def:lcm/supremum}{def:lcm/ideals} We can show that \( \braket{ l } \) is the largest principal ideal contained in \( \braket{ x } \cap \braket{ y } \) similar to the case of GCDs in \cref{def:gcd}.

  Now let \( z = ax = by \) be a member of the intersection. Then it is a common multiple of \( x \) and \( y \), hence \( l \) must divide \( z \).

  Generalizing on \( z \), we conclude that \( \braket{ x } \cap \braket{ y } \subseteq \braket{ l } \). Since we already have the converse inclusion, it follows that the two ideals are equal.

  \ImplicationSubProof{def:lcm/ideals}{def:lcm/direct} Same as in the case of GCDs in \cref{def:gcd}.
\end{proof}

\begin{lemma}\label{thm:common_divisor_to_multiple_lemma}
  If \( d \) is a common divisor of \( x \) and \( y \), then \( xy / d \) is a common multiple.
\end{lemma}
\begin{proof}
  Let \( d \) be a common divisor of \( x \) and \( y \). Obviously \( d \) is a divisor of \( xy \). Denote their quotient by \( m \). Then
  \begin{equation*}
    xy = \parens[\Big]{ d \cdot \frac x d } \cdot y = d \cdot \frac {xy} d = dm.
  \end{equation*}

  Furthermore,
  \begin{equation*}
    m = \frac {xy} d = x \frac y d,
  \end{equation*}
  hence \( m \) is a multiple of \( x \), and we can analogously show that it is a multiple of \( y \).
\end{proof}

\begin{lemma}\label{thm:lcm_to_gcd_lemma}
  If \( l \) be the \hyperref[def:lcm]{LCM} of \( x \) and \( y \). Then \( xy / l \) is a \hyperref[def:gcd]{GCD}.
\end{lemma}
\begin{proof}
  Since \( l \) is an LCM, it divides \( xy \). Then \( g \coloneqq xy / l \) is a divisor of \( x \) because we can cancel \( y \) in
  \begin{equation*}
    xy = dl = g \cdot \parens*{ \frac {xy} l } = g \cdot \frac x l \cdot y
  \end{equation*}
  to obtain
  \begin{equation*}
    x = g \cdot \frac x l.
  \end{equation*}

  We can similarly conclude that \( g \) is a divisor of \( y \). It remains to show that it is a GCD.

  Let \( d \) be a common divisor of \( x \) and \( y \). \Cref{thm:common_divisor_to_multiple_lemma} implies that \( m \coloneqq xy / d \) is a common multiple, and hence \( l \) divides \( m \). It follows that \( d = xy / m \) divides \( g = xy / l \), implying that \( g \) is indeed the GCD of \( x \) and \( y \).
\end{proof}

\begin{proposition}\label{thm:gcd_and_lcm}
  If \( g \) is a \hyperref[def:gcd]{GCD} of \( x \) and \( y \) and if \( l \) is an \hyperref[def:lcm]{LCM}, then \( xy \) and \( gl \) are \hyperref[def:domain_divisibility/associates]{associates}.
\end{proposition}
\begin{proof}
  Since \( l \) is an LCM, \cref{thm:lcm_to_gcd_lemma} implies that \( g' \coloneqq xy / l \) is a GCD of \( x \) and \( y \).

  \Cref{thm:def:gcd/associates} implies that \( g \) and \( g' \) are associates, thus \( g' = ga \) for some invertible \( a \). Then
  \begin{equation*}
    xy = lg' = lga,
  \end{equation*}
  hence \( xy \) and \( lg \) are associates.
\end{proof}

\begin{remark}\label{rem:gcd_but_no_lcm}
  From \cref{thm:gcd_and_lcm} it may seem that we are always able to recover a GCD from a LCM, and it is indeed so, but the existence of a LCM does not necessarily follow from the existence of a GCD. The precise existence conditions are described in \cref{thm:gcd_and_lcm_existence}.

  \incite[thm. 4]{Khurana2003GCD} provides counterexamples --- in \( \BbbZ[\sqrt{-3}] \), the elements \( x = 2 \) and \( y = 1 + \sqrt{-3} \) have a GCD, but not a LCM.

  Fortunately, if \hi{every} pair of elements has a GCD, then every pair also has a LCM.
\end{remark}

\begin{lemma}\label{thm:gcd_of_multiple}\mcite[lemma 1]{Khurana2003GCD}
  Let \( x \), \( y \) and \( r \) be nonzero elements of some integral domain. If \( g \) is a \hyperref[def:gcd]{greatest common divisor} for \( rx \) and \( ry \), then \( r \) divides \( g \) and their \hyperref[def:domain_quotient]{quotient} is a greatest common divisor of \( x \) and \( y \).
\end{lemma}
\begin{proof}
  Since \( r \) is a common divisor of \( rx \) and \( ry \), it follows that \( r \) divides \( g \). Thus, \( g = r \cdot g / r \) divides both \( rx \) and \( ry \). We can cancel \( r \) to obtain that \( g / r \) divides both \( x \) and \( y \).

  Furthermore, if \( d \) is also a common divisor for \( x \) and \( y \), then \( rd \) divides both \( rx \) and \( ry \) and hence also \( g \). Again, cancelling \( r \), we obtain that \( d \) divides \( g / r \).
\end{proof}

\begin{proposition}\label{thm:gcd_and_lcm_existence}\mcite[thm. 2]{Khurana2003GCD}
  For an arbitrary \hyperref[def:integral_domain]{integral domain}, two elements \( x \) and \( y \) have a \hyperref[def:lcm]{least common multiple} if and only if, for every nonzero \( r \), the elements \( rx \) and \( ry \) have a \hyperref[def:gcd]{greatest common divisor}.
\end{proposition}
\begin{proof}
  \SufficiencySubProof

  \SubProof*{Proof that \( x \) and \( y \) have a GCD} \Cref{thm:lcm_to_gcd_lemma} implies that, if \( l \) is an LCM of \( x \) and \( y \), then \( xy / l \) is a GCD.

  \SubProof*{Proof that \( rx \) and \( ry \) have a GCD} Since \( l \) is a LCM of \( x \) and \( y \), it is natural to suppose that \( rl \) will be an LCM of \( rx \) and \( ry \).

  It is clearly a common multiple. Furthermore, if \( m \) is also a common multiple, \( m / r \) is a common multiple for \( x \) and \( y \) and thus \( l \) divides \( m / r \). Hence, \( rl \) divides \( m / r \), making \( rl \) a least common multiple of \( rx \) and \( ry \).

  Therefore, by what we have already shown, the following is a GCD for \( rx \) and \( ry \):
  \begin{equation*}
    \frac {r^2 xy} {rl} = \frac {rxy} l.
  \end{equation*}

  \NecessitySubProof Suppose that \( rx \) and \( ry \) have a GCD for every nonzero \( r \) and let \( g \) be a GCD for \( x \) and \( y \) themselves.

  Define
  \begin{equation*}
    l \coloneqq g \cdot \frac x g \cdot \frac y g = x \cdot \frac y g = \frac x g \cdot y.
  \end{equation*}

  Thus, \( l \) is a common multiple for \( x \) and for \( y \). Let \( m \) also be a common multiple of \( x \) and \( y \). We will show that \( l \) divides \( m \).

  Let \( g' \) a greatest common divisor for \( mx \) and \( my \). The product \( xy \) is also a common divisor for \( mx \) and \( my \), hence \( xy \) divides \( g' \).

  \Cref{thm:gcd_of_multiple} implies that \( g' / m \) is a greatest common divisor for \( x \) and \( y \). \Cref{thm:def:gcd/associates} implies that \( g \) and \( g' / m \) are associates. Then \( gm \) and \( g' \) are also associates, from which it follows that \( gm \) divides \( xy = gl \). Therefore, \( m \) divides \( l \).

  Generalizing on \( m \), we conclude that \( l \) is a least common multiple.
\end{proof}
