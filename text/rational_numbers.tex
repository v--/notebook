\section{Rational numbers}\label{sec:rational_numbers}

\paragraph{Field of rational numbers}

\begin{definition}\label{def:rational_numbers}
  We define the \hyperref[def:field]{field} \( \BbbQ \) of \term[bg=рационални числа (\cite[18]{Тагамлицки1971Диф}), ru=рациональные числа (\cite[def. 22.1]{АлександровМаркушевичХинчин1951ЭнциклопедияТом1})]{rational numbers} as the \hyperref[thm:field_of_fractions]{field of fractions} of the \hyperref[def:ring]{ring} \( \BbbZ \) of \hyperref[def:integers]{integers}.
\end{definition}
\begin{comments}
  \item We will prefer representations of rational numbers \hyperref[def:lowest_terms]{in lowest terms} with positive denominators. \Fullref{def:lowest_terms} implies that such a representation is unique.

  \item We consider the following an indispensable part of \( \BbbQ \):
  \begin{itemize}
    \item The ordering defined in \fullref{def:rational_numbers_ordering}, which makes \( \BbbQ \) an \hyperref[def:ordered_semiring]{ordered (semi)ring}.
    \item The order topology defined via \fullref{def:order_topology}.
  \end{itemize}
\end{comments}

\paragraph{Ordering of integers}

\begin{definition}\label{def:rational_numbers_ordering}\mcite[lemma 5QH]{Enderton1977Sets}
  We extend the \hyperref[def:integer_ordering]{integer ordering} \( \leq_\BbbZ \) to the \hyperref[def:rational_numbers]{rational numbers} \( \BbbQ \) as follows:
  \begin{equation*}
    \frac a b \leq \frac c d \quad\T{if}\quad ad \leq_\BbbZ bc.
  \end{equation*}
\end{definition}

\begin{definition}\label{def:archimedean_field}\mcite[22]{Kelley1975Topology}
  We say that an \hyperref[def:ordered_semiring]{ordered} \hyperref[def:field]{field} is \term{Archimedean} if, for every pair of positive field elements \( x \) and \( y \), there exists a positive integer \( n \) such that \( nx > y \).
\end{definition}

\begin{proposition}\label{thm:def:rational_numbers_ordering}
  \hyperref[def:rational_numbers_ordering]{Rational number ordering} has the following basic properties:
  \begin{thmenum}
    \thmitem{thm:def:rational_numbers_ordering/inverse} We have
    \begin{equation*}
      \frac a b \leq \frac c d \T{if and only if} -\frac c d \leq -\frac a b.
    \end{equation*}

    \thmitem{thm:def:rational_numbers_ordering/total} It is a \hyperref[def:totally_ordered_set]{total order}.
    \thmitem{thm:def:rational_numbers_ordering/dense} As a totally ordered set, \( \BbbQ \) is \hyperref[def:dense_total_order]{dense-in-itself}.
    \thmitem{thm:def:rational_numbers_ordering/ordered_ring} It makes \( \BbbQ \) an \hyperref[def:ordered_semiring]{ordered (semi)ring}.
    \thmitem{thm:def:rational_numbers_ordering/archimedean} It makes \( \BbbQ \) an \hyperref[def:archimedean_field]{Archimedean field}.

    \thmitem{thm:def:rational_numbers_ordering/archimedean_exponentiation} For every pair \( a / b > 1 \) and \( c / d > 0 \), there exists some integer \( n \) such that \( (a / b)^n > c / d \).

    \thmitem{thm:def:rational_numbers_ordering/power_monotone} If \( a / b < c / d \), then, for any positive integer \( n \), we have \( (a / b)^n < (c / d)^n \).

    \thmitem{thm:def:rational_numbers_ordering/reciprocal} For positive rational numbers \( a / b \) and \( c / d \), we have
    \begin{equation}\label{eq:thm:def:rational_numbers_ordering/reciprocal}
      \frac a b < \frac c d \T{if and only if} \frac b a > \frac d c.
    \end{equation}
  \end{thmenum}
\end{proposition}
\begin{proof}
  \SubProofOf{thm:def:rational_numbers_ordering/inverse} Let \( a / b \leq c / d \). Then \( ad \leq_\BbbZ bc \). \Fullref{thm:def:integer_ordering/inverse} implies that \( -bc \leq_\BbbZ -ad \), and hence \( -c / d \leq - a / b \).

  The converse direction of the proof is identical.

  \SubProofOf{thm:def:rational_numbers_ordering/total} All necessary conditions for \( \leq \) to be a total order follow from the analogous property for integers --- \fullref{thm:def:integer_ordering/total} --- similarly to our proof in \fullref{thm:def:rational_numbers_ordering/inverse}.

  \SubProofOf{thm:def:rational_numbers_ordering/dense} Let
  \begin{equation*}
    \frac a b < \frac c d.
  \end{equation*}

  Then \( ad < bc \), hence
  \begin{equation*}
    \frac a b = \frac {2d \cdot a} {2d \cdot b} < \frac {ad + bc} {2bd} < \frac {2b \cdot c} {2b \cdot d} = \frac c d.
  \end{equation*}

  \SubProofOf{thm:def:rational_numbers_ordering/ordered_ring} We will show that \( \BbbQ \) is an ordered ring:
  \SubProofOf*[def:ordered_semigroup]{addition compatibility} If \( a / b \leq c / d \), we will prove that, for any rational number \( e / f \),
  \begin{equation}\label{eq:thm:def:rational_numbers_ordering/ordered_ring/additive_hypothesis}
    \underbrace{\frac a b + \frac e f}_{\frac {af + be} {bf}} \leq \underbrace{\frac c d  + \frac e f}_{\frac {cf + de} {df}}.
  \end{equation}

  We have \( ad \leq bc \). Because \( \BbbZ \) is itself an ordered ring by \fullref{thm:def:integer_ordering/ordered_ring}, this inequality will be preserved if we multiply it by \( f^2 \) and then add \( bdef \) so that
  \begin{equation*}
    (af + be) df = adf^2 + bdef \leq bcf^2 + bdef (cf + de) bf.
  \end{equation*}

  Thus, \eqref{eq:thm:def:rational_numbers_ordering/ordered_ring/additive_hypothesis} follows.

  \SubProofOf*[def:ordered_semiring]{multiplication compatibility} If \( a / b \leq c / d \), we will prove that, for any rational number \( e / f \),
  \begin{equation}\label{eq:thm:def:rational_numbers_ordering/ordered_ring/multiplicative_hypothesis}
    \frac a b \cdot \frac e f \leq \frac c d \cdot \frac e f.
  \end{equation}

  We have \( ad \leq bc \). Then we can multiply by \( ef \) to obtain \( ae \cdot df \leq ce \cdot bf \), which is a restatement of \eqref{eq:thm:def:rational_numbers_ordering/ordered_ring/multiplicative_hypothesis}.

  \SubProofOf{thm:def:rational_numbers_ordering/archimedean} If both \( a / b \) and \( c / d \) are positive, all \( a \), \( b \), \( c \) and \( d \) are positive integers. We are looking for a positive integer \( n \) such that
  \begin{equation*}
    n \cdot \frac a b > \frac c d,
  \end{equation*}
  that is,
  \begin{equation*}
    n \cdot ad > bc.
  \end{equation*}

  \Fullref{alg:integer_division} gives us integers \( q \) and \( r \), where \( 0 \leq r < ad \) and
  \begin{equation*}
    bc = q \cdot ad + r.
  \end{equation*}

  Since \( bc \leq q \cdot ad \), we can conclude that \( bc < (q + 1) \cdot ad \).

  \SubProofOf{thm:def:rational_numbers_ordering/archimedean_exponentiation} We will use an auxiliary statement: for any nonnegative integer \( n \), we have
  \begin{equation}\label{eq:thm:def:rational_numbers_ordering/archimedean_exponentiation/aux}
    \parens[\Big]{ \frac a b }^n > n\parens[\Big]{ \frac a b - 1 }.
  \end{equation}

  The base case for \( n = 0 \) is trivial. If we suppose that \eqref{eq:thm:def:rational_numbers_ordering/archimedean_exponentiation/aux} holds for some concrete \( n \), then
  \small
  \begin{equation*}
    \parens[\Big]{ \frac a b }^{n+1} = \parens[\Big]{ \frac a b }^n\parens[\Big]{ \frac a b - 1 } + \parens[\Big]{ \frac a b }^n > \parens[\Big]{ \frac a b }^n\parens[\Big]{ \frac a b - 1 } + n\parens[\Big]{ \frac a b - 1 } > \parens[\Big]{ \frac a b - 1 } + n\parens[\Big]{ \frac a b - 1 } = (n + 1)\parens[\Big]{ \frac a b - 1 }.
  \end{equation*}
  \normalsize

  Having that in mind, \fullref{thm:def:rational_numbers_ordering/archimedean} implies that, for some positive integer \( n \), we have
  \begin{equation*}
    n\parens[\Big]{\frac a b - 1} > \frac c d.
  \end{equation*}

  Then \eqref{eq:thm:def:rational_numbers_ordering/archimedean_exponentiation/aux} implies that \( (a / b)^n > c / d \).

  \SubProofOf{thm:def:rational_numbers_ordering/power_monotone} Suppose that \( a / b < c / d \). We will show by induction on positive integers \( n \) that \( (a / b)^n < (c / d)^n \).

  The case \( n = 1 \) is trivial. If \( (a / b)^n < (c / d)^n \) holds, then we can multiply both sides by \( c / d \) so that
  \begin{equation}\label{eq:thm:def:rational_numbers_ordering/power_monotone/right}
    \frac c d \cdot \parens[\Big]{ \frac a b }^n < \parens[\Big]{ \frac c d }^{n+1}.
  \end{equation}

  We can multiply \( a / b < c / d \) by \( (a / b)^n \):
  and by \( a / b \) so that
  \begin{equation}\label{eq:thm:def:rational_numbers_ordering/power_monotone/left}
    \parens[\Big]{ \frac a b }^{n + 1} < \frac c d \cdot \parens[\Big]{ \frac a b }^n.
  \end{equation}

  The result follows by combining \eqref{eq:thm:def:rational_numbers_ordering/power_monotone/right} and \eqref{eq:thm:def:rational_numbers_ordering/power_monotone/left}.

  \SubProofOf{thm:def:rational_numbers_ordering/reciprocal} Both directions have identical proofs, so we will only prove that \( a / b < c / d \) implies \( b / a > d / c \).

  This is done simply by multiplying both sides by \( {bd} / {ac} \):
  \begin{equation*}
    \frac d c = \frac {a \cdot bd} {b \cdot ac} < \frac {c \cdot bd} {d \cdot ac} = \frac b a.
  \end{equation*}
\end{proof}

\begin{proposition}\label{thm:rational_number_signum}\mimprovised
  For the \hyperref[def:totally_ordered_ring_signum]{signum} function of \( \BbbQ \), we have
  \begin{equation*}
    \sgn_\BbbQ\parens*{ \frac a b } = \frac {\sgn_\BbbZ(a)} {\sgn_\BbbZ(b)}.
  \end{equation*}
\end{proposition}
\begin{proof}
  Simple case-by-case verification.
\end{proof}

\paragraph{Dedekind incompleteness}

\begin{proposition}\label{thm:nth_root_is_not_rational}
  For any integer \( n \geq 2 \), there exists no \hyperref[def:rational_numbers]{rational number} whose \( n \)-th power is a \hyperref[def:prime_number]{prime number}.
\end{proposition}
\begin{comments}
  \item \Fullref{thm:nth_root_is_irrational} establishes that this fails for \hyperref[def:real_numbers]{real numbers}.
\end{comments}
\begin{proof}
  Fix a prime number \( p \). Aiming at a contradiction, suppose that there exists a rational number \( a / b \) whose \( n \)-th power is \( p \). Let \( c / d \) be an irreducible fraction (in which \( c \) and \( d \) are coprime) corresponding to \( a / b \).

  We have
  \begin{equation*}
    c^n = p d^n.
  \end{equation*}

  Since \( p \) is prime, \fullref{thm:euclids_lemma} implies that \( p \) divides \( c \). Let \( c' \) be the quotient of \( c \) by \( p \). Then
  \begin{equation*}
    p^n c'^n = p d^n.
  \end{equation*}

  We see that \( p^{n-1} \) divides \( d^n \) and, again \fullref{thm:euclids_lemma}, \( p \) divides \( d \).

  Then \( p \) is a common factor of both \( c \) and \( d \). But we have assumed that \( c \) and \( d \) are coprime.

  The obtained contradiction shows that there is no rational number whose \( n \)-th power is \( p \).
\end{proof}

\begin{corollary}\label{thm:rational_numbers_not_dedekind_complete}
  The \hyperref[def:rational_numbers]{rational numbers} are not Dedekind complete.
\end{corollary}
\begin{proof}
  Fix a prime number \( p \) and consider the \hyperref[def:dedekind_cut]{Dedekind cut} \( (A, B) \), where
  \begin{align*}
    A \coloneqq \set*{ \frac a b \given* \parens[\Big]{ \frac a b }^2 \leq p }
    &&
    B \coloneqq \set*{ \frac a b \given* \parens[\Big]{ \frac a b }^2 \geq p }.
  \end{align*}

  \Fullref{thm:nth_root_is_not_rational} implies that neither \( A \) has a minimum nor \( B \) has a maximum.

  Therefore, the rational numbers are not Dedekind complete.
\end{proof}
