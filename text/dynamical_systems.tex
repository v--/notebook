\chapter{Dynamical systems}\label{ch:dynamical_systems}

Dynamical systems model phenomena evolving with time. We dedicate much of \fullref{sec:abstract_dynamical_systems} to the question about what a dynamical system is, and then used the established formalisms to study objects which can, in some favorable cases, be considered dynamical systems:
\begin{itemize}
  \item \Fullref{sec:cellular_automata}.
  \item \Fullref{sec:recurrence_relations}.
  \item \Fullref{sec:difference_equations}.
\end{itemize}

The concept of a dynamical system is very general, and there are little commonalities between their different kinds. The commonalities are mostly terminological, and even thus, as mentioned in remarks over the chapter, terminology often differs too. For example, as mentioned in \cref{rem:dynamical_system_trajectory_terminology}, the \hyperref[def:dynamical_system_trajectory]{trajectories} of an autonomous differential equation encode local information, while the trajectories of a cellular automaton encode global information and are thus not called trajectories.

An obvious relationship exists between difference equations and (ordinary) differential equations, however their behavior and properties differ by much, as well as the tools used to study them.
