\section{Prime numbers}\label{sec:prime_numbers}

\paragraph{Prime numbers}

\begin{definition}\label{def:prime_number}\mcite[2]{Apostol1976AnalyticNumberTheory}
  A \term[ru=простое число (\cite[45]{Зорич2019АнализЧасть1})]{prime number} is an integer greater than \( 1 \) whose only \hi{positive} proper \hyperref[def:divisibility]{divisors} is \( 1 \). Non-prime integers greater than \( 1 \) are called \term{composite numbers}.
\end{definition}

\begin{remark}\label{rem:prime_numbers_and_prime_elements}
  The definition of a prime number given in \fullref{def:prime_number} is standard, however it seems inconsistent with \fullref{sec:integral_domains}.

  First, \fullref{sec:integral_domains} actually defines \hyperref[def:domain_divisibility/irreducible]{irreducible elements} rather than \hyperref[def:domain_divisibility/prime]{prime elements} of the domain \( \BbbZ \). Second, if \( p \) is a prime number, \( -p \) is also a prime number.

  Fortunately, prime and irreducible elements coincide in \hyperref[def:gcd_domain]{GCD domains} due to \fullref{thm:def:gcd_domain/irreducible_is_prime} and \fullref{thm:def:gcd_domain/irreducible_is_prime}. Unfortunately, calling negative prime elements of \( \BbbZ \) \enquote{prime numbers} is not accepted.

  Coprime integers are, fortunately, defined as in general GCD domains via \fullref{def:coprime_elements}.
\end{remark}

\begin{lemma}[Euclid's lemma]\label{thm:euclids_lemma}
  If \( p \) is a \hyperref[def:prime_number]{prime number}, then \( p \mid nm \) implies \( p \mid n \) or \( p \mid m \).
\end{lemma}
\begin{proof}
  Since \( \BbbZ \) is a GCD domain, the lemma follows from \fullref{thm:def:gcd_domain/irreducible_is_prime}.
\end{proof}

\begin{theorem}[Euclid's prime infinitude theorem]\label{thm:euclids_prime_infinitude_theorem}\mcite[62]{Кострикин2000АлгебраЧасть1}
  There exist infinitely many \hyperref[def:prime_number]{prime number}.
\end{theorem}
\begin{comments}
  \item Thus, we can recursively construct a sequence of all prime numbers.
\end{comments}
\begin{proof}
  Suppose that \( p_1, \ldots, p_n \) is an enumeration of all prime numbers and consider the number
  \begin{equation*}
    m \coloneqq p_1 \cdots p_n + 1.
  \end{equation*}

  \begin{itemize}
    \item If \( m \) is prime, then there exists a prime not in the list \( p_1, \ldots, p_n \), contradicting our assumption.

    \item If \( m \) is not prime, then one of the primes, say \( p_k \), divides \( m \). Let \( q \) be their quotient.

    Then
    \begin{equation*}
      1 = m - (m - 1) = p_k \parens[\Big]{ q - \prod_{i \neq k}^n p_i }.
    \end{equation*}

    It follows that \( p_k \) is invertible in \( \BbbZ \), which is a contradiction.
  \end{itemize}

  Obtaining a contradiction in both cases, we conclude that there is no finite list of prime numbers.
\end{proof}

\paragraph{Integer parity}

This paragraph is here because we prove it via \fullref{thm:euclids_lemma}.

\begin{definition}\label{def:integer_parity}\mimprovised
  We associate with each \hyperref[def:integers]{integer} its \term{parity} --- we say that the integer is \term{even} if it is \hyperref[def:divisibility]{divisible} by \( 2 \) and \term{odd} otherwise.
\end{definition}

\begin{proposition}\label{thm:integer_product_parity}
  The product of two integers is \hyperref[def:integer_parity]{odd} if and only if both are odd.
\end{proposition}
\begin{proof}
  Since \( 2 \) is prime, \fullref{thm:euclids_lemma} implies that \( 2 \) divides \( nm \) if and only if \( 2 \) divides \( n \) or \( m \).
\end{proof}

\begin{corollary}\label{thm:integer_parity_grading}
  \hyperref[def:integer_parity]{Integer parity} induces a \( \BbbF_2 \)-\hyperref[def:graded_ring]{ring grading} on \( \BbbZ \) when assigning even elements to \( 0 \).
\end{corollary}
\begin{proof}
  This is simply a restatement of \fullref{thm:integer_product_parity}.
\end{proof}

\begin{lemma}\label{thm:integer_successor_parity}
  The \hyperref[def:integers]{integer} \( n \) is \hyperref[def:integer_parity]{even} (resp. odd) if and only if \( n + 1 \) is odd (resp. even).
\end{lemma}
\begin{comments}
\end{comments}
\begin{proof}
  \SufficiencySubProof
  \SubProof*{Proof if \( n \) is even} Let \( n = 2m \) for some integer \( m \).

  Suppose also that \( n + 1 \) is also even; i.e. \( n + 1 = 2k \) for some integer \( k \).

  Then \( 2m + 1 = 2k \), hence \( 2(m - k) = 1 \), implying that \( 2 \) divides \( 1 \). But \fullref{thm:integer_multiplicative_invertibility} implies that \( 2 \) is not a unit.

  Then \( n + 1 \) cannot be even, i.e. it is odd.

  \SubProof*{Proof if \( n \) is odd} Suppose that \( 2 \) does not divide \( n \).

  \Fullref{thm:integers_archimedean} implies that there exists a positive integer multiple of \( 2 \) exceeding \( n \); let \( m \) b
  e the smallest such multiple. Then \( 2m > n \) and \( 2(m - 1) < n \) (we exclude the case \( 2(m - 1) = n \) since \( n \) is odd).

  Then \( 2m - 2 < n < 2m \), and \fullref{thm:ordered_ring/additive_inversion} implies that \( -2m < -n < -2m + 2 \). By adding \( 2m \), we obtain \( 0 < 2m - n < 2 \).

  Therefore, \( 2m - n = 1 \); that is, \( n + 1 = 2m \) is even.

  \NecessitySubProof
  \SubProof*{Proof if \( n \) is odd} We have already shown that \( n + 1 \) is even whenever \( n \) is odd. If \( n + 1 \) is even, so is \( n - 1 = n + 1 - 2 \).

  \SubProof*{Proof if \( n \) is even} If we suppose that both \( n \) and \( n - 1 \) are even, we would obtain a contradiction with what we have already shown --- that \( n = (n - 1) + 1 \) must be odd.
\end{proof}

\begin{proposition}\label{thm:integer_sum_parity}
  The sum of two integers is \hyperref[def:integer_parity]{odd} if and only if exactly one of them is odd.
\end{proposition}
\begin{proof}
  \SufficiencySubProof Fix integers \( n \) and \( m \). We consider distinct cases.

  \begin{itemize}
    \item If \( n \) and \( m \) are both even, then \( n = 2k \) and \( m = 2p \) for some integers, and \( n + m = 2(k + p) \); that is, their sum is also even.

    \item If \( n \) and \( m \) are both odd, \( n + 1 \) and \( m - 1 \), both of which are even by \fullref{thm:integer_successor_parity}. Then their sum \( (n + 1) + (m - 1) = n + m \) is also even.

    \item If \( n \) is odd and \( m \) is even, by \fullref{thm:integer_successor_parity}, \( n - 1 \) is even, and hence so is the sum \( (n - 1) + m \). Again by \fullref{thm:integer_successor_parity}, the sum \( (n - 1) + m + 1 = n + m \) is odd.

    \item If \( n \) is even and \( m \) is odd, we proceed analogously.
  \end{itemize}

  \NecessitySubProof In the proof of sufficiency, we have covered all possible cases.
\end{proof}

\paragraph{Valuations}

\incite*[255]{Aluffi2009Algebra} remarks that \enquote{Entire libraries have been written on the subject of valuations}. We provide \fullref{def:exponential_valuation} and otherwise try to avoid the term as much as possible.

\begin{definition}\label{def:non_archimedean_absolute_value}\mcite[561]{Jacobson1989BasicAlgebraII}
  If an \hyperref[def:absolute_value]{absolute value} function \( \norm{\anon}: \BbbK \to \BbbR \) satisfies
  \begin{equation}\label{eq:def:non_archimedean_absolute_value}
    \norm{x + y} \leq \max\set{ \norm{x}, \norm{y} }.
  \end{equation}
  we say that it is a \term{non-Archimedean}.
\end{definition}
\begin{comments}
  \item \incite[465]{Lang2002Algebra} calls such functions \enquote{valuations}, while \cite[def. 9.4']{Jacobson1989BasicAlgebraII} generalizes them to Abelian groups and calls them \enquote{additive valuations}.
\end{comments}

\begin{definition}\label{def:exponential_valuation}
  An \term[ru=нормирование (\cite[437]{Винберг2014КурсАлгебры}), en=exponential valuation (\cite[def. 9.4']{Jacobson1989BasicAlgebraII})]{exponential valuation} on an \hyperref[def:integral_domain]{integral domain} \( D \) is a map \( v: D \setminus \set{ 0 } \to \BbbZ_{>0} \) that satisfies the following properties:
  \begin{thmenum}
    \thmitem{def:exponential_valuation/homomorphism} \( v(x \cdot y) = v(x) + v(y) \).
    \thmitem{def:exponential_valuation/min} \( v(x + y) \geq \min\set{ v(x), v(y) } \).
  \end{thmenum}
\end{definition}
\begin{comments}
  \item Our definition is based the following:
  \begin{itemize}
    \item \incite[def. 9.4']{Jacobson1989BasicAlgebraII}, where Jacobson defines an \enquote{exponential valuation} on fields with values in an abelian group having an additional value adjoined, \( \infty \), such that \( v(x) = \infty \) if and only if \( x = 0 \).

    \item \incite[437]{Винберг2014КурсАлгебры}, where Vinberg defines an \enquote{нормирование} (\enquote{norming}) on integral domains as surjective functions with positive integer values.
  \end{itemize}
\end{comments}

\begin{proposition}\label{thm:exponential_and_additive_valuations}
  Fix a \hyperref[def:field]{field} \( \BbbK \) and two functions, \( \norm{\anon}: \BbbK \to \BbbR \) and \( v: \BbbK \setminus \set{ 0 } \to \BbbR \). Suppose that there exists a constant \( 0 < c < 1 \) such that, for nonzero \( x \) in \( \BbbK \), we have
  \begin{equation}\label{eq:thm:exponential_and_additive_valuations}
    \norm{x} = c^{v(x)}
  \end{equation}

  Then \( \norm{\anon} \) is a \hyperref[def:non_archimedean_absolute_value]{non-Archimedean} \hyperref[def:absolute_value]{absolute value} if and only if \( v \) is an \hyperref[def:exponential_valuation]{exponential valuation}.
\end{proposition}
\begin{proof}
  Follows from our definition of the \hyperref[def:logarithm]{natural logarithm} as a (selection of the) inverse of the \hyperref[def:exponential_function]{exponential function}.
\end{proof}

\begin{definition}\label{def:padic_valuation}\mcite[437]{Винберг2014КурсАлгебры}
  For every \hyperref[def:domain_divisibility/prime]{prime element} \( p \) in an \hyperref[def:integral_domain]{integral domain} \( D \), we define the \( p \)-adic \hyperref[def:exponential_valuation]{exponential valuation} on \( D \) as
  \begin{equation}\label{eq:def:padic_valuation}
    v_p(x) \coloneqq \max\set{ k = 0, 1, 2, \ldots \given p^k \mid x }.
  \end{equation}
\end{definition}
\begin{comments}
  \item We define \( v_p \) only for nonzero elements of \( D \).
\end{comments}
\begin{defproof}
  We must verify that \( v_p \) really is an exponential valuation. If \( p^k \) divides \( x \) and \( p^m \) divides \( y \), then \( p^k p^m \) divides \( xy \), thus
  \begin{equation*}
    v_p(x \cdot y)
    =
    p^k p^m
    \reloset {\eqref{eq:thm:semigroup_exponentiation_properties/distributive}} =
    p^{k + m}
    =
    v_p(x) + v_p(y).
  \end{equation*}

  This verifies \fullref{def:exponential_valuation/homomorphism}.

  Furthermore, if \( p^k \) divides \( x \) and \( p^m \) divides \( y \), then \( p^{\min\set{ k, m }} \) divides \( x + y \), although it is possible that a larger power of \( p \) divides \( x + y \) (e.g. \( 2^3 \div 3 + 5 \)). Then
  \begin{equation*}
    v_p(x + y)
    \geq
    \min\set{ k, m }
    =
    \min\set{ v_p(x), v_p(y) }.
  \end{equation*}

  This verifies \fullref{def:exponential_valuation/min} and completes the proof.
\end{defproof}

\begin{theorem}[Fundamental theorem of arithmetic]\label{thm:fundamental_theorem_of_arithmetic}
  We can use the \hyperref[def:padic_valuation]{\( p \)-adic valuations} to decompose a positive integer \( n \) as follows:
  \begin{equation}\label{eq:thm:fundamental_theorem_of_arithmetic}
    n = \prod_{p \T{is prime}} p^{v_p(n)}.
  \end{equation}

  Furthermore, only finitely many of the valuations have positive values.
\end{theorem}
\begin{comments}
  \item Consequentially, every integer greater than \( 1 \) can be \hyperref[def:irreducible_factorization]{factored} uniquely into a product of prime powers.

  \item The valuations with positive values give us an \hyperref[def:irreducible_factorization]{irreducible factorization} of \( n \) that is unique. As discussed in \fullref{rem:integer_domain_chain}, \( \BbbZ \) is a factorial domain, so existence and uniqueness up to an equivalence of factorizations is automatic. Using prime numbers, however, allows stating a more powerful uniqueness result.

  \item The factorization can also unique when regarded as a \hyperref[def:multiset]{multiset} of prime numbers.

  \item As in \cite[18]{Apostol1976AnalyticNumberTheory}, we will sometimes find it convenient to use the increasing sequence of all prime numbers.
\end{comments}
\begin{proof}
  Let \( u p_1 \cdots p_r \) be an irreducible factorization of \( n \). All irreducible factors are either prime numbers or additive inverses of prime numbers, thus, without loss of generality, we can assume that \( u = 1 \) and that all irreducible factors are prime numbers.

  Furthermore, every irreducible factorization of \( n \) is simply a permutation of this one.

  Then \( v_p(n) \) is simply the number of occurrences of \( p \) among \( p_1, \ldots, p_r \).

  This concludes the proof.
\end{proof}

\begin{corollary}\label{thm:integer_divisibility_via_primes}
  For positive integers \( n \) and \( m \), \( n \) divides \( m \) if and only if for every \hyperref[def:padic_valuation]{\( p \)-adic valuation} we have \( v_p(n) \leq v_p(m) \).
\end{corollary}
\begin{proof}
  \SufficiencySubProof Suppose that \( n \) divides \( m \). Suppose that, for some prime number \( p \), we have \( v_p(n) > v_p(m) \). Then \( p^{v_p(n)} \) divides \( n \) but not \( m \), which contradicts the transitivity of divisibility shown in \fullref{thm:semiring_divisibility_order}.

  Then \( v_p(n) \leq v_p(m) \) for every prime number \( p \).

  \NecessitySubProof Suppose that \( v_p(n) \leq v_p(m) \) for every prime number \( p \).

  \Fullref{thm:fundamental_theorem_of_arithmetic} implies that \( v_p(n) \) is nonzero for only finitely many primes, say \( p_1, \ldots, p_r \). Then
  \begin{equation}\label{eq:thm:integer_divisibility_via_primes/proof/decomposition}
    n = p_1^{v_{p_1}(n)} \cdots p_r^{v_{p_r}(n)}.
  \end{equation}

  Via induction on \( r \) it follows that
  \begin{equation*}
    n = p_1^{v_{p_1}(n)} \cdots p_{r-1}^{v_{p_r}(n)}
  \end{equation*}
  divides
  \begin{equation*}
    p_1^{v_{p_1}(m)} \cdots p_{r-1}^{v_{p_r}(m)}.
  \end{equation*}

  The latter divides \( m \), hence \( n \) divides \( m \).
\end{proof}

\begin{corollary}\label{thm:integer_gcd_lcm_via_primes}
  For positive integers \( n \) and \( m \), we have
  \begin{equation*}
    \gcd(n, m) = \prod_{p \T{is prime}} p^{\min\set{ v_p(n), v_p(m) }}
  \end{equation*}
  and
  \begin{equation*}
    \lcm(n, m) = \prod_{p \T{is prime}} p^{\max\set{ v_p(n), v_p(m) }}.
  \end{equation*}
\end{corollary}
\begin{proof}
  Follows from \fullref{thm:integer_divisibility_via_primes}.
\end{proof}

\paragraph{Prime counting}

\begin{proposition}\label{thm:prime_number_sqrt}
  An integer \( n > 1 \) is \hyperref[def:prime_number]{prime} if and only if no integer \( k \) with \( 1 < k \leq \op{isqrt}(n) \) divides \( n \).
\end{proposition}
\begin{comments}
  \item A consequence of this is that a brute-force primality check, in which we check whether \( k \) divides \( n \) for all \( k < n \), the condition on \( k \) can be simplified to \( k \leq \op{isqrt}(n) \).
\end{comments}
\begin{proof}
  \SufficiencySubProof Suppose that \( n \) is prime. Since \( n > 1 \), then \( \op{isqrt}(n) < n \), and thus \( 1 < k \leq \op{isqrt}(n) \) implies \( k < n \). Then, since \( n \) is prime, no such \( k \) can divide \( n \) since its only positive divisors are \( 1 \) and \( n \) itself.

  \NecessitySubProof Suppose that, for some \( n > 1 \), from \( 1 < k \leq \op{isqrt}(n) \) it follows that \( k \) does not divide \( n \).

  Let \( n = ab \). \Fullref{thm:int_sqrt_leq} implies that either \( a \leq \op{isqrt}(n) \), in which case \( a = 1 \) and \( b = n \), or \( b \leq \op{isqrt}(n) \), in which case \( b = 1 \) and \( a = n \).

  Therefore, we conclude that \( n \) is prime.
\end{proof}

\begin{definition}\label{def:int_sqrt}\mimprovised
  Based on \cite{PythonDocs:3.13:math}, we will introduce an \term{integer square root} \hyperref[def:arithmetic_function]{arithmetic function}
  \begin{equation*}
    \op{isqrt}(n) \coloneqq \min\set{ k = 1, 2, \ldots \given k^2 \leq n }.
  \end{equation*}

  If \( n = \op{isqrt}(n)^2 \), we call \( n \) a \term[en=perfect square (\cite[example 1.7.1]{Rosen2019DiscreteMathematics})]{perfect square}.
\end{definition}
\begin{comments}
  \item In terms of the real number \( n \)-th root function, which we will discuss in \fullref{def:principal_nonnegative_nth_root}, we can define the integer square root as
  \begin{equation*}
    \op{isqrt}(n) \coloneqq \floor\parens[\Big]{ \sqrt{n} }.
  \end{equation*}
\end{comments}

\begin{proposition}\label{thm:int_sqrt_leq}
  If for some positive integers we have \( n = ab \), we have the following possibilities:
  \begin{thmenum}
    \thmitem{thm:int_sqrt_leq/equal} \( a = \op{isqrt}(n) = b \).
    \thmitem{thm:int_sqrt_leq/a_leq} \( a \leq \op{isqrt}(n) < b \).
    \thmitem{thm:int_sqrt_leq/b_leq} \( a > \op{isqrt}(n) \geq b \).
  \end{thmenum}
\end{proposition}
\begin{proof}
  We have the following options in general for \( n = ab \):
  \begin{itemize}
    \item If \( a = b \), then \( n \) is a perfect square, hence \fullref{thm:int_sqrt_leq/equal} holds.
    \item If \( a < b \), then we must investigate how they are related to \( \op{isqrt}(n) \).

    If we suppose that \( b \leq \op{isqrt}(n) \), then \( ab < b^2 \leq \op{isqrt}(n)^2 \leq n \), which is a contradiction.

    Similarly, if we suppose that \( a > \op{isqrt}(n) \), then both \( a \geq \op{isqrt}(n) + 1 \), hence
    \begin{equation*}
      (\op{isqrt}(n) + 1)^2 < ab = n,
    \end{equation*}
    which contradicts the maximality of \( \op{isqrt}(n) \).

    This leads to \fullref{thm:int_sqrt_leq/a_leq}.

    \item If \( a > b \), we can argue by \fullref{thm:preorder_duality} to conclude that \fullref{thm:int_sqrt_leq/b_leq} holds.
  \end{itemize}
\end{proof}

\begin{corollary}\label{thm:prime_number_sqrt_prime}
  An integer \( n > 1 \) is \hyperref[def:prime_number]{prime} if and only if no \hi{prime} \( p \) with \( 1 < p \leq \op{isqrt}(n) \) divides \( n \).
\end{corollary}
\begin{proof}
  Follows from \fullref{thm:prime_number_sqrt} by noticing that \( k \) divides \( n \) whenever the prime factors of \( p \) do.
\end{proof}

\begin{algorithm}[Sieve of Eratosthenes]\label{alg:sieve_of_eratosthenes}\mcite[exerc. 4.5.4.8]{Knuth1997ArtVol2}
  We can construct approximations to the sequence
  \begin{equation*}
    s_k \coloneqq \begin{cases}
      1, &k \T{is prime}, \\
      0, &\T{otherwise}.
    \end{cases}
  \end{equation*}

  Via recursion on \( n \), we will build the sequence \( \seq{ s_{n,k} }_{k=1}^\infty \) such that, for \( 1 \leq k \leq n^2 \), \( s_{n,k} \) is \( 1 \) if \( k \) is prime and \( 0 \) otherwise.

  \begin{thmenum}
    \thmitem{alg:sieve_of_eratosthenes/one} We start with the sequence
    \begin{equation*}
      s_{1,k} \coloneqq \begin{cases}
        0, &k = 1, \\
        1, &\T{otherwise.}
      \end{cases}
    \end{equation*}

    \thmitem{alg:sieve_of_eratosthenes/recursive} For \( n > 1 \), let \( s_{n,k} = s_{n-1,k} \) if \( s_{n-1,n} = 0 \) and otherwise let
    \begin{equation*}
      s_{n,k} \coloneqq \begin{cases}
        0,         &k > n \T{and} n \T{divides} k \\
        s_{n-1,k}, &\T{otherwise.}
      \end{cases}
    \end{equation*}
  \end{thmenum}
\end{algorithm}
\begin{comments}
  \item This algorithm can be found as \identifier{arithmetic.primes.build_erathostenes_sieve} in \cite{notebook:code}.
\end{comments}
\begin{proof}
  Correctness follows from \fullref{thm:prime_number_sqrt_prime}.
\end{proof}

\begin{definition}\label{def:arithmetic_function}\mcite[24]{Apostol1976AnalyticNumberTheory}
  We use the term \term[en=arithmetical function (\cite[24]{Apostol1976AnalyticNumberTheory})]{arithmetic function} to refer to complex-valued functions defined on the positive integers.
\end{definition}

\begin{definition}\label{def:prime_counting_function}\mcite[8]{Apostol1976AnalyticNumberTheory}
  Denote by \( \pi(n) \) the \hyperref[def:arithmetic_function]{arithmetic function} giving the number of primes less than or equal to \( n \).
\end{definition}

\begin{proposition}\label{thm:prime_counting_sieve}
  We can use \fullref{alg:sieve_of_eratosthenes} to compute the \hyperref[def:prime_counting_function]{prime counting function}:
  \begin{equation*}
    \pi(n) \coloneqq \sum_{k=1}^n s_{\op{isqrt}(n),k}
  \end{equation*}
\end{proposition}
\begin{comments}
  \item We can use \fullref{thm:inclusion_exclusion_principle} to compute \( \pi(n) \) --- see \fullref{ex:thm:inclusion_exclusion_eratosthenes}.
\end{comments}
\begin{proof}
  Trivial.
\end{proof}

\begin{proposition}\label{thm:inclusion_exclusion_eratosthenes}\mcite[\S 8.6.3]{Rosen2019DiscreteMathematics}
  Denote by \( p_1, p_2, \ldots \) the sequence of all \hyperref[def:prime_number]{prime numbers}. Then the \hyperref[def:prime_counting_function]{prime counting function} can be expressed as
  \begin{equation}\label{eq:thm:inclusion_exclusion_eratosthenes}
    \pi(n) = (n - 1) + \sum_{m=1}^{\mathclap{\pi(\op{isqrt}(n))}} (-1)^m \sum_{\mathclap{i_1 < \cdots < i_m}} \floor\parens[\Big]{ \frac n {p_{i_1} \cdots p_{i_m}} } + \pi(\op{isqrt}(n)).
  \end{equation}
\end{proposition}
\begin{proof}
  We will adapt the reasoning of \fullref{alg:sieve_of_eratosthenes} so that it can be used with \fullref{thm:inclusion_exclusion_principle}.

  Fix a positive integer \( n \), let \( N \coloneqq \set{ 1, 2, \ldots, n } \) and, for a positive integer \( k \), let \( A_k \coloneqq k\BbbZ \cap N \) be the subset of multiples of \( k \). Finally, let \( P \) be the set of prime numbers less than or equal to \( \op{isqrt}(n) \).

  \Fullref{thm:prime_number_sqrt_prime} implies that \( m \in N \setminus \set{ 1 } \) is composite if and only if there exists some prime \( p \leq \op{isqrt}(m) \) such that \( m \in A_p \). We can extend \( p \) to range up to \( \op{isqrt}(n) \), however we must then exclude the possibility that \( m \) is itself prime, in which case \( m \in A_m \) without \( m \) being composite.

  Thus, the composite numbers in \( N \) are
  \begin{equation*}
    C \coloneqq \parens[\Big]{ \parens[\Big]{ \bigcup_{p \in P} A_p } \setminus P }.
  \end{equation*}

  Then
  \begin{equation*}
    \pi(n) = \card(N \setminus (C \cup \set{ 1 })).
  \end{equation*}

  Due to additivity of \( \card \),
  \begin{equation}\label{eq:thm:inclusion_exclusion_eratosthenes/sets}
    \pi(n) = \underbrace{\card N}_n - 1 - \card\parens[\Big]{ \bigcup_{p \in P} A_p } + \underbrace{\card P}_{\pi(\op{isqrt}(n))}.
  \end{equation}

  It remains to apply \fullref{thm:inclusion_exclusion_principle}. \Fullref{thm:natural_number_divisibility_lattice/join} implies that
  \begin{equation*}
    A_{k_1} \cap \cdots \cap A_{k_m} = A_{\lcm(k_1, \ldots,k_m)},
  \end{equation*}
  which for prime numbers becomes
  \begin{equation*}
    A_{p_1} \cap \cdots \cap A_{p_m} = A_{p_1 \cdots p_m}.
  \end{equation*}

  Finally, note that
  \begin{equation*}
    \card(A_k) = \floor\parens[\Big]{ \frac n k }.
  \end{equation*}

  Then
  \begin{equation*}
    \card\parens[\Big]{ \bigcup_{p \in P} A_p } = \sum_{m=1}^{\mathclap{\pi(\op{isqrt}(n))}} (-1)^m \sum_{\mathclap{i_1 < \cdots < i_m}} \floor\parens[\Big]{ \frac n {p_{i_1} \cdots p_{i_m}} }
  \end{equation*}
  and, after substituting in \eqref{eq:thm:inclusion_exclusion_eratosthenes/sets}, we obtain \eqref{eq:thm:inclusion_exclusion_eratosthenes}.
\end{proof}

\begin{example}\label{ex:thm:inclusion_exclusion_eratosthenes}.
  As discussed in \fullref{thm:inclusion_exclusion_eratosthenes}, we can count the number of primes less than or equal to \( n \) using only our knowledge of much smaller primes.

  For example, let \( n = 25 \). The primes less than or equal to \( \op{isqrt}(n) = 5 \) are \( 2 \), \( 3 \) and \( 5 \), and thus
  \begin{align*}
    \pi(25) &= 25 - 1 - \parens[\Big]{ \floor\parens[\Big]{ \frac 2 {25} } - \floor\parens[\Big]{ \frac 3 {25} } - \floor\parens[\Big]{ \frac 5 {25} } } + \\
            &+ \parens[\Big]{ \floor\parens[\Big]{ \frac 6 {25} } - \floor\parens[\Big]{ \frac {10} {25} } - \floor\parens[\Big]{ \frac {15} {25} } } + \\
            &- \floor\parens[\Big]{ \frac {30} {25} } - 3
            = \\ &= 25 - 1 - (12 + 8 + 5) + (4 + 2 + 1) - 0 + 3
            = \\ &= 9
  \end{align*}

  Indeed,
  \begin{equation*}
    \begin{array}{r *{23}{c}}
      1  & \hi{2} & \hi{3} & 4       & \hi{5} & 6       & \hi{7} & 8  & 9  & 10      & \hi{11} & 12 & \hi{13} \\
      14 & 15     & 16     & \hi{17} & 18     & \hi{19} & 20     & 21 & 22 & \hi{23} & 24      & 25 &
    \end{array}
  \end{equation*}
\end{example}

\begin{theorem}[Matiyasevich-Stechkin sieve]\label{thm:matiyasevich_stechkin_sieve}\mcite[61]{БарановСтечкин2004ЭкстремальнаяКомбинаторика}
  Consider the \hyperref[def:parabola]{parabola} \( y = x^2 \) with positive integer coordinates. For any two integers \( a > 1 \) and \( b > 1 \), consider the \hyperref[def:line_segment]{line segment} connecting the points \( (-a, a^2) \) and \( (b, b^2) \) of the parabola.

  Then \( k > 1 \) is a \hyperref[def:prime_number]{composite number} if and only if some segment passes through the point \( (0, k) \) on the ordinate.

  \begin{figure}[!ht]
    \centering
    \includegraphics[page=1]{output/thm__matiyasevich_stechkin_sieve}
    \caption{An illustration of the \hyperref[thm:matiyasevich_stechkin_sieve]{Matiyasevich-Stechkin sieve}.}\label{fig:thm:matiyasevich_stechkin_sieve}
  \end{figure}
\end{theorem}
\begin{proof}
  Fix positive integers \( a \) and \( b \). We are interested in the point in which the segment between \( (-a, a^2) \) and \( (b, b^2) \) passes though the ordinate. We must thus find the real number \( t \in [0, 1] \) such that
  \begin{equation*}
    tb + (1 - t)(-a) = 0.
  \end{equation*}

  This reduces to
  \begin{equation*}
    t(a + b) = a,
  \end{equation*}
  hence \( t = a / (a + b) \).

  Then
  \begin{equation*}
    tb^2 + (1 - t)a^2
    =
    \frac {ab^2} {a + b} + \frac {ba^2} {a + b}
    =
    ab \frac {b + a} {a + b}.
  \end{equation*}

  Hence, the intersection point is \( (0, ab) \).

  We conclude that the ordinate point \( k \) intersects the segment corresponding to \( a \) and \( b \) if and only if \( k = ab \). Furthermore, \( k > 1 \) is composite if both \( a \) and \( b \) are greater than \( 1 \).
\end{proof}

\paragraph{Coprime numbers}

\begin{definition}\label{def:coprime_numbers}
  We say that the integers \( n \) and \( m \) are \term[ru=взаимно простые (числа) (\cite[45]{Зорич2019АнализЧасть1}), en=coprime / relatively prime (\cite[231]{RosenEtAl2018DiscreteMathematicsHandbook})]{coprime} if any of the following equivalent conditions hold:
  \begin{thmenum}
    \thmitem{def:coprime_numbers/abstract}\mimprovised They are \hyperref[def:coprime_elements]{coprime elements} of \( \BbbZ \), that is, they satisfy any of the equivalent definitions in \fullref{def:coprime_elements}.

    \thmitem{def:coprime_numbers/concrete}\mcite[45]{Зорич2019АнализЧасть1} Their only \hi{positive} common divisor is \( 1 \).

    \thmitem{def:coprime_numbers/prime}\mimprovised No \hyperref[def:prime_number]{prime number} divides both \( n \) and \( m \).
  \end{thmenum}
\end{definition}
\begin{defproof}
  \ImplicationSubProof{def:coprime_numbers/abstract}{def:coprime_numbers/concrete} If \( n \) and \( m \) are coprime elements of \( \BbbZ \), they satisfy \fullref{def:coprime_elements/greatest}, thus every common divisor is invertible. The only invertible positive element of \( \BbbZ \) is \( 1 \), thus the only positive common divisor of \( n \) and \( m \) is \( 1 \).

  \ImplicationSubProof{def:coprime_numbers/concrete}{def:coprime_numbers/prime} Suppose that the only positive common divisor of \( n \) and \( m \) is \( 1 \). We have purposely excluded \( 1 \) from the list of prime numbers, thus no prime number divides both \( n \) and \( m \).

  \ImplicationSubProof{def:coprime_numbers/prime}{def:coprime_numbers/abstract} Suppose that no prime number divides both \( n \) and \( m \).

  Suppose that \( d \) is a common divisor of \( n \) and \( m \).

  \begin{itemize}
    \item If \( d > 0 \), \fullref{thm:fundamental_theorem_of_arithmetic} implies that \( d \) has a prime factorization; thus, if \( d > 1 \), then some prime number divides both \( n \) and \( m \), which is a contradiction. So \( d = 1 \) is the only positive common divisor.

    \item If \( d < 0 \), then \( -d \) is a positive common divisor, and we conclude that \( d = -1 \).
  \end{itemize}

  Therefore, the only common divisors of \( n \) and \( m \) are invertible elements of \( \BbbZ \). Then \( n \) and \( m \) satisfy \fullref{def:coprime_elements/divisors}.
\end{defproof}

\begin{proposition}\label{thm:n_plus_1_coprime}
  Given any integer greater \( n \) than \( 1 \), the numbers \( n \) and \( n + 1 \) are \hyperref[def:coprime_elements]{coprime}.
\end{proposition}
\begin{proof}
  Let \( m \) be a common divisor of \( n \) and \( n + 1 \). Then there exists some integer \( k \) such that \( n = km \) and some integer \( l \) such that \( n + 1 = lm \). Then
  \begin{equation*}
    n + 1 = lm = km + 1,
  \end{equation*}
  thus \( m \) divides \( 1 \).

  Therefore, \( m = 1 \).
\end{proof}

\begin{proposition}\label{thm:primality_via_coprimality}
  The (positive) integer \( n \) is \hyperref[def:prime_number]{prime} if and only if \( n \) is \hyperref[def:coprime_numbers]{coprime} with every \( m < n \).
\end{proposition}
\begin{proof}
  \SufficiencySubProof Trivial.
  \NecessitySubProof If \( m \mid n \), then \( \gcd(m, n) = 1 \), hence \( m = 1 \).
\end{proof}

\paragraph{Euler's totient theorem}

\begin{definition}\label{def:eulers_totient_function}\mcite[25]{Apostol1976AnalyticNumberTheory}
  Denote by \( \varphi(n) \) the \hyperref[def:arithmetic_function]{arithmetic function} giving the number of positive integers smaller than or equal to \( n \) that are \hyperref[def:coprime_elements]{coprime} to \( n \). We call \( \varphi \) \term{Euler's totient function}.
\end{definition}
\begin{comments}
  \item In particular, we have \( \varphi(1) = 1 \). It is for this reason that we count integers smaller than \hi{or equal to} \( n \).
  \item \Fullref{thm:totient_function_product_formula} provides a useful expression in terms of prime factors.
\end{comments}

\begin{proposition}\label{thm:division_by_three_digits}
  A positive integer is divisible by \( 3 \) if and only if the sum of its \hyperref[def:positional_number_system/decimal]{decimal digits} is divisible by \( 3 \).
\end{proposition}
\begin{comments}
  \item We will discuss decimal \hyperref[def:integer_radix_expansion]{integer expansions} in \fullref{sec:positional_number_systems}. For the purpose of this proposition, a decimal expansion of the positive integer \( n \) is a polynomial \( p(X) \) such that \( p(10) = n \).
\end{comments}
\begin{proof}
  For any positive integer \( k \), \fullref{thm:xn_minus_yn_factorization} implies that
  \begin{equation*}
    10^k - 1 = 9 \cdot \sum_{i=0}^{k-1} 10^i,
  \end{equation*}
  hence
  \begin{equation*}
    10^k = 1 \mod 3.
  \end{equation*}

  Given the decimal expansion
  \begin{equation*}
    n = \sum_{k=0}^m a_0 10^k
  \end{equation*}
  we thus have
  \begin{equation*}
    n = \sum_{k=0}^m a_0 10^k = \sum_{k=0}^m a_0 \mod 3.
  \end{equation*}
\end{proof}

\begin{proposition}\label{thm:def:eulers_totient_function}
  \hyperref[def:eulers_totient_function]{Euler's totient function} \( \varphi \) has the following basic properties:
  \begin{thmenum}
    \thmitem{thm:def:eulers_totient_function/one} \( \varphi(1) = 0 \).
    \thmitem{thm:def:eulers_totient_function/prime} \( p \) is \hyperref[def:prime_number]{prime} if and only if \( \varphi(p) = p - 1 \).
    \thmitem{thm:def:eulers_totient_function/zn} The \hyperref[def:semiring]{multiplicative group} \( \BbbZ_n^\times \) of the ring \hyperref[def:ring_of_integers_modulo]{\( \BbbZ_n \)} of integers modulo \( n > 1 \) has order \( \varphi(n) \).
  \end{thmenum}
\end{proposition}
\begin{proof}
  \SubProofOf{thm:def:eulers_totient_function/one} There are no positive integers smaller than \( 1 \).

  \SubProofOf{thm:def:eulers_totient_function/prime}

  \SufficiencySubProof* Every positive integer smaller than \( p \) is coprime to \( p \), and there are exactly \( p - 1 \) positive integers smaller than \( p \) --- \( 1, 2, \ldots, p - 1 \).

  \NecessitySubProof* If \( \varphi(p) = p - 1 \), the \( p \) is coprime to every positive integer smaller than \( p \). \Fullref{thm:primality_via_coprimality} implies that \( p \) is prime.

  \SubProofOf{thm:def:eulers_totient_function/zn} Follows from \fullref{thm:multiplicative_group_of_integers_modulo}.
\end{proof}

\begin{theorem}[Totient divisor sum]\label{thm:totient_divisor_sum}
  For any positive integer \( n \), we have
  \begin{equation}\label{eq:thm:totient_divisor_sum}
    n = \sum_{d \mid n} \varphi(d),
  \end{equation}
  where \( \varphi \) \hyperref[def:eulers_totient_function]{Euler's totient function}.
\end{theorem}
\begin{proof}
  Consider the \hyperref[def:cyclic_group]{cyclic group} \( C_n \).

  Every element \( x \) of \( C_n \) generates a cyclic group. \Fullref{thm:cyclic_subgroup_classification/finite} implies that all elements of order \( d \) generate the same subgroup, \( H_d \), and \fullref{thm:def:group_element_order/group_order} implies that \( d \) divides \( n \).

  Denote by \( G_d \) the set of generators of \( H_d \), i.e. the set of elements of order \( d \). Then we have the partition
  \begin{equation*}
    C_n = \bigcup \set{ G_d \given d \mid n },
  \end{equation*}
  and thus the cardinality \( n \) of \( C_n \) is the sum of the cardinalities of \( G_d \).

  \Fullref{thm:def:cyclic_group/generators_cardinality} implies that \( H_d \) has \( \varphi(d) \) generators, hence \eqref{eq:thm:totient_divisor_sum} follows.
\end{proof}

\begin{theorem}[Totient function product formula]\label{thm:totient_function_product_formula}
  For every positive integer \( n \), \hyperref[def:eulers_totient_function]{Euler's totient function} can be expressed as
  \begin{equation}\label{eq:thm:totient_function_product_formula}
    \varphi(n) = n \cdot \prod_{v_p(n) > 0} \parens[\Big]{ 1 - \frac 1 p }.
  \end{equation}
\end{theorem}
\begin{comments}
  \item Since we want to cover the case \( n = 1 \), where \( \varphi(1) = 1 \), we must allow \( r = 0 \) and agree that the empty product has value \( 1 \).
\end{comments}
\begin{proof}
  It will be simpler for us to work with an prime factorization \( n = p_1^{a_1} \cdots p_r^{a_r} \).

  Let \( N \) and \( A_k \) be defined as in \fullref{thm:inclusion_exclusion_eratosthenes}. Then
  \begin{equation*}
    \varphi(n) = \card N - \card\parens[\Big]{ \bigcup_{k=1}^r A_{p_k} }.
  \end{equation*}

  \Fullref{thm:inclusion_exclusion_principle} implies that
  \begin{equation}\label{eq:thm:totient_function_product_formula/intermediate}
    \varphi(n) = n + \sum_{m=1}^r (-1)^m \sum_{i_1 < \cdots < i_m} \frac n {p_{i_1} \cdots p_{i_m}}.
  \end{equation}

  We now use induction on \( r \).
  \begin{itemize}
    \item If \( r = 0 \), since we have agreed on \( 1 \) as the product of the empty set, \( \varphi(1) = 1 \).
    \item If \( r = 1 \), then
    \begin{equation*}
      \varphi(n) = n + (-1) \frac n {p_1} = n \parens[\Big]{ 1 - \frac 1 {p_1} }.
    \end{equation*}

    \item Suppose that \eqref{eq:thm:totient_function_product_formula} holds for all positive integers whose prime factorization has \( r \) terms. Let \( n = p_1^{a_1} \cdots p_r^{a_r} p_{r+1}^{a_{r+1}} \). Then
    \begin{align*}
      \varphi(n)
      &\reloset{\eqref{eq:thm:totient_function_product_formula/intermediate}} =
      n + \sum_{m=1}^{r+1} (-1)^m \sum_{i_1 < \cdots < i_m} \frac n {p_{i_1} \cdots p_{i_m}}
      = \\ &=
      n + \sum_{m=1}^r \underbrace{(-1)^m} \sum_{i_1 < \cdots < i_m \leq r} \frac n {p_{i_1} \cdots p_{i_m}} + \\ &\quad\quad + \frac n {p_{r+1}} + \sum_{m=1}^r \underbrace{(-1)^{m+1}} \sum_{i_1 < \cdots < i_m \leq r} \frac n {p_{i_1} \cdots p_{i_m}} \cdot \frac 1 {p_{r+1}}
      = \\ &=
      \parens[\Big]{ 1 - \frac 1 {p_{r+1}} } \cdot \parens[\Big]{ n + \sum_{m=1}^r (-1)^m \sum_{i_1 < \cdots < i_m \leq r} \frac n {p_{i_1} \cdots p_{i_m}} }
      \reloset {\T{ind.}} = \\ &=
      \parens[\Big]{ 1 - \frac 1 {p_{r+1}} } \cdot n \cdot \prod_{k=1}^r \parens[\Big]{ 1 - \frac 1 {p_k} }.
    \end{align*}
  \end{itemize}
\end{proof}

\begin{example}\label{ex:thm:totient_function_product_formula} We can use \fullref{thm:totient_function_product_formula} to compute the totient function of a number if we know its prime factors. For example,
  \begin{equation*}
    \varphi(25) = 25 \cdot \parens[\Big]{ 1 - \frac 1 5 } = \frac {25 \cdot 4} 5 = 20.
  \end{equation*}
\end{example}

\begin{definition}\label{def:multiplicative_arithmetic_function}\mcite[33]{Apostol1976AnalyticNumberTheory}
  If an \hyperref[def:arithmetic_function]{arithmetic function} \( f \) satisfies
  \begin{equation}\label{eq:def:multiplicative_arithmetic_function}
    f(nm) = f(n) \cdot f(m)
  \end{equation}
  whenever \( n \) and \( m \) are \hyperref[def:coprime_numbers]{coprime}, we say that \( f \) is \term[ru=мультипликативная (функция) (\cite[def. 36]{Бухштаб1966ТеорияЧисел})]{multiplicative}.

  If \eqref{eq:def:multiplicative_arithmetic_function} is satisfied for all positive integers \( n \) and \( m \), we say that \( f \) is \term[ru=вполне мультипликативная (функция) (\cite[def. 37]{Бухштаб1966ТеорияЧисел})]{completely multiplicative}.
\end{definition}
\begin{comments}
  \item There is a clash in terminology. \Fullref{def:multiplicative_function} generalizes completely multiplicative arithmetic functions, but refers to them simply as \enquote{multiplicative}. Thus, a function may be multiplicative in the sense of \fullref{def:multiplicative_arithmetic_function} but not \fullref{def:multiplicative_function}.

  To resolve this issue, we use \enquote{multiplicative on coprimes} and \enquote{completely multiplicative} whenever confusion is possible.
\end{comments}

\begin{proposition}\label{thm:totient_function_multiplicative}
  \hyperref[def:eulers_totient_function]{Euler's totient function} is \hyperref[def:multiplicative_arithmetic_function]{multiplicative on coprimes}.
\end{proposition}
\begin{proof}
  Follows from \fullref{thm:totient_function_product_formula}.
\end{proof}

\begin{theorem}[Euler's totient theorem]\label{thm:eulers_totient_theorem}
  For positive coprime integers \( n \) and \( x \), we have
  \begin{equation*}
    x^{\varphi(n)} \cong 1 \pmod n,
  \end{equation*}
  where \( \varphi \) is \hyperref[def:eulers_totient_function]{Euler's totient function}.
\end{theorem}
\begin{proof}
  This is vacuous for \( n = 1 \) since all integers are equal modulo \( 1 \).

  Suppose that \( n > 1 \). First, use \fullref{alg:integer_division} to obtain integers \( q \) and \( y < n \) such that
  \begin{equation*}
    x = nq + y.
  \end{equation*}

  Since \( x \) is, by assumption, coprime with \( n \), then \( y \) is also coprime with \( n \). Indeed, every common divisor \( d \) of \( y \) and \( n \) is also a common divisor \( x \), and the largest such possible value is \( \gcd(n, x) = 1 \).

  Now consider the \hyperref[def:semiring]{multiplicative group} \( \BbbZ_n^\times \) of the ring \hyperref[def:ring_of_integers_modulo]{\( \BbbZ_n \)} of integers modulo \( n \) and the \hyperref[def:cyclic_group]{cyclic subgroup} \( \set{ 1, y, y^2, \ldots } \) (modulo \( n \)). It is necessarily finite as a subgroup of \( \BbbZ_n^\times \). Furthermore, by \fullref{thm:lagranges_subgroup_theorem}, its order \( k \) divides the order of \( \BbbZ_n^\times \). By \fullref{thm:def:eulers_totient_function/zn}, the order of \( \BbbZ_n^\times \) is \( \varphi(n) \).

  We have \( y^k \cong 1 \pmod n \) since \( k \) is the order of a cyclic group. If \( \varphi(n) = km \), then
  \begin{equation*}
    y^{\varphi(n)}
    =
    y^{km}
    \reloset {\eqref{eq:thm:semigroup_exponentiation_properties/repeated}} =
    (y^k)^m
    \cong
    1^m
    \pmod n.
  \end{equation*}
\end{proof}

\begin{corollary}\label{thm:division_modulo}
  Given positive integers \( n \) and \( m \), we can apply \fullref{alg:integer_division} to obtain \( n = q \cdot \varphi(m) + r \), where \( \varphi \) is \hyperref[def:eulers_totient_function]{Euler's totient function}.

  Then, for a positive integer \( x \) coprime to \( m \), we have
  \begin{equation*}
    x^n \cong x^r \pmod m,
  \end{equation*}
\end{corollary}
\begin{proof}
  By \fullref{thm:eulers_totient_theorem}, \( x^{\varphi(m)} \cong 1 \pmod m \). Then
  \begin{equation*}
    x^n = (x^{\varphi(m)})^q x^r \cong x^r \pmod m.
  \end{equation*}
\end{proof}

\begin{example}\label{ex:thm:division_modulo}
  The integers \( 9 \) and \( 10 \) are coprime. We have \( \varphi(10) = 4 \) and \( 1000 = 250 \cdot 4 \). By \fullref{thm:division_modulo},
  \begin{equation*}
    9^{1000} \cong 9^0 = 1 \pmod {10}.
  \end{equation*}

  We can thus sometimes vastly simplify finding the last digit of the \hyperref[def:positional_number_system/decimal]{decimal expansion} of \( 9^{1000} \).
\end{example}

\begin{theorem}[Fermat's little theorem]\label{thm:fermats_little_theorem}
  For a \hyperref[def:prime_number]{prime number} \( p \) and for any positive integer \( x \), we have
  \begin{equation*}
    x^p \cong x \pmod p.
  \end{equation*}
\end{theorem}
\begin{proof}
  If \( p \mid x \), then both \( x^p \) and \( x \) and congruent to \( 0 \) modulo \( p \).

  Otherwise, by \fullref{thm:eulers_totient_theorem}, we have \( x^{\varphi(p) + 1} \cong x \pmod p \), and by \fullref{thm:def:eulers_totient_function/prime}, we have \( \varphi(p) + 1 = p \).
\end{proof}
