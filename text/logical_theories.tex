\subsection{Logical theories}\label{subsec:logical_theories}

As in \fullref{subsec:first_order_models}, we will only be interested in (sets of) closed first-order formulas.

\begin{definition}\label{def:first_order_theory}\mcite[def. 1.4.5]{Hinman2005}
  The \term{closure} of the set \( \Gamma \) of closed formulas in the \hyperref[def:first_order_syntax]{first-order language} \( \mscrL \) is the set
  \begin{equation*}
    \cl(\Gamma) \coloneqq \set{ \varphi \in \boldop{Form} \given \Gamma \vDash \varphi }.
  \end{equation*}

  If necessary, we may distinguish between the syntactic closure \( \cl^\vdash(\Gamma) \) and the semantic closure \( \cl^\vDash(\Gamma) \) of \( \Gamma \).

  \begin{thmenum}
    \thmitem{def:first_order_theory/axiomatized}\mcite[def. 1.4.37]{Hinman2005} We say that \( \Gamma \) is \term{axiomatized} by \( \Delta \) if \( \Gamma = \cl(\Delta) \).

    \medskip

    \thmitem{def:first_order_theory/complete}\mcite[def. 1.4.8]{Hinman2005} The set \( \Gamma \) of closed formulas is said to be \term{complete} if every for every formula in \( \varphi \), either \( \Gamma \vDash \varphi \) or \( \Gamma \vDash \neg \varphi \).

    \thmitem{def:first_order_theory/consistent} Given a deductive system, the set \( \Gamma \) of formulas is said to be \term[bg=противоречива,ru=противоречивая]{inconsistent} if \( \Gamma \vdash \bot \) and \term{consistent} otherwise.
  \end{thmenum}
\end{definition}
\begin{comments}
  \item Closedness of a set of closed formulas should not be confused with \fullref{def:derivability_and_satisfiability/completeness}, which defines completeness of a deductive system via how it relates to semantics.

  \item Peter Hinman in \cite[def. 1.4.5]{Hinman2005} calls \( \cl(\Gamma) \) the \enquote{theory generated by \( \Gamma \)}, but we avoid the term \enquote{theory} altogether as much as possible.

  \item This definition straightforwardly adapts to propositional formulas.
\end{comments}

\begin{proposition}\label{thm:formulas_unsatisfiable_iff_inconsistent}
   Assuming \hyperref[rem:classical_logic]{classical logic}, a theory \( \Gamma \) is \hyperref[def:propositional_model]{unsatisfiable} if and only if it is semantically \hyperref[def:first_order_theory/consistent]{inconsistent}.
\end{proposition}
\begin{proof}
  \SufficiencySubProof Assume first that \( \Gamma \) is unsatisfiable. Then for all zero models of \( \Gamma \) any formula is satisfied vacuously, in particular that any model of \( \Gamma \) satisfies \( \bot \). Thus, \( \Gamma \vDash \bot \) and, by \fullref{thm:classical_first_order_logic_is_sound_and_complete}, \( \Gamma \vdash \bot \). Therefore, \( \Gamma \) is inconsistent.

  \NecessitySubProof Let \( \Gamma \) be inconsistent and suppose that \( \mscrX = (X, I) \) is a model of \( \Gamma \). Fix any valuation \( v \) in \( \mscrX \). Since \( \Gamma \vdash \bot \), \fullref{thm:classical_first_order_logic_is_sound_and_complete} implies that \( \Bracks{\bot}_v = T \).

  By the \hyperref[def:first_order_valuation/formula_valuation]{definition of formula valuation}, however, we have \( \Bracks{\bot}_v = F \). The obtained contradiction shows that \( \mscrX \) cannot be a model of \( \Gamma \) and since this structure was chosen arbitrarily, we conclude that \( \Gamma \) is unsatisfiable.
\end{proof}

\begin{theorem}[First-order compactness theorem]\label{thm:first_order_compactness_theorem}\mcite[thm. 3.1.2]{Hinman2005}
  If every finite subset of the set \( \Gamma \) of closed formulas is satisfiable, then \( \Gamma \) itself is satisfiable.
\end{theorem}
\begin{comments}
  \item The theorem also applies, with adaptations based on \fullref{rem:propositional_logic_as_first_order_logic}, to propositional formulas.
\end{comments}

\begin{theorem}[Downward L\"owenheim-Skolem theorem]\label{thm:downward_lowenheim_skolem_theorem}\mcite[thm. 2.3.35]{Hinman2005}
  Let \( \Gamma \) be a set of closed first-order formulas over some arbitrary language. Suppose that \( \Gamma \) has a \hyperref[def:first_order_model]{model} of \hyperref[def:set_finiteness]{infinite} \hyperref[thm:cardinality_existence]{cardinality}.

  Then \( \Gamma \) also has a \hyperref[def:set_countability]{countable} model.
\end{theorem}

\begin{example}[Skolem's paradox]\label{ex:skolems_paradox}
  From \fullref{thm:downward_lowenheim_skolem_theorem} it follows that \hyperref[def:zfc]{\logic{ZFC}}, if it is consistent, has a model at is at most countably infinite. The \hyperref[def:zfc/infinity]{axiom of infinity} states that no model of \logic{ZFC} is finite. Therefore, there exists a model of \logic{ZFC} that is countably infinite. But we can construct uncountable sets in \logic{ZFC}.

  Therefore, either:
  \begin{itemize}
    \item Uncountable sets within the metatheory are fundamentally different from those within the object theory.
    \item Uncountable sets are paradoxical and \logic{ZFC} must disallow them in order to be consistent.
    \item \logic{ZFC} is inconsistent even when restricted to countable sets.
  \end{itemize}
\end{example}

\begin{theorem}[Upward L\"owenheim-Skolem theorem]\label{thm:upward_lowenheim_skolem_theorem}\mcite[thm. 3.4.9]{Hinman2005}
  Let \( \Gamma \) be a set of closed first-order formulas over some arbitrary language. Suppose that \( \Gamma \) has a \hyperref[def:first_order_model]{model} of \hyperref[def:set_finiteness]{infinite} \hyperref[thm:cardinality_existence]{cardinality} \( \kappa \).

  Then for every \hyperref[def:set_finiteness]{infinite} \hyperref[def:cardinal]{cardinal} \( \mu \), the theory \( \Gamma \) also has a model of cardinality \( \mu \).
\end{theorem}

\begin{definition}\label{def:category_of_small_first_order_models}\mimprovised
  Let \( \mscrL \) be a first-order language and let \( \Gamma \) be a nonempty set of formulas. Suppose that we are given a \hyperref[def:grothendieck_universe]{Grothendieck universe} \( \mscrU \), which is safe to assume to be the smallest suitable one as explained in \fullref{def:large_and_small_sets}. We define the \term{category of \( \mscrU \)-small models} for \( \Gamma \) as the following \hyperref[rem:concrete_categories]{concrete category}:

  \begin{itemize}
    \item The \hyperref[def:category/objects]{objects} are the \( \mscrU \)-small models of \( \Gamma \).

    \item The \hyperref[def:category/morphisms]{morphisms} between two models are the \hyperref[def:first_order_homomorphism]{structure homomorphisms} between them.
  \end{itemize}
\end{definition}

\begin{remark}\label{rem:positive_formulas_in_category_of_models}
  We usually want \( \Gamma \) to be a set of \hyperref[def:positive_formula]{positive formulas} because, in general, homomorphisms are not injective and the homomorphic image of a model may fail to be a model. \Fullref{thm:positive_formulas_preserved_under_homomorphism} shows that if all formulas in \( \Gamma \) are positive, the image of any homomorphism is again a model and thus an object in the category.
\end{remark}

\begin{example}\label{ex:def:category_of_small_first_order_models}
  This is an incomplete list of categories of small models of \hyperref[def:first_order_theory]{first-order theories} that can be found in this document:
  \begin{itemize}
    \item The categories \hyperref[rem:pointed_set/category]{\( \cat{Set_*} \)}, \hyperref[def:set_with_involution/category]{\( \cat{Inv} \)}, \hyperref[def:magma/category]{\( \cat{Mag} \)}, \hyperref[def:monoid/category]{\( \cat{Mag}_* \)}, \hyperref[def:monoid]{\( \cat{Mon} \)}, \hyperref[def:group/category]{\( \cat{Grp} \)} and \hyperref[def:abelian_group]{\( \cat{Ab} \)}, whose relations are crucial for the definition and properties of groups.

    \item The category \hyperref[def:semiring/category]{\( \cat{SRing} \)} of semirings, \hyperref[def:semimodule]{\( \cat{SMod}_R \)} of semimodules, \hyperref[def:ring/category]{\( \cat{Ring} \)} of rings and \hyperref[def:module/category]{\( \cat{Mod}_R \)} of modules, which are all based on commutative monoids.

    \item The categories \hyperref[def:partially_ordered_set]{\( \cat{Pos} \)} in order theory and the related \hyperref[def:preordered_set/category]{\( \cat{PreOrd} \)}, \hyperref[def:totally_ordered_set]{\( \cat{Tos} \)}, as well as \hyperref[def:semilattice/category]{\( \cat{Lat} \)}, \hyperref[def:heyting_algebra/category]{\( \cat{Heyt} \)} and \hyperref[def:boolean_algebra/category]{\( \cat{Bool} \)} in lattice theory.
  \end{itemize}

  In contrast:
  \begin{itemize}
    \item We define the category \hyperref[def:category_of_small_topological_spaces]{\( \cat{Top} \)}  of topological spaces and all of its related categories within set theory without a corresponding first-order theory, and similarly for the \hyperref[def:category_of_small_affine_spaces]{category of affine spaces}.

    \item The category \hyperref[def:category_of_small_sets]{\( \cat{Set} \)} of sets with respect to either \logic{ZFC}, \logic{ZFC+U} or na\"ive set theory is not the same as the category of \( \mscrU \)-small models of set theory. Instead, it is a category within a fixed set theory. Within the metatheory of this document, we work within a fixed model of \logic{ZFC+U} with respect to \hyperref[rem:classical_logic]{classical logic}.

    \item Similarly, we do not care about models of \hyperref[def:peano_arithmetic]{Peano arithmetic} enough to study its category of \( \mscrU \)-small models. Instead, we only use a single model and denote it by \( \BbbN \).
  \end{itemize}
\end{example}

\begin{definition}\label{def:subobject_and_quotient}\mcite[126]{MacLane1998}
  For an object \( X \) in some \hyperref[def:category]{category}, define an equivalence relation on \hyperref[def:morphism_invertibility/left_cancellative]{monomorphisms} \( f: A \to X \) and \( g: B \to X \) with codomain \( X \) to hold if there exists an isomorphism \( h: A \to B \) such that \( f = g \bincirc h \). We call the equivalence classes of this relation \term{subobjects} of \( X \).

  \hyperref[thm:categorical_principle_of_duality]{Dually}, for the relation where for two epimorphisms \( f: X \to A \) and \( g: X \to B \) there exists an isomorphism \( h: A \to B \) such that \( f = h \bincirc g \), we say that the equivalence classes are \term{quotient objects} of \( X \).
\end{definition}

\begin{proposition}\label{thm:first_order_categorical_invertibility}
  Fix a \hyperref[def:category_of_small_first_order_models]{category of small models} over a \hyperref[def:first_order_theory]{first-order theory}.

  \begin{thmenum}
    \thmitem{thm:first_order_categorical_invertibility/injective} Every \hyperref[def:multi_valued_function/empty]{\hi{nonempty}} \hyperref[def:first_order_embedding]{structure embedding} (injective homomorphism) is a \hyperref[def:morphism_invertibility/left_cancellative]{categorical monomorphism}.

    In particular, via the canonical inclusion, every submodel of \( \mscrX \) is a \hyperref[def:subobject_and_quotient]{categorical subobject} of \( \mscrX \).

    A partial converse always holds --- every \hyperref[def:morphism_invertibility/left_cancellative]{split monomorphism} is injective.

    If the \hyperref[def:concrete_category]{forgetful functor} \( U: \cat{C} \to \cat{Set} \) has a \hyperref[def:category_adjunction]{left adjoint}, as it often does, then every monomorphism is injective. Otherwise, a homomorphism may be left invertible with respect to homomorphisms but not general functions.

    \thmitem{thm:first_order_categorical_invertibility/surjective} \hyperref[thm:categorical_principle_of_duality]{Dually}, every surjective homomorphism is a \hyperref[def:morphism_invertibility/right_cancellative]{categorical epimorphism}.

    In particular, via the canonical inclusion, every surjective function of \( \mscrX \) is a \hyperref[def:subobject_and_quotient]{categorical quotient object} of \( \mscrX \).

    A partial converse always holds --- every \hyperref[def:morphism_invertibility/right_cancellative]{split epimorphism} is surjective.

    If the \hyperref[def:concrete_category]{forgetful functor} \( U: \cat{C} \to \cat{Set} \) has a \hyperref[def:category_adjunction]{right adjoint}, then every epimorphism is surjective.

    \thmitem{thm:first_order_categorical_invertibility/bijective} Every \hyperref[def:first_order_embedding]{structure isomorphism} is a \hyperref[def:morphism_invertibility/isomorphism]{categorical isomorphism} and vice versa.
  \end{thmenum}
\end{proposition}
\begin{proof}
  All follow from \fullref{thm:concrete_category_function_invertibility}.
\end{proof}

\begin{definition}\label{def:lindenbaum_tarski_algebra}\mcite[def. 1.7.4]{Hinman2005}
  Assume some fixed \hyperref[def:deductive_system]{deductive system} in propositional or first-order logic. Let \( \Gamma \) be a set of \hyperref[def:first_order_syntax/closed_formula]{closed formulas} within the corresponding logic, and suppose that \( \Gamma \) is closed in the sense of \fullref{def:first_order_theory}.

  Then \( (\Gamma, \vdash) \) is a \hyperref[def:preordered_set]{preordered set}. We define the \term{Lindenbaum-Tarski algebra} of the theory \( \Gamma \) is the partially ordered set obtained from \( (\Gamma, \vdash) \) using \fullref{thm:preorder_to_partial_order}.

  More concretely, the Lindenbaum-Tarski algebra of \( \Gamma \) is a quotient set of \( \Gamma \) by the equivalence relation \hyperref[thm:equivalence_closure]{induced by} \( \vdash \) and endowed with the partial order
  \begin{equation}\label{eq:def:lindenbaum_tarski_algebra/order}
    [\varphi] \leq [\psi] \T{if and only if} \varphi \vdash \psi.
  \end{equation}
\end{definition}
\begin{comments}
  \item Of course, we can define the algebra using entailment rather than derivability, but in the cases we consider, the two are equivalent and derivability is simpler to work with.
\end{comments}
\begin{proof}
  The correctness of \eqref{eq:def:lindenbaum_tarski_algebra/order}, i.e. the fact that the relation \( \leq \) does not depend on the choice of representatives from the quotient sets, follows from \fullref{thm:preorder_to_partial_order}.

  We must only demonstrate that \( (\Gamma, \vdash) \) is indeed a preordered set. Reflexivity of \( \vdash \) follows from \fullref{def:axiomatic_deductive_system} and transitivity follows from \fullref{thm:deductive_system_transitivity}.
\end{proof}

\begin{proposition}\label{thm:intuitionistic_lindenbaum_tarski_algebra}
  Assume that we are working in the \hyperref[def:intuitionistic_propositional_deductive_systems]{intuitionistic propositional natural deduction system}. A \hyperref[def:lindenbaum_tarski_algebra]{Lindenbaum-Tarski algebra} is then is a \hyperref[def:heyting_algebra]{Heyting algebra}.

  In the \hyperref[def:classical_propositional_deductive_systems]{classical derivation system}, the algebra is instead a \hyperref[def:boolean_algebra]{Boolean algebra}.

  In the \hyperref[def:minimal_propositional_hilbert_system]{minimal derivation system}, we have an unbounded lattice with only a top element, but no bottom. Consequently, conditionals and pseudocomplements may fail to exist.

  Explicitly:
  \begin{thmenum}
    \thmitem{thm:intuitionistic_lindenbaum_tarski_algebra/join} The \hyperref[def:semilattice/join]{join} of the equivalence classes \( [\psi_1] \) and \( [\psi_2] \) is the class \( [\psi_1 \vee \psi_2] \) of their \hyperref[def:propositional_language/connectives/disjunction]{disjunction}.

    \thmitem{thm:intuitionistic_lindenbaum_tarski_algebra/bottom} The class of \hyperref[def:propositional_semantics/contradiction]{contradictions} \( [\bot] \) is the \hyperref[def:extremal_points/top_and_bottom]{bottom element}.

    \thmitem{thm:intuitionistic_lindenbaum_tarski_algebra/meet} Similarly to joins, the \hyperref[def:semilattice/meet]{meet} of \( [\psi_1] \) and \( [\psi_2] \) is the equivalence class \( [\psi_1 \wedge \psi_2] \) of their \hyperref[def:propositional_language/connectives/conjunction]{conjunction}.

    \thmitem{thm:intuitionistic_lindenbaum_tarski_algebra/top} The class of \hyperref[def:propositional_semantics/tautology]{tautologies} \( [\top] \) is the \hyperref[def:extremal_points/top_and_bottom]{top element}.

    \thmitem{thm:intuitionistic_lindenbaum_tarski_algebra/conditional} The \hyperref[eq:def:heyting_algebra/conditional]{conditional} of \( [\psi_1] \) and \( [\psi_2] \) is the equivalence class \( [\psi_1 \rightarrow \psi_2] \).

    \thmitem{thm:intuitionistic_lindenbaum_tarski_algebra/complement} The \hyperref[eq:def:heyting_algebra/pseudocomplement]{pseudocomplement} \( \widetilde{[\psi]} \) of \( [\psi] \) equals \( [\neg \psi] \).

    In the classical derivation system this pseudocomplement is a complement, i.e. it satisfies \eqref{def:bounded_lattice_complement/join} and \eqref{def:bounded_lattice_complement/meet}.
  \end{thmenum}
\end{proposition}
\begin{proof}
  \SubProofOf{thm:intuitionistic_lindenbaum_tarski_algebra/join} We will show that \( [\psi_1 \vee \psi_2] \) is the supremum of \( [\psi_1] \) and \( [\psi_2] \).

  The inference rule \eqref{eq:def:minimal_propositional_natural_deduction_system/or/intro_left} implies that \( \psi_1 \vdash \psi_1 \vee \psi_2 \) and \eqref{eq:def:minimal_propositional_natural_deduction_system/or/intro_right} implies that \( \psi_2 \vdash \psi_1 \vee \psi_2 \). Thus, \( \psi_1 \vee \psi_2 \) is an upper bound for both \( \psi_1 \) and \( \psi_2 \) under the ordering \( \vdash \).

  Let \( \varphi \) be any formula in \( \Gamma \) such that \( \psi_1 \vdash \varphi \), \( \psi_2 \vdash \varphi \) and \( \varphi \vdash (\psi_1 \vee \psi_2) \). Then the following instance of \eqref{eq:def:minimal_propositional_natural_deduction_system/or/elim}
  \begin{equation*}
    \begin{prooftree}
      \hypo{ [\psi_1 \vee \psi_2] }
      \hypo{ [\psi_1]^1 }
      \ellipsis {} { \varphi }
      \hypo{ [\psi_2]^1 }
      \ellipsis {} { \varphi }
      \infer[left label=\( 1 \)]3[\ref{eq:def:minimal_propositional_natural_deduction_system/or/elim}]{ \varphi }
    \end{prooftree}
  \end{equation*}
  demonstrates that \( \psi_1, \psi_2, (\psi_1 \vee \psi_2) \vdash \varphi \). Hence, from \fullref{thm:deductive_system_transitivity} it follows that
  \begin{equation*}
    (\psi_1 \vee \psi_2) \vdash \varphi.
  \end{equation*}

  Our choice of representatives from \( [\psi_1] \) and \( [\psi_2] \) does not matter for derivability, hence \( [\varphi] = [\psi_1 \vee \psi_2] \) and this is indeed the supremum of \( [\psi_1] \) and \( [\psi_2] \).

  \SubProofOf{thm:intuitionistic_lindenbaum_tarski_algebra/top} The inference rule \eqref{eq:def:minimal_propositional_natural_deduction_system/top/intro} shows that \( [\varphi] \leq [\top] \) for any formula \( \varphi \) and \( [\top] \) is the top element.

  \SubProofOf{thm:intuitionistic_lindenbaum_tarski_algebra/meet} Let \( \psi_1 \) and \( \psi_2 \) be any formulas in \( \Gamma \). The inference rule \eqref{eq:def:minimal_propositional_natural_deduction_system/and/elim_left} implies that \( \psi_1 \wedge \psi_2 \vdash \psi_2 \) and \eqref{eq:def:minimal_propositional_natural_deduction_system/and/elim_right} implies that \( \psi_1 \wedge \psi_2 \vdash \psi_1 \). Thus, \( \psi_1 \wedge \psi_2 \) is a lower bound for both \( \psi_1 \) and \( \psi_2 \) under the ordering \( \vdash \).

  We must show that \( \psi_1 \wedge \psi_2 \) is derives the greatest lower bound and vice versa. Let \( \varphi \) be a formula in \( \Gamma \) such that \( \varphi \vdash \psi_1 \), \( \varphi \vdash \psi_2 \) and \( (\psi_1 \wedge \psi_2) \vdash \varphi \). We will show that \( \varphi \vdash (\psi_1 \wedge \psi_2) \).

  The rule \eqref{eq:def:minimal_propositional_natural_deduction_system/and/intro} implies that
  \begin{equation*}
    \psi_1, \psi_2 \vdash \psi_1 \wedge \psi_2.
  \end{equation*}

  But \( \varphi \) derives both \( \psi_1 \) and \( \psi_2 \), hence due to \fullref{thm:deductive_system_transitivity},
  \begin{equation*}
    \varphi \vdash (\psi_1 \wedge \psi_2).
  \end{equation*}

  Analogously to our proof of \fullref{thm:intuitionistic_lindenbaum_tarski_algebra/join}, we conclude that the choice of representatives of the equivalence classes is irrelevant and that \( [\varphi] = [\psi_1 \wedge \psi_2] \) is the infimum of \( [\psi_1] \) and \( [\psi_2] \).

  \SubProofOf{thm:intuitionistic_lindenbaum_tarski_algebra/bottom} That \( [\bot] \) is the bottom is a restatement of \eqref{eq:thm:minimal_propositional_negation_laws/efq}.

  \SubProofOf{thm:intuitionistic_lindenbaum_tarski_algebra/conditional} Restating \eqref{eq:def:heyting_algebra/conditional}, we must prove that \( [\psi_1 \rightarrow \psi_2] \) equals
  \begin{equation*}
    ([\psi_1] \rightarrow [\psi_2]) = \underbrace{\sup\set[\Big]{ [\varphi] \given* \varphi \in \Gamma \T{and} (\varphi \wedge \psi_1) \vdash \psi_2 }}_{\Phi}.
  \end{equation*}

  An equivalent condition for \( \varphi \) to be in \( \Phi \) is, due to \fullref{thm:conjunction_of_premises},
  \begin{equation*}
    \varphi, \psi_1 \vdash \psi_2.
  \end{equation*}

  \Fullref{thm:syntactic_deduction_theorem} implies that
  \begin{equation*}
    \varphi \vdash (\psi_1 \rightarrow \psi_2).
  \end{equation*}

  Thus, \( [\psi_1 \rightarrow \psi_2] \) is an upper bound of \( \Psi \).

  It remains to show that \( (\psi_1 \rightarrow \psi_2) \in \Psi \). Both \( \psi_1 \) and \( (\psi_1 \rightarrow \psi_2) \) follow from the formula \( \parens[\Big]{ (\psi_1 \rightarrow \psi_2) \wedge \psi_1 } \) and by applying \eqref{eq:def:def:axiomatic_deductive_system/mp}, we obtain
  \begin{equation*}
    \psi_1, (\psi_1 \rightarrow \psi_2) \vdash \psi_2.
  \end{equation*}

  From \fullref{thm:conjunction_of_premises},
  \begin{equation*}
    \parens[\Big]{ (\psi_1 \rightarrow \psi_2) \wedge \psi_1 } \vdash \psi_2.
  \end{equation*}

  Therefore, \( [\psi_1 \rightarrow \psi_2] \in \Phi \) and it is indeed the supremum of \( \Psi \).

  \SubProofOf{thm:intuitionistic_lindenbaum_tarski_algebra/complement} The pseudocomplement \( \widetilde{[\psi]} \) is, by definition,
  \begin{equation*}
    \widetilde{[\psi]}
    =
    [\psi] \rightarrow [\bot].
  \end{equation*}

  From what we have already proved, we can conclude that \( \widetilde{[\psi]} = [\psi \rightarrow \bot] \). From \fullref{def:minimal_propositional_hilbert_system/negation} it follows that the formula \( \psi \rightarrow \bot \) derives \( \neg \psi \) and vice versa, thus \( \widetilde{[\psi]} = [\neg \psi] \).

  If we are working in classical logic where \eqref{eq:thm:minimal_propositional_negation_laws/lem} holds, then
  \begin{equation*}
    \sup\set{ [\psi], \widetilde{[\psi]} }
    \reloset {\ref{thm:intuitionistic_lindenbaum_tarski_algebra/join}} =
    [\psi \vee \neg \psi]
    \reloset {\eqref{eq:thm:minimal_propositional_negation_laws/lem}} =
    [\top],
  \end{equation*}
  which proves \eqref{def:bounded_lattice_complement/join}.

  The dual law shows \eqref{def:bounded_lattice_complement/meet}:
  \begin{equation*}
    \inf\set{ [\psi], \widetilde{[\psi]} }
    \reloset {\ref{thm:intuitionistic_lindenbaum_tarski_algebra/meet}} =
    [\psi \wedge \neg \psi]
    \reloset {\eqref{eq:thm:minimal_propositional_negation_laws/lnc}} =
    [\neg \top]
    =
    \widetilde{[\top]}
    \reloset {\eqref{eq:def:heyting_algebra/pseudocomplement}} =
    [\bot].
  \end{equation*}
\end{proof}

\begin{remark}\label{rem:thm:intuitionistic_lindenbaum_tarski_algebra/syntactic_proof}
  Notice that our proof of \fullref{thm:intuitionistic_lindenbaum_tarski_algebra} relies entirely on the derivation system. On the other hand, we define \hyperref[def:propositional_heyting_algebra_semantics]{Heyting semantics} for intuitionistic formulas.

  Thus, we have shown that Heyting algebras arise naturally from the intuitionistic natural deduction system and that their role as a semantical framework is justified.
\end{remark}

\begin{proposition}\label{thm:filters_intuitionistic_lindenbaum_tarski_algebra}
  Fix some first-order language \( \mscrL \) and let \( L \) be the \hyperref[thm:filters_intuitionistic_lindenbaum_tarski_algebra]{classical Lindenbaum-Tarski algebra} over all formulas of \( \mscrL \).

  Given a \hyperref[def:first_order_theory/complete]{complete set of closed formulas} \( \Gamma \), the set of (equivalence classes of) all formulas derivable from \( \Gamma \) is an \hyperref[def:ultrafilter]{ultrafilter} in \( L \).
\end{proposition}
\begin{proof}
  Simply a reformulation of the definition of completeness.
\end{proof}
