\subsection{Totally ordered sets}\label{subsec:totally_ordered_sets}

\paragraph{Totally orders}

\begin{definition}\label{def:totally_ordered_set}
  We say that a partially ordered set \( (P, \leq) \) is \term[ru=линейно упорядоченое множество (\cite[def. 3.4]{Гуров2013Решётки})]{linearly} or \term{totally ordered} if any of the following equivalent conditions hold:
  \begin{thmenum}
    \thmitem{def:totally_ordered_set/connected}\mcite[2]{Birkhoff1967Lattices} The nonstrict relation \( \leq \) is \hyperref[def:binary_relation/connected]{connected}.

    \thmitem{def:totally_ordered_set/trichotomic}\mcite[4]{Engelking1989Topology} The strict relation \( < \) satisfies \hyperref[def:binary_relation/trichotomy]{trichotomy}.

    \thmitem{def:totally_ordered_set/chain}\mcite[2]{Birkhoff1967Lattices} The set \( P \) is itself a \hyperref[def:partial_order_chain/chain]{chain}.

    \thmitem{def:totally_ordered_set/width} The set \( P \) has \hyperref[def:partial_order_chain/width]{width} \( 1 \).
  \end{thmenum}
\end{definition}
\begin{comments}
  \item The metamathematical properties are inherited from \fullref{def:partially_ordered_set} with the additional axiom \eqref{eq:def:binary_relation/connected}. We denote the corresponding category via \( \cat{Tos} \).
\end{comments}
\begin{defproof}
  \ImplicationSubProof{def:totally_ordered_set/connected}{def:totally_ordered_set/trichotomic} By inspecting the compatibility condition \eqref{eq:def:preordered_set/compatibility_nonstrict} we conclude that \( x \leq y \) implies \( x < y \) or \( x = y \) and \( x \geq y \) implies \( x > y \) or \( x = y \).

  Hence, if the nonstrict relation is connected, the strict relation satisfies trichotomy.

  \ImplicationSubProof{def:totally_ordered_set/trichotomic}{def:totally_ordered_set/chain} by inspecting the compatibility condition \eqref{eq:def:preordered_set/compatibility_strict} we conclude that \( x < y \) implies \( x \leq y \), \( x > y \) implies \( x \geq y \), and \( x = y \) implies both.

  Hence, if the strict relation satisfies trichotomy, every two elements are comparable.

  \ImplicationSubProof{def:totally_ordered_set/chain}{def:totally_ordered_set/width} If every two elements are comparable, no two are incomparable, hence antichains are either the empty set or are singleton sets.

  Hence, \( P \) has width \( 1 \).

  \ImplicationSubProof{def:totally_ordered_set/width}{def:totally_ordered_set/connected} Suppose that \( P \) has width \( 1 \). For any two distinct elements \( x \) and \( y \), the set \( \set{ x, y } \) is either a chain or an antichain. But it cannot be an antichain since then \( P \) would have width \( 2 \). It remains for \( \set{ x, y } \) to be a chain, that is, it remains for \( x \) and \( y \) to be comparable.

  Generalizing on \( x \) and \( y \), we conclude that the nonstrict relation is connected.
\end{defproof}

\begin{proposition}\label{thm:def:totally_ordered_set}
  \hyperref[def:totally_ordered_set]{Totally ordered sets} have the following basic properties:
  \begin{thmenum}
    \thmitem{thm:def:totally_ordered_set/embedding_iff_strict} An \hyperref[def:order_function/preserving]{order-preserving map} between totally ordered sets is an \hyperref[def:preordered_set/homomorphism]{order embedding} if and only if it is \hyperref[def:order_function/preserving]{strict}.

    \thmitem{thm:def:totally_ordered_set/maximal_iff_greatest} In a totally ordered set, \( x \) is \hyperref[def:extremal_points/maximal_and_minimal_element]{maximal} (resp. maximal) in the set \( A \) if and only if \( x \) is the \hyperref[def:extremal_points/greatest_and_least]{greatest} (resp. least) element of \( A \).

    \thmitem{thm:def:totally_ordered_set/initial_segment} Let \( (P, \leq) \) be a totally ordered set. Let \( Q \) be the family of \hyperref[def:order_interval/unbounded]{strict initial segments} \( P_{<x} \) of members \( x \) of \( P \).

    Then \( (P, \leq) \) is \hyperref[def:preordered_set/homomorphism]{order-isomorphic} to \( (Q, \subseteq) \).

    \thmitem{thm:def:totally_ordered_set/lexicographic} If \( (P, \leq_P) \) and \( (Q, \leq_Q) \) are totally ordered sets, then the \hyperref[eq:def:lexicographic_order]{lexicographic} and \hyperref[eq:def:lexicographic_order/reverse]{reverse lexicographic} orders on \( P \times Q \) are strict total order relations.

    \thmitem{thm:def:totally_ordered_set/cofinal_iff_unbounded} If a totally ordered set is \hyperref[def:extremal_points/bounds]{unbounded from above}, then a subset is \hyperref[def:cofinal_set]{cofinal} if and only if it is itself unbounded from above.
  \end{thmenum}
\end{proposition}
\begin{proof}
  \SubProofOf{thm:def:totally_ordered_set/embedding_iff_strict} Fix an order-preserving map \( f: P \to Q \) between totally ordered sets.

  \SufficiencySubProof* If \( f \) is an order embedding, it is injective, and it follows from \fullref{thm:def:order_function/injective_implies_strict} that \( f \) is strict.

  \NecessitySubProof* Suppose that \( f \) is strict.

  We will first show that \( f \) is injective. Every two elements \( x_1 \) and \( x_2 \) of \( P \) are comparable, thus, if \( f(x_1) = f(x_2) \), since \( f \) is strict, we obtain contradictions with both \( x_1 < x_2 \) and \( x_1 > x_2 \). Thus, it remains for \( x_1 \) and \( x_2 \) to be equal.

  Furthermore, \( f \) is also order-reflecting. Indeed, if \( y_1 \leq y_2 \), then \( f^{-1}(y_1) \leq f^{-1}(y_2) \) because \( f^{-1}(y_1) > f^{-1}(y_2) \) would imply
  \begin{equation*}
     y_1 = f(f^{-1}(y_1)) > f(f^{-1}(y_2)) = y_2.
  \end{equation*}

  We conclude that \( f \) is an order embedding.

  \SubProofOf{thm:def:totally_ordered_set/maximal_iff_greatest}

  \SufficiencySubProof* If \( x \) is maximal in \( A \), then \( y \geq x \) implies \( x = y \) for every \( y \) in \( A \), hence, by trichotomy, \( y \leq x \). Hence, \( x \) is an upper bound of \( A \) that belongs to \( A \), that is, a maximum.

  \NecessitySubProof* If \( x \) is a maximum of \( A \), \fullref{thm:def:extremal_points/greatest_is_maximal} implies that is is maximal in \( A \).

  \SubProofOf{thm:def:totally_ordered_set/initial_segment} Consider the map
  \begin{equation*}
    \begin{aligned}
      &f: P \to Q \\
      &f(x) \coloneqq P_{<x} = \set{ y \in P \given y < x }.
    \end{aligned}
  \end{equation*}

  Note that \( f \) is \hyperref[def:order_function/preserving]{strictly order-preserving}. Indeed, if \( x < y \), then \( x \in P_{<y} \), but \( x \not\in P_{<x} \) and hence \( P_{<x} \) is a strict subset of \( P_{<y} \).

  Then \fullref{thm:def:totally_ordered_set/embedding_iff_strict} implies that \( f \) is an embedding. Since \( f \) is also surjective by definition of \( Q \), it follows that it is a strict isomorphism between \( (P, \leq) \) and \( (Q, \subseteq) \).

  \SubProofOf{thm:def:totally_ordered_set/lexicographic} We have already shown in \fullref{thm:lexicographic_order_is_partial_order} that the lexicographic order \( \prec \) on \( P \times Q \) is a partial order. It only remains to check trichotomy.

  Let \( (a, b) \) and \( (c, d) \) be pairs in \( P \times Q \). Since \( <_P \) and \( <_Q \) are strict total orders, we only have the following possibilities:
  \begin{itemize}
    \item If \( a = c \) and \( b = d \), then \( (a, b) = (c, d) \).
    \item If \( a = c \) and \( b <_Q d \), then \( (a, b) \prec (c, d) \).
    \item If \( a = c \) and \( b >_Q d \), then \( (a, b) \succ (c, d) \).
    \item If \( a <_P c \), then \( (a, b) \prec (c, d) \).
    \item If \( a >_P c \), then \( (a, b) \succ (c, d) \).
  \end{itemize}

  Therefore, trichotomy holds in \( (P \times Q, \prec) \).

  The proof for the reverse lexicographic order is analogous.

  \SubProofOf{thm:def:totally_ordered_set/cofinal_iff_unbounded} We will prove the converse - a set is bounded if and only if it is not cofinal.

  \SufficiencySubProof* Let \( A \) be bounded from above by \( x \). \Fullref{thm:def:partial_order_chain/unbounded} then implies that there exists some \( y \) such that \( x < y \). But no element of \( A \) is greater than \( y \), implying that \( A \) is not cofinal.

  \NecessitySubProof* Suppose that \( A \) is not cofinal. Let \( x \in P \). Then, for every \( y \in A \), \( x < y \) does not hold, that is, \( y \leq x \) for every \( y \in A \). Therefore, \( x \) is an upper bound of \( A \).
\end{proof}

\paragraph{Well-ordered sets}

\begin{definition}\label{def:well_ordered_set}\mcite[180]{Birkhoff1967Lattices}
  We say that a \hyperref[def:totally_ordered_set]{totally ordered set} \( (P, \leq) \) is \term[bg=добра (верига) (\cite[14]{Проданов1982ФункАнализ}), ru=вполне упорядоченое (множество) (\cite[def. 3.15]{Гуров2013Решётки})]{well-ordered} if it satisfies the \hyperref[def:chain_condition]{descending chain condition}.
\end{definition}

\begin{proposition}\label{thm:def:well_ordered_set}
  \hyperref[def:well_ordered_set]{Well-ordered sets} have the following basic properties:
  \begin{thmenum}
    \thmitem{thm:def:well_ordered_set/finite_chain} Every finite \hyperref[def:totally_ordered_set]{totally ordered set} is well-ordered.
    \thmitem{thm:def:well_ordered_set/embedding_extensional} Every \hyperref[def:preordered_set/homomorphism]{order embedding} of a well-ordered set to itself is \hyperref[def:extensive_function]{extensive}, that is, if \( f: P \to P \) is an embedding, we have \( x \leq f(x) \) for every \( x \).

    \thmitem{thm:def:well_ordered_set/unique_isomorphism} There is at most one isomorphism between any pair of well-ordered sets.

    \thmitem{thm:def:well_ordered_set/lexicographic} If \( (P, \leq_P) \) and \( (Q, \leq_Q) \) are well ordered sets, then the \hyperref[eq:def:lexicographic_order]{lexicographic} and \hyperref[eq:def:lexicographic_order/reverse]{reverse lexicographic} orders on \( P \times Q \) are strict total order relations.
  \end{thmenum}
\end{proposition}
\begin{proof}
  \SubProofOf{thm:def:well_ordered_set/finite_chain} Let \( (P, \leq) \) be totally ordered set of finite cardinality \( n \). Let \( A \) be any nonempty subset of \( P \). We will show by induction on \( \card(A) \) that \( A \) has a minimum.

  \begin{itemize}
    \item If \( A = \set{ a } \), then \( a \) is vacuously a minimum.
    \item Suppose that all subsets of \( P \) of cardinality \( k - 1 \) have a minimum. Let \( \card(A) = k \). Pick an arbitrary element \( a \in A \). By the inductive hypothesis, \( A \setminus \set{ a } \) has a minimum, say \( b \). Then \( \min\set{ a, b } \) is a minimum of \( A \).
  \end{itemize}

  \SubProofOf{thm:def:well_ordered_set/embedding_extensional} We proceed by induction on \( P \). Fix \( x_0 \) and suppose that \( y \leq f(y) \) for all \( y < x_0 \).

  Aiming at a contradiction, suppose that \( f(x_0) < x_0 \). Thus, there exists some \( y_0 < x_0 \) such that \( f(x_0) = y_0 \). By the inductive hypothesis we have \( f(x_0) = y_0 \leq f(y_0) \). Thus, either \( f(x_0) < f(y_0) \), which contradicts that \( f \) is an order homomorphism, or \( f(x_0) = f(y_0) \), which contradicts the injectivity of \( f \).

  The obtained contradiction demonstrates that \( x \leq f(x) \) for all \( x \in P \).

  \SubProofOf{thm:def:well_ordered_set/unique_isomorphism} Let \( f: P \to Q \) and \( g: P \to Q \) be two order isomorphisms between well-ordered sets.

  For any member \( x \) of \( P \) we have the following possibilities:
  \begin{itemize}
    \item If \( f(x) < g(x) \), then \( g^{-1}(f(x)) < x \) and \( f(g^{-1}(f(x))) < f(x) \). Then we can use \hyperref[rem:natural_number_recursion]{natural number recursion} to build an strictly descending sequence of members of \( Q \). But this is a contradiction because \( Q \) satisfies the descending chain condition.

    \item We can build a similar sequence if \( g(x) < f(x) \).

    \item It remains for \( f(x) = g(x) \) to hold.
  \end{itemize}

  Since \( x \in P \) was arbitrary, we conclude that \( f = g \).

  \SubProofOf{thm:def:well_ordered_set/lexicographic} We have already shown in \fullref{thm:def:totally_ordered_set/lexicographic} that these are total orders. It only remains to check the \hyperref[def:chain_condition]{descending chain condition}.

  Let \( \prec \) be the lexicographic order on \( P \times Q \).

  Suppose that there exists an strictly descending sequence
  \begin{equation*}
    \cdots \prec (a_3, b_3) \prec (a_2, b_2) \prec (a_1, b_1).
  \end{equation*}

  Since \( P \) satisfies the descending chain condition, the corresponding sequence \( \seq{ a_k }_{k=1}^\infty \). Then there exists an index \( k_0 \) such that \( a_{k_0} = a_k \) for \( k \geq k_0 \). Therefore, the sequence \( \set{ b_k }_{k=k_0}^\infty \) must be infinitely descending. But this contradicts the chain condition imposed on \( Q \).

  Therefore, no strictly descending sequence exists in the totally ordered set \( (P \times Q, \prec) \), and thus it is well-ordered.
\end{proof}

\paragraph{Order topology}

\begin{definition}\label{def:order_topology}\mcite[57]{Kelley1975Topology}
  Let \( P \) be a \hyperref[def:partially_ordered_set]{totally ordered set} with more than one element\fnote{If \( P \) is a singleton set, the supposed subbase \eqref{eq:def:order_topology/subbase} fails to cover \( P \) and thus does not satisfy \fullref{thm:topology_from_subbase}. On the other hand, if \( P \) is either empty or a singleton set, \fullref{thm:empty_set_discrete_and_indiscrete_topologies} implies that its topology is unique, so adding a special case to this definition will not introduce anything new.}. The \term{order topology} induced by \( \leq \) is the topology generated by the \hyperref[def:topological_subbase]{subbase} of \hyperref[def:order_interval/unbounded]{strict initial and final segments}
  \begin{equation}\label{eq:def:order_topology/subbase}
    \mscrS \coloneqq \set[\Big]{ P_{>a} \given* a \in P } \cup \set[\Big]{ P_{<b} \given* b \in P }.
  \end{equation}

  The \hyperref[def:topological_base]{base} corresponding to this subbase adds \hyperref[def:order_interval/open]{open segments}:
  \begin{equation}\label{eq:def:order_topology/base}
    \mscrB = \mscrS \cup \set[\Big]{ \varnothing } \cup \set[\Big]{ (a, b) \given a, b \in P \T{and} a < b }.
  \end{equation}
\end{definition}
\begin{defproof}
  \SubProof{Proof of compatibility of \( \mscrS \) and \( \mscrB \)} Define
  \begin{equation*}
    \mscrC = \set*{ \bigcap \mscrS \given* \mscrS \T{is a nonempty finite subset of} \mscrS }.
  \end{equation*}

  We will show that \( \mscrB = \mscrC \).

  \SubProof*{Proof that \( \mscrB \subseteq \mscrC \)} Let \( B \in \mscrB \). The definition of \( \mscrB \) suggests the following possibilities:
  \begin{itemize}
    \item If \( B \in \mscrS \), there is nothing to prove.
    \item If \( B = \varnothing \), then, for any element \( a \) of \( P \), we have \( B = P_{<a} \cap P_{>a} \).
    \item If \( B = (a, b) \) for some \( a < b \), then \( B = (a, b) = P_{<b} \cap P_{>a} \).
  \end{itemize}

  In all cases, \( B \) is an intersection of either one or two members of \( \mscrS \), hence \( B \in \mscrC \).

  Generalizing on \( B \), we conclude that \( \mscrB \subseteq \mscrC \).

  \SubProof*{Proof that \( \mscrC \subseteq \mscrB \)} Now let \( C = S_1 \cap \cdots \cap S_n \), where \( S_1, \ldots, S_n \) are members of \( \mscrS \). We will show by induction on \( n > 0 \) that \( C \in \mscrB \).

  The case \( n = 1 \) is trivial. Let \( n > 1 \), suppose that all \( n \)-ary intersections belong to \( \mscrB \) and consider
  \begin{equation*}
    C = S_1 \cap \cdots \cap S_n \cap S_{n+1}.
  \end{equation*}

  By the inductive hypothesis, we have that \( D \coloneqq S_1 \cap \cdots \cap S_n \) belongs to \( \mscrB \). We have the following possibilities:
  \begin{itemize}
    \item If \( D = P_{<b} \), then:
    \begin{itemize}
      \item If \( S_{n+1} = P_{<c} \), then \( D \cap S_{n+1} = P_{<\min\set{c, b}} \), which belongs to \( \mscrS \).
      \item If \( S_{n+1} = P_{>c} \), then \( D \cap S_{n+1} = (c, b) \) (empty if \( b \leq c \)), which belongs to \( \mscrB \).
    \end{itemize}

    \item If \( D \) is empty, then \( D \cap S_{n+1} \) is empty, hence it belongs to \( \mscrB \).

    \item If \( D = (a, b) \), then:
    \begin{itemize}
      \item If \( S_{n+1} = P_{<c} \), then \( D \cap S_{n+1} = (\min\set{a, c}, \min\set{b, c}) \) (possibly empty), which belongs to \( \mscrB \).
      \item If \( S_{n+1} = P_{>c} \), then \( D \cap S_{n+1} = (\max\set{a, c}, \max\set{b, c}) \), which also belongs to \( \mscrB \).
    \end{itemize}
  \end{itemize}

  Therefore, \( C = D \cap S_{n+1} \) belongs to \( \mscrB \).

  Generalizing on \( C \), we conclude that \( \mscrC \subseteq \mscrB \).

  \SubProof{Proof that \( \mscrB \) is a base} We will show that the axioms in \fullref{thm:topology_from_base} hold.

  \SubProofOf*[thm:topology_from_base/B1]{B1} Let \( x \in P \).

  \begin{itemize}
    \item If \( x \) is a \hyperref[def:extremal_points/maximum_and_minimum]{maximum}, then take any other value \( y < x \) and the final segment \( P_{>y} \) will contain \( x \). We use here that there is more than one element in \( P \).

    \item If \( x \) is not a maximum, then \( x \) belongs to any initial segment \( P_{<y} \) whenever \( y > x \).
  \end{itemize}

  In both cases there exists some set in \( \mscrS \) containing \( x \). Thus, \( \bigcup S = P \).

  \SubProofOf*[thm:topology_from_base/B2]{B2} Let \( U \) and \( V \) be members of \( \mscrB \). We consider \( 9 \) cases:
  \begin{itemize}
    \item If either \( U = \varnothing \) or \( V = \varnothing \), then \( U \cap V = \varnothing \).
    \item If \( U = P_{<u} \) and \( V = P_{>v} \), then \( U \cap V = (v, u) \) (empty if \( v \geq u \)).
    \item If \( U = P_{<u} \) and \( V = (v_1, v_2) \), then \( U \cap V = (\min{v_1, u}, \min{v_2, u}) \) (possibly empty).
    \item If \( U = (u_1, u_2) \) and \( V = P_{>v} \), then \( U \cap V = (\max{u_1, v}, \max{u_2, v}) \) (possibly empty).
    \item If \( U = (u_1, u_2) \) and \( V = (v_1, v_2) \), then:
    \begin{itemize}
      \item If \( u_2 < v_1 \), then \( U \cap V = \varnothing \).
      \item If \( u_1 < v_1 < u_2 < v_2 \) then \( U \cap V = (v_1, u_2) \).
      \item If \( u_1 < v_1 < v_2 < u_2 \) then \( U \cap V = V = (v_1, v_2) \).
      \item If \( v_1 < u_1 < u_2 < v_2 \) then \( U \cap V = U = (u_1, u_2) \).
      \item If \( v_1 < u_1 < v_2 < u_2 \) then \( U \cap V = (u_1, v_2) \).
    \end{itemize}
  \end{itemize}

  In all the cases, the intersection \( U \cap V \) belongs to \( \mscrB \).
\end{defproof}

\begin{proposition}\label{thm:order_topology_intervals}
  Under the \hyperref[def:order_topology]{order topology} on a \hyperref[def:totally_ordered_set]{totally ordered set}:
  \begin{thmenum}
    \thmitem{thm:order_topology_intervals/open} \hyperref[def:order_interval/unbounded]{Open initial and final segments} and \hyperref[def:order_interval/open]{open intervals} are open sets.

    \thmitem{thm:order_topology_intervals/closed} \hyperref[def:order_interval/unbounded]{Closed initial and final segments} and \hyperref[def:order_interval/closed]{closed intervals} are closed sets.
  \end{thmenum}
\end{proposition}
\begin{proof}
  \SubProofOf{thm:order_topology_intervals/open} Open initial and final segments are members of the subbase \eqref{eq:def:order_topology/subbase}, which makes them open.

  The open interval \( (a, b) \) is a member of the base \eqref{eq:def:order_topology/base}. Hence, it is also open.

  \SubProofOf{thm:order_topology_intervals/closed} The closed ray \( P_{\leq b} \) is the complement of the open ray \( P_{>b} \), which makes it a closed set. Similarly, \( P_{\geq a} \) is the complement of \( P_{<a} \), which also makes it a closed set.

  For the closed interval \( [a, b] \), we have
  \begin{equation*}
    [a, b] = P_{\leq b} \cap P_{\geq a}
  \end{equation*}

  Both \( P_{\leq b} \) and \( P_{\geq a} \) are closed rays, and hence their union is also a closed set. Hence, \( [a, b] \) is closed.
\end{proof}

\begin{example}\label{ex:def:order_topology}
  Examples of \hyperref[def:order_topology]{order topologies} include:
  \begin{itemize}
    \item The order topology on \( \BbbR \), which is equivalent to the \hyperref[def:metric_topology]{metric topology} as shown in \fullref{thm:real_metric_and_order_topologies_coincide}.

    \item All \hyperref[def:ordinal]{ordinals} greater than \( 1 \) induce topological spaces called the \hyperref[def:ordinal_space]{ordinal spaces}.
  \end{itemize}
\end{example}

\begin{definition}\label{def:ordinal_space}\mcite[40]{SteenSeebach1995TopCounterexamples}
  Let \( \alpha \) be an \hyperref[def:ordinal]{ordinal} grater than \( 1 \). When regarded as the set of smaller ordinals, as justified by \fullref{thm:ordinal_is_set_of_smaller_ordinals}, \( \alpha \) is a \hyperref[def:totally_ordered_set]{totally order set} and hence we can endow it with the \hyperref[def:order_topology]{order topology} \( \mscrT \) to obtain a \hyperref[def:topological_space]{topological space}.

  We call the space \( (\alpha, \mscrT) \) an (open) \term{ordinal space}.
\end{definition}

\begin{proposition}\label{thm:limit_ordinal_order_topology}
  In an \hyperref[def:ordinal_space]{ordinal space} \( (\alpha, \mscrT) \), a nonzero ordinal \( \beta \in \alpha \) is a \hyperref[def:successor_and_limit_ordinal]{limit ordinal} if and only if it is a \hyperref[def:set_cluster_point]{cluster point} of \( \alpha \).
\end{proposition}
\begin{proof}
  \SufficiencySubProof Let \( \beta \) be a limit ordinal. We need to show that every neighborhood of \( \beta \) contains points distinct from \( \beta \).

  \Fullref{thm:properties_via_bases/cluster} allows us to only consider neighborhood in a local base at \( \beta \). We will use the base of open intervals containing \( \beta \).
  \begin{itemize}
    \item If \( \beta \) belongs to the initial segment \( \alpha_{<\gamma} \), then every ordinal smaller than \( \beta \) also belongs to \( \alpha_{<\gamma} \), hence the latter is nonempty.

    \item If \( \beta \) belongs to the final segment \( \alpha_{>\gamma} \), then \( \op{sc}(\beta) \) also belongs to this segment, and thus it is nonempty.

    \item If \( \beta \) belongs to the open interval \( (\gamma, \delta) \), this interval must also contain \( \op{sc}(\gamma) \) because \( \beta \) is a limit ordinal and hence \( \gamma < \beta \) implies \( \op{sc}(\gamma) < \beta \).
  \end{itemize}

  Therefore, every neighborhood of this local base at \( \beta \) contains points distinct from \( \beta \), making \( \beta \) a cluster point.

  \NecessitySubProof Let \( \beta \) be a cluster point and let \( \gamma < \beta \). Note that \( (\gamma, \op{sc}(\beta)) \) is an interval containing \( \beta \). The set
  \begin{equation*}
    (\gamma, \op{sc}(\beta)) \setminus \set{ \beta }
    =
    (\gamma, \beta)
  \end{equation*}
  is nonempty, implying that \( \beta \) is not the successor of \( \gamma \). Since \( \gamma \) was arbitrary, we conclude that \( \beta \) is a limit ordinal.
\end{proof}
