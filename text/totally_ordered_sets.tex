\subsection{Totally ordered sets}\label{subsec:totally_ordered_sets}

\begin{definition}\label{def:totally_ordered_set}\mcite[2]{Roman2008}
  We say that a partially ordered set is \term{totally ordered} if either the nonstrict order \( \leq \) is \hyperref[def:binary_relation/total]{total} or if the strict order \( < \) is \hyperref[def:binary_relation/trichotomic]{trichotomic}.

  The theory, homomorphisms and category \( \cat{Tos} \) are inherited from \fullref{def:partially_ordered_set} with one additional axiom added.
\end{definition}
\begin{defproof}
  Equivalence between nonstrict and strict total orders follows directly from the compatibility condition \eqref{eq:def:preordered_set/compatibility_nonstrict}.
\end{defproof}

\begin{proposition}\label{thm:total_order_embedding_iff_strict}
  An \hyperref[def:order_homomorphism/increasing]{order homomorphism} between \hyperref[def:totally_ordered_set]{totally ordered sets} is an \hyperref[def:order_homomorphism/embedding]{order embedding} if and only if it is \hyperref[def:order_homomorphism/increasing]{strict}.
\end{proposition}
\begin{proof}
  \SufficiencySubProof Follows from \fullref{thm:order_embedding_is_strict}.

  \NecessitySubProof
  \SubProof*{Proof that \( f \) is injective} Let \( f: P \to Q \) be a strict order homomorphism and suppose that \( f(x_1) = f(x_2) \) for some \( x_1 \) and \( x_2 \) in \( P \). We will use the \hyperref[def:binary_relation/trichotomic]{trichotomy} of \( < \).
  \begin{itemize}
    \item If \( x_1 < x_2 \), then \( f(x_1) < f(x_2) \) since \( f \) is strictly monotone, which contradicts our assumption \( f(x_1) = f(x_2) \).

    \item If \( x_1 > x_2 \), similarly \( f(x_1) > f(x_2) \) and we again obtain a contradiction.

    \item It remains for \( x_1 \) to be equal to \( x_2 \).
  \end{itemize}

  Since \( x_1 \) and \( x_2 \) were arbitrary, we conclude that \( f \) is injective.

  \SubProof*{Proof that \( f^{-1} \) is order-preserving} Since \( f \) is injective, its inverse \( f^{-1} \) is a single-valued partial function, and is total and injective on \( f(P) \). It then follows from \fullref{thm:order_embedding_is_strict} that the restriction of \( f^{-1} \) to \( f(P) \) is order-preserving.
\end{proof}

\begin{corollary}\label{thm:totally_ordered_strict_isomorphisms}
  An \hyperref[def:order_homomorphism/increasing]{order homomorphism} between \hyperref[def:totally_ordered_set]{totally ordered sets} is an \hyperref[def:order_homomorphism/embedding]{order isomorphism} if and only if it is \hyperref[def:order_homomorphism/increasing]{strict} and \hyperref[def:function_invertibility/surjective]{surjective}.
\end{corollary}
\begin{proof}
  Follows from \fullref{thm:total_order_embedding_iff_strict}.
\end{proof}

\begin{proposition}\label{thm:totally_ordered_minimal_element_is_minimum}
  In a \hyperref[def:totally_ordered_set]{totally ordered set}, every \hyperref[def:extremal_points/maximal_and_minimal_element]{minimal element} is a \hyperref[def:extremal_points/maximum_and_minimum]{minimum}.
\end{proposition}
\begin{proof}
  Let \( x_0 \) be a minimal element of \( (P, \leq) \). By definition of total order, for any \( x \in P \) either \( x \leq x_0 \) or \( x_0 \leq x \). If \( x \leq x_0 \), then since \( x_0 \) is a minimal element, we have \( x = x_0 \).

  Therefore, for any \( x \in P \), either \( x = x_0 \) or \( x > x_0 \). That is, \( x \leq x_0 \) always holds, proving that \( x_0 \) is a minimum.
\end{proof}

\begin{proposition}\label{thm:totally_ordered_segment_isomorphism}
  Let \( (P, \leq) \) be a \hyperref[def:totally_ordered_set]{totally-ordered set}. Let \( Q \) be the set containing the \hyperref[def:order_interval/ray]{strict initial segment} \( P_{<x} \) for every member \( x \) of \( P \).

  Then \( (P, \leq) \) is \hyperref[def:order_homomorphism/isomorphism]{order-isomorphic} to \( (Q, \subseteq) \).
\end{proposition}
\begin{proof}
  Explicitly define the isomorphism
  \begin{equation*}
    \begin{aligned}
      &f: P \to Q \\
      &f(x) \coloneqq P_{<x} = \set{ y \in P \given y < x }.
    \end{aligned}
  \end{equation*}

  Note that \( f \) is \hyperref[def:order_homomorphism/increasing]{strictly order-preserving}. Indeed, if \( x < y \), then \( x \in P_{<y} \), but \( x \not\in P_{<x} \) and hence \( P_{<x} \) is a strict subset of \( P_{<y} \).

  Then \fullref{thm:total_order_embedding_iff_strict} implies that \( f \) is an embedding. Since \( f \) is also surjective by definition of \( Q \), it follows that it is a strict isomorphism between \( (P, \leq) \) and \( (Q, \subseteq) \).
\end{proof}

\begin{lemma}\label{thm:unbounded_totally_ordered_set}
  A totally ordered set is \hyperref[def:extremal_points/upper_and_lower_bounds]{unbounded from above} if and only if, for every element, there exists some strictly larger element.
\end{lemma}
\begin{proof}
  Fix a totally ordered set \( (P, \leq) \).

  Expanding the definition of boundedness, we conclude that
  \begin{displayquote}
    \( P \) is unbounded if and only if there does not exist \( x \in P \) such that, for every \( y \in P \), \( y \leq x \).
  \end{displayquote}

  The consequent of the above statement has a negation in front of its outermost quantifier, hence we can move the negation inwards\footnote{This is justified by \fullref{thm:first_order_quantifiers_are_dual}} to obtain
  \begin{displayquote}
    \( P \) is unbounded if and only if, for every \( x \in P \), there exist \( y \in P \) such that \( y \leq x \) does not hold.
  \end{displayquote}

  Taking into account that \( P \) is totally ordered, \( y \leq x \) does not hold if and only if \( y > x \) holds. Then
  \begin{displayquote}
    \( P \) is unbounded if and only if, for every \( x \in P \), there exist \( y \in P \) such that \( y > x \).
  \end{displayquote}
\end{proof}

\begin{proposition}\label{thm:totally_ordered_cofinal_equivalences}
  Let \( (P, \leq) \) be an \hyperref[def:extremal_points/upper_and_lower_bounds]{unbounded from above} totally ordered set and let \( A \subseteq P \). Then \( A \) is \hyperref[def:cofinal_set]{cofinal} if and only if it is itself unbounded from above.
\end{proposition}
\begin{comments}
  \item There is a somewhat similar result in \fullref{thm:partially_ordered_cofinal_equivalences}.
  \item This equivalence is useful for \hyperref[def:regular_cardinal]{regular cardinals} --- for example \fullref{thm:cardinal_cofinality}.
\end{comments}
\begin{proof}
  We will prove the converse - a set is bounded if and only if it is not cofinal.

  \SufficiencySubProof Let \( A \) be bounded from above by \( x \). \Fullref{thm:unbounded_totally_ordered_set} then implies that there exists some \( y \) such that \( x < y \). But no element of \( A \) is greater than \( y \), implying that \( A \) is not cofinal.

  \NecessitySubProof Suppose that \( A \) is not cofinal. Let \( x \in P \). Then, for every \( y \in A \), \( x < y \) does not hold. That is, \( y \leq x \) for every \( y \in A \). Therefore, \( x \) is bounded.
\end{proof}

\begin{proposition}\label{thm:total_lexicographic_order_is_total_order}
  If \( (P, \leq_P) \) and \( (Q, \leq_Q) \) are \hyperref[def:totally_ordered_set]{totally ordered sets}, then the \hyperref[eq:def:lexicographic_order]{lexicographic} and \hyperref[eq:def:lexicographic_order/reverse]{reverse lexicographic} orders on \( P \times Q \) are \hyperref[def:totally_ordered_set]{strict total order} relations.
\end{proposition}
\begin{comments}
  \item An analogous result holds for partial orders (\fullref{thm:partial_lexicographic_order_is_partial_order}) and well-ordered sets (\fullref{thm:well_ordered_lexicographic_order_is_well_ordered}).
\end{comments}
\begin{proof}
  We have already shown in \fullref{thm:lexicographic_order_is_partial_order} and these are partial orders. It only remains to check trichotomy.

  \SubProofOf[def:binary_relation/trichotomic]{trichotomy} Let \( \prec \) be the lexicographic order on \( P \times Q \). Let \( (a, b) \) and \( (c, d) \) be pairs in \( P \times Q \). Since \( <_P \) and \( <_Q \) are strict total orders, we only have the following possibilities:
  \begin{itemize}
    \item If \( a = c \) and \( b = d \), then \( (a, b) = (c, d) \).
    \item If \( a = c \) and \( b <_Q d \), then \( (a, b) \prec (c, d) \).
    \item If \( a = c \) and \( b >_Q d \), then \( (a, b) \succ (c, d) \).
    \item If \( a <_P c \), then \( (a, b) \prec (c, d) \).
    \item If \( a >_P c \), then \( (a, b) \succ (c, d) \).
  \end{itemize}

  The proof for the reverse lexicographic order is analogous.
\end{proof}

\begin{proposition}\label{thm:finite_totally_ordered_set_is_well_ordered}
  Every finite \hyperref[def:totally_ordered_set]{totally ordered set} is \hyperref[def:well_ordered_set]{well-ordered}.
\end{proposition}
\begin{proof}
  Let \( (P, \leq) \) be totally ordered set of finite cardinality \( n \). Let \( A \) be any nonempty subset of \( P \). We will show by induction on \( \card(A) \) that \( A \) has a minimum.

  \begin{itemize}
    \item If \( A = \set{ a } \), then \( a \) is vacuously a minimum.
    \item Suppose that all subsets of \( P \) of cardinality \( k - 1 \) have a minimum. Let \( \card(A) = k \). Pick an arbitrary element \( a \in A \). By the inductive hypothesis, \( A \setminus \set{ a } \) has a minimum, say \( b \). Then \( \min\set{ a, b } \) is a minimum of \( A \).
  \end{itemize}
\end{proof}

\begin{definition}\label{def:order_topology}\mcite[exer. 1.7.4]{Engelking1989}
  Let \( P \) be a \hyperref[def:partially_ordered_set]{totally ordered set} with more than one element. The \term{order topology} induced by \( \leq \) is the topology generated by the \hyperref[def:topological_subbase]{subbase} of open \hyperref[def:order_interval/ray]{rays}
  \begin{equation}\label{eq:def:order_topology/subbase}
    \mscrS \coloneqq \set[\Big]{ (a, \infty) \given a \in P } \cup \set[\Big]{ (-\infty, b) \given b \in P }.
  \end{equation}

  The \hyperref[def:topological_base]{base} corresponding to this subbase is
  \begin{equation}\label{eq:def:order_topology/base}
    \mscrB = \mscrS \cup \set[\Big]{ \varnothing } \cup \set[\Big]{ (a, b) \given a, b \in P \T{and} a < b }.
  \end{equation}

  See the proof of \hyperref[thm:topology_from_base/B1]{B1} for why \( P \) must have more than one element.
\end{definition}
\begin{defproof}
  \SubProof{Proof of compatibility of \( \mscrS \) and \( \mscrB \)} Define
  \begin{equation*}
    \mscrC = \set*{ \bigcap \mscrS \given* \mscrS \T{is a nonempty finite subset of} \mscrS }.
  \end{equation*}

  We will show that \( \mscrB = \mscrC \).

  Let \( B \in \mscrB \). The cases \( B \in \mscrS \) and \( B = \varnothing \) are trivial. Suppose that \( B \not\in \mscrS \). Then there exist points \( a < b \) such that
  \begin{equation*}
    B = (a, b) = (-\infty, b) \cap (a, \infty).
  \end{equation*}

  This is an intersection of members of \( \mscrS \), hence \( B \in \mscrC \). Therefore, \( \mscrB \subseteq \mscrC \).

  Now let \( C = S_1 \cap \cdots \cap S_n \), where \( S_1, \ldots, S_n \) are members of \( \mscrS \). We will show by induction on \( n > 0 \) that \( C \in \mscrB \). The case \( n = 1 \) is trivial. Suppose that all \( n \)-ary intersections belong to \( \mscrB \) and let
  \begin{equation*}
    C = S_1 \cap \cdots \cap S_n \cap S_{n+1}.
  \end{equation*}

  By the inductive hypothesis, we have that \( D \coloneqq S_1 \cap \cdots \cap S_n \) belongs to \( \mscrB \) and thus we have three cases:
  \begin{itemize}
    \item If either \( D = (a, \infty) \) or \( D = (-\infty, b) \), then \( D \in \mscrS \).
    \item If \( D = \varnothing \), then \( D = (-\infty, a) \cap (a, \infty) \) for some \( a \in P \).
    \item If \( D = (a, b) \), then \( D = (-\infty, b) \cap (a, \infty) \).
  \end{itemize}

  In all cases both \( D \) and \( C = D \cap S_{n+1} \) are finite intersection of members of \( \mscrS \). Therefore, \( \mscrC \subseteq \mscrB \). Since we already have the inclusion in the other direction, we conclude that \( \mscrC = \mscrB \).

  \SubProof{Proof that \( \mscrB \) is a base} We will show that the axioms in \fullref{thm:topology_from_base} hold.

  \SubProofOf*[thm:topology_from_base/B1]{B1} Let \( x \in P \).

  If \( x \) is a \hyperref[def:extremal_points/maximum_and_minimum]{maximum}, then take any other value \( y < x \) and the set \( (y, \infty) \) will contain \( x \). We use here that there is more than one element in \( P \).

  If \( x \) is not a maximum, then \( x \) belongs to any interval \( (-\infty, y) \) whenever \( y > x \).

  In both cases there exists an interval in \( S \) containing \( x \). Thus, \( \bigcup S = P \).

  \SubProofOf*[thm:topology_from_base/B2]{B2} Let \( U \) and \( V \) be members of \( \mscrB \). We consider \( 14 \) cases:
  \begin{itemize}
    \item If either \( U = \varnothing \) or \( V = \varnothing \), then \( U \cap V = \varnothing \).
    \item If \( U = (-\infty, u) \) and \( V = (v, \infty) \), then
    \begin{itemize}
      \item If \( u \leq v \), then \( U \cap V = \varnothing \).
      \item If \( v < u \), then \( U \cap V = (v, u) \).
    \end{itemize}

    \item If \( U = (-\infty, u) \) and \( V = (v_1, v_2) \), then
    \begin{itemize}
      \item If \( u \leq v_1 \), then \( U \cap V = \varnothing \).
      \item If \( v_1 < v_2 \leq u \), then \( U \cap V = V =  (v_1, v_2) \).
      \item If \( v_1 \leq u < v_2 \), then \( U \cap V = (v_1, u) \).
    \end{itemize}

    \item If \( U = (u_1, u_2) \) and \( V = (v, \infty) \), then
    \begin{itemize}
      \item If \( u_2 \leq v \), then \( U \cap V = \varnothing \).
      \item If \( u_1 \leq v < u_2 \), then \( U \cap V = (v, u_2) \).
      \item If \( v \leq u_1 < u_2 \), then \( U \cap V = U = (u_1, v_1) \).
    \end{itemize}

    \item If \( U = (u_1, u_2) \) and \( V = (v_1, v_2) \), then
    \begin{itemize}
      \item If \( u_2 < v_1 \), then \( U \cap V = \varnothing \).
      \item If \( u_1 < v_1 < u_2 < v_2 \) then \( U \cap V = (v_1, u_2) \).
      \item If \( u_1 < v_1 < v_2 < u_2 \) then \( U \cap V = V = (v_1, v_2) \).
      \item If \( v_1 < u_1 < u_2 < v_2 \) then \( U \cap V = U = (u_1, u_2) \).
      \item If \( v_1 < u_1 < v_2 < u_2 \) then \( U \cap V = (u_1, v_2) \).
    \end{itemize}
  \end{itemize}

  In all cases, the intersection \( U \cap V \) belongs to \( \mscrB \).
\end{defproof}

\begin{proposition}\label{thm:order_topology_intervals}
  Under the \hyperref[def:order_topology]{order topology} \( \mscrT \) on a \hyperref[def:totally_ordered_set]{totally ordered set} \( (P, \leq) \):
  \begin{thmenum}
    \thmitem{thm:order_topology_intervals/open} \hyperref[def:order_interval/ray]{Open rays} and \hyperref[def:order_interval/open]{open intervals} are open sets.

    \thmitem{thm:order_topology_intervals/closed} \hyperref[def:order_interval/ray]{Closed rays} and \hyperref[def:order_interval/closed]{closed intervals} are closed sets.
  \end{thmenum}
\end{proposition}
\begin{proof}
  \SubProofOf{thm:order_topology_intervals/open} Open rays are members of the subbase \eqref{eq:def:order_topology/subbase}, which makes them open.

  The open interval \( (a, b) \) is a members of the base \eqref{eq:def:order_topology/base}. Hence, it is also open.

  \SubProofOf{thm:order_topology_intervals/closed} The closed ray \( (-\infty, b] \) is the complement of the open ray \( (a, \infty) \), which makes it a closed set. Similarly, \( [a, \infty) \) is the complement of \( (-\infty, b) \), making it a closed set.

  For the closed interval \( [a, b] \), we have
  \begin{equation*}
    [a, b] = (-\infty, b] \cap [a, \infty).
  \end{equation*}

  Both \( (-\infty, b] \) and \( [a, \infty) \) are closed rays, and hence their union is also a closed set. Hence, \( [a, b] \) is closed.
\end{proof}

\begin{example}\label{ex:def:order_topology}
  Examples of \hyperref[def:order_topology]{order topologies} include:
  \begin{itemize}
    \item The order topology on \( \BbbR \), which is equivalent to the \hyperref[def:metric_topology]{metric topology} as shown in \fullref{thm:real_metric_and_order_topologies_coincide}.

    \item All \hyperref[def:ordinal]{ordinals} greater than \( 1 \) induce topological spaces called the \hyperref[def:ordinal_space]{ordinal spaces}.
  \end{itemize}
\end{example}

\begin{definition}\label{def:ordinal_space}
  Let \( \alpha \) be an \hyperref[def:ordinal]{ordinal}. When regarded as the set of smaller ordinals, as shown valid in \fullref{thm:ordinal_is_set_of_smaller_ordinals}, \( \alpha \) is a \hyperref[def:totally_ordered_set]{totally order set} and hence we can endow it with the \hyperref[def:order_topology]{order topology} \( \mscrT \) to obtain a \hyperref[def:topological_space]{topological space}. We call the space \( (\alpha, \mscrT) \) an \term{ordinal space}.
\end{definition}

\begin{proposition}\label{thm:limit_ordinal_order_topology}
  In an \hyperref[def:ordinal_space]{ordinal space} \( (\alpha, \mscrT) \), a nonzero ordinal \( \beta \in \alpha \) is a \hyperref[def:successor_and_limit_ordinal]{limit ordinal} if and only if it is the \hyperref[def:cluster_point]{cluster point} of \( \alpha \).
\end{proposition}
\begin{proof}
  \SufficiencySubProof Let \( \beta \) be a limit ordinal. We need to show that every neighborhood of \( \beta \) contains points distinct from \( \beta \).

  \Fullref{thm:properties_via_bases/cluster} allows us to only consider neighborhood in a local base at \( \beta \). We will use the base of open intervals containing \( \beta \).
  \begin{itemize}
    \item For \( \gamma < \beta < \delta \), the interval \( (\gamma, \delta) \) must contain \( \op{succ}(\gamma) \) because, as a limit ordinal, \( \beta \) contains the successors of all smaller ordinals.

    \item The same proof applies to the interval \( (\gamma, \infty) \).

    \item For \( \beta < \delta \), the interval \( (-\infty, \delta) \) contains all predecessors of \( \beta \). It is important here that \( \beta \) is nonzero, which we implicitly assume for limit ordinals.
  \end{itemize}

  Therefore, every neighborhood of the local base at \( \beta \) contains points distinct from \( \beta \), making \( \beta \) a cluster point.

  \NecessitySubProof Let \( \beta \) be a cluster point and let \( \gamma < \beta \). Then the set
  \begin{equation*}
    (\gamma, \op{succ}(\beta)) \setminus \set{ \beta }
    =
    (\gamma, \beta)
  \end{equation*}
  is nonempty, implying that \( \beta \) is not the successor of \( \gamma \). Since this holds for all smaller ordinals, we conclude that \( \beta \) is a limit ordinal.
\end{proof}
