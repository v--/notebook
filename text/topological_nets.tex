\subsection{Topological nets}\label{subsec:topological_nets}

In this section, \( X \) will denote an arbitrary nonempty topological space.

\begin{definition}\label{def:topological_net}
  A \term{net} or \term{generalized \hyperref[def:sequence]{sequence}} or \term{Moore-Smith sequence} in a nonempty set \( S \) is a family of elements of \( S \) indexed by a nonempty \hyperref[def:cartesian_product/indexed_family]{directed set}, i.e. a function from a nonempty directed \hyperref[def:directed_set]{set} \( (\mscrK, \leq) \) to \( S \). We use the conventional notation for indexed families:
  \begin{equation*}
    \{ x_k \}_{k \in \mscrK},
  \end{equation*}
  because the preorder on the domain \( \mscrK \) is usually clear from the context.

  If we know that the net is a sequence, we will usually use the notation for sequences given in \fullref{def:sequence}.

  Note that this definition does not actually require a topology on \( S \). Some other important definitions also do not require topologies:
  \begin{thmenum}
    \thmitem{def:topological_net/frequently_in} We say that \( \{ x_k \}_{k \in \mscrK} \) is \term{frequently in} the set \( A \subseteq S \) if for every index \( k_0 \in \mscrK \) there exists an index \( k \geq k_0 \) such that \( x_k \in A \).

    \thmitem{def:topological_net/eventually_in} We say that \( \{ x_k \}_{k \in \mscrK} \) is \term{eventually in} the set \( A \subseteq S \) if there exists an index \( k_0 \) such that \( x_k \in A \) whenever \( k \geq k_0 \). This is obviously a stronger condition.

    \thmitem{def:topological_net/subnet}\mcite[50]{Engelking1989}We say that the net \( \{ y_m \}_{m \in M} \subseteq S \) is a \term{subnet} of \( \{ x_k \}_{k \in \mscrK} \subseteq S \) if there exists an embedding function \( \varphi: M \to \mscrK \) such that
    \begin{thmenum}
      \thmitem{def:topological_net/subnet/directed} To every \( k \in \mscrK \) there corresponds \( m \in M \) such that \( \varphi(m) \geq k \).
      \thmitem{def:topological_net/subnet/identity} For every \( m \in M \) we have \( x_{\varphi(m)} = y_m \).
    \end{thmenum}
  \end{thmenum}
\end{definition}

\begin{proposition}\label{thm:def:topological_net}
  \hyperref[def:topological_net]{Nets} have the following basic properties:

  \begin{thmenum}
    \thmitem{thm:def:topological_net/eventually_in_implies_frequently_in} \enquote{\hyperref[def:topological_net/eventually_in]{Eventually in}} implies \enquote{\hyperref[def:topological_net/frequently_in]{frequently in}}.

    \thmitem{thm:def:topological_net/net_eventually_in_iff_not_frequently_in_complement} The net \( \{ x_k \}_{k \in \mscrK} \subseteq S \) is eventually in \( A \subseteq S \) if and only if it is not frequently in \( S \setminus A \).
  \end{thmenum}
\end{proposition}
\begin{proof}
  \SubProofOf{thm:def:topological_net/eventually_in_implies_frequently_in} Suppose that the net \( \{ x_k \}_{k \in \mscrK} \subseteq S \) is eventually in \( A \subseteq S \). Then there exists an index \( k_0 \) such that \( x_k \in A \) for all \( k \geq k_0 \).

  Given any index \( k_1 \), we choose \( k_2 \) such that \( k_2 \geq k_0 \) and \( k_2 \geq k_1 \) (this is possible by the definition of a directed set). Then \( x_{k_2} \in A \) and \( k_2 \) satisfies the existence quantifier in \fullref{def:topological_net/frequently_in}.

  \SubProofOf{thm:def:topological_net/net_eventually_in_iff_not_frequently_in_complement} Suppose that \( \{ x_k \}_{k \in \mscrK} \) is both eventually in \( A \) and frequently in \( S \setminus A \).

  Since the net is eventually in \( A \), we can fix an index \( k_0 \) such that \( x_k \in A \) whenever \( k \geq k_0 \).

  Since the net is frequently in \( A \), we can fix an index \( k_1 \geq k_0 \) such that \( k_1 \in S \setminus A \), which is a contradiction.

  This proves that the two conditions are incompatible.
\end{proof}

\begin{definition}\label{def:net_convergence}
  Let \( X \) be a topological space and \( \{ x_k \}_{k \in \mscrK} \subseteq X \) be a net.

  \begin{thmenum}
    \thmitem{def:net_convergence/cluster} If the net is frequently in every neighborhood of \( x_0 \in X \), we say that \( x_0 \) is a \term{cluster point} or an \term{accumulation point} of \( \{ x_k \}_{k \in \mscrK} \subseteq X \).

    \thmitem{def:net_convergence/limit} If the net is eventually in every neighborhood of \( x_0 \in X \), we say that \( x_0 \) is a \term{limit point} of \( \{ x_k \}_{k \in \mscrK} \subseteq X \).
  \end{thmenum}

  In general, there can exist multiple limit points (see \fullref{ex:multiple_limit_points_of_net}) and even more cluster points (see \fullref{ex:cluster_points/sine}). In Hausdorff spaces, however, limits are unique by \fullref{thm:t2_iff_singleton_limits}.

  If \( \{ x_k \}_{k \in \mscrK} \subseteq X \) has a unique limit, we say that the net \term[bg=схожда,ru=сходится]{converges} to \( x_0 \) use the notation
  \begin{equation*}
    x_0 = \lim_{k \in \mscrK} x_k.
  \end{equation*}

  If the net is a \hyperref[def:sequence]{sequence}, we also use the following notations:
  \begin{itemize}
    \item \( x_0 = \lim_{k \to \infty} x_k \)
    \item \( x_0 = \lim x_k \)
    \item \( x_k \xrightarrow[k \to \infty]{} x_0 \)
    \item \( x_k \to x_0 \)
  \end{itemize}
\end{definition}

\begin{example}\label{ex:multiple_limit_points_of_net}
  Even limits of sequences need not be unique in arbitrary topological spaces. Let \( X = \{ y, z \} \) be a binary set with the indiscrete \hyperref[def:standard_topologies/indiscrete]{topology} \( \{ \varnothing, X \} \). L

  Define the following \hyperref[def:sequence]{sequence}
  \begin{balign*}
    x_k \coloneqq \begin{cases}
      y, & k \text{ is even}, \\
      z, & k \text{ is odd}.
    \end{cases}
  \end{balign*}

  The only neighborhood of \( y \), the whole space \( X \), contains all members of the sequence, therefore \( y \) is a limit point of the sequence. The same is true for \( z \), however.
\end{example}

\begin{example}\label{ex:cluster_points/sine}
  Consider the net \( \{ \sin(k) \}_{k \in \BbbR} \). It has no limit point, yet every real number in the interval \( [-1, 1] \) is a cluster point.
\end{example}

\begin{example}\label{ex:reverse_inclusion_net}
  A commonly used technique is to use a variation of a \term{reverse set inclusion net}.

  Fix an element \( x_0 \in X \) of any topological space and choose an element \( x_U \) our of every neighborhood \( U \) of \( x_0 \). Consider the directed set \( (\mscrT(x), \leq) \) consisting of all neighborhoods of \( x_0 \) ordered by \term{reverse inclusion}, i.e. \( U \leq V \iff U \supseteq V \).

  Choose an element \( x_U \) from each neighborhood \( U \) of \( x_0 \). Then, by construction, \( x_0 \) is a limit point of the net \( \{ x_U \}_{U \in T(x_0)} \).
\end{example}

\begin{proposition}\label{thm:def:net_convergence}
  Convergence of \hyperref[def:net_convergence]{nets} has the following basic properties:

  \begin{thmenum}
    \thmitem{thm:def:net_convergence/sequence_converges_iff_almost_entirely_in_neighborhood} The point \( x_0 \in X \) is a limit point of the sequence \( \{ x_k \}_{k=1}^\infty \subseteq X \) if and only if, given a neighborhood \( U \) of \( x_0 \), only finitely many elements of the sequence are outside \( U \).

    \thmitem{thm:def:net_convergence/limit_point_is_cluster_point} Every limit point is a cluster point.

    \thmitem{thm:def:net_convergence/cluster_point_iff_subnet_limit_point} A point \( x_0 \in X \) is a cluster point of the net \( \{ x_k \}_{k \in \mscrK} \subseteq X \) if and only if \( x_0 \) is a limit point of some subnet.

    \thmitem{thm:def:net_convergence/limit_implies_no_proper_cluster_points} If a net has a limit point, all of its cluster points are limit points.

    \thmitem{thm:def:net_convergence/unique_limit_point_iff_unique_cluster_point} A net has a unique limit point if and only if has a unique cluster point.

    \thmitem{thm:def:net_convergence/unique_limit_point_iff_subnets_have_same_limit} A net has a unique limit point if and only if all subnets have the same limit point.
  \end{thmenum}
\end{proposition}
\begin{proof}
  \SubProofOf{thm:def:net_convergence/sequence_converges_iff_almost_entirely_in_neighborhood} This is simply a restatement of \fullref{def:net_convergence/limit} for the special case of sequences.

  \SubProofOf{thm:def:net_convergence/limit_point_is_cluster_point} Follows from \fullref{thm:def:topological_net/eventually_in_implies_frequently_in}.

  \SubProofOf{thm:def:net_convergence/cluster_point_iff_subnet_limit_point}
  The definition of a cluster point (\fullref{def:net_convergence/cluster}) allows us to build a reverse inclusion net in the style of \fullref{ex:reverse_inclusion_net}.

  \SubProofOf{thm:def:topological_net/eventually_in_implies_frequently_in} Suppose that the net \( \{ x_k \}_{k \in \mscrK} \subseteq S \) is eventually in \( A \subseteq S \). Then there exists an index \( k_0 \) such that \( x_k \in A \) for all \( k \geq k_0 \).

  Given any index \( k_1 \), we choose \( k_2 \) such that \( k_2 \geq k_0 \) and \( k_2 \geq k_1 \) (this is possible by the definition of a directed set). Then \( x_{k_2} \in A \) and \( k_2 \) satisfies the existence quantifier in \fullref{def:topological_net/frequently_in}.

  \SubProofOf{thm:def:topological_net/net_eventually_in_iff_not_frequently_in_complement} Suppose that \( \{ x_k \}_{k \in \mscrK} \) is both eventually in \( A \) and frequently in \( S \setminus A \).

  Since the net is eventually in \( A \), we can fix an index \( k_0 \) such that \( x_k \in A \) whenever \( k \geq k_0 \).

  Since the net is frequently in \( A \), we can fix an index \( k_1 \geq k_0 \) such that \( k_1 \in S \setminus A \), which is a contradiction.

  This proves that the two conditions are incompatible.
\end{proof}

\begin{proposition}\label{thm:limit_point_iff_in_closure}\mcite[prop. 1.6.3]{Engelking1989}
  Fix a set \( A \subseteq X \). A point \( x_0 \in X \) belongs to \( \cl{A} \) if and only if there exists a net \( \{ x_k \}_{k \in \mscrK} \subseteq A \) for which \( x_0 \) is a limit point.

  By \fullref{thm:def:net_convergence/cluster_point_iff_subnet_limit_point}, we can consider cluster points of nets rather than limit points.
\end{proposition}
\begin{proof}
  The complement of the empty set is the empty set, hence the statement of the proposition holds vacuously. Assume that \( A \) is nonempty.

  \SufficiencySubProof Suppose that \( x_0 \in \cl{A} \). If \( x_0 \in A \), then the one-element net \( (x_0) \) converges to \( x_0 \).

  If \( x_0 \in \fr{A} \), by \fullref{def:topological_boundary_operator/neighborhoods}, every neighborhood of \( x_0 \) contains points from \( A \). Therefore, we can build reverse inclusion net in the style of \fullref{ex:reverse_inclusion_net} that converges to \( x_0 \).

  \NecessitySubProof Let \( x_0 \) be a limit point of \( \{ x_k \}_{k \in \mscrK} \subseteq A \). We will show that \( x_0 \) belongs every closed set that contains \( A \).

  Let \( F \supseteq A \) be a closed set. Denote \( U \coloneqq X \setminus F \). Suppose that \( x_0 \in U \). Then \( U \) is a neighborhood \( x_0 \) and, by \fullref{def:net_convergence/cluster}, the net \( \{ x_k \}_{k \in \mscrK} \subseteq A \) is eventually in \( U \). But \( U \) does not contains \( A \).

  The obtained contradiction shows that \( x_0 \) belongs to every closed set containing \( A \) and hence to their intersection, the closure \( \cl A \).
\end{proof}

\begin{proposition}\label{thm:cluster_point_of_set_iff_limit_point_of_net}
  The point \( x_0 \in X \) is a cluster \hyperref[def:cluster_point/cluster_point]{point} of the set \( A \) if and only if it is a limit \hyperref[def:net_convergence/cluster]{point} of some net in \( A \setminus \{ x_0 \} \) (or, equivalently, a cluster point of some net in \( A \setminus \{ x_0 \} \)).
\end{proposition}
\begin{proof}
  \SufficiencySubProof Let \( x_0 \in \drv(A) \). By \fullref{thm:def:derived_set/cluster_via_neighborhoods}, every neighborhood \( U \) of \( x_0 \) intersects \( A \setminus \{ x_0 \} \). Choose \( x_U \in U \cap (A \setminus \{ x_0 \}) \) for every neighborhood \( U \) of \( x_0 \) and form the reverse inclusion \hyperref[ex:reverse_inclusion_net]{net} \( \{ x_U \}_{U \in T(x)} \). Then \( x_0 \) is a limit point of this net. Furthermore, the net is contained in \( A \setminus \{ x_0 \} \).

  \NecessitySubProof Conversely, if \( \{ x_k \}_{k \in \mscrK} \subseteq A \setminus \{ x_0 \} \) is a net and if \( x_0 \) is a limit point of this net, then for every neighborhood \( U \) of \( x_0 \) there exists an index \( k_U \) such that for \( k \geq k_U \) we have \(  x_k \in U \). In particular, \( U \cap A \) contains elements other than \( x_0 \). Since this is true for any neighborhood \( U \) of \( x_0 \), by \fullref{thm:def:derived_set/cluster_via_neighborhoods} we conclude that \( x_0 \) is a cluster point of the set\( A \).
\end{proof}

\begin{corollary}\label{thm:closed_iff_contains_all_net_cluster_points}
  A set is closed if and only if it contains the limit points of all of its nets (or, equivalently, the cluster points of all of its nets).
\end{corollary}
\begin{proof}
  By \fullref{thm:def:derived_set/closed_iff_contains_all_cluster_points}, the set \( A \) is closed if and only if it contains all of its cluster points. By \fullref{thm:cluster_point_of_set_iff_limit_point_of_net}, this is equivalent to \( A \) containing all limit points of its nets.
\end{proof}

\begin{proposition}\label{thm:net_convergence_via_subbases}
  Fix a topological space \( X \), a point \( x_0 \) and a local \hyperref[def:topological_local_subbase]{subbase} \( P(x_0) \). The point \( x_0 \) is a limit of the net \( \{ x_k \}_{k \in \mscrK} \subseteq X \) if and only if it is eventually in every element \( U_P \) of the local subbase \( P(x_0) \).
\end{proposition}
\begin{proof}
  \SufficiencySubProof Obvious consequence of the definition of local subbase.
  \NecessitySubProof Fix a neighborhood \( U \) of \( x_0 \). By \fullref{def:topological_local_subbase}, there exists a finite family \( \{ U_k \}_{k=1}^n \subseteq P(x_0) \) such that \( \bigcap_{k=1}^n U_k \subseteq U \). Since the net \( \{ x_k \}_{k \in \mscrK} \subseteq X \) is eventually in each of \( U_k, k = 1, \ldots, n \), from transitivity of inclusion it follows that the net is eventually in \( U \).
\end{proof}

\begin{definition}\label{def:sequential_topological_closure_operator}
  In analogy to \fullref{def:topological_closure_operator}, we define the \term{sequential closure operator}
  \begin{balign*}
     & \cl^S: \pow(X) \to \pow(X)                                                                                                          \\
     & \cl^S(A) \coloneqq \left\{ x \in X \colon x \text{ is a limit point of some sequence } \{ x_k \}_{k=1}^\infty \subseteq A \right\}.
  \end{balign*}

  If \( \cl^S(A) = A \), we say that \( A \) is \term{sequentially closed}.
\end{definition}

\begin{definition}\label{def:sequential_space}
  A topological space is called \term{sequential} if every sequentially \hyperref[def:sequential_topological_closure_operator]{closed} set is closed.
\end{definition}

\begin{remark}\label{rem:sequential_spaces}
  By \fullref{thm:limit_point_iff_in_closure}, in a \hyperref[def:sequential_space]{sequential space}, a set is closed if and only if it is sequentially closed.

  By \fullref{thm:closed_iff_contains_all_net_cluster_points}, a set is closed if and only if it contains the limit points of all of its nets.

  Therefore, a set in a sequential space is closed if and only if it contains the limit points of all of its sequences.

  Since we are able to define a topology in terms of closed sets, this means that the topology in a sequential space is completely determined by convergent sequences rather than convergent nets as in general topological spaces.

  This allows us to restrict ourselves only to sequences rather than arbitrary nets in certain spaces like \hyperref[def:metric_space]{metric spaces}.
\end{remark}

\begin{lemma}\label{thm:sequential_space_convergence}
  Let \( X \) be a sequential space and \( x_0 \) limit point of the net \( \{ x_k \}_{k \in \mscrK} \), then we can define a sequence
  \begin{equation*}
    \{ x_k \}_{k=1}^\infty \subseteq \{ x_k \colon k \in \mscrK \},
  \end{equation*}
  consisting of members of the net, for which \( x_0 \) is a limit point.
\end{lemma}
\begin{proof}
  Let \( X \) be a first-countable space and let \( x_0 \) be a limit point of the net \( \{ x_k \}_{k \in \mscrK} \).

  Since \( X \) is a first countable space, we can fix a countable local \hyperref[def:topological_local_base]{base} \( \{ U_k \}_{k=1}^\infty \) at \( x_0 \). For each \( k = 1, 2, \ldots \), define the neighborhood \( V_k \coloneqq \bigcap_{m=1}^k U_m \), so that \( V_k \subseteq V_m \) whenever \( k \geq m \).

  For each neighborhood \( V_k \), since \( \{ x_k \}_{k \in \mscrK} \) is eventually in \( V_k \), there exists an index \( k_k \) such that \( x_{k_k} \in V_k \).

  Thus, we obtain a sequence \( \{ x_{k_k} \}_{k=1}^\infty \) that is eventually in every neighborhood of the local base \( \{ V_k \}_{k=1}^\infty \) of \( x_0 \), which by \fullref{thm:net_convergence_via_subbases} is sufficient for \( x_0 \) to be a limit point of the sequence.
\end{proof}

\begin{proposition}\label{thm:first_countable_spaces_are_sequential}
  Every first-countable space is sequential.
\end{proposition}
\begin{proof}
  Let \( X \) be a first-countable space and let \( A \subseteq X \) be a sequentially \hyperref[def:sequential_topological_closure_operator]{closed} set. We must show that it is closed.

  Fix a point \( x_0 \in \cl(A) \). We will show that \( x_0 \in A \). By \fullref{thm:limit_point_iff_in_closure}, there is a net \( \{ x_k \}_{k \in \mscrK} \subseteq A \) for which \( x_0 \) is a limit point.

  By \fullref{thm:sequential_space_convergence}, we can choose a sequence \( \{ x_k \}_{k=1}^\infty \) that converges to \( x_0 \) out of elements of the net. But since \( X \) is a sequential space, the limit points of any sequence are contained in the sequentially closed set \( A \).

  We showed that \( A = \cl(A) \). Since \( A \) was an arbitrary sequentially closed set, we conclude that the space \( X \) is sequential.
\end{proof}
