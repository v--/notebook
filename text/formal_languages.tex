\subsection{Formal languages}\label{subsec:formal_languages}

\paragraph{Languages}

Languages are used to define formulas for expressing the \hyperref[def:zfc]{axioms of set theory}. Here, sets are used to formally define languages. A simple way out of this vicious cycle is via the theory-metatheory relationship discussed in \fullref{rem:metalogic} and \fullref{rem:set_definition_recursion}. In short, we define languages within the metatheory using the already available concept of set, and we later define formulas, again in the metatheory, which allows us to subsequently formally define sets via axioms within the object logic.

\begin{definition}\label{def:formal_language}\mcite[3; 4]{Salomaa1987}
  Fix a nonempty set \( \mscrA \), which we will call an \term[ru=алфавит (\cite[19]{Гладкий1973})]{alphabet}. Unless explicitly noted otherwise, like in \fullref{subsec:free_groups}, we will assume that \( \mscrA \) is finite.

  \begin{thmenum}
    \thmitem{def:formal_language/symbol} We call each element of \( \mscrA \) a \term[ru=символ (\cite[19]{Гладкий1973})]{symbol}.

    \thmitem{def:formal_language/word} We call a \hyperref[def:sequence]{finite sequence} of symbols a \term[ru=слово (\cite[19]{Гладкий1973})]{word} or \term[ru=цепочка (\cite[19]{Гладкий1973})]{string}. If \( (a, b, c) \) is a word, for convenience we use the notation \( abc \). This notation only makes sense if each symbol of the language is actually represented by one typographic symbol.

    \thmitem{def:formal_language/empty_word} We denote the empty word via \( \varepsilon \)\fnote{This notation is used, for example, by \incite[9]{Savage1998}. \incite[3]{Salomaa1987} uses \( \lambda \) instead, while \incite[19]{Гладкий1973} uses \( \Lambda \).}.

    \thmitem{def:formal_language/word_length} We define the \term[ru=длина (\cite[19]{Гладкий1973})]{length} \( \len(w) \) of a word \( w \) as the number of elements of the corresponding tuple.

    \thmitem{def:formal_language/concatenation} We define \term[ru=конкатенация (\cite[19]{Гладкий1973})]{concatenation} of the words \( v = (v_1, \ldots, v_n) \) and \( w = (w_1, \ldots, w_m) \) as the word
    \begin{equation*}
      v \cdot w \coloneqq (v_1, \ldots, v_n, w_1, \ldots, w_m).
    \end{equation*}

    We abbreviate \( \smash{\overbrace{w \ldots w}^{k \T*{times}}} \) as \( w^k \).\fnote{This is only a notational shortcut within the metalogic. We do not distinguish, formally, between the words \( aaabbaa \) and \( a^3 b^2 a^2 \), nor between \( a \varepsilon b \) and \( ab \).}

    \thmitem{def:formal_language/reverse}\mimprovised We define the \term{reverse word} of \( w = (w_1, \ldots, w_n) \) as
    \begin{equation*}
      \op{rev}(w) \coloneqq (w_n, \ldots, w_1).
    \end{equation*}

    \thmitem{def:formal_language/prefix}\mimprovised We say that word \( p = (p_1, \ldots, p_m) \) is a \term[ru=начало (\cite[20]{Гладкий1973})]{prefix} of \( w = (w_1, \ldots, w_n) \) if
    \begin{equation*}
      w = (\underbrace{p_1, \ldots, p_m}_p, w_{m+1}, \ldots, w_n).
    \end{equation*}

    \thmitem{def:formal_language/suffix}\mimprovised We say that the word \( s \) is a \term[ru=конец (\cite[20]{Гладкий1973})]{suffix} of \( w \) if \( \op{rev}(s) \) is a prefix of \( \op{rev}(w) \).

    \thmitem{def:formal_language/subword} We say that the word \( v \) is a \term{subword} of \( w \) if there exists a prefix \( p \) and a suffix \( s \) of \( v \) such that
    \begin{equation*}
      w = p v s.
    \end{equation*}

    \thmitem{def:formal_language/kleene_star}\mcite[158]{Savage1998} We define the \term{Kleene star}\fnote{The Kleene star is a monoid --- see \fullref{def:free_monoid}.} \( \mscrA^* \) of \( \mscrA \) as the set of all words over \( \mscrA \). If we wish to exclude the empty word, like we often do, we instead write \( \mscrA^+ \) for the set of all non-empty words over \( \mscrA \).

    \thmitem{def:formal_language/language} We call any subset of \( \mscrA^* \) a \term{language} over \( \mscrA \).
  \end{thmenum}
\end{definition}
\begin{comments}
  \item Note that in some contexts like \hyperref[subsec:propositional_logic]{propositional logic} or \hyperref[subsec:first_order_logic]{first-order logic}, the term \enquote{language} may refer to the alphabet itself.
\end{comments}

\begin{example}\label{ex:def:formal_language}
  We list several examples of \hyperref[def:formal_language]{formal languages}:
  \begin{thmenum}
    \thmitem{ex:def:formal_language/full} The simplest examples are the empty language and the Kleene star itself.

    \thmitem{ex:def:formal_language/an} For any alphabet, we can pick one letter \( a \) and form the language
    \begin{equation*}
      \mscrL \coloneqq \set{ a^n \given n \geq 0 }
    \end{equation*}
    consisting of all finite repetitions of the symbol \( a \).

    We can redefine all operations from \fullref{subsec:natural_numbers} to hold for words in \( \mscrL \). For example, addition of \( a^n \) and \( a^m \) is their concatenation \( a^{n + m} \), while the exponentiation \( (a^n)^m \) corresponds to \( m \) repetitions of the word \( a^n \), that is, to \( a^{nm} \).

    We can identify the language with the \hyperref[def:labeled_set]{edge-labeled} \hyperref[def:directed_graph]{directed graph}
    \begin{equation}\label{eq:def:formal_language/an}
      \begin{aligned}
        \includegraphics[page=1]{output/ex__def__formal_language}
      \end{aligned}
    \end{equation}

    Each word in \( \mscrL \) corresponds to a \hyperref[def:graph_walk/directed]{walk} in \eqref{eq:def:formal_language/an} and vice versa.

    \thmitem{ex:def:formal_language/anbn} A slightly more complicated language is
    \begin{equation*}
      \mscrL \coloneqq \set{ a^n b^n \given n \geq 0 }.
    \end{equation*}

    It can encode natural number operations just as well. It cannot, however, be represented via a directed graph like \eqref{eq:def:formal_language/an}. This will be made precise and proved in \fullref{ex:def:finite_automaton/anbn}.

    \thmitem{ex:def:formal_language/even} Even numbers in binary notation are described by the language
    \begin{equation*}
      \mscrL \coloneqq \set[\Big]{ w0 \given w \in \set{ 0, 1 }^* }.
    \end{equation*}

    \thmitem{ex:def:formal_language/leucine}\mcite{codon_charts} The \term{Leucine} amino-acid is encoded by the language
    \begin{equation*}
       \mscrL \coloneqq \set{ \texttt{TTA}, \texttt{TTG}, \texttt{CTT}, \texttt{CTC}, \texttt{CTA}, \texttt{CTG} }.
    \end{equation*}
  \end{thmenum}
\end{example}

\paragraph{Finite automata}

\begin{definition}\label{def:finite_automaton}\mcite[27]{Salomaa1987}
  Fix an \hyperref[def:formal_language]{alphabet} \( \Sigma \). Let \( Q \) be a finite nonempty set, whose members we will call \term{states}. Let \( \delta: Q \times \Sigma \multto Q \) be a \hyperref[def:set_valued_map]{set-valued map}, which we will call a \term{transition function} because, depending on a state in \( Q \) and a symbol in \( \Sigma \), \( \delta \) gives us the possible states towards we can transition.

  Finally, let \( S \) and \( T \) be nonempty sets of states, which we call an \term{initial} and \term{terminal states}, correspondingly.

  We call this entire contraption \( (\Sigma, Q, \delta, I, T) \) a \term[ru=конечный автомат (\cite[159]{Гладкий1973})]{finite automaton}. It models a real-world device that starts its work from some initial state and, via a sequence of state transitions, reaches some terminal state.

  \begin{figure}[!ht]
    \hfill
    \includegraphics[page=1]{output/def__finite_automaton}
    \hfill
    \includegraphics[page=2]{output/def__finite_automaton}
    \hfill\hfill
    \caption{A \hyperref[def:finite_automaton/determinism]{nondeterministic finite automaton} accepting the language \( \set{ a } \cup \set{ b^n \given n > 0 } \cup \set{ aa b^n \given n > 0 } \) and its \hyperref[alg:determinization_of_finite_automata]{determinization}.}
    \label{fig:def:finite_automaton}
  \end{figure}

  \begin{thmenum}
    \thmitem{def:finite_automaton/graph}\mimprovised Regard \( \delta \) as a set of triples \( (h_0, l_0, t_0) \). Denote by \( h: \delta \to Q \), \( l: \delta \to \Sigma \) and \( t: \delta \to Q \) the functions that take the corresponding entry out of each triple.

    Then the quadruple \( (Q, \delta, h, t) \) is a \hyperref[def:directed_multigraph]{directed multigraph}, whose arcs are \hyperref[def:labeled_set]{labeled} by \( l \).

    We can identify the automaton with its (multi)graph. When drawing this graph, for example in \cref{fig:def:finite_automaton}, we denote initial states via inward arrows without a source and terminal states via double circles.

    \thmitem{def:finite_automaton/determinism} If there is only one initial state and if \( \delta \) is a \hyperref[def:set_valued_map/partial]{single-valued partial function}, we say that the automata is \term{deterministic}.

    Determinism ensures that there is at most one possible state to transition to, given a current state and a symbol.

    \thmitem{def:finite_automaton/recognition}\mcite[def. 4.1.1]{Savage1998} We say that the automaton \term{accepts} or \term{recognizes} the \hyperref[def:formal_language/word]{word} \( a_1 \cdots a_n \) over \( \mscrA \) if there exists a \hyperref[def:graph_walk/directed]{walk}
    \begin{equation*}
      s \reloset {e_1} \to \anon \reloset {e_2} \to \cdots \reloset {e_{n-1}} \to \anon \reloset {e_n} \to \anon.
    \end{equation*}
    such that
    \begin{itemize}
      \item The label \( l(e_k) \) is \( a_k \) for every \( k = 1, \ldots, n \).
      \item The tail \( t(e_n) \) of the walk is a terminal state from \( T \).
    \end{itemize}

    \thmitem{def:finite_automaton/language} The set of all words recognized by the automaton is called the language \term{accepted} or \term[ru=(язык) распознается (автоматом) (\cite[45]{Гладкий1973})]{recognized} by the automaton. We denote this language via \( \mscrL(F) \).

    \thmitem{def:finite_automaton/equivalent}\mcite[152]{Savage1998} We say that two finite automata are \term{equivalent} if they recognize the same language.
  \end{thmenum}
\end{definition}

\begin{definition}\label{def:reverse_language}\mimprovised
  We define the \term{reverse language} of \( \mscrL \) as
  \begin{equation*}
    \op{rev}(\mscrL) \coloneqq \set{ \op{rev}(w) \given w \in \mscrL }.
  \end{equation*}
\end{definition}

\begin{proposition}\label{thm:reverse_language_involution}
  The \hyperref[def:reverse_language]{reverse} of the reverse of a language is the original language.
\end{proposition}
\begin{proof}
  Trivial.
\end{proof}

\begin{definition}\label{def:reverse_finite_automaton}\mimprovised
  We define the \term{reverse automaton} of \( F = (\Sigma, Q, \delta, I, T) \) as
  \begin{equation*}
    \op{rev}(F) \coloneqq (\Sigma, Q, \op{rev}(\delta), T, I),
  \end{equation*}
  where
  \begin{equation*}
    \op{rev}(\delta) \coloneqq \set{ (q, a, p) \given (p, a, q) \in \delta }.
  \end{equation*}
\end{definition}
\begin{comments}
  \item We have
  \begin{equation*}
    p \in \op{rev}(\delta)(q, a) \T{if and only if} q \in \delta(p, a).
  \end{equation*}
\end{comments}

\begin{proposition}\label{thm:reverse_finite_automaton_graph}
  The \hyperref[def:finite_automaton/graph]{multigraph} of a \hyperref[def:reverse_finite_automaton]{reverse automaton} is its \hyperref[def:opposite_directed_multigraph]{opposite multigraph}.
\end{proposition}
\begin{proof}
  Trivial.
\end{proof}

\begin{proposition}\label{thm:reverse_finite_automaton_language}
  For a given \hyperref[def:finite_automaton]{finite automaton} \( F = (\Sigma, Q, \delta, I, T) \), we have
  \begin{equation*}
    \mscrL(\op{rev}(F)) = \op{rev}(\mscrL(F)).
  \end{equation*}
\end{proposition}
\begin{proof}
  Trivial.
\end{proof}

\begin{example}\label{ex:def:finite_automaton}
  We list several examples of \hyperref[def:finite_automaton]{finite automata}:
  \begin{thmenum}
    \thmitem{ex:def:finite_automaton/even} Consider the language form \fullref{ex:def:formal_language/leucine} describing even binary numbers. The language can be recognized by the automaton with multigraph
    \begin{equation*}
      \includegraphics[page=2]{output/ex__def__finite_automaton}
    \end{equation*}

    \thmitem{ex:def:finite_automaton/leucine} Consider the language form \fullref{ex:def:formal_language/leucine} describing Leucine. It can be recognized by the nondeterministic finite automaton
    \begin{equation*}
      \includegraphics[page=1]{output/ex__def__finite_automaton}
    \end{equation*}

    \thmitem{ex:def:finite_automaton/anbn} Consider the language
    \begin{equation*}
      \mscrL \coloneqq \set{ a^n b^n \given n \geq 0 }
    \end{equation*}
    from \fullref{ex:def:formal_language/anbn}.

    As we will see in \fullref{alg:determinization_of_finite_automata}, a deterministic automaton exists accepting a language if and only if a nondeterministic one exists. Aiming at a contradiction, suppose that there exists some deterministic finite automaton \( F = (\Sigma, Q, \delta, \set{ s }, T) \) whose language is \( \mscrL \). Let \( G \) be its multigraph.

    Since \( s \) is the only initial state, and since \( \mscrL \) contains the empty string, then \( s \) must also be a terminal state.

    Furthermore, \( \mscrL \) contains the word \( ab \), hence there must exist a terminal state \( t_1 \) and an intermediate state \( q_1 \) such that
    \begin{equation*}
      \includegraphics[page=3]{output/ex__def__finite_automaton}
    \end{equation*}

    Furthermore, the above is a full subgraph of \( G \) --- none of the states above are interconnected by any additional arcs.

    Then, in order for \( F \) to accept \( aabb \), it must have more states. Since the automaton is deterministic, there cannot be another arc with label \enquote{\( a \)} starting at \( s \). Hence, \( q_1 \) is the only node where it is possible to have another arc with label \enquote{\( a \)}.

    Hence, \( G \) either has as a subgraph either
    \begin{equation*}
      \includegraphics[page=4]{output/ex__def__finite_automaton}
    \end{equation*}
    or
    \begin{equation*}
      \includegraphics[page=5]{output/ex__def__finite_automaton}
    \end{equation*}

    In particular, \( F \) must have at least five states in order to recognize \( a^2 b^2 \).

    Continuing by induction, we conclude that in order for \( F \) to recognize \( a^n b^n \), it must have at least \( 2n + 1 \) states. For example, the following automaton recognizes \( a^n b^n \):
    \begin{equation*}
      \includegraphics[page=6]{output/ex__def__finite_automaton}
    \end{equation*}

    But \( \mscrL \) contains word of arbitrary length. Therefore, no finite automaton is able to recognize \( \mscrL \).
  \end{thmenum}
\end{example}

\begin{algorithm}[Determinization of finite automata]\label{alg:determinization_of_finite_automata}
  Let \( N = (\Sigma, Q, \delta, I, T) \) be a \hyperref[def:finite_automaton]{finite automaton}. We will build an \hyperref[def:finite_automaton/equivalent]{equivalent} \hyperref[def:finite_automaton/determinism]{deterministic automaton} \( \det(N) \).

  This can be achieved by grouping states that would otherwise make the automaton nondeterministic. We will first recursively construct the operator \( \mscrD(P, V) \), which, given a set of states \( P \) and a family of \enquote{visited} sets of states \( V \), produces a family of triples describing the transitions of the deterministic automaton. The family \( V \) helps us avoid cycles when traversing \( N \).

  \begin{thmenum}
    \thmitem{alg:determinization_of_finite_automata/step} Suppose we are given a subset \( P \) of \( P \) and a subset \( V \) of \( \pow(Q) \).

    For every symbol \( a \) in \( \Sigma \), consider the set of all states that we can transition to via \( a \) from some state in \( P \):
    \begin{equation*}
      \delta(P, a) = \bigcup_{p \in P} \delta(p, a) = \set{ q \in Q \given \qexists* {p \in P} q \in \delta(p, a) }.
    \end{equation*}

    Define the set of triples that would become part the graph of the new transition function:
    \begin{equation*}
      E_P \coloneqq \set{ (P, a, \delta(P, a)) \given \delta(P, a) \neq \varnothing }.
    \end{equation*}

    Finally, define
    \begin{equation*}
      \mscrD(P, V) \coloneqq E_P \cup \bigcup \set[\Big]{ \mscrD\parens[\Big]{ \delta(P, a), V \cup \set{ P } } \given* \delta(P, a) \T{is nonempty and is not in} V }.
    \end{equation*}

    Note how we used \( V \) to filter out only those sets of states that have not yet been visited.

    \thmitem{alg:determinization_of_finite_automata/run} Let \( \delta' \coloneqq \mscrD(I, \varnothing) \). Define the new set of states
    \begin{equation*}
      Q' \coloneqq \set[\Big]{ P \given* (P, a, O) \in \delta' } \cup \set[\Big]{ O \given* (P, a, O) \in \delta' }.
    \end{equation*}

    Define also the set of initial states \( I' \coloneqq \set{ I } \) and of terminal states
    \begin{equation*}
      T' \coloneqq \set{ P \in Q' \given P \cap T \neq \varnothing }.
    \end{equation*}

    Then \( \det(N) \coloneqq (\Sigma, Q', \delta', I', T') \) is the desired finite automaton.
  \end{thmenum}
\end{algorithm}
\begin{comments}
  \item This algorithm can be found as \texttt{automata.finite.determinize} in \cite{code}.
\end{comments}
\begin{defproof}
  We have explicitly made \( I \) the only initial state, and we have grouped arcs with identical labels. Hence, \( F \) is indeed deterministic.

  We must show that \( \mscrL(N) = \mscrL(F) \). We have introduced a special state for every occurrence of several vertices that have incoming arcs with identical labels. Hence, we replace every such group of arcs with a single arc. The two automata should then accept identical languages.
\end{defproof}

\paragraph{Formal grammars}

\begin{definition}\label{def:formal_grammar}\mcite[def. 4.9.1]{Savage1998}
  Let \( \Sigma \) and \( V \) be disjoint nonempty subsets of some \hyperref[def:formal_language]{alphabet}, whose members we call \term[ru=основные (символы) (\cite[27]{Гладкий1973})]{terminals} and \term[ru=вспомогательные (символы) (\cite[27]{Гладкий1973})]{nonterminals}, respectively. Fix some \term{starting nonterminal} \( S \in V \).

  Let \( \to \) be some \hyperref[def:binary_relation]{binary relation} over \( (V \cup \Sigma)^* \), whose members we call \term[ru=правила (\cite[27]{Гладкий1973})]{production rules}. We impose the restriction that, for every rule \( v \to w \), the word \( v \) contains at least one nonterminal.

  We call the quadruple \( (V, \Sigma, \to, S) \) a \term[ru=(формальная) генеративная грамматика (\cite[10]{Гладкий1973})]{formal generative grammar}.

  \begin{thmenum}
    \thmitem{def:formal_grammar/derivation} We define the binary relation \( \Rightarrow \) on the Kleene star \( (V \cup \Sigma)^* \) by declaring that, for every two \hyperref[def:formal_language/word]{words} \( p \) and \( s \) over \( V \cup \Sigma \) and every production rule \( v \to w \), we have \( pvs \Rightarrow pws \). We call this an \term[ru=непосредственный (вывод) (\cite[28]{Гладкий1973})]{immediate derivation}.

    A \term[ru=вывод (\cite[27]{Гладкий1973})]{derivation} of length \( m \) of the word \( w_m \) from \( w_0 \) is a finite sequence of words such that
    \begin{equation}\label{eq:def:formal_grammar/derivation}
      w_0
      \Rightarrow
      w_1
      \Rightarrow
      \cdots
      \Rightarrow
      w_{n-1}
      \Rightarrow
      w_m.
    \end{equation}

    We say that \( w_m \) is (immediately) \term[ru=выводимая (цепочка) (\cite[27]{Гладкий1973})]{derivable} from \( w_0 \) if there exists a corresponding derivation

    We denote the \hyperref[def:relation_closures/transitive]{transitive closure} of \( \Rightarrow \) by \( \reloset + \Rightarrow \) and the \hyperref[def:relation_closures/reflexive]{reflexive} closure of \( \reloset + \Rightarrow \) by \( \reloset {*} \Rightarrow \). Clearly \( w_1 \) is derivable from \( w_n \) if and only if \( w_1 \reloset {*} \Rightarrow w_m \).

    \thmitem{def:formal_grammar/language} We associate the following language with the grammar \( G \):
    \begin{equation*}
      \mscrL(G) \coloneqq \set{ w \in \Sigma^* \given S \reloset + \Rightarrow w }.
    \end{equation*}

    It consists of all words that are derivable from \( S \) and contain only terminals.

    We say that words in \( \mscrL(G) \) are \term{generated} by \( G \).

    \thmitem{def:formal_grammar/equivalent} We say that two grammars are \term{equivalent} if they generate the same language.

    \thmitem{def:formal_grammar/graph}\mimprovised We can regard the derivation relation \( \Rightarrow \) as a set of \hyperref[def:directed_graph/arcs]{arcs} over the grammar language \( \mscrL(G) \). Thus, \( (\mscrL(G), \Rightarrow) \) is a \hyperref[def:directed_graph]{directed graph}\fnote{It is possible for multiple rules to produce the same immediate derivation, however \( \Rightarrow \) is only concerted that at least one exists.}, whose nonempty walks are precisely the derivations in \( G \). Furthermore, we can \hyperref[def:labeled_set]{label} each arc with the rule applied.

    We call it the \term{derivation graph} of \( G \).

    \thmitem{def:formal_grammar/schema}\mcite[27]{Гладкий1973} There are different grammars sharing the same alphabet and rules, but having different starting nonterminals. We will call the set of rules a \term[ru=схема]{grammar schema}.
  \end{thmenum}
\end{definition}

\begin{proposition}\label{thm:derivation_graph_connected}
  Every \hyperref[def:formal_grammar/graph]{derivation graph} is \hyperref[def:graph_connectedness/weak]{weakly connected}.
\end{proposition}
\begin{proof}
  A \hyperref[def:formal_grammar/language]{grammar's language} is defined as the set of all words that can be derived from the starting nonterminal.
\end{proof}

\begin{example}\label{ex:natural_number_arithmetic_grammar/schema}
  We define a \hyperref[def:formal_grammar/schema]{grammar schema} for arithmetic of \hyperref[def:natural_numbers]{natural numbers}. We will use binary notation for simplicity.

  Let \( V \coloneqq \set{ N, O, E } \) and \( \Sigma \coloneqq \set{ 0, 1, +, \times, (, ) } \). Consider the derivation rules
  \begin{equation}\label{eq:ex:natural_number_arithmetic_grammar/schema/simple}
    \begin{aligned}
      N &\to 0   & \quad B &\to 0   & \quad O &\to +      & \quad E &\to N \\
      N &\to 1   & \quad B &\to 0 B & \quad O &\to -      & \quad E &\to (E O E) \\
      N &\to 1 B & \quad B &\to 1   & \quad   &           &         & \\
        &        & \quad B &\to 1 B & \quad   &           &         & \\
    \end{aligned}
  \end{equation}

  It is convenient to use the following shorthands:
  \begin{equation}\label{eq:ex:natural_number_arithmetic_grammar/schema/shorthand}
    \begin{aligned}
      N &\to 0 \mid 1 \mid 1 B \\
      B &\to 0 \mid B 0 \mid 1 \mid B 1 \\
      O &\to + \mid \times \\
      E &\to N \mid (E O E)
    \end{aligned}
  \end{equation}

  Choosing a different starting nonterminal generates different languages. The symbol \( N \) corresponds to numbers, \( O \) corresponds to operations, and \( E \) can be either a number or a binary expression.

  \begin{figure}[!ht]
    \centering
    \includegraphics[page=1]{output/ex__natural_number_arithmetic_grammar__rules}
    \caption{A fragment of the \hyperref[def:formal_grammar/graph]{derivation graph} of the binary natural number arithmetic grammar from \fullref{ex:natural_number_arithmetic_grammar/schema}.}
    \label{fig:ex:natural_number_arithmetic_grammar/schema}
  \end{figure}
\end{example}

\paragraph{Length-increasing grammars}

\begin{remark}\label{rem:length_increasing_grammar}
  Given a finite set of grammar rules, there may be \hyperref[def:formal_grammar/derivation]{derivations} of arbitrary length. This can happen even if we remove cycles from the \hyperref[def:formal_grammar/graph]{derivation graph} --- see \fullref{ex:unboudned_grammar_derivation_length}. We will discuss a way to restrict this behavior and ensure that it is possible to determine whether a derivation exists in finitely many steps.

  In general, given the grammar rule \( v \to w \), it is possible that \( \len(v) \geq \len(w) \). This is always true for \hyperref[def:epsilon_free_grammar]{\( \varepsilon \) rules}, for example.

  Noam Chomsky in \cite[361]{MathPsychology1963Vol2} defines \enquote{type 1} grammars as those satisfying the inequality \( \len(v) \leq \len(w) \) for every rule \( v \to w \). These grammars still have cycles like \( A \to B \to A \), but they avoid more convoluted cases and lead to \Fullref{alg:length_increasing_grammar}.

  This restriction excludes the empty string from the grammar's language. \incite[15]{Salomaa1987} additionally allows the rule \( S \to \varepsilon \), but only if \( S \) does not occur on the right side of any derivation. \Fullref{ex:unboudned_grammar_derivation_length} highlights the importance of the latter assumption.

  Salomaa calls these grammars \enquote{length-increasing}, preferring to use \enquote{type 1} for what we call \hyperref[def:chomsky_hierarchy/context_sensitive]{context-sensitive} grammars. The latter are, up to nuances of \( \varepsilon \) rules, called \enquote{type 1} grammars by Chomsky earlier in \cite[142]{Chomsky1959}. John Savage in \cite[def. 4.9.2]{Savage1998} defines \enquote{context-sensitive grammars} as what Salomaa calls length-increasing. We introduce the term \enquote{essentially length-increasing} in \fullref{def:length_increasing_grammar} and generally avoid \enquote{type 1} when referring to grammars or languages to circumvent ambiguity.

  (What we call) context-sensitive \hi{grammars} is a strict subset of the essentially length-increasing grammars, but \fullref{thm:context_sensitive_languages} shows that context-sensitive \hi{languages} and length-increasing languages coincide. Hence, the term \enquote{type 1 language} makes sense, although we prefer \enquote{context-sensitive language}.
\end{remark}

\begin{definition}\label{def:epsilon_free_grammar}\mcite[54]{Salomaa1987}
  Fix a formal grammar \( G = (V, \Sigma, \to, S) \). Rules of the form \( A \to \varepsilon \), where \( A \) is a nonterminal, play a special role. We call them \term{\( \varepsilon \) rules}.

  We say that \( G \) is \term{\( \mathbf{\varepsilon} \)-free} if there are no \( \varepsilon \) rules.

  This condition sometimes turns out to be too restrictive because it disallows the grammar to generate empty words. Thus, we introduce another concept. We say that \( G \) is \term{essentially \( \mathbf{\varepsilon} \)-free} if \( S \to \varepsilon \) is the only \( \varepsilon \) rule allowed, and if it is present, \( S \) must not be on the right side of any rule.
\end{definition}
\begin{comments}
  \item This definition for essentially \( \varepsilon \)-free grammars is our own generalization of the concept of essentially length-increasing grammars defined in \fullref{def:length_increasing_grammar}.
\end{comments}

\begin{definition}\label{def:length_increasing_grammar}\mcite[15]{Salomaa1987}
  We say that the grammar \( G = (V, \Sigma, \to, S) \) is \term[ru=неукорачивающая (граматика) (\cite[83]{Гладкий1973})]{length-increasing} if \( \len(v) \leq \len(w) \) for every rule \( v \to w \) and \term{essentially length-increasing} if it is \hyperref[def:epsilon_free_grammar]{essentially \( \varepsilon \)-free} and if \( \len(v) \leq \len(w) \) for any non-\( \varepsilon \) rule.
\end{definition}

\begin{lemma}\label{thm:length_increasing_grammar}
  Fix an \hyperref[def:length_increasing_grammar]{essentially length-increasing} grammar \( G = (V, \Sigma, \to, S) \). Let \( m \) be the cardinality of \( \Sigma \cup V \).

  If the word \( w \) over \( \Sigma \) of length \( n \) is derivable, there exists a derivation of length at most
  \begin{equation*}
    \sum_{k=0}^n m^k.
  \end{equation*}
\end{lemma}
\begin{comments}
  \item If \( m > 1 \), we have
  \begin{equation*}
    \sum_{k=0}^n m^k
    \reloset {\ref{thm:geometric_series_properties/finite_sum}} =
    \frac {1 - m^{n+1}} {1 - m}.
  \end{equation*}

  \item Chomsky introduced length-increasing grammars in \cite[360]{MathPsychology1963Vol2} and on the same page hinted at this property of theirs, which we will use in \fullref{alg:length_increasing_grammar}.
\end{comments}
\begin{proof}
  Suppose that the word \( w \) of length \( n \) is derivable in \( G \) and consider the derivation
  \begin{equation*}
    S = w_0 \Rightarrow w_1 \Rightarrow \cdots \Rightarrow w_r = w.
  \end{equation*}

  The case where \( w \) is empty should be handled separately. Since \( G \) is essentially \( \varepsilon \)-free, the only possible derivation of \( \varepsilon \) follows the rule \( S \to \varepsilon \) and thus has length
  \begin{equation*}
    1 = m^0 = \sum_{k=0}^0 m^k.
  \end{equation*}

  Now suppose that \( n > 0 \). Let \( i_0 \coloneqq 0 \) and, for every \( s = 1, \ldots, n \), let \( i_s \) be the index of the last word of length \( s \) in the derivation (if no word of length \( s \) exists, let \( i_s \) match \( i_{s-1} \)). Note that there are exactly \( m^s \) possible words of length \( s \), and thus there must exist a derivation of \( w_{i_s} \) from \( w_{i_{s-1} + 1} \) in at most \( m^s - 1 \) steps. Any derivation longer than that necessarily follows a \hyperref[def:graph_cycle]{cycle} in the \hyperref[def:formal_grammar/graph]{derivation graph}.

  Therefore, \( w_{i_s} \) can be derived from \( w_{i_0} = S \) in at most
  \begin{equation*}
    \sum_{k=0}^s (m^k - 1) \leq \sum_{k=0}^s m^k
  \end{equation*}
  steps.
\end{proof}

\begin{example}\label{ex:unboudned_grammar_derivation_length}
  Consider the grammar
  \begin{equation}\label{eq:ex:unboudned_grammar_derivation_length/bad}
    \begin{aligned}
      S &\to \varepsilon \\
      S &\to a, \\
      S &\to SS.
    \end{aligned}
  \end{equation}

  It is not \hyperref[def:epsilon_free_grammar]{essentially epsilon-free}, hence also not \hyperref[def:length_increasing_grammar]{essentially length-increasing}, and thus \fullref{thm:length_increasing_grammar} does not apply. The lemma would imply that the acyclic derivation length should be bounded by \( 2 \), while there exist countably many acyclic derivations of the word \( a \):
  \begin{equation*}
    \begin{aligned}
      S &\Rightarrow a \\
      S &\Rightarrow SS \Rightarrow Sa \Rightarrow a \\
      S &\Rightarrow SS \Rightarrow SSS \Rightarrow SSa \Rightarrow Sa \Rightarrow a \\
        &\vdots
    \end{aligned}
  \end{equation*}

  This is problematic for parsing algorithms because we cannot check in finitely many steps whether a grammar generates a word. \Fullref{alg:epsilon_rule_removal} suggests instead the essentially length-increasing grammar
  \begin{equation*}
    \begin{aligned}
      S &\to \varepsilon \\\
      S &\to A, \\
      A &\to a, \\
      A &\to AA,
    \end{aligned}
  \end{equation*}
  which disallows the aforementioned derivations.
\end{example}

\begin{algorithm}[Word membership in context-sensitive languages]\label{alg:length_increasing_grammar}
  Let \( G = (V, \Sigma, \to, S) \) be an \hyperref[def:length_increasing_grammar]{essentially length-increasing} grammar for \( \mscrL \). Denote by \( m \) the cardinality of \( V \cup \Sigma \). Fix some upper bound \( n \) on lengths of words. We can construct a set \( L_n \) that contains all words in \( \mscrL(G) \) of length at most \( n \) (it may also contain longer words).

  \begin{thmenum}
    \thmitem{alg:length_increasing_grammar/start} Start with a set \( W_0 \coloneqq \set{ S } \) of words derivable in zero steps.

    \thmitem{alg:length_increasing_grammar/step} Given \( W_k \), define the set of words derivable in \( k + 1 \) steps:
    \begin{equation*}
      W_{k+1} \coloneqq \set{ p v s \given p u s \in W_k \T{and} u \to v }.
    \end{equation*}

    \thmitem{alg:length_increasing_grammar/union} \Fullref{thm:length_increasing_grammar} gives us an upper bound on the derivation length of words of length \( n \). Denote this bound by \( u \). We can thus take the union of all words derivable in at most \( u \) steps and ignore those that contain nonterminals:
    \begin{equation*}
      L_n \coloneqq \bigcup_{k=1}^u \set{ w \in W_k \given w \T{has only terminals} }.
    \end{equation*}

    \thmitem{alg:length_increasing_grammar/membership} If we are interested in whether a particular word \( w \) of length at most \( n \) is in \( \mscrL(G) \), we can simply check if it is in \( L_n \), or, even better, at every step of the algorithm check if it is in \( W_k \).
  \end{thmenum}
\end{algorithm}
\begin{comments}
  \item This algorithm can be used to test whether a word belongs to a \hyperref[def:chomsky_hierarchy/context_sensitive]{context-sensitive language}, hence the name.

  \item Compare this to the more complicated \fullref{alg:brute_force_parsing} that is intended for more restricted cases, but also handles \( \varepsilon \) rules.
\end{comments}

\paragraph{Hierarchy of grammars}

\begin{definition}\label{def:chomsky_hierarchy}\mcite[15]{Salomaa1987}
  We can classify \hyperref[def:formal_grammar]{formal grammars} and \hyperref[def:formal_language]{languages} to form the \term{Chomsky hierarchy}. Chomsky himself in \cite[def. 6]{Chomsky1959} defined a hierarchy of grammars consisting of four types --- \enquote{type 0} through \enquote{type 3}. He also defined a parallel hierarchy of languages, in which \( \mscrL \) is a \enquote{type \( i \) language} if there exists a type \( i \) grammar \hyperref[def:formal_grammar/language]{generating it}.

  Unfortunately, these definitions later evolved to be inconsistent across authors, and even among different publications by Chomsky. We thus entirely avoid numeric grammar and language types, and use more descriptive names instead. The grammars no longer form a hierarchy, but the corresponding languages do.

  \begin{thmenum}
    \thmitem{def:chomsky_hierarchy/unrestricted} When no additional restrictions are imposed on the rules of the grammar, we call it \term{unrestricted}. We call the corresponding languages \term{recursively enumerable} following \cite[thm 5.4.1; thm 5.4.2]{Savage1998}.

    \thmitem{def:chomsky_hierarchy/context_sensitive} We say that the grammar is \term{context-sensitive} if it is \hyperref[def:epsilon_free_grammar]{essentially \( \varepsilon \)-free} and if every non-\( \varepsilon \) rule has the form
    \begin{equation*}
      p A s \to p w s
    \end{equation*}
    for a nonterminal \( A \), arbitrary words \( p \) and \( s \) and a nonempty\footnote{Since \( w \) is nonempty, a context-sensitive grammar is \hyperref[def:length_increasing_grammar]{essentially length-increasing}.} word \( w \). Of course, there may be multiple such representations for a single rule.

    We call the corresponding languages \term{context-sensitive}. The kerfuffle surrounding the term \enquote{context-sensitive} is discussed in \fullref{rem:length_increasing_grammar}. \Fullref{thm:context_sensitive_languages} better characterizes these languages.

    \thmitem{def:chomsky_hierarchy/context_free} We say that the grammar is \term[ru=безконтекстная / контекстно-свободная (грамматика) (\cite[29]{Гладкий1973})]{context-free} if every rule has the form
    \begin{equation*}
      A \to w,
    \end{equation*}
    where \( A \) is a nonterminal and \( w \) is an arbitrary word.

    We call the corresponding languages \term{context-free}. While it is possible for a context-free \hi{grammar} to not be \hyperref[def:epsilon_free_grammar]{essentially \( \varepsilon \)-free} and hence not context-sensitive, a context-free \hi{language} is context-sensitive. This is shown in \fullref{thm:context_free_languages_are_context_sensitive} and discussed in \fullref{rem:chomsky_hierarchy_failure}.

    \thmitem{def:chomsky_hierarchy/regular}\mcite[44]{Salomaa1987} Finally, we call the grammar \term{left linear} if every rule has one of the forms
    \begin{itemize}
      \item \( A \to w \),
      \item \( A \to B w \),
    \end{itemize}
    where \( w \) is an arbitrary word consisting entirely of terminals. Similarly, it is \term{right linear} if every rule has one of the forms
    \begin{itemize}
      \item \( A \to w \),
      \item \( A \to w B \).
    \end{itemize}

    We refer to the two types of grammars collectively as \term{regular grammars}.

    We call the language \term{regular} if it can be generated by a regular grammar. \Fullref{thm:regular_languages} better characterizes these languages.
  \end{thmenum}
\end{definition}
\begin{comments}
  \item In \cite[142]{Chomsky1959}, Chomsky calls context-free languages \enquote{type 2} and regular languages \enquote{type 3}, while in \cite[366]{MathPsychology1963Vol2} he calls context-free languages \enquote{type 4} and gives no number for regular languages.
\end{comments}

\begin{proposition}\label{thm:non_recursively_enumerable_language}\mcite[thm. 5.7.4]{Savage1998}
  There exists a formal language that cannot be generated by a grammar.
\end{proposition}

\begin{example}\label{ex:def:chomsky_hierarchy}
  We give several examples of grammars in the \hyperref[def:chomsky_hierarchy]{Chomsky hierarchy}.

  \begin{thmenum}
    \thmitem{ex:def:chomsky_hierarchy/non_enumerable} While every grammar is an unrestricted grammar, \fullref{thm:non_recursively_enumerable_language} shows that not every language is recursively enumerable.

    \thmitem{ex:def:chomsky_hierarchy/an} The \hyperref[def:chomsky_hierarchy/regular]{right linear grammar}
    \begin{equation*}
      \begin{aligned}
        S &\to aS \mid \varepsilon
      \end{aligned}
    \end{equation*}
    describes the language \( \mscrL = \set{ a^n \given n \geq 0 } \) discussed in \fullref{ex:def:formal_language/an}.

    It can also be described via the left linear grammar
    \begin{equation*}
      \begin{aligned}
        S &\to Sa \mid \varepsilon.
      \end{aligned}
    \end{equation*}

    Neither of these grammars is \hyperref[def:epsilon_free_grammar]{essentially \( \varepsilon \)-free}. \Fullref{ex:alg:epsilon_rule_removal/an} proposes using \Fullref{alg:epsilon_rule_removal} to obtain the essentially length-increasing grammar
    \begin{equation*}
      \begin{aligned}
        S &\to A \mid \varepsilon, \\
        A &\to aA \mid a.
      \end{aligned}
    \end{equation*}

    \thmitem{ex:def:chomsky_hierarchy/anbn} Consider the \hyperref[def:chomsky_hierarchy/context_free]{context-free grammar}
    \begin{equation*}
      \begin{aligned}
        S &\to A \mid \varepsilon, \\
        A &\to aAb \mid ab
      \end{aligned}
    \end{equation*}
    describing \( \mscrL = \set{ a^n b^n \given n \geq 0 } \) from \fullref{ex:def:formal_language/anbn}.

    We have shown in \fullref{ex:def:finite_automaton/anbn} that this language cannot be recognized by a finite automaton. \Fullref{thm:regular_languages} suggests that it is not \hyperref[def:chomsky_hierarchy/regular]{regular}.

    Hence, \( \mscrL \) is a context-free language that is not regular.

    \thmitem{ex:def:chomsky_hierarchy/length_increasing_not_context_sensitive} In a \hyperref[def:chomsky_hierarchy/context_sensitive]{context-sensitive grammar}, the only possible production rules replace a nonterminal with some nonempty word. In general, it is possible to replace terminal symbols with arbitrary words. For example, consider the grammar
    \begin{equation*}
      \begin{aligned}
         S &\to aA, \\
        aA &\to BB \\
         B &\to b \\
      \end{aligned}
    \end{equation*}

    The language generated by this grammar is \( \mscrL = \set{ bb } \). The entirety of \( aA \) gets replaced by a new word not starting with \( a \), while in a context-sensitive grammar the second rule would need to include the terminal \( a \) as the first symbol --- only the nonterminal \( A \) would get replaced, and only when it is preceded by \( a \). Of course, the grammar can be vastly simplified:
    \begin{equation*}
      S \to bb.
    \end{equation*}

    It is reasonable to expect that grammars are context-sensitive, i.e. that terminals should never get replaced. Unfortunately, as we just saw, this is not so general \hyperref[def:length_increasing_grammar]{length-increasing} grammars. \Fullref{alg:length_increasing_to_context_sensitive} however allows us to convert general (essentially) length-increasing grammars to equivalent context-sensitive grammars.
  \end{thmenum}
\end{example}

\paragraph{Context-sensitive languages}

\begin{algorithm}[Length-increasing to context-sensitive grammar]\label{alg:length_increasing_to_context_sensitive}\mcite[thm. 9.2]{Salomaa1987}
  Fix an \hyperref[def:length_increasing_grammar]{essentially length-increasing} grammar \( G = (V, \Sigma, \to, S) \). We will build an \hyperref[def:formal_grammar/equivalent]{equivalent} (essentially length-increasing) \hyperref[def:chomsky_hierarchy/context_sensitive]{context-sensitive} grammar \( G' = (V', \Sigma, \to', S) \) with the same terminals.

  Enumerate all non-\( \varepsilon \) rules of \( \to \) from \( 1 \) to \( n \).

  \begin{thmenum}
    \thmitem{alg:length_increasing_to_context_sensitive/init} For every terminal \( a \) in \( \Sigma \), let \( a' \) be a new nonterminal not in \( V \). Let
    \begin{equation*}
      V_0 \coloneqq V \cup \set{ a' \given a \in \Sigma }.
    \end{equation*}

    For every nonterminal \( A \) in \( V_0 \), let \( A' \) refer to \( A \) itself.

    Let \( {\to_0} \) consist only of the rules \( a' \mapsto a \) for every terminal \( a \) in \( \Sigma \).

    \thmitem{alg:length_increasing_to_contexs_sensitive/step} At step \( k \), we consider the rule
    \begin{equation*}
      r_1 \cdots r_{m_k} \to s_1 \cdots s_{l_k},
    \end{equation*}
    where \( r_1 \cdots r_{m_k} \) and \( s_1 \cdots s_{l_k} \) are either terminal or nonterminal symbols.

    Let \( C_1, \ldots, C_{m_k} \) be new nonterminal symbols, and let
    \begin{equation*}
      V_k \coloneqq V_{k-1} \cup \set{ C_1, \ldots, C_{m_k} }.
    \end{equation*}

    Define \( \to'_k \) as \( \to'_{k-1} \) with the following additional rules:
    \begin{align*}
      r_1' r_2' \cdots r_{m_k}'                                      &\to'_k C_1 r_2' \cdots r_{m_k}', \\
      C_1 r_2' \cdots r_{m_k}'                                       &\to'_k C_1 C_2 \cdots r_{m_k}', \\
                                                                     &\vdots, \\
      C_1 C_2 \cdots C_{m_k-1} r_{m_k}'                              &\to'_k C_1 C_2 \cdots C_{m_k-1} C_{m_k} s_{m_k+1}' \cdots s_{l_k}', \\
      C_1 C_2 \cdots C_{m_k-1} C_{m_k} s_{m_k+1}' \cdots s_{l_k}'    &\to'_k s_1' C_2 \cdots C_{m_k-1} C_{m_k} s_{m_k+1}' \cdots s_{l_k}', \\
                                                                     &\vdots, \\
      s_1' s_2' \cdots s_{m_k-1}' C_{m_k} s_{m_k+1}' \cdots s_{l_k}' &\to'_k s_1' s_2' \cdots s_{m_k-1}' B_{m_k} s_{m_k+1}' \cdots s_{l_k}'.
    \end{align*}

    Each of these rules replaces exactly one nonterminal with a nonempty word.

    \thmitem{alg:length_increasing_to_context_sensitive/finish} Finally, let \( V' \coloneqq V'_n \) and \( {\to'} \coloneqq {\to'_n} \). In the case where \( S \to \varepsilon \), let \( S \to' \varepsilon \). Then \( G' = (V', \Sigma, \to', S) \) is the desired grammar.
  \end{thmenum}
\end{algorithm}

\begin{proposition}\label{thm:context_sensitive_languages}
  The class of languages generated by \hyperref[def:chomsky_hierarchy/context_sensitive]{context-sensitive grammars} and by \hyperref[def:length_increasing_grammar]{essentially length-increasing} grammars coincide.
\end{proposition}
\begin{comments}
  \item As discussed in \fullref{rem:length_increasing_grammar} and \fullref{def:chomsky_hierarchy/context_sensitive}, we call this class of languages \enquote{context-sensitive languages}.
\end{comments}
\begin{proof}
  Every context-sensitive grammar is, by definition, essentially length-increasing. \Fullref{alg:length_increasing_to_context_sensitive} shows that every essentially length-increasing grammar has an equivalent context-sensitive grammar.
\end{proof}

\begin{remark}\label{rem:chomsky_hierarchy_failure}
  Every \hyperref[def:chomsky_hierarchy/regular]{regular grammar} is \hyperref[def:chomsky_hierarchy/context_free]{context-free}, but not every context-free grammar is \hyperref[def:epsilon_free_grammar]{essentially \( \varepsilon \)-free} and \hyperref[def:chomsky_hierarchy/context_sensitive]{context-sensitive}. Chomsky disallowed \( \varepsilon \) rules in \cite[def. 6]{Chomsky1959}, and this led to a tidy hierarchy because it made context-free grammars necessarily context-sensitive.

  Fortunately, context-free \hi{languages} are context-sensitive as a consequence of \fullref{thm:context_free_languages_are_context_sensitive}.

  Unfortunately, for an arbitrary context-free grammar, \fullref{thm:length_increasing_grammar} does not hold, and neither does \fullref{alg:length_increasing_grammar}. This is not a problem because:
  \begin{itemize}
    \item For theoretical purposes, we can use \fullref{alg:epsilon_rule_removal} to adapt context-free languages to \fullref{alg:length_increasing_grammar}.

    \item For practical purposes, parsing context-free languages is a topic in itself. These algorithms are mostly restricted to certain classes of context-free grammars; see e.g. \cite[ch. 6]{Salomaa1987}. General algorithms for parsing all context-free grammars are scarcer; several are discussed in \cite{Economopoulos2006}. We present \fullref{alg:brute_force_parsing}, which handles arbitrary context-free grammars, but is too inefficient to use in practice.
  \end{itemize}
\end{remark}

\begin{algorithm}[Epsilon rule removal]\label{alg:epsilon_rule_removal}\mcite[thm. 6.2]{Salomaa1987}
  Fix a \hyperref[def:chomsky_hierarchy/context_free]{context-free} grammar \( G = (V, \Sigma, \to, S) \). We will build an \hyperref[def:formal_grammar/equivalent]{equivalent} context-free grammar \( G' = (V', \Sigma, \to', S') \) with the same terminals, such that \( G' \) is \hyperref[def:epsilon_free_grammar]{essentially \( \varepsilon \)-free}, and thus context-sensitive.

  \begin{thmenum}
    \thmitem{alg:epsilon_rule_removal/init} We will recursively construct a set \( U \) so that \( A \reloset {*} \Rightarrow \varepsilon \) if and only if \( A \in U \).

    First, define
    \begin{equation*}
      U_k \coloneqq \begin{cases}
        \set[\Big]{ A \in V \given A \to \varepsilon },                               &k = 0, \\
        U_{k-1} \cup \set[\Big]{ A \in V \given \qexists*{w \in U_{k-1}^*} A \to w }, &k > 0.
      \end{cases}
    \end{equation*}

    At each step, we \( U_k \) is a subset of \( V \). Since \( V \) has only finitely many nonterminals, the sequence \hyperref[def:stabilizing_sequence]{stabilizes} --- there exists some index \( m \) such that \( U_m = U_k \) for any \( k > m \). Denote \( U_m \) via \( U \).

    Then \( A \reloset {*} \Rightarrow \varepsilon \) if and only if \( A \in U \).

    \thmitem{alg:epsilon_rule_removal/rules} Now our goal is to define the rules of \( G' \) based on the rules of \( G \), but with zero or more \enquote{nullable} nonterminals removed from the right side of any rule.

    Let \( S' \) be an entirely new start symbol and let \( V' \coloneqq V \cup \set{ S' } \). Define the production relation \( A \to' w \) to hold if \( w \) is \hi{nonempty} and if there exist words \( w_0, \ldots, w_n \) over \( V \cup \Sigma \) and nonterminals \( B_1, \ldots, B_n \) from \( U \) such that
    \begin{equation*}
      A \to w_0 B_1 w_1 B_2 w_2 \cdots B_n w_n.
    \end{equation*}
    and
    \begin{equation*}
      w = w_0 w_1 w_2 \cdots w_n.
    \end{equation*}

    \thmitem{alg:epsilon_rule_removal/finish} Define \( S' \to S \). If \( S \reloset {*} \Rightarrow \varepsilon \), also add the rule \( S' \to \varepsilon \). Then \( G' = (V', \Sigma, \to', S') \) is the desired grammar.
  \end{thmenum}
\end{algorithm}

\begin{example}\label{ex:alg:epsilon_rule_removal}
  We list several examples demonstrating the operation of \fullref{alg:epsilon_rule_removal}:
  \begin{thmenum}
    \thmitem{ex:alg:epsilon_rule_removal/an} We discussed the grammar
    \begin{equation*}
      S \to aS \mid \varepsilon
    \end{equation*}
    in \fullref{ex:def:chomsky_hierarchy/an}.

    Using \fullref{alg:epsilon_rule_removal}, we conclude that the only nonterminal \( S \) belongs to \( U_0 \). Thus, \fullref{alg:epsilon_rule_removal/rules} suggests instead the rules
    \begin{equation*}
      S \to' aS \mid a
    \end{equation*}
    and \fullref{alg:epsilon_rule_removal/finish} suggests
    \begin{equation*}
      S' \to' S \mid \varepsilon,
    \end{equation*}
    where \( S' \) is the new starting nonterminal.

    The obtained grammar is essentially \( \varepsilon \)-free.

    \thmitem{ex:alg:epsilon_rule_removal/natural} Given the rules
    \begin{equation*}
      \begin{aligned}
        N &\to 0 \mid 1 B, \\
        B &\to \varepsilon \mid 0 B \mid 1 B
      \end{aligned}
    \end{equation*}
    and starting nonterminal \( N \), the algorithm suggests instead
    \begin{equation*}
      \begin{aligned}
        S' &\to' N, \\
        N  &\to' 0 \mid 1 \mid 1 B, \\
        B  &\to' 0 \mid 0 B \mid 1 \mid 1 B.
      \end{aligned}
    \end{equation*}

    The only member of \( U \) is \( B \), and we add a new rule of every instance of \( B \) where it is removed.

    This motivated our choice for rules in \fullref{ex:natural_number_arithmetic_grammar/schema}. The obtained grammar is \( \varepsilon \)-free, not merely essentially \( \varepsilon \)-free.

    \thmitem{ex:alg:epsilon_rule_removal/dead} It is possible to obtain \enquote{dead} rules via \fullref{alg:epsilon_rule_removal}. For example, for
    \begin{equation*}
      \begin{aligned}
        S &\to A B, \\
        A &\to \varepsilon \mid a, \\
        B &\to \varepsilon
      \end{aligned}
    \end{equation*}
    with starting nonterminal \( S \), the algorithm produces
    \begin{equation*}
      \begin{aligned}
        S' &\to S \mid \varepsilon, \\
        S &\to A B \mid A \mid B, \\
        A &\to a. \\
      \end{aligned}
    \end{equation*}

    The derivations
    \begin{equation*}
      S' \Rightarrow S \Rightarrow B
    \end{equation*}
    and
    \begin{equation*}
      S' \Rightarrow S \Rightarrow A B \Rightarrow a B
    \end{equation*}
    cannot be expanded further in order to reach a word consisting entirely of terminals.

    Thus, the rules \( S \to B \) and \( S \to A B \) are essentially useless, but do no harm unless the number of rules matters as in \fullref{alg:brute_force_parsing}.
  \end{thmenum}
\end{example}

\begin{proposition}\label{thm:context_free_languages_are_context_sensitive}
  \hyperref[def:chomsky_hierarchy/context_free]{Context-free} \hi{languages} are \hyperref[def:chomsky_hierarchy/context_sensitive]{context-sensitive}.
\end{proposition}
\begin{proof}
  \Fullref{alg:epsilon_rule_removal} shows that every context-free grammar can be converted to another context-free grammar that is also context-sensitive.
\end{proof}

\begin{definition}\label{def:renaming_rule}\mimprovised
  A \term{renaming rule} in a formal grammar is a production rule \( A \to B \), where both \( A \) and \( B \) are nonterminals.
\end{definition}

\begin{algorithm}[Renaming rule collapse]\label{alg:renaming_rule_collapse}
  Fix an \hyperref[def:epsilon_free_grammar]{essentially \( \varepsilon \)-free} \hyperref[def:chomsky_hierarchy/context_free]{context-free} grammar \( G = (V, \Sigma, \to, S) \). We will build an \hyperref[def:formal_grammar/equivalent]{equivalent} context-free grammar \( G' = (V, \Sigma, \to', S) \) without \hyperref[def:renaming_rule]{renaming rules}.

  \begin{thmenum}
    \thmitem{alg:renaming_rule_collapse/rules} For every rule \( A \to w \), we define a \hyperref[def:function]{set-valued map} \( C(A \to w, U) \) whose values are words \( v \) over \( V \cup \Sigma \) for which \( A \to v \) is a non-renaming rule. The parameter \( U \) is a set of nonterminals \enquote{already traversed} during recursion, we use it to avoid unbounded recursion like in the case of \( A \to B \to A \). Let
    \begin{equation*}
      C(A \to w, U) \coloneqq \begin{cases}
        \varnothing,                                    &A \in U, \\
        \bigcup_{B \to v} C(B \to v, U \cup \set{ A }), &w = B, \\
        \set{ w },                                      &\T{otherwise.}
      \end{cases}
    \end{equation*}

    \thmitem{alg:renaming_rule_collapse/new_grammar} Let \( A \to' v \) hold if and only if \( v \in C(A \to w, \varnothing) \) for some rule \( A \to w \). The obtained quadruple \( G' = (V', \Sigma, \to', S) \) is the desired grammar without renaming rules.
  \end{thmenum}
\end{algorithm}
\begin{comments}
  \item This algorithm can be found as \texttt{grammars.renaming\_rules.collapse\_renaming\_rules} in \cite{code}.
  \item Our goal is to \enquote{collapse} derivations like \( A \to B \to C \to w \) into \( A \to' w \).
\end{comments}

\paragraph{Regular languages}

\begin{definition}\label{def:reverse_grammar}\mcite[17]{Salomaa1987}
  We define the \term{reverse grammar} of the \hyperref[def:chomsky_hierarchy/context_free]{context-free} \( G = (V, \Sigma, \to, S) \) as
  \begin{equation*}
    \op{rev}(G) \coloneqq (V, \Sigma, \to_{\op{rev}}, S),
  \end{equation*}
  where
  \begin{equation*}
    A \to_{\op{rev}} \op{rev}(w) \T{if} A \to w.
  \end{equation*}
\end{definition}
\begin{comments}
  \item For example, the rule \( A \to Ba_1 \cdots a_n \) becomes \( A \to_{\op{rev}} a_n \cdots a_1 B \).
\end{comments}

\begin{proposition}\label{thm:reverse_linear_grammar}
  The \hyperref[def:reverse_grammar]{reverse} of a \hyperref[def:chomsky_hierarchy/regular]{left linear grammar} is \hyperref[def:chomsky_hierarchy/regular]{right linear} and vice versa.
\end{proposition}
\begin{proof}
  Trivial.
\end{proof}

\begin{proposition}\label{thm:reverse_grammar_language}
  For a given \hyperref[def:chomsky_hierarchy/context_free]{context-free} grammar \( G = (V, \Sigma, \to, S) \), we have
  \begin{equation*}
    \mscrL(\op{rev}(G)) = \op{rev}(\mscrL(G)).
  \end{equation*}
\end{proposition}
\begin{proof}
  Trivial.
\end{proof}

\begin{algorithm}[Finite automaton from regular grammar]\label{alg:finite_automaton_from_regular_grammar}
  Let \( G = (V, \Sigma, \to, S) \) be a \hyperref[def:chomsky_hierarchy/regular]{regular grammar}. We will construct a \hyperref[def:finite_automaton]{finite automaton} that \hyperref[def:finite_automaton/language]{accepts} \( \mscrL(G) \).

  \begin{thmenum}
    \thmitem{alg:finite_automaton_from_regular_grammar/init} Let \( G_1 = (V, \Sigma, \to_1, S) \) be \( G_1 \) if it is right linear and \( \op{rev}(G) \) if it is left linear. Then \( G_1 \) is necessarily right linear.

    \thmitem{alg:finite_automaton_from_regular_grammar/epsilon} Let \( G_2 = (V_2, \Sigma, \to_2, S_2) \) be the grammar obtained from \( G_1 \) by removing \( \varepsilon \) rules via \Fullref{alg:epsilon_rule_removal}.

    \thmitem{alg:finite_automaton_from_regular_grammar/collapse} Let \( G_3 = (V_2, \Sigma, \to_3, S_2) \) be the grammar obtained from \( G_2 \) by collapsing renaming rules via \fullref{alg:renaming_rule_collapse}.

    \thmitem{alg:finite_automaton_from_regular_grammar/intermediate} Build another intermediate grammar \( G_4 = (V_4, \Sigma, \to_4, S_2) \) as follows:
    \begin{itemize}
      \item For each rule \( A \to_3 w \), \( w = a_1 \cdots a_n \), instead consider the sequence of rules
      \begin{align*}
        A       &\to_4 a_1 A_1, \\
        A_1     &\to_4 a_2 A_2, \\
                &\vdots \\
        A_{n-1} &\to_4 a_n,
      \end{align*}
      where \( A_1, \ldots, A_{n-1} \) are new nonterminals.

      \item For each rule \( A \to_3 wB \) with \( \len(w) > 0 \), consider a similar sequence, but the last rule being
      \begin{equation*}
        A_{n-1} \to_4 a_n B.
      \end{equation*}
    \end{itemize}

    Thus, every rule in \( G_4 \) has one of the forms \( A \to_4 a \) or \( A \to_4 a B \) or \( A \to_4 B \).

    \thmitem{alg:finite_automaton_from_regular_grammar/automaton}\mcite[thm. 4.10.1]{Savage1998} Let \( F \) be some new nonterminal symbol. Then the following is a finite automaton that accepts \( \mscrL(G_4) \), and hence also \( G_2 \) and \( G_1 \):
    \begin{itemize}
      \item \( \Sigma \) is the alphabet.
      \item \( V_4 \cup \set{ F } \) is the set of states.
      \item \( S_2 \) the only starting state.
      \item \( F \) is a final state. \( S_2 \) is also a final state if \( \varepsilon \in \mscrL(G) \).
      \item Add the following transitions:
      \begin{itemize}
        \item \( \delta(A, a) \coloneqq F \) if \( A \to_4 a \).
        \item \( \delta(A, a) \coloneqq B \) if \( A \to_4 aB \).
      \end{itemize}
    \end{itemize}

    \thmitem{alg:finite_automaton_from_regular_grammar/reverse} If \( G \) is right linear, then \( F \) is the desired automaton because
    \begin{equation*}
      \mscrL(F) = \mscrL(G) = \mscrL(G_4).
    \end{equation*}

    Otherwise, we take the \hyperref[def:reverse_finite_automaton]{reverse automaton} \( \op{rev}(F) \) because
    \begin{equation*}
      \mscrL(\op{rev}(F))
      \reloset {\ref{thm:reverse_finite_automaton_language}} =
      \op{rev}(\mscrL(F))
      =
      \op{rev}(\mscrL(G_4))
      =
      \op{rev}(\mscrL(\op{rev}(G)))
      \reloset {\ref{thm:reverse_grammar_language}} =
      \op{rev}(\op{rev}(\mscrL(G)))
      \reloset {\ref{thm:reverse_language_involution}} =
      \mscrL(G).
    \end{equation*}
  \end{thmenum}
\end{algorithm}
\begin{comments}
  \item This algorithm can be found as \texttt{grammars.regular.to\_finite\_automaton} in \cite{code}.
\end{comments}

\begin{algorithm}[Right-linear grammar from finite automaton]\label{alg:right_linear_grammar_from_finite_automaton}\mcite[thm. 4.10.1]{Savage1998}
  Let \( F = (\Sigma, Q, \delta, I, T) \) be a \hyperref[def:finite_automaton]{finite automaton}. We will build a \hyperref[def:chomsky_hierarchy/regular]{right linear grammar} \( G = (V, \Sigma, \to, S) \) that \hyperref[def:formal_grammar/language]{generates} \( \mscrL(F) \).

  \begin{thmenum}
    \thmitem{alg:right_linear_grammar_from_finite_automaton/determinize} Use \fullref{alg:determinization_of_finite_automata} to obtain a deterministic automaton \( \det(F) = (\Sigma, Q', \delta', \set{ I }, T') \) equivalent to \( F \).

    \thmitem{alg:right_linear_grammar_from_finite_automaton/grammar} The following describes the desired grammar:
    \begin{itemize}
      \item \( \Sigma \) is the set of terminals.
      \item \( Q' \) is the set of nonterminals.
      \item \( I \) is the starting nonterminal.
      \item The following are rules:
      \begin{itemize}
        \item \( A \to aB \) if \( \delta'(A, a) = B \).
        \item \( A \to \varepsilon \) for each terminal state \( A \in T' \).
      \end{itemize}
    \end{itemize}
  \end{thmenum}
\end{algorithm}
\begin{comments}
  \item This algorithm can be found as \texttt{grammars.regular.from\_finite\_automaton} in \cite{code}.
\end{comments}

\begin{proposition}\label{thm:regular_languages}
  The following are equivalent for a given \hyperref[def:formal_language/language]{language}:
  \begin{thmenum}
    \thmitem{thm:regular_languages/right} It is \hyperref[def:formal_grammar/language]{generated} by a \hyperref[def:chomsky_hierarchy/regular]{right linear grammar}.
    \thmitem{thm:regular_languages/left} It is \hyperref[def:formal_grammar/language]{generated} by a \hyperref[def:chomsky_hierarchy/regular]{left linear grammar}.
    \thmitem{thm:regular_languages/automata} It is \hyperref[def:finite_automaton/language]{recognized} by a \hyperref[def:finite_automaton]{finite automaton}.
  \end{thmenum}
\end{proposition}
\begin{proof}
  \ImplicationSubProof{thm:regular_languages/right}{thm:regular_languages/left} Let \( G = (V, \Sigma, \to, S) \) be a right-regular grammar. We will describe a procedure for obtaining an equivalent left-regular grammar.

  \begin{itemize}
    \item \Fullref{alg:finite_automaton_from_regular_grammar} gives us a finite automaton \( F = (\Sigma, Q, \delta, \set{ S }, T) \) such that \( \mscrL(F) = \mscrL(G) \).

    \item Take the \hyperref[def:reverse_finite_automaton]{reverse automaton} \( \op{rev}(F) \) of \( F \).

    \item Determinize \( \op{rev}(F) \) via \fullref{alg:determinization_of_finite_automata}.

    \item Use \fullref{alg:right_linear_grammar_from_finite_automaton} to convert \( \det(\op{rev}(F)) \) to a right-regular grammar \( G' \). At this point, we have
    \begin{equation*}
      \mscrL(G')
      =
      \mscrL(\det(\op{rev}(F)))
      =
      \mscrL(\op{rev}(F))
      \reloset {\ref{thm:reverse_finite_automaton_language}} =
      \op{rev}(\mscrL(F)).
    \end{equation*}

    \item Finally, take the \hyperref[def:reverse_grammar]{reverse grammar} \( \op{rev}(G') \). It is a left-regular grammar as a consequence of \fullref{thm:reverse_linear_grammar}. Furthermore,
    \begin{equation*}
      \mscrL(\op{rev}(G'))
      \reloset {\ref{thm:reverse_grammar_language}} =
      \op{rev}(\mscrL(G'))
      =
      \op{rev}(\op{rev}(\mscrL(F)))
      \reloset {\ref{thm:reverse_language_involution}} =
      \mscrL(F).
    \end{equation*}

    Hence, \( \op{rev}(G') \) is the desired left linear grammar.
  \end{itemize}

  \ImplicationSubProof{thm:regular_languages/left}{thm:regular_languages/automata} \Fullref{alg:finite_automaton_from_regular_grammar} allows us to convert every left linear grammar to a finite automaton.

  \ImplicationSubProof{thm:regular_languages/automata}{thm:regular_languages/right} \Fullref{alg:right_linear_grammar_from_finite_automaton} allows us to convert every finite automaton to a right linear grammar.
\end{proof}
