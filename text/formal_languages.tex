\subsection{Formal languages}\label{subsec:formal_languages}

Languages are used to define formulas for expressing the \hyperref[def:zfc]{axioms of set theory}. Here, sets are used to formally define languages. A simple way out of this vicious cycle is via the theory-metatheory relationship discussed in \fullref{rem:metalogic} and \fullref{rem:set_definition_recursion}. In short, we define languages within the metatheory using the already available concept of set, and we later define formulas, again in the metatheory, which allows us to subsequently formally define sets via axioms within the object logic.

\begin{definition}\label{def:formal_language}
  Fix a nonempty set \( \mscrA \).

  \begin{thmenum}
    \thmitem{def:formal_language/alphabet} We call \( \mscrA \) an \term{alphabet}.

    \thmitem{def:formal_language/symbol} We call each element of \( \mscrA \) a \term{symbol}.

    \thmitem{def:formal_language/word} A \term{string} over \( \mscrA \) is a \hyperref[def:sequence]{finite sequence} of symbols. If \( (a, b, c) \) is a word, for convenience we write it as the string \( abc \). This is the reason words are also referred to as \term{strings}. This notation only makes sense if each symbol of the language is actually represented by one typographic symbol.

    The term \enquote{string} is common in programming practice. In the context of formal languages, strings are often called \term{words}. We will avoid the later term since it does not correspond to the everyday use of the term \enquote{word}.

    \thmitem{def:formal_language/empty_word} We denote the empty string by \( \varepsilon \).

    \thmitem{def:formal_language/word_length} The \term{length} \( \len(w) \) of a word \( w \) is the number of elements of the tuple \( w \).

    \thmitem{def:formal_language/concatenation} The \term{concatenation} of the words \( v = (v_1, \ldots, v_n) \) and \( w = (w_1, \ldots, w_m) \) is the word
    \begin{equation*}
      vw \coloneqq (v_1, \ldots, v_n, w_1, \ldots, w_m).
    \end{equation*}

    We abbreviate \( \overbrace{w w \ldots w}^{k \T{times}} \) as \( w^k \). This is only a notation. We do not distinguish, formally, between the words \( aaabbaa \) and \( a^3 b^2 a^2 \), nor between \( a \varepsilon b \) and \( ab \).

    \thmitem{def:formal_language/reverse} The \term{reverse word} of \( w = (w_1, \ldots, w_n) \) is
    \begin{equation*}
      \op{rev}(w) \coloneqq (w_n, \ldots, w_1).
    \end{equation*}

    \thmitem{def:formal_language/prefix} The word \( p = (p_1, \ldots, p_m) \) is a \term{prefix} of \( w = (w_1, \ldots, w_n) \) if
    \begin{equation*}
      w = (\underbrace{p_1, \ldots, p_m}_p, w_{m+1}, \ldots, w_n).
    \end{equation*}

    \thmitem{def:formal_language/suffix} The word \( s \) is a \term{suffix} of \( w \) if \( \op{rev}(s) \) is a prefix of \( \op{rev}(w) \).

    \thmitem{def:formal_language/subword} The word \( v \) is a \term{subword} of \( w \) if there exists a prefix \( p \) and a suffix \( s \) of \( v \) such that
    \begin{equation*}
      w = pvs.
    \end{equation*}

    \thmitem{def:formal_language/kleene_star} The \term{Kleene star} \( \mscrA^* \) of \( \mscrA \) is the set of all words over \( \mscrA \). If we wish to exclude the empty word, like we often do, we instead write \( \mscrA^+ \) for the set of all non-empty words over \( \mscrA \).

    The Kleene star is also called the free monoid on \( A \) --- see \fullref{thm:free_monoid_universal_property}.

    \thmitem{def:formal_language/language} A \term{language} over \( \mscrA \) is any subset of \( \mscrA^* \). Note that in some contexts like \hyperref[subsec:propositional_logic]{propositional logic} or \hyperref[subsec:first_order_logic]{first-order logic} the term \enquote{language} may refer to the alphabet itself; see \fullref{rem:propositional_language_is_alphabet} for a further discussion.
  \end{thmenum}
\end{definition}

\begin{definition}\label{def:formal_grammar}\mcite[def. 2.2]{Sipser2013}
  Let \( V \) and \( \Sigma \) be disjoint nonempty subsets of some \hyperref[def:formal_language/alphabet]{alphabet}.

  \begin{thmenum}
    \thmitem{def:formal_grammar/terminals} We call elements of \( \Sigma \) \term{terminals}. We denote terminals in abstract grammars using lowercase Greek letters, and we denote words using lowercase Latin letters.

    \thmitem{def:formal_grammar/non_terminals} We call elements of \( V \) \term{non-terminals}. By convention, variables are denoted using capital letters.

    \thmitem{def:formal_grammar/start} We assume that a special \term{start symbol} \( S \in V \) is fixed.

    \thmitem{def:formal_grammar/production_rules} We define a binary \hyperref[def:relation]{relation} \( \to \) of \term{production rules} over \( (V \cup \Sigma)^* \).

    We impose the restriction that no rules of the form \( \varepsilon \to v \) exist for any word \( v \). We do allow, however, production rules of the form \( v \to \varepsilon \). Such rules are called \term{\( \varepsilon \)-rules}.

    Rules describe transformations that define how a language is \enquote{generated} starting from \( S \). See \fullref{def:grammar_derivation} and \fullref{ex:natural_number_arithmetic_grammar/derivation}.

    \thmitem{def:formal_grammar/grammar} The quadruple \( G \coloneqq (V, S, \Sigma, \to) \) is called a \term{formal grammar} or simply a \term{grammar}.
  \end{thmenum}
\end{definition}

\begin{definition}\label{def:chomsky_hierarchy}
  We will define the \term{Chomsky hierarchy} of \hyperref[def:formal_grammar]{formal grammars}. We can classify a grammar \( G = (V, S, \Sigma, \to) \) as follows, based on their rules:
  \begin{thmenum}
    \thmitem{def:chomsky_hierarchy/unrestricted} In general, every production rule has the form \( v \to w \), where both \( v \) and \( w \) are words consisting of terminal and non-terminals, and \( v \) is nonempty.

    When no additional restrictions are imposed on the rules of the grammar, we call it \term{unrestricted grammar}. The other levels of the hierarchy are subsets of the unrestricted grammars.

    \thmitem{def:chomsky_hierarchy/non_contracting} The grammar is \term{non-contracting} if \( \len(v) \leq \len(w) \) for every production rule \( v \to w \).

    \thmitem{def:chomsky_hierarchy/context_sensitive} The grammar is \term{context-sensitive} if every rule has the form \( aAb \to w \) for some non-terminal \( A \), arbitrary words \( a \) and \( b \), and a nonempty word \( w \).

    The requirement for \( w \) to be nonempty is set up so that context-sensitive grammars are non-contracting.

    \thmitem{def:chomsky_hierarchy/context_free} The grammar is \term{context-free} every rule has the form \( A \to w \) for some non-terminal \( A \) and a nonempty word \( w \).

    Unlike for context-sensitive languages, \( w \) is allowed to be empty. Thus, a context-free grammar is non-contracting, however it may not be context-sensitive if it has \( \varepsilon \)-rules.

    \thmitem{def:chomsky_hierarchy/regular} Finally, the grammar is \term{regular} if every rule has one of the forms
    \begin{align*}
      &A \to \varepsilon, \\
      &A \to B \tau, \\
      &A \to \tau B, \\
      &A \to \tau,
    \end{align*}
    where \( A \) and \( B \) are non-terminals and \( \tau \) is a terminal.

    Regular grammars are obviously context-free.
  \end{thmenum}
\end{definition}

\begin{example}\label{ex:natural_number_arithmetic_grammar/backus_naur_form}
  We define a grammar for primary school notation of multiplication and division of \hyperref[def:natural_numbers]{natural numbers}. Note that we consider the numbers in \( \BbbN \) only as symbols, without any regard to semantics.

  Let \( V \coloneqq \set{ N, O, M, E } \) and \( \Sigma \coloneqq \BbbN \cup \set{ \times, \div, (, ) } \). Define the grammar
  \begin{equation}\label{eq:ex:natural_number_arithmetic_grammar/backus_naur_form/simple}
    \begin{aligned}
      N &\to 0 \\
      N &\to 1 \\
      \phantom{N} &\vdots \\
      N &\to n \\
      \phantom{N} &\vdots \\
      O &\to \times \\
      O &\to \div \\
      E &\to N \\
      E &\to (E O E)
    \end{aligned}
  \end{equation}

  It is convenient to use the following shorthands:
  \begin{equation}\label{eq:ex:natural_number_arithmetic_grammar/backus_naur_form/shorthand}
    \begin{aligned}
      N &\to 0 \mid 1 \mid 2 \mid \ldots \\
      O &\to \times \mid \div \\
      E &\to N \mid (E O E)
    \end{aligned}
  \end{equation}

  We can choose different non-terminals as the starting symbol. The symbol \( N \) corresponds to numbers, \( O \) corresponds to operations, and \( E \) can be either a number or an expression. We say that \eqref{eq:ex:natural_number_arithmetic_grammar/backus_naur_form/shorthand} specifies a \term{grammar schema}. With any starting symbol, the grammar is clearly \hyperref[def:chomsky_hierarchy]{context-free}.
\end{example}

\begin{remark}\label{rem:backus_naur_form}
  The infinitude of possible rules in \fullref{ex:natural_number_arithmetic_grammar/backus_naur_form} may not bother us formally, but when dealing with software implementations, we must have a finite number of rules. An example of a nontrivial grammar in the wild is the Python grammar that can be found in \cite{Python39Grammar}. There are also other advantages of introducing a more convenient metasyntax (a syntax for describing language syntax).

  For \hyperref[def:chomsky_hierarchy/context_free]{context-free grammars}, is often convenient to use the \term{Backus-Naur form (BNF)}. For \fullref{ex:natural_number_arithmetic_grammar/backus_naur_form}, the BNF is
  \begin{bnf*}
    \bnfprod{nonzero digit} {\bnfts{1} \bnfor \bnfts{2} \bnfor \bnfts{3} \bnfor \bnfts{4} \bnfor \bnfts{5} \bnfor \bnfts{6} \bnfor \bnfts{7} \bnfor \bnfts{8} \bnfor \bnfts{9}} \\
    \bnfprod{digit}         {\bnfts{0} \bnfor \bnfpn{nonzero digit}} \\
    \bnfprod{number}        {\bnfpn{nonzero digit} \bnfor \bnfpn{number} \bnfsp \bnfpn{digit}} \\
    \bnfprod{operation}     {\bnfts{\( \times \)} \bnfor \bnfts{\( \div \)}} \\
    \bnfprod{expression}    {\bnfpn{number} \bnfor \bnfts{(} \bnfsp \bnfpn{number} \bnfsp \bnfpn{operation} \bnfsp \bnfpn{number} \bnfsp \bnfts{)}}.
  \end{bnf*}

  The obvious difference is that we explicitly define numbers via their decimal representation, which means that we get a finite amount of rules. Compared to \eqref{eq:ex:natural_number_arithmetic_grammar/backus_naur_form/simple} some other differences are:
  \begin{itemize}
    \item Variables are denoted by \( \langle \)words enclosed in angle brackets\( \rangle \), so that we can name variables more descriptively using more than one symbol.
    \item Terminals are, by convention, put in \enquote{quotes}. In human-readable rich text documents like this one, it is sometimes possible to use different fonts, and so instead of using \enquote{quotes} we specify terminals using an \texttt{upright typewriter font}.
    \item Free-text rules can be specified using a normal font. This is also only used in human-readable rich text documents, however this usage is justified because such rules are only beneficial for human understanding and not for machine parsing.
    \item By convention, the symbol \( \Coloneqq \) is used instead of \( \to \) for specifying transition rules.
    \item Different rules with the same, source are concatenated as in \eqref{eq:ex:natural_number_arithmetic_grammar/backus_naur_form/shorthand}.
    \item In order to fully describe a context-free grammar, we must only specify its Backus-Naur form and its starting variable.
  \end{itemize}
\end{remark}

\begin{definition}\label{def:grammar_derivation}
  Fix a \hyperref[def:formal_grammar]{formal grammar} \( G = (V, S, \Sigma, \to) \).

  \begin{thmenum}
    \thmitem{def:grammar_derivation/derivation} We define the binary relation \( \Rightarrow \) on the Kleene star \( (V \cup \Sigma)^* \) by declaring that, for every two \hyperref[def:formal_language/word]{words} \( p \) and \( s \) over \( V \cup \Sigma \) and every production rule \( v \to w \), we have \( pvw \Rightarrow pws \). We also define the relation \( \Rightarrow_L \) as a restriction of \( \Rightarrow \) to the cases where \( p \) contains only terminal symbols and \( \Rightarrow_R \) --- if \( s \) contains only terminal symbols.

    A \term{derivation} of the word \( w_n \) from \( w_1 \) is a \hyperref[def:quiver_path/directed]{directed path} in the quiver induced by the relation \( \Rightarrow \), i.e.
    \begin{equation}\label{eq:def:grammar_derivation/derivation}
      w_1
      \reloset {u_1 \to v_1} \implies
      w_2
      \reloset {u_2 \to v_2} \implies
      \cdots
      \reloset {u_{n-2} \to v_{n-2}} \implies
      w_{n-1}
      \reloset {u_{n-1} \to v_{n-1}} \implies
      w_n.
    \end{equation}

    A \term{leftmost derivation} is a derivation performed using \( \Rightarrow_L \) rather than \( \Rightarrow \). \term{Rightmost derivations} are defined analogously.

    We say that \( w_n \) is \term{derivable} from \( w_1 \) if there exists a derivation from \( w_1 \) to \( w_n \).

    We denote the \hyperref[def:relation_closures/transitive]{transitive closure} of \( \Rightarrow \) by \( \reloset + \Rightarrow \) and the \hyperref[def:relation_closures/reflexive]{reflexive} closure of \( \reloset + \Rightarrow \) by \( \reloset {*} \Rightarrow \). Clearly \( w_1 \) is derivable from \( w_n \) if and only if \( w_1 \reloset {*} \Rightarrow w_n \).

    The leftmost and rightmost derivations generate the same derivability relation --- the only potential difference is in the order of rule applications in the derivation itself.

    \thmitem{def:grammar_derivation/unambiguous}\mcite[def. 2.7]{Sipser2013} We say that the word \( w \) can be derived \term{unambiguously} if it has a unique leftmost derivation.

    Define the set
    \begin{equation*}
      D \coloneqq \set{ w \in (V \cup \Sigma)^* \colon S \reloset + \Rightarrow w }
    \end{equation*}
    of all words derivable from the starting symbol \( S \).

    If every word in \( D \) can be derived unambiguously, we say that the grammar itself is \term{unambiguous}.

    In an unambiguous grammar, \fullref{thm:structural_induction_on_unambiguous_grammars} can be used on the \hyperref[def:quiver/simple]{simple directed graph} \( (D, \Rightarrow_L) \). Indeed, every word in \( D \) is derivable from \( S \) and every leftmost derivation is unique.

    \thmitem{def:grammar_derivation/grammar_language} The \term{language} of the grammar is the set
    \begin{equation*}
      \mscrL(G) \coloneqq \set{ w \in \Sigma^* \colon S \reloset + \Rightarrow w }
    \end{equation*}
    of all terminal-only words derivable from the starting symbol \( S \).

    We also say that strings in \( \mscrL(G) \) are \term{generated} by the grammar \( G \).

    If a language can be generated by a \hyperref[def:chomsky_hierarchy/regular]{regular} grammar, we say that the language itself is regular, and similarly for \hyperref[def:chomsky_hierarchy/context_free]{context-free} and \hyperref[def:chomsky_hierarchy/context_sensitive]{context-sensitive} grammars.

    For other grammars, for example \hyperref[def:chomsky_hierarchy/unrestricted]{unrestricted} or \hyperref[def:chomsky_hierarchy/non_contracting]{non-contracting}, such a terminology is not established.
  \end{thmenum}
\end{definition}

\begin{example}\label{ex:natural_number_arithmetic_grammar/derivation}
  We continue \fullref{ex:natural_number_arithmetic_grammar/backus_naur_form}. Depending on our choice of starting symbol, we can derive different sets of words.

  For the sake of simplifying our exposition and proof, however, we will assume the simpler grammar described in \eqref{eq:ex:natural_number_arithmetic_grammar/backus_naur_form/simple}.

  Choose the starting symbol to be \( E \). We will show that this grammar is unambiguous.

  \Cref{fig:ex:natural_number_arithmetic_grammar/derivation/ambiguous} demonstrates that removing the parentheses makes even this simple grammar ambiguous.

  \begin{figure}[!ht]
    \hfill
    \includegraphics[page=1]{output/ex__natural_number_arithmetic_grammar__derivation.pdf}
    \hfill\hfill
    \caption{The unique leftmost derivation of the parenthesized arithmetic expression \( ((6 \div 2) \times 3) \)}
    \label{fig:ex:natural_number_arithmetic_grammar/derivation/unambiguous}
  \end{figure}

  \begin{figure}[!ht]
    \hfill
    \includegraphics[page=2]{output/ex__natural_number_arithmetic_grammar__derivation.pdf}
    \hfill\hfill
    \caption{Different leftmost derivations of the parenthesis-less arithmetic expression \( 6 \div 3 \times 2 \)}
    \label{fig:ex:natural_number_arithmetic_grammar/derivation/ambiguous}
  \end{figure}

  We will show that \( G \) is unambiguous. Let \( w \) be a word in \( \mscrL(G) \). We explicitly build the leftmost derivation of \( w \) using recursion on \( \len(w) \):
  \begin{itemize}
    \item If \( \len(w) = 1 \), then \( w = n \in \BbbN \), and the word has been derived as \( E \to N \to n \).

    \item Assume that \( w \) is unambiguously derived for \( \len(w) < m + 2 \) and let \( \len(w) = m + 2 \), then \( w \) is necessarily enclosed in parentheses. Let \( w = ( \sigma_1 \ldots \sigma_m ) \) be the symbols of \( w \). Because of the parentheses, the only possibility for \( \sigma_1 \ldots \sigma_m \) is that it consists of two words in \( \mscrL(G) \) with either a multiplication symbol \( \times \) or a division symbol \( \div \) between them. Let \( k \) be the index of the operator symbol, that is, the index such that \( \sigma_1 \ldots \sigma_{k-1} \) and \( \sigma_{k+1} \ldots \sigma_m \) both belong to \( \mscrL(G) \).

    By the inductive hypothesis, both \( \sigma_1 \ldots \sigma_{k-1} \) and \( \sigma_{k+1} \ldots \sigma_m \) are unambiguously derived from \( E \). Then \( w \) is generated by the rule \( E \to (E O E) \), where the operator symbol \( \sigma_k \) determines the terminal of \( O \). Therefore, the derivation of \( w \) is also unambiguous.
  \end{itemize}
\end{example}

\begin{definition}\label{def:ordered_arborescence}
  An \term{ordered arborescence} is an \hyperref[def:arborescence]{arborescence} \( T = (G, A) \) with a \hyperref[def:partially_ordered_set]{partial order} \( \leq \) such that every set of \hyperref[def:arborescence/ancestry]{siblings} is a \hyperref[def:partially_ordered_set_chain_and_antichain]{chain}.
\end{definition}

\begin{definition}\label{def:concrete_syntax_tree}\mimprovised
  Fix a \hyperref[def:chomsky_hierarchy/context_free]{context-free} \hyperref[def:formal_grammar]{formal grammar} \( G = (V, S, \Sigma, \to) \).

  For every word \( w \) in \( \mscrL(G) \), we will build an \hyperref[def:ordered_arborescence]{ordered} \hyperref[def:arborescence/undirected]{rooted tree} whose \hyperref[def:arborescence/ancestry]{leaves} are the symbols of \( w \) and whose root is \( S \). We will call this a \term{concrete syntax tree} for \( w \).

  \begin{figure}[!ht]
    \hfill
    \includegraphics[page=1]{output/def__concrete_syntax_tree.pdf}
    \hfill\hfill
    \caption{A concrete syntax tree for the expression \( (6 \div (3 \times 2)) \) from \cref{fig:ex:natural_number_arithmetic_grammar/derivation/unambiguous}}
    \label{fig:def:concrete_syntax_tree}
  \end{figure}

  Fix a derivation
  \begin{equation}\label{eq:def:concrete_syntax_tree/derivation}
    S \Rightarrow w_1 \Rightarrow \cdots \Rightarrow w_{n-1} \Rightarrow w_n.
  \end{equation}

  In practice, we want this to be unique and hence we can restrict ourselves to leftmost derivations in \hyperref[def:grammar_derivation/unambiguous]{unambiguous} grammars.

  We use \hyperref[rem:natural_number_recursion]{natural number recursion} on \( n \) to build the tree. Note that, for the purposes of recursion, we allow \( w_n \) to contain non-terminals.

  \begin{itemize}
    \item In the trivial case where \( n = 0 \), and there is no actual derivation, we build a single-vertex tree with root \( S \).

    \item Suppose that we can build a tree for all derivations of length \( m - 1 \) and fix a derivation \eqref{eq:def:concrete_syntax_tree/derivation} of length \( n \).

    First, build a tree \( T \) from the derivation
    \begin{equation*}
      S \Rightarrow w_1 \Rightarrow \cdots \Rightarrow w_{n-2} \Rightarrow w_{n-1}.
    \end{equation*}

    There must exist words \( p \), \( s \) and a rule \( A \to v \) such that
    \begin{equation*}
      w_{n-1} = pAs \Rightarrow pvs = w_n.
    \end{equation*}

    There already exists a leaf for \( A \) in \( T \). For every symbol in \( v \), add a new node as a child of this node.
  \end{itemize}
\end{definition}

\begin{theorem}[Structural induction on unambiguous grammars]\label{thm:structural_induction_on_unambiguous_grammars}\mimprovised
  Unlike for the other induction principles in \fullref{rem:induction}, we will not formulate this one via logical formulas. This will complicate us unnecessarily. We will instead describe how the principle is used. It should be noted that, in applications of this principle, we prefer using \hyperref[rem:abstract_syntax_tree]{abstract syntax trees} to \hyperref[def:concrete_syntax_tree]{concrete syntax trees}, but formulating the principle itself is easier via concrete syntax trees.

  Let \( G = (V, S, \Sigma, \to) \) be an \hyperref[def:chomsky_hierarchy/context_free]{unambiguous} \hyperref[def:chomsky_hierarchy/context_free]{context-free} \hyperref[def:formal_grammar]{formal grammar}.

  Suppose that we want to prove a statement for every word in \( \mscrL(G) \). It is sufficient to perform the following for every word \( w \):
  \begin{displayquote}
    Let \( A_1, \ldots, A_n \) be all non-terminals of \( w \) and let \( u_0, \ldots, u_n \) be subwords of \( w \) such that
    \begin{equation*}
      w = u_0 A_1 u_1 A_2 \ldots A_n u_n.
    \end{equation*}

    Let \( v_1, \ldots, v_n \) be arbitrary words in \( \mscrL(G) \) derivable from \( A_1, \ldots, A_n \), respectively, so that we have the concrete syntax tree
    \begin{equation*}
      \begin{aligned}
        \includegraphics[page=1]{output/thm__structural_induction_on_unambiguous_grammars.pdf}
      \end{aligned}
    \end{equation*}

    Then we must prove the statement for the word
    \begin{equation*}
      u_0 v_1 u_1 \ldots v_n u_n.
    \end{equation*}
  \end{displayquote}

  Compare this principle to the more general \fullref{thm:well_founded_induction}.
\end{theorem}
\begin{proof}
  Clearly every word in \( \mscrL(G) \) can be obtained in this way. The principle itself follows from \fullref{thm:well_founded_induction} even without non-ambiguity.

  Non-ambiguity ensures that every word has a unique concrete syntax tree, and prevents us from proving the statement for one tree of a word and disproving it for another.
\end{proof}

\begin{remark}\label{rem:abstract_syntax_tree}
  Unlike \hyperref[def:concrete_syntax_tree]{concrete syntax trees}, which focus on how words are built from symbols, \term{abstract syntax trees} focus on how words are built with respect to semantics. For example, natural number arithmetic is not concerned with the symbols themselves but only in how operations are applied to numbers. This is better expressed via the abstract syntax tree in \cref{fig:rem:abstract_syntax_tree} rather than the concrete syntax tree in \cref{fig:def:concrete_syntax_tree}.

  A formal definition for an abstract syntax tree would be clunky. It is an \hyperref[def:ordered_arborescence]{ordered} \hyperref[def:arborescence/undirected]{rooted tree} of words that unambiguously encodes a concrete syntax tree. The structure of an abstract syntax tree is determined entirely by the concrete application, and we will not attempt to give a more precise definition. Dirk van Dalen in \cite{VanDalen2004} uses the term \term{parsing tree} for a similar notion.

  Applications of \fullref{thm:structural_induction_on_unambiguous_grammars} use abstract syntax trees as can be seen in \fullref{subsec:propositional_logic} and \fullref{subsec:first_order_logic}, and a broader discussion on the properties of certain trees can be found in \fullref{rem:binary_operation_syntax_trees}.

  \begin{figure}[!ht]
    \hfill
    \includegraphics[page=1]{output/rem__abstract_syntax_tree.pdf}
    \hfill\hfill
    \caption{An abstract syntax tree for the expression \( (6 \div (3 \times 2)) \) from \cref{fig:ex:natural_number_arithmetic_grammar/derivation/unambiguous}}
    \label{fig:rem:abstract_syntax_tree}
  \end{figure}
\end{remark}

\begin{remark}\label{rem:binary_operation_syntax_trees}
  \hyperref[def:magma]{Binary operations} are easily extended to higher arities. Given a binary operation \( +: M \times M \to M \), we can extend it via \hyperref[rem:natural_number_recursion]{natural number recursion} to arbitrary \( n \)-tuples \( x_1, \ldots, x_n \) as
  \begin{equation}\label{eq:rem:binary_operation_syntax_trees/expression}
    (x_1 + (x_2 + \cdots + (x_{n-1} + x_n) \cdots )).
  \end{equation}

  This expression corresponds to the \hyperref[rem:abstract_syntax_tree]{abstract syntax tree}
  \begin{equation}\label{eq:rem:binary_operation_syntax_trees/tree/basic}
    \begin{aligned}
      \includegraphics[page=1]{output/rem__binary_operation_syntax_trees.pdf}
    \end{aligned}
  \end{equation}

  Several things can be noted here.
  \begin{thmenum}
    \thmitem{rem:binary_operation_syntax_trees/associativity} When exchanging the order of the parentheses in the expression \eqref{eq:rem:binary_operation_syntax_trees/expression}, only the root is changed in the syntax tree \eqref{eq:rem:binary_operation_syntax_trees/tree/basic}. Therefore, for an \hyperref[def:magma/associative]{associative} binary operation, abstract syntax trees can instead be represented as finite ordered rooted trees like
    \begin{equation}\label{eq:rem:binary_operation_syntax_trees/tree/associative}
      \begin{aligned}
        \includegraphics[page=2]{output/rem__binary_operation_syntax_trees.pdf}
      \end{aligned}
    \end{equation}

    Of course, \( n \)-ary trees can still be used for non-associative binary operations, as long as we have selected a strategy for evaluation. If we evaluate \eqref{eq:rem:binary_operation_syntax_trees/tree/associative} as \eqref{eq:rem:binary_operation_syntax_trees/expression}, we say that the operation is \term{right associative}. Dually, if we evaluate \eqref{eq:rem:binary_operation_syntax_trees/tree/associative} as
    \begin{equation*}
      (( \cdots (x_1 + x_2) + \cdots + x_{n-1}) + x_n),
    \end{equation*}
    we say that the operation is \term{left associative}.

    Note that, unlike associativity, left and right associativity are not properties of the operation, but rather conventions on how to evaluate expressions without explicit parentheses. For example, in a \hyperref[def:heyting_algebra]{Heyting algebra}, the \hyperref[eq:def:heyting_algebra/conditional]{conditional} \( \rightarrow \) is not associative, but it is often taken to be right associative so that
    \begin{equation*}
      x \rightarrow y \rightarrow z
    \end{equation*}
    is evaluated as
    \begin{equation*}
      (x \rightarrow (y \rightarrow z)).
    \end{equation*}

    \thmitem{rem:binary_operation_syntax_trees/commutativity} If, additionally, the operation is \hyperref[def:magma/commutative]{commutative}, we can regard the syntax tree as unordered.

    Note that extending operations can be confusing for commutative but non-associative operations. Commutativity allows us to swap summands inside parentheses, and associativity is needed to \enquote{remove} the parentheses.

    Consider the real number midpoint operation
    \begin{equation*}
      x \oplus y \coloneqq \frac {x + y} 2
    \end{equation*}
    from \fullref{ex:def:magma/midpoint}. It is commutative and not associative, and in the expression \( x \oplus y \oplus z \), we can swap \( x \) with \( y \) but not with \( z \). This can be very unintuitive. We aim to always write parentheses for non-associative operations.

    \thmitem{rem:binary_operation_syntax_trees/infinite} Suppose that \( + \) is associative and commutative. Suppose also that we are given an \hyperref[def:cartesian_product/indexed_family]{indexed family} \( \seq{ x_k }_{k \in \mscrK} \) of elements of \( M \). We can obviously construct a tree with root \( + \) and children \( \seq{ x_k }_{k \in \mscrK} \).

    This is not strictly a syntax tree. Syntax trees are necessarily finite, and the family may even be uncountable. Nevertheless, we can sometimes evaluate this tree to obtain a member of the monoid.
    \begin{thmenum}
      \thmitem{rem:binary_operation_syntax_trees/infinite/lattice} The operations of a \hyperref[def:semilattice/lattice]{lattice} arise by specializing \hyperref[def:partially_ordered_set_extremal_points/supremum_and_infimum]{suprema and infima} to binary sets. Conversely, as discussed in \fullref{thm:binary_lattice_operations/new_lattice}, the binary lattice operations induce a \hyperref[def:partially_ordered_set]{partial order}, and we can define suprema and infima. If the lattice happens to be \hyperref[def:semilattice/complete]{complete}, instead of the binary operations, we can use
      \begin{equation*}
        \bigvee_{k \in \mscrK} x_k = \sup\set{ x_k \given k \in \mscrK }.
      \end{equation*}

      \thmitem{rem:binary_operation_syntax_trees/infinite/convergence} If \( M \) is a \hyperref[rem:topological_first_order_structures]{topological monoid} and if \( \mscrK = \set{ 1, 2, 3, \ldots } \), the family is a \hyperref[def:sequence]{sequence}, and we can define the sequence of partial sums
      \begin{equation*}
        \seq*{ \sum_{k=1}^n x_k }_{n=1}^\infty
      \end{equation*}

      This gives rise to \hyperref[def:convergent_series]{series} discussed in \fullref{subsec:series} and \fullref{subsec:real_series}. A limit may not exist for the net, unfortunately, and if it does, it may not be unique (if the topology is not \hyperref[def:separation_axioms/T2]{Hausdorff}).

      \thmitem{rem:binary_operation_syntax_trees/infinite/direct_sum} Suppose that \( M \) has an \hyperref[def:monoid]{identity element} \( 0_M \). If only finitely many elements of the family are different from \( 0_M \), we regard the ordinary summation operation as well-defined on the whole family and write
      \begin{equation*}
        s \coloneqq \sum_{k \in \mscrK} x_k.
      \end{equation*}

      Technically, this involves selecting a \hyperref[def:well_ordered_set]{well-ordering} \( x_{1_n}, \ldots, x_{k_n} \) on the set
      \begin{equation*}
        \set{ x_k \given k \in \mscrK \T{and} x_k \neq 0 }
      \end{equation*}
      and assigning to \( s \) the result of the iterated binary operation
      \begin{equation*}
        (x_{1_n} + (x_{2_n} + \cdots + (x_{k_{n-1}} + x_{k_n}) \cdots)).
      \end{equation*}

      Commutativity ensures that the sum \( s \) does not depend on the order of summands (and hence on the well-order we have chosen). Adding any member of the family would not change the sum, which justifies this shorthand definition. Furthermore, since we only sum finitely many summands, we can construct a well-ordering using \hyperref[rem:natural_number_recursion]{natural number recursion} without relying on the \hyperref[def:zfc/choice]{axiom of choice}.

      This is fundamental for the definition of \hyperref[def:monoid_direct_product]{direct sums}, which in turn are used to define \hyperref[rem:linear_combinations]{linear combinations} and \hyperref[def:polynomial_algebra]{polynomials}.
    \end{thmenum}
  \end{thmenum}
\end{remark}
