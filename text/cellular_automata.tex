\section{Cellular automata}\label{sec:cellular_automata}

\paragraph{General cellular automata}

\begin{definition}\label{def:cellular_automaton}\mcite[def. 2.4.1]{HadelerMüller2017CellularAutomata}
  Fix a \hyperref[def:cayley_graph]{Cayley graph} \( \Gamma(G, C) \), whose vertices we will call \term[en=cell (\cite[def. 2.1.3]{HadelerMüller2017CellularAutomata})]{cells}. Fix a nonempty set \( E \), whose elements we will call \term[en=elementary state (\cite[def. 2.3.1]{HadelerMüller2017CellularAutomata})]{elementary states}.

  A \term{cellular automaton} is a \hyperref[def:discrete_dynamical_system]{discrete-time} \hyperref[def:dynamical_system]{dynamical systems} whose phase space is \( E^G \), the set of functions assigning states to cells. We will call such a function a \term[en=configuration (\cite[def. 2.3.2]{HadelerMüller2017CellularAutomata})]{configuration}.

  We will assume that the \hyperref[def:one_step_evolution_function]{one-step evolution function} \( \varphi: E^G \to E^G \) is determined by its action on some \hyperref[def:cayley_graph_neighborhood]{neighborhood} \( D_e \) of the origin. More precisely, we need a function \( \varphi_e: E^{D_e} \to E \), which we call a \term[en=local function (\cite[def. 2.4.1]{HadelerMüller2017CellularAutomata})]{local evolution function}. By shifting the neighborhood, we are able to determine \( \varphi \) at any cell. In this context, we will call \( \varphi \) the \term[en=global function (\cite[def. 2.4.1]{HadelerMüller2017CellularAutomata})]{global evolution function}.
\end{definition}
\begin{comments}
  \item We can regard a cellular automaton as a triple \( (\Gamma(G, C), E, \varphi_e) \).
  \item Since the phase space consists of global configuration, \hyperref[def:dynamical_system_trajectory]{trajectories} are sequences of global configuration.
\end{comments}

\paragraph{Automata on the square grid}

\begin{remark}\label{rem:automata_on_square_grid}
  \hyperref[def:cellular_automaton]{Cellular automata} on \hyperref[def:free_abelian_group]{free abelian group} \( \BbbZ^2 \) are particularly useful and simple to visualize.

  Consider the \hyperref[def:parallelogram_tiling]{parallelogram tiling} of the \hyperref[def:euclidean_plane]{Euclidean plane} with \hyperref[def:unit_hypercube]{unit square}. To each cell of the automata there corresponds exactly one square, which we can \hyperref[def:set_coloring]{color} based on its state.

  In this context, we refer to the tiling as a \term{square grid}, and we assume that the connection set of the \hyperref[def:cayley_graph]{Cayley graph} is the standard basis.
\end{remark}

\begin{definition}\label{def:game_of_life}\mimprovised
  One particular \hyperref[def:cellular_automaton]{cellular automaton} on the \hyperref[rem:automata_on_square_grid]{square grid} is called the \term[en=Game of Life (\cite[3]{HadelerMüller2017CellularAutomata})]{game of life}.

  We assume that the states are \hyperref[con:boolean_value]{Boolean values}, where truth corresponds to a cell that is considered \term{alive} and falsity corresponds to one that is considered \term{dead}. We refer to the moments of the automaton as \term[en=generation (\cite[120]{Gardner1970GameOfLife})]{generations}.

  Given the local state \( u: D_{i, j} \to \set{ T, F } \) at the cell \( (i, j) \), the local evolution function simply counts the number of \hi{other} live cells in the \hyperref[def:cayley_graph_neighborhood/moore]{Moore neighborhood} \( D_{i, j} \) as follows:
  \begin{equation}\label{eq:def:game_of_life}
    \varphi_{i,j}(u) \coloneqq \begin{cases}
      u(i, j), &(i, j) \T{has} 2 \T{live neighbors}, \\
      T,       &(i, j) \T{has} 3 \T{live neighbors}, \\
      F,       &\T{otherwise}.
    \end{cases}
  \end{equation}
\end{definition}
\begin{comments}
  \item The game of life was originally published by \incite[120]{Gardner1970GameOfLife} as a puzzle called \enquote{life} and attributed to John Horton Conway. The article suggests placing counters on a physical board, and Gardner describes the rules as follows:
  \begin{displayquote}
    \begin{enumerate}
      \item Survivals. Every counter with two or three neighboring counters survives for the next generation.
      \item Deaths. Each counter with four or more neighbors dies (is removed) from overpopulation. Every counter with one neighbor or none dies from isolation.
      \item Births. Each empty cell adjacent to exactly three neighbors --- no more, no fewer --- is a birth cell. A counter is placed on it at the next move.
    \end{enumerate}
  \end{displayquote}

  It later became popular enough for encyclopedic knowledge to be gathered on it. One such resource is \cite{ConwayLife:home}.

  \cite{CGSE:game_of_life_digital_clock} demonstrates how the evolution of a certain initial configuration can visualize a digital clock.
\end{comments}

\begin{example}\label{ex:def:game_of_life}
  We list examples related to the \hyperref[def:game_of_life]{game of life}, with names for the configurations taken from Gardner's article \cite[120]{Gardner1970GameOfLife}:
  \begin{thmenum}
    \thmitem{ex:def:game_of_life/blinker} The \term{blinker} is a neighborhood of three cells shown in \fullref{fig:ex:def:game_of_life/blinker/0}.

    With a blinker as an initial configuration of live cells, game of life is \hyperref[def:dynamical_system_periodicity]{periodic} with fundamental period \( 2 \).

    \begin{figure}[!ht]
      \begin{subcaptionblock}{0.15\textwidth}
        \centering
        \includegraphics[page=1]{output/ex__def__game_of_life__blinker}
        \caption{\( t = 0 \)}\label{fig:ex:def:game_of_life/blinker/0}
      \end{subcaptionblock}
      \hfill
      \begin{subcaptionblock}{0.15\textwidth}
        \centering
        \includegraphics[page=2]{output/ex__def__game_of_life__blinker}
        \caption{\( t = 1 \)}\label{fig:ex:def:game_of_life/blinker/1}
      \end{subcaptionblock}
      \hfill
      \begin{subcaptionblock}{0.15\textwidth}
        \centering
        \includegraphics[page=3]{output/ex__def__game_of_life__blinker}
        \caption{\( t = 2 \)}\label{fig:ex:def:game_of_life/blinker/2}
      \end{subcaptionblock}
      \hfill
      \begin{subcaptionblock}{0.15\textwidth}
        \centering
        \includegraphics[page=4]{output/ex__def__game_of_life__blinker}
        \caption{\( t = 3 \)}\label{fig:ex:def:game_of_life/blinker/3}
      \end{subcaptionblock}
      \hfill
      \begin{subcaptionblock}{0.15\textwidth}
        \centering
        \includegraphics[page=5]{output/ex__def__game_of_life__blinker}
        \caption{\( t = 4 \)}\label{fig:ex:def:game_of_life/blinker/4}
      \end{subcaptionblock}
      \caption{The evolution of a \hyperref[ex:def:game_of_life/blinker]{blinker} in the \hyperref[def:game_of_life]{game of life}.}\label{fig:ex:def:game_of_life/blinker}
    \end{figure}

    \thmitem{ex:def:game_of_life/loaf} The \term{loaf} is a neighborhood of seven cells shown in \cref{fig:ex:def:game_of_life/loaf/3}.

    As a configuration of live cells, it is \hyperref[def:dynamical_system_fixed_point]{fixed}. When starting from the configuration in \cref{fig:ex:def:game_of_life/loaf/0}, it takes three generations to each the loaf.

    \begin{figure}[!ht]
      \begin{subcaptionblock}{0.15\textwidth}
        \centering
        \includegraphics[page=1]{output/ex__def__game_of_life__loaf}
        \caption{\( t = 0 \)}\label{fig:ex:def:game_of_life/loaf/0}
      \end{subcaptionblock}
      \hfill
      \begin{subcaptionblock}{0.15\textwidth}
        \centering
        \includegraphics[page=2]{output/ex__def__game_of_life__loaf}
        \caption{\( t = 1 \)}\label{fig:ex:def:game_of_life/loaf/1}
      \end{subcaptionblock}
      \hfill
      \begin{subcaptionblock}{0.15\textwidth}
        \centering
        \includegraphics[page=3]{output/ex__def__game_of_life__loaf}
        \caption{\( t = 2 \)}\label{fig:ex:def:game_of_life/loaf/2}
      \end{subcaptionblock}
      \hfill
      \begin{subcaptionblock}{0.15\textwidth}
        \centering
        \includegraphics[page=4]{output/ex__def__game_of_life__loaf}
        \caption{\( t = 3 \)}\label{fig:ex:def:game_of_life/loaf/3}
      \end{subcaptionblock}
      \hfill
      \begin{subcaptionblock}{0.15\textwidth}
        \centering
        \includegraphics[page=5]{output/ex__def__game_of_life__loaf}
        \caption{\( t = 4 \)}\label{fig:ex:def:game_of_life/loaf/4}
      \end{subcaptionblock}
      \caption{An evolution in the \hyperref[def:game_of_life]{game of life} that reaches a \hyperref[ex:def:game_of_life/loaf]{loaf}.}\label{fig:ex:def:game_of_life/loaf}
    \end{figure}

    \thmitem{ex:def:game_of_life/r_pentomino} The \term{\( R \)-pentomino}, shown in \cref{fig:ex:def:game_of_life/r_pentomino/0}, is a neighborhood with a more complicated evolution. Gardner states that \enquote{Its fate is not yet known. Conway has tracked it for \( 460 \) moves.}

    \cite{ConwayLife:r_pentomino} claims that it is \hyperref[def:dynamical_system_eventual_periodicity]{eventually periodic} with period \( 2 \) (due to several blinkers) and preperiod \( 1103 \).

    \begin{figure}[!ht]
      \begin{subcaptionblock}{0.15\textwidth}
        \centering
        \includegraphics[page=1]{output/ex__def__game_of_life__r_pentomino}
        \caption{\( t = 0 \)}\label{fig:ex:def:game_of_life/r_pentomino/0}
      \end{subcaptionblock}
      \hfill
      \begin{subcaptionblock}{0.15\textwidth}
        \centering
        \includegraphics[page=2]{output/ex__def__game_of_life__r_pentomino}
        \caption{\( t = 1 \)}\label{fig:ex:def:game_of_life/r_pentomino/1}
      \end{subcaptionblock}
      \hfill
      \begin{subcaptionblock}{0.15\textwidth}
        \centering
        \includegraphics[page=3]{output/ex__def__game_of_life__r_pentomino}
        \caption{\( t = 2 \)}\label{fig:ex:def:game_of_life/r_pentomino/2}
      \end{subcaptionblock}
      \hfill
      \begin{subcaptionblock}{0.15\textwidth}
        \centering
        \includegraphics[page=4]{output/ex__def__game_of_life__r_pentomino}
        \caption{\( t = 3 \)}\label{fig:ex:def:game_of_life/r_pentomino/3}
      \end{subcaptionblock}
      \hfill
      \begin{subcaptionblock}{0.15\textwidth}
        \centering
        \includegraphics[page=5]{output/ex__def__game_of_life__r_pentomino}
        \caption{\( t = 4 \)}\label{fig:ex:def:game_of_life/r_pentomino/4}
      \end{subcaptionblock}
      \caption{The evolution of an \hyperref[ex:def:game_of_life/r_pentomino]{\( R \)-pentomino} in the \hyperref[def:game_of_life]{game of life}.}\label{fig:ex:def:game_of_life/r_pentomino}
    \end{figure}

    \thmitem{ex:def:game_of_life/glider} The \term{glider}, shown in \cref{fig:ex:def:game_of_life/loaf/0}, is more subtle. If we start with a glider as a configuration of live cells, after four generations, we reach a configuration with the glider translated diagonally.

    \begin{figure}[!ht]
      \begin{subcaptionblock}{0.20\textwidth}
        \centering
        \includegraphics[page=1]{output/ex__def__game_of_life__glider}
        \caption{\( t = 0 \)}\label{fig:ex:def:game_of_life/glider/0}
      \end{subcaptionblock}
      \hfill
      \begin{subcaptionblock}{0.20\textwidth}
        \centering
        \includegraphics[page=2]{output/ex__def__game_of_life__glider}
        \caption{\( t = 1 \)}\label{fig:ex:def:game_of_life/glider/1}
      \end{subcaptionblock}
      \hfill
      \begin{subcaptionblock}{0.20\textwidth}
        \centering
        \includegraphics[page=3]{output/ex__def__game_of_life__glider}
        \caption{\( t = 2 \)}\label{fig:ex:def:game_of_life/glider/2}
      \end{subcaptionblock}
      \hfill
      \begin{subcaptionblock}{0.20\textwidth}
        \centering
        \includegraphics[page=4]{output/ex__def__game_of_life__glider}
        \caption{\( t = 3 \)}\label{fig:ex:def:game_of_life/glider/3}
      \end{subcaptionblock}
      \\[\baselineskip]
      \begin{subcaptionblock}{0.20\textwidth}
        \centering
        \includegraphics[page=5]{output/ex__def__game_of_life__glider}
        \caption{\( t = 4 \)}\label{fig:ex:def:game_of_life/glider/4}
      \end{subcaptionblock}
      \hfill
      \begin{subcaptionblock}{0.20\textwidth}
        \centering
        \includegraphics[page=6]{output/ex__def__game_of_life__glider}
        \caption{\( t = 5 \)}\label{fig:ex:def:game_of_life/glider/5}
      \end{subcaptionblock}
      \hfill
      \begin{subcaptionblock}{0.20\textwidth}
        \centering
        \includegraphics[page=7]{output/ex__def__game_of_life__glider}
        \caption{\( t = 6 \)}\label{fig:ex:def:game_of_life/glider/6}
      \end{subcaptionblock}
      \hfill
      \begin{subcaptionblock}{0.20\textwidth}
        \centering
        \includegraphics[page=8]{output/ex__def__game_of_life__glider}
        \caption{\( t = 7 \)}\label{fig:ex:def:game_of_life/glider/7}
      \end{subcaptionblock}
      \caption{The \enquote{movement} of a \hyperref[ex:def:game_of_life/glider]{glider} in the \hyperref[def:game_of_life]{game of life}.}\label{fig:ex:def:game_of_life/glider}
    \end{figure}
  \end{thmenum}
\end{example}

\paragraph{Elementary cellular automata}

\begin{definition}\label{def:elementary_cellular_automaton}\mcite{Wolfram1983StatisticalMechanicsOfCellularAutomata}
  An \term{elementary cellular automaton} is a two-state \hyperref[def:cellular_automaton]{cellular automaton} on the \hyperref[def:cayley_graph]{Cayley graph} \( \Gamma(\BbbZ, \set{ 1 }) \) with a local evolution function specified by the three-cell \hyperref[def:cayley_graph_neighborhood/moore]{Moore neighborhood}.

  As in the \hyperref[def:game_of_life]{game of life}, we regard the (elementary) states as \hyperref[con:boolean_value]{Boolean values}, but it will be important to regard \( T \) as \( 1 \) and \( F \) as \( 0 \).

  There are \( 8 = 2^3 \) possible configurations of the Moore neighborhood at a cell, and thus \( 2^8 = 256 \) possible local evolution functions. Fix some nonnegative integer \( n \) less than \( 256 \). We will determine an evolution function based on \( n \), and call it \term{rule \( n \)}.

  Consider the \( 8 \)-bit \hyperref[def:endianness/big]{big endian} \hyperref[def:ring_of_unsigned_integers]{binary encoding} \( a_0 \cdots a_7 \) of \( n \). If the Moore neighborhood at some cell \( k \) has values \( (l, c, r) \), regard them as the \( 3 \)-bit encoding of some integer \( i \), and define the value of the local evolution function at this neighborhood as \( a_i \).
\end{definition}
\begin{comments}
  \item \incite[6]{HadelerMüller2017CellularAutomata} calls elementary cellular automata \enquote{Wolfram automata} because of Stephen Wolfram's paper \cite{Wolfram1983StatisticalMechanicsOfCellularAutomata} attempting to classify them.

  \item By abuse of language, we will also refer to the automaton itself as \enquote{rule \( n \)}.

  \item We can visualize a rule more easily; refer to \fullref{ex:def:elementary_cellular_automaton/30}.
\end{comments}

\begin{definition}\label{def:doubly_infinite_sequence}\mcite[87]{Feller1971ProbabilityTheoryVol2}
  A \term{doubly infinite sequence} is a \hyperref[def:indexed_family]{family} indexed by the set \( \BbbZ \) of \hyperref[def:integers]{integers}.

  As with \hyperref[def:sequence]{ordinary sequences}, we will use the notation
  \begin{equation*}
    \seq{ a_k }_{k=-\infty}^\infty.
  \end{equation*}
\end{definition}

\begin{remark}\label{rem:elementary_cellular_automaton_visualization}
  Similarly to \hyperref[rem:automata_on_square_grid]{cellular automata on a square grid}, \hyperref[def:elementary_cellular_automaton]{elementary cellular automaton} can also be visualized on a square grid.

  The global state at any given step is a \hyperref[def:doubly_infinite_sequence]{doubly infinite sequence}. Instead of the grid being the configuration of the elementary automaton, we simply let the \( k \)-th row represent step \( k \) of the evolution.

  The \hyperref[def:dynamical_system_trajectory]{trajectories} of the automaton are then sequences of rows.
\end{remark}

\begin{example}\label{ex:def:elementary_cellular_automaton}
  We list examples of \hyperref[def:elementary_cellular_automaton]{elementary cellular automata}:
  \begin{thmenum}
    \thmitem{ex:def:elementary_cellular_automaton/30} We show the evolution of rule \( 30 \) in \fullref{fig:ex:def:elementary_cellular_automaton/30}, visualized row-wise as per \fullref{rem:elementary_cellular_automaton_visualization}.

    The corresponding big endian bit string of \( 30 \) is \( 00011110 \), meaning that the local evolution at the origin is
    \begin{equation*}
      \varphi_0(l, c, r) \coloneqq \begin{cases}
        1, &\T{only one of} l, c \T{or} r \T{is} 1 \\
        1, &l = 0 \T{and} c = r = 1 \\
        0, &l.
      \end{cases}
    \end{equation*}

    As suggested by \incite[604]{Wolfram1983StatisticalMechanicsOfCellularAutomata}, we can visualize the rule more easily as follows:
    \begin{align*}
      \frac {000} 0
      &&
      \frac {001} 1
      &&
      \frac {010} 1
      &&
      \frac {011} 1
      &&
      \frac {100} 1
      &&
      \frac {101} 0
      &&
      \frac {110} 0
      &&
      \frac {111} 0.
    \end{align*}

    \begin{figure}[!ht]
      \begin{subcaptionblock}{\textwidth}
        \centering
        \includegraphics[page=1]{output/ex__def__elementary_cellular_automaton__30__zero}
        \caption{Only the cell at \( 0 \) is marked initially.}\label{fig:ex:def:elementary_cellular_automaton/30/zero}
      \end{subcaptionblock}
      \\[\baselineskip]
      \begin{subcaptionblock}{\textwidth}
        \centering
        \includegraphics[page=1]{output/ex__def__elementary_cellular_automaton__30__unit}
        \caption{The cells at \( -1 \) and \( 1 \) are marked initially.}\label{fig:ex:def:elementary_cellular_automaton/30/unit}
      \end{subcaptionblock}
      \caption{The evolution of \hyperref[ex:def:elementary_cellular_automaton/30]{rule \( 30 \)} with different initial configurations.}\label{fig:ex:def:elementary_cellular_automaton/30}
    \end{figure}

    \thmitem{ex:def:elementary_cellular_automaton/diagonal} Some rules allow \hyperref[def:dynamical_system_periodicity]{periodic} configurations, for example \cref{fig:ex:def:elementary_cellular_automaton/diagonal/1}.

    Others like \cref{fig:ex:def:elementary_cellular_automaton/diagonal/3} have configurations are not periodic but translate the initial configuration on regular periods, similarly to \hyperref[ex:def:game_of_life/glider]{gliders} in the \hyperref[def:game_of_life]{game of life}.

    \begin{figure}[!ht]
      \begin{subcaptionblock}{\textwidth}
        \centering
        \includegraphics[page=1]{output/ex__def__elementary_cellular_automaton__diagonal}
        \caption{Rule \( 1 \).}\label{fig:ex:def:elementary_cellular_automaton/diagonal/1}
      \end{subcaptionblock}
      \\[\baselineskip]
      \begin{subcaptionblock}{\textwidth}
        \centering
        \includegraphics[page=2]{output/ex__def__elementary_cellular_automaton__diagonal}
        \caption{Rule \( 2 \).}\label{fig:ex:def:elementary_cellular_automaton/diagonal/2}
      \end{subcaptionblock}
      \\[\baselineskip]
      \begin{subcaptionblock}{\textwidth}
        \centering
        \includegraphics[page=3]{output/ex__def__elementary_cellular_automaton__diagonal}
        \caption{Rule \( 3 \).}\label{fig:ex:def:elementary_cellular_automaton/diagonal/3}
      \end{subcaptionblock}
      \caption{The evolution of several \hyperref[def:elementary_cellular_automaton]{elementary cellular automata}.}\label{fig:ex:def:elementary_cellular_automaton/diagonal}
    \end{figure}

    \thmitem{ex:def:elementary_cellular_automaton/18} Rule 18 allows for a peculiar pattern visualized in \cref{fig:ex:def:elementary_cellular_automaton/18} that resemble \hyperref[def:sierpinski_triangle]{Sierpinski triangles}.

    \begin{figure}[!ht]
      \centering
      \includegraphics[page=1]{output/ex__def__elementary_cellular_automaton__18}
      \caption{The evolution of \hyperref[ex:def:elementary_cellular_automaton/18]{rule 18} with only the zero marked initially.}\label{fig:ex:def:elementary_cellular_automaton/18}
    \end{figure}
  \end{thmenum}
\end{example}
