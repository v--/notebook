\section{Recurrent relations}\label{sec:recurrent_relations}

\paragraph{Progressions}\hfill

Progressions are an elementary concept that happens to be quite useful. There is no definition of progression, but rather the term \enquote{progression} refers to specific recursively defined \hyperref[def:sequence]{sequences}.

\begin{definition}\label{def:arithmetic_progression}
  In an \hyperref[def:abelian_group]{abelian group}, by default taken to be the field of complex numbers, an \( n \)-term (resp. infinite) \term[ru=арифметическая прогрессия (\cite[143]{АлександровМаркушевичХинчин1952ЭнциклопедияТом3}), en=arithmetic progression (\cite[def. 2.4.3]{Rosen2019DiscreteMathematics})]{arithmetic progression} with \term[en=common difference (\cite[def. 2.4.3]{Rosen2019DiscreteMathematics})]{difference} \( d \) is a sequence \( \seq{ a_k }_{k=0}^{n-1} \) (resp. \( \seq{ a_k }_{k=0}^\infty \)) satisfying any of the following equivalent conditions:
  \begin{thmenum}
    \thmitem{def:arithmetic_progression/implicit}\mcite[\S 227]{Киселёв2004Геометрия} The difference of any two consecutive elements equals \( d \).
    \thmitem{def:arithmetic_progression/direct}\mcite[595]{ButlerEtAl2016Progressions} We have \( a_k = a_0 + kd \) for every index \( k \), where \( kd \) is \hyperref[con:additive_semigroup/multiplication]{iterated addition}.
    \thmitem{def:arithmetic_progression/recursive}\mcite[143]{АлександровМаркушевичХинчин1952ЭнциклопедияТом3} We have
    \begin{equation}\label{eq:def:arithmetic_progression/recursive}
      a_k \coloneqq \begin{cases}
        a_0,         &k = 0, \\
        a_{k-1} + d, &k > 0.
      \end{cases}
    \end{equation}
  \end{thmenum}
\end{definition}

\begin{proposition}\label{thm:arithmetic_progression_partial_sums}
  The \hyperref[def:convergent_series]{series} constructed from the arithmetic progression \( \seq{ a_0 + kd }_{k=0}^\infty \) has partial sums
  \begin{equation}\label{eq:thm:arithmetic_progression_partial_sums}
    2 \sum_{k=0}^n a_k = (n + 1) (a_n - a_0).
  \end{equation}
\end{proposition}
\begin{proof}
  \begin{balign*}
    2 \sum_{k=0}^n a_k
     & =
    2 \sum_{k=0}^n (a_0 + kd)
    =    \\ &=
    \sum_{k=0}^n (a_0 + kd) + \sum_{k=0}^n (a_0 + (n-k)d)
    =    \\ &=
    \sum_{k=0}^n (2 a_0 + nd)
    =    \\ &=
    (n + 1) (a_0 + a_n).
  \end{balign*}
\end{proof}

\begin{corollary}\label{thm:numeric_arithmetic_progression_partial_sums}
  For any positive integer \( n \), we have
  \begin{equation}\label{eq:thm:numeric_arithmetic_progression_partial_sums}
    \sum_{k=0}^n k = \sum_{k=1}^n k = \frac {n (n + 1)} 2 = \binom {n+1} 2.
  \end{equation}
\end{corollary}
\begin{comments}
  \item These are precisely the \hyperref[def:triangular_number]{triangular numbers}, as shown in \fullref{thm:triangular_point_configuration_cardinality}.
\end{comments}
\begin{proof}
  Follows from either \fullref{thm:arithmetic_progression_partial_sums} or \fullref{thm:vandermonde_convolution}.
\end{proof}

\begin{definition}\label{def:geometric_progression}
  In a \hyperref[def:field]{field}, by default taken to be the field of complex numbers, an \( n \)-term (resp. infinite) \term[ru=геометрическая прогрессия (\cite[144]{АлександровМаркушевичХинчин1952ЭнциклопедияТом3}), en=geometric progression (\cite[def. 2.4.2]{Rosen2019DiscreteMathematics})]{geometric progression} with \term[ru=знаменатель (прогрессии) (\cite[\S 227]{Киселёв2004Геометрия}), en=common ratio (\cite[def. 2.4.2]{Rosen2019DiscreteMathematics})]{quotient} \( q \) is a sequence \( \seq{ a_k }_{k=0}^{n-1} \) (resp. \( \seq{ a_k }_{k=0}^\infty \)) of \hi{nonzero elements} satisfying any of the following equivalent conditions:
  \begin{thmenum}
    \thmitem{def:geometric_progression/implicit}\mcite[\S 227]{Киселёв2004Геометрия} The quotient of two consecutive elements equals \( q \).
    \thmitem{def:geometric_progression/direct}\mcite[144]{АлександровМаркушевичХинчин1952ЭнциклопедияТом3} We have \( a_k = a_0 q^k \) for every index \( k \).
    \thmitem{def:geometric_progression/recursive}\mcite[144]{АлександровМаркушевичХинчин1952ЭнциклопедияТом3} We have
    \begin{equation}\label{eq:def:geometric_progression/recursive}
      a_k \coloneqq \begin{cases}
        a_0,       &k = 0, \\
        a_{k-1} q, &k > 0.
      \end{cases}
    \end{equation}
  \end{thmenum}
\end{definition}

\begin{proposition}\label{thm:arithmetic_to_geometric_progression}
  Given a complex arithmetic progression \( \seq{ a_k }_{k=0}^\infty \) with difference \( d \), for any complex number \( z \), the sequence \( \seq{ z^{a_k} }_{k=0}^\infty \) is a geometric progression with quotient \( z^d \).
\end{proposition}
\begin{proof}
  Trivial.
\end{proof}

\paragraph{Geometric series}

\begin{definition}\label{def:geometric_series}\mcite[61]{Rudin1976AnalysisPrinciples}
  A \term{geometric series} is \hyperref[def:convergent_series]{series} whose coefficients come from a complex-valued \hyperref[def:geometric_progression]{geometric progression} with base \( 1 \):
  \begin{equation}\label{eq:def:geometric_series}
    \sum_{k=0}^\infty z^k.
  \end{equation}
\end{definition}
\begin{comments}
  \item Rudin defines the series only for real numbers, for the generalization is straightforward.
\end{comments}

\begin{proposition}\label{thm:xn_minus_yn_factorization}
  For arbitrary \hyperref[def:ring]{ring} elements \( x \) and \( y \) and for every nonnegative integer \( n \), we have
  \begin{equation}\label{eq:thm:xn_minus_yn_factorization}
    x^{n + 1} - y^{n + 1} = (x - y)(x^n + x^{n-1} y + \cdots + y^n) = (x - y) \sum_{k=0}^n x^k y^{n-k}.
  \end{equation}
\end{proposition}
\begin{proof}
  \begin{align*}
    (x - y) \sum_{k=0}^n x^k y^{n-k}
    &=
    \sum_{k=0}^n x^{k+1} y^{n-k} - \sum_{k=0}^n x^k y^{n-k+1}
    = \\ &=
    \sum_{k=1}^{n+1} x^k y^{(n+1)-k} - \sum_{k=0}^n x^k y^{(n+1)-k}
    =
    x^{n+1} - y^{n+1}.
  \end{align*}
\end{proof}

\begin{proposition}\label{thm:def:geometric_series}
  The geometric series \eqref{eq:def:geometric_series} has the following basic properties:
  \begin{thmenum}
    \thmitem{thm:def:geometric_series/finite_sum} For all \( z \in \BbbC \setminus \set{ 1 } \), the geometric series \eqref{eq:def:geometric_series} has partial sums
    \begin{equation}\label{eq:thm:def:geometric_series/finite_sum}
      \sum_{k=0}^n z^k = \frac {1 - z^{n+1}} {1 - z}.
    \end{equation}

    \thmitem{thm:def:geometric_series/degenerate} In the degenerate case \( z = 1 \), the progression itself is constant, and its partial sums are instead
    \begin{equation}\label{eq:thm:def:geometric_series/degenerate}
      \sum_{k=0}^n z^k = n + 1.
    \end{equation}

    \thmitem{thm:def:geometric_series/series_sum_exterior} The series diverges when \( \abs{z} \geq 1 \).

    \thmitem{thm:def:geometric_series/series_sum_interior} For \( 0 < \abs{z} < 1 \), the series converges absolutely with limit
    \begin{equation}\label{eq:thm:def:geometric_series/series_sum_interior}
      \sum_{k=0}^\infty z^k = \frac 1 {1 - z}.
    \end{equation}
  \end{thmenum}
\end{proposition}
\begin{proof}
  \SubProofOf{thm:def:geometric_series/finite_sum} Follows from \fullref{thm:xn_minus_yn_factorization}.

  \SubProofOf{thm:def:geometric_series/degenerate} Obvious.

  \SubProofOf{thm:def:geometric_series/series_sum_exterior}

  \SubProof*{Proof for \( z = 1 \)} If \( z = 1 \), \fullref{thm:def:geometric_series/degenerate} implies that the series diverges because it grows indefinitely.

  \SubProof*{Proof for \( \abs{z} = 1 \) and \( z \neq 1 \)} In this case the integer powers \( z^k \) are rotations around the complex plane unit circle, which do not tend to a limit. Hence, the series diverges again.

  \SubProof*{Proof for \( \abs{z} > 1 \)} In this case \( \abs{z^n} \) grows indefinitely with \( n \), and it follows that
  \begin{equation*}
    \sum_{k=m}^n z^k
    =
    z^m \sum_{k=0}^{n-m} z^k
    =
    z^m \frac {1 - z^{n-m+1}} {1 - z}
    =
    \frac {z^m - z^{n+1}} {1 - z}.
  \end{equation*}
  can get arbitrarily far from \( 0 \). Therefore, in this case the series also diverges.

  \SubProofOf{thm:def:geometric_series/series_sum_interior} Fix \( z \in B(0, 1) \). Since only \( z^{n + 1} \) depends on \( n \) in \eqref{eq:thm:def:geometric_series/finite_sum}, we obtain \eqref{eq:thm:def:geometric_series/series_sum_interior} by simply noting that \( z^n \to 0 \) when \( n \to \infty \).
\end{proof}

\begin{example}\label{ex:def:geometric_series}
  We list some examples of \hyperref[def:geometric_series]{geometric series}:
  \begin{thmenum}
    \thmitem{ex:def:geometric_series/two} A surprisingly useful series occurs for \( z = 1 / 2 \), where
    \begin{equation}\label{eq:ex:def:geometric_series/two}
      \sum_{k=0}^\infty \frac 1 {2^k}
      \reloset {\eqref{eq:thm:def:geometric_series/series_sum_interior}} =
      \frac 1 {1 - 1 / 2}
      =
      2.
    \end{equation}

    Its usefulness comes from the fact that
    \begin{equation}\label{eq:ex:def:geometric_series/two/one}
      \sum_{k=1}^\infty \frac 1 {2^k}
      =
      \sum_{k=0}^\infty \frac 1 {2^k} - 1
      =
      1,
    \end{equation}
    hence we can use the terms for generalized \hyperref[def:convex_hull]{convex combinations}.

    \thmitem{ex:def:geometric_series/interest} In this example we will exploit the equivalence between the closed form representations in \fullref{def:arithmetic_progression} and \fullref{def:geometric_progression} and the corresponding inductive definitions. The equivalences are obvious from a mathematical standpoint, however outside of mathematics they have highly nontrivial consequences. Indeed, they highlight the difference between simple interest and compound interest.

    Consider a savings account with \( 1000\$ \). A simple monthly interest of \( 2\% \) will earn \( 240\$ \) over a year:
    \begin{equation*}
      1000 (1 + 12 \cdot 2 / 100) = 1240.
    \end{equation*}

    The same account with a compound monthly interest of \( 2\% \) will earn a bit more - about \( 268\$ \):
    \begin{equation*}
      1000 (1 + 2 / 100)^{12} \approx 1268.24.
    \end{equation*}

    Over the course of ten years, however, simple interest will earn a total of \( 2400\$ \), while compound interest will earn \( \approx 9765\$ \).

    The difference between linear and exponential growth appears staggering in a real-world situation even though the difference may not be very noticeable in the short-term.
  \end{thmenum}
\end{example}

\paragraph{Catalan numbers}

\begin{definition}\label{def:catalan_number}\mcite[188]{Stanley2023EnumCombinatoricsVol2}
  We define the \term[ru=число Каталана (\cite[\S 5.7.4]{Новиков2013ДискретнаяМатематика})]{Catalan numbers} \hyperref[rem:natural_number_recursion]{recursively} as
  \begin{equation}\label{eq:def:catalan_number}
    C_n \coloneqq \begin{cases}
      1,                                    &n = 0, \\
      \sum_{k=0}^{n-1} C_k \cdot C_{n-k-1}, &n > 0.
    \end{cases}
  \end{equation}
\end{definition}
\begin{comments}
  \item Richard Stanley lists many characterizations of Catalan numbers in the exercises to \cite[ch. 6]{Stanley2023EnumCombinatoricsVol2}.
\end{comments}

\begin{proposition}\label{thm:catalan_number_via_binomial_coefficients}
  We can characterize \hyperref[def:catalan_number]{Catalan numbers} via \hyperref[def:binomial_coefficient]{binomial coefficients}:
  \begin{equation}\label{eq:thm:catalan_number_via_binomial_coefficients}
    C_n = \frac 1 {n + 1} \binom {2n} n
  \end{equation}
\end{proposition}
\begin{proof}
  \todo{Prove}
\end{proof}

\begin{proposition}\label{thm:full_binary_tree_count}
  The number of \hyperref[def:n_ary_tree]{full binary trees} with \( n \) \hyperref[def:rooted_tree/internal]{internal nodes} is described by the \hyperref[def:catalan_number]{Catalan number} \( C_n \).
\end{proposition}
\begin{proof}
  We will use induction on \( n \). The case \( n = 0 \) is vacuous. For the inductive step, suppose that the hypothesis holds for less than \( n \) internal nodes.

  Let \( T \) be a full binary tree with \( n \) internal nodes. The \hyperref[def:rooted_tree/subtree]{immediate subtrees} of \( T \) have a total of \( n - 1 \) internal nodes. If the left subtree has \( k \) internal nodes, the right subtree has \( n - k - 1 \) internal nodes.

  Denote by \( t_n \) the number of full binary trees with \( n \) internal nodes. Then, by what we have shown,
  \begin{equation*}
    t_n = \sum_{k=0}^{n-1} t_k t_{n-k-1}.
  \end{equation*}

  By the inductive hypothesis, \( t_n \) satisfies the recursive clause in \eqref{eq:def:catalan_number}, hence it equals \( C_n \).
\end{proof}

\begin{lemma}\label{thm:dyck_language_length}
  Every string in the \hyperref[def:dyck_language]{Dyck language} has even length.
\end{lemma}
\begin{proof}
  We will use \fullref{thm:induction_on_rooted_trees} on the \hyperref[def:parse_tree]{parse tree} of the word \( w \):
  \begin{itemize}
    \item If \( w \) is the empty word, its length is clearly even.

    \item If \( w = (w') \), then \( w \) is longer than \( w' \) by two symbols. By the inductive hypothesis, \( w \) has even length.

    \item If \( w = w_1 w_2 \), then \( w \) has even length if and only if both \( w_1 \) and \( w_2 \) have even length.

    Again, by the inductive hypothesis, \( w \) has even length.
  \end{itemize}
\end{proof}

\begin{corollary}\label{thm:dyck_language_word_count}
  The number of strings of length \( 2n \) in the \hyperref[def:dyck_language]{Dyck language} is described by the \hyperref[def:catalan_number]{Catalan number} \( C_n \).
\end{corollary}
\begin{proof}
  Given a full binary tree, if we label the left and right edges of an internal node with the corresponding parenthesis, via \hyperref[def:ordered_tree_enumeration]{pre-order enumeration} we obtain a word from the Dyck language. Conversely, for every word we can construct a full binary tree.
\end{proof}
