\subsection{Lambda calculus reductions}\label{subsec:lambda_calculus_reductions}

\paragraph{\( \beta \) and \( \eta \)-reductions}

\begin{definition}\label{def:alpha_reflexive_closure}\mimprovised
  We say that the binary relation \( {\sim} \) between \( \synlambda \)-terms is \term{\( \alpha \)-reflexive} if \( M \aequiv N \) implies \( M \sim N \).

  Correspondingly, we define the \term{\( \alpha \)-reflexive closure} of a binary relation as its set-theoretic union with \( {\aequiv} \).
\end{definition}
\begin{comments}
  \item This definition is not established. It is based on reflexive closures discussed in \fullref{def:relation_closures/reflexive}. It allows us to make explicit the dependency on \( \alpha \)-equivalence, which is implicit in our primary sources --- \cite[ch. 3]{Barendregt1984}, \cite[1B2]{Hindley1997} and \cite[191]{Герасимов2011}.
\end{comments}

\begin{definition}\label{def:lambda_reduction}\mcite[315]{Barendregt1984}
  We will define different kinds of \enquote{\( \rho \)-reduction} relations simultaneously. The symbol \enquote{\( \rho \)} is inessential to the formalisms presented here; we use it as a placeholder for \enquote{\( \beta \)}, \enquote{\( \eta \)}, \enquote{\( \beta\eta \)} and potentially other kinds of reductions.

  \begin{thmenum}
    \thmitem{def:lambda_reduction/single} A \term{single-step reduction} is a relation generated by or otherwise satisfying the following rules:
    \begin{ThreeColumns}
      \begin{equation*}\taglabel[\ensuremath{ \logic{App}_\rho^L }]{def:lambda_reduction/app_left}
        \begin{prooftree}
          \hypo{ M \rred N }
          \infer1[\ref{def:lambda_reduction/app_left}]{ LM \rred LN }
        \end{prooftree}
      \end{equation*}
    \BeginSecondColumn
      \begin{equation*}\taglabel[\ensuremath{ \logic{App}_\rho^R }]{def:lambda_reduction/app_right}
        \begin{prooftree}
          \hypo{ M \rred N }
          \infer1[\ref{def:lambda_reduction/app_right}]{ ML \rred NL }
        \end{prooftree}
      \end{equation*}
    \BeginThirdColumn
      \begin{equation*}\taglabel[\ensuremath{ \logic{Abs}_\rho }]{def:lambda_reduction/abs}
        \begin{prooftree}
          \hypo{ M \rred N }
          \infer1[\ref{def:lambda_reduction/abs}]{ \qabs x M \rred \qabs x N }
        \end{prooftree}
      \end{equation*}
    \end{ThreeColumns}

    We make the reliance on \( \alpha \)-equivalence explicit via the following rule:
    \begin{equation*}\taglabel[\ensuremath{ \logic{Alpha}_\rho }]{def:lambda_reduction/alpha}
      \begin{prooftree}
        \hypo{ A \aequiv B }
        \hypo{ B \rred C }
        \hypo{ C \aequiv D }
        \infer3[\ref{def:lambda_reduction/alpha}]{ A \rred D }.
      \end{prooftree}
    \end{equation*}

    \thmitem{def:lambda_reduction/reduction} For each single-step reduction \( {\rred} \) defined above, we define the corresponding \term{multi-step \( \rho \)-reduction} \( {\rred*} \) as the \hyperref[def:relation_closures/transitive]{transitive} and \hyperref[def:alpha_reflexive_closure]{\( \alpha \)-reflexive closure} of \( {\rred} \). Which closure is taken first is immaterial due to \ref{def:lambda_reduction/alpha} --- \( A \aequiv B \rred C \) already implies \( A \rred C \).

    Without further context, \enquote{\( \rho \)-reduction} we will refer to the multi-step \( \rho \)-reduction.

    \thmitem{def:lambda_reduction/equivalence} If we instead take the
    \hyperref[def:relation_closures/transitive]{transitive} and \hyperref[def:alpha_reflexive_closure]{\( \alpha \)-reflexive} closure of the \hyperref[def:relation_closures/symmetric]{symmetric closure} of \( \rred \), we obtain \term{\( \rho \)-equivalence}, which we denote by \( {\equivrel{\rho}} \).
  \end{thmenum}
\end{definition}
\begin{comments}
  \item These relations can be obtained via \fullref{thm:least_fixed_point_recursion}, similarly to what is described in \fullref{ex:fixed_point_recursion_for_relations}.
\end{comments}

\begin{definition}\label{def:beta_eta_reduction}\mcite[1B1; 1C1]{Hindley1997}
  Consider the following rules:
  \begin{TwoColumns}
    \begin{equation*}\taglabel[\ensuremath{ \logic{Red}_\beta }]{def:beta_eta_reduction/beta}
      \begin{prooftree}
        \infer0[\ref{def:lambda_reduction/beta}]{ \underbrace{(\qabs x M) N}_{\T{\term{\( \beta \)-redex}}} \bred \underbrace{M[x \mapsto N]}_{\T{\term{\( \beta \)-contractum}}} }.
      \end{prooftree}
    \end{equation*}
  \BeginSecondColumn
    \begin{equation*}\taglabel[\ensuremath{ \logic{Red}_\eta }]{def:beta_eta_reduction/eta}
      \begin{prooftree}
        \hypo{ x \not\in \op*{Free}(M) }
        \infer1[\ref{def:lambda_reduction/eta}]{ \underbrace{\qabs x M x}_{\T{\term{\( \eta \)-redex}}} \ered \underbrace{M}_{\mathclap{\T{\term{\( \eta \)-contractum}}}} }.
      \end{prooftree}
    \end{equation*}
  \end{TwoColumns}

  We define \term{\( \beta \)-reduction} \( {\bred} \) to be the relation generated by the rules of \fullref{def:lambda_reduction/single} extended with \ref{def:lambda_reduction/beta}. We similarly define \term{\( \eta \)-reduction} \( {\ered} \) based on \ref{def:lambda_reduction/eta}. Combining both, we obtain \term{\( \beta\eta \)-reduction} \( {\bered} \).
\end{definition}
\begin{comments}
  \item Of course, we also consider equivalences and multi-step reductions based on the above single-step reductions.
\end{comments}

\begin{example}\label{ex:def:beta_eta_reduction}
  We list examples of \hyperref[def:lambda_reduction]{\( \beta \) and \( \eta \)-reductions}:
  \begin{thmenum}
    \thmitem{ex:def:beta_eta_reduction/i} Consider the combinator \( \ref{eq:ex:def:lambda_term/combinator/i} = \qabs \synx \synx \).

    For every \( \synlambda \)-term \( A \) we have \( IA \bred A \). This explains the naming --- \( I \) stands for \enquote{identity}.

    \thmitem{ex:def:beta_eta_reduction/k} Consider the combinator \( \ref{eq:ex:def:lambda_term/combinator/k} = \qabs \synx \qabs \syny \syny \synx \).

    For every pair of combinators \( A \) and \( B \) where \( \syny \) is not free, we have
    \begin{equation*}
      KAB
      =
      (\qabs {\hi{\synx}} \qabs \syny \hi{\synx}) \hi{A} B
      \bred
      (\qabs {\hi{\syny}} A) \hi{B}
      \bred
      A.
    \end{equation*}

    \thmitem{ex:def:beta_eta_reduction/s} Consider the combinator \( \ref{eq:ex:def:lambda_term/combinator/s} = \qabs \synx \qabs \syny \qabs \synz \synx \synz (\syny \synz) \).

    For every triple of \( \synlambda \)-terms \( A \), \( B \) and \( C \) where \( \syny \) and \( \synz \) are not free, we have
    \begin{balign*}
      SABC
      &=
      \parens[\Big]{ \qabs {\hi{\synx}} \qabs \syny \qabs \synz \hi{\synx} \synz (\syny \synz) } \hi{A}BC
      \bred \\ &\bred
      \parens[\Big]{ \qabs {\hi{\syny}} \qabs \synz A \synz (\hi{\syny} \synz) } \hi{B}C
      \bred \\ &\bred
      \parens[\Big]{ \qabs {\hi{\synz}} A \hi{\synz} (B \hi{\synz}) } \hi{C}
      \bred \\ &\bred
      A C (B C).
    \end{balign*}

    \thmitem{ex:def:beta_eta_reduction/skk} \Fullref{ex:def:beta_eta_reduction/k} and \fullref{ex:def:beta_eta_reduction/s} imply that
    \begin{equation*}
      SKK \bred* \qabs \synz K \synz (K \synz) \bred* \qabs \synz \synz \aequiv I.
    \end{equation*}

    \thmitem{ex:def:beta_eta_reduction/boolean}\mcite[6.2.2]{Barendregt1984} We can encode the \hyperref[con:boolean_value]{Boolean values} via the combinators
    \begin{subequations}
      \begin{align}
        T \coloneqq \overbrace{\qabs \synx \qabs \syny \synx}^{\ref{eq:ex:def:lambda_term/combinator/k}} \label{ex:def:beta_eta_reduction/boolean/t}, \\
        F \coloneqq \qabs \synx \qabs \syny \syny \label{ex:def:beta_eta_reduction/boolean/f}.
      \end{align}

      Fix \( \synlambda \)-terms \( A \) and \( B \) where \( \syny \) is not free. If the term \( C \) can be either \( T \) or \( F \), \( \beta \)-reduction allows us to \enquote{select} either \( A \) or \( B \) based on the value of \( V \):
      \begin{equation}\label{ex:def:beta_eta_reduction/boolean/if}
        VAB \bred* \begin{cases}
          A, &C = T, \\
          B, &C = F.
        \end{cases}
      \end{equation}

      This allows us to implement \hyperref[def:boolean_function]{Boolean functions}, for example negation:
      \begin{equation}\label{ex:def:beta_eta_reduction/boolean/negation}
        NV \coloneqq VFT.
      \end{equation}

      Indeed, \( NT = TFT \bred* F \) and \( NF = FFT \bred* T \).
    \end{subequations}

    \thmitem{ex:def:beta_eta_reduction/numerals}\mcite[6.4.4]{Barendregt1984} For every natural number \( n \), we can define the \term{Church numeral}
    \begin{equation}\label{eq:ex:def:beta_eta_reduction/numerals}
      c_n \coloneqq \qabs \synx \qabs \syny \underbrace{\synx^n \syny}_{\mathclap{(\synx \cdots (\synx (\synx \syny)) \cdots)}}.
    \end{equation}

    Note that in this definition \( \synx^n \syny \) associates to the right, technically breaking our convention from \fullref{rem:lambda_term_parentheses/left_associative}.

    For the base case we have \( c_0 = F \).

    Consider the \term{successor} term
    \begin{equation}\label{eq:ex:def:beta_eta_reduction/succ}
      S^+ \coloneqq \qabs \synf \qabs \synx \qabs \syny \synx (\synf \synx \syny).
    \end{equation}

    Then
    \begin{balign*}
      S^+ c_n
      &=
      \parens[\Big]{ \qabs {\hi{\synf}} \qabs \synx \qabs \syny \synx (\hi{\synf} \synx \syny) } \hi{c_n}
      \bred \\ &\bred
      \qabs \synx \qabs \syny \synx \parens[\Big]{ \parens[\Big]{ \qabs {\hi{\synx}} \qabs \syny \hi{\synx}^n \synf } \hi{\synx} \syny }
      \bred \\ &\bred
      \qabs \synx \qabs \syny \synx \parens[\Big]{ \parens[\Big]{ \qabs {\hi{\syny}} \synx^n \hi{\syny} } \hi{\syny} }
      \bred \\ &\bred
      \underbrace{\qabs \synx \qabs \syny \synx (\synx^n \syny)}_{c_{n+1}}.
    \end{balign*}

    \thmitem{ex:def:beta_eta_reduction/omega} Consider the combinator \( \ref{eq:ex:def:lambda_term/combinator/omega_n} = (\qabs \synx \synx^n) \).

    We have
    \begin{equation*}
      \omega_n \omega_n
      =
      (\qabs {\hi{\synx}} \hi{\synx^n}) \hi{\omega}
      \bred
      \omega_n^n.
    \end{equation*}

    In particular, for \( \Omega = \omega \omega = \omega_2 \omega_2 \),
    \begin{equation*}
      \Omega \bred \Omega \bred \Omega \bred \Omega \bred \cdots
    \end{equation*}
    and, for \( \omega_3 \omega_3 \),
    \begin{equation*}
      (\omega_3)^2 \bred (\omega_3)^3 \bred (\omega_3)^4 \bred \cdots.
    \end{equation*}
  \end{thmenum}
\end{example}

\begin{lemma}\label{thm:beta_reduction_free_variables}
  \begin{subequations}
    Let \( M = (\qabs x N) K \). Then
    \begin{equation}\label{eq:thm:beta_reduction_free_variables/equality}
      \op*{Free}(N[x \mapsto K]) = \begin{cases}
        \op*{Free}(M), &x \in \op*{Free}(N), \\
        \op*{Free}(N), &\T{otherwise.} \\
      \end{cases}
    \end{equation}

    Both cases can be summarized via the inequality
    \begin{equation}\label{eq:thm:beta_reduction_free_variables/inequality}
      \op*{Free}(N[x \mapsto K]) \subseteq \op*{Free}(M).
    \end{equation}
  \end{subequations}
\end{lemma}
\begin{proof}
  The equality \eqref{eq:thm:beta_reduction_free_variables/equality} follows from \eqref{eq:thm:lambda_substitution_free_variables_single} by noting that
  \begin{equation*}
    \op*{Free}(M)
    =
    \op*{Free}((\qabs x N) K)
    =
    \op*{Free}(\qabs x N) \cup \op*{Free}(K)
    =
    \parens[\Big]{ \op*{Free}(N) \setminus \set{ x } } \cup \op*{Free}(K).
  \end{equation*}
\end{proof}

\begin{proposition}\label{thm:eta_reduction_free_variables_equal}
  If \( M \ered N \), then \( \op*{Free}(M) = \op*{Free}(N) \).
\end{proposition}
\begin{proof}
  Follows via \fullref{thm:least_fixed_point_induction} by noting that \( x \) is not free in \( M \) if and only if it is not free in \( \qabs x M x \).
\end{proof}

\begin{proposition}\label{thm:beta_eta_reduction_free_variables_contra}
  Let \enquote{\( \rho \)} be \enquote{\( \beta \)}, \enquote{\( \eta \)} or \enquote{\( \beta\eta \)}.

  If \( M \rred N \), then \( \op*{Free}(N) \subseteq \op*{Free}(M) \).
\end{proposition}
\begin{proof}
  Follows from \fullref{thm:beta_reduction_free_variables} and \fullref{thm:eta_reduction_free_variables_equal}.
\end{proof}

\begin{example}\label{ex:thm:beta_reduction_free_variables}
  We list examples related to \fullref{thm:beta_reduction_free_variables}:
  \begin{thmenum}
    \thmitem{ex:thm:beta_reduction_free_variables/free} The \( \beta \)-contraction of \( (\qabs \synx \synx \syny) \synz \) is \( \synz \syny \). Both have \( \syny \) and \( \synz \) as their free variables.

    \thmitem{ex:thm:beta_reduction_free_variables/not_free} The \( \beta \)-contraction of \( (\qabs \synx \synz \syny) \synz \) is \( \synz \synz \). Both have \( \synz \) as a free variable, but the \( \beta \)-redex also has \( \syny \).
  \end{thmenum}
\end{example}

\begin{proposition}\label{thm:substitution_on_beta_eta_reduction}
  Let \enquote{\( \rho \)} be \enquote{\( \beta \)}, \enquote{\( \eta \)} or \enquote{\( \beta\eta \)}.

  Fix \( \synlambda \)-terms \( M \) and \( N \) such that \( M \rred N \). Then, for any substitution \( \sigma \), we have \( M[\sigma] \rred N[\sigma] \).
\end{proposition}
\begin{proof}
  We will use \fullref{thm:least_fixed_point_induction} on \( M \rred N \) simultaneously on all substitutions to show that \( M[\sigma] \rred N[\sigma] \).
  \begin{itemize}
    \item If \( M \rred N \) due to \ref{def:lambda_reduction/app_left}, then \( M = AB \) and \( N = AD \), where \( B \rred D \) and the inductive hypothesis holds for the latter. Then
    \begin{equation*}
      M[\sigma]
      =
      A[\sigma] \thinspace B[\sigma]
      \reloset {\T{ind.}} \rred
      A[\sigma] \thinspace D[\sigma]
      =
      N[\sigma].
    \end{equation*}

    \item If \( M \rred N \) due to \ref{def:lambda_reduction/app_right}, we proceed analogously.

    \item If \( M \rred N \) due to \ref{def:lambda_reduction/alpha}, we use \fullref{thm:substitution_on_alpha_equivalent_terms}.

    \item If \( M \rred N \) due to \ref{def:lambda_reduction/abs}, we have \( M = \qabs x A \) and \( N = \qabs x B \), where \( A \rred B \) and the inductive hypothesis holds for the latter. \Fullref{thm:lambda_substitution_single_rule} implies that
    \begin{align*}
      M[\sigma] &= \qabs u A[\sigma_{x \mapsto u}], &&u \not\in \op*{Free}(M) \cup \op*{Free}(M[\sigma]), \\
      N[\sigma] &= \qabs v B[\sigma_{x \mapsto v}], &&v \not\in \op*{Free}(N) \cup \op*{Free}(N[\sigma]).
    \end{align*}

    Furthermore, \fullref{thm:eta_reduction_free_variables_equal} implies that the free variables of \( M \) and \( N \) coincide, thus \( u = v \).

    The inductive hypothesis implies that \( A[\sigma_{x \mapsto u}] \rred B[\sigma_{x \mapsto u}] \), hence we can apply \ref{def:lambda_reduction/alpha} to obtain \( M[\sigma] \rred N[\sigma] \).
  \end{itemize}
\end{proof}

\paragraph{Fixed point combinators}

\begin{definition}\label{def:fixed_point_combinator}\mimprovised
  We say that the combinator \( M \) is a \term{fixed point combinator} if, for every \( \synlambda \)-term \( F \), we have \( MF \beequiv F(MF) \).
\end{definition}
\begin{comments}
  \item The definition is based on \cite[6.1.2]{Barendregt1984}, but is adapted to handle equivalences explicitly.
\end{comments}

\begin{example}\label{ex:def:fixed_point_combinator}
  We list examples of how \hyperref[def:fixed_point_combinator]{fixed point combinators} can be used:
  \begin{thmenum}
    \thmitem{ex:def:fixed_point_combinator/neg} Let \( M \) be a fixed point combinator. Consider the negation term \( N = VFT \) from \eqref{ex:def:beta_eta_reduction/boolean/negation}.

    Then \( MN \) is its own negation because \( MN \beequiv N(MN) \).

    If we regard \( N \) as a function on \( \synlambda \)-terms, \( MN \) corresponds to a fixed point of \( N \). Since neither Boolean value is a fixed point of negation, we conclude that \( N \) actually corresponds to generalization of negation from \hyperref[def:classical_semantics]{classical semantics}.

    \thmitem{ex:def:fixed_point_combinator/succ} Again, let \( M \) be a fixed point combinator.

    For the successor combinator \( S^+ \) from \eqref{eq:ex:def:beta_eta_reduction/succ}, the fixed point \( MS^+ \) corresponds to a generalized numeral that is its own successor.
  \end{thmenum}
\end{example}

\begin{proposition}\label{thm:y_is_a_fixed_point_combinator}
  The term \( \ref{eq:ex:def:lambda_term/combinator/y} = \qabs \synx (\qabs \syny \synx \syny \syny) (\qabs \syny \synx \syny \syny) \) is a \hyperref[def:fixed_point_combinator]{fixed point combinator}.
\end{proposition}
\begin{proof}
  Let \( F \) be a \( \synlambda \)-term where \( \synx \) and \( \syny \) are not free. We have
  \begin{equation*}
    YF
    =
    \parens[\Big]{ \qabs {\hi{\synx}} (\qabs \syny \hi{\synx} \syny \syny) (\qabs \syny \hi{\synx} \syny \syny) } \hi{F}
    \bred
    (\qabs v F v v) (\qabs v F v v),
  \end{equation*}
  where \( v = \syny \) if \( \syny \) is not free in \( F \) and is a fresh variable otherwise.

  Denote this reduct by \( P \). Then
  \begin{equation*}
    P =
    (\qabs {\hi{v}} F \hi{v} \hi{v}) \hi{(\qabs v F v v)}
    \bred
    F (\qabs v F v v) (\qabs v F v v)
    =
    FP.
  \end{equation*}

  Since \( YF \bequiv P \) and \( P \bequiv FP \), we conclude that \( YF \bequiv F(YF) \).
\end{proof}

\paragraph{Confluence}

\begin{definition}\label{def:relation_confluence}\mimprovised
  We say that the binary relation \( {\to} \) is (single-step) \term{confluent} if, whenever \( M \to N \) and \( M \to K \), there exists some \( L \) such that \( N \to L \) and \( K \to L \). We say that \( L \) is a \term{confluence point} of \( N \) and \( K \).

  \begin{figure}[!ht]
    \centering
    \includegraphics[page=1]{output/def__relation_confluence}
    \caption{\hyperref[def:relation_confluence]{Single-step confluence} as a \hyperref[def:hasse_diagram]{Hasse diagram}.}\label{fig:def:relation_confluence}
  \end{figure}
\end{definition}
\begin{comments}
  \item It is natural to define multi-step confluence of \( {\to} \) as single-step confluence of the reflexive and transitive closure of \( {\to} \). This makes little sense for reduction relations, since it would leave us with \enquote{multi-step confluence of single-step reduction relations} and \enquote{single-step confluence of multi-step reduction relations}.

  Furthermore, by having chosen to make \( \alpha \)-equivalence explicit, we must rely on the \hyperref[def:alpha_reflexive_closure]{\( \alpha \)-reflexive} closure instead of the reflexive closure, which forces us to restrict our attention to reduction relations rather than arbitrary relations.

  We chose to not introduce multi-step confluence.

  \item \incite[ch. 3]{Barendregt1984} uses the term \enquote{diamond property} for what we call \enquote{single-step confluence} and \enquote{Church-Rosser property} for what we would call \enquote{multi-step confluence}. \incite[3.4.1]{Mimram2020} uses \enquote{confluence} for what we would call \enquote{multi-step confluence}, while \incite[14]{TroelstraSchwichtenberg2000} use both \enquote{confluence} and \enquote{Church-Rosser property} for what we call \enquote{single-step confluence}.
\end{comments}

\begin{proposition}\label{thm:confluence_of_reflexive_transitive_closure}
  If a relation is \hyperref[def:relation_confluence]{single-step confluent}, so is its \hyperref[def:relation_closures/transitive]{reflexive} and \hyperref[def:relation_closures/transitive]{transitive closure}.

  More generally, let \( {\to} \) be a relation and let \( {\redrel*{}} \) be its corresponding closure. Suppose that, whenever \( M \to N \) and \( M \to K \), there exists some \( L \) such \( N \redrel*{} L \) and \( K \redrel*{} L \).

  Then \( {\redrel*{}} \) is confluent.
\end{proposition}
\begin{proof}
  Suppose that \( M \redrel*{} N \) and \( M \redrel*{} K \). Then there exist sequences
  \begin{equation*}
    M = N_0 \to N_1 \to \cdots \to N_n = N
  \end{equation*}
  and
  \begin{equation*}
    M = K_0 \to K_1 \to \cdots \to K_k = K.
  \end{equation*}

  We will use induction on \( i = 0, \ldots, \min\set{ n, k } \) to show that there exists a \( {\redrel*{}} \)-confluence point \( L_i \) such that \( N_i \redrel*{} L_i \) and \( K_i \redrel*{} L_i \).

  \begin{itemize}
    \item In the base case \( i = 0 \), \( M = N_i = K_i \), thus we can take \( L_i \) to be \( M \) itself.
    \item Suppose that the inductive hypothesis holds for \( i < \min\set{ n, k } \) and let \( L_i \) be the \( {\redrel*{}} \)-confluence point of \( N_i \) and \( K_i \).

    We thus have \( N_i \to N_{i+1} \) and \( N_i \redrel*{} L_i \). Since \( \to \) is single-step confluent, there exists a \( {\to} \)-confluence point \( A \) such that \( N_{i+1} \to A \) and \( L_i \to A \). Let \( B \) be a \( {\to} \)-confluence point of \( K_{i+1} \) and \( L_i \).

    Finally, let \( L_{i+1} \) be a \( {\to} \)-confluence point of \( A \) and \( B \). Then \( N_{i+1} \to A \to L_{i+1} \) and \( K_{i+1} \to B \to L_{i+1} \).

    \begin{figure}[!ht]
      \centering
      \includegraphics[page=1]{output/fig__thm__single_step_confluent_implies_multi_step}
      \caption{An illustration of the inductive step in the proof of \fullref{thm:confluence_of_reflexive_transitive_closure}}\label{fig:thm:confluence_of_reflexive_transitive_closure}
    \end{figure}
  \end{itemize}

  We now have the following possibilities:
  \begin{itemize}
    \item If \( n = k \), let \( L \coloneqq L_n = L_k \).
    \item If \( n > k \), for \( i > k \) let \( L_{i+1} \) be a \( {\to} \)-confluence point of \( N_{i+1} \) and \( L_i \), and let \( L \coloneqq L_n \).
    \item If \( n < k \), for \( i > n \) let \( L_{i+1} \) be a \( {\to} \)-confluence point of \( K_{i+1} \) and \( L_i \), and let \( L \coloneqq L_k \).
  \end{itemize}

  In all cases, \( N \redrel*{} L \) and \( K \redrel*{} L \).
\end{proof}

\begin{corollary}\label{thm:confluence_of_multi_step_reduction}
  If a \hyperref[def:lambda_reduction]{single-step reduction relation} is \hyperref[def:relation_confluence]{confluent}, so is the corresponding multi-step reduction relation.

  More generally, suppose that, whenever \( M \rred N \) and \( M \rred K \), there exists some \( \synlambda \)-term \( L \) such \( N \rred* L \) and \( K \rred* L \).

  Then \( {\rred*} \) is confluent.
\end{corollary}
\begin{proof}
  \Fullref{thm:confluence_of_reflexive_transitive_closure} implies that its reflexive and transitive closure \( {\rred*_R} \) of \( {\rred} \) is confluent.

  But \( {\rred*} \) is the \( \alpha \)-reflexive and transitive closure of \( {\rred} \). Thus, we have the following possibilities:
  \begin{itemize}
    \item If \( M \aequiv N \) and \( M \aequiv K \), let \( L \coloneqq M \). Then \( N \rred* L \) and \( K \rred* L \).

    \item If \( M \aequiv N \) and \( M \rred*_R K \), let \( L \coloneqq K \). Clearly \( K \rred* L \). To show that \( N \rred* L \), consider a concrete sequence
    \begin{equation*}
      M = K_0 \rred K_1 \rred \cdots \rred K_k = K.
    \end{equation*}

    \begin{itemize}
      \item If \( k = 0 \), then \( M = K \), and \( N \aequiv L \).
      \item If \( K > 0 \), then \( N \aequiv M \rred K_1 \), hence \ref{def:parallel_beta_reduction/alpha} allows us to conclude that \( N \rred K_1 \). By transitivity, \( N \rred* K = L \).
    \end{itemize}

    \item If \( M \rred*_R N \) and \( M \aequiv K \), we swap \( N \) and \( K \) and use the previous case.

    \item If \( M \rred*_R N \) and \( M \rred*_R K \), let \( L \) be a \( {\rred*_R} \)-confluence point of \( N \) and \( K \).
  \end{itemize}
\end{proof}

\paragraph{Confluence of \( \beta \)-reduction}\hfill

We will introduce a definition that we will find useful mostly for proving \fullref{thm:church_rosser_theorem_for_beta_reduction}.

\begin{definition}\label{def:parallel_beta_reduction}
  We define \term{parallel \( \beta \)-reduction} via the following rules:

  \begin{equation*}\taglabel[\ensuremath{ \logic{Red}_\Vert }]{def:parallel_beta_reduction/red}
    \begin{prooftree}
      \hypo{ A \pred C }
      \hypo{ B \pred D }
      \infer2[\ref{def:parallel_beta_reduction/red}]{ (\qabs x A)B \pred C[x \mapsto D] }
    \end{prooftree}
  \end{equation*}

  \begin{ThreeColumns}
    \begin{equation*}\taglabel[\ensuremath{ \logic{Var}_\Vert }]{def:parallel_beta_reduction/var}
      \begin{prooftree}
        \infer0[\ref{def:parallel_beta_reduction/var}]{ x \pred x }
      \end{prooftree}
    \end{equation*}
  \BeginSecondColumn
    \begin{equation*}\taglabel[\ensuremath{ \logic{App}_\Vert }]{def:parallel_beta_reduction/app}
      \begin{prooftree}
        \hypo{ A \pred C }
        \hypo{ B \pred D }
        \infer2[\ref{def:parallel_beta_reduction/app}]{ AB \pred CD }
      \end{prooftree}
    \end{equation*}
  \BeginThirdColumn
    \begin{equation*}\taglabel[\ensuremath{ \logic{Abs}_\Vert }]{def:parallel_beta_reduction/abs}
      \begin{prooftree}
        \hypo{ A \pred B }
        \infer1[\ref{def:parallel_beta_reduction/abs}]{ \qabs x A \pred \qabs x B }
      \end{prooftree}
    \end{equation*}
  \end{ThreeColumns}

  \begin{equation*}\taglabel[\ensuremath{ \logic{Alpha}_\Vert }]{def:parallel_beta_reduction/alpha}
    \begin{prooftree}
      \hypo{ A \aequiv B }
      \hypo{ B \pred C }
      \hypo{ C \aequiv D }
      \infer3[\ref{def:parallel_beta_reduction/alpha}]{ C \pred D }
    \end{prooftree}
  \end{equation*}
\end{definition}
\begin{comments}
  \item The rules are based on \cite[3.2.3]{Barendregt1984}, but with \ref{def:parallel_beta_reduction/alpha} being made explicit and with \ref{def:parallel_beta_reduction/var} holding only for variables.
\end{comments}

\begin{proposition}\label{thm:def:parallel_beta_reduction}
  \hyperref[def:parallel_beta_reduction]{Parallel \( \beta \)-reduction} has the following basic properties:
  \begin{thmenum}
    \thmitem{thm:def:parallel_beta_reduction/reflexive} It is \hyperref[def:alpha_reflexive_closure]{\( \alpha \)-reflexive}.
    \thmitem{thm:def:parallel_beta_reduction/free} If \( M \pred N \), then \( \op*{Free}(N) \subseteq \op*{Free}(M) \).
  \end{thmenum}
\end{proposition}
\begin{proof}
  \SubProofOf{thm:def:parallel_beta_reduction/reflexive} We will use \fullref{thm:least_fixed_point_induction} on \( M \aequiv N \) to show that \( M \pred N \):
  \begin{itemize}
    \item If \( M \aequiv N \) due to \ref{def:lambda_term_alpha_equivalence/var}, then \( M \) and \( N \) are equal variables. The rule \ref{def:parallel_beta_reduction/var} implies that \( M \pred M \).

    \item If \( M \aequiv N \) due to \ref{def:lambda_term_alpha_equivalence/app}, then \( M = AB \) and \( N = CD \), where \( A \equiv C \) and \( B \aequiv D \). The inductive hypothesis implies that \( A \pred C \) and \( B \pred D \). We then use \ref{def:parallel_beta_reduction/app} to conclude that \( M = AB \pred CD = N \).

    \item If \( M \aequiv N \) due to \ref{thm:alpha_equivalence_simplified/lift}, then \( M = \qabs x A \) and \( N = \qabs x B \), where \( A \aequiv B \). The inductive hypothesis implies that \( A \pred B \), and \ref{def:parallel_beta_reduction/abs} implies that \( M \pred N \).

    \item Finally, if \( M \aequiv N \) due to \ref{thm:alpha_equivalence_simplified/ren}, then \( M = \qabs a A \), \( N = \qabs b B \), \( a \) is not free in \( B \) and \( A \aequiv B[b \mapsto a] \).

    The inductive hypothesis implies that \( A \pred B[b \mapsto a] \) and \ref{def:parallel_beta_reduction/abs} implies that
    \begin{equation*}
      \qabs a A \pred \qabs a B[b \mapsto a].
    \end{equation*}

    Furthermore, \fullref{thm:alpha_conversion} implies that \( \qabs b B \aequiv \qabs a B[b \mapsto a] \). We can thus use \ref{def:parallel_beta_reduction/alpha} to conclude that
    \begin{equation*}
      M = \qabs a A \pred \qabs b B.
    \end{equation*}
  \end{itemize}

  \SubProofOf{thm:def:parallel_beta_reduction/free} We will use \fullref{thm:least_fixed_point_induction} on \( M \pred N \):
  \begin{itemize}
    \item If \( M \pred N \) due to \ref{def:parallel_beta_reduction/var}, then \( M \) and \( N \) are identical variables. Vacuously, their free variables coincide.

    \item If \( M \pred N \) due to \ref{def:parallel_beta_reduction/app}, the inductive hypothesis implies that their free variables coincide.

    \item If \( M \pred N \) due to \ref{def:parallel_beta_reduction/abs}, the inductive hypothesis implies that their free variables coincide.

    \item If \( M \pred N \) due to \ref{def:parallel_beta_reduction/alpha}, then \( M \aequiv B \pred C \aequiv N \). Suppose that the inductive hypothesis holds for \( B \pred C \).

    Then
    \begin{equation*}
      \op*{Free}(N)
      \reloset {\ref{thm:def:lambda_term_alpha_equivalence/free}} =
      \op*{Free}(C)
      \reloset {\T{ind.}} \subseteq
      \op*{Free}(B)
      \reloset {\ref{thm:def:lambda_term_alpha_equivalence/free}} =
      \op*{Free}(M).
    \end{equation*}

    \item Finally, if \( M \pred N \) due to \ref{def:parallel_beta_reduction/red}, then \( M = (\qabs x A)B \) and \( N = C[x \mapsto D] \), where \( A \pred C \) and \( B \pred D \). Suppose that the inductive hypothesis holds for \( A \pred C \) and \( B \pred D \).

    Then
    \begin{align*}
      \op*{Free}(C[x \mapsto D])
      &\reloset {\eqref{eq:thm:beta_reduction_free_variables/inequality}} \subseteq
      \op*{Free}((\qabs x C) D)
      = \\ &=
      \parens[\Big]{ \op*{Free}(C) \setminus \set{ x } } \cup \op*{Free}(D)
      \reloset {\T{ind.}} \subseteq \\ &\subseteq
      \parens[\Big]{ \op*{Free}(A) \setminus \set{ x } } \cup \op*{Free}(B)
      = \\ &=
      \op*{Free}((\qabs x A) B).
    \end{align*}
  \end{itemize}
\end{proof}

\begin{proposition}\label{thm:parallel_beta_reduction_transitive_closure}\mcite[3.2.7]{Barendregt1984}
  The \hyperref[def:relation_closures/transitive]{transitive closure} of the \hyperref[def:parallel_beta_reduction]{parallel \( \beta \)-reduction} \( {\pred} \) coincides with the \hyperref[def:lambda_reduction/reduction]{multi-step \( \beta \)-reduction} \( {\bred*} \).
\end{proposition}
\begin{proof}
  We will show that, when regarded as sets, the relations obey the following inequalities:
  \begin{equation*}
    {\bred} \subseteq {\pred} \subseteq {\bred*}.
  \end{equation*}

  Taking their transitive closures will then give us
  \begin{equation*}
    {\bred*} \subseteq \cl_T({\pred}) \subseteq {\bred*},
  \end{equation*}
  which is the required result.

  \SubProof{Proof that \( M \bred N \) implies \( M \pred N \)} We will use \fullref{thm:least_fixed_point_induction} on \( M \bred N \):
  \begin{itemize}
    \item If \( M \bred N \) due to \ref{def:lambda_reduction/app_left}, then \( M = LA \) and \( N = LB \), where \( A \bred B \). Suppose that the inductive hypothesis holds for the latter, so that \( A \pred B \).

    Since \( {\pred} \) is reflexive, we have \( L \pred L \). Then \ref{def:parallel_beta_reduction/app} implies that \( M \pred N \).

    \item If \( M \bred N \) due to \ref{def:lambda_reduction/app_right}, we act analogously.

    \item If \( M \bred N \) due to \ref{def:lambda_reduction/abs}, we use \ref{def:parallel_beta_reduction/abs} to conclude that \( M \pred N \).

    \item If \( M \bred N \) due to \ref{def:lambda_reduction/alpha}, we use \ref{def:parallel_beta_reduction/alpha}.

    \item Finally, if \( M \bred N \) due to \ref{def:lambda_reduction/beta}, then \( M = (\qabs x A) B \) and \( N = A[x \mapsto B] \). Again, since \( {\pred} \) is reflexive, we have \( A \pred A \) and \( B \pred B \), hence \ref{def:parallel_beta_reduction/red} allows us to conclude that \( M \pred N \).
  \end{itemize}

  \SubProof{Proof that \( M \pred N \) implies \( M \bred* N \)} We will use \fullref{thm:least_fixed_point_induction} on \( M \pred N \):
  \begin{itemize}
    \item If \( M \pred N \) due to \ref{def:parallel_beta_reduction/var}, then \( M = N \) and, since \( {\bred*} \) is reflexive by definition, \( M \bred* N \).

    \item If \( M \pred N \) due to \ref{def:parallel_beta_reduction/app}, then \( M = AB \) and \( N = CD \), where \( A \pred C \) and \( B \pred D \) and the inductive hypothesis holds for the latter two.

    Then \( A \bred* C \) and \( B \bred* D \). If
    \begin{equation*}
      A = A_1 \bred \cdots \bred A_n = C,
    \end{equation*}
    then \ref{def:lambda_reduction/app_left} allows us to conclude that
    \begin{equation*}
      AB = A_1 B \bred \cdots \bred A_n B = CB.
    \end{equation*}

    If
    \begin{equation*}
      B = B_1 \bred \cdots \bred B_m = D,
    \end{equation*}
    then \ref{def:lambda_reduction/app_right} allows us to conclude that
    \begin{equation*}
      CB = C B_1 \bred \cdots \bred C B_m = CD.
    \end{equation*}

    Therefore, \( AB \bred* CD \).

    \item If \( M \pred N \) due to \ref{def:parallel_beta_reduction/abs}, we can similarly chain the individual reductions to conclude that \( M \bred* N \) using \ref{def:lambda_reduction/abs}.

    \item If \( M \pred N \) due to \ref{def:parallel_beta_reduction/alpha}, then \( M \aequiv A \pred B \aequiv N \). The inductive hypothesis on \( A \pred B \) gives us \( A \bred* B \). Suppose that
    \begin{equation*}
      A = A_1 \bred \cdots \bred A_n = B.
    \end{equation*}

    \begin{itemize}
      \item If \( n = 1 \), then \( A = B \) and thus \( M \aequiv N \). Then \( M \bred* N \) since the latter is an \hyperref[def:alpha_reflexive_closure]{\( \alpha \)-reflexive closure}.

      \item Otherwise, we have
      \begin{equation*}
        M \aequiv A_1 \bred A_2 \aequiv A_2,
      \end{equation*}
      hence \ref{def:lambda_reduction/alpha} implies \( M \bred A_2 \). Analogously, we can apply \ref{def:lambda_reduction/alpha} to
      \begin{equation*}
        A_{n-1} \bred A_n = B \aequiv N
      \end{equation*}
      to obtain \( A_{n-1} \bred N \).

      By transitivity, we conclude \( M \bred* N \).
    \end{itemize}

    \item Finally, if \( M \pred N \) due to \ref{def:parallel_beta_reduction/red}, then \( M = (\qabs x A)B \) and \( N = C[x \mapsto D] \), where \( A \pred C \) and \( B \pred D \). The inductive hypothesis implies that \( A \bred* C \) and \( B \bred* D \).

    Applying \ref{def:lambda_reduction/abs} and \ref{def:lambda_reduction/app_left} successively, we conclude that
    \begin{equation*}
      (\qabs x A) B \bred* (\qabs x C) B.
    \end{equation*}

    Similarly, applying \ref{def:lambda_reduction/app_right} successively, we conclude that
    \begin{equation*}
      (\qabs x C) B \bred* (\qabs x C) D.
    \end{equation*}

    Then \ref{def:lambda_reduction/beta} implies that
    \begin{equation*}
      (\qabs x C) D \bred C[x \mapsto D].
    \end{equation*}

    Chaining the above, we obtain \( M \bred* N \).
  \end{itemize}
\end{proof}

\begin{proposition}\label{thm:substitution_on_parallel_reduction}\mcite[3.2.4]{Barendregt1984}
  Fix \( \synlambda \)-terms \( M \) and \( N \) such that \( M \pred N \). Fix substitutions \( \sigma \) and \( \rho \) such that \( \sigma(u) \pred \rho(u) \) for every variable \( u \).

  Then \( M[\sigma] \pred N[\rho] \).
\end{proposition}
\begin{proof}
  We will use \fullref{thm:least_fixed_point_induction} on \( M \pred N \) simultaneously on all compatible substitutions:
  \begin{itemize}
    \item If \( M \pred N \) due to \ref{def:parallel_beta_reduction/var}, then \( M \) and \( N \) are identical variables. Thus, \( \sigma(M) \pred \rho(N) \).

    \item If \( M \pred N \) due to \ref{def:parallel_beta_reduction/app}, then \( M = AB \) and \( N = CD \), where \( A \pred C \) and \( B \pred D \). Suppose that the inductive hypothesis holds for \( A \pred C \) and \( B \pred D \). Then \( A[\sigma] \pred C[\rho] \) and \( B[\sigma] \pred D[\rho] \), hence \( M[\sigma] \pred N[\rho] \).

    \item If \( M \pred N \) due to \ref{def:parallel_beta_reduction/abs}, then \( M = \qabs x A \) and \( N = \qabs x B \) and \( A \pred B \). Suppose the inductive hypothesis holds for \( A \pred B \).

    We will utilize \fullref{thm:lambda_substitution_single_rule} twice:
    \begin{align*}
      M[\sigma] &= \qabs u A[\sigma_{x \mapsto u}], &&u \not\in \op*{Free}(M) \cup \op*{Free}(M[\sigma]), \\
      N[\rho]   &= \qabs v B[\rho_{x \mapsto v}],   &&v \not\in \op*{Free}(N) \cup \op*{Free}(N[\rho]).
    \end{align*}

    The inductive hypothesis implies that
    \begin{equation*}
      A[\sigma_{x \mapsto u}] \pred B[\sigma_{x \mapsto u}].
    \end{equation*}

    \begin{itemize}
      \item If \( u \) is free in \( B[\rho_{x \mapsto v}] \), then it must be equal to \( v \), and \ref{def:parallel_beta_reduction/abs} implies that \( M[\sigma] \pred N[\rho] \).

      To show that \( u = v \), suppose instead that \( u \neq v \). Then \( u \) is free in \( \rho(w) \) for some \( w \) free in \( B \), and \fullref{thm:def:parallel_beta_reduction/free} implies that \( u \) is free in \( \sigma(w) \). But, again by \fullref{thm:def:parallel_beta_reduction/free}, the free variables of \( B \) are free in \( A \). Then \( u \) must be free in \( \sigma(w) \) for \( w \) free in \( A \). But this contradicts our assumption that \( u \) is not free in \( \op*{Free}(M[\sigma]) \).

      \item If \( u \) is not free in \( B[\rho_{x \mapsto v}] \), \fullref{thm:alpha_conversion_modified} implies that
      \begin{equation*}
        \qabs u B[\rho_{x \mapsto u}]
        \aequiv
        \qabs v B[\rho_{x \mapsto v}].
      \end{equation*}

      Then
      \begin{equation*}
        M[\sigma]
        =
        \qabs u A[\sigma_{x \mapsto u}]
        \pred
        \qabs u B[\sigma_{x \mapsto v}]
        \aequiv
        \qabs v B[\rho_{x \mapsto v}]
        =
        N[\rho]
      \end{equation*}
      and \ref{def:parallel_beta_reduction/alpha} implies that \( M[\sigma] \pred N[\rho] \).
    \end{itemize}

    \item If \( M \pred N \) due to \ref{def:parallel_beta_reduction/alpha}, then \( M \aequiv B \pred C \aequiv N \), where the inductive hypothesis holds for \( B \pred C \).

    \Fullref{thm:substitution_on_alpha_equivalent_terms} implies that \( M[\sigma] \aequiv B[\sigma] \) and \( C[\rho] \aequiv N[\rho] \), and the inductive hypothesis implies that \( B[\sigma] \pred C[\rho] \). Then \ref{def:parallel_beta_reduction/alpha} allows us to conclude that \( M[\sigma] \aequiv N[\rho] \).

    \item Finally, if \( M \pred N \) due to \ref{def:parallel_beta_reduction/red}, then \( M = (\qabs x A)B \) and \( N = C[x \mapsto D] \), where \( A \pred C \) and \( B \pred D \). Suppose that the inductive hypothesis holds for \( A \pred C \) and \( B \pred D \).

    \Fullref{thm:lambda_substitution_single_rule} implies that
    \begin{equation*}
      M[\sigma] = \parens[\Big]{ \qabs u A[\sigma_{x \mapsto u}] } B[\sigma],
    \end{equation*}
    where \( u \not\in \op*{Free}(\qabs x A) \cup \op*{Free}_\sigma(\qabs x A) \).

    The inductive hypothesis implies that \( A[\sigma_{x \mapsto u}] \pred C[\rho_{x \mapsto u}] \) and \( B[\sigma] \pred D[\rho] \). Then \ref{def:parallel_beta_reduction/red} implies that
    \begin{equation*}
      M[\sigma]
      =
      \parens[\Big]{ \qabs u A[\sigma_{x \mapsto u}] } B[\sigma]
      \pred
      C[\rho_{x \mapsto u}][u \mapsto D[\rho]].
    \end{equation*}

    Furthermore,
    \begin{equation*}
      N[\rho]
      =
      C[x \mapsto D][\rho]
      \reloset {\eqref{eq:thm:substitution_composition_is_alpha_equivalent}} \aequiv
      C[\rho_{x \mapsto D[\rho]}]
      \reloset {\eqref{eq:thm:substitution_chain_contraction/contraction}} \aequiv
      C[\rho_{x \mapsto u}][u \mapsto D[\rho]].
    \end{equation*}

    Therefore, \ref{def:parallel_beta_reduction/alpha} allows us to conclude that \( M[\sigma] \pred N[\rho] \).
  \end{itemize}
\end{proof}

\begin{lemma}\label{thm:parallel_beta_reduction_deconstruction}
  If \( M \pred N \), depending on the structure of \( M \), we have the following possibilities:
  \begin{thmenum}
    \thmitem{thm:parallel_beta_reduction_deconstruction/app} If \( M = AB \), we have two possibilities:
    \begin{thmenum}
      \thmitem{thm:parallel_beta_reduction_deconstruction/app/app} \( N = CD \), where \( A \pred C \) and \( B \pred D \).

      \thmitem{thm:parallel_beta_reduction_deconstruction/app/red} \( A = \qabs x E \) and \( N \aequiv F[x \mapsto G] \), where \( B \pred G \) and \( E \pred F \).
    \end{thmenum}

    \thmitem{thm:parallel_beta_reduction_deconstruction/abs} If \( M = \qabs x A \), then \( N \aequiv \qabs x B \), where \( A \pred B \).
  \end{thmenum}
\end{lemma}
\begin{comments}
  \item The main purpose of this lemma is to simplify handling \ref{def:parallel_beta_reduction/alpha} in the proof of \fullref{thm:parallel_beta_church_rosser}.
\end{comments}
\begin{proof}
  \SubProofOf{thm:parallel_beta_reduction_deconstruction/app} Let \( M = AB \). We will use \fullref{thm:least_fixed_point_induction} on \( AB \pred N \):
  \begin{itemize}
    \item If \( AB \pred N \) due to \ref{def:parallel_beta_reduction/app}, then \fullref{thm:parallel_beta_reduction_deconstruction/app/app} holds.

    \item If \( AB \pred N \) due to \ref{def:parallel_beta_reduction/red}, then \fullref{thm:parallel_beta_reduction_deconstruction/app/red} holds.

    \item If \( AB \pred N \) due to \ref{def:parallel_beta_reduction/alpha}, then \( AB \aequiv A' B' \) and \( N \aequiv N' \), where \( A' B' \pred N' \) and the inductive hypothesis holds for the latter.

    \Fullref{thm:parallel_beta_reduction_deconstruction/app} lists the following possibilities:
    \begin{itemize}
      \item The possibility \fullref{thm:parallel_beta_reduction_deconstruction/app/app} for \( A' B' \pred N' \) implies that \( N' = C' D' \), where \( A' \pred C' \) and \( B' \pred D' \). Furthermore, due to \ref{def:lambda_term_alpha_equivalence/app}, we conclude that \( N = CD \), where \( C \aequiv C' \) and \( D \aequiv D' \).

      Then \ref{def:parallel_beta_reduction/alpha} implies that \( A \pred C \) and \( B \pred D \), which is the desired result.

      \item The possibility \fullref{thm:parallel_beta_reduction_deconstruction/app/red} for \( A' B' \pred N' \) implies that \( A' = \qabs y E' \) and \( N' = F'[y \mapsto G'] \), where \( B' \pred G' \) and \( E' \pred F' \).

      \begin{itemize}
        \item If \( A \aequiv A' \) due to \ref{thm:alpha_equivalence_simplified/lift}, then \( A = \qabs y E \), where \( E \aequiv E' \). Via \ref{def:parallel_beta_reduction/alpha} we conclude that \( E \pred F' \).

        Then \fullref{thm:parallel_beta_reduction_deconstruction/app/red} holds with \( x \coloneqq y \), \( G \coloneqq G' \) and \( F \coloneqq F' \).

        \item If \( A \aequiv A' \) due to \ref{thm:alpha_equivalence_simplified/ren}, then \( A = \qabs x E \), where \( x \) is not free in \( E' \) and \( E \aequiv E'[y \mapsto x] \).

        Let \( F \coloneqq F'[y \mapsto x] \). \Fullref{thm:substitution_on_parallel_reduction} implies that \( E'[y \mapsto x] \pred F \) and \ref{def:parallel_beta_reduction/alpha} implies that \( E \pred F \).

        By \fullref{thm:def:parallel_beta_reduction/free}, \( x \) is not free in \( F' \) since otherwise it would be free in \( E' \). Let \( G \coloneqq G' \). Then \fullref{thm:substitution_chain_contraction/contraction} implies that
        \begin{equation*}
          N'
          =
          F'[y \mapsto G']
          \aequiv
          F'[y \mapsto x][x \mapsto G']
          =
          F[x \mapsto G],
        \end{equation*}
        thus \fullref{thm:parallel_beta_reduction_deconstruction/app/red} holds.
      \end{itemize}
    \end{itemize}
  \end{itemize}

  \SubProofOf{thm:parallel_beta_reduction_deconstruction/abs} Let \( M = \qabs x A \). We will use \fullref{thm:least_fixed_point_induction} on \( (\qabs x A) \pred N \):
  \begin{itemize}
    \item If \( (\qabs x A) \pred N \) due to \ref{def:parallel_beta_reduction/abs}, then \fullref{thm:parallel_beta_reduction_deconstruction/abs} holds.

    \item If \( (\qabs x A) \pred N \) due to \ref{def:parallel_beta_reduction/alpha}, then \( \qabs x A = M \aequiv M' \) and \( N \aequiv N' \), where \( M' \pred N' \) and the inductive hypothesis holds for the latter.

    \begin{itemize}
      \item If \( M \aequiv M' \) due to \ref{thm:alpha_equivalence_simplified/lift}, then \( M' = \qabs x A' \), where \( A \aequiv A' \).

      The inductive hypothesis on \( (\qabs x A') \pred N' \) implies that \( N' \aequiv \qabs x B' \), where \( A' \pred B' \). Let \( B \coloneqq B' \). Then \ref{def:parallel_beta_reduction/alpha} implies that \( A \pred B \).

      Since \( N \aequiv N' \aequiv \qabs x B \), we conclude that \fullref{thm:parallel_beta_reduction_deconstruction/abs} holds.

      \item If \( M \aequiv M' \) due to \ref{thm:alpha_equivalence_simplified/ren}, then \( M' = \qabs y A' \), where \( x \) is not free in \( A' \) and we have \( A \aequiv A'[y \mapsto x] \).

      The inductive hypothesis on \( (\qabs y A') \pred N' \) implies that \( N' \aequiv \qabs y B' \), where \( A' \pred B' \). \Fullref{thm:substitution_on_parallel_reduction} implies that \( A'[y \mapsto x] \pred B'[y \mapsto x] \).

      Let \( B \coloneqq B'[y \mapsto x] \). Then \ref{def:parallel_beta_reduction/alpha} implies that \( A \pred B \).

      Furthermore, \fullref{thm:alpha_conversion} implies that \( N' \aequiv \qabs x B \) and thus \fullref{thm:parallel_beta_reduction_deconstruction/abs} holds.
    \end{itemize}
  \end{itemize}
\end{proof}

\begin{proposition}\label{thm:parallel_beta_church_rosser}
  \hyperref[def:parallel_beta_reduction]{Parallel \( \beta \)-reduction} is \hyperref[def:relation_confluence]{confluent}.
\end{proposition}
\begin{proof}
  Suppose that \( M \pred N \) and \( M \pred K \). We will use \fullref{thm:least_fixed_point_induction} on \( M \pred N \) simultaneously on all \( K \) to show the existence of the desired confluence point \( L \) such that \( N \pred L \) and \( N \pred L \):
  \begin{itemize}
    \item If \( M \pred N \) due to \ref{def:parallel_beta_reduction/var}, then \( M \) and \( N \) are identical variables. Furthermore, \ref{def:parallel_beta_reduction/var} is the only rule allowing to conclude \( M \pred K \), hence \( M = N = K \).

    Define \( L \coloneqq M \) so that \( N \pred L \) and \( K \pred L \), again due to \ref{def:parallel_beta_reduction/var}.

    \item If \( M \pred N \) due to \ref{def:parallel_beta_reduction/app}, we have \( M = AB \) and \( N = CD \), where \( A \pred C \) and \( B \pred D \) and the inductive hypothesis holds for the latter two.

    \Fullref{thm:parallel_beta_reduction_deconstruction/app} lists the following possibilities for \( AB \pred K \):
    \begin{itemize}
      \item In the case \fullref{thm:parallel_beta_reduction_deconstruction/app/app}, we have \( K = EF \), where \( A \pred E \) and \( B \pred F \).

      The inductive hypothesis implies the existence of confluence points \( G \) of \( C \) and \( E \) and \( H \) of \( D \) and \( F \).

      Let \( L \) be \( GH \). Then \ref{def:parallel_beta_reduction/app} allows us to conclude that \( N \pred L \) and \( K \pred L \).

      \item In the case \fullref{thm:parallel_beta_reduction_deconstruction/app/red}, we have \( A = \qabs x E \) and \( K \aequiv F[x \mapsto G] \), where \( B \pred G \) and \( E \pred F \).

      We have \( A = \qabs x E \pred \qabs x F \) due to \ref{def:parallel_beta_reduction/abs}. By the inductive hypothesis on \( A \pred C \), there exists a confluence point \( P \) of \( C \) and \( \qabs x F \).

      \Fullref{thm:parallel_beta_reduction_deconstruction/abs} implies that \( P \aequiv \qabs x Q \), where \( F \pred Q \).

      Furthermore, by the inductive hypothesis on \( B \pred D \), there exists a confluence point \( R \) of \( D \) and \( G \).

      Let \( L \coloneqq Q[x \mapsto R] \). We will show that this is the desired term.
      \begin{itemize}
        \item Since \( A \pred C \), \fullref{thm:parallel_beta_reduction_deconstruction/abs} implies that \( C \aequiv \qabs x S \), where \( E \pred S \). Furthermore, since \( \qabs x S \aequiv C \pred P \aequiv \qabs x Q \), \ref{def:parallel_beta_reduction/alpha} and \ref{def:parallel_beta_reduction/abs} imply that \( S \pred Q \).

        Then we can use \ref{def:parallel_beta_reduction/red} to reduce \( N = CD \aequiv (\qabs x S) D \) to \( L = Q[x \mapsto R] \). Thus, \( N \pred L \).

        \item Since \( F \pred Q \), \fullref{thm:substitution_on_parallel_reduction} implies that
        \begin{equation*}
          K \aequiv F[x \mapsto G] \pred Q[x \mapsto R] \aequiv L
        \end{equation*}

         Then \ref{def:parallel_beta_reduction/alpha} implies that \( K \pred L \).
      \end{itemize}
    \end{itemize}

    \item If \( M \pred N \) due to \ref{def:parallel_beta_reduction/abs}, then \( M = \qabs x A \) and \( N = \qabs x B \), where \( A \pred B \) and the inductive hypothesis holds for the latter.

    \Fullref{thm:parallel_beta_reduction_deconstruction/abs} implies that \( K \aequiv x C \), where \( A \pred C \). By the inductive hypothesis, there exists a confluence point \( D \) of \( B \) and \( C \)

    Let \( L \coloneqq \qabs x D \). Then \ref{def:parallel_beta_reduction/abs} implies that \( N \pred L \) and \( K \pred L \).

    \item If \( M \pred N \) due to \ref{def:parallel_beta_reduction/alpha}, then \( M \aequiv B \pred C \aequiv N \), where the inductive hypothesis holds for \( B \pred C \).

    Then \( B \pred K \) due to \ref{def:parallel_beta_reduction/alpha}, and the inductive hypothesis implies that there exists some \( \synlambda \)-term \( L \) such that \( C \pred L \) and \( K \pred L \).

    We can use \ref{def:parallel_beta_reduction/alpha} again to conclude that \( N \pred L \).

    \item Finally, if \( M \pred N \) due to \ref{def:parallel_beta_reduction/red}, then \( M = (\qabs x A)B \) and \( N = C[x \mapsto D] \), where \( A \pred C \) and \( B \pred D \) and the inductive hypothesis holds for the latter two.

    \Fullref{thm:parallel_beta_reduction_deconstruction/app} lists the following possibilities for \( M \pred K \):
    \begin{itemize}
      \item In the case \fullref{thm:parallel_beta_reduction_deconstruction/app/app}, we have \( K = EF \), where \( \qabs x A \pred E \) and \( B \pred F \).

      \Fullref{thm:parallel_beta_reduction_deconstruction/abs} implies that \( E \aequiv \qabs x G \), where \( A \pred G \). By the inductive hypothesis on \( A \pred C \), there exists a confluence point \( P \) of \( C \) and \( G \). By the inductive hypothesis on \( B \pred D \), there exists a confluence point \( Q \) of \( D \) and \( F \).

      Let \( L \coloneqq P[x \mapsto Q] \).
      \begin{itemize}
        \item We can use \fullref{thm:substitution_on_parallel_reduction} to conclude that
        \begin{equation*}
          N = C[x \mapsto D] \pred P[x \mapsto Q] = L.
        \end{equation*}

        \item We can use \ref{def:parallel_beta_reduction/red} to conclude that
        \begin{equation*}
          K = EF \aequiv (\qabs x G) F \pred P[x \mapsto Q] = L.
        \end{equation*}

        Then \ref{def:parallel_beta_reduction/alpha} implies that \( K \pred L \).
      \end{itemize}

      \item In the case \fullref{thm:parallel_beta_reduction_deconstruction/app/red}, we have \( K \aequiv F[x \mapsto G] \), where \( B \pred G \) and \( A \pred F \).

      By the inductive hypothesis on \( A \pred C \), there exists a confluence point \( P \) of \( C \) and \( F \) and a confluence point \( Q \) of \( D \) and \( G \).

      Let \( L \coloneqq P[x \mapsto Q] \). Then \fullref{thm:substitution_on_parallel_reduction} implies that
      \begin{equation*}
        N = C[x \mapsto D] \pred P[x \mapsto Q] = L
      \end{equation*}
      and
      \begin{equation*}
        K = F[x \mapsto G] \pred P[x \mapsto Q] = L.
      \end{equation*}
    \end{itemize}
  \end{itemize}
\end{proof}

\begin{theorem}[Church-Rosser theorem for beta reduction]\label{thm:church_rosser_theorem_for_beta_reduction}
  \hyperref[def:beta_eta_reduction]{Multi-step \( \beta \)-reduction} is \hyperref[def:relation_confluence]{confluent}.
\end{theorem}
\begin{proof}
  By \fullref{thm:parallel_beta_church_rosser}, \hyperref[def:parallel_beta_reduction]{parallel \( \beta \)-reduction} is single-step confluent. By \fullref{thm:confluence_of_reflexive_transitive_closure}, so is its reflexive and transitive closure. Since it is reflexive, it is invariant under the reflexive closure.

  By \fullref{thm:parallel_beta_reduction_transitive_closure}, the transitive closure of parallel \( \beta \)-reduction coincides with multi-step \( \beta \)-reduction; thus multi-step \( \beta \)-reduction is confluent.
\end{proof}

\paragraph{Confluence of \( \eta \)-reduction}

\begin{lemma}\label{thm:eta_reduction_deconstruction}
  If \( M \ered N \), depending on the structure of \( M \), we have the following possibilities:
  \begin{thmenum}
    \thmitem{thm:eta_reduction_deconstruction/app} If \( M = AB \), we have two possibilities:
    \begin{thmenum}
      \thmitem{thm:eta_reduction_deconstruction/app/app_left} \( N \aequiv AD \), where \( B \ered D \).

      \thmitem{thm:eta_reduction_deconstruction/app/app_right} \( N \aequiv CB \), where \( A \ered C \).
    \end{thmenum}

    \thmitem{thm:eta_reduction_deconstruction/abs} If \( M = \qabs x A \), we also have two possibilities:
    \begin{thmenum}
      \thmitem{thm:eta_reduction_deconstruction/abs/lift} \( N \aequiv \qabs x B \), where \( A \ered B \).

      \thmitem{thm:eta_reduction_deconstruction/abs/red} \( A \aequiv Nx \) and \( x \not\in \op*{Free}(N) \).
    \end{thmenum}
  \end{thmenum}
\end{lemma}
\begin{proof}
  \SubProofOf{thm:eta_reduction_deconstruction/app} Let \( M = AB \). We will use \fullref{thm:least_fixed_point_induction} on \( AB \ered N \):
  \begin{itemize}
    \item If \( AB \ered N \) due to \ref{def:lambda_reduction/app_left}, then \fullref{thm:eta_reduction_deconstruction/app/app_left} holds.

    \item If \( AB \ered N \) due to \ref{def:lambda_reduction/app_right}, then \fullref{thm:eta_reduction_deconstruction/app/app_right} holds.

    \item If \( AB \ered N \) due to \ref{def:lambda_reduction/alpha}, then \( AB \aequiv A' B' \) and \( N \aequiv N' \), where \( A' B' \ered N' \) and the inductive hypothesis holds for the latter.

    \Fullref{thm:eta_reduction_deconstruction/app} lists the following possibilities for \( A' B' \ered N' \):
    \begin{itemize}
      \item In the case \fullref{thm:eta_reduction_deconstruction/app/app_left}, we have \( N' \aequiv A' D' \), where \( B' \ered D' \).

      Let \( D \coloneqq D' \). We have \( B \aequiv B' \ered D' = D \), hence \ref{def:parallel_beta_reduction/alpha} implies that \( B \ered D \).

      Furthermore, since \( A \aequiv A' \), we have \( N \aequiv N' \aequiv A D \).

      \item In the case \fullref{thm:eta_reduction_deconstruction/app/app_right}, we proceed analogously.
    \end{itemize}
  \end{itemize}

  \SubProofOf{thm:eta_reduction_deconstruction/abs} Let \( M = \qabs x A \). We will use \fullref{thm:least_fixed_point_induction} on \( (\qabs x A) \ered N \):
  \begin{itemize}
    \item If \( (\qabs x A) \ered N \) due to \ref{def:lambda_reduction/abs}, then \fullref{thm:eta_reduction_deconstruction/abs/lift} holds.

    \item If \( (\qabs x A) \ered N \) due to \ref{def:beta_eta_reduction/eta}, then \fullref{thm:eta_reduction_deconstruction/abs/red} holds.

    \item If \( (\qabs x A) \ered N \) due to \ref{def:lambda_reduction/alpha}, then \( \qabs x A = M \aequiv M' \) and \( N \aequiv N' \), where \( M' \pred N' \) and the inductive hypothesis holds for the latter.

    Furthermore, \ref{def:lambda_term_alpha_equivalence/abs} implies that \( M' = \qabs y A' \), where \( A[x \mapsto n] \aequiv A'[y \mapsto n] \) for every \( n \) not free in \( M \). In particular, \( A \aequiv A'[y \mapsto x] \).

    \Fullref{thm:eta_reduction_deconstruction/abs} lists the following possibilities for \( (\qabs y A') \ered N' \):
    \begin{itemize}
      \item In the case \fullref{thm:eta_reduction_deconstruction/abs/lift}, we have \( N' \aequiv \qabs y B' \), where \( A' \ered B' \).

      Let \( B \coloneqq B'[z \mapsto x] \). Then \fullref{thm:substitution_on_beta_eta_reduction} implies that \( A'[y \mapsto x] \ered B \) and \ref{def:parallel_beta_reduction/alpha} implies that \( A \ered B \).

      Furthermore, we have
      \begin{equation*}
        N
        \aequiv
        N'
        \aequiv
        \qabs y B'
        \reloset {\eqref{eq:thm:alpha_conversion}} \aequiv
        \qabs x B.
      \end{equation*}

      We conclude that \fullref{thm:eta_reduction_deconstruction/abs/lift} holds.

      \item In the case \fullref{thm:eta_reduction_deconstruction/abs/red}, we have \( A' \aequiv N'y \) and \( y \not\in \op*{Free}(N') \). Then, due to \fullref{thm:lambda_substitution_noop},
      \begin{equation*}
        \underbrace{A'[y \mapsto x]}_A
        =
        N'[y \mapsto x] x
        =
        N' x.
      \end{equation*}

      Applying \ref{def:lambda_term_alpha_equivalence/app}, we obtain \( N' x \aequiv N x \). Therefore,
      \begin{equation*}
        A \aequiv Nx.
      \end{equation*}

      Since \( y \) is not free in \( N' \), \fullref{thm:lambda_substitution_free_variables} implies that \( x \) is not free in \( N' = N'[y \mapsto x] \), and \fullref{thm:def:lambda_term_alpha_equivalence/free} implies that \( x \) is not free in \( N \).

      Thus, \fullref{thm:eta_reduction_deconstruction/abs/red} follows.
    \end{itemize}
  \end{itemize}
\end{proof}

\begin{theorem}[Church-Rosser theorem for eta reduction]\label{thm:church_rosser_theorem_for_eta_reduction}
  \hyperref[def:beta_eta_reduction]{Multi-step \( \eta \)-reduction} is \hyperref[def:relation_confluence]{confluent}.
\end{theorem}
\begin{proof}
  \SubProof{Proof for one step} We will use \fullref{thm:least_fixed_point_induction} on \( M \ered N \) to show that \( M \ered K \) implies the existence of a multi-step confluence point \( L \) such that \( N \ered* L \) and \( K \ered* L \).

  \begin{itemize}
    \item If \( M \ered N \) due to \ref{def:lambda_reduction/app_left}, then \( M = PB \) and \( N = PD \), where \( B \ered D \) and the inductive hypothesis holds for the latter.

    \Fullref{thm:eta_reduction_deconstruction/app} lists the following possibilities for \( PB \pred K \):
    \begin{itemize}
      \item The possibility \fullref{thm:eta_reduction_deconstruction/app/app_left} implies that \( K \aequiv PF \), where \( B \ered F \).

      By the inductive hypothesis on \( B \ered D \), there exists a multi-step confluence point \( R \) of \( D \) and \( F \). Let \( L \coloneqq PR \). Then, due to \ref{def:lambda_reduction/app_left}, by induction on the number of steps in \( {\ered*} \) we obtain
      \begin{equation*}
        N = PD \ered* PR = L
      \end{equation*}
      and
      \begin{equation*}
        K \aequiv PF \ered* PR = L,
      \end{equation*}
      which due to \ref{def:lambda_reduction/alpha} implies \( K \ered* L \).

      \item The possibility \fullref{thm:eta_reduction_deconstruction/app/app_right} implies that \( K \aequiv QB \), where \( P \ered* Q \). We proceed as follows:
      \begin{equation*}
        \includegraphics[page=1]{output/thm__church_rosser_theorem_for_eta_reduction}
      \end{equation*}
    \end{itemize}

    \item If \( M \ered N \) due to \ref{def:lambda_reduction/app_right}, we proceed analogously.

    \item If \( M \ered N \) due to \ref{def:lambda_reduction/abs}, then \( M = \qabs x A \) and \( N = \qabs x B \), where \( A \ered B \) and the inductive hypothesis holds for the latter.

    \Fullref{thm:eta_reduction_deconstruction/abs} lists the following possibilities for \( (\qabs x A) \pred K \):
    \begin{itemize}
      \item The possibility \fullref{thm:eta_reduction_deconstruction/abs/lift} implies that \( K \aequiv \qabs x C \), where \( A \ered C \). The inductive hypothesis on \( A \ered B \) implies the existence of a multi-step confluence point \( P \) of \( B \) and \( C \). Let \( L \coloneqq \qabs x P \)

      By induction on the number of steps in \( B \ered* P \), via \ref{def:lambda_reduction/abs} we obtain
      \begin{equation*}
        N = \qabs x B \ered* \qabs x P = L
      \end{equation*}
      and similarly
      \begin{equation*}
        K \aequiv \qabs x C \ered* \qabs x P = L,
      \end{equation*}
      which due to \ref{def:lambda_reduction/alpha} implies \( K \ered* L \).

      \item The possibility \fullref{thm:eta_reduction_deconstruction/abs/red} implies that \( A \aequiv Kx \) and that \( x \) is not free in \( K \).

      Then \( Kx \ered B \) and, by \ref{thm:eta_reduction_deconstruction/app}, there exists a term \( L \) such that \( B \aequiv Lx \) and \( K \ered L \). This is a single-step confluence point of \( K \) and \( N \) because
      \begin{equation*}
        N
        =
        \qabs x B
        \aequiv
        \qabs x L x
        \ered
        L.
      \end{equation*}
    \end{itemize}

    \item If \( M \ered N \) due to \ref{def:lambda_reduction/alpha}, then \( M \aequiv M' \ered N' \aequiv N \), where \( M' \ered N' \) and the inductive hypothesis holds for the latter.

    Then \ref{def:lambda_reduction/alpha} allows us to conclude that \( M' \ered K \), hence we can use the inductive hypothesis on \( M' \ered N' \) to obtain a multi-step confluence point \( L \) of \( N' \) and \( K \).

    Again via \ref{def:lambda_reduction/alpha}, we conclude that \( N \ered* L \). Therefore, \( L \) is the desired confluence point.

    \item If \( M \ered N \) due to \ref{def:beta_eta_reduction/eta}, then \( M = \qabs x N x \).

    \Fullref{thm:eta_reduction_deconstruction/abs} lists the following possibilities for \( (\qabs x N x) \pred K \):
    \begin{itemize}
      \item The possibility \fullref{thm:eta_reduction_deconstruction/abs/lift} implies that \( K \aequiv \qabs x C \), where \( Nx \ered C \). \Fullref{thm:eta_reduction_deconstruction/app} implies that \( C \ered Lx \).

      Then \( K \ered L \) and \( N \ered C \ered L \).

      \item The possibility \fullref{thm:eta_reduction_deconstruction/abs/red} implies that \( N x \ered K x \), and \ref{def:beta_eta_reduction/app_right} implies that \( N \ered K \). Then \( K \) is a multi-step confluence point of \( N \) and \( K \).
    \end{itemize}
  \end{itemize}

  \SubProof{Proof for multiple steps} If \( M \ered* N \) and \( M \ered* K \), \fullref{thm:confluence_of_multi_step_reduction} implies that we can use the one-step result to obtain a \( \synlambda \)-term \( L \) such that \( N \ered* L \) and \( K \ered* L \).
\end{proof}
