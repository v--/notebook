\section{Packing and tiling}\label{sec:packing_and_tiling}

\paragraph{Packing}

\begin{definition}\label{def:topological_space_packing}\mimprovised
  A \term{packing} in a \hyperref[def:topological_space]{topological space} is a family of closed sets with pairwise disjoint interiors.
\end{definition}
\begin{comments}
  \item We base our definition on the definition of tilings by \incite[611]{Arenas1999Tilings}, but without the requirement of covering, analogously to how \incite[16]{GrünbaumShephard1987TilingsAndPatterns} define a packing in the Euclidean plane as a tiling that is not a covering.
\end{comments}

\begin{definition}\label{def:lattice_packing}\mimprovised
  Fix a \hyperref[def:point_lattice]{point lattice} \( L \) and a closed set \( T \) in the \hyperref[def:euclidean_space]{Euclidean space} \( \BbbR^n \). If the family of \hyperref[def:rigid_motion/translation]{translates} of \( T \) by vectors in \( L \) is a \hyperref[def:topological_space_packing]{packing}, that is, if their interiors are pairwise disjoint, we call it a \term{lattice packing} of \( \BbbR^n \) with \( T \).
\end{definition}
\begin{comments}
  \item It is immaterial whether we use a point lattice or a \hyperref[def:euclidean_space_grid]{grid} in this definition because we can translate \( P \) itself. We prefer working with lattices to not deviate from the established terminology too much.

  \item Our definition of lattice packing is based on the following:
  \begin{itemize}
    \item \incite[223]{Cassels1997NumberGeometry} defines a \enquote{packing} in an Euclidean space as a disjoint family of translates. We want to keep the definition close to that of a tiling, so we require the sets to be closed but allow their boundaries to intersect. Cassels then defines a lattice packing as a packing based on a point lattice.

    \item \incite[1]{Rogers1964PackingAndCovering} also defines a \enquote{lattice packing} similarly, but requires the sets to be \hyperref[def:lebesgue_measure]{Lebesgue-measurable}.
  \end{itemize}
\end{comments}

\begin{remark}\label{rem:sphere_packing}
  A \hyperref[def:topological_space_packing]{packing} of a space with \hyperref[def:metric_space/ball]{balls} is sometimes called \enquote{sphere packing}. This term presupposes that spheres cannot be contained within other spheres, which is easily achieved without further assumptions by using balls instead. The discussion in \cite{MathOF:sphere_packing_vs_ball_packing} suggests that this is a historical artifact from before the modern understanding of topology.

  John Conway and Neil Sloane have dedicated an entire book on sphere packing, \cite{ConwaySloane1998SpherePackings}, while \incite[ch. 7]{Rogers1964PackingAndCovering} discusses \enquote{packings of spheres}. The term \enquote{ball packing} is used by \incite[327]{Gruber2007ConvexAndDiscreteGeometry}.

  Similarly, the term \enquote{circle packing} is sometimes used instead of \enquote{ball packing}.
\end{remark}

\begin{definition}\label{def:lattice_ball_packing}\mimprovised
  Fix a \hyperref[def:point_lattice]{point lattice} \( L \) in the \hyperref[def:euclidean_space]{Euclidean space} \( \BbbR^n \) with \hyperref[def:minimal_lattice_vector]{minimal norm} \( 2r \). By placing a \hyperref[def:metric_space/ball]{(closed) ball} of radius \( r \) at each point of \( L \), we obtain a \hyperref[def:lattice_packing]{lattice packing} in \( \BbbR^n \).

  We call it the \term{lattice ball packing} (or \term{lattice disk packing} if \( n = 2 \)) induced by \( L \).

  \begin{figure}[!ht]
    \begin{subcaptionblock}[t]{0.3\textwidth}
      \centering
      \includegraphics[page=1]{output/def__lattice_ball_packing}
      \caption{Packing via the \hyperref[def:integer_point_lattice]{integer point lattice}.}\label{fig:def:lattice_ball_packing/integer}
    \end{subcaptionblock}
    \hfill
    \begin{subcaptionblock}[t]{0.3\textwidth}
      \centering
      \includegraphics[page=2]{output/def__lattice_ball_packing}
      \caption{Packing via a lattice without an \hyperref[def:equinormed]{equinormed} basis.}\label{fig:def:lattice_ball_packing/non_uniform}
    \end{subcaptionblock}
    \hfill
    \begin{subcaptionblock}[t]{0.3\textwidth}
      \centering
      \includegraphics[page=3]{output/def__lattice_ball_packing}
      \caption{Packing via the \hyperref[def:hexagonal_point_lattice]{standard hexagonal point lattice}.}\label{fig:def:lattice_ball_packing/hexagonal}
    \end{subcaptionblock}
    \caption{Several \hyperref[def:lattice_packing]{lattice packings} of the unit disk.}\label{fig:def:lattice_ball_packing}
  \end{figure}
\end{definition}
\begin{defproof}
  The distance between any two points of the lattice is at least twice the radius of a ball, thus the interiors of the balls are disjoint.
\end{defproof}

\begin{remark}\label{rem:lattice_packing_density}
  As discussed in \fullref{thm:fundamental_parallelotope_measure}, the \hyperref[def:point_lattice_determinant]{determinant} of the \hyperref[def:point_lattice]{point lattice} \( L \) is the volume of a \hyperref[def:fundamental_parallelotope]{fundamental parallelotope}. In dimension \( n \), each fundamental parallelotope contains exactly one copy of \( T \), sliced into \( 2^n \) pieces. This leads to \fullref{def:lattice_packing_density}.

  \begin{figure}[!ht]
    \begin{subcaptionblock}{0.45\linewidth}
      \centering
      \includegraphics[page=1]{output/rem__lattice_packing_density}
    \end{subcaptionblock}
    \hfill
    \begin{subcaptionblock}{0.45\linewidth}
      \centering
      \includegraphics[page=2]{output/rem__lattice_packing_density}
    \end{subcaptionblock}
    \caption{In a \hyperref[def:lattice_packing]{lattice packing} of the plane with a set \( T \), every \hyperref[def:fundamental_parallelotope]{fundamental parallelotope} contains exactly one copy of \( T \) sliced into four parts.}\label{fig:rem:lattice_packing_density}
  \end{figure}
\end{remark}

\begin{definition}\label{def:lattice_packing_density}\mcite[7]{ConwaySloane1998SpherePackings}
  We define the \term{packing density} of a \hyperref[def:lattice_packing]{lattice packing} with a \hyperref[def:lebesgue_measure]{Lebesgue-measurable set} \( T \) and \hyperref[def:point_lattice]{point lattice} \( L \) as
  \begin{equation}\label{eq:def:lattice_packing_density}
    \delta(L, T) \coloneqq \frac {\lambda(T)} {\det(L)}.
  \end{equation}
\end{definition}
\begin{comments}
  \item The density of more general packings is discussed by \incite{Rogers1964PackingAndCovering}.
\end{comments}

\begin{proposition}\label{thm:disk_packing_density}
  In the \hyperref[def:euclidean_plane]{Euclidean plane}, fix a \hyperref[def:point_lattice]{point lattice} \( L \) with minimal norm \( 2r \), and fix a \hyperref[def:point_lattice_basis]{basis} \( \set{ u, v } \) of \( L \).

  Then the \hyperref[def:lattice_packing_density]{density} of the corresponding \hyperref[def:lattice_ball_packing]{lattice ball packing} is
  \begin{equation}\label{eq:def:disk_packing_density}
    \delta(L, B(0, r)) \coloneqq \frac {\pi r^2} {\norm{u} \cdot \norm{v} \cdot \abs{\sin\measuredangle(u, v)}}.
  \end{equation}
\end{proposition}
\begin{proof}
  A disk's area is given by \fullref{thm:area_of_circle} and the corresponding fundamental parallelogram's area is given by \fullref{thm:area_of_parallelogram}.
\end{proof}

\begin{proposition}\label{thm:minimal_disk_packing_density}
  The minimal \hyperref[def:lattice_packing_density]{density} of a \hyperref[def:lattice_ball_packing]{lattice ball packing} for lattices with \hyperref[def:minimal_lattice_vector]{minimal norm} \( 2r \) is
  \begin{equation}\label{eq:thm:minimal_disk_packing_density}
    \delta(L, B(0, r)) = \frac \pi {2 \sqrt 3} \approx 0.9069
  \end{equation}
  and is attained by \hyperref[def:hexagonal_point_lattice]{hexagonal lattices}.
\end{proposition}
\begin{comments}
  \item This result is demonstrated visually in \cref{fig:def:lattice_ball_packing}.
\end{comments}
\begin{proof}
  Fix a lattice \( L \) and a basis \( \set{ u, v } \) of \( L \).

  We want to maximize \eqref{eq:def:lattice_packing_density}. Since the numerator is fixed, we should minimize the denominator
  \begin{equation*}
    \det L = \norm{u} \cdot \norm{v} \cdot \abs{\sin\measuredangle(u, v)}.
  \end{equation*}

  By \fullref{thm:point_lattice_determinant}, the determinant does not depend on the basis, so we may choose a basis where \( u \) is a minimal vector and \( v \) has minimal norm among all vectors linearly independent from \( u \).

  Then, since \( u - v \) is also linearly independent from \( u \), we have
  \begin{equation*}
    \norm{u - v} \geq \norm{v} \geq \norm{u} = 2r.
  \end{equation*}

  \Fullref{thm:law_of_cosines_for_vectors} implies that
  \begin{equation*}
    \norm{u - v}^2 = \norm{u}^2 + \norm{v}^2 - 2 \cdot \norm{u} \cdot \norm{v} \cdot \cos \measuredangle(u, v).
  \end{equation*}

  The above is bounded from below by \( \norm{v}^2 \). We can thus cancel \( \norm{v}^2 \) to obtain the inequality
  \begin{equation*}
    0 \leq \norm{u}^2 - 2 \cdot \norm{u} \cdot \norm{v} \cdot \cos \measuredangle(u, v).
  \end{equation*}

  By cancelling \( \norm{u} \), we obtain
  \begin{equation*}
    2 \cdot \norm{v} \cdot \cos \measuredangle(u, v) \leq \norm{u},
  \end{equation*}
  hence
  \begin{equation*}
    \cos \measuredangle(u, v) \leq \frac {\norm{u}} {2\norm{v}}.
  \end{equation*}

  Then
  \begin{equation*}
    \abs{\sin \measuredangle(u, v)}
    =
    \sqrt{ 1 - (\cos \measuredangle(u, v))^2 }
    \geq
    \frac 1 {2\norm{v}} \cdot \sqrt{ 4\norm{v}^2 - \norm{u}^2 }.
  \end{equation*}

  We have thus obtained the lower bound
  \begin{equation*}
    \det L = \norm{u} \cdot \norm{v} \cdot \abs{\sin\measuredangle(u, v)} \geq \frac {\norm {u}} 2 \cdot \sqrt{ 4\norm{v}^2- \norm{u}^2 }.
  \end{equation*}

  Since the lower bound is an increasing function of \( \norm{v} \), we conclude that the minimal bound is achieved when \( \norm{u} = \norm{v} = 2r \), in which case
  \begin{equation*}
    \det L \geq \norm{u}^2 \cdot \frac {\sqrt 3} 2 = 2\sqrt 3 \cdot r^2.
  \end{equation*}

  This bound is attained when \( \measuredangle(u, v) = \pi / 3 \).

  In \fullref{def:hexagonal_point_lattice}, we have defined hexagonal lattices as those having a basis of minimal vectors and an angle \( \pi / 3 \) between them. This completes the proof.
\end{proof}

\paragraph{Tiling}

\begin{definition}\label{def:topological_space_tiling}\mcite[611]{Arenas1999Tilings}
  A \term[ru=паркет / мозайка (\cite[465]{Маркушевич1967АналитическиеФункцииТом2})]{tiling} of a \hyperref[def:topological_space]{topological space} is a \hyperref[def:topological_space_packing]{packing} that \hyperref[def:set_cover]{covers} the entire space.

  We refer to sets in the tiling as \term[ru=плитка (\cite[465]{Маркушевич1967АналитическиеФункцииТом2})]{tiles}.
\end{definition}
\begin{comments}
  \item We use the definition of tiling by \incite[611]{Arenas1999Tilings}, but split it in two parts -- we define packing in \fullref{def:topological_space_packing} and them, based on the more concrete case in \incite[16]{GrünbaumShephard1987TilingsAndPatterns}, we define a tiling as a packing that covers the entire space.

  \item We discuss some more other definitions in \fullref{rem:tiling_definition}.
\end{comments}

\begin{remark}\label{rem:tiling_definition}
  Tilings are abstractions of familiar concepts from architecture and art. As such, the study of tilings is primarily restricted to the \hyperref[def:euclidean_plane]{Euclidean plane}.

  There are dozens of kinds of tilings floating around in the literature. We give a very general definition in \fullref{def:topological_space_tiling}, and briefly mention how several authors treat tilings.

  \begin{thmenum}
    \thmitem{rem:tiling_definition/topological} \incite[611]{Arenas1999Tilings} defines tilings in general topological spaces. We use his definition. At this level of generality meaningful regularity conditions are difficult to introduce.

    \thmitem{rem:tiling_definition/grunbaum} \incite[16]{GrünbaumShephard1987TilingsAndPatterns} restrict themselves to the Euclidean plane, and then proceed to classify tilings based on how they are constructed.

    \thmitem{rem:tiling_definition/berger} \incite[596]{Berger2010GeometryRevealed} restricts himself to the Euclidean plane and requires all tiles to be compact and to have a nonempty interior. In his earlier book, in \cite[\S 1.7.3]{Berger1987GeometryI}, he initially restricts himself to tilings generated from a single connected tile by an \hyperref[def:isometry_group]{isometry group}. These restrictions allow him to prove classification theorems.
  \end{thmenum}
\end{remark}

\begin{definition}\label{def:prototile}\mimprovised
  Fix a \hyperref[def:topological_space_tiling]{tiling} \( \mscrT \) of the \hyperref[def:euclidean_space]{Euclidean space} \( \BbbR^n \).

  We say that the subfamily \( \mscrP \) of \( \mscrT \) is a system of \term[en=prototile (\cite[20]{GrünbaumShephard1987TilingsAndPatterns})]{prototiles} if every tile in \( \mscrT \) is \hyperref[def:isometry]{isometric} to \hi{exactly one} tile in \( \mscrP \).
\end{definition}
\begin{comments}
  \item We extend the definition by \incite[20]{GrünbaumShephard1987TilingsAndPatterns} that holds for only one tile. The authors rely on an implicit generalization when defining \( n \)-hedral tilings several pages later.
\end{comments}

\begin{definition}\label{def:k_hedral_tiling}\mcite[23]{GrünbaumShephard1987TilingsAndPatterns}
  For a positive integer \( k \), we say that a \hyperref[def:topological_space_tiling]{tiling} of the \hyperref[def:euclidean_space]{Euclidean space} \( \BbbR^n \) is \( k \)-hedral if there exists a family of \( k \) \hyperref[def:prototile]{prototiles}.

  If \( k = 1 \), we call the tiling \term{monohedral}.
\end{definition}
\begin{defproof}
  There may be infinitely many families of prototiles, but every the families are pairwise \hyperref[def:equinumerosity]{equinumerous}.
\end{defproof}

\begin{definition}\label{def:parallelogram_tiling}\mcite[exerc. 1.1.10]{GrünbaumShephard1987TilingsAndPatterns}
  Fix a \hyperref[def:parallelogram]{parallelogram} \( ABCD \) and consider the \hyperref[def:point_lattice]{point lattice} \( L \) with \hyperref[def:point_lattice_basis]{basis vectors} \( \vect{AB} \) and \( \vect{AC} \), so that \( ABCD \) is (congruent to) a \hyperref[def:fundamental_parallelotope]{fundamental parallelepiped} of \( L \).

  \begin{figure}[!ht]
    \begin{subcaptionblock}{0.45\linewidth}
      \centering
      \includegraphics[page=1]{output/def__parallelogram_tiling}
      \caption{Tiling with the \hyperref[def:unit_hypercube]{unit square}, a \hyperref[def:fundamental_parallelotope]{fundamental parallelogram} of the \hyperref[def:integer_point_lattice]{integer point lattice}.}\label{fig:def:parallelogram_tiling/square}
    \end{subcaptionblock}
    \hfill
    \begin{subcaptionblock}{0.45\linewidth}
      \centering
      \includegraphics[page=2]{output/def__parallelogram_tiling}
      \caption{Tiling with a \hyperref[def:parallelogram/rhombus]{rhombus} with angle \( \pi / 3 \), a fundamental parallelogram of a \hyperref[def:hexagonal_point_lattice]{hexagonal point lattice}.}\label{fig:def:parallelogram_tiling/rhombus}
    \end{subcaptionblock}
    \caption{Two \hyperref[def:parallelogram_tiling]{parallelogram tilings} of the plane.}\label{fig:def:parallelogram_tiling}
  \end{figure}

  Then the family of translations of \( ABCD \) by vectors in \( L \) is a \hyperref[def:k_hedral_tiling]{monohedral} \hyperref[def:topological_space_tiling]{tiling} of \( \BbbR^2 \). We call it the \term{parallelogram tiling} by \( ABCD \).
\end{definition}
\begin{comments}
  \item The parallelogram \( ABCD \) is itself a tile, however it is only a fundamental parallelepiped of \( L \) if \( A \) is the zero vector.
\end{comments}
\begin{defproof}
  We will work in the \hyperref[def:affine_coordinate_system]{affine coordinate system} with origin \( A \) and basis vectors \( \vect{AB} \) and \( \vect{AC} \). With respect to this system we have the following:
  \begin{itemize}
    \item The parallelogram is simply the unit square
    \begin{equation*}
      \set{ (x, y) \given 0 \leq x \leq 1 \T{and} 0 \leq y \leq 1 }.
    \end{equation*}

    \item \( L \) reduces to the \hyperref[def:integer_point_lattice]{integer point lattice} \( \BbbZ^2 \).
  \end{itemize}

  Fix a point \( (x, y) \) in \( \BbbR^2 \). Then the fractional parts of \( x \) and \( y \) uniquely determine coordinates inside the parallelogram, while the integral parts determine the corresponding translation. Therefore, the family of translations is indeed a tiling.
\end{defproof}

\begin{lemma}\label{thm:triangle_rotation_around_midpoint}
  Fix a \hyperref[def:triangle]{triangle} \( ABC \) and let \( f \) be the \hyperref[def:rigid_motion/rotation]{rotation} by \( \pi \) radians around the \hyperref[thm:segment_midpoint]{midpoint} of \( AC \).

  \begin{figure}[!ht]
    \centering
    \includegraphics[page=1]{output/thm__triangle_rotation_around_midpoint}
    \caption{An illustration of \fullref{thm:triangle_rotation_around_midpoint}.}\label{fig:thm:triangle_rotation_around_midpoint}
  \end{figure}

  Then \( f \) swaps \( A \) and \( C \) and sends \( B \) to \( D \coloneqq \tau_{\vect{BC}}(A) \).
\end{lemma}
\begin{comments}
  \item In particular, \( ABC \) and \( CAD \) are \hyperref[def:congruent_shapes]{congruent}.
  \item We can thus form the \hyperref[def:parallelogram]{parallelogram} \( ABCD \).
\end{comments}
\begin{proof}
  Consider the standard coordinate system \( Oxy \).

  The midpoint \( M \) of \( AB \) satisfies \( \vect{OM} \coloneqq (\vect{OA} + \vect{OC}) / 2 \). By \fullref{thm:plane_rotation_matrix_angle}, rotation by \( \pi \) about the origin maps \( \vect{OP} \) to \( -\vect{OP} \). Then
  \begin{equation*}
    f(\vect{OP}) = \vect{OM} - (\vect{OP} - \vect{OM}) = 2\vect{OM} - \vect{OP} = \vect{OA} + \vect{OC} - \vect{OP}.
  \end{equation*}
  which automatically implies \( f(\vect{OA}) = \vect{OC} \) and \( f(\vect{OC}) = \vect{OA} \). Furthermore,
  \begin{equation*}
    f(\vect{OB}) = \vect{OA} + \vect{OC} - \vect{OB} = \vect{OA} + \vect{BC} = \vect{OD}.
  \end{equation*}
\end{proof}

\begin{definition}\label{def:triangular_tiling}\mimprovised
  Fix a \hyperref[def:triangle]{triangle} \( ABC \) and consider the \hyperref[def:point_lattice]{point lattice} \( L \) with \hyperref[def:point_lattice_basis]{basis vectors} \( \vect{AB} \) and \( \vect{AC} \).

  \begin{figure}[!ht]
    \begin{subcaptionblock}[t]{0.45\linewidth}
      \centering
      \includegraphics[page=1]{output/def__triangular_tiling}
      \caption{Tiling with the triangle formed by the standard basis vectors, which generates the \hyperref[def:integer_point_lattice]{integer point lattice}.}\label{fig:def:triangular_tiling/right}
    \end{subcaptionblock}
    \hfill
    \begin{subcaptionblock}[t]{0.45\linewidth}
      \centering
      \includegraphics[page=2]{output/def__triangular_tiling}
      \caption{Tiling with an \hyperref[def:triangle/equilateral]{equilateral triangle}, which generates a \hyperref[def:hexagonal_point_lattice]{hexagonal point lattice}.}\label{fig:def:triangular_tiling/equilateral}
    \end{subcaptionblock}
    \caption{Two \hyperref[def:triangular_tiling]{triangular tilings} of the plane.}\label{fig:def:triangular_tiling}
  \end{figure}

  \Fullref{thm:triangle_rotation_around_midpoint} implies that the triangle \( CAD \), where \( D \coloneqq \tau_{\vect{BC}}(A) \), is a \hyperref[def:rigid_motion/rotation]{rotation} of \( ABC \). Then the family of translations of \( ABC \) and \( CAD \) by vectors in \( L \) is a \hyperref[def:k_hedral_tiling]{monohedral} \hyperref[def:topological_space_tiling]{tiling} of \( \BbbR^2 \). We call it the \term{triangular tiling} by \( ABC \).
\end{definition}
\begin{comments}
  \item We chose our rotation so that \( D \) is antipodal to \( B \). We would obtain the same tiling if we instead rotated \( ABC \) along the midpoint of \( AB \) or \( BC \).

  \item The triangle \( ABC \) is itself a tile, however it is only a fundamental simplex of \( L \) if \( A \) is the zero vector.
\end{comments}
\begin{defproof}
  The triangular tiling by \( ABC \) is a refinement of the \hyperref[def:parallelogram_tiling]{parallelogram tiling} by \( ABCD \), hence it is indeed a tiling.
\end{defproof}

\paragraph{Voronoi cells}

\begin{definition}\label{def:voronoi_cell}
  We define the \term{Voronoi cell} of a point \( x \) with respect to a \hyperref[def:discrete_set]{discrete} set of points \( \mscrP \) in the \hyperref[def:euclidean_space]{Euclidean space} \( \BbbR^n \) as
  \begin{equation}\label{eq:def:voronoi_cell}
    V(x, \mscrP) \coloneqq \bigcap_{p \in \mscrP} \set{ q \in \BbbR^n \given \norm{ q - x } \leq \norm{ y - p } }.
  \end{equation}

  If \( \mscrP \) is empty, \( V(x, \mscrP) \) coincides with \( \BbbR^n \).
\end{definition}
\begin{comments}
  \item Our definitions extend the following:
  \begin{itemize}
    \item \incite[74]{Rogers1964PackingAndCovering} additionally requires \( \mscrP \) to be countable, and additionally assume the existence of a radius \( r \) such that the balls \( B(p, r) \) with \( p \in \mscrP \) cover \( \BbbR^n \).

    \item \incite[exerc. 2.9]{BoydVandenberghe2004ConvexOptimization} additionally require \( \mscrP \) to be a finite set.
  \end{itemize}
\end{comments}

\begin{proposition}\label{thm:voronoi_cell_union}
  The \hyperref[def:voronoi_cell]{Voronoi cell} of a point with respect to the discrete set \( \mscrP \cup \mscrQ \) is the intersection of cells for \( \mscrP \) and \( \mscrQ \).
\end{proposition}
\begin{proof}
  Obvious from \eqref{eq:def:voronoi_cell}.
\end{proof}

\begin{corollary}\label{thm:voronoi_cell_self}
  The \hyperref[def:voronoi_cell]{Voronoi cell} of \( x \) with respect to \( \mscrP \cup \set{ x } \) coincides with its cell with respect to \( \mscrP \).
\end{corollary}
\begin{proof}
  Follows from \fullref{thm:voronoi_cell_union} by noting that
  \begin{equation*}
    V(x, \set{ x }) = \set{ q \in \BbbR^n \given \norm{ q - x } \leq \norm{ q - x } } = \BbbR^n.
  \end{equation*}
\end{proof}

\begin{proposition}\label{thm:equidistant_point_line}
  Fix two points \( A \) and \( B \) in the \( n \)-dimensional \hyperref[def:euclidean_space]{Euclidean space}. The set of points equidistant from \( A \) and from \( B \) is a hyperplane passing through the midpoint \( M \) of the segment \( AB \) and orthogonal to \( AB \).

  \begin{figure}[!ht]
    \begin{subcaptionblock}[t]{0.45\linewidth}
      \centering
      \includegraphics[page=1]{output/thm__equidistant_point_line}
    \end{subcaptionblock}
    \hfill
    \begin{subcaptionblock}[t]{0.45\linewidth}
      \centering
      \includegraphics[page=2]{output/thm__equidistant_point_line}
    \end{subcaptionblock}
    \caption{A \hyperref[def:affine_hyperplane]{line} separating the points closer to \( A \) from those closer to \( B \).}\label{fig:thm:equidistant_point_line}
  \end{figure}
\end{proposition}
\begin{proof}
  The midpoint \( M \) is, by definition, equidistant from \( A \) and from \( B \).

  Let \( P \) be any other equidistant point. The triangles \( AMP \) and \( BMP \) have three equal sides, and \fullref{thm:triangle_congruence/three_sides} implies that they are congruent.

  Then \( MP \) is a median of the isosceles triangle \( ABP \), and \fullref{thm:isosceles_median_and_altitude} implies that \( PM \) is a height. Therefore, the angle \( PMB \) is right.

  For any other point \( P' \) equidistant from \( A \) and from \( B \), the angle \( P'MB \) is also right. Then the lines defined by the segments \( MP \) and \( MP' \) are parallel and they have a common point, thus they coincide.

  We conclude that the line defined by \( MP \) contains all points equidistance from \( A \) and from \( B \), and that it is orthogonal to \( AB \).

  Conversely, fix a point \( P \) on that line distinct from \( M \). The triangles \( AMP \) and \( BMP \) have two equal sides --- \( AM \) and \( BM \), as well as their common side \( MP \). \Fullref{thm:triangle_congruence/two_sides_and_angle} implies that the triangles are congruent, hence their remaining sides \( AP \) and \( BP \) are equal.
\end{proof}

\begin{proposition}\label{thm:voronoi_cell_polygon}
  The \hyperref[def:voronoi_cell]{Voronoi cell} of \( x \) with respect to \( \mscrP \) is a \hyperref[def:polytope]{polytope} defined by the following intersection of \hyperref[def:half_space]{half-planes}:
  \begin{equation}\label{eq:thm:voronoi_cell_polygon}
    V(x, \mscrP) = \bigcap_{p \in \mscrP} \set[\Big]{ q \in \BbbR^n \given* \inprod { p - x } q \leq \frac {\norm{p}^2 - \norm{x}^2} 2 }.
  \end{equation}

  When regarding \( \BbbR^n \) as the affine space \( \BbbA^n \), with respect to the affine coordinate system \( Ae_1,\ldots,e_n \) this becomes
  \begin{equation}\label{eq:thm:voronoi_cell_polygon/affine}
    V(A, \mscrP) = \bigcap_{P \in \mscrP} \set[\Big]{ Q \in \BbbA^n \given* \inprod {\vect{AP}} {\vect{AQ}} \leq \frac {\norm{\vect{AP}}^2} 2 }.
  \end{equation}
\end{proposition}
\begin{proof}
  For each individual \( p \in \mscrP \), we can square
  \begin{equation*}
    \norm{ q - x } \leq \norm{ q - p }
  \end{equation*}
  to obtain
  \begin{equation*}
    \inprod { q - x } { q - x } \leq \inprod { q - p } { q - p },
  \end{equation*}
  which we can expand to
  \begin{equation*}
    \cancel{\norm{q}^2} - 2 \inprod q x + \norm{x}^2 \leq \cancel{\norm{q}^2} - 2 \inprod q p + \norm{p}^2.
  \end{equation*}

  Then
  \begin{equation*}
    2 \inprod q { p - x } \leq \norm{p}^2 - \norm{x}^2.
  \end{equation*}

  This concludes the proof.
\end{proof}

\begin{definition}\label{def:voronoi_tiling}\mimprovised
  Every \hyperref[def:discrete_set]{discrete} set of points \( \mscrP \) in the \hyperref[def:euclidean_space]{Euclidean space} \( \BbbR^n \) induces a \hyperref[def:topological_space_tiling]{tiling} by \hyperref[def:voronoi_cell]{Voronoi cells}.

  We call it the \term{Voronoi tiling} of \( \BbbR^n \) induced by \( \mscrP \).
\end{definition}

\begin{proposition}\label{thm:hexagonal_point_lattice_voronoi_cell}
  Fix a \hyperref[def:hexagonal_point_lattice]{hexagonal point lattice} \( L \) with \hyperref[def:minimal_lattice_vector]{minimal norm} \( 2r \) and a point \( A \) in \( L \).

  \Fullref{thm:hexagonal_point_lattice_minimal_vectors} implies that, after translating \( A \) by the minimal vectors, we obtain a regular hexagon whose vertices are at minimal distance from \( A \). Let \( V_1, \ldots, V_6 \) be the vertices of this hexagon.

  For each \( V_k \), let \( M_k \) be the \hyperref[thm:segment_midpoint]{midpoint} of the segment \( AV_k \) and let \( C_k \) be the \hyperref[thm:medicenter]{medicenter} of the triangle \( A V_k V_{k+1} \)\fnote{The index arithmetic here is performed modulo \( 6 \).}.

  Then the convex hull of the points \( C_1, \ldots, C_6 \) is a \hyperref[def:regular_polygon]{regular hexagon} with radius \( r / {\sqrt 3} \), which coincides with the \hyperref[def:voronoi_cell]{Voronoi cell} of \( P \).

  \begin{figure}[!ht]
    \begin{subcaptionblock}[t]{0.45\linewidth}
      \centering
      \includegraphics[page=1]{output/thm__hexagonal_point_lattice_voronoi_cell}
      \caption{The intersection of two \hyperref[def:half_space]{half-planes}.}\label{fig:thm:hexagonal_point_lattice_voronoi_cell/construction}
    \end{subcaptionblock}
    \hfill
    \begin{subcaptionblock}[t]{0.45\linewidth}
      \centering
      \includegraphics[page=2]{output/thm__hexagonal_point_lattice_voronoi_cell}
      \caption{The intersection of all half-planes.}\label{fig:thm:hexagonal_point_lattice_voronoi_cell/result}
    \end{subcaptionblock}
    \caption{Structure of the \hyperref[def:voronoi_cell]{Voronoi cell} of \( A \) in \fullref{thm:hexagonal_point_lattice_voronoi_cell}.}\label{fig:thm:hexagonal_point_lattice_voronoi_cell}
  \end{figure}
\end{proposition}
\begin{proof}
  \Fullref{thm:voronoi_cell_polygon} implies that the Voronoi cell of \( A \) is
  \begin{equation}\label{eq:thm:hexagonal_point_lattice_voronoi_cell/proof/cell}
    V(A, L) = \bigcap_{P \in L} \set[\Big]{ Q \in \BbbA^n \given \inprod {\vect{AP}} {\vect{AQ}} \leq \frac {\norm{\vect{AP}}^2} 2 }.
  \end{equation}

  \SubProof{Proof that \( V_1, \ldots, V_6 \) determine the Voronoi cell} We will first show that
  \begin{equation*}
    V(A, \set{ V_1, \ldots, V_6 }) = V(A, L).
  \end{equation*}

  We only need to prove
  \begin{equation*}
    V(A, \set{ V_1, \ldots, V_6 }) \subseteq V(A, L)
  \end{equation*}
  since the inverse inclusion follows from \fullref{thm:voronoi_cell_union}.

  Fix a point \( P \) from \( L \) and \( Q \) from \( V(A, \set{ V_1, \ldots, V_6 }) \) and suppose that both are distinct from \( A \) and from \( V_1, \ldots, V_6 \).

  Since \( Q \) belongs to \( V(A, \set{ V_1, \ldots, V_6 }) \), from \eqref{eq:thm:hexagonal_point_lattice_voronoi_cell/proof/cell} it follows that, for any vertex \( V_k \),
  \begin{equation}\label{eq:thm:hexagonal_point_lattice_voronoi_cell/proof/vk_q_half_space}
    \inprod {\vect{AV_k}} {\vect{AQ}} \leq \frac {\norm{\vect{AV_k}}^2} 2.
  \end{equation}

  \Fullref{thm:cosine_of_angle_measure} implies that
  \begin{equation}\label{eq:thm:hexagonal_point_lattice_voronoi_cell/proof/avk_aq_prod}
    \inprod {\vect{AV_k}} {\vect{AQ}}
    =
    \norm{\vect{AV_k}} \cdot \norm{\vect{AQ}} \cdot \cos\measuredangle(\vect{AV_k}, \vect{AQ}).
  \end{equation}

  Plugging \eqref{eq:thm:hexagonal_point_lattice_voronoi_cell/proof/avk_aq_prod} into \eqref{eq:thm:hexagonal_point_lattice_voronoi_cell/proof/vk_q_half_space}, we obtain
  \begin{equation*}
    \norm{\vect{AV_k}} \cdot \norm{\vect{AQ}} \cdot \cos\measuredangle(\vect{AV_k}, \vect{AQ}) \leq \frac {\norm{\vect{AV_k}}^2} 2,
  \end{equation*}
  where we can divide by \( \norm{\vect{AV_k}} \) to conclude that
  \begin{equation}\label{eq:thm:hexagonal_point_lattice_voronoi_cell/proof/aq_cos_avk_ineq}
    \norm{\vect{AQ}} \cdot \cos\measuredangle(\vect{AV_k}, \vect{AQ}) \leq \frac {\norm{\vect{AV_k}}} 2.
  \end{equation}

  Fix a vertex \( V_{k_0} \) such that
  \begin{equation*}
    \measuredangle(\vect{AV_{k_0}}, \vect{AQ}) \leq \frac \pi 6 \T{or} \measuredangle(\vect{AV_{k_0}}, \vect{AQ}) \geq \frac {5\pi} 6
  \end{equation*}

  Such a choice is possible because the vertices \( V_1, \ldots, V_6 \) are rotations by \( \pi / 3 \) of each other about \( A \).

  Then, since the cosine decreases from \( 0 \) to \( \pi \) and increases from \( \pi \) to \( 2\pi \), we have
  \begin{equation*}
    \cos \measuredangle(\vect{AV_{k_0}}, \vect{AQ}) \geq \frac {\sqrt 3} 2,
  \end{equation*}
  hence we can extend the inequality \eqref{eq:thm:hexagonal_point_lattice_voronoi_cell/proof/aq_cos_avk_ineq} to
  \begin{equation*}
    \frac {\sqrt 3 \cdot \norm{\vect{AQ}}} 2 \leq \norm{\vect{AQ}} \cdot \cos\measuredangle(\vect{AV_k}, \vect{AQ}) \leq \frac {\norm{\vect{AV_k}}} 2
  \end{equation*}
  and conclude that
  \begin{equation}\label{eq:thm:hexagonal_point_lattice_voronoi_cell/proof/aq_norm_via_avk0}
    \norm{\vect{AQ}} \leq \frac {\norm{\vect{AV_{k_0}}}} {\sqrt 3} = \frac {2r} {\sqrt 3}.
  \end{equation}

  By assumption, \( \vect{AP} \) is not zero nor a minimal vector, and \fullref{thm:hexagonal_point_lattice_shortest_non_minimal_vectors} implies that
  \begin{equation}\label{eq:thm:hexagonal_point_lattice_voronoi_cell/proof/ap_norm}
    \norm{\vect{AP}} \geq 2 \sqrt 3 \cdot r.
  \end{equation}

  We can thus combine \eqref{eq:thm:hexagonal_point_lattice_voronoi_cell/proof/aq_norm_via_avk0} and \eqref{eq:thm:hexagonal_point_lattice_voronoi_cell/proof/ap_norm} to obtain
  \begin{equation}\label{eq:thm:hexagonal_point_lattice_voronoi_cell/proof/aq_norm_via_ap}
    \norm{\vect{AQ}} \leq \frac {2r} {\sqrt 3} \leq \frac {\vect{AP}} 3.
  \end{equation}

  \Fullref{thm:cosine_of_angle_measure} implies that
  \begin{equation*}
    \inprod {\vect{AP}} {\vect{AQ}}
    =
    \norm{\vect{AP}} \cdot \norm{\vect{AQ}} \cdot \cos\measuredangle(\vect{AP}, \vect{AQ}),
  \end{equation*}
  where we can use \eqref{eq:thm:hexagonal_point_lattice_voronoi_cell/proof/aq_norm_via_ap} to conclude
  \begin{equation*}
    \inprod {\vect{AP}} {\vect{AQ}}
    \leq
    \norm{\vect{AP}} \cdot \frac {\norm{\vect{AP}}} 3 \cdot \cos\measuredangle(\vect{AP}, \vect{AQ}).
  \end{equation*}

  The maximal value of \( \cos \) is \( 1 \), thus
  \begin{equation*}
    \inprod {\vect{AP}} {\vect{AQ}} \leq \frac {\norm{\vect{AP}}^2} 3 \leq \frac {\norm{\vect{AP}}^2} 2.
  \end{equation*}

  This is the desired inequality that must hold for \( Q \) to belongs to the half-plane generated by \( P \) in the Voronoi cell. Since \( P \) was arbitrary, we conclude that \( Q \) belongs to \( V(A, L) \).

  \SubProof{Proof that the cell is the convex hull of \( C_1, \ldots, C_6 \)} \Fullref{thm:equidistant_point_line} implies that the set of points equidistant from \( A \) and from \( V_k \) coincides with the line through \( M_k \) orthogonal to \( A V_k \). Refer to \cref{fig:thm:hexagonal_point_lattice_voronoi_cell/construction} for an illustration.

  The midpoint \( M_k \) of \( A V_k \) lies in the segment \( V_{k-1} V_{k+1} \), and is actually its midpoint also. Indeed, the triangles \( A V_k V_{k-1} \) and \( A V_k V_{k+1} \) are both equilateral and have a common side, hence they are congruent.

  \Fullref{thm:isosceles_median_and_altitude} implies that the median \( V_{k-1} M_k \) is also a height, thus \( AV_k \) and \( V_{k-1} V_{k+1} \) are perpendicular. \Fullref{thm:equidistant_point_line} then implies that the line of equidistant points from \( A \) and from \( V_k \) is the line through \( V_{k-1} \) and \( V_{k+1} \).

  Furthermore, as the medicenter of \( A V_k V_{k+1} \), \( C_k \) it is the intersection of the medians \( V_{k+1} M_k \) and \( V_k M_{k+1} \), therefore it is the intersection of the equidistant lines for \( A \) and \( V_k \) and for \( A \) and \( V_{k+1} \).

  Therefore, \( C_1, \ldots, C_6 \) are intersections of equidistant lines for \( A \) and for the vertices \( V_1, \ldots, V_6 \). Picking the half-plane containing \( A \) for each of the equidistant lines and taking their intersection, we obtain the Voronoi cell of \( A \).

  The Voronoi cell is compact and convex, and is thus a convex hull of its extremal points, which we have determined to be \( C_1, \ldots, C_6 \).

  \SubProof{Proof that \( \vect{AC_k} \) has norm \( r / \sqrt 3 \)} Finally,
  \begin{equation*}
    \norm{\vect{AC_k}}
    =
    \frac 1 3 \cdot \norm{\vect{AV_k} + \vect{AV_{k+1}}}.
  \end{equation*}

  We have concluded in \fullref{thm:hexagonal_point_lattice_shortest_non_minimal_vectors} that the norm of \( \vect{AV_k} + \vect{AV_{k+1}} \) is \( \sqrt 3 r \), hence
  \begin{equation*}
    \norm{\vect{AC_k}} = \frac r {\sqrt 3}.
  \end{equation*}
\end{proof}

\begin{definition}\label{def:hexagonal_tiling}\mimprovised
  The \hyperref[def:voronoi_tiling]{Voronoi tiling} of a \hyperref[def:hexagonal_point_lattice]{hexagonal point lattice} \( L \) consists, as shown in \fullref{thm:hexagonal_point_lattice_voronoi_cell}, of hexagons. We thus call it the \term{hexagonal tiling} of \( L \).

  \begin{figure}[!ht]
    \begin{subcaptionblock}[t]{0.45\linewidth}
      \centering
      \includegraphics[page=1]{output/def__hexagonal_tiling}
      \caption{A \hyperref[def:hexagonal_tiling]{hexagonal tiling} of the plane.}\label{fig:def:hexagonal_tiling/tiling}
    \end{subcaptionblock}
    \hfill
    \begin{subcaptionblock}[t]{0.45\linewidth}
      \centering
      \includegraphics[page=2]{output/def__hexagonal_tiling}
      \caption{The corresponding \hyperref[def:lattice_ball_packing]{disk packing}.}\label{fig:def:hexagonal_tiling/packing}
    \end{subcaptionblock}
  \end{figure}
\end{definition}
\begin{comments}
  \item Unlike for \hyperref[def:parallelogram_tiling]{parallelogram} and \hyperref[def:triangular_tiling]{triangular} tilings, the hexagons are uniquely defined by the lattice structure.
\end{comments}
