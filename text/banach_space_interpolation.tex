\section{Banach space interpolation}\label{sec:banach_space_interpolation}

\begin{definition}\label{def:interpolated_topological_vector_space}\mcite[24]{BerghLöfstrom1976Interpolation}
  Let \( \BbbK \) be either the \hyperref[def:real_numbers]{field \( \BbbR \) of real numbers} or the \hyperref[def:real_numbers]{field \( \BbbC \) of complex numbers}.

  \begin{thmenum}
    \thmitem{def:interpolated_topological_vector_space/compatibility} We say that two \hyperref[def:topological_vector_space]{topological vector spaces} \( X_0 \) and \( X_1 \) are \term{compatible} if they can both be \hyperref[def:morphism_invertibility/left_cancellative]{embedded} \hyperref[def:global_continuity]{continuously} into a \hyperref[def:separation_axioms/T2]{Hausdorff} topological vector space \( \mscrU \), in which case we can regard them as subspaces of \( \mscrU \).

    In particular, both \( X_0 \) and \( X_1 \) are Hausdorff. We write \( \oline{X} \coloneqq (X_0, X_1) \).

    \thmitem{def:interpolated_topological_vector_space/intersection} Denote by
    \begin{equation*}
      \Delta \oline{X} \coloneqq X_0 \cap X_1
    \end{equation*}
    the \term{intersection} of \( X_0 \) and \( X_1 \) (when regarded as subspaces of \( \mscrU \)).

    \thmitem{def:interpolated_topological_vector_space/sum} Denote by
    \begin{equation*}
      \Sigma \oline{X} \coloneqq ( X_0 + X_1 )
    \end{equation*}
    the \term{sum} of \( X_0 \) and \( X_1 \). If \( x \in \Sigma \oline{X} \), then there exist (possibly nonunique) vectors \( x_0 \in X_0 \) and \( x_1 \in X_1 \) such that \( x = x_0 + x_1 \).

    \thmitem{def:interpolated_topological_vector_space/intermediate_space} Let \( \oline{X} \) be a pair of compatible spaces. We say that the space \( X \) is an \term{intermediate} space for \( \oline{X} \) if \( \Delta \oline{X} \subseteq X \subseteq \Sigma \oline{X} \) with continuous linear inclusions.

    \thmitem{def:interpolated_topological_vector_space/morphisms} We introduce \hyperref[def:category/morphisms]{morphisms} between two compatible pairs \( \oline{X} \) and \( \oline{Y} \) that are, strictly speaking, not \hyperref[def:function]{functions} between the pairs themselves. We define an \term{operator} \( T: \oline{X} \to \oline{Y} \) between compatible pairs to be a function \( T \) from \( \Sigma \oline{X} \) to \( \Sigma \oline{Y} \) that satisfies the additional conditions
    \begin{align*}
      T(X_0) \subseteq Y_0
      &&
      T(X_1) \subseteq Y_1.
    \end{align*}

    \thmitem{def:interpolated_topological_vector_space/category} If \( \cat{C} \) is a \hyperref[def:subcategory]{subcategory} of the category \hyperref[def:category_of_topological_vector_spaces]{\( \cat{TopVect}_{\BbbK} \)} of topological vector spaces. We define the category \( \cat{Interp}_{\cat{C}} \) as the product category \( \cat{TopVect}_{\BbbK} \times \cat{TopVect}_{\BbbK} \). More explicitly:
    \begin{refenum}
      \refitem{def:category/objects} The \hyperref[def:set]{class} of objects is the class of all pairs of \hyperref[def:interpolated_topological_vector_space/compatibility]{compatible spaces}.
      \refitem{def:category/morphisms} The morphisms between two compatible pairs are the \hyperref[def:interpolated_topological_vector_space/morphisms]{continuous linear operators} \( T: \oline{X} \to \oline{Y} \) between them.
      \refitem{def:category/composition} Composition of morphisms is the usual \hyperref[def:set_valued_map/composition]{function composition} if we regard a morphism \( T: \oline{X} \to \oline{Y} \) as a function from \( \Sigma \oline{X} \) to \( \Sigma \oline{Y} \).
    \end{refenum}

    \thmitem{def:interpolated_topological_vector_space/interpolation_space} We say that the intermediate spaces \( X \) for \( \oline{X} \) and \( Y \) for \( \oline{Y} \) are a pair of \term{interpolation spaces} with respect to \( \oline{X} \) and \( \oline{Y} \) if, for any continuous linear \hyperref[def:interpolated_topological_vector_space/morphisms]{operator} \( T: \oline{X} \to \oline{Y} \) between the compatible pairs, we have \( T(X) \subseteq Y \).
  \end{thmenum}
\end{definition}

\begin{lemma}\label{thm:ordered_semigroup_max_distributivity}
  In a \hyperref[def:ordered_semigroup]{ordered semigroup} \( M \),
  \begin{equation}\label{eq:thm:ordered_semigroup_max_distributivity}
    \max \set{a b, c d} \leq \max \set{a, c} \cdot \max \set{b, d}.
  \end{equation}
\end{lemma}
\begin{proof}
  Since \( a \leq \max \set{a, c} \), then
  \begin{equation*}
    ab
    \leq
    \max \set{a, c} \cdot b
    \leq
    \max \set{a, c} \cdot \set{b, d}
  \end{equation*}

  Analogously, \( cd \leq \max \set{a, c} \cdot \set{b, d} \) and
  \begin{equation*}
    \max \set{a b, c d} \leq \max \set{a, c} \cdot \set{b, d}.
  \end{equation*}
\end{proof}

\begin{proposition}\label{def:banach_space_sum_and_intersection_norms}\mcite[24]{BerghLöfstrom1976Interpolation}
  Let \( X \coloneqq (X_0, X_1) \) be a \hyperref[def:interpolated_topological_vector_space/compatibility]{compatible pair} of \hyperref[def:banach_space]{Banach spaces}.

  \begin{thmenum}
    \thmitem{def:banach_space_sum_and_intersection_norms/intersection} The intersection \( \Delta \oline{X} = X_0 \cap X_1 \) is a Banach space with norm
    \begin{equation}\label{eq:def:banach_space_sum_and_intersection_norms/intersection}
      \norm{x}_{\Delta \oline{X}} \coloneqq \max \set{ \norm{x}_{X_0}, \norm{x}_{X_1} }.
    \end{equation}

    \thmitem{def:banach_space_sum_and_intersection_norms/sum} The sum \( \Sigma \oline{X} = X_0 + X_1 \) is a Banach space with norm
    \begin{equation}\label{eq:def:banach_space_sum_and_intersection_norms/sum}
      \norm{x}_{\Delta \oline{X}} \coloneqq \inf \set{ \norm{x_0}_{X_0} + \norm{x_1}_{X_1} : x_0 + x_1 = x }.
    \end{equation}
  \end{thmenum}
\end{proposition}
\begin{proof}
  \SubProofOf{def:banach_space_sum_and_intersection_norms/intersection} We will first show that \( \norm{x}_{\Delta \oline{X}} \) is indeed a norm.
  \begin{refenum}
    \refitem{def:norm/N1} We have
    \begin{equation}\label{eq:def:banach_space_sum_and_intersection_norms/intersection/zero}
      \norm{x}_{\Delta \oline{X}} = \max \set{ \norm{x}_{X_0}, \norm{x}_{X_1} } = 0 \T{if and only if} \norm{x}_{X_0} = \norm{x}_{X_1} = 0.
    \end{equation}

    Clearly \( 0 \) belongs to both \( X_0 \) and \( X_1 \) hence to their intersection. Therefore, \eqref{eq:def:banach_space_sum_and_intersection_norms/intersection/zero} is satisfied if and only if \( x = 0 \).

    \refitem{def:norm/N2} Absolute homogeneity follows from
    \begin{equation*}
      \norm{tx}_{\Delta \oline{X}}
      =
      \max \set{ \norm{tx}_{X_0}, \norm{tx}_{X_1} }
      \reloset {\ref{def:norm/N2}} =
      \abs{t} \max \set{ \norm{x}_{X_0}, \norm{x}_{X_1} }
      =
      \abs{t} \norm{x}_{\Delta \oline{X}}.
    \end{equation*}

    \refitem{def:norm/N3} Subadditivity follows from
    \begin{balign*}
      \norm{x + y}_{\Delta \oline{X}}
      &=
      \max \set{ \norm{x + y}_{X_0}, \norm{x + y}_{X_1} }
      \reloset {\ref{def:norm/N3}} \leq \\ &\leq
      \max \set{ \norm{x}_{X_0} + \norm{y}_{X_0}, \norm{x}_{X_1} + \norm{y}_{X_1} }
      \reloset {\ref{eq:thm:ordered_semigroup_max_distributivity}} \leq \\ &\leq
      \max \set{ \norm{x}_{X_0}, \norm{x}_{X_1} } + \max \set{ \norm{y}_{X_0}, \norm{y}_{X_1} }
      = \\ &=
      \norm{x}_{\Delta \oline{X}} + \norm{y}_{\Delta \oline{X}}.
    \end{balign*}
  \end{refenum}

  We will now show the completeness of \( \norm{\cdot}_{\Delta \oline{X}} \) directly. Let \( \seq{ x_k }_{k=1}^\infty \subseteq \Delta \oline{X} \) be a \hyperref[def:fundamental_net]{fundamental sequence}. Both \( X_0 \) and \( X_1 \) are complete, therefore \( \seq{ x_k }_{k=1}^\infty \) converges to the same value. Both are subspaces of \( \mscrU \), therefore the limit of the sequence is the same in both. In particular, it belongs to the intersection \( \Delta \oline{X} \).

  Denote the limit of \( \seq{ x_k }_{k=1}^\infty \) by \( x \). Let \( \varepsilon > 0 \) and let \( k_0 \) be an index such that both \( \norm{x_k - \xi_0}_{X_0} < \varepsilon \) and \( \norm{x_k - \xi_1}_{X_1} < \varepsilon \) whenever \( k \geq k_0 \). Then, for any \( k \geq k_0 \),
  \begin{equation*}
    \norm{x_k - x}_{\Delta \oline{X}}
    =
    \max\set{\norm{x_k - x}_{X_0}, \norm{x_k - x}_{X_1}}
    <
    \varepsilon.
  \end{equation*}

  Therefore, the sequence \( \seq{ x_k }_{k=1}^\infty \) converges to \( x_0 \) in \( \Delta \oline{X} \).

  \SubProofOf{def:banach_space_sum_and_intersection_norms/sum} Again, we will first show that \( \norm{x}_{\Sigma \oline{X}} \) is indeed a norm.
  \begin{refenum}
    \refitem{def:norm/N1} Analogously to \ref{def:banach_space_sum_and_intersection_norms/sum},
    \begin{equation*}
      \norm{x}_{\Sigma \oline{X}} = \inf \set{ \norm{x_0}_{X_0} + \norm{x_1}_{X_1} : x_0 + x_1 = x } = 0
    \end{equation*}
    if and only if
    \begin{equation*}
      \norm{x}_{X_0} = \norm{x}_{X_1} = 0.
    \end{equation*}

    \refitem{def:norm/N2} Absolute homogeneity follows from
    \begin{equation*}
      \norm{tx}_{\Sigma \oline{X}}
      =
      \inf \set{ \norm{tx_0}_{X_0} + \norm{tx_1}_{X_1} | x_0 + x_1 = x }
      \reloset {\ref{def:norm/N2}} =
      \abs{t} \norm{x}_{\Sigma \oline{X}}.
    \end{equation*}

    \refitem{def:norm/N3} Subadditivity follows from
    \begin{align*}
      &\phantom{{}={}}
      \norm{x + y}_{\Sigma \oline{X}}
      = \\ &=
      \inf \set{ \left( \norm{x_0}_{X_0} + \norm{x_1}_{X_1} \right) + \left( \norm{y_0}_{X_0} + \norm{y_1}_{X_1} \right) | \substack{\textstyle{x_0 + x_1 = x} \\ \textstyle{y_0 + y_1 = y}} }
      \leq \\ &\leq
      \norm{x}_{\Sigma \oline{X}} + \norm{y}_{\Sigma \oline{X}}.
    \end{align*}
  \end{refenum}

  It remains to prove the completeness of \( \norm{\cdot}_{\Sigma \oline{X}} \). Let \( \{ x_0^{(k)} + x_1^{(k)} \}_{k=1}^\infty \subseteq \Sigma \oline{X} \) be a \hyperref[def:fundamental_net]{fundamental sequence}. Fix \( \varepsilon > 0 \). Then there exists an index \( m_0 \) such that \( k, m \geq k_0 \) implies
  \begin{equation*}
    \norm{x_0^{(k)} + x_1^{(k)} - x_0^{(m)} + x_1^{(m)}}_{\Sigma \oline{X}} < \varepsilon.
  \end{equation*}

  But
  \begin{equation*}
    \norm{x_0^{(k)} - x_0^{(m)}}_{X_0}
    \leq
    \norm{\left(x_0^{(k)} + x_1^{(k)} \right) - \left( x_0^{(m)} + x_1^{(m)} \right)}_{\Sigma \oline{X}}
    <
    \varepsilon,
  \end{equation*}
  hence the sequence \( \{ x_0^{(k)} \}_{k=1}^\infty \) is fundamental. Since \( X_0 \) is complete, this sequence has a limit, which we will denote by \( \xi_0 \). We define \( \xi_1 \) analogously.

  With the same \( \varepsilon \), denote by \( k_0 \) an index such that both \( \norm{x_0^{(k)} - \xi_0}_{X_0} < \tfrac \varepsilon 2 \) and \( \norm{x_1^{(k)} - \xi_1}_{X_1} < \tfrac \varepsilon 2 \) whenever \( k \geq k_0 \).

  Then
  \begin{balign*}
    &\phantom{{}={}}
    \norm{\left( x_0^{(k)} + x_1^{(k)} \right) - \left( \xi_0 + \xi_1 \right)}_{\Sigma \oline{X}}
    = \\ &=
    \inf \set{ \norm{x_0}_{X_0} + \norm{x_1}_{X_1} : x_0 + x_1 = \left( x_0^{(k)} + x_1^{(k)} \right) - \left( \xi_0 + \xi_1 \right) }
    \leq \\ &\leq
    \norm{x_0^{(k)} - \xi_0}_{X_0} + \norm{x_1^{(k)} - \xi_1}_{X_1}
    <
    \tfrac \varepsilon 2 + \tfrac \varepsilon 2
    =
    \varepsilon.
  \end{balign*}

  Therefore, \( \xi_0 + \xi_1 \) is the limit of the sequence \( \{ x_0^{(k)} + x_1^{(k)} \}_{k=1}^\infty \subseteq \Sigma \oline{X} \) in \( \Sigma \oline{X} \).
\end{proof}

\begin{example}\label{thm:lp_interpolation_spaces/definition}
  The spaces \( L^p(\BbbR) \) are interpolation spaces for the pair \( (L^1(\BbbR), L^\infty(\BbbR)) \). The pair is compatible because both are subspaces of the space \( S(\BbbR) \) of all Lebesgue-measurable real function with metric
  \begin{equation*}
    \rho(f, g) \coloneqq \int_{\BbbR} \frac {\abs{f(x) - g(x)}} {1 + \abs(f(x) - g(x))} d\lambda.
  \end{equation*}
\end{example}

\begin{definition}\label{def:lebesgue_space}\mcite[6]{BerghLöfstrom1976Interpolation}
  Let \( \mu: U \to [0, \infty] \) be a positive measure and \( p \) be a positive real number. The \term{Lebesgue space} \( L_p \) is defined as the set of bounded functions \( f: U \mapsto \BbbK \) such that the norm
  \begin{equation*}
    \norm{f}_{L_p} \coloneqq \begin{dcases}
      \parens[\Big]{\int_U \abs{f(t)}^p dt}^{1/p}, &0 < p < \infty \\
      \ess\sup_{t \in U} \abs{f(t)} , &p = \infty
    \end{dcases}
  \end{equation*}
\end{definition}

\begin{theorem}[The Riesz-Thorin interpolation theorem]\label{thm:riesz_thorin}\mcite[24]{BerghLöfstrom1976Interpolation}
  Fix two measure spaces \( (U, \mu) \) and \( (V, \nu) \). Let \( T: S(U, \mu) \to S(V, \nu) \) be a continuous linear map between the corresponding spaces of measurable functions.

  Suppose that for some real numbers \( p_0, p_1, q_0, q_1 \geq 1 \) we have
  \begin{equation*}
    T(L^{p_j}(U, \mu)) \subseteq T(L^{q_j}(V, \nu)), j = 0, 1.
  \end{equation*}

  Additionally, let \( \theta \in (0, 1) \) and
  \begin{align*}
    \frac 1 p = \frac {1 - \theta} {p_0} + \frac {\theta} {p_1}
    &&
    \frac 1 q = \frac {1 - \theta} {q_0} + \frac {\theta} {q_1}.
  \end{align*}

  Then
  \begin{equation*}
    T(L^p(U, \mu)) \subseteq L^q(V, \nu)
  \end{equation*}
  and
  \begin{equation*}
    \norm{T}_{\hom(L^p, L^q)} \leq \norm{T}_{\hom(L^{p_0}, L^{q_0})}^{1 - \theta} \norm{T}_{\hom(L^{p_1}, L^{q_1})}^\theta.
  \end{equation*}
\end{theorem}

\begin{definition}\label{def:distribution_function}\mcite[6]{BerghLöfstrom1976Interpolation}
  Let \( \mu: U \to [0, \infty] \) be a positive measure and \( p \) be a positive real number.

  \begin{thmenum}
    \thmitem{def:distribution_function/distribution_function} Given a scalar-valued function \( f: U \mapsto \BbbK \), we define its \term{distribution function} as
    \begin{align*}
      &m_f: [0, \infty] \to \BbbK \\
      &m_f(\sigma) \coloneqq \mu(\set{ x :  > \sigma }).
    \end{align*}

    \thmitem{def:distribution_function/rearrangement} We define the \term{decreasing rearrangement} of \( f \) as
    \begin{equation*}
      f^*(t) \coloneqq \inf\set{ \sigma : m_f(\sigma) \leq t }.
    \end{equation*}

    \thmitem{def:distribution_function/lorenz_space} The \( (p, q)-\)Lorenz space, for potentially infinite positive \( q > 0 \), is the set of functions \( f: U \mapsto \BbbK \) for which the quasinorm
    \begin{equation*}
      \norm{f}_{L_{p,q}} \coloneqq \begin{dcases}
        \parens[\Big]{\int_0^\infty \parens[\Big]{\frac {f^*(\tau)} {\tau^p}}^q \frac {d t} t}^{\frac 1 q}, &1 \leq q < \infty \\
        \ess\sup\parens[\Big]{\frac {f^*(t)} {t^p}}, &q = \infty
      \end{dcases}
    \end{equation*}
    is finite.

    In particular, when \( q = \infty \), we use the notation
    \begin{equation*}
      \norm{f}_{L^{p*}} \coloneqq \parens[\Big]{p \int_0^\infty \sigma^p m_f(\sigma) \frac {d \sigma} \sigma}^{\frac 1 p}
    \end{equation*}
  \end{thmenum}
\end{definition}

\begin{theorem}[The Marcinkiewicz interpolation theorem]
  Fix two measure spaces \( (U, \mu) \) and \( (V, \nu) \). Let \( T: S(U, \mu) \to S(V, \nu) \) be a continuous linear map between the corresponding spaces of measurable functions.

  Suppose that for some real numbers \( p_0, p_1, q_0, q_1 \geq 1 \) we have
  \begin{equation*}
    T(L^{p_j}(U, \mu)) \subseteq T(L^{q_j *}(V, \nu)), j = 0, 1.
  \end{equation*}

  Additionally, let \( \theta \in (0, 1) \) and
  \begin{align*}
    \frac 1 p = \frac {1 - \theta} {p_0} + \frac {\theta} {p_1}
    &&
    \frac 1 q = \frac {1 - \theta} {q_0} + \frac {\theta} {q_1}.
  \end{align*}

  Then, if \( p \leq q \),
  \begin{equation*}
    T(L^p(U, \mu)) \subseteq L^q(V, \nu)
  \end{equation*}
  and
  \begin{equation*}
    \norm{T}_{\hom(L^p, L^q)} \leq C_\theta \norm{T}_{\hom(L^{p_0}, L^{q_0*})}^{1 - \theta} \norm{T}_{\hom(L^{p_1}, L^{q_1*})}^\theta
  \end{equation*}
  for some constant \( C_\theta \).
\end{theorem}

\begin{definition}\label{def:banach_interpolation_space_exponent}\mcite[27]{BerghLöfstrom1976Interpolation}
  Let \( \oline{X} \coloneqq ( X_0, X_1 ) \) and \( \oline{Y} \coloneqq ( Y_0, Y_1 ) \) be compatible pairs of Banach spaces. If \( X \) and \( Y \) are a pair of interpolation spaces and, additionally, the inequality
  \begin{equation}\label{da:def:banach_interpolation_space_exponent}
    \norm{T}_{\hom(X, Y)} \leq C \norm{T}_{\hom(X_0, Y_0)}^{1-\theta} \cdot \norm{T}_{\hom(X_1, Y_1)}^{\theta}
  \end{equation}
  holds for some constant \( C > 0 \) and \( \theta \in [0, 1] \), we say that the pair \( (X, Y) \) are \term{interpolation spaces of exponent} \( \theta \).

  If, additionally, \( C = 1 \), we say that \( (X, Y) \) is an \term{exact pair} of interpolation spaces.
\end{definition}

\begin{definition}\label{def:k_functional}\mcite[38]{BerghLöfstrom1976Interpolation}
  Let \( \oline{X} \coloneqq ( X_0, X_1 ) \) be a compatible pair of Banach spaces. Instead of the norm \( \norm{\cdot}_{X_1} \) in \( X_1 \), we can consider \hyperref[def:equivalent_metrics]{equivalent norms} of the type \( t\norm{\cdot}_{X_1} \) for \( t \geq 0 \). Furthermore, we can also introduce equivalent norms in \( \Sigma \oline{X} \) via the \term{\( K \)-functional}
  \begin{equation}\label{eq:def:k_functional}
    \begin{aligned}
      &K: (0, \infty) \times {\Sigma \oline{X}} \\
      &K(t, x) \coloneqq \inf \set{ \norm{x_0}_{X_0} + t\norm{x_1}_{X_1} : x_0 + x_1 = x }.
    \end{aligned}
  \end{equation}

  See \fullref{def:k_functional_properties/equivalent_norm} for a proof that \( x \mapsto K(t, x) \) for a fixed \( t \geq 0 \) is an equivalent norm in the sum \( \Sigma \oline{X} \).
\end{definition}

\begin{proposition}\label{def:k_functional_properties}\mcite[38]{BerghLöfstrom1976Interpolation}
  The \hyperref[def:k_functional]{\( K \)-functional} has the following basic properties:

  \begin{thmenum}
    \thmitem{def:k_functional_properties/basic} For any fixed \( x \in \Sigma \oline{X} \), the function \( t \mapsto K(t, x) \) is positive, \hyperref[def:order_function]{monotone} and \hyperref[def:convex_function]{concave}.

    \thmitem{def:k_functional_properties/inequality} For positive real numbers \( t, s > 0 \), we have the following inequality:
    \begin{equation}\label{eq:def:k_functional_properties/inequality}
      K(t, x) \leq \max\set{1, \frac t s} K(s, x).
    \end{equation}

    \thmitem{def:k_functional_properties/equivalent_norm} For any fixed \( t > 0 \), the function \( x \mapsto K(t, x) \) is an \hyperref[def:equivalent_metrics]{equivalent norm} in the sum \( \Sigma \oline{X} \).
  \end{thmenum}
\end{proposition}
\begin{proof}
  \SubProofOf{def:k_functional_properties/basic} That \( t \mapsto K(t, x) \) is positive is a slight generalization of \fullref{def:norm/N1}, which can be proved as in \fullref{def:banach_space_sum_and_intersection_norms/sum}.

  Monotonicity follows from the monotonicity of the infimum.

  To see that \( t \mapsto K(t, x) \) is concave, fix \( x \), \( \lambda \in [0, 1] \) and \( t, s > 0 \). We have
  \begin{align*}
    &\phantom{{}={}}
    K(\lambda t + (1 - \lambda) s, x)
    = \\ &=
    \inf \set{ \norm{x_0}_{X_0} + (\lambda t + (1 - \lambda) s)\norm{x_1}_{X_1} | x_0 + x_1 = x }
    = \\ &=
    \inf \set{ \lambda \left(\norm{x_0}_{X_0} + t \norm{x_1}_{X_1} \right) + (1 - \lambda) \left(\norm{x_0}_{X_0} + s \norm{x_1}_{X_1} \right) | x_0 + x_1 = x }
    \geq \\ &\geq
    \lambda K(t, x) + (1 - \lambda) K(s, x).
  \end{align*}

  \SubProofOf{def:k_functional_properties/inequality} Fix positive real numbers \( t, s > 0 \).
  \begin{itemize}
    \item If \( t \leq s \), by monotonicity we have
    \begin{equation}\label{eq:def:k_functional_properties/inequality/monotonicity}
      K(t, x) \leq K(s, x)
    \end{equation}

    \item If \( t > s \), we use concavity with
    \begin{equation*}
      s = \frac s t t + \left(1 - \frac s t \right) 0
    \end{equation*}
    to obtain
    \begin{equation*}
      K(s, x) \geq \frac s t K(t, x) + \left(1 - \frac s t \right) K(0, x).
    \end{equation*}

    By positivity of \( K \), we have \( K(t, x) = 0 \) if and only if \( t = 0 \), hence
    \begin{equation}\label{eq:def:k_functional_properties/inequality/concavity}
      K(t, x) \leq \frac t s K(s, x).
    \end{equation}
  \end{itemize}

  Combining \eqref{eq:def:k_functional_properties/inequality/monotonicity} and \eqref{eq:def:k_functional_properties/inequality/concavity}, we obtain \eqref{eq:def:k_functional_properties/inequality}.

  \SubProofOf{def:k_functional_properties/equivalent_norm} That \( x \mapsto K(t, x) \) for a fixed \( t > 0 \) is a slight generalization of the proof in \fullref{def:banach_space_sum_and_intersection_norms/sum}.

  That the norms \( \norm{\cdot}_{\Sigma \oline{X}} \) and \( K(t, \cdot) \) are equivalent follows from \eqref{eq:def:k_functional_properties/inequality} with \( s = 1 \) for the upper bound and \( t = 1, s = t \) for the lower bound. That is,
  \begin{equation*}
    \min\set{1, t} \underbrace{K(1, x)}_{\norm{x}_{\Sigma \oline{X}}} \leq K(t, x) \leq \max\set{1, t} \underbrace{K(1, x)}_{\norm{x}_{\Sigma \oline{X}}}.
  \end{equation*}
\end{proof}

\begin{example}\label{thm:lp_interpolation_spaces/k_functional}
  The \hyperref[def:k_functional]{\( K \)-functional} for the pair \( (L_1(\BbbR), L_\infty(\BbbR)) \) from \fullref{thm:lp_interpolation_spaces/definition} is
  \begin{equation*}
    K(t, f) \coloneqq \int_0^t f^*(\tau) d\tau.
  \end{equation*}
\end{example}

\begin{definition}\label{def:lorenz_quasinorm}
  For \( \theta \in \BbbR \), \( q \in (0, \infty] \) and nonnegative functions \( g: [0, \infty) \to [0, \infty] \) we define
  \begin{equation}\label{eq:def:lorenz_quasinorm}
    \Phi_{\theta,q}(g) \coloneqq \begin{dcases}
      \left( \int_0^\infty \left( \frac {g(\tau)} {\tau^\theta} \right)^q \frac {d\tau} \tau \right)^{\tfrac 1 q}, &0 < q < \infty \\
      \ess\sup_{t \geq 0} \left( \frac {g(t)} {t^\theta} \right),                                                &q = \infty
    \end{dcases}
  \end{equation}
  and
  \begin{equation}\label{eq:def:lorenz_quasinorm/gamma}
    \gamma_{\theta,q} \coloneqq \Phi_{\theta,q}(\min(t, 1)).
  \end{equation}
\end{definition}

\begin{proposition}\label{thm:def:lorenz_quasinorm}
  The function \hyperref[def:lorenz_quasinorm]{\( \Phi_{\theta,q} \)} has the following basic properties:

  \begin{thmenum}
    \thmitem{thm:def:lorenz_quasinorm/reciprocal} For \( s > 0 \) and \( h(t) \coloneqq g(\tfrac t s) \) we have
    \begin{equation}\label{eq:thm:def:lorenz_quasinorm/reciprocal}
      \Phi_{\theta,q}(h) = \frac 1 {s^{\theta}} \Phi_{\theta,q}(g).
    \end{equation}

    \thmitem{thm:def:lorenz_quasinorm/gamma} For finite \( q \) we have
    \begin{equation}\label{eq:thm:def:lorenz_quasinorm/gamma}
      \gamma_{\theta,q} = \left( \frac 1 {q \theta (1 - \theta)} \right)^{\tfrac 1 q}.
    \end{equation}
  \end{thmenum}
\end{proposition}
\begin{proof}
  \SubProofOf{thm:def:lorenz_quasinorm/reciprocal} The case \( q = \infty \) is obvious. For \( 0 < q < \infty \), we have
  \begin{balign*}
    \Phi_{\theta,q}(h)
    &=
    \left( \int_0^\infty \left( \frac {h(\tau)} {\tau^\theta} \right)^q \frac {d\tau} \tau \right)^{\tfrac 1 q}
    = \\ &=
    \left( \frac 1 {s^{\theta q}} \int_0^\infty \left( \frac {g(\tfrac \tau s)} {\left(\tfrac \tau s \right)^\theta} \right)^q \frac {d{\tfrac \tau s}} {\tfrac \tau s} \right)^{\tfrac 1 q}
    = \\ &=
    \frac 1 {s^{\theta}} \Phi_{\theta,q}(g).
  \end{balign*}

  \SubProofOf{thm:def:lorenz_quasinorm/gamma} We can raise \( \gamma_{\theta,q} \) to the \( q \)-th power for brevity of notation:
  \begin{balign*}
    \gamma_{\theta,q}^q
    &=
    \Phi_{\theta,q}(\min(t, 1))^q
    = \\ &=
    \int_0^1 \left( \frac {\tau} {\tau^\theta} \right)^q \frac {d\tau} \tau + \int_1^\infty \left( \frac {1} {\tau^\theta} \right)^q \frac {d\tau} \tau
    = \\ &=
    \int_0^1 \tau^{(1 - \theta) q - 1} d\tau + \int_1^\infty \tau^{-\theta q - 1} d\tau
    = \\ &=
    \frac {1 - 0} {(1 - \theta) q} + \frac {\lim_{\tau \to \infty} \tau^{-\theta q} - 1} {-\theta q}
    = \\ &=
    \frac 1 {(1 - \theta) q} - \frac 1 {-\theta q}
    = \\ &=
    \frac {-\theta q - (1 - \theta) q} {(1 - \theta) (-\theta) q^2}
    = \\ &=
    \frac 1 {(1 - \theta) \theta q}
  \end{balign*}
\end{proof}

\begin{definition}\label{def:k_functional_interpolation_space}\mcite[40]{BerghLöfstrom1976Interpolation}
  Let \( \oline{X} \coloneqq ( X_0, X_1 ) \) be a compatible pair of Banach spaces.

  For \( \theta \in (0, \infty) \), \( q \in (0, \infty] \), we introduce the following norm:
  \begin{equation}\label{eq:def:k_functional_interpolation_space/norm}
    \norm{x}_{\theta,q,K} \coloneqq \Phi_{\theta,q}(K(t, x)).
  \end{equation}

  The subspace of \( \Sigma\oline{X} \) for which this norm is finite is denoted by either
  \begin{align*}
    K_{\theta,q}(\oline{X})
    &&
    X_{\theta,q,K}.
  \end{align*}
\end{definition}

\begin{theorem}\label{thm:k_functional_interpolation}\mcite[thm. 3.1.2]{BerghLöfstrom1976Interpolation}
  Let \( \theta \in (0, 1) \) and \( q \in (0, \infty] \). The space \( X_{\theta,q,K} \) defined in \fullref{eq:def:k_functional_interpolation_space/norm} is an \hyperref[def:banach_interpolation_space_exponent]{exact interpolation space} of exponent \( \theta \). Furthermore,
  \begin{equation}\label{eq:thm:k_functional_interpolation/inequality}
    K(s, x) \leq (\gamma_{\theta,q})^{-1} s^\theta \norm{x}_{\theta,q,K}.
  \end{equation}
\end{theorem}
\begin{proof}
  Note that \( K(s, x) \) is a norm on \( \Sigma \oline{X} \) by \fullref{def:k_functional_properties/equivalent_norm}. Therefore, \( \norm{\cdot}_{\theta,q,K} \), the composition of \( K(s, x) \) with the \hyperref[def:lorenz_quasinorm]{Lorenz quasinorm} \( \Phi_{\theta,q} \), is a norm.

  We denote by \( X_{\theta,q,K} \) the space consisting of all vectors from \( \Sigma \oline{X} \) where the norm \eqref{eq:def:k_functional_interpolation_space/norm} is finite.

  From \eqref{eq:def:k_functional_properties/inequality} it follows that
  \begin{equation*}
    \min(1, \tfrac t s) K(s, x) \leq K(t, x)
  \end{equation*}
  and hence
  \begin{equation*}
    \underbrace{\Phi_{\theta,q}}_{\text{depends on } t} \parens[\Big]{ \min(1, \tfrac t s) K(s, x) } \leq \underbrace{\Phi_{\theta,q}(K(s, x))}_{\text{norm in } X_{\theta,q,K}}.
  \end{equation*}

  By \eqref{eq:thm:def:lorenz_quasinorm/reciprocal}, we have
  \begin{equation*}
    \Phi_{\theta,q}(\min(1, \tfrac t s)) = \tfrac 1 {s^\theta} \underbrace{\Phi_{\theta,q}(\min(1, t))}_{\hyperref[eq:def:lorenz_quasinorm/gamma]{\gamma_{\theta,q}}},
  \end{equation*}
  and \eqref{eq:thm:k_functional_interpolation/inequality} follows.

  It remains to show that \( X_{\theta,q,K} \) is an exact interpolation space of exponent \( \theta \).

  Note that \( K(1, x) = \norm{x}_{\Sigma \oline{X}} \) and thus \eqref{eq:thm:k_functional_interpolation/inequality} with \( s = 1 \) implies that
  \begin{equation*}
    \gamma_{\theta,q} \norm{x}_{\Sigma \oline{X}} \leq \norm{x}_{\theta,q,K},
  \end{equation*}
  which shows that \( X_{\theta,q,K} \) can be embedded continuously into \( \Sigma \oline{X} \).

  On the other hand, for \( x \in \Delta \oline{X} \) we have
  \begin{equation*}
    K(t, x) \leq \norm{x} \leq \norm{x}_{\Delta \oline{X}} \T{since} x = x + 0
  \end{equation*}
  and
  \begin{equation*}
    K(t, x) \leq \norm{x} \leq t \norm{x}_{\Delta \oline{X}} \T{since} x = 0 + x.
  \end{equation*}

  Therefore,
  \begin{equation*}
    K(t, x) \leq \min(1, t) \norm{x}_{\Delta \oline{X}},
  \end{equation*}
  which after applying \( \Phi_{\theta,q} \) becomes
  \begin{equation*}
    \norm{x}_{\theta,q,K} \leq \gamma_{\theta,q} \norm{x}_{\Delta \oline{X}}.
  \end{equation*}

  Hence, we have the chain of continuous linear inclusions of Banach spaces
  \begin{equation*}
    \Delta \oline{X} \subseteq X \subseteq \Sigma \oline{X}.
  \end{equation*}

  Finally, to show that \( X \) is an interpolation space of exponent \( \theta \), fix a linear operator \( T: \oline{X} \mapsto \oline{CY} \) between compatible pairs and let \( Y \) be an intermediate space for \( \oline{CY} \).

  Then
  \begin{align*}
    K(t, Tx)_{\oline{CY}}
    &\leq
    \inf \set{ \norm{y_0}_{Y_0} + t \norm{y_1}_{Y_1} : y_0 + y_1 = Tx }
    \leq \\ &\leq
    \inf \set{ \norm{T}_{\hom(X_0, Y_0)} \norm{x_0}_{X_0} + t \norm{T}_{\hom(X_1, Y_1)} \norm{x_1}_{Y_1} : x_0 + x_1 = x }
    = \\ &=
    \norm{T}_{\hom(X_0, Y_0)} K\parens[\Bigg]{\frac {\norm{T}_{\hom(X_1, Y_1)}} {\norm{T}_{\hom(X_0, Y_0)}} t, x}.
  \end{align*}

  By applying \( \Phi_{\theta,q} \) to both sides and using \eqref{eq:thm:def:lorenz_quasinorm/reciprocal}, we obtain
  \begin{equation*}
    \norm{Tx}_{\oline{Y}_{\theta,q,K}}
    \leq
    {\norm{T}_{\hom(X_1, Y_1)}}^{1 - \theta} {\norm{T}_{\hom(X_0, Y_0)}}^{\theta} \norm{x}_{\oline{Y}_{\theta,q,K}}.
  \end{equation*}

  Thus, \( X \) satisfies \fullref{def:banach_interpolation_space_exponent} for being an exact interpolating space with exponent \( \theta \).
\end{proof}

\begin{definition}\label{def:discrete_k_interpolation_space}
  For positive numbers \( \theta \) and \( q \), we denote by \( \lambda^{\theta,q} \) the set of all doubly-infinite real sequences \( \seq{ x_k }_{k=-\infty}^\infty \) such that the norm
  \begin{equation}\label{eq:def:discrete_k_interpolation_space}
    \norm{\seq{ x_k }_{k=-\infty}^\infty}_{\lambda^{\theta,q}} \coloneqq \left( \sum_{k=-\infty}^\infty \left( \frac {\abs{x_k}} {2^{k\theta}} \right)^q \right)^{\tfrac 1 q}
  \end{equation}
  is finite.
\end{definition}

\begin{theorem}\label{thm:discrete_k_interpolation}\mcite[lemma 3.1.3]{BerghLöfstrom1976Interpolation}
  The vector \( x \in \Sigma\oline{X} \) belongs to \hyperref[def:k_functional_interpolation_space]{\( X_{\theta,q,K} \)} if and only if the sequence \( \seq{ x_k }_{k=-\infty}^\infty \) defined as
  \begin{equation}\label{eq:thm:discrete_k_interpolation/sequence}
    x_k \coloneqq K(2^k, x)
  \end{equation}
  belongs to \hyperref[def:discrete_k_interpolation_space]{\( \lambda^{\theta,q} \)}.

  Furthermore, for any integer \( k \) the following inequalities hold:
  \begin{equation}\label{eq:thm:discrete_k_interpolation/inequalities}
    \frac 1 {2^\theta} \ln 2 \norm{x_k}_{\lambda^{\theta,q}}
    \leq
    \norm{x}_{\theta,q,K}
    \leq
    2 \cdot \ln 2 \norm{x_k}_{\lambda^{\theta,q}}.
  \end{equation}
\end{theorem}
\begin{proof}
  We have
  \begin{equation*}
    \norm{x}_{\theta,q,K}^q
    =
    \int_0^\infty \left( \frac {K(\tau, x)} {\tau^\theta} \right)^q \frac {d\tau} \tau
    =
    \sum_{k=-\infty}^\infty \int_{2^k}^{2^{k+1}} \left( \frac {K(\tau, x)} {\tau^\theta} \right)^q \frac {d\tau} \tau.
  \end{equation*}

  By \fullref{def:k_functional_properties/inequality}, for each integer \( k \),
  \begin{equation*}
    K(2^k, x) \leq 2 K(2^k, x).
  \end{equation*}

  By the \hyperref[def:k_functional_properties/basic]{monotonicity} of \( K \), for \( t \in [2^k, 2^{k+1}] \) we have
  \begin{equation*}
    K(2^k, x) \leq K(t, x) \leq 2 K(2^k, x).
  \end{equation*}

  Denote \( x_k \coloneqq K(2^k, x) \). For \( 2^k \leq t \leq 2^{k+1} \) we have
  \begin{equation*}
    \frac{x_k}{2^{(k+1)\theta}} \leq \frac{K(t, x)}{t^\theta} \leq 2 \frac{x_k}{2^{k\theta}}
  \end{equation*}

  Therefore,
  \begin{align*}
    \norm{x}_{\theta,q,K}^q
    &=
    \sum_{k=-\infty}^\infty \int_{2^k}^{2^{k+1}} \left( \frac {K(\tau, x)} {\tau^\theta} \right)^q \frac {d\tau} \tau
    \leq \\ &\leq
    2^q \sum_{k=-\infty}^\infty \left(\frac{x_k}{2^{k\theta}} \right)^q \cdot \ln \tau \mid_{\tau=2^k}^{2^{k+1}}
    = \\ &=
    \ln 2 \cdot 2^q \sum_{k=-\infty}^\infty \left(\frac{x_k}{2^{k\theta}} \right)^q
    = \\ &=
    2^q \ln 2 \norm{\seq{ x_k }_{k=-\infty}^\infty}_{\lambda^{\theta,q}}^q
  \end{align*}
  and similarly for the lower bound.
\end{proof}

\begin{definition}\label{def:e_functional}\mcite[174]{BerghLöfstrom1976Interpolation}
  Let \( \oline{X} = (X_0, X_1) \) be a compatible pair of Banach spaces. Let \( x \in \Sigma \oline{X} \). Put
  \begin{equation}\label{eq:def:e_functional}
    \begin{aligned}
      &E: (0, \infty) \times {\Sigma \oline{X}} \\
      &E(t, x) \coloneqq \inf \set{ \norm{x - x_0}_{X_1} : \norm{x_0}_{X_0} \leq t }.
    \end{aligned}
  \end{equation}
\end{definition}

\begin{proposition}\label{thm:def:e_functional}\mcite[lemma 7.1.3]{BerghLöfstrom1976Interpolation}
  When \( \oline{X} \) are quasi-Banach spaces, the \hyperref[def:e_functional]{\( E \)-functional} has the following basic properties:

  \begin{thmenum}
    \thmitem{def:k_functional_properties/decreasing} For fixed \( x \in \Sigma\oline{X} \), the function \( t \mapsto E(t, x) \) is decreasing.

    \thmitem{def:k_functional_properties/subaditive} For \( \varepsilon \in (0, 1) \), we have
    \begin{equation*}
      E(t, x + y) \leq E(\varepsilon t, x) + E((1 + \varepsilon) t, y).
    \end{equation*}

    \thmitem{def:k_functional_properties/positive} \( x = 0 \) if and only if \( E(t, x) = 0 \) for all \( t > 0 \).

    \thmitem{def:k_functional_properties/k_functional_connection}\mcite[thm. 7.1.4]{BerghLöfstrom1976Interpolation}
    \begin{equation*}
      E(t, x) = \sup \set{ \frac {K(s, x) - t} s : s > 0 }.
    \end{equation*}
  \end{thmenum}
\end{proposition}

\begin{definition}\label{def:approximation_space}\mcite[def. 7.1.5]{BerghLöfstrom1976Interpolation}
  Let \( \oline{X} = (X_0, X_1) \) be a compatible pair of Banach spaces. We define an \term{approximation space} \( E_{\alpha,q}(\oline{X}) \) for \( x \in \Sigma\oline{X} \) as the space of all members of \( \Sigma\oline{X} \) for which the following norm
  \begin{equation}\label{eq:def:approximation_space/norm}
    \norm{x}_{\alpha,q,E} \coloneqq \Phi_{-\alpha,q}(E(t,a))
  \end{equation}
  is finite.

  Here \( \alpha \) and \( q \) are both positive real numbers and \( q \) is potentially \( \infty \).
\end{definition}

\begin{theorem}\label{thm:interpolation_space_and_approximation_space}\mcite[thm. 7.1.6]{BerghLöfstrom1976Interpolation}
  Let \( X \) be a compatible pair of Banach spaces. Let \( \alpha \) and \( q \) be positive real numbers and define
  \begin{align*}
    \theta \coloneqq \frac 1 {\alpha + 1},
    &&
    r \coloneqq \theta q.
  \end{align*}

  Then
  \begin{equation*}
    (E_{\alpha,\theta q}(\oline{X}))^\theta = K_{\theta,q}(\oline{X}).
  \end{equation*}
\end{theorem}

\begin{theorem}\label{thm:interpolation_space_and_approximation_space_reiteration}\mcite[thm. 7.1.8]{BerghLöfstrom1976Interpolation}
  Let \( X \) be a compatible pair of Banach spaces. Let \( \theta, \alpha_0, \alpha_1, r_0, r_1 \) and \( q \) be positive real numbers such that \( \alpha_0 \neq \alpha_1 \) and define \( r \coloneqq \theta q \) and
  \begin{align*}
    \alpha \coloneqq (1 - \theta) \alpha_0 + \theta \alpha_1,
    &&
    \beta \coloneqq - \frac {\alpha_1 - \alpha} {\alpha_0 - \alpha}.
  \end{align*}

  Then
  \begin{equation*}
    K_{\theta,q}(E_{\alpha_0,r_0}(\oline{X}), E_{\alpha_1,r_1}(\oline{X})) = E_{\alpha,q}(\oline{X})
  \end{equation*}
  and
  \begin{equation*}
    E_{\beta,r}(E_{\alpha_0,r_0}(\oline{X}), E_{\alpha_1,r_1}(\oline{X}))^\theta = E_{\alpha,q}(\oline{X}).
  \end{equation*}
\end{theorem}
