\section{Difference equations}\label{sec:difference_equations}

\paragraph{Difference operators}

\begin{remark}\label{rem:finite_difference_literature}
  Difference operators are studied as approximations of univariate real functions. Given such a function \( f: \BbbR \to \BbbR \), we can form the finite difference \( \Delta = f(x + h) - f(x) \) with step \( h \), and the second difference becomes \( \Delta^2 = f(x + 2h) - 2f(x + h) + f(x) \). This is done in books dedicated to finite differences like \incite[3]{Jordan1950FiniteDifferences} and \incite[12]{Гельфонд1959ИсчислениеКонечныхРазностей}.

  Numeric method literature focuses on finding \( f \) based on a given sequence of values, thus on the case \( N = \BbbN \) and \( R = \BbbR \) by \incite[5]{LevyLessman1961FiniteDifferenceEquations}, \incite[65]{БахваловЖидковКобельков2015ЧисленныеМетоды} and \incite[21]{Боянов2008ЧислениМетоди}.

  When defining finite difference operators in \fullref{def:difference_operator}, we follow and slightly generalize the definition of \incite[77]{Stanley2012EnumerativeCombinatoricsVol1}, treating \( \Delta \) as an abstract operator on \( \fun(\BbbZ, \BbbK) \), where \( \BbbK \) is a field of characteristic zero.
\end{remark}

\begin{definition}\label{def:difference_operator}\mimprovised
  Let \( N \) be either the \hyperref[def:natural_numbers]{semiring \( \BbbN \) of natural numbers} or the \hyperref[def:integers]{ring \( \BbbZ \) of integers}, and let \( R \) be any \hyperref[def:ring]{ring}. Consider the set \( \fun(N, R) \) of all functions between them --- its elements are either \hyperref[def:sequence]{infinite sequences} or \hyperref[def:doubly_infinite_sequence]{doubly infinite sequences}.

  We define the \term[bg=крайна разлика от ред \( k \) (\cite[21]{Боянов2008ЧислениМетоди}), ru=конечная разность (\cite[11]{Гельфонд1959ИсчислениеКонечныхРазностей}), en=finite difference (\cite[77]{Stanley2012EnumerativeCombinatoricsVol1})]{finite difference} operator \( \Delta \) on \( \fun(N, R) \) as
  \begin{equation}\label{eq:def:difference_operator}
    \Delta f(n) \coloneqq f(n + 1) - f(n).
  \end{equation}

  It is then straightforward to define the \( k \)-th order difference operator \( \Delta^k \) by iterated composition.
\end{definition}
\begin{comments}
  \item We slightly generalize the definition by \incite[77]{Stanley2012EnumerativeCombinatoricsVol1} in order to reinterpret some \hyperref[def:recurrence_relation]{recurrence relations}. See \fullref{rem:finite_difference_literature} for more general remarks on the definition.
\end{comments}

\begin{definition}\label{def:shift_operator}\mimprovised
  We can map the indices of the \hyperref[def:doubly_infinite_sequence]{doubly infinite sequence}
  \begin{equation*}
    \ldots, x_{-2}, x_{-1}, x_0, x_1, x_2, \ldots
  \end{equation*}
  via \( n \mapsto n - 1 \) to shift it to the left and obtain
  \begin{equation*}
    \ldots, x_{-1}, x_0, x_1, x_2, x_3, \ldots
  \end{equation*}
  or via \( n \mapsto n + 1 \) to shift it the right and obtain
  \begin{equation*}
    \ldots, x_{-3}, x_{-2}, x_{-1}, x_0, x_1, \ldots
  \end{equation*}

  This defines two operators on the set \( \fun(\BbbZ, A) \) of doubly infinite sequences over some set \( A \), which we call \term[en=left shift (operator) (\cite[exerc. 7.26]{FabianEtAl2001FunctionalAnalysis}), ru=левый (оператор сдвига) (\cite[example 1.3.3]{Хелемский2014ФункциональныйАнализ})]{left shift} \( L \) and \term[en=right shift (operator) (\cite[exerc. 7.26]{FabianEtAl2001FunctionalAnalysis}), ru=правый (оператор сдвига) (\cite[example 1.3.3]{Хелемский2014ФункциональныйАнализ})]{right shift} \( R \).

  An important variation is more widespread: given a sequence
  \begin{equation*}
    x_1, x_2, x_3, x_4, \ldots
  \end{equation*}
  left shift gives
  \begin{equation*}
    x_2, x_2, x_4, x_5, \ldots
  \end{equation*}
  but right shift is not obvious. When working over a \hyperref[def:ring]{ring}, an easy choice is to pad with zeros:
  \begin{equation*}
    0, x_1, x_2, x_3, \ldots
  \end{equation*}
\end{definition}
\begin{comments}
  \item Shift operators are closely related to shifts of \hyperref[def:positional_number_system]{digit string}, defined in \fullref{def:digit_shift}.

  \item For doubly infinite sequences, \incite[77]{Stanley2012EnumerativeCombinatoricsVol1} uses a right shift operator in order to use \fullref{thm:binomial_theorem} on \hyperref[def:difference_operator]{difference operators}. He denotes it by \( E \) and calls it simply a \enquote{shift operators} --- both conventions are unfortunately ambiguous for us.

  The usage for sequences is popular, but inconsistent across the literature, and serves mostly for illustrative examples.

  This is how \incite[exerc. 7.26]{FabianEtAl2001FunctionalAnalysis} define the \enquote{left shift operator} \( L \) and \enquote{right shift operator} in the \hyperref[def:lebesgue_space]{Lebesgue (sequence) space} \( l_2 \). \incite[example 1.3.3]{Хелемский2014ФункциональныйАнализ} uses the same terminology (\enquote{оператор сдвига} in Russian) in the case of \( l_p \), and denotes the operators by \( T_l \) and \( T_r \). \incite[exerc. 2.1.21]{FriedbergInselSpence2018LinearAlgebra} use the same terminology for general sequences over a field, but denotes them by \( T \) and \( U \).

  \incite[84]{Lang2002Algebra} denotes the right shift operator on integers by \( T \) and calls it \enquote{the} shift operator. \incite[347]{Rudin1987RealAndComplexAnalysis} denotes the right shift operator on complex numbers by \( S \) and also calls it \enquote{the} shift operator.
\end{comments}

\begin{proposition}\label{thm:difference_operator_factorization}\mcite[77]{Stanley2012EnumerativeCombinatoricsVol1}
  For the \( k \)-th \hyperref[def:difference_operator]{difference operator}, we have the following factorization:
  \begin{equation}\label{eq:thm:difference_operator_factorization}
    \Delta^k f(n) = \sum_{i=0}^k (-1)^{k-i} \binom k i f(n + i)
  \end{equation}
\end{proposition}
\begin{proof}
  In the \hyperref[def:endomorphism_ring]{endomorphism ring} of \( \fun(\BbbZ, A) \), the identity operator \( \id \) on \( \fun(\BbbZ, A) \) commutes with any other operator, in particular the right \hyperref[def:shift_operator]{shift operator} \( R \).

  Furthermore, we can express \( \Delta \) as \( R - \id \). Thus, \fullref{thm:binomial_theorem} gives us
  \begin{equation*}
    \Delta^k
    =
    (R - \id)^k
    =
    \sum_{i=0}^k \binom k i R^i (-\id)^{k-i}
    =
    \sum_{i=0}^k (-1)^{k-i} \binom k i R^i \id^{k-i}.
  \end{equation*}

  Then \eqref{eq:thm:difference_operator_factorization} follows directly.
\end{proof}
