\section{Difference equations}\label{sec:difference_equations}

\paragraph{Difference operators}

\begin{remark}\label{rem:finite_difference_literature}
  Difference operators are studied as approximations of univariate real functions. Given such a function \( f: \BbbR \to \BbbR \), we can form the finite difference \( \Delta = f(x + h) - f(x) \) with step \( h \), and the second difference becomes \( \Delta^2 = f(x + 2h) - 2f(x + h) + f(x) \). This is done in books dedicated to finite differences like \incite[3]{Jordan1950FiniteDifferences} and \incite[12]{Гельфонд1959ИсчислениеКонечныхРазностей}.

  Numeric method literature focuses on finding \( f \) based on a given sequence of values, thus on the case \( N = \BbbN \) and \( R = \BbbR \) by \incite[5]{LevyLessman1961FiniteDifferenceEquations}, \incite[65]{БахваловЖидковКобельков2015ЧисленныеМетоды} and \incite[21]{Боянов2008ЧислениМетоди}.

  When defining finite difference operators in \fullref{def:finite_difference_operator}, we follow and slightly generalize the definition of \incite[77]{Stanley2012EnumerativeCombinatoricsVol1}, treating \( \Delta \) as an abstract operator on \( \fun(\BbbZ, \BbbK) \), where \( \BbbK \) is a field of characteristic zero.
\end{remark}

\begin{definition}\label{def:finite_difference_operator}\mimprovised
  Let \( N \) be either the \hyperref[def:natural_numbers]{semiring \( \BbbN \) of natural numbers} or the \hyperref[def:integers]{ring \( \BbbZ \) of integers}, and let \( R \) be any \hyperref[def:ring]{ring}. Consider the set \( \fun(N, R) \) of all functions between them --- its elements are either \hyperref[def:sequence]{infinite sequences} or \hyperref[def:doubly_infinite_sequence]{doubly infinite sequences}.

  We define the \term[bg=крайна разлика от ред \( k \) (\cite[21]{Боянов2008ЧислениМетоди}), ru=конечная разность (\cite[11]{Гельфонд1959ИсчислениеКонечныхРазностей}), en=finite difference (\cite[77]{Stanley2012EnumerativeCombinatoricsVol1})]{finite difference} operator \( \Delta \) on \( \fun(N, R) \) as
  \begin{equation}\label{eq:def:finite_difference_operator}
    \Delta f(n) \coloneqq f(n + 1) - f(n).
  \end{equation}

  It is then straightforward to define the \( k \)-th order difference operator \( \Delta^k \) by iterated composition.
\end{definition}
\begin{comments}
  \item We slightly generalize the definition by \incite[77]{Stanley2012EnumerativeCombinatoricsVol1} in order to reinterpret some \hyperref[def:recurrence_relation]{recurrence relations}. See \fullref{rem:finite_difference_literature} for more general remarks on the definition.
\end{comments}

\begin{definition}\label{def:shift_operator}\mimprovised
  Fix a \hyperref[def:monoid]{monoid} \( M \) and a set \( A \). For every element \( h \) of \( M \), we define the \term[ru=оператор сдвига (\cite[9]{Левитан1973ТеорияОператоровОбобщенногоСдвига}), en=shift operator (\cite{Stanley2012EnumerativeCombinatoricsVol1})]{shift} by \( h \) on functions from \( M \) to \( A \) as
  \begin{equation}\label{eq:def:shift_operator}
    S^h(f(x)) \coloneqq f(x + h).
  \end{equation}
\end{definition}
\begin{comments}
  \item Thus, \( S \) is a \hyperref[def:monoid_action]{monoid action} of \( M \) on \( \fun(M, A) \).

  \item Boris Levitan has a monograph on shift operators, \cite{Левитан1973ТеорияОператоровОбобщенногоСдвига}, based on previous work by Jean Dieudonné. Although only shift operators on univariate real functions are considered in the book, it is suggested that a lot of the theory works for greater generality when \( M \) is a group. We simply generalize this to monoids.

  \incite[12]{LevyLessman1961FiniteDifferenceEquations} provide a similar definition for functions on real numbers.
\end{comments}

\begin{definition}\label{def:sequence_shift_operator}
  Consider the \hyperref[def:shift_operator]{shift operator} \( S^n \) on \hyperref[def:doubly_infinite_sequence]{doubly infinite sequences} over some set \( A \). It is given by translating indices via \( i \mapsto i + n \), which is the \( n \)-th iterate of \( S^1 \).

  We denote \( S^1 \) by \( R \) and call it the \term[en=right shift operator (\cite[exerc. 7.26]{FabianEtAl2001FunctionalAnalysis}), ru=правый оператор сдвига (\cite[example 1.3.3]{Хелемский2014ФункциональныйАнализ})]{right shift operator}. We denote its inverse, \( S^{-1} \), by \( L \), and call it the \term[en=left shift operator (\cite[exerc. 7.26]{FabianEtAl2001FunctionalAnalysis}), ru=левый оператор сдвига (\cite[example 1.3.3]{Хелемский2014ФункциональныйАнализ})]{left shift operator}.

  It is straightforward to generalize left shift operators to \hyperref[def:sequence]{infinite sequences}, but right shifting is not obvious because it requires a choice for the prepended element. In case \( A \) is a ring, we define right shifts on infinite sequences by prepending a zero.
\end{definition}
\begin{comments}
  \item For doubly infinite sequences, \incite[77]{Stanley2012EnumerativeCombinatoricsVol1} uses a right shift operator in order to use \fullref{thm:binomial_theorem} on \hyperref[def:finite_difference_operator]{difference operators}. He denotes it by \( E \) and calls it simply a \enquote{shift operators} --- both conventions are unfortunately ambiguous for us.

  The usage for sequences is popular, but inconsistent across the literature, and serves mostly for illustrative examples.

  This is how \incite[exerc. 7.26]{FabianEtAl2001FunctionalAnalysis} define the \enquote{left shift operator} \( L \) and \enquote{right shift operator} in the \hyperref[def:lebesgue_space]{Lebesgue (sequence) space} \( l_2 \). \incite[example 1.3.3]{Хелемский2014ФункциональныйАнализ} uses the same terminology (\enquote{оператор сдвига} in Russian) in the case of \( l_p \), and denotes the operators by \( T_l \) and \( T_r \). \incite[exerc. 2.1.21]{FriedbergInselSpence2018LinearAlgebra} use the same terminology for general sequences over a field, but denotes them by \( T \) and \( U \).

  \incite[84]{Lang2002Algebra} denotes the right shift operator on integers by \( T \) and calls it \enquote{the} shift operator. \incite[347]{Rudin1987RealAndComplexAnalysis} denotes the right shift operator on complex numbers by \( S \) and also calls it \enquote{the} shift operator.

  \item Left and right shift operators are closely related to shifts of \hyperref[def:positional_number_system]{digit string}, defined in \fullref{def:digit_shift}.
\end{comments}

\begin{proposition}\label{thm:difference_operator_factorization}\mcite[77]{Stanley2012EnumerativeCombinatoricsVol1}
  For the \( k \)-th \hyperref[def:finite_difference_operator]{difference operator}, we have the following factorization:
  \begin{equation}\label{eq:thm:difference_operator_factorization}
    \Delta^k f(n) = \sum_{i=0}^k (-1)^{k-i} \cdot \binom k i \cdot f(n + i).
  \end{equation}
\end{proposition}
\begin{proof}
  In the \hyperref[def:endomorphism_semiring]{endomorphism ring} of \( \fun(\BbbZ, A) \), the operator \( (-\id) \) on \( \fun(\BbbZ, A) \) commutes with the \hyperref[def:sequence_shift_operator]{right shift operator} \( R \). Indeed,
  \begin{equation*}
    [R \bincirc (-\id)](f(n))
    =
    R(-f(n))
    =
    -f(n + 1)
    =
    [-\id](f(n + 1))
    =
    [(-\id) \bincirc R](f(n)).
  \end{equation*}

  Furthermore, we can express \( \Delta \) as \( R - \id \). Thus, \fullref{thm:binomial_theorem} gives us
  \begin{equation*}
    \Delta^k
    =
    (R - \id)^k
    =
    \sum_{i=0}^k \binom k i R^i (-\id)^{k-i}
    =
    \sum_{i=0}^k (-1)^{k-i} \binom k i R^i \id^{k-i}.
  \end{equation*}

  Then \eqref{eq:thm:difference_operator_factorization} follows directly.
\end{proof}

\begin{corollary}\label{thm:difference_operator_linearity}
  \hyperref[def:finite_difference_operator]{Difference operators} are \hyperref[def:linear_function]{linear}.
\end{corollary}
\begin{proof}
  Follows from \fullref{thm:difference_operator_factorization} since both shift operators and the identity are linear.
\end{proof}

\begin{proposition}\label{thm:successive_value_via_difference_operator}
  The \hyperref[def:sequence_shift_operator]{right shift operator} can be expressed via \hyperref[def:finite_difference_operator]{difference operators} as follows:
  \begin{equation}\label{eq:thm:successive_value_via_difference_operator}
    R^k f(n)
    =
    f(n + k)
    =
    \sum_{i=0}^k \binom k i \cdot \Delta^i f(n).
  \end{equation}
\end{proposition}
\begin{proof}
  Similarly to \fullref{thm:difference_operator_factorization}, we use \fullref{thm:binomial_theorem}:
  \begin{equation*}
    R^k
    =
    (\Delta + \id)^k
    =
    \sum_{i=0}^k \binom k i \Delta^i \id^{k-i}.
  \end{equation*}
\end{proof}

\paragraph{Difference equations}

\begin{definition}\label{def:difference_equation}\mimprovised
  A \term[ru=разностное уравнение (\cite[307]{Гельфонд1959ИсчислениеКонечныхРазностей})]{difference equation} of \term[ru=порядок (разностного уравнения) (\cite[307]{Гельфонд1959ИсчислениеКонечныхРазностей})]{order} \( k \) is an \hyperref[def:equation]{equation} whose \hyperref[con:indeterminate]{indeterminate} is an \hyperref[def:sequence]{infinite} or \hyperref[def:doubly_infinite_sequence]{doubly infinite sequence} and its first \( k \) \hyperref[def:finite_difference_operator]{finite differences}.

  More precisely, let \( T \) be either the \hyperref[def:natural_numbers]{semiring \( \BbbN \) of natural numbers} or the \hyperref[def:integers]{ring \( \BbbZ \) of integers}, and let \( R \) be any \hyperref[def:ring]{ring}. As in \hyperref[def:dynamical_system]{dynamical systems}, we interpret the set \( T \) as time.

  This allows us to characterize a difference equation as
  \begin{equation}\label{eq:def:difference_equation}
    F(N, \Delta^k X_N, \ldots, \Delta X_N, X_N) = 0
  \end{equation}
  for some function \( F: T \times R^{k+1} \to R \). Here \( F \) is a \hyperref[con:expression]{metalinguistic expression} and \( N \) and \( X \) are indeterminates; we use the capital letter convention from \fullref{rem:conventions_for_indeterminates} to make the exposition more precise.

  \begin{thmenum}
    \thmitem{def:difference_equation/solution} We say that the sequence \( x: T \to R \) is a \term{solution} to the equation \eqref{eq:def:difference_equation} if, for any moment \( t \), we have
    \begin{equation*}
      F(t, \Delta^k x_t, \ldots, \Delta x_t, x_t) = 0.
    \end{equation*}

    \thmitem{def:difference_equation/initial} Since \eqref{eq:def:difference_equation} in general has multiple solutions, we may introduce additional constants. If, for some moment \( t_0 \), we are given the \( k \) values \( x_{t_0}, \Delta x_{t_0}, \cdots, \Delta^{k-1} x_{t_0} \), we call them \term{initial values} and call the obtained \hyperref[def:equation/system]{system of equations} an \term{initial value problem}.

    \thmitem{def:difference_equation/autonomous} We say that the equation is \term{autonomous} if it does not depend on \( N \).
  \end{thmenum}
\end{definition}
\begin{comments}
  \item Our definition is a formalization of \cite[307]{Гельфонд1959ИсчислениеКонечныхРазностей}. A brief literature overview is given in \fullref{rem:recurrence_relations_and_difference_equations}.

  \item In practice, we try to rewrite the expression \( F \) so that \( \Delta^K X_N \) is on the left of the equation and everything else is on the right. See \fullref{ex:def:difference_equation} for examples.
\end{comments}

\begin{remark}\label{rem:recurrence_relations_and_difference_equations}
  Recurrence relations, which we define in \fullref{def:recurrence_relation}, and difference equations, which we define in \fullref{def:difference_equation} are often conflated in the literature. We show their equivalence in \fullref{thm:recurrence_relations_and_difference_equations}, but nonetheless treat them differently.

  What we call recurrence relations are defined slightly differently:
  \begin{itemize}
    \item \incite[3]{LevyLessman1961FiniteDifferenceEquations} defines an \enquote{ordinary difference equation of the \( r \)-th order} as a \enquote{relation of the form}
    \begin{equation*}
      y_{n+r} = F(n, y_n, y_{n+1}, \ldots, y_{n+r-1}).
    \end{equation*}

    Later, in \cite[75]{LevyLessman1961FiniteDifferenceEquations}, after discussing \hyperref[def:finite_difference_operator]{finite differences}, they redefine a difference equation as \enquote{a relation between the differences of a function at one or more general values of the independent variable.}

    Our definition of recurrence relations closely resembles theirs.

    \item \incite[395]{LidlNiederreiter1997FiniteFields} define a \enquote{linear recurrent sequence} in the \hyperref[def:finite_field]{finite field} \( \BbbF_q \) as a sequence satisfying
    \begin{equation*}
      x_{n+k} = a_{k-1} s_{n+k-1} + a_{k-2} s_{n+k-2} + \cdots + a_0 s_n + a \quad\T{for} n = 0, 1, 2, \ldots
    \end{equation*}

    We base our exposition of recurrences in finite fields in \fullref{sec:recurrence_relations} on their book, but also consider more general recurrent sequences, and emphasize the difference between recurrence relations and sequences satisfying them.

    \item \incite[21]{Юмагулов2015ДинамическиеСистемы} uses \enquote{рекуррентное уравнение} (\enquote{recurrent equation}) for the \hyperref[def:equation]{equation}
    \begin{equation*}
      x_{n+1} = f(x_n, x_{-1}, \ldots, x_{n-k}), \quad n = 0, 1, 2, \ldots
    \end{equation*}

    We also allow non-autonomous recurrences where \( f \) depends on \( n \).

    \item \incite[def. 2.4.4]{Rosen2019DiscreteMathematics} provides the following informal definition:
    \begin{displayquote}
      A recurrence relation for the sequence \( \seq{ x_n } \) is an equation that expresses \( a_n \) in terms of one or more of the previous terms of the sequence, namely, \( a_0, a_1, \ldots, a_{n-1} \), for all integers \( n \) with \( n \geq n_0 \), where \( n_0 \) is a nonnegative integer.
    \end{displayquote}
  \end{itemize}

  Out of the authors we cite, what we call a different equation is only discussed by \incite[307]{Гельфонд1959ИсчислениеКонечныхРазностей}, who considers the equation
  \begin{equation*}
    F(x, f(x), \Delta f(x), \ldots, \Delta^n f(x)) = 0.
  \end{equation*}
\end{remark}

\begin{proposition}\label{thm:recurrence_relations_and_difference_equations}
  Using \fullref{thm:successive_value_via_difference_operator}, we can express every indeterminate in a \hyperref[def:recurrence_relation]{recurrence relation} via \hyperref[def:finite_difference_operator]{difference operators}, thus turning it into an \hyperref[def:difference_equation]{difference equation}.

  Conversely, using \fullref{thm:difference_operator_factorization}, we can express every finite difference in an difference equation via consecutive elements of the sequence. If the largest power \( \Delta^k x_t \) is a function of the lower-order differences, we obtain a recurrence relation.
\end{proposition}
\begin{comments}
  \item Initial value problems in the sense of \fullref{def:recurrence_relation/initial} correspond to initial value problems in the sense of \fullref{def:difference_equation/initial}.

  \item \Fullref{ex:def:difference_equation/fibonacci} demonstrates how a recurrence relation transforms into a difference equation. \Fullref{ex:def:difference_equation/no_solutions} shows how the converse may fail.
\end{comments}

\begin{example}\label{ex:def:difference_equation}
  We list examples of \hyperref[def:difference_equation]{difference equations}:
  \begin{thmenum}
    \thmitem{ex:def:difference_equation/arithmetic} The \hyperref[def:arithmetic_progression]{arithmetic progression}
    \begin{equation*}
      X_{N+1} = X_N + d
    \end{equation*}
    can be turned into a constant difference equation by subtracting \( X_N \):
    \begin{equation*}
      \Delta X_N = d.
    \end{equation*}

    \thmitem{ex:def:difference_equation/geometric} Similarly, the \hyperref[def:geometric_progression]{geometric progression}
    \begin{equation*}
      X_{N+1} = X_N q
    \end{equation*}
    can be turned into the difference equation
    \begin{equation*}
      \Delta X_N = (q - 1) X_N.
    \end{equation*}

    This equation is discussed in greater detail in \fullref{def:discrete_malthusian_model}.

    Multiples of \( 2^n \) are thus solutions to \( \Delta X_n = X_n \). This is the discrete analogue of the exponential function \( e^x \) solving the differential equation \( D f(x) = f(x) \).

    \thmitem{ex:def:difference_equation/fibonacci} A slightly more substantial example is the \hyperref[def:fibonacci_numbers]{Fibonacci sequence} initial value problem
    \begin{empheq}[left=\empheqlbrace]{align*}
      &X_{N+2} = X_{N+1} + X_N, \\
      &X_0 = 0, \\
      &X_1 = 1.
    \end{empheq}

    \Fullref{thm:successive_value_via_difference_operator} implies that
    \begin{equation*}
      X_{N+1} = X_N + \Delta X_N
    \end{equation*}
    and
    \begin{equation*}
      X_{N+2} = \sum_{i=0}^2 \binom 2 i \cdot \Delta^i X_N = X_N + 2 \Delta X_N + \Delta^2 X_N.
    \end{equation*}

    Thus, the Fibonacci recurrence becomes
    \begin{equation*}
      \Delta^2 X_N + 2 \Delta X_N + X_N = \Delta X_N + X_N
    \end{equation*}
    which simplifies to
    \begin{equation}\label{eq:ex:def:difference_equation/fibonacci}
      \Delta^2 X_N = -\Delta X_N + X_N.
    \end{equation}

    This is easily verified by noting that, by \fullref{thm:difference_operator_factorization},
    \begin{equation*}
      \Delta^2 X_N = X_{N+2} - 2X_{N+1} + X_N
    \end{equation*}
    and
    \begin{equation*}
      -\Delta X_N + X_N = -(X_{N+1} - X_N) + X_N = -X_{N+1} + 2X_N,
    \end{equation*}
    hence
    \begin{equation*}
      X_{N+2} - 2X_{N+1} + X_N = -X_{N+1} + 2X_N,
    \end{equation*}
    which simplifies to the initial recurrence
    \begin{equation*}
      X_{N+2} = X_{N+1} + X_N.
    \end{equation*}

    We can also verify the transformation via any triple from the Fibonacci sequence. For example, taking \( F_4 = 3, F_5 = 5 \) and \( F_6 = 8 \), we have
    \begin{equation*}
      \Delta^2 F_4 = F_6 - 2F_5 + F_4 = 8 - 10 + 3 = 1
    \end{equation*}
    and
    \begin{equation*}
      -\Delta F_4 + F_4 = -(F_5 - F_4) + F_4 = 2F_4 - F_5 = 1.
    \end{equation*}

    For the \hyperref[def:lucas_numbers]{Lucas numbers} \( L_4 = 7 \), \( L_5 = 11 \) and \( L_6 = 18 \), we have
    \begin{equation*}
      \Delta^2 L_4 = 3 = -\Delta L_4 + L_4.
    \end{equation*}

    \thmitem{ex:def:difference_equation/non_autonomous} Consider the non-autonomous difference equation
    \begin{equation*}
      \Delta X_N = N.
    \end{equation*}

    It corresponds to the non-autonomous recurrence relation
    \begin{equation*}
      X_{N+1} = X_N + N.
    \end{equation*}

    As shown in \fullref{ex:def:recurrence_relation/triangular_numbers}, the \hyperref[def:triangular_number/arithmetic]{triangular numbers} are a solution with initial value \( x_0 = 0 \).

    \thmitem{ex:def:difference_equation/no_solutions} Over the real numbers, consider the difference equation
    \begin{equation*}
      \Delta X_N \cdot X_N = 1.
    \end{equation*}

    Assuming that \( x_0 \) is nonzero, we have
    \begin{equation*}
      x_1 = x_0 + \frac 1 {x_0},
    \end{equation*}
    and \( x_1 \) is thus also nonzero. This generalizes to any moment \( t \): we have
    \begin{equation*}
      x_{t+1} = x_t + \frac 1 {x_t}.
    \end{equation*}

    Thus, the equation has a solution if and only if \( x_0 \) is nonzero. In this case it is expressible as a recurrence relation.

    Since every recurrence relation initial value problem has a solution, we conclude that no recurrence corresponds to this difference equation when \( x_0 = 0 \). This provides a counterexample for \fullref{thm:recurrence_relations_and_difference_equations}.
  \end{thmenum}
\end{example}

\paragraph{Discrete Malthusian model}
