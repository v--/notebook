\section{Discrete dynamical systems}\label{sec:discrete_dynamical_systems}

\paragraph{General dynamical systems}

\begin{definition}\label{def:dynamical_system}\mimprovised
  Fix an object \( X \) in some \hyperref[def:concrete_category]{concrete category} and an \hyperref[con:additive_semigroup]{additive} \hyperref[def:monoid]{monoid} (resp. \hyperref[def:group]{group}) \( T \).

  A \term{dynamical system} with \term{phase space} \( X \) and \term{time system} \( T \) is a \hyperref[def:monoid_action]{monoid action} (resp. \hyperref[def:group_action]{group action}) \( \Phi: T \times X \to X \). We refer to \( \Phi \) itself as the \term{evolution function} of the system.
\end{definition}
\begin{comments}
  \item The purpose of this definition is to make concrete any discussion of dynamical systems. Formally, it is the same concept as that of a monoid or group action.
\end{comments}

\begin{example}\label{ex:def:dynamical_system}
  We list examples of \hyperref[def:dynamical_system]{dynamical systems}:
  \begin{thmenum}
    \thmitem{ex:def:dynamical_system/rotation} Let our phase space be the \hyperref[def:euclidean_plane]{Euclidean plane}. A dynamical system then corresponds to movement of points in the plane.

    For example, in \fullref{ex:def:group_action_orbit/rotation} we discussed that rotation is a group action on the plane, and hence we can view rotation as a dynamical system whose time is given by the \hyperref[def:circle_group]{circle group}.
  \end{thmenum}
\end{example}

\begin{definition}\label{def:dynamical_system_time_classification}\mimprovised
  We say that a \hyperref[def:dynamical_system]{dynamical system} has \term{discrete time} if \( T \) is a \hyperref[def:monoid/submodel]{submonoid} of the additive group of \hyperref[def:integers]{integers} \( \BbbZ \) and \term{continuous time} if \( T \) is a submonoid of the additive group of the \hyperref[def:real_numbers]{real numbers} \( \BbbR \).
\end{definition}
\begin{comments}
  \item We choose whether the system is a monoid action or a group action based on the context. Since subgroups are submonoids, the definition holds generally.
\end{comments}

\begin{proposition}\label{thm:discrete_dynamical_system}
  \hyperref[def:dynamical_system_time_classification]{Discrete-time} \hyperref[def:dynamical_system]{dynamical systems} are particularly simple.

  The evolution function \( \Phi: T \times X \to X \) is entirely determined by \( \Phi_1 \) in the following sense: for any integer \( n \), we have
  \begin{equation}\label{eq:thm:discrete_dynamical_system}
    \Phi_n(x) = \Phi_1^n(x).
  \end{equation}

  Therefore, the entire evolution function is determined by a single endofunction on \( X \).
\end{proposition}
\begin{proof}
  We will first show \eqref{eq:thm:discrete_dynamical_system} for nonnegative integers via \hyperref[con:induction/peano_arithmetic]{natural number induction}:
  \begin{itemize}
    \item For the base case, \eqref{eq:def:monoid_action/family/identity} implies that \( \Phi_0(x) = x = \Phi_1^0(x) \).

    \item Suppose that \eqref{eq:thm:discrete_dynamical_system} holds. Then
    \begin{equation*}
      \Phi_{n + 1}(x)
      \reloset {\eqref{eq:def:monoid_action/family/compatibility}} =
      \Phi_n(x) \bincirc \Phi_1(x)
      \reloset {\T{ind.}} =
      \Phi^n(x) \bincirc \Phi_1(x)
      =
      \Phi^{n+1}(x).
    \end{equation*}
  \end{itemize}

  For negative numbers (if the time system is a group) we can also use induction, with \eqref{eq:def:group_action/family/compatibility} instead of \eqref{eq:def:monoid_action/family/compatibility}, but with a different hypothesis: for every nonnegative integer \( n \), we have
  \begin{equation}\label{eq:thm:discrete_dynamical_system/nonpositive}
    \Phi_{-n}(x) = \Phi_1^{-n}(x)
  \end{equation}

  In order for \eqref{eq:thm:discrete_dynamical_system/nonpositive} to make sense, \( \Phi_1(x) \) must be invertible. This follows from \eqref{eq:def:group_action/family/compatibility} by noting that
  \begin{equation*}
    \id(x) = \Phi_0(x) = \Phi_1(x) \bincirc \Phi_{-1}(x),
  \end{equation*}
  implying that \( \Phi_1(x)^{-1} = \Phi_{-1}(x) \).
\end{proof}

\begin{definition}\label{def:dynamical_system_trajectory}\mimprovised
  Fix a \hyperref[def:dynamical_system]{dynamical system} with evolution function \( \Phi: T \times X \to X \).

  A \term{trajectory} starting at the \term{initial state} \( x_0 \in X \) is an \hyperref[def:indexed_family]{indexed family} \( \seq{ x_t }_{t \in T} \) obtained as
  \begin{equation*}
    x_t \coloneqq \Phi_t(x_0).
  \end{equation*}
\end{definition}
\begin{comments}
  \item This notation is consistent because \eqref{eq:def:monoid_action/family/identity} implies that \( \Phi_0(x_0) = x_0 \).
\end{comments}

\begin{proposition}\label{thm:def:dynamical_system_trajectory}
  \hyperref[def:dynamical_system_trajectory]{Dynamic system trajectories} have the following basic properties:
  \begin{thmenum}
    \thmitem{thm:def:dynamical_system_trajectory/composition} For a dynamical system with evolution function \( \Phi: T \times X \to X \), the trajectory of \( x_0 \) satisfies, for every \( t \in X \),
    \begin{equation*}
      x_{t+s} = \Phi_s(x_t).
    \end{equation*}

    The order of \( s \) and \( t \) is important unless \( T \) is commutative.

    \thmitem{thm:def:dynamical_system_trajectory/discrete} For a \hyperref[def:dynamical_system_time_classification]{discrete-time} dynamical system with evolution function \( \Phi: T \times X \to X \), the trajectory of \( x_0 \) is a sequence, perhaps two-sided, such that, for any integer \( n \) in \( T \), we have
    \begin{equation*}
      x_n = \Phi_1^n(x_0).
    \end{equation*}
  \end{thmenum}
\end{proposition}
\begin{proof}
  \SubProofOf{thm:def:dynamical_system_trajectory/composition} Follows from \eqref{eq:def:monoid_action/family/compatibility}.

  \SubProofOf{thm:def:dynamical_system_trajectory/discrete} Follows from \fullref{thm:def:dynamical_system_trajectory/composition} and \fullref{thm:discrete_dynamical_system}.
\end{proof}

\begin{definition}\label{def:autonomous_system}
  \todo{Autonomous systems}
\end{definition}
