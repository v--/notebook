\section{Point lattices}\label{sec:point_lattices}

\paragraph{Point configurations}

\begin{definition}\label{def:point_configuration}\mimprovised
  A \term{discrete} (resp. \term{finite}) \term{point configuration} in an \hyperref[def:euclidean_space]{Euclidean space} is simply a nonempty \hyperref[def:discrete_set]{discrete} (resp. finite) subset of the space.
\end{definition}
\begin{comments}
  \item Finite configurations are, of course, discrete.

  \item These concepts are not established. We give a definition because we want a formalism for \hyperref[def:triangular_point_configuration]{triangular}, \hyperref[def:hypercubic_point_configuration]{square} and other similar configurations, which are rarely discussed formally.

  \item \incite[405]{BoydVandenberghe2004ConvexOptimization} define a \enquote{configuration} as finite subset of an Euclidean space, in the context of convex optimization. Ileana Streinu in \cite[395]{Rosen2018DiscreteHandbook}, like us, uses the term \enquote{point configuration} instead, in the context of discrete geometry.
\end{comments}

\begin{proposition}\label{thm:point_configuration_finiteness}
  Fix a \hyperref[def:point_configuration]{discrete point configuration} \( C \) in the \hyperref[def:euclidean_space]{Euclidean space} \( \BbbR^n \).

  For every point \( P \) of \( \BbbR^n \), there exists at least one point in \( C \) of minimal distance to \( P \).
\end{proposition}
\begin{proof}
  Fix any point \( R \) in \( C \). Since \( C \) is discrete, the intersection \( C' \) of \( C \) with the ball \( B(0, \norm{P - R}) \) is finite.

  Then we can enumerate all points in \( C' \) to find their distance from \( P \) and choose any of the points of minimal distance.
\end{proof}

\paragraph{Point lattices}

\begin{definition}\label{def:point_lattice}\mimprovised
  A \term{point lattice} in an \hyperref[def:euclidean_space]{Euclidean space} is a \hyperref[def:discrete_set]{discrete} \hyperref[def:group/submodel]{subgroup} of its \hyperref[def:semiring]{additive group}.
\end{definition}
\begin{comments}
  \item What we call \enquote{point lattice} is generally called a \enquote{lattice}, which clashes with order-theoretic lattices defined in \fullref{def:lattice}. We make good use of order-theoretic lattices, so we prefer to be as unambiguous as possible, although ambiguity is still present. See the discussion in \fullref{rem:lattice_terminology}.

  \item This definition is stated as a characterization by \incite[thm. 21.2(ii)]{Gruber2007Geometry}. The aforementioned authors define lattices via explicit bases and proceed to study basis changes. We prefer a basis-free definition.

  \item Clearly every point lattice is a \hyperref[def:point_configuration]{discrete point configuration}.
\end{comments}

\begin{remark}\label{rem:lattice_terminology}
  There is an unfortunate terminological clash between order-theoretic lattices as defined in \fullref{def:lattice} and point lattices as defined in \fullref{def:point_lattice}. We, as well as most of the authors we cite, give preference to order-theoretic lattices.

  We use the term \enquote{point lattice} to disambiguate. \incite[356]{Gruber2007Geometry} suggests using \enquote{geometric} for possible disambiguation, while Marcel Berger refers to \enquote{plane lattices}. Both \enquote{geometric lattice} and \enquote{point lattice} are also concepts in lattice theory, and are discussed in \cite[80]{Birkhoff1967Lattices}, however we decided that \enquote{point lattice} is unambiguous enough.

  We list which authors prefer using \enquote{lattice} for one concept or for the other.
  \begin{itemize}
    \item Only Donald Knuth uses both, although in different books --- order-theoretic lattices in \cite[exerc. 2.3.2.19]{Knuth1997ArtVol2} and point lattices in \cite[97]{Knuth1997ArtVol2}.

    \item Explicit definitions of point lattices are given by
    \incite[\S I.2]{Cassels1971NumberGeometry},
    \incite[\S 9.14.29]{Berger1987GeometryI},
    \incite[28]{GrünbaumShephard1987Tilings},
    \incite[1]{Rogers1964Packing},
    \incite[356]{Gruber2007Geometry},
    \incite[376]{Винберг2014Алгебра} (as \enquote{решётка}) and
    \incite[155]{Шафаревич1999Алгебра} (as \enquote{решётка}).

    \item Explicit definitions for order-theoretic lattices are given by
    \incite[6]{Birkhoff1967Lattices},
    \incite[9]{Grätzer2011Lattices},
    \incite[53]{Roman2008Lattices},
    \incite[def. 6.14]{Harzheim2005OrderedSets},
    \incite[def. 2.4]{DaveyPriestley2002Lattices},
    \incite[40]{HalmosGivant2009BooleanAlgebras},
    \incite[3]{Johnstone1982StoneSpaces},
    \incite[80]{Kelley1975Topology},
    \incite[145]{Bourbaki2004Sets},
    \incite[174]{Riehl2016Categories},
    \incite[exerc. VI.2.1]{MacLane1998Categories},
    \incite[248]{Stanley2012EnumCombinatoricsVol1},
    \incite[192]{Carothers2000RealAnalysis},
    \incite[3]{HillePhillips1996FunctionalAnalysis},
    \incite[2]{Yoshida1980FunctionalAnalysis},
    \incite[210]{Rotman2015AlgebraPart1},
    \incite[def. 8.2]{Jacobson1985AlgebraI},
    \incite[171]{Bennett1995Geometry},
    \incite[227]{Cohn2013Measures} (only for lattices of vectors),
    \incite[def. 6.5.1]{Bobrowski2005FunctionalAnalysis} (only for lattices of continuous functions),
    \incite[174]{GelbaumOlmsted2003AnalysisCounterexamples} (also only for lattices of continuous functions),
    \incite[46]{CitkinMuravitsky2022ConsequenceRelations} (only for distributive lattices),
    \incite[10]{Barendregt1984LambdaCalculus} (only for complete lattices).

    In Russian, \enquote{решётка} (\enquote{lattice}) is mostly used, for example by
    \incite[127]{Гуров2013Решётки},
    \incite[129]{Мальцев1970ОбщаяАлгебра},
    \incite[79]{БогомоловСалий1997ОбщаяАлгебра},
    \incite[\S 2.1.2]{КусраевКутателадзе2005БулевозначныйАнализ},
    \incite[212]{БелоусовТкачёв2004ДискретнаяМатематика} and
    \incite[360]{КанторовичАкилов1984ФункАнализ},
    however \enquote{структура} (\enquote{structure}) is sometimes used in older literature, for example by
    \incite[16]{Владимиров1969БулевыАлгебры}
    \incite[58]{Ляпин1960Полугруппы} and
    \incite[178]{Курош1973ОбщаяАлгебра} (who doesn't even mention \enquote{решётка}).
  \end{itemize}
\end{remark}

\begin{definition}\label{def:point_sublattice}\mimprovised
  A \term{sublattice} of a \hyperref[def:point_lattice]{point lattice} is simply an additive subgroup.
\end{definition}
\begin{comments}
  \item We thus refer to \hyperref[def:group/generated]{generated subgroups} as \enquote{generated sublattices}. Other concepts also transfer naturally. Additive groups are closed under multiplication with integers, hence the generated sublattice of a set of points is simply the set of their linear combinations with integer coefficients.
\end{comments}

\begin{definition}\label{def:point_lattice_basis}\mcite[9]{Cassels1971NumberGeometry}
  We say that a \hyperref[def:hamel_basis]{Hamel basis} \( v_1, \ldots, v_n \) of \( \BbbR^n \) is a \term{basis} of the \hyperref[def:point_lattice]{point lattice} \( L \) if
  \begin{equation}\label{eq:def:point_lattice_basis}
    L = \set[\Big]{ \sum_{k=1}^n a_k v_k \given a_1, \ldots, a_n \T{are integers} }.
  \end{equation}

  \begin{figure}[!ht]
    \begin{subcaptionblock}{0.3\textwidth}
      \centering
      \includegraphics[page=1]{output/def__point_lattice_basis}
    \end{subcaptionblock}
    \hfill
    \begin{subcaptionblock}{0.3\textwidth}
      \centering
      \includegraphics[page=2]{output/def__point_lattice_basis}
    \end{subcaptionblock}
    \hfill
    \begin{subcaptionblock}{0.3\textwidth}
      \centering
      \includegraphics[page=3]{output/def__point_lattice_basis}
    \end{subcaptionblock}
    \caption{Different \hyperref[def:point_lattice_basis]{bases} for the same \hyperref[def:point_lattice]{point lattice}.}\label{fig:def:point_lattice_basis}
  \end{figure}
\end{definition}
\begin{comments}
  \item Every point lattice has a basis, as shown in \cite[thm. 21.2(ii)]{Gruber2007Geometry}. We will not rely on the existence of bases but rather provide them explicitly.
\end{comments}

\begin{definition}\label{def:primitive_lattice_vector}\mimprovised
  Fix a \hyperref[def:point_lattice]{point lattice} \( L \) and let \( v \) be a vector of \( L \). Define
  \begin{equation*}
    s \coloneqq \min\set{ t > 0 \given tu \in L }.
  \end{equation*}

  We call \( sv \) the \term{primitive vector} of \( v \). If \( s = 1 \), we call \( v \) itself \term{primitive}.
\end{definition}
\begin{comments}
  \item We have given a basis-independent definition, but bases can simplify determining primitive vectors --- see the characterization in \fullref{thm:primitive_lattice_vector_via_basis}.
  \item Primitive vectors as standalone objects are discussed in \incite[24]{Cassels1971NumberGeometry} and \incite[358]{Gruber2007Geometry}, however we wanted to highlight the construction of primitive vectors from existing ones.
\end{comments}

\begin{proposition}\label{thm:primitive_lattice_vector_via_basis}
  In a \hyperref[def:point_lattice]{point lattice} \( L \) with \hyperref[def:point_lattice_basis]{basis} \( v_1, \ldots, v_n \), the \hyperref[def:primitive_lattice_vector]{primitive vector} of
  \begin{equation*}
    w = \sum_{k=1}^n a_k v_k
  \end{equation*}
  is
  \begin{equation}\label{eq:thm:primitive_lattice_vector_via_basis}
    w' \coloneqq \frac w {\gcd(a_1, \ldots, a_n)}.
  \end{equation}
\end{proposition}
\begin{comments}
  \item In particular, \( w \) is primitive if and only if \( a_1, \ldots, a_n \) are \hyperref[def:coprime_elements]{coprime}.
\end{comments}
\begin{proof}
  Let
  \begin{equation*}
    g \coloneqq \gcd(a_1, \ldots, a_n).
  \end{equation*}

  We must show that \( 1 / g \) is the minimal positive real number such that \( w / g \) is in \( L \).

  First note that \( w / g \) is in \( L \) as a linear combination of \( v_1, \ldots, v_n \) with integer coefficients.

  Suppose that \( w / d \) is in \( L \) for some \( d \geq g \). Then \( a_k / d \) is an integer for any \( k = 1, \ldots, n \), hence \( d \) is a common divisor. But \( g \) is the greatest common divisor, thus \( d \) divides \( g \), and \fullref{thm:natural_number_divisibility_lattice} implies that \( d \leq g \).

  Since we have assumed that \( d \geq g \), it follows that \( d = g \).

  Then \( 1 / g \) is indeed the minimal positive real number such that \( w / g \) is in \( L \).
\end{proof}

\begin{definition}\label{def:minimal_lattice_vector}\mcite[42]{ConwaySloane1998SpherePackings}
  We define the \term{minimal norm} of a \hyperref[def:point_lattice]{point lattice} as the minimal norm of nonzero vectors in the lattice. We will refer to lattice vectors attaining this norm as \term{minimal vectors}.
\end{definition}
\begin{comments}
  \item A minimal vector is necessarily \hyperref[def:primitive_lattice_vector]{primitive}, but not vice versa.
\end{comments}

\begin{proposition}\label{thm:point_lattice_via_basis}
  A \hyperref[def:semimodule/generated]{spanning subset} \( L \) of the \hyperref[def:euclidean_space]{Euclidean space} \( \BbbR^n \) is a \hyperref[def:point_lattice]{point lattice} if and only if there exists a basis \( v_1, \ldots, v_n \) of \( \BbbR^n \) so that \eqref{eq:def:point_lattice_basis} holds.
\end{proposition}
\begin{comments}
  \item If \( L \) is not a spanning subset, we simply need to consider a lower-dimensional space.
\end{comments}
\begin{proof}
  \SufficiencySubProof\mcite{MathSE:point_lattice_basis_existence} We will use recursion on \( m \leq n \) to build a basis \( v_1, \ldots, v_m \) for the \hyperref[def:point_sublattice]{sublattice} \( L_m \coloneqq L \cap V_m \), where \( V_m \) is the linear span of \( v_1, \ldots, v_m \).

  The intersection of subgroups is a subgroup by \fullref{thm:intersection_substructure}, so \( L_m \) is indeed a discrete subgroup of \( L \). Furthermore, since \( L \) spans \( \BbbR^n \), \( L_m \) spans \( V_m \). Thus, the problem is well-posed, and we can begin.

  \SubProof*{Proof of base case} For the case \( m = 1 \), let \( v_1 \) be a \hyperref[def:minimal_lattice_vector]{minimal vector} (although choosing a \hyperref[def:primitive_lattice_vector]{primitive vector} works also).

  We can regard the vectors of \( L \cap V_m \) as real numbers. Every bounded neighborhood of \( 0 \) has only finitely many members of \( L_m \), and thus we can pick a minimal positive number \( t \) in \( L_m \).

  It is clear that integer multiples of \( t \) are in \( L_m \). We need to show the converse.

  Conversely, fix a point \( x \) in \( L_m \).
  \begin{itemize}
    \item If \( x = 0 \), it is clearly a multiple of \( t \).
    \item If \( x > 0 \), then \( x \geq t \) since otherwise \( t \) would not be minimal.

    Let \( m \) be the largest integer such that \( mt \leq x \). The case \( mt = x \) is clear, so suppose that \( mt < x \).

    Then \( x - mt > 0 \) is also in \( L_m \), thus \( x - mt \geq t \), implying \( x \geq (m + 1)t \). But this contradicts the maximality of \( m \).

    Therefore, \( x \) is necessarily an integer multiple of \( t \).

    \item If \( x < 0 \), then \( -x > 0 \) is an integer multiple of \( t \), and thus so is \( x \) itself.
  \end{itemize}

  Therefore, as desired,
  \begin{equation*}
    L \cap \linspan{ v_m } = L_m = \set{ at \given a \in \BbbZ }.
  \end{equation*}

  \SubProof*{Proof of inductive step} Suppose that \( v_1, \ldots, v_{m-1} \) is a basis of the lattice \( L_{m-1} \).

  Every bounded neighborhood of \( V_{m-1} \) has only finitely many members of \( L \), and thus, by \fullref{thm:point_configuration_finiteness}, there exists a member \( v_m \) of \( L \setminus V_{m-1} \) that minimizes its distance to \( V_{m-1} \).

  It this point \( V_m = \linspan{ v_1, \ldots, v_m } \) and \( L_m = V_m \cap L \) are defined.

  Clearly \( v_1, \ldots, v_{m-1}, v_m \) is a basis of \( V_m \). It remains to show that it is also a basis of \( L \).

  Since \( L_m \) is a subgroup of \( L \), integer multiples of each of the basis vectors belong to \( L_m \), and thus so do sums of integer multiples, i.e. integer linear combinations. We need to show the converse.

  Fix a vector
  \begin{equation*}
    x = \sum_{k=1}^m a_k v_k
  \end{equation*}
  from \( L_m \). We will show that \( a_1, \ldots, a_m \) are integers.

  If \( a_m = 0 \), then \( x \) itself is in \( L_{m-1} \), and we are done. Suppose that \( a_m \neq 0 \).

  In order to use the inductive hypothesis, we need to show that \( a_m \) is an integer\fnote{Even if \( v_m \) is a \hyperref[def:primitive_lattice_vector]{primitive vector}, it may happen that \( a_m \) is not an integer. See \fullref{ex:primitive_lattice_vectors_not_basis} for a counterexample. It is important to choose \( v_m \) carefully, as we do here.}.

  Suppose that \( a_m \) is not an integer. We have chosen \( v_m \) so that it minimizes the distance to \( V_{m-1} \) among vectors in \( L \setminus V_{m-1} \). Let \( a_m' \) be the floor of \( a_m \) if \( a_m > 0 \) and the ceiling of \( a_m \) otherwise. Then \( 0 < \abs{a_m - a_m'} < 1 \).

  We have
  \begin{balign*}
    \op{dist}\parens[\Big]{ (a_m - a_m') v_m, V_{m-1} }
    &=
    \inf_{y \in V_{m-1}} \norm{ (a_m - a_m') v_m - y }
    = \\ &=
    \abs{a_m - a_m'} \inf_{y \in V_{m-1}} \norm[\Big]{ v_m - \frac y {a_m - a_m'} }
    = \\ &=
    \abs{a_m - a_m'} \inf_{y \in V_{m-1}} \norm{v_m - y}
    = \\ &=
    \abs{a_m - a_m'} \op{dist}(v_m, V_{m-1})
    < \\ &<
    \op{dist}(v_m, V_{m-1}).
  \end{balign*}

  But we have chosen \( v_m \) so that it minimizes the distance to \( V_{m-1} \) among vectors in \( L \setminus V_{m-1} \); hence, we have obtained a contradiction.

  It remains for \( a_m \) to be an integer. Then \( a_m v_m \) is in \( L_m \), and hence so is \( x - a_m v_m \). But \( x - a_m v_m \) is in \( V_{m-1} \), thus we can use the inductive hypothesis to conclude that \( a_1, \ldots, a_{m-1} \) are integer.

  Therefore, \( a_1, \ldots, a_m \) are integers. Generalizing on \( x \), we conclude that every member of \( L_m \) is a linear combination of \( v_1, \ldots, v_m \) with integer coefficients.

  This concludes the proof of the inductive step.

  \NecessitySubProof Suppose that \eqref{eq:def:point_lattice_basis} holds for a subset \( L \) of \( \BbbR^n \) and a basis \( v_1, \ldots, v_n \). We must show that \( L \) is a point lattice.

  \begin{itemize}
    \item By assumption \( v_1, \ldots, v_n \) is a basis of \( \BbbR^n \), so \( L \) spans \( \BbbR^n \).
    \item The sum and difference of vectors with integer coordinates is also a vector with integer coordinates, hence \( L \) is a subgroup under addition.
    \item Additionally, \( L \) is discrete because every point is isolated in a ball whose radius is the \hyperref[def:minimal_lattice_vector]{minimal norm} of \( L \).
  \end{itemize}

  This concludes the proof.
\end{proof}

\paragraph{Point lattice determinants}

\begin{proposition}\label{thm:point_lattice_determinant}
  The \hyperref[con:change_of_basis]{change of basis} matrices of a \hyperref[def:point_lattice]{point lattice} all have \hyperref[def:matrix_determinant]{determinant} \( 1 \) or \( -1 \).
\end{proposition}
\begin{proof}
  The product of a change of basis matrix and its inverse has determinant \( 1 \). Two bases of a point lattice have integer coordinates in each other, so the determinant is also an integer. It can only be \( 1 \) or \( -1 \).
\end{proof}

\begin{definition}\label{def:point_lattice_determinant}\mcite[10]{Cassels1971NumberGeometry}
  We define the \term{determinant} of a \hyperref[def:point_lattice]{point lattice} \( L \) as
  \begin{equation}\label{eq:def:point_lattice_determinant}
    \det L \coloneqq \abs{ \det(v_1, \cdots, v_n) },
  \end{equation}
  where \( v_1, \ldots, v_n \) is any \hyperref[def:point_lattice_basis]{basis} of \( L \).
\end{definition}
\begin{comments}
  \item \incite[5]{ConwaySloane1998SpherePackings} instead define the determinant by squaring instead of taking the absolute value.
\end{comments}
\begin{defproof}
  \Fullref{thm:point_lattice_determinant} implies that the determinant of different bases only changes its sign, so by taking the absolute value we make \eqref{eq:def:point_lattice_determinant} invariant under change of bases.
\end{defproof}

\begin{definition}\label{def:fundamental_paralellotope}\mcite[4]{ConwaySloane1998SpherePackings}
  A \term{fundamental parallelotope} in a \hyperref[def:point_lattice]{point lattice} is a \hyperref[def:parallelotope]{parallelotope} generated by a \hyperref[def:point_lattice_basis]{basis} of the lattice.

  We adopt the terminology from \fullref{def:parallelotope_terminology} based on the dimension.
\end{definition}
\begin{comments}
  \item \incite[69]{Cassels1971NumberGeometry} uses the term \enquote{fundamental parallelepiped} for arbitrary dimensions.
\end{comments}

\begin{proposition}\label{thm:fundamental_paralellotope_measure}
  The \hyperref[def:lebesgue_measure]{Lebesgue measure} of a \hyperref[def:fundamental_paralellotope]{fundamental parallelotope} coincides with the \hyperref[def:point_lattice_determinant]{lattice determinant}.
\end{proposition}
\begin{proof}
  Follows from \fullref{thm:volume_of_parallelotope}.
\end{proof}

\begin{definition}\label{def:fundamental_simplex}\mcite[4]{ConwaySloane1998SpherePackings}
  A \term{fundamental simplex} in a \hyperref[def:point_lattice]{point lattice} is a \hyperref[def:standard_simplex]{standard simplex} generated by a \hyperref[def:point_lattice_basis]{basis} of the lattice.

  We adopt the terminology from \fullref{def:simplex_terminology} based on the dimension.
\end{definition}
\begin{comments}
  \item \incite[69]{Cassels1971NumberGeometry} uses the term \enquote{fundamental parallelepiped} for arbitrary dimensions.
\end{comments}

\begin{proposition}\label{thm:fundamental_simplex_measure}
  The \hyperref[def:lebesgue_measure]{Lebesgue measure} of a \hyperref[def:fundamental_simplex]{fundamental simplex} is half of the \hyperref[def:point_lattice_determinant]{lattice determinant}.
\end{proposition}
\begin{proof}
  Follows from \fullref{thm:fundamental_paralellotope_measure}.
\end{proof}

\begin{definition}\label{def:euclidean_space_grid}\mcite[303]{Cassels1971NumberGeometry}
  A \term{grid} in the \hyperref[def:euclidean_space]{Euclidean space} \( \BbbR^n \) is a \hyperref[def:rigid_motion/translation]{translation} of a \hyperref[def:point_lattice]{point lattice}.
\end{definition}
\begin{comments}
  \item The term \enquote{grid} is for the most part used informally in the literature. This definition by Cassels seems to encompass our desired meaning.

  \item Grids whose origin is the zero vector are simply point lattices, for which reason Cassels also suggests calling grids \enquote{inhomogeneous lattices}. The latter term is also used by \incite[396]{Gruber2007Geometry}.
\end{comments}

\paragraph{Hypercube lattices}

\begin{definition}\label{def:hypercubic_point_lattice}
  We say that a \hyperref[thm:span_via_linear_combinations]{spanning} \hyperref[def:point_lattice]{point lattice} in the \hyperref[def:euclidean_plane]{Euclidean space} \( \BbbR^d \) is \term{hypercubic} of dimension \( d \) if it has a \hyperref[def:point_lattice_basis]{basis} of \( d \) pairwise \hyperref[def:orthogonality]{orthogonal} \hyperref[def:minimal_lattice_vector]{minimal vectors}.

  Based on \fullref{def:parallelotope_terminology}, we call hypercube lattices \enquote{square} in two dimensions and \enquote{cubical} in three dimensions.
\end{definition}

\begin{proposition}\label{thm:hypercubic_point_sublattice}
  Let \( v_1, \ldots, v_d \) be an orthogonal basis of minimal vectors a \hyperref[def:hypercubic_point_lattice]{hypercubic point lattice}. Then the sublattice generated by any subset of \( v_{i_1}, \ldots, v_{i_n} \) is an \( n \)-dimensional hypercubic lattice.
\end{proposition}
\begin{proof}
  Trivial.
\end{proof}

\begin{proposition}\label{thm:hypercubic_point_lattice_vector_rotation}
  In the \hyperref[def:hypercubic_point_lattice]{square point lattice}, a rotation of a lattice vector by a multiple of \( \pi / 2 \) is also lattice vector.
\end{proposition}
\begin{proof}
  By definition, there exists a basis \( \set{ u, v } \) of orthogonal minimal vectors. Without loss of generality, suppose that \( \measuredangle(u, v) = \pi / 2 \) (if \( \measuredangle(u, v) = 3\pi / 2 \), \fullref{thm:sum_of_angles_measure} implies that \( \measuredangle(v, u) = \pi / 2 \), and we can swap them).

  The rotation of \( v \) by \( \pi / 2 \) is \( -u \). Indeed, \fullref{thm:sum_of_angles_measure} implies that the rotation is forms a straight angle with \( u \), which by \fullref{thm:straight_iff_opposite_rays} is an oppositely directed vector to \( u \).

  Similarly, we conclude that the rotation of \( -u \) by \( \pi / 2 \) is \( -v \), and that the rotation of \( -v \) is \( u \).

  It remains to note that
  \begin{equation*}
    au + bv = bv - a(-u) = -a(-u) - b(-v) = -b(-v) + au.
  \end{equation*}
\end{proof}

\begin{proposition}\label{thm:hypercubic_point_lattice_minimal_vectors}
  There are \( 2d \) \hyperref[def:minimal_lattice_vector]{minimal vectors} in a \( d \)-dimensional \hyperref[def:hypercubic_point_lattice]{hypercubic point lattice}, and every two of them are either additive inverses or are orthogonal.
\end{proposition}
\begin{proof}
  Let \( v_1, \ldots, v_d \) be an orthogonal basis of minimal vectors of \( L \). Then \( -v_1, \ldots, -v_d \) are also minimal and pairwise orthogonal, hence \( \pm v_1, \ldots, \pm v_d \) are \( 2d \) minimal vectors of \( L \) satisfying the proposition.

  Fix a minimal vector
  \begin{equation*}
    w = \sum_{k=1}^d a_k v_k.
  \end{equation*}

  Then
  \begin{equation*}
    \norm{w}^2 = \inprod w w = \sum_{i=1}^d \sum_{j=1}^d a_i a_j \inprod {v_i} {v_j}.
  \end{equation*}

  Since the basis vectors are orthogonal,
  \begin{equation*}
    \norm{w}^2 = \sum_{k=1}^d a_k^2 \norm{v_k}^2.
  \end{equation*}

  We can divide by the minimal norm to conclude that
  \begin{equation*}
    \sum_{k=1}^d a_k^2 = 1.
  \end{equation*}

  The coefficients \( a_1, \ldots, a_d \) are integers, so in order to sum to \( 1 \), exactly one of them must be nonzero. If \( a_k \) is the nonzero coefficient, then \( w = v_k \) if \( a_k = 1 \) and \( w = -v_k \) otherwise.

  This concludes the proof.
\end{proof}

\begin{corollary}\label{thm:hypercubic_point_lattice_minimal_basis}
  In a \( d \)-dimensional \hyperref[def:hypercubic_point_lattice]{hypercubic point lattice}, every \( d \)-tuple of linearly independent \hyperref[def:minimal_lattice_vector]{minimal vectors} is a \hyperref[def:point_lattice_basis]{lattice basis}.
\end{corollary}
\begin{comments}
  \item We have an analogous statement for hexagonal lattices --- see \fullref{thm:hexagonal_point_lattice_minimal_basis}.
\end{comments}
\begin{proof}
  Let \( v_1, \ldots, v_d \) be an orthogonal basis of minimal vectors of the hypercubic lattice \( L \).

  Let \( u_1, \ldots, u_d \) be linearly independent minimal vectors. We will use induction on \( d \) to show that \( u_1, \ldots, u_d \) is a basis of \( L \).

  The base case \( d = 0 \) is vacuous, so suppose that the statement holds for \( (d - 1) \)-dimensional hypercubic lattices. Denote by \( L' \) the sublattice generated by \( v_1, \ldots, v_{d-1} \). \Fullref{thm:hypercubic_point_sublattice} implies that \( L' \) is hypercubic, so the inductive hypothesis holds for \( L' \).

  Then there exists some sublist of \( d - 1 \) elements of \( u_1, \ldots, u_d \) that form a basis of \( L' \). Without loss of generality, suppose that this is \( u_1, \ldots, u_{d-1} \).

  \Fullref{thm:hypercubic_point_lattice_minimal_vectors} implies that either \( u_d = \pm v_d \) or they are orthogonal. If they are orthogonal, they are linearly independent, and thus \( u_d \) belongs to \( L' \). But that would imply that it is linearly dependent with \( u_1, \ldots, u_{d-1} \), contradicting our initial assumption.

  Then \( u_d = \pm v_d \), implying that \( u_1, \ldots, u_d \) is a basis of \( L \).
\end{proof}

\begin{definition}\label{def:integer_point_lattice}\mimprovised
  The \hyperref[def:free_semimodule]{free module} \( \BbbZ^n \) can be regarded as a \hyperref[def:point_lattice]{point lattice} in \( \BbbR^n \) generated by the \hyperref[def:sequence_space]{standard basis}. This motivates calling \( \BbbZ^n \) the \term{integer point lattice}.

  \begin{figure}[!ht]
    \begin{subcaptionblock}{0.3\textwidth}
      \centering
      \includegraphics[page=1]{output/def__integer_point_lattice}
    \end{subcaptionblock}
    \hfill
    \begin{subcaptionblock}{0.3\textwidth}
      \centering
      \includegraphics[page=2]{output/def__integer_point_lattice}
    \end{subcaptionblock}
    \hfill
    \begin{subcaptionblock}{0.3\textwidth}
      \centering
      \includegraphics[page=3]{output/def__integer_point_lattice}
    \end{subcaptionblock}
    \caption{Two \hyperref[def:point_lattice_basis]{bases} of the \hyperref[def:integer_point_lattice]{integer point lattice} \( \BbbZ^2 \).}\label{fig:def:integer_point_lattice}
  \end{figure}
\end{definition}
\begin{comments}
  \item Integer lattices are the simplest \hyperref[def:hypercubic_point_lattice]{hypercube lattices}.

  \item Since we make good use order-theoretic lattices, which \( \BbbZ^n \) can be considered an instance of, we prefer the term \enquote{integer point lattice} to the simpler and more established \enquote{integer lattice} in order to emphasize which kinds of lattice we are talking about.

  \item The term \enquote{integer lattice} for \( \BbbZ^n \) is often regarded as a whole, without discussion of general point lattices. Such usage can be found in
  \begin{itemize}
    \item \incite[296]{Aluffi2009Algebra} in the context of \hyperref[def:polytope]{polytopes}.
    \item \incite[77]{Hatcher2002AlgTopology} in the context of \hyperref[def:group_action]{group actions}.
    \item Patrick Jaillet in \cite[503]{Rosen2018DiscreteHandbook} in the context of \hyperref[def:gaussian_integers]{Gaussian integers}.
    \item \incite[92]{ГашковЧубариков2005Сложность} (as \enquote{целочисленная решётка}) in the context of number geometry (which applies Euclidean geometry to number theory).
    \item Boltyanskii and Yaglom in \cite[256]{АлександровМаркушевичХинчин1966ЭнциклопедияТом5} (as \enquote{целочисленная решётка}) in the context of \hyperref[def:euclidean_plane]{plane geometry}.
    \item \incite[348]{Зыков2004Графы} (as \enquote{целочисленная решётка}) in the context of \hyperref[def:graph_embeddings]{graph embeddings}.
  \end{itemize}

  Authors who discuss general point lattices and also explicitly use the term \enquote{integer lattice} include
  \begin{itemize}
    \item \incite[106]{ConwaySloane1998SpherePackings} in the context of \hyperref[def:topological_space_packing]{packing}.
    \item \incite[322]{Cassels1971NumberGeometry} in the context of number geometry.
    \item \incite[310]{Gruber2007Geometry} in the context of \hyperref[def:polytope]{polytopes}.
    \item \incite[fig. 3.8.5]{GrünbaumShephard1987Tilings} in the context of \hyperref[def:topological_space_tiling]{tilings}.
  \end{itemize}
\end{comments}

\paragraph{Hexagonal point lattices}

\begin{definition}\label{def:hexagonal_point_lattice}\mimprovised
  We say that a \hyperref[def:point_lattice]{point lattice} in the \hyperref[def:euclidean_plane]{Euclidean plane} is \term{hexagonal} if it has a \hyperref[def:point_lattice_basis]{basis} of \hyperref[def:minimal_lattice_vector]{minimal vectors} \( u \) and \( v \) such that the \hyperref[def:angle/measure]{angle measure} \( \angle(u, v) \) is \( \pi / 3 \).

  \begin{figure}[!ht]
    \begin{subcaptionblock}{0.5\textwidth}
      \centering
      \includegraphics[page=1]{output/def__hexagonal_point_lattice}
    \end{subcaptionblock}
    \begin{subcaptionblock}{0.5\textwidth}
      \centering
      \includegraphics[page=2]{output/def__hexagonal_point_lattice}
    \end{subcaptionblock}
    \caption{The \hyperref[def:hexagonal_point_lattice]{standard hexagonal point lattice} and the corresponding \hyperref[def:hexagonal_tiling]{tiling}.}\label{fig:def:hexagonal_point_lattice}
  \end{figure}

  We will refer to the hexagonal lattice with basis \( u = (1, 0) \) and \( v = (1, \sqrt 3) / 2 \) as \term{standard}.
\end{definition}
\begin{comments}
  \item For this definition we generalize \enquote{the} hexagonal lattice discussed by \incite[\S 6.2]{ConwaySloane1998SpherePackings}.

  \item We will prove in \fullref{thm:hexagonal_point_lattice_minimal_basis} that every pair of linearly independent minimal vectors in a hexagonal point lattice forms a basis. We can thus, after the proof of the corollary, weaken the assumption
  \begin{displayquote}
    Let \( \set{ u, v } \) be a basis of minimal vectors such that \( \measuredangle(u, v) = \pi / 3 \)
  \end{displayquote}
  to
  \begin{displayquote}
    Let \( u \) and \( v \) be minimal vectors such that \( \measuredangle(u, v) = \pi / 3 \).
  \end{displayquote}

  \item Hexagonal point lattices have several remarkable properties:
  \begin{itemize}
    \item As shown in \fullref{thm:hexagonal_point_lattice_voronoi_cell}, their \hyperref[def:voronoi_tiling]{Voronoi tiling} consists of \hyperref[def:regular_polygon]{regular hexagons}.
    \item As shown in \fullref{thm:minimal_disk_packing_density}, they \hyperref[def:lattice_ball_packing]{pack balls} of fixed radius most efficiently.
  \end{itemize}
\end{comments}

\begin{example}\label{ex:primitive_lattice_vectors_not_basis}
  Fix a \hyperref[def:hexagonal_point_lattice]{hexagonal point lattice} \( L \) and let \( \set{ u, v } \) be a basis of minimal vectors such that \( \measuredangle(u, v) = \pi / 3 \).

  \Fullref{thm:primitive_lattice_vector_via_basis} implies that the vectors \( u + v \) and \( u - v \) are \hyperref[def:primitive_lattice_vector]{primitive}. They generate a \hyperref[def:point_sublattice]{sublattice}, which does not coincide with \( L \) itself, because their \hyperref[def:fundamental_parallelotope]{fundamental parallelepiped} contains the point \( v \) in its interior. Then \( \set{ u + v, u - v } \) is not a basis of \( L \).

  This is illustrated in \fullref{fig:ex:primitive_lattice_vectors_not_basis/primitive}.

  The same figure also illustrates why, in our proof of the inductive step in \fullref{thm:point_lattice_via_basis}, the choice of vector is important. If, in the base step, we choose the first basis vector to be \( u + v \), we must choose either \( u \) or \( v \) in the inductive step since they minimize the distance to the subspace spanned by \( u + v \). Even though, by \fullref{thm:v_minus_u_as_rotation}, the vector \( u - v \) is minimal, it does not minimize this distance, and \( \set{ u + v, u - v } \) is not a basis of \( L \).

  \begin{figure}[!ht]
    \begin{subcaptionblock}{0.45\textwidth}
      \centering
      \includegraphics[page=1]{output/ex__primitive_lattice_vectors_not_basis}
      \caption{The hexagonal point lattice with basis \( \set{ u, v } \).}\label{fig:ex:primitive_lattice_vectors_not_basis/basis}
    \end{subcaptionblock}
    \hfill
    \begin{subcaptionblock}{0.45\textwidth}
      \centering
      \includegraphics[page=2]{output/ex__primitive_lattice_vectors_not_basis}
      \caption{The sublattice generated by the primitive vectors \( u + v \) and \( u - v \).}\label{fig:ex:primitive_lattice_vectors_not_basis/primitive}
    \end{subcaptionblock}
    \caption{An illustration of \hyperref[def:primitive_lattice_vector]{primitive vectors} not forming a basis in \fullref{ex:primitive_lattice_vectors_not_basis}.}\label{fig:ex:primitive_lattice_vectors_not_basis}
  \end{figure}

  Thus, it is possible for primitive linearly independent vectors to not form a basis. Even minimal vectors may fail to form a basis. \incite*{ConwaySloane1995MinimalVectorBases} give an example of a particular lattice in \( \BbbR^{11} \) where no \( 11 \) \hyperref[def:minimal_lattice_vector]{minimal vectors} form a basis.

  In hexagonal lattices, \fullref{thm:hexagonal_point_lattice_minimal_basis} ensures that linearly independent minimal vectors always form a basis.
\end{example}

\begin{lemma}\label{thm:v_minus_u_as_rotation}
  For equinormed vectors \( u \) and \( v \) in the \hyperref[def:euclidean_plane]{Euclidean plane}, if \( v \) is a rotation of \( u \) by \( \pi / 3 \) radians, then \( v - u \) is a rotation of \( v \) by \( \pi / 3 \) radians.
\end{lemma}
\begin{proof}
  \SubProof{Proof that \( \norm{v - u} = \norm{u} \)} \Fullref{thm:law_of_cosines_for_vectors} implies that
  \begin{equation*}
    \norm{v - u}^2 = \norm{v}^2 + \norm{u}^2 - 2 \cdot \norm{u} \cdot \norm{v} \cdot \cos\measuredangle(u, v).
  \end{equation*}

  From \eqref{eq:thm:trigonometric_function_secondary_roots/third} it follows that \( \cos\measuredangle(u, v) = 1 / 2 \), hence
  \begin{equation*}
    \norm{v - u}^2 = \norm{v}^2 + \norm{u}^2 - \norm{u} \cdot \norm{v}.
  \end{equation*}

  Since \( u \) and \( v \) have the same norm, we can thus conclude
  \begin{equation}\label{eq:thm:v_minus_u_as_rotation/v_minus_u_norm}
    \norm{v - u} = \norm{u}.
  \end{equation}

  \SubProof{Proof that \( \cos \measuredangle(v, v - u) = 1 / 2 \)} \fullref{thm:cosine_of_angle_measure} implies
  \begin{equation}\label{eq:thm:v_minus_u_as_rotation/proof/cos_v_and_minus_u/main}
    \cos \measuredangle(v, v - u)
    =
    \frac {\inprod v {v - u}} {\norm v \cdot \norm {v - u}}
    =
    \frac {\inprod v v} {\norm v \cdot \norm v}
    -
    \frac {\inprod v u} {\norm v \cdot \norm u}
    =
    1 - \cos\measuredangle(v, -u).
  \end{equation}

  Furthermore,
  \begin{multline*}
    \cos \measuredangle(v, -u)
    \reloset {\eqref{eq:thm:angle_of_opposite_ray/cos}} =
    -\cos \measuredangle(v, u)
    \reloset {\eqref{eq:thm:angle_measure_swap}} =
    -\cos(2\pi - \measuredangle(u, v))
    \reloset {\ref{thm:trigonometric_function_period}} = \\ =
    -\cos(-\measuredangle(u, v))
    \reloset {\ref{thm:def:trigonometric_function/parity}} =
    -\cos(\measuredangle(u, v)).
  \end{multline*}

  Then
  \begin{equation*}
    \cos \measuredangle(v, v - u)
    \reloset {\eqref{eq:thm:v_minus_u_as_rotation/proof/cos_v_and_minus_u/main}} =
    1 - \cos \measuredangle(v, -u)
    =
    1 + \cos\parens[\Big]{ \frac {2\pi} 3 }
    \reloset {\eqref{eq:thm:trigonometric_function_secondary_roots/two_thirds}}
    1 - \frac 1 2,
  \end{equation*}
  hence
  \begin{equation}\label{eq:thm:v_minus_u_as_rotation/v_and_v_minus_u_cosine}
    \cos \measuredangle(v, v - u) = \frac 1 2.
  \end{equation}

  \SubProof{Proof that \( \measuredangle(v, v - u) = \pi / 3 \)} Based on \eqref{eq:thm:v_minus_u_as_rotation/v_and_v_minus_u_cosine}, \fullref{thm:trigonometric_angle_equations} implies that the possible values for the angle measure \( \measuredangle(v, v - u) \) are \( \pi / 3 \) and \( 5\pi / 3 \).

  Similarly to \eqref{eq:thm:v_minus_u_as_rotation/proof/cos_v_and_minus_u/main}, we can derive
  \begin{equation*}
    \cos \measuredangle(u, v - u)
    =
    \cos \measuredangle(u, v) - 1
    =
    -\frac 1 2,
  \end{equation*}
  hence the possible values for \( \measuredangle(u, v - u) \) are \( 2\pi / 3 \) and \( 4\pi / 3 \).

  \Fullref{thm:sum_of_angles_measure} implies that
  \begin{equation*}
    \measuredangle(u, v - u) = \measuredangle(u, v) + \measuredangle(v, v - u) \pmod {2\pi}.
  \end{equation*}

  If we divide by \( \pi / 3 \), we obtain the equation
  \begin{equation}\label{eq:thm:v_minus_u_as_rotation/proof/cos_v_and_minus_u/equation}
    \underbrace{\frac 3 \pi \cdot \measuredangle(u, v - u)}_a - \underbrace{\frac 3 \pi \cdot \measuredangle(v, v - u)}_b = 1 \pmod 6,
  \end{equation}
  where \( a \) is \( 2 \) or \( 4 \) and \( b \) is \( 1 \) or \( 5 \).

  \begin{table}[!ht]
    \begin{equation*}
      \begin{array}{*{3}{l}}
        \toprule
        a  & b  & a - b \pmod 6 \\
        \midrule
        2  & 1  & 1            \\
        2  & 5  & -3 = 3       \\
        4  & 1  & 3            \\
        4  & 5  & -1 = 5       \\
        \bottomrule
      \end{array}
    \end{equation*}
    \caption{The possible solution of the equation \eqref{eq:thm:v_minus_u_as_rotation/proof/cos_v_and_minus_u/equation} in our proof of \fullref{thm:v_minus_u_as_rotation}}\label{tab:thm:v_minus_u_as_rotation/proof/cos_v_and_minus_u}
  \end{table}

  We show in \cref{tab:thm:v_minus_u_as_rotation/proof/cos_v_and_minus_u} that the only possible solution is \( a = 2 \) and \( b = 1 \). Therefore,
  \begin{equation*}
    \measuredangle(v, v - u) = \frac \pi 3,
  \end{equation*}
  that is, \( v - u \) is a rotation of \( v \) by \( \pi / 3 \) radians.
\end{proof}

\begin{proposition}\label{thm:hexagonal_point_lattice_vector_rotation}
  In a \hyperref[def:hexagonal_point_lattice]{hexagonal point lattice}, a rotation of a lattice vector by a multiple of \( \pi / 3 \) is also lattice vector.
\end{proposition}
\begin{proof}
  By definition, there exists a basis \( \set{ u, v } \) of minimal vectors where \( \measuredangle(u, v) = \pi / 3 \).

  Let \( x = au + bv \) be any vector in \( L \). Denote by \( y_m \) its rotation by \( m\pi / 3 \), and by \( w_m \) the rotation of \( u \).

  We will first show by two-step induction on \( m \) that \( w_m \) is a minimal vector of \( L \).
  \begin{itemize}
    \item The base cases \( w_0 = u \) and \( w_1 = v \) are clear.
    \item Suppose that \( w_{m-2} \) and \( w_{m-1} \) are minimal vectors.

    \fullref{thm:v_minus_u_as_rotation} implies that
    \begin{equation*}
      w_m = w_{m-1} - w_{m-2},
    \end{equation*}
    hence \( w_m \) is in \( L \).

    It is a minimal vector because, as a rotation of \( w_{m-1} \), it has the same norm.
  \end{itemize}

  Now we can show by induction that
  \begin{equation}\label{eq:thm:hexagonal_point_lattice_vector_rotation/recurrence}
    y_m = a w_m + b w_{m+1}.
  \end{equation}

  \begin{itemize}
    \item The base case \( m = 0 \) is simply
    \begin{equation*}
      y_0 = x = au + bv = a w_0 + b w_1.
    \end{equation*}

    \item If \( y_{m-1} = a w_{m-1} + b w_m \), since rotation about the origin is linear, we conclude that
    \begin{equation*}
      y_m = a w_m + b w_{m+1}
    \end{equation*}

    Since both \( w_m \) and \( w_{m+1} \) are vectors from \( L \), then so is \( y_m \).
  \end{itemize}
\end{proof}

\begin{proposition}\label{thm:hexagonal_point_lattice_minimal_vectors}
  A \hyperref[def:hexagonal_point_lattice]{hexagonal point lattice} has six \hyperref[def:minimal_lattice_vector]{minimal vectors}, and their \hyperref[def:convex_hull]{convex hull} is a \hyperref[def:regular_polygon]{regular hexagon}.
\end{proposition}
\begin{comments}
  \item As can be seen by comparing \cref{fig:thm:hexagonal_point_lattice_minimal_vectors} and \cref{fig:thm:hexagonal_point_lattice_voronoi_cell/result}, this hexagon is different from the hexagons in the corresponding \hyperref[def:hexagonal_tiling]{hexagonal tiling}.
\end{comments}
\begin{proof}
  Let \( L \) be a hexagonal point lattice. By definition, then, there exists a basis \( \set{ u, v } \) of minimal vectors where \( \measuredangle(u, v) = \pi / 3 \).

  \begin{figure}[!ht]
    \centering
    \includegraphics[page=1]{output/thm__hexagonal_point_lattice_minimal_vectors}
    \caption{The minimal vectors in \fullref{thm:hexagonal_point_lattice_minimal_vectors}.}\label{fig:thm:hexagonal_point_lattice_minimal_vectors}
  \end{figure}

  \SubProof{Proof that there are six minimal vectors} We are interested in the solutions to the equation
  \begin{equation}\label{eq:thm:hexagonal_point_lattice_minimal_vectors/proof/eq_norm}
    \norm x = \norm u.
  \end{equation}

  Obviously \( u \) and \( v \) are solutions. To seek general solutions, first note that
  \begin{align*}
    \norm{au + bv}^2
    &=
    \inprod {au + bv} {au + bv}
    = \\ &=
    a^2 \cdot \norm{u}^2 + 2ab \cdot \inprod u v + b^2 \cdot \norm{v}^2
    \reloset{\eqref{eq:thm:cosine_of_angle_measure/cos}} = \\ &=
    a^2 \cdot \norm{u}^2 + 2ab \cdot \norm{u} \cdot \norm{v} \cdot \cos\measuredangle(u, v) + b^2 \cdot \norm{v}^2
    \reloset{\norm{u} = \norm{v}} = \\ &=
    \norm{u}^2 (a^2 + 2ab \cos\measuredangle(u, v) + b^2).
  \end{align*}

  We have chosen \( u \) and \( v \) such that \( \measuredangle(u, v) = \pi / 3 \), and \eqref{eq:thm:trigonometric_function_secondary_roots/third} implies that \( \cos \measuredangle(u, v) = 1 / 2 \), thus
  \begin{equation}\label{eq:thm:hexagonal_point_lattice_minimal_vectors/proof/norm_squared}
    \norm{au + bv}^2 = \norm{u}^2 (a^2 + ab + b^2).
  \end{equation}

  Then the squared equation \eqref{eq:thm:hexagonal_point_lattice_minimal_vectors/proof/eq_norm} becomes
  \begin{equation*}
    \norm{au + bv}^2 = \norm{u}^2 (a^2 + ab + b^2) = \norm{u}^2.
  \end{equation*}

  Dividing by \( \norm{u}^2 \), we obtain
  \begin{equation}\label{eq:thm:hexagonal_point_lattice_minimal_vectors/proof/eq_square}
    a^2 + ab + b^2 = 1.
  \end{equation}

  We can regard \eqref{eq:thm:hexagonal_point_lattice_minimal_vectors/proof/eq_square} as an equation in \( a \). \Fullref{thm:real_quadratic_discriminant} then implies that \eqref{eq:thm:hexagonal_point_lattice_minimal_vectors/proof/eq_square} has a real root if and only if the discriminant \( b^2 - 4(b^2 - 1) \) is nonnegative, which is equivalent to \( 3b^2 \leq 4 \). Thus, \( b \) can be either \( 0 \), \( 1 \) or \( -1 \).

  Similarly, \( a \) can be either \( 0 \), \( 1 \) or \( -1 \). For any vectors \( x \) in \( L \), there exist integers \( a \) and \( b \) such that \( x = au + bv \). \Cref{tab:thm:hexagonal_point_lattice_minimal_vectors/proof/eq_square} lists all possible options for \( x \), whose norms are calculated using \eqref{eq:thm:hexagonal_point_lattice_minimal_vectors/proof/norm_squared}.

  \begin{table}[!ht]
    \begin{equation*}
      \begin{array}{*{5}{l}}
        \toprule
        a  & b  & au + bv  & a^2 + ab + b^2 \T{is minimal} \\
        \midrule
        0  & 0  & \vect 0  & 0            & \T{no}  \\
        0  & -1 & -v       & 1            & \T{yes} \\
        0  & 1  & v        & 1            & \T{yes} \\
        -1 & 0  & -u       & 1            & \T{yes} \\
        -1 & -1 & -(u + v) & 3            & \T{no}  \\
        -1 & 1  & -u + v   & 1            & \T{yes} \\
        1  & 0  & u        & 1            & \T{yes} \\
        1  & -1 & u - v    & 1            & \T{yes} \\
        1  & 1  & u + v    & 3            & \T{no}  \\
        \bottomrule
      \end{array}
    \end{equation*}
    \caption{All linear combinations of \( u \) and \( v \) satisfying \eqref{eq:thm:hexagonal_point_lattice_minimal_vectors/proof/eq_square} in our proof of \fullref{thm:hexagonal_point_lattice_minimal_vectors}}\label{tab:thm:hexagonal_point_lattice_minimal_vectors/proof/eq_square}
  \end{table}

  \SubProof{Proof that their convex hull is a regular hexagon} We will show that the obtained minimal vectors are all rotations of \( u \) by multiples of \( \pi / 3 \) radians.

  \begin{itemize}
    \item The vector \( v \) is, by definition, a rotation of \( u \) by \( \pi / 3 \) radians.

    \item \Fullref{thm:v_minus_u_as_rotation} implies the same for \( v \) and \( v - u \).

    \item We have just concluded that
    \begin{equation*}
      \measuredangle(u, v - u) = \frac {2\pi} 3.
    \end{equation*}

    \Fullref{thm:angle_of_opposite_ray} implies that
    \begin{equation*}
      \measuredangle(-u, v - u) = \measuredangle(u, v - u) + \pi = \frac {5\pi} 3.
    \end{equation*}

    \Fullref{thm:angle_measure_swap} implies that
    \begin{equation*}
      \measuredangle(v - u, u) = 2\pi - \measuredangle(u, v - u) = \frac \pi 3.
    \end{equation*}

    Then \( -u \) is a rotation of \( v - u \) by \( \pi / 3 \) radians.

    \item Applying \fullref{thm:angle_of_opposite_ray} twice, we conclude that
    \begin{equation*}
      \measuredangle(-u, -v) = \measuredangle(u, v),
    \end{equation*}
    hence \( -v \) is a rotation of \( -u \) by \( \pi / 3 \) radians.

    \item Again via \fullref{thm:angle_of_opposite_ray}, we conclude that \( u - v = -(v - u) \) is a rotation of \( -v \) by \( \pi / 3 \) radians.

    \item Finally, we use that
    \begin{equation*}
      \measuredangle(u - v, u) = \measuredangle(v - u, -u)
    \end{equation*}
    to conclude that \( u \) is a rotation of \( u - v \) by \( \pi / 3 \) radians.
  \end{itemize}
\end{proof}

\begin{corollary}\label{thm:hexagonal_point_lattice_minimal_basis}
  In a \hyperref[def:hexagonal_point_lattice]{hexagonal point lattice}, every pair of linearly independent \hyperref[def:minimal_lattice_vector]{minimal vectors} is a \hyperref[def:point_lattice_basis]{lattice basis}.
\end{corollary}
\begin{comments}
  \item We have an analogous statement for hypercubic lattices --- see \fullref{thm:hypercubic_point_lattice_minimal_basis}.
\end{comments}
\begin{proof}
  Fix a hexagonal point lattice \( L \). By definition, there exists a basis \( \set{ u, v } \) of minimal vectors where \( \measuredangle(u, v) = \pi / 3 \).

  \Fullref{thm:hexagonal_point_lattice_minimal_vectors} implies that there are six minimal vectors in \( L \) --- \( \pm u \), \( \pm v \) and \( \pm (v - u) \) (\fullref{thm:v_minus_u_as_rotation} implies that \( v - u \) is minimal).

  The linearly independent pairs of minimal vectors are those not containing two additive inverses. Pairs containing only \( \pm u \) and \( \pm v \) are obvious bases.

  Out of the other possible pairs, due to their similarity, we will only consider \( x = u \) and \( y = v - u \). Then \( u = x \) and \( v = y - x \), thus we can express any integer linear combination of \( x \) and \( y \) via \( u \) and \( v \).

  This concludes the proof.
\end{proof}

\begin{proposition}\label{thm:hexagonal_point_lattice_shortest_non_minimal_vectors}
  In a \hyperref[def:hexagonal_point_lattice]{hexagonal point lattice} with a minimal vectors \( u \) and \( v \) such that \( \measuredangle(u, v) = \pi / 3 \), the shortest non-minimal vectors have norm \( \sqrt 3 \cdot \norm{u} \). The vector \( u + v \) attains this norm.
\end{proposition}
\begin{proof}
  \Fullref{thm:hexagonal_point_lattice_minimal_basis} implies that \( u \) and \( v \) form a basis.

  As in \fullref{thm:hexagonal_point_lattice_minimal_vectors}, we start with \eqref{eq:thm:hexagonal_point_lattice_minimal_vectors/proof/norm_squared}:
  \begin{equation*}
    \norm{au + bv}^2 = \norm{u}^2 \cdot (a^2 + ab + b^2).
  \end{equation*}

  We want to minimize this norm, which is equivalent to minimizing the integer function
  \begin{equation}\label{eq:thm:hexagonal_point_lattice_shortest_non_minimal_vectors/proof/min}
    a^2 + ab + b^2.
  \end{equation}

  In both cases, we seek only solutions such that \( a^2 + ab + b^2 > 1 \).

  For \( u + v \) we have \( a^2 + ab + b^2 = 3 \) and hence
  \begin{equation*}
    \norm{u + v} = \sqrt 3 \cdot \norm{u},
  \end{equation*}
  as desired.

  It remains to prove that no \( a \) and \( b \) exist such that
  \begin{equation}\label{eq:thm:hexagonal_point_lattice_shortest_non_minimal_vectors/proof/eq}
    a^2 + ab + b^2 = 2.
  \end{equation}

  \Fullref{thm:real_quadratic_discriminant} implies that \eqref{eq:thm:hexagonal_point_lattice_shortest_non_minimal_vectors/proof/eq} has a real root if and only if the discriminant \( b^2 - 4(b^2 - 2) \) is nonnegative, which is equivalent to \( 3b^2 \leq 8 \).

  As in \fullref{thm:hexagonal_point_lattice_minimal_vectors}, the only solution candidates for \( b \) are \( -1 \), \( 0 \) and \( 1 \), and similarly for \( a \). The case analysis in \cref{tab:thm:hexagonal_point_lattice_minimal_vectors/proof/eq_square} suggests, however, that no solutions to \eqref{eq:thm:hexagonal_point_lattice_shortest_non_minimal_vectors/proof/eq} exist.

  Therefore, \( \sqrt 3 \cdot \norm{u} \) is the second smallest norm.
\end{proof}
