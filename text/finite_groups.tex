\section{Finite groups}\label{sec:finite_groups}

\paragraph{Cauchy's subgroup theorem}

\begin{lemma}\label{thm:stable_points_of_prime_group_action}\mcite[thm. IV.1.3]{Aluffi2009Algebra}
  Fix a \hyperref[def:group_action]{group action} \( \Phi: G \times X \to X \) of a \hi{nontrivial} finite group \( G \) on a finite set \( S \). Suppose that the order of \( G \) is a power of a \hyperref[def:prime_number]{prime number} \( p \).

  Consider the set
  \begin{equation}\label{eq:thm:stable_points_of_prime_group_action/fixed}
    Z \coloneqq \set{ x \in X \given \Phi(g, x) = x }
  \end{equation}
  of all points of \( X \) fixed by \( \Phi \).

  Then
  \begin{equation}\label{eq:thm:stable_points_of_prime_group_action}
    \card(X) \cong \card(Z) \pmod p.
  \end{equation}
\end{lemma}
\begin{proof}
  Let \( A \) be a set with one representative from each \hi{disjoint nontrivial} orbit of \( X \). Then
  \begin{equation*}
    X = \parens*{ \bigcup_{a \in A} O(a) } \cup Z.
  \end{equation*}

  \Fullref{thm:cardinality_sum_rule} implies that
  \begin{equation}\label{eq:thm:stable_points_of_prime_group_action/proof}
    \card(X) = \sum_{a \in A} \card(O(a)) + \card(Z).
  \end{equation}

  \Fullref{thm:orbit_stabilizer_theorem} implies that \( \card(O(a)) = [G: S(a)] \) for every \( a \in A \), and \fullref{thm:lagranges_subgroup_theorem} implies that \( \card(O(a)) \) divides the order of \( G \).

  The latter is a power of \( p \). Since \( \card(O(a)) \) is by assumption a nontrivial orbit, it must have at least two elements, and thus it must itself be a power of \( p \). Thus, all terms of \eqref{eq:thm:stable_points_of_prime_group_action/proof} except for the last are multiples of \( p \).

  Then \eqref{eq:thm:stable_points_of_prime_group_action} follows.
\end{proof}

\begin{theorem}[Cauchy's subgroup theorem]\label{thm:cauchys_subgroup_theorem}\mcite[thm. IV.2.1]{Aluffi2009Algebra}
  If the \hyperref[def:prime_number]{prime number} \( p \) divides the order of the finite \hyperref[def:group]{group} \( G \), then \( G \) contains an element of \hyperref[def:group_element_order]{order} \( p \).
\end{theorem}
\begin{proof}
  Denote the order of \( G \) by \( n \).

  Consider the set \( S \) of all \( p \)-tuples \( (x_1, \ldots, x_p) \) of elements of \( G \) such that \( x_1 \cdots x_p = e \).

  We note that \( x_p = (x_1 \cdots x_{p-1})^{-1} \), thus the last coordinate is uniquely determined by the rest, which may be arbitrary. \Fullref{thm:combinatorial_variation_count/repetition} implies that \( S \) has \( n^{p-1} \) elements.

  Next, consider the \hyperref[def:cyclic_shift]{right cyclic shift} \( (x_{k_1}, \ldots, x_{k_p}) \) of \( (x_1, \ldots, x_p) \) by the nonnegative integer \( c \) (where \( k_i = \rem_1(i + c, p) \), \( \rem_1 \) being the \hyperref[def:shifted_remainder]{shifted remainder}). We will show by induction on \( c \) that \( (x_{k_1}, \ldots, x_{k_p}) \) is also in \( S \):
  \begin{itemize}
    \item The case \( c = 0 \) is vacuous.
    \item Otherwise, if \( (x_{k_1}, \ldots, x_{k_p}) \) is a shift by \( c \) for which the inductive hypothesis holds, we know that \( x_{k_1} \cdots x_{k_p} = e \), and we can multiply by \( x_{k_p} \) on the left and \( x_{k_p}^{-1} \) on the right to obtain
    \begin{equation*}
      x_{k_p} x_{k_1} \cdots x_{k_{p-1}} = e.
    \end{equation*}

    This is a cyclic shift by \( c + 1 \); it clearly also belongs to \( S \).
  \end{itemize}

  Thus, \( S \) is closed under right cyclic shifts. By \fullref{thm:cyclic_shift_is_action}, the cyclic group \( C_p \) \hyperref[def:group_action]{acts} on \( S \) by right cyclic shifts.

  For this action, \fullref{thm:stable_points_of_prime_group_action} implies that
  \begin{equation*}
    \card(S) \cong \card(Z) \pmod p,
  \end{equation*}
  where \( Z \) is the set fixed of points, as defined in \eqref{eq:thm:stable_points_of_prime_group_action/fixed}.

  We have already concluded that \( \card(S) = n^{p-1} \). Furthermore, by assumption \( p \) divides \( n \), so \( p \) divides \( \card(S) \). Thus, it divides \( \card(Z) \).

  We know that \( Z \) necessarily contains the tuple \( (e, e, \ldots, e) \), so it is nonempty. Since \( p > 1 \), it must also contain another element.

  Each tuple in \( Z \) is fixed by all cyclic shifts, hence all components of the tuple must be equal. Thus, \( Z \) contains some tuple \( (a, a, \ldots, a) \), where \( a \neq e \).

  This is the desired element of order \( p \) because \( a^p = e \).
\end{proof}

\begin{corollary}\label{thm:weak_cauchys_subgroup_theorem}
  A finite \hyperref[def:group]{group} \( G \) has at least one \hyperref[def:group/submodel]{subgroup} of order \( p \) for any \hyperref[def:prime_number]{prime number} \( p \) dividing the order of \( G \).
\end{corollary}
\begin{proof}
  \Fullref{thm:cauchys_subgroup_theorem} implies that \( G \) has an element \( x \) of order \( p \). The cyclic subgroup \( \braket{ x } \) then has order \( p \).
\end{proof}

\paragraph{Classification of small finite groups}

\begin{proposition}\label{thm:prime_groups_are_simple}
  All groups of \hyperref[def:prime_number]{prime} \hyperref[def:group_order]{order} are \hyperref[def:simple_object]{simple}.
\end{proposition}
\begin{proof}
  Suppose that \( G \) has order \( p \) for some prime \( p \) and let \( N \) be a normal subgroup.

  From \fullref{thm:lagranges_subgroup_theorem} it follows that \( \ord(N) \) divides \( p \). But \( p \) is prime, hence \( N \) is either the trivial group or the full group.

  Therefore, \( G \) is a simple group.
\end{proof}

\begin{proposition}\label{thm:single_group_of_prime_order}
  All groups of \hyperref[def:prime_number]{prime} \hyperref[def:group_order]{order} are \hyperref[def:cyclic_group]{cyclic}.
\end{proposition}
\begin{proof}
  Fix a group \( G \) of order \( p \).

  By \fullref{thm:lagranges_subgroup_theorem}, for every element \( a \) of \( G \), the cyclic subgroup \( \braket{ a } \) has an order dividing \( p \). But \( p \) is prime, thus either \( \braket{ a } \) is trivial (in which case \( a = e \)) or \( \braket{ a } \) is the entire group.

  Then \( a \) is a generator of \( G \), i.e. \( G \) is cyclic.
\end{proof}

\begin{proposition}\label{thm:clasification_of_small_groups}
  The \hyperref[def:group]{groups} up to order \( 7 \) can be classified as follows:
  \begin{thmenum}
    \thmitem{thm:clasification_of_small_groups/trivial} All groups with one element are \hyperref[def:group/homomorphism]{isomorphic} to the trivial \hyperref[def:cyclic_group]{cyclic group} \( C_1 = \braket{ e } \).

    \thmitem{thm:clasification_of_small_groups/prime} All groups of prime order are \hyperref[def:prime_number]{prime} \hyperref[def:group_order]{order} are isomorphic to the (canonical) cyclic group \( C_p = \braket{ a \given a^p = e } \).

    \thmitem{thm:clasification_of_small_groups/4} Every group of order \( 4 \) is isomorphic to either the cyclic group \( C_4 \) or to the \hyperref[def:klein_four_group]{Klein four-group} \( V_4 \). The groups \( C_4 \) and \( V_4 \) are not isomorphic.

    Note that \( V_4 \) is isomorphic to the \hyperref[def:dihedral_group]{dihedral group} \( D_2 \), as shown in \fullref{thm:klein_four_group_presentation}. \Fullref{thm:dihedral_group_not_cyclic} implies that \( C_4 \) and \( D_2 \) (and hence \( C_4 \) and \( V_4 \)) are not isomorphic.

    \thmitem{thm:clasification_of_small_groups/6} Every group of order \( 6 \) is isomorphic to either the cyclic group \( C_6 \) or to the \hyperref[def:symmetric_group]{symmetric group} \( S_3 \).

    The groups \( C_6 \) and \( S_3 \) are not isomorphic since, as shown in \fullref{ex:s3}, \( S_3 \) is not abelian.

    We also note here that \( S_3 \) is isomorphic to the \hyperref[def:dihedral_group]{dihedral group} \( D_6 \), as shown in \fullref{thm:s3_and_d6}.
  \end{thmenum}
\end{proposition}
\begin{comments}
  \item As a corollary, \( V_4 \) is the smallest non-cyclic group, while \( S_3 \) is the smallest non-abelian group
\end{comments}
\begin{proof}
  \SubProofOf{thm:clasification_of_small_groups/trivial} There is a straightforward isomorphism between any pair of groups of one element.

  \SubProofOf{thm:clasification_of_small_groups/prime} It follows from \fullref{thm:single_group_of_prime_order} that every group of prime order is cyclic, and from \fullref{thm:cyclic_monoid_classification/finite_group} that all finite cyclic groups are isomorphic.

  \SubProofOf{thm:clasification_of_small_groups/4} Let \( G \) be a group of order \( 4 \).

  The group \( G \) necessarily has a neutral element \( e \). \Fullref{thm:cauchys_subgroup_theorem} implies that \( G \) has an element \( a \) of order \( 2 \).

  Let \( b \) be any of the two remaining elements. \Fullref{thm:def:group_element_order/group_order} implies that the order of \( b \) is either \( 2 \) or \( 4 \).

  We thus have two possibilities:
  \begin{itemize}
    \item If \( b \) has order \( 2 \), then \( (ab)^2 = a^2 b^2 = e \), hence \( ab \) is an \hyperref[def:group_element_order]{involution}.

    Then \( a \) and \( b \) satisfy the \hyperref[def:group_presentation]{presentation} \eqref{eq:thm:klein_four_group_presentation} of Klein's four-group \( V_4 \). Thus, \( G \) is isomorphic to \( V_4 \).

    \item If \( b \) has order \( 4 \), it generates \( G \). Hence, by \fullref{thm:cyclic_monoid_classification/finite_group}, \( G \) is isomorphic to \( C_4 \).
  \end{itemize}

  \SubProofOf{thm:clasification_of_small_groups/6} Let \( G \) be a group of order \( 6 \).

  \Fullref{thm:cauchys_subgroup_theorem} implies that \( G \) has an element \( a \) of order \( 3 \) and an element \( b \) of order \( 2 \). Of course, \( G \) also has a neutral element \( e \).

  Consider the cyclic subgroup \( \braket{ a } \). Either \( b \) belongs to \( \braket{ a } \) or not.

  We thus have two possibilities:
  \begin{itemize}
    \item If \( b \) belongs to \( \braket{ a } \), then \( b \) equals some multiple \( ca \) of \( a \).

    Since \( b^2 = e \), we have \( (ca)^2 = e \). But \( a^3 = e \). Thus, \( c^2 a^2 = a^3 \), i.e. \( a = c^2 \).

    \Fullref{thm:def:group_element_order/group_order} implies that the order of \( c \) is either \( 2 \), \( 3 \) or \( 6 \). It cannot be \( 2 \) because then \( a \) would equal \( e \). It cannot be \( 3 \) because then \( a^3 = ca = b \) would equal \( e \).

    It remains for the order of \( c \) to be \( 6 \). Then \( c \) generates \( G \), and \fullref{thm:cyclic_monoid_classification/finite_group} implies that \( G \) is isomorphic to \( C_6 \).

    \item If \( b \) does not belong to \( \braket{ a } \), then the (right) coset \( \braket{ a } b \) is disjoint from \( \braket{ a } \).

    Both sets have \( 3 \) elements, so they form a partition of \( G \), that is,
    \begin{equation*}
      G = \set{ e, a, a^2, b, ab, a^2 b }.
    \end{equation*}

    To show that \( G \) satisfies the Dihedral group presentation \eqref{eq:def:dihedral_group}, we must show that \( bab = a^2 \). We will do a case analysis of \( ba \) (and in one case discover that \( G \) is actually cyclic):
    \begin{itemize}
      \item If \( ba = e \), then \( a \) and \( b \) are inverses. \Fullref{thm:def:group_element_order/inverse} implies that they have the same order, which contradicts our assumptions about them.

      \item If \( ba = a \), then \( b = e \), which again contradicts our assumptions.

      \item If \( ba = a^2 \), then \( a = b \), which is again a contradiction.

      \item If \( ba = b \), then \( a = e \), which is again a contradiction.

      \item If \( ba = ab \), then \( G \) is generated by \( ab \). Indeed, we will show that \( ab \) has order \( 6 \):
      \begin{itemize}
        \item \( ab \) is by itself clearly distinct from \( e \).
        \item \( (ab)^2 = a^2 b^2 = a^2 \).
        \item \( (ab)^3 = a^3 b^3 = b \).
        \item \( (ab)^4 = a \).
        \item \( (ab)^5 = a^2 b \).
      \end{itemize}

      Clearly \( (ab)^6 = e \). Then \( 6 \) is the smallest power of \( ab \) that gives \( e \), i.e. the order of \( ab \). Then \( ab \) generates \( G \), and \fullref{thm:cyclic_monoid_classification/finite_group} implies that \( G \) is isomorphic to \( C_6 \).

      \item Finally, if \( ba = a^2 b \), then \( bab = a^2 bb = a^2 \).

      In this case the equations in \eqref{eq:def:dihedral_group} are satisfied, and \( G \) is thus (isomorphic to) the Dihedral group \( D_6 \).

      The statement of the proposition requires us to prove that \( G \) is isomorphic to \( S_3 \) instead, but this follows easily since, by \fullref{thm:s3_and_d6}, \( S_3 \) and \( D_6 \) are isomorphic.
    \end{itemize}
  \end{itemize}
\end{proof}
