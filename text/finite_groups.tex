\section{Finite groups}\label{sec:finite_groups}

\paragraph{Cauchy's subgroup theorem}

\begin{lemma}\label{thm:stable_points_of_prime_group_action}\mcite[thm. IV.1.3]{Aluffi2009Algebra}
  Fix a \hyperref[def:group_action]{group action} \( \Phi: G \times X \to X \) of a \hi{nontrivial} finite group \( G \) on a finite set \( S \). Suppose that the order of \( G \) is a power of a \hyperref[def:prime_number]{prime number} \( p \).

  Consider the set
  \begin{equation}\label{eq:thm:stable_points_of_prime_group_action/fixed}
    Z \coloneqq \set{ x \in X \given \Phi(g, x) = x }
  \end{equation}
  of all points of \( X \) fixed by \( \Phi \).

  Then
  \begin{equation}\label{eq:thm:stable_points_of_prime_group_action}
    \card(X) \cong \card(Z) \pmod p.
  \end{equation}
\end{lemma}
\begin{proof}
  Let \( A \) be a set with one representative from each \hi{disjoint nontrivial} orbit of \( X \). Then
  \begin{equation*}
    X = \parens[\Big]{ \bigcup_{a \in A} O(a) } \cup Z.
  \end{equation*}

  \Fullref{thm:cardinality_sum_rule} implies that
  \begin{equation}\label{eq:thm:stable_points_of_prime_group_action/proof}
    \card(X) = \sum_{a \in A} \card(O(a)) + \card(Z).
  \end{equation}

  \Fullref{thm:orbit_stabilizer_theorem} implies that \( \card(O(a)) = [G: S(a)] \) for every \( a \in A \), and \fullref{thm:lagranges_subgroup_theorem} implies that \( \card(O(a)) \) divides the order of \( G \).

  The latter is a power of \( p \). Since \( \card(O(a)) \) is by assumption a nontrivial orbit, it must have at least two elements, and thus it must itself be a power of \( p \). Thus, all terms of \eqref{eq:thm:stable_points_of_prime_group_action/proof} except for the last are multiples of \( p \).

  Then \eqref{eq:thm:stable_points_of_prime_group_action} follows.
\end{proof}

\begin{theorem}[Cauchy's subgroup theorem]\label{thm:cauchys_subgroup_theorem}\mcite[thm. IV.2.1]{Aluffi2009Algebra}
  If the \hyperref[def:prime_number]{prime number} \( p \) divides the order of the finite \hyperref[def:group]{group} \( G \), then \( G \) contains an element of \hyperref[def:group_element_order]{order} \( p \).
\end{theorem}
\begin{proof}
  Denote the order of \( G \) by \( n \).

  Consider the set \( S \) of all \( p \)-tuples \( (x_1, \ldots, x_p) \) of elements of \( G \) such that \( x_1 \cdots x_p = e \).

  We note that \( x_p = (x_1 \cdots x_{p-1})^{-1} \), thus the last coordinate is uniquely determined by the rest, which may be arbitrary. \Fullref{thm:combinatorial_variation_count/repetition} implies that \( S \) has \( n^{p-1} \) elements.

  Next, consider the \hyperref[def:cyclic_shift]{right cyclic shift} \( (x_{k_1}, \ldots, x_{k_p}) \) of \( (x_1, \ldots, x_p) \) by the nonnegative integer \( c \) (where \( k_i = \rem_1(i + c, p) \), \( \rem_1 \) being the \hyperref[def:shifted_remainder]{shifted remainder}). We will show by induction on \( c \) that \( (x_{k_1}, \ldots, x_{k_p}) \) is also in \( S \):
  \begin{itemize}
    \item The case \( c = 0 \) is vacuous.
    \item Otherwise, if \( (x_{k_1}, \ldots, x_{k_p}) \) is a shift by \( c \) for which the inductive hypothesis holds, we know that \( x_{k_1} \cdots x_{k_p} = e \), and we can multiply by \( x_{k_p} \) on the left and \( x_{k_p}^{-1} \) on the right to obtain
    \begin{equation*}
      x_{k_p} x_{k_1} \cdots x_{k_{p-1}} = e.
    \end{equation*}

    This is a cyclic shift by \( c + 1 \); it clearly also belongs to \( S \).
  \end{itemize}

  Thus, \( S \) is closed under right cyclic shifts. By \fullref{thm:cyclic_shift_is_action}, the cyclic group \( C_p \) \hyperref[def:group_action]{acts} on \( S \) by right cyclic shifts.

  For this action, \fullref{thm:stable_points_of_prime_group_action} implies that
  \begin{equation*}
    \card(S) \cong \card(Z) \pmod p,
  \end{equation*}
  where \( Z \) is the set fixed of points, as defined in \eqref{eq:thm:stable_points_of_prime_group_action/fixed}.

  We have already concluded that \( \card(S) = n^{p-1} \). Furthermore, by assumption \( p \) divides \( n \), so \( p \) divides \( \card(S) \). Thus, it divides \( \card(Z) \).

  We know that \( Z \) necessarily contains the tuple \( (e, e, \ldots, e) \), so it is nonempty. Since \( p > 1 \), it must also contain another element.

  Each tuple in \( Z \) is fixed by all cyclic shifts, hence all components of the tuple must be equal. Thus, \( Z \) contains some tuple \( (a, a, \ldots, a) \), where \( a \neq e \).

  This is the desired element of order \( p \) because \( a^p = e \).
\end{proof}

\begin{corollary}\label{thm:weak_cauchys_subgroup_theorem}
  A finite \hyperref[def:group]{group} \( G \) has at least one \hyperref[def:group/submodel]{subgroup} of order \( p \) for any \hyperref[def:prime_number]{prime number} \( p \) dividing the order of \( G \).
\end{corollary}
\begin{proof}
  \Fullref{thm:cauchys_subgroup_theorem} implies that \( G \) has an element \( x \) of order \( p \). The cyclic subgroup \( \braket{ x } \) then has order \( p \).
\end{proof}
