\subsection{Boolean algebras}\label{subsec:boolean_algebras}

\paragraph{Heyting algebras}

\begin{proposition}\label{thm:heyting_conditional_set_is_ideal}
  In any \hyperref[def:lattice]{lattice} \( X \), for any two elements \( x \) and \( y \), the following set is nonempty and \hyperref[def:closed_ordered_subset]{downward closed}:
  \begin{equation*}
    H_{x,y} \coloneqq \set{ c \in X \given x \wedge c \leq y }.
  \end{equation*}

  If \( X \) is \hyperref[def:distributive_lattice]{distributive}, then \( H_{x,y} \) is \hyperref[def:directed_set]{upward directed} and hence a \hyperref[def:lattice_ideal]{lattice ideal}.
\end{proposition}
\begin{proof}
  The set \( H_{x,y} \) is nonempty --- it contains \( y \) because \( x \wedge y \leq y \). Furthermore, if \( a \) belongs to \( H_{x,y} \) and \( b \leq a \), then \( x \wedge b \leq x \wedge a \leq y \), thus \( b \) also belongs to \( H_{x,y} \).

  If \( X \) is distributive and if \( a \) and \( b \) belong to \( H_{x,y} \), then
  \begin{equation*}
    x \wedge (a \vee b)
    \reloset {\eqref{eq:def:distributive_lattice/meet_over_join}} =
    (x \wedge a) \vee (x \wedge b)
    \reloset {\ref{thm:def:lattice/operations_preserve_order}} \leq
    y \vee y
    =
    y,
  \end{equation*}
  hence their least upper bound \( a \vee b \) is also in \( H_{x,y} \).
\end{proof}

\begin{example}\label{ex:heyting_conditional_ideal}
  We will list several examples related to \fullref{thm:heyting_conditional_set_is_ideal}.

  \begin{thmenum}
    \thmitem{ex:heyting_conditional_ideal/pentagon} Without the assumption of distributivity, \fullref{thm:heyting_conditional_set_is_ideal} may not hold. Indeed, consider the pentagon lattice \eqref{eq:ex:def:modular_lattice/pentagon}. We have
    \begin{equation*}
      H_{a,b} = \set{ \bot, b, c },
    \end{equation*}
    but
    \begin{equation*}
      a \wedge (b \vee c) = a \wedge \top = a
    \end{equation*}
    is not in \( H_{a,b} \).

    Nevertheless, \( H_{b,c} = \set{ \bot, c } \) is an ideal --- in fact, the principal ideal of \( c \).

    \thmitem{ex:heyting_conditional_ideal/diamond} Consider the diamond lattice \eqref{eq:ex:def:distributive_lattice/diamond}. We have
    \begin{equation*}
      H_{a,c} = \set{ \bot, b, c }
    \end{equation*}
    but, again,
    \begin{equation*}
      a \wedge (b \vee c) = a \wedge \top = a,
    \end{equation*}
    so \( H_{a,c} \) is also not an ideal.

    \thmitem{ex:heyting_conditional_ideal/nonprincipal_ideal} Adjoin to the \hyperref[def:ordinal]{ordinal} \( \omega + 1 \) an auxiliary element \( a \) that is comparable only to the bottom \( 0 \) and the top \( \omega \). The following is a fragment of the Hasse diagram of \( \omega \cup \set{ a } \):
    \begin{equation*}
      \includegraphics[page=2]{output/ex__heyting_conditional_ideal}
    \end{equation*}

    It is a distributive lattice as a consequence of \fullref{thm:distributive_lattice_characterization}.

    We have \( H_{a,0} = \omega \), which is an ideal of \( \omega \cup \set{ a } \). It is not, however, a principal ideal, because it contains no greatest element.

    On the other hand, the ideal \( H_{\omega,a} = \set{ 0, a } \) is principal.
  \end{thmenum}
\end{example}

\begin{proposition}\label{thm:heyting_conditional_set}
  For every two \hyperref[def:lattice]{lattice} elements \( x \) and \( y \), the following are equivalent for a third element \( z \):
  \begin{thmenum}
    \thmitem{thm:heyting_conditional_set/iff} We have \( c \leq z \) if and only if \( x \wedge c \leq y \).
    \thmitem{thm:heyting_conditional_set/greatest} We have \( x \wedge z \leq y \) and \( z \) is the greatest among all such elements.
    \thmitem{thm:heyting_conditional_set/ideal} The set\fnote{We have not required the lattice to be distributive, hence \( H_{x,y} \) may fail to be an ideal in general.} \( H_{x,y} \) from \fullref{thm:heyting_conditional_set_is_ideal} is a \hyperref[def:lattice_ideal/principal]{principal ideal} generated by \( z \).
  \end{thmenum}
\end{proposition}
\begin{comments}
  \item In particular, if \( z \) exists, it is unique.
\end{comments}
\begin{proof}
  \ImplicationSubProof{thm:heyting_conditional_set/iff}{thm:heyting_conditional_set/greatest} If \( c \) is an element such that \( x \wedge c \leq y \), by assumption \( c \leq z \). Then \( z \) is the greatest among all such elements.

  \ImplicationSubProof{thm:heyting_conditional_set/greatest}{thm:heyting_conditional_set/ideal} Trivial.

  \ImplicationSubProof{thm:heyting_conditional_set/ideal}{thm:heyting_conditional_set/iff} If \( c \leq z \), then \( c \) is in \( H_{x,y} \) and hence \( x \wedge c \leq y \). Conversely, if \( x \wedge c \leq y \), then \( c \) is in \( H_{x,y} \) and thus \( c \leq z \).
\end{proof}

\begin{definition}\label{def:heyting_algebra}\mcite[331]{PicadoPultr2012}
  A \term{Heyting algebra} is a \hyperref[def:extremal_points/bounds]{bounded} \hyperref[def:lattice]{lattice} \( X \) with an additional \hyperref[def:binary_operation]{binary operation} \( {\rightarrow} \), which we call a \term{relative pseudocomplement} in accordance to \cite[51]{Birkhoff1967}, such that any of the following equivalent conditions hold:
  \begin{thmenum}[series=def:heyting_algebra]
    \thmitem{def:heyting_algebra/ideal} The element \( x \rightarrow y \) satisfies the equivalent conditions in \fullref{thm:heyting_conditional_set}.

    \thmitem{def:heyting_algebra/axioms} The following first-order axioms hold:
    \begin{thmenum}
      \thmitem{def:heyting_algebra/axioms/self} The following adaptation of \fullref{thm:boolean_equivalences/self_conditional}:
      \begin{equation}\label{eq:def:heyting_algebra/axioms/self}
        \mathllap{\xi \rightarrow \xi} \syneq \mathrlap{\top.}
      \end{equation}

      \thmitem{def:heyting_algebra/axioms/modus_ponens} The following variation of \eqref{eq:def:def:axiomatic_deductive_system/mp}:
      \begin{equation}\label{eq:def:heyting_algebra/axioms/modus_ponens}
        \mathllap{\xi \wedge (\xi \rightarrow \eta)} \syneq \mathrlap{\xi \wedge \eta.}
      \end{equation}

      \thmitem{def:heyting_algebra/axioms/modular} The following consequence of the modular identity \eqref{eq:def:modular_lattice}\fnote
        {
          If \( \varphi \) and \( \psi \) are \hyperref[def:propositional_grammar_schema/formula]{propositional formulas}, then
          \begin{equation*}
            \varphi \synwedge (\psi \synimplies \varphi)
            \reloset {\eqref{eq:thm:boolean_equivalences/conditional_as_disjunction}} \gleichstark
            \varphi \synwedge (\synneg \psi \synvee \varphi)
            \reloset {\eqref{eq:def:modular_lattice}} \gleichstark
            (\varphi \synwedge \synneg \psi) \synvee \varphi
            \gleichstark
            \varphi.
          \end{equation*}
        }:
      \begin{equation}\label{eq:def:heyting_algebra/axioms/modular}
        \mathllap{\eta \wedge (\xi \rightarrow \eta)} \syneq \mathrlap{\eta.}
      \end{equation}

      \thmitem{def:heyting_algebra/axioms/distributive} Distributivity of the relative pseudocomplement over conjunction:
      \begin{equation}\label{eq:def:heyting_algebra/axioms/distributive}
        \mathllap{\xi \rightarrow (\eta \wedge \zeta)} \syneq \mathrlap{(\xi \rightarrow \eta) \wedge (\xi \rightarrow \zeta).}
      \end{equation}
    \end{thmenum}
  \end{thmenum}

  Heyting algebras also have the following additional structure:
  \begin{thmenum}[resume=def:heyting_algebra]
    \thmitem{def:heyting_algebra/pseudocomplement} A unary operation \( {\widetilde {\anon}} \), which we call the \term{pseudocomplement}, for which the following axiom holds:
    \begin{equation}\label{eq:def:heyting_algebra/pseudocomplement}
      \widetilde \xi \syneq \xi \rightarrow \bot.
    \end{equation}
  \end{thmenum}

  Heyting algebras have the following metamathematical properties:
  \begin{thmenum}[resume=def:heyting_algebra]
    \thmitem{def:heyting_algebra/theory} We extend the \hyperref[def:lattice/theory]{first-order theory of lattices} by adding:
    \begin{itemize}
      \item The nullary functional symbols \( \top \) and \( \bot \) along with the axioms \eqref{eq:thm:def:lattice/bounded_absorption/join} and \eqref{eq:thm:def:lattice/bounded_absorption/meet}.

      \item The infix binary functional symbol \( \rightarrow \) along with the axioms \eqref{eq:def:heyting_algebra/axioms/self}, \eqref{eq:def:heyting_algebra/axioms/modus_ponens}, \eqref{eq:def:heyting_algebra/axioms/modular} and \eqref{eq:def:heyting_algebra/axioms/distributive}.

      \item The unary functional symbol \( {\widetilde {\anon}} \) along with the axiom \eqref{eq:def:heyting_algebra/pseudocomplement}.
    \end{itemize}

    \thmitem{def:heyting_algebra/submodel} In addition to containing the joins and meets of all its members, a \hyperref[def:first_order_submodel]{first-order submodel} of a Heyting algebra must also contain the top, bottom and be closed with respect to the relative pseudocomplement\fnote{It follows from \eqref{eq:def:heyting_algebra/pseudocomplement} that if a Heyting subalgebra is closed under relative pseudocomplements and it contains the bottom element, then it is also closed under pseudocomplements.}. We call such submodels \term{Heyting subalgebras}.

    \thmitem{def:heyting_algebra/homomorphism} A function \( f: X \to Y \) between Heyting algebras is a \hyperref[def:first_order_homomorphism]{first-order homomorphisms} if it is a \hyperref[def:lattice/homomorphism]{lattice homomorphism} that additionally satisfies
    \begin{align}\label{eq:def:heyting_algebra/homomorphism/top_bottom}
      f(\top) = \top
      &&
      f(\bot) = \bot
    \end{align}
    and
    \begin{equation}\label{eq:def:heyting_algebra/homomorphism/operation}
      f(x_1 \rightarrow x_2) = f(x_1) \rightarrow f(x_2).
    \end{equation}

    \thmitem{def:heyting_algebra/opposite} The \hyperref[def:lattice/opposite]{dual lattice} \( X^\oppos \) of a Heyting algebra may not be a Heyting algebra.

    \thmitem{def:heyting_algebra/category} We denote the \hyperref[def:category_of_small_first_order_models]{category of \( \mscrU \)-small models} for Heyting algebras by \( \cat{Heyt} \). It is a subcategory of the \hyperref[def:lattice/category]{category \( \cat{Lat} \) of lattices}.
  \end{thmenum}
\end{definition}
\begin{comments}
  \item What we call a Heyting algebra is called a \enquote{Browerian lattice} by \incite[147]{Birkhoff1967} and a \enquote{browerian algebra} \incite[7]{Golan2010}. The latter uses \enquote{Heyting algebras} for what we call a \hyperref[def:complete_lattice]{complete} Heyting algebra.
\end{comments}
\begin{defproof}
  \ImplicationSubProof{def:heyting_algebra/ideal}{def:heyting_algebra/axioms} Suppose that, for any elements \( x \) and \( y \), we have
  \begin{equation}\label{eq:def:heyting_algebra/proof/iff}
    c \leq (x \rightarrow y) \T{if and only if} x \wedge c \leq y.
  \end{equation}

  \SubProofOf*{def:heyting_algebra/axioms/self} Clearly \( x \wedge \top = x \), thus \( \top \leq (x \rightarrow x) \) and hence \( x \rightarrow x \) is the top element.

  \SubProofOf*{def:heyting_algebra/axioms/modus_ponens} \eqref{eq:def:heyting_algebra/proof/iff} implies
  \begin{equation*}
    x \wedge (x \rightarrow y) \leq y.
  \end{equation*}

  Then, since \( x \wedge \anon \) preserves order,
  \begin{equation*}
    \underbrace{x \wedge (x \wedge (x \rightarrow y))}_{x \wedge (x \rightarrow y)} \leq x \wedge y.
  \end{equation*}

  Conversely, since \( x \wedge y \leq y \), \eqref{eq:def:heyting_algebra/proof/iff} implies that \( y \leq (x \rightarrow y) \). Hence,
  \begin{equation*}
    x \wedge y \leq x \wedge (x \rightarrow y).
  \end{equation*}

  Then \eqref{eq:def:heyting_algebra/axioms/modus_ponens} follows.

  \SubProofOf*{def:heyting_algebra/axioms/modular} Since \( y \leq (x \rightarrow y) \), obviously
  \begin{equation*}
    y \wedge (x \rightarrow y) = y.
  \end{equation*}

  \SubProofOf*{def:heyting_algebra/axioms/distributive} \eqref{eq:def:heyting_algebra/proof/iff} implies
  \begin{equation*}
    x \wedge (x \rightarrow y) \leq y
  \end{equation*}
  and similarly
  \begin{equation*}
    x \wedge (x \rightarrow z) \leq z.
  \end{equation*}

  Then \fullref{thm:def:lattice/operations_preserve_order} implies that
  \begin{equation*}
    x \wedge \parens[\Big]{ (x \rightarrow y) \wedge (x \rightarrow z) } \leq y \wedge z.
  \end{equation*}

  Hence,
  \begin{equation}\label{eq:def:heyting_algebra/proof/distributive_backward}
    (x \rightarrow y) \wedge (x \rightarrow z) \leq x \rightarrow (y \wedge z).
  \end{equation}

  Conversely, we have
  \begin{equation*}
    x \wedge (x \rightarrow (y \wedge z)) \leq y \wedge z \leq y,
  \end{equation*}
  thus
  \begin{equation*}
    x \rightarrow (y \wedge z) \leq (x \rightarrow y)
  \end{equation*}
  and similarly
  \begin{equation*}
    x \rightarrow (y \wedge z) \leq (x \rightarrow z).
  \end{equation*}

  Therefore,
  \begin{equation}\label{eq:def:heyting_algebra/proof/distributive_forward}
    x \rightarrow (y \wedge z) \leq (x \rightarrow y) \wedge (x \rightarrow z).
  \end{equation}

  Combining \eqref{eq:def:heyting_algebra/proof/distributive_backward} and \eqref{eq:def:heyting_algebra/proof/distributive_forward}, we obtain \eqref{eq:def:heyting_algebra/axioms/distributive}.

  \ImplicationSubProof{def:heyting_algebra/axioms}{def:heyting_algebra/ideal} Suppose that the axioms from \fullref{def:heyting_algebra/axioms} hold.

  First let \( c \leq (x \rightarrow y) \). \Fullref{thm:def:lattice/operations_preserve_order} implies that
  \begin{equation*}
    x \wedge c
    \leq
    x \wedge (x \rightarrow y)
    \reloset {\eqref{eq:def:heyting_algebra/axioms/modus_ponens}} =
    x \wedge y
    \leq
    y.
  \end{equation*}

  Conversely, if \( x \wedge c \leq y \), we have
  \begin{equation*}
    c
    \reloset {\eqref{eq:def:heyting_algebra/axioms/modular}} =
    c \wedge (x \rightarrow c)
    \leq
    \top \wedge (x \rightarrow c)
    \reloset {\eqref{eq:def:heyting_algebra/axioms/self}} =
    (x \rightarrow x) \wedge (x \rightarrow c)
    \reloset {\eqref{eq:def:heyting_algebra/axioms/distributive}} =
    x \rightarrow (\underbrace{x \wedge c}_{\leq y})
    \leq
    x \rightarrow y.
  \end{equation*}
\end{defproof}

\begin{proposition}\label{thm:def:heyting_algebra}
  \hyperref[def:heyting_algebra]{Heyting algebras} have the following basic properties:
  \begin{thmenum}
    \thmitem{thm:def:heyting_algebra/distributive} Every Heyting algebra is a \hyperref[def:distributive_lattice]{distributive lattice}.
    \thmitem{thm:def:heyting_algebra/leq} We always have \( y \leq (x \rightarrow y) \).
  \end{thmenum}
\end{proposition}
\begin{proof}
  \SubProofOf{thm:def:heyting_algebra/distributive} \Fullref{ex:heyting_conditional_ideal/pentagon} implies that the pentagon lattice \eqref{eq:ex:def:modular_lattice/pentagon} is not a Heyting algebra, and \fullref{ex:heyting_conditional_ideal/diamond} implies that the diamond lattice \eqref{eq:ex:def:distributive_lattice/diamond} is not a Heyting algebra. Therefore, no Heyting algebra contains as a sublattice either \( N_5 \) or \( M_3 \).

  Then \fullref{thm:distributive_lattice_characterization} implies that Heyting algebras are distributive.

  \SubProofOf{thm:def:heyting_algebra/leq} Clearly \( y \in H_{x,y} \) because \( x \wedge y \leq y \).
\end{proof}

\begin{proposition}\label{thm:complete_heyting_algebra}
  A \hyperref[def:complete_lattice]{complete lattice} is a \hyperref[def:heyting_algebra]{Heyting algebra} if and only if the following infinite distributive law holds:
  \begin{equation}\label{eq:thm:complete_heyting_algebra/infinite_distributive}
    x \wedge \parens[\Big]{ \bigvee_{k \in \mscrK} y_k } = \bigvee_{k \in \mscrK} (x \wedge y_k).
  \end{equation}

   Furthermore, we have
  \begin{equation}\label{eq:thm:complete_heyting_algebra/relative_pseudocomplement}
    (x \rightarrow y) \coloneqq \bigvee\set{ c \in X \given x \wedge c \leq y }.
  \end{equation}
\end{proposition}
\begin{proof}
  Fix a complete lattice \( X \).

  \SufficiencySubProof Suppose that \( X \) is a Heyting algebra.

  Fix some element \( y \) and an arbitrary indexed family \( \seq{ x_k }_{k \in \mscrK} \) from \( X \). We have, for every \( a \) in \( X \),
  \begin{equation}\label{eq:thm:complete_heyting_algebra/proof/first}
    x \wedge \parens[\Big]{ \bigvee_{k \in \mscrK} y_k } \leq a
  \end{equation}
  if and only if
  \begin{equation*}
    \bigvee_{k \in \mscrK} y_k \leq (x \rightarrow a)
  \end{equation*}
  if and only if, for every \( k \in \mscrK \),
  \begin{equation*}
    y_k \leq (x \rightarrow a)
  \end{equation*}
  if and only if, for every \( k \in \mscrK \),
  \begin{equation}\label{eq:thm:complete_heyting_algebra/proof/last}
    x \wedge y_k \leq a.
  \end{equation}

  Taking \( a \) to be \( \bigvee_{k \in \mscrK} (x \wedge y_k) \), we obtain a true statement in \eqref{eq:thm:complete_heyting_algebra/proof/last}, hence \eqref{eq:thm:complete_heyting_algebra/proof/first} holds:
  \begin{equation*}
    \parens[\Big]{ \bigvee_{k \in \mscrK} y_k } \leq \bigvee_{k \in \mscrK} (x \wedge y_k).
  \end{equation*}

  The converse follows from \fullref{thm:def:complete_lattice/distributive_inequality}.

  \NecessitySubProof Suppose that \eqref{eq:thm:complete_heyting_algebra/infinite_distributive} holds. Consider the definition \eqref{eq:thm:complete_heyting_algebra/relative_pseudocomplement}.

  Obviously \( a \wedge x \leq y \) implies \( a \leq (x \rightarrow y) \).

  Conversely, suppose that \( a \leq (x \rightarrow y) \). Then
  \begin{equation*}
    x \wedge a
    \leq
    x \wedge (x \rightarrow y)
    =
    x \wedge \bigvee\set{ c \in X \given x \wedge c \leq y }
    \reloset {\eqref{eq:thm:complete_heyting_algebra/infinite_distributive}} =
    \bigvee\set{ x \wedge c \given x \wedge c \leq y }
    =
    y.
  \end{equation*}

  Generalizing on \( x \) and \( y \), we conclude that \( {\rightarrow} \) is indeed a relative pseudocomplement and thus \( X \) is a Heyting algebra.
\end{proof}

\begin{example}\label{ex:def:heyting_algebra}
  We list examples of \hyperref[def:heyting_algebra]{Heyting algebras}:
  \begin{thmenum}
    \thmitem{ex:def:heyting_algebra/lindenbaum_tarski} As shown in \fullref{thm:intuitionistic_lindenbaum_tarski_algebra}, every \hyperref[def:lindenbaum_tarski_algebra]{Lindenbaum-Tarski algebra} for the \hyperref[def:intuitionistic_propositional_deductive_systems]{intuitionistic propositional deduction system} is a Heyting algebra.

    \thmitem{ex:def:heyting_algebra/topology} The topology \( \mscrT \) of a \hyperref[def:topological_space]{topological space} \( (X, \mscrT) \) is a complete Heyting algebra.

    Indeed,
    \begin{itemize}
      \item \hyperref[def:lattice/join]{Arbitrary joins} are given by \hyperref[def:basic_set_operations/union]{unions}.
      \item \hyperref[def:lattice/meet]{Finite meets} are given by \hyperref[def:basic_set_operations/intersection]{intersections}.
      \item The \hyperref[def:extremal_points/top_and_bottom]{top element} is the entire domain \( L \).
      \item The \hyperref[def:extremal_points/top_and_bottom]{bottom element} is the empty set.
      \item The \hyperref[def:heyting_algebra]{relative pseudocomplement} \( U \rightsquigarrow V \) is then
      \begin{equation*}
        \bigcup\set[\Big]{ A \in T \given \underbrace{A \cap U}_{A \setminus (X \setminus U)} \subseteq V }
        =
        \bigcup\set[\Big]{ A \in T \given A \subseteq V \cup (X \setminus U) }
        =
        \Int((X \setminus U) \cup V),
      \end{equation*}
      which is actually similar to \fullref{thm:boolean_equivalences/conditional_as_disjunction} despite the fact that arbitrary topologies are not Boolean algebras.

      \item As a result, the \hyperref[def:heyting_algebra/pseudocomplement]{pseudocomplement} is
      \begin{equation*}
        \widetilde U = \Int(X \setminus U).
      \end{equation*}
    \end{itemize}

    This is actually used in topological semantics --- see \fullref{def:propositional_topological_semantics}.
  \end{thmenum}
\end{example}

\paragraph{Boolean algebra}

\begin{definition}\label{def:bounded_lattice_complement}\mcite[16]{Birkhoff1967}
  In a \hyperref[def:extremal_points/bounds]{bounded} \hyperref[def:lattice]{lattice}, a \term[ru=дополнение (\cite[def. 1.1]{Гуров2013})]{complement} of an element \( x \) is another element \( y \) such that \( x \wedge y = \bot \) and \( x \vee y = \top \).
\end{definition}

\begin{proposition}\label{thm:distributive_bounded_lattice_unique_complement}
  In a \hyperref[def:extremal_points/bounds]{bounded} \hyperref[def:distributive_lattice]{distributive lattice}, each element has at most one complement.
\end{proposition}
\begin{proof}
  If \( y \) and \( z \) are both complements of \( x \), then
  \begin{balign*}
    y
    &\reloset {\eqref{eq:thm:def:lattice/bounded_absorption/meet}} =
    y \wedge \top
    = \\ &=
    y \wedge (z \vee x)
    = \\ &\reloset {\eqref{eq:def:distributive_lattice/meet_over_join}} =
    (y \wedge z) \vee (y \wedge x)
    = \\ & =
    y \wedge z
    = \\ & =
    (x \wedge z) \vee (y \wedge z)
    = \\ &\reloset {\eqref{eq:def:distributive_lattice/meet_over_join}} =
    (x \vee y) \wedge z
    = \\ & =
    z.
  \end{balign*}
\end{proof}

\begin{definition}\label{def:boolean_algebra}\mcite[18]{Birkhoff1967}
  A \term[ru=булева алгебра (\cite[def. 1.1]{Гуров2013})]{Boolean algebra} is a \hyperref[def:extremal_points/bounds]{bounded} \hyperref[def:distributive_lattice]{distributive lattice} with an additional unary operation \( {\oline \anon} \), such that \( {\oline x} \) is a \hyperref[def:bounded_lattice_complement]{complement} of \( x \).

  Existence of the complement is provided by the operation itself, while uniqueness follows from \fullref{thm:distributive_bounded_lattice_unique_complement}.

  Boolean algebras have the following metamathematical properties:
  \begin{thmenum}[resume=def:boolean_algebra]
    \thmitem{def:boolean_algebra/theory} We extend the \hyperref[def:lattice/theory]{first-order theory of lattices} by adding:
    \begin{itemize}
      \item The nullary functional symbols \( \top \) and \( \bot \) along with the axioms \eqref{eq:thm:def:lattice/bounded_absorption/join} and \eqref{eq:thm:def:lattice/bounded_absorption/meet}.

      \item The unary functional symbol \( {\oline {\anon}} \) along with the axioms
      \begin{subequations}
        \begin{align}
          \xi \vee \oline \xi \syneq \top, \label{eq:def:boolean_algebra/join} \\
          \xi \wedge \oline \xi \syneq \bot. \label{eq:def:boolean_algebra/meet}
        \end{align}
      \end{subequations}
    \end{itemize}

    \thmitem{def:boolean_algebra/submodel}\mcite[18]{Birkhoff1967} In addition to containing the joins and meets of all its members, a \hyperref[def:first_order_submodel]{first-order submodel} of a Boolean algebra must also contain the complement of each of its members. \fnote{Since the axioms \eqref{eq:def:boolean_algebra/join} and \eqref{eq:def:boolean_algebra/meet} hold, it is redundant to require that a Boolean subalgebra contains the top and bottom elements}. We call such submodels \term{Boolean subalgebras}.

    \thmitem{def:boolean_algebra/homomorphism} A \hyperref[def:first_order_homomorphism]{first-order homomorphism} between Boolean algebras is a lattice homomorphism that preserve top and bottom elements. Complements are automatically preserved, as we shall see in \fullref{thm:distributive_bounded_lattice_unique_complement}.

    \thmitem{def:boolean_algebra/opposite} The \hyperref[def:lattice/opposite]{dual lattice} of a Boolean algebra is again a Boolean algebra. We will call it the \term{dual Boolean algebra}.

    \thmitem{def:boolean_algebra/category} We denote the \hyperref[def:category_of_small_first_order_models]{category of \( \mscrU \)-small models} for Boolean algebras via \( \cat{Bool} \).
  \end{thmenum}
\end{definition}

\begin{example}\label{ex:def:boolean_algebra}
  We list examples of \hyperref[def:boolean_algebra]{Boolean algebras}:

  \begin{thmenum}
    \thmitem{ex:def:boolean_algebra/lindenbaum_tarski} As shown in \fullref{thm:intuitionistic_lindenbaum_tarski_algebra}, every \hyperref[def:lindenbaum_tarski_algebra]{Lindenbaum-Tarski algebra} for the \hyperref[def:classical_propositional_deductive_systems]{classical propositional deductive system} is a Boolean algebra.

    \thmitem{ex:def:boolean_algebra/f2} The \hyperref[def:finite_field]{finite field} \( \BbbF_2 = \set{ 0, 1 } \) is a Boolean algebra. Indeed, it is clearly a bounded lattice, and since it doesn't contain neither the pentagon lattice \eqref{eq:ex:def:modular_lattice/pentagon} nor the diamond lattice \eqref{eq:ex:def:distributive_lattice/diamond}, \fullref{thm:distributive_lattice_characterization} implies that \( \BbbF_2 \) is distributive. Finally, the complement operation can be defined in the obvious way --- by exchanging the two elements.

    \Fullref{thm:two_element_lattice} implies that every two-element lattice is isomorphic to \( \BbbF_2 \).

    \thmitem{ex:def:boolean_algebra/power_set} The power set of any set is a \hyperref[def:complete_lattice]{complete} Boolean algebra --- see \fullref{thm:boolean_algebra_of_subsets}.
  \end{thmenum}
\end{example}

\begin{proposition}\label{thm:def:boolean_algebra}
  \hyperref[def:boolean_algebra]{Boolean algebras} have the following basic properties:
  \begin{thmenum}
    \thmitem{thm:def:boolean_algebra/involution} Complementation is an \hyperref[def:involution]{involution}.

    \thmitem{thm:def:boolean_algebra/opposite_complement} The complementation operation in a Boolean algebra coincides with complementation in its \hyperref[def:boolean_algebra/opposite]{dual}.

    \thmitem{thm:def:boolean_algebra/heyting} Every Boolean algebra is a \hyperref[def:heyting_algebra]{Heyting algebra}. Furthermore, we have the following variation of \eqref{eq:thm:boolean_equivalences/conditional_as_disjunction}:
    \begin{equation*}
      (x \rightarrow y) \coloneqq \oline {x} \vee y.
    \end{equation*}

    \thmitem{thm:def:boolean_algebra/distributive}\mcite[lemma 162]{Gratzer2011} In a \hyperref[def:complete_lattice]{complete} Boolean algebra, for any element \( x \) and any family \( \seq{ y_k }_{k \in \mscrK} \), we have
    \begin{subequations}
      \begin{align}
        x \vee \parens[\Big]{ \bigwedge_{k \in \mscrK} y_k } &= \bigwedge_{k \in \mscrK} (x \vee y_k), \label{eq:thm:def:boolean_algebra/distributive/join_over_meet} \\
        x \wedge \parens[\Big]{ \bigvee_{k \in \mscrK} y_k } &= \bigvee_{k \in \mscrK} (x \wedge y_k). \label{eq:thm:def:boolean_algebra/distributive/meet_over_join}
      \end{align}
    \end{subequations}
  \end{thmenum}
\end{proposition}
\begin{proof}
  \SubProofOf{thm:def:boolean_algebra/involution} We have
  \begin{equation*}
    x
    \reloset {\eqref{eq:thm:lattice_operation_characterization/compatibility/meet}} =
    x \wedge \top
    \reloset {\eqref{eq:def:boolean_algebra/join}} =
    x \wedge (\oline x \vee \doline x)
    \reloset {\eqref{eq:def:distributive_lattice/meet_over_join}} =
    (x \wedge \oline x) \vee (x \wedge \doline x)
    \reloset {\eqref{eq:def:boolean_algebra/meet}} =
    \bot \vee (x \wedge \doline x)
    \reloset {\eqref{eq:thm:lattice_operation_characterization/compatibility/join}} =
    x \wedge \doline x
    \reloset {\eqref{eq:thm:lattice_operation_characterization/compatibility/meet}} \leq
    \doline x.
  \end{equation*}

  Analogously,
  \begin{equation*}
    \doline x
    \reloset {\eqref{eq:thm:lattice_operation_characterization/compatibility/meet}} =
    \doline x \wedge \top
    \reloset {\eqref{eq:def:boolean_algebra/join}} =
    \doline x \wedge (x \vee \oline x)
    \reloset {\eqref{eq:def:distributive_lattice/meet_over_join}} =
    (\doline x \wedge x) \vee (\doline x \wedge \oline x)
    \reloset {\eqref{eq:def:boolean_algebra/meet}} =
    (\doline x \wedge x) \vee \bot
    \reloset {\eqref{eq:thm:lattice_operation_characterization/compatibility/join}} =
    \doline x \wedge x
    \reloset {\eqref{eq:thm:lattice_operation_characterization/compatibility/meet}} \leq
    x.
  \end{equation*}

  Therefore,
  \begin{equation*}
    x = \doline x
  \end{equation*}

  \SubProofOf{thm:def:boolean_algebra/opposite_complement} Note that \eqref{eq:def:boolean_algebra/join} in a Boolean algebra corresponds to \eqref{eq:def:boolean_algebra/meet} in its dual.

  \SubProofOf{thm:def:boolean_algebra/heyting} We will show that
  \begin{equation*}
    c \leq \oline {x} \vee y \T{if and only if} x \wedge c \leq y.
  \end{equation*}

  First, if \( c \leq \oline x \vee y \), we have
  \begin{equation*}
    x \wedge c
    \leq
    x \wedge (\oline x \vee y)
    \reloset {\eqref{eq:def:distributive_lattice/meet_over_join}} =
    (\underbrace{x \wedge \oline x}_{\bot}) \vee (x \wedge y)
    =
    x \wedge y
    \leq
    y.
  \end{equation*}

  Conversely, if \( x \wedge c \leq y \), we have
  \begin{equation*}
    \oline x \vee (x \wedge c) \leq \oline x \vee y,
  \end{equation*}
  which again due to distributivity implies
  \begin{equation*}
    (\underbrace{\oline x \vee x}_{\top}) \wedge (\oline x \vee c) \leq \oline x \vee y
  \end{equation*}
  and
  \begin{equation*}
    c \leq \oline x \vee c \leq \oline x \vee y.
  \end{equation*}

  \SubProofOf{thm:def:boolean_algebra/distributive} Let \( X \) be a Boolean algebra. \Fullref{thm:def:boolean_algebra/heyting} implies that it is a Heyting algebra, hence \eqref{eq:thm:def:boolean_algebra/distributive/join_over_meet} holds as a restatement of \eqref{eq:thm:complete_heyting_algebra/infinite_distributive}. Furthermore, the opposite Boolean algebra \( X^\oppos \) is also a Heyting algebra and \eqref{eq:thm:complete_heyting_algebra/infinite_distributive} holds in \( X^\oppos \), hence \eqref{eq:thm:def:boolean_algebra/distributive/join_over_meet} holds in \( X \).
\end{proof}

\begin{theorem}[De Morgan's laws]\label{thm:de_morgans_laws}
  In a \hyperref[def:boolean_algebra]{Boolean algebra}, the following hold for any finite family \( \set{ x_k }_{k \in \mscrK} \):
  \begin{subequations}
    \begin{align}
      \oline{\bigvee_{k \in \mscrK} x_k}   &= \bigwedge_{k \in \mscrK} \oline{x_k}, \label{eq:thm:de_morgans_laws/complement_of_join} \\
      \oline{\bigwedge_{k \in \mscrK} x_k} &= \bigvee_{k \in \mscrK} \oline{x_k}.   \label{eq:thm:de_morgans_laws/complement_of_meet}
    \end{align}
  \end{subequations}

  If the algebra is a \hyperref[def:complete_lattice]{complete lattice}, \( \mscrK \) can be any family, not necessarily finite.
\end{theorem}
\begin{comments}
  \item See also the syntactic analog, \fullref{thm:boolean_equivalences/de_morgan}.
\end{comments}
\begin{proof}
  We will show that \( \bigwedge_{m \in \mscrK} \oline{x_m} \) is the complement of \( \bigvee_{k \in \mscrK} x_k \). Indeed, we have
  \begin{equation*}
    \parens*{ \bigvee_{k \in \mscrK} x_k } \vee \parens*{ \bigwedge_{m \in \mscrK} \oline{x_m} }
    \reloset {\eqref{eq:thm:def:boolean_algebra/distributive/join_over_meet}} =
    \bigwedge_{m \in \mscrK} \parens*{ \bigvee_{k \in \mscrK} x_k } \vee \oline {x_m}
    =
    \bigwedge_{m \in \mscrK} \parens*{ \bigvee_{k \neq m} x_k } \vee \underbrace{x_m \vee \oline {x_m}}_{\top}
    =
    \bigwedge_{m \in \mscrK} \top
    =
    \top
  \end{equation*}
  and,
  \begin{equation*}
    \parens*{ \bigwedge_{m \in \mscrK} \oline{x_m} } \wedge \parens*{ \bigvee_{k \in \mscrK} x_k }
    \reloset {\eqref{eq:thm:def:boolean_algebra/distributive/meet_over_join}} =
    \bigvee_{k \in \mscrK} \parens*{ \bigwedge_{m \in \mscrK} x_m } \wedge \oline {x_k}
    =
    \bigvee_{k \in \mscrK} \parens*{ \bigwedge_{m \neq k} x_m } \wedge \underbrace{x_k \wedge \oline {x_k}}_{\bot}
    =
    \bigvee_{k \in \mscrK} \bot
    =
    \bot.
  \end{equation*}

  This demonstrates \eqref{eq:thm:de_morgans_laws/complement_of_join}. We can analogously prove \eqref{eq:thm:de_morgans_laws/complement_of_meet}.
\end{proof}

\begin{theorem}[Principle of duality for Boolean algebras]\label{thm:boolean_algebra_duality}\mcite{Gottschalk1953}
  Consider the \hyperref[def:lattice/theory]{first-order theory of Boolean algebras}. Within it, consider the \hyperref[def:first_order_syntax/closed_formula]{closed formula} \( \varphi \). Denote by \( \varphi^C \) the dual formula in the sense of \fullref{thm:lattice_duality}, in which we swap all connectives --- all instances \( \vee \) and \( \wedge \), as well as \( \leq \) and \( \geq \). Denote by \( \varphi^V \) the formula obtained from \( \varphi \) by swapping each variable with its complement.

  If every Boolean algebra \hyperref[def:first_order_model]{satisfies} \( \varphi \), then every Boolean algebra also satisfies \( \varphi^C \), \( \varphi^V \) and \( \varphi^{CV} \).

  More generally, the following are equivalent for a Boolean algebra \( X \):
  \begin{TwoColumns}
    \begin{itemize}
      \item \( X \) satisfies \( \varphi \).
      \item \( X \) satisfies \( \varphi^{CV} \).
    \end{itemize}
    \BeginSecondColumn
    \begin{itemize}
      \item \( X^\oppos \) satisfies \( \varphi^C \).
      \item \( X^\oppos \) satisfies \( \varphi^V \).
    \end{itemize}
  \end{TwoColumns}
\end{theorem}
\begin{comments}
  \item Similar statements hold more generally --- see \fullref{thm:preorder_duality} and \fullref{thm:lattice_duality}.

  \item This result is richer than the aforementioned ones --- \incite{Gottschalk1953} refers to it as \enquote{quaterniality} rather than \enquote{duality}.
\end{comments}
\begin{proof}
  \Fullref{thm:def:boolean_algebra/opposite_complement} implies that complementation coincides in \( X \) and \( X^\oppos \), thus, if \( X \) satisfies \( \varphi \), then, as in \fullref{thm:lattice_duality}, \( X^\oppos \) satisfies \( \varphi^C \).

  On the other hand, a \hyperref[def:first_order_valuation/variable_assignment]{variable assignment} in \( X \) corresponds to the valuation of the complemented variables in \( X^\oppos \), thus \( X \) satisfies \( \varphi \) if and only if \( X^\oppos \) satisfies \( \varphi^V \).

  Finally, \( X \) satisfies \( \varphi \) if and only if \( X^\oppos \) satisfies \( \varphi^C \) if and only if \( X = (X^\oppos)^\oppos \) satisfies \( \varphi^{CV} \).
\end{proof}

\paragraph{Ultrafilters}

\begin{remark}\label{rem:boolean_algebra_ideal}
  \Fullref{thm:lattice_ideal_as_semiring_ideal} demonstrates that a subset of a Boolean algebra is a \hyperref[def:lattice_ideal]{lattice ideal} (resp. filter) if and only if it is a \hyperref[def:semiring_ideal]{semiring ideal} of the \hyperref[ex:def:semiring/lattice]{join-meet semiring} (resp. meet-join semiring).
\end{remark}

\begin{proposition}\label{thm:improper_boolean_ideal}
  The only \hyperref[def:lattice_ideal]{lattice ideal} or \hyperref[def:lattice_ideal]{filter} in a \hyperref[def:boolean_algebra]{Boolean algebra} that contains both an element and its complement is the algebra itself.
\end{proposition}
\begin{proof}
  Suppose that some ideal \( I \) in \( X \) contains both \( x \) and \( \oline x \). Then \( I \) must contain their join \( \top \), and since \( I \) is closed under arbitrary meets, for every element \( y \) of the \( X \), \( I \) must contain \( y \wedge \top = y \). Therefore, \( I = X \).

  The proof for filters follows via \fullref{thm:boolean_algebra_duality}.
\end{proof}

\begin{proposition}\label{thm:boolean_prime_iff_maximal}
  A \hyperref[def:lattice_ideal]{lattice ideal} or \hyperref[def:lattice_ideal]{filter} in a \hyperref[def:boolean_algebra]{Boolean algebra} is \hyperref[def:lattice_ideal/prime]{prime} if and only if it is \hyperref[def:lattice_ideal/maximal]{maximal}.
\end{proposition}
\begin{proof}
  We will consider only ideals. The proof for filters follows via \fullref{thm:boolean_algebra_duality}.

  \SufficiencySubProof Let \( P \) be a prime ideal. \Fullref{thm:maximal_ideal_theorem} gives us a maximal ideal \( M \) containing \( I \).

  Suppose that \( M \) has some element \( x \) not in \( P \). Then \( x \wedge \oline x = \bot \) is in \( P \), thus \( \oline x \) must also be in \( P \) because the latter is prime. Then \( M \) contains both \( x \) and \( \oline x \), and \fullref{thm:improper_boolean_ideal} implies that \( M \) coincides with the ambient Boolean algebra. But \( M \) must be proper, giving us a contradiction with the existence of \( x \).

  Therefore, \( M \) and \( P \) coincide, thus \( P \) is maximal.

  \NecessitySubProof Follows from \fullref{thm:lattice_ideal_as_semiring_ideal} and \fullref{thm:def:semiring_ideal/maximal_is_prime}.
\end{proof}

\begin{definition}\label{def:ultrafilter}
  We say that a proper \hyperref[def:lattice_ideal]{filter} \( F \) in a \hyperref[def:boolean_algebra]{Boolean algebra} is an \term[bg=ултрафилтер (\cite[18]{Проданов1982}), ru=ультрафильтр (\cite[182]{Гуров2013})]{ultrafilter} if any of the following equivalent conditions hold:
  \begin{thmenum}
    \thmitem{def:ultrafilter/direct}\mcite[182]{Гуров2013} For every algebra element \( x \), either\fnote{Both \( x \in F \) and \( \oline x \in F \) cannot hold because of \fullref{thm:improper_boolean_ideal}.} \( x \in F \) or \( \oline x \in F \).

    \thmitem{def:ultrafilter/prime} \( F \) is a \hyperref[def:lattice_ideal/prime]{prime filter}.

    \thmitem{def:ultrafilter/maximal}\mcite[233]{DaveyPriestley2002} \( F \) is a \hyperref[def:lattice_ideal/maximal]{maximal filter}.
  \end{thmenum}
\end{definition}
\begin{proof}
  \ImplicationSubProof{def:ultrafilter/direct}{def:ultrafilter/prime} Suppose that, for every algebra element \( x \), either \( x \in F \) or \( \oline x \in F \).

  Let \( x \vee y \in F \). If \( x \not\in F \), then \( \oline x \in F \) and hence the following is also a member of \( F \):
  \begin{equation*}
    \oline x \vee (x \vee y)
    =
    (\oline x \vee x) \vee y
    =
    \top \vee y
    =
    y.
  \end{equation*}

  Hence, if \( x \not\in F \), then \( y \in F \).

  Since \( x \) was chosen arbitrarily, we conclude that \( F \) is a prime filter.

  \ImplicationSubProof{def:ultrafilter/prime}{def:ultrafilter/maximal} Follows from \fullref{thm:boolean_prime_iff_maximal}.

  \EquivalenceSubProof{def:ultrafilter/maximal}{def:ultrafilter/direct} Let \( F \) be a maximal ideal. Fix an arbitrary algebra element \( x \). Either \( x \) is in \( F \), or is in the complement of \( F \), in which case \fullref{thm:improper_boolean_ideal} implies that \( \oline x \) is in \( F \).
\end{proof}

\begin{definition}\label{def:principal_ultrafilter}\mcite[example 1.6.11]{Hinman2005}
  Fix an arbitrary \hyperref[def:set]{set} \( A \) and consider its \hyperref[thm:boolean_algebra_of_subsets]{Boolean algebra of subsets} \( \pow(A) \).

  For every element \( x \) of \( A \), the following family is an \hyperref[def:ultrafilter]{ultrafilter} in \( \pow(A) \):
  \begin{equation*}
    \mscrF_x \coloneqq \set{ B \subseteq A \given x \in B }.
  \end{equation*}

  We call \( \mscrF_x \) the \term{principal ultrafilter} of \( x \).
\end{definition}
\begin{comments}
  \item \incite[234]{DaveyPriestley2002} define \enquote{principal ultrafilters} to be maximal \hyperref[def:lattice_ideal/principal]{principal filters}, which notion differs from our definition.
\end{comments}

\begin{lemma}[Ultrafilter lemma]\label{thm:ultrafilter_lemma}
  Every proper \hyperref[def:lattice_ideal]{filter} in a \hyperref[def:boolean_algebra]{Boolean algebra} is contained in an \hyperref[def:ultrafilter]{ultrafilter}.
\end{lemma}
\begin{proof}
  Follows from \fullref{thm:lattice_ideal_as_semiring_ideal} and \fullref{thm:maximal_ideal_theorem}.
\end{proof}
