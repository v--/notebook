\section{Boolean algebras}\label{sec:boolean_algebras}

\paragraph{Boolean algebra}

\begin{definition}\label{def:bounded_lattice_complement}\mcite[16]{Birkhoff1967LatticeTheory}
  In a \hyperref[def:bounded_lattice]{bounded lattice}, a \term[ru=дополнение (\cite[def. 1.1]{Гуров2013ТеорияРешёток})]{complement} of an element \( x \) is another element \( y \) such that \( x \wedge y = \bot \) and \( x \vee y = \top \).
\end{definition}

\begin{proposition}\label{thm:distributive_bounded_lattice_unique_complement}
  In a \hyperref[def:extremal_points/bounds]{bounded} \hyperref[def:distributive_lattice]{distributive lattice}, each element has at most one complement.
\end{proposition}
\begin{proof}
  If \( y \) and \( z \) are both complements of \( x \), then
  \begin{balign*}
    y
    &\reloset {\eqref{eq:def:bounded_lattice/theory/bot}} =
    y \wedge \top
    = \\ &=
    y \wedge (z \vee x)
    = \\ &\reloset {\eqref{eq:def:distributive_lattice/meet_over_join}} =
    (y \wedge z) \vee (y \wedge x)
    = \\ & =
    y \wedge z
    = \\ & =
    (x \wedge z) \vee (y \wedge z)
    = \\ &\reloset {\eqref{eq:def:distributive_lattice/meet_over_join}} =
    (x \vee y) \wedge z
    = \\ & =
    z.
  \end{balign*}
\end{proof}

\begin{definition}\label{def:boolean_algebra}\mcite[18]{Birkhoff1967LatticeTheory}
  A \term[ru=булева алгебра (\cite[def. 1.1]{Гуров2013ТеорияРешёток})]{Boolean algebra} is a \hyperref[def:extremal_points/bounds]{bounded} \hyperref[def:distributive_lattice]{distributive lattice} with an additional unary operation \( {\oline \anon} \), such that \( {\oline x} \) is a \hyperref[def:bounded_lattice_complement]{complement} of \( x \).

  Existence of the complement is provided by the operation itself, while uniqueness follows from \cref{thm:distributive_bounded_lattice_unique_complement}.

  Boolean algebras have the following metamathematical properties:
  \begin{thmenum}
    \thmitem{def:boolean_algebra/theory}\mimprovised We extend the \hyperref[def:lattice/theory]{first-order theory of lattices} by adding:
    \begin{itemize}
      \item The nullary functional symbols \( \top \) and \( \bot \) along with the axioms \eqref{eq:def:bounded_lattice/theory/top} and \eqref{eq:def:bounded_lattice/theory/bot}.

      \item The unary functional symbol \( {\oline {\anon}} \) along with the axioms
      \begin{align}
        \xi \vee \oline \xi \syneq \top, \label{eq:def:boolean_algebra/join} \\
        \xi \wedge \oline \xi \syneq \bot. \label{eq:def:boolean_algebra/meet}
      \end{align}
    \end{itemize}

    \thmitem{def:boolean_algebra/submodel}\mcite[18]{Birkhoff1967LatticeTheory} In addition to containing the joins and meets of all its members, a \hyperref[def:first_order_submodel]{first-order submodel} of a Boolean algebra must also contain the complement of each of its members. We call such submodels \term{Boolean subalgebras}.

    See \cref{thm:boolean_subalgebra} for simplified equivalent conditions.

    \thmitem{def:boolean_algebra/homomorphism} A \hyperref[def:first_order_homomorphism]{first-order homomorphism} between Boolean algebras is a lattice homomorphism that preserve top and bottom elements. Complements are automatically preserved, as we shall see in \cref{thm:distributive_bounded_lattice_unique_complement}.

    \thmitem{def:boolean_algebra/opposite} The \hyperref[def:lattice/opposite]{dual lattice} of a Boolean algebra is again a Boolean algebra. We will call it the \term{dual Boolean algebra}.

    \thmitem{def:boolean_algebra/category}\mcite[\S I.3.2]{Johnstone1982StoneSpaces} We denote the \hyperref[def:category_of_small_first_order_models]{category of \( \mscrU \)-small models} for Boolean algebras via \( \cat{Bool} \).
  \end{thmenum}
\end{definition}

\begin{example}\label{ex:def:boolean_algebra}
  We list examples of \hyperref[def:boolean_algebra]{Boolean algebras}:

  \begin{thmenum}
    \thmitem{ex:def:boolean_algebra/lindenbaum_tarski} As shown in \cref{inf:def:propositional_natural_deduction_systems/bot/dne}, every \hyperref[def:lindenbaum_tarski_algebra]{Lindenbaum-Tarski algebra} for the \hyperref[def:propositional_natural_deduction_systems]{classical natural deduction system} is a Boolean algebra.

    \thmitem{ex:def:boolean_algebra/f2} The \hyperref[def:finite_field]{finite field} \( \BbbF_2 = \set{ 0, 1 } \) is a Boolean algebra. Indeed, it is clearly a bounded lattice, and since it doesn't contain neither the pentagon lattice \eqref{eq:ex:def:modular_lattice/pentagon} nor the diamond lattice \eqref{eq:ex:def:distributive_lattice/diamond}, \cref{thm:distributive_lattice_characterization} implies that \( \BbbF_2 \) is distributive. Finally, the complement operation can be defined in the obvious way --- by exchanging the two elements.

    \Cref{thm:two_element_lattice} implies that every two-element lattice is isomorphic to \( \BbbF_2 \).

    \thmitem{ex:def:boolean_algebra/power_set} The power set of any set is a \hyperref[def:complete_lattice]{complete} Boolean algebra --- see \cref{thm:boolean_algebra_of_subsets}.

    \thmitem{ex:def:boolean_algebra/interval} For every \hyperref[def:real_numbers]{real number} \( b \), the \hyperref[def:order_interval/closed]{closed interval} \( [0, b] \) is a Boolean algebra.

    \Cref{thm:def:lattice/closed_interval} implies that \( [0, b] \) is a sublattice of \( \BbbR \). We can define the complement of \( x \) in \( [0, b] \) as \( b - x \).

    This is indeed a complement in the sense of \cref{def:bounded_lattice_complement} because \( \min\set{ x, a + (b - x) } \).
  \end{thmenum}
\end{example}

\begin{theorem}[De Morgan's laws]\label{thm:de_morgans_laws}
  In a \hyperref[def:boolean_algebra]{Boolean algebra}, the following hold for any finite family \( \set{ x_k }_{k \in \mscrK} \):
  \begin{align}
    \oline{\bigvee_{k \in \mscrK} x_k}   &= \bigwedge_{k \in \mscrK} \oline{x_k}, \label{eq:thm:de_morgans_laws/complement_of_join} \\
    \oline{\bigwedge_{k \in \mscrK} x_k} &= \bigvee_{k \in \mscrK} \oline{x_k}.   \label{eq:thm:de_morgans_laws/complement_of_meet}
  \end{align}

  If the algebra is a \hyperref[def:complete_lattice]{complete lattice}, \( \mscrK \) can be any family, not necessarily finite.
\end{theorem}
\begin{comments}
  \item See \cref{rem:de_morgans_laws} for a list of related theorems.
\end{comments}
\begin{proof}
  The equality \eqref{eq:thm:de_morgans_laws/complement_of_join} is a restatement of \eqref{eq:thm:de_morgans_laws_for_heyting_algebras/complement_of_join}, while \eqref{eq:thm:de_morgans_laws/complement_of_meet} follows via \fullref{thm:boolean_algebra_duality}.
\end{proof}

\begin{theorem}[Principle of duality for Boolean algebras]\label{thm:boolean_algebra_duality}\mcite{Gottschalk1953Quaternality}
  Consider the \hyperref[def:lattice/theory]{first-order theory of Boolean algebras}. Within it, consider the \hyperref[def:first_order_syntax/closed_formula]{closed formula} \( \varphi \). Denote by \( \varphi^C \) the dual formula in the sense of \fullref{thm:lattice_duality}, in which we swap all connectives --- all instances \( \vee \) and \( \wedge \), as well as \( \leq \) and \( \geq \). Denote by \( \varphi^V \) the formula obtained from \( \varphi \) by swapping each variable with its complement.

  If every Boolean algebra \hyperref[def:first_order_model]{satisfies} \( \varphi \), then every Boolean algebra also satisfies \( \varphi^C \), \( \varphi^V \) and \( \varphi^{CV} \).

  More generally, the following are equivalent for a Boolean algebra \( X \):
  \begin{paracol}{2}
    \begin{leftcolumn}
      \begin{itemize}
        \item \( X \) satisfies \( \varphi \).
        \item \( X \) satisfies \( \varphi^{CV} \).
      \end{itemize}
    \end{leftcolumn}

    \begin{rightcolumn}
      \begin{itemize}
        \item \( X^\oppos \) satisfies \( \varphi^C \).
        \item \( X^\oppos \) satisfies \( \varphi^V \).
      \end{itemize}
    \end{rightcolumn}
  \end{paracol}
\end{theorem}
\begin{comments}
  \item Similar statements hold more generally --- see \fullref{thm:preorder_duality} and \fullref{thm:lattice_duality}.

  \item This result is richer than the aforementioned ones --- \incite{Gottschalk1953Quaternality} refers to it as \enquote{quaterniality} rather than \enquote{duality}. See \cref{ex:def:group_action/formula_duality} for a discussion of how it relates to the \hyperref[def:klein_four_group]{Klein four-group}.
\end{comments}
\begin{proof}
  \Cref{thm:def:boolean_algebra/opposite_complement} implies that complementation coincides in \( X \) and \( X^\oppos \), thus, if \( X \) satisfies \( \varphi \), then, as in \fullref{thm:lattice_duality}, \( X^\oppos \) satisfies \( \varphi^C \).

  On the other hand, a \hyperref[def:first_order_valuation/variable_assignment]{variable assignment} in \( X \) corresponds to the valuation of the complemented variables in \( X^\oppos \), thus \( X \) satisfies \( \varphi \) if and only if \( X^\oppos \) satisfies \( \varphi^V \).

  Finally, \( X \) satisfies \( \varphi \) if and only if \( X^\oppos \) satisfies \( \varphi^C \) if and only if \( X = (X^\oppos)^\oppos \) satisfies \( \varphi^{CV} \).
\end{proof}

\begin{proposition}\label{thm:def:boolean_algebra}
  \hyperref[def:boolean_algebra]{Boolean algebras} have the following basic properties:
  \begin{thmenum}
    \thmitem{thm:def:boolean_algebra/involution} Complementation is an \hyperref[def:morphism_invertibility/involution]{involution}.

    \thmitem{thm:def:boolean_algebra/opposite_complement} The complementation operation in a Boolean algebra coincides with complementation in its \hyperref[def:boolean_algebra/opposite]{dual}.

    \thmitem{thm:def:boolean_algebra/heyting} Every Boolean algebra is a \hyperref[def:heyting_algebra]{Heyting algebra}. Furthermore, we have the following variation of \eqref{eq:thm:classical_equivalences/conditional_as_disjunction}:
    \begin{equation*}
      (x \rightarrow y) \coloneqq \oline {x} \vee y.
    \end{equation*}

    \thmitem{thm:def:boolean_algebra/distributive}\mcite[lemma 162]{Grätzer2011LatticeTheory} In a \hyperref[def:complete_lattice]{complete} Boolean algebra, for any element \( x \) and any family \( \seq{ y_k }_{k \in \mscrK} \), we have
    \begin{align}
      x \vee \parens[\Big]{ \bigwedge_{k \in \mscrK} y_k } &= \bigwedge_{k \in \mscrK} (x \vee y_k), \label{eq:thm:def:boolean_algebra/distributive/join_over_meet} \\
      x \wedge \parens[\Big]{ \bigvee_{k \in \mscrK} y_k } &= \bigvee_{k \in \mscrK} (x \wedge y_k). \label{eq:thm:def:boolean_algebra/distributive/meet_over_join}
    \end{align}
  \end{thmenum}
\end{proposition}
\begin{proof}
  \SubProofOf{thm:def:boolean_algebra/involution} We have
  \begin{equation*}
    x
    \reloset {\eqref{eq:thm:lattice_operation_characterization/compatibility/meet}} =
    x \wedge \top
    \reloset {\eqref{eq:def:boolean_algebra/join}} =
    x \wedge (\oline x \vee \doline x)
    \reloset {\eqref{eq:def:distributive_lattice/meet_over_join}} =
    (x \wedge \oline x) \vee (x \wedge \doline x)
    \reloset {\eqref{eq:def:boolean_algebra/meet}} =
    \bot \vee (x \wedge \doline x)
    \reloset {\eqref{eq:thm:lattice_operation_characterization/compatibility/join}} =
    x \wedge \doline x
    \reloset {\eqref{eq:thm:lattice_operation_characterization/compatibility/meet}} \leq
    \doline x.
  \end{equation*}

  Analogously,
  \begin{equation*}
    \doline x
    \reloset {\eqref{eq:thm:lattice_operation_characterization/compatibility/meet}} =
    \doline x \wedge \top
    \reloset {\eqref{eq:def:boolean_algebra/join}} =
    \doline x \wedge (x \vee \oline x)
    \reloset {\eqref{eq:def:distributive_lattice/meet_over_join}} =
    (\doline x \wedge x) \vee (\doline x \wedge \oline x)
    \reloset {\eqref{eq:def:boolean_algebra/meet}} =
    (\doline x \wedge x) \vee \bot
    \reloset {\eqref{eq:thm:lattice_operation_characterization/compatibility/join}} =
    \doline x \wedge x
    \reloset {\eqref{eq:thm:lattice_operation_characterization/compatibility/meet}} \leq
    x.
  \end{equation*}

  Therefore,
  \begin{equation*}
    x = \doline x
  \end{equation*}

  \SubProofOf{thm:def:boolean_algebra/opposite_complement} Note that \eqref{eq:def:boolean_algebra/join} in a Boolean algebra corresponds to \eqref{eq:def:boolean_algebra/meet} in its dual.

  \SubProofOf{thm:def:boolean_algebra/heyting} We will show that
  \begin{equation*}
    c \leq \oline {x} \vee y \T{if and only if} x \wedge c \leq y.
  \end{equation*}

  First, if \( c \leq \oline x \vee y \), we have
  \begin{equation*}
    x \wedge c
    \leq
    x \wedge (\oline x \vee y)
    \reloset {\eqref{eq:def:distributive_lattice/meet_over_join}} =
    (\underbrace{x \wedge \oline x}_{\bot}) \vee (x \wedge y)
    =
    x \wedge y
    \leq
    y.
  \end{equation*}

  Conversely, if \( x \wedge c \leq y \), we have
  \begin{equation*}
    \oline x \vee (x \wedge c) \leq \oline x \vee y,
  \end{equation*}
  which again due to distributivity implies
  \begin{equation*}
    (\underbrace{\oline x \vee x}_{\top}) \wedge (\oline x \vee c) \leq \oline x \vee y
  \end{equation*}
  and
  \begin{equation*}
    c \leq \oline x \vee c \leq \oline x \vee y.
  \end{equation*}

  \SubProofOf{thm:def:boolean_algebra/distributive} Let \( X \) be a Boolean algebra. \Cref{thm:def:boolean_algebra/heyting} implies that it is a Heyting algebra, hence \eqref{eq:thm:def:boolean_algebra/distributive/join_over_meet} holds as a restatement of \eqref{eq:thm:complete_heyting_algebra/infinite_distributive}. Furthermore, the opposite Boolean algebra \( X^\oppos \) is also a Heyting algebra and \eqref{eq:thm:complete_heyting_algebra/infinite_distributive} holds in \( X^\oppos \), hence \eqref{eq:thm:def:boolean_algebra/distributive/join_over_meet} holds in \( X \).
\end{proof}

\begin{proposition}\label{thm:boolean_algebra_heyting_characterization}
  A \hyperref[def:heyting_algebra]{Heyting algebra} is a \hyperref[def:boolean_algebra]{Boolean algebra} if and only if \( \doline{x} = x \) for any element.
\end{proposition}
\begin{comments}
  \item \Cref{thm:def:heyting_algebra/dni} implies that \( x \leq \doline{x} \) in any Heyting algebra, so, when using this characterization, there is only one inequality to show.
\end{comments}
\begin{proof}
  \SufficiencySubProof Follows from the definition of a Boolean algebra, considering \cref{thm:def:boolean_algebra/heyting}.
  \NecessitySubProof Suppose that \( \doline{x} = x \) for any element in a Heyting algebra. It is a bounded lattice by definition, and \cref{thm:def:heyting_algebra/distributive} implies that it is distributive. We must only show that \( \oline{x} \) is a \hyperref[def:bounded_lattice_complement]{complement} of \( x \).

  \Cref{thm:def:heyting_algebra/semicomplement} implies that \( x \wedge \oline{x} = \bot \) for any Heyting algebra. On the other hand,
  \begin{equation*}
    \oline{x \vee \oline{x}}
    \reloset {\eqref{eq:thm:de_morgans_laws_for_heyting_algebras/complement_of_meet}} \geq
    \oline{x} \wedge \doline{x}
    =
    \oline{x} \wedge x
    =
    \bot
  \end{equation*}
  and
  \begin{equation*}
    x \vee \oline{x}
    =
    \doline{x \vee \oline{x}}
    =
    \oline{\bot}
    \reloset {\ref{thm:def:heyting_algebra/extrema_complement}} =
    \top.
  \end{equation*}

  Then \( \oline{x} \) is the complement of \( x \).
\end{proof}

\paragraph{Ultrafilters}

\begin{remark}\label{rem:boolean_algebra_ideal}
  \Cref{thm:lattice_ideal_as_semiring_ideal} demonstrates that a subset of a Boolean algebra is a \hyperref[def:lattice_ideal]{lattice ideal} (resp. filter) if and only if it is a \hyperref[def:semiring_ideal]{semiring ideal} of the \hyperref[ex:def:semiring/lattice]{join-meet semiring} (resp. meet-join semiring).
\end{remark}

\begin{proposition}\label{thm:improper_boolean_ideal}
  The only \hyperref[def:lattice_ideal]{lattice ideal} or \hyperref[def:lattice_ideal]{filter} in a \hyperref[def:boolean_algebra]{Boolean algebra} that contains both an element and its complement is the algebra itself.
\end{proposition}
\begin{proof}
  Suppose that some ideal \( I \) in \( X \) contains both \( x \) and \( \oline x \). Then \( I \) must contain their join \( \top \), and since \( I \) is closed under arbitrary meets, for every element \( y \) of the \( X \), \( I \) must contain \( y \wedge \top = y \). Therefore, \( I = X \).

  The proof for filters follows via \fullref{thm:boolean_algebra_duality}.
\end{proof}

\begin{proposition}\label{thm:boolean_prime_iff_maximal}
  A \hyperref[def:lattice_ideal]{lattice ideal} or \hyperref[def:lattice_ideal]{filter} in a \hyperref[def:boolean_algebra]{Boolean algebra} is \hyperref[def:lattice_ideal/prime]{prime} if and only if it is \hyperref[def:lattice_ideal/maximal]{maximal}.
\end{proposition}
\begin{proof}
  We will consider only ideals. The proof for filters follows via \fullref{thm:boolean_algebra_duality}.

  \SufficiencySubProof Let \( P \) be a prime ideal. \Fullref{thm:maximal_ideal_theorem} gives us a maximal ideal \( M \) containing \( I \).

  Suppose that \( M \) has some element \( x \) not in \( P \). Then \( x \wedge \oline x = \bot \) is in \( P \), thus \( \oline x \) must also be in \( P \) because the latter is prime. Then \( M \) contains both \( x \) and \( \oline x \), and \cref{thm:improper_boolean_ideal} implies that \( M \) coincides with the ambient Boolean algebra. But \( M \) must be proper, giving us a contradiction with the existence of \( x \).

  Therefore, \( M \) and \( P \) coincide, thus \( P \) is maximal.

  \NecessitySubProof Follows from \cref{thm:lattice_ideal_as_semiring_ideal} and \cref{thm:def:semiring_ideal/maximal_is_prime}.
\end{proof}

\begin{definition}\label{def:ultrafilter}
  We say that a proper \hyperref[def:lattice_ideal]{filter} \( F \) in a \hyperref[def:boolean_algebra]{Boolean algebra} is an \term[bg=ултрафилтер (\cite[18]{Проданов1982ФункционаленАнализЧаст1}), ru=ультрафильтр (\cite[182]{Гуров2013ТеорияРешёток})]{ultrafilter} if any of the following equivalent conditions hold:
  \begin{thmenum}
    \thmitem{def:ultrafilter/direct}\mcite[182]{Гуров2013ТеорияРешёток} For every algebra element \( x \), either\fnote{Both \( x \in F \) and \( \oline x \in F \) cannot hold because of \cref{thm:improper_boolean_ideal}.} \( x \in F \) or \( \oline x \in F \).

    \thmitem{def:ultrafilter/prime} \( F \) is a \hyperref[def:lattice_ideal/prime]{prime filter}.

    \thmitem{def:ultrafilter/maximal}\mcite[233]{DaveyPriestley2002LatticeTheory} \( F \) is a \hyperref[def:lattice_ideal/maximal]{maximal filter}.
  \end{thmenum}
\end{definition}
\begin{proof}
  \ImplicationSubProof{def:ultrafilter/direct}{def:ultrafilter/prime} Suppose that, for every algebra element \( x \), either \( x \in F \) or \( \oline x \in F \).

  Let \( x \vee y \in F \). If \( x \not\in F \), then \( \oline x \in F \) and hence the following is also a member of \( F \):
  \begin{equation*}
    \oline x \vee (x \vee y)
    =
    (\oline x \vee x) \vee y
    =
    \top \vee y
    =
    y.
  \end{equation*}

  Hence, if \( x \not\in F \), then \( y \in F \).

  Since \( x \) was chosen arbitrarily, we conclude that \( F \) is a prime filter.

  \ImplicationSubProof{def:ultrafilter/prime}{def:ultrafilter/maximal} Follows from \cref{thm:boolean_prime_iff_maximal}.

  \EquivalenceSubProof{def:ultrafilter/maximal}{def:ultrafilter/direct} Let \( F \) be a maximal ideal. Fix an arbitrary algebra element \( x \). Either \( x \) is in \( F \), or is in the complement of \( F \), in which case \cref{thm:improper_boolean_ideal} implies that \( \oline x \) is in \( F \).
\end{proof}

\begin{definition}\label{def:principal_ultrafilter}\mcite[example 1.6.11]{Hinman2005Logic}
  Fix an arbitrary \hyperref[def:set]{set} \( A \) and consider its \hyperref[thm:boolean_algebra_of_subsets]{power set Boolean algebra} \( \pow(A) \).

  For every element \( x \) of \( A \), the following family is an \hyperref[def:ultrafilter]{ultrafilter} in \( \pow(A) \):
  \begin{equation*}
    \mscrF_x \coloneqq \set{ B \subseteq A \given x \in B }.
  \end{equation*}

  We call \( \mscrF_x \) the \term{principal ultrafilter} of \( x \).
\end{definition}
\begin{comments}
  \item \incite[234]{DaveyPriestley2002LatticeTheory} define \enquote{principal ultrafilters} to be maximal \hyperref[def:lattice_ideal/principal]{principal filters}, which notion differs from our definition.
\end{comments}

\begin{lemma}[Ultrafilter lemma]\label{thm:ultrafilter_lemma}
  Every proper \hyperref[def:lattice_ideal]{filter} in a \hyperref[def:boolean_algebra]{Boolean algebra} is contained in an \hyperref[def:ultrafilter]{ultrafilter}.
\end{lemma}
\begin{proof}
  Follows from \cref{thm:lattice_ideal_as_semiring_ideal} and \fullref{thm:maximal_ideal_theorem}.
\end{proof}
