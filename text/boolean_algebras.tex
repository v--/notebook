\section{Boolean algebras}\label{sec:boolean_algebras}

\paragraph{Boolean algebra}

\begin{definition}\label{def:bounded_lattice_complement}\mcite[16]{Birkhoff1967LatticeTheory}
  In a \hyperref[def:bounded_lattice]{bounded lattice}, a \term[ru=дополнение (\cite[def. 1.1]{Гуров2013ТеорияРешёток})]{complement} of an element \( x \) is another element \( y \) such that \( x \wedge y = \bot \) and \( x \vee y = \top \).
\end{definition}

\begin{proposition}\label{thm:distributive_bounded_lattice_unique_complement}
  In a \hyperref[def:extremal_points/bounds]{bounded} \hyperref[def:distributive_lattice]{distributive lattice}, each element has at most one complement.
\end{proposition}
\begin{proof}
  If \( y \) and \( z \) are both complements of \( x \), then
  \begin{balign*}
    y
    &\reloset {\eqref{eq:def:bounded_lattice/theory/bot}} =
    y \wedge \top
    = \\ &=
    y \wedge (z \vee x)
    = \\ &\reloset {\eqref{eq:def:distributive_lattice/meet_over_join}} =
    (y \wedge z) \vee (y \wedge x)
    = \\ & =
    y \wedge z
    = \\ & =
    (x \wedge z) \vee (y \wedge z)
    = \\ &\reloset {\eqref{eq:def:distributive_lattice/meet_over_join}} =
    (x \vee y) \wedge z
    = \\ & =
    z.
  \end{balign*}
\end{proof}

\begin{definition}\label{def:boolean_algebra}\mcite[18]{Birkhoff1967LatticeTheory}
  A \term[ru=булева алгебра (\cite[def. 1.1]{Гуров2013ТеорияРешёток})]{Boolean algebra} is a \hyperref[def:extremal_points/bounds]{bounded} \hyperref[def:distributive_lattice]{distributive lattice} with an additional unary operation \( {\oline \anon} \), such that \( {\oline x} \) is a \hyperref[def:bounded_lattice_complement]{complement} of \( x \).

  Existence of the complement is provided by the operation itself, while uniqueness follows from \cref{thm:distributive_bounded_lattice_unique_complement}.

  Boolean algebras have the following metamathematical properties:
  \begin{thmenum}
    \thmitem{def:boolean_algebra/theory}\mimprovised We extend the \hyperref[def:lattice/theory]{first-order theory of lattices} by adding:
    \begin{itemize}
      \item The nullary functional symbols \( \top \) and \( \bot \) along with the axioms \eqref{eq:def:bounded_lattice/theory/top} and \eqref{eq:def:bounded_lattice/theory/bot}.

      \item The unary functional symbol \( {\oline {\anon}} \) along with the axioms
      \begin{align}
        \xi \vee \oline \xi \syneq \top, \label{eq:def:boolean_algebra/join} \\
        \xi \wedge \oline \xi \syneq \bot. \label{eq:def:boolean_algebra/meet}
      \end{align}
    \end{itemize}

    \thmitem{def:boolean_algebra/submodel}\mcite[18]{Birkhoff1967LatticeTheory} In addition to containing the joins and meets of all its members, a \hyperref[def:first_order_submodel]{first-order submodel} of a Boolean algebra must also contain the complement of each of its members. We call such submodels \term{Boolean subalgebras}.

    See \cref{thm:boolean_subalgebra} for simplified equivalent conditions.

    \thmitem{def:boolean_algebra/homomorphism}\mcite[89]{HalmosGivant2009BooleanAlgebras} A \hyperref[def:fol_homomorphism]{first-order homomorphism} between Boolean algebras is a bounded lattice homomorphism that preserves complements.

    \thmitem{def:boolean_algebra/category}\mcite[\S I.3.2]{Johnstone1982StoneSpaces} We denote the \hyperref[def:fol_theory_model_functor/obj]{category of \( \mscrU \)-small models} for Boolean algebras via \( \cat{Bool} \).

    \thmitem{def:boolean_algebra/opposite}\mimprovised The categories of Boolean algebras have a natural \hyperref[con:opposite_object]{opposite object functor}. We define \( (X, \wedge, \vee, \top, \bot, \oline \anon)^{\oppos} \) as \( (X, \vee, \wedge, \bot, \top, \oline \anon) \) --- the \hyperref[def:bounded_lattice/opposite]{opposite bounded lattice}, but with the same complementation operation. We say that complementation is \term[en=self-dual (\cite[193]{Gottschalk1953Quaternality})]{self-dual}.

    We show in \cref{thm:def:boolean_algebra/opposite_order} that the induced order in the opposite Boolean algebra is the \hyperref[def:preordered_set/opposite]{opposite order}.

    \thmitem{def:boolean_algebra/complete}\mimprovised If \( X \) is simultaneously a Boolean algebra and a \hyperref[def:complete_lattice]{complete lattice}, we call it a \term{complete Boolean algebra}.
  \end{thmenum}
\end{definition}

\begin{algorithm}\label{alg:propositional_formula_to_boolean_term}
\end{algorithm}

\begin{example}\label{ex:def:boolean_algebra}
  We list examples of \hyperref[def:boolean_algebra]{Boolean algebras}:

  \begin{thmenum}
    \thmitem{ex:def:boolean_algebra/lindenbaum_tarski} As shown in \cref{thm:lindenbaum_tarski_algebras/classical}, every \hyperref[def:lindenbaum_tarski_algebra]{Lindenbaum-Tarski algebra} for the \hyperref[def:propositional_natural_deduction]{classical natural deduction system} is a Boolean algebra.

    \thmitem{ex:def:boolean_algebra/f2} The \hyperref[def:finite_field]{finite field} \( \BbbF_2 = \set{ 0, 1 } \) is a Boolean algebra. Indeed, it is clearly a bounded lattice, and since it doesn't contain neither the pentagon lattice \eqref{eq:ex:def:modular_lattice/pentagon} nor the diamond lattice \eqref{eq:ex:def:distributive_lattice/diamond}, \cref{thm:distributive_lattice_characterization} implies that \( \BbbF_2 \) is distributive. Finally, the complement operation can be defined in the obvious way --- by exchanging the two elements.

    Every two-element lattice is obviously isomorphic to \( \BbbF_2 \).

    \thmitem{ex:def:boolean_algebra/power_set} The power set of any set is a \hyperref[def:complete_lattice]{complete} Boolean algebra --- see \cref{thm:boolean_algebra_of_subsets}.

    \thmitem{ex:def:boolean_algebra/interval} For every \hyperref[def:real_numbers]{real number} \( b \), the \hyperref[def:order_interval/closed]{closed interval} \( [0, b] \) is a Boolean algebra.

    \Cref{thm:def:lattice/closed_interval} implies that \( [0, b] \) is a sublattice of \( \BbbR \). We can define the complement of \( x \) in \( [0, b] \) as \( b - x \).

    This is indeed a complement in the sense of \cref{def:bounded_lattice_complement} because \( \min\set{ x, a + (b - x) } \).
  \end{thmenum}
\end{example}

\begin{proposition}\label{thm:def:boolean_algebra}
  \hyperref[def:boolean_algebra]{Boolean algebras} have the following basic properties:
  \begin{thmenum}
    \thmitem{thm:def:boolean_algebra/involution} As a function, complementation is \hyperref[def:morphism_invertibility/involution]{involutive}, i.e. its own inverse.

    \thmitem{thm:def:boolean_algebra/opposite_order} The induced order in the \hyperref[def:boolean_algebra/opposite]{opposite Boolean algebra} \( X^{\oppos} \) of \( X \) is the \hyperref[def:preordered_set/opposite]{opposite} of the induced order of \( X \).

    \thmitem{thm:def:boolean_algebra/heyting} Every Boolean algebra is a \hyperref[def:heyting_algebra]{Heyting algebra}. Furthermore, we have the following variation of \eqref{eq:thm:classical_equivalences/conditional_as_disjunction}:
    \begin{equation}\label{eq:thm:def:boolean_algebra/heyting}
      (x \rightarrow y) \coloneqq \oline {x} \vee y.
    \end{equation}

    \thmitem{thm:def:boolean_algebra/complement_contravariant} The complement is contravariant:
    \begin{equation}\label{eq:thm:def:boolean_algebra/complement_contravariant}
      x \leq y \T{if and only if} \oline x \geq \oline y
    \end{equation}

    \thmitem{thm:def:boolean_algebra/distributive}\mcite[lemma 162]{Grätzer2011LatticeTheory} In a \hyperref[def:complete_lattice]{complete} Boolean algebra, for any element \( x \) and any family \( \seq{ y_k }_{k \in \mscrK} \), we have
    \begin{align}
      \parens*{ \bigwedge_{k \in \mscrK} x_k } \vee y &= \bigwedge_{k \in \mscrK} (x_k \vee y), \label{eq:thm:def:boolean_algebra/distributive/join_over_meet} \\
      \parens*{ \bigvee_{k \in \mscrK} x_k } \wedge y &= \bigvee_{k \in \mscrK} (x_k \wedge y). \label{eq:thm:def:boolean_algebra/distributive/meet_over_join}
    \end{align}
  \end{thmenum}
\end{proposition}
\begin{proof}
  \SubProofOf{thm:def:boolean_algebra/involution} We have
  \begin{equation*}
    x
    \reloset {\eqref{eq:thm:lattice_operation_characterization/compatibility/meet}} =
    x \wedge \top
    \reloset {\eqref{eq:def:boolean_algebra/join}} =
    x \wedge (\oline x \vee \doline x)
    \reloset {\eqref{eq:def:distributive_lattice/meet_over_join}} =
    (x \wedge \oline x) \vee (x \wedge \doline x)
    \reloset {\eqref{eq:def:boolean_algebra/meet}} =
    \bot \vee (x \wedge \doline x)
    \reloset {\eqref{eq:thm:lattice_operation_characterization/compatibility/join}} =
    x \wedge \doline x
    \reloset {\eqref{eq:thm:lattice_operation_characterization/compatibility/meet}} \leq
    \doline x.
  \end{equation*}

  Analogously,
  \begin{equation*}
    \doline x
    \reloset {\eqref{eq:thm:lattice_operation_characterization/compatibility/meet}} =
    \doline x \wedge \top
    \reloset {\eqref{eq:def:boolean_algebra/join}} =
    \doline x \wedge (x \vee \oline x)
    \reloset {\eqref{eq:def:distributive_lattice/meet_over_join}} =
    (\doline x \wedge x) \vee (\doline x \wedge \oline x)
    \reloset {\eqref{eq:def:boolean_algebra/meet}} =
    (\doline x \wedge x) \vee \bot
    \reloset {\eqref{eq:thm:lattice_operation_characterization/compatibility/join}} =
    \doline x \wedge x
    \reloset {\eqref{eq:thm:lattice_operation_characterization/compatibility/meet}} \leq
    x.
  \end{equation*}

  Therefore,
  \begin{equation*}
    x = \doline x
  \end{equation*}

  \SubProofOf{thm:def:boolean_algebra/opposite_order} Follows from \cref{thm:def:bounded_lattice/opposite_order}.

  \SubProofOf{thm:def:boolean_algebra/heyting} We will show that
  \begin{equation*}
    c \leq \oline {x} \vee y \T{if and only if} x \wedge c \leq y.
  \end{equation*}

  First, if \( c \leq \oline x \vee y \), we have
  \begin{equation*}
    x \wedge c
    \leq
    x \wedge (\oline x \vee y)
    \reloset {\eqref{eq:def:distributive_lattice/meet_over_join}} =
    (\underbrace{x \wedge \oline x}_{\bot}) \vee (x \wedge y)
    =
    x \wedge y
    \leq
    y.
  \end{equation*}

  Conversely, if \( x \wedge c \leq y \), we have
  \begin{equation*}
    \oline x \vee (x \wedge c) \leq \oline x \vee y,
  \end{equation*}
  which again due to distributivity implies
  \begin{equation*}
    (\underbrace{\oline x \vee x}_{\top}) \wedge (\oline x \vee c) \leq \oline x \vee y
  \end{equation*}
  and
  \begin{equation*}
    c \leq \oline x \vee c \leq \oline x \vee y.
  \end{equation*}

  \SubProofOf{thm:def:boolean_algebra/complement_contravariant}
  \SufficiencySubProof* Suppose that \( x \leq y \). Then
  \begin{equation}\label{eq:thm:def:boolean_algebra/complement_contravariant/proof/ineq}
    x \wedge \oline y
    \reloset {\ref{thm:def:lattice/operations_preserve_order}} \leq
    y \wedge \oline y
    \reloset {\eqref{eq:def:boolean_algebra/meet}} =
    \bot,
  \end{equation}
  hence
  \begin{equation*}
    \oline x \vee \oline y
    \reloset {\ref{def:bounded_lattice/neutral}} =
    \underbrace{(\oline x \vee x)}_\top \wedge (\oline x \vee \oline y)
    \reloset {\eqref{eq:thm:def:lattice/distributive_inequality/join_over_meet}} =
    \oline x \vee (x \wedge \oline y)
    \reloset {\eqref{eq:thm:def:boolean_algebra/complement_contravariant/proof/ineq}} =
    \bot \vee \oline x
    \reloset {\ref{def:bounded_lattice/neutral}} =
    \oline x.
  \end{equation*}

  Therefore, \( \oline x \geq \oline y \).

  \NecessitySubProof* Conversely, if \( \oline x \geq \oline y \), then \( \doline x \leq \doline y \), and \cref{thm:def:boolean_algebra/involution} implies that \( x \leq y \).

  \SubProofOf{thm:def:boolean_algebra/distributive} \Cref{thm:def:boolean_algebra/heyting} implies that every Boolean algebra is a Heyting algebra, hence \eqref{eq:thm:def:boolean_algebra/distributive/meet_over_join} holds as a restatement of \eqref{eq:thm:complete_heyting_algebra/infinite_distributive}.

  Via \fullref{thm:lattice_duality/tautology} we conclude that \eqref{eq:thm:def:boolean_algebra/distributive/join_over_meet} also holds.
\end{proof}

\begin{theorem}[De Morgan's laws]\label{thm:de_morgans_laws}
  In a \hyperref[def:boolean_algebra]{Boolean algebra}, the following hold for any finite family \( \set{ x_k }_{k \in \mscrK} \):
  \begin{align}
    \oline{\bigvee_{k \in \mscrK} x_k}   &= \bigwedge_{k \in \mscrK} \oline{x_k}, \label{eq:thm:de_morgans_laws/complement_of_join} \\
    \oline{\bigwedge_{k \in \mscrK} x_k} &= \bigvee_{k \in \mscrK} \oline{x_k}.   \label{eq:thm:de_morgans_laws/complement_of_meet}
  \end{align}

  If the algebra is a \hyperref[def:complete_lattice]{complete lattice}, \( \mscrK \) can be any family, not necessarily finite.
\end{theorem}
\begin{comments}
  \item This is one of several variants of De Morgan's laws presented here --- see \cref{rem:de_morgans_laws/boolean}.
\end{comments}
\begin{proof}
  The equality \eqref{eq:thm:de_morgans_laws/complement_of_join} is a restatement of \eqref{eq:thm:de_morgans_laws_for_heyting_algebras/complement_of_join}. We will prove \eqref{eq:thm:de_morgans_laws/complement_of_meet} explicitly.

  We have
  \begin{equation*}
    \parens*{ \bigwedge_{j \in \mscrK} x_j } \wedge \parens*{ \bigvee_{i \in \mscrK} \oline{x_i} }
    \reloset {\eqref{eq:thm:def:boolean_algebra/distributive/meet_over_join}} =
    \bigvee_{i \in \mscrK} \parens*{ \parens*{ \bigwedge_{j \in \mscrK} x_j } \wedge \oline{x_i} }
    =
    \bigvee_{i \in \mscrK} \parens[\vast]{ \parens*{ \bigwedge_{j \neq i} x_j }\wedge \underbrace{\oline{x_i} \wedge x_i}_{\bot} }
    =
    \bot
  \end{equation*}
  and similarly
  \begin{equation*}
    \parens*{ \bigwedge_{j \in \mscrK} x_j } \vee \parens*{ \bigvee_{i \in \mscrK} \oline{x_i} }
    \reloset {\eqref{eq:thm:def:boolean_algebra/distributive/join_over_meet}} =
    \bigwedge_{j \in \mscrK} \parens*{ x_j \vee \bigvee_{i \in \mscrK} \oline{x_i} }
    =
    \bigwedge_{j \in \mscrK} \parens[\vast]{ \underbrace{\oline{x_j} \vee x_j}_{\top} \vee \bigvee_{i \neq j} \oline{x_i} }
    =
    \top,
  \end{equation*}
  hence the complement of \( \bigwedge_{j \in \mscrK} x_j \) is \( \bigvee_{i \in \mscrK} \oline{x_i} \).
\end{proof}

\begin{theorem}[Principle of duality for Boolean algebras]\label{thm:boolean_algebra_duality}
  There are multiple ways to extend \fullref{thm:order_duality} to \hyperref[def:boolean_algebra]{Boolean algebras}. We list two of them that can be stated easily for \hyperref[def:fol_formula]{first-order formulas}, and we will discuss more in \cref{def:boolean_polynomial_opposite} and \fullref{alg:propositional_formula_dualization}.

  \begin{thmenum}
    \thmitem{thm:boolean_algebra_duality/identity} The opposite of a Boolean algebra is precisely its \hyperref[def:bounded_lattice/opposite]{opposite bounded lattice}, hence the identity map relates a Boolean algebra with its opposite via \fullref{thm:lattice_duality}.

    \thmitem{thm:boolean_algebra_duality/complement} The complement \( \oline{\anon} \) is an isomorphism between a Boolean algebra and its opposite; in particular, the two are \hyperref[def:elementary_equivalence]{elementarily equivalent}.
  \end{thmenum}
\end{theorem}
\begin{proof}
  \SubProofOf{thm:boolean_algebra_duality/identity} Follows from \cref{thm:lattice_duality}.

  \SubProofOf{thm:boolean_algebra_duality/complement} We must show that \( \oline{\anon} \) is an isomorphism from a Boolean algebra \( X \) to its opposite to \( X^{\oppos} \).

  \Fullref{thm:de_morgans_laws} implies that \( \oline{x \wedge y} = \oline x \vee \oline y \), and similarly \( \oline{x \vee y} = \oline x \wedge \oline y \).

  \Cref{thm:def:heyting_algebra/extrema_pseudocomplement} implies that \( \oline \top = \bot \) and \( \oline \bot = \top \).

  Hence, complementation is a homomorphism from \( X \) to \( X^{\oppos} \). \Cref{thm:def:boolean_algebra/involution} implies that it is invertible as a function, its inverse being the complementation homomorphism from \( X^{\oppos} \) to \( X \).

  Therefore, \( \oline{\anon} \) is an isomorphism.
\end{proof}

\begin{proposition}\label{thm:boolean_algebra_heyting_characterization}
  A \hyperref[def:heyting_algebra]{Heyting algebra} is a \hyperref[def:boolean_algebra]{Boolean algebra} if and only if \( \doline{x} = x \) for any element.
\end{proposition}
\begin{comments}
  \item \Cref{thm:def:heyting_algebra/dni} implies that \( x \leq \doline{x} \) in any Heyting algebra, so, when using this characterization, there is only one inequality to show.
\end{comments}
\begin{proof}
  \SufficiencySubProof Follows from the definition of a Boolean algebra, considering \cref{thm:def:boolean_algebra/heyting}.
  \NecessitySubProof Suppose that \( \doline{x} = x \) for any element in a Heyting algebra. It is a bounded lattice by definition, and \cref{thm:def:heyting_algebra/distributive} implies that it is distributive. We must only show that \( \oline{x} \) is a \hyperref[def:bounded_lattice_complement]{complement} of \( x \).

  \Cref{thm:def:heyting_algebra/semicomplement} implies that \( x \wedge \oline{x} = \bot \) for any Heyting algebra. On the other hand,
  \begin{equation*}
    \oline{x \vee \oline{x}}
    \reloset {\eqref{eq:thm:de_morgans_laws_for_heyting_algebras/complement_of_meet}} \geq
    \oline{x} \wedge \doline{x}
    =
    \oline{x} \wedge x
    =
    \bot
  \end{equation*}
  and
  \begin{equation*}
    x \vee \oline{x}
    =
    \doline{x \vee \oline{x}}
    =
    \oline{\bot}
    \reloset {\ref{thm:def:heyting_algebra/extrema_pseudocomplement}} =
    \top.
  \end{equation*}

  Then \( \oline{x} \) is the complement of \( x \).
\end{proof}

\paragraph{Boolean algebra polynomials}

\begin{definition}\label{def:free_boolean_algebra}\mcite[257]{HalmosGivant2009BooleanAlgebras}
  We refer to the \hyperref[def:fol_free_term_model]{free term model} \( F(\mscrX) \) of the \hyperref[def:boolean_algebra/theory]{theory of Boolean algebras} in the set of indeterminates \( \mscrX \) as the \term[en=free lattice (\cite[36]{Birkhoff1967LatticeTheory})]{free Boolean algebra} on \( \mscrX \).

  \begin{thmenum}
    \thmitem{def:free_boolean_algebra/polynomial}\mcite[61]{HalmosGivant2009BooleanAlgebras} We call the elements of the term model \term[en=Boolean polynomial (\cite[29]{Birkhoff1967LatticeTheory})]{Boolean polynomials}, in analogy with \hyperref[def:polynomial_algebra/polynomial]{algebraic polynomials} and \hyperref[def:free_lattice/polynomial]{lattice polynomials}.

    We will use the familiar functional notation like \( f(X_1, \ldots, X_n) \) to make explicit the indeterminates on which the polynomial \( f \) depends.
  \end{thmenum}
\end{definition}
\begin{comments}
  \item With the representation of Boolean functions as \hyperref[def:zhegalkin_polynomial]{Zhegalkin polynomials} shown in \cref{def:standard_boolean_functions}, we can convert any Boolean polynomial to a Zhegalkin polynomial.
\end{comments}

\begin{proposition}\label{thm:boolean_polynomial_complement}
  Consider the following operation on \hyperref[def:free_boolean_algebra/polynomial]{Boolean polynomials}:
  \begin{equation}\label{eq:thm:boolean_polynomial_complement}
    \widetilde f = \begin{cases}
      \latbot                             &f = \lattop, \\
      \lattop                             &f = \latbot, \\
      \latneg X,                          &f = X \in \mscrX, \\
      X,                                  &f = \latneg X, \\
      \latneg \widetilde g,               &f = \latneg g \T{and} g \not\in \mscrX, \\
      \widetilde g \latvee \widetilde h   &f = g \latwedge h, \\
      \widetilde g \latwedge \widetilde h &f = g \latvee h.
    \end{cases}
  \end{equation}

  We claim that \( \widetilde f \) coincides with the complement \( \oline f = \latneg f \).
\end{proposition}
\begin{proof}
  We will use \fullref{thm:induction_on_abstract_syntax} on \( f \):
  \begin{itemize}
    \item The cases \( f = \lattop \) and \( f = \latbot \) follow from \cref{thm:def:heyting_algebra/extrema_pseudocomplement}.

    \item If \( f = X \) for some indeterminate \( X \), then \( \oline X \) is by definition \( \latneg X \).

    \item If \( f = \latneg X \), then
    \begin{equation*}
      \oline f
      =
      \doline X
      \reloset {\ref{thm:def:boolean_algebra/involution}} =
      X.
    \end{equation*}

    \item If \( f = \latneg g \), where the inductive hypothesis holds for \( g \), we have
    \begin{equation*}
      \oline f
      =
      \oline {\latneg g}
      =
      \latneg \latneg g
      \reloset {\T{ind.}} =
      \latneg {\widetilde g}.
    \end{equation*}

    \item If \( f = g \latwedge h \), where the inductive hypothesis hold for \( g \) and \( h \), then
    \begin{equation*}
      \oline f
      \reloset {\eqref{eq:thm:de_morgans_laws/complement_of_wedge}} =
      \oline g \latvee \oline h
      \reloset {\T{ind.}} =
      \widetilde g \latvee \widetilde h
      =
      \widetilde f.
    \end{equation*}

    \item If \( f = g \latvee h \), we proceed analogously.
  \end{itemize}
\end{proof}

\begin{definition}\label{def:opposite_boolean_polynomial}\mcite[ch. 4]{HalmosGivant2009BooleanAlgebras}
  To every \hyperref[def:free_boolean_algebra/polynomial]{Boolean polynomial} \( f(X_1, \ldots, X_n) \) in \( F(\mscrX) \), there correspond several polynomials that can be considered \hyperref[con:opposite_object]{opposite}.

  \begin{thmenum}
    \thmitem{def:opposite_boolean_polynomial/contradual} We define the \term{contradual} as \( f(\oline{X_1}, \ldots, \oline{X_n}) \), by negating all indeterminates.

    The map sending a polynomial to its contradual is an automorphism of \( F(\mscrX) \).

    \thmitem{def:opposite_boolean_polynomial/complement} At the other extreme, we can simply consider the complement \( \oline{f(X_1, \ldots, X_n)} \).

    \Cref{thm:boolean_polynomial_complement} shows that \( \oline{f(X_1, \ldots, X_n)} \) can be defined recursively via \eqref{eq:thm:boolean_polynomial_complement}; this is utilized in \fullref{alg:propositional_formula_dualization}.

    The map sending a polynomial to its complement is an isomorphism between \( F(\mscrX) \) and its opposite.

    \thmitem{def:opposite_boolean_polynomial/dual} Combining the two, we obtain the \term{dual} \( \oline{f(\oline{X_1}, \ldots, \oline{X_n})} \).

    The dual extends the \hyperref[def:opposite_lattice_polynomial]{opposite lattice polynomial} by preserving complements.

    The map sending a polynomial to its dual is also an isomorphism between \( F(\mscrX) \) and its opposite.
  \end{thmenum}
\end{definition}
\begin{comments}
  \item \incite*[ch. 4]{HalmosGivant2009BooleanAlgebras} write\fnote{In their notation, \( p' \) is the complement of \( p \).}
  \begin{displayquote}
    A slight misunderstanding can arise about the meaning of duality, and often does. It is well worthwhile to clear it up once and for all, especially since the clarification is quite amusing in its own right. If an experienced Boolean algebraist is asked for the dual of a Boolean polynomial, such as say \( p \vee q \), his answer might be \( p \wedge q \) one day and \( p' \wedge q' \) another day; the answer \( p' \vee q' \) is less likely but not impossible. Let us restrict attention to the completely typical case of a polynomial \( f(p, q) \) in two variables. The \textit{complement} of \( f(p, q) \) is by definition \( (f(p, q))' \), abbreviated \( f'(p, q) \); the \textit{dual} of \( f(p, q) \) is \( f'(p', q') \); the \textit{contradual} of \( f(p, q) \) is \( f(p', q') \).
  \end{displayquote}
\end{comments}

\begin{proposition}\label{thm:klein_four_group_action_on_boolean_polynomials}
  The \hyperref[def:klein_four_group]{Klein four group} \( V_4 = \set{ e, a, b, ab } \) \hyperref[def:group_action]{acts} on \hyperref[def:free_boolean_algebra/polynomial]{Boolean polynomials} via the opposites defined in \cref{def:opposite_boolean_polynomial}.

  Namely, \( e \) preserves the polynomial \( f \), \( a \) sends \( f \) to its contradual and \( b \) sends \( f \) to its dual. Then it remains for \( ab \) to send \( f \) to its complement.
\end{proposition}
\begin{comments}
  \item \incite{Gottschalk1953Quaternality} calls this result the \enquote{theory of quaternality}, alluding to \hyperref[con:duality]{duality}.
\end{comments}
\begin{proof}
  Straightforward.
\end{proof}

\paragraph{Ultrafilters}

\begin{remark}\label{rem:boolean_algebra_ideal}
  \Cref{thm:lattice_ideal_as_semiring_ideal} demonstrates that a subset of a Boolean algebra is a \hyperref[def:lattice_ideal]{lattice ideal} (resp. filter) if and only if it is a \hyperref[def:semiring_ideal]{semiring ideal} of the \hyperref[ex:def:semiring/lattice]{join-meet semiring} (resp. meet-join semiring).
\end{remark}

\begin{proposition}\label{thm:improper_boolean_ideal}
  The only \hyperref[def:lattice_ideal]{lattice ideal} or \hyperref[def:lattice_ideal]{filter} in a \hyperref[def:boolean_algebra]{Boolean algebra} that contains both an element and its complement is the algebra itself.
\end{proposition}
\begin{proof}
  Suppose that some ideal \( I \) in \( X \) contains both \( x \) and \( \oline x \). Then \( I \) must contain their join \( \top \), and since \( I \) is closed under arbitrary meets, for every element \( y \) of the \( X \), \( I \) must contain \( y \wedge \top = y \). Therefore, \( I = X \).

  The proof for filters follows via \fullref{thm:boolean_algebra_duality}.
\end{proof}

\begin{proposition}\label{thm:boolean_prime_iff_maximal}
  A \hyperref[def:lattice_ideal]{lattice ideal} or \hyperref[def:lattice_ideal]{filter} in a \hyperref[def:boolean_algebra]{Boolean algebra} is \hyperref[def:lattice_ideal/prime]{prime} if and only if it is \hyperref[def:lattice_ideal/maximal]{maximal}.
\end{proposition}
\begin{proof}
  We will consider only ideals. The proof for filters follows via \fullref{thm:boolean_algebra_duality}.

  \SufficiencySubProof Let \( P \) be a prime ideal. \Fullref{thm:maximal_ideal_theorem} gives us a maximal ideal \( M \) containing \( I \).

  Suppose that \( M \) has some element \( x \) not in \( P \). Then \( x \wedge \oline x = \bot \) is in \( P \), thus \( \oline x \) must also be in \( P \) because the latter is prime. Then \( M \) contains both \( x \) and \( \oline x \), and \cref{thm:improper_boolean_ideal} implies that \( M \) coincides with the ambient Boolean algebra. But \( M \) must be proper, giving us a contradiction with the existence of \( x \).

  Therefore, \( M \) and \( P \) coincide, thus \( P \) is maximal.

  \NecessitySubProof Follows from \cref{thm:lattice_ideal_as_semiring_ideal} and \cref{thm:def:semiring_ideal/maximal_is_prime}.
\end{proof}

\begin{definition}\label{def:ultrafilter}
  We say that a proper \hyperref[def:lattice_ideal]{filter} \( F \) in a \hyperref[def:boolean_algebra]{Boolean algebra} is an \term[bg=ултрафилтер (\cite[18]{Проданов1982ФункционаленАнализЧаст1}), ru=ультрафильтр (\cite[182]{Гуров2013ТеорияРешёток})]{ultrafilter} if any of the following equivalent conditions hold:
  \begin{thmenum}
    \thmitem{def:ultrafilter/direct}\mcite[182]{Гуров2013ТеорияРешёток} For every algebra element \( x \), either\fnote{Both \( x \in F \) and \( \oline x \in F \) cannot hold because of \cref{thm:improper_boolean_ideal}.} \( x \in F \) or \( \oline x \in F \).

    \thmitem{def:ultrafilter/prime} \( F \) is a \hyperref[def:lattice_ideal/prime]{prime filter}.

    \thmitem{def:ultrafilter/maximal}\mcite[233]{DaveyPriestley2002LatticeTheory} \( F \) is a \hyperref[def:lattice_ideal/maximal]{maximal filter}.
  \end{thmenum}
\end{definition}
\begin{proof}
  \ImplicationSubProof{def:ultrafilter/direct}{def:ultrafilter/prime} Suppose that, for every algebra element \( x \), either \( x \in F \) or \( \oline x \in F \).

  Let \( x \vee y \in F \). If \( x \not\in F \), then \( \oline x \in F \) and hence the following is also a member of \( F \):
  \begin{equation*}
    \oline x \vee (x \vee y)
    =
    (\oline x \vee x) \vee y
    =
    \top \vee y
    =
    y.
  \end{equation*}

  Hence, if \( x \not\in F \), then \( y \in F \).

  Since \( x \) was chosen arbitrarily, we conclude that \( F \) is a prime filter.

  \ImplicationSubProof{def:ultrafilter/prime}{def:ultrafilter/maximal} Follows from \cref{thm:boolean_prime_iff_maximal}.

  \EquivalenceSubProof{def:ultrafilter/maximal}{def:ultrafilter/direct} Let \( F \) be a maximal ideal. Fix an arbitrary algebra element \( x \). Either \( x \) is in \( F \), or is in the complement of \( F \), in which case \cref{thm:improper_boolean_ideal} implies that \( \oline x \) is in \( F \).
\end{proof}

\begin{definition}\label{def:principal_ultrafilter}\mcite[example 1.6.11]{Hinman2005Logic}
  Fix an arbitrary \hyperref[def:set]{set} \( A \) and consider its \hyperref[thm:boolean_algebra_of_subsets]{power set Boolean algebra} \( \pow(A) \).

  For every element \( x \) of \( A \), the following family is an \hyperref[def:ultrafilter]{ultrafilter} in \( \pow(A) \):
  \begin{equation*}
    \mscrF_x \coloneqq \set{ B \subseteq A \given x \in B }.
  \end{equation*}

  We call \( \mscrF_x \) the \term{principal ultrafilter} of \( x \).
\end{definition}
\begin{comments}
  \item \incite[234]{DaveyPriestley2002LatticeTheory} define \enquote{principal ultrafilters} to be maximal \hyperref[def:lattice_ideal/principal]{principal filters}, which notion differs from our definition.
\end{comments}

\begin{lemma}[Ultrafilter lemma]\label{thm:ultrafilter_lemma}
  Every proper \hyperref[def:lattice_ideal]{filter} in a \hyperref[def:boolean_algebra]{Boolean algebra} is contained in an \hyperref[def:ultrafilter]{ultrafilter}.
\end{lemma}
\begin{proof}
  Follows from \cref{thm:lattice_ideal_as_semiring_ideal} and \fullref{thm:maximal_ideal_theorem}.
\end{proof}
