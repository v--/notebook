\subsection{Boolean algebras}\label{subsec:boolean_algebras}

\begin{definition}\label{def:heyting_algebra}\mcite[147]{Birkhoff1948}
  A \term{Heyting algebra} or \term{relatively pseudo-complemented algebra} is a \hyperref[def:semilattice/bounded]{bounded} \hyperref[def:semilattice/distributive_lattice]{distributive lattice} \( L \) with a binary operation \( \rightarrow \) defined as
  \begin{equation}\label{eq:def:heyting_algebra/conditional}
    (x \rightarrow y) \coloneqq \bigvee\set{ a \in L \given a \wedge x \leq y }.
  \end{equation}

  In order for the operation to be well-defined, we require that the corresponding join exists for all \( y \) and \( z \). We call this operation the \term{conditional} in analogy with the \hyperref[def:propositional_language/connectives/conditional]{propositional connective}, although this operation is often called \enquote{implication} because of \hyperref[def:material_implication]{material implication}.

  Heyting algebras are useful for defining truth values for intuitionistic logic -- see \fullref{def:propositional_heyting_algebra_semantics} --- and also appear as the \hyperref[def:lindenbaum_tarski_algebra]{Lindenbaum-Tarski algebra} for propositional intuitionistic theories --- see \fullref{thm:intuitionistic_lindenbaum_tarski_algebra}.

  \begin{thmenum}[series=def:heyting_algebra]
    \thmitem{def:heyting_algebra/pseudocomplement} For any element \( x \), we define its \term{pseudocomplement} as
    \begin{equation}\label{eq:def:heyting_algebra/pseudocomplement}
      \widetilde x
      \coloneqq
      (x \rightarrow \bot)
      =
      \bigvee\set{ a \in L \given a \wedge x = \bot }.
    \end{equation}
  \end{thmenum}

  Heyting algebras have the following metamathematical properties:
  \begin{thmenum}[resume=def:heyting_algebra]
    \thmitem{def:heyting_algebra/theory} We extend the language of the \hyperref[def:semilattice/theory]{theory of lattices} with the binary infix functional symbol \( \rightarrow \) and the unary functional symbol \( \widetilde{\anon} \). By adding the axiom \eqref{eq:def:heyting_algebra/conditional} to the theory of bounded distributive lattices, we obtain the theory of Heyting algebras.

    \thmitem{def:heyting_algebra/submodel} The Heyting subalgebras are the \hyperref[def:semilattice/submodel]{bounded sublattices} for which the conditional is well-defined.

    \thmitem{def:heyting_algebra/trivial} The \hyperref[rem:trivial_structure]{trivial} Heyting algebra is the \hyperref[def:semilattice/trivial]{one-element bounded lattice}.

    \thmitem{def:boolean_algebra/initial} The \hyperref[thm:substructures_form_complete_lattice/bottom]{initial substructure} of a Heyting algebra is the \hyperref[def:semilattice/trivial]{two-element bounded lattice} (whose elements are equal only of the lattice is trivial).

    \thmitem{def:heyting_algebra/homomorphism} \hyperref[def:first_order_homomorphism]{First-order homomorphisms} between Heyting algebras are lattice homomorphisms with the additional requirement that homomorphisms preserve conditionals.

    \thmitem{def:heyting_algebra/category} The \hyperref[def:category_of_small_first_order_models]{category of \( \mscrU \)-small models} for Heyting algebras is denoted by \( \cat{Heyt} \). It is a subcategory of the \hyperref[def:semilattice/category]{category \( \cat{Lat} \) of lattices}.

    \thmitem{def:heyting_algebra/opposite} The \hyperref[def:semilattice/duality]{principle of duality for lattices} does not hold for Heyting algebras.
  \end{thmenum}
\end{definition}

\begin{example}\label{ex:topological_space_is_heyting_algebra}
  Somewhat similar to how the power set of a nonempty set is a Boolean algebra, as shown in \fullref{thm:boolean_algebra_of_subsets}, the topology \( \mscrT \) of a \hyperref[def:topological_space]{topological space} \( (X, \mscrT) \) is a Heyting algebra. This is actually used in topological semantics --- see \fullref{def:propositional_topological_semantics}.

  Indeed,
  \begin{itemize}
    \item \hyperref[def:semilattice/join]{Arbitrary joins} are given by \hyperref[def:basic_set_operations/union]{unions \( \bigcap \)}.
    \item \hyperref[def:semilattice/meet]{Finite meets} are given by \hyperref[def:basic_set_operations/intersection]{intersections \( \bigcup \)}.
    \item The \hyperref[def:extremal_points/top_and_bottom]{top} is the entire domain \( L \).
    \item The \hyperref[def:extremal_points/top_and_bottom]{bottom} is the empty set.
    \item The \hyperref[eq:def:heyting_algebra/conditional]{conditional \( U \leadsto V \)} is then
    \begin{equation*}
      \bigcup\set[\Big]{ A \in T \given \underbrace{A \cap U}_{A \setminus (X \setminus U)} \subseteq V }
      =
      \bigcup\set[\Big]{ A \in T \given A \subseteq V \cup (X \setminus U) }
      =
      \Int((X \setminus U) \cup V),
    \end{equation*}
    which is actually similar to \fullref{thm:boolean_equivalences/conditional_as_disjunction} despite the fact that arbitrary topologies are not Boolean algebras.

    \item As a result, the \hyperref[def:heyting_algebra/pseudocomplement]{pseudocomplement} is
    \begin{equation*}
      \widetilde U = \Int(X \setminus U).
    \end{equation*}
  \end{itemize}
\end{example}

\begin{definition}\label{def:bounded_lattice_complement}
  Let \( L \) be a \hyperref[def:semilattice/bounded]{bounded} \hyperref[def:semilattice/lattice]{lattice} and fix an element \( x \in L \). A \term{complement} of \( x \) is an element \( y \) such that
  \begin{align}
    x \vee y = \top \label{def:bounded_lattice_complement/join}, \\
    x \wedge y = \bot \label{def:bounded_lattice_complement/meet}.
  \end{align}

  Due to the commutativity of both \( \vee \) and \( \wedge \), \( y \) is a complement of \( x \) if and only if \( x \) is a complement of \( y \).
\end{definition}

\begin{proposition}\label{thm:distributive_bounded_lattice_unique_complement}
  In a \hyperref[def:semilattice/bounded]{bounded} \hyperref[def:semilattice/distributive_lattice]{distributive lattice} \( L \), each \( x \in L \) has at most one complement.

  Thus, complementation can be regarded as a \hyperref[def:partial_function]{partial operation}.
\end{proposition}
\begin{proof}
  If \( y \) and \( z \) are both complements of \( x \), then
  \begin{balign*}
    y
    &\reloset {\eqref{eq:thm:binary_lattice_operations/identity/meet}} =
    y \wedge \top
    = \\ &\reloset {\eqref{def:bounded_lattice_complement/join}} =
    y \wedge (z \vee x)
    = \\ &\reloset {\eqref{eq:def:semilattice/distributive_lattice/finite/meet_over_join}} =
    (y \wedge z) \vee (y \wedge x)
    = \\ &\reloset {\eqref{def:bounded_lattice_complement/meet}} =
    y \wedge z
    = \\ &\reloset {\eqref{def:bounded_lattice_complement/meet}} =
    (x \wedge z) \vee (y \wedge z)
    = \\ &\reloset {\eqref{eq:def:semilattice/distributive_lattice/finite/meet_over_join}} =
    (x \vee y) \wedge z
    = \\ &\reloset {\eqref{eq:thm:binary_lattice_operations/identity/meet}} =
    z.
  \end{balign*}
\end{proof}

\begin{definition}\label{def:boolean_algebra}\mcite[def. II.5.2]{Вулих1961}
  A \term{Boolean algebra} is a \hyperref[def:semilattice/bounded]{bounded} \hyperref[def:semilattice/distributive_lattice]{distributive lattice} in which every element has a \hyperref[def:bounded_lattice_complement]{complement}. The complement of each element is unique due to \fullref{thm:distributive_bounded_lattice_unique_complement}. We define a unary function that gives to every element \( x \) its complement \( \overline \anon \). By definition, this function is an \hyperref[def:set_with_involution]{involution}.

  \begin{thmenum}[series=def:boolean_algebra]
    \thmitem{def:boolean_algebra/conditional} We also define the binary operation \term{conditional} (\( \rightarrow \)) via
    \begin{equation}\label{eq:def:boolean_algebra/conditional}
      (x \rightarrow y) \coloneqq (\overline x \vee y)
    \end{equation}
    in analogy with \fullref{thm:boolean_equivalences/conditional_as_disjunction}. This operation highlights that Boolean algebras are a special case of \hyperref[thm:boolean_algebras_are_heyting_algebras]{Heyting algebras}.

    \thmitem{def:boolean_algebra/biconditional} It remains to define a binary operation corresponding to the \hyperref[def:propositional_language/connectives/biconditional]{propositional biconditional}. Inspired by \fullref{thm:boolean_equivalences/biconditional_via_conditionals}, define
    \begin{equation}\label{eq:def:boolean_algebra/biconditional}
      (x \leftrightarrow y) \coloneqq (x \rightarrow y) \wedge (y \rightarrow x).
    \end{equation}
  \end{thmenum}

  Boolean algebras have the following metamathematical properties:
  \begin{thmenum}[resume=def:boolean_algebra]
    \thmitem{def:boolean_algebra/theory} To obtain the theory of Boolean algebras, we replace the unary functional symbol \( \widetilde{\anon} \) with \( \overline \anon \) in the language of the \hyperref[def:heyting_algebra/theory]{theory of Heyting algebras} and then add the axioms \eqref{def:bounded_lattice_complement/join} and \eqref{def:bounded_lattice_complement/meet} to the theory. We may also replace \eqref{eq:def:heyting_algebra/conditional} defining \( \rightarrow \) with the simpler axiom \eqref{eq:def:boolean_algebra/conditional}.

    \thmitem{def:boolean_algebra/submodel} The Boolean subalgebras are the \hyperref[def:semilattice/submodel]{bounded sublattices} which are closed under compementation.

    \thmitem{def:boolean_algebra/trivial} The \hyperref[rem:trivial_structure]{trivial} Boolean algebra is the \hyperref[def:semilattice/trivial]{one-element bounded lattice}.

    \thmitem{def:boolean_algebra/initial} The \hyperref[thm:substructures_form_complete_lattice/bottom]{initial substructure} of a Boolean algebra is the \hyperref[def:semilattice/trivial]{two-element bounded lattice} (whose elements are equal only of the lattice is trivial).

    \thmitem{def:boolean_algebra/homomorphism} \hyperref[def:first_order_homomorphism]{First-order homomorphisms} between Boolean algebras are simply lattice homomorphisms.

    Complements are automatically preserved because for any lattice homomorphism \( \varphi \) between the Boolean algebras \( L \) and \( Y \),
    \begin{equation*}
      \varphi(x) \vee_Y \varphi(\overline x) = \varphi(x \vee_X \overline x) = \varphi(\top_X) = \top_Y,
    \end{equation*}
    and similarly for \( \wedge \), hence, due to \fullref{thm:distributive_bounded_lattice_unique_complement},
    \begin{equation*}
      \varphi(\overline x) = \overline {\varphi(x)}.
    \end{equation*}

    Implications are also automatically preserved because of \eqref{eq:def:boolean_algebra/conditional}.

    \thmitem{def:boolean_algebra/category} The \hyperref[def:category_of_small_first_order_models]{category of \( \mscrU \)-small models} for Boolean algebras \( \cat{Bool} \) is a full subcategory the \hyperref[def:heyting_algebra/category]{category \( \cat{Heyt} \) of Heyting algebras}.

    \thmitem{def:boolean_algebra/opposite} The \hyperref[def:semilattice/duality]{principle of duality for lattices} holds for Boolean algebras without interchanging complements.
  \end{thmenum}
\end{definition}

\begin{example}\label{ex:boolean_algebras}
  Examples of \hyperref[def:boolean_algebra]{Boolean algebras} include:

  \begin{itemize}
    \thmitem{ex:boolean_algebras/lindenbaum_tarski} The \hyperref[def:lindenbaum_tarski_algebra]{Lindenbaum-Tarski algebras} for classical logic. We prove in \fullref{thm:intuitionistic_lindenbaum_tarski_algebra} that it is a Boolean algebra.
    \thmitem{ex:boolean_algebras/f2} The \hyperref[thm:finite_fields]{prime field} \( \BbbF_2 \) with suitably defined operations discussed in \fullref{thm:f2_is_boolean_algebra}.
    \thmitem{ex:boolean_algebras/power_set} The power set of any set, usually taken to be a space with additional structure (see \fullref{thm:boolean_algebra_of_subsets}).
  \end{itemize}
\end{example}

\begin{corollary}\label{thm:f2_is_boolean_algebra}
  For certain purposes, for example \hyperref[def:zhegalkin_polynomial]{Zhegalkin polynomials}, we may regard the \hyperref[thm:finite_fields]{prime field} \( \BbbF_2 \) as a Boolean algebra with \( 1 \) as the \hyperref[def:extremal_points/top_and_bottom]{top} and \( 0 \) as the \hyperref[def:extremal_points/top_and_bottom]{bottom element}.
\end{corollary}
\begin{proof}
  Follows from \fullref{def:boolean_algebra/trivial}.
\end{proof}

\begin{proposition}\label{thm:boolean_algebras_are_heyting_algebras}
  Every \hyperref[def:boolean_algebra]{Boolean algebra} is a \hyperref[def:heyting_algebra]{Heyting algebra} with an identification given by \eqref{eq:def:boolean_algebra/conditional}.
\end{proposition}
\begin{proof}
  Fix any \( x, y \in L \) in a Boolean algebra \( L \). We will show that \( x \rightarrow y \) as defined in \eqref{eq:def:boolean_algebra/conditional} satisfies \eqref{eq:def:heyting_algebra/conditional}.

  Let
  \begin{equation*}
   A \coloneqq \set{ a \in L \given a \wedge x \leq y }
  \end{equation*}
  be the set from \eqref{eq:def:heyting_algebra/conditional}.

  We will show that \( \overline x \vee y \) is an \hyperref[def:extremal_points/upper_and_lower_bounds]{upper bound} of \( A \).

  Fix some \( a_0 \in A \). By definition of \( A \), we have
  \begin{equation*}
   a_0 \wedge x \leq y.
  \end{equation*}

  But this means that
  \begin{equation*}
   \underbrace{(a_0 \wedge x) \vee \overline x}_{a_0 \vee \overline x} \leq y \vee \overline x.
  \end{equation*}

  Since \( a_0 \leq a_0 \vee b \) for any \( b \in L \), it follows that \( a_0 \leq y \vee \overline x \). Therefore, \( \overline x \vee y \) is indeed an upper bound of \( A \).

  Also note that
  \begin{equation*}
   (\overline x \vee y) \wedge x = \underbrace{(\overline x \vee x)}_{\top} \wedge (y \wedge x) = y \wedge x \leq y,
  \end{equation*}
  hence \( \overline x \vee y \in A \).

  Thus, \( \overline x \vee y \) is both an upper bound of \( A \) and an element of \( A \), i.e. it is the least upper bound of \( A \). Therefore,
  \begin{equation*}
   \overline x \vee y = \bigvee A.
  \end{equation*}
\end{proof}

\begin{theorem}[De Morgan's laws]\label{thm:de_morgans_laws}
  If \( L \) is a \hyperref[def:boolean_algebra]{Boolean algebra}, the following hold for any finite \hyperref[def:cartesian_product/indexed_family]{family} \( \set{ x_k }_{k \in \mscrK} \subseteq X \):
  \begin{align}
    \overline{\bigvee_{k \in \mscrK} x_k} = \bigwedge_{k \in \mscrK} \overline{x_k} \label{eq:thm:de_morgans_laws/complement_of_join} \\
    \overline{\bigwedge_{k \in \mscrK} x_k} = \bigvee_{k \in \mscrK} \overline{x_k} \label{eq:thm:de_morgans_laws/complement_of_meet}
  \end{align}

  If \( L \) is \hyperref[def:semilattice/complete]{complete}, \( \mscrK \) may be taken to be any family, not necessarily finite.
\end{theorem}
\begin{proof}
  We will only show \eqref{eq:thm:de_morgans_laws/complement_of_join} since \eqref{eq:thm:de_morgans_laws/complement_of_meet} is dual.

  In order for \( \wedge_{k \in \mscrK} \overline{x_k} \) to be the complement of \( \vee_{k \in \mscrK} x_k \), the conditions \eqref{def:bounded_lattice_complement/join} and \eqref{def:bounded_lattice_complement/meet} need to be satisfied.

  From distributivity, we have
  \begin{equation*}
    \parens*{ \bigwedge_{k \in \mscrK} \overline{x_k} } \vee \parens*{ \bigvee_{m \in \mscrK} x_m }
    \reloset {\eqref{eq:def:semilattice/distributive_lattice/arbitrary/join_over_meet}} =
    \bigwedge_{k \in \mscrK} \parens*{ \overline{x_k} \vee \bigvee_{m \in \mscrK} x_m }
    =
    \bigwedge_{k \in \mscrK} \parens*{ \underbrace{\overline{x_k} \vee x_k}_{\top} \vee \bigvee_{\mathclap{m \in K \setminus \set{k}}} x_m }
    =
    \bigwedge_{k \in \mscrK} \top
    =
    \top,
  \end{equation*}
  which proves \eqref{def:bounded_lattice_complement/join}. The proof of \eqref{def:bounded_lattice_complement/meet} is analogous.
\end{proof}

\begin{proposition}\label{thm:boolean_prime_iff_maximal}
  An  \hyperref[def:lattice_ideal/ideal]{ideal} (resp. filter) in a \hyperref[def:boolean_algebra]{Boolean algebra} is \hyperref[def:lattice_ideal/prime]{prime} if and only if it is \hyperref[def:lattice_ideal/maximal]{maximal}.
\end{proposition}
\begin{proof}
  \SufficiencySubProof Let \( P \) be a prime filter.

  Suppose that \( P \) is strictly contained in some other filter \( F \). Let \( x \in F \setminus P \). We have \( x \vee \overline x = \top \) and \( \top \in P \), hence primality implies \( \overline x \in P \).

  But \( P \subseteq F \), thus \( x \in F \) and \( \overline x \in F \), which via \fullref{thm:improper_boolean_ideal} implies that \( F = L \).

  Therefore, \( P \) is maximal.

  \NecessitySubProof Follows from \fullref{thm:semilattice_ideal_as_semiring_ideal} and \fullref{thm:def:semiring_ideal/maximal_is_prime}.
\end{proof}

\begin{lemma}\label{thm:improper_boolean_ideal}
  The only \hyperref[def:lattice_ideal/ideal]{ideal or filter} that contains both an element and its complement is the algebra itself.
\end{lemma}
\begin{proof}
  If \( x \in I \) and \( \overline x \in I \) for some ideal \( I \) in \( L \), since \( x \wedge \overline x = \bot \) and since ideals are downward closed, \( I = L \).

  The proof for filters is dual.
\end{proof}

\begin{definition}\label{def:ultrafilter}\mimprovised
  We say that a proper \hyperref[def:lattice_ideal/ideal]{filter} \( F \) in a \hyperref[def:boolean_algebra]{Boolean algebra} \( L \) is an \term{ultrafilter} if any of the following equivalent conditions hold:
  \begin{thmenum}
    \thmitem{def:ultrafilter/direct} For every \( x \in L \), either \( x \in F \) or \( \overline x \in F \).

    Both \( x \in F \) and \( \overline x \in F \) cannot hold because of \fullref{thm:improper_boolean_ideal}.

    \thmitem{def:ultrafilter/prime} \( F \) is a \hyperref[def:lattice_ideal/prime]{prime filter}.

    \thmitem{def:ultrafilter/maximal} \( F \) is a \hyperref[def:lattice_ideal/maximal]{maximal filter}.
  \end{thmenum}
\end{definition}
\begin{proof}
  \ImplicationSubProof{def:ultrafilter/direct}{def:ultrafilter/prime} Suppose that either \( x \in F \) or \( \overline x \in F \) for every \( x \in L \).

  Let \( x \vee y \in F \). If \( x \not\in F \), then \( \overline x \in F \) and hence the following is also a member of \( F \):
  \begin{equation*}
    \overline x \vee (x \vee y)
    =
    (\overline x \vee x) \vee y
    =
    \top \vee y
    =
    y.
  \end{equation*}

  Hence, if \( x \not\in F \), then \( y \in F \).

  Since \( x \) was chosen arbitrarily, we conclude that \( F \) is a prime filter.

  \ImplicationSubProof{def:ultrafilter/prime}{def:ultrafilter/direct} Let \( P \) be a prime filter.

  Suppose that \( x \not\in P \). Then \( x \vee \overline x = \top \in P \), which via primality implies that \( \overline x \in P \).

  \EquivalenceSubProof{def:ultrafilter/prime}{def:ultrafilter/maximal} Follows from \fullref{thm:boolean_prime_iff_maximal}.
\end{proof}

\begin{example}\label{ex:principal_ultrafilter}
  For every element \( x \) of a nonempty set \( X \), we define its \term{principal ultrafilter}
  \begin{equation*}
    F \coloneqq \set{ A \subseteq X \given x \in A }.
  \end{equation*}

  It is an \hyperref[def:ultrafilter]{ultrafilter} in the \hyperref[thm:boolean_algebra_of_subsets]{Boolean algebra of subsets} \( \pow(X) \). Indeed, if \( A \subseteq X \) is not in the filter \( F \), then \( x \in X \setminus A \).

  The concept should not be confused with \hyperref[def:lattice_ideal/principal]{principal filters}.
\end{example}

\begin{lemma}[Ultrafilter lemma]\label{thm:ultrafilter_lemma}
  Every proper \hyperref[def:lattice_ideal/ideal]{filter} in a \hyperref[def:boolean_algebra]{Boolean algebra} is contained in an \hyperref[def:ultrafilter]{ultrafilter}.
\end{lemma}
\begin{proof}
  Follows from \fullref{thm:semilattice_ideal_as_semiring_ideal} and \fullref{thm:maximal_ideal_theorem}.
\end{proof}
