\subsection{Boolean algebras}\label{subsec:boolean_algebras}

\paragraph{Heyting algebras}

\begin{proposition}\label{thm:heyting_conditional_set_is_ideal}
  In any \hyperref[def:lattice]{lattice} \( X \), for any two elements \( x \) and \( y \), the following set is nonempty and \hyperref[def:closed_ordered_subset]{downward closed}:
  \begin{equation*}
    H_{x,y} \coloneqq \set{ c \in X \given x \wedge c \leq y }.
  \end{equation*}

  If \( X \) is \hyperref[def:distributive_lattice]{distributive}, then \( H_{x,y} \) is \hyperref[def:directed_set]{upward directed} and hence a \hyperref[def:lattice_ideal]{lattice ideal}.
\end{proposition}
\begin{proof}
  The set \( H_{x,y} \) is nonempty --- it contains \( y \) because \( x \wedge y \leq y \). Furthermore, if \( a \) belongs to \( H_{x,y} \) and \( b \leq a \), then \( x \wedge b \leq x \wedge a \leq y \), thus \( b \) also belongs to \( H_{x,y} \).

  If \( X \) is distributive and if \( a \) and \( b \) belong to \( H_{x,y} \), then
  \begin{equation*}
    x \wedge (a \vee b)
    \reloset {\eqref{eq:def:distributive_lattice/meet_over_join}} =
    (x \wedge a) \vee (x \wedge b)
    \reloset {\ref{thm:def:lattice/operations_preserve_order}} \leq
    y \vee y
    =
    y,
  \end{equation*}
  hence their least upper bound \( a \vee b \) is also in \( H_{x,y} \).
\end{proof}

\begin{example}\label{ex:heyting_conditional_ideal}
  We will list several examples related to \fullref{thm:heyting_conditional_set_is_ideal}.

  \begin{thmenum}
    \thmitem{ex:heyting_conditional_ideal/pentagon} Without the assumption of distributivity, \fullref{thm:heyting_conditional_set_is_ideal} may not hold. Indeed, consider the pentagon lattice \eqref{eq:ex:def:modular_lattice/pentagon}. We have
    \begin{equation*}
      H_{a,b} = \set{ \bot, b, c },
    \end{equation*}
    but
    \begin{equation*}
      a \wedge (b \vee c) = a \wedge \top = a
    \end{equation*}
    is not in \( H_{a,b} \).

    Nevertheless, \( H_{b,c} = \set{ \bot, c } \) is an ideal --- in fact, the principal ideal of \( c \).

    \thmitem{ex:heyting_conditional_ideal/diamond} Consider the diamond lattice \eqref{eq:ex:def:distributive_lattice/diamond}. We have
    \begin{equation*}
      H_{a,c} = \set{ \bot, b, c }
    \end{equation*}
    but, again,
    \begin{equation*}
      a \wedge (b \vee c) = a \wedge \top = a,
    \end{equation*}
    so \( H_{a,c} \) is also not an ideal.

    \thmitem{ex:heyting_conditional_ideal/nonprincipal_ideal} Adjoin to the \hyperref[def:ordinal]{ordinal} \( \omega + 1 \) an auxiliary element \( a \) that is comparable only to the bottom \( 0 \) and the top \( \omega \). The following is a fragment of the Hasse diagram of \( \omega \cup \set{ a } \):
    \begin{equation*}
      \begin{aligned}
        \includegraphics[page=2]{output/ex__heyting_conditional_ideal}
      \end{aligned}
    \end{equation*}

    It is a distributive lattice as a consequence of \fullref{thm:distributive_lattice_characterization}.

    We have \( H_{a,0} = \omega \), which is an ideal of \( \omega \cup \set{ a } \). It is not, however, a principal ideal, because it contains no greatest element.

    On the other hand, the ideal \( H_{\omega,a} = \set{ 0, a } \) is principal.
  \end{thmenum}
\end{example}

\begin{proposition}\label{thm:heyting_conditional_set}
  For every two \hyperref[def:lattice]{lattice} elements \( x \) and \( y \), the following are equivalent for a third element \( z \):
  \begin{thmenum}
    \thmitem{thm:heyting_conditional_set/iff} We have \( c \leq z \) if and only if \( x \wedge c \leq y \).
    \thmitem{thm:heyting_conditional_set/greatest} We have \( x \wedge z \leq y \) and \( z \) is the greatest among all such elements.
    \thmitem{thm:heyting_conditional_set/ideal} The set\footnote{We have not required the lattice to be distributive, hence \( H_{x,y} \) may fail to be an ideal in general.} \( H_{x,y} \) from \fullref{thm:heyting_conditional_set_is_ideal} is a \hyperref[def:lattice_ideal/principal]{principal ideal} generated by \( z \).
  \end{thmenum}
\end{proposition}
\begin{comments}
  \item In particular, if \( z \) exists, it is unique.
\end{comments}
\begin{proof}
  \ImplicationSubProof{thm:heyting_conditional_set/iff}{thm:heyting_conditional_set/greatest} If \( c \) is an element such that \( x \wedge c \leq y \), by assumption \( c \leq z \). Then \( z \) is the greatest among all such elements.

  \ImplicationSubProof{thm:heyting_conditional_set/greatest}{thm:heyting_conditional_set/ideal} Trivial.

  \ImplicationSubProof{thm:heyting_conditional_set/ideal}{thm:heyting_conditional_set/iff} If \( c \leq z \), then \( c \) is in \( H_{x,y} \) and hence \( x \wedge c \leq y \). Conversely, if \( x \wedge c \leq y \), then \( c \) is in \( H_{x,y} \) and thus \( c \leq z \).
\end{proof}

\begin{definition}\label{def:heyting_algebra}\mcite[331]{PicadoPultr2012}
  A \term{Heyting algebra} is a \hyperref[def:extremal_points/bounds]{bounded} \hyperref[def:lattice]{lattice} \( X \) with an additional \hyperref[def:binary_operation]{binary operation} \( {\rightarrow} \), which we will call it the \term{relative pseudocomplement} in accordance to \cite[51]{Birkhoff1967}, such that any of the following equivalent conditions hold:
  \begin{thmenum}[series=def:heyting_algebra]
    \thmitem{def:heyting_algebra/ideal} The element \( x \rightarrow y \) satisfies the equivalent conditions in \fullref{thm:heyting_conditional_set}.

    \thmitem{def:heyting_algebra/axioms} The following first-order axioms hold:
    \begin{thmenum}
      \thmitem{def:heyting_algebra/axioms/self} The following adaptation of \fullref{thm:boolean_equivalences/self_conditional}:
      \begin{equation}\label{eq:def:heyting_algebra/axioms/self}
        \mathllap{\xi \rightarrow \xi} \doteq \mathrlap{\top.}
      \end{equation}

      \thmitem{def:heyting_algebra/axioms/modus_ponens} The following variation of \eqref{eq:def:def:axiomatic_deductive_system/mp}:
      \begin{equation}\label{eq:def:heyting_algebra/axioms/modus_ponens}
        \mathllap{\xi \wedge (\xi \rightarrow \eta)} \doteq \mathrlap{\xi \wedge \eta.}
      \end{equation}

      \thmitem{def:heyting_algebra/axioms/modular} The following consequence of the modular identity \eqref{eq:def:modular_lattice}\footnote
        {
          If \( \varphi \) and \( \psi \) are \hyperref[def:propositional_syntax/formula]{propositional formulas}, then
          \begin{equation*}
            \varphi \wedge (\psi \rightarrow \varphi)
            \reloset {\eqref{eq:thm:boolean_equivalences/conditional_as_disjunction}} \gleichstark
            \varphi \wedge (\neg \psi \vee \varphi)
            \reloset {\eqref{eq:def:modular_lattice}} \gleichstark
            (\varphi \wedge \neg \psi) \vee \varphi
            \gleichstark
            \varphi.
          \end{equation*}
        }:
      \begin{equation}\label{eq:def:heyting_algebra/axioms/modular}
        \mathllap{\eta \wedge (\xi \rightarrow \eta)} \doteq \mathrlap{\eta.}
      \end{equation}

      \thmitem{def:heyting_algebra/axioms/distributive} Distributivity of the relative pseudocomplement over conjunction:
      \begin{equation}\label{eq:def:heyting_algebra/axioms/distributive}
        \mathllap{\xi \rightarrow (\eta \wedge \zeta)} \doteq \mathrlap{(\xi \rightarrow \eta) \wedge (\xi \rightarrow \zeta).}
      \end{equation}
    \end{thmenum}
  \end{thmenum}

  Heyting algebras also have the following additional structure:
  \begin{thmenum}[resume=def:heyting_algebra]
    \thmitem{def:heyting_algebra/pseudocomplement} A unary operation \( {\widetilde {\anon}} \), which we call the \term{pseudocomplement}, for which the following axiom holds:
    \begin{equation}\label{eq:def:heyting_algebra/pseudocomplement}
      \widetilde \xi \doteq \xi \rightarrow \bot.
    \end{equation}
  \end{thmenum}

  Heyting algebras have the following metamathematical properties:
  \begin{thmenum}[resume=def:heyting_algebra]
    \thmitem{def:heyting_algebra/theory} We extend the \hyperref[def:lattice/theory]{first-order theory of lattices} by adding:
    \begin{itemize}
      \item The nullary functional symbols \( \top \) and \( \bot \) along with the axiom \eqref{eq:thm:def:lattice/bounded_absorption/join} and \eqref{eq:thm:def:lattice/bounded_absorption/meet}.

      \item The infix binary functional symbol \( \rightarrow \) along with the axioms \eqref{eq:def:heyting_algebra/axioms/self}, \eqref{eq:def:heyting_algebra/axioms/modus_ponens}, \eqref{eq:def:heyting_algebra/axioms/modular} and \eqref{eq:def:heyting_algebra/axioms/distributive}.

      \item The unary functional symbol \( {\widetilde {\anon}} \) along with the axiom \eqref{eq:def:heyting_algebra/pseudocomplement}.
    \end{itemize}

    \thmitem{def:heyting_algebra/submodel} In addition to containing the joins and meets of all its members, a \hyperref[def:first_order_submodel]{first-order submodel} of a Heyting algebra must also contain the top, bottom and be closed with respect to the relative pseudocomplement\footnote{It follows from \eqref{eq:def:heyting_algebra/pseudocomplement} that if a Heyting subalgebra is closed under relative pseudocomplements and it contains the bottom element, then it is also closed under pseudocomplements.}. We call such submodels \term{Heyting subalgebras}.

    \thmitem{def:heyting_algebra/homomorphism} A function \( f: X \to Y \) between Heyting algebras is a \hyperref[def:first_order_homomorphism]{first-order homomorphisms} if it is a \hyperref[def:lattice/homomorphism]{lattice homomorphism} that additionally satisfies
    \begin{align}\label{eq:def:heyting_algebra/homomorphism/top_bottom}
      f(\top_X) = \top_Y
      &&
      f(\bot_X) = \bot_Y
    \end{align}
    and
    \begin{equation}\label{eq:def:heyting_algebra/homomorphism/operation}
      f(x_1 \rightarrow_X x_2) = f(x_1) \rightarrow_Y f(x_2).
    \end{equation}

    \thmitem{def:heyting_algebra/opposite} The \hyperref[def:lattice/opposite]{dual lattice} \( X^{\opcat} \) of a Heyting algebra may not be a Heyting algebra.

    \thmitem{def:heyting_algebra/category} We denote the \hyperref[def:category_of_small_first_order_models]{category of \( \mscrU \)-small models} for Heyting algebras by \( \cat{Heyt} \). It is a subcategory of the \hyperref[def:lattice/category]{category \( \cat{Lat} \) of lattices}.
  \end{thmenum}
\end{definition}
\begin{comments}
  \item What we call a Heyting algebra is called a \enquote{Browerian lattice} by \incite[147]{Birkhoff1967} and a \enquote{browerian algebra} \incite[7]{Golan2010}. The latter uses \enquote{Heyting algebras} for what we call a \hyperref[def:complete_lattice]{complete} Heyting algebra.
\end{comments}
\begin{defproof}
  \ImplicationSubProof{def:heyting_algebra/ideal}{def:heyting_algebra/axioms} Suppose that, for any elements \( x \) and \( y \), we have
  \begin{equation}\label{eq:def:heyting_algebra/proof/iff}
    c \leq (x \rightarrow y) \T{if and only if} x \wedge c \leq y.
  \end{equation}

  \SubProofOf*{def:heyting_algebra/axioms/self} Clearly \( x \wedge \top = x \), thus \( \top \leq (x \rightarrow x) \) and hence \( x \rightarrow x \) is the top element.

  \SubProofOf*{def:heyting_algebra/axioms/modus_ponens} \eqref{eq:def:heyting_algebra/proof/iff} implies
  \begin{equation*}
    x \wedge (x \rightarrow y) \leq y.
  \end{equation*}

  Then, since \( x \wedge \anon \) preserves order,
  \begin{equation*}
    \underbrace{x \wedge (x \wedge (x \rightarrow y))}_{x \wedge (x \rightarrow y)} \leq x \wedge y.
  \end{equation*}

  Conversely, since \( x \wedge y \leq y \), \eqref{eq:def:heyting_algebra/proof/iff} implies that \( y \leq (x \rightarrow y) \). Hence,
  \begin{equation*}
    x \wedge y \leq x \wedge (x \rightarrow y).
  \end{equation*}

  Then \eqref{eq:def:heyting_algebra/axioms/modus_ponens} follows.

  \SubProofOf*{def:heyting_algebra/axioms/modular} Since \( y \leq (x \rightarrow y) \), obviously
  \begin{equation*}
    y \wedge (x \rightarrow y) = y.
  \end{equation*}

  \SubProofOf*{def:heyting_algebra/axioms/distributive} \eqref{eq:def:heyting_algebra/proof/iff} implies
  \begin{equation*}
    x \wedge (x \rightarrow y) \leq y
  \end{equation*}
  and similarly
  \begin{equation*}
    x \wedge (x \rightarrow z) \leq z.
  \end{equation*}

  Then \fullref{thm:def:lattice/operations_preserve_order} implies that
  \begin{equation*}
    x \wedge \parens[\Big]{ (x \rightarrow y) \wedge (x \rightarrow z) } \leq y \wedge z.
  \end{equation*}

  Hence,
  \begin{equation}\label{eq:def:heyting_algebra/proof/distributive_backward}
    (x \rightarrow y) \wedge (x \rightarrow z) \leq x \rightarrow (y \wedge z).
  \end{equation}

  Conversely, we have
  \begin{equation*}
    x \wedge (x \rightarrow (y \wedge z)) \leq y \wedge z \leq y,
  \end{equation*}
  thus
  \begin{equation*}
    x \rightarrow (y \wedge z) \leq (x \rightarrow y)
  \end{equation*}
  and similarly
  \begin{equation*}
    x \rightarrow (y \wedge z) \leq (x \rightarrow z).
  \end{equation*}

  Therefore,
  \begin{equation}\label{eq:def:heyting_algebra/proof/distributive_forward}
    x \rightarrow (y \wedge z) \leq (x \rightarrow y) \wedge (x \rightarrow z).
  \end{equation}

  Combining \eqref{eq:def:heyting_algebra/proof/distributive_backward} and \eqref{eq:def:heyting_algebra/proof/distributive_forward}, we obtain \eqref{eq:def:heyting_algebra/axioms/distributive}.

  \ImplicationSubProof{def:heyting_algebra/axioms}{def:heyting_algebra/ideal} Suppose that the axioms from \fullref{def:heyting_algebra/axioms} hold.

  First let \( c \leq (x \rightarrow y) \). \Fullref{thm:def:lattice/join_preserves_order} implies that
  \begin{equation*}
    x \wedge c
    \leq
    x \wedge (x \rightarrow y)
    \reloset {\eqref{eq:def:heyting_algebra/axioms/modus_ponens}} =
    x \wedge y
    \leq
    y.
  \end{equation*}

  Conversely, if \( x \wedge c \leq y \), we have
  \begin{equation*}
    c
    \reloset {\eqref{eq:def:heyting_algebra/axioms/modular}} =
    c \wedge (x \rightarrow c)
    \leq
    \top \wedge (x \rightarrow c)
    \reloset {\eqref{eq:def:heyting_algebra/axioms/self}} =
    (x \rightarrow x) \wedge (x \rightarrow c)
    \reloset {\eqref{eq:def:heyting_algebra/axioms/distributive}} =
    x \rightarrow (\underbrace{x \wedge c}_{\leq y})
    \leq
    x \rightarrow y.
  \end{equation*}
\end{defproof}

\begin{proposition}\label{thm:def:heyting_algebra}
  \hyperref[def:heyting_algebra]{Heyting algebras} have the following basic properties:
  \begin{thmenum}
    \thmitem{thm:def:heyting_algebra/distributive} Every Heyting algebra is a \hyperref[def:distributive_lattice]{distributive lattice}.
    \thmitem{thm:def:heyting_algebra/leq} We always have \( y \leq (x \rightarrow y) \).
  \end{thmenum}
\end{proposition}
\begin{proof}
  \SubProofOf{thm:def:heyting_algebra/distributive} \Fullref{ex:heyting_conditional_ideal/pentagon} implies that the pentagon lattice \eqref{eq:ex:def:modular_lattice/pentagon} is not a Heyting algebra, and \fullref{ex:heyting_conditional_ideal/diamond} implies that the diamond lattice \eqref{eq:ex:def:distributive_lattice/diamond} is not a Heyting algebra. Therefore, no Heyting algebra contains as a sublattice either \( N_5 \) or \( M_3 \).

  Then \fullref{thm:distributive_lattice_characterization} implies that Heyting algebras are distributive.

  \SubProofOf{thm:def:heyting_algebra/leq} Clearly \( y \in H_{x,y} \) because \( x \wedge y \leq y \).
\end{proof}

\begin{proposition}\label{thm:complete_heyting_algebra}
  A \hyperref[def:complete_lattice]{complete lattice} is a \hyperref[def:heyting_algebra]{Heyting algebra} if and only if the following infinite distributive law holds:
  \begin{equation}\label{eq:thm:complete_heyting_algebra/infinite_distributive}
    x \wedge \parens[\Big]{ \bigvee_{k \in \mscrK} y_k } = \bigvee_{k \in \mscrK} (x \wedge y_k).
  \end{equation}
\end{proposition}
\begin{proof}
  Fix a complete lattice \( X \).

  \SufficiencySubProof Suppose that \( X \) is a Heyting algebra.

  Fix some element \( y \) and an arbitrary indexed family \( \seq{ x_k }_{k \in \mscrK} \) from \( X \). We have, for every \( a \) in \( X \),
  \begin{equation}\label{eq:thm:complete_heyting_algebra/proof/first}
    x \wedge \parens[\Big]{ \bigvee_{k \in \mscrK} y_k } \leq a
  \end{equation}
  if and only if
  \begin{equation*}
    \bigvee_{k \in \mscrK} y_k \leq (x \rightarrow a)
  \end{equation*}
  if and only if, for every \( k \in \mscrK \),
  \begin{equation*}
    y_k \leq (x \rightarrow a)
  \end{equation*}
  if and only if, for every \( k \in \mscrK \),
  \begin{equation}\label{eq:thm:complete_heyting_algebra/proof/last}
    x \wedge y_k \leq a.
  \end{equation}

  Taking \( a \) to be \( \bigvee_{k \in \mscrK} (x \wedge y_k) \), we obtain a true statement in \eqref{eq:thm:complete_heyting_algebra/proof/last}, hence \eqref{eq:thm:complete_heyting_algebra/proof/first} holds:
  \begin{equation*}
    \parens[\Big]{ \bigvee_{k \in \mscrK} y_k } \leq \bigvee_{k \in \mscrK} (x \wedge y_k).
  \end{equation*}

  The converse follows from \fullref{thm:def:complete_lattice/distributivity_inequality}.

  \NecessitySubProof Suppose that \eqref{eq:thm:complete_heyting_algebra/infinite_distributive} holds. Define
  \begin{equation*}
    (x \rightarrow y) \coloneqq \bigvee\set{ c \in X \given x \wedge c \leq y }.
  \end{equation*}

  Obviously \( a \wedge x \leq y \) implies \( a \leq (x \rightarrow y) \).

  Conversely, suppose that \( a \leq (x \rightarrow y) \). Then
  \begin{equation*}
    x \wedge a
    \leq
    x \wedge (x \rightarrow y)
    =
    x \wedge \bigvee\set{ c \in X \given x \wedge c \leq y }
    \reloset {\eqref{eq:thm:complete_heyting_algebra/infinite_distributive}} =
    \bigvee\set{ x \wedge c \given x \wedge c \leq y }
    =
    y.
  \end{equation*}

  Generalizing on \( x \) and \( y \), we conclude that \( {\rightarrow} \) is indeed a relative pseudocomplement and thus \( X \) is a Heyting algebra.
\end{proof}

\begin{example}\label{ex:def:heyting_algebra}
  We list examples of \hyperref[def:heyting_algebra]{Heyting algebras}:
  \begin{thmenum}
    \thmitem{ex:def:heyting_algebra/lindenbaum_tarski} As shown in \fullref{thm:intuitionistic_lindenbaum_tarski_algebra}, every \hyperref[def:lindenbaum_tarski_algebra]{Lindenbaum-Tarski algebra} for the \hyperref[def:intuitionistic_propositional_deductive_systems]{intuitionistic propositional natural deduction system} is a Heyting algebra.

    \thmitem{ex:def:heyting_algebra/topology} The topology \( \mscrT \) of a \hyperref[def:topological_space]{topological space} \( (X, \mscrT) \) is a complete Heyting algebra.

    Indeed,
    \begin{itemize}
      \item \hyperref[def:lattice/join]{Arbitrary joins} are given by \hyperref[def:basic_set_operations/union]{unions}.
      \item \hyperref[def:lattice/meet]{Finite meets} are given by \hyperref[def:basic_set_operations/intersection]{intersections}.
      \item The \hyperref[def:extremal_points/top_and_bottom]{top element} is the entire domain \( L \).
      \item The \hyperref[def:extremal_points/top_and_bottom]{bottom element} is the empty set.
      \item The \hyperref[eq:def:heyting_algebra/conditional]{conditional \( U \rightsquigarrow V \)} is then
      \begin{equation*}
        \bigcup\set[\Big]{ A \in T \given \underbrace{A \cap U}_{A \setminus (X \setminus U)} \subseteq V }
        =
        \bigcup\set[\Big]{ A \in T \given A \subseteq V \cup (X \setminus U) }
        =
        \Int((X \setminus U) \cup V),
      \end{equation*}
      which is actually similar to \fullref{thm:boolean_equivalences/conditional_as_disjunction} despite the fact that arbitrary topologies are not Boolean algebras.

      \item As a result, the \hyperref[def:heyting_algebra/pseudocomplement]{pseudocomplement} is
      \begin{equation*}
        \widetilde U = \Int(X \setminus U).
      \end{equation*}
    \end{itemize}

    This is actually used in topological semantics --- see \fullref{def:propositional_topological_semantics}.
  \end{thmenum}
\end{example}

\paragraph{Boolean algebra}

\begin{definition}\label{def:extremal_points/bounds_complement}
  Let \( L \) be a \hyperref[def:extremal_points/bounds]{bounded} \hyperref[def:lattice]{lattice} and fix an element \( x \in L \). A \term{complement} of \( x \) is an element \( y \) such that
  \begin{align}
    x \vee y = \top \label{def:extremal_points/bounds_complement/join}, \\
    x \wedge y = \bot \label{def:extremal_points/bounds_complement/meet}.
  \end{align}

  Due to the commutativity of both \( \vee \) and \( \wedge \), \( y \) is a complement of \( x \) if and only if \( x \) is a complement of \( y \).
\end{definition}

\begin{proposition}\label{thm:distributive_bounded_lattice_unique_complement}
  In a \hyperref[def:extremal_points/bounds]{bounded} \hyperref[def:distributive_lattice]{distributive lattice} \( L \), each \( x \in L \) has at most one complement.

  Thus, complementation can be regarded as a \hyperref[def:set_valued_map/partial]{partial operation}.
\end{proposition}
\begin{proof}
  If \( y \) and \( z \) are both complements of \( x \), then
  \begin{balign*}
    y
    &\reloset {\eqref{eq:thm:def:lattice/bounded_absorption/meet}} =
    y \wedge \top
    = \\ &\reloset {\eqref{def:extremal_points/bounds_complement/join}} =
    y \wedge (z \vee x)
    = \\ &\reloset {\eqref{eq:def:distributive_lattice/meet_over_join}} =
    (y \wedge z) \vee (y \wedge x)
    = \\ &\reloset {\eqref{def:extremal_points/bounds_complement/meet}} =
    y \wedge z
    = \\ &\reloset {\eqref{def:extremal_points/bounds_complement/meet}} =
    (x \wedge z) \vee (y \wedge z)
    = \\ &\reloset {\eqref{eq:def:distributive_lattice/meet_over_join}} =
    (x \vee y) \wedge z
    = \\ &\reloset {\eqref{eq:thm:def:lattice/bounded_absorption/meet}} =
    z.
  \end{balign*}
\end{proof}

\begin{definition}\label{def:boolean_algebra}\mcite[def. 1.7.1]{Hinman2005}
  A \term{Boolean algebra} is a \hyperref[def:extremal_points/bounds]{bounded} \hyperref[def:distributive_lattice]{distributive lattice} in which every element has a \hyperref[def:extremal_points/bounds_complement]{complement}. The complement of each element is unique due to \fullref{thm:distributive_bounded_lattice_unique_complement}. We define a unary function that gives to every element \( x \) its complement \( \overline \anon \). By definition, this function is an \hyperref[def:involution]{involution}.

  \begin{thmenum}[series=def:boolean_algebra]
    \thmitem{def:boolean_algebra/conditional} We also define the binary operation \term{conditional} (\( \rightarrow \)) via
    \begin{equation}\label{eq:def:boolean_algebra/conditional}
      (x \rightarrow y) \coloneqq (\overline x \vee y)
    \end{equation}
    in analogy with \fullref{thm:boolean_equivalences/conditional_as_disjunction}. This operation highlights that Boolean algebras are a special case of \hyperref[thm:boolean_algebras_are_heyting_algebras]{Heyting algebras}.

    \thmitem{def:boolean_algebra/biconditional} It remains to define a binary operation corresponding to the \hyperref[def:propositional_language/connectives/biconditional]{propositional biconditional}. Inspired by \fullref{thm:boolean_equivalences/biconditional_via_conditionals}, define
    \begin{equation}\label{eq:def:boolean_algebra/biconditional}
      (x \leftrightarrow y) \coloneqq (x \rightarrow y) \wedge (y \rightarrow x).
    \end{equation}
  \end{thmenum}

  Boolean algebras have the following metamathematical properties:
  \begin{thmenum}[resume=def:boolean_algebra]
    \thmitem{def:boolean_algebra/theory} To obtain the theory of Boolean algebras, we replace the unary functional symbol \( \widetilde{\anon} \) with \( \overline \anon \) in the language of the \hyperref[def:heyting_algebra/theory]{theory of Heyting algebras} and then add the axioms \eqref{def:extremal_points/bounds_complement/join} and \eqref{def:extremal_points/bounds_complement/meet} to the theory. We may also replace \eqref{eq:def:heyting_algebra/conditional} defining \( \rightarrow \) with the simpler axiom \eqref{eq:def:boolean_algebra/conditional}.

    \thmitem{def:boolean_algebra/submodel} The Boolean subalgebras are the \hyperref[def:lattice/submodel]{bounded sublattices} which are closed under compementation.

    \thmitem{def:boolean_algebra/homomorphism} \hyperref[def:first_order_homomorphism]{First-order homomorphisms} between Boolean algebras are simply lattice homomorphisms.

    Complements are automatically preserved because for any lattice homomorphism \( \varphi \) between the Boolean algebras \( L \) and \( Y \),
    \begin{equation*}
      \varphi(x) \vee_Y \varphi(\overline x) = \varphi(x \vee_X \overline x) = \varphi(\top_X) = \top_Y,
    \end{equation*}
    and similarly for \( \wedge \), hence, due to \fullref{thm:distributive_bounded_lattice_unique_complement},
    \begin{equation*}
      \varphi(\overline x) = \overline {\varphi(x)}.
    \end{equation*}

    Implications are also automatically preserved because of \eqref{eq:def:boolean_algebra/conditional}.

    \thmitem{def:boolean_algebra/category} The \hyperref[def:category_of_small_first_order_models]{category of \( \mscrU \)-small models} for Boolean algebras \( \cat{Bool} \) is a full subcategory the \hyperref[def:heyting_algebra/category]{category \( \cat{Heyt} \) of Heyting algebras}.

    \thmitem{def:boolean_algebra/opposite} The \hyperref[thm:lattice_duality]{principle of duality for lattices} holds for Boolean algebras without interchanging complements.
  \end{thmenum}
\end{definition}

\begin{example}\label{ex:boolean_algebras}
  Examples of \hyperref[def:boolean_algebra]{Boolean algebras} include:

  \begin{itemize}
    \thmitem{ex:boolean_algebras/lindenbaum_tarski} The \hyperref[def:lindenbaum_tarski_algebra]{Lindenbaum-Tarski algebras} for classical logic. We prove in \fullref{thm:intuitionistic_lindenbaum_tarski_algebra} that it is a Boolean algebra.
    \thmitem{ex:boolean_algebras/f2} The isomorphic (as per \fullref{thm:two_element_lattice}) Boolean algebras \hyperref[def:boolean_value]{\( \set{ T, F } \)} and \hyperref[def:finite_field]{\( \BbbF_2 \)} are used extensively in \fullref{sec:mathematical_logic}.
    \thmitem{ex:boolean_algebras/power_set} The power set of any set, usually taken to be a space with additional structure (see \fullref{thm:boolean_algebra_of_subsets}).
  \end{itemize}
\end{example}

\begin{proposition}\label{thm:boolean_algebras_are_heyting_algebras}
  Every \hyperref[def:boolean_algebra]{Boolean algebra} is a \hyperref[def:heyting_algebra]{Heyting algebra} with an identification given by \eqref{eq:def:boolean_algebra/conditional}.
\end{proposition}
\begin{proof}
  Fix any \( x, y \in L \) in a Boolean algebra \( L \). We will show that \( x \rightarrow y \) as defined in \eqref{eq:def:boolean_algebra/conditional} satisfies \eqref{eq:def:heyting_algebra/conditional}.

  Let
  \begin{equation*}
   A \coloneqq \set{ a \in L \given a \wedge x \leq y }
  \end{equation*}
  be the set from \eqref{eq:def:heyting_algebra/conditional}.

  We will show that \( \overline x \vee y \) is an \hyperref[def:extremal_points/bounds]{upper bound} of \( A \).

  Fix some \( a_0 \in A \). By definition of \( A \), we have
  \begin{equation*}
   a_0 \wedge x \leq y.
  \end{equation*}

  But this means that
  \begin{equation*}
   \underbrace{(a_0 \wedge x) \vee \overline x}_{a_0 \vee \overline x} \leq y \vee \overline x.
  \end{equation*}

  Since \( a_0 \leq a_0 \vee b \) for any \( b \in L \), it follows that \( a_0 \leq y \vee \overline x \). Therefore, \( \overline x \vee y \) is indeed an upper bound of \( A \).

  Also note that
  \begin{equation*}
   (\overline x \vee y) \wedge x = \underbrace{(\overline x \vee x)}_{\top} \wedge (y \wedge x) = y \wedge x \leq y,
  \end{equation*}
  hence \( \overline x \vee y \in A \).

  Thus, \( \overline x \vee y \) is both an upper bound of \( A \) and an element of \( A \), i.e. it is the least upper bound of \( A \). Therefore,
  \begin{equation*}
   \overline x \vee y = \bigvee A.
  \end{equation*}
\end{proof}

\begin{theorem}[De Morgan's laws]\label{thm:de_morgans_laws}
  If \( L \) is a \hyperref[def:boolean_algebra]{Boolean algebra}, the following hold for any finite \hyperref[def:cartesian_product/indexed_family]{family} \( \set{ x_k }_{k \in \mscrK} \subseteq X \):
  \begin{align}
    \overline{\bigvee_{k \in \mscrK} x_k} = \bigwedge_{k \in \mscrK} \overline{x_k} \label{eq:thm:de_morgans_laws/complement_of_join} \\
    \overline{\bigwedge_{k \in \mscrK} x_k} = \bigvee_{k \in \mscrK} \overline{x_k} \label{eq:thm:de_morgans_laws/complement_of_meet}
  \end{align}

  If \( L \) is \hyperref[def:complete_lattice]{complete}, \( \mscrK \) may be taken to be any family, not necessarily finite.
\end{theorem}
\begin{proof}
  We will only show \eqref{eq:thm:de_morgans_laws/complement_of_join} since \eqref{eq:thm:de_morgans_laws/complement_of_meet} is dual.

  In order for \( \wedge_{k \in \mscrK} \overline{x_k} \) to be the complement of \( \vee_{k \in \mscrK} x_k \), the conditions \eqref{def:extremal_points/bounds_complement/join} and \eqref{def:extremal_points/bounds_complement/meet} need to be satisfied.

  From distributivity, we have
  \begin{equation*}
    \parens*{ \bigwedge_{k \in \mscrK} \overline{x_k} } \vee \parens*{ \bigvee_{m \in \mscrK} x_m }
    \reloset {\eqref{eq:def:distributive_lattice/arbitrary/join_over_meet}} =
    \bigwedge_{k \in \mscrK} \parens*{ \overline{x_k} \vee \bigvee_{m \in \mscrK} x_m }
    =
    \bigwedge_{k \in \mscrK} \parens*{ \underbrace{\overline{x_k} \vee x_k}_{\top} \vee \bigvee_{\mathclap{m \in K \setminus \set{k}}} x_m }
    =
    \bigwedge_{k \in \mscrK} \top
    =
    \top,
  \end{equation*}
  which proves \eqref{def:extremal_points/bounds_complement/join}. The proof of \eqref{def:extremal_points/bounds_complement/meet} is analogous.
\end{proof}

\begin{proposition}\label{thm:boolean_prime_iff_maximal}
  An  \hyperref[def:lattice_ideal]{ideal} (resp. filter) in a \hyperref[def:boolean_algebra]{Boolean algebra} is \hyperref[def:lattice_ideal/prime]{prime} if and only if it is \hyperref[def:lattice_ideal/maximal]{maximal}.
\end{proposition}
\begin{proof}
  \SufficiencySubProof Let \( P \) be a prime filter.

  Suppose that \( P \) is strictly contained in some other filter \( F \). Let \( x \in F \setminus P \). We have \( x \vee \overline x = \top \) and \( \top \in P \), hence primality implies \( \overline x \in P \).

  But \( P \subseteq F \), thus \( x \in F \) and \( \overline x \in F \), which via \fullref{thm:improper_boolean_ideal} implies that \( F = L \).

  Therefore, \( P \) is maximal.

  \NecessitySubProof Follows from \fullref{thm:lattice_ideal_as_semiring_ideal} and \fullref{thm:def:lattice_ideal/principal/maximal_is_prime}.
\end{proof}

\begin{lemma}\label{thm:improper_boolean_ideal}
  The only \hyperref[def:lattice_ideal]{ideal or filter} that contains both an element and its complement is the algebra itself.
\end{lemma}
\begin{proof}
  If \( x \in I \) and \( \overline x \in I \) for some ideal \( I \) in \( L \), since \( x \wedge \overline x = \bot \) and since ideals are downward closed, \( I = L \).

  The proof for filters is dual.
\end{proof}

\begin{definition}\label{def:ultrafilter}\mimprovised
  We say that a proper \hyperref[def:lattice_ideal]{filter} \( F \) in a \hyperref[def:boolean_algebra]{Boolean algebra} \( L \) is an \term{ultrafilter} if any of the following equivalent conditions hold:
  \begin{thmenum}
    \thmitem{def:ultrafilter/direct} For every \( x \in L \), either \( x \in F \) or \( \overline x \in F \).

    Both \( x \in F \) and \( \overline x \in F \) cannot hold because of \fullref{thm:improper_boolean_ideal}.

    \thmitem{def:ultrafilter/prime} \( F \) is a \hyperref[def:lattice_ideal/prime]{prime filter}.

    \thmitem{def:ultrafilter/maximal} \( F \) is a \hyperref[def:lattice_ideal/maximal]{maximal filter}.
  \end{thmenum}
\end{definition}
\begin{proof}
  \ImplicationSubProof{def:ultrafilter/direct}{def:ultrafilter/prime} Suppose that either \( x \in F \) or \( \overline x \in F \) for every \( x \in L \).

  Let \( x \vee y \in F \). If \( x \not\in F \), then \( \overline x \in F \) and hence the following is also a member of \( F \):
  \begin{equation*}
    \overline x \vee (x \vee y)
    =
    (\overline x \vee x) \vee y
    =
    \top \vee y
    =
    y.
  \end{equation*}

  Hence, if \( x \not\in F \), then \( y \in F \).

  Since \( x \) was chosen arbitrarily, we conclude that \( F \) is a prime filter.

  \ImplicationSubProof{def:ultrafilter/prime}{def:ultrafilter/direct} Let \( P \) be a prime filter.

  Suppose that \( x \not\in P \). Then \( x \vee \overline x = \top \in P \), which via primality implies that \( \overline x \in P \).

  \EquivalenceSubProof{def:ultrafilter/prime}{def:ultrafilter/maximal} Follows from \fullref{thm:boolean_prime_iff_maximal}.
\end{proof}

\begin{example}\label{ex:principal_ultrafilter}
  For every element \( x \) of a nonempty set \( X \), we define its \term{principal ultrafilter}
  \begin{equation*}
    F_x \coloneqq \set{ A \subseteq X \given x \in A }.
  \end{equation*}

  It is an \hyperref[def:ultrafilter]{ultrafilter} in the \hyperref[thm:boolean_algebra_of_subsets]{Boolean algebra of subsets} \( \pow(X) \). Indeed, if \( A \subseteq X \) is not in the filter \( F \), then \( x \in X \setminus A \).

  The concept should not be confused with \hyperref[def:lattice_ideal/principal]{principal filters}.
\end{example}

\begin{lemma}[Ultrafilter lemma]\label{thm:ultrafilter_lemma}
  Every proper \hyperref[def:lattice_ideal]{filter} in a \hyperref[def:boolean_algebra]{Boolean algebra} is contained in an \hyperref[def:ultrafilter]{ultrafilter}.
\end{lemma}
\begin{proof}
  Follows from \fullref{thm:lattice_ideal_as_semiring_ideal} and \fullref{thm:maximal_ideal_theorem}.
\end{proof}
