\section{Curry-Howard correspondence}\label{sec:curry_howard_correspondence}

\begin{concept}\label{con:curry_howard_correspondence}
  \incite[479]{Howard1980FormulasAsTypes} writes
  \begin{displayquote}
    H. Curry (1958) has observed that there is a close correspondence between \textit{axioms} of positive implicational propositional logic, on the one hand, and \textit{basic combinators} on the other hand. For example, the combinator \( K = \qabs X \qabs Y X \) corresponds to the axiom \( a \rightimply (b \rightimply a) \).
  \end{displayquote}

  The publication he refers to is \cite[312]{CurryFeysCraig1958CombinatoryLogicVol1}, where Curry begins the section with
  \begin{displayquote}
    We shall study here a striking analogy between the theory of and the theory of functionality and the theory of implication in propositional algebra.
  \end{displayquote}

  The core idea is that the type \( \alpha \synimplies \rho \) can be regarded as a propositional formula, and its \hyperref[def:type_derivation_tree]{type derivation trees} then correspond to \hyperref[def:natural_deduction_proof_tree]{natural deduction proof trees}.

  These ideas are later developed and vastly amplified by Per Martin-L\"of, whose publication \cite{MartinLöf1975IntuitionisticTypeTheory} introduces \enquote{dependent types} (see \fullref{con:dependent_types}) for handling higher-order logic, at which Howard only hints. Because of the latter development, \enquote{intuitionistic type theory}, \enquote{dependent type theory} and \enquote{Martin-L\"of type theory} refer to this more abstract setting. \enquote{Simple type theory} continues to refer to the theory with only arrow types. Accidentally, the adjective \enquote{simple} is a contradistinction with Russell's type theory rather than Martin-L\"of's.

  We will call the overall concept of identifying types with formulas the \term{Curry-Howard correspondence}. This phrase is used by \incite[74]{Hindley1997BasicSTT}, \incite[45]{AwodeyWarren2009HoTT} and \incite[def. 4.1.7]{Mimram2020ProgramEqualsProof}. \incite[341]{BarendregtDekkersStatman2013LambdaCalculusWithTypes} extend this to \enquote{Curry-de Bruijn-Howard correspondence}. An alternative suggested by Howard himself is \enquote{formulas-as-types correspondence}, also used by \incite[74]{Hindley1997BasicSTT}, \incite[\S 1.3.4]{TroelstraSchwichtenberg2000BasicProofTheory} and \incite[572]{Barendregt1984LambdaCalculus}. Another variation, \enquote{propositions-as-types correspondence}, is used by \incite[def. 5.4.14]{Barendregt1992LambdaCalculiWithTypes}, \incite[prop. 6.3.11]{BarendregtDekkersStatman2013LambdaCalculusWithTypes}, \incite[8]{AwodeyWarren2009HoTT} and \incite[def. 4.1.7]{Mimram2020ProgramEqualsProof}.

  We formalize the base ideas via \fullref{alg:proof_tree_to_type_derivation} and \fullref{alg:type_derivation_to_proof_tree}. First, however, we must introduce additional types.
\end{concept}
\begin{comments}
  \item From the perspective of first-order logic, the correspondence is described in \incite[def. 5.1.11]{Mimram2020ProgramEqualsProof}.
\end{comments}

\paragraph{Extended simple type theory}\hfill

\begin{remark}\label{rem:type_theory_rule_classification}
\end{remark}

\begin{definition}\label{def:empty_type}\mcite[\S 4.3.4]{Mimram2020ProgramEqualsProof}
  The \term{empty type} \( \syn\Bbbzero \).
\end{definition}

\begin{definition}\label{def:unit_type}\mcite[\S 4.3.2]{Mimram2020ProgramEqualsProof}
  The \term{unit type} \( \syn\Bbbone \).
\end{definition}

\begin{definition}\label{def:product_type}
\end{definition}

\begin{definition}\label{def:sum_type}
\end{definition}

\begin{concept}\label{con:dependent_types}
  \todo{Discuss dependent types}
\end{concept}

\begin{concept}\label{con:homotopy_type_theory}
  \todo{Homotopy type theory}
\end{concept}

\paragraph{Curry-Howard correspondence}\hfill

\begin{example}\label{ex:con:curry_howard_correspondence}
  We list examples related to the \hyperref[con:curry_howard_correspondence]{Curry-Howard correspondence}:
  \begin{thmenum}
    \thmitem{ex:con:curry_howard_correspondence/minimal_implicational}
  \end{thmenum}
\end{example}

\begin{algorithm}\label{alg:type_derivation_to_proof_tree}
\end{algorithm}

\begin{algorithm}\label{alg:proof_tree_to_type_derivation}
\end{algorithm}
