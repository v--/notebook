\section{Curry-Howard correspondence}\label{sec:curry_howard_correspondence}

\begin{concept}\label{con:curry_howard_correspondence}
  \incite[479]{Howard1980FormulasAsTypes} writes
  \begin{displayquote}
    H. Curry (1958) has observed that there is a close correspondence between \textit{axioms} of positive implicational propositional logic, on the one hand, and \textit{basic combinators} on the other hand. For example, the combinator \( K = \qabs X \qabs Y X \) corresponds to the axiom \( a \rightimply (b \rightimply a) \).
  \end{displayquote}

  The publication he refers to is \cite[312]{CurryFeysCraig1958CombinatoryLogicVol1}, where Curry begins the section with
  \begin{displayquote}
    We shall study here a striking analogy between the theory of and the theory of functionality and the theory of implication in propositional algebra.
  \end{displayquote}

  Haskell Curry is credited for the realization that, in modern terms, the \hyperref[def:simple_type]{arrow type} \( \alpha \synimplies \rho \) can be regarded as a \hyperref[def:propositional_alphabet/connectives/conditional]{conditional formula}, and its \hyperref[def:type_derivation_tree]{type derivation trees} then correspond to \hyperref[def:natural_deduction_proof_tree]{natural deduction proof trees}. In \fullref{alg:proof_tree_to_type_derivation} and \fullref{alg:type_derivation_to_proof_tree}, we will extend this to conjunctions via \hyperref[def:product_type]{product types} and to disjunctions via \hyperref[def:sum_type]{sum types}.

  William Howard is credited for extending this analogy to \hyperref[sec:first_order_logic]{first-order logic} via what are now called \enquote{dependent types} (see \fullref{con:dependent_types}).

  Honoring them, we will refer to the overall identification of types and formulas as the \term[en=Curry-Howard correspondence (\cite[def. 4.1.7]{Mimram2020ProgramEqualsProof})]{Curry-Howard correspondence}.
\end{concept}
\begin{comments}
  \item From the perspective of first-order logic, the correspondence is described in \incite[def. 5.1.11]{Mimram2020ProgramEqualsProof}.

  \item Different authors refer to this concept slightly differently:
  \begin{itemize}
    \item This phrase \enquote{Curry-Howard correspondence} is used by \incite[45]{AwodeyWarren2009HoTT} and \incite[def. 4.1.7]{Mimram2020ProgramEqualsProof}.

    \item \incite[341]{BarendregtDekkersStatman2013LambdaCalculusWithTypes} extends this to \enquote{Curry-de Bruijn-Howard correspondence}.

    \item A variation, the \enquote{Curry-Howard isomorphism}, is used by \incite[74]{Hindley1997BasicSTT} and \incite[14]{GirardEtAl1989ProofsAndTypes}.

    \item  An alternative suggested by Howard himself is \enquote{formulas-as-types correspondence}, also used by \incite[74]{Hindley1997BasicSTT}, \incite[\S 1.3.4]{TroelstraSchwichtenberg2000BasicProofTheory} and \incite[572]{Barendregt1984LambdaCalculus}.

    \item Another variation, the \enquote{propositions-as-types correspondence}, is used by \incite[def. 5.4.14]{Barendregt1992LambdaCalculiWithTypes}, as well as by the aforementioned \incite[prop. 6.3.11]{BarendregtDekkersStatman2013LambdaCalculusWithTypes}, \incite[8]{AwodeyWarren2009HoTT} and \incite[def. 4.1.7]{Mimram2020ProgramEqualsProof}.
  \end{itemize}
\end{comments}

\paragraph{Extended simple type theory}\hfill

\begin{remark}\label{rem:type_theory_rule_classification}
  \incite[24]{MartinLöf1984IntuitionisticTypeTheory} uses (metatheoretic) \hyperref[def:inference_rule]{inference rules} to build his entire type theory. He classifies the different rules for each family of types as follows:
  \begin{thmenum}
    \thmitem{rem:type_theory_rule_classification/formation} A \term[en=formation (rule) (\cite[\S 8.1.9]{Mimram2020ProgramEqualsProof})]{type formation} rule allows us to introduce a new type given some premises.

    For example, in \fullref{def:simple_type}, we have introduced arrow types via the \hyperref[def:formal_grammar/schema]{formal grammar} rule
    \begin{bnf*}
      \bnfprod{arrow type} {\bnftsq{(} \bnfsp \bnfpn{type} \bnfsp \bnftsq{\( \synimplies \)} \bnfsp \bnfpn{type} \bnfsp \bnftsq{)}}
    \end{bnf*}
    which we can recast using \hyperref[rem:type_universes]{type universes} as the inference rule
    \begin{equation*}
      \begin{prooftree}
        \hypo{ \tau: \op*{Type}_n }
        \hypo{ \sigma: \op*{Type}_n }
        \infer2{ \tau \synimplies \sigma: \op*{Type}_n }
      \end{prooftree}
    \end{equation*}

    Note that, since the rules are part of the metalanguage, we do not try to formalize them and hence do not use the dot convention from \fullref{rem:object_language_dots}.

    \thmitem{rem:type_theory_rule_classification/introduction} An \term{introduction rule} specifies which terms have a type from the family. In the words of Martin-L\"of, \enquote{The introduction rules say what are the canonical elements}.

    We have already used such a rule for arrow types --- \ref{inf:def:arrow_typing_rules/intro/explicit} for typed and \ref{inf:def:arrow_typing_rules/intro/implicit} for untyped terms. Unlike type formation rules, we have completely formalized these; the first of them is
    \begin{equation*}
      \begin{prooftree}
        \hypo{ [x: \tau] }
        \ellipsis {} { M: \sigma }
        \infer1{ \qabs {x^{\tau}} M: \tau \synimplies \sigma }
      \end{prooftree}
    \end{equation*}

    \thmitem{rem:type_theory_rule_classification/elimination} Conversely, an \term{elimination rule} allows us to deconstruct a term. In the words of Martin-L\"of, \enquote{The elimination rule shows how we may define functions on the set defined by the introduction rules}.

    Again, we have already used such a rule, \ref{inf:def:arrow_typing_rules/elim}, which we defined as
    \begin{equation*}
      \begin{prooftree}
        \hypo{ M: \tau \synimplies \sigma }
        \hypo{ N: \tau }
        \infer2{ M N: \sigma }
      \end{prooftree}
    \end{equation*}

    \thmitem{rem:type_theory_rule_classification/equality} Finally, an \term{equality rule} demonstrates compatibility of \hyperref[con:equality]{definitional equality} with the other rules.

    To enable substituting definitionally equal terms and types and making type derivation \hyperref[con:extensionality]{extensional}, we must introduce two rules. If, in accordance with \fullref{con:equality}, we denote the equality judgment
    \begin{center}
      \( x \) and \( y \) are definitionally equal terms of type \( \tau \)
    \end{center}
    by
    \begin{equation*}
      x \coloneqq y: \tau,
    \end{equation*}
    we can formulate the rules as follows:
    \begin{paracol}{2}
      \begin{leftcolumn}
        \ParacolAlignmentHack
        \begin{equation*}
          \begin{prooftree}
            \hypo{ M: \tau }
            \hypo{ \tau \coloneqq \tau': \op*{Type}_n }
            \infer2{ M: \tau' }
          \end{prooftree}
        \end{equation*}
      \end{leftcolumn}

      \begin{rightcolumn}
        \ParacolAlignmentHack
        \begin{equation*}
          \begin{prooftree}
            \hypo{ M \coloneqq M': \tau }
            \hypo{ \tau \coloneqq \tau': \op*{Type}_n }
            \infer2{ M \coloneqq M': \tau' }
          \end{prooftree}
        \end{equation*}
      \end{rightcolumn}
    \end{paracol}

    Furthermore, to ensure that \( {\coloneqq} \) is an \hyperref[def:equivalence_relation]{equivalence relation}, we need inferences rules resembling those from \fullref{ex:recursively_defined_relation}:
    \columnratio{0.25,0.25,0.5}
    \begin{paracol}{3}
      \begin{nthcolumn}{0}
        \ParacolAlignmentHack
        \begin{equation*}
          \begin{prooftree}
            \hypo{ M: \tau }
            \infer1{ M \coloneqq M: \tau }
          \end{prooftree}
        \end{equation*}
      \end{nthcolumn}

      \begin{nthcolumn}{1}
        \ParacolAlignmentHack
        \begin{equation*}
          \begin{prooftree}
            \hypo{ M \coloneqq N: \tau }
            \infer1{ N \coloneqq M: \tau }
          \end{prooftree}
        \end{equation*}
      \end{nthcolumn}

      \begin{nthcolumn}{2}
        \ParacolAlignmentHack
        \begin{equation*}
          \begin{prooftree}
            \hypo{ M \coloneqq N: \tau }
            \hypo{ N \coloneqq K: \tau }
            \infer2{ M \coloneqq K: \tau }
          \end{prooftree}
        \end{equation*}
      \end{nthcolumn}
    \end{paracol}
    \columnratio{}

    These five \enquote{common} equality rules are given in \cite[433]{UnivalentProject2024OctoberHoTT} and, in a less unrefined form, they are also discussed by \incite[14,15]{MartinLöf1984IntuitionisticTypeTheory}.

    Unlike in Martin-L\"of's presentation, in \cite[27]{UnivalentProject2024OctoberHoTT}, the other equality rules are split into two kinds. \incite[\S 8.1.9]{Mimram2020ProgramEqualsProof} also distinguishes congruence rules, which the Univalent Foundations Project assumes implicit:
    \begin{thmenum}
      \thmitem{rem:type_theory_rule_classification/equality/computation} A \term{computation rule} expresses how an elimination rule acts on the corresponding canonical objects obtained from introduction rules.

      Such rules generalize \hyperref[def:beta_eta_reduction]{\( \beta \)-reduction} of untyped \( \synlambda \)-terms. For example, for arrow types we have
      \begin{equation*}
        \begin{prooftree}
          \hypo{ [x: \tau] }
          \ellipsis {} { M: \sigma }

          \hypo{ N: \tau }
          \infer2{ (\qabs {x^\tau} M) N \coloneqq M[x \mapsto N]: \sigma }.
        \end{prooftree}
      \end{equation*}

      \thmitem{rem:type_theory_rule_classification/equality/uniqueness} A \term{uniqueness rule} expresses how an arbitrary term of a given type can be expressed in the canonical form given by introduction rules.

      Such rules generalize \hyperref[def:beta_eta_reduction]{\( \eta \)-expansion} of untyped \( \synlambda \)-terms. For example, for arrow types we have
      \begin{equation*}
        \begin{prooftree}
          \hypo{ M: \tau \synimplies \sigma }
          \infer1{ \qabs {x^\tau} M x: \tau \synimplies \sigma }.
        \end{prooftree}
      \end{equation*}

      \thmitem{rem:type_theory_rule_classification/equality/congruence} Finally, a \term{congruence rule} simply propagates definitional equality.

      For example, we can formulate the following congruence rules for arrow types:
      \begin{equation*}
        \begin{prooftree}
          \hypo{ \tau \coloneqq \tau': \op*{Type}_n }
          \hypo{ \sigma \coloneqq \sigma': \op*{Type}_n }
          \infer2{ \tau \synimplies \sigma \coloneqq \tau' \synimplies \sigma': \op*{Type}_n }
        \end{prooftree}
      \end{equation*}

      \begin{paracol}{2}
        \begin{leftcolumn}
          \ParacolAlignmentHack
          \begin{equation*}
            \begin{prooftree}
              \hypo{ M \coloneqq M': \sigma }
              \infer1{ \qabs {x^{\tau}} M \coloneqq \qabs {x^{\tau}} M': \tau \synimplies \sigma }
            \end{prooftree}
          \end{equation*}
        \end{leftcolumn}

        \begin{rightcolumn}
          \ParacolAlignmentHack
          \begin{equation*}
            \begin{prooftree}
              \hypo{ M \coloneqq M': \tau \synimplies \sigma }
              \hypo{ N \coloneqq N': \tau }
              \infer2{ M N \coloneqq M' N': \sigma }
            \end{prooftree}
          \end{equation*}
        \end{rightcolumn}
      \end{paracol}
    \end{thmenum}
  \end{thmenum}
\end{remark}
\begin{comments}
  \item Our list of rules is based on the corresponding rules for dependent products in \incite{MartinLöf1984IntuitionisticTypeTheory}, \cite[\S A.2]{UnivalentProject2024OctoberHoTT} and \cite[\S 8.1.10]{Mimram2020ProgramEqualsProof}.

  \item In the cited lectures, Martin-L\"of refers to types as \enquote{sets} and to the corresponding rules as \enquote{set formation} rules, but in later works like \cite{MartinLöf1994TypeJudgments} he already shifts to \enquote{types} and \enquote{type formation}.
\end{comments}

\begin{definition}\label{def:empty_type}\mcite[\S 4.3.4]{Mimram2020ProgramEqualsProof}
  The \term{empty type} \( \syn\Bbbzero \).
\end{definition}

\begin{definition}\label{def:unit_type}\mcite[\S 4.3.2]{Mimram2020ProgramEqualsProof}
  The \term{unit type} \( \syn\Bbbone \).
\end{definition}

\begin{definition}\label{def:product_type}
\end{definition}

\begin{definition}\label{def:sum_type}
\end{definition}

\begin{concept}\label{con:identity_types}
  \todo{Identity types}

  Martin-L\"of distinguishes between several kinds of equality and introduces \enquote{identity types} (see \fullref{con:identity_types}), allowing him to develop his \enquote{intuitionistic type theory} --- a fusion of type theory and higher-order logic based on the \hyperref[con:curry_howard_correspondence]{Curry-Howard correspondence}. His theory first appears as \cite{MartinLöf1975IntuitionisticTypeTheory} and, in a more refined later form, can be found in \cite{MartinLöf1984IntuitionisticTypeTheory}.
\end{concept}

\begin{concept}\label{con:homotopy_type_theory}
  \todo{Homotopy type theory}
\end{concept}

\begin{concept}\label{con:dependent_types}
  \todo{Discuss dependent types}

  \Fullref{thm:set_of_all_functions_via_cartesian_product}.
\end{concept}

\begin{remark}\label{rem:dependent_type_theory}
  Because of this latter development, \enquote{intuitionistic type theory}, \enquote{dependent type theory} and \enquote{Martin-L\"of type theory} refer to this more abstract setting. \enquote{Simple type theory} continues to refer to the theory with only arrow types. Accidentally, the adjective \enquote{simple} is a contradistinction with Russell's type theory rather than Martin-L\"of's.
\end{remark}

\paragraph{Curry-Howard correspondence}\hfill

\begin{example}\label{ex:con:curry_howard_correspondence}
  We list examples related to the \hyperref[con:curry_howard_correspondence]{Curry-Howard correspondence}:
  \begin{thmenum}
    \thmitem{ex:con:curry_howard_correspondence/minimal_implicational}
  \end{thmenum}
\end{example}

\begin{algorithm}\label{alg:type_derivation_to_proof_tree}
\end{algorithm}

\begin{algorithm}\label{alg:proof_tree_to_type_derivation}
\end{algorithm}
