\subsection{Enumerative combinatorics}\label{subsec:enumerative_combinatorics}

This subsection lists several results of various importance that don't really belong to any other section.

\paragraph{Pigeonhole principle}

\begin{theorem}[Dirichlet's pigeonhole principle]\label{thm:pigeonhole_principle}
  If \( \card(A) > \card(B) \), then there exists no injective function from \( A \) to \( B \).
\end{theorem}
\begin{comments}
  \item The name of the theorem comes from the observation that if we are given more pigeons than pigeonholes, then at least one pigeonhole must contain multiple pigeons in it.
\end{comments}
\begin{proof}
  This is a corollary of \fullref{thm:set_domination_relation_trichotomy}.
\end{proof}

\paragraph{Factorial}

\begin{definition}\label{def:factorial}\mcite[46]{Knuth1997ArtVol1}
  We define the \term[bg=факториел (\cite[129]{Тагамлицки1971Диф}), ru=факториал (\cite[145]{АлександровМаркушевичХинчин1952ЭнциклопедияТом3})]{factorial} of a \hyperref[def:integer_signum]{nonnegative integer} \( n \) recursively as follows:
  \begin{equation*}
    n! \coloneqq \begin{cases}
      1,          &n = 0, \\
      (n - 1)! n, &n > 0.
    \end{cases}
  \end{equation*}
\end{definition}

\begin{proposition}\label{thm:gamma_function_interpolates_factorial}
  For every \hyperref[def:integer_signum]{nonnegative integer} \( n \) we have
  \begin{equation*}
    \Gamma(n + 1) \coloneqq n!,
  \end{equation*}
  where \( \Gamma \) is the Gamma function defined in \fullref{def:gamma_function}.
\end{proposition}
\begin{proof}
  We use induction on \( n \).
  \begin{itemize}
    \item If \( n = 0 \), then
    \begin{equation*}
      \Gamma(1)
      =
      \int_0^\infty x^0 e^{-x} \dl x
      =
      -e^{-x}\restr_{x=0}^\infty
      =
      -\underbrace{\lim_{x \to \infty} e^{-x}}_{0} + 1
      =
      1
      =
      0!
    \end{equation*}

    \item If \( n > 0 \) and \( \Gamma(n) = (n - 1)! \), then
    \begin{balign*}
      \Gamma(n + 1)
      &=
      \int_0^\infty x^n \cdot e^{-x} \dl x
      = \\ &=
      \underbrace{(- x^n e^{-x})\restr_{x=0}^\infty}_{-(0 - 0)} + n \int_0^\infty e^{-x} x^{n-1} \dl x
      = \\ &=
      n \Gamma(n)
      = \\ &=
      n (n - 1)!
      = \\ &=
      n!
    \end{balign*}
  \end{itemize}
\end{proof}

\begin{theorem}[Stirling's factorial approximation]\label{thm:stirlings_factorial_approximation}
  For the factorial function we have
  \begin{equation*}
    n! = \sqrt{2 \pi n} \cdot \parens*{ \frac n e }^n \cdot e^{\mu(n)},
  \end{equation*}
  where \( \mu \), defined by \eqref{eq:thm:stirlings_gamma_approximation/mu}, satisfies
  \begin{equation*}
    0 < \mu(n) < \frac 1 {12n}.
  \end{equation*}
\end{theorem}
\begin{proof}
  Follows from \fullref{thm:gamma_function_interpolates_factorial} and \fullref{thm:stirlings_gamma_approximation}.
\end{proof}

\paragraph{Binomial coefficients}

\begin{definition}\label{def:binomial_coefficient}\mcite[53]{Knuth1997ArtVol1}
  We define the \term[ru=биномиальный коеффициент (\cite[100]{БелоусовТкачёв2004ДискретнаяМатематика})]{binomial coefficient} of the \hyperref[def:integer_signum]{nonnegative integers} \( n \geq k \) as the number
  \begin{equation*}
    \binom n k \coloneqq \frac {n!} {k!(n-k)!},
  \end{equation*}
\end{definition}
\begin{comments}
  \item Binomial coefficients are motivated by \fullref{thm:binomial_theorem}.
\end{comments}

\begin{theorem}[Pascal's binomial recurrence]\label{thm:pascals_binomial_recurrence}
  For nonnegative integers \( 1 < k < n \), \hyperref[def:binomial_coefficient]{binomial coefficients} have the following property:
  \begin{equation}\label{eq:thm:pascals_binomial_recurrence}
    \binom n k = \binom {n - 1} k + \binom {n - 1} {k - 1}.
  \end{equation}
\end{theorem}
\begin{comments}
  \item \incite[thm. 6.4.2]{Rosen2019DiscreteMathematics} calls this recurrence \enquote{Pascal's identity} after Pascal because of its relation to \hyperref[con:pascals_triangle]{Pascal's triangle}. \incite[56]{Knuth1997ArtVol1} calls it the \enquote{addition formula} for binomial coefficients.
\end{comments}
\begin{proof}
  \begin{balign*}
    \binom {n - 1} k + \binom {n - 1} {k - 1}
    &=
    \frac {(n - 1)!} {k! (n - 1 - k)!} + \frac {(n - 1)!} {(k - 1)! (n - k)!}
    = \\ &=
    \frac {(n - 1)!} {(k - 1)! (n - 1 - k)!} \bracks*{ \frac 1 k + \frac 1 {n - k} }
    = \\ &=
    \frac {(n - 1)!} {(k - 1)! (n - 1 - k)!} \frac n {k(n - k)}
    = \\ &=
    \frac {n!} {k! (n - k)!}
    = \\ &=
    \binom n k.
  \end{balign*}
\end{proof}

\begin{theorem}[Newton's binomial theorem]\label{thm:binomial_theorem}
  If, in some \hyperref[def:semiring]{semiring}, the members \( x \) and \( y \) commute (i.e. \( xy = yx \)), then
  \begin{equation}\label{eq:thm:binomial_theorem}
    (x + y)^n = \sum_{k=0}^n \binom n k x^k y^{n-k}.
  \end{equation}
\end{theorem}
\begin{proof}
  We use induction on \( n \). For \( n = 0 \), the theorem trivially holds. Assume that the theorem holds for \( 1, \ldots, n \). Then
  \begin{balign*}
    (x + y)^{n+1}
     & =
    x (x + y)^n + y (x + y)^n
    = \\ &=
    \sum_{k=0}^n \binom n k x^{k+1} y^{n-k} + y \sum_{k=0}^n \binom n k x^k y^{n-k}
    = \\ &=
    x^{n+1} + y \sum_{k=0}^{n-1} \binom n k x^{k+1} y^{n-(k+1)} + y \sum_{k=0}^n \binom n k x^k y^{n-k}
    = \\ &=
    x^{n+1} + y \parens*{ \sum_{k=1}^n \binom n {k-1} x^k y^{n-k} + y^n \sum_{k=1}^n \binom n k x^k y^{n-k} } + y^{n+1}
    = \\ &\reloset {\eqref{eq:thm:pascals_binomial_recurrence}} =
    x^{n+1} + y \sum_{k=1}^n \binom {n+1} k x^k y^{n-k} + y^{n+1}
    = \\ &=
    \sum_{k=0}^n \binom {n+1} k x^k y^{(n+1)-k}.
  \end{balign*}
\end{proof}

\begin{theorem}[Vandermonde convolution]\label{thm:vandermonde_convolution}\mcite[59]{Knuth1997ArtVol1}
  For nonnegative integers \( n \), \( m \) and \( k \leq n + m \), we have
  \begin{equation}\label{eq:thm:vandermonde_convolution}
    \binom {n + m} k = \sum_{\substack{i+j=k \\ i \leq n \T*{and} j \leq m}} \binom n i \cdot \binom m j.
  \end{equation}
\end{theorem}
\begin{proof}
  Consider the polynomial \( X + 1 \) over any commutative ring.

  By \fullref{thm:binomial_theorem}, we have
  \begin{equation*}
    (X + 1)^{n+m} = \sum_{k=0}^{n+m} x^k 1^{n+m-k}.
  \end{equation*}

  Using the \hyperref[def:semigroup_algebra]{convolution product} instead, we obtain
  \begin{equation*}
    (X + 1)^n \cdot (X + 1)^m = \sum_{k=0}^{n+m} \parens*{ \sum_{\substack{i+j=k \\ i \leq n \T*{and} j \leq m}} \binom n i \cdot \binom m j } x^k.
  \end{equation*}

  Since the two are equal, we obtain \eqref{eq:thm:vandermonde_convolution}.
\end{proof}

\begin{concept}\label{con:pascals_triangle}\mcite[fig. 6.4.1]{Rosen2019DiscreteMathematics}
  \term[ru=треугольник Паскаля (\cite[\S 5.3.4]{Новиков2013ДискретнаяМатематика})]{Pascal's triangle} is a colloquial name for schematic triangles a-la \cref{fig:con:pascals_triangle}, in which every number is the sum of the two numbers above, with ones on the boundary.

  \begin{figure}[!ht]
    \centering
    \includegraphics[page=1]{output/con__pascals_triangle}
    \caption{\hyperref[con:pascals_triangle]{Pascal's triangle}.}\label{fig:con:pascals_triangle}
  \end{figure}

  By \enquote{schematic triangle} we mean that every object showing the same recurrence can again be called \enquote{Pascal's triangle}. For example, \incite[54]{Knuth1997ArtVol1} and \incite[179]{Яблонский2003ДискретнаяМатематика} use the term to refer to \hyperref[def:pascal_matrix/lower]{Pascal lower triangular matrices}.
\end{concept}

\begin{definition}\label{def:pascal_matrix}\mcite[1]{EdelmanStrang2018PascalMatrices}
  We will define three families of square matrices, which we will collectively call \term{Pascal matrices}.
  \begin{thmenum}
    \thmitem{def:pascal_matrix/lower} The \hyperref[def:triangular_matrix]{lower triangular} matrix \( L_n = \seq{ l_{i,j} }_{i,j=1}^n \), where
    \begin{equation*}
      l_{i,j} \coloneqq \begin{cases}
        0,                       &j > i, \\
        1,                       &j = 1, \\
        1,                       &j = i, \\
        l_{i-1,j} + l_{i-1,j-1}, &1 < j < i.
      \end{cases}
    \end{equation*}

    \thmitem{def:pascal_matrix/upper} Its \hyperref[def:transpose_matrix]{transpose}, the \hyperref[def:triangular_matrix]{upper triangular} matrix \( U_n = \seq{ u_{i,j} }_{i,j=1}^n \).

    \thmitem{def:pascal_matrix/symmetric} The symmetric matrix \( S_n = \seq{ s_{i,j} }_{i,j=1}^n \), where
    \begin{equation*}
      s_{i,j} \coloneqq \begin{cases}
        1,                     &i = 1, \\
        1,                     &j = 1, \\
        s_{i-1,j} + s_{i,j-1}, &i > 1 \T{and} j > 1.
      \end{cases}
    \end{equation*}
  \end{thmenum}
\end{definition}
\begin{comments}
  \item These matrices in \identifier{combinatorics.binomial} in \cite{notebook:code}.
\end{comments}

\begin{example}\label{ex:con:pascals_triangle}
  We give examples of \hyperref[def:pascal_matrix]{Pascal matrices}:
  \begin{equation*}
    L_7
    =
    \begin{pmatrix}
      1        &          &           &        &        &      &   \\
      \fbox{1} & \fbox{1} &           &        &        &      &   \\
      1        & \fbox{2} & 1         &        &        &      &   \\
      1        & 3        & 3         & 1      &        &      &   \\
      1        & 4        & 6         & 4      & 1      &      &   \\
      1        & \fbox{5} & \fbox{10} & 10     & 5      & 1    &   \\
      1        & 6        & \fbox{15} & 20     & 15     & 6    & 1
    \end{pmatrix}
  \end{equation*}
  \begin{equation*}
    U_7
    =
    \begin{pmatrix}
      1        & \fbox{1} & 1         & 1      & 1      & 1         & 1         \\
               & \fbox{1} & \fbox{2}  & 3      & 4      & \fbox{5}  & 6         \\
               &          & 1         & 3      & 6      & \fbox{10} & \fbox{15} \\
               &          &           & 1      & 4      & 10        & 20        \\
               &          &           &        & 1      & 5         & 15        \\
               &          &           &        &        & 1         & 6         \\
               &          &           &        &        &           & 1
    \end{pmatrix}
  \end{equation*}
  \begin{equation*}
    S_7
    =
    \begin{pmatrix}
      1        & \fbox{1} & 1         & 1      & 1      & 1      & 1   \\
      \fbox{1} & \fbox{2} & 3         & 4      & 5      & 6      & 7   \\
      1        & 3        & 6         & 10     & 15     & 21     & 28  \\
      1        & 4        & \fbox{10} & 20     & 35     & 56     & 84  \\
      1        & \fbox{5} & \fbox{15} & 35     & 70     & 126    & 210 \\
      1        & 6        & 21        & 56     & 126    & 252    & 462 \\
      1        & 7        & 28        & 84     & 210    & 462    & 924
    \end{pmatrix}
  \end{equation*}
\end{example}

\begin{proposition}\label{thm:pascal_matrix_binomial}
  We have the following relation between \hyperref[def:binomial_coefficient]{Binomial coefficients} and elements of \hyperref[def:pascal_matrix]{Pascal matrices}:
  \begin{align*}
    l_{i,j} &= \binom i j \thickspace\T{if}\thickspace i \leq j, \\
    u_{i,j} &= \binom j i \thickspace\T{if}\thickspace i \geq j, \\
    s_{i,j} &= \binom {i + j} j = \binom {i + j} i = \frac {(i + j)!} {i! j!}. \\
  \end{align*}
\end{proposition}
\begin{proof}
  Follows from \fullref{thm:pascals_binomial_recurrence}.
\end{proof}

\begin{proposition}\label{thm:pascal_matrix_product}
  For \hyperref[def:pascal_matrix]{Pascal matrices}, we have \( S_n = L_n \cdot U_n \).
\end{proposition}
\begin{proof}
  The \( (i, j) \)-th entry of \( L_n \cdot U_n \) is
  \begin{equation*}
    \sum_{k=1}^n l_{i,k} \cdot u_{k,j}
    =
    \sum_{k=1}^n l_{i,k} \cdot l_{j,k}
    =
    \sum_{k=1}^{\min\set{ i, j }} \binom i k \cdot \binom j k
    \reloset {\eqref{eq:thm:vandermonde_convolution}} =
    \binom {i + j} i
    =
    s_{i,j}.
  \end{equation*}
\end{proof}

\paragraph{Progressions}\hfill

Progressions are an elementary concept that happens to be quite useful. There is no definition of progression, but rather the term \enquote{progression} refers to specific recursively defined \hyperref[def:sequence]{sequences}.

\begin{definition}\label{def:arithmetic_progression}
  In an \hyperref[def:abelian_group]{abelian group}, by default taken to be the field of complex numbers, an \( n \)-term (resp. infinite) \term[ru=арифметическая прогрессия (\cite[143]{АлександровМаркушевичХинчин1952ЭнциклопедияТом3}), en=arithmetic progression (\cite[def. 2.4.3]{Rosen2019DiscreteMathematics})]{arithmetic progression} with \term[en=common difference (\cite[def. 2.4.3]{Rosen2019DiscreteMathematics})]{difference} \( d \) is a sequence \( \seq{ a_k }_{k=0}^{n-1} \) (resp. \( \seq{ a_k }_{k=0}^\infty \)) satisfying any of the following equivalent conditions:
  \begin{thmenum}
    \thmitem{def:arithmetic_progression/implicit}\mcite[\S 227]{Киселёв2009Геометрия} The difference of any two consecutive elements equals \( d \).
    \thmitem{def:arithmetic_progression/direct}\mcite[1]{ButlerEtAl2016Progressions} We have \( a_k = a_0 + kd \) for every index \( k \), where \( kd \) is \hyperref[con:additive_semigroup/multiplication]{iterated addition}.
    \thmitem{def:arithmetic_progression/recursive}\mcite[143]{АлександровМаркушевичХинчин1952ЭнциклопедияТом3} We have
    \begin{equation}\label{eq:def:arithmetic_progression/recursive}
      a_k \coloneqq \begin{cases}
        a_0,         &k = 0, \\
        a_{k-1} + d, &k > 0.
      \end{cases}
    \end{equation}
  \end{thmenum}
\end{definition}

\begin{proposition}\label{thm:arithmetic_progression_partial_sums}
  The \hyperref[def:convergent_series]{series} constructed from the arithmetic progression \( \seq{ a_0 + kd }_{k=0}^\infty \) has partial sums
  \begin{equation}\label{eq:thm:arithmetic_progression_partial_sums}
    2 \sum_{k=0}^n a_k = (n + 1) (a_n - a_0).
  \end{equation}
\end{proposition}
\begin{proof}
  \begin{balign*}
    2 \sum_{k=0}^n a_k
     & =
    2 \sum_{k=0}^n (a_0 + kd)
    =    \\ &=
    \sum_{k=0}^n (a_0 + kd) + \sum_{k=0}^n (a_0 + (n-k)d)
    =    \\ &=
    \sum_{k=0}^n (2 a_0 + nd)
    =    \\ &=
    (n + 1) (a_0 + a_n).
  \end{balign*}
\end{proof}

\begin{corollary}\label{thm:numeric_arithmetic_progression_partial_sums}
  For any positive integer \( n \), we have
  \begin{equation}\label{eq:thm:numeric_arithmetic_progression_partial_sums}
    \sum_{k=0}^n k = \sum_{k=1}^n k = \frac {n (n + 1)} 2 = \binom {n+1} 2.
  \end{equation}
\end{corollary}
\begin{proof}
  Follows from either \fullref{thm:arithmetic_progression_partial_sums} or \fullref{thm:vandermonde_convolution}.
\end{proof}

\begin{definition}\label{def:geometric_progression}
  In a \hyperref[def:field]{field}, by default taken to be the field of complex numbers, an \( n \)-term (resp. infinite) \term[ru=геометрическая прогрессия (\cite[144]{АлександровМаркушевичХинчин1952ЭнциклопедияТом3}), en=geometric progression (\cite[def. 2.4.2]{Rosen2019DiscreteMathematics})]{geometric progression} with \term[ru=знаменатель (прогрессии) (\cite[\S 227]{Киселёв2009Геометрия}), en=common ratio (\cite[def. 2.4.2]{Rosen2019DiscreteMathematics})]{quotient} \( q \) is a sequence \( \seq{ a_k }_{k=0}^{n-1} \) (resp. \( \seq{ a_k }_{k=0}^\infty \)) of \hi{nonzero elements} satisfying any of the following equivalent conditions:
  \begin{thmenum}
    \thmitem{def:geometric_progression/implicit}\mcite[\S 227]{Киселёв2009Геометрия} The quotient of two consecutive elements equals \( q \).
    \thmitem{def:geometric_progression/direct}\mcite[144]{АлександровМаркушевичХинчин1952ЭнциклопедияТом3} We have \( a_k = a_0 q^k \) for every index \( k \).
    \thmitem{def:geometric_progression/recursive}\mcite[144]{АлександровМаркушевичХинчин1952ЭнциклопедияТом3} We have
    \begin{equation}\label{eq:def:geometric_progression/recursive}
      a_k \coloneqq \begin{cases}
        a_0,       &k = 0, \\
        a_{k-1} q, &k > 0.
      \end{cases}
    \end{equation}
  \end{thmenum}
\end{definition}

\begin{proposition}\label{thm:arithmetic_to_geometric_progression}
  Given a complex arithmetic progression \( \seq{ a_k }_{k=0}^\infty \) with difference \( d \), for any complex number \( z \), the sequence \( \seq{ z^{a_k} }_{k=0}^\infty \) is a geometric progression with quotient \( z^d \).
\end{proposition}
\begin{proof}
  Trivial.
\end{proof}

\paragraph{Geometric series}

\begin{definition}\label{def:geometric_series}\mcite[61]{Rudin1976AnalysisPrinciples}
  A \term{geometric series} is \hyperref[def:convergent_series]{series} whose coefficients come from a complex-valued \hyperref[def:geometric_progression]{geometric progression} with base \( 1 \):
  \begin{equation}\label{eq:def:geometric_series}
    \sum_{k=0}^\infty z^k.
  \end{equation}
\end{definition}
\begin{comments}
  \item Rudin defines the series only for real numbers, for the generalization is straightforward.
\end{comments}

\begin{proposition}\label{thm:xn_minus_yn_factorization}
  For arbitrary \hyperref[def:ring]{ring} elements \( x \) and \( y \) and for every nonnegative integer \( n \), we have
  \begin{equation}\label{eq:thm:xn_minus_yn_factorization}
    x^{n + 1} - y^{n + 1} = (x - y)(x^n + x^{n-1} y + \cdots + y^n) = (x - y) \sum_{k=0}^n x^k y^{n-k}.
  \end{equation}
\end{proposition}
\begin{proof}
  \begin{align*}
    (x - y) \sum_{k=0}^n x^k y^{n-k}
    &=
    \sum_{k=0}^n x^{k+1} y^{n-k} - \sum_{k=0}^n x^k y^{n-k+1}
    = \\ &=
    \sum_{k=1}^{n+1} x^k y^{(n+1)-k} - \sum_{k=0}^n x^k y^{(n+1)-k}
    =
    x^{n+1} - y^{n+1}.
  \end{align*}
\end{proof}

\begin{proposition}\label{thm:def:geometric_series}
  The geometric series \eqref{eq:def:geometric_series} has the following basic properties:
  \begin{thmenum}
    \thmitem{thm:def:geometric_series/finite_sum} For all \( z \in \BbbC \setminus \set{ 1 } \), the geometric series \eqref{eq:def:geometric_series} has partial sums
    \begin{equation}\label{eq:thm:def:geometric_series/finite_sum}
      \sum_{k=0}^n z^k = \frac {1 - z^{n+1}} {1 - z}.
    \end{equation}

    \thmitem{thm:def:geometric_series/degenerate} In the degenerate case \( z = 1 \), the progression itself is constant, and its partial sums are instead
    \begin{equation}\label{eq:thm:def:geometric_series/degenerate}
      \sum_{k=0}^n z^k = n + 1.
    \end{equation}

    \thmitem{thm:def:geometric_series/series_sum_exterior} The series diverges when \( \abs{z} \geq 1 \).

    \thmitem{thm:def:geometric_series/series_sum_interior} For \( 0 < \abs{z} < 1 \), the series converges absolutely with limit
    \begin{equation}\label{eq:thm:def:geometric_series/series_sum_interior}
      \sum_{k=0}^\infty z^k = \frac 1 {1 - z}.
    \end{equation}
  \end{thmenum}
\end{proposition}
\begin{proof}
  \SubProofOf{thm:def:geometric_series/finite_sum} Follows from \fullref{thm:xn_minus_yn_factorization}.

  \SubProofOf{thm:def:geometric_series/degenerate} Obvious.

  \SubProofOf{thm:def:geometric_series/series_sum_exterior}

  \SubProof*{Proof for \( z = 1 \)} If \( z = 1 \), \fullref{thm:def:geometric_series/degenerate} implies that the series diverges because it grows indefinitely.

  \SubProof*{Proof for \( \abs{z} = 1 \) and \( z \neq 1 \)} In this case the integer powers \( z^k \) are rotations around the complex plane unit circle, which do not tend to a limit. Hence, the series diverges again.

  \SubProof*{Proof for \( \abs{z} > 1 \)} In this case \( \abs{z^n} \) grows indefinitely with \( n \), and it follows that
  \begin{equation*}
    \sum_{k=m}^n z^k
    =
    z^m \sum_{k=0}^{n-m} z^k
    =
    z^m \frac {1 - z^{n-m+1}} {1 - z}
    =
    \frac {z^m - z^{n+1}} {1 - z}.
  \end{equation*}
  can get arbitrarily far from \( 0 \). Therefore, in this case the series also diverges.

  \SubProofOf{thm:def:geometric_series/series_sum_interior} Fix \( z \in B(0, 1) \). Since only \( z^{n + 1} \) depends on \( n \) in \eqref{eq:thm:def:geometric_series/finite_sum}, we obtain \eqref{eq:thm:def:geometric_series/series_sum_interior} by simply noting that \( z^n \to 0 \) when \( n \to \infty \).
\end{proof}

\begin{example}\label{ex:def:geometric_series}
  We list some examples of \hyperref[def:geometric_series]{geometric series}:
  \begin{thmenum}
    \thmitem{ex:def:geometric_series/two} A surprisingly useful series occurs for \( z = 1 / 2 \), where
    \begin{equation}\label{eq:ex:def:geometric_series/two}
      \sum_{k=0}^\infty \frac 1 {2^k}
      \reloset {\eqref{eq:thm:def:geometric_series/series_sum_interior}} =
      \frac 1 {1 - 1 / 2}
      =
      2.
    \end{equation}

    Its usefulness comes from the fact that
    \begin{equation}\label{eq:ex:def:geometric_series/two/one}
      \sum_{k=1}^\infty \frac 1 {2^k}
      =
      \sum_{k=0}^\infty \frac 1 {2^k} - 1
      =
      1,
    \end{equation}
    hence we can use the terms for generalized \hyperref[def:convex_hull]{convex combinations}.

    \thmitem{ex:def:geometric_series/interest} In this example we will exploit the equivalence between the closed form representations in \fullref{def:arithmetic_progression} and \fullref{def:geometric_progression} and the corresponding inductive definitions. The equivalences are obvious from a mathematical standpoint, however outside of mathematics they have highly nontrivial consequences. Indeed, they highlight the difference between simple interest and compound interest.

    Consider a savings account with \( 1000\$ \). A simple monthly interest of \( 2\% \) will earn \( 240\$ \) over a year:
    \begin{equation*}
      1000 (1 + 12 \cdot 2 / 100) = 1240.
    \end{equation*}

    The same account with a compound monthly interest of \( 2\% \) will earn a bit more - about \( 268\$ \):
    \begin{equation*}
      1000 (1 + 2 / 100)^{12} \approx 1268.24.
    \end{equation*}

    Over the course of ten years, however, simple interest will earn a total of \( 2400\$ \), while compound interest will earn \( \approx 9765\$ \).

    The difference between linear and exponential growth appears staggering in a real-world situation even though the difference may not be very noticeable in the short-term.
  \end{thmenum}
\end{example}
