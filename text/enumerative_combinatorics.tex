\subsection{Enumerative combinatorics}\label{subsec:enumerative_combinatorics}

This subsection lists several results of various importance that don't really belong to any more consistent theory.

\begin{theorem}[Dirichlet's pigeonhole principle]\label{def:pigeonhole_principle}
  If we are given more pigeons than pigeonholes, then at least one pigeonhole must contain multiple pigeons in it.

  More formally, if \( \card(A) > \card(B) \), then there exists no injective function from \( A \) to \( B \).
\end{theorem}
\begin{proof}
  This is a corollary of \fullref{thm:set_domination_relation_trichotomy}.
\end{proof}

\begin{definition}\label{def:binomial_coefficient}
  We define the \term{binomial coefficient} of the \hyperref[def:integer_signum]{nonnegative integers} \( n \) and \( k \) as the number
  \begin{equation*}
    \binom n k \coloneqq \frac {n!} {k!(n-k)!}
  \end{equation*}
\end{definition}
\begin{comments}
  \item Binomial coefficients are motivated by \fullref{thm:binomial_theorem}.
\end{comments}

\begin{theorem}[Pascal's identity]\label{thm:pascals_identity}
  \hyperref[def:binomial_coefficient]{Binomial coefficients} have the following property:
  \begin{equation}\label{eq:thm:pascals_identity}
    \binom n k = \binom {n - 1} k + \binom {n - 1} {k - 1}.
  \end{equation}
\end{theorem}
\begin{proof}
  \begin{balign*}
    \binom {n - 1} k + \binom {n - 1} {k - 1}
    &=
    \frac {(n - 1)!} {k! (n - 1 - k)!} + \frac {(n - 1)!} {(k - 1)! (n - k)!}
    = \\ &=
    \frac {(n - 1)!} {(k - 1)! (n - 1 - k)!} \bracks*{ \frac 1 k + \frac 1 {n - k} }
    = \\ &=
    \frac {(n - 1)!} {(k - 1)! (n - 1 - k)!} \frac n {k(n - k)}
    = \\ &=
    \frac {n!} {k! (n - k)!}
    = \\ &=
    \binom n k.
  \end{balign*}
\end{proof}

\begin{theorem}[Newton's binomial theorem]\label{thm:binomial_theorem}
  If, in some \hyperref[def:semiring]{semiring}, the members \( x \) and \( y \) commute (i.e. \( xy = yx \)), then
  \begin{equation}\label{eq:thm:binomial_theorem}
    (x + y)^n = \sum_{k=0}^n \binom n k x^k y^{n-k}.
  \end{equation}
\end{theorem}
\begin{proof}
  We use induction on \( n \). For \( n = 0 \), the theorem trivially holds. Assume that the theorem holds for \( 1, \ldots, n \). Then
  \begin{balign*}
    (x + y)^{n+1}
     & =
    x (x + y)^n + y (x + y)^n
    = \\ &=
    \sum_{k=0}^n \binom n k x^{k+1} y^{n-k} + y \sum_{k=0}^n \binom n k x^k y^{n-k}
    = \\ &=
    x^{n+1} + y \sum_{k=0}^{n-1} \binom n k x^{k+1} y^{n-(k+1)} + y \sum_{k=0}^n \binom n k x^k y^{n-k}
    = \\ &=
    x^{n+1} + y \parens*{ \sum_{k=1}^n \binom n {k-1} x^k y^{n-k} + y^n \sum_{k=1}^n \binom n k x^k y^{n-k} } + y^{n+1}
    = \\ &\reloset {\eqref{eq:thm:pascals_identity}} =
    x^{n+1} + y \sum_{k=1}^n \binom {n+1} k x^k y^{n-k} + y^{n+1}
    = \\ &=
    \sum_{k=0}^n \binom {n+1} k x^k y^{(n+1)-k}.
  \end{balign*}
\end{proof}

\begin{proposition}\label{thm:xn_minus_yn_factorization}
  For every \hyperref[def:ring]{ring} element \( x \) and every nonnegative integer \( n \), we have
  \begin{equation}\label{eq:thm:xn_minus_yn_factorization}
    x^{n + 1} - y^{n + 1} = (x - y)(x^n + x^{n-1} y + \cdots + y^n) = (x - y) \sum_{k=0}^n x^k y^{n-k}.
  \end{equation}
\end{proposition}
\begin{proof}
  \begin{align*}
    (x - y) \sum_{k=0}^n x^k y^{n-k}
    &=
    \sum_{k=0}^n x^{k+1} y^{n-k} - \sum_{k=0}^n x^k y^{n-k+1}
    = \\ &=
    \sum_{k=1}^{n+1} x^k y^{(n+1)-k} - \sum_{k=0}^n x^k y^{(n+1)-k}
    =
    x^{n+1} - y^{n+1}.
  \end{align*}
\end{proof}

\begin{definition}\label{def:factorial}
  We define the \term{factorial} of a \hyperref[def:integer_signum]{nonnegative integer} \( n \) recursively as follows:
  \begin{equation*}
    n! \coloneqq \begin{cases}
      1,          &n = 0, \\
      (n - 1)! n, &n > 0.
    \end{cases}
  \end{equation*}
\end{definition}

\begin{proposition}\label{thm:gamma_function_interpolates_factorial}
  For every \hyperref[def:integer_signum]{nonnegative integer} \( n \) we have
  \begin{equation*}
    \Gamma(n + 1) \coloneqq n!,
  \end{equation*}
  where \( \Gamma \) is the Gamma function defined in \fullref{def:gamma_function}.
\end{proposition}
\begin{proof}
  We use induction on \( n \).
  \begin{itemize}
    \item If \( n = 0 \), then
    \begin{equation*}
      \Gamma(1)
      =
      \int_0^\infty x^0 e^{-x} \dl x
      =
      -e^{-x}\restr_{x=0}^\infty
      =
      -\underbrace{\lim_{x \to \infty} e^{-x}}_{0} + 1
      =
      1
      =
      0!
    \end{equation*}

    \item If \( n > 0 \) and \( \Gamma(n) = (n - 1)! \), then
    \begin{balign*}
      \Gamma(n + 1)
      &=
      \int_0^\infty x^n \cdot e^{-x} \dl x
      = \\ &=
      \underbrace{(- x^n e^{-x})\restr_{x=0}^\infty}_{-(0 - 0)} + n \int_0^\infty e^{-x} x^{n-1} \dl x
      = \\ &=
      n \Gamma(n)
      = \\ &=
      n (n - 1)!
      = \\ &=
      n!
    \end{balign*}
  \end{itemize}
\end{proof}

\begin{theorem}[Stirling's factorial approximation]\label{thm:stirlings_factorial_approximation}
  For every \hyperref[def:integer_signum]{nonnegative integer} \( n \) there exists some constant \( \theta \in (0, 1) \) such that
  \begin{equation*}
    n! = \sqrt{2 \pi n} \cdot \parens*{ \frac n e }^n \cdot e^{\frac 1 {12n + \theta}}.
  \end{equation*}
\end{theorem}
\begin{proof}
  Follows from \fullref{thm:gamma_function_interpolates_factorial} and \fullref{thm:stirlings_gamma_approximation}.
\end{proof}
